
\addchap{Abbreviations and symbols}

\begin{multicols}{2}
\begin{tabbing}
AC \hspace{6ex} \= Attributive Construction \\
Adj. \> Adjective \\
ALC \> Analytic Linker Construction \\
Ap. \> Apocope (Construct State) \\
alt. \> alternative form \\
BCE \> Before the Common Era \\
C \> Christian \\
CE \> Common Era \\ 
ch. \> chapter (in references) \\
Cl. \> Clause \\
Conj. \> Conjunction \\
CSC \> Construct State Construction \\
DAC \> Double Annexation \\ \> Construction \\
\DiyZ \> \Diy (\ili{NENA} \\ \> dialect) \\
DLC \> Dative Linker Construction \\
ed. \> edited (by), editor \\
f., ff. \> and following page(s) \\
fn. \> footnote (in references)\\
Intrg. \> Interrogative pronoun \\
J \> Jewish \\
lit. \> literally \\
N \> Noun / Nominal \\
NE \> North-Eastern \\
\ili{NENA} \> North-Eastern Neo-Aramaic \\
NP \> Noun Phrase \\
NW \> North-Western \\
\ili{NWNA} \> North-Western Neo-Aramaic \\
p.c. \> personal communication \\ 
PP \> Prepositional Phrase \\
Prep. \> Preposition \\
Q. \> Quantification \\
SE \> South-Eastern \\
W. \> Western \\
\end{tabbing}
\end{multicols}

\section*{Gloss labels}
\label{ap:glosses}
The glossing of the examples follows the Leipzig Glossing Rules \citep{LeipzigGlossingRules}, with some additions. Proper nouns are abbreviated in the glosses.
The following gloss labels are used:

\begin{multicols}{2}
\begin{tabbing
1, 2, 3 \hspace{4ex} \= \first, \second, \third person \\%
\agent \> agent-like argument \\%
\abs \> absolute state \\
\acc \> accusative \\
\adj \> adjectival derivation \\
\agr \> agreement features \\
\aux \> auxiliary \\
\caus \> causative \\
\comp \> complementizer \\
\compr \> comparative \\
\cop \> copula \\
\cst \> construct state \\%
\dat \> dative \\
\definite \> definite  \\%
\dem \> demonstrative \\%
\textsc{det} \> determiner \\
\dir \> directional \\
\emp \> emphatic state \\
\far \> distal \\%
\ez \> Ezafe \\%
\exist \> existential particle ($\exists$) \\
\textsc{f}, \fem \> feminine (singular)\\%
\free \> free state \\
\fut \> future \\
\gen \> genitive  \\%
\imp \> imperative \\
\iprf \> imperfect \\
\ind \> indicative \\%
\indef \> indefinite \\%
\inf \> infinitive \\%
\invar \> invariable form \\
\lnk \> linker \\%
\textsc{m}, \masc \> masculine (singular) \\%
\nom \> nominative \\
\neg \> negation \\%
\obl \> oblique \\
\ord \> ordinal \\%
\patient \> patient-like argument \\%
\pass \> passive \\
\pl \> plural \\%
\poss \> possessive pronominal suffix \\%
\pro \> pronominal \prim \\
\prog \> progressive \\
\pst \> past \\%
\prtc \> active participle \\%
\prf \> perfect \\
\near \> proximal \\%
\refl \> reflexive \\%
\rel \> relativizer \\
\resl \> resultative participle \\%
\subj \> subjunctive \\%
\sg \> singular \\%
\supr \> superlative \\
\schwa \> vocalic nucleus of a \cst\ suffix \\
\end{tabbing}

\end{multicols}

\paragraph*{Notes regarding the glossing of verbs}

\begin{enumerate
\item The present and preterite bases of \ili{NENA} and Kurdish verbs are not glossed explicitly. Instead, the verbal base is glossed by an \ili{English} verb in base form (\textit{do}) or past form (\textit{did}) respectively. The past participle (\textit{done}) is used as gloss for the \ili{NENA} resultative (passive/perfect) \isi{participles}, followed by the \resl\ gloss
\item The explicit glossing of agent (\agent) and patient (\patient) pronominal arguments of the verb is only done when both arguments appear
\item A verb is glossed as subjunctive (\subj) or indicative (\ind) only when the two forms are different


\end{enumerate

The general format of examples is detailed \vpageref{ex:example}.

\newpage

\section*{Brackets and symbols}

\begin{itemize}
\item[C, V] Consonant, Vowel
\item[X, Y] \Prim, \Secn
\item[( )] In gloss: gender of nouns \\ In text or translation: context of an example \\ In translation only: material added to clarify the translation \\ In tables: form with restricted use
\item[(?)] Uncertain gloss 
\item[{[ ]}] Important constituent 
\item[\{ \}] Optional element
\item[/] Alternative formulations
\item[*] In examples: unattested or ungrammatical form \\
In historical discussions: reconstructed form
\item[.] Gloss separator
\item[-] Morpheme boundary 
\item[\cb] Clitic boundary
\item[ˈ] Intonation boundary
\item[⁺, \parplus] Phonological velarization
\item[V́, V̀] Word-stress, utterance-stress
\item[\ldots] Hesitation in speech; elided material
\item[\zero] Paradigmatic/Morphological Zero (lack of overt element)
\item[$\updownarrow, \leftrightarrow$] Apposition
\item[$\mapsto$] Dependency (Attributive) relation
\item[$\Longleftrightarrow$] Co-reference
\end{itemize}


\textsc{small caps} are used to introduce key concepts.