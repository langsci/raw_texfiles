\chapter{Attributive constructions in Syriac}
\label{ch:Syriac}
\renewcommand{\defaultDialect}{\Syr}

\section{Introduction}

The point of departure of this study is \Syr, taken to be an approximation of the precursor(s) of the \ili{NENA} dialects of the \CAram period. While it is sometimes assumed that all \ili{NENA} dialects developed from a unique undocumented Proto-\ili{NENA} dialect \citep[cf.][]{Hoberman1988history}, such an assumption is uncertain at the current state of our knowledge. A plausible alternative assumption is that the dialectal continuum observed in \ili{NENA} existed also in the \CAram period \citep[cf.][512]{KimStammbaum}. Be that as it may, only a few Eastern dialects of that period left traces as literary languages. Arguably, among these Syriac is the best documented. Thus, in the absence of contrary evidence, I assume that any constructions extant in Syriac existed as well in the pre-\ili{NENA} dialects. This assumption is supported by the fact that the constructions surveyed in this chapter are by and large extant also in two other documented Eastern \CAram languages, namely \JBA and \CMand. Where relevant, some comparative notes regarding these two languages are given as well. 



The research into ACs in \Syr, often termed \enquote{genitive} in the literature, is of course old and vast, and in the following I cannot expect to innovate much. Rather, the aim of the current section is to position the data about the Syriac attributive system in the framework described in \sref{ss:typology_here}, to facilitate comparison with the \ili{NENA} dialects and contact languages. 

My point of departure is the seminal article of \citet{GoldenbergAttribution}, \enquote{Attribution in \ili{Semitic} Languages}, in which he masterfully analyses the basic constructions available in Syriac. A further extension of these ideas is given in \citet[Ch.\ 14]{GoldenbergSemitic}, particularly in pp.\ 236ff.\ regarding Syriac.

The data in this survey are drawn from two types of sources: On the one hand, I have consulted Syriac grammars, notably the classical grammars of \citet{NoldekeSyriac}, \citet{DuvalSyriaque} and the pedagogical grammars of \citet{MuraokaSyriac, MuraokaHebraists}. On the other hand, I have drawn extensively on textual studies of various Syriac texts of the \Pesh\ -- \textit{The book of 1 Kings} \citep{WilliamsKings}, \textit{The Gospel of Matthew} \citep{JoostenMatthew},
\textit{Sirach (The Wisdom of Ben-Sira)} \citep{PeursenBenSira} and \textit{The Prayer of Manasseh} \citep{GutmanVanPeursen} -- as well as \textit{The Book of the Laws of the Countries} of Bardaisan \citep{BakkerBardaisan}. In all cited examples I have indicated both the primary source (if given) and the secondary source in which I have found the example. Whenever possible, I have tried to verify the correctness of the example in the primary source.\footnote{I would like to thank my colleague \name{Ralph}{Barczok} for helping me in finding some of the more obscure primary sources.} To round off the picture, I have also gathered numerous examples directly from the first part of the \textit{Acts of Thomas} published by \citet[\syr{172}--\syr{184}]{WrightActs}\footnote{The page numbers of the Syriac section of this edition are given in Syriac letters, a convention which I have kept in the citations below. The same text is reproduced in the chrestomathy of \cite[30*--40*]{MuraokaSyriac}.}, as well as some examples from the Syriac dictionary of \citet{CSD}.\footnote{In all cases the Syriac text is reproduced as it appears in the source cited. For the sake of consistency, however, it is always transcribed according to the East-Syriac vocalisation system, which indicates length only for the vowels \phonemic{a} and \phonemic{e}. Unpronounced letters are generally not transcribed except in some suffixes and clitics in which they are printed as superscript letters. Spirantization, which is  usually not  phonemic, is not transcribed.} Some further examples were taken from specialized articles cited below. 

The chapter is organised as follows: The next section gives a brief reminder regarding the three morphological states present in Syriac. \Sref{ss:Syriac_poss} discusses the use of possessive pronominal suffixes, while the three subsequent sections deal with the three main attributive constructions of Syriac, namely the \isi{construct state} construction (\sref{ss:syr_cst}), the \isi{analytic linker construction} (\sref{ss:syr_ALC}), and the \isi{double annexation} construction (\sref{ss:syr_DAC}). \Sref{ss:dat_lnk} deals with the marginal dative linker construction, while \sref{ss:syr_adj} presents the \isi{juxtaposition-cum-agreement} construction used by adjectival \secns. \Sref{ss:syr_conclusions} concludes this chapter with some general remarks.


\section{The three states in Syriac}

Following the discussion in \sref{ss:state}, recall that Syriac, like other Aramaic varieties of antiquity, possesses a 3-way state distinction in nouns, namely the \concept{construct state},  and two free states: the \concept{absolute state} and the \concept{emphatic state}, the latter being the citation form of nouns and adjectives. In Syriac, the \isi{absolute state}  is used only in specific syntactic environments (especially with adjectives used predicatively), while the \isi{emphatic state} is the commonly used form of the (free) noun \citep[22]{MuraokaSyriac}. Morphologically the \isi{emphatic state} is marked by means of an \transc{-ā} suffix (in the \sg*), which is absent in the absolute and construct states. The \masc\ absolute and construct states, moreover, are identical in form. Examples of these forms are given in \vref{tb:syr_states}.

\begin{table}[h!]
 \centering
 \begin{tabular}{l c c c c}
 \toprule
	 & \abs & \cst & \emp & \transl{his} \\
 \midrule
 \transl{name} & \multicolumn{2}{c}{\transc{šem}} & \transc{šm-ā} & \transc{šm-ēh} \\
 \transl{king} & \multicolumn{2}{c}{\transc{mlek}} & \transc{malk-ā} & \transc{malk-ēh} \\
 \transl{queen} & \transc{malkā} & \transc{malka-t} & \transc{malkt-ā} & \transc{malkt-ēh} \\
 \bottomrule
 \end{tabular}
 \caption{The three states of Syriac singular nouns, as well as their pre-suffixal forms with the 3\masc\ possessive pronoun} \label{tb:syr_states}
 \end{table}

\section{Possessive pronominal suffixes (X-y.\poss)} 
\label{ss:Syriac_poss}

Syriac, as all \ili{Semitic} languages, has a set of pronominal suffixes, which attach to nouns and prepositions \parencites[19]{MuraokaSyriac}[88]{GoldenbergSemitic}. Following the conventional terminology, I shall call these suffixes \concept{possessive pronoun}\textsc{s}. These suffixes attach to the \concept{pre-suffixal} nominal stem \citep[xix]{GoldenbergSemitic}, which can be derived from the \emp* by dropping off the \emp* suffix \transc{-ā}.  Consequently, the possessive pronouns, just like the \masc\ \cst* \zero\ suffix (i.e.\ lack of suffix), stand in opposition to the \emp* suffix \transc{-ā} (see \ref{tb:syr_states}).

\syacex{Noun}{Pronoun}{984}
{ܗܲܝܡܵܢܘܼܬ݂ܹܗ}
{haymānut-ēh}
{faith-\poss.3\masc}
{his faith}
{\cite[70, \S 91e]{MuraokaSyriac}}

\syacex{Noun}{Pronoun}{1031}
{ܒܫܡܗ}
{ba\cb{} šm-ēh}
{in\cb{} name-\poss.3\masc}
{by his name}
{\Pesh, Prayer of Manasseh, ed.\ \cite[A3]{BaarsSchneider}; \cite[90 (3A)]{GutmanVanPeursen}}

\newpage 
The possessive pronouns are strongly bound to the \isi{head noun}, and have scope over it alone. Thus, they should be seen as morphological word-level inflectional suffixes (see \sref{ss:clitics_affixes}). Whenever two nouns are conjoined, each noun must have its own \isi{possessive suffix} (contrast with \example{1030}):

\syacex{Conjoined Nouns}{Pronoun}{1032}
{ܪܘ\hspace{-0.7ex}ܓܙ\hspace{-0.4ex}ܟ ܘܚܡ̣ܬ\hspace{-0.3ex}ܟ}
{rugz-āk w\cb{}ḥemt-āk}
{rage-\poss.2\masc{} and\cb{}fury-\poss.2\masc}
{your rage and your fury}
{\Pesh, Prayer of Manasseh, ed.\ \cite[A3]{BaarsSchneider}; \cite[90 (4A)]{GutmanVanPeursen}}

 While syntactically the N+\poss\ construction is parallel to an NP, morphologically it is equivalent to a single noun.\footnote{The behaviour of the suffixed noun as a single noun can be illustrated by the observation of \citet[187, fn. 21]{PeursenBenSira}, who notes that in the corpus of Sirach the maximal chain of \cst* nouns consists of two \cst* nouns and one final noun, irrespectively of the question whether the final noun bears a \isi{possessive suffix} or not. \label{ft:SirachSuffix} } 


\section{The construct state construction (X.\textsc{cst} Y)} \label{ss:syr_cst}

The formally simplest, though not most frequent, \isi{attributive construction} in Syriac is the \cst* construction (CSC), in which the \prim appears in the \cst*. 

\syacex{Noun}{Noun}{960}
{ܡܲܠܟ݁ܘܼܬ݂ ܫܡܲܝܵܐ}
{malkut šmayyā}
{kingdom.\cst{} heaven}
{kingdom of heaven}
{\Pesh, Matthew 11:11; \cite[61, \S 73a]{MuraokaSyriac}}


\citet[61]{MuraokaSyriac} notes that this construction \enquote{tends to be confined to standing phrases verging on compound nouns}, citing the following examples:

\syacex{Noun}{Noun}{959}
{ܓܙܵܪ ܕܝܼܢܵܐ}
{gzār dinā}
{decision.\cst{} judgement}
{verdict}
{\cite[61, \S 73a]{MuraokaSyriac}}\antipar\newpage 

\syacex{Noun}{Noun}{963}
{ܒܲܪ ܚܹܐܖܹ̈ܐ}
{bar ḥērē}
{son.\cst{} free.\pl}
{a free-born man}
{\cite[61, \S 73b]{MuraokaSyriac}}

In such idiomatic cases, the \secn is non-referential. Furthermore, in idiomatic usage, one finds also cases where the \secn is a PP:

\syacex{Infinitive}{\PP}{1131}
{ܡܣܳܡ ܒܪܻܝܫܳܐ}
{msām b\cb{} rišā}
{put.\inf.\cst{} in\cb{} head}
{death penalty}
{\cite[338, \S 357a]{DuvalSyriaque}}

\syacex{Noun}{\PP}{1132}
{ܡܰܦ݁ܰܩ ܒܪܽܘܚܳܐ}
{mappaq b\cb{} ruḥā}
{utterance.\cst{} in\cb{} spirit}
{an excuse}
{\cite[338, \S 357a]{DuvalSyriaque}}



The compounding is sometimes reflected in the orthography when the expression is spelt as one word (with possible further phonetic reductions), such as \textsyriac{ܡܣܡܒܪܫܐ} \foreign{msām-b-rišā}{death penalty} used as an alternative spelling of example \ref{ex:1131} \citep[285]{CSD}, or the frequently occurring \textsyriac{ܒܪܢܫܐ} \foreign{bar-nāšā}{man} (lit. son-man).

The morpho-syntactic independence of the two members of the construction is apparent, on the other hand, when they are separated by intervening material, notably second position clitics, be they certain conjunctive adverbs\footnote{See \citet[67]{PeursenFalla} for the term \concept{conjunctive adverb}.} or the \isi{enclitic} personal pronoun. It should be noted, however, that such cases are not so frequent and these clitics tend to normally appear after the entire CSC \citep[69]{PeursenFalla}.\footnote{See further \citet[183]{PeursenBenSira} who defines the CSC as an indivisible \concept{phrase atom}. }

\syacex{Noun}{Noun}{967}
{ܒ̈ܢܱܝ ܕܶܝܢ ܒܱܠܗܳܐ}
{bnay \cb{}dēn Balhā}
{son.\pl.\cst{} \cb{}however B.}
{the sons of Bala, however}
{\textit{Zachariae Episcopi Mitylenes}, \textit{Anectoda Syriaca} Vol.\ 3, ed.\ \cite[39, 16]{Land} \apud \cite[157, \S 208A]{NoldekeSyriac}}

\syacex{Noun}{Noun}{968}
{ܕܰܒܢܱܝ̈ ܐܷܢܘܿܢ ܙܰܕܺܝ̈ܩܷܐ}
{bnay \cb{}ʾennon zaddiqē
{son.\pl.\cst{} \cb{}3\pl{} righteous.\pl{}}
{They are sons of the righteous.}
{\textit{S. Ephræmi Opera} vol.\ 2 ed.\ \cite[384 D]{BenedictusEphraem}; \cite[158, \S 208A]{NoldekeSyriac}}


In these examples, the \prims must carry word-stress, otherwise clitics could not attach to them. Thus, following the discussion in \sref{ss:cst_Semitic}, they confirm the view that \cst* in Syriac as in many other \ili{Semitic} languages is not merely a phonological artefact resulting from stress shift, but is rather a marker of a morpho-syntactic category. Further evidence to this is adduced by the quite rare case where two conjoined nouns appear in \cst*.\footnote{Recall that while this usage is very rare in Syriac and other classical \ili{Semitic} languages, a similar construction is quite frequent in \MHeb (mostly in the written style, see \example{MHeb_conj}) and in Standard (modern) \ili{Arabic} \citep[138f.]{BadawiCarter}, hinting that this is in fact a natural development of the language.}

\syacex{Conjoined nouns}{Noun}{1009}
{ܟܳܬ̈ܒܱܝ ܘܩܳܪ̈ܝܱܝ ܫܡܳܗܰܝ̈ܗܘܿܢ}
{[kātbay w\cb{} qāryay] šmāhay-hon}
{writing.\cst{} and\cb{} reading.\cst{} names-\poss.3\mpl}
{writing and reading their names}
{\textit{Zachariae Episcopi Mitylenes}, \textit{Anectoda Syriaca} Vol.\ 3, ed.\ \cite[136, 14]{Land} \apud \cite[158, \S 208A]{NoldekeSyriac}}


Noun phrases, including inflected possessed nouns, can regularly appear as \secns of the CSC, whether they represent an \isi{attributive construction} (leading to a chain of constructs\footnote{\citet[187]{PeursenBenSira} notes that at most one embedded CSC occurs as the \secn of the CSC in the corpus of Sirach. See also \vref{ft:SirachSuffix}.}), a conjoined NP, or a combination of both.

\syacex{Noun}{Possessed Noun}{1144}
{ܢܐ\hspace{-0.7ex}ܓܪܬ ܪܘܚܟ}
{naʾgrat ruḥ-āk}
{prolonging.\cst{} spirit-\poss.2\masc}
{your patience\footnotemark}
{\textit{Acts of Thomas}, ed.\ \cite[\syr{179}]{WrightActs} = \cite[37*]{MuraokaSyriac}}
\footnotetext{The form \textsyriac{ܢܐ\hspace{-0.7ex}ܓܪܬ} \transc{naʾgrat}, unattested elsewhere, is understood to be a synonym of \textsyriac{ܢܓܝܪܘܬ} \foreign{naggirut}{patience}, or a corruption thereof \citep[37*, fn. 62]{MuraokaSyriac}.}


\syacex{Noun}{Noun Phrase}{1001}
{ܠܥܝܢ ܒܢ̈ܝ ܐܢܫܐ}
{l\cb{} ʿayn [bnay nāšā]}
{to\cb{} eye son.\pl.\cst{} man}
{in the eyes of men}
{\Pesh, Sirach 1:29 \apud \cite[186]{PeursenBenSira}}


\syacex{Noun}{Conjoined Nouns}{1020}
{ܣܘ̈ܟܝ ܬܫܒܘܚܬܐ ܘܐܝܩܪܐ}
{sawkay [tešbuḥtā w\cb{} ʾiqārā]}
{branch.\pl.\cst{} praise and\cb{} honour}
{branches of praise and honour}
{\Pesh, Sirach 24:16 \apud \cite[210]{PeursenBenSira}}


\syacex{Noun}{Conjoined Noun Phrases}{1000}
{ܘܒܝܬ ܡܣܡܟ ܬܫܒܘܚܬܐ ܘܐܝܩܪܐ ܕܠܥܠܡ}
{bēt [[mesmak tešboḥtā] w\cb{} [iqārā da\cb{} l\cb{} ʿālam]]}
{house.\cst{} support.\inf.\cst{} praise and\cb{} honour \lnk\cb{} to\cb{} eternity.\abs}
{a house of support of praise and eternal honour}
{\Pesh, Sirach 1:19 \apud \cite[210]{PeursenBenSira}}


Moreover, the \secn can be determined by an attributive demonstrative: 



\syacex{Noun}{Determined noun}{1043}
{ܪܽܘܡܻܝ ܐܰܢ̱ܬܰܬ ܗܰܘ ܡܰܠܟܳܐ}
{Rumi ʾattat haw malkā}
{R. wife.\cst{} \dem.\masc{} king}
{Rumi, wife of this king}
{Simeon Beth-Arsamensis, \textit{Homeritarum martyrium}, ed.\ \cite[368, 2]{AssemanusBibliotheca}; \cite[339, \S 357f]{DuvalSyriaque}\,\footnotemark}
\footnotetext{\citeauthor{DuvalSyriaque} mistakenly gives the wrong page number (365).}

In the context of the ALC, such a demonstrative is arguably a \isi{definite article} (\cite[65]{PatElCorrelative}; see discussion in \sref{ss:syr_corr}. 

\subsection{Adjectives and participles as \prims}

The CSC can be headed by other part-of-speech categories as well, notably \isi{participles} and adjectives. A CSC headed by a participle yields a nominalisation of a verbal phrase, where the \secn corresponds to an argument (not necessarily a direct one) of the \prim.

\syacex{Participle}{Noun Phrase}{1012}
{ܡܢ ܝܬܒܝ ܟܘܪ̈ܣܘܬܐ ܕܡ̈ܠܟܐ}
{men yātbay [kursāwātā d\cb{} malkā]}
{from sit.\pl.\prtc.\cst{} thrones \lnk\cb{} king}
{from those who sit on royal thrones}
{\Pesh, Sirach 40:3; \cite[203]{PeursenBenSira}}


A CSC composed of an adjectival \prim and nominal \secn is quite peculiar in its semantics, as it is the \prim (the adjective) which qualifies the \secn (the noun), yet the entire phrase acts as an adjective phrase. Thus, in the following example, \foreign{saggi ḥnānā}{great.\cst{} compassion} should be understood as \transl{bearer of great compassion}. The example shows, moreover, the equivalence of such CSCs to regular adjectives.\footnote{See \citet{GoldenbergAdjectivization} for an analysis of the phenomenon in \Arab, and \citet{DoronAdjectival} for a analysis of the phenomenon in \MHeb, cast in formal semantics terminology. In \ili{NENA}, on the other hand, such examples are very rare (see the \Qar \example{557}), and in most dialects virtually non-existent.} 

\syacex{Adjective}{Noun}{1039}
{ܕܐܢܬ ܗܘ ܡܪܝܐ ܢܓܝܪ ܪܘܚܐ܁ ܘܡܪܚܡܢܐ ܘܣ̇ܓܝ ܚܢܢܐ}
{ʾat \cb{}\textsuperscript{h}u māryā ngir ruḥā wa\cb{} mraḥmānā w\cb{} saggi ḥnānā} 
{2\masc{} \cb{}3\masc{} Lord long.\cst{} spirit and\cb{} merciful and\cb{} great.\cst{} compassion}
{You are the Lord, long-suffering and merciful and of great compassion.}
{\Pesh, Prayer of Manasseh, ed. \cite[A7]{BaarsSchneider};  \cite[217 (7a)]{GutmanVanPeursen}}


\subsection{Adverbial \secns}

Deviating further from the typical noun+noun AC, adjectives and \isi{participles} in \cst* can be followed also by adverbials (PPs or adverbs), including infinitives headed by the preposition \transc{l-} \parencites[76, \S 96b]{MuraokaSyriac}[53ff.]{BrockConstruct}.

Of particular interest is the usage of the \secn \foreign{b\cb{} kull}{in all},  which according to \citet[54f.]{BrockConstruct}  was used in sixth- and seventh-century translations as the equivalent of  Greek superlatives. 

\syacex{Adjective}{\PP}{1116}
{ܕܚܟܡ̈ܝ ܒܟܠ ܝܘܢ̈ܝܐ}
{d\cb{} ḥakkimay b\cb{} kull (yawnāye)}
{\lnk\cb{} wise.\mpl.\cst{} in\cb{} all Greek.\pl}
{of the most wise (the Greeks)}
{Eusebius, \textit{Theophania} ed.\ \cite[II.\syr{81}]{Lee}; \cite[55]{BrockConstruct}\footnotemark}
\footnotetext{Brock mistakenly attributes the edition to Cureton. Moreover, he cites the words in wrong order, putting the appositive \foreign{yawnāye}{Greeks} at the beginning of the phrase.}
A similar construction has been preserved in \ili{NENA} to express superlatives, using however nominal \secns including the pronoun \foreign{kull}{all} itself; see \examples{335}{465} for \JZax and \example{122} for \JUrm.

Another noteworthy usage of \isi{adverbial} \secns is the usage of  infinitives headed by the preposition \transc{l-}, which seems to have entered regular usage in Syriac in the sixth or seventh century as well \citep[57f.]{BrockConstruct}:

\syacex{Participle}{Infinitive}{1119}
{ܘܝ̈ܕܥܝ ܠܡܒܐܫܘ}
{w\cb{} yādʿay l\cb{} mabʾāšu}
{and\cb{} know.\ptcp.\mpl.\cst{} to\cb{} harm.\inf{}}
{and who know how to harm}
{Babai, \textit{Commentary on Evagrius' Centuries}, Cod.\ Vatic.\ syr.\ N.\ 178, f.\ 8b, ed.\ \cite[22, 2]{Frankenberg}; \cite[58]{BrockConstruct}}



Finally, there are other examples of \isi{adverbial} \secns, headed by adjectives or \isi{participles}:


\syacex{Adjective}{\PP}{1115}
{ܫܱܦܺܝܪܱܬ ܒܚܶܙܘܳܐ}
{šappirat b\cb{} ḥezwā}
{beautiful.\fpl.\cst{} in\cb{} appearance}
{beautiful in appearance}
{\Pesh, Genesis 12:11; \cite[158, \S 206]{NoldekeSyriac}}



\syacex{Participle}{\PP}{1151}
{ܥ̇ܐ̈ܠܝ ܠܓܢܘܢܐ}
{ʿāʾellay la\cb{} gnonā}
{enter.\ptcp.\mpl.\cst{} to\cb{} bridal\_chamber}
{those who enter the bridal chamber}
{\textit{Acts of Thomas}, ed.\ \cite[\syr{182}]{WrightActs}}


\syacex{Participle}{Adverb}{1117}
{ܡܳܝܬܱܝ ܩܱܠܺܝܠܴܐܺܝܬ}
{māytay qalilāʾit
{die.\ptcp.\mpl.\cst{} quickly}
{those who die quickly}
{\textit{Acta Martyrum Orientalium et Occidentalium} Vol. 1, ed.\ \cite[79, 10]{AssemanusActa}; \cite[157, \S 207]{NoldekeSyriac}}


\subsection{Adverbial \prims}

As for \isi{adverbial} \prims, many prepositions of Syriac can be analysed as being \isi{adverbial} nouns in \cst*. Thus, the preposition \textsyriac{ܩܕܡ} \foreign{qdām}{before} is the \cst* of \textsyriac{ܩܕܡܐ} \foreign{qdāmā}{front}.

\syacex{Adverbial Noun}{Noun}{1052}
{ܩܕܳܡ ܝܰܘܡܳܐ}
{qdām yawmā}
{front.\cst{} day}
{yesterday}
{\cite[490]{CSD}}

Similarly, nouns in \cst* may join basic prepositions to form \isi{adverbial} expressions:

\syacex{Adverbial Phrase}{Noun}{1152}
{ܒܝܕ ܡܪܢ}
{[b\cb{} yad] mār-an}
{in\cb{} hand.\cst{} master-\poss.1\pl}
{by our Lord}
{\textit{Acts of Thomas}, ed.\ \cite[\syr{182}]{WrightActs}}


\subsection{The proclitic \d as a pronominal \prim}

A category which is quite restricted from appearing as \prim, if not completely absent, is that of pronouns. According to my survey, none of the independent personal pronouns, demonstrative pronouns or interrogative pronouns can appear as a \prim of the CSC construction. This is not surprising, given that in general these elements do not show state distinctions. However, according to \citet{GoldenbergAttribution} one element, namely the \isi{proclitic} \transc{d-}\~\transc{da-} can serve as a \isi{pronominal head}. Thus, he draws a parallel between the following two cases, arguing that both instantiate an \isi{attributive relationship} (a \concept{genitive construction} in his words). 

\acex{Noun}{Noun}{1040}
{deḥlat ʾalāhā}
{fear.\cst{} God}
{fear of God}
{GoldenbergAttribution}{4}

\acex{Pronoun}{Noun}{1041}
{d\cb{} alāhā}
{\textsc{pro}\cb{} God}
{that\footnotemark of God}
{GoldenbergAttribution}{4}
\footnotetext{\citeauthor{GoldenbergAttribution} translates this phrase as \enquote{N of God}.}

The syntactic equivalence between the \d \isi{proclitic} and a \cst* noun is especially clear when it is used to  repeat anaphorically a \cst* \prim, as in the following conjoined NP:\footnote{\textit{Pace} \citet[8]{WilliamsKings} this should not be seen as a \enquote{mixed construction} (as is the case with \example{1045}), but rather as two conjoined ACs. This structure should furthermore be contrasted with \example{1020} in which we have one AC, consisting of a primary modified by two conjoined \secns.}

\syacex{Noun}{Conjoined Nouns}{1130}
{ܢܒ̈ܝܝ ܒܥܠܐ ܘܕܚ̈ܘ\hspace{-0.7ex}ܓܒܐ}
{nbiyay Baʿlā wa\cb{} d\cb{} hugbe}
{prophets.\cst{} B. and\cb{} \textsc{pro}\cb{} shrines}
{the prophets of the Baal and those of the shrines}
{\Pesh, 1 Kings 19:1; \cite[21]{WilliamsKings}}



While the \isi{proclitic} \d certainly qualifies as being pronominal by virtue of replacing a noun, 
 it is hardly justifiable to see it morphologically (rather than syntactically) as being in \cst*, 
 since it does not have a corresponding \isi{free state}.\footnote{It can be traced back to the Northwest \ili{Semitic} pronominal 
 \transc{*ðū}, of which Aramaic retained the fossilized genitive form \transc{*ðī} which evolved into \d 
 \citep[437]{GzellaNWS}. As such, it is etymologically related to the demonstrative pronouns, as is evident from the \fem\ \isi{demonstrative pronoun}  \transc{hāḏē}, but synchronically it is hard to see it as the \cst* form of the former. See also \citet[297, \S 316]{DuvalSyriaque}:\enquote{Le pronom \textsyriac{ܕ} [\transc{d-}] est un ancien démonstratif, qui se subordonne un mot ou une phrase, comme un nom à l'état construit}.} Therefore, while \example{1041} clearly represents an AC, it is not exactly an instance of the CSC.\footnote{\citet{BarAsherAdnominal} presents a competing analysis, according to which \d is in essence a subordinating particle, introducing always clausal \secns (cf. \sref{ss:syr_ALC_clausal}). A noun following \d should be understood, according to this idea, as a clause representing a \textit{predicative possessive construction}, in which only the possessor is overtly expressed as a \textit{topic}. While this idea is thought provoking, it suffers from some shortcomings: first, it requires the postulation of a null existential particle in each such case. More importantly, the expression of possessors as topics is unknown in Aramaic outside this context. \label{fn:syr_BarAsher_theory}} 

As we shall see in the following section, the most frequent function of the \transc{d-} \isi{proclitic} is to stand between an overt NP and its attribute. Therefore, as discussed in \sref{ss:Analytic_AC}, we shall call it a \concept{pronominal linker} or simply \concept{linker} (glossed \lnk). In cases such as \example{1041}, we may say that it links between an implicit referent and the \secn.\footnote{Goldenberg calls this element a \concept{pronominal head}, a term which may raise some confusion since every pronoun serves as a head of its phrase. Another term found in the \ili{Semitic} literature, following the work of \citet{PennachiettiPronomi} is \concept{determinative pronoun}. \citet{WertheimerFunctions}, using a somewhat different perspective, analyses the \d as a \textit{translatif}, a term due to Tesnière denoting a conversion morpheme, as the \d can convert a noun to an an attribute, a clause to a noun, etc.}

\section{The analytic linker construction (X \textsc{lnk} Y)} \label{ss:syr_ALC}

\largerpage[2] 
Probably the most frequent AC in Syriac is the construction where the \prim and the \secn are mediated by the \isi{pronominal linker} \transc{d(a)-}. This construction, which we term the \concept{analytic linker construction} (=ALC)  is illustrated by the following examples:\footnote{See \vref{fn:ALC} for the use of the term \concept{analytic genitive construction}. The very same construction with the \lnk* \d is attested in other Eastern \il{Aramaic!Classical}Classical Aramaic languages, such as \CMand \citep{HaberlRelative}, and \JBA \citep[93, \S 4.3]{BarAsherJBA}.}

\acex{Noun}{Noun}{1042}
{deḥltā d\cb{} alāhā}
{fear.\emp{} \lnk{} God}
{fear of God\footnotemark}
{GoldenbergAttribution}{4}\antipar\antipar

\footnotetext{Goldenberg translates this example as \enquote{fear \textsuperscript{\tiny N}of God}, emphasizing the nominal nature of the \lnk*, which is not apparent in the \ili{English} translation otherwise.}

\newpage
\syacex{Noun}{Noun}{962}
{ܡܲܠܟ݁ܘܼܬ݂ܵܐ ܕܲܫܡܲܝܵܐ}
{malkutā da\cb{} šmayyā}
{kingdom.\emp{} \lnk\cb{} heaven}
{kingdom of heaven}
{\Pesh, Matthew 11:21; \cite[61, \S 73a]{MuraokaSyriac}}

\syacex{Noun}{Noun}{1137}
{ܫܠܝܚܐ ܕܐܠܗܐ}
{šliḥā d\cb{} alāhā}
{apostle \lnk\cb{} God}
{apostle of God}
{\textit{Acts of Thomas}, ed.\ \cite[\syr{178}]{WrightActs}}


 \citet{GoldenbergAttribution} analyses this construction as two constituents standing in \isi{apposition} to each other, only the latter being a \concept{genitive construction}. In the terminology used here, however, the entire construction qualifies as being an AC, the \prim and \secn standing in \concept{indirect attributive relationship}.\footnote{Compare with the  formulation of \citet[79]{GoldenbergEarly} discussing this construction: \enquote{We may call \enquote{indirect annexion} a construction in which the \isi{head noun} is represented by a formal head substitute [...] and the full noun which is replaced by that formal substitute precedes the kernel annexion as in \textsyriac{ܪܘܚܐ ܕ-ܩܘܕܫܐ} [\textit{rūḥa} [\textit{d-quḏša}]]}. The bracketing of the example clearly shows the unity of the whole construction.} Goldenberg's analysis, contrasted with this study's terminology, is presented in the \vref{tb:syr_ALC}:
 
 \begin{table}[h!]
 \centering
 \begin{tabular}{l l l}
 \toprule
 \multicolumn{3}{c}{Goldenberg's terminology} \\
 \midrule
 Apposition 	& \multicolumn{2}{|c}{{\small Genitive Construction}} \\
				& \multicolumn{1}{|l}{\hspace{0.1in}Head} & Attribute \\
 \midrule	
 deḥltā		&	d- & alāhā \\
 malkutā		& 	da- & šmayyā \\
 \midrule
 Primary	& Linker & Secondary \\
 \multicolumn{3}{c}{Attributive Construction} \\
 \midrule
 \multicolumn{3}{c}{Terminology used in this study} \\
 \bottomrule
 \end{tabular}
 \caption{Goldenberg's analysis of the ALC, contrasted with terminology used in this study} \label{tb:syr_ALC}
 \end{table}
 

 The pronominal nature of the \lnk* \d becomes evident when it appears without an immediate nominal antecedent, such as in \examples{1041}{1130}. In this work such cases are treated as having a zero (\zero) primary,  and we shall indicate the position where a nominal \prim could have appeared by the symbol \zero.\footnote{This \zero\ is thus a paradigmatic zero. In Saussurian terms it relates \textit{in absentia} to a possible antecedent. Cf. \citet[70]{MuraokaSyriac}: \enquote{At times the nucleus noun phrase to be qualified by the following Dalath [=\d linker] phrase is wanting}.} The following famous quotations illustrate this (and see also \example{1163a}):
 
 \syacex{\zero}{Noun}{987}
 {ܠܐ ܡܸܬ݂ܪܲܥܹܐ ܐܲܢ̄ܬ݁ ܕܲܐܠܗܐ ܐܸܠܵܐ ܕܲܒܢܲܝ̈ܢܵܫܵܐ}
 {lā metraʿē \cb{}ʾat \zero{} d\cb{} alāhā ʾellā \zero{} da\cb{} bnay\cb{} nāšā}
 {\neg{} think \cb{}2\sg{} \zero{} \lnk\cb{} God but \zero{} \lnk\cb{} son.\pl.\cst\cb{} man}
 {You are not thinking of things of God but of things of men.}
 {\Pesh, Matthew 16:23; \cite[71]{MuraokaSyriac}}

 
 \syacex{\zero}{Noun}{986}
 {ܗܲܒ݂ܘ ܗܵܟܸܝܠ ܕܩܸܣܲܪ ܠܩܸܣܲܪ ܘܕܲܐܠܵܗܵܐ ܠܲܐܠܵܗܵܐ}
 {habaw hākēl \zero{} d\cb{} qesar l\cb{} qesar w\cb{} \zero{} d\cb{} alāhā l\cb{} alāhā}
 {give.\imp.\pl{} then \zero{} \lnk\cb{} Caesar to\cb{} Caesar and\cb{} \zero{} \lnk\cb{} God to\cb{} God}
 {Give then that which is of Caesar to Caesar and that which is of God to God.}
 {\Pesh, Matthew 22:21; \cite[71]{MuraokaSyriac}}

\largerpage
 An intervening \isi{clitic} can easily be attached to the \prim in this constriction:
 
 \syacex{Noun}{Noun}{1165}
 {ܣܘܓܐܐ ܓܝܪ ܕܒ̈ܢܝܐ}
 {sogā \cb{}gēr da\cb{} bnayā}
 {multitude \cb{}indeed \lnk\cb{} sons}
 {many children, indeed}
 {\textit{Acts of Thomas}, ed.\ \cite[\syr{181}]{WrightActs}}
 
 
 
 Since the \prim does not directly govern  the \secn, but rather stands in \isi{apposition} with the \lnk*, it must appear in the \isi{free state}, typically being the \isi{emphatic state} in Syriac.
 There are, however, some rare exceptions to this rule:\footnote{\citet[155, \S 205B, fn. 1]{NoldekeSyriac}, however, sees such cases as textual errors. Similarly, \citet[25, fn. 6]{HopkinsName} comments: \enquote{The correctness of this construction is not well established and most of the examples alleged in the literature are plain blunders occurring in unreliable sources or the result of mistaken analysis of the text.} In the context of the \JUrm \ili{NENA} dialect, where such constructions are regular, I shall analyse  this phenomenon as \concept{agreement in state} (see \sref{ss:JUrm_cst_lnk}).}
 
 \syacex{Noun}{Possessed Noun}{1045}
 {ܝܰܘܡ̈ܰܝ ܕܛܰܠܝܽܘܬܝ}
 {yawmay d\cb{} ṭaliut-\textsuperscript{i}} 
 {day.\pl.\cst{} \lnk\cb{} youth-\poss.1\sg}
 {the days of my youth}
 {\textit{S. Ephræmi Opera} vol.\ 3 ed.\ \cite[429]{AssemanusEphraem}; \cite[339, \S 357g]{DuvalSyriaque}}
 
 As \citet{DuvalSyriaque} notes, this construction should be kept apart from the similar-looking sequence of morphemes corresponding to a simple CSC, in which the linker is an integral part of the \secn, lacking an explicit primary (here noted as \zero). 
 
 \syacex{Noun}{Noun Phrase}{1046}
{ܝܰܘܡܰܝ̈ ܕܒܶܝܬ ܕܽܘܩܠܬܝܰܢܳܘܣ}
 {yawmay [\zero{} d\cb{} bēt Dokletiyanos]}
 {day.\pl.\cst{} \hspace{0.7ex}\zero{} \lnk\cb{} house.\cst{} D. }
 {the days of those of the house of D.}
 {\textit{Julianos der Abtrünnige}, ed.\ \cite[24, 9]{Hoffmann} \apud\ \cite[339, \S 357g]{DuvalSyriaque}}

 
 Whenever two conjoined nouns appear as \secns, the \lnk* is normally repeated:
 
 \syacex{Noun}{Conjoined Nouns}{1017}
 {ܩܠܐ ܕܨ̈ܦܘܢܘܬܐ ܘܕܗܕܪ̈ܘܠܐ}
 {qālā d\cb{} ṣepunwātā w\cb{} d\cb{} hadrulē}
 {voice \lnk= bagpipes.\pl{} and\cb{} \lnk\cb{} water\_organ.\pl}
 {sound of pipes and organs}
 {\textit{Acts of Thomas}, ed.\ \cite[\syr{174}]{WrightActs}}
 
 Cases without the repetition of the \lnk* appear as well, especially  if the conjoined nouns form an idiomatic expression, as in the first of the following two examples:
 
 \syacex{Noun}{Conjoined Nouns}{1018}
 {ܕ̈ܪܐ ܕܒܣܪܐ ܘܕܡܐ}
 {dārē d\cb{} [besrā wa\cb{} dmā]}
 {generation.\pl{} \lnk\cb{} flesh and\cb{} blood}
 {the generations of flesh and blood}
 {\Pesh, Sirach 14:18 \apud\linebreak \cite[209]{PeursenBenSira}}

 
\syacex{Noun}{Conjoined Nouns}{1053}
{ܢܡ̈ܘܣܐ ܕܪ̈ܩܡܝܐ ܘܐܘܪ̈ܣܝܐ ܘܥܪ̈ܒܝܐ}
{nāmusē d\cb{} [rāqāmāyē w\cb{} ursāyē w\cb{} ʿarbāyē]}
{laws \lnk\cb{} Rakamaens and\cb{} Edessans and\cb{} Arabs}
{the laws of the Rakamaens, the Edessans and the Arabs}
{Bardaisan, \textit{Book of the Laws of the Countries} ed. \cite[46:13]{Drijvers} \apud \cite[125]{BakkerBardaisan}}
 
 In contrast to the CSC, both the \prim and the \secn can be expanded to multi-word NPs, as in the following example:
 






\syacex{Noun Phrase}{Noun Phrase}{1143}
{ܒܪܐ ܡܫܠܡܢܐ ܕܪ̈ܗܡ̣ܐ ܡ̈ܫܠܡܢܐ}
{[brā mšalmānā] d\cb{} [raḥmē mšalmānē]}
{son perfect.\masc{} \lnk\cb{} mercy(\pl) perfect.\mpl}
{perfect son of perfect mercy}
{\textit{Acts of Thomas}, ed.\ \cite[\syr{179}]{WrightActs}}


The \secn itself can be a CSC:

\syacex{Noun}{Noun Phrase}{1148}
{ܘܬܪܥܐ ܕܒܝܬ ܓܢܘܢܐ}
{tarʿā d\cb{} [bēt gnonā]}
{door \lnk\cb{} house.\cst{} bridal\_bed }
{the door of the bridal chamber}
{\textit{Acts of Thomas}, ed.\ \cite[\syr{180}]{WrightActs}}
 
 Similarly, a noun inflected with a possessive \isi{pronominal suffix} can act as a \prim or as a \secn of the ALC. Its usage as a \prim should not be confused with the DAC discussed in \sref{ss:syr_DAC}. 
 
 \syacex{Possessed Noun}{Noun}{1112part}
 {ܐܝܼܕܗ ܕܝܡܝܢܐ}
 {ʾid-ēh d\cb{} yamminā}
 {hand-\poss.3\masc{} \lnk\cb{} right(\fem)}
 {his right hand}
 {\textit{Acts of Thomas}, ed.\ \cite[\syr{178}]{WrightActs}}
 
 \syacex{Noun}{Possessed Noun}{1138}
 {ܠܘܝܐ ܕܥ̈ܒ݂ܕܘܗܝ}
 {lewyā d\cb{} ʿabd-aw}
 {companion \lnk\cb{} servant-\pl.\poss.3\masc}
 {companion of His servants}
 {\textit{Acts of Thomas}, ed.\ \cite[\syr{179}]{WrightActs}}\antipar
 \newpage 
 
  It is difficult to come across cases where two conjoined nouns act as a single \prim of the \isi{analytic linker construction}. \citet[204]{PeursenBenSira} notes that in the book of \textit{Ben-Sira} such potential constructions are rendered instead by a conjunction of two ACs, the first being the ALC and the second the \isi{possessive suffix} construction:
 
\syacex{Conjoined Nouns}{Noun}{1051}
{ܟܘܠ ܡܬܠܐ ܘܚܟܝܡ̈ܐ ܘܐܘܚ̈ܕܬܗܘܢ}
{[kol matlē d\cb{} ḥakkimē] w\cb{} [ʾuḥdātat-hon]}
{all proverbs \lnk= wise.\pl{} and\cb{} riddles-\poss.3\pl}
{all the proverbs of the wise and their riddles}
{\Pesh, Sirach 50:27; \cite[204]{PeursenBenSira}}
 
This may, however, be an artefact of this text being translated from \ili{Hebrew}. In the source text this construction would have been rendered by a CSC which prohibits (in classical \ili{Hebrew}) conjoined \prims. Indeed, in other sources one finds such cases regularly:

\syacex{Conjoined Nouns}{Noun}{1055}
{ܬܪܒܝܬܐ ܘܫܡܠܝܐ ܕܦܓܪܐ}
{[tarbitā w\cb{} šumlāyā] d\cb{} pagrā}
{growth and\cb{} perfection \lnk\cb{} body}
{the growth and perfection of the body}
{Bardaisan, \textit{Book of the Laws of the Countries} ed. \cite[34:14]{Drijvers} \apud \cite[123]{BakkerBardaisan}}


\syacex{Conjoined Noun}{Noun Phrase}{1139} 
{ܘܗ̣ܕܝܐ ܘܡܕܒܪܢܐ ܕܐܝܠܝܢ ܕܡܗܝܡܢܝܢ ܒܗ}
{w\cb{} [hadāyā wa\cb{} mdabrānā] d\cb{} [aylēn da\cb{} [mhaymēnin b-ēh]]}
{and\cb{} guide and\cb{} conductor \lnk\cb{} who.\pl{} \lnk\cb{} believe.\ptcp.\mpl.\abs{} in-3\masc}
{and guide and conductor of those who believe in Him}
{\textit{Acts of Thomas}, ed.\ \cite[\syr{179}]{WrightActs}}

\syacex{Conjouned NPs}{Participle}{1140}
{ܘܒܝܬ ܓܘܣܐ ܘܢܝ̇ܚܐ ܕܐܠܝ̈ܨ\hspace{-0.7ex}ܐ}
{[[bēt gawsā] wa\cb{} nyāḥā] d\cb{} ʾaliṣē}
{house.\cst{} refuge and\cb{} rest \lnk\cb{} afflict.\pass.\ptcp.\emp.\mpl}
{refuge and repose of the afflicted} 
{\textit{Acts of Thomas}, ed.\ \cite[\syr{179}]{WrightActs}}

 
A somewhat unusual usage of the \lnk* construction is to introduce a \secn noun which is appositive to the \prim. It could tentatively be assimilated with cases of adjectival \secns in the ALC, which are normally analysed as reduced relative clauses (see \sref{ss:syr_adjSecn}), but in contrast to those cases, the \secn is in the \emp*. 

\syacex{Noun}{Noun}{1038}
{ܓܒܪ̈ܝܗܝܢ ܕ\hspace{-0.6ex}ܓ̈ܠܝܐܐ}
{gabrey-hen d\cb{} gelyāʾē}
{man.\pl-\poss.3\fpl{} \lnk\cb{} G.\textsc{m}.\pl}
{their men, (who are) the Gelians}
{Bardaisan, \textit{Book of the Laws of the Countries} ed.\ \cite[44:17]{Drijvers} \apud \cite[121]{BakkerBardaisan}}

\largerpage
In this case, the \lnk* stands not only in \isi{apposition} with the \prim, as is always the case in the \isi{analytic linker construction}, but also with the \secn, as all nominal expressions in this example -- the \prim, the linker and the \secn, have the same referent. 
 
 \subsection{Pronominal \secns}
 \largerpage
 Pronominal \secns are realized in the ALC by means of the possessive pro\-nom\-inal suffixes (see \sref{ss:Syriac_poss}) attached to an allomorph of the \lnk*, namely \transc{dil-}. Diachronically \transc{dil-} can be analysed as a combination of the  \lnk* \d with the \isi{dative preposition} \transc{l-} but synchronically it is simply the allo-form the \lnk* takes when it attaches to the pronominal suffixes.\footnote{The same form is found in  \CMand \citep[404, \S 260]{MacuchHandbook}. In \JBA, on the other hand, the form is normally \transc{dīd-} \citep[108]{BarAsherJBA}, though one finds the  form \transc{dīl-} as well in \enquote{rare and dialectal use} \citep[331]{SokoloffJBA}. \citet[332, fn.\ 2]{NoldekeMandaic}  proposes to analyse the form \transc{dīd-} as originating in \foreign{d + yād}{\lnk+hand.\cst} \citep[cf.][60]{GarbellUrmi}, but it seems more plausible to explain it as a cognate of \transc{dil-} mutated by assimilation \citep[108]{BarAsherJBA}. The earliest attested form, from the \il{Aramaic!Early}{Early Aramaic} period, is \transc{ðīl-}. \label{ft:Noldeke_did}} 
 
 
 As \citet[90]{GutmanVanPeursen} note, it is difficult to establish a functional difference between this construction and the \isi{possessive suffix} construction, as different manuscripts of the same text may use one or the other construction.  \citet[71]{MuraokaSyriac}, on the other hand, states that this construction puts some \concept{emphasis} on the \secn. This may be related to the fact that unlike the pronominal suffixes, the base \transc{dil-} can bear stress.  Contrast \examples{984}{1032} with the following:
 
  \syacex{Noun}{Pronoun}{985}
  {ܗܲܝܡܵܢܘܼܬ݂ܵܐ ܕܝܼܠܹܗ}
  {haymānutā dil-ēh}
  {faith.\emp{} \lnk-\poss.3\masc}
  {his faith}
  {\cite[70, \S 93e]{MuraokaSyriac}}
  
   
  \syacex{Noun}{Pronoun}{1033}
  {ܒܫܡܐ ܕܝܠܟ}
  {ba\cb{} šmā dil-āk}
  {in\cb{} name.\emp{} \lnk-\poss.2\masc}
  {by your name}
  {\Pesh, Prayer of Manasseh, ed.\ \cite[B3]{BaarsSchneider}; \cite[90 (3B)]{GutmanVanPeursen}}
  
  \syacex{Noun Phrase}{Pronoun}{1030}
  {ܚܡ̣ܬܐ ܘܪܘ\hspace{-0.7ex}ܓܙܐ ܕܝܠܟ}
  {ḥemtā w\cb{} rugzā dil-āk}
  {fury.\emp{} and\cb{} rage.\emp{} \lnk-\poss.2\masc }
  {your fury and rage}
  {\Pesh, Prayer of Manasseh, ed.\ \cite[B5]{BaarsSchneider}, \cite[90 (4B)]{GutmanVanPeursen}}
  
  
  As the last example shows, in contrast to the \isi{possessive suffix} construction, when the \lnk* construction is used, one pronominal \secn is enough for a conjoined NP \prim, as the \lnk*  can be appositive to an entire NP (compare also with \examples{1051}{1055}). 
  
  Moreover, the syntactic autonomy of the \lnk* permits it to precede the \prim:
  
  \syacex{Noun}{Pronoun}{988}
  {ܕܝܠܗ ܣܦܪܐ}
  {dil-ēh seprā}
  {\lnk-\poss.3\masc{} book}
  {his book}
  {\cite[71, \S 91f]{MuraokaSyriac}}

  
  \subsection{Clausal \secns} \label{ss:syr_ALC_clausal}
  
  Relative clauses in Syriac cannot be introduced by the CSC; Rather, the ALC is obligatory used \citep[236f.]{GoldenbergSemitic}.\footnote{Other classical \ili{Semitic} languages allow the usage of the CSC with clausal \secns, notably \ili{Akkadian}, but also Classical \ili{Hebrew}, \ili{Arabic} and Ge'ez \citep[Ch.\ 14]{GoldenbergSemitic}.} For this reason, the \lnk* \d is referred by some authors also as \concept{relative pronoun} \citep[21, \S 15]{MuraokaSyriac}. It can be co-referent with any argument (or adjunct) of the \isi{relative clause}.\footnote{The term \concept{clause} is used here to cover verbal clauses with finite verbs. Nominal clauses, and in particular clauses with \isi{participial} predicates are treated in the next section.}
  
  \syacex{Noun (subject)}{Clause}{972}
  {ܢܒܝܐ ܕܸܐܬ݂ܵܐ ܠܘܵܬܲ\hspace{-0.7ex}ܢ}
  {nabyā d\cb{} etā lwāt-an}
  {prophet \lnk\cb{} came to-\poss.1\pl}
  {the prophet who came to us}
  {\cite[63, \S 77]{MuraokaSyriac}}
  
  
  \syacex{Noun (object)}{Clause}{971}
  {ܢܒ݂ܝܵܐ ܕܫܲܕܲܪܬ݀ܗ ܠܘܵܬ݂ܵ\hspace{-0.3ex}ܟ݂}
  {nabyā d\cb{} šaddar-t-ēh lwāt-āk}
  {prophet \lnk\cb{} sent-\agent1\sg-\patient3\masc{} to-\poss.2\masc}
  {the prophet whom I sent to you}
  {\cite[63, \S 77]{MuraokaSyriac}}
  
  
 
  While I agree with \citet{GoldenbergAttribution} that \d represents one and the same morpheme regardless of the material that follows it, it is worthwhile noting that the distribution of \D+Clause is somewhat different from \D+Noun, implying that these combinations are not equivalent  \citep[245f.]{PeursenBenSira}. This can be illustrated by cases in which the same \prim is expanded both by a nominal and a clausal \secn. In such cases, the two \d phrases are not conjoined (contrast with \example{1017}):\footnote{One may argue that the NP \foreign{šoʿitē d\cb{}sābē}{discourse of the elders} is the \prim of the \isi{relative clause}, rather than the noun \transc{šoʿitē}. Yet in the dependency model I use, as well as from the semantic point of view, the \isi{relative clause} is an expansion of \foreign{šoʿitē}{discourse} alone.}
  
  \syacex{Noun}{Noun+Clause}{1029}
  {ܒܫܘܥ̈ܝܬܐ ܕܣܒ̈ܐ ܕܫܡܥܘ ܡܢ ܐܒܗ̈ܝܗܘܼ\hspace{-0.7ex}ܢ}
  {b\cb{} [šoʿitē d\cb{} sābē] da\cb{} [šmaʿ-\textsuperscript{u} men ʾabāh-ayhon]}
  {in\cb{} tales \lnk\cb{} elders \lnk\cb{} heard-3\pl{} from fathers-\poss.3\pl}
  {the discourse of the elders, which they have heard from their fathers}
  {\Pesh, Sirach 8:9 \apud \cite[232]{PeursenBenSira}}

  
  Notwithstanding the pronominal nature of the \lnk* \d, in its usage as a relative pronoun it does not co-occur with a zero \prim, according to my survey. Instead, an explicit pronominal \prim can occur in this construction, yielding either a \concept{free relative}, or, less frequently, a non-restrictive \isi{relative clause}.
  
  \syacex{Pronoun}{Clause}{973}
  {ܡܲܢ ܕܠܝܼ ܡܩܲܒܸ݁ܠ ܠܡܲܢ ܕܫܲܠܚܲܢܝ ܡܩܲܒܸ݁ܠ}
  {man d\cb{} [l-i mqabbel] l-man d\cb{} [šalḥa-n\textsuperscript{i}] mqabbel}
  {who \lnk\cb{} to-1\sg{} receive.\ptcp.\masc{}  to-who \lnk\cb{} sent-\patient1\sg{} receive.\ptcp.\masc{}}
  {he who receives me receives him who has sent me\footnotemark}
  {\Pesh, Matthew 10:40; \cite[87, \S 111]{MuraokaSyriac}}
\footnotetext{Note that the first \isi{relative clause} is in fact a \isi{participial} clause, treated in the next section.}
  
  \syacex{Pronoun}{Clause}{1036}
  {ܐܢܬ ܕܐܣܪܬܝܗܝ ܠܝܡܐ}
  {ʾat d\cb{} esar-t-āy l-yammā}
  {2\masc{} \lnk\cb{} bound-\agent2\masc{}-\patient3\masc{} \acc-sea}
  {You, who bound the sea}
  {\Pesh, Prayer of Manasseh, ed.\ \cite[B3]{BaarsSchneider}; \cite[221 (3a)]{GutmanVanPeursen}}
  
\largerpage[-1]
  As \citet[87]{GutmanVanPeursen} note, the \isi{interrogative pronoun} \foreign{man}{who}, which typically introduces non-specific free relatives as in example \ref{ex:973}, can in fact also introduce free relatives referring to specific referents, as in the following example:\footnote{Peripherally to this, note that in Modern \ili{Hebrew} the pronouns introducing free relatives are frequently preceded by the definite accusative marker \transc{\texthebrew{את} ʾet}, rendering them syntactically (but not necessarily semantically) definite.}
  
  \syacex{Pronoun}{Clause}{1037}
  {ܡܢ ܕܐܚ̣ܕܬ ܠܬܗܘܡܐ}
  {man d\cb{} eḥad-t la-thomā}
  {who \lnk\cb{} held-\agent2\masc{} \acc-abyss}
  {Who held\footnotemark the abyss}
  {\Pesh, Prayer of Manasseh, ed.\ \cite[B3]{BaarsSchneider}; \cite[221 (3b)]{GutmanVanPeursen}}
  
  \footnotetext{In the Syriac text, the verb appears in the \second person, yet the grammatical context seems to require a \third person. Indeed, in another version of the same text, the verb appears in the \third person; see discussion of \citet[87f.]{GutmanVanPeursen}.}

\largerpage[2]
  A quite distinct usage of the \D+Clause pattern occurs when \d serves as a \isi{complementizer}. This is the case when \d introduces complements of verbs, be they direct object or \isi{adverbial} complements. 
   These constructions are arguably not ACs at all, as their function is not to modify an implied referent. Indeed, in these cases, no nominal antecedent appears before the \d, and one could argue that no nominal \prim is possible at all in this position. Nonetheless, I list these examples for the sake of completeness, but I gloss the \d in this function as \lnkcomp.\footnote{\citet[275]{WertheimerFunctions} argues that both functions of \d, introducing relative clauses or complement clauses, stem from its more general function as a conversion morpheme (\textit{translatif} in her terms), and indeed both uses are in fact nominalizations: as a \isi{relativizer} \d nominalizes clauses into adjectives, while as a \comp* it nominalizes clauses into nouns (or \textit{substantives} in her terms). \label{ft:Wertheimer_complement} }
  
  \syacex{Verb}{Clause}{978}
  {ܫܲܪܝܼ ܕܲܢܡܲܠܸܠ}
  {šarri da\cb{} nmalel}
  {begin.\prf.3\masc{} \lnkcomp\cb{} speak.\iprf.3\masc{}}
  {He began to speak.}
  {\Pesh, Mark 12:1; \cite[65, \S 82]{MuraokaSyriac}}

  
  \syacex{Verb}{Clause}{979}
  {ܐܵܙܸܠ ܐ̄ܢܵܐ ܕܸܐܛܲܝܸܒ ܠܟ݂ܘܿܢ ܐܲܬ݂ܪܵܐ}
  {ʾāzel \cb{}nā  d\cb{} eṭayyeb l-kon ʾatrā}
  {go.\ptcp{} \cb{}1\sg{}  \lnkcomp\cb{} prepare.\iprf.1\sg{} to-\poss.2\mpl{} place}
  {I go to prepare a place for you.}
  {\Pesh, John 14:2; \cite[65, \S 82]{MuraokaSyriac}}
  
  The distinct functions of \d serving either as a \isi{complementizer} or as pronominal \lnk*  are especially clear in the rare cases where two \d morphemes follow, each with another function, as in the following example:\footnote{The \zero\ \prim refers to the noun \foreign{ʾidā}{hand}, which can be analysed as having been raised out of the CP to serve as the subject of the matrix-clause. See also \citet[35*, fn. 51]{MuraokaSyriac}. The continuation of this sentence is given in \example{1163b}.}
  
  \syacex{\zero}{Noun}{1163a}
  {ܐܫܬ̣ܟܚ̣ܬ݀ ܐܝܼܕܐ ܕܕܫ̣̇ܩܝܐ ܗܝ}
  {ʾeštakḥat ʾidā d\cb{} \zero{} d\cb{} šāqyā \cb{}\textsuperscript{h}i}
  {was\_found.3\fem{} hand(\fem) \lnkcomp\cb{} \zero{} \lnk\cb{} cupbearer \cb{}3\fem{}}
  {the hand was found to be that of the cupbearer}
  {\textit{Acts of Thomas}, ed.\ \linebreak \cite[\syr{178}]{WrightActs}}
  
  \largerpage
  The \d morpheme functions likewise as a \comp* when it follows an \isi{adverbial} acting as a conjunction. Note that in such cases the  \isi{attributive relationship} is already marked by the \cst* of the \isi{adverbial}. Compare the following with \example{1052}:
  
  \syacex{Adverbial Noun}{Clause}{975}
  {ܩܕܳܡ ܕܢܶܩܪܶܐ ܬܰܪܢܳܓܠܴܐ}
  {qdām [d\cb{} neqrē tarnāglā]}
  {front.\cst{} \hspace{0.7ex}\lnkcomp\cb{} call.\iprf.3\masc{} cock}
  {before a cock crows}
  {\cite[490]{CSD}}
  
  The structure of this example is superficially parallel to that of \examples{1045}{1046}, in that the \d follows an element in \cst*. The fine difference is that here it acts as a \comp* (nominalizing an event), and thus does not designate any implied referent. Note also that the \cst* marking is needed in order to transform the noun \textsyriac{ܩܕܳܡܳܐ} \foreign{qdāmā}{front}, which can by itself be used adverbially, into
  a conjunction requiring a complement phrase. 
  

\subsection{Adjectives and participles as \secns} \label{ss:syr_adjSecn}

Adjectives and \isi{participles} are closely related in Syriac. Both show nominal inflection, and are characterised by the fact that in predicative position they commonly appear in \abs*. Moreover, with a \third person subject, they can dispense with the \isi{enclitic} personal pronoun (e.p.p.) which normally appears after predicative nominals.

  In the following examples, the two categories are kept separate, but in some cases it is difficult to tease them apart. 

As explained in \sref{ss:syr_adj}, adjectives used attributively stand in \isi{apposition} with their \isi{head noun} and thus agree with it in state, being most commonly the \emp*. The usage of \abs*, on the other hand, is typical of the predicative use of adjectives and \isi{participles}.\footnote{This distinction is also true in \JBA; see \citet[63]{BarAsherJBA}.}
Moreover, in  \abs* they can appear as \secns in the ALC. As the \abs* is typical of predicative function, these \secns are often considered in Syriac grammars as elliptical relative clauses, lacking an explicit subject argument \parencites[66, \S 94]{MuraokaHebraists}[211ff.]{PeursenBenSira}.\footnote{Alternatively, \citet[115, \S\S 9--10]{GoldenbergSyriac} analyses these elements  as being quasi-verbal conjugated predicates, of which the \third person marker is a \zero. Yet appositive nominal \secns in \emp* can also appear in a similar syntactic structure, as  \example{1038} shows. \label{ft:syr_abs_adj_zero}}

It is quite difficult to find cases of single-word adjectival \secns  following a nominal \prim in this construction. \citet{PeursenBenSira} gives the following example as a case of \D+Adjective:

\syacex{Noun}{Participle}{1021}
{ܢܫܐ ܕܣܐܒ}
{nāšā d\cb{} sāb}
{man \lnk\cb{} old.\abs.\masc}
{an old person}
{\Pesh, Sirach 8:6 \apud \cite[211]{PeursenBenSira}}

Yet  most cases of adjectival \secns introduced in this construction are multi-word expressions:

\syacex{Noun}{Adjective Phrase}{1023}
{ܡܣܟܢܐ ܕܚܝ ܘܥܙܝܙ ܒܓܘܫܡܗ}
{meskēnā d\cb{} [ḥay w\cb{} ʿazziz b\cb{} gušm-ēh]}
{miserable \lnk\cb{} alive.\abs{} and\cb{} vigorous.\abs{} in\cb{} body-\poss.3\masc}
{a poor man who is alive and sound in his body}
{\Pesh, Sirach 30:14 \apud \cite[212]{PeursenBenSira}}

\syacex{Noun}{Adjective Phrase}{1058}
{ܠܥܒܘܬܐ ܘܐܣܘܛܘܬܐ ܕܠܐ ܡܬܒܥܝܐ}
{laʿbutā w\cb{} āsoṭutā d\cb{} [lā metbaʿāyā]}
{avidity(\fem) and\cb{} intemperance(\fem) \lnk\cb{} \neg{} necessary.\abs.\fem}
{intemperance and unnecessary luxury\footnotemark}
{Bardaisan, \textit{Book of the Laws of the Countries} ed.\ \cite[34:25]{Drijvers} \apud \cite[129]{BakkerBardaisan}}\vspace*{-.6mm
     

This tendency is inverted whenever there is no overt \prim. In such cases, one can easily find single-word adjectives following the \lnk*. The adjective in these cases is effectively nominalized, as the following example demonstrates:

\footnotetext{\citet[129, fn.\ 117]{BakkerBardaisan} argues that the \secn modifies only the second noun, arguing that \enquote{it would seem superfluous to specify \textit{intemperance} with the notion of not being necessary}. Note, however, that it is the second noun that means \enquote{intemperance}. Nonetheless, given that the adjective has a \sg* form, it is probably correct that it modifies only one of the nouns.

\syacex{\zero}{Adjective}{993}
{ܥܒܸܕ݂ܘ ܕܫܲܦ݁ܝܼܪ}
{ʿbed-\textsuperscript{u} \zero{} d\cb{} šappir}
{do.\imp-\pl{} \zero{} \lnk\cb{} beautiful.\masc.\abs}
{Do what is good}
{\Pesh, Matthew 5:44; \cite[87, \S 111]{MuraokaSyriac}}

\largerpage
According to \citet[271]{WertheimerFunctions}, who discusses similar cases with clausal \secns, the nominalisation is achieved exactly due to the lack of an overt \prim. 


One finds similar expressions with the pronominal \prim \textsyriac{ܡܕܡ} \foreign{meddem}{something}:

\syacex{Pronoun}{Adjective}{1025}
{ܡܕܡ ܕܫܦܝܪ}
{meddem d\cb{} šappir}
{something \lnk\cb{} beautiful.\masc.\abs}
{something (that is) beautiful}
{\Pesh, Sirach 23:5; \cite[211]{PeursenBenSira}}

Longer, phrasal adjectival \secns can also follow \zero{} or pronominal \prims:

\syacex
{\zero}{Adjective Phrase}{1006}
{ܘܕܩܠܝܠ ܛܒ}
{w\cb{} d\cb{} [qalil tāb]}
{and\cb{} \lnk\cb{} light.\masc.\abs{} very}
{and what is very light}
{\Pesh, Sirach 22:18 \textit{apud} \cite[199]{PeursenBenSira}}

Participial phrases  following pronominal \prims are quite regular. See the \isi{participial} \secns in \example{973} or \example{1139}, or the following example:

\syacex{Pronoun}{Participial Phrase}{1157}
{ܗܠܝܢ ܕܥܒܪ̈ܢ \ldots\ ܘܠܗܠܝܢ ܕܠܐ ܥܒܪ̈ܢ}
{hālēn d\cb{} [\opt{lā} ʿābr-ān]}
{\dem.\pl{} \lnk\cb{} \opt\neg{} pass.\ptcp-\abs.\fpl}
{those that are \opt{not} transitory}
{\textit{Acts of Thomas}, ed.\ \cite[\syr{183}]{WrightActs}}

In contrast to the case of clausal \secns with finite verbs (see \sref{ss:syr_ALC_clausal}),  \isi{participial} \secns can co-occur with \zero\ \prims, though this does not happen very frequently:

\syacex{\zero}{Participial Phrase}{994}
{ܘܸܐܡܲܪ ܠܕ݂ܵܐܬܹܝܢ ܥܲܡܹܗ}
{(w\cb{} emar l\cb{}) \zero{} d\cb{} [ātēn ʿamm-ēh]}
{and\cb{} said to\cb{} \zero{} \lnk\cb{} come.\ptcp.\abs.\mpl{} with-\poss.3\masc}
{and he said to those who were coming with him}
{\Pesh, Matthew 8:10; \cite[87, \S 111]{MuraokaSyriac}}

\syacex{\zero}{Participial Phrase}{992}
{ܕܫܲܠܝܼܛ ܓܹܝܪ ܒܟ݂ܠ ܚܲܕ݂ ܗ̄ܘܼ}
{\zero{} d\cb{} [šalliṭ gēr b\cb{} kol] (ḥad \cb{}\textsuperscript{h}u)}
{\zero{} \lnk\cb{} rule.\ptcp.\masc.\abs{} however in\cb{} all one \cb{}3\masc}
{He who controls all (is one).}
{\cite[87, \S 111]{MuraokaSyriac}}

For discussion of the factors motivating the appearance of adjectives in the ALC versus simple \isi{apposition}, see \sref{ss:syr_adj_vs}.
 

\subsection{Adverbial \secns}

Similarly to adjectival \secns, \isi{adverbial} \secns are usually considered to be reduced clauses. Being in fact PPs, they always consist of multiple words (considering the preposition itself, being a \isi{proclitic}, as a separate syntactic word):

\syacex{Noun}{\PP}{991}
{ܐܝܼܠܵܢܹ̈ܐ ܕܲܒ݂ܦ݂ܲܪܕ݁ܲܝܣܵܐ}
{ʾilānē da\cb{} b\cb{} pardaysā}
{tree.\pl{} \lnk\cb{} in\cb{} garden}
{the trees in the garden}
{\Pesh, Genesis 3:2; \cite[72]{MuraokaSyriac}}

 
The preposition \textsyriac{ܐܝܟ} \foreign{ʾak}{as} is also found
followed by \D+PP. If the PP is seen as a  reduced clause, such cases should be treated as similar in structure to \example{975}, where the \d serves as a \comp*.

\syacex{Conjunction}{\PP}{976}
{ܐܲܝܟ݂ ܕܒ݂ܲܫܡܲܝܵܐ}
{ʾak d\cb{} ba\cb{} šmayyā}
{as  \lnkcomp\cb{} in\cb{} heaven}
{as in heaven}
{\cite[64, \S 78]{MuraokaSyriac}}

\syacex{Conjunction}{\PP}{1161}
{ܐܝܟ ܕܒܐܟܣܢܝܐ}
{ʾak  d\cb{} b\cb{} aksnāyā}
{as  \lnkcomp\cb{} in\cb{} stranger}
{as upon a stranger}
{\textit{Acts of Thomas}, ed.\ \cite[\syr{175}]{WrightActs}}

\subsection{Numerals as ordinal \secns} \label{ss:syr_lnk_ord}

An noteworthy usage of this construction is to form ordinal numerals out of the cardinal numerals. The construction is especially interesting, since the \secn, i.e.\ the \isi{numeral}, agrees in gender with the \prim.\footnote{This construction is preserved in many \ili{NENA} dialects, although the gender distinction is lost in some; see \sref{ss:NENA_high_ordinals}.}  Example \ref{ex:1048} can be directly contrasted with \example{1047}.\footnote{When a \isi{numeral} functions as a cardinal, it typically precedes the quantified noun without any marking of an \isi{attributive relation}:


\syacex[(i)]{Noun}{Cardinal}{Card}
{ܚܕܳܐ ܠܰܬܠܳܬܳܐ ܝܰܘܡܺܝܢ}
{xdā la\cb{} tlātā yawmēn}
{one.\textsc{f} to\cb{} three.\textsc{m} day.\pl.\abs}
{once in three days}
{\cite[614]{CSD}}\antipar 

Similarly to an ordinal \isi{numeral}, a cardinal \isi{numeral} agrees in gender with the modified noun. The linker \d appears only sometimes following the cardinal \textsyriac{ܐܠܦ} \foreign{ʾālef}{thousand}, but in this construction the cardinal acts syntactically as a \prim \citep[see][177, \S239]{NoldekeSyriac}.}

\syacex{Noun}{Ordinal}{1048}
{ܝܱܘܡܳܐ ܕܰܬܖܷ̈ܝܢ}
{yawmā da\cb{} trēn}
{day(\masc) \lnk\cb{} two.\textsc{m}}
{the second day}
{\cite[178, \S 239]{NoldekeSyriac}}

\syacex{Noun}{Ordinal}{1111}
{ܒܫܲܢ̄ܬ݂ܵܐ ܕܲܬ݂ܠܵܬ݂}
{b\cb{} šattā da\cb{} tlāt}
{in\cb{} year(\fem) \lnk\cb{} three.\textsc{f}}
{in the third year}
{\Pesh, Deuteronomy 26:12; \cite[38, \S 44b]{MuraokaSyriac}}


This construction too can be used without an explicit \prim:

\syacex{\zero}{Ordinal}{1049}
{ܘܕܬܠܬܐ}
{da\cb{} tlātā}
{\lnk\cb{} third.\textsc{m}}
{a third one}
{\Pesh, Sirach 23:16 \apud \cite[258]{PeursenBenSira}}

\subsection{The analytic linker construction with a correlative} \label{ss:syr_corr}
\largerpage[2]

The ALC exhibits a variant construction in which the \lnk* is  preceded by a demonstrative or \isi{interrogative pronoun}, traditionally termed  \enquote{correlative} \citep[175f., \S 236]{NoldekeSyriac}. This happens especially frequently with clausal \secns. \citet{PatElCorrelative} discusses this construction, bringing \textit{inter alia}, the following example:

\syacex{Noun Phrase}{Clause}{1112}
{ܐܝܼܕܗ ܕܝܡܝܢܐ܇ ܗ̇ܝ ܕܐܪܝܡ ܥܠ ܝܗܘܕܐ}
{[ʾid-ēh d\cb{} yamminā] hāy d\cb{} [arim ʿal Yhudā]}
{hand-\poss.3\masc{} \lnk\cb{} right(\fem) \dem.\fem{} \lnk\cb{} struck.3\masc{} on Judas}
{his right hand with which he had struck Judas}
{\textit{Acts of Thomas}, ed.\  \cite[\syr{178}]{WrightActs}}

 
In the same textual source there is another such example, but with an \isi{enclitic} personal pronoun intervening between the \prim and the \isi{demonstrative pronoun}. This example is the continuation of \example{1163a}:

\syacex{Noun}{Clause}{1163b}
{ܕܕܫܲܩܝܐ ܗܝ. ܗ̇ܘ ܕܡܚ̣ܝܗܝ ܗܘܐ ܠܝܗܘܕܐ}
{d\cb{} šāqyā \cb{}\textsuperscript{h}i haw  da\cb{} mḥā-y \cb{}\textsuperscript{h}wa l-Yhudā}
{\lnk\cb{} cupbearer \cb{}3.\fem{} \dem.\masc{} \lnk\cb{} struck-\patient.3\masc{} \cb{}\pst{} \acc-J.}
{it was that of the cupbearer who had smitten Judas}
{\textit{Acts of Thomas}, ed.\ \cite[\syr{178}]{WrightActs}}

While structurally the pronoun may be analysed as a pronominal \prim \citep[see the analysis of][274]{WertheimerFunctions},  functionally \citeauthor{PatElCorrelative} argues that it should be seen as a \isi{definite article}, marking the attribute, and thus the entire AC, as definite.\footnote{It is interesting to note the parallelism between this construction and the classical \ili{Semitic} CSC, in which the \isi{definite article} is attached to the \secn, as discussed in \sref{ss:CSCdet}. See in this context also \example{1043}}. Indeed, such a demonstrative, acting as a \isi{definite article}, can precede nominal \prims as well (see also examples of \cite[67]{PatElCorrelative}):

\syacex{Noun}{Noun}{1034}
{ܪܘܡܐ ܗ̇ܘ ܕܫܡܝܐ}
{rawmā haw da\cb{} šmayyā}
{height \dem.\masc{} \lnk\cb{} heaven}
{the height of heaven}
{\Pesh, Prayer of Manasseh, ed.\ \cite[A10]{BaarsSchneider}; \cite[91 (7A)]{GutmanVanPeursen}}\antipar\newpage

This last example is parallel functionally to \example{1113}, which uses instead a \isi{proleptic pronoun} to render the AC definite. This is discussed in the following section.





\section{The double annexation construction (X-y.\poss\ \textsc{lnk} Y)} \label{ss:syr_DAC}

\subsection{Plain construction}

Another frequent AC used in Syriac is a variant of the ALC with a proleptic (i.e.\ cataphoric) possessive \isi{pronominal suffix} attached to the \prim, indexing the \secn. 
Following \citet[234, fn. 15]{GoldenbergSemitic} I shall term this construction the \concept{double annexation construction} (=DAC).\footnote{See \vref{ft:DAC_term} for further information on this term.}


\syacex{Noun}{Noun}{1136}
{ܪ̈\hspace{-0.8ex}ܓܠܘܗܝ ܕܫܠܝܚܐ}
{regl-aw\textsuperscript{hi} da\cb{} šliḥā}
{foot-\pl.\poss.3\masc{} \lnk\cb{} apostle}
{the feet of the apostle}
{\textit{Acts of Thomas}, ed.\ \cite[\syr{178}]{WrightActs}}

\syacex{Noun}{Noun}{995}
{ܡܸ̈ܠܲܘܗ̄ܝ ܕܡܵܪܝܵܐ}
{mell-aw d\cb{} māryā}
{word-\pl.\poss.3\masc{} \lnk\cb{} Lord}
{the words of the Lord}
{\cite[88, \S 112d]{MuraokaSyriac}}

\acex{Noun}{Noun}{ExNum}
{šm-āh d\cb{} attətā}
{name-\poss.3\fem{} \lnk\cb{} woman}
{the name of the woman}
{GoldenbergSemitic}{236}

\citet[234]{GoldenbergSemitic} analyses this construction as \enquote{a complex construction made of a sequence of two correlated annexions \textsc{n}\textsuperscript{1}--\textsc{pron}\textsuperscript{2} \~ \textsc{pron}\textsuperscript{1}--\textsc{n}\textsuperscript{2}, identical indices indicateting coreferentiality}. Schematically, he represents the construction as if there were two appositions involved:\footnote{The \isi{apposition} between the two attributes must be understood as an indirect \isi{apposition}, or merely co-reference, since each attribute is governed by another head.}

\begin{table}[h]
\centering
\begin{tabular}{c c}
\toprule
Head & Attribute \\
\midrule 
šm-	& -āh \\
$\updownarrow$	& $\updownarrow$	 \\
d-	& attetā \\
\bottomrule
\end{tabular}
\caption[Goldenberg's representation of the Double Annexation Construction]{Goldenberg's representation of the Double Annexation Construction} \label{tb:DAC}
\end{table}

The syntactic independence of the two phrases is demonstrated by cases where an intervening \isi{clitic} appears: 

\syacex{Noun}{Noun}{1150}
{ܐܚܘܗܝ ܐܢܐ ܕܝܗܘܕܐ}
{ʾaḥ-aw \cb{}na d\cb{} Yhudā}
{brother-\poss.3\masc{} \cb{}1\sg{} \lnk\cb{} Judas}
{I'm the brother of Judas}
{\textit{Acts of Thomas}, ed.\ \cite[\syr{181}]{WrightActs}}

According to \citet[61f.]{MuraokaSyriac} the DAC   is used \enquote{when both nouns ... are logically determined}. Indeed, this construction is used as an alternative to marking definite determination by means of a \enquote{correlative} \isi{demonstrative pronoun} (contrast with \example{1034}):

\syacex{Noun}{Noun}{1113}
{ܪܘܡܗ̇ ܕܫܡܝܐ}
{rawm-āh da\cb{} šmayyā}
{height-\poss.3\fem{} \lnk\cb{} heaven(\fem)}
{the height of heaven}
{\Pesh, Prayer of Manasseh, ed.\ \cite[B9]{BaarsSchneider}; \cite[90 (7B)]{GutmanVanPeursen}}

It is not accurate, however, to claim that both constituent nouns are determined. Rather, it is the AC as a whole which is determined. Thus, we can find cases where the \secn is indefinite, albeit with a generic reading. 

\syacex{Noun}{Noun}{1015part}
{ܕܢܦܫܗ ܕܡܣܟܢܐ}
{napš-ēh d\cb{} meskēnā}
{soul-\poss.3\masc{} \lnk\cb{} poor}
{the soul of a poor man}
{\Pesh, Sirach 35:20 \apud \cite[207]{PeursenBenSira}}

\syacex{Noun}{Noun}{1016}
{ܥ̇ܒܕܗ ܕܢܓܪܐ}
{ʿbād-ēh d\cb{} naggārā}
{work-\poss.3\masc{} \lnk\cb{} carpenter}
{the business of a carpenter}
{\textit{Acts of Thomas}, ed.\ \cite[\syr{174}]{WrightActs}}



The \secn may be expanded into a multi-word noun phrase or a possessed noun:

\syacex{Noun}{Noun Phrase}{969}
{ܒܪܹܗ ܕܲܐܠܵܗܵܐ ܚܲܝܵܐ}
{br-ēh d\cb{} [alāhā ḥayyā]}
{son-\poss.3\masc{} \lnk\cb{} God alive.\masc}
{the son of the living God}
{\Pesh, Matthew 16:16; \cite[62, \S 73f]{MuraokaSyriac}}

\syacex{Noun}{Possessed Noun}{1147}
{ܒܩܢܘܡܗ̇ ܕܐܠܗܘܬ\hspace{-0.3ex}ܟ}
{ba\cb{} qnom-āh d\cb{} alāhut-āk}
{in\cb{} nature-\poss.3\fem{} \lnk\cb{} divinity(\fem)-\poss.2\masc}
{in the nature of Your Godhead}
{\textit{Acts of Thomas}, ed.\ \cite[\syr{180}]{WrightActs}}

The \prim, on the other hand, cannot be a possessed noun, as the \isi{possessive suffix} is a marker of the construction itself (contrast with \example{1112part}). Whenever the \prim is expanded to a multi-word NP, the \isi{possessive suffix}, being a nominal suffix, must be attached to the \isi{head noun} itself, or if some head nouns are conjoined, to each of them \citep[cf.][340, \S 359b]{DuvalSyriaque}:

\syacex{Noun Phrase}{Noun}{999}
{ܚܲܝܠܹܗ ܪܲܒ݁ܵܐ ܕܲܐܠܵܗܵܐ}
{[ḥayl-ēh rabbā] d\cb{} alāhā}
{\hspace{0.7ex}power-\poss.3\masc{} great.\masc{} \lnk\cb{} God}
{the great power of God}
{\Pesh, Acts 8:10; \cite[89, \S 112j]{MuraokaSyriac}}

\syacex{Conjoined Nouns}{Noun}{1054}
{ܙܒ̈ܢܘܗܝ ܘܙܢܘ̈ܗܝ ܕܟܝܢܐ}
{zabn-aw wa\cb{} zn-aw da\cb{} kyānā}
{time-\pl.\poss.3\masc{} and\cb{} manner-\pl.\poss.3\masc{} \lnk\cb{} nature}
{the periods and modes of nature}
{Bardaisan, \textit{Book of the Laws of the Countries} ed.\ \cite[34:10--11]{Drijvers}, \cite[123]{BakkerBardaisan}}

The DAC can also be embedded in a larger AC. This  is the case of \example{1015part}, which is embedded in the following example. Note that the definite value of the DAC is  propagated to the entire AC:

\syacex{Noun}{Noun Phrase}{1015}
{ܡܪܪܐ ܕܢܦܫܗ ܕܡܣܟܢܗ}
{mrārā d\cb{} [napš-ēh d\cb{} meskēnā]}
{bitterness \lnk\cb{} soul-\poss.3\masc{} \lnk\cb{} poor}
{the bitterness of the soul of a poor man}
{\Pesh, Sirach 35:20 \textit{apud}\linebreak \cite[207]{PeursenBenSira}}

The \secn can itself be pronominal, leaning on the linker \transc{dil-}:

\syacex{Noun}{Pronoun}{989}
{ܣܸܦ݂ܪܹܗ ܕܝܠܗ}
{sepr-ēh dil-ēh}
{book-\poss.3\masc{} \lnk-\poss.3\masc}
{his book}
{\cite[71, \S 91f]{MuraokaSyriac}}

Only in such cases can we find a \isi{possessive pronoun} on the \prim which is not of the \third person:

\syacex{Noun}{Pronoun}{1158}
{ܠܙܥܘܪܘܬܝ ܕܝܠܝ}
{zʿorut-\textsuperscript{i} dil-\textsuperscript{i}}
{littleness-\poss.1\sg{} \lnk-\poss.1\sg}
{my littleness}
{\textit{Acts of Thomas}, ed.\ \cite[\syr{183}]{WrightActs}}

The DAC  is found also with \isi{adverbial} heads, serving to mark the \secn as definite \citep[371]{MengozziExtended}. \citet[324]{PatElContact} suggests that it spread from nominal \prims to prepositional \prims due to the fact that most of the prepositions in \ili{Semitic} languages are derived from nominal forms. 

\syacex{Preposition}{Noun}{996}
{ܥܲܡܗܹܝܢ ܕܲܒ݂̈ܢܵܬ݂ܵܗ}
{ʿamm-hēn da\cb{} bnāt-ēh}
{with-\poss.\fpl{} \lnk\cb{} daughters-\poss.3\masc{}}
{together with his daughters}
{\cite[88, \S 111e]{MuraokaSyriac}}

\acex{Preposition}{Noun}{1120}
{ʿamm-ēh d\cb{} malkā}
{with-\poss.\masc{} \lnk\cb{} king}
{with the king}
{MengozziExtended}{377}

In such cases, one may question whether the \lnk* can be analysed as standing in \isi{apposition} with the prepositional \prim, as it does with nominal \prims. \citet[372]{MengozziExtended}, drawing a parallel between the cases of nominal \prims and prepositional \prims, suggests that the answer is positive: \blockquote{The construction [with a prepositional primary] is a variant of the genitive phrase with proleptic
pronoun [= DAC with nominal primary]. The determinative pronoun [= \d] functions in [the former case] as a \enquote{pro-preposition}, in that it resumes the head of the \isi{prepositional phrase}, i.e.\ the preposition itself.}

\citet[118]{CohenEzafe}, on the other hand, writing on a similar construction occurring in the \ili{NENA} \JZax dialect, suggests that such an analysis is implausible: \blockquote{In this position, there is no motivation for the pronoun \d to occur, since there is no sense in pronominally representing the preposition (as there is, e.g.\ between two nouns, where \d perfectly represents the first noun).} 

Indeed, Mengozzi's position is somewhat contradictory: A pronoun cannot become a \enquote{pro-preposition} without losing its pronominal status. Thus, his analysis in fact implies that the \d morpheme in this position is no more pronominal, but rather serves as a pure linker connecting the preposition to its complement. An alternative solution reveals itself if we observe carefully the linguistic facts: In Syriac, a \D+Noun combination never occurs directly after a bare preposition, but only after a preposition followed by a \isi{proleptic pronoun}.\footnote{This is also true of \JBA \citep[95]{BarAsherJBA}.} Thus, it seems reasonable to postulate that the \d represents not the proposition but rather the referent introduced by the \isi{proleptic pronoun}. As for the \secn, it could be analysed as a reduced equational \isi{relative clause}, specifying the referent of these pronouns, somewhat similarly to \example{1038}.\footnote{But note that in this example the \secn specifies the nominal \prim, and not the \isi{possessive suffix}.} Thus, \example{1120} should be literally translated as \transl{with him, who is the king}, but, of course, the heavy pragmatic markedness that is associated with such an \ili{English} translation is not present in the Syriac original (except for the marking of the \secn as definite).\footnote{This analysis is in some respects similar to the analysis of \citet{BarAsherAdnominal} of \d followed by nouns (see \vref{fn:syr_BarAsher_theory}), in that both assume that \D+Noun can be interpreted as a clause. However, the type of clause involved, and the scope of this analysis (which is in our case quite limited) marks a clear difference between the two approaches.}

 
As the \d \lnk* is co-referential both with the \isi{possessive suffix} and the \secn, we get schematically a skewed picture of the grammatical relations in this construction, as compared to \vref{tb:DAC}:



\begin{table}[h]
\centering
\begin{tabular}{l l l l}
\toprule
Head 	& Attr. & & \\
\midrule 
ʿamm- 	& -ēh 	& & \\
		& $\updownarrow $ & & \\
		& d- & $\Longleftrightarrow$ 	& malkā	 \\
\midrule
		& Head &				& Attr. \\
\bottomrule
\end{tabular}
\caption[Relations within a DAC headed by a preposition]{Relations within a DAC headed by a preposition}
\end{table}

\subsection[Variants of the DAC (X-\opt{y.{\poss}} \textsc{lnk}-y.\poss\ \textsc{lnk} Y)]{Variants of the double annexation construction (X-\opt{y.{\poss}} \textsc{lnk}-y.\poss\ \textsc{lnk} Y)}

A variant of the DAC is a construction in which the \isi{possessive suffix} is not attached to the \prim noun, but rather to the linker \transc{dil-}, yielding a quite elaborate structure: 

\syacex{Noun}{Noun}{998}
{ܡܫܲܡܫܵܢܹ̈ܐ ܕܝܼܠܵܗ ܕܡܸܠܬ݂ܵܐ}
{mšamsānē dil-āh d\cb{} meltā}
{ministers \lnk-\poss.3\fem{} \lnk\cb{} word(\fem)}
{ministers of the word}
{\Pesh, Luke 1:2; \cite[88, \S 112h]{MuraokaSyriac}}


As the linker is syntactically independent it can precede, together with the \secn, the \prim (compare with \example{988}):

\syacex{Noun}{Noun Phrase}{997}
{ܕܝܼܠܗܿܢ ܕܲܬ݂ܖܸ̈ܥܣܲܪ ܫ̈ܠܝܼܚܹܐ ܫ̈ܡܵܗܹܐ}
{[dil-hon da\cb{} treʿsar šliḥē] šmāhē}
{\lnk-\poss.3\mpl{} \lnk\cb{} twelve apostles names}
{the names of the twelve apostles}
{\Pesh, Matthew 10:2; \cite[88, \S 112h]{MuraokaSyriac}}

\largerpage[-1]
This construction can even appear without a primary (i.e.\ with a \zero\ \prim)  in predicative position. This is the case in the following example, in which the inflected \transc{dil-} \lnk* is separated from the \d \lnk* by an \isi{enclitic} personal pronoun, which serves to mark the AC as being predicative:\footnote{Cf. \citet[33*, fn. 24]{MuraokaSyriac}.}

\syacex{\zero}{Noun}{1160}
{ܘܩܠܐ ܗܢܐ ܕܚܕܘܬܐ ܕܝܠܗ̇ ܗܘ ܕܡܤܬܘܬܐ}
{(w\cb{} qālā hānā d\cb{} ḥadutā) \zero{} dil-āh \cb{}\textsuperscript{h}u d\cb{} meštutā}
{and\cb{} sound \dem.\masc{} \lnk\cb{} joy \zero{} \lnk-\poss.3\fem{} \cb{}3\masc{} \lnk\cb{} feast(\fem)}
{and this sound of rejoicing is that of the wedding-feast}
{\textit{Acts of Thomas}, ed.\ \cite[\syr{174}]{WrightActs}}

A highly elaborate variant  of this construction occurs when the \isi{possessive suffix} is attached both to the \prim noun and to a \lnk*. If not for its rareness, one might term this a \concept{triple \isi{annexation} construction}:

\syacex{Noun}{Noun}{1129}
{ܡܗ̈ܝܡܢܝܗ̇ ܕܝܠܗ̇ ܕܐܣܬܝܪ}
{mhaymanay-āh dil-āh d\cb{} Estēr}
	{eunuchs-\poss.3\fem{} \lnk-\poss.3\fem{} \lnk\cb{} E.}
{the eunuchs of Esther}
{\Pesh, Esther 4:4; \cite[8]{WilliamsKings}}

	
At the other extreme, a variant of the DAC in which the \lnk*
 is completely lacking does not exist in Syriac as such, although it is attested in the contemporary \il{Aramaic!Galilean}{Galilean Aramaic}, a western \il{Aramaic!Classical}Classical Aramaic language \citep[25]{HopkinsName}. Only the word \textsyriac{ܟܘܠ} \foreign{kul}{all}, which could be analysed as standing in \isi{attributive relation} to its complement, shows this syntax regularly:\footnote{It is interesting to note that this peculiar syntax of the word \transc{kul} is conserved in many \ili{NENA} dialects. A survey of  different \Syr constructions involving \transc{kul} can be found in \citet[Ch.\ 3]{WilliamsKings}.  }

\syacex{All}{Noun}{1146}
{ܒܟܠܗܝܢ ܒܪ̈ܝܬܟ}
{b\cb{} kul-hin beryāt-āk}
{in\cb{} all-\poss.3\fpl{} creature.\fpl-\poss.2\masc}
{in all Your creatures}
{\textit{Acts of Thomas}, ed.\ \cite[\syr{180}]{WrightActs}}\antipar

\section{The dative linker construction (X-\opt{y.\poss} \dat\ Y)} \label{ss:dat_lnk}

As an alternative to the usage of the \d pronominal \lnk*, one finds cases where the dative/allative preposition \textsyriac{ܠ}  \foreign{l-}{to} (glossed here \dat) is used \citep[\S 362]{DuvalSyriaque}.\footnote{As for the usage of prepositional linkers, \citet[8]{WilliamsKings} lists also the \enquote{the partitive construction with \textsyriac{ܡܢ} [=\foreign{men}{from}]} as a \enquote{genitive construction}. In this case, however,  the preposition contributes semantically to the partitive reading, and should therefore be analysed as a contentful head of a PP, rather than an AC marker. See \citet[56]{JoostenMatthew} for more examples of the partitive construction.}

\syacex{Noun}{Noun Phrase}{1134}
{ܒܢܶܫ̈ܶܐ ܠܡܰܠܟܳܐ ܕܗܽܘܢܳܝ̈ܶܐ}
{b\cb{} nešē l\cb{} [malkā d\cb{} hunāyē]}
{in\cb{} women \dat\cb{} king \lnk\cb{} Huns}
{amongst the women of the king of Huns}
{\textit{Chronique de Josué le Stylite} ed.\ \cite[18, 1]{Martin}; \cite[342, \S 362]{DuvalSyriaque}}


Such cases are, however, quite rare, and often it is difficult to say whether the preposition is a pure marker of the AC, or rather contributes some semantic content.\footnote{Thus, in the second example cited by \citet[342, \S 362]{DuvalSyriaque}, \textsyriac{ܩܽܘܒܫܳܐ ܠܖ̈ܶܓܠܰܝܟ} \foreign{qubšā l-reglayk}{a footstool for your feet} (\Pesh, Acts 2:35), the \transc{l-} seems to fulfil its ordinary function as a contentful preposition rather than a marker of an AC. This phrase, moreover, is a literal translation of the \BHeb \transc{\texthebrew{הֲדֹם לְרַגְלֶיךָ} hădom lə-raglɛ̄kā} (Psalms 110:1). One may  speculate that such \BHeb constructions may indeed be the source for the Syriac construction. The \BHeb construction is said to be used especially when a definite \secn follows an indefinite \prim \parencites[157]{WaltkeOconnor}[63]{JenniLehrbuch}. Given the rarity of the Syriac construction, it is difficult to tell whether this is true in Syriac as well. As for the colloquial \ili{Arabic} usage of a similar construction, see below.}



The above construction can be seen as a parallel of the ALC, but with a dative linker instead of a pronominal one. Similarly, a rare alternative to the DAC exists as well. In this construction the \secn is indexed by a \isi{possessive pronoun} on the \prim, followed by the preposition \transc{l-}:

\syacex{Noun}{Noun}{1125}
{ܫܡܗ ܠܐܡܗ}
{šm-āh l\cb{} emm-ēh}
{name-\poss.3\fem{} \dat\cb{} mother-\poss.3\masc}
{the name of his mother}
{Matthew 13:55, Curetonian \citep[ed.][]{Burkitt} and Sinaitic \citep[ed.][]{Lewis} manuscripts \apud \cite[56]{JoostenMatthew}}

\syacex{Noun}{Noun}{1164}
{ܐܡܗ̇ ܠܟܠܬܐ}
{ʾemm-āh l\cb{} kaltā}
{mother-\poss.3\fem{} \dat\cb{} bride}
{the mother of the bride}
{\textit{Acts of Thomas}, ed.\ \cite[\syr{182}]{WrightActs}}


This construction is discussed in length 
by \citet{HopkinsName}, who gives credit to \citet[324]{GoldenbergUllendorff} for being the first to note it, as only three examples of it are attested in the standard version of the \Pesh\  (Ruth 1:2; 2:19; Luke 1:27). In all of these cases, as in \example{1125}, the \prim is the noun \textsyriac{ܫܡܐ}  \foreign{šmā}{name}.



The prepositional linker \transc{l-} differs from the \isi{pronominal linker} \d, in that it does not represent a noun. In this sense, it is a truly a pure marker of the AC. In this respect one can cite \citet[254]{PolotskySchneider}, who writes regarding a similar construction in Ge'ez: \blockquote{The
complement introduced by \textit{la-} therefore lacks the ability, which an \isi{apposition} ought to  possess, of leading a separate syntactic existence; and this accounts for the fact the analytical construction really makes the impression of a unified whole, rather than of two separable elements in \isi{apposition}.}

\citet[30ff.]{HopkinsName} suggests that the origin of this construction belongs to a colloquial register of Aramaic, and for this reason it is nearly absent from literary sources. He attributes the existence of a similar construction in the vernacular Eastern \ili{Arabic} dialects to an Aramaic substratum.\footnote{\citet[380]{ErwinIraqi} notes  that in \Iraq this construction (with the \isi{possessive suffix}) has always a definite  \secn. See for instance the following example:
\acexfn[\Iraq]
{Noun}{Noun}{2027}
{ṣadīq-a l\cb{} ʿali}
{friend-\poss.3\masc{} \dat\cb{} A.}
{Ali's friend}
{ErwinIraqi}{380}

} Moreover, he shows that this construction gave rise to the  \enquote{normal possessive construction} of \WNA, in which the \transc{l-} preposition has been encliticized to the head, yielding an \transc{-il} suffix functioning as a \cst* marker (see \example{1867}), parallel to the \ili{NENA} \ed suffix (see \sref{ss:d_vs_ed}). 


\section{Adjectival attribution by apposition (X Y.\agr)} \label{ss:syr_adj}

\subsection{The juxtaposition-cum-agreement construction}

\citet[8]{GoldenbergAttribution} qualifies adjectives as follows: \blockquote{If we admit that
adjectives have to do both with the carrier of the quality etc. and with the attributed
quality itself, then the form \enquote{adjective} is recognized as an attributive complex
with pronominal reference and attribute as distinguishable components, the former
represented by the inflectional markers and the latter given in the lexeme involved.
The implied \isi{attributive relation} marks the adjective as the morphological exponent
of that relation, and consequently as the morphological correlate of the genitive
complex.}


Goldenberg's \enquote{pronominal reference} is our \isi{pronominal linker}, while his \enquote{attribute} is our \secn. Thus, using the current terminology, he equates an adjective to a \lnk+\secn phrase. Indeed, just as the \lnk* stands in \isi{apposition} with the \prim to be qualified, the adjective stands in \isi{apposition} with the noun it qualifies, giving raise effectively to a \isi{juxtaposition-cum-agreement} construction. Nonetheless, for simplicity I shall refer to adjectives as being simple \secns.\footnote{While Goldenberg's conception of the adjective is appealing in its structural elegance, it does not provide us with any operational criterion to distinguish between adjectives and inflecting nouns (e.g.\ those which designate animate beings). The key difficulty lies in the fact that are no clear criteria to demarcate  words which designate a \enquote{carrier of quality} (=adjectives) from those which designate directly a \enquote{substance} or \enquote{entity} (=nouns). For example, if the \ili{Hebrew} adjective \texthebrew{שָׂב} \foreign{śāb}{aged man} can be analysed as  \texthebrew{אִישׁ שֵׂיבָה} \foreign{ʾīš śebā}{a man of old age} \citep[9]{GoldenbergAttribution}, shouldn't also the noun \texthebrew{יֶלֶד} \foreign{yeled}{child} be analysable along similar lines as \texthebrew{אִישׁ יַלְדוּת} \foreign{ʾīš yaldūt}{a man of childhood}? Note also that \transc{yeled}, denoting an animate noun, can inflect for gender and number just as the adjective \transc{śāb}. Indeed, the apparent difference between the two lies not in their structure but rather in their distribution, as \transc{yeled} is rarely used as a modifier of another noun.} 

The equivalence between the adjective and the \lnk* phrase is especially clear in the case of ordinal \secns. These can be realised using the \lnk* construction (see \ref{ss:syr_lnk_ord} and especially \example{1048}), or by adjectival derivation of the ordinal:

\syacex{Noun}{Ordinal}{1047}
{ܝܘܡܐ ܬܪܱܝܳܢܳܐ}
{yawmā trayānā}
{day(\masc) second.\masc}
{the second day}
{\cite[178, \S 239]{NoldekeSyriac}}

The equivalence is schematized in \vref{tb:second_day}, to be contrasted with \vref{tb:syr_ALC} \citep[cf.][236]{GoldenbergSemitic}.

\begin{table}[h!]
\centering
\begin{tabular}{l l l}
\toprule
Apposition & Head & Attr. \\
\midrule
yawmā & da- & trēn \\
yawmā & \multicolumn{2}{c}{tarayānā} \\
\midrule 
Primary & \multicolumn{2}{c}{Secondary} \\
\bottomrule
\end{tabular}
\caption[Adjectival attribution according to Goldenberg]{Adjectival attribution according to Goldenberg, contrasted with the terminology used in this study} \label{tb:second_day}
\end{table}


Notwithstanding the constitutional equivalence between adjectives and the \lnk* phrase, in Syriac, as in other \ili{Semitic} languages, there is a morphological difference between the two: while  the \d \lnk*  is an uninflecting particle, the \enquote{pronominal reference} within the adjective is made overt by the very inflecting character of the adjective, which agrees with its \prim in gender, number and determination.\footnote{In Syriac, the agreement in determination is apparent by the agreement in state (absolute or emphatic). This is also true in principal in \JBA, although some examples seem to indicate that attributive adjectives are always in \emp*, irrespective of the state of the \prim \citep[64]{BarAsherJBA}.}

\syacex{Noun}{Adjective}{1121}
{ܡܲܠܟ݁ܵܐ ܛܵܒ݂ܵܐ ܇ ܡܲܠ̈ܟ݁ܵܬ݂ܵܐ ܛܵܒ݂ܵܬ݂ܵܐ}
{malk-ā ṭāb-ā \linebreak malk-ātā ṭāb-ātā}
{monarch-\masc.\emp{} good-\masc.\emp{} \linebreak monarch-\fpl.\emp{} good-\fpl.\emp}{a good king / good queens}
{\cite[72, \S 92.1]{MuraokaSyriac}}


On the other hand, adjectives and \lnk* phrases show many similar syntactic properties. Just as the \lnk* phrase can stand alone without any explicit \prim, so too can an adjective be used independently without any antecedent, as in the following example (the adjectives are marked as \textbf{bold}):

\syacex{\zero}{Adjective}{1122}
{ܗܲܘ ܕܡܲܕ݂ܢܲܚ ܫܸܡܫܹܗ ܥܲܠ ܛܵܒ݂ܹ̈ܐ ܘܥܲܠ ܒܝܼܫܹ̈ܐ}
{haw d\cb{} madnaḥ šemš-ēh ʿal \textbf{ṭāb-ē} w\cb{} ʿal \textbf{biš-ē}}
{\dem.\masc{} \lnk\cb{} rise.\caus.\ptcp.\masc{} sun-\poss.3\masc{} on {good-\mpl.\emp}{} and\cb{} on {bad.\mpl{}.\emp}}
{He who makes his sun rise above the good ones and the evil ones}
{\Pesh, Matthew 5:45; \cite[76, \S 96d]{MuraokaSyriac}}

Similarly, the adjective can sometimes precede its \prim, just as the \lnk* can (see \example{997}):

\syacex{Noun}{Adjective}{981}
{ܩܲܕ݂ܡܵܝܬ݁ܵܐ ܫܸܬܸܐܣܬ݂ܵܐ}
{qadmāytā šetestā}
{first.\fem{} foundation(\fem)}
{the first foundation}
{\cite[69, \S 91a]{MuraokaSyriac}}


Finally, in parallel to cases where the ALC has a demonstrative preceding the \lnk* phrase, thus rendering it definite (see \sref{ss:syr_corr}), the same pattern occurs with adjectival \secns, as \citet[66--67]{PatElCorrelative} notes. 

\syacex{Noun}{Conjoined adjectives}{2029}
{ܠܨܒܝܢܐ ܓܝܪ ܗ̇ܘ ܪܒܐ ܘܩܕܝܫܐ}
{l\cb{} ṣebyānā \cb{}gēr haw [rabbā w\cb{} qaddišā]}
{to\cb{} will(\masc) \cb{}indeed \dem.\masc{} big.\masc{} and\cb{} holy.\masc{}}
{to that great and holy will}
{Bardaisan, \textit{Book of the Laws of the Countries} ed. \cite[62:2--3]{Drijvers} \apud \cite[137]{BakkerBardaisan}; \cite[67]{PatElCorrelative}}

In spite of all these similarities, it is worth noting that the adjective itself can appear as part of a \lnk* phrase, i.e.\ as the \secn of the ALC. The conditions governing this usage are briefly discussed in the next section.

\subsection{Juxtaposition vs.\ the analytic linker construction} \label{ss:syr_adj_vs}

The usage of the \isi{juxtaposition-cum-agreement} construction (in either order), as well as the independent usage of adjectives, should be regarded as the default AC for adjectival \secns. But, as we saw in \sref{ss:syr_adjSecn}, adjectives can also appear as \secns of the ALC, with or without overt primaries. For example, \citet[211]{PeursenBenSira} states that \textsyriac{ܚܟܝܡܐ} \foreign{hakkimā}{wise.\emp} alternates freely with \textsyriac{ܕܚܟܝܡ} \foreign{d\cb{}hakkim}{\lnk\cb{}wise.\abs}, both corresponding to \ili{Hebrew} \texthebrew{חכם} \foreign{ḥaḵam}{a wise person}. 

\citet[212]{PeursenBenSira} also notes that it is difficult to establish a \enquote{functional difference} between the two constructions, but rather the \lnk* construction is more frequent in certain contexts. In my interpretation, these contexts are (a) the occurrence of a (multi-word) AdjP as \secn or (b) the occurrence of a pronominal \prim, \zero\ included.\footnote{If no explicitly primary appears, the \lnk* fully assumes its pronominal role.} 

The usage of the \lnk* construction in cases like (a) may be motivated by the speaker's desire to delineate the phrasal nature of the \secn, and thus avoid any ambiguity as for the scope of modifiers of the adjective. The motivation for (b) may lie in her desire to clearly express the \isi{pronominal head} extant in the adjective. This is achieved by attaching to the adjective an explicit \isi{pronominal head}, namely the \lnk*.\footnote{One may tentatively analyse an adjective appearing in \abs* as expressing the adjectival lexeme alone without a \isi{pronominal head}, in contrast to the \emp* of the adjective, which is \enquote{the formal expression of its structure as a nomen adiectivum, which includes an inherent pronominal reference to the qualified substantival entity} \citep[718]{GoldenbergPredicative}. If this is true, the \lnk* is effectively an extraction of the \isi{pronominal head} from the adjective. \label{ft:syr_abs_adj}}


\section{Conclusions} \label{ss:syr_conclusions}
\largerpage
This chapter gave a survey of the various ACs of Syriac. Three main construction are used in this domain: the \isi{construct state} construction, the \isi{analytic linker construction} and the \isi{double annexation} construction. The former is the least productive of the three, being used mostly with fixed expressions or specific \prims, while the latter two are used more frequently. The alternation between these two seems to be chiefly related to questions of determination. The latter two constitute the source for the Neo-CSC present in \ili{NENA} dialects, as is discussed in \sref{ss:NeoCSC_Origin}.

Some marginal attributive constructions of Syriac are variants of the DAC as well as the dative linker construction. The latter may be the source of Neo-CSC in \WNA. 

Beyond these constructions one finds the \isi{juxtaposition-cum-agreement} construction used with attributive adjectives, a construction which is extant in all strata of Aramaic. 
