%\documentclass[output=paper]{LSP/langsci} 
\addchap{Preface to the new edition}
%\author{Oliver Čulo
% \and Silvia Hansen-Schirra\affiliation{University of %Mainz, Germersheim} 
% \lastand Stella Neumann\affiliation{RWTH Aachen 
%University}
%}
%\title{Preface to the new edition} 
%\abstract{Goodbye TC3, welcome TMNLP! A welcome address from the previous TC3 editors.}
%\maketitle

%\begin{refsection}

%\section{Introduction} 

\lehead{Oliver Czulo, Silvia Hansen-Schirra, Stella Neumann}

This volume is a re-issue of the second of three volumes made up of previous issues of the open access journal "Translation: Computation, Corpora, Cognition" (TC3) which was transformed into the book series "Translation and Multilingual Natural Language Processing" (TMNLP) at LangSci Press. The underlying TC3 issue focused on the potential of exchange between the three fields Contrastive Linguistics, Translation Studies and Machine Translation. Today, we can look back and ask what has changed in the last five years. 

The corpus paradigm proved a way to establish an intensified exchange between Translation Studies and Contrastive Linguistics, both in terms of methodology and theory. It seems, however, that the flow of information is stronger in one direction: translation scholars have adapted and evolved approaches from Contrastive Linguistics, yet our observation is that this has happened to a lesser extent in the opposite direction.

In Contrastive Linguistics, there has been a growing interest in the analysis of translations from various viewpoints, though apparently not so much in translation theory itself. Among those who integrate translations in their analysis of language contrasts is \citet{Gast2015} who makes use of the Europarl corpus to study impersonalisation strategies in English and to contrast those to strategies found in the German data. \citet{Levshina2017} investigates the use of \textit{t/v} forms (formal vs. informal forms of addressing) in various languages by means of a parallel corpus of subtitles. While the design of these studies may be carefully crafted around the specificities of translated language, one should not forget the methodological drawbacks of using parallel data for contrastive studies: Translated language is influenced by the translation process and potentially by the source language. If not analysed with this in mind, translation data can be misleading, as \citet{Neumann_HansenSchirra2013} show in their discussion of the differences between the translational and the contrastive perspective.

For Machine Translation and Translation Studies, the topic of post-editing was and is a point of convergence. After all, this is the point at which humans and machines meet. Typically, the topic of machine translation is addressed from angles such as quality issues, evaluation of machine translation or the impact of machine translation and post-editing on translations and translators. These topics are much discussed - and rightly so! - in the field of post-editing. However, little work has been done so far on how to model machine translation in terms of translation-theoretic models. Exceptions are, for instance, \citet{Rozmyslowicz2014} who discusses whether translated texts produced by a machine can be counted as translations even though functional translation theory requires agency for the production of a translation, or \citet{Lapshinova2013} who treats machine-translated texts as a text type in its own right and analyses them for their linguistic variation. While the latter is not actually a contribution in terms of theoretical modelling of machine translation, it does represent an interesting shift in perspective on how machine-translated texts are treated. \citet{Culo2014} makes an attempt at classifying machine translation from the viewpoint of functional theories, albeit on a rather abstract level.
\par

This re-issue contains two new paradigms besides the corpus paradigm which have gained ground in all of the fields addressed here: the cognitive and process-based research paradigm, which interact heavily. With techniques such as eye tracking and key logging, we gain insight into micro and macro processes in various types of text processing, including post-editing, which also brings Machine Translation into focus \cite[see e.g. Carl \& Dragsted, this volume;][]{CarlEtAl2016}. We hope that this re-edition will give another impulse for connecting Translation Studies with its neighbouring fields, potentially discovering new (common?) grounds.

\bigskip
\hfill Leipzig, Germersheim and Aachen, September 2017 \\
\bigskip
\hfill Oliver Czulo, Silvia Hansen-Schirra, Stella Neumann
\sloppy
\printbibliography[heading=subbibliography,notkeyword=this]

%\end{refsection}