\chapter{Preface}

This book is an extended and revised version of my German book \emph{Grammatiktheorie}
\citep{MuellerGTBuch2}. It introduces various grammatical theories that play a role in current
theorizing or have made contributions in the past which are still relevant today. I explain some foundational
assumptions and then apply the respective theories to what can be called the ``core grammar'' of
German. I have decided to stick to the object language that I used in the German version of this
book since many of the phenomena that will be dealt with cannot be explained with English as the object
language. Furthermore, many theories have been developed by researchers with English as their native
language and it is illuminative to see these theories applied to another language.
I show how the theories under consideration deal with arguments and adjuncts, active/passive
alternations, local reorderings (so"=called scrambling), verb position, and fronting of phrases over
larger distances (the verb second property of the Germanic languages without English).

The second part deals with foundational questions that are important for developing theories.
This includes a discussion of the question of whether we have innate domain specific knowledge of
language (UG), the discussion of psycholinguistic evidence concerning the processing of language by
humans, a discussion of the status of empty elements and of the question whether we construct and perceive utterances 
holistically or rather compositionally, that is, whether we use phrasal or lexical
constructions. The second part is not intended as a standalone book although the printed version of
the book is distributed this way for technical reasons (see below). Rather it contains topics that are discussed again and again when frameworks are
compared. So instead of attaching these discussions to the individual chapters they are organized in
a separate part of the book.

Unfortunately, linguistics is a scientific field 
with a considerable amount of terminological chaos. I therefore wrote an introductory
chapter that introduces terminology in the way it is used later on in the book. The second chapter
introduces phrase structure grammars, which plays a role for many of the theories that are covered
in this book. I use these two chapters (excluding the Section~\ref{sec-PSG-Semantik} on interleaving
phrase structure grammars and semantics) in introductory courses of our BA curriculum for German
studies. Advanced readers may skip these introductory chapters. The following chapters are
structured in a way that should make it possible to understand the introduction of the theories
without any prior knowledge. The sections regarding new developments and classification are more
ambitious: they refer to chapters still to come and also point to other publications that are
relevant in the current theoretical discussion but cannot be repeated or summarized in this
book. These parts of the book address advanced students and researchers. I use this book for teaching
the syntactic aspects of the theories in a seminar for advanced students in our BA. The slides are
available on my web page. The second part of the book, the general discussion, is more ambitious and contains the discussion
of advanced topics and current research literature.

This book only deals with relatively recent developments. For a historical overview, see for instance
\citew{Robins97a-u,JL2006a-u}. I am aware of the fact that chapters on
Integrational Linguistics\is{Integrational Linguistics}
\citep{Lieb83a-u,Eisenberg2004a,Nolda2007a-u}, Optimality Theory\indexot (\citealp{PS93a-u};
\citealp{Grimshaw97a-u}; G.\ \citealp{GMueller2000a-u}), Role and Reference Grammar\is{Role and
  Reference Grammar} \citep{vanValin93a-ed} and Relational Grammar\is{Relational Grammar}
\citep{Perlmutter83a-ed,PR84a-ed} are missing. I will leave these theories for later editions.

The original German book was planned to have 400 pages, but it finally was much bigger: the first
German edition has 525 pages and the second German edition has 564 pages. I
added a chapter on Dependency Grammar and one on Minimalism to the English version and now the
book has \pageref{LastPage} pages. I tried to represent the chosen theories appropriately and to cite all important work. Although the list of
references is over 85 pages long, I was probably not successful.
I apologize for this and any other shortcomings.

%%% -*- coding:utf-8 -*-

\section*{Available versions of this book}

The canonical version of this book is the PDF document available from the Language Science Press
webpage of this book\footnote{%
\url{\lsURL}
}. This page also links to a Print on Demand version. Since the book is very long, we decided to
split the book into two volumes. The first volume contains the description of all theories and the
second volume contains the general discussion. Both volumes contain the complete list of references
and the indices. The second volume starts with page~\pageref{part-discussion}. The printed volumes
are therefore identical to the parts of the PDF document.







%      <!-- Local IspellDict: en_US-w_accents -->


\addchap{Acknowledgments} 
%content goes here
The help and support of Martin Haspelmath and Sebastian Nordhoff in the preparation of this volume is gratefully acknowledged. 

We would also like to thank the authors of the chapters in this volume for their cooperation during the editing process and especially for their input to the reviewing of chapters by their peers. 

We especially thank the following additional external reviewers, %individuals, 
who contributed their time and expertise to provide independent peer review for the papers in this collection: Lisa Bonnici, Jason Brown, Elisabet Engdahl, Marieke Hoetjes, Beth Hume, Anne O'Keefe, Adam Schembri, Thomas Stolz, Andy Wedel and Shuly Wintner.
 

\section*{本书的出版过程}
%\section*{On the way this book is published}

我从1994年开始写我的毕业论文,并在1997年成功通过答辩。这一阶段的手稿可以在我的网页上获取。在答辩之后,我必须要找到出版商。我很高兴收到了Niemeyer的“语言学研究”系列丛书的邀请,但是同时我对价格感到震惊不已,当时每本书需要186德国马克,这还是在我没有出版商的任何帮助的情况下,自己写书和排版的价格(这个价格是纸版小说的二十倍)。\dotfootnote{%
与此同时,Niemeyer被de Gruyter收购,并停止营业了。这本书的价格现在是139.95欧元 / 196.00美元。欧元的价格相当于273.72德国马克。
} 这基本上意味着我的书是没有出版的:直到1998年,才能在我的网站上看到这本书,并随后在图书馆可以查询到。我的教授转正著作由CSLI出版社出版,价格相对来说合理多了。在我开始写教科书的时候,我就寻找不同的出版渠道,并跟无名印刷需求的出版社协商。Brigitte Narr负责管理Stauffenburg出版集团,她说服我在他们的出版社出版HPSG的教材。这本书的德语版属于我,这样我就可以在我的主页上出版。这一合作是成功的,由此我还可以跟Stauffenburg出版我的第二本关于语法理论的教科书。我想这本书具有更为广泛的相关性,并且可以供非德语的读者阅读。由此,我决定将它翻译为英语。不过,Stauffenburg重点出版德语书籍,我必须找到另一家出版社。幸运的是,出版界的情况与1997年相比发生了戏剧性的翻天覆地的变化:我们现在有高水平的出版社,不仅有严格的同行评审,还有着完全公开的途径。我很高兴Brigitte Narr将本书版权卖回给我,我现在就可以在CC-BY版权下由语言科学出版社出版这本英文版教材了。
%I started to work on my dissertation in 1994 and defended it in 1997. During the whole time the
%manuscript was available on my web page. After the defense, I had to look for a publisher. I was
%quite happy to be accepted to the series \emph{Linguistische Arbeiten} by Niemeyer, but at the same time I
%was shocked about the price, which was 186.00 DM for a paperback book that was written and typeset
%by me without any help by the publisher (twenty times the price of a paperback novel).\footnote{%
 % As a side remark: in the meantime Niemeyer was bought by de Gruyter and closed down. The price of the book is now
 % 139.95 \euro / \$ 196.00. The price in Euro corresponds to 273.72 DM. 
%%This is a price increase of 47\,\%.
%} This
%basically meant that my book was depublished: until 1998 it was available from my web page and after
%%this it was available in libraries only. My Habilitationsschrift was published by CSLI Publications
%for a much more reasonable price. When I started writing textbooks, I was looking for alternative
%distribution channels and started to negotiate with no-name print on demand publishers. Brigitte Narr,
%who runs the Stauffenburg publishing house, convinced me to publish my HPSG textbook with her. The
%\textsc{cop}yrights for the German version of the book remained with me so that I could publish it on my web page. The collaboration was successful so that I also published my second textbook about
%grammatical theory with Stauffenburg. I think that this book has a broader relevance and should be
%accessible for non-German-speaking readers as well. I therefore decided to have it translated into
%English. Since Stauffenburg is focused on books in German, I had to look for another publisher. Fortunately the situation in the publishing sector changed quite dramatically in comparison
%to 1997: we now have high profile publishers with strict peer review that are entirely open access. I am very
%glad about the fact that Brigitte Narr sold the rights of my book back to me and that I can now 
%publish the English version with Language Science Press under a CC-BY license.

%      <!-- Local IspellDict: en_US-w_accents -->

\section*{Language Science Press: scholar-owned high quality linguistic books}

In 2012 a group of people found the situation in the publishing business so unbearable that they
agreed that it would be worthwhile to start a bigger initiative for publishing linguistics books in
platinum open access, that is, free for both readers and authors. I set up a web page and collected
supporters, very prominent linguists from all over the world and all subdisciplines and Martin
Haspelmath and I then founded Language Science Press. At about the same time the DFG had announced
a program for open access monographs and we applied \citep{MH2013a} and got funded (two out of 18 applications got
funding). The money was used for a coordinator (Dr.\ Sebastian Nordhoff) and an economist (Debora
Siller), two programmers (Carola Fanselow and Dr.\ Mathias Schenner), who worked on the publishing
plattform Open Monograph Press (OMP) and on conversion software that produces various formats (ePub, XML,
HTML) from our \LaTeX{} code. Svantje Lilienthal worked on the documentation of OMP, produced
screencasts and did user support for authors, readers and series editors.

OMP was extended by open review facilities and community-building gamification tools
\citep{MuellerOA,MH2013a}. All Language Science Press books are reviewed by at least two external
reviewers. Reviewers and authors may agree to publish these reviews and thereby make the whole
process more transparent (see also \citew{Pullum84a} for the suggestion of open reviewing of journal
articles). In addition there is an optional second review phase: the open
review (see the blog posts by Sebastian Nordhoff about the reviewing options at Language Science
Press\footnote{%
\url{https://userblogs.fu-berlin.de/langsci-press/2015/05/27/axes-of-open-review/}, 2020-09-03.
}). This second optional reviewing phase is completely open to everybody. The whole community may comment on the document
that is published by Language Science Press. After this second review phase, which usually lasts for
two months, authors may revise their publication and an improved version will be published. The
English version of this book was the first book to go through this open review phase. The Chinese
translation was also open for comments on Paperhive. Readers left more than 2500 comments\footnote{%
\url{https://paperhive.org/documents/items/Zf2Qf47i6nf2}, 2020-09-03.}, which were automatically fed into the version control
and bug tracking system used by Language Science Press\footnote{%
\url{https://github.com/langsci/177/}, 2020-09-03.
}.

Currently, Language Science Press has 26 series on various subfields of linguistics with high
profile series editors from all continents. There are 437 members in the respective editorial boards
coming from 49 countries. We have 134 published books with more than 1 Mio downloads.\footnote{%
Downloads by robots excluded, the English version of this textbook was downloaded over 40,000 times
since 2016.
} 1196 authors from 53 countries have published books or chapters with Language Science Press as of March 2020 and there are
572 expressions of interest. 
%Two multi-volume handbooks, one on HPSG and one on LFG, are in
%preparation \citep{HPSGHandbook,LFGhandbook}.


Series editors are responsible for delivering manuscripts that are typeset in \LaTeX{}, but they are
supported by a web-based typesetting infrastructure that was set up by Language Science Press and
there is also conversion software converting Word manuscripts into \LaTeX{}. Proofreading is
community-based. Until now 224 people helped improve our books. Their work is documented in the
Hall of Fame: \url{http://langsci-press.org/hallOfFame}.


Language Science Press is a community"=based publisher, but apart from the press managers Martin
Haspelmath and me, there are two people who are employed for the central organization and
typesetting: Sebastian Nordhoff, who is also a press manager, and Felix Kopecky, who does 
typesetting. Both have 50\,\% positions. In the period of 2018--2020, these two positions got payed
with the help of financial support by 115 academic institutions including  
Harvard, the MIT, and Berkeley and by societies like EuroSLA.\footnote{%
  A full list of supporting institutions is available at:
  \url{http://langsci-press.org/knowledgeunlatched}.
} The Language Science Press approach is endorsed by the leading scholars Noam Chomsky, Adele
Goldberg, and Steven Pinker, who sent letters of support in 2017.\footnote{%
``Very pleased to learn about this fine initiative, a most valuable way to
bring to the general public the results of scholarly work.  It's a
cliché, but true, that we all stand on the shoulders of giants, and rely
on the cultural wealth provided to everyone by past generations.  It is
only proper that the public should gain access to whatever contemporary
scholarship can contribute, and the ideas outlined here seem to be a
very promising way to realize this ideal.'' Noam Chomsky, 2017-02-01.
 
``Language Science Press is setting a standard for freely accessible
articles and books that are carefully reviewed.'' Adele Goldberg, 2017-05-02. 

``Sharing data and methods is one of the pillars of scholarly inquiry. The knowledge created by
scholars belongs to everyone, and open access publications are a major pathway to realizing that
ideal. Language Science Press, together with Knowledge Unlatched, provides an excellent way for us
to make our findings available to the global public.'' Steven Pinker, 2017-01-22. 
} The fundraising for the period 2021--2023 is ongoing.

If you think that textbooks like this one should be freely available to whoever wants to read them
and that publishing scientific results should not be left to profit-oriented publishers, then you
can join the Language Science Press community and support us in various ways: you can register with Language Science Press and have your name
listed on our supporter page with more than 1000 other enthusiasts, you may devote your time and help
with proofreading. We are also looking for institutional supporters like foundations,
societies, linguistics departments or university libraries. Detailed information on how to support
us is provided at the following webpage: \url{http://langsci-press.org/supportUs}.
In case of questions, please contact me or the Language Science Press coordinator at \href{mailto:contact@langsci-press.org}{contact@langsci-press.org}.


~\medskip

\noindent
Berlin, September 04, 2020\hfill Stefan Müller


%      <!-- Local IspellDict: en_US-w_accents -->


%% -*- coding:utf-8 -*-
\section*{Foreword of the second edition}

The second edition comes with a lot of small improvements: the index has been improved, typos have
been fixed, and ORCIDs were added to authors and are displayed on the title pages of the papers now.

% order: 04.01.22
% Gray -> gray
% der Frau -> dem Kind
% added \ref{ex-schema-hc-flat-synsem-sign}

% complex-predicates 04.01.22
% unified synsems2signs. The relation has the same name now in order.tex and complex-predicates.tex

% relative-clauses.tex 05.01.22
% \trace -> \trace{}
% glosses aligned in {x:rc-129}
% added language tag
% fixed index entry for Bavarian German


% 18.01.22 added language info for German examples

% 25.01.22 Footnote~\ref{fn-hf-schema} was missing. % in udc

% 03.02.22 Idioms: NP in (8) too much, REL bad feature name, ref to Krenn&Erbach added

% 08.02.22 Information structure: added page numbers for Bildhauer & Cook 2010
%          fixed layout issue with Head-Dislocation Schema for Catalan
% 09.02.22 Added sentence about diff-list and reference to copestake2002.
%
% 14.02.22 Added glosses to helfen in chapter on processing
%
% 30.03.22 Figure 4, Mary is NP not N
%
% 26.10.22 (38) used to be phrase => but since the constraint referred to HD-DTR this would cause a
% conflict for unheaded phrases. Noticed by student.
% The left-hand daughter in (38b) must be SYNSEM X, noted by St.Mü.


~\medskip

\noindent
Berlin, \today\hfill Stefan Müller, Anne Abeillé, Robert D. Borsley \& Jean-​Pierre Koenig


%      <!-- Local IspellDict: en_US-w_accents -->

%% -*- coding:utf-8 -*-

\section*{Foreword of the third edition}

% fixed \forwardt for Harry in 7-cg.tex Ina Baier 18.06.2018

% added mention of Rizzi2014a 09.07.2018

% fixed 0,045 which should have been 0.045 in innateness chapter.

% added reference/source for she smiled herself an upgrade.

% added reference to Chesi2015a because of infinite sentences

% changed the discussion of passive in German in GB a bit to make things clearer, 10.10.2018

% added references to GSag2000a and NK2019a in the discussion of start symbol and utterance
% fragments 18.01.2019

% fixed Baumgärtner's dicussion of rule schema rather than rule 09.02.2019

% fixed missing italics in Figure for Max sleeps.

% fixed ungrammatical Chinese example in phrasal.tex 30.07.2019, Thanks Wang Lulu 

Since more and more researchers and students are using the book now, I get feedback that helps
improve it. For the third edition I added references, expanded the discussion of the passive in GB (Section~\ref{sec-passive-gb})
a bit and fixed typos.\footnote{%
  A detailed list of issues and fixes can be found in the GitHub repository of this book at
  \url{https://github.com/langsci/25/}.%
}

Chapter~\ref{chap-mp} contained figures from different chapters of \citew{Adger2003a}. Adger
introduces the DP rather late in the book and I had a mix of NPs and DPs in figures. I fixed this in
the new edition. I am so used to talking about NPs that there were references to NP in the general
discussion that should have been references to DP. I fixed this as well. I added a figure explaining
the architecture in the Phase model of Minimalism and since the figures mention the concept of
\emph{numeration}, I added a footnote on numerations. I also added a figure depicting the
architecture assumed in Minimalist theories with Phases (right figure in Figure~\ref{fig-architecture-minimalism}).

I thank Frank Van Eynde for pointing out eight typos in his review of the first edition. They have
been fixed. He also pointed out that the placement of \argst in the feature geometry of signs in
HPSG did not correspond to \citew{GSag2000a-u}, where \argst is on the top level rather than under
\cat. Note that earlier versions of this book had \argst under \cat and there had never been proper
arguments for why it should not be there, which is why many practitioners of HPSG have kept it in
that position \citep{MuellerLFGphrasal}. One reason to keep \argst on the top level is that \argst is appropriate
for lexemes only. If \argst is on the sign level, this can be represented in the type hierarchy:
lexemes and words have an \argst feature, phrases do not. If \argst is on the \cat level, one would
have to distinguish between \catvs that belong to lexemes and words on the one hand and phrasal
\catvs on the other hand, which would require two additional subtypes of the type \type{cat}. 
The most recent version of the computer implementation done in Stanford by Dan Flickinger has \argst
under \local (2019-01-24). So, I was tempted to leave everything as it was in the second edition of
the book. However, there is a real argument for not having \argst under \cat. \cat is assumed to be
shared in coordinations and \cat contains valence features for subjects and complements. The values of
these valence features are determined by a mapping from \argst. In some analyses, extracted elements
are not mapped to the valence features and the same is sometimes assumed for omitted elements. To
take an example consider (\mex{1}):
\ea
He saw and helped the hikers.
\z
\emph{saw} and \emph{helped} are coordinated and the members in the valence lists have to be
compatible. Now if one coordinates a ditransitive verb with one omitted argument with a strictly
transitive verb, this would work under the assumption that the omitted argument is not part of the
valence representation. But if \argst is part of \cat, coordination would be made impossible since a
three-place argument structure list would be incompatible with a two-place list. Hence I decided to
change this in the third edition and represent \argst outside of \cat from now on.

I changed the section about Sign-Based Construction Grammar (SBCG) again. An argument about nonlocal
dependencies and locality was not correct, since \citet[\page 166]{Sag2012a} does not share all
information between filler and extraction side. The argument is now revised and presented as
Section~\ref{sec-local-feature-sbcg}. Reviewing \citew{MuellerCxG}, Bob Borsley pointed out to me that the \xargf is a way to
circumvent locality restrictions that is actually used in SBCG. I added a footnote to the section on
locality in SBCG.

A brief discussion of \citegen{Welke2019a-u} analysis of the German clause structure was added to the
chapter about Construction Grammar (see Section~\ref{sec-verb-position-cxg}).

The analysis of a verb-second sentence in LFG is now part of the LFG chapter
(Figure~\ref{Abbildung-V2-LFG} on page~\pageref{Abbildung-V2-LFG}) and not just an
exercise in the appendix. A new exercise was designed instead of the old one and the old one was
integrated into the main text.

I added a brief discussion of \citegen{Osborne2019a} claim that Dependency Grammars are simpler than
phrase structure grammars (p.\,\pageref{page-simplicity-dg}).

Geoffrey Pullum pointed out at the HPSG conference in 2019 that the label \emph{constraint"=based}
may not be the best for the theories that are usually referred to with it. Changing the term in
this work would require to change the title of the book. The label \emph{model theoretic} may be
more appropriate but some implementational work in HPSG and LFG not considering models may find the
term inappropriate. I hence decided to stick to the established term.

I followed the advice by Lisbeth Augustinus and added a preface to Part~II of the book that gives
the reader some orientation as to what to expect.

I thank Mikhail Knyazev for pointing out to me that the treatment of V to I to C movement in the
German literature differs from the lowering that is assumed for English and that some further
references are needed in the chapter on Government \& Binding. 

Working on the Chinese translation of this book, Wang Lulu pointed out some
typos and a wrong example sentence in Chinese. Thanks for these comments! 

I thank Bob Borsley, Gisbert Fanselow, Hubert Haider and Pavel Logacev for discussion and Ina Baier for a mistake
in a CG proof and Jonas Benn for pointing out some typos to me. Thanks to Tabea Reiner for a comment
on gradedness. Thanks also to Antonio Machicao y Priemer for yet another set of comments on the
second edition and to Elizabeth Pankratz for proofreading parts of what I changed.

~\medskip

\noindent
Berlin, 15th August 2019\hfill Stefan Müller



%      <!-- Local IspellDict: en_US-w_accents -->

%% -*- coding:utf-8 -*-

\section*{Foreword of the fourth edition}



% reference to Sag2020

% We thank Shalom Lappin and Richard Sproat for discussion of implementation issues.

% added footnote to gb chapter regarding the assignment of semantic role accross phrase boundary

% Thanks Andreas Pankau

% trincated English -> truncated English 19.01.2020

% fixed URLs, added DOIs

% e or t for trace -> t for trace

% be- und ent-laden sind keine Partikel sondern Präfixe
% Namen ersetzt und Frauen durch Eichhörnchen

% added references for DP approach (Brame, Hudson)

% Due to the work on Chinese, some index entries were fixed (Phenomenon/phenomenon)

I fixed several typos, added and updated URLs and DOIs in the book and in the list of references.
I added a footnote to Chapter~\ref{chap-GB} concerning the assignment of semantic roles
across phrase boundaries (footnote~\ref{fn-semantic-role-phrase-boundary} on
p.\,\pageref{fn-semantic-role-phrase-boundary}). I thank Andreas Pankau for discussion on this point.

I added a paragraph discussing John Torr's implementational work (pages~\pageref{page-torr-implementation-beginning}--\pageref{page-torr-implementation-end}). I thank Shalom Lappin and Richard Sproat for discussion of implementation issues.

A small paragraph for further reading was added to Chapter~\ref{chap-phrasal} on phrasal vs.\ lexical analyses.

Language Science Press will publish a handbook on Head-Driven Phrase Structure Grammar hopefully
later this year \citep*{HPSGHandbook}. It contains several chapters comparing other syntactic
theories to HPSG. I added the respective references to the further readings sections of the chapters
for Lexical Functional Grammar, Categorial Grammar, Construction Grammar, and Minimalism.

This edition is the first edition that uses precompiled trees. Setting this up was not
straightforward. I am really grateful to Sašo Živanović for helping me and adapting the
\texttt{forest} package so that everything runs smoothly and efficiently. This saves me a lot of
time and reduces the energy consumption of my computer dramatically.


~\medskip

\noindent
Berlin, 2nd September 2020\hfill Stefan Müller



%      <!-- Local IspellDict: en_US-w_accents -->

%% -*- coding:utf-8 -*-

\section*{Foreword of the fifth edition}

I want to thank Philip Kime for help with biber, the tool that Language Science Press is using for
creating lists of referecnes and for manipulating bibliography databases. The bibliography was
entirely updated and manually checked since this was done for the HPSG Handbook \citep*{HPSGHandbook}. Papers
now have DOIs wherever possible.

Ladis Duffet pointed out a mistake in Section~\ref{sec-valence-classes}, which probably confused many who tried to make
sense of this section in earlier editions.

% added Ajdukiewicz35a-u to DP authors.

% changed some examples in 1-begriffe and other files

% 23.02.2021 changed AdvP -> [very] to AdvP -> [very,roof]

% 23.02.2021 fn regarding triangles in trees

% 26.01.2021 added Lötscher85 to dg chapter.

% 05.11.2021 fixed Peter Cullicover's findings gloss in dg.tex
% and missing "d" in 3-gb.tex

% 03.12.2021 added LS-GRAM for German HPSG implementations

% 06.12.2021 fied typo in index head"complement"=phrase

% fixed typos from email in 1-begriffe.tex (patch)

% 04.04.2022
% typos in 2-psg. Klammern in {ex-einige-kluge-Frauen-und-Maenner}
% 12.04.2022
% typos in 3-gb.tex meet- statt meet
% Titel des zitierten Buches in Leseliste war falsch, noch aus deutscher GT falsch
% 19.04.2022 3-minimalism
% S. 180 \_ fehlte bei ATB extraction
% EPP erklärt und Quelle
% 20.04.22 mögen -> like
%
% 27.04. female person = woman
%        electrical appliance = electrical device
%        PREDspecification war ab Ausgabe 4 kaputt
% 06.05. 7-cg.tex removed lots of brackets around cg-expressions. Thanks to anonymous reader.
%
% 15.05. 8-hpsg.tex Strange brackets in Figure {verb-movement-syn-simple}.
%        loc should be LOC in figure 9.15
%        edits due to comments by anonymous reader.
% replaced Mann by Roman

I fixed a mistake at the beginning of Section~\ref{sec-typeraising}: it now reads ``backward
application'' instead of ``forward application''.

I fixed the Case Principle in the chapter on HPSG. The first two clauses did not mention that they
only apply to verbal heads.

As pointed out to me by an anonymous reader, the type of the AVM in (\ref{avm-woman}) should have
been \type{woman} rather than \type{female person}. The top-most type in Figure~\ref{fig-electric-appliance} has to be
\type{electric device} rather than \type{electrical appliance}, since this is the name used in the text.

I fixed some brackets in the Categorial Grammar derivation in Figure~\ref{Abbildung-CG-isst-der-junge-den-kuchen-jacobs}. There were
just too many brackets to keep track of everything \ldots. Thanks to Matthew Korte and Pascal
Hohmann for spotting this (independently)!
Léonie Cujé found superflous brackets in Figure~\ref{abb-CG-Adjunktion}. They were removed. Thanks!

Figure~\ref{verb-movement-syn-simple} on page~\pageref{verb-movement-syn-simple} contained some strange brackets, which I removed now.

I also want to thank an anonymous reader for sending patches to the \LaTeX{} files correcting some
typos and wrong or missing words in glosses.

Since the last two reviews of the book complained about the classification and new developments
sections referring to material not introduced yet, I decided to make the structure of the book more
explicit by repeating the introductory remark from page~\pageref{page:structure-of-book} at the
beginning of all the advanced sections. I still think that this is the correct structure of the book to introduce a
certain framework and then evaluate it. The only way to fairly evaluate a theory is to compare it to
other theories. This cannot be done without knowledge of the theories to be compared. So readers
interested in such comparisons should read the introductory parts of the chapters and then come back
to the evaluation part and the parts discussing further developments. \citet{Culicover2021a}
remarked that it is unclear how the book is supposed to be used for teaching. The book is already
used at many, many universities worldwide, but those who want to know how I use it may check out my
slides, which are available both as PDF and source code on GitHub:
\url{https://github.com/stefan11/grammatical-theory-slides}. During Corona times I also put
recordings of my lessons online: \url{https://www.youtube.com/watch?v=_W6nVRnC0NA&list=PLXwGGsuPxWRotmEg5LStGTxZWEkqKXmrh&index=1}.

~\medskip

\noindent
Berlin, \today\hfill Stefan Müller



%      <!-- Local IspellDict: en_US-w_accents -->


%      <!-- Local IspellDict: en_US-w_accents -->
