% revised version
% Langsci pdf comments worked in 2017-06-20
% \begingroup
% \let\clearpage\relax
\chapter{Introduction}
\label{cha:intro}
% \endgroup
% \noindent
This work is concerned with the notion of semantic transparency and
its relation to the semantics of compound nouns. On the one hand, my
aim is to give a comprehensive overview of the phenomenon of semantic
transparency in compound nouns, discussing its role in models of
morphological processing, giving an overview of existing theories of
compound semantics and discussing previous models of the semantic
transparency of compounds.  On the other hand, I will discuss in
detail new empirical investigations into the nature of semantic
transparency and the factors that make compounds % complex nominals
appear more or less transparent. This part focuses on English noun
noun combinations.

\section{A first notion of semantic transparency}

\is{semantic transparency!{a first notion of}|(}
\is{semantic transparency!{in terms of relatedness to constituent meanings}|(}
\is{compound!{constituent meanings}!{relation to compound meaning and semantic transparency}|(}
Semantic transparency is a measure of the degree to which the meaning
of a multimorphemic combination can be synchronically related to the meaning of its
constituents and the typical way of combining the constituent
meanings. Semantic transparency is a scalar notion. 
\is{semantic transparency!{in terms of meaning predictability}|(}
\is{meaning predictability|(}
At the top
end of the scale are combinations whose meaning is fully transparent,
that is, combinations whose meaning is predictable. Conversely, at the bottom end are combinations whose
meaning is opaque. Their meaning cannot be predicted, and a link between
the meaning of the constituents and the meaning of the resulting combination can hardly be established.
 In between, there are
combinations with varying degrees of relatedness between the
constituents' meaning and the meaning of the whole, and with varying
degrees of predictability based on typical ways of combining these constituents.\footnote{Note that this  view combines 2 lines of thinking about
  semantic transparency. In particular, \citet[46]{Plag:2003}, in
  discussing derivations, links semantic
  transparency to meaning predictability, whereas
  \citet[344]{Zwitserlood:1994} understands the semantic transparency of
  compounds in terms of the synchronic relatedness between the meaning of their
  constituents and the compound meaning.}

Examples of English
compounds with different degrees of semantic trans\-par\-en\-cy are given in \Next.

\ex. % USE EXAMPLES FROM COCA!
\a. \label{ex:silk_fabrics}
For example, in the letters between Lady Sabine Winn and her mil\-li\-ner, Ann Charlton, sets of samples were sent, divided between gauzes, ribbons and \textbf{silk fabrics}. COCA
% Date 	2015
% Publication information 	Mar2015, Vol. 65 Issue 3, p30-36. 7p.
% Title 	Shopping, Spectacle & the Senses
% Author 	Dyer, Serena;
% Source 	MAG: History Today 
% \a. She wore gabardine trousers and a \textbf{silk shirt}, all in the mauve/beige family. 
% Date 	2012
% Publication information 	New York : Five Spot,Edition: 1st ed.
% Title 	Henny on the couch :a novel
% Author 	Soodak, Rebecca Land.
% Source 	FIC: Henny on the couch :a novel
% \a.  % \emph{superconducting cable} %, \emph{dead squirrel}
% Researchers there conducted an experiment in superconductivity that
% produced a current of 287,000 amperes, the highest current ever sent
% through a \textbf{superconducting cable}. COCA
% % Date 	1993 (Fall)
% % Publication information 	Fall93, Vol. 30 Issue 3, p4, 4p, 3bw
% % Title 	News notes.
% % Source 	Science Activities
\b.  \label{ex:bronze_lion}% \emph{bronze lion}
 The \textbf{bronze lion} was placed in the palace's foundations to please the
 gods. COCA
% Date 	2008
% Publication information 	Jul/Aug2008, Vol. 61 Issue 4, p46-52, 6p, 10 color
% Title 	Who Were the Hurrians? (cover story)
% Author 	Lawler, Andrew
% Source 	Archaeology
\c. \label{ex:school_teacher} % \emph{school teacher}
His dad worked for John Deere, his mother was a \textbf{school teacher}. COCA
% Date 	2012 (120318)
% Publication information 	FEATURES; Pg. 6E
% Title 	Off the football field, with a new goal in mind
% Author 	Ray Mark Rinaldi
% Source 	Denver Post
\d. \label{ex:milk_man} % \emph{milkman}
 I am the proud son of a hardworking \textbf{milkman}. COCA
% Date 	2012 (120227)
% Publication information 	A-SECTION; Pg. A11
% Title 	Tattooist^s art helps breast cancer patients
% Author 	Thomas Heath
% Source 	Washington Post
\d. \label{ex:butter_cup} % \emph{buttercup} 
 The creeping \textbf{buttercup} and Virginia creeper weren't as plentiful as
 she'd thought. COCA
% Date 	2006
% Publication information 	New York : Pocket Books, Edition: Pocket Books pbk. ed.
% Title 	Pretty woman /
% Author 	Michaels, Fern.
% Source 	Pretty woman
 \d. \label{ex:hog_wash} % \emph{hogwash} 
But experts call the hypothesis \textbf{hogwash}. COCA
% Date 	2003 (Mar)
% Publication information 	Vol. 234, Iss. 3; pg. 172
% Title 	Scary e-mail hoaxes
% Author 	Catherine Dennis
% Source 	Cosmopolitan
\d. \label{ex:cloud_nine}% \emph{cloud nine}
To stay on postcoital \textbf{cloud nine}, stick to no-brainer subjects that
won't make him think that this one night of passion has changed
everything. COCA
% Date 	2001 (Jun)
% Publication information 	Vol. 230, Iss. 6; pg. 198, 4 pgs
% Title 	The biggest communication mistakes women make
% Author 	Michele Bender
% Source 	Cosmopolitan

% The meaning of \Last[a], \emph{superconducting cable}, appears to be
% predictable based on the meaning of its parts and the typical, or
% standard way of combining the modifier \emph{superconducting} with the
% head \emph{cable}, namely by simple set intersection: a
% superconducting cable is a cable and is superconducting. 
The meaning of \emph{silk fabric} in \Last[a] appears to be
predictable based on the meaning of its parts and the typical, or
standard way of combining the modifier \emph{silk} with the
head \emph{fabric}. This standard way can in this case either be seen
as simple set intersection (a silk fabric is a fabric and is silk) or
as an instantiation of some relation between the 2 constituents,
here the \textsc{made of}-relation (a fabric made of silk).
The meanings
of the following 3 items, \emph{bronze lion}, \emph{school
  teacher}, and \emph{milk man}, are somewhat less predictable: a
\emph{bronze lion} might have the corresponding color, or might be
made out of bronze. In the latter case, he would not be a real lion,
but the image of one. \emph{School} and \emph{teacher} can be linked
by a local relation (teacher at a school), but both are not restricted in their combinatorics to a local
relation, cf. the occurrences of the 2 constituents in other compounds: \emph{geography teacher} or \emph{school
  finances}. Likewise, neither \emph{milk} nor \emph{man} seem to
suggest an interpretation along the lines of `HEAD who goes from house
to house delivering MODIFIER', cf. \emph{milkmaid, milk-soup, woodman,
  sandman, snowman}, and \emph{garbage-man}.
For \emph{buttercup}, some people might see a synchronic
relatedness between its constituents and the whole compound, pointing to the resemblance of the color of a buttercup's petals to the color of butter and the resemblance of the petals' arrangement to the shape of a cup.
Only the 2 final items in \Last,  \emph{hogwash} and \emph{cloud
  nine}, show no synchronic relation between their constituents and
the respective compound meanings.
% Other examples: \emph{ladybird, honeymoon,butterfly}
\is{semantic transparency!{in terms of meaning predictability}|)}
\is{meaning predictability|)}

% Complex words, that is, words consisting of more than one morpheme,
% differ in the to which their meaning can be predicted from the meaning
% of their parts. In English, inflectional morphology is largely
% transparent. That is, adding the plural morpheme ``s" to a noun
% regularly add the plural semantics to that noun. Leaving the domain of
% inflectional morphology, this kind of full meaning predictability does
% hardly exists. Especially noun noun combinations are notorious for
% allowing different interpretations, even if both its constituents are
% not ambiguous. Thus, a \emph{government advisor} can at least be an
% advisor for the government, i.e., someone advising governemnts, or
% someone from a government, i.e., an advisor provided by the
% government. 

% \ex. ADD CORPUS EXAMPLE!
% % oneThus, a combination like
% \emph{aquarium computer} can mean 

% \ex. \a. 
% All electronic components of the aquarium computer must only be operated in sound condition.
% % User manual, Dupla Multicontrol, Directions for Use 

 
% 1. Several possibilities
% 2. Contextual driven adjustments
% 3. Lexicalization/Semantic drift

% Typically, new compounds can be interpreted in a number of ways
% (\textbf{IS THERE AN EXAMPLE IN DOWNING?})

% \textbf{[HIER WEITER: SEMANTIC DRIFT/ARONOFF]}

% In addition, compounds are subject to semantic drift
% In derivational
% morphology, we regularly get deviations from full predictability. Thus,
% the suffix \emph{-able} is listed in the OED with the meaning
% ``Forming adjectives denoting the capacity for or capability of being
% subjected to or (in some compounds) performing the action denoted or
% implied by the first element of the compound." % "-able, suffix." OED Online. Oxford University Press, March 2015. Web. 8 May 2015.
% That is, from the very beginning, we have the possibility of two
% different resulting word-formations, that is, we cannot predict the
% meaning but in fact two possible meanings.

% Thus, we can predict that \emph{drinkable} should mean either
% \emph{having the capacity/capability of being subjected to drinking or
% of drinking}, and in fact the actual meaning of the word in, e.g.,
% \Next, corresponds
% to the first possible interpretation. 

% \ex. Left to its own devices, real ale stays in a drinkable
% condition for about a week.  % A0A 135 	

% An example for the second possible interpretation is a word like
% \emph{breathable} in \Next, where it is the mesh itself that breathes,
% albeit metaphorically.

% \ex. The upper is made from breathable nylon mesh with man-made
% leather reinforcements. % CFT 3037

% Note that in both cases only one of the possibilities gives the
% intended reading; in th


% The term `semantic transparency' aims to capture the intuitively felt differences with regard to 
% between noun-noun combinations like \emph{hogwash} in contrast to \emph{stone building}, meaning `nonsense', and a compound like
% \emph{milkman}, meaning `man who goes from house to house delivering milk',
% cf. \Next and \NNext for examples in context.

% \textbf{START WITH PLAG FIRST, USE CLEAR TRANSPARENT EXAMPLE!}

% \ex. This idea that we can't have 5 percent growth in America is \textbf{hogwash}. COCA
% % Date 	2011 (110615)
% % Title 	STEVE @!DOOCY, FOX NEWS ANCHOR: Let's bring in Newt Gingrich right now. He's down in our D.C. bureau
% % Source 	Fox_Live

% \ex. The \textbf{milkman} who hasn't been paid calls you bad names. COCA
% % Date 	2006
% % Publication information 	# 6/26/2006, Vol. 82 Issue 19, p66-74, 9p, 4bw
% % Title 	INNOCENCE.
% % Author 	Jhabvala, Ruth Prawer
% % Source 	New Yorker

 % second meaning from the OALD
% The meaning `nonsense' for \emph{hogwash} is not at all
% straightforward; in particular, the meanings of \emph{hog} and
% of \emph{wash} seem unrelated to the compound's
% meaning. \emph{Hogwash} is a semantically opaque compound. In contrast, the meaning of \emph{milkman} is clearly connected to
% the meaning of its two constituents, as witnessed by the fact that
% they both occur in the definition
% given above. Thus, this compound is semantically transparent. One way
% of defining semantic transparency proceeds exactly along the reasoning
% outlined so far, cf. the following quote taken from
% \citet[344]{Zwitserlood:1994}. 

% \begin{quote}
% ``The meaning of a fully transparent compound is synchronically
%   related to the meaning of its composite words (e.g. milkman). Semantic opacity refers to the situation in which the relation
% between the meaning of the whole compound and (one of) its constituents is not
% apparent.'' \citet[344]{Zwitserlood:1994}
% \end{quote}



 % second meaning from the OALD
% The meaning `nonsense' for \emph{hogwash} is not at all
% straightforward; in particular, the meanings of \emph{hog} and
% of \emph{wash} seem unrelated to the compound's
% meaning. \emph{Hogwash} is a semantically opaque compound.

Note that for combinations like \emph{hogwash} the qualification that the meanings of the compound and its
constituents must be synchronically related becomes important. Thus, it is not a coincidence
that \emph{hogwash} means nonsense, and neither of its 2 constituents are
arbitrarily chosen terms. Rather, the `nonsense' meaning is etymologically
well motivated: According to the
 OED, it was originally used to refer to kitchen refuse that was used as food
 for pigs, as illustrated by the following quote.

\ex.    Cooks who were not
thrifty put all the kitchen leavings into a bucket. The content was called
‘wash’, and the washman visited regularly to buy it: he then sold it as
‘\textbf{hog-wash}’, or pigswill.\\
J. Flanders Victorian House (2004) iii. 87 OED
%  OED
% 2003   J. Flanders Victorian House (2004) iii. 87
%"hogwash, n.". OED Online. June 2013. Oxford University
%Press. http://www.oed.com/view/Entry/87638?redirectedFrom=hogwash (accessed
%June 20, 2013). 

Probably via the intermediate step of the second meaning reported in the OED,
`Any liquid for drinking that is of very poor quality, as cheap beer, wine,
etc.', \emph{hogwash} then came to be used with its now most frequent meaning,
`nonsense'. Both of these 2 last steps, that is, from liquid waste for pigs
to cheap alcohol and again from cheap alcohol to nonsense are metaphorical
extensions that are easy to follow; its current meaning is therefore quite well motivated on the basis of its
historical origin.

For \emph{cloud nine}, not even a good etymological explanation is
available. In addition, it is more restricted in typically appearing following the preposition \emph{on}, and, perhaps bearing witness to its unclear etymology, an alternative, \emph{on cloud seven}, is available, apparently with exactly the same meaning, compare the 2 earliest quotes from the OED in \Next.

\ex. \a. Oh, she's off on \textbf{Cloud Seven}—doesn't even know we exist.\\1956   O. Duke Sideman ix. 120  OED
\b. I don't like strange music, I'm not on \textbf{Cloud Nine}.\\1959   Down Beat 14 May 20 OED
% \b. 1960   H. Wentworth \& S. B. Flexner Dict. Amer. Slang 110/2   Cloud seven, on, completely happy, perfectly satisfied; in a euphoric state. 

Even though neither \emph{cloud nine} nor \emph{cloud seven} have been
attested for long, their etymology remains unclear; the best one can
find are statements like the following attempt for \emph{cloud nine}:
``the number nine is said by some to come from a meteorologist’s
classification of a very high type of cloud" \citep{Walter:2014}.
\is{semantic transparency!{a first notion of}|)}
\is{semantic transparency!{in terms of relatedness to constituent meanings}|)}
\is{compound!{constituent meanings}!{relation to compound meaning and semantic transparency}|)}

% Zwitserlood's definition thus allows one to classify \emph{hogwash} as an
% opaque compound, whereas \emph{milkman} is classified as a transparent
% compound. One major problem with this view of semantic transparency
% becomes apparent even when only looking at a larger set of English
% complex nominals, cf. \Next.
% % compounds, cf. \Next.


% One first such criterion is given when looking at a stricter
% understanding of semantic transparency, where semantic transparency is
% linked to meaning predictability, cf. e.g. the quote below, taken from
% \citet[46]{Plag:2003}.
% \begin{quotation}
%   ``[\dots], these forms are also semantically transparent, i.e. their
%   meaning is predictable on the basis of the word-formation rule
%   according to which they have been formed.'' \citet[46]{Plag:2003}
% \end{quotation}
% \noindent
% On this understanding of semantic transparency, \emph{buttercup} joins \emph{cloud
%   nine} and \emph{hogwash} in the opaque group: English does not
% provide a word formation rule of the type \emph{flower whose head has the
% color of A and the shape of B}.
% % Already the example given by \citet{Zwitserlood:1994} is enough
% % to demonstrate the difference between the two definitions. A standard
% % dictionary entry for \emph{milkman}, here taken from the Oxford
% % Advanced Learner's dictionary, % 5th edition, 1995
% % gives the following definition: \emph{a person who
% %   regularly delivers milk to people's houses}. Clearly, this
% % definition matches Zwitserlood's requirements for a fully transparent
% % compound: \emph{milkman} is related to \emph{milk}, because that is
% % what is being delivered, and it is related to \emph{man}, in that
% % \emph{man} can be used interchangably with person in this
% % context. However, 
% What about \emph{milkman}? Again, it does not qualify as semantically
% transparent according to Plag's definition. There is no word-formation rule
% that would help us to predict the exact meaning of \emph{milkman}, given that
% all that is additionally available to us are the meanings of \emph{milk} and
% \emph{man}. The problem is not that it is impossible to formulate such a rule
% (e.g. `expand \emph{AB} to \emph{B who regularly delivers A to people's
%   houses}'), the problem is that this rule would only apply to the single case
% of \emph{milkman}, but not to any other NN combination,
% cf. e.g. \emph{milkmaid, woodman, sandman, snowman, garbage-man}. A milkmaid
% milks cows, a woodman has something to do with forests, a sandman might dig
% sand (or worse, personifies sleep), a snowman is made from snow, and a garbabe-man collects the
% garbage. What they have in common is that their meaning can in no case be
% derived via the AB = B who regularly delivers A to people's houses
% scheme. \emph{Bronze lion} and \emph{school teacher} are a bit more
% difficult. \emph{Bronze lion} is member of a larger group of cases that
% roughly show the form material noun + solid, cf. \Next.
% \textbf{GIVE THE RULE!}

% \ex. % EXAMPLES material noun + solid
% \a. Candles in glass holders burned before the \textbf{stone angel}. COCA
% % Date 	2009
% % Publication information 	New York : Pocket Books,Edition: 1st Pocket Books trade pbk. ed.
% % Title 	In the blood
% % Author 	Phoenix, Adrian.
% \b. What's one more \textbf{paper cup} or \textbf{plastic bag}? COCA
% % Date 	1990 (Jul/Aug)
% % Publication information 	Vol. 45 Issue 1, p8, 8p, 10c
% % Title 	Let's Not Go Overboard!
% % Author 	Smith, Libby
% % Source 	Conservationist

% However, in contrast to e.g. \emph{paper cup/bag}, \emph{bronze lion} can be
% argued to contain one further semantic operation, namely the step from `real
% lion' to the metaphorical `image in the likeness of a lion'.

% \emph{School teacher} also falls into a larger class of constructions that
% behave uniformly, cf. e.g. \emph{kindergarten/college/university
%   teacher}. %all attested in
%                                 %Coca, with university teacher at 60, school
%                                 %teacher at 1359
% In all cases, the readings can be paraphrased along the pattern of `teacher who
% teaches at X'.
% However, the N \emph{teacher} pattern has also a large number of combinations
% where the first noun specifies the subject the teacher teaches, as in \emph{music/science/math
%   teacher}. %Coca 1134,463,394
% %  in \Next.

% % \ex. \a. 
% % \b.
% % \c.
% % \d. 

% Finally, \emph{superconducting cable} allows the prediction of its
% meaning with the help of a very simple rule, as it displays an
% intersective semantics: AB is A and is B. Other classic examples of
% adjectives that allow the application of this rule are \emph{dead} and
% color terms like \emph{green}, illustrated in
% \Next and \NNext.


% % on the list, \emph{superconducting cable},
% % shows the same intersective semantics like \emph{dead} and
% % \emph{green} in the examples discussed earlier. In contrast, the first
% % item, \emph{cloud nine}, is arguably more opaque than hogwash because
% % there are no traces of a motivation left. Before I come to a proposal
% % of how to order these items in terms of their internal semantic
% % structure, the next section discusses the measures that have been
% % proposed and used so far in order to quantify the notion of semantic transparency.
% % In addition to the fact that they hold for different sets of data, the
% % two definitions lack any indications of what exactly they would accept
% % as possible word formation rule on the one hand or as synchronic
% % relation on the other hand.

% % In the following, we will take a more detailed not only on these two
% % definitions, but also at other definitions of semantic transparency
% % given in the literature as well as the factors that are usually
% % mentioned in the discussion of semantic transparency.



% % It seems, then, that \emph{milkman} is only transparent in terms of the
% % Zwitserlood definition but not in terms of Plag definition. In contrast,
% % according to both approaches typical examples of intersective AN constructions
% % seem to be clearly transparent, cf. e.g. \emph{dead} and \emph{green} in \Next
% % and \NNext.

% \ex. \a. " He'll eat anything -- - a \textbf{dead squirrel} in the road, tomatoes in
% your garden, berries, your cat. " COCA
% % Date 	2012 (120226)
% % Publication information 	FEATURES; Pg. 1E
% % Title 	Predator now prey
% % Author 	Mark Davis; Staff
% % Source 	Atlanta Journal Constitution
% \b.  
% % Officers and men were dressed in drab khaki uniforms, instead of the
% % scarlet I had seen in England, and this somewhat disappointed me as it seemed
% % to detract from the glamour of war
% ; but worse still was the sight of the \textbf{dead soldiers}. COCA
% %  Date 	2012
% % Publication information 	Spring 2012
% % Title 	First Blood
% % Author 	Reitz, Deneys
% % Source 	MHQ : The Quarterly Journal of Military History

% \ex.
% \a. In the basement, accessed by a sliding glass door off the back yard,
% workers were arranging a \textbf{green suede sofa} on the beige carpet in the TV
% room. COCA
% %   Date 	2012 (120616)
% % Publication information 	; Pg. E01
% % Title 	Model home shows off its green side
% % Author 	V. Dion Haynes
% % Source 	Chicago Sun-Times
% \b.  A small projector, hidden within a \textbf{green plastic watering can}, shows a
% five-minute video in which hands arrange flowers, clean a vase and pick up
% bird eggs. COCA
% % Date 	2012 (120615)
% % Publication information 	Style; Pg. C10
% % Title 	^Projected Images^ catches flash, fade of life experience
% % Author 	Mark Jenkins
% % Source 	Washington Post

% A dead squirrel is a squirrel and dead, and
% a green suede sofa is green and is a suede sofa. We can use the rule
% `\emph{AB} expands to \emph{is A
% and is B}' to cover the semantics of the adjective noun constructions in all four cases in
% \Last and \LLast.





\section{Compounds and complex nominals}
\label{sec:intro-complex-nominals}

\is{compound!{vs. other constructions}|(}
Compounds share many properties with other complex constructions
having a nominal head. The term `complex nominal' is used in this work
to refer to
constructions of the general format MODIFIER HEAD, with the head always being a
noun and the resulting construction likewise being substitutable in noun contexts. It is a cover term that subsumes constructions that are traditionally called
compounds (e.g. \emph{blackbird, railway}, and \emph{volcano ash}) as well as constructions
that are traditionally considered as phrases (e.g.
\emph{superconducting cable} and \emph{brown hair}), extending on the
usage of the term in \citet[1--2]{Levi:1978}, where it was used to encompass
nominal compounds as well as combinations of nonpredicating adjectives
with nouns (e.g.  \emph{electric clock} or \emph{musical
  talent}).\footnote{\citet[1--2]{Levi:1978} specifically mentions a
  third group of constructions where the head noun is a deverbal nominalization (e.g.
\emph{presidential refusal} or \emph{metal detection}). However, as far
as I can tell these constructions are always a subset of either of the first 2 constructions.}

\is{stress pattern!{compound status and}|(}
For English, with no binding elements nor specific word forms as formal
markers of compoundhood, stress placement is often accepted as the
only fail-safe criterion for compoundhood: if an X-N construction is
stressed on the first constituent, then it is a compound (this
has been most famously formalized by \citealt[17--18]{ChomskyandHalle:1968}, who
distinguish between a nuclear stress rule and a compound stress
rule). However, as
\citet[761]{Plagetal:2008} point out after listing the many authors
stating exceptions to this rule, there is a considerable number of constructions that
are typically regarded as compounds but that do not show fore-stress,
compare the examples in \Next, drawn from (1) in
\citet{Plagetal:2008}.
 % write, ``should be regarded as compounds by
                       % most analysts''

\ex. apple p\'ie, Michigan h\'ospital, summer n\'ight, aluminum f\'oil,
spring br\'eak, silk t\'ie\\
(the acute accent marks the vowel of the
most prominent syllable)

\is{stress pattern!{compound status and}|)}
In this work, all these constructions are complex nominals and the
term \emph{compound} is also used with the wider, more general usage in
mind. In the
discussion of other criteria that have been introduced to diagnose
compoundhood the main focus has been on noun noun constructions. \citet{Bauer:1998} shows that none of the criteria traditionally employed to distinguish between 2 constructions (listedness, orthography, stress, syntactic isolation of the
first constituent, constituent coordination, \emph{one}-substitution)
yields strong evidence for a distinction between 2 types of noun noun
constructions. \citet{Bell:2011} follows \citet{Bauer:1998} in that
the criteria do not allow to distinguish between 2 different
categories and argues for
the analysis of all noun noun constructions as compounds. In a similar vein,
\citet[434]{Baueretal:2013} acknowledge that ``there seems to be no
established set of trustworthy procedures that could tell us reliably
and theory-neutrally for a given NN construction whether it is a noun
or a phrase", arguing for a maximally inclusive approach in assigning
compound status.

Note that the 2 major academic reference grammars of English both
maintain a distinction between 2 different categorical types of noun noun combinations: \citet[1332]{Quirketal:1985} distinguish between phrasal
and compound noun noun (N + N) constructions (they explicitly name stress and
\emph{one}-substitution as indicating compound- and phrasehood respectively), \citet[448--451]{HuddlestonandPullum:2002} distinguish and
discuss the difference between `composite nominals' and `compound
nouns'.
\is{compound!{vs. other constructions}|)}

\section{Aims and Goals}
\label{sec:aims_goals}
% \textbf{CHECK!}

% \textbf{WHAT EXACTLY IS SEMANTIC TRANSPARENCY? USE MARELLIS IDEAS from Marelli et al 2014 and Marelli and Luzatti 2014!}

This work has 2 main goals. Firstly, I want to show why the semantic
transpareny of complex nominals, and more specifically, of compounds, is an important topic in current linguistic
research. Secondly, I want to explore to what extent a more fine-grained analysis of the
factors involved in establishing semantic transparency allows one to
predict the 
semantic transparency of compounds. % even for new compounds. 
% The main hypotheses that I am
% going to explore/develop are the following:
%  \begin{enumerate}
%   \item The internal semantic makup of complex nominals plays key role for the
%     overall semantic transparency attributed to that complex
%     nominal. \textbf{[case study 1]}
%   \item The internal semantic parameters that determine the overall semantic
%     transparency are partially construction specific. That is, what counts
%     as semantically transparent for a compound might not be viewed as
%     semantically transparent in the case of phrases. 
%   \item Different constructions, e.g. phrasal and compound adjective noun and
%     noun noun combinations, can be captured via the same general framework.
%   \item While the semantic factors that play a role with regard to semantic
%     transparency are the same
%     crosslinguistically, their weight and their link to different
%     language-specific constructions might differ. \textbf{[case study 2]} 
%   \end{enumerate}

  As far as the data coverage is concerned, I will be mainly concerned with
  English noun noun constructions. 
% When convenient, these will be contrasted
%   to similar constructions in German, a language that employs more inflectional
%   morphology and therefore allows to see certain patterns more clearly.
% , namely , and a fully isolating language, namely Mandarin
%   Chinese.

\section{Structure}
\label{sec:struc}

Chapter 2 discusses the role and nature of semantic transparency in
psycholinguistics. Chapter 3 discusses the role of semantic
transparency in so far as it pertains to phenomena of interest to
theoretical linguistics. In addition, it situates semantic
transparency with respect to related terms. Chapter 4 is
concerned with the semantics of compounds and complex nominals. Chapter 5
discusses 3 previous attempts at modelling semantic transparency. 

The following 2 chapters are concerned with 2 new empirical investigations into
semantic transparency. Both chapters introduce statistical models for
semantic tranparency ratings on both compounds and their
constituents that make use of the semantic structure of the
compounds. Chapter \ref{cha:empirical-1} discusses models that use
properties derived from just the set of compounds for which the models
predict the ratings. In contrast, Chapter \ref{cha:empirical-2} introduces
models in which the semantic predictors take the distribution of the
semantic structure across a compound's constituent families into
account.

Chapter \ref{cha:conclusion} summarizes the main points and
gives an outlook to further research. 

The webpage for this book is
\url{http://www.martinschaefer.info/publications/semTranBook.html}.
%  It contains up-to-date links to the data used in the 
% \part{Preliminaries}
% \label{part:preliminaries}


%%% Local Variables: 
%%% mode: latex
%%% TeX-master: "habil-master_rev-1"
%%% End: 
