\documentclass[output=paper]{langsci/langscibook} 
\ChapterDOI{10.5281/zenodo.4727659}
\author{Aleksandar Trklja\orcid{}\affiliation{Universität Innsbruck} and Łukasz Grabowski\orcid{}\affiliation{University of Opole}}
\title{Introduction} 
\abstract{\noabstract}
\IfFileExists{../localcommands.tex}{
  \addbibresource{localbibliography.bib}
  \usepackage{langsci-optional}
\usepackage{langsci-gb4e}
\usepackage{langsci-lgr}

\usepackage{listings}
\lstset{basicstyle=\ttfamily,tabsize=2,breaklines=true}

%added by author
% \usepackage{tipa}
\usepackage{multirow}
\graphicspath{{figures/}}
\usepackage{langsci-branding}

  
\newcommand{\sent}{\enumsentence}
\newcommand{\sents}{\eenumsentence}
\let\citeasnoun\citet

\renewcommand{\lsCoverTitleFont}[1]{\sffamily\addfontfeatures{Scale=MatchUppercase}\fontsize{44pt}{16mm}\selectfont #1}
   
  %% hyphenation points for line breaks
%% Normally, automatic hyphenation in LaTeX is very good
%% If a word is mis-hyphenated, add it to this file
%%
%% add information to TeX file before \begin{document} with:
%% %% hyphenation points for line breaks
%% Normally, automatic hyphenation in LaTeX is very good
%% If a word is mis-hyphenated, add it to this file
%%
%% add information to TeX file before \begin{document} with:
%% %% hyphenation points for line breaks
%% Normally, automatic hyphenation in LaTeX is very good
%% If a word is mis-hyphenated, add it to this file
%%
%% add information to TeX file before \begin{document} with:
%% \include{localhyphenation}
\hyphenation{
affri-ca-te
affri-ca-tes
an-no-tated
com-ple-ments
com-po-si-tio-na-li-ty
non-com-po-si-tio-na-li-ty
Gon-zá-lez
out-side
Ri-chárd
se-man-tics
STREU-SLE
Tie-de-mann
}
\hyphenation{
affri-ca-te
affri-ca-tes
an-no-tated
com-ple-ments
com-po-si-tio-na-li-ty
non-com-po-si-tio-na-li-ty
Gon-zá-lez
out-side
Ri-chárd
se-man-tics
STREU-SLE
Tie-de-mann
}
\hyphenation{
affri-ca-te
affri-ca-tes
an-no-tated
com-ple-ments
com-po-si-tio-na-li-ty
non-com-po-si-tio-na-li-ty
Gon-zá-lez
out-side
Ri-chárd
se-man-tics
STREU-SLE
Tie-de-mann
} 
  \togglepaper[1]%%chapternumber
}{}

\begin{document}
\maketitle 
%\shorttitlerunninghead{}%%use this for an abridged title in the page headers

 



The notion of formulaicity has received increasing attention in disciplines and areas as diverse as  linguistics, literary studies, art theory and art history. In recent years, linguistic studies of formulaicity have been flourishing (e.g. ~\citealt{Wray2002,Wray2008,Wray2009,SchmittCarter2004,Wood2010a,Wood2010b,Wood2015,Kecskes2016,MylesCordier2017,PiirainenEtAl2020}), and the very notion of formulaicity has been approached from various methodological and theoretical perspectives and with various purposes in mind, be it descriptive, exploratory or applied.



The %
%objects?
%Microsoft Office User
%April 6, 2021, 11:31 AM
object of investigation in linguistic studies are multiword expressions (%
%MWEs
%Microsoft Office User
%April 6, 2021, 11:32 AM
MWE) but individual approaches and models differ in how %
%MWEs
%Microsoft Office User
%April 6, 2021, 11:32 AM
MWE are defined and identified in language. For these reasons, it would be wrong to claim that all linguistic studies of formulaicity constitute a uniform field of research. There is no such a thing as 'formulaicity linguistics'. Linguistic formulaicity has become a superordinate term for the view that a large proportion of natural language consists of repetitive lexical units. This makes %
%MWEs
%Microsoft Office User
%April 6, 2021, 11:34 AM
MWE somehow special with respect to alternative linguistic units of analysis that have theoretical foundations in formal syntactic, semantic or lexical structures. Such structures can be and are often included in the study of linguistic formulaicity but they do not provide the minimum necessary conditions against which %
%MWEs
%Microsoft Office User
%April 6, 2021, 11:35 AM
MWE are set as linguistic units. In fact, there are authors who proposed new approaches or models that deny the existence of such structures. The minimum assumption shared by all studies of linguistic formulaicity is that a MWE is considered a unit because it is a linguistic expression that has been repeatedly %
%I think reused and re{}-used are both acceptable but consistency should be maintained.
%Microsoft Office User
%April 6, 2021, 11:37 AM
reused. The very fact that a linguistic expression is %
%See above
%Microsoft Office User
%April 6, 2021, 11:39 AM
re-used across different situations and by different language users constitutes a good ground to treat it as a unit of analysis. It is therefore no wonder that the main focus in the study of linguistic formulaicity is on the investigation of the effect repetition has on various language issues such as idiomaticity, language acquisition, formation of social discourses, translation-related issues etc. As one can see, the novelty of these studies does not lie in the introduction of new issues they address but rather in a new treatment of established issues.



Linguists of various schools have studied linguistic formulaicity using different approaches and research perspectives, and with different purposes in mind. In an attempt to provide a useful generalization and conceptual clarification, \citet[163--164]{galkowski_kompetencja_nodate} argues that it is possible to distinguish between three major approaches to linguistic formulaicity, namely a linguistic, psycholinguistic and sociolinguistic one. The focus of the purely linguistic approach is on the investigation of formulaicity in terms of lexical and grammatical categories identified primarily using formal grammatical or functional lexical criteria. The psycholinguistic approach is primarily concerned with the study of how linguistic data is stored, processed as well as retrieved from the mental lexicon. Finally, the sociolinguistic approach explores situational and cultural aspects tied to the use of formulaic language (Gałkowski, ibid.). In reality, most studies combine these approaches as illustrated in \citet{SchmittCarter2004,Wood2010a,Wood2010b,Wood2015,Wray2002} or \citet{UnderwoodEtAl2004,PiirainenEtAl2020}, among others. Also, there has been a plethora of research conducted in recent years by specialists in corpus and computational linguistics, who study formulaic language with primarily applied purposes in mind, such as development of %
%natural language processing (NLP) tools
%Microsoft Office User
%April 6, 2021, 11:41 AM
natural language processing tools (NLP) or machine translation tools, fine-tuning textual classification methods etc. (cf. \citealt{ForsythGrabowski2015}; Pęzik 2018). Given such a proliferation of research perspectives, it is no surprise that formulaic language has been defined, labelled and operationalized in many different ways (cf. \citealt{WrayPerkins2000}; \citealt{Wray2002,Wray2009}), and each approach brings new insights into this interesting, yet at the same time, not fully and comprehensively explored phenomenon. This observation provided the main rationale for the present volume. We invited specialists that cover the whole spectrum of relevant issues and thus showcase their state-of-the-art research.



Thus, we present a selection of studies into formulaic language arranged into complementary sections. The first section with three chapters presents new theoretical and methodological insights as well as their practical application in the development of custom-designed software tools for identification and exploration of formulaic language in texts. The second section with two chapters presents examples of innovative research into formulaic language in language learning contexts. Finally, the third section with three chapters showcases research on formulaic language conducted primarily %
%from the perspectives of corpus linguistic, discourse studies and translation studies.
%
%Microsoft Office User
%April 6, 2021, 11:53 AM
from corpus linguistic, discourse studies and translation studies perspectives.



The first chapter by Joan Bybee and Ricardo Napoleão de Souza focuses on the relation between frequency effects typical of linguistic prefabrication and phonetic effects. By exploring a sample of adjective-noun sequences extracted from a conversational corpus, Bybee and de Souza show that certain phonetic effects, such as vowel duration, correspond to conventionalized structures found in prefabricated expressions. They also argue that phonetic effects are promising in view of future studies focusing on the notion of conventionality of prefabricated expressions. The authors demonstrate that prefabricated expressions constitute the conventional means of referring to these entities or concepts of some cultural importance despite being semantically compositional. In addition, they show that %
%Maybe scare{}-quotes or “such prefabs”
%Microsoft Office User
%April 6, 2021, 11:56 AM
prefabs form clusters of semantically related word sequences and that they can contribute to creativity in language use.



Richard Forsyth looks into formulaic language from a corpus-driven perspective and proposes a set of computational procedures to quantify the degree of formulaic language in individual texts and language corpora. Forsyth implements his approach into a custom-designed freely available software written in Python and shows - using an additional criterion of coverage - how n-grams of various lengths  emerge from the data and facilitate determination of the degree to which texts are permeated with recurrent sequences of words.



The chapter by Piotr Pęzik focuses on the identification of prefabricated expressions in dependency-annotated corpora. More precisely, he investigates restrictions on the valency of binary collocations and their tendency to be regularly subsumed by larger collocational chains. Specific examples from Polish and English are followed by a presentation of Treelets software, where the Author’s approach has been implemented, which illustrates in practical terms how recurrent multi-word items may be systematically explored using dependency-based methods.



The second section opens with a contribution by Stephen Cutler, who deals with an important problem of how new formulaic language is acquired and stored by L2 learners of English. In these studies, two different learning paths are contrasted: fusion (operationalized through a focus on the sequence’s elements and structure) versus holistic acquisition (operationalized through a focus on the spoken sound form of the sequence as a whole). Cutler argues that the findings provide further support to the claim that regular retrieval and simple corrective feedback help consolidate recall of the sequences learnt by L2 learners.



Ying Wang undertakes a successful attempt at a comparison of ideational functions of formulaic language in native student and expert academic writing. The chapter presents unique features of formulaic sequences identified in each text variety and shows that native student writing is more characteristic of everyday and highly idiomatic formulaic sequences, among others, while expert academic writing abounds in formulaic language associated with research and scientific argumentation. In conclusion, Ying Wang presents an informative discussion on how the research findings translate into formal instruction.



The last section of the volume starts with a chapter by Andreas Buerki, who shows how changes in social discourse are reflected in phraseology. Taking the 2016 referendum on the United Kingdom’s membership in the European Union as reflected in a large, tailor-made corpus of media texts, Buerki identifies various discursive strategies reflected in recurrent phraseologies and compares their use across time and specific topics. The results obtained in the study demonstrate how phraseological units reflect specific ideological positions. In addition, the present data indicates that formulaicity plays an important role in the Brexit discourse because it is more formulaic than %
%unclear which comparable discourse is being referred to
%Microsoft Office User
%April 6, 2021, 12:08 PM
the comparable discourse. Finally, the chapter casts new methodological insights into how phraseology, and formulaicity in general, can be used in discourse analytical research.



Łukasz Grabowski and Nicholas Groom undertake an attempt at employing the concept of grammar patterns in descriptive research on formulaic language in English-to-Polish translation. Their aim is to verify whether the Polish equivalents are realized with the same level of regularity. The detailed findings show that grammar patterns can be useful as a unit of analysis and a starting point for exploration of formulaicity in translation, and that they may cast more light onto some more general differences between semantics and pragmatics in source texts and translations.



Finally, the last chapter in the volume, by Mikhail Mikhailov,  takes under scrutiny the concept of syntactic idioms and explores through a corpus linguistic analysis the structure, meaning and use of the Russian construction N-\textit{s}{}-N and its English and Finnish matches. These counterparts are identified in parallel corpora. Mikhailov argues that the Construction Grammar approach used in his study helps make syntactic idioms more explicit for descriptive purposes, also when explored with the use of parallel and comparable corpora.



We believe that such a selection of original studies collected in this book will provide more insights into a fascinating phenomenon of formulaicity in language explored from both a systemic and textual angle. We sincerely hope that the volume will therefore come in useful for anyone interested in formulaic language, from both a theoretical and practical perspective.



Obviously enough, this volume would not have been possible without many people involved in its preparation, compilation and production. First of all, we would like to thank the Authors of the chapters for accepting our invitation and for further smooth collaboration through the entire production process, from the initial submission, review stage, revision stage to the very preparation of final versions of the chapters. We would also like to cordially thank our reviewers (Mikhail Kopotev, Stephen Jeaco, Francis Bond, Janusz Malak, Tadeusz Piotrowski, Stanisław Goźdź-Roszkowski, Łucja Biel, Laura Vilkaite, Jiří Milicka, Larisa Leisiö, Rita Jukneviciene, Magda Stroińska, Martin Hilpert, Cristiano Brocas, Alexander Rosen), who gave of their time for careful inspection and evaluation of all submitted chapters. Last but least, special thanks are extended to Editors of the series “Phraseology and Multiword Expressions” at Language Science Press, in particular to Michael Rosner, Manfred Sailer and Agata Savary, for giving us a green light to prepare and publish the volume, as well as to Sebastian Nordhoff and Felix Kopecky for their invaluable help in typesetting and technical matters. In particular, our sincere thanks are extended to Michael Rosner, who successfully and flexibly co-ordinated the entire volume preparation process despite difficult pandemic-related circumstances. 



Aleksandar Trklja, Łukasz Grabowski (Volume editors)

 
\sloppy\printbibliography[heading=subbibliography,notkeyword=this]
\end{document} 
