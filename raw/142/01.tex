\chapter{Why (I) care about Darwin's Problem}

This essay discusses in a somewhat informal way, with far too few references and well short of any adequate acknowledgements, what thinking about ``Darwin's Problem"---the problem of language evolution---has taught me about the nature of language and the landscape of the language sciences.\footnote{An important caveat: The reflections that follow are necessarily biased by lots of factors, not least of which is my professional training in a particular linguistic/cognitive tradition. I apologize if at times this professional deformation leads me to use a generic term like ``linguistics'' when sometimes I means ``the particular type of linguistic tradition I grew up in'' (for background, see my own work: \cite{boeckx2006linguistic,boeckx2009language}, as well as the thoughtful reflections in \cite{marantz2019linguists}). While I am fully aware of the severe limitations of this characterization, readers familiar with other linguistic frameworks should not feel alienated. I am only using the generative tradition as a foil, and advocate a far more inclusive vision for the language sciences in this essay.}

I like the term ``Darwin's Problem" as a way to refer to the problem of human language evolution, because it echoes the nomenclature introduced in \cite{chomsky1986knowledge}, to talk about the logical problems of language acquisition (``Plato's Problem"), language knowledge (``Humboldt's Problem") and use (``Descartes' Problem"). The term ``Darwin's Problem'' makes clear that this evolutionary focus is part of a family of questions that constitutes a research program very much in line with Tinbergen's ``Four Questions" program for ethology, which integrates mechanism, function, ontogeny, and phylogeny \citep{tinbergen1963aims}. Indeed, different approaches to Darwin's problem often go hand in hand with specific approaches to other issues such as language acquisition.\footnote{To give one example, I don't think that the difficulties faced by the standard generative treatment in the context of Darwin's problem are totally independent of the difficulties the standard generative treatment faces in the context of language acquisition. A much more comprehensive essay than the one I am able to offer here is required to articulate such interdependencies.}

Although other questions have occupied pride of place in linguistics, Darwin's Problem is my favorite, not only because I'd love to know how the modern human language faculty came to be, but also, and perhaps more importantly, because methodologically speaking it is the question that unambiguously makes the language sciences part of the biological sciences. Studying human language means different things to different people, and that's perfectly legitimate. A focus on the underlying biology is by no means the only option. I got into linguistics not because of an inordinate love for languages, but because of the promise (going all the way back to Descartes, and likely even earlier philosophers, \cite{chomsky2009cartesian}) that understanding this capacity that we have to develop at least one language is bound to tell us something deep about who we are. That's a humanities question alright, but everybody's guess is that the answer is ultimately rooted in biology; and this means, in light of Dobzhansky's famous dictum (``Nothing makes sense in biology except in the light of evolution", \cite{dobzhansky1973nothing}), evolutionary biology. Chomsky's frequent point (see, e.g., \cite{chomsky2012science}) that even when placed in the same environment, only a child, but not a kitten or a rock, ends up acquiring a language is deep down about Darwin's Problem: it compares creatures and points to the species-specific character of the trait in question. It's the quintessential question of what makes (made) us human.

Of all the Tinbergian questions on the agenda of the language sciences, Darwin's Problem is the most interdisciplinary one of all. You can't investigate it in the privacy of your linguistics office. That is unlike, say, the problem of what knowledge of language is. For the latter, linguists have (successfully) proceeded pretty much like philologists did in the past: analyzing (parts of) sentences, across languages. Familiar data, modulo the odd sentence construction. You can't (seriously) do this in the context of Darwin's Problem. I know some people have tried: they have looked for ``fossils" --- modern language constructions that (they claim) linguistic theory would single out as ``simpler"; and interpreted these essentially as relics or vestiges of a simpler, pre-linguistic/proto-linguistic system. \cite{bickerton1984language}, \cite{jackendoff1999possible}, and, in a book-length format, \cite{progovac2015evolutionary} are among those who have advocated this approach. I have written about why I find this non-compelling \citep{boeckx2016ljiljana}. At the end of the day, these ``fossils" are modern language constructions, and their proto-linguistic status rests on some speculation about what proto-language was (as well as on the researchers' analytic biases regarding what counts as ``simple constructions"). But how do we know? Indeed, how could we possibly know in the absence of linguistic documents from that long-gone era?  

This \textit{modus operandi} is very different from the approach pioneered in \cite{kirby2001spontaneous}, and now pursued by many researchers: the iterated learning paradigm looks at language(-like) data (artificial grammars), and tests participants that have a modern language capacity, but the goal is to distill generic biases that drive the learning process (and the results are crucially validated in computational models where biases can be controlled for). Unlike the search for language fossils, the iterated learning paradigm can be readily exported to other domains, and indeed the iterated learning paradigm has been applied to non-language material (whistles, drawings, etc.) \citep{cornish2013systems,verhoef2013combinatorial}. The language(-like) data is not essential to the iterated learning paradigm. But it is essential for the fossilized-construction studies (which invariably rely on jargon that is not free from theoretical dispute). In the latter case philologists/traditional grammarians feel at home. In the former, maybe less so.

The iterated learning paradigm is just one of the many ways in which Darwin's Problem has become empirically tractable, ``experimentable" in the lab. Progress in genetics offers other experimental opportunities. Refined methods in comparative psychology offer yet others. All of these options are now open to language scientists. But they won't attract the linguists only at ease amidst (parts of) sentences. That's the great value of thinking about Darwin's Problem: it forces you to make a choice: which draws you more? The nature of language data, or the nature of cognition? Do you take language to be the ultimate goal, or the means to get there? If you have to make constructive comparisons with species that don't have language, the answer is inescapable.

Darwin's Problem is also great at forcing linguists to be specific about what Gallistel called ``the foundational abstractions" \citep{gallistel2009foundational}. Along with \cite{krakauer2017neuroscience}, I agree that the cognitive descriptions of behavior have a lot to contribute to work in other disciplines. But we can't just ask the folks across the border to read our textbooks. The textbooks train their readers in a particular discipline. We must get these texts down to basics; stick to the essentials. Ideally, frame these in generic terms; otherwise, they won't ``get past customs'', as it were. This is a massive ``mapping'' problem, as David Poeppel has called it \citep{poeppel2012maps}, expanding on the important reflections in \cite{poeppel2005defining}. It is of the utmost importance. For language, I side with \cite{fitch2014toward} and \cite{uriagereka2008syntactic}, and think that some of the earliest descriptions of linguistic computations, such as some of those found in \cite{Chomsky1957} and reviewed in the first chapters of \cite{lasnik2000syntactic}, constitute a rock-solid foundation. Notice that in those early studies, actual (parts of) sentences played no role. It was all algebraic: terminal symbols, non-terminal symbols, transformations, monostrings, etc.\footnote{Indeed, on the first page of \cite{Chomsky1957} one reads: ``The ultimate outcome of these investigations should be a theory of linguistic structure in which the descriptive devices utilized in particular grammars are presented and studied abstractly, with no specific reference to particular languages." I do not think that Chomsky's statement is an encouragement to ignore data from languages, but rather (and more interestingly) an invitation to develop a linguistic theory that remains useful even when traditional data points are not available.} I find this ideally suited for fruitful comparisons with species that don't manipulate (parts of) sentences of the familiar sort.

Of course, some might say this is not ``core'' linguistics. That's fine. Language is such a rich and complex phenomenon that different people are entitled to different opinions about language. What's clear in the context of Darwin's Problem is that language is not a thing. It is many things put together: it's a mosaic, a patchwork, a complex system -- a conjunction of many parts that have come together in the course of evolution. Linguists would call it a compound.

Interestingly, linguists distinguish between two types of compound. There are compounds like \textit{handbag}, where one of the parts is clearly dominant (a handbag is a bag, not a hand). Such compounds are called endocentric. There are other compounds, like \textit{football} (the game), that are called exocentric, where all the parts are equally important. In light of \cite{hauser2002faculty}, one could say that linguists tend to think of the human faculty of language as an endocentric compound. Sure, they say, the language faculty consists of many parts, but some parts are more important than others. These would constitute the core, and the rest would be ``externalized'' to the periphery. The bet here is that the core is species-unique. I think this renders cross-species comparison particularly difficult. It's too easy to turn the core of the compound into a mountain that is too tall to climb for other species. It leads to a kind of exceptional nativism---something irreducibly unique about human language. Comparative psychology becomes necessarily contrastive. The alternative, which I favor, is one that takes the language faculty to be akin to an exocentric compound: all parts are needed to make a unique whole, but none of the parts, on their own, are unique. As such, it's just a matter of identifying them, across cognitive domains, scattered among organisms. I think that's the only way to climb ``mount improbable'', to use Richard Dawkins' apt phrase.

The leitmotivs that animate and structure this essay are thus: 
\begin{enumerate}
\renewcommand{\labelenumi}{(\roman{enumi})}
\item renewed appreciation for the comparative method applied to cognitive questions, leading to the identification of elementary but fundamental abstractions in non-linguistic species relevant to language
\item awareness of the conceptual gaps between disciplines, and the need to carefully link genotype and phenotype without bypassing any ``intermediate'' levels of description (certainly not the brain)
\item adoption of a ``philosophical'' outlook that puts the complexity of biological entities front and center
\end{enumerate}
I see these three themes as the ingredients of the current zeitgeist, which is aimed at reducing distance between species and levels of analysis. Hopefully, the discussion that follows will encourage linguists to take part in this interdisciplinary enterprise. 

At the end of the day, Darwin's Problem is a question that opens the field of language studies like no other I know of. That's why I agree with Steve Levinson's assertion that ``real progress is likely to come from an evolutionary perspective".\footnote{\url{https://www.mpi.nl/imprs100/the-germ-of-an-idea}} Darwin's problem is the only one that has made me revise my understanding of language based on progress in other fields; progress that seems so fundamental that it requires a shift of perspective in order to be integrated (the \textit{FOXP2} literature being a prime example; \cite{fisher2019human}). It's the only one that expanded my data set (filling it with data of different kinds, from different species, from birds to bats to baboons). It's the only question that has left me without any excuse for not doing biology.








