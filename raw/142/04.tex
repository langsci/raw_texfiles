\chapter{Richer fossils}\label{chap4}

For much of their history the language sciences have been dominated by a philological methodology, and a reliance on (narrowly defined) linguistic data. Including in the generative era, when the goal of the enterprise changed dramatically to the cognitive, the methodology remained largely philological, in the sense that traditional linguistic data (words, sentences, etc.) were used to distill cognitive principles, and eventually, one hopes, insights into the nature and content of the biological endowment. Even in more interdisciplinary domains like neurolinguistics or developmental linguistics, traditional data points, in the form of stimuli or child utterances, remain central. But in the absence of such data from the fossil record, or from ``non-linguistic'' creatures, this methodology comes to a stop.\footnote{I am aware of attempts to apply concepts from linguistic theorizing in the analysis of communicative acts in non-human animals \citep{schlenker2016formal,pleyer2020construction}. I have not yet been able to see clearly what such approaches could tell us about stages of language evolution whose users went extinct and left no linguistic document to apply these concepts to.}

I find this state of affairs well worth reflecting on, for in a certain sense, it illustrates the limits of a certain research program in linguistics, one that was designed to reach precisely this point of inquiry into the (evolutionary) biological foundations of language. It is perhaps for this reason that \cite{piattelli2011geneticist} characterized the work on \textit{FOXP2} as ``a geneticist's dream", but ``a linguist's nightmare". Just when the linguists found themselves confronted with first-rate molecular evidence bearing on our language capacity \citep{lai2001forkhead}, they found their methodology unable to handle it properly. As indicated briefly in chapter \ref{chap2}, since the \textit{FOXP2} discovery over two decades ago, there has been massive progress in many fields whose results bear on language and ``the human condition" (i.e.,   the quintessential focus of generativism, under Chomsky's vision, in my opinion). This provides a unique opportunity for linguists, but it requires a shift in methodology, because in order to exploit all these discoveries (and those yet to come), research in linguistics must proceed in a way that differs sharply from the way it has been done over the last half-century. As they say, in order to dig deeper, one needs a new shovel. 

To repeat a key point in chapter \ref{chap3}, I do not mean to throw the baby out with the bathwater. Insights from theoretical linguistics are here to stay. But it is the perfect time to take seriously Chomsky's assertion that in the future (which I think is now), ``it will be necessary to move to a much broader domain of evidence. What linguistics should try to provide is an abstract characterization of (particular and universal) grammar that will serve as a guide and framework for this more general inquiry" \citep{chomsky1981lectures}. Results concerning the format of rules and representations of the sort I touched on in chapter \ref{chap3} can indeed guide experimental inquiry seeking to establish linking hypotheses across levels of analysis. 

Unlike the early days of generative grammar, the aim is no longer to shift the focus of inquiry (while sticking to standard methodological tools); rather, the aim is to keep the (bio-cognitive) focus, but shift the methodology to give it a more decidedly (recognizable) biological character. 

Having more data, from a wide range of sources, certainly helps, but more data does not necessarily provide a fast lane to understanding. To get a good grasp of the biological foundations of language, one must combine ``big data'' and ``big theory''. Linguists might feel inadequate when it comes to gathering  data ``of this new kind'', although I think they shouldn’t feel that way. Moreover, there is still a vital role for them to play in influencing the design of future experiments, and in interpreting the results of past and current ones.

Sidney Brenner put it best when he wrote: ``As was predicted at the beginning of the Human Genome Project, getting the sequence will be the easy part as only technical issues are involved. The hard part will be finding out what it means, because this poses intellectual problems of how to understand the participation of the genes in the functions of living cells" \citep{brenner1995loose}. What is true of genes is also true of neural circuits, as \cite{krakauer2017neuroscience} rightly stress. Theoretical linguists should not fear engaging with more experimental fields, for as Hopfield pointed out, ``too few biologists attempt to deduce broad biological principles from the experimental facts at hand. Indeed the constant quest for new data can distract researchers from the important job of fitting the facts together in a coherent picture."\footnote{\url{https://www.princeton.edu/news/2000/12/15/neuroscience-contest-prompts-thinking-about-thinking}}

With these remarks in mind, I would like to provide a brief overview of some of the research opportunities that arise to study language evolution ``in the laboratory", as \cite{scott2010language} put it, or rather ``across laboratories'', for as we will see, new opportunities require integrating multiple domains of expertise, and no single laboratory can house all of them at once. Certainly, no single individual can be an expert in all of them.

It is likely that the specific illustrations I will use here will quickly become obsolete, so readers should keep their eyes on the main lessons. One key message is certainly that it is no longer productive to keep repeating that ``the basic difficulty with studying the evolution of language is that the evidence is so sparse" \citep{jackendoff2006did}. Yes, it is true that linguistic behavior does not fossilize, but indirect evidence can exist, especially if we learn about how to manipulate it carefully, avoiding jumping to conclusions, and instead assemble an explanatory chain of argument across levels of analysis. It turns out that aspects of language evolution can be reconstructed. They are not lost forever.

In my own work I have mostly relied on the evidence coming from ancient DNA, which I regard as a game changer. I think that paleogenetics/genomics, together with even more recent work on paleoproteomics, enriches the fossil record significantly. It does not reveal anything immediately about the evolution of cognition, but it provides key elements to reconstruct aspects of brain evolution that in turn can be related to cognitive capacity. This is why work on the neurobiological foundations of language is such a key level of inquiry: it is the main bridge between the molecular data and cognitive science. It is what makes ``molecular archaeology'' \citep{paabo2014human} possible.

\section{On language evolution and change}

Before embarking on a few illustrations of this line of research, I want to emphasize how the statements above complement (and do not replace or conflict with) work that seeks to find alternative ways to reconstruct language evolution. Here I have in mind the line of work that (in the words of Marieke Schouwstra and Simon Kirby\footnote{\url{https://blogs.ed.ac.uk/wegrowlanguages/about}}), ``grow miniature languages in the lab", by asking people to improvise and communicate with artificial signs. Researchers do so ``because [they] want to study how languages can start from scratch, and to see how the languages we know today could have gotten their rules and their rich structure." Such work essentially tries to create the necessary and sufficient conditions for cognitive biases brought to the task by individual learners to modify the raw input of data and shape it in a way that captures key properties of natural languages.

This process of grammar building in the course of interactions (learning/use) is sometimes dismissed as being concerned with processes of language change, not language evolution, since the individuals come to the task already equipped with a human language faculty \citep{berwick2016only}. This criticism can be mitigated by resorting to a complementary method of grammar formation using computational models where the biases of interacting (artificial) agents can be controlled for \citep{kirby2001spontaneous,thompson2016culture}. However, critics are quick to point out that this line of work implements the cognitive biases by brute force, and does not show how these evolve organically, as must have happened in the course of (biological) evolution.

I find this type of criticism unfair. To begin with, every experimental method has limitations. \textit{In vitro} work in the life sciences does not attempt to reconstruct all the processes that took place in evolution. The point is to create the conditions that make it possible to isolate at least one factor. Likewise, \textit{in vivo} work with animal models is not claimed to capture all aspects of the species these animals are trying to model. Limitations are opportunities for complementary approaches to arise and fill in the gaps.

As for the claim that what is being studied ``in the lab'' is language change (``glossogeny") as opposed to language evolution
(``phylogeny"), I remain unconvinced, for reasons that I think matter when thinking about language as a biological object of study and how it evolved. On the face of it, the difference can be clearly stated, as Martin Haspelmath has done\footnote{\url{https://dlc.hypotheses.org/894}} (see also \cite{mendivil2019did}): language evolution refers to the emergence of a cognitive capacity to acquire at least one language, whereas language change refers to the emergence of a new language that differs from the one that preceded it. Crucially, this new language and the one it replaced were both acquired by generations of individuals equipped with the language capacity whose evolution is the central topic of ``language evolution''. Put differently, one domain of research studies the difference between a non-linguistic creature and a linguistic creature, whereas the other domain of research focuses on how different generations of linguistic creatures exercise their (by hypothesis, invariant) language capacity. But to my mind researchers framing the issue in this way have underestimated the difficulty of a new question that arises by doing so: What is a linguistic creature once we recognize (as discussed earlier in this essay) that language is an evolutionary mosaic?

It seems to me that many of the researchers who insist upon a sharp separation between language evolution and language change also insist (tacitly) on a sharp contrast between ``us" (humans) and the other animals (some kind of ``FLN''). That is to say, the distinction between language phylogeny and glossogeny is not theory-free and goes well beyond somewhat arbitrary decisions about lexical conventions (evolution vs. change).

In a world where the notion of language is fixed\footnote{One where humans are still seen as the pinnacle of evolution?} (as in the orthodox characterization of ``Universal Grammar''), the distinction between language evolution and language change appears to be a no-brainer. But if the concept of language is far less uniform, biologically speaking---in other words, if the concept of language leaks (across species)---, then the gap between linguistic and non-linguistic creatures is reduced (dramatically so in the case of our closest relatives with whom we interbred), and once this gap is reduced, so is the gap between (the processes underlying) language evolution and language change. What emerges instead is a much more gradual picture or continuum for language, pretty much like the one already entertained for key aspects of language such as vocal learning \citep{petkov2012birds,martins2020vocal}.

Instead of thinking of the language faculty as a trait that emerged abruptly, and that did not change once it emerged \citep{berwick2016only}, I find it more useful to think of our language capacity as a collection of (generic) cognitive biases
\citep{christiansen2016creating,gervain2010speech}\footnote{I do not think we yet have a very good idea of what this catalog of cognitive biases may consist of. I suspect there are likely to be very many, associated with general notions like memory, attention, salience, etc.} put to the task of acquiring and using ``an art", as \cite{darwindescent} defined our ``language instinct''. Such biases may be more or less ``primitive'' (widely shared across species) or ``derived'' (substantially modified over the course of evolution of our lineage),\footnote{In other words, the emergence of some biases may depend on particular anatomical developments (brain growth trajectory, etc.).} but without a sharp discontinuity from the cognitive capacities of our ancestors and living relatives (contra \cite{hauser2002faculty}, as we saw in chapter \ref{chap3}). As such, the picture that emerges is not radically different from the cognitive continuity observed across the communities of language users that constitute the focus on research on language change. The changes may look more or less dramatic but that is more a subjective (non-theory-free) assessment than an objective truth.

While it is tempting to define our ``modern'' human language capacity as the full collection of cognitive biases that reliably leads to the acquisition and use of natural languages, I doubt that this statement is either necessary or sufficient. It is not necessary because perhaps not all biases are absolutely needed to reliably learn a language, and it is not sufficient, because talk of cognitive biases too quickly leads one to think of ``internal'' factors, at the expense of ``external'' factors.

The constructive role of the environment (the context of acquisition and use\footnote{Acquisition and use are not different things. Acquisition is use with a greater degree of uncertainty \citep{mccauley2019language}, much as evolution is akin to building an airplane while flying it.}) may well contribute significantly to the reliable emergence of properties once too quickly built into ``Universal Grammar'' (and attributed ``ultimately" to the genome) \citep{kirby2017culture,raviv2020language}, so much so that the environment may act as a buffer and compensate for a large amount of biological variation among language learners. In other words, it may be that some cognitive biases are not so necessary,\footnote{They could be said to be defeasible (in a sense reminding one abstractly of how grammar is organized according to Optimality Theory, \cite{smolensky1993optimality}).} with effects felt only in ``exceptional'' circumstances of acquisition and use. After all, even ``core'' aspects of language may come from \textit{tendencies} (e.g. the ``dendrophilia'' hypothesis in \cite{fitch2014toward} mentioned in chapter \ref{chap3} and defined as a propensity for hierarchical structuring), rather than hard, all-or-nothing constraints. There may be a fair amount of redundancy among biases.

Over and above computational models and more realistic experimental settings, the literature on emerging sign languages makes clear that the ``arena of use" (to use a phrase from Hurford's lucid and prescient essay; \cite{hurford1990nativist}) matters in shaping grammars,\footnote{Consider also works showing how different environmental conditions correlate with certain typological properties, e.g. \cite{everett2015climate}.} and reveals that one should not insist (contra \cite{mendivil2019did}) on a radical separation between language evolution and language change, not because\footnote{Contra T. Scott-Phillips, in \url{https://dlc.hypotheses.org/894}.} ``language emergence'' (``The change from a very simple system into a system that is `characteristically linguistic'{}") is a third, mid\-dle-ground scenario, but because the notion of ``characteristically linguistic'' is much harder to define in a Darwinian context than it is in a ``Cartesian'' context \citep{chomsky2009cartesian}.

Consider for instance the grammatical differences between the songs of white-rumped munias and the songs of Bengalese finches. 
As is well-known thanks to the groundbreaking work of Kazuo Okanoya \citep{okanoya2004bengalese,okanoya2017sexual}, the domesticated strain of the munia, known as the Bengalese finch, exhibits greater song variation and complexity (greater variation in transition between notes, making the structure of the munia song more linear). If we were to refer to these song repertoires as ``languages'', would we treat the change in song structure from the munia to the finch as a case of language evolution or language change? That there are genetic differences between the wild munia and the domesticated Bengalese finch would maybe lead one to talk about language evolution, although the core song circuit of the Bengalese finch does not differ in fundamental ways from that of the munia. The environmental context clearly differs, and so perhaps one would speak of new song emergence, or song change. In the case of differences like Middle English vs Modern English, few doubt that we are dealing with a case of language change, but I think that our faltering intuitions about evolution or change in the munia vs. Bengalese finch situation point to these processes occupying a continuum. This very much matters for the treatment of language \textit{evolution}, in light of the growing evidence that a fair amount of ``language-readiness" must have been in place in the common ancestor we shared with the Neanderthals \citep{dediu2013antiquity}.

Consider another scenario. It has been claimed that aspects of our biology (jaw size, vocal tract configuration, possible brain-related mutations harbored by microcephaly candidate genes like \textit{ASPM}; \cite{dediu2017language,blasi2019human,dediu2007linguistic,dediu2021tone}) may impact typological properties of our language (e.g., presence of certain classes of consonants, or tonal contrasts). Given the relevance of biological mutations, would we speak of language evolution? Is the difference between a system with certain consonants or with tone ``enough'' to qualify as language evolution, or do we take these differences to still fall within a certain type of linguistic system and speak of language change instead? Would our intuition carry over to situations of, say, pervasive congenital deafness in a population forcing a modality change in the way the language users communicate? Perhaps we would still say that the ``underlying'' system remains the same, but aren't transitions from a gesture-dominant to a speech-dominant system treated as language evolution in the ``protolanguage'' literature \citep{fitch2010evolution}? This is particularly important in light of the Darwinian take on selection, which necessarily works on standing variation.

There are of course non-cognitive (philological) ways of studying language change that look totally inadequate when applied to language evolution, but it seems to me that if one adopts a cognitive approach to language change, one that focuses on process rather than state \citep{heine18}, then the most sensible approach is to drop any sharp dividing line between change and evolution, and view ``linguistic'' differences across species and communities along a continuum.

I have sometimes heard\footnote{I recall the Necker cube metaphor being used in this sense by T. Scott-Phillips in one of his presentations.} that choosing to focus on the ``biological'' foundations or the ``cultural'' foundations of language is a bit like the two ways of viewing the Necker cube: both are valid perspectives. But I think this is the wrong metaphor to use, as it suggests that you must do one or the other (our visual system does not let you entertain both perspectives at the same time). Instead, if, as I suggest here, one must bear in mind the neurobiological foundations of cultural learning as well as the role of culture in giving meaning and direction to the learning biases we are endowed with, then the  biological/cultural divide is more like another optical illusion: the Penrose triangle, i.e., an impossibility.

Part of the resistance towards the view advocated here may stem from the failure on the part of the generative tradition to recognize the critical, structuring socio-cultural aspect of core properties of language (this is especially true in the domain of syntax). One may insist on language being ``for thought'' and not ``for communication'', as Chomsky and followers have done, but clearly language exists, and survives, thanks to its communal use. There is no faculty of language that is not instantiated in and by a specific language used by more than one language user. There is no ``parameter'' that can be set without cues from usage data. It is just wrong to say that language evolution is about the evolution of a mental organ, whereas language change is about the way in which this organ is put to use. They are not dissociable things: an organ without use is no organ at all. True, external stimuli don't contain ``grammar'', but nor does the genome.

\section{Self-domestication}

Part of the reason why I have devoted a fair amount of research time to the topic of ``self-domestication" is precisely because it offers a very concrete way to understand better the interaction between biological and cultural evolution. Self-domestication refers to the hypothesis that humans (specifically, \textit{Homo sapiens}) went through a process similar to that which morphed wolves into dogs, and that this matters for understanding human cognition and indeed some aspects of our language faculty. This process is, I think, best characterized as a reduction in reactive aggression \citep{wrangham2018two}. To strengthen the case for self-domestication, it is usually pointed out \citep{theofanopoulou2017self} that anatomical changes in our lineage are reminiscent of a set of phenotypical traits that tend to characterize domesticated species, collectively referred to as the ``domestication syndrome'' \citep{wilkins2014domestication}: reductions in skull and brain size, changes to braincase shape, reductions in tooth size, shortening of the muzzle/flattening of the face, and the development of floppy ears.

Self-domestication is hypothesized to have contributed to our ultra-social phenotype \citep{hare2017survival}. Crucially, for present purposes, this change in temperament modified the context in which humans communicated, learned from one another, and shared knowledge.

As \cite{thomas2014self} put it, as soon as we recognize the importance of cultural transmission for language evolution, it becomes important to ask about the origin of the traits that enabled that ``process of structure-creating cultural evolution" \citep{thomas2018self}. Eventually, this question leads to the neurobiological foundations of specific cognitive (learning) biases. Thus, cultural evolution and biological evolution cannot be kept distinct for long; there are clear feedback loops between the two. Self-domestication is a hypothesis regarding these neurobiological foundations.

Work on self-domestication over the past five years or so illustrates a handful of themes that reveal how much has changed in the context of language evolution. 

Until recently, the most successful branch of evolutionary linguistics from a comparative perspective was clearly the literature on vocal learning. Though rare among animals, vocal (production) learning (ability to modify vocal output based on experience) is not an ability unique to humans, and its existence in at least a few species has led to some impressive results at multiple levels of analysis, not only at the behavioral level, or developmental level, but also right down to neurogenetics (for a survey, see \cite{jarvis2019evolution}). I believe the self-domestication hypothesis opens the door to similar progress, now that there is a growing database of paleogenomes allowing one to probe the earliest stages of domestication, as well as a growing understanding of the neurological bases of tameness, which is the central unifying trait of domesticates. 

To be sure, progress does not entail lack of controversy: even in the domain of vocal learning, which builds on decades on intensive investigation, the exact set of vocal learners is still up for grabs, and the necessary and sufficient neurological mechanisms are still a matter of debate \citep{martins2020vocal}. The same is to be expected for the self-domestication hypothesis. Work over the past 5 years has been driven by an influential hypothesis (the neural crest based hypothesis put forth by \cite{wilkins2014domestication}) that ties the domestication syndrome (traits associated with tameness) to a mild neural crest deficit (`neurocristopathy'), conceptually similar to the role played by the hypothesis that vocal production learning depends on a direct cortico-laryngeal connection (J\"{u}rgens-Kuypers hypothesis, as per \cite{fitch2010evolution}. Both hypotheses are contested \citep{lord2020history,johnsson2021neural,lameira2017bidding}, but what is not up for debate is their usefulness in shaping experimental work \citep{pfenning2014convergent,zanella2019dosage,wilkins2021neural}. 

There may well be multiple paths to vocal learning
(multiple mechanisms at work) \citep{martins2020vocal,wirthlin2019modular}, just as different stages of domestication may require distinct explanations \citep{o2020glutamate}. It is likely that for both vocal learning and domestication the notion of ``continuum'' will be needed. This is just the fractal nature of scientific explanation at work. What matters, and what is the true sign of progress, is that it is now possible to move beyond claims that language is exclusive to us, and that careful experimental testing can be carried out.

Working on the self-domestication hypothesis has taught me several important lessons. First, it is possible to join forces, working across laboratories, to validate \textit{in vitro} hypotheses first generated \textit{in silico}. In my particular case, a close look at genetic differences between domesticates and their closest wild relatives \citep{theofanopoulou2017self} and between modern humans and their closest extinct relations \citep{kuhlwilm2019catalog} led us to zoom in on a region of the genome implicated in various neurodevelopmental disorders including the Williams-Beuren syndrome (known to give rise to a hypersocial phenotype), and study the impact of differential expression of a gene called \textit{BAZ1B} in neural crest development \citep{zanella2019dosage}. Though it is by no means the only relevant gene, we argued that it contributed to the retraction of the modern human face, which may thus underlie key traits of the domestication syndrome. 

In subsequent work \citep{andirko2021fine} we tested the claim that the modern human face emerges significantly earlier than other aspects of our ``modern'' anatomy, such as our characteristically globular braincase \citep{hublin2017new}. The central message of this work is that the \textit{sapiens} lineage has a more complex evolutionary history than previously assumed (see also \cite{scerri2018did,Bergstrom}), and that quite a few important things happened in the nearly 500k years after the split from the Neanderthals and the Denisovans \citep{stringer2016origin}. Accordingly, if at least some of the changes impacted cognition and our language capacity, as I currently think they did, they add dimensions of variation in the context of the ``antiquity'' of the language faculty \citep{dediu2013antiquity}: there is a lot of hypothesis space between ``exclusively'' modern/recent evolutionary changes and ``shared with our closest extinct relatives''. All of this contributes to a significantly more gradual narrative for language evolution.

Our attempt to pinpoint genetic changes associated with self-domestication \citep{theofanopoulou2017self}, has also taught me that although the initial focus may be on the domestication syndrome,
presumably the result of mutations impacting the neural crest, other changes, at the level of the brain, particularly those harbored by various receptors regulating stress circuits (glutamate receptors, oxytocin receptors) likely played a crucial role \citep{o2020glutamate,theofanopoulou2017hypothesis} in giving rise to the cognitive biases that became part of our ``domesticated phenotype'' (reduction in reactive aggression being in my opinion the most important one). Accordingly, when trying to model \textit{in vitro} some of the aspects of self-domestication, both the ``face'' and the ``brain'' and how these two interface must be taken into account. Our best bet right now (ongoing work with with Alessandro Vitriolo and Giuseppe Testa) is neuruloids, the organoid structures designed by \cite{haremaki2019self} to capture the developmental stages at which brain and face are about to embark on distinct trajectories. At bottom, we are trying to understand how \textit{sapiens} grew a small face, but maintained a big brain.

It is true that some of this experimental work may seem remote from linguistic concerns. But one must bear in mind the indirect connection between genotype and phenotype. The paleogenetic revolution opens the door to an unprecedented range of experimental opportunities to shed light on the human condition \citep{paabo2014human}, but it in no way reduces the gap between genes and cognition. As a result, we must learn to carefully and patiently build linking hypotheses step by step, and understand that for many of these steps, the overall shape of the explanatory link won't be obvious. The same is true in architecture: it took a long time for the outline of the Eiffel tower to emerge from the scaffoldings. It would have been a mistake to try to speed things up just to make the end result more transparent more quickly.

Still, in the context of self-domestication, we are beginning to understand how changes at the level of neurotransmitters impact specific circuits (especially the basal ganglia) that help us understand how the songs of the (domesticated) Bengalese finches become more varied, and in some sense more complex than the songs of the wild munias \citep{tomtics}. It is also clear that changes of facial morphology, e.g., the disappearance of prominent browridges, opened up new possibilities for facial expressions, reshaping social dynamics \citep{godinho2018supraorbital} and communication that must be understood in a multi-modal context. If social pressures truly impact grammatical structure (as evidenced in \cite{raviv2020language}), then the changes in social dynamics brought about by self-domestication must have modified our language capacity (in ways that could be revealed by work on other domesticates not known for vocal learning, such as dogs, bonobos, etc.).

\section{Brain development}

As stated above, there is evidence that our facial phenotype evolved earlier than other aspects of our cranium. In particular, our species-specific globular skull shape emerged in the last 100k years \citep{neubauer2018evolution}. I have long been interested in this characteristic skull shape \citep{boeckx2013biolinguistics,boeckx2014shape}, because much like facial reduction and retraction, which are potentially linked to a change in social cognition (`the self-domestication hypothesis'), a globular neurocranium points to a distinct perinatal brain growth trajectory in our species \citep{gunz2010brain}. Whereas the face of a \textit{sapiens} newborn is already characteristically ``small'', and ``modern-looking'', it takes longer for the neurocranium to acquire its distinctive shape, and regional brain growth changes appear to be the primary determinant.

Phillip Gunz, Jean-Jacques Hublin and colleagues have suggested that late-expanding, posterior structures like the cerebellum played a major role in this reshaping \citep{hublin2015brain}.\footnote{In our work, building on (paleo)genetic datasets, we indeed find evidence for this claim: ``modern''-derived, nearly fixed expression quantitative trait loci accumulate in the cerebellum more than in other structures, in a way that is statistically significant \citep{andirko2019derived}. Regions of the modern genome associated with signals of positive selection and embedded in larger introgression deserts (regions depleted of introgressed variants from other hominins) have a distinctive expression profile in the cerebellum \citep{raul}, and finally machine-learning approaches assigning an age of emergence to nearly-fixed mutations in the modern genome point to an enrichment for the cerebellum around 90kya \citep{andirko2021fine}.} Together with a few other structures, namely the precuneus/superior parietal area \citep{pereira2020morphometric,bruner2021evolving}, the cerebellum stands out in the context of \textit{sapiens} brain evolution (see also \cite{dumas2021systematic,weiss2021cis,gunz2019neandertal}).

The derived status of the cerebellum raises a lot of interesting new questions in the context of language evolution. To begin with, the modern-specific cerebellar expansion appears to be more pronounced for the right hemisphere \citep{kochiyama2018reconstructing}, and given that this is the dominant cerebellar hemisphere for language functions, it raises the possibility that this anatomical trait had cognitive import of great relevance for us. The fact that we can already identify candidate mutations for this differential growth of the cerebellum provides yet another piece of evidence for the truly transformative role of paleogenetics: even without ancient genomes, one could single out the cerebellum (as \cite{gunz2010brain} did on the basis of detailed virtual reconstructions of endocasts), but genetic information opens the possibility of going beyond size criteria and attributing differential growth trajectories to specific cellular phenotypes. 

In the context of the cerebellum, the granule cells constitute our best bet so far for a candidate cell population that may explain this cerebellar enlargement. This is important because such information can guide further inquiry: the question changes from ``What can an enlarged cerebellum do?'' to something more precise about the role of an expanded granular layer. For instance, \cite{straub2020gradients} point to an increased storage capacity associated with an expanded granular layer in mammals, opening up the possibility of amplified representational capacities in our species. Could this have led to an expanded range of representations (cf. our discussion of Graf's hypothesis in chapter \ref{chap3}), modulating cortical output in our species, adding tiers to the strictly local tree-based representations constructed in the temporo-frontal network linked by an expanded direct arcuate tract \citep{rilling2008evolution,friederici2017language,eichert2019special,balezeau2020primate}? Such questions highlight the need to investigate an ``extended'' language network, well beyond the classical (Broca's and Wernicke's) regions.

The evidence pointing to a distinctive role of the cerebellum also raises important new questions for comparative neuroscience, where the cerebellum is still all too frequently left out of the equation, as is the case for our circuit-level characterization of vocal learning, with only a few exceptions \citep{pidoux2018subcortical,Hoeksema2020.12.19.423579,wirthlin2018parrot,gutierrez2018parrots}. The same is true for developmental neuroscience, where certain (posterior) brain regions may be more important than previously thought (witness \cite{orpella2020integrating,irurtzun2015globularization}).

Last, but not least, it forces one to think about what the behavioral-cognitive contribution of these neuroanatomical changes was: now that we have a better appreciation of capacities of other hominins, are there some behavioral practices that could be imputed to these changes? I suspect that there are. If the brain grows differently, it wires differently, and thus functions distinctively. Providing detailed linking hypotheses addressing these questions is an important task for the years to come.

One of the interesting possibilities emerging from the different timing of modified ontogenies for the face and the brain is that the two-stage hypothesis put forward in \citep{okanoya2017sexual} to capture the structural differences between the songs of the Bengalese finches and those of the munias (first, a domestication/taming phase, followed by a sexual selection phase resulting in more varied and elaborate songs for the Bengalese finches) may guide hypothesis-construction for human language evolution. Could the self-domestication phase set the stage for further elaboration, made possible by changes in specific brain structures? How could this be tested?

\section{Language-ready ``mini-brains''?}

To my mind one of the most exciting possibilities for evolutionary studies arises in the context of impressive progress in the field of synthetic embryology, or, as it is more popularly known, of ``organoids". Organoids are three-dimensional culture systems consisting initially of homogeneous populations of stem cells that ``self-organize'' in complex ways. As they do so, they produce patterns that are similar to those found \textit{in vivo} during embryogenesis. As such, they offer manipulable ``miniaturized'' model systems of organs \citep{huch2017hope}.

\hspace*{-2pt} Thanks to the rapid advances in ``brain organoid'' studies (beginning with \cite{lancaster2013cerebral}), it is now possible to consider that aspects of our language-ready brain, especially those that arise early in development, may be examined and manipulated in an experimental context. This is particularly the case when combined with the use of gene-editing techniques (`CRISPR-cas9') to model in three dimensions the effects of variants found in species whose brains are otherwise lost to us forever, as is the  case for the Neanderthals and Denisovans \citep{Trujilloeaax2537}. In a way, such work would complement the efforts by evolutionary linguists to grow mini-languages in the lab.

Comparative work on brain organoids using closely related species such as chimpanzees and bonobos has already made interesting discoveries \citep{mora2016differences,kanton2019organoid}, but up to now it has mostly focused on cortical aspects, and sought to model factors that led to cortical expansion in the \textit{Homo} lineage \citep{heide2018brain,pollen2019establishing}. As discussed above, for \textit{sapiens}-specific aspects, we will need to develop new organoid models (most molecular events associated with cortical expansion are present in all hominin genomes currently available; \cite{florio2018evolution}). But it strikes me that there is a genuine possibility to capture aspects of human brain development \textit{in vitro} \citep{giandomenico2017probing,muchnik2019modeling,benito2020early}. For instance, it is now possible to generate ``assembloids" (fused organoids made up of distinct parts) for cortico-striatum structures \citep{miura2020generation} and cortico-spinal cord-muscle structures \citep{andersen2020generation}. Both structures figure prominently in discussion of vocal learning \citep{jarvis2019evolution}, and I think it is not unreasonable to anticipate that the assembloids just mentioned will enable us to probe the development of circuitry that provides the neurobiological foundations for speech. 

To be sure, organoid technology is not without challenges (reproducibility being the major one) or limitations (it can only capture the very early developmental stages, and it can as of now only test the effect of a few mutations at a time), but this is true of all models, and we should take advantage of the opportunities they offer. In particular, the (still distant) hope of constructing ``giant'' assembloids bringing together some of the most derived aspects of human brain development could provide a decisive step in ``brain-gene-ering''\footnote{\url{https://braingeneers.ucsc.edu}} the evolution of the language-ready brain.

To be very clear, the point is not to expect these organoids to ``speak'', but rather to reconstruct key aspects of the neurobiological scaffolding of our linguistic ability that the fossil record is inherently incapable of capturing. By exploiting paleogenetic information to grow brain organoids with ancestral mutations in them, we can, as it were, enrich the fossil record, and avoid the facile conclusion that ``languages don't fossilize". Aspects of brain development making language possible may be reconstructed from an expanded fossil record.

In so doing, we would be contributing to the ``exciting challenge" laid out in \cite{fisher2015translating}: ``to distil all these As, Gs, Ts, and Cs into meaningful insights regarding the biological underpinnings of some of our most mysterious traits, such as speech and language. By taking advantage of an ever-growing tool kit for investigating gene function, it will at last be possible to bridge the mechanistic gaps between DNA, neurons, circuits, brains, and cognition."
