\chapter{Conclusion}\label{ch:conclusion}

The research on different aspects of quantification and part-whole structures in natural language is abundant and since the early days of formal semantics it has continually led to a better understanding of certain properties of the language faculty. Nevertheless, one perspective seems to be somewhat neglected, namely to what extent the way in which the relationships between parts within a whole are conceptualized is relevant for grammar. The main aim of this book was to contribute to our understanding of quantification in natural language by exploring the so far understudied domain of subatomic quantification, i.e., quantification over parts of singular individuals. The mainstream perspective is that singular individuals are atomic building blocks of the denotations of singular count nouns. The consequence of such an atomicity-based approach is that such entities are also assumed to lack linguistically relevant internal structure. I provided what I believe is compelling evidence for an alternative view. 

The relevance of the phenomenon of subatomic quantification for natural language semantics suggests that the material part-whole structure of singular individuals is linguistically accessible. Therefore, the denotations of nouns such as \textit{cat} and \textit{apple} are treated as consisting of mereological atoms, i.e., entities with no proper parts. Instead, I argued for an approach which allows us to distinguish between different types of part-whole structures based on the way in which parts are arranged with respect to each other. In order to support this view, I explored various aspects of meaning of a broad range of linguistic expressions, such as different types of partitive constructions, certain types of adjectives, as well as multipliers, from a cross-linguistic perspective. 

One of the key findings presented in this book is the significance of the topological notion of integrity with respect to part-whole structures encoded in nominal semantics. Non-trivial properties of different types of expressions investigated in this book suggest that there is one unified parthood relation for various types of entities. At the same time, different part-whole structures result from distinct topological relations holding between particular elements. Entities conceptualized as integrated wholes, i.e., objects whose parts stick together, differ from those whose parts are not bound by any topological constraints or, alternatively, are arranged in some other type of spatial configuration. This concerns both wholes, e.g., the prototypical referents of singular count nouns and regular plurals, and parts, e.g., spatially continuous pieces as opposed to arbitrary discontinuous portions of matter. 

The major claim of the book is that subatomic quantification is subject to the very same constraints as quantification over wholes. Specifically, only entities that are conceptualized as non-overlapping integrated contiguous objects can be counted as one, be it either wholes or parts. In order to capture this generalization, I proposed a universal mechanism which allows for counting of both entire objects and their parts. Its formal implementation is based on the theory of wholes called mereotopology, which extends standard mereological parthood with topological notions such as connectedness. Building on this framework, I developed a system which derives the predicted semantics of representative constructions discussed in the empirical part of the book. Different aspects of their meaning arise as a result of the interaction between topology, partitivity, and numerical quantification. An important advantage of the proposed approach is that it does not define countability in terms of atomicity, since devising counting as quantification over mereological atoms, i.e., entities that have no proper parts, is very problematic for subatomic quantification.

This chapter concludes the book by summarizing its empirical insights as well as theoretical contributions to the study of quantification in natural language. First, I will provide an overall summary, then I will discuss some of the open questions and possible topics for future investigations.

\section{Overall summary}\label{sec:overall-summary}

As outlined in Chapter \ref{ch:introduction}, this book was intended to contribute to the field of formal semantics by exploring subatomic quantification. Throughout its pages, I provided a broad range of novel evidence indicating its significance in natural language and developed a mereotopological approach accounting for the observed phenomena. The book is divided into two conceptual parts. In the first part comprising Chapters \ref{ch:partitives-and-part-whole-structures}, \ref{ch:exploring-topological-sensitivity}, and \ref{ch:multipliers}, I provided a broad range of evidence indicating that natural language involves elaborate means to express subatomic quantification. The second part involving Chapters \ref{ch:conceptual-background}, \ref{ch:theory-of-parts-and-wholes}, and \ref{ch:mereotopological-account-for-subatomic-quantification} was dedicated to developing a conceptual framework that would provide means to model quantification over both wholes and parts, and a formal approach based on it that would account for a representative subset of the observed phenomena.

The data discussed in Chapter \ref{ch:partitives-and-part-whole-structures} provide cross-linguistic evidence that the same partitive word, e.g., \textit{part}, \textit{half}, and \textit{most}, can simultaneously appear both in entity and set partitives, i.e., constructions that refer to parts of singularities and pluralities and typically combine with singular and plural DPs, respectively. This fact indicates that such expressions are able to operate both at the atomic and subatomic level depending on what structure is provided by the embedded DP \citep[see][]{moltmann1997parts,moltmann1998part}. On the other hand, explicit partitives modified by cardinal numerals can only quantify over material parts of singular individuals, irrespective of whether the complement DP is singular or plural, see  \tabref{tab:properties-of-partitive-words2}, where `bare' and `count' refer to bare explicit and count explicit partitives, respectively. This might suggest that since partitive words have distinct properties in different structures, they differ significantly after all \citep{schwarzschild1996pluralities}.

    \begin{table}[h!]
    \centering
\begin{tabular}{lcccc}
\lsptoprule
                           & \multicolumn{2}{c}{\textsc{singulars}}          & \multicolumn{2}{c}{\textsc{plurals}} \\
                           & bare & count & bare & count \\ \midrule
subatomic quantification   & $\checked$                   & $\checked$                    & *                              & $\checked$ \\
quantification over wholes & *                              & *                               & $\checked$                   & * \\ \lspbottomrule
\end{tabular}
\caption{Properties of partitive words in explicit partitives}
\label{tab:properties-of-partitive-words2}
\end{table}

Nonetheless, the crucial piece of data coming from Italian partitives involving irregular plurals, i.e., expressions that imply not only plurality, but also cohesion or integrity of parts \citep{ojeda1995semantics,acquaviva2008lexical}, provides evidence to the contrary. In particular, when such a partitive is modified by the cardinal numeral, the whole phrase can get either a part-of-a-singularity reading or a part-of-a-plurality interpretation as long as the parts making up the plurality constitute an integrated entity. Therefore, countability results from the interplay between the meaning of a particular partitive word and the semantic properties of a singular or plural DP it combines with. If a plural expression denotes a plural entity constituted by a cohesive, i.e., spatially related, sum of parts, quantification over continuous parts of such a sum is possible, see \tabref{tab:properties-of-italian-parte2}.
	
    \begin{table}[h!]
    \centering
\begin{tabular}{lcccccc}
\lsptoprule
                           & \multicolumn{2}{c}{\textsc{singulars}}          & \multicolumn{2}{c}{\textsc{regular pl}}    & \multicolumn{2}{c}{\textsc{irregular pl}}  \\
                           & bare & count & bare & count & bare & count \\ \midrule
subatomic quantification   & $\checked$                   & $\checked$                    & *                              & $\checked$                    & $\checked$                   & $\checked$                    \\
quantification over wholes & *                              & *                               & $\checked$                   & *                               & $\checked$                   & $\checked$                    \\ \lspbottomrule
\end{tabular}
\caption{Properties of Italian \textit{parte} `part' in explicit partitives}
\label{tab:properties-of-italian-parte2}
\end{table}

Furthermore, I showed that at least in some languages the parallelism between entity and set partitives cannot be explained in terms of semantic ambiguity of partitive words. Instead, I argued that what allows for the cross-linguistically widely attested distribution of those expressions is the fact that singulars and plurals actually involve part-whole structures that are based on the same unified parthood relation. Crucially, what distinguishes the two is an additional notion that is responsible for how parts are topologically arranged with respect to each other. In other words, I posited that the intuitive distinction between integrated wholes and arbitrary sums of parts is reflected in grammar by the syntactic distinction between count singulars and regular plurals, respectively. In addition, the evidence from Italian irregular plurals shows that there are natural language expressions designating entities that are similar to plurals in that they comprise a number of integrated objects, but at the same time their sum is arranged in such a way that it constitutes a cluster.

Chapter \ref{ch:exploring-topological-sensitivity} was dedicated to the systematic exploration of the role the topological notion of integrity plays with respect to part-whole structures and subatomic quantification in particular. I provided evidence that shows that the distinction between topology-neutral and topology-sensitive partitive words is lexicalized in Polish. Specifically, Polish has three morphologically distinct half-words, i.e., \textit{połowa}, \textit{pół}, and \textit{połówka}, all `half'. When applied to a DP denoting an entity, they all yield its part constituting approximately 50\% of the whole. However, where they differ is the type of entity they select for as well as the type of part they return. Given different distributional and referential properties of each expression in question, I interpreted the alternation in terms of topological sensitivity. While the half-word \textit{połowa} is topology-neutral, i.e., it simply measures halves of any type of entity and yields a topologically indeterminate portion of a whole, \textit{pół} and \textit{połówka} are semantically more complex. Similarly to its topology-neutral counterpart, in the case of \textit{pół} the outcome of quantification is either a continuous or a discontinuous half. However, unlike \textit{połowa} it selects only for integrated wholes, i.e., individuated entities that come in one piece. On the other hand, the meaning of the morphologically most complex \textit{połówka} is even more restricted. While it shares selectional restrictions with \textit{pół}, in addition it imposes topological constraints on the extension of the resulting partitive construction. As a result, the whole partitive can only be true of a continuous integrated part constituting approximately 50\% of the volume of an individuated object, see \tabref{tab:denotations-of-polish-half-words2}.

\begin{table}[h!]
			\centering
			\begin{tabular}{lcc}
				\lsptoprule
				& \textsc{continuous part} & \textsc{discontinous part} \\ \midrule
				połowa  & $\checked$    & $\checked$      \\
				pół     & $\checked$    & $\checked$      \\
				połówka & $\checked$    & *                 \\ \lspbottomrule
			\end{tabular}
			\caption{Denotations of Polish half-words}\label{tab:denotations-of-polish-half-words2}
		\end{table}

Similar alternations are attested in different classes of other partitive expressions in Polish, e.g., piece-words and quarter-words. In each of the examined cases, the notion of integrity played a crucial role in predicting possible denotations of distinct types of partitives. Even more interestingly, cross-linguistic investigation revealed that the phenomenon observed is not a Polish idiosyncrasy. Instead, it appears in several other languages. Though the means particular languages employ to express certain spatial configurations differ, e.g., English utilizes determiners whereas Mandarin uses classifier semantics, the resulting effect is similar and demonstrates the relevance of the topological notion of integrity in subatomic quantification.    

In addition, I discussed the contrast between two types of Polish whole-adjec\-tives \citep[see][]{moltmann1997parts,morzycki2002wholes}, i.e., \textit{cały} `whole' and \textit{kompletny} `complete, which emphasizes two distinct aspects of wholeness, namely maximality and integrity. While the former involves both, the latter does imply that no part is missing but does not indicate that they remain in a particular topological configuration, see \ref{tab:interpretations-of-polish-whole-adjectives2}. The significance of the presented data lies primarily in revealing the relevance of topological relations holding between parts of individuals forcing us to recognize that natural language semantics is sensitive to whether parts come in one piece or constitute discontinuous entities. 

		\begin{table}[h!]
			\centering
			\begin{tabular}{lcc}
				\lsptoprule
				      & \textsc{maximality} & \textsc{integrity} \\ \midrule
				cały  & $\checked$    & $\checked$  \\
				kompletny & $\checked$   & * \\ \lspbottomrule
			\end{tabular}
			\caption{Interpretations of Polish whole-adjectives}
            \label{tab:interpretations-of-polish-whole-adjectives2}
		\end{table}

The final piece of data was discussed in Chapter \ref{ch:multipliers}. It concerns a neglected class of numerical expressions, namely multipliers such as English \textit{double}, that display non-trivial quantificational properties. Specifically, I provided evidence demonstrating that multipliers are specialized for subatomic quantification, i.e., they count cognitively salient parts that are essential for a whole to be considered having a certain property. In many cases, those parts are self-sufficient, i.e., have themselves a property comparable to that of the whole, but sometimes they are just intuitively the most salient elements. Though the distribution of multipliers involves also abstract entities denoted by event nominals as well as role nouns, I argued that the semantic behavior observed in combination with singular count nouns constitutes the basic quantificational mechanism that can be further extended to other types of entities. Interestingly, when the multiplier combines with the mass term, it forces a portion, i.e., count interpretation of the modified noun.

Furthermore, I discussed Slavic multipliers that display morphological complexity suggesting semantic compositionality. For instance, the Russian multiplier \textit{dvojnoj} `double' consists of the numeral root corresponding to the number 2 as well as some additional morphology including a special multiplicative affix. This fact indicates that multipliers share with cardinals reference to integers, but differ in that they are devised to count entities of a different type, see \tabref{tab:properties-of-cardinals-and-multipliers2}. In particular, cardinals are semantically equipped to count wholes. Though they can be used in subatomic quantification, e.g., in count explicit partitives, this is only possible when they combine with a partitive word. In such a case, however, entities designated by the whole partitive phrase are treated as objects in their own right. Hence, the source of subatomic quantification is the partitive, and the cardinal simply counts provided entities in their relative entirety. On the other hand, multipliers always quantify over parts of objects referred to by a modified nominal. Given the cross-linguistically widespread appearance of multipliers, the examined data set shows that analogously to cardinals, whose purpose is to count wholes, natural language developed expressions dedicated to numerical subatomic quantification.

    \begin{table}[h!]
    \centering
\begin{tabular}{lcc}
\lsptoprule
                           & \textsc{cardinals} & \textsc{multipliers}  \\ \midrule
subatomic quantification   & *               & $\checked$    \\
quantification over wholes & $\checked$               & *    \\ \lspbottomrule
\end{tabular}
\caption{Properties of cardinals and multipliers}
\label{tab:properties-of-cardinals-and-multipliers2}
\end{table}

In Chapter \ref{ch:conceptual-background}, I provided a general conceptual framework intended as a basis for developing a formal account for subatomic quantification in natural language. I suggested how the linguistic evidence discussed in Chapters \ref{ch:partitives-and-part-whole-structures}, \ref{ch:exploring-topological-sensitivity}, and \ref{ch:multipliers} could be linked with insights provided by cognitive psychology. Specifically, extensive research on perception and cognitive development indicates that since early childhood we conceptualize objects, i.e., integrated wholes, in a different way than other types of entities. Furthermore, even young children possess the ability to simultaneously perceive a whole as an object in its own right, as well as a collection of parts, and this capacity guides their linguistic development. Finally, human number sense seems to be sensitive to the intuitive notion of integrity. Specifically, young children always count each separate integrated physical entity as one. 

Based on both linguistic and psychological evidence, I presented three claims constituting the conceptual core of the book. I postulated that natural language is sensitive to the topological notion of integrity, which manifests itself in how nominal lexicon is classified into different grammatical categories. For instance, count singular nouns designate entities conceptualized as integrated wholes. On the other hand, plural nominals presuppose parts of their referents to be integrated wholes but impose no topological constraints on them, i.e., denote arbitrary sums of individuals. The second claim concerns the general counting principles, i.e., a set of rules constraining what can be counted. Specifically, the principle of non-overlap ensures that enumerated entities are disjoint, i.e., each thing can be counted once and once only. The principle of maximality demands that counting involves entities in their mereological entirety, i.e., no part is left out. Finally, the principle of integrity guarantees that an entity that can be put in a one-to-one correspondence with numbers needs to be conceptualized as an object that comes in one piece. All things considered, the general counting principles ensure that only sets consisting exclusively of elements that are conceptualized as discrete integrated objects can be enumerated. Importantly, there is no reference to atomicity in the postulated quantificational mechanism. Finally, the third claim extends the general counting principles to subatomic quantification. I posit that the set of constraints described above is a universal mechanism applicable both at the level of wholes and at the level of parts.

In Chapter \ref{ch:theory-of-parts-and-wholes}, I introduced a theory of wholes called mereo\-topology which enhances standard mereology based on the parthood relation with additional topological features (\citealt{casati_varzi1999parts}; for linguistic application, see \citealt{grimm2012degrees,grimm2012number}). Within this framework, the primitive notion of connectedness enables us to derive complex notions capturing distinct configurations of parts within a part-whole structure. One such notion is a property of being maximally strongly self-connected (\textsc{mssc}), which allows us to capture an intuition on what it means to be an integrated whole. Specifically, an \textsc{mssc} entity is an object whose every part overlaps the whole, and anything else that has a relevant property and overlaps it is part of it. Such a perspective seems to correspond neatly to the intuitive view on individuals as configurations of parts arranged in a certain manner rather than atomic elements with no proper parts, and thus constitutes an advantageous alternative for accounts based on atomicity. This alternative view turns out to be of significant value in modeling quantification over both wholes and parts.

Finally, in Chapter \ref{ch:mereotopological-account-for-subatomic-quantification} I developed a formal account for subatomic quantification based on the conceptual background and mereo\-topological notion of \textsc{mssc} discussed above. In particular, I argued that an approach building on \textsc{mssc} entities instead of atoms is more auspicious in modeling quantification over parts. For instance, a significant advantage of a mereotopological account is that it does not require postulating sorted domains with different parthood relations defined over them \citep{link1983logical}. The distinction between part-whole structures associated with singular count nouns and regular plurals can be recast in terms of connectedness. While the former encode that the elements of their denotations are \textsc{mssc} entities, the latter require parts of their referents to be such individuals but impose no topological constraints on pluralities thereof. Furthermore, I proposed that numerical expressions including cardinals and multipliers are complex semantic expressions derived from names of numbers via different overt or covert classifiers specialized in counting distinct types of objects, e.g., \textsc{mssc} entities and cognitively salient parts thereof, respectively. The last piece of the puzzle concerned how to capture different semantic interpretations of distinct types of topology-neutral and topology-sensitive partitive words. For this purpose, I proposed a special individuating element that can be either expressed formally, e.g., as a special suffix on Polish topology-sensitive partitive words, or remain null. In any case, what this individuating element does is that it partitions the set of entities denoted by a partitive and introduces the \textsc{mssc} constraint relative to a given partition. This guarantees that the extension of a whole partitive involves only non-overlapping continuous parts. Such integrated parts are conceptualized as objects in their own right within a part-whole structure of an individual, and thus can be quantified over, similarly to other \textsc{mssc} entities. With these components in place, I provided step-by-step derivations of multiplier phrases as well as different types of partitives that account for distinct interpretations of constructions involving topology-neutral and topology-sensitive partitive words. In the proposed system, distinct semantic effects arise as a result of different interactions between partitivity, topology, and quantification. Its great advantage is that it provides an explanation for the topological phenomena concerning reference to continuous and discontinuous parts in different types of partitives, an issue that previous approaches failed not only to account for but even to acknowledge \citep[e.g.,][]{chierchia2010mass}.

\section{Open questions and future research}\label{sec:open-questions-and-future-research}

The account developed in this book provides an explanation for a number of phenomena in natural language. I believe its contribution is valuable both from a theoretical perspective and in terms of data it covers. However, some questions remain and many issues require further investigation. One of the most serious challenges concerns the interpretation of count explicit partitives with the embedded plural DP, as discussed in \sectref{sec:partitivity-and-countability}. Unlike standard explicit set partitives, such expressions only get a part-of-a-singularity reading. I argued that this effect is due to the general counting principles which require that a counted entity constitutes an integrated whole. Since pluralities fail to satisfy this condition, the only possibility how a relevant phrase could be interpreted involves quantification over material parts. Nonetheless, the issue concerning the interpretation of the morpho-syntactic plural on the downstairs nominal remains open. The question of what the exact semantic contribution of number in such a structure is (if any) requires further investigation. It might be the case that in such constructions the plural is semantically void. However, there is the intuition that within a set of parts denoted by a count explicit partitive each element is part of a different object. At this point, I can only hypothesize that the mechanism deriving this effect is pragmatic in nature. However, it might also be the case that the plural is interpreted as a distributive operator, analogously to what has been postulated to account for covariational uses of \textit{different} \citep[see, e.g.,][]{beck2000semantics,brasoveanu2008sentence,dotlacil2010anaphora}. Before a detailed analysis could be proposed, definitely more research is required.

As indicated in \sectref{sec:possible-extensions}, the proposed account can be easily extended to topology-sensitive partitives in other languages than Polish and German as well as to other types of constructions involving subatomic quantification. However, the question whether such an extension is empirically appropriate will in each case require careful consideration. After examining the cross-linguistic data presented in  \sectref{sec:cross-linguistic-parallels}, at first sight it seems plausible that, e.g., in English explicit and proportional partitives it is the determiner that introduces the individuating element. It appears that the same semantics is associated with a special classifier that follows a partitive word in Mandarin. However, more tests need to be applied to reveal the real nature of the interaction between topology, partitivity, and the determiner or classifier semantics, respectively. 

A further issue concerns a detailed analysis of Italian irregular plurals \citep{ojeda1995semantics,acquaviva2008lexical}. Though the data from count explicit partitives involving such expressions were of great significance for capturing the nature of the interaction between partitivity and numerical quantification, the proper semantic treatment of such constructions would require accommodating more complex topological notions. Although this task calls for another enterprise, I believe this study provides a neat framework for pursuing it, see  \sectref{sec:italian-partitives-with-irregular-plurals}. A potentially interesting topic concerns the investigation into partitives in languages displaying the collective-singulative number alternation \citep{grimm2012degrees,grimm2012number}. Moreover, if the proposed analysis is on the right track, effects similar to what we have observed in partitives with irregular plurals in Italian should also be detectable in Slavic partitives with topologically sensitive derived mass nouns that denote a plurality of discrete objects forming a cluster \citep{grimm_docekal-toappear-counting} as well as swarm nouns \citep{henderson2017swarms}. A pilot study concerning Czech and Polish suggests that this is in fact the case, yet more work needs to be done before a conclusion can be drawn.

So far, whole-adjectives have not received a lot of attention in the semantic literature. In this book, I argued that different types of Polish expressions of this kind provide evidence for distinguishing between maximality and integrity, see \sectref{sec:integrity-vs-maximality}. However, a detailed analysis of the meaning of whole-adjectives awaits to be developed. As a starting point, one could consider an interaction between universal quantification at the subatomic level and the individuating element encoding topological integrity or some similar component. The idea that whole-adjectives could be interpreted as universal quantifiers over parts has already been proposed \citep{moltmann1997parts}. Nevertheless, careful investigations into scope dependencies and related factors are required in order to test whether such treatment is empirically plausible (see \citealt{morzycki2002wholes} for potentially problematic data).

Another area where the proposed perspective could be applied concerns adverbials quantifying over parts of a singularity such as \textit{mostly}, \textit{wholly}, \textit{largely} \citep[see][]{morzycki2002wholes} as well as \textit{all wet} \citep[see][pp. 162--170]{schwarzschild1996pluralities}. Such explorations would most probably require extending the account advocated here with the insights of the prolific research on scale structures \citep[e.g.,][]{kennedy_mcnally1999event,kennedy_mcnally2005scale}. If both approaches turn out to be compatible, it potentially opens up a new perspective on the opposition between partial and total predicates, e.g., \textit{dirty}, \textit{wet}, and \textit{touch}, as opposed to \textit{clean}, \textit{dry}, and \textit{eat} \citep[e.g.,][]{yoon1996total,rotstein_winter2004total}.

A separate subject concerns the interpretation of multipliers with event nominals, role and abstract nouns as well as collectives. In  \sectref{sec:less-obvious-cases}, I suggested that extending ontology with additional types for events as well as social roles will allow us to extend the basic mechanism proposed to account for the semantic behavior of multipliers modifying concrete nouns to more abstract types of objects. However, the exact implementation of the idea requires careful consideration and rigorous study. It might turn out that a more general notion is necessary or, alternatively, that several derived concepts will prove more advantageous in tackling the distribution of multipliers across the board. In any case, figuring this out seems to be an intriguing challenge that would almost certainly provide a novel perspective on quantification in natural language.

If the approach I have argued for is correct, I expect that scrupulous cross-linguistic investigation will reveal even more types of constructions indicating the relevance of topological relations within part-whole structures. Though a number of directions worth examining has already been indicated, e.g., adjectival half-words in German and Romance, I believe that there are even more parts of the lexicon that are sensitive to topological relations at the subatomic level. One potential example is a class of expressions I refer to as verbs of separation such as \textit{separate}, \textit{dismantle}, and \textit{dissolve}, mentioned briefly in  \sectref{sec:aggregate-meaning}. At first blush, their meaning seems to affect the topological component of part-whole structures. Examining them from the perspective developed here might be a promising enterprise that would shed light on subatomic quantification in the verbal domain. It seems that an especially auspicious area of research concerns Slavic verbal morphology with its various prefixes and aspectual distinctions \citep[e.g.,][]{filip1999aspect}.

\begin{sloppypar}
Furthermore, I believe that rigorous experimental investigation could also shed new light on subatomic quantification, as it did in other areas of the field. For instance, what would be of significant interest is the exploration of the relative strength of topological inferences in different types of constructions cross-linguistically. It might be informative to compare structures that encode the individuating element formally such as Polish and German topology-sensitive proportional partitives with constructions where a similar effect arises pragmatically. 
\end{sloppypar}

To conclude, I hope that this study provides a valuable perspective on previously neglected semantic phenomena in natural language. Though I presented a number of novel observations concerning distinct types of relatively well-studied constructions as well as genuine insights concerning expressions that so far have not received attention in the study of meaning, I expect that many more obscure regions await to be explored. I believe that a detailed map of the phenomena related to subatomic quantification in natural language can only emerge as a result of careful systematic typological as well as experimental research. This is, however, where other journeys are to begin.
