\addchap{\lsAcknowledgementTitle} 

This monograph is based on my Ph.D. thesis defended at the Department of Linguistics and Baltic Languages at the Masaryk University in Brno under the supervision of Mojmír Dočekal in 2019 (committee members: Pavel Caha, Jakub Dotlačil, Scott Grimm, Jiří Raclavský, Viola Schmitt, Ondřej Šefčík, and Ludmila Veselovská). It received the E.\,W. Beth Dissertation Prize awarded by the Association for Logic, Language and Information, as well as the Dean’s Award for the Best Ph.D. Thesis at the Masaryk University in Brno. It is also one of the outcomes of the research project \textit{Formal Approaches to Number in Slavic} (GA17-16111S; \url{https://sites.google.com/view/number-in-slavic/home}), funded by the Czech Science Foundation (GAČR) and carried out at the Department of Linguistics and Baltic Languages at the Masaryk University in Brno, in cooperation with researchers from the Center for Language and Cognition at the University of Groningen, the Department of German Studies at the University of Vienna, and the Center for Experimental Research on Natural Language at the University of Wrocław.

This book is a result of a collaborative effort and I would have never finished it without the inspiration, support, and advice of many people I met in the last several years. First of all, I would like to thank Mojmír Dočekal for introducing me to how fascinating linguistics can be and raising my initial interest in the subject of pluralities and quantification. I would have never become a semanticist if I had not met him. Second, I am deeply grateful to Viola Schmitt for so many things that it is hard to list them. She convinced me that my idea of accounting for subatomic quantification is actually interesting and constantly kept me motivated during the whole process of writing my thesis. Her constructive criticism and inspiring comments were of incalculable value and this book would have never come into existence if it weren't for her support and advice. Third, I would like to say ``thank you'' like one million times to Berit Gehrke who was the handling editor of this book. She did much more for it than I could have ever imagined. I wish everyone such a great editor. Next, I am very thankful to Nina Haslinger with whom I had many thought-provoking discussions concerning the data, theoretical implications, and formalism. Without those conversations this book most definitely would not be what it is. Furthermore, I am also very grateful to the reviewers of this book, Adam Przepiórkowski and Peter Sutton, whose detailed, inspiring, and often critical comments pushed me to improve the content of this monograph so that it is much better than it could have been. Their feedback was invaluable. I would also like to thank Scott Grimm who inspired me to radically change my way of thinking about nominal semantics. Finally, I would wish to acknowledge the technical support of the entire Language Science Press editorial team as well as the help of Radek Šimík.

The process of clarifying what this book should be about and how to achieve its aim was long. During that time I stayed at several academic locations, each of which had a considerable impact on my linguistic development. Therefore, I would like to thank the following people to whom I owe for what I know now (always in alphabetical order). Many thanks to my friends and colleagues at the Masaryk University in Brno, especially Pavel Caha, Hana Strachoňová, and Markéta Ziková for their support and making Brno a stimulating place to do linguistics. Next, I am very grateful to the people with whom I interacted during my stay at the Humboldt-Universität zu Berlin, namely Manfred Bierwisch, Aleksandra Gogłoza, Robert Hammel, Denisa Lenertová, Manfred Krifka, Roland Meyer, Uli Sauerland, Stephanie Solt, Luka Szucsich, Kazuko Yatsushiro, Karolina Zuchewicz, and especially Radek Šimík. Due to those discussions I have clarified many vague ideas and learned a great deal of semantics. During two stays at the University of Vienna I met a number of people with whom discussing linguistics was so much more than I had ever experienced. Specifically I would like to thank Muriel Assmann, Daniel Büring, Enrico Flor, Izabela Jordanoska, Chiara Masnovo, Martin Prinzhorn, Maximilian Prüller, Magdalena Roszkowski, Jakob Steixner, and Ede Zimmermann for helping me discover so many new things and realize what I want to say and how. Finally, for a long time parallel to my Ph.D. program I was working at the Palacký University in Olomouc. During that period, I have learned much due to interactions with Jakub Bortlík, Petra Charvátová, Joseph Emonds, Anna Kozánková, Jeffrey Parrott, and Lída Veselovská. I am also grateful for support to Ivana Dobrotová, Michał Hanczakowski, and Jan Jeništa with whom I worked at that time. I would also like to sincerely thank Jakub Dotlačil, Berit Gehrke, and Yasu Sudo whom I met several times at different occasions during my PhD program. Their comments and inspiring insights as well as interesting discussions were always of great help. 

At different conferences, workshops, and other events I met people whose questions and comments were of extreme importance for developing my understanding of the phenomena described in this book as well as realizing problems I have not even noticed before. Sometimes it was just a question or a remark, sometimes it was a longer discussion or even several conversations. In each case, however, it stimulated me to clarify what I actually want to say. Other people helped me a lot as informants providing examples from their native languages. Therefore, I would like to thank David Adger, Curt Anderson, Boban Arsenijević, Petr Biskup, Joanna Błaszczak, Jonathan Bobaljik, Natalie Boll-Avetisyan, Olga Borik, Ana Bosnić, Roslyn Burns, Lisa Bylinina, Bożena Cetnarowska, Lucas Champollion, Guglielmo Cinque, Miloje Despić, Dominika Dziubała-Szrejbrowska, Regine Eckardt, Kurt Erbach, Katalin É. Kiss, Hana Filip, Salvatore Florio, Katherine Fraser, Ljudmila Geist, Dariusz Głuch, Julie Goncharov, Justyna Grudzińska, Veronika Hegedűs, Gillen Martinez de la Hidalga, Daniel Hole, Krzysztof Hwaszcz, Tetiana Kamyshanova, Jiří Kašpar, Hanna Kędzierska, Dorota Klimek-Jankowska, Predrag Kovačević, Jakub Kozakoszczak, Ivona Kučerová, Fred Landman, Daniel Lassiter, Tomas Lentz, Chang Liu, Aitor Lizardi Ituarte, Sebastian Löbner, Marijana Marelj, Radvan Markus, Lanko Marušič, Ora Matushansky, Louise McNally, Erlinde Meertens, Katarzyna Miechowicz-Mathiasen, Petra Mišmaš, Maša Močnik, Marcin Morzycki, Olav Mueller-Reichau, Andrew Nevins, David Nicolas, Rick Nouwen, Itziar Orbegozo Arrizabalaga, Agnieszka Patejuk, Roumyana Pancheva, Asya Pereltsvaig, Lilla Pintér, Hagen Pitsch, Adam Przepiórkowski, Wiktor Pskit, Zorica Puškar, Agata Renans, Susan Rothstein, Marta Ruda, Paweł Rutkowski, Piotr Stalmaszczyk, Balázs Surányi, Branimir Stanković, Peter Sutton, Iveta Šafratová, Guy Tabachnick, Barbara Tomaszewicz, Jan Winkowski, Jacek Witkoś, Jan Wiślicki, Kata Wohlmuth, Henk Zeevat, Sarah Zobel, Yulia Zinova, and Eytan Zweig. All errors are of course my own.

Finally, many thanks to my family and friends for their support and not asking too often whether I'm already done. In particular, I am grateful to my Mum and Dad for teaching me how to think independently, to Piotr and Zosia, Herok, Bacha, Byrtek, and the Gwizdońs. Last, but not least, I am deeply thankful to Izabela for her love and unconditional support as well as tons of delicious plum cake that kept me going during hard times. In the last several years, we have achieved together so much more than this book.

\null

\hfill Marcin Wągiel

\hfill Brno, 13 July 2021

 
 


