\documentclass[output=paper]{langsci/langscibook} 
\ChapterDOI{10.5281/zenodo.5675859}
\author{Kirsten Jeppesen Kragh\affiliation{University of Copenhagen}}
\title[The importance of paradigmatic analyses]{The importance of paradigmatic analyses: From one lexical input into multiple grammatical paradigms}
\abstract{I take as my starting point that when lexical entities grammaticalize, they enter preexisting paradigms. Therefore, grammatical paradigms are important for the understanding of the reanalyses leading to grammaticalization. In the line of Henning Andersen’s thinking I propose to conceive of grammar as composed of sets of paradigms \citep{Nørgård-Sørensen2011}. The term \emph{paradigm} is used not in the narrow sense of inflectional paradigm, nor entirely in the line of the ``classical'' grammaticalization approach of \citet{Lehmann1985}, but in the more general sense of a selectional set, composed of marked or unmarked members \citep[19]{Andersen2008}. The lexical input that I use to illustrate my point is the French verb \textit{voir} ‘to see’, in order to show the pathway of a multifunctional lexical item into grammar, i.e. into a number of individual paradigms.

My approach combines synchronic and diachronic investigations on electronic corpora. Each paradigm presents the actual synchronic status of diachronic grammaticalization processes. By distinguishing the different contexts (labelled domains) in which the given forms appear, and state which semantic fields they cover (labelled frames), I can generate synchronic paradigms of which the grammatical entities are members. I aim to demonstrate that synchronic paradigms provide a precise and relatively simple presentation of what otherwise would seem utterly diverse usages of a lexical entity.}

\begin{document}
\maketitle 

\section{Introduction} \label{kragh:1}

I\footnote{An extensive part of this work has been done in collaboration with Lene Schøsler. I am deeply grateful for her valuable suggestions and comments on this article.} take as my starting point that when lexical entities grammaticalize, they enter pre-existing or new grammatical paradigms. Therefore, paradigms are important for the understanding of the reanalyses leading to grammaticalization.  

In the line of Henning Andersen’s thinking I propose to conceive of grammar as composed of sets of paradigms \citep{Nørgård-Sørensen2011}. I here use the term \textit{paradigm} not in the narrow sense of inflectional paradigm, nor entirely in the line of the ``classical'' grammaticalization approach of  \citet{Lehmann1985}, but in the more general sense of a selectional set, composed of marked and unmarked members \citep[19]{Andersen2008}. In previous studies \citep{KraghSchosler2014, KraghSchosler2015, KraghSchosler2016, KraghSchosler2019, KraghSchosler2020} I have shown that the notion of a paradigm is useful for the understanding of grammatical structure. 

In order to show the pathway of a lexical item into grammar, i.e. into a number of different paradigms, I will use the lexical French verb \textit{voir} ‘to see’ and the derived phrases, \textit{voici} and \textit{voilà} to illustrate my point. When aiming at analysing a polysemous and multifunctional lexical entity like this verb, the researcher can choose between a polysemic or a monosemic approach, see \citet{Waltereit2002, Waltereit2006}. But when studying how a lexical item grammaticalizes, I do not consider this discussion to be the essential one. Rather, I think that the most interesting point is how the lexical item enters different grammatical paradigms.

\begin{sloppypar}
My approach combines synchronic and diachronic investigations on electronic corpora. Each paradigm shown in the following presents the synchronic results of diachronic grammaticalization processes, based on synchronic paradigmatic analyses of very different functions. The paradigms comprise the following grammatical categories: verbal categories: tense, aspect and mood, i.e. progression (\textit{je le vois qui arrive} ‘I see him arriving’), voice (\textit{il se voit refuser l’accès} ‘he is refused entrance’), pragmatic categories: presentatives (\textit{voilà le bateau} ‘here is the boat’), focus constructions (\textit{voici le bateau qui arrive} ‘here is the boat arriving’), and discourse markers (\textit{le bateau arrive, tu vois} ‘the boat is arriving, in fact’), and the category of particles, i.e. the preposition \textit{vu} ‘considering’ and the subordinate conjunction \textit{vu que} ‘considering that’. I shall illustrate my paradigmatic approach by means of three cases: the progressive paradigm, the presentative paradigm, and the focalization paradigm. I aim to show that the progressive and the presentative paradigms are preconditions for \textit{voir}’s entrance in the focalization paradigm. By distinguishing the different contexts (labelled \textit{syntactic} \textit{domains}) in which the forms appear, and state which semantic fields they cover (labelled \textit{frames}), I identify synchronic paradigms of which the above exemplified grammatical entities are members. I claim that synchronic paradigms provide a precise and relatively simple presentation of what otherwise would seem utterly diverse usages of a lexical entity.\footnote{An alternative way of presenting a selection of diverse usages is the lexical approach provided in dictionaries. Please, see \url{https://robert-correcteur.lerobert.com} for an illustration of a rather unsystematic presentation with a mix of lexical and grammatical information on the different functions of \textit{voir} in the digital dictionary Robert Connecteur.}
\end{sloppypar}


\section{Definition of paradigmatic level} \label{kragh:2}

The grammatical paradigm can be identified through a set of five defining features \citep[5--6]{Nørgård-Sørensen2011}.\footnote{This section is a revised version of section 2.1 in \citet{KraghSchosler2015}.}

First, the grammatical paradigm is a \textit{closed set of items}, the number of members being fixed at a given language stage.\footnote{In fact, over time most paradigms change their member list, so a \textit{given language stage} is an abstract notion to be defined for each paradigm at a specifically defined synchrony. Since languages are always changing, it is not an easy task to identify the relevant synchronic stages permitting the establishment of a paradigm, without the risk of circular argumentation.} In Modern French, for instance, the category of tense, aspect, and mood (TAM) comprises the simple verb forms present, past, future tenses: \textit{voit}, \textit{vit}/\textit{voyait}, \textit{verra/verrait}, and a number of analytical forms.

Secondly, for every paradigm, the \textit{syntactic} \textit{domain}, i.e. the syntagmatic context in which it applies must be specified. Thus, in Modern French, the domain of the category TAM is the simple or composed finite verb form of a given sentence.

Thirdly, a paradigm has a \textit{semantic} \textit{frame}, i.e. a common semantic denominator, here tense, aspect and mood, within which the content of its members is defined in opposition to one another. Semantic frames are language specific and cannot be taken to be equivalent to the cognitive networks of \citet{Croft2001} and \citet{CroftCruse2004}.

Fourthly, the choice between the members is \textit{obligatory}, in the sense that in the given syntagmatic context that defines the domain of the paradigm, speakers cannot avoid selecting one of the members; they must choose for example one verbal form -- simple or analytical, the present, the past, the future, etc. -- to express the temporal and aspectual dimension of a given utterance.

\begin{sloppypar}
Fifthly, a paradigm is \textit{asymmetric}, distinguishing between marked and unmarked members, possibly in a hierarchical structure. In Modern French, the present tense is the unmarked form, because it appears in both perfective and imperfective contexts and displays such values as present, habitual, progressive, recent past, near future, etc. Compared to the present tense, the simple past, the imperfect, the future, and the conditional are all marked, i.e. restricted, both in regard to the type of context in which they appear, and in regard to their temporal and aspectual values.
\end{sloppypar}

Whereas \textit{the domain} refers to the syntagmatic delimitation of the paradigm on the expression level, \textit{the frame} is its counterpart on the content level. A paradigm is not a pure expression system, but a sign system with the domain-and-frame pair constituting a unity of expression and content (according to the terminology of the Danish Functional tradition, see \citealt{Engberg-PedersenEtAl1996}).

\section{Lexical level: Level 0} \label{kragh:3}

Before studying the processes of grammaticalization, we must start at the lexical level, i.e. the valency level with its possible constituents in free use. It is crucial here to distinguish the lexical level from the construction level; it is the former that provides the input to a grammaticalization process.

\textit{Voir} is a highly polysemous verb, characterized by its frequency in usage. Such high frequency verbs are likely to grammaticalize or pragmaticalize (cf. \citealt[674--675]{Bolly2010}). 

This makes \textit{voir} an obvious choice as object in an attempt to demonstrate how one single lexical item tends to enter a number of new constructions, thus becoming member of a wide range of grammatical paradigms. Each new usage of \textit{voir} being routinized is a candidate for entering a new paradigm. \citet{Craig1991} introduces the term \textit{polygrammaticalization} to refer to this phenomenon where one particular lexical entity is the source of multiple grammaticalization chains.

As I see it, the function of a given lexical item in a given context triggers the specific meaning of this lexical item. Thus, when \textit{voir} is used in new contexts, it is ascribed new functions (e.g. presentation, focus, progression, passive voice, discourse marker, etc.); they provide new meanings. I have shown in previous studies that secondary features of the original construction turn into primary features during the process of grammaticalization (cf. for instance \citealt{KraghSchosler2015}). In what follows, I will first present the lexical level (level 0), and subsequently a number of paradigms into which forms of the verb \textit{voir} have entered as a result of the grammaticalizations.

\begin{figure}
%\includegraphics[width=\textwidth]{figures/Kragh-fig1.png}
\begin{tikzpicture}
  \tikzstyle{every node}=[rectangle,draw,text width=4.7cm,align=center]
  \node(0){Level 0\\Lexical level\\Valency-bound constituents};
  \node(A1)[below left=2\baselineskip and 5mm of 0.south]{Level 1};
  \node(A2)[below right=2\baselineskip and 5mm of 0.south]{Level 1};
  \draw[->](0)-|(A1.north);
  \draw[->](0)-|(A2.north);
\end{tikzpicture}
 \caption{Lexical level (level 0)\label{fig:kragh:1}}
\end{figure}

As indicated in \figref{fig:kragh:1}, level 0 is lexical, it comprises the valency-bound constituents that combine with the verb \textit{voir}, e.g. noun phrases \REF{ex:kragh:1}, possibly with a subordinate relative clause \REF{ex:kragh:2}, complement clauses \REF{ex:kragh:3}; a number of nexus constructions,\footnote{The term \textit{nexus} describes the relation of interdependency with the antecedent/referent in contrast e.g. to subordinate relative clauses. This implies that the antecedent/referent cannot be omitted, e.g. *\textit{Je vois perdu}/\textit{président} etc.}  with infinitive \REF{ex:kragh:4}, with deictic relative clause \REF{ex:kragh:5}, with present or perfect participles, \REF{ex:kragh:6} and \REF{ex:kragh:7}, with adjectives \REF{ex:kragh:8}, with nouns \REF{ex:kragh:9}, or with a prepositional phrase as object complement, \REF{ex:kragh:10} and \REF{ex:kragh:11}: 

\begin{exe}
    \ex \label{ex:kragh:1} Je vois Jean. \\
    `I see John.'

    \ex \label{ex:kragh:2}
    Je vois la maison qui est rouge. \\
    ‘I see the house which is red’
    
    \ex \label{ex:kragh:3} Je vois que Jean arrive\\
    ‘I see that John arrives’

    \ex \label{ex:kragh:4} Je le vois arriver\\
    ‘I see him arrive’

    \ex \label{ex:kragh:5} Je le vois qui arrive\\
    ‘I see him arriving’

    \ex \label{ex:kragh:6} Je le vois jouant le football\\
    ‘I see him playing soccer’

    \ex \label{ex:kragh:7} Je le vois perdu\\
    lit. ‘I see him lost’

    \ex \label{ex:kragh:8} Je le vois heureux\\
    lit. ‘I see him happy’

    \ex \label{ex:kragh:9} Je le vois président\\
    lit. ‘I see him being president’

    \ex \label{ex:kragh:10} Je le vois en bonne humeur\\
    ‘I see him in a good mood’

    \ex \label{ex:kragh:11} Je le vois en vainqueur/comme vainqueur\\
    lit. ‘I see him the winner’
\end{exe}

No other verb of visual perception has such a large number of possible constituents \citep[10]{WillemsDecfrancq2000}. 

We now proceed from the lexical level to the levels of grammatical paradigms.

\section{Paradigms: Levels 1 to 3} \label{kragh:4}
\subsection{Level 1: Reanalysis of the subordinate relative clause into the deictic relative clause} \label{kragh:4.1}

Levels 1 and 2 and later on also level 3 illustrate the use of \textit{voir} in contexts where the original meaning of the verb \textit{voir} is bleached due to a number of reanalyses and grammaticalizations.

Level 1 comprises the initial reanalysis and grammaticalization that are a prerequisite of the grammaticalizations at level 2 and 3.

At level 1 I find the grammaticalization of the deictic relative clause due to a reanalysis of the relative subordinate (see \citealt{KraghSchosler2014, KraghSchosler2015}), i.e. reanalysis of level 0. This is an important step for the subsequent grammaticalizations that involve \textit{voir} in a progressive context (level 2).

\begin{figure}
%\includegraphics[width=.5\textwidth]{figures/Kragh-fig2.png}
\begin{tikzpicture}
  \tikzstyle{every node}=[rectangle,draw,text width=9cm,align=center]
  \node(A1){Level 1\\Reanalysis of the deictic relative as construction};
  \node(B1)[below=of A1]{Level 2\\Paradigm~1 };
  \node(0)[above=of A1]{Level 0\\Lexical level\\valency-bound constituents};
  \draw[->](0.south)-|(A1.north);
  \draw[->](A1)--(B1);
\end{tikzpicture}
\caption{From lexical level to grammatical level\label{fig:kragh:2}}
\end{figure}

In example \REF{ex:kragh:2}, we had a clear subordinate relative clause which was part of the NP. In certain contexts, such subordinate clauses are reanalysed. I have found bridging examples in which the hearer may interpret the message of the relative clause in two different ways, see example \REF{ex:kragh:12}:

\ea \label{ex:kragh:12} S’i erent venu apoier;/quant le cuens vit son escuier/qui sor le noir destrier estoit \\
    ‘they have come to rest there/when the count saw his horseman/who was sitting on the black horse’\\
    (Les romans de Chrétien de Troyes, Erec et Enide, ca. 1213, p. 98, vers 3207–3210, Frantext)
\z


The interpretation of this example is twofold. The point may be that the count firstly catches sight of his horseman and subsequently discovers him sitting on the black horse. However, another interpretation is also possible, providing a bridging or critical context, which permits reanalysis because of the ambiguity (see \citealt[117]{Diewald2002}, \citealt{Heine2002}), namely a holistic (progressive) perception of the horseman sitting on the horse.

I will consider the second interpretation of example \REF{ex:kragh:12} to be the result of the speaker reanalysing the subordinate type of relative clause, in the following way: A (subordinate relative clause specifying an NP) > B (deictic relative construction), i.e. into a new type of verbal complementation, without immediate change of the surface manifestation. This implies that the construction has acquired not only a) a different function, which is not a subordinate, but a nexus relation, but also b) a different meaning. This meaning has been described tentatively in terms of a holistic vision. Moreover, this vision is progressive, by which term I refer to an ongoing process performed by the referent of the direct object of the verb of perception.\footnote{For a more detailed description of the origin of the deictic relative, I refer to \citet[178--182]{KraghSchosler2014}.}

 \subsubsection{From level 1 to level 2, paradigm 1} \label{kragh:4.1.1}


As the deictic relative construction (B) is accepted in the speech community and increases its use, it is \textit{embedded,} i.e. integrated into grammar (Herzog et al., 1968:  185). Once embedded, it can be considered as yet another way of expressing progressivity: (\textit{Je vois}) \textit{Pierre qui chante}, ‘I see Peter singing’, cf. \figref{fig:kragh:3}.

  
\begin{figure}
%\includegraphics[width=.5\textwidth]{figures/Kragh-fig3.png}
\begin{tikzpicture}
  \tikzstyle{every node}=[rectangle,draw,text width=7cm,align=center]
  \node(A1){Level 1\\Reanalysis of the deictic relative as construction};
  \node(B1)[below=of A1]{Level 2\\Paradigm~1\\Progression};
  \node(0)[above=of A1]{Level 0\\Lexical level\\valency-bound constituents};
  \draw[->](0.south)-|(A1.north);
  \draw[->](A1)--(B1);
\end{tikzpicture}
 \caption{From level 0 to paradigm 1\label{fig:kragh:3}}
 \end{figure}

Paradigm 1 shows the inventory of the progressive constructions in French in form of a paradigm. Please note that this paradigm has a diachronic dimension. From the early times until the end of the 17\textsuperscript{th} century, \textit{Pierre va chantant} is the unmarked form, also diatopically. \textit{Pierre est à chanter} and \textit{Pierre est après chanter} are diatopically and diastratically marked, whereas \textit{Pierre est en train de chanter} turns into the unmarked form of progression from the 19\textsuperscript{th} century. The type \textit{Je vois Pierre qui chante} is a marked member of the paradigm from the 17\textsuperscript{th} century: They all mean `Peter is singing'. It is the only member that provides a holistic perception of the activity, perceived in its progression, and which has a different referent for the subjects of the two verbs ($\text{S1} \neq \text{S2}$).

\begin{table}
\fittable{\begin{tabular}{lll}
\lsptoprule
\multicolumn{3}{l}{\emph{Syntactic domain}: V1+V2, S1=S2/S2${\neq}$S2; \emph{Semantic frame}: +Progressivity}\\
{Diachr. perspective} & {Expression} & {Content}\\\midrule
unmarked form\footnote{${\rightarrow}$1700} & \textit{Pierre va} \textit{chantant} & S1=S2, +progressivity, ±durativity\\
marked form & \textit{Pierre est à chanter} & S1=S2, +progressivity, +durativity\\
marked form & \textit{Pierre est après chanter} & S1=S2, +progressivity, +durativity\\
unmarked form\footnote{1800${\rightarrow}$} & \textit{Pierre est en train de chanter} & S1=S2, +progressivity, +durativity\\
marked form & \textit{Je vois Pierre qui chante} & S1${\neq}$S2, +progressivity,  ±durativity,\\ & &  holistic vision\\
\lspbottomrule
\end{tabular}}
\caption{Progressivity in French, progressive constructions (Kragh \& Schøsler, 2015:  290)\label{tab:kragh:1}}
\end{table}

\subsubsection{From level 1 to level 2, Paradigm 2} \label{kragh:4.1.2}


Let us now consider another paradigm, that of the fossilized imperative form of \textit{voir,} \textit{voici}/\textit{voilà}, reanalyzed as member of the presentation paradigm.  

French has a number of ways of expressing presentation: \textit{c’est X}, \textit{il y a X}, \textit{il est X}, \textit{voici/voilà X}, \textit{X} being the presented entity. These constructions have been examined in \citet{KraghSchosler2019}, where the inventory of presentation is discussed and a list has been established in accordance with \citet{Lambrecht2000, Lambrecht2001} and \citet{RiegelEtAl2009}:\\\\
\noindent\textit{C’est X:}

\begin{exe}
    \ex \label{ex:kragh:13} Ce n’est pas eux\\
    ‘It’s not them’
\end{exe}

\noindent\textit{Il y a X:}

\begin{exe}
    \ex \label{ex:kragh:14} Il y a quelqu’un\\
    ‘There is someone’
\end{exe}

\noindent\textit{Il est X:}

\begin{exe}
    \ex \label{ex:kragh:15} Il était une fois une belle princesse\\
    ‘Once upon a time there was a beautiful princess’
\end{exe}

\noindent\textit{Voici/voilà:}

\begin{exe}
    \ex \label{ex:kragh:16} Voilà une belle fleur\\
    ‘Here is a beautiful flower’

    \ex \label{ex:kragh:17} Voici mon ami Pierre\\
    ‘Here is my friend Peter’

    \ex \label{ex:kragh:18} Voilà ma soeur que tu as rencontrée hier\\
    ‘Here is my sister whom you met yesterday’

    \ex \label{ex:kragh:19} Voilà qu’il neige\\
    ‘It is snowing’
    
    \ex \label{ex:kragh:20} Voilà comment faire\\
    ‘This is how to do it'
\end{exe}


These presentatives can be listed in a paradigm (see Paradigm 2). In the following, I will focus on the presentatives \textit{voici} and \textit{voilà}, derived from the verb \textit{voir}.

In this construction, \textit{Voici}/\textit{Voilà X} presents a referent \textit{X}, known or unknown to the hearer, examples \REF{ex:kragh:16} and \REF{ex:kragh:17}. This kind of constructions is also called \textit{neutral focus structures}. \textit{X} can be an NP, \REF{ex:kragh:16} and \REF{ex:kragh:17}, possibly with a subordinate relative clause \REF{ex:kragh:18}, a complement clause \REF{ex:kragh:19}, or an interrogative clause \REF{ex:kragh:20}.

It is characteristic for such constructions that they address explicitly the hearer and thus have the feature of +deixis, since they presuppose the presence of the hearer in the factual or fictive conversation space. In Modern French, \textit{Voilà X} is more frequent than \textit{Voici X}, and the latter is marked, since it has reduced possibilities of usage. The constructions are especially frequent in oral, informal communication, in accordance with the deictic character of the forms.

I consider the presentative construction exemplified in \REF{ex:kragh:16} to \REF{ex:kragh:20} a reanalysis of the lexical usage of the verb \textit{voir}, level 0. This reanalysis is illustrated by a bridging example \REF{ex:kragh:21} in which the hearer may interpret the message of the utterance in two different ways. Example \REF{ex:kragh:21} is from Chanson de Roland, 1100 (cited in \citealt[79]{OppermannMarsaux2006}):

\begin{exe}
    \ex \label{ex:kragh:21} Dreiz emperere, veiz me ci en present\\
    ‘Rightful Emperor, see me present here' (interpretation 1)\\
    `Rightful Emperor, here I am before you' (interpretation 2)
\end{exe}

According to interpretation 1, the speaker addresses the Emperor by saying ‘see me being present here’; thus, \textit{veiz} expresses visual perception. However, another interpretation is also possible, interpretation 2, which is a way of attracting the attention of the emperor. Thus, example \REF{ex:kragh:21} provides a bridging or critical context which permits reanalysis because of the ambiguity (cf. \citealt{Diewald2002, Heine2002}), namely an intention of attracting attention, i.e. a purely pragmatic function. I will consider the second interpretation to be the result of the speaker reanalysing the imperative form of \textit{voir} followed by the particle \textit{ci}, in the following way: A (imperative form of the verb of perception \textit{voir} followed by a locative particle \textit{ci}, meaning ‘see here’) > B (presentative), i.e. into a new way of attracting attention to a given item, without immediate change of the surface manifestation. This implies that the construction has acquired not only a) a different function. Thus, it is no longer a finite verb\,+\,a particle, but it is reanalysed as a fixed form with a particle, with b) a different meaning, i.e. presentation or attracting attention. Consequently, the new function is pragmatic. 

This process of reanalysis and grammaticalization of the imperative verbal form in the second person singular and merge with the two adverbs \textit{{}-ci} and \textit{{}-là}, losing its full lexical meaning, is schematized in \figref{fig:kragh:4}. I refer to \citet[212--216]{KraghStrudsholm2013} and \citet[190--191]{KraghSchosler2014} for a detailed account of the reanalysis from level 0 to level 1 and to \citet{OppermannMarsaux2006} for an account of the initial steps of this process. In the course of the subsequent reanalysis from level 1 to level 2, \textit{voici}/\textit{voilà} \textit{X} is reanalysed as a member of the presentative paradigm (\figref{fig:kragh:4})

  
\begin{figure}
%\includegraphics[width=.5\textwidth]{figures/Kragh-fig4.png}
\begin{tikzpicture}
  \tikzstyle{every node}=[rectangle,draw,text width=7cm,align=center]
  \node(A1){Level 1\\From lexicon to construction \textit{voici/voilà}};
  \node(B1)[below=    of A1]{Level 2\\Paradigm~2 -- Presentation\\\textit{voici/voilà X}};
  \node(0)[above=of A1]{Level 0\\Lexical level\\valency-bound constituents};
  \draw[->](0.south)-|(A1.north);
  \draw[->](A1)--(B1);
\end{tikzpicture}
 \caption{From level 0 to paradigm 2\label{fig:kragh:4}}
 \end{figure}

The presentative paradigm is listed in paradigm 2 and contains the constructions exemplified above in \REF{ex:kragh:13} to \REF{ex:kragh:20}.

The syntactic domain, i.e. the syntagmatic context, for the presentative paradigm (S)VX is a verb with or without a subject, followed by X as the presented entity. The semantic frame is \textit{presentation}, in the sense of an introduction of important and new information to the hearer about a new or already known entity. With respect to the content, the members of the paradigm are, in addition to \textit{presentation}, characterized by two features, namely the option of identification/opposition and deixis. \textit{Identification} refers to the designation of a referent, possibly combined with the designation in opposition to one of more other potential referents (\textit{opposition}). The feature of \textit{deixis} refers to the possibility of addressing explicitly a hearer and it presupposes the presence of the hearer in the factual or the fictive conversation room. 

\begin{table}
\begin{tabularx}{\textwidth}{l>{\raggedright}p{\widthof{\textit{Voici/voilà ma sœur}}}Q}
\lsptoprule
\multicolumn{3}{l}{\emph{Syntactic domain}: (S)VX; \emph{Semantic frame}: Presentation}\\
{Member} & {Expression} & {Content}\\
\midrule
\textit{C’est X}\textsuperscript{u} & \textit{Ce n’est pas eux} & Presentation/identification, ${\pm}$opposition, ${\pm}$deixis\\
\textit{Il est X}\textsuperscript{m,}\footnote{Fossilized variant of \textit{il y a}.} & \textit{Il était [une fois] une belle princesse} & Presentation, −identification, −deixis \\
\textit{Il y a X}\textsuperscript{m} & \textit{Il y a quelqu’un} & Presentation, −identification, −deixis\\
\textit{Voici}/\textit{voilà X}\textsuperscript{m} & \textit{Voici/voilà ma sœur} & Presentation, −identification, +deixis\\
\lspbottomrule
\end{tabularx}
\caption{Presentative paradigm. \textsuperscript{m}: marked member of paradigm; \textsuperscript{u}: unmarked member of paradigm}
\label{tab:kragh:2}
\end{table}

Having the least restrictions of the four members of the paradigm, \textit{c’est X} is the unmarked member \citep{KraghSchosler2019}. In addition to presentation, \textit{c’est X} is characterized by its capacity of identifying, possibly with specification of opposition. Deixis is possible. \textit{Il y a} can only mean presentation, not identification, and deixis is not required. \textit{Il est} is the fossilized variant of the productive \textit{il y a}; it has the same content as \textit{il y a}, but is mainly used in introductions of fairy tales or to express time. \textit{Voici}/\textit{voilà} expresses presentation, not identification, but does, on the other hand, express deixis. The three latter are thus marked in proportion to \textit{c’est X}. 

\subsubsection{From level 2 to level 3, Paradigm 3} \label{kragh:4.1.2.1}

Members of the presentative paradigm can occur with a subordinate or deictic relative clause and thereby enter a paradigm of focalization: \textit{c’est X qui/que}…, \textit{il y a X qui/que}…, \textit{voici/voilà X qui/que}… In addition to these, I have focalization constructions which are not derived from a presentative construction: \textit{il a X qui/que}…, \textit{X est là qui/que}…

The focalization paradigm comprises the following members:\largerpage


\ea\label{ex:kragh:22}C’est X qui/que…:\\ Ce n’est pas eux qui arrivent \\
    ‘It is not they who arrive’
\ex \label{ex:kragh:23}Il y a X qui/que…:\\ Il y a quelqu’un qui arrive\\
    ‘There is someone coming’
\ex\label{ex:kragh:24}Il est X qui/que…:\\ Il était une fois une belle pricesse qui vivait dans un vieux château\\
    ‘Once upon a time there was a beautiful princess who lived in an old castle’
\ex\label{ex:kragh:25}Voici/voilà X qui/que…:\\ Voilà ma sœur qui arrive\\
    ‘Here is my sister arriving’
\ex\label{ex:kragh:26}Il a X qui/que…:\\ Il a les cheveux qui tombent\\
    ‘He has his hair falling off’\footnote{Example citated from \citet{Conti2010}.}
\ex\label{ex:kragh:27}X est là qui/que…:\\ Elle est là qui pleure\\
    ‘There she is crying’\footnote{Example citated from \citet[104]{Furukawa2000}.}
\z

As illustrated in example \REF{ex:kragh:25}, the presentative \textit{voici}/\textit{voilà} can occur with a deictic relative. This is a new reanalysis which presupposes a number of previous reanalyses, presented in the preceding sections. I hypothesize that the reanalyses at the constructional level 1 from subordinate to deictic relative, and the grammaticalization of \textit{voici}/\textit{voilà} are more or less parallel processes during the Middle Ages, because I have no textual evidence that one should precede the other. The subsequent reanalyses as members of paradigms, Paradigm 1 and 2 (level 2), respectively, take place from the 16\textsuperscript{th} century onwards \citep{OppermannMarsaux2006, KraghSchosler2015}.\largerpage

This means that both the ideas of progression and simultaneity expressed in the deictic relative, and presentation expressed by the grammaticalized form \textit{voici}/\textit{voilà} are preconditions for the grammaticalization of the focus construction composed of \textit{voici}/\textit{voilà} and a deictic relative. 

\textit{Voici/voilà} are deictic expressions rooted in the time of the utterance, and in order to express simultaneity, typical of holistic constructions, the verb of the deictic relative must be in the present tense. This is in accordance not only with my analysis of \textit{voici/voilà X} and a deictic relative, where they are part of a progressive, holistic and deictic construction \citep{KraghSchosler2014}, but also with the view of \citet[50--51]{Lambrecht2000}, who states that the function of this type of focus construction is to present an entity and to express new information about it at the same time. Furthermore, the structure presupposes a known referent \citep[456]{RiegelEtAl2009}.

Examples \REF{ex:kragh:28} to \REF{ex:kragh:30} illustrate the difference between presentation and focalization. Example \REF{ex:kragh:28} shows a presentative construction with a pronominalization of the object \textit{ma} \textit{sœur}. Example \REF{ex:kragh:29} is also a presentative construction where the NP is specified by a subordinate relative, and the NP including the relative is pronominalized in the object pronoun \textit{la}.

In contrast, example \REF{ex:kragh:30} illustrates grammaticalized focalisation. Here I find a dislocation of the object \textit{ma sœur} specified not by a subordinate relative, but by a deictic relative being part of a nexus construction. In this case, pronominalization by means of \textit{la} of the antecedent \textit{ma sœur} concerns only the NP \textit{ma sœur}, not the deictic relative. 

\begin{exe}
    \ex \label{ex:kragh:28} Voilà ma sœur  ${\rightarrow}$ la voilà\\
    ‘Here is my sister ${\rightarrow}$ here she is’

    \ex \label{ex:kragh:29} Voilà ma sœur que tu as rencontrée hier ${\rightarrow}$ la voilà\\
    ‘Here is my sister whom you met yesterday ${\rightarrow}$ here she is’

    \ex \label{ex:kragh:30} Voilà ma sœur qui arrive ${\rightarrow}$ la voilà qui arrive\\
    ‘Here is my sister arriving ${\rightarrow}$ here she is arriving’
\end{exe}

The possibility of a personal pronoun as antecedent is a specific feature of a deictic relative; this possibility does not exist for subordinate relative clauses, cf. example \REF{ex:kragh:29}. 

Summing up the reanalyses leading to the grammaticalized focalization paradigm: this paradigm is the result of reanalyses of two constructions at a higher level, each with its own paradigmatic structure (Paradigm 1 and Paradigm 2). This implies that characteristic features of Paradigm 1, e.g. progression, and of Paradigm 2, e.g. that \textit{voici} and \textit{voilà} function as presentatives are inherited in Paradigm 3, as schematized in \figref{fig:kragh:5}.

\begin{figure}
%\includegraphics[width=.5\textwidth]{figures/Kragh-fig5.png}
\begin{tikzpicture}
  \tikzstyle{every node}=[rectangle,draw,text width=4.7cm,align=center]
  \node(0) {Level 0\\Lexical level};
  \node(A1)[below left=2\baselineskip and 5mm of 0.south]{Level 1\\From subordinate to deictic relative construction};
  \node(A2)[below right=2\baselineskip and 5mm of 0.south]{Level 1\\From lexicon to construction \textit{voici/voilà}};
  \node(B1)[below=of A1]{Level 2\\Paradigm~1\\Progression};
  \node(B2)[below=of A2]{Level 2\\Paradigm~2\\Presentation\vphantom{g}};
  \node(3)[below=12.5\baselineskip of 0]{Level 3\\Paradigm 3\\Focalization};
  \draw[->](0)-|(A1.north);
  \draw[->](0)-|(A2.north);
  \draw[->](A1)--(B1);
  \draw[->](A2)--(B2);
  \draw[->](B1.south)|-(3.west);
  \draw[->](B2.south)|-(3.east);
\end{tikzpicture}
\caption{From Paradigm 1 and 2 to Paradigm 3\label{fig:kragh:5}}
\end{figure}

The focalization paradigm is presented in Paradigm 3 and is composed of the constructions exemplified above (examples \ref{ex:kragh:22}--\ref{ex:kragh:27}).

The \textit{syntactic} \textit{domain} of the paradigm of grammaticalized focalization differs from that of presentative (neutral focus) because it contains an obligatory \textit{qui}\slash\textit{que}-clause, which is presented in the figure as follows: (S)VX \textit{qui/que}{}-clause. The \textit{semantic frame}, i.e. the grammatical meaning, is focalization in the sense that the structures provide important and new information to the interlocutor on new or already known entities. 

With respect to the content, the members of the paradigm are, in addition to \textit{focalization}, characterized by four features, namely the option of expressing \textit{opposition}, further information by means of a subordinate relative vs. a nexus construction introduced by \textit{qui/que} expressing progressivity, the option of deictic concord, and the option of deixis. The feature of \textit{deictic concord} refers to coincidence with respect to person, time and place between the matrix and the subordinate or deictic relative. Thus, in example \REF{ex:kragh:25} and \REF{ex:kragh:30} I find deictic concord between \textit{Voilà} and \textit{qui arrive}, because of the simultaneity of the perception (\textit{voilà}) and the perceived object (\textit{ma sœur qui arrive}). This concept should not be confused with \textit{deixis}, which refers to the possibility of addressing a hearer explicitly and presupposes the presence of the hearer in the factual or the fictive conversation room. Thus, there is \textit{deixis} in example \REF{ex:kragh:25} and \REF{ex:kragh:30}, because the speaker is addressing the hearer explicitly, which presupposes the presence of the hearer in the factual or the fictive conversation space. 

\begin{table}
\small
\caption{Paradigm of focalization \citep{KraghSchosler2019}. \textsuperscript{m}: marked member of paradigm; \textsuperscript{u}: unmarked member of paradigm.}
\label{tab:kragh:3}
\begin{tabularx}{\textwidth}{lQQ}
\lsptoprule
\multicolumn{3}{l}{\emph{Syntactic domain}: (S)VX(\textit{qui/que}−clause); \emph{Semantic frame}: Focalization}\\
{Member} & {Expression} & {Content}\\
\midrule
\textit{C’est X} \textit{qui}/\textit{que}\textsuperscript{u} & \textit{C’est n’est pas eux qui arrivent} & focus ±opposition, subordinate relative, ${\pm}$deictic concord, −deixis\\\tablevspace
\textit{Il est X qui/que}\textsuperscript{m} & \textit{Il était [une fois] une belle princesse qui vivait dans un vieux château} & focus −opposition, −deixis\\\tablevspace
\textit{Il y a X qui}/\textit{que}\textsuperscript{m} & \textit{Il y a quelqu’un qui arrive}  & focus −opposition, nexus construction, +deictic concord (?), −deixis\\\tablevspace
\textit{Voici}\slash\textit{voilà X} (\textit{qui}/\textit{que})\textsuperscript{m} & \textit{Voici/voilà ma sœur qui arrive}  & focus −opposition, nexus construction, +deictic concord, +deixis\\\tablevspace
\textit{Il a X qui/que}\textsuperscript{m} & \textit{Il a les cheveux qui tombent} \newline ‘he has his hair falling off’ & focus −opposition, nexus construction, +deictic concord, −deixis, Object related to subject, e.g. body part, family member, etc.\\\tablevspace
\textit{Il est là qui/que}\textsuperscript{m} & \textit{Elle est là qui pleure} ‘there she is crying’ & focus −opposition, nexus construction, +deictic concord, −deixis, Presupposes prior indication of spatial location\\
\lspbottomrule
\end{tabularx}
\end{table}

The structures identified as focus constructions have been characterized by means of the following criteria: focalization ±opposition, subordination \textit{versus} nexus relation, ±deictic concord between the matrix and the relative clause, with restrictions on tense, mood, etc., and ±deixis in the construction. According to these criteria, \textit{C’est X} \textit{qui}/\textit{que} clause is the unmarked member of the paradigm; since it has fewest restrictions. It expresses focalization with or without opposition to another referent and can have deictic concord between the matrix and the relative, but has no restrictions on tense and mood, etc., and it has no deixis in the construction. The relation between the relative clause and the antecedent is a relation of subordination. The other structures are opposed to this unmarked construction as marked members. Among the marked members, the \textit{il y a qui/que} structure is less marked than the \textit{voici}\slash\textit{voilà} \textit{qui, il a X qui}, and \textit{il est là qui} structures because it has fewer restrictions on tense and mood. It does not express deixis. The three last mentioned structures share the following criteria: like \textit{il y a qui/que} they focus without indication of opposition and they form a nexus construction. In contrast with \textit{il y a X qui/que}, they have deictic concord, but only \textit{voici/voilà X qui} has deixis, i.e. presupposes the presence of the hearer in the same factual or fictive conversation room as the speaker. 

As illustrated in \figref{fig:kragh:5}, level 3 presupposes level 2, and not vice-versa. In chronological terms this relation of precondition is confirmed by my corpus investigations, which show that level 2 can be found from the 13\textsuperscript{th} century, whereas level 3 occurs by the end of the 17\textsuperscript{th} century, spreading in the 19\textsuperscript{th} century. Therefore, it is reasonable to conclude that grammaticalized focalization is the result of a reanalysis of the presentative structure, with the consequence of focalization being clearly marked. 

It is my hypothesis that other usages of the verb \textit{voir} have led to pathways similar to the ones presented so far and into other grammatical paradigms as suggested in \sectref{kragh:1}. 

I shall now give a brief overview of four of these processes of reanalyses and grammaticalizations leading to \textit{tu vois} as a discourse marker (\sectref{kragh:4.2}), \textit{se voir} as a member of the voice paradigm (\sectref{kragh:4.3}), \textit{vu} as preposition, and \textit{vu que} as member of the paradigm of subordinate conjunctions (\sectref{kragh:4.3.2}), in order to show further perspectives of my approach.

\subsection{Level 1: Reanalyses to constructions and sub-levels} \label{kragh:4.2}

The origin of verbal discourse markers like \textit{Tu vois, mon bonheur passe} ‘You know, my happiness is waning’ is widely debated (\citealt{KraghToAppear}). My hypothesis is that this type of discourse markers originates as a reanalysis of a complex phrase of the type \textit{Tu vois que mon bonheur passe} ${\rightarrow}$ \textit{tu vois, mon bonheur passe} (level 1).

 \begin{figure}
%\includegraphics[width=.5\textwidth]{figures/Kragh-fig6.png}
\begin{tikzpicture}
  \tikzstyle{every node}=[rectangle,draw,text width=7cm,align=center]
  \node(A1){Level 1\\From lexical level to constructional level};
  \node(B1)[below=    of A1]{Level 2\\Paradigm~4\\Discourse marker};
  \node(0)[above=of A1]{Level 0\\Lexical level\\valency-bound constituents};
  \draw[->](0.south)-|(A1.north);
  \draw[->](A1)--(B1);
\end{tikzpicture}
 \caption{From constructional level to paradigmatic level, Paradigm 4}
 \label{fig:kragh:6}

\end{figure}

Based on this first reanalysis, I have observed a subsequent reanalysis of \textit{tu vois, vous voyez, voyons} and \textit{voilà} leading to the creation of a discourse marker paradigm. For a detailed account of this process, I refer to \citet{KraghToAppear,KraghToAppearb}.

\subsection{Level 1: Reanalyses to constructions and sub-levels} \label{kragh:4.3}

As illustrated in \figref{fig:kragh:7}, I assume that different reanalyses also precede the voice paradigm (Paradigm 5), and the preposition and conjunction paradigms (Paradigms 6 and 7), respectively.  

\begin{figure}
%\includegraphics[width=\textwidth]{figures/Kragh-fig7.png}

\begin{tikzpicture}
  \tikzstyle{every node}=[rectangle,draw,text width=2.7cm,align=center]
  \node(A1){Level 1\\Reanalysis};
  \node(A2)[right=2mm of A1]{Level 1\\Reanalysis};
  \node(A3)[right=2mm of A2]{Level 1\\Reanalysis};
  \node(B1)[below=    of A1]{Level 2\\Paradigm~5\\Voice\vphantom{jp}};
  \node(B2)[right=2mm of B1]{Level 2\\Paradigm~6\\Preposition\vphantom{jp}};
  \node(B3)[right=2mm of B2]{Level 2\\Paradigm~4\\Conjunction\vphantom{jp}};
  \node(0)[above=of A2]{Level 0\\Lexical level};
  \draw[->](0)-|(A1.north);
  \draw[->](0.south)-|(A2.north);
  \draw[->](0)-|(A3.north);
  \draw[->](A1)--(B1);
  \draw[->](A2)--(B2);
  \draw[->](A3)--(B3);
\end{tikzpicture}
 \caption{From constructional level to paradigmatic level, Paradigms 5, 6, and 7\label{fig:kragh:7}}
 \end{figure}


 \subsubsection{Level 2, Paradigm 5} \label{kragh:4.3.1}


As part of the voice paradigm, the reflexive form of \textit{voir} (see example \ref{ex:kragh:31}) competes not only with the active construction (\textit{Le gouvernement augmente les prix} ‘The government raises the prices’), but also with other ways of expressing a state of affairs without an agent or an active subject, e.g. the passive construction (\textit{Les prix sont augmentés} ‘The prices have gone up’), the unaccusative construction (\textit{Les prix augmentent} ‘The prices go up’), the reflexive unaccusative construction (\textit{Les prix s’augmentent} ‘The prices go up’), and the deontic reflexive passive (\textit{Le vin blanc se boit frais} ‘White wine should be served chilled’), and a number of impersonal constructions (e.g. \textit{On augmente les prix} ‘The prices have gone up’). Periphrastic reflexive passives, typically with an affected person as subject are found with the verbs \textit{faire} (‘to make’) and \textit{voir}, and express mainly activities that are adverse to an affected person:

\begin{exe}
    \ex \label{ex:kragh:31} Pierre se voit refuser l’accès\\
    ‘Peter is denied entrance…’
\end{exe}

Thus, they differ from both active, unaccusative, and passive constructions with regard to types of agent and patient, and to the event described. It should be noted that \textit{se voir} has been included among passive constructions by a few other scholars, e.g. by \citet[1051]{GrevisseGoosse2008} where it is called \emph{auxiliaire du passif} and classified among the semi-auxiliaries, whereas \citet[188]{Defrancq2000} uses the term \textit{passif de l’objet prépositionnel}. 


\subsubsection{Level 2, Paradigms 6 and 7} \label{kragh:4.3.2}


Paradigms 6 and 7 have in common that they are formed from the past participle of \textit{voir}. Preliminary results indicate that \textit{vu} as a preposition occurs from the 14\textsuperscript{th} century \citep{Rey1986}, and suggest that this precedes the conjunction \textit{vu que}, of which the first occurrences found are from the 15\textsuperscript{th} century (Frantext). Both the preposition \textit{vu}  and the conjunction \textit{vu que} are grammatical entities; they have no lexical, but only grammatical meaning. Therefore, I do not consider them to be cases of lexicalization. Whether \textit{vu que} is the result of a regrammation of the preposition \textit{vu} or rather of the participle (as in \textit{Il a vu que…}) remains to be investigated.

Paradigm 6 comprises the paradigm of prepositions. This paradigm includes simple forms like \textit{à}, \textit{de}, \textit{en}, \textit{dans}, \textit{pour}, etc., complex formations such as \textit{à côté de}, \textit{au} \textit{lieu} \textit{de}, \textit{pour} \textit{cause} \textit{de}, etc., and forms derived from past participles like \textit{vu}, \textit{attendu}, \textit{exepté}, \textit{compris}, \textit{hormis}, as well as present participles such as \textit{suivant}, \textit{durant}, \textit{moyennant}, etc.  An exemple is:

\begin{exe}
    \ex \label{ex:kragh:32} \emph{Vu} la situation actuelle, il faut partir au plus vite\\
    `\emph{Considering} the actual situation, we must leave as quickly as possible’
\end{exe}


Paradigm 7 is the paradigm of subordinate conjunctions. It includes the simple conjunctions \textit{que}, \textit{si}, \textit{comme}, \textit{quand}, etc., complex conjunctions like \textit{à mesure que}, \textit{avant} \textit{que}, \textit{dès} \textit{que}, \textit{bien} \textit{que}, \textit{à} \textit{la} \textit{condition} \textit{que}, etc., and conjunctions derived from verbal forms \textit{excepté que}, \textit{vu} \textit{que}, \textit{suppose} \textit{que,} and \textit{suivant} \textit{que}, \textit{pourvu que}, \textit{attendu que}:

\begin{exe}
    \ex \label{ex:kragh:33} \emph{Vu que} le texte de la recommendation n’est pas encore prêt, il est assez difficile de poursuivre l’analyse\\
    `\emph{As} the text of the recommendation is not yet ready, it is rather difficult to undertake further analysis’
\end{exe}

\section{Conclusions} \label{kragh:5}

As stated at the beginning of this paper, I think that when lexical entities grammaticalize, they enter pre-existing or new grammatical paradigms, and that therefore the concept of paradigm is important if we wish to understand the reanalyses that lead to grammaticalization. I claim that grammar is composed of sets of paradigms in the general sense of selectional sets, composed of marked and unmarked members, cf. \figref{fig:kragh:8} (page \pageref{fig:kragh:8}).

\begin{sidewaysfigure}
%\includegraphics[width=.5\textwidth,angle=270]{figures/Kragh-fig8.png}
\small
\begin{tikzpicture}
  \tikzstyle{every node}=[rectangle,draw,text width=2.5cm,align=center,anchor=north]
  \node(A1){Level 1\\Reanalysis};
  \node(A2)[right=2mm of A1]{Level 1\\Reanalysis};
  \node(A3)[right=2mm of A2]{Level 1\\Reanalysis};
  \node(A4)[right=2mm of A3]{Level 1\\Reanalysis};
  \node(A5)[right=2mm of A4]{Level 1\\Reanalysis};
  \node(A6)[right=2mm of A5]{Level 1\\Reanalysis};
%
  \node(B1)[below=    of A1]{Level 2\\Paradigm~1\\Progression\vphantom{gj}};
  \node(B2)[right=2mm of B1]{Level 2\\Paradigm~2\\Presentation\vphantom{gj}};
  \node(B3)[right=2mm of B2]{Level 2\\Paradigm~4\\Disc. mark.\vphantom{gj}};
  \node(B4)[right=2mm of B3]{Level 2\\Paradigm~5\\Voice\vphantom{gj}};
  \node(B5)[right=2mm of B4]{Level 2\\Paradigm~6\\Preposition\vphantom{gj}};
  \node(B6)[right=2mm of B5]{Level 2\\Paradigm~7\\Conjunction\vphantom{gj}};
%
  \node(0)[above=of A3,xshift=1.475cm,text width=2cm]{Level 0\\Lexical level};
  \node(3)[below=of B1,xshift=1.475cm,text width=2cm]{Level 3\\Paradigm 3\\Focalization};
%
  \draw[->](0)-|(A1.north);
  \draw[->](0)-|(A2.north);
  \draw[->](0)-|(A3.north);
  \draw[->](0)-|(A4.north);
  \draw[->](0)-|(A5.north);
  \draw[->](0)-|(A6.north);
%
  \draw[->](A1)--(B1);
  \draw[->](A2)--(B2);
  \draw[->](A3)--(B3);
  \draw[->](A4)--(B4);
  \draw[->](A5)--(B5);
  \draw[->](A6)--(B6);
%
  \draw[->](B1.south)|-(3.west);
  \draw[->](B2.south)|-(3.east);
\end{tikzpicture}
 \caption{Overview\label{fig:kragh:8}}
 \end{sidewaysfigure}

An approach of grammar as sets of paradigms provides a precise and relatively straight forward presentation of what otherwise would seem utterly diverse usages of a lexical entity, see the illustration of a lexical approach in the the digital dictionary Robert Connecteur\footnote{\url{https://dictionnaire.lerobert.com/}}, which confuses very different levels of usage, lexical, semi-grammatical, and grammatical.

I hope to have provided convincing evidence in favour of this claim. 

{\sloppy\printbibliography[heading=subbibliography,notkeyword=this]}
\end{document} 
