\documentclass[output=paper, colorlinks,citecolor=brown]{langsci/langscibook} 
\ChapterDOI{10.5281/zenodo.5675841}
\author{Henning Andersen\affiliation{University of California, Los Angeles}}
\title{Paradigms of paradigms}
\abstract{This paper is concerned with inflectional morphology. Its point of departure is the old insight that paradigms of content categories are typically nested inside other paradigms forming hierarchical structures, e.g., case paradigms are included in number paradigms: \textsc{case}] \textsc{number}]. Similarly, say, in the Latin verb, \textsc{person}] \textsc{number}] \textsc{tense}] \textsc{aspect}] \textsc{mood}] \textsc{voice]} (\sectref{andersen_sec1}). As for exponence, paradigms (selectional sets) of inflectional classes (declensions, conjugations) are more often flat (linear) and asymmetrical with respect to one or more characteristics, e.g., meaning, stem shape, or productivity. But they may include selectional sets of allomorphs at a lower level, say, in individual cases or tenses, presenting paradigms in paradigms within paradigms (...) (\sectref{content_paradigm}). A third dimension in inflectional systems is the typological paradigm of morphological techniques commonly reflected in synchronic variation; English verb morphology, for instance, comprises analytic (\textit{will call, got arrested}), agglutinative (\textit{waded, jogged}), fusional (\textit{kept, built}), and introflective (\textit{sang, hung}) formations (\sectref{english_conjugation}). As additional examples of typological paradigms the conjugations of French, Latin, and Russian are examined (Sections \ref{english_conjugation}--\ref{russian_conjugation}). It is shown that such typological variation may reflect the historicity of an inflectional system. Since this variation is part of speakers' competence, it should be recognized as an inomissible part of synchronic description (\sectref{andersen_sec3}).}

\begin{document}
\maketitle

\section{Introduction} \label{andersen_sec1}
\subsection{Content paradigm} \label{content_paradigm}\largerpage

The purpose of this paper is to draw attention to paradigmatic relations in inflectional morphology that have traditionally been overlooked. The background for this examination is the distinction between content paradigms and exponent paradigms. We begin with paradigms of grammatical categories.

Quite naturally the notion of paradigm – what \citet{Saussure1959} called associative relations – has historically implied a focus on grammatical content categories and their members. Saussure illustrated the notion with members of inflectional categories and derivational categories and, to include the exponent level, sets of homonyms. When \citet{Jakobson1956a} characterized speech as a product of the dual processes of selecting and combining linguistic entities, the selectional sets he referred to were evidently paradigms of lexical and grammatical categories (see also \citealt{Bloomfield1933}: 164; \citealt{Hamp1966}, s.v. \textit{selection}). Such categories serve well to illustrate the diversity of selectional sets.

Scalar paradigms perhaps best illustrate Saussure's understanding that linguistic categories are imposed on the world of experience and thereby shape our conceptual categories. Consider the colors in the world around us, which in the world of languages are represented by anywhere from two to a dozen simple color terms. The modern understanding of the hues of the color wheel has facilitated the identification of a multitude of colors. Their multifarious exploitation in design and fashion, as in interior decorating and in clothing, suggests that this perceptual dimension has a potential for infinitely differentiated paradigms of hues and subparadigms of shades, tints, and tones -- through the imposition of color names (or code numbers). Many other scalar dimensions of experience are expressed with paradigms of contrary opposites, e.g., \textit{large} vs \textit{small}, \textit{wide} vs \textit{narrow}, \textit{dark} vs \textit{light}, \textit{loud} vs \textit{soft}, which are employed relative to explicit or implicit standards in both literal and metaphorical senses.

Graded (step-wise) categories are as common. The number words from \textit{one} to \textit{ten}, which recursively name units, tens, hundreds, thousands, and so on, illustrate how paradigms can be nested inside paradigms, in this instance ad infinitum; see \REF{ex:andersen_1}. Decimal fractions of each whole number likewise form nested paradigms stretching to the infinitely small, a conceptual counterpart to the potentially infinite differentiation of the realm of colors. 

\begin{exe}
    \ex \label{ex:andersen_1} \textsc{units} ] \textsc{tens} ] \textsc{hundreds} ] \textsc{thousands} ] \textsc{ten-thousands} ] ...
\end{exe}

Grammatical categories typically form paradigms of contradictory opposites. These paradigms do not have infinitely many members, but a hierarchical ordering of paradigms is commonly in evidence. Typically, for instance, in languages with grammatical cases, the case paradigm is subordinated to a number paradigm; see \tabref{tab:andersen_1}. 

\begin{table}
\caption{Latin First declension\label{tab:andersen_1}}
\begin{tabular}{>{\scshape}lll} 
\lsptoprule
& Singular & Plural\\\midrule
nom & \textit{tabul-a} & \textit{tabul-a-e}\\
acc & \textit{tabul-a-m} & \textit{tabul-ā-s}\\
gen & \textit{tabul-a-e} & \textit{tabul-ā-r-um} \\
dat & \textit{tabul-a-e} & \textit{tabul-ī-s}\\
abl & \textit{tabul-ā} & \textit{tabul-ī-s}\\
\lspbottomrule
\end{tabular}
\end{table}

This relation of subordination, which can be summed up as \textsc{case} ] \textsc{number} ], becomes manifest in historical change when \textsc{case} is lost while \textsc{number} remains, as in many European languages. 

\begin{exe}
    \ex \label{ex:andersen_2} \textsc{person} ] \textsc{number} ] \textsc{tense} ] \textsc{aspect} ] \textsc{mood} ] \textsc{voice} ] 
\end{exe}


Similarly in verbs, the category of \textsc{person} is often subordinated to that of \textsc{number}. In Latin this relationship is realized in each of the three tenses Present, Imperfect, Future, in both aspects, Infective and Perfective (where the tenses are called Perfect, Pluperfect, and Future Perfect), in both moods, Indicative and Subjunctive, and in both voices, Active and Passive, yielding a structure that is easily represented by a tree-diagram, but can be summed up as in \REF{ex:andersen_2}.

\subsection{Exponent paradigms} \label{exponent_paradigm}

When we turn to the expression side of languages, we find paradigms of less consistently hierarchical structure. A language may have extensive allomorphy in its inflectional categories; see, for example the Latin case allomorphy in \tabref{tab:andersen_2}. 

%%please move \begin{table} just above \begin{tabular
\begin{table}
\caption{Allomorphy in Latin noun declension. Singular\label{tab:andersen_2}}
\begin{tabular}{>{\scshape}ll}
\lsptoprule
nom.sg & -a/-e/-es/-ēs/-is/-s/-ū/-um/-us/-∅\\
acc.sg & -am/-e/-em/-im/-ū/-um/-us/-∅\\
gen.sg & -ae/-eī/-ī/-is/-ūs\\
dat.sg & -ae/-eī/-ī/-u/-uī\\
abl.sg & -ā/-e/-ē/-i/-ō/-ū\\
\lspbottomrule
\end{tabular}
\end{table}


The allomorphs are organized into several partly overlapping classes, the  ``declensions''. The grammatical tradition recognizes five declensions (\tabref{tab:andersen_3}), that is, a paradigm of five noun classes forming a flat structure But several of the declensions have cases with notable internal allomorphy, that is, allomorph paradigms within case paradigms within the paradigm of declensions; see \tabref{tab:andersen_3}. 

\begin{table}
\caption{Allomorphy in Latin case desinences\label{tab:andersen_3}}
\begin{tabular}{>{\scshape}llllll}
\lsptoprule
& Decl. 1 & Decl. 2 & Decl. 3 & Decl. 4 & Decl. 5\\\midrule
nom.sg & {}-a & {}-um/-us/-∅ & {}-e/-es/-is/-s/-∅ & {}-ū/-us & {}-ēs\\
acc.sg & {}-am & {}-um/-us & {}-e/-em/-im/-∅ & {}-ū/-um & {}-em\\
gen.sg & {}-ae & {}-ī & {}-is & {}-ūs & {}-ei\\
dat.sg & {}-ae & {}-ō & {}-ī & {}-u/-uī & {}-eī\\
abl.sg & {}-ā & {}-ō & {}-e/-i & {}-ū & {}-ē\\
\lspbottomrule
\end{tabular}
\end{table}

\begin{sloppypar}
In Decl. 2, for instance, feminines and masculines have syncretic desinences in all cases; e.g., \textit{fīcus}.\textsc{f.nom.sg}, \textit{fīcum}.\textsc{f.acc.sg}, \textit{fīcī}.\textsc{f.gen.sg} ‘fig', \textit{servus}.\textsc{m.nom.sg}, \textit{servum}.\textsc{m.acc.sg}, \textit{servī}.\textsc{m.gen.sg} ‘slave'. Both masculines and neuters have additional, lexically conditioned allomorphs: Some Decl. 2 masculines have a \mbox{-∅} \textsc{nom.sg} desinence (\textit{vir} `man', \textit{liber} `book'), and a few Decl. 2 neuters have \mbox{\textit{-us}.\textsc{nom}}\slash\textsc{acc.sg} (\textit{vulgus} `crowd', \textit{vīrus} `poison'). This allomorphy is conditioned by the gender of lexical stems; hence each of the endings \textit{-um}.\textsc{nom.sg} and \textit{\mbox{-us}}.\textsc{acc.sg} points to (or indicates) the neuter gender of its noun stem; see \tabref{tab:andersen_4}.
\end{sloppypar}\largerpage

\begin{table}
\caption{Allomorphy paradigms in Latin Second-declension desinences\label{tab:andersen_4}}
\begin{tabular}{ccc ccc}
\lsptoprule
\multicolumn{3}{c}{\scshape nom.sg} & \multicolumn{3}{c}{\scshape acc.sg}\\\cmidrule(lr){1-3}\cmidrule(lr){4-6}
\scshape m & \scshape f & \scshape n & \scshape m & \scshape f & \scshape n\\\midrule
-us/-∅ & -us  &    -um/-us & -um & -um & -um/-us\\
\lspbottomrule
\end{tabular}
\end{table}

\begin{table}
\caption{Allomorphy paradigms in Russian First-declension desinences\label{tab:andersen_5}}
\begin{tabular}{cc cc cc cc}
\lsptoprule
\multicolumn{2}{c}{\scshape nom.sg}    &  \multicolumn{2}{c}{\scshape acc.sg} &  \multicolumn{2}{c}{\scshape gen.sg} & \multicolumn{2}{c}{\scshape loc.sg}\\\cmidrule(lr){1-2}\cmidrule(lr){3-4}\cmidrule(lr){5-6}\cmidrule(lr){7-8}
\scshape m & \scshape n        &   \scshape m &\scshape n &      \scshape m & \scshape n &  \scshape  m & \scshape n \\\midrule
   -∅      &  -o     &  -∅/-a & -o  & -a/-u & -a & -e/-u & -e\\
\lspbottomrule
\end{tabular}
\end{table}

\begin{sloppypar}
A similar case paradigm in Russian (\tabref{tab:andersen_5}) includes paradigms of allomorphs that indicate the gender of the stems: \textsc{nom.sg} \textit{-o} (\textit{okn-o} ‘window') indicates neuter, \textsc{nom.sg} -∅ (\textit{gorod-∅} ‘town') indicates masculine. But additional allomorphs within the masculine point to referential features such as animacy (\textsc{acc.sg} \textit{syn-a} ‘son' vs \textit{gorod-∅}) or mass vs countable (\textsc{gen.sg} \textit{čaj-u} ‘tea' vs \textit{gorod-a}) or material vs nonmaterial referent (\textsc{loc.sg} \textit{v sneg-ú} ‘in the snow' vs \textit{v sneg-e} ‘in snow (as concept or word)'). 
\end{sloppypar}

In terms of the theory of semiotics of Charles S. Peirce, the semiotic value of this low level allomorphy is identified as indexical (\citealt{Shapiro1969}; \citealt{Anttila1972, Anttila1989}; \citealt{Andersen1980, Andersen2020}): individual allomorphs point to, indicate, or are indexes of phonological, grammatical, or lexical features of given noun stems or features of their referents or combinations of such subsidiary information.\largerpage

A systematic investigation of the contribution of this secondary level of signaling in communication is a task for the future. But the index values of allomorphs or morphological processes such as mutation or truncation will be relevant repeatedly below.

\section{Typological paradigms} \label{andersen_sec2}

Here we turn to yet another paradigmatic relation in inflection, one that opens up a neglected perspective on morphological systems.

For the purposes of this exposition it is useful to be able to refer to \citegen[120--146]{Sapir1921} typology of morphological techniques. It is presented in \tabref{tab:andersen_6} with minimal characterizations of the individual types, derived from Sapir's text; I use \textit{introflection} for Sapir's \textit{symbolism}. 

\begin{table}
\caption{Basic morphological techniques}
\label{tab:andersen_6}
\begin{tabularx}{\textwidth}{llQ}
\lsptoprule
A.& Analytic & Constructions of lexical and grammatical words, free or clitic.\\
B.& Synthetic & \\
  & ~~Agglutination & Simple juxtaposition of bound lexical and grammatical morphemes. \newline ± Cross-boundary phonological or phonotactic indexing.\\
  & ~~Fusion & Grammatical morphemes with cumulative grammatical content.  \newline ± Cross-boundary grammatical and/or lexical content indexing.\\
  & ~~Introflection & Lexical morphemes with grammatical content.\\
C.& Isolating & Grammatical exponents not constructed with lexical morphemes.\\
\lspbottomrule
\end{tabularx}
\end{table}

\begin{sloppypar}
The major types (A, B, C) form a paradigm, and within the synthetic macrotype, agglutination, fusion, and introflection form another paradigm. In the following pages we will see some examples of how these paradigms are exploited in morphological systems. The examples are taken from English (\sectref{english_conjugation}), French (\sectref{french_conjugation}), Latin (\sectref{latin_conjugation}), and Russian (\sectref{russian_conjugation}).
\end{sloppypar}

\subsection{English conjugation} \label{english_conjugation}\largerpage

It is convenient to begin with the conjugation of the English verb; see the major verbal categories in \tabref{tab:andersen_7}. Infinitive and imperative are uninflected. They have identical lexical content; their distinct grammatical content is expressed solely by their syntactic properties. In the Present Indicative, the \textsc{3sg} is suffixed /-ǝz/ \mbox{({\textbar}{\textbar} /\textsc{i}z/)} after sibilants, /-s/ after voiceless stops, and otherwise /-z/, e.g., \textit{pitch-es}, \textit{bat-s}, \textit{jog-s}, \textit{run-s}. The content of this desinence is debatable; some would consider it cumulative (`\textsc{3sg} Present Indicative'), but it might be just `Indicative', specifying a predicate as asserted, in the least marked Person–Number–Tense environment. Assuming that cumulative exponence is characteristic of fusion (cf. \tabref{tab:andersen_6}), this noncumulative interpretation would be compatible with agglutination, as are the phonotactic adjustment after sibilant stems and the phonological adjustment after voiceless stops. The Present participle in /-\textsc{i}ŋ/ is agglutinative. 

\begin{table}
\caption{English verb morphology: Categories\label{tab:andersen_7}}
\begin{tabular}{ll}
\lsptoprule
Present system & Preterite system  \\
\midrule
Infinitive & \\
Imperative & \\
Present tense & Preterite\\
Present participle & Past participle\\
Auxiliated: Progressive and Futures & Auxiliated: Perfect and Passives\\
\lspbottomrule
\end{tabular}
\end{table}

The Past tense and Past participle are expressed by three formations that differ in morphological technique. They form a selectional set, a paradigm; data from \citet{Bloch1947}, \citet[249–257]{Palmer1987}.

Preterite 1, the productive pattern characteristic of thousands of verbs and applied to all new verbs, is agglutinative. It has a phonotactic adjustment /-ǝd/ \mbox{({\textbar}{\textbar} /\textsc{i}d/)} after stem-final dental plosives (here written ...T), a voiceless allomorph /-t/ after stem-final voiceless consonants, and otherwise /-d/; see \tabref{tab:andersen_8}.

Preterite 2 is fusional. It comprises verbs with the regular /-t/ and /-d/ allomorphs and some that have a lexically conditioned /-t/. Some of these verbs have a vowel mutation in the Preterite (\textit{keep–kept}, \textit{flee–fled, mean–meant}), some have both a vowel and a coda mutation (\textit{lose–lost}), some have a vowel mutation and coda truncation (\textit{can–could}, \textit{catch–caught}); a couple have just coda truncation (\textit{make–made}); see \tabref{tab:andersen_9}. 

\begin{table}
\caption{English Preterite 1: Agglutinative\label{tab:andersen_8}}
\begin{tabular}{lll}
\lsptoprule
...T{\longrule} & ...vl C{\longrule} & Default\\\midrule
\textit{bat}: -ǝd (\textit{{}-ed}) & \textit{pitch}: -t (\textit{{}-ed}) & \textit{wail}: -d (\textit{{}-ed})\\
\textit{bat} - \textit{batted} & \textit{pitch} - \textit{pitched} & \textit{wail} - \textit{wailed}\\
\textit{wade} - \textit{waded}  & \textit{miss} - \textit{missed} & \textit{jog} - \textit{jogged}\\
\lspbottomrule
\end{tabular}
\end{table}


\begin{table}
\caption{English Preterite 2: Fusional\label{tab:andersen_9}}
\begin{tabularx}{\textwidth}{QQQ}
\lsptoprule
Regular \textit{–t} & Regular \textit{–d} & Irregular \textit{–t}\\
\midrule
Vowel mutation &  & \\
\textit{keep}, \textit{leap}, \textit{sleep}, \textit{weep}, ... & \textit{flee}, \textit{say}, \textit{hear}, \textit{tell}, \textit{sell}, ... & \textit{dream}, \textit{mean}, \textit{feel}, \textit{kneel}, ...\\
Vowel mutation and coda mutation & \multicolumn{2}{c}{Vowel mutation and coda truncation}\\
\textit{cleave}, \textit{leave}, \textit{lose},... & \textit{can}, \textit{shall}, \textit{will} & \textit{catch}, \textit{teach}, \textit{bring}, ...\\
& Coda truncation & \\
& \textit{have}, \textit{make} & \\
\lspbottomrule
\end{tabularx}
\end{table}

The mutations in the stems of these verbs are conditioned by the Past-tense category. Since Past tense is separately expressed by the distinct Past tense marker in all these verbs, the mutations are indexical, and the morphological type of these forms is fusional; cf. \tabref{tab:andersen_6}. Some of these verbs have an agglutinative Preterite 1 variant in which the mutation is omitted, e.g., \textit{leaped}, \textit{dreamed}, \textit{kneeled}, \textit{cleaved}. There are also basically agglutinative verbs with a variant ‘Irregular \textit{-t}' Preterite 2 and no mutation, e.g. \textit{spell–spelt}, \textit{spill–spilt}.\largerpage[2]

There are no verbs in Preterite 2 that have the phonotactic vowel epenthesis (-ǝd) found in Preterite 1. Instead, Preterite 2 verbs in stem-final dental plosive (here written ...T) truncate the ...t or ...d after the appropriate Preterite allomorph has been selected, that is, ...t-t → \textit{-t}, ...d-d → \textit{-d}, ...d-t → \textit{-t}; see \tabref{tab:andersen_10}. Some English experts view the ending in these verbs as -∅ \citep{Palmer1987}. The verbs with regular \textit{-t} or -\textit{d} might be described that way; but the verbs with `Irregular \textit{-t}' suggest that it is indeed the stem-final dental plosive that is truncated.

\begin{table}
\caption{English Preterite 2$'$: Fusional\label{tab:andersen_10}}
\begin{tabularx}{\textwidth}{QQQ}
\lsptoprule
Regular \textit{-t} & Regular \textit{-d} & Irregular \textit{-t}\\
...T truncation &  & \\
\midrule
\textit{slit}, \textit{split}, \textit{put}, \textit{bet}, \textit{let}, \textit{cut}, \textit{hurt}, \textit{cost}, \textit{must}, \textit{burst}, ... & \textit{bid}, \textit{rid}, \textit{shed}, \textit{spread}, ... & \textit{bend}, \textit{rend}, \textit{send}, ̛\textit{build}, ...\\
Vowel mutation and ...T truncation &  & \\
\textit{meet} & \textit{bleed}, \textit{lead}, \textit{read}, \textit{slide},... & \\
\lspbottomrule
\end{tabularx}
\end{table}

\begin{sloppypar}
Preterite 3 comprises verbs with unsuffixed and with suffixed \textsc{pst.ptcp}. Besides this distinction there are verbs with three alternating vowels (\textit{sing–sang–sung}, \textit{drive–drove–driv-en}), with two vowels, identical in \textsc{prs} and \textsc{pst.ptcp} (\textit{run–ran–run}, \textit{know–knew–know-n}) or identical in \textsc{pst} and \textsc{pst.ptcp} (\textit{hang–hung–hung}, \textit{speak–spoke–spok-en}), and with a single vowel throughout (\textit{beat–beat–beat-en}); see \tabref{tab:andersen_11}.
\end{sloppypar}

Preterite 3 is typologically diverse. In the unsuffixed lexemes past tense and past-participle function are expressed through introflection (cells 1.1–1.3). In the suffixed lexemes, past tense is expressed through introflection (cells 2.1–2.4), and past-participle functions by the suffix alone (cells 2.2, 2.4; agglutination) or by the suffix accompanied by stem-vowel mutation (cells 2.1, 2.3); this is fusion. 

Casual speaking styles show a strong tendency to extend the introflective past-tense forms to past-participle function, e.g. \textit{I would`ve did it differently} (cell 2.1), \textit{you could`ve came earlier} (cell 1.2), \textit{they should`ve took the other one} (cell 2.2). 

\begin{table}
\caption{English Preterite 3}
\label{tab:andersen_11}
\begin{tabularx}{\textwidth}{QQ}
\lsptoprule
1. No past-participle suffix & 2. Suffixed participle\\
1.1 Three alternating vowels & 2.1 Three alternating vowels\\
\textit{sing} (+ 9 more) & \textit{drive} (+ 8 more), \textit{fly}, \textit{do}\\
1.2 Distinct vowel in Pst & 2.2 Distinct vowel in Pst\\
\textit{come}, \textit{run} & \textit{know} (+ 4), \textit{take} (+ 1), \textit{slay}, \textit{eat}, \textit{give}, \textit{see}, \textit{bid}, \textit{fall}, \textit{draw}\\
1.3 Distinct vowel in Pst and \textsc{pst.ptcp} & 2.3 Distinct vowel in Pst and \textsc{pst.ptcp}\\
\textit{cling} (+ 10), \textit{hang}, \textit{strike}, \textit{sneak}, \textit{shine}, \textit{bind} (+ 3), \textit{hold}, \textit{sit} (+2), \textit{shoot}, \textit{fight}, \textit{light} & \textit{speak} (+ 5), \textit{break} (+ 1), \textit{choose}, \textit{lie}, \textit{get} (+ 4), \textit{bite} (+ 4)\\
& 2.4 One vowel in all three stems\\
& \textit{beat}\\
\lspbottomrule
\end{tabularx}
\end{table}


The different formations of the English Preterite form a typological paradigm as in \tabref{tab:andersen_12}. As a selectional set they are evidenced in synchronic (stylistic, social) variation.  Preterite 1 {\textasciitilde} Preterite 2: \textit{dreamed} {\textasciitilde} \textit{dreamt}, \textit{kneeled} {\textasciitilde} \textit{knelt}; Preterite 1 {\textasciitilde} Preterite 3: \textit{strived} {\textasciitilde} \textit{strove}, \textit{thrived} {\textasciitilde} \textit{throve}, (metaphorical) \textit{weaved} (through traffic) {\textasciitilde} \textit{wove}, \textit{slayed} (an audience) {\textasciitilde} \textit{slew}; Preterite 2´ {\textasciitilde} Preterite 3: \textit{(for)bid} {\textasciitilde} \textit{(for)bade}; note Preterite 1 {\textasciitilde} 2 {\textasciitilde} 3 in \textit{cleave}–\textit{cleaved} {\textasciitilde} \textit{cleave}–\textit{cleft} {\textasciitilde} \textit{cleave}–\textit{clove–cloven}; all variants cited from \textit{The American Heritage Dictionary}.\largerpage
 
\tabref{tab:andersen_12} displays the paradigm of morphological techniques of English conjugation.{\interfootnotelinepenalty=10000\footnote{English noun plurals exemplify the same variation. Analytic: \textit{heads of cattle}, \textit{... cabbage}; \textit{pairs of scissors}, \textit{... trousers}, \textit{cloves of garlic}. Agglutination: \textit{horse-s}, \textit{cat-s}, \textit{dog-s}, \textit{cow-s}; \textit{ox-en}. Fusion: \textit{calf–calv-es}, \textit{house–hous-es}, \textit{youth–youth-s}; \textit{child–childr-en}. Introflection: \textit{woman–women}, \textit{man–men}; \textit{foot–feet}, also \textit{tooth}, \textit{goose}; \textit{mouse–mice}, also \textit{louse}; \textit{crisis–crises}, \textit{alumna–alumnae}; \textit{sheep–sheep}, also \textit{deer}, \textit{grouse}, \textit{trout}, \textit{fish}; \textit{sail}, \textit{cannon}.}}
 

\begin{table}
\caption{English conjugation in typological perspective}
\label{tab:andersen_12}
\begin{tabularx}{\textwidth}{lQ}
\lsptoprule
Analytic & Futures (\textit{will}, \textit{is going to work}), Continuous (\textit{is}, \textit{was}, \textit{has been} \textit{working}), Retrospective (\textit{has}, \textit{had worked}), Passive (\textit{was}, \textit{got fired})\\
Agglutination & Present: 3sg Indicative; Prs.ptcp;\newline Preterite 1. Preterite 3: suffixed \textsc{pst.ptcp} (types \textit{known}, \textit{beaten})\\
Fusion & Preterite 2. Preterite 3: suffixed \textsc{pst.ptcp} (types \textit{driven}, \textit{spoken})\newline Modal verbs, \textit{have} (\textit{has}, \textit{had}), \textit{been}\\
Introflection & Preterite 3: Pst, unsuffixed \textsc{pst.ptcp} 
\textit{be} (\textit{am}, \textit{is}, \textit{are}, \textit{was}, \textit{were})\\
\lspbottomrule
\end{tabularx}
\end{table}



Here it is worth noting that the vowel alternations in Preterite 3 verbs reflect those of the Indo-European Present vs. Perfect formations. These apophonic alternations, which Germanic shares with all the other Indo-European languages, may be at least 7000 years old. The agglutinative Preterite 1, by contrast, is Common Germanic heritage, perhaps less than 2500 years old. With the exception of the ‘be' passive, the analytic, auxiliated formations are much younger. 

It appears that this typological perspective reflects what one can call the historicity of the system of verbal morphology. 

Now, we know enough of the history of English to recognize that this is not the same as reflecting the history of the language. A historical account will acknowledge (i) that Preterite 2 developed from Preterite 1 thanks to a variety of conditioned sound changes, so it is younger; (ii) that in Old English our apophonic Preterite 3 verbs had a separate set of past-tense desinences, that is, they were fusional; they became introflective only when this `strong' past-tense inflection was lost; but also (iii) that more Preterite 3 verbs have changed to Preterite 1 or 2 since Old English, than vice versa, and (iv) that there is a similar predominance of Preterite 2 verbs transitioning to Preterite 1. 

None of these details can be read off the overview in \tabref{tab:andersen_12}. Still the typological paradigm undeniably suggests a generalized historical perspective on this synchronic system.  

In the following pages I will look at a few other languages whose history is known, to see to what extent such a historical perspective may be a common feature of morphological systems.

\subsection{French conjugation} \label{french_conjugation}
French has two regular conjugations, one productive, exemplified by \textit{chanter} `sing', the other practically unproductive, typified by \textit{finir} `finish'. In addition there is a number of irregular verbs. The system of categories can be seen in \tabref{tab:andersen_13}. 

An important feature of French conjugation is that finite verbs are obligatorily accompanied by subject clitics; consequently the (contingent) suffixal Participant (Person, Number) marking is strictly an agreement feature. 

\begin{table}
\caption{French tense system}
\label{tab:andersen_13}
\begin{tabularx}{\textwidth}{QQl}
\lsptoprule
Present system & Preterite system & Infinitive system\\
Present indicative & Past indicative & Future\\
Present subjunctive & Past subjunctive & Conditional\\
Imperfect &  & \\
Present participle & Past participle & Infinitive  \\
\lspbottomrule
\end{tabularx}
\end{table}


Conj. 1 is agglutinative: Suffixes for tense, mood, person, and the nonfinite categories simply follow the stem. There is some stem allomorphy: Some verb stems with mid vowels have a regular alternation between pretonic and tonic (final, closed) syllable: /e/, /ǝ/ {\textasciitilde} /ɛ/ (\textit{céder–cède} `cede', \textit{jeter–jette} `throw', \textit{appeler–appelle} `name, call'), /ø/ {\textasciitilde} /œ/ (\textit{beurrer–beurre} `butter'), and /o/ {\textasciitilde} /ɔ/ (\textit{coller–colle} `glue'). The alternation has no apparent synchronic motivation; contrast \textit{aider–aide} `help' with /ɛ/, \textit{sauver–sauve} `save' with /o/; but it is phonologically (prosodically) conditioned, and it is irrelevant to the stem–desinence boundary. Thus it is compatible with agglutination.

Conj. 2 is similarly agglutinative. But it is characterized by two truncations that produce distinct stems for (i) \textsc{Prs.ind.123sg} and (ii) the Preterite and Infinitive systems; see (3.a–b), where superscript 0, 00 represent coda and rhyme truncation, respectively; the basic stem ends in /s/, e.g., /finis-/. The truncations produce stem allomorphs with specific grammatical meaning: This is a fusional feature. 

\ea\label{ex:andersen_3}\ea
\glt \textsc{Prs.ind}: fini\textsuperscript{0}{}-∅.123\textsc{sg} vs finis-õ.1\textsc{p}l, finis-∅.3\\textsc{pl}; \textsc{Prs.sbj}: finis-∅.123\textsc{sg.3pl}, finis-j-õ.\textsc{1pl}; \textsc{Impf}: finis-ɛ.\textsc{123sg.3pl};  \textsc{Prs.ptcp}: finis-ã\\
\ex\label{ex:andersen_3b}
\glt \textsc{st.ind}: fin\textsuperscript{00}{}-i-∅.\textsc{123.sg}, fin\textsuperscript{00}{}-i-m.\textsc{1pl}, fin\textsuperscript{00}{}-i-t.\textsc{2pl}, fin\textsuperscript{00}{}-i-r.\textsc{3pl}; \textsc{Pst.sbj}: fin\textsuperscript{00}{}-i-s-∅.12\textsc{sg.3pl}, fin\textsuperscript{00}{}-i-s-j-õ.\textsc{1pl}, fin\textsuperscript{00}{}-i.\textsc{pst.ptcp}, fin\textsuperscript{00}{}-ir.\textsc{inf} \z\z

\begin{sloppypar}
The contrast between the stem-final ...s- in finis-∅.\textsc{prs.sbj.123sg.3pl} (\textit{finisse(nt)}) \REF{ex:andersen_3} and the \textsc{Pst.sbj} morpheme -s- in fin\textsuperscript{00}{}-i-s-∅.\textsc{pst.sbj.12sg.3pl} (\textit{finisse(nt)}) \REF{ex:andersen_3b} is recognized by French grammarians (see \citealt[588–589]{Grevisse1961}). These grammatical forms are systematically homophonous in Conj. 2 verbs, but since they have different morpheme constituency they are not homonymous. 
\end{sloppypar}

Irregular verbs have largely the same agglutinative suffixations as Conj. 1 and 2 verbs; but they have different allomorphs in the Preterite and Future systems; and they are characterized by stem mutations and truncations, as well as by stem suppletion. The lexical distribution of these features is irregular. A systematic presentation of the whole picture would exceed the space available here. The following subregularities and a few illustrations in \REF{ex:andersen_4} will suffice for the present purpose.

\ea \label{ex:andersen_4}
\ea \label{ex:andersen_4a}
Vowel mutations, homologous to those in Conj. 1, but involving different vowels. /e/ → /jɛ/, /ǝ/ → /wa/, /y/ → /wa/, /u/ → /ø {\textasciitilde} œ/; phonotactically, /jɛ/ and /wa/ count as single segments. In some verbs the vowel mutation affects \textsc{Prs.ind.123sg.3pl}; e.g., \textit{acquérir} ‘get': aker-õ.\textsc{1pl}, akjɛr-∅.\textsc{3sg}, akjɛr-∅.\textsc{3pl} (\textit{acquérons}, \textit{acquiert, acquièrent}), \textit{mourir} ‘die': mur-õ, mœr-∅, mœr-∅ (\textit{mourons}, \textit{meurt}, \textit{meurent}).
\ex \label{ex:andersen_4b}
      Coda truncation in \textsc{Prs.ind.sg}; e.g., \textit{dormir} ‘sleep': dɔrm-õ.\textsc{1pl}, dɔrm-∅.\textsc{3pl}, but dɔr\textsuperscript{0}{}-∅.\textsc{3sg} (\textit{dormons}, \textit{dorment}, \textit{dort}); \textit{bouillir} ‘boil': buj-õ, buj-∅, bu\textsuperscript{0}{}-∅ (\textit{bouillons}, \textit{bouillent}, \textit{bout}); \textit{lire} ‘read': liz-õ, liz-∅, li\textsuperscript{0}{}-∅ (\textit{lisons}, \textit{lisent}, \textit{lit}).
\ex \label{ex:andersen_4c}
      Coda truncation in \textsc{Prs.ind.sg} also occurs in some verbs with vowel mutation (but not in \textsc{Prs.sbj.1–3sg}): \textit{recevoir} ‘receive': rǝsǝv-õ.\textsc{1pl}, rǝswav-∅.\textsc{3pl}, rǝswa\textsuperscript{0}{}-∅.\textsc{3sg} (\textit{recevons, reçoivent, reçoit}); \textit{devoir} ‘ought; owe': dǝv-õ, dwav-∅, dwa\textsuperscript{0}{}-∅ (\textit{devons}, \textit{doivent}, \textit{doit}); \textit{boire} ‘drink': byv-õ, bwav-∅, bwa\textsuperscript{0}{}-∅ (\textit{buvons}, \textit{boivent}, \textit{boit}); \textit{mouvoir} ‘move': muv-õ, mœv-∅, mø\textsuperscript{0}{}-∅ (\textit{mouvons}. \textit{meuvent}, \textit{meut}).
\ex \label{ex:andersen_4d}
      Rhyme truncation in \textsc{pst.ind.sbj}, as in Conj. 2; e.g., \textit{acquérir}: aker-õ.1pl, ak\textsuperscript{00}{}-i-∅.\textsc{3sg} (\textit{acquérons}, \textit{acqui}); \textit{voir} ‘see': vwaj-õ, v\textsuperscript{00}{}-i-∅ (\textit{voyons}, \textit{vi}); \textit{recevoir}: rǝsǝv-õ, rǝs\textsuperscript{00}{}-y-∅ (\textit{recevons}, \textit{reçu}); \textit{devoir}: dǝv-õ, d\textsuperscript{00}{}-y-1∅1 (\textit{devons}, \textit{du}); \textit{boire:} byv-õ, b\textsuperscript{00}{}-y-∅ (\textit{buvons}, \textit{bu}); \textit{lire}: liz-õ, l\textsuperscript{00}{}-y-∅ (\textit{lisons}, \textit{lu}); \textit{savoir} ‘know': sav-õ, s\textsuperscript{00}{}-y-∅ (\textit{savons, su}).
\ex \label{ex:andersen_4e}
      Preterite allomorphy. -i- {\textasciitilde} -y-; e.g., (i) -i- in both \textsc{pst.ind.sbj} and \textsc{pst.ptcp}: e.g., \textit{dormir}: dɔrm-i-∅, dɔrm-i (\textit{dormi}, \textit{dormi}); \textit{bouillir}: buj-i-∅, buj-i (\textit{bouilli}, \textit{bouilli}); (ii) -i- in \textsc{pst.ind-sbj}, -y- in \textsc{pst.ptcp}: \textit{voir}: v\textsuperscript{00}i-∅ and v\textsuperscript{00}{}-y (\textit{vi}, \textit{vu}), \textit{rompre} ‘break': rõp-i-∅, rõp-y (\textit{rompi}, \textit{rompu}), \textit{battre} ‘beat': bat-i-∅, bat-y (\textit{batti}, \textit{battu}). (iii) -y- in both \textsc{pst.ind-sbj} and \textsc{pst.ptcp}: \textit{lire}: l\textsuperscript{00}{}-y-∅ and l\textsuperscript{00}{}-y (\textit{lu}, \textit{lu}), \textit{courir} ‘run': kur-y-∅ and kur-y (\textit{couru}, \textit{couru}).
\ex \label{ex:andersen_4f}
      Interfixed consonants in \textsc{Inf} and/or \textsc{Fut.Cond}; e.g., \textit{connaître} ‘know': kɔnɛs-õ, kɔnɛ\textsuperscript{0}{}-t-r.\textsc{inf} (\textit{connaissons}); \textit{coudre} ‘sew': kuz-õ, ku\textsuperscript{0}{}-d-r (\textit{cousons}); \textit{moudre} ‘grind': mul-õ, mu\textsuperscript{0}{}-d-r (\textit{moulons}); \textit{tenir} ‘hold': tǝn-õ, tjẽ-d-r-ɛ.cond.\textsc{3sg} (\textit{tenons}, \textit{tiendrait}); \textit{vouloir} ‘will, want': vul-õ, vu\textsuperscript{0}{}-d-r-ɛ (\textit{voulons}, \textit{voudrait}).
    \z
\z\pagebreak

The features in (\ref{ex:andersen_4a}--\ref{ex:andersen_4f}) are relevant to many irregular verbs. They define some stem allomorphy indicating desinential grammatical content and some desinence allomorphy indicating the lexical content of the stem: This is fusion.

The following examples in \REF{ex:andersen_5} illustrate wordforms that combine lexical and grammatical content: Introflection.

\ea \label{ex:andersen_5} \ea
      A specific stem for \textsc{Prs.sbj}: \textit{aller} ‘go': al-õ.\textsc{prs.ind.1pl} but aj-∅.\textsc{prs.sbj.123sg.3pl} (\textit{allons}, \textit{aille(nt)}); \textit{vouloir}: vul-õ, vøl.\textsc{prs.ind.3pl} but vœj-∅ (\textit{voulons}, \textit{veulent,} \textit{veuille}(\textit{nt})); \textit{savoir}: sav-õ, sav-∅ but saš-∅ (\textit{savons}, \textit{savent}, \textit{sache(nt)}).
\ex \label{ex:andersen_5b}
      A specific stem for \textsc{Fut} and \textsc{Cond}. \textit{tenir}: tǝn-õ, tjẽ-d-r-e.\textsc{fut.1sg} (\textit{tenons}, \textit{tiendrai}), \textit{aller}: al-õ, ir-e (\textit{allons}, \textit{irai}), \textit{voir}: vwaj-õ, vɛr-e (\textit{voyons}, \textit{verrai}).
\ex \label{ex:andersen_5c}
      A specific wordform for \textsc{pst.ptcp} (feminine endings in parentheses). \textit{Offrir} ‘offer': ɔfr-õ, ɔfɛr(-t) (\textit{offrons}, \textit{offert}); \textit{ouvrir} ‘open': uvr-õ, uvɛr(-t) (\textit{ouvrons}, \textit{ouvert}); \textit{mourir}: mur-õ, mɔr(-t) (\textit{mourons}, \textit{mort}); combined with rhyme truncation: \textit{acquérir}: aker-õ, ak\textsuperscript{00}{}-i(-z) (\textit{acqérons}, \textit{acquis}); \textit{mettre} ‘put': mɛt-õ, m\textsuperscript{00}{}-i(-z) (\textit{mettons}, \textit{mis}); \textit{écrire} ‘write': ekriv-õ, ekr\textsuperscript{00}{}-i(-t) (\textit{écrivons}, \textit{écrit}).
\ex \label{ex:andersen_5d}
      Other grammatically specific stems or wordforms. (i) \textit{haïr} ‘hate': ais-õ, ɛ.\textsc{prs.123sg} (\textit{haïssons}, \textit{hait});  (ii) \textit{pouvoir} ‘can, be able': pɥi.\textsc{prs.ind.1sg}, pɥis–.\textsc{prs.sbj} (\textit{puis}, \textit{puisse}(\textit{nt})); (iii) \textit{savoir}: sav–.\textsc{prs.ind.pl/inf}, se.\textsc{prs.ind.sg}, saš–\textsc{prs.sbj/ptcp}, s\textsuperscript{00}–.\textsc{pst.ind/sbj/ptcp}, so-r–.\textsc{fut/cond} (\textit{savons}, \textit{savais}, \textit{sais}, \textit{sache}(\textit{nt}), \textit{sachant}, \textit{su}s, \textit{saurai}); (iv) \textit{aller}: vɛ.\textsc{prs.ind.1sg}, va.\textsc{23sg}, võ.\textsc{3pl}, aj–.\textsc{prs.sbj}, ir–.\textsc{fut/cond} (\textit{vais}, \textit{va}, \textit{allons}, \textit{vont}, \textit{aille}, \textit{ir-ai});  (v) \textit{avoir} ‘have': e.\textsc{prs.ind.1sg}, a.\textsc{23sg}, õ.\textsc{3pl}, ɛ.\textsc{prs.sbj.123sg.3pl}, y-.\textsc{pst/pst.ptcp}, or–.\textsc{fut/cond} (\textit{avons}, \textit{ai}, \textit{a}, \textit{ont}, \textit{aie}, \textit{eus}, \textit{aurai}); (vi) \textit{être} ‘be': sɥi.\textsc{prs.ind.1sg}, ɛ.\textsc{23sg}, som.\textsc{1pl}, ɛt.\textsc{2pl}, sõ.\textsc{3pl}, swa–.\textsc{prs.sbj}, fy–.\textsc{pst}, ɛt–.\textsc{inf}, sǝr–.\textsc{fut/cond}, ete.\textsc{pst.ptcp} (\textit{suis}, \textit{est}, \textit{sommes}, \textit{sont}, \textit{sois}, \textit{fus}, \textit{êt-re}, \textit{se-r-ai}, \textit{été}); (vii) \textit{faire} ‘do, make': fǝz–.\textsc{prs.ind.1pl/ptcp/impf}, fɛ.\textsc{prs.123sg/inf}, fɛt.\textsc{prs.ind.2pl}, fõ.\textsc{3pl} fas–.\textsc{prs.sbj, fi.pst}, fǝr–.\textsc{fut/cond}, fɛ(-t).\textsc{pst.ptcp} (\textit{faisons}, \textit{fait}, \textit{faites}, \textit{font}, \textit{fasse(nt)}, \textit{fi}, \textit{ferai}, \textit{fait}).
    \z
    \z

The irregular lexical distribution of the many subregularities in the morphology of these verbs makes for some complexity; in \citegen{StumpFinkel2017} approach, French has 72 conjugations. Still, it is clear that features in (\ref{ex:andersen_4a}--\ref{ex:andersen_4f}) produce allomorphy, in stems or suffixes, that amounts to cross-boundary indexes; they exemplify the technique of fusion. The examples in \REF{ex:andersen_5} are stems or wordforms that combine lexical and grammatical content; this is introflection. Many irregular verbs have no introflective forms at all, but several have a handful or more.

\begin{table}
\caption{French conjugation in typological perspective}
\label{tab:andersen_14}
\begin{tabularx}{\textwidth}{lQ}
\lsptoprule
Analytic & Obligatory pronominal subject clitics. Passive: \textit{être} + \textsc{p.p.p.}; Perfect: \textit{avoir}/\textit{être} + \textsc{p.p.p.}; Auxiliated Future: \textit{aller} + \textsc{inf}; Causative: \textit{faire} + \textsc{inf}.\\
Agglutination & Only productive type: \textit{chanter}; regular, prosodically conditioned stem-internal V-mutations (type \textit{céder–cède}).\\
Fusion & Stem allomorphy. V-mutations, coda truncation, rhyme truncation, V/C-interfixation: \textit{finir}: finis fini\textsuperscript{0} fin\textsuperscript{00}{}-i-; \textit{mouvoir}: muv mœv mø\textsuperscript{0} m\textsuperscript{00}{}-y-; \textit{recevoir}: rǝsǝv rǝswav rǝswa\textsuperscript{0} rǝs\textsuperscript{00}{}-y-; \textit{devoir}: dǝv dwav dwa\textsuperscript{0} d\textsuperscript{00}{}-y-; \textit{lire}: liz li\textsuperscript{0} l\textsuperscript{00}{}-y-; \textit{écrire}: ekriv ekri\textsuperscript{0} ekr\textsuperscript{00}{}-i-. \newline Lexically conditioned suffix allomorphy: \textsc{Pst} -e/a/ɛ-, -i-, -y-, \textsc{p.p.p}. -e-, -i-,
-y-; \textsc{Inf} -e, -r, -ir, -war.\\
Introflection & \textit{haïr}: ɛ.\textsc{prs.sg}; \textit{pouvoir}: pɥi pɥis; \textit{vouloir}: vœj; \textit{savoir}: se saš– s– sor–; \textit{aller}: vɛ va võ aj– ir–; \textit{faire}: fǝz– fɛ– fɛt fõ fas– f– fɛ(-t) fǝr-; \textit{avoir}: e a õ aj– y– or–; \textit{être}: sɥi ɛ som ɛt sõ swa– f– ete sǝr–; lexicalized \textsc{pst.ptcp}: \textit{mɔr}(\textit{{}-t}), \textit{ɔfɛr}(\textit{{}-t}), \textit{mi}(\textit{{}-z}), \textit{ekri}(\textit{{}-t}), \textit{aki}(\textit{{}-z}) ....\\
\lspbottomrule
\end{tabularx}
\end{table}

In \tabref{tab:andersen_14}, only the Passive is old; the other analytic formations have developed since the early Middle Ages. The productive, agglutinative pattern is the descendant of the Latin productive Conj. 1. Among the fusional verbs only the regular but unproductive \textit{finir} conjugation continues a productive Latin formation (Late Lat. \textit{finīscō–finīvī–finīre}). The other fusional patterns as well as all the introflective ones go back to pre-Latin formations that had ceased to be productive in classical Latin; this is true also of some post-Latin suppletive verbs, e.g., Fr. \textit{aller} (< \textit{ambulāre}, \textit{vadere}, \textit{īre}) and \textit{être} (< \textit{esse}(\textit{re}), \textit{stare}). 

The analytic formations are of different age. But the synthetic part of the paradigm largely reflects the historicity of the system. 

\subsection{Latin conjugation}  \label{latin_conjugation}

The hierarchy of Latin verbal categories was briefly summarized in \REF{ex:andersen_2}, repeated here as \REF{ex:andersen_6}.

\begin{exe}
    \ex \label{ex:andersen_6} \textsc{person} ] \textsc{number} ] \textsc{tense} ] \textsc{aspect} ] \textsc{mood} ] \textsc{voice} ]  
\end{exe}

\begin{sloppypar}
In Latin, verbal inflection is organized in a paradigm of four conjugation classes, traditionally numbered 1 to 4. Conj. 1 is fully productive, Conj. 4 less so; Conj. 2 and 3 are unproductive, the latter, with the exception of inchoative verbs formed with the suffix \textit{{}-sc-} (\textit{senēscō} `age', Late Lat. \textit{finīscō} `finish'). 
\end{sloppypar}

Basic-stem formation. Verb stems are derived from lexical morphemes with interfixes (stem formatives),  in Conj. 1 with \textit{-ā-}, in Conj. 4 with \textit{-ī-}, in Conj. 2 with \textit{-ē-} ({\textasciitilde} -∅- {\textasciitilde} \textit{-i-}; see below). In Conj. 3, some basic (Infective) stems are derived with several lexically conditioned affixes, other stems are bare, e.g., interfixed (\textit{cap-i-ō} `catch'), infixed (\textit{ru-m-p-ō} `break'), and bare stem (\textit{ag-ō} `lead'). The interfixation, being lexically conditioned, is fusional derivation. Verb stems serve as bases for inflection for Voice, Mood, Aspect, Tense, Number and Person as well as the derivation of a roster of deverbal nominal (Infinitive, Gerund), adjectival (Gerundive, Present and Past Participle), and adverbial (Supines 1 and 2) derivatives, as well as deverbal verbs (e.g., frequentative, desiderative). Here we focus on the obligatory verbal categories. 

\begin{description}
\item[Voice:] In the Passive, the Perfective tenses are analytic (auxiliary ‘be' + \textsc{p.p.p.}); in the Infective tenses, Passive morphs are joined with Participant (Person\slash Number) exponents (see below) \REF{ex:andersen_7g}. 

\item[Mood:] In the Subjunctive there is no distinct Future or Future perfect;  but Subjunctive is expressed cumulatively with the other tenses (see below). In the Imperative there is no Perfective aspect, no Imperfect tense, only second person forms; the Future is expressed by the suffix \textit{-to-}. Negative Imperative is auxiliated, \textit{noli} ‘do not' + \textsc{Inf}.

\item[Aspect:] In Conj. 1, 2, 4 Infective stems are identical with the \textsc{Inf} stem. In Conj. 1 and 4, the Perfective exponent is \textit{-v-}, in Conj. 2, where the class suffix is -∅- in the Perfective, the suffix is \textit{-u-}: the \textit{-v-} {\textasciitilde} \textit{-u-} alternation is phonologically conditioned: it is agglutinative \REF{ex:andersen_7}. 
\end{description}

In Conj. 3 Perfective stems are related to Infective stems by (i) deaffixation (\textit{cap-i-ō–cēp-ī}, \textit{ru-m-p-ō–rūp-ī}), (ii) vowel mutation and/or (iii) quantity change (\textit{ag-ō–ēg-ī}), (iv) reduplication (\textit{curr-ō–cu-curr-ī} `run'), (v) a combination of some of these (\textit{ta-n-g-ō–te-tig-ī} `touch'), (vi) affixation (\textit{scrīb-ō–scrīp-s-ī} `write', \textit{ping-ō–pinx-ī} `paint'), or (vii) invariant-stem inflection (\textit{scand-ō–scand-ī} `ascend'). Types (i–v) are fusional: the stem allomorph points to the Perfect desinences; (vi) and (vii) are agglutinative, (vi) with an overt \textit{-s-}, (vii) with zero Perfect suffix. 

\ea \label{ex:andersen_7}
\ea\label{ex:andersen_7a}
      Present. Conj. 1 \textit{amā-re}.\textsc{inf}., \textit{amā-v-ī}.\textsc{prs.pfv.1.sg}; Conj. 2 \textit{monē-re}, \textit{mon-u-ī}; Conj. 3 \textit{capi-ō}, \textit{cēp-ī}; \textit{ag-ō}, \textit{ēg-ī}; Conj. 4 \textit{audī-re}, \textit{audī-v-ī}.
\ex \label{ex:andersen_7b}
      Past. Conj. 1 \textit{amā-b-a-m}.\textsc{pst.infv}, \textit{amā-v-er-a-m}.\textsc{pst.pfv}; Conj. 2 \textit{monē-b-a-m}, \textit{mon-u-er-a-m}; Conj. 3 \textit{capi-ē-b-a-m}, \textit{cēp-er-a-m}; \textit{ag-ē-b-a-m}, \textit{ēg-er-a-m}; Conj. 4 \textit{audi-ē-b-a-m}, \textit{audī-v-er-a-m}.
\ex \label{ex:andersen_7c}
      Future. Conj. 1, 2 \textit{amā-b-ō}.\textsc{fut.inf}, \textit{amā-v-er-ō}.\textsc{fut.pfv}; \textit{monē-b-ō}, \textit{mon-u-er-ō}; \textit{vs.} Conj. 3, 4 \textit{capi-a-m}, \textit{...-e-s, ...-e-nt}, \textit{cēp-er-ō}; \textit{ag-a-m}, \textit{ēg-er-ō}; \textit{audi-a-m,} \textit{audī-v-er-ō}.
\ex \label{ex:andersen_7d}
      Prs.subj \textit{am\textsuperscript{0}}\textit{{}-e-m, mone-a-m}, \textit{capi-a-m}, \textit{audi-a-m}; \textsc{Impf.subj} \textit{amā-r-e-m, monē-r-e-m}, \textit{cap-er-e-m}, \textit{audī-r-e-m}; \textsc{Prf.subj} \textit{amā-v-er-i-m, mon-u-er-i-m}, \textit{cēp-er-i-m}, \textit{audī-v-er-i-m}; \textsc{Plup.subj} \textit{amā-v-iss-e-m}, \textit{mon-u-iss-e-m}, \textit{cēp-iss-e-m}, \textit{audī-v-iss-e-m}.
\ex \label{ex:andersen_7e}
      Conj. 1: \textit{amā-re}.\textsc{inf}. but \textit{am\textsuperscript{0}}\textit{{}-ō}.\textsc{prs.ind.1sg}, \textit{am}\textsuperscript{0}\textit{{}-em}.\textsc{prs.sbj.1.sg}.
\ex \label{ex:andersen_7f}
      Prs \textit{amā-s}.\textsc{2sg}, \textit{ama-t}.\textsc{3sg}, \textit{amā-mus}.\textsc{1pl}, \textit{amā-tis}.\textsc{2pl}, \textit{ama-nt}.\textsc{3pl}.
\ex \label{ex:andersen_7g}
      Conj. 1: \textit{amā-re}.\textsc{inf}, \textit{amā-v-ī}.\textsc{prf.1.sg}, \textit{amā-t-us}.\textsc{pst.pass.ptcp}; Conj. 2: \textit{monē-re}, \textit{mon-u-ī}, \textit{moni-t-us}; Conj. 3 \textit{cap-e-re}, \textit{cēp-ī}, \textit{cap-t-us}; Conj. 4: \textit{finī-re}, \textit{finī-v-ī}, \textit{finī-t-us}.
\ex \label{ex:andersen_7h}
      \textit{am\textsuperscript{0}}\textit{{}-o-r}, \textit{amā-r-is}, \textit{amā-t-ur}, \textit{amā-m-ur}, \textit{amā-mini}, \textit{ama-nt-ur}.
\z
\z

\begin{description}
\item[Tense:] Indicative: In both aspects, Present tense has a zero exponent. Past Infective (Imperfect) and Perfective (Pluperfect) are expressed by \textit{-b-ā-} and \textit{-er-ā-}, respectively; in Conj. 3, 4 the Infective \textit{-b-} is affixed to an \textit{-ē-} interfix \REF{ex:andersen_7b}. Future Infective (Future) in Conj. 1, 2 is expressed by \textit{-b-ō}/\textit{i}/\textit{u-} and Future Perfective (Future Perfect) of all verbs, by \textit{-er-ō}/\textit{i}/\textit{u-}. In Conj. 3, 4 Future Infective is expressed by \textit{-a}/\textit{e-} \REF{ex:andersen_7c}. The structure of all these forms is transparent and mainly agglutinative. The \textit{-b-} {\textasciitilde} \textit{-er-} allomorphs (`nonPresent') indicate Aspect; the allomorphy is fusional. The \textit{-b-} vs \textit{–a}/\textit{e-} allomorphy in the Infective Future indicates Conj. class; in Conj. 3, 4 the \textit{–a-} {\textasciitilde} \textit{-e-} allomorphy indicates Person (\textit{-a-m}, \textit{-e-s}, ...): both these alternations are fusional.
\end{description}

In the Subjunctive, Present is expressed by \textsuperscript{0}{}\textit{-e-} in Conj. 1, otherwise by \textit{-a-}; Imperfect is expressed by \textit{-r-e-} in Conj. 1, 2, 4 and by \textit{-er-e-} in Conj. 3: phonological conditioning. Perfect is \textit{-er-i-} and Pluperfect, \textit{-iss-e-} \REF{ex:andersen_7d}.  In sum, Tense is largely expressed agglutinatively, but indicates  Mood and Aspect: Fusion. Person and Number allomorphy is conditioned by Voice, Aspect and Tense \REF{ex:andersen_7e}. 

In the Active Participant desinences, separate plural exponents can be recognized in the final \textit{-s} of \textit{-mu-s}.\textsc{1pl}, \textit{-ti-s}.\textsc{2pl}, and in the longer desinence \textit{-nt}.\textsc{3pl} vs \textit{-t-}.\textsc{3sg} \REF{ex:andersen_7f}. Passive suffixes are partly fused with Participant desinences: \textit{-o-r}.\textsc{1sg.pass}, \textit{-r-is}.\textsc{pass.2sg}, \textit{-t-ur}.\textsc{3sg.pass}, and \textit{-nt-ur}.\textsc{3pl.pass} are agglutinative; in \textit{-m-ur}.\textsc{1pl.pass}, Person and Number are cumulative; in \textit{-mini}.\textsc{pass.2pl}, similarly, Voice, Person, and Number are cumulative \REF{ex:andersen_7g}: Fusion. 

In the Perfective Present (the Perfect), \textit{-ī}.\textsc{1sg} is a covariant of the Infective Present and Future \textit{-ō}.\textsc{1sg} and the default \textit{-m}.\textsc{1sg}. The desinences \textit{-is-tī}.\textsc{2sg} and \textit{-is-ti-s}.\textsc{2pl} contain the Perfect suffix \textit{-is-} {\textasciitilde} \textit{-er-}; \textit{-ēre}.\textsc{3pl} is cumulative: Fusion. 

Phonotactic and phonological adjustments. In the Infective, Conj. 1 interfix \textit{-ā-} is truncated before vocalic endings \REF{ex:andersen_7e} and exemplifies a general alternation in quantity phonologically conditioned by following desinences \REF{ex:andersen_7f}. The long interfix vowels of Conj. 2, 4 shorten in hiatus, e.g., \textit{monē-re}.\textsc{inf} \textit{mone-ō}.\textsc{prs.infv.1sg}, \textit{finī-re}, \textit{fini-ō}. Inflection within each of the aspects is fairly transparent. 

Among the irregular verbs there are instances of (i) stem suppletion yielding wordforms that combine lexical and grammatical content, (ii) stems with ambiguous aspect (\textit{pluit}.\textsc{prs/prf.3sg} `rains/rained'), and (iii) a few stems with aspect or voice meaning that overrides that of their inflection, e.g., \textit{ōd-ī} `hate', \textit{me-min-ī} `remember' (Infective meaning despite Perfective form), \textit{ūt-or} `use', \textit{fru-or} `enjoy' (Active meaning despite Passive form): Introflection. 

The typological paradigm of Latin verb inflection is in \tabref{tab:andersen_15}. In the typological paradigm (\tabref{tab:andersen_15}) I leave aside basic-stem formation to focus on obligatory grammatical categories. The analytic formations, the mainly agglutinative productive conjugations, the fusional Conj. 3 and the introflective suppletive verbs make for an apparent historical perspective. The unproductive patterns reflect (original aorist and perfect) formations from the distant past of the language. The productive formations may have ancient ancestors too \citep{Sihler2010}, as may the Perfective Passive. But in the synchronic view, the transparent productive patterns of inflection and the unproductive ones form a clear reflection of the historicity of the system of conjugation. 

\begin{table}
\caption{Latin conjugation in typological perspective\label{tab:andersen_15}}
\begin{tabularx}{\textwidth}{lQ}
\lsptoprule
Analytic & Auxiliated Perfective Passive and Perfective of deponent verbs: ‘be' + \textsc{pst.pass.ptcp}; auxiliated Future: ‘be' + \textsc{Fut.ptcp} \\
Agglutination & Suffixal Perfective \textit{-v-}/\textit{-u-} in Conj. 1, 2, 4. 
Regular Tense and Participant inflection within each aspect. Mainly agglutinative; some phonotactic and phonological adjustments. \\
Fusion & Tense and Indicative vs Subjunctive are cumulative; Tense suffixes indicate Aspect; \newline
        Conj. 3, 4 Future \textit{-a}/\textit{e-} allomorphy indicates person; \newline
        \textsc{P.p.p}.: \textit{-t-} {\textasciitilde} \textit{-s-}, e.g., \textit{mittō–missus}, \textit{cadō–cāsus}, \textit{tendō–tensus}.\newline
        Participant desinences: Some indicate Aspect; some are cumulative, especially Passive. Unproductive Perfective stem formation (Conj. 3): Types \textit{capiō–cēpī}, \textit{rumpō–rūpī}, \textit{agō–ēgī}, \textit{currō–cucurrī}, \textit{tangō–tetigī}, ....\\
Introflection & (i) Suppletion: \textit{ferō–tulī–lātum}, \textit{tollō–sustulī–sublātum}; \textit{sum}, \textit{es-}, \textit{sunt}, \textit{eram}, \textit{fuī}, \textit{possum–potuī}; \textit{volō}, \textit{vīs}, \textit{vult}, \textit{velle} ...; (ii) Stems combining lexical and grammatical content, despite inflection: \textit{ōdī}; \textit{frūor}\\
\lspbottomrule
\end{tabularx}
\end{table}

\subsection{Russian conjugation}  \label{russian_conjugation}

The hierarchy of obligatory verbal categories in Russian \REF{ex:andersen_7} \citep{Jakobson1956b} is similar to that of Latin:

\ea
    \label{ex:andersen_8}
          \textsc{person} ] \textsc{number} ] \textsc{tense} ] \textsc{aspect} ] \textsc{mood} ] \textsc{voice} ]   
\z

\begin{description}
\item[Voice:] The Passive is analytic: ‘be' + \textsc{pst.pass.ptcp}; the Passive–middle voice is agglutinative, expressed by a fixed verbal clitic with phonologically conditioned allomorphy (=\textit{s{\textquotesingle}a} {\textasciitilde} =\textit{s}{\textquotesingle}).

\item[Mood:] The Irrealis is agglutinative; it is expressed by Past tense or Infinitive plus the movable enclitic =\textit{by}. The Imperative–Hortative is inflected for Person and enclitic Number (=\textit{te}; see below). The clitics follow person, gender, and (in participles) case desinences; possible clitic orders are: \textit{=sja/s{\textquotesingle}=by}, \textit{=te=s{\textquotesingle}}.

\item[Aspect:] Russian aspect is often characterized as derivational (thus \citetv{chapters/04_wiemer}). It is in fact expressed by stem affixation, but Aspect differs from all lexical derivational categories in the language by being an obligatory grammatical category. 
\end{description}

In a discussion of morphological techniques a first distinction must be made between the non-obligatory semantic categories of essentially monoaspectual “procedural” verbs\footnote{R \textit{sposoby glagol`nogo dejstvija}; \citet[385--418]{Issatschenko1962}; \url{https://russkiyyazik.ru/889/}}, which will not be discussed here. We will focus on the “basic” non-procedural verbs. As Imperfectives these represent states or activities, and as Perfectives they represent results of activities. 

The foundation of this lexico-grammatical category is a large and open class of simplex Imperfective verbs \REF{ex:andersen_9a} and a few dozen simplex Perfective verbs \REF{ex:andersen_9b} \citep[352–355, 381–385]{Issatschenko1962}. These two groups of primary simplicia, in which each lexeme combines lexical and aspectual meaning, can be considered introflective. Secondary Perfective verbs are formed from simplex Imperfectives by prefixation \citep{JandaEtAl2013}. In \REF{ex:andersen_9c} perfectivization is agglutinative. Secondary Imperfective verbs are formed from both primary and secondary Perfectives by suffixation (\textit{-a, -va, -iva}), regularly accompanied by mutation of stem vowel and/or stem-final consonant and stress displacement. In \REF{ex:andersen_9d} imperfectivization is fusional. It must be acknowledged that the vast majority of Russian verbs are old and replete with codified semantic extensions. But the processes of Perfectivization \REF{ex:andersen_9c} and Imperfectivization \REF{ex:andersen_9d} are perfectly productive and apply to neologisms. 

\ea \label{ex:andersen_9}
    \ea \label{ex:andersen_9a} Primary Imperfective. E.g., \textit{pisá-t'} `write', \textit{rabóta-t'} ‘work';
    \ex \label{ex:andersen_9b} Primary Perfective. E.g., \textit{liší-t'} ‘deprive', \textit{da-t'} ‘give';
    \ex \label{ex:andersen_9c} Secondary Perfective. E.g., \textit{na-pisá-t'} ‘write', \textit{s-pisá-t'} ‘copy', \textit{pod-pisá-t`} ‘sign', \textit{za-rabóta-t'} ‘earn (lit.: work in)', \textit{pro-rabóta-t'} ‘study, analyze (lit.: work through)';
    \ex \label{ex:andersen_9d} Secondary Imperfective. E.g. \textit{liš-á-t'} ‘deprive', \textit{da-vá-t'} `give', \textit{s-pís-yva-t'} ‘copy', \textit{pod-pís-yva-t'} ‘sign', \textit{za-rabát-yva-t'} ‘earn', \textit{pro-rabát-yva-t'} ‘study, analyse'.
    \z
    \z

\begin{description}
\item[Tense:] The two tenses are Past and Present. The verb \textit{byt'} in addition has a future tense \textit{búd-u} ‘will be', which serves as auxiliary with Imperfective verbs to present a state or acivity as future. The morphological present of perfective verbs regularly has future reference.

\item[Participant categories:] Tense suffixes are followed by participant suffixes, Person and Number in the Present tense, Gender or Number in the Past tense. These are mainly agglutinative, but Person and Number are cumulative in the Present. 
\end{description}

To these can be added the deverbal (nominal) Infinitive, (adjectival) participles, and (adverbial) gerund. Their expression of Aspect and Tense (or Taxis; \citealt{Jakobson1956b}) is fusional.

A key morphophonemic fact of Russian verb inflection is that Infinitive and Past endings begin with consonants, whereas Present and Imperative endings begin with vowels (in the Imperative, alternating with -∅). The endings entail different phonotactic adjustments of stems that end in a consonant and stems that end in a vowel. 

There are four productive inflection classes (\ref{ex:andersen_10a}--\ref{ex:andersen_10d}). The majority of Russian verbs, which includes most secondary Perfectives, most primary, and all secondary Imperfectives \REF{ex:andersen_10a}, have a stem in ...\textit{a-} or ...\textit{e-} in Infinitive and Past and a stem in ...\textit{aj-} or ...\textit{ej-} in Present and Imperative. There is evidence from historical morphology and from child speech that synchronically the /j/ is inserted between a stem-final vowel and a vocalic ending (cf. \citealt{Andersen1980}). The /j/ epenthesis counts as a phonotactic adjustment, compatible with agglutination. The three other productive patterns are fusional: A basic vowel-final stem is modified in Present and Imperative by a suffix-allomorph replacement \REF{ex:andersen_10b}, by truncation \REF{ex:andersen_10c}, or by truncation and a mutation of the \textsc{Prs.1sg} stem-final consonant \REF{ex:andersen_10d}. The mutation is one of several consonant mutations that reduce the number of phonological distinctions in specific derivational and inflectional environments \citep{Andersen1995}, and which are regular in verb inflection. 

\ea \label{ex:andersen_10} \ea
   \label{ex:andersen_10a} \textit{déla-t'–déla-j-ut} \textsc{prs.3pl} ‘do, make', \textit{belé-t'–belé-j-ut} ‘whiten';
 \ex \label{ex:andersen_10b} \textit{tolk-ová-t'}–\textit{tolk-új-ut} \textsc{prs.3pl} ‘interpret';
 \ex \label{ex:andersen_10c} \textit{mók-nu-t'}–\textit{mók-n\textsuperscript{0}}\textit{{}-ut.}\textsc{prs.3pl} ‘get wet'; 
 \ex \label{ex:andersen_10d} \textit{prosí-t'}–\textit{proš\textsuperscript{0}}\textit{{}-ú}.\textsc{prs.1sg} ‘ask'
 \z
 \z


There are (synchronically) underived verbs that pattern with the productive formations \REF{ex:andersen_10a} and \REF{ex:andersen_10d}. In addition to these, there are some two dozen groups of additional simplex verbs with unproductive subregularities. In those, stem allomorphy indicates the grammatical content of endings, that is, they exemplify fusion. A few of them have suppletive stems, e.g., \textit{sést`} `sit down': \textit{s{\textquotesingle}é-}.\textsc{inf/pst}, \textit{s{\textquotesingle}ád-}.\textsc{prs/impv}; (\textit{po})\textit{nját'} `understand': \textit{-n{\textquotesingle}a-}.\textsc{inf/pst}, \textit{-jm-}.\textsc{prs-impv}. 

Besides these suppletive verbs there are some that differ from the regular alternation of Infinitive–Past stem and Present–Imperative stem: \textit{éxat`} `ride': \textit{jéxa–}.\textsc{inf/pst}, \textit{jéd-}.\textsc{prs}, \textit{pojezžáj-}.\textsc{impv}; \textit{idtí} `go, walk': \textit{i-}.\textsc{inf}, \textit{id-}.\textsc{prs/impv}, \textit{šol-}.\textsc{pst}; \textit{léč'} `lie down': \textit{l{\textquotesingle}é-}.\textsc{inf}, \textit{l{\textquotesingle}ág-}.\textsc{prs/impv}, \textit{l{\textquotesingle}og-}.\textsc{pst}; \textit{ést} `eat': \textit{jé-} default, \textit{jed'-}.\textsc{prs.pl}; \textit{dát'} `give': \textit{da-} default, \textit{dad-}.\textsc{prs.pl}, \textit{daj-}.\textsc{impv}; \textit{být'} `be`: \textit{jést'}.\textsc{prs.3sg}, \textit{sút'}.\textsc{prs.3pl} (bookish), \textit{búd-}.\textsc{fut}. 

These suppletive stems cooccur with grammatical morphemes in concatenations that can be viewed as fusional. But some of their alternant stems are limited to, and hence are indexes of, specific grammatical content. In practical terms, then, they can be considered introflective; compare the similarly ambiguous status of the English Preterite 2 verbs (\sectref{andersen_sec3}, note 3). 

\begin{table}
\caption{Russian verb inflection in typological perspective}
\label{tab:andersen_16}
\begin{tabularx}{\textwidth}{lQQ}
\lsptoprule
Analytic & Passive: `be' + \textsc{pst.pass.ptcp}. Passive-Middle: verb + \textit{=sja}/\textit{s'} & Irrealis: Past or Inf + \textit{=by} Futures: \textit{búdut}, \textit{stánut} + Inf\\
Agglutination & Aspect: Prefixal Perfectivization; & Infinitive …V-\textit{t'}.\\
 & Present, Imperative:  /j/ epenthesis: Productive: \textit{déla-t'–déla-j-ut}, \textit{belé-t'–belé-j-ut} \REF{ex:andersen_10a}; unproductive \textit{zna-t'} type (>20) & Past. ...V\textit{-l-} + Gender/Number\\
Fusion & \multicolumn{2}{p{9.5cm}}{Aspect: Suffixal Imperfectivization in /\textit{-a -va -iva}/ with mutations and displacement;}\\
  & \multicolumn{2}{p{9.5cm}}{Productive: \textit{tolk-ová-t'}–\textit{tolk-új-ut}: suffix allomorphy \REF{ex:andersen_10b}; \textit{mók-nu-t'}–\textit{mók-n\textsuperscript{0}}\textit{{}-ut}: V truncation \REF{ex:andersen_10c}; \textit{prosí-t'}–\textit{proš\textsuperscript{0}}\textit{{}-ú}: V truncation and C mutation \REF{ex:andersen_10d}.}\\
  & \multicolumn{2}{p{9.5cm}}{Unproductive: \textit{víde-t'–víž\textsuperscript{0}}\textit{{}-u} (> 50), \textit{deržá-t'–derž\textsuperscript{0}}\textit{{}-ú} (>30).} \\
  & \multicolumn{2}{p{9.5cm}}{Unproductive patterns (>180 verbs): \textit{davá-t'} (+ 2 more), \textit{krý-t'} (+ 6), \textit{bí-t'} (+ 4), \textit{ží-t'} (+ 2), \textit{dé-t'} (+ 3), \textit{žá-t'} (+ 4), \textit{ple-stí} (+ 17), \textit{pé-č'} (+ 12), \textit{nes-tí} (+ 6), \textit{teré-t'} (+ 4), \textit{móknu-t'} (+ 59), \textit{pisá-t'} (+ 50), \textit{ždá-t'} (+ 9).} \\
Introflection & \multicolumn{2}{p{9.5cm}}{Aspect: Simplex Perfectives and Imperfectives.}   \\
 & \multicolumn{2}{p{9.5cm}}{Irregular stem alternants (>20 verbs): \textit{jéxa–/jed–/pojezžaj–; i–/id–/šol–; l{\textquotesingle}é–/l{\textquotesingle}og–/l{\textquotesingle}ág–; s{\textquotesingle}é–/s{\textquotesingle}ád–; -n'a–/-jm–; -jé–/jed'–; da–/dad–/daj–; bi–/jést'/sút'/búd–}.}\\
\lspbottomrule
\end{tabularx}
\end{table}

The synchronic overview of Russian verb inflection in \tabref{tab:andersen_16} only partly corresponds to the historical perspective it suggests. 

The analytic Futures are quite young, first attested in the 1300s, though as a category, with different auxiliaries, the analytic Future must have originated before the 1000s \citep{Andersen2006a}. Also, the productive agglutinative verbs with /j/ epenthesis in the Present are likely younger than the unproductive fusional patterns. At the other extreme, the introflective verbs do include some of the oldest, and long since unproductive, formations, originally athematic (\textit{ést'} `eat', \textit{dát'} `give') and infixed presents (\textit{léč'–ljágut} `lie down', \textit{sést'–sjádut} `sit down'). 

But the agglutinative prefixation (serving Perfectivization) is as old as preverbs in other Indo-European languages; they were grammatized as Perfectivizers in recent prehistory. The fusional Imperfectivizing suffixation is specific to Slavic and cannot be much younger. Again, the agglutinative Russian Past tense developed as a Perfect in the Middle Ages, being regrammatized as a simple past no later than the 1200s, whereas the analytic \textsc{pst.pass} with participles in \textit{-en-} and \textit{-t-} has ancient origins. 

We are reminded of the English verb system, in which the fusional Preterite 2 is younger than the agglutinative Preterite 1, and the apparent historicity of the system to some extent is at odds with its known history. 

\section{Conclusion} \label{andersen_sec3}

In this study of paradigms of paradigms I have drawn attention to notable differences between the hierarchical structures constituted by paradigms of morphological categories (\sectref{content_paradigm}) and the mainly flat or mixed flat–hierarchical paradigms of exponent allomorphs (\sectref{exponent_paradigm}). Such language-particular paradigms of paradigms will long occupy students of morphology, and although they are language-particular a detailed study of them can be expected to yield insights into common and perhaps universal principles that underlie their organization. 

Against this background I have looked at the presumably universal paradigm of  morphological techniques that defines synchronic typological variation (\sectref{andersen_sec2}). I have limited the definitions of these techniques to the bare bones (\tabref{tab:andersen_1}); a discussion of details will be offered elsewhere. 

My aim in this regard has been twofold: 

First of all, I wished to highlight the fact that synchronic typological paradigms may afford an extraordinary perspective on the historicity of inflectional systems. In each of the systems sketched here the typological paradigm points up the contrast between the less restricted, productive formations and the more restricted, unproductive ones, between the systems' younger and older parts. This imprint of history by and large reflects the well-known Morphological Cycle \citep{Hodge1970}. True enough, as emphasized in \sectref{english_conjugation}, a synchronic paradigm of morphological techniques is unlikely to directly reflect the historical development of the given system. For languages whose history we know it is clear that besides a main-stream development from analysis to agglutination to fusion to introflection, there are many renewals that loop back from each of the synthetic types to structurally simpler techniques (\citealt{Werner1987}, \citealt{Igartua2015}).  

Still, the perceived historicity of any typological paradigm similar to \tabref{tab:andersen_12} and Tables~\ref{tab:andersen_14}--\ref{tab:andersen_16} implies hypotheses about past developments. In any language that lacks a historical record, this perspective extends an invitation to historical linguists to uncover the actual historical past of the given language through internal reconstruction. 

But more importantly, I wished to advocate for an approach to morphological description that acknowledges typological paradigms. In simple, practical terms, the paradigms of morphological techniques that can be observed in language after language show us that only an approach to morphological analysis that captures this synchronic variation can attain descriptive adequacy. 

A theory of morphology that presumed all inflection to be agglutinative would be artificial and inadequate \citep{Hockett1954}, not to say useless. A Word-and-Par\-a\-digm approach that operates with unanalysed wordforms as if all inflectional systems were introflective is no better. The recent advance into the dead end of Word-and-Paradigm theory by \citet{StumpFinkel2017} divides the wordforms of inflectional paradigms into stable and alternating fragments (termed ``themes” and ``plats”) that are divorced from both lexical and grammatical content; e.g., Fr. \textit{mouvoir} ‘move': \textsc{m}{}- + -uvwar, -ø, -uv, -œv, ...; \textit{mourir} ‘die': \textsc{m}{}- + urir, -urõ, -œr, -ɔr; or \textit{moudre} ‘grind': \textsc{mu}{}- + -dr, -l, -ly...). This approach achieves descriptions that are truly meaningless.

Due attention to synchronic variation in inflectional morphology – stylistic, inflection-class, allomorphic, and typological – and to the innovations that give rise to such variation will convince the interested linguist of the priority – in the minds of speakers – of productive patterns over unproductive ones, of regular patterns over irregular ones, and of speakers' concern with ultimate elements of exponence and their correlations with elements of meaning (symbolic as well as indexical). 

Since all such variation reflects the speakers' competence, it must be acknowledged in any theory of morphology, and recognized as an inomissible part of any adequate synchronic description.\footnote{I am grateful to Lars Heltoft and Lene Schøsler for their insightful comments on an early draft of this paper.}

\section*{Abbreviations}
\begin{multicols}{2}
\begin{tabbing}
\textsc{cond}\hspace{1em}\= participle\kill
\scshape 1  \> first person\\
\scshape 2  \> second person\\
\scshape 3  \> third person\\
\scshape abl  \> ablative\\
\scshape acc  \> accusative\\
\scshape cond  \> conditional\\
\scshape conj  \> conjugation\\
\scshape dat  \> dative\\
\scshape f  \> feminine\\
\scshape fr.  \> French\\
\scshape fut  \> future\\
\scshape gen  \> genitive\\
\scshape impf  \> imperfect\\
\scshape impv  \> imperative\\
\scshape ind  \> indicative\\
\scshape inf  \> infinitive\\
\scshape infv  \> infective\\
\scshape ipfv  \> imperfective\\
\scshape lat.  \> Latin\\
\scshape loc  \> locative\\
\scshape lit.  \> literally\\
\scshape m  \> masculine\\
\scshape n  \> neuter\\
\scshape nom  \> nominative\\
\scshape p.p.p.  \> past passive participle\\
\scshape pass  \> passive\\
\scshape pfv  \> perfective\\
\scshape pl  \> plural\\
\scshape plup  \> pluperfect\\
\scshape prf  \> perfect\\
\scshape prs  \> present\\
\scshape pst  \> past\\
\scshape ptcp  \> participle\\
\scshape r  \> Russian\\
\scshape sbj  \> subjunctive\\
\scshape sg  \> singular\\
\scshape vl  \> voiceless
\end{tabbing}
\end{multicols}

{\sloppy\printbibliography[heading=subbibliography,notkeyword=this]}
\end{document}
