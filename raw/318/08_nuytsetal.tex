\documentclass[output=paper]{langsci/langscibook} 
\ChapterDOI{10.5281/zenodo.5675853}
\author{Jan Nuyts\affiliation{University of Antwerp} and Wim Caers\affiliation{University of Antwerp} and Henri-Joseph Goelen\affiliation{University of Antwerp}}
\title{The Dutch modals, a paradigm?}
\abstract{This paper addresses the question how to define the notion of a “paradigm”, as a cognitively real phenomenon. The discussion is based on the case of a set of forms from a linguistic class that is not part of the traditional domain of “paradigmhood” (i.e. inflectional morphology): the modal auxiliaries in Dutch. The paper presents the results of a few studies into the diachronic evolution of these forms, grammatically and semantically, showing how a subset of them has gradually accumulated shared features and developed an internal division of labor, thus displaying active group behavior.}

\begin{document}
\maketitle 

\section{Introduction}\label{nuyts:1}\largerpage

In this paper we address the question how to define the notion of a “paradigm”, as a cognitively real phenomenon. We do so by means of the concrete case of a set of forms that may arguably be classified as a paradigm, from a linguistic form class that does not belong in the traditional domain of “paradigmhood” (i.e. inflectional morphology): the modal auxiliaries in Dutch. We present the results of a few studies into the diachronic evolution of these forms, grammatically and semantically, showing how a subset of them has gradually accumulated shared features and developed an internal division of labor, thus displaying active group behavior.

The paper is organized as follows. In \sectref{nuyts:2} we address the issue of how to define a paradigm, as a cognitively real phenomenon. In \sectref{nuyts:3} we briefly lay out the methodology used in the studies into the diachronic evolution of the Dutch modals. \sectref{nuyts:4} presents a bird's-eye overview of the grammatical evolutions in this set of verbs. \sectref{nuyts:5} surveys the semantic developments. In \sectref{nuyts:6} we formulate the conclusions.

\section{Paradigms} \label{nuyts:2}

One of the central issues addressed in the present volume is: how cognitively real is the concept of a linguistic paradigm? The only way to answer this question is to look at the linguistic behavior of the members of a candidate for the label.

This, however, invokes another crucial, and maybe more controversial, issue: what are the necessary and sufficient conditions for categorizing a set of forms as a paradigm? We assume that a diagnosis should be based on three very elementary criteria: (i) Do the forms share properties or characteristics, and do they show a tendency towards increasing convergence over time? (ii) Is there a kind of internal organization within the group of forms, and do the members show developments sensitive to it, for instance in view of optimizing the division of labor between them? And (iii) does the set of forms occupy a distinct position, structurally and/or functionally, in the overall linguistic system of the language? 

If the answer to at least some of these questions is “sufficiently positive” (see below), one is entitled to call the set of forms a paradigm. Moreover, one then has to consider it cognitively real: it is demonstrably a significant element in the organization of the verbal behavior produced by the cognitive systems for language use implemented in the brains of the speakers of the language, hence it is somehow “represented” in the latter (in a non-literal sense of “represented”: it somehow has a specific, recognizable status in the language users’ “cognitive grammar”).

This answer raises at least three new questions, however, to which the answer is in part less straightforward.

First of all, what is a “sufficiently positive” answer to the above questions? How many and what kinds of features and tendencies do the forms have to share in order for the set to be called a paradigm? How stringent does the internal organization of the set have to be? How distinct should it be in the overall linguistic system? This may be the wrong way to formulate the issue, though: paradigmhood is not a black-and-white matter, but a graded one (hence the concept of “necessary and sufficient conditions” for paradigmhood is quite relative). Paradigms come in degrees of integration, internal organization, and distinctness in the linguistic system, and there is no hard cut-off point for membership of the category. This corresponds to how paradigms come into existence diachronically. They do not appear all of a sudden in one fell swoop. They emerge and develop gradually, potentially over a long period of time (and they will also gradually and slowly disintegrate and disappear again). It is pointless to try to determine one specific point on this developmental “cline” at which paradigmhood starts/stops.

Secondly, what types of sets of linguistic forms may be considered candidates for paradigmhood? The notion of a paradigm is traditionally more or less confined to obligatory inflectional systems in a language (cf. e.g. \citealt{DiewaldSmirnova2010, Blevins2015}). These nearly automatically satisfy the criteria of sharing sufficient features, showing a stringent internal organization, and taking a distinct position in the overall linguistic system of the language. Yet the definition of a paradigm in terms of the above diagnostic criteria does not imply in any way that only an inflectional system, or only an obligatory system, can count as such. There is no reason why other types of linguistic forms could not fulfil the criteria for paradigmhood as well, including non-inflectional grammatical forms such as auxiliaries, or forms of which the status as grammatical vs lexical is controversial such as adverbs and adjectives (cf. e.g. the modal adverbs/adjectives in languages), or even clearly lexical categories such as main verbs (cf. e.g. the perception verbs, communication verbs, or mental state verbs in languages), even if these are hardly ever obligatory elements in the grammar of a language. The fact that, diachronically, inflection often emerges from auxiliaries, which in turn typically develop out of main verbs (i.e. one of the classical examples of a grammaticalization path), further underscores this point (cf. the preceding issue regarding the gradual emergence of paradigms).

This raises a third question, pertaining to another central issue addressed in this volume: is there a necessary link between paradigms and grammaticalization? Probably not. There is a tendency in the literature to strongly associate paradigms with grammaticalization processes (cf. the fact that \citealt{Lehmann1982} considers paradigmatization a key feature of grammaticalization; see e.g. \citealt{Diewald2009, Nørgård-Sørensen2011}). This is to a large extent justified: grammaticalization typically does result in, and may even be the most important trigger for, the formation of  paradigms (all inflectional paradigms are due to grammaticalization). (This, however, raises a chicken-and-egg question: does the strive in a linguistic system to form paradigms trigger grammaticalization, or is the formation of paradigms a side-effect of grammaticalization?) Nevertheless, there is nothing in the three diagnostic criteria implying that the formation of paradigms must necessarily always involve grammaticalization.\footnote{Some authors would seem to consider grammaticalization a necessary condition for a set of forms to be called a paradigm. We do not adopt this perspective. Paradigmhood is a basic analytical notion, required to describe a type of pattern observable in collections of forms in a language. Tying it up with another analytical notion, in casu grammaticalization, means reducing its value as a descriptive notion (and in a way even means making it superfluous, since what it describes is also covered by the notion of grammaticalization). It also means that if one encounters sets of non-grammaticalized forms with essentially the same basic properties (in terms of the criteria specified earlier), one needs to introduce another term for these, and one thereby looses the generalization that one is dealing with the same basic phenomenon.} 

In this paper we offer an example of a set of forms that may arguably be called a paradigm and that demonstrates the import of the three questions just raised. It concerns a grammatical system in a domain other than inflection, which illustrates the gradualness of the notion of a paradigm as defined in terms of the three diagnostic criteria mentioned earlier, and which shows that grammaticalization is not a necessary correlate of paradigmhood. Our casus belli is the set of modal auxiliaries in Dutch. In the traditional view there are six such verbs in the language: \textit{kunnen} ‘can’, \textit{mogen} ‘may’, \textit{moeten} ‘must’, \textit{hoeven} ‘need’, \textit{zullen} ‘shall/will’, and \textit{willen} ‘will/want’ (see e.g. \citealt[383--437]{Duinhoven1997}). Corpus investigations into the diachronic evolution of these verbs reveal that four of them, \textit{kunnen}, \textit{mogen}, \textit{moeten} and \textit{hoeven}, show systematic behavior, with a tendency to increase and strengthen shared properties over time, grammatically as well as semantically, while at the same time avoiding functional “conflict” within the set. The other two verbs, \textit{zullen} and \textit{willen}, do not participate in these evolutions. In this paper we will present an overview of the facts to this effect.

\section{Data} \label{nuyts:3}

The considerations in this paper are predominantly based on corpus studies into the grammatical and (in part) the semantic developments of the six modal verbs. Although we will only offer a summary overview of the observations from these studies relevant for our present purpose, we should briefly sketch the method applied in them (more details can be found in the references given below).

Our primary source of information are investigations into the global evolution of each of the six modal verbs from the earliest known documents till the present (see \citealt{Nuyts2013, BylooNuyts2014, NuytsByloo2015, NuytsEtAl2018, NuytsEtAlinPrep}). These studies all used the same method and analytical categories.

They compared the grammatical and, in part (see below), the semantic properties of the verbs in samples of (in principle) 200 instances per modal from four main “time slices” (based on the generally accepted division in main stages in the history of the language): Old Dutch (OD, >\,1150), Early Middle Dutch (EMD, 1250–1300, the start of the Middle Dutch period), Early New Dutch (END, 1550–1650, the start of the New Dutch period), and Present Day Dutch (PDD, <\,1950).\footnote{There is some dispute over the precise temporal demarcation of the stages, the studies followed \citet{vandenToorn1997}. Occasionally samples are smaller than 200, because the materials for the period did not contain more instances of the modal.} For PDD the studies worked with separate samples of 200 instances for the written language (PDDW) and the spoken language (PDDS).

The samples were drawn from a self-compiled balanced corpus of materials covering these periods. The selection of texts for inclusion in the corpus, and of instances for the samples, was based on criteria such as representativity (in terms of text genres and regional spreading, among others) and comparability across the periods. Within these confines, the selection was random.

The data were analyzed paying due attention to inter-rater-reliability and making use of statistical tools (Fisher's exact and the Spearman rank coefficient of correlation).

As a secondary source, we will occasionally refer to a follow-up study which focused on the grammatical developments from END onwards in \textit{kunnen}, \textit{mogen}, \textit{moeten} and \textit{hoeven} (see \citealt{CaersNuyts2021}). In comparison with the earlier studies, it considered an additional language stage, half way between the stages of END and PDD, labeled New Dutch (ND, 1750–1850), and it worked with much larger samples of (in principle) 1000 instances. Otherwise the methodology was the same as in the earlier studies.

In this paper we will only present the relevant headlines emerging from these studies. For the full story (including tables with detailed frequency information) the reader is referred to the publications mentioned above.

The semantic evolution of \textit{zullen} and \textit{willen} was not investigated in the studies mentioned. Our discussion of this issue in \sectref{nuyts:5} is based on the information about these verbs provided in the major general and historical dictionaries for Dutch: the \textit{Oudnederlands woordenboek} (\citeyear{Oudnederlandswoordenboek2012}) for OD, the \textit{Vroegmiddelnederlands woordenboek} \citep{PijnenburgEtAl2000} for EMD, the \textit{Middelnederlandsch woordenboek} (\citealt{Verwijs1885}) for Middle Dutch, and the \textit{Woordenboek der Nederlandsche taal} (\citealt{DeVriesTeWinkel1864}) for the developments from Middle Dutch till the 20th century.\largerpage

All examples provided in this paper are from the corpus data used in the studies mentioned above (they were not necessarily cited in the publications, though). The sample from which they were taken (OD, EMD, END, etc.) is mentioned between brackets after the example. The relevant form is boldfaced. In examples from the spoken data we maintain the transcription conventions used in the source corpus (no capitals, no punctuation, etc.; the spoken samples were drawn from the \textit{Corpus Gesproken Nederlands} (\citeyear{CorpusgesprokenNederlands2004}), but the transcript has occasionally been simplified somewhat by omitting, among others, irrelevant pause fillers, repetitions, or back channel cues (without this being marked in the example).

\section{The grammatical developments} \label{nuyts:4}

The grammatical developments of the modals, from OD till PDD, were outlined and discussed in detail in \citet{Nuyts2013} for \textit{kunnen}, \textit{mogen} and \textit{moeten}, in \citet{NuytsEtAl2018} for \textit{hoeven}, and in \citet{NuytsEtAlinPrep} for \textit{zullen} and \textit{willen}.

Like most auxiliaries in the languages of the world, the Dutch modals have all emerged from main verbs through a regular process of grammaticalization (\citealt{HopperTraugott2003}). (See further below in this section on the definition of the main verbal vs auxiliary status of the modals.) \textit{Kunnen}, \textit{mogen}, \textit{moeten} and \textit{zullen} are preterit-presents, verbs of which the current present tense stem was originally the past tense stem. \textit{Hoeven} and \textit{willen}, however, are not (although \textit{willen} has adopted some of the characteristics of the preterit-presents, see \citealt{DeVriesTeWinkel1864}). The timing of the evolution towards an auxiliary is not the same in all the verbs, however.

In \textit{mogen}, \textit{moeten} and \textit{zullen} the auxiliarization process is more or less completed already in the OD sources. There is no trace of the original main verbal use of \textit{zullen} (which meant ‘to owe something (to someone)’) in our data for any of the periods.\footnote{The \textit{Oudnederlands woordenboek} (\citeyear{Oudnederlandswoordenboek2012}) mentions one single instance, occurring in the \textit{Mittelfränkische Reimbibel}. The linguistic status of this text is under dispute, however (Old Dutch or Old High German?). For that reason it was not included in the data in \citet{NuytsEtAlinPrep}.}  In the data for \textit{mogen} and \textit{moeten} (which originally meant, respectively, ‘to have power’ and, probably, ‘to measure’) there are possible relics of the original main verb (with a substantially different meaning, though), even until today. But these are marginal, in all time slots, and these modals, too, were nearly exclusively auxiliary already in OD and EMD.

In \textit{kunnen}, however, the auxiliarization process is in full course in OD and EMD. The auxiliary use is already dominant then, but the old main verbal use, with the meaning ‘to know’, is still prominently present. It even exists until today, even if it is fairly rare now – \REF{ex:nuyts:1} is a PDD example.

\ea
    \label{ex:nuyts:1}
    \gll hij \textit{kon} al een beetje Spaans \\
       he could already a bit Spanish   \\\jambox*{(PDDS)}
    \glt [lit.] `He could some Spanish already.' [i.e.] `He already knew some Spanish.'
    \z

In \textit{hoeven} the evolution is even more recent. This modal has only emerged around the start of the END period, out of the main verb \textit{behoeven}, which meant, and still means, ‘to need’. \textit{Hoeven} shows clear traces of its main verbal origins until today, illustrated in \REF{ex:nuyts:2} (\textit{hoeven} can be replaced freely by \textit{behoeven} in the example). Nevertheless, from its conception in END onwards it predominantly behaves as an auxiliary.

\ea
    \label{ex:nuyts:2}
    \gll Anderen vonden oplossingen: ze lezen grootgedrukte boeken of kijken enkel naar Nederlands gesproken programma's waarvoor ze geen ondertitels \textit{hoeven}. \\
 others found solutions they read large.print books or watch only to Dutch spoken programs for.which they no subtitles need\\\jambox*{(PDDW)}
    \glt  `Others [elderly people with poor eye sight] found solutions: they read books in large print or only watch programs spoken in Dutch for which they do not need subtitles.'
\z


\textit{Willen}, finally, is predominantly an auxiliary from the oldest sources onwards, but the original main verbal use, meaning ‘to wish’, ‘to desire’, was still very prominent in EMD, and, even if declining through time, remains present until today. \REF{ex:nuyts:3} is an example.

\ea
    \label{ex:nuyts:3}
    \gll Ik heb deze situatie niet \textit{gewild}.\\
         I have this situation not wanted \\\jambox*{(PDDW)}
    \glt  `I did not want this situation.'
    \z




In sum, in terms of its origins the set of modal verbs in Dutch (as traditionally conceived) emerged only very gradually and unsystematically.

In more recent grammatical developments, however, a subpart of the traditional set (if it has ever been a real set at all)\footnote{There are many more auxiliary verbs in Dutch, hence the fact alone that the six modal verbs have auxiliarized is not sufficient to call them a paradigm. As we will argue in \sectref{nuyts:5}, the semantic criterion often adduced for considering them a system is not convincing either.}  – specifically, \textit{kunnen}, \textit{mogen}, \textit{moeten} and \textit{hoeven} – starts to behave in a very similar way, both in terms of the direction and of the timing of the changes. \textit{Zullen} and \textit{willen}, however, do not participate in the evolutions, or at least not clearly so. The fact that it concerns quite remarkable developments makes the observation even more significant.

Thus, in the course of the New Dutch period, \textit{kunnen}, \textit{mogen}, \textit{moeten} and \textit{hoeven} start showing a distinct tendency to regain independence, and to get used again as an autonomous verb in the clause.\footnote{These changes make these four modal verbs in Dutch very different from their equivalents even in closely related languages such as English and German, which do not show comparable developments. They make them special even among the world’s languages, since we appear to be dealing with a process of collective degrammaticalization, considered highly unusual in the literature (cf. \citealt{Norde2009}; see \citealt{Nuyts2013}, \citealt{CaersNuyts2021} for discussion of this issue).} Moreover, the properties of the new autonomous uses are exactly the same in all four verbs. Strongly simplifying matters, these uses come in two main types.

On the one hand there are instances that are presumably still auxiliary, but with an elided main verb. Very occasionally, the elision is due to the fact that the implied main verb has been mentioned in the preceding clause, as in \REF{ex:nuyts:4} (the elided main verb is [\textit{in de tuin}] \textit{werken} ‘work [in the garden]’). In by far most cases, the implied main verb has not been mentioned in the preceding discourse, but is more or less clearly imaginable, as in \REF{ex:nuyts:5} and \REF{ex:nuyts:6} (in both examples the main verb \textit{gaan} ‘go’ is understood).

\ea
    \label{ex:nuyts:4}
    \gll weet je waar da{\textquotesingle}k zin in heb in de tuin te werken maar ja {\textquotesingle}k heb geen gerief dus {\textquotesingle}k \textit{kan} niet\\
  know you where that.I desire in have in the garden to work but yes I have no tools so I can not\\\jambox*{(PDDS)}
\glt   `You know what I’d like to do, work in the garden. But I have no tools so I can’t [elided: work in the garden].'
\ex
    \label{ex:nuyts:5}
    \gll die op vijfentwintig zes geboren zijn die \textit{mogen} niet mee\\
       those on twenty.five six born are those may not with   \\\jambox*{(PDDS)}
    \glt  [lit.] `Those born on June 25th may not [implied: go] along.' [i.e.] `Those ... may not join.'
\ex
    \label{ex:nuyts:6}
    \gll je \textit{moet} ook weer met de mode en met de kleur mee\\
    you must also again with the fashion and with the color with\\\jambox*{(PDDS)}
    \glt  [lit.] `You must [implied: go] along with fashion and with the colors.' [i.e.] `You must follow fashion and the popular colors.'
    \z

The fully implicit type in \REF{ex:nuyts:5} and \REF{ex:nuyts:6} is special, though, and differs from the contextual type in \REF{ex:nuyts:4}, in that making the implied main verb explicit usually sounds unnatural and forced to native speakers. The sentence simply feels better without it. Also note that elision of the kind in \REF{ex:nuyts:5} and \REF{ex:nuyts:6} is impossible in English, unlike that in \REF{ex:nuyts:4}.\footnote{The use of a modal without a main verb in the clause (without the latter having been mentioned in the preceding discourse) is to some extent possible in other Germanic languages, including German. Usually this concerns the elision of a motion verb in the presence of a directional phrase, as in German \textit{ich muss jetzt nach Hause} [lit.] ‘I must [implied: go] home now’. The possibilities to omit the main verb in Dutch extend far beyond those in German or other Germanic languages, however (instances such as \REF{ex:nuyts:5} or \REF{ex:nuyts:6}, e.g., would seem impossible in German). The presence of a directional is not required. Main verbal uses of the type in (\ref{ex:nuyts:7}--\ref{ex:nuyts:8}) below would even seem entirely absent in other Germanic languages (and these never feature directionals). (See \citealt{CaersNuyts2021} on the role of directionals in the re-autonomization process in the Dutch modals.)} Possibly, the position of the main verb is getting unstable in these uses, and the modal verb is regaining independence.

On the other hand, there are autonomous instances in which one cannot imagine a main verb next to the modal verb anymore, as in \REF{ex:nuyts:7} and \REF{ex:nuyts:8}. In such cases, the modal itself must be considered the main verb of the clause.\footnote{There is not a sharp borderline between the autonomous uses of the type with an elided main verb and the main verbal type. There are quite a few cases of doubt between the two in the data. This may not be accidental, but may be a sign of a diachronic relationship: presumably, if a verb changes from an auxiliary to a main verbal status, or vice versa, it passes through the stage with an implied main verb.}

\ea%7
    \label{ex:nuyts:7}
   \gll Wat u doet \textit{kan} helemaal niet, een klooster bouwen op het grootste Joodse kerkhof ter wereld!\\
  what you do can at.all not a monastery build on the biggest Jewish cemetery on earth\\\jambox*{(PDDW)}
\glt   [lit.] `What you are doing cannot at all, building a monastery on the largest Jewish cemetery on earth!' [i.e.] `What you are doing is totally unacceptable, ....'
\ex%8
    \label{ex:nuyts:8}
    \gll          En rijdt er tussen het feestgedruis door toch nog soms eens een trein, dan is dat mooi meegenomen. Maar het \textit{hoeft} niet meer per se.\\
  and rides there between the festivities through nevertheless still sometimes once a train then is that nicely taken.along but it need not anymore per se\\\jambox*{(PDDW)}
\glt   [lit.] `And if there is occasionally still a train during the festivities, that is an asset. But it need not absolutely anymore.' [i.e.] `... But it is not absolutely necessary/indispensable anymore.'
\z

The overwhelming majority of these new main verbal uses – including the examples in \REF{ex:nuyts:7} and \REF{ex:nuyts:8} – has a valency pattern that is very different from that of the original main verbal uses illustrated in \REF{ex:nuyts:1} and \REF{ex:nuyts:2} above (hence these new main verbs cannot be considered a continuation of the original main verbs, i.e. the development is not a case of “retraction” in \citegen{Haspelmath2004} sense). The original main verbs are transitive, in all four modals, with a first argument referring to an entity, very often a living being. But the new main verbal uses of the four modals are nearly always intransitive, and their only argument refers to a state of affairs, deictically, or in a complement clause or an equivalent. The difference between the old and the new main verbal use is illustrated again for \textit{kunnen} in (respectively) \REF{ex:nuyts:9} and \REF{ex:nuyts:10}.\footnote{The old and new main verbs are also semantically different. The original meaning of the old main verbal uses (mentioned earlier in this section) is entirely absent in the new main verbs. The latter exclusively feature modal and related meanings: sometimes they express dynamic modality, as in example \REF{ex:nuyts:10}, or in \REF{ex:nuyts:8} above, but they are much more often deontic modal, as in \REF{ex:nuyts:7} above, or directive (expressing a permission or obligation). (See \sectref{nuyts:5} on these meanings; and see \citet{Nuyts2013} for a more elaborate discussion of the meaning difference between the old and new main verbal uses of the modals.)}

\ea%9
    \label{ex:nuyts:9}
    \gll          Dat voel liede sijn die en geen latijn en \textit{conen} noch en verstaen.\\
  that many people are who not no Latin not can nor not understand\\\jambox*{(EMD)}
\glt   [lit.] `That there are many people who can nor understand Latin.' [i.e.] `That there are many people who do not know or understand Latin.'
\ex%10
    \label{ex:nuyts:10}
\gll          iemand van twintig die beroemd wil worden dat \textit{kan} op allerlei manieren\\
  somebody of twenty who famous wants become that can on various ways\\\jambox*{(PDDS)}
\glt   [lit.] `Someone aged twenty who wants to become famous, that can in different ways.' [i.e.] `..., that is possible in different ways.'
\z

There are minor differences between the four modals participating in the re-autonomization process, in terms of the precise timing and/or in terms of how intensively they participate in the developments, as the follow-up study in \citet{CaersNuyts2021} has shown (cf. \sectref{nuyts:3}). The process has started after 1850 in \textit{kunnen}, \textit{mogen} and \textit{moeten}, but already between 1750 and 1850 in \textit{hoeven} (parallel with the ongoing process of auxiliarization in this modal, which only started in END). Moreover, the increase in frequency of the autonomous uses is very substantial in all four verbs (it is statistically highly significant in all of them), but it is most prominent in \textit{hoeven}, followed by \textit{kunnen} in the written data, but by \textit{mogen} in the spoken data, and it is overall weakest in \textit{moeten}.\footnote{See the references given earlier for detailed frequency data. To give an impression of the figures: in the data in \citet{CaersNuyts2021}, in \textit{hoeven} the new autonomous uses jump from app. 1\% of all occurrences of the modal in END to slightly over 8\% in ND and PDDW and to more than 28\% in PDDS. (See \citet{Nuyts2013} and \citet{CaersNuyts2021} for arguments why the more drastic increase in the spoken PDD data, which occurs in all four modals, is not due to sloppiness but signals the direction into which the language is evolving.)}

In spite of these small differences, there is a clear common line in the – from a regular grammaticalization perspective quite unexpected – developments in these four modals. They appear to behave as a system, displaying a collective dynamics, in which analogy between the individual members may play an important role.\footnote{The timing and intensity of the process in the four forms would seem to suggest that \textit{hoeven} has the leading role in it. Yet the question is how this is compatible with the fact that this is by far the youngest among the modals, still in the process of auxiliarizing in the relevant period, as well as with the fact that it is much less frequent than the other modals, and occupies a special position among them as a negative polarity item. These remain open questions.} 

\textit{Zullen} and \textit{willen} do not show a comparable evolution, however. \textit{Zullen} does have a new autonomous use, as illustrated in \REF{ex:nuyts:11}.

\ea%11
    \label{ex:nuyts:11}
    \gll          ik weet niet of ie {\textquotesingle}t heel druk heeft maar {\textquotesingle}t \textit{zal} wel\\
  I know not whether he it very busy has but it shall rather\\\jambox*{(PDDS)}
\glt   [lit.] `I do not know whether he is very busy but it probably will [implied: be] so.' [i.e.] `... but he probably is.'
\z

\noindent It only occurs in the PDDS data, however, and it is very marginal even there (2\% of the instances). It is moreover exclusively of the type with an implied main verb, so these instances are arguably still auxiliary. Also, it exclusively occurs in combination with the modal particle \textit{wel}, as in \REF{ex:nuyts:11}. Maybe the few cases in the data indicate that this verb will ultimately join the others in the process of re-autonomization, but if so, it is at least highly reluctant to do so.

\textit{Willen} also shows autonomous uses, but it featured them from the earliest documents onwards (their frequency fluctuates through time), and these uses have not changed in nature over time. They are all remains of the original main verbal use of the verb. New main verbal uses of the kind in \REF{ex:nuyts:7} and \REF{ex:nuyts:8} (or of any other type) do not occur, in any of the time slices, not even in the PDDS data. Hence this verb shows no signs of a participation in the group dynamics characterizing \textit{kunnen}, \textit{mogen}, \textit{moeten} and \textit{hoeven}.

\section{The semantic developments} \label{nuyts:5}

The semantic developments, from OD till the present, are discussed in detail in \citet{BylooNuyts2014} and \citet{NuytsByloo2015} for \textit{kunnen}, \textit{mogen} and \textit{moeten}, and in \citet{NuytsEtAl2018} for \textit{hoeven}. We have not made an equivalent diachronic meaning analysis for \textit{zullen} and \textit{willen}. Yet the rough outline of the semantic profile of these verbs in different stages of Dutch emerging from the historical dictionaries (see \sectref{nuyts:3}) will suffice for the present purpose.

The semantic developments in the different verbs perfectly mirror the grammatical ones described in \sectref{nuyts:4}: \textit{kunnen}, \textit{mogen}, \textit{moeten} and \textit{hoeven} show similar evolutions, but these are not shared, or at least not to the same extent, by \textit{zullen} and \textit{willen}.

Although the original main verbs from which \textit{kunnen}, \textit{mogen}, \textit{moeten}, and \textit{hoeven} emerged had quite different meanings (cf. \sectref{nuyts:4}: resp. ‘know’, ‘have power’, ‘measure’, and ‘need’), as auxiliaries these verbs have developed more or less the same full range of modal and related meanings typically associated with modal verbs in the languages of the world (see \citealt{Nuyts2006,Nuyts2016} for elaborate definitions and discussion). This includes, in all four verbs, different types of dynamic modal meanings (an ascription of a capacity or possibility, or of a need or necessity, to a participant in the state of affairs, or of a potential or inevitability to the state of affairs as a whole), as in \REF{ex:nuyts:12} (see also \REF{ex:nuyts:4}, \REF{ex:nuyts:8} and \REF{ex:nuyts:10} above).

\ea%12
    \label{ex:nuyts:12}
    \gll           kijk als gij een huishouden hebt en ge \textit{moet} nog vanalles d{\textquotesingle}rbij doen dan is {\textquotesingle}t heel wat anders he\\
  look if you a household have and you must still different.things on.top.of.it do then is it entirely something else right\\\jambox*{(PDDS)}
\glt   `Look, if you are managing a household and you have to do several things on top of it, then you have a different story, right?'
\z

It includes a deontic modal meaning in all four verbs (an assessment of the degree of moral acceptability of the state of affairs), as in \REF{ex:nuyts:13} (see also \REF{ex:nuyts:6} and \REF{ex:nuyts:7} above).

\ea%13
    \label{ex:nuyts:13}
\gll          We onderzoeken nu de authenticiteit van de lak en of deze geretoucheerd is. Bij zo{\textquotesingle}n dure aankoop \textit{mag} je geen risico nemen.\\
  we investigate now the authenticity of the paint and if this retouched is with such.an expensive acquisition may you no risk take\\\jambox*{(PDDW)}
\glt  `We are now investigating whether the paint is authentic and has not been retouched. When buying something so expensive one shouldn’t [lit. may not] take a risk.' 
  \z

Three of the four verbs have also developed an epistemic modal or inferential (evidential) meaning (an assessment of, respectively, the degree of likelihood of the state of affairs, or the degree of reliability of the information resulting in the postulation of the state of affairs), as in \REF{ex:nuyts:14} (the inferential meaning occurs in \textit{moeten}, the epistemic one in the other verbs). This meaning type is missing in \textit{hoeven} – but it is also a very minor one in the three other verbs (in \textit{mogen} it has even disappeared again in PDD).

\ea%14
    \label{ex:nuyts:14}
\gll           Iets minder zon morgen, en er \textit{kan} een buitje vallen.\\
  somewhat less sun tomorrow and there can a small.shower fall\\\jambox*{(PDDW)}
\glt  `Somewhat less sunny tomorrow, and there may be an occasional shower.'
\z

All four verbs have moreover acquired a directive meaning (marking a permission, obligation, advice, etc.), as in \REF{ex:nuyts:15} (see also \REF{ex:nuyts:5} above).\footnote{Directivity is often considered part of deontic modality, but there are good arguments to keep the two categories separated. See \citet{NuytsEtAl2010}. This is of no further importance here.}

\ea%15
    \label{ex:nuyts:15}
\gll          ik heb begrepen dat {\textquotesingle}t in het paspoort niet \textit{hoeft} te worden ingeschreven dus als je dat geheim wilt houden kan dat\\
  I have understood that it in the passport not need to become registered so if you that secret want keep can that\\\jambox*{(PDDS)}
\glt  `I understand that it [one’s marital status] need not be mentioned in the passport [i.e. it is not compulsory], hence if you want to keep it secret that is possible.'
\z

Finally, three of the verbs have developed a volitional meaning (expressing a wish), as in \REF{ex:nuyts:16}. This use is missing in \textit{kunnen}, but it is also minor in the other three verbs.\footnote{Next to the meanings and uses mentioned above, some of the modals have developed yet other minor ones.}

\ea%16
    \label{ex:nuyts:16}
\gll       dat zijn gedemodeerde spullen dat je zegt ja dat \textit{hoef} ik niet meer hé dat \textit{moet} ik niet meer hebben in feite\\
  that are old-fashioned things that you say yes that need I not anymore right that must I not anymore have in fact\\\jambox*{(PDDS)}
\glt   `Those are old-fashioned things, so you think ``I don’t want [lit. need] them anymore'', right, ``I don’t want to [lit. must] own them anymore'', essentially.'
\z

There are considerable differences between the four verbs in terms of when, and how fast, the developments towards these different meanings have happened. In \textit{mogen} and \textit{moeten} the full range of modal and related meanings sketched above is already present in OD and EMD, and the original main verbal meaning is obsolete even then. In \textit{kunnen} a considerable part of the developments happened after OD and EMD. Only the dynamic modal meanings are present then, along with the original meaning ‘to know’. The deontic, epistemic and directive meanings emerge in the course of the evolutions towards END and PDD. In \textit{hoeven}, finally, the evolution happens helter-skelter: all modal and related meanings emerge more or less simultaneously, at the moment when the form arises as an auxiliary, around the start of the END period. These uses immediately assume a dominant position in the semantic profile of the verb, and they gain further ground in the course of the New Dutch period. 

In sum, there appears to have been a semantic “unification” process among these four verbs, aiming to form a linguistic system that allows the expression of complementary meanings within the same range of modal and related semantic dimensions. The process seems to have stepped up after \textit{kunnen} joined \textit{mogen} and \textit{moeten}: it stands to reason that the blitz evolution in \textit{hoeven} is due to a strive in this form to semantically adapt as quickly as possible to the profile of the other three forms. It is significant that the re-autonomization process in the system, as described in \sectref{nuyts:4}, sets in while the semantic evolution in \textit{hoeven} is still in full progress. This confirms that the four verbs at least from then onwards behave as a full-fledged paradigm.

There are indications that interactions between the members of the set already started much earlier, however. There are, for instance, semantic changes in the four verbs that may be the result of a tendency to avoid synonymy between them, within the range of meanings they share (cf. \citealt{NuytsByloo2015}). Thus, \textit{moeten} has evolved from a weak modal (expressing ability, possibility, potential, etc.) to a strong modal (expressing need, necessity, inevitability, etc.) in OD (with last traces of the process in early EMD), possibly in order to avoid semantic overlap with weak \textit{mogen}. Weak \textit{kunnen} has gradually taken over meanings from weak \textit{mogen} from OD and EMD onwards, which may explain why since then \textit{mogen} is increasingly focusing on its directive meaning of permission. And strong \textit{hoeven} may have developed into a negative polarity item (a process which started immediately when this modal emerged around the beginning of the END period) in order to avoid conflict with strong \textit{moeten} (which subsequently has come to dislike negative contexts in PDD).

\textit{Zullen} and \textit{willen}, however, do not seem to participate in the systematic semantic evolutions observed in \textit{kunnen}, \textit{mogen}, \textit{moeten} and \textit{hoeven}.

The earliest meaning evolutions in \textit{zullen} happened before the time of the oldest documents, hence they are unknown. But from OD onwards, and continuously until today, this verb predominantly expresses a temporal meaning, as the marker of the future tense in Dutch. \REF{ex:nuyts:17} is an EMD example (from \citet{PijnenburgEtAl2000}, entry \textit{sullen} in the online edition).

\ea%17
    \label{ex:nuyts:17}
    \gll           Wi debroeders ende desustre van sente ians hus jn ghent doen cont alledenghenen die dese letteren \textit{sullen} zien dat ...\\
  we the.brothers and the.sisters of Saint John’s house in Ghent do announcement all.those who these letters shall see that\\
\glt   `We, brothers and sisters of Saint John’s house in Ghent, announce to all those who will see this letter, that ...'
\z

This verb probably never developed a dynamic modal meaning (the \textit{Oudnederlands woordenboek} (\citeyear{Oudnederlandswoordenboek2012}) offers one or two ambiguous examples in which one of the meanings could possibly be situational necessity, as a special type of dynamic modality). In OD and EMD it also featured a deontic and a directive meaning.\footnote{In the literature it is often suggested that the temporal meaning (future) emerged from the deontic and/or directive meaning in this verb (see e.g. \citealt{Duinhoven1997}: 428; a similar claim has been made regarding English \textit{shall}, cf. \citealt{BybeeBybee1987, BybeeBybee1991}). Yet authors do not provide proof for this assumption: they only cite individual instances that existed side by side in the oldest documented stages of the language, but they do not offer evidence that demonstrates a diachronic order between the meanings. It is perfectly imaginable that both meaning types developed in a parallel evolution out of the original main verbal meaning ‘to owe something (to someone)’. The development from the original meaning directly to the future meaning involves a straightforward metonymic path. If one owes someone something, this implies that one has to do something in the future to settle one’s debt. The change to the future meaning is a small step (it follows the same logic as that offered in the literature for the presumed change from a directive to a future meaning). Offering formal proof for one or the other scenario may be impossible, however, at least for Dutch, since these meanings emerged before the oldest documents.}  How important these were is hard to assess: the dictionaries offer numerous examples, but many or most of them can just as well be interpreted temporally. Moreover, these meanings have more or less disappeared since then (the directive meaning is still mentioned in dictionaries for the present day language, but the examples sound fairly archaic). The verb did develop an epistemic meaning (cf. \ref{ex:nuyts:11}), but the source is no doubt the temporal meaning (since the future is inherently uncertain, markers of the future generally show a strong tendency to acquire an epistemic meaning). This is unlike the epistemic (or evidential) meaning in the other modals, which emerged from other modal meanings (most probably from the dynamic modal ones, see \citealt{BylooNuyts2014}).

\textit{Willen} does not seem to have had any distinct meanings other than the present: from OD onwards it marks a wish or desire, i.e. volition. \REF{ex:nuyts:18} is an EMD illustration (from \citealt{PijnenburgEtAl2000}, entry \textit{willen} in the online edition).

\ea%18
    \label{ex:nuyts:18}
\gll           Aldaer si tesamen waren, so sprac Symon Petrus: Jc \textit{wille} gaen veschen.\\
  when they together were so said Simon Petrus I want go fish\\
\glt   `When they were together, Simon Petrus said: I want to go fishing.'
\z

\noindent The dictionaries do mention some other meanings, but these can more or less all be accounted for as contextual implicatures from the volitional meaning.\footnote{For example, \citet{PijnenburgEtAl2000} mention ‘to demand’ as a meaning in EMD (i.e. directivity), but from the examples it is obvious that this always concerns a use of the verb with a volitional meaning, yet in a context in which fulfilling the wish is inevitable for the addressee (all examples are from legal texts such as wills, ordinances, or decrees). A somewhat more doubtful case is the future meaning in EMD listed by \citet{PijnenburgEtAl2000} (which they call “infrequent”). Here, too, most dictionary examples can be read volitionally, but in just a few instances a temporal reading is more obvious than a volitional one. \REF{ex:nuyts:fn} is one of the very few illustrations (adapted from \citealt{PijnenburgEtAl2000}, entry \textit{willen} in the online edition): 

\ea 
    \label{ex:nuyts:fn}
\gll Alsoe wat si daer af segghen \textit{willen} ende ordineren, dat sal ic houden ende doen met ghoeden paise.\\
Thus what they there of say want and order that shall I maintain and do with good peace.\\
\glt `So what they want to/will say and order about it, that I will obey and do in good spirit.'
\z

\noindent So maybe there has been a minor tendency in this verb in Middle Dutch to develop a temporal meaning.} In any case, the verb never developed any of the prototypical modal and related meanings central in \textit{kunnen}, \textit{mogen}, \textit{moeten} and \textit{hoeven}, such as a dynamic, deontic or epistemic/inferential one. As indicated above, volition also occurs as a meaning in these other modals, but only as a minor one. Although it is arguably related to directivity (see \citealt{Nuyts2008}), this meaning is not central to the system of the modal and related meanings.

In sum, \textit{zullen} and \textit{willen} have a semantic profile and development very different from that of \textit{kunnen}, \textit{mogen}, \textit{moeten} and \textit{hoeven}. Hence, also semantically, \textit{zullen} and \textit{willen} are not part of the system constituted by the other four modal verbs. Maybe the reason why they did not join the latter is that their dominant or exclusive meaning (future tense marking and volition), present from OD/END onwards, is too remote from the classical range of modal and related meanings, and/or too “unnatural” as a source for developing these meanings. (Volition may emerge out of, but is an unlikely source for a dynamic, deontic or epistemic meaning. Time is an unlikely source for a dynamic or deontic meaning. See \citealt{BylooNuyts2014}.)

\section{Conclusion} \label{nuyts:6}

Dutch {\textit{kunnen}}{,} {\textit{mogen}}{,} {\textit{moeten}}{ and} {\textit{hoeven}}{ show clear signs of “group behavior”, both in the grammatical domain (cf. \sectref{nuyts:4}) and in the semantic domain (cf. \sectref{nuyts:5}), hence there is every reason to consider them a paradigm. This conclusion is further underscored by the fact that} {\textit{zullen}}{ and} {\textit{willen}}{, even though they are grammatically and semantically somewhat related (and therefore traditionally considered part of the system of the modals in Dutch), do not participate in the dynamics shared by} {\textit{kunnen}}{,} {\textit{mogen}}{,} {\textit{moeten}}{ and} {\textit{hoeven}}{. The latter four verbs constitute a significant cluster in the linguistic behavior of speakers of Dutch, hence we cannot but conclude that they play a distinctive role in the cognitive grammar coded in the brains of those speakers.}

As such, this set of four modal verbs also offers an illustration of the different issues raised in \sectref{nuyts:2} regarding what may count as a paradigm.

First of all, the set illustrates the gradualness of the notion as defined by the three diagnostic criteria mentioned in \sectref{nuyts:2}. The four verbs clearly satisfy the criterion of increasing convergence. Over the past 1000 years, they have gradually grown closer together, at an increasing speed, both semantically and grammatically, showing more and more signs of system-bound dynamics, such as analogy effects, or avoidance of synonymy or functional overlap within the system. This process has gained momentum in the last 200 years, causing the helter-skelter grammatical and semantic evolutions observed in \textit{hoeven}, as well as the rapidly evolving, and linguistically exceptional, re-autonomization process in all four verbs.

The fact that there are mechanisms of avoidance of synonymy and functional overlap at work also signals that there is at least to some extent a strive for a division of labor among the members of the set (cf. the second diagnostic criterion). As indicated in \sectref{nuyts:5}, the result in the present day language is that there are two strong verbs (\textit{moeten} and \textit{hoeven}) and two weak verbs (\textit{mogen} and \textit{kunnen}). The two strong verbs differ in terms of their relation with negation,\footnote{This is true at least in the standard language and in the Northern Dutch dialects. Interestingly, in the Southern Dutch dialects \textit{moeten} does not show a preference for positive contexts, but \textit{hoeven} barely exists in these language varieties hence there is not really a competitor for \textit{moeten} in them (see \citealt{DiepeveenDiepeveen2006}).} the two weak verbs in terms of which modal or related meanings are most prominent in their semantic profile (cf. \sectref{nuyts:5}). The internal organization in this paradigm is not fully systematic (the division of labor is along different criteria between the weak and between the strong modals) nor very strict (\textit{hoeven} is a pure negative polarity item but \textit{moeten} does not entirely exclude negation; and \textit{kunnen} and \textit{mogen} share some meanings). As such the organization is probably weaker than, for instance, in an inflectional paradigm such as a case marking system in which (in the “perfect” situation) all participating forms are more or less complementary and together cover the entire functional “territory” (i.e. coding all semantically and/or syntactically relevant roles of participants in a state of affairs). Nevertheless, the four Dutch modals are relatively complementary in expressing aspects of the shared range of modal and related meanings, witness the fact that there are relatively very few usage contexts in which two of the forms would be mutually exchangeable without altering the meaning or communicative effect of an utterance.

Also the third diagnostic criterion, distinctiveness in the overall linguistic system, is satisfied to some extent. Again, the set of four forms is less unique in the system of Dutch than, for instance, an inflectional paradigm such as the case marking system in Latin, for which there are no alternatives at all in that language. Structurally, there are many other auxiliary forms in Dutch beyond the modals. And semantically or functionally, there are many other (sets of) forms for expressing modal and related meanings in the language, including adverbs and adjectives, and full verbs (e.g. some mental state predicates). Yet none of the latter types of expressive devices covers the same range of meanings as the four modal auxiliaries (modal adverbs and adjectives, for instance, typically focus on one modal category, most often epistemic modality or inferentiality; the same applies for the mental state verbs). Moreover, the modals are the only set of forms covering this semantic domain within the range of grammatical devices in the language (assuming that adverbs are not grammatical forms, a view that is not generally shared).

The gradualness of the notion of a paradigm, and of the three diagnostic criteria for it, is also manifest if one considers the diachronic evolution of the set of four modals in Dutch: it has taken a long time for this system to emerge, and determining a cut-off point on the historical cline for calling it a paradigm is arbitrary.

Secondly, the set of four modals in Dutch also illustrates that not only inflectional systems deserve to be called paradigms. As suggested above, it is beyond doubt that inflectional systems (obligatory or non-obligatory ones) more readily qualify for paradigmhood than systems of forms from other parts of speech. (Not all inflectional systems are necessarily “perfect” paradigms, though. For instance, in many languages, systems of verbal affixes for marking categories such as tense, aspect, mood, modality, or evidentiality, even obligatory ones, are not the unique markers for these meanings, and/or are not better organized internally than non-inflectional systems.) Still, in our present case we are dealing with a system of non-inflectional grammatical markers, and not even very strongly grammaticalized ones (the Dutch modals are much less grammaticalized than their English counterparts, for instance).\footnote{For instance, unlike the English modals, the Dutch modals still have productive infinitive and past tense forms, and they are still inflected for person and number.}  Moreover, in view of the re-autonomization trend and the strong increase in main verbal uses of these forms in Dutch, the status of the system as a grammatical one would seem to be weakened.

Thirdly and finally, the latter point also illustrates the last issue regarding paradigmhood raised in \sectref{nuyts:2}: the link with grammaticalization. Although, as indicated, more grammaticalized systems no doubt stand a better chance to qualify for paradigmhood than less grammaticalized ones, the set of four Dutch modals shows that grammaticalization is not a necessary correlate of paradigmhood. The process of re-autonomization in the four forms, which substantially increases their mutual tie, hence strengthens their status as a paradigm, is even, arguably, an instance of degrammaticalization (see \citealt{Nuyts2013}, \citealt{CaersNuyts2021}), i.e. an evolution in the opposite direction.

\section*{Acknowledgments}

Research for this paper was supported by a project financed jointly by the Belgian and Dutch science agencies FWO-Vlaanderen and NWO (“lead agency” project G.0A18.15N).

{\sloppy\printbibliography[heading=subbibliography,notkeyword=this]}
\end{document}
