\documentclass[output=paper]{langsci/langscibook} 
\ChapterDOI{10.5281/zenodo.5675839}
\author{Gabriele Diewald\affiliation{Leibniz Universität Hannover} and {Katja Politt}\affiliation{{Leibniz Universität Hannover}}}
\title{Paradigms regained}
\abstract{The contributions in this volume are anchored in the notions of paradigm and the paradigmatic organisation of linguistic items. The papers united here are substantial elaborations and enhancements of concepts as well as case studies that were presented at a workshop “Paradigms regained” held at the 52nd SLE Annual Meeting (SLE 2019), which took place from 21st--24th August 2019 at Leipzig University.

Its background is a long-lasting project aiming at assessing the cognitive reality of (grammatical) paradigms throughout various linguistic domains, thereby testing this notion for its ability to allow for “graceful integration” \citep{Jackendoff2011}. A notion like this should be able to account for empirical findings and general cognitive mechanisms. In this volume, different domains of grammatical phenomena are investigated to illustrate what the concept of grammatical paradigms can and cannot – yet – explain. The theoretical and conceptual foundations of this project are grammaticalisation theory, implicational morphology, usage-based constructional approaches, cognitive semantics, as well as corpus-based and experimental approaches to grammatical structures in diachronic and synchronic phenomena.}

\begin{document}
\maketitle

\section*{Definitions and positions}

The notion of paradigm is primarily discussed in morphological theories, where it plays a central role as a tool for describing the structures in which inflectional forms are organised. The members of inflectional paradigms are primarily identified by their formal properties (cf. \citealt[7]{Fabri1998}). Each member of a paradigm corresponds to a cell, which can be either filled by a form or by a form-feature pair \citep{Lieb2005,Werner1994,Wurzel1984}. Lately, work on relational structures in morphological paradigms \citep{AckermanEtAl2009,Blevins2015,Blevins2016} has shown that this purely instrumental conception of paradigms as nothing but a useful descriptive convention clearly underestimates its cognitive foundation and functional importance. Paradigms in this sense are structures which provide “cohesive wholes” \citep[94]{Blevins2015} for the paradigm members. These structures consist of relations and associations between the individual cells within and in between paradigms. As such, they are part of speaker knowledge, because they provide necessary generalisations that allow speakers to infer previously unencountered forms of lexical items (\citealt[54]{AckermanEtAl2009}). Knowing the overall organisational structure of the forms allows for inferring forms and their functions from one another, i.e., putting them into relation to one another. 

This inferential nature of paradigms is what can be generalised as a structuring principle to other areas of grammar (\citealt[xi]{Nørgård-Sørensen2011}). For this it is necessary to expand the notion of paradigms from a purely inflectional notion to a broader sense: It is assumed that grammatical items are structured in \textit{grammatical paradigms}. Grammatical paradigms in this sense are functional sets (\citealt{Andersen2008, Diewald2020a, Nørgård-Sørensen2011, Politt2021}).  They are holistic semiotic structures, consisting of ordered bundles of oppositions between all marked and unmarked members of the category in question (which in grammaticalisation are modified in various ways). Take the grammatical category \textsc{tense} as an example: The members of the \textsc{tense} paradigm share a common categorical function, namely situating events relative to the speech time. The unmarked zero in \textsc{tense} is the present. The specific function of all other members of the category – like past and future forms – can be described relative to that unmarked zero, i.e., in opposition to it and of course also in opposition to one another. Those oppositions serve a similar function as the aforementioned relations within inflectional paradigms; they (i) relate the members of a grammatical paradigm to each other and (ii) define the specific categorical function of each member relative to the categorical function of the other members In short: The oppositions and relations between the members of a grammatical paradigm are the very essence of grammatical structures (cf. \citealt{Politt2021}). They “cannot be described without reference to the paradigmatic organisation that lies behind the syntagmatic realisations” (\citealt[71]{Nørgård-Sørensen2011}).\largerpage

It is because of this internal relational structure that grammatical paradigms are an invaluable tool for describing the target structures of grammaticalisation processes and assessing the status of a grammaticalising element. These elements acquire a place in such a structure or change their place within it. By entering grammatical paradigms, elements form new oppositional pairs with other grammatical elements that are members of the same superordinate category, e.g. \textsc{tense} (\citealt[4]{DiewaldSmirnova2010}). By developing this opposition, the newly grammaticalised item becomes a member of a grammatical paradigm \citep{BybeeEtAl1994,Lehmann2015}. 

While it is undisputed that “grammar” is the target domain of grammaticalisation processes, and that “paradigms” play a role in the development of Indo-European languages, the exact extent of the notion of paradigm and grammatical paradigm and its usefulness for languages with little or no inflectional morphology has been under dispute for some time now. For once, there is fundamental criticism concerning the lack of an exact definition of “grammar”, as put forth by Himmelmann: “[w]ork in grammaticalisation […] hardly ever makes explicit the concept of grammar underlying a given investigation” \citep[2]{Himmelmann1992}. Furthermore, there is a lively discussion about (i) whether the notion of paradigm should be extended to syntagmatic linguistic structures beyond bound morphology and periphrastic forms, as for example in Construction Morphology (\citealt{Booij2010,Booij2016, Booij2018}), and include, for example, grammatical oppositions on the level of the whole clause, like the opposition between sentences particles and modal particles. In constructional approaches, paradigms are often “marginalized or even lost” \citep[277]{Diewald2020a}. Another hotly disputed issue is (ii) what the benefit of such an extension might be (\citealt{Bisang2014,Diewald2020a,Haspelmath2000,WiemerBisang2004}).

This discussion, which arose in typological research and grammaticalisation studies, meets with current questions and challenges in construction grammar. The latter aims at describing grammatical structures in their entirety. If grammatical paradigms are indeed structures of the internal organisation of grammatical categories, it must be possible to describe them in constructional terms as well. The goal is therefore to find an integrative approach that combines both construction grammar and paradigms as organisational structures of grammar (\citealt{Diewald2009, Diewald2015, Diewald2020a, DiewaldSmirnova2010, Politt2021}). In such an approach, paradigms are not only the aforementioned generalisations of associative structures but they can be seen as constructions “whose function and meaning is defined by the specific number and constellation of [their] components”, which “mutually define each other’s values” through their inherent indexical structure \citep[303]{Diewald2020a}.

The basic assumptions derived from this background are:\largerpage

\begin{itemize}
\item paradigms are necessary generalisations of grammatical structures,
\item paradigms are part of the grammatical knowledge of speakers, and
\item paradigms are what makes grammaticalisation processes structured processes.
\end{itemize}

These basic assumptions are to be tested and refined based on the case studies and theoretical reflections offered by the contributions of this volume. 

\section*{The papers}

The contributions in this volume explore and test these assumptions, raising questions like the following ones:

\begin{itemize}
\item Can research from different linguistic subdisciplines underpin the importance of the notion of paradigms?
\item What are the advantages (and limitations) of such an integrative approach of describing grammatical structures as paradigmatic, i.e., as consisting of oppositions and relations?
\item Is there independent evidence from neighboring disciplines supporting the assumption of paradigms as cognitive entities?
\end{itemize}

\begin{sloppypar}The contributions range from diachronic and synchronic case studies to broadly scaled surveys of different types of paradigmatic organisation to theoretical reflections of relevant notions within this field of research. This allows for an arrangement of the contributions to this volume into three sections: The first section, containing two papers, deals with general terminological, definitional, and theoretical issues. A broad survey on large-scale diachronic mechanisms and drifts building up inflectional paradigms of various types (\textsc{Andersen}) is followed by a theoretical reflection on the status of paradigms as metaconstructions in the construction grammatical approach (\textsc{Leino}). The second section consists of two papers paying close attention to the details of particular mechanisms and (diachronic) processes steering the organisation of morphological paradigms and more extended constructions, with one of them investigating the interplay of inflection and derivation in Slavic languages (\textsc{Wiemer}), and the other one dealing with the morphological process of recursion in relation to composition, mainly drawing on examples from Turkish (\textsc{Reiner}). The third section consists of six contributions offering detailed language specific case studies, taking up linguistic phenomena of Danish (\textsc{Hansen}, \textsc{Heltoft}), Dutch (\textsc{Nuyts,} \textsc{Caers} \textsc{\&} \textsc{Goelen}), German (\textsc{Hartmann} \textsc{\&} \textsc{Neels}), Norwegian (\textsc{Kurek-Przybilski}) and French (\textsc{Kragh}), most of them addressing or focusing on diachronic issues. The following are brief outlines of the contributions in the order in which they appear.\end{sloppypar}

In “Paradigms of paradigms” \textsc{Henning} \textsc{Andersen} provides a broad overview of the types of organisational structures of paradigms in inflectional paradigms (selectional sets) arising in the course of diachronic processes. The article provides a classification of formal and functional complexities of paradigms of verbal and nominal categories (\textsc{case,} \textsc{number,} \textsc{person,} \textsc{tense,} \textsc{aspect,} \textsc{mood,} \textsc{voice}) that are due to hierarchical nesting or embedding of their morphological exponents. Attention is given to the interaction of different techniques within one grammatical paradigm in a particular language, e.g., the “typological gradation” in the paradigms of verbal categories in English, French, and Latin conjugations, which includes the phenomena of auxiliarisation, agglutination, irregular forms, fusion, ablaut and suppletion. Contending that morphological systems are typologically diverse (due to diachronic processes), it is argued that morphological theory – also in synchronic analysis – must take care of the fundamental feature of typological diversity in its theoretical and methodological layout from the start.

In “Formalizing paradigms in Construction Grammar” \textsc{Jaakko} \textsc{Leino} discusses the question of how constructions in a language are organised. He draws on both Construction and Cognitive Grammar to explore similarities and differences – relations and oppositions – between constructions and introduces the notion of \textit{metaconstructions} (\citealt{Leino2003, LeinoÖstman2005}) as a generalisation of constructions on a more schematic level. \textsc{leino~}contrasts the two notions of metaconstructions and grammatical paradigms with each other and explores how metaconstructions can serve as a means to describe the internal organisation of grammar and as a base for the formation and integration of new constructions in(to) the system.

The comprehensive contribution by \textsc{Björn} \textsc{Wiemer} “No paradigms without classification: How stem-derivation develops into grammatical aspect” develops detailed suggestions on the subclassification of complex paradigms of verbal aspect in Slavic languages. Based on data from several Slavic languages, it proposes a layered conception of the notion of paradigm. The first layer rests on the principal binary distinction between perfective and imperfective aspect, which is realised by the inherent aspectual features of the verb stems and their associated derivational pattern. The second layer is constituted by subparadigms, which are triggered by specific, mutually exclusive bundles of regular usage conditions and contextual features. Drawing on notions from Construction Grammar approaches and Word-and-Paradigm models, it is suggested that these bundles of features can be conceived as constructional templates for individual aspectual values within subparadigms, which operate on an underlying binary system of aspectual distinctions based on verb stems. Thus, the paper puts forward far-reaching suggestions for analysis of grammatical distinctions that integrate lexical, constructional, and contextual features.

\begin{sloppypar}
In “Recursion and paradigms” \textsc{Tabea} \textsc{Reiner} discusses morphological paradigms from both a constructional and compositional perspective. By comparing the status of inflectional paradigms in Constructional, Distributed, and Autonomous Morphology, she raises the question what they can contribute to a model of inflectional recursion. Namely, paradigms could serve as a means of modelling restrictions on recursion patterns in inflectional morphology and therefore constitute a fundamental unit of morphological description. 
\end{sloppypar}

In “Redundant indexicality and paradigmatic reorganisations in the Middle Danish case system” \textsc{Bjarne} \textsc{Simmelkjær} \textsc{Sandgaard} \textsc{Hansen} investigates the fundamental changes of the Danish case system during the Middle Danish period as an instance of grammaticalisation. The central issue is the increasing topological fixation inside the noun phrase and its interaction with phrase internal agreement and case marking. The newly established system of noun phrase marking is shown to be an instance of grammatical change proper (instead of e.g. phonologically induced change) and provides a prime example of the claim that grammaticalisation is inextricably linked with paradigmaticisation.

The contribution by \textsc{Lars Heltoft} “The semantic reorganisation of case paradigms and word order paradigms in the history of Danish” investigates the interaction of word order change, namely the topological fixation of the subject position and serialisation rules in the middle field as well as change in the case system in the history of Danish. Assuming that inflectional and (pluri-item) constructional representations of grammatical information alike are organised in paradigms, this investigation highlights the shifts in the “co-organisation” of the expression of grammatical content. It is suggested that this type of complex collaboration between morphological and topological marking techniques can be called second-order paradigms or hyperparadigms.

In raising the question “The Dutch modals, a paradigm?” \textsc{Jan Nuyts}, \textsc{Wim Caers} and \textsc{Henri-Joseph Goelen} depart from morphology-based definitions of “paradigm” and adopt a cognitive perspective, whereby a paradigm is defined as a “cognitively real phenomenon”. Relevant criteria for a paradigm, more precisely the gradual rise of paradigms, are seen in the gradual accumulation of shared grammatical and semantic features, and an increasingly pronounced “divisions of labor”, i.e. a stricter internal functional organisation, among the entities involved. In presenting a “meta study” of several diachronic investigations of the development of the Dutch modal verbs \textit{kunnen} ‘can’, \textit{mogen} ‘may’, \textit{moeten} ‘must’, and \textit{hoeven} ‘need’, the broad lines of change and convergence in structural and semantic features of this group are taken as an instance of this type of paradigmaticisation. Furthermore, the authors raise the question whether grammaticalisation should always be seen as a necessary correlate of paradigmhood, thus, offering arguments for further discussion on the theoretical issue concerning the distinction between paradigmatic relations on the one hand and narrowly defined paradigms on the other hand. 

\textsc{Stefan} \textsc{Hartmann} and \textsc{Jakob} \textsc{Neels} analyze the grammaticalisation of a family of constructions in “Grammaticalisation, schematisation and paradigmaticisation: How they intersect in the development of German degree modifiers”. Drawing on both synchronic and diachronic corpus data, they explore the gradual context expansion of German degree modifier-constructions such as [\textit{ein wenig}~X] (‘a little’), [\textit{ein bissche}n X] (‘a bit’), [\textit{ein Quäntchen} X] (lit. ‘a quantum’), [\textit{ein Tick}~X] (lit. ‘a tick’) and [eine Idee X] (lit. ‘an idea’). They aim to show that paradigmaticisation leads to multiple interrelated paradigms with varying levels of schematicity, similar to the differences of higher and lower level constructions discussed by \citet{Traugott2007}. 

\begin{sloppypar}
In “Generics as a paradigm: A corpus-based study of Norwegian” \textsc{Anna} \textsc{Kurek-Przybilski} investigates how the notion of grammatical paradigms can help in modelling language specific grammatical categories. Looking at encyclopedic texts from Nynorsk, she develops a genericity paradigm that can serve as a baseline for investigating co-existing varieties of a language and helps in understanding the grammaticalisation process of generic contexts and expressions. 
\end{sloppypar}

\textsc{Kirsten} \textsc{Jeppesen} \textsc{Kragh} draws on French diachronic data to illustrate “The importance of paradigmatic analyses: From one lexical input into multiple grammatical paradigms”. By following the grammaticalisation path of the French verb \textit{voir} ‘to see’ and its polygrammaticalisation into multiple grammatical categories, Kragh shows that the target structures of grammaticalisation do not necessarily have to belong to the same grammatical areas. For \textit{voir}, she illustrates grammaticalisation paths into \textsc{tense,} \textsc{aspect,} \textsc{mood,} \textsc{voice,} as well as discourse markers and prepositions. Synchronic paradigms serve as the target structure of these grammaticalisation paths and allow for precise descriptions of the newly acquired grammatical functions of the grammaticalised elements due to their inherent relational structure. 

\begin{verbatim}
    
\end{verbatim}
{\sloppy\printbibliography[heading=subbibliography]}
\end{document}
