\documentclass[output=paper,colorlinks,citecolor=brown]{langscibook}
\author{Luis Eguren\affiliation{Universidad Autónoma de Madrid} and Alberto Pastor\affiliation{Southern Methodist University}}
\title{(Non)-pied-piping adjectival wh-comparatives in Spanish}

\abstract{This paper analyses pied-piping and non-pied-piping structures in adjectival wh-comparatives in Spanish. Under the assumption that optional pied-piping is not possible (\citeauthor{heck2008pied} \citeyear{heck2008pied,heck2009certain}), it is proposed that the non-pied-piping structure (e.g., \textit{¿Cuánto es esta mesa más larga?} `lit. How much is this table more long') results from an underlying syntactic configuration in which the differential wh-item is merged as the specifier of a functional phrase in the extended projection of the adjective, whereas the pied-piping structure (e.g., \textit{¿Cuánto más larga es esta mesa?} `lit How much more long is this table?') obtains by means of reprojection \citep{hornstein2002reprojections}, whereby the wh-form heads the comparative construction. This analysis is further extended to Spanish (non)-pied-piping wh-comparatives with nouns.}

\IfFileExists{../localcommands.tex}{%hack to check whether this is being compiled as part of a collection or standalone
   \usepackage{langsci-optional}
\usepackage{langsci-gb4e}
\usepackage{langsci-lgr}

\usepackage{listings}
\lstset{basicstyle=\ttfamily,tabsize=2,breaklines=true}

%added by author
% \usepackage{tipa}
\usepackage{multirow}
\graphicspath{{figures/}}
\usepackage{langsci-branding}

   
\newcommand{\sent}{\enumsentence}
\newcommand{\sents}{\eenumsentence}
\let\citeasnoun\citet

\renewcommand{\lsCoverTitleFont}[1]{\sffamily\addfontfeatures{Scale=MatchUppercase}\fontsize{44pt}{16mm}\selectfont #1}
  
   %% hyphenation points for line breaks
%% Normally, automatic hyphenation in LaTeX is very good
%% If a word is mis-hyphenated, add it to this file
%%
%% add information to TeX file before \begin{document} with:
%% %% hyphenation points for line breaks
%% Normally, automatic hyphenation in LaTeX is very good
%% If a word is mis-hyphenated, add it to this file
%%
%% add information to TeX file before \begin{document} with:
%% %% hyphenation points for line breaks
%% Normally, automatic hyphenation in LaTeX is very good
%% If a word is mis-hyphenated, add it to this file
%%
%% add information to TeX file before \begin{document} with:
%% \include{localhyphenation}
\hyphenation{
affri-ca-te
affri-ca-tes
an-no-tated
com-ple-ments
com-po-si-tio-na-li-ty
non-com-po-si-tio-na-li-ty
Gon-zá-lez
out-side
Ri-chárd
se-man-tics
STREU-SLE
Tie-de-mann
}
\hyphenation{
affri-ca-te
affri-ca-tes
an-no-tated
com-ple-ments
com-po-si-tio-na-li-ty
non-com-po-si-tio-na-li-ty
Gon-zá-lez
out-side
Ri-chárd
se-man-tics
STREU-SLE
Tie-de-mann
}
\hyphenation{
affri-ca-te
affri-ca-tes
an-no-tated
com-ple-ments
com-po-si-tio-na-li-ty
non-com-po-si-tio-na-li-ty
Gon-zá-lez
out-side
Ri-chárd
se-man-tics
STREU-SLE
Tie-de-mann
}
    \bibliography{localbibliography}
    \togglepaper[23]
}{}
\begin{document}
\maketitle

\section{Introduction}
\label{sec:eguren:1}

As shown in the examples in \ref{ex:eguren:1}, both non-pied-piping and pied-piping structures are possible in adjectival wh-comparatives in Spanish: in \ref{ex:eguren:1a}, the differential \textit{cuánto} `how much' moves alone to the left-periphery of the sentence; in \ref{ex:eguren:1b}, it pied-pipes \textit{más}+A.

\begin{exe} % declares the example environment
    \ex\label{ex:eguren:1} % sets the number, in this case \ref{ex:eguren:1}
    \begin{xlist} % declares a list of sub-examples
        \ex \label{ex:eguren:1a}
            \gll   ¿Cuánto	es	Ana	más	alta?\\   % target language line
                   {how much}	is	Ann	more	tall\\    % interlinear gloss
        \ex\label{ex:eguren:1b}
            \gll   ¿Cuánto	más	alta	es	Ana?\\
                   {how much}	more	tall	is	Ann\\
\end{xlist}
\end{exe}

Assuming that optional pied-piping does not exist \citep{heck2008pied,heck2009certain}, in this paper we will argue that wh-clauses like the ones in \ref{ex:eguren:1} have their origin in two different structures. We will propose, in particular, that the non-pied-piping structure in \ref{ex:eguren:1a} underlies a syntactic configuration in which the differential wh-word is generated in the specifier position of a Degree Phrase in the extended projection of the adjective, whereas the pied-piping structure in \ref{ex:eguren:1b} results from reprojection \citep{hornstein2002reprojections}, which turns the wh-item in [Spec, DegP] into the head of the comparative construction.

The content of the paper is organized as follows. \sectref{sec:eguren:2} describes the facts under discussion. \sectref{sec:eguren:3} sets out our assumptions on optional pied-piping and on the structure of adjectival comparatives. In \sectref{sec:eguren:4}, we present our analysis of (non)-pied-piping structures in Spanish adjectival comparatives in detail. In \sectref{sec:eguren:5}, our proposal is extended to wh-comparatives in the nominal domain. \sectref{sec:eguren:6} finally contains the conclusions of this research.


\section{The facts}
\label{sec:eguren:2}

As illustrated in \ref{ex:eguren:2}, the differential in adjectival wh-comparatives can take three different lexical forms in current Spanish: \textit{cuánto} `how much', \textit{qué tanto} `what so much' and \textit{cómo de} `how of'.\footnote{\
Following \cite{moron2004frase}, we take \textit{de} `of' in \textit{cómo de} `how of' to be a partitive Case marker.}

\begin{exe} % declares the example environment
    \ex\label{ex:eguren:2} % sets the number, in this case \ref{ex:eguren:1}
    \begin{xlist} % declares a list of sub-examples
        \ex % sets the sub-example letter, in this case (a)
            \gll ¿Cuánto	es	Ana	más	alta	que	María?\\ 
            {how much}		is	Ann	more	tall	than	Mary\\    % interlinear gloss
                 \ex
            \gll  {¿}{Qué}	{tanto} es Ana más alta que	María?\\
            what 	{so much}	is	Ann	more tall than	Mary\\
            \ex\label{ex:eguren:2c}
            \gll¿Cómo	es	Ana	de	más	alta	que	María?\\
            how	is	Ann	of	more	tall	than	Mary\\
\end{xlist}
\end{exe}

\newpage

The use of the wh-expressions in \ref{ex:eguren:2} is subject to dialectal and/or idiolectal variation: \textit{qué tanto} is a form of Latin American Spanish, and, at least in Peninsular Spanish, speakers vary in using \textit{cuánto} or \textit{cómo de} as their preferred (or only) option.\footnote{On the origin, history and current distribution of the Spanish adjectival wh-forms \textit{cuán(to)}, \textit{qué tan(to)} and \textit{cómo de}, see \citet*{de2006cuantificadores}. A reviewer, who claims to be a speaker of Peninsular Spanish, takes the sentences with \textit{cómo} in \ref{ex:eguren:2c} and \ref{ex:eguren:3c}, and also the examples with \textit{así} in \ref{ex:eguren:7}, to be ungrammatical. For us, as speakers of the same variety of Spanish, all these sentences are well-formed. As pointed out in the text, there is thus ideolectal variation in the use of adjectival wh-comparatives with \textit{cómo} (and of the focus movement comparative construction with \textit{así} as well). All the data with \textit{qué tanto} in this paper have been checked with speakers of Mexican and Venezuelan Spanish.}

All these wh-forms can move to the CP-domain on their own, giving rise to a non-pied-piping (discontinuous) structure, as in \ref{ex:eguren:2}, and they can also pied-pipe the rest of the adjectival comparative construction, as in the examples in \ref{ex:eguren:3}:\footnote{The patterns in \ref{ex:eguren:2} and \ref{ex:eguren:3} are replicated in wh-exclamatives with adjectival comparatives. In non-comparative adjectival degree wh-clauses in Spanish, all wh-words pied-piped the adjective \ref{ex:eguren:fn1a}, but only \textit{cómo} can move alone \ref{ex:eguren:fn1b}--\ref{ex:eguren:fn1c}:
 \begin{exe}
      \ex\label{ex:eguren:fn1}
       \begin{xlist} % declares a list of sub-examples
        \ex \label{ex:eguren:fn1a}
            \gll   {¿}{Cuán/Qué tan/Cómo} de	alta	es	Ana?\\   % target language line
            {how much/what so/how} of tall is	Ann\\
            \ex\label{ex:eguren:fn1b}
            \gll  * {¿Cuán/Qué} tan es	Ana	alta?\\
            {} {how much/what} so is	Ann	tall\\
            \ex\label{ex:eguren:fn1c}
            \gll ¿Cómo	es	Ana	de	alta?\\
            how	is	Ann	of	tall\\
\end{xlist}
 \end{exe}
These constructions are analysed in \citet{eguren2020pied}.}

\begin{exe} 
    \ex\label{ex:eguren:3} 
    \begin{xlist}
            \ex
            \gll   ¿Cuánto	más	alta	que	María	es	Ana?\\   % target language line
{how much}	more	tall	than	Mary	is	Ann\\    % interlinear gloss
            \ex
            \gll ¿Qué	tanto	más	alta	que	María	es	Ana?\\
                what	{so much}	more tall	than	Mary		is	Ann\\
            \ex\label{ex:eguren:3c}
            \gll ¿Cómo	de	más	alta	que	María	es	Ana?\\
              how	of	more	tall	than	Mary		is	Ann\\
    \end{xlist}
\end{exe}


This is not an exclusive feature of Spanish. Other Romance languages, like Italian \ref{ex:eguren:4} or Catalan \ref{ex:eguren:5}, also allow for non-pied-piping and pied-piping structures in adjectival wh-comparatives:\footnote{We thank Lorenzo Bartoli and Françesc Roca for their help with the Italian and Catalan data. For one of the reviewers, the sentence in \ref{ex:eguren:4a} is odd, which may indicate that the acceptance of non-pied-piping adjectival comparatives is subject to dialectal/ideolectal variation in Italian. We assume that, in sentences like the ones in \ref{ex:eguren:4} and \ref{ex:eguren:5}, the comparative complement (\textit{di Maria/que la Maria}) is extraposed.}


\begin{exe} 
    \ex\label{ex:eguren:4} 
    \begin{xlist}
            \ex\label{ex:eguren:4a}
\gll 	Quanto 		è	piú	alta	Ana	di		Maria?\\
        {how much}	is	more	tall	Ann	than	Mary\\
            \ex\label{ex:eguren:4b}
\gll 	Quanto		piú	alta	è	Ana	di		Maria?\\
        {how much}	more	tall	is	Ann	than	Mary\\
    \end{xlist}
\end{exe}

\begin{exe} 
    \ex\label{ex:eguren:5} 
    \begin{xlist}
            \ex\label{ex:eguren:5a}
\gll   Quant és	més	alta	l’Anna		que	la Maria?\\
       {how much}	is	more	tall	the-Ann		than	the Mary\\
            \ex\label{ex:eguren:5b}
\gll Quant		més	alta	és	l’Anna	que	la Maria?\\
     {how much}	more	tall	is	the-Ann	than	the Mary\\
    \end{xlist}
\end{exe}

Moreover, as illustrated in \ref{ex:eguren:6} and \ref{ex:eguren:7}, the paradigm in \ref{ex:eguren:2}--\ref{ex:eguren:3} also obtains in cases of focus movement of the differential modifier in Spanish adjectival comparatives:

\begin{exe} 
    \ex\label{ex:eguren:6} 
    \begin{xlist}
            \ex
\gll UN METRO es	esta	mesa	más	larga.\\
      	one	meter			is	this	table	more	long\\
            \ex
\gll UN	METRO	más	Larga	es	esta	mesa.\\
      one	meter		more	long		is	this	table\\
    \end{xlist}
\end{exe}

\begin{exe} 
    \ex\label{ex:eguren:7} 
    \begin{xlist}
            \ex
\gll ASÍ	es	esta	mesa	de más	larga.\\
     so		is	this	table	of	more	long\\
            \ex
\gll ASÍ 	DE	más	Larga	es	esta	mesa.\\
    so		of		more	long		is	this	table\\
    \end{xlist}
\end{exe}

In this paper, we will center on (non)-pied-piping structures in wh-clauses with adjectival differential comparatives in Spanish \ref{ex:eguren:2}--\ref{ex:eguren:3}, but our account also applies to similar patterns in other languages \ref{ex:eguren:4}--\ref{ex:eguren:5}, and to the Spanish constructions with focus movement in \ref{ex:eguren:6} and \ref{ex:eguren:7}.

\section{Background and assumptions}
\label{sec:eguren:3}

Our analysis of (non)-pied-piping structures in adjectival wh-comparatives in Spanish will be based on two main assumptions. We will first assume, following \citet{heck2008pied,heck2009certain}, that pied-piping cannot be optional. \citeauthor{heck2008pied} (\citeyear{heck2008pied,heck2009certain}) claims that pied-piping is a last resort operation, so that ``pied-piping is only possible if forced'' \citep[257]{heck2008pied}. This insight is precisely phrased in what he calls the Repair Generalization in \ref{ex:eguren:8} \citep[117]{heck2008pied} \citep[92]{heck2009certain}:\footnote{The idea that pied-piping cannot be optional has been independently put forward in \cite{ocon2008movimiento}, on the basis that optional pied-piping violates a minimalist economy condition whereby movement involves as less material as possible for convergence \citep[262]{chomsky1995b}}

\begin{exe} 
    \ex\label{ex:eguren:8} 
   \textit{The Repair Generalization}\\
   Pied-piping of $\beta$ by $\alpha$ is possible only if movement of $\alpha$ from $\beta$ is blocked.\\
\end{exe}

Under the view that pied-piping is a last resort operation, in Heck's words,``one would not expect pied-piping and stranding to coexist: stranding would always block pied-piping. Only if stranding is not available can pied-piping become an option'' \citep[95]{heck2009certain}. Cases of optional pied-piping would thus be always apparent, and may have a different source in each case. To illustrate how alleged instances of optional pied-piping can be tackled under this approach, we will briefly review Heck’s account of one well-known case of apparent optionality: possessor extraction in Slavic languages.

As shown in \ref{ex:eguren:9}, in Slavic languages like Russian, the wh-possessor can either move alone or pied-pipe the whole nominal expression:

\begin{exe} 
    \ex\label{ex:eguren:9} 
    \begin{xlist}
            \ex
\gll Ja	sprosil	[čju$_i$ ty	čital {[ t$_i$} knigu]].\\
     I	asked	 whose	you	read {} book\\
            \ex
\gll Ja	sprosil	{[[ čju	knigu]$_i$}	ty	čital  t$_i$].\\
    I	asked  {whose	book} you	read {}\\
    \glt `I asked whose book you read.'	
    \end{xlist}
\end{exe}

Building on the insights in \citet{bovskovic2004careful} on a similar pattern in Serbo-Croatian,  Heck (\citeyear[294]{heck2008pied}; \citeyear[fn. 50]{heck2009certain}) argues that a Russian nominal expression can be either a DP or an NP: if it is an NP, the possessor can be subextracted \ref{ex:eguren:10a}; if it is a DP, the possessor cannot move alone, and piped-pipes the noun \ref{ex:eguren:10b}, the reason being that, in this case, subextraction violates the Left Branch Condition in \ref{ex:eguren:11} \citep[121]{heck2008pied}. 

\begin{exe} 
    \ex\label{ex:eguren:10} 
    \begin{xlist}
            \ex\label{ex:eguren:10a}
\gll Ja	sprosil	{[čju$_i$} ty	čital{ [$_{NP}$ t$_i$} [$_{NP}$ knigu]]].\\
     	I	asked whose 	you	read	{}			book\\
            \ex\label{ex:eguren:10b}
\gll Ja	sprosil [{[$_{DP}$} {čju} {[$_{NP}$} {knigu}]]$_i$	ty	čital  t$_i$].\\
   I	asked {} whose {} book	you	read\\
\end{xlist}
\end{exe}

\begin{exe} 
    \ex\label{ex:eguren:11} 
   \textit{Left Branch Condition}\\
   If $\alpha$ is the leftmost category within DP, then $\alpha$ can’t undergo movement from DP\\
\end{exe}

Together with the assumption that pied-piping cannot be optional, in our proposal, we will also adopt the structural analysis of adjectival comparatives in \cite{svenonius2006northern} represented in \ref{ex:eguren:12}:

% figure 1
\begin{exe}
\ex\label{ex:eguren:12}
\begin{forest}  
[DegP
  [\textit{8 months}]
  [Deg'
    [Deg\\\textit{Meas}, align=center, base=bottom]
    [QP
      [Q'
        [Q\\\textit{er}, align=center, base=bottom]
        [AP\\\textit{old}, align=center, base=bottom]]
      [PP\\\textit{than Sam}, align=center, base=bottom]]
  ]
]
\end{forest}
\end{exe}

\cite{svenonius2006northern} follow \cite{kennedy2005scale} in claiming that comparative items are not true degree morphemes, but expressions that map adjective meanings to new adjective meanings: in the structure in \ref{ex:eguren:12}, the comparative morphology is thus not inserted under Deg$^0$, but heads a functional projection between AP and DegP instead. Within the framework of \cite{kennedy1999projecting}'s compositional semantic analysis of gradable adjectives, Svenonius and Kennedy further argue that the comparative+adjective constituent, which denotes a function from individuals to degrees, must combine with a Deg head to derive a property of individuals. As shown in \ref{ex:eguren:12}, they therefore propose that differential comparatives with measure phrases, in particular, involve the combination with a Deg head they dub ``Meas''. 

In our view, Svenonius and Kennedy's analysis of adjectival comparatives with measure phrases can be extended to all cases of adjectival differential comparatives, including those in which the differential is a wh-expression, just by positing that the null Deg head in \ref{ex:eguren:12} is always endowed with a [+differential] feature, as in \ref{ex:eguren:13}:

% figure 2
\begin{exe}
\ex\label{ex:eguren:13}
\begin{forest}  
[DegP
  [QP\\\textit{cuánto/qué tanto/cómo}, align=center, base=bottom]
  [Deg'
    [Deg\textsubscript{\textit{[+dif]}}\\Ø, align=center, base=bottom]
    [QP
      [Q'
        [Q\\\textit{(de) más}, align=center, base=bottom]
        [AP\\\textit{alta}, align=center, base=bottom]]
      [XP\\\textit{que Maria}, align=center, base=bottom]]
  ]
]
\end{forest}
\end{exe}

We will next present our proposal, assuming that wh-forms in adjectival differential comparatives are merged in the specifier position of DegP in the structural representation in \ref{ex:eguren:13}.

\section{The proposal}
\label{sec:eguren:4}

With the ideas above in mind, we propose that the non-pied-piping structures in Spanish wh-clauses with adjectival comparatives in \ref{ex:eguren:2}, reproduced in \ref{ex:eguren:14}, directly obtain from the structure in \ref{ex:eguren:13}: in this structure, the differential wh-expression is merged as a maximal projection in [Spec, DegP], and, as depicted in \ref{ex:eguren:15}, it raises alone to the CP-domain,\footnote{Assuming \cite{rizzi1997fine}'s theory of the left periphery of the sentence, we follow in particular the view in \citet[710--711]{bosque2009fundamentos} that wh-words first move to [Spec, FocP] to have its focus feature checked, and then raise to Force Phrase to check its wh-feature: [{$_{ForceP}$} WhP$_i$ [{$_{FocP}$} t$_i$ [\dots t$_i$ ]]].} stranding all the other elements in the comparative construction.

\begin{exe} 
    \ex\label{ex:eguren:14} 
    \begin{xlist}
            \ex
\gll {¿Cuánto/Qué tanto} es	Ana	más	alta	que	María?\\
      {how much/what so	much}	is	Ann	more tall than	Mary\\
            \ex
\gll¿Cómo	es	Ana	de	más	alta	que	María?\\
   how	is	Ann	of	more	tall	than	Mary\\
\end{xlist}
\end{exe}

% figure 3
\begin{exe}
\ex\label{ex:eguren:15}
\begin{forest}  
[DegP
  [QP,name=qp]
  [Deg' [Deg] [QP]]]
\node (a) at (-1, 0.3) {};
\draw[-Latex] (qp.south) -- +(0,-0.3)-| node[below right]{} (a);
\end{forest}
\end{exe}

Under the assumption that pied-piping is a last resort operation that is only possible if forced, the structure in \ref{ex:eguren:13} cannot be, however, the source of the pied-piping adjectival wh-comparatives in \ref{ex:eguren:3}, repeated in \ref{ex:eguren:16}: in \ref{ex:eguren:13}, subextraction of the wh-item can take place, and pied-piping, being a more costly operation, is blocked.

\begin{exe} 
    \ex\label{ex:eguren:16} 

\gll {¿Cuánto/Qué tanto/Cómo} de más alta que	María	es	Ana?\\
{how much/what so	much/how} of	more tall than Mary is Ann\\
\end{exe}

The construction in \ref{ex:eguren:16} must thus have its origin in a different structure in which pied-piping is the only option. We would like to argue that this structure is the result of the label-change mechanism \citet{hornstein2002reprojections} call ``reprojection''.

\citet{hornstein2002reprojections} claim that a phrase marker's label can change in the course of the derivation, which captures the fact that, in some cases, given two merged categories X and Y, ``at some stage it is sound to consider their mother a YP, whereas at some other stage, instead, their mother is seen by the system as an XP'' \citep[107]{hornstein2002reprojections}. In their paper, Hornstein and Uriagereka make use of reprojection, in particular, to allow the selectional requirements of binary quantifiers like most to be satisfied in (standard) configurational terms: once reprojection has applied \ref{ex:eguren:17b}, both its first argument (the restriction = people) and its second argument (the scope = love children) are syntactic dependents of the quantifier.\footnote{\citet[fn. 3]{hornstein2002reprojections} point out that reprojection has an interpretive effect: ``\dots  reproject is nothing but mere project, which happens again as different interpretive demands of the elements that merge arise''.}

\begin{exe} 
    \ex\label{ex:eguren:17} 
     \begin{xlist}
    \ex
   {[{{}$_{IP}$} [{{}$_{QP}$} most people]$_i$ [{{}$_{I'}$} I [{{}$_{VP}$} t$_i$ love children]]]} 
    \ex\label{ex:eguren:17b}
   {[{{}$_{QP}$} [{{}$_{Q'}$} most people]$_i$ [{{}$_{IP}$} I [{{}$_{VP}$} t$_i$ love children]]]}
    \end{xlist}
\end{exe}

As illustrated in \ref{ex:eguren:17}, reprojection thus involves heads and their specifiers, so that, after reprojection, the head of a YP in [Spec, XP] is turned into the head of the whole phrase, and the XP is now its specifier. This is depicted in \ref{ex:eguren:18}:

% figure 4
\begin{exe}
\ex\label{ex:eguren:18}
\begin{forest}  
[XP [YP] [X' [X]]]
\node (a) at (1.5, -1) {};
\node (b) at (3.5, -1) {};
\draw[-Latex] (a) -- (b);
\end{forest}
\quad
\begin{forest}
[YP [Y' [Y]] [XP]]
\end{forest}
\end{exe}

We consider that reprojection gives rise to the structure from which the pied-piping adjectival wh-comparatives in \ref{ex:eguren:16} are derived. Our account goes as follows. As shown in \ref{ex:eguren:19}, once reprojection has applied, the head of the QP in [Spec, DegP] in the structure in \ref{ex:eguren:13} heads the construction, and the DegP becomes its specifier: 

\begin{exe} 
\ex\label{ex:eguren:19} 
% Original:
% \rnode{A1}{} \quad {[\rnode{A2}{{}$_{QP}$} [{{}$_{Q'}$} cuánto/qué tanto/cómo] [{{}$_{DegP}$} (de) más alta que María]]}\\
% \ncbar[angle=-90]{->}{A2}{A1}
\begin{forest}
[{[\capsub{QP} [\capsub{Q'} cuánto/qué tanto/cómo] [\capsub{DegP} (de) más alta que María]]},align=left]
\node (start) at (-4.6, 0) {};
\node (end) at (-5.4, 0.4) {};
\draw[-Latex] (start.south) -- +(0,-0.2)-| node[below right]{} (end);
\end{forest}
\end{exe}


In the reprojected structure in \ref{ex:eguren:19}, the wh-expression has the categorial status of an intermediate (X') projection and, given the condition that intermediate projections are not subject to movement operations (Chomsky \citeyear[4]{chomsky1986barriers}; Chomsky \citeyear[396]{chomsky1995a}), it obligatorily pied-pipes the DegP in [Spec, QP].\footnote{The pied-piping structure in \ref{ex:eguren:19} could also be obtained, in principle, by using \cite{citko2008missing}'s 
`Project Both' labeling algorithm, whereby Merge allows that any of the two merged elements (or both) projects as the label. However, \cite[926]{citko2008missing} claims that, in the case of External Merge, ``the elements that can undergo Project Both are categorically non-distinct'', whereas the two merged elements in the constructions under study have a different categorial status: the comparative+adjective constituent is a DegP and the differential is a QP.}

In support of our proposal, we would like to point out that a reprojection analysis of differential comparatives like the one above has also been advocated by \cite{lopez2014left} in order to account for the mixed properties of the correlative \textit{cuanto} `how much' and the correlatum \textit{tanto} `so much' in Spanish comparative correlatives, illustrated in \ref{ex:eguren:20} (all the examples below are taken from \cite{lopez2014left}):

\begin{exe} 
    \ex\label{ex:eguren:20} 
\gll Conozco (tantas)	más	culturas	diferentes cuantos	más	países	visito.\\
I-know 	{so many}	more	cultures	different how-many	more	countries I-visit\\
\glt `The more countries I visit, the more different cultures I know.'
\end{exe}

In the construction in \ref{ex:eguren:20}, \textit{tanto} and \textit{cuanto} are, on the one hand, the differentials of their respective comparative operators, thus denoting a quantity that defines an interval. This is shown by the fact that they cannot co-occur with a differential modifier:


\begin{exe} 
    \ex\label{ex:eguren:21} 
\gll	Cuanto		{(*diez	centímetros)}	más	alta	es	Ana, tanto (*mucho)	más	atractiva	es.\\
how-much	  {ten		centimeters} 	more	tall	is	Ana, so-much		much 		more	attractive	she-is\\
\end{exe}

On the other hand, just like in the Spanish correlative structure with \textit{tanto} and \textit{cuanto} in \ref{ex:eguren:22}, also in the comparative correlative in \ref{ex:eguren:20}, the correlatum and the correlative sentence are mutually dependent:

\begin{exe} 
    \ex\label{ex:eguren:22} 
\gll Tú		conoces		a	tantos		hombres		cuantas		mujeres	conozco		yo.\\
you	know.2sg	to	so-many	men	how-many	women	know.1sg	I\\
\glt `You know as many men as I know women.'\\
\end{exe}

As shown in the simplified structural representations in \ref{ex:eguren:23}, to capture these facts, \cite[169--170]{lopez2014left} argues that \textit{tanto} and \textit{cuanto} in comparative correlatives are first merged as differential phrases in [Spec, DegP] (left), and then become heads after reprojection (right), thus allowing for the selection relation between the two quantifiers to be properly established in structural terms.

% figure 5
\begin{exe}
\ex\label{ex:eguren:23}
\begin{forest}  
[DegP [QP\\\textit{tanto/cuanto}, align=center, base=bottom] 
      [Deg' [Deg$^0$\\\textit{más}, align=center, base=bottom] [\dots]]]
\node (a) at (2, -1.5) {};
\node (b) at (4, -1.5) {};
\draw[-Latex] (a) -- (b);
\end{forest}
\quad
\begin{forest}
[QP [Q'\\\textit{tanto/cuanto}, align=center, base=bottom] 
      [Deg' [Deg$^0$\\\textit{más}, align=center, base=bottom] [\dots]]]
\end{forest}
\end{exe}

\citeauthor{lopez2014left} proposes, in particular, that in the constructions in \ref{ex:eguren:20} and \ref{ex:eguren:22}, the correlation \textit{tanto}\dots \textit{cuanto} is a comparative structure of equality involving a predication relation, which guarantees that the correlatum (\textit{tanto}) and the relative clause introduced by \textit{cuanto} denote the same quantity. By means of reprojection, the two members in correlative comparatives, which are basically DegPs expressing a comparison of superiority/inferiority, are therefore reinterpreted as QPs that can enter into an equality comparison structure. \cite{lopez2014left}'s reprojection-based analysis of Spanish comparative correlatives is thus consistent with the view in \cite[fn. 3]{hornstein2002reprojections} that, for reprojection to take place, it must have an interpretive effect.

To complete our analysis of (non)-pied-piping adjectival wh-comparatives in Spanish, we would like to claim that reprojection also has an interpretive effect in this case, which now has to do with the different information-structural content of the structures in \ref{ex:eguren:14} and \ref{ex:eguren:16}, repeated below, using the constructions with \textit{cuánto} `how much' for illustration: 

\begin{exe} 
    \ex\label{ex:eguren:24} 
    \begin{xlist}
            \ex\label{ex:eguren:24a}
\gll ¿Cuánto es	Ana más		alta	que	María?\\
     {how much}		is	Ann more	tall	than	Mary\\
            \ex\label{ex:eguren:24b}
\gll ¿Cuánto más	alta	que	María	es	Ana?\\
   {how much}		more	tall	than	Mary		is	Ann\\
\end{xlist}
\end{exe}

In the discontinuous structure in \ref{ex:eguren:24a}, only the differential \textit{cuánto} is asked for. Under the standard assumption that wh-elements ask for new information and thus carry focal prominence in the sentence (see, e.g., \citeauthor{rizzi1997fine} \citeyear{rizzi1997fine}), in this structure \textit{cuánto} is focused, and the fact that Ann is taller than Mary is presupposed. In the pied-piping structure in \ref{ex:eguren:24b}, however, the whole comparative phrase is now asked for (and thereby focused), so that it is not presupposed that Ann is taller than Mary. That this is the case is shown by the contrast in examples like the ones in \ref{ex:eguren:25} and \ref{ex:eguren:26}: 

\begin{exe} 
    \ex\label{ex:eguren:25} 
    \begin{xlist}
            \ex
\gll¿Cuánto			más	alta	que	María	es	Ana?\\
     {how much}		more	tall	than	Mary		is	Ann\\
            \ex
\gll Nada.	Son		exactamente	iguales.\\
  nothing	are.3pl	exactly	{the same}\\
\end{xlist}
\end{exe}

\begin{exe} 
    \ex\label{ex:eguren:26} 
    \begin{xlist}
            \ex
\gll ¿Cuánto es	Ana más		alta	que	María?\\
     {how much}	is	Ann more	tall	than	Mary\\
            \ex
\gll \#Nada.		Son		exactamente	iguales.\\
 nothing	are.3pl	exactly	{the same}\\
\end{xlist}
\end{exe}

The answer (B) to the questions in \ref{ex:eguren:25} and \ref{ex:eguren:26}, stating that there is no difference in height between the two individuals at hand, is appropriate in \ref{ex:eguren:25}, and sounds odd in \ref{ex:eguren:26}, because only in the first case it is not presupposed that Ann is taller than Mary.\footnote{We owe both this idea and the examples in \ref{ex:eguren:25}--\ref{ex:eguren:26} to Cristina Sánchez López (p.c.).} Reprojection in adjectival wh-comparatives in Spanish thus has the interpretive effect of making the whole comparative phrase become the focus of the question.\footnote{\cite[110]{hornstein2002reprojections} claim that reprojection induces an intervention effect in split constructions like Negative Polarity Licensing (e.g., \textit{*Nobody gave most children a red cent}). Juan Uriagereka (p.c.) points to us that, under his judgements (that we share), a Spanish sentence like the one in \ref{ex:eguren:fn2a} is worse than the one in \ref{ex:eguren:fn2b}, which would count as additional support for our reprojection-based analysis of the pied-piping wh-comparative structure in \ref{ex:eguren:fn2a}:
\begin{exe}
      \ex\label{ex:eguren:fn2}
       \begin{xlist} % declares a list of sub-examples
        \ex\label{ex:eguren:fn2a} % sets the sub-example letter, in this case (a)
            \gll   No sé cuánto 		más	rico	que	yo	pueda	ser	maestro	alguno.\\   % target language line
           	not know.1sg {how much} 	more	rich	than	I	can.3sg.subj	be	teacher	any\\
            \ex\label{ex:eguren:fn2b}
            \gll No	sé 	cuánto	pueda			ser	maestro	alguno		más	rico	que	yo.\\
            not know.1sg	{how much}		can.3sg.subj	be	teacher	any		more	rich	than	I\\
\end{xlist}
 \end{exe}
}

In the next section, we will extend our analysis of (non)-pied-piping adjectival wh-comparatives to (non)-pied-piping structures in Spanish wh-comparatives with nouns. 

\section{(Non)-pied-piping wh-comparatives in the nominal domain}
\label{sec:eguren:5}

As illustrated in \ref{ex:eguren:27} and \ref{ex:eguren:28}, non-pied-piping and pied-piping structures are also well-formed in nominal wh-comparatives in Spanish, which are now headed by the general Spanish wh-item \textit{cuántos} `how many' or the American Spanish wh-expression \textit{qué tantos} `what so many':\footnote{On \textit{qué tanto}(s)+N in Old and American Spanish, see \cite{de2006cuantificadores}. The patterns in \ref{ex:eguren:27} and \ref{ex:eguren:28} also obtain with adverbial wh-comparatives:
\begin{exe}
      \ex\label{ex:eguren:fn3}
       \begin{xlist} % declares a list of sub-examples
        \ex % sets the sub-example letter, in this case (a)
            \gll  {¿Cuánto/Qué tanto}			corre	más	este	coche?\\   % target language line
            {how much/what so much}	runs	more	 this	car\\
            \ex
            \gll {¿Cuánto/Qué tanto}	más	corre	este	coche?\\
            {how much/what so much}	more	runs	this	car\\
\end{xlist}
 \end{exe}
}

\begin{exe} 
    \ex\label{ex:eguren:27} 
    \begin{xlist}
            \ex
\gll ¿Cuántos		libros		leyó	Pedro	más	que	Juan?\\
     {how many}	books	read	Peter		more	than	John\\
            \ex
\gll ¿Cuántos		libros		más	que	Juan	leyó	Pedro?\\
  {how many}	books	more	than	John	read	Peter\\
\end{xlist}
\end{exe}


\begin{exe} 
    \ex\label{ex:eguren:28} 
    \begin{xlist}
            \ex
\gll ¿Qué		tantos		libros		leyó	Pedro	más	que	Juan?\\
   what	{so many}	books read	Peter more	than	John \\
            \ex
\gll ¿Qué		tantos		libros		más	que	Juan	leyó	Pedro?\\
    what {so many}	books	more	than	John	read	Peter\\
\end{xlist}
\end{exe}

Furthermore, as in adjectival differential comparatives (see the examples in \ref{ex:eguren:6} and \ref{ex:eguren:7}), both non-pied-piping and pied-piping structures are once again possible in cases of focus movement in Spanish comparatives with nouns:

\begin{exe} 
    \ex\label{ex:eguren:29} 
    \begin{xlist}
            \ex
\gll DOS LIBROS	leyó	Pedro	más	que	Juan.\\
   two	books	read	Peter	more than John\\
            \ex
\gll DOS LIBROS	MÁS	QUE	JUAN leyó	Pedro.\\
    two	books	more than	John read	Peter\\
\end{xlist}
\end{exe}

Focusing on wh-comparatives, and assuming that pied-piping cannot be optional, our analysis of the non-pied-piping and pied-piping structures in \ref{ex:eguren:27} and \ref{ex:eguren:28} goes in the same line as the one we have proposed for (non)-pied-piping adjetival wh-comparatives in the previous section. We will first adopt \cite{brucart2003adicion}'s structural analysis of differential comparatives with nouns, represented below:\footnote{For other analyses of nominal comparatives, which share the basic insights in \cite{brucart2003adicion}, see \cite{gallego2013iv} and the references therein. In all these analyses, the differential phrase also ends up being located in the Spec position of the highest functional head in the comparative construction.}

\begin{exe} 
\ex\label{ex:eguren:30} 
\begin{xlist}
\ex\label{ex:eguren:30a} 
\gll muchos	libros	más	de	los		cuatro	previstos\\
   many	books	more	of	the	four	envisaged\\
\ex\label{ex:eguren:30b} 
%%%Original:
% \jtree[scaleby=1.5, labelgap=0]
% \! = {DegP}
%     <wideleft>{QP}({\textit{muchos libros$_i$}})             ^<wideright>{Deg'}
%     <wideleft>{Deg'}(<left>{Deg}({\textit{más}})
%     ^<right>{QP}({\textit{t$_i$}}))
%     ^<wideright>{PP}      
%     <left>{{P}}({\textit{de}})  ^<right>{QP}({\textit{los cuatro previstos}}).
% \endjtree
\begin{forest}  
[DegP
  [QP\\\textit{muchos libros}$_i$, align=center, base=bottom]
  [Deg' 
    [Deg'
      [Deg\\\textit{más}, align=center, base=bottom]
      [QP\\\textit{t}$_i$, align=center, base=bottom]]
    [PP
      [P\\\textit{de}, align=center, base=bottom]
      [QP\\\textit{los cuatro previstos}, align=center, base=bottom]]]]
\end{forest}
\end{xlist}
\end{exe}

Brucart takes the comparative morpheme to be an additive (or substractive) operator heading a DegP that underlyingly selects two complements: a differential QP including a quantifier and the noun, and a base, corresponding to the comparative code, which contains another QP. As shown in \ref{ex:eguren:30b}, in Brucart's view, the differential phrase further raises to [Spec, DegP] to check a [+differential] feature of the Deg head.\footnote{\
According to \cite{brucart2003adicion}, in comparatives with bare nouns (\textit{más libros que Juan} `more books than John'), this feature is checked by LF-movement of an empty quantifier in the differential QP. Moreover, in some cases, the overt quantifier in the differential phrase can also raise alone, leaving the noun \textit{in situ} (\textit{muchos/pocos/bastantes más libros que Juan} `many/few/enough more books than John').}

Under Brucart's analysis of nominal comparatives, the non-pied-piping construction in \ref{ex:eguren:31a} originates in the structure in \ref{ex:eguren:31b}: in \ref{ex:eguren:31b}, the differential wh-phrase is first internally merged in [Spec, Deg], and then moves alone to the sentential left-periphery, stranding the comparative operator and the comparative code: 

\begin{exe} 
\ex\label{ex:eguren:31} 
\begin{xlist}
\ex\label{ex:eguren:31a} 
\gll ¿Cuántos		libros		leyó	Pedro	más	que	Juan?\\
  {how many}	books	read	Peter		more	than	John \\
\ex\label{ex:eguren:31b} 
%%%Original:
% {\rnode{A1}{}}
% \jtree[scaleby=1.5, labelgap=0]
% \! =  {DegP}
%   <wideleft>{QP}({\rnode{A2}{\textit{cuantos libros$_i$}}})         ^<wideright>{Deg'}
%     <left>{Deg'}(<left>{Deg}({\textit{más}})
%     ^<right>{QP}({\textit{t$_i$}}))
%     ^<right>{XP}({qué juan}).
% \ncbar[angle=-90]{->}{A2}{A1}
% \endjtree
\begin{forest}  
[DegP
  [QP\\\textit{cuantos libros}$_i$, align=center, base=bottom, name=qp]
  [Deg' 
    [Deg'
      [Deg\\\textit{más}, align=center, base=bottom]
      [QP\\\textit{t}$_i$, align=center, base=bottom]]
    [XP\\\textit{qué juan}, align=center, base=bottom]]]
\node (a) at (-2.2, 0.3) {};
\draw[-Latex] (qp.south) -- +(0,-0.3)-| node[below right]{} (a);
\end{forest}\\
\end{xlist}
\end{exe}

As for the pied-piping wh-comparative in \ref{ex:eguren:32a}, this structure, just like the pied-piping structure in adjectival wh-comparatives, results from reprojection: as shown in \ref{ex:eguren:32b}, after reprojection, the quantifier in the differential QP heads the construction, and pied-pipes the whole comparative phrase.

\begin{exe} 
\ex\label{ex:eguren:32} 
\begin{xlist}
\ex\label{ex:eguren:32a} 
% Original:
% \gll ¿Cuántos		libros		más	que	Juan	leyó	Pedro?\\
%   {how many}	books	more	than	John	read	Peter\\
% \rnode{A1}{} \quad [\rnode{A2}{{}$_{QP}$} [{{}$_{Q'}$} cuántos libros] [{{}$_{DegP}$} más que Juan]]\\
% \ncbar[angle=-90]{->}{A2}{A1}

\gll ¿Cuántos		libros		más	que	Juan	leyó	Pedro?\\
   {how many}	books	more	than	John	read	Peter\\
\ex \label{ex:eguren:32b} 
\begin{forest}
[{[\capsub{QP} [\capsub{Q'} cuántos libros] [\capsub{DegP} más que Juan]]},align=left]
\node (start) at (-3.1, 0) {};
\node (end) at (-4, 0.4) {};
\draw[-Latex] (start.south) -- +(0,-0.2)-| node[below right]{} (end);
\end{forest}
\end{xlist}
\end{exe}


As in adjectival wh-comparatives, in wh-comparatives with nouns, reprojection also has the interpretive effect of making the whole comparative phrase become the focus of the question: in the non-pied-piping structure in \ref{ex:eguren:31a}, the differential \textit{cuántos libros } is focused, and the fact that Peter read more books than John is presupposed; in the pied-piping structure in \ref{ex:eguren:32a}, the whole comparative phrase is now focused, and it is not presupposed that Peter read more books than John.

To conclude this section, we will discuss an apparent counter-example to our proposal on (non)-pied-piping structures in Spanish wh-comparatives with nouns. Like other Spanish nominal quantifiers (see fn. 14), both \textit{cuántos} and \textit{qué tantos} can also raise to [Spec, DegP] in the structure in \ref{ex:eguren:30b} on its own, stranding the noun. In this case, as illustrated in \ref{ex:eguren:33}, \textit{cuántos} and \textit{qué tantos} pied-pipe the comparative phrase, and the corresponding non-pied-piping structure in \ref{ex:eguren:34} is not well-formed:\footnote{The examples in \ref{ex:eguren:33a} and \ref{ex:eguren:33b} are taken from \citeauthor{espanola2000corpus} and \cite[1027]{de2006cuantificadores}, respectively. There seems to be variation among Spanish speakers as regards the acceptability of this construction.}  

\begin{exe} 
    \ex\label{ex:eguren:33} 
    \begin{xlist}
\ex\label{ex:eguren:33a}
\gll ¿Y		cuántos		más	saquitos	tiene			de		estos?\\
   and	{how many}	more	{little bags}	have.3sg	of	these \\
\ex\label{ex:eguren:33b}
\gll ¿Qué	tantos	más	guardabosques	tenemos?\\
  what	{so many}	more	rangers		have.1pl\\
\end{xlist}
\end{exe}

\begin{exe} 
    \ex\label{ex:eguren:34} 
 \gll {*} {¿Cuántos/Qué tantos}	ha		leído	más	libros?\\
{} {how many/what so many}	has	read	more	books\\
\end{exe}

The pattern in \ref{ex:eguren:33} and \ref{ex:eguren:34} \textit{prima facie} questions our analysis of non-pied-piping nominal wh-comparatives in \ref{ex:eguren:31b}: under this analysis, the quantifier in the [\textit{cuántos/qué tantos más} N] comparative construction, being located in [Spec, DegP], should move alone, but it does not, as can be seen in \ref{ex:eguren:34}.

To account for examples like the ones in \ref{ex:eguren:33} and \ref{ex:eguren:34}, we tentatively suggest that  the ungrammaticality of the sentences in \ref{ex:eguren:34} is related to the fact that the Spanish nominal quantifiers \textit{cuántos} and \textit{qué tantos} agree in gender and number with the noun. Assuming that this agreement relation must always hold within a local domain, the quantifier cannot move alone, and pied-piping, as in \ref{ex:eguren:33}, is the only option.

That this idea is on the right track is supported by the fact that \textit{cuánto} and \textit{qué tanto} in adjetival wh-comparatives are invariable wh-forms not showing agreement with the adjective, and can be extracted, as shown in \ref{ex:eguren:35}, on their own:

\begin{exe} 
    \ex\label{ex:eguren:35} 
\gll {¿Cuánto/Qué tanto} son	Ana	y	María	más	altas?\\
 {how much/what so much} are	Ann	and	Mary more tall.fem.pl\\
\end{exe}

\section{Conclusions}
\label{sec:eguren:6}

In this paper, adhering to the view in \citeauthor{heck2008pied} (\citeyear{heck2008pied,heck2009certain}) that optional pied-piping cannot exist, we have argued that pied-piping and non-pied-piping adjectival wh-comparatives in Spanish originate from two different structures. Non-pied-piping structures (e.g., \textit{¿Cúanto es Ana más alta?} `lit. How much is Ann more tall?') underly a syntactic configuration in which the differential wh-form is generated as a maximal projection in [Spec, DegP], thus moving alone. Pied-piping structures (e.g., \textit{¿Cuánto más alta es Ana?} `lit. How much more tall is Ann?') are derived through reprojection \citep{hornstein2002reprojections}, by which the head of the QP in [Spec, DegP] becomes the head of the comparative construction, and pied-pipes \textit{más}+A. We have further shown that an analysis along these lines also applies to (non)-pied-piping structures in Spanish wh-comparatives with nouns. Our analyses of (non)-discontinuous wh-comparatives in Spanish thus support \citeauthor{heck2008pied} (\citeyear{heck2008pied,heck2009certain})'s claim that alleged cases of optional pied-piping are always apparent.

\section*{Acknowledgements}
We would like to thank Cristina Sánchez López and two anonymous reviewers for their comments on the content of this paper, which has been supported by a grant to the project FFI2017-87140-C4-4-P and by the Salvador de Madariaga grant PRX18/00009.

\printbibliography[heading=subbibliography,notkeyword=this]

\end{document}