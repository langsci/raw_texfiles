%%%%%%%%%%%%%
%%% Editor Notes
%%% 1. Bibliography has to be reformatted
%%%%%%%%%%%%%


\documentclass[output=paper,colorlinks,citecolor=brown,
% hidelinks,
% showindex
]{langscibook}

\author{Kate Bove\affiliation{New Mexico State University}}
\title{(Non)veridicality and mood selection in doxastic predicates in Yucatec Spanish}
\abstract{In Yucatec Spanish, a contact variety spoken on the Yucatán Peninsula of Mexico, speaker mood selection in the complement of doxastic (belief) predicates does not pattern with other varieties of Spanish. To account for this mood selection, I move towards a modal analysis to explain the potential expansion of the semantic space occupied by subjunctive in this contact variety. I adopt the framework of nonveridicality \citep{Giannakidou1998,Giannakidou2002,Giannakidou2006,Giannakidou2009,Giannakidou2015,GiannakidouMari2020,Mari2016} and argue that mood selection in Yucatec Spanish is motivated by the interpretation of the embedded proposition.
\citet{Giannakidou2015} and \citet{Mari2016} argue that mood selection reflects this epistemic certainty (or lack thereof) that denotes a set of possible worlds that includes worlds in which the proposition \textit{p} is true and worlds in which \textit{p} is false. As such, the use of the subjunctive indicates epistemic weakening to the commitment of \textit{p}. Using data from Yucatec Spanish, I propose a formal analysis for doxastic predicates like \textit{creer} `to believe' which involves two evaluations: (1) the epistemic model of the referent denoted by the NP of the matrix clause and (2) the model of the speaker, which licenses the mood selected in the embedded clause. To reflect the speaker´s evaluation, I propose what is called a \textsc{reflect} operator, which therefore triggers the subjunctive mood in doxastic utterances where the speaker is epistemically uncertain. Lastly, there is a brief discussion of language contact.}

\IfFileExists{../localcommands.tex}{
 \addbibresource{../localbibliography.bib}
 \usepackage{langsci-optional}
\usepackage{langsci-gb4e}
\usepackage{langsci-lgr}

\usepackage{listings}
\lstset{basicstyle=\ttfamily,tabsize=2,breaklines=true}

%added by author
% \usepackage{tipa}
\usepackage{multirow}
\graphicspath{{figures/}}
\usepackage{langsci-branding}

 
\newcommand{\sent}{\enumsentence}
\newcommand{\sents}{\eenumsentence}
\let\citeasnoun\citet

\renewcommand{\lsCoverTitleFont}[1]{\sffamily\addfontfeatures{Scale=MatchUppercase}\fontsize{44pt}{16mm}\selectfont #1}
   
 %% hyphenation points for line breaks
%% Normally, automatic hyphenation in LaTeX is very good
%% If a word is mis-hyphenated, add it to this file
%%
%% add information to TeX file before \begin{document} with:
%% %% hyphenation points for line breaks
%% Normally, automatic hyphenation in LaTeX is very good
%% If a word is mis-hyphenated, add it to this file
%%
%% add information to TeX file before \begin{document} with:
%% %% hyphenation points for line breaks
%% Normally, automatic hyphenation in LaTeX is very good
%% If a word is mis-hyphenated, add it to this file
%%
%% add information to TeX file before \begin{document} with:
%% \include{localhyphenation}
\hyphenation{
affri-ca-te
affri-ca-tes
an-no-tated
com-ple-ments
com-po-si-tio-na-li-ty
non-com-po-si-tio-na-li-ty
Gon-zá-lez
out-side
Ri-chárd
se-man-tics
STREU-SLE
Tie-de-mann
}
\hyphenation{
affri-ca-te
affri-ca-tes
an-no-tated
com-ple-ments
com-po-si-tio-na-li-ty
non-com-po-si-tio-na-li-ty
Gon-zá-lez
out-side
Ri-chárd
se-man-tics
STREU-SLE
Tie-de-mann
}
\hyphenation{
affri-ca-te
affri-ca-tes
an-no-tated
com-ple-ments
com-po-si-tio-na-li-ty
non-com-po-si-tio-na-li-ty
Gon-zá-lez
out-side
Ri-chárd
se-man-tics
STREU-SLE
Tie-de-mann
} 
 \togglepaper[23]%%chapternumber
}{}

\begin{document}
\maketitle

\section{Introduction}

In Spanish, utterances containing doxastic predicates such as \textit{creer} `to believe' or \textit{pensar }`to think' introduce or report the belief of an individual. The embedded proposition, or in this case what is believed, is marked with either the subjunctive or indicative moods:

\begin{exe} % declares the example environment
    \ex\label{ex:bove:1} 
    \begin{xlist} % declares a list of sub-examples
        \ex \label{ex:bove:1a}
          Creo que es (\textsc{ind})/ *sea (\textsc{subj})  la respuesta correcta.   % target language line
            \glt `I think that it is the right answer.'
        \ex\label{ex:bove:1b}
            No creo que *es (\textsc{ind})/ sea (\textsc{subj})  la respuesta correcta.
            \glt `I do not think that it is the right answer.'
\end{xlist}
\end{exe}

In this example, only the indicative is acceptable under the affirmative belief predicate in \ref{ex:bove:1a} while negation triggers the subjunctive in \ref{ex:bove:1b}. This is referred to as polarity subjunctive xxx and can be observed in many varieties of Spanish. This mood pattern has been referred to as the assertion/non-assertion divide  in which the affirmative belief statement asserts a proposition while the negated belief statement communicates a 
`dubitative flavor'. However, under negation, alternation is permitted:


\begin{exe} % declares the example environment
    \ex\label{ex:bove:2} 
    \begin{xlist} % declares a list of sub-examples
        \ex \label{ex:bove:2a}
          No creo que es la respuesta correcta.   % target language line
                   \glt `I do not think that it is the right answer.'
        \ex\label{ex:bove:2b}
           No creo que sea la respuesta correcta.
                   \glt `I do not think that it is the right answer.'
\end{xlist}
\end{exe}


The motivation for this alternation has been credited to pragmatic presupposition \citep{Haverkate2002,Lunn1989} Mejías-Bikandi 1994.\todo{Bibliography entry for Mejías-Bikandi (1994) missing} Within these analyses, pragmatically presupposed utterances license the subjunctive, and a lack of this presupposition, which is considered an assertion, licenses the indicative. More recently, several researchers \citep{Gielau2015,Quer2001,Villalta2008} have adopted an intensional worlds analysis of this alternation that shifts the evaluation of the proposition to other worlds.

Yucatec Spanish presents an interesting case of mood selection under doxastic predicates. For speakers of this contact variety, alternation is permitted under both affirmative and negated matrix clauses that contain doxastic predicates like \textit{creer }`to believe' (see \citealt{Bove2020}). The alternation observed in such statements is contextually determined, making only one mood appropriate when given a specific situation. Many previous accounts of Yucatec Spanish show that this contact variety spoken in Yucatán, Mexico, has many features due to language contact with Yucatec Maya, demonstrating differences between bilingual and monolingual speakers (i.e. \citealt{Bove2019,Bove2020,Michnowicz2011,Michnowicz2015,Solomon1999}). However, it has been shown that many features of this variety are observed in both bilingual and monolingual speakers, making it markedly \textit{yucateco}. The previous accounts of Spanish mood selection that are driven primarily by pragmatically motivated mood are not able to fully capture the observed alternation in Yucatec Spanish. Instead, I turn to the notion of (non)veridicality \citep{Giannakidou1998,Giannakidou2002,Giannakidou2006,Giannakidou2009,Giannakidou2015,GiannakidouMari2020,Mari2016}. At its most basic, there is a tripartite classification: veridical, anti-veridical, and nonveridical. A statement that is \textit{veridical }is true while a statement that is \textit{anti-veridical }is false. The third category, \textit{nonveridical}, communicates uncertainty with regard to the truth of a proposition. A speaker makes an evaluation of veridicality based on their perceptions of truth, referred to as the epistemic model. Using this framework, I argue that a speaker's perception of (non)veridicality licenses mood in doxastic statements in Yucatec Spanish. 


The following data were collected during acceptability judgement interviews conducted in Valladolid, Yucatán. Twenty Yucatec Spanish consultants participated, and all were born and raised in Yucatán and had spent their entire lives in Valladolid or a Maya community within 10 kilometers of Valladolid. There were ten bilingual and ten monolingual participants who took part in this study. In the following presentation of research, I group together all speakers, but it is important to note that the unanticipated mood selection occurs at highest rates in bilingual, non-university educated speakers. However, a statistical analysis of acceptability is outside the scope of the present analysis and will be reserved for future work. This manuscript is organized in the following way: Section two presents previous literature on the subjunctive in Spanish and the notion of (non)veridicality. Section three presents Yucatec Spanish mood data. Section four presents a formalization of (non)veridicality as applied to Yucatec Spanish, and conclusions, implications, and directions for further research are discussed in section five.
 
\section{The subjunctive in Spanish and modal approaches to analyzing mood}

Descriptive and functional analyses of the Spanish subjunctive (i.e. Bull 1960; Gili y Gaya 1969; Rivero 1971  \citealt{TerrellHooper1974,GarciaTracey1997,Bell1980,Lunn1989,Lunn1995}; Studerus 1995)\todo{Bibliography entries missing for Bull, Gili y Gaya, Rivero, Studerus} are in agreement that mood selection can be explained in terms of assertions and presuppositions. When a speaker's utterance is an assertion, the speaker presents this information as fact. In Spanish, an assertion licenses the indicative. On the other hand, a non-assertion does not present any information as fact, nor does it necessarily describe it as false. A non-assertion often licenses the subjunctive in Spanish. This can be seen in examples \ref{ex:bove:3} and \ref{ex:bove:4}, which demonstrate the effects of negation on assertion and dubitative meaning \citep[490]{TerrellHooper1974}:


\begin{exe} % declares the example environment
    \ex\label{ex:bove:3} 
    \begin{xlist} % declares a list of sub-examples
        \ex \label{ex:bove:3a}
           Creo que Martín ha (\textsc{ind}) leído ese libro.   % target language line
                   \glt `I think that Martin has read that book.'
        \ex\label{ex:bove:3b}
           No creo que sea la respuesta correcta.
                   \glt `I do not think that it is the right answer.'
\end{xlist}
\end{exe}

\begin{exe} % declares the example environment
    \ex\label{ex:bove:4} 
    \begin{xlist} % declares a list of sub-examples
        \ex \label{ex:bove:4a}
            Dudo que Consuelo sea (\textsc{subj}) culpable.  % target language line
                   \glt `I doubt that Consuelo is guilty.'
        \ex\label{ex:bove:4b}
            No dudo que Consuelo es (\textsc{ind}) culpable.
                   \glt `I do not doubt that Consuelo is guilty.'
\end{xlist}
\end{exe}


The shifts in certainty (and, therefore, in mood) in examples \ref{ex:bove:3} and \ref{ex:bove:4} demonstrate that mood in some Spanish predicates can be sensitive to linguistic elements such as negation. \citeauthor{TerrellHooper1974}'s (\citeyear{TerrellHooper1974}) work on mood was paramount in that it presented a succinct account of mood selection in several predicate types in Spanish. There are, however, a few shortcomings within their analysis. Most notably for the current analysis, while \citeauthor{TerrellHooper1974} state that belief predicates are assertive, Mejías-Bikandi (1994)\todo{Bibliography entry missing} points out that belief predicates cannot be totally assertive; the act of embedding a proposition under the matrix \textit{I believe} expresses some doubt of the truth of the complement clause. 




Like \citet{TerrellHooper1974}, \citet{Lunn1989} investigates mood selection from a pragmatic perspective and theorizes that pragmatically presupposed propositions appear in the subjunctive. Her approach argues that subjunctive-marked VPs share low information value through which the speaker is able to communicate a belief that the utterance is flawed in news value. \citeauthor{Lunn1995}'s (\citeyear{Lunn1995}) account of the subjunctive is more detailed than both previous and later accounts of the subjunctive and includes relative clauses such as those with future temporal reference like \textit{cuando} `when'. Overall, Lunn states that the reason for such a complex mood system in Spanish is that mood cannot be explained exclusively through any one area of linguistic structure; instead, mood alternation reflects semantic, syntactic, and pragmatic changes, and mood selection can vary significantly even between native Spanish speakers.



Accounts of (non)assertion and presupposition capture mood selection in many varieties of Spanish, and several newer accounts of the Spanish subjunctive (i.e. \citealt{Mejias-Bikandi1998,Haverkate2002}) discuss the pragmatically motivated mood alteration in negated doxastic predicates. Under negation, speakers can communicate commitment to the proposition through mood change. In such cases, the subjunctive allows speakers to withhold commitment to the proposition \citep{Mejias-Bikandi1998} or communicates a lack of knowledge of veracity of a proposition \citep{Haverkate2002}. However, none of the previously mentioned theories account for alternation under affirmative doxastic predicates. 



Instead of approaching this case using traditional analyses of Spanish subjunctive, I argue that mood selection can be analyzed from a modal approach, in which the quantification over possible worlds and anchoring of models of evaluation allow for interpretation of mood. In possible worlds semantics, a proposition's truth value is connected to individuals and their respective worlds. With regard to doxastic predicates, Farkas (1999)\todo{Bibliography entry missing; did you mean Farkas (1992)?} argues that verbs of belief are similar to assertions in that they are evaluated in an epistemic context, which \citet[7]{Farkas2003} defines as follows:

\begin{exe}
    \ex\label{ex:bove:5}
    E$^{i,w}$ is the set of worlds compatible with what i believes in w
\end{exe}


Within the definition in \ref{ex:bove:5}, E is the epistemic context, \textit{i }refers to the individual anchor, and \textit{w} refers to the actual world. An utterance S is true in the actual world w iff it is also true in the context set E, that is, iff it is true in the set of worlds compatible with what \textit{i }believes in the actual world w. Belief predicates reflect an individual's perspective within his/her epistemic context. This notion of an individual's worldview specifies only the worlds within the epistemic context that align with the individual's perspective of truth. If the Spanish subjunctive is analyzed as a reflection of doubt, a belief statement that licenses the subjunctive cannot also have assertive force, as \citet{Farkas2003} posits.



Like \citeauthor{Farkas2003}, \citeauthor{Quer2001}'s (\citeyear{Quer2001}) modal analysis relies on a model of evaluation through which a speaker evaluates a proposition. According to Quer, the indicative is the default mood that marks that a proposition is evaluated with respect to the actual world, which he argues is considered the default world, and the use of the indicative indicates that the model in which the embedded proposition should be evaluated is M\textsubscript{E}(speaker). M\textsubscript{E}(speaker) and M\textsubscript{E}(x) (or the model of a referent denoted by an NP x) are the defaults since assertion is viewed as a basic conversational move \citep[87]{quer1998}. However, a change of mood from the indicative to the subjunctive indicates a modal shift, overtly marking that the embedded proposition should be evaluated in a world other than the real world. The differentiation between extensionally and intensionally anchored propositions allows for a modal interpretation; while extensionally anchored propositions limit interpretation to only events that occur in the real world, intensionality allows interpretation to extend to other possible worlds (see \citealt{Kemchinsky2009} for an analysis of strong intensional modals in Spanish). 



With regards to mood alternation, \citet[91]{Quer2001} argues that there are two possibilities in negated epistemics, as seen in the following examples from Catalán:


\begin{exe} % declares the example environment
    \ex\label{ex:bove:6} 
    \begin{xlist} % declares a list of sub-examples
        \ex \label{ex:bove:6a} 
            \gll  El jurat    no creu 	  [que sigui     innocent.]\capsub{ME(Jury)}\\  
            The jury not believe.\textsc{3sg} that be.\textsc{sub}.\textsc{3sg} innocent\\% target language line
                   \glt `The jury doesn't believe that s/he is innocent.'\\
        \ex\label{ex:bove:6b} 
            \gll  El jurat    no creu 	  [que és     innocent.]\capsub{ME(Speaker)}\\
            The jury not believe.\textsc{3sg} that be.\textsc{ind}.\textsc{3sg} innocent\\
                   \glt `The jury doesn't believe that s/he is innocent.'\\
\end{xlist}
\end{exe}


Both \ref{ex:bove:6a} and \ref{ex:bove:6b} contain epistemic predicates, and negation induces a shift in anchor that triggers different moods. According to \citet{quer1998}, the use of the subjunctive in \ref{ex:bove:6a} shows that the proposition \textit{sigui innocent }`she is innocent' is evaluated in the model M\textsubscript{E}(Jury). In \ref{ex:bove:6b} the individual anchor shifts to the model of the speaker M\textsubscript{E}(Speaker). In Yucatec Spanish, the subjunctive is permitted when the referent of the matrix clause (the believer) and the referent of the embedded clause (the NP of the belief utterance) are coreferential, which cannot be accounted for under Quer's theory. Additionally, \citet{quer1998,Quer2001} posits that the shift in individual models is only available under negation, which is not the case for Yucatec Spanish \citep{Bove2020}. It appears that mood selection in doxastic utterances in this contact variety pattern more closely with Italian \textit{credere}. For example, \citet[23]{Homer2007} presents the following triplet:



\begin{exe} % declares the example environment
    \ex\label{ex:bove:7} 
    \begin{xlist} % declares a list of sub-examples
        \ex \label{ex:bove:7a} 
            \gll  Io non conosco l' amica di Gianni, Maria. Lui crede che lei sia/??è incinta.\\  
           I NEG know the friend of Gianni Maria he believes that she be.\textsc{subj}/\textsc{ind} pregnant\\% target language line
                   \glt `I don't know Gianni's Friend Maria. He believes that she is pregnant.'\\
        \ex\label{ex:bove:7b}
            \gll  Sono content che finalmente io e lui siamo d'accordo: Gianni crede che Maria ?sia/è incinta.\\
           I-am glad that finally I and he be.\textsc{subj} agreed Gianni believes that Maria be.\textsc{subj}/\textsc{ind} pregnant\\
                   \glt `I'm glad that he and me finally agree: Gianni believes that Maria is pregnant.'\\
          \ex\label{ex:bove:7c}
                   \gll Gianni deve essere impazzito! Crede che Maria sia/??è incinta.\\
                   Gianni must be crazy he-believes that Maria be.\textsc{subj}/\textsc{ind} pregnant \\
                   `Gianni must have lost his mind! He believes that Mary is pregnant.'
\end{xlist}
\end{exe}
\newpage

In \ref{ex:bove:7a}, the speaker lacks doxastic certainty, which the speaker makes clear by stating that he does not know Mary. In this case, Homer doubts the acceptability of the use of the subjunctive. Contrastingly, in \ref{ex:bove:7b} it is clear that the speaker and Gianni both believe that Maria is pregnant. In such a situation, Homer argues that indicative is the most acceptable mood selection. Lastly, in context where the speaker strongly disagrees with the believer, as seen in \ref{ex:bove:7c}, there is a preference for the subjunctive, which Homer attributes to the lack of agreement between the speaker and Gianni. In the following section, we will see similarities between Yucatec Spanish and Italian in cases similar to \ref{ex:bove:7a}--\ref{ex:bove:7b}. However, cases such as \ref{ex:bove:7c} cannot be accounted for Homer's proposal when analyzing Yucatec Spanish, which will be presented in the next section. As \ neither theory proposed in \ref{ex:bove:6} or \ref{ex:bove:7} can fully explain mood alternation in Yucatec Spanish, I turn instead to a framework that can account for mood alternation in both contexts. 



The notion of veridicality was first discussed by \citet{Montague1969} and has been primarily shaped by \citet{Zwarts1995} and Giannakidou (1994, 2002, 2006, 2011).\todo{Bibliography entries for 1994, 2011 missing} Giannakidou's (\citeyear{Giannakidou1998,Giannakidou2002,Giannakidou2009,Giannakidou2015}) framework relies on a three-part distinction: veridicality, anti-veridicality, and nonveridicality. According to \citet[17]{Giannakidou2015}, a proposition is veridical if, when evaluated in an epistemic state (which is a set of possible worlds) all worlds in M(i) are \textit{p}{}-worlds, which communicates full commitment to the proposition \textit{p}. A proposition is anti-veridical if all worlds in M(i) are $\neg$\textit{p} worlds, which communicates a counter-commitment to \textit{p}. Lastly, a proposition is nonveridical if there is at least one world in M(i) that is a $\neg$\textit{p} world, which communicates a weakened commitment to \textit{p}. The individual model M\textsubscript{E}(x) is essential for interpretation and is similar to the anchor incorporated in several modal theories (i.e. \citealt{Farkas1992,quer1998}). The proposition is evaluated in this epistemic model, which represents what is true to the speaker \citep{Giannakidou2015}:

\begin{exe}
    \ex\label{ex:bove:8}
   \textbf{ Epistemic state of an individual anchor i}: An epistemic state M(i) is a set of worlds associated with an individual i representing worlds compatible with what i knows or believes.
\end{exe}


The proposition must obligatorily be true in both the epistemic model of the speaker and the actual world. 



Embedding a proposition under a belief predicate adds an element of uncertainty within the matrix clause. An utterance containing a doxastic predicate can be an assertion of what the speaker knows to be true, or the speaker may be reporting a belief belonging to the referent denoted by the NP in the matrix clause. The following is an example from Greek (in 9a) and its truth conditions (in 9b) \citep[21]{Giannakidou2015}:




\begin{exe} % declares the example environment
    \ex\label{ex:bove:9} 
    \begin{xlist} % declares a list of sub-examples
        \ex \label{ex:bove:9a}
            \gll  O Nicholas pistevi oti efije i Ariadne.\\  
           the Nicholas believe.\textsc{3sg} that-\textsc{ind} left-\textsc{3sg} the Ariadne\\% target language line
                   \glt `Nicholas believes that Ariadne left.'\\
               \ex\label{ex:bove:9b}
               ⟦Nicholas believes that Ariadne left⟧  M(Nicholas) =1\\
               
iff ${\forall}$w [w ${\in}$ M(Nicholas) $\rightarrow$ w ${\in}$ \{w'. Ariadne left in w'\}]
               
\end{xlist}
\end{exe}




Belief predicates like \ref{ex:bove:9} have two possible anchors: the speaker and the referent denoted by the NP in the main clause. It is not necessary that the proposition be true in both models for the utterance to be felicitous. Formally, \citet[57]{GiannakidouMari2020} define verdicality as the following:

\begin{exe}
    \ex\label{ex:bove:10}
   \textit{Subjective veridicality:}
     \begin{xlist}
         \ex  A function F that takes a sentence  ${\emptyset}$ as its argument -- where  ${\emptyset}$ stands for p or $\neg$p -- is subjectively veridical with respect to an individual anchor
i and an epistemic state M(i) iff F  ${\emptyset}$ is homogenous.
\ex M(i) is homogenous iff ${\forall}$w' [w' ${\in}$ M(i) $\rightarrow$ p(w')], or ${\forall}$w' [w' ${\in}$ M(i) $\rightarrow$ $\neg$p(w')].
     \end{xlist}
 \end{exe}
  


Like the idea proposed in Giannakidou and Mari (2015)\todo{Bibliography entry missing, did you mean 2016?}, this definition highlights importance of the subjectivity of (non) veridicality as the truth of a proposition may be true in one individual's epistemic model and false in another's. A proposition is veridical if and only if the belief space is homogeneously comprised of only worlds in which \textit{p} is true or only worlds in which \textit{p} is false. Unlike a veridical proposition, a subjectively nonveridical proposition is true in some possible worlds and false in others, making the epistemic model nonhomogeneous. \citet[61]{GiannakidouMari2020} define subjective nonveridicality as the following:

\begin{exe}
    \ex\label{ex:bove:11}
   \textit{Subjective nonveridicality:}\\
     A function F that takes a proposition p as its argument is subjectively nonveridical with respect to an individual anchor i and an epistemic state 
M(i) iff F p is not homogenous, i.e., iff ${\exists}$w' [w' ${\in}$ M(i) p(w') $\wedge$ ${\exists}$w'' ${\in}$ M(i) $\neg$p(w')].
 \end{exe}
  

This (lack of) homogeneity of the epistemic model (Giannakidou 2013)\todo{Bibliography entry missing} is the key distinction between veridical and nonverididal evaluations. A nonveridical proposition does not entail \textit{p }and is therefore epistemically weaker than a veridical proposition. While some propositions can be objectively nonveridical (i.e. a modal that implies both \textit{p }and $\neg$\textit{p}), the notion of subjective (non)veridicality accounts for differing evaluations of a proposition based on an individual's epistemic model.



Crosslinguistically, the affirmative is not always licensed in the complements of affirmative belief matrix clauses. Namely, Italian belief predicates select both indicative and subjunctive moods (see Quer 2002,\todo{Bibliography entry missing, did you mean Quer 2001?} \citealt{Portner1997,Mari2016}), reflecting a speaker's degree of commitment. \citet{Giannakidou2015} and \citet{Mari2016} argue that mood selection reflects this epistemic certainty (or lack thereof), which \citet[12]{Giannakidou2015} defines as the following:

\begin{exe}
\ex\label{ex:bove:12}
 ⟦Epistemic Subjunctive⟧\textsuperscript{c,w,g} = $\lambda$q{\textless}st{\textgreater}. ${\cap}$f(w) ${\cap}$q(w') is not ${\emptyset}$;\\
 ${\cap}$f(w) is a nonveridical epistemic modal base because not all worlds are p worlds.
\end{exe}



According to this definition, the epistemic subjunctive denotes a set of possible worlds that includes worlds in which \textit{p} is true and worlds in which \textit{p} is untrue. As such, the use of the subjunctive indicates epistemic weakening.


\section{
Yucatec Spanish}

As \citet{Homer2007} notes, not all epistemic predicates are equal. Homer, like others, argues that there is a certainty scale among doxastic predicates. Predicates like `to be certain', which communicate certainty usually lack the scalar implicature of `I am unsure' that predicates like \textit{believe} trigger. In many cases, this lack of uncertainty would license the indicative. Nevertheless, through an experimental approach, \citet{bove2020a} notes both a difference mood selection in predicates as well as structures. When asked about the predicate of certainty \textit{estar seguro }`to be sure', bilingual speakers of Yucatec Spanish preferred the subjunctive 25\% of the time in affirmative statements (\textit{estoy segura que x }`I am sure that x'\textit{)}, 60\% in interrogatives (\textit{¿está segura que x?} `is she sure that x?'), and 26.1\% in utterances with past temporal reference (\textit{estaba segura que x }`she was sure that x'). In many other varieties of Spanish, subjunctive would not be selected for any of these structures. This suggests that scalar implicature of the predicate is not the only factor in mood selection in this contact variety.



While previous accounts of Yucatec Spanish have noted the use of the subjunctive in epistemics such as \textit{estar seguro }`to be sure' \citep{bove2020a} and the role of interrogatives and possibility adverbs \citep{Bove2020}, it is important to note that the full range of epistemics merits future investigation. However, the current analysis focuses on the doxastic predicates \textit{creer }`to believe' and \textit{pensar }`to think'. According to \citet[133]{GiannakidouMari2020}, there are two possible patterns for doxastic predicates: Doxastics may select only the indicative (as seen in Greek and most Romance languages) or doxastics may `flexibly allow both moods' (as seen in Italian, some cases Portuguese, and in this case of Yucatec Spanish). In the formalization proposed, I argue that Yucatec Spanish, like Italian, has a flexible mood selection.



As previous accounts for Spanish mood selection are not able to fully explain patterns in this contact variety, I use data from Yucatec Spanish in this section to demonstrate that in this variety, mood selection is based on the evaluation of the embedded proposition. When presented with contextualized utterances in which the Yucatec Spanish speaker is uncertain of the truth of a proposition, s/he prefers the use of the subjunctive. Consider the following examples:

\begin{exe} % declares the example environment
    \ex\label{ex:bove:13} 
 Mi hermano me contó de su amigo. Nunca aprueba los exámenes, pero ayer, aprobó el examen con un 9.5 sin estudiar la información que estaba en el examen.
 \glt `My brother told me about his friend. He never passes his exams, but yesterday, he passed his exam with a 9.5 without studying the information that was on the exam.'
 
    \begin{xlist} % declares a list of sub-examples
        \ex \label{ex:bove:13a}
            Él cree que su amigo es (\textsc{ind}) tramposo.% target language line
                   \glt `He thinks that his friend is a cheater.'
        \ex\label{ex:bove:13b}
           Él cree que su amigo sea (\textsc{subj}) tramposo.
             \glt `He thinks that his friend is a cheater.'
\end{xlist}
\end{exe}

\begin{exe} % declares the example environment
    \ex\label{ex:bove:14} 
  Le pregunto a mi amiga si van a visitar a sus padres este sábado. Me dice que ella sí va pero su esposo no está seguro.
 \glt `I ask my friend if she and her husband are going to visit her parents this Saturday. She tells me that she is going to go, but her husband is not sure.'
 
    \begin{xlist} % declares a list of sub-examples
        \ex \label{ex:bove:14a}
            Ella no cree que él va (\textsc{ind}) este fin de semana.% target language line
                   \glt `She doesn’t believe that he is going this weekend.'
        \ex\label{ex:bove:14b}
           Ella no cree que él vaya (\textsc{subj}) este fin de semana.
             \glt `She doesn’t believe that he is going this weekend.'
\end{xlist}
\end{exe}


The purpose of contextualization is to control the epistemic model of each Yucatec Spanish consultant. When presented with these utterances, all consultants preferred the subjunctive for both despite the fact that \ref{ex:bove:13} is affirmative and \ref{ex:bove:14} is negated. This selection reflects a nonveridical interpretation in which the epistemic model contains worlds in which \textit{p }is true and those in which \textit{p} is false. There are several notable similarities between these sentences: Both have third-person matrices which indicates evaluation in two epistemic models, and both have a context that does not allow the speaker to fully evaluate the truth of the embedded proposition. In these cases, the use of the subjunctive communicates a non-homogeneous belief space that includes both worlds in which \textit{p} is true (\textit{p }worlds) and worlds in which p is false ($\neg$\textit{p }worlds).



A similar evaluation can be seen in the following affirmative example that is contextualized by uncertainty:

\begin{exe} % declares the example environment
    \ex\label{ex:bove:15} 
  Tengo que mover el sofá, y le pregunto a mi vecina si su esposo me puede ayudar. Me dice que ahora no porque no está en casa, pero\dots
 \glt `I have to move the sofa, and I ask my neighbor if her husband can help me. She tells me not now because he’s not home, but\dots'
    \begin{xlist} % declares a list of sub-examples
        \ex \label{ex:bove:15a}
           Ella cree que él está (\textsc{ind}) disponible esta tarde.% target language line
                   \glt `She thinks that he is available this afternoon.'
        \ex\label{ex:bove:15b}
           Ella cree que él esté (\textsc{subj}) disponible esta tarde.
             \glt `She thinks that he is available this afternoon.'
\end{xlist}
\end{exe}




Neither the speaker nor the referent denoted by the NP of the matrix clause is certain of the husband's schedule, clearly encouraging a nonveridical interpretation. While previous accounts of Spanish predict that an affirmative predicate will license the indicative in the subordinate clause, participants preferred the use of the subjunctive in \ref{ex:bove:15b} over the indicative in \ref{ex:bove:15a}. When asked about the use of the subjunctive in this statement, participants commented on the uncertainty of the husband's schedule as the motivation for mood selection. Like the previous examples, the speaker's epistemic uncertainty licenses the use of the subjunctive.



As seen in the previous examples, a lack of certainty in items presented during elicitations with the Yucatec Spanish consultants encourages a nonveridical interpretation that licenses the subjunctive. In such cases, the uncertainty is triggered by the presupposition of the embedded clause. However, there are several cases in which consultants accepted both the subjunctive and indicative comparably but noted slightly different interpretations. This is similar to \citeauthor{Mari2016}'s (\citeyear{Mari2016}) account of Italian subjunctive in which the subjunctive can communicate both doxastic certainty as well as epistemic uncertainty. The next two examples of Yucatec Spanish are presented with a context in which there was information missing from the common ground, and the participants accommodated for this:

\begin{exe} % declares the example environment
    \ex\label{ex:bove:16} 
 Mi amiga me contó ayer de sus problemas con su esposo. Su comportamiento es sospechoso—dice que siempre sale sin decirle adónde va por mucho tiempo.
 \glt `My friend told me yesterday about problems with her husband. His behavior is very suspicious -- she says that he always goes out for a long time without telling her where he goes.'
 
    \begin{xlist} % declares a list of sub-examples
        \ex \label{ex:bove:16a}
           Ella piensa que él tiene (\textsc{ind}) amante.% target language line
                   \glt `She thinks that he has a lover.'
        \ex\label{ex:bove:16b}
           Ella piensa que él tenga (\textsc{subj}) amante.
             \glt `She thinks that he has a lover.'
\end{xlist}
\end{exe}


In these cases, Yucatec Spanish consultants accepted both the indicative in \ref{ex:bove:16a} and the subjunctive in \ref{ex:bove:16b}. When discussing the situation with individual participants, it was clear that the evaluation of the proposition could shift when more propositions were added to the context set. If it was made clear that the speaker knew her friend's husband had a lover by adding more information to the context set, mood preference shifted toward the indicative. If the speaker maintained that the friend did not know if her husband was a cheater, subjunctive acceptance remained high. Thus, the subjunctive selection reflects this epistemic uncertainty.



Based on these previous acceptability judgements, I propose that it is the speaker's epistemic model that is used to evaluate veridicality and therefore mood is selected. In the discussion of affirmative doxastic predicates, I presented the importance of context in interpretations involving truth value. Further, I argue that the same is also true for negated belief predicates. Consider the following example:

\begin{exe} % declares the example environment
    \ex\label{ex:bove:17} 
	Alguien nos robó anoche, y la policía encontró a un joven en la calle corriendo afuera de nuestra casa. Sin embargo, una vecina dice que no puede ser él porque lo vio media hora antes en el centro.
 \glt `Someone robbed us last night, and the police found a young guy in the street running outside of our house. Nevertheless, a neighbor says that it couldn’t be him because she saw him half an hour before downtown.'
 
    \begin{xlist} % declares a list of sub-examples
        \ex \label{ex:bove:17a}
          Ella no cree que él sea (\textsc{subj}) culpable.% target language line
                   \glt `She doesn’t believe that he is guilty.'
        \ex\label{ex:bove:17b}
           Ella no cree que él es (\textsc{ind}) culpable.
             \glt `She doesn’t believe that he is guilty.'
\end{xlist}
\end{exe}



Like example \ref{ex:bove:16} above, in utterance \ref{ex:bove:17}, consultants accepted both the indicative (in \ref{ex:bove:17b}) and the subjunctive (in \ref{ex:bove:17a}). According to \citet{Mari2016}, negated doxastic utterances have a dubitative flavor rather than a belief of $\neg$\textit{ p}, resulting in the interpretation of she doubts that he is guilty. 


Saurland (2008)\todo{Bibliography entry missing} argues that there is an implicated presupposition that the truth of \textit{p} is not a common ground. Then, the following utterances were added to the common ground to see how each affected mood selection:


\begin{exe} % declares the example environment
    \ex\label{ex:bove:18} 
    \begin{xlist} % declares a list of sub-examples
        \ex \label{ex:bove:18a}
           Puede creer eso, pero en realidad no fue así.  % target language line
                   \glt `She can believe that, but in reality, it wasn’t like that.'
        \ex\label{ex:bove:18b}
            Puede creer eso, y de hecho tiene razón. Lo vi también muy lejos de la casa.
                   \glt `She can believe that, and actually, she’s right. I, too, saw him far away from the house.'
\end{xlist}
\end{exe}


When presented with the additional information in \ref{ex:bove:18}, consultant evaluations demonstrated a high preference for the indicative. The utterance in \ref{ex:bove:18a} provides evidence that the proposition is false, and \ref{ex:bove:18b} provides evidence that the proposition is true. The change towards preference for the indicative communicates that the belief space has been updated, and now homogeneously consists of $\neg$\textit{p} or \textit{p }worlds, respectively. The change in the epistemic model demonstrated by acceptability judgments in \ref{ex:bove:17} and \ref{ex:bove:18} suggests that with additional information added to the context set, participant interpretations can also shift. 



The speaker's epistemic model, which reflects their perspective of the actual world and real world knowledge, helps contribute and/or eliminate possible worlds from the context set. In the following context, there is some uncertainty communicated, but mood choice reflects the speaker's epistemic knowledge of the actual world:


\begin{exe} % declares the example environment
    \ex\label{ex:bove:19} 
   Mis amigas quieren saber si nuestra amiga, Sara, ya ha regresado de vacaciones. No me ha mandado ningún mensaje así que\dots
    \glt `My friends want to know if our friend Sara has come back from vacation. She hasn’t sent me any messages, so\dots'
    \begin{xlist} % declares a list of sub-examples
        \ex \label{ex:bove:19a}
           No pienso que Sara esté (\textsc{subj}) en Valladolid de nuevo.  % target language line
                   \glt `I don’t think that she is in Valladolid again.'
        \ex\label{ex:bove:19b}
            No pienso que Sara está (\textsc{ind}) en Valladolid de nuevo.
                   \glt `I don’t think that she is in Valladolid again.'
\end{xlist}
\end{exe}


The presented context does not provide evidence to allow for the interpretation of the veracity of the proposition \textit{no está en Valladolid de nuevo }`she is not in Valladolid again'. As such, all participants accepted the use of the subjunctive in \ref{ex:bove:19a}, but most of the participants also accepted the use of the indicative in \ref{ex:bove:19b}. During elicitations, all participants pointed to the epistemic uncertainty to explain why the subjunctive in \ref{ex:bove:19a} was accepted, but when asked why indicative was also acceptable, they cited real their perceptions of the actual world. Many stated that it could be said that the lack of contact between friends meant that she had not returned. For those who rejected the indicative, the uncertain context only allowed the selection of the subjunctive. When asked about the perceived real world knowledge (for example, if she was in town she would call), the subjunctive only group did not believe these ideas to be true in their perceived actual world. Therefore, it can be concluded that, for these speakers, this real world knowledge was not a part of their epistemic model. 



Overall, if a context allows for several possible outcomes in Yucatec Spanish, speakers use this epistemic uncertainty to motivate their subjunctive selection; in contrast, if a context set provides sufficient information for the speaker to interpret a proposition as veridical or anti-veridical, the epistemic certainty licenses the selection of the indicative, indicating that the speaker commits to the truth or falsity of \textit{p}. The epistemic model, comprised of context and perceived knowledge of the real world, is a key component to the individual model by which a speaker can evaluate the truth of a proposition, and thus, it plays a significant role in mood selection in Yucatec Spanish. In situations that present a clear context and, therefore, a clear evaluation of the proposition as veridical, there is little alternation in mood selection. However, when the context set does not provide sufficient information for a speaker to evaluate the truth of a given proposition, both the subjunctive and indicative are accepted. This suggests that epistemic certainty plays a significant role in mood selection in Yucatec Spanish. 

\section{Formalization of mood in Yucatec Spanish}

Thus far, I have presented data that suggest that mood selection in Yucatec Spanish is motivated by speaker interpretation of a proposition: If a speaker interprets a proposition as true, she is epistemically certain of \textit{p}. In such cases, the epistemic model contains only \textit{p }worlds\textit{ }and therefore selects the indicative. On the other hand, epistemic uncertainty resulting from an epistemic model with both \textit{p} and $\neg$\textit{p} worlds licenses the subjunctive. In this section, I adopt a Montagovian (1974)\todo{Bibliography entry missing} truth conditional composition framework to create a formalization for mood selection in this variety of Spanish. A compositional semantics for a natural language translates natural language expressions into (unambiguous) logical language expressions. Following \citet{Giannakidou2002,Giannakidou2006,Giannakidou2015} I argue that each proposition is anchored to an epistemic model to be evaluated, for which I follow Giannakidou's (\citeyear[1889]{Giannakidou2009}):

\begin{exe}
    \ex\label{ex:bove:20}
     \textbf{Epistemic state of an individual anchor i:} an epistemic state M(\textit{i}) is a set of worlds associated with an individual \textit{i} representing worlds compatible with what i knows or believes. 
\end{exe}



As doxastic predicates do not entail event realization, a speaker can report another individual's beliefs that she knows to be false. For this reason, each belief statement has two important factors: doxastic and epistemic certainty. This is similar to \citeauthor{Homer2007}'s (\citeyear{Homer2007}) idea of \textit{Strength of Belief }and \textit{Speaker Congruence}. The first doxastic element, or strength of belief, is expressed through the matrix verb. Following \citet{AnandHacquard2014} and \citet{Mari2016}, I differentiate between doxastic predicates that assert a proposition and those that report a belief. The primary distinction between the meanings is the speaker's ability to calculate the truth value of the proposition embedded under the belief verb. An individual can express a doxastic opinion such as \textit{I believe in Santa Claus}, which cannot be evaluated as true or false; it expresses a doxastic perspective. Contrastingly, an epistemic doxastic such as \textit{I believe it is raining }allows a speaker to communicate how she perceives the world, and this doxastic can be evaluated as true or false. When reporting the beliefs of another person, the speaker can communicate perceptions regarding the truth of the proposition, but she does not necessarily report the truth of the proposition. 

The second factor to consider when defining \textit{creer }is similar to \citeauthor{Homer2007}'s (\citeyear{Homer2007}) idea of speaker congruence in that the speaker is also able to communicate epistemic certainty. While Italian´s (lack of) speaker congruence contributes to mood selection, in the case of Yucatec Spanish, the speaker´s interpretation of epistemic (un)certainty is what licenses mood. Therefore, I propose that a belief statement has two important models of evaluation: (1) the model of the subject of the matrix that introduces the belief and (2) the model of the speaker which is reflected in the mood selected in the embedded clause. I adopt the following lexical entry for the doxastic predicate \textit{creer }`to believe'.

\begin{exe}
    \ex\label{ex:bove:21}
     ⟦\textsc{creer}⟧\textsuperscript{c,w,g} =  $\lambda$p$\lambda$x[believe' (x,w) ${\wedge}$ ${\forall}$w'(w'${\in}$M(x) $\rightarrow$ p(w')) ${\wedge}$ \textsc{reflect} (M(i) ${\wedge}$ (p(w')))]
\end{exe}



The meaning of \textit{creer} is the set of propositions \textit{p }that the subject (x) believes to be true at w, and that for all worlds w', if they are compatible with the epistemic model of the subject, then the set of propositions \textit{p }are true in w'. This first model is doxastic and is able to report if a proposition is in the doxastic space of the NP of the matrix clause (x believes p, x does not believe p). In the second model, the speaker is able to communicate her own epistemic certainty that results from an epistemic model M(i) that is comprised of \textit{p} or $\neg$\textit{p} worlds. 



This second model (of the speaker) is what triggers mood in doxastic statements. The \textsc{Reflect} operator has the following mood selection potential on a proposition \textit{p}: a speaker can withhold commitment to the truth of \textit{p }and communicate epistemic uncertainty which in turn licenses mood. If the speaker evaluates a proposition \textit{p} as veridical or anti-veridical, the indicative will be selected in the embedded clause. If she evaluates the proposition as nonveridical, the subjunctive will be selected. This operator is based on \citeauthor{AnandHacquard2014}'s (\citeyear[78]{AnandHacquard2014}) \textsc{assert }operator that gives the issues \textit{p} priority, directing conversation toward \textit{p} and adding it to all future common grounds. \textsc{Reflect }can be defined as the following:

\begin{exe}
    \ex\label{ex:bove:22}
    \textsc{Reflect }(\textit{i, p, w}), where ${\forall}$w: w ${\in}$ M(i)
\end{exe}



While the definition of \textit{creer}, as seen in \ref{ex:bove:25} accounts for doxastic commitment, this operator allows for the communication of speaker congruence through mood selection. It is important to note that the \textsc{Reflect} operator is the trigger for mood selection. In Yucatec Spanish, this \textsc{Reflect }operator is crucial for interpretation of all \textit{creer} utterances as this is the element that licenses mood. In other varieties of Spanish that allow for pragmatic mood alternation under negation, I argue that this \textsc{Reflect }operator is also present. However, Spanish and other languages that do not allow this alternation in affirmative utterances, the doxastic strength entails the epistemic certainty and therefore the \textsc{Reflect }operator is not needed to communicate speaker congruence. In the following section, I present the formalization for \textit{\ Juan cree que su amigo es }(\textsc{ind})/ \textit{sea }(\textsc{subj}) \textit{tramposo }`Juan thinks that his friend is a cheater'. As both the indicative and the subjunctive were felicitous, I present the formalizations for each in the following sections.



\textbf{4.1} \textsc{Veridical interpretations}. A speaker interprets a proposition as veridical if and only if the epistemic model (M(i)) contains only worlds in which the proposition is true. I define the indicative as the following:


\begin{exe}
    \ex\label{ex:bove:23}
    ⟦\textsc{indicative}⟧\textsuperscript{c,w,g}: $\lambda$p$\lambda $w'[p(w')]
\end{exe}


The meaning of the indicative states that, given a context \textit{c}, a world \textit{w}, and the assignment function \textit{g}, lambdas introduce sets: $\lambda$p is a set of propositions p and $\lambda$w' is a set of worlds. In the formula, the set of p are true in the set of worlds w'. The definition of the indicative is a function from propositions to a function from worlds to truth values {\textless}t',{\textless}w,t{\textgreater}{\textgreater}. Indicative takes something of type t' and returns something of type {\textless}w,t{\textgreater}; and then {\textless}w,t{\textgreater}  is the semantic type of what \textit{creer} then takes as its argument. If correct, these truth conditions support the previous prediction that the indicative is selected when the epistemic model of the speaker is homogeneously comprised of only worlds in which \textit{p} is true. 


\begin{exe}
    \ex\label{ex:bove:24}
        Juan cree\capsub{M(x)} que su amigo es tramposo.\\
    ⟦Juan cree\capsub{M(x)} que su amigo es tramposo⟧\textsuperscript{c,w,g} = 1 iff = [believe' (j,w) ${\wedge}$ ${\forall}$w'(w'${\in}$ M(j) $\rightarrow$
be.cheater' (a) (w')) ${\wedge}$ \textsc{reflect}(M(i) ${\wedge}$ be.cheater' (a) (w'))]
\end{exe}



The \textsc{reflect} operator indicates that, in the model of the speaker, all worlds are such that \textit{a} is a cheater. This demonstrates that the predictions made regarding veridical analyses are accurate. 


\textbf{4.2 }\textsc{Nonveridical interpretations.} When \textit{creer} embeds a proposition that has a nonveridical interpretation, the speaker communicates her epistemic uncertainty. The predictions posit that if a speaker selects the subjunctive, the speaker has evaluated the proposition as nonveridical. This indicates that, within the model of the speaker, there are worlds in which the proposition is true and worlds in which it is false. In \ref{ex:bove:25}, I introduce the translation and meaning of the subjunctive in Yucatec Spanish:


\begin{exe}
    \ex\label{ex:bove:25}
  
    ⟦\textsc{Subjunctive}⟧\textsuperscript{c,w,g} = $\lambda$p
    $\lambda$w[${\exists}$w' [p(w') ${\wedge}$ ${\exists}$w'' ($\neg$p(w''))]]
\end{exe}




The subjunctive means that, given context \textit{c, }world \textit{w, }and assignment function \textit{g, }there is a set of propositions and a set of worlds \textit{w}, compatible with the actual world, such that there exists a world in which \textit{p} is true and there exists a world in which \textit{p} is false. If the predictions made are correct, the subjunctive is selected when the speaker evaluates a given proposition as nonveridical, representing a non-homogeneous belief space. 

\begin{exe}
    \ex\label{ex:bove:25}
        Juan cree\capsub{M(x)} que su amigo sea tramposo.\\
    ⟦Juan cree\capsub{M(x)} que su amigo sea tramposo⟧\textsuperscript{c,w,g} = 1 iff =
    [believe' (j,w) ${\wedge}$ ${\forall}$w'(w'${\in}$ M(j) $\rightarrow$ ${\exists}$w'([be.cheater' (a) (w')])) ${\wedge}$ ${\exists}$w''($\neg$($\lambda$w[be.cheater' (a) (w)]) (w'')) ${\wedge}$ \textsc{reflect} (M(i) ${\wedge}$ ([${\exists}$w'([be.cheater' (a) (w')]) ${\wedge}$ ${\exists}$w'' ( $\neg$([be.cheater' (a) (w'')]]))))]
\end{exe}
\todo{Labels for examples 25 and 26 are the same, please disambiguate}



When the indicative was used above, both the subject and speaker evaluated the proposition as true. The subjunctive is introduced, presenting a set of possible worlds including those in which the proposition \textit{su amigo sea tramposo} is true and those in which the proposition is false. Both the epistemic model of the subject (M(x)) and the model of the speaker (M(i)) contain both \textit{p} and $\neg$\textit{p}worlds. Crucially, this model is able to show the epistemic model of two individuals through the use of the \textsc{reflect} operator. While \textit{creer} introduces the evaluation by the argument of the NP (in this case Juan), it is the \textsc{reflect} operator that communicates the evaluation of the speaker. 

\section{Conclusions and implications}

In Yucatec Spanish, mood selection is unique in that the subjunctive is licensed in affirmative doxastic predicates, an environment known to require the use of the indicative in other varieties. To account for this, I argue that the indicative or the subjunctive mood in an utterance containing a matrix doxastic predicate is selected in Yucatec Spanish based on a speaker's evaluation of an embedded proposition, which follow \citeauthor{Giannakidou2015}'s (\citeyear{Giannakidou2015}) notion of (non)veridicality. The current analysis of Yucatec Spanish adopts a new lexical entry for \textit{creer} in which both the model of the speaker and the model of the subject of the matrix clause are represented, but speaker evaluation licenses mood. In cases in which the speaker evaluates the embedded proposition as veridical, both the model of the speaker and the model of the subject are homogeneously comprised of worlds in which \textit{p }is true. On the other hand, when a speaker evaluates a proposition as nonveridical, the speaker can communicate this epistemic uncertainty by selecting the subjunctive in the embedded proposition. 



The logical question arises: is this a result of language contact? I argue that the unique subjunctive use in Yucatec Spanish is not due to transfer of the subjunctive from Yucatec Maya; the subjunctive in Yucatec Maya is not used to communicate doubt and uncertainty as it can be used in Spanish. However, I do think this use of subjunctive may be a transfer of conjecture marking. With regards to epistemicity, speakers of Yucatec Maya are able to use the grammatical reportative marker \textit{bini} to communicate a lack of first-hand knowledge (see \citealt{LeGuen2018} for more information of epistemicity in Yucatec Maya). Therefore, speakers are able to communicate (lack of) doubt through this preverbal modal marker. It is possible that speakers use mood in Spanish to communicate epistemic certainty when an aspect-mood marker would be appropriate in Yucatec Maya. Further research will hopefully be able to answer this question more fully. 



Overall, there are several implications for this research. First, the semantic notion of nonveridicality can explain non-standard mood alternation in this Spanish contact variety. While this theory has not been previously applied to mood analysis in Spanish, it has been applied cross-linguistically, and I argue that it successfully captures mood choice in Yucatec Spanish. The current project analyzes affirmative and negated utterances. It is necessary to note that the scope of the current analysis is limited. If the theory of (non)veridicality can be used to explain the mood patterns of this contact variety, there should also be an extension of the use of the subjunctive in other nonveridical environments (i.e. interrogatives, downward entailing quantifiers, modals). Preliminary work with this using \textit{creer} and \textit{pensar} can be found in \citet{Bove2020}, but work with similar predicates merit future investigation. Additionally, this research demonstrates that non-standard varieties of Spanish can shed light on mood theories. The mood patterns in doxastic predicates suggest potential evidence of contact effects. A logical extension of the current research is to investigate other epistemic predicates including those that express certainty. While factive predicates (i.e. \textit{to know}) would presumably not allow for alternation as the proposition is always evaluated as veridical, other epistemic matrices that express doxastic perspective should be included in future research. 

\printbibliography[heading=subbibliography,notkeyword=this]
\end{document}
