\chapter{Word order}\label{ch:3}

This chapter reviews different accounts of word order which have been proposed in the \ac{PandP} framework. In the \ac{PandP} approach, linguistic universals or common structural features that are found across languages are explained by a set of finite \textit{principles}. On the other hand, linguistic variation can be explained by different parameters, with which a particular language is set. The ultimate goal in the \ac{PandP} framework is to find such principles and parameters that are unique to human language.

As explained in Chapter \ref{ex:1}, the universal approach to \ac{CS} is advocated in this monograph, which views that both monolingual and bilingual grammars are subject to the same grammatical principles. Thus, an account of \ac{OV}-\ac{VO} variation in \ac{CS} will be also based on existing theories and proposals on \ac{OV} and \ac{VO} orders in the linguistics literatures. 

To begin with, I provide a brief overview of different approaches to word order in the \ac{PandP} theory and support a derivational approach proposed by \citet{Kayne1994}, who argues that both \ac{OV} and \ac{VO} orders share the same underlying \ac{VO} order and \ac{OV} order is derived from \ac{VO} via object movement. This idea will be taken up to describe \ac{OV} order in Korean and Japanese in contrast with \ac{VO} order of English, and \ac{OV}-\ac{VO} contrast between Korean/Japanese and English is argued to be due to different feature specifications on the functional category \textit{v}: while \textit{v} in Korean and Japanese is specified for \ac{EPP}, which triggers object movement, \textit{v} in English lacks \ac{EPP}. Also I propose that the Korean light verb \textit{ha} and Japanese light verb \textit{su} represent the functional category \textit{v}, but English-type light verbs characterize another functional category, Asp(ect), which is projected between \textit{v} and V. The intricate interplay between \textit{v} and A\textsc{sp} will be explained further in the feature inheritance system in Chapter \ref{ch:4}.

\section{Non-derivational approach}\label{ch3:sect:3.1}

In early generative grammar, word order was stipulated according to phrase structure rules. With the advent of the notion of \textit{parameters} \citep{Chomsky1976}, which may vary from one language to another, it was assumed that languages were parameterized according to the directionality of a head: heads may precede or follow its complement (e.g. \citealt{Chomsky1981}, \citealt{Stowell1981}). In English, for instance, all heads normally precede their complements, as exhibited in \REF{ex:43}.

\ea\label{ex:43}\ea
close the door
\ex on the table
\ex an argument with Bibi
\ex curious about Bibi \z
\z

The verbal head \textit{close} precedes its complement \textit{the door} in (\ref{ex:43}a), the prepositional head \textit{on} precedes its complement \textit{the table} in (\ref{ex:43}b), the nominal head \textit{argument} precedes its complement \textit{with Bibi} in (\ref{ex:43}c), and the adjectival head \textit{curious} precedes its complement \textit{about Bibi} in (\ref{ex:43}d). Contrary to the \textit{head-initial} structure in English, Korean shows \textit{head-final} structure, in which all heads uniformly follow their complements and reflect a mirror image of English word order. Korean counterparts to the examples in \REF{ex:43} are provided in \REF{ex:44}.

\ea\label{ex:44}\ea
\gll mwun-ul  tat-ala \\
door-\textsc{acc} close-\textsc{imp} \\
\glt `close the door'
\ex \gll   takca (wi)-ey \\
 table  top-\textsc{loc} \\
\glt `on the table'
\ex \gll   Bibi wa-(uy)   encayng \\
Bibi with-\textsc{gen} argument
\\
\glt `an argument with Bibi'
\ex \gll Bibi eytayhay kwungkumha-ta \\
Bibi about    curious-\textsc{decl} \\
\glt `curious about Bibi'
\z \z

Under the head parameter approach, the \ac{VO} sequence reflects head-initial structure where the verbal head precedes its complement/direct object. The \ac{OV} order, on the contrary, exhibits head-final structure where the verbal head follows its complement. (\ref{ex:45}a) and (\ref{ex:45}b) represent the syntactic structure of \ac{VO} and \ac{OV}, respectively.

\ea\label{ex:45}
\begin{tabular}[t]{llll}
a.   &  
\begin{forest}
[VP[V][OBJ]]
\end{forest} &
b.   &
\begin{forest}
[VP[OBJ][V]]
\end{forest}
\end{tabular}
\z

In a given language, heads may consistently occur either in the initial position or in the final position within a phrase, regardless of their category, as in English and Korean/Japanese, respectively.\footnote{\textrm{Japanese also exhibits head-final structure, similar to Korean.}} However, the positioning of a head may also vary with respect to its complement. For instance, in Chinese, the verbal head precedes its complement, but the nominal head follows its complement \citep{Huang1982}. Similarly, in Dutch and German, verbs canonically follow their complements, but other heads are arguably positioned before their complements in their canonical order (\citealt{Koster1975}). 

One may argue that this is due to the fact that the directionality parameter can be set differently for different heads: in Chinese, for example, the verbal head has the head-initial setting of the parameter, but the nominal head is equipped with the head-final setting. Although the (category-specific) head parameter approach may be descriptively adequate to explain various word order patterns found within a language as well as across languages, it is still problematic. As \citet{Dryer1992} notes, “disharmonic” systems or languages exhibiting a mix of head-initial and head-final orders in fact outnumber harmonic ones or languages with a more rigid word order in the world, thus it raises questions regarding the role of parameter setting in these languages and across languages.

While Dryer’s criticism is concerned with the surface order parameterized by the head parameter, which does not seem to be uniform, the head directionality parameter was identified either as a surface structure condition or a deep structure condition in the Government and Binding theory. At the surface structure level, the directionality parameter was stated over Case assignment. For instance, \citet{Koopman1984} and \citet{Travis1984} argue that Case assignment is directional, which is parameterized differently from language to language. At the deep structure, the directionality parameter was formulated in terms of the directionality of government \citep{Kayne1983} or theta-role assignment (\citealt{Koopman1984,Travis1984}), which was considered to be parameterizable at that time. Under these views, the head-initial vs head-final structure in (\ref{ex:45}a) and (\ref{ex:45}b) can be restated that the verbal head governs/Case-assigns/theta-role assigns the object to the right in the former and to the left in the latter, which is subject to parametric variation. 

Whether the head directionality is parameterized at the surface structure or the deep structure, such parameterization cannot be sustained in modern syntactic frameworks such as minimalist syntax (\citealt{Epsteinetal1996}). With the introduction of the Minimalist Program, the notion of government, which played an essential role in Government and Binding theory, has been abolished, thus the directionality of government is no longer expressible. In addition, Case is no longer viewed to be assigned by a head, but is restated as feature matching between a probe (a functional head) and a goal. Most importantly, the notion of parameters was restricted by \citet{Borer1984} to ``the idiosyncratic properties of lexical items'' where lexical items are equivalent to grammatical elements such as inflection. This idea was endorsed by Chomsky in the Minimalist Program, which \citet{Baker2008} calls \textit{The Borer-Chomsky Conjecture} \REF{ex:BorChomCon}.

\ea\label{ex:BorChomCon} \textbf{The Borer-Chomsky Conjecture} \\
All parameters of variation are attributable to differences in the features of particular items (e.g. the functional heads) in the lexicon. 
\z

In other words, parametric variation is confined to morphosyntactic features of functional categories. As a result, the directionality parameter cannot be stated over theta-role assignment either, since theta-roles are assigned by lexical categories and the old head parameter has been modified by setting parameters on functional heads rather than lexical heads. As we will see in subsequent chapters, the Borer-Chomsky Conjecture in \REF{ex:BorChomCon} will be one of the most important theoretical notions that are adopted in this monograph to explain \ac{OV}-\ac{VO} variation across languages as well as in \ac{CS}. 

\section{Derivational approach}\label{ch3:sect:3.2}

With the postulation of deep structure vs surface structure in generative linguistics, linguists started to postulate the possibility to derive surface structure from its deep structure via transformations in order to explain various sentence types. \citet{Bach1962} proposed the \ac{VO} order of German is derived from \ac{OV} via a transformation. This suggests that \ac{OV} and \ac{VO} orders may share the same underlying structure, which is in this case \ac{OV}. \citet{Bach1968} extended this idea and proposed the Universal Base Hypothesis, which states that all languages have identical deep structures and surface structures are derived via language-specific transformation. Such a transformational approach can explain unexpected/exceptional word orders besides the orders of cross-linguistic tendencies, which the head parameter approach fails to describe: ``Languages are consistent at Deep Structure in having head-initial or head-final characteristics, but transformations may give rise to surface inconsistencies'' \citep[4]{Svenonius2000}. 

\largerpage%longdistance
However, due to poor understanding of the exact nature of the so-called lan\-guage-specific transformations (e.g. What triggers transformations? What constrains them?), the Universal Base Hypothesis faced criticism and was not in vogue for a long time. Yet, transformational grammar has become considerably more restrictive and principled over the years, and the idea that all languages have the same underlying structure has been revived, especially with the advent of the Minimalist Program.

Researchers who adopt a derivational approach to deriving \ac{OV} vs \ac{VO} order are roughly divided into two groups; one who argues that \ac{OV} is derived from \ac{VO} (e.g. \citealt{Kayne1994}) and another who argues that \ac{VO} is derived from \ac{OV} (e.g. \citealt{Haider1992}). There are also a small number of scholars who also take an intermediary position between these two competing views, claiming that surface word order can be base-generated, as the head parameter approach suggests, or one order can be derived from the other (e.g. \citealt{Vicente2004}). However, as discussed earlier, the head parameter approach is no longer formulable on minimalist assumptions due to the fact that it has lost its theoretical foundations, and the hybrid approach combining the head parameter approach and the derivational approach does not provide any substantial advantage over a pure derivational approach. Thus, I will not discuss the hybrid approach here and will review the two competing views on deriving \ac{OV} and \ac{VO} order in the following sub-sections, namely (i) \ac{OV} is derived from \ac{VO} and (ii) \ac{VO} is derived from \ac{OV}.   

\subsection{OV is derived from VO}\label{ch3:sect:3.2.1}
\largerpage%longdistance
In his seminal work, \textit{The Antisymmetry of Syntax}, \citet{Kayne1994} proposes, among other things, that the sequence of Specifier-Head-Complement is the universal order in all languages, which is imposed by \ac{UG}. He argues that: 

\begin{quote}
    If \ac{UG} unfailingly imposes \textsc{s-h-c} order, there cannot be any directionality parameter in the standard sense of the term. The difference between so-called head-initial languages and so-called head-final languages cannot be due to a parametric setting whereby complement positions in the latter type precede their associated heads. Instead, we must think of word order variation in terms of different combinations of movements. \\
    \hspace*{\fill} (\citealt[47]{Kayne1994})
\end{quote}


According to Kayne, \ac{UG} only allows the ({Spec})-Head-Complement sequence underlyingly, and the surface Complement-Head order must be derived from it. This proposal is part and parcel of his \acf{LCA}, which states that asymmetric c-command invariably maps into linear precedence, and word order is determined by hierarchical syntactic structure. Kayne further argues that surface \ac{OV} order is derived from its underlying \ac{VO} structure via object movement; the object raises to the left of the position where the verb ends up (1994: 48). He does not specify where exactly the object raises in the structure, but it is argued to move leftward past the verbal head into some specifier position. On this view, languages exhibiting either \ac{OV} or \ac{VO} as their canonical word order have the universal \ac{VO} structure underlyingly, and \ac{OV} languages (e.g. Korean, Japanese) invariably involve object movement to a position to the left of the verb (\citealt{Kayne2003}).

 However, researchers note that the object may also move from its base position in \ac{VO} languages . In Scandinavian languages, for instance, objects can move clause-internally to a position outside \ac{VP}, which is referred to as \textit{object shift} (\citealt{Holmberg1986}). The relevant examples are provided in \REF{ex:47} and \REF{ex:48} below.

\ea\label{ex:47}
 \glllll nemandinn 	las 	ekki 	bókina {} {} {} {}	Icelandic \\
 studenten   	laeste 	ikke 	bogen {} {} {} {}	Danish\\
 naemingurin	las	ikki  	bókina	{} {} {} {}	Faroese\\
 studenten	läste	inte  	boken {} {} {} {} Swedish \\
student-the	read	not	book-the {~~~~~~} {} {} {} \\
\glt `The student didn't read the book.' 
\ex\label{ex:48} \glllll nemandinn 	las \textbf{hana}$_i$	ekki t$_i$ {} {} {}	Icelandic \\
 studenten   	laeste \textbf{den}$_i$	ikke t$_i$ {} {} {}	Danish\\
 naemingurin	las	\textbf{hana}$_i$ ikki	t$_i$ {} {} {}	Faroese\\
 studenten	läste \textbf{den}$_i$	inte t$_i$ {} {} {} Swedish \\
student-the	read	it not {} {~~~~~~~~} {} {} \\
\glt `The student didn't read the book.'  adopted from \citet{Thrainsson1996}
\z

\largerpage[2]
In \REF{ex:47} the full \ac{NP} object follows the verb and negation.\footnote{The object can appear between the verb and the negation in Icelandic, similar to (\ref{ex:47}a), whereas this is not possible in the other languages.} But when the object is realized as an unstressed definite pronoun as in \REF{ex:48},  it precedes negation (and adverbs) but follows the subject and the verb. It is generally agreed in the literature that the pronominal object in \REF{ex:48}  has moved out of its base position into a position outside the \ac{VP} along with verb movement, and the landing site of the shifted object is argued to be Spec, AgrOP, a specifier position of a functional projection outside the \ac{VP} (\citealt{Deprez1989,JonasBobaljik1993,Ferguson1996,Thrainsson1996}).\footnote{Assuming that a sentential adverb or the negation is adjoined to \ac{VP}, some researchers propose an alternative analysis that the object in object shift constructions in \REF{ex:48} moves to a \ac{VP} adjoined position, as in \REF{ex:fn27} (\citealt{Holmberg1986,HolmbergPlatzack1995}).

\ea\label{ex:fn27}
~[\textsubscript{VP} OBJ\textit{\textsubscript{i}} [\textsubscript{VP} AdvP [ V t\textit{\textsubscript{i}} ]
\zlast}

\clearpage

The derivation in \REF{ex:49} illustrates object shift in \ac{VO} languages.

\ea \label{ex:49}
\begin{forest}
[AgrOP  [OBJ\textsubscript{i}, name = tgt]    
[VP  [AdvP]      
[VP [V] [t$_i$, name = src]]]]
\draw[->,dashed] (src) to[out=south,in=south] (tgt);
\end{forest}
\z     

In \REF{ex:49} the object moves out of its post-verbal base position, whish results in surface \ac{OV} order in \ac{VO} languages. Chomsky initially identifies AgrO (Agreement Object) as a functional category that triggers object shift. If the (Case) feature on AgrO is strong, it triggers object shift, resulting in \ac{OV} order whenever the verb does not raise to a position higher than the landing site of the shifted object. If the (Case) feature of AgrO is weak, the object remains in situ, therefore the underlying \ac{VO} order surfaces. In other words, the \ac{OV}-\ac{VO} distinction results from \textit{overt} vs \textit{covert} object movement due to the strong vs weak feature on a functional head above \ac{VP}. 

Chomsky argues that syntactic movement is a result of feature checking. Only functional categories could arguably host strong features, thus, a strong feature on a functional head triggers overt movement in the derivation, while weak features of a functional head do not trigger overt syntactic operations, thus movement is covert.\footnote{\textrm{(Overt) object movement, (overt) object raising, and object shift are used interchangeably in this monograph.}} According to Chomsky, the term \textit{features} refers to the properties of language that enter into two interface levels, \acf{PF} and \acf{LF}, and the computational system that generates them (2000: 91).\footnote{\textrm{Chomsky adopts Aristotle’s view of language as sound with meaning and argues that I(nternal)-language, which is a hierarchically structured expression (syntax), provides instructions to the thought system (or the Conceptual-Intentional system or semantics) and the sound system (or the Articulatory-Perceptual system or phonology). This is often called the Y-model in the field of generative linguistics.} } 


The motivation for AgrO (along with AgrS(ubject)) was that objects (and subjects) may agree with the finite verb in heavily inflected languages like Xhosa (\ref{ex:50}a) or Quechua (\ref{ex:50}b).

\ea\label{ex:50}\ea
\gll u-mama  u-ya-wu-phek-a  um-ngqsuaho \\
1a-mother  1a-\textsc{pres}-3-cook-\textsc{asp}  3-samp \\
\hfill Xhosa
\glt  ‘Mother cooks samp.’
\ex \gll \textit{pro}  riku-wa-rqa-nki  \textit{pro} \\
{} see-1\textsc{sg}-\textsc{past}-2 {}
\\
\hfill Quechua
\glt `You saw me.' \citep{DenDikken2016}
\z
\z

Chomsky later replaced AgrO with \textit{v} (\citeyear{Chomsky1995,Chomsky1998}).\footnote{\label{fn:30}According to Chomsky, \textit{v} has two sub-types,\textit{v}  and \textit{v}*. While \textit{v} heads intransitive constructions and does not assign (accusative) Case, \textit{v}* \textrm{heads transitive constructions and assigns (accusative) Case. Under this view, it is} \textit{v}* \textrm{, not} \textrm{\textit{v}}\textrm{, which has a strong feature and triggers object shift. I will not distinguish} \textrm{\textit{v}} \textrm{and} \textrm{\textit{v}}\textrm{\textsuperscript{*}} \textrm{in this monograph for the sake of simplicity.}} In addition, strong vs weak features are formulated in different terms, as is the presence vs the absence of the \ac{EPP} feature; while a functional category with the \ac{EPP} feature triggers an overt syntactic operation, a functional category lacking the \ac{EPP} feature does not do so. The term \textit{\ac{EPP}} stands for the \textit{\acl{EPP}}, which originally demanded simply that a clause must have a subject \citep{Chomsky1982}. Since the nineties, generative linguists have extensively subscribed to the view that subjects originate as Specifiers of \textit{v}P/\acp{VP} (\ac{VP} internal subject hypothesis). Under this view, the \ac{EPP} requires that the subject that is base-generated at Spec, \textit{v}P raises to Spec, \ac{TP}. Chomsky reformulates the \ac{EPP} as a morphological property of T with a  strong (D-) feature, which forces Spec, \ac{TP} to be lexicalized by raising an element. The \ac{EPP} feature was considered as an independent feature on T, triggering syntactic movement of a phrase to its specifier position. However, the application of the \ac{EPP} feature has been extended, and Chomsky started using the term \textit{\ac{EPP} feature} to refer the property of a functional head that triggers overt syntactic movement to its specifier position in general.\footnote{There have been attempts to eliminate the \ac{EPP} (e.g. \citealt{Boskovic2002,GrohmannEtAl2000}), yet the \ac{EPP} is widely assumed and in practice in current generative syntax.}

Although the structure in \REF{ex:49} was originally proposed for object shift in \ac{VO} languages, we can extend this analysis to \ac{OV} languages as well, such as Korean and Japanese. Following Kayne’s derivarational approach that surface \ac{OV} order is derived from its underlying \ac{VO} order in \ac{OV} languages, it is likely that the mechanism that is responsible for object shift in \ac{VO} languages is similar to, or even quite possibly identical to the mechanism that is responsible for the derivation of \ac{OV} order from \ac{VO} order in \ac{OV} languages; what is common is that the object leaves its base-generated position and raises to a position higher than the verb both in object shift in \ac{VO} languages and \ac{OV} languages. Thus, one can say that the landing site for the derived object proposed for object shift in \REF{ex:49} is also a possible landing site for the moved object for \ac{OV} languages, which is endorsed by \citet{Kayne2003} and many others.\footnote{\citet{Ochi2009}, for instance, proposes that the \ac{DP} object overtly raises to Spec, \textit{v}P from its underlying position inside the \ac{VP} in Japanese.} On this assumption, the contrast between \ac{OV} languages (e.g. Korean and Japanese) and \ac{VO} languages (e.g. English) can be also explained on the hypothesis that while the \ac{EPP} feature on \textit{v} triggers object raising in \ac{OV} languages, in \ac{VO} languages, the \ac{EPP} feature is absent on \textit{v}.\footnote{In
    all Scandinavian languages, indefinite quantified negative objects move to a pre-verbal position, which \citet{Christensen2004} calls \textsc{neg}-shift, showing that \ac{VO} languages may also exhibit \ac{OV} order on the surface.
    \ea\label{ex:fn33}
    \glll a. {*} jeg har faktisk [\textsubscript{NegP} {} [\textsubscript{\textit{v}P} set \uline{ingenting} ]] {\ldots} {\ldots og det har du heller ikke}.\\
          b. {} jeg har faktisk [\textsubscript{NegP}  \uline{ingenting}$_i$  [\textsubscript{\textit{v}P}  set t$_i$ ]] {\ldots} {\ldots and that have you neither not}\\
          {} {} ~I have actually {} nothing {} seen {}  {}\\\hfill Danish\vspace*{-\baselineskip}

    \glt `I haven't actually seen anything and neither have you.'  (\citealt[1, (1)]{Christensen2004})
    \z

    The fact that \ac{VO} languages may also exhibit \ac{OV} order as in Scandinavian languages shown above seems to suggest that the \ac{EPP} property on \textit{v} may not be entirely absent in all \ac{VO} languages, especially under the assumption that the mechanisms responsible for object shift in \ac{OV} languages and \ac{VO} languages are alike or even identical, as I assumed in this monograph. Under this approach, the key difference of object movement between \ac{OV} and \ac{VO} languages is that the object moves along with verb movement in \ac{VO} languages. However, movement of the bare quantified object in \REF{ex:fn33} differs from object shift, which depends on verb movement, and the object is generally assumed to move to Spec, NegP (\citealt{Haegeman1995,HaegemanZanuttini1991,Jonsson1996,Kayne1998,Platzack1998,Rognvaldsson1987,Sells2000,Svenonius2002}). \citet{Christensen2004} argues that this movement is driven by the \ac{EPP} of an uninterpretable feature [\textit{u}Quant] on C (more precisely, the Fin head) via Spec, \textit{v}P as an escape hatch. Taking these lines of thought together, it is reasonable to assume that the mechanisms responsible for object shift in \ac{OV} and \ac{VO} languages are not identical and the \ac{EPP} on \textit{v} is absent in \ac{VO} languages altogether and object shift is triggered by something else in \ac{VO} languages. I will leave this for future research.
} The tree structures in \REF{ex:51} illustrate overt vs covert object movement, triggered by the [+\ac{EPP}] vs [$-$\ac{EPP}] features on \textit{v}. One cautionary note made here is that $\pm$ features indicate the presence and the absence of the features, which does not imply that features are binary.
\newpage

\ea\label{ex:51}\adjustbox{width=0.9\textwidth}{
\begin{tabular}[t]{llll}
a. & \textsc{ov} languages  & b. & \textsc{vo} languages  \\ && \\
   & \begin{forest}
  [\textit{v}P,s sep=12mm [OBJ$_i$,name=tgt]
  [\textit{v}’,s sep=2mm [\textit{v}\\{[+\textsc{epp}]}]
  [VP [V][t$_i$,name=src]]]]
  \draw[->, dashed] (src) to[in=south,out=south west] (tgt);
  \end{forest}
&   & \begin{forest}
   [\textit{v}P, s sep=6mm [\textit{v}\\{[-\textsc{epp}]}]
   [VP[V][OBJ]]]
   \end{forest}
\end{tabular} }
\z

In this monograph, I adopt the original proposal on \textit{v} by Chomsky in the Minimalist Program, in which  \textit{v} is regarded as a Case-checking/assigning light verb.\footnote{This view is also shared by \citet{HaleKeyser1993}, who consider that light verbs are \textit{i} and lexical verbs are V.} I assume that \textit{v} is one of possible functional categories that represent a light verb and besides \textit{v}, other verbal functional categories can or may correspond to light verbs. Also the precise syntax of light verbs differs across languages (\citealt{Adger2003,Butt2003}). Based on this, I propose that Korean/Japanese light verbs and English light verbs represent different functional categories in the structure, and their syntax also differs, as described in \REF{ex:52} and \REF{ex:53}.

\ea\label{ex:52}
\ea The Korean and Japanese light verbs \textit{ha} and \textit{su} in the [bare verbal noun + \textit{ha/su}] construction lexicalize the functional category \textit{v}.
\ex  In English, \textit{v} is never overtly lexicalized (cf \citealt[351]{Chomsky1995}). Instead, English light verbs represent a different functional category, A\textsc{spect}, which is projected between \textit{v} and V.\footnote{The presence of the functional category \ac{ASP} between \textit{v} and V was also proposed by \citet{Richardson2003} for Russian and by \citet{Travis2000,Travis2010} as a general \ac{VP} structure across languages.}
\z
\ex\label{ex:53}
 \begin{tabular}[t]{ll}
 a. & Korean and Japanese  \\
   & \begin{forest}
  [\textit{v}P [SUB] 
  [\textit{v}$'$ [\textit{v}\textsuperscript{[+\textsc{epp}]} \\ \textit{ha}\textsuperscript{KR}/\textit{su}\textsuperscript{JP}]
  [A\textsc{sp}P [A\textsc{sp} \\ $\varnothing$]
  [VP [V][OBJ] ]]] ]
  \end{forest}
  \\
b. & English \\
& \begin{forest}
  [\textit{v}P [SUB] 
  [\textit{v}$'$ [\textit{v}\textsuperscript{[-\textsc{epp}]} \\ $\varnothing$ ]
  [A\textsc{sp}P [A\textsc{sp} \\ LV\textsuperscript{ENG}]
  [VP [V][OBJ] ]]] ]
  \end{forest}  
 \end{tabular}
\z

In (\ref{ex:53}a) the Korean light verb \textit{ha} and the Japanese light verb \textit{su} lexicalize \textit{v}, which takes the \ac{ASP} as its complement in which the underlying order is \textit{ha}-V-O in Korean and \textit{su}-V-O in Japanese. However, surface order is O-V-\textit{ha} in Korean and O-V-\textit{su} in Japanese, as indicated in \REF{ex:54}. The surface \ac{OV}-\textit{ha}/\textit{su} order in Korean and Japanese is derived from the underlying \textit{ha}/\textit{su}-\ac{VO} order via two steps: (i) First, the object raises to Spec, \ac{ASP}, which results in \ac{OV} order within \ac{ASP}, as in \REF{ex:55}. After that, (ii) the entire \ac{ASP} raises to Spec, \textit{v}P, yielding \ac{OV}-\textit{ha}/\textit{su} order, as in \REF{ex:56}. Both of these movements, object movement and \ac{ASP} raising, are a consequence of feature specifications on \textit{v} and valuation of these features in a derivation, which will be discussed in detail in Chapter \ref{ch:4}. 

\ea\label{ex:54}
    \ea{
    \gll  ~~nemwu manh-un   ton-ul     sopi   hayss-eyo \\
    ~~too   much-\textsc{rel} money-\textsc{acc} spend \textsc{do}.\textsc{past}{}-\textsc{decl} \\} \hfill Korean
    \glt ~~`He spent too much money.'
    \ex[*]{\gll  hayss-eyo nemwu manh-un   ton-ul sopi    \\
    \textsc{do}.\textsc{past}{}-\textsc{decl} too much-\textsc{rel} money-\textsc{acc} spend  \\} 
    \ex{  \gll   ~~ichi-mon daisu-no   mondai-o       saiten site   \\
    ~~one-\textsc{clf}   algebra-\textsc{gen} question-\textsc{acc} mark   \textsc{do} \\} \hfill Japanese
    \glt ~~`You mark one algebra question.'
     \ex[*]{\gll  site ichi-mon daisu-no   mondai-o       saiten    \\
     \textsc{do} one-\textsc{clf}  algebra-\textsc{gen} question-\textsc{acc} mark \\}
    \z
\ex\label{ex:55}
\begin{forest}
[\textit{v}P, s sep = 45mm [\textit{v} {=} \textit{ha}\textsuperscript{KR}/\textit{su}\textsuperscript{JP}]
[A\textsc{sp}P [OBJ$_i$, name=tgt]
[A\textsc{sp}P [A\textsc{sp} \\ $\varnothing$]
[VP [V][t$_i$,name=src]]]]]
\draw[->,dashed] (src) to [in=south,out=south west] (tgt);
\end{forest}
\ex\label{ex:56}
\begin{forest}
[\textit{v}P 
[A\textsc{sp}P$_k$,name=tgt [\textbf{OBJ}$_i$]
[A\textsc{sp}P [A\textsc{sp} \\ $\varnothing$]
[VP [V][t$_i$]]]]
[\textit{v}$'$[\textit{v} {=} \textit{\textbf{ha}}\textsuperscript{KR}/\textit{\textbf{su}}\textsuperscript{JP}][t$_k$,name=src]]
]
\draw[->,dashed] (src) to [in=south east,out=south west] (tgt);
\end{forest}
\z

Before we discuss the structure of English light verbs in (\ref{ex:53}b) where light verbs exemplify the functional category \ac{ASP}, let me say a few words about aspect. The term \textit{aspect} here refers to the properties of the event-structural organization of a verb phrase, and various terms, such \textit{lexical aspect}, \textit{semantic aspect}, \textit{situational aspect}, \textit{inner aspect}, \textit{event structure}, \textit{Aktionsart}, have been proposed in the literature, referring to \citegen{Vendler1957} classification of verb types: states, activities, achievements, and accomplishments.\footnote{This is distinguished from \textit{grammatical aspect}, which has also been referred to as \textit{viewpoint aspect} or \textit{outer aspect} or \textit{morphological aspect}.} In the past, the event structure of a verb phrase was typically considered to belong to semantics, not syntax. However, as \citet[18]{TennyPustejovsky2000} point out, event structure has been directly encoded in syntax as well with recent developments in syntactic theory, especially with the articulation of extended \ac{VP}s and functional projections. Although researchers have different opinions about where an aspect node is projected in the syntax, such as above \textit{v}P/\ac{VP} \citep{Borer1998}, on \textit{v}P/AgrOP (\citealt{RitterRosen2000}), or between \textit{v}P and \ac{VP} (\citealt{Richardson2003,Travis2000,Travis2010}; cf \citealt{Ramchand2008}), it is generally agreed that an aspect node is a functional category of the extended verbal projection.

The presence of the functional category \ac{ASP} between \textit{v} and V, as proposed in \REF{ex:53}, was also suggested by \citet{Richardson2003} for Russian and by \citet{Travis2000,Travis2010} as a general \ac{VP} structure across languages. Richardson assumes that \ac{VP}s in Russian per se are not an aspectual domain and argues that with the projection of \ac{ASP} above \ac{VP}, the event structure of the \ac{VP} is calculated.\footnote{\textrm{However, Richardson also argues that there are inherently telic verbs in Russian, and A}\textrm{\textsc{sp}}\textrm{P can sometimes merge in a derivation with a telicity feature whose value is not set, therefore having uninterpretable aspectual feature (2003: 59).}} Richardson’s event structure is illustrated in \REF{ex:57} below. 

\ea\label{ex:57}
\begin{forest}
[\textit{v}P [~~~]
[A\textsc{sp}P [~~~] [VP]]{ \draw (.east) node[right]{ $\rightarrow$ Semeantic aspect/event structure}; }
    ] { \draw (.east) node[right]{ $\rightarrow$ Light verb phrase}; }
\end{forest}
\z

\citet{Travis2000} also provides morphological and syntactic evidence from languages like Tagalog and Navajo, where an aspectual morpheme may appear between the two verbs (V\textsubscript{1} and V\textsubscript{2}) in reduplication, and proposes the layered \ac{VP} structure in \REF{ex:58}.

\ea\label{ex:58}
\begin{forest}
[V$_1$P [SUB]
[V$_1'$ [V$_1$]
[A\textsc{sp}P [~~~]
[A\textsc{sp}' [A\textsc{sp}]
[V$_2$P [OBJ]
[V$_2'$ [V$_2$][PP]]]]]]]
\end{forest}
\z

\largerpage[-1]
Travis assumes the \ac{VP} structure is layered as in \citet{Larson1988}, in which the direct object merges at [Spec, \ac{VP}] while the indirect object that is headed by a preposition appears as the complement of the verb in double object constructions. Within the layered \ac{VP} in \REF{ex:58}, there is a functional category \ac{ASP} between V\textsubscript{1} and V\textsubscript{2}. Though V\textsubscript{1} seems to correspond to \textit{v}, Travis claims that both V\textsubscript{1} and V\textsubscript{2} are lexical categories, following the more restricted distinction between lexical and functional categories; only lexical categories introduce arguments (\citealt{Abney1987}). Travis states that “V\textsubscript{1,} although lexical, is closer to a light verb” (\citeyear[12]{Travis2000}).

Returning to the structure in (\ref{ex:53}b), now I proceed to explain why English-type light verbs are analyzed as lexical roots corresponding to the functional category \ac{ASP} rather than lexicalizing the verbal head V in \ac{VP}, which was discussed in \citet{DenDikkenShim2011}. Aktionsart or the event structure of a \ac{VP} is not an inherent property of a verb but is normally determined jointly by the verbal head and its complement (\citealt{Dowty1979,Tenny1994,VanVoorst1988,Verkuyl1972,Verkuyl1993}). For instance, while \textit{he ate an apple} is categorized as an accomplishment in Vendler’s term, having both an initial and an end point, \textit{he ate apples}, with a bare-plural object, is categorized as an activity, which is not bounded terminally. However, in a light verb construction with the light verb \textit{take}, for instance, the aspectual properties of the verb phrase are not decided by the light verb and its complement combined. Instead the aspectual constitution of the light verb + complement combination is the same as that of the corresponding ‘simple’ verb construction where the light verb’s complement is used as a verb. In \REF{ex:59}, for example, the object of \textit{take} is systematically an indefinite singular \ac{NP} (e.g. \textit{a look}, \textit{a walk}, \textit{a bath}, \textit{a decision}), but the aspectual properties of the \ac{VP} are not constant. In (\ref{ex:59}a-c) \textit{take}-\acp{LVC} (e.g. \textit{take a look}, \textit{take a walk}, \textit{take a bath}) denote activities or atelic, which is compatible with a durative temporal \textit{for}{}-phrase. On the other hand, \textit{take} \textit{a} \textit{decision} in (\ref{ex:59}d) indicates accomplishment or telic, thus incompatible with a \textit{for}-phrase, and an \textit{in}-phrase is an appropriate time-frame adverbial. Such different aspectual properties among \textit{take}-\acp{LVC}, (\ref{ex:59}a-c) vs (\ref{ex:59}s), are in fact in concert with the aspectual class of their corresponding simple verb constructions as shown in \REF{ex:60}: the \ac{VP}s in (\ref{ex:60}a-c) are atelic whereas the \ac{VP} in (\ref{ex:60}d) is telic. 
  
\ea\label{ex:59} 
    \ea I took a look at it for/*in two seconds.
    \ex I took a walk for/*in half an hour.
    \ex I took a bath for/*in an hour.
    \ex I took a decision in/*for one minute.
    \z

\ex \label{ex:60}
    \ea  I looked at it for/*in two seconds.
    \ex I walked for/*in half an hour.
    \ex I bathed for/*in an hour.
    \ex I decided in/*for one minute.
    \z
\z

\largerpage[-1]
What is particularly interesting here is the fact that the verb \textit{walk} in (\ref{ex:60}b) can be made compatible with an \textit{in}-phrase by adding an event-delimiting \ac{PP} in the complement of \textit{walk} such as \textit{around the block} as in (\ref{ex:61}a). Likewise, the effect of the \ac{PP} is the same both in the corresponding in the light verb case, as a comparison of (\ref{ex:61}a) and (\ref{ex:61}b) shows.\footnote{\textrm{It seems that the the aspectual properties may not be the same between (\ref{ex:61}a) and (\ref{ex:61}b) if the tense is changed into future:} \textrm{(i)  I will walk around the block in five minutes}\textrm{(ii)  I will talk a walk around the block in five minutes} While both (i) and (ii) can mean the speaker will begin to walk around the block in five minutes, there is an additional ‘telic’ meaning in (i), in which the event of walking will be terminated in five minutes. While this was not shared by all four native speakers of English that I consulted, the speaker who suggested this possibility mentioned that in order to get this additional reading, there should be a prosodic emphasis in speech, which indicates that the informational structure may not be the same. While I maintain that English light verbs do not contribute to the event structure, some researchers have argued that light verbs may contribute to the event structure based on languages other than English such as Hindi (\citealt{Mohanan2006}). }

\newpage
\ea\label{ex:61}
\ea I walked around the block in five minutes.
\ex  I took a walk around the block in five minutes
\z
\z

We see that the aspect properties of \textit{take} light verb constructions are entirely a function of the aspect properties of the nominal and its complement. The light verb itself takes no controlling part in this. On the assumption that Aktionsart is determined compositionally by material contained in the lexical projection of the predicate head, this informs us that in light verb constructions, the light verb is not contained in the lexical projection of the predicate head.\footnote{\textrm{I consider \ac{VP} to be the domain of aspect computation by assuming that \ac{VP} has [Asp] feature. This assumption will play an important role in Chapter 4 where the \textit{v} -A}\textrm{\textsc{sp}} \textrm{structure in Korean and Japanese is developed in great detail under the feature inheritance system.}} In effect, it tells us that the light verb itself is not the predicate head: instead the noun heading the light verb’s complement is. But of course the light verb does have a close relationship with that noun: \textit{take} selects \textit{a} \textit{walk}. So the light verb must be merged directly with the projection of the noun.\footnote{\textrm{This idea will be} \textrm{reflected in the tree structure later where \ac{ASP}} \textrm{lexicalized by an English light verb directly takes the \ac{DP} object phrase as its complement without there being the projection of \ac{VP} in (\ref{ex:67}).}} This is guaranteed if the light verb realizes \ac{ASP}, the functional head that takes the lexical projection of the predicate head as its complement. The light verb itself does not participate in the determination of the aspectual properties. Rather, it lexicalizes the functional head by whose complement these properties are determined. It is this on this basis that I propose that English-type light verbs lexicalize \ac{ASP}.

\subsection{VO is derived from OV}\label{ch3:sect:3.2.2}


While the view that \ac{VO} is the underlying order for \ac{OV} (à la Kayne) has prevailed in generative syntax, a small number of researchers proposed an opposite view that \ac{VO} order is derived from \ac{OV} order, based on an observation of peculiar properties about word orders across languages. Cross-linguistic data show that not only the direct object but also other verbal complements precede the verb in \ac{OV} languages while they follow the verb in \ac{VO} languages, as presented in \REF{ex:62}. In (\ref{ex:62}a), the direct object \textit{Joa}, the indirect object \textit{a present} and the prepositional phrase \textit{to her house} all follow the verb \textit{sent} in a \ac{VO} language like English whereas they all precede the verb in \ac{OV} languages such as in Korean (\ref{ex:62}b) or Japanese (\ref{ex:62}c).

\ea\label{ex:62}\ea Bibi sent Joa a present to her house.
\ex \gll  Bibi-ka     Joa-eykey senmwul-ul cip-ulo      ponay-ess-ta \\
Bibi-\textsc{nom} Joa-\textsc{dat}  present-\textsc{acc} house-\textsc{loc} send-\textsc{past-decl} \\
\newpage
\ex \gll Bibi-ka     Joa-ni     purezento-o ie-ni   okuta-ta \\
Bibi-\textsc{nom} Joa-\textsc{dat} present-\textsc{acc}  house-\textsc{loc} send-\textsc{past} \\
\glt  ‘Bibi sent Joa a present to (her) house.’
\z \z

The placement of verbal complements with respect to the verb in \REF{ex:62} can be predicted by Kayne's theory: it is a result of raising verbal complements out of \ac{VP} in the case of \ac{OV} languages while such movement does not happen in \ac{VO} languages. In addition to Korean and Japanese, which are typologically unrelated to English, Germanic \ac{OV} languages such as German and Dutch also support Kayne’s prediction that verbal complements precede the verb in \ac{OV} languages, as in \REF{ex:63} and \REF{ex:64}, respectively. 

\ea\label{ex:63} \gll \ldots daß  sie    jedem       ein Paket     an seine Privatadresse  schicken werden \\ 
\ldots that they everybody a    package to his     home.address send        will \\ \hfill German
\glt `\ldots that they will send everybody a package to his home address.’  \citep{Haider1992}
\ex\label{ex:64} 
    \ea \gll \ldots~  dat  Jan   het  boek aan Marie gaf \\
\ldots~ that John the book to Mary gave \\ \hfill Dutch
\glt `\ldots~ that John gave the book to Mary.'
    \ex \gll \ldots~  dat  Jan   de doos op de tafel zette \\
\ldots~ that John the box on the table put \\ 
\glt `\ldots~ that John put the book on the table.'   \citep{Barbiers2000}
    \z
\z

The fact that all verbal complements precede the verb in various \ac{OV} languages can be explained as a result of multiple application of raising to the left to the verb in Kayne’s approach.\footnote{Marcel den Dikken (p.c.) points out that the \ac{PP} in (\ref{ex:64}b) can optionally surface to the right of the verb in Dutch (and to some extent in German as well). Also \ac{CP} complements generally must appear to the right of the verb in Dutch and German. The fact that \ac{PP} complements and \ac{CP} complements surface to the right of the verb in Dutch and German can be accounted for by the object shift-based analysis: \ac{PP}s and \acp{CP} do not have a Case feature to check against AgrO and hence are expected to stay in situ.} However, \citet{Haider1992,Haider2000} addresses peculiar properties found in German which challenge Kayne’s analysis. \citet{Haider1992} observed that the linear order of preverbal arguments in German is the same as that of postverbal arguments in English; in \REF{ex:63} the complements of the verb are ordered with respect to one another in the following order both in German and English: \acf{IO} - \acf{DO} - \acf{PP}.\footnote{The linear order of \ac{IO}-\ac{DO}-\ac{PP} is also observed in Korean and Japanese in (\ref{ex:62}b, c) and Dutch in \REF{ex:64}. In addition, \citet[183]{Barbiers2000} shows that in double object constructions, the only allowed order is \ac{IO}-\ac{DO} both in Dutch and English. 

\ea
\gllll  a.  {} dat  Jan  {\longrule}  Marie  het boek  gaf. \\
        a.$'$ {} that  John  gave  Mary  the book    {\longrule}  \\
        b.  * dat  Jan  {\longrule}  het boek    Marie  gaf \\
        b$'$. * that  John  gave  the book   Mary \\
\zlast
}

If we assume that the ordering of the arguments inside the \ac{VP} in English is the underlying structure, as \citet{Kayne1994} claims, one must provide further explanation of how this initial order must be preserved after a series of movement operations in German, which is lacking in his proposal. There are proposals in the literature to explain order preservation along with object shift in Scandinavian languages where the initial order inside the \ac{VP} is preserved after the object raises. \citet{FoxPesetsky2005}, for instance, propose a mapping mechanism between syntax and phonology, which determines the linear ordering of words. Such linearization is restricted by two constraints; (a) the relative ordering of words is fixed at the end of each phase (or spell-out domain) and (b) ordering established in an earlier phase may not be revised or contradicted in a later phase. According to their proposal, the fact that the initial order V-\ac{IO}-\ac{DO}-\ac{PP} within the \ac{VP} is preserved can be explained by a combination of object shift and \ac{VP}-remnant movement. Although their proposal accounts for order preservation along with object shift in Scandinavian languages where the verb still precedes the object after the object raises as exemplified in \REF{ex:47}, it fails to describe order preservation in \ac{OV} languages where the initial order starts as \ac{VO} but ends up in \ac{OV}.

To account for the same linear order effect in German and English, Haider proposes as an alternative view that \ac{VO} is derived from \ac{OV} via V head movement, which keeps the underlying order of the verbal arguments intact after transformation \REF{ex:65}. 

\ea\label{ex:65}
    \ea  ~[ IO [ DO [ PP V ]]]
    \ex ~[ V\textit{\textsubscript{i}} [ IO t\textit{\textsubscript{i$''$}} [ DO t\textit{\textsubscript{i$'$}} [ PP t\textit{\textsubscript{i}} ]]]]
    \z
\z

The structure in (\ref{ex:65}a) is the head-final structure where V takes all of its complements to the left.\footnote{\citet{Haider1992} allows V to take its \ac{PP} argument to the right under his own theory, which does not concern us here.} According to Haider, the head-final structure in (\ref{ex:65}a), which he calls the right-branching structure, is the only structure that \ac{UG} allows: the structural build-up of phrases is universally right branching (Basic Branching Constraint: \citealt{Haider1992,Haider2000}). On this proposal, the head-initial structure is not allowed by \ac{UG} and \ac{VO} order is derived from leftward movement of the verb as shown in (\ref{ex:65}b).\footnote{However, linguists implement different mechanisms to derive \ac{VO} from \ac{OV}, either via head movement (raising V to the left of its complements: \citealt{Barbiers2000,Haider1992,Haider2000}) or phrasal movement (remnant \ac{VP} movement: \citealt{Taraldsen2000}).} What is also different between (\ref{ex:65}a) and (\ref{ex:65}b) is that the \ac{VP} structure in \ac{OV} languages is simpler than that of \ac{VO} languages, which shows a Larsonian \ac{VP} shell structure.  

In the Minimalist Program, \textit{v} (and its variants) is a place where light verbs appear. On the other hand, \textit{v} has no special status in Haider’s proposal; it is one of the V positions in the shell structure of complex head-initial \ac{VP} and \textit{v} is entirely absent in \ac{OV} languages. Thus, it is not clear where a light verb is projected in an \ac{OV} language in (\ref{ex:65}a), which has a simple \ac{VP} structure. Since this monograph aims to investigate the role of light verbs in Korean-English and Japanese-English \ac{CS} in relation to deriving \ac{OV} and \ac{VO} orders, I will not consider Haider’s model where the position of light verbs of \ac{OV} languages is not identified in the syntax.\footnote{\citet{Haider2013} argues that \ac{OV} and \ac{VO} orders are not complementary and there is a third category which is underspecified for directionality (Type III), based on the diachronic Germanic word order split between \ac{OV} and \ac{VO} orders. As reported in Chapter 2, the distribution of \ac{OV} and \ac{VO} orders is not also perfectly complementary in Korean-English and Japanese-English \ac{CS}. It is especially true in light verb constructions with heavy/lexical verbs, which may alternate between \ac{OV} and \ac{VO} orders (Chapter \ref{ch:5}), and Haider’s proposal might be handy to account for this fact. However, as the reasons stated above, I will not follow this direction in this monograph. I thank the anonymous reviewer who referred to this work.}

\section{Minimalist approach}\label{ch3:sect:3.3}
\largerpage%longdistance
Generative linguists agree on the view that one of the most fundamental aspects of human language is its hierarchical structure. Yet, how hierarchical structure built with lexical items is mapped into a linear sequence is still a matter of debate \citep{Barrie2012}. \citet{Kayne1994} proposes that not only the hierarchical structure but also the linear order of combined words is established in syntax, as expressed in the \ac{LCA}: asymmetric c-command invariably maps into linear precedence and word order is determined by hierarchical syntactic structure. In the Minimalist Program, on the other hand, Chomsky proposes the bare phrase structure where structure is built via \textit{Merge}, which takes two lexical items ${\alpha}$ and ${\beta}$ and the linear order between them is unspecified but decided later at the syntax-\ac{PF} (Phonetic Form) interface. In other words, structure is built up hierarchically but actual word order is established after syntax when the lexical items are spelled out/pronounced. Yet, it is not clear how the linearization procedures occurs. Chomsky does not discuss it any further for it is beyond the syntax. 


\largerpage%longdistance
Nonetheless, Chomsky accepts Kayne’s idea of the universal head-complement order (\acs{SVO}) and any deviation from this results from movement (\citeyear{Chomsky1995}: 340). Yet, the core idea of the \ac{LCA} seems to be incompatible with the bare phrase structure. Chomsky writes, ``the bare theory structure lacks much of the structure of the standard X-bar theory that plays a crucial role in Kayne’s analysis'' (\citeyear{Chomsky2015}: 208): there are no bar levels and no distinction between lexical items and heads projected from them (\citeyear{Chomsky2015}: 228). As mentioned earlier, the \ac{LCA} dictates that there exists an inherent asymmetry among lexical items. Under this view, it is not clear how a symmetric structure (between ${\alpha}$ and ${\beta}$) is converted into an ordered string in the bare phrase structure (\citealt{Zwart2011}).\footnote{Zwart proposes that Merge itself yields an ordered pair rather than an unordered set, which is expressed in his \textit{structure-to-order conversion}: The structure-to-order coversion is a correspondence rule, and (i) reads that the two lexical items ${\alpha}$ and ${\beta}$ are ordered as ${\alpha}$-${\beta}$.

\ea  Structure-to-order conversion ~~~~~~~~~~ (\citealt[101]{Zwart2011}) \\
\textless ${\alpha}$, ${\beta}$ \textgreater = /${\alpha}$, ${\beta}$/
\zlast
} 

To accommodate the \ac{LCA} in the bare phrase structure, the \ac{LCA} has been re-analyzed as an operation that applies to \ac{PF} and the violation of the \ac{LCA} (e.g. symmetric relations between two lexical items) must be eliminated before the structure is spelled out at \ac{PF} (\citealt{Chomsky1995,Moro2000}). One way to turn a symmetric relation between two lexical items into an asymmetric relation is to move (or the term \textit{internal merge} in the Minimalist Program) one lexical item, which leaves a trace. The trace of the moved element, which has no phonological value, is ignored/deleted at \ac{PF}, thus does not violate the \ac{LCA}: only a moved element, not the trace of it, is spelled out at \ac{PF}. 


Depending on the branch of Minimalism one chooses to employ, one could choose between these two approaches to Merge: either (a) Merge itself imposes a linear order (as well as the hierarchical order) on the constituents it combines or (b) Merge imposes the hierarchical order but not the linear order \citep{OsborneEtAl2011}. Between these two options, the first one is more compatible with Kayne’s take on deriving word order.

In this monograph, I adopt general assumptions shared in the minimalist work. Although I assume that structure is built via Merge between two lexical items, I will not commit myself to the bare phrase structure in this monograph, but continue to use  tree structure as in the X-bar theory. There are a few reasons to do so. First, it is for expository/notational purposes. Following the minimalist view that the locus of linguistic variation is due to different morphological features of a functional category rather than a lexical category (the Borer-Chomsky Conjecture), it is easy for me to explain if a clear distinction is made between the projection of functional categories (such as C, T, \textit{v}, \ac{ASP}) and that of lexical categories. In addition, if word order/linear order is regarded entirely as the property of \ac{PF}/phonology, as assumed in the bare phrase structure, it is not clear how parametric variations such as word order can be explained in terms of morphological features of functional categories, which should not play a role at \ac{PF}. Yet, the assumption that structure is built via Merge, which takes two lexical items, will be reflected in syntactic trees and this has a consequence in representation by dispensing an unnecessary projection of syntactic categories. For instance, the structure in \REF{ex:66}, which is repeated from (\ref{ex:53}b), where an English light verb directly lexicalizes \ac{ASP}, not V, thus leaving the V head empty, is simplified into the structure in \REF{ex:67} in which the empty V head is not projected. Instead, the light verb and the object merge, which is depicted in the structure where \ac{ASP} takes the \ac{DP}/\ac{NP} object as its complement.
\largerpage

\ea\label{ex:66}
\begin{forest}
  [\textit{v}P [SUB] 
  [\textit{v}$'$ [\textit{v}\textsuperscript{[-\textsc{epp}]} \\ $\varnothing$ ]
  [A\textsc{sp}P [A\textsc{sp} \\ LV\textsuperscript{ENG}]
  [VP [V][OBJ] ]]] ]\end{forest}
 
\ex\label{ex:67} \textit{v}P structure with an English light verb 	~~~~	(final version)	 \\
\begin{forest}
  [\textit{v}P [SUB] 
  [\textit{v}$'$ [\textit{v}\textsuperscript{[-\textsc{epp}]} \\ $\varnothing$ ]
  [A\textsc{sp}P [A\textsc{sp} \\ LV\textsuperscript{ENG}]
  [OBJ] ]]]\end{forest}
\z
\clearpage

But notice that the functional category \textit{v} is empty in \REF{ex:67} and still projected. Likewise, when the functional category \ac{ASP} is null, it will be projected as well, as illustrated in \REF{ex:68}, which represents the underlying \textit{v}P structure with an English heavy verb.

\ea\label{ex:68} \textit{v}P structure with an English heavy verb \\
\begin{forest}
  [\textit{v}P [SUB] 
  [\textit{v}$'$ [\textit{v}\textsuperscript{[-\textsc{epp}]} \\ $\varnothing$ ]
  [A\textsc{sp}P [A\textsc{sp} \\$\varnothing$]
  [VP [ HV\textsuperscript{ENG}][OBJ] ]]] ]\end{forest}
\z

In Chapter \ref{ch:4}, the special status of these two functional categories, \textit{v} and A\textsc{sp,} will be discussed in detail in the phase theory where \textit{v} is defined as a phase head and features are passed down from \textit{v} to \ac{ASP} via feature inheritance. And it will be argued that how features are specified on \textit{v} and valued via feature inheritance will lead to \ac{OV} and \ac{VO} variation in monolingual and bilingual grammars alike. 

One final comment is in order. Although I take Kayne’s approach that the sequence of Specifier-Head-Complement is the universal order in all languages, I adopt the minimalist take on it: the \ac{LCA} is a constraint at \ac{PF}. This will provide an important set up for the next chapter, which introduces \textit{\acl{FI}}. 

\subsection{Chapter summary and conclusion}\label{ch3:sect:3.4}

This chapter provided a short overview of different approaches to word order, with particular focus on \ac{OV} and \ac{VO} orders. After a close examination of different approaches to \ac{OV} and \ac{VO} orders, I have adopted Kayne’s proposal that both \ac{OV} and \ac{VO} languages have the same underlying \ac{VO} order and \ac{OV} order is derived from \ac{VO} by object movement to the left of the verb.

I have also proposed the syntax of light verbs in Korean, Japanese, and English, where Korean light verb \textit{ha} and the Japanese light verb \textit{su} represent the functional category \textit{v} whereas English light verbs exemplify the functional category \ac{ASP} as illustrated below. 

\ea\label{ex:69}
 \begin{tabular}[t]{ll}
 a. & Korean and Japanese  \\
   & \begin{forest}
  [\textit{v}P [SUB] 
  [\textit{v}$'$ [\textit{v}\textsuperscript{[+\textsc{epp}]} \\ \textit{ha}\textsuperscript{KR}/\textit{su}\textsuperscript{JP}]
  [A\textsc{sp}P [A\textsc{sp} \\ $\varnothing$]
  [VP [V][OBJ] ]]] ]
  \end{forest}
  \\
b. & English \\
& \begin{forest}
  [\textit{v}P [SUB] 
  [\textit{v}$'$ [\textit{v}\textsuperscript{[-\textsc{epp}]} \\ $\varnothing$ ]
  [A\textsc{sp}P [A\textsc{sp} \\ LV\textsuperscript{ENG}]
  [VP [V][OBJ] ]]] ]
  \end{forest}  
 \end{tabular}
\z


While the underlying \ac{VO} order remains in English (\ref{ex:69}b) in a syntactic derivation, the object raises to the left of the verb, targeting Spec, \ac{ASP} in Korean and Japanese (\ref{ex:69}a), resulting in \ac{OV} order within \ac{ASP}. After the object moves to Spec, \ac{ASP}, the entire \ac{ASP} moves to Spec, \textit{v}P, as a result of which the surface order of O-V-\textit{ha} in Korean and O-V-\textit{su} in Japanese is derived (recall the structures in \REF{ex:55} and \REF{ex:56} for this). I have argued that both object movement and \ac{ASP} raising in Korean and Japanese are due to feature specifications on \textit{v}, which are different from feature specifications on \textit{v} in English. Yet, I have not shown how \textit{v}’s features are different in Korean/Japanese and English, and will discuss this in next chapter.
