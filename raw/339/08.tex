\chapter[Clitics in a corpus of a spoken variety]{Clitics in a corpus of a spoken variety (Bosnian)}
\label{Clitics in a corpus of a spoken variety}
\section{Introduction}

This chapter contains a pilot study on the usage of CLs in a spoken variety of BCS. This topic has so far remained untouched, probably due to the lack of good spoken data. On the basis of a corpus of Bosnian interviews, we study the inventory and, in particular, the internal organisation of the CL cluster and the position of the CL or cluster. First, we are interested in the inventory of CLs and the types of simple and mixed clusters found in this variety. Second, we would like to give a data-driven account of CL placement. In this regard we inspect the heaviness of the constituents preceding the CLs. 

As this chapter is dedicated to spoken language, we annotate the data with respect to syntactic features typical of spoken language which complicate determination of clause boundaries, such as disfluency phenomena, right dislocation, and others. In the final step, we thus look for potential correlation between the position of the CL or CL cluster and these syntactic structures. 

This chapter has the following structure: in Section \ref{State of the art: Clitics in spoken BCS} we start with a short overview of the state of the art concerning CLs in spoken BCS. In Section \ref{RQ:9} we formulate our research questions. Then in Section \ref{Corpus of Bosnian interviews} we describe the analysed Corpus of Bosnian Interviews (for additional information see also Chapter Overview of Corpora available for BCS). The general principles of spoken language analysis are discussed in Section \ref{Principles of analysis of spoken language} This feeds into our data preparation, including our annotation scheme, as presented in Section \ref{Data preparation}. The results of our data-driven study are discussed in the order that we used in Chapters \ref{Our terms and concepts}, \ref{Clitics and variation in grammaticography and related work} and \ref{Clitics in dialects}. We focus only on those parameters of variation for which we had enough data. The inventory of CLs and attested CL clusters is presented in Section \ref{Inventory:9}. Section \ref{Internal organisation of the clitic cluster:9} is devoted to internal organisation of CL clusters, while the positioning of single CLs and CL clusters is the focus of Section \ref{Positioning of single clitics and clitic clusters}. This section is followed by Section \ref{Diaclisis:9} on diaclisis. In Section \ref{Impact} we discuss the impact of certain syntactic structures on CL positioning. The final Section \ref{Summary:9} contains a summary of the findings.

\section{State of the art: Clitics in spoken BCS}
\label{State of the art: Clitics in spoken BCS}
We are aware of the peculiarities of the syntax of spoken language, where intonation plays a crucial role and where syntactic features diverging from written language may be found. Unfortunately, we are confronted with the fact that beyond dialectology (see Section \ref{Available data}), spoken BCS is seriously understudied. What we do find in the literature, however, are some scattered conjectures based on linguists’ intuition. Here it is important to underline that we do not have any theoretical basis specifically for Bosnian; the claims below were made for the Croatian spoken variety. 

Silić was the first to touch upon the differences in CL placement between the spoken and written Croatian varieties. He claimed that the written variety is subject to rhythmic rules, and consequently CLs should follow the first stressed word or be inserted into the first phrase \citep[cf.][391]{Silic78}. This means that phrase splitting is not only a norm, but also natural and expected in the written Croatian variety \citep[cf.][225]{Silic06}. 

In contrast, CL placement in the spoken variety is claimed to be freer than in written language, since the spoken variety is governed by “logical factors” (\citealt[cf.][391]{Silic78}, \citeyear[28]{Silic84}).\footnote{Silić does not define logical factors.} Furthermore, \citet[225]{Silic06} argues that CLs after (heavy) phrases are common and natural in the spoken variety and in some registers which are similar to the spoken variety. \citet[63]{Alexander09} also claims that placement of CLs after (heavy) phrases is within the norm of the spoken Croatian variety. A third indication of differences in CL placement between the spoken and written Croatian varieties can be found in \citet[464]{KedvesWerkmann13}. They cite observations made by teachers in Croatian high schools on allegedly incorrect CL placement as one of the most common mistakes in students’ speech \citep[464]{KedvesWerkmann13}. The teachers’ observations are based on their normativist expectations of CL placement (CLs should follow the first stressed word and are not to be placed after heavy constituents) on the one hand, and on the other hand on the fact that students’ speech (i.e. the spoken variety) obviously does not meet these expectations. 

\section{Research questions}
\label{RQ:9}
Based on the observations on CLs in standard BCS varieties and in dialects presented in Chapters \ref{Clitics and variation in grammaticography and related work} and \ref{Clitics in dialects}, as well as on scattered conjectures about CLs in spoken BCS (Croatian) varieties presented in the section above, in this chapter we address the following research questions:

\begin{enumerate}[label=RQ\arabic*:,ref=RQ\arabic*]
\item\label{8RQ1} Does the inventory attested in the spoken Bosnian variety differ from the inventory in written standard Bosnian (and other standard varieties)? 
\item\label{8RQ2} What is the maximum size of CL clusters in the spoken Bosnian variety? 
\item\label{8RQ3} Does CL ordering in the spoken Bosnian variety differ from the ordering in written standard Bosnian (and other standard varieties)?
\item\label{8RQ4} Do morphonological processes like haplology of unlikes or suppletion take place within clusters in the spoken Bosnian variety?\footnote{For more information on haplology of unlikes see Section \ref{Morphonological processes within the cluster}.} 
\item\label{8RQ5} Which type of placement, 1P, 2P or DP, is dominant in the spoken Bosnian variety? 
\item\label{8RQ6} How heavy are constituents which precede CLs in the spoken Bosnian variety? 
\item\label{8RQ7} Does the spoken Bosnian variety allow phrase splitting and (pseudo)\-di\-a\-cli\-sis?\footnote{For more information on phrase splitting see Section \ref{Phrase splitting (2W)}.}\textsuperscript{,}\footnote{For more information on (pseudo)diaclisis see Section \ref{Diaclisis and pseudodiaclisis}.}
\item\label{8RQ8} Are syntactic features typical of spoken language like disfluency phenomena and right dislocation correlated with CL placement in the spoken Bosnian variety?
\end{enumerate}

\section{Corpus of Bosnian interviews}
\label{Corpus of Bosnian interviews}
\subsection{Data quality}
\label{Data Quality}

Like for dialectological studies, a major obstacle for the study of CLs in spoken BCS varieties is availability and quality of data resources.\footnote{For a discussion of this problem with respect to the study of CLs in BCS dialects see \ref{Available data}.} Namely, as explained in Section \ref{Corpora for spoken BCS}, only very few corpora with data from spoken BCS varieties are available. We analyse the Bosnian Interviews corpus \citep{RaeckeStevanovic01}, which contains 13 narrative interviews conducted with refugees from the territory of Bosnia (for the available sociolinguistic data see the next section) in 1994.\footnote{The reasons for choosing this particular corpus for the study of CLs in a spoken variety can be found in Section \ref{Parameters of variation in spoken BCS}.} The minimal corpus annotation includes only deictics and regional features which mainly include pronunciation, but we introduce some additional layers of annotation (see Sections \ref{Principles of analysis of spoken language} and \ref{Annotation scheme}). 

However, there are two major obstacles affecting our study. The biggest is lack of access to the audio recordings, which would make possible the disambiguation of some unclear parts of the transcript. This is particularly important as the transcription does not meet fully the standards which in recent years have been achieved in the burgeoning field of spoken language study.\footnote{ For an example of a consistent annotation system for spoken German (GAT) see \citet{SABBCGMQSU98} and for spoken Russian see the system proposed by \citet{KibrikPodlesskaya03, KibrikPodlesskaya06} and presented concisely below. } Second, the main problem for our analysis of CL placement is the lack of consistent annotation of breaks which would allow the identification of the intonational units which deviate from clausal units as the main interactional unit. Instead, punctuation signs, in particular commas, are used for segmentation in an unsystematic way. The following examples show how breaks after \textit{u} \textit{stvari} ‘in fact’ were marked with a comma (\ref{(9.X1)}), three dots (\ref{(9.X3)}), hyphen (\ref{(9.X4)}) or were not marked at all (\ref{(9.X2)}).\footnote{The code given in brackets matches the code of speaker in Bosnian Interviews. This is why we keep the round brackets. For details consult Table \ref{T9.1}.}

\begin{exe}\ex\label{(9.X1)} 
\gll [\dots] \minsp{[} u stvari], sad \textbf{je} u {zarobljeništvu  [\dots].} \\
{} {} in matter now be.\textsc{3sg} in captivity \\
\glt ‘[\dots] in fact, he is now in captivity [\dots].’
\hfill (BG1) 

\ex\label{(9.X3)} 
\gll [\dots] \minsp{[} u stvari]... od kad \textbf{su} bili ovi {izbori  [\dots].} \\
{} {} in matter from when be.\textsc{3pl} be.\textsc{ptcp.pl.m} those elections \\
\glt ‘[\dots] in fact \dots since this election was held [\dots].’
\hfill (BG1) 

\ex\label{(9.X4)} 
\gll [\dots] \minsp{[} u stvari] -- obeležavali \textbf{su} naša {vrata  [\dots].} \\
{} {} in matter {} mark.\textsc{ptcp.pl.m} be.\textsc{3pl} our doors \\
\glt ‘[\dots] in fact - they marked our doors [\dots].’
\hfill  (TZ)

\ex\label{(9.X2)} 
\gll [\dots] \minsp{[} u stvari] dva mjeseca \textbf{sam} bio bez {posla [\dots].} \\ 
{} {} in matter two month be.\textsc{1sg} be.\textsc{ptcp.sg.m} without job \\
\glt ‘[\dots]  in fact, I was without a job for two months [\dots].’
\hfill (BG1) 
\end{exe}


\noindent This kind of unsystematic annotation is especially clear in the case of breathing breaks, which, according to the rules of standard BCS orthography, should be represented with commas, but which are missing from the studied corpus.

\subsection{Sociolinguistic features of the corpus}
\label{Sociolinguistic paramters of the corpus}
In all, 16 people were interviewed. However, two of them (second speaker in transcript DJ and second speaker in transcript IL) played a secondary role, since the length of their utterances counted in words is much smaller in comparison to that of the remaining 14 participants. Table \ref{T9.1} summarises the socio-linguistic metadata from the Bosnian Interviews corpus on sex, age, profession or education of the interviewees, and nationality and religious background of their family members (in as much detail as available).

\begin{table}
\caption{Sociolinguistic information about participants.\label{T9.1}}
\begin{tabularx}{\textwidth}{lllQQr}
\lsptoprule
Code  & Sex & Born & Information\footnote{Available information on nationality and religious background.} & Profession or education & Transcript size\footnote{In words spoken only by interviewees.} \\
\midrule
BG1 & M & 1957 & Croat, Catholic & architectural technician&2,271 \\
BG2&M&1940&Croat, Catholic&driver&2,362 \\
BH&F&NA\footnote{Born in Serbia, lived in Bosnia for 20 years.}&atheist, husband Muslim&textile technician&3,279 \\
BJ&F&1950&not declared Muslims&NA&1,970 \\
BL&F&NA&Serbian, Orthodox&NA&2,892 \\
BR&M&1948&atheist, parents Muslims&higher administrative school&3,109 \\
DJ&M&1941&Muslim&mine worker&2,310 \\
&NA&NA&NA&NA&196 \\
DO&F&1962&Muslim&economic technician&2,463 \\
IL1&F&1964&father Orthodox, mother Catholic&pedagogical academy&1,756 \\
IL2&M&1958&parents Muslims&NA&789 \\
KR&F&1949&Croat, Catholic&special needs educator&2,537 \\
MO1&M&1957&atheist Yugoslav parents Croats, Catholics&PhD political sciences&2,013 \\
MO2&F&1959&atheist Yugoslav, parents Serbians, Orthodox&lawyer&919 \\
TZ&F&1973&parents Serbians&medical high school&2,920 \\
VI&F&NA&Orthodox, Yugoslav&NA&6,632 \\
&&&&&$= 38,422$ \\
\lspbottomrule
\end{tabularx}
\end{table}

The transcripts were anonymised, so we cannot draw any conclusions about the exact place of origin and areas inhabited by the interviewees. In other words, establishing the interviewees’ dialectal backgrounds is impossible. 

It is important to understand the political and cultural background of Bosnia within Yugoslavia before the war broke out in 1991. Many people had ancestors and relatives from different Yugoslavian countries and of different religions. Since they all lived in Yugoslavia, many interviewees considered themselves Yugoslavs and atheists, as that was a common political orientation at the time. Nevertheless, upon the interviewer’s request to specify their background in more detail, in most cases they provided the ethnic identity and the religion of their parents. Although, as it is clear from Table \ref{T9.1}, interviewees have different social and ethnic backgrounds, the corpus compilers labelled the variety spoken by them as spoken Bosnian since most of them were born in Bosnia and they all lived there for years before they came to Germany. 

Further, we can see that the group varies with respect to many sociolinguistic factors. The age spread is at least 33 years, and the speakers represent different layers of society, which we can conclude from their professions and education. Thus, the group is heterogeneous with respect to sociolinguistic factors, so there is no such factor which could clearly influence the linguistic results.

\section{Principles of analysis of spoken language}
\label{Principles of analysis of spoken language}
As mentioned above, with the exception of dialectology, research on spoken BCS has developed poorly. Therefore, as will become clear in the following, the principles of analysis in this chapter are based on influential literature from other linguistic areas. 

The first important issue in the analysis of spoken language concerns segmentation. As mentioned in Section \ref{Data Quality}, due to the lack of consistent annotation of breaks, segmentation is based on syntactic criteria. We follow the view of \citet[484]{ThompsonCouper05} that “the clause is in fact the locus of interaction in everyday conversation”. The authors continue: “[t]he clause, then, with its crucial predicate, appears to be a unit which facilitates the monitoring of talk for social actions” \citep[cf.][485]{ThompsonCouper05}. In our annotation scheme, we thus focus on syntactic clauses, but additionally take into consideration some other structures that do not coincide with clauses, as they are characteristic of spoken language. This is necessary to appropriately determine the position of a CL in a clause (see Section \ref{Data preparation}). Our annotation scheme is inspired by an approach which distinguishes types of \textsc{elementary} \textsc{discourse} \textsc{units} (EDUs), presented by the Russian linguists \citep{KibrikPodlesskaya03, KibrikPodlesskaya06}, but has a stronger focus on purely structural features. In their approach, \citet{KibrikPodlesskaya03, KibrikPodlesskaya06} combine formal, semantic and cognitive features which are often difficult to distinguish in our data. Therefore, we also draw on the work by \citet{Crible16, Crible18}, who designed a more consistent annotation system for various disfluency phenomena. 

In the first step of data processing, we split the transcript into syntactic clauses. We focused on those which contain CLs, as illustrated in (\ref{(9.1)}):\footnote{The ekavian pronunciation is most probably due to the fact that this interviewee was born and lived in Serbia for 15 years (see Table \ref{T9.1}).}

\protectedex{\begin{exe}\ex\label{(9.1)}
\gll U Bosni \textbf{sam} živela dvadeset {godina  [\dots].} \\
in Bosnia be.1\textsc{sg} live.\textsc{ptcp}.\textsc{sg}.\textsc{f} twenty years \\
\glt ‘I lived in Bosnia for twenty years  [\dots].’
\hfill (BH)
\end{exe}
}

\noindent The analysis of such clauses does not cause any problems. However, as \citet[38]{Crible16} points out, spoken language in its spontaneous forms is characterised by “the frequent occurrence of so-called disfluencies, which are generally considered to be cues of ongoing processes of language production and comprehension”. In the following typology, types 1--4 represent fluencemes as proposed by \citet{Crible18}. In addition to EDUs defined as syntactic clauses we distinguish further types (see 5--7), based on \citet{KibrikPodlesskaya06}, that are smaller than syntactic clauses and are not necessarily linked to disfluency. Further we added our own types 8--10.


\begin{enumerate}
	\setcounter{enumi}{0}
\item \textsc{identical repetitions} “include any words formally similar to each other and directly contiguous” \citep[73]{Crible18}, as in \ref{(9.2)}, where the predicate \textit{dala} ‘gave’ is repeated:


\protectedex{\begin{exe}\ex\label{(9.2)}
\gll Sada \textbf{je} \minsp{[} dala, dala] treći u izbjeglištvu. \\
now be.3\textsc{sg} {} give.\textsc{ptcp}.\textsc{sg}.\textsc{f} give.\textsc{ptcp}.\textsc{sg}.\textsc{f} third in exile \\
\glt ‘Now she finished, finished the third (grade) in exile.’
\hfill (BG1)
\end{exe}
}

\item \textsc{partial repetitions} are structures in which a word or phrase is broken and repeated in such a way that the broken part could be omitted, resulting in a well-formed unit, as in (\ref{(9.3)}) where the interviewee corrected the broken part with \textit{večera} ‘dinner’. 


\protectedex{\begin{exe}\ex\label{(9.3)}
\gll I napravi \textbf{se} nekakva \minsp{[} veče... večera] a conto {Božića  [\dots].} \\
 and make.3\textsc{prs} \textsc{refl} some {} din... dinner a conto Christmas \\
\glt ‘And some din… dinner is made a conto Christmas  [\dots].’
\hfill (KR)
\end{exe}
}

\item \textsc{false starts} “are interruptions that leave a segment syntactically and/or semantically incomplete and where no elements from the previous, abandoned context are taken up in what follows” \citep[73]{Crible18}. This is illustrated in example (\ref{(9.4)}) in which the interviewee interrupts the segment \textit{niku} and continues with \textit{ovdje} ‘here’.

\protectedex{\begin{exe}\ex\label{(9.4)}
\gll \minsp{[} Nismo niku...] ovdje, dok \textbf{smo} ovdje eto u {Njemačkoj  [\dots].} \\
{} \textsc{neg}.be.3\textsc{pl} anywh... here while be.1\textsc{pl} here well in Germany\\
\glt `We were not anywh… here, while we are here, well, in Germany  [\dots].'
\hfill (VI)
\end{exe}
}

\item \textsc{substitutions} correspond to any segment replaced by another with semantic and/or syntactic modification.\footnote{\citet[74]{Crible18} distinguishes between morphological and propositional substitutions, which we do not consider necessary.} Unlike in repetition, in substitutions the repaired segment does not have to be contiguous. In (\ref{(9.5)}) the interviewee substitutes the segment \textit{ulazila sam} `I was coming in' with the contiguous segment \textit{ušla sam} ‘I came in’.

\begin{exe}\ex\label{(9.5)}
\gll \minsp{[} Ulazila \textbf{sam}, ušla \textbf{sam}] kad \textbf{sam} {htjela  [\dots].}\\
{} coming.in.\textsc{ptcp}.\textsc{sg}.\textsc{f} be.1\textsc{sg} come.in.\textsc{ptcp}.\textsc{sg}.\textsc{f} be.1\textsc{sg} when be.1\textsc{sg} want.\textsc{ptcp}.\textsc{sg}.\textsc{f}\\
\glt `I was coming in, I came in when I wanted  [\dots].'
\hfill (BJ)
\end{exe}

\item \textsc{rendered topic} (left or right dislocation) is a nominal (including prepositional) phrase that precedes or follows the clause and is not syntactically integrated.\footnote{\citet[7]{KibrikPodlesskaya06} do not mention the possibility of a nominal phrase following the clause in their definition, but from our examples it is obvious that such instances can occur.} A reliable feature which helps to qualify a noun phrase as dislocation, and not as an actant or an adjunct in the clause, is the presence of an anaphoric pronoun of the third person which is co-referent with the topic \citep[cf.][7]{KibrikPodlesskaya06}. In example (\ref{(9.6)}) below the anaphoric pronoun \textit{ona} ‘she’ signals that the nominal phrase \textit{komunistička policija} ‘communist police’ is a rendered topic.

\begin{exe}\ex\label{(9.6)}
\gll Jer \minsp{[} komunistička policija], ona \textbf{je} progonila {svakoga  [\dots].} \\
because {} communist police she be.3\textsc{sg} pursue.\textsc{ptcp}.\textsc{sg}.\textsc{f} everyone \\
\glt ‘Because the communist police, it pursued everyone  [\dots].’
\hfill (BG1)
\end{exe}

\item \begin{sloppypar}\textsc{retrospective edus} (afterthoughts) are units smaller than syntactic clauses which follow the clause or phrase they refer to. In contrast to \citet{KibrikPodlesskaya06} we do not distinguish any subtypes depending on different cognitive processes \citep[cf.][10]{KibrikPodlesskaya06}. Unlike in dislocation, in retrospective EDUs there is no anaphoric pronoun which is co-referent with the element in the afterthought. In example (\ref{(9.7)}) the afterthought noun phrase \textit{praznik rada} ‘Labour Day’ follows \textit{prvi maj} ‘first of May’.\end{sloppypar}

\begin{exe}\ex\label{(9.7)}
\gll Prvi maj, \minsp{[} praznik rada], \textbf{se} proslavljao, na {primjer  [\dots].} \\
first May {} holiday work \textsc{refl} celebrate.\textsc{ptcp}.\textsc{sg}.\textsc{m} on example \\
\glt ‘May First, Labour Day, was celebrated, for example  [\dots].’
\hfill (BR)
\end{exe}

\item \textsc{Discourse structuring elements} (DSEs) can consist of single words or phrases.\footnote{The term ``regulatory EDU'' for this type of small EDU originated with \citet[63ff]{Chafe94} and was adapted by \citet[12f]{KibrikPodlesskaya06}. However, we prefer the term “discourse structuring element” (DSE) proposed by \citet{Birzer15}.} They differ functionally from other types of small EDUs, as they do not carry any propositional information. Therefore, they can be eliminated from the clause due to irrelevance to the propositional content; furthermore, they cannot be subject to questions and cannot be negated (\citealt[see][85f]{Birzer15} for a consistent account). Example (\ref{(9.8)}) contains three DSEs: \textit{ali} ‘but’, \textit{recimo} ‘let’s say’, \textit{evo} ‘here’:

\begin{exe}\ex\label{(9.8)}
\gll \minsp{[} Ali] \minsp{[} recimo], \minsp{[} evo] starija {kćerka  [\dots].} \\
{} but {} say.\textsc{imp}.1\textsc{pl} {} here older daughter \\
\glt ‘But, let’s say, here, the older daughter  [\dots].’
\hfill (BL)
\end{exe}

\item \textsc{omissions} can be divided into ellipsis and aposiopesis. The term \textsc{ellipsis} refers to cases where the missing element can be restituted semantically. \textsc{aposiopesis} is used in a purely syntactic and not rhetorical sense: it represents syntactic units where constituents are missing and are not recoverable from the context \citep[cf.][]{Karlik17a}. In (\ref{(9.9)}) we can reconstruct a predicate related to ‘studying’. In contrast, in example (\ref{(9.10)}) the missing element cannot be reconstructed. We annotated such instances as aposiopesis.

\begin{exe}\ex\label{(9.9)}
\gll Ova druga kćerka []\textsubscript{ellipsis} na fakultetu u Z. a sin završava treću godinu zanata u L.\\
this other daughter {} on faculty in Z. and son finish.3\textsc{prs} third year craft in L.\\
\glt ‘This other daughter [studies] at the faculty in Z., and the son is finishing the third year of craft (school) in L.’
\hfill (BG1)

\ex\label{(9.10)}
\gll  [\dots] samo da prestane da puca, da \textbf{se} može []\textsubscript{aposiopesis} i žao {\textbf{mi}  [\dots].}\\
{} just that stop.3\textsc{prs} that shoot.3\textsc{prs} that \textsc{refl} can.3\textsc{prs} {} and sorry me.\textsc{dat} \\
\glt ‘[\dots] only to stop shooting, to be able to [?]… and I am sorry  [\dots].’ \\
\hfill (DO)
\end{exe}

\item \textsc{anacolutha} appear in utterances where the speaker switches to a different construction within one clause (or in a broader syntactic context), which leads to a grammatically and/or semantically ill-formed structure \citep[cf.][]{Karlik17b}. In example (\ref{(9.11)}) the interviewee first doubles the \textsc{refl\textsubscript{lex}} CL \textit{se} which belongs to the verb \textit{izraziti se} ‘express’ and then omits the \textsc{refl\textsubscript{lex}} CL \textit{se} which belongs to the verb \textit{sjetiti se} ‘remember’.

\protectedex{\begin{exe}\ex\label{(9.11)}
\gll  [\dots] kako \textbf{se}\textsubscript{?} sada da \textbf{se}\textsubscript{1} izrazim\textsubscript{1} ne mogu\textsubscript{2} da []\textsubscript{anacoluthon} {sjetim\textsubscript{3}  [\dots].}\\
{} how \textsc{refl} now that \textsc{refl} express.1\textsc{prs} \textsc{neg} can.1\textsc{prs} that {} remember.1\textsc{prs}\\
\glt ‘[\dots] how myself now to express myself, I can’t remember  [\dots].’   \\
\hfill (BL)
\end{exe}
}

\item \textsc{inserted clause} Like \citet[75]{Crible18} we distinguish insertions in the sense of what she calls parenthetical insertions – “propositional segments functioning as a ‘parenthetical aside’ [\dots] – located in the sequence of fluencemes to which it adds some background information without directly modifying the content of the utterance”. In our data insertions are mainly relative clauses or parentheticals, such as \textit{gdje smo živjeli} ‘where we lived’ in (\ref{(9.41)}) and \textit{zna se dobro} ‘it is well known’ in (\ref{(9.42)}). 

\begin{exe}\ex\label{(9.41)}
\gll  [\dots] u okruženju, \minsp{[} gdje \textbf{smo} živjeli], \textbf{smo} bili ređi. \\
{} in environment {} where be.1\textsc{pl} live.\textsc{ptcp}.\textsc{pl}.\textsc{m} be.1\textsc{pl} be.\textsc{ptcp}.\textsc{pl}.\textsc{m} rarer \\
\glt ‘[\dots] in the environment where we lived, we were rarer.’
\hfill (KR)

\ex\label{(9.42)}
\gll U prošlom ratu, \minsp{[} zna \textbf{se} dobro], nije [\dots] ne bi bilo ništa {novo  [\dots].} \\
in previous war {} know.\textsc{3prs} \textsc{refl} well \textsc{neg}.be.3\textsc{sg} {} \textsc{neg} \textsc{cond}.\textsc{3sg} be.\textsc{ptcp}.\textsc{sg}.\textsc{n} nothing new \\
\glt ‘During the last war – this is well known – there would not have been anything new  [\dots].’ 
\hfill (BG1)
\end{exe}
\end{enumerate}

\section{Data preparation}
\label{Data preparation}
\subsection{Annotation scheme}
\label{Annotation scheme}
Our objective for the annotation was economy, transparency, and adequacy of analysis. We distinguished three main steps in the annotation process: segmentation into clauses, annotation of categories related to CLs (inventory and position-related phenomena mentioned in Chapter \ref{Our terms and concepts}), and annotation of syntactic structures described in Section \ref{Principles of analysis of spoken language}. The full coding scheme is given in Tables \ref{T9.2}--\ref{T9.2b}.

\subsection{Inventory-related categories}
\label{Inventory-related categories}

\begin{table}
\caption{The coding scheme of inventory-related categories\label{T9.2}}
\begin{tabularx}{\textwidth}{Xl}
\lsptoprule
  Category & Values\\\midrule
  CL & VERB $=$ any verbal CL except \textit{je}\\
     & VERB\_je $=$ verbal CL \textit{je}\\
     & PRON\_acc\\
     & PRON\_dat\\
     & PRON\_gen\\
     & REFL $=$ reflexive CL\\
     & REFL-X $=$ reflexive CL used in haplology of unlikes\\
     & QUEST $=$ the polar question marker \textit{li}\\
     & any combination of the above CLs\\
  cluster & $0 = \text{no cluster}$\\
     & $1 = \text{cluster}$\\
  (pseudo)diaclisis & $0 = \text{no diaclisis}$\\
     & $1 = \text{diaclisis}$\\
     & $2 = \text{pseudodiaclisis}$\\
\lspbottomrule
\end{tabularx}
\end{table}

The first topic investigated is inventory-related categories which include distribution of CL types, the types of clusters and the ordering of CLs in clusters in spoken Bosnian in comparison to standard written Bosnian (and other standard varieties). We include this in Table \ref{T9.2}. 

In the CL Type category, we annotated not only single CLs but also all occurrences of two and more CLs in a clause. The distinction between clusters and (pseudo)diaclisis was annotated separately. This allowed us to obtain information on the types of CLs which clusterise, and compute the maximal size of a cluster. 

As to morphonological processes, we are particularly interested in the interaction of the verbal CL \textit{je} and the reflexive CL \textit{se}. In written language their co-occurrence usually leads to haplology of unlikes.\footnote{Haplology of unlikes and pseudodiaclisis are the only phenomena related to CC which are included in the coding scheme. Skipping the annotation of other phenomena related to CC is motivated by the overall small size of the corpus. We devote Part \ref{part3} to CC and base the discussion on empirical material retrieved from large web corpora.} Therefore, the reflexive CL \textit{se} appearing without the verbal CL \textit{je}, which has been haplologised, was annotated separately as \textsc{refl-x}.\footnote{Note that this element of annotation relates to the mere surface structure of haplology. It does not refer to the typology of reflexives proposed in Section \ref{Different types of reflexives}.}  

Some CL forms, in particular the \textsc{refl\textsubscript{2nd}} \textit{si} and the pronominal \textit{ju}, were not included in the annotation scheme presented in Table \ref{T9.2} due to their infrequency. However, they were observed in the data and we comment on them in Section \ref{Inventory:9}. 

\subsection{Position-related categories}
\label{Position-related categories}

\begin{table}[b]
\caption{The coding scheme of position-related categories\label{T9.2a}}
\begin{tabularx}{\textwidth}{>{\raggedright}p{\widthof{position of CL in relation to the}}Q}
\lsptoprule
Category & Values \\\midrule
  position of CL & 1P $= \text{initial position}$\\
 & 2P $= \text{second position}$\\
 & DP $= \text{delayed position}$\\
  position of CL in relation to the number of words & number of stressed words (prosodic units) counted from the beginning of the clause or from the beginning of the insertion if a CL occurs after an insertion \\
  length of host (only for DP) & number of graphemes of the host constituent in DP\\
  length of the initial constituent & number of graphemes of the initial constituent (for 2P the length of the initial constituent coincides with the length of the host constituent)\\
  phrase splitting & $0 = \text{phrase splitting impossible}$\\
 & $1 = \text{phrase splitting}$\\
 & $2 = \text{no phrase splitting, although possible}$\\
\lspbottomrule
\end{tabularx}
\end{table}

\begin{table}
\caption{The coding scheme of syntactic structures\label{T9.2b}}
\begin{tabular}{cll}
\lsptoprule
Category & Values \\\midrule
A. & \multicolumn{2}{l}{position of a syntactic structure in relation to CL(s)} \\
& identical repetition & $0 = \text{no}$ \\
&& $1 = \text{yes, before CL}$ \\
&& $2 = \text{yes, after CL}$ \\
&& $3 = \text{yes, including CL}$ \\
& partial repetition & $0 = \text{no}$ \\
&& $1 = \text{yes, before CL}$ \\
&& $2 = \text{yes, after CL}$ \\
&& $3 = \text{yes, including CL}$ \\
& false start & $0 = \text{no}$ \\
&& $1 = \text{yes, before CL}$ \\
&& $2 = \text{yes, after CL}$ \\
&& $3 = \text{yes, including CL}$ \\
& substitution & $0 = \text{no}$ \\
&& $1 = \text{yes, before CL}$ \\
&& $2 = \text{yes, after CL}$ \\
&& $3 = \text{yes, including CL}$ \\
B.&\multicolumn{2}{l}{presence of a syntactic structure before CL(s)}\\
& rendered topic & $0 = \text{no}$ \\
&& $1 = \text{yes}$ \\
& Retrospective&$0 = \text{no}$ \\
&& $1 = \text{yes}$ \\
& DSE & $0 = \text{no}$ \\
&& $1 = \text{yes}$ \\
& ellipsis & $0 = \text{no}$ \\
&& $1 = \text{yes}$ \\
& aposiopesis & $0 = \text{no}$ \\
&& $1 = \text{yes}$ \\
& anacoluthon & $0 = \text{no}$ \\
&& $1 = \text{yes}$ \\
& inserted clause & $0 = \text{no}$ \\
&& $1 = \text{yes}$ \\
\lspbottomrule
\end{tabular}
\end{table}

Secondly, we are interested in CL position in the clause. Categories which had to be annotated for the study of this topic are shown in Table \ref{T9.2a}. One of the phenomena considered under positioning is phrase splitting, where the CL is placed within the hosting phrase. This, however, can take place only if a hosting phrase can be split. Therefore, in the coding scheme we distinguished clauses which do not satisfy the conditions for phrase splitting (coded as 0), clauses where phrase splitting takes place (coded as 1), and clauses where phrase splitting could theoretically be possible (phrases appeared in a position before the CL, but phrase splitting did not take place; coded as 2).

We distinguished three types of positions of the CL or cluster: 1P (where there is no initial host constituent in the clause or alternatively the CLs directly follow an insertion), 2P and DP. In order to establish the placement type, we took into account the special syntactic structures discussed above in Section \ref{Principles of analysis of spoken language} and listed in Table \ref{T9.2b}. In most cases we were simply interested in whether these phenomena, including DSEs or inserted clauses, appeared before the CLs in the utterance (Part B of Table \ref{T9.2b}.).\footnote{Examining the potential impact of inserted clauses on the CLs which appear in the main clause after the insertion is one of the objectives of this research. While inserted clauses can include CLs too, they also contain predicates, so they are categorised as separate clauses.} Formally they are not integrated into the clause, but they may potentially have an impact on CL placement. Disfluencies (Part A of Table \ref{T9.2b}.) such as repetition, substitution and false start may appear before a CL or involve a CL. To permit examination of this, the coding scheme included information on where a fluenceme is placed relative to the CL. 

When measuring the length of constituents preceding CLs (i.e. their heaviness) we followed the solutions which \citet{KCN18} proposed for measuring the positions of pronominal CLs in Old Czech Bible translations. In the case of 2P we measured the length of the initial constituent, which coincides with the host, while in the case of DP we measured the length of the initial constituent and the length of the host appearing directly before the CL. The unit of measurement is the grapheme, and we applied it only to clauses which do not contain anonymised elements before CLs.\footnote{Anonymised versions of city and person names usually contained only one grapheme so we were unable to ascertain exactly how long the constituents preceding CLs were. } The task was relatively straightforward as in most cases one letter corresponds to one grapheme; the exceptions are the letter combinations \textit{nj, lj, dž} and the variants for jat \textit{i, e, je, ije} which we treated as one grapheme. When measuring the heaviness of constituents preceding CLs, information provided in <reg> tags was particularly valuable. It suggested that speakers chose nonstandard lexico-syntactic elements and/or pronounced some units differently than they would be pronounced in standard Bosnian. For instance, in (\ref{(9.12)}) additional information about the host element is preserved in the <reg> tag. We can use it as a basis for measuring heaviness allowing us to establish that the initial constituent in (\ref{(9.12)}) is two graphemes long and not three, as would be the case in standard usage.

\protectedex{\begin{exe}\ex\label{(9.12)}
\gll \minsp{<reg\_orig="Đe">} gdje \textbf{ti } \textbf{je } potvrda? \\
{} where you.\textsc{dat}  be.3\textsc{sg}  confirmation \\
\glt ‘Where is your confirmation?’
\hfill (DJ)
\end{exe}
}

\noindent We also determined the position of CLs relative to the beginning of the clause, which we understood as the number of preceding stressed words (i.e. prosodic units) which can serve as hosts to CLs. Therefore, certain conjunctions such as \textit{i} ‘and’ and \textit{a} ‘and/but’, all prepositions and the negations \textit{ne} ‘no/not’ and \textit{ni} ‘nor/not even’ were not included as independent words in this measurement, because they cannot host CLs. 

\section{Inventory}
\label{Inventory:9}
\subsection{Distribution of clitics in the corpus of spoken Bosnian}
\label{Distribution of clitics in the corpus of spoken Bosnian}

The annotated clauses contain 4727 CLs. It is interesting to note that CLs are very rarely attested in repetitions (both identical and partial), false starts or substitutions. Speakers repeated or replaced (when self-correcting) a total of 132 CLs in 117 clauses. This number accounts for only 3\% of all CL usages in the corpus. In the majority of cases the replacement in partial repetitions and substitutions contains a verbal CL (100 of 117 clauses). We did not count CLs appearing in repetitions and substitutions in the overall distribution shown in Figure \ref{F.9.1}.

\begin{figure}
\caption{Distribution of single CLs in the corpus}
\label{F.9.1}
\includegraphics[width=.49\textwidth]{f91}
\end{figure}

To start with, in general the CL inventory coincides with the inventory attested in written standard Bosnian (and other standard varieties). Nevertheless, the reflexive CL \textit{si} was identified as an additional CL element in the inventory of the spoken Bosnian variety that is not present in the inventory of either standard Bosnian or standard Serbian. We shortly comment on this CL in Section \ref{Reflexive clitics:9}. Verbal CLs are the most frequent CL and they make up 69\% of the sample ($N=3210$; almost half of that ($N=1341$) are occurrences of the CL \textit{je} ‘is’). The second most frequent type is reflexive CLs ($N=750$), followed by pronominal CLs ($N=565$). The question marker \textit{li} ($N=79$) is mostly used in its reduced form, transcribed as \textit{l’}, like in (\ref{(9.13)}). 

\begin{exe}\ex\label{(9.13)}
\gll E, sad da \textbf{l’}           \textbf{će } te biti mržnje, da \textbf{l’ } neće, ne {znam  [\dots]}  \\
          eh now that \textsc{q} \textsc{fut}.3\textsc{sg} this be.\textsc{inf} hatred that \textsc{q} \textsc{neg}.\textsc{fut}.3\textsc{sg} \textsc{neg} know.1\textsc{prs} \\
\glt ‘Eh, now, will there be hatred, or not, I don’t know…
\hfill (VI)
\end{exe}

\subsection{Pronominal clitics}
\label{Pronominal clitics:9}
The frequencies of pronominal CLs are as follows: dative CLs are the most frequent ($N=350$), followed by accusative ($N=198$), and genitive ($N=15$). The high frequency of dative CLs is related to the non-argumental, i.e., possessive (\ref{(9.X5)}) and ethical (\ref{(9.X6)}) dative. 

\begin{exe}\ex\label{(9.X5)}
\gll  [\dots] muž \textbf{mi} \textbf{je} tu išo u {školu  [\dots].} \\
{} husband me.\textsc{dat} be.3\textsc{sg} here go.\textsc{ptcp}.\textsc{sg}.\textsc{m} to school \\
\glt ‘[\dots] my husband went to school here  [\dots].’ 
\hfill (DO)

\ex\label{(9.X6)}
\gll I ne znam \textbf{ti} koje već, šta {već  [\dots].}  \\
and \textsc{neg} know.1\textsc{prs} you.\textsc{dat} which already what already \\
\glt ‘And I honestly do not know whichever, whatever  [\dots].’ 
\hfill (BH)
\end{exe}

\noindent As mentioned in Section \ref{Inventory of pronominal clitics in BCS standard varieties} written standard BCS varieties differ with respect to their usage of the accusative pronominal CLs \textit{ju} and \textit{je} ‘her’. Therefore, we examined the corpus to determine the distribution of these forms in the spoken Bosnian variety. We found 13 clauses with pronominal CLs in the accusative third person singular feminine form. However, all of them contained the CL form \textit{je}, like in (\ref{(9.14)}). We thus found no empirical evidence for the CL \textit{ju} being part of the CL inventory in the language of the recorded speakers.

\protectedex{\begin{exe}\ex\label{(9.14)}
\gll  [\dots] ne vidim  \textbf{je} nikako. \\
{} \textsc{neg} see.1\textsc{prs}  her.\textsc{acc}  nohow \\
\glt ‘[\dots] I don’t see it at all.’
\hfill (BG1)
\end{exe}}

\noindent Nevertheless, regardless of the absence of the CL form \textit{ju} from the corpus of spoken Bosnian, we should refrain from generalisations concerning the usage of this CL in spoken Bosnian as such.\footnote{Caution is necessary here. Only 16 informants contributed to the analysed corpus of spoken Bosnian. Our statement does not mean that there is no CL \textit{ju} at all in any Bosnian varieties, nor that the CL \textit{ju} cannot generally be attested in the spoken Bosnian variety. The reason for this caution is the size of the corpus of spoken Bosnian on the one hand and data from bsWaC on the other. If we compare the distribution of accusative CLs \textit{ju} \texttt{[tag="Pp3fsa" \& word="ju"]} and \textit{je} \texttt{[tag="Pp3fsa" \& word="je"]} in bsWaC, we get the following results: 27,433 occurrences of \textit{ju} (95.6 per million) and 33,305 occurrences of \textit{je} (116.1 per million). As we can see, the difference in the distribution of the competing forms is not that extensive at all. Moreover, the CL \textit{ju} is attested in dialectological data from the language territory of Bosnia and Herzegovina presented in Section \ref{Feminine pronominal clitics}.}

Further, we notice that the CL for third person plural accusative is reduced by some speakers to a form transcribed as \textit{i’} instead of \textit{ih} ‘them’: see example (\ref{(9.15)}) below. This is in line with forms found in most dialects (see Section \ref{Plural pronominal clitics in the accusative}).

\protectedex{\begin{exe}\ex\label{(9.15)}
\gll Od koga \textbf{i’} brani -- od komšija.\\
from whom them.\textsc{acc} protect.3\textsc{prs} {} from neighbours\\
\glt ‘Whom is he protecting them from – from the neighbours.'
\hfill  (VI)
\end{exe}}

\subsection{Verbal clitics}
\label{Verbal clitics:9}
As pointed out above, verbal CLs are quantitatively the most frequent CL type in the whole corpus of spoken Bosnian, with the CL \textit{je} ‘is’ as the most frequent CL form ($N=1341$, 29\% of all CL occurrences in the corpus).

We already indicated in Section \ref{Verbal clitics:8} that in some Štokavian dialects there is only one form of the conditional auxiliary for all persons and that this form is spreading from dialects into spoken BCS varieties. This syncretism is also attested in the corpus data: the interviewees use the CL form \textit{bi} ‘would’ for all persons in the conditional. This is nicely illustrated in (\ref{(9.16)}), where the interviewee uses the CL form \textit{bi} and not the CL form \textit{bih} which is prescribed in standard BCS varieties. 

\protectedex{\begin{exe}\ex\label{(9.16)}
\gll  [\dots] kad \textbf{bi} ja stvarno mogla još više {dati  [\dots].}  \\
{} when \textsc{cond}.1\textsc{sg} I really can.\textsc{ptcp}.\textsc{sg}.\textsc{f} still  more give \\
\glt  ‘[\dots] if I could really give even more  [\dots].’ 
\hfill (VI)
\end{exe}}

\noindent Inflected forms of the conditional occur 5 times, only as forms of the first person singular and plural \textit{bih} (\ref{(9.17)}) and \textit{bismo} (\ref{(9.18)}), whereas the uninflected form \textit{bi} for those and the second person plural is used 14 times.

\begin{exe}\ex\label{(9.17)}
\gll  [\dots] dodao \textbf{bih} još možda malo tačniji odgovor.  \\
{} add.\textsc{ptcp}.\textsc{sg}.\textsc{m} \textsc{cond}.1\textsc{sg} still perhaps little more.precise answer \\
\glt ‘[\dots] I would add maybe a slightly more accurate answer.’
\hfill  (MO1)

\ex\label{(9.18)}
\gll  [\dots] ono što \textbf{bismo} mi {željeli [\dots]} \\
      {} That what \textsc{cond}.1.\textsc{pl} we wish.\textsc{ptcp}.\textsc{pl}.\textsc{m} \\
\glt ‘[\dots] the thing we would want…’
\hfill  (MO1)
\end{exe}

\noindent Regardless of the small number of observations, we may assume that they are probably a case of diastratic variation. Namely, the inflected forms prescribed in standard BCS are used by spouses who obtained a higher education than the rest of the interviewees. The male speaker (\ref{(9.17)}) had a PhD degree and the female (\ref{(9.18)}) was a lawyer.

\subsection{Reflexive clitics}
\label{Reflexive clitics:9}

As stated in Section \ref{Reflexive markers se and si in BCS standard varieties} there is diatopic variation in the inventory of reflexive CLs between the BCS standard varieties. The \textsc{refl\textsubscript{2nd}} \textit{si} is only recognised by authors describing the Croatian standard. It does occur in the analysed corpus, but with a very low frequency. Namely, in 750 occurrences of reflexive CLs we find two instances of \textsc{refl\textsubscript{2nd}} \textit{si}, which is 0.3\% of all occurrences. Both of the following utterances were produced by the same speaker. 

\begin{exe}\ex\label{(9.19)}
\gll  [\dots] mogli  \textbf{smo}  \textbf{si}  dozvoliti  pristojne  uvjete  {života  [\dots].}  \\
{} can.\textsc{ptcp}.\textsc{pl.m} be.1\textsc{pl}  \textsc{refl}  allow.\textsc{inf}  decent conditions life \\
\glt ‘We were able to afford decent living conditions  [\dots].’
\hfill (KR)

\ex\label{(9.20)}
\gll  [\dots] i mogao  \textbf{si}  \textbf{je}  priuštiti.  \\
{} and can.\textsc{ptcp}.\textsc{sg}.\textsc{m}  \textsc{refl}  be.3\textsc{sg}  afford.\textsc{inf} \\
\glt ‘[a]nd he could allow himself.’
\hfill (KR)
\end{exe}

\noindent As mentioned in Section \ref{Reflexive clitic si}, the occurrence of the \textsc{refl\textsubscript{2nd}} CL \textit{si} is also reported for idioms of Western Herzegovina and Northern Bosnia. Considering the frequency distribution in the corpus, we have to admit that when compared with the reflexive CL \textit{se}, the reflexive CL \textit{si} is not frequent in Croatian either. In hrWaC, the occurrences of the CL \textit{si} make up only 1.19\% of all reflexive CL occurrences.\footnote{95,016 out of 7,969,617 occurrences of reflexive CLs.\textcolor{black}{The low frequency of the reflexive CL \textit{si} can probably partially be attributed to its homonymy with the verbal CL \textit{si} and tagging inaccuracy.}} The difference between the standard BCS varieties is that in contrast to Bosnian and Serbian, the Croatian standard recognises the \textsc{refl\textsubscript{2nd}} \textit{si} as part of the codified system. However, the form as such occurs not only in spoken Croatian but also in spoken Bosnian \citep[pace][440]{Ridjanovic12}. 

\subsection{Clusters}

Clusters are attested in 461 clauses. In total, 454 clusters consisted of 2 CLs and only 7, of 3 CLs, which sheds new empirical light on the size of the CL cluster. Note that \citet[451f]{PiperKlajn14} claim that a CL cluster usually consists of two or three elements. Our corpus data show that at least in the spoken Bosnian variety CL clusters with three CLs are much rarer than CL clusters with two CLs. 

The frequency distribution of the most frequent types is shown  below in Figure \ref{F9.2}. Note that, as already mentioned in Table \ref{T9.2}, we annotated the verbal CL \textit{je} separately from the other verbal CLs (V\_je vs V).

\begin{figure}
\caption{Distribution of cluster types in the corpus of spoken Bosnian}
\label{F9.2}
\includegraphics[width=.9\textwidth]{f92}
\end{figure}

\largerpage
Fifteen different combinations for 2-CL clusters and 5 different combinations for 3-CL clusters are attested. The most frequent types are $\text{V(erbal)}+\text{REFL}$ $(N=141)$ and $\text{PRON}\_\text{dat}+\text{V(erbal)}\_\text{je}$ $(N=122)$ – note that this combination with a dative CL is much more frequent than the combination $\text{PRON}\_\text{acc}+\text{V(erbal)}\_\text{je}$ $(N=19)$.\footnote{These 19 occurrences also include CL clusters with the verbal CL \textit{je} and a pronominal CL in the accusative in an order which diverges from the CL order attested in the standard BCS varieties. For more information see below.} In contrast, the combinations $\text{V(erbal)}+\text{PRON}\_\text{dat}$ $(N=46)$ and  $\text{V(erbal)}+\text{PRON}\_\text{acc}$ $(N=51)$ are similarly frequent. As already observed in Section \ref{Pronominal clitics:9} above, the high frequency of dative CLs is due to occurrences of possessive and ethical dative. These numbers, however, indicate that possessive dative might be more frequent than ethical dative, since possessive dative is typically used with the verbal CL \textit{je} as in example (\ref{(9.X6)}) provided above.\footnote{\textcolor{black}{Bear in mind that} in our annotation scheme we did not distinguish between possessive and ethical dative CLs.} 

The combination $\text{PRON}\_\text{dat} + \text{REFL}$ appears 9 times; as expected there are no combinations of reflexive CLs with the accusative.\footnote{This is not very surprising since \textsc{refl\textsubscript{lex}} have genitive and not accusative complements.}

\section{Internal organisation of the clitic cluster}
\label{Internal organisation of the clitic cluster:9}
\subsection{Clitic ordering within the cluster}

In Section \ref{Clitic ordering within the cluster in BCS standard varieties} we pointed out that in BCS standard varieties the ordering sequence of CLs in clusters does not differ. In our data, however, we do find two types of CL order in the cluster which diverge from the order attested in standard BCS varieties. 

The first and by far the most common CL order diverging from the sequence given by \citet[29]{FranksKing00} and presented in Section \ref{Clitic ordering within the cluster} involves the verbal CL \textit{je} and the reflexive CL \textit{se}. The expected CL order \textit{se} \textit{je} allowed in standard Croatian and Bosnian appears only six times: one of those utterances is presented in (\ref{(9.21)}).

\protectedex{\begin{exe}\ex\label{(9.21)}
\gll  [\dots] onda \textbf{se} \textbf{je} Irma u mene {pre [\dots]} \\
{} then \textsc{refl} be.3\textsc{sg} Irma in me.\textsc{gen} get.scare \\
\glt ‘[\dots] then my Irma got scare...’
\hfill (BR)
\end{exe}
}

\noindent In contrast, the reversed CL order \textit{je se} is found 25 times in clusters with 2 CLs (\ref{(9.22)}), making this the fifth most frequent type of cluster, and once in a cluster with 3 CLs (\ref{(9.23)}).

\begin{exe}\ex\label{(9.22)}
\gll Gore \textbf{je} \textbf{se} {oženio  [\dots].} \\
up.there be.3\textsc{sg} \textsc{refl} marry.\textsc{ptcp}.\textsc{sg}.\textsc{m} \\
\glt ‘He got married up there  [\dots].’
\hfill (DJ)

\ex\label{(9.23)}
\gll  [\dots] nego muž \textbf{mi} \textbf{je} \textbf{se} {prep’o  [\dots].}  \\
{} but husband me.\textsc{dat} be.3\textsc{sg} \textsc{refl} get.scared.\textsc{ptcp}.\textsc{sg}.\textsc{m} \\
\glt ‘[\dots] but my husband got scared  [\dots].’
\hfill  (BJ)
\end{exe}

\noindent Still, both patterns are much less frequent than the haplologised structure (which omits the verbal CL \textit{je}), discussed in the next section. 

The second CL ordering sequence which diverges from the order attested in standard BCS varieties also involves the verbal CL \textit{je}. Although the established CL order in standard BCS has the CL \textit{je} appearing in the final position of the ordering sequence, we found 5 clusters in which the verbal CL \textit{je} precedes a pronominal accusative CL, like in example (\ref{(9.24)}).

\protectedex{\begin{exe}\ex\label{(9.24)}
\gll  [\dots] koji \textbf{je} \textbf{me} dobro poznavao i {cijenio  [\dots].} \\
{} which be.3\textsc{sg} me.\textsc{acc} good know.\textsc{ptcp}.\textsc{sg}.M and respect.\textsc{ptcp}.\textsc{sg}.\textsc{m} \\
\glt ‘[\dots] who knew me very well and respected me  [\dots].’
\hfill (BG1)
\end{exe}
}

\noindent Although these two types of CL ordering in a cluster attested in spoken Bosnian diverge from the standard BCS varieties, they do not come as a surprise. As already mentioned in Section \ref{Clitic ordering within the cluster:8}, they are also attested in dialects spoken on BSC territory.

\subsection{Morphonological processes within the cluster}
\label{Chapter9:Morphonological processes within the cluster}

In order to identify possible microvariation, we analysed reflexive pronouns with regard to the following categories:
\begin{enumerate}
 \item haplology of unlikes (\textit{se} appearing alone where a \textit{se je} cluster is expected),
 \item co-occurrence with \textit{je} (\textit{se je} appearing in a cluster),
 \item \textit{se} and \textit{je} appearing in (pseudo)diaclisis. 
\end{enumerate}
 
 The data contain 122 clauses with possible co-occurrence of the reflexive CL \textit{se} and the verbal CL \textit{je}, with the distribution in \figref{F.9.3}.

\begin{figure}[ht]
\caption{Distribution of different constructions with the reflexive CL \textit{se} and the verbal CL \textit{je} in clauses}
\label{F.9.3}
\includegraphics[width=.6\textwidth]{f93}
\end{figure}

In Figure \ref{F.9.3} we see that in 68.8\% of cases the reflexive CL \textit{se} appears without \textit{je} and in 25.4\% of cases, with \textit{je}. This speaks for the preference of haplology in spoken Bosnian. All \textit{je se} clusters are simple clusters. Similarly, the haplological forms are generated by one verb. All analysed instances are in a past-tense context. Our data thus show that haplology of unlikes described by many grammarians of the standard BCS varieties (\citealt[e.g.][246]{TezakBabic96}, \citealt[596]{Baric97}, \citealt[471]{JHP00}, \citealt[302, 333]{Ridjanovic12}, \citealt[450]{PiperKlajn14}) is not the rule in spoken Bosnian.\footnote{See also Section \ref{Haplology of unlikes}.} Although \textit{je} is often omitted as an auxiliary and in simple clusters, the non-haplological forms are only twice less frequent. Our data do not allow for any conclusions about mixed clusters or usage as a copula.

As already mentioned in the previous section, the combination of the CLs \textit{je} and \textit{se} mostly appears in an order (\ref{(9.25)}) which is reversed in comparison to the one attested in standard Croatian and Bosnian varieties. However, even this reversed order is less frequent than haplology of unlikes (\ref{(9.26)}).

\begin{exe}\ex\label{(9.25)}
\gll  [\dots] znalo \textbf{je} {\textbf{se}  [\dots].} \\
{} know.\textsc{ptcp}.\textsc{sg}.\textsc{n} be.3\textsc{sg}  \textsc{refl}  \\
\glt ‘[\dots] it was known  [\dots].’
\hfill (BR)

\ex\label{(9.26)}
\gll  [\dots] znalo \textbf{se} {uvijek  [\dots].} \\
{} know.\textsc{ptcp}.\textsc{sg}.\textsc{n} \textsc{refl}  always \\
\glt ‘[\dots] it was always known  [\dots].’
\hfill (BL)
\end{exe}

\noindent Other types of variation with respect to morphonological processes within the cluster are not evident in the data. The morphonological process of suppletion, in which the feminine accusative pronominal CL \textit{je} is replaced with its counterpart \textit{ju} when followed by its homonym, the verbal CL \textit{je}, is not attested at all. Note that in \textcolor{black}{revised} dialectological data we find no evidence for suppletion either. We may assume that the Bosnian linguist \citet[434]{Ridjanovic12} could be right in his claim that suppletion is a feature of deliberate speech, \textcolor{black}{but more robust data is definitely needed on this matter.} 

\section{Position of the clitic or the clitic cluster}
\label{Positioning of single clitics and clitic clusters}
\subsection{General distribution}

The following discussion on CL positioning is based on a subsample of 3829 clauses. This choice is motivated by the fact that on the basis of anonymised transcripts only, some clauses were impossible to interpret and a recording would have been needed. The 22 cases of (pseudo)diaclisis are not included in the sample and are discussed separately in Section \ref{Diaclisis:9}. For the time being, we also excluded all cases where CLs were substituted by other CLs as a type of disfluency phenomenon.

We analysed the positions of single CLs and clusters separately and compared them to establish whether any differences could be observed. In total, we analysed 3,399 single CLs (26 in 1P and 164 in DP) and 430 clusters (3 in 1P, 17 in DP). The logarithmic frequencies of the position of single CLs and CL clusters are as given in \figref{F.9.4}.

\begin{figure}
\caption{Frequencies of placement types for single CLs and CL clusters (normalised to natural logarithm for conciseness)}
\label{F.9.4}
\includegraphics[width=.7\textwidth]{f94}
\end{figure}

Pearson’s chi-squared test does not show any significant difference between single CLs and clusters with respect to placement ($p = 0.1991$). Regardless of whether the CLs appear in clusters or as single CLs, a retrograde fall, i.e., a reduction in occurrences from 2P (94\% for single CLs, 95\% for clusters), through 3P (4\% for both single CLs and clusters) to 1P ($<\,1\%$ for both single CLs and clusters), is noticeable for all CLs.

The category labelled 1P was mostly recorded for the position after an insertion; we discuss it in Section \ref{Impact}. Among single CLs in 1P ($N=26$) we identify only verbal ($N=19$) and reflexive ($N=7$) CLs, in particular the verbal CL \textit{je}, which appears 15 times.

\subsection{Placement of single CLs}

We now turn to differences in the positioning of individual CLs compared to cluster types. We found occurrences of delayed placement for all types of single CLs except the polar question CL \textit{li}. In Chapter \ref{Our terms and concepts} we pointed out that the polar question marker \textit{li} differs significantly from other CLs because it does not have a non-clitic equivalent. In our data we annotated 79 appearances of the polar question marker \textit{li} in total. The only significant fact we would like to raise is that the CL \textit{li} takes the second position in 100\% of cases. Moreover, in 100\% of cases \textit{li} follows one very short word (only 2 to 4 graphemes long). This is in line with the observation of \citet[119]{SiewierskaUhlirova98} who call this CL “inflexible”.

\begin{figure}
\caption{Placement of single CLs}
\label{F.9.5}
\includegraphics[width=.7\textwidth]{f95}
\end{figure}

Little variation is observed for pronominal CLs. While accusative and dative pronominal CLs were attested not only in 2P but also in DP, genitive CLs were attested only in 2P. The generally more frequent verbal and reflexive CLs differ somewhat from pronominal CLs and the polar question marker \textit{li}. As well as in delayed placement, in rare cases they also appear in the first position (see more below). Nonetheless, delayed placement seems equally rare for all the three CL types: verbal, reflexive and pronominal.

\subsection{Placement of clusters}

The distribution of cluster types across positions is shown in Figure \ref{F.9.6}. In all, only 17 of 455 analysed clusters occupy DP. The two most frequent clusters V(erbal), REFL ($N=5$) and PRON\_dat, V(erbal)\_je ($N=6$), also have the highest frequency in DP. For six clusters only single occurrences in DP are observed. 

Although reflexive CLs can take not only 2P but also DP and 1P, clusters starting with a reflexive CL show no variation in this respect. Namely, they always appear in 2P in the data. However, other CL clusters containing the reflexive CL \textit{se} do show variation. For instance, the cluster \textit{je se} was attested not only in 2P, but also in DP and 1P.

Clusters containing the polar question marker \textit{li} are, similarly to \textit{li} occurring as a single CL, observed in the data only in 2P.

\begin{figure}
\caption{Placement of CL clusters}
\label{F.9.6}
\includegraphics[width=.8\textwidth]{f96}
\end{figure}

\subsection{Relationship between the length of preceding phrases and clitic placement}

The previous sections showed no substantial difference in terms of placement in clauses between CLs and CL clusters. Second position is by far the dominant position in spoken Bosnian. Delayed placement represents about 4\% of CL occurrences. 

In the following lines, we focus on the differences between 2P and DP. As already discussed in Sections \ref{Placement with respect to breaks in BCS standard varieties} and Section \ref{Second position, second word and delayed placement}, delayed placement is usually associated with breathing breaks and long initial or heavy phrases. We measure the length of constituents according to the number of intonational words and the number of graphemes as suggested in \citet{KCN18} and thoroughly explained in Section \ref{Position-related categories}. This approach is rather new in studies of South Slavic languages, where most authors argue in favour of a mixed prosodic and syntactic approach.\footnote{However, none of these authors offer solutions for how exactly to empirically distinguish (heavy) phrases which cannot host CLs from phrases which can host CLs. For more information on approaches to 2P effects see Sections \ref{Second position} and \ref{Approaches to 2P effects: syntax, phonology and information structure}.} Figure \ref{F.9.7} shows the differences in placement related to the number of preceding words counted from the beginning of a clause or the end of an insertion. 

\begin{figure}
\caption{Frequencies of CLs in DP and 2P (normalised to natural logarithm for conciseness. The abscissa shows the number of words preceding the CL).}
\label{F.9.7}
\includegraphics[width=.7\textwidth]{f97}
\end{figure}

One preceding word is by definition necessary for 2P. The studied data contain 2,982 such observations. Thus,  $\text{2P}=\text{2W}$ placement holds for 77\% of the data.\footnote{For more information on 2W see Section \ref{Second position vs second word in BCS standard varieties}}

According to Radanović-Kocić (\citeyear[108ff]{RadanovicKocic88}, \citeyear[435]{RadanovicKocic96}) CLs do not usually follow an initial phrase longer than two words, unless that phrase is a subject. Only in the latter case can a phrase longer than two intonational words be a potential CL host. We found 175 observations with CL placement in the second position after two stressed words, and 31 observations including initial host constituents which are 3--5 words long like \textit{do aprila devedeset i druge} `until April 1992' in (\ref{(9.27)}): 

\protectedex{\begin{exe}\ex\label{(9.27)}
\gll \minsp{[} Do aprila devedeset i druge] \textbf{je} podnošljivo bilo. \\
{} until April ninety and two be.3\textsc{sg} bearable be.\textsc{ptcp}.\textsc{sg}.\textsc{n} \\
\glt ‘Until April 1992 it was bearable.’ 
\hfill (KR)
\end{exe}
}

\noindent This example proves that also longer, non-subject-initial constituents may host CLs in spoken Bosnian Radanović-Kocić (pace \citealt[108ff]{RadanovicKocic88}, \citeyear[435]{RadanovicKocic96}).

Delayed placement appears in the transcripts of all speakers, with the exception of the very short transcript of the second speaker in interview DJ. One hundred eighty one clauses contain a CL or cluster in DP. The part preceding the delayed CL is 2 to 7 words long. Figure \ref{F.9.7} suggests that when the CL is placed after the third or further words, the probability is in favour of DP; that is, in such cases at least two constituents are usually involved. Nonetheless, the constituents preceding 2P CLs may be relatively long when counted in graphemes. This is shown in Figure \ref{F.9.8}.

\begin{figure}
\caption{Box plot representing the frequency distribution for the length of constituents preceding CLs measured in graphemes (ordinate). 2P – initial constituents for 2P, DP\textsubscript{i} – initial constituent in DP, DP\textsubscript{h} – host constituent in DP. }
\label{F.9.8}
\includegraphics[width=.97\textwidth]{f98}
\end{figure}

In Figure \ref{F.9.8} we present box plots where the lower whisker represents the minimum value. The edges of the box show the upper and the lower quartile (25\textsuperscript{th} and 75\textsuperscript{th} percentile), while the thick line represents the median. The upper whisker represents the trimmed estimator based on interquartile range, allowing the outlier values to be seen.\footnote{Interquartile range is the difference between the upper and the lower quartile multiplied by 1.5.}

The minimum, first quantile and mode of 2P are all equal to 2 (see also Table \ref{T9.3}). The most frequent length of the initial constituent preceding CLs in 2P is two graphemes, as seen in 30\% of observations. Twenty per cent of observations in that group are three graphemes long, which is also the median. Another 25\% of observations are 4--5 graphemes long. Only 28 observations ($<\,1\%$) are longer than 9 graphemes. As given in Table \ref{T9.3}, the mean is 4.12 and standard deviation (SD) 2.60.

As mentioned in Section \ref{Position-related categories}, we follow \citet{KCN18} in describing DP. We computed two parameters: the length of the initial constituent (DP\textsubscript{i}) and the length of the actual host (DP\textsubscript{h}). The two types of constituents have some common distributional properties: the minimum value (2), the lower quartile (3) and the upper quantile (6). Nonetheless, they differ as to mode and median, which are equal to each other for both types of constituents. The most frequent initial constituent length is 3 graphemes (18\% of observations with DP). In the case of the host it is 4 (23\% of observations with DP). In both cases, only single observations exceed the value of 10. However, the outliers in host constituents are shorter (the maximum value is 29 graphemes) than the outliers for initial constituents, which reach a length of up to 38 graphemes. Thus, we observe that both types of constituents appearing in DP are, in general, longer than the initial constituents that host CLs in 2P. 

Although the most frequent host in DP is longer than the initial constituent in DP, its values are more “compact”, which is visible when the standard deviation (SD) in Table \ref{T9.3} is compared. It takes the highest value for DP\textsubscript{i}, the middle for DP\textsubscript{h}, and the lowest for 2P. Importantly, SD of DP\textsubscript{h} is much closer to SD of 2P than of DP\textsubscript{i}. This result suggests that the length of the host counted in graphemes is limited in some way.

\begin{table}
\caption{Descriptive statistics for the length of the constituent preceding a CL in a clause}
\label{T9.3}
\centering
\begin{tabularx}{.8\textwidth}{XYYYY}
\lsptoprule
&\multicolumn{4}{c}{Measurement in graphemes} \\\cmidrule(rl){2-5}
&Mean&SD&Median&Mode \\
\midrule
2P &4.12 &2.60 &3 &2 \\
DP\textsubscript{i} &6.03 &6.25 &3 &3 \\
DP\textsubscript{h} &5.38 &3.15 &4 &4 \\
\lspbottomrule
\end{tabularx}
\end{table}

We tested the results for significance. None of the three distributions come from the normal distribution which can be tested with the Shapiro-Wilk normality test ($\text{2P: } W = 0.74724, p < 2.2e\text{-}16\text{; }  \text{DP\textsubscript{i}: } W = 0.63624, p < 2.2e\text{-}16\text{; } \text{DP\textsubscript{h}: } W = 0.84325, p = 1.13e\text{-}12$). Therefore, we investigated the differences in lengths of particular constituents using non-parametric tests. The difference between the distribution of a DP initial constituent and a 2P initial constituent is significant according to the Kolmogorov-Smirnov test ($D = 0.24803, p  = 1.384e\text{-}05$). We made sure that the difference is not a result of location shift. To this end we performed the Wilcoxon Rank-Sum Test, which confirmed that the true location shift is not equal to 0 ($W = 18794, p = 0.01412$). 

The same holds for the difference between the DP host and the 2P initial constituent (Kolmogorov-Smirnov test: $D = 0.23772, p = 6.914e\text{-}09$; Wilcoxon Rank-Sum Test: $W = 421830, p = 2.214e\text{-}11$). Thus the length of an initially positioned host for 2P CLs and the length of the two constituents distinguished for DP differ significantly.

We now examine the relationship between the DP constituents. The observations should be treated pairwise, as this is the way these constituents occur. We first show them in Figure \ref{F.9.9}. The 181 observations are sorted according to the length of all constituents preceding the delayed CL. Circles and triangles represent the actual constituent lengths, while the black and grey lines depict the main trend in the data.

\begin{figure}
\caption{Pairwise length of initial and host constituents in DP}
\label{F.9.9}
\includegraphics[width=.7\textwidth]{f99}
\end{figure}

The longer the constituents, the bigger the difference between the initial constituent and the host. In very short constituents (up to six graphemes long) the host is often longer than the initial constituent. However, when constituents become longer, host length remains at the same level, while the initial constituent may still lengthen. Because the deviations in the data are obviously caused by very long initial constituents, we tested for the significance of the difference between the host and the initial constituent leaving out the nine (less than 5\%) of the longest initial constituents, that is, the constituents with over 20 graphemes. We used the Wilcoxon signed-rank test for paired vectors with the alternative hypothesis that the median for the initial constituent is lower than the median for the host. The result of the test ($V = 4687.5, p = 0.01018$) allows us to reject the null-hypothesis.

Therefore, we conclude that the phenomenon of DP in spoken Bosnian is a result of significantly long initial constituents which block 2P placement. Surprisingly, the actual hosts are, in most cases, even longer than initial constituents. According to the trend visible in Figure \ref{F.9.9}, this regularity applies to clauses where both the host and the initial constituent are about six graphemes, like in example (\ref{(9.X7)}) where the initial constituent is four graphemes long, whereas the actual host is eight graphemes long. 

\protectedex{\begin{exe}\ex\label{(9.X7)}
\gll Ne da mislim, \minsp{[} nego]\textsubscript{phrase1} \minsp{[} sto posto]\textsubscript{host} \textbf{sam} {ubijeđen  [\dots].} \\
\textsc{neg} that think.1\textsc{prs} {} but {} hundred percent be.1\textsc{sg} convinced.\textsc{ptcp} \\
\glt ‘Not that I assume, but I am one hundred percent convinced  [\dots].’  
\hfill (BG1)
\end{exe}
}

In the subset of initial constituents longer than 6 graphemes, the host is shorter, and it stays at a length of 3--6 graphemes (\ref{(9.X8)}). Hence, the length of the host remains at the same level. 

\protectedex{\begin{exe}\ex\label{(9.X8)}
\gll  [\dots] \minsp{[} drugarica najbolja]\textsubscript{phrase1} \minsp{[} bila]\textsubscript{host} \textbf{mi} \textbf{je} {Srpkinja  [\dots].} \\
{} {} friend best {} be.\textsc{ptcp}.\textsc{sg}.\textsc{f} me.\textsc{dat} be.3\textsc{sg} Serbian \\
\glt `[\dots] my best (girl)friend was a Serbian  [\dots].'
\hfill (TZ)
\end{exe}
}

\noindent This phenomenon cannot be explained by the syntactic properties of constituents. It is important to observe that hosts prefer even numbers of graphemes, since two and four are the modes, while the mode of the initial constituent is an odd number, 3. Since the numbers represent word length, it is clear that when DP occurs, the host type changes. The two-grapheme words are usually grammatical words such as pronouns or determiners.

\begin{exe}\ex\label{(9.X9)}
\gll  [\dots] tako da \minsp{[} ja]\textsubscript{host} \textbf{sam} {tu  [\dots].} \\
{} so that {} I be.1\textsc{sg} here \\
\glt ‘[\dots] so that I am here  [\dots].’ 
\hfill (MO1)

\ex\label{(9.X10)}
\gll  [\dots] do temelja \minsp{[} to]\textsubscript{host} \textbf{je} {uništeno  [\dots].} \\
{} until foundation {} that be.3\textsc{sg} destroyed.\textsc{pass}.\textsc{ptcp} \\
\glt ‘[\dots] that is completely destroyed  [\dots].’ 
\hfill (BJ)
\end{exe}

\noindent The four-grapheme words are lexical words, for example verbs and particles like in (\ref{(9.X11)}) and (\ref{(9.X12)}). 

\begin{exe}\ex\label{(9.X11)}
\gll  [\dots] da to \minsp{[} vodi]\textsubscript{host} \textbf{nas} u {propast  [\dots].} \\
{} that that {} lead.3\textsc{prs} us.\textsc{acc} in failure \\
\glt ‘[\dots] that this leads us to failure  [\dots].’ 
\hfill (BG1)

\ex\label{(9.X12)}
\gll Moj muž \minsp{[} isto]\textsubscript{host} \textbf{je} {ranjen  [\dots].} \\
my husband {} also  be.3\textsc{sg} injured.\textsc{pass}.\textsc{ptcp} \\
\glt ‘My husband is also injured  [\dots].’
\hfill  (BJ)
\end{exe}

%The short initial constituents where DP occurs include, for example, connectors \ref{(9.X13)}.
%
%\protectedex{\begin{exe}\ex\label{(9.X13)}
%\gll  [\dots] kao da oni osjećaju \textbf{se}  [\dots] \\
%{} like that they feel.3\textsc{prs} \textsc{refl} \\
%\glt ‘[\dots] like they feel  [\dots]’ 
%\hfill (BG1)
%\end{exe}
%}

\noindent Since no acoustic data are available, we could speculate that CL placement in the case of very short, one-word constituents might be highly phonologically motivated. As suggested by \citet[71f]{DFZ09}, it may be related to intonational contour.

\subsection{Phrase splitting}
\label{Phrase splitting:9}
\subsubsection{Inventory of clitics participating in phrase splitting}
\label{Inventory of clitics participating in phrase splitting}

We now proceed to the analysis of phrase splitting in the spoken Bosnian variety.\footnote{More information on phrase splitting in written standard BCS varieties can be found in Section \ref{The limits of phrase splitting in BCS standard varieties}, whereas more information on phrase splitting in BCS dialects can be found in Section \ref{Phrase splitting}.} Out of 4106 annotated clauses with CLs, 260 contained a compound phrase as a potential CL host. In other words, 260 clauses in the corpus are potential contexts for phrase splitting. However, only every fifth clause ($N=53$) was actually split by a CL.

\begin{figure}
\caption{Inventory of CLs appearing in contexts where a phrase could be split}
\label{F.9.10}
\includegraphics[width=.6\textwidth]{f910}
\end{figure}

Figure \ref{F.9.10} shows which CL types are used in the corpus as elements inserted into a phrase. Phrase splitting is possible mainly by verbal CLs ($N=31$). The most common type is phrase splitting with verbal CLs \textit{su} ($N=11$) and \textit{je} ($N=15$), as previously observed among others by \citet[174f]{PetiStantic02}. We observe no difference between the behaviour of the CL \textit{je} and other verbal CLs with respect to phrase splitting.

Further, in the case of pronominal CLs, phrase splitting is attested only with dative CLs ($N=5$). We find two occurrences of genitive CLs in the context of multiword phrases, but without phrase splitting. Our data confirm that reflexive CLs may also split a phrase ($N=5$). The differences between verbal and other CLs in the context of phrase splitting which can be seen in Figure \ref{F.9.10} are presumably motivated by frequency. Namely, verbal CLs are generally more frequent. We have no evidence for structural restrictions. For instance, accusative CLs were not attested in our data either as CLs which split phrases or as CLs which occur in the context of phrases which could be split but were not. Nonetheless, both variants can be easily retrieved from bsWaC.\footnote{An example of phrase splitting with the accusative CL \textit{me} is given in (\ref{psacc}): \\
\protectedex{\begin{exe}\ex\label{psacc}
\gll \minsp{[} Moja \textbf{me} porodica] nije čula na dan muzičkog nastupa. \\
{} my me.\textsc{acc} family \textsc{neg}.be.\textsc{3sg} hear.\textsc{ptcp}.\textsc{sg}.\textsc{f} on day music performance \\
\glt ‘My family did not hear me on the day of the musical performance.’ 
\hfill [bsWaC 1.2]
\end{exe}
}
    }

According to some authors (\citealt[e.g.][]{Progovac96}, \citealt{RadanovicKocic88}, \citeyear{RadanovicKocic96}), clusters are not used as splitting elements. However, the corpus of spoken Bosnian contains seven occurrences of the cluster \textit{mi je} (PRON\_dat $+$ V\_je) inserted into a phrase, as in (\ref{(9.28)}). 

\protectedex{\begin{exe}\ex\label{(9.28)}
\gll \minsp{[} Jedan \textbf{mi} \textbf{je} sin] bio {otišao  [\dots].}  \\
{} one me.\textsc{dat} be.3\textsc{sg} son be.\textsc{ptcp}.\textsc{sg}.\textsc{m} leave.\textsc{ptcp}.\textsc{sg}.\textsc{m} \\
\glt ‘One of my sons had left  [\dots]’ 
\hfill (DJ1)
\end{exe}
}

\noindent The possibility of phrase splitting with diaclisis, as in (\ref{(9.29)}), is not mentioned in the literature at all.

\protectedex{\begin{exe}\ex\label{(9.29)}
\gll  [\dots] \minsp{[} moja \textbf{je} mater] \textbf{se} {udala  [\dots].} \\
{} {} my be.3\textsc{sg} mother \textsc{refl} marry.\textsc{ptcp}.\textsc{sg}.\textsc{f} \\
\glt ‘[\dots] my mother got married  [\dots].’
\hfill (VI)
\end{exe}
}

\subsubsection{Split phrases}
\label{Split phrases}
Typical split phrases are subject noun phrases, consisting of a possessive attribute and a noun as in (\ref{(9.30)}). This kind of phrase splitting is considered uncontroversial not only in standard Croatian, but also in standard Bosnian and Serbian.

\protectedex{\begin{exe}\ex\label{(9.30)}
\gll  [\dots] \minsp{[} moja  \textbf{je} majka]  mene  rodila  u  {bolnici  [\dots].}  \\
{} {} my be.3\textsc{sg} mother me.\textsc{acc} birth.\textsc{ptcp}.\textsc{sg}.\textsc{f} in hospital \\
\glt ‘[\dots] my mother gave birth to me in the hospital  [\dots].’
\hfill (BR)
\end{exe}
}

\noindent Furthermore, in the data we find split adverb phrases, similar to (\ref{(9.32)}). This kind of phrase splitting is also found in standard BCS varieties.

\protectedex{\begin{exe}\ex\label{(9.32)}
\gll  [\dots] \minsp{[} vrlo  \textbf{su} rijetko]  išli  u  {džamiju  [\dots].} \\
{} {} very be.3\textsc{pl}  rarely go.\textsc{ptcp}.\textsc{pl}.\textsc{m} in mosque  \\
\glt ‘[\dots] they very rarely went to the mosque  [\dots].’
\hfill (IL)
\end{exe}
}

\noindent However, splitting is not restricted to NP subject and adverb phrases only. In (\ref{(9.31)}) the verbal CL \textit{su} ‘are’ splits a modifier in the prepositional phrase \textit{na istim} ‘on same’ from its noun \textit{linijama} ‘lines’.

\protectedex{\begin{exe}\ex\label{(9.31)}
\gll Svaki  puta  kad  zovem \minsp{[} na istim  \textbf{su} {linijama]  [\dots].} \\
every time when call.1\textsc{prs}  {} on same be.3\textsc{prs}  lines \\
\glt  ‘Every time I call, they are on the same lines (of front)  [\dots]’
\hfill (DO)
\end{exe}
}

\noindent The example above clearly contradicts \citet[436]{RadanovicKocic96}, who claims that a sentence is ungrammatical when a CL is placed between a noun and its modifier in a prepositional phrase. 

\subsubsection{Clitic position and phrase splitting}

The most frequent CL position within a split phrase is after the first stressed word ($N=46$). However, in the corpus of spoken Bosnian there are two cases where phrases which consist of more than two stressed words are split. In those utterances the verbal CL \textit{je} is placed after the second stressed word of the phrase, as in (\ref{(9.33)}).

\protectedex{\begin{exe}\ex\label{(9.33)}
\gll  [\dots] i  \minsp{[} još  mnogo  \textbf{je}  toga]  {izgorjelo  [\dots].} \\
{} and {} more much be.3\textsc{sg} that burn.down.\textsc{ptcp}.\textsc{sg}.\textsc{n} \\
\glt ‘[\dots] and much more of that burned down  [\dots].’
\hfill  (KR)
\end{exe}
}

\noindent When the CL is placed after the first word in the initial phrase, its position is naturally  $\text{2P}=\text{2W}$. However, split phrases (even prepositional phrases) which are not initial are also possible, as shown in (\ref{(9.34)}).

\protectedex{\begin{exe}\ex\label{(9.34)}
\gll  [\dots] \minsp{[} svako]\textsubscript{phrase1} \minsp{[} na svom \textbf{je} koritu]\textsubscript{phrase2} {jači  [\dots].}  \\
  {} {} everybody {} on own be.3\textsc{sg} trough stronger \\
\glt ‘[\dots] everyone is stronger on his own trough (on his own territory)  [\dots].’  \\
\strut\hfill (VI)
\end{exe}
}

\noindent Since phrase splitting is possible in DP, it is not necessarily motivated by 2P. However, this phenomenon is very rare, as in the whole corpus we have only six cases of  non-initial phrase splitting. 

\section{Diaclisis}
\label{Diaclisis:9}

In this section we discuss the attested cases of diaclisis. Twenty-two such utterances are attested, three of which contain a matrix verb and its complement (pseudodiaclisis). As these numbers are small we restrict ourselves to some general observations without analysing frequencies. We identified the following combinations of CLs which do not form a cluster:
\begin{enumerate}
	
\item QUEST, V(erbal), REFL
\item V(erbal), PRON\_dat, REFL
\item V(erbal)\_je, REFL, PRON\_dat
\item QUEST, V(erbal)
\item V(erbal)\_je, PRON\_acc
\item V(erbal), PRON\_dat
\item V(erbal)\_je, REFL 
\item V(erbal), REFL

\end{enumerate}

We see that diaclisis and pseudodiaclisis always involve an interaction between a verbal CL and another CL type, most frequently a reflexive ($N=17$). Further, we observe that two (\ref{(9.35)}) or three CLs (\ref{(9.36)}) can appear in diaclisis. In the latter case two of them clusterise.

\begin{exe}\ex\label{(9.35)}
\gll  [\dots] po  gradovima  \textbf{su}  predsednici opština  \textbf{se}  odjednom {opredjeljivali   [\dots]} \\
{} in cities be.3\textsc{pl}  presidents counties \textsc{refl}  suddenly determine.\textsc{ptcp}.\textsc{pl}.\textsc{m} \\
\glt ‘[\dots] in the cities, the county presidents were suddenly determining  [\dots].’ \\
\strut\hfill (BH)

\ex\label{(9.36)}
\gll Da  \textbf{li}  Bosna  i  Hercegovina  \textbf{će}  \textbf{se}  {osamostaliti  [\dots].} \\
That  \textsc{q}  Bosnia and Herzegovina \textsc{fut}.3\textsc{sg}  \textsc{refl}  become.independent.\textsc{inf} \\
\glt ‘Will Bosnia and Herzegovina become independent  [\dots].’
\hfill (DJ)
\end{exe}

\noindent As mentioned in Section \ref{Diaclisis and pseudodiaclisis} we use the term pseudodiaclisis for matrix-em\-bed\-ding structures in which CC does not occur and a CL is present in the matrix. We found only three clear cases of pseudodiaclisis, including:

\protectedex{\begin{exe}\ex\label{(9.37)}
\gll  [\dots] pa  \textbf{sam}\textsubscript{1}  uspio\textsubscript{1}  \textbf{se}\textsubscript{2}  izvuć’\textsubscript{2}  kroz  {bašče  [\dots].} \\
{} so be.1\textsc{sg}  manage.\textsc{ptcp}.\textsc{sg}.\textsc{m} \textsc{refl}  extract.\textsc{inf}  through gardens \\
\glt ‘[\dots] so I managed to get myself out through the gardens  [\dots].’
\hfill (BG1)
\end{exe}
}

\noindent Note that we did not annotate pseudodiaclisis with \textit{da}-complements. Interestingly, pseudodiaclisis was also attested in a CC utterance. In (\ref{(9.38)}) the dative pronominal CL climbs from its infinitive complement, as it is placed before the particle \textit{i}. However, it does not form a cluster with the verbal CL in the matrix. The possibility of such cases has not been reported before.

\protectedex{\begin{exe}\ex\label{(9.38)}
\gll  [\dots] evo ja \textbf{sam}\textsubscript{1} trebao\textsubscript{1} \textbf{vam}\textsubscript{2} i donjet\textsubscript{2} baš {pismo  [\dots].} \\
{} here I be.1\textsc{sg} need.\textsc{ptcp}.\textsc{sg}.\textsc{m} you.\textsc{dat} bring.\textsc{inf} really letter \\
\glt ‘[\dots] here I should really have brought you the letter  [\dots].’ 
\hfill (BR)
\end{exe}
}

\noindent Finally, it is important to note that (pseudo)diaclisis does not seem to be linked to any fluencemes, i.e. specific structures typical of spoken language, which we discuss in the following section.

\section{Impact of certain syntactic structures on clitic placement}
\label{Impact}
\subsection{Impact of structures occurring before clitics}

As the first to address the impact of disfluency and structures typical of spoken language on CL positioning, we are able to present a few observations. Table \ref{T.9.4} summarises the frequencies of individual types of occurrences.

\begin{table}
\caption{Special syntactic structures occurring before CL placement in the data in non-anonymised utterances. Values in brackets are frequencies relative to the frequency of a particular placement type.\label{T.9.4}}
\begin{tabularx}{\textwidth}{lYYYYY}
\lsptoprule
Type of structure & 1P\footnote{$N=29$} & \multicolumn{2}{c}{2P\footnote{$N=3619$}} & \multicolumn{2}{c}{DP\footnote{$N=181$}}\\\cmidrule(lr){3-4}\cmidrule(lr){5-6}
&& $n$ & \% & $n$ & \%\\\midrule
repetition identical & 1 &32 & 1.9 & 2 & 1.1 \\
repetition partial   & 0 &20 & 1.2 &3  & 1.6 \\
aposiopesis & 0 & 72 & 1.9 & 4 & 2.2 \\
anacoluthon & 0 & 74 & 2.0 & 6 & 3.3 \\
retrospective EDU & 7 & 213 & 5.8 & 18 & 9.9 \\
inserted clause & 5 &120 & 3.3 & 0 & 0.0 \\
rendered topic & 0 & 56 & 1.5& 2 & 1.1 \\
DSE & 8 &390 & 10.7 & 18 & 9.9 \\
substitution & 0 & 30 & 0.8 &1 & 0.5 \\
false start & 0 & 42 & 1.1 &3 & 1.6 \\
ellipsis & 1 & 83 & 2.2 &3 & 1.6 \\
\tablevspace
total & 22 & 1132 & 31.2 &42 & 23.2 \\
\lspbottomrule
\end{tabularx}
\end{table}

We annotated inserted clauses, two types of repetition, false starts, substitutions, rendered topics, retrospective EDUs, DSEs, omissions, and anacolutha. This is a necessary step for determining the position of the CLs in the clause correctly. 

In the next lines we refer to the same sample as in Section \ref{Positioning of single clitics and clitic clusters}, so that we can address the length of constituents. Most types of annotated structures are quite rare in the data, mostly accounting for around 1\% of observations. Only retrospective EDUs and DSEs cross the threshold of 5\%. All of the special syntactic structures appear in the context of DP or 2P. However, no difference in the relative frequency of the two types of placement can be inferred from the data. Only retrospective EDUs occur twice as often in DP clauses than in 2P clauses, and could therefore be a potential topic for further study.

\protectedex{\begin{exe}\ex\label{(9.X14)}
\gll Ali u školi tako, \minsp{[} u gimnaziji,] mnogo \textbf{je} bilo Muslimana  [\dots]. \\
but in school so {} in secondary.school many be.3\textsc{sg} be.\textsc{ptcp}.\textsc{sg}.\textsc{n} Muslims \\
\glt ‘But in school like, in secondary school, there were many Muslims  [\dots].’ \\
\hfill (TZ)
\end{exe}
}

\noindent With respect to 1P, we observe that CLs take it when they are preceded by inserted clauses ($N=5$; example (\ref{(9.X15)})), DSEs ($N=8$) or retrospectives ($N=7$; example (\ref{(9.40)})).

\begin{exe}\ex\label{(9.X15)}
\gll  [\dots] niko \minsp{[} ko god dobije tu vojnu obavezu] \textbf{se} ne treba {javljat  [\dots].} \\
{} nobody {} who ever get.3\textsc{prs} that military obligation \textsc{refl} \textsc{neg} need.3\textsc{prs} apply.\textsc{inf} \\
\glt ‘[\dots] nobody who gets invited to military service ever has to apply  [\dots].’ \\
\hfill (VI)

\ex\label{(9.40)}
\gll Jedan drug, \minsp{[} Musliman,] \textbf{me} \textbf{je} {zvao  [\dots].}  \\
one friend {} muslim me.\textsc{acc} be.3\textsc{sg} call.\textsc{ptcp}.\textsc{sg}.\textsc{m} \\
\glt ‘One friend, a Muslim, called me  [\dots].’ 
\hfill (TZ)
\end{exe}

\noindent We do not observe instances of absolute 1P defined as a true sentence-initial position. From this result, we can conclude that insertions, DSEs and retrospectives are the main triggers for 1P. This means that 1P is restricted to syntactic structures typical of spoken language.

When it comes to the usage of the question marker \textit{li}, it is worth mentioning that speakers tend to avoid potential delays in its placement caused e.g. by long phrases, insertions and special syntactic structures. In fact, not a single insertion, special syntactic structure or any other element or category which could endanger the 2P of \textit{li} is found. Only DSEs are used twice as in  (\ref{(9.13)}), which has already been discussed in Section \ref{Distribution of clitics in the corpus of spoken Bosnian}:

\protectedex{\begin{exe}\exr{(9.13)}
\gll E, sad da \textbf{l’} \textbf{će} te biti mržnje, da \textbf{l’} neće, ne {znam [\dots]} \\
  eh now that \textsc{q} \textsc{fut}.3\textsc{sg} this be.\textsc{inf} hatred that \textsc{q} \textsc{neg}.\textsc{fut}.3\textsc{sg} \textsc{neg} know.1\textsc{prs} \\
\glt ‘Eh, now, will there be hatred, or not, I don’t know…’  
\hfill (VI)
\end{exe}
}

\subsection{Positioning of clitics within fluencemes}

As mentioned in Section \ref{Distribution of clitics in the corpus of spoken Bosnian}, 132 CLs in 117 clauses are repeated or substituted with other CLs. Disfluency involving CLs does not seem to have an impact on CL positioning. In repetitions involving CLs ($N=5$ for partial, $N=19$ for identical), the CL is never delayed. In 44 cases of substitutions we find only one CL in DP, which is caused by a very long initial constituent.

\section{Summary}
\label{Summary:9}
We can now answer our research questions presented in Section \ref{RQ:9}:

\begin{enumerate}[label=A\arabic*:]
\item Our data from the corpus of spoken Bosnian do not reveal any major peculiarities in the inventory of CLs. Apart from cases of phonologically reduced forms, which can also be found in dialects, the inventory coincides with the inventory of standard Bosnian and Serbian varieties. We identified only one additional item, the \textsc{refl\textsubscript{2nd}} CL \textit{si}, recognised only by the Croatian standard. This CL is very rare, which is in line with the claims made in the literature.

\item We have seen that in spoken Bosnian the most representative clusters consist of 2 elements (99\%) and those with 3 components are the exception (1\%). Combinations of 4 or more CLs, whether in clusters or in (pseudo)diaclisis, are not found at all. Further, in the corpus of spoken Bosnian language 18 different cluster combinations of 2 CLs and 6 different combinations of 3 CLs are attested. We have shown that the most frequent cluster combinations of two and three CLs include a pronominal CL in the dative or the verbal CL \textit{je}. 

\item Generally, CL ordering in clusters coincides with the ordering found in standard BCS varieties. There are, however, two divergent types of CL cluster ordering. The first involves the verbal CL \textit{je} and the reflexive CL \textit{se}. Interestingly, when these CLs co-occur in a cluster, they are 4 times as likely to be attested in the non-standard CL order, that is, with \textit{je} preceding \textit{se}. The second case of diaphasic variation is the order of the verbal CL \textit{je} and pronominal accusative CLs. Namely, we found examples of the verbal CL \textit{je} preceding pronominal accusative CLs \textit{me} and \textit{ga}, the reverse of the order established in written standard Bosnian (and other standard varieties).

\item Verbal and reflexive CLs are by far the most frequent CLs in the whole corpus. However, the combination of \textit{je} and \textit{se} in a CL cluster is quite rare. We have shown that the Bosnian spoken variety prefers haplologised structures, i.e. where only the reflexive CL \textit{se} occurs while the verbal CL \textit{je} is omitted. Furthermore, haplology of unlikes is attested in every interview of the corpus. In contrast, there is not a single example of suppletion. Moreover, we did not find any occurrences of the pronominal accusative CL \textit{ju}.  

\item In spoken Bosnian 2P is the dominant CL position. Interestingly, we found no statistically significant difference in the placement of single CLs and CL clusters. They both show a strong tendency towards 2P. Moreover, the findings concerning differences in the placement of individual CL types are of major theoretical interest. First, the polar question marker \textit{li} differs from all the other CL types as it is placed in 2P after one short word (2 to 4 graphemes) in 100\% of cases. Its positioning is thus much more unified than the positioning of verbal, pronominal and reflexive CLs. Second, in our corpus 1P is restricted to verbal and reflexive CLs, as neither \textit{li} nor pronominal CLs seem to allow it. Finally, we would like to emphasise that 1P is connected to contexts of insertions, DSEs and retrospectives; i.e., there are no utterances with absolute sentential 1P.

\item As to the nature of 2P in spoken Bosnian, we saw a strong tendency towards 2W: the CL occupies the position after the first word in 77\% of all observations (single CLs and clusters). Our data provide interesting insights into the heaviness of a constituent in the spoken variety. The typical CL position in the clause is after the first word, which is most frequently two graphemes long. The most frequent initial constituent in DP is three graphemes long, but in general its length is not limited, while the most frequent host in DP is four graphemes long, making it longer than the initial constituent. However, even in the case of DP host length is more limited than the length of the initial constituent. The current study provides only statistical tendencies. In our view, two further types of investigations should be undertaken in the future. First, studies based on acoustic data are necessary to allow examination of the role of phonological contour. Second, a study of written language should investigate whether similar statistical regularities can be obtained.\footnote{Although \citet{Reinkowski01} analyses the positioning of CLs in newspapers and magazines, her results cannot be directly compared with ours. In her study, three CL positions are distinguished: initial (after the first word), middle (any position before the predicate, but not immediately after the first word) and final (behind the predicate), which does not coincide with our coding scheme. Additionally, many types of initial constituents allowed in our study fall outside the scope of Reinkowski’s study.}

\item Our analysis of phrase splitting brings new insights, as we see that in the spoken Bosnian variety, like in the written Croatian standard, splitting is possible not only for 2P but also for DP. Further, in spoken Bosnian splitting occurs not only in subject, but also in prepositional phrases, which is in line with the findings of \citet{DiesingZec17} for written Serbian.\footnote{\textcolor{black}{The reader should, however, bear in mind that \citet[9f]{DiesingZec17} differentiate between predicate and argument initial prepositional phases, which a CL can split. The latter was accepted by more than 67\% of participants in the acceptability judgment experiment, but it had very low scores in the production experiment \citep[9f]{DiesingZec17}. Therefore, \citet[11f]{DiesingZec17} ascribe the ungrammatical status to split prepositional arguments in Serbian. In contrast, we do not believe that such structures have an ungrammatical status in the spoken Bosnian variety.}}   Contrary to the observations of some authors, we find that clusters can take part in phrase splitting. The examples attested involve possessive dative. We did not find a single example of an accusative or genitive pronominal CL taking part in splitting. There are also few examples of diaclisis with phrase splitting and even fewer examples of pseudodiaclisis. 

\item We have shown that the analysis of CL positioning in spoken language has to take into consideration disfluency phenomena and some discourse-organising structures because they affect the surface structure of the sentences. They have, however, no direct impact on CL positioning beyond 1P after insertions, DSEs and retrospectives. Neither do they trigger (pseudo)\-di\-a\-cli\-sis. 

The restricted empirical base notwithstanding, we would argue that our small pilot study could serve as the point of departure for future studies on CL positioning in spoken languages, not only in BCS but also beyond. We have prepared a scheme for the annotation of disfluency and other phenomena typical of spoken language, which is a conditio sine qua non for the analysis of CL positioning in spoken language. 
\end{enumerate}
