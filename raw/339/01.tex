\part{Preliminaries}
\label{part1}
\chapter{Introduction and overview}
\label{Introduction with overview}
%CC is a problematic part here
%\stepcounter{footnote}
%\footnote{This study was carried out within the research project `Microvariation of the Pronominal and Auxiliary Clitics in Bosnian, Croatian, and Serbian. Empirical Studies of Spoken Languages, Dialects and Heritage Languages' funded by the Deutsche Forschungsgemeinschaft (HA 2659/6-1, 2015--2018). Furthermore we express our gratitude for the financial support received from BAYHOST.} %Including the footnote into the title repeats the footnote every time the title is referenced (top of the left page), so the standard solution with \protect doesn`t work. We may have to think of something here.
\section{Topic of the book}
The present monograph is a data-oriented, empirical in-depth study of the system of clitics in Bosnian, Croatian, and Serbian.\footnote{For detailed information on clitics see Section \ref{Clitics}.} Clitics are elements which, like affixes, cannot occur freely in a clause but need a host to lean on. The book deals with expressions such as those highlighted in the following examples (the hosts are \textit{spremna} `ready' in example (\ref{(1.1)}) and \textit{jučer} `yesterday' in example (\ref{(1.2)})):

\begin{exe}
\ex\label{(1.1)}
\gll Spremna \textbf{sam} \textbf{mu} \textbf{se} vratiti i {oprostiti [\dots].}\\
ready be.\textsc{1sg} him.\textsc{dat} \textsc{refl} return.\textsc{inf} and forgive.\textsc{inf}\\
\glt `I am ready to return to him and forgive him [\dots].'
\hfill [hrWaC v2.2]

\ex\label{(1.2)}
\gll Jučer \textbf{ga} \textbf{je} Marin snimao na središnjem terenu.\\
yesterday him.\textsc{acc}  be.\textsc{3sg} Marin record.\textsc{ptcp.sg.m} on central site\\
\glt `Yesterday Marin was recording him on the central site.'
\hfill [hrWaC v2.2]
\end{exe}

\noindent Since the seminal work \textit{Über} \textit{ein} \textit{Gesetz} \textit{der} \textit{indogermanischen} \textit{Wortstellung} by the Swiss linguist Jakob \citet{Wackernagel92}, clitics have received continuous attention from linguists.\footnote{\textcolor{black}{\citet{WSM20} is a recent English translation of Wackernagel's (\citeyear{Wackernagel92}) work.}} They are particularly well described in the Romance languages, (Ancient) Greek, and Czech, for instance. 

Cross-linguistically, \textsc{clitics} (CLs) can be defined as ``elements with some of the properties charac­teristic of independent words and some characteristic of affixes, in particular, inflec­­tional affixes within words. Such elements act like single-word syntactic constituents in that they function as heads, arguments, or modifiers within phrases, but like affixes in that they are  ``dependent'', in some way or another, on adjacent words'' \citet[xii]{Zwicky94}.

CLs are interesting for several reasons. First, they have a special phonological structure and they combine features of both syntactic words and affixes, thus blurring the boundary between the morphological and the syntactic system of the language. Second, in languages like Bosnian, Croatian, and Serbian (BCS), which are usually claimed to have so-called free word order, allowing for positional permutations of phrases depending on information structure, CLs differ from other elements with a similar syntactic function in that their position is fixed. The placement of these CLs is usually associated with the left edge of the sentence, the so-called ``second position''.\footnote{For detailed information on the second position see Section \ref{Position of the clitic or the clitic cluster}.} Third, some CLs have non-clitic counterparts which have the same meaning and syntactic function but differ as to word order. They are what \citet[3]{Zwicky77} calls \textsc{special clitics}; i.e. unaccented bound forms which act as ``variant[s] of a stressed free form[s] with the same cognitive meaning and a similar phonological makeup''. This is a hard nut to crack for functional frameworks which tend to explain structures by functional or cognitive mechanisms. The second position effect, however, seems to have a purely formal syntactic and/or prosodic basis. Nevertheless, formal approaches also have to struggle with the idiosyncratic word order behaviour setting CLs apart from other syntactic elements. A major problem is the ordering of the CLs in clusters, where verbal CLs show up in two different positions (CL \textit{je} `is' vs all other verbal CLs in BCS). As \citet[12]{FJL04} argue, ``the study of clitics can shed light on the interfaces between syntactic, morphological, and phonological linguistic representations.'' We should add that they are also an ideal test case for usage-based approaches and/or explanations connected to the notions of repeated morphs, syntactic complexity and long-distance dependencies.

\section{Clitics and microvariation}
\label{Clitics and microvariation}
Our starting point is the observation that there is a high degree of variation in the CL system of Bosnian, Croatian, and Serbian. First, as argued by \citet{WaldenfelsEder16}, these Neo-Štokavian varieties share a largely convergent grammatical, lexical, stylistic, and orthographic basis but show multiple types of variation of a multifactorial nature. Second, there seems to be relevant variation within varieties. We acknowledge that there is a considerable body of research dedicated specifically to CLs in BCS. However, the research on the syntax of Bosnian, Croatian, and Serbian is divided into works with a formal theoretical orientation on the one hand, and descriptive studies on the other. Considering this split, it comes as no surprise that in the literature we find largely contradictory statements concerning the acceptability of certain structures. Moreover, most authors rely exclusively on linguistic intuition and work with constructed examples. In footnotes, authors sometimes admit that the data they are discussing in order to develop certain theoretical claims are either marginal or are rejected right away by other native speakers. 

A good illustration of disagreements as to the grammaticality of certain examples is provided by the question whether CLs can climb out of the so-called \textit{da}-construction.\footnote{For basic information on clitic climbing see Section \ref{Clitic climbing} and for thorough theoretical and empirical data on clitic climbing out of \textit{da}\textsubscript{2} and infinitive complements see Part \ref{part3}.}\textsuperscript{,}\footnote{For basic information on \textit{da}-complements see Section \ref{Types of complements}, and for detailed information on CC out of \textit{da}-complements based on empirical evidence see Chapter \ref{A corpus-based study on CC in da constructions and the raising-control  distinction (Serbian)}.} For instance, the formal linguist \citet[174, 197]{Stjepanovic04} argues that semi-finite \textit{da}\textsubscript{2}-complements and infinitive complements allow climbing in a similar way, as in the following constructed example (\ref{(1.3)}):\footnote{We use the abbreviations Cr, Sr, and Bs for the varieties Croatian, Serbian, and Bosnian, respectively.}\textsuperscript{,}\footnote{Stjepanović uses the label Serbo-Croatian.}\textsuperscript{,}\footnote{In example (\ref{(1.3)}), the pronominal CL \textit{ga} `him', which is generated by the \textit{da}\textsubscript{2}-complement \textit{posjeti} `visits', climbs out and appears in the matrix clause, to the left of the matrix predicate \textit{mora} `must' and/or \textit{želi} `wants'.}

\begin{exe}
\ex\label{(1.3)}
\gll Marija \textbf{ga} mora / želi da  posjeti. \\
Mary him.\textsc{acc} must.\textsc{3prs} {} want.\textsc{3prs} that visit.\textsc{3prs} \\
\glt `Mary has to/wants to visit him.'
\hfill (BCS; \citealt[174]{Stjepanovic04})
\end{exe}

\noindent In contrast, \citet[448]{CavarWilder94} argue that clitic climbing out of finite complements is ``blocked in all dialects'' of BCS. Others, like \citet[146]{Progovac05}, refer to individual variation in the sense that some speakers of Serbian do and others do not accept such sentences. All the above-mentioned authors rely exclusively on constructed examples. This indicates that we not only encounter conscious pre-selection of data best fitting the theoretical claims, but also have to deal with the question of data quality. We agree with \citet[60]{DFZ09} who emphasise that current research on CLs ``has [\dots] relied heavily on native speaker judgments that have been culled primarily from previously published work, or from interrogating native speaker linguists. While these are not uncommon methods in theoretical linguistics, it is well worth augmenting the database with other sources [\dots].''

We focus both on language-internally motivated variation (systemic microvariation) and on selected cases of sociolinguistic microvariation in the diatopic and the diaphasic dimensions. This distinction is meant to capture the fact that there are two basic types of conditioning factors: whereas systemic microvariation is triggered by purely language-internal factors, sociolinguistic microvariation in the narrow sense depends on features relating to space (diatopic dimension: three standard languages, dialects), or to the modes of language use (diaphasic: e.g. standard vs non-standard, written vs spoken language). We do not deal with variation in the language use of social groups (diastratic dimension). Throughout the book we use the terms variation and microvariation interchangeably.

One example of sociolinguistic microvariation concerns the CL of the third person feminine accusative pronoun: whereas the Croatian handbook \textit{Hrvatski jezični savjetnik} by \citet[173]{BHMV99} recommends the form \textit{je}, \citet[74]{HMBO14} favour \textit{ju}; compare the examples in (\ref{(2609)}). 

\begin{exe}\ex\label{(2609)}\begin{xlist}
\ex\label{(2609a)}
\gll Vidjela \textit{sam} \textit{je}. \\
see\textsc{.ptcp.sg.f} be\textsc{.1sg} her\textsc{.acc} \\
\glt `I saw her.'
\hfill (Cr; \citealt[173]{BHMV99})
\ex\label{(2609b)}
\gll Vidim \textit{ju}. \\
see\textsc{.1prs} her\textsc{.acc} \\
\glt `I see her.'
\hfill (Cr; \citealt[74]{HMBO14})
\end{xlist}
\end{exe}

\noindent We are interested both in the prescriptive norms of the three standard languages, Bosnian, Croatian, and Serbian, and in real language usage as found in web corpora and in dialects\textcolor{black}{, that is, observable language data}.\footnote{\textcolor{black}{The type, quality, and quantity of data available for different variants of BCS vary considerably. We discuss these topics in Chapter \ref{Corpora for Bosnian, Croatian and Serbian} \textcolor{black}{as well as} in Sections \ref{Available data} and \ref{Data Quality}}.} The main focus is on the dominant varieties based on Neo-Štokavian, but we also allow for short side-glances at other dialects like Old-/Middle \textcolor{black}{(Torlak)} Štokavian, Kajkavian and Čakavian. Throughout this book, we use the label Bosnian\slash Croatian\slash Serbian (BCS) to refer to the Štokavian language usage common to the varieties used in Croatia, Serbia, and Bosnia-Herzegovina. We do not examine standard language use in Montenegro because first, efforts to create a standard for the variety spoken there are still in their infancy and, second, there are considerably fewer resources and specific studies. When we refer to language structures as codified in national handbooks we use the single label: Croatian, Serbian or Bosnian. The same holds for language usage patterns found in web corpora; i.e. in texts from the top-level domains .hr, .sr, and .ba. The question whether we are dealing with independent languages or with national variants of a so-called polycentric language is not relevant to our study. 

\textcolor{black}{As the focus of the present study is microvariation within the CL system of Serbian, Croatian, and Bosnian and not a cross-linguistic typology of CL systems, we are quite cautious with regard to data and findings from other languages. We agree with \citet[4]{RosenHana17} that in many respects, CLs in different languages are similar with regard to inventory, positioning, and internal order (CL clusters), but the mix of properties in each language may be unique. Therefore, we will focus on Serbian, Croatian, and Bosnian data and refrain from conjectures concerning larger groups of languages. Thus, we will not comment on South Slavic or Slavic in general. Readers with a particular interest in contrastive studies can consult the existing, rich literature: the general handbook of \citet{FranksKing00} on systems of Slavic CLs, \citet{Bozovic21} on clustering phenomena in South Slavic, and \citet{Migdalski16} on second position cliticisation in Slavic, which emphasises the diachronic perspective. Although we acknowledge findings and theoretical insights on CLs in languages including Russian, Polish, Bulgarian, Slovene, and Portuguese, we will use these data only in exceptional cases to generate research hypotheses. In the case of Portuguese, we have come across studies showing that register may have a significant effect on clitic climbing. This observation was used to generate research hypotheses for our corpus-based study on clitic climbing in infinitive complements in relation to the diaphasic variation presented in Chapter \ref{A corpus-based study on clitic climbing in infinitive complements in relation to the raising-control dichotomy and diaphasic variation (Croatian)}. No comparison with clitic climbing in Portuguese is offered. 
Similarly, we consider the linguistic material from Russian and Bulgarian, e.g. in \citet{Landau00, Landau04, Landau13}, to be irrelevant for our study because Russian CLs differ fundamentally from Serbian, Croatian, and Bosnian CLs with respect to inventory (no CL pronouns, no CL reflexive), position (second position not obligatory), and cluster formation (not present). Polish does have CL pronouns and a reflexive, but these allow both second position and verb-adjacent position. Further differences include the presence of the conditional CL \textit{by} and past tense endings, but no present tense forms of the copula/auxiliary are available. Moreover, there are no CL clusters in Polish. Bulgarian is utterly different as it shows CL doubling, while Slovene has proclitics. We make an exception for Czech, taking it into consideration in the case of clitic climbing, which is exceptionally well described for this language. The Czech CL system is highly comparable indeed as it shares with BCS its CL inventory (verbal, reflexive, and pronominal CLs) and some other crucial features such as cluster formation, second position effects, and morphological processes within the cluster.}\footnote{\textcolor{black}{Czech clitics can phonologically encliticise or procliticize \citep[295 and citations therein]{Lenertova01}, in contrast to BCS CLs. We think that this factor is irrelevant for the phenomenon of CC, which is observed in languages with phonologically diverse types of CLs. See Chapters \ref{Approaches to clitic climbing} and \ref{Constraints on clitic climbing in Czech compared to Bosnian, Croatian and Serbian (theory and observations)} for further discussion on this matter.}}

\section{Empirical orientation}
\label{Empirical orientation}
The monograph offers an account of the range of language-internal and sociolinguistic microvariation in this component of the language system by integrating large amounts of data and findings from descriptive and prescriptive works on the one hand, and from theoretically oriented studies on the other. A selection of structures is tested in an array of corpus and experimental studies. Our aim is to bridge the gulf between fine-grained description and syntactic generalisation by putting traditional work by Croatian, Serbian, and Bosnian scholars on an equal footing with general linguistic studies with a purely theoretical orientation. 

As to the empirical approach chosen, the current work is usage-based oriented and we use triangulation of methods: intuition/theory – observation – experiment.\footnote{For details on the empirical approach chosen see Chapter \ref{Empirical approach to clitics in BCS}.} The first step always involves a thorough analysis of the whole body of existing research literature, independently of the respective theoretical framework, which is quite unique in syntax research. We also systematically document the approaches advocated by the leading normativists of Croatian, Serbian, and Bosnian who frequently discuss or evaluate variants.

\textcolor{black}{The state-of-the-art literature review shows that with the exception of a few studies the previously analysed data lack precise characteristics and descriptions of amount and origin, which calls into question their replicability.\footnote{\textcolor{black}{ Under ``with exceptions'' we refer to the studies of \citet{DFZ09}, \citet{ZFD17}, and \citet{DiesingZec17}; see Section \ref{Barriers and delayed placement of clitics (DP)}.}} We have not come across many studies on CLs in BCS combining the theoretical literature with either corpus or experimental evidence. Thus, we conclude that in contrast to our study, most of the previously undertaken efforts did not include the kinds of standard types of empirical evidence currently acknowledged in linguistics.
Therefore, we believe it is necessary to verify the often contradictory theoretical claims against empirical data collected primarily from corpora – our first source of observations.
} 
 Since corpora allow the application of statistical methods, some hypotheses can be verified already at this stage. A selection of hypotheses concerning factors determining variation in the usage of CLs, formulated on the basis of corpus material, are further tested in acceptability judgment experiment where the level of control can be adjusted for individual factors. We are convinced that corpora as recordings of natural language production can be supplemented with experimental data such as acceptability judgment data because they both provide evidence about syntax. However, they offer different kinds of evidence: while corpora reflect language production, acceptability data primarily reflect language comprehension. \textcolor{black}{The corroborating results from studies using different kind of data and methods provide more insightful and reliable linguistic evidence in comparison to studies using only one type of data.}

Our aim is to give an account of the range of variation encountered in the real usage of the CL systems of Bosnian, Croatian, and Serbian. We restrict ourselves to the three main types of CLs, namely pronominal, reflexive and verbal CLs, thus excluding  the polar interrogative marker \textit{li}. Finally, in our empirical studies we pay only marginal attention to the question of phrase splitting as this phenomenon has already been studied extensively elsewhere.\footnote{For basic information on phrase splitting see Section \ref{Phrase splitting (2W)}.}

\section{Structure of the volume}

The parts \ref{part1},  \ref{part2}, and  \ref{part3} are the three main parts which form the core structure of this monograph.
 
Part \ref{part1} covers Chapters \ref{Introduction with overview}--\ref{Corpora for Bosnian, Croatian and Serbian}. Chapter \ref{Our terms and concepts} introduces the most important concepts and terms used in the monograph, presents the parameters of variation, \textcolor{black}{and discusses} the most influential works that examine \textcolor{black}{BCS} CLs within formal theoretical frameworks. Departing from theoretical approaches to CLs which are \textcolor{black}{usually} based on \textcolor{black}{limited} numbers of constructed examples, we decided to investigate the phenomena of interest empirically. Our approach is explained in detail in Chapter \ref{Empirical approach to clitics in BCS}. In the subsequent Chapter \ref{Corpora for Bosnian, Croatian and Serbian} we first present electronically stored corpora that are easily accessible to the research community and then discuss which of them are the most suitable for our empirical studies. 

\textcolor{black}{As we explain in Chapter \ref{Parameters of variation: Inventory, internal organisation of cluster and position},} Part \ref{part2} focuses on the parameters of microvariation identified in Chapter \ref{Our terms and concepts}. The structure of Chapter \ref{Clitics and variation in grammaticography and related work} and Chapter \ref{Clitics in dialects} follows the parameters of variation identified in Chapter \ref{Our terms and concepts}. In Chapter \ref{Clitics and variation in grammaticography and related work} the parameters of variation are explored in detail at the level of standard languages. The goal of that chapter is to identify possible diatopic variation between BCS standard varieties, i.e. between different standard languages. Furthermore, where information is available in the literature, we comment on diaphasic variation within one BCS (standard) variety. However, the process of identifying variation is based solely on descriptions in the literature. Nevertheless, Chapter \ref{Clitics in dialects} offers deeper analysis of the identified parameters of variation with respect to diatopic variation. Moreover, we elaborate on some factors of variation such as CL inventory, CL placement, and morphological processes within the CL cluster, based on the empirical data in Chapter \ref{Clitics in a corpus of a spoken variety}.

The only parameter of variation which we do not examine in Chapter \ref{Clitics and variation in grammaticography and related work} is clitic climbing. The large number of mainly theoretical studies on CLs notwithstanding, clitic climbing has not received much attention. Moreover, as a phenomenon it has also been overlooked in grammar books and related works written by native authors. This is why we decided to dedicate one whole part of the book to this topic. Therefore, drawing on empirical studies on clitic climbing in Czech, in Chapters \ref{Approaches to clitic climbing}--\ref{Experimental study on constraints on clitic climbing out of infinitive complements} we study clitic climbing mechanisms in some detail and propose a series of constraints it is subject to. Finally, we offer an explanation for constraints on clitic climbing in terms of complexity in Chapter \ref{On the heterogeneous nature of constraints on clitic climbing: complexity effects}. \textcolor{black}{Our data-driven study gives new insights into the understanding of clitic climbing achieved through probabilistic modelling. We hope that in the future this can also feed into existing formal theories of clitic climbing.} Chapter \ref{Summary and outlook} recapitulates the main findings and gives an outlook for further studies.
