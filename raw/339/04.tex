\chapter{Corpora for Bosnian, Croatian, and Serbian}
\label{Corpora for Bosnian, Croatian and Serbian}
\section{Introduction}
The goal of this chapter is to explain the choice of corpora used to extract linguistic evidence, formulate further hypotheses and find examples of the language structures in focus. As explained in the previous chapter our approach to CLs in BCS is primarily empirical and not oriented towards any particular working grammatical framework. Therefore, the role of data in our research is not limited to extracting examples confirming or contradicting the existing theories: we principally use corpora inductively to identify patterns which form regularities and exceptions concerning the behaviour of CLs. 

The rest of this chapter is structured as follows: Section \ref{Some remarks on corpus types} discusses the double meaning of the term \textit{corpus} in linguistics and briefly summarises types of electronic text sources. In Section \ref{Overview of traditionally compiled corpora for BCS} we present an overview of the most important corpora for BCS with a special focus on web corpora in Section \ref{WAC}. Section \ref{Corpora for spoken BCS} discusses available corpora of spoken language. Section \ref{Discussion} presents some concluding remarks. 

\section{Some remarks on corpus types}
\label{Some remarks on corpus types}
\subsection{The meanings of the term \textit{corpus} in modern linguistics}

In order to analyse the advantages and disadvantages of the available corpora we should first review the term corpus and discuss types of corpora, as this kind of collection of data forms a very heterogeneous group. Let us start from a very broad characteristic of the term corpus given by \citet[21]{McEneryWilson96}:
\begin{quotation} In principle, any collection of more than one text can be called a corpus: the term `corpus' is simply the Latin for `body', hence a corpus may be defined as any body of text. It need imply nothing more. But the term `corpus' when used in the context of modern linguistics tends most frequently to have more specific connotations than this simple definition provides for. These may be considered under four main headings:
	\begin{itemize}
		\item sampling and representativeness,
 \item finite size,
 \item machine-readable form,
 \item a standard reference.
 	\end{itemize}
\end{quotation}

Thus, a corpus is a collection of naturally-occurring language documentation, gathered with respect to some particular framework. This framework can be either oriented towards characterising a particular type of language \citet[171]{Sinclair91} and results in both general reference corpora and small, specialised corpora, or towards studying a particular linguistic phenomenon, in which case often only particular types of structures are stored. 

In order not to confuse these two approaches to data collection, we call the former type corpus\textsubscript{1}, whereas a corpus constructed in order to test a research hypothesis will be called corpus\textsubscript{2}.

These two approaches to data collection should be kept separate, as a corpus\textsubscript{1} can be a potential source of material for a corpus\textsubscript{2}. A corpus\textsubscript{2}, however, can rarely be used in the function of a corpus\textsubscript{1}, unless the linguistic phenomenon under study is a certain variety of a language as a whole.

	Hence, in the present chapter we focus on the available corpora in the broader sense of corpus\textsubscript{1} in order to find out which of them can best serve the extraction of a representative collection of naturally occurring utterances, relevant to our project on CLs in BCS.

\subsection{Types of corpora as text collections}

When describing corpora as text collections, it is important to cover several parameters. First, whether the corpus contains written or spoken texts. Secondly, whether the corpus is monolingual or multilingual. Among multilingual varieties there are parallel corpora – where one piece of semantic content is represented in several languages (source language and one or more translations), and comparable corpora – where texts with similar characteristics (register, lexical content, style, genre) have been collected from several languages. In the case of comparable corpora it is important for the proportions of features according to which the stratification takes place to be preserved in all languages.

In the present study we focus on the microvariation of CLs in BCS, which excludes the use of parallel corpora as a source because of the interference from the source text. This is because “phenomena pertaining to the make-up of the source text tend to be transferred to the target text” \citep[275]{Toury95}. Therefore, in the next part we will focus on the description of monolingual corpora of BCS, including both written and spoken varieties.

The features important for linguistic studies are size, content and sampling principles, which allow to assess for what language varieties the given corpus is representative and which research questions can be studied with its help. In this respect modern corpora can be divided into traditional sources which follow certain priorly defined principles and criteria (stratified sample), and opportunistic collections compiled from what is easily available and accessible (convenience sample). Stratified samples are typical of reference and monitor corpora, whose task is to reflect the “real-life” state of language, and of small corpora compiled for specific research purposes.

However, most publicly available corpora are nowadays collected for purposes of computational linguistics and here corpus size is the deciding factor. As \citet[120]{ManningSchutze99} point out, “in Statistical NLP, one commonly receives as a corpus a certain amount of data from a certain domain of interests, without having any say how it is constructed.” The most popular solution is crawling the internet. We elaborate on this approach in Section \ref{WAC}.

In the overview below, we will describe both the smaller traditional and the more impressive in size opportunistic sources, and discuss their utility for studying microvation of CLs in BCS. The overview represents the state-of-the-art for the period 2015--2018 when the study data were retrieved. Due to rapid technological development, mainly the increase of storage possibilities and computational capacities, new resources appear very quickly, so some sources available now are not mentioned.

\section{Overview of traditionally compiled corpora for BCS}
\label{Overview of traditionally compiled corpora for BCS}
\subsection{Bosnian corpora}

Unfortunately, even now the range of available text collections for the Bosnian language is rather narrow. 

\begin{table}
\caption{Bosnian corpora}
\label{table:bos1}
\centering
\begin{tabularx}{.97\textwidth}{lll>{\raggedright\arraybackslash}Xlll}
\lsptoprule
Corpus	&	Words		&	Text age	&	Text types												&	Lemm.		&	PoS	&	MSD \\
\midrule
OCBT		&	1,500,000	&	1989--1997		&fiction, essays, newspapers, children's books, Islamic texts, legal texts, folklore 	&	yes	&	no	&	no\\
\lspbottomrule
\end{tabularx}
\end{table}

The first widely available digital corpus of Bosnian is the Oslo Corpus of Bosnian Texts (OCBT, \citealt{Santos98}). It was the only larger corpus of Bosnian for a long time until recently, when bsWaC and SETimes \citep[30]{LjubesicKlubicka14} were compiled. 

OCBT was created as a joint project of the Department for East European and Oriental Studies and the Text Laboratory of the University of Oslo. The main goal was to make Bosnian texts from the period 1989--1997 available for linguistic research \citep{Santos98}. The corpus is accessible for online search after registering for a free account. Table \ref{table:bos1} summarises the most important facts about the OCBT.\footnote{MSD stands for ``morphosyntactic descriptioons''.} The OCBT contains written texts belonging to different genres and its estimated size is 1,500,000 words \citep{Santos98}. It is searchable online through the Corpus Query Processor (CQP). The interface is rather simple, as it provides only concordances of words, phrases, suffixes, prefixes, or their combinations \citep{Santos98}. Functionalities often considered a standard, such as sorting and filtering, are not available. As to metainformation, the origins of texts are provided, which is a great advantage, but no morphosyntactic annotation is applied. Moreover, the OCBT is useful mostly for studies of standard language, as can be read from the content description in Table \ref{table:bos1}.

Summing up, it appears that the only traditionally compiled, monolingual source of Bosnian is the OCBT and therefore this language is definitely under-resourced. The Oslo Corpus of Bosnian Texts is certainly quite diversified with respect to the functional styles which it includes, but the texts are quite old, as they originate from the first development phase of standard Bosnian. However, the biggest objection to using this corpus in our study is that morphosyntactic annotation, which would allow efficient searching, is not available.

\subsection{Croatian corpora}

The two publicly available corpora of standard Croatian are \textit{Hrvatski nacionalni korpus} \citep{Tadic02, Tadic20}, that is, the Croatian National Corpus (CNC) and \textit{Hrvatska jezična riznica} \citep{CavarRoncevic11, CavarBrozovic12}, that is, the  Croatian Language Corpus (Riznica). Table \ref{table:cro1} gives basic information about these sources.

\begin{table}
\caption{Croatian corpora}
\label{table:cro1}
\centering
\begin{tabularx}{.97\textwidth}{ll>{\raggedright\arraybackslash}X>{\raggedright\arraybackslash}Xlll}
\lsptoprule
Corpus	&	Words		&	Text age	&	Text types												&	Lemm.		&	PoS	&	MSD \\
\midrule
CNC		&	2,559,160	&	1990--								&Croatian literature, journals, and newspapers, booklets, official letters 	&	yes	&	yes	&	yes\\
Riznica	&	84,536,657	&	since 2\textsuperscript{nd} half of the 19\textsuperscript{th} century		&Croatian literature, non-fiction: scientific publications, online journals, and newspapers 	&	yes	&	yes	&	yes\\
\lspbottomrule
\end{tabularx}
\end{table}

CNC was the first widely available digital corpus of contemporary Croatian language. The most current, third version comprises 216.8 million tokens. It is available via a NoSketchEngine interface, which allows complex queries to be constructed using the syntax of Corpus Query Languages (CQL).

The main goal of the project initiated at the Department of Linguistics (University of Zagreb) was to construct a corpus which would be big enough to cover the whole scope of standard Croatian in order to generate a primary source of linguistic data for lexicographical, orthographical, morphological, syntactic, and semantic research on contemporary Croatian (\citealt[339]{Tadic98}). During compilation, special attention was paid to the desired ideal corpus structure, according to which reference corpora such as the British National Corpus are built. This aim has not been achieved yet, mainly because a spoken corpus has yet to be created \parencites[346]{Tadic98}[446]{Tadic02}.

The sampling frame was based on a variety of media, text types, genres, fields, and topics \citep[442]{Tadic02} according to the standards for text typologies \citep{eagles96}. It is important to emphasise that only texts from 1990 on were incorporated into the corpus since the Croatian language could only develop without any obstructions starting from that period \citep[442]{Tadic02}.

The CNC is lemmatised and morphosyntactically annotated. However, from the user perspective we have to make the objection that neither is it easy to find a description of  the morphosyntactic tagset in use, nor to follow how attributes should be built.\footnote{\textcolor{black}{For a thorough insight into tagset visit} \url{http://nl.ijs.si/ME/V4/msd/html/msd-hr.html}.} In many cases the theoretically available opportunities fail, because for example overly long concordances in CQL seem to be too complex for the corpus.

Riznica was compiled at the Institute of Croatian Language and Linguistics in Zagreb.\footnote{\textcolor{black}{Over the period of the current project, Riznica went through an incredible change concerning metainformation and the corpus manager.}} The goal was to produce a publicly available linguistic resource on the Croatian language and to provide crucial information about the Croatian language standard \citep[51]{CavarBrozovic12}. The collection covers texts written in various functional domains and genres and dated from the second half of the 19\textsuperscript{th} century onwards.\footnote{The official description of the corpus states that it is compiled from texts from the period of the standardisation of Croatian (\citealt[52]{CavarBrozovic12}, \url{http://riznica.ihjj.hr/dokumentacija/index.en.html}). However, texts from previous centuries, such as \textit{Planine} by P. Zoranić, or \textit{Judita} by M. Marulić, may be found during querying.} It includes essential Croatian literature, including poetry, scientific publications from various domains, online journals, and newspapers. In contrast to the CNC, Riznica also contains translated literature from outstanding Croatian translators.\footnote{\textcolor{black}{For more information visit} \url{http://riznica.ihjj.hr/dokumentacija/index.en.html}.} Because of its rigorously selected texts, Riznica as a corpus could be an interesting object of linguistic research as long as the intention was to explore how the desired structures should behave in proper, standardised Croatian. Riznica only became an attractive source of standardised language in 2018, that is in the last phase of our study, when its new release allowed for part-of-speech searches as well as for queries concerning morphosyntactic structures.\footnote{According to the official description of the corpus, \textcolor{black}{the first release} should be annotated for lemma and word-class \citep[52]{CavarBrozovic12}; however, when queries are made through \url{http://riznica.ihjj.hr/index.hr.html} that kind of functionality is not available. The newest version was annotated with ReLDI tagger \citep{LjubesicErjavec16}, and is available \textcolor{black}{via CLARIN-Sl} at \url{https://www.clarin.si/noske/run.cgi/corp\_info?corpname=riznica\&struct\_attr\_stats=1}.}

\subsection{Serbian corpora}

Even though the range of available corpora of Serbian is not that narrow, in the literature we often find statements that Serbian is an under-resourced language with respect to the availability of electronic corpora \parencites[685]{Dobric12}[4106]{BSM14}. Table \ref{table:ser1} gives an overview of the best-known available digital corpora of the Serbian language. 

\begin{table}
\caption{Serbian corpora}
\label{table:ser1}
\centering
\begin{tabularx}{.97\textwidth}{lY>{\raggedright\arraybackslash}X>{\raggedright\arraybackslash}Xlll}
\lsptoprule
Corpus	&	\multicolumn{1}{l}{Words}		&	Text age	&	Text types												&	Lemm.		&	PoS	&	MSD \\
\midrule
NETK & 22,000,000&since 1920&literature, scientific, and journalistic texts&no&no&no\\
SrpKor2003&22,000,000&since 1920&literature, scientific, and journalistic texts&no&no&no\\
SrpKor2013&122,000,000&from 20\textsuperscript{th} and 21\textsuperscript{st} century&literature, scientific, publicistic, administrative, and other texts&yes&yes&no\\
SrpLemKor&3,763,352&from 20\textsuperscript{th} and 21\textsuperscript{st} century&general fiction, literature, scientific, and legislative texts&yes&yes&no\\
KSJ&11,000,000&from 12\textsuperscript{th} up to 20\textsuperscript{th} century&literature, scientific, and legislative texts&yes&yes&yes\\
\lspbottomrule
\end{tabularx}
\end{table}

\textit{Korpus savremenog srpskog jezika}, that is the Corpus of contemporary Serbian (SrpKor2003), has been accessible online since 2002 \citep[41a]{Utvic11}. Its first version, NETK, lacked information about text sources, which was incorporated into the newer version. Both corpora are still available online; nevertheless, authorisation is required to use them. NETK and SrpKor2003 are monolingual corpora of raw texts written in the 20\textsuperscript{th} century and belonging to different functional styles \citep{KrstevVitas05}.\footnote{\textcolor{black}{It is important to emphasise that both corpora are available only} without an annotation layer.} They contain 22,000,000 words. The concepts which were important for the development of these corpora are described in \citet{VKP00}. 

\begin{sloppypar}
The latest version, SrpKor2013, was released in 2013. It can be queried through an interface available for non-commercial purposes after registration. SrpKor2013 contains 122,000,000 words and it is the largest corpus of Serbian compiled in a traditional way. SrpKor2013 is lemmatised, and annotated for parts-of-speech. The markup scheme contains 16 tags \citep[43a]{Utvic11}. The existing documentation is very limited, so it is hard to draw many conclusions about the existence of further, for instance morphosyntactic, markup. SrpKor2013 is composed exclusively of written language and its texts can be divided into five functional styles (literary, scientific, publicistic, administrative, and others) \citep[42a]{Utvic11}. Bibliographic metainformation is provided for all texts and searches may only be performed separately for individual styles. From the description of the corpus it may be concluded that it also contains translation and some texts from online portals. Nonetheless, local experts consider the existing version of the Corpus of contemporary Serbian to be insufficient. In their opinion, in a new release more attention should be paid to achieving a balance between registers \citep[42a]{Utvic11}.
\end{sloppypar}

One part of SrpKor2013 has been extracted as a separate corpus under the name SrpLemKor. It is a lemmatised and PoS annotated corpus which consists of 3,763,352 words. This is the only part of the corpus for which the proportions of texts from particular registers are given in the documentation.

Korpus srpskog jezika (KSJ) is well described in \citet{Kostic03}. It comprises 11,000,000 words. \citet[47]{Dobric09} emphasises its diachronic dimension, which is reflected in the inclusion of texts dating back to the 12\textsuperscript{th} century. On the other hand, KSJ does not cover spoken language or many contemporary texts. Additionally no clear line can be drawn between Croatian, Serbian, and Serbo-Croatian where texts originating from the second half of the 20\textsuperscript{th} century are concerned. Undoubtedly the biggest advantage of KSJ is its detailed annotation, consisting of the grammatical status of each word, number of graphemes, syllable division and phonological structure, which was completed manually. Nevertheless, the main problem with KSJ is its accessibility. In personal communication with Dušica Filipović Đurđević and Aleksandar Kostić, son of the corpus compiler Dorđe Kostić, we found out that querying the corpus is possible only indirectly. One has to contact Aleksandar Kostić with a precise description of the data necessary and then the members of the Department of Psychology of the University of Belgrade extract concordances and send them back. Needless to say, first, such a mode of work is not very convenient and secondly, it is very likely that only simple queries are possible. 

\begin{sloppypar}
Summing up, considerable resources exist for Serbian, but similarly to other BCS corpora, they suffer from certain drawbacks. First, the annotation and searchability do not fulfil current standards. In this respect, the biggest problem seems to be the extremely limited possibility of using morphosyntactic annotation in queries. Secondly, many sources also contain diachronic data, or possibly include other varieties of BCS which remain unannotated.
\end{sloppypar}

\section{\{bs,hr,sr\}WaC}
\label{WAC}
\subsection{The concept of the Web as a Corpus}

\{bs,hr,sr\}WaC \citep{LjubesicKlubicka14, LjubesicErjavec16c, LjubesicErjavec16a, LjubesicErjavec16b} belong to the Web as Corpus family of corpora, first popularized by WaCky \citep{BBFZ09}. The idea of using the internet as a source of linguistic data was controversial at first and generated a discussion about the content of web pages, since in such cases the acquisition of material is  less controlled than in the case of traditional corpora. However, within the last decade the concept has become more and more popular, in particular because it is faster and cheaper in comparison to the traditional way of compiling a corpus \citep[43]{Benko17}. 

The lack of resources for most of the South Slavic languages, which we hope we managed to demonstrate above, has also been recognised by the group of linguists behind the Regional Linguistic Data Initiative ReLDI. Furthermore, for smaller languages we do not have the luxury of text sampling, since the amount of data written in these languages is limited by their population, in comparison to, for example, English or Spanish. On the other hand, this can be a point in favour of web data since a large part of all writings are available online and can be turned into a language corpus \citep{LjubesicErjavec11}. Therefore, treating web corpora as fully-fledged language resources is certainly appropriate in the case of South Slavic languages.

We will now provide the key data about \{bs,hr,sr\}WaC, and discuss the problems and limitations of these three corpora.

\subsection{\{bs,hr,sr\}WaC in a nutshell}
The \{bs,hr,sr\}WaC corpora are undoubtedly the largest existing corpora for each of the three languages. Some key statistical data are presented below in Table~\ref{table:wac1}.

\begin{table}
\caption{\{bs,hr,sr\}WaC corpora\label{table:wac1}}
\begin{tabular}{lrrrr}
\lsptoprule
Corpus & Words & Tokens & Documents & Compiled\\
\midrule
bsWaC&	248,478,730	&	286,865,790	&	896,059 &	10/28/2017 17:23:26\\
hrWaC&	1,210,021,198 &	1,397,757,548 &	3,611,090 &	10/28/2017 01:27:08\\
srWaC&	476,888,297	&	554,627,647	&	1,353,238 &	10/28/2017 04:17:28\\
\lspbottomrule
\end{tabular}
\end{table}

Numerical data show that hrWaC is definitely the biggest of them, as it is more than double the size of srWaC and nearly five times bigger than bsWaC. We can identify several reasons for this state of affairs. First, the size of the Croatian economy and market. Second, the proportion of content written in closely related languages which appears in the Bosnian web and which had to be eliminated. And last but not least, the fact that the authors of the project are Croatians, and therefore may be more dedicated to the development of tools for studying Croatian.

\begin{sloppypar}
\{bs,hr,sr\}WaC are a family of top-level-domain corpora of Bosnian, Croatian, and Serbian, which are available for download and online work via the  NoSketchEngine concordancer.\footnote{Other top-level-domains are: .ba, .hr, .rs, .biz, .com, .eu, .info, .net.} They are currently accessible through the same platform as the latest version of Riznica. Like Riznica, they have been automatically lemmatised and morphosyntactically annotated with the unified tagset pattern according to MULTEXT-East Morphosyntactic Specifications.\footnote{\textcolor{black}{For more information visit} \url{http://nl.ijs.si/ME/V5/msd/html/msd-hr.html}. The newest version has been released in 2019, see \url{http://nl.ijs.si/ME/V6/msd/html/msd-hbs.html\#msd.msds-hbs}.} The tagsets for Croatian and Serbian are identical on the morphosyntactic level, apart from one additional subset of tags for the synthetic future tense in Serbian \citep[31]{LjubesicKlubicka14}. 
\end{sloppypar}

The morphological annotation was performed automatically with an accuracy estimated at 92.5\%, which fulfils the current standards in NLP \citep[4269]{LKAJ16}

\subsection{WaC content}
\label{WaC content}
The main problem of corpora compiled from the web is the lack of metadata on corpus composition. This applies to all possible categories which are used to characterise traditionally compiled corpora (sociolinguistic information, text age, style, genre, and register). Web corpora can be characterised with technical information (domain, URL, date of update or upload, and size), and, if additionally processed, with internal linguistic factors such as size of lexicon and frequency of grammatical features. Such analyses are nevertheless time-consuming and can usually, due to the massive volume of data, explain only part of variance. Additionally, in order to evaluate web corpora as a source, similar analysis must be performed on traditionally compiled, representative sources, which, as stated above, barely exist in BCS.

\citet{Benko17} shows that, regardless of problems with characterising their exact content, web corpora should not be treated as an inferior type of data, but simply a different one. Furthermore, experiments conducted for English \citep{BiberEgbert16} and Czech \citep{Cvrcek18} show that as far as internal linguistic features are in question, web data and traditional corpora overlap to a large extent.

\citet[62f]{Gato14} observes that although web corpora cannot cover all possible registers, they provide quite a wide spectrum, starting with formal legal texts on the one hand, and ending with informal blogs, and chat rooms on the other. It seems that the web contains both traditional genres adapted to the new medium, like newspaper and academic articles, and entirely new ones, such as tweets or Facebook entries, rarely included in traditionally compiled sources.

Finally, \citet[106]{SchafferBildhauer13}, authors of the German web corpus, come to the conclusion that web corpora do not generally perform noticeably worse than traditional ones of the same size. In addition, since size matters, it has to be said that large web corpora frequently outperform smaller traditional corpora \citep[106]{SchafferBildhauer13}. In other words, although the contents of web corpora cannot be described in traditional terms, there are good reasons to assume that with respect to linguistic structure a massive corpus is better than a small one.

\subsection{Sources of noise}

\subsubsection{Closely related languages}

Another point of critique towards web corpora is noisy data. The creators of South-Slavic WaC corpora indicate two main sources of noise: first, documents written in other, closely related languages and secondly, texts of low quality \citep[29]{LjubesicKlubicka14}. 

	In order to solve the problem of closely related languages, the creators used two classifiers: a blacklist classifier and unigram-level language models \citep[32]{LjubesicKlubicka14}. Table \ref{table:dist} shows what share of documents in each corpus was identified as written in a closely related language \citep[cf.][33]{LjubesicKlubicka14}. The authors used a ternary classifier in bsWaC, where the share of foreign documents was the highest, and assumed that a binary classifier for hrWaC and srWac, which distinguishes only between Serbian and Croatian, is sufficiently informative.

\begin{table}
\caption{Distribution of identified languages through the three corpora}
\label{table:dist}
\centering
\begin{tabularx}{.5\textwidth}{lYYY}
\lsptoprule
& bs & hr & sr\\
\midrule
bsWaC & 78.0\% & 16.5\% & 5.5\%\\
hrWaC & - &99.7\% & 0.3\%\\
srWaC & - & 1.3\% & 98.7\%\\
\lspbottomrule
\end{tabularx}
\end{table}

Nonetheless, although most documents in \{bs,hr,sr\}WaC are classified cor\-rect\-ly, one should be aware that single paragraphs  in closely related languages might still appear. This is mostly the result of reader comments, where the content of the document is generated by many users. Still, we think that such appearances should not affect our results because all unexpected occurrences can be checked manually before they are included in the data set.

An issue that is linked to content written in closely related languages is the occasional appearance of lexical elements from other South Slavic varieties. We must point out that even strictly monitored corpora such as the CNC contain words which, according to handbooks, do not belong to the Croatian standard, such as: \textit{opšti} `general' (\textit{opći} in Croatian), \textit{januar} `January' (\textit{siječanj} in Croatian), \textit{sveštenik} `priest' (\textit{svećenik} in Croatian), \textit{tačka} `dot, point' (\textit{točka} in Croatian). Although the authors of the CNC tried to minimise this phenomenon by selecting texts written after 1990, such word forms are present, for instance because academic texts which discuss differences between Croatian and Serbian have been included. Similar evidence of non-Croatian word forms can be found also in Riznica, where the lemma \textit{tačka} typical of Serbian is attested not only in academic texts, but also in texts by 19\textsuperscript{th} century Croatian writers. In the same manner, Croatian word forms such as \textit{nogomet} `football' (\textit{fudbal} in Serbian), \textit{glazba} `music' (\textit{muzika} in Serbian) and \textit{zrakoplov} `aircraft' (\textit{vazduhoplov}, \textit{avion} in Serbian) can be found in SrpLemKor. A similar problem applies to web corpora, but on a larger scale. 

\subsubsection{Non-standard language use and low quality data}

The authors of the corpora approach the problem of low quality data with the assumption that most of the content of each web corpus can be qualified as good \citep[33]{LjubesicKlubicka14}. In order to easily detect low quality text the most frequent types of deviation must be identified and classified. Above all, non-standard usage of the upper case, lower case and punctuation, and usage of non-standard language, understood as slang and dialects, belong here \citep[33]{LjubesicKlubicka14}. 

Not all these problems are easy to solve, but during the procedure of noise removal from the first release of \{bs,hr,sr\}WaC, it was postulated that a low percentage of diacritic characters should reflect less standard language usage and this assumption was used as a very simple estimate of text quality \citep[34]{LjubesicKlubicka14}. In the second release, the REDI tool was used to restore diacritics, so that the texts could be correctly lemmatised and part-of-speech annotated.\footnote{The REDI tool is available at \url{https://github.com/clarinsi/redi}.}

To this we can add the problem of avoiding standard punctuation, which can be partly related to specific writing style. It is commonly known that some texts, for instance those written on discussion fora, are characterised by a relaxed approach to punctuation and the use of symbols so even when those texts are built of several sentences they can contain hardly any periods at all \citep[90]{SchafferBildhauer13}. \citet[43]{Gato14} also points out that online texts often contain misspelled words and grammatical mistakes, or include improper usage by non-natives.

Non-standard data could be an interesting object of CL research as they represent a very spontaneous, non-planned channel of communication, but they must be approached cautiously. Hence, results arising from non-standard language used online must always be checked manually to decide whether a particular divergence is caused by the relaxed use of language or carelessness of the language user, or else whether its source is non-native language use.

The second kind of low quality data typical of web documents are URLs, automatic translations, words split into fragments, and emoticons, but they do not affect our research much as they can be easily filtered out.

\section{Corpora for spoken BCS}
\label{Corpora for spoken BCS}
\subsection{Bosnian}

In the area of spoken varieties the availability of resources is even lower than for written varieties. Building spoken corpora is related to higher costs understood in terms such as time, money, and manpower. Moreover, the workflow is more complicated since compiling a corpus of spoken data requires the same steps as in the case of written varieties, but additionally recordings must be obtained and transcribed. Spoken data is also further from theoretical language models and normative description, so many structures not included in normative descriptions, such as (dis)fluencemes, occur.\footnote{\textcolor{black}{For more information on such structures} see Section \ref{Principles of analysis of spoken language}.} This poses a challenge for both human annotators and automatic taggers and lemmatisers. It comes as no surprise that only a handful of spoken corpora, compiled mainly for specific research purposes, are currently available.

The only corpus of Bosnian that we found was a corpus of narrative interviews compiled within the DFG-funded project \textit{Corpus-based} \textit{analysis} \textit{of} \textit{local} \textit{and} \textit{temporal} \textit{deictics} \textit{in} \textit{(spontaneously)} \textit{spoken} \textit{and} \textit{(reflected)} \textit{written} \textit{language}. The corpus is called Bosnian Interviews \citep{RaeckeStevanovic01} and was mainly transcribed and annotated by Slavica Stevanović. It used to be available for searches through an online interface, but currently access to its XML files can be obtained only on request. The data consist of 13 narrative conversation-situations with Bosnian refugees. The corpus is neither PoS annotated nor lemmatised, but tagging of v/t/n-deictics was performed for purposes of the above-mentioned research goal. An additional meta-layer of regional pronunciation is also featured. The formal description of the corpus is, nevertheless, very vague as, for example, the size of the corpus is not stated. We provide more details on this corpus in Chapter \ref{Clitics in a corpus of a spoken variety}.

\subsection{Croatian}

The Croatian Adult Spoken Language Corpus HrAL \citep{KuvacHrzica16} was built by sampling spontaneous conversations of 617 speakers from all Croatian counties, and it comprises over 250,000 tokens and over 100,000 types in 165 transcripts annotated with the ages and genders of the speakers, as well as the location of the conversation. It was compiled in three periods: 2010--2012, 2014--2015 and 2016. Croatian speakers from different parts of Croatia with access to groups of speakers (friends and families) were recruited and trained to collect samples of spoken language. Sampling was performed in informal situations, predominantly spontaneous conversations among friends, relatives or acquaintances during family meals, informal gatherings, and socialising. Thus the corpus contains rather short, often interrupted utterances.

\subsection{Serbian}

We are not aware of any publicly available, electronically stored corpus of spoken Serbian. Nevertheless, this gap might be filled in the near future, as some efforts towards building both Serbian and Bosnian spoken sources are being made, e.g. in a project at Humboldt-Universität zu Berlin.\footnote{\textcolor{black}{For more information visit} \url{https://www.slawistik.hu-berlin.de/de/fachgebiete/suedslawsw/colabnet/projects/spoc/spoc}.}

\section{Discussion}
\label{Discussion}
\subsection{The scope of available data}

This section compares the properties of traditionally compiled corpora and web corpora for BCS  with the goals of the study. As shown above, sources for corpus analysis in BCS are certainly limited. On the one hand, we have at our disposal rather sparse traditionally compiled corpora. They mostly represent language strongly influenced by normative prescription. Additionally, the languages are not equally represented if we compare size and type of data and the extent of annotation, which implies an individual approach to working with each corpus. The worst situation is in the area of spoken language and dialects, where little or no data can be identified.\footnote{\textcolor{black}{Some corpora such as \citet[]{Vukovic21} were developed after our project was finished.}}

On the other hand, large data sets of unknown composition obtained from the web can be easily accessed and processed in a comparable way for all three South Slavic varieties in question. Although the language of web corpora cannot be described in traditional terms, a considerable share of the language represented in web corpora is not influenced by normative prescription, but is probably not worse as concerns linguistic richness than traditional data, as we hope we have shown in Section \ref{WaC content}. 

Additionally, the analysis of url domain lists shows that web corpora not only cover texts typically included in corpora of standard language such as literary, journalistic, administrative,  academic, and popular scientific texts, but also contain very new registers and genres that appear in user-generated content such as blogs and fora which are much closer to spontaneous language, even though written and not spoken \citep[4]{SchafferBildhauer13}. This type of data is a valuable source of colloquial language and as such certainly relevant for studies of microvariation. Next to the available meta-information (allowing to track where the texts come from), size, and accessibility, such variety of data is a great advantage of web corpora  which, at the same time, is hard to obtain from traditional sources.\footnote{We are aware that the internet is often criticized for poor quality of texts, which includes numerous spelling errors, omission of diacritic signs and non-standard use of the upper and lower case and that this critique also pertains to texts from \{bs, hr, sr\}WaC. However, we would like to point out that for us these corpora are a source of authentic, spontaneously produced written texts, which were not under strict influence of the norm or externally corrected to look like prescribed standard Bosnian, Croatian or Serbian.}

The question of the extent to which the available data can be considered representative appears. Following \citet[243]{Biber05}, “representativeness refers to the extent to which a sample includes the full range of variability in population”. Ironic as it may seem, ideal representativeness is not possible to achieve. This is because however much corpus constructors try, they can only create a corpus which is the representation of itself, \citet[1]{KilgarriffGrefenstette03} claim.\footnote{\citet[8f]{KilgarriffGrefenstette03} list several reasons why corpora fail to represent real language usage. They draw attention to the arbitrariness which dominates in the text sampling, i.e. it is literally impossible to include all text types and topics \citep[cf.][9]{KilgarriffGrefenstette03}.}  Furthermore, the representativeness criterion seems useless nowadays, because web corpora do not contain that sort of metadata. Therefore, neither is it possible to check the range of text types they cover, nor can one be sure about the population of text types themselves, since the web covers a considerable, but not systematised, share of texts. Nonetheless, its variety and particularly its size counterbalance the limited information about its representativeness \citep[45]{Gato14}. 

\citet[120]{ManningSchutze99} argue that “having more training data is normally more useful than any concerns of balance, and one should simply use all the text that is available”. Furthermore, we agree with what Kilgarriff already noticed, namely that “it is the web that presents the most provocative questions about the nature of language” \citep[cf.][344]{Kilgarriff01}. 

Therefore, we follow \citet{ManningSchutze99} and try to use a possibly broad scope of available corpora according to our goals. Apart from providing naturally-occurring, non-externally normativised and proofread language, corpora benefit our work on two topics. The first is CL placement and inventory in spoken varieties (see Part \ref{part2}). The second is an empirical approach to CC (see Part \ref{part3}), a controversial topic which has so far been studied exclusively in terms of theoretical syntax. In the two sections below we explain our choices as to the studied sources.

It is important to remember that currently digital sources develop very dynamically. During the period when the current project was conducted, some significant changes could be observed, such as the improvements in the morphosyntactic tagger for BCS and the new release of Riznica. Due to practical reasons, primarily time constraints, we could not benefit from all the advancements. Some parts of study were conducted on the old versions. Some corpora were rejected due to their poor quality at the time.

\subsection{Variation in spoken BCS}
\label{Parameters of variation in spoken BCS}
As presented in Section \ref{Corpora for spoken BCS}, the most under-resourced area of BCS is spoken sources. Thus, making statements about CL behaviour in spoken varieties and dialects based on corpus data is barely possible at the moment. The available spoken sources neither meet the standards applied to written corpora with regard to morphosyntactic annotation level, nor are they preprocessed with regard to  transcript standards.

Work with both the Bosnian Interviews and HrAL corpora would require performing a high load of additional preprocessing. Importantly, the two corpora are not comparable. While Croatian transcripts contain mostly conversations, Bosnian Interviews are rather spoken narratives. As a consequence studying CLs is more feasible in the case of Bosnian data. Therefore, we decided that the first attempt to study the behaviour and distribution of CLs would be in spoken Bosnian, in particular concerning the influence of discourse structuring elements and disfluencemes on CL placement.\footnote{\textcolor{black}{For more information on discourse structuring elements and disfluencemes} see Section \ref{Principles of analysis of spoken language}.} The results of this study, as well as a more detailed description of the corpus, based on our own explorations, are described in Chapter \ref{Clitics in a corpus of a spoken variety}.

Given the lack of sufficient spoken dialectological corpora, we decided to work with the written sources described in detail in Chapter \ref{Clitics in dialects}.

\subsection{Clitic climbing in BCS}
\label{Clitic climbing in BCS}

In order to study the variation in constructions featuring verbal embeddings in the three South-Slavic varieties, a similar kind of data should be acquired for each variety. In that respect \{bs,hr,sr\}WaC are superior to other sources because, as explained above, a quite similar type of language variety is represented in all three web corpora. Additionally, the tagset and the query syntax are identical, so the results of searches are also comparable. Comparison in that respect across standard varieties on the basis of traditionally compiled corpora is barely possible, mainly due to the very limited searchability. 

Web corpora are also unbeatable in terms of size. This increases the chances that even very rare variants of studied constructions will occur. For this reason they provide the best environment for examining the possibilities of CC from \textit{da}\textsubscript{2}-complements in Serbian, as described in Chapter \ref{A corpus-based study on CC in da constructions and the raising-control distinction (Serbian)} and in \citet*{JHK17b}.

Finally, as already mentioned, web corpora include user-generated content which represents spontaneous, non-edited, and thus very authentic language typical of ordinary users, present in fora, blogs, and reader comments. This type of language is not represented in the traditionally compiled corpora of BCS.\footnote{In some languages, e.g. Czech, this type of language is  steadily coming to be incorporated also in monitor corpora, e.g. Koditex \citep{Zasina18}.} Since WaC are in a sense anonymous, as we rarely have access to sociolinguistic metadata, the possibilities for in-depth study of sociolinguistic variation are extremely limited. On the other hand, because Riznica has been available on the same online platform as hrWaC since spring 2018 and since it uses the same tagset as WaC, some conclusions can be drawn as to the factor of standard vs colloquial variety. Therefore in Chapter \ref{A corpus-based study on clitic climbing in infinitive complements in relation to the raising-control dichotomy and diaphasic variation (Croatian)} we study CC from infinitive complements in Croatian in the forum.hr URL domain and in Riznica.
