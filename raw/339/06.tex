\chapter[Clitics and variation in grammaticography and related work]
        {Clitics and variation in grammaticography and related work (Bosnian, Croatian, Serbian)}
\label{Clitics and variation in grammaticography and related work}
\section{Introduction}

The goal of this literature-based chapter is to present the current state of the art on CL systems in grammaticography and related work, with reference to variation. Furthermore, we compile information some authors give concerning different registers, i.e. diaphasic variation. It should be noted that there are actually no works on CLs focusing specifically on variation; neither are there any empirical variationist studies on CLs.\footnote{Information on this topic is scattered around the quoted works. Furthermore, we were able to find some information on CLs in respect of diachronic variation, but none of the authors mentioned diastratic variation, i.e. differences in the language use of individual social groups. }

Since our aim is а deeper empirical investigation of CLs, at this stage the most important goal is to detect possible instances of microvariation in the CL system with the help of the parameters of variation outlined in Chapter \ref{Our terms and concepts}. Afterwards the selected CL phenomena recognised as a source of variation can be thoroughly investigated. It goes without saying that we must discuss here approaches authors adhere to. Some of them favour the phonological one whereas others share a more formal syntactic orientation.

In the next section, we explain our strategy, which allowed us to gather rather the scattered data on variation. Subsequent sections of this chapter follow the order in which we presented the parameters of microvariation in Chapter \ref{Our terms and concepts}. Section \ref{Inventory:7} gives an overview of the inventory of CLs. In Section \ref{Internal organisation of the clitic cluster:7} we present the rules of CL clusterisation and morphonological changes which occur within CL clusters. Section \ref{Position of the clitic or clitic cluster} on position of the CL or CL cluster follows, in which we focus on the placement of CLs after breaks, heavy phrases, conjunctions, and complementisers. In the same section, we discuss in detail 2P, phrase splitting and DP of CLs. Special attention is given to the concept of 2P and different views on it. The final subsection addresses the problem of phrase splitting and the syntactic contexts which allow it. 

\section{Detecting variation}
\label{Detecting variation}
This chapter follows the first step of our empirical approach presented in Chapter \ref{Introduction with overview} and described in some detail in Chapter \ref{Empirical approach to clitics in BCS}. We approach this goal by inspecting grammar books for Bosnian, Croatian, and Serbian written by native authors, i.e. first we try to detect both systemic and sociolinguistic microvariation in each standard variety. A comparison of the existing descriptions of CLs in the grammar books, which are widely used in Bosnia, Croatia, and Serbia, enable us to detect the first level of variation in the CL system: diatopic variation between standard varieties of BCS on the level of prescribed language usage.\footnote{We must point out that although the grammar books used imply the description of some kind of standard variety (the examples which authors provide are definitely neither colloquial nor dialectal), they are not labeled as normative grammar books, except \citet{PiperKlajn14}.} We chose grammar books which are currently in use as handbooks in schools or at universities. We believe that the grammars on our list reflect the current state of the CL system in standard varieties of the mentioned three countries. 

Since there is no long tradition of Bosnian grammaticography, we chose the first  post-Yugoslavian grammar book, \textit{Gramatika bosanskoga jezika} by \citet{JHP00}.\footnote{The first grammar book which has the word Bosnian in its title was \textit{Gramatika bosanskoga jezika za srednje škole} \citep{Vuletic90}.} In addition to this work, we decided to take into consideration the descriptions of CLs in \textit{Bosnian for foreigners with a comprehensive grammar} \citep{Ridjanovic12} written in English. 

Among Croatian grammar books we thoroughly examined Katičić's (\citeyear{Katicic86}) highly influential \textit{Sintaksa hrvatskoga književnog jezika}. Further, we observed how CLs are presented in \textit{Gramatika hrvatskoga jezika}, a handbook used in elementary education \citep{TezakBabic96}. We also took into consideration \textit{Gramatika hrvatskoga jezika: za gimnazije i visoka učilišta} \citep{SilicPranjkovic07} and \textit{Hrvatska gramatika} \citep{Baric97}, which are used in high school education and by students of the Croatian language at Croatian universities.

As to Serbian grammars, we deliberately started the analysis from an older book, namely \citeposst{Stevanovic75} \textit{Savremeni srpskohrvatski jezik}, in order to see if there have been any changes in the CL system or in the norm. Next we included the high school grammar handbook \textit{Gramatika srpskog jezika} written by \citet{StanojcicPopovic02}. Since CLs are a phenomenon which lies at the intersection of syntax with other disciplines, we analysed \textit{Sintaksa savremenoga srpskog jezika} \citep{PiperIvic05} as well. In addition, we took into consideration the recent \textit{Normativna gramatika srpskog jezika} by \citet{PiperKlajn14}. Like for Bosnian, we analysed descriptions of CLs in one grammar book for foreigners, \textit{Gramatika srpskog jezika za strance} by native authors \citet{MrazovicVukadinovic09}.  

\hspace*{-1.5mm}Besides variation between BCS standard varieties, browsing the grammar books revealed some diaphasic or diatopic variation within individual varieties. We were especially interested in so-called instances of common mistakes or deviations from standard language use. To us this actually indicates that there is some kind of variation within one variety, since native speakers of BCS normally have to learn the standard because they usually speak a dialect or some other kind of non-standard variety at home. We also inspected \textit{jezični savjetnici} `language guidebooks' for the same reason – they helped us detect both diatopic variation between BCS varieties, and diaphasic and diatopic variation within one variety.

However, since, as we already emphasised, we are not only interested in the CL system in the prescriptive norms of standard varieties of BCS. After grammar books and language guidebooks we consulted some other papers on CLs in which we could find scattered information on diatopic variation between standard varieties of BCS and on diaphasic variation (i.e. in different registers) in the CL system of BCS. It should be pointed out that most theoretical works on CLs in BCS presented in Chapter \ref{Theoretical Aproaches to the Study of Clitics in BCS} do not mention variation or are not interested in it at all. Here, we refer not only to works addressing this topic like \citet{RadanovicKocic88} and \citet{Milicevic07}, but also to works which do not address it directly. We found the latter especially interesting if they contained statements on CLs which contradict the statements of other scholars.

\section{Inventory}
\label{Inventory:7}
\subsection{Inventory of pronominal clitics in BCS standard varieties}
\label{Inventory of pronominal clitics in BCS standard varieties}
The following table lists the CL and full forms of all personal pronouns in the genitive, accusative, and dative (\citealt[cf.][208f]{Baric97}, \citealt[363f]{MrazovicVukadinovic09}).\footnote{Although \citet[363f]{MrazovicVukadinovic09} list only forms with a short rising accent in the paradigm, they admit that a short falling accent is also possible.}

\begin{table}
\caption{Pronominal CLs in BCS and their corresponding full forms\label{table:clcorr}}
\begin{tabular}{llllll}
\lsptoprule
\multicolumn{2}{c}{ } & \multicolumn{2}{c}{Genitive/accusative} & \multicolumn{2}{c}{Dative} \\
\cmidrule(lr){3-4}\cmidrule(lr){5-6}
\multicolumn{2}{c}{} & CL form & Full form & CL form& Full form\\
\midrule
 singular& I & \textit{me} 	& \textit{mȅne, mène} & \textit{mi} & \textit{mȅni, mèni} \\
 & you & \textit{te} & \textit{tȅbe, tèbe} & \textit{ti}	& \textit{tȅbi, tèbi} \\
 & he, it & \textit{ga/ga, nj} & \textit{njȅga, njèga} & \textit{mu} & \textit{njȅmu, njèmu} \\
 & she & \textit{je/ju, je} & \textit{njȇ/njû} & \textit{joj} & \textit{njȏj} \\
 \tablevspace
plural & we	& \textit{nas} & \textit{nȃs} & \textit{nam} & \textit{nȁma}\\
 & you & \textit{vas} & \textit{vȃs} & \textit{vam} & \textit{vȁma}\\
 & they& \textit{ih} & \textit{njȋh} & \textit{im} & \textit{njȉma}\\
\lspbottomrule
\end{tabular}
\end{table}



It is worth noting that the pronominal CLs for the third person singular have two CL variants in the accusative: \textit{ga} and \textit{nj} `him' for masculine and neuter, and \textit{ju} and \textit{je} `her' for feminine \citep[cf.][366]{MrazovicVukadinovic09}. The CL \textit{nj} differs syntactically from other CLs since it follows prepositions, does not clusterise and is not associated with the second position. \citet[128]{MrazovicVukadinovic09} emphasise that it is felt to be archaic in Serbian.

We noticed some diatopic variation concerning the accusative CL \textit{ju} between the three BCS standard varieties. \citet[56]{RadanovicKocic88} states that “The accusative singular feminine CL \textit{ju} is completely replaced by the genitive \textit{je} in the Eastern variant, while it is still used in the Western variant.''\footnote{It is not completely clear what the term ``Eastern variant'' actually means since Radanović-Kocić does not define it. From her observations on the usage of the accusative CL \textit{ju}, we infer that she uses this term for some kind of standard Serbian since her statement definitely cannot apply to all Serbian/Eastern varieties. In Chapter \ref{Clitics in dialects} we provide data on the usage of the pronominal CL \textit{ju} in Eastern Štokavian dialects.} Whereas in the Yugoslavian period Croatian grammar authors of the post-Maretić era tried to allow wider use of the CL \textit{ju}, for instance after verbs and words ending with \textit{-e/ -je} \citep[cf.][187]{Mamic95}, Serbian linguists were not keen to accept such an idea.\footnote{In 1899 Maretić published two grammar books which became highly influential (ten adapted editions for use in schools until 1928) in which he argued for the usage of the CL \textit{ju} only in suppletion contexts.} The Serbian linguist \citet[306]{Stevanovic75} claimed that the pronominal CL \textit{ju} `her' could only be used in combination with the verbal CL \textit{je} `is', arguing that all other uses of \textit{ju}, those in which it does not follow the verbal CL \textit{je}, are dialectal. He criticised the Croatian authors Brabec, Hraste \& Živković who introduced the broader use of \textit{ju} \citep[cf.][306]{Stevanovic75}. The Serbian linguists \citet[97]{PiperKlajn14} explicitly state that the usage of CL \textit{ju} (apart from instances of suppletion in the combination with the verbal CL \textit{je} `is', \textit{nije} `is not' and other \textit{-je} ending verbs) is not correct in standard Serbian. The usage of \textit{ju} beyond these contexts of suppletion is to be considered a foreign, i.e. Croatian, construction \citep[cf.][97]{PiperKlajn14}.\footnote{\textcolor{black}{For more information on suppletion see Section \ref{Morphonological processes within the cluster}}.} Accordingly, the stance of Croatian linguists \citet[143]{FrancicPetrovic13} is that both \textit{ju} and \textit{je} are acceptable in standard Croatian; however, in certain contexts they recommend only one of them. If the host ends in \textit{-ju} (like \textit{skrivaju} `are hiding'), they advise the use of \textit{je}, whereas if it ends in \textit{-je} (as \textit{prije} `before' and \textit{nije} `is not'), they recommend the use of \textit{ju} \citep[cf.][143]{FrancicPetrovic13}. However, native speakers in BCS standard varieties do not always follow those recommendations and there is a great deal of diaphasic and diatopic variation within Croatian. \citet[143]{FrancicPetrovic13} estimate that deviations from the above-mentioned rules are very frequent in the standard Croatian language. As we already pointed out in Section \ref{Detecting variation}, we assume that this is caused by other Croatian varieties, which are spoken by native speakers whose language differs from the standard in respect of the mentioned rule. 

To sum up, the distribution of the pronominal CLs \textit{ju} and \textit{je} is sometimes attributed to the diatopic differences in inventory and sometimes to suppletion.

\subsection{Inventory of verbal clitics in BCS standard varieties}
\label{Inventory of verbal clitics in BCS standard varieties}
Most authors (e.g. \citealt[72]{TezakBabic96}, \citealt[72]{Baric97}, \citealt[284]{Popovic04}, \citealt[21]{SilicPranjkovic07}, \citealt[28]{PiperKlajn14}) agree that there are three types of verbal CLs: unstressed present tense and aorist forms of the verb \textit{biti} `be' and unstressed present tense forms of the verb \textit{ht(j)eti} `will'.\footnote{Unlike other authors, \citet[350]{Stevanovic75}, \citet[97]{StanojcicPopovic02}, \citet{MrazovicVukadinovic09}, \citet[271]{JHP00} consider the forms \textit{sam, si, je}... to be unstressed present tense forms of the verb \textit{jesam}. The question whether the present tense forms \textit{sam, si, je}... and \textit{budem, budeš, bude}... are different forms of one verb \textit{biti} or whether there are two verbs, \textit{biti} with the present tense \textit{budem, budeš, bude}... and \textit{jesam} with the present tense \textit{sam, si, je}... is not relevant to our study.}

\subsubsection{Present tense clitics of \textit{biti} `be'}

Table \ref{table:clbiti} contains the forms of the present tense CLs \textit{biti} `be' as presented in the grammar books of all three standards (\citealt[cf.][271]{Baric97}, \citealt[128]{MrazovicVukadinovic09}). There are no diatopic differences between CL systems of BCS standard varieties.\footnote{In Table \ref{table:clbiti} we can see that there is diatopic difference between the BCS standard varieties in the third person singular full form of verb \textit{biti} `be'. While \textit{jȅst} is proposed for standard Croatian, \textit{jèste} is proposed in standard Bosnian and Serbian. However, it seems that the current state is a consequence of diachronic variation within standard Serbian. \citet[198]{PiperKlajn14} claim that in previous periods of Serbian language the full third person singular form of the verb \textit{biti} was \textit{jȅst}, while today the form \textit{jèste} is more common. It is interesting to note that \citet[128]{MrazovicVukadinovic09} provide only the form \textit{jȅst} in the complete present tense paradigm, without mentioning \textit{jèste}. However, all the examples which they provide on the same page contain only \textit{jèste} as the full form third person singular present tense and none include the form \textit{jȅst}.}


\begin{table}
\caption{Clitic and full forms of the present tense of \textit{biti} `be'\label{table:clbiti}}
\begin{tabular}{llllll}
\lsptoprule
\multicolumn{3}{c}{singular} & \multicolumn{3}{c}{plural}\\\cmidrule(lr){1-3}\cmidrule(lr){4-6}
& CL form & Full form & & CL form & Full form\\\midrule
be.\textsc{1sg} & \textit{sam} & \textit{jèsam} & be.\textsc{1pl} & \textit{smo} & \textit{jèsmo} \\
be.\textsc{2sg} & \textit{si} & \textit{jèsi}   & be.\textsc{2pl} & \textit{ste} & \textit{jèste} \\
be.\textsc{3sg} & \textit{je} & \textit{jé}     & be.\textsc{3pl} & \textit{su} & \textit{jèsu} \\
&& \textit{jȅst, jéste} &&& \\
\lspbottomrule
\end{tabular}
\end{table}


The only difference we noticed is in views on whether \textit{je} `is' can bear an accent or not. \citet[128]{MrazovicVukadinovic09} claim that \textit{je} is always unstressed, but unlike other CLs it can be used at the beginning of a sentence together with the question particle \textit{li}.\footnote{Although they claim that the unstressed form of \textit{je} is an exception and is the only present tense CL which can take the first position in a sentence, in the example they provide on the same page there is an accent symbol on \textit{jé}, i.e. \textit{Jé li ovo dobro?} `Is this ok?' \citep[cf.][128]{MrazovicVukadinovic09}.}  A slightly different explanation is offered by \citet[272]{JHP00}, who say that \textit{je} at the beginning of a sentence is used as an accented word. In contrast, \citet[496]{Katicic86} and \citet[198]{PiperKlajn14} argue that there are two forms: one is the CL \textit{je}, which is unstressed, and the other, the full form \textit{jé}, which is stressed and used in questions. In other words, they claim that in the question expression \textit{jé li} the form \textit{jé} bears its own accent, and can therefore be placed at the beginning of a sentence (\citealt[cf.][496]{Katicic86}, \citealt[198]{PiperKlajn14}). 

\subsubsection{Aoristal/conditional clitics of \textit{biti} `be'}

As may be seen in Table \ref{table:clbitiao}, there is no diatopic variation between the three standard varieties with respect to the CL aoristal forms of \textit{biti} `be' (\citealt[cf.][271]{Baric97}, \citealt[128]{MrazovicVukadinovic09}).\footnote{The conditional is formed with the third person plural form \textit{bi}, the full form \textit{bȉše} is only part of the aorist tense.} 


\begin{table}
\caption{Clitic and full forms of aorist of \textit{biti} `be'\label{table:clbitiao}}
\begin{tabular}{llllll}
\lsptoprule
\multicolumn{3}{c}{singular} & \multicolumn{3}{c}{plural} \\\cmidrule(rl){1-3}\cmidrule(rl){4-6}
 &CL form & Full form && CL form & Full form\\\midrule
be.\textsc{1aor} & \textit{bih}	& \textit{bȉh} & be.\textsc{1aor} & \textit{bismo} & \textit{bȉsmo}\\
be.\textsc{2aor} & \textit{bi}	& \textit{bȉ} & be.\textsc{2aor} & \textit{biste} & \textit{bȉste}\\
be.\textsc{3aor} & \textit{bi}	& \textit{bî} & be.\textsc{3aor} & \textit{bi} & \textit{bȉše, bȉšē}\\
\lspbottomrule
\end{tabular}
\end{table}


Again, the only difference is in the interpretation: some authors believe that aoristal forms are not stressed, even when they take 1P in the sentence, while others claim that aoristal forms can be stressed. With respect to this \citet[28]{PiperKlajn14} state that the CLs \textit{bih}, \textit{bi}, etc. and \textit{li} do not have stressed counterparts. In contrast, many other authors of Bosnian, Croatian, and Serbian grammar books (e.g. \citealt[498]{Katicic86}, \citealt[246]{TezakBabic96}, \citealt[272]{JHP00}, \citealt[284]{Popovic04}, \citealt[124]{MrazovicVukadinovic09}, \citealt[264, 565]{Ridjanovic12}) explicitly state that stressed counterparts of aoristal forms of the verb \textit{biti} do exist.\footnote{\citet[271]{Baric97} list \textit{bi} in the paradigm for stressed forms of the aorist for second and third person singular, but without any accentuation symbol.} \citet[498]{Katicic86} claims that the auxiliary forms of \textit{biti}, with which the conditional is compounded, are stressed forms when followed by \textit{li}. Furthermore, in one Croatian grammar the following example (\ref{(6.1)}) of the stressed \textit{bih} can be found in a contrasting conditional subordinative sentence:

\begin{exe}
\ex\label{(6.1)}
\gll Àko tȋ\textsubscript{1} nè bi\textsubscript{1}, jȃ\textsubscript{2} bȉh\textsubscript{2}\\
 if you \textsc{neg} \textsc{cond.3sg} I \textsc{cond.1sg}\\
\glt ‘If you would not, I would.’
\hfill (Cr; \citealt[72]{Baric97})
\end{exe}

\noindent Although there is no diatopic variation between the BCS standard varieties, diaphasic variation does exist, since in the colloquial language of all three varieties the invariable form \textit{bi} is used for all persons (\citealt[cf.][106]{FHM06}, \citealt[159]{MrazovicVukadinovic09}). 

\subsubsection{Present tense/future clitics of \textit{ht(j)eti} `will'}

In comparison with the full and CL forms of the verb \textit{biti}, the picture of \textit{ht(j)eti} `will' seems quite clear.\footnote{Only \textit{htjeti} is used in standard Croatian and standard Bosnian, while both \textit{htjeti} and \textit{hteti} are allowed in standard Serbian, although the latter clearly dominates in the Serbian WaC corpus.} \textit{Ht(j)eti} has full and CL forms in the present tense \citep[199]{PiperKlajn14}. The CL forms are used to build the future tense. The full forms have a different meaning, i.e. `wish', but they can be used in the future tense as well, for instance to form questions \citep[199]{PiperKlajn14}. Table \ref{table:clhtjeti} presents CL and full forms of \textit{ht(j)eti} (\citealt[cf.][272]{Baric97}, \citealt[125]{MrazovicVukadinovic09}).


\begin{table}
\caption{Clitic and full forms of the present tense of \textit{ht(j)eti} `will'\label{table:clhtjeti}}
\begin{tabular}{llllll}
\lsptoprule
\multicolumn{3}{c}{Singular} & \multicolumn{3}{c}{Plural} \\\cmidrule(lr){1-3}\cmidrule(lr){4-6}
&CL form & Full form && CL form & Full form\\\midrule
will.\textsc{1sg}	&\textit{ću}		&\textit{hòću}		&will.\textsc{1pl}	&\textit{ćemo}	&\textit{hȍćemo, hòćemo}	\\
will.\textsc{2sg}	&\textit{ćeš}		&\textit{hȍćeš}		&will.\textsc{2pl}	&\textit{ćete}	&\textit{hȍćete, hòćete}	\\
will.\textsc{3sg}	&\textit{će}		&\textit{hȍće}		&will.\textsc{3pl}	&\textit{će}		&\textit{hòćē}		\\
\lspbottomrule
\end{tabular}
\end{table}


The only diatopic variation attested in the BCS standard varieties with respect to \textit{ht(j)eti} is in the accent in the full forms, i.e. \textit{hȍćemo} (Croatian) vs \textit{hòćemo} (Serbian) (\citealt[cf.][272]{Baric97}, \citealt[125]{MrazovicVukadinovic09}). Apart from this, there is purely orthographic variation between BCS standard varieties: while in standard Serbian the CL forms of \textit{ht(j)eti} merge together with the \textit{-ti} infinitive, i.e. \textit{písaću}, \textit{písaćeš}, this is not the case in standard Croatian, i.e. \textit{písat} \textit{ću}, \textit{písat} \textit{ćeš} `I will write, you will write' (cf. \citealt[155]{MrazovicVukadinovic09}, \citealt[241]{Baric97}). There is no difference in pronunciation.

\subsection{Reflexive markers \textit{se} and \textit{si} in BCS standard varieties}
\label{Reflexive markers se and si in BCS standard varieties}
There is some diatopic difference in the inventory of reflexive CLs between BCS standard varieties. None of the Serbian authors list \textsc{refl\textsubscript{2nd}} \textit{si} as a CL dative form, and \citet[56]{RadanovicKocic88} explicitly says, “The dative clitic of reflexive pronoun has been completely lost in the Eastern variant, but it is still used in Western variant […]''. While the past 60 years have seen some vacillation in Croatian grammaticography regarding the standardness of the \textsc{refl\textsubscript{2nd}} CL \textit{si}, the stance of Serbian authors on the issue was rather stable: they \citep[e.g.][97]{StanojcicPopovic02} either did not list the reflexive CL form \textit{si} or they \citep[e.g.][367]{MrazovicVukadinovic09} explicitly stated that there is no such form in standard Serbian.\footnote{In the former Yugoslavia some Croatian grammarians \citep[e.g.][96]{BHZ63}, like all Serbian grammarians, did not consider the reflexive CL form \textit{si} to be part of the standard language and they did not list it with other CLs. In their publications from the former Yugoslavian period, other Croatian authors \citep[e.g.][364]{BMPV71} tried to defend the standardness of the reflexive CL \textit{si}. \citet[][364]{BMPV71} claimed that the CL dative form \textit{si} is correct and necessary in the language. The latter view prevailed in Croatian grammaticography since all grammar books published after 1990 list the CL reflexive form \textit{si}.} 


\begin{table}
\caption{Reflexive \textit{se} in BCS and its corresponding full forms\label{table:reflex}}
\begin{tabular}{lllllll}
\lsptoprule
&\multicolumn{2}{c}{Genitive} & \multicolumn{2}{c}{Accusative}& \multicolumn{2}{c}{Dative}\\\cmidrule(lr){2-3}\cmidrule(lr){4-5}\cmidrule(lr){6-7}
& CL form & Full form & CL form & Full form & CL form & Full form\\\midrule
\textsc{refl} & \textit{se} & \textit{sȅbe, sèbe} & \textit{se} & \textit{sȅbe, sèbe} & \textit{si} & \textit{sȅbi, sèbi}\\
\lspbottomrule
\end{tabular}
\end{table}


However, it is clear from Table \ref{table:reflex} that the authors (\citealt[e.g.][208]{Baric97}, \citealt[367]{MrazovicVukadinovic09}) ascribe case to the reflexive markers. While \citet[367]{MrazovicVukadinovic09} say that there is only one CL \textit{se} which is in the accusative case, \citet[208]{Baric97} claim that there are three CL forms: in the accusative, genitive and dative. Without discussing this issue here, we can say that BCS grammarians traditionally distinguish between the reflexive pronoun \textit{se} (\textsc{refl\textsubscript{2nd}} in the terminology proposed in Section \ref{Conclusion: how many types of se do we need to distinguish?}) and the reflexive particle \textit{se} (\textsc{refl\textsubscript{lex}}). \citet[558f]{Ridjanovic12} is the only author who explicitly claims that the \textsc{refl\textsubscript{lex}} \textit{se} has only the CL form and no corresponding full form. As we already pointed out in Chapter \ref{Our terms and concepts}, we do not use the term reflexive pronoun. Instead we use the term reflexive marker.

The diatopic variation in the inventory of BCS standard varieties is nicely summed up by \citet[440]{Ridjanovic12} who claims that the \textsc{refl\textsubscript{2nd}} CL \textit{si}, which is widely used in Croatian, can hardly be found elsewhere in BCS territory.\footnote{As we show in Section \ref{Reflexive clitic si}, this claim is not completely true; while it may apply to standard varieties, there is undeniable diatopic/dialectal variation – we provide data on the usage of the \textsc{refl\textsubscript{2nd}} CL \textit{si} outside Croatian territory. Moreover, the form in question is present also in the spoken Bosnian variety, for more details see Section \ref{Reflexive clitics:9}.}

\subsection{Polar question marker \textit{li}}

Finally, we would like to observe that there is no variation in respect of the CL particle \textit{li} in BCS varieties. This is why it will not be part of our research focus, as we already pointed out in Chapter \ref{Our terms and concepts}.

\section{Internal organisation of the clitic cluster}
\label{Internal organisation of the clitic cluster:7}
\subsection{Clitic ordering within the cluster in BCS standard varieties}
\label{Clitic ordering within the cluster in BCS standard varieties}
In this section we will summarise the main information on CL clusters found in literature. Several authors (e.g. \citealt[471]{JHP00}, \citealt[284, 289]{Popovic04}, \citealt[451]{PiperKlajn14}) observe that if there is more than one CL in the same simple clause, CLs will group and linearise. \citet[451f]{PiperKlajn14} point out that the CL cluster usually consists of two or three elements, rarely four, while groups of five or more CLs are quite infrequent. Similarly, \citet[558]{Ridjanovic12} claims that the maximum number of CLs in one cluster is five, commenting that such clusters are very rare. We will start our discussion with the cluster ordering presented in Section \ref{Clitic ordering within the cluster}:

\begin{exe}\sn
\textit{li} ${}>{}$ \textsc{verbal}* ${}>{}$ \textsc{pron\textsubscript{dat}} ${}>{}$ \textsc{pron\textsubscript{acc}} ${}>{}$ \textsc{pron\textsubscript{gen}} ${}>{}$ \textsc{refl} ${}>{}$ \textit{je} \\
* except \textit{je} = \textsc{prs.3sg} of \textit{biti} `be'
\end{exe}

While reviewing the grammar books, we encountered divergent information concerning the ordering of pronouns in the accusative and genitive. Note that these are homophonous forms. Some authors (e.g. \citealt[246]{TezakBabic96}, \citealt[596]{Baric97}, \citealt[472]{JHP00}, \citealt[659]{MrazovicVukadinovic09}) propose the order dative > genitive > accusative. \citet[596f]{Baric97} support this order with the following two examples (presented in (\ref{(6.2)}) and (\ref{(6.3)})) which contain \textsc{refl\textsubscript{lex}}.

\begin{exe}\ex\label{(6.2)}
\gll On \textbf{joj} \textbf{se} nasmiješio. \\
 he her\textsc{.dat} \textsc{refl} smile\textsc{.ptcp.sg.m} \\
 \glt ‘He smiled at her.’ 
\hfill  (Cr; \citealt[596]{Baric97})

\ex\label{(6.3)}
\gll Djeca \textbf{su} \textbf{ga} \textbf{se} nagledala. \\
 children be\textsc{.3pl} him\textsc{.gen} \textsc{refl} look.at\textsc{.ptcp.pl.n} \\
 \glt ‘The children saw enough of him.’ 
\hfill  (Cr; \citealt[597]{Baric97})
\end{exe}

\noindent However, we would like to point out that this argumentation hinges solely on the doubtful interpretation that \textit{se} is marked for accusative case. With respect to CL order in the cluster, some authors (e.g. \citealt[104]{PiperIvic05}, \citealt[564]{Ridjanovic12}, \citealt[451]{PiperKlajn14}) even claim that the dative comes first and then the accusative or genitive, which basically means that one of the two can be expected. However, the Serbian linguists \citet[452]{PiperKlajn14} later explicitly state that the pronominal accusative CL stands before the genitive one, like in the permuted example (\ref{(6.4a)}) in which the genitive complement \textit{svoje pažnje} ‘their own attention’ of the verb \textit{lišiti} ‘deprive’ has been replaced by the genitive CL \textit{je} ‘her’.

\begin{exe}\ex\begin{xlist}
\ex\label{(6.4a)}
\gll Lišili \textbf{su} \textbf{ih} svoje pažnje. \\
 deprive\textsc{.ptcp.pl.m} be\textsc{.3pl} them\textsc{.acc} own attention \\
\glt ‘They deprived them of their attention.’ 
\ex\label{(6.4b)}
\gll Lišili \textbf{su} \textbf{ih} \textbf{je}. \\
 deprive\textsc{.ptcp.pl.m} be\textsc{.3pl} them\textsc{.acc} her\textsc{.gen} \\
 \glt ‘They deprived them of it.’ 
\hfill (Sr; \citealt[452]{PiperKlajn14})
\end{xlist}
\end{exe}

\noindent In contrast two other Serbian linguists, \citet[659]{MrazovicVukadinovic09}, claim that there are no patterns combining accusative and genitive CLs in standard Serbian. They emphasise that the CL form of the genitive is never used together with the accusative: for example, they say that in standard Serbian the sentence presented in (\ref{(6.5a)}) cannot be paraphrased as (\ref{(6.5b)}) but only as (\ref{(6.5c)}) \citep[cf.][659]{MrazovicVukadinovic09}. 
 
\begin{exe}\ex\begin{xlist}
\ex[]{\label{(6.5a)}
\gll Vlasti \textbf{su} lišile narod slobode govora. \\
 authorities be\textsc{.3pl} deprive\textsc{.ptcp.pl.f} folk freedom speech\\ 
\glt ‘The authorities deprived the people of freedom of speech.’}
\ex[*]{\label{(6.5b)}
\gll Vlasti \textbf{su} \textbf{ga} \textbf{je} lišile. \\
 authorities be\textsc{.3pl} him\textsc{.acc} her\textsc{.gen} deprive\textsc{.ptcp.pl.f}\\ 
\glt Intended: ‘The authorities deprived him of it.’}
\ex[]{\label{(6.5c)}
\gll Vlasti \textbf{su} \textbf{ga} lišile nje / toga / slobode govora.\\
 authorities be\textsc{.3pl} him\textsc{.acc} deprive\textsc{.ptcp.pl.f} her\textsc{.gen} {} that {} freedom of.speech\\ 
\glt ‘The authorities deprived him of it/that/freedom of speech.’ \\}
\strut\hfill (Sr; \citealt[659]{MrazovicVukadinovic09})
\end{xlist}
\end{exe}

\noindent The Bosnian linguist \citet[565]{Ridjanovic12} shares the latter point of view that genitive and accusative CLs never occur together. An interesting discussion of this problem is offered by \citet[104ff]{Milicevic07}, who presents an account of the constitution of the CL cluster within the framework of Meaning-Text Theory. She claims that in most cases genitive and accusative CLs may come in either order. For her either of the variants presented in (\ref{(6.6a)}--\ref{(6.6d)}) is acceptable:

\begin{exe}
\ex\begin{xlist}
\ex[]{\label{(6.6a)}
\gll Lišili su \textbf{ih} \textbf{ga}. \\
 deprive\textsc{.ptcp.pl.m} be\textsc{.3pl} them\textsc{.acc} him\textsc{.gen} \\
\glt ‘They deprived them of it/him.’\\}
\ex[?]{\label{(6.6c)}
\gll Lišili su \textbf{ih} \textbf{ga}. \\
deprive\textsc{.ptcp.pl.m} be\textsc{.3pl} them\textsc{.gen\textsubscript{?}} him\textsc{.acc\textsubscript{?}} \\
\glt ‘They deprived him of them.’\\}
\ex[?]{\label{(6.6b)}
\gll  Lišili su \textbf{ga} \textbf{ih}. \\
 deprive\textsc{.ptcp.pl.m} be\textsc{.3pl} him\textsc{.acc} them\textsc{.gen} \\
\glt ‘They deprived him of them.’ \\}
\ex[?]{\label{(6.6d)}
\gll  Lišili su \textbf{ga} \textbf{ih}. \\
deprive\textsc{.ptcp.pl.m} be\textsc{.3pl} him\textsc{.gen\textsubscript{?}} them\textsc{.acc\textsubscript{?}} \\
\glt ‘They deprived them of it/him.’
\hfill (Sr; \citealt[105]{Milicevic07})}
\end{xlist}
\end{exe}

\noindent She claims that this sentence can also be read either way. Due to this ambiguity, she argues that the order accusative > genitive seems to be “the default case”.

The Serbian linguist \citet[291]{Popovic04} is the only one who notes that the presented CL order within the cluster stays the same even if the cluster consists of CLs which are governed by two different verbs, i.e. the CL order in simple and mixed clusters is the same.\footnote{For the difference between simple and mixed clusters see Section \ref{Clitic ordering within the cluster}.} \citet[660]{MrazovicVukadinovic09} are the only ones who claim that permutations of CL order in the cluster are not possible and that no other element can be inserted into the CL cluster in standard Serbian. 

At the end of this subsection we would like to point out that we have not observed any diatopic variation between BCS standard varieties in respect of CL order within the cluster. Any observed differences actually arise from obviously disparate interpretations, primarily with respect to the realisation of clusters with both genitive and accusative CLs.

There is no information on diaphasic variation in the grammar books. We have only come across a short article by \citet{Ondrus57}, who claims that in the Serbian colloquial register the verbal CL \textit{je} ‘is’ can precede pronominal CLs \citep[cf.][517f]{Ondrus57}. He provides the following example (\ref{(6.7)}) of the reversed order from colloquial Serbian:\footnote{This reversed order in CL clusters can be quickly verified in srWaC and hrWaC. In srWaC v1.2 we found 152 (0.3 per million) examples with \textit{je}\textsubscript{be.\textsc{3sg}} \textit{ga}\textsubscript{him.\textsc{acc}} and 183 (0.3 per million) examples with \textit{je}\textsubscript{be.\textsc{3sg}} \textit{mu}\textsubscript{him.\textsc{dat}} word order. It seems that the mentioned word order is even more frequent in Croatian, since \textit{je}\textsubscript{be.\textsc{3sg}} \textit{ga}\textsubscript{him.\textsc{acc}} is attested by 681 (0.5 per million) and \textit{je}\textsubscript{be.\textsc{3sg}} \textit{mu}\textsubscript{him.\textsc{dat}} by 531 (0.4 per million) examples.}\textsuperscript{,}\footnote{This kind of reversed CL order in which pronominal CLs are preceded by the verbal CL \textit{je} is also attested in dialects. For more information see Section \ref{Clitic ordering within the cluster:8}.}

\protectedex{\begin{exe}
\ex\label{(6.7)}
\gll Žao \textbf{joj} \textbf{je} \textbf{ga}. \\
 sorry her\textsc{.dat} be\textsc{.3sg} him\textsc{.acc} \\
\glt ‘She is sorry for him.’ 
\hfill (Sr; \citealt[518]{Ondrus57})
\end{exe}
}

\subsection{Morphonological processes within the cluster}
\subsubsection{Suppletion}
\label{Suppletion}
Most authors (e.g. \citealt[306]{Stevanovic75}, \citealt[210, 597]{Baric97}, \citealt[472]{JHP00}, \citealt[366]{MrazovicVukadinovic09}, \citealt[434]{Ridjanovic12}, \citealt[28, 97]{PiperKlajn14}) generally agree that if the pronominal CL \textit{je} ‘her’ precedes the verbal CL \textit{je} ‘is’, the former will be replaced by its alternative form \textit{ju}. However, the Bosnian linguist \citet[434]{Ridjanovic12} claims that this dissimilated form is a feature of deliberate speech. Furthermore, he insists that in everyday colloquial language Bosnians use just one \textit{je} instead of two CLs, meaning they prefer haplology \citep[cf.][434]{Ridjanovic12}.\footnote{It seems that Ridjanović’s observation might be correct. While analysing language material for Chapter \ref{Clitics in dialects} we could not find examples from local idioms (dialects) in which suppletion does take place. Moreover, our colleagues from Croatia and Serbia who specialise in dialectology could not provide us with examples of suppletion from their transcripts either. However, there are local idioms (dialects) in which the string \textit{ju je} occurs, but not as a consequence of suppletion since the CL \textit{ju} is the only third person singular feminine accusative CL form available.} Unfortunately, he does not provide any examples for this, but \citet[97]{PiperKlajn14} observe a similar phenomenon among Serbian native speakers and label it as a common mistake in standard Serbian, see (\ref{(6.13)}):

\protectedex{\begin{exe}
\ex[*]{\label{(6.13)}
\gll Teorija nosi ime naučnika koji \textbf{je} prvi formulisao. \\
 theory carry\textsc{.3prs} name scientist which be\textsc{.3sg} first formulate\textsc{.ptcp.sg.m} \\ 
\glt Intended: ‘The theory carries the name of the scientist who first formulated it.’ 
\hfill (Sr; \citealt[97]{PiperKlajn14})}
\end{exe}
}

\noindent It seems that the mentioned feature, which is not accepted in standard Bosnian and Serbian, is an actual piece of evidence for diaphasic variation, since it occurs in non-standard varieties spoken by native speakers. 

Suppletion also occurs in both standard Serbian and standard Croatian if the third person feminine accusative CL stands after \textit{nije} ‘is not’ or another verb which ends with \textit{-je} (\citealt[cf.][210, 597]{Baric97}, \citealt[366]{MrazovicVukadinovic09}, \citealt[97]{PiperKlajn14}), as in example (\ref{(6.14)}).

\protectedex{\begin{exe}
\ex\label{(6.14)}
\gll Ne smije \textbf{ju} ni vidjeti.\\
 \textsc{neg} may\textsc{.3prs} her\textsc{.acc} \textsc{neg} see\textsc{.inf} \\
\glt ‘He may not even see her.’ 
\hfill (Cr; \citealt[597]{Baric97})
\end{exe}
}

\noindent It seems that in Serbian the suppletion in the last two cases mentioned is the result of a change in the norm. The rule that in standard Serbian the accusative CL \textit{je} must be replaced with its counterpart CL \textit{ju} after verbs which end in \textit{-je}, or after \textit{nije} ‘is not’ emerged during recent decades, since in the 1970s \citet[306]{Stevanovic75} claimed that \textit{ju} could be used only in combination with \textit{je}. As we have already pointed out in Section \ref{Inventory:7}, \citet[306]{Stevanovic75} argued that all uses of \textit{ju} which are not the result of its placement after the verbal CL \textit{je} are dialectal. 

At the end of this section we would like to point out that we did not observe any diatopic variation between the BCS standard varieties with respect to suppletion, which as a phenomenon exists in all BCS standard varieties. However, we have demonstrated that BCS standard varieties do differ as to the range of contexts in which suppletion is recommended. 

\subsubsection{Haplology of unlikes}
\label{Haplology of unlikes}
Several BCS grammar books (e.g. \citealt[246]{TezakBabic96}, \citealt[596]{Baric97}, \citealt[471]{JHP00}, \citealt[302, 333]{Ridjanovic12}, \citealt[450]{PiperKlajn14}) mention haplology of unlikes, i.e. that the verbal CL \textit{je} ‘is’ is deleted if it would follow the reflexive CL \textit{se}.\footnote{For more information on the \textcolor{black}{haplology of unlikes} see Section \ref{Morphonological processes within the cluster}.} The Croatian linguist \citet[497]{Katicic86} claims that such deletion usually occurs, but it is not necessarily a general rule. He thinks that keeping the reflexive CL \textit{se} is a feature of a pedantic and explicit style of expression \citep[cf.][497]{Katicic86}. Similar statements can be found in descriptions of standard Serbian. Namely, \citet[452]{PiperKlajn14} are even stricter when it comes to the \textit{se} \textit{je} cluster: they explicitly mark example (\ref{(6.8a)}) as incorrect, but consider example (\ref{(6.8b)}) to be correct in standard Serbian.

\begin{exe}\ex\begin{xlist}
\ex[*]{\label{(6.8a)}
\gll  On \textbf{se} \textbf{je} obradovao. \\
 he \textsc{refl} be\textsc{.3sg} gladden\textsc{.ptcp.sg.m} \\ 
\glt Intended: ‘He was gladdened.’ }
\ex[]{\label{(6.8b)}
\gll On \textbf{se} obradovao. \\
 he \textsc{refl} gladden\textsc{.ptcp.sg.m} \\ 
\glt ‘He was gladdened.’
\hfill (Sr; \citealt[452]{PiperKlajn14})}
\end{xlist}
\end{exe}

\noindent Regarding the omission of the verbal CL \textit{je}, which is in contact position with the reflexive CL \textit{se}, the Bosnian linguist \citet{Ridjanovic12} offers a syntactic explanation. He claims that not every verbal CL \textit{je} is omitted in the combination with the reflexive CL \textit{se}  \citep[cf.][564]{Ridjanovic12}. According to him, in standard Bosnian omission is possible as long as the verbal CL \textit{je} is a past tense auxiliary  \citep[cf.][564]{Ridjanovic12}. However, if the verbal CL \textit{je} is a copula like in example (\ref{(6.9)}), it will not be omitted  \citep[cf.][564]{Ridjanovic12}. %very often the exact same source

\protectedex{\begin{exe}\ex\label{(6.9)}
\gll Dobro \textbf{se} \textbf{je} nadati. \\
 good \textsc{refl} be\textsc{.3sg} hope\textsc{.inf} \\
\glt ‘It is good to hope.’ 
\hfill (Bs; \citealt[][564]{Ridjanovic12})
\end{exe}}

\noindent While the combination of the reflexive CL \textit{se} and auxiliary CL \textit{je} in the simple CL cluster (\ref{(6.8a)}) leads to the deletion of the auxiliary CL \textit{je} (\ref{(6.8b)}), Ridjanović’s example (\ref{(6.9)}) with the CL copula \textit{je} is a case of a mixed cluster.\footnote{For more information on the simple and mixed CL cluster see Section \ref{Internal organisation of the clitic cluster}.} It seems that whereas the auxiliary CL \textit{je} is regularly omitted in simple CL clusters in BCS standard varieties, in standard Bosnian the CL copula \textit{je} is preserved in mixed clusters if it co-occurs with the reflexive CL \textit{se}.\footnote{It is interesting to note that in the grammar books we did not find any information about co-occurrence of the auxiliary CL \textit{je} in mixed clusters with the reflexive CL \textit{se}. However, in bsWaC we found both examples with haplology of unlikes (\ref{14102021a}) and without it (\ref{14102021b}). 

\begin{exe}\ex\label{14102021a}
\gll Uspio \textbf{se} izvući. \\
 Manage\textsc{.ptcp.sg.m} \textsc{refl} get.out\textsc{.inf} \\
\glt ‘He managed to get out.’ 
\hfill [bsWaC v1.2]

\ex\label{14102021b}
\gll Baš ovdje uspio \textbf{se} \textbf{je} pokazati i kao pjesnik. \\
 exactly here manage\textsc{.ptcp.sg.m} \textsc{refl} be\textsc{.3sg} show.\textsc{inf} and as poet \\
\glt ‘Exactly here he managed to show himself as a poet.’
\hfill [bsWaC v1.2]
\end{exe}}
\citet[246]{TezakBabic96} add that the verbal CL \textit{je} is also often omitted after CLs \textit{me} ‘me’ and \textit{te} ‘you’ in Croatian – see the example presented in (\ref{(6.10)}).

\protectedex{\begin{exe}\ex\label{(6.10)}
\gll Gizela \textbf{me} čekala u posječenom parku. \\
 Gizela me\textsc{.acc} wait\textsc{.ptcp.sg.f} in trimmed park\\
\glt ‘Gizela was waiting for me in the trimmed park.’ 
\hfill (Cr; \citealt[596]{Baric97})
\end{exe}
 }

\noindent This usage should be examined in the context of the so-called truncated perfect (Serbian \textit{krnji perfekat}). In headlines or certain contexts of spoken language the auxiliary can be omitted for all persons irrespective of the presence of other CLs. \citet[98f]{MeermannSonnenhauser16} who analysed this usage in spoken Serbian claim that it involves a distancing mechanism. The truncated perfect indicates a lack of anchoring to the point of speech and serves as a means for the speaker to distance himself or herself from what has been said. While \citet[99f]{MeermannSonnenhauser16} claim that the truncated perfect produces effects of surprise or indignation, some Croatian authors (e.g. \citealt[41, 52, 55]{Katicic86}, \citealt[404, 596]{Baric97}) argue that omitting the auxiliary CLs \textit{je} and \textit{su} brings a stylistic value of greater brevity and expressivity. 

While the omission of the verbal CL \textit{je} after the reflexive CL \textit{se} and pronominal CLs \textit{me} and \textit{te} is phonologically motivated, i.e. to avoid duplication of the same vowel, according to several Croatian authors (e.g. \citealt[129]{TezakBabic96}, \citealt[41, 52, 55]{Katicic86}, \citealt[404, 596]{Baric97}), not only the verbal CL \textit{je} but also \textit{su} as an auxiliary can be omitted, even if there is no pronominal or reflexive CL in the sentence. 

Here we would like to emphasise that we did not observe any diatopic variation between the BCS standard varieties with respect to haplology. The only observed discrepancies concern the interpretation of whether the haplology is obligatory or optional. 

\section{Position of the clitic or the clitic cluster}
\label{Position of the clitic or clitic cluster}
\subsection{General remarks on clitic placement in BCS standard varieties}

Many grammarians comment on the peculiarity of CL placement. In comparison to conjunctions and prepositions, CLs do have greater freedom of positioning \citep[cf.][246]{TezakBabic96}. Therefore, it can be said that the place in the sentence which CLs take is relatively free \citep[cf.][450]{PiperKlajn14}. However, CLs cannot be placed in any position in a sentence \citep[cf.][246]{TezakBabic96}. \citet[470]{JHP00} also emphasise that obligatory word order in Bosnian, which determines the place of clitics (proclitics and enclitics) in a sentence, is controlled only by prosodic rules. In this vein, the Croatian linguist \citet[495]{Katicic86} claims that the positioning of CLs is strictly and mechanically determined, which makes it stylistically neutral \citep[495]{Katicic86}. 

Taking into consideration all the above-mentioned factors, below we present the treatment of CL placement in standard BCS varieties in detail. We discuss the following interrelated factors: breaks (or punctuation), conjunctions and complementisers, 2P, DP and phrase splitting, including the limits of the latter. Phrase splitting will receive the most attention because it is a major source of microvariation in CL placement and it has been studied and discussed in quite some detail. 

\subsection{Placement with respect to breaks in BCS standard varieties}
\label{Placement with respect to breaks in BCS standard varieties}

With regard to CL placement, many authors emphasise that CLs cannot follow a break, which is in line with the so-called phonological and mixed formal approaches to 2P.\footnote{\textcolor{black}{For more information on phonological and mixed formal approaches to 2P see Section \ref{Approaches to 2P effects: syntax, phonology and information structure}}.} A physiological break is a pause needed for normal breathing, and the shortest is realised after a prosodic unit \citep[e.g.][243]{TezakBabic96}. Bosnian, Croatian, and Serbian linguists (\citealt[cf.][246]{TezakBabic96}, \citealt[471]{JHP00}, \citealt[371]{StanojcicPopovic02}, \citealt[283, 303]{Popovic04}, \citealt[105]{PiperIvic05}, \citealt[450]{PiperKlajn14}) agree that CLs cannot follow a break. Therefore the sentence provided in (\ref{(6.15)}) should be considered incorrect (a break is marked by  |)

\protectedex{\begin{exe}
\ex[*]{\label{(6.15)}
\gll Ivica | \textbf{je} potrčao.\\
 Ivica {} be\textsc{.3sg} run\textsc{.ptcp.sg.m}\\ 
\glt Intended: ‘Ivica started running.’
\hfill (Cr; \citealt[243]{TezakBabic96})}
\end{exe}
}

\noindent In some cases in written language breaks are visible in orthography (comma, full stop, colon, etc.), but not always, since orthography only partially correlates with prosody. It seems that there is no diatopic variation between BCS standard varieties in respect of CL position after a break. Specifically, all scholars (e.g. \citealt[246]{TezakBabic96}, \citealt[371]{StanojcicPopovic02}, \citealt[303]{Popovic04}, \citealt[105]{PiperIvic05}, \citealt[450]{PiperKlajn14}) agree that CLs cannot be placed directly after physiological breaks, orthographically represented by a comma (\ref{(6.16b)}) or bracket (\ref{(6.17b)}), for example, or after any other kind of insertion and/or punctuation. 

\begin{exe}\ex\begin{xlist}
\ex[]{\label{(6.16a)}
\gll Taj profesor, poštovana koleginice, napisao \textbf{je} udžbenik za svaki predmet koji je predavao. \\
 that professor respected colleague write\textsc{.ptcp.sg.m} be\textsc{.3sg} textbook for every subject which be\textsc{.3sg} teach\textsc{.ptcp.sg.m}\\ 
 \glt ‘That professor, respected colleague, wrote the textbook for every subject he taught.’}
\ex[*]{\label{(6.16b)}
\gll Taj profesor, poštovana koleginice, \textbf{je} {napisao [\dots].}\\
 that professor respected colleague be\textsc{.3sg} write\textsc{.ptcp.sg.m} \\}
\hfill (Sr; \citealt[450]{PiperKlajn14})
\end{xlist}
\ex\begin{xlist}
\ex[]{\label{(6.17a)}
\gll Genitiv partitivni {(}piti                vode) uzmanje \textbf{je} dijelova od cjeline. \\
 genitive partial       \,drink\textsc{.inf} water taking be\textsc{.3sg} parts of whole. \\
\glt ‘The partial genitive (drink water) is used for parts of a whole.’}
\ex[*]{\label{(6.17b)}
\gll Genitiv partitivni {(}piti          vode{)} \textbf{je}     {uzimanje [\dots].}\\
     genitive partial   \,drink\textsc{.inf} water be\textsc{.3sg} taking\\}
\hfill (Cr, \citealt[246]{TezakBabic96}) 
\end{xlist}
\end{exe}

\noindent The Serbian linguist \citet[307]{Popovic04} notes that people tend to place CLs after commas although it is against the norms of standard language. One of the Serbian grammar books \citep[e.g.][105]{PiperIvic05} recommends the use of full instead of CL forms directly after a break, like in example (\ref{(6.18)}) below.

\begin{exe}\ex\label{(6.18)}
\gll Tobožnji snimak, uprkos svim uveravanjima, \textbf{jeste} falsifikat. \\
 supposed recording despite all assurances be\textsc{.3sg} counterfeit \\
\glt ‘The supposed recording, despite all assurances, is counterfeit.’ \\
\hfill (Sr; \citealt[105]{PiperIvic05})
\end{exe}

\noindent Heavy phrases are also recognised as a factor influencing CL placement. The Croatian authors \citet[246]{TezakBabic96} emphasise that CLs cannot follow longer syntagms in standard Croatian. As an illustration, they provide examples with a heavy phrase presented in (\ref{(6.19a)}) and (\ref{(6.19b)}). 

\begin{exe}\ex\begin{xlist}\ex[]{\label{(6.19a)}
\gll \minsp{``} Tri dana kod sina'' mala \textbf{j}e, ali značajna pripovijetka. \\
 {} three day at son small be\textsc{.3sg} but important story \\ 
 \glt`\textit{Three days with the son} is a small, but important story.’}
\ex[*]{\label{(6.19b)}
\gll \minsp{``} Tri dana kod sina'', \textbf{je} {mala [\dots].}\\
 {} three day at son be\textsc{.3sg} small \\ }
\hfill (Cr; \citealt[246]{TezakBabic96})
\end{xlist}
\end{exe}

\noindent Similarly, the Serbian authors \citet[450]{PiperKlajn14} point out that in standard Serbian, CLs do not follow a long initial phrase after which the break is – as they say – more expressed. In such cases, CLs follow the next stressed word, like in example (\ref{(6.20)}) provided below \citep[cf.][450]{PiperKlajn14}.

\begin{exe}\ex\label{(6.20)}
\gll Profesor uvoda u lingvistiku dobar \textbf{je} čovek.\\
 professor introduction in linguistics good be\textsc{.3sg} man \\
\glt ‘The Introduction to Linguistics professor is a good man.’ \\
\hfill (Sr; \citealt[450]{PiperKlajn14})
\end{exe}

\noindent Both Croatian and Serbian authors offer a prosodic explanation (physiological break) for this particular case of CL placement, which is supported by syntactic arguments (long initial phrase). However, it is important to emphasize that neither the Croatian nor the Serbian authors specify how to determine longer syntagms. From the treatment of DP in the grammars we can only infer that there must be language-internal variation (within one standard variety).\footnote{For more information on this see Section \ref{Second position, second word and delayed placement}. } 

\subsection{Placement with regard to different types of hosts in BCS standard varieties}
\label{Placement with regard to different types of hosts in BCS standard varieties}
CL positioning after conjunctions and complementisers varies. CLs cannot be placed either after the negative particle \textit{ne} ‘not’ or after the conjunctions \textit{a} and \textit{i} ‘and’, but they can directly follow the conjunctions \textit{pa} and \textit{te} ‘so, and, then’ (\citealt[cf.][595]{Baric97}, \citealt[471]{JHP00}, \citealt[371]{StanojcicPopovic02}, \citealt[297]{Popovic04}, \citealt[451]{PiperKlajn14}).\footnote{Examples \textcolor{black}{with CLs placed after the conjunctions \textit{a} and \textit{i}} can be found in \textit{Šumadijsko-vojvođanski}, \textit{Istočnohercegovački}, and \textit{Srednjobosanski} dialects. For more information see Section \ref{Clitic first}.} The negative coordinative conjunction \textit{ni} ‘neither/nor’ cannot as a proclitic be a host for enclitics, whereas \textit{niti} ‘neither/nor’ can (\citealt[cf.][537, 562]{Ridjanovic12}, \citealt[297]{Popovic04}, \citealt[451]{PiperKlajn14}). The coordinative conjunction \textit{no} behaves ambiguously: when synonymous with \textit{ali} ‘but’ it cannot host CLs, but when synonymous with \textit{nego} and \textit{već} ‘than’, it can (\citealt[cf.][371]{StanojcicPopovic02}, \citealt[298]{Popovic04}). The bookish particle \textit{pak} ‘however’ can only follow CLs \citep[cf.][451]{PiperKlajn14}, as in example (\ref{(6.21)}) presented below.

\begin{exe}\ex\label{(6.21)}
\gll Ona \textit{je} htela da \textit{im} pomogne, on \textbf{joj} \minsp{[} pak] to nije dozvolio. \\
 she be\textsc{.3sg} want\textsc{.ptcp.sg.f} that them\textsc{.dat} help\textsc{.3prs} he her\textsc{.dat} {} but that \textsc{neg.}be\textsc{.3sg} allow\textsc{.ptcp.sg.m}\\
\glt ‘She wanted to help them, but he did not allow her to.’ \\
\hfill (Sr; \citealt[][451]{PiperKlajn14})
\end{exe}

\noindent CLs can follow the coordinating conjunctions \textit{ali} and \textit{ili}, and according to Bosnian and Serbian literature (e.g. \citealt[471]{JHP00}, \citealt[371]{StanojcicPopovic02}, \citealt[284, 298f]{Popovic04}, \citealt[562]{Ridjanovic12}, \citealt[451]{PiperKlajn14}) they must follow all complementisers.\footnote{Croatian authors \citet[595]{Baric97} only claim that CLs follow question and relative complementisers, but from their statement it is not clear whether these complementisers are the only complementisers which CLs follow, nor whether CLs must or can follow them.}\textsuperscript{,}\footnote{This may be true for standard Bosnian and standard Serbian. However, as dialectological data show for the \textit{Šumadijsko-vojvođanski} dialect, CLs do not always follow the complementiser \textit{da}. Moreover, our data from srWaC with CC out of \textit{da}\textsubscript{2}-complements also show that CLs do not always follow the complementiser \textit{da}. For more information and examples see Section \ref{Delayed placement of clitics} and Chapter \ref{A corpus-based study on CC in da constructions and the raising-control distinction (Serbian)}.} In example (\ref{(6.22)}) presented below the verbal CL \textit{je} `is’ directly follows the complementiser \textit{koji} `which’.

\begin{exe}\ex\label{(6.22)}
\gll Čita roman koji \textbf{je} napisao jedan mladi pisac.\\
 read\textsc{.3prs} novel which be\textsc{.3sg} write\textsc{.ptcp.sg.m} one young writer\\
\glt ‘He is reading a novel written by a young writer.’  \\
\hfill (Sr; \citealt[450]{PiperKlajn14})
\end{exe}

\noindent The Serbian author \citet[300f]{Popovic04} is the only one who notes the difference between the two kinds of \textit{jer} ‘because’. He claims that CLs follow the causal complementiser \textit{jer}, while \textit{jer} as a connector (\textit{nadovezivački veznik}) is not followed by CLs since a break can be felt after it \citep[cf.][300f]{Popovic04}. 

We can sum this subsection up as follows: there is no diatopic variation between standard BCS varieties as regards CL placement in the case of the negative particle \textit{ne} and the conjunctions \textit{a}, \textit{i}, and \textit{ni}. They cannot host CLs. By contrast, all South-Slavonic grammarians recognise that the conjunctions \textit{pa}, \textit{te}, \textit{niti}, \textit{ali}, and \textit{ili} can be hosts to CLs. However, since they do not state that the mentioned conjunctions must host CLs, we can expect variation within each standard variety. Furthermore, Bosnian and Serbian authors emphasise that CLs must follow all complementisers. In contrast, Croatian authors \citet[][595]{Baric97} only state that CLs follow question and relative complementisers, but they do not specify whether such complementisers are obligatory hosts to CLs. Hence, here we can expect variation within one standard variety and possibly diatopic variation between different BCS standard varieties.  

\subsection[Second position, second word, and delayed placement]{Second position, second word and delayed placement}
\label{Second position, second word and delayed placement}

\subsubsection{Second position vs second word in BCS standard varieties}
\label{Second position vs second word in BCS standard varieties}
While above we presented the problem of CL placement in BCS in a broader context, i.e. with respect to elements which can host CLs, this section deals specifically with the vague treatment of 2P.\footnote{For basic information \textcolor{black}{and theoretical discussion} on 2P and DP see \textcolor{black}{Sections \ref{Second position}, \ref{Approaches to 2P effects: syntax, phonology and information structure}, and \ref{Barriers and delayed placement of clitics (DP)}}.} Croatian and Serbian authors have different approaches to what 2P actually is. As we demonstrate in the following, Serbian linguists tend to interpret 2P as the position after the first phrase, which can but does not need to be compound, while Bosnian and Croatian authors take it to mean the position after the first word (the 2W solution).\footnote{The term ``compound phrase'' refers to a phrase which consists of at least two content words.} 

All the analysed grammar books of BCS standard varieties (e.g. \citealt[495]{Katicic86}, \citealt[471]{JHP00}, \citealt[17]{Popovic04}, \citealt[29, 450]{PiperKlajn14}) state that CLs attach to the preceding stressed word, and that they are consequently placed in the second position in the sentence, in principle after the first stressed word. However, in some grammar books \citep[e.g][105]{PiperIvic05} it is emphasised that the rule in question refers to the simple clause. Moreover, as we show, there is continuous discussion on the insertion of CLs after initial compound phrases, even after those which consist of only two stressed content words like \textit{moj prijatelj} ‘my friend’ in example (\ref{(6.23)}) below.\footnote{In idioms of the Neo-Štokavian \textit{Istočnohercegovački} dialect, which was one of the dialects that served as a base for standard Croatian, CLs can follow phrases which consist of two content words, for more information see Section \ref{Second position:8}. }

%1958/59 and 64/65 ...no
Even in Yugoslavian times is was clear that the 2P after the first stressed word, i.e. 2W, was typical of the Western part of Serbo-Croatian language territory \citep[cf.][307]{Pesikan58}. The Croatian linguist \citet[154f]{Babic64}, for instance, rejected any possibility of CL insertion after a syntagm, even one containing only two stressed content words. However, it seems that not all Croatian linguists of the time agreed with Babić. \citet[434]{BMPV71} and \citet[146f]{Brabec65}, for instance, allowed sentences in which a CL follows a two-word phrase, like in the example provided below where the verbal CL \textit{je} ‘is’ attaches to the initial compound phrase \textit{moj prijatelj}.

\begin{exe}\ex\label{(6.23)}
\gll \minsp{[} Moj prijatelj]\textsubscript{phrase1} \textbf{je} jučer došao k nama.\\
 {} my friend be\textsc{.3sg} yesterday come\textsc{.ptcp.sg.m} to us\\
\glt ‘My friend came to us yesterday.’
\hfill (Cr; \citealt[434]{BMPV71})
\end{exe}

\largerpage
\noindent Half a century later, Croatian linguists still disagree on the question whether CLs can follow compound phrases of two stressed content words or not. While \citet[344]{Raguz97} allows it, \citet[182]{FHM06} reject such a possibility. \citet[496f]{Katicic86} claims that placing the CL directly after the first compound phrase bears the hallmarks of a substandard colloquial expression.\footnote{This does not mean that instead of 2P, Croatian authors prefer DP of CLs. As we have already pointed out in the previous lines, \citet[307]{Pesikan58} observed that placing CLs after the first stressed word was typical of the Western part of Serbo-Croatian language territory. In the next sections it becomes obvious that in standard Croatian 2W is not less preferred than DP of CLs.} However, unlike Croatian linguists, the Serbian authors \citet[29, 450]{PiperKlajn14} and \citet[161]{IKPB11} underline that the 2P rule should not be taken literally, because it is possible to place a CL after a compound phrase in standard Serbian. Moreover, Serbian linguists provide examples with CLs which follow initial phrases, which contain more than two content words, such as in examples (\ref{(6.24)}) and (\ref{(6.25)}) provided below. 

\begin{exe}\ex\label{(6.24)}
\gll Manji deo takmičara \textbf{je} iz Beograda.\\
 smaller part contestants be\textsc{.3sg} from Beograd \\
\glt ‘The smaller part of the contestants is from Beograd.’ \\
\hfill (Sr; \citealt[161]{IKPB11})
\ex\label{(6.25)}
\gll Moj prijatelj s trećeg sprata \textbf{je} došao.\\
 my friend from third floor be\textsc{.3sg} come\textsc{.ptcp.sg.m} \\
\glt ‘My friend from the third floor came.’ 
\hfill (Sr; \citealt[450]{PiperKlajn14})
\end{exe}

\subsubsection{Heavy initial phrases}

The Serbian linguist \citet[308]{Pesikan58} admits that it is difficult to give concrete examples of longer initial phrases which cannot be followed by CLs in Serbian; in his opinion CL placement also depends on word length.\footnote{\textcolor{black}{For theoretical discussion and the definition of the term \textcolor{black}{heavy initial phrase} in this work see Section \ref{Barriers and delayed placement of clitics (DP)}.}} In a later work, \citet{RadanovicKocic88} explains this in more detail. She claims that CLs usually do not directly follow an initial phrase longer than two words, but if the first long phrase is the subject, CLs can lean on it optionally (\citealt[cf.][108ff]{RadanovicKocic88}, \citealt[435]{RadanovicKocic96}).\footnote{In the examples by \citet{IKPB11} and \citet{PiperKlajn14} provided in (\ref{(6.24)}) and (\ref{(6.25)}) the subject is the first long phrase and therefore it can host CLs.} She supports her claims with examples (\ref{(6.26a)}) and its permutation (\ref{(6.26b)}): 

\begin{exe}\ex\begin{xlist}\ex\label{(6.26a)}
\begin{xlist}
\gll Kolutovi plavičastog dima penjali \textbf{su} {\textbf{se} [\dots].} \\
 rings blueish smoke climb\textsc{.ptcp.pl.m} be\textsc{.3pl} \textsc{refl} \\
\ex\label{(6.26b)}
\gll Kolutovi plavičastog dima \textbf{su} \textbf{se} {penjali [\dots].} \\
rings blueish smoke be\textsc{.3pl} \textsc{refl} climb\textsc{.ptcp.pl.m}\\
\end{xlist}
\glt ‘Rings of blueish smoke were climbing [\dots].’ \\
\strut\hfill (BCS; \citealt[110]{RadanovicKocic88})


\ex[]{\label{(6.27a)}
\gll Svoje probleme i dileme lingvistika \textbf{će} {rešavati [\dots]}. \\
 own problems and dilemmas linguistics \textsc{fut.3sg} solve\textsc{.inf} \\ 
\glt ‘Linguistics will be solving its problems and dilemmas [\dots].’}
\ex[*]{\label{(6.27b)}
\gll Svoje probleme i dileme \textbf{će} lingvistika {rešavati [\dots]}. \\
own problems and dilemmas \textsc{fut.3sg} linguistics solve\textsc{.inf} \\ }
\hfill (BCS; \citealt[119f]{RadanovicKocic88})
\end{xlist}
\end{exe}

\noindent \citet[][435]{RadanovicKocic96} is the only author who claims that long initial objects cause DP of CLs – compare examples (\ref{(6.27a)}) and (\ref{(6.27b)}) provided above. Unlike her, \citet[658]{MrazovicVukadinovic09} and \citet[450]{PiperKlajn14} do not explicitly distinguish between long initial subjects and other kinds of phrases. According to them, in Serbian the 2P reserved for CLs is after the first stressed unit, i.e. if there is a compound phrase at the beginning of a sentence, the CL does not follow the first word, but the first phrase (\citealt[658]{MrazovicVukadinovic09}, \citealt[450]{PiperKlajn14}). Since Serbian authors differ in their opinion, we inevitably expect variation within the Serbian language. 

\subsubsection{Delayed placement}

In contrast to \citet{MrazovicVukadinovic09} and \citet{PiperKlajn14}, \citet[48]{Alexander09} claims that CLs do not have to follow either the first word or the first phrase: their placing can be delayed. If the language user does not want to split the first phrase or splitting is not appropriate to that particular register, CL placement can be delayed \citep[cf.][48]{Alexander09}. See the Croatian example provided below in (\ref{(6.28)}).

\begin{exe}\ex\label{(6.28)}
\gll Vaša naklapanja zaista \textbf{mi} dosađuju. \\
 your ramblings really me\textsc{.dat} bore\textsc{.3prs} \\
\glt ‘Your ramblings really bore me.’ 
\hfill (Cr; \citealt[496]{Katicic86})
\end{exe}

\noindent Moreover, if the second phrase is a compound, it can be split as well, which results in DP of the CL combined with phrase splitting. See the examples from Croatian provided below in (\ref{(6.29)}) and (\ref{(6.30)}).  

\begin{exe}\ex\label{(6.29)}
\gll Psunj, Papuk i Krndija tvrdo \textbf{su} eruptivno gorje.\\
 Psunj Papuk and Krndija hard be\textsc{.3pl} eruptive mountains\\
\glt ‘Psunj, Papuk and Krndija are hard volcanic mountains.’  \\
\hfill  (Cr; \citealt[597]{Baric97})

\ex\label{(6.30)}
\gll Od toga doba mnogo \textbf{je} vode proteklo. \\
 from that time much be\textsc{.3sg} water flow\textsc{.ptcp.sg.n}\\
\glt ‘Much water has flowed (much time has elapsed) from that time.’  \\
\hfill  (Cr; \citealt[496]{Katicic86})
\end{exe}

\noindent Similarly, Bosnian authors \citep[e.g.][471]{JHP00} claim that if CLs do not follow the first stressed word in a sentence, they will directly follow the predicate, i.e. placement will be delayed. But, as we can see in the examples provided above, the second phrase is not always a predicate. 

During past decades, Croatian authors (e.g. \citealt[150--152]{Weber59}, \citealt[150]{Jonke53}, \citealt[182]{FHM06}) considered the DP of CLs to be fully acceptable when one did not want to separate syntactically or semantically tightly bounded words, i.e. if one did not want to split compound phrases. Conversely, the Serbian linguist \citet[308]{Pesikan58} claimed that it is better to place CLs after a two-word phrase than to use DP of CLs. \citet[364]{Popovic04} sees the Croatian tendency to delay CL placement as a major factor in the growing divergence between Serbian and Croatian writers. 


We can recapitulate this subsection with the following observations: Serbian grammarians differ from Bosnian and Croatian grammarians in their comprehension of the 2P; consequently we can expect diatopic variation between the BCS standard varieties. Furthermore, while Bosnian and Croatian authors recommend delaying the placement of CLs as a better alternative to placing CLs after compound phrases, as we saw Serbian authors propose quite the opposite. Therefore, we can assume that there is diatopic variation between BCS standard varieties in respect of CL placement. These tendencies are corroborated by \citeauthor{Reinkowski01}'s (\citeyear[]{Reinkowski01}) diachronic corpus study, which analysed newspaper articles from the years 1905, 1935, 1965 and 1995. She found that in both Serbian and Croatian journalistic registers the DP is dominant \citep[cf.][183, 202]{Reinkowski01}.\footnote{In following years, in her own independent study, \citet[14]{Alexander08} proved and verified Reinkowski’s results.} Furthermore, she established that phrase splitting had also been present in the Croatian journalistic register for over the previous 100 years, although it reached its lowest point in 1965  \citep[cf.][191--195]{Reinkowski01}. In contrast, phrase splitting has been slowly disappearing from the Serbian journalistic register over the analysed period of 90 years \citep[191--195]{Reinkowski01}. 

\subsection{The limits of phrase splitting in BCS standard varieties}
\label{The limits of phrase splitting in BCS standard varieties}
As phrase splitting has attracted the attention of both normativists and formal linguists, we would like to give an account of the data discussed in the literature. \citet[81]{Reinkowski01} claims that \citet[289]{MeilletVaillant24} were the first to notice and mention the fine diatopic variation between Croatian and Serbian: in their own words, the language of Belgrade prefers no splitting of the subject phrases. Likewise \citet[11]{Alexander08} emphasises that Croatian and Serbian differ as to phrase splitting and that this difference was already noticeable well before the break up of Serbo-Croatian. Even those Croatian linguists who were not determined apologists of phrase splitting and 2W CL placement like \citet[145]{Brabec65} admit that phrase splitting has always been common in Croatian texts from all periods and all regions. The Serbian linguist \citet[309]{Pesikan58} finds the Croatian tendency to insert CLs after the first stressed word and to split phrases is an exaggeration. \citet[111]{RadanovicKocic88} claims that there is an important difference between initial two-word subject phrases and non-subject phrases: in her dialect only subject phrases can be split, whereas others cannot.\footnote{She does not explain what “my dialect” actually means. It could mean the Eastern variant of Serbo-Croatian with all its varieties, only the standard Serbian variety or the \textit{Istočnohercegovački} dialect, since she was born where this dialect is spoken. The problem is that her thesis is called \textit{The grammar of Serbo-Croatian clitics}, so on the one hand she examines Serbo-Croatian as one abstract system, while on the other hand she admits that variation exists, but ultimately she uses her own language feeling and her own dialect as a baseline of comparison when she claims that something is ungrammatical.} Furthermore, she believes that the placement of CLs after the first word of a two-word initial subject or after the whole phrase depends on the structure of the subject phrase \citep[cf.][112]{RadanovicKocic88}. However, \citet[52--55]{Alexander09} notes that not only is phrase splitting more frequent in Croatian than in Bosnian and Serbian, it is also found in more contexts, i.e. in more types of phrase structures in Croatian. \citet[182]{FHM06} give precise examples of phrases which can be split by CLs in Croatian: CLs can split an adjective, numeral, pronoun or noun from a noun, and a forename from a family name. Regarding Bosnian, \citet[196f]{Cedic01} admits that phrase splitting does occur, but he sees it rather as a Croatian import than a Bosnian feature. 

As will become evident in this subsection, \citet[4]{FranksPeti06} correctly noticed that “there is a high degree of variation in judgments about the acceptability of different kinds of splitting, both across speakers and across languages”. Therefore, in the following we compare cases in which phrase splitting is possible in all three standard varieties. Before further elaborating on this phenomenon we must point out that the Serbian linguist \citet[306]{Popovic04} is the only one to notice that CLs can be inserted only after the first word in a compound phrase.\footnote{As our data from spoken Bosnian indicate, this is not completely true. For examples of phrase splitting in which CLs are not inserted after the first stressed word in a phrase see Section \ref{Split phrases}.}  In both Croatian and Serbian, CLs can split an adjective and a noun: compare examples (\ref{(6.31)})--(\ref{(6.33)}) (\citealt[cf.][246]{TezakBabic96}, \citealt[450]{PiperKlajn14}). 

\begin{exe}\ex\label{(6.31)}
\gll Motovunske \textbf{su} ulice vrvjele pukom. \\
 Motovunian be\textsc{.3pl} streets buzz\textsc{.ptcp.pl.f} people\\
\glt ‘Motovunian streets were buzzing with people.’ \\
\hfill  (Cr; \citealt[246]{TezakBabic96})

\ex\label{(6.32)}
\gll Dobra \textbf{se} roba brzo proda.\\
 good \textsc{refl} wares quickly sell\textsc{.3prs} \\
\glt ‘Good wares sell quickly.’ 
\hfill  (Sr; \citealt[450]{PiperKlajn14})

\ex\label{(6.33)}
\gll Anina \textbf{mi} \textbf{ga} \textbf{je} sestra poklonila. \\
 Ana’s me\textsc{.dat} it\textsc{.acc} be\textsc{.3sg} sister gift\textsc{.ptcp.sg.f}\\
\glt ‘Ana’s sister gave it to me as a present.’ 
\hfill (BCS; \citealt[419]{Progovac96})
\end{exe}

\noindent \citet[112]{RadanovicKocic88} claims that two-word initial subject phrases with an attribute-noun structure represent the only real case of microvariation, because CLs can follow the first word or first phrase in both Croatian and Serbian variants. Although as a rule the majority of formal linguists tend to discount both syntactic microvariation and sociolinguistic variation, \citet[135]{RadanovicKocic88} assumes that in the case of the adjective attribute and noun, phrase splitting is more frequent in Croatian. She argues that in her dialect CLs can be placed after the adjective only if the adjective carries the phrasal stress, while in other dialects this condition is not necessary \citep[cf.][134]{RadanovicKocic88}. However, in a later paper \citep[435]{RadanovicKocic96} she states that such examples are marginal and that more complex CL clusters are not allowed in that position.\footnote{As we show later in this section, \citet{Progovac96} also comments that sentences become worse if CL clusters, and not single CLs, split phrases. However, dialectal and spoken data speak against those claims made by theoretical syntacticians – see Sections \ref{Phrase splitting} and \ref{Inventory of clitics participating in phrase splitting}.} She provides the example presented in (\ref{(6.34a)}) and its permutation (\ref{(6.34b)}):\footnote{This dative CL \textit{ti} ‘you’ in this example is called the ethical dative (not an argument) and is not easily translatable into English. It is used in spoken language, in directed speech and signals closeness  \citep[220]{SilicPranjkovic07}.}

\begin{exe}\ex\begin{xlist}\ex[?]{\label{(6.34a)}
\gll Moj \textbf{ga} \textbf{se} brat sjeća. \\
 my him\textsc{.gen} \textsc{refl} brother remember\textsc{.3prs} \\ 
\glt ‘My brother remembers him.’}
\ex[*]{\label{(6.34b)}
\gll Moj \textbf{ti} \textbf{ga} \textbf{se} brat sjeća.\\
 my you\textsc{.dat} him\textsc{.gen} \textsc{refl} brother remember\textsc{.3prs} \\
 \glt Intended: ‘My brother remembers him, you know.’}
\strut\hfill (BCS; \citealt[435]{RadanovicKocic96})
\end{xlist}
\end{exe}


%\citet[306f]{Pesikan58} also admits that adjective attributes can be affected by splitting in Serbian, but he dislikes such examples. \citet[70]{FranksProgovac94}, like Pešikan, generally disapprove of splitting adjectives from nouns, but they claim that such structures reflect the general possibility of splitting adjectives from the nouns they modify.

\noindent \citet[114]{RadanovicKocic88} claims that unlike subjects with an adjective-noun structure, subjects with other structures are rarely split by CLs. Furthermore, from the perspective of her dialect she evaluates \citeauthor{Katicic86}'s (\citeyear{Katicic86}) sentences with a non-subject initial split phrase presented in (\ref{(6.35)}) as grammatically questionable. 

\begin{exe}\ex[?]{\label{(6.35)}
\gll Takvoj \textbf{se} definiciji može staviti prigovor. \\
 such \textsc{refl} definition can\textsc{.3prs} put\textsc{.inf} complaint \\
 \glt ‘Such a definition is subject to complaint.’ \\}
\strut\hfill  (Cr; \citealt[111]{RadanovicKocic88})
\end{exe}

\noindent From the discussion presented above it seems that insertion of a CL between an adjective and a noun is less restricted in standard Croatian than in standard Serbian. 
As we can see from the literature, other kinds of attributes can be split from their head noun by a CL as well. We find examples of a CL splitting a demonstrative pronoun and a noun in both standard Croatian (\ref{(6.36)}) and standard Serbian (\ref{(6.37)}) (\citealt[cf.][597]{Baric97}, \citealt[450]{PiperKlajn14}).\footnote{We found this kind of split phrase in dialectological data. For more information see Section \ref{Phrase splitting}.}


\begin{exe}\ex\label{(6.36)}
\gll Taj \textbf{će} \textbf{se} režim prije ili kasnije naći pod ruševinama svoje nasilne politike. \\
 that \textsc{fut.3sg} \textsc{refl} regime sooner or later find\textsc{.inf} under ruins own violent politics \\
\glt `This regime will sooner or later find itself under the ruins of its own violent politics.' 
\hfill (Cr; \citealt[597]{Baric97})

\ex\label{(6.37)}
\gll Taj \textbf{nam} predlog ne odgovara.\\
 that us.\textsc{dat} suggestion \textsc{neg} answer\textsc{.3prs} \\
\glt ‘That suggestion does not suit us.’ 
\hfill  (Sr; \citealt[450]{PiperKlajn14})
\end{exe}

\noindent In the previous century the Serbian linguist \citet[306f]{Pesikan58} claimed that in Serbian CLs cannot separate a noun from its pronoun attribute. As we saw above, contrary to him, currently the Serbian authors \citet[450]{PiperKlajn14} allow such a possibility in standard Serbian.

Furthermore, CLs can split adverbial phrases in both Croatian and Serbian standard language – see the example provided below in (\ref{(6.38)}).

\begin{exe}\ex\label{(6.38)}
\gll Vrlo \textbf{su} hrabro to uradili.\\
 very be\textsc{.3pl} bravely that do\textsc{.ptcp.pl.m}\\
\glt ‘They did that very bravely.’ 
\hfill  (Sr; \citealt[450]{PiperKlajn14})
\end{exe}

\largerpage[2]
\noindent \citet[450]{PiperKlajn14} state that CLs can directly follow the modifiers \textit{samo} and \textit{jedino} ‘only’. It is important to emphasise that in example (\ref{(6.39)}) provided by \citet[450]{PiperKlajn14}, CLs are inserted into a prepositional phrase just like in the Croatian example (\ref{(6.40)}). Unlike \citet{PiperKlajn14}, Radanović-Kocić (\citeyear[114]{RadanovicKocic88}, \citeyear[436]{RadanovicKocic96}) believes that there are very few cases in which a CL can split a head noun and its modifier. She adds that examples in which CLs are placed between a noun and its modifier in a PP are ungrammatical in her dialect \citep[cf.][436]{RadanovicKocic96}.\footnote{As we already emphasised, she does not really explain what “my dialect” means. Since it could mean the \textit{Istočnohercegovački} dialect, we would like to point out that data from dialectological literature show that these kinds of splitting are possible in the \textit{Istočnohercegovački} dialect; for more information see Section \ref{Phrase splitting}.}

\begin{exe}\ex\label{(6.39)}
\gll Samo / Jedino \textbf{se} iz Niša nije niko javio.\\
 only {} only \textsc{refl} from Niš \textsc{neg.}be\textsc{.3sg} no.one reply\textsc{.ptcp.sg.m}\\
\glt ‘Only from Niš no one replied.’ 
\hfill  (Sr; \citealt[450]{PiperKlajn14})

\ex\label{(6.40)}
\gll Neki \textbf{su} od njih sada čučali na pragu, drugi \textbf{se} razvalili u hladovini. \\
 some be\textsc{.3pl} from them now crouch\textsc{.ptcp.pl.m} on doorstep others \textsc{refl} lay.down\textsc{.ptcp.pl.m} in shade \\
\glt ‘Some of them were now crouching on the doorstep, others were lying in the shade.’
\hfill  (Cr; \citealt[597]{Baric97})
\end{exe}

% \noindent
\citet[560]{Ridjanovic12} claims that phrase splitting in Bosnian has its limits: namely, a NP which consists of a noun and an adverbial, nominal complement or modifier cannot be split by a CL. Therefore in such cases the CL will follow that phrase  \citep[560]{Ridjanovic12} – see example (\ref{(6.41)}) provided below.

\begin{exe}\ex\label{(6.41)}
\gll Avion u letu \textbf{je} slikalo nekoliko turista.\\
 airplane in flight be\textsc{.3sg} photograph\textsc{.ptcp.sg.n} several tourists\\
\glt ‘The airplane in flight was photographed by several tourists.’ \\
\hfill  (Bs; \citealt[560]{Ridjanovic12})
\end{exe}

\noindent Only for standard Croatian did we find information that a CL can split a noun from its postmodifying genitive attribute – see example (\ref{(6.42)}) provided below. In contrast, according to \citet[306f]{Pesikan58} in standard Serbian CLs cannot separate a noun and its postmodifying genitive attribute – see (\ref{(6.43)}).

\begin{exe}\ex[]{\label{(6.42)}
\gll Kontrast \textbf{je} ovih fakata očigledan.\\
 contrast be\textsc{.3sg} these facts obvious\\
\glt ‘The contrast between these facts is obvious.’ 
\hfill  (Cr; \citealt[597]{Baric97})}

\ex[*]{\label{(6.43)}
\gll Čovek \textbf{će} visoke inteligencije uvek uspeti.\\
 man \textsc{fut.3sg} high intelligence always succeed\textsc{.inf}\\ 
\glt Intended: ‘A man of high intelligence will always succeed.’ \\}
\strut\hfill  (Sr; \citealt[306f]{Pesikan58})
\end{exe}

\noindent The problem of phrase splitting in the context of a noun and its modifier in case has been thoroughly discussed in the theoretical literature. \citet[70]{FranksProgovac94} and \citet[522]{MiseskaTomic96} have stated that examples with CLs inserted between a noun and its modifier in the genitive are incorrect – compare the examples provided below in (\ref{(6.45a)}), (\ref{(6.45b)}) and (\ref{(6.46a)}). Two years later, however, \citet[419]{Progovac96} admitted that the example presented in (\ref{(6.45b)}) is possible, but the phenomenon is extremely marginal.\footnote{The very same example was deemed unacceptable in \citet{FranksProgovac94}.}

\begin{exe}\ex\begin{xlist}
\ex[]{\label{(6.45a)}
\gll Prijatelji moje sestre \textbf{su} upravo stigli.\\
 friends my sister be\textsc{.3pl} just arrive\textsc{.ptcp.pl.m}\\ 
\glt ‘My sister’s friends just arrived.’}
\ex[*]{\label{(6.45b)}
\gll Prijatelji \textbf{su} moje sestre upravo stigli.\\
 friends be\textsc{.3pl} my sister just arrive\textsc{.ptcp.pl.m}\\ }
\hfill (BCS; \citealt[522]{MiseskaTomic96})
\end{xlist}
\end{exe}

\noindent \citet[418]{Progovac96} rejects examples in which more than one CL splits a noun and its modifier in a case as ungrammatical. Furthermore, she claims that the examples become worse when more than one CL is inserted \citep[cf.][419]{Progovac96} – compare her examples (\ref{(6.46a)}) and (\ref{(6.46b)}) below. However, she observes that conversely, an insertion of more CLs in a possessive phrase makes no difference, like in her example (\ref{(6.33)}) above. 

\begin{exe}\ex[*]{\label{(6.46a)}
\gll Roditelji \textbf{su} \textbf{se} uspešnih studenata razišli.\\
 parents be\textsc{.3pl} \textsc{refl} successful students separate\textsc{.ptcp.pl.m}\\ 
\glt Intended: ‘Parents of successful students separated.’\\}
\strut\hfill  (BCS; \citealt[418]{Progovac96})

\ex[?*]{\label{(6.46b)}
\gll Prijatelji \textbf{su} \textbf{mi} \textbf{ga} moje sestre poklonili.\\
 friends be\textsc{.3pl} me\textsc{.dat} it\textsc{.acc} my sister gift\textsc{.ptcp.pl.m}\\
\glt Intended: ‘My sister’s friends gifted it to me.’ 
\hfill  (BCS; \citealt[419]{Progovac96})}
\end{exe}

\noindent Similarly but going into less detail,
\citeauthor{RadanovicKocic88} (\citeyear[114]{RadanovicKocic88}, \citeyear[436]{RadanovicKocic96}) believes that there are very few cases in which a CL splits a head noun and its modifier in the genitive. She even emphasises that in most cases, in her dialect CLs have to follow such phrases and adds that sentences in which CLs are placed between a noun and its modifier in the genitive are ungrammatical in her dialect \citep[cf.][436]{RadanovicKocic96}. However, \citet[54]{Alexander09} uses examples from \textit{Hrvatska} \textit{gramatika} \citep{Baric97} to argue that such restrictions do not apply to standard Croatian, i.e. CLs can be inserted between a head noun and its modifier in the genitive, as we already demonstrated in example (\ref{(6.42)}) provided above.

Only for standard Croatian do we find information that CLs can split an apposition from a noun – see the example provided below. 

\begin{exe}\ex\label{(6.47)}
\gll Gospoja \textbf{ih} \textbf{se} Olivija naprosto plašila.\\
 madam them\textsc{.gen} \textsc{refl} Olivia simply afraid\textsc{.ptcp.sg.f}\\
\glt ‘Madam Olivia was simply afraid of them.’ 
\hfill (Cr; \citealt[597]{Baric97})
\end{exe}

\noindent A CL can separate parts of indefinite pronouns and adverbs (\citealt[cf.][496]{Katicic86}, \citealt[207]{Baric97}, \citealt[295f]{Popovic04}). However, it seems that in both Croatian and Serbian the version without splitting presented in (\ref{(6.48a)}) is more frequent than the version with splitting presented in (\ref{(6.48b)}) (\citealt[cf.][496]{Katicic86}, \citealt[207]{Baric97}, \citealt[295f, 323]{Popovic04}).

\begin{exe}\ex\begin{xlist}
\ex\label{(6.48a)}
\gll Tko god \textbf{je} vidio njegove slike, bio \textbf{je} zadivljen. \\
 who ever be\textsc{.3sg} see\textsc{.ptcp.sg.m} his paintings be\textsc{.ptcp.sg.m} be\textsc{.3sg} impressed\\
\glt ‘Whoever saw his paintings was impressed.’
\ex\label{(6.48b)}
\gll Tko \textbf{je} god vidio njegove {slike [\dots].} \\
who be\textsc{.3sg} ever see\textsc{.ptcp.sg.m} his paintings\\
\glt ‘Whoever saw his paintings [\dots].’ 
\hfill  (Cr; \citealt[207]{Baric97})
\end{xlist}
\end{exe}

\noindent In both Bosnian and Croatian scholarly literature and textbooks (e.g. \citealt[64]{Babic63}, \citealt[496]{Katicic86}, \citealt[598]{Baric97}, \citealt[471]{JHP00}, \citealt[182]{FHM06}, \citealt[195]{FrancicPetrovic13}) it is claimed that even forenames can be separated from family names by CLs – see example (\ref{(6.49)}) below. Furthermore, it is claimed that such positioning, which strictly follows the rule of CL placing, is a hallmark of stylistically polished expression \citep[cf.][496]{Katicic86}.


\begin{exe}\ex\label{(6.49)}
\gll Luka \textbf{bi} Šušmek polazio u šetnju da namigne kojoj curi.\\
 Luka \textsc{cond.3sg} Šušmek depart\textsc{.ptcp.sg.m} in walk that wink\textsc{.3prs} which girl\\
 \glt ‘Luka Šušmek would go on walks to wink at some girl.’ \\
\hfill  (Cr; \citealt[598]{Baric97})
\end{exe}

\noindent Unlike Bosnian and Croatian scholars, Serbian linguists \citep[e.g.][319]{Popovic04} do not allow splitting of forenames and family names in contemporary Serbian, although they admit that such occurrences were possible in earlier periods of Serbian.\footnote{Not only was this kind of phrase splitting possible in earlier periods of Serbian, but it is also now present in dialects spoken on the Serbian territory, see Section \ref{Phrase splitting}.} \citet[][306]{Pesikan58} provides the following example presented in (\ref{(6.50)}) which was previously acceptable in Serbian.

\begin{exe}\ex\label{(6.50)}
\gll Matija \textbf{je} Benadić čovek sasvim star.\\
 Matija be\textsc{.3sg} Benadić man very old\\
\glt ‘Matija Benadić is a very old man.’
\hfill  (Sr; \citealt[306]{Pesikan58})
\end{exe}

\noindent \citet[37]{CavarWilder94} believe that cases in which a verbal CL splits a forename and a family name are marginal for most speakers. \textcolor{black}{\citet[3]{Boskovic01} claims that splitting a first and last name by a CL is an eccentricity generally possible in South Slavic. Following \citet[116]{Franks97}, \citet[16f, 29]{Boskovic01} argues that a CL can split the first name and the last name only when both names are inflected for structural case. According to \citet[116]{Franks97}, splitting of proper names can only occur when both first and last name are treated as separate heads.}\footnote{\textcolor{black}{We completely agree that \textit{Lav} in example (\ref{(28.05.21a)}) is not a head, which is probably one of the elements contributing to the unacceptability of the example in question.}
 
\begin{exe}\ex[*]{\label{(28.05.21a)}
\gll Lav sam Tolstoja čitala. \\
Leo.\textsc{nom} be.\textsc{1sg} Tolstoj.\textsc{acc} read.\textsc{ptcp.sg.f}\\ 
\glt Intended: ‘I read Leo Tolstoj’.
\hfill (BCS, \citealt[17]{Boskovic01})}
\end{exe}


\noindent Note, however, that \citet[116]{Franks97} and \citet[16]{Boskovic01} admit that declining only one part is, in fact, marginally possible in BCS and that this marginality is independent of splitting. \textcolor{black}{Moreover, we would like to point out that the “case test” cannot be applied in the straightforward manner assumed by \citet{Boskovic01} and \citet{Franks97}, since examples of phrase splitting like (\ref{(28.05.21b)}) in which seemingly only one part of the proper name phrase is inflected can be attested in corpora.}
 
\begin{exe}\ex\label{(28.05.21b)}
\gll S Jadrankom je Kosor godinama radio i {surađivao [\dots].}\\
 with Jadranka.\textsc{ins} be.\textsc{3sg} Kosor.\textsc{nom} years.\textsc{ins} work.\textsc{ptcp.sg.m} and cooperate.\textsc{ptcp.sg.m}\\
\glt ‘He worked and cooperated with Jadranka Kosor for years.’
\hfill  [hrWaC v2.2]
\end{exe}

\noindent \textcolor{black}{ Moreover, we believe that in future, theoretical assumptions on the range and limits of proper name splitting should be verified against robust empirical data. For instance, \citet[116]{Franks97} considers that examples with splitting of proper names in which both parts are in the nominative case are not completely acceptable, although as we show in this section, such examples appear in the normative and descriptive BCS literature: see examples in (\ref{(6.49)}) and (\ref{(6.50)}).}} A structurally similar case is compound geographical names and terms, which according to \citet[306f]{Pesikan58} and \citet[116]{RadanovicKocic88} cannot be split either.  

The Serbian linguists \citet[307]{Pesikan58}, \citet[116]{RadanovicKocic88}, and \citet[523]{MiseskaTomic96} observe that conjoined NPs in general are never split by CLs: compare the examples presented in (\ref{(6.51)}) and (\ref{(6.52b)}) with the example in (\ref{(6.52a)}).
 
\begin{exe}\ex[*]{\label{(6.51)}
\gll Petar \textbf{će} i Marko doći. \\
 Petar \textsc{fut.3sg} and Marko come\textsc{.inf} \\ 
\glt Intended: ‘Petar and Marko will come.’\hfill  (Sr; \citealt[307]{Pesikan58})\hbox{}}

\ex\begin{xlist}
\ex[]{{\label{(6.52a)}
\gll O Veri i Jani \textbf{si} \textbf{mi} govorio.\\
 about Vera and Jana be\textsc{.2sg} me\textsc{.dat} speak\textsc{.ptcp.sg.m}\\ }
\glt ‘You told me about Vera and Jana.’ }
\ex[*]{\label{(6.52b)}
\gll O Veri \textbf{si} \textbf{mi} i Jani govorio.\\
about Vera be\textsc{.2sg} me and Jana speak\textsc{.ptcp.sg.m}\\ }
\hfill  (BCS; \citealt[523]{MiseskaTomic96})
\end{xlist}
\end{exe}

\noindent However, it seems that not all facts are this clear cut, even in the case of Serbian. \citet[418f]{Progovac96}, for instance, claims that examples of conjoined NPs with one inserted CL (as in (\ref{(6.53a)})) are marginal, and those with an inserted CL cluster (as in (\ref{(6.53b)})) are outright ungrammatical in her Serbian. 

\begin{exe}\ex[??]{\label{(6.53a)}
\gll Sestra \textbf{će} i njen muž doći u utorak.\\
 sister \textsc{fut.3sg} and her husband come\textsc{.inf} in Tuesday \\
\glt Intended: ‘My sister and her husband will come on Tuesday.’ \\
\hfill (BCS; \citealt[418]{Progovac96})}

\ex[*]{\label{(6.53b)}
\gll Sestra \textbf{će} \textbf{mi} \textbf{ga} i njen muž pokloniti.\\
 sister \textsc{fut.3sg} me\textsc{.dat} it\textsc{.acc} and her husband gift\textsc{.inf}\\ 
 \glt Intended: ‘My brother remembers him.’\hfill(BCS; \citealt[418f]{Progovac96})\hbox{}}
\end{exe}

\noindent In her opinion, the examples become worse when more than one CL is inserted \citep[cf.][419]{Progovac96}. However, in contrast to all the theoretical linguists who reject phrase splitting in the case of conjoined NPs (e.g. \citealt[116f]{RadanovicKocic88}, \citealt[66ff]{Schutze94}, \citealt[418f]{Progovac96};),\citealt[5, 11]{FranksPeti06}) claim that splitting of conjoined NPs is perfectly fine for many native speakers of Croatian. \citet[320]{Popovic04} claims that inserting CLs into conjoined phrases is very rare in contemporary Serbian, but he does not call it ungrammatical.\footnote{The mentioned structure, controversial from the theoretical point of view, is according to dialectological data widespread in the \textit{Istočnohercegovački} dialect and has been attested on Serbian territory, for more information see Section \ref{Phrase splitting}.}

According to \citet[458]{Ridjanovic12} the most frequent cases of phrase splitting in Bosnian are those with interrogative pronouns and their adjective postmodifiers: see the example in provided in (\ref{(6.54)}).

\begin{exe}\ex\label{(6.54)}
\gll Čemu \textbf{se} dobrom možemo nadati?\\
 what \textsc{refl} good can\textsc{.1prs} hope\textsc{.inf} \\
\glt ‘What good can we hope for?’ 
\hfill (Bs; \citealt[458]{Ridjanovic12})
\end{exe}

\noindent If there is a relative or question pronoun (question word) in a Bosnian sentence, CLs can be placed directly after it, but this is not obligatory \citep[563]{Ridjanovic12}. This kind of phrase splitting is not considered controversial in Serbian literature (e.g. \citealt[307]{Pesikan58}, \citealt[294]{Popovic04}), moreover it is claimed to be quite common. Compare examples (\ref{(6.55a)}), (\ref{(6.55b)}) and (\ref{(6.56)}) provided below.

\begin{exe}\ex\begin{xlist}\ex\label{(6.55a)}
\gll Koji \textbf{ste} grad posjetili? \\
 which be\textsc{.2pl} city visit\textsc{.ptcp.pl.m} \\
\ex\label{(6.55b)}
\gll Koji grad \textbf{ste} posjetili? \\
 which be\textsc{.2pl} city visit\textsc{.ptcp.pl.m} \\
\end{xlist}
\glt ‘Which city did you visit?’ 
\hfill  (Bs; \citealt[563]{Ridjanovic12})

\ex\label{(6.56)}
\gll Koja \textbf{ga} \textbf{je} žena pozdravila? \\
 which him\textsc{.acc} be\textsc{.3sg} woman greet\textsc{.ptcp.sg.f} \\
\glt ‘Which woman greeted him?’ 
\hfill (Sr; \citealt[307]{Pesikan58})
\end{exe}

\noindent As we have shown above, not all phrase splitting possibilities are mentioned in grammar books of all standard varieties. We can infer that there is some microvation with respect to phrase splitting. Moreover, \citet[][105]{PiperIvic05} also claim that phrase splitting in Serbian is possible, but they clearly favour the examples presented in (\ref{(6.57a)}) and (\ref{(6.57b)}) and judge them to be far better than the phrase splitting version presented in (\ref{(6.57c)}). 


\begin{exe}\ex\begin{xlist}\ex\label{(6.57a)}
\gll Novi igrači \textbf{su} bili uspešniji.\\
 new players be\textsc{.3pl} be\textsc{.ptcp.pl.m} more.successful\\
\ex\label{(6.57b)}
\gll Novi igrači bili \textbf{su} uspešniji.\\
 new players be\textsc{.ptcp.pl.m} be\textsc{.3pl} more.successful\\
\ex\label{(6.57c)}
\gll Novi \textbf{su} igrači bili uspešniji.\\
 new be\textsc{.3pl} players be\textsc{.ptcp.pl.m} more.successful\\
\end{xlist}
\glt‘New players were more successful.’ 
\hfill (Sr; \citealt[105]{PiperIvic05})
\end{exe}

\noindent In contrast, \citet[371]{StanojcicPopovic02} also mention the possibility of inserting CLs into a syntagm, but they do not specify in what cases it is possible and they do not state whether splitting or not splitting is better in Serbian. 

Before we conclude this section, we would like to point out one more interesting fact. In most cases phrases are split by verbal CLs (compare examples presented in this section). This is also noted by \citet{PetiStantic02} for Croatian. She observes that verbal CLs very often split phrases in standard Croatian, which is not the case for pronominal ones \citep[cf.][174f]{PetiStantic02}.\footnote{In this respect dialects and spoken data do not differ much from standard BCS varieties. For more information, see Sections \ref{Phrase splitting}. However, \citet{PetiStantic02} uses absolute \textcolor{black}{and not relative} values. For more discussion see \ref{Phrase splitting:9}.}

We can conclude this section on splitting by referring to \citet[50]{Alexander09}, who notices that there is still great need to investigate to what extent 2P after long phrases, and different types of phrase splitting are acceptable. 

\section{Summary}
\subsection{Clitic inventory}

This part can be summed up as follows. Undoubtedly, from the descriptions of Serbian and Croatian linguists it can be seen that there is one important difference in the CL inventory of BCS standard varieties, since only Croatian grammarians accept the standardness of the reflexive CL \textit{si}. The only scholar who clearly spells out this difference is the Bosnian author \citet[440]{Ridjanovic12}. 

Moreover, Croatian and Serbian authors differ in their recommendations for the usage of the third person singular feminine accusative CL \textit{ju} and \textit{je}. According to some Croatian authors, \textit{ju} can be treated as a separate unit of the inventory (and not only as the result of suppletion, for which see below).

\subsection{Clitic cluster and morphonological processes within it}

We would like to reiterate the following facts from this part. First of all, the linearisation of pronominal CLs presented in Bosnian, Croatian, and Serbian grammar books differs from the one shown in \citet[29]{FranksKing00} since in the former the authors claim that genitive precedes accusative. Further, there is some disagreement among Serbian authors regarding the realization of the hypothetically possible combination of genitive and accusative pronominal CLs within the CL cluster. While \citet[451]{PiperKlajn14} provide an example of this, \citet[659]{MrazovicVukadinovic09} strongly refuse such a possibility. It might be relevant to point out that the CLs in question are homophones.

Regarding morphonological processes, it is not very clear if haplology of unlikes is obligatory or not in standard Croatian. The assertions that the auxiliary CL \textit{je} can be deleted and that it is deleted after the reflexive CL \textit{se} are found in \citet[246]{TezakBabic96} and \citet[596]{Baric97}. Unlike them, \citet[497]{Katicic86} does not consider haplology to be the rule. In opposition to Croatian authors, \citet[452]{PiperKlajn14} are explicit in considering the sequence \textit{se je} to be incorrect in standard Serbian. However, \citet{Ridjanovic12} observes that haplology does not always occur in standard Bosnian: the exception to haplology is cases in which the verbal CL \textit{je} has the function of a copula.

The third person feminine accusative pronominal CL \textit{ju} can be used in standard Serbian only in the case of suppletion: direct contact with the verbal CL \textit{je}, verbs ending in \textit{-je} and \textit{nije} \citep[cf.][97]{PiperKlajn14}. The usage of the CL \textit{ju} in standard Croatian is not restricted to contexts of suppletion.  

\subsection{Position of clitics or clitic cluster: second position}

This part can be summed up as follows. BCS authors emphasise that CLs cannot follow breathing breaks, i.e. they cannot follow punctuation symbols, brackets, inserted parts of a sentence, inserted sentences and listing. \citet[246]{TezakBabic96} also underline that CLs cannot follow a longer syntagm. Using full forms instead of the CL ones directly after a breathing break is recommended  \citep[cf.][105]{PiperIvic05}. CLs can follow the coordinative conjunctions \textit{pa}, \textit{te}, \textit{niti}, \textit{ali}, and \textit{ili} and they can never directly follow \textit{a}, \textit{i}, and \textit{ni}. In all three standard varieties CLs are posterior to the subordinating conjunctions. Serbian authors (e.g. \citealt[450]{PiperKlajn14}, \citealt[371]{StanojcicPopovic02}) claim that the right-most position of CL pronouns and reflexives is after their governing verb. 
The most interesting facts we found in Croatian and Serbian grammar books are those which concern different types of variation. \citet[374]{SilicPranjkovic07} ascribe the differences in CL placement to the spoken and written language register, i.e. diamesic variation. Similarly, \citet[452]{PiperKlajn14} consider that the variation in the placement of CLs depends on the type of CL, the sentence structure and the functional register in use, i.e. diaphasic variation. 

\subsection{Second position, delayed placement and phrase splitting}

We would like to highlight the following facts in this part. While it seems that in Croatian and Bosnian the second position rule is understood as 2W, in Serbian literature it is emphasised that 2P is normally understood as the position posterior to the first phrase. However, even some Serbian authors acknowledge that it is possible to split a phrase by CL insertion, but this is less preferred. \citet[450]{PiperKlajn14} specify the conditions under which CLs can be inserted into the first phrase in standard Serbian. 

In contrast to Serbian, in which phrase splitting is uncontroversial only in cases of adjective attributes, adverbs, and the words \textit{samo} and \textit{jedino} \citep[cf.][450]{PiperKlajn14}, Croatian and Bosnian standards allow the insertion of CLs in far more contexts. For instance, CLs can be inserted between a head noun and its noun attribute, even if the latter is a PP, between an apposition and a noun, between parts of an indefinite compound pronoun and an adverb, and between a question pronoun and a noun. Only in Bosnian and Croatian grammar books is it stated that CLs can split a forename from a family name. The data show some disagreement among scholars as to whether a phrase can be split by more than one CL. 

We would like to conclude this chapter by pointing out that in the works analysed above, CC, diaclisis and pseudodiaclisis have almost completely escaped the attention of the normativist authors, i.e. they were touched upon in only very few cases and superficially. This might be explained by the fact that these concepts are not established in traditional grammaticography. 
