\chapter{Acknowledgements}

This research was carried out within the project “Microvariation of the Pronominal and Auxiliary Clitics in Bosnian, Croatian and Serbian. Empirical Studies of Spoken Languages, Dialects and Heritage Languages”, in the years 2015 until 2019 generously supported by the German Research Foundation (HA 2659/6-1). Further, we received financial support from BAYHOST, DAAD and the women’s affairs officer at the Faculty for Linguistics, Literatures and Cultural Studies at the University of Regensburg \textcolor{black}{as well as from the Faculty of Humanities of the University of Klagenfurt}. 

It took slightly longer to complete than we expected, and would not have been possible without the contribution of many people, whom we would like to thank here.

We are honoured to publish our results with a high quality open-access publisher. Thus, we wish to express our gratitude to the two anonymous reviewers who provided us with valuable comments enriching the contents of the book, and to editorial board members of Open Slavic Linguistics, especially Roland Meyer, Luka Szucsich, and Radek Šimík, and Sebastian Nordhoff from Language Science Press for words of encouragement and support while editing this book.

Our project benefited considerably from seminars on statistics, programming, natural language processing, and corpus linguistics organized by Nikola Ljubešić, Maja Miličević, and Tanja Samardžić on behalf of the Regional Linguistic Data Initiative (\url{https://reldi.spur.uzh.ch/}) financed by the Swiss Science Foundation.

Our workflow would not have been as smooth without our undergraduate student assistants Theodora Tiha Loos and Eberhard Gade, who performed the data extraction and annotation, and the manuscript conversion to \LaTeX. We are very grateful to have had the book proofread by Krystyna Kupiszewska, an amazing and patient translator. Branimir Brgles from the Department of Onomastics and Etymology at the Institute of Croatian Language and Linguistics in Zagreb plotted the dialectological maps for Chapters \ref{Clitics in dialects} and \ref{Parameters of variation: conclusions}.

Naturally, we benefited from the feedback we received at conferences, round tables, seminars, and talks all over Europe. In particular, we owe thanks to Alexandr Rosen, Uwe Junghanns, Irenäus Kulik, Petr Karlík, and Anna Dušková for their invaluable comments and notes contributing to Chapters \ref{Approaches to clitic climbing} and \ref{Constraints on clitic climbing in Czech compared to Bosnian, Croatian and Serbian (theory and observations)}. We would like to recognise the impact of brainstorming with Václav Cvrček, Jana Pekarovičová, Petar Vuković, Alexandr Rosen, Jiří Hana, Pavel Kosek, Anna Dušková, Lenka Nerlich, Marek Nekula, participants of the round table \textit{Language, culture and variation} held in Regensburg on 4\textsuperscript{th} November 2016 and of the round table \textit{Mechanisms and constraints on Clitic Climbing} held in Regensburg on 19\textsuperscript{th} October 2017. It influenced our approach to clitics and variation outlined in Chapter \ref{Our terms and concepts} and the data gathering and annotation strategy, which were crucial for Chapters \ref{Clitics in a corpus of a spoken variety} and \ref{Experimental study on constraints on clitic climbing out of infinitive complements}. We applied Pavel Kosek’s advice on measuring heaviness of initial constituents in the data annotation for Chapter \ref{Clitics in a corpus of a spoken variety}. 
    
Moreover, we had the honour of presenting our project at the seminar \textit{Jezikoslovne rasprave} at the Institute of Croatian Language and Linguistics on 14\textsuperscript{th} December 2017 and at the Department of Slavonic Studies of Humboldt Universität zu Berlin on 13\textsuperscript{th} February 2019. Invaluable comments, suggestions and ideas for improvement made among others by Anita Peti-Stantić, Roland Meyer, Luka Szucsich, and Radek Šimík, who as an enthusiastic audience definitely had an impact on our work in Chapters \ref{Experimental study on constraints on clitic climbing out of infinitive complements} and \ref{Summary and outlook}. 

Finally, we were always able to count on constructive discussion with the participants of the research seminar at the Department of Slavonic Studies of the University of Regensburg who followed our work on the project through all these years. 

Jasmina Moskovljević Popović answered our questions regarding raising and control and gave us her monograph \citetitle{Moskovljevic07} which was more than useful during work on Part \ref{part3} of our book dedicated to clitic climbing. Ivana Kurtović Budja, Željka Brlobaš, Ljiljana Kolenić, Anita Celinić, Perina Vukša Nahod, and Milica Dinić Marinković provided us with dialectal data needed for Chapter \ref{Clitics in dialects}. Thanks to professor Tilman Berger from the University of Tübingen we were able to work with the full corpus Bosnian Interviews \citep{RaeckeStevanovic01}.

And last but not least, in the preparatory and final stages of the project we received useful comments from our consultants Petar Vuković and Petar Kehayov. 

The corpus studies would not have been possible without two very supportive figures from the NLP world: Tomaž Erjavec and Nikola Ljubešić. They helped us solve various problems with CQL queries and other technical issues. 

The speeded yes-no psycholinguistic acceptability judgment experiment with 336 native speakers as participants presented in Chapter \ref{Experimental study on constraints on clitic climbing out of infinitive complements} was a serious undertaking that would have been impossible if not for many colleagues who helped us find valuable contacts at different universities and faculties. These were, first and foremost, Ivana Brač, Tomislava Bošnjak Botica, Ana Ostroški Anić, and Siniša Runjić from the Institute of Croatian Language and Linguistics in Zagreb. 

The experiment would not have been possible without great effort and enthusiasm of staff members at various institutions of higher education in Croatia. For organising the necessary permissions, providing quiet rooms, additional computers, and IT support for conducting the experiment, as well as for encouraging students to participate in the study, we would like to express our gratitude to:
\begin{itemize}
\renewcommand\labelitemi{-}
    \item  Boris Lazarević, Milan Poljak, and Klaudija Carović-Stanko from the Faculty of Agriculture, University of Zagreb,
    \item  Ivan Botica, Zrinka Jelaska, Barbara Kušević, and Iva Nazalević Čučević from the Faculty of Humanities and Social Sciences, University of Zagreb,
    \item  Tea Žakula from the Faculty of Mechanical Engineering and Naval Architecture, University of Zagreb,
    \item  Đuro Njavro, Robert Manestar, and Ivanka Rajh from the Zagreb School of Economics and Management,
    \item  Marina Novak, Sanja Kiš Žuvela, Monika Jurić Janjik, Ivan Ćurković, and Petra Mitrović from the Academy of Music, University of Zagreb,
    \item  Ivan Grgurević, Borna Abramović, and Jasmina Pašagić Škrinjar from the Faculty of Transport and Traffic Sciences, University of Zagreb,
    \item  Tereza Rogić Lugarić from the Faculty of Law, University of Zagreb
    \item  Mirko Bošnjak, Boženko Ćosić, and Ružica Rajšić, from the Student Centre, University of Zagreb,
    \item  Krunoslav Zmaić, Siniša Ozimec, and Tihomir Florijančić from the Faculty of Agriculture, University of Osijek,
    \item  Senka Blažetić, Irena Labak, Ljiljana Krstin, Kristina Mandić, Tanja Žuna Pfeiffer, and Mario Dunić from the Department of Biology, University of Osijek,
    \item  Časlav Livada, Drago Žagar, Marijana Širić, Tomislav Matić, Mario Miloloža, Dalibor Buljić, Kruno Miličević, Dario Došen, Ana Šokčević, Luka Kruljac, and Tomislav Lovrić from the Faculty of Electrical Engineering, University of Osijek,
    \item  Mirjana Milić and Ana Penjak from the Faculty of Kinesiology, University of Split,
    \item  Josip Lasić, Marko Rimac, Renata Relja, Snježana Dobrota, Marita Brčić Kuljiš, Marija Lončar, Snježana Žana Dimzov, and Nataša Torlak from the Faculty of Humanities and Social Sciences, University of Split,
    \item  Anđelko Domazet, Domagoj Runje, Željko Matas, and Danko Kovačević from the Catholic Faculty of Theology, University of Split.
\end{itemize}
Part of the participants were recruited in Rokovci-Andrijaševci during the 2017 Christmas holidays. This was possible thanks to Damir Dekanić, Ante Rajković, Martin Majer, Martina Markota, Blanka Vincetic, Sanja Uremovic, and Ana Koprtla from the Rokovci-Andrijaševci municipality . 

Finally, we can say without any exaggeration that it was not easy to deliver this book. We would not have succeeded in this task without the people who stood by us during the past few years. Being there for us took a lot of patience, care, love and tolerance. We owe our children, spouses, parents, parents-in-law, relatives, friends, and colleagues a debt of gratitude for their help and understanding.



