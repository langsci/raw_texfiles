\documentclass[output=paper,
modfonts
]{LSP/langsci}

\ChapterDOI{10.5281/zenodo.495437}
%\usepackage{langsci-optional}
\usepackage{langsci-gb4e}
\usepackage{langsci-lgr}

\usepackage{listings}
\lstset{basicstyle=\ttfamily,tabsize=2,breaklines=true}

%added by author
% \usepackage{tipa}
\usepackage{multirow}
\graphicspath{{figures/}}
\usepackage{langsci-branding}

%
\newcommand{\sent}{\enumsentence}
\newcommand{\sents}{\eenumsentence}
\let\citeasnoun\citet

\renewcommand{\lsCoverTitleFont}[1]{\sffamily\addfontfeatures{Scale=MatchUppercase}\fontsize{44pt}{16mm}\selectfont #1}
  

\title{The domain of stress assignment: Word-boundedness and frequent collocation}

\author{%
Ellen Kaisse\affiliation{University of Washington}
}

% \chapterDOI{} %will be filled in at production
\epigram{Phenomena that a theory of the human language faculty ought to accommodate might well happen never to be attested because there is no course of change or borrowing by which they could arise. \citep[336]{anderson1992}}

\abstract{%
The linguistic literature treats hundreds of processes that apply between adjacent, open class content words. Overwhelmingly, these processes are local, segmental adjustments applying between the last segment of one word and the first segment of the next, such as voicing or place assimilation. Most other kinds of processes are profoundly underrepresented. Notably, iterative processes that eat their way across words such as vowel harmony, consonant harmony, or footing (assignment of rhythmic stress) typically do not extend beyond the word or count material outside the word when locating a particular stressable syllable, such as the penult. This result becomes more understandable when one considers how processes are phonologized from their phonetic precursors. Precursors are articulatorily or perceptually motivated and are strongest in temporally adjacent segments; they lose their force as the distance between segments increases, so there are no strong precursors for iteration outside of a word. Furthermore, frequent repetition leads to phonologization \citep{bybee2006a}. Any content word in a lexicon occurs next to any other content word much less frequently than do roots plus their affixes, hosts plus clitics, or any combination that includes at least one closed class item. So we should expect roots and affixes or hosts and clitics to be much more common as domains for stress assignment or other iterative rules than are strings of independent lexical items. In this paper, I concentrate on the near non-existence of stress assignment rules that span a domain larger than a morphological word plus clitics. We look at one revealing case that does treat some pairs of content words as a single span: stress assignment in literary Macedonian \citep{lunt1952}. The spans involve frequent collocations -- groups of already frequent words that are frequently heard together, sometimes to the extent that they have developed a lexicalized meaning. The other interword Macedonian cases involve closed class item such as prepositions and interrogatives. Finally, we consider evidence that rhythm can be evidenced statistically in the syntactic choices speakers make \citep{anttila2010,shih2016}, concluding that there may be rhythmic pressures from the lexicon to make phrases eurhythmic as well as eurhythmic pressures from phrases that can be phonologized to create rules of lexical stress assignment.
}

\begin{document}
\maketitle

\section{Introduction: what kinds of processes are postlexical and where do stress rules fit in?}\label{sec:kaisse:1}
Some types of phonological processes seem always to apply solely within a single word, not taking into account any material in adjacent words, while other types can apply between words. In work to appear \citep{kaisseforthcoming} I surveyed the literature on lexical (word-bounded) and postlexical (non-word-bounded) rules, sampling about 70 careful descriptions of non-tonal processes that make up their focus and environment from material that spans more than one content word.\footnote{Some tonal processes are well known to span large numbers of syllables, both within and between words \citep{hyman2011}, though many are also word-bounded. In Kaisse (forthcoming) I endorse David Odden’s (p.c. 2015) speculation about the long- distance behavior of tonal adjustments in \ili{Bantu} languages, which offer the most numerous and robust examples of processes where several tones in one word can be affected by a distant tone in an adjacent word. Odden cites the perceptual difficulty of locating tone in \ili{Bantu}, with its long words and subtle cues for tone, and tone’s low functional load there, since only H, not L tone needs to be marked.} Only a few involve anything other than strictly local adjustments between the last segment of one word and the first segment of the next. They might require agreement in voicing, place of articulation, or other features. Or they might avoid onset-less syllables\is{syllable} by deleting or desyllabifying a word final vowel when the next word begins with a vowel, or by moving a word final consonant into the \isi{onset} of the next, vowel-initial word. \REF{ex:kaisse:1} contains some fairly familiar examples from \ili{Spanish}. \REF{ex:kaisse:1a} illustrates postlexical voice \isi{assimilation} of /s/ to [z] before voiced consonants and \isi{assimilation} of continuancy (/g/ → [ɣ] and /b/ → [β]) after that /s/; \REF{ex:kaisse:1b} shows place \isi{assimilation} of a nasal to a following consonant and the retention of an underlying stop after non-continuant /n/; \REF{ex:kaisse:1c} shows reduction of hiatus; and \REF{ex:kaisse:1d} shows re\isi{syllabification} of a word-final consonant:

\begin{exe}
\ex \label{ex:kaisse:1} \ili{Spanish} (personal knowledge)
\begin{xlist}
\ex \label{ex:kaisse:1a} \gll /los ˈgato-s ˈbjen-en/ \ → \ [loz ˈɣatoz ˈβjenen]\\
the  cat-\textsc{pl}  come-3\textsc{pl}\\
\glt `The cats are coming.'
\ex \label{ex:kaisse:1b} \gll /ˈbjen-en ˈgato-s/ \ → \ [ˈbjeneŋ ˈgatos]\\
come-3\textsc{pl} cat-\textsc{pl}\\
\glt `Cats are coming.'
\ex \label{ex:kaisse:1c} \gll /ˈteŋ-o  ˈotro-s/ \ → \ [ˈteŋ ˈotros]\\
have-\textsc{1sg} other-\textsc{pl}\\
\glt `I have others.'
\ex \label{ex:kaisse:1d} \gll /ˈtjen-es ˈotro-s/ \ → \ [ˈtje.ne.ˈso.tros]\\
have-\textsc{2sg} other-\textsc{pl}\\
\glt `You have others.'
\end{xlist}
\end{exe}

\noindent Really, any local adjustment that is found within words can be found between words. This exuberance of types is probably due to the fact that most phonologized processes start life as natural local effects and these effects are not sensitive to grammatical information but rather to temporal adjacency (\citealt{kiparsky1982b} et seq.). However, I found almost no vowel harmony processes that extended into an adjacent content word, no consonant harmony processes \citep{hansson2010}, and, crucially for the current paper, no \isi{stress} rules with a domain larger than the morphological word or the morphological word plus clitics. Compare the familiar types of examples in \REF{ex:kaisse:1} with the fanciful ones in \REF{ex:kaisse:2}, where something resembling the \ili{English} rule that constructs moraic trochees from the end of a word takes a whole noun phrase as its domain, resulting in feet\is{foot} that span syllables\is{syllable} belonging to different words and in wide-scale \isi{allomorphy}:

\ea \label{ex:kaisse:2} Fanciful \ili{English} with noun phrase as initial domain of footing
	\ea \label{ex:kaisse:2a}\begin{tikzpicture}[baseline=(2ni.base)]
\matrix [matrix of nodes, nodes={inner xsep=2pt, inner ysep=0pt, outer sep=0pt}]
{\mgmorph{taj} & \mgmorph{ni} & {}  & \mgmorph{dag}\\
};
\mgone{taj}\mgtwo{ni}\mgone[.]{dag}
\mgfoot{taj}{taj}\mgfoot{ni}{dag}
\mgldelim{/}{taj}\mgrdelim{/}{dag}
\end{tikzpicture}\\
{[ˌtajˈnidəg]}\\
`tiny dog'	
	\ex \label{ex:kaisse:2b} \begin{tikzpicture}[baseline=(2kɪ.base)]
	\matrix [matrix of nodes, nodes={inner xsep=1pt, inner ysep=0pt, outer sep=0pt}]
	{\mgmorph{taj} & \mgmorph{ni} & {}  & \mgmorph{kɪ} & \mgmorph{tɛn} \\
	};
	\mgone{taj}\mgone[.]{ni}\mgtwo{kɪ}\mgone[.]{tɛn}
	\mgfoot{taj}{ni}\mgfoot{kɪ}{tɛn}
	\mgldelim{/}{taj}\mgrdelim{/}{tɛn}
	\end{tikzpicture}\\
	{[ˌtajniˈkɪɾən]}\\
	`tiny kitten'

	\ex \label{ex:kaisse:2c} \begin{tikzpicture}[baseline=(2ɹa.base)]
	\matrix [matrix of nodes, nodes={inner xsep=3pt, inner ysep=0pt, outer sep=0pt}]
	{\mgmorph{taj} & \mgmorph{ni} & {}  & \mgmorph{pɪ} & \mgmorph{ɹa} & \mgmorph{na} \\
	};
	\mgone{taj}\mgone{ni}\mgone[.]{pɪ}\mgtwo{ɹa}\mgone[.]{na}
	\mgfoot{taj}{taj}\mgfoot{ni}{pɪ}\mgfoot{ɹa}{na}
	\mgldelim{/}{taj}\mgrdelim{/}{na}
	\end{tikzpicture}\\
	{[ˌtajˌnipəˈɹanə]}\\
	`tiny piranha'
	\z
\z

\noindent In this paper, I will describe the continuum of \isi{stress} behavior of cohering and non-cohering affixes\is{affix} (i.e.\ affixes that do and do not interact phonologically with their bases), clitics of varying degrees of stress-interaction\is{stress} with their hosts, \isi{compound} words, and the one detailed presentation of a \isi{stress} rule I have encountered where initial \isi{stress} assignment takes some strings of content words and treats them as a single domain, ignoring any \isi{stress} the component words might otherwise have in isolation. That case comes from literary Macedonian \citep{lunt1952,franks1987,franks1989}.

Before continuing, I should make it clear that there are many rhythmic adjustments that do apply between content words -- cases like the Rhythm Rule in \ili{English} \citep{hayes1984k} which is responsible for the retraction of the secondary \isi{stress} in ˌ\textit{Japanese ˈlanguage} (vs.\ ˌ\textit{Japaˈnese}) or \textit{ˌMississippi ˈmud} vs.\ \textit{ˌMissiˈssippi}. These kinds of cases are postlexical and involve prosodically self-sufficient content words that have had their own stresses\is{stress} assigned lexically, independent of the larger context in which they find themselves. There is then a rhythmic adjustment that demotes or moves a nonprimary \isi{stress} in order to avoid \isi{stress} clash. Like the invented example \REF{ex:kaisse:2}, the Macedonian case that I will look at instead involves assigning a single antepenultimate \isi{stress} to a string of two content words which, in other syntactic contexts, would each receive a \isi{stress} of their own. Often the single \isi{stress} does not fall on any of the syllables that would have been stressed\is{stress} in isolation. This is clearly different from the way a rhythm rule works. 

Another example of post\isi{lexical stress}\is{stress} adjustment, as opposed to the first pass of \isi{stress} assignment, involves \isi{compound} \isi{stress} rules. Again, these are not relevant to my claim that pairs of content words are almost never the domain of the first pass of \isi{stress} assignment. The most well-known of these \isi{compound} \isi{stress} rules, like that of \ili{English}, also simply adjust the relative primary stresses\is{stress} of the members, rather than treating them as a single unit for the lexical footing process. Thus, if we put together \textit{linˈguistics} and \textit{ˈscholar} we get \textit{linˈguistics ˌscholar}, with the primary \isi{stress} of \textit{ˈscholar} subordinated to that of \textit{linˈguistics}, but we do not \isi{stress} the whole string as a single \isi{prosodic word}, which might result in antepenultimate \isi{stress} falling on the last syllable of the first member, with the bizarre (for \ili{English}) output \textit{linguiˈstics scholar}. We shall see, however, that occasionally languages (such as \ili{Modern Greek}) do take \isi{compound} words as the initial domain of \isi{stress} assignment. So while prosodically independent content words are not commonly the domain for the first pass of \isi{stress} assignment, some languages stretch that domain to include both members of a \isi{compound} word. 

Yet another \isi{aspect} of the postlexical adjustment of \isi{stress} is offered by Newman’s \citeyearpar[28--29]{newman1944} insightful early discussion of \isi{stress} domains in Yokuts, pointed out to me by an anonymous referee. Nouns and verbs maintain their lexical stresses\is{stress} in connected speech but \isi{function words} tend to lean on them and to lack stresses\is{stress} of their own, and the faster the speech, the bigger the phrasing groups and the fewer the perceived stresses.\is{stress} Newman’s description is perforce rather general and impressionistic, but it summarizes well the plasticity of post\isi{lexical stress}\is{stress} adjustments, which favor cliticization of \isi{function words} and variable rhythmic groupings dependent on tempo and number of syllables.\is{syllable}\footnote{I cannot tell from Newman’s description how content words that are not nouns or verbs might behave. I tentatively conclude that he is referring only to lack of stress on \isi{function words} in rapid connected speech, not to complete loss of stress on open class modifiers or other content words.}

While truly grammaticized rhythmic \isi{stress} assignment almost never takes an initial domain beyond a word and its affixes and clitics, there are now known to be gradient rhythmic effects that extend beyond the word. They simply don’t seem to be able to rise to the level of \isi{phonologization} in that larger domain. \citet{martin2007}, \citet{anttila2010}, and \citet{shih2016} have found that vowel harmony (Martin) and optimization of rhythm can have statistical reflections in \ili{Turkish} \isi{compound} words and in \ili{English} word order.\is{word order} That is, to use Martin’s felicitous phrasing, lexical generalizations that are part of the grammar can “leak” into larger domains, causing statistical preferences for outputs that are consonant with the \isi{phonology} of a language’s words. I would add that it is hard to know what the primary direction of leakage really is: the leakage Martin and Shih posit from smaller domains to larger ones probably co-exists with the direction of larger-to-smaller domain \isi{phonologization}, which I posit in \citet{kaisseforthcoming} and which is the staple view of Lexical Phonology\is{phonology} and its descendants \citep{kiparsky1982b,bermudez-Otero2015} and of most traditional approaches to \isi{sound change}. In \citet{kaisseforthcoming} I used the example of the phonetic precursor of vowel harmony, namely the vowel-to-vowel coarticulation that peters out as one gets farther away from the source vowel \citep{ohman1966,ohala1992,flemming1997}, which can be grammaticized within words because stems\is{stem} and their affixes\is{affix} are in frequent collocation \citep{bybee2006a}\footnote{\citet{barnes2006} cites \citet{inkelas2001k} for evidence that phonetic vowel-to-vowel coarticulation is problematic as a simple, unaided source for vowel harmony. Their argument comes from \ili{Turkish}, where anticipatory phonetic effects are stronger than perseveratory ones, but the phonologized harmony system is perseveratory. He instead attributes the \isi{phonologization} of vowel harmony to vowel-to-vowel coarticulation coupled with lengthening of the trigger \isi{syllable} and \isi{paradigm} uniformity effects allowing longer-distance effects on distant affixes.}. Classical Lexical Phonology\is{phonology} postulates this one-way direction (from postlexical and variable or gradient to lexical and regular and obligatory), but there is no reason that once a rule is \isi{lexicalized}, it cannot then generalize postlexically again \citep{kaisse1993}. One can imagine a feedback loop, where small postlexical \isi{variation} in favor of alternating stresses and avoidance of \isi{stress} clash starts to be phonologized as iterative footing, while iterative footing starts to make syntactic phrases that are eurhythmic more desirable as choices for speakers in real time. 

There is a continuum of likely domains for foot-construction.\is{foot} \citet{selkirk1995}, \citet{peperkamp1997} and \citet{anderson2011} demonstrate that there are various types of clitics that range from more to less rhythmically cohesive (interactive) with their hosts. I will extend the continuum, showing that in literary Macedonian \citep{lunt1952,franks1987,franks1989} the \isi{lexical stress}\is{stress} rule assigning antepenultimate \isi{stress} has stretched its domain to include even two content words, but only when they are in frequent collocation and, in some of the cases, have taken on a more \isi{lexicalized}, semantically less \isi{compositional} meaning. This is the “exception that proves the rule.” In general, only the supremely frequent collocation of a closed class \isi{bound morpheme} -- an \isi{affix} -- with another affix or the root it can attach to provides the frequent collocation that allows for \isi{phonologization}. However, occasionally even two content words can appear together so frequently that they become subject to the first pass of the \isi{lexical stress}\is{stress} rules of the language; they act like a single word for the purposes of building a \isi{foot} at the edge of a word. Clitics and \isi{function words} in Macedonian also form part of the initial domain for this foot-building.\is{foot} This accords with the general observation that the \isi{phonology} of clitics is like the \isi{phonology} of affixes\is{affix} in many cases -- they are ‘phrasal affixes’ \citep{anderson1992}. Clitics are usually less cohering than stem-level affixes but, as we shall illustrate in \sectref{sec:kaisse:2}, there are even a few that act like they are inside the \isi{phonological word} for the purpose of foot-construction.\is{foot} But basic \isi{foot} building algorithms do not typically extend beyond the morphological word. Occasionally they extend even less far than that, as in the case of non-cohering (stress neutral)\is{stress} suffixes of \ili{English}, and occasionally they extend into the larger \isi{phonological word}, which can include clitics and other \isi{function words} that can have stressless variants  -- i.e.\ be \isi{prosodically deficient}, in Anderson’s \citeyearpar{anderson2011}’s terms. Indeed, the failure of an \isi{affix} to participate in the lexical footing domain is one of the main ways in which non-cohering affixes\is{affix} have been defined. Note for instance that \citet[84ff]{chomskyhalle1968} use the term “stress neutral”\is{stress} for the group of suffixes\is{suffixation|(} which, to anticipate the later interpretation of their intentions, are outside the \isi{prosodic word}. These suffixes, which include all the inflectional\is{inflection} affixes,\is{affix} as well as a number \isi{derivational affixes}\is{affix}\is{derivation} such as \#\textit{ly, \#ness,} agentive \textit{\#er, \#ful,} and many others, not only fail to affect \isi{stress} but also don’t induce word-internal rules like Trisyllabic Shortening and Sonorant Syllabification. (See \citealt{kiparsky1982b} for a full discussion of Trisyllabic Laxing and \citealt{kaisse2005} for a summary of the characteristics of \ili{English} cohering suffixes.) Indeed, we might ask why \isi{stress} is one of the most typical diagnostics for cohering suffixes, and not only in \ili{English}. I would speculate that rhythm is hard to grammaticize as a phonological process without frequent repetition of the same strings. Like clitics, word-level suffixes have less stringent \isi{subcategorization} restrictions. \citet{fabb1988}, which we will discuss in a bit more detail in \sectref{sec:kaisse:2} discovered that the possible combinations of \ili{English} stem-level suffixes with other suffixes or with roots are very restricted. On the other hand, word-level suffixes can combine freely with many suffixes and words, and therefore are not in as frequent collocation as the stem level ones, which are heard over and over again with the same preceding morpheme, be it an \isi{affix} or a root.\is{suffixation|)} Clitics are even less demanding of the preceding host -- indeed in some cases, such as \isi{special clitics}, the host can belong to any part of speech and can even be a phrase -- so they are less likely to be phonologized as part of the \isi{stress} domain. But because they are prosodically unable to stand on their own, they must lean on a host and thus may sometimes be phonologized as part of the \isi{stress} domain. 

\section[Clitics and the stress domain continuum]{Clitics and the stress domain continuum}\label{sec:kaisse:2}

There has been considerable attention paid to the various ways that clitics -- \isi{prosodically deficient} items that must lean on a host in some fashion -- interact with the \isi{lexical stress}\is{stress} assignment rules of a language and fall into their domain. I will summarize some recent results here because I believe that the insights from clitics can help us understand how \isi{compound} words and even some phrases can come to behave in the same way as clitics do with respect to their hosts.

It is worth reviewing some of the typical diagnostics of clitics (paraphrased from \citealt{zwickypullum1983}):

\ea \label{ex:kaisse:3} Characteristics of clitics
	\ea \label{ex:kaisse:3a} clitics have a low degrees of \isi{selection} with respect to their hosts; affixes\is{affix} have a higher degree of \isi{selection}. 
	\ex \label{ex:kaisse:3b} affixed words are more likely to have idiosyncratic \isi{semantics} than host + \isi{clitic} combinations
	\ex \label{ex:kaisse:3c} affixed words are treated as a unit by the syntax while hosts plus their clitics are not. 
	\z
\z

\noindent To these we can add that while the boundary between a \isi{root} and stem-level (cohering) \isi{affix} is the most favorable position for phonological\is{phonology} interactions to occur, and the boundary between base and word-level (non-cohering) suffixes somewhat less favorable, clitics are sometimes phonologically even less connected -- less cohesive -- with their hosts; they might fail, for instance, to undergo vowel harmony, the last segment of their hosts might undergo word-final devoicing, as if there were no following segment, and as we shall see, in some languages, the clitics might not be visible to the \isi{lexical stress}\is{stress} rules or at least to the first pass of those rules.

Beginning with Anderson’s \citeyearpar{anderson1992} classification of clitics as phrasal affixes as a foundation, \citet{selkirk1995} and \citet{peperkamp1997}, elaborated in \citet{anderson2011}, propose a hierarchy of \isi{clitic} types. Ranked from least-cohering to most cohering with respect to phonological interaction with their host, we have the continuum in \REF{ex:kaisse:4}:

\ea \label{ex:kaisse:4} The \isi{clitic} continuum\\
\isi{prosodic word} \isi{clitic} {\textgreater} free \isi{clitic} {\textgreater} affixal \isi{clitic} {\textgreater} internal \isi{clitic}\\
\begin{tabularx}{\linewidth-0.1em}[t]{lXlX}
	a. & \begin{tabular}[t]{l}
		Prosodic Word Clitic \\
	\begin{forest}
	where n children=0{tier=word}{}
	[PPh [PWd [Host] ] [PWd [Clitic] ] ]
	\end{forest}
	\end{tabular} &

	b. & \begin{tabular}[t]{l}Free Clitic \\
	\begin{forest}
	where n children=0{tier=word}{}
	[PPh [PWd [Host] ] [Clitic ] ] 
	\end{forest}
	\end{tabular}\\

\end{tabularx}

\begin{tabularx}{\linewidth-0.1em}[t]{lXlX}
	c. &  \begin{tabular}[t]{l}Affixal Clitic \\
	\begin{forest}
		where n children=0{tier=word}{}
	[PPh [PWd [PWd [Host] ] [Clitic ] ] ]
	\end{forest}
	\end{tabular} &
	
	d. & \begin{tabular}[t]{l}Internal Clitic \\
	\begin{forest}
	where n children=0{tier=word}{}
	[PPh [PWd [Host ] [Clitic ] ] ]
	\end{forest}
	\end{tabular} \\
\end{tabularx}
\z

\noindent Intuitively speaking, we know that clitics usually fall somewhere in between the phonological cohesiveness of stem-level, highly cohering affixes\is{affix} and the phonological non-cohesiveness of independent prosodic words. But in this elaborated hierarchy, not all clitics do. There are clitics termed ‘internal’ by Selkirk that act, at least for some processes, exactly like stem-level affixes.\is{affix} We will extend this notion to the \isi{compound} words of \ili{Modern Greek} and the extended \isi{stress} domains of Macedonian -- these involve normally prosodically independent words that nonetheless act as a single domain and receive one \isi{stress} as if they were a single word. On the other end, there are clitics that Selkirk terms ‘\isi{prosodic word} clitics;’ these are forced in some cases to behave like independent prosodic words despite their underlying prosodic deficiency. In \REF{ex:kaisse:5}, I illustrate the internal type with the example of verbal clitics in \ili{Formentera Catalan} \citep{torres-Tamarit2015}, where the whole verb-\isi{clitic} complex is used to assign moraic trochees from the right. This can result in no \isi{stress} falling on the verb \isi{stem} itself if there are two monosyllabic clitics \REF{ex:kaisse:5d} or one monosyllabic \isi{clitic} followed by a \isi{clitic} consisting of a single consonant that makes a heavy syllable \REF{ex:kaisse:5e}. The raising of the root /o/ of the imperative verb in (\ref{ex:kaisse:5b}--\ref{ex:kaisse:5e}), which only occurs to unstressed mid vowels, suggests that \isi{stress} is not assigned cyclically to the \isi{root} and then again to the host-\isi{clitic} complex, but rather all in one go to the entire string:

\ea \label{ex:kaisse:5} \ili{Formentera Catalan}  \citep{torres-Tamarit2015} internal clitics\\
/kompr-ə/ 

	\ea \label{ex:kaisse:5a}	\gll ˈkompr \squish-ə  \\
		buy \squish-\textsc{2sg.imp}\\
		\glt `Buy!'

	\ex \label{ex:kaisse:5b}	\gll kumˈpr \squish-ə  \squish=l\\
		buy \squish-\textsc{2sg.imp} \squish=\textsc{3sg.m.acc}\\
		\glt `Buy it/him!'

	\ex \label{ex:kaisse:5c}	\gll kumˈpr \squish-ə  \squish=n\\ 
		buy \squish-\textsc{2sg.imp} \squish\textsc{=part}\\
		\glt `Buy some!'

	\ex \label{ex:kaisse:5d}	\gll kumpr \squish-ə \squish=ˈmə \squish=lə \\
		buy \squish\textsc{-2sg.imp} \squish\textsc{=1sgdat} \squish\textsc{=3sg.f.acc}\\
		\glt `Buy it/her for me!'

	\ex \label{ex:kaisse:5e}	\gll kumpr \squish-ə  \squish=ˈmə  \squish=l\\ 
		buy \squish-\textsc{2sg.imp} \squish\textsc{=1sg.dat} \squish\textsc{=3m.acc}\\
		\glt `Buy it/him for me!'
	\z
\z

\noindent Internal clitics meet the criteria of low \isi{selection}, predictable \isi{semantics}, and so forth, but phonologically\is{phonology} they are unusual -- they act like stem-level affixes. The situation in Formentera is similar to that described for Lucanian \ili{Italian} clitics by \citet{peperkamp1997}. Formentera and Lucanian internal clitics are wholly included within the \isi{phonological word} and count from the outset in the calculation of where a right edge trochaic \isi{foot} should be built.

More generally speaking, internal clitics are \isi{prosodically deficient} words that act more cohesively in their phonological interaction with their host than their other characteristics would suggest they should. This is captured in Selkirk’s framework by having them form a single \isi{prosodic word} in combination with their host. Their hallmark is that they are stressed\is{stress} and generally processed by the word-bounded \is{phonology} on the first pass, acting just like a stem-level affix. This means that they can never receive a second \isi{stress} all their own, but they may happen to be in position to take the \isi{stress} of the whole host-\isi{clitic} string, as illustrated in the Formentera form \REF{ex:kaisse:5d} [kumprə=ˈmə=lə], where the \isi{clitic} /mə/ receives the penultimate \isi{stress} for the whole host-\isi{clitic} complex. I suggest that we can profitably extend this insight onto bigger, more independent words than clitics. We will see in \sectref{sec:kaisse:3} that \ili{Modern Greek} treats the otherwise independent pieces of a \isi{compound} word as a single domain for \isi{stress} assignment -- a comparatively rare phenomenon. And Macedonian (\sectref{sec:kaisse:4}) treats prepositions, even polysyllabic ones, as a single \isi{stress} domain with their objects and treats certain common collocations of adjective + noun as a single \isi{stress} domain as well. To put it another way, one occasionally encounters situations where a pair of words that should be expected to be prosodically independent can act like ‘internal words,’ parallel to the notion of internal clitics. I will link this unusual phenomenon to frequency of collocation. The more exemplars there are of a string of words, repeatedly heard together, the more likely they are to be internalized as a rhythmic group and to be reified as a domain for the assignment of a single stress.\is{stress}

At the other end of the continuum of independence of clitics lie the Prosodic Word clitics, which receive postlexical treatment as independent prosodic words even though they are underlyingly \isi{prosodically deficient}. They are illustrated by the prevocalic clitics of \ili{Bilua} (\citealt{obata2003}, reported in \citealt{anderson2011}), where preconsonantal proclitics are normally free  --  they are outside the domain of \isi{stress} assignment, which targets the first syllable of the word: 

\ea \label{ex:kaisse:6} \ili{Bilua} (\citealt{obata2003}, \citealt[2004]{anderson2011}) free \isi{clitic}\\
% 	\ea \label{ex:kaisse:6a} 
	\gll o=  {\squish ˈβouβaɛ} \squish=k \squish=a\\
	\textsc{3sg.m}= {\squish kill} \squish=\textsc{3sg.fem.obj} \squish\textsc{=tns}\\
	\glt ‘He killed it.’
% 	\z
\z

\noindent When a host word begins with a vowel, however, the usually stressless clitics instead take on a derived prosodic word-hood of their own and receive an independent stress: \is{stress}

\ea \label{ex:kaisse:7} \ili{Bilua} (\citealt{obata2003}, \citealt{anderson2011}) PWd \isi{clitic} under duress\\
% 	\ea \label{ex:kaisse:} 
	\gll  ˈo=  {\squish ˈodiɛ} \squish=k \squish=a\\
	3\textsc{sg.m}=  {\squish call} \squish=\textsc{3sg.fem.obj} \squish\textsc{=tns}\\
	\glt  ‘He called her.’
% 	\z
\z
\noindent In this case, the \isi{clitic} is forced to act like content words do normally.

A referee has expressed some doubt about the proper analysis of \ili{Bilua} clitics and, by extension, whether the category of prosodic clitics needs to exist at all. But it is useful to suspend our skepticism and contemplate such an entity because it illustrates the opposite side of a prosodic coin. The putative prosodic clitics are elements that are inherently \isi{prosodically deficient}, but that can be forced under certain circumstances to act as independent prosodic words with their own stress.\is{stress} Macedonian nouns and adjectives in certain frequent collocations are inherently prosodically independent elements that can be forced under certain circumstances to act as \isi{prosodically deficient} pieces of a single \isi{prosodic word}, as are the words that make up a \isi{compound} in \ili{Modern Greek}.

To complete the hierarchy, there are clitics whose characteristics of independence place them between internal and Prosodic Word clitics. The affixal clitics and the \isi{free clitics} are probably the ones that linguists are most used to encountering. An affixal \isi{clitic} shows some prosodic independence from its host: it is not included in the domain of footing on the first pass. In Selkirk’s framework it forms a recursive \isi{prosodic word} with that host, and it therefore may receive or induce an additional \isi{stress} or \isi{stress shift} once it is included in a second pass of \isi{stress} assignment. When word \isi{stress} falls on the penult or antepenult of a content word, we find an additional \isi{stress} added to a string of clitics if they can form a foot,\is{foot} as shown in the third column of \REF{ex:kaisse:8} (Binary footing is indicated with parentheses.):

\ea \label{ex:kaisse:8} Neapolitan \ili{Italian} \citep{peperkamp1997} affixal clitics\\
\begin{tabular}{llll}
a. & ˈkonta & ˈkonta=lə & (ˈkonta)=(ˈti=llə)\\
	& tell.\textsc{imp} & tell.\textsc{imp} \squish=it &  tell.\textsc{imp}=you=it\\
b. & ˈpettina &  ˈpettina =lə &  (ˈpetti)na=(ˈti=llə)\\
	&comb.\textsc{imp} & comb.\textsc{imp}=it &  comb.\textsc{imp}=you=it\\
\end{tabular}
\z

\noindent Neapolitan \ili{Italian} treats the morphological word, sans clitics, as the basic, initial domain of footing, but clitics are partially cohering, falling into a larger domain that still follows the basic principles of word-internal \isi{stress} assignment. In Neapolitan, if the \isi{stress} assigned to the \isi{clitic} sequence clashes with the \isi{stress} of the host, the host's \isi{stress} is lost:

\begin{exe}
\ex \label{ex:kaisse:9}  Neapolitan \ili{Italian} \citep{peperkamp1997} affixal clitics with clash reduction

\gll ˈfa {\quad\quad} ˈfal \squish=lə {\quad\quad} fa \squish=ˈtti  \squish=llə\\
	do.\textsc{imp} {} do.\textsc{imp} \squish=it {} do.\textsc{imp} \squish=you \squish=it\\
\z

\noindent The parallel for us here is that the pieces of \isi{compound} words and larger phrases usually receive \isi{stress} treatment individually for their component parts, but a postlexical pass of \isi{stress} assignment can promote, move, or demote one of the stresses\is{stress} once the two elements are considered together in a larger domain. In \ili{English} compounding,\is{compound} as we have mentioned, the component prosodic words each receive a \isi{lexical stress}\is{stress} but the rhythm of the two is adjusted when they come together, promoting the first \isi{stress} to primary:

\ea \label{ex:kaisse:10} \ili{English} compound \\
linˈguistics deˈpartment  → linˈguistics deˌpartment
\z 

\noindent And in \ili{English} phrases, which typically have the most prominent \isi{stress} on the final content word, the Rhythm Rule \citep{hayes1984k} adjusts the tertiary and secondary stresses\is{stress} of the first word in that phrase to avoid \isi{stress} clash with the primary stress.\is{stress}

\noindent\begin{minipage}{.99\linewidth}
\ea \label{ex:kaisse:11} \ili{English} phrase\\
\begin{tikzpicture}[baseline=(Ja.base)]
\matrix [matrix of nodes, nodes={inner sep=0pt, outer sep=0pt}]
{{[} & \mgmorph{Ja} & \mgmorph{pa} & \mgmorph{nese} & {] [} & \mgmorph{lan} & \mgmorph{guage} & {]}\\
};
\mgone{Ja}\mgone[.]{pa}\mgtwo{nese}\mgthree{lan}\mgone[.]{guage}
\mgfoot{Ja}{pa}\mgfoot{nese}{nese}\mgfoot{lan}{guage}
\end{tikzpicture}
→
\begin{tikzpicture}[baseline=(Ja.base)]
\matrix [matrix of nodes, nodes={inner sep=0pt, outer sep=0pt}]
{{[} & \mgmorph{Ja} & \mgmorph{pa} & \mgmorph{nese} & {] [} & \mgmorph{lan} & \mgmorph{guage} & {]}\\
};
\mgtwo{Ja}\mgone[.]{pa}\mgone{nese}\mgthree{lan}\mgone[.]{guage}
\mgfoot{Ja}{pa}\mgfoot{nese}{nese}\mgfoot{lan}{guage}
\end{tikzpicture}
\z
\end{minipage}

The last type of \isi{clitic} we need to discuss is ‘free.’ Such clitics simply do not interact with their host for the purposes of \isi{stress} assignment. They are not themselves independent prosodic words but do not fall inside of any \isi{prosodic word} and thus are invisible to every pass of footing. For instance, \ili{Barcelona Catalan} verbs \citep{torres-Tamarit2015} show the same \isi{stress} whether or not they have enclitics, and the \isi{stress} can fall three or even four syllables from the end of the word, in contravention to the generalization that there is a three-syllable final \isi{stress} window and that \isi{stress} is typically penultimate. 

\ea \label{ex:kaisse:12} \ili{Barcelona Catalan} \citep{torres-Tamarit2015}
	\ea \label{ex:kaisse:12a} \gll ˈkompr \squish-ə\\
	buy \squish-\textsc{2sg.imp}\\
	\glt  `Buy!'
	
	\ex \label{ex:kaisse:12b} \gll ˈkompr \squish-ə  \squish=mə  \squish=lə\\
	buy \squish\textsc{-2sg.imp} \squish\textsc{=1sgdat} \squish\textsc{=3sg.f.acc}\\
	\glt `Buy it/her for me!'

	\ex \label{ex:kaisse:12c} \gll ˈkompr \squish-ə  \squish=mə  \squish=l\\
	buy \squish-\textsc{2sg.imp} \squish\textsc{=1sgdat} \squish\textsc{=3sg.m.acc}\\
	\glt `Buy it/him for me!'
	\z
\z

\noindent The parallel for larger units is this: in \sectref{sec:kaisse:4.4} we will see that  while some Macedonian prepositions are inherently stressless and are included in a stress-assignment\is{stress} domain with their objects (i.e.\ they are ‘internal’), others always remain stressless and do not form a domain with their objects (i.e.\ they are free). 

Even though \isi{stress} is very often iterative within a word and its stem-level affixes,\is{affix} it can be unable to cross even the relatively weak boundary of a word level suffix. Consider for instance the lack of movement of \isi{stress} from the base in \ili{English} words suffixed by word-level suffixes such as \textit{\#hood} and \textit{\#ly,} and the concomitant location of \isi{stress} outside the lexically mandated window of the last three syllables:

\ea \label{ex:kaisse:13} \ili{English}
	\ea \label{ex:kaisse:13a} ˈdialect\#hood (c.f.\ ˈdialect, ˈdiaˈlect-al)
	\ex \label{ex:kaisse:13b} ˈmanifestly (c.f.\ ˈmanifest, ˌmanifesˈtation)
	\z
\z

\noindent If even an \isi{affix} can be outside the domain of \isi{stress} assignment and other phonological processes, it is not surprising that \isi{stress} assignment often does not count clitics on the initial pass (or ever), let alone treat them as single domain compound words, words plus function items leaning on them, or the strings of prosodically independent content words. 

Let us tie these increasingly unlikely domains for \isi{stress} assignment to frequency of collocation. Cohesive affixes\is{affix} are typically less productive and more selective with respect to the stems\is{stem} they combine with than are non-cohesive affixes,\is{affix} which are somewhat clitic-like. There is thus a hierarchy of selectiveness tied to frequency of collocation:\footnote{A referee raises the excellent question of “whether this hierarchy predicts an implicational \isi{typology} of \isi{stress} domains, e.g.\ if a language treats \isi{compound} words as a single stress domain, then clitics and all affixes\is{affix} should be parsed within the same stress domain as their host.” I do not know the answer to this question, though it does seem to be true for \ili{Modern Greek} (discussed for instance in \citealt{anderson2011} as having affixal clitics which affect the \isi{stress} of their hosts) and for Macedonian.} 

\ea \label{ex:kaisse:14} Hierarchy of selectiveness
\begin{itemize}
\item stem-level cohering affixes \is{affix}
\item word-level non-cohering affixes \is{affix}
\item clitics
\item \isi{compound} words
\item words in fixed expressions 
\item collocations involving closed class items, such as prepositions, even if they are polysyllabic
\item truly novel or unpredictable collocations that are clearly not listed in the lexicon
\end{itemize}
\z

\noindent For instance, \citet{fabb1988} discovered that the \ili{English} stem-level adjective-forming suffix\is{suffixation} \textit{{}-ous} has very narrow sectional restrictions. It cannot attach after any other \isi{affix} at all. It’s clear that \form{-ous} is stem level because it affects the \isi{stress} of its base (\textit{ˈ}\textit{moment, mo}\textit{ˈ}\textit{mentous}), triggers Trisyllabic Laxing (\textit{ˈ}\textit{\=o}\textit{men,} \textit{ˈ}\textit{\u{o}}\textit{minous)} and meets the various other tests for cohesion in the literature. Similarly, the stem-level adjective-forming suffix\is{suffixation} \textit{{}-ic} attaches only after the suffix \textit{{}-ist (capitalistic),} and the stem-level adjective-forming suffix \textit{{}-ary} attaches only after the suffix\textit{ -ion (revolutionary)}. On the other hand, word-level adjective-forming \#\textit{y} can attach productively to virtually any noun (\textit{chocolate\#y}, or \textit{protein\#y,} recently heard on a yoghurt commercial), and it can occur in novel coinages after most other noun-forming suffixes (\textit{musi-cian\#y, profess-or\#y.}) Like clitic-host \isi{semantics}, the meanings of words with word-level affixes\is{affix} are usually transparent. Finally, notice that ˈ\textit{chocolate\#y} has primary \isi{stress} in the same place as the base ˈ\textit{chocolate}, even though, if ˈ\textit{chocolate} is pronounced with three syllables, \isi{stress} in ˈ\textit{chocolate\#y} falls in an otherwise impossible pre-antepenultimate position. The same is true for ˈ\textit{manifestly} \REF{ex:kaisse:13b}. Non-cohesive affixes\is{affix} are in a sense invisible to the lexical \isi{phonology} of the language and can violate various of its requirements. 

Clitics are even less selective than word-level affixes. For example, many Romance, Greek and Slavic clitics attach to any verb and after any set of affixes. Similarly, the \ili{English} contracted auxiliary \textit{=s} can be found leaning leftward on words of almost any category. 

\ea \label{ex:kaisse:15} \ili{English}
	\ea \label{ex:kaisse:15a} That man’s a linguist; 
	\ex \label{ex:kaisse:15b} The thing you are leaning on’s not safe; 
	\ex \label{ex:kaisse:15c} What you think’s not important;
	\ex \label{ex:kaisse:15d} Skin that looks pink’s an indication of good circulation.
	\z
\z

\noindent And both within compounds and within sentences almost any word can be followed by almost any other word -- there are almost no selectional requirements. However, some words belong to closed classes -- prepositions, relative and interrogative elements, and other \isi{function words} and therefore will recur sequences more frequently. This cline lines up with the likelihood of two elements being taken into a single \isi{stress} domain.

The hierarchy in \REF{ex:kaisse:14} will take us through the extended domain cases of Macedonian, where prepositions -- even polysyllabic and semantically rich ones -- are unstressable on their own, always forming a \isi{stress} domain with their complements\is{complement} or being unstressable, and where frequent collocations of content words, particularly involving frequent words or collocations with unpredictable or frozen \isi{semantics}, are stressed\is{stress} as if they were single words. 

\section[Compounds and the stress domain continuum]{Compounds and the stress domain continuum}\label{sec:kaisse:3}

\is{compound|(}The continuum of domain size we have observed for clitics continues outward into compounds. To review, there are clitics of various sorts: the ‘internal’ type which is considered in the first pass of \isi{stress} assignment, the more familiar type which figures in a second pass that takes into account previously assigned stresses,\is{stress} and a third, free type, which is never counted for stress.\is{stress} My impression is that it is even more uncommon to find compound words -- i.e.\ words made of otherwise prosodically independent elements -- forming a single domain for the first pass of footing. Rather, as we noted earlier, members of a compound are rather like affixal clitics, where a postlexical instantiation of rhythm may adjust the stresses\is{stress} on the basis of the newly available material but does not erase earlier, lexically assigned footing. But this is not always the case.

\ili{Modern Greek} demonstrates the comparatively rare type of compounding where a compound word is stressed\is{stress} as a single domain. It has been shown instrumentally by \citet{athanasopoulou2014} that the well-known traditional description, sharpened in \citet{ralli2013}, of a single, usually antepenultimate primary stress, is correct. The \isi{stress} is placed without regard to where the stresses would fall in the individual members in isolation. A compounding morpheme \textit{{}-o} is often inserted at the end of the first member (replacing any final inflectional ending) and \isi{stress} falls on one of the last three syllables of the whole compound, most commonly the antepenult:

\ea \label{ex:kaisse:16} \ili{Modern Greek} \citep{athanasopoulou2014,ralli2013}
	\ea \label{ex:kaisse:16a} \gll ˈlik \squish-os \\
 	wolf \squish-\textsc{m.sg.nom}\\
	
	\ex \label{ex:kaisse:16b} \gll ˈskil \squish-os\\
	dog \squish-\textsc{m.sg.nom} \\

	\ex \label{ex:kaisse:16c} \gll liˈk \squish-o- {\squish skil} \squish-o\\
	wolf \squish-\textsc{cmpd-} {\squish dog} \squish\textsc{-n.sg.nom}\\
	\glt ‘wolfhound’

	\ex \label{ex:kaisse:16d} \gll ˈriz \squish-i \\
	rice \squish-\textsc{n.sg.nom}\\

	\ex \label{ex:kaisse:16e} \gll ˈɣala\\
	milk \\

	\ex \label{ex:kaisse:16f} \gll riˈz \squish-o- {\squish ɣal} \squish-o\\
	rice \squish\textsc{-cmpd-} {\squish milk} \squish-\textsc{n.sg.nom} \\
	\glt  ‘rice pudding’
	\z
\z

\noindent Tokyo Japanese \citep{poser1990,kubozono2008} has a similar phenomenon whereby a single pitch accent is assigned within a compound based, they argue, on \isi{foot} structure. 

A referee points out that \ili{English} compounds, especially those ending in the element \form{-man,} can sometimes have only a single stress,\is{stress} reminiscent of the Greek case here and of some Macedonian cases to come. While I have not found a published study on this phenomenon, \emph{Language Log} \citep{liberman2015} has an informative post that discusses the unstressed, hence reduced, final vowels in such words as \textit{fireman, clansman, gentleman} versus the full final vowels in words like \textit{caveman, handyman,} and \textit{weatherman}. A lively set of reader responses about which words have a schwa versus the expected compound \isi{stress} and full final vowel seems to lead to the conclusion that the longer a \form{-man} compound has been in \ili{English}, the more likely it is to have a single stress.\is{stress} The situation here may be the demotion of a compound to a single word over time rather than a Greek-like treatment of a compound word as a single \isi{stress} domain, but it falls under the general rubric of familiarity breeding unitary \isi{stress} domains. However, the reader consensus is that there is no simple connection to actual contemporary frequency in the behavior of \form{-man}. \is{compound|)}

I had believed that the Greek case was as far as regular \isi{stress} domains ever extend. However, there is at least one case I am now aware of where the domain extends into some combinations of prosodically independent words -- \ili{Literary Macedonian}.\footnote{ I am very grateful to Ryan Bennett for bringing case to my attention.}

\section[Enlarged stress domains in Macedonian]{Enlarged stress domains in Macedonian}\label{sec:kaisse:4}
\subsection[Overview]{Overview}\label{sec:kaisse:4.1}


We have seen that while stem level affixes\is{affix} virtually always are taken into account in the first pass of \isi{foot} building, many languages have \isi{stress} neutral word-level affixes\is{affix} like \ili{English} \form{\#ly,} \form{\#ness,} that are not part of the \isi{stress} domain. Indeed, \isi{stress} neutrality seems to be a recurrent, if not definitional characteristic of non-cohesiveness. This suggests that rhythm is not easily maintained or phonologized over large domains. Continuing along this cline, we saw that there are clitics of various sorts, some of which are considered in the first pass of \isi{stress} assignment, some in a second pass that takes into account previously assigned stresses,\is{stress} and some of which are never counted for stress. Finally, we saw that \isi{compound} words are only rarely the domain of initial footing. 

A survey of \isi{stress} assignment, beginning with the compendious \citet{hayes1995} confirms the general impression that stress-assigning\is{stress} processes are lexical, not postlexical. But there is an exception, well-known among Slavicists, that in a sense proves the rule. Macedonian as described by \citet{lunt1952} and \citet{koneski1976} (analyzed in generative terms by \citealt{franks1987,franks1989} can build initial feet\is{foot} over certain units larger than the word. The example treated by Lunt and Koneski comes from what is termed ‘\ili{Literary Macedonian}.’ At first I was concerned that this might be an artificial language and that the cases of large domain \isi{stress} could be the creation of scholars. However, \citet[5--6]{lunt1952} and \citet{franks1987} explain that the literary language is simply a pan-dialectal\is{dialects} creation-by-commission that takes features from several western and central \isi{dialects} but does not invent them. The large domain \isi{stress} effects are found in several dialects,\is{dialects} though some of the details differ from spoken dialect to spoken dialect.\is{dialects} 

\ili{Literary Macedonian} has regular antepenultimate stress:\is{stress}

\ea \label{ex:kaisse:17} \ili{Literary Macedonian} \citep{lunt1952,franks1987,franks1989}
	\ea \label{ex:kaisse:17a} \gll  ˈbeseda\\
	lecture\\
	
	\ex \label{ex:kaisse:17b} \gll beˈseda \squish-ta\\
	lecture \squish-\textsc{def}\\
	
	\ex \label{ex:kaisse:17c} \gll voˈdeniʧar\\
	miller\\
	
	\ex \label{ex:kaisse:17d} \gll vodeniˈʧar \squish-i \squish-te\\
	miller \squish-\textsc{pl} \squish\textsc{-def}\\
	\z
\z

\noindent As is usual in such systems, monosyllables\is{syllable} are stressed\is{stress} and disyllables are stressed\is{stress} on the penult. (See \citealt[53]{halle1987} for a full analysis.) But some syntactically complex\is{complexity} strings can be stressed\is{stress} as single units, termed ‘enlarged \isi{stress} domains’. These domains are not, for the most part, unfamiliar to phonologists, as they involve potentially \isi{prosodically deficient} \isi{function words} such as negative or interrogative particles and pronouns plus the following word. However, there are also some\textit{} strings of modifiers plus nouns. There are also polysyllabic propositions, which are cross-linguistically unlikely to be \isi{prosodically deficient} (i.e., to be clitics), yet are stressed\is{stress} as a unit with their objects. I summarize Lunt’s list in \REF{ex:kaisse:18}:

\ea \label{ex:kaisse:18} \ili{Literary Macedonian} enlarged domains \citep[23--25]{lunt1952}
	\ea \label{ex:kaisse:18a} monosyllabic words which have no accent of their own, both pro\isi{clitic} and en\isi{clitic}. The proclitics include personal pronouns, particles and prepositions. The enclitics include definite articles and certain indirect personal pronouns.
	\ex \label{ex:kaisse:18b} the negative particle plus the verb, and any elements that fall between them, such as the present tense of the verb ‘to be,’ even though these normally have their own accent. 
	\ex \label{ex:kaisse:18c} the interrogatives meaning `what,' `how,' `where,' and `how many,' plus the following verb, and any stressless elements between them.
	\ex \label{ex:kaisse:18d} prepositions and their pronominal objects.
	\ex \label{ex:kaisse:18e} prepositions “when used with concrete, spatial meanings” “and in a number of set phrases” when their object is non-definite.
	\ex \label{ex:kaisse:18f} a numeral and the noun it modifies.
	\ex \label{ex:kaisse:18g} some combinations of adjectives and the nouns they modify.
	\z
\z

\noindent Let us begin with \REF{ex:kaisse:18g}, which is the most typologically unusual. We will return to the also-surprising prepositions 18d) in \sectref{sec:kaisse:4.3}. 

\subsection[Prenominal adjectives ]{Prenominal adjectives}\label{sec:kaisse:4.2}

It is worth quoting Lunt’s remarks about the combination of adjectives and nouns in their entirety. 

\begin{indentquote}
The combination of adjective+substantive under a single accent is common to many, but not all, of the central dialects on which the literary language is based, and in any case it is not productive. Such a shift of accent is impossible if either the noun or adjective come from outside the narrow sphere of daily life. Therefore this usage is not recommended. Conversational practice is extremely varied. Place names tend to keep the old [i.e.\ single domain, antepenultimate-EMK] accent… Often used combinations tend to keep the single accent: ‘soured milk’ [yoghurt], ‘dry grapes= raisins’, ‘the left foot,' ‘the lower door,' ‘(he didn’t see a) living soul’. Still one usually hears [the words stressed as separate domains]. Only with the numbers and perhaps a few fixed phrases [dry grapes] is the single \isi{stress} widespread in the speech of Macedonian intellectuals. \citep[24--25]{lunt1952}
\end{indentquote}
What we should note then, is that open class content words are only grouped into a single \isi{stress} domain when the words are frequent (“narrow sphere of everyday life”), especially when such words are also in frequent collocation with one another (“often-used combinations”). And even though this system arose in some of the dialects on which the new literary language was based, it apparently is not easily learned. Lunt here reports that the single domain \isi{stress} on attributive adjective plus noun has not been taken up by the educated speakers who adopted it. 

\citet{franks1987,franks1989} combines Lunt’s examples with those of \citet{koneski1976} and others he and colleagues elicited. Here are several representative ones, including those mentioned in the above quotation.

\ea \label{ex:kaisse:19} \ili{Literary Macedonian} (\citealt{lunt1952}; \citealt[989]{franks1987})
	\ea \label{ex:kaisse:19a} \gll dolnaˈta porta\\
	lower gate\\
	
	\ex \label{ex:kaisse:19b} \gll kiseˈlo mleko\\
	sour milk\\
	
	\glt ‘yoghurt’
	
	\ex \label{ex:kaisse:19c} suˈvo grozje\\
	dry grapes\\
	\glt ‘raisins’

	\ex \label{ex:kaisse:19d} Crˈvena voda\\
	red water\\
	\glt ‘name of a village’

	\ex \label{ex:kaisse:19e} \gll ˈstar  ʧovek\\ 
	old man\\
	
	\ex \label{ex:kaisse:19f} \gll novaˈta kukʲa\\
	new house\\
	\z
\z
The examples in \REF{ex:kaisse:19} are not argued to be compound words by the various sources, probably because they have the regular syntax of noun phrases and many such as (a), (e) and (f) have \isi{compositional} \isi{semantics}; however, they are strings that that are in frequent collocation and may have come to take on a specialized, less predictable meaning. Given the silence of the sources on the question of \isi{compound} versus phrase, perhaps the most noncommittal and appropriate term for them is Erman and Warren’s \citeyearpar{erman2000} “prefabs.” Prefabs are not idioms or \isi{lexicalized} compounds with unpredictable meaning and peculiar syntax but simply common and conventional collocations that can be stored in the \isi{lexicon} while having \isi{compositional} meaning and normal syntax. 

The matter of whether a string of words that occur together frequently is a \isi{compound} or a phrase is a vexed one, even for well-studied languages like \ili{English}. (See \citealt{plag2008} for an overview of the controversy.) 

\subsection{Another compound or adjective-noun case}\label{sec:kaisse:4.3}

Ryan Bennett informs me that some dialects of \ili{Modern Irish} show a similar phenomenon in their adjective plus noun pairs. For instance, Mhac An \citet{fhailigh1968} states for the dialect of Erris that while such a phrase typically has word \isi{stress} on each element, frequent collocations can show a single stress.\is{stress} Thus the infrequent [ˈdrɑx ˈxlɑdəx] `bad shore' has \isi{stress} on each member, but [ˈʃɑn vʲɑn] `old woman,' has only a single stress.\is{stress} Mhac An Fhailigh calls these ‘loose’ versus ‘close’ compounds,\is{compound} rather than phrases vs.\ compounds,\is{compound} but does not give clear criteria for what defines a \isi{compound} versus a phrase. I have not yet found an extended description of this phenomenon in Irish so I mention it  only in passing here. 

The adjective plus noun enlarged domains of Macedonian (and perhaps of Erris Irish) are the exception that proves the rule -- phrases are not normally the domain of \isi{stress} assignment. If what leads to \isi{phonologization} of rhythmic tendencies within words is frequent collocation, it is gratifying that the only extensively described example of an enlarged domain that I could find is one involving frequent collocation of common words, and it is consistent with my hypothesis that the Erris Irish seems to accord with the same generalization. But Lunt tells us that the Macedonian example is hard to learn and is being eliminated over time. Stress\is{stress} rules really don’t like to apply outside a single content word, its affixes\is{affix} (or some of them) and, sometimes, its clitics. 

\subsection[Macedonian prepositions and their objects]{Macedonian prepositions and their objects}\label{sec:kaisse:4.4}

The \isi{stress} behavior of prepositions in Macedonian is also worth looking at in a bit of detail. Since prepositions are the heads of phrases and especially since they can be polysyllabic in Macedonian, one might expect they would be independent prosodic words on their own. Looking at a more familiar case, \ili{English} monosyllabic prepositions like \textit{to} and\textit{ for} are optionally pro\isi{clitic} on their complements,\is{complement} but polysyllabic prepositions like \textit{behind} and \textit{above} are not. But Macedonian prepositions, regardless of their apparent prosodic heft in terms of syllable count, never receive \isi{stress} as a domain on their own. They are, apparently, as \isi{prosodically deficient} as clitics. Lunt devotes several pages (52--65) to the individual behavior of some two dozen prepositions because their \isi{stress} behaviors are somewhat idiosyncratic. The basic generalization is that prepositions have or receive no accent of their own. They either form a single \isi{stress} domain with their nominal or pronominal complement,\is{complement} acting like internal clitics \REF{ex:kaisse:20a}; or, in some cases, they behave like \isi{free clitics}, never receiving a \isi{stress} of their own nor receiving the single \isi{stress} of the enlarged domain \REF{ex:kaisse:20b}. 

\ea \label{ex:kaisse:20} \ili{Literary Macedonian} \citep{franks1989}
	\ea \label{ex:kaisse:20a} \gll okoˈlu selo\\
	near village\\
	\glt  `near the village'

	\ex \label{ex:kaisse:20b} otkaj ˈgradot\\
	from direction\\
	\glt `from the direction (of)'
	\z
\z
Prosodic deficiency in prepositions makes sense from the point of view of the frequency of collocations. Prepositions are closed class items and they require a nominal or pronominal complement.\is{complement} Thus, when they are heard, they are always in collocation with a noun phrase.

\subsection[Summary]{Summary}\label{sec:kaisse:4.5}

As we mentioned earlier, it is helpful to think of the Macedonian collocations of full words that are stressed\is{stress} as a single word as the inverse of the Prosodic Word clitics of \ili{Bilua} in \REF{ex:kaisse:7}, re-illustrated below in \REF{ex:kaisse:21}. PWd clitics are underlyingly \isi{prosodically deficient} but can receive their own \isi{stress} under duress. The Macedonian adjectives are elements that are \textit{not} inherently \isi{prosodically deficient}, even in Macedonian, yet in certain common collocations, they are being treated as a piece of a single \isi{prosodic word}, like internal clitics. This is illustrated in \REF{ex:kaisse:22}. From a cross-linguistic perspective, the Macedonian polysyllabic prepositions are a bit unexpected in being inherently \isi{prosodically deficient}, parallel either to internal \REF{ex:kaisse:20a} or free \REF{ex:kaisse:20b} proclitics. These are illustrated in the second line of \REF{ex:kaisse:22} and in \REF{ex:kaisse:23}.

\ea \label{ex:kaisse:21} Prosodic independence of underlyingly \isi{prosodically deficient} \isi{clitic} \\
	\begin{forest}
		where n children=0{tier=word}{}
	[PPh [PWd [Clitic [{ˈo}, name=o ] ] ] [PWd [Host [{ˈodie}, name=odie ] ] ] ] 
	\node at (odie.base east) [right=1em, anchor=base west] {`he called'};
	\node at (o.base west) [below=3ex, anchor=base west] {3\textsc{sg.m}=};
	\node at (odie.base west) [below=3ex, anchor=base west] {call};
	\end{forest}
\z

\ea \label{ex:kaisse:22} Prosodic dependence of underlyingly prosodically independent words and of ‘internal’ prepositions\\
	\begin{forest}
	where n children=0{tier=word}{}
	[PPh [PWd [? [kiseˈlo, name=kiselo] ] [? [mleko, name=mleko ] ] ] ]
	\node at (mleko.base east) [right=1em, anchor=base west] (gloss) {`sour milk'};
	\node at (kiselo.base west) [below=3ex, anchor=base west] {okoˈlu};
	\node at (mleko.base west) [below=3ex, anchor=base west] {selo};
	\node at (gloss.base west) [below=3ex, anchor=base west] {`near the village'};
	\end{forest}\\
\z

\ea \label{ex:kaisse:23} Macedonian unstressable \isi{preposition} parallel to free pro\isi{clitic}\\
	\begin{forest}
	where n children=0{tier=word}{}
	[PPh [Preposition, name=P] [PWd [N, name = N] ] ]
	\node at (P) [below=1ex] {otkaj};
	\node at (N) [below=1ex] (gradot) {ˈgradot};
	\node at (gradot.base east) [right=1em, anchor=base west] {`from the direction'};
	\end{forest}
\z

\section[Conclusion]{Conclusion}\label{sec:kaisse:5}

Why is \isi{stress} assignment almost always word-bounded? Why is it restricted only to cohering affixes\is{affix} in some languages, and how does it manage to extend outward to clitics in others? The hypothesis offered in this paper and in Kaisse (forthcoming) is that rhythm, like long-distance harmony, is grammaticized when certain syllables\is{syllable} are heard together over and over.\footnote{Although tone rules are more powerful in their ability to escape the word, for reasons that, as far as I know, are not well understood (see footnote 2), word-boundedness is nonetheless certainly robustly attested for tone rules as well as other iterative processes, and I would propose that the same factors of frequent collocation should underlie that domain restriction.} This usually only happens within words, including their affixes,\is{affix} and, occasionally, with words plus slightly more independent closed class items such as clitics. Because word-level affixes\is{affix} have fewer selectional restrictions than stem-level ones, stem-level affixes \is{affix}are the most likely to be included in the initial domain of footing. And since clitics have even fewer selectional restrictions than word-level affixes,\is{affix} they are even less likely to be included. Since there are thousands of independent content words, the collocation of any two is much less likely to have many exemplars. However, both \isi{compound} words (as in \ili{Modern Greek}) and ‘prefabs’ -- collocations of frequent words in frequently heard phrases like ‘old man’ -- are more prone to becoming a single domain than just any sequence of content words. Similarly, collocations of closed class items like prepositions may cross the border and form a single initial \isi{stress} domain. 

There may also be a functional motivation for the prosodic independence of content words. Extending the domain for the initial building of feet\is{foot} would eliminate the demarcative function linguists often ascribe to stress.\is{stress} Because \isi{stress} is culminative, with exactly one primary stressed syllable per content word, it allows us to identify the independent lexical words of a phrase. And because \isi{stress} is often predictably located on an initial, final, or penultimate syllable, it also allows us to identify word boundaries, helping the listener locate the end of one word and the beginning of the next. 

Let us return to a question I raised in \sectref{sec:kaisse:1}. Does the regular, alternating \isi{stress} assignment found within words stem from the grammaticization of rhythmic tendencies in sentences? Or are rhythmic tendencies in sentences the result of “leakage” from regular \isi{phonology} within the word? The counterparts of word-bounded phonological\is{phonology} processes are present and, in the current century, detectable, as statistical, gradient tendencies in larger domains such as compounds\is{compound} and phrases. \citet{shih2016} shows that syntactic choices are gradiently influenced by the rhythmic principles that we see operating obligatorily inside words as alternating \isi{stress} assignment, rhythm rules, \isi{stress} clash avoidance, lapse avoidance, and so on. She argues that \isi{word order} and construction choice can be recruited in response to these phonological pressures from inside the lexical phonology,\is{phonology} unearthing small effects in the choice of optional syntactic variants that favor more eurhythmic outputs. Thus she finds that the genitive constrictions \textit{X’s Y} and\textit{ Y of X} are skewed toward avoidance of \isi{stress} clashes and lapses. Similarly, the dative \isi{alternation} \textit{verb IO DO vs.\ verb DO to IO} trends in a small but discoverable way towards phrases that are more eurhythmic \citep{anttila2010,shih2016}. Along the same lines, \citet{hammond2013} finds that the Brown and the Buckeye corpora contain statistically fewer instances of underlyingly clashing prenominal adjective-noun pairs than would be expected, so that adjectives like \textit{aloof} and \textit{unknown,} with final primary stress,\is{stress} are underrepresented when they are prenominal,\textit{} while adjectives like \textit{happy} and \textit{finite,} with initial primary \isi{stress} are not. Here we run into a chicken and egg problem. The more traditional view in historical \isi{linguistics} is the Neogrammarian one: these pressures are there all along as variation,\is{variation} with the syntax reflecting them before they are phonologized. This is how I envisioned the progression in Kaisse (forthcoming). Along the same lines, \citet{myers2015} note the tendency of learners to extend phenomena like utterance-final devoicing to the word-level. Participants in their artificial language-learning experiments generalized beyond the training data, applying word-final devoicing inside of utterances in novel strings, not just the kind of utterance-final devoicing they were trained on. \citet[399]{myers2015} conclude that “learners are biased towards word-based distributional patterns.” They speculate that this is because we hear and store many more exemplars of words than we do of utterances. \citet{baumann2012} take a similar view of the direction of development of \isi{lexical stress}\is{stress} patterns, investigating through game theoretic simulations the \isi{lexicalization} of stress in \ili{English} bisyllabic words, which can have either initial (\textit{ˈlentil, ˈresearch}\textsubscript{N}) or final (\textit{hoˈtel, reˈsearch}\textit{\textsubscript{V}}) prominence. They argue that “words adopt, on average, those \isi{stress} patterns that produce, on average, the best possible phrasal level rhythm.” But Shih and Martin’s views are equally plausible: the obligatory, regular \isi{stress} system of the language spreads from words into gradient choices in the syntax. I suspect that both directions are operating at the same time, since language change results from a constant push and pull of \isi{variation} and optimization. 

Whatever the direction of change, we have seen that it is rare for rhythmic tendencies beyond the word to become phonologized and to be reflected as \isi{foot} construction processes that take material outside of words as their domain. I have argued that there are simply not enough exemplars of adjacent words heard repeatedly together to result in the reification of such a large domain.

%\section*{Abbreviations}
\section*{Acknowledgements}
I am grateful to Ryan Bennett, Larry Hyman, Andrew Livingston, David Odden, the referees for this paper, and the audience at the Annual Meeting on Phonology 2015 (University of British Columbia).

\printbibliography[heading=subbibliography,notkeyword=this]


\end{document}