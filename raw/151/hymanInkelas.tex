\documentclass[output=paper,
modfonts
]{LSP/langsci}
% \bibliography{localbibliography}

% \usepackage{langsci-optional}
\usepackage{langsci-gb4e}
\usepackage{langsci-lgr}

\usepackage{listings}
\lstset{basicstyle=\ttfamily,tabsize=2,breaklines=true}

%added by author
% \usepackage{tipa}
\usepackage{multirow}
\graphicspath{{figures/}}
\usepackage{langsci-branding}

% 
\newcommand{\sent}{\enumsentence}
\newcommand{\sents}{\eenumsentence}
\let\citeasnoun\citet

\renewcommand{\lsCoverTitleFont}[1]{\sffamily\addfontfeatures{Scale=MatchUppercase}\fontsize{44pt}{16mm}\selectfont #1}
  


\ChapterDOI{10.5281/zenodo.495445}
\title{Multiple exponence in the Lusoga verb stem}

\author{%
Larry M. Hyman\affiliation{University of California, Berkeley}\lastand 
Sharon Inkelas\affiliation{University of California, Berkeley}
\newlineCover\newlineCover 
with Fred Jenga\affiliation{University of Texas, Austin}
}

% \sectionDOI{} %will be filled in at production
% \epigram{}

\abstract{
In this paper we address an unusual pattern of multiple exponence in Lusoga, a Bantu language spoken in Uganda, which bears on the questions of whether affix order is reducible to syntactic structure, whether derivation is always ordered before inflection, and what motivates multiple exponence in the first place. In Lusoga, both derivational and inflectional categories may be multiply exponed. The trigger of multiple exponence is the reciprocal suffix, which optionally triggers the doubling both of preceding derivational suffixes and of following inflectional suffixes. In these cases, each of the doubled affixes appear both before (closer to the root) and after the reciprocal. We attribute this pattern to restructuring, arguing that the inherited Bantu stem consisting of a root + suffixes has been reanalyzed as a compound-like structure with two internal constituents, the second headed by the reciprocal morpheme, each potentially undergoing parallel derivation and inflection.
}

\begin{document}
\maketitle
\lehead{Larry M. Hyman \& Sharon Inkelas with Fred Jenga}
\section{Introduction}

Among the most important contributions of Steve Anderson's realizational
approach to \isi{morphology} have been his early insistence that \isi{morphology} is
not reducible to syntax, his argument that formal theoretical models of
\isi{morphology} need to take different approaches to \isi{derivation} and
\isi{inflection} (``\isi{split morphology}''), his development of morphological rule
ordering as the mechanism of ordering affixes,\is{affix} and his postulation that
redundant (inflectional) morphological \isi{exponence} is actively avoided by
grammars. According to \citet{anderson1992}, derivational\is{derivation} \isi{morphology} takes
place in the lexicon,\is{lexicon} while inflectional\is{inflection} \isi{morphology} takes place in the
syntax. Inflectional\is{inflection} \isi{morphology} is realized by the application of
ordered rules which spell out \isi{features} supplied by syntactic principles
such as agreement. The best evidence that the ordering of inflectional\is{inflection}
affixes\is{affix} cannot simply be read off of syntactic structure comes from
morphotactics which have no analogue or simple justification in syntax.

In this paper we address some rather unusual facts from \ili{Lusoga}, a \ili{Bantu}
language spoken in Uganda, which bear on the questions of whether \isi{affix}
order is reducible to syntactic structure, and whether \isi{derivation} is
always ordered before inflection,\is{inflection} particularly as concerns multiple
\isi{exponence}. In \sectref{sec:hyman:2} we introduce the \ili{Bantu} verb stem and briefly summarize
what has been said about the ordering of derivational suffixes within
it. After reviewing the findings that much of this ordering is strictly
morphotactic, not following from syntactic \isi{scope} or semantic
compositionality, in \sectref{sec:hyman:3} we discuss multiple \isi{exponence} among the \ili{Lusoga}
derivational verb extensions. In \sectref{sec:hyman:4} we then turn to the original
contribution of \ili{Lusoga}, which shows multiple \isi{exponence} of inflectional\is{inflection}
agreement as well as unexpected intermingling of inflectional\is{inflection} and
derivational\is{derivation} affixation. We present our analysis in \sectref{sec:hyman:5} and conclude with
a few final thoughts in \sectref{sec:hyman:6}.

\section{The \ili{Bantu} verb stem}\label{sec:hyman:2}

Most overviews of the \ili{Bantu} verb \isi{stem} assume a structure with an
obligatory verb \isi{root} followed by possible derivational suffixes\is{suffixation}
(``extensions''), and ending with an inflectional final vowel (FV)
morpheme.\is{morpheme} As shown in \REF{ex:hyman:1}, the verb \isi{stem} may in turn be preceded by a
string of inflectional prefixes to form a word:

\ea \label{ex:hyman:1}
\begin{forest}
for tree={fit=band}
[word
	[{inflectional prefixes\\{\footnotesize subject-TAM-object-etc.}} ]
	[stem
		[{root-extensions-FV}, roof]
	]
]
\end{forest}
\z

\noindent While this structure has been reconstructed for Proto-\ili{Bantu} \citep{meeussen1967b}, there is much variation on how the different derivational ``verb
extensions'' are ordered. As shown in \citet{hyman2003}, most \ili{Bantu}
languages show at least a tendency to favor the ``CARP'' template\is{templates} in
\REF{ex:hyman:2}, for which we give reflexes in several \ili{Bantu} languages:
%\newpage

\ea \label{ex:hyman:2}\begin{tabular}[t]{@{}lcccc}
 & C(ausative) & A(pplicative) & R(eciprocal) & P(assive) \\
\ili{Shona} & -is- & -il- & -an- & -w- \\
\ili{Makua} & -ih- & -il- & -an- & -iw- \\
\ili{Chichewa} & -its- & -ir- & -an- & -idw- \\
\ili{Lusoga} & -is- & -ir- & -agan- & -(ib)w- \\
Proto-\ili{Bantu} & *-ɪc- & *-ɪd- & *-an- & *-ɪC-ʊ-
\end{tabular}
\z

\noindent The arguments for recognizing the CARP template\is{templates} include the following:

(i) Certain pairs of co-occurring suffixes\is{suffixation} must appear in a fixed
surface order. This is true of the \isi{causative} + \isi{applicative} (CA), which
can co-occur only in this order, independent of their relative \isi{scope}.
Compare the following two examples from \ili{Chichewa}, in which \isi{applicative}
\form{-ir-} introduces an instrument (\citealt{hyman1992,hyman2003}). Scope (schematized on the right) varies across the two examples,
but surface order is the same:

\ea
\begin{tabular}[t]{@{}llll}
a. & \multicolumn{3}{l}{applicativized \isi{causative}:}\\
& lil-its-ir- & `cause to cry with' & {[} {[} cry {]} -cause-with {]}\\
b. & \multicolumn{3}{l}{causativized \isi{applicative}:}\\
& takas-its-ir- & `cause to stir with' & {[} {[} stir-with {]} -cause {]}
\end{tabular}
\z

(ii) Non-templatic orders which are driven by \isi{scope} can occur with
certain sets of suffixes,\is{suffixation} but are typically limited and show a
``\isi{compositional} asymmetry'': The a-templatic order is restricted to the
reading in which the surface order corresponds to relative \isi{scope}, while
the templatic\is{templates} order can be interpreted with either possible \isi{scope}
relations (e.g.\ reciprocalized \isi{causative}, causativized reciprocal). The
two orders of \isi{causative} and reciprocal (CR, RC) illustrate this property
in \REF{ex:hyman:4}, again from \ili{Chichewa}:

\ea \label{ex:hyman:4}
\begin{tabular}[t]{@{}llll}
	a. & \multicolumn{3}{l}{templatic CR:}\\
	& mang-its-an- & `cause each other to tie' & [ [ tie ] -cause-e.o. ] \\
	& & `cause to tie each other' & [ [ tie-e.o. ] -cause ] \\
	b. & \multicolumn{3}{l}{a-templatic RC:}\\
	& mang-an-its- & `cause to tie each other' & [ [ tie-e.o. ] -cause ] \\
	& & \ljudge{*}`cause each other to tie' & \\
\end{tabular}
\z


\noindent As seen in (4a), the templatic CR order allows either \isi{scope}
interpretation, while the a-templatic RC order in (4b) can only be used
to express a causativized reciprocal. The same facts are observed in
cases where the a-templatic order of \isi{applicative} and reciprocal is
reinforced by an A-B-A ``copied'' sequence:

\ea \label{ex:hyman:5}
\begin{tabular}[t]{@{}llll}
	a. & \multicolumn{3}{l}{templatic AR:}\\
	& mang-ir-an- & `tie (sth.) for each other' & [ [ tie ] -for-e.o. ] \\
	& & `tie each other for (s.o.)' & [ [ tie-e.o. ] -for ] \\
	b. & \multicolumn{3}{l}{a-templatic RAR:}\\
	& mang-an-ir-an- & `tie each other for (s.o.)' & [ [ tie-e.o. ] -for ] \\
	& & \ljudge{*}`tie (sth.) for each other' & \\
\end{tabular}
\z

\noindent Again, as seen in (5a), the templatic AR order can have either \isi{scope}
(reciprocalized \isi{applicative}, applicativized reciprocal), while in (5b)
the a-templatic (RA) + copy (R) sequence can only be \isi{compositional}, hence
an applicativized reciprocal. (We will see such A-B-A sequences in
\ili{Lusoga} in \sectref{sec:hyman:3}.)

(iii) A third argument for CARP is that at least one language,
\ili{Chimwiini},  allows only this order, whereas no \ili{Bantu} language allows verb
extensions to be freely ordered by \isi{scope}. Thus, \citet[28]{abasheikh1978}
writes:

\begin{quote}
``In Chimwi:ni, unlike some other \ili{Bantu} languages, the order of the
extensions is restricted. The following ordering of the extensions
mentioned above is as follows: - Verb Stem\is{stem} - Causative - Applied -
Reciprocal - Passive. It is not possible to put these extensions in any
other order.''
\end{quote}

\noindent Other than \isi{stative} \form{-ik-}, which is more restricted in its
co-occurrence with other suffixes, the above summarizes the general
picture for the productive extensions which are involved in valence.
Even given the occasional variations, e.g.\ \ili{Kitharaka} \citep{muriungi2003},
which reverses the \isi{applicative} and reciprocal, hence the order CRAP, the
evidence points unequivocally to the fact that extension order is
determined primarily by template.\is{templates}

The importance of templaticity\is{templates} is also seen from the existence of one
other valence-related suffix, the short \isi{causative} \form{-i-} (I) which
typically occurs between the reciprocal and passive, hence CARIP (see
also \citealt{bastin1986}, \citealt{good2005}). Although both \form{*-ɪc}- (\textgreater{}
\form{-ɪs-, -is-}) and \form{*-i-} were present in Proto-\ili{Bantu},
\form{*-ɪc-} occurred only in combination with \form{*-i-}, hence
\form{*-ɪc-i-} (cf.\ \citet{bastin1986}. However, as summarized in \REF{ex:hyman:6}, the
current distribution of the two extensions (as well as the \isi{productivity}
of \form{-i-}) varies considerably across different \ili{Bantu} languages
\citep[261]{hyman2003}:

\ea \label{ex:hyman:6} \begin{tabular}[t]{@{}ll@{ : }l}
a. & -is-i- and -i- & \ili{Kinande}, \ili{Luganda}, \ili{Lusoga}\\
b. & -is- only       & \ili{Chichewa}, \ili{Shona}, \ili{Zulu}\\
c. &  -i- only (or almost only) & \ili{Nyamwezi}, \ili{Nyakyusa}\\
\end{tabular}
\z

\noindent The fact that \form{-is-} is the linearly first extension and \form{-i-}
a quite later extension in CARIP, for reasons not motivated by \isi{scope},
presents one more reason to accept a templatic,\is{templates} rather than
\isi{compositional} approach to \ili{Bantu} verb extensions. However, this
conclusion is not without interesting complications. As shown in such
studies as \citet{hyman1994,hyman2003c} and \citet{downing2005j}, \form{-i-} frequently
produces frication of a preceding consonant (a.k.a.\ \ili{Bantu}
spirantization) with potential multiple (cyclic) effects, as seen from
the following examples in which \form{-i-} co-occurs with the
(non-fricativizing) \isi{applicative} \form{-il-} suffix in \REF{ex:hyman:7} from \ili{Cibemba}:

\ea \label{ex:hyman:7}
% Table generated by Excel2LaTeX from sheet 'Sheet1'
\begin{tabular}[t]{@{}lllll}
	lub-  & `be lost' & lil-  & `cry' & UR \\
	lub-i- & `lose' & lil-i- & `make cry' & Morphology  (I)\\
	luf-i- &       & lis-i- &       & Phonology \\
	luf-il-i- & `lose for/at' & lis-il-i- & `make cry for/at' & Morphology (A) \\
	luf-is-i- &       & lis-is-i- &       & Phonology \\
\end{tabular}%
\z


In both outputs, the \isi{applicative} and short \isi{causative} exhibit the
expected surface AI order. However, the frication of \form{lub-} `be
lost' and \form{lil-} `cry' to \form{luf-} and \form{lis-} suggests that
at some level of representation, \form{-i-} is root adjacent. \citet{hyman1994} adopts the above cyclic analysis in which \isi{morphology} and
\isi{phonology} are interleaved (see e.g.\ \citealt{kiparsky1982h}): \form{-i-} combines
with the \isi{root} on the first morphological cycle, triggering a
phonological application of frication on the root.\is{root} When the \isi{applicative}
is added on the next cycle of morphology,\is{morphology} it is ``interfixed'' between
the \isi{root} and the short \isi{causative}, in conformity with the AI order
required by the CARIP template.\is{templates} This example illustrates the surface
nature of the template.\is{templates}

Although it is not part of the CARIP template\is{templates} of valence-changing
derivational suffixes,\is{suffixation} the ``final vowel'' (FV) inflectional ending
position is also templatic in that it is required in most \ili{Bantu}
languages. The set of suffixes that may appear in the FV position
includes \isi{past tense} \form{*-ɪ-}, subjunctive \form{*-ɛ}, and (in most
other contexts) default \form{*-a}. The \form{-ɛ} portion of perfective
\form{*-il-ɛ}, which we will encounter in \sectref{sec:hyman:4}, is also in this slot, even
as the \form{-il-} portion is sometimes considered to be part of the
extension system. The customary reason for assuming bimorphemic
\form{*-il-ɛ} is that the short \isi{causative} (I) and passive (P) occur
between the two parts, hence \form{*-il-i-ɛ} and \form{*-il-ʊ-ɛ} \citep{bastin1983}. If we assumed that \form{*-il-ɛ} was monomorphemic, we would have
to assume some kind of exfixation or \isi{metathesis} of the \isi{causative} and
passive with the {[}il{]} portion of \form{-ilɛ}. There is a second
argument from \ili{Lusoga} (and \ili{Luganda}): Whenever \isi{causative} \form{-i-} or
passive \form{-u-} is present, the FV of the perfective complex is
\form{-a} (see \REF{ex:hyman:9} and note 5 below). We assume that \form{-il-}
occurs in the template ordered before -I-P- with the function of
perfectivizing the extended derivational base so it can accept \form{-ɛ}
or \form{-a} (cf.\ \sectref{sec:hyman:4}).) With this established, we are ready to go on to
the issues that arise in \ili{Lusoga}.

\section{\ili{Lusoga} verb extensions}\label{sec:hyman:3}

As mentioned above, \ili{Lusoga} is spoken in Uganda and is the \ili{Bantu} language
most closely related to \ili{Luganda}. The data cited in this study were
contributed by Fr.\ Fred Jenga, a native speaker from Wairaka (Jinja
District).

\subsection{Long and Short Causatives}

\ili{Lusoga} exhibits the CARIP template\is{templates} discussed above, where  C
refers to the long \isi{causative} \form{-is-} and I refers to the short
\isi{causative} extension \form{-i-}. In fact, \ili{Lusoga} uses both \form{-is-i-}
and \form{-i-} productively and often interchangeably, to express both
causation and instrumentals: \form{-lim-is-i-}, \form{-lim-i-} `cause to
cultivate, cultivate with (sth.)'. As indicated, \form{-is-} cannot
occur without \form{-i-}, while the reverse is possible. The two
\isi{causative} morphs are quite consistent in their CARIP templatic ordering
with respect to the \isi{applicative}, namely, \form{-is-il-i-} (CAI),
\form{-il-i-} (AI), which are realized as \form{-is-iz-} and \form{-iz-}
by the following processes:

\ea\label{ex:hyman:8}
`make cultivate for/at'\\
% Table generated by Excel2LaTeX from sheet 'Sheet1'
\begin{tabular}[t]{@{}lll@{}}
	lim-is-il-i-a & lim-il-i-a & UR \\
	lim-is-iz-i-a & lim-iz-i-a & frication \\
	lim-is-iz-y-a & lim-iz-y-a & gliding \\
	lim-is-iz-a & lim-iz-a & glide-absorption \\
\end{tabular}%
\z

\subsection{Reciprocal + Short Causative}\label{sec:hyman:3.2}

Challenges to the CARIP template\is{templates} arise with the reciprocal suffix,\is{suffixation} which
in \ili{Lusoga} has the long reflex \form{-agan-} of Proto-\ili{Bantu}
\form{*-an-}.\footnote{While it is marginally possible for the
  reciprocal and passive to co-occur in some \ili{Bantu} languages, typically
  with an impersonal subject, e.g.\ \ili{Ndebele} \form{kw-a-sik-w-an-a}
  \textasciitilde{} \form{kw-a-sik-an-w-a} `there was stabbing
  [stabbed] of each other' \citep[66]{sibanda2004}, we have thus far not
  been able to get the two to co-occur in \ili{Lusoga} and will therefore ignore
  the passive extension in what follows.} In the next few subsections we
will consider how the reciprocal combines with its fellow extensions in
the CARIP template,\is{templates} including both ordering flexibility as well as affix
doubling.

We begin with the short \isi{causative}, \form{-i-}. When used alone, without
the long \isi{causative}, we observe flexible ordering possibilities, well
beyond what would be expected from the CARIP template.\is{templates} In these and
subsequent examples, a left bracket indicates the boundary between
inflectional prefixes and the beginning of the verb stem:

\ea \label{ex:hyman:9} `they make each other sew'\\
\begin{tabular}[t]{@{}llll@{}}
a. & bà-{[}tùùng-ágán-y-á & /tùùng-agan-i-a/ & RI \\
b. & bà-{[}tùùnz-ágán-á & /tùùng-i-agan-a/  & IR \\
c. & bà-{[}tùùnz-ágán-y-á & /tùùng-i-agan-i-a/ & IRI \\
\end{tabular}
\z

\noindent In none of (9a-c) does the short \isi{causative} \form{-i-} surface as a
vowel. Nonetheless, its presence is clearly felt. In (9a) it glides,
preceding a following vowel; in (9b) and (9c) it spirantizes the final
\uf{g} of \uf{-tùung-} `sew' to {[}z{]} by a general process in the language,
and is otherwise deleted before the following vowel (of the reciprocal).
The reciprocal suffix \form{-agan-} does not trigger compensatory
lengthening when vowels glide or delete before it, as also seen in the
examples with root-final vowels immediately followed by \form{-agan-},
below:

\ea \label{ex:hyman:10}\begin{tabular}[t]{@{}lllll}
a. & bà-{[}mw-àgán-á & `they shave each other' & /-mo-/ & `shave'\\
b. & bà-{[}ty-àgán-á & `they fear each other' & /-tì-/ & `fear'
\end{tabular}
\z

\noindent Note that (9c) appears to exhibit \emph{two} instances of the short
\isi{causative}: \isi{root} spirantization indicates a following short \isi{causative},
and the glide following the reciprocal also indicates a following short
\isi{causative}. These two surface reflexes of the short \isi{causative} could
result from input suffix doubling, something that is attested elsewhere
in \ili{Lusoga}, as shown in the UR given for (9c). Alternatively, the double
reflex of the short \isi{causative} could be the result of a-templatic IR
order, in which the single short \isi{causative} spirantizes the \isi{root} and then
the reciprocal is interfixed inside of it, an analysis Hyman has
supported for Chibemba \REF{ex:hyman:7}. On this account, short \isi{causative} doubling
(IRI) is illusory. We leave open for now whether the IRI ordering is
required; what is clear is that both RI and IR are possible.

\subsection{Reciprocal + Long Causative}

We turn next to the long \isi{causative} \form{-is-,} which, as we have seen,
must co-occur with the short \isi{causative} \form{-i-}. The most common
realization when reciprocal and long \isi{causative} are both present is for
\form{-agan-} to appear between \form{-is-} and \form{-i-}, as in (11a),
exhibiting the CRI order expected given the CARIP template.\is{templates} However, two
other surface realizations are also possible:\footnote{Since \ili{Lusoga} has
  a \uf{L} vs.\ Ø tone system \citep{hyman2016}, only L(ow) vowels are marked
  with a grave accent in underlying forms. Vowels without an accent
  receive their surface tones by specific rules. H(igh) tone is marked
  with an acute in output forms.}

\ea `they make each other sew' \\
\begin{tabular}{@{}llll@{}}
a. & bà-{[}tùùng-ís-ágán-y-á & /tùùng-is-agan-i-a/ & CRI\\
b. & bà-{[}tùùng-ís-ágán-á & /tùùng-is-i-agan-a/ & CIR\\
c. & bà-{[}tùùng-ágán-ís-á & /tùùng-agan-is-i-a/ & RCI\\
\end{tabular}
\z

\noindent In (11b), \form{-agan-} follows \form{-is-i-} (CIR). In (11c)
\form{-agan-} precedes \form{-is-i-} (RCI). This variation reveals the
same freedom with respect to the ordering of the long \isi{causative} and
reciprocal extensions that we observed in \sectref{sec:hyman:3.2} with respect to the
ordering of the short \isi{causative} and reciprocal extensions.

Note that for phonological reasons, it is impossible to distinguish
between the inputs \form{-is-} and \form{-is-i-} before \form{-agan-.}
The reason is that, sandwiched between long \isi{causative} \form{-is-} and
following vowel-initial \form{-agan-,} short \isi{causative} \form{-i-} would
glide to \form{-y-} and then get absorbed into the preceding {[}s{]},
without leaving a trace. As was seen in \REF{ex:hyman:10}, \isi{compensatory lengthening}
is not expected before \form{-agan-}. However, it can be detected
between \form{-i-} and a FV when an en\isi{clitic} such as locative class 17
\form{=kò} `on it, a little' is added:

\ea `they make each other sew a little'\\
\begin{tabular}[t]{@{}llll@{}}
a. & bà-{[}tùùng-ís-ágán-y-áá =kò & /tùùng-is-i-agan-i-a =kò/ & CIRI + encl\\
b. & bà-{[}tùùng-ís-ágan-á =kò & /tùùng-is-i-agan-a =kò/ & CIR + encl\\
c. & bà-{[}tùùng-ágán-ís-áá =kò & /tùùng-agan-is-i-a =kò/ & RCI + encl\\
\end{tabular}
\z

\noindent In (12a), the final length on \form{-aa} can be directly attributed to
the gliding of the preceding \form{-i-}, since there is a surface {[}y{]}, as
can be the final length in (12c), where the glide has been absorbed into
the preceding {[}s{]}. Although (12b) does not show a surface reflex of
the internal \form{-i-}, we continue to assume that \form{-is-} must be
accompanied by \form{-i-}, as also reconstructed for Proto-\ili{Bantu} \citep{bastin1986}.

While there are three possible realizations when reciprocal
\form{-agan-} combines with the long and short \isi{causative} suffixes, the
preferred surface orders are IRI in (9c), and CRI, in (11a). RI and CRI
are of course predicted straightforwardly from CARIP, while the IR of
IRI is not. Both early placement of C (\form{-is-}) in the CARIP
template\is{templates} and the early realization of the first \form{-i-} of the
hypothesized a-templatic IRI ordering discussed in this section are
consistent with a generalization that \citet[272]{hyman2003} has
characterized as ``causativize first!'': Both \form{-is-} and \form{-i-}
are spelled out early, but later affixation may result in two surface
reflexes of \form{-i-}, either because of interfixation of subsequently
added extension suffixes\is{suffixation} or because of outright morphological \form{-i-}
doubling of the kind seen in the \ili{Chichewa} RAR case illustrated in (5b).

\subsection{Reciprocal + Applicative}

The CARIP template\is{templates} is complicated further by the behavior of the
\isi{applicative}, represented by ``A'' in CARIP. In all three of the following
examples, the transitive verb \form{kùb-} `beat' is both reciprocalized
`beat each other' and applicativized. Applicative \form{-ir-} licenses a
locative argument, expressed by the en\isi{clitic} \form{=wà} `where'. Here
again we observe alternative \isi{affix} orders:

\ea \label{ex:hyman:13}`where do they beat each other?'\\
\begin{tabular}{@{}lll@{}}
a. & bà-{[}kùb-ír-ágán-á =wà & AR  \\
b. & bà-{[}kùb-ágán-ír-á =wà & RA \\
c. & bà-{[}kùb-ír-ágán-ír-á =wà & ARA \\
\end{tabular}
\z

\noindent (13a) represents the expected AR order of CARIP, while the RA order of
(13b) represents an order which is closer to the \isi{compositional}
interpretation of the resulting verb. In (13c) \form{-ir-agan-ir-} has
both the AR and RA orders. The variation between AR, RA and ARA orders
represents a competition between the demand of the CARIP template\is{templates} for
one order and the requirement for affixes\is{affix} to appear in a surface order
that reflects their relative \isi{scope}. The AR order (13a) is templatic; the
RA order in (13b) is a scope-based or \isi{compositional} override. As
suggested by \citet{hyman2003}, ABA \isi{affix doubling} can thus be interpreted
as a means of satisfying both template\is{templates} and compositionality
considerations; if the template\is{templates} wants AR and \isi{scope} wants RA, then ARA,
in some manner, satisfies both.\footnote{The questions in \REF{ex:hyman:13}
  unambiguously ask where the action took place and could therefore be
  answered ``in Jinga'' or ``in the house''. The absence of the
  \isi{applicative} in the corresponding question \form{bà-{[}kùb-agan-a =wà}
  `where do they beat each other?' more narrowly asks what spot or area
  of the body was hit. An appropriate answer would therefore be ``on the
  head''. Finally, the double reflex of \isi{applicative} \form{-ir-} of ARA
  \form{-ir-agan-ir-} in (13c) is reminiscent of the double reflex of
  RAR \form{-an-ir-an-} in \ili{Chichewa} in (5c): the sequence
  \form{-ir-agan-} is licensed by CARIP, while \form{-agan-ir-}
  represents the \isi{scope} override. Concerning ABA suffix\is{suffixation} ordering, one
  might note that \ili{Lusoga} (13c) violates Hyman's \citeyear{hyman2003} generalization,
  observable in \ili{Chichewa} (5c), that AB always reflects the \isi{scope}, while
  BA is templatic.}

An illustrative pair of examples is presented in \REF{ex:hyman:14}, based on the
transitive verb \form{bal-} `count', which is reciprocalized and
applicativized. In this instance, \isi{applicative} \form{-ir-} licenses a
benefactive object:

\ea \label{ex:hyman:14}\begin{tabularx}{\linewidth}[t]{@{}lll>{\raggedright\arraybackslash}X}
a. & bà-bì-{[}bál-ír-ágán-á & AR & `they count them {[}inanimate class 8{]} for each other' \\
b. & bà-tù-{[}bál-ír-ágán-á & AR & `they {[}animate{]} count each other for us' \textasciitilde{} `they count us for each other' \\ 
\end{tabularx}
\z

\noindent By varying the animacy of the object pronouns in \REF{ex:hyman:14}, it is possible to
bias the \isi{scope} interpretation of reciprocal and \isi{applicative} in opposite
directions. In (14a) the object prefix \form{-bi-} `them' (class 8)
represents an inanimate object such as \form{èbitabo} `books' or
\form{èbikopò} `cups', hence animate `each other' (referring back to
\form{bà-} `they') claims the benefactive rôle over inanimate
\form{-bì-} `them'. In this sentence the AR order \form{-ir-agan-}
satisfies both the CARIP template\is{templates} and \isi{scope}: {[}{[}count them{]} for
each other{]}. In (14b), animate first person object \form{-tù-} `us'
preferentially claims the benefactive role over third person
\form{-agan-}, again referring back to \form{bà-} `they'. The
\form{-ir-agan-} order in this sentence is also templatic, but this time
need not reflect \isi{scope}: Although the preferred interpretation is
{[}{[}count each other{]} for us{]}, the other \isi{scope} ({[}{[}count us{]} for each other{]}) is also possible, though pragmatically less
likely. It is thus not surprising that the two alternatives are also
possible in \REF{ex:hyman:15} with the same meaning:

\ea \label{ex:hyman:15}`they {[}animate{]} count each other for us'\\
\begin{tabular}{@{}lll@{}}
a. & bà-tù-{[}bál-ágán-ír-a & RA \\
b. & bà-tù-{[}bál-ír-ágán-ír-á & ARA \\
\end{tabular}
\z

\noindent Parallel to (12b,c), (15a) is a \isi{scope} override, while
\form{-ir-agan-ir-} satisfies both CARIP and \isi{scope} in (15b). What is
surprising is that the same possibilities are at least marginally
acceptable in \REF{ex:hyman:16}, both sentences having the same meaning.

\ea \label{ex:hyman:16} `they count them {[}inanimate cl. 8{]} for each other'\\
\begin{tabular}{@{}lll@{}}
a. & bà-bì-{[}bál-ágán-ír-á & RA \\
b. & \ljudge{??}bà-bì-{[}bál-ír-ágán-ír-á & ARA \\
\end{tabular}
\z

\noindent As in (15a,b), the RA sequence occurs perfectly well in (16a), while the
doubled RAR sequence in (16b) was judged as sounding ``Lugandish,''
perhaps OK to use, but seems a little funny, ``like a foreigner learning
\ili{Lusoga}.'' While we have an explanation for the variation in (13b,c) and
(15a,b), neither CARIP nor \isi{scope} predicts that (16a,b) should be
possible. We thus arrive at a major divergence from the template\is{templates} + \isi{scope}
approach that accounts for the variations considered above in \ili{Lusoga}, as
well as \ili{Chichewa}, Chibemba, and other \ili{Bantu} languages. We now address
why this may be so in the next section.

\section{Inflectional FV suffixes in Lusoga}\label{sec:hyman:4}

In \sectref{sec:hyman:3} we were largely able to account for surface variations in verb
extension order in \ili{Lusoga} by appealing to a tradeoff between the CARIP
template\is{templates} and \isi{scope} considerations: While the templatic CARIP is always
available and represents the default order of affixes,\is{affix} conflicting
orders may be licensed by \isi{scope}, and template-\isi{scope}\is{templates} interactions can
even result in ABA sequences. The one major exception concerns cases of
atemplatic (A)RA \form{-(ir)-agan-ir-,} in which a-templatic RA
\form{-agan-ir-} cannot be said to be a \isi{compositional} override. In this
section we show that this unexpected ordering likely owes its existence
to an optional \isi{restructuring} of reciprocal \form{-agan-}.

To illuminate this hypothesis, we now turn to the interaction of
reciprocal \form{-agan-} with the set of complementary inflectional
``final vowel'' (FV) suffixes.\is{suffixation} Every verb must end in one of these.
While most verbs end in the default FV \form{-a}, specific TAM
categories require one of two other finals, the FV \form{-e} or the FV
complex \form{-ir-e}, which have the following distributions:

\ea\begin{tabularx}{\linewidth}[t]{@{}lll@{ : }>{\raggedright\arraybackslash}X}
 a. & ``irrealis'' & -e & hortative/subjunctive, affirmative imperative singular with an object prefix, affirmative imperative \isi{plural}, negative near future (F1)\\
b. & ``perfective'' & -ir-e & perfect/today past (P1), yesterday past (P2)\\
c. & ``default'' & -a & elsewhere\\
\end{tabularx}
\z

\noindent As summarized in (17a) and exemplified in \REF{ex:hyman:18}, what unifies the uses of
\form{-e} is its use in a subset of irrealis constructions:

\ea \label{ex:hyman:18}\begin{tabularx}{\linewidth}[t]{@{}lr@{}ll>{\raggedright\arraybackslash}X}
a. & bì- & {[}bál-è & `count them!' & (singular imperative with an object prefix; cf.\ bàl-à `count!')\\
b. & mù- & {[}bál-è & `count (pl.)!' & (\isi{plural} imperative)\\
c. & tù- & {[}bál-è & `let's count!'  & (hortative/subjunctive)\\
d. & tì-bá-á- & {[}bál-è & `they will not count' & (negative near future F1) \\
\end{tabularx}
\z

\noindent As per the general \ili{Bantu} stem structure in \REF{ex:hyman:1}, the FV follows the verb
extensions, e.g.\ \isi{applicative} \form{-ir-} in \REF{ex:hyman:19}.

\ea \label{ex:hyman:19}\begin{tabular}[t]{@{}lr@{}ll}
a. & bì-tù- & {[}bàl-ír-è & `count them for us!'\\
b. & mù-tù- & {[}bàl-ír-è & `count (pl.) for us!'\\
c. & tù-bà- & {[}bàl-ír-è & `let's count for them!'\\
d. & tì-bá-á-tú- & {[}bál-ìr-é & `they will not count for us' \\
\end{tabular}
\z

\noindent However, two options are attested when the extension is \form{-agan-}:

\ea \label{ex:hyman:20}\begin{tabular}[t]{@{}lr@{}ll}
a. & mù- & {[}bàl-ágàn-é & `count each other!'\\
 & tù- & {[}bàl-ágàn-é & `let's count each other!'\\
 & tì-bá-á- & {[}bál-àgàn-é & `they will not count each other'\\
b. & mù- & {[}bàl-é-gàn-é & `count (pl.) each other!'\\
 & tù- & {[}bàl-é-gàn-é & `let's count each other!'\\
 & tì-bá-á- & {[}bál-è-gàn-é & `they will not count each other' \\
 \end{tabular}
 \z

\noindent The expected forms are in (20a), where reciprocal \form{-agan-} is
followed by FV \form{-e}. Surprisingly, the alternatives in (20b) show
the FV \form{-e} occurring both before and after the reciprocal. In
these forms we have segmented off the first FV as \form{-e-}, which
means that the reciprocal \isi{allomorph} is \form{-gan-} in this context.
The alternative would be to recognize a reciprocal \isi{allomorph}
\form{-egan-} which is used whenever there is an upcoming FV
\form{-e}.\footnote{It is important to note that \form{-e-gan-} cannot
  be used if the FV is \form{-a}: \form{ò-kú-{[}bál-ágán-á} `to count
  each other', \form{bà-{[}bàl-ágán-á} `they count each other' vs.
  \form{*ò-kú-{[}bál-é-gán-á}, \form{*bà-{[}bàl-é-gán-á.}} We will see
in the discussion of perfective \form{-ir-e} below that the first
\form{-e-} is correctly interpreted as a copy agreeing with the final
\form{-e}.

The same variation obtains when the \isi{applicative} suffix is present:

\ea \label{ex:hyman:21}\begin{tabularx}{\linewidth}[t]{@{}lr@{}ll>{\raggedright\arraybackslash}X@{}}
a. & mù-bì- & {[}bál-ìr-àgàn-é & `count (pl.) them for each other!'\\
 & tù-bì- & {[}bál-ìr-àgàn-é & `let's count them for each other!'\\
 & tì-bá-á-bí- & {[}bál-ìr-àgàn-é & `they will not count them for each other'\\
b. & mù-bì- & {[}bál-ìr-è-gàn-é & `count (pl.) them for each other!'\\
 & tù-bì- & {[}bál-ìr-è-gàn-é & `let's count them for each other!'\\
 & tì-bá-á-bí- & {[}bál-ìr-è-gàn-é & `they will not count them for each other'\\
\end{tabularx}
\z

\noindent In \REF{ex:hyman:21}, the \isi{applicative} \form{-ir-} precedes the reciprocal, showing
the AR order predicted by the CARIP template,\is{templates} but the presence of the FV
between the two in the forms in (21b) is highly unusual from a \ili{Bantu}
point of view.

Exactly the same phenomenon of FV doubling occurs with the perfective
\form{-ir-e} FV complex. As in \ili{Luganda}, \ili{Lusoga} \form{-ir-e} has several
allomorphs. These are presented in \REF{ex:hyman:22} in the form they take prior to
the application of phonological rules:\footnote{As was discussed at the
  end of \sectref{sec:hyman:2} with respect to Proto-\ili{Bantu}, we represent \form{-ir-e} as
  bimorphemic. In \REF{ex:hyman:22} we omit the passive and \isi{causative} forms that
  occur with final \form{-a}, thereby providing even more allomorphs,
  e.g.\ the perfective of the \isi{lexicalized} passive verb \form{\uf{-lùm-u-}}
  `be in pain' is \form{tù-{[}lùm-íír-w-à} `s/he was in pain', while the
  perfect of the \isi{lexicalized} \isi{causative} verb \form{\uf{-tèm-i-}} `blink' is
  \form{tù-{[}tèm-ííz-à} `we blinked', where \form{r → z} is triggered
  by the \isi{causative} suffix \uf{-i-}. Both occur with a long \form{-iir-}
  morph followed by \form{-a}. As seen in these examples, the fact that
  \form{-a} is used with passive \form{-u-} and \isi{causative} \form{-i-}
  provides additional evidence that \form{-ir-} is a separate morpheme
  from \form{-e} or \form{-a}.}

\ea \label{ex:hyman:22}\begin{tabular}[t]{@{}ll@{}l@{}l}
a. & -ir-e & {\ :\ } & after a CV- verb root\\
b. & -i- \dots{} -e & {\ :\ } & when fused (``imbricated'') into a longer verb base\\
c. & -i-e & {\ :\ } & after a labial consonant and /n/\\
d. & -i-e & {\ :\ } & after a fricated consonant {[}s{]} or {[}z{]}, where -i- → y → Ø
\end{tabular}
\z

\noindent The above four allomorphs are illustrated in the perfect/today past (P1)
tense below:

\ea \label{ex:hyman:23}\begin{tabular}[t]{@{}lllll}
a. & /tù-{[}tì-ir-e/ & → & tù-{[}tì-ìr-é & `we feared'\\
b. & /tù-{[}tomer-i-e/ & → & tù-{[}tómèìr-é & `we ran into (s.o./sth.)' \\
c. & /tù-{[}tùm-i-e/ & → & tù-{[}tùm-y-é & `we sent' \\
d. & /tù-{[}bal-i-e/ & → & tù-{[}báz-è & `we counted'
\end{tabular}
\z

\noindent In (23a), the \uf{-ir-e} \isi{allomorph} is realized after the CV verb \uf{-tì-}
`fear'. In (23b), longer verb bases that end in a coronal consonant
undergo \isi{imbrication} whereby \form{-i-} metathesizes with the consonant.
We will see in further examples that the reciprocal \form{-agan-}
extension also undergoes \isi{imbrication} to become \form{-again-e}. In
(23c), the \uf{-i-} of \uf{-i-e} glides to {[}y{]}.\footnote{The following
  \form{-e} actually lengthens, but then is shortened by a rule of final
  vowel shortening (FVS), which converts \form{à-lím-y-èè} to
  \form{à-lím-y-è}. Thus compare the long vowel in
  \form{à-{[}lím-y-\textsuperscript{\small{↓}}éé =kò} which is realized when an
  en\isi{clitic} follows. (\textsuperscript{↓} indicates a downstepped high
  tone).} In (23d), \form{-i-} fricates the preceding \uf{l} to {[}z{]},
yielding the same derivation as in \REF{ex:hyman:8}: \uf{-bal-i-e} → \emph{baz-i-e →
baz-y-e → baz-e}, the {[}y{]} being absorbed into the preceding
fricative.

We will now illustrate each of the above allomorphs of \form{-ir-e} in
\REF{ex:hyman:23} as they are realized with the reciprocal extension. We start with
the reciprocalized version of (23b), which exhibits the imbricating
\form{-i-e} perfective FV \isi{allomorph}. The historically conservative
variant, in which the \isi{root} is followed directly by the reciprocal suffix\is{suffixation}
and then the \form{-i-e} FV, is shown in (24a). However, the preferred
alternative is (24b), in which the perfective \form{-i-e} appears,
imbricated, both immediately following the \isi{root} \textsc{and} immediately
following the reciprocal. URs showing both a single and a doubled FV
complex are provided for each form:

\ea `we ran into each other'\\
\begin{tabular}[t]{@{}lll}
a. & /tù-{[}tomer-agan-i-e/ & tù-{[}tómèr-àgàìn-é \\
b. & /tù-{[}tomer-i-e-agan-i-e/ & tù-{[}tómèìr-è-gàìn-é \\
\end{tabular}
\z

\noindent A parallel situation obtains in \REF{ex:hyman:25}, which corresponds to (23c):

\ea \label{ex:hyman:25}`we sent each other'\\
\begin{tabular}{@{}lll}
a. & /tù-{[}tùm-agan-i-e/ & tù-{[}tùm-àgàìn-é \\
b. & /tù-{[}tùm-i-e-agan-i-e/ & tù-{[}tùm-y-è-gàìn-é \\
\end{tabular}
\z

\noindent Example \REF{ex:hyman:26}, based on (23d), shows similar facts, the main difference
being the frication triggered by \isi{causative} \form{-i-} on the verb \isi{root}
\form{-bal-} `count':

\ea \label{ex:hyman:26}`we counted each other'\\
\begin{tabular}{@{}lll}
a. & /tù-{[}bal-agan-i-e/ & tù-{[}bál-àgàìn-é \\
b. & /tù-{[}bal-i-e-gan-i-e/ & tù-{[}báz-è-gàìn-é \\
\end{tabular}
\z

\noindent Finally, in \REF{ex:hyman:27}, we see a reciprocalized version of the \isi{root} in (23a),
which, on its own, would take the \form{-ir-e} FV \isi{allomorph}. The
historical variant is shown in (27a), but the preferred variant, with
doubled FV, is given in (27b):

\ea \label{ex:hyman:27} `we feared each other'\\
\begin{tabular}{@{}lll}
a. & /tù-{[}tì-agan-i-e/ & tù-ty-àgàìn-é \\
b. & /tù-{[}tì-ir-e-gan-i-e/ & tù-tì-ìr-è-gàìn-é \\
\end{tabular}
\z

\noindent As before there are two instances of the perfective in (27b), vs.\ one in
(27a). In this case of doubling, however, the \isi{allomorphy} of the
perfective is different in the two copies. The first copy of the FV
follows a CV \isi{root} and assumes the expected \form{-ir-e} form; the second
copy, following the longer \form{-agan-}, assumes the imbricating
\form{-i-e} form. The fact that the allomorphs are different suggests
that the two copies are generated independently.

In sum, both the irrealis \form{-e} FV and the perfective FV allomorphs
can appear once in a reciprocalized verb, or twice, with the double
spell-out being clearly preferred. We now turn to an analysis of these
facts in \sectref{sec:hyman:5}.

\section{Towards an analysis}\label{sec:hyman:5}

From the perspective of familiar cross-linguistic principles of \isi{affix}
ordering (derivation\is{derivation} closer to the \isi{root} than inflection;\is{inflection} prohibition on
multiple \isi{exponence}), \ili{Lusoga} presents two interesting puzzles: (i)
derivational and inflectional suffixes\is{suffixation} both double; (ii) when
inflectional suffixes double, they do so on either side of derivation,\is{derivation}
violating the ``\isi{split morphology}''\is{morphology} hypothesis. Thus, in a form like
\form{tù-{[}bàl-é-gàn-é} `let's count each other' from (20b), the
irrealis FV \form{-e} occurs both before and after the derivational
reciprocal suffix \form{-gan-}. While doubling of derivational suffixes\is{suffixation}
has been previously discussed in the \ili{Bantu} literature \citep{hyman2003}, the
doubling of \isi{inflection} has not. This is the final focus of this study.
Given that the doubling occurs in verbs containing the reciprocal suffix
\form{-agan-}, the question we face is what it is about this suffix that
triggers the phenomenon. Why is it only the reciprocal that does this?

Our hypothesis is that the phonological\is{phonology} form of the reciprocal has led
to a \isi{reanalysis} of the internal morphological\is{morphology} structure of the
reciprocalized \ili{Lusoga} verb stem.\is{stem} The reciprocal suffix \form{-agan-} is
the only \ili{Lusoga} derivational suffix which is both disyllabic and
\form{a}-initial. Taken together, these phonological facts are
consistent with a \isi{reanalysis} of the verb \isi{stem} in which the reciprocal
suffix\is{suffixation} is bimorphemic, \form{-a-gan}. Because of its phonological
identity, the \form{-a-} portion became identified with the default FV
\form{-a}. At the same time this permitted the reanalyzed reciprocal
suffix, \form{-gan-}, to conform to the default -CVC- verb root
structure.

As a result of this \isi{reanalysis}, the verb structure in (28a) became
reinterpreted as in (28b), where we use \# to indicate the internal stem
boundary:

\ea\begin{tabular}[t]{@{}llrl}
a. & \emph{Expected (inherited)} & b. &  \emph{Unexpected (innovated)}  \\
& \textsc{Root}-Reciprocal-FV & & \textsc{Root}-FV\#Reciprocal-FV \\
 & \textsc{Root}-agan-a & & \textsc{Root}-a\#gan-a
\end{tabular}
\z

\noindent From this step, the following \isi{analogical} reanalyses follow
straightforwardly, with \isi{allomorph} \isi{variation} in (29b) conditioned by the
phonological\is{phonology} size and shape of the root:\is{root}

\ea\begin{tabular}[t]{@{}llrll}
a. & \emph{Expected (inherited)} & b. & \emph{Unexpected (innovated)} & \\
& \textsc{Root}-agan-e & & \textsc{Root}-e\#gan-e & (irrealis) \\
 & \textsc{Root}-agan-ir-e & & \textsc{Root}-ir-e\#gan-ir-e & (perfective)
 \end{tabular}
 \z

\noindent In (29a) inflectional \form{-e} and \form{-ir-e} are suffixed after
derivational \form{-agan-}. (We show the perfective as \form{-ir-e} in
the above, although its exact \isi{allomorph} will vary, as pointed out in
\REF{ex:hyman:22}.) In (29b) we see the \isi{reanalysis} brought on by \isi{analogy}. As a
result, from the simple right-branching suffixing\is{suffixation} construction in (29a),
reciprocal verb stems\is{stem} became reanalyzed, optionally, as compounding,\is{compound}
with two roots:\is{root} the verb root, and \form{-gan-}. Both are inflectable
(29b), though it is possible also to inflect only the verb \isi{stem} as a
whole (29a).

As indicated, the compounding account allows us to account for the
apparent affixation\is{affix} of the inflectional suffixes\is{suffixation} \form{-e} and
\form{-ir-e} inside of a derivational suffix, the restructured
reciprocal \form{-gan-}. These suffixes\is{suffixation} also potentially precede the
short \isi{causative} \form{-i-}. The inflection of stems containing both
\form{-(a)gan-} and the short \isi{causative} is seen in the following six
alternants, based on the \isi{causative} verb \form{-lùm-i-} `injure', where
-i- glides to {[}y{]} before the following vowel:\footnote{Although the
  verb root \form{-lùm-} means `bite', the \isi{semantics} of the \isi{lexicalized}
  \isi{causative} verb \form{-lùm-ì-} `injure, cause pain' is most clearly
  seen in the corresponding \isi{lexicalized} passive verb \form{-lùm-ù-} `to
  ache, be in pain'.}

\ea \label{ex:hyman:30}`let's injure each other'\\
\begin{tabular}{@{}ll}
a. &  tù-{[}lùm-y-ágàn-é \\
 &	tù-{[}lùm-ágàn-y-é \\
 &	tù-{[}lùm-y-ágàn-y-é  \\
b. &  tù-{[}lùm-y-é-gàn-é  \\
 &	tù-{[}lùm-é-gàn-y-é \\
 & 	tù-{[}lùm-y-é-gàn-y-é \\
\end{tabular}
\z

\noindent The options in (30a) all follow the expected parsing, with \form{-agan-}
treated as a derivational suffix.\is{suffixation} Those in (30b) represent the claimed
\isi{restructuring} in which the FV \form{-e} occurs both before and after
reciprocal \form{-gan-}. In each set, \isi{causative} \form{-i-} appears
immediately after the root in the first example, after the reciprocal in
the second, and both before and after in the third. In the last two
examples of (30b), the first (inflectional) \form{-e} occurs not only
before \form{-gan-}, but also before the (derivational) \isi{causative}
\form{-i-} suffix.\is{suffixation} Parallel cases could be illustrated in which
\form{-i-} combines with the various perfective allomorphs. Our
analysis, which assumes a double or \isi{compound} \isi{stem} structure, each of
which is independently inflected, thus nicely accounts for the above
(and other) cases where the inflectional FV linearly precedes
(restructured) reciprocal \form{-gan-} and potentially other
derivational suffixes.\is{suffixation}

\ea
\begin{tikzpicture}[baseline=(current bounding box.north)]
\coordinate (root) at (3,3);
\coordinate (ol) at (0,0);
\coordinate (ir) at (2.75,0);
\coordinate (il) at (3.25,0);
\coordinate (or) at (6,0);
\draw (ol) -- node[below] {\textsc{Root}-(ext\subit{i})-(FV\subit{j})} (ir);
\draw (il) -- node[below] {\form{gan}-ext\subit{i}-FV\subit{j}} (or);
\draw (ol) -- (root) -- (or);
\draw (ir) -- ($(ol)!(ir)!(root)$);
\draw (il) -- ($(or)!(il)!(root)$);
\end{tikzpicture}
\z

Before moving on to our conclusion, we briefly cite phonological
evidence for our analysis from closely related Lulamogi, which also
optionally realizes the inflectional FV both before and after reciprocal
\form{-gan-} \citep{hymaninpress}. In this language, there are two facts
concerning \isi{vowel length} and (pre-)penultimate position that are relevant
to the analysis of the reciprocal. First, a word-initial V- prefix
lengthens if it is followed by a monosyllabic \isi{stem} (i.e.\ if it is in
penultimate position). This is seen in (32a):

\ea\begin{tabular}[t]{@{}lllll}
a. & /a-{[}ti-â/ & → & àà-{[}ty-â & `s/he fears' \\
b. & /ba-{[}ti-â/ & → & bà-{[}ty-â & `they fear' \\
c. & /a-{[}sék-a/ & → & à-{[}sék-à & `s/he laughs'
\end{tabular}
\z

\noindent As seen in (32b), if the word-initial prefix has the shape CV-, its
vowel doesn't lengthen, while in (32c) \uf{a-} fails to lengthen because it
is in pre-penul\-ti\-mate position. The second length-related phenomenon is
exemplified in \REF{ex:hyman:33}:

\ea \label{ex:hyman:33}\begin{tabular}[t]{@{}lllll}
a. & /tu-{[}á-ti-a/ &  → & tw-áá-{[}ty-à & `we will fear' \\
b. & /tu-á-{[}sek-a/ & → & tw-á-{[}sèk-á & `we will laugh'
\end{tabular}
\z

\noindent In (33a), the prefix sequence \uf{tu-á-} (\textsc{1pl-fut}) undergoes
gliding + \isi{compensatory lengthening} to be realized {[}tw-áá-{]} in
penultimate position. In (33b), on the other hand, the same gliding
process applies, but the result is short {[}tw-á-{]}, since prefixal
vowel sequences are realized short in pre-penultimate position.

A systematic exception to both penultimate prefixal V-lengthening and
pre-penul\-ti\-mate prefixal V+V shortening occurs when reciprocal
\form{-agan-} is suffixed to a monosyllabic verb root:

\ea\begin{tabular}[t]{@{}lll}
a. & àà-{[}ty-ágán-à & `s/he often fears' \\
b. & tw-áá-{[}ty-àgàn-á & `we will fear each other' \\
\end{tabular}
\z

\noindent In (34a), where \form{-agan-} is used as a frequentative suffix, the
initial subject prefix \form{à-} lengthens even though it is in
pre-penultimate position. In (34b), the {[}tw-áá-{]} sequence remains
long even though it too is in pre-penultimate position. Note also that
the first vowel of the \form{-ty-àgàn-} sequence is short, i.e.
\isi{compensatory lengthening} appears not to apply. All of these observations
can be accounted for if we assume the same analysis as in \ili{Lusoga}:

\ea \label{ex:hyman:35}\begin{tabular}[t]{@{}lllll@{}}
a. & /a-ti-a\#gan-a/ & → & àà-ty-ágán-à & `s/he often fears' \\
b. & /tu-á-ti-a\#gan-a/ & → & tw-áá-ty-àgàn-á & `we will fear each other'
\end{tabular}
\z

\noindent In \REF{ex:hyman:35} the \# symbol again represents the boundary between the two
stems. The result in (35a) is that the initial \uf{a-} is now in
penultimate position in the first stem and is thus free to lengthen. In
(35b) the \uf{tu-á-} is now also in penultimate position, and so
{[}tw-áá-{]} fails to shorten. Taken alone, either our \ili{Lusoga} analysis
or this Lulamogi analysis of \citet{hymaninpress} might seem overly
speculative---and especially surprising from a traditional \ili{Bantu}
perspective. However, taken together, the two sets of facts support each
other. In fact, Lulamogi is the only other \ili{Bantu} language we are aware
of that allows the option of spelling out the FV both before and after
the reciprocal extension. Thus compare the following with \ili{Lusoga}
(20a,b):

\ea `let's count each other'\\
\begin{tabular}[t]{@{}ll}
a. & tú-{[}bàl-àgàn-é \\
b. & tú-{[}bàl-è-gàn-é \\
\end{tabular}
\z

\noindent As we stated earlier, we think this reconceptualization is due to the
fact that \form{-agan-} is the only highly productive suffix\is{suffixation} that could
be re-interpreted in the way we have suggested. It is significant that
the historical \ili{Bantu} reciprocal suffix \form{*-an-} often joins with
other suffixes to make a -VCVC- conglomerate (cf.\ \citealt[1289--91]{bostoen2010} for further discussion). In \ili{Lusoga}, Lulamogi, \ili{Luganda}, and
many other \ili{Bantu} languages, \form{*-an-} has joined with an archaic
\form{*-ang-} or \form{*-ag-} extension which likely had an original
pluractional interpretation.\footnote{While the most general realization
  of the reciprocal is \form{-agan-} in \ili{Luganda}, the form is regularly
  \form{-aŋŋan-} after CV verb roots, e.g.\ \form{mw-aŋŋan-} `shave each
  other'. Since \form{-aŋŋan-} derives from \form{*-angan-} via
  Meinhof's Law \citep[192--193]{katamba1991}, this provides evidence
  that the earlier bimorphemic form was likely \form{*-ang-an-} in all
  three closely related languages.} As we have suggested, the shape and
``weightiness'' of the resulting \form{-agan-} has led to multiple
\isi{exponence} and inflectional ``entrapment'' within the derivational
\isi{morphology} of the verb \isi{stem} in \ili{Lusoga} (and Lulamogi). We consider
further implications in the next section.

\section{Conclusion}\label{sec:hyman:6}

In the preceding sections we have documented multiple \isi{exponence} of
derivational suffixes\is{suffixation} (\sectref{sec:hyman:3}) and inflectional suffixes (\sectref{sec:hyman:4}) in \ili{Lusoga}, and
have proposed a \isi{restructuring} analysis of \form{*-agan-} \textgreater{}
\form{-a-gan-} in \sectref{sec:hyman:5} to account for the multiple copies of the
inflectional FV in \form{-e-gan-} sequences. \citet{harris2006}
discuss instances in which grammaticalization of an outer \isi{affix}
``traps'' an inner one, with the result that the two affixes\is{affix} occur in an
unexpected order. Loss of the trapped \isi{affix} is an attested diachronic\is{diachrony}
repair for this ``entrapment'' situation; doubling (by addition of an
outer inflectional affix)\is{affix} is another. \ili{Lusoga}, however, appears to
illustrate \isi{reanalysis} of a different kind, in which an existing \isi{affix} is
reanalyzed as a root,\is{root} and doubling represents agreement in a
compounding-like structure of the sort proposed by \citet{inkelas2005} for \isi{reduplication}, in which doubled morphemes\is{morpheme} can also show
divergent \isi{allomorphy} of the kind displayed by the perfective complex in
\ili{Lusoga}. If correct, the \ili{Lusoga} facts are important both from a
synchronic and diachronic\is{diachrony} point of view. An historical change of *affix\is{affix}
\textgreater{} \isi{root} would contradict the more broadly attested
grammaticalization pattern *root\is{root} \textgreater{} affix\is{affix} (but see \citealt{norde2009}). Synchronically, multiple \isi{exponence} of the inflectional ending is
quite different from the doubling of derivational suffixes.\is{suffixation} While the
latter has been interpreted as the resolution of a template-\isi{scope}\is{templates}
mismatch, perhaps spelled out cyclically, this cannot work for
inflectional doubling. In the examples in \REF{ex:hyman:30} above, it was seen that
the derivational \isi{causative} \form{-i-} can appear once or twice: It can
appear either before the reciprocal (\form{-i-agan-}), after it
(\form{-agan-i-}), or both before or after (\form{-i-agan-i-}). However,
we have thus only shown two possibilities concerning inflectional FVs
such as subjunctive \form{-e}. In \REF{ex:hyman:20}, repeated as (37a,b), we saw that
\form{-e} can appear either after \form{-agan-} or both before and after
\form{-agan-}:

\ea \label{ex:hyman:37}\begin{tabular}[t]{@{}lr@{}ll}
a. & mù- & {[}bàl-ágàn-é & `count each other!' \\
 & tù- &{[}bàl-ágàn-é & `let's count each other!' \\
 & tì-ba-a- &{[}bál-àgàn-é  & `they will not count each other' \\
b. & mù- &{[}bàl-é-gàn-é & `count (pl.) each other!' \\
 & tù- &{[}bàl-é-gàn-é & `let's count each other!' \\
 & tì-bá-á- &{[}bál-è-gàn-á & `they will not count each other'\\
c. & *mù- &{[}bàl-é-gàn-á & `count each other!' \\
 & *tù- &{[}bàl-é-gàn-á & `let's count each other!' \\
 & *tì-bá-á- &{[}bál-è-gàn-á & `they will not count each other'
\end{tabular}
\z

\noindent However, (37c) shows that it is not possible to express the \isi{inflection}
only on the first stem. These facts motivate the compounding\is{compound} structure
we have offered for the \ili{Lusoga} verb stem,\is{stem} and suggest that the second
member, on which inflection is obligatory, is the head, and agreement in
derivational and inflectional properties is optionally enforced,
explaining the presence of duplicate \isi{morphology} on the first
constituent. The structures in \REF{ex:hyman:37} are not amenable to a cyclic
analysis proceeding bottom-up from the verb root.

In \ili{Lusoga}, compounding, derivation and inflection are intermingled in
typologically unusual ways. The complexities of the system -- and of multiple exponence in general \citep[21]{dimensions} --  give credence
to views in which \isi{morphology} is a component of grammar with its own
internal morphotactic organization; it does not mirror syntax directly
and thus cannot be reduced to syntactic principles. This is a result of
which we think Steve would approve.

%\subsection*{Acknowledgments} 

%\subsection*{Abbreviations}

{\sloppy
\printbibliography[heading=subbibliography,notkeyword=this]
}

\end{document}
