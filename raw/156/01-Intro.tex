\documentclass[output=paper]{LSP/langsci} 
\author{Carola Trips\affiliation{Universität Mannheim} 
 \lastand Jaklin Kornfilt\affiliation{Syracuse University}
}
\title{Further insights into phrasal compounding} 
\abstract{\noabstract}

\ChapterDOI{10.5281/zenodo.885107}
\maketitle

\begin{document}

\section{Further insights into phrasal compounding from a typological
  and theoretical perspective} 

%\section*{State of the art}
%\label{state-of-the-art}

This collection of papers on phrasal compounds is part of a bigger project
whose aims are twofold: First, it seeks to broaden the typological
perspective by providing data for as many different languages as
possible to gain a better understanding of the phenomenon
itself. Second, based on these data, which clearly show interaction
between \isi{syntax} and morphology, it aims to discuss theoretical models
which deal with this kind of interaction in different ways. For
example, models like Generative Grammar assume components of grammar and a
clear-cut distinction between the lexicon (often including
morphology) and grammar which mostly stands for the
computational system (\isi{syntax}). Other models, like \isi{construction grammar} do not assume
such components and are rather based on a lexicon including constructs. A comparison of
these models makes it then possible to assess their explanatory power.  

The field of morphology and \isi{syntax} started to acknowledge the existence of
phrasal compounds predominantly in the context of Lexicalist theories because a
number of authors realised that they are not easy to
handle in models of linguistic theory which demarcate the lexicon
(morphology) from \isi{syntax}. Commenting on the difference
between base and derived forms Chomsky said in his ``Remarks on
Nominalization'':
  
\begin{quote}
``However, when the lexicon
is separated from the categorial component of the base and its entries are
analyzed in terms of contextual features, this difficulty disappears.''\\
{\small (\citealp[190]{Chomsky1970})}
\end{quote}

This assumption was dubbed The \isi{Lexicalist Hypothesis} and in the course
of time a number of different versions surfaced. For example,
\citet[8]{Lapointe1980} put forward the Generalized Lexicalist
Hypothesis which stated that ``No syntactic rule can refer to elements of morphological structure.''
\citet[18]{Botha1981} took the perspective from morphology and established The No Phrase
Constraint which postulated that ``Syntactic phrases cannot occur inside of root compounds.''
In 1987, Di Sciullo \& Williams summarised these hypotheses and
constraints in their Atomicity Thesis:

\begin{quote}
``Words are ``atomic'' at the level of phrasal \isi{syntax} and phrasal semantics. The
words have ``features'' or properties, but these features have no structure, and
the relation of these features to the internal composition of the word cannot be
relevant in \isi{syntax} -- this is the thesis of the atomicity of words, or the lexical
integrity hypothesis, or the strong lexicalist hypothesis (as in Lapointe 1980), or
a version of the lexicalist hypothesis of Chomsky (1970), Williams (1978;
1978a), and numerous others.''\\
{\small (Di Sciullo \& Williams 1987:49) \nocite{disciuwill1987}}
\end{quote}

Some of these authors commented on instances of phrasal compounding
like \cite{Botha1980} (who coined the term ``phrasal compounds'') and
\cite{Savini1984} and came to the conclusion that they constitute
negative evidence for these constraints because they clearly showed
interaction between \isi{syntax} and morphology (see the following examples
from Dutch): 

\ea\label{pcbothasavini}
 \ea
 \gll uit-die-bottel-drink alkoholis\\
from-the-bottle-drink alcoholic\\
\glt `alcoholic who drinks straight from the bottle'\\
\glend (Botha 1980:143)
 \ex 
\gll laat-in-die-aand drankie\\
late-in-the-evening drink\\
\glt `drink taken late in the evening'\\
\glend \citep[39]{Savini1984}
\z
\z

In the same vein, \citet{Lieber1988,Lieber1992} put forward examples
for \ili{English} and came to the conclusion that they violate these
constraints, or in more general terms, the \isi{Lexical Integrity Hypothesis}:

% \begin{examplea}
% \label{pclieber}
% \item a slept all day look
% \item a who's the boss wink\\
% (Lieber 1992:11)
% \end{examplea}

\ea\label{pclieber}
 \ea slept all day look
 \ex a who's the boss wink\\
(Lieber 1992:11)
\z
\z

But despite these rather sporadic discussions of the phenomenon no comprehensive study of phrasal compounds in individual languages or cross-linguisti\-cally
existed. 

Fortunately, with a growing interest in compounding as an interface
pheno\-me\-non the situation has changed in the last five years. This can
be seen by the publication of a number of volumes dedicating
themselves explicitly to this type of word formation by providing
detailed accounts of types of compounds across languages (see e.g. \citealp{ScaliseVogelEd2010,LieberStekauer2009}), and
this development brings phrasal compounds now to the fore as well.

% \section*{Information about our project on phrasal compounds and the
%   proposed volume}
% \label{info-pc}

To gain a better understanding of phrasal compounds, in 2013 a
workshop with the topic ``Phrasal compounds from a typological and
theoretical perspective'' brought together scholars who had been
working on (phrasal) compounding in different languages and from
different theoretical perspectives. The outcome of this fruitful
workshop was a collection on the topic which was published in 2015 as a
special edition of STUF \citep{TripsKornfilt2015volume}. The languages
under investigation were \ili{German}, \ili{English}, \ili{Italian}, \ili{Turkish}, some additional
\ili{Turkic} languages and \ili{Greek}. Concerning the approaches chosen for an
analysis of the phenomenon, some authors (Pafel, Göksel)
analysed the phrasal non-head of phrasal compounds in terms of quotes,
quotations, citations whereas authors like Meibauer and Trips favoured
a semantic analysis which attributes an important role to pragmatics
(Trips to some degree in the form of coercion, Meibauer even more so
in terms of pragmatic enrichment). Some of the authors (Bisetto,
Ba\v{g}r{\i}a\c{c}{\i}k \& Angela Ralli) made a distinction between
phrasal compounds that are lexical/morphological and syntactic (either
within one and the same language or comparing languages) and some authors
(Trips \& Kornfilt) found similar semantic restrictions in diverse
languages (\ili{Germanic}, \ili{Turkish}) but also clear structural differences.

Despite this valuable contribution to a phenomenon underrepresented in
current research, it became evident quickly that to come closer to
fulfilling the aims defined above it would be necessary to add further
languages, on the one hand, and to deepen the theoretical discussion,
on the other hand. 

Concerning the typological aspect of (phrasal) compounding we wanted
to include further languages which had not been investigated so far;
especially interesting are, for example, \ili{Slavic} languages, because they
seem to exhibit compounds, but they occur less frequently than for
example in the \ili{Germanic} languages. Another aspect worth investigating
is whether all \ili{Germanic} languages behave in the same way. One very
interesting example is \ili{Icelandic} which has much more
inflectional morphology than the other contemporary \ili{Germanic} languages. Can we then
expect that \ili{Icelandic} behaves differently because of different
morphology? Another, more general question is if languages which are
of the same syntactic type (e.g. SOV) behave in the same way when it
comes to PCs. Would we, for example, expect to find the same patterns
we identified for \ili{German} as an SOV language in another SOV language
like \ili{Japanese}? And what about languages in contact? Would we expect to
find the borrowing of phrasal compounding from a source language to a
recipient language since, after all, they are complex (under the
assumption that contact generally leads to simplification)?

Concerning questions relevant for linguistic theory it would be
worthwhile investigating if there is a correlation between the
morphological and syntactic typology of a language. So for example is
the rightheadedness in morphology (always) related to SOV? Or is a
rich inflectional system a prerequisite for rightheadedness in
morphology? Another interesting question is whether the distinction
between PCs containing a predicate and PCs not containing a predicate
made by Trips related to the property of the nominal head requiring an
argument (or not) as the non-head?  Focussing on the semantic relation
between the non-head and the head in languages like \ili{English} and \ili{German}
we find a tight semantic relation. The same is true for \ili{Turkish}, but
in addition we have selectional restrictions. In contrast, languages
like Sakha (\ili{Turkic}) show looser semantic relations between the
non-head and head. So would we find these similarities/differences in
other language pairs? And, from a more general point of view, are
there theories which model the general properties of phrasal compounds more
adequately than others? And if so, which properties would such a theory
have?

Our interest in these questions made us open up our workshop in 2015
as well as this special issue to papers conceived in different
frameworks. While we cannot answer these evaluative questions yet, we
hope that this collection of case studies conducted in a variety of
models will bring us closer to such answers.

Turning back to structural and semantic properties of phrasal
compounds, questions about the relationship of the head and the
non-head of phrasal compounds were addressed by the presentations at
the workshop and continue to be a focus in the contributions to this
special issue. In many simple as well as phrasal compounds, the
semantics appear to be similar to that of a predicate — argument
relationship, as in \ili{Turkish} and \ili{German}:


\ea\label{ex:t1}
\ili{Turkish}\\
\gll dilbilim 	ö\u{g}renci-si 	\\		
linguistics	student-\textsc{cm}		\\		
\glt `linguistics student'			\\	
\z

\ea\label{ex:t2}
\ili{German}\\
	\gll Linguistikstudent      \\
linguistics-student \\
\glt	`linguistics student'\\
\z

However, especially with respect to quotative phrasal compounds, it is
clear that much more general semantic relationships must be allowed to
hold. This is shown quite clearly in the examples above, especially by
those in (\ref{pclieber}).

Another issue that contributions have focused on is the overt
(syntactic and/or morphological) expression of the head — non-head
relationship in compounds, and in phrasal compounds in particular. As
illustrated in (\ref{ex:t1}), \ili{Turkish} (nominal) compounds have a compound \isi{marker}
(CM) on their head; similar compounds in \ili{German} and \ili{English} don’t have such
a \isi{marker}; \ili{Greek} does, as well as Pharasiot, a variety of Asia Minor
\ili{Greek} influenced by \ili{Turkish}. However, the compound markers of these
\ili{Greek} varieties differ with respect to their sources and their shapes
— one of the issues discussed in one of the contributions in this
volume. Does the presence versus absence of a compound \isi{marker}
determine other properties of a compound, whether phrasal or
otherwise? This is a fascinating question whose answer has been
attempted in the contribution on Pharasiot, but one which can only be
answered more definitively after a good deal of further
cross-linguistic research.

One property which appears to hold
cross-linguistically is adjacency between the head and the non-head in
compounds, setting them apart from phrases:\largerpage

\ea\label{ex:t3}
\ea\label{ex:t3a} 
	\gll 	(\c{c}al{\i}\c{s}kan)	dilbilim		(*\c{c}al{\i}\c{s}kan)	ö\u{g}renci-si	(\ili{Turkish})\\
                (diligent)	linguistics	   diligent	student-\textsc{cm}\\
\glt `diligent linguistics (*diligent) student'


\ex\label{ex:t3b}
		\gll 	der 	(fleißige) 	Linguistik(*fleißige)student		(\ili{German})\\
	the  	diligent		linguistics-diligent-student\\
\glt	`the diligent linguistics (*diligent) student'

\ex\label{ex:t3c}
 	 the 	(diligent) 	linguistics (*diligent)        student		(\ili{English})\\
\z
\z

Thus, adjacency turns out to be a reliable diagnostic device for
distinguishing compounds from phrases. This becomes particularly
important when distinguishing phrasal compounds from phrases, given
that in both, the non-head constituent is phrasal, making the relevant
distinction less clear at first glance.

The non-head in phrasal compounds can be expressed in a variety of different
ways cross-linguistically. Limiting attention to clausal non-heads in
phrasal compounds, we see that in some languages, that constituent can
be either identical to a root clause (and thus a ``quotative''), or it
can show up in the typical shape of an embedded clause in the language
in question. Thus, in \ili{Turkic} languages, embedded clauses typically
show up as gerund-like nominalizations, and this is a pattern that
shows up in \ili{Turkish} phrasal (\isi{non-quotative}) compounds:

\ea\label{ex:t4}
	\gll [en     \c{c}abuk   nas{\i}l   zengin   ol	      -un   -du\u{g}	     -u]	(*ilgin\c{c})	soru	-su\\
	most  fast       how    rich       become \textsc{-pass} \textsc{-fact-nom} \textsc{-3.sg}
        (interesting)   question -\textsc{cm}\\
\glt	`The (interesting) question (of) how one gets rich fastest'
\z

        In \ili{German}, on the other hand, embedded clauses typically show
        up as fully finite, verb-final clauses, in contrast to root
        clauses which are verb-second; not surprisingly, this is a
        pattern that shows up in \ili{German} phrasal (\isi{non-quotative})
        compounds:

\ea\label{ex:t5}
	\gll die (interessante) [wie man am schnellsten reich wird] (*interessante) Frage\\
	the interesting	  how one  the fastest	   rich  gets	interesting    question\\
\glt	`The (interesting) question (of) how one gets rich fastest'
\z

        In quotative phrasal compounds, we find the non-head
        exhibiting the morphosyntactic properties of the root clause;
        this appears to be similar cross-lingui\-sti\-cally, as
        illustrated in (\ref{ex:t6a}) for \ili{Turkish}, \ili{German}, and \ili{English}:
       
\ea\label{ex:t6}
\ea\label{ex:t6a}
\ili{Turkish}\\
\gll [en     \c{c}abuk   nas{\i}l   zengin   ol	      -un -ur]	(*ilgin\c{c}) 	soru-su 	   \\
         most  fast       how    rich       become \textsc{-pass} \textsc{-aorist}	interesting	question-\textsc{cm}\\
\glt	`The ``how does one get rich fastest'' (*interesting) question'
    

\ex\label{ex:t6b}
\ili{German}\\
\gll die [wie wird man am schnellsten reich] (*interessante) Frage \\
        the  how become one the fastest   rich	interesting   question\\
\glt	`The ``how does one get rich fastest'' (*interesting) question'
\z
\z
        Similar semantics can be expressed by phrases rather than
        compounds in many instances. Often, a preposition or a
        postposition is involved in the equivalent phrase, heading the
        clause; this is illustrated in (9) for \ili{Turkish} and \ili{German},
        respectively:

\ea\label{ex:t7}
\ea\label{ex:t7a}
\gll  [en    \c{c}abuk   nas{\i}l   zengin   ol	   -un   -du\u{g}       -u]     hakk{\i}nda (ilgin\c{c})	soru-lar\\
	most  fast       how    rich       become \textsc{-pass} \textsc{-fact-nom} \textsc{-3.sg} about      (interesting)   question-pl\\
\glt	`(interesting) questions about how one gets rich fastest'\\

\ex\label{ex:t7b}
\gll	(interessante) Fragen darüber, [wie man am schnellsten reich wird]\\
	interesting    questions about   how one  the fastest        rich   becomes\\
\glt	`(interesting) questions about how one becomes rich fastest'\\
\z
\z
        The possibility of non-adjacency between the phrasal (here,
        clausal) non-head and the head shows, for both \ili{Turkish} and
        \ili{German}, that these constructions are not compounds, but rather
        phrases. In addition, the fact that in the \ili{Turkish} example
        there is no compound \isi{marker} strengthens this observational
        claim.

We thus see that phrasal compounds exhibit similarities as well as
differences cross-linguistically. Among the latter, we saw that in
\ili{Turkish}, clausal non-heads in phrasal compounds can be nominalized;
this is not an option in \ili{German} and \ili{English} phrasal
compounds. Furthermore, \ili{Turkish} phrasal compounds exhibit a compound
\isi{marker} attached to the head; no such \isi{marker} is ever found in \ili{German} or
\ili{English} phrasal compounds. Future research will, we hope, show
explanations for these differences, beyond those we were able to
sketch in this brief overview.

To come closer to an answer to these questions, a second workshop on
phrasal compounding from a typological and theoretical perspective
took place in 2015 adding further languages and theoretical
models. The present volume is a collection of these contributions. 

\textbf{Kristín Bjarnadóttir} provides a description of compounding in
\ili{Icelandic} in general terms including phrasal compounding as a marked
case. She shows that compounds are extremely productive in
  \ili{Icelandic} and are traditionally grouped into a class containing
  stems and a class containing inflected words (mainly \isi{genitive}) as
  non-heads. Phrasal compounds are also found, and a more common type,
  well established in the vocabulary, can be distinguished from a more
  current, complex type. Interestingly, phrasal compounds may also
  contain a \isi{genitive} non-head and then the question arises how they
  can be distinguished from the genitival non-phrasal compounds. 

  \textbf{Bogdan Szymanek} discusses compounding in \ili{Polish} (and more
  generally, in \ili{Slavic}). He shows that compounds exists in \ili{Polish} but
  that they are much less productive than in \ili{German} or
  \ili{English}. Phrasal compounds do not seem to occur at all, as in all
  the other \ili{Slavic} languages. The author identifies a number of
  reasons why this type of word formation is absent, for example the
  presence of `multi-word units' that are frequently used to express
  complex nominal concepts. 

  \textbf{Alexandra Bagasheva} provides a study of phrasal compounds
  in \ili{Bulgarian}. Despite the fact that this type of compound is said
  not to exist in \ili{Slavic} languages she shows that they do, especially
  so in life style magazines. The author discusses her data in the
  constructionalist framework and proposes the process of ``pattern''
  borrowing from \ili{English} as an explanation of why phrasal compounds
  have started to emerge in \ili{Bulgarian}.

\textbf{Katrin Hein} provides a comprehensive description of phrasal
compounds in \ili{German} and models the different types found in
\isi{construction grammar}. She prefers this model because ``traditional''
generative approaches do not allow for \isi{syntax} in morphology and
because such an approach also fails to explain why a speaker chooses
to use a phrasal compound instead of a nominal compound. Based on a
corpus study she shows that the types of phrasal compounds she found
can all be captured as form-meaning pairings in this model and that
their frequency and productivity justify defining them as
constructions. In addition, she notes that the model serves well to
explain why the second constituent with its semantic properties has to
be seen as the main element and not the first constituent with its
abstract syntactic properties. 


\textbf{Kunio Nishiyama} describes and categorizes various types of
compounds in \ili{Japanese} whose non-heads are phrasal. Nishiyama proposes
that the main criterion of categorization is whether noun
\isi{incorporation} is involved or not in the formation of a given phrasal
compound in \ili{Japanese}. The author is careful not to take a stand on
whether an explicit Baker-type \isi{incorporation} is involved or not, but
the \isi{derivation} he assumes is based on a head-movement approach,
similar to a Baker-type noun \isi{incorporation}, given that the evidence
for noun \isi{incorporation} having taken place is the appearance of
``\isi{modifier} \isi{stranding}'' effects, i.e. that a ``\isi{modifier}'' can be separated
from its head only when it is stranded (as a result of
\isi{incorporation}). If noun \isi{incorporation} has applied in the \isi{derivation} of
a phrasal compound, a further division is made according to whether
the ``predicate'', i.e. the \isi{verbal noun} which is the host of the
incorporated noun, is of \ili{Sino-Japanese} or of native origin. Nishiyama
proposes that there are two licensing conditions for \isi{modifier}
\isi{stranding}: the complement of the \isi{verbal noun}, i.e. the left-hand
element of the compound, should be a \isi{relational noun} or a part of a
cliché.

If no noun \isi{incorporation} is involved, there are four subclasses,
depending on the phrasal non-head: a modifying non-head, a coordinate
structure as a non-head, phrasal non-heads to which prefixes (which
the author is inclined to analyze as proclitics) are attached, and
non-heads to which suffixes (which, again, the author suggests are
enclitics in contemporary \ili{Japanese}) are attached. Nishiyama further
proposes that in phrasal compounds whose non-heads are modifying
structures and coordinate structures, the licensing condition is again
cliché.

\textbf{Metin Ba\v{g}r{\i}a\c{c}{\i}k, Asl{\i} Göksel \& Angela Ralli}
The paper argues that compounding in \ili{Pharasiot Greek} (PhG), an
endangered Asia Minor \ili{Greek} variety, is selectively copied from
\ili{Turkish}, based on differences between PhG compounds and \ili{Hellenic}
compounds on the one hand, and similar properties between PhG
compounds and \ili{Turkish} compounds, on the other: As opposed to various
other \ili{Hellenic} varieties, compounds in PhG are exclusively composed of
two fully inflected nouns, where the non-head, the left-hand
constituent, is marked with one of the two compound markers, -u and
-s, whose shape is conditioned morphologically. According to the
authors, these compound markers have been exapted from the \isi{genitive}
markers in PhG. \ili{Hellenic} compounds have a compound \isi{marker}, as well,
located similarly between the head and the non-head, but it is quite a
different \isi{marker}, with a different history; it has been exapted from
an \ili{Ancient Greek} thematic vowel. Furthermore, in \ili{Hellenic} compounds,
there has to be at least one (uninflected) stem. Similarities between
PhG and \ili{Turkish} compounds include, in addition to certain structural
common features, the provenance of the respective compound markers: in
\ili{Turkish}, the compound \isi{marker} is identical to the third person singular
\isi{possessive} (agreement) \isi{marker} and is placed, just like that agreement
\isi{marker} in \isi{possessive} constructions, on the head, i.e. the rightmost
nominal element. In PhG, the compound \isi{marker} has the shape of a
\isi{genitive} \isi{marker} and is placed, just like the \isi{genitive}, on the
non-head. A parallel is drawn by the authors between the respective
sources of the compound markers in \ili{Turkish} and PhG (i.e. the
\isi{possessive} agreement \isi{marker} in \ili{Turkish}, and the \isi{genitive} \isi{marker} in
PhG), basing their view on a possible identification of the \isi{genitive}
in PhG with the \ili{Turkish} \isi{possessive} agreement \isi{marker} (rather than with
the \isi{genitive} in \ili{Turkish}, which is placed on the non-head in \ili{Turkish}
possessives). The paper discusses, in addition to the similarities
between PhG and \ili{Turkish} compounds, also differences between them:
\ili{Turkish} compounds can have phrasal (and even clausal) non-heads, while
PhG compounds cannot. This difference is attributed mainly to the
location of the compound \isi{marker} within the compound: the PhG compound
\isi{marker}, being a purely morphological \isi{affix}, attaches to stems, similar
to all affixes in the language (as well as in all \ili{Hellenic}
varieties). Therefore, no phrasal constituent can be hosted in the
position to which the compound \isi{marker} attaches. In \ili{Turkish}, on the
other hand, since the compound \isi{marker} attaches to the head, the
non-head can host phrasal constituents. This correlation is claimed to
also hold in \ili{Khalkha} \ili{Mongolian}, an \ili{Altaic} language like \ili{Turkish}, in
which, however, the compound \isi{marker} attaches to the non-head. The
authors claim that similar to PhG, but unlike \ili{Turkish}, phrasal
constituents cannot be hosted in the non-head position in \ili{Mongolian},
thus supporting the correlation they propose between the locus of the
compound \isi{marker} and the availability of phrasal non-heads. Apparent
counterexamples in \ili{Khalkha}, they argue, involve a covert preposition
which assigns \isi{genitive} Case, thus imposing a phrasal, rather than a
compound, structure on these counterexamples.

\textbf{Jürgen Pafel} takes a theoretical stance and discusses the
morphology-\isi{syntax} relation in modular approaches. He analyses phrasal
compounds in the \isi{conversion approach} in a number of languages and
shows, contra the \isi{Lexical Integrity Hypothesis}, that morphology and
\isi{syntax} are separate levels of grammar with separate structures and
distinct properties. Further, the properties of phrasal compounding
speak in favour of a parallel architecture framework, where general
interface relations constrain their properties. 


%\section{Conclusions}

\section*{Acknowledgements}

We would like to thank the participants of the workshop for
interesting talks and fruitful discussions. 

\printbibliography[heading=subbibliography,notkeyword=this]

\end{document}


