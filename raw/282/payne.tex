\documentclass[output=paper,colorlinks,citecolor=brown]{langscibook}
\author{Doris L. Payne\affiliation{University of Oregon \& SIL International}\orcid{}\lastand Alejandra Vidal\affiliation{Universidad Formosa \& CONICET}\orcid{}}
\title{Pilagá determiners and demonstratives: Discourse use and grammaticalisation}
\abstract{Guaykuruan languages of the South American Chaco have rich sets of classifiers and demonstratives, marking deictic, visibility, postural and recognitional meanings. There is lack of consensus in the Guaykuruan literature about determiner and demonstrative elements, even across closely related dialects. This chapter explores them in Pilagá, including their structure, discourse profile, extension into the tense-evidentiality domain, and grammaticalisation as subordinators. Corpus data show that \textit{mʔe} is best viewed as ‘distance neutral’, contrasting with \textit{hoʔ} ‘proximal’ (which also has adverbial uses), \textit{tʃaʔa} ‘distal visible’, \textit{maʕa} ‘unseen’ and \textit{naqae} ‘recognitional’. \textit{Mʔe} is dominantly endophoric and has grammaticalised as a relativiser. The ‘vertical’ classifying determiner \textit{daʔ} has grammaticalised as a general subordinator.}
\IfFileExists{../localcommands.tex}{
  \usepackage{langsci-optional}
\usepackage{langsci-gb4e}
\usepackage{langsci-lgr}

\usepackage{listings}
\lstset{basicstyle=\ttfamily,tabsize=2,breaklines=true}

%added by author
% \usepackage{tipa}
\usepackage{multirow}
\graphicspath{{figures/}}
\usepackage{langsci-branding}

  
\newcommand{\sent}{\enumsentence}
\newcommand{\sents}{\eenumsentence}
\let\citeasnoun\citet

\renewcommand{\lsCoverTitleFont}[1]{\sffamily\addfontfeatures{Scale=MatchUppercase}\fontsize{44pt}{16mm}\selectfont #1}
   
  %% hyphenation points for line breaks
%% Normally, automatic hyphenation in LaTeX is very good
%% If a word is mis-hyphenated, add it to this file
%%
%% add information to TeX file before \begin{document} with:
%% %% hyphenation points for line breaks
%% Normally, automatic hyphenation in LaTeX is very good
%% If a word is mis-hyphenated, add it to this file
%%
%% add information to TeX file before \begin{document} with:
%% %% hyphenation points for line breaks
%% Normally, automatic hyphenation in LaTeX is very good
%% If a word is mis-hyphenated, add it to this file
%%
%% add information to TeX file before \begin{document} with:
%% \include{localhyphenation}
\hyphenation{
affri-ca-te
affri-ca-tes
an-no-tated
com-ple-ments
com-po-si-tio-na-li-ty
non-com-po-si-tio-na-li-ty
Gon-zá-lez
out-side
Ri-chárd
se-man-tics
STREU-SLE
Tie-de-mann
}
\hyphenation{
affri-ca-te
affri-ca-tes
an-no-tated
com-ple-ments
com-po-si-tio-na-li-ty
non-com-po-si-tio-na-li-ty
Gon-zá-lez
out-side
Ri-chárd
se-man-tics
STREU-SLE
Tie-de-mann
}
\hyphenation{
affri-ca-te
affri-ca-tes
an-no-tated
com-ple-ments
com-po-si-tio-na-li-ty
non-com-po-si-tio-na-li-ty
Gon-zá-lez
out-side
Ri-chárd
se-man-tics
STREU-SLE
Tie-de-mann
} 
  \togglepaper[1]%%chapternumber
}{}

\begin{document}
\maketitle 
\shorttitlerunninghead{Pilagá determiners and demonstratives}

%orphan control
%keep examples together
%adjust formatting of tables
%adjust placement of abbreviations table

\section{Introduction}\label{sec:payne:1}

Pilagá (ISO 639-3: plg) is an endangered Guaykuruan language, spoken by around 5,000 people in Formosa, northeastern Argentina, in the South American Gran Chaco.\footnote{Authors are listed in alphabetical order and the chapter is fully co-authored.} Guaykuruan languages have rich sets of determiners. Nearly all nouns require one, so they are ubiquitous in discourse. Pilagá determiners include forms that also function demonstratively, pronominally, adverbially, have incipient nominal tense and evidential functions, and two have developed clausal subordination functions. Some determiners are simple, involving only what we call classifiers (\textsc{clf}),\footnote{Classifiers are usually pro- or enclitics. We write them with the clitic boundary = as part of demonstrative constructions, but as separate orthographic words before a noun in accord with Pilagá orthographic practice. Some nouns with possessor prefixes lack determiners, though they can co-occur.} highlighted in \REF{ex:payne:1}; others are demonstrative word-level constructions, highlighted in \REF{ex:payne:2}.\footnote{Examples use a modified IPA representation with <y> for IPA /j/, <ñ> for /ɲ/, <č> for /tʃ/, <\textrm{b̶}> for the bilabial fricative allophone of /w/; <λ> represents a palatal lateral sonorant. These are adaptations to the practical orthography.}

\ea\label{ex:payne:1} (190Verbos2 165)\footnotemark{}\\
\gll  s-anem  \textbf{he-ʔn}  nsedaʕanaʕat  \textbf{daʔ}  ya-qaya-di-pi\\
 \textsc{a1}-give \textsc{m-clf}:near  pole \textsc{clf:ver} \textsc{pos1}-brother-\textsc{pauc-col}\\
\glt ‘I give the pole to my brothers.’ 
\z
\footnotetext{All data were collected by Alejandra Vidal with Pilagá native speakers in Formosa, between 1988 and the present. Data citations like “190Verbos2 165” refer to line “165” in file or text number “190” in our Pilagá FLEx database. The database contains narrative and expository texts, and some elicited material. Examples with no citation are elicited and not in the database.}

\ea\label{ex:payne:2} \citep[123]{Vidal2001}\\
\gll  \textbf{ha-da=ča-lo}  yawo-ʔ\\
     \textsc{f-clf:ver=dem1:dist.vis-pl} woman-\textsc{pauc}\\
\glt ‘those women (standing)’ 
\z

This chapter addresses the morphosyntax, meaning, and discourse uses of simple and complex Pilagá determiners and demonstratives.\footnote{Previous studies of determiners and demonstratives in Guaykuruan discourse have focused on Toba, especially \citet{Carpio2012} on Western Toba (which Vidal assesses as very close to Pilagá), \citet{González2015} on an eastern variety of Toba, and \citet{MessineoCúneo2019} on Toba generally.}  The study is based on a corpus of over 70 texts plus elicited data. \sectref{sec:payne:2} discusses definitions and terminology, and presents the three paradigms of key morphemes that figure in determiner and demonstrative constructions. Sections 3 through 6 focus on the morphosyntax and semantics of the constructions, supporting the claim that three distinct paradigms of key morphemes are involved. \sectref{sec:payne:7} discusses extensions into tense-evidentiality, and grammaticalisation of the ‘neutral’ demonstrative root \textit{mʔe} as a relativiser and of the ‘vertical’ \textsc{clf} \textit{daʔ} as a more general subordinator. Throughout, issues of semantics and function are addressed, including how interaction among morphemes may affect interpretation. A conclusion is in \sectref{sec:payne:8}.

\section{Classifier and demonstrative roots}\label{sec:payne:2}

There is lack of consensus in the Guaykuruan literature about what are called “demonstratives”. The issues concern terminology for cognate forms and the inventory of relevant elements and complex structures, which may vary by dialect and language \citep{Vidal1997,Vidal2001,Carpio2012,González2015,MessineoEtAl2016,Cúneo2016}. We thus first clarify key terms as used in this work.

\begin{itemize}
\item \textsc{classifier} (\textsc{clf}): Any of the six deictic or posture/shape clitics in \tabref{tab:payne:1}.
\item \textsc{demonstrative} \textsc{root} (\textsc{dem1}, \textsc{dem2}): Any of the morphemes in \tabref{tab:payne:2} and \tabref{tab:payne:3}, which have deictic, pointing-out, or joint-attention functions.
\item \textsc{demonstrative} \textsc{construction} \textsc{(dem)}: A word-level construction that contains a deictic or joint-attention establishing root \textbf{other} \textbf{than} just a classifier. All but one demonstrative constructions contain a classifier; may function adnominally, pronominally, and in one case adverbially; and may be endophoric or exophoric to the discourse.\footnote{For Pilagá, we use “demonstrative root” to designate a root from \tabref{tab:payne:2} or \tabref{tab:payne:3}, and “demonstrative” to designate a demonstrative construction.}
\item \textsc{determiner} \textsc{(det)}: Any classifier or demonstrative construction \textbf{when} \textbf{functioning} \textbf{adnominally}. All determiners syntactically allow the noun (phrase) they accompany to function as a syntactic argument and/or as a referring expression in discourse. They may or may not be deictic.\footnote{\citet[57]{Diessel1999Book} uses “demonstrative determiner” for adnominal demonstratives not found in other syntactic contexts. In Pilagá, all demonstratives that function adnominally can also function pronominally. \textsc{clf}s also (but rarely) function pronominally.}
\end{itemize}

As the first two bullet points above suggest, we distinguish what we call classifiers (\textsc{clf}s) from two sets of demonstrative roots (\tabref{tab:payne:1}-\tabref{tab:payne:3}). Simple \textsc{clf}s are the default determiner form in discourse (\sectref{sec:payne:3}). Aside from a demonstrative construction with adverbial function (\sectref{sec:payne:4}), all demonstrative constructions include a \textsc{clf}. What we call \textsc{dem1} roots are preceded by a \textsc{clf} (\sectref{sec:payne:5}), while the \textsc{dem2} root is followed by a \textsc{clf} (\sectref{sec:payne:6}).

All the sets in \tabref{tab:payne:1}-\tabref{tab:payne:3} have some deictic semantics, and \textsc{clf}s and \textsc{dem1} roots include visibility contrasts. The deictic overlaps might lead one to consider all three sets to be demonstrative morphemes. But there are functional reasons to distinguish classifiers from demonstrative roots. It would be unusual for a language to require every nominal to have a demonstrative, and this is one reason not to consider the default and ubiquitous classifier determiners to be demonstratives. The two sets of demonstrative roots are stronger orienting devices, pointing the hearer’s attention to a participant, place, or time, usually (but not always) via deixis or visibility features.

\tabref{tab:payne:1} presents the Pilagá singular classifiers. Underlyingly they contain glottals, but they often surface with weak to no glottalisation. C\textsc{lf}s with /a/ and /i/ often undergo vowel harmony alternation to /o/. For instance, \textit{diʔ} has allomorphs \textit{dyo} and \textit{doʔ}.\footnote{Also note that \textit{do}(\textit{ʔ}) is a dialect variant of \textit{daʔ} ‘vertical; abstract’.} \textit{Soʔ} may undergo vowel harmony to \textit{saʔ}, and sometimes we find \textit{esoʔ}. We write the variations where they surface. The plural counterparts lengthen the vowel \citep{Vidal2001}, though this is optional (especially when there is a plural affix on a noun).

%Adjust table formatting
\begin{table}
\begin{tabularx}{\textwidth}{lQlQ}
\lsptoprule
\multicolumn{2}{c}{{\bfseries Deictic direction/Visibility}} & \multicolumn{2}{c}{{\bfseries Posture/Shape}}\\
\midrule
{\textit{naʔ}} & ‘near’; ‘coming’ to the reference point & \textit{diʔ} & {‘horizontally extended’ (line or plane)} \\
\tablevspace
{\textit{soʔ}} & {‘far’; ‘departing’ from the reference point; ‘past’} & \textit{daʔ} & {‘vertically extended’; ‘abstract’}\\
\tablevspace
{\textit{gaʔ}} & {‘unseen, absent’; ‘unknown, generic, non-referential’; ‘irrealis/future’} & \textit{ñiʔ} & {‘non-extended, bunched up, sitting’}\\
\lspbottomrule
\end{tabularx}
\caption{Pilagá singular classifier (\textsc{clf}) clitics}
\label{tab:payne:1}
\end{table}

As \tabref{tab:payne:1} shows, the \textsc{clf} paradigm has two semantic subsets (they are not contrastive in morphosyntactic distribution). Guaykuruan cognates of these morphemes have fascinated scholars due to the relatively unusual combination of their meanings, both basic and metaphorical \citep{Klein1979,MessineoEtAl2016}. Relative to the physical world, the first semantic subset has deictic and/or visibility features, and in some contexts allows inference of motion semantics. The deictic meanings fit with \citegen{Diessel1999Book} definition of demonstrative elements, which leads some researchers to refer to all \tabref{tab:payne:1} morphemes as “demonstratives” for related languages \citep{MessineoEtAl2016}.\footnote{\citet{MessineoEtAl2016} do not mention cognates of the demonstrative roots we present in \tabref{tab:payne:2} and \tabref{tab:payne:3}.} \citet{Carpio2012} refers to the Western Toba cognates as “demonstrative roots” (she also identifies a distinct set of morphemes – cognate with what we call “demonstrative roots” – to which the \textsc{clf} cognates can attach). \citet[153]{González2015} rejects calling the cognate Eastern Toba morphemes “classifiers” because, though they communicate a certain kind of nominal classification, a given nominal can occur with one or another according to the speaker’s perspective. (See also discussion in \citet[145]{Messineo2003}, who nevertheless uses the term “nominal classifier”.) However, in many classifier languages, classifier choice can be sensitive to varying speaker conceptions of the configuration of a concept – e.g. in Yagua, ‘water’ can be conceptualised as long+horizontal or as round; ‘wood’ can be conceptualised as upright or as short+small \citep{Payne1986}.

The second semantic subset most concretely indicates salient shape or postural configuration of a referent. For instance, \textit{daʔ} in its concrete sense indicates vertically extended items like upright trees and people. It is also used for abstract nouns and has grammaticalised as a general subordinator (\sectref{sec:payne:7.3}). The shape semantics lead \citet{Klein1979}, \citet{Vidal1997,Vidal2001}, and \citet{MessineoCúneo2019} to call all six “classifiers”. Our primary point here is not to argue that these six morphemes are (not) classifying or are (not) deictic in nature as the paradigm clearly has both types of semantic features. Rather, we wish to clarify that these comprise a distinct paradigm from what we call “demonstrative roots”, to which we now turn.

Pilagá demonstrative roots divide into two paradigms based on how they combine with classifiers: \textsc{dem1} roots follow \textsc{clf}s, but the \textsc{dem2} root precedes \textsc{clf}s. The \textsc{dem1} roots are deictic, indicating ‘proximal’, possibly ‘medial’, and ‘distal’ distinctions relative to a reference point, as well as visibility contrasts. To give an initial sense of their differing discourse profiles, \tabref{tab:payne:2} and \tabref{tab:payne:3} show the number of instances of the demonstrative roots in the corpus (whether as part of complex demonstrative constructions or not).

%adjust table formatting
\begin{table}
\begin{tabularx}{\textwidth}{lQr}
\lsptoprule
{\bfseries \textsc{dem1} Root} & {\bfseries Major Senses} & {\bfseries Instances in Corpus}\\
\midrule
{\textit{hoʔ}} & {exophoric adverbial; ‘proximal’ (\textsc{prox}); current discourse topic} & 363\\
\tablevspace
{\textit{mʔe}} & exophoric ‘medial visible’; endophoric ‘neutral’ (\textsc{neut}) & 241\\
\tablevspace
{\textit{čaʔa}} & {‘distal visible’ (\textsc{dist.vis})} & 18\\
\tablevspace
{\textit{maʕa}} & {‘unseen (\textsc{nvis})’; ‘inferential, uncertain’} & 6\\
\lspbottomrule
\end{tabularx}
\caption{Pilagá deictic and visibility demonstrative roots (\textsc{dem1})}
\label{tab:payne:2}
\end{table}

The \textsc{dem1} roots can function exophorically. This is most common for \textit{hoʔ} ‘proximal’, which is dominantly exophoric and neutral for visibility, and for \textit{čaʔa} (variant \textit{čʔa}) ‘distal visible’.\footnote{For Western Toba, \citet[47-49]{Carpio2012} identifies -\textit{ha} ‘non-visible exophoric’ as a suffix on what we call \textsc{clf}s. A possibly cognate Pilagá form surfaces in the frozen combination \textit{hoʔ daha} ‘there, a place very far away’. We do not treat \textit{ha} further here but note its analogous position to \textit{hoʔ}. \textit{Čaʔa} (variant \textit{čʔa}) comes from a motion verb and sometimes carries ‘itive’ and ‘ventive’ directionals that are characteristic of verbs, as in \REF{ex:payne:38} and \REF{ex:payne:39}.} The exophoric uses optionally occur with pointing gestures. In contrastive elicitation contexts, \textit{mʔe} can indicate exophoric referents medially distant between those marked with \textit{hoʔ} and \textit{čaʔa}. \textit{Maʕa} refers to something unseen; the referent may be inferred or something about it is uncertain. 

Demonstratives with \textit{hoʔ} can also function endophorically to refer to discourse-anaphoric referents. Endophoric use is also possible for \textit{čaʔa}, but the referent is not considered close to the speaker or reference point. \textit{Mʔe} is primarily endophoric, either anaphoric or cataphoric. Especially in its endophoric distribution, \textit{mʔe} is best viewed as ‘distance neutral’ \citep[211]{Himmelmann1996} since it can occur with all the \textsc{clf}s to mark referents as ‘proximal/(coming) in the visual field’, ‘distal/(going) out of the visual field’, ‘never seen’, or depending on the particular classifier to refer to ‘horizontal’, ‘vertical’, or ‘bunched up’ referents. It is not accompanied by pointing gestures. It has also grammaticalised as a relativiser (\sectref{sec:payne:7.4}).

The \textsc{dem2} set contains just the root \textit{naqae} (\tabref{tab:payne:3}). It takes \textsc{clf}s as enclitics, unlike the \textsc{dem1} roots which take \textsc{clf}s as proclitics. We analyse it as a ‘recognitional‘ (\textsc{rcg}) demonstrative root, but in some contexts it may function more emphatically or mark unexpected information (\sectref{sec:payne:6}).

%adjust table formatting
\begin{table}
\begin{tabularx}{\textwidth}{lQr}
\lsptoprule
{\bfseries \textsc{dem2} root} & {\bfseries Major sense} & {\bfseries Instances in corpus}\\
\midrule
{\textit{Naqae}} & {‘this/that familiar but previously inactive; recognitional (\textsc{rcg})’} & 84\\
\lspbottomrule
\end{tabularx}
\caption{Pilagá recognitional demonstrative root (\textsc{dem2:rcg})}
\label{tab:payne:3}
\end{table}

Having now introduced the \textsc{clf}s and demonstrative roots in \tabref{tab:payne:1}-\tabref{tab:payne:3}, \sectref{sec:payne:3}-\sectref{sec:payne:6} will discuss the morphosyntax and functions of four constructions that employ them. In anticipation, \tabref{tab:payne:4} overviews the grammatical functions of the basic classifier (\textsc{bclf}), simple demonstrative (\textsc{sdem}), deictic demonstrative (\textsc{ddem}), and recognitional demonstrative (\textsc{rdem}) constructions. A dash in \tabref{tab:payne:4} indicates the morpheme in the first column lacks the adverbial function.

%adjust table formatting
\begin{table}
\begin{tabularx}{\textwidth}{Qllll}
\lsptoprule
& {\bfseries Adverbial} & {\bfseries Pronominal} & {\bfseries Determiner} & {\bfseries Other}\\
\midrule 
\textit{hoʔ} \textsc{dem1} ‘proximal/ unspecified’ & \textsc{sdem}, \textsc{ddem} & \textsc{sdem} (rare), \textsc{ddem} & \textsc{ddem} & \\
\tablevspace
\textit{mʔe} \textsc{dem1} ‘medial visible’; ‘neutral’ & \textsc{–} & \textsc{ddem} & \textsc{ddem} & relativiser\\
\tablevspace
\textit{čaʔa} \textsc{dem1} ‘distal visible’ & \textsc{–} & \textsc{ddem} & \textsc{ddem} & \\
\tablevspace
\textit{maʕa} \textsc{dem1} ‘non-visible’ & \textsc{–} & \textsc{ddem} & \textsc{ddem}   & \\
\tablevspace
\textit{naqa}(\textit{e}) \textsc{dem2} ‘recognitional’ & \textsc{–} & \textsc{ddem} & \textsc{rdem} & \\
\tablevspace
\textit{daʔ} \textsc{clf} ‘vertical; abstract’ & \textsc{–} & \textsc{bclf} & \textsc{bclf} & subordinator\\
Other \textsc{clf}s & \textsc{–} & \textsc{bclf} (rare) & \textsc{bclf} & \\
\lspbottomrule
\end{tabularx}
\caption{Syntactic distribution of basic classifier and demonstrative constructions}
\label{tab:payne:4}
\end{table}

Across languages, demonstrative morphemes may have differing syntactic functions (\citealt[4]{Diessel1999Book}; \citealt{Krasnoukhova2012}). For example, in one language a single paradigm might function as demonstrative pronouns for participants or objectified concepts, as adnominal demonstratives, and as demonstrative adverbs for location or time. The Pilagá morpheme \textit{hoʔ} distributes like this, though the particular construction it appears in (\textsc{sdem} or \textsc{ddem}) matters for syntactic function. In another language, a given demonstrative paradigm may have only a subset of functions. In English, for instance, \textit{here/there} are adverbial demonstrative proforms for locations,\footnote{This sets aside dialectal uses like \textit{this here dog}, where \textit{here} doubles \textit{this} as a proximal determiner.} and \textit{now/then} are adverbial demonstrative proforms for time. But \textit{this/that/these/those} function as both demonstrative participant pronouns and as demonstrative determiners.\footnote{\citet[90]{Diessel1999Book} also discusses presentational (what some call “predicational” or “identificational”) and other functions of demonstratives.} The \textsc{dem1} roots and the \textsc{dem2} root distribute like these last English morphemes when in particular constructions. Classifiers in the \textsc{bclf} construction function primarily as determiners, and more rarely as pronouns.

\section{Basic classifier construction}\label{sec:payne:3}

In Pilagá discourse, determiners most frequently have the structure in \REF{ex:payne:3}. We call this the basic classifier construction (\textsc{bclf}). The only required element is one of the six clitics in \tabref{tab:payne:1}, or a plural counterpart. \textsc{bclf}s functioning as determiners are highlighted in \REF{ex:payne:1} above and in the examples below.

\ea\label{ex:payne:3} {Basic classifier construction (\textsc{bclf})}
\glt \textsc{(gender-)classifier}
\z

The \textsc{bclf} construction is illustrated in \REF{ex:payne:4}–\REF{ex:payne:6} with the posture/shape \textsc{clf}s.

\ea\label{ex:payne:4} (028SanMartin2 1.5)\\
\gll  \textbf{diʔ}  naʔa-ik\\
\textsc{clf:hor}  road-\textsc{m}\\
\glt ‘road’ 
\z

\ea\label{ex:payne:5} (190Verbos2 565)\\
\gll  da=mʔe  yi-la-ʔa  \textbf{daʔ}  epaq\\
\textsc{clf:ver=dem1:neut} \textsc{a3}-find-\textsc{obj.sg} \textsc{clf:ver} tree\\
\glt ‘She/He sees a tree.’ 
\z

\ea\label{ex:payne:6} (028SanMartin2 1.2)\\
\gll  se-b̶ide-wʔo  \textbf{ñiʔ}  tamnaʕa-ki\\
\textsc{a1}-arrive-\textsc{loc}:outward  \textsc{clf:no.ext} religion-place\\
\glt ‘I arrive at the church.’ 
\z

Examples \REF{ex:payne:7}–\REF{ex:payne:8} illustrate vowel-lengthened plural \textsc{clf}s. \textit{Saaʔ} occurs in \REF{ex:payne:8}, rather than \textit{sooʔ}, due to vowel harmony with the following noun. Recall that the plural \textsc{clf} forms are optional (especially when the noun is marked for plurality).

\ea\label{ex:payne:7}
\gll  \textbf{naaʔ}  y-ʔaiʔte\\
\textsc{clf}:near.\textsc{pl}  \textsc{pos1}-eyes\\
\glt ‘my eyes’
\z

\ea\label{ex:payne:8}  (008ZorroPato 1)\\
\gll  qančʔe  yi-laeyʔa-lo  \textbf{saaʔ}  taʕañi  kʼoqte-l\\
then \textsc{3}-see.ahead-\textsc{pl}  \textsc{clf}:far.\textsc{pl} rosy.billed.duck  offspring-\textsc{pl}\\
\glt ‘He suddenly saw some rosy-billed ducklings.’
\z

Examples \REF{ex:payne:9}–\REF{ex:payne:11} show the \textsc{bclf} with gender prefixes. Masculine is usually unmarked (formally and functionally), but an overt prefix \textit{ho-/(h)e}- can be added for clarity.

\ea\label{ex:payne:9}
\gll  \textbf{ho-gaʔ}  emek\\
\textsc{m-clf}:absent  house\\
\glt ‘that (unknown) house’
\z

\ea\label{ex:payne:10} (011Kitilipi 1.20)\\
\gll  qačʔe  qo-i-law-lo    \textbf{ho-ʔn}  l-ʔaiʔte  ekey qo-d-ʔoya-lo    \textbf{soʔ}  l-ʔaiʔte\\
\textsc{conj}   \textsc{sbj.indf-a3}-see-\textsc{pl}  \textsc{m-clf}:near  \textsc{pos3}-eye.\textsc{pl} \textsc{intj} \textsc{sbj.indf-a3}-fear-\textsc{pl}  \textsc{clf}:far  \textsc{pos3}-eye.\textsc{pl}\\
\glt ‘They saw the eyes (coming) and they got scared.’ 
\z

\ea\label{ex:payne:11} (005ZorroAvispa 1.1)\\
\gll  yi-la-ʔa  \textbf{ha-soʔ}  waʕatʔo\\
\textsc{a3}-find-\textsc{obj.sg}    \textsc{f-clf}:far    wasp\\
\glt ‘They found a wasp (in the forest).’ 
\z

Members of the deictic/visibility \textsc{clf} subset in \tabref{tab:payne:1} can express metaphorical or cognitive distance, and sometimes a kind of evidentiality (\sectref{sec:payne:7}).\footnote{\citet{MessineoEtAl2016} observe similar uses for the Toba cognates.} Example \REF{ex:payne:12} describes customary actions. No particular mothers or carandillo palm leaves are physically near the narrator, yet the ‘near’ \textsc{clf} \textit{naʔ} occurs. Example \REF{ex:payne:13} is the first line of a folktale in which the participants are not departing from view within the world of discourse, though they are apparently conceptualised as distal and hence coded with the ‘far’ \textsc{clf} \textit{soʔ}.


\ea\label{ex:payne:12} (039Artesania 1.1)\\
\gll  \textbf{naʔ}  qad-atʔe-l-pi  daʔ  set-ake  d-ʔoʕo-n-aʕan načʔe  wʔae-ñe  yi-lake  \textbf{naʔ}  laqata  l-awa\\
\textsc{clf}:near  \textsc{pos1pl}-mother-\textsc{pl-col} \textsc{sub}  want-\textsc{des}  \textsc{a3}-weave-\textsc{nprog-caus} then be.first-\textsc{compl}  \textsc{a3}-look.for  \textsc{clf}:near  carandillo  \textsc{pos3}-leaf\\
\glt ‘When our mothers want to weave (make handicrafts), first they look for carandillo (\textit{trithrinax schizophylla}) leaves.’ 
\z

\ea\label{ex:payne:13} (003Zorro Paloma 1.1)\\
\gll  wʔo  \textbf{so}ʔ  n-loʔ  \textbf{so}ʔ  waʕayaqalʔačiyi  qataʕa  \textbf{so}ʔ  doqotoʔ\\
\textsc{exist}  \textsc{clf}:far  \textsc{pos.indf}-day  \textsc{clf}:far fox and  \textsc{clf}:far  dove\\
\glt ‘There was a day the fox and dove met each other.’ 
\z

Similarly, \REF{ex:payne:14} and \REF{ex:payne:15} are the initial sentences of two different explanations of fishing customs. They seem to be situationally identical in objective deictic/visibility features, but in \REF{ex:payne:14} generic ‘people’ who go fishing take the ‘far’ \textsc{clf}, while in \REF{ex:payne:15} generic ‘people’ who go fishing take the ‘near’ \textsc{clf} \textit{naʔ}. Presumably they are conceptualised differently within the world of discourse.

\ea\label{ex:payne:14} (013Pesca2 1.1)\\
\gll  \textbf{soʔ}  \textbf{siyaʕa-di-pi}  daʔ  set-ake  di-yʔako\\
\textsc{clf}:far person-\textsc{pauc-col}  \textsc{sub}   want-\textsc{des}  \textsc{a3}-fish\\
\glt ‘When \textbf{the} \textbf{people} want to go to fish, …’ 
\z

\ea\label{ex:payne:15} (14Pesca4 1.1)\\
\gll  daʔ  ni-yʔakoʕo-k  daʔ  čʔe    n-piyae-yi daʔ  di-yʔako  \textbf{naʔ}    \textbf{siyaʕa-di-pi}\\
\textsc{sub}  \textsc{b3}-fish-\textsc{m}  \textsc{sub}  suddenly  \textsc{b3}-gather-\textsc{pl} \textsc{sub}  \textsc{a3}-fish    \textsc{clf}:near person-\textsc{pauc-col}\\
\glt ‘When it is fishing (time), when \textbf{the} \textbf{people} spontaneously gather to go hunter-gather (in general, lit. ‘fish’), …’ 
\z

The \textsc{clf}s can also show psychological deixis in the sense of empathy or point-of-view. For instance, in \REF{ex:payne:16} from a folktale, the skunk beats both the peccary (by killing the peccary with its odour and then eating it) and the fox (by outsmarting the fox). With one exception, the poor peccary is consistently referred to with the ‘near’ \textsc{clf} in the story, while the skunk and the fox who eat or attempt to eat the peccaries are referred to with the ‘far’ \textsc{clf}.

\ea\label{ex:payne:16} (004 ZorrinoZorro 1.4)\\
\gll  načʔe  daʔ  yi-lew  \textbf{naʔ}  owaqae,  načʔe \textbf{soʔ} koñem ya-lik  \textbf{ha-na=mʔe} owaqae\\
then  \textsc{sub}  \textsc{a3}-die  \textsc{clf}:near  peccary  then \textsc{clf}:far skunk \textsc{a3}-eat  \textsc{f-clf}:near=\textsc{dem1:neut} peccary\\
\glt ‘When the peccary dies, then the skunk eats this peccary.’ 
\z

The third member of the deictic/visibility \textsc{clf} subset is \textit{gaʔ} ‘absent, unseen’. Its meaning ranges from ‘unseen now’ (i.e. absent, remote) to ‘never seen’ and hence ‘unknown’. Thus, it can indicate nonidentifiablity or nonreferentiality, as in \REF{ex:payne:17}.\footnote{Also, some interrogative roots take the \textsc{clf} \textit{gaʔ} ‘unseen’, as in \REF{ex:payne:22}.}

\ea\label{ex:payne:17} (013Pesca2 1.1)\\
\gll  yi-kʔataʕa-som-ʔa  \textbf{gaʔ}  lačiyaʔge\\
\textsc{a3}-go-\textsc{loc}:down-\textsc{obj.sg}  \textsc{clf}:absent stream\\
\glt ‘They (prepare to) go to a/some stream.’ 
\z

Finally, a diminutive \textit{tʔae}(\textit{ʔ}) can intervene between a \textsc{clf} and a noun. As we will see below, this diminutive is becoming morphologised as part of demonstrative constructions.

\ea\label{ex:payne:18} (001ZorroPichi 2.10)\\
\gll  soʔ  \textbf{tʔaeʔ}  napam\\
 \textsc{clf}:far  little  armadillo\\
\glt ‘the distant little armadillo’ 
\z

We now turn to demonstrative constructions employing the roots in \tabref{tab:payne:2} and \tabref{tab:payne:3}.

\section{Simple demonstrative construction}\label{sec:payne:4}

The simple demonstrative (\textsc{sdem}) construction contains only a demonstrative root and functions as an adverbial proform (cf. \tabref{tab:payne:4}). This construction is limited to the ubiquitous \textsc{dem1} root \textit{hoʔ}. Our understanding is that it is primarily used to draw the hearer’s attention to something in the context, much as a pointing gesture does. In fact, the \textsc{sdem} is often, but not always, accompanied by a physical gesture. As a simple demonstrative, \textit{hoʔ} mostly functions as an exophoric adverbial locative, as in \REF{ex:payne:19}–\REF{ex:payne:21}. \textit{Hoʔ} is often translated as \textit{aqui} (‘here’) but also as \textit{allí} (‘there’) in Spanish. As an attention drawing form, it allows some locational range; but it is primarily proximal, so we gloss it consistently as ‘proximal’ to reflect this dominant use.\footnote{\textit{Heʔn,} as in \REF{ex:payne:20}, is a common contraction from \textit{he-naʔ}; the two forms are equivalent in meaning.}

\ea\label{ex:payne:19} (006ZorroCompanero 1.8)\\
\gll  a-wʔaʕa-nyi  \textbf{hoʔ}  naʔ  yi-če\\
\textsc{a2-hit-loc:middle}  \textsc{dem1:prox}  \textsc{clf:near}  \textsc{pos1-leg}\\
\glt ‘Hit \textbf{here} (on) my leg!’ 
\z

\ea\label{ex:payne:20} (107Ethno26Grasa 3)\\
\gll  he-ʔn  četa  ho-ga-mʔe  siyak  qanačʔe qo-y-ača-n-yi    \textbf{hoʔ}  ha-gaʔ  alewanʔoʕona\\
\textsc{m-clf}:near grease  \textsc{m-clf}:absent-\textsc{dem1:neut}  animal then \textsc{sbj.indf-a3}-put-\textsc{nprog-loc}  \textsc{dem1:prox}  \textsc{f-clf}:absent vessel\\
\glt ‘Then the fat of whatever animal they put \textbf{there} in an (earthenware) vessel (\textit{alewanʔoʕona}).’ 
\z

\ea\label{ex:payne:21} (007ZorroWaqaw 1.8)\\
\gll  čʔe  Ø-ek  \textbf{hoʔ}  de-mače-tape-get soʔ=n-egaʕa-wa  waqaʔw\\
 soon \textsc{a3}-go \textsc{dem1:prox}  \textsc{a3}-hear-\textsc{prog-ven} \textsc{clf}:far=\textsc{pos.indf}-friend-\textsc{hum} bird.species\\
\glt ‘Then, he (Fox) went away \textbf{there} [indicating the place where Waqaw was; not necessarily far or close], he heard his friend Waqaw (bird species) coming.’ 
\z

\textit{Hoʔ} can also have a temporal function, as in \REF{ex:payne:22}. (It also occurs in \textit{ho(ʔ)kalʔioʔ} meaning ‘before, long ago’.)

\ea\label{ex:payne:22} (001ZorroPichi 1.7)\\
\gll qančʔe  naeʔ=gaʔ  aw-men  \textbf{hoʔ}  ñ-egaʕa-wa\\
then  \textsc{intg=clf}:absent \textsc{a2}-sell  \textsc{dem1:prox}  \textsc{pos1}-companion-\textsc{hum}\\
\glt ‘So what will you sell \textbf{now}, my companion?’ 
\z

Though the \textsc{sdem} with \textit{hoʔ} is primarily exophoric, it can be endophoric. In \REF{ex:payne:23}, it functions as a discourse anaphoric form, referring back to the situation of being authorised to find a particular document.

\ea\label{ex:payne:23} (067ToribiaAcosta.46-48)
\ea\label{ex:payne:23a}
\gll   hayem  kaʔ  sepa  čʔe  algún  documento daʔ  daʔ  Ø-ek-a  soʔ  saλa-nek\\
 \textsc{1sg} before seem.to.me then some  document \textsc{sub} \textsc{sub} \textsc{3}-go-\textsc{loc}:specific  \textsc{clf}:far chief-\textsc{agent}\\
\glt ‘I believed that, that the chief came back with the document’
\ex\label{ex:payne:23b}
\gll   daʔ    qomiʔ  y-aloʕo-na-lo\\
\textsc{sub}  \textsc{1pl}  \textsc{3}-show-\textsc{nprog-pl}\\
\glt ‘so that he could show us.’
\ex\label{ex:payne:23c}
\gll   daʔ    kaʔ  epaʕa  autorisaw  \textbf{hoʔ}    eta-t\\
  \textsc{sub} before it.seems authorised \textsc{dem1:prox} say-\textsc{prog}\\
\glt ‘He was saying that he seems authorised for \textbf{this}.’ 
\z
\z

In the more complex demonstrative construction next discussed in \sectref{sec:payne:5}, we find \textit{hoʔ} in both adverbial and non-adverbial functions.

\section{Deictic demonstrative construction}\label{sec:payne:5}

Pilagá has a complex deictic demonstrative (\textsc{ddem}) construction involving the \textsc{dem1} roots (\tabref{tab:payne:2}) plus the \textsc{clf}s (\tabref{tab:payne:1}). The elements of complex demonstratives show dialect and idiolect variation and may vary by speaker’s age. As noted in \sectref{sec:payne:2}, some elements can undergo vowel harmony. The ‘near’ \textsc{clf} \textit{naʔ} often reduces to (\textit{ʔ})\textit{n}, and the ‘neutral’ \textsc{dem1} root \textit{mʔe} often reduces to (\textit{ʔ})\textit{m}. There is considerable variation in the text corpus especially for \textit{mʔe}. For instance, \textit{hogamʔe}, \textit{hogamʔoʔ}, and \textit{hoganʔe} all contain \textit{mʔe} and are alternative forms of ‘that absent/unknown’. \textit{Hoʔn} is a contraction from \textit{ho-naʔ=mʔe} (\textsc{m-clf}:near=\textsc{dem1:neut}). According to the consultant Ignacio Silva, some of the variant forms are “old words”, rarely heard now. All these factors result in a great variety of surface forms.

The deictic demonstrative (\textsc{ddem}) construction has the structure in \REF{ex:payne:24}.

\ea\label{ex:payne:24}
Deictic demonstrative construction (\textsc{ddem})\\
(\textsc{gender}-)\textsc{classifier=}(diminutive=)\textsc{dem1}(-plural)\\
\z

The gender markers in the \textsc{ddem} are \textit{ha}- ‘feminine’ and (\textit{h})\textit{e-}/\textit{ho}-/\textit{Ø} ‘masculine’, illustrated in \REF{ex:payne:25}–\REF{ex:payne:28}. Sometimes the masculine is left unmarked for gender. We do not write the zero form in examples. Plural can be marked by lengthening the \textsc{clf} vowel, and some \textsc{ddem}s add -\textit{lo} or -\textit{wa} ‘plural’.

\ea\label{ex:payne:25} (004ZorrinoZorro 1.4)\\
\gll  \textbf{ha-na=mʔe}  owaqae\\
\textsc{f-clf}:near=\textsc{dem1:neut} peccary\\
\glt ‘\textbf{this} peccary’ 
\z

\ea\label{ex:payne:26} (136ethnograph55 6)\\
\gll  naqae=ga    \textbf{ho-ga=maʕa}  piyʔoʕonaq\\
\textsc{dem2:rcg=clf}:absent \textsc{m-clf}:absent=\textsc{dem1:nvis} shaman\\
\glt ‘(Death could result from the action of) \textbf{some/any} shaman.’ 
\z

\ea\label{ex:payne:27} (001ZorroPichi 2.14)\\
\gll  yeči  ki  hora  daʔ  \textbf{ho-da=maʕa}  y-em\\
 evident what hour \textsc{sub} \textsc{m-clf:ver=dem1:nvis}  \textsc{a3}-end\\
\glt ‘(I don’t know) what time \textbf{that} (the story) ends…’ 
\z

\ea\label{ex:payne:28} (032ColoniaEnsanchez 1.3)\\
\gll  \textbf{ñi=maʕa}  ñiʔ  qan-saλaʕa-nek\\
 \textsc{clf:no.ext=dem1:nvis}  \textsc{clf:no.ext}  \textsc{pos1pl}-chief-\textsc{m}\\
\glt ‘\textbf{that} our chief’ (not present at the time of utterance)\footnote{A native speaker said this text line sounded redundant, apparently due to both the \textsc{ddem} and the separate \textsc{bclf} before ‘our chief’.} 
\z

‘Shape’ (rounded) or ‘size’  appears as a semantic extension of ‘feminine’ gender. However, not all nominals in Pilagá are marked for a particular gender distinction, regardless of their shape, nor is such marking synchronically predictable. As \REF{ex:payne:24} indicates, a diminutive can occur between the \textsc{clf} and demonstrative root, as in \REF{ex:payne:29}–\REF{ex:payne:30}. The diminutive is acceptable after a \textsc{dem1} root only if the diminutive is preceded by a \textsc{clf} (as if the diminutive morpheme is nominal); compare \REF{ex:payne:31}–\REF{ex:payne:32}. The diminutive can communicate that one is feeling sorry for a referent.

\ea\label{ex:payne:29}
\gll  ñiʔ=tʔae=mʔe\\
\textsc{clf:no.ext=dim=dem1:neut}\\
\glt ‘that little rounded/sitting one’ (I may be seeing it or not)
\z

\ea\label{ex:payne:30}
\gll  daʔ=tʔae=čaʔa\\
 \textsc{clf:ver=dim=dem1:dist.vis}\\
\glt ‘that far little one’ (I see it)
\z

\ea\label{ex:payne:31}
\gll  ñiʔ=mʔe  ñiʔ  taʔe\\
\textsc{clf:no.ext=dem1:neut}  \textsc{clf:no.ext}  little\\
\glt ‘that little rounded/sitting one’ (I may be seeing it or not)
\z

\ea\label{ex:payne:32}
 *daʔ=čaʔa  tʔae\\
\z

All members of the \textsc{dem}1 paradigm (\tabref{tab:payne:2}) occur in the \textsc{ddem} construction. We illustrate this in combination with the ‘horizontal’ \textsc{clf} \textit{diʔ}. In \REF{ex:payne:33}, \textit{hoʔ} indicates the object is close to the speaker and visible at the time of utterance. \textit{Mʔe} is neutral in \REF{ex:payne:34} about whether the object is visible at speech time. \textit{Čaʔa} in \REF{ex:payne:35} requires that the object be visible at speech time. \textit{Maʕa} in \REF{ex:payne:36} indicates the object is not present/visible to the speaker at speech time.

\ea\label{ex:payne:33}
\gll  yi-laʔa  \textbf{di=hoʔ}  siyaʕawa\\
 \textsc{a3}-see \textsc{clf:hor=dem1:prox} person\\
\glt ‘She/He saw \textbf{this} person lying down/asleep/dead.’ (The person is visible now and close; but need not currently be horizontal/dead.)
\z

\ea\label{ex:payne:34}
\gll  yi-laʔa  \textbf{di=mʔe}  siyaʕawa\\
\textsc{a3}-see  \textsc{clf:hor=dem1:neut} person\\
\glt ‘She/He saw \textbf{a/that} person lying/sleeping/dead.’ (The person may or may not be in sight at the time of speaking.)
\z

\ea\label{ex:payne:35}
\gll  yi-laʔa  \textbf{di=čaʔa}  siyaʕawa\\
 \textsc{a3}-see \textsc{clf:hor=dem1:dist.vis} person\\
\glt ‘She/He saw \textbf{that} far-away lying-down/asleep/dead person.’ (The person is visible now and far away; pointing to the person.)
\z

\ea\label{ex:payne:36}
\gll  yi-laʔa  \textbf{di=maʕa}  siyaʕawa\\
 \textsc{a3}-see \textsc{clf:hor=dem1:nvis} person\\
\glt ‘She/He saw \textbf{that} person lying down/asleep/dead.’ (The person is not visible to the speaker.)
\z

Though all \textsc{dem1} roots occur in the \textsc{ddem} construction, there are some co-occurrence restrictions with particular \textsc{clf}s to avoid semantic clashes. This is particularly relevant for the deictic/visibility \textsc{clf}s, as the posture/shape \textsc{clf}s do not lend deictic information to the overall meaning of the demonstrative construction (as seen in \REF{ex:payne:33}–\REF{ex:payne:36} with \textit{diʔ} ‘horizontal’).

\textit{Čaʔa} ‘distant visible’ only occurs with \textsc{clf}s that allow interpretation of a visible referent, i.e. \textit{naʔ} ‘near, coming’, \textit{soʔ} ‘far, departing’, and the posture/shape \textsc{clf}s, as in \REF{ex:payne:37}–\REF{ex:payne:41}. Examples \REF{ex:payne:37}–\REF{ex:payne:38} have a distal+visible referent, marked by \textit{čaʔa}. The fact that it is approaching the reference point (potentially communicated by \textit{naʔ}) may be communicated with or without the ‘ventive’ suffix -\textit{get} on \textit{čaʔa}. The ‘itive’ -\textit{ge}(\textit{ʔ}) is not possible with \textit{naʔ=čaʔa}, but the ‘itive’ is possible with \textit{soʔ=čaʔa}, as in \REF{ex:payne:39}.

\ea\label{ex:payne:37}
\gll  naʔ=čʔa\\
     \textsc{clf}:near=\textsc{dem1:dist.vis}\\
     \glt ‘far referent coming near’
\z

\ea\label{ex:payne:38}
\gll  naʔ=čʔa-get\\
     \textsc{clf}:near=\textsc{dem1:dist.vis-ven}\\
     \glt ‘far referent coming near’
\z

\ea\label{ex:payne:39}
\gll soʔ=čaʔa-geʔ\\
 \textsc{clf}:far=\textsc{dem1:dist.vis}-\textsc{it}\\
\glt ‘far referent going away’
\z

\ea\label{ex:payne:40}
\gll do=čaʔa\\
 \textsc{clf:ver=dem1:dist.vis}\\
\glt ‘that upright far referent’
\z

\ea\label{ex:payne:41}
\gll  ña=čʔa-lo\\
     \textsc{clf:no.ext=dem1:dist.vis-pl}\\
\glt ‘those sitting there’
\z

The three examples sets between \REF{ex:payne:42} and \REF{ex:payne:51} illustrate additional combinations of the deictic/visibility \textsc{clf}s with the more frequent \textsc{dem1} roots. The specific interpretation of a combination may depend on pragmatic context. Examples \REF{ex:payne:42}–\REF{ex:payne:44} carry the ‘near, coming’ \textsc{clf} \textit{naʔ}. In \REF{ex:payne:44}, the ‘distal’ feature of the \textsc{dem1} root \textit{čaʔa} over-rides any ‘near, proximal, coming’ meaning that might otherwise be associated with \textit{naʔ}; this suggests that \textit{naʔ} may be bleaching of its spatial semantics. Along with a wave of the hand, \REF{ex:payne:44} could serve as an answer to the question ‘Where is José?’

\ea\label{ex:payne:42}
\gll  \textbf{noʔ=hoʔ}  naʔ  taʔe\\
 \textsc{clf}:near\textsc{=dem1:prox}  \textsc{clf}:near  little\\
\glt ‘\textbf{this} little one’ (\textbf{right here} beside me and I \textbf{see} it)
\z

\ea\label{ex:payne:43}
\gll  \textbf{naʔ=mʔe}  naʔ    taʔe\\
 \textsc{clf}:near\textsc{=dem1:neut}  \textsc{clf}:near  little\\
\glt ‘\textbf{this} little one’ (the item \textbf{may be present or not}; the expression could refer to something I have been talking about)
\z

\ea\label{ex:payne:44}
\gll  \textbf{naʔ=čaʔa}  naʔ  taʔe\\
 \textsc{clf}:near\textsc{=dem1:dist.vis} \textsc{clf}:near  little\\
\glt ‘(he’s) \textbf{that} little one’ (\textbf{there}, not moving)
\z

Examples \REF{ex:payne:45}–\REF{ex:payne:48} combine \textit{soʔ} with \textsc{dem1} roots. \REF{ex:payne:45} is unacceptable to our consultant with the explanation that it contradictorily combines \textit{soʔ} ‘far’ with \textit{hoʔ} ‘proximal’ (we return to this combination further below). When \textit{soʔ} combines with ‘neutral’ \textit{mʔe}, as in \REF{ex:payne:46}, the result indicates a visible or identifiable referent departing from the deictic center; thus with \textit{mʔe}, the \textsc{clf} yields the primary deixis/visibility meaning. In \REF{ex:payne:47} with \textit{čaʔa} ‘distal visible’ plus the ‘itive’ -\textit{ge}(\textit{ʔ}), the overall reading is of an already distal but visible participant moving away. Without the ‘itive’, one consultant finds \textit{soʔ} incompatible with \textit{čaʔa}. This is because \textit{soʔ} can sometimes be interpreted as ‘(going) out of view’, while \textit{čaʔa} specifically indicates ‘visible’; but the combination was acceptable in \REF{ex:payne:39}. In \REF{ex:payne:48} with \textit{maʕa} ‘non-visible’, the speaker could possibly know the non-visible referent, though there is something uncertain about it in the speaker’s mind.

\ea\label{ex:payne:45}
\gll  *soʔ=taʔe=hoʔ\\
     \textsc{clf}:far=\textsc{dim=dem1:prox}\\
\glt {}
\z

\ea\label{ex:payne:46}
\gll  soʔ=taʔe=mʔe\\
     \textsc{clf}:far=\textsc{dim=dem1:neut}\\
\glt ‘that small visible/identifiable departing referent’
\z

\ea\label{ex:payne:47}
\gll  soʔ=taʔe=čaʔa-ge (*čaʔa)\\
 \textsc{clf}:far=\textsc{dim=dem1:dist.vis-it}\\
\glt ‘that small visible far-away departing referent’
\z

\ea\label{ex:payne:48}
\gll  soʔ=taʔe=maʕa\\
     \textsc{clf}:far=\textsc{dim=dem1:nvis}\\
\glt ‘that little (stationary) unseen referent’ (perhaps I know it)
\z

Possible interpretations of the ‘absent’ \textsc{clf} \textit{gaʔ} include unknown, non-specific, and non-referential readings, as in \REF{ex:payne:49}–\REF{ex:payne:52}. It may combine with the ‘proximal’ and ‘neutral’ \textsc{dem1} roots, but not with \textit{čaʔa} ‘distal+visible’, as shown by \REF{ex:payne:49}–\REF{ex:payne:51}. This restriction is due to the semantic clash between the ‘visible’ feature of \textit{čaʔa} and the ‘absent’ feature of \textit{gaʔ}. The perhaps surprising example in this set is \REF{ex:payne:49}, as it might seem that the ‘absent’ feature of \textit{gaʔ} should conflict with \textit{hoʔ}. However, its acceptability reveals the expanding semantic domain of polysemous \textit{hoʔ}; in particular, with a \textsc{clf,} \textit{hoʔ} can be used endophorically for a participant currently under discussion. In this use, it participates in topic marking. Example \REF{ex:payne:50} shows that with the ‘neutral’ \textsc{dem1} root, the semantic features of the \textsc{clf} again become especially evident.

\ea\label{ex:payne:49}
\gll  \textbf{gaʔ=taʔe=hoʔ}\\
 \textsc{clf}:absent\textsc{=dim=dem1:prox}\\
\glt ‘\textbf{this} little one’ (referring to something/somebody under discussion that is \textbf{far} or I \textbf{do not remember well})
\z

\ea\label{ex:payne:50}
\gll  \textbf{gaʔ=taʔe=mʔe}\\
 \textsc{clf}:absent\textsc{=dim=dem1:neut}\\
\glt ‘\textbf{that} little one’ (\textbf{not in view}, never seen, or unknown, but I \textbf{remember} it)
\z

\ea\label{ex:payne:51}
  *gaʔ=taʔe=čaʔa\\
\z

If under the scope of negation, \textit{gaʔ} plus \textit{mʔe} may indicate ‘nothing, nobody’, as in \REF{ex:payne:52}.

\ea\label{ex:payne:52}
\gll  Qaya  \textbf{gaʔ=mʔe}\\
\textsc{nexist.hum}  \textsc{clf}:absent=\textsc{dem1:neut}\\
\glt ‘There is \textbf{nobody}.’
\z

In \REF{ex:payne:44}, we saw that the meaning of \textsc{dem1} \textit{čaʔa} overrides the spatial meaning that \textsc{clf} \textit{naʔ} might otherwise carry. However, in some situations the meaning of a \textsc{clf} can override that of a \textsc{dem1} root. Thus, \REF{ex:payne:53}–\REF{ex:payne:56} were said to mean “basically the same” in terms of spatial/visibility deixis. They all carry the \textsc{clf} \textit{soʔ} ‘far, departing (potentially to the point of being absent)’, regardless of choice of the demonstrative root. Notably, \REF{ex:payne:54} was judged as fine, while \REF{ex:payne:45} with the same key elements was rejected. We analyse the variability in speaker’s judgments as reflecting the polysemous nature of \textit{hoʔ}: on one occasion its exophoric ‘proximal’ feature is conceptually prominent and thus it is viewed as conflicting with \textit{soʔ}, but on another – as in \REF{ex:payne:54} – \textit{hoʔ} is interpreted endophorically to indicate the participant ‘under discussion’ in the discourse, so there is no spatial deixis conflict. The \textsc{dem2} root in \REF{ex:payne:56} is discussed in \sectref{sec:payne:6}.

\ea\label{ex:payne:53}
\gll  \textbf{soʔ}  y-alek\\
\textsc{clf}:far  \textsc{pos1}-son\\
\glt ‘my son (distant/absent)’
\z

\ea\label{ex:payne:54}
\gll  \textbf{so=hoʔ}  y-alek\\
\textsc{clf}:far=\textsc{dem1:prox}  \textsc{pos1}-son\\
\glt ‘\textbf{that} my son (departing)’
\z

\ea\label{ex:payne:55}
\gll  \textbf{so=mʔe}  y-alek\\
\textsc{clf}:far=\textsc{dem1:neut}  \textsc{pos1}-son\\
\glt ‘\textbf{that} my son (distant/absent)’
\z

\ea\label{ex:payne:56}
\gll  \textbf{naqae=soʔ}  y-alek\\
\textsc{dem2:rcg=clf}:far  \textsc{pos1}-son\\
\glt ‘\textbf{that} (is) my son’ (understood to not be present)
\z

The preceding discussion has focused on structure of the \textsc{ddem} and meanings of composing morphemes. We now more explicitly address grammatical and discourse functions of this construction (cf. \tabref{tab:payne:4}). D\textsc{dem} constructions serve as adverbial and participant proforms or as determiners. The proform function is illustrated in \REF{ex:payne:57}-\REF{ex:payne:59} with the root \textit{hoʔ}. The two senses of \REF{ex:payne:57} show the adverbial exophoric locative function of the \textsc{ddem} with \textit{hoʔ}, and its participant reference function. In \REF{ex:payne:58}, the \textsc{ddem} refers exophorically to an inanimate entity. In \REF{ex:payne:59}, it refers exophorically to a location.

\ea\label{ex:payne:57}
\gll  \textbf{so=hoʔ}\\
\textsc{clf}:far=\textsc{dem1:prox}\\
\glt ‘\textbf{there}’ (Spanish \textit{allá}) / ‘\textbf{one} (who is) \textbf{departing}’
\z

\ea\label{ex:payne:58}
\gll  \textbf{ha-n=hoʔ}  mate\\
 \textsc{f-clf}:near-\textsc{dem1:prox} mate(drink)\\
\glt ‘This is a mate (container).’
\z

\ea\label{ex:payne:59} (060TrabajoMadera 12)\\
\gll  maλaʕa  qaga  \textbf{ha-ño=hoʔ}    naʕa na-ñ-ʔa  dyo=hoʔ  Campo\\
yet  \textsc{nexist}  \textsc{f-clf:no.ext=dem1:prox}  now \textsc{b3}-sit-\textsc{obj.sg}  \textsc{clf:hor=dem1:prox} place.name\\
\glt ‘\textbf{This} [pointing to the place of the community] did not yet exist (which) is now (the spread-out community of) Estanislao del Campo.’ 
\z

Example \REF{ex:payne:58} is a zero-copula equational clause and the \textsc{ddem} is not in the same phrase as \textit{mate}. The pronominal \textsc{ddem} with \textit{naʔ} ‘near’ plus \textit{hoʔ} indicates an item close enough to touch. The feminine gender prefix occurs due to the rounded shape of the container. In \REF{ex:payne:59}, \textit{ha-ño=hoʔ} indicates a non-extended referent, pointing to the particular location (rather than extended shape) of the community.

\textsc{Ddem} constructions with \textsc{dem1} roots other than \textit{hoʔ} function only pronominally and as determiners (not adverbially; cf. \tabref{tab:payne:4}). In \REF{ex:payne:60}, \textit{he-n=mʔe} functions as a text-internal anaphoric pronominal. It refers to the story the speaker is in the midst of relating. (\textit{He-n=hoʔ} with the ‘proximal’ \textsc{dem1} root in this context would mean ‘here, the place where I the speaker am’.)

\ea\label{ex:payne:60} (001ZorroPichi 2.13)\\
\gll  \textbf{he-n=mʔe}  huw!\\
\textsc{m-clf}:near=\textsc{dem1:neut} wow\\
\glt ‘\textbf{this} (story), wow!’ (meaning ‘this story I am telling you’) 
\z

In \REF{ex:payne:61}, the highlighted \textsc{ddem} functions as a cataphoric pronominal. \textit{Mʔe} carries the ‘vertical’ \textsc{clf} \textit{daʔ}, but in this context it designates a propositional event which, as a whole, is an abstract concept.

\ea\label{ex:payne:61} (048RecoleccionMiel 1.1)
\ea\label{ex:payne:61a}
\gll  so=mʔe  siyaʕa-di-pi  daʔ=mʔe qo-ila-ʔa  soʔ  konʼayaʕapoλoʔ\\
 \textsc{clf}:far\textsc{=dem1:neut} person-\textsc{pauc-col} \textsc{clf:ver=dem1:neut} \textsc{sbj.indf}-see-\textsc{obj.sg}  \textsc{clf}:far bee.hive\\
\glt ‘When the men find the bee hive,
\ex\label{ex:payne:61b}
\gll načʼe  wʼae-ñe  \textbf{daʔ=mʔe}\\
 soon be.first-\textsc{compl}  \textsc{clf:ver=dem1:neut}\\
\glt ‘they first do \textbf{this}:’
\ex\label{ex:payne:61c}
\gll qo-ya-lo-n  soʔ  doleʔ\\
  \textsc{sbj.indf-a3}-stir-\textsc{nprog}  \textsc{clf}:far fire\\
\glt ‘they stir up the fire.’ 
\z
\z

A \textsc{ddem} with \textit{čaʔa} may function exophorically or endophorically. The exophoric function is dominant, but in \REF{ex:payne:62d}, from a story about competition between Fox and Toad, \textit{ñiʔ=čaʔa} is endophoric, referring to the toad. Line \REF{ex:payne:62d} also shows the pronominal \textsc{ddem} \textit{soʔ=tʔae=mʔe} functioning anaphorically.

\ea\label{ex:payne:62} (002SapoZorro 1.11-1.14)
\ea\label{ex:payne:62a}
\gll degesesow  eso  wayqalʼačiyi  yači  enaʕaye-ik\\
quickly \textsc{clf}:far fox  certain dusty-\textsc{aug}\\
\glt ‘Quickly, it is clear/certain that the fox stirred up a lot of dust.’

\ex\label{ex:payne:62b}
\gll soʔ    qololo  daʔ  Ø-wenot  qanačʔe  yitaʕa ne-noʕo-segem  soʔ  qololo  l-qaya\\
  \textsc{clf}:far toad \textsc{sub}  \textsc{a3}-jump  then  again  \textsc{b3}-move-upward \textsc{clf}:far toad \textsc{pos3}-sibling\\
\glt ‘But when the toad jumped, another toad appeared.’

\ex\label{ex:payne:62c}
\gll ta  Ø-wenot  ta  ne-noʕo-segem  soʔ  qololo l-qaya  ye-dʔ-a-ta\\
  \textsc{again} \textsc{a3}-jump again \textsc{b3}-move-upward \textsc{clf}:far toad
\textsc{pos3}-sibling \textsc{a3}-arrive-\textsc{obj.sg}-other.side\\
\glt ‘He jumped again, and another toad (appeared until) it reached the finish.’

\ex\label{ex:payne:62d}
\gll qančʔe  \textbf{soʔ=tʔae=mʔe}  qančʼe  yeči, n-selka-pe-get  \textbf{ñiʔ=čaʔa=w}  ñiʔ n-qomit-aʕa-wa  ñiʔ  yači  yi-weʔen\\
  then \textsc{clf}:far\textsc{=dim}=\textsc{dem1:neut} then certain \textsc{b3}-see-\textsc{prog-ven}  \textsc{clf:no.ext=dem1:dist.vis=intsf}  \textsc{clf:no.ext} \textsc{b3}-compete-\textsc{nmlz-hum}  \textsc{clf:no.ext} certain \textsc{a3}-laugh\\
\glt ‘He (Fox) certainly keeps looking for that far-distant one (toad) coming towards him and so he (toad, referenced throughout by \textit{ñiʔ}) certainly/evidently laughs at the competitor (Fox).’ 
\z
\z

Examples \REF{ex:payne:63}–\REF{ex:payne:64} show pronominal \textsc{ddem}s with the ‘nonvisible’ root \textit{maʕa}.

\ea\label{ex:payne:63} (032ColoniaEnsanchez 1.1)\\
\gll  \textbf{diʔ=maʕa}  diʔ-ae  qad-ʔačaqaʔ le-naʕat  Colonia  Ensanchez Ø-naʔa-ge  naʔ  seʔw\\
\textsc{clf:hor=dem1:nvis}  \textsc{clf:hor-f}  \textsc{pos1pl}-community \textsc{pos3}-name  Colonia Ensanchez \textsc{a3}-be-it  \textsc{clf}:near  north\\
\glt ‘This our community, its name (is) Colonia Ensanchez, is towards the north.’ (Context: The speaker is in a workshop talking about his far-distant community, probably looking at a map.) 
\z

\ea\label{ex:payne:64} (032ColoniaEnsanchez 1.3)\\
\gll  \textbf{ñiʔ=maʕa}  ñiʔ  qan-saλaʕa-nek l-sek  ha-ñiʔ  tamnaʕa-ki\\
\textsc{clf:no.ext=dem1:nvis}  \textsc{clf:no.ext}  \textsc{pos1pl}-chief-\textsc{m}
\textsc{pos3}-neighbor  \textsc{f-clf:no.ext}  religion-\textsc{loc}  \\
\glt ‘That (far away house) (that I’m talking about from memory) of our chief is between the church (and the school).’ 
\z

We now briefly comment on adnominal \textsc{ddem} uses. In \REF{ex:payne:65}, \textit{dyo=hoʔ} refers exophorically to a concrete participant. In \REF{ex:payne:66}, \textit{he-n=hoʔ} refers to a time.

\ea\label{ex:payne:65}
\gll  \textbf{dyo=hoʔ}  pioq  čeʔeda  weta-ñ-ʔa  kaliʔo\\
\textsc{clf:hor-dem1:prox}  dog be.first  be-\textsc{loc}:below-\textsc{obj.sg}  long.ago\\
\glt ‘\textbf{This} dog (present, that I am signaling) has been (lying) here a long time.’
\z

\ea\label{ex:payne:66} (077Sent09Cantidad 2)\\
\gll  soʔ  l-aqaya  setaeʔ  na-paʕagen-a  daʔ paʕagentanaʕaik  \textbf{he-n=hoʔ}  woʔe\\
\textsc{clf}:far  \textsc{pos2}-brother want \textsc{b3}-learn-\textsc{obj.sg}    \textsc{sub} teacher.\textsc{m}  \textsc{m-clf}:near=\textsc{dem1:prox} year\\
\glt ‘Your brother wants to study teaching (to be a teacher) \textbf{this} year.’ 
\z

Adnominal \textsc{ddem}s with \textit{mʔe} often mark already-mentioned participants, as in \REF{ex:payne:16}. But this is not always the case. In \REF{ex:payne:67}, \textit{diʔ=mʔe} occurs on the first mention of ‘garden/field’; the consultant expressed the view that the sentence would mean essentially the same thing if a \textsc{bclf} with just \textit{diʔ} occurred instead.

\ea\label{ex:payne:67} (035Linea 1.4)\\
\gll  daʔ  set-ake  a-e-ye  \textbf{diʔ=mʔe} qad-an-aʕan-qaʔ  qanačʼe  o-ketʼa-ge diʔ  naʔa-ik  Ø-lekaʔa-ege\\
 \textsc{sub} want-\textsc{des}  \textsc{a3}-go-in.line \textsc{clf:hor=dem1:neut} \textsc{pos1pl}-plant-\textsc{caus-loc} then \textsc{a2}-continue-\textsc{it} \textsc{clf:hor}  road-\textsc{m} \textsc{3a}-be.big-forward\\
\glt ‘If you want to get to our vegetable garden, you have to continue along the wide path.’ 
\z

In \REF{ex:payne:68}, \textit{mʔe} combines with the ‘absent’ \textsc{clf} \textit{gaʔ}, to determine the nonreferential phrase ‘our thought’

\ea\label{ex:payne:68} (001ZorroPichi 1.1)\\
\gll  čaqaga  daʔ  \textbf{gaʔ=mʔe}  qad-enat-aʕak\\
what \textsc{sub}    \textsc{clf}:absent=\textsc{dem1:neut}  \textsc{pos1pl}-think-\textsc{nmlz}\\
\glt ‘What (is) our thought? (i.e. ‘What shall we do?)’ \\
\z

\section{Recognitional demonstrative construction}\label{sec:payne:6}

A second demonstrative construction has not, to our knowledge, been noted in previous Guaykuruan literature. We call this a recognitional demonstrative (\textsc{rdem}) construction. In it, the root \textit{naqae} co-occurs with a \textsc{clf,} but \textit{naqae} differs from the \textsc{dem1} set in taking the \textsc{clf} as an enclitic, yielding the structure in \REF{ex:payne:69}.

\ea\label{ex:payne:69}
 Recognitional demonstrative construction (\textsc{rdem})\\
\textsc{dem2}.root=\textsc{classifier}(-plural)\\
\z

\textit{Naqae} indicates that the speaker anticipates the hearer already knows or is familiar with the identity of the referent (whether or not it has already been mentioned in the discourse), but wishes to activate it in the hearer’s mind. There may be an assumption of shared knowledge about the referent, but there may be doubt or even disbelief that the hearer is currently attending to it, so the speaker is activating it for the hearer. This is similar to what \citet{Himmelmann1996} and \citet{Diessel1999Book} call a “recognitional” demonstrative. We consider \textit{naqae} to be a demonstrative root as it orients the hearer’s attention to a participant.

Though consultants specifically comment that \textit{naqae} means the hearer knows the referent, in some contexts we think \textit{naqae} would be better characterized as indicating a familiar concept, as it can also be used for non-referential mentions. Speakers suggest it sometimes indicates a note of surprise or unexpectedness about a known but previously inactive referent, as if something has just activated it in the mind of the speaker. This is the case in \REF{ex:payne:70}, which stacks \textit{naqae=ñi} together with \textit{hoʔ} and \textit{ñiʔ}. Here, \textit{hoʔ} is verbally signalling (“pointing”) to the person, who is sitting. 

\ea\label{ex:payne:70}
\gll  naqae=ñi  hoʔ  ñiʔ  siyaʕawa\\
 \textsc{dem2:rcg=clf:no.ext}  \textsc{dem1:prox}  \textsc{clf:no.ext}  person\\
\glt ‘Ah, \textbf{that/this} is the person!’ (I see him/her, sitting)
\z

The \textsc{rdem} construction is attested in pronominal and adnominal functions (cf. \tabref{tab:payne:4}). In \REF{ex:payne:71}, \textit{naqae=na-wa} functions pronominally.\footnote{The second instance of the \textsc{clf} \textit{naaʔ} in \REF{ex:payne:71} functions like a relativiser to introduce a clause modifying \textit{naqae=na-wa}.} \textit{Naqa=ñi} is also pronominal in \REF{ex:payne:72}. However, \textit{naqae=naʔ} in \REF{ex:payne:72} appears to be adnominal. In the discourse just prior to \REF{ex:payne:72}, the fox is annoyed by a wasp and says, “Why are you always in my path? I’m going to hit you.” Fox then utters \REF{ex:payne:72}. Here, \textit{naqaenaʔ} indicates some emotiveness or unexpectedness.

\ea\label{ex:payne:71} (017Pesca1 1.4)\\
\gll  qataʕa  daʔ  an-awa-ʔ-n  naaʔ  l-ʔawaʕak-o \textbf{naqae=na-wa}  naaʔ  n-aya-pe-egʔa-lo he-n  ñiyaq-pi \\
and \textsc{sub}  \textsc{b2}-watch-\textsc{pl-nprog}  \textsc{clf}:near.\textsc{pl}  \textsc{pos3}-water.channel-\textsc{pl} \textsc{dem2:rcg=clf}:near-\textsc{pl}  \textsc{clf}:near.\textsc{pl}  \textsc{b3}-leave-\textsc{prog-loc}:specific-\textsc{pl}  \textsc{m-clf}:near  fish-\textsc{col}\\
\glt ‘Also to watch the water channels, \textbf{these} are where from the fish emerge.’ 
\z

\ea\label{ex:payne:72} (005ZorroAvispa 1.2)\\
\gll  lʔeʔ  \textbf{naqa=ñi}  y-ʔata-ʔnyi  \textbf{naqae=naʔ}  y-adik\\
why  \textsc{dem2:rcg=clf:no.ext}  \textsc{a3}-move-\textsc{loc}:middle  \textsc{dem2:rcg=clf}:near  \textsc{pos1}-path\\
\glt ‘Why does \textbf{this} \textbf{one} move (be) in \textbf{this} my path?’ 
\z

The \textsc{rdem} construction can be anaphoric. In \REF{ex:payne:73d}, \textit{naqae=na-wa} refers back to ‘the place where the fish pass’ mentioned in \REF{ex:payne:73b}.

\ea\label{ex:payne:73} (017Pesca1 1.1-104)
\ea\label{ex:payne:73a}
\gll wʼae-ñi  qomi  qo-ya-paʕage-nek-e daʔ  qo-y-eʔet  naʔ  čikena\\
 be.first-\textsc{compl}  \textsc{1pl}    \textsc{sbj.indf-a3}-teach-\textsc{agent-pl}  \textsc{sub}  \textsc{sbj.indf-a3}-prepare \textsc{clf}:near  arrow\\
\glt ‘First they taught us to prepare the arrows’

\ex\label{ex:payne:73b}
\gll qataʕa  naʔ  Ø-wapiñʼa-lo  qataʕa  naʔ n-aeya-pe-ege-ʔa  naʔ  ñiyaqa-pi\\
and \textsc{clf}:near \textsc{3}-be.place-\textsc{pl} and \textsc{clf}:near  \textsc{b3}-go-\textsc{ipfv}-opposite-\textsc{obj.sg}  \textsc{clf}:near fish-\textsc{col}\\
\glt ‘and (to know) the places and where the fish pass by.’

\ex\label{ex:payne:73c}
\gll qataʕa  qomi  qo-ya-paʕage-nek-e daʔ  qo-ya-ye-n  naʔ  ñiyaq\\
and  \textsc{1pl}  \textsc{sbj.indf-a3}-teach-\textsc{agent-pl} \textsc{sub}  \textsc{sbj.indf-a3}-throw-\textsc{nprog}  \textsc{clf}:near fish\\
\glt ‘Also they taught us how to stab a fish’

\ex\label{ex:payne:73d}
\gll qataʕa  daʔ  an-awa-ʔ-n  naaʔ  l-ʔawaʕako \textbf{naqae=na-wa}  naaʔ  n-aya-p-ege-lo he-ʔn  ñiyaqa-pi\\
and \textsc{sub}  \textsc{b2}-watch\textsc-{pl-nprog}  \textsc{clf}:near.\textsc{pl} \textsc{pos3}-caudal \textsc{dem2:rcg=clf}:near-\textsc{pl}  \textsc{clf}:near.\textsc{pl}  \textsc{b3}-go-\textsc{prog}-opposite-\textsc{pl} \textsc{m-clf}:near  fish-\textsc{col}\\
\glt ‘and how to watch the flow where the fish come out.’ 
\z
\z

Finally, \REF{ex:payne:26} suggests that Pilagá allows stacking of \textsc{rdem} and \textsc{ddem}.

\section{Further grammaticalisation: Nominal TAM and subordination}\label{sec:payne:7}

\subsection{Overview}\label{sec:payne:7.1}

Having now discussed the morphosyntax and basic functions of \textsc{clf}s and demonstratives, we turn to extended uses for nominal tense, mood/evidentiality, and clausal subordination. Pilagá adds to the body of data showing how demonstratives and determiners can further grammaticalise \citep{Diessel1999Book,Diessel2003,Gildea1993,Aikhenvald2015}.

\subsection{Incipient nominal tense, mood and evidentiality}\label{sec:payne:7.2}

Like other Guaykuruan languages, Pilagá lacks grammatical tense forms. However, some \textsc{clf}s and \textsc{dem1} roots implicate temporal meanings in certain contexts, and visible versus inferred source of evidence or (un)certainty. The temporal and evidentiality/modality meanings sometimes relate to evaluation of a nominal referent and sometimes to the proposition. Pilagá thus pertains to the set of languages having nominal TAM \citep{NordlingerSadler2004}. The role of \textsc{clf}s in conveying temporal, modal and evidential meanings in Guaykuruan has been discussed in other works \citep{MessineoEtAl2016,MessineoCúneo2019}, including for Pilagá \citep{VidalKlein1998,VidalGutiérrez2010}. Here we also note the role of \textsc{dem1} roots in marking these concepts.

In Pilagá, temporal use of \textsc{clf}s and \textsc{dem1} roots is pragmatic rather than fully grammaticalised, and interpretations interact with person and lexical meanings. First, \REF{ex:payne:74}–\REF{ex:payne:75} reveal the possible present-time interpretation of posture/shape \textsc{clf}s versus the past-time effect of \textit{soʔ} ‘far, departing’. \textit{Daʔ} is the \textsc{clf} for abstract nouns like \textit{lasook} ‘custom’, as well as for vertical ‘person’. Tthe overall interpretation in \REF{ex:payne:74} is present time. In \REF{ex:payne:75}, \textit{soʔ} occurs with both nouns. Given the abstract concept of ‘custom’, \textit{soʔ} cannot be interpreted as meaning that \textit{lasook} is spatially distant or moving away, so a space-to-time metaphorical inference yields the understanding of a ‘distant’ or past time situation. This likely also affects the use and interpretation of \textit{soʔ} with ‘person’.

\ea\label{ex:payne:74}
\gll  eta hoʔ  \textbf{daʔ}  lasook  \textbf{daʔ}  siyaʕawa\\
it.is.said \textsc{dem1:prox}  \textsc{clf:ver} custom \textsc{clf:ver} person\\
\glt ‘This \textbf{is} the custom of the person.’ (present)
\z

\ea\label{ex:payne:75}
\gll  eta  hoʔ  \textbf{soʔ}  lasook  \textbf{soʔ}  siyaʕa-di-pi\\
it.is.said  \textsc{dem1:prox}  \textsc{clf}:far  custom \textsc{clf}:far  person-\textsc{pauc-col}\\
\glt ‘This \textbf{was} the custom (of) the people.’
\z

To more clearly see the possible temporal effect of \textit{soʔ} when applied to concrete objects, consider \REF{ex:payne:76}. \textit{Hoʔ} occurs in \textit{soʔ=hoʔ} because ‘my son’ is in the speaker’s vicinity at the time of utterance. Since ‘my son’ is locally present, \textit{soʔ} ‘far’ can only be interpreted as indicating a temporally distant or past event. In this instance the \textsc{clf} has propositional/event-scope, while the \textsc{dem1} root has nominal scope related to the speech time.

\ea\label{ex:payne:76} (052RelatoAnciana 62)\\
\gll  n-oye-tak  naʕa  \textbf{soʔ=hoʔ}  y-alek n-woʕom  daʔ  l-qowaʕa\\
 \textsc{b3}-cry-\textsc{prog} now \textsc{clf}:far=\textsc{dem1:prox}  \textsc{pos1}-son \textsc{b3}-feel \textsc{clf:ver}  \textsc{pos3}-hunger\\
\glt ‘My son here/now was crying because he felt hunger.’ 
\z

\textit{Soʔ} does not obligate a past-time propositional interpretation if contextual factors indicate otherwise. Because of \textit{qomle} ‘later’ in \REF{ex:payne:77}, \textit{soʔ} is interpreted as applying to the past-time of the events involving ‘our ancestors’ and not to the event of telling.

\ea\label{ex:payne:77} \citep[1353]{VidalGutiérrez2010}\\
\gll  qomle  s-aqtanaʕan  soʔ  qadetalpi\\
later  \textsc{a1}-tell  \textsc{clf}:far our.grandparents\\
\glt ‘I’m going to tell you about our ancestors.’ 
\z

In contrast to \textit{soʔ}, the \textsc{clf} \textit{gaʔ} ‘unseen’ pragmatically allows that the “event in which it is embedded is an expression of the ignorance, the desires, or the intentions of the speaker, rather than a realized event” \citep[176]{VidalKlein1998}. \textit{Gaʔ} often occurs in clauses with conditional, obligation, or prospective meaning, as in \REF{ex:payne:78}.

\ea\label{ex:payne:78} (025EspirituSuri 1.3)\\
\gll  awa-wʔo-e  \textbf{gaʔ}  ade-wo …\\
 \textsc{a2}-make-\textsc{pl}  \textsc{clf}:absent \textsc{pos2}-clothes\\
\glt ‘you have to make your costumes …’ 
\z

Temporal interpretation is affected by pragmatic interaction between person, proximity of a referent to the speaker versus to other referents, and the semantics of lexemes, \textsc{clf}s and \textsc{dem} roots. In \REF{ex:payne:79}–\REF{ex:payne:81}, the speaker and the grammatical subject are the same person. \textit{Taqa} ‘talk’ plus \textit{ño=hoʔ} ‘non.extended=proximal’ implies a present-time action because the first-person speaker can talk ‘now’ to someone who is physically near.

\ea\label{ex:payne:79}
\gll  se-taqa-tap-ege  \textbf{ño=hoʔ}  siyaʕawa\\
 \textsc{a1}-talk-\textsc{prog-it}  \textsc{clf:no.ext=dem1:prox} person\\
\glt ‘I am talking to \textbf{a/this} person (sitting next/close to me).’
\z

In \REF{ex:payne:80} with ‘talk’, the ‘neutral’ \textsc{dem1} root \textit{mʔe} with a posture/shape \textsc{clf} allows a present or past interpretation. In \REF{ex:payne:81}, ‘distal+visible’ \textit{čaʔa} implies a past event because – ignoring telephones – one cannot talk ‘now’ to someone far away.

\ea\label{ex:payne:80}
\gll  se-taqa-tap-ege  \textbf{ñi=mʔe}  siyaʕawa\\
 \textsc{a1}-talk-\textsc{prog-it}  \textsc{clf:no.ext=dem1:neut} person\\
\glt ‘I am talking now to \textbf{a/that} person (who is sitting).’ /\\
‘I talked to \textbf{that} person (who is now sitting).’
\z

\ea\label{ex:payne:81}
\gll  se-taqa-tap-ege  \textbf{ñi=čaʔa}  siyaʕawa\\
 \textsc{a1}-talk-\textsc{prog-it}  \textsc{clf:no.ext=dem1:dist.vis} person\\
\glt ‘I was talking to \textbf{that} person now sitting (far from me).’ (Since he/she is far away, it is impossible to be talking to him/her right now.)
\z

\textsc{Dem}1 roots also play a role in expressing a speaker’s (un)certainty. Compare \REF{ex:payne:82}–\REF{ex:payne:83}, which show that \textit{maʕa} is a marker of uncertainty compared to \textit{mʔe}.

\ea\label{ex:payne:82}
\gll  eta  \textbf{ho-da=maʕa}  lasook\\
it.is.said \textsc{m-clf:ver=dem1:nvis} custom\\
\glt ‘It is said \textbf{that} is how the custom must have been.’ /\\
‘\textbf{That} seems to have been (how) the custom (was).’
\z

\ea\label{ex:payne:83}
\gll  eta  \textbf{ho-da=mʔe}  lasook\\
it.is.said \textsc{m-clf:ver=dem1:neut} custom\\
\glt ‘It is said \textbf{this} is what the custom is (like).’ (speaker is certain)
\z

 \subsection{\textit{Daʔ} as clausal subordinator}\label{sec:payne:7.3}

Elements of the determiner and demonstrative systems have become markers of subordination \citep{Vidal2001}. The \textsc{clf} \textit{daʔ} ‘vertically extended’ introduces clauses with a variety of adverbial, complement, and nominal-modifying functions. This needs more exposition than can be taken up here, but we note that it introduces readings of at least adverbial ‘when’ in \REF{ex:payne:15}–\REF{ex:payne:16}, ‘conditional’ in \REF{ex:payne:67}, and ‘purpose’ in \REF{ex:payne:84}. In \REF{ex:payne:76}, \textit{daʔ} occurs before an abstract nominal ‘hunger’, but the phrase with \textit{daʔ} communicates an adverbial ‘because’ notion. The complement function is illustrated in \REF{ex:payne:23c}, \REF{ex:payne:73c}, and \REF{ex:payne:85}, and a nominal-modifying (i.e. relative) function surfaces in \REF{ex:payne:27} and \REF{ex:payne:86}.

\ea\label{ex:payne:84} (004ZorrinoZorro 1.1)\\
\gll  soʔ  koñem  wʔo  soʔ  maečʔe  la-wa-naʕanqaʔ \textbf{daʔ}  na-wa-n  naʔ  owaqae\\
\textsc{clf:far} skunk  exist \textsc{clf}:far own  \textsc{pos3}-trap-\textsc{nmlz}:place \textsc{sub}  \textsc{b3}-trap-\textsc{nprog}  \textsc{clf}:near  peccary\\
\glt ‘The skunk had his own trapping place \textbf{in} \textbf{order} \textbf{to} trap the peccary.’ 
\z

\ea\label{ex:payne:85} (015Pesca3 1.2)\\
\gll  wačʔe  qo-d-ʔoya  \textbf{daʔ}  ne-matae-yi  ga=mʔ  n-oʕonek\\
because  \textsc{sbj.indf-a3}-fear  \textsc{sub}  \textsc{b3}-puncture-\textsc{pl}  \textsc{clf}:absent=\textsc{dem1:neut} \textsc{pos.indf}-fish\\
\glt ‘Because they feared \textbf{that} the fish would damage it.’ 
\z

\ea\label{ex:payne:86} (003ZorroPaloma 1.2)\\
\gll  yi-pit-etpa-lo  sa-wa  l-ʔaiʔte  soʔ  doqotoʔ \textbf{daʔ}  toʕomaqčiglo\\
\textsc{a3}-want-\textsc{prog-pl}  \textsc{clf}:far-\textsc{pl}  \textsc{pos3}-eyes  \textsc{clf}:far dove \textsc{sub} be.red\\
\glt ‘He wanted dove eyes \textbf{that} were red.’ 
\z

Historically, the \textit{daʔ} subordinator is likely connected to the ‘abstract’ nominal determining function of \textit{daʔ}. As \textit{daʔ} is the \textsc{clf} for abstract nominal referents, it is well-suited to mark nominalised propositions, which are typically rather abstract conceptual entities. These then come to serve as subordinate clauses.

 \subsection{\textit{Mʔe} as a relativiser}\label{sec:payne:7.4}

\textit{Mʔe} is a highly frequent demonstrative root (\tabref{tab:payne:2}). We have seen that in contrastive elicitation, it allows a ‘medial’ spatial contrast between \textit{hoʔ} ‘proximal’ and \textit{čaʔa} ‘distal visible’, and a visibility contrast with \textit{maʕa} ‘not visible’. However, it can occur with all deictic \textsc{clf}s to mark referents as ‘proximal/in the visual field’, ‘distal/(going) out of the visual field’, or ‘never seen/absent/nonreferential’; and it occurs with all posture/shape \textsc{clf}s. We also noted that \textit{mʔe} demonstratives can be used cataphorically, as in \REF{ex:payne:61}, though they are usually anaphoric. Given its range of collocations and uses, we conclude that \textit{mʔe} has developed a ‘distance/deictically neutral’ role \citep[211]{Himmelmann1996}.

Perhaps concomitant with its neutral deictic use, \textit{mʔe} has developed as the most common relativiser. It follows a head noun to anaphorically introduce a modifying relative clause, as in \REF{ex:payne:87}–\REF{ex:payne:88}. As a relativiser, it does not occur with classifiers or gender affixes (this is also true of the Western Toba cognate; \citealt[53]{Carpio2012}). It thus diverges from the \textsc{ddem} construction involving this root, which requires a classifier.

\ea\label{ex:payne:87} (052RelatoAnciana 52)\\
\gll  ad-apenaʔ  l-tʔa  diʔ=m ad-apenaʔ  \textbf{mʔe}  yi-wa\\
\textsc{pos2}-grandfather  \textsc{pos3}-father  \textsc{clf:hor=dem1:neut} \textsc{pos2}-grandfather   \textsc{dem1:neut}  \textsc{pos1}-spouse\\
\glt ‘the father of your grandfather (deceased) \textbf{that} was (my) husband’ 
\z

\ea\label{ex:payne:88} (071Sent03Comunidad 7)\\
\gll  naegaʔ  waʔa-ege  nqoʔ  gaʔ=nadik \textbf{mʔe}  yi-lot-ʔa  gaʔ=Joel \\
where be-opposite  when   \textsc{clf}:absent=road \textsc{dem1:neut}  \textsc{a3}-see-\textsc{obj.sg}   \textsc{clf}:absent=Joel\\
\glt ‘Where is the road \textbf{that} goes directly to (lit. sees) (the house of) Joel?’ 
\z

\textit{Mʔe} also introduces headless relative clauses, as in \REF{ex:payne:89}.

\ea\label{ex:payne:89} (013Pesca2 1.4)\\
\gll  yi-laʔa-ge  načʔe  yi-loʔt-ege \textbf{mʔe}  t-a-y-ʔa\\
 \textsc{a3}-see-\textsc{it} soon \textsc{a3}-see-opposite \textsc{dem1:neut}  \textsc{a3}-go-inside-\textsc{obj.sg}\\
\glt ‘He follows it (a bee, with his gaze) to see (the place) \textbf{where} it goes inside (of the honeycomb).’ 
\z

The relativising use of \textit{mʔe} might at first appear to be the \textsc{sdem} construction; but by itself, \textit{mʔe} is not synchronically attested as a proform. Nevertheless, it is largely associated with discourse anaphoricity. It has become the unmarked means to refer to a just-mentioned referent. Historically, this may have come about via an adjoined clause. That is, a conceivable earlier analysis of \REF{ex:payne:88} might have been ‘Where is the road, \textbf{that.one} (i.e. ‘road’) sees Joel?’ The relativiser function then developed by reanalysing the modifying clause (‘that [one] sees Joel’) as embedded. If this scenario is correct, then contra \citegen{Himmelmann1996} suggestion, it is not the distal member of the demonstrative paradigm which has extended its meaning to become grammaticalised as a relativiser, but the ‘middle (visible)’ and/or ‘neutral’ member of the paradigm.

\section{Conclusions}\label{sec:payne:8}

This study contributes to our understanding of the typological range of determiner and demonstrative systems. It has especially highlighted the demonstrative roots, which have not received much previous study in Guaykuruan languages.

Anyone who has examined a substantive discourse sample for any language, and over that sample tried to pecify “all and only” the componential semantic features that distinct demonstrative forms have, can surely attest that choice among demonstrative morphemes cannot be tied exclusively to literal spatial deixis nor to “clean” endophoric versus exophoric factors. The choices are always sensitive to speaker’s conceptualisation of referents on particular occasions of speaking, and to assumptions about the hearer’s continually changing state of mind in the endeavor to establish joint attention. With these important cautions in mind, the following is nevertheless a summary of our understanding of the prototypical functions of the demonstrative roots presented in \tabref{tab:payne:2} and \tabref{tab:payne:3}.

\begin{itemize}
\item \textit{hoʔ}  Adverbial; extended to participants when combined with \textsc{clf}s; visually or conceptually proximal (e.g. in the flow of the discourse); typically exophoric
\item \textit{mʔe}  Cognitively activated for speaker; assumed to be already activated for hearer; mostly endophoric and anaphoric
\item \textit{čaʔa}  Visually distal; typically exophoric
\item \textit{maʕa}  Unseen, uncertain; inferred
\item \textit{naqae}  Speaker instructs hearer to activate information that is assumed be already identifiable, known, or familiar
\end{itemize}

Relative to syntactic function, both deictic demonstratives (\textsc{ddem;} with all \textsc{dem1} roots) and the recognitional demonstrative (\textsc{rdem)} serve as determiners and as participant pronominals; but only (combinations with) \textit{hoʔ} function adverbially to signal location and time. The \textsc{sdem} with \textit{hoʔ} and some \textsc{ddem}s with \textit{hoʔ} function adverbially. It has been suggested that such a system, where the number of deictic distinctions in the pronominal domain supersedes the number of distinctions in the adverbial domain, may be comparatively rare \citep[19]{Levinson2018}. However, the Pilagá system somewhat corresponds to the most frequent type found in \citegen{Krasnoukhova2012} South American study, namely a system in which the same demonstrative form is used in participant-pronominal and adnominal functions (i.e. the \textsc{ddem}), but not in adverbial functions (which in Pilagá mostly uses the \textsc{sdem} with \textit{hoʔ}). Clearly, \textit{hoʔ} is a versatile element, occurring in the \textsc{sdem} construction as an adverbial pro-form and in the \textsc{ddem} construction for participant pronominal and determiner functions. \textit{Naqae} functions as part of a recognitional demonstrative. 

In our database, the demonstrative root tokens with exophoric function outnumber the tokens with endophoric function. Anaphoric uses are much more attested than cataphoric uses. Anaphora has been pointed out as a possible source for further grammaticalisation of \textit{mʔe} as a relativiser. This development suggests that it is not always the most distal (nor proximal) member of a demonstrative system that is subject to further grammaticalisation \citep[217]{Himmelmann1996}. What appears significant in the development of \textit{mʔe} as a relativiser is its endophoric+anaphoric profile, not a distal/proximal feature. If the subordinator \textit{daʔ} is historically related to the \textsc{clf} \textit{daʔ} ‘vertically extended’, the semantic pathway must be via the extension of \textit{daʔ} for abstract nominal concepts.

Corpus examination shows that essentially all determiners contain a \textsc{clf}. In fact, the basic determiner is just a \textsc{clf}, either deictic or postural. It would be communicatively unusual for essentially every nominal in discourse to be marked by a demonstrative; therefore we conclude that \textsc{clf}s do not have the typical usage profile of demonstratives.

The extension of some \textsc{clf}s and demonstrative roots into temporal and evidential/certainty meanings does not appear to be a widespread cross-linguistic feature of demonstrative systems. However, it is found in nearby Nivaĉle and Wichí; in Chorote \citep{Carol2011}; and in Movima, possibly Chapacuran Wari’, and some other South American languages \citep{Krasnoukhova2014}. The postural information found in the Guaykuruan determiner and demonstrative systems is rare, but it is also attested elsewhere, for example in the demonstrative system of the Chadic language Goemai \citep{Hellwig2018}. The evidential/(un)certainty semantics found in the Pilagá system is connected to speaker-anchored distance/non-visibility of referents. Evidential functions of demonstratives and determiners also appear to be typological rare. Further study is merited as to what extent these relatively unusual features occur in other languages of the Chaco, South America, and beyond.

\section*{Acknowledgments}

We are grateful to Pilagá speakers who have assisted with this research, especially Ignacio Silva and José Miranda. We also thank Belén Carpio, Olga Krasnoukhova, Manuel Otero, Yvonne Treis, and an anonymous reviewer for comments. This research is partially supported by NSF grant BCS 1263817 and by the Argentinian Consejo Nacional de Investigaciones Científicas y Técnicas (CONICET)

\section*{Abbreviations}

\begin{tabularx}{.45\textwidth}{lQ}
\textsc{a} & roughly active set of verbal person markers\\
\textsc{aug} & augmentative\\
\textsc{b} & roughly stative set of verbal person markers\\
\textsc{bclf} & basic classifier construction\\
\textsc{caus} & causative\\
\textsc{clf} & classifier\\
\textsc{col} & collective\\
\textsc{compl} & completive\\
\textsc{conj} & conjunction\\
\textsc{dem} & demonstrative\\
\textsc{ddem} & complex deictic demonstrative\\
\textsc{des} & desiderative\\
\textsc{dim} & diminutive\\
\textsc{dist} & distal\\
\textsc{f} & feminine\\
\textsc{hor} & horizontal\\
\textsc{hum} & human\\
\textsc{indf} & indefinite/nonspecific\\
\textsc{intg} & interrogative\\
\textsc{intj} & interjection\\
\textsc{intsf} & intensifier\\
\end{tabularx}
\begin{tabularx}{.45\textwidth}{lQ}
\textsc{it} & itive\\
\textsc{loc} & locative\\
\textsc{m} & masculine\\
\textsc{neut} & neutral deixis\\
\textsc{nexist} & non-existing\\
\textsc{nmlz} & nominaliser\\
\textsc{no.ext} & non-extended\\
\textsc{nprog} & non-progressive\\
\textsc{nvis} & non-visible\\
\textsc{obj} & object\\
\textsc{pauc} & paucal\\
\textsc{pl} & plural\\
\textsc{pos} & possessive\\
\textsc{prog} & progressive\\
\textsc{prox} & proximal\\
\textsc{rcg} & recognitional\\
\textsc{rdem} & complex recognitional demonstrative\\
\textsc{sbj} & subject\\
\textsc{sdem} & simple demonstrative\\
\textsc{sg} & singular\\
\textsc{sub} & subordinator\\
\textsc{ven} & ventive\\
\textsc{ver} & vertical\\
\textsc{vis} & visible\\
\end{tabularx}

\sloppy\printbibliography[heading=subbibliography,notkeyword=this]
\end{document}
