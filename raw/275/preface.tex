\documentclass[output=paper]{langsci/langscibook}
\ChapterDOI{10.5281/zenodo.3972826}

\author{András Bárány\affiliation{Leiden University}\and Theresa Biberauer\affiliation{University of Cambridge, Stellenbosch University, University of the West Cape}\and Jamie Douglas\affiliation{University of Cambridge}\lastand Sten Vikner\affiliation{Aarhus University}}

\title{Introduction}

\abstract{\noabstract}

\maketitle

\begin{document}

\noindent The three volumes of \emph{Syntactic architecture and its
consequences} present contributions to comparative generative linguistics that
\enquote{rethink} existing approaches to an extensive range of phenomena,
domains, and architectural questions in linguistic theory. At the heart of the
contributions is the tension between descriptive and explanatory adequacy which
has long animated generative linguistics and which continues to grow thanks to
the increasing amount and diversity of data available to us. As the three
volumes show, such data from a large number of understudied languages as well
as diatopic and diachronic varieties of well-known languages are being used to
test previously stated hypotheses, develop novel ideas and expand on our
understanding of linguistic theory.

The volumes feature a combination of squib- and regular-length discussions
addressing research questions with foci which range from micro to macro in
scale. We hope that together, they provide a valuable overview of issues that
are currently being addressed in generative linguistics, broadly defined,
allowing readers to make novel analogies and connections across a range of
different research strands. The chapters in Volume 2, \emph{Between syntax and
morphology}, and Volume 3, \emph{Inside syntax}, develop novel insights into
phenomena such as syntactic categories, relative clauses, constituent orders,
demonstrative systems, alignment types, case, agreement, and the syntax of null
elements.

The contributions to the present, first volume, \emph{Syntax inside the
grammar}, address research questions on the relation of syntax to other aspects
of grammar and linguistics more generally. The volume is divided into two
parts, dealing with language acquisition, variation and change (Part I), and
syntactic interfaces (Part II).

The chapters in Part I, \emph{Language acquisition, variation and change},
address questions such as the role of random drift in language change (Clark),
complexity in grammars (Bejar, Massam, Pérez-Leroux, and Roberge), and the
modelling of syntactic micro- and macro-variation across languages
synchronically in Bantu and Polynesian languages (van der Wal; Travis),
diachronically (Schifano and Cognola), and also across frameworks (Borsley;
Vincent and Börjars). The chapters by Haeberli and Ihsane, Fuß and Trips, and
van Kemenade provide novel insights into the diachrony of English verbs,
subjects, and prepositions, respectively, while Vincent and Börjars’
contribution shines light on the general notion of \enquote{heads} across time
and across current syntactic frameworks, and Roussou focuses on the diachrony
and grammaticalisation of complementisers.

Several chapters in Part II, \emph{Syntactic interfaces}, explore how syntax
and semantics interact in the context of decomposed functional structure,
expanding on influential proposals on fine-grained distinctions in the
\emph{v}-domain \parencite{Chomsky1995,Kratzer1996} and the structuring of
events
\parencite{Borer2003,Borer2005a,Borer2005b,Borer2013,Ramchand2008,Ramchand2018}.
Specific cases discussed here are the decomposition of passives (Biggs; Fadlon,
Horvath, Siloni, and Wexler), telicity (Hu), split intransitivity (J. Baker),
and verb-internal modifiers (Song). Questions about higher levels of clausal
architecture, such as the lack of verbal wh-expressions (Irurtzun) and
potential violations of the Final-over-Final Condition (Aboh) also feature in
this part. Other chapters, in turn, tackle issues in the nominal domain, such
as the syntax of nominal predication (Adger), a novel perspective on Binding
Principles A and B (Richards), and questions on the syntax of classifiers and
classifier languages (Lam). Finally, the syntax–phonology interface in several
Bantu languages is the topic of Hyman’s
chapter.\enlargethispage{1\baselineskip}

Taken together, then, the contributions to this volume, many of which have
clearly been influenced and inspired by
\textcite{Roberts2010,Roberts2012,Roberts2014,Roberts2019},~\textcite{RobRou2003},~\textcite{RobHol2010},
\textcite{BibRob2012,BibRob2015}, and \textcite{BibHolRob2014} give the reader
a sense of the lively nature of current discussion of topics in synchronic and
diachronic comparative syntax ranging from the core verbal domain to higher,
propositional domains.

{\sloppy
    \printbibliography[heading=subbibliography,notkeyword=this]
}

\end{document}
