\documentclass[output=paper]{langsci/langscibook}
\ChapterDOI{10.5281/zenodo.3972846}

\author{Ans van Kemenade\affiliation{Radboud University}}
\title{All those years ago: Preposition stranding in Old English}

% \chapterDOI{} %will be filled in at production

\abstract{This squib revisits the case for preposition stranding (P-stranding)
    in Old English as it was argued in the hot debate on \emph{wh}-movement in
    the 1980s. It looks at more recent literature on the relevant issues,
    finding that P-stranding in Old English warrants an analysis in terms of
    \emph{wh}-movement, which should allow for movement of a zero\is{null
    pronouns} prepositional object out of PP. Examination of the York corpus of
    Old English adds more detail to the picture known, but largely confirms the
findings so far.}

\maketitle

\begin{document}\glsresetall

\section{Background}

This squib follows up the discussion and analysis of preposition stranding
(P-stranding)  in specific types of \ili{Old English} \isi{relative clauses} in
\citet{vanKemenade1987}, which has featured in discussion of various issues in
more recent literature (\citealt{Alcorn2014}; \citealt{EmoFaa2014}).  My
treatment here is based on examination of the \textit{York corpus of Old
English} (\glsunset{YCOE}\gls{YCOE}) \parencite{Tayloretal2003}; it
re-addresses some of the theoretical issues, and reconsiders the analysis.

Examples of P-stranding\is{preposition stranding} in present-day English are
given in (\ref{ex:11.1}a--b), exemplifying P-stranding\is{preposition stranding} by
\textit{wh}-movement in \textit{wh}-relative clauses. \textit{Wh}-movement in
\isi{relative clauses} moves a constituent to Spec,CP (in modern terms), and
may involve long \textit{wh}-movement through an intermediate Spec,CP
(\ref{ex:11.1}b). This \textit{wh}-movement strategy allows preposition
stranding relatively freely in present-day English, as in
(\ref{ex:11.1}a,b):\pagebreak

\ea%1
    \label{ex:11.1}
	\ea That’s the guy [\textsubscript{CP} who\textsubscript{i} I told you about t\textsubscript{i}]
	\ex That’s the guy [\textsubscript{CP} who\textsubscript{i} I thought [\textsubscript{CP} t\textsubscript{i} I had told you about t\textsubscript{i}]]
	\z
\z

Preposition stranding in \ili{Old English} is, however, not allowed in constructions
comparable to \REF{ex:11.1}. Relative clauses\is{relative clauses} that involve \isi{movement} of an
overt relative pronoun are common in \ili{Old English} texts, but they do not feature
P-stranding (this is also true of \textit{wh}-questions, \citealt{Allen1977};
1980). When a prepositional object is relativised, it pied-pipes the
preposition along to Spec, CP, as in \REF{ex:11.2}:

\ea Blickling, 89.13 \parencite[270]{Allen1980}\label{ex:11.2}\\
    \gll Gehyr ðu arfæsta God mine stefne, [\textsubscript{CP} [mid ðære]\textsubscript{i}   ic earm {}  to ðe cleopie  t\textsubscript{i}]\\
    Hear thou merciful God my voice, {} with \hphantom{[}which I, poor (one), to thee call\\
\z

There are several other types of \isi{relative clauses} in \ili{Old English}
that do allow P-stranding, and in which stranding is indeed obligatory. These
share the property that they do not have an overt relative pronoun. I give
examples of relatives with the invariant complementiser\is{complementizers} \textit{þe}, with short
and long relativisation, in \REF{ex:11.3} (both from
\citealt[147--148]{vanKemenade1987}), of an infinitival relative in
\REF{ex:11.4}, and an example of an adjective$+$infinitive construction in
\REF{ex:11.5}.

\ea%3
    \label{ex:11.3} Orosius, 141, 22 \parencite[147]{vanKemenade1987}
    \ea
	\gll \& het         forbærnan þæt gewrit [\textsubscript{CP} 0\textsubscript{i} þe   hit  t\textsubscript{i} on awriten wæs]\\
		and ordered burn          the   writ {} {}           that it  {}    in written   was\\
	\glt ‘and ordered to burn the writ that it was written in’
	\ex
	\gll Đonne hie lecgeað ða tiglan beforan  hie    [\textsubscript{CP} 0\textsubscript{i} þe    him beboden wæs [\textsubscript{CP} 0\textsubscript{i} ðæt hie sceoldon ða ceastre Hierusalem  t\textsubscript{i} on awritan]]\\
		then    they put     the tiles   before    them  {} {}        that  them ordered was {} {} that they should the city Jerusalem             {} on  draw\\
	\glt ‘Then they put in front of them the tiles that they were ordered to draw the city of Jerusalem on.’
	\z
\ex Blickling, 157 \parencite[151]{vanKemenade1987}\label{ex:11.4}\\
    \gll Drihten, þu   þe           gecure þæt fæt      [\textsubscript{CP} 0\textsubscript{i}     t\textsubscript{i} on to eardienne ]\\
        Lord,      you yourself chose   that vessel  {} {} {}                  in to live\\
    \glt ‘Lord, you chose for yourself that vessel to live in.’
\pagebreak
\ex LS8 (Eust) 315 \parencite[266]{Fischeretal:2000}%5
    \label{ex:11.5}\\
    \gll Wæs seo wunung       þær    swyþe wynsum on  to wicenne \\
        was the dwelling-place  there very   pleasant  in  to live\\
    \glt `The dwelling-place there was very pleasant to live in.'
\z

A special case are relatives with \textit{that} as the relative pronoun form,
as we will see below.

P-stranding in constructions such as (\ref{ex:11.3}--\ref{ex:11.5}) featured prominently in the
1970’s and 1980’s debate on whether preposition stranding in the North and West
Germanic languages is derived by \textit{wh}-movement
(\citealt{Chomsky1977,ChomskyLasnik1977}; \citealt[286--297]{vanRiemsdijk1978};
\citealt{Vat1978}; \citealt{vanKemenade1987}), or by a second relativisation
strategy of deletion over a variable \citep{Maling1976,BresnanGrimshaw1978,Allen1977,Allen1980}, which may
involve long-distance deletion. This debate has been resolved to the extent
that, as far as the data can show us, both strategies are subject to subjacency
(\citealt{Allen1980}; \citealt{vanKemenade1987}): they both respect the complex
NP constraint and the \textit{wh}-island constraint, and occur only in
constructions that allow COMP to COMP \isi{movement}. In the terms of
\textcite{Chomsky1977,ChomskyLasnik1977}, this means that they must result from
\textit{wh}-movement. \citet{Vat1978}, in the wake of
\citet{vanRiemsdijk1978} follows \citet{Allen1977} in showing that Old English
has the same type of P-stranding\is{preposition stranding} by R-pronouns such
as \textit{þær} ‘there’, satisfying subjacency, and argues that
P-stranding\is{preposition stranding} in relatives without an overt pronoun
must be due to \textit{wh}-movement of \textit{þær}, with subsequent deletion
under identity with the antecedent.

\Citet{vanKemenade1987} presents another variant of this analysis. The general
ban on P-stranding\is{preposition stranding} in \citegen{vanRiemsdijk1978} analysis is accounted for by
the status of PP as a bounding node for subjacency. \ili{Dutch} P-stranding\is{preposition stranding} is
allowed because \ili{Dutch} allows an “escape hatch” to this ban in the form of
positions on the left of the preposition in (\ref{ex:11.6}a--b) that are designated for
R-pronouns, the only (overt) items in \ili{Dutch} grammar that strand a preposition:

\ea%6
    \label{ex:11.6}
    	\ea
	\gll Jan heeft het gisteren [\textsubscript{PP} \textit{daar} \textit{over} (*daar)] gehad.\\
        		Jan has    it   yesterday {}    there  about   {}         had\\
	\glt ‘Jan talked about that yesterday.’
	\ex Jan heeft het \textit{daar} gisteren \textit{over} gehad.
	\ex \textit{Daar} heeft Jan het gisteren \textit{over} gehad.
	\z
\z

(\ref{ex:11.6}c) shows that R-pronouns also move to Spec,CP. \Citet{vanKemenade1987}
proposes a parallel analysis for preposition stranding by \textit{þær} and by
various types of pronouns in \ili{Old English}: this is obligatory when the object of
the preposition is \textit{þær} ‘there’, and optional when the object is a
personal pronoun (both examples from
\citealt[117]{vanKemenade1987}):

\ea%7
    \label{ex:11.7}
	\ea Boeth, XXVII, 61, 20\\
	\gll þæt \textit{þær}   nane oðre    \textit{on} ne sæton   \\
       		that there no    others on  not sat \\
   	\glt ‘that no others sat (on) there’
	\ex WSgospel, Mt\\
	\gll þa    genealæhte  \textit{him} an man \textit{to}  \\
		then approached him  a   man to\\
	\glt ‘then a man approached him’
	\z
\z

\Citet[126--35]{vanKemenade1987} proposes that this type of pronoun
fronting\is{fronting!of pronouns}
represents a form of cliticisation\is{clitics} that is compatible with
\textit{wh}-movement, inspired by the fact that it applies to personal pronouns
as well, and by the fact that the positions where \textit{þær} and pronouns
occur in \ili{Old English} are special positions in \ili{Dutch} syntax more generally. She
extends this analysis to P-stranding\is{preposition stranding} in relatives without an overt pronoun as
zero cliticisation,\is{clitics} that is, P-stranding\is{preposition stranding} in the constructions exemplified in
(\ref{ex:11.3}--\ref{ex:11.5}) are cases of \textit{wh}-movement of a phonetically null variant of
\textit{þær} or a personal pronoun.

Let us now turn to a consideration of the merits of this approach in the light
of more recent literature, and based on an examination of the \textit{York
corpus of Old English} (YCOE, \citealt{Tayloretal2003}). These concern a number
of issues, which I would like to address in turn:

\begin{itemize}

    \item the locality conditions at play in the various constructions;

    \item other instances of P-stranding\is{preposition stranding} in
        relatives;

    \item the parallelism between P-stranding\is{preposition stranding} by
        \textit{þær} and pronouns, and P-strand\-ing\is{preposition stranding} in
        relatives (and related constructions) without an overt pronoun.

\end{itemize}

\section{Locality conditions}

There is no evidence that the relation between the CP of \textit{þe}-relatives
and the variable with which they are associated in any way violates the
subjacency condition, as noted above in relation to (\ref{ex:11.3}b). This
would indicate that \textit{þe}-relatives and related constructions in Old
English are derived by \textit{wh}-movement, of a zero\is{null pronouns}
clitic,\is{clitics} or a zero operator. Note that the \textit{þe}-relative is
by far the most frequent relative in \ili{Old English} (over 13,000 examples in
YCOE, including some 500 examples of P-stranding),  but long
\textit{wh}-movement is generally rare in Old English, and (\ref{ex:11.3}b)
is one of only two examples in the YCOE corpus of a \textit{þe}-relative with a
long-distance dependency. Note, nevertheless, that the facts are compatible
with subjacency, and I therefore assume, with \citet{vanKemenade1987}, that
they are derived by \textit{wh}-movement of a zero\is{null pronouns} element
that is identified under identity with the antecedent. This is in line with the
fact that they are most typically restrictive relatives.

\section{P-stranding by pro-forms and zero\is{null pronouns} pro-forms}

I now turn to a renewed assessment of the question to what extent it is
justified to parallel P-stranding\is{preposition stranding} by pronouns and \textit{þær}-\isi{adverbs} with
P-stranding  in constructions with an invariant complementiser\is{complementizers}. An argument in
favour of this parallel might be an observation in \citet{Alcorn2014} that
there are two spelling variants of the \ili{Old English} antecedents of the
prepositions \textit{by} and \textit{for}, \{be\} and \{for\} for unstranded
prepositions, and \{bi (big, bii, by, bie\} and \{fore\} for stranded
prepositions.  She argues that the choice between the two is prosodically
conditioned, with the stranded variant being prosodically independent. This
observation applies equally to prepositions stranded by \textit{þær} and
personal pronouns, and those stranded in \textit{þe}-relatives. This suggests
that the prepositions involved behave similarly. Observe, however, that this
does not necessarily mean that the stranding strategies are the same, it could
rather be determined by their pre-verbal or clause-final position.

\citet{Allen1980} argues against the parallelism between stranding by
\textit{þær} and personal pronouns and stranding in \textit{þe}-relatives:
\textit{þær}-relatives, which also involve stranding, had been introduced into
the debate by \citet{Vat1978}, who argues that \textit{þe}-relatives are really
\textit{þær}-relatives with subsequent deletion of \textit{þær} in Spec,CP
under identity with the antecedent. Allen argues that \textit{þær}-relatives
and \textit{þe}-relatives take different antecedents, with
\textit{þær}-relatives occurring with inanimate antecedents only, while
\textit{þe}-relatives take any antecedent. This observation is borne out by
examination of the YCOE corpus: \textit{þær}-relatives, totalling 315 in
number, are frequently found with NP antecedents that have no locative
connotation, but these are not animate,\is{animacy} they rather comprise rather
diverse notions such as \enquote*{(utter) darkness}, \enquote*{the heavenly
kingdom}, \enquote*{eternal life}, \enquote*{the course of things},
\enquote*{hellfire}, \enquote*{tortures}, \enquote*{the fairness of glory},
\enquote*{wedlock}, and so on. Allen also argues that a parallel between
stranding by personal pronouns and stranding in \textit{þe}-relatives is
problematic in view of the fact that the range of prepositions stranded by
pronouns is limited, whereas this is not in the case of
\textit{þe}-relatives.

An argument not mentioned by Allen which may also be important is that
\textit{þe}-relatives are dominantly restrictive, whereas
\textit{þær}-relatives are often non-re\-stric\-tive.

\ea Or\_6:3.136.4.2863%8
    \label{ex:11.8}\\
    \gll \& on oþerre wæs an gewrit, þær   wæron on awritene ealra þara ricestena monna noman\\
        and on other   was  a  writ,     there were    on written   all     the    richest men’s  names\\
    \glt ‘and in the other was a writ, on which were written the names of all the richest men’
\z

Note that the clause introduced by \textit{þær} in \REF{ex:11.8} is
ambiguous between a V2 main clause and a non-restrictive relative. This is
frequently the case in \textit{se}-relatives and \textit{þær}-relatives (cf.\
\citealt{LosvanKemenade2018}).  Surely the identification with the antecedent
must be subject to tighter restrictions in restrictive relatives, where the
relative clause serves to further identify the antecedent.

On the basis of the arguments reviewed so far, we may dismiss an analysis in
terms of  overt \textit{þær}/pronoun \isi{movement} to Spec,CP and subsequent
deletion, \emph{pace} \citet{Vat1978}, since on this analysis we would expect a
complete parallelism between \textit{ðær} and \textit{þe}-relatives, and this
is not feasible. \Citeauthor{vanKemenade1987}’s zero\is{null pronouns} cliticisation approach
allows a broader set of contexts for extraction, including personal pronouns.
Let us suppose that the zero\is{null pronouns} clitic\is{clitics} is in effect a zero\is{null pronouns} operator which
piggybacks on the escape hatch out of PP that is overtly around in the grammar,
and which can be used more liberally in restrictive relatives with a zero
operator, and other clauses where the identifying context for the zero\is{null pronouns} operator
is strict. There are several analyses to this effect available in the
literature. One is \textcite{Abels2003,Abels2012}, who casts the escape hatch
in terms of phase\is{phases} theory, making crucial use of a zero\is{null pronouns} parallel to R-stranding
in \ili{Dutch}. He proposes that \ili{Dutch} R-pronouns (including their zero\is{null pronouns} variant) are
base-generated on the left of P of a special class of zero\is{null pronouns} place prepositions.
An argument against this analysis is thus again that it works for some
prepositions only, whereas stranding in \textit{þe}-relatives is general for
all prepositions. Another analysis to the same effect is \citet{Matsumoto2013}.
He argues for a cyclic linearisation analysis that capitalises on the idea that
(zero) prepositional objects can be extracted in contexts where V and P have
the same head-complement \isi{parameters}. In effect, this means that extraction is
only possible when the complement of P is on its left (for whatever reason).
All analyses along these lines thus make use of a position on the left of P
that allows an escape hatch for extraction of the (zero) prepositional object.

At this point, it is also interesting to look at Old Norse, which has a
relativisation strategy with an invariant complementiser\is{complementizers} \textit{er} or
\textit{sem}, in which (zero) prepositional objects are relativised, stranding
the preposition (\citealt[260]{Faarlund2004}, see \citealt{Maling1976} for
present-day \ili{Icelandic}). Interestingly, Old Norse also has some form of
stranding by pronouns, although apparently on a more limited scale:
\citet{Faarlund2004} cites an example of pronoun topicalisation\is{topicalization} with stranding
(\citeyear{Faarlund2004}: 233, his (98)), and of an R-pronoun stranding a
preposition in a nonroot question (\citeyear{Faarlund2004}: 258, his (32c)).

\citet{EmoFaa2014} assume that \ili{Old English} had no preposition stranding, based
on \citegen{vanKemenade1987}’s analysis of stranding in relatives with
invariant complementisers\is{complementizers} as zero\is{null pronouns} cliticisation.\is{clitics} This glosses over the fact
that zero\is{null pronouns} cliticisation\is{clitics} is in fact van Kemenade’s analysis of P-stranding\is{preposition stranding} in
relatives with invariant complementisers\is{complementizers}, a construction clearly shared by Old
English and Old Norse.

An important remaining point are locality conditions: the evidence underlying
\citegen{Allen1980} and \citegen{vanKemenade1987} conclusion that the various
relativisation strategies respect subjacency is far from robust, although it is
consistent across clause types and extraction sites. \citet[181--186]{Abels2003}
argues that comparatives of inequality provide the one context which can only
involve operator \isi{movement}. Here, we run into a robustness problem once again:
there is only one relevant example of a comparative of inequality with
P-stranding in the YCOE corpus:

\ea Or\_2:5.48.36.938%9
    \label{ex:11.9}\\
     \gll to beteran tidun  þonne we nu    on sint \\
        to better    times than    we now in  are \\
    \glt ‘in better times than we are in now’
\z

We can conclude that the evidence is consistent with subjacency, although we
would like to base this on more robust data. I nevertheless maintain that
relatives with invariant complementisers\is{complementizers} and other \textit{wh}-related
constructions with zero\is{null pronouns} operators are \isi{movement} constructions. There is a
general ban on P-stranding\is{preposition stranding}, and I follow \textcite{Abels2003,Abels2012} in
taking PP to be a phase\is{phases} head. A zero\is{null pronouns} operator can be extracted out of PP, via
its Spec, or a Phase edge. I leave the details for further research (see e.g.\
\citealt{Walkden2017}, CGSW abstract). The fact that there was stranding was an
important basis for extension of stranding to other contexts over the Middle
English period.

\section{Other instances of P-stranding\is{preposition stranding} in relatives}

I now turn to further evidence for stranding in \ili{Old English}, which also occurs
in \textit{that}-relatives, albeit to a limited extent. This is an interesting
construction to consider, since the \textit{þe}-relative is presumably the
historical precursor of the present-day English \textit{that}-relative, which
is also typically assumed to involve \textit{wh}-movement, either of a null
operator, or of a \textit{wh}-pronoun with subsequent deletion under identity
with the antecedent. \ili{Old English} \textit{that}-relatives are ambiguous: we
could regard \textit{that} as an overt demonstrative\is{demonstratives} pronoun, which would make
the \textit{that}-relative a neuter gender instance of the \textit{se}-relative
(which is usually non-restrictive); we could alternatively regard it as an
early instance of an invariant complementiser\is{complementizers}. There is evidence both ways: of
the total of 2,743 examples of \textit{se}-relatives in the YCOE corpus, I
found 42 coded as \textit{se}-relatives with stranding. All of these have a
demonstrative as relative pronoun, and the complementiser\is{complementizers} \textit{þe}.  21 of
the cases have \textit{ðæt} as the relative marker, and have straightforward
neuter antecedents, such as \REF{ex:11.9b} with neuter \textit{sweord} as
antecedent; a further 12 have two \textit{þæt} forms, the neuter demonstrative\is{demonstratives}
pronoun \textit{ðæt} as antecedent, and \textit{þæt} as the relative marker, as
exemplified in \REF{ex:11.10}; two examples have a feminine antecedent
\REF{ex:11.11}. Four examples have a relative form other than \textit{þæt},
viz.  \textit{þære} (feminine genitive/dative singular, with a feminine
antecedent); \textit{þæm} (masculine/neuter dative\is{dative case} singular), or \textit{þa}
(masculine/neuter nominative/accusative plural). This once again includes
\REF{ex:11.11}, remarkably with a feminine antecedent
\textit{mægþe}.

\ea Bede\_2:10.138.4.1327%9
    \label{ex:11.9b}\\
    \gll  Þa     sealde se  cyning him  {}    \textit{sweord}, \textit{þæt} he hine      mid  gyrde; …  \\
        then gave   the king    him (a) sword,   that he himself with girded\\
    \glt ‘Then the king gave him a sword, which he girded himself with’
\z

\ea CP:46.351.5.2368%10
    \label{ex:11.10}\\
    \gll …, sua him læs    licað    \textit{ðæt} \textit{ðæt} hie  to   gelaðode sindon, \\
        …  so   them less pleases that that they to called      are\\
    \glt `…, the less they are pleased with that to which they are called’
\z

\ea Bede\_5:22.478.23.4805%11
    \label{ex:11.11}\\
    \gll Ond þonne Norþanhymbra   \textit{mægþe}   \textit{þæm} Ceolwulf se    cyning in cynedome ofer is,   \\
        and then    Northumbrians’ province that   Ceolwulf  the king  in    kingship over is\\
    \glt ‘And in the province of Northumbria, over which King Ceolwulf reigns’.
\z

The majority of these examples ($21+12$) is thus compatible, on the one hand,
with a pronominal interpretation of \textit{that} (since in most cases the
antecedent is neuter) and on the other hand with that of an invariant
complementiser (assuming that P-stranding\is{preposition stranding} involves
zero operator \isi{movement}).  The cases with a feminine antecedent (2 in total)
suggest that \textit{that} is an invariant complementiser\is{complementizers}, since a gender
mismatch between antecedent and relative pronoun would not be expected. The
cases with pronominal forms other than \textit{that} (4 in total) suggest, on
the other hand, that \isi{movement} of the pronoun strands the preposition, since the
form of the pronoun is incompatible with an interpretation as invariant
complementiser. Old English \textit{that}-relatives with stranding thus suggest
some evidence for P-stranding by an overt relative pronoun, in a specific
context.

The YCOE corpus also features two examples of relatives coded as \textit{se} \textit{þe} relatives with P-stranding\is{preposition stranding}. One of these seems to be unreliable, as it is presumably not a \textit{se}-relative but a \textit{þe}-relative on an antecedent that is appositive in the context. \REF{ex:11.13} looks like a bona fide case of a \textit{se} \textit{þe} relative with P-stranding\is{preposition stranding}.

\ea Bede\_4:31.376.2.3751%13
    \label{ex:11.13}\\
    \gll Swylce eac ealle ða  hrægl, þa       ðe   he mid gegearwad wæs, \\
        such    also all    the robes, which that he with attired was \\
    \glt ‘Also all the robes in which he was attired, …’
\z

The observations about \emph{that}-relatives fit well with the analysis
sketched here: \textit{þæt} is at this stage of the language clearly to some
extent ambiguous between relative pronoun status and its later grammaticalised\is{grammaticalization}
complementiser status, witness the fact that it features a substantial number
of cases of P-stranding\is{preposition stranding}. We also find the first
instances of unambiguous P-stranding by a relative pronoun as in
\REF{ex:11.13}.

In conclusion, we can say that the findings of the 1980’s literature on
P-strand\-ing largely hold up. This applies to the theoretical analysis (any
analysis must somehow allow for relatively free extraction out of PP when the
prepositional object is a zero\is{null pronouns} element), as well as to the
factual coverage now allowed by the \gls{YCOE} corpus (we can present more
detail now, but there are no facts that were glossed over earlier).



\printchapterglossary{}

{\sloppy
\printbibliography[heading=subbibliography,notkeyword=this]
}

\end{document}
