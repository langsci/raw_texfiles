\documentclass[output=paper]{langsci/langscibook} 
\ChapterDOI{10.5281/zenodo.5524286}
\author{Koji Shimamura\affiliation{Kobe City University of Foreign Studies}}
\title[Against embedded modal as control in Japanese]
      {Against embedded modal as control in Japanese: Its relevance to the implicational complementation hierarchy}  
\abstract{In this chapter, I will investigate the nature of one specific sentential complementation in Japanese that has been considered to be a case of obligatory control in the syntactic literature: embedding the modal element, \textit{yoo}. I will propose, contrary to \citet{fujii2006,fujii2010,takano2010} among others, that it does not exemplify a case of such a control construction, giving another way to get it via indexical shifting. Then, I will also discuss the relevance of the analysis to be proposed in terms of the implicational complementation hierarchy put forth by \citet{wurmbrandlohninger2020}.}

\begin{document}
\SetupAffiliations{mark style=none}
\maketitle

\section{Introduction}
This chapter reconsiders one specific construction in Japanese that has been analyzed as a control complement and hence it has been assumed to involve a PRO subject (\citealt{fujii2006,fujii2010,takano2010,uchibori2000} among others), where the volitional modal \textit{yoo} is embedded under several kinds of matrix predicates.\footnote{\textit{yoo} has two instantiations, which are phonologically conditioned: \textit{yoo} appears when a verb stem ends with a vowel whereas \textit{oo} appears when it ends with a consonant. However, the polite suffix \textit{mas} is exceptional due to its irregular inflectional paradigm, and we have \textit{mas-yoo} in lieu of the expected \textit{mas-oo}.} As we will see, all the instances of the \textit{yoo} complement are \textit{prima facie} the same, but a closer look into their semantic properties divulges that they are different in accordance with the types of matrix selecting verbs. Specifically, verbs like \textit{kime-} `decide' exhibit more signatures of clausal complexity in their \textit{yoo} complements than verbs like \textit{kokoromi-} `try'. This syntactic disparity regarding the complexity of the \textit{yoo} complement clause is traceable in terms of the semantic properties of the embedded clauses in general. This correlation between the semantic properties of a given complement clause and its syntactic realization is now captured in terms of the universal generalization proposed by \citet{wurmbrandlohninger2020}, viz. the the implicational complementation hierarchy.

This chapter goes as follows: in Section \ref{shimamuS2}, I will go over the analysis of English infinitives proposed by \citet{Wurmbrand2014}, discussing how the semantic properties of embedded clauses affect the syntactic architecture of them as well as a recent argument made by \citet{wurmbrandlohninger2020} regarding the implicational complementation hierarchy in terms of the clause size of such infinitival complements. Then, Section \ref{shimamuS3} will look into the nature of the \textit{yoo} complement connecting to different matrix verbs, showing that the \textit{yoo} complement to verbs like \textit{kime-} `decide' is compatible with independent temporal construal and enjoys various subject interpretations, which state of affairs is however not observed in the \textit{yoo} complement clause selected by other verbs like \textit{kokoromi-} `try'. In Section \ref{shimamuS4}, I will put forth my analysis, contending that the pertinent contrast is due to the size of the embedded \textit{yoo} complement. The \textit{yoo} complement of \textit{kokoromi-} `try' is very small, so that it is, as we will see, able to undergo long passivization. In Section \ref{shimamuS5}, I then show that the size of the embedded \textit{yoo} is not absolute, and the clause size can be expanded even for `try' verbs if other syntactic/semantic factors such as the presence of an overt embedded subject and the possibility of temporal independence of the embedded clause are taken into consideration. Section \ref{shimamuS6} will then conclude.
 
\section{English infinitives and the implicational complementation hierarchy}\label{shimamuS2}
\citet{Wurmbrand2014} proposes an intriguing proposal regarding what has been called the control infinitive (CI). Her approach posed a significant challenge to the widely accepted perspective that the CI is tensed whereas other instances of infinitives (i.e. ECM/raising) are untensed. This disparity is most conspicuously expressed in the ``null Case'' approach to licensing a PRO (see \citealt{martin2001} and references therein). That is, the subject of the CI complement is licensed as a PRO due to the availability of the pertinent null Case while the subject of the ECM/raising counterpart must enter into a structural Case dependency with the matrix $v$ (ECM) or T (raising) to be Case-marked.

However, even though the following three alleged CI complements are \textit{prima facie} similar for their verbal/infinitival morphology, they exhibit different properties for their temporal interpretations.

\ea\label{shimamu1}
\begin{xlist}
\ex\label{shimamu1a} Yesterday, John decided/wanted/planned to leave tomorrow.
\ex\label{shimamu1b} Yesterday, John tried/began/managed to leave (*tomorrow).
\ex\label{shimamu1c} Yesterday, John claimed to be leaving $\{$right then/tomorrow$\}$/*to leave tomorrow.\hfill\citep[][408]{Wurmbrand2014}
\end{xlist}
\z
In \REF{shimamu1a}, the CI complement of e.g. \textit{decide} denotes future irrealis, allowing modification by \textit{tomorrow}. In contrast, such an interpretation is prohibited in \REF{shimamu1b} and \REF{shimamu1c}: the CI complement of \REF{shimamu1b} is \textit{simultaneous} in the sense that the matrix verb and the embedded verb do not permit independent adverbial modification, and the same holds for \textit{claim} in \REF{shimamu1c}, which is a case of the propositional CI, according to \citet{Wurmbrand2014}. It is not a \textit{bonafide} future irrealis complementation due to the impossibility of *\textit{to leave tomorrow}. Rather, the CI complement of \REF{shimamu1c} is construed as temporally simultaneous with the matrix predicate with the adverb \textit{right then}, or as a planned/scheduled future with \textit{tomorrow} (like \textit{I'm leaving tomorrow} in the matrix context).

\citet{Wurmbrand2014} contends that all the CI complements in \REF{shimamu1}, even \REF{shimamu1a}, are tenseless, with the structure where the finite future tense is decomposed into T and \textit{woll}P (see \citealt{abusch1985,Abusch1988} for \textit{woll}P). When T is encoded as \textsc{[past]}, the combination of T and  \textit{woll} will be spelled out as \textit{would}.

\ea\label{shimamu2} Finite \textit{will}\\
\Tree [.TP {T\newline\textsc{[pres]}} [.\textit{woll}P \textit{woll} \qroof{\ldots}.$v$P ] ]
\begin{tikzpicture}[overlay]
\draw [thick,decorate,decoration={brace,amplitude=10pt}]
(-0.9,-2.5) -- (-3.2,-2.5) node [black,midway,yshift=-0.6cm,xshift=-0.15cm] 
{\itshape will};
\end{tikzpicture}
\z
Given \REF{shimamu2}, the CI complement in \REF{shimamu1a} is \textit{woll}P and that of \REF{shimamu1b} and \REF{shimamu1c} lacks both TP and \textit{woll}P. Since TP is absent in \REF{shimamu1a}, the future orientation of the embedded \textit{woll}P is not absolute. In this connection, consider \REF{shimamu3}. In \REF{shimamu3a}, the matrix verb and the embedded verb are each modified by different adverbs, \textit{a week ago} and \textit{yesterday}, respectively. This is not possible in \REF{shimamu3b}. This is because the embedded clause has finite \textit{will} that results from \REF{shimamu2}, and \textit{will}'s T is absolute in the sense that it refers to the utterance time (the speaker's now). The availability of \textit{yesterday} in \REF{shimamu3a} thus indicates that the future orientation of the CI complement in \REF{shimamu3a} is ``relativized'' to the matrix past tense (Leo's now). Therefore, such a complement lacks tense (hence TP), and the future construal is rendered by the modal \textit{woll}.

\ea\label{shimamu3}
\begin{xlist}
\ex\label{shimamu3a} Leo decided a week ago to go to the party yesterday.
\ex\label{shimamu3b} Leo decided a week ago that he will go to the party (*yesterday).
\end{xlist}\hfill\citep[][413]{Wurmbrand2014}
\z 

Turning to the other CI complements in \REF{shimamu1}, the simultaneous interpretation comes, under \citeauthor{Wurmbrand2014}'s analysis, in the form of bare VP \citep[][cf.]{wurmbrand2001}. Details aside, we have at least the following two types of infinitives in English.\footnote{\citet{Wurmbrand2014} also discusses the structure where an aspectual projection, AspP, is projected in tenseless (simultaneous) infinitives. However, I abstract away from it in this paper.} 

\ea\label{shimamu4}
\begin{multicols}{2}
\begin{xlist}
\ex\label{shimamu4a} \Tree [.VP [.V \textit{decide/want/plan} ] [.\textit{woll}P \textit{woll} \qroof{\ldots}.$v$P ] ]
\ex\label{shimamu4b} \Tree [.VP [.V \textit{try/begin/manage/claim} ] \qroof{\ldots}.VP ]
\end{xlist}
\end{multicols}
\z

Now, what is interesting at this point is that the infinitive morphology \textit{per se} does not tell us much about the syntactic structure of a given CI complement. Rather, its syntactic interior becomes discernible through examining the properties of selecting verbs.

In this connection, \citet{wurmbrandlohninger2020}, examining various European languages, put forth a hypothesis concerning the size of complement clauses that is defined in terms of semantics. According to them, there are three types of complements: \textit{propositions}, \textit{situations} and \textit{events}. Propositions involve speech/epistemic contexts, and they are temporally independent and anchored to the embedding context. Situations denote emotive and irrealis contexts. They lack speaker/utterance-oriented properties, but they have their own time and world parameters. Events are semantically a property of events, lacking their own context/time/world parameters. Then, with this trichotomy, clauses that denote propositions are more clausal than those which denote situations, which are in turn more clausal than those which denote events. This structural differences are reflected in various syntactic, morphological and semantic properties, and the presence of some property X in one type of complement implies X's existence or absence in another type of complement left/right-adjacent to it in the clause-size-defining scale, termed the implicational complementation hierarchy (ICH) \citep[][6]{wurmbrandlohninger2020}.

\begin{table}
\caption{Implicational complementation hierarchy}
\label{shimamutab:1:ICH}
 \fittable{\begin{tabular}{lcl} 
  \lsptoprule
   \textsc{most independent} &   & \textsc{least independent} \\ 
  \textsc{least transparent}  &   Proposition $\gg$ Situation $\gg$ Event  & \textsc{most transparent}  \\
   \textsc{least integrated}  &   &    \textsc{most integrated} \\
  \lspbottomrule
 \end{tabular}}
\end{table}

For instance, if a languages allows clitic climbing from the situation complement, then it should be the case that the event complement also allows it. According to \citet{wurmbrandlohninger2020}, the minimal structures of the three types of complements are the following \citep[][33]{wurmbrandlohninger2020}:

\begin{multicols}{3}
\ea\label{shimamu4.1}
\begin{xlist} 
\ex \Tree [.\textit{Proposition} \textit{believe} \qroof{\ldots}.Op ]
\ex \Tree [.\textit{Situation} \textit{decide} \qroof{\ldots}.TMA ]
\ex \Tree [.\textit{Event} \textit{try} \qroof{\ldots}.Theta ]
\end{xlist}
\z
\end{multicols}

\noindent Op stands for the operator domain, CP, and TMA signifies the tense-modal-as\-pect domain. Theta corresponds to the argument structure domain so that it is defined in terms of $v$P (VP). Since these are the minimal structures, it is still possible to have e.g the situation complement structured as a CP, but it will never be the case that the situation complement comprises only the Theta structure.

As we will see, the same state of affairs holds for what has been analyzed as control in Japanese.

\section{Control in Japanese?}\label{shimamuS3}
\subsection{Embedding \textit{yoo}}
Although there is no sign of non-finiteness in Japanese in the sense of European languages like English, it has sometimes been argued that Japanese has PROs and hence control constructions. One such case to be discussed throughout the rest of this paper involves a volitional modal element, \textit{yoo} \citep{fujii2006,fujii2010,uchibori2000,shimamura2015}. Observe:\footnote{Regarding the matrix selecting verbs, all the above English examples cannot be replicated in Japanese. This is because some of them cannot take a \textit{yoo} complement, and selects a different type of embedded clause. For instance, `start' in Japanese is \textit{hazime-}, and the complement clause of this verb is rendered via the complex predicate formation/bare VP-complementation or Restructuring in the sense of \citet{wurmbrand2001}. I thus discuss Japanese data only for those verbs that are compatible with the \textit{yoo} complement.}
\ea\label{shimamu5}
\begin{xlist}
\ex\label{shimamu5a} \gll Kinoo, Taroo-wa [ asu syuppatu-si-yoo-to ] $\{$kime/omot$\}$-ta.\\
yesterday Taro-\textsc{top} {} tomorrow departure-do-\textsc{mod-rep} {} \phantom{$\{$}decide/think-\textsc{past}\\
\glt `Yesterday, Taro $\{$decided to leave/thought about leaving$\}$ tomorrow.'
\ex\label{shimamu5b} \gll Kinoo, Taroo-wa [ (*asu) syuppatu-si-yoo-to ] $\{$kokoromi/si$\}$-ta.\\
yesterday Taro-\textsc{top} {} \phantom{*(}tomorrow departure-do-\textsc{mod-rep} {} \phantom{$\{$}try/do-\textsc{past}\\
\glt `Yesterday, Taro tried to leave (*tomorrow).'
\end{xlist}
\z
\REF{shimamu5a} is just like \REF{shimamu1a} in English, allowing two independent time adverbs to occur. In contrast, the event of \textit{leaving} in the complement clause must be simultaneous with the matrix verbs in \REF{shimamu5b} (but see \REF{shimamu34a} below). However, notwithstanding this apparent similarity between in English and Japanese, I will argue that \REF{shimamu5} does not substantiate control constructions, at least in the sense that it does not involve an obligatorily controlled PRO.

\subsection{\textit{Yoo} in the matrix context and the interpretation of the agent}
The first task I would like to undertake to do is consider whether the complement clauses in \REF{shimamu5} are really control complements. In this connection, note that the clause suffixed by \textit{yoo} is in fact used as a root clause that expresses the speaker's intention. Therefore, it is rather difficult to have non-1st person subjects in the \textit{yoo} sentence, and previous researches discussing this modal observe that the 2nd or 3rd person pronouns are incompatible with \textit{yoo} (but see \citealt{moriyama1990} and \citealt{Narrog2009}). The following judgment represents the standard (widely accepted) observation reported in the literature \citep[cf.][]{fujii2006}.

\ea\label{shimamu6}
\gll $\{$Boku/\#kimi/\#kanozyo$\}$-wa syuppatu-si-yoo.\\
\phantom{$\{$}I/You/She-\textsc{top} departure-do-\textsc{mod}\\
\glt `$\{$I/\#You/\#She$\}$ will leave.'
\z
However, as \citet{shimamura2015} points out, the choice of \textit{kimi} `you' and \textit{kare} `she' becomes possible when an appropriate context is set up. For instance, if I am in a privileged position by which I can command `you' or `her' to leave, then I can utter \REF{shimamu6} with \textit{kimi} or \textit{kanozyo}. 

Also equally important to mention here is the fact that plural subjects are possible, and again, the intention to make a given action (here, \textit{leaving}) to happen is ascribed to the speaker:

\ea\label{shimamu7}
\gll $\{$Boku/kimi/kanozyo$\}$-tati-wa syuppatu-si-yoo.\\
\phantom{$\{$}I/You/She-\textsc{pl-top} departure-do-\textsc{mod}\\
\glt `$\{$We/You (\textsc{pl})/They$\}$ will leave.'
\z
Therefore, we need to dissociate the intention holder from the actual doer; the simplest case is such that the former and the latter are identical, hence the case of \textit{I will leave} in \REF{shimamu6}. Then, I assume that \textit{yoo} has its person parameter fixed to the 1st person, denoting the speaker's volitional attitude as shown in \figref{shimamu8}, where the attitude holder of \textit{yoo} is expressed in terms of the person feature on the modal head, and the actual doer (agent) is merged to Spec-$v$P.
Therefore, the agent can be anything, be it 1st person, 2nd person, 3rd person, singular or plural.

\begin{figure}
\caption{Structure of \textit{yoo} clause\label{shimamu8}}
\Tree [.TP [.ModP [.$v$P DP [.$v'$ \qroof{\textit{leave}}.VP $v$ ] ] [.Mod {\textit{yoo}\newline\textsc{[person: 1]}} ] ] T ] 
\end{figure}

Then, what is expected is that when embedded as a(n apparent) control complement, the embedded agent does not have to be identical to the matrix attitude holder. This prediction is borne out, insofar as the selecting verb is \textit{kime-} `decide' or \textit{omow-} `think' among others. Observe \REF{shimamu9}, where I give a silent subject in the complement clause as $e$. The embedded agent has other members in addition to Taro (represented as +).

\ea\label{shimamu9} Context: Taro, who is the leader of his trekking team, was thinking about when they should leave, and he reached the conclusion:\\
\gll Taroo$_1$-wa [ asu $e_1$+ syuppatu-si-yoo-to ] $\{$kime/omot$\}$-ta.\\
Taro-\textsc{top} {} tomorrow {} departure-do-\textsc{mod-rep} {} \phantom{$\{$}decide/think-\textsc{past}\\
\glt `Taro$_1$ $\{$decided $e_1$+ to leave/thought about $e_1$+ leaving$\}$ tomorrow.'
\z
This is like partial control \citep[cf.][]{landau2000}. Embedding \textit{yoo} also yields a split-control-like construction as in \REF{shimamu10}, but it is also possible to include additional members other than Taro and Jiro for the embedded agent interpretation.

\ea\label{shimamu10} Context: Taro, who is the leader of his trekking team, was thinking about when they should leave, and he reached the conclusion, which he told to Jiro:\\
\gll Taroo$_1$-wa Ziroo$_2$-ni [ asu $e_{1+2}$+ syuppatu-si-yoo-to ] it-ta.\\
Taro-\textsc{top} Jiro-\textsc{dat} {} tomorrow {} departure-do-\textsc{mod-rep} {} say-\textsc{past}\\
\glt Lit. `Taro$_1$ told Jiro$_2$ $e_{1+2}$+ to leave tomorrow.'
\z
In passing, \REF{shimamu10} is also fine in the context where Taro commands Jiro to leave (with or without other members) tomorrow. In this case, the embedded agent does not include the attitude holder, reminiscent of the `you' option in \REF{shimamu6} and \REF{shimamu7}. In contrast, the simultaneous complement in \REF{shimamu5b} (like English) never allows partial control, whence it must be like a case of exhaustive subject control.\footnote{Note that \textit{kokoromi-}/\textit{su-} cannot take a dative argument, so that they never allow an object-control-like interpretation.}

\ea\label{shimamu11} \gll Taroo-wa [  $e_1$(*+) syuppatu-si-yoo-to ] $\{$kokoromi/si$\}$-ta.\\
Taro-\textsc{top} {} {} departure-do-\textsc{mod-rep} {} \phantom{$\{$}try/do-\textsc{past}\\
\glt `Taro$_1$ tried $e_1$(*+) to leave.'
\z

To recap, the modal \textit{yoo} is not limited to the embedded context, which is different from the CI in English, and the silent subject (agent) of the control-like construction in Japanese readily accommodates various interpretations like partial, split and even partial split control. This suggests that \textit{yoo} is not a case of obligatory control (OC), for the OC PRO is not assumed to support such a wide variety of interpretational options of the silent agent (see \citealt{landau2000} and \citet{hornstein1999,hornstein2003} for the opposing views regarding whether split control exists and (if so) is a case of OC). As we will see next, the Japanese construction under discussion passes other OC diagnostics. However, I suggest that this state of affairs is illusory, due to the shifted person parameter of \textit{yoo} via indexical shifting.

\subsection{Obligatory control diagnostics and indexical shifting of \textit{yoo}}\largerpage[2]
The wide range of interpretational possibilities for the embedded agent strongly suggests that embedding \textit{yoo} is not a case of OC. However, other diagnostics such as the availability of \textit{de se}/\textit{de te} seem to tell us that it is an instance of OC. For instance, \citet{fujii2006} gives:

\ea\label{shimamu12} Context: Hiroshi planned to go abroad. He had already got his passport and made a visa available recently. One day, he went to drinking and came home badly drunk. He found the passport on the table, without remembering that this was what he himself got from the embassy. Looking at the picture on the passport and the visa, he thinks, ``I don't know who this guy is, but he seems to be planning to go abroad soon. I wish I could!''\\
\gll \# Hirosi$_1$-wa [ $e_1$ gaikoku-ni ik-oo-to ] omot-te-i-ru.\\
{} Hiroshi-\textsc{top} {} {} foreign.country-to go-\textsc{mod-rep} {} think-\textsc{asp-cop-pres}\\
\glt `Hiroshi thinks of going abroad.' \citep[][106]{fujii2006}
\z
In the provided context in \REF{shimamu12}, the sentence sounds infelicitous. Also, \citet{fujii2006} shows, among others, the antecedent of the alleged OC PRO of the \textit{yoo} complement must be ``one-clause up'', namely, the ban on long-distance antecedents. Witness:

\ea[*]{\label{shimamu13}\gll Karera-wa [ Hirosi-ni [ $e$ otagai-o naguri-a-oo-to ] omow ]-ase-ta.\\
they-\textsc{top} {} Hiroshi-\textsc{dat} {} {} each.other-\textsc{acc} hit-\textsc{recip-mod-rep} {} think \phantom{]-}\textsc{caus-past}\\
\glt Lit. `They$_1$ made Hiroshi think $e_1$ to hit each other.' \citep[][104]{fujii2006}}
\z
This example shows that the highest subject cannot be the antecedent of the silent subject in the most embedded clause. These data plus the other tests \citet{fujii2006} discusses may lead us to conclude that the \textit{yoo} complement can be an OC complement (setting aside partial control discussed above).
 
However, recall that \textit{yoo} in the matrix context must have the attitude holder is the 1st person, and this restriction is lifted when \textit{yoo} is embedded. In this sense, it can be a case of indexical shifting, and relevant to this, Japanese allows imperatives to be embedded, concerning which \citet{saueryatsu2014} propose that it is possible due to the indexical shifting of the imperative verb. Given this, I assume with \citet{saueryatsu2014} that in the context where the reporting particle \textit{to} is employed indexical shifting applies obligatorily.\footnote{\citet{sudo2012} observes that indexical shifting is optional. However, his discussion is based on the shifted indexicality of the 1st person pronoun, and its shiftablity is controversial; see \citet{saueryatsu2014} and \citet{shimamura2018} for a detailed discussion on this.} I assume that the monster operator is located in the reporting particle.

\begin{figure}
\caption{\textit{yoo} complement and indexical shifting\label{shimamu14}} 
\Tree [.RepP [.{\ldots} [.ModP \qroof{\ldots}.$v$P [.Mod {\textit{yoo}\newline\textsc{[person: 1]}} ] ] {\ldots} ] [.Rep \textit{to}$_{\includegraphics[width=3mm]{figures/monster}}$ ] ]
\end{figure}

Then, the obligatory \textit{de se} construal is due to indexical shifting, since it has been shown that the first person pronoun of Zazaki, when shifted, must be interpreted as a self-ascription by the matrix subject. Observe:\footnote{However, see \citet{deal2020} for the cases where indexical shifting does not lead to the obligatory \textit{de se} interpretation.}

\ea\label{shimamu15} Zazaki's indexical shift  \citep[][79]{anand2006}\\ 
\gll Heseni va [ {kɛ} {ɛz} newɛsha. ]\\
Hesen.\textsc{obl} said {} that I be.sick.\textsc{pres} {} \\
\glt `Hesen said that he was.'\\
\begin{xlist}
\ex  Hesen says, ``I am sick today.''
\ex[\#]{Hesen, at the hospital for a checkup, happens to glance at the chart of a patient's blood work. Hesen, a doctor himself, sees that the patient is clearly sick, but the name is hard to read. He says to the nurse when she comes in, ``This guy is really sick.''}
\end{xlist}
\z

Turning to the ``one-clause up'' requirement, the locus of the reporting particle accounts for the impossibility of long-distance antecedents. That is, since indexical shifting via \textit{to} in \REF{shimamu13} is implemented relative to Hiroshi's context, it is impossible to have the silent subject interpreted relative to the matrix subject, to the extent that the former is identical to Hiroshi, the most natural interpretation.

Then, what is the silent subject? I suggest that it is a silent pronoun, \textit{pro}, readily available in the Japanese grammar. In the default cases of subject-control-like examples i.e. \REF{shimamu5}, the attitude holder of the shifted \textit{yoo} and the embedded agent should be regarded as identical, so that we apparently get the obligatory \textit{de se} reading. However, as we saw above, the agent does not have to be identical to the attitude holder, and such being the case, it is like a command. For instance:

\ea\label{shimamu16} \gll {Yamada sensei}-wa Taroo$_1$-ni [ $e_{1}$ motto ronbun-o kak-oo-to ] it-ta.\\
{Prof. Yamada}-\textsc{top} Taro-\textsc{dat} {} {} more paper-\textsc{acc} write-\textsc{mod-rep} {} say-\textsc{past}\\
\glt Lit. `Prof. Yamada told Taro$_1$ $e_{1}$ to write more papers.'
\z
This example seems to be a case of obligatory object control, hence the obligatory \textit{de te} reading. Nevertheless, we can come up with the following example:

\ea\label{shimamu17} Context: Yuta is hosting a party. He hears that a certain waiter named Yusuke is being a nuisance. Yuta tells the nearest waiter, ``Yusuke has to go.'' Unbeknownst to him, he's talking to Yusuke.\\
\gll Yuuta-wa Yuusuke$_1$-ni [ $e_1$ koko-kara dete-ik-oo-to ] it-ta.\\
Yuta-\textsc{top} Yusuke-\textsc{dat} {} {} here-from leave-go-\textsc{mod-rep} {} say-\textsc{past}\\
\glt `Yuta said to Yusuke$_1$ $e_1$ to leave here.'
\z
This example clearly shows that the pertinent \textit{de te} reading can be absent.\largerpage

Another example that can be regarded as problematic to the OC approach to the embedded \textit{yoo} is concerned with the sloppy reading under ellipsis. Building on the fact that OC only allows the sloppy reading in the context of ellipsis \citep{hornstein1999}, \citet{fujii2006} observes that examples like \REF{shimamu18} only support the sloppy reading. 

\ea\label{shimamu18}
\begin{xlist}
\ex \gll Taroo-wa Ziroo-ni [ $e$ ie-ni kaer-oo-to ] it-ta.\\
Taro-\textsc{top} Jiro-\textsc{dat} {} {} house-to return-\textsc{mod-rep} {} say-\textsc{past}\\
\glt `Taro told Jiro to go home.'
\ex \gll Saburoo-ni-mo da.\\
Saburo-\textsc{dat}-also \textsc{cop.pres}\\
\glt `Saburo, too.' (Lit. Taro also told Saburo [$\{$Saburo/*Jiro$\}$ to go home].)
\end{xlist}
\z
However, we can have another example, where the strict reading is possible (or sounds more natural). As in \REF{shimamu19b}, the elided doer is most naturally interpreted as Saburo, not his parents since the common sense says that his parents are not supposed to write any papers to get their son's academics right. Note that making a command to a 3rd person individual is possible as in \REF{shimamu20}; also, see \REF{shimamu6} and \REF{shimamu7}.

\ea\label{shimamu19}
\begin{xlist}
\ex \gll {Yamada sensei}-wa Taroo$_1$-ni [ $e_{1}$ motto ronbun-o kak-oo-to ] it-ta.\\
{Prof. Yamada}-\textsc{top} Taro-\textsc{dat} {} {} more paper-\textsc{acc} write-\textsc{mod-rep} {} say-\textsc{past}\\
\glt Lit. `Prof. Yamada told Taro$_1$ $e_{1}$ to write more papers.'
\ex\label{shimamu19b} \gll Kare-no oya-ni-mo da.\\
he-\textsc{gen} parent-\textsc{dat}-also \textsc{cop.pres}\\
\glt `His parents, too.' (Lit. Prof. Yamada also told his (Saburo's) parents [Saburo to write more papers].)\\
\end{xlist}
\z
\ea\label{shimamu20} \gll Otaku-no musuko-san-wa motto ronbun-o kak-oo.\\
you-\textsc{gen} son-\textsc{pol-top} more paper-\textsc{acc} write-\textsc{mod}\\
\glt `Your son should write more papers.'
\z

Given the above discussion, the complement clauses in \REF{shimamu5} do not host a(n OC) PRO, but the silent subjects are silent pronouns, namely, \textit{pro}.

\section{Proposal}\label{shimamuS4}
The aim of this section is to explain why the examples in \REF{shimamu5}, repeated here in \REF{shimamu21}, behave differently for their temporal and subject interpretations.

\ea\label{shimamu21}
\begin{xlist}
\ex\label{shimamu21a} \gll Kinoo, Taroo-wa [ asu syuppatu-si-yoo-to ] $\{$kime/omot$\}$-ta.\\
yesterday Taro-\textsc{top} {} tomorrow departure-do-\textsc{mod-rep} {} \phantom{$\{$}decide/think-\textsc{past}\\
\glt `Yesterday, Taro $\{$decided to leave/thought about leaving$\}$ tomorrow.'
\ex\label{shimamu21b} \gll Kinoo, Taroo-wa [ (*asu) syuppatu-si-yoo-to ] $\{$kokoromi/si$\}$-ta.\\
yesterday Taro-\textsc{top} {} \phantom{*(}tomorrow departure-do-\textsc{mod-rep} {} \phantom{$\{$}try/do-\textsc{past}\\
\glt `Yesterday, Taro tried to leave (*tomorrow).'
\end{xlist}
\z
To capture the differences under discussion, I propose the two structures in Figures~\ref{shimamu22a} and~\ref{shimamu22b}.

\begin{figure}
\caption{TP-complementation\label{shimamu22a}}
\Tree [.VP [.RepP [.TP [.ModP [.$v$P \qroof{\textit{pro}}.DP [.$v'$ \qroof{\ldots}.VP $v$ ] ] [.Mod {\textit{yoo}\newline\textsc{[person: 1]}} ] ] T ] [.Rep \textit{to}$_{\includegraphics[width=3mm]{figures/monster}}$ ] ][.V \textit{decide}/\textit{think} ] ]
\end{figure}

\begin{figure}
\caption{ModP/VP-complementation\label{shimamu22b}}
\Tree [.VP [.RepP [.ModP \qroof{\ldots}.VP [.Mod {\textit{yoo}\newline\textsc{[person: 1]}} ] ]  [.Rep \textit{to}$_{\includegraphics[width=3mm]{figures/monster}}$ ] ][.V \textit{try}/\textit{do} ] ]
\end{figure}

In \figref{shimamu22a}, the embedded clause has $v$P as well as TP, whereas \figref{shimamu22b} lacks them. Unlike \REF{shimamu4a}, I assume that TP is present. Note that in Japanese, the finite future can be expressed by the present form if a given verb is eventive, and when embedded, it is interpreted relative to the matrix reference time (Taro's now in \REF{shimamu23}). Observe:

\ea\label{shimamu23}
\gll Taroo-wa [ Ziroo-ga koko-ni ku-ru-to ] it-ta.\\
Taro-\textsc{top} {} Jiro-\textsc{nom} this.place-to come-\textsc{pres-rep} {} say-\textsc{past}\\
\glt `Taro said that Jiro would come here.' 
\z
Therefore, Japanese is not like English in this respect, but just like English in \REF{shimamu3a} it is possible to utter:

\ea\label{shimamu24} \gll Sensyuu, Taroo-wa [ kinoo syuppatu-si-yoo-to ] $\{$kime/omot$\}$-te-i-ta.\\
last.week Taro-\textsc{top} {} yesterday departure-do-\textsc{mod-rep} {} \phantom{$\{$}decide/think-\textsc{asp-cop-past}\\
\glt `Last week, Taro $\{$decided to leave/thought about leaving$\}$ yersterday.'
\z
I thus assume that the future tense is always relative, unlike English (see \citealt{ogihara1995}), so the presence of TP is not problematic even for \REF{shimamu24}.

Also worthwhile to note here is that I do not assume that Rep is a complementizer, contrary to the widely accepted view; see \citet{shimamura2018} and references therein. For instance, \citeauthor{shimamura2018} discusses a case where non-clausal items like names are embedded:

\ea\label{shimamu24.5} \begin{xlist}
 \ex[]{\gll Kare-wa zibun-no misume-o Aoi-to nazuke-ta.\\
he-\textsc{top} self-\textsc{gen} daughter-\textsc{acc} Aoi-\textsc{rep} name-\textsc{past}\\
\glt `He named his daughter Aoi.'}
\ex[*]{\label{shimamu24.5b}\gll Kare-wa zibun-no misume-$\{$ga/o$\}$ Aoi-da-to nazuke-ta.\\
he-\textsc{top} self-\textsc{gen} daughter-\textsc{nom/acc} Aoi-\textsc{cop.pres-rep} name-\textsc{past}\\
\glt  Intended: `He named his daughter Aoi.'}
\end{xlist}
\z
As in \REF{shimamu24.5b}, any property that signalizes a clausal structure is excluded: that is, nominative case, which is assumed to be assigned by the (finite) C-T association \citep{chomsky2008}, is impossible and the copula cannot appear either. Thus, this means that Rep directly attaches to the name. Note also that this is not a case of direct quotation since we can ask the name as follows:

\ea \gll Kare-wa zibun-no misume-o nan-to nazuke-ta-no.\\
he-\textsc{top} self-\textsc{gen} daughter-\textsc{acc} what-\textsc{rep} name-\textsc{past-q}\\
\glt `What did he name his daughter?'
\z

Then, the structure in \figref{shimamu22a} can be considered to be more ``biclausal'' than that in \figref{shimamu22b}, which is supported by the fact about the negative concord items (NCI); the combination of wh-pronouns and \textit{-mo} `also' yields NCIs such as \textit{dare-mo} (who-also) `anyone', which requires the presence of a negation as in \REF{shimamu25}, and unlike negative polarity items, NCIs need a given negation to be in the same clause where they are located, as shown in \REF{shimamu26}.

\ea\label{shimamu25} \begin{xlist}
\ex[]{\gll Taroo-wa dare-mo seme-naka-ta.\\
Taro-\textsc{top} who-also blame-\textsc{neg-past}\\
\glt `Taro didn't blame anyone.'}
\ex[*]{\gll Taroo-wa dare-mo seme-ta.\\
Taro-\textsc{top} who-also blame-\textsc{past}\\}
\end{xlist}
\ex\label{shimamu26} \begin{xlist}
\ex[]{ \gll Taroo-wa [ Ziroo-ga dare-mo seme-naka-ta-to ] it-ta.\\
Taro-\textsc{top} {} Jiro-\textsc{nom} who-also blame-\textsc{neg-past-rep} {} say-\textsc{past}\\
\glt `Taro said that Jiro didn't blame anyone.'}
\ex[*]{\gll Taroo-wa [ Ziroo-ga dare-mo seme-ta-to ] iw-anakat-ta.\\
Taro-\textsc{top} {} Jiro-\textsc{nom} who-also blame-\textsc{past-rep} {} say-\textsc{neg-past}\\}
\end{xlist}
\z
Then, consider the following contrast:

\ea\label{shimamu27}
\begin{xlist} 
\ex\label{shimamu27a}  {\gll Taroo-wa [ dare-ni-mo aw-oo-to. ] $\{$*kime-nakat/?omow-anakat$\}$-ta.\\
Taro-\textsc{top} {} who-\textsc{dat}-also see-\textsc{mod-rep} {} \phantom{$\{$*}decide-\textsc{neg}/think-\textsc{neg}-\textsc{past}\\
\glt Lit. `Taro didn't $\{$decide to meet/think about meeting$\}$ anyone.'}
\ex\label{shimamu27b} \gll Taroo-wa [ dare-ni-mo aw-oo-to ] $\{$?kokoromi/si$\}$-nakat-ta.\\
Taro-\textsc{top} {} who-\textsc{dat}-also see-\textsc{mod-rep} {} \phantom{$\{$?}try/do-\textsc{neg-past}\\
\glt Lit. `Taro didn't try to meet anyone.'
\end{xlist}
\z
In \REF{shimamu27a}, the NCI cannot be licensed by the matrix negation with \textit{kime-} `decide'. Note that \textit{omow-} `think' is relatively fine, but it is a typical neg-raising verb, so it may be irrelevant here. What is crucial is then that the NCI in \REF{shimamu27b} is licensed. It is obvious that both \textit{kokoromi-} `try' and \textit{si-} `do' are not neg-raising verbs, yet the NCI is possible. This suggests that the embedded clause in \REF{shimamu27a} is smaller than that in \REF{shimamu27b}.

At this point, the structures in Figures~\ref{shimamu22a} and~\ref{shimamu22b} give us another interesting prediction: that is, the accusative case that is assigned to the embedded object stems from the embedded verb in \figref{shimamu22a} and the matrix verb in \figref{shimamu22b} since an accusative case, by assumption, is assigned by (transitive) $v$, so that long passive should be possible in \figref{shimamu22b}, but not in \figref{shimamu22a}. This prediction turns out to be true as follows:\footnote{Some of my language consultants did not like \REF{shimamu28b}, but they still saw the clear contrast between \REF{shimamu28a} and \REF{shimamu28b}, observing that \REF{shimamu28b} is much more acceptable than \REF{shimamu28a}.}

\ea\label{shimamu28}
\begin{xlist} 
\ex[*]{\label{shimamu28a}\gll Sono kenkyuusya-niyotte sin'yaku-ga umidas-oo-to $\{$kime-rare/omow-are$\}$-te-i-ta.\\
that researcher-by new.medicine-\textsc{nom} create-\textsc{mod-rep} \phantom{$\{$}decide-\textsc{pass}/think-\textsc{pass}-\textsc{asp-cop-past}\\
\glt Lit. `A new medicine had been $\{$decided to create/thought about creating$\}$ by the researcher.'}
\ex[]{\label{shimamu28b}\gll Sono kenkyuusya-niyotte sin'yaku-ga umidas-oo-to $\{$kokoromi-rare/s-are$\}$-te-i-ta.\\
that researcher-by new.medicine-\textsc{nom} create-\textsc{mod-rep} \phantom{$\{$}try-\textsc{pass}/do-\textsc{pass}-\textsc{asp-cop-past}\\
\glt Lit. `A new medicine was being tried to create by the researcher.'}
\end{xlist}
\z 
Having established that the two \textit{yoo} complements are different in their sizes, we are ready to explain why temporal interpretations and subject (agent) interpretations are different between them. That is, since the \textit{yoo} complement in \figref{shimamu22a} hosts T and \textit{pro}, it is compatible with two independent time adverbs and various kinds of agent interpretations. In contrast, \figref{shimamu22b} lacks T and \textit{pro}, so that it must be temporally simultaneous with the matrix event time, and the agent of the embedded event must be the same as the matrix subject just like \textit{Restructuring} discussed by \citet{wurmbrand2001}.

\section{How much structure we need in the \textit{yoo} complement}\label{shimamuS5}
Before I conclude, I will discuss a couple of empirical issues of the proposed analysis. Although Japanese does not have e.g. clitic climbing, it has scrambling. It is widely known that scrambling out of the proposition complement must be an instance of A$'$-movement (i.e. A$'$-scrambling) \citep[][among many others]{saito1992}. In contrast, the situation complement is transparent to A-scrambling (\citealt{nemoto1991}; but see \citealt{takano2010}), so that the event complement should also allow A-scrambling from it. This is indeed the case: in \REF{shimamu29}, the pronoun \textit{soko} `that place' needs to be A-bound by some quantified expression in order to function as a bound variable. In \REF{shimamu29a}, since it is not A-bound by any quantifiers, the bound variable interpretation is not possible, whereas A-scrambling the embedded object in front of the matrix subject that has the relevant pronoun makes the bound variable reading possible.

\ea\label{shimamu29} 
\begin{xlist} 
\ex\label{shimamu29a}  \gll Soko$_{*2}$-no bengosi$_1$-ga [ $e_1$ mittu-izyoo-no kigyoo$_2$-o uttae-yoo-to ] kime-ta.\\
that.place-\textsc{gen} lawyer-\textsc{nom} {} {} 3.\textsc{cl}-more.than-\textsc{gen} company-\textsc{acc} sue-\textsc{mod-rep} {} decide-\textsc{past}\\
\glt Lit. `Their$_{*2}$ lawyers decided to sue [more than three companies]$_{2}$.'
\ex\label{shimamu29b}  \gll [Mittu-izyoo-no kigyoo$_2$-o]$_3$ soko$_{2}$-no bengosi$_1$-ga [ $e_1$ $t_3$ uttae-yoo-to ] kime-ta.\\
\phantom{[}3.\textsc{cl}-more.than-\textsc{gen} company-\textsc{acc} that.place-\textsc{gen} lawyer-\textsc{nom} {} {} {} sue-\textsc{mod-rep} {} decide-\textsc{past}\\
\glt Lit. `[More than three companies$_{2}$]$_{3}$, their$_{2}$ lawyers decided to sue $t_3$.'
\end{xlist}
\z
Then, it follows that the event complement is also transparent for A-scrambling. For the current discussion, the complement clause of \textit{kokoromi-} `try' and \textit{su-} `do' should be of this kind since it is tenseless/simultaneous.

\ea\label{shimamu30} 
\begin{xlist} 
\ex \gll Soko$_{*2}$-no bengosi$_1$-ga [ $e_1$ mittu-izyoo-no kigyoo$_2$-o uttae-yoo-to ] $\{$kokoromi/si$\}$-ta.\\
that.place-\textsc{gen} lawyer-\textsc{nom} {} {} 3.\textsc{cl}-more.than-\textsc{gen} company-\textsc{acc} sue-\textsc{mod-rep} {} \phantom{$\{$}try/do-\textsc{past}\\
\glt Lit `Their$_{*2}$ lawyers tried to sue [more than three companies]$_{2}$.'
\ex \gll [Mittu-izyoo-no kigyoo$_2$-o]$_3$ soko$_{2}$-no bengosi$_1$-ga [ $e_1$ $t_3$ uttae-yoo-to ] $\{$kokoromi/si$\}$-ta.\\
\phantom{[}3.\textsc{cl}-more.than-\textsc{gen} company-\textsc{acc} that.place-\textsc{gen} lawyer-\textsc{nom} {} {} {} sue-\textsc{mod-rep} {} \phantom{$\{$}try/do-\textsc{past}\\
\glt Lit. `[More than three companies$_{2}$]$_{3}$, their$_{2}$ lawyers tried to sue $t_3$.'
\end{xlist}
\z

Now, let us consider the proposed analysis of the control-like construction in Japanese in light of the ICH. As is obvious from Figures~\ref{shimamu22a} and~\ref{shimamu22b}, what I have argued is that embedding \textit{yoo} involves reduced clausal complements. Since they are transparent to A-scrambling, it should be that CP is absent (unless we assume that the CP that embeds \textit{yoo} is somehow transparent, and this is like what \citealt{uchibori2000} claims). In contrast, the NCI licensing is different between Figures~\ref{shimamu22a} and~\ref{shimamu22b}, and this is another instance of the ICH effect. Also, the differences in the temporal/subject interpretations are also understood in terms of the ICH. \figref{shimamu22a} is a situation complement, realized as a TP; \figref{shimamu22b} is an event complement, which is however realized as a ModP. I assume that modals are relative to an event rather than a world of evaluation \citep{hacquard2006}, so that it is still possible to have the \textit{yoo} complement tenseless. In a sense, assuming that \textit{yoo} can be with or without T is tantamount to decomposing \textit{will}/\textit{would} into T and \textit{woll}, although unlike \textit{woll}, \textit{yoo} itself does not contribute to the future interpretation, only expressing the speaker's volition. Anyway, the differences in the temporal/subject interpretations follow from the size of the complement clause.

However, things are not so simple as we expect; for instance, it is predicted that \figref{shimamu22a}, but not \figref{shimamu22b}, is compatible with an overt embedded subject. Notwithstanding this prediction, my language consultants and I do not see any robust contrast between the `decide/think' complement and the `try/do' complement.

\ea\label{shimamu31} \gll Taroo$_1$-wa [  yotee-doori-ni zibun$_1$-ga syuppatu-si-yoo-to ] $\{$kime/omot/?kokoromi/?si$\}$-ta.\\
Taro-\textsc{top} {} plan-way-\textsc{cop.inf} self-\textsc{nom} departure-do-\textsc{mod-rep} {} \phantom{$\{$}decide/think/try/do-\textsc{past}\\
\glt Lit. `Taro$_1$ $\{$decided self $e_1$ to leave/think of self$_1$ leaving/tried self$_1$ to leave$\}$ as planned.'
\z
However, the ICH does not say that the event complement must be the Theta domain. Since it is concerned with the minimal structure, such a complement can still be organized as some structure bigger than $v$P/VP. As we have seen, the NCI can be licensed when the selecting verbs are \textit{kokoromi-} `try' and \textit{si-} `do'. However, even those verbs seem incompatible with an NCI downstairs and its licensing negation upstairs when the embedded subject is overt.

\ea\label{shimamu32} \gll Taroo-wa [ (?*zibun$_1$-ga) dare-ni-mo aw-oo-to ] $\{$?kokoromi/si$\}$-nakat-ta.\\
Taro-\textsc{top} {} \phantom{?*}self-\textsc{nom} who-\textsc{dat}-also see-\textsc{mod-rep} {} \phantom{$\{$?}try/do-\textsc{neg-past}\\
\glt Lit. `Taro didn't try to meet anyone.'
\z
Although I would not say that the overt embedded subject renders \REF{shimamu32} completely ungrammatical, its presence makes it much harder to accept it. Also, long passive becomes impossible if the embedded subject is overt:\footnote{I assume with \citet{sudo2012} that indexical shifting of pronouns are optional in Japanese.}

\ea\label{shimamu33} 
\begin{xlist} 
\ex[]{\gll Taroo$_1$-wa [  yotee-doori-ni kare$_1$-ga sono sigoto-o si-yoo-to ] $\{$kokoromi/si$\}$-ta.\\
Taro-\textsc{top} {} plan-way-\textsc{cop.inf} he-\textsc{nom} that job-\textsc{acc} do-\textsc{mod-rep} {} \phantom{$\{$}try/do-\textsc{past}\\
\glt Lit. `Taro tried to do the job as planned.'}
\ex[*]{\gll [Sono sigoto-ga]$_2$ Taroo$_1$-niyotte [  yotee-doori-ni kare$_1$-ga $t_2$ si-yoo-to ] $\{$kokoromi-rare/s-are$\}$-te-i-ta.\\
 \phantom{[}that job-\textsc{nom} Taro-by {} plan-way-\textsc{cop.inf} he-\textsc{nom} {} do-\textsc{mod-rep} {} \phantom{$\{$}try-\textsc{pass}/do-\textsc{pass}-\textsc{asp-cop-past}\\
\glt Lit. `That job was tried by Taro that he would do as planned.'}
\end{xlist}
\z
In a similar vein, some of my informants reported that they can have two independent time adverbs even with \textit{kokoromi-} `try'/\textit{si-} `do' as in:

\ea\label{shimamu34} 
\judgewidth{?}
\begin{xlist} 
\ex[?]{\label{shimamu34a}\gll Kyoo Taroo$_1$-wa [  yotee-doori-ni asu sono sigoto-o si-yoo-to ] $\{$kokoromi/si$\}$-ta.\\
today Taro-\textsc{top} {} plan-way-\textsc{cop.inf} tomorrow that job-\textsc{acc} do-\textsc{mod-rep} {} \phantom{$\{$}try/do-\textsc{past}\\
\glt Lit. `Today Taro tried to do the job tomorrow as planned.'}
\ex[*]{\label{shimamu34b}\gll Kyoo [sono sigoto-ga]$_2$ Taroo$_1$-niyotte [  yotee-doori-ni asu $t_2$ si-yoo-to ] $\{$kokoromi-rare/s-are$\}$-te-i-ta.\\
 today \phantom{[}that job-\textsc{nom} Taro-by {} plan-way-\textsc{cop.inf} tomorrow {} do-\textsc{mod-rep} {} \phantom{$\{$}try-\textsc{pass}/do-\textsc{pass}-\textsc{asp-cop-past}\\
\glt Lit. `Today, that job was tried by Taro that he will do tomorrow as planned.'}
\end{xlist}
\z
Although cases like \REF{shimamu5b} are bad, \REF{shimamu34a} can still sound possible if the intended construal is such that Taro's attempt to arrange something for him to do the job tomorrow was done today. However, long passive is excluded as \REF{shimamu34b} shows. In addition, the NCI licensing, as is expected, also becomes impossible:

\ea[*]{\label{shimamu35}\gll Kyoo Taroo$_1$-wa [  asu dono sigoto-mo si-yoo-to ] $\{$kokoromi/si$\}$-nakat-ta.\\
today Taro-\textsc{top} {} tomorrow which job-also do-\textsc{mod-rep} {} \phantom{$\{$}try/do-\textsc{neg-past}\\
\glt Lit. `Today Taro didn't try to do any jobs tomorrow.'}
\z
These indicate that even the complement clauses of `try' verbs in Japanese can have more structure than what is given in \figref{shimamu22b}. However, this is fine under the ICH, since it is concerned with, as I said, the minimal structure, and the clause size can vary across languages or even within a language (or among speakers of a given language), to the extent that it obeys the ICH (e.g. no situation complement that is organized only in the form of the Theta domain).

\section{Conclusion}\label{shimamuS6}
In this chapter, I have investigated the nature of one specific sentential complementation in Japanese that has been considered to be a case of (OC) control: embedding the modal element, \textit{yoo}. I have argued contrary to the literature that it does not exemplify a case of control, proposing a way to get such a construction via indexical shifting. It has also been argued throughout this chapter that the size of the complement clause can vary in accordance with a given selecting (matrix) predicate. This is captured by the ICH proposed by \citet{wurmbrandlohninger2020}. Although their discussion is mainly concerned with data from several European languages, Japanese, as we have seen, nicely fits the relevant generalization, so the validity of it is now reinforced by one of the east Asian languages. 

\section*{Acknowledgements}

This chapter is a ``syntactically'' extended version of my semantics generals paper submitted to UConn (\citealt{shimamura2015}). I thank two anonymous reviewers for comments and suggestions, which improved this chapter a lot. Also, \textit{Danke schön} to Susi, whose inspiration and tutelage during my time at UConn had and still have enormous influence on me, so I dedicate this chapter to her. This research is funded by JSPS KAKENHI (20K13017), so I hereby acknowledge it. 

\section*{Abbreviations}
\begin{multicols}{3}
\begin{tabbing}
\textsc{recip}\hspace{1ex}\=reciprocal\kill
\textsc{acc} \> accusative \\
\textsc{caus} \> causative \\
\textsc{cl} \> classifier \\
\textsc{cop} \> copula \\
\textsc{gen} \> genitive \\
\textsc{dat} \> dative \\
\textsc{inf} \> infinitive \\
\textsc{mod} \> modal \\
\textsc{nom} \> nominative \\
\textsc{pass} \> passive \\
\textsc{past} \> past tense \\
\textsc{pol} \> polite \\
\textsc{pres} \> present tense \\
\textsc{recip} \> reciprocal \\
\textsc{rep} \> reporting particle \\
\textsc{top} \> topic
\end{tabbing}
\end{multicols}

{\sloppy\printbibliography[heading=subbibliography,notkeyword=this]}

\end{document}
