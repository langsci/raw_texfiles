%!TEX root = ../main.tex

\chapter{Information structure} \label{cha:infostructure}

\section{Introduction} \label{intro-information-structure}

This chapter should be seen as a preliminary study of those linguistic structures captured under the rubric of \isi{information structure}. I address a number of mechanisms which are employed to create textual cohesion, emphasis, and event sequencing. In linguistic theory, the notions of topicalisation, emphasis, \isi{focus}, fore- and backgrounding have been used to analyse \isi{information structure}. As in many other languages, the correlates of these abstract concepts are drawn from a wide range of linguistic phenomena. They may be expressed by nuances in intonation, designated morphology, specific particles, syntactic constructions, or an exploitation of the rich TAM system. Some of these mechanisms are typical of certain text genres while others are more pervasive.

I will describe different particles and enclitics that are used to mark \isi{focus}, intensification and emphasis in {\S}\ref{info-foc-emph} and briefly point to the narrative paragraph marker \emph{watik} in {\S}\ref{watik}. This is followed by a discussion of topicalisation in {\S}\ref{info-cleft}. The chapter closes with a description of how Komnzo speakers exploit their complex TAM system to sequence event descriptions in {\S}\ref{info-tam-event}.

\section{Clitics and particles} \label{info-foc-emph}

There are a number of particles, enclitics, and affixes that are used for \isi{focus}. These are sometimes glossed as intensifiers, emphasisers, or they are sometimes translated into \ili{English} by `only' or `also'. All of these interact with \isi{focus}, but it might be premature to analyse them purely as \isi{focus} markers. By looking at a longer piece of text, I will describe the \isi{intensifier} \emph{fof}, the \isi{emphatic} \isi{enclitic} \emph{=wä}, the contrastive markers \emph{komnzo} and \emph{=nzo}, and the \isi{particle} \emph{we}. All of these elements are pervasive in the language and not preferred in any particular text genre.

Following König (\citeyear{Konig:1991vy}), who discusses \isi{focus} particles, I draw a distinction in function between presentational, contrastive and additive \isi{focus}. König states that: ``[a] \isi{focus} \isi{particle} relates the value of the focused expression to a set of alternatives'' (\citeyear[32]{Konig:1991vy}). A contrastive \isi{focus} excludes all alternatives,	 while presentational \isi{focus} emphasises whatever lies within its scope. Additive \isi{focus} presupposes a previous proposition and highlights that the same applies to another referent. We find that Komnzo employs the \isi{particle} \emph{fof} and the \isi{enclitic} \emph{=wä} for presentational \isi{focus}, the \isi{particle} \emph{komnzo} and the related \isi{enclitic} \emph{=nzo} for contrastive \isi{focus}, and the \isi{particle} \emph{we} for additive \isi{focus}.

These mechanisms may be categorised according to their scope. The \isi{particle} \emph{fof} usually has scope over the element which it follows. This may be a whole clause if it occurs post-verbally. More commonly, it is found after demonstratives, deictics or complete noun phrases in which case it has scope over these elements ({\S}\ref{discourse-particles}). The \isi{enclitic} \emph{=wä} attaches to noun phrases, but is most commonly found with pronouns. The \isi{particle} \emph{komnzo} occurs pre-verbally and has scope over the predicate, while the \isi{enclitic} \emph{=nzo} attaches mostly to \isi{nominal}s and noun phrases and, thus, has scope over arguments or adjuncts. The \isi{particle} \emph{we} occurs in front of a clause over which it has scope or is sometimes used twice bracketing an element.

A third criterion for categorising these elements is according to their semantic content. König points out that \ili{English} words like `even, just, only' have a lexical meaning, whereas \isi{focus} particles in other languages mark `pure \isi{focus}' (\citeyear[29ff.]{Konig:1991vy}). He cites \ili{Somali} (\citealt[21ff.]{Saeed:1984wm}) and \ili{Manam} (\citealt[476ff.]{Lichtenberk:1983vh}) as languages where \isi{focus} particles have been described as being lexically empty. We can attribute such a \isi{characteristic} to the \isi{particle} \emph{fof}. It is the word which occurs with the highest frequence in the corpus. Informants often found it hard to give a separate translation of \emph{fof}, and when pressed to do so often translated it with `really'. As there are two adverbs \emph{fefe} `really' and \emph{minzü} `very' expressing the same, I take \emph{fof} to have no lexical meaning. This holds not true for the other elements discussed here. The \isi{particle} \emph{komnzo} as well as the \isi{enclitic} \emph{=nzo} are often translated as `only'. The \isi{particle} \emph{we} is, often translated as `also' or `too'.

I will make use of a text excerpt to explain how these mechanisms are put to use in Komnzo. The example text in (\ref{nzurnatext}) is the last part of a \emph{nzürna} story which is a common narrative in the Morehead region with numerous local variants. The \emph{nzürna} character is a female being who can change her appearance. Although these stories are often comical, the \emph{nzürna} poses some kind of a threat to the protagonists of the story. She is said to kill and eat especially small children. Mary Ayres roughly translated \emph{nzürna ŋare} as ``devil woman'' (\citeyear[93]{Ayres:ws}). In contrast to mythical stories, or knowledge about magic and sorcery, \emph{nzürna} stories are public stories, which are often retold and joked about. This particular \emph{nzürna} story is set in Firra, a now abandoned village about 15km south of Morehead. The narrator is Maraga Kwozi, a man who used to live in Firra. The \emph{nzürna} used to help and look after the people of Firra until the day that she killed and ate a stranger who was visiting the village. Outraged at this vicious incident, the village people took revenge and burned the tree in which she and her husband Nagawa were living. Nagawa escaped from the fire, but his wife was killed. The text excerpt picks up where main action is over. Nagawa returns to their home in Waisam to find out if his wife has survived the attack by the villagers. The elements to be discussed are underlined:

\begin{exe}
	\ex
	\begin{xlist}

	\exi{1} \emph{ane thrma mni fthé zäbtha.}
	\trans `After this, the fire had finished.'

	\exi{2} \emph{\uline{wati} nagawa ŋabrigwa ... sir}
	\trans `Then Nagawa returned ... to see'

	\exi{3} \emph{``\uline{komnzo} rä o z kwarsir mnin?''}
	\trans ``Is she still alive or did she burn in the fire?''

	\exi{4} \emph{ŋabrigwa ... bobomr \uline{we} waisam wäsü fthé sanmara.}
	\trans `He went back ... there he also saw that Wäsü tree in Waisam.'

	\exi{5} \emph{\uline{watik} fi ``nafazfthen\uline{wä}.''}
	\trans `Then he (said): ``It was all her own fault.'''

	\exi{6} \emph{ya\uline{nzo} bobo yanora ... nafaŋareanema.}
	\trans `He was just crying ... for his wife.'

	\exi{7} \emph{\uline{wati}, fi näbi zäbrima.}
	\trans `Thus, he went back for good.'

	\exi{8} \emph{zmbo yamnzr ane woga oten.}
	\trans `This man lives now here in Ote.'

	\exi{9} \emph{emoth fäthä ämnzr.}
	\trans `He lives with his daughters.'

	\exi{10} \emph{\uline{watik}, kabeyé komnzo fä nomai sumarwre ... ymarwre fthé ...}
	\trans `Well, the people still see him there ... we see him when ...'

	\exi{11} \emph{fä ŋaritakwr nima firrafo yak ... \uline{we} nima ŋabrigwr.}
	\trans `he crosses (the river) on his way to Firra ... and also when he returns.'

	\exi{12} \emph{tnz fäth ane kabe yé}
	\trans `He is a short man.'

	\exi{13} \emph{ane nzürna ŋareane zokwasi nimame \uline{fof} rä \uline{fof}.}
	\trans `That Nzürna woman's story is just like that.'

	\exi{14} \emph{mane bobo firran zwamnzrm.}
	\trans `the one, who was staying in Firra.'

	\exi{15} \emph{tüfr yam nä fefe thwafiyokwrm ...}
	\trans `She did many things, ...'

	\exi{16} \emph{fi fathfa ane \uline{fof} wäfiyokwa ...}
	\trans `but this one thing she did in public ...'

	\exi{17} \emph{nä karma kabe mane yanatha mogarkamma}
	\trans `eating that man from another village ... from Mogarkam.'

	\exi{18} \emph{nafane zokwasi ... ane trikasi fobo\uline{nzo} wythk \uline{fof} brä.}
	\trans `her story ... that story ends there. It is over.'

	\exi{19} \emph{ane nzürna ŋareanema}
	\trans `about that Nzürna woman.'

	\exi{20} \emph{\uline{watik}, fobo \uline{fof} zräkoré}
	\trans `Well, that is what I told you.'

	\exi{21} \emph{nä karen nima nä buné bänema ... }
	\trans `In other villages (there are) others ...'

	\exi{22} \emph{nä nzürna ŋare zokwasi trikasi bä räro ...}
	\trans `other Nzürna woman stories are there ...'

	\exi{23} \emph{fi ane kar woga mane erä fi ane miyatha erä}
	\trans `but it is those village people who know about these.'

	\exi{24} \emph{nzefé nzüwäbragwé nima ni miyatha nrä}
	\trans `I followed like we know (this story).'

	\exi{25} \emph{nzekaren ane yam kwafiyokwrm ...}
	\trans `She did this in our village ...'

	\exi{26} \emph{nzenme ŋafyé mä thwamnzrm}
	\trans `where our fathers lived.'

	\exi{27} \emph{ŋafyé \uline{we} nzenm natrikwath}
	\trans `The fathers also told us (about it).'

	\exi{28} \emph{nima zbo zf zakoré. \uline{fof} zäbthé}
	\trans `I have said it now. I am finished.'
	\end{xlist}
	\Corpus{tci20120901-01}{MAK \#201-238}
	\label{nzurnatext}
\end{exe}

The \isi{intensifier} \emph{fof} occurs in lines 13, 16, 18, 20, and 28. In line 13, the narrator marks the end of the story by stating the story is ``just like that'' and \emph{fof} occurs twice. In the first instance, it has scope over \emph{nima=me} `like.this={\Ins}'. In the second instance, it occurs postverbally and has scope over the whole proposition. It is very common to give an affirmative reply by saying \emph{nima fof} or \emph{nimame fof} `just like this'. Such a reply rarely occurs without \emph{fof}. In lines 16 and 20, \emph{fof} occurs after the demonstratives \emph{ane} ({\Dem}) and \emph{fobo} (\Dist.\All) which is also very common. In line 16, the narrator emphasises that amongst many things that she did, it was this one incident where she stepped out of line. In line 20, he literally says ``\uline{to there}, I spoke'' emphasising the point where his story has come to an end now. In lines 18 and 28, \emph{fof} has scope over the predicate which in this case is the whole proposition. In line 18, the verb form is \emph{wythk} `it comes to an end.' In 28, the verb \emph{zäbthé} `I am finished' follows and finally closes the narration. In each case, \emph{fof} sets a mark which can be compared to a gesture like slamming one's hand on the table. It underlines and emphasises whatever lies in its scope.

The particle \emph{komnzo} and the \isi{enclitic} \emph{=nzo} occur in lines 3, 6, and 18. In line 3, \emph{komnzo} occurs in a question: `Is she still alive or did she burn in the fire?' The first clause only contains \emph{komnzo} and the copula \emph{rä} which translates literally as `she only exists'. In line 6, \emph{=nzo} is cliticised to \emph{ya} `cry, wail' and thus translates literally as `he was shouting out \uline{only wails}'. In line 18, \emph{=nzo} is attached to a \isi{demonstrative} \emph{fobo} \Dist.{\All}. The narrator stresses the fact that the story ends at that point and does not continue. Thus, with all three examples, we find \emph{komnzo} and \emph{=nzo} have a contrastive function, i.e. setting something apart from other options.

The \isi{particle} \emph{we} functions as an additive marker like the \ili{English} \isi{particle} \emph{also}. It occurs in lines 4, 11, and 27. In line 4, it introduces the account of Nagawa's return: that of seeing the Wäsü tree. In line 11, the narrator talks first about Nagawa crossing the river and then adds another clause about his return trip when he crosses the river again. The function of additive \isi{focus} becomes particularly clear in line 27. After the narrator explains that he is entitled and knowledgeable to tell the story because it took place in his village (lines 24-26), he adds another piece of justification, namely that his fathers told him the story.

\largerpage
The emphasising suffix \emph{=wä} occurs only once in the text (line 5). In his pain and sadness, Nagawa realises that it was his wife's action that had led to the act of revenge. This comment could have been expressed as \emph{nafa-zfth-en} \Third.\Poss-fault-{\Loc} `her fault', but the speaker adds \emph{=wä} \emph{nafa-zfth-en=wä} which can be translated as `her \uline{own} fault.' For a more detailed discussion of \emph{=wä} ({\S}\ref{emphathicwae}).

\section{The paragraph marker \emph{watik}} \label{watik}

The word \emph{watik} or sometimes \emph{wati} means `enough'. I often overhead it being used with together the \isi{adjectivaliser} suffix \emph{-thé} and the instrumental \emph{=me}. Thus, \emph{watikthéme} `(I have) enough' is a common reply to an offer to have more food or more tea. In narratives or procedural texts, \emph{watik} is often used to mark a new thought or the begining of a paragraph. Its use is typically followed by a short pause similar to the \ili{English} expressions `well', `and then', `thus', or `next'. We find such instances of \emph{watik} or \emph{wati} in the text excerpt (\ref{nzurnatext}) in lines 2, 5, 7, 10, and 20. \emph{Watik} introduces new episodes in each of these lines.

\section{Fronted relative clauses} \label{info-cleft}

Relative clauses are right-adjoined ({\S}\ref{relclauses}), and an example of a \isi{relative clause} is given in (\ref{ex483}). The matrix noun phrase \emph{bäne dgwr} `that orchid' is followed by the \isi{relative clause} [in square brackets]. Usually the \isi{relative clause} follows the matrix clause.

\begin{exe}
	\ex \emph{dgwrfa enrgegwr bäne dgwr} [\emph{boba mane themare}] \emph{berä.}\\
	\gll dgwr=fa en\stem{rgeg}wr bäne dgwr boba mane the\stem{mar}e b=e\stem{rä}\\
	orchid={\Abl} \Stsg:\Sbj>\Stpl:\Obj:\Nonpast:\Ipfv:\Venit/pull-off \Dem:\Med{} orchid \Med.{\Abl} which  \Fpl:\Sbj>\Stpl:\Obj:\Rpst:\Pfv/see \Med=\Stpl:\Sbj:\Nonpast:\Ipfv/be\\
	\trans `(The bowerbird) pulls them off the orchid. That orchid, which we saw over there.'\\\Corpus{tci20120815}{ABB \#32}
	\label{ex483}
\end{exe}

In public speeches, one often hears \isi{topic} constructions such as (\ref{ex133}) where the speaker proclaims to the people gathered at a feast that it is time to sing and dance (and not to fight). Literally, this sentence can be translated as: `The drums which resonate, they resonate for the dance ... only for this.' Formally, this is a fronted noun phrase with a following \isi{relative clause}. In most cases, the following \isi{relative clause} consists of \emph{mane} `what, which' and the \isi{copula} (\ref{ex717}). As a convention, I translate this with the \ili{English} phrases `as for X', `concerning X' or `when it comes to X'.

\begin{exe}
	\ex {\emph{brubru} [\emph{mane änor}] \emph{wathma änor ... zane frümöwä}}\\
	\gll brubru mane ä\stem{nor} wath=ma ä\stem{nor} (.) zane frü=me=wä\\
	drum which \Stpl:\Sbj:\Nonpast:\Ipfv/shout dance={\Char} \Stpl:\Sbj:\Nonpast:\Ipfv/shout (.) \Dem:{\Prox} alone=\Ins=\Emph\\
	\trans `As for the drums, they are resonating for the dance ... only for this.'\\\Corpus{tci20121019-04}{ABB \#46}
	\label{ex133}
\end{exe}

\begin{exe}
	\ex \emph{komnzo zokwasi} [\emph{mane rä}] \emph{... faremane zokwasi fefe ane fof rä ... komnzo.}\\
	\gll komnzo zokwasi mane \stem{rä} (.) farem=ane zokwasi fefe ane fof \stem{rä} (.) komnzo\\
	komnzo language which \Tsg.\F:\Sbj:\Nonpast:\Ipfv/be (.) farem=\Poss.{\Sg} language real {\Dem} {\Emph} \Tsg.\F:\Sbj:\Nonpast:\Ipfv/be (.) komnzo\\
	\trans `When it comes to Komnzo, this is the Farem's real language ... Komnzo!'\\\Corpus{tci20120924-02}{ABM \#4-5}
	\label{ex717}
\end{exe}

As we see in (\ref{ex717}), the \isi{relative clause} often contains the \isi{copula} (lit. `Komnzo language which is ...'). The result is that it contributes nothing to the state of affairs, but its main function is pragmatic. Therefore, I analyse the fronted noun phrase together with the \isi{relative clause} under the label fronted \isi{relative clause}, i.e. fronted with respect to the matrix clause, and I put both together in bracket in the following examples. Note that there may also be no matrix noun phrase in cases where it is the event that is topicalised, for example in (\ref{ex718}).

\begin{exe}
	\ex {[\emph{mane ynzänza}]\emph{ ... büdisn mä nzrugrm ... oroman fä fof samara ... ŋafe}}\\
	\gll mane yn\stem{zä}nza (.) büdisn mä nz\stem{rugr}m (.) oroman fä fof sa\stem{mar}a (.) ŋafe\\
	who \Sg:\Sbj>\Tsg.\Masc:\Obj:\Pst:\Ipfv:\Venit/carry (.) büdisn where \Fpl:\Sbj:\Pst:\Dur/sleep (.) old.man {\Dist} {\Emph} \Sg:\Sbj>\Tsg.\Masc:\Pst:\Ipfv/see (.) father\\
	\trans `As he was carrying him ... at Büdisn where we were sleeping ... the old man, father, saw him there.'\Corpus{tci20110810-02}{MAB \#55-56}
	\label{ex718}
\end{exe}

Fronted relative clauses are the main strategy to introduce or reactivate topics in the sense described by Keenan and Schieffelin (\citeyear[342]{Keenan:1976wo}). We find them not only in public speeches, but also in narratives, where speakers employ them to indicate a change in \isi{topic} or to introduce a \isi{topic}. I will describe this function by taking the reader through a particular narrative. Example sentence (\ref{ex134}) introduces the protagonist of the story, a man named Kukufia.

\begin{exe}
	\ex {[\emph{kukufia mane yara}] \emph{masun swamnzrm.}}\\
	\gll kukufia mane ya\stem{r}a masu=n swa\stem{m}nzrm\\
	kukufia which \Tsg.\Masc:\Pst:\Ipfv/be masu={\Loc} \Tsg.\Masc:\Pst:\Dur/dwell\\
	\trans `Kukufia lived in Masu.'\Corpus{tci20100905}{ABB \#8-9}
	\label{ex134}
\end{exe}

In order to state the simple fact that Kukufia lived in Masu, it would be sufficient to say \emph{kukufia masun swamnzrm} `Kukufia lived in Masu'. But because the sentence establishes the \isi{topic} (Kukufia), a fronted \isi{relative clause} is used. This is a very common way to introduce a character to a story.

Kukufia is a malicious character who comes to Rouku and tortures two children while their parents are away at a sago camp. Kukufia takes the two children fishing in his canoe. He pokes the small boy with the bones of a fish. One day, the father of the two children returns looking for them. Example (\ref{ex135}) shows, how this change in \isi{topic} is expressed.

\begin{exe}
	\ex \label{ex135}
	\begin{xlist}
	\ex \label{ex135a}
	\emph{fafen nge zi swathizrm ... ekri zi ... kofä ysma.}\\
	\gll fafen nge zi swa\stem{thi}zrm (.) ekri zi (.) kofä ys=ma\\
	meanwhile child pain \Tsg.\Masc:\Sbj:\Pst:\Dur/die (.) body pain (.) fish bone=\Char\\
	\trans `In the meantime, the child was in pain ... body pain from the fish bones.'
	\ex \label{ex135b} \emph{watik} [\emph{nafaŋafe mane yanra}] \emph{nagayé thrathorthm.}\\
	\gll watik nafa-ŋafe mane yan\stem{r}a nagayé thra\stem{thorthm}\\
	then \Third.\Poss-father which \Tsg.\Masc:\Sbj:\Pst:\Ipfv:\Venit/be children \Stsg:\Sbj>\Stpl:\Obj:\Irr:\Pfv/search\\
	\trans `Then ... As for their father, he was looking for the children.'\\\Corpus{tci20100905}{ABB \#90-95}
	\end{xlist}
\end{exe}

Again, the change in \isi{topic} is marked by a fronted \isi{relative clause} (\ref{ex135b}). The construction is not purely pragmatic here, as there is a \isi{venitive} marker on the copula (\emph{ya\uline{n}ra}) which indicates that the father is coming.

Further along in the story, the father finds his children locked inside the house. He finds out about Kukufia's visits and decides to hide underneath the house. When Kukufia returns later in the day, the father shoots him with an arrow. Kukufia runs away to Masu where his two wifes live. The father follows the trail of blood. In Masu, Kukufia transforms into a little baby boy hanging on the breast of one the wives. This is the point in the text where we find the next fronted \isi{relative clause} (\ref{ex137b}).

\begin{exe}
	\ex \label{ex137}
	\begin{xlist}
	\ex \label{ex137a}
	\emph{kukufia näbi zamatha dunzikarä ... ŋakwir e Masu kräkwther.}\\
	\gll kukufia näbi za\stem{math}a dunzi=karä (.) ŋa\stem{kwi}r e masu krä\stem{kwther}\\
	kukufia one \Tsg:\Sbj:\Pst:\Pfv/run arrow={\Prop} (.) \Tsg:\Sbj:\Nonpast:\Ipfv/run until masu \Tsg:\Irr:\Pfv/change \\
	\trans `Kukufia ran away with the arrow (inside him) ... He was running until Masu where he changed (his appearance).'
	\ex \label{ex137b} {[\emph{nafane ŋare mane zfrärm}] \emph{... edama ... thrma ŋare. wati mämen fobo zämira fof.}}\\
	\gll nafane ŋare mane zf\stem{rä}rm (.) eda=ma (.) thrma ŋare wati mäme=n fobo zä\stem{mir}a fof\\
	\Tsg.{\Poss} woman which \Tsg.\F:\Sbj:\Pst:\Dur/be (.) two={\Char} (.) after woman then breast={\Loc} \Dist.{\All} \Stsg:\Sbj:\Pst:\Pfv/hang {\Emph}\\
	\trans `It was his wife ... the second ... the latter wife. He was hanging on her breast.'\\\Corpus{tci20100905}{ABB \#117-121}
	\end{xlist}
\end{exe}

The narrator first describes Kukufia's escape in (\ref{ex137a}) and then changes the \isi{topic} to the wife on whose breast the little baby boy is hanging (\ref{ex137b}). The new \isi{topic} is again introduced by a fronted \isi{relative clause}. Kukufia's fate is sealed as the father quickly recognises the small boy. He kills Kukufia and his two wives on the spot and the story ends.

Fronted relative clauses of this type are used both to topicalise an expression, as in the introductory example to this section (\ref{ex133}), but also to indicate a change in the \isi{topic}, as in the examples above. The relative \isi{pronoun} used for this type of construction is always \emph{mane} `who, which'.

\section{TAM categories and event-sequencing} \label{info-tam-event}

Foley points out that Papuan languages often exploit their rich TAM systems for pragmatic purposes (\citeyear[389]{Foley:2000uh}). TAM marking and discourse notions such as foregrounding has been discussed by many authors, for example by Hopper (\citeyear{Hopper:1979us}). One such example from the Papuan language \ili{Sentani} comes from Hartzler (\citeyear{Hartzler:1983wm}) who has shown that clauses in \isi{irrealis} are commonly used for \isi{backgrounded}, presupposed propositions, whereas \isi{realis} is used for foregrounded, asserted propositions. Komnzo puts its TAM system to the same pragmatic use in order to create textual cohesion, but in Komnzo more TAM categories are involved ({\S}\ref{TAMsemantics}). This pragmatic use is often found in texts or parts of texts where the sequence of events is important, for example in procedurals, and descriptions of a path.

I will begin by comparing the above-mentioned realis-\isi{irrealis} distinction. Consider the following text (\ref{fathasitext}) which describes the first part of a wedding ceremony. This procedural was given by Abia Bai. The actual wedding took place two days after the recording was made. Therefore, the description of the event is set in the \isi{future}, which reduces the number of possible TAM categories. The speaker may only choose between the indicative non-past and the \isi{irrealis} verbal inflection.\footnote{Future reference is expressed periphrastically with the particle \emph{kwa} which may occur with non-past indicative and irrealis inflections.} In (\ref{fathasitext}), I have underlined the verbs in \isi{irrealis} \isi{mood} in Komnzo as well as in the \ili{English} translation. All other verbs are in non-past and indicative \isi{mood}.

\begin{exe}
\ex
\begin{xlist}
	\exi{1} \emph{wati foba nimame kwa ŋathkärwr.}
	\trans `Well, it will begin like this:'
	\exi{2} \emph{dagon rthé \uline{thrarakthkwrth} \uline{thräbthth}}
	\trans `The food \uline{will be placed} on the platform. That \uline{will be finished}.'
	\exi{3} \emph{zöbthé fefe kwa ... chris e nafaŋare maki ernth fof.}
	\trans `First, they are putting the paint on Chris and his wife.'
	\exi{4} \emph{maki fthé \uline{thrarnth} ... fthé \uline{thrabthth} ...}
	\trans `When they \uline{have put on} the paint ... when they \uline{have finished} ...'
	\exi{5} \emph{watik, foba kwa änrokonth.}
	\trans `next they will escort them this way.'
	\exi{6} \emph{fthé \uline{thrnthbth} nima ...}
	\trans `When they \uline{will bring} them in ... '
	\exi{7} \emph{faf mä kwa nge fathasi zn rä fof ...}
	\trans `to the place where the children's feast will take place ...'
	\exi{8} \emph{kwa änrokonth kwot bobomr ...}
	\trans `they will escort them up until ...'
	\exi{9} \emph{\uline{thranthaifth} faf znfo.}
	\trans `they \uline{will arrive} at the place.'
	\exi{10} \emph{watik kwa emsakrnth.}
	\trans `Next, they will sit them down.'
	\exi{11} \emph{\uline{thramsth} \uline{kramsth}}
	\trans `They \uline{will sit} them down. They \uline{will sit} down.'
	\exi{12} \emph{watik, zöbthé fefe kwa äyoknth a ätriknth nima:}
	\trans `Well, first, they will advise them and they will say:'
\end{xlist}
\Corpus{tci20110817-02}{ABB \#22-40}
\label{fathasitext}
\end{exe}

The content of this little excerpt is quickly summarised: After the food preparations, the bride and the groom will be decorated and painted. The women will escort the couple to the village square where they will be placed on a bench only to be lectured about codes of conduct and the expected behaviour.

We find that the speaker alternates between \isi{realis} and irrealis \isi{mood}. Realis occurs with the painting (line 3), the escorting (line 5), the escorting again (line 8), the sitting down (line 10) and the advising (line 12). Irrealis occurs with the finishing of the food preparations (line 2), the painting and the finishing thereof (line 4), the bringing (line 6), the arriving (line 8) and the sitting down (line 11). This alternation in TAM categories is congruent with an alternation between foregrounded, asserted events and \isi{backgrounded}, presupposed events. In some instances, the verb in \isi{realis} is repeated in irrealis, e.g. the sitting down in lines 11 and 12. Additionally, the repetition of one part of a proposition in the next proposition can be described as kind of tail-head-linkage.\footnote{De Vries (\citeyear{deVries:2005tm}) offers a typology for tail-head-linkage in Papuan languages. However, for the most part his sample consists of languages where this is achieved by using (parts of) serial verb constructions.} Thus, we find a rhetorical device that is used both for textual cohesion and foregrounding.

As for stories in the past, speakers have more TAM values to choose from. They may alternate again between irrealis and \isi{realis}, but they may also exploit the aspectual categories: \isi{perfective} and \isi{imperfective}. As was described in {\S}\ref{TAMsemaspect}, the \isi{imperfective} is divided again into a basic \isi{imperfective} and \isi{durative}. Thus, the richness of the TAM system allows speakers to make finer distinctions.

I will show this in another text excerpt (\ref{masentext}). This text is part of a story about a man who fell off a coconut palm and died. It was told by Marua Bai who remembers this incident well. The protagonist of the story used to wander around in the night and steal other people's palm wine. Palm wine is produced by cutting a fresh shoot up in the palm. A bamboo container which is tied underneath the shoot captures the sap. The sap slowly ferments and turns into an alcoholic substance. The main character of the story sets off alone in the night. He climbs and raids a number of palms. At the third palm, a coconut leaf breaks and he falls some twenty meters into a pineapple plant. Even though he survives his severe injuries, he dies about a week later. For each verb in each of the lines of text, the TAM value is given on the right. Where there are two verbs in a line, the underlined segments show which verb belongs to which translation and TAM value.

\begin{exe}
	\ex
	\exi{1}
	    $\begin{array}{l}
	   	\parbox{9cm}{\emph{wati fam änatha:}} \parbox{0,3cm}{\hfill}\parbox{4cm}{\Pst:\Ipfv}\\
	    \parbox{9cm}{He was thinking:} \parbox{0,3cm}{\hfill}\parbox{4cm}{\hfill}\\
	    \end{array}$
	\exi{2}
	    $\begin{array}{l}
	   	\parbox{9cm}{\emph{``kwa ŋabrigwé skerur.''}} \parbox{0,3cm}{\hfill}\parbox{4cm}{\Nonpast:\Ipfv}\\
	    \parbox{9cm}{``I will go back for coconut wine.''} \parbox{0,3cm}{\hfill}\parbox{4cm}{\hfill}\\
	    \end{array}$
	\exi{3}
	    $\begin{array}{l}
	   	\parbox{9cm}{\emph{zbär kretharuf gardafo.}}\parbox{0,3cm}{\hfill} \parbox{4cm}{\Irr:\Pfv}\\
	    \parbox{9cm}{In the night, he got into the canoe.} \parbox{0,3cm}{\hfill}\parbox{4cm}{\hfill}\\
	    \end{array}$
	\exi{4}
	    $\begin{array}{l}
	   	\parbox{9cm}{\emph{kwanrafinzrm gardame.}} \parbox{0,3cm}{\hfill}\parbox{4cm}{\Pst:\Dur}\\
	    \parbox{9cm}{He was paddling here with the canoe.} \parbox{0,3cm}{\hfill}\parbox{4cm}{\hfill}\\
	    \end{array}$
	\exi{5}
	    $\begin{array}{l}
	   	\parbox{9cm}{\emph{mane yanra zäzr mnz ... finzo ... kabe matak}} \parbox{0,3cm}{\hfill}\parbox{4cm}{\Pst:\Ipfv}\\
	    \parbox{9cm}{When he got to Zäzr Mnz ... (it was) only him ... nobody else} \parbox{0,3cm}{\hfill}\parbox{4cm}{\hfill}\\
	    \end{array}$
	\exi{6}
	    $\begin{array}{l}
	   	\parbox{9cm}{\emph{yokwa kar ane fof ... matak}} \parbox{0,3cm}{\hfill}\parbox{4cm}{no verb}\\
	    \parbox{9cm}{the same thing in Yokwa ... nobody} \parbox{0,3cm}{\hfill}\parbox{4cm}{\hfill}\\
	    \end{array}$
	\exi{7}
	    $\begin{array}{l}
	   	\parbox{9cm}{\emph{garda \uline{sräzin} ... yaniyak aki kwayanen ... mnz.}} \parbox{0,3cm}{\hfill}\parbox{4cm}{\uline{\Irr:\Pfv} \Nonpast:\Ipfv}\\
	    \parbox{9cm}{He \uline{put down} the canoe ... and came in the moonlight ... to the house.} \parbox{0,3cm}{\hfill}\parbox{4cm}{\hfill}\\
	    \end{array}$
	\exi{8}
	    $\begin{array}{l}
	   	\parbox{9cm}{\emph{nä skeru ŋasongwr.}} \parbox{0,3cm}{\hfill}\parbox{4cm}{\Nonpast:\Ipfv}\\
	    \parbox{9cm}{He climbed a (coconut) wine palm.} \parbox{0,3cm}{\hfill}\parbox{4cm}{\hfill}\\
	    \end{array}$
	\exi{9}
	    $\begin{array}{l}
	   	\parbox{9cm}{\emph{warfo ... fä ŋonathr.}} \parbox{0,3cm}{\hfill}\parbox{4cm}{\Nonpast:\Ipfv}\\
	    \parbox{9cm}{Up there ... he was drinking.} \parbox{0,3cm}{\hfill}\parbox{4cm}{\hfill}\\
	    \end{array}$
	\exi{10}
	    $\begin{array}{l}
	   	\parbox{9cm}{\emph{zrämbth we nä ŋazifo kresöbäth.}} \parbox{0,3cm}{\hfill}\parbox{4cm}{2x \Irr{}:\Pfv}\\
	    \parbox{9cm}{He finished and climbed another coconut.} \parbox{0,3cm}{\hfill}\parbox{4cm}{\hfill}\\
	    \end{array}$
	\exi{11}
	    $\begin{array}{l}
	   	\parbox{9cm}{\emph{fä ŋonathr.}} \parbox{0,3cm}{\hfill}\parbox{4cm}{\Nonpast:\Ipfv}\\
	    \parbox{9cm}{He was drinking.} \parbox{0,3cm}{\hfill}\parbox{4cm}{\hfill}\\
	    \end{array}$
	\exi{12}
	    $\begin{array}{l}
	   	\parbox{9cm}{\emph{we nä kabeane ŋazifo kresöbäth}} \parbox{0,3cm}{\hfill}\parbox{4cm}{\Irr:\Pfv}\\
	    \parbox{9cm}{and again he climbed another man's coconut.} \parbox{0,3cm}{\hfill}\parbox{4cm}{\hfill}\\
	    \end{array}$
	\exi{13}
	    $\begin{array}{l}
	   	\parbox{9cm}{\emph{mane ŋasogwa warfo ...}} \parbox{0,3cm}{\hfill}\parbox{4cm}{\Pst:\Ipfv}\\
	    \parbox{9cm}{As he climbed on top ...} \parbox{0,3cm}{\hfill}\parbox{4cm}{\hfill}\\
	    \end{array}$
	\exi{14}
	    $\begin{array}{l}
	   	\parbox{9cm}{\emph{\uline{kräms} drari wrbr.}} \parbox{0,3cm}{\hfill}\parbox{4cm}{\uline{\Irr:\Pfv} \Nonpast{}}\\
	    \parbox{9cm}{He \uline{sat down} and untied the bamboo container.} \parbox{0,3cm}{\hfill}\parbox{4cm}{\hfill}\\
	    \end{array}$
	\exi{15}
	    $\begin{array}{l}
	   	\parbox{9cm}{\emph{fof n zäznoba.}} \parbox{0,3cm}{\hfill}\parbox{4cm}{\Pst:\Pfv}\\
	    \parbox{9cm}{He was about to drink.} \parbox{0,3cm}{\hfill}\parbox{4cm}{\hfill}\\
	    \end{array}$
	\exi{16}
	    $\begin{array}{l}
	   	\parbox{9cm}{\emph{zamthetha drari.}} \parbox{0,3cm}{\hfill}\parbox{4cm}{\Pst:\Pfv}\\
	    \parbox{9cm}{He lifted up the bamboo container.} \parbox{0,3cm}{\hfill}\parbox{4cm}{\hfill}\\
	    \end{array}$
	\exi{17}
	    $\begin{array}{l}
	   	\parbox{9cm}{\emph{bäw! ŋazi tafokarä ane zägarnza.}} \parbox{0,3cm}{\hfill}\parbox{4cm}{\Pst:\Ipfv}\\
	    \parbox{9cm}{Bang! The coconut leaf broke off (with him).} \parbox{0,3cm}{\hfill}\parbox{4cm}{\hfill}\\
	    \end{array}$
	\exi{18}
	    $\begin{array}{l}
	   	\parbox{9cm}{\emph{zane zäkurfa ziyé}} \parbox{0,3cm}{\hfill}\parbox{4cm}{\Pst:\Pfv}\\
	    \parbox{9cm}{This one here split.} \parbox{0,3cm}{\hfill}\parbox{4cm}{\hfill}\\
	    \end{array}$
	\exi{19}
	    $\begin{array}{l}
	   	\parbox{9cm}{\emph{zenta ŋagarwa}} \parbox{0,3cm}{\hfill}\parbox{4cm}{\Pst:\Ipfv}\\
	    \parbox{9cm}{He split his crotch.} \parbox{0,3cm}{\hfill}\parbox{4cm}{\hfill}\\
	    \end{array}$
	\exi{20}
	    $\begin{array}{l}
	   	\parbox{9cm}{\emph{fainr fr sazika}} \parbox{0,3cm}{\hfill}\parbox{4cm}{\Pst:\Pfv}\\
	    \parbox{9cm}{He went into the pineapple plant.} \parbox{0,3cm}{\hfill}\parbox{4cm}{\hfill}\\
	    \end{array}$
	\exi{21}
	    $\begin{array}{l}
	   	\parbox{9cm}{\emph{fä swanorm ``ara ara'' ... kambe matak}} \parbox{0,3cm}{\hfill}\parbox{4cm}{\Pst:\Dur}\\
	    \parbox{9cm}{There he was shouting ``ah ah'' ... no people (heard him)} \parbox{0,3cm}{\hfill}\parbox{4cm}{\hfill}\\
	    \end{array}$
	\\\Corpus{tci20120904-01}{MAB \#42-69}
	\label{masentext}
\end{exe}

Several observations which pertain to event sequencing as well as foregrounding can be made from this text. First, the narrator uses \isi{non-past} \isi{tense} for several clauses: the walking to the house (line 7), the climbing (line 8), the drinking (lines 9 and 11) and the untying (line 14). In some cases, the \isi{non-past} alternates again with the irrealis \isi{perfective} forms (line 10, 12, and 14) as we have seen in the wedding text above. The use of a \isi{non-past} \isi{tense} in a story which is otherwise told in \isi{recent past} or \isi{past} is quite common. In these cases, the \isi{non-past} is used to foreground or emphasise the clauses in question.

Secondly, we find that it is the \isi{past} \isi{imperfective} which is used for the foregrounded clauses (in lines 13, 17, and 19). In line 17, the breaking of the coconut leaf is in the \isi{imperfective}, whereas the preceding events in lines 15 and 16 are in the \isi{perfective}. This might seem to contradict the notion of perfectivity, but the reader should keep in mind that the \isi{perfective} in Komnzo focusses more on the beginning of an event (inceptive, or punctual) rather than the completion of an event. See {\S}\ref{TAMsemaspect} for a description of the semantics of \isi{aspect} in Komnzo. Lines 18 and 19 both describe the severe injury which the protagonist received from his fall. Again the \isi{imperfective} \isi{aspect} is used for the foregrounded clause which provides more detail about the injury (i.e. that he split his crotch).

Although preliminary at this stage of research, we may attempt to build a hierarchy of TAM values with respect to foregrounding. In such a hierarchy,s irrealis inflections are more backgrounding than \isi{realis} inflections. All \isi{past} tenses are more backgrounding than the \isi{non-past}. Finally, as we have seen, the \isi{perfective} is more backgrounding than the \isi{imperfective}. It follows that the most foregrounding TAM value is the \isi{non-past}, while the irrealis (\isi{perfective}) is the most backgrounding TAM value. The pragmatic functions of the TAM system in Komnzo provide a rich field for future research.