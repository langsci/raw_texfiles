%!TEX root = ../main.tex

\chapter{Aspects of the lexicon} \label{cha:lexicon}

\section{Introduction}

This chapter brings together two topics which can be roughly subsumed under the rubric of lexicology. First, I describe sign \isi{metonymy} and \isi{metaphor} expressed by \isi{reduplication} ({\S}\ref{redupmeton}). These are found especially in terms for plants and birds. The second part is a description of the conceptualisation of \isi{landscape} ({\S}\ref{landscapeterminology}). These sections are sprinkled with anthropological comments.

\section{Sign metonymies}\label{redupmeton}

\subsection{Overview}

This section builds on Evans (\citeyear{Evans:1997vj}), who discusses `sign metonymies' in Australian languages. He points out how biota of different species, families or even kingdoms are connected through sharing a linguistic sign, i.e. they are referred to by the same word or they share a stem. One observation that can be made for Komnzo is the high number of reduplications that are found in plant names, and to some extent in names for animals, especially bird and fish species. In some cases, we have a reduplicative orphan, because the base is missing. In other cases, the base exists only in another language. Most of the time, however, there is a base in the lexicon.

The semantic link between the two referents shows a wide range of complexity. At the lower end, \isi{reduplication} can single out some salient part of one plant, usually the fruit, establishing a relation of non-prototypicality. For example, \emph{mefa} and \emph{mefamefa} refer to two chestnut species (Semecarpus sp), but the nuts of \emph{mefa} are roasted and eaten, while the nuts of \emph{mefamefa} are much smaller. Note that non-prototypicality is a general feature of \isi{reduplication} in Komnzo ({\S}\ref{nomreduplication}). At the upper end of complexity, the \isi{reduplication} pattern links referents through several steps of technical or cultural practices. One example is \emph{ruga} `pig' and \emph{rugaruga} `tree species' (Gmelina ledermannii). The two biota are linked in the following way: \emph{rugaruga} is the tree from which \emph{brubru} `kundu drums' are made. These drums are used for \emph{wath} `dance' or \emph{ruga wath} `pig dance', because a pig will be killed and distributed in the morning hours after the dance. Thus, the technical concept of `drum' and the cultural practice of `dance' mediate between \emph{ruga} `pig' and \emph{rugaruga} `tree species'.

Examples of this type have to be checked thoroughly with several speakers. Otherwise, we run the risk of either (i) documenting folk etymologies, or (ii) not recognizing existing links at all. In an early stage of my fieldwork, the connection between \emph{ruga} and \emph{rugaruga} was explained in terms of spatial relations: the pig is often found in the vicinity of this tree. We will see below that this is true for other connections, but not for this particular example. During a plant walk, I was shown the \emph{rugaruga} tree, and when I invoked the spatial explanation, my informants ruled out that explanation by saying ``pigs roam around anywhere''.

In the cases involving \isi{reduplication}, there is a clear direction from base > \isi{reduplication}. In such cases, we may ask if there are any detectable patterns in the direction of the semantic extensions. Most examples follow the animacy hierarchy in the way that what ranks higher is the base and what ranks lower is the reduplicated form. A list of examples is given in the following sections. For now, we can list the \emph{züm} `centipede', which reduplicates to \emph{zümzüm} `grass species' (\ref{ex615}), or \emph{kwazür} `fish species', which reduplicates to \emph{kwazürkwazür} `grass species' (\ref{ex632}), or \emph{zuaku} `widow(er)', which reduplicates to \emph{zuakuzuaku} `Fly River Anchovy' (\ref{ex642}). Those examples which violate this rule involve \isi{inanimate} referents, like \emph{karo} `anthill' and \emph{karokaro} `grassland goana' (\ref{ex636}). Some of them can be explained by invoking relative salience or importance in every-day life.

Patterns of shared stems not only allow us to gain insight into the local classification of plants and animals, but can also reveal culturally significant connections from plant usage to esoteric knowledge. The following description will group examples by the type of semantic connection. Note that, under Evans' definition, \isi{reduplication} is only one type; identical forms or inflected forms are also included \citep[136]{Evans:1997vj}.

\subsection{Metaphor}\label{redupmetaphor}

Metaphorical links between different biota can be based on movement (\ref{ex613}), appearance (\ref{ex616}), colour (\ref{ex622}), taste (\ref{ex623}), feeling (\ref{ex627}), hearing/sounds (\ref{ex645}), or patterns of human interaction (\ref{ex628}). Note that a few examples link biota to non-biological concepts (\ref{ex617}, \ref{ex620}, \ref{ex627}, \ref{ex629}), and in example (\ref{ex618}) the base is a \ili{Nama} word.

\sloppy
\begin{exe}
\ex \label{ex613}
\begin{xlist}
	\ex	\label{ex614} \emph{dö} `monitor lizard'; \emph{dödö} `broom, plant species' (Melaleuca sp). \textsc{movement:} the lizards ``sweeps'' the floor with its tail when it walks.
	\ex \label{ex615} \emph{züm} `centipede'; \emph{zümzüm} `grass species'; \textsc{movement:} the grass grows along the ground in curves and has little spikes like the centipede.
\end{xlist}
\end{exe}%movement
\begin{exe}
\ex \label{ex616}
	\begin{xlist}
		\ex \label{ex617} \emph{toku} `piggy-back ride (but on the shoulders rather than the back)'; \emph{tokutoku} `Bar-shouldered Dove'; \textsc{appearance:} the bird has a thick brownish line on the back of its neck at the same place where one would carry a child.
		\ex \label{ex618} \emph{min} `nose' in \ili{Nama}; \emph{minmin} `Purple-tailed Imperial Pigeon'). \textsc{appearance:} the bird has large nose-like beak.
		\ex \label{ex619} \emph{msar} `weaver ant'; \emph{msarmsar} `insect larvae, esp. bee larvae' \textsc{appearance:} the bee larvae look like little ants.
		\ex \label{ex620} \emph{garda} `canoe'; \emph{gardagarda} `tree species'; \textsc{appearance:} the seeds ot this tree are long and thin; they crack open lengthwise resembling the shape of a canoe.
	\end{xlist}
\end{exe}%appearance
\begin{exe}
	\ex \label{ex622} \emph{yem} `cassowary' (Casuarius casuarius); \emph{yemyem} `tree species' (Aceratium sp); \textsc{colour:} the fruit of this tree is bright red as the cassowary's skin on its throat.
\end{exe}%colour
\begin{exe}
\ex \label{ex623}
	\begin{xlist}
		\ex \label{ex624} \emph{thatha} `sugarcane' (Poaceae sp); \emph{thathathatha} `grass species'; \textsc{taste:} the grass tastes as sweet as the sugarcane. In the neighbouring variety \ili{Wära}, the grass species is \emph{kthkokthko}, while the word for sugarcane is \emph{kthko}.
		\ex \label{ex625} \emph{with} `banana'; \emph{withwith} `tree species' (Pseudouvaria sp); \textsc{taste:} fruit tastes sweet like a banana.
	\end{xlist}
	\end{exe}%taste
	\begin{exe}
		\ex \label{ex627} \emph{kata} `bamboo knife'; \emph{katakata} `grass species' (Carex sp); \textsc{feeling:} the grass is as sharp as a bamboo knife.
	\end{exe}%tactile
	\begin{exe}
		\ex \label{ex645} \emph{ŋatha} `dog'; \emph{ŋathaŋatha} `Bronze Quoll' (Dactylopsila trivirgata); \textsc{sound:} the bronze quoll barks like a dog.
	\end{exe}%auditory
	\begin{exe}
	\ex \label{ex628}
	\begin{xlist}
		\ex \label{ex629} \emph{tafko} `hat'; \emph{tafkotafko} `tree species' (Macaranga sp); \textsc{interaction:} the large leaves of this tree can be used as a hat against rain or sun.
		\ex \label{ex630} \emph{ŋazi} `coconut' (Cocos nucifera); \emph{ŋaziŋazi} `grass species' (Exocarpus largifolius); \textsc{interaction:} the grass is put on the flowers of a coconut when it flowers for the first time to make it grow strong.
	\end{xlist}
\end{exe}%pattern of human interaction
\fussy 

\subsection{Metonymy}\label{redupmetonymy}

Metonymic links between animals and plants can be of three types: \isi{temporal} (\ref{ex631}), spatial (\ref{ex635}) and technical/cultural (\ref{ex639}). Note that for some examples, the link involves a biological term and a non-biological term, as in \emph{zuaku} `widower, orphan' and \emph{zuakuzuaku} `fly river anchovy' (\ref{ex642}).

\sloppy
\begin{exe}
\ex \label{ex631}
	\begin{xlist}
		\ex \label{ex632} \emph{kwazür} `Narrow-fronted Tandan' (Neosilurus ater); \emph{kwazürkwazür} `grass species' (Helminthostachis zeylanica). \textsc{\isi{temporal}:} the flowering of this grass signals that the fish is greasy; \textsc{human interaction:} fishnets and fishhooks are painted with the root of this plant to ensure a good catch.
		\ex \label{ex633} \emph{tauri} `wallaby'; \emph{tauritauri} `tree species' (Diplanchia hetrophila); \textsc{\isi{temporal}:} In June/July, when the tree flowers, wallabies like to stay close to this tree; people set traps in its vicinity or hide there for hunting wallabies.
		\ex \label{ex634} \emph{dbän} `tree species (Lamiodendron sp)'; \emph{dbän tayo} `yam harvest season' (lit. `weak, ripe \emph{dbän}'); \textsc{\isi{temporal}:} The dry leaves of this tree signal the begin of the yam harvest.
	\end{xlist}
\end{exe}%\isi{temporal}
\begin{exe}
\ex \label{ex635}
	\begin{xlist}
		\ex \label{ex636} \emph{karo} `anthill; ground oven'; \emph{karokaro} `monitor lizard (grassland)' \textsc{spatial:} during the dry season, the grassland goanna likes to dig a hole and hide inside the anthill.
		\ex \label{ex637} \emph{nzöyar} `Fawn-breasted Bowerbird' (Chlamydera cerviniventris); \emph{nzöyarnzöyar} `tree species' (Elaeocarpus sp); \textsc{spatial:} the bowerbird collects the branches and fruit of this tree to build its display area.
		\ex \label{ex638} \emph{dagu} `tree species' (Banksia dentata); \emph{dagu} `python species'; \textsc{spatial:} the python sleeps on the tree. \textsc{appearance:} the bark of the tree looks like the python.
	\end{xlist}
\end{exe}%spatial
\fussy 

More complex connections involve technical concepts (\ref{ex640}) or references to cultural concepts (\ref{ex641}-d).

\begin{exe}
\ex
\label{ex639}
\begin{xlist}
	\ex \label{ex640} \emph{tru} `palm species' (Hydriastele sp); \emph{kwartru} `thin long trough which collects the sago'; \emph{trutru} `current, stream of water' \textsc{technical:} \emph{Kwartru} is made from the palm leaf. While washing the sago pulp, a stream of water runs along the trough collecting the sago flour.
	\ex \label{ex641} \emph{ruga} `pig' > \emph{rugaruga} `tree species' (Gmelina ledermannii); \textsc{cultural:} Pigs are killed during dances, which are often called \emph{ruga wath} `pig dance(s)'. At such dances, \emph{brubru} `kundu drum(s)' are used and the tree \emph{rugaruga} provides the best timber for carving drums.
	\sloppy
	\ex \label{ex642} \emph{zuaku} `widow(er), orphan'; \emph{zuakuzuaku} `Fly River Anchovy' (Thryssa rastrosa); \textsc{cultural:} Widows wear a woven mourning dress from one week up a year after the death of a relative. The bones of the fish look like the woven mourning dress.
	\fussy 
	\ex \label{ex643} \emph{bidr} `flying fox'; \emph{bidr} `joking name for woman'; \textsc{cultural:} This builds on the tree metaphor in which the tree is the origin of people. It may stand for a mythical origin or for one's place of birth. Since women are expected to shift to their husband's village, they behave like flying foxes, who move from one tree to another.
\end{xlist}
\end{exe}%techical/cultural

The most complex connections involve esoteric knowledge. A particularly puzzling example involves the link between the names of two birds and the word for `vulva'. The reduplications \emph{ktikti} and \emph{dirdir} refer to two birds, the  `Greater Streaked Lory' and the `Red-cheeked Parrot', respectively.\footnote{\emph{Ktikti} refers to either the Greater Streaked Lory (Chalcopsitta scintillata) or the Rainbow Lorikeet (Trichoglossus haematodus), or the term covers both. \emph{Dirdir} is the Red-cheeked Parrot (Geoffroyus geoffroyi).} Both words lack a corresponding base in Komnzo. However, the word \emph{dir} [{ⁿ}dır] means `vulva' in \ili{Blafe} and there is a cognate in \ili{Nen} \emph{kter} [kəter] `vulva'. This is to say that the two bird names as well as the words in \ili{Blafe} and \ili{Nen} are cognate, while \emph{nzga}, the Komnzo word for `vulva', is probably not. Note that \ili{Blafe} and \ili{Nen} are spoken about 60km to the west and east, respectively.

The link between the two bird names and the word `vulva' can be explained by the \emph{fütha} myth, which talks about the origin of the bullroarer \citep[80]{Ayres:ws}. \emph{Fütha} is a story place in Rouku. This story also appears in \citep[307]{Williams:1936transfly} as an episode of the Kwavaru myth. According to the myth, a man hears a roaring noise coming from his wife's belly. He wonders what is causing the noise. He wants to have this object. So he calls several birds  to fetch that object from his wife's vagina. Many of them fail, but in their attempts, they spill blood on themselves. That is why their plumage contains patches of red. Finally, one of the birds is successful. It steals the bullroarer and brings it to the man. Since then, the bullroarer is a sacred object, only for initiated men. In William's version, the woman breaks down bleeding and crying, and thus, the story also explains the origin of menstruation.

The \isi{reduplication} pattern makes reference to the red plumage of the two birds \emph{ktikti} and \emph{dirdir}. Moreover, there are other small birds with red colour that involve these words. For example, \emph{kti tharthar} `Spangled Drongo' has bright red eyes, and the word \emph{tharthar} means `side, next to'. The `Red-flanked Lorikeet' \emph{ŋazi dirdir} has a red beak and red sides, but the connection to \emph{ŋazi} `coconut' can be explained by its behaviour rather than relating to the red colour. These birds like to sit in coconut palms. The `Orange-breasted Fig-parrot' \emph{kor dirdir} has red cheeks, but the meaning of \emph{kor} is unclear.\footnote{Spangled Drongo (Dicrurus bracteatus), Red-flanked Lorikeet (Charmosyna placentis) and Orange-breasted Fig-parrot (Cyclopsitta gulielmiterti).} The shared linguistic sign links these bird names to bases meaning `vulva'. But this is esoteric knowledge, which should not be shared with women or uninitiated men. Therefore, the link is hidden by using words from distant languages: not from Komnzo, nor from neighbouring languages.

\subsection{Conclusion}\label{redupconcl}

This has been a preliminary analysis of the data on sign metonymies. Many examples have been collected, but more comparative data is needed to explain the semantic links. Data from the surrounding languages can provide two kinds of evidence; first, there will be more cases in which the base comes from another language, as in (\ref{ex618}), or in the myth described above. Secondly, we may find that the same biota are linked in other languages. Two examples of this come from \ili{Wära} and \ili{Blafe}. In \ili{Wära}, the link between sugarcane and a particular grass species (\ref{ex624}) is established by the \isi{reduplication} of the non-cognate word \emph{kthko} [kə̆θko]. In \ili{Blafe}, the \isi{temporal} link between the fish and the grass species (\ref{ex632}) is established by the cognate word \emph{bäwr} [bæwə̆r].

\section{Landscape terminology}\label{landscapeterminology}

\subsection{Conceptualisation of landscape}\label{cncptlndscp}

Williams opens his monograph about the Morehead district with the following description of the \isi{landscape}: ``Its scenery often has a mild, almost dainty, attractiveness in detail, but represents on the whole the extreme of monotony'' {\citeyear[1]{Williams:1936transfly}}. The Komnzo terminology reflects Williams' observation. There are general terms for \isi{landscape} types, but we also find words expressing very specific local arrangements. For example, while there is a general distinction between \emph{fz} `forest', \emph{ksi kar} `open grassland' and \emph{fath} `clear place', we also find fine-grained distinctions like \emph{fokufoku} `small patch of forest', \emph{fz minz} `thin strip of forest' (lit. `forest vine'), \emph{thaba} `clearing surrounded by forest' and \emph{morthr} `edge of forest with a smaller patch forest close by'. Some of the more general terms are shown with pictures in {\S}\ref{geographyenviro}.

Large parts of the Morehead district are inundated by rising water during the wet season. This usually takes place between January and June, but there is some fluctuation from year to year. It is hardly surprising that this regular cycle has found its way into the lexicon of Komnzo. I invite the reader on a walk from the high ground down to the river. I translate the term \emph{töna} as `high ground'. It is that part of the land, regardless of vegetation type, which is virtually never covered by water. Settlements and yam gardens are located on \emph{töna}. Small hills are referred to by \emph{märmär} or the Motu loan \emph{ororo}.\footnote{Nowadays, Komnzo speakers refer to people from the highlands as \emph{märmär kabe} `hill people'.} These areas may become islands (\emph{bod}) during high floods. Wide, gentle slopes (\emph{rsrs}) lacerated by many small creeks (\emph{ttfö}) lead to lower areas. It is often along creeks where people plant sago palms or sometimes taro. Closer to the river, the ground can be very uneven and bumpy due to running water. This is called \emph{kore}. A little lower lies that part of the land which is always covered by water during the rainy season. Often backwater stays in stagnant pools, which dry up only during the height of the dry season. These places are called \emph{zra}, which I translate with `swamp', but maybe the term `billabong', commonly used in Australian \ili{English}, is more fitting. In this area, we find smaller pools of water which dry up  (\emph{nawan}) and larger pools which are permanent (\emph{dmgu}). The ankle-deep, muddy water covered with leaves is called \emph{nzäwi}. Walking towards the river, the land rises again in many places. This difference in elevation is almost unnoticeable, but it is enough so that this area dries up first at the end of the wet season. These area between the swamp and the river are called \emph{for} and people plant cassava, sweet potato and taro here. The steep riverbanks along the Morehead river are called \emph{rokuroku}, a word from which the village name Rouku originates. The sides of the river are covered with patches of \emph{süfi} `floating grass', and in some places this layer is called \emph{tüf} when it is thick enough to support the cultivation of sweet potatoes. Finally, there is the river which is called \emph{ŋars}. Although found only in the southwest around Bensbach, large open lagoons are called \emph{füwä} in Komnzo.

Especially in dry season much of people's daily life involves coming and going from the high ground to the river. This movement has left some impact in the verb lexicon. For example, the stem \emph{frezsi} usually means `take something out of the water'. In a middle template it means `come up from the river' and can be used when disembarking a canoe, or walking back from a river camp to the village.

There are numerous creeks leading to the Morehead river. The mouth of a creek or a river is referred to by \emph{zfth} `base'. This word can refer to the base of a tree, but it can also mean `origin, reason'. Interestingly, the smaller creeks may be called \emph{ttfö tuti} `creek branches, creek twigs' or \emph{ttfö minz} `creek vines'. The place where the creeks start can be called either \emph{ttfö ker} `creek tail' or \emph{ttfö zrminz} `creek root'. The same can be said about the Morehead river. Thus waterways are often conceptualised by a tree \isi{metaphor}. This stems from the \emph{kwafar} myth, which associates the origin of all people with a tree. Kwafar is located somewhere in the Arafura Sea between Papua New Guinea and Australia. In the myth, the tree burns down and a flood caused by killing a mythical creature forces people to retreat northwards and southwards. The roots in the ground also burn and with the rising water they become creeks and rivers. In other versions of the myth, the tree falls northwards and the creeks and rivers are formed from the burned stem, branches, and twigs of the tree.

\subsection{Place names}\label{placenames}

Place names in the Morehead district are both numerous and densely clustered \citep[129]{Ayres:ws}.\footnote{I would like to thank Mary Ayres for giving me access to her fieldnotes which proved to be enormously helpful during the elicitation and investigation of place names.} The village of Rouku alone consists of some three dozen named places. The knowledge of most place names is common knowledge, for example Williams notes that ``if you ask your guide where you stand at any moment, he will be able to give a name to the land.'' (\citeyear[207]{Williams:1936transfly}). However, the details of every small track and the stories that belong to it is something only known by the rightful owners of that piece of land. In that sense, knowledge about place names can be compared to a proof of ownership. Therefore, I deliberately do not include a complete list of collected place names, nor do I provide a detailed map. Below, I address selected topics related to place names.

All place names in Komnzo are proper nouns, but they differ with respect to their meaning. Some place names have no meaning other than the places they designate, for example \emph{fthi}, \emph{kanathr} or \emph{ŋazäthe}. At some point in the past, they might have been segmentable into meaningful parts or constitute a meaningful word in themselves, but this knowledge has faded away.\footnote{Two of these examples look like inflected verb forms; \emph{kanathr} is similar to an imperative form of `eat' in a middle template: \emph{kanathé} `eat yourself!'; \emph{ŋazäthe} contains the middle prefix \emph{ŋ-}, a possible non-dual marker \emph{-th} and the first non-singular suffix \emph{-e}. However, the assumed verb stem \emph{zä-} does not exist in modern Komnzo.} Place names commonly preserve features which have become non-productive or lexemes which have become archaic. This can also be found in Komnzo. For example, the \isi{place name} \emph{thmefi}, meaning `moustache', can be split into the components \emph{thm} `nose' and \emph{efi} `hair'. However, the word \emph{efi} is archaic, and instead \emph{thäbu} is used. In fact, some speakers are unaware of the possible segmentation.

More commonly, Komnzo place names consist of two elements, which usually form a nominal compound. These compounds range from rather dry descriptions, like \emph{gani zfth} `base of the \emph{gani} tree' (Endiandra brassii), to the most colourful illustrations, as in \emph{nzga warsi} `vulva chewing', \emph{kwanz fath} `bald head clearing'. Many nominal compounds consist of a plant name plus a \isi{landscape} term or a term used for the part of a plant. The most common \isi{landscape} terms are \emph{zra} `swamp, waterhole' and \emph{ttfö} `creek'. The most common plant part terms are \emph{zfth} `base' and \emph{fr} `stem, grove'.\footnote{The word \emph{zfth} can mean (i) `base of a plant, tree', (ii) `rivermouth', (iii) `origin' or (iv) `reason'.} A few examples are: \emph{karesa zfth} `\emph{karesa} base' (Melaleuca sp), \emph{atätö fr} `\emph{atätö} stem' (Pouteria sp), \emph{wsws zra} `\emph{wsws} swamp' (Combretum sp). These are not descriptions of places, but place names. A phrase like \emph{karesa zfth} can refer to the base of any \emph{karesa} tree, but it refers only to one named place.

A few place names are inflected verb forms, for example \emph{karifthe} `you two send each other off!'.\footnote{\parbox{0.02cm}{\hfill}\parbox{6cm}{ka\stem{rifth}e}\\ \parbox{0.001cm}{\hfill}\parbox{6.06cm}{\Stdu:\Sbj:\Imp:\Pfv/send}} This place connects to a myth in which the ancestor of the Garaita people and the ancestor of the Rouku people were fighting. At the end of the story, they depart in opposite directions from \emph{karifthe}. Another name which includes an inflected verb is \emph{kafthé fr}. The first element is means `take off your bag!' and the second means `stem, grove'. Interestingly, \emph{kafthé} is not Komnzo, but \ili{Wartha}.\footnote{Imperative perfectives in Komnzo mark dual versus non-dual with a vowel change in the prefix, and the suffix is zero for second singular. The corresponding Komnzo verb form would be \emph{käthf}.} I address the topic of mixed language place names in \S\ref{mixedplacenames}. For some place names, there is no etymology available, for example \emph{yrn} `they are many'.

Simpson and Hercus (\citeyear{Hercus:2002ul}) provide a list of differences between introduced and indigenous place names in Australia. In the following, I apply some points of their typology to the Morehead district. The first point which Hercus and Simpson discuss is the difference between a system and a local network. The former is meant to provide an overview, a kind of standardised template for naming places, which can be applied universally and is open to everyone. Komnzo place names, like indigenous place names in Australia, differ in that they often constitute smaller networks of place names. For example, the number of named places is much denser in the vicinity of inhabited places or previously inhabited places. Moreover, places or tracks of land belong to a particular clan, and the detailed knowledge about these places, which may sometimes include place names, is not meant for the public.

A second difference raised by Hercus and Simpson is that between local mnemonics and mnemotechnics. They point out that place names have developed organically over a long time as local mnemonics to refer to places. This applies to places in Europe and Australia (or the Morehead district) alike, but not to introduced \isi{place name} systems. For example, the Komnzo \isi{place name} \emph{swäri zfth} `\emph{swäri} base' must have started as the description of a place with an especially large or beautiful \emph{swäri} tree (Alstonia actinifila), but over time it has lost its descriptive function. Today it is used even though the \emph{swäri} tree was cut down decades ago. Francesca Merlan has described \isi{place name} systems of this kind as being ``non-arbitrary'', because they establish a direct relationship to the designated places \citep{Merlan:2001wp}. In contrast to local mnemonics, technological advances like writing and mapping provides a kind of mnemotechnics, which opens the possibility to include arbitrary place names like \emph{Sydney} or \emph{Port Moresby}.

Simpson and Hercus outline three naming strategies that are rarely found in indigenous Australia: commemoration strategies, topographic descriptors and relative location. Commemoration strategies are wholly absent in Komnzo place names. They are only found in those names introduced by Europeans. For example, the Morehead river was named after B. D. Morehead, who was the premier of Queensland between 1888 and 1890. The Bensbach river was named during a joint expedition in 1895 by W. MacGregor and J. Bensbach who was the Dutch Resident at Ternate at the time. While Hercus and Simpson point out that topographic descriptors are rare in indigenous Australia, they are quite common in Komnzo. However, as pointed out above, they include only a small set of words (\emph{zfth} `tree base', \emph{fr} `stem, grove' or \emph{ttfö} `creek'). Relative terms like \emph{North Melbourne} or \emph{West Berlin} are almost completely absent in Komnzo, as they are in Australian languages. The only example in which a \isi{place name} establishes a relation to another place is \emph{fthiker}. The link here is a creek which has its mouth at place called \emph{fthi} and its starting point at \emph{fthiker} `\emph{fthi} tail'. Note that creeks themselves are usually not named, but the word \emph{ttfö} `creek' can be added to a place situated on a creek.

\subsection{Mixed place names}\label{mixedplacenames}

An interesting phenomenon that sheds some light on \isi{multilingualism} is the fact that many place names are composed of words from two languages. I refer to these as mixed place names. Most of them involve one Komnzo word. But in a few place names both words are from different languages even though the place is located on Komnzo speaking territory. The basic principle of mixed place names is shown in Figure \ref{duplaplace} for \emph{fotnz}. This is a place near Rouku village, which can be parsed as one word from \ili{Wartha} Thuntai and one word from Komnzo.

\begin{figure}[H]
\begin{center}%
\begin{tikzpicture}
	\draw[->,thick] (3.8,1.5) --(5.2,0);
	\node[right] at (0,1.5) {\framebox[3.5cm]{\emph{fo} [ɸo:] `coconut'}};
	\node[right] at (0,0) {\framebox[3.5cm]{\emph{ŋazi} `coconut'}};
	\node[right] at (5.2,1.5) {\framebox[3cm]{\emph{tg} [tə̆{\ᵑ}k] `short'}};
	\node[right] at (5.2,0) {\framebox[3cm]{\emph{tnz} `short'}};
	\node[right] at (-2.3,1.5) {Wartha};
	\node[right] at (-2.3,0) {Komnzo};
	\node[right] at (-2.3,2.7) {\isi{place name}: \emph{fotnz} `short coconut'};
\end{tikzpicture}
\end{center}
\caption{The principle of mixed place names: \emph{fotnz}}\label{duplaplace}
\end{figure}%The principle of mixed place names

This principle is rather pervasive. A quarter of recorded place names involve a word from another language. I give a few examples in (\ref{ex656}-\ref{ex658}). These are sorted according to whether the Komnzo word is the first (\ref{ex656}) or last element (\ref{ex657}). I show the \isi{place name} as a single word in most cases, because often speakers only realised their segmentability when I prompted them. This is followed by a literal \ili{English} translation of the contributing elements, after which the two languages are given. In parentheses, I provide the two words in each language. Note that I follow the Komnzo orthography here, because with the exception of \ili{Nama} there is no orthography available for these varieties. A few cases are problematic because one of the two words is identical in the contributing languages (\ref{ex658}). However, all examples designate places on Komnzo speaking territory.

\begin{exe}
\ex
\label{ex656}
\begin{xlist}
	\ex \emph{fotnz} `coconut + short'; \ili{Wartha} Thuntai (\emph{\textbf{fo}} \emph{tg}) + Komnzo (\emph{ŋazi} \emph{\textbf{tnz}}).
	\ex \emph{säzäri} `paperbark + bending over'; \ili{Wartha} Thuntai (\emph{\textbf{sä}} \emph{ytho}) + Komnzo (\emph{karesa} \emph{\textbf{zäri}}); the word \emph{zäri} `bending (branches)' is considered archaic, but there is the modern word \emph{zäre} `shade'.
	\ex \emph{tratrabäk} `bird species + back'; \ili{Kánchá} (\emph{\textbf{tratra} bak}) + Komnzo (\emph{drädrä \textbf{bäk}}).
	\ex \emph{makozanzan} `vagina + beating'; \ili{Arammba} (\emph{\textbf{mako} kamakama}) + Komnzo (\emph{nzga \textbf{zanzan}}).
	\ex \emph{füsari} `garden row + axe'; \ili{Nama} (\emph{\textbf{fü}} \emph{bilé}) + Komnzo (\emph{ŋanz} \emph{\textbf{sari}}); The word \emph{sari} is considered archaic.
	\ex \emph{düdüsam} `broom + liquid'; \ili{Nama} (\emph{\textbf{düdü}} \emph{wkwr}) + Komnzo (\emph{dödö} \emph{\textbf{sam}}).
	\ex \emph{fakwr} `after + ashes'; \ili{Nama} (\emph{\textbf{fa}} \emph{fak}) + Komnzo (\emph{thrma} \emph{\textbf{kwr}}).
	\ex \emph{wästhak} `tree species (Ficus elastica) + place'; \ili{Nama} (\emph{\textbf{wäs} näk}) + Komnzo (\emph{wäsü \textbf{thak}}); the word \emph{thak} is archaic in Komnzo and only found in \emph{mni thak} `fire place'.\label{ex659}
\end{xlist}
\end{exe}%second element is Komnzo
\begin{exe}
\ex
\label{ex657}
\begin{xlist}
	\ex \emph{zthékabir} `penis + sleep(n)'; Komnzo (\emph{\textbf{zthé}} \emph{etfth}) + \ili{Wära} (\emph{zthk} \emph{\textbf{kabir}}).
	\ex \emph{snzäzwär} `river crayfish + base'; Komnzo (\emph{\textbf{snzä} zfth}) + \ili{Wartha} (\emph{dawi \textbf{zwär}}).\label{ex660}
	\ex \emph{ormogo} `Emerald Dove + house'; Komnzo (\emph{\textbf{or}} \emph{mnz}) + \ili{Nama} (\emph{bänz} \emph{\textbf{mogo}}).
	\ex \emph{yem gi faf} `cassowary killing place'; Komnzo (\emph{\textbf{yem} zan \textbf{faf}}) + \ili{Nama} (awyé \textbf{gi faf}).
	\ex \emph{märofak} `tree species (Dillenia ensifolia) + ashes'; Komnzo (\emph{\textbf{märo} kwr}) + \ili{Nama} (\emph{mane \textbf{fak}}).
\end{xlist}
\end{exe}%first element is Komnzo
\begin{exe}
\ex
\label{ex658}
\begin{xlist}
	\ex \emph{sizwär} `eye + base'; Komnzo (\emph{\textbf{si} zfth}) + \ili{Wartha} Thuntai (\emph{\textbf{si zwär}}).
	\ex \emph{gawe} `I + also'; Komnzo (\emph{nzä \textbf{we}}) + \ili{Wartha} Thuntai (\emph{\textbf{ga we}}).
	\ex \emph{mnzärfr} `ant + stem'; \ili{Nama} (\emph{\textbf{mnzär fr}}) + Komnzo (\emph{msar \textbf{fr}}).
	\ex \emph{zöfäthak} `bird + place'; \ili{Wära} (\emph{\textbf{zöfä thak}}) + Komnzo (\emph{ymd \textbf{thak}}); the word \emph{thak} is archaic in Komnzo and only found in \emph{mni thak} `fire place'.
\end{xlist}
\end{exe}%one of the two elements exists in Komnzo AND the other language

Mixed place names pattern roughly according to geography. For example, place names containing \ili{Nama} words are mostly found east of Rouku, while place names involving \ili{Wartha} Thuntai words are mostly found to the southwest. There are many exceptions, where (i) the place does not fit geography or (ii) the `foreign' word could be from more than one language. However, the overall pattern suggests that geography plays a role. Thus, if we showed these places on a map and marked them for the contributing `foreign' languages, we could geographically visualise speech varieties. Data from other villages and their place names is needed to corroborate this observation.

The pattern of mixed place names calls for an investigation of naming customs. However, as with most place names, the point in time when such double language names were coined is far removed. Most of my informants did not remember anyone giving a name to these places. A common response was ``we learned them from our fathers''. In fact, most informants find the idea of naming a place somewhat strange. That being said, we can still draw some conclusions about naming customs. Mixed place names differ from the monolingual, descriptive place names in one important aspect. One can imagine a gradual transition from a description to a proper name like \emph{swäri zfth} `\emph{swäri} base' mentioned above. With mixed place names such a transition is an unlikely scenario. Instead, a more deliberate act of coining the name has to be assumed. Note that we also find monolingual place names, where a transition from description to proper name can be ruled out on semantic grounds, for example \emph{nzarga wth} (`tree species + faeces') or \emph{zäzr mnz} (`lazy + house'). However, the point here is that a gradual transition is unlikely because two languages are involved, even if the name is of a more descriptive nature like (\ref{ex659}) and (\ref{ex660}). These observations authenticate the importance of place in Morehead culture, an argument that was put forward by Ayres (\citeyear{Ayres:ws}).

Mixed place names can shed some light on the degree of \isi{multilingualism} in the language communities concerned. There are varying degrees of metalinguistic awareness both between different place names and between different speakers. That is to say that speakers differ in their language profiles, and ultimately differ in how much access they have to the word in the ``foreign language''. Moreover, some place names are easier to parse, while others have undergone phonological reduction or one of the segments has become archaic. Generally speaking, most speakers are aware of these double language names and the meaning in the respective languages. One observation that can be made is the complete absence of doublets, that is cases were both terms refer to the same referent, but in different languages. There are examples of doublets in Komnzo, but not for place names. For example, there is a cassava species called \emph{ubi biskar}. The word \emph{ubi} is from Malay and the word \emph{biskar} is a Komnzo word, but both mean `cassava'. This type of doublet is to be expected if the speakers who coined the name did not know the meaning of the foreign word. The pattern that we find with place names suggests the opposite. At the time of coinage, one has to assume a degree of \isi{multilingualism} at least as high as today.

\subsection{Social landscape}\label{sociallandscape}

This section addresses the topic of social \isi{landscape}, by which I mean the reference system used for people in relation to space. The Komnzo terms for this domain conceptualise either pure geography or what we may call kinship-dependent geography. The importance of place in the Morehead district has been described in great detail by Mary Ayres. I sketch out the sister-exchange system only where it is relevant to the discussion. Otherwise I refer the reader to Ayres (\citeyear{Ayres:ws}) and \S\ref{exogamy}.

The purely geographic terms are based on an east-west axis. The people who live in the east are referred to with the word \emph{nzödmä}, while the people in the west are called \emph{smärki}. These labels are often only applied to people living two villages away. They are rarely used for one's immediate neighbours. The system is ego-centric in that the same labels or cognate terms are applied in other villages. If one moves further west, the term \emph{güdmä} [{\ᵑ}gʏ{ⁿ}dmæ] is used for everyone to the east, including the people of Rouku.\footnote{The word \emph{güdmä} in \ili{Nama} and \ili{Blafe} are cognate with Komnzo \emph{nzödmä}. In Komnzo, \ili{Wära}, \ili{Anta}, and \ili{Wèré} velar stops have undergone palatalisation before front vowels, for example [{\ᵑ}g] > [{ⁿ}dʒ].} Likewise the people in the east would call everyone who lives west of them \emph{smärki}. Thus, the terms \emph{nzödmä} and \emph{smärki} do not refer to a specific group, but mean `people from the east' and `people from the west', respectively. This has caused some confusion for early ethnographers \citep[36]{Williams:1936transfly}, but was explained by Ayres (\citeyear[132]{Ayres:ws}). Furthermore, the east-west axis is validated by the term \emph{tharthar kabe} `people on the side', which is used for the \ili{Arammba} speakers in the north.\footnote{Often the \ili{Arammba} phrase \emph{sarsar ŋar} is used, which has the same meansing.} The naming system is a correlate of a fact about the region's geography. Most villages are built on what is called the ``Morehead ridge'' (\citealt[15]{Paijmans:1971morehead}), a slighty elevated ridge that runs in east-west direction. Further north, the speakers of \ili{Suki} and Gogodala are collectively labelled with the proper name \emph{wiram}. Also the people in the south do not fit in the east-west schema, but are instead referred to by proper names, for example \emph{wartha}, or they may be called \emph{mazo kabe} `coast people'. Groups which live further away have proper names, for example the \ili{Kanum} and \ili{Marind} speakers in the west are called \emph{kodomarid}, and the speakers of \ili{Kiwai} are called \emph{turéd}.

As pointed out by Ayres, people define themselves as belonging to a particular origin place. The ancestors of different clans and sections might have arrived from different directions, but they ``spread out'' from the same origin place. Hence, people can be referred to by their origin place. For Komnzo speakers, this is \emph{farem kar} `farem place', which is situated about 3km northwest of Rouku. Other examples are \emph{mät} for the people of Yokwa or \emph{thamga} for the people of Uparua.\footnote{\emph{mät} is a term referring to the red colour of the ground, and the villaga \emph{Mata} in the east derives its name from the same word. There is no etymology for \emph{thamga} or \emph{farem}.} Origin places usually overlap with language variety, in that a speaker of \ili{Wära} belongs to \emph{mät kar}, whereas a speaker of \ili{Anta} belongs to \emph{thamga kar}.

The \isi{kinship} system gives rise to yet another, very common way of referring to people. The rules of \isi{exogamy} involve a number of factors. Some are related to place, for example identification with a particular origin place establishes an exogamous group. Some are related to the section system, for example the Mayawa section regardless of place forms an exogamous group. The section classification cross-cuts the place system, i.e. one may not marry people from the same origin place, but also not from a different place if they belong to the same section. Additionally, people who ``share a land boundary'' may not intermarry. That is to say that two individuals may not marry even if they belong to different places and different sections, if their land is adjacent. Ayres argues convincingly that locality forms the most important factor in the complicated definitions of exogamous groupings (\citeyear[{\S}5]{Ayres:ws}). If \isi{kinship} is conceptualised in terms of space, it follows that \isi{kinship} terms can be used to refer to people of a particular place. I often overheard people talking about their \emph{ngom kar} `brother-in-law place' or \emph{thuft kar} `in-law place'. Note that the calculations one has to make to arrive at the correct referent are rather complex. Not only does one individual normally have several brothers-in-law, but that different individuals have different in-laws. Nevertheless, such knowledge is common ground for the people of Rouku. Although I often found it difficult to identify the referent in an utterance like (\ref{ex661}), every child in Rouku could make the correct deduction without effort.

\begin{exe}
	\ex \emph{watik kraritth bern ... sukufa ärithr \textbf{nafathufthnm} ... \textbf{nafangom karnm}.}\\
	\gll watik kra\stem{rit}th b=e\stem{rn} (.) sukufa ä\stem{ri}thr nafa-thufth=nm (.) nafa-ngom kar=nm\\
	then \Stdu:\Sbj:\Irr:\Pfv/go.across \Med=\Stdu:\Sbj:\Nonpast:\Ipfv/be (.) tobacco \Stsg:\Sbj>\Stpl:\Io:\Nonpast:\Ipfv/give \Third.\Poss-{in.law}=\Dat.{\Nsg} (.) \Third.\Poss-{brother.in.law} village=\Dat.{\Nsg}\\
	\trans `Then they went across there ... He shared tobacco with his in-laws ... with the people of his brother-in-law place.'\Corpus{tci20111119-01}{ABB \#88-91}
	\label{ex661}
\end{exe}
