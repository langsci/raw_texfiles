\section{Case}\label{sec:7}



\subsection{Overview}\label{sec:7.1}


Ik has a \textsc{case} system. This means that every noun has a special marking to show what role it has in the sentence. The language marks this role by means of a set of case \textsc{suffixes} (endings). Four of the cases are marked with suffixes consisting of a single vowel, while for three others, the suffix consists of /k/ plus a vowel. An eighth case, the oblique, is marked by the absence of any suffix. In the following examples, \REF{ex:case:1}-\REF{ex:case:8}, notice how the word \textit{ŋókí-} ‘dog’ at the end of each sentence has a different ending depending on the case for which it is marked:



 
\ea\label{ex:case:1}
\gll Atsa     ŋókᵃ. \\ 
come:\textsc{3sg}   dog:\textsc{nom}    \\
\glt ‘The dog comes.’ 
\z


 
\ea\label{ex:case:2}
\gll Cɛa     boroka     ŋókíkᵃ. \\ 
kill:\textsc{3sg}   bushpig:\textsc{nom}   dog:\textsc{acc}    \\
\glt ‘The bushpig kills the dog.’ 
\z



 
\ea\label{ex:case:3}
\gll Maa     eméá     ŋókíkᵉ. \\ 
give:\textsc{3sg}   meat:\textsc{acc}   dog:\textsc{dat}    \\
\glt ‘He gives meat to the dog.’ 
\z



 
\ea\label{ex:case:4}
\gll Mɨta     ima     ŋókí. \\ 
be:\textsc{3sg}   child:\textsc{nom}   dog:\textsc{gen}    \\
\glt ‘It is the child of the dog.’ 
\z



 
\ea\label{ex:case:5}
\gll Xɛɓa     ŋókú. \\ 
fear:\textsc{3sg}   dog:\textsc{abl}    \\
\glt ‘He fears the dog.’ 
\z


 
\ea\label{ex:case:6}
\gll Ƙaa     ŋókᵒ. \\ 
go:\textsc{3sg}  dog:\textsc{ins}    \\
\glt ‘He goes with the dog.’ 
\z



 
\ea\label{ex:case:7}
\gll Bɛna     ŋókúkᵒ. \\ 
not.be:\textsc{3sg}  dog:\textsc{cop}    \\
\glt ‘It is not a dog.’ 
\z



 
\ea\label{ex:case:8}
\gll Mɨta     ŋókⁱ. \\ 
be:\textsc{3sg}  dog:\textsc{obl}    \\
\glt ‘It is a dog.’ 
\z


Eight examples are given above because Ik has eight cases: nominative, accusative, \isi{dative}, genitive, ablative, instrumental, copulative, and oblique. \tabref{tab:case:suffixes} presents the non-final and final forms of the suffixes that mark all eight of these cases. Keep in mind that the null symbol <Ø> signifies either 1) that the case suffix is inaudible or, for the \isi{oblique case}, 2) that there is no case suffix:


\begin{table}
\caption{Ik case suffixes}
\label{tab:case:suffixes}


\begin{tabularx}{\textwidth}{XXXX}
\lsptoprule

Case & Abbreviation & Non-final & Final\\
\midrule
Nominative & \textsc{nom} & {}-a & {}-ᵃ/-\textsuperscript{Ø}\\
Accusative & \textsc{acc} & {}-a & {}-kᵃ\\
Dative & \textsc{dat} & {}-e & {}-kᵉ\\
Genitive & \textsc{gen} & {}-e & {}-e/-\textsuperscript{Ø}\\
Ablative & \textsc{abl} & {}-o & {}-ᵒ/-\textsuperscript{Ø}\\
Instrumental & \textsc{ins} & {}-o & {}-ᵒ/-\textsuperscript{Ø}\\
Copulative & \textsc{cop} & {}-o & {}-kᵒ\\
Oblique & \textsc{obl} & {}-Ø & {}-\textsuperscript{Ø}\\
\lspbottomrule
\end{tabularx}
\end{table}
From \tabref{tab:case:suffixes}, there may appear to be significant ambiguity in the Ik case system. For instance, the non-final forms of the nominative and accusative suffixes, the \isi{dative} and genitive suffixes, and the ablative, instrumental, and copulative suffixes all look the same, respectively. In most cases, the key to disambiguating the suffixes is called ‘subtractive’ morphology. Two of the Ik case suffixes (namely nominative and instrumental) are subtractive in that they subtract or delete the final vowel of the noun to which they attach. So, for example, while the non-final forms of the nominative and accusative are identical, their morphological behavior is not: the nominative \{-a\} subtracts the noun’s final vowel, as when \textit{ŋókí-} ‘dog’ becomes \textit{ŋók-á} ‘dog:\textsc{nom}’; by contrast, the accusative suffix is non-subtractive, as in \textit{ŋókí-à} ‘dog:\textsc{acc}’. Other case ambiguities like genitive versus \isi{dative} and ablative versus copulative, in their non-final forms, can be resolved in the context of the sentence. Different verbs require different cases.

Since every Ik noun ends in a vowel, and since that vowel can be any of the nine (/i, ɨ, e, ɛ, a, ɔ, o, ʉ, u/), the collision of nouns and case suffixes gives rise to all kinds of \isi{vowel assimilation} (see \sectref{sec:2.4.4}). The next two tables present declensions of two nouns illustrating \isi{vowel assimilation}. \tabref{tab:case:feti} shows the noun \textit{fetí-} ‘sun’ declined for all eight cases. In particular, notice how the vowel /o/ in the ablative and copulative suffixes partially assimilate the /i/ in \textit{fetí-} to become /u/:


\begin{table}
\caption{Case declension of \textit{fetí-} ‘sun’}
\label{tab:case:feti}
\begin{tabularx}{.66\textwidth}{XXX}
\lsptoprule
Case & Non-final & Final\\
\midrule
\textsc{nom} & feta & fetᵃ\\
\textsc{acc} & fetíá & fetíkᵃ\\
\textsc{dat} & fetíé & fetíkᵉ\\
\textsc{gen} & fetíé & fetí\\
\textsc{abl} & fetúó & fetú\\
\textsc{ins} & feto & fetᵒ\\
\textsc{cop} & fetúó & fetúkᵒ\\
\textsc{obl} & feti & fetⁱ\\
\lspbottomrule
\end{tabularx}
\end{table}
While \tabref{tab:case:feti} shows partial \isi{vowel assimilation} caused by case suffixation, \tabref{tab:case:kija} reveals an instance of total assimilation. In this table, the noun \textit{kíʝá-} ‘land’ is declined for all eight cases. Note how the final /a/ of \textit{kíʝá-} becomes totally assimilated by the non-final \isi{dative}, genitive, ablative, and copulative suffixes.


\begin{table}
\caption{Case declension of \textit{kíʝá-} ‘land’}
\label{tab:case:kija}
\begin{tabularx}{.66\textwidth}{XXX}
\lsptoprule
Case & Non-final & Final\\
\midrule
\textsc{nom} & kíʝá & kíʝᵃ\\
\textsc{acc} & kíʝáà & kíʝákᵃ\\
\textsc{dat} & kíʝéè & kíʝákᵉ\\
\textsc{gen} & kíʝéè & kíʝáᵉ\\
\textsc{abl} & kíʝóò & kíʝáᵒ\\
\textsc{ins} & kíʝó & kíʝᵒ\\
\textsc{cop} & kíʝóò & kíʝákᵒ\\
\textsc{obl} & kíʝá & kíʝᵃ\\
\lspbottomrule
\end{tabularx}
\end{table}




\subsection{Nominative (\textsc{nom})}\label{sec:7.2}
\largerpage

The \textsc{nominative} case, marked by the suffix \{-a\}, is the ‘naming’ case, whose role is to: 1) mark the subject of main clauses, 2) mark the subject of sequential clauses (see \sectref{sec:8.10.7}), and 3) mark the direct object of clauses with 1\textsuperscript{st} and 2\textsuperscript{nd} person subjects (‘I’, ‘we’, ‘you’). Three examples (\REF{ex:case:9}-\REF{ex:case:11}) are provided below, each one illustrating one of the three grammatical roles of the \isi{nominative case}. The third example contains seven sentences that show how Ik object marking is \textsc{split}: objects after 3-person subjects ((s)he/it, they) take the \isi{accusative case}, while 1- or 2-person subjects (you, we) take objects in the \isi{nominative case}:\\




Subject of a \isi{main clause}
\ea\label{ex:case:9}
\gll Atsáá   lɔŋ\'{ɔ}t-\textbf{\ᵃ}! \\
come:\textsc{prf}   enemies-\textsc{nom}    \\
\glt ‘The enemies have come!’ 
\z




Subject of a sequential clause
\ea\label{ex:case:10}
\gll Toɓuo   ƙaƙaam-\textbf{a}   kʉláɓákᵃ. \\
spear:\textsc{seq}   hunter-\textsc{nom}   bushbuck:\textsc{acc}    \\
\glt ‘And the hunter speared the bushbuck.’ 
\z




Object of a clause with a 1/2-person subject




\ea\label{ex:case:11}
  \ea
  \gll Ŋƙ{\Í}á   tɔbɔŋ-\textbf{a}=na. \\
eat:\textsc{1sg}   mush-\textsc{nom}=this    \\   
  \glt ‘I eat this meal mush.’
  \ex
  \gll Ŋƙ{\Í}da   tɔbɔŋ-\textbf{a}=na. \\
eat:\textsc{2sg}   mush-\textsc{nom}=this    \\
  \glt ‘You eat this meal mush.’
  \ex
  \gll Ŋƙa   tɔbɔŋ\'{ɔ}-á=na. \\
eat:\textsc{3sg}   mush-\textsc{acc}=this    \\
  \glt ‘She eats this meal mush.’
  \ex
  \gll Ŋƙ{\Í}má     tɔbɔŋ-\textbf{a}=na \\
eat:\textsc{1pl.exc}   mush-\textsc{nom}=this    \\
  \glt ‘We eat this meal mush.’
  \ex
  \gll Ŋƙ{\Í}s{\Í}na     tɔbɔŋ-\textbf{a}=na. \\
eat:\textsc{1pl.inc}   mush-\textsc{nom}=this    \\
  \glt ‘We all eat this meal mush.’
  \ex
  \gll Ŋƙ{\Í}tá   tɔbɔŋ-\textbf{a}=na. \\
eat:\textsc{2pl}   mush-\textsc{nom}=this    \\
  \glt ‘You all eat this meal mush.’
  \ex
  \gll Ŋƙáta   tɔbɔŋ\'{ɔ}-á=na. \\
  eat:\textsc{3pl}   mush-\textsc{acc}=this    \\
  \glt ‘They eat this meal mush.’

  \z
\z







\subsection{Accusative case (\textsc{acc})}\label{sec:7.3}


The \textsc{accusative} case, marked by the suffix \{-ka\}, is also split with regard to its basic function. One of its basic functions, that for which it is named, is to mark the direct object of any clause with a 3-person subject ((s)he/it, they). Its other common function is to mark the subject \textit{and} any object of several kinds of subordinate clauses (including relative and temporal clauses). Each of these functions is exemplified by one of the sentences in examples \REF{ex:case:12}-\REF{ex:case:15}. In the first example, a sentence with a 1-person subject is also given to show the contrast:\\




Direct object of a clause with a 3-person subject
\ea\label{ex:case:12}
\gll Wetésátà   m\`{ɛ}s\`{ɛ}-\textbf{à}   mùɲ. \\
drink:\textsc{fut:3pl}   beer-\textsc{acc}   all    \\
\glt ‘They will drink all the beer.’ 
\z




\ea\label{ex:case:13}
\gll Wetésímà     m\`{ɛ}s-à     mùɲ. \\
drink:\textsc{fut:1pl.exc}   beer-\textsc{nom}   all    \\
\glt ‘We will drink all the beer.’ 
\z





Subject and object of a \isi{subordinate clause}
\ea\label{ex:case:14}
\gll Mee   k\'{ɔ}r\'{ɔ}ɓ\'{a}di=[náa   ɲci-\textbf{a}   detí.]   \\
give:\textsc{imp}   thing:\textsc{obl}=that\textsc{}  I-\textsc{acc}   bring:\textsc{1sg}   \\   
\glt ‘Give me the thing that I brought earlier.’ 
\z




\ea\label{ex:case:15}
\gll [Noo   ŋgó-\textbf{á}     b\'{ɛ}ɗ{\Í}mɛɛ     bi-\textbf{a}], {\dots} \\
when   we-\textsc{acc}   want:\textsc{1pl.exc}   you-\textsc{acc}    \\
\glt ‘When we were looking for you, {\dots}’ 
\z






\subsection{Dative (\textsc{dat})}\label{sec:7.4}


The \textsc{dative} case, marked by the suffix \{-ke\}, is the ‘to’ or ‘in’ case, whose role is to mark indirect objects (also called ‘extended’ or ‘secondary’ objects). These indirect objects may encode semantic notions like destination, location, \isi{recipient}, experiencer, \isi{possession}, and purpose. These are illustrated in examples \REF{ex:case:16}-\REF{ex:case:21}:\\




Destination
\ea\label{ex:case:16}
\gll Ƙeesíá   awá-\textbf{k\ᵉ}. \\
go:\textsc{fut:1sg}   home-\textsc{dat}    \\
\glt ‘I’m going home.’ 
\z




Location
\ea\label{ex:case:17}
\gll Ia     sédà-\textbf{k\ᵉ}. \\
be:\textsc{3sg}   garden-\textsc{dat}    \\
\glt ‘She’s in the garden.’ 
\z


Recipient
\ea\label{ex:case:18}
\gll Tɔkɔráta   kabasáá   ròɓà-\textbf{k\ᵉ}. \\
divide:\textsc{3pl}   flour:\textsc{acc}   people-\textsc{dat}    \\
\glt ‘They are dividing out flour to people.’ 
\z




Experiencer
\ea\label{ex:case:19}
\gll Ɨɓálá     ɲcì-\textbf{è}   zùkᵘ. \\
appall:\textsc{3sg}   I-\textsc{dat}   very    \\
\glt ‘It really appalls me.’ (Lit: ‘It is very appalling to me.’) 
\z




Possession
\ea\label{ex:case:20}
\gll Ia     ɦyɔa     ntsí-\textbf{k\ᵉ}. \\
be:\textsc{3sg}   cattle:\textsc{nom}    he-\textsc{dat}    \\
\glt ‘He has cattle.’ (Lit: ‘There are cattle to him.’) 
\z



\newpage 

Purpose
\ea\label{ex:case:21}
\gll Ƙaa     ɲera     dakúáƙ\`{ɔ}-\textbf{k\ᵋ}. \\
go:\textsc{3sg}   girls:\textsc{nom}   wood:inside-\textsc{dat}    \\
\glt ‘The girls go for firewood.’ 
\z






\subsection{Genitive (\textsc{gen})}\label{sec:7.5}


The \textsc{genitive} case, marked by the suffix \{-e\}, is the ‘of’ case, whose role is to encode a possessive or associative relationship a noun has with another noun (or, in rare cases, with a verb). Within the broad notions of \isi{possession} and association are finer nuances such as: ownership, part-whole relationship, kinship, and attribution. These nuances are illustrated in examples \REF{ex:case:22}-\REF{ex:case:25}:\\




Ownership
\ea\label{ex:case:22}
\gll H\'{ɔ}nɨnɨ   ɦyɔa     ńtí-\textbf{e}     ɓórékᵉ. \\
drive:\textsc{seq}   cattle:\textsc{acc}   they-\textsc{gen}   corral:\textsc{dat}    \\
\glt ‘And they drove their cattle to the corral.’ 
\z




Part-whole relationship
\ea\label{ex:case:23}
\gll Wasá     dɛɛdɛɛ   kwará-\textbf{\ᵉ}. \\
stand:\textsc{3sg}   foot:\textsc{dat}   mountain-\textsc{gen}    \\
\glt ‘He’s standing at the foot of the mountain.’ 
\z




Kinship
\ea\label{ex:case:24}
\gll M{\Í}ná     cekíá     ntsí-\textbf{é}     zùkᵘ. \\
love:\textsc{3sg}   wife:\textsc{acc}   he-\textsc{gen}   very    \\
\glt ‘He loves his wife very much.’ 
\z




Attribution
\ea\label{ex:case:25}
\gll Maráŋá   muceá   bì-\textsuperscript{Ø}. \\
good:\textsc{3sg}   way:\textsc{nom}   you-\textsc{gen}    \\
\glt ‘Your luck is good.’ (lit: Your way is good.) 
\z


The \isi{genitive case} has two further roles. One is the \textsc{nominalization} of clauses, that is, the process by which a whole clause is changed into a \isi{noun phrase} that can be used as a subject or object in another clause. For example, the clause \textit{Cɛɨƙɔta náa eakwa ídèmèk\ᵃ} ‘The man killed the snake’ can be compressed into the nominalized \textit{cɛ\'{ɛ}s\'{ʉ}ƙɔta eakwéé ídèmè} ‘the killing of the man of the snake’ or ‘the man’s killing of the snake’. The other secondary role of the genitive has to do with verb \textit{ƙámón} ‘to be like’. For unknown historical reasons, this particular verb requires \isi{genitive case} marking on its \isi{complement}, as in \textit{Ƙámá ròɓèè mùɲ} ‘He’s like all people’, where \textit{ròɓè-è} is analyzed as ‘people-\textsc{gen’} or `of people'.





\subsection{Ablative (\textsc{abl})}\label{sec:7.6}


The \textsc{ablative} case, marked by the suffix \{-o\}, is the ‘from’ case (or in some situations `at' or `in'), whose function is to mark objects with the following semantic roles: origin/source, cause, stimulus, source of judgment, location of activity (versus static location, which is covered by the \isi{dative} case). Each of these semantic roles of the ablative are illustrated among example sentences \REF{ex:case:26}-\REF{ex:case:30}:\\



Origin/source
\ea\label{ex:case:26}
\gll Atsía     awá-\textbf{\ᵒ}. \\
come:\textsc{1sg}   home-\textsc{abl}    \\
\glt ‘I come from home.’ 
\z




Cause
\ea\label{ex:case:27}
\gll Baduƙota=noo   ɲ\'{ɛ}ƙ\`{ɛ}{-}\textbf{\ᵓ}. \\
die:\textsc{3sg}=\textsc{pst}     hunger-\textsc{abl}    \\
\glt ‘He died from hunger.’ 
\z




Stimulus
\ea\label{ex:case:28}
\gll Xɛɓa     ɲérà-\textbf{\ᵒ}. \\
fear:\textsc{3sg}   girls-\textsc{abl}    \\
\glt ‘He’s shy of girls.’ 
\z




Source of judgment
\ea\label{ex:case:29}
\gll Daa     \'{ɲ}cù-\textsuperscript{Ø}. \\
nice:\textsc{3sg}   I-\textsc{abl}    \\
\glt ‘It’s nice to me.’ 
\z




Location of activity
\ea\label{ex:case:30}
\gll Cɛmáta   sédìkà-\textbf{\ᵒ}. \\
fight:\textsc{3pl}   gardens-\textsc{abl}    \\
\glt ‘They are fighting in the gardens.’ 
\z






\subsection{Instrumental (\textsc{ins})}\label{sec:7.7}


The \textsc{instrumental} case, marked by the suffix \{-o\}, is the ‘by’ or ‘with’ case. Unlike the ablative suffix \{-o\}, the instrumental suffix is subtractive, meaning that it first deletes the noun’s final vowel. The function of the \isi{instrumental case} is to mark secondary objects with such semantic roles as instrument/means, pathway, accompaniment, manner, time, and occupation. Each of these nuances are illustrated by one sentence each in example sentences \REF{ex:case:31}-\REF{ex:case:36}:\\




Instrument/means
\ea\label{ex:case:31}
\gll Toɓíá=noo     gasoa       ɓɨs-\textbf{\ᵓ}. \\
spear:\textsc{1sg}=\textsc{pst}   warthog:\textsc{nom}   spear-\textsc{ins}    \\
\glt ‘I speared a warthog with a spear.’ 
\z




Pathway
\ea\label{ex:case:32}
\gll Ƙaini     fots-\textbf{o}     gígìròkᵉ. \\
go:\textsc{3pl}   ravine-\textsc{ins}   downside:\textsc{dat}    \\
\glt ‘And they went down by way of the ravine.’ 
\z



Accompaniment
\ea\label{ex:case:33}
\gll Atsímá=naa     kúrúɓád-\textbf{o}   ŋgóᵉ. \\
come:\textsc{1pl}=\textsc{pst}   things-\textsc{ins}   we:\textsc{gen}    \\
\glt ‘We came with our things.’ 
\z




Manner
\ea\label{ex:case:34}
\gll Ráʝétuo   ɲcie   gáánàs-\textbf{\ᵓ}. \\
answer:\textsc{3sg}   I:\textsc{dat}   badness-\textsc{ins}    \\
\glt ‘And he answered me with hostility.’ 
\z




Time
\ea\label{ex:case:35}
\gll Bɨraa     ɲɛƙa     ódoicik-\textbf{ó}=ni. \\
lack:\textsc{3sg}   hunger:\textsc{nom}   days-\textsc{ins}=these    \\
\glt ‘There is no hunger these days.’ 
\z




Occupation
\ea\label{ex:case:36}
\gll Cɛma     fítés-\textbf{o}   ƙwázìkàᵉ. \\
fight:\textsc{3sg}   washing-\textsc{ins}   clothes:\textsc{gen}    \\
\glt ‘She’s washing clothes.’ (lit: ‘She is fighting with the washing of clothes.’) 
\z






\subsection{Copulative (\textsc{cop})}\label{sec:7.8}


The \textsc{copulative} case, marked by the suffix \{-ko\}, is the ‘is’ or ‘coupling’ case, whose function is to link one noun to another in a relationship of exact identity. In this function, the copulative marks three kinds of nouns: 1) a focused (fronted) noun, 2) the \isi{complement} of a verbless \textsc{copula} (linking verb) clause, and 3) the \isi{complement} of a \isi{negative copula} of identity clause. These different uses of the copulative are illustrated in examples sentences \REF{ex:case:37}-\REF{ex:case:41}:\\





Fronted subject


\ea\label{ex:case:37}
\gll Ŋ{gó-}\textbf{ó}=naa   wetím. \\
 we-\textsc{cop}=\textsc{pst}   drink:\textsc{1pl.exc}   \\
\glt ‘It was we (who) drank (it).’ 
\z




Fronted object
\ea\label{ex:case:38}
\gll Emó-\textbf{ó}     b\'{ɛ}ɗ{\Í}. \\
meat-\textsc{cop}   want:\textsc{1sg}    \\
\glt ‘It is meat (that) I want.’ 
\z




Fronted secondary object
\ea\label{ex:case:39}
\gll Ɲɛƙɔ{-}\textbf{ɔ}     ƙaiátèè   ƙàƙààƙ\`{ɔ}k\ᵋ. \\
hunger-\textsc{cop}   go:\textsc{plur:3pl}   hunt:inside:\textsc{dat}    \\
\glt ‘It is (due to) hunger (that) they keep going hunting.’ 
\z





Verbless \isi{copula} \isi{complement}
\ea\label{ex:case:40}
  \ea
  \gll Ìsù-\textbf{k\ᵒ}?   Ámó-\textbf{o}   keɗe {\dots}?\\
what-\textsc{cop}   person-\textsc{cop}   or     \\   
  \glt ‘What is it? A person or {\dots}?'
  \ex
  \gll Ámá-\textbf{k\ᵒ}. \\
person-\textsc{cop}    \\
  \glt `It’s a person.’
  \z
\z




Negative \isi{copula} \isi{complement}
\ea\label{ex:case:41}
\gll Bɛna=náá     \'{ɲ}cù-\textbf{k\ᵒ}. \\
not.be:\textsc{3sg}=\textsc{pst}   I-\textsc{cop}    \\
\glt ‘It was not me!’ 
\z






\subsection{Oblique (\textsc{obl})}\label{sec:7.9}


The \textsc{oblique} case, marked by the absence of any suffix, is the ‘leftover’ case. As such, it is employed to mark nouns in a variety of disparate grammatical roles and functions. Among these are the following: 1) The subject and/or object of an \isi{imperative} clause, 2) the subject and/or object of an optative clause, 3) the object of a \isi{preposition}, and 4) a vocative noun (used when calling someone). Each of these uses of the \isi{oblique case} are demonstrated in examples \REF{ex:case:42}-\REF{ex:case:46}:\\




Subject and/or object of an \isi{imperative} clause
\ea\label{ex:case:42}
\gll Deté     bi     cue=dííǃ \\
bring:\textsc{imp}   you:\textsc{obl}   water:\textsc{obl}=those    \\
\glt ‘You bring that water!’ 
\z

Subject and/or object of an optative clause
\ea\label{ex:case:43}
\gll \'{Ɲ}ci   nesíbine     emuti     ntsí. \\
I:\textsc{obl}   listen:\textsc{1sg:opt}   story:\textsc{obl}   he:\textsc{gen}    \\
\glt ‘Let me listen to her story.’ 
\z




Object of a \isi{preposition}
\ea\label{ex:case:44}
\gll Túbia     ima     \'{ɲ}cia   páka   awᵃ. \\
follow:\textsc{3sg}   child:\textsc{nom}   I:\textsc{acc}   until   home:\textsc{obl}    \\
\glt ‘The child follows me up to home.’ 
\z




\ea\label{ex:case:45}
\gll Kirotánía  kóteré   ɦyekesí   bì. \\
sweat:\textsc{1sg}   for     life:\textsc{obl}   you:\textsc{gen}    \\
\glt ‘I sweat for your survival.’ 
\z





Vocative
\ea\label{ex:case:46}
\gll Éé   wice,     atsúǃ \\
hey   children:\textsc{obl}  come:\textsc{imp}    \\
\glt ‘Hey children, come!’ 
\z




