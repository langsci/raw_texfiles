\section{Phonology}\label{sec:2}
 
\subsection{Consonants and vowels}\label{sec:2.1}


Ik has an array of thirty consonants and nine vowels, which are presented in \tabref{tab:phon:sounds}. In the table’s first column are shown the alphabetical letters used to represent these sounds. The second column shows the phonetic symbol for the sound used by the International Phonetic Alphabet (IPA). Then in the third column, an approximate English equivalent is given in bold typeface, or else an explanation of how the sound is made if there is no English approximation.


\begin{table}
\caption{Ik sound inventory}
\label{tab:phon:sounds}
\small
\begin{tabularx}{\textwidth}{llX}
\lsptoprule
Alphabetic & Phonetic & English equivalent\\
\midrule
A a & [a] & as in ‘f\textbf{a}ther’\\
B b & [b] & as in ‘\textbf{b}oy’\\
Ɓ  ɓ & [ɓ] & as an English \textbf{b} but with air sucked in\\
C c & [tʃ] & as in ‘\textbf{ch}ild’\\
D d & [d̻] & as in ‘\textbf{d}aughter’\\
Ɗ ɗ & [ɗ] & as an English \textbf{d} but with air sucked in\\
Dz dz & [ʣ̻] & as in ‘a\textbf{dz}e’\\
E e & [e] & as in ‘b\textbf{ai}t’ with a shorter, crisper sound\\
Ɛ ɛ & [ɛ] & as in ‘b\textbf{e}t’\\
F f & [f] & as in ‘\textbf{f}ood’\\
G g & [ɡ] & as in ‘\textbf{g}ood’\\
H h & [h] & as in ‘\textbf{h}appy’\\
Hy ɦy & [ɦʲ] & as an English \textbf{h} but with a raspy sound\\
I i & [i] & as in ‘b\textbf{ea}t’ with a shorter, crisper sound\\
Ɨ ɨ & [ɪ] & as in ‘b\textbf{i}t’\\
J j & [ʤ] & as in ‘\textbf{j}oy’\\
Jʼ ʝ & [ʄ] & as a \textbf{dy} sound but with air sucked in\\
K k & [k] & as in ‘\textbf{k}arma’\\
Ƙ ƙ & [kʼ] & 1) as an English \textbf{k} with a popping release\\
& [ɠ] & 2) as an English \textbf{g} with air sucked in\\
L l & [l] & as in ‘\textbf{l}ove’\\
M m & [m] & as in ‘\textbf{m}an’\\
N n & [n̻] & as in ‘\textbf{n}ature’\\
Ɲ ɲ & [ɲ] & as in ‘o\textbf{ni}on’\\
Ŋ ŋ & [ŋ] & as in ‘si\textbf{ng}’\\
O o & [o] & as in ‘b\textbf{oa}t’ with a shorter, crisper sound\\
Ɔ ɔ & [ɔ] & as in ‘b\textbf{ough}t’\\
P p & [p] & as in ‘\textbf{p}lay’\\
R r & [ɾ] & 1) as a Spanish or Swahili flapped \textbf{r}\\
& [r] & 2) as a Spanish or Swahili trilled \textbf{r}\\
S s & [s] & as in ‘\textbf{s}orrow’\\
Ts ts & [ʦ] & as in ‘bli\textbf{tz}’\\
Tsʼ tsʼ & [ʦʼ] & as an English \textbf{ts}/\textbf{tz} with a hissing release\\
T t & [t̻] & as in ‘\textbf{t}error’\\
U u & [u] & as in ‘b\textbf{oo}t’\\
Ʉ ʉ & [ʊ] & as in ‘p\textbf{u}t’\\
W w & [w] & as in ‘\textbf{w}onder’\\
X x & [ʃ] & as in ‘\textbf{sh}oulder’\\
Y y & [j] & as in ‘\textbf{y}es’\\
Z z & [z] & as in ‘\textbf{z}ebra’\\
Ʒ ʒ & [ʒ] & as in ‘plea\textbf{s}ure’\\
\lspbottomrule
\end{tabularx}

\end{table}

Those sounds in \tabref{tab:phon:sounds} that have a small square under the IPA symbol are pronounced with the tip of the tongue a bit farther forward than in English. Especially [d̻], [n̻], and [t̻] are affected; sometimes they are fronted so much that they touch the back of the front teeth. It is important not to pronounce [d̻] exactly like an English ‘d’ as this sounds more like the Ik sound [ɗ] which contrasts with [d̻]. The sounds [ɓ, ɗ, ɠ, ʝ] are called \textsc{implosives} because they are made by ‘imploding’ or sucking air into the mouth rather than expelling air from the lungs. The sounds [kʼ] and [tsʼ] are called \textsc{ejectives} because they are made by ejecting air from the throat cavity instead of from the lungs. Lastly, the sound [ɦʲ], unlike an [h], is made with the vocal chords vibrating, giving it a raspy, throaty sound. It only occurs at the beginning of words. The nine Ik vowels – [a, e, ɛ, i, ɨ, ɔ, o, ʉ, u] – operate in a \isi{vowel harmony} system, which is discussed in \sectref{sec:2.5}.
 
\subsection{Consonant devoicing}\label{sec:2.2}


At the end of an Ik word, if silence immediately follows, voiced consonants are \isi{devoiced}. In other words, they sound more like unvoiced consonants in that environment. This is similar to German, for instance, where the word \textit{Tag} ‘day’ is pronounced as [tak]. Consonant \isi{devoicing} most noticeably affects /d/ and /g/ in Ik, as when \textit{êd} ‘name’ sounds like [êt] or when \textit{h\`{ɛ}g} ‘marrow’ sounds like [h\`{ɛ}k]. 
 
\subsection{Vowel devoicing}\label{sec:2.3}


Ik vowels are also \isi{devoiced} before a pause. This is important because every word in every grammatical context – without exception – ends in a vowel. If that final vowel is not immediately followed by another sound, then it is whispered or even left totally inaudible (for example, after the consonants /f, m, n, ɲ, ŋ, r, s, z, ʒ/). It has become a tradition in scholarly writing on Ik to write whispered vowels with the following raised (superscript) symbols: <\ⁱ,\ᶤ,\ᵉ,\ᵋ,\ᵃ,\ᵓ,\ᵒ,\ᶶ,\ᵘ>.

\subsection{Morphophonology}\label{sec:2.4}
\subsubsection{Deaffrication}\label{sec:2.4.1}

The affricates /c/ and /j/ are occasionally deaffricated or ‘hardened’ into their non-affricate counterparts /k/ and /g/, respectively. This is not a general phonological tendency in the language but is, rather, limited to a small handful of words. Moreover, the principle is applied in different ways to different words. For instance, in the word \textit{muceé-} ‘path, way’, the /c/ is hardened to /k/ when the word is used in the \isi{instrumental case} (see \sectref{sec:7.7}): \textit{muko} ‘on the way’. Secondly, as an instance of idiolectal variation, the plural inclusive pronoun \textit{ɲjíní-} ‘we all (including addressees)’ is pronounced idiosyncratically as \textit{ŋgíní-} by a minority of speakers. Thirdly, when the words \textit{Icé-} ‘Ik people’ and \textit{wicé-} ‘children’ are declined for the nominative or instrumental cases, their /c/ hardens to /k/. This type of deaffrication can be clearly seen in a case declension, like the one in \tabref{tab:phon:wice}. Note that, as explained later in \sectref{sec:2.4.3}, all cases have non-final and final forms:


\begin{table}
\caption{Case declension of \textit{Icé-} ‘Ik’ and \textit{wicé-} ‘children’}
\label{tab:phon:wice}


\begin{tabularx}{\textwidth}{XXXXX}
\lsptoprule

& \multicolumn{2}{X}{‘Ik’} & \multicolumn{2}{X}{‘children’}\\
\midrule
& Non-final & Final & Non-final & Final\\
\midrule
Nominative & Ika & Ik\ᵃ & wika & wik\ᵃ\\
Accusative & Icéá & Icék\ᵃ & wicéá & wicék\ᵃ\\
Dative & Icéé & Icékᵉ & wicéé & wicékᵉ\\
Genitive & Icéé & Icé & wicéé & wicé\\
Ablative & Icóó & Icéᵒ & wicóó & wicéᵒ\\
Instrumental & Ico/Iko & Icᵒ/Ikᵒ & wico/wiko & wicᵒ/wikᵒ\\
Copulative & Icóó & Icékᵒ & wicóó & wicékᵒ\\
Oblique & Ice & Ice & wice & wice/wicᵉ\\
\lspbottomrule
\end{tabularx}
\end{table}


\subsubsection{Haplology}\label{sec:2.4.2}

In Ik, when a consonant in one morpheme is made at the same place of articulation as a consonant in the next morpheme, \textsc{haplology} may occur – the deletion of the first of the two similar consonants. One example of this involves the venitive suffix \{-ét-\} and the an\isi{dative} suffix \{-uƙot-\}, both of which end in /t/. If another suffix containing /t/, /d/, or /s/ is attached to either of these, their final /t/ may be omitted. To illustrate this, \tabref{tab:phon:haplog} presents a conjugation of the verb \textit{ŋat\'{ɛ}t\'{ɔ}n} ‘to run this way’. Notice how the /t/ in \{-ét-\} disappears from the suffix in the forms for \textsc{2sg} (‘you’), \textsc{1pl.inc} (‘we all’), and \textsc{2pl} (‘you all’). The 3\textsc{pl} form (‘they’) is an exception as it does not drop its final /t/ in the same environment.


\begin{table}
\caption{Haplology in \textit{ŋat\'{ɛ}t\'{ɔ}n} ‘to run this way’}
\label{tab:phon:haplog}


\begin{tabularx}{\textwidth}{XXXXl}
\lsptoprule

\textsc{1sg} & ŋat-ɛt-{\Í} &  & ŋat-ɛt-{\Í} & ‘I run this way.’\\
\textsc{2sg} & ŋat-ɛt-{\Î}d & → & ŋat-\'{ɛ}{}-{\Î}d & ‘You run this way.’\\
\textsc{3sg} & ŋat-ɛt &  & ŋat-ɛt & ‘(S)he/it runs this way.’\\
\textsc{1pl.exc} & ŋat-ɛt-{\Í}m &  & ŋat-ɛt-{\Í}m & ‘We run this way.’\\
\textsc{1pl.inc} & ŋat-ɛt-{\Í}s{\Í}n & → & ŋat-ɛ{}-{\Í}s{\Í}n & ‘We all run this way.’\\
\textsc{2pl} & ŋat-\'{ɛ}t-{\Í}t & → & ŋat-\'{ɛ}{}-{\Í}t & ‘You all run this way.’\\
\textsc{3pl} & ŋat-ɛt-át &  & ŋat-ɛt-át & ‘They run this way.’\\
\lspbottomrule
\end{tabularx}
\end{table}

A second example of \isi{haplology} occurs when a verb root ending in /g/, /k/, or /ƙ/ is followed directly by the an\isi{dative} suffix \{-uƙot-\}. When this happens, the final velar consonant of the verb root gets omitted in anticipation of the velar /ƙ/ in \{-uƙot-\}. \tabref{tab:phon:haplology} illustrates this by listing a few verbs ending in /g/, /k/, or /ƙ/, which disappear when the next morpheme is the an\isi{dative} suffix \{-uƙot-\}.


\begin{table}[t]
\caption{Haplology in verbs ending in a velar consonant}
\label{tab:phon:haplology}


\begin{tabularx}{\textwidth}{XXXX}
\lsptoprule

ɦyɔt\'{ɔ}g-ʉƙɔt- & → & ɦyɔt\'{ɔ}-ɔƙɔt- & ‘go near’\\
iɓók-uƙot- & → & iɓó-óƙot- & ‘shake off’\\
ɨpák-ʉƙɔt- & → & ɨpá-áƙot- & ‘swipe off’\\
kɔk-ʉƙɔt- & → & kɔ-ɔƙɔt- & ‘close up’\\
ŋƙáƙ-uƙot- & → & ŋƙá-áƙot- & ‘eat up’\\
oƙ-uƙot- & → & o-oƙot- & ‘put aside’\\
torík-uƙot- & → & torí-íƙot- & ‘lead away’\\
\lspbottomrule
\end{tabularx}
\end{table}

\subsubsection{Non-final consonant deletion}\label{sec:2.4.3}
 
Ik makes a clear distinction between \textsc{non-final} and \textsc{final} forms of all morphemes and words. Presumably this is to delineate syntactic boundaries, often with stylistic overtones. Non-final forms are those that occur within a string of speech, with at least one element immediately following them. Final forms, by contrast, are those that occur at the end of a string of speech, before a pause, with nothing immediately following. This basic distinction was already shown to affect the voicing of vowels in \sectref{sec:2.3}. In the case of a small number of morphemes, it also affects consonants. \tabref{tab:phon:consdel} presents a few of these morphemes whose final forms contain consonants that are omitted in their non-final forms. The first column of the table shows the underlying form (\textsc{uf}) of the morpheme in question. This is followed in the next two columns by the non-final (\textsc{nf}) and final (\textsc{ff}) forms that actually occur in speech. Notice how the non-final forms are missing one consonant that is fully present in the \textsc{uf} and the \textsc{ff}.


\begin{table}
\caption{Consonant deletion in non-final forms}
\label{tab:phon:consdel}


\begin{tabularx}{\textwidth}{XXXl}
\lsptoprule

\textsc{uf} & \textsc{nf} & \textsc{ff} & Morpheme description\\
\midrule
{}-ka & {}-a & {}-k\ᵃ & \isi{accusative case} suffix\\
{}-ke & {}-e & {}-kᵉ & \isi{dative} case suffix\\
{}-ko & {}-o & {}-kᵒ & \isi{copulative case} suffix\\
{}-\'{} ka & {}-\'{} a & {}-\'{} k\ᵃ & present perfect suffix\\
{}-\'{} de & {}-\'{} e & {}-\'{} dᵉ & \isi{dummy pronoun} suffix\\
nákà & náà & nák\ᵃ & ‘earlier today’\\
bàtsè & bèè & bàtsᵉ & ‘yesterday’\\
nòkò & nòò & nòkᵒ & ‘long ago’\\
ʝ{\Ì}k\`{ɛ} & ʝ{\Ì}{\Ì} & ʝ{\Ì}k\ᵋ & ‘also, too’\\
ɲákà & ɲáà & ɲák\ᵃ & ‘just’\\
\lspbottomrule
\end{tabularx}
\end{table}

\subsubsection{Vowel assimilation}\label{sec:2.4.4}

In addition to consonants, Ik vowels also undergo phonological changes at morpheme boundaries. For instance, when two dissimilar vowels come in contact with each other as a result of two morphemes joining together, there is a powerful urge for them to become more like each other. This \textsc{vowel assimilation} was already seen at work in \tabref{tab:phon:haplology}, as when putting the root \textit{torík-} ‘lead’ and affix \textit{{}-uƙot-} ‘away’ together led to \textit{tor}\textit{íí}\textit{ƙot-} instead of *\textit{tor}\textit{íú}\textit{ƙot-}. It is also seen in \tabref{tab:phon:vowelassim} where the ‘yester-’ \isi{adverb} \textit{bàtsè} becomes \textit{b}\textit{èè} in its non-final form instead of *\textit{b}\textit{àè}. Ik \isi{vowel assimilation} only takes place between morphemes and not inside morphemes. Inside morphemes, many combinations of dissimilar vowels are allowed, for example in \textit{kaɨn} ‘year’, \textit{m\`{ɛ}\`{ʉ}r} ‘drongo’, and \textit{kɔ{\Í}n} ‘scent’. 

Ik \isi{vowel assimilation} can be clearly seen throughout the lexicon, as when the transitive \isi{infinitive} suffix \{-és\} and the \isi{intransitive} \isi{infinitive} suffix \{-òn\} are affixed to verb roots. If the verb root that these suffixes attach to ends in /a/ or /e/, the vowel of the suffix fully assimilates it. \tabref{tab:phon:vowelassim} offers a few examples of this kind of \isi{vowel assimilation} in verbal infinitives.


\begin{table}
\caption{Vowel assimilation in verbal infinitives}
\label{tab:phon:vowelassim}


\begin{tabularx}{\textwidth}{XXXX}
\lsptoprule

Transitive &  &  & \\
\midrule
fá-és & → & féés & ‘to boil’\\
ɨsá-és & → & ɨsɛɛs & ‘to miss’\\
ɨt{\Í}ŋá-és & → & ɨt{\Í}ŋ\'{ɛ}\'{ɛ}s & ‘to force’\\
tamá-és & → & tamɛɛs & ‘to extol’\\
wa-és & → & weés & ‘to harvest’\\
% \midrule %better use more vertical space instead of extra midrules
\\
Intransitive &  &  & \\
\midrule
ƙà-òn & → & ƙòòn & ‘to go’\\
ŋká-ón & → & ŋkóón & ‘to stand up’\\
tsá-ón & → & tsóón & ‘to be dry’\\
tsè-òn & → & tsòòn & ‘to dawn’\\
zè-òn & → & zòòn & ‘to be big’\\
\lspbottomrule
\end{tabularx}
\end{table}

Another environment illustrating Ik \isi{vowel assimilation} is the case declension of nouns. Since all Ik nouns end in a vowel, and since seven of the eight case suffixes consist of or contain a vowel, case suffixation creates a fertile ground for \isi{vowel assimilation}. For example, as \tabref{tab:phon:vowelassimdog} illustrates, in the declension of the noun root \textit{ŋókí-} ‘dog’, the /o/ in the \isi{ablative case} suffix \{-o\} and the \isi{copulative case} suffix \{-ko\} partially assimilate the final /i/ of \textit{ŋókí-} to /u/.


\begin{table}[t]
\caption{Vowel assimilation in the declension of \textit{ŋókí-} ‘dog’}
\label{tab:phon:vowelassimdog}


\begin{tabularx}{\textwidth}{XXX}
\lsptoprule

Case & \textsc{nf} & \textsc{ff}\\
\midrule
Nominative & ŋók-á & ŋók-\ᵃ\\
Accusative & ŋókí-à & ŋókí-k\ᵃ\\
Dative & ŋókí-è & ŋókí-kᵉ\\
Genitive & ŋókí-è & ŋókí-\textsuperscript{Ø}\\
Ablative & ŋókú-ò & ŋókú-\textsuperscript{Ø}\\
Instrumental & ŋók-ó & ŋók-ᵒ\\
Copulative & ŋókú-ò & ŋókú-kᵒ\\
Oblique & ŋókí & ŋókⁱ\\
\lspbottomrule
\end{tabularx}
\end{table}

 
Other \isi{vowel assimilation} effects are shown in the case declension of a noun like \textit{ŋʉrá-} ‘cane rat’, as in \tabref{tab:phon:vowelassimrat}, where the final /a/ of \textit{ŋʉrá-} is assimilated by the \isi{dative}, genitive, ablative, and \isi{copulative case} suffixes in their non-final forms.


\begin{table}
\caption{Vowel assimilation in the declension of \textit{ŋʉrá-} ‘cane rat’}
\label{tab:phon:vowelassimrat}


\begin{tabularx}{\textwidth}{XXX}
\lsptoprule

Case & \textsc{nf} & \textsc{ff}\\
\midrule
Nominative & ŋʉr-a & ŋʉr-\textsuperscript{Ø}\\
Accusative & ŋʉrá-á & ŋʉrá-k\ᵃ\\
Dative & ŋʉr\'{ɛ}-\'{ɛ} & ŋʉrá-k\ᵋ\\
Genitive & ŋʉr\'{ɛ}-\'{ɛ} & ŋʉrá-\ᵋ\\
Ablative & ŋʉr\'{ɔ}-\'{ɔ} & ŋʉrá-ᵓ\\
Instrumental & ŋʉr-ɔ & ŋʉr-ᵓ\\
Copulative & ŋʉr\'{ɔ}-\'{ɔ} & ŋʉrá-kᵓ\\
Oblique & ŋʉra & ŋʉr\\
\lspbottomrule
\end{tabularx}
\end{table}

\newpage
Ik \isi{vowel assimilation} may be \textsc{partial}, as when the word \textit{ŋókí-k\ᵒ} ‘It is a dog’ is rendered as \textit{ŋókú-k\ᵒ}. There, the /i/ at the end of \textit{ŋókí-} ‘dog’ only moves back in the mouth to become /u/; it does not fully assimilate to become identical to the /o/ in the suffix. But \isi{vowel assimilation} can also be \textsc{total}, as when \textit{ŋʉrá-\'{ɛ}} ‘of the cane rat’ becomes \textit{ŋʉr\'{ɛ}-\'{ɛ}}. In that instance, the /a/ at the end of \textit{ŋʉrá}{}- becomes fully identical to the vowel in the suffix. 

Ik \isi{vowel harmony} can be \textsc{regressive} as in both prior examples, where a vowel exerts pressure on a preceding one. But it can also be \textsc{progressive}, as in the example of \textit{torí-úƙot-} becoming \textit{torí-íƙot}{}-, where the /i/ acts ahead on the /u/.
 
\subsubsection{Vowel desyllabification}\label{sec:2.4.5}

When the back-of-the-mouth vowels /ɔ/, /o/, /ʉ/ or /u/ wind up next to another vowel across a morpheme boundary, they may lose their status as the nucleus of a \isi{syllable} and become the \isi{semi-vowel} /w/ instead. When such vowel \textsc{desyllabification} occurs, the syllabic ‘weight’ of the vowel gets transferred to the following vowel in a process called \textsc{compensatory lengthening}. This phonological change is evident in the transitive infinitives of verbs ending in a back vowel. \tabref{tab:phon:desyllabv} depicts how the back vowel at the end of the verb root changes to /w/ and then lengthens the vowel in the transitive suffix \{-és\}.


\begin{table}
\caption{Vowel desyllabification in verbs}
\label{tab:phon:desyllabv}


\begin{tabularx}{\textwidth}{XXXX}
\lsptoprule

tʉtsʉ-ɛs & → & tʉtswɛɛs & ‘to wring’\\
r\'{ɔ}-\'{ɛ}s & → & rw\'{ɛ}\'{ɛ}s & ‘to string’\\
ho-és & → & hweés & ‘to cut’\\
ó-és & → & wéés & ‘to call’\\
ru-és & → & rweés & ‘to uproot’\\
\lspbottomrule
\end{tabularx}
\end{table}

Vowel desyllabification also takes place in the case declensions of nouns. Any noun root that ends in a back vowel can have that vowel desyllabified to /w/, with the result that the following case suffix is lengthened. As \tabref{tab:phon:desyllabn} demonstrates, this happens with a noun like \textit{dakú-} ‘plant, tree’ which ends with the back vowel /u/. In five of the eight cases – accusative, \isi{dative}, genitive, ablative, copulative – the final /u/ of \textit{dakú-} changes to /w/ and then lengthens the case suffix. Note that in the \isi{nominative case}, the /u/ of \textit{dakú-} is desyllabified but does not lengthen the nominative suffix \{-a\}. This irregularity is a peculiarity of the \isi{nominative case} only and is seen in many other noun declensions.


\begin{table}
\caption{Vowel desyllabification in nouns}
\label{tab:phon:desyllabn}


\begin{tabularx}{\textwidth}{XXXX}
\lsptoprule

Case & \multicolumn{3}{X}{ Non-final}\\
\midrule
Nominative & dakw-a &  & \\
Accusative & dakú-á & → & dakw-áá\\
Dative & dakú-é & → & dakw-éé\\
Genitive & dakú-é & → & dakw-éé\\
Ablative & dakú-ó & → & dakw-óó\\
Instrumental & dak-o &  & \\
Copulative & dakú-ó & → & dakw-óó\\
Oblique & daku &  & \\
\lspbottomrule
\end{tabularx}
\end{table}

\subsection{Vowel harmony}\label{sec:2.5}


Ik vowels participate in a phonological system called \textsc{vowel harmony}. This means that the language’s sound system seeks vocalic ‘harmony’ by ensuring that all vowels in a single word belong to the same vowel class. The vowel classes involved are the following: 1) the [+\isi{ATR}] or ‘heavy’ vowels /i, e, o, u/ that are made with a larger cavity in the throat, giving them a ‘heavier’, more resonant sound, and 2) the [-\isi{ATR}] or ‘light’ vowels /ɨ, ɛ, ɔ, ʉ/ that are made with a smaller cavity in the throat, giving them a ‘lighter’, less resonant sound. Where the ninth vowel /a/ fits in with these two classes is a theoretical question that has not been conclusively resolved. However, what is clear is that in Ik, /a/ sometimes behaves as a [+\isi{ATR}] vowel and other times as a [-\isi{ATR}] vowel. And it certainly is found together with vowels from both classes within a single word. The Ik vowel classes anchored by the low vowel /a/ are depicted in \tabref{tab:phon:vowelclasses}:


\begin{table}
\caption{Ik vowel classes}
\label{tab:phon:vowelclasses}
\begin{tabular}{ccccc}
\lsptoprule
\multicolumn{2}{c}{ \textsc{[+\isi{ATR}]}} &  & \multicolumn{2}{c}{ \textsc{[-\isi{ATR}]}}\\
\midrule
 i & u &  & ɨ & ʉ\\
 e & o &  & ɛ & ɔ\\
&  & a &  & \\
\lspbottomrule
\end{tabular}
\end{table}

Because of \isi{vowel harmony}, all the vowels in a single word will generally belong to one of the vowel classes shown in \tabref{tab:phon:vowelclasses}. This is clearly evident in the lexicon where verbs consisting of multiple syllables and morphemes contain either [+\isi{ATR}] or [-\isi{ATR}] vowels, but not both. \tabref{tab:phon:vowelharmony} shows an opposing set of such verbs. Notice how all the vowels in each word belong to one vowel class.


\begin{table}[p]
\caption{Vowel harmony in the lexicon}
\label{tab:phon:vowelharmony}
\begin{tabularx}{\textwidth}{XX}
\lsptoprule {}
[+\isi{ATR}] & \\
\midrule
béberés & ‘to pull’\\
béberetés & ‘to pull this way’\\
béberésúƙot\ᵃ & ‘to pull that way’\\
% \midrule {}
\\{}
[-\isi{ATR}] & \\
\midrule
b\'{ɛ}ɗ\'{ɛ}s & ‘to want’\\
bɛɗɛt\'{ɛ}s & ‘to look for’\\
b\'{ɛ}ɗ\'{ɛ}sʉƙɔt\ᵃ & ‘to go look for’\\
\lspbottomrule
\end{tabularx}
\end{table}

In some situations though, /a/ blocks \isi{vowel harmony} from spreading to all the morphemes in a word. For example, when the stative suffix \{-án-\} falls between a verb with [-\isi{ATR}] vowels and the \isi{intransitive} suffix \{-òn-\}, the /a/ in \{-án-\} prevents the spread of harmony to the whole word. \tabref{tab:phon:vowelharmA} gives a few examples of the harmony-blocking behavior of /a/. Notice how [-\isi{ATR}] vowels are found to the left of \{-án-\} (in bold), while the [+\isi{ATR}] /o/ in \{-òn-\} comes after it.


\begin{table}[p]
\caption{Vowel harmony blocking behavior of /a/}
\label{tab:phon:vowelharmA}
\begin{tabularx}{\textwidth}{XX}
\lsptoprule
akw\'{ɛ}t\'{ɛ}kw\'{ɛ}t\textbf{án}ón & ‘to writhe around’\\
ɓɛl\'{ɛ}ɓ\'{ɛ}l\textbf{án}ón & ‘to be cracked’\\
g\'{ɔ}l\'{ɔ}gɔl\textbf{án}ón & ‘to be crooked’\\
ɨl\'{ɔ}ɗ{\Í}ŋ\textbf{án}ón & ‘to be discriminatory’\\
ŋ\'{ʉ}zʉm\textbf{án}ón & ‘to bicker’\\
\lspbottomrule
\end{tabularx}
\end{table}
Ik has three suffixes which are said to be \textsc{dominant} in that they always spread their [+\isi{ATR}] value as far as they can within a word. These include the pluractional suffix \{-í-\}, the middle suffix \{-ím-\}, and the \isi{plurative} suffix \{-íkó-\}, all of which contain the vowel /i/. Unless an /a/ blocks the way, these three suffixes will cause all the vowels in the word they are found in to harmonize to [+\isi{ATR}]. This dominant behavior is illustrated in \tabref{tab:phon:dominant}. Notice how the [-\isi{ATR}] vowels in the first column all become [+\isi{ATR}] in the third column as a result of the dominance of the suffixes (in bold typeface).


\begin{table}[p]
\caption{Ik dominant suffixes}
\label{tab:phon:dominant}
\begin{tabularx}{\textwidth}{XXXXl}
\lsptoprule
abʉtɛs & ‘to sip’ & → & abut\textbf{i}és & ‘to sip continuously\\
k\`{ɔ}n\`{ɔ}n & ‘to be one’ & → & kón\textbf{í}ón & ‘to be one-by-one’\\
&  &  &  & \\
ɨlɔɛs & ‘to defeat’ & → & ilo\textbf{im}étòn & ‘to be defeated’\\
kɔk\'{ɛ}s & ‘to close’ & → & kok\textbf{ím}étòn & ‘to close (alone)’\\
&  &  &  & \\
ɔrɔr & ‘stream’ & → & orór\textbf{íkw}\ᵃ & ‘streams’\\
w\`{ɛ}l & ‘opening’ & → & wél\textbf{íkw}\ᵃ & ‘openings’\\
\lspbottomrule
\end{tabularx}
\end{table}

Two other issues surrounding \isi{vowel harmony} deserve mention. First, when two nouns are joined together to form a compound word (\sectref{sec:4.3}), \isi{vowel harmony} does not occur between them. For example, the noun roots \textit{rébè-} ‘millet’ and \textit{m\`{ɛ}s\`{ɛ}-} ‘beer’ can be joined into the compound \textit{rébèm\`{ɛ}s\`{ɛ}-} ‘millet beer’, in which, notice, the vowels belong to two different [\isi{ATR}] vowel classes. An exception to this rule is when the second noun in the compound begins with the vowel /i/, in which case /i/ harmonizes the last vowel of the first noun, as when \textit{ɲ\'{ɔ}kɔkɔrɔ-ímà-} ‘chick’ becomes \textit{ɲ\'{ɔ}kɔkɔró-ímà-} (where the first noun’s /ɔ/ is harmonized to /o/). Second, many of Ik’s clitics take on the [\isi{ATR}] value of their host word, for example when the \isi{anaphoric} pronoun \textit{déé} becomes \textit{d\'{ɛ}\'{ɛ}} in the phrase \textit{mɔƙɔr\'{ɔ}\'{ɛ}=d\'{ɛ}\'{ɛ}} ‘in that rock pool’. Again, the exception is when the \isi{clitic} contains /i/, in which case it becomes dominant, harmonizing its host, as when \textit{bár{\Í}t{\Í}n\'{ʉ}ɔ=díí} ‘from those corrals’ becomes \textit{bár{\Í}t{\Í}núo=díí} (where the vowels /\'{ʉ}ɔ/ become /úo/).
 
\subsection{Tone}\label{sec:2.6}
\subsubsection{Tone inventory}\label{sec:2.6.1}

Ik is a tonal language. In terms of acoustics, this means that every vowel is identified not only by where it is formed in the vocal chamber but also by the \textsc{pitch} with which it is uttered. This further entails that every \isi{syllable}, morpheme, word, and phrase exhibits a specific and indispensable \textsc{tone} pattern. At a phonological (or psychological) level, Ik has just two tones: \textsc{high} (H) and \textsc{low (L)}. All other tones that one hears can be traced back to these two. However, for practical applications like \isi{orthography} and language learning, four sub-tones must be recognized. These include: \textsc{high}, \textsc{high-falling}, \textsc{mid}, and \textsc{low}. High tone is pronounced with a level, relatively high pitch. High-falling tone falls quickly from relatively high to relatively low pitch, often in the presence of a \isi{depressor consonant} (see \sectref{sec:2.6.4}). Mid tone is a level, relatively medium-height pitch, while low tone is either relatively low and flat or tapering off before a pause. \tabref{tab:phon:tones} presents the Ik tones with their names in the first column, pitch profiles in the second, and the orthographic diacritics for writing them in the third (the same diacritics employed throughout the foregoing dictionary sections):


\begin{table}
\caption{Ik tones}
\label{tab:phon:tones}


\begin{tabularx}{\textwidth}{XXX}
\lsptoprule

Tone & Pitch & Symbol\\
\midrule
\textsc{high} & [\raisebox{1.5mm}{--}] & Á á\\
\textsc{high-falling} &  [$\setminus$] & \^{A} â\\
\textsc{mid} & [--] & A a\\
\textsc{low} & [\raisebox{-1.5mm}{--}] & \`{A} à\\
\lspbottomrule
\end{tabularx}
\end{table}


\subsubsection{Lexical tone}\label{sec:2.6.2}

As mentioned above, every word in the Ik lexicon has a tone pattern or ‘melody’. That is, Ik words are not identified solely on the basis of consonants and vowels (as in non-tonal languages like English) but also on their tone pattern, which must be learned. Since every vowel and therefore every \isi{syllable} bears a tone, the combination of many syllables in words produces a large inventory of tone patterns. And since the tone pattern of a word is totally unpredictable, language learners must resort to memorizing the pattern with the word. \tabref{tab:phon:tonepatterns} gives a sample of the lexical tone patterns found on some short words in Ik:


\begin{table}
\caption{Ik lexical tone patterns}
\label{tab:phon:tonepatterns}


\begin{tabularx}{\textwidth}{XXX}
\lsptoprule
Nouns &  & \\
\midrule
HH & ámá- & ‘person’\\
HL & \'{ɛ}bà- & ‘horn’\\
MH & cekí- & ‘woman’\\
LL & ɲèrà- & ‘girls’\\
\\
Verbs &  & \\
\midrule
H & ŋáɲ- & ‘open’\\
H(L) & éd\`{} - & ‘carry on back’\\
L & àts- & ‘come’\\
\lspbottomrule
\end{tabularx}
\end{table}

\subsubsection{Grammatical tone}\label{sec:2.6.3}
\largerpage
Ik does not have grammatical tone, whereby tone alone can carry out a grammatical function. But tone often accompanies other grammatical signals, thereby reinforcing them. Thus, in that regard, it could be said that Ik has ‘semi-gramma\-tical’ tone. For example, when the suffix \{-íkó-\} is used to pluralize a singular noun, the tone over the singular root usually changes, as when \textit{kɔl} ‘ram’ becomes \textit{kólíkw\ᵃ}. Similarly, when the venitive suffix \{-ét-\} is added to a verb stem, it often changes the overall tone pattern, as when \textit{b\'{ɛ}ɗ\'{ɛ}s} ‘to want’ becomes \textit{bɛɗɛt\'{ɛ}s} ‘to look for’, whereby the tone of the root \textit{b\'{ɛ}ɗ-} goes from \textsc{high} to \textsc{mid}. Indeed, many of the suffixes of the language are associated with significant tone changes to the stem. So even if one learns the tonal melodies of nouns and verbs on their own, these melodies may change in particular grammatical contexts. This type of tone changeability is one of the system’s more difficult aspects.

The Ik tone system is challenging for foreigners and is not yet fully understood from an analytical point of view. Still, the good news is that with lots of practice, language learners can reasonably expect to develop a certain degree of communicative competency. For the most complete description of the tone system to date, the reader is invited to consult \sectref{sec:3.2} in \textit{A grammar of Ik (Icé-tód)} \citep{Schrock2014}. That section expands on what has been presented here and includes more detailed discussions of other features of the Ik tone system.
 
\subsubsection{Depressor consonants}\label{sec:2.6.4} 

In Ik, the voiced consonants /b, d, dz, g, ɦy, j, z, ʒ/ plus /h/ act as \textsc{depressor consonants}. Depressor consonants are so-called because they ‘depress’ or pull down the pitch of neighboring vowels. In doing so, they act almost as if they had a very low tone of their own. The effect of Ik depressors is so strong that, over time, it led to the creation of a whole new set of lexical tone patterns. For instance, all Ik verbs with a HL pattern in their roots have a depressor as the first consonant after the initial high tone: \textit{d\'{ɛ}g\`{ɛ}m-} ‘crouch’, \textit{g\'{ʉ}g\`{ʉ}r-} ‘hunched’, \textit{íbòt-} ‘jump’, \textit{kídzìm-} ‘descend’, and \textit{tsʼágwà-} ‘be raw’. This is because, in anticipation of the extra-low pitch of the depressor, the language compensated by putting a high tone before it where there used to be none. As another example, all nouns with the root tone pattern HL have a depressor as the only consonant between two vowels, as in: \textit{d\'{ɔ}bà-} ‘mud’, \textit{\'{ɛ}bà-} ‘horn’, \textit{édì-} ‘name’, \textit{nébù-} ‘body’, and \textit{wídzò-} ‘evening’. And when these types of nouns lose their final vowel due to vowel \isi{devoicing}, that is when the \textsc{high-falling} contour tone comes into play, as in \textit{d\^{ɔ}b\ᵃ} ‘mud’, \textit{\^{ɛ}b\ᵃ} ‘horn’, \textit{êd\ᵃ} ‘name’, \textit{nêb\ᵃ} `body', and \textit{wîdz\ᵃ} ‘evening’.