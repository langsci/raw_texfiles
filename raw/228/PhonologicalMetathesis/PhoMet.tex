\chapter{Phonologically conditioned metathesis and associated processes}\label{ch:PhoMet}

\section{Introduction}\label{sec:Int ch:PhoMet}
In this chapter I discuss the use of M\=/forms (metathesised forms)
before vowel-initial enclitics,
as well as the other phonological process with which it occurs.
Such M\=/forms occur only before vowel-initial enclitics
and it is possible to analyse them as conditioned by the this environment.\footnote{
		The processes described in this chapter only occur before vowel-initial enclitics.
		Comparable environments do not exist in Amarasi.
		Vowel-initial suffixes -- of which there are only two
		(see \srf{sec:PeoGroSuf} and \srf{sec:Suf-aq}) --
		only occur on VVC{\#} final stems
		which remain unchanged before vowel-initial enclitics.}
M\=/forms before an enclitic boundary are glossed {\Mvv}
(M with an equals sign for clitic above it).
Three phonological processes, including metathesis,
are triggered before vowel-initial enclitics.
These processes are summarised in \qf{ex:ProEncBou} below.

\begin{exe}
	\ex{Processes at Enclitic Boundaries}\label{ex:ProEncBou}
	\begin{xlist}
		\ex{Metathesis}
		\ex{Consonant Insertion}
		\ex{Vowel Assimilation}
	\end{xlist}
\end{exe}

In this chapter I present an analysis of these processes.
Metathesis is triggered by \tsc{Crisp-Edge};
the need to keep prosodic words phonologically distinct (\srf{sec:Met ch:PhoMet}).
Consonant insertion is triggered by the need for feet to have an onset consonant (\srf{sec:ConIns}).
Vowel assimilation is triggered by metathesis occurring after consonant insertion (\srf{sec:VowAss ch:PhoMet}).

The behaviour of stems before vowel-initial enclitics
is quite diverse between different varieties of Meto
and it must be emphasised that the discussion here
only describes Kotos Amarasi as spoken in the hamlet of Koro{\Q}oto.
Occasional notes on other varieties of Meto occur at some points as relevant
and more extensive comparative data is presented in \srf{sec:HisDev}.

The Amarasi vowel-initial enclitics, all of which trigger these processes,
are given in \trf{tab2:AmaVowIniEnc}.\footnote{
		In \cite{ed16b} I analysed many of these
		vowel-initial enclitics as containing only a single vowel,
		in line with their usual phonetic realisation.
		Since then, I have collected comparative data
		from other varieties of Meto which is best analysed
		by positing that \emph{all} these enclitics have two underlying vowels.
		This evidence is presented in \srf{sec:HisDev}.}
The enclitics \ve{=ii}, \ve{=ein} and \ve{=ee} have
different uses when attached to verbs than they do with nouns.
The enclitic \ve{=ein} displays some complex allomorphy
and is associated with unusual consonant insertion (\srf{sec:PluEnc}).
The function and syntactic behaviour of most of these enclitics
is discussed in Chapter \ref{ch:SynMet}.

\begin{table}[h]
	\caption{Amarasi vowel-initial enclitics}\label{tab2:AmaVowIniEnc}
	\centering
		\begin{tabular}{lll}\lsptoprule
			Form						&Gloss	& Use\\ \midrule
			\ve{=ii}				&{\ii}	& definite referent near/relevant to speaker\\
											&{\ii}	& raises discourse prominence\\
			\ve{=aan/=ana}	&{\aan}	&	definite referent near/relevant to addressee\\
			\ve{=ee}				&{\ee}	& definite referent near/relevant to a third person\\
											&{\eeV}	&	third person P argument (object) of verb\\
			\ve{=aa}				&{\aa}	&	definite referent near/relevant to no one (≈obviative)\\
			\ve{=ein/=eni}	&{\ein}	&	definite plural\\
											&{\einV}&	third person plural verbal argument (S/A/P)\\
			\ve{=ees/=esa}	&{\es}	&	the numeral one (1); indefinite singular\\
			\ve{=een/=ena}	&{\een}	&	inceptive, beginning of state/event \\
			\ve{=aah/=aha}	&just		&	restrictive\\
			\ve{=oo-n}				&{\oo}	& reflexive	(takes genitive suffix \srf{sec:GenSuf}) \\\lspbottomrule
		\end{tabular}
\end{table}

In addition to these vowel-initial enclitics,
the vowel-initial forms of the sentence enclitics \ve{=ma/=ama} `and'
and \ve{=te/=ate} {\te} occasionally, though not obligatorily,
trigger metathesis on their host. The vowel-initial allomorphs
of these enclitics only occur after consonant-final stems in my data.

The structure of the M\=/form of stems
before these vowel-initial enclitics
is summarised in \trf{tab:AmaMfoEnc}
according to the nine unique surface phonotactic
shapes of U\=/forms which can be identified.
The different M\=/forms in \trf{tab:AmaMfoEnc}
are completely predictable based on the corresponding U\=/form,
while the M\=/forms are not fully predictive of the U\=/forms.

\begin{table}[ht]
	\centering
	\caption{Amarasi M-forms before enclitics}\label{tab:AmaMfoEnc}
		\stl{0.5em}\begin{tabular}{lrclrcll}\lsptoprule
							&	U\=/form																	&		&M\=/form																&	U\=/form		&		&M\=/form				&gloss		\\ \midrule
						1.&V\sub{1}{\sA}C\sub{1}V\sub{2}{\sB}				&\ra&V\sub{1}{\sA}V{\sA}C\sub{1}C{\sB}		&\ve{fafi}	&\ra&\ve{faaf\j=}	&`pig'		\\
						2.&V\sub{1}C\sub{1}V\sub{2}C\sub{2}					&\ra&V\sub{1}V\sub{2}C\sub{1}C\sub{2}			&\ve{muʔit}	&\ra&\ve{muiʔt=}	&`animal'	\\
						3.&V\sub{1}\sub{\A\tsc{hi}}C\sub{1}V\sub{2}\sub{\B\tsc{mid}}
																												&\ra&V\sub{1}{\sA}V{\sA}C\sub{1}C{\sB}		&\ve{ume}		&\ra&\ve{uum\j=}	&`house'	\\
						4.&V\sub{1}{\sA}C\sub{1}a\sub{2}(C\sub{2})	&\ra&V\sub{1}{\sA}V{\sA}C\sub{1}(C\sub{2})&\ve{n-sosa}&\ra&\ve{n-soos=}	&`buy'	\\
						5.&V\sub{1}V\sub{2}C\sub{1}V\sub{3}{\sA}		&\ra&V\sub{1}V\sub{2}C\sub{1}C{\sA}				&\ve{n-aiti}	&\ra&\ve{n-ait\j=}	&`pick up'	\\
						6.&V\sub{1}V\sub{2}C\sub{1}V\sub{3}C\sub{2}	&\ra&V\sub{1}V\sub{2}C\sub{1}C\sub{2}			&\ve{nautus}&\ra&\ve{nauts=}	&`beetle'	\\
						7.&V\sub{1}V\sub{2}C\sub{1}									&\ra&V\sub{1}V\sub{2}C\sub{1}							&\ve{kaut}	&\ra&\ve{kaut=}		&`papaya'	\\
						8.&V\sub{1}{\sA}V\sub{2}{\sB}								&\ra&V\sub{1}{\sA}V{\sA}C{\sB}						&\ve{ai}		&\ra&\ve{aa\j=}		&`fire'		\\
%						9.&V\sub{1}C\sub{1}V\sub{2}-ʔ								&\ra&V\sub{1}V\sub{2}C\sub{1}							&\ve{mabe-ʔ}&\ra&\ve{maeb=}		&`time'		\\
						\lspbottomrule
		\end{tabular}
\end{table}

In this chapter I describe each of these M\=/forms in detail
and analyse the ways in which these M\=/forms are derived from the U\=/form.
This analysis has two fundamental elements:
feet should begin with a consonant
and prosodic words should
be phonologically separate from one another.

\section{Metathesis}\label{sec:Met ch:PhoMet}
Metathesis is obligatorily triggered before vowel-initial enclitics.
Examples of CVC{\#} final stems before the enclitic
\ve{=ee} are given in \qf{ex2:VCVC->VVCC=}.
In each example metathesis of the penultimate consonant
and final vowel occurs before the enclitic.

\begin{exe}
	\ex{{\ldots}V\sub{1}C\sub{1}V\sub{2}C\sub{2} {\ra} {\ldots}V\sub{1}V\sub{2}C\sub{1}C\sub{2}=V}\label{ex2:VCVC->VVCC=}
		\sn{\gw\begin{tabular}{rcclll}
			\ve{ra\tbr{mu}p}	&+&\ve{=ee}&{\ra}&\ve{ra\tbr{um}p=ee}	&`lamp'			\\
			\ve{mu\tbr{ʔi}t}	&+&\ve{=ee}&{\ra}&\ve{mu\tbr{iʔ}t=ee}	&`animal'		\\
			\ve{te\tbr{nu}k}	&+&\ve{=ee}&{\ra}&\ve{te\tbr{un}k=ee}	&`umbrella'	\\
			\ve{te\tbr{no}ʔ}	&+&\ve{=ee}&{\ra}&\ve{te\tbr{on}ʔ=ee}	&`egg'			\\
			\ve{u\tbr{ku}m}		&+&\ve{=ee}&{\ra}&\ve{u\tbr{uk}m=ee}	&`cuscus'		\\
			\ve{po\tbr{ʔo}n}	&+&\ve{=ee}&{\ra}&\ve{po\tbr{oʔ}n=ee}	&`garden'	\\
			\ve{o\tbr{ʔo}f}		&+&\ve{=ee}&{\ra}&\ve{o\tbr{oʔ}f=ee}	&`pen, corral'	\\
			\ve{ma\tbr{nu}s}	&+&\ve{=ee}&{\ra}&\ve{ma\tbr{un}s=ee}	&`betel vine'	\\
			\ve{a\tbr{na}h}		&+&\ve{=ee}&{\ra}&\ve{a\tbr{an}h=ee}	&`child'				\\
			\ve{mo\tbr{to}r}	&+&\ve{=ee}&{\ra}&\ve{mo\tbr{ot}r=ee}	&`motorbike'	\\
		\end{tabular}}
\end{exe}

I analyse this metathesis as occurring due to a crisp edge constraint.
This constraint prohibits a single element from
being linked to more than one prosodic category.
It is given in \qf{as:CrispEdge} below, as first described by \citet[208]{itme99}.
(The symbol `{$\mathfrak{C}$}' represents a prosodic category
such as a foot, a syllable, or a prosodic word
and `{\A}' is an element which is linked
to more than one prosodic category.)

\begin{exe}
	\ex{\tsc{Crisp-Edge}: ``Multiple linking between prosodic categories is prohibited''}\label{as:CrispEdge}
	\sna{\xy
		<0em,1cm>*\as{*}="star",<1em,1cm>*\as{\hp{\sub{1}}{$\mathfrak{C}$}\sub{1}}="s1",<3em,1cm>*\as{\hp{\sub{2}}{$\mathfrak{C}$}\sub{2}}="s2",
		<0em,0cm>*\as{}="c1",<1em,0cm>*\as{}="c2",<2em,0cm>*\as{\A}="c3",<3em,0cm>*\as{}="c4",<4em,0cm>*\as{}="c5",
		<0.5em,0cm>*\as{\ldots}="dots1",<3.5em,0cm>*\as{\ldots}="dots2",
		"c3"+U;"s1"+D**\dir{-};"c3"+U;"s2"+D**\dir{-};
		"c1"+U;"s1"+D**\dir{-};"c2"+U;"s1"+D**\dir{-};"c4"+U;"s2"+D**\dir{-};"c5"+U;"s2"+D**\dir{-};
		"c1"+U;"c2"+U**\dir{-};"c4"+U;"c5"+U**\dir{-};
	\endxy}
\end{exe}

The constraint in Amarasi is \tsc{Crisp-Edge}[PrWd], given in \qf{as:CrispEdge[S]} below,
which prohibits elements from being linked to more than one prosodic word.
Other prosodic categories in Amarasi can have multiple elements
such as the medial C-slot of a foot which is ambisyllabic
and linked to two syllables (\srf{sec:Syl}).

\begin{exe}
	\ex{\tsc{Crisp-Edge}[PrWd]}\label{as:CrispEdge[S]}
	\sna{\xy
		<0em,1cm>*\as{*}="star",<1.5em,1cm>*\as{\hp{\sub{1}}PrWd\sub{1}}="s1",<4.5em,1cm>*\as{\hp{\sub{2}}PrWd\sub{2}}="s2",
		<0em,0cm>*\as{}="c1",<1.5em,0cm>*\as{}="c2",<3em,0cm>*\as{\A}="c3",<4.5em,0cm>*\as{}="c4",<6em,0cm>*\as{}="c5",
		<0.75em,0cm>*\as{\ldots}="dots1",<5.25em,0cm>*\as{\ldots}="dots2",
		"c3"+U;"s1"+D**\dir{-};"c3"+U;"s2"+D**\dir{-};
		"c1"+U;"s1"+D**\dir{-};"c2"+U;"s1"+D**\dir{-};"c4"+U;"s2"+D**\dir{-};"c5"+U;"s2"+D**\dir{-};
		"c1"+U;"c2"+U**\dir{-};"c4"+U;"c5"+U**\dir{-};
	\endxy}
\end{exe}

Recall from \srf{sec:Str} that enclitics are extra-metrical and do not count for stress.
Instead, stress is assigned to the penultimate syllable of the clitic host.
I analyse this otherwise aberrant stress pattern as resulting
from a recursive prosodic word structure \sub{PrWd}[\sub{PrWd}[HOST]=cl],
in which the clitic does not form an independent prosodic word
but is parsed together with the clitic host.
Stress is then assigned to the most deeply embedded prosodic word.\footnote{
		Thanks goes to Daniel Kaufman for suggesting this analysis.}

When a consonant-initial enclitic attaches to
a host the resulting prosodic structure
does not violate \tsc{Crisp-Edge}[PrWd] as no segments are linked
to both the internal and external prosodic word.
Examples of consonant-initial enclitics (before which metathesis
is not obligatory) are given in \qf{ex:VCVC->VCVC=C} below,
which shows a number of verbs with consonant-initial pronominal enclitics attached.\footnote{
		CV{\#} final enclitics take the U\=/form or M\=/form
		according to discourse structures (Chapter \ref{ch:DisMet}).}

\newpage
\begin{exe}
	\ex{No metathesis before ={{\#}C}}\label{ex:VCVC->VCVC=C}
		\sn{\stl{0.4em}\gw\begin{tabular}{rcllll}
			\ve{n-roroʔ}		&+&\ve{=kau}	&\ra&\ve{na-ʔkoroʔ=kau}	&`tricks me'	\\
			\ve{na-naniʔ}		&+&\ve{=koo}	&\ra&\ve{na-naniʔ=koo}	&`moves you'	\\
			\ve{na-fani-ʔ}	&+&\ve{=kii}	&\ra&\ve{na-fani-ʔ=kii}	&`returns it to you (pl.)'	\\
			\ve{n-oʔen}			&+&\ve{=kiit}	&\ra&\ve{n-oʔen=kiit}	&`calls to us (incl.)'	\\
			\ve{na-barab}		&+&\ve{=kai}	&\ra&\ve{na-barab=kai}	&`prepares us (excl.)'	\\
			\ve{na-retaʔ}		&+&\ve{=siin}	&\ra&\ve{na-retaʔ=siin}	&`tells them a story'	\\
			\ve{t-biku}			&+&\ve{=siin}	&\ra&\ve{t-biku=siin}	&`curse them'	\\
			\ve{au uisneno}	&+&\ve{=kau}	&\ra&\ve{au uisneno=kau}	& `I am God'\\
			\ve{hii maufinu}&+&\ve{=kii}	&\ra&\ve{hii maufinu=kii}	& `you are evil'\\
			%\ve{na-traak}	&+&\ve{=kau}&\ra&\ve{na-traak=kau}	&`accuses me'	\\
			%\ve{na-fani-ʔ}	&+&\ve{=kiti}&\ra&\ve{na-fani-ʔ=kiti}	&`returns (it) to us (incl.)'	\\
			%\ve{n-batis}	&+&\ve{=kiit}&\ra&\ve{n-batis=kiit}	&`separates us (incl.)'	\\
		\end{tabular}}
\end{exe}

The prosodic structures and morphological structures of \ve{na-naniʔ=koo} `moves you'
and \ve{na-barab=kai} `prepares us' are given in \qf{as:nananiq=koo}.
These show that no element is attached to both prosodic words.
There is a crisp-edge after the internal prosodic word boundary,
as indicated by the dashed line.

Similarly, given the presence of empty C-slots (\srf{sec:EmpCSlo}),
a structure such as \ve{t-biku=siin} `curse them' also has a crisp
edge between the host and enclitic, as illustrated in \qf{as:tbiku=siin} below.

\begin{multicols}{2}
	\begin{exe}
		\exa{\xy
			<4em,5.5cm>*\as{{\hp{\sub{1}}PrWd\sub{1}}}="PrWd1",<6.5em,6.5cm>*\as{{\hp{\sub{2}}PrWd\sub{2}}}="PrWd2",
			<5em,4.5cm>*\as{\hp{\sub{1}}Ft\sub{1}}="Ft1",<10em,4.5cm>*\as{\hp{\sub{2}}Ft\sub{2}}="Ft2",
			<2em,3.5cm>*\as{\hp{\sub{1}}σ\sub{1}}="s1",<4em,3.5cm>*\as{\hp{\sub{2}}σ\sub{2}}="s2",<6em,3.5cm>*\as{\hp{\sub{3}}σ\sub{3}}="s3",<9em,3.5cm>*\as{\hp{\sub{4}}σ\sub{4}}="s4",<11em,3.5cm>*\as{\hp{\sub{5}}σ\sub{5}}="s5",
			<1em,2.5cm>*\as{C}="CV1",<2em,2.5cm>*\as{V}="CV2",<3em,2.5cm>*\as{C}="CV3",<4em,2.5cm>*\as{V}="CV4",<5em,2.5cm>*\as{C}="CV5",<6em,2.5cm>*\as{V}="CV6",<7em,2.5cm>*\as{C}="CV7",
			<8em,2.5cm>*\as{C}="CV8",<9em,2.5cm>*\as{V}="CV9",<10em,2.5cm>*\as{C}="CV10",<11em,2.5cm>*\as{V}="CV11",<12em,2.5cm>*\as{C}="CV12",
			<1em,1.5cm>*\as{n}="cv1",<2em,1.5cm>*\as{a}="cv2",<3em,1.5cm>*\as{b}="cv3",<4em,1.5cm>*\as{a}="cv4",<5em,1.5cm>*\as{r}="cv5",<6em,1.5cm>*\as{a}="cv6",<7em,1.5cm>*\as{b}="cv7",
			<8em,1.5cm>*\as{k}="cv8",<9em,1.5cm>*\as{a}="cv9",<10em,1.5cm>*\as{ }="cv10",<11em,1.5cm>*\as{i}="cv11",<12em,1.5cm>*\as{ }="cv12",
			<1em,1cm>*\as{n}="cv1.2",<2em,1cm>*\as{a}="cv2.2",<3em,1cm>*\as{n}="cv3.2",<4em,1cm>*\as{a}="cv4.2",<5em,1cm>*\as{n}="cv5.2",<6em,1cm>*\as{i}="cv6.2",<7em,1cm>*\as{ʔ}="cv7.2",
			<8em,1cm>*\as{k}="cv8.2",<9em,1cm>*\as{o}="cv9.2",<10em,1cm>*\as{ }="cv10.2",<11em,1cm>*\as{o}="cv11.2",<12em,1cm>*\as{ }="cv12.2",
			<1.5em,0cm>*\as{\hp{\sub{1}}M\sub{1}}="m1",<5em,0cm>*\as{\hp{\sub{2}}M\sub{2}}="m2",<9.5em,0cm>*\as{\hp{\sub{3}}M\sub{3}}="m3",<2.5em,0cm>*\as{-}="-",<7.5em,0cm>*\as{=}="=",
			"m1"+U;"cv1.2"+D**\dir{-};"m1"+U;"cv2.2"+D**\dir{-};"m2"+U;"cv3.2"+D**\dir{-};"m2"+U;"cv4.2"+D**\dir{-};"m2"+U;"cv5.2"+D**\dir{-};"m2"+U;"cv6.2"+D**\dir{-};"m2"+U;"cv7.2"+D**\dir{-};
			"m3"+U;"cv8.2"+D**\dir{-};"m3"+U;"cv9.2"+D**\dir{-};"m3"+U;"cv11.2"+D**\dir{-};
			"cv1"+U;"CV1"+D**\dir{-};"cv2"+U;"CV2"+D**\dir{-};"cv3"+U;"CV3"+D**\dir{-};"cv4"+U;"CV4"+D**\dir{-};"cv5"+U;"CV5"+D**\dir{-};"cv6"+U;"CV6"+D**\dir{-};"cv7"+U;"CV7"+D**\dir{-};
			"cv8"+U;"CV8"+D**\dir{-};"cv9"+U;"CV9"+D**\dir{-};"cv11"+U;"CV11"+D**\dir{-};
			"CV1"+U;"s1"+D**\dir{-};"CV2"+U;"s1"+D**\dir{-};"CV3"+U;"s1"+D**\dir{-};"CV3"+U;"s2"+D**\dir{-};"CV4"+U;"s2"+D**\dir{-};"CV5"+U;"s2"+D**\dir{-};
			"CV5"+U;"s3"+D**\dir{-};"CV6"+U;"s3"+D**\dir{-};"CV7"+U;"s3"+D**\dir{-};
			"CV8"+U;"s4"+D**\dir{-};"CV9"+U;"s4"+D**\dir{-};"CV10"+U;"s4"+D**\dir{-};
			"CV10"+U;"s5"+D**\dir{-};"CV11"+U;"s5"+D**\dir{-};"CV12"+U;"s5"+D**\dir{-};
			"s1"+U;"PrWd1"+D**\dir{-};"s2"+U;"Ft1"+D**\dir{-};"s3"+U;"Ft1"+D**\dir{-};"s4"+U;"Ft2"+D**\dir{-};"s5"+U;"Ft2"+D**\dir{-};
			"Ft1"+U;"PrWd1"+D**\dir{-};"PrWd1"+U;"PrWd2"+D**\dir{-};"Ft2"+U;"PrWd2"+D**\dir{-};
			<7.5em,3.25cm>*\as{\tikz[red,thick,dashed,baseline=0.9ex]\draw (0,0) -- (0,5cm);}="line",
		\endxy}\label{as:nananiq=koo}
	\end{exe}
	\begin{exe}
		\exa{\xy
			<4em,5cm>*\as{{\hp{\sub{1}}PrWd\sub{1}}}="PrWd1",<5.5em,6cm>*\as{{\hp{\sub{2}}PrWd\sub{2}}}="PrWd2",
			<4em,4cm>*\as{\hp{\sub{1}}Ft\sub{1}}="Ft1",<8.5em,4cm>*\as{\hp{\sub{\tsc{m}}}Ft\sub{\tsc{m}}}="Ft2",
			<3em,3cm>*\as{\hp{\sub{1}}σ\sub{1}}="s1",<5em,3cm>*\as{\hp{\sub{2}}σ\sub{2}}="s2",<7.5em,3cm>*\as{\hp{\sub{3}}σ\sub{3}}="s3",<9.5em,3cm>*\as{\hp{\sub{4}}σ\sub{4}}="s4",
			<1em,2cm>*\as{C}="CV1",<2em,2cm>*\as{C}="CV2",<3em,2cm>*\as{V}="CV3",<4em,2cm>*\as{C}="CV4",<5em,2cm>*\as{V}="CV5",<6em,2cm>*\as{C}="CV6",
			<7em,2cm>*\as{C}="CV7",<8em,2cm>*\as{V}="CV8",<9em,2cm>*\as{V}="CV9",<10em,2cm>*\as{C}="CV10",
			<1em,1cm>*\as{t}="cv1",<2em,1cm>*\as{b}="cv2",<3em,1cm>*\as{i}="cv3",<4em,1cm>*\as{k}="cv4",<5em,1cm>*\as{u}="cv5",<6em,1cm>*\as{ }="cv6",
			<7em,1cm>*\as{s}="cv7",<8em,1cm>*\as{i}="cv8",<9em,1cm>*\as{i}="cv9",<10em,1cm>*\as{n}="cv10",
			<1em,0cm>*\as{\hp{\sub{1}}M\sub{1}}="m1",<3.5em,0cm>*\as{\hp{\sub{2}}M\sub{2}}="m2",<8.5em,0cm>*\as{\hp{\sub{3}}M\sub{3}}="m3",<2em,0cm>*\as{-}="-",<6em,0cm>*\as{=}="=",
			"m1"+U;"cv1"+D**\dir{-};"m2"+U;"cv2"+D**\dir{-};"m2"+U;"cv3"+D**\dir{-};"m2"+U;"cv4"+D**\dir{-};"m2"+U;"cv5"+D**\dir{-};
			"m3"+U;"cv7"+D**\dir{-};"m3"+U;"cv8"+D**\dir{-};"m3"+U;"cv9"+D**\dir{-};"m3"+U;"cv10"+D**\dir{-};
			"cv1"+U;"CV1"+D**\dir{-};"cv2"+U;"CV2"+D**\dir{-};"cv3"+U;"CV3"+D**\dir{-};"cv4"+U;"CV4"+D**\dir{-};"cv5"+U;"CV5"+D**\dir{-};"cv6"+U;"CV6"+D**\dir{};
			"cv7"+U;"CV7"+D**\dir{-};"cv8"+U;"CV8"+D**\dir{-};"cv9"+U;"CV9"+D**\dir{-};"cv10"+U;"CV10"+D**\dir{-};
			"CV2"+U;"s1"+D**\dir{-};"CV3"+U;"s1"+D**\dir{-};"CV4"+U;"s1"+D**\dir{-};
			"CV4"+U;"s2"+D**\dir{-};"CV5"+U;"s2"+D**\dir{-};"CV6"+U;"s2"+D**\dir{-};
			"CV7"+U;"s3"+D**\dir{-};"CV8"+U;"s3"+D**\dir{-};"CV9"+U;"s4"+D**\dir{-};"CV10"+U;"s4"+D**\dir{-};
			"s1"+U;"Ft1"+D**\dir{-};"s2"+U;"Ft1"+D**\dir{-};"s3"+U;"Ft2"+D**\dir{-};"s4"+U;"Ft2"+D**\dir{-};
			"CV1"+U;"PrWd1"+D**\dir{-};"Ft1"+U;"PrWd1"+D**\dir{-};"PrWd1"+U;"PrWd2"+D**\dir{-};"Ft2"+U;"PrWd2"+D**\dir{-};
			<6.5em,3cm>*\as{\tikz[red,thick,dashed,baseline=0.9ex]\draw (0,0) -- (0,4.5cm);}="line",
		\endxy}\label{as:tbiku=siin}
	\end{exe}
\end{multicols}

However, vowel-initial enclitics have a
defective foot structure with no initial C-slot.
That is, rather than taking the otherwise obligatory
CVCVC foot their foot structure is VCVC (\srf{sec:TheFoo}).
As a result, the final C-slot of the clitic host
is the onset C-slot for the initial syllable of the enclitic.

This is illustrated in \qf{as:muqit=ee} below,
which shows the structures of \ve{muʔit} `animal' + \ve{=ee}
and \ve{ukum} `cuscus' + \ve{=ee} before metathesis takes place.
The final C-slot of the host is ultimately linked to both the
internal prosodic word (PrWd\sub{1}) and the external prosodic word
(PrWd\sub{2}) as mediated by the syllabic and foot
structures of the host and enclitic,
thus violating \tsc{Crisp-Edge}[PrWd].
This illicit structure is resolved by final CV {\ra} VC metathesis
of the clitic host. This yields the structure in \qf{as:muiqt=ee}
in which the final C-slot of the host is only linked to the following
prosodic structures; the prosodic structures of the enclitic.

\begin{multicols}{2}
\begin{exe}\ex{\label{as:muqit+=ee}
	\begin{xlist}
	\exa{\xy
		<3em,5.5cm>*\as{\hp{\sub{1}}PrWd\sub{1}}="PrWd1",<5em,6.5cm>*\as{\hp{\sub{2}}PrWd\sub{2}}="PrWd2",
		<3em,4.5cm>*\as{\hp{\sub{1}}Ft\sub{1}}="Ft1",<7em,4.5cm>*\as{\hp{\sub{2}}Ft\sub{2}}="Ft2",
		<2em,3.5cm>*\as{\hp{\sub{1}}σ\sub{1}}="s1",<4em,3.5cm>*\as{\hp{\sub{2}}σ\sub{2}}="s2",<6em,3.5cm>*\as{\hp{\sub{3}}σ\sub{3}}="s3",<8em,3.5cm>*\as{\hp{\sub{4}}σ\sub{4}}="s4",
		<1em,2.5cm>*\as{C}="CV1",<2em,2.5cm>*\as{V}="CV2",<3em,2.5cm>*\as{C}="CV3",<4em,2.5cm>*\as{V}="CV4",<5em,2.5cm>*\as{C}="CV5",
		<6em,2.5cm>*\as{V}="CV6",<7em,2.5cm>*\as{C}="CV7",<8em,2.5cm>*\as{V}="CV8",<9em,2.5cm>*\as{C}="CV9",
		<1em,1.5cm>*\as{m}="cv1",<2em,1.5cm>*\as{u}="cv2",<3em,1.5cm>*\as{ʔ}="cv3",<4em,1.5cm>*\as{i}="cv4",<5em,1.5cm>*\as{t}="cv5",
		<6em,1.5cm>*\as{e}="cv6",<7em,1.5cm>*\as{ }="cv7",<8em,1.5cm>*\as{e}="cv8",<9em,1.5cm>*\as{ }="cv9",
		<1em,1cm>*\as{[ʔ]}="cv1.2",<2em,1cm>*\as{u}="cv2.2",<3em,1cm>*\as{k}="cv3.2",<4em,1cm>*\as{u}="cv4.2",<5em,1cm>*\as{m}="cv5.2",
		<6em,1cm>*\as{e}="cv6.2",<7em,1cm>*\as{ }="cv7.2",<8em,1cm>*\as{e}="cv8.2",<9em,1cm>*\as{ }="cv9.2",
		<3em,0cm>*\as{\hp{\sub{1}}M\sub{1}}="m1",<7em,0cm>*\as{\hp{\sub{2}}M\sub{2}}="m2",<5.5em,0cm>*\as{=}="=",
		"m1"+U;"cv1.2"+D**\dir{-};"m1"+U;"cv2.2"+D**\dir{-};"m1"+U;"cv3.2"+D**\dir{-};"m1"+U;"cv4.2"+D**\dir{-};"m1"+U;"cv5.2"+D**\dir{-};
		"m2"+U;"cv6.2"+D**\dir{-};"m2"+U;"cv8.2"+D**\dir{-};
		"cv1"+U;"CV1"+D**\dir{-};"cv2"+U;"CV2"+D**\dir{-};"cv3"+U;"CV3"+D**\dir{-};"cv4"+U;"CV4"+D**\dir{-};"cv5"+U;"CV5"+D**\dir{-};
		"cv6"+U;"CV6"+D**\dir{-};"cv7"+U;"CV7"+D**\dir{};"cv8"+U;"CV8"+D**\dir{-};"cv9"+U;"CV9"+D**\dir{};
		"CV1"+U;"s1"+D**\dir{-};"CV2"+U;"s1"+D**\dir{-};"CV3"+U;"s1"+D**\dir{-};"CV3"+U;"s2"+D**\dir{-};"CV4"+U;"s2"+D**\dir{-};"CV5"+U;"s2"+D**\dir{-};
		"CV5"+U;"s3"+D**\dir{-};"CV6"+U;"s3"+D**\dir{-};"CV7"+U;"s3"+D**\dir{-};"CV7"+U;"s4"+D**\dir{-};"CV8"+U;"s4"+D**\dir{-};"CV9"+U;"s4"+D**\dir{-};
		"s1"+U;"Ft1"+D**\dir{-};"s2"+U;"Ft1"+D**\dir{-};"s3"+U;"Ft2"+D**\dir{-};"s4"+U;"Ft2"+D**\dir{-};
		"Ft1"+U;"PrWd1"+D**\dir{-};"PrWd1"+U;"PrWd2"+D**\dir{-};"Ft2"+U;"PrWd2"+D**\dir{-};
		<5em,1.75cm>*\as{\tikz[red,thick,dashed,baseline=0.9ex]\draw (0,0) rectangle (0.4cm,2cm);}="box",
	\endxy}\label{as:muqit=ee}
	\exa{\xy
		<2.5em,5.5cm>*\as{\hp{\sub{1}}PrWd\sub{1}}="PrWd1",<4.5em,6.5cm>*\as{\hp{\sub{2}}PrWd\sub{2}}="PrWd2",
		<2.5em,4.5cm>*\as{\hp{\sub{\tsc{m}}}Ft\sub{\tsc{m}}}="Ft1",<7em,4.5cm>*\as{\hp{\sub{2}}Ft\sub{2}}="Ft2",
<1.5em,3.5cm>*\as{\hp{\sub{1}}σ\sub{1}}="s1",<3.5em,3.5cm>*\as{\hp{\sub{2}}σ\sub{2}}="s2",<6em,3.5cm>*\as{\hp{\sub{3}}σ\sub{3}}="s3",<8em,3.5cm>*\as{\hp{\sub{4}}σ\sub{4}}="s4",
		<1em,2.5cm>*\as{C}="CV1",<2em,2.5cm>*\as{V}="CV2",<3em,2.5cm>*\as{V}="CV3",<4em,2.5cm>*\as{C}="CV4",<5em,2.5cm>*\as{C}="CV5",
		<6em,2.5cm>*\as{V}="CV6",<7em,2.5cm>*\as{C}="CV7",<8em,2.5cm>*\as{V}="CV8",<9em,2.5cm>*\as{C}="CV9",
		<1em,1.5cm>*\as{m}="cv1",<2em,1.5cm>*\as{u}="cv2",<3em,1.5cm>*\as{i}="cv3",<4em,1.5cm>*\as{ʔ}="cv4",<5em,1.5cm>*\as{t}="cv5",
		<6em,1.5cm>*\as{e}="cv6",<7em,1.5cm>*\as{ }="cv7",<8em,1.5cm>*\as{e}="cv8",<9em,1.5cm>*\as{ }="cv9",
		<1em,1cm>*\as{[ʔ]}="cv1.2",<2em,1cm>*\as{u}="cv2.2",<3em,1cm>*\as{u}="cv3.2",<4em,1cm>*\as{k}="cv4.2",<5em,1cm>*\as{m}="cv5.2",
		<6em,1cm>*\as{e}="cv6.2",<7em,1cm>*\as{ }="cv7.2",<8em,1cm>*\as{e}="cv8.2",<9em,1cm>*\as{ }="cv9.2",
		<3em,0cm>*\as{\hp{\sub{1}}M\sub{1}}="m1",<7em,0cm>*\as{\hp{\sub{2}}M\sub{2}}="m2",<5.5em,0cm>*\as{=}="=",
		"m1"+U;"cv1.2"+D**\dir{-};"m1"+U;"cv2.2"+D**\dir{-};"m1"+U;"cv3.2"+D**\dir{-};"m1"+U;"cv4.2"+D**\dir{-};"m1"+U;"cv5.2"+D**\dir{-};
		"m2"+U;"cv6.2"+D**\dir{-};"m2"+U;"cv8.2"+D**\dir{-};
		"cv1"+U;"CV1"+D**\dir{-};"cv2"+U;"CV2"+D**\dir{-};"cv3"+U;"CV3"+D**\dir{-};"cv4"+U;"CV4"+D**\dir{-};"cv5"+U;"CV5"+D**\dir{-};
		"cv6"+U;"CV6"+D**\dir{-};"cv7"+U;"CV7"+D**\dir{};"cv8"+U;"CV8"+D**\dir{-};"cv9"+U;"CV9"+D**\dir{};
		"CV1"+U;"s1"+D**\dir{-};"CV2"+U;"s1"+D**\dir{-};"CV3"+U;"s2"+D**\dir{-};"CV4"+U;"s2"+D**\dir{-};
		"CV5"+U;"s3"+D**\dir{-};"CV6"+U;"s3"+D**\dir{-};"CV7"+U;"s3"+D**\dir{-};"CV7"+U;"s4"+D**\dir{-};"CV8"+U;"s4"+D**\dir{-};"CV9"+U;"s4"+D**\dir{-};
		"s1"+U;"Ft1"+D**\dir{-};"s2"+U;"Ft1"+D**\dir{-};"s3"+U;"Ft2"+D**\dir{-};"s4"+U;"Ft2"+D**\dir{-};
		"Ft1"+U;"PrWd1"+D**\dir{-};"PrWd1"+U;"PrWd2"+D**\dir{-};"Ft2"+U;"PrWd2"+D**\dir{-};
		<4.5em,3.25cm>*\as{\tikz[red,thick,dashed,baseline=0.9ex]\draw (0,0) -- (0,5cm);}="line",
	\endxy}\label{as:muiqt=ee}
	\end{xlist}}
\end{exe}
\end{multicols}

This analysis is dependent on the analysis of intervocalic
consonants as ambisyllabic (\srf{sec:Syl}) --
that is being both the coda for the previous syllable
and onset for the following syllable.
If the intervocalic consonant were not ambisyllabic,
\tsc{Crisp-Edge}[PrWd] would not be violated
and metathesis would not be necessary.
The analysis of intervocalic consonants as ambisyllabic in Amarasi
finds independent support from reduplication (\srf{sec:Red}),
in which the initial syllable -- including the intervocalic consonant --
is the reduplicant.

In the case of root final consonants,
as in \qf{as:muqit+=ee} above, metathesis results in a mismatch between the prosodic
and morphological structures of the clitic host and enclitic.
While the final consonant is morphologically a member of
the clitic host, phonologically it is a member of the enclitic.

Such a mismatch does not occur when the final
consonant of the host is a suffix.
The structures of two stems with a final suffix and vowel-initial
enclitic are shown in \qf{as:moni-t+=ee} below;
both before metathesis in \qf{as:moni-t=ee} and after metathesis in \qf{as:moin-t=ee}.
These words are \ve{{\rt}moni} `live' + \ve{-t} `{\at}' {\ra} \ve{moin-t=ee} `the life'
and \ve{feto} `man's sister' + \ve{-f} `{\F}' + {\ee} {\ra} \ve{feot-f=ee} `the sister'.

\begin{multicols}{2}
\begin{exe}\ex{\label{as:moni-t+=ee}
	\begin{xlist}
	\exa{\xy
		<3em,5.5cm>*\as{\hp{\sub{1}}PrWd\sub{1}}="PrWd1",<5em,6.5cm>*\as{\hp{\sub{2}}PrWd\sub{2}}="PrWd2",
		<3em,4.5cm>*\as{\hp{\sub{1}}Ft\sub{1}}="Ft1",<7em,4.5cm>*\as{\hp{\sub{2}}Ft\sub{2}}="Ft2",
		<2em,3.5cm>*\as{\hp{\sub{1}}σ\sub{1}}="s1",<4em,3.5cm>*\as{\hp{\sub{2}}σ\sub{2}}="s2",<6em,3.5cm>*\as{\hp{\sub{3}}σ\sub{3}}="s3",<8em,3.5cm>*\as{\hp{\sub{4}}σ\sub{4}}="s4",
		<1em,2.5cm>*\as{C}="CV1",<2em,2.5cm>*\as{V}="CV2",<3em,2.5cm>*\as{C}="CV3",<4em,2.5cm>*\as{V}="CV4",<5em,2.5cm>*\as{C}="CV5",
		<6em,2.5cm>*\as{V}="CV6",<7em,2.5cm>*\as{C}="CV7",<8em,2.5cm>*\as{V}="CV8",<9em,2.5cm>*\as{C}="CV9",
		<1em,1.5cm>*\as{m}="cv1",<2em,1.5cm>*\as{o}="cv2",<3em,1.5cm>*\as{n}="cv3",<4em,1.5cm>*\as{i}="cv4",<5em,1.5cm>*\as{t}="cv5",
		<6em,1.5cm>*\as{e}="cv6",<7em,1.5cm>*\as{ }="cv7",<8em,1.5cm>*\as{e}="cv8",<9em,1.5cm>*\as{ }="cv9",
		<1em,1cm>*\as{f}="cv1.2",<2em,1cm>*\as{e}="cv2.2",<3em,1cm>*\as{t}="cv3.2",<4em,1cm>*\as{o}="cv4.2",<5em,1cm>*\as{f}="cv5.2",
		<6em,1cm>*\as{e}="cv6.2",<7em,1cm>*\as{ }="cv7.2",<8em,1cm>*\as{e}="cv8.2",<9em,1cm>*\as{ }="cv9.2",
		<2.5em,0cm>*\as{\hp{\sub{1}}M\sub{1}}="m1",<5em,0cm>*\as{\hp{\sub{2}}M\sub{2}}="m2",<7em,0cm>*\as{\hp{\sub{3}}M\sub{3}}="m3",
		<4em,0cm>*\as{-}="-",<6.25em,0cm>*\as{=}="=",
		"m1"+U;"cv1.2"+D**\dir{-};"m1"+U;"cv2.2"+D**\dir{-};"m1"+U;"cv3.2"+D**\dir{-};"m1"+U;"cv4.2"+D**\dir{-};"m2"+U;"cv5.2"+D**\dir{-};
		"m3"+U;"cv6.2"+D**\dir{-};"m3"+U;"cv8.2"+D**\dir{-};
		"cv1"+U;"CV1"+D**\dir{-};"cv2"+U;"CV2"+D**\dir{-};"cv3"+U;"CV3"+D**\dir{-};"cv4"+U;"CV4"+D**\dir{-};"cv5"+U;"CV5"+D**\dir{-};
		"cv6"+U;"CV6"+D**\dir{-};"cv7"+U;"CV7"+D**\dir{};"cv8"+U;"CV8"+D**\dir{-};"cv9"+U;"CV9"+D**\dir{};
		"CV1"+U;"s1"+D**\dir{-};"CV2"+U;"s1"+D**\dir{-};"CV3"+U;"s1"+D**\dir{-};"CV3"+U;"s2"+D**\dir{-};"CV4"+U;"s2"+D**\dir{-};"CV5"+U;"s2"+D**\dir{-};
		"CV5"+U;"s3"+D**\dir{-};"CV6"+U;"s3"+D**\dir{-};"CV7"+U;"s3"+D**\dir{-};"CV7"+U;"s4"+D**\dir{-};"CV8"+U;"s4"+D**\dir{-};"CV9"+U;"s4"+D**\dir{-};
		"s1"+U;"Ft1"+D**\dir{-};"s2"+U;"Ft1"+D**\dir{-};"s3"+U;"Ft2"+D**\dir{-};"s4"+U;"Ft2"+D**\dir{-};
		"Ft1"+U;"PrWd1"+D**\dir{-};"PrWd1"+U;"PrWd2"+D**\dir{-};"Ft2"+U;"PrWd2"+D**\dir{-};
		<5em,1.75cm>*\as{\tikz[red,thick,dashed,baseline=0.9ex]\draw (0,0) rectangle (0.4cm,2cm);}="box",
	\endxy}\label{as:moni-t=ee}
	\exa{\xy
		<2.5em,5.5cm>*\as{\hp{\sub{1}}PrWd\sub{1}}="PrWd1",<4.5em,6.5cm>*\as{\hp{\sub{2}}PrWd\sub{2}}="PrWd2",
		<2.5em,4.5cm>*\as{\hp{\sub{\tsc{m}}}Ft\sub{\tsc{m}}}="Ft1",<7em,4.5cm>*\as{\hp{\sub{2}}Ft\sub{2}}="Ft2",
<1.5em,3.5cm>*\as{\hp{\sub{1}}σ\sub{1}}="s1",<3.5em,3.5cm>*\as{\hp{\sub{2}}σ\sub{2}}="s2",<6em,3.5cm>*\as{\hp{\sub{3}}σ\sub{3}}="s3",<8em,3.5cm>*\as{\hp{\sub{4}}σ\sub{4}}="s4",
		<1em,2.5cm>*\as{C}="CV1",<2em,2.5cm>*\as{V}="CV2",<3em,2.5cm>*\as{V}="CV3",<4em,2.5cm>*\as{C}="CV4",<5em,2.5cm>*\as{C}="CV5",
		<6em,2.5cm>*\as{V}="CV6",<7em,2.5cm>*\as{C}="CV7",<8em,2.5cm>*\as{V}="CV8",<9em,2.5cm>*\as{C}="CV9",
		<1em,1.5cm>*\as{m}="cv1",<2em,1.5cm>*\as{o}="cv2",<3em,1.5cm>*\as{i}="cv3",<4em,1.5cm>*\as{n}="cv4",<5em,1.5cm>*\as{t}="cv5",
		<6em,1.5cm>*\as{e}="cv6",<7em,1.5cm>*\as{ }="cv7",<8em,1.5cm>*\as{e}="cv8",<9em,1.5cm>*\as{ }="cv9",
		<1em,1cm>*\as{f}="cv1.2",<2em,1cm>*\as{e}="cv2.2",<3em,1cm>*\as{o}="cv3.2",<4em,1cm>*\as{t}="cv4.2",<5em,1cm>*\as{f}="cv5.2",
		<6em,1cm>*\as{e}="cv6.2",<7em,1cm>*\as{ }="cv7.2",<8em,1cm>*\as{e}="cv8.2",<9em,1cm>*\as{ }="cv9.2",
		<2.5em,0cm>*\as{\hp{\sub{1}}M\sub{1}}="m1",<5em,0cm>*\as{\hp{\sub{2}}M\sub{2}}="m2",<7em,0cm>*\as{\hp{\sub{3}}M\sub{3}}="m3",
		<4em,0cm>*\as{-}="-",<6.25em,0cm>*\as{=}="=",
		"m1"+U;"cv1.2"+D**\dir{-};"m1"+U;"cv2.2"+D**\dir{-};"m1"+U;"cv3.2"+D**\dir{-};"m1"+U;"cv4.2"+D**\dir{-};"m2"+U;"cv5.2"+D**\dir{-};
		"m3"+U;"cv6.2"+D**\dir{-};"m3"+U;"cv8.2"+D**\dir{-};
		"cv1"+U;"CV1"+D**\dir{-};"cv2"+U;"CV2"+D**\dir{-};"cv3"+U;"CV3"+D**\dir{-};"cv4"+U;"CV4"+D**\dir{-};"cv5"+U;"CV5"+D**\dir{-};
		"cv6"+U;"CV6"+D**\dir{-};"cv7"+U;"CV7"+D**\dir{};"cv8"+U;"CV8"+D**\dir{-};"cv9"+U;"CV9"+D**\dir{};
		"CV1"+U;"s1"+D**\dir{-};"CV2"+U;"s1"+D**\dir{-};"CV3"+U;"s2"+D**\dir{-};"CV4"+U;"s2"+D**\dir{-};
		"CV5"+U;"s3"+D**\dir{-};"CV6"+U;"s3"+D**\dir{-};"CV7"+U;"s3"+D**\dir{-};"CV7"+U;"s4"+D**\dir{-};"CV8"+U;"s4"+D**\dir{-};"CV9"+U;"s4"+D**\dir{-};
		"s1"+U;"Ft1"+D**\dir{-};"s2"+U;"Ft1"+D**\dir{-};"s3"+U;"Ft2"+D**\dir{-};"s4"+U;"Ft2"+D**\dir{-};
		"Ft1"+U;"PrWd1"+D**\dir{-};"PrWd1"+U;"PrWd2"+D**\dir{-};"Ft2"+U;"PrWd2"+D**\dir{-};
		<4.5em,3.25cm>*\as{\tikz[red,thick,dashed,baseline=0.9ex]\draw (0,0) -- (0,5cm);}="line",
	\endxy}\label{as:moin-t=ee}
	\end{xlist}}
\end{exe}
\end{multicols}

Similarly, for vowel-final hosts after which consonant insertion occurs (\srf{sec:ConIns}),
the consonant which occurs in this C-slot is neither morphologically a
memebr of the host or the enclitic.

\section{Consonant insertion}\label{sec:ConIns}
When a vowel-initial enclitic is attached to a vowel-final stem,
a consonant conditioned by the final vowel of the stem is inserted.
After the front vowels /i/ and /e/ the inserted consonant is /\j/.
After the back rounded vowels /u/ and /o/ the inserted consonant is /ɡw/.
Examples are given in \qf{ex:VV->VVC=V} below.
(Consonant insertion after /a/ is discussed in \srf{sec:CliHosFinA} below.)

\begin{exe}
	\ex{VV[+\A\tsc{place}]+=V {\ra} VVC[+\A\tsc{place}]=V} \label{ex:VV->VVC=V}
	\sn{\gw\begin{tabular}{rlllll}
		\ve{nii} 	&+&\ve{=ee}&{\ra}&\ve{nii\tbr{\j}=ee}	& `the pole' \\
		\ve{fee} 	&+&\ve{=ee}&{\ra}&\ve{fee\tbr{\j}=ee}	& `the wife' \\
		\ve{kfuu} &+&\ve{=ee}&{\ra}&\ve{kfuu\tbr{gw}=ee}& `the star' \\
		\ve{oo} 	&+&\ve{=ee}&{\ra}&\ve{oo\tbr{gw}=ee}	& `the bamboo' \\
	\end{tabular}}
\end{exe}

This consonant insertion takes place because
feet in Amarasi require an onset consonant.
The requirement for an onset is a very common cross-linguistically \citep{mcpr93,prsm93}.

The requirement for feet to begin with an onset consonant
is also the reason for glottal stop insertion (\srf{sec:GloStoIns2}),
as illustrated for the vowel-initial stem \ve{ukum} `cuscus' in \qf{as:uukm=ee} below.
Because feet require an initial consonant, a glottal stop is inserted in \qf{as:quukm=ee}.
The glottal stop is the default initial consonant.
(This representations in \qf{as:uukm=ee>quukm=ee}
have been simplified by removing the tiers showing
the prosodic words and morphemes.)

\begin{multicols}{2}
\begin{exe}\ex{\label{as:uukm=ee>quukm=ee}
	\begin{xlist}
	\exa{\xy
		<2.5em,3cm>*\as{\hp{\sub{\tsc{m}}}Ft\sub{\tsc{m}}}="Ft1",<7em,3cm>*\as{\hp{\sub{2}}Ft\sub{2}}="Ft2",
		<1.5em,2cm>*\as{\hp{\sub{1}}σ\sub{1}}="s1",<3.5em,2cm>*\as{\hp{\sub{2}}σ\sub{2}}="s2",<6em,2cm>*\as{\hp{\sub{3}}σ\sub{3}}="s3",<8em,2cm>*\as{\hp{\sub{4}}σ\sub{4}}="s4",
		<1em,1cm>*\as{C}="CV1",<2em,1cm>*\as{V}="CV2",<3em,1cm>*\as{V}="CV3",<4em,1cm>*\as{C}="CV4",<5em,1cm>*\as{C}="CV5",
		<6em,1cm>*\as{V}="CV6",<7em,1cm>*\as{C}="CV7",<8em,1cm>*\as{V}="CV8",<9em,1cm>*\as{C}="CV9",
		<1em,0cm>*\as{\0}="cv1",<2em,0cm>*\as{u}="cv2",<3em,0cm>*\as{u}="cv3",<4em,0cm>*\as{k}="cv4",<5em,0cm>*\as{m}="cv5",
		<6em,0cm>*\as{e}="cv6",<7em,0cm>*\as{ }="cv7",<8em,0cm>*\as{e}="cv8",<9em,0cm>*\as{ }="cv9",
		"cv1"+U;"CV1"+D**\dir{-};"cv2"+U;"CV2"+D**\dir{-};"cv3"+U;"CV3"+D**\dir{-};"cv4"+U;"CV4"+D**\dir{-};"cv5"+U;"CV5"+D**\dir{-};
		"cv6"+U;"CV6"+D**\dir{-};"cv7"+U;"CV7"+D**\dir{};"cv8"+U;"CV8"+D**\dir{-};"cv9"+U;"CV9"+D**\dir{};
		"CV1"+U;"s1"+D**\dir{-};"CV2"+U;"s1"+D**\dir{-};"CV3"+U;"s2"+D**\dir{-};"CV4"+U;"s2"+D**\dir{-};
		"CV5"+U;"s3"+D**\dir{-};"CV6"+U;"s3"+D**\dir{-};"CV7"+U;"s3"+D**\dir{-};"CV7"+U;"s4"+D**\dir{-};"CV8"+U;"s4"+D**\dir{-};"CV9"+U;"s4"+D**\dir{-};
		"s1"+U;"Ft1"+D**\dir{-};"s2"+U;"Ft1"+D**\dir{-};"s3"+U;"Ft2"+D**\dir{-};"s4"+U;"Ft2"+D**\dir{-};
		<1em,0.5cm>*\as{\tikz[red,thick,dashed,baseline=0.9ex]\draw (0,0) rectangle (0.4cm,1.5cm);}="box",
	\endxy}\label{as:uukm=ee}
	\exa{\xy
		<2.5em,3cm>*\as{\hp{\sub{\tsc{m}}}Ft\sub{\tsc{m}}}="Ft1",<7em,3cm>*\as{\hp{\sub{2}}Ft\sub{2}}="Ft2",
		<1.5em,2cm>*\as{\hp{\sub{1}}σ\sub{1}}="s1",<3.5em,2cm>*\as{\hp{\sub{2}}σ\sub{2}}="s2",<6em,2cm>*\as{\hp{\sub{3}}σ\sub{3}}="s3",<8em,2cm>*\as{\hp{\sub{4}}σ\sub{4}}="s4",
		<1em,1cm>*\as{C}="CV1",<2em,1cm>*\as{V}="CV2",<3em,1cm>*\as{V}="CV3",<4em,1cm>*\as{C}="CV4",<5em,1cm>*\as{C}="CV5",
		<6em,1cm>*\as{V}="CV6",<7em,1cm>*\as{C}="CV7",<8em,1cm>*\as{V}="CV8",<9em,1cm>*\as{C}="CV9",
		<1em,0cm>*\as{\tbr{ʔ}}="cv1",<2em,0cm>*\as{u}="cv2",<3em,0cm>*\as{u}="cv3",<4em,0cm>*\as{k}="cv4",<5em,0cm>*\as{m}="cv5",
		<6em,0cm>*\as{e}="cv6",<7em,0cm>*\as{ }="cv7",<8em,0cm>*\as{e}="cv8",<9em,0cm>*\as{ }="cv9",
		"cv1"+U;"CV1"+D**\dir{-};"cv2"+U;"CV2"+D**\dir{-};"cv3"+U;"CV3"+D**\dir{-};"cv4"+U;"CV4"+D**\dir{-};"cv5"+U;"CV5"+D**\dir{-};
		"cv6"+U;"CV6"+D**\dir{-};"cv7"+U;"CV7"+D**\dir{};"cv8"+U;"CV8"+D**\dir{-};"cv9"+U;"CV9"+D**\dir{};
		"CV1"+U;"s1"+D**\dir{-};"CV2"+U;"s1"+D**\dir{-};"CV3"+U;"s2"+D**\dir{-};"CV4"+U;"s2"+D**\dir{-};
		"CV5"+U;"s3"+D**\dir{-};"CV6"+U;"s3"+D**\dir{-};"CV7"+U;"s3"+D**\dir{-};"CV7"+U;"s4"+D**\dir{-};"CV8"+U;"s4"+D**\dir{-};"CV9"+U;"s4"+D**\dir{-};
		"s1"+U;"Ft1"+D**\dir{-};"s2"+U;"Ft1"+D**\dir{-};"s3"+U;"Ft2"+D**\dir{-};"s4"+U;"Ft2"+D**\dir{-};
		<1em,0.5cm>*\as{\tikz[red,thick,dashed,baseline=0.9ex]\draw (0,0) rectangle (0.4cm,1.5cm);}="box",
	\endxy}\label{as:quukm=ee}
	\end{xlist}}
\end{exe}
\end{multicols}

Instead of inserting a glottal stop,
empty C-slots at clitic boundaries
are usually filled by the features of the previous vowel spreading.
Before vowel-initial enclitics this results in either /\j/ or /ɡw/,
depending on the quality of the vowel which spreads.
The way this works is illustrated in \qf{as:niij=ee} below for
the words \ve{nii} {\ra} \ve{nii\j=ee} `pole' and \ve{fee} {\ra} \ve{fee\j=ee} `wife'.

Example \qf{as:niij=ee1} shows the structure of these words before metathesis.
The initial C-slot of the foot containing the enclitic is empty.
In order to provide this foot with an onset, the feature \tsc{[+front]} of the previous
V-slot spreads, resulting in the consonant /\j/ in \qf{as:niij=ee2}.
%Metathesis would then occur as described in \srf{sec:Met ch:PhoMet} above.

\begin{multicols}{2}
\begin{exe}\ex{\label{as:niij=ee}
	\begin{xlist}
	\exa{\xy
%		<3em,5.5cm>*\as{\hp{\sub{1}}PrWd\sub{1}}="PrWd1",<5em,6.5cm>*\as{\hp{\sub{2}}PrWd\sub{2}}="PrWd2",
		<3em,4.5cm>*\as{\hp{\sub{1}}Ft\sub{1}}="Ft1",<7em,4.5cm>*\as{\hp{\sub{2}}Ft\sub{2}}="Ft2",
		<2em,3.5cm>*\as{\hp{\sub{1}}σ\sub{1}}="s1",<4em,3.5cm>*\as{\hp{\sub{2}}σ\sub{2}}="s2",<6em,3.5cm>*\as{\hp{\sub{3}}σ\sub{3}}="s3",<8em,3.5cm>*\as{\hp{\sub{4}}σ\sub{4}}="s4",
		<1em,2.5cm>*\as{C}="CV1",<2em,2.5cm>*\as{V}="CV2",<3em,2.5cm>*\as{C}="CV3",<4em,2.5cm>*\as{V}="CV4",<5em,2.5cm>*\as{C}="CV5",
		<6em,2.5cm>*\as{V}="CV6",<7em,2.5cm>*\as{C}="CV7",<8em,2.5cm>*\as{V}="CV8",<9em,2.5cm>*\as{C}="CV9",
		<1em,1.5cm>*\as{n}="cv1",<2em,1.5cm>*\as{i}="cv2",<3em,1.5cm>*\as{ }="cv3",<4em,1.5cm>*\as{i}="cv4",<5em,1.5cm>*\as{ }="cv5",
		<6em,1.5cm>*\as{e}="cv6",<7em,1.5cm>*\as{ }="cv7",<8em,1.5cm>*\as{e}="cv8",<9em,1.5cm>*\as{ }="cv9",
		<1em,1cm>*\as{f}="cv1.2",<2em,1cm>*\as{e}="cv2.2",<3em,1cm>*\as{ }="cv3.2",<4em,1cm>*\as{e}="cv4.2",<5em,1cm>*\as{ }="cv5.2",
		<6em,1cm>*\as{e}="cv6.2",<7em,1cm>*\as{ }="cv7.2",<8em,1cm>*\as{e}="cv8.2",<9em,1cm>*\as{ }="cv9.2",
		<4em,0cm>*\as{\tsc{[+fr.]}}="f","f"+U;"cv4.2"+D**\dir{-};"f"+U;"cv5.2"+D**\dir{.};"cv5.2"+D;"CV5"+D**\dir{.};"cv4"+D;"cv5"+U**\dir{.};
%		<2.5em,0cm>*\as{\hp{\sub{1}}M\sub{1}}="m1",<7em,0cm>*\as{\hp{\sub{2}}M\sub{2}}="m2",<5.5em,0cm>*\as{=}="=",
%		"m1"+U;"cv1.2"+D**\dir{-};"m1"+U;"cv2.2"+D**\dir{-};"m1"+U;"cv3.2"+D**\dir{};"m1"+U;"cv4.2"+D**\dir{-};"m1"+U;"cv5.2"+D**\dir{};
%		"m2"+U;"cv6.2"+D**\dir{-};"m2"+U;"cv8.2"+D**\dir{-};
		"cv1"+U;"CV1"+D**\dir{-};"cv2"+U;"CV2"+D**\dir{-};"cv3"+U;"CV3"+D**\dir{};"cv4"+U;"CV4"+D**\dir{-};"cv5"+U;"CV5"+D**\dir{};
		"cv6"+U;"CV6"+D**\dir{-};"cv7"+U;"CV7"+D**\dir{};"cv8"+U;"CV8"+D**\dir{-};"cv9"+U;"CV9"+D**\dir{};
		"CV1"+U;"s1"+D**\dir{-};"CV2"+U;"s1"+D**\dir{-};"CV3"+U;"s1"+D**\dir{-};"CV3"+U;"s2"+D**\dir{-};"CV4"+U;"s2"+D**\dir{-};"CV5"+U;"s2"+D**\dir{-};
		"CV5"+U;"s3"+D**\dir{-};"CV6"+U;"s3"+D**\dir{-};"CV7"+U;"s3"+D**\dir{-};"CV7"+U;"s4"+D**\dir{-};"CV8"+U;"s4"+D**\dir{-};"CV9"+U;"s4"+D**\dir{-};
		"s1"+U;"Ft1"+D**\dir{-};"s2"+U;"Ft1"+D**\dir{-};"s3"+U;"Ft2"+D**\dir{-};"s4"+U;"Ft2"+D**\dir{-};
%		"Ft1"+U;"PrWd1"+D**\dir{-};"PrWd1"+U;"PrWd2"+D**\dir{-};"Ft2"+U;"PrWd2"+D**\dir{-};
		<4.5em,1.75cm>*\as{\tikz[red,thick,dashed,baseline=0.9ex]\draw (0,0) rectangle (0.8cm,2cm);}="box",
	\endxy}\label{as:niij=ee1}
	\exa{\xy
%		<3em,5.5cm>*\as{\hp{\sub{1}}PrWd\sub{1}}="PrWd1",<5em,6.5cm>*\as{\hp{\sub{2}}PrWd\sub{2}}="PrWd2",
		<3em,4.5cm>*\as{\hp{\sub{1}}Ft\sub{1}}="Ft1",<7em,4.5cm>*\as{\hp{\sub{2}}Ft\sub{2}}="Ft2",
		<2em,3.5cm>*\as{\hp{\sub{1}}σ\sub{1}}="s1",<4em,3.5cm>*\as{\hp{\sub{2}}σ\sub{2}}="s2",<6em,3.5cm>*\as{\hp{\sub{3}}σ\sub{3}}="s3",<8em,3.5cm>*\as{\hp{\sub{4}}σ\sub{4}}="s4",
		<1em,2.5cm>*\as{C}="CV1",<2em,2.5cm>*\as{V}="CV2",<3em,2.5cm>*\as{C}="CV3",<4em,2.5cm>*\as{V}="CV4",<5em,2.5cm>*\as{C}="CV5",
		<6em,2.5cm>*\as{V}="CV6",<7em,2.5cm>*\as{C}="CV7",<8em,2.5cm>*\as{V}="CV8",<9em,2.5cm>*\as{C}="CV9",
		<1em,1.5cm>*\as{n}="cv1",<2em,1.5cm>*\as{i}="cv2",<3em,1.5cm>*\as{ }="cv3",<4em,1.5cm>*\as{i}="cv4",<5em,1.5cm>*\as{\j}="cv5",
		<6em,1.5cm>*\as{e}="cv6",<7em,1.5cm>*\as{ }="cv7",<8em,1.5cm>*\as{e}="cv8",<9em,1.5cm>*\as{ }="cv9",
		<1em,1cm>*\as{f}="cv1.2",<2em,1cm>*\as{e}="cv2.2",<3em,1cm>*\as{ }="cv3.2",<4em,1cm>*\as{e}="cv4.2",<5em,1cm>*\as{\j}="cv5.2",
		<6em,1cm>*\as{e}="cv6.2",<7em,1cm>*\as{ }="cv7.2",<8em,1cm>*\as{e}="cv8.2",<9em,1cm>*\as{ }="cv9.2",
		<4.5em,0cm>*\as{\tsc{[+fr.]}}="f","f"+U;"cv4.2"+D**\dir{-};"f"+U;"cv5.2"+D**\dir{-};
%		<2.5em,0cm>*\as{\hp{\sub{1}}M\sub{1}}="m1",<7em,0cm>*\as{\hp{\sub{2}}M\sub{2}}="m2",<5.5em,0cm>*\as{=}="=",
%		"m1"+U;"cv1.2"+D**\dir{-};"m1"+U;"cv2.2"+D**\dir{-};"m1"+U;"cv3.2"+D**\dir{};"m1"+U;"cv4.2"+D**\dir{-};"m1"+U;"cv5.2"+D**\dir{};
%		"m2"+U;"cv6.2"+D**\dir{-};"m2"+U;"cv8.2"+D**\dir{-};
		"cv1"+U;"CV1"+D**\dir{-};"cv2"+U;"CV2"+D**\dir{-};"cv3"+U;"CV3"+D**\dir{};"cv4"+U;"CV4"+D**\dir{-};"cv5"+U;"CV5"+D**\dir{-};
		"cv6"+U;"CV6"+D**\dir{-};"cv7"+U;"CV7"+D**\dir{};"cv8"+U;"CV8"+D**\dir{-};"cv9"+U;"CV9"+D**\dir{};
		"CV1"+U;"s1"+D**\dir{-};"CV2"+U;"s1"+D**\dir{-};"CV3"+U;"s1"+D**\dir{-};"CV3"+U;"s2"+D**\dir{-};"CV4"+U;"s2"+D**\dir{-};"CV5"+U;"s2"+D**\dir{-};
		"CV5"+U;"s3"+D**\dir{-};"CV6"+U;"s3"+D**\dir{-};"CV7"+U;"s3"+D**\dir{-};"CV7"+U;"s4"+D**\dir{-};"CV8"+U;"s4"+D**\dir{-};"CV9"+U;"s4"+D**\dir{-};
		"s1"+U;"Ft1"+D**\dir{-};"s2"+U;"Ft1"+D**\dir{-};"s3"+U;"Ft2"+D**\dir{-};"s4"+U;"Ft2"+D**\dir{-};
%		"Ft1"+U;"PrWd1"+D**\dir{-};"PrWd1"+U;"PrWd2"+D**\dir{-};"Ft2"+U;"PrWd2"+D**\dir{-};
		<5em,1.75cm>*\as{\tikz[red,thick,dashed,baseline=0.9ex]\draw (0,0) rectangle (0.4cm,2cm);}="box",
	\endxy}\label{as:niij=ee2}
	\end{xlist}}
\end{exe}
\end{multicols}

The process is the same when /ɡw/ is inserted.
This is shown for \ve{kfuu} {\ra} \mbox{\ve{kfuugw=ee}} `star'
and \ve{oo} {\ra} \ve{oogw=ee} `bamboo' in \qf{as:kfuugw=ee} below.
In \qf{as:kfuugw=ee1} the initial C-slot of the second foot is empty.
As a result, the features \tsc{[+back,+round]} of the previous vowel spread
producing the consonant /ɡw/ in \qf{as:kfuugw=ee2}.
The initial empty C-slot of \ve{oo} `bamboo'
is also filled by a glottal stop in \qf{as:kfuugw=ee2}.

\begin{multicols}{2}
\begin{exe}\ex{\label{as:kfuugw=ee}
	\begin{xlist}
	\ex\raisebox{\dimexpr-\totalheight+5ex\relax}{\xy
%		<3em,5.75cm>*\as{\hp{\sub{1}}PrWd\sub{1}}="PrWd1",<5em,6.75cm>*\as{\hp{\sub{2}}PrWd\sub{2}}="PrWd2",
		<3em,4.75cm>*\as{\hp{\sub{1}}Ft\sub{1}}="Ft1",<7em,4.75cm>*\as{\hp{\sub{2}}Ft\sub{2}}="Ft2",
		<2em,3.75cm>*\as{\hp{\sub{1}}σ\sub{1}}="s1",<4em,3.75cm>*\as{\hp{\sub{2}}σ\sub{2}}="s2",<6em,3.75cm>*\as{\hp{\sub{3}}σ\sub{3}}="s3",<8em,3.75cm>*\as{\hp{\sub{4}}σ\sub{4}}="s4",
		<1em,2.75cm>*\as{C}="CV1",<2em,2.75cm>*\as{V}="CV2",<3em,2.75cm>*\as{C}="CV3",<4em,2.75cm>*\as{V}="CV4",<5em,2.75cm>*\as{C}="CV5",
		<6em,2.75cm>*\as{V}="CV6",<7em,2.75cm>*\as{C}="CV7",<8em,2.75cm>*\as{V}="CV8",<9em,2.75cm>*\as{C}="CV9",
		<1em,1.75cm>*\as{kf}="cv1",<2em,1.75cm>*\as{u}="cv2",<3em,1.75cm>*\as{ }="cv3",<4em,1.75cm>*\as{u}="cv4",<5em,1.75cm>*\as{ }="cv5",
		<6em,1.75cm>*\as{e}="cv6",<7em,1.75cm>*\as{ }="cv7",<8em,1.75cm>*\as{e}="cv8",<9em,1.75cm>*\as{ }="cv9",
		<1em,1.25cm>*\as{ }="cv1.2",<2em,1.25cm>*\as{o}="cv2.2",<3em,1.25cm>*\as{ }="cv3.2",<4em,1.25cm>*\as{o}="cv4.2",<5em,1.25cm>*\as{ }="cv5.2",
		<6em,1.25cm>*\as{e}="cv6.2",<7em,1.25cm>*\as{ }="cv7.2",<8em,1.25cm>*\as{e}="cv8.2",<9em,1.25cm>*\as{ }="cv9.2",
		<4em,0cm>*\as{{$\left[\hspace{-2mm}\begin{array}{l}\textrm{\tsc{+ba.}}\\\textrm{\tsc{+ro.}}\end{array}\hspace{-2mm}\right]$}}="f","f"+U;"cv4.2"+D**\dir{-};"f"+U;"cv5.2"+D**\dir{.};"cv5.2"+D;"CV5"+D**\dir{.};"cv4"+D;"cv5"+U**\dir{.};
%		<2.5em,0cm>*\as{\hp{\sub{1}}M\sub{1}}="m1",<7em,0cm>*\as{\hp{\sub{2}}M\sub{2}}="m2",<5.5em,0cm>*\as{=}="=",
%		"m1"+U;"cv1.2"+D**\dir{-};"m1"+U;"cv2.2"+D**\dir{-};"m1"+U;"cv3.2"+D**\dir{};"m1"+U;"cv4.2"+D**\dir{-};"m1"+U;"cv5.2"+D**\dir{};
%		"m2"+U;"cv6.2"+D**\dir{-};"m2"+U;"cv8.2"+D**\dir{-};
		"cv1"+U;"CV1"+D**\dir{-};"cv2"+U;"CV2"+D**\dir{-};"cv3"+U;"CV3"+D**\dir{};"cv4"+U;"CV4"+D**\dir{-};"cv5"+U;"CV5"+D**\dir{};
		"cv6"+U;"CV6"+D**\dir{-};"cv7"+U;"CV7"+D**\dir{};"cv8"+U;"CV8"+D**\dir{-};"cv9"+U;"CV9"+D**\dir{};
		"CV1"+U;"s1"+D**\dir{-};"CV2"+U;"s1"+D**\dir{-};"CV3"+U;"s1"+D**\dir{-};"CV3"+U;"s2"+D**\dir{-};"CV4"+U;"s2"+D**\dir{-};"CV5"+U;"s2"+D**\dir{-};
		"CV5"+U;"s3"+D**\dir{-};"CV6"+U;"s3"+D**\dir{-};"CV7"+U;"s3"+D**\dir{-};"CV7"+U;"s4"+D**\dir{-};"CV8"+U;"s4"+D**\dir{-};"CV9"+U;"s4"+D**\dir{-};
		"s1"+U;"Ft1"+D**\dir{-};"s2"+U;"Ft1"+D**\dir{-};"s3"+U;"Ft2"+D**\dir{-};"s4"+U;"Ft2"+D**\dir{-};
%		"Ft1"+U;"PrWd1"+D**\dir{-};"PrWd1"+U;"PrWd2"+D**\dir{-};"Ft2"+U;"PrWd2"+D**\dir{-};
		<4.5em,2cm>*\as{\tikz[red,thick,dashed,baseline=0.9ex]\draw (0,0) rectangle (0.8cm,2cm);}="box",
	\endxy}\label{as:kfuugw=ee1}
	\ex\raisebox{\dimexpr-\totalheight+5ex\relax}{\xy
%		<3em,5.75cm>*\as{\hp{\sub{1}}PrWd\sub{1}}="PrWd1",<5em,6.75cm>*\as{\hp{\sub{2}}PrWd\sub{2}}="PrWd2",
		<3em,4.75cm>*\as{\hp{\sub{1}}Ft\sub{1}}="Ft1",<7em,4.75cm>*\as{\hp{\sub{2}}Ft\sub{2}}="Ft2",
		<2em,3.75cm>*\as{\hp{\sub{1}}σ\sub{1}}="s1",<4em,3.75cm>*\as{\hp{\sub{2}}σ\sub{2}}="s2",<6em,3.75cm>*\as{\hp{\sub{3}}σ\sub{3}}="s3",<8em,3.75cm>*\as{\hp{\sub{4}}σ\sub{4}}="s4",
		<1em,2.75cm>*\as{C}="CV1",<2em,2.75cm>*\as{V}="CV2",<3em,2.75cm>*\as{C}="CV3",<4em,2.75cm>*\as{V}="CV4",<5em,2.75cm>*\as{C}="CV5",
		<6em,2.75cm>*\as{V}="CV6",<7em,2.75cm>*\as{C}="CV7",<8em,2.75cm>*\as{V}="CV8",<9em,2.75cm>*\as{C}="CV9",
		<1em,1.75cm>*\as{kf}="cv1",<2em,1.75cm>*\as{u}="cv2",<3em,1.75cm>*\as{ }="cv3",<4em,1.75cm>*\as{u}="cv4",<5em,1.75cm>*\as{ɡw}="cv5",
		<6em,1.75cm>*\as{e}="cv6",<7em,1.75cm>*\as{ }="cv7",<8em,1.75cm>*\as{e}="cv8",<9em,1.75cm>*\as{ }="cv9",
		<1em,1.25cm>*\as{ʔ}="cv1.2",<2em,1.25cm>*\as{o}="cv2.2",<3em,1.25cm>*\as{ }="cv3.2",<4em,1.25cm>*\as{o}="cv4.2",<5em,1.25cm>*\as{ɡw}="cv5.2",
		<6em,1.25cm>*\as{e}="cv6.2",<7em,1.25cm>*\as{ }="cv7.2",<8em,1.25cm>*\as{e}="cv8.2",<9em,1.25cm>*\as{ }="cv9.2",
		<4.5em,0cm>*\as{{$\left[\hspace{-2mm}\begin{array}{l}\textrm{\tsc{+ba.}}\\\textrm{\tsc{+ro.}}\end{array}\hspace{-2mm}\right]$}}="f","f"+U;"cv4.2"+D**\dir{-};"f"+U;"cv5.2"+D**\dir{-};
%		<2.5em,0cm>*\as{\hp{\sub{1}}M\sub{1}}="m1",<7em,0cm>*\as{\hp{\sub{2}}M\sub{2}}="m2",<5.5em,0cm>*\as{=}="=",
%		"m1"+U;"cv1.2"+D**\dir{-};"m1"+U;"cv2.2"+D**\dir{-};"m1"+U;"cv3.2"+D**\dir{};"m1"+U;"cv4.2"+D**\dir{-};"m1"+U;"cv5.2"+D**\dir{};
%		"m2"+U;"cv6.2"+D**\dir{-};"m2"+U;"cv8.2"+D**\dir{-};
		"cv1"+U;"CV1"+D**\dir{-};"cv2"+U;"CV2"+D**\dir{-};"cv3"+U;"CV3"+D**\dir{};"cv4"+U;"CV4"+D**\dir{-};"cv5"+U;"CV5"+D**\dir{-};
		"cv6"+U;"CV6"+D**\dir{-};"cv7"+U;"CV7"+D**\dir{};"cv8"+U;"CV8"+D**\dir{-};"cv9"+U;"CV9"+D**\dir{};
		"CV1"+U;"s1"+D**\dir{-};"CV2"+U;"s1"+D**\dir{-};"CV3"+U;"s1"+D**\dir{-};"CV3"+U;"s2"+D**\dir{-};"CV4"+U;"s2"+D**\dir{-};"CV5"+U;"s2"+D**\dir{-};
		"CV5"+U;"s3"+D**\dir{-};"CV6"+U;"s3"+D**\dir{-};"CV7"+U;"s3"+D**\dir{-};"CV7"+U;"s4"+D**\dir{-};"CV8"+U;"s4"+D**\dir{-};"CV9"+U;"s4"+D**\dir{-};
		"s1"+U;"Ft1"+D**\dir{-};"s2"+U;"Ft1"+D**\dir{-};"s3"+U;"Ft2"+D**\dir{-};"s4"+U;"Ft2"+D**\dir{-};
%		"Ft1"+U;"PrWd1"+D**\dir{-};"PrWd1"+U;"PrWd2"+D**\dir{-};"Ft2"+U;"PrWd2"+D**\dir{-};
		<5em,2cm>*\as{\tikz[red,thick,dashed,baseline=0.9ex]\draw (0,0) rectangle (0.525cm,2cm);}="box",
	\endxy}\label{as:kfuugw=ee2}
	\end{xlist}}
\end{exe}
\end{multicols}

The newly-inserted consonant in \qf{as:niij=ee} and \qf{as:kfuugw=ee}
is shared between the prosodic word containing the host
and the prosodic word containing the enclitic,
as illustrated in \qf{as:niij=ee,kfuugw=ee1} below.
This is resolved by metathesis, which yields the structure
in \qf{as:niij=ee,kfuugw=ee2} with a crisp edge after the internal prosodic word.

\begin{multicols}{2}
\begin{exe}\ex{\label{as:niij=ee,kfuugw=ee}
	\begin{xlist}
	\exa{\xy
		<3em,6.5cm>*\as{\hp{\sub{1}}PrWd\sub{1}}="PrWd1",<4.5em,7.5cm>*\as{\hp{\sub{2}}PrWd\sub{2}}="PrWd2",
		<3em,5.5cm>*\as{\hp{\sub{1}}Ft\sub{1}}="Ft1",<7em,5.5cm>*\as{\hp{\sub{2}}Ft\sub{2}}="Ft2",
		<2em,4.5cm>*\as{\hp{\sub{1}}σ\sub{1}}="s1",<4em,4.5cm>*\as{\hp{\sub{2}}σ\sub{2}}="s2",<6em,4.5cm>*\as{\hp{\sub{3}}σ\sub{3}}="s3",<8em,4.5cm>*\as{\hp{\sub{4}}σ\sub{4}}="s4",
		<1em,3.5cm>*\as{C}="CV1",<2em,3.5cm>*\as{V}="CV2",<3em,3.5cm>*\as{C}="CV3",<4em,3.5cm>*\as{V}="CV4",<5em,3.5cm>*\as{C}="CV5",
		<6em,3.5cm>*\as{V}="CV6",<7em,3.5cm>*\as{C}="CV7",<8em,3.5cm>*\as{V}="CV8",<9em,3.5cm>*\as{C}="CV9",
		<1em,2.5cm>*\as{n}="cv1",<2em,2.5cm>*\as{i}="cv2",<3em,2.5cm>*\as{ }="cv3",<4em,2.5cm>*\as{i}="cv4",<5em,2.5cm>*\as{\j}="cv5",
		<6em,2.5cm>*\as{e}="cv6",<7em,2.5cm>*\as{ }="cv7",<8em,2.5cm>*\as{e}="cv8",<9em,2.5cm>*\as{ }="cv9",
		<1em,2cm>*\as{f}="cv1.3",<2em,2cm>*\as{e}="cv2.3",<3em,2cm>*\as{ }="cv3.3",<4em,2cm>*\as{e}="cv4.3",<5em,2cm>*\as{\j}="cv5.3",
		<6em,2cm>*\as{e}="cv6.3",<7em,2cm>*\as{ }="cv7.3",<8em,2cm>*\as{e}="cv8.3",<9em,2cm>*\as{ }="cv9.3",
		<1em,1.5cm>*\as{kf}="cv1.4",<2em,1.5cm>*\as{u}="cv2.4",<3em,1.5cm>*\as{ }="cv3.4",<4em,1.5cm>*\as{u}="cv4.4",<5em,1.5cm>*\as{ɡw}="cv5.4",
		<6em,1.5cm>*\as{e}="cv6.4",<7em,1.5cm>*\as{ }="cv7.4",<8em,1.5cm>*\as{e}="cv8.4",<9em,1.5cm>*\as{ }="cv9.4",
		<1em,1cm>*\as{ʔ}="cv1.2",<2em,1cm>*\as{o}="cv2.2",<3em,1cm>*\as{ }="cv3.2",<4em,1cm>*\as{o}="cv4.2",<5em,1cm>*\as{ɡw}="cv5.2",
		<6em,1cm>*\as{e}="cv6.2",<7em,1cm>*\as{ }="cv7.2",<8em,1cm>*\as{e}="cv8.2",<9em,1cm>*\as{ }="cv9.2",
		<2.5em,0cm>*\as{\hp{\sub{1}}M\sub{1}}="m1",<7em,0cm>*\as{\hp{\sub{2}}M\sub{2}}="m2",<5em,0cm>*\as{=}="=",
		"m1"+U;"cv1.2"+D**\dir{-};"m1"+U;"cv2.2"+D**\dir{-};"m1"+U;"cv3.2"+D**\dir{ };"m1"+U;"cv4.2"+D**\dir{-};
		"m2"+U;"cv6.2"+D**\dir{-};"m2"+U;"cv8.2"+D**\dir{-};
		"cv1"+U;"CV1"+D**\dir{-};"cv2"+U;"CV2"+D**\dir{-};"cv3"+U;"CV3"+D**\dir{};"cv4"+U;"CV4"+D**\dir{-};"cv5"+U;"CV5"+D**\dir{-};
		"cv6"+U;"CV6"+D**\dir{-};"cv7"+U;"CV7"+D**\dir{};"cv8"+U;"CV8"+D**\dir{-};"cv9"+U;"CV9"+D**\dir{};
		"CV1"+U;"s1"+D**\dir{-};"CV2"+U;"s1"+D**\dir{-};"CV3"+U;"s1"+D**\dir{-};"CV3"+U;"s2"+D**\dir{-};"CV4"+U;"s2"+D**\dir{-};"CV5"+U;"s2"+D**\dir{-};
		"CV5"+U;"s3"+D**\dir{-};"CV6"+U;"s3"+D**\dir{-};"CV7"+U;"s3"+D**\dir{-};"CV7"+U;"s4"+D**\dir{-};"CV8"+U;"s4"+D**\dir{-};"CV9"+U;"s4"+D**\dir{-};
		"s1"+U;"Ft1"+D**\dir{-};"s2"+U;"Ft1"+D**\dir{-};"s3"+U;"Ft2"+D**\dir{-};"s4"+U;"Ft2"+D**\dir{-};
		"Ft1"+U;"PrWd1"+D**\dir{-};"PrWd1"+U;"PrWd2"+D**\dir{-};"Ft2"+U;"PrWd2"+D**\dir{-};
		<5em,2.25cm>*\as{\tikz[red,thick,dashed,baseline=0.9ex]\draw (0,0) rectangle (0.525cm,3cm);}="box",
	\endxy}\label{as:niij=ee,kfuugw=ee1}
	\exa{\xy
		<2.5em,6.5cm>*\as{\hp{\sub{1}}PrWd\sub{1}}="PrWd1",<4.5em,7.5cm>*\as{\hp{\sub{2}}PrWd\sub{2}}="PrWd2",
		<2.5em,5.5cm>*\as{\hp{\sub{\tsc{m}}}Ft\sub{\tsc{m}}}="Ft1",<7em,5.5cm>*\as{\hp{\sub{2}}Ft\sub{2}}="Ft2",
<1.5em,4.5cm>*\as{\hp{\sub{1}}σ\sub{1}}="s1",<3.5em,4.5cm>*\as{\hp{\sub{2}}σ\sub{2}}="s2",<6em,4.5cm>*\as{\hp{\sub{3}}σ\sub{3}}="s3",<8em,4.5cm>*\as{\hp{\sub{4}}σ\sub{4}}="s4",
		<1em,3.5cm>*\as{C}="CV1",<2em,3.5cm>*\as{V}="CV2",<3em,3.5cm>*\as{V}="CV3",<4em,3.5cm>*\as{C}="CV4",<5em,3.5cm>*\as{C}="CV5",
		<6em,3.5cm>*\as{V}="CV6",<7em,3.5cm>*\as{C}="CV7",<8em,3.5cm>*\as{V}="CV8",<9em,3.5cm>*\as{C}="CV9",
		<1em,2.5cm>*\as{n}="cv1",<2em,2.5cm>*\as{i}="cv2",<3em,2.5cm>*\as{i}="cv3",<4em,2.5cm>*\as{ }="cv4",<5em,2.5cm>*\as{\j}="cv5",
		<6em,2.5cm>*\as{e}="cv6",<7em,2.5cm>*\as{ }="cv7",<8em,2.5cm>*\as{e}="cv8",<9em,2.5cm>*\as{ }="cv9",
		<1em,2cm>*\as{f}="cv1.3",<2em,2cm>*\as{e}="cv2.3",<3em,2cm>*\as{e}="cv3.3",<4em,2cm>*\as{ }="cv4.3",<5em,2cm>*\as{\j}="cv5.3",
		<6em,2cm>*\as{e}="cv6.3",<7em,2cm>*\as{ }="cv7.3",<8em,2cm>*\as{e}="cv8.3",<9em,2cm>*\as{ }="cv9.3",
		<1em,1.5cm>*\as{kf}="cv1.4",<2em,1.5cm>*\as{u}="cv2.4",<3em,1.5cm>*\as{u}="cv3.4",<4em,1.5cm>*\as{ }="cv4.4",<5em,1.5cm>*\as{ɡw}="cv5.4",
		<6em,1.5cm>*\as{e}="cv6.4",<7em,1.5cm>*\as{ }="cv7.4",<8em,1.5cm>*\as{e}="cv8.4",<9em,1.5cm>*\as{ }="cv9.4",
		<1em,1cm>*\as{ʔ}="cv1.2",<2em,1cm>*\as{o}="cv2.2",<3em,1cm>*\as{o}="cv3.2",<4em,1cm>*\as{ }="cv4.2",<5em,1cm>*\as{ɡw}="cv5.2",
		<6em,1cm>*\as{e}="cv6.2",<7em,1cm>*\as{ }="cv7.2",<8em,1cm>*\as{e}="cv8.2",<9em,1cm>*\as{ }="cv9.2",
		<2em,0cm>*\as{\hp{\sub{1}}M\sub{1}}="m1",<7em,0cm>*\as{\hp{\sub{2}}M\sub{2}}="m2",<5em,0cm>*\as{=}="=",
		"m1"+U;"cv1.2"+D**\dir{-};"m1"+U;"cv2.2"+D**\dir{-};"m1"+U;"cv3.2"+D**\dir{-};
		"m2"+U;"cv6.2"+D**\dir{-};"m2"+U;"cv8.2"+D**\dir{-};
		"cv1"+U;"CV1"+D**\dir{-};"cv2"+U;"CV2"+D**\dir{-};"cv3"+U;"CV3"+D**\dir{-};"cv4"+U;"CV4"+D**\dir{};"cv5"+U;"CV5"+D**\dir{-};
		"cv6"+U;"CV6"+D**\dir{-};"cv7"+U;"CV7"+D**\dir{};"cv8"+U;"CV8"+D**\dir{-};"cv9"+U;"CV9"+D**\dir{};
		"CV1"+U;"s1"+D**\dir{-};"CV2"+U;"s1"+D**\dir{-};"CV3"+U;"s2"+D**\dir{-};"CV4"+U;"s2"+D**\dir{-};
		"CV5"+U;"s3"+D**\dir{-};"CV6"+U;"s3"+D**\dir{-};"CV7"+U;"s3"+D**\dir{-};"CV7"+U;"s4"+D**\dir{-};"CV8"+U;"s4"+D**\dir{-};"CV9"+U;"s4"+D**\dir{-};
		"s1"+U;"Ft1"+D**\dir{-};"s2"+U;"Ft1"+D**\dir{-};"s3"+U;"Ft2"+D**\dir{-};"s4"+U;"Ft2"+D**\dir{-};
		"Ft1"+U;"PrWd1"+D**\dir{-};"PrWd1"+U;"PrWd2"+D**\dir{-};"Ft2"+U;"PrWd2"+D**\dir{-};
		<4.4em,4cm>*\as{\tikz[red,thick,dashed,baseline=0.9ex]\draw (0,0) -- (0,6.5cm);}="line",
	\endxy}\label{as:niij=ee,kfuugw=ee2}
	\end{xlist}}
\end{exe}
\end{multicols}

\newpage
For words which contain a surface vowel sequence,
the C-slot affected by metathesis is empty.
As a result, metathesis has no discernible effect on the surface structure of such words.
However, in \srf{sec:VowAss ch:PhoMet} I show that we can still detect metathesis
for words in which the surface vowel sequence involves vowels of different qualities.

The reason vowel features spread at clitic
boundaries rather than simply inserting a glottal stop is not immediately clear.
One possibility is that the glottal stop is only inserted word initially.
However, this does not account for glottal stop insertion in examples
such as \ve{n-ita} `see' {\ra} \ve{na-\tbr{ʔ}ita-b} `show' (\srf{sec:GloStoInsVocPre}),
in which case the inserted glottal stop is not word-initial.
\label{WhyNotGlottal?}

Another possible reason could be due to the differing morphological structures.
Glottal stop insertion happens at affix boundaries
while vowel features spread at clitic boundaries.
However, this runs counter to the general principle
that the phonological structures of Amarasi are `blind'
to the morphological structures.

Another possible reason why vowel features spread at clitic
boundaries could be that the C-slot which is filled by /\j/ or /ɡw/
is the final C-slot of the previous foot.
This is the analysis I currently favour,
though it seems somewhat counter intuitive
given that the whole reason vowel features spread is to
provide the \emph{following} clitic with an onset.

Given the presence of glottal stop insertion
word initially in forms such as \ve{ukum} {\ra} [ˈ\tbr{ʔ}ʊkʊm] `cuscus'
and foot initially in words such as \ve{n-ita} `see' {\ra} \ve{na-\tbr{ʔ}ita-b} `show',
there does not currently seem to be a good phonological
reason why such glottal stop insertion does not also happen at clitic boundaries.
It may simply be a fact of Amarasi that at clitic
boundaries vowel features spread to produce /\j/ and /ɡw/.

\subsection{Location of the inserted consonant}\label{sec:LocInsCon}
Amarasi consonant insertion can be analysed as a result of
vowel features spreading into an adjacent empty C-slot.
However, this empty C-slot could logically originate with the foot containing the clitic host,
or the foot containing the enclitic.
There are at least three reasons for analysing this empty C-slot
as originating with the foot of the clitic host rather than the enclitic:

\begin{exe}
	\exi{i.}{It simplifies the analysis of consonant-final words.}
	\exi{ii.}{There are varieties of Meto in which consonant insertion
						occurs with no enclitic present (\srf{sec:WorFinConIns}).}
	\exi{iii.}{It can provide a reason vowel features spread to produce
						an onset rather than glottal stop insertion.}
\end{exe}

Regarding the first point above, if the empty C-slot originated with the enclitic, forms such as
\ve{muʔit}+\ve{=ee} `the animal' {\ra} \ve{muiʔt=ee} would be underlyingly
\ve{muʔit}+\ve{=Cee} and we would probably expect something like \ve{\tcb{*}muʔittee},
(cf. gemination in Seri, discussed by \citealt[631]{mast83}).
Additional rules would then have to be introduced to avoid such forms.

While the empty C-slot probably originates with the foot of the clitic host,
the consonant inserted in this C-slot
is not a member of the same morpheme as the clitic host.
Instead, it is an epenthetic segment which does not
belong to either the previous or following morpheme,
much like epenthetic glottal stops (\srf{sec:GloStoIns}).

Nonetheless, the syllabification of Amarasi words (\srf{sec:Syl})
means that when a vowel-initial enclitic is attached to a stem
the final C-slot of this stem is ambisyllabic,
occurring as the coda of the initial foot
and as the onset of the foot containing the enclitic.
As a result, it is a member of more than one prosodic word.
This dual membership is the reason why metathesis
is triggered before vowel-initial enclitics in Amarasi (\srf{sec:Met ch:PhoMet}).
Metathesis rearranges the phonotactic structure
of the host and enclitic such that after
metathesis this C-slot is the onset to only one prosodic word.

	%\subsection{No insertion after U\=/form suffix \it{-ʔ}}
%An additional complication in the derivation of the M\=/form
%is only revealed when the data from vowel-initial enclitics is considered.
%As discussed in \srf{sec:Met ch:PhoMet} above, any final consonant
%of the host is retained after attachment of a vowel-initial enclitic.
%
%However, there are at least two words in my data which have a final
%glottal stop in the U\=/form which does not surface
%after a vowel-initial enclitic has been attached.
%These words are \ve{uaba-\tbr{ʔ}} `speech, language'
%and \ve{mabe-\tbr{ʔ}} `evening, time',
%neither of which retains this glottal stop before an enclitic:
%\ve{uaba-ʔ} `speech, language' + \ve{=ees} `one' {\ra} \ve{uab=ees} `one speech'
%and \ve{mabe-ʔ} `evening, time' + \ve{=ees} `one' {\ra} \ve{maeb=ees} `one evening'.
%I analyse the final glottal stop as a suffix redundantly marking the U\=/form.
%
%It must be emphasised here that the vast majority of final glottal stops
%are members of the root and are not U\=/form suffixes.
%There is thus a difference between examples such as
%\ve{ma\tbr{be}-\tbr{ʔ}} {\ra} \ve{ma\tbr{eb}=ees} `evening'
%in which the glottal stop is a suffix and does not occur in the M\=/form
%and examples such as \ve{sba\tbr{keʔ}} {\ra} \ve{sba\tbr{ekʔ}=ees} `branch'
%in which the glottal stop \emph{is} part of the root and
%\emph{does} occur in the M\=/form before enclitics.
%
%While the U\=/form suffix on \ve{mabe-ʔ} `evening'
%does not surface when a vowel-initial enclitic
%is attached, neither does consonant insertion occur
%as would be expected for vowel-final roots.
%This indicates that the enclitic is attached
%to the U\=/form of the stem with the glottal
%stop suffix blocking consonant insertion.
%After metathesis the suffix is removed.
%This analysis is somewhat ad-hoc.
%However, there does not currently seem to be an alternate analysis
%which accounts for the data.
\section{Vowel assimilation}\label{sec:VowAss ch:PhoMet}
When a vowel-initial enclitic attaches to a stem which ends in a vowel sequence
in which the vowels are of a different quality,
the final vowel conditions insertion of /\j/ or /ɡw/,
and then assimilates to the quality of the previous vowel.
Examples are given in \qf{ex:VaVb+=V->VaVaCb=V} below.

\begin{exe}
	\ex{V{\sA}V{\sB}+=V {\ra} V{\sA}V{\sA}C{\sB}+=V}\label{ex:VaVb+=V->VaVaCb=V}
		\sn{\gw\begin{tabular}{rlllll}
			\ve{kre\tbr{i}} &+&\ve{=ee}&{\ra}&\ve{kre\tbr{e\j}=ee}	& `the church/week' \\
			\ve{fa\tbr{i}} 	&+&\ve{=ee}&{\ra}&\ve{fa\tbr{a\j}=ee}		& `the night' \\
			\ve{n-ro\tbr{i}} 	&+&\ve{=ee}&{\ra}&\ve{n-ro\tbr{o\j}=ee}	& `carries it' \\
			\ve{pu\tbr{i}} 	&+&\ve{=ee}&{\ra}&\ve{pu\tbr{u\j}=ee}		& `the quail' \\
			\ve{ma\tbr{e}}	&+&\ve{=ee}&{\ra}&\ve{ma\tbr{a\j}=ee}		& `the taro' \\
			\ve{o\tbr{e}} 	&+&\ve{=ee}&{\ra}&\ve{o\tbr{o\j}=ee}		& `the water' \\
			\ve{ki\tbr{u}}	&+&\ve{=ee}&{\ra}&\ve{ki\tbr{igw}=ee}		& `the tamarind' \\
			\ve{n-ke\tbr{u}}	&+&\ve{=ee}&{\ra}&\ve{n-ke\tbr{egw}=ee}	& `shaves it' \\
			\ve{ha\tbr{u}}	&+&\ve{=ee}&{\ra}&\ve{ha\tbr{agw}=ee}		& `the wood/tree' \\
			\ve{me\tbr{o}}	&+&\ve{=ee}&{\ra}&\ve{me\tbr{egw}=ee}		& `the cat' \\
			\ve{a\tbr{o}} 	&+&\ve{=ee}&{\ra}&\ve{a\tbr{agw}=ee}		& `the slaked lime' \\
		\end{tabular}}
\end{exe}

When a vowel-initial enclitic attaches to a stem which ends in CV{\#},
the final vowel conditions insertion of /\j/ or /ɡw/,
metathesis takes place, and the vowel which conditioned consonant
insertion assimilates to the previous vowel.
%Examples are given in \qf{ex:VaCVbr+=V->VaVaCb=V} below.

\begin{exe}
	\ex{V{\sA}CV{\sB}+=V {\ra} V{\sA}V{\sA}CC{\sB}=V}\label{ex:VaCVbr+=V->VaVaCb=V}
		\sn{\gw\begin{tabular}{rlllll}
			\ve{kbit\tbr{i}}&+&\ve{=ee}&\ra&\ve{kbiit\tbr{\j}=ee}				& `the scorpion' \\
			\ve{kren\tbr{i}}&+&\ve{=ee}&\ra&\ve{kre\tbr{e}n\tbr{\j}=ee}	& `the ring' \\
			\ve{faf\tbr{i}}	&+&\ve{=ee}&\ra&\ve{fa\tbr{a}f\tbr{\j}=ee}	& `the pig' \\
			\ve{on\tbr{i}} 	&+&\ve{=ee}&\ra&\ve{o\tbr{o}n\tbr{\j}=ee}		& `the bee; the sugar' \\
			\ve{uk\tbr{i}} 	&+&\ve{=ee}&\ra&\ve{u\tbr{u}k\tbr{\j}=ee}		& `the banana' \\
			\ve{kep\tbr{e}} &+&\ve{=ee}&\ra&\ve{keep\tbr{\j}=ee}				& `the tick (parasite)' \\
			\ve{bar\tbr{e}} &+&\ve{=ee}&\ra&\ve{ba\tbr{a}r\tbr{\j}=ee}	& `the place' \\
			\ve{nop\tbr{e}}	&+&\ve{=ee}&\ra&\ve{no\tbr{o}p\tbr{\j}=ee}	& `the cloud' \\
			\ve{bik\tbr{u}} &+&\ve{=ee}&\ra&\ve{bi\tbr{i}k\tbr{gw}=ee}	& `the curse' \\
			\ve{tef\tbr{u}} &+&\ve{=ee}&\ra&\ve{te\tbr{e}f\tbr{gw}=ee}	& `the sugar-cane' \\
			\ve{fat\tbr{u}} &+&\ve{=ee}&\ra&\ve{fa\tbr{a}t\tbr{gw}=ee}	& `the stone' \\
			\ve{nop\tbr{u}} &+&\ve{=ee}&\ra&\ve{no\tbr{o}p\tbr{gw}=ee}	& `the grave' \\
			\ve{hut\tbr{u}} &+&\ve{=ee}&\ra&\ve{huut\tbr{gw}=ee}				& `louse' \\
			\ve{nef\tbr{o}} &+&\ve{=ee}&\ra&\ve{ne\tbr{e}f\tbr{gw}=ee}	& `the lake' \\
			\ve{knaf\tbr{o}}&+&\ve{=ee}&\ra&\ve{kna\tbr{a}f\tbr{gw}=ee}	& `the mouse' \\
			\ve{kor\tbr{o}} &+&\ve{=ee}&\ra&\ve{koor\tbr{gw}=ee}				& `the bird' \\
		\end{tabular}}
\end{exe}

This vowel assimilation can be analysed as an automatic
result of metathesis after consonant insertion.
This is illustrated in \qf{as:faafj=ee} below
for \ve{fafi} {\ra} \ve{faaf\j=ee} `pig',
and \ve{nope} {\ra} \ve{noop\j=ee} `cloud'.
In \qf{as:faafj=ee1} the second foot begins with an empty C-slot.
Because feet require an onset, the features
(abbreviated as \tsc{+fr.} = front)
of the previous vowel spread,
producing the obstruent /\j/ in \qf{as:faafj=ee2}.

\begin{multicols}{2}
\begin{exe}\ex{\label{as:faafj=ee}
	\begin{xlist}
	\exa{\xy
%		<3em,5.5cm>*\as{\hp{\sub{1}}PrWd\sub{1}}="PrWd1",<5em,6.5cm>*\as{\hp{\sub{2}}PrWd\sub{2}}="PrWd2",
		<3em,4.5cm>*\as{\hp{\sub{1}}Ft\sub{1}}="Ft1",<7em,4.5cm>*\as{\hp{\sub{2}}Ft\sub{2}}="Ft2",
		<2em,3.5cm>*\as{\hp{\sub{1}}σ\sub{1}}="s1",<4em,3.5cm>*\as{\hp{\sub{2}}σ\sub{2}}="s2",<6em,3.5cm>*\as{\hp{\sub{3}}σ\sub{3}}="s3",<8em,3.5cm>*\as{\hp{\sub{4}}σ\sub{4}}="s4",
		<1em,2.5cm>*\as{C}="CV1",<2em,2.5cm>*\as{V}="CV2",<3em,2.5cm>*\as{C}="CV3",<4em,2.5cm>*\as{V}="CV4",<5em,2.5cm>*\as{C}="CV5",
		<6em,2.5cm>*\as{V}="CV6",<7em,2.5cm>*\as{C}="CV7",<8em,2.5cm>*\as{V}="CV8",<9em,2.5cm>*\as{C}="CV9",
		<1em,1.5cm>*\as{f}="cv1",<2em,1.5cm>*\as{a}="cv2",<3em,1.5cm>*\as{f}="cv3",<4em,1.5cm>*\as{i}="cv4",<5em,1.5cm>*\as{ }="cv5",
		<6em,1.5cm>*\as{e}="cv6",<7em,1.5cm>*\as{ }="cv7",<8em,1.5cm>*\as{e}="cv8",<9em,1.5cm>*\as{ }="cv9",
		<1em,1cm>*\as{n}="cv1.2",<2em,1cm>*\as{o}="cv2.2",<3em,1cm>*\as{p}="cv3.2",<4em,1cm>*\as{e}="cv4.2",<5em,1cm>*\as{ }="cv5.2",
		<6em,1cm>*\as{e}="cv6.2",<7em,1cm>*\as{ }="cv7.2",<8em,1cm>*\as{e}="cv8.2",<9em,1cm>*\as{ }="cv9.2",
		<4em,0cm>*\as{\tsc{[+fr.]}}="f","f"+U;"cv4.2"+D**\dir{-};"f"+U;"cv5.2"+D**\dir{.};"cv5.2"+D;"CV5"+D**\dir{.};"cv4"+D;"cv5"+U**\dir{.};
%		<2.5em,0cm>*\as{\hp{\sub{1}}M\sub{1}}="m1",<7em,0cm>*\as{\hp{\sub{2}}M\sub{2}}="m2",<5.5em,0cm>*\as{=}="=",
%		"m1"+U;"cv1.2"+D**\dir{-};"m1"+U;"cv2.2"+D**\dir{-};"m1"+U;"cv3.2"+D**\dir{};"m1"+U;"cv4.2"+D**\dir{-};"m1"+U;"cv5.2"+D**\dir{};
%		"m2"+U;"cv6.2"+D**\dir{-};"m2"+U;"cv8.2"+D**\dir{-};
		"cv1"+U;"CV1"+D**\dir{-};"cv2"+U;"CV2"+D**\dir{-};"cv3"+U;"CV3"+D**\dir{-};"cv4"+U;"CV4"+D**\dir{-};"cv5"+U;"CV5"+D**\dir{};
		"cv6"+U;"CV6"+D**\dir{-};"cv7"+U;"CV7"+D**\dir{};"cv8"+U;"CV8"+D**\dir{-};"cv9"+U;"CV9"+D**\dir{};
		"CV1"+U;"s1"+D**\dir{-};"CV2"+U;"s1"+D**\dir{-};"CV3"+U;"s1"+D**\dir{-};"CV3"+U;"s2"+D**\dir{-};"CV4"+U;"s2"+D**\dir{-};"CV5"+U;"s2"+D**\dir{-};
		"CV5"+U;"s3"+D**\dir{-};"CV6"+U;"s3"+D**\dir{-};"CV7"+U;"s3"+D**\dir{-};"CV7"+U;"s4"+D**\dir{-};"CV8"+U;"s4"+D**\dir{-};"CV9"+U;"s4"+D**\dir{-};
		"s1"+U;"Ft1"+D**\dir{-};"s2"+U;"Ft1"+D**\dir{-};"s3"+U;"Ft2"+D**\dir{-};"s4"+U;"Ft2"+D**\dir{-};
%		"Ft1"+U;"PrWd1"+D**\dir{-};"PrWd1"+U;"PrWd2"+D**\dir{-};"Ft2"+U;"PrWd2"+D**\dir{-};
		<4.5em,1.75cm>*\as{\tikz[red,thick,dashed,baseline=0.9ex]\draw (0,0) rectangle (0.8cm,2cm);}="box",
	\endxy}\label{as:faafj=ee1}
	\exa{\xy
%		<3em,5.5cm>*\as{\hp{\sub{1}}PrWd\sub{1}}="PrWd1",<5em,6.5cm>*\as{\hp{\sub{2}}PrWd\sub{2}}="PrWd2",
		<3em,4.5cm>*\as{\hp{\sub{1}}Ft\sub{1}}="Ft1",<7em,4.5cm>*\as{\hp{\sub{2}}Ft\sub{2}}="Ft2",
		<2em,3.5cm>*\as{\hp{\sub{1}}σ\sub{1}}="s1",<4em,3.5cm>*\as{\hp{\sub{2}}σ\sub{2}}="s2",<6em,3.5cm>*\as{\hp{\sub{3}}σ\sub{3}}="s3",<8em,3.5cm>*\as{\hp{\sub{4}}σ\sub{4}}="s4",
		<1em,2.5cm>*\as{C}="CV1",<2em,2.5cm>*\as{V}="CV2",<3em,2.5cm>*\as{C}="CV3",<4em,2.5cm>*\as{V}="CV4",<5em,2.5cm>*\as{C}="CV5",
		<6em,2.5cm>*\as{V}="CV6",<7em,2.5cm>*\as{C}="CV7",<8em,2.5cm>*\as{V}="CV8",<9em,2.5cm>*\as{C}="CV9",
		<1em,1.5cm>*\as{f}="cv1",<2em,1.5cm>*\as{a}="cv2",<3em,1.5cm>*\as{f}="cv3",<4em,1.5cm>*\as{i}="cv4",<5em,1.5cm>*\as{\j}="cv5",
		<6em,1.5cm>*\as{e}="cv6",<7em,1.5cm>*\as{ }="cv7",<8em,1.5cm>*\as{e}="cv8",<9em,1.5cm>*\as{ }="cv9",
		<1em,1cm>*\as{n}="cv1.2",<2em,1cm>*\as{o}="cv2.2",<3em,1cm>*\as{p}="cv3.2",<4em,1cm>*\as{e}="cv4.2",<5em,1cm>*\as{\j}="cv5.2",
		<6em,1cm>*\as{e}="cv6.2",<7em,1cm>*\as{ }="cv7.2",<8em,1cm>*\as{e}="cv8.2",<9em,1cm>*\as{ }="cv9.2",
		<4.5em,0cm>*\as{\tsc{[+fr.]}}="f","f"+U;"cv4.2"+D**\dir{-};"f"+U;"cv5.2"+D**\dir{-};
%		<2.5em,0cm>*\as{\hp{\sub{1}}M\sub{1}}="m1",<7em,0cm>*\as{\hp{\sub{2}}M\sub{2}}="m2",<5.5em,0cm>*\as{=}="=",
%		"m1"+U;"cv1.2"+D**\dir{-};"m1"+U;"cv2.2"+D**\dir{-};"m1"+U;"cv3.2"+D**\dir{};"m1"+U;"cv4.2"+D**\dir{-};"m1"+U;"cv5.2"+D**\dir{};
%		"m2"+U;"cv6.2"+D**\dir{-};"m2"+U;"cv8.2"+D**\dir{-};
		"cv1"+U;"CV1"+D**\dir{-};"cv2"+U;"CV2"+D**\dir{-};"cv3"+U;"CV3"+D**\dir{-};"cv4"+U;"CV4"+D**\dir{-};"cv5"+U;"CV5"+D**\dir{-};
		"cv6"+U;"CV6"+D**\dir{-};"cv7"+U;"CV7"+D**\dir{};"cv8"+U;"CV8"+D**\dir{-};"cv9"+U;"CV9"+D**\dir{};
		"CV1"+U;"s1"+D**\dir{-};"CV2"+U;"s1"+D**\dir{-};"CV3"+U;"s1"+D**\dir{-};"CV3"+U;"s2"+D**\dir{-};"CV4"+U;"s2"+D**\dir{-};"CV5"+U;"s2"+D**\dir{-};
		"CV5"+U;"s3"+D**\dir{-};"CV6"+U;"s3"+D**\dir{-};"CV7"+U;"s3"+D**\dir{-};"CV7"+U;"s4"+D**\dir{-};"CV8"+U;"s4"+D**\dir{-};"CV9"+U;"s4"+D**\dir{-};
		"s1"+U;"Ft1"+D**\dir{-};"s2"+U;"Ft1"+D**\dir{-};"s3"+U;"Ft2"+D**\dir{-};"s4"+U;"Ft2"+D**\dir{-};
%		"Ft1"+U;"PrWd1"+D**\dir{-};"PrWd1"+U;"PrWd2"+D**\dir{-};"Ft2"+U;"PrWd2"+D**\dir{-};
		<5em,1.75cm>*\as{\tikz[red,thick,dashed,baseline=0.9ex]\draw (0,0) rectangle (0.4cm,2cm);}="box",
	\endxy}\label{as:faafj=ee2}
	\end{xlist}}
\end{exe}
\end{multicols}

The third C-slot is shared between the external and internal prosodic
words as shown in (\ref{as:faafj=ee}d).
Because fuzzy borders are not allowed between prosodic words,
metathesis is triggered, yielding the form in (\ref{as:faafj=ee}e) with
a crisp edge between the clitic host and enclitic.

\begin{multicols}{2}
\begin{exe}\exr{as:faafj=ee}{
	\begin{xlist}
		\exi{d.}\exia{\xy
		<3em,5.5cm>*\as{\hp{\sub{1}}PrWd\sub{1}}="PrWd1",<4.5em,6.5cm>*\as{\hp{\sub{2}}PrWd\sub{2}}="PrWd2",
		<3em,4.5cm>*\as{\hp{\sub{1}}Ft\sub{1}}="Ft1",<7em,4.5cm>*\as{\hp{\sub{2}}Ft\sub{2}}="Ft2",
		<2em,3.5cm>*\as{\hp{\sub{1}}σ\sub{1}}="s1",<4em,3.5cm>*\as{\hp{\sub{2}}σ\sub{2}}="s2",<6em,3.5cm>*\as{\hp{\sub{3}}σ\sub{3}}="s3",<8em,3.5cm>*\as{\hp{\sub{4}}σ\sub{4}}="s4",
		<1em,2.5cm>*\as{C}="CV1",<2em,2.5cm>*\as{V}="CV2",<3em,2.5cm>*\as{C}="CV3",<4em,2.5cm>*\as{V}="CV4",<5em,2.5cm>*\as{C}="CV5",
		<6em,2.5cm>*\as{V}="CV6",<7em,2.5cm>*\as{C}="CV7",<8em,2.5cm>*\as{V}="CV8",<9em,2.5cm>*\as{C}="CV9",
		<1em,1.5cm>*\as{f}="cv1",<2em,1.5cm>*\as{a}="cv2",<3em,1.5cm>*\as{f}="cv3",<4em,1.5cm>*\as{i}="cv4",<5em,1.5cm>*\as{\j}="cv5",
		<6em,1.5cm>*\as{e}="cv6",<7em,1.5cm>*\as{ }="cv7",<8em,1.5cm>*\as{e}="cv8",<9em,1.5cm>*\as{ }="cv9",
		<1em,1cm>*\as{n}="cv1.2",<2em,1cm>*\as{o}="cv2.2",<3em,1cm>*\as{p}="cv3.2",<4em,1cm>*\as{e}="cv4.2",<5em,1cm>*\as{\j}="cv5.2",
		<6em,1cm>*\as{e}="cv6.2",<7em,1cm>*\as{ }="cv7.2",<8em,1cm>*\as{e}="cv8.2",<9em,1cm>*\as{ }="cv9.2",
		<2.5em,0cm>*\as{\hp{\sub{1}}M\sub{1}}="m1",<7em,0cm>*\as{\hp{\sub{2}}M\sub{2}}="m2",<5em,0cm>*\as{=}="=",
		"m1"+U;"cv1.2"+D**\dir{-};"m1"+U;"cv2.2"+D**\dir{-};"m1"+U;"cv3.2"+D**\dir{-};"m1"+U;"cv4.2"+D**\dir{-};
		"m2"+U;"cv6.2"+D**\dir{-};"m2"+U;"cv8.2"+D**\dir{-};
		"cv1"+U;"CV1"+D**\dir{-};"cv2"+U;"CV2"+D**\dir{-};"cv3"+U;"CV3"+D**\dir{-};"cv4"+U;"CV4"+D**\dir{-};"cv5"+U;"CV5"+D**\dir{-};
		"cv6"+U;"CV6"+D**\dir{-};"cv7"+U;"CV7"+D**\dir{};"cv8"+U;"CV8"+D**\dir{-};"cv9"+U;"CV9"+D**\dir{};
		"CV1"+U;"s1"+D**\dir{-};"CV2"+U;"s1"+D**\dir{-};"CV3"+U;"s1"+D**\dir{-};"CV3"+U;"s2"+D**\dir{-};"CV4"+U;"s2"+D**\dir{-};"CV5"+U;"s2"+D**\dir{-};
		"CV5"+U;"s3"+D**\dir{-};"CV6"+U;"s3"+D**\dir{-};"CV7"+U;"s3"+D**\dir{-};"CV7"+U;"s4"+D**\dir{-};"CV8"+U;"s4"+D**\dir{-};"CV9"+U;"s4"+D**\dir{-};
		"s1"+U;"Ft1"+D**\dir{-};"s2"+U;"Ft1"+D**\dir{-};"s3"+U;"Ft2"+D**\dir{-};"s4"+U;"Ft2"+D**\dir{-};
		"Ft1"+U;"PrWd1"+D**\dir{-};"PrWd1"+U;"PrWd2"+D**\dir{-};"Ft2"+U;"PrWd2"+D**\dir{-};
		<5em,1.75cm>*\as{\tikz[red,thick,dashed,baseline=0.9ex]\draw (0,0) rectangle (0.4cm,2cm);}="box",
	\endxy}
	\exi{e.}\exia{\xy
		<2.5em,5.5cm>*\as{\hp{\sub{1}}PrWd\sub{1}}="PrWd1",<4.5em,6.5cm>*\as{\hp{\sub{2}}PrWd\sub{2}}="PrWd2",
		<2.5em,4.5cm>*\as{\hp{\sub{\tsc{m}}}Ft\sub{\tsc{m}}}="Ft1",<7em,4.5cm>*\as{\hp{\sub{2}}Ft\sub{2}}="Ft2",
<1.5em,3.5cm>*\as{\hp{\sub{1}}σ\sub{1}}="s1",<3.5em,3.5cm>*\as{\hp{\sub{2}}σ\sub{2}}="s2",<6em,3.5cm>*\as{\hp{\sub{3}}σ\sub{3}}="s3",<8em,3.5cm>*\as{\hp{\sub{4}}σ\sub{4}}="s4",
		<1em,2.5cm>*\as{C}="CV1",<2em,2.5cm>*\as{V}="CV2",<3em,2.5cm>*\as{V}="CV3",<4em,2.5cm>*\as{C}="CV4",<5em,2.5cm>*\as{C}="CV5",
		<6em,2.5cm>*\as{V}="CV6",<7em,2.5cm>*\as{C}="CV7",<8em,2.5cm>*\as{V}="CV8",<9em,2.5cm>*\as{C}="CV9",
		<1em,1.5cm>*\as{f}="cv1",<2em,1.5cm>*\as{a}="cv2",<3em,1.5cm>*\as{i}="cv3",<4em,1.5cm>*\as{f}="cv4",<5em,1.5cm>*\as{\j}="cv5",
		<6em,1.5cm>*\as{e}="cv6",<7em,1.5cm>*\as{ }="cv7",<8em,1.5cm>*\as{e}="cv8",<9em,1.5cm>*\as{ }="cv9",
		<1em,1cm>*\as{n}="cv1.2",<2em,1cm>*\as{o}="cv2.2",<3em,1cm>*\as{e}="cv3.2",<4em,1cm>*\as{p}="cv4.2",<5em,1cm>*\as{\j}="cv5.2",
		<6em,1cm>*\as{e}="cv6.2",<7em,1cm>*\as{ }="cv7.2",<8em,1cm>*\as{e}="cv8.2",<9em,1cm>*\as{ }="cv9.2",
		<2.5em,0cm>*\as{\hp{\sub{1}}M\sub{1}}="m1",<7em,0cm>*\as{\hp{\sub{2}}M\sub{2}}="m2",<5em,0cm>*\as{=}="=",
		"m1"+U;"cv1.2"+D**\dir{-};"m1"+U;"cv2.2"+D**\dir{-};"m1"+U;"cv3.2"+D**\dir{-};"m1"+U;"cv4.2"+D**\dir{-};
		"m2"+U;"cv6.2"+D**\dir{-};"m2"+U;"cv8.2"+D**\dir{-};
		"cv1"+U;"CV1"+D**\dir{-};"cv2"+U;"CV2"+D**\dir{-};"cv3"+U;"CV3"+D**\dir{-};"cv4"+U;"CV4"+D**\dir{-};"cv5"+U;"CV5"+D**\dir{-};
		"cv6"+U;"CV6"+D**\dir{-};"cv7"+U;"CV7"+D**\dir{};"cv8"+U;"CV8"+D**\dir{-};"cv9"+U;"CV9"+D**\dir{};
		"CV1"+U;"s1"+D**\dir{-};"CV2"+U;"s1"+D**\dir{-};"CV3"+U;"s2"+D**\dir{-};"CV4"+U;"s2"+D**\dir{-};
		"CV5"+U;"s3"+D**\dir{-};"CV6"+U;"s3"+D**\dir{-};"CV7"+U;"s3"+D**\dir{-};"CV7"+U;"s4"+D**\dir{-};"CV8"+U;"s4"+D**\dir{-};"CV9"+U;"s4"+D**\dir{-};
		"s1"+U;"Ft1"+D**\dir{-};"s2"+U;"Ft1"+D**\dir{-};"s3"+U;"Ft2"+D**\dir{-};"s4"+U;"Ft2"+D**\dir{-};
		"Ft1"+U;"PrWd1"+D**\dir{-};"PrWd1"+U;"PrWd2"+D**\dir{-};"Ft2"+U;"PrWd2"+D**\dir{-};
		<4.5em,3.25cm>*\as{\tikz[red,thick,dashed,baseline=0.9ex]\draw (0,0) -- (0,5cm);}="line",
	\endxy}
	\end{xlist}}
\end{exe}
\end{multicols}

Metathesis results in the features of the final vowel of the clitic host
shared across an intervening consonant.
This results in `lines crossing', as shown in (\ref{as:faafj=ee}f),
with the intervening consonantal features represented by \tsc{[+c.]}.
A prohibition against association lines crossing is one of the fundamental
principles of autosegmental phonology \citep[48]{go76}.
Thus, the vowel features de-link
yielding an empty V-slot in (\ref{as:faafj=ee}g) into which the previous vowel spreads,
yielding the final output with a double vowel in (\ref{as:faafj=ee}h).

\begin{multicols}{2}
\begin{exe}\exr{as:faafj=ee}{
	\begin{xlist}
	\exi{f.}\exia{\xy
		<2.5em,4.5cm>*\as{\hp{\sub{\tsc{m}}}Ft\sub{\tsc{m}}}="Ft1",<7em,4.5cm>*\as{\hp{\sub{2}}Ft\sub{2}}="Ft2",
<1.5em,3.5cm>*\as{\hp{\sub{1}}σ\sub{1}}="s1",<3.5em,3.5cm>*\as{\hp{\sub{2}}σ\sub{2}}="s2",<6em,3.5cm>*\as{\hp{\sub{3}}σ\sub{3}}="s3",<8em,3.5cm>*\as{\hp{\sub{4}}σ\sub{4}}="s4",
		<1em,2.5cm>*\as{C}="CV1",<2em,2.5cm>*\as{V}="CV2",<3em,2.5cm>*\as{V}="CV3",<4em,2.5cm>*\as{C}="CV4",<5em,2.5cm>*\as{C}="CV5",
		<6em,2.5cm>*\as{V}="CV6",<7em,2.5cm>*\as{C}="CV7",<8em,2.5cm>*\as{V}="CV8",<9em,2.5cm>*\as{C}="CV9",
		<1em,1.5cm>*\as{f}="cv1",<2em,1.5cm>*\as{a}="cv2",<3em,1.5cm>*\as{\xc{\,i\,}}="cv3",<4em,1.5cm>*\as{f}="cv4",<5em,1.5cm>*\as{\j}="cv5",
		<6em,1.5cm>*\as{e}="cv6",<7em,1.5cm>*\as{ }="cv7",<8em,1.5cm>*\as{e}="cv8",<9em,1.5cm>*\as{ }="cv9",
		<1em,1cm>*\as{n}="cv1.2",<2em,1cm>*\as{o}="cv2.2",<3em,1cm>*\as{\xc{\,e\,}}="cv3.2",<4em,1cm>*\as{p}="cv4.2",<5em,1cm>*\as{\j}="cv5.2",
		<6em,1cm>*\as{e}="cv6.2",<7em,1cm>*\as{ }="cv7.2",<8em,1cm>*\as{e}="cv8.2",<9em,1cm>*\as{ }="cv9.2",
		<4em,0cm>*\as{\tsc{[+fr.]}}="f",{\ar@{-}|-(.425)*@{|} |-{\hole} |-(.575)*@{|} "f"+U;"cv3.2"+D};"f"+U;"cv5.2"+D**\dir{-};
		<1.9em,0cm>*\as{\tsc{[+c.]}}="f2","f2"+U;"cv4.2"+D**\dir{-};
		"cv1"+U;"CV1"+D**\dir{-};"cv2"+U;"CV2"+D**\dir{-};"cv4"+U;"CV4"+D**\dir{-};"cv5"+U;"CV5"+D**\dir{-};{\ar@{-}|-(.425)*@{|} |-{\hole} |-(.575)*@{|} "cv3"+U;"CV3"+D};
		"cv6"+U;"CV6"+D**\dir{-};"cv7"+U;"CV7"+D**\dir{};"cv8"+U;"CV8"+D**\dir{-};"cv9"+U;"CV9"+D**\dir{};
		"CV1"+U;"s1"+D**\dir{-};"CV2"+U;"s1"+D**\dir{-};"CV3"+U;"s1"+D**\dir{};"CV3"+U;"s2"+D**\dir{-};"CV4"+U;"s2"+D**\dir{-};"CV5"+U;"s2"+D**\dir{};
		"CV5"+U;"s3"+D**\dir{-};"CV6"+U;"s3"+D**\dir{-};"CV7"+U;"s3"+D**\dir{-};"CV7"+U;"s4"+D**\dir{-};"CV8"+U;"s4"+D**\dir{-};"CV9"+U;"s4"+D**\dir{-};
		"s1"+U;"Ft1"+D**\dir{-};"s2"+U;"Ft1"+D**\dir{-};"s3"+U;"Ft2"+D**\dir{-};"s4"+U;"Ft2"+D**\dir{-};
		<4em,1cm>*\as{\tikz[red,thick,dashed,baseline=0.9ex]\draw (0,0) rectangle (1.2cm,2.5cm);}="box",
	\endxy}
	\exi{g.}\exia{\xy
		<2.5em,4.5cm>*\as{\hp{\sub{\tsc{m}}}Ft\sub{\tsc{m}}}="Ft1",<7em,4.5cm>*\as{\hp{\sub{2}}Ft\sub{2}}="Ft2",
<1.5em,3.5cm>*\as{\hp{\sub{1}}σ\sub{1}}="s1",<3.5em,3.5cm>*\as{\hp{\sub{2}}σ\sub{2}}="s2",<6em,3.5cm>*\as{\hp{\sub{3}}σ\sub{3}}="s3",<8em,3.5cm>*\as{\hp{\sub{4}}σ\sub{4}}="s4",
		<1em,2.5cm>*\as{C}="CV1",<2em,2.5cm>*\as{V}="CV2",<3em,2.5cm>*\as{V}="CV3",<4em,2.5cm>*\as{C}="CV4",<5em,2.5cm>*\as{C}="CV5",
		<6em,2.5cm>*\as{V}="CV6",<7em,2.5cm>*\as{C}="CV7",<8em,2.5cm>*\as{V}="CV8",<9em,2.5cm>*\as{C}="CV9",
		<1em,1.5cm>*\as{f}="cv1",<2em,1.5cm>*\as{a}="cv2",<3em,1.5cm>*\as{ }="cv3",<4em,1.5cm>*\as{f}="cv4",<5em,1.5cm>*\as{\j}="cv5",
		<6em,1.5cm>*\as{e}="cv6",<7em,1.5cm>*\as{ }="cv7",<8em,1.5cm>*\as{e}="cv8",<9em,1.5cm>*\as{ }="cv9",
		<1em,1cm>*\as{n}="cv1.2",<2em,1cm>*\as{o}="cv2.2",<3em,1cm>*\as{ }="cv3.2",<4em,1cm>*\as{p}="cv4.2",<5em,1cm>*\as{\j}="cv5.2",
		<6em,1cm>*\as{e}="cv6.2",<7em,1cm>*\as{ }="cv7.2",<8em,1cm>*\as{e}="cv8.2",<9em,1cm>*\as{ }="cv9.2",
		<5em,0cm>*\as{\tsc{[+fr.]}}="f","f"+U;"cv5.2"+D**\dir{-};
		"cv1"+U;"CV1"+D**\dir{-};"cv2"+U;"CV2"+D**\dir{-};"cv2"+U;"CV3"+D**\dir{.};"cv4"+U;"CV4"+D**\dir{-};"cv5"+U;"CV5"+D**\dir{-};"cv2.2"+U;"cv3"+U**\dir{.};
		"cv6"+U;"CV6"+D**\dir{-};"cv7"+U;"CV7"+D**\dir{};"cv8"+U;"CV8"+D**\dir{-};"cv9"+U;"CV9"+D**\dir{};
		"CV1"+U;"s1"+D**\dir{-};"CV2"+U;"s1"+D**\dir{-};"CV3"+U;"s1"+D**\dir{};"CV3"+U;"s2"+D**\dir{-};"CV4"+U;"s2"+D**\dir{-};"CV5"+U;"s2"+D**\dir{};
		"CV5"+U;"s3"+D**\dir{-};"CV6"+U;"s3"+D**\dir{-};"CV7"+U;"s3"+D**\dir{-};"CV7"+U;"s4"+D**\dir{-};"CV8"+U;"s4"+D**\dir{-};"CV9"+U;"s4"+D**\dir{-};
		"s1"+U;"Ft1"+D**\dir{-};"s2"+U;"Ft1"+D**\dir{-};"s3"+U;"Ft2"+D**\dir{-};"s4"+U;"Ft2"+D**\dir{-};
		<3em,1.75cm>*\as{\tikz[red,thick,dashed,baseline=0.9ex]\draw (0,0) rectangle (0.4cm,2cm);}="box",
	\endxy}
	\end{xlist}}
\end{exe}
\end{multicols}

\begin{exe}\sn{
	\begin{xlist}
	\exi{h.}\exia{\xy
		<2.5em,5.5cm>*\as{\hp{\sub{1}}PrWd\sub{1}}="PrWd1",<4.5em,6.5cm>*\as{\hp{\sub{2}}PrWd\sub{2}}="PrWd2",
		<2.5em,4.5cm>*\as{\hp{\sub{\tsc{m}}}Ft\sub{\tsc{m}}}="Ft1",<7em,4.5cm>*\as{\hp{\sub{2}}Ft\sub{2}}="Ft2",
<1.5em,3.5cm>*\as{\hp{\sub{1}}σ\sub{1}}="s1",<3.5em,3.5cm>*\as{\hp{\sub{2}}σ\sub{2}}="s2",<6em,3.5cm>*\as{\hp{\sub{3}}σ\sub{3}}="s3",<8em,3.5cm>*\as{\hp{\sub{4}}σ\sub{4}}="s4",
		<1em,2.5cm>*\as{C}="CV1",<2em,2.5cm>*\as{V}="CV2",<3em,2.5cm>*\as{V}="CV3",<4em,2.5cm>*\as{C}="CV4",<5em,2.5cm>*\as{C}="CV5",
		<6em,2.5cm>*\as{V}="CV6",<7em,2.5cm>*\as{C}="CV7",<8em,2.5cm>*\as{V}="CV8",<9em,2.5cm>*\as{C}="CV9",
		<1em,1.5cm>*\as{f}="cv1",<2.5em,1.5cm>*\as{a}="cv2",<3em,1.5cm>*\as{ }="cv3",<4em,1.5cm>*\as{f}="cv4",<5em,1.5cm>*\as{\j}="cv5",
		<6em,1.5cm>*\as{e}="cv6",<7em,1.5cm>*\as{ }="cv7",<8em,1.5cm>*\as{e}="cv8",<9em,1.5cm>*\as{ }="cv9",
		<1em,1cm>*\as{n}="cv1.2",<2.5em,1cm>*\as{o}="cv2.2",<3em,1cm>*\as{ }="cv3.2",<4em,1cm>*\as{p}="cv4.2",<5em,1cm>*\as{\j}="cv5.2",
		<6em,1cm>*\as{e}="cv6.2",<7em,1cm>*\as{ }="cv7.2",<8em,1cm>*\as{e}="cv8.2",<9em,1cm>*\as{ }="cv9.2",
		<2.5em,0cm>*\as{\hp{\sub{1}}M\sub{1}}="m1",<7em,0cm>*\as{\hp{\sub{2}}M\sub{2}}="m2",<5em,0cm>*\as{=}="=",
		"m1"+U;"cv1.2"+D**\dir{-};"m1"+U;"cv2.2"+D**\dir{-};"m1"+U;"cv4.2"+D**\dir{-};
		"m2"+U;"cv6.2"+D**\dir{-};"m2"+U;"cv8.2"+D**\dir{-};
		"cv1"+U;"CV1"+D**\dir{-};"cv2"+U;"CV2"+D**\dir{-};"cv2"+U;"CV3"+D**\dir{-};"cv4"+U;"CV4"+D**\dir{-};"cv5"+U;"CV5"+D**\dir{-};
		"cv6"+U;"CV6"+D**\dir{-};"cv7"+U;"CV7"+D**\dir{};"cv8"+U;"CV8"+D**\dir{-};"cv9"+U;"CV9"+D**\dir{};
		"CV1"+U;"s1"+D**\dir{-};"CV2"+U;"s1"+D**\dir{-};"CV3"+U;"s1"+D**\dir{};"CV3"+U;"s2"+D**\dir{-};"CV4"+U;"s2"+D**\dir{-};"CV5"+U;"s2"+D**\dir{};
		"CV5"+U;"s3"+D**\dir{-};"CV6"+U;"s3"+D**\dir{-};"CV7"+U;"s3"+D**\dir{-};"CV7"+U;"s4"+D**\dir{-};"CV8"+U;"s4"+D**\dir{-};"CV9"+U;"s4"+D**\dir{-};
		"s1"+U;"Ft1"+D**\dir{-};"s2"+U;"Ft1"+D**\dir{-};"s3"+U;"Ft2"+D**\dir{-};"s4"+U;"Ft2"+D**\dir{-};
		"s1"+U;"Ft1"+D**\dir{-};"s2"+U;"Ft1"+D**\dir{-};"s3"+U;"Ft2"+D**\dir{-};"s4"+U;"Ft2"+D**\dir{-};
		"Ft1"+U;"PrWd1"+D**\dir{-};"PrWd1"+U;"PrWd2"+D**\dir{-};"Ft2"+U;"PrWd2"+D**\dir{-};
		%<4em,1cm>*\as{\tikz[red,thick,dashed,baseline=0.9ex]\draw (0,0) rectangle (1.2cm,2.5cm);}="box",
	\endxy}
	\end{xlist}}
\end{exe}

The reason vowel features de-link rather than consonant
features is probably due to vowel assimilation/deletion
being preferred over consonant assimilation in Amarasi.
Vowel assimilation is attested in at least three other parts of the grammar of Amarasi
while consonant assimilation is almost unattested.\footnote{
		The only example of consonant assimilation in Amarasi 
		is phonetic assimilation of /n/ to the place of
		any following non-labial obstruent (\srf{sec:Con}).}
Other examples of vowel assimilation in Amarasi include the following:

\begin{exe}
	\exi{i.}{Assimilation of /a/ after metathesis;
							e.g. \ve{nim\tbr{a}} {\ra} \ve{ni\tbr{i}m} `five' (\srf{sec:AssOfA})}
	\exi{ii.}{Height assimilation of mid vowels after metathesis;
							e.g. \ve{um\tbr{e}} {\ra} \ve{u\tbr{i}m} `house' (\srf{sec:MidVowAss})}
	\exi{iii.}{Phonetic partial height assimilation of mid vowels before high vowels;
							e.g. \ve{koʔu} `big' {\ra} [ˈk\tbr{o}ʔʊ] *[ˈkɔʔʊ] (\srf{sec:Vow}).}
\end{exe}

By making use of empty C-slots,
the analysis of vowel assimilation before vowel-initial enclitics
as being triggered by an intervening consonant can be extended
to words which end in a vowel sequence.
This is illustrated in \qf{as:faaj=ee} below
for the words \ve{fai} {\ra} \ve{faa\j=ee} `night',
and \ve{mae} {\ra} \ve{maa\j=ee} `taro'.

In \qf{as:faaj=ee1} the second foot begins with an empty C-slot.
Because feet require an initial consonant,
the features of the previous vowel spread,
producing the obstruent /\j/ in \qf{as:faaj=ee2}.

\begin{multicols}{2}
\begin{exe}\ex{\label{as:faaj=ee}
	\begin{xlist}
	\exa{\xy
%		<3em,5.5cm>*\as{\hp{\sub{1}}PrWd\sub{1}}="PrWd1",<5em,6.5cm>*\as{\hp{\sub{2}}PrWd\sub{2}}="PrWd2",
		<3em,4.5cm>*\as{\hp{\sub{1}}Ft\sub{1}}="Ft1",<7em,4.5cm>*\as{\hp{\sub{2}}Ft\sub{2}}="Ft2",
		<2em,3.5cm>*\as{\hp{\sub{1}}σ\sub{1}}="s1",<4em,3.5cm>*\as{\hp{\sub{2}}σ\sub{2}}="s2",<6em,3.5cm>*\as{\hp{\sub{3}}σ\sub{3}}="s3",<8em,3.5cm>*\as{\hp{\sub{4}}σ\sub{4}}="s4",
		<1em,2.5cm>*\as{C}="CV1",<2em,2.5cm>*\as{V}="CV2",<3em,2.5cm>*\as{C}="CV3",<4em,2.5cm>*\as{V}="CV4",<5em,2.5cm>*\as{C}="CV5",
		<6em,2.5cm>*\as{V}="CV6",<7em,2.5cm>*\as{C}="CV7",<8em,2.5cm>*\as{V}="CV8",<9em,2.5cm>*\as{C}="CV9",
		<1em,1.5cm>*\as{f}="cv1",<2em,1.5cm>*\as{a}="cv2",<3em,1.5cm>*\as{ }="cv3",<4em,1.5cm>*\as{i}="cv4",<5em,1.5cm>*\as{ }="cv5",
		<6em,1.5cm>*\as{e}="cv6",<7em,1.5cm>*\as{ }="cv7",<8em,1.5cm>*\as{e}="cv8",<9em,1.5cm>*\as{ }="cv9",
		<1em,1cm>*\as{m}="cv1.2",<2em,1cm>*\as{a}="cv2.2",<3em,1cm>*\as{ }="cv3.2",<4em,1cm>*\as{e}="cv4.2",<5em,1cm>*\as{ }="cv5.2",
		<6em,1cm>*\as{e}="cv6.2",<7em,1cm>*\as{ }="cv7.2",<8em,1cm>*\as{e}="cv8.2",<9em,1cm>*\as{ }="cv9.2",
		<4em,0cm>*\as{\tsc{[+fr.]}}="f","f"+U;"cv4.2"+D**\dir{-};"f"+U;"cv5.2"+D**\dir{.};"cv5.2"+D;"CV5"+D**\dir{.};"cv4"+D;"cv5"+U**\dir{.};
%		<2.5em,0cm>*\as{\hp{\sub{1}}M\sub{1}}="m1",<7em,0cm>*\as{\hp{\sub{2}}M\sub{2}}="m2",<5.5em,0cm>*\as{=}="=",
%		"m1"+U;"cv1.2"+D**\dir{-};"m1"+U;"cv2.2"+D**\dir{-};"m1"+U;"cv3.2"+D**\dir{};"m1"+U;"cv4.2"+D**\dir{-};"m1"+U;"cv5.2"+D**\dir{};
%		"m2"+U;"cv6.2"+D**\dir{-};"m2"+U;"cv8.2"+D**\dir{-};
		"cv1"+U;"CV1"+D**\dir{-};"cv2"+U;"CV2"+D**\dir{-};"cv3"+U;"CV3"+D**\dir{};"cv4"+U;"CV4"+D**\dir{-};"cv5"+U;"CV5"+D**\dir{};
		"cv6"+U;"CV6"+D**\dir{-};"cv7"+U;"CV7"+D**\dir{};"cv8"+U;"CV8"+D**\dir{-};"cv9"+U;"CV9"+D**\dir{};
		"CV1"+U;"s1"+D**\dir{-};"CV2"+U;"s1"+D**\dir{-};"CV3"+U;"s1"+D**\dir{-};"CV3"+U;"s2"+D**\dir{-};"CV4"+U;"s2"+D**\dir{-};"CV5"+U;"s2"+D**\dir{-};
		"CV5"+U;"s3"+D**\dir{-};"CV6"+U;"s3"+D**\dir{-};"CV7"+U;"s3"+D**\dir{-};"CV7"+U;"s4"+D**\dir{-};"CV8"+U;"s4"+D**\dir{-};"CV9"+U;"s4"+D**\dir{-};
		"s1"+U;"Ft1"+D**\dir{-};"s2"+U;"Ft1"+D**\dir{-};"s3"+U;"Ft2"+D**\dir{-};"s4"+U;"Ft2"+D**\dir{-};
%		"Ft1"+U;"PrWd1"+D**\dir{-};"PrWd1"+U;"PrWd2"+D**\dir{-};"Ft2"+U;"PrWd2"+D**\dir{-};
		<4.5em,1.75cm>*\as{\tikz[red,thick,dashed,baseline=0.9ex]\draw (0,0) rectangle (0.8cm,2cm);}="box",
	\endxy}\label{as:faaj=ee1}
	\exa{\xy
%		<3em,5.5cm>*\as{\hp{\sub{1}}PrWd\sub{1}}="PrWd1",<5em,6.5cm>*\as{\hp{\sub{2}}PrWd\sub{2}}="PrWd2",
		<3em,4.5cm>*\as{\hp{\sub{1}}Ft\sub{1}}="Ft1",<7em,4.5cm>*\as{\hp{\sub{2}}Ft\sub{2}}="Ft2",
		<2em,3.5cm>*\as{\hp{\sub{1}}σ\sub{1}}="s1",<4em,3.5cm>*\as{\hp{\sub{2}}σ\sub{2}}="s2",<6em,3.5cm>*\as{\hp{\sub{3}}σ\sub{3}}="s3",<8em,3.5cm>*\as{\hp{\sub{4}}σ\sub{4}}="s4",
		<1em,2.5cm>*\as{C}="CV1",<2em,2.5cm>*\as{V}="CV2",<3em,2.5cm>*\as{C}="CV3",<4em,2.5cm>*\as{V}="CV4",<5em,2.5cm>*\as{C}="CV5",
		<6em,2.5cm>*\as{V}="CV6",<7em,2.5cm>*\as{C}="CV7",<8em,2.5cm>*\as{V}="CV8",<9em,2.5cm>*\as{C}="CV9",
		<1em,1.5cm>*\as{f}="cv1",<2em,1.5cm>*\as{a}="cv2",<3em,1.5cm>*\as{ }="cv3",<4em,1.5cm>*\as{i}="cv4",<5em,1.5cm>*\as{\j}="cv5",
		<6em,1.5cm>*\as{e}="cv6",<7em,1.5cm>*\as{ }="cv7",<8em,1.5cm>*\as{e}="cv8",<9em,1.5cm>*\as{ }="cv9",
		<1em,1cm>*\as{m}="cv1.2",<2em,1cm>*\as{a}="cv2.2",<3em,1cm>*\as{ }="cv3.2",<4em,1cm>*\as{e}="cv4.2",<5em,1cm>*\as{\j}="cv5.2",
		<6em,1cm>*\as{e}="cv6.2",<7em,1cm>*\as{ }="cv7.2",<8em,1cm>*\as{e}="cv8.2",<9em,1cm>*\as{ }="cv9.2",
		<4.5em,0cm>*\as{\tsc{[+fr.]}}="f","f"+U;"cv4.2"+D**\dir{-};"f"+U;"cv5.2"+D**\dir{-};
%		<2.5em,0cm>*\as{\hp{\sub{1}}M\sub{1}}="m1",<7em,0cm>*\as{\hp{\sub{2}}M\sub{2}}="m2",<5.5em,0cm>*\as{=}="=",
%		"m1"+U;"cv1.2"+D**\dir{-};"m1"+U;"cv2.2"+D**\dir{-};"m1"+U;"cv3.2"+D**\dir{};"m1"+U;"cv4.2"+D**\dir{-};"m1"+U;"cv5.2"+D**\dir{};
%		"m2"+U;"cv6.2"+D**\dir{-};"m2"+U;"cv8.2"+D**\dir{-};
		"cv1"+U;"CV1"+D**\dir{-};"cv2"+U;"CV2"+D**\dir{-};"cv3"+U;"CV3"+D**\dir{};"cv4"+U;"CV4"+D**\dir{-};"cv5"+U;"CV5"+D**\dir{-};
		"cv6"+U;"CV6"+D**\dir{-};"cv7"+U;"CV7"+D**\dir{};"cv8"+U;"CV8"+D**\dir{-};"cv9"+U;"CV9"+D**\dir{};
		"CV1"+U;"s1"+D**\dir{-};"CV2"+U;"s1"+D**\dir{-};"CV3"+U;"s1"+D**\dir{-};"CV3"+U;"s2"+D**\dir{-};"CV4"+U;"s2"+D**\dir{-};"CV5"+U;"s2"+D**\dir{-};
		"CV5"+U;"s3"+D**\dir{-};"CV6"+U;"s3"+D**\dir{-};"CV7"+U;"s3"+D**\dir{-};"CV7"+U;"s4"+D**\dir{-};"CV8"+U;"s4"+D**\dir{-};"CV9"+U;"s4"+D**\dir{-};
		"s1"+U;"Ft1"+D**\dir{-};"s2"+U;"Ft1"+D**\dir{-};"s3"+U;"Ft2"+D**\dir{-};"s4"+U;"Ft2"+D**\dir{-};
%		"Ft1"+U;"PrWd1"+D**\dir{-};"PrWd1"+U;"PrWd2"+D**\dir{-};"Ft2"+U;"PrWd2"+D**\dir{-};
		<5em,1.75cm>*\as{\tikz[red,thick,dashed,baseline=0.9ex]\draw (0,0) rectangle (0.4cm,2cm);}="box",
	\endxy}\label{as:faaj=ee2}
	\end{xlist}}
\end{exe}
\end{multicols}

The recently filled C-slot is shared between the external and internal prosodic words,
as shown in (\ref{as:faaj=ee}d).
Because fuzzy borders are not allowed at prosodic word boundaries,
metathesis is triggered, resulting in the form in (\ref{as:faaj=ee}e),
in which there is a crisp edge between the host and enclitic.

\begin{multicols}{2}
\begin{exe}\exr{as:faaj=ee}{
	\begin{xlist}
		\exi{d.}\exia{\xy
		<3em,5.5cm>*\as{\hp{\sub{1}}PrWd\sub{1}}="PrWd1",<4.5em,6.5cm>*\as{\hp{\sub{2}}PrWd\sub{2}}="PrWd2",
		<3em,4.5cm>*\as{\hp{\sub{1}}Ft\sub{1}}="Ft1",<7em,4.5cm>*\as{\hp{\sub{2}}Ft\sub{2}}="Ft2",
		<2em,3.5cm>*\as{\hp{\sub{1}}σ\sub{1}}="s1",<4em,3.5cm>*\as{\hp{\sub{2}}σ\sub{2}}="s2",<6em,3.5cm>*\as{\hp{\sub{3}}σ\sub{3}}="s3",<8em,3.5cm>*\as{\hp{\sub{4}}σ\sub{4}}="s4",
		<1em,2.5cm>*\as{C}="CV1",<2em,2.5cm>*\as{V}="CV2",<3em,2.5cm>*\as{C}="CV3",<4em,2.5cm>*\as{V}="CV4",<5em,2.5cm>*\as{C}="CV5",
		<6em,2.5cm>*\as{V}="CV6",<7em,2.5cm>*\as{C}="CV7",<8em,2.5cm>*\as{V}="CV8",<9em,2.5cm>*\as{C}="CV9",
		<1em,1.5cm>*\as{f}="cv1",<2em,1.5cm>*\as{a}="cv2",<3em,1.5cm>*\as{ }="cv3",<4em,1.5cm>*\as{i}="cv4",<5em,1.5cm>*\as{\j}="cv5",
		<6em,1.5cm>*\as{e}="cv6",<7em,1.5cm>*\as{ }="cv7",<8em,1.5cm>*\as{e}="cv8",<9em,1.5cm>*\as{ }="cv9",
		<1em,1cm>*\as{m}="cv1.2",<2em,1cm>*\as{a}="cv2.2",<3em,1cm>*\as{ }="cv3.2",<4em,1cm>*\as{e}="cv4.2",<5em,1cm>*\as{\j}="cv5.2",
		<6em,1cm>*\as{e}="cv6.2",<7em,1cm>*\as{ }="cv7.2",<8em,1cm>*\as{e}="cv8.2",<9em,1cm>*\as{ }="cv9.2",
		<2.5em,0cm>*\as{\hp{\sub{1}}M\sub{1}}="m1",<7em,0cm>*\as{\hp{\sub{2}}M\sub{2}}="m2",<5em,0cm>*\as{=}="=",
		"m1"+U;"cv1.2"+D**\dir{-};"m1"+U;"cv2.2"+D**\dir{-};"m1"+U;"cv3.2"+D**\dir{};"m1"+U;"cv4.2"+D**\dir{-};
		"m2"+U;"cv6.2"+D**\dir{-};"m2"+U;"cv8.2"+D**\dir{-};
		"cv1"+U;"CV1"+D**\dir{-};"cv2"+U;"CV2"+D**\dir{-};"cv3"+U;"CV3"+D**\dir{};"cv4"+U;"CV4"+D**\dir{-};"cv5"+U;"CV5"+D**\dir{-};
		"cv6"+U;"CV6"+D**\dir{-};"cv7"+U;"CV7"+D**\dir{};"cv8"+U;"CV8"+D**\dir{-};"cv9"+U;"CV9"+D**\dir{};
		"CV1"+U;"s1"+D**\dir{-};"CV2"+U;"s1"+D**\dir{-};"CV3"+U;"s1"+D**\dir{-};"CV3"+U;"s2"+D**\dir{-};"CV4"+U;"s2"+D**\dir{-};"CV5"+U;"s2"+D**\dir{-};
		"CV5"+U;"s3"+D**\dir{-};"CV6"+U;"s3"+D**\dir{-};"CV7"+U;"s3"+D**\dir{-};"CV7"+U;"s4"+D**\dir{-};"CV8"+U;"s4"+D**\dir{-};"CV9"+U;"s4"+D**\dir{-};
		"s1"+U;"Ft1"+D**\dir{-};"s2"+U;"Ft1"+D**\dir{-};"s3"+U;"Ft2"+D**\dir{-};"s4"+U;"Ft2"+D**\dir{-};
		"Ft1"+U;"PrWd1"+D**\dir{-};"PrWd1"+U;"PrWd2"+D**\dir{-};"Ft2"+U;"PrWd2"+D**\dir{-};
		<5em,1.75cm>*\as{\tikz[red,thick,dashed,baseline=0.9ex]\draw (0,0) rectangle (0.4cm,2cm);}="box",
	\endxy}
	\exi{e.}\exia{\xy
		<2.5em,5.5cm>*\as{\hp{\sub{1}}PrWd\sub{1}}="PrWd1",<4.5em,6.5cm>*\as{\hp{\sub{2}}PrWd\sub{2}}="PrWd2",
		<2.5em,4.5cm>*\as{\hp{\sub{\tsc{m}}}Ft\sub{\tsc{m}}}="Ft1",<7em,4.5cm>*\as{\hp{\sub{2}}Ft\sub{2}}="Ft2",
<1.5em,3.5cm>*\as{\hp{\sub{1}}σ\sub{1}}="s1",<3.5em,3.5cm>*\as{\hp{\sub{2}}σ\sub{2}}="s2",<6em,3.5cm>*\as{\hp{\sub{3}}σ\sub{3}}="s3",<8em,3.5cm>*\as{\hp{\sub{4}}σ\sub{4}}="s4",
		<1em,2.5cm>*\as{C}="CV1",<2em,2.5cm>*\as{V}="CV2",<3em,2.5cm>*\as{V}="CV3",<4em,2.5cm>*\as{C}="CV4",<5em,2.5cm>*\as{C}="CV5",
		<6em,2.5cm>*\as{V}="CV6",<7em,2.5cm>*\as{C}="CV7",<8em,2.5cm>*\as{V}="CV8",<9em,2.5cm>*\as{C}="CV9",
		<1em,1.5cm>*\as{f}="cv1",<2em,1.5cm>*\as{a}="cv2",<3em,1.5cm>*\as{i}="cv3",<4em,1.5cm>*\as{ }="cv4",<5em,1.5cm>*\as{\j}="cv5",
		<6em,1.5cm>*\as{e}="cv6",<7em,1.5cm>*\as{ }="cv7",<8em,1.5cm>*\as{e}="cv8",<9em,1.5cm>*\as{ }="cv9",
		<1em,1cm>*\as{m}="cv1.2",<2em,1cm>*\as{a}="cv2.2",<3em,1cm>*\as{e}="cv3.2",<4em,1cm>*\as{ }="cv4.2",<5em,1cm>*\as{\j}="cv5.2",
		<6em,1cm>*\as{e}="cv6.2",<7em,1cm>*\as{ }="cv7.2",<8em,1cm>*\as{e}="cv8.2",<9em,1cm>*\as{ }="cv9.2",
		<2em,0cm>*\as{\hp{\sub{1}}M\sub{1}}="m1",<7em,0cm>*\as{\hp{\sub{2}}M\sub{2}}="m2",<5em,0cm>*\as{=}="=",
		"m1"+U;"cv1.2"+D**\dir{-};"m1"+U;"cv2.2"+D**\dir{-};"m1"+U;"cv3.2"+D**\dir{-};"m1"+U;"cv4.2"+D**\dir{};
		"m2"+U;"cv6.2"+D**\dir{-};"m2"+U;"cv8.2"+D**\dir{-};
		"cv1"+U;"CV1"+D**\dir{-};"cv2"+U;"CV2"+D**\dir{-};"cv3"+U;"CV3"+D**\dir{-};"cv4"+U;"CV4"+D**\dir{};"cv5"+U;"CV5"+D**\dir{-};
		"cv6"+U;"CV6"+D**\dir{-};"cv7"+U;"CV7"+D**\dir{};"cv8"+U;"CV8"+D**\dir{-};"cv9"+U;"CV9"+D**\dir{};
		"CV1"+U;"s1"+D**\dir{-};"CV2"+U;"s1"+D**\dir{-};"CV3"+U;"s2"+D**\dir{-};"CV4"+U;"s2"+D**\dir{-};
		"CV5"+U;"s3"+D**\dir{-};"CV6"+U;"s3"+D**\dir{-};"CV7"+U;"s3"+D**\dir{-};"CV7"+U;"s4"+D**\dir{-};"CV8"+U;"s4"+D**\dir{-};"CV9"+U;"s4"+D**\dir{-};
		"s1"+U;"Ft1"+D**\dir{-};"s2"+U;"Ft1"+D**\dir{-};"s3"+U;"Ft2"+D**\dir{-};"s4"+U;"Ft2"+D**\dir{-};
		"Ft1"+U;"PrWd1"+D**\dir{-};"PrWd1"+U;"PrWd2"+D**\dir{-};"Ft2"+U;"PrWd2"+D**\dir{-};
		<4.5em,3.25cm>*\as{\tikz[red,thick,dashed,baseline=0.9ex]\draw (0,0) -- (0,5cm);}="line",
	\endxy}
	\end{xlist}}
\end{exe}
\end{multicols}

Metathesis results in the features of the final vowel of the clitic host
being shared across an intervening C-slot.
In this case the C-slot is `filled' by a null consonant,
whose features are represented as \tsc{[-c.]} in (\ref{as:faaj=ee}f).
Because of this intervening consonant, the vowel features de-link
yielding an empty V-slot in (\ref{as:faaj=ee}g)
into which the previous vowel spreads,
giving the final outputs in (\ref{as:faaj=ee}h).

\begin{multicols}{2}
\begin{exe}\exr{as:faaj=ee}{
	\begin{xlist}
	\exi{f.}\exia{\xy
		<2.5em,4.5cm>*\as{\hp{\sub{\tsc{m}}}Ft\sub{\tsc{m}}}="Ft1",<7em,4.5cm>*\as{\hp{\sub{2}}Ft\sub{2}}="Ft2",
<1.5em,3.5cm>*\as{\hp{\sub{1}}σ\sub{1}}="s1",<3.5em,3.5cm>*\as{\hp{\sub{2}}σ\sub{2}}="s2",<6em,3.5cm>*\as{\hp{\sub{3}}σ\sub{3}}="s3",<8em,3.5cm>*\as{\hp{\sub{4}}σ\sub{4}}="s4",
		<1em,2.5cm>*\as{C}="CV1",<2em,2.5cm>*\as{V}="CV2",<3em,2.5cm>*\as{V}="CV3",<4em,2.5cm>*\as{C}="CV4",<5em,2.5cm>*\as{C}="CV5",
		<6em,2.5cm>*\as{V}="CV6",<7em,2.5cm>*\as{C}="CV7",<8em,2.5cm>*\as{V}="CV8",<9em,2.5cm>*\as{C}="CV9",
		<1em,1.5cm>*\as{f}="cv1",<2em,1.5cm>*\as{a}="cv2",<3em,1.5cm>*\as{\xc{\,i\,}}="cv3",<4em,1.5cm>*\as{\0}="cv4",<5em,1.5cm>*\as{\j}="cv5",
		<6em,1.5cm>*\as{e}="cv6",<7em,1.5cm>*\as{ }="cv7",<8em,1.5cm>*\as{e}="cv8",<9em,1.5cm>*\as{ }="cv9",
		<1em,1cm>*\as{m}="cv1.2",<2em,1cm>*\as{a}="cv2.2",<3em,1cm>*\as{\xc{\,e\,}}="cv3.2",<4em,1cm>*\as{\0}="cv4.2",<5em,1cm>*\as{\j}="cv5.2",
		<6em,1cm>*\as{e}="cv6.2",<7em,1cm>*\as{ }="cv7.2",<8em,1cm>*\as{e}="cv8.2",<9em,1cm>*\as{ }="cv9.2",
		<4em,0cm>*\as{\tsc{[+fr.]}}="f",{\ar@{-}|-(.425)*@{|} |-{\hole} |-(.575)*@{|} "f"+U;"cv3.2"+D};"f"+U;"cv5.2"+D**\dir{-};
		<1.9em,0cm>*\as{\tsc{[-c.]}}="f2","f2"+U;"cv4.2"+D**\dir{-};
		"cv1"+U;"CV1"+D**\dir{-};"cv2"+U;"CV2"+D**\dir{-};"cv4"+U;"CV4"+D**\dir{-};"cv5"+U;"CV5"+D**\dir{-};{\ar@{-}|-(.425)*@{|} |-{\hole} |-(.575)*@{|} "cv3"+U;"CV3"+D};
		"cv6"+U;"CV6"+D**\dir{-};"cv7"+U;"CV7"+D**\dir{};"cv8"+U;"CV8"+D**\dir{-};"cv9"+U;"CV9"+D**\dir{};
		"CV1"+U;"s1"+D**\dir{-};"CV2"+U;"s1"+D**\dir{-};"CV3"+U;"s1"+D**\dir{};"CV3"+U;"s2"+D**\dir{-};"CV4"+U;"s2"+D**\dir{-};"CV5"+U;"s2"+D**\dir{};
		"CV5"+U;"s3"+D**\dir{-};"CV6"+U;"s3"+D**\dir{-};"CV7"+U;"s3"+D**\dir{-};"CV7"+U;"s4"+D**\dir{-};"CV8"+U;"s4"+D**\dir{-};"CV9"+U;"s4"+D**\dir{-};
		"s1"+U;"Ft1"+D**\dir{-};"s2"+U;"Ft1"+D**\dir{-};"s3"+U;"Ft2"+D**\dir{-};"s4"+U;"Ft2"+D**\dir{-};
		<4em,1cm>*\as{\tikz[red,thick,dashed,baseline=0.9ex]\draw (0,0) rectangle (1.2cm,2.5cm);}="box",
	\endxy}
	\exi{g.}\exia{\xy
		<2.5em,4.5cm>*\as{\hp{\sub{\tsc{m}}}Ft\sub{\tsc{m}}}="Ft1",<7em,4.5cm>*\as{\hp{\sub{2}}Ft\sub{2}}="Ft2",
<1.5em,3.5cm>*\as{\hp{\sub{1}}σ\sub{1}}="s1",<3.5em,3.5cm>*\as{\hp{\sub{2}}σ\sub{2}}="s2",<6em,3.5cm>*\as{\hp{\sub{3}}σ\sub{3}}="s3",<8em,3.5cm>*\as{\hp{\sub{4}}σ\sub{4}}="s4",
		<1em,2.5cm>*\as{C}="CV1",<2em,2.5cm>*\as{V}="CV2",<3em,2.5cm>*\as{V}="CV3",<4em,2.5cm>*\as{C}="CV4",<5em,2.5cm>*\as{C}="CV5",
		<6em,2.5cm>*\as{V}="CV6",<7em,2.5cm>*\as{C}="CV7",<8em,2.5cm>*\as{V}="CV8",<9em,2.5cm>*\as{C}="CV9",
		<1em,1.5cm>*\as{f}="cv1",<2em,1.5cm>*\as{a}="cv2",<3em,1.5cm>*\as{ }="cv3",<4em,1.5cm>*\as{ }="cv4",<5em,1.5cm>*\as{\j}="cv5",
		<6em,1.5cm>*\as{e}="cv6",<7em,1.5cm>*\as{ }="cv7",<8em,1.5cm>*\as{e}="cv8",<9em,1.5cm>*\as{ }="cv9",
		<1em,1cm>*\as{m}="cv1.2",<2em,1cm>*\as{a}="cv2.2",<3em,1cm>*\as{ }="cv3.2",<4em,1cm>*\as{ }="cv4.2",<5em,1cm>*\as{\j}="cv5.2",
		<6em,1cm>*\as{e}="cv6.2",<7em,1cm>*\as{ }="cv7.2",<8em,1cm>*\as{e}="cv8.2",<9em,1cm>*\as{ }="cv9.2",
		<5em,0cm>*\as{\tsc{[+fr.]}}="f","f"+U;"cv5.2"+D**\dir{-};
		"cv1"+U;"CV1"+D**\dir{-};"cv2"+U;"CV2"+D**\dir{-};"cv2"+U;"CV3"+D**\dir{.};"cv4"+U;"CV4"+D**\dir{};"cv5"+U;"CV5"+D**\dir{-};"cv2.2"+U;"cv3"+U**\dir{.};
		"cv6"+U;"CV6"+D**\dir{-};"cv7"+U;"CV7"+D**\dir{};"cv8"+U;"CV8"+D**\dir{-};"cv9"+U;"CV9"+D**\dir{};
		"CV1"+U;"s1"+D**\dir{-};"CV2"+U;"s1"+D**\dir{-};"CV3"+U;"s1"+D**\dir{};"CV3"+U;"s2"+D**\dir{-};"CV4"+U;"s2"+D**\dir{-};"CV5"+U;"s2"+D**\dir{};
		"CV5"+U;"s3"+D**\dir{-};"CV6"+U;"s3"+D**\dir{-};"CV7"+U;"s3"+D**\dir{-};"CV7"+U;"s4"+D**\dir{-};"CV8"+U;"s4"+D**\dir{-};"CV9"+U;"s4"+D**\dir{-};
		"s1"+U;"Ft1"+D**\dir{-};"s2"+U;"Ft1"+D**\dir{-};"s3"+U;"Ft2"+D**\dir{-};"s4"+U;"Ft2"+D**\dir{-};
		<3em,1.75cm>*\as{\tikz[red,thick,dashed,baseline=0.9ex]\draw (0,0) rectangle (0.4cm,2cm);}="box",
	\endxy}
	\end{xlist}}
\end{exe}
\end{multicols}
\begin{exe}\sn{
	\begin{xlist}
	\exi{h.}\exia{\xy
		<2.5em,5.5cm>*\as{\hp{\sub{1}}PrWd\sub{1}}="PrWd1",<4.5em,6.5cm>*\as{\hp{\sub{2}}PrWd\sub{2}}="PrWd2",
		<2.5em,4.5cm>*\as{\hp{\sub{\tsc{m}}}Ft\sub{\tsc{m}}}="Ft1",<7em,4.5cm>*\as{\hp{\sub{2}}Ft\sub{2}}="Ft2",
<1.5em,3.5cm>*\as{\hp{\sub{1}}σ\sub{1}}="s1",<3.5em,3.5cm>*\as{\hp{\sub{2}}σ\sub{2}}="s2",<6em,3.5cm>*\as{\hp{\sub{3}}σ\sub{3}}="s3",<8em,3.5cm>*\as{\hp{\sub{4}}σ\sub{4}}="s4",
		<1em,2.5cm>*\as{C}="CV1",<2em,2.5cm>*\as{V}="CV2",<3em,2.5cm>*\as{V}="CV3",<4em,2.5cm>*\as{C}="CV4",<5em,2.5cm>*\as{C}="CV5",
		<6em,2.5cm>*\as{V}="CV6",<7em,2.5cm>*\as{C}="CV7",<8em,2.5cm>*\as{V}="CV8",<9em,2.5cm>*\as{C}="CV9",
		<1em,1.5cm>*\as{f}="cv1",<2.5em,1.5cm>*\as{a}="cv2",<3em,1.5cm>*\as{ }="cv3",<4em,1.5cm>*\as{ }="cv4",<5em,1.5cm>*\as{\j}="cv5",
		<6em,1.5cm>*\as{e}="cv6",<7em,1.5cm>*\as{ }="cv7",<8em,1.5cm>*\as{e}="cv8",<9em,1.5cm>*\as{ }="cv9",
		<1em,1cm>*\as{m}="cv1.2",<2.5em,1cm>*\as{a}="cv2.2",<3em,1cm>*\as{ }="cv3.2",<4em,1cm>*\as{ }="cv4.2",<5em,1cm>*\as{\j}="cv5.2",
		<6em,1cm>*\as{e}="cv6.2",<7em,1cm>*\as{ }="cv7.2",<8em,1cm>*\as{e}="cv8.2",<9em,1cm>*\as{ }="cv9.2",
		<1.75em,0cm>*\as{\hp{\sub{1}}M\sub{1}}="m1",<7em,0cm>*\as{\hp{\sub{2}}M\sub{2}}="m2",<5em,0cm>*\as{=}="=",
		"m1"+U;"cv1.2"+D**\dir{-};"m1"+U;"cv2.2"+D**\dir{-};"m2"+U;"cv6.2"+D**\dir{-};"m2"+U;"cv8.2"+D**\dir{-};
		"cv1"+U;"CV1"+D**\dir{-};"cv2"+U;"CV2"+D**\dir{-};"cv2"+U;"CV3"+D**\dir{-};"cv4"+U;"CV4"+D**\dir{};"cv5"+U;"CV5"+D**\dir{-};
		"cv6"+U;"CV6"+D**\dir{-};"cv7"+U;"CV7"+D**\dir{};"cv8"+U;"CV8"+D**\dir{-};"cv9"+U;"CV9"+D**\dir{};
		"CV1"+U;"s1"+D**\dir{-};"CV2"+U;"s1"+D**\dir{-};"CV3"+U;"s1"+D**\dir{};"CV3"+U;"s2"+D**\dir{-};"CV4"+U;"s2"+D**\dir{-};"CV5"+U;"s2"+D**\dir{};
		"CV5"+U;"s3"+D**\dir{-};"CV6"+U;"s3"+D**\dir{-};"CV7"+U;"s3"+D**\dir{-};"CV7"+U;"s4"+D**\dir{-};"CV8"+U;"s4"+D**\dir{-};"CV9"+U;"s4"+D**\dir{-};
		"s1"+U;"Ft1"+D**\dir{-};"s2"+U;"Ft1"+D**\dir{-};"s3"+U;"Ft2"+D**\dir{-};"s4"+U;"Ft2"+D**\dir{-};
		"s1"+U;"Ft1"+D**\dir{-};"s2"+U;"Ft1"+D**\dir{-};"s3"+U;"Ft2"+D**\dir{-};"s4"+U;"Ft2"+D**\dir{-};
		"Ft1"+U;"PrWd1"+D**\dir{-};"PrWd1"+U;"PrWd2"+D**\dir{-};"Ft2"+U;"PrWd2"+D**\dir{-};
		%<4em,1cm>*\as{\tikz[red,thick,dashed,baseline=0.9ex]\draw (0,0) rectangle (1.2cm,2.5cm);}="box",
	\endxy}
	\end{xlist}}
\end{exe}

Evidence that both consonant insertion \emph{and} metathesis
are required for vowel assimilation comes from the process of
consonant insertion in the variety of Kotos Amarasi spoken in Fo{\Q}asa{\Q} hamlet.
As discussed in \srf{sec:FoqConIns} below, in Fo{\Q}asa{\Q} hamlet
consonant insertion before enclitics is not conditioned by vowel features
spreading; instead a default consonant /ɡ/ is simply inserted.
When metathesis then takes place, vowel assimilation does not occur.
One example is Fo{\Q}asa{\Q} \ve{umi} {\ra} \ve{u\tbr{i}mg=ee} `house'.

In Nai{\Q}bais Amfo{\Q}an, vowel assimilation does not occur
after consonant insertion for words which end in a vowel sequence.
\citet[32]{cu18} gives many examples including
\ve{a\tbr{i}} + \ve{=ees} {\ra} \ve{a\tbr{i}\j=ees} `one fire',
and \ve{ha\tbr{u}} + \ve{=ees} {\ra} \ve{ha\tbr{u}gw=ees} `one tree'.
However, vowel assimilation \emph{does} occur for CV{\#} final words.
Examples include \ve{uk\tbr{i}} + \ve{=ees} {\ra} \ve{u\tbr{u}k\j=ees} `one banana'
and \ve{nef\tbr{o}} + \ve{=ees} {\ra} \ve{ne\tbr{e}fgw=ees} `one lake' \citep[32]{cu18}.

This Amfo{\Q}an data provides evidence
\emph{against} my analysis of vowel assimilation in Amarasi
words with a final vowel sequence being due to
metathesis of an intervening empty C-slot.
However, it is possible to posit that in Amfo{\Q}an
intervening empty C-slots do not trigger de-linking
of vowel features, or that words with a surface
final vowel sequence do not undergo metathesis
before vowel-initial enclitics.

\subsection{Clitic hosts with final VVCV{\#}}
After words which end in VVCV{\#} (\srf{sec:SurVVCVWor}),
consonant insertion is triggered, but vowel assimilation does not take place.
Examples are given in \qf{ex2:V1V2C1V3a->V1V2C1Ca=} below.

\begin{exe}
	\ex{{\ldots}V\sub{1}V\sub{2}C\sub{1}V\sub{3}{\sA} {\ra} {\ldots}V\sub{1}V\sub{2}C\sub{1}C{\sA}=}\label{ex2:V1V2C1V3a->V1V2C1Ca=}
		\sn{\gw\begin{tabular}{rlll}
			U\=/form			&			&M\=/form								&\\
			\ve{aunu}		&{\ra}&\ve{au{\ng}\tbr{gw}=ee}			&`spear'	\\
			\ve{n-aiti}	&{\ra}&\ve{n-ait\tbr{\j}=ee}		&`picks it up'	\\
			\ve{n-eiti}	&{\ra}&\ve{n-eit\tbr{\j}=een}	&`has travelled'	\\
		\end{tabular}}
\end{exe}

This is explained by the fact that the first two vowels
of such words are assigned to a single V-slot, as illustrated
for \ve{n-aiti} {\ra} \ve{n-ait\j=ee} `picks it up' in \qf{as:naitj=ee} below.
Consonant insertion then occurs in \qf{as:naitj=ee1}--\qf{as:naitj=ee2}
to provide the second foot with an initial consonant.

\begin{multicols}{2}
\begin{exe}\ex{\label{as:naitj=ee}
	\begin{xlist}
	\exa{\xy
		<3em,4cm>*\as{\hp{\sub{1}}Ft\sub{1}}="Ft1",<7em,4cm>*\as{\hp{\sub{2}}Ft\sub{2}}="Ft2",
		<2em,3cm>*\as{\hp{\sub{1}}σ\sub{1}}="s1",<4em,3cm>*\as{\hp{\sub{2}}σ\sub{2}}="s2",<6em,3cm>*\as{\hp{\sub{3}}σ\sub{3}}="s3",<8em,3cm>*\as{\hp{\sub{4}}σ\sub{4}}="s4",
		<1em,2cm>*\as{C}="CV1",<2em,2cm>*\as{V}="CV2",<3em,2cm>*\as{C}="CV3",<4em,2cm>*\as{V}="CV4",<5em,2cm>*\as{C}="CV5",
		<6em,2cm>*\as{V}="CV6",<7em,2cm>*\as{C}="CV7",<8em,2cm>*\as{V}="CV8",<9em,2cm>*\as{C}="CV9",
		<1em,1cm>*\as{n}="cv1",<1.75em,1cm>*\as{a}="cv2",<2.25em,1cm>*\as{i}="cv2.2",<3em,1cm>*\as{t}="cv3",<4em,1cm>*\as{i}="cv4",<5em,1cm>*\as{ }="cv5",
		<6em,1cm>*\as{e}="cv6",<7em,1cm>*\as{ }="cv7",<8em,1cm>*\as{e}="cv8",<9em,1cm>*\as{ }="cv9",
		<4em,0cm>*\as{\tsc{[+fr.]}}="f","f"+U;"cv4"+D**\dir{-};"f"+U;"cv5"+D**\dir{.};"cv5"+D;"CV5"+D**\dir{.};"cv4"+D;"cv5"+U**\dir{.};
		"cv1"+U;"CV1"+D**\dir{-};"cv2"+U;"CV2"+D**\dir{-};"cv2.2"+U;"CV2"+D**\dir{-};"cv3"+U;"CV3"+D**\dir{-};"cv4"+U;"CV4"+D**\dir{-};"cv5"+U;"CV5"+D**\dir{};
		"cv6"+U;"CV6"+D**\dir{-};"cv7"+U;"CV7"+D**\dir{};"cv8"+U;"CV8"+D**\dir{-};"cv9"+U;"CV9"+D**\dir{};
		"CV1"+U;"s1"+D**\dir{-};"CV2"+U;"s1"+D**\dir{-};"CV3"+U;"s1"+D**\dir{-};"CV3"+U;"s2"+D**\dir{-};"CV4"+U;"s2"+D**\dir{-};"CV5"+U;"s2"+D**\dir{-};
		"CV5"+U;"s3"+D**\dir{-};"CV6"+U;"s3"+D**\dir{-};"CV7"+U;"s3"+D**\dir{-};"CV7"+U;"s4"+D**\dir{-};"CV8"+U;"s4"+D**\dir{-};"CV9"+U;"s4"+D**\dir{-};
		"s1"+U;"Ft1"+D**\dir{-};"s2"+U;"Ft1"+D**\dir{-};"s3"+U;"Ft2"+D**\dir{-};"s4"+U;"Ft2"+D**\dir{-};
		<4.5em,1.5cm>*\as{\tikz[red,thick,dashed,baseline=0.9ex]\draw (0,0) rectangle (0.8cm,1.5cm);}="box",
	\endxy}\label{as:naitj=ee1}
	\exa{\xy
		<3em,4cm>*\as{\hp{\sub{1}}Ft\sub{1}}="Ft1",<7em,4cm>*\as{\hp{\sub{2}}Ft\sub{2}}="Ft2",
		<2em,3cm>*\as{\hp{\sub{1}}σ\sub{1}}="s1",<4em,3cm>*\as{\hp{\sub{2}}σ\sub{2}}="s2",<6em,3cm>*\as{\hp{\sub{3}}σ\sub{3}}="s3",<8em,3cm>*\as{\hp{\sub{4}}σ\sub{4}}="s4",
		<1em,2cm>*\as{C}="CV1",<2em,2cm>*\as{V}="CV2",<3em,2cm>*\as{C}="CV3",<4em,2cm>*\as{V}="CV4",<5em,2cm>*\as{C}="CV5",
		<6em,2cm>*\as{V}="CV6",<7em,2cm>*\as{C}="CV7",<8em,2cm>*\as{V}="CV8",<9em,2cm>*\as{C}="CV9",
		<1em,1cm>*\as{n}="cv1",<1.75em,1cm>*\as{a}="cv2",<2.25em,1cm>*\as{i}="cv2.2",<3em,1cm>*\as{t}="cv3",<4em,1cm>*\as{i}="cv4",<5em,1cm>*\as{\j}="cv5",
		<6em,1cm>*\as{e}="cv6",<7em,1cm>*\as{ }="cv7",<8em,1cm>*\as{e}="cv8",<9em,1cm>*\as{ }="cv9",
		<4.5em,0cm>*\as{\tsc{[+fr.]}}="f","f"+U;"cv4"+D**\dir{-};"f"+U;"cv5"+D**\dir{-};
		"cv1"+U;"CV1"+D**\dir{-};"cv2"+U;"CV2"+D**\dir{-};"cv2.2"+U;"CV2"+D**\dir{-};"cv3"+U;"CV3"+D**\dir{-};"cv4"+U;"CV4"+D**\dir{-};"cv5"+U;"CV5"+D**\dir{-};
		"cv6"+U;"CV6"+D**\dir{-};"cv7"+U;"CV7"+D**\dir{};"cv8"+U;"CV8"+D**\dir{-};"cv9"+U;"CV9"+D**\dir{};
		"CV1"+U;"s1"+D**\dir{-};"CV2"+U;"s1"+D**\dir{-};"CV3"+U;"s1"+D**\dir{-};"CV3"+U;"s2"+D**\dir{-};"CV4"+U;"s2"+D**\dir{-};"CV5"+U;"s2"+D**\dir{-};
		"CV5"+U;"s3"+D**\dir{-};"CV6"+U;"s3"+D**\dir{-};"CV7"+U;"s3"+D**\dir{-};"CV7"+U;"s4"+D**\dir{-};"CV8"+U;"s4"+D**\dir{-};"CV9"+U;"s4"+D**\dir{-};
		"s1"+U;"Ft1"+D**\dir{-};"s2"+U;"Ft1"+D**\dir{-};"s3"+U;"Ft2"+D**\dir{-};"s4"+U;"Ft2"+D**\dir{-};
		<5em,1.5cm>*\as{\tikz[red,thick,dashed,baseline=0.9ex]\draw (0,0) rectangle (0.4cm,1.5cm);}="box",
	\endxy}\label{as:naitj=ee2}
	\end{xlist}}
\end{exe}
\end{multicols}

The recently filled C-slot is shared between the internal and external prosodic
words, as shown in (\ref{as:naitj=ee}d).
Because fuzzy borders are not allowed at prosodic word boundaries,
metathesis is triggered, which yields the form in (\ref{as:naitj=ee}e)
which has a crisp edge at the prosodic word boundary.

\begin{multicols}{2}
\begin{exe}\exr{as:naitj=ee}{
	\begin{xlist}
		\exi{d.}\exia{\xy
		<3em,5cm>*\as{\hp{\sub{1}}PrWd\sub{1}}="PrWd1",<4.5em,6cm>*\as{\hp{\sub{2}}PrWd\sub{2}}="PrWd2",
		<3em,4cm>*\as{\hp{\sub{1}}Ft\sub{1}}="Ft1",<7em,4cm>*\as{\hp{\sub{2}}Ft\sub{2}}="Ft2",
		<2em,3cm>*\as{\hp{\sub{1}}σ\sub{1}}="s1",<4em,3cm>*\as{\hp{\sub{2}}σ\sub{2}}="s2",<6em,3cm>*\as{\hp{\sub{3}}σ\sub{3}}="s3",<8em,3cm>*\as{\hp{\sub{4}}σ\sub{4}}="s4",
		<1em,2cm>*\as{C}="CV1",<2em,2cm>*\as{V}="CV2",<3em,2cm>*\as{C}="CV3",<4em,2cm>*\as{V}="CV4",<5em,2cm>*\as{C}="CV5",
		<6em,2cm>*\as{V}="CV6",<7em,2cm>*\as{C}="CV7",<8em,2cm>*\as{V}="CV8",<9em,2cm>*\as{C}="CV9",
		<1em,1cm>*\as{n}="cv1",<1.75em,1cm>*\as{a}="cv2",<2.25em,1cm>*\as{i}="cv2.2",<3em,1cm>*\as{t}="cv3",<4em,1cm>*\as{i}="cv4",<5em,1cm>*\as{\j}="cv5",
		<6em,1cm>*\as{e}="cv6",<7em,1cm>*\as{ }="cv7",<8em,1cm>*\as{e}="cv8",<9em,1cm>*\as{ }="cv9",
		<1em,0cm>*\as{\hp{\sub{1}}M\sub{1}}="m1",<2.75em,0cm>*\as{\hp{\sub{2}}M\sub{2}}="m2",<7em,0cm>*\as{\hp{\sub{3}}M\sub{3}}="m3",
		<5em,0cm>*\as{=}="=",<2em,0cm>*\as{-}="-",
		"m1"+U;"cv1"+D**\dir{-};"m2"+U;"cv2"+D**\dir{-};"m2"+U;"cv2.2"+D**\dir{-};"m2"+U;"cv3"+D**\dir{-};"m2"+U;"cv4"+D**\dir{-};
		"m3"+U;"cv6"+D**\dir{-};"m3"+U;"cv8"+D**\dir{-};
		"cv1"+U;"CV1"+D**\dir{-};"cv2"+U;"CV2"+D**\dir{-};"cv2.2"+U;"CV2"+D**\dir{-};"cv3"+U;"CV3"+D**\dir{-};"cv4"+U;"CV4"+D**\dir{-};"cv5"+U;"CV5"+D**\dir{-};
		"cv6"+U;"CV6"+D**\dir{-};"cv7"+U;"CV7"+D**\dir{};"cv8"+U;"CV8"+D**\dir{-};"cv9"+U;"CV9"+D**\dir{};
		"CV1"+U;"s1"+D**\dir{-};"CV2"+U;"s1"+D**\dir{-};"CV3"+U;"s1"+D**\dir{-};"CV3"+U;"s2"+D**\dir{-};"CV4"+U;"s2"+D**\dir{-};"CV5"+U;"s2"+D**\dir{-};
		"CV5"+U;"s3"+D**\dir{-};"CV6"+U;"s3"+D**\dir{-};"CV7"+U;"s3"+D**\dir{-};"CV7"+U;"s4"+D**\dir{-};"CV8"+U;"s4"+D**\dir{-};"CV9"+U;"s4"+D**\dir{-};
		"s1"+U;"Ft1"+D**\dir{-};"s2"+U;"Ft1"+D**\dir{-};"s3"+U;"Ft2"+D**\dir{-};"s4"+U;"Ft2"+D**\dir{-};
		"Ft1"+U;"PrWd1"+D**\dir{-};"PrWd1"+U;"PrWd2"+D**\dir{-};"Ft2"+U;"PrWd2"+D**\dir{-};
		<5em,1.5cm>*\as{\tikz[red,thick,dashed,baseline=0.9ex]\draw (0,0) rectangle (0.4cm,1.5cm);}="box",
	\endxy}
	\exi{e.}\exia{\xy
		<2.5em,5cm>*\as{\hp{\sub{1}}PrWd\sub{1}}="PrWd1",<4.5em,6cm>*\as{\hp{\sub{2}}PrWd\sub{2}}="PrWd2",
		<2.5em,4cm>*\as{\hp{\sub{\tsc{m}}}Ft\sub{\tsc{m}}}="Ft1",<7em,4cm>*\as{\hp{\sub{2}}Ft\sub{2}}="Ft2",
<1.5em,3cm>*\as{\hp{\sub{1}}σ\sub{1}}="s1",<3.5em,3cm>*\as{\hp{\sub{2}}σ\sub{2}}="s2",<6em,3cm>*\as{\hp{\sub{3}}σ\sub{3}}="s3",<8em,3cm>*\as{\hp{\sub{4}}σ\sub{4}}="s4",
		<1em,2cm>*\as{C}="CV1",<2em,2cm>*\as{V}="CV2",<3em,2cm>*\as{V}="CV3",<4em,2cm>*\as{C}="CV4",<5em,2cm>*\as{C}="CV5",
		<6em,2cm>*\as{V}="CV6",<7em,2cm>*\as{C}="CV7",<8em,2cm>*\as{V}="CV8",<9em,2cm>*\as{C}="CV9",
		<1em,1cm>*\as{n}="cv1",<1.75em,1cm>*\as{a}="cv2",<2.25em,1cm>*\as{i}="cv2.2",<3em,1cm>*\as{i}="cv3",<4em,1cm>*\as{t}="cv4",<5em,1cm>*\as{\j}="cv5",
		<6em,1cm>*\as{e}="cv6",<7em,1cm>*\as{ }="cv7",<8em,1cm>*\as{e}="cv8",<9em,1cm>*\as{ }="cv9",
		<1em,0cm>*\as{\hp{\sub{1}}M\sub{1}}="m1",<2.75em,0cm>*\as{\hp{\sub{2}}M\sub{2}}="m2",<7em,0cm>*\as{\hp{\sub{3}}M\sub{3}}="m3",
		<5em,0cm>*\as{=}="=",<2em,0cm>*\as{-}="-",
		"m1"+U;"cv1"+D**\dir{-};"m2"+U;"cv2"+D**\dir{-};"m2"+U;"cv2.2"+D**\dir{-};"m2"+U;"cv3"+D**\dir{-};"m2"+U;"cv4"+D**\dir{-};
		"m3"+U;"cv6"+D**\dir{-};"m3"+U;"cv8"+D**\dir{-};
		"cv1"+U;"CV1"+D**\dir{-};"cv2"+U;"CV2"+D**\dir{-};"cv2.2"+U;"CV2"+D**\dir{-};"cv3"+U;"CV3"+D**\dir{-};"cv4"+U;"CV4"+D**\dir{-};"cv5"+U;"CV5"+D**\dir{-};
		"cv6"+U;"CV6"+D**\dir{-};"cv7"+U;"CV7"+D**\dir{};"cv8"+U;"CV8"+D**\dir{-};"cv9"+U;"CV9"+D**\dir{};
		"CV1"+U;"s1"+D**\dir{-};"CV2"+U;"s1"+D**\dir{-};"CV3"+U;"s2"+D**\dir{-};"CV4"+U;"s2"+D**\dir{-};
		"CV5"+U;"s3"+D**\dir{-};"CV6"+U;"s3"+D**\dir{-};"CV7"+U;"s3"+D**\dir{-};"CV7"+U;"s4"+D**\dir{-};"CV8"+U;"s4"+D**\dir{-};"CV9"+U;"s4"+D**\dir{-};
		"s1"+U;"Ft1"+D**\dir{-};"s2"+U;"Ft1"+D**\dir{-};"s3"+U;"Ft2"+D**\dir{-};"s4"+U;"Ft2"+D**\dir{-};
		"Ft1"+U;"PrWd1"+D**\dir{-};"PrWd1"+U;"PrWd2"+D**\dir{-};"Ft2"+U;"PrWd2"+D**\dir{-};
		<4.5em,3cm>*\as{\tikz[red,thick,dashed,baseline=0.9ex]\draw (0,0) -- (0,4.5cm);}="line",
	\endxy}
	\end{xlist}}
\end{exe}
\end{multicols}

The final vowel of the clitic host then de-links in (\ref{as:naitj=ee}f).
This is both because it shares features with /\j/ across an intervening C-slot,
and because sequences of three vowels are not allowed in Amarasi.
After this vowel de-links, the previous vowel spreads into the empty V-slot in (\ref{as:naitj=ee}g),
yielding the final output in (\ref{as:naitj=ee}h).

\begin{multicols}{2}
\begin{exe}\exr{as:naitj=ee}{
	\begin{xlist}
	\exi{f.}\exia{\xy
		<2.5em,4cm>*\as{\hp{\sub{\tsc{m}}}Ft\sub{\tsc{m}}}="Ft1",<7em,4cm>*\as{\hp{\sub{2}}Ft\sub{2}}="Ft2",
<1.5em,3cm>*\as{\hp{\sub{1}}σ\sub{1}}="s1",<3.5em,3cm>*\as{\hp{\sub{2}}σ\sub{2}}="s2",<6em,3cm>*\as{\hp{\sub{3}}σ\sub{3}}="s3",<8em,3cm>*\as{\hp{\sub{4}}σ\sub{4}}="s4",
		<1em,2cm>*\as{C}="CV1",<2em,2cm>*\as{V}="CV2",<3em,2cm>*\as{V}="CV3",<4em,2cm>*\as{C}="CV4",<5em,2cm>*\as{C}="CV5",
		<6em,2cm>*\as{V}="CV6",<7em,2cm>*\as{C}="CV7",<8em,2cm>*\as{V}="CV8",<9em,2cm>*\as{C}="CV9",
		<1em,1cm>*\as{n}="cv1",<1.75em,1cm>*\as{a}="cv2",<2.25em,1cm>*\as{i}="cv2.2",<3em,1cm>*\as{\xc{\,i\,}}="cv3",<4em,1cm>*\as{t}="cv4",<5em,1cm>*\as{\j}="cv5",
		<6em,1cm>*\as{e}="cv6",<7em,1cm>*\as{ }="cv7",<8em,1cm>*\as{e}="cv8",<9em,1cm>*\as{ }="cv9",
		<4em,0cm>*\as{\tsc{[+fr.]}}="f",{\ar@{-}|-(.425)*@{|} |-{\hole} |-(.575)*@{|} "f"+U;"cv3"+D};"f"+U;"cv5"+D**\dir{-};
		<1.9em,0cm>*\as{\tsc{[+c.]}}="f2","f2"+U;"cv4"+D**\dir{-};
		"cv1"+U;"CV1"+D**\dir{-};"cv2"+U;"CV2"+D**\dir{-};"cv2.2"+U;"CV2"+D**\dir{-};"cv4"+U;"CV4"+D**\dir{-};
		"cv5"+U;"CV5"+D**\dir{-};{\ar@{-}|-(.425)*@{|} |-{\hole} |-(.575)*@{|} "cv3"+U;"CV3"+D};
		"cv6"+U;"CV6"+D**\dir{-};"cv7"+U;"CV7"+D**\dir{};"cv8"+U;"CV8"+D**\dir{-};"cv9"+U;"CV9"+D**\dir{};
		"CV1"+U;"s1"+D**\dir{-};"CV2"+U;"s1"+D**\dir{-};"CV3"+U;"s1"+D**\dir{};"CV3"+U;"s2"+D**\dir{-};"CV4"+U;"s2"+D**\dir{-};"CV5"+U;"s2"+D**\dir{};
		"CV5"+U;"s3"+D**\dir{-};"CV6"+U;"s3"+D**\dir{-};"CV7"+U;"s3"+D**\dir{-};"CV7"+U;"s4"+D**\dir{-};"CV8"+U;"s4"+D**\dir{-};"CV9"+U;"s4"+D**\dir{-};
		"s1"+U;"Ft1"+D**\dir{-};"s2"+U;"Ft1"+D**\dir{-};"s3"+U;"Ft2"+D**\dir{-};"s4"+U;"Ft2"+D**\dir{-};
		<4em,0.75cm>*\as{\tikz[red,thick,dashed,baseline=0.9ex]\draw (0,0) rectangle (1.2cm,2cm);}="box",
	\endxy}
	\exi{g.}\exia{\xy
		<2.5em,4cm>*\as{\hp{\sub{\tsc{m}}}Ft\sub{\tsc{m}}}="Ft1",<7em,4cm>*\as{\hp{\sub{2}}Ft\sub{2}}="Ft2",
<1.5em,3cm>*\as{\hp{\sub{1}}σ\sub{1}}="s1",<3.5em,3cm>*\as{\hp{\sub{2}}σ\sub{2}}="s2",<6em,3cm>*\as{\hp{\sub{3}}σ\sub{3}}="s3",<8em,3cm>*\as{\hp{\sub{4}}σ\sub{4}}="s4",
		<1em,2cm>*\as{C}="CV1",<2em,2cm>*\as{V}="CV2",<3em,2cm>*\as{V}="CV3",<4em,2cm>*\as{C}="CV4",<5em,2cm>*\as{C}="CV5",
		<6em,2cm>*\as{V}="CV6",<7em,2cm>*\as{C}="CV7",<8em,2cm>*\as{V}="CV8",<9em,2cm>*\as{C}="CV9",
		<1em,1cm>*\as{n}="cv1",<1.75em,1cm>*\as{a}="cv2",<2.25em,1cm>*\as{i}="cv2.2",<3em,1cm>*\as{ }="cv3",<4em,1cm>*\as{t}="cv4",<5em,1cm>*\as{\j}="cv5",
		<6em,1cm>*\as{e}="cv6",<7em,1cm>*\as{ }="cv7",<8em,1cm>*\as{e}="cv8",<9em,1cm>*\as{ }="cv9",
		<5em,0cm>*\as{\tsc{[+fr.]}}="f","f"+U;"cv5"+D**\dir{-};
		"cv1"+U;"CV1"+D**\dir{-};"cv2"+U;"CV2"+D**\dir{-};{\ar@{-}|-(.425)*@{|} |-{\hole} |-(.525)*@{|} "cv2.2"+U;"CV2"+D};
		"cv2.2"+U;"CV3"+D**\dir{.};"cv4"+U;"CV4"+D**\dir{-};"cv5"+U;"CV5"+D**\dir{-};
		"cv6"+U;"CV6"+D**\dir{-};"cv7"+U;"CV7"+D**\dir{};"cv8"+U;"CV8"+D**\dir{-};"cv9"+U;"CV9"+D**\dir{};
		"CV1"+U;"s1"+D**\dir{-};"CV2"+U;"s1"+D**\dir{-};"CV3"+U;"s1"+D**\dir{};"CV3"+U;"s2"+D**\dir{-};"CV4"+U;"s2"+D**\dir{-};"CV5"+U;"s2"+D**\dir{};
		"CV5"+U;"s3"+D**\dir{-};"CV6"+U;"s3"+D**\dir{-};"CV7"+U;"s3"+D**\dir{-};"CV7"+U;"s4"+D**\dir{-};"CV8"+U;"s4"+D**\dir{-};"CV9"+U;"s4"+D**\dir{-};
		"s1"+U;"Ft1"+D**\dir{-};"s2"+U;"Ft1"+D**\dir{-};"s3"+U;"Ft2"+D**\dir{-};"s4"+U;"Ft2"+D**\dir{-};
		<3em,1.5cm>*\as{\tikz[red,thick,dashed,baseline=0.9ex]\draw (0,0) rectangle (0.4cm,1.5cm);}="box",
	\endxy}
	\end{xlist}}
\end{exe}
\end{multicols}

\begin{exe}\sn{
	\begin{xlist}
	\exi{h.}\exia{\xy
		<2.5em,5cm>*\as{\hp{\sub{1}}PrWd\sub{1}}="PrWd1",<4.5em,6cm>*\as{\hp{\sub{2}}PrWd\sub{2}}="PrWd2",
		<2.5em,4cm>*\as{\hp{\sub{\tsc{m}}}Ft\sub{\tsc{m}}}="Ft1",<7em,4cm>*\as{\hp{\sub{2}}Ft\sub{2}}="Ft2",
<1.5em,3cm>*\as{\hp{\sub{1}}σ\sub{1}}="s1",<3.5em,3cm>*\as{\hp{\sub{2}}σ\sub{2}}="s2",<6em,3cm>*\as{\hp{\sub{3}}σ\sub{3}}="s3",<8em,3cm>*\as{\hp{\sub{4}}σ\sub{4}}="s4",
		<1em,2cm>*\as{C}="CV1",<2em,2cm>*\as{V}="CV2",<3em,2cm>*\as{V}="CV3",<4em,2cm>*\as{C}="CV4",<5em,2cm>*\as{C}="CV5",
		<6em,2cm>*\as{V}="CV6",<7em,2cm>*\as{C}="CV7",<8em,2cm>*\as{V}="CV8",<9em,2cm>*\as{C}="CV9",
		<1em,1cm>*\as{n}="cv1",<2em,1cm>*\as{a}="cv2",<3em,1cm>*\as{i}="cv3",<4em,1cm>*\as{t}="cv4",<5em,1cm>*\as{\j}="cv5",
		<6em,1cm>*\as{e}="cv6",<7em,1cm>*\as{ }="cv7",<8em,1cm>*\as{e}="cv8",<9em,1cm>*\as{ }="cv9",
		<1em,0cm>*\as{\hp{\sub{1}}M\sub{1}}="m1",<3em,0cm>*\as{\hp{\sub{2}}M\sub{2}}="m2",<7em,0cm>*\as{\hp{\sub{3}}M\sub{3}}="m3",
		<2em,0cm>*\as{-}="-",<5em,0cm>*\as{=}="=",
		"m1"+U;"cv1"+D**\dir{-};"m2"+U;"cv2"+D**\dir{-};"m2"+U;"cv3"+D**\dir{-};"m2"+U;"cv4"+D**\dir{-};
		"m3"+U;"cv6"+D**\dir{-};"m3"+U;"cv8"+D**\dir{-};
		"cv1"+U;"CV1"+D**\dir{-};"cv2"+U;"CV2"+D**\dir{-};"cv3"+U;"CV3"+D**\dir{-};"cv4"+U;"CV4"+D**\dir{-};"cv5"+U;"CV5"+D**\dir{-};
		"cv6"+U;"CV6"+D**\dir{-};"cv7"+U;"CV7"+D**\dir{};"cv8"+U;"CV8"+D**\dir{-};"cv9"+U;"CV9"+D**\dir{};
		"CV1"+U;"s1"+D**\dir{-};"CV2"+U;"s1"+D**\dir{-};"CV3"+U;"s1"+D**\dir{};"CV3"+U;"s2"+D**\dir{-};"CV4"+U;"s2"+D**\dir{-};"CV5"+U;"s2"+D**\dir{};
		"CV5"+U;"s3"+D**\dir{-};"CV6"+U;"s3"+D**\dir{-};"CV7"+U;"s3"+D**\dir{-};"CV7"+U;"s4"+D**\dir{-};"CV8"+U;"s4"+D**\dir{-};"CV9"+U;"s4"+D**\dir{-};
		"s1"+U;"Ft1"+D**\dir{-};"s2"+U;"Ft1"+D**\dir{-};"s3"+U;"Ft2"+D**\dir{-};"s4"+U;"Ft2"+D**\dir{-};
		"s1"+U;"Ft1"+D**\dir{-};"s2"+U;"Ft1"+D**\dir{-};"s3"+U;"Ft2"+D**\dir{-};"s4"+U;"Ft2"+D**\dir{-};
		"Ft1"+U;"PrWd1"+D**\dir{-};"PrWd1"+U;"PrWd2"+D**\dir{-};"Ft2"+U;"PrWd2"+D**\dir{-};
	\endxy}
	\end{xlist}}
\end{exe}
\section{Clitic hosts with final /a/}\label{sec:CliHosFinA}
When an enclitic attaches to stems which end in the vowel /a/,
the clitic host undergoes metathesis and no consonant is inserted.
Examples of vowel-initial enclitics attached to stems which
end in /Ca/ are given in \qf{ex:VCa+=V->VVC=V} below.

As discussed in \srf{sec:AssOfA}, when a word which ends in surface
/Ca/ undergoes metathesis, the vowel /a/ undergoes complete assimilation.
Assimilation of /a/ in metathesised forms is a derived
environment effect and should not be confused with assimilation of vowels
after consonant insertion discussed in \srf{sec:VowAss ch:PhoMet} above.
Although the results are similar, assimilation of /a/ and assimilation
after consonant insertion are triggered by different factors.

\begin{exe}
	\ex{V{\sA}Ca{\#} + =V {\ra} V{\sA}V{\sA}C=V \label{ex:VCa+=V->VVC=V}}
		\sn{\gw\begin{tabular}{rlllll}
			\ve{n-bi\tbr{ba}} &+&\ve{=ee}&{\ra}&\ve{n-bi\tbr{ib}=ee}	& `massages her/him' \\
			\ve{n-ne\tbr{na}}	&+&\ve{=ee}&{\ra}&\ve{n-ne\tbr{en}=ee}	& `hears it/her/him' \\
			\ve{n-pa\tbr{ha}}	&+&\ve{=ee}&{\ra}&\ve{n-pa\tbr{ah}=ee}	& `splits it' \\
			\ve{n-so\tbr{sa}}	&+&\ve{=ee}&{\ra}&\ve{n-so\tbr{os}=ee}	& `buys it' \\
			\ve{n-su\tbr{ba}}	&+&\ve{=ee}&{\ra}&\ve{n-su\tbr{ub}=ee}	& `buries it/her/him' \\
		\end{tabular}}
\end{exe}

The lack of consonant insertion in such examples 
can be accounted for because the vowel /a/
is featureless regarding the relevant vocalic place features which spread.
The vowel /a/ is \tsc{[-front, -back, -round]} (\srf{sec:Vow}).
Thus, it can provide no features to fill a following empty C-slot.

The way this works for \ve{n-biba} + \ve{=ee} {\ra} \ve{n-biib=ee}
`massages her/him' and \ve{n\=/nena} + \ve{=ee} {\ra} \ve{n-neen=ee}
`hears her/him/it' is illustrated in \qf{as:nbiib=ee} below.
In \qf{as:nbiib=ee1} the initial C-slot of the
second foot begins is empty.
Because feet require an onset, the features
of the previous vowel spread.
However, the features of /a/ are insufficient to produce a consonant
and the onset C-slot of the enclitic remains empty in \qf{as:nbiib=ee2}.

\begin{multicols}{2}
\begin{exe}\ex{\label{as:nbiib=ee}
	\begin{xlist}
	\ex\raisebox{\dimexpr-\totalheight+6.37ex\relax}{\xy
		<3em,5cm>*\as{\hp{\sub{1}}Ft\sub{1}}="Ft1",<7em,5cm>*\as{\hp{\sub{2}}Ft\sub{2}}="Ft2",
		<2em,4cm>*\as{\hp{\sub{1}}σ\sub{1}}="s1",<4em,4cm>*\as{\hp{\sub{2}}σ\sub{2}}="s2",<6em,4cm>*\as{\hp{\sub{3}}σ\sub{3}}="s3",<8em,4cm>*\as{\hp{\sub{4}}σ\sub{4}}="s4",
		<1em,3cm>*\as{C}="CV1",<2em,3cm>*\as{V}="CV2",<3em,3cm>*\as{C}="CV3",<4em,3cm>*\as{V}="CV4",<5em,3cm>*\as{C}="CV5",
		<6em,3cm>*\as{V}="CV6",<7em,3cm>*\as{C}="CV7",<8em,3cm>*\as{V}="CV8",<9em,3cm>*\as{C}="CV9",
		<1em,2cm>*\as{nb\hp{n}}="cv1",<2em,2cm>*\as{i}="cv2",<3em,2cm>*\as{b}="cv3",<4em,2cm>*\as{a}="cv4",<5em,2cm>*\as{\0}="cv5",
		<6em,2cm>*\as{e}="cv6",<7em,2cm>*\as{ }="cv7",<8em,2cm>*\as{e}="cv8",<9em,2cm>*\as{ }="cv9",
		<1em,1.5cm>*\as{nn\hp{n}}="cv1.2",<2em,1.5cm>*\as{e}="cv2.2",<3em,1.5cm>*\as{n}="cv3.2",<4em,1.5cm>*\as{a}="cv4.2",<5em,1.5cm>*\as{\0}="cv5.2",
		<6em,1.5cm>*\as{e}="cv6.2",<7em,1.5cm>*\as{ }="cv7.2",<8em,1.5cm>*\as{e}="cv8.2",<9em,1.5cm>*\as{ }="cv9.2",
		<4em,0cm>*\as{{$\left[\hspace{-2mm}\begin{array}{l}\textrm{\tsc{-fr.}}\\\textrm{\tsc{-ba.}}\\\textrm{\tsc{-ro.}}\end{array}\hspace{-2mm}\right]$}}="f",
		"f"+U;"cv4.2"+D**\dir{-};"f"+U;"cv5.2"+D**\dir{.};%"cv5.2"+D;"CV5"+D**\dir{.};"cv4"+D;"cv5"+U**\dir{.};
		"cv1"+U;"CV1"+D**\dir{-};"cv2"+U;"CV2"+D**\dir{-};"cv3"+U;"CV3"+D**\dir{-};"cv4"+U;"CV4"+D**\dir{-};"cv5"+U;"CV5"+D**\dir{-};
		"cv6"+U;"CV6"+D**\dir{-};"cv7"+U;"CV7"+D**\dir{};"cv8"+U;"CV8"+D**\dir{-};"cv9"+U;"CV9"+D**\dir{};
		"CV1"+U;"s1"+D**\dir{-};"CV2"+U;"s1"+D**\dir{-};"CV3"+U;"s1"+D**\dir{-};"CV3"+U;"s2"+D**\dir{-};"CV4"+U;"s2"+D**\dir{-};"CV5"+U;"s2"+D**\dir{-};
		"CV5"+U;"s3"+D**\dir{-};"CV6"+U;"s3"+D**\dir{-};"CV7"+U;"s3"+D**\dir{-};"CV7"+U;"s4"+D**\dir{-};"CV8"+U;"s4"+D**\dir{-};"CV9"+U;"s4"+D**\dir{-};
		"s1"+U;"Ft1"+D**\dir{-};"s2"+U;"Ft1"+D**\dir{-};"s3"+U;"Ft2"+D**\dir{-};"s4"+U;"Ft2"+D**\dir{-};
		<4.5em,2.25cm>*\as{\tikz[red,thick,dashed,baseline=0.9ex]\draw (0,0) rectangle (0.8cm,2cm);}="box",
	\endxy}\label{as:nbiib=ee1}
	\ex\raisebox{\dimexpr-\totalheight+6.37ex\relax}{\xy
		<3em,5cm>*\as{\hp{\sub{1}}Ft\sub{1}}="Ft1",<7em,5cm>*\as{\hp{\sub{2}}Ft\sub{2}}="Ft2",
		<2em,4cm>*\as{\hp{\sub{1}}σ\sub{1}}="s1",<4em,4cm>*\as{\hp{\sub{2}}σ\sub{2}}="s2",<6em,4cm>*\as{\hp{\sub{3}}σ\sub{3}}="s3",<8em,4cm>*\as{\hp{\sub{4}}σ\sub{4}}="s4",
		<1em,3cm>*\as{C}="CV1",<2em,3cm>*\as{V}="CV2",<3em,3cm>*\as{C}="CV3",<4em,3cm>*\as{V}="CV4",<5em,3cm>*\as{C}="CV5",
		<6em,3cm>*\as{V}="CV6",<7em,3cm>*\as{C}="CV7",<8em,3cm>*\as{V}="CV8",<9em,3cm>*\as{C}="CV9",
		<1em,2cm>*\as{nb\hp{n}}="cv1",<2em,2cm>*\as{i}="cv2",<3em,2cm>*\as{b}="cv3",<4em,2cm>*\as{a}="cv4",<5em,2cm>*\as{\0}="cv5",
		<6em,2cm>*\as{e}="cv6",<7em,2cm>*\as{ }="cv7",<8em,2cm>*\as{e}="cv8",<9em,2cm>*\as{ }="cv9",
		<1em,1.5cm>*\as{nn\hp{n}}="cv1.2",<2em,1.5cm>*\as{e}="cv2.2",<3em,1.5cm>*\as{n}="cv3.2",<4em,1.5cm>*\as{a}="cv4.2",<5em,1.5cm>*\as{\0}="cv5.2",
		<6em,1.5cm>*\as{e}="cv6.2",<7em,1.5cm>*\as{ }="cv7.2",<8em,1.5cm>*\as{e}="cv8.2",<9em,1.5cm>*\as{ }="cv9.2",
		<4.5em,0cm>*\as{{$\left[\hspace{-2mm}\begin{array}{l}\textrm{\tsc{-fr.}}\\\textrm{\tsc{-ba.}}\\\textrm{\tsc{-ro.}}\end{array}\hspace{-2mm}\right]$}}="f",
		"f"+U;"cv4.2"+D**\dir{-};"f"+U;"cv5.2"+D**\dir{-};
		"cv1"+U;"CV1"+D**\dir{-};"cv2"+U;"CV2"+D**\dir{-};"cv3"+U;"CV3"+D**\dir{-};"cv4"+U;"CV4"+D**\dir{-};"cv5"+U;"CV5"+D**\dir{-};
		"cv6"+U;"CV6"+D**\dir{-};"cv7"+U;"CV7"+D**\dir{};"cv8"+U;"CV8"+D**\dir{-};"cv9"+U;"CV9"+D**\dir{};
		"CV1"+U;"s1"+D**\dir{-};"CV2"+U;"s1"+D**\dir{-};"CV3"+U;"s1"+D**\dir{-};"CV3"+U;"s2"+D**\dir{-};"CV4"+U;"s2"+D**\dir{-};"CV5"+U;"s2"+D**\dir{-};
		"CV5"+U;"s3"+D**\dir{-};"CV6"+U;"s3"+D**\dir{-};"CV7"+U;"s3"+D**\dir{-};"CV7"+U;"s4"+D**\dir{-};"CV8"+U;"s4"+D**\dir{-};"CV9"+U;"s4"+D**\dir{-};
		"s1"+U;"Ft1"+D**\dir{-};"s2"+U;"Ft1"+D**\dir{-};"s3"+U;"Ft2"+D**\dir{-};"s4"+U;"Ft2"+D**\dir{-};
		<5em,2.25cm>*\as{\tikz[red,thick,dashed,baseline=0.9ex]\draw (0,0) rectangle (0.4cm,2cm);}="box",
	\endxy}\label{as:nbiib=ee2}
	\end{xlist}}
\end{exe}
\end{multicols}

Because this empty C-slot is shared between two prosodic words,
as shown in (\ref{as:nbiib=ee}d),
metathesis is triggered to resolve the fuzzy border.
This produces a crisp edge after the internal prosodic word,
as illustrated in (\ref{as:nbiib=ee}e).

\begin{multicols}{2}
\begin{exe}\exr{as:nbiib=ee}{
	\begin{xlist}
		\exi{d.}\exia{\xy
		<4em,5.5cm>*\as{\hp{\sub{1}}PrWd\sub{1}}="PrWd1",<5.5em,6.5cm>*\as{\hp{\sub{2}}PrWd\sub{2}}="PrWd2",
		<4em,4.5cm>*\as{\hp{\sub{1}}Ft\sub{1}}="Ft1",<8em,4.5cm>*\as{\hp{\sub{2}}Ft\sub{2}}="Ft2",
		<3em,3.5cm>*\as{\hp{\sub{1}}σ\sub{1}}="s1",<5em,3.5cm>*\as{\hp{\sub{2}}σ\sub{2}}="s2",<7em,3.5cm>*\as{\hp{\sub{3}}σ\sub{3}}="s3",<9em,3.5cm>*\as{\hp{\sub{4}}σ\sub{4}}="s4",
		<1em,2.5cm>*\as{C}="CV1",<2em,2.5cm>*\as{C}="CV2",<3em,2.5cm>*\as{V}="CV3",<4em,2.5cm>*\as{C}="CV4",<5em,2.5cm>*\as{V}="CV5",<6em,2.5cm>*\as{C}="CV6",
		<7em,2.5cm>*\as{V}="CV7",<8em,2.5cm>*\as{C}="CV8",<9em,2.5cm>*\as{V}="CV9",<10em,2.5cm>*\as{C}="CV10",
		<1em,1.5cm>*\as{n}="cv1",<2em,1.5cm>*\as{b}="cv2",<3em,1.5cm>*\as{i}="cv3",<4em,1.5cm>*\as{b}="cv4",<5em,1.5cm>*\as{a}="cv5",<6em,1.5cm>*\as{\0}="cv6",
		<7em,1.5cm>*\as{e}="cv7",<8em,1.5cm>*\as{ }="cv8",<9em,1.5cm>*\as{e}="cv9",<10em,1.5cm>*\as{ }="cv10",
		<1em,1cm>*\as{n}="cv1.2",<2em,1cm>*\as{n}="cv2.2",<3em,1cm>*\as{e}="cv3.2",<4em,1cm>*\as{n}="cv4.2",<5em,1cm>*\as{a}="cv5.2",<6em,1cm>*\as{\0}="cv6.2",
		<7em,1cm>*\as{e}="cv7.2",<8em,1cm>*\as{ }="cv8.2",<9em,1cm>*\as{e}="cv9.2",<10em,1cm>*\as{ }="cv10.2",
		<1em,0cm>*\as{\hp{\sub{1}}M\sub{1}}="m1",<3.5em,0cm>*\as{\hp{\sub{2}}M\sub{2}}="m2",<8em,0cm>*\as{\hp{\sub{3}}M\sub{3}}="m3",
		<2em,0cm>*\as{-}="-",<6em,0cm>*\as{=}="=",
		"m1"+U;"cv1.2"+D**\dir{-};"m2"+U;"cv2.2"+D**\dir{-};"m2"+U;"cv3.2"+D**\dir{-};"m2"+U;"cv4.2"+D**\dir{-};"m2"+U;"cv5.2"+D**\dir{-};"m3"+U;"cv7.2"+D**\dir{-};"m3"+U;"cv9.2"+D**\dir{-};
		"cv1"+U;"CV1"+D**\dir{-};"cv2"+U;"CV2"+D**\dir{-};"cv3"+U;"CV3"+D**\dir{-};"cv4"+U;"CV4"+D**\dir{-};"cv5"+U;"CV5"+D**\dir{-};"cv6"+U;"CV6"+D**\dir{-};
		"cv7"+U;"CV7"+D**\dir{-};"cv8"+U;"CV8"+D**\dir{};"cv9"+U;"CV9"+D**\dir{-};"cv10"+U;"CV10"+D**\dir{};
		"CV2"+U;"s1"+D**\dir{-};"CV3"+U;"s1"+D**\dir{-};"CV4"+U;"s1"+D**\dir{-};"CV4"+U;"s2"+D**\dir{-};"CV5"+U;"s2"+D**\dir{-};"CV6"+U;"s2"+D**\dir{-};
		"CV6"+U;"s3"+D**\dir{-};"CV7"+U;"s3"+D**\dir{-};"CV8"+U;"s3"+D**\dir{-};"CV8"+U;"s4"+D**\dir{-};"CV9"+U;"s4"+D**\dir{-};"CV10"+U;"s4"+D**\dir{-};
		"s1"+U;"Ft1"+D**\dir{-};"s2"+U;"Ft1"+D**\dir{-};"s3"+U;"Ft2"+D**\dir{-};"s4"+U;"Ft2"+D**\dir{-};
		"CV1"+U;"PrWd1"+D**\dir{-};"Ft1"+U;"PrWd1"+D**\dir{-};"PrWd1"+U;"PrWd2"+D**\dir{-};"Ft2"+U;"PrWd2"+D**\dir{-};
		<6em,1.75cm>*\as{\tikz[red,thick,dashed,baseline=0.9ex]\draw (0,0) rectangle (0.4cm,2cm);}="box",
	\endxy}
	\exi{e.}\exia{\xy
		<3.5em,5.5cm>*\as{\hp{\sub{1}}PrWd\sub{1}}="PrWd1",<5.5em,6.5cm>*\as{\hp{\sub{2}}PrWd\sub{2}}="PrWd2",
		<3.5em,4.5cm>*\as{\hp{\sub{\tsc{m}}}Ft\sub{\tsc{m}}}="Ft1",<8em,4.5cm>*\as{\hp{\sub{2}}Ft\sub{2}}="Ft2",
		<2.5em,3.5cm>*\as{\hp{\sub{1}}σ\sub{1}}="s1",<4.5em,3.5cm>*\as{\hp{\sub{2}}σ\sub{2}}="s2",<7em,3.5cm>*\as{\hp{\sub{3}}σ\sub{3}}="s3",<9em,3.5cm>*\as{\hp{\sub{4}}σ\sub{4}}="s4",
		<1em,2.5cm>*\as{C}="CV1",<2em,2.5cm>*\as{C}="CV2",<3em,2.5cm>*\as{V}="CV3",<4em,2.5cm>*\as{V}="CV4",<5em,2.5cm>*\as{C}="CV5",<6em,2.5cm>*\as{C}="CV6",
		<7em,2.5cm>*\as{V}="CV7",<8em,2.5cm>*\as{C}="CV8",<9em,2.5cm>*\as{V}="CV9",<10em,2.5cm>*\as{C}="CV10",
		<1em,1.5cm>*\as{n}="cv1",<2em,1.5cm>*\as{b}="cv2",<3em,1.5cm>*\as{i}="cv3",<4em,1.5cm>*\as{a}="cv4",<5em,1.5cm>*\as{b}="cv5",<6em,1.5cm>*\as{\0}="cv6",
		<7em,1.5cm>*\as{e}="cv7",<8em,1.5cm>*\as{ }="cv8",<9em,1.5cm>*\as{e}="cv9",<10em,1.5cm>*\as{ }="cv10",
		<1em,1cm>*\as{n}="cv1.2",<2em,1cm>*\as{n}="cv2.2",<3em,1cm>*\as{e}="cv3.2",<4em,1cm>*\as{a}="cv4.2",<5em,1cm>*\as{n}="cv5.2",<6em,1cm>*\as{\0}="cv6.2",
		<7em,1cm>*\as{e}="cv7.2",<8em,1cm>*\as{ }="cv8.2",<9em,1cm>*\as{e}="cv9.2",<10em,1cm>*\as{ }="cv10.2",
		<1em,0cm>*\as{\hp{\sub{1}}M\sub{1}}="m1",<3.5em,0cm>*\as{\hp{\sub{2}}M\sub{2}}="m2",<8em,0cm>*\as{\hp{\sub{3}}M\sub{3}}="m3",
		<2em,0cm>*\as{-}="-",<6em,0cm>*\as{=}="=",
		"m1"+U;"cv1.2"+D**\dir{-};"m2"+U;"cv2.2"+D**\dir{-};"m2"+U;"cv3.2"+D**\dir{-};"m2"+U;"cv4.2"+D**\dir{-};"m2"+U;"cv5.2"+D**\dir{-};"m3"+U;"cv7.2"+D**\dir{-};"m3"+U;"cv9.2"+D**\dir{-};
		"cv1"+U;"CV1"+D**\dir{-};"cv2"+U;"CV2"+D**\dir{-};"cv3"+U;"CV3"+D**\dir{-};"cv4"+U;"CV4"+D**\dir{-};"cv5"+U;"CV5"+D**\dir{-};"cv6"+U;"CV6"+D**\dir{-};
		"cv7"+U;"CV7"+D**\dir{-};"cv8"+U;"CV8"+D**\dir{};"cv9"+U;"CV9"+D**\dir{-};"cv10"+U;"CV10"+D**\dir{};
		"CV2"+U;"s1"+D**\dir{-};"CV3"+U;"s1"+D**\dir{-};"CV4"+U;"s2"+D**\dir{-};"CV5"+U;"s2"+D**\dir{-};
		"CV6"+U;"s3"+D**\dir{-};"CV7"+U;"s3"+D**\dir{-};"CV8"+U;"s3"+D**\dir{-};"CV8"+U;"s4"+D**\dir{-};"CV9"+U;"s4"+D**\dir{-};"CV10"+U;"s4"+D**\dir{-};
		"s1"+U;"Ft1"+D**\dir{-};"s2"+U;"Ft1"+D**\dir{-};"s3"+U;"Ft2"+D**\dir{-};"s4"+U;"Ft2"+D**\dir{-};
		"CV1"+U;"PrWd1"+D**\dir{-};"Ft1"+U;"PrWd1"+D**\dir{-};"PrWd1"+U;"PrWd2"+D**\dir{-};"Ft2"+U;"PrWd2"+D**\dir{-};
		<5.5em,3.25cm>*\as{\tikz[red,thick,dashed,baseline=0.9ex]\draw (0,0) -- (0,5cm);}="line",
	\endxy}
	\end{xlist}}
\end{exe}
\end{multicols}

The features of the penultimate vowel of the M-foot (metathesised foot)
then spread in (\ref{as:nbiib=ee}f) due to the morphemically
conditioned rule of /a/ assimilation (\srf{sec:MorRulAssOfA}).
This produces the final outputs with double vowels in (\ref{as:nbiib=ee}g).

\begin{multicols}{2}
\begin{exe}\exr{as:nbiib=ee}{
	\begin{xlist}
		\exi{f.}\exia{\xy
		<3.5em,5.5cm>*\as{\hp{\sub{1}}PrWd\sub{1}}="PrWd1",<5.5em,6.5cm>*\as{\hp{\sub{2}}PrWd\sub{2}}="PrWd2",
		<3.5em,4.5cm>*\as{\hp{\sub{\tsc{m}}}Ft\sub{\tsc{m}}}="Ft1",<8em,4.5cm>*\as{\hp{\sub{2}}Ft\sub{2}}="Ft2",
		<2.5em,3.5cm>*\as{\hp{\sub{1}}σ\sub{1}}="s1",<4.5em,3.5cm>*\as{\hp{\sub{2}}σ\sub{2}}="s2",<7em,3.5cm>*\as{\hp{\sub{3}}σ\sub{3}}="s3",<9em,3.5cm>*\as{\hp{\sub{4}}σ\sub{4}}="s4",
		<1em,2.5cm>*\as{C}="CV1",<2em,2.5cm>*\as{C}="CV2",<3em,2.5cm>*\as{V}="CV3",<4em,2.5cm>*\as{V}="CV4",<5em,2.5cm>*\as{C}="CV5",<6em,2.5cm>*\as{C}="CV6",
		<7em,2.5cm>*\as{V}="CV7",<8em,2.5cm>*\as{C}="CV8",<9em,2.5cm>*\as{V}="CV9",<10em,2.5cm>*\as{C}="CV10",
		<1em,1.5cm>*\as{n}="cv1",<2em,1.5cm>*\as{b}="cv2",<3em,1.5cm>*\as{i}="cv3",<4em,1.5cm>*\as{a}="cv4",<5em,1.5cm>*\as{b}="cv5",<6em,1.5cm>*\as{\0}="cv6",
		<7em,1.5cm>*\as{e}="cv7",<8em,1.5cm>*\as{ }="cv8",<9em,1.5cm>*\as{e}="cv9",<10em,1.5cm>*\as{ }="cv10",
		<1em,1cm>*\as{n}="cv1.2",<2em,1cm>*\as{n}="cv2.2",<3em,1cm>*\as{e}="cv3.2",<4em,1cm>*\as{a}="cv4.2",<5em,1cm>*\as{n}="cv5.2",<6em,1cm>*\as{\0}="cv6.2",
		<7em,1cm>*\as{e}="cv7.2",<8em,1cm>*\as{ }="cv8.2",<9em,1cm>*\as{e}="cv9.2",<10em,1cm>*\as{ }="cv10.2",
		<3em,0cm>*\as{\tsc{[+fr.]}}="f1","f1"+U;"cv3.2"+D**\dir{-};"f1"+U;"cv4.2"+D**\dir{.};
		"cv1"+U;"CV1"+D**\dir{-};"cv2"+U;"CV2"+D**\dir{-};"cv3"+U;"CV3"+D**\dir{-};"cv4"+U;"CV4"+D**\dir{-};"cv5"+U;"CV5"+D**\dir{-};"cv6"+U;"CV6"+D**\dir{-};
		"cv7"+U;"CV7"+D**\dir{-};"cv8"+U;"CV8"+D**\dir{};"cv9"+U;"CV9"+D**\dir{-};"cv10"+U;"CV10"+D**\dir{};
		"CV2"+U;"s1"+D**\dir{-};"CV3"+U;"s1"+D**\dir{-};"CV4"+U;"s2"+D**\dir{-};"CV5"+U;"s2"+D**\dir{-};
		"CV6"+U;"s3"+D**\dir{-};"CV7"+U;"s3"+D**\dir{-};"CV8"+U;"s3"+D**\dir{-};"CV8"+U;"s4"+D**\dir{-};"CV9"+U;"s4"+D**\dir{-};"CV10"+U;"s4"+D**\dir{-};
		"s1"+U;"Ft1"+D**\dir{-};"s2"+U;"Ft1"+D**\dir{-};"s3"+U;"Ft2"+D**\dir{-};"s4"+U;"Ft2"+D**\dir{-};
		"CV1"+U;"PrWd1"+D**\dir{-};"Ft1"+U;"PrWd1"+D**\dir{-};"PrWd1"+U;"PrWd2"+D**\dir{-};"Ft2"+U;"PrWd2"+D**\dir{-};
		<3.5em,1cm>*\as{\tikz[red,thick,dashed,baseline=0.9ex]\draw (0,0) rectangle (0.8cm,1.5cm);}="box",
	\endxy}
	\exi{g.}\exia{\xy
		<3.5em,5.5cm>*\as{\hp{\sub{1}}PrWd\sub{1}}="PrWd1",<5.5em,6.5cm>*\as{\hp{\sub{2}}PrWd\sub{2}}="PrWd2",
		<3.5em,4.5cm>*\as{\hp{\sub{\tsc{m}}}Ft\sub{\tsc{m}}}="Ft1",<8em,4.5cm>*\as{\hp{\sub{2}}Ft\sub{2}}="Ft2",
		<2.5em,3.5cm>*\as{\hp{\sub{1}}σ\sub{1}}="s1",<4.5em,3.5cm>*\as{\hp{\sub{2}}σ\sub{2}}="s2",<7em,3.5cm>*\as{\hp{\sub{3}}σ\sub{3}}="s3",<9em,3.5cm>*\as{\hp{\sub{4}}σ\sub{4}}="s4",
		<1em,2.5cm>*\as{C}="CV1",<2em,2.5cm>*\as{C}="CV2",<3em,2.5cm>*\as{V}="CV3",<4em,2.5cm>*\as{V}="CV4",<5em,2.5cm>*\as{C}="CV5",<6em,2.5cm>*\as{C}="CV6",
		<7em,2.5cm>*\as{V}="CV7",<8em,2.5cm>*\as{C}="CV8",<9em,2.5cm>*\as{V}="CV9",<10em,2.5cm>*\as{C}="CV10",
		<1em,1.5cm>*\as{n}="cv1",<2em,1.5cm>*\as{b}="cv2",<3em,1.5cm>*\as{i}="cv3",<4em,1.5cm>*\as{i}="cv4",<5em,1.5cm>*\as{b}="cv5",<6em,1.5cm>*\as{\0}="cv6",<7em,1.5cm>*\as{e}="cv7",<8em,1.5cm>*\as{ }="cv8",<9em,1.5cm>*\as{e}="cv9",<10em,1.5cm>*\as{ }="cv10",
		<1em,1cm>*\as{n}="cv1.2",<2em,1cm>*\as{n}="cv2.2",<3em,1cm>*\as{e}="cv3.2",<4em,1cm>*\as{e}="cv4.2",<5em,1cm>*\as{n}="cv5.2",<6em,1cm>*\as{\0}="cv6.2",<7em,1cm>*\as{e}="cv7.2",<8em,1cm>*\as{ }="cv8.2",<9em,1cm>*\as{e}="cv9.2",<10em,1cm>*\as{ }="cv10.2",
		<3.5em,0cm>*\as{\tsc{[+fr.]}}="f1","f1"+U;"cv3.2"+D**\dir{-};"f1"+U;"cv4.2"+D**\dir{-};
		"cv1"+U;"CV1"+D**\dir{-};"cv2"+U;"CV2"+D**\dir{-};"cv3"+U;"CV3"+D**\dir{-};"cv4"+U;"CV4"+D**\dir{-};"cv5"+U;"CV5"+D**\dir{-};"cv6"+U;"CV6"+D**\dir{-};
		"cv7"+U;"CV7"+D**\dir{-};"cv8"+U;"CV8"+D**\dir{};"cv9"+U;"CV9"+D**\dir{-};"cv10"+U;"CV10"+D**\dir{};
		"CV2"+U;"s1"+D**\dir{-};"CV3"+U;"s1"+D**\dir{-};"CV4"+U;"s2"+D**\dir{-};"CV5"+U;"s2"+D**\dir{-};
		"CV6"+U;"s3"+D**\dir{-};"CV7"+U;"s3"+D**\dir{-};"CV8"+U;"s3"+D**\dir{-};"CV8"+U;"s4"+D**\dir{-};"CV9"+U;"s4"+D**\dir{-};"CV10"+U;"s4"+D**\dir{-};
		"s1"+U;"Ft1"+D**\dir{-};"s2"+U;"Ft1"+D**\dir{-};"s3"+U;"Ft2"+D**\dir{-};"s4"+U;"Ft2"+D**\dir{-};
		"CV1"+U;"PrWd1"+D**\dir{-};"Ft1"+U;"PrWd1"+D**\dir{-};"PrWd1"+U;"PrWd2"+D**\dir{-};"Ft2"+U;"PrWd2"+D**\dir{-};
		%<5.5em,3.25cm>*\as{\tikz[red,thick,dashed,baseline=0.9ex]\draw (0,0) -- (0,5cm);}="line",
	\endxy}
	\end{xlist}}
\end{exe}
\end{multicols}

Metathesis before vowel-initial enclitics operates at the consonant-vowel tier.
It is blind to the contents of the C-slots and V-slots.
Thus, that the C-slot shared between the clitic hosts and enclitic is empty
in \qf{as:nbiib=ee} is irrelevant, or unseen, by the constraint requiring a crisp edge.

Nonetheless, metathesis is still somewhat successful in creating a crisp edge.
A word such as \ve{n-biib=ee}, in which the clitic host ends in a surface consonant,
arguably has a greater phonological separation between host and enclitic
than potential \ve{\tcb{*}n-biba=ee} in which the host ends in a surface vowel.

\subsection{Clitic hosts with final /Va/}
After stems which end in Va{\#},
/ɡw/ is inserted, but vowel assimilation does not take place.
Examples are given in \qf{ex:Va+=V->Vagw=V} below.

\begin{exe}
	\ex{Va+=V {\ra} Vagw=V \label{ex:Va+=V->Vagw=V}}
		\sn{\gw\begin{tabular}{rlllll}
			\ve{mria}		&+&\ve{=een}&{\ra}&\ve{na-mria\tbr{gw}=een}	& `fertile, lush' \\
			\ve{tea}		&+&\ve{=een}&{\ra}&\ve{n-tea\tbr{gw}=een}		& `arrived' \\
			\ve{haa}		&+&\ve{=een}&{\ra}&\ve{haa\tbr{gw}=een}			& `four' \\
			\ve{tua}		&+&\ve{=ee}	&{\ra}&\ve{na-tua\tbr{gw}=ee}		& `occupies it' \\
			\ve{pentua}	&+&\ve{=ee}	&{\ra}&\ve{pentua\tbr{gw}=ee}		& `the church elder' \\
		\end{tabular}}
\end{exe}

The consonant /ɡw/ is inserted in this environment
to break up the underlying sequence of three vowels.
However, the reason /ɡw/ is inserted rather than /ʔ/
not fully explained in more or less the same
way that insertion of /\j/ or /ɡw/ at clitic boundaries
is not fully explained (see the discussion on \prf{WhyNotGlottal?}).

Nonetheless, because /ɡw/ is not inserted in such
examples to provide an onset consonant, but rather
to break up a sequence of three vowels,
identifying it as the default foot-final consonant is probably the best solution.
My analysis is illustrated in \qf{as:natuagw=ee}
below, which is followed by a discussion of this
analysis and its possible implications.

The underlying structure of \ve{na-tua} + \ve{=ee}
{\ra} \ve{na-tuagw=ee} `occupies it' is given in  \qf{as:natuagw=ee1}.
This form has a sequence of three vowels.
As a result consonant insertion occurs to resolve this disallowed sequence.
Because this C-slot is the final C-slot of the initial foot,
the default final consonant /ɡw/ is inserted in \qf{as:natuagw=ee2}.
This C-slot is also shared with the following foot.
As a result, metathesis then occurs to resolve the fuzzy border after
the internal prosodic word, producing the final output in (\ref{as:natuagw=ee}c).

Insertion of /ɡw/ in \qf{as:natuagw=ee} and parallel forms occurs primarily
to resolve a disallowed sequence of three vowels.
However, it also has the added benefit of
providing the enclitic with an onset consonant.

\begin{multicols}{2}
\begin{exe}\ex{\label{as:natuagw=ee}
	\begin{xlist}
	\exa{\xy
		<5em,5cm>*\as{\hp{\sub{1}}PrWd\sub{1}}="PrWd1",<7em,6cm>*\as{\hp{\sub{2}}PrWd\sub{2}}="PrWd2",
		<5em,4cm>*\as{\hp{\sub{1}}Ft\sub{1}}="Ft1",<9em,4cm>*\as{\hp{\sub{2}}Ft\sub{2}}="Ft2",
		<2em,3cm>*\as{\hp{\sub{1}}σ\sub{1}}="s1",<4em,3cm>*\as{\hp{\sub{2}}σ\sub{2}}="s2",<6em,3cm>*\as{\hp{\sub{3}}σ\sub{3}}="s3",<8em,3cm>*\as{\hp{\sub{4}}σ\sub{4}}="s4",<10em,3cm>*\as{\hp{\sub{5}}σ\sub{5}}="s5",
		<1em,2cm>*\as{C}="CV1",<2em,2cm>*\as{V}="CV2",<3em,2cm>*\as{C}="CV3",<4em,2cm>*\as{V}="CV4",<5em,2cm>*\as{C}="CV5",<6em,2cm>*\as{V}="CV6",<7em,2cm>*\as{C}="CV7",<8em,2cm>*\as{V}="CV8",<9em,2cm>*\as{C}="CV9",<10em,2cm>*\as{V}="CV10",<11em,2cm>*\as{C}="CV11",
		<1em,1cm>*\as{n}="cv1",<2em,1cm>*\as{a}="cv2",<3em,1cm>*\as{t}="cv3",<4em,1cm>*\as{u}="cv4",<5em,1cm>*\as{ }="cv5",<6em,1cm>*\as{a}="cv6",<7em,1cm>*\as{ }="cv7",<8em,1cm>*\as{e}="cv8",<9em,1cm>*\as{ }="cv9",<10em,1cm>*\as{e}="cv10",<11em,1cm>*\as{ }="cv11",
		<1.5em,0cm>*\as{\hp{\sub{1}}M\sub{1}}="m1",<4.5em,0cm>*\as{\hp{\sub{2}}M\sub{2}}="m2",<9em,0cm>*\as{\hp{\sub{3}}M\sub{3}}="m3",<2.5em,0cm>*\as{-}="-",<7em,0cm>*\as{=}="=",
		"m1"+U;"cv1"+D**\dir{-};"m1"+U;"cv2"+D**\dir{-};"m2"+U;"cv3"+D**\dir{-};"m2"+U;"cv4"+D**\dir{-};"m2"+U;"cv6"+D**\dir{-};"m3"+U;"cv8"+D**\dir{-};"m3"+U;"cv10"+D**\dir{-};
		"cv1"+U;"CV1"+D**\dir{-};"cv2"+U;"CV2"+D**\dir{-};"cv3"+U;"CV3"+D**\dir{-};"cv4"+U;"CV4"+D**\dir{-};"cv5"+U;"CV5"+D**\dir{};"cv6"+U;"CV6"+D**\dir{-};"cv7"+U;"CV7"+D**\dir{};"cv8"+U;"CV8"+D**\dir{-};"cv9"+U;"CV9"+D**\dir{};"cv10"+U;"CV10"+D**\dir{-};"cv11"+U;"CV11"+D**\dir{};
		"CV1"+U;"s1"+D**\dir{-};"CV2"+U;"s1"+D**\dir{-};"CV3"+U;"s1"+D**\dir{-};"CV3"+U;"s2"+D**\dir{-};"CV4"+U;"s2"+D**\dir{-};"CV5"+U;"s2"+D**\dir{-};
		"CV5"+U;"s3"+D**\dir{-};"CV6"+U;"s3"+D**\dir{-};"CV7"+U;"s3"+D**\dir{-};"CV7"+U;"s4"+D**\dir{-};"CV8"+U;"s4"+D**\dir{-};"CV9"+U;"s4"+D**\dir{-};
		"CV9"+U;"s5"+D**\dir{-};"CV10"+U;"s5"+D**\dir{-};"CV11"+U;"s5"+D**\dir{-};
		"s2"+U;"Ft1"+D**\dir{-};"s3"+U;"Ft1"+D**\dir{-};"s4"+U;"Ft2"+D**\dir{-};"s5"+U;"Ft2"+D**\dir{-};
		"s1"+U;"PrWd1"+D**\dir{-};"Ft1"+U;"PrWd1"+D**\dir{-};"PrWd1"+U;"PrWd2"+D**\dir{-};"Ft2"+U;"PrWd2"+D**\dir{-};
		<6em,1.5cm>*\as{\tikz[red,thick,dashed,baseline=0.9ex]\draw (0,0) rectangle (2cm,1.5cm);}="box",
	\endxy}\label{as:natuagw=ee1}
	\exa{\xy
		<5em,5cm>*\as{\hp{\sub{1}}PrWd\sub{1}}="PrWd1",<7em,6cm>*\as{\hp{\sub{2}}PrWd\sub{2}}="PrWd2",
		<5em,4cm>*\as{\hp{\sub{1}}Ft\sub{1}}="Ft1",<9em,4cm>*\as{\hp{\sub{2}}Ft\sub{2}}="Ft2",
		<2em,3cm>*\as{\hp{\sub{1}}σ\sub{1}}="s1",<4em,3cm>*\as{\hp{\sub{2}}σ\sub{2}}="s2",<6em,3cm>*\as{\hp{\sub{3}}σ\sub{3}}="s3",<8em,3cm>*\as{\hp{\sub{4}}σ\sub{4}}="s4",<10em,3cm>*\as{\hp{\sub{5}}σ\sub{5}}="s5",
		<1em,2cm>*\as{C}="CV1",<2em,2cm>*\as{V}="CV2",<3em,2cm>*\as{C}="CV3",<4em,2cm>*\as{V}="CV4",<5em,2cm>*\as{C}="CV5",<6em,2cm>*\as{V}="CV6",<7em,2cm>*\as{C}="CV7",<8em,2cm>*\as{V}="CV8",<9em,2cm>*\as{C}="CV9",<10em,2cm>*\as{V}="CV10",<11em,2cm>*\as{C}="CV11",
		<1em,1cm>*\as{n}="cv1",<2em,1cm>*\as{a}="cv2",<3em,1cm>*\as{t}="cv3",<4em,1cm>*\as{u}="cv4",<5em,1cm>*\as{ }="cv5",<6em,1cm>*\as{a}="cv6",<7em,1cm>*\as{ɡw}="cv7",<8em,1cm>*\as{e}="cv8",<9em,1cm>*\as{ }="cv9",<10em,1cm>*\as{e}="cv10",<11em,1cm>*\as{ }="cv11",
		<1.5em,0cm>*\as{\hp{\sub{1}}M\sub{1}}="m1",<4.5em,0cm>*\as{\hp{\sub{2}}M\sub{2}}="m2",<9em,0cm>*\as{\hp{\sub{3}}M\sub{3}}="m3",<2.5em,0cm>*\as{-}="-",<7em,0cm>*\as{=}="=",
		"m1"+U;"cv1"+D**\dir{-};"m1"+U;"cv2"+D**\dir{-};"m2"+U;"cv3"+D**\dir{-};"m2"+U;"cv4"+D**\dir{-};"m2"+U;"cv6"+D**\dir{-};"m3"+U;"cv8"+D**\dir{-};"m3"+U;"cv10"+D**\dir{-};
		"cv1"+U;"CV1"+D**\dir{-};"cv2"+U;"CV2"+D**\dir{-};"cv3"+U;"CV3"+D**\dir{-};"cv4"+U;"CV4"+D**\dir{-};"cv5"+U;"CV5"+D**\dir{};"cv6"+U;"CV6"+D**\dir{-};"cv7"+U;"CV7"+D**\dir{-};"cv8"+U;"CV8"+D**\dir{-};"cv9"+U;"CV9"+D**\dir{};"cv10"+U;"CV10"+D**\dir{-};"cv11"+U;"CV11"+D**\dir{};
		"CV1"+U;"s1"+D**\dir{-};"CV2"+U;"s1"+D**\dir{-};"CV3"+U;"s1"+D**\dir{-};"CV3"+U;"s2"+D**\dir{-};"CV4"+U;"s2"+D**\dir{-};"CV5"+U;"s2"+D**\dir{-};
		"CV5"+U;"s3"+D**\dir{-};"CV6"+U;"s3"+D**\dir{-};"CV7"+U;"s3"+D**\dir{-};"CV7"+U;"s4"+D**\dir{-};"CV8"+U;"s4"+D**\dir{-};"CV9"+U;"s4"+D**\dir{-};
		"CV9"+U;"s5"+D**\dir{-};"CV10"+U;"s5"+D**\dir{-};"CV11"+U;"s5"+D**\dir{-};
		"s2"+U;"Ft1"+D**\dir{-};"s3"+U;"Ft1"+D**\dir{-};"s4"+U;"Ft2"+D**\dir{-};"s5"+U;"Ft2"+D**\dir{-};
		"s1"+U;"PrWd1"+D**\dir{-};"Ft1"+U;"PrWd1"+D**\dir{-};"PrWd1"+U;"PrWd2"+D**\dir{-};"Ft2"+U;"PrWd2"+D**\dir{-};
		<7em,1.5cm>*\as{\tikz[red,thick,dashed,baseline=0.9ex]\draw (0,0) rectangle (0.525cm,1.5cm);}="box",
	\endxy}\label{as:natuagw=ee2}
	\end{xlist}}
\end{exe}
\end{multicols}

\begin{exe}\sn{
	\begin{xlist}
	\exi{c.}\exia{\xy
		<4.5em,5cm>*\as{\hp{\sub{1}}PrWd\sub{1}}="PrWd1",<6.75em,6cm>*\as{\hp{\sub{2}}PrWd\sub{2}}="PrWd2",
		<4.55em,4cm>*\as{\hp{\sub{1}}Ft\sub{1}}="Ft1",<9em,4cm>*\as{\hp{\sub{2}}Ft\sub{2}}="Ft2",
		<2em,3cm>*\as{\hp{\sub{1}}σ\sub{1}}="s1",<3.5em,3cm>*\as{\hp{\sub{2}}σ\sub{2}}="s2",<5.5em,3cm>*\as{\hp{\sub{3}}σ\sub{3}}="s3",<8em,3cm>*\as{\hp{\sub{4}}σ\sub{4}}="s4",<10em,3cm>*\as{\hp{\sub{5}}σ\sub{5}}="s5",
		<1em,2cm>*\as{C}="CV1",<2em,2cm>*\as{V}="CV2",<3em,2cm>*\as{C}="CV3",<4em,2cm>*\as{V}="CV4",<5em,2cm>*\as{V}="CV5",<6em,2cm>*\as{C}="CV6",<7em,2cm>*\as{C}="CV7",<8em,2cm>*\as{V}="CV8",<9em,2cm>*\as{C}="CV9",<10em,2cm>*\as{V}="CV10",<11em,2cm>*\as{C}="CV11",
		<1em,1cm>*\as{n}="cv1",<2em,1cm>*\as{a}="cv2",<3em,1cm>*\as{t}="cv3",<4em,1cm>*\as{u}="cv4",<5em,1cm>*\as{a}="cv5",<6em,1cm>*\as{ }="cv6",<7em,1cm>*\as{ɡw}="cv7",<8em,1cm>*\as{e}="cv8",<9em,1cm>*\as{ }="cv9",<10em,1cm>*\as{e}="cv10",<11em,1cm>*\as{ }="cv11",
		<1.5em,0cm>*\as{\hp{\sub{1}}M\sub{1}}="m1",<4em,0cm>*\as{\hp{\sub{2}}M\sub{2}}="m2",<9em,0cm>*\as{\hp{\sub{3}}M\sub{3}}="m3",<2.5em,0cm>*\as{-}="-",<7em,0cm>*\as{=}="=",
		"m1"+U;"cv1"+D**\dir{-};"m1"+U;"cv2"+D**\dir{-};"m2"+U;"cv3"+D**\dir{-};"m2"+U;"cv4"+D**\dir{-};"m2"+U;"cv5"+D**\dir{-};"m3"+U;"cv8"+D**\dir{-};"m3"+U;"cv10"+D**\dir{-};
		"cv1"+U;"CV1"+D**\dir{-};"cv2"+U;"CV2"+D**\dir{-};"cv3"+U;"CV3"+D**\dir{-};"cv4"+U;"CV4"+D**\dir{-};"cv5"+U;"CV5"+D**\dir{-};"cv6"+U;"CV6"+D**\dir{};"cv7"+U;"CV7"+D**\dir{-};"cv8"+U;"CV8"+D**\dir{-};"cv9"+U;"CV9"+D**\dir{ };"cv10"+U;"CV10"+D**\dir{-};"cv11"+U;"CV11"+D**\dir{};
		"CV1"+U;"s1"+D**\dir{-};"CV2"+U;"s1"+D**\dir{-};"CV3"+U;"s1"+D**\dir{-};"CV3"+U;"s2"+D**\dir{-};"CV4"+U;"s2"+D**\dir{-};
		"CV5"+U;"s3"+D**\dir{-};"CV6"+U;"s3"+D**\dir{-};"CV7"+U;"s4"+D**\dir{-};"CV8"+U;"s4"+D**\dir{-};"CV9"+U;"s4"+D**\dir{-};
		"CV9"+U;"s5"+D**\dir{-};"CV10"+U;"s5"+D**\dir{-};"CV11"+U;"s5"+D**\dir{-};
		"s2"+U;"Ft1"+D**\dir{-};"s3"+U;"Ft1"+D**\dir{-};"s4"+U;"Ft2"+D**\dir{-};"s5"+U;"Ft2"+D**\dir{-};
		"s1"+U;"PrWd1"+D**\dir{-};"Ft1"+U;"PrWd1"+D**\dir{-};"PrWd1"+U;"PrWd2"+D**\dir{-};"Ft2"+U;"PrWd2"+D**\dir{-};
		<6.4em,3cm>*\as{\tikz[red,thick,dashed,baseline=0.9ex]\draw (0,0) -- (0,4.5cm);}="line",
	\endxy}
	\end{xlist}}
\end{exe}

If /ɡw/ is the default foot-final consonant, as I have proposed,
we would expect it to also be inserted after Ca{\#} final words
when a vowel-initial enclitic is attached (\ve{n-biib=ee} `massages her/him')
in order to provide the enclitic with an onset consonant.
The reason this does not occur probably has to do
with the motivation for /ɡw/ insertion in each case.

After Va{\#} final words, such as 
\ve{na-tua\tbr{gw}=ee} `occupies it',
/ɡw/ is not primarily inserted to provide the enclitic with an onset,
but rather the break up the underlying sequence of three vowels.
Put another way, lack of an onset consonant
after Ca{\#}-final words is better than insertion of /ɡw/,
while the presence of a sequence of three
vowels in Va{\#} final words is worse than insertion of /ɡw/. 

However, given the data from the M\=/forms of VVCV{\#} forms,
such as \ve{aunu} {\ra} \ve{aun} `spear'
and \ve{kaunaʔ} {\ra} \ve{kaun} `snake; creature'
in which the sequence of three vowels created after metathesis
is resolved by vowel deletion (\srf{sec:VowDel}),
we might expect the underlying sequence of vowels
in examples such as \ve{na-tua=ee} {\ra} \ve{na-tua\tbr{gw}=ee} `occupies it'
to be similarly resolved by vowel deletion.

The reason this does not occur
is because vowel deletion would have to occur
twice in order to resolve such sequences
as there are four underlying vowels in such forms.
One instance of epenthesis is preferred over two instances of deletion.

\subsection{Fo{\Q}asa{\Q} consonant insertion}\label{sec:FoqConIns}
Evidence in favour of analysing /ɡw/
as a default consonant comes from the variety of
Kotos Amarasi spoken in Fo{\Q}asa{\Q} hamlet,
one of the four hamlets unified to form the village of Nekmese{\Q} (\srf{sec:LanBac}).
In Fo{\Q}asa{\Q}, when a vowel-initial enclitic is attached to a vowel-final stem,
a velar obstruent /ɡ/ (without [w]) is inserted.
Metathesis also occurs, but does not trigger vowel assimilation.
Examples of Fo{\Q}asa{\Q} consonant insertion are given in \qf{ex:ConInsFoq} below.
\begin{exe}
	\ex{Consonant insertion in Fo{\Q}asa{\Q}}\label{ex:ConInsFoq}
			\sn{\gw\begin{tabular}{rllllll}
											&	&					&			& Fo{\Q}asa{\Q}				& Koro{\Q}oto 					& gloss\\
				\ve{umi} 			&+&\ve{=ee}&{\ra}&\ve{uim\tbr{g}=ee}		&\ve{uum\tbr{\j}=ee}			&`the house'\\
				\ve{peti} 		&+&\ve{=ee}&{\ra}&\ve{peit\tbr{g}=ee}		&\ve{peet\tbr{\j}=ee}		&`the box'\\
				\ve{n-rari} 	&+&\ve{=ee}&{\ra}&\ve{n-rair\tbr{g}=ee}	&\ve{n-raar\tbr{\j}=ee}	&`finishes it'\\
				\ve{n-so{ʔ}i}	&+&\ve{=ee}&{\ra}&\ve{n-soiʔ\tbr{g}=ee}	&\ve{n-sooʔ\tbr{\j}=ee}	&`counts it'\\
				\ve{fee} 			&+&\ve{=ee}&{\ra}&\ve{fee\tbr{g}=ee}		&\ve{fee\tbr{\j}=ee}			&`the wife'\\
				\ve{n-mo{ʔ}e}	&+&\ve{=ee}&{\ra}&\ve{n-moeʔ\tbr{g}=ee}	&\ve{n-mooʔ\tbr{\j}=ee}	&`does it'\\
				\ve{hau}			&+&\ve{=ii}&{\ra}&\ve{hau\tbr{g}=ii}				&\ve{haa\tbr{gw}=ii}			&`the tree'\\
				\ve{neno}			&+&\ve{=ees}&{\ra}&\ve{neo\ng\tbr{g}=ees}		&\ve{nee\ng\tbr{gw}=ees}	&`one day'\\
			\end{tabular}}
\end{exe}

Roots which end in final /a/ have variation between
no consonant insertion, or insertion of /ɡ/ in Fo{\Q}asa{\Q}.
Two exmples are \ve{na-ʔura} + \ve{=een} {\ra} \ve{na-ʔuurg=een} {\tl} \ve{na-ʔuur=een}
`its started raining', and \ve{n-sosa} + \ve{=ee} {\ra} \ve{n-soosg=ee}
{\tl} \ve{n-soos=ee} `buys it'.

In present day Nekmese{\Q}, this so-called ``Fo{\Q}asa{\Q} \it{ge}''
also occurs in the speech of those who
originally lived in Koro{\Q}oto{\Q},
particularly in certain set phrases.
It is also common in the speech of the generation
which was born after the creation of Nekmese{\Q}.
One phrase in which this Fo{\Q}asa{\Q} ge
is common is the phrase used to take leave,
given in \qf{ex:KorJeen} below.

\begin{exe}
\let\eachwordone=\textnormal
	\ex{\gllll Koro{\Q}oto: \ve{au} \ve{ʔ-faan\j=een} \ve{tua} \\
						 Fo{\Q}asa{\Q}: \ve{au} \ve{ʔ-fai{\ng}g=een} \ve{tua} \\
						{} au ʔ-fani=ena tua \\
						{} {\au} {\q}-return{\Mv}={\een} {\tua}\\
			\glt \lh{Koro{\Q}oto:} `I'm going to go back (home) now.'}\label{ex:KorJeen}
\end{exe}

The process of consonant insertion for Fo{\Q}asa{\Q}
\ve{umi} {\ra} \ve{uimg=ee} `house' is illustrated in \qf{as:uimg=ee} below.
In \qf{as:uimg=ee1} each foot begins with an empty C-slot.
As a result, consonants are inserted in \qf{as:uimg=ee2}.
The glottal stop is selected to fill the first empty C-slot
as this is the default initial consonant.
The velar obstruent is selected to fill the second
empty C-slot, as this C-slot is foot final
and /ɡ/ is the default final consonant.

Because the third C-slot is shared between two prosodic
words, metathesis then occurs to produce
a crisp edge after the internal prosodic word,
yielding the final output in (\ref{as:uimg=ee}c).

\begin{multicols}{2}
\begin{exe}\ex{\label{as:uimg=ee}
	\begin{xlist}
	\exa{\xy
		<3em,5cm>*\as{\hp{\sub{1}}PrWd\sub{1}}="PrWd1",<4.5em,6cm>*\as{\hp{\sub{2}}PrWd\sub{2}}="PrWd2",
		<3em,4cm>*\as{\hp{\sub{1}}Ft\sub{1}}="Ft1",<7em,4cm>*\as{\hp{\sub{2}}Ft\sub{2}}="Ft2",
		<2em,3cm>*\as{\hp{\sub{1}}σ\sub{1}}="s1",<4em,3cm>*\as{\hp{\sub{2}}σ\sub{2}}="s2",<6em,3cm>*\as{\hp{\sub{3}}σ\sub{3}}="s3",<8em,3cm>*\as{\hp{\sub{4}}σ\sub{4}}="s4",
		<1em,2cm>*\as{C}="CV1",<2em,2cm>*\as{V}="CV2",<3em,2cm>*\as{C}="CV3",<4em,2cm>*\as{V}="CV4",<5em,2cm>*\as{C}="CV5",
		<6em,2cm>*\as{V}="CV6",<7em,2cm>*\as{C}="CV7",<8em,2cm>*\as{V}="CV8",<9em,2cm>*\as{C}="CV9",
		<1em,1cm>*\as{ }="cv1",<2em,1cm>*\as{u}="cv2",<3em,1cm>*\as{m}="cv3",<4em,1cm>*\as{i}="cv4",<5em,1cm>*\as{}="cv5",
		<6em,1cm>*\as{e}="cv6",<7em,1cm>*\as{ }="cv7",<8em,1cm>*\as{e}="cv8",<9em,1cm>*\as{ }="cv9",
		<3em,0cm>*\as{\hp{\sub{1}}M\sub{1}}="m1",<7em,0cm>*\as{\hp{\sub{2}}M\sub{2}}="m2",<5em,0cm>*\as{=}="=",
		"m1"+U;"cv2"+D**\dir{-};"m1"+U;"cv3"+D**\dir{-};"m1"+U;"cv4"+D**\dir{-};"m2"+U;"cv6"+D**\dir{-};"m2"+U;"cv8"+D**\dir{-};
		"cv1"+U;"CV1"+D**\dir{};"cv2"+U;"CV2"+D**\dir{-};"cv3"+U;"CV3"+D**\dir{-};"cv4"+U;"CV4"+D**\dir{-};"cv5"+U;"CV5"+D**\dir{};"cv6"+U;"CV6"+D**\dir{-};"cv7"+U;"CV7"+D**\dir{};"cv8"+U;"CV8"+D**\dir{-};"cv9"+U;"CV9"+D**\dir{};
		"CV1"+U;"s1"+D**\dir{-};"CV2"+U;"s1"+D**\dir{-};"CV3"+U;"s1"+D**\dir{-};"CV3"+U;"s2"+D**\dir{-};"CV4"+U;"s2"+D**\dir{-};"CV5"+U;"s2"+D**\dir{-};
		"CV5"+U;"s3"+D**\dir{-};"CV6"+U;"s3"+D**\dir{-};"CV7"+U;"s3"+D**\dir{-};"CV7"+U;"s4"+D**\dir{-};"CV8"+U;"s4"+D**\dir{-};"CV9"+U;"s4"+D**\dir{-};
		"s1"+U;"Ft1"+D**\dir{-};"s2"+U;"Ft1"+D**\dir{-};"s3"+U;"Ft2"+D**\dir{-};"s4"+U;"Ft2"+D**\dir{-};
		"Ft1"+U;"PrWd1"+D**\dir{-};"PrWd1"+U;"PrWd2"+D**\dir{-};"Ft2"+U;"PrWd2"+D**\dir{-};
		<5em,1.5cm>*\as{\tikz[red,thick,dashed,baseline=0.9ex]\draw (0,0) rectangle (0.4cm,1.5cm);}="box",
		<1em,1.5cm>*\as{\tikz[red,thick,dashed,baseline=0.9ex]\draw (0,0) rectangle (0.4cm,1.5cm);}="box",
	\endxy}\label{as:uimg=ee1}
	\exa{\xy
		<3em,5cm>*\as{\hp{\sub{1}}PrWd\sub{1}}="PrWd1",<4.5em,6cm>*\as{\hp{\sub{2}}PrWd\sub{2}}="PrWd2",
		<3em,4cm>*\as{\hp{\sub{1}}Ft\sub{1}}="Ft1",<7em,4cm>*\as{\hp{\sub{2}}Ft\sub{2}}="Ft2",
		<2em,3cm>*\as{\hp{\sub{1}}σ\sub{1}}="s1",<4em,3cm>*\as{\hp{\sub{2}}σ\sub{2}}="s2",<6em,3cm>*\as{\hp{\sub{3}}σ\sub{3}}="s3",<8em,3cm>*\as{\hp{\sub{4}}σ\sub{4}}="s4",
		<1em,2cm>*\as{C}="CV1",<2em,2cm>*\as{V}="CV2",<3em,2cm>*\as{C}="CV3",<4em,2cm>*\as{V}="CV4",<5em,2cm>*\as{C}="CV5",
		<6em,2cm>*\as{V}="CV6",<7em,2cm>*\as{C}="CV7",<8em,2cm>*\as{V}="CV8",<9em,2cm>*\as{C}="CV9",
		<1em,1cm>*\as{ʔ}="cv1",<2em,1cm>*\as{u}="cv2",<3em,1cm>*\as{m}="cv3",<4em,1cm>*\as{i}="cv4",<5em,1cm>*\as{g}="cv5",
		<6em,1cm>*\as{e}="cv6",<7em,1cm>*\as{ }="cv7",<8em,1cm>*\as{e}="cv8",<9em,1cm>*\as{ }="cv9",
		<3em,0cm>*\as{\hp{\sub{1}}M\sub{1}}="m1",<7em,0cm>*\as{\hp{\sub{2}}M\sub{2}}="m2",<5em,0cm>*\as{=}="=",
		"m1"+U;"cv2"+D**\dir{-};"m1"+U;"cv3"+D**\dir{-};"m1"+U;"cv4"+D**\dir{-};"m2"+U;"cv6"+D**\dir{-};"m2"+U;"cv8"+D**\dir{-};
		"cv1"+U;"CV1"+D**\dir{-};"cv2"+U;"CV2"+D**\dir{-};"cv3"+U;"CV3"+D**\dir{-};"cv4"+U;"CV4"+D**\dir{-};"cv5"+U;"CV5"+D**\dir{-};"cv6"+U;"CV6"+D**\dir{-};"cv7"+U;"CV7"+D**\dir{};"cv8"+U;"CV8"+D**\dir{-};"cv9"+U;"CV9"+D**\dir{};
		"CV1"+U;"s1"+D**\dir{-};"CV2"+U;"s1"+D**\dir{-};"CV3"+U;"s1"+D**\dir{-};"CV3"+U;"s2"+D**\dir{-};"CV4"+U;"s2"+D**\dir{-};"CV5"+U;"s2"+D**\dir{-};
		"CV5"+U;"s3"+D**\dir{-};"CV6"+U;"s3"+D**\dir{-};"CV7"+U;"s3"+D**\dir{-};"CV7"+U;"s4"+D**\dir{-};"CV8"+U;"s4"+D**\dir{-};"CV9"+U;"s4"+D**\dir{-};
		"s1"+U;"Ft1"+D**\dir{-};"s2"+U;"Ft1"+D**\dir{-};"s3"+U;"Ft2"+D**\dir{-};"s4"+U;"Ft2"+D**\dir{-};
		"Ft1"+U;"PrWd1"+D**\dir{-};"PrWd1"+U;"PrWd2"+D**\dir{-};"Ft2"+U;"PrWd2"+D**\dir{-};
		<5em,1.5cm>*\as{\tikz[red,thick,dashed,baseline=0.9ex]\draw (0,0) rectangle (0.4cm,1.5cm);}="box",
		<1em,1.5cm>*\as{\tikz[red,thick,dashed,baseline=0.9ex]\draw (0,0) rectangle (0.4cm,1.5cm);}="box",
	\endxy}\label{as:uimg=ee2}
	\end{xlist}}
\end{exe}
\end{multicols}

\begin{exe}\sn{
	\begin{xlist}
	\exi{c.}\exia{\xy
		<2.5em,5cm>*\as{\hp{\sub{1}}PrWd\sub{1}}="PrWd1",<4.5em,6cm>*\as{\hp{\sub{2}}PrWd\sub{2}}="PrWd2",
		<2.5em,4cm>*\as{\hp{\sub{\tsc{m}}}Ft\sub{\tsc{m}}}="Ft1",<7em,4cm>*\as{\hp{\sub{2}}Ft\sub{2}}="Ft2",
		<1.5em,3cm>*\as{\hp{\sub{1}}σ\sub{1}}="s1",<3.5em,3cm>*\as{\hp{\sub{2}}σ\sub{2}}="s2",<6em,3cm>*\as{\hp{\sub{3}}σ\sub{3}}="s3",<8em,3cm>*\as{\hp{\sub{4}}σ\sub{4}}="s4",
		<1em,2cm>*\as{C}="CV1",<2em,2cm>*\as{V}="CV2",<3em,2cm>*\as{V}="CV3",<4em,2cm>*\as{C}="CV4",<5em,2cm>*\as{C}="CV5",
		<6em,2cm>*\as{V}="CV6",<7em,2cm>*\as{C}="CV7",<8em,2cm>*\as{V}="CV8",<9em,2cm>*\as{C}="CV9",
		<1em,1cm>*\as{ʔ}="cv1",<2em,1cm>*\as{u}="cv2",<3em,1cm>*\as{i}="cv3",<4em,1cm>*\as{m}="cv4",<5em,1cm>*\as{g}="cv5",
		<6em,1cm>*\as{e}="cv6",<7em,1cm>*\as{ }="cv7",<8em,1cm>*\as{e}="cv8",<9em,1cm>*\as{ }="cv9",
		<3em,0cm>*\as{\hp{\sub{1}}M\sub{1}}="m1",<7em,0cm>*\as{\hp{\sub{2}}M\sub{2}}="m2",<5em,0cm>*\as{=}="=",
		"m1"+U;"cv2"+D**\dir{-};"m1"+U;"cv3"+D**\dir{-};"m1"+U;"cv4"+D**\dir{-};"m2"+U;"cv6"+D**\dir{-};"m2"+U;"cv8"+D**\dir{-};
		"cv1"+U;"CV1"+D**\dir{-};"cv2"+U;"CV2"+D**\dir{-};"cv3"+U;"CV3"+D**\dir{-};"cv4"+U;"CV4"+D**\dir{-};"cv5"+U;"CV5"+D**\dir{-};"cv6"+U;"CV6"+D**\dir{-};"cv7"+U;"CV7"+D**\dir{};"cv8"+U;"CV8"+D**\dir{-};"cv9"+U;"CV9"+D**\dir{};
		"CV1"+U;"s1"+D**\dir{-};"CV2"+U;"s1"+D**\dir{-};"CV3"+U;"s2"+D**\dir{-};"CV4"+U;"s2"+D**\dir{-};
		"CV5"+U;"s3"+D**\dir{-};"CV6"+U;"s3"+D**\dir{-};"CV7"+U;"s3"+D**\dir{-};"CV7"+U;"s4"+D**\dir{-};"CV8"+U;"s4"+D**\dir{-};"CV9"+U;"s4"+D**\dir{-};
		"s1"+U;"Ft1"+D**\dir{-};"s2"+U;"Ft1"+D**\dir{-};"s3"+U;"Ft2"+D**\dir{-};"s4"+U;"Ft2"+D**\dir{-};
		"Ft1"+U;"PrWd1"+D**\dir{-};"PrWd1"+U;"PrWd2"+D**\dir{-};"Ft2"+U;"PrWd2"+D**\dir{-};
		<4.5em,3cm>*\as{\tikz[red,thick,dashed,baseline=0.9ex]\draw (0,0) -- (0,4.5cm);}="line",
	\endxy}
	\end{xlist}}
\end{exe}

Consonant insertion in Fo{\Q}asa{\Q} hamlet
is different from Koro{\Q}oto hamlet in two ways.
Firstly, in Fo{\Q}asa{\Q} the default
foot-final consonant is the velar obstruent /ɡ/,
while in Koro{\Q}oto it is the labio-velar obstruent /ɡw/.

Secondly, in Koro{\Q}oto hamlet word-medial consonant insertion
is conditioned by the quality of the previous vowel.
In Fo{\Q}asa{\Q} hamlet, the quality of the previous vowel plays no role, and
instead the default final consonant is inserted.
Because no features are shared between the inserted
consonant and the previous vowel,
vowel assimilation is not triggered by metathesis.
\section{The plural enclitic}\label{sec:PluEnc}
The plural enclitic has a number of allomorphs and variant forms,
partly depending on the shape of the host to which it attaches.
This enclitic marks plurals for nouns and for verbs it marks that one
or more of the core verbal arguments (subject or object) is plural.
The allomorphy of the plural enclitic for verbs and nouns is similar, though not identical.
This allomorphy is summarised in \trf{tab:PluEncAll}.
The main difference is the allomorphs taken by
nouns and verbs ending in a vowel sequence.

\begin{table}[h]
	\caption{Plural enclitic allomorphy}\label{tab:PluEncAll}
	\centering
		\begin{tabular}{rll}\lsptoprule
							Stem	& Nominals				&Verbs	 \\ \midrule
			{\ldots}C{\#}	& \ve{=ein/=eni},	&\ve{=ein/=eni}, \\
										& \ve{=enu/=uun}	&\ve{=enu/=uun}\\
			{\ldots}CV{\#}& \ve{=n}					&\ve{=n}\\
			{\ldots}VV{\#}& \ve{=n=gwein},	&\ve{=n}\\
										& \ve{=nu}				&\\
			\lspbottomrule
		\end{tabular}
\end{table}

After consonant-final stems, the plural enclitic
usually has the form \ve{=ein/=eni}.
The M\=/form \ve{=ein} is usually realised as [ɪn], and U\=/form \ve{=eni} as [ɛni].
The choice between the U\=/form and M\=/form of this enclitic
is discourse driven (Chapter \ref{ch:DisMet}) and
the M\=/form is the default form (\srf{sec:DefFor1}).
Before this enclitic CVC{\#} stems undergo metathesis,
as is expected before vowel-initial enclitics.
Examples of pluralised consonant-final verbs and nouns
are given in \qf{ex:pl->-ein/C} below.

\begin{exe}
	\ex{\{\tsc{pl}\} {\ra} \ve{=ein} /C{\#}{\gap}}\label{ex:pl->-ein/C}
		\sn{\gw\begin{tabular}{rcll}
			\ve{anah}				&\ra&\ve{aanh=\tbr{ein}}				&`children'\\
			\ve{kaes mutiʔ}	&\ra&\ve{kaes muitʔ=\tbr{ein}}	&`Europeans'\\
			\ve{enoʔ}				&\ra&\ve{eonʔ=\tbr{ein}}				&`doors'\\
			\ve{tuaf}			&\ra&\ve{tuaf=\tbr{ein}}				&`people'\\
			\ve{kuan}				&\ra&\ve{kuan=\tbr{ein}}				&`villages'\\
			\ve{n-fesat}		&\ra&\ve{n-feest=\tbr{ein}}			&`(they) throw a party'\\
			\ve{na-barab}		&\ra&\ve{na-baarb=\tbr{ein}}		&`(they) prepare'\\
			\ve{n-ʔonen}		&\ra&\ve{n-ʔoenn=\tbr{ein}}			&`(they) pray'\\
			\ve{na-tuin}		&\ra&\ve{na-tuin=\tbr{ein}}			&`(they) follow'\\
%			\ve{}	&\ra&\ve{=n}	&`'\\
%			\ve{}	&\ra&\ve{=n}	&`'\\
	\end{tabular}}
\end{exe}

This enclitic also has the variant forms \ve{=uun} and \ve{=enu},
of which the form \ve{=uun} is the M\=/form of \ve{=enu}.
The expected M\=/form \ve{\tcb{*}=eun} does not occur in my data,
thus \ve{=enu} {\ra} \ve{=uun} is an irregular M\=/form (\srf{sec:IrrMfor}).
The forms \ve{=enu} and \ve{=uun} are rare in my data.
There are twelve attestations of \ve{=uun} in my corpus and three attestations of \ve{=enu}.
This is compared with 157 attestations of \ve{=ein}
and 21 attestations of \ve{=eni}.
Examples of \ve{=uun} and \ve{=enu} are given in \qf{ex:pl->-uun,enu/C} below.
The clitic hosts shown in \qf{ex:pl->-uun,enu/C} also occur with \ve{=ein}.\footnote{
		In some varieties of Amanuban the plural enclitic
		usually has the form \ve{=enu/=eun}.}

\begin{exe}
	\ex{\{\tsc{pl}\} {\ra} \ve{=uun {\tl} =enu} /C{\#}{\gap}}\label{ex:pl->-uun,enu/C}
		\sn{\gw\begin{tabular}{rcll}
			\ve{abas}				&\ra&\ve{aabs=\tbr{uun}}			&`threads'\\
			\ve{na-ʔkoroʔ}	&\ra&\ve{na-ʔkoorʔ=\tbr{uun}}	&`(they) hide'\\
			\ve{Timor}			&\ra&\ve{Tiamr=\tbr{uun}}			&`Timorese people'\\
			\ve{faif anaʔ}	&\ra&\ve{faif aanʔ=\tbr{enu}}	&`piglets'\\
			\ve{kana-k}			&\ra&\ve{kaan-k=\tbr{enu}}		&`their names'\\
	\end{tabular}}
\end{exe}

After stems which end in CV,
the plural enclitic usually takes the form \ve{=n}.
Examples are given in \qf{ex:pl->-n/CV} below.

\begin{exe}
	\ex{\{\tsc{pl}\} {\ra} \ve{=n} /CV{\#}{\gap}}\label{ex:pl->-n/CV}
		\sn{\gw\begin{tabular}{rcll}
			\ve{kase}		&\ra&\ve{kase=\tbr{n}}	&`foreigners'\\
			\ve{hutu}		&\ra&\ve{hutu=\tbr{n}}	&`head-lice'\\
			\ve{kbiti}	&\ra&\ve{kbiti=\tbr{n}}	&`scorpions'\\
			\ve{koro}		&\ra&\ve{koro=\tbr{n}}	&`birds'\\
			\ve{tuni}		&\ra&\ve{tuni=\tbr{n}}	&`eels'\\
			\ve{n-moʔe}	&\ra&\ve{n-moʔe=\tbr{n}}	&`(they) do/make'\\
			\ve{na-tona}	&\ra&\ve{na-tona=\tbr{n}}	&`(they) tell'\\
			\ve{n-eki}		&\ra&\ve{n-eki=\tbr{n}}		&`(they) bring'\\
			\ve{na-hana}	&\ra&\ve{na-hana=\tbr{n}}	&`(they) cook'\\
%			\ve{}	&\ra&\ve{=n}	&`'\\
%			\ve{}	&\ra&\ve{=n}	&`'\\
	\end{tabular}}
\end{exe}

Similarly, after verbs which end in a vowel sequence,
the plural enclitic also has the form \ve{=n}.
A number of examples are given in \qf{ex:pl->=n/VVverb} below.

\begin{exe}
	\ex{\{\tsc{pl}\} {\ra} \ve{=n} Verb, /VV{\#}{\gap}}\label{ex:pl->=n/VVverb}
		\sn{\gw\begin{tabular}{rcll}
			\ve{n-sii}		&\ra&\ve{n-sii=\tbr{n}}	&`(they) sing'\\
			\ve{n-murai}	&\ra&\ve{n-murai=\tbr{n}}	&`(they) start'\\
			\ve{n-tui}		&\ra&\ve{n-tui=\tbr{n}}	&`(they) write'\\
			\ve{n-kae}		&\ra&\ve{n-kae=\tbr{n}}	&`(they) cry'\\
			\ve{n-nao}		&\ra&\ve{n-nao=\tbr{n}}	&`(they) go'\\
			\ve{na-niu}		&\ra&\ve{na-niu=\tbr{n}}	&`(they) bathe'\\
			\ve{na-mnau}	&\ra&\ve{na-mnau=\tbr{n}}	&`(they) remember'\\
			\ve{n-poi}		&\ra&\ve{n-poi=\tbr{n}}	&`(they) exit/go out'\\
%			\ve{}	&\ra&\ve{=n}	&`'\\
%			\ve{}	&\ra&\ve{=n}	&`'\\
	\end{tabular}}
\end{exe}

When nouns which end in a vowel sequence are pluralised,
a number of different forms occur.
Firstly, there is the form \ve{=nu}
which I have encountered once as a simple plural 
during my fieldwork. This example
is given in \qf{ex:Obs06/10/14} below.

\begin{exe}
	\ex{\gll	hiit t-hormaat hau=\tbr{nu}!\\
						{\hiit} {\t}-honour tree={\ein}\\
			\glt	`We're giving honour to the trees!'
						(Joke made when ducking branches of trees while riding in the back of a truck.)
						\txrf{Observation 06/10/14}}\label{ex:Obs06/10/14}
\end{exe}

The clitic \ve{=nu} also attaches to VV{\#} final pronouns
to mark an otherwise unexpressed plural possessum.
Thus, \ve{au=nu} `mine/my things', \ve{hoo=nu} `yours/your (sg.) things',
\ve{hai=nu} `ours (excl.)/our things', \ve{hii=nu} `yours (pl.)/your things'.
See \srf{sec:PosDet} for more discussion.

However, the normal way in which VV{\#} final nouns mark plural
and the normal way VV{\#} final pronouns mark plural possessums
is with a form [ŋɡwɪn].
This is analysable as a combination of \ve{=nu} + \ve{=ein}
with insertion of /ɡw/ before the second enclitic.
Examples with are given in \qf{ex:pl->=n/VVnoun} below.\footnote{
		In the Baikeno variety of Meto the plural enclitic has the form \ve{=mbini}
		after words which end in a vowel sequence, e.g. \ve{bi{\j}ae=mbini} `cows'.
		Insertion of Baikeno /b/ also corresponds 
		to insertion of Amarasi /ɡw/ in other environments.}

\begin{exe}
	\ex{\{\tsc{pl}\} {\ra} \ve{=ŋgwein} /VV{\#}{\gap}}\label{ex:pl->=n/VVnoun}
		\sn{\gw\begin{tabular}{rcll}
			\ve{bifee}	&{\ra}&\ve{bifee=\tbr{ŋgwein}}		&`women'	\\
			\ve{bi\j ae}&{\ra}&\ve{bi\j ae=\tbr{ŋgwein}}	&`cows'	\\
			\ve{oe}			&{\ra}&\ve{oe=\tbr{ŋgwein}}				&`kinds of water'	\\
			\ve{pentua}	&{\ra}&\ve{pentua=\tbr{ŋgwein}}		&`church elders'	\\
			\ve{too}		&{\ra}&\ve{too=\tbr{ŋgwein}}			&`citizens'	\\
			\ve{hau}		&{\ra}&\ve{hau=\tbr{ŋgwein}}			&`trees'	\\
%		\end{tabular}}
%\end{exe}
%\begin{exe}
%	\ex{\{\tsc{pl}\} {\ra} \ve{=ŋgwein} /VV{\#}{\gap}}\label{ex:pl->=n/VVpronoun}
%		\sn{\gw\begin{tabular}{rcll}
			\ve{au}		&{\ra}&\ve{au=\tbr{ŋgwein}}		&`mine/my things'	\\
			\ve{hoo}	&{\ra}&\ve{hoo=\tbr{ŋgwein}}	&`yours (sg.)/your things'	\\
			\ve{hai}	&{\ra}&\ve{hai=\tbr{ŋgwein}}	&`ours (excl.)/our things'	\\
			\ve{hii}	&{\ra}&\ve{hii=\tbr{ŋgwein}}	&`yours (pl.)/your things'	\\
		\end{tabular}}
\end{exe}

The noun \ve{kfuu} `star' is an exception.
This word has the plural form \ve{kfuu=n} `stars' for some speakers.
In this case singular \ve{kfuu} `star' is a back formation,
as the final /n/ of plural \ve{kfuu=n} is a reflex
of the final consonant of Proto-Malayo-Polynesian *bituqən.
Similarly, while the loan word \ve{partei} `friend'
(from Dutch \it{partij} [partɛi]) usually has the plural
\ve{partei=ŋgwein} `friends', it has been attested once
with \ve{=n}; thus \ve{partei=n} `friends' \citep[3]{or16}.\footnote{
		Another allomorph for VV{\#} final nouns
		is \ve{=ŋgonu/=ŋgoun} which is attested
		from a single speaker, and then only on the loan \ve{oraŋ tua} `parents'
		(from Malay \it{orang tua}).
		There is one example each in my data of \ve{oraŋ tua=ŋgonu}
		and \ve{oraŋ tua=ŋgoun} `parents'.}

There are also three examples in which
a CV{\#} or C{\#} final noun takes double plural marking
with both \ve{=n} and \ve{=ein},
given in \qf{ex:130914-2, 1.17}--\qf{ex:130902-1, 3.28} below.

\begin{exe}
	\ex{\glll	feʔe n-ʔoban naan rauk=\tbr{n}=\tbr{ein}, nopu nua mes \hspace{20mm} ka= n-eku =f, n-ʔoobn=aah.\\
						feʔe n-ʔoban naan raku=\tbr{n}=\tbr{ein} nopu nua mes {} ka= n-eku =f, n-ʔoban=aah.\\
						earlier \n-furrow {\naan} sweet.potato=\tbr{\ein}=\tbr{\ein} hole two but {} {\ka}= \n-eat ={\fa} \n-furrow=just\\
			\glt	`Earlier it had dug up the sweet potatoes, there were two holes
						but it hadn't eaten anything, it just dug around.'
						\txrf{130914-2, 1.17} {\emb{130914-2-01-17.mp3}{\spk{}}{\apl}}}\label{ex:130914-2, 1.17}
	\ex{\glll	hoo m-fee areʔ kana=n hau fua-f maut \hspace{30mm} he koor=\tbr{n}=\tbr{ein} bisa n-eku=n.\\
						hoo m-fee areʔ kana=n hau fua-f maut {} he koro=\tbr{n}=\tbr{ein} bisa n-eku=n.\\
						{\hoo} \m-give every name={\ein} tree fruit-{\f} let {}
						{\he} bird=\tbr{\ein}=\tbr{\ein} can \n-eat={\einV}\\
			\glt	`You gave all kinds of fruit trees in order that all the different birds could eat.'
						\hfill{\citep[11]{or16b}}}\label{ex:BigBook}
	\ex{\glll	rari =te, n-ma-taeb n-ok ahh baroit=\tbr{n}=\tbr{eni} =ma\\
						rari =te, n-ma-tabe n-oka {} baroit=\tbr{n}=\tbr{eni} =ma\\
						finish ={\te} \n-{\mak}-shake.hands \n-{\ok} {} bride/groom=\tbr{\ein}=\tbr{\ein} =and\\
			\glt	`After that he shook hands with both the bride and groom and'
						\txrf{130902-1, 3.28} {\emb{130902-1-03-28.mp3}{\spk{}}{\apl}}}\label{ex:130902-1, 3.28}
\end{exe}

Despite the complexities in the data,
the forms of the plural enclitic(s) can
be mostly described as allomorphy,
as summarised in \trf{tab:PluEncAll}.
For verbs the analysis is a straightforward case
of phonologically conditioned allomorphy.
Vowel-final stems take \ve{=n} while
consonant-final stems take \ve{=ein/=eni},
or its variant \ve{=uun/=enu}.
For nouns the data is more complex.
Consonant-final nouns take \ve{=ein/=eni},
CV{\#} final nouns take \ve{=n},
and VV{\#} final nouns normally take
\ve{=ŋgwein} (analysable as \ve{=nu} + \ve{=ein})
but also are attested with \ve{=n} or \ve{=nu}.
Double plural marking with \ve{=n=ein} also occasionally
occurs with nouns.

The examples with double plural marking may
indicate that the different allomorphs have come from different sources
and may have once been different morphemes with different functions.
While there may be traces of these different functions
in some of the synchronic data, in most cases, and for most speakers,
they appear to have semantically merged and both mark plural.\footnote{
		The allomorph \ve{=n} may have originally marked plurals with
		an emphasis on the group as a collection of individuals,
		thus paralleling the use of the quantifier \ve{areʔ} `every, all'
		while \ve{=ein} marked plurals as a whole mass,
		thus paralleling the use of the quantifier \ve{okeʔ} `all'.}

That the vowel-initial forms of the plural enclitic \ve{=ein}
do not occur with vowel-final stems means that consonant
insertion is not usually observed before this enclitic.
There is one exception in my database:
the verb \ve{na-ʔbaʔe} `play', which has been attested once
with the plural enclitic allomorph \ve{=ein} as \ve{na-ʔbaaʔ\j=ein}.
This verb is also exceptional in not otherwise taking M\=/forms.\footnote{
		There are also three other vowel-final stems occurring with
		the enclitic \ve{=ein} in the Amarasi Bible translation:
		\ve{na-ʔtaʔi} `trembles' + \ve{=ein} {\ra} \it{<na{\Q}tai{\Q} jein>} (one example),
		\ve{koʔu} `big' + \ve{=ein} {\ra} \it{<kou{\Q} guin>} (five examples)
		and \ve{na-ʔseʔ{\tl}seʔo} `whispers' {\ra} \it{<na{\Q}se{\Q}-seo{\Q} guin>} (two examples).}

\subsection{Consonant insertion after \it{=n}}\label{sec:ConInsPluEnc}
When the \ve{=n} allomorph of the plural enclitic
attaches to a stem which ends in a vowel sequence,
any subsequent clitic triggers insertion of /ɡw/.
This is analysable as resulting from historic/underlying \ve{=nu}.

Such consonant insertion does not occur 
when \ve{=n} attaches to a CV{\#} final stem.
Instead, when a vowel-initial enclitic follows,
the host is treated like a CVC{\#} stem with regular metathesis.
Examples are given in \qf{ex:CV=n+=V->VC=n=V} below.

\begin{exe}
	\ex{CV\ve{=n} + =V {\ra} VC\ve{=n}=V}\label{ex:CV=n+=V->VC=n=V}
	\sn{\stl{0.31em}\gw\begin{tabular}{rclclcll}
				\ve{sepa\tbr{tu}}	&+&\ve{=n}&+&\ve{=ii}	&\ra&\ve{sepa\tbr{ut}=n=ii}	&`the shoes'\\
				\ve{hu\tbr{tu}}		&+&\ve{=n}&+&\ve{=aan}	&\ra&\ve{hu\tbr{ut}=n=aan}	&`the head-lice'\\
				\ve{ka\tbr{se}}		&+&\ve{=n}&+&\ve{=ee}	&\ra&\ve{ka\tbr{es}=n=ee}	&`the foreigners'\\
				\ve{kbi\tbr{ti}}	&+&\ve{=n}&+&\ve{=ee}	&\ra&\ve{kbi\tbr{it}=n=ee}	&`the scorpions'\\
				\ve{ko\tbr{ro}}		&+&\ve{=n}&+&\ve{=ee}	&\ra&\ve{ko\tbr{or}=n=ee}	&`the birds'\\
				\ve{fa\tbr{fi}}		&+&\ve{=n}&+&\ve{=ee}	&\ra&\ve{fa\tbr{if}=n=ee}	&`the pigs'\\
				\ve{n-to\tbr{ti}}	&+&\ve{=n}&+&\ve{=aah}	&\ra&\ve{n-to\tbr{it}=n=aah}	&`(they) just ask'\\
				\ve{n-he\tbr{ra}}	&+&\ve{=n}&+&\ve{=ee}	&\ra&\ve{n-he\tbr{er}=n=ee}	&`(they) pull it'\\
				\ve{n-fa\tbr{ni}}	&+&\ve{=n}&+&\ve{=een}	&\ra&\ve{n-fa\tbr{in}=n=een}	&`(they've) now returned'\\
				\ve{na-hi\tbr{ni}}&+&\ve{=n}&+&\ve{=ii}	&\ra&\ve{na-hi\tbr{in}=n=een}	&`(they) now know'\\
		%		\ve{}	&+&\ve{=n}&+&\ve{=ii}	&\ra&\ve{}=n=}	&`'\\
		%		\ve{}	&+&\ve{=n}&+&\ve{=ii}	&\ra&\ve{}=n=}	&`'\\
	\end{tabular}}
\end{exe}

\largerpage
As discussed above, the regular allomorph of
the plural enclitic on VV{\#} verbs is \ve{=n}.
When either of the enclitics \ve{=een} or \ve{=aah}
follows, /ɡw/ usually occurs before the second enclitic.
Examples are given in \qf{ex:VV=ngw=een} below.

\newpage
\begin{exe}
	\ex{VV + \ve{=n} + =V {\ra} VV\ve{=ŋgw}V}\label{ex:VV=ngw=een}
		\sn{\stl{0.2em}\gw\begin{tabular}{rclclcll}
			\ve{n-sii}	&+&\ve{=n}&+&\ve{=een}&{\ra}&\ve{n-sii=ŋ\tbr{gw}=een}&`(they've) now sung'\\
			\ve{n-murai}&+&\ve{=n}&+&\ve{=een}&{\ra}&\ve{n-murai=ŋ\tbr{gw}=een}&`(they've) now started'\\
			\ve{n-tui}	&+&\ve{=n}&+&\ve{=een}&{\ra}&\ve{n-tui=ŋ\tbr{gw}=een}&`(they've) now written'\\
			\ve{n-kae}	&+&\ve{=n}&+&\ve{=een}&{\ra}&\ve{n-kae=ŋ\tbr{gw}=een}&`(they've) now cried'\\
			\ve{n-tea}	&+&\ve{=n}&+&\ve{=een}&{\ra}&\ve{n-tea=ŋ\tbr{gw}=een}&`(they've) now arrived'\\
			\ve{na-bua}	&+&\ve{=n}&+&\ve{=een}&{\ra}&\ve{na-bua=ŋ\tbr{gw}=een}&`(they've) now gathered'\\
			\ve{n-nao}	&+&\ve{=n}&+&\ve{=een}&{\ra}&\ve{n-nao=ŋ\tbr{gw}=een}&`(they've) now gone'\\
			\ve{na-niu}	&+&\ve{=n}&+&\ve{=een}&{\ra}&\ve{na-niu=ŋ\tbr{gw}=een}&`(they've) now bathed'\\
%			\ve{na-mnau}&+&\ve{=n}&+&\ve{=een}&{\ra}&\ve{na-mnau=ŋ\tbr{gw}=een}&`(they've) now remembered'\\
			\ve{n-poi}	&+&\ve{=n}&+&\ve{=aah}&{\ra}&\ve{n-poi=ŋ\tbr{gw}=aah}&`(they) just went out'\\
		\end{tabular}}
\end{exe}

Such insertion of /ɡw/ does not occur for verbs before other enclitics.
There are five examples in my corpus,
two with the discourse marker \ve{=ii}
and three with the \tsc{3sg.acc} pronoun \ve{=ee}.
These examples are given in \qf{ex:VV+-n+=ee/i/a->VV-n=ee/i/a} below.\footnote{
		There is also one example in my corpus of \ve{=n} + \ve{=ein} on a VV{\#}
		final host without consonant insertion: \ve{n-tea=n=ein} `they've arrived'.}

\begin{exe}
	\ex{VV + =n + =ee/=ii {\ra} VV=n=ee/=ii \label{ex:VV+-n+=ee/i/a->VV-n=ee/i/a}}
		\sn{\gw\begin{tabular}{lll}
			\ve{na-ʔ-rau=n}				&+ \ve{=ii} {\ra}			&\ve{na-ʔ-rau=n=ii}	\\ \hhline{~}
			{\na-\qV-bite=\einV}	&\hp{+ }{\ii}					&`made these ones bite'\\
			\ve{m-foe{\tl}foe=n}	&+ \ve{=ii} {\ra}			&\ve{m-foe{\tl}foe=n=ii}\\ \hhline{~}
			{\m-{\frd}move=\einV}	&\hp{+ }{\ii}					&`(we've) worked hard'\\
			\ve{n-nao=n}					&+ \ve{=ee} {\ra}			&\ve{n-nao=n=ee}\\ \hhline{~}
			{\n-go=\einV}					&\hp{+ }{\eeV}				&`went to him/her'\\
			\ve{n-sae=n}					&+ \ve{=ee} {\ra}			&\ve{n-sae=n=ee}\\ \hhline{~}
			{\n-go.up=\einV}			&\hp{+ }{\eeV}				&`increased for him/her'\\
			\ve{t-fee=n}					&+ \ve{=ee=siin} {\ra}&\ve{t-fee=n=ee=siin}\\ \hhline{~}
			{\t-give	=\einV}			&\hp{+ }{\eeV}={\siin}&`gave it to them'\\
		\end{tabular}}
\end{exe}

\largerpage
When a combination of a plural enclitic
and a vowel-initial enclitic occur on a VV{\#} noun,
the plural enclitic takes the form \ve{=n}
with insertion of /ɡw/ before the second enclitic.
For nouns, this includes enclitics other than
\ve{=een} and \ve{=aah}.
Examples are given in \qf{ex:VV=ngw=ee} below.

\begin{exe}
	\ex{VV + =n + =V {\ra} VV=ŋgw=V}\label{ex:VV=ngw=ee}
		\sn{\gw\begin{tabular}{rlllll}
			\ve{oe=n}		&+&\ve{=aan}	&{\ra}&\ve{oe=ŋ\tbr{gw}=aan}&`the kinds of water'\\
			\ve{mei=n}	&+&\ve{=ee}	&{\ra}&\ve{mei=ŋ\tbr{gw}=ee}&`the tables'\\
			\ve{too=n}	&+&\ve{=ii}	&{\ra}&\ve{too=ŋ\tbr{gw}=ii}&`the citizens'\\
%			\ve{=n}	&+&\ve{=ii}	&{\ra}&\ve{=n=gw}&`the citizens'\\
		\end{tabular}}
\end{exe}

\newpage
Examples of \ve{=n} and another enclitic
are judged as ungrammatical without insertion of /ɡw/.
Two examples are \ve{\tcb{*}n-sii=n=een} `sung' and \ve{\tcb{*}n-kae=n=een} `cried'.
This creates near-minimal pairs between forms in which a final /n/
is part of the root and ones in which it is the plural enclitic.
Thus, \ve{n-sii=n} + \ve{=een} {\ra} \ve{n-sii=ŋgw=een} `they've sung'
can be compared with \ve{n-pina} + \ve{=een} {\ra} \ve{n-piin=een} `blazed'.

Similarly, among nouns, insertion of /ɡw/ occurs after plural \ve{=n},
but not after the \tsc{3sg.gen} suffix \ve{-n}.
Thus \ve{too=ŋgw=ee} `citizen={\ein}={\ee}' (`the citizens')
with insertion of /ɡw/ can be compared
with \ve{ao-n=ee} `body-{\N}={\ee}' (`someone's body') without insertion.

When the \ve{=n} allomorph of the plural enclitic
attaches to a stem which ends in a vowel sequence,
any subsequent clitic usually triggers insertion of /ɡw/.
This is regular for nouns before all vowel-initial
enclitics and regular for verbs before \ve{=een} `{\een}'
and \ve{=aah} `just'.

	\subsection{Analysis of /ɡw/ insertion after VV=\it{n}}\label{sec:AnaGwInsVVn}
Insertion of /ɡw/ after \ve{=n} can be analysed by positing
metathesis of underlying (or historic) \ve{=nu}.
While rare, this form \emph{is} attested without any following enclitic.
I propose that this is (or was) the allomorph of this enclitic taken by
words which end in a vowel sequence.\footnote{
		Evidence that this enclitic consists of only a single syllable
		comes from its likely etymon:
		PAN *-nu `marker of uncertainty' \citep{bltr}
		or *nu `genitive marker' \citep[914]{wo10}.
		One use of this suffix that has been reconstructed
		is as place-holder for an unexpressed possessum,
		such as in *a-nu-ku `my unnamed thing: mine' \citep{bltr}.
		This reconstructed function matches almost exactly use of \ve{=nu}
		seen in phrases such as \ve{au=nu} `my things, mine'.
		See \srf{sec:PosDet} for more details on this construction.}

The full analysis is illustrated in \qf{as:too=ngw=ii} below for
\ve{too} `citizen' + \ve{=nu} `{\ein}' + \ve{=ii} `{\ii}' {\ra} \ve{too=ŋgw=ii} `the citizens'.
The first step is for the enclitic \ve{=nu} to be attached,
as illustrated in \qf{as:too=ngw=ii1}.
Because this enclitic is a single syllable,
it is directly linked to the prosodic word containing the host,
in the same way pre-foot material in words greater than two syllables
is directly linked to the prosodic word (\srf{sec:PrWd}).
The second enclitic is then attached,
as shown in \qf{as:too=ngw=ii2}, which also shows that the
foot containing this second enclitic begins with an empty C-slot.

\begin{multicols}{2}
\begin{exe}\ex{\label{as:too=ngw=ii}
	\begin{xlist}
	\exa{\xy
		<4.5em,5cm>*\as{\hp{\sub{1}}PrWd\sub{1}}="PrWd1",
		<3em,4cm>*\as{\hp{\sub{1}}Ft\sub{1}}="Ft1",
		<2em,3cm>*\as{\hp{\sub{1}}σ\sub{1}}="s1",<4em,3cm>*\as{\hp{\sub{2}}σ\sub{2}}="s2",<7em,3cm>*\as{\hp{\sub{3}}σ\sub{3}}="s3",
		<1em,2cm>*\as{C}="CV1",<2em,2cm>*\as{V}="CV2",<3em,2cm>*\as{C}="CV3",<4em,2cm>*\as{V}="CV4",<5em,2cm>*\as{C}="CV5",
		<6em,2cm>*\as{C}="CV6",<7em,2cm>*\as{V}="CV7",<8em,2cm>*\as{C}="CV8",
		<1em,1cm>*\as{t}="cv1",<2em,1cm>*\as{o}="cv2",<3em,1cm>*\as{ }="cv3",<4em,1cm>*\as{o}="cv4",<5em,1cm>*\as{ }="cv5",
		<6em,1cm>*\as{n}="cv6",<7em,1cm>*\as{u}="cv7",<8em,1cm>*\as{ }="cv8",
		<2.5em,0cm>*\as{\hp{\sub{1}}M\sub{1}}="m1",<6.5em,0cm>*\as{\hp{\sub{2}}M\sub{2}}="m2",<5.5em,0cm>*\as{=}="=",
		"m1"+U;"cv1"+D**\dir{-};"m1"+U;"cv2"+D**\dir{-};"m1"+U;"cv3"+D**\dir{};"m1"+U;"cv4"+D**\dir{-};"m1"+U;"cv5"+D**\dir{};
		"m2"+U;"cv6"+D**\dir{-};"m2"+U;"cv7"+D**\dir{-};
		"cv1"+U;"CV1"+D**\dir{-};"cv2"+U;"CV2"+D**\dir{-};"cv3"+U;"CV3"+D**\dir{};"cv4"+U;"CV4"+D**\dir{-};"cv5"+U;"CV5"+D**\dir{};
		"cv6"+U;"CV6"+D**\dir{-};"cv7"+U;"CV7"+D**\dir{-};"cv8"+U;"CV8"+D**\dir{};
		"CV1"+U;"s1"+D**\dir{-};"CV2"+U;"s1"+D**\dir{-};"CV3"+U;"s1"+D**\dir{-};"CV3"+U;"s2"+D**\dir{-};"CV4"+U;"s2"+D**\dir{-};"CV5"+U;"s2"+D**\dir{-};
		"CV6"+U;"s3"+D**\dir{-};"CV7"+U;"s3"+D**\dir{-};"CV8"+U;"s3"+D**\dir{-};
		"s1"+U;"Ft1"+D**\dir{-};"s2"+U;"Ft1"+D**\dir{-};
		"Ft1"+U;"PrWd1"+D**\dir{-};"s3"+U;"PrWd1"+D**\dir{-};
	\endxy}\label{as:too=ngw=ii1}
	\exa{\xy
		<4.05em,5cm>*\as{\hp{\sub{1}}PrWd\sub{1}}="PrWd1",<5.85em,6cm>*\as{\hp{\sub{2}}PrWd\sub{2}}="PrWd2",
		<2.7em,4cm>*\as{\hp{\sub{1}}Ft\sub{1}}="Ft1",<9em,4cm>*\as{\hp{\sub{2}}Ft\sub{2}}="Ft2",
		<1.8em,3cm>*\as{\hp{\sub{1}}σ\sub{1}}="s1",<3.6em,3cm>*\as{\hp{\sub{2}}σ\sub{2}}="s2",<6.3em,3cm>*\as{\hp{\sub{3}}σ\sub{3}}="s3",
		<8.1em,3cm>*\as{\hp{\sub{4}}σ\sub{4}}="s4",<9.9em,3cm>*\as{\hp{\sub{5}}σ\sub{5}}="s5",
		<0.9em,2cm>*\as{C}="CV1",<1.8em,2cm>*\as{V}="CV2",<2.7em,2cm>*\as{C}="CV3",<3.6em,2cm>*\as{V}="CV4",<4.5em,2cm>*\as{C}="CV5",
		<5.4em,2cm>*\as{C}="CV6",<6.3em,2cm>*\as{V}="CV7",<7.2em,2cm>*\as{C}="CV8",
		<8.1em,2cm>*\as{V}="CV9",<9em,2cm>*\as{C}="CV10",<9.9em,2cm>*\as{V}="CV11",<10.8em,2cm>*\as{C}="CV12",
		<0.9em,1cm>*\as{t}="cv1",<1.8em,1cm>*\as{o}="cv2",<2.7em,1cm>*\as{ }="cv3",<3.6em,1cm>*\as{o}="cv4",<4.5em,1cm>*\as{ }="cv5",
		<5.4em,1cm>*\as{n}="cv6",<6.3em,1cm>*\as{u}="cv7",<7.2em,1cm>*\as{ }="cv8",
		<8.1em,1cm>*\as{i}="cv9",<9em,1cm>*\as{ }="cv10",<9.9em,1cm>*\as{i}="cv11",<10.8em,1cm>*\as{ }="cv12",
		<2.25em,0cm>*\as{\hp{\sub{1}}M\sub{1}}="m1",<4.95em,0cm>*\as{=}="=",<5.85em,0cm>*\as{\hp{\sub{2}}M\sub{2}}="m2",
		<7.2em,0cm>*\as{=}="=",<9em,0cm>*\as{\hp{\sub{3}}M\sub{3}}="m3",
		"m1"+U;"cv1"+D**\dir{-};"m1"+U;"cv2"+D**\dir{-};"m1"+U;"cv3"+D**\dir{};"m1"+U;"cv4"+D**\dir{-};"m1"+U;"cv5"+D**\dir{};
		"m2"+U;"cv6"+D**\dir{-};"m2"+U;"cv7"+D**\dir{-};"m3"+U;"cv9"+D**\dir{-};"m3"+U;"cv11"+D**\dir{-};
		"cv1"+U;"CV1"+D**\dir{-};"cv2"+U;"CV2"+D**\dir{-};"cv3"+U;"CV3"+D**\dir{};"cv4"+U;"CV4"+D**\dir{-};"cv5"+U;"CV5"+D**\dir{};
		"cv6"+U;"CV6"+D**\dir{-};"cv7"+U;"CV7"+D**\dir{-};"cv8"+U;"CV8"+D**\dir{};
		"cv9"+U;"CV9"+D**\dir{-};"cv10"+U;"CV10"+D**\dir{};"cv11"+U;"CV11"+D**\dir{-};"cv12"+U;"CV12"+D**\dir{};
		"CV1"+U;"s1"+D**\dir{-};"CV2"+U;"s1"+D**\dir{-};"CV3"+U;"s1"+D**\dir{-};"CV3"+U;"s2"+D**\dir{-};"CV4"+U;"s2"+D**\dir{-};"CV5"+U;"s2"+D**\dir{-};
		"CV6"+U;"s3"+D**\dir{-};"CV7"+U;"s3"+D**\dir{-};"CV8"+U;"s3"+D**\dir{-};"CV8"+U;"s4"+D**\dir{-};
		"CV9"+U;"s4"+D**\dir{-};"CV10"+U;"s4"+D**\dir{-};"CV10"+U;"s5"+D**\dir{-};"CV11"+U;"s5"+D**\dir{-};"CV12"+U;"s5"+D**\dir{-};
		"s1"+U;"Ft1"+D**\dir{-};"s2"+U;"Ft1"+D**\dir{-};"s4"+U;"Ft2"+D**\dir{-};"s5"+U;"Ft2"+D**\dir{-};
		"Ft1"+U;"PrWd1"+D**\dir{-};"s3"+U;"PrWd1"+D**\dir{-};"PrWd1"+U;"PrWd2"+D**\dir{-};"Ft2"+U;"PrWd2"+D**\dir{-};
		<7.2em,1.5cm>*\as{\tikz[red,thick,dashed,baseline=0.9ex]\draw (0,0) rectangle (0.35cm,1.5cm);}="box",
	\endxy}\label{as:too=ngw=ii2}
	\end{xlist}}
\end{exe}
\end{multicols}

In order to provide the second foot with an onset,
the features of the previous vowel spread in (\ref{as:too=ngw=ii}c),
in which the features \tsc{[+back]} and \tsc{[+round]} of the vowel /u/
are abbreviated to \tsc{[+v.]}.
This produces the obstruent /ɡw/
as the onset of the second foot in (\ref{as:too=ngw=ii}d).

\begin{multicols}{2}
\begin{exe}\exr{as:too=ngw=ii}{
	\begin{xlist}
	\exi{c.}\exia{\xy
		<4.05em,5cm>*\as{\hp{\sub{1}}PrWd\sub{1}}="PrWd1",<5.85em,6cm>*\as{\hp{\sub{2}}PrWd\sub{2}}="PrWd2",
		<2.7em,4cm>*\as{\hp{\sub{1}}Ft\sub{1}}="Ft1",<9em,4cm>*\as{\hp{\sub{2}}Ft\sub{2}}="Ft2",
		<1.8em,3cm>*\as{\hp{\sub{1}}σ\sub{1}}="s1",<3.6em,3cm>*\as{\hp{\sub{2}}σ\sub{2}}="s2",<6.3em,3cm>*\as{\hp{\sub{3}}σ\sub{3}}="s3",
		<8.1em,3cm>*\as{\hp{\sub{4}}σ\sub{4}}="s4",<9.9em,3cm>*\as{\hp{\sub{5}}σ\sub{5}}="s5",
		<0.9em,2cm>*\as{C}="CV1",<1.8em,2cm>*\as{V}="CV2",<2.7em,2cm>*\as{C}="CV3",<3.6em,2cm>*\as{V}="CV4",<4.5em,2cm>*\as{C}="CV5",
		<5.4em,2cm>*\as{C}="CV6",<6.3em,2cm>*\as{V}="CV7",<7.2em,2cm>*\as{C}="CV8",
		<8.1em,2cm>*\as{V}="CV9",<9em,2cm>*\as{C}="CV10",<9.9em,2cm>*\as{V}="CV11",<10.8em,2cm>*\as{C}="CV12",
		<0.9em,1cm>*\as{t}="cv1",<1.8em,1cm>*\as{o}="cv2",<2.7em,1cm>*\as{ }="cv3",<3.6em,1cm>*\as{o}="cv4",<4.5em,1cm>*\as{ }="cv5",
		<5.4em,1cm>*\as{n}="cv6",<6.3em,1cm>*\as{u}="cv7",<7.2em,1cm>*\as{ }="cv8",
		<8.1em,1cm>*\as{i}="cv9",<9em,1cm>*\as{ }="cv10",<9.9em,1cm>*\as{i}="cv11",<10.8em,1cm>*\as{ }="cv12",
		<6.3em,0cm>*\as{\tsc{[+v.]}}="f","f"+U;"cv7"+D**\dir{-};"f"+U;"cv8"+D**\dir{.};"cv8"+D;"CV8"+D**\dir{.};
		"cv1"+U;"CV1"+D**\dir{-};"cv2"+U;"CV2"+D**\dir{-};"cv3"+U;"CV3"+D**\dir{};"cv4"+U;"CV4"+D**\dir{-};"cv5"+U;"CV5"+D**\dir{};
		"cv6"+U;"CV6"+D**\dir{-};"cv7"+U;"CV7"+D**\dir{-};"cv8"+U;"CV8"+D**\dir{};
		"cv9"+U;"CV9"+D**\dir{-};"cv10"+U;"CV10"+D**\dir{};"cv11"+U;"CV11"+D**\dir{-};"cv12"+U;"CV12"+D**\dir{};
		"CV1"+U;"s1"+D**\dir{-};"CV2"+U;"s1"+D**\dir{-};"CV3"+U;"s1"+D**\dir{-};"CV3"+U;"s2"+D**\dir{-};"CV4"+U;"s2"+D**\dir{-};"CV5"+U;"s2"+D**\dir{-};
		"CV6"+U;"s3"+D**\dir{-};"CV7"+U;"s3"+D**\dir{-};"CV8"+U;"s3"+D**\dir{-};"CV8"+U;"s4"+D**\dir{-};
		"CV9"+U;"s4"+D**\dir{-};"CV10"+U;"s4"+D**\dir{-};"CV10"+U;"s5"+D**\dir{-};"CV11"+U;"s5"+D**\dir{-};"CV12"+U;"s5"+D**\dir{-};
		"s1"+U;"Ft1"+D**\dir{-};"s2"+U;"Ft1"+D**\dir{-};"s4"+U;"Ft2"+D**\dir{-};"s5"+U;"Ft2"+D**\dir{-};
		"Ft1"+U;"PrWd1"+D**\dir{-};"s3"+U;"PrWd1"+D**\dir{-};"PrWd1"+U;"PrWd2"+D**\dir{-};"Ft2"+U;"PrWd2"+D**\dir{-};
		<6.75em,1.5cm>*\as{\tikz[red,thick,dashed,baseline=0.9ex]\draw (0,0) rectangle (0.7cm,1.5cm);}="box",
	\endxy}
	\exi{d.}\exia{\xy
		<4.05em,5cm>*\as{\hp{\sub{1}}PrWd\sub{1}}="PrWd1",<5.85em,6cm>*\as{\hp{\sub{2}}PrWd\sub{2}}="PrWd2",
		<2.7em,4cm>*\as{\hp{\sub{1}}Ft\sub{1}}="Ft1",<9em,4cm>*\as{\hp{\sub{2}}Ft\sub{2}}="Ft2",
		<1.8em,3cm>*\as{\hp{\sub{1}}σ\sub{1}}="s1",<3.6em,3cm>*\as{\hp{\sub{2}}σ\sub{2}}="s2",<6.3em,3cm>*\as{\hp{\sub{3}}σ\sub{3}}="s3",
		<8.1em,3cm>*\as{\hp{\sub{4}}σ\sub{4}}="s4",<9.9em,3cm>*\as{\hp{\sub{5}}σ\sub{5}}="s5",
		<0.9em,2cm>*\as{C}="CV1",<1.8em,2cm>*\as{V}="CV2",<2.7em,2cm>*\as{C}="CV3",<3.6em,2cm>*\as{V}="CV4",<4.5em,2cm>*\as{C}="CV5",
		<5.4em,2cm>*\as{C}="CV6",<6.3em,2cm>*\as{V}="CV7",<7.2em,2cm>*\as{C}="CV8",
		<8.1em,2cm>*\as{V}="CV9",<9em,2cm>*\as{C}="CV10",<9.9em,2cm>*\as{V}="CV11",<10.8em,2cm>*\as{C}="CV12",
		<0.9em,1cm>*\as{t}="cv1",<1.8em,1cm>*\as{o}="cv2",<2.7em,1cm>*\as{ }="cv3",<3.6em,1cm>*\as{o}="cv4",<4.5em,1cm>*\as{ }="cv5",
		<5.4em,1cm>*\as{n}="cv6",<6.3em,1cm>*\as{u}="cv7",<7.2em,1cm>*\as{ɡw}="cv8",
		<8.1em,1cm>*\as{i}="cv9",<9em,1cm>*\as{ }="cv10",<9.9em,1cm>*\as{i}="cv11",<10.8em,1cm>*\as{ }="cv12",
		<6.75em,0cm>*\as{\tsc{[+v.]}}="f","f"+U;"cv7"+D**\dir{-};"f"+U;"cv8"+D**\dir{-};
		"cv1"+U;"CV1"+D**\dir{-};"cv2"+U;"CV2"+D**\dir{-};"cv3"+U;"CV3"+D**\dir{};"cv4"+U;"CV4"+D**\dir{-};"cv5"+U;"CV5"+D**\dir{};
		"cv6"+U;"CV6"+D**\dir{-};"cv7"+U;"CV7"+D**\dir{-};"cv8"+U;"CV8"+D**\dir{-};
		"cv9"+U;"CV9"+D**\dir{-};"cv10"+U;"CV10"+D**\dir{};"cv11"+U;"CV11"+D**\dir{-};"cv12"+U;"CV12"+D**\dir{};
		"CV1"+U;"s1"+D**\dir{-};"CV2"+U;"s1"+D**\dir{-};"CV3"+U;"s1"+D**\dir{-};"CV3"+U;"s2"+D**\dir{-};"CV4"+U;"s2"+D**\dir{-};"CV5"+U;"s2"+D**\dir{-};
		"CV6"+U;"s3"+D**\dir{-};"CV7"+U;"s3"+D**\dir{-};"CV8"+U;"s3"+D**\dir{-};"CV8"+U;"s4"+D**\dir{-};
		"CV9"+U;"s4"+D**\dir{-};"CV10"+U;"s4"+D**\dir{-};"CV10"+U;"s5"+D**\dir{-};"CV11"+U;"s5"+D**\dir{-};"CV12"+U;"s5"+D**\dir{-};
		"s1"+U;"Ft1"+D**\dir{-};"s2"+U;"Ft1"+D**\dir{-};"s4"+U;"Ft2"+D**\dir{-};"s5"+U;"Ft2"+D**\dir{-};
		"Ft1"+U;"PrWd1"+D**\dir{-};"s3"+U;"PrWd1"+D**\dir{-};"PrWd1"+U;"PrWd2"+D**\dir{-};"Ft2"+U;"PrWd2"+D**\dir{-};
		<7.2em,1.5cm>*\as{\tikz[red,thick,dashed,baseline=0.9ex]\draw (0,0) rectangle (0.45cm,1.5cm);}="box",
	\endxy}
	\end{xlist}}
\end{exe}
\end{multicols}

Because the C-slot containing the newly-inserted consonant
is shared between the internal and external prosodic word,
metathesis occurs in (\ref{as:too=ngw=ii}e) to create a crisp edge
after the internal prosodic word, as shown in (\ref{as:too=ngw=ii}f).

\begin{multicols}{2}
\begin{exe}\exr{as:too=ngw=ii}{
	\begin{xlist}
	\exi{e.}\exia{\xy
		<4.05em,6cm>*\as{\hp{\sub{1}}PrWd\sub{1}}="PrWd1",<5.85em,7cm>*\as{\hp{\sub{2}}PrWd\sub{2}}="PrWd2",
		<2.7em,5cm>*\as{\hp{\sub{1}}Ft\sub{1}}="Ft1",<9em,5cm>*\as{\hp{\sub{2}}Ft\sub{2}}="Ft2",
		<1.8em,4cm>*\as{\hp{\sub{1}}σ\sub{1}}="s1",<3.6em,4cm>*\as{\hp{\sub{2}}σ\sub{2}}="s2",<6.3em,4cm>*\as{\hp{\sub{3}}σ\sub{3}}="s3",
		<8.1em,4cm>*\as{\hp{\sub{4}}σ\sub{4}}="s4",<9.9em,4cm>*\as{\hp{\sub{5}}σ\sub{5}}="s5",
		<0.9em,3cm>*\as{\x}="x1",<1.8em,3cm>*\as{\x}="x2",<2.7em,3cm>*\as{\x}="x3",<3.6em,3cm>*\as{\x}="x4",<4.5em,3cm>*\as{\x}="x5",
		<5.4em,3cm>*\as{\x}="x6",<6.3em,3cm>*\as{\x}="x7",<7.2em,3cm>*\as{\x}="x8",
		<8.1em,3cm>*\as{\x}="x9",<9em,3cm>*\as{\x}="x10",<9.9em,3cm>*\as{\x}="x11",<10.8em,3cm>*\as{\x}="x12",
		<0.9em,2cm>*\as{C}="CV1",<1.8em,2cm>*\as{V}="CV2",<2.7em,2cm>*\as{C}="CV3",<3.6em,2cm>*\as{V}="CV4",<4.5em,2cm>*\as{C}="CV5",
		<5.4em,2cm>*\as{C}="CV6",<6.3em,2cm>*\as{V}="CV7",<7.2em,2cm>*\as{C}="CV8",
		<8.1em,2cm>*\as{V}="CV9",<9em,2cm>*\as{C}="CV10",<9.9em,2cm>*\as{V}="CV11",<10.8em,2cm>*\as{C}="CV12",
		<0.9em,1cm>*\as{t}="cv1",<1.8em,1cm>*\as{o}="cv2",<2.7em,1cm>*\as{ }="cv3",<3.6em,1cm>*\as{o}="cv4",<4.5em,1cm>*\as{ }="cv5",
		<5.4em,1cm>*\as{n}="cv6",<6.3em,1cm>*\as{u}="cv7",<7.2em,1cm>*\as{ɡw}="cv8",
		<8.1em,1cm>*\as{i}="cv9",<9em,1cm>*\as{ }="cv10",<9.9em,1cm>*\as{i}="cv11",<10.8em,1cm>*\as{ }="cv12",
		<6.75em,0cm>*\as{\tsc{[+v.]}}="f","f"+U;"cv7"+D**\dir{-};"f"+U;"cv8"+D**\dir{-};
		"cv1"+U;"CV1"+D**\dir{-};"cv2"+U;"CV2"+D**\dir{-};"cv3"+U;"CV3"+D**\dir{};"cv4"+U;"CV4"+D**\dir{-};"cv5"+U;"CV5"+D**\dir{};
		"cv6"+U;"CV6"+D**\dir{-};"cv7"+U;"CV7"+D**\dir{-};"cv8"+U;"CV8"+D**\dir{-};
		"cv9"+U;"CV9"+D**\dir{-};"cv10"+U;"CV10"+D**\dir{};"cv11"+U;"CV11"+D**\dir{-};"cv12"+U;"CV12"+D**\dir{};
		"CV1"+U;"x1"+D**\dir{-};"CV2"+U;"x2"+D**\dir{-};"CV3"+U;"x3"+D**\dir{-};"CV4"+U;"x4"+D**\dir{-};"CV5"+U;"x5"+D**\dir{-};
		"CV6"+U;"x7"+D**\dir{.};"CV7"+U;"x6"+D**\dir{.};"CV8"+U;"x8"+D**\dir{-};
		"CV9"+U;"x9"+D**\dir{-};"CV10"+U;"x10"+D**\dir{-};"CV11"+U;"x11"+D**\dir{-};"CV12"+U;"x12"+D**\dir{-};
		"x1"+U;"s1"+D**\dir{-};"x2"+U;"s1"+D**\dir{-};"x3"+U;"s1"+D**\dir{-};"x3"+U;"s2"+D**\dir{-};"x4"+U;"s2"+D**\dir{-};"x5"+U;"s2"+D**\dir{-};
		"x6"+U;"s3"+D**\dir{-};"x7"+U;"s3"+D**\dir{-};"x8"+U;"s3"+D**\dir{-};"x8"+U;"s4"+D**\dir{-};
		"x9"+U;"s4"+D**\dir{-};"x10"+U;"s4"+D**\dir{-};"x10"+U;"s5"+D**\dir{-};"x11"+U;"s5"+D**\dir{-};"x12"+U;"s5"+D**\dir{-};
		"s1"+U;"Ft1"+D**\dir{-};"s2"+U;"Ft1"+D**\dir{-};"s4"+U;"Ft2"+D**\dir{-};"s5"+U;"Ft2"+D**\dir{-};
		"Ft1"+U;"PrWd1"+D**\dir{-};"s3"+U;"PrWd1"+D**\dir{-};"PrWd1"+U;"PrWd2"+D**\dir{-};"Ft2"+U;"PrWd2"+D**\dir{-};
		<5.85em,2.5cm>*\as{\tikz[red,thick,dashed,baseline=0.9ex]\draw (0,0) rectangle (0.7cm,1.5cm);}="box",
	\endxy}
		\exi{f.}\exia{\xy
		<3.6em,6cm>*\as{\hp{\sub{1}}PrWd\sub{1}}="PrWd1",<5.85em,7cm>*\as{\hp{\sub{2}}PrWd\sub{2}}="PrWd2",
		<2.7em,5cm>*\as{\hp{\sub{1}}Ft\sub{1}}="Ft1",<9em,5cm>*\as{\hp{\sub{2}}Ft\sub{2}}="Ft2",
		<1.8em,4cm>*\as{\hp{\sub{1}}σ\sub{1}}="s1",<3.6em,4cm>*\as{\hp{\sub{2}}σ\sub{2}}="s2",<5.4em,4cm>*\as{\hp{\sub{3}}σ\sub{3}}="s3",
		<8.1em,4cm>*\as{\hp{\sub{4}}σ\sub{4}}="s4",<9.9em,4cm>*\as{\hp{\sub{5}}σ\sub{5}}="s5",
		<0.9em,3cm>*\as{\x}="x1",<1.8em,3cm>*\as{\x}="x2",<2.7em,3cm>*\as{\x}="x3",<3.6em,3cm>*\as{\x}="x4",<4.5em,3cm>*\as{\x}="x5",
		<5.4em,3cm>*\as{\x}="x6",<6.3em,3cm>*\as{\x}="x7",<7.2em,3cm>*\as{\x}="x8",
		<8.1em,3cm>*\as{\x}="x9",<9em,3cm>*\as{\x}="x10",<9.9em,3cm>*\as{\x}="x11",<10.8em,3cm>*\as{\x}="x12",
		<0.9em,2cm>*\as{C}="CV1",<1.8em,2cm>*\as{V}="CV2",<2.7em,2cm>*\as{C}="CV3",<3.6em,2cm>*\as{V}="CV4",<4.5em,2cm>*\as{C}="CV5",
		<5.4em,2cm>*\as{V}="CV6",<6.3em,2cm>*\as{C}="CV7",<7.2em,2cm>*\as{C}="CV8",
		<8.1em,2cm>*\as{V}="CV9",<9em,2cm>*\as{C}="CV10",<9.9em,2cm>*\as{V}="CV11",<10.8em,2cm>*\as{C}="CV12",
		<0.9em,1cm>*\as{t}="cv1",<1.8em,1cm>*\as{o}="cv2",<2.7em,1cm>*\as{ }="cv3",<3.6em,1cm>*\as{o}="cv4",<4.5em,1cm>*\as{ }="cv5",
		<5.4em,1cm>*\as{u}="cv6",<6.3em,1cm>*\as{n}="cv7",<7.2em,1cm>*\as{ɡw}="cv8",
		<8.1em,1cm>*\as{i}="cv9",<9em,1cm>*\as{ }="cv10",<9.9em,1cm>*\as{i}="cv11",<10.8em,1cm>*\as{ }="cv12",
		<6.3em,0cm>*\as{\tsc{[+v.]}}="f","f"+U;"cv6"+D**\dir{-};"f"+U;"cv8"+D**\dir{-};
		"cv1"+U;"CV1"+D**\dir{-};"cv2"+U;"CV2"+D**\dir{-};"cv3"+U;"CV3"+D**\dir{};"cv4"+U;"CV4"+D**\dir{-};"cv6"+U;"CV6"+D**\dir{-};
		"cv7"+U;"CV7"+D**\dir{-};"cv8"+U;"CV8"+D**\dir{-};
		"cv9"+U;"CV9"+D**\dir{-};"cv10"+U;"CV10"+D**\dir{};"cv11"+U;"CV11"+D**\dir{-};"cv12"+U;"CV12"+D**\dir{};
		"CV1"+U;"x1"+D**\dir{-};"CV2"+U;"x2"+D**\dir{-};"CV3"+U;"x3"+D**\dir{-};"CV4"+U;"x4"+D**\dir{-};"CV5"+U;"x5"+D**\dir{-};
		"CV6"+U;"x6"+D**\dir{-};"CV7"+U;"x7"+D**\dir{-};"CV8"+U;"x8"+D**\dir{-};"CV9"+U;"x9"+D**\dir{-};
		"CV10"+U;"x10"+D**\dir{-};"CV11"+U;"x11"+D**\dir{-};"CV12"+U;"x12"+D**\dir{-};
		"x1"+U;"s1"+D**\dir{-};"x2"+U;"s1"+D**\dir{-};"x3"+U;"s1"+D**\dir{-};"x3"+U;"s2"+D**\dir{-};"x4"+U;"s2"+D**\dir{-};"x5"+U;"s2"+D**\dir{-};
		"x5"+U;"s3"+D**\dir{-};"x6"+U;"s3"+D**\dir{-};"x7"+U;"s3"+D**\dir{-};"x8"+U;"s4"+D**\dir{-};"x8"+U;"s4"+D**\dir{-};
		"x9"+U;"s4"+D**\dir{-};"x10"+U;"s4"+D**\dir{-};"x10"+U;"s5"+D**\dir{-};"x11"+U;"s5"+D**\dir{-};"x12"+U;"s5"+D**\dir{-};
		"s1"+U;"Ft1"+D**\dir{-};"s2"+U;"Ft1"+D**\dir{-};"s4"+U;"Ft2"+D**\dir{-};"s5"+U;"Ft2"+D**\dir{-};
		"Ft1"+U;"PrWd1"+D**\dir{-};"s3"+U;"PrWd1"+D**\dir{-};"PrWd1"+U;"PrWd2"+D**\dir{-};"Ft2"+U;"PrWd2"+D**\dir{-};
		<6.75em,3.75cm>*\as{\tikz[red,thick,dashed,baseline=0.9ex]\draw (0,0) -- (0,5cm);}="line",
	\endxy}
	\end{xlist}}
\end{exe}
\end{multicols}

However, metathesis results in the features of final
vowel of \ve{=nu} and the inserted consonant /ɡw/ being
shared across an intervening consonant,
as shown in (\ref{as:too=ngw=ii}g).
As a result the vowel de-links.
This results in the third syllable having an empty V-slot.
Amarasi allows empty C-slots, but not empty V-slots.
Normally empty V-slots are filled by spreading of
an adjacent vowel, as discussed in \srf{sec:VowAss}.
However, there is no adjacent vowel in (\ref{as:too=ngw=ii}h).
Thus, this V-slot is deleted.

\begin{multicols}{2}
\begin{exe}\exr{as:too=ngw=ii}{
	\begin{xlist}	
	\exi{g.}\exia{\xy
		<3.6em,5cm>*\as{\hp{\sub{1}}PrWd\sub{1}}="PrWd1",<5.85em,6cm>*\as{\hp{\sub{2}}PrWd\sub{2}}="PrWd2",
		<2.7em,4cm>*\as{\hp{\sub{1}}Ft\sub{1}}="Ft1",<9em,4cm>*\as{\hp{\sub{2}}Ft\sub{2}}="Ft2",
		<1.8em,3cm>*\as{\hp{\sub{1}}σ\sub{1}}="s1",<3.6em,3cm>*\as{\hp{\sub{2}}σ\sub{2}}="s2",<5.4em,3cm>*\as{\hp{\sub{3}}σ\sub{3}}="s3",
		<8.1em,3cm>*\as{\hp{\sub{4}}σ\sub{4}}="s4",<9.9em,3cm>*\as{\hp{\sub{5}}σ\sub{5}}="s5",
		<0.9em,2cm>*\as{C}="CV1",<1.8em,2cm>*\as{V}="CV2",<2.7em,2cm>*\as{C}="CV3",<3.6em,2cm>*\as{V}="CV4",<4.5em,2cm>*\as{C}="CV5",
		<5.4em,2cm>*\as{V}="CV6",<6.3em,2cm>*\as{C}="CV7",<7.2em,2cm>*\as{C}="CV8",
		<8.1em,2cm>*\as{V}="CV9",<9em,2cm>*\as{C}="CV10",<9.9em,2cm>*\as{V}="CV11",<10.8em,2cm>*\as{C}="CV12",
		<0.9em,1cm>*\as{t}="cv1",<1.8em,1cm>*\as{o}="cv2",<2.7em,1cm>*\as{ }="cv3",<3.6em,1cm>*\as{o}="cv4",<4.5em,1cm>*\as{ }="cv5",
		<5.4em,1cm>*\as{u}="cv6",<6.3em,1cm>*\as{n}="cv7",<7.2em,1cm>*\as{ɡw}="cv8",
		<8.1em,1cm>*\as{i}="cv9",<9em,1cm>*\as{ }="cv10",<9.9em,1cm>*\as{i}="cv11",<10.8em,1cm>*\as{ }="cv12",
		<6.3em,0cm>*\as{\tsc{[+v.]}}="f","f"+U;"cv8"+D**\dir{-};{\ar@{-}|-(.425)*@{|} |-{\hole} |-(.575)*@{|} "f"+U;"cv6"+D};
		<4em,0cm>*\as{\tsc{[+c.]}}="f2","f2"+U;"cv7"+D**\dir{-};
		"cv1"+U;"CV1"+D**\dir{-};"cv2"+U;"CV2"+D**\dir{-};"cv3"+U;"CV3"+D**\dir{};"cv4"+U;"CV4"+D**\dir{-};"cv5"+U;"CV5"+D**\dir{};
		"cv7"+U;"CV7"+D**\dir{-};"cv8"+U;"CV8"+D**\dir{-};{\ar@{-}|-(.425)*@{|} |-{\hole} |-(.575)*@{|} "cv6"+U;"CV6"+D};
		"cv9"+U;"CV9"+D**\dir{-};"cv10"+U;"CV10"+D**\dir{};"cv11"+U;"CV11"+D**\dir{-};"cv12"+U;"CV12"+D**\dir{};
		"CV1"+U;"s1"+D**\dir{-};"CV2"+U;"s1"+D**\dir{-};"CV3"+U;"s1"+D**\dir{-};"CV3"+U;"s2"+D**\dir{-};"CV4"+U;"s2"+D**\dir{-};"CV5"+U;"s2"+D**\dir{-};
		"CV5"+U;"s3"+D**\dir{-};"CV6"+U;"s3"+D**\dir{-};"CV7"+U;"s3"+D**\dir{-};"CV8"+U;"s4"+D**\dir{-};"CV8"+U;"s4"+D**\dir{-};
		"CV9"+U;"s4"+D**\dir{-};"CV10"+U;"s4"+D**\dir{-};"CV10"+U;"s5"+D**\dir{-};"CV11"+U;"s5"+D**\dir{-};"CV12"+U;"s5"+D**\dir{-};
		"s1"+U;"Ft1"+D**\dir{-};"s2"+U;"Ft1"+D**\dir{-};"s4"+U;"Ft2"+D**\dir{-};"s5"+U;"Ft2"+D**\dir{-};
		"Ft1"+U;"PrWd1"+D**\dir{-};"s3"+U;"PrWd1"+D**\dir{-};"PrWd1"+U;"PrWd2"+D**\dir{-};"Ft2"+U;"PrWd2"+D**\dir{-};
		<5.4em,1.5cm>*\as{\tikz[red,thick,dashed,baseline=0.9ex]\draw (0,0) rectangle (0.35cm,1.5cm);}="box",
	\endxy}
	\exi{h.}\exia{\xy
		<3.6em,5cm>*\as{\hp{\sub{1}}PrWd\sub{1}}="PrWd1",<5.85em,6cm>*\as{\hp{\sub{2}}PrWd\sub{2}}="PrWd2",
		<2.7em,4cm>*\as{\hp{\sub{1}}Ft\sub{1}}="Ft1",<9em,4cm>*\as{\hp{\sub{2}}Ft\sub{2}}="Ft2",
		<1.8em,3cm>*\as{\hp{\sub{1}}σ\sub{1}}="s1",<3.6em,3cm>*\as{\hp{\sub{2}}σ\sub{2}}="s2",<5.4em,3cm>*\as{\hp{\sub{3}}σ\sub{3}}="s3",
		<8.1em,3cm>*\as{\hp{\sub{4}}σ\sub{4}}="s4",<9.9em,3cm>*\as{\hp{\sub{5}}σ\sub{5}}="s5",
		<0.9em,2cm>*\as{C}="CV1",<1.8em,2cm>*\as{V}="CV2",<2.7em,2cm>*\as{C}="CV3",<3.6em,2cm>*\as{V}="CV4",<4.5em,2cm>*\as{C}="CV5",
		<5.4em,2cm>*\as{V}="CV6",<6.3em,2cm>*\as{C}="CV7",<7.2em,2cm>*\as{C}="CV8",
		<8.1em,2cm>*\as{V}="CV9",<9em,2cm>*\as{C}="CV10",<9.9em,2cm>*\as{V}="CV11",<10.8em,2cm>*\as{C}="CV12",
		<0.9em,1cm>*\as{t}="cv1",<1.8em,1cm>*\as{o}="cv2",<2.7em,1cm>*\as{ }="cv3",<3.6em,1cm>*\as{o}="cv4",<4.5em,1cm>*\as{ }="cv5",
		<5.4em,1cm>*\as{ }="cv6",<6.3em,1cm>*\as{n}="cv7",<7.2em,1cm>*\as{ɡw}="cv8",
		<8.1em,1cm>*\as{i}="cv9",<9em,1cm>*\as{ }="cv10",<9.9em,1cm>*\as{i}="cv11",<10.8em,1cm>*\as{ }="cv12",
		<2.5em,0cm>*\as{\hp{\sub{1}}M\sub{1}}="m1",<4.95em,0cm>*\as{=}="=",<6.3em,0cm>*\as{\hp{\sub{2}}M\sub{2}}="m2",
		<7.65em,0cm>*\as{=}="=",<9em,0cm>*\as{\hp{\sub{3}}M\sub{3}}="m3",
		"m1"+U;"cv1"+D**\dir{-};"m1"+U;"cv2"+D**\dir{-};"m1"+U;"cv3"+D**\dir{};"m1"+U;"cv4"+D**\dir{-};"m1"+U;"cv5"+D**\dir{};
		"m2"+U;"cv7"+D**\dir{-};"m3"+U;"cv9"+D**\dir{-};"m3"+U;"cv11"+D**\dir{-};
		"cv1"+U;"CV1"+D**\dir{-};"cv2"+U;"CV2"+D**\dir{-};"cv3"+U;"CV3"+D**\dir{};"cv4"+U;"CV4"+D**\dir{-};"cv5"+U;"CV5"+D**\dir{};
		"cv7"+U;"CV7"+D**\dir{-};"cv8"+U;"CV8"+D**\dir{-};
		"cv9"+U;"CV9"+D**\dir{-};"cv10"+U;"CV10"+D**\dir{};"cv11"+U;"CV11"+D**\dir{-};"cv12"+U;"CV12"+D**\dir{};
		"CV1"+U;"s1"+D**\dir{-};"CV2"+U;"s1"+D**\dir{-};"CV3"+U;"s1"+D**\dir{-};"CV3"+U;"s2"+D**\dir{-};"CV4"+U;"s2"+D**\dir{-};"CV5"+U;"s2"+D**\dir{-};
		"CV5"+U;"s3"+D**\dir{-};"CV7"+U;"s3"+D**\dir{-};"CV8"+U;"s4"+D**\dir{-};"CV8"+U;"s4"+D**\dir{-};{\ar@{-}|-(.425)*@{|} |-{\hole} |-(.575)*@{|} "CV6"+U;"s3"+D};
		"CV9"+U;"s4"+D**\dir{-};"CV10"+U;"s4"+D**\dir{-};"CV10"+U;"s5"+D**\dir{-};"CV11"+U;"s5"+D**\dir{-};"CV12"+U;"s5"+D**\dir{-};
		"s1"+U;"Ft1"+D**\dir{-};"s2"+U;"Ft1"+D**\dir{-};"s4"+U;"Ft2"+D**\dir{-};"s5"+U;"Ft2"+D**\dir{-};{\ar@{-}|-(.475)*@{|} |-{\hole} |-(.525)*@{|} "s3"+U;"PrWd1"+D};
		"Ft1"+U;"PrWd1"+D**\dir{-};"PrWd1"+U;"PrWd2"+D**\dir{-};"Ft2"+U;"PrWd2"+D**\dir{-};
		<5.4em,2.5cm>*\as{\tikz[red,thick,dashed,baseline=0.9ex]\draw (0,0) rectangle (0.35cm,1.5cm);}="box",
	\endxy}
	\end{xlist}}
\end{exe}
\end{multicols}

\newpage
This produces a structure in which the second syllable
contains a final cluster in (\ref{as:too=ngw=ii}i).
Normally final clusters are resolved by deletion
of the final C-slot of the cluster,
as seen in the derivation M\=/forms (\srf{sec:MetConDel}).
However, in this case the final C-slot contains
all that is left of the plural morpheme. I propose
that the preservation of this morpheme motivates deletion
of the penultimate C-slot (\ref{as:too=ngw=ii}i) instead.
This produces the final output, given in (\ref{as:too=ngw=ii}j).

\begin{multicols}{2}
\begin{exe}\exr{as:too=ngw=ii}{
	\begin{xlist}
	\exi{i.}\exia{\xy
		<3em,5cm>*\as{\hp{\sub{1}}PrWd\sub{1}}="PrWd1",<6em,6cm>*\as{\hp{\sub{2}}PrWd\sub{2}}="PrWd2",
		<3em,4cm>*\as{\hp{\sub{1}}Ft\sub{1}}="Ft1",<9em,4cm>*\as{\hp{\sub{2}}Ft\sub{2}}="Ft2",
		<2em,3cm>*\as{\hp{\sub{1}}σ\sub{1}}="s1",<4em,3cm>*\as{\hp{\sub{2}}σ\sub{2}}="s2",
		<8em,3cm>*\as{\hp{\sub{4}}σ\sub{4}}="s4",<10em,3cm>*\as{\hp{\sub{5}}σ\sub{5}}="s5",
		<1em,2cm>*\as{C}="CV1",<2em,2cm>*\as{V}="CV2",<3em,2cm>*\as{C}="CV3",<4em,2cm>*\as{V}="CV4",<5em,2cm>*\as{\xc{C}}="CV5",
		<6em,2cm>*\as{C}="CV6",<7em,2cm>*\as{C}="CV7",
		<8em,2cm>*\as{V}="CV8",<9em,2cm>*\as{C}="CV9",<10em,2cm>*\as{V}="CV10",<11em,2cm>*\as{C}="CV11",
		<1em,1cm>*\as{t}="cv1",<2em,1cm>*\as{o}="cv2",<3em,1cm>*\as{ }="cv3",<4em,1cm>*\as{o}="cv4",<5em,1cm>*\as{ }="cv5",
		<6em,1cm>*\as{n}="cv6",<7em,1cm>*\as{ɡw}="cv7",
		<8em,1cm>*\as{i}="cv8",<9em,1cm>*\as{ }="cv9",<10em,1cm>*\as{i}="cv10",<11em,1cm>*\as{ }="cv11",
		<2.5em,0cm>*\as{\hp{\sub{1}}M\sub{1}}="m1",<5em,0cm>*\as{=}="=",<6em,0cm>*\as{\hp{\sub{2}}M\sub{2}}="m2",
		<7.5em,0cm>*\as{=}="=",<9em,0cm>*\as{\hp{\sub{3}}M\sub{3}}="m3",
		"m1"+U;"cv1"+D**\dir{-};"m1"+U;"cv2"+D**\dir{-};"m1"+U;"cv3"+D**\dir{};"m1"+U;"cv4"+D**\dir{-};"m1"+U;"cv5"+D**\dir{};
		"m2"+U;"cv6"+D**\dir{-};"m3"+U;"cv8"+D**\dir{-};"m3"+U;"cv10"+D**\dir{-};
		"cv1"+U;"CV1"+D**\dir{-};"cv2"+U;"CV2"+D**\dir{-};"cv3"+U;"CV3"+D**\dir{};"cv4"+U;"CV4"+D**\dir{-};"cv5"+U;"CV5"+D**\dir{};
		"cv6"+U;"CV6"+D**\dir{-};"cv7"+U;"CV7"+D**\dir{-};"cv8"+U;"CV8"+D**\dir{-};
		"cv9"+U;"CV9"+D**\dir{};"cv10"+U;"CV10"+D**\dir{-};"cv11"+U;"CV11"+D**\dir{};
		"CV1"+U;"s1"+D**\dir{-};"CV2"+U;"s1"+D**\dir{-};"CV3"+U;"s1"+D**\dir{-};"CV3"+U;"s2"+D**\dir{-};"CV4"+U;"s2"+D**\dir{-};
		"CV6"+U;"s2"+D**\dir{-};{\ar@{-}|-(.45)*@{|} |-{\hole} |-(.575)*@{|} "CV5"+U;"s2"+D};
		"CV7"+U;"s4"+D**\dir{-};"CV8"+U;"s4"+D**\dir{-};"CV9"+U;"s4"+D**\dir{-};"CV9"+U;"s5"+D**\dir{-};"CV10"+U;"s5"+D**\dir{-};"CV11"+U;"s5"+D**\dir{-};
		"s1"+U;"Ft1"+D**\dir{-};"s2"+U;"Ft1"+D**\dir{-};"s4"+U;"Ft2"+D**\dir{-};"s5"+U;"Ft2"+D**\dir{-};
		"Ft1"+U;"PrWd1"+D**\dir{-};"PrWd1"+U;"PrWd2"+D**\dir{-};"Ft2"+U;"PrWd2"+D**\dir{-};
		<5.5em,2cm>*\as{\tikz[red,thick,dashed,baseline=0.9ex]\draw (0,0) rectangle (0.8cm,0.5cm);}="box",
	\endxy}
	\exi{j.}\exia{\xy
		<3em,5cm>*\as{\hp{\sub{1}}PrWd\sub{1}}="PrWd1",<5.5em,6cm>*\as{\hp{\sub{2}}PrWd\sub{2}}="PrWd2",
		<3em,4cm>*\as{\hp{\sub{1}}Ft\sub{1}}="Ft1",<8em,4cm>*\as{\hp{\sub{2}}Ft\sub{2}}="Ft2",
		<2em,3cm>*\as{\hp{\sub{1}}σ\sub{1}}="s1",<4em,3cm>*\as{\hp{\sub{2}}σ\sub{2}}="s2",
		<7em,3cm>*\as{\hp{\sub{4}}σ\sub{4}}="s4",<9em,3cm>*\as{\hp{\sub{5}}σ\sub{5}}="s5",
		<1em,2cm>*\as{C}="CV1",<2em,2cm>*\as{V}="CV2",<3em,2cm>*\as{C}="CV3",<4em,2cm>*\as{V}="CV4",<5em,2cm>*\as{C}="CV5",
		<6em,2cm>*\as{C}="CV6",<7em,2cm>*\as{V}="CV7",<8em,2cm>*\as{C}="CV8",<9em,2cm>*\as{V}="CV9",<10em,2cm>*\as{C}="CV10",
		<1em,1cm>*\as{t}="cv1",<2em,1cm>*\as{o}="cv2",<3em,1cm>*\as{ }="cv3",<4em,1cm>*\as{o}="cv4",<5em,1cm>*\as{n}="cv5",
		<6em,1cm>*\as{ɡw}="cv6",<7em,1cm>*\as{i}="cv7",<8em,1cm>*\as{ }="cv8",<9em,1cm>*\as{i}="cv9",<10em,1cm>*\as{ }="cv10",
		<2.5em,0cm>*\as{\hp{\sub{1}}M\sub{1}}="m1",<4em,0cm>*\as{=}="=",<5em,0cm>*\as{\hp{\sub{2}}M\sub{2}}="m2",
		<6.5em,0cm>*\as{=}="=",<8em,0cm>*\as{\hp{\sub{3}}M\sub{3}}="m3",
		"m1"+U;"cv1"+D**\dir{-};"m1"+U;"cv2"+D**\dir{-};"m1"+U;"cv3"+D**\dir{};"m1"+U;"cv4"+D**\dir{-};"m1"+U;"cv5"+D**\dir{};
		"m2"+U;"cv5"+D**\dir{-};"m3"+U;"cv7"+D**\dir{-};"m3"+U;"cv9"+D**\dir{-};
		"cv1"+U;"CV1"+D**\dir{-};"cv2"+U;"CV2"+D**\dir{-};"cv3"+U;"CV3"+D**\dir{};"cv4"+U;"CV4"+D**\dir{-};"cv5"+U;"CV5"+D**\dir{-};
		"cv6"+U;"CV6"+D**\dir{-};"cv7"+U;"CV7"+D**\dir{-};"cv8"+U;"CV8"+D**\dir{};"cv9"+U;"CV9"+D**\dir{-};"cv10"+U;"CV10"+D**\dir{};
		"CV1"+U;"s1"+D**\dir{-};"CV2"+U;"s1"+D**\dir{-};"CV3"+U;"s1"+D**\dir{-};"CV3"+U;"s2"+D**\dir{-};"CV4"+U;"s2"+D**\dir{-};
		"CV5"+U;"s2"+D**\dir{-};"CV6"+U;"s4"+D**\dir{-};"CV7"+U;"s4"+D**\dir{-};"CV8"+U;"s4"+D**\dir{-};"CV8"+U;"s5"+D**\dir{-};"CV9"+U;"s5"+D**\dir{-};"CV10"+U;"s5"+D**\dir{-};
		"s1"+U;"Ft1"+D**\dir{-};"s2"+U;"Ft1"+D**\dir{-};"s4"+U;"Ft2"+D**\dir{-};"s5"+U;"Ft2"+D**\dir{-};
		"Ft1"+U;"PrWd1"+D**\dir{-};"PrWd1"+U;"PrWd2"+D**\dir{-};"Ft2"+U;"PrWd2"+D**\dir{-};
		%<6em,2.5cm>*\as{\tikz[red,thick,dashed,baseline=0.9ex]\draw (0,0) rectangle (0.35cm,1.5cm);}="box",
	\endxy}
	\end{xlist}}
\end{exe}
\end{multicols}

Under this analysis it is only nominal stems ending in a surface vowel sequence
which take (or took) the allomorph \ve{=nu}.
When another enclitic is then added, the process
illustrated in \qf{as:too=ngw=ii1}--(\ref{as:too=ngw=ii}j) above occurs.

Similarly, for verbs a historic trace of the allomorph
\ve{=nu} is only preserved on VV{\#} final stems
when the enclitics \ve{=een} or \ve{=aah} are attached.
Other vowel-final verbs take the allomorph \ve{=n}.

Nominal stems ending in CV{\#} on the other hand,
take the allomorph \ve{=n},
which simply fills the final C-slot,
as illustrated for \ve{kase=n} `foreigners' in \qf{as:kasen} below.
When a vowel-initial enclitic is added, such stems then undergo metathesis.
The structure of \ve{kaes=n=ee} `the foreigners'
is shown in \qf{as:kaes=n=ee} below to illustrate.

\newpage
\begin{multicols}{2}
\begin{exe}
	\exa{\xy
		<3em,5cm>*\as{\hp{\sub{1}}PrWd\sub{1}}="PrWd",
		<3em,4cm>*\as{\hp{\sub{1}}Ft\sub{1}}="ft1",
		<2em,3cm>*\as{\hp{\sub{1}}σ\sub{1}}="s1",<4em,3cm>*\as{\hp{\sub{2}}σ\sub{2}}="s2",
		<1em,2cm>*\as{C}="CV1",<2em,2cm>*\as{V}="CV2",<3em,2cm>*\as{C}="CV3",<4em,2cm>*\as{V}="CV4",<5em,2cm>*\as{C}="CV5",
		<1em,1cm>*\as{k}="cv1",<2em,1cm>*\as{a}="cv2",<3em,1cm>*\as{s}="cv3",<4em,1cm>*\as{e}="cv4",<5em,1cm>*\as{n}="cv5",
		<2.5em,0cm>*\as{\hp{\sub{1}}M\sub{1}}="m1",<5em,0cm>*\as{\hp{\sub{2}}M\sub{2}}="m2",<4em,0cm>*\as{=}="=2",
		"m1"+U;"cv1"+D**\dir{-};"m1"+U;"cv2"+D**\dir{-};"m1"+U;"cv3"+D**\dir{-};"m1"+U;"cv4"+D**\dir{-};"m2"+U;"cv5"+D**\dir{-};
		"cv1"+U;"CV1"+D**\dir{-};"cv2"+U;"CV2"+D**\dir{-};"cv3"+U;"CV3"+D**\dir{-};"cv4"+U;"CV4"+D**\dir{-};"cv5"+U;"CV5"+D**\dir{-};
		"CV1"+U;"s1"+D**\dir{-};"CV2"+U;"s1"+D**\dir{-};"CV3"+U;"s1"+D**\dir{-};"CV3"+U;"s2"+D**\dir{-};"CV4"+U;"s2"+D**\dir{-};"CV5"+U;"s2"+D**\dir{-};
		"s1"+U;"ft1"+D**\dir{-};"s2"+U;"ft1"+D**\dir{-};
		"ft1"+U;"PrWd"+D**\dir{-};
	\endxy}\label{as:kasen}
	\exa{\xy
		<2.5em,5cm>*\as{\hp{\sub{1}}PrWd\sub{1}}="PrWd1",<5em,6cm>*\as{\hp{\sub{2}}PrWd\sub{2}}="PrWd2",
		<2.5em,4cm>*\as{\hp{\sub{\tsc{m}}}Ft\sub{\tsc{m}}}="ft1",<7em,4cm>*\as{\hp{\sub{2}}Ft\sub{2}}="ft2",
		<1.5em,3cm>*\as{\hp{\sub{1}}σ\sub{1}}="s1",<3.5em,3cm>*\as{\hp{\sub{2}}σ\sub{2}}="s2",<6em,3cm>*\as{\hp{\sub{3}}σ\sub{3}}="s3",<8em,3cm>*\as{\hp{\sub{4}}σ\sub{4}}="s4",
		<1em,2cm>*\as{C}="CV1",<2em,2cm>*\as{V}="CV2",<3em,2cm>*\as{V}="CV3",<4em,2cm>*\as{C}="CV4",<5em,2cm>*\as{C}="CV5",
		<6em,2cm>*\as{V}="CV6",<7em,2cm>*\as{C}="CV7",<8em,2cm>*\as{V}="CV8",<9em,2cm>*\as{C}="CV9",
		<1em,1cm>*\as{k}="cv1",<2em,1cm>*\as{a}="cv2",<3em,1cm>*\as{e}="cv3",<4em,1cm>*\as{s}="cv4",<5em,1cm>*\as{n}="cv5",
		<6em,1cm>*\as{e}="cv6",<7em,1cm>*\as{ }="cv7",<8em,1cm>*\as{e}="cv8",<9em,1cm>*\as{ }="cv9",
		<2.5em,0cm>*\as{\hp{\sub{1}}M\sub{1}}="m1",<5em,0cm>*\as{\hp{\sub{2}}M\sub{2}}="m2",<7em,0cm>*\as{\hp{\sub{3}}M\sub{3}}="m3",
		<4em,0cm>*\as{=}="=",<6.2em,0cm>*\as{=}="=2",
		"m1"+U;"cv1"+D**\dir{-};"m1"+U;"cv2"+D**\dir{-};"m1"+U;"cv3"+D**\dir{-};"m1"+U;"cv4"+D**\dir{-};"m2"+U;"cv5"+D**\dir{-};
		"m3"+U;"cv6"+D**\dir{-};"m3"+U;"cv8"+D**\dir{-};
		"cv1"+U;"CV1"+D**\dir{-};"cv2"+U;"CV2"+D**\dir{-};"cv3"+U;"CV3"+D**\dir{-};"cv4"+U;"CV4"+D**\dir{-};"cv5"+U;"CV5"+D**\dir{-};
		"cv6"+U;"CV6"+D**\dir{-};"cv8"+U;"CV8"+D**\dir{-};
		"CV1"+U;"s1"+D**\dir{-};"CV2"+U;"s1"+D**\dir{-};"CV3"+U;"s2"+D**\dir{-};"CV4"+U;"s2"+D**\dir{-};
		"CV5"+U;"s3"+D**\dir{-};"CV6"+U;"s3"+D**\dir{-};"CV7"+U;"s3"+D**\dir{-};"CV7"+U;"s4"+D**\dir{-};"CV8"+U;"s4"+D**\dir{-};"CV9"+U;"s4"+D**\dir{-};
		"s1"+U;"ft1"+D**\dir{-};"s2"+U;"ft1"+D**\dir{-};"s3"+U;"ft2"+D**\dir{-};"s4"+U;"ft2"+D**\dir{-};
		"ft1"+U;"PrWd1"+D**\dir{-};"PrWd1"+U;"PrWd2"+D**\dir{-};"ft2"+U;"PrWd2"+D**\dir{-};
	\endxy}\label{as:kaes=n=ee}
\end{exe}
\end{multicols}

The remaining piece of the puzzle is the observation that,
in the vast majority of my data,
nouns (but not verbs) which end in a surface vowel sequence
are pluralised with a double marking of the plural:
\ve{=nu} + \ve{=ein} {\ra} \ve{=ŋgwein}.
The reason for this double plural marking is unexplained.
Examples from \prf{ex:pl->=n/VVnoun}
are repeated in \qf{ex:pl->=n/VVnoun-2} below.

\begin{exe}
	\ex{\{\tsc{pl}\} {\ra} =ŋgwein Nominal, /VV{\#}{\gap}}\label{ex:pl->=n/VVnoun-2}
		\sn{\gw\begin{tabular}{rcll}
			\ve{bifee}	&{\ra}&\ve{bifee=\tbr{ŋgwein}}		&`women'	\\
			\ve{bi\j ae}&{\ra}&\ve{bi\j ae=\tbr{ŋgwein}}	&`cows'	\\
			\ve{oe}			&{\ra}&\ve{oe=\tbr{ŋgwein}}				&`kinds of water'	\\
			\ve{pentua}	&{\ra}&\ve{pentua=\tbr{ŋgwein}}		&`church elders'	\\
			\ve{too}		&{\ra}&\ve{too=\tbr{ŋgwein}}			&`citizens'	\\
			\ve{hau}		&{\ra}&\ve{hau=\tbr{ŋgwein}}			&`trees'	\\
		\end{tabular}}
\end{exe}
\section{Multiple enclitics}\label{sec:ConInsConIns}
Amarasi allows sequences of enclitics to occur.
When the second enclitic is vowel-initial,
it usually triggers the normal processes of metathesis
and consonant insertion on the previous enclitic,
though there are some exceptions.

A number of examples of \ve{=een} `{\een}'
attached to \ve{=ee} `{\ee}/{\eeV}'
are given in \qf{ex:=ee+en->=eej=een} below.
All these examples show expected insertion of /\j/
before \ve{=een} as conditioned by the final vowel of \ve{=ee}.\footnote{
		There is one example in my corpus in which /ɡw/
		is irregularly inserted after \ve{=ee}. This is \ve{hai mi-ʔuab=eegw=een}
		`we've already spoken about it' ({\hai} \mi-speak={\ee=\een}).}

\begin{exe}
	\ex{\ve{=ee} + \ve{=een} {\ra} \ve{=ee\j=een} \label{ex:=ee+en->=eej=een}}
		\sn{\stl{0.35em}\gw\begin{tabular}{llll}
			\ve{na-sopu}		&+ \ve{=ee} + \ve{=een} \ra& \ve{na-soopgw=ee\j=een}	& `finished it' \\
			\ve{buku}				&+ \ve{=ee} + \ve{=een} \ra& \ve{buukgw=ee\j=een}		& `the book (already)' \\
			\ve{mepu}				&+ \ve{=ee} + \ve{=een} \ra& \ve{meepgw=ee\j=een}		& `the work (already)' \\
			\ve{na-kratiʔ}	&+ \ve{=ee} + \ve{=een} \ra& \ve{na-kraitʔ=ee\j=een} & `destroyed it' \\
			\ve{n-porin} 		&+ \ve{=ee} + \ve{=een} \ra& \ve{n-poirn=ee\j=een} 		& `threw it' \\
			\ve{n-isa} 			&+ \ve{=ee} + \ve{=een} \ra& \ve{n-iis=ee\j=een} 			& `defeated him' \\
			\ve{ʔsobeʔ} 		&+ \ve{=ee} + \ve{=een} \ra& \ve{ʔsoebʔ=ee\j=een} & `the hat (already)' \\
		\end{tabular}}
\end{exe}

In such examples the metathesis of the penultimate clitic
is not detectable as the first clitic has a sequence of two identical vowels.
Examples of vowel-initial enclitics attached
to pronominal enclitics showing vowel assimilation are given in
\qf{ex:130825-7, 0.11} and \qf{ex:130825-6, 19.24} below.

\begin{exe}
	\ex{\glll	n-aan=\tbr{kaagw}=\tbr{ii} onai, \sf{pak} \\
						n-ana=\tbr{kau}=\tbr{ii} onai \sf{pak} \\
						\n-get={\kau}={\ii} like.this dad \\
			\glt	`S/he got me like this, dad.'
						\txrf{130825-7, 0.11}{\emb{130825-7-00-11.mp3}{\spk{}}{\apl}}}\label{ex:130825-7, 0.11}
	\ex{\glll a|n-\sf{koleŋ}=\tbr{kaa\j}=\tbr{ena} =m he t-pasan reʔ ʔfutuʔ \\
						{\a}n-\sf{koleŋ}=\tbr{kai}=\tbr{ena} =ma he t-pasan reʔ ʔfutuʔ \\
						{\a\n}-call={\kai=\een} =and {\he} \t-tie {\reqt} belt \\
			\glt \lh{a|}`They started calling (to) us to tie the seatbelt.'
						\txrf{130825-6, 19.24}{\emb{130825-6-19-24.mp3}{\spk{}}{\apl}}}\label{ex:130825-6, 19.24}
\end{exe}

Under the analysis of vowel assimilation given in \srf{sec:VowAss},
the assimilation of the final vowels of \ve{=ka\tbr{u}} `{\kau}' {\ra} \ve{=ka\tbr{a}gw}
and \ve{=ka\tbr{i}} {\ra} \ve{=ka\tbr{a}\j} `{\kai}' in such examples
is due to metathesis of the medial empty C-slot.
Similarly, as discussed in \srf{sec:ConIns}, the insertion of the consonant
in all such examples occurs to provide the following foot with an onset consonant.

That consonant insertion and metathesis affect
enclitics when an additional enclitic is added
has exactly the same explanation as that for every other clitic host.
Phrases such as \ve{n-aan=kaagw=ii} `got me'
or \ve{meepgw=ee\j=een} `the work (already)'
both have two internal prosodic words,
with a crisp edge required after each.
The phonological and morphological structures
of \ve{meepgw=ee\j=een} `the work (already)' are shown in \qf{as:meepgw=eej=een}
below with the crisp edges indicated.\footnote{
		The use of the M\=/form for the enclitic \ve{=ena/=een} `{\een}' in \qf{as:meepgw=eej=een}
		is due to discourse structures, as discussed in Chapter \ref{ch:DisMet}}

\begin{exe}
	\exa{\xy
		<2.5em,5cm>*\as{\hp{\sub{1}}PrWd\sub{1}}="PrWd1",<4.5em,6cm>*\as{\hp{\sub{2}}PrWd\sub{2}}="PrWd2",<6em,7cm>*\as{\hp{\sub{3}}PrWd\sub{3}}="PrWd3",
		<2.5em,4cm>*\as{\hp{\sub{\tsc{m}}}Ft\sub{\tsc{m}}}="ft1",
		<6.5em,4cm>*\as{\hp{\sub{\tsc{m}}}Ft\sub{\tsc{m}}}="ft2",<10.5em,4cm>*\as{\hp{\sub{\tsc{m}}}Ft\sub{\tsc{m}}}="ft3",
		<1.5em,3cm>*\as{\hp{\sub{1}}σ\sub{1}}="s1",<3.5em,3cm>*\as{\hp{\sub{2}}σ\sub{2}}="s2",
		<5.5em,3cm>*\as{\hp{\sub{3}}σ\sub{3}}="s3",<7.5em,3cm>*\as{\hp{\sub{4}}σ\sub{4}}="s4",
		<9.5em,3cm>*\as{\hp{\sub{5}}σ\sub{5}}="s5",<11.5em,3cm>*\as{\hp{\sub{6}}σ\sub{6}}="s6",
		<1em,2cm>*\as{C}="CV1",<2em,2cm>*\as{V}="CV2",<3em,2cm>*\as{V}="CV3",<4em,2cm>*\as{C}="CV4",<5em,2cm>*\as{C}="CV5",
		<6em,2cm>*\as{V}="CV6",<7em,2cm>*\as{V}="CV7",<8em,2cm>*\as{C}="CV8",
		<9em,2cm>*\as{C}="CV9",<10em,2cm>*\as{V}="CV10",<11em,2cm>*\as{V}="CV11",<12em,2cm>*\as{C}="CV12",
		<1em,1cm>*\as{m}="cv1",<2.5em,1cm>*\as{e}="cv2",<3em,1cm>*\as{ }="cv3",<4em,1cm>*\as{p}="cv4",<5em,1cm>*\as{gw}="cv5",
		<6.5em,1cm>*\as{e}="cv6",<7em,1cm>*\as{}="cv7",<8em,1cm>*\as{}="cv8",<9em,1cm>*\as{\j}="cv9",
		<10em,1cm>*\as{e}="cv10",<11em,1cm>*\as{e}="cv11",<12em,1cm>*\as{n}="cv12",
		<2.5em,0cm>*\as{\hp{\sub{1}}M\sub{1}}="m1",<6.5em,0cm>*\as{\hp{\sub{2}}M\sub{2}}="m2",<11em,0cm>*\as{\hp{\sub{3}}M\sub{3}}="m3",
		<5em,0cm>*\as{=}="=1",<9em,0cm>*\as{=}="=2",
		"m1"+U;"cv1"+D**\dir{-};"m1"+U;"cv2"+D**\dir{-};"m1"+U;"cv4"+D**\dir{-};"m2"+U;"cv6"+D**\dir{-};
		"m3"+U;"cv10"+D**\dir{-};"m3"+U;"cv11"+D**\dir{-};"m3"+U;"cv12"+D**\dir{-};
		"cv1"+U;"CV1"+D**\dir{-};"cv2"+U;"CV2"+D**\dir{-};"cv2"+U;"CV3"+D**\dir{-};"cv4"+U;"CV4"+D**\dir{-};"cv5"+U;"CV5"+D**\dir{-};
		"cv6"+U;"CV6"+D**\dir{-};"cv6"+U;"CV7"+D**\dir{-};"cv9"+U;"CV9"+D**\dir{-};
		"cv10"+U;"CV10"+D**\dir{-};"cv11"+U;"CV11"+D**\dir{-};"cv12"+U;"CV12"+D**\dir{-};
		"CV1"+U;"s1"+D**\dir{-};"CV2"+U;"s1"+D**\dir{-};"CV3"+U;"s2"+D**\dir{-};"CV4"+U;"s2"+D**\dir{-};
		"CV5"+U;"s3"+D**\dir{-};"CV6"+U;"s3"+D**\dir{-};"CV7"+U;"s4"+D**\dir{-};"CV8"+U;"s4"+D**\dir{-};
		"CV9"+U;"s5"+D**\dir{-};"CV10"+U;"s5"+D**\dir{-};"CV11"+U;"s6"+D**\dir{-};"CV12"+U;"s6"+D**\dir{-};
		"s1"+U;"ft1"+D**\dir{-};"s2"+U;"ft1"+D**\dir{-};"s3"+U;"ft2"+D**\dir{-};"s4"+U;"ft2"+D**\dir{-};"s5"+U;"ft3"+D**\dir{-};"s6"+U;"ft3"+D**\dir{-};
		"ft1"+U;"PrWd1"+D**\dir{-};"PrWd1"+U;"PrWd2"+D**\dir{-};"ft2"+U;"PrWd2"+D**\dir{-};"PrWd2"+U;"PrWd3"+D**\dir{-};"ft3"+U;"PrWd3"+D**\dir{-};
		<8.5em,3.5cm>*\as{\tikz[red,thick,dashed,baseline=0.9ex]\draw (0,0) -- (0,4.5cm);}="line",
		<4.5em,3.5cm>*\as{\tikz[red,thick,dashed,baseline=0.9ex]\draw (0,0) -- (0,4.5cm);}="line",
	\endxy}\label{as:meepgw=eej=een}
\end{exe}

While the normal process occurs in most cases
when a second vowel-initial enclitic is added,
there are a small number of exceptions.
The first exception is insertion of /ɡw/
when an enclitic attaches to an enclitic which has
already triggered insertion of /\j/.

Examples are scarce.
I have located only four in \citet{or16c} and nine
in the Amarasi Bible translation,
yielding the five unique examples given in \qf{ex:j=ee+en->=ee=gwen} below.
Nonetheless, native speakers reject forms with two insertions of /\j/ as ungrammatical, thus
\ve{\tcb{*}oo\j=ee\j=een} `the water already' or \ve{\tcb{*}n-raar\j=ee\j=een}.
There is, however, a single example in my corpus: \ve{n-heek\j=ee\j=een} `caught it already'.

\begin{exe}
	\ex{\ve{\j=ee} + \ve{=een} {\ra} \ve{\j=eegw=een} \label{ex:j=ee+en->=ee=gwen}}
		\sn{\stl{0.4em}\gw\begin{tabular}{llll}
			\ve{oe}			&+ \ve{=ee} + \ve{=een} \ra& \ve{oo\tbr{\j}=ee\tbr{gw}=een}			& `the water (already)' \\
			\ve{ʔ-piri}	&+ \ve{=ee} + \ve{=een} \ra& \ve{ʔ-piir\tbr{\j}=ee\tbr{gw}=een}	& `(I've) chosen him' \\
			\ve{n-moʔe}	&+ \ve{=ee} + \ve{=een} \ra& \ve{n-mooʔ\tbr{\j}=ee\tbr{gw}=een}	& `(s/he's) made it' \\
			\ve{ʔ-eki}	&+ \ve{=ee} + \ve{=een} \ra& \ve{ʔ-eek\tbr{\j}=ee\tbr{gw}=een}	& `(I've) brought him' \\
			\ve{n-rari}	&+ \ve{=ee} + \ve{=een} \ra& \ve{n-raar\tbr{\j}=ee\tbr{gw}=een}	& `finished it' \\
		\end{tabular}}
\end{exe}

Insertion of /ɡw/ after insertion of /\j/ is probably a kind of dissimilation.
After /\j/ has been inserted, insertion of a second /\j/ is blocked.
Thus, the default medial consonant /ɡw/ is inserted at the second clitic boundary.

Another possible case of dissimilatory consonant insertion occurs
when the inceptive enclitic \ve{=een} occurs attached
to the Indonesian loanword \it{estiga} `PhD, doctoral degree'.\footnote{
		The phrase \it{estiga} is borrowed from Indonesian S3,
		an abbreviation of \it{sarjana tiga} `third bachelors/scholar'.}
In this case the consonant /\j/ is inserted,
perhaps due to the presence of [ɡ] in the clitic host
and/or the penultimate vowel being /i/.
I heard this phrase not infrequently during my fieldwork
after explaining I was learning Amarasi for my PhD.
It is given in \qf{ex:estiga=jen} below.

\begin{exe}
	\ex{\gll	\sf{estiga}\j=een\\
						PhD={\een}\\
			\glt	`(So, you're) now doing a PhD?' \txrf{observation}}\label{ex:estiga=jen}
\end{exe}

\largerpage
The second exception involving multiple
enclitics is when an enclitic attaches to the plural enclitic \ve{=eni/=ein},
or the pronominal enclitics
\ve{=kiti/=kiit} `{\kiit}', or \ve{=sini/=siin} `{\siin}'.
These enclitics occur in the M\=/form
before anther enclitic, but consonant insertion
does not usually occur.
Examples are given in \qf{ex:NoIns} below.

\begin{exe}
	\ex{No consonant insertion after \ve{=ein}, \ve{=kiit} and \ve{=siin}\label{ex:NoIns}}
		\sn{\stl{0.3em}\gw\begin{tabular}{rclclcll}
			\ve{anah}		&+&\ve{=ein}&+&\ve{=aa}	&\ra&\ve{aanh=ein=aa}		&`the children'\\
			\ve{bareʔ}	&+&\ve{=ein}&+&\ve{=ee}	&\ra&\ve{baerʔ=ein=ee}	&`the stuff'\\
			\ve{upu-ʔ}	&+&\ve{=ein}&+&\ve{=ee}	&\ra&\ve{uup-ʔ=ein=ee}	&`the grandchildren'\\
			%\ve{bae-f}	&+&\ve{=ein}&+&\ve{=ee}	&\ra&\ve{bae-f=ein-=ee}	&`brothers-in-law/mates'\\
			\ve{papaʔ}	&+&\ve{=ein}&+&\ve{=ii}	&\ra&\ve{paapʔ=ein=ii}	&`the wounds'\\
			\ve{neka-m}	&+&\ve{=ein}&+&\ve{=ii}	&\ra&\ve{neek-m=ein=ii}	&`your feelings'\\
			\ve{tua-k}	&+&\ve{=ein}&+&\ve{=ii}	&\ra&\ve{tua-k=ein=ii}	&`their-selves'\\
			\ve{n-saen=n}	&+&\ve{=kiit}&+&\ve{=een}&\ra&\ve{n-sae=n=kiit=een}&`loaded on us'\\
			\ve{na-pein}	&+&\ve{=siin}&+&\ve{=een}&\ra&\ve{n-pein=siin=een}&`has got them'\\
	%		\ve{=n}&+&\ve{=ein}&+&\ve{=}&\ra&\ve{=n=gwen}&`'\\
		\end{tabular}}
\end{exe}

As discussed in Chapter \ref{ch:DisMet}, the M\=/form
of these enclitics (and a number of other word classes) is the default form.
Thus, we can propose that subsequent enclitics attach to the
(consonant-final) default form of these enclitics.

Verbs also have a default M\=/form,
but consonant insertion \emph{does} occur after verbs.
The difference between verbs and enclitics is probably
due to the productivity of the U\=/form/M\=/form alternation.
For verbs this alternation is completely productive
while for enclitics it is less productive
and the U\=/forms are rarely used.
Thus for verbs, the U\=/form is still the underlying
morphological form, while for enclitics the M\=/form
may be the underlying morphological form.

A second reason that consonant insertion does not
occur after the pronominal enclitics \ve{=kiit} `{\kiit}' and \ve{=siin} `{\siin}',
is because these have alternate U\=/forms
with a final /a/: \ve{=kita} and \ve{=sina} (\srf{sec:IrrMfor}).
These forms are more conservative than the U\=/forms \ve{=kiti} and \ve{=sini},
as can be seen by comparing them with their Proto-Malayo-Polynesian etyma *kita and *sida.
Before the innovation of the U\=/forms \ve{=kiti} and \ve{=sini},
a second enclitic would have attached to /a/ final forms
after which consonant insertion does not occur (\srf{sec:CliHosFinA}).
This older pattern has been retained after
the innovation of \ve{=kiti} and \ve{=sini}.

The usual pattern after \ve{=ein} `{\ein}',
\ve{=kiit} `{\kiit}' and \ve{=siin} `{\siin}' is
for no consonant to be inserted after the attachment of a vowel-initial enclitic.
However, there are sporadic examples in which /ɡw/ is inserted after \ve{=ein}.
I have so far found three examples, one in my corpus
and two in the Amarasi Bible translation,
all given in \qf{ex:=ein=gwX} below.
In this case the unexpected insertion
of /ɡw/ may be by analogy with insertion of /ɡw/
after \ve{=n} and \ve{=nu} (\srf{sec:ConInsPluEnc}).\footnote{
		There is also one example of insertion of /ɡw/ in my corpus
		after the alternate plural enclitic \ve{=enu/=uun}:
		\ve{oemetan} `dirty' + \ve{=uun} `{\ein}' + \ve{ii} `{\ii}' {\ra}
		\ve{oemeetn=uuŋg=ii} `the dirty one'}

\begin{exe}
	\ex{=ein + =V {\ra} =eiŋgwV\label{ex:=ein=gwX}}
		\sn{\stl{0.15em}\gw\begin{tabular}{rclclcll}
%			\ve{oemetan}	&+&\ve{=uun}&+&\ve{=ii}	&{\ra}&\ve{oemeetn=uu{\ng}gw=ii}&`the dirty (ones)'\\
			\ve{skora-m}	&+&\ve{=ein}&+&\ve{=ii}	&{\ra}&\ve{skoor-m=ei{\ng}gw=ii}&`your schooling'\\
			\ve{anah}			&+&\ve{=ein}&+&\ve{=aa}	&{\ra}&\ve{aanh=ei{\ng}gw=aa}		&`the children'\\
			%\ve{a-toup} 	& &					& &					&			&\ve{a-toup}							&								\\ \hhline{~}
			%\ve{noniʔ}		&+&\ve{=ein}&+&\ve{=aa}	&{\ra}&\ve{noinʔ=ei{\ng}gw=aa}	&`the disciples'\\
			\ve{a-toup noniʔ}		&+&\ve{=ein}&+&\ve{=aa}	&{\ra}&\ve{a-toup noinʔ=ei{\ng}gw=aa}	&`the disciples'\\
		\end{tabular}}
\end{exe}


\section{Historical development}\label{sec:HisDev}
In this section I present some comparative data from
other varieties of Meto which indicates
that Amarasi consonant insertion before vowel-initial enclitics
arose through fortition of an earlier glide.
This comparative data also provides the crucial
data which leads me to analyse vowel-initial enclitics
as containing at least two vowels.

In Amanuban the attachment of a vowel-initial
enclitic to a VV{\#} final host triggers insertion
of the glide /w/ after back vowels and /j/ after front vowels
Examples are given in \qf{ex:AmanuGLiIns} below.
Data comes from a speaker from Niki Niki
(central Amanuban) and Noemuke (south Amanuban).

\begin{exe}
	\ex{Amanuban glide insertion}\label{ex:AmanuGLiIns}
	\sn{\gw\begin{tabular}{rlllll}
		\ve{ai} 	&+&\ve{=ees}&{\ra}&\ve{ai\tbr{j}ees}		& `one fire' \\
		\ve{tei} 	&+&\ve{=ees}&{\ra}&\ve{tei\tbr{j}ees}	& `one (pile of) dung' \\
		\ve{oe} 	&+&\ve{=ees}&{\ra}&\ve{oe\tbr{j}ees}		& `one (body of) water' \\
		\ve{fee} 	&+&\ve{=ees}&{\ra}&\ve{fee\tbr{j}ees}	& `one wife' \\
		\ve{ao} 	&+&\ve{=ees}&{\ra}&\ve{ao\tbr{w}ees}		& `one (container of) slaked lime' \\
		\ve{too} 	&+&\ve{=ees}&{\ra}&\ve{too\tbr{w}ees}	& `one population' \\
		\ve{hau} 	&+&\ve{=ees}&{\ra}&\ve{hau\tbr{w}ees}	& `one tree/piece of wood' \\
		\ve{kiu} 	&+&\ve{=ees}&{\ra}&\ve{kiu\tbr{w}ees}	& `one tamarind tree' \\
		%\ve{} 	&+&\ve{=ees}&{\ra}&\ve{\tbr{j}=ee}	& `one' \\
	\end{tabular}}
\end{exe}

Consonant insertion in Amarasi after VV{\#} sequences
is probably a development from an Amanuban-like system,
with the addition of vowel assimilation 
and insertion of voiced obstruents rather than glides.

Regarding the voiced obstruents, Amarasi /\j/ and /ɡw/
result from fortition of *j > \ve{\j} and *w > \ve{gw}.
Other Amanuban/Amarasi cognates showing such glide fortition
include Amanuban \ve{ai\tbr{j}oo}, Amarasi \ve{ai\tbr{\j}oʔo} `casuarina tree'
and Amanuban \ve{nai\tbr{j}eeʔ} Amarasi \ve{nai\tbr{\j}eer} `ginger'.

In Amanuban when a vowel-initial enclitic is attached
to a CV{\#} final word the first vowel of the enclitic assimilates to the quality
of the final vowel of the host, the host undergoes metathesis,
and the final vowel of the host assimilates to the quality of the previous vowel.
Examples are given in \qf{ex:AmaVowAss} below.
Vowel assimilation does not otherwise affect nouns
after metathesis in Amanuban, thus \ve{fa\tbr{fi}} + \ve{anaʔ}
`small, baby' {\ra} \ve{fa\tbr{if} anaʔ} `piglet'.\footnote{
		The way the process exemplified in \qf{ex:AmaVowAss}
		may be connected with the metathesis with
		final non-syllabic glides attested in
		some varieties of Amanuban (see \srf{sec:OriMetAma})
		deserves further investigation.}

\begin{exe}
	\ex{Amanuban vowel assimilation}\label{ex:AmaVowAss}
	\sn{\gw\begin{tabular}{rlllll}
		\ve{faf\tbr{i}}	&+&\ve{=\tbr{e}es}&{\ra}&\ve{fa\tbr{afi}es}	& `one pig' \\
		\ve{uk\tbr{i}}	&+&\ve{=\tbr{e}es}&{\ra}&\ve{u\tbr{uki}es}		& `one banana tree' \\
		\ve{bes\tbr{i}}	&+&\ve{=\tbr{e}es}&{\ra}&\ve{be\tbr{esi}es}	& `one knife' \\
		\ve{mon\tbr{e}}	&+&\ve{=\tbr{e}es}&{\ra}&\ve{mo\tbr{one}es}	& `one husband' \\
		\ve{um\tbr{e}}	&+&\ve{=\tbr{e}es}&{\ra}&\ve{u\tbr{ume}es}		& `one house' \\
		\ve{nen\tbr{o}}	&+&\ve{=\tbr{e}es}&{\ra}&\ve{ne\tbr{eno}es}	& `one day' \\
		\ve{kol\tbr{o}}	&+&\ve{=\tbr{e}es}&{\ra}&\ve{ko\tbr{olo}es}	& `one bird' \\
		%\ve{non\tbr{o}}	&+&\ve{=\tbr{e}es}&{\ra}&\ve{no\tbr{ono}es}	& `one stem/stick' \\
		\ve{as\tbr{u}} 	&+&\ve{=\tbr{e}es}&{\ra}&\ve{a\tbr{asu}es}	& `one dog' \\
		\ve{han\tbr{u}} &+&\ve{=\tbr{e}es}&{\ra}&\ve{ha\tbr{anu}es}	& `one mortar' \\
		\ve{tef\tbr{u}}	&+&\ve{=\tbr{e}es}&{\ra}&\ve{te\tbr{efu}es}	& `one sugar-cane stalk' \\
		%\ve{} 	&+&\ve{=ees}&{\ra}&\ve{}e}es}	& `one' \\
	\end{tabular}}
\end{exe}

The first vowel of the enclitic in such examples
is often quite short, and in the case of /i/ sometimes reduced to a glide.
Thus, \ve{uki} `banana' + \ve{=ees} `one' {\ra} \ve{uukies} {\ra} [ˈʔʊːkĭɛs] {\tl} [ˈʔʊːkjɛs].

Regarding stems ending in /u/, there seems to be variation
between metathesis and vowel assimilation, as illustrated
above, and simple attachment of the enclitic with no further changes.
An example of the latter in my data
is \ve{fatu} + \ve{=ees} {\ra} \ve{fatu=ees} [ˈfatʊɛs]{\tl}[ˈfatʊwɛs] `one stone/rock'.

The vowel assimilations which take place in Amanuban
when a vowel-initial enclitic is attached to a CV{\#}
stem could also represent a precursor to the process
of consonant insertion and vowel assimilation in Amarasi.

A system intermediate between that of Amanuban,
with assimilation of the first vowel of the enclitic,
and that of Kotos Amarasi, with consonant insertion,
is attested in Ro{\Q}is Amarasi from Buraen.

In Buraen Ro{\Q}is Amarasi /b/ is inserted after back vowels
and /\j/ after front vowels.
The first vowel of the enclitic also assimilates
as conditioned by the quality of the vowels of the enclitic.
If the enclitic is \ve{=aa} `{\aa}' the first vowel undergoes
complete assimilation, while if the enclitic is \ve{=ii} `{\ii}'
\ve{=ee} `{\ee}/{\eeV}' or \ve{=ees} `one' the first vowel
assimilates to the backness, but not the height of the final vowel of the clitic host.
Examples are given in \qf{ex:Buraen} below.
I do not have sufficient data on the behaviour of other
enclitics in Buraen Ro{\Q}is after vowel-final hosts to state their behaviour.\footnote{
		There are also two examples in my Buraen Ro{\Q}is data from a single speaker
		in which consonant insertion does not take place
		after the phrase \ve{aan feto} `daughter' + \ve{=ee}
		{\ra} \ve{aan feet=oe} `the daughter' and + \ve{=ees}
		{\ra} \ve{aan feet=oes} `one daughter'.
		These examples probably represent a more conservative pattern.
		The same speaker has consonant insertion in other situations,
		e.g. \ve{noo tenu} + \ve{=ii} {\ra} \ve{noo teen\tbr{b}ui} `the third time'.}

\begin{exe}
	\ex{Buraen Ro{\Q}is consonant insertion and vowel assimilation}\label{ex:Buraen}
		\sn{\stl{0.4em}\gw\begin{tabular}{rlllll}
		\ve{kor\tbr{o}} 	&+&\ve{=\tbr{a}a}&{\ra}&\ve{ko\tbr{orbo}a}		& `the bird' \\
		\ve{nen\tbr{o}} 	&+&\ve{=\tbr{e}e}&{\ra}&\ve{ne\tbr{enbo}e}		& `the sky' \\
		\ve{n-top\tbr{u}} &+&\ve{=\tbr{e}e}&{\ra}&\ve{n-to\tbr{opbo}e}	& `receives it' \\
		\ve{nif\tbr{u}} 	&+&\ve{=\tbr{e}es}&{\ra}&\ve{ni\tbr{ifbo}es}	& `one thousand' \\
		\ve{aan fet\tbr{o}} 	&+&\ve{=\tbr{i}i}&{\ra}&\ve{aan fe\tbr{etbu}i}		& `the daughter' \\
		\ve{ten\tbr{u}} 	&+&\ve{=\tbr{i}i}&{\ra}&\ve{te\tbr{enbu}i}		& `the third' \\
		\ve{braf\tbr{i}} 	&+&\ve{=\tbr{a}a}&{\ra}&\ve{bra\tbr{af{\j}i}a}	& `sea cucumber' \\
		\ve{te\tbr{i}} 		&+&\ve{=\tbr{a}a}&{\ra}&\ve{te\tbr{e{\j}i}a}		& `the faeces' \\
		\ve{meʔ\tbr{e}} 	&+&\ve{=\tbr{a}a}&{\ra}&\ve{me\tbr{eʔ{\j}e}a}	& `the red ones' \\
		\ve{mon\tbr{e}} 	&+&\ve{=\tbr{a}a}&{\ra}&\ve{mo\tbr{on{\j}e}a}	& `the husband' \\
		\ve{fe\tbr{e}} 		&+&\ve{=\tbr{a}a}&{\ra}&\ve{fe\tbr{e{\j}e}a}		& `the wife' \\
	\end{tabular}}
\end{exe}

As in Amanuban, the first vowel of the enclitic 
is usually extremely short,
and in Buraen Ro{\Q}is when this vowel is back rounded 
it can be realised as a glide [w],
thus \ve{aan feto} `daughter' + \ve{=ii} {\ra}
\ve{aan feetbui} [ˌʔaˑnˈfɛːtbʊ̆i] {\tl} [ˌʔaˑnˈfɛːtbwi]. 

Assimilation of the first vowel of \ve{=aa} after front
vowels does not seem to be obligatory in Buraen Ro{\Q}is.
One example from my data is \ve{umi} `house' + \ve{=aa}
`{\aa}' {\ra} \ve{uum\j=aa} `the house' in my data.

Ro{\Q}is Amarasi from Tunbaun is similar,
though after back rounded vowels the first vowel
of the enclitic does not undergo assimilation,
and /ɡw/ is usually inserted.
Historically, the /ɡw/ at clitic boundaries in 
Tunabun and Kotos Amarasi may come from re-analysis of
[ɡ] and an initial back vowel of the following clitic,
though examples such as Kotos \ve{ai\j oʔo} + \ve{=esa}
`one{\U}' {\ra} \ve{ai\j ooʔgw=esa} `one casuarina tree'
rather than \ve{*ai\j ooʔg=osa} indicate that this
glide can no longer be analysed as an underlying vowel.

A system similar to that of Ro{\Q}is Amarasi
operates in the variety of Meto spoken in Oepaha,
though in this case I only have data for the enclitic \ve{=aa}
and one example of \ve{=ii}.
In Oepaha /b/ is inserted after /o/, /l/ after /e/
and /\j/ is inserted after /i/.
Examples are given in \qf{ex:Oepaha} below.
Assimilation of the first vowel of the enclitic does
not take place in all examples, though data is too
scarce to state any conditions.\footnote{
		Oepaha data is limited, coming from a single wordlist and text.
		Possessed nouns were usually cited with the
		first person inclusive pronoun \ve{hiit} as a default possessor.}

\begin{exe}
	\ex{Oepaha consonant insertion and vowel assimilation}\label{ex:Oepaha}
	\sn{\stl{0.4em}\gw\begin{tabular}{rlllll}
		\ve{kmi\tbr{i}} &+ \ve{=aa} {\ra}&\ve{hiti kmi\tbr{i{\j}i}a}&[hɪt̪ɪkˈmiːʒia]	&\emb{hiti-kmiij-ia.mp3}{\spk{}}{\apl}		& `our urine' \\
		\ve{te\tbr{i}} 	&+ \ve{=aa} {\ra}&\ve{hiit te\tbr{e{\j}i}a}	&[hiˈt̪ːeː{{\j}}ɪa]&\emb{hit-teej-ia.mp3}{\spk{}}{\apl}			& `our faeces' \\
		\ve{uk\tbr{i}} 	&+ \ve{=aa} {\ra}&\ve{u\tbr{uk{\j}i}a}			&[ˈʔʊːkʒɪa]			&\emb{uukj-ia.mp3}{\spk{}}{\apl}					& `banana' \\
		\ve{o\tbr{o}} 	&+ \ve{=aa} {\ra}&\ve{o\tbr{obo}a}					&[ˈʔɔːβwɐ]			&\emb{oob-oa.mp3}{\spk{}}{\apl}						& `bamboo' \\
		\ve{nen\tbr{o}} &+ \ve{=aa} {\ra}&\ve{ne\tbr{enb}aa}				&[ˈnɛːnbɐt̪ʊnːa]	&\emb{neenb-oa-tuunn-aa.mp3}{\spk{}}{\apl}& `sky' \\ \hhline{~}
							&								 			 &\,\ve{tuun-n=aa}					&[ˈnɛːnβɐt̪ʊnːa]	&\emb{neenb-oa-tuunn-aa2.mp3}{\spk{}}{\apl}& \\
		\ve{mon\tbr{e}} &+ \ve{=aa} {\ra}&\ve{hiit mo\tbr{onle}a}		&[hit̪̚ ˈmɔːnlɛa]	&\emb{hit-moonl-ea.mp3}{\spk{}}{\apl}			& `our husband' \\
		\ve{fe\tbr{e}} 	&+ \ve{=aa} {\ra}&\ve{hiit fe\tbr{el}aa}			&[hɪt̪ˈfɛːla]		&\emb{hit-feel-aa.mp3}{\spk{}}{\apl}			& `our wife' \\
		\ve{fe\tbr{e}} 	&+ \ve{=ii} {\ra}&\ve{fe\tbr{el}ii}					&			&& `the wife' \\
		%\ve{\tbr{}} 	&+&\ve{=aa}&{\ra}&\ve{\tbr{e}a}	& `' \\
	\end{tabular}}
\end{exe}

Finally, although slightly orthogonal to the development
of consonant insertion in Kotos Amarasi,
the processes described for Amanuban CV{\#}
stems and vowel-initial enclitics also affect stems
with a final glottal stop CVʔ{\#}.
The first vowel of the enclitic assimilates to the quality
of the final vowel of the host, the host undergoes metathesis,
and the final vowel of the host assimilates to the quality of the previous vowel.
Examples are given in \qf{ex:AmaVowAssGlot} below.\footnote{
		I do not have data from Amanuban for the behaviour
		of words with a final consonant other than the glottal stop.}

\newpage
\begin{exe}
	\ex{Amanuban vowel assimilation with /ʔ/}\label{ex:AmaVowAssGlot}
	\sn{\gw\begin{tabular}{rlllll}
		\ve{as\tbr{i}ʔ} 	&+&\ve{=\tbr{e}es}&{\ra}&\ve{a\tbr{a}sʔ=\tbr{i}es}		& `one flea' \\
		\ve{mas\tbr{i}ʔ} 	&+&\ve{=\tbr{e}es}&{\ra}&\ve{ma\tbr{a}sʔ=\tbr{i}es}		& `one packet of salt' \\
		\ve{sun\tbr{i}ʔ} 	&+&\ve{=\tbr{e}es}&{\ra}&\ve{su\tbr{u}nʔ=\tbr{i}es}		& `one sword' \\
		\ve{kbat\tbr{e}ʔ} &+&\ve{=\tbr{e}es}&{\ra}&\ve{kba\tbr{a}tʔ=\tbr{e}es}	& `one grub' \\
		\ve{ten\tbr{o}ʔ} 	&+&\ve{=\tbr{e}es}&{\ra}&\ve{te\tbr{e}nʔ=\tbr{o}es}		& `one egg' \\
		\ve{en\tbr{o}ʔ} 	&+&\ve{=\tbr{e}es}&{\ra}&\ve{e\tbr{e}nʔ=\tbr{o}es}		& `one door' \\
		\ve{es\tbr{u}ʔ} 	&+&\ve{=\tbr{e}es}&{\ra}&\ve{e\tbr{e}sʔ=\tbr{u}es}		& `one mortar' \\
		\ve{ʔsun\tbr{u}ʔ} &+&\ve{=\tbr{e}es}&{\ra}&\ve{ʔsu\tbr{u}nʔ=\tbr{u}es}	& `one spoon' \\
		%\ve{\tbr{}ʔ} 	&+&\ve{=\tbr{e}es}&{\ra}&\ve{\tbr{}ʔ=\tbr{e}es}	& `one' \\
	\end{tabular}}
\end{exe}

Ro{\Q}is Amarasi (both from Buraen and Tunbaun)
shows a similar process of vowel assimilation
when a vowel initial enclitic attaches to a CVʔ{\#} word.
The process in Ro{\Q}is is different as the first
vowel of the clitic retains its height,
with the exception of \ve{=aa} in which complete assimilation takes place.
Examples are shown in \qf{ex:RoqVowAssGlot} below.\footnote{
		The Ro{\Q}is data for stems whose final vowel is a front vowel
		is somewhat ambiguous as I only have one example
		in which the first vowel of the enclitic has clearly assimilated:
		\ve{atoniʔ} + \ve{=aa} {\ra} \ve{atoonʔ=ia} `the man'.}

\begin{exe}
	\ex{Ro{\Q}is Amarasi vowel assimilation with /ʔ/}\label{ex:RoqVowAssGlot}
	\sn{\gw\begin{tabular}{rlllll}
		\ve{n-sen\tbr{u}ʔ} 		&+&\ve{=\tbr{e}e}&{\ra}&\ve{n-se\tbr{e}nʔ\tbr{o}e}	& `replaces it' \\
		\ve{na-knin\tbr{u}ʔ} 	&+&\ve{=\tbr{e}e}&{\ra}&\ve{na-kni\tbr{i}nʔ\tbr{o}e}& `cleans it' \\
		\ve{na-ser\tbr{o}ʔ} 	&+&\ve{=\tbr{e}e}&{\ra}&\ve{na-se\tbr{e}rʔ\tbr{o}e}	& `mixes it' \\
		\ve{un\tbr{u}ʔ} 			&+&\ve{=\tbr{i}i}&{\ra}&\ve{uunʔ\tbr{u}i}					& `long ago' \\
		\ve{mnan\tbr{u}ʔ} 		&+&\ve{=\tbr{i}i}&{\ra}&\ve{mna\tbr{a}nʔ\tbr{u}i}		& `the length' \\
		\ve{met\tbr{o}ʔ} 			&+&\ve{=\tbr{i}i}&{\ra}&\ve{me\tbr{e}tʔ\tbr{u}i}		& `the dry land' \\
		\ve{mor\tbr{o}ʔ} 			&+&\ve{=\tbr{i}i}&{\ra}&\ve{mo\tbr{o}rʔ\tbr{u}i}		& `the yellow one' \\
		\ve{b{\j}akas\tbr{e}ʔ}&+&\ve{=\tbr{e}es}&{\ra}&\ve{b{\j}aka\tbr{a}sʔ\tbr{e}es}		& `one horse' \\
		\ve{na-suk\tbr{i}ʔ}		&+&\ve{=\tbr{e}e}&{\ra}&\ve{na-su\tbr{u}kʔ\tbr{e}e}	& `supports it' \\
		\ve{aton\tbr{i}ʔ} 		&+&\ve{=\tbr{i}i}&{\ra}&\ve{ato\tbr{o}nʔ\tbr{i}i}		& `the man' \\
		\ve{aton\tbr{i}ʔ} 		&+&\ve{=\tbr{a}a}&{\ra}&\ve{ato\tbr{o}nʔ\tbr{i}a}		& `the man' \\
		%\ve{\tbr{}ʔ} 	&+&\ve{=ees}&{\ra}&\ve{\tbr{}ʔ=\tbr{e}es}	& `one' \\
	\end{tabular}}
\end{exe}


The data in which the first vowel of the enclitic
undergoes assimilation to the final vowel of the host
provides the crucial evidence which has swayed me to analyse
all vowel-initial enclitics as containing two vowels,
rather than a single vowel as I proposed in my PhD thesis \citep{ed16b}.
Under an analysis in which these enclitics contain a double vowel,
this process can be explained as a case of the first
vowel of the enclitic undergoing assimilation.
However, if such enclitics contained only a single vowel,
it is difficult to explain the presence of the additional vowel in these examples.

\section{Conclusion}
Metathesis before vowel-initial enclitics can be
analysed as phonologically conditioned.
When a vowel-initial enclitic is added to a stem
this triggers a number of phonological processes: metathesis,
consonant insertion, and vowel assimilation.

The first process is consonant insertion (\srf{sec:ConIns}).
Consonant insertion occurs because feet require an onset.
The next process is metathesis (\srf{sec:Met ch:PhoMet}).
Metathesis occurs before enclitics to create a crisp edge after an internal prosodic word.
Analysing metathesis as motivated by \tsc{Crisp Edge} is cucially depenent
on the analysis of intervocalic consonants as ambisyllabic (\srf{sec:Syl}).
The final process is vowel assimilation (\srf{sec:VowAss ch:PhoMet}),
under which any vowel which conditioned insertion of a consonant assimilates.
This occurs because after metathesis any such vowel shares
features with the inserted consonant across another C-slot.

In one environment Amarasi metathesis is phonologically conditioned.
It occurs to create a phonological boundary between two prosodic words.
However, as discussed in Chapter \ref{ch:SynchMet},
just because \emph{some} instances of metathesis in a language
are phonologically conditioned, does not mean \emph{all}
instances of metathesis in that language are phonologically conditioned.
In addition to phonologically conditioned metathesis,
Amarasi also has instances of metathesis which
cannot be accounted for by reference to phonology alone.
Amarasi has two kinds of morphological metathesis:
metathesis marking syntactic structures (Chapter \ref{ch:SynMet})
and metathesis marking discourse structures (Chapter \ref{ch:DisMet}).
