\subsection{Reduplication}\label{sec:Red}
Reduplication provides support for
the CVC syllable and CVCVC foot as distinct domains of Amarasi word structure.
It also provides support for ambisyllabic intervocalic consonants
as such consonants are copied in reduplication.
Amarasi has two kinds of reduplication: full reduplication and partial reduplication.
In full reduplication the entire word is copied.
Examples include \ve{reko} `good' {\ra} \ve{reko{\tl}reko} `properly',
and \ve{neno} `day' {\ra} \ve{neno{\tl}neno} `every day'.

In partial reduplication the initial
syllable of the final foot is copied and prefixed to this final foot.
That the reduplicant is CVC is evidence for identifying a CVC syllable
with the intervocalic consonant as ambisyllabic.
For roots which consist of a single foot,
the reduplicant is simply placed to the left of the stem.
Examples are given in \qf{ex:ParRed} below.

\begin{exe}
	\ex{Partial reduplication:}\label{ex:ParRed}
		\sn{\gw\begin{tabular}{llll}
			\ve{baʔuk}	&{\ra}& \ve{\tbr{baʔ}{\tl}baʔuk} 	& `many' \\
			\ve{reko}		&{\ra}& \ve{\tbr{rek}{\tl}reko}		& `good' \\
			\ve{koʔu}		&{\ra}& \ve{\tbr{koʔ}{\tl}koʔu} 		& `big' \\
%			\ve{n-nenuk}&{\ra}& \ve{n-\tbr{nen}{\tl}nenuk} & `(go for a) walk' \\
			\ve{n-mate}	&{\ra}& \ve{n-\tbr{mat}{\tl}mate} 	& `die' \\
			\ve{n-nao}	&{\ra}& \ve{n-\tbr{na}{\tl}nao}		& `go' \\
			\ve{n-tae}	&{\ra}& \ve{n-\tbr{ta}{\tl}tae}		& `look down' \\
			\ve{okeʔ}		&{\ra}& \ve{\tbr{ok}{\tl}okeʔ}			& `all' \\
			\ve{anaʔ}		&{\ra}& \ve{\tbr{an}{\tl}anaʔ}			& `small' \\
		\end{tabular}}
\end{exe}

In the case of phonemically vowel-initial roots
which begin with a predictable glottal stop (\srf{sec:GloStoIns}),
this epenthetic glottal stop is the onset of both the reduplicant and following foot.
Two examples are \ve{ok{\tl}okeʔ} `all' {\ra}
[\tbr{ʔ}ɔkˈ\tbr{ʔ}ɔkɛʔ] {\emb{ok-okeq.mp3}{\spk{}}{\apl}}
and \ve{an{\tl}anaʔ} `small' {\ra} [\tbr{ʔ}anˈ\tbr{ʔ}anɐʔ] {\emb{an-anaq.mp3}{\spk{}}{\apl}}.

When the medial C-slot of the foot is empty,
the final C-slot of the reduplicant is filled by
the final consonant of the foot.
Examples are given in \qf{ex:ParRedEmpMedCSlo} below.

\newpage
\begin{exe}
	\ex{Partial reduplication with empty medial C-slots:}\label{ex:ParRedEmpMedCSlo}
		\sn{\gw\begin{tabular}{llll}
			\ve{fauk} 	&{\ra}& \ve{\tbr{fak}{\tl}fauk} 		& `several' \\
			\ve{buaʔ} 	&{\ra}& \ve{\tbr{buʔ}{\tl}buaʔ} 	& `together' \\
			\ve{na-tuin}&{\ra}& \ve{na-\tbr{tun}{\tl}tuin} & `follows; because of' \\
			\ve{kais}		&{\ra}& \ve{\tbr{kas}{\tl}kais} 		& `don't, \tsc{prohibitive}' \\
			\ve{na-ʔuab}		&{\ra}& \ve{na-\tbr{ʔub}{\tl}ʔuab} 		& `speaks' \\
%			\ve{mfaun} &{\ra}& \ve{m\tbr{fa}{\tl}faun} 		& `many' \\
%			\ve{ʔnaef} &{\ra}& \ve{ʔ\tbr{na}{\tl}naef} 	& `old man' \\
		\end{tabular}}
\end{exe}

Suffixes or enclitics attached to a stem do not appear in the reduplicant in partial reduplication.
Two examples include \ve{n-poi=n} `{\n}-exit={\einV}' {\ra} \ve{n-po{\tl}poi=n},
and \ve{na-breo=n} `{\na}-grope.around={\einV}' {\ra} \ve{na-bre{\tl}reo=n}.

There are two CCVVC{\#} roots in my corpus in which the final consonant does
not appear in the reduplicant:
\ve{ʔnaef} `old man' {\ra} \ve{ʔ\tbr{na}{\tl}naef}
and \ve{mfaun} `many' {\ra} \ve{m\tbr{fa}{\tl}faun}.
In both cases the final consonant is probably frozen morphology:
the plural enclitic  \ve{=n} (\srf{sec:PluEnc}) for \ve{mfaun} `many'
and the genitive suffix \ve{-f} (\srf{sec:GenSuf}) for \ve{ʔnaef} `old man'.
Cognates of \ve{ʔnaef} `old man' include
\ve{na-ʔnae} `grow' and the poetic word \ve{ʔnaek} `great, big'.

Reduplication provides evidence for identifying
the foot as a distinct unit of phonological structure
as for roots which are larger than a single foot
the CVC reduplicant is placed after the
pre-foot material and prefixed to the foot,
thus as a kind of infix.
Examples are given in \qf{ex:ParRedConClu} below.

\begin{exe}
	\ex{Partial reduplication with pre-foot material:}\label{ex:ParRedConClu}
		\sn{\gw\begin{tabular}{llll}
			\ve{ʔroo} 			&{\ra}& \ve{ʔ\tbr{ro}{\tl}roo} 			& `far, distant' \\
%			\ve{na-ʔnenuʔ} 	&{\ra}& \ve{na-ʔ\tbr{nen}{\tl}nenuʔ} & `turn' \\
			\ve{na-kberoʔ} 	&{\ra}& \ve{na-k\tbr{ber}{\tl}beroʔ} & `move' \\
			\ve{na-msena} 	&{\ra}& \ve{na-m\tbr{sen}{\tl}sena}	& `full, satiated' \\
			\ve{na-thoe} 		&{\ra}& \ve{na-t\tbr{ho}{\tl}hoe}		& `inundate, bless' \\
			\ve{maʔfenaʔ} 	&{\ra}& \ve{maʔ\tbr{fen}{\tl}fenaʔ}	& `heavy' \\
			\ve{taikobi} 		&{\ra}& \ve{tai\tbr{kob}{\tl}kobi}		& `fall down' \\
			\ve{paumakaʔ} 	&{\ra}& \ve{pau\tbr{mak}{\tl}makaʔ}	& `near' \\
		\end{tabular}}
\end{exe}

The prosodic and morphological structures of \ve{maʔfenaʔ} `heavy'
and its reduplicated counterpart \ve{maʔfen{\tl}fenaʔ} `very heavy'
are given in \qf{as:maqfenaq} below.
Example \qf{as:maqfenfenaq} shows the CVC reduplicant (\ve{fen})
occurs as prefix to the final foot within the prosodic structure
and thus as an in infix within the morphological structure.

\begin{multicols}{2}
	\begin{exe}
		\ex{\label{as:maqfenaq}\begin{xlist}
			\exa{\xy
				<4.5em,6cm>*\as{PrWd}="PrWd",<6em,5cm>*\as{Ft}="ft1",
				<2em,4cm>*\as{σ}="s1",<5em,4cm>*\as{σ}="s2",<7em,4cm>*\as{σ}="s3",
				<1em,3cm>*\as{C}="CV1",<2em,3cm>*\as{V}="CV2",<3em,3cm>*\as{C}="CV3",
				<4em,3cm>*\as{C}="CV4",<5em,3cm>*\as{V}="CV5",<6em,3cm>*\as{C}="CV6",<7em,3cm>*\as{V}="CV7",<8em,3cm>*\as{C}="CV8",
				<1em,2cm>*\as{m}="cv1",<2em,2cm>*\as{a}="cv2",<3em,2cm>*\as{ʔ}="cv3",
				<4em,2cm>*\as{f}="cv4",<5em,2cm>*\as{e}="cv5",<6em,2cm>*\as{n}="cv6",<7em,2cm>*\as{a}="cv7",<8em,2cm>*\as{ʔ}="cv8",
				<4.5em,1cm>*\as{M}="m1",<6em,0cm>*\as{}="m2",
				"m1"+U;"cv1"+D**\dir{-};"m1"+U;"cv2"+D**\dir{-};"m1"+U;"cv3"+D**\dir{-};"m1"+U;"cv4"+D**\dir{-};
				"m1"+U;"cv5"+D**\dir{-};"m1"+U;"cv6"+D**\dir{-};"m1"+U;"cv7"+D**\dir{-};"m1"+U;"cv8"+D**\dir{-};
				"cv1"+U;"CV1"+D**\dir{-};"cv2"+U;"CV2"+D**\dir{-};"cv3"+U;"CV3"+D**\dir{-};
				"cv4"+U;"CV4"+D**\dir{-};"cv5"+U;"CV5"+D**\dir{-};"cv6"+U;"CV6"+D**\dir{-};"cv7"+U;"CV7"+D**\dir{-};"cv8"+U;"CV8"+D**\dir{-};
				"CV1"+U;"s1"+D**\dir{-};"CV2"+U;"s1"+D**\dir{-};"CV3"+U;"s1"+D**\dir{-};
				"CV4"+U;"s2"+D**\dir{-};"CV5"+U;"s2"+D**\dir{-};"CV6"+U;"s2"+D**\dir{-};
				"CV6"+U;"s3"+D**\dir{-};"CV7"+U;"s3"+D**\dir{-};"CV8"+U;"s3"+D**\dir{-};
				"s1"+U;"PrWd"+D**\dir{-};"s2"+U;"ft1"+D**\dir{-};"s3"+U;"ft1"+D**\dir{-};
				"ft1"+U;"PrWd"+D**\dir{-};
			\endxy}
			\exa{\label{as:maqfenfenaq}\xy
				<6em,6cm>*\as{PrWd}="PrWd",<9em,5cm>*\as{Ft}="ft1",
				<2em,4cm>*\as{σ}="s1",<5em,4cm>*\as{σ}="s2",<8em,4cm>*\as{σ}="s3",<10em,4cm>*\as{σ}="s4",
				<1em,3cm>*\as{C}="CV1",<2em,3cm>*\as{V}="CV2",<3em,3cm>*\as{C}="CV3",<4em,3cm>*\as{C}="CV4",<5em,3cm>*\as{V}="CV5",<6em,3cm>*\as{C}="CV6",
				<7em,3cm>*\as{C}="CV7",<8em,3cm>*\as{V}="CV8",<9em,3cm>*\as{C}="CV9",<10em,3cm>*\as{V}="CV10",<11em,3cm>*\as{C}="CV11",
				<1em,2cm>*\as{m}="cv1",<2em,2cm>*\as{a}="cv2",<3em,2cm>*\as{ʔ}="cv3",<4em,2cm>*\as{f}="cv4",<5em,2cm>*\as{e}="cv5",<6em,2cm>*\as{n}="cv6",
				<7em,2cm>*\as{f}="cv7",<8em,2cm>*\as{e}="cv8",<9em,2cm>*\as{n}="cv9",<10em,2cm>*\as{a}="cv10",<11em,2cm>*\as{ʔ}="cv11",
				<6em,0cm>*\as{M}="m1",<5.5em,1cm>*\as{M}="m2",
				"m1"+U;"cv1"+D**\dir{-};"m1"+U;"cv2"+D**\dir{-};"m1"+U;"cv3"+D**\dir{-};"m1"+U;"cv7"+D**\dir{-};
				"m1"+U;"cv8"+D**\dir{-};"m1"+U;"cv9"+D**\dir{-};"m1"+U;"cv10"+D**\dir{-};"m1"+U;"cv11"+D**\dir{-};
				"m2"+U;"cv4"+D**\dir{-};"m2"+U;"cv5"+D**\dir{-};"m2"+U;"cv6"+D**\dir{-};
				"cv1"+U;"CV1"+D**\dir{-};"cv2"+U;"CV2"+D**\dir{-};"cv3"+U;"CV3"+D**\dir{-};"cv4"+U;"CV4"+D**\dir{-};"cv5"+U;"CV5"+D**\dir{-};"cv6"+U;"CV6"+D**\dir{-};
				"cv7"+U;"CV7"+D**\dir{-};"cv8"+U;"CV8"+D**\dir{-};"cv9"+U;"CV9"+D**\dir{-};"cv10"+U;"CV10"+D**\dir{-};"cv11"+U;"CV11"+D**\dir{-};
				"CV1"+U;"s1"+D**\dir{-};"CV2"+U;"s1"+D**\dir{-};"CV3"+U;"s1"+D**\dir{-};
				"CV4"+U;"s2"+D**\dir{-};"CV5"+U;"s2"+D**\dir{-};"CV6"+U;"s2"+D**\dir{-};
				"CV7"+U;"s3"+D**\dir{-};"CV8"+U;"s3"+D**\dir{-};"CV9"+U;"s3"+D**\dir{-};
				"CV9"+U;"s4"+D**\dir{-};"CV10"+U;"s4"+D**\dir{-};"CV11"+U;"s4"+D**\dir{-};
				"s1"+U;"PrWd"+D**\dir{-};"s2"+U;"PrWd"+D**\dir{-};"s3"+U;"ft1"+D**\dir{-};"s4"+U;"ft1"+D**\dir{-};"ft1"+U;"PrWd"+D**\dir{-};
			\endxy}
		\end{xlist}}
	\end{exe}
\end{multicols}
 
%That the reduplicant in partial reduplication
%occurs between the foot and any pre-foot material
%provides evidence that the foot constitutes a
%distinct domain of Amarasi word structure.
%That the reduplicant consists of CVC provides
%support for analysing the Amarasi syllable as having this structure.
%%It also provides some support for analysing the medial C-slot of the foot as ambisyllabic.
%%Additional evidence for the medial C-slot of the foot being ambisyllabic comes
%%from metathesis before vowel-initial enclitics as discussed in Chapter \ref{ch:PhoMet}.