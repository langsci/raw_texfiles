\section{Consonant deletion}\label{sec:ConDel ch:Phon}
Kotos Amarasi does not allow word-final consonant clusters.
Thus, the addition of a suffix consisting of a single consonant
to a consonant-final stem is not straightforward.
In most cases any stem-final consonant is deleted when
a mono-consonantal suffix is added.
Examples are given in \qf{ex:GloStoRepGenSuf} below
which shows a number of consonant-final stems with
a genitive suffix (\srf{sec:GenSuf}).
In all cases the final consonant of the stem
is replaced by the genitive suffix, including
when that final consonant is itself a suffix.\footnote{
		Apart from stems with another suffix attached,
		the only final consonant so far attested
		with roots which take genitive suffixes is the glottal stop.
		Historically, this glottal stop is probably itself a suffix \citep[77]{ed18d},
		though this analysis does not seem a possible synchronic analysis.}

\newpage
\begin{exe}
	\ex{Final consonant replacement after genitive suffix}\label{ex:GloStoRepGenSuf}
	\sn{\gw\stl{0.5em}\begin{tabular}{rlllll}
		\ve{enoʔ}	&+&\ve{-n}&{\ra}& \ve{iin eno-n} 	& `its door'\\
		\ve{retaʔ}	&+&\ve{-n}&{\ra}& \ve{iin reta-n} 	& `her/his story'\\
		\ve{humaʔ}	&+&\ve{-k}&{\ra}& \ve{au huma-k} 	& `my face'\\
		\ve{ʔnakaʔ} &+&\ve{-k}&{\ra}& \ve{au ʔnaka-k} & `my head'\\
		\ve{a-m-nema-t}	&+&\ve{-n}&{\ra}& \ve{iin a-m-nema-n} & `her/his arrival, origins'\\
		\ve{a-reko-t}	&+&\ve{-n}&{\ra}& \ve{iin a-reko-n} & `her/his goodness'\\
		\end{tabular}}
\end{exe}

When the people group suffix \ve{-s}
attaches to CVC{\#} final stems this suffix replaces the final consonant.
However, after VVC{\#} final stems this suffix has the allomorph \ve{-as}.
Examples of \ve{-s} are given in \qf{ex:PeoGroSuf1} below.

\begin{exe}
	\ex{People group suffix \ve{-s}}\label{ex:PeoGroSuf1}
	\sn{\gw\stl{0.275em}\begin{tabular}{rlclcll}
		`Savu'				&\ve{Sapu}			&+&\ve{-s} &\ra&\ve{Sapu-s} 		& `person from Savu'\\
		`Rote'				&\ve{Rote}			&+&\ve{-s} &\ra&\ve{Rote-s} 		& `person from Rote'\\
		`Koro{\Q}oto'	&\ve{Koorʔoto}	&+&\ve{-s} &\ra&\ve{Koorʔoto-s}	& `person from Koro{\Q}oto'\\
		`Belu'				&\ve{Beru}			&+&\ve{-s} &\ra&\ve{Beru-s}			& `person from Belu'\\
		`Kupang'			&\ve{Kopan} 		&+&\ve{-s} &\ra&\ve{Kopa-s}			& `person from Kupang'\\
		`Helong'			&\ve{ʔHeroʔ}		&+&\ve{-s} &\ra&\ve{ʔHero-s}		& `Helong person'\\
		`Buraen'			&\ve{Buraen} 		&+&\ve{-as}&\ra&\ve{Buraen-as}	& `person from Buraen'\\
		`Naet'				&\ve{Naet} 			&+&\ve{-as}&\ra&\ve{Naet-as}		& `person from Naet'\\
		`east'				&\ve{neon sae-t}&+&\ve{-as}&\ra&\ve{neon sae-t-as} & `easterner'\footnotemark\\
		\end{tabular}}
\end{exe}
\footnotetext{ The form \ve{neon sae-t-as} `easterner' specifically
		refers to someone from the north-eastern Atoni (Meto speaking)
		regions: Oecusse (Baikeno), Miomafo, Insana, and Biboki.}

These different morphophonemic processes apply with the aim
of fitting the derived word into the
canonical disyllabic foot shape (\srf{sec:TheFoo}).
Thus, for \ve{Kopan} `Kupang' {\ra} \ve{Kopa-s}
`person from Kupang', deletion of the root final consonant
means that the derived word, \ve{Kopa-s} is a disyllabic foot.
The prosodic and morphological structures of structures
of \ve{Kopan} `Kupang' and \ve{Kopa-s} `person from Kupang'
are shown in \qf{as:Kopan->Kopas} below.

\newpage
\begin{multicols}{2}
	\begin{exe}\ex{\label{as:Kopan->Kopas}
		\begin{xlist}
			\exa{\label{as:Kopan}\xy
				<3em,4cm>*\as{Ft}="ft",
				<2em,3cm>*\as{σ}="s1",<4em,3cm>*\as{σ}="s2",
				<1em,2cm>*\as{C}="CV1",<2em,2cm>*\as{V}="CV2",<3em,2cm>*\as{C}="CV3",<4em,2cm>*\as{V}="CV4",<5em,2cm>*\as{C}="CV5",
				<1em,1cm>*\as{k}="cv1",<2em,1cm>*\as{o}="cv2",<3em,1cm>*\as{p}="cv3",<4em,1cm>*\as{a}="cv4",<5em,1cm>*\as{n}="cv5",
				<3em,0cm>*\as{M}="m1",
				"m1"+U;"cv1"+D**\dir{-};"m1"+U;"cv2"+D**\dir{-};"m1"+U;"cv3"+D**\dir{-};"m1"+U;"cv4"+D**\dir{-};"m1"+U;"cv5"+D**\dir{-};
				"cv1"+U;"CV1"+D**\dir{-};"cv2"+U;"CV2"+D**\dir{-};"cv3"+U;"CV3"+D**\dir{-};
				"cv4"+U;"CV4"+D**\dir{-};"cv5"+U;"CV5"+D**\dir{-};
				"CV1"+U;"s1"+D**\dir{-};"CV2"+U;"s1"+D**\dir{-};"CV3"+U;"s1"+D**\dir{-};
				"CV3"+U;"s2"+D**\dir{-};"CV4"+U;"s2"+D**\dir{-};"CV5"+U;"s2"+D**\dir{-};
				"s1"+U;"ft"+D**\dir{-};"s2"+U;"ft"+D**\dir{-};
			\endxy}
			\exa{\label{as:Kopas}\xy
				<3em,4cm>*\as{Ft}="ft",
				<2em,3cm>*\as{σ}="s1",<4em,3cm>*\as{σ}="s2",
				<1em,2cm>*\as{C}="CV1",<2em,2cm>*\as{V}="CV2",<3em,2cm>*\as{C}="CV3",<4em,2cm>*\as{V}="CV4",<5em,2cm>*\as{C}="CV5",
				<1em,1cm>*\as{k}="cv1",<2em,1cm>*\as{o}="cv2",<3em,1cm>*\as{p}="cv3",<4em,1cm>*\as{a}="cv4",<5em,1cm>*\as{s}="cv5",
				<2.5em,0cm>*\as{M}="m1",<5em,0cm>*\as{M}="m2",
				"m1"+U;"cv1"+D**\dir{-};"m1"+U;"cv2"+D**\dir{-};"m1"+U;"cv3"+D**\dir{-};"m1"+U;"cv4"+D**\dir{-};"m2"+U;"cv5"+D**\dir{-};
				"cv1"+U;"CV1"+D**\dir{-};"cv2"+U;"CV2"+D**\dir{-};"cv3"+U;"CV3"+D**\dir{-};
				"cv4"+U;"CV4"+D**\dir{-};"cv5"+U;"CV5"+D**\dir{-};
				"CV1"+U;"s1"+D**\dir{-};"CV2"+U;"s1"+D**\dir{-};"CV3"+U;"s1"+D**\dir{-};
				"CV3"+U;"s2"+D**\dir{-};"CV4"+U;"s2"+D**\dir{-};"CV5"+U;"s2"+D**\dir{-};
				"s1"+U;"ft"+D**\dir{-};"s2"+U;"ft"+D**\dir{-};
			\endxy}
		\end{xlist}}
	\end{exe}
\end{multicols}

For VVC{\#} final words use of the allomorph \ve{-as} does not
result in an increase in word size
as both the penultimate and final vowel of the stem
can be assigned to a single V-slot (\srf{sec:SurVVCVWor}).
The prosodic and morphological structures of \ve{Naet} {\ra} \ve{Naet-as}
`person from Naet' are shown in \qf{as:Naet->Naetas} below to illustrate.\footnote{
		Naet is one of the four hamlets which was unified to
		form the village of Nekmese{\Q} (\srf{sec:Amarasi}).}
Both words are disyllabic feet, with \ve{Naet} having an empty
medial C-slot, as seen in \qf{as:Naet},
while \ve{Naet-as} has both its vowels
assigned to a single V-slot, as seen in \qf{as:Naetas}.

\begin{multicols}{2}
	\begin{exe}\ex{\label{as:Naet->Naetas}
		\begin{xlist}
			\exa{\label{as:Naet}\xy
				<3em,4cm>*\as{Ft}="ft",
				<2em,3cm>*\as{σ}="s1",<4em,3cm>*\as{σ}="s2",
				<1em,2cm>*\as{C}="CV1",<2em,2cm>*\as{V}="CV2",<3em,2cm>*\as{C}="CV3",<4em,2cm>*\as{V}="CV4",<5em,2cm>*\as{C}="CV5",
				<1em,1cm>*\as{n}="cv1",<2em,1cm>*\as{a}="cv2",<3em,1cm>*\as{}="cv3",<4em,1cm>*\as{e}="cv4",<5em,1cm>*\as{t}="cv5",
				<3em,0cm>*\as{M}="m1",
				"m1"+U;"cv1"+D**\dir{-};"m1"+U;"cv2"+D**\dir{-};"m1"+U;"cv4"+D**\dir{-};"m1"+U;"cv5"+D**\dir{-};
				"cv1"+U;"CV1"+D**\dir{-};"cv2"+U;"CV2"+D**\dir{-};
				"cv4"+U;"CV4"+D**\dir{-};"cv5"+U;"CV5"+D**\dir{-};
				"CV1"+U;"s1"+D**\dir{-};"CV2"+U;"s1"+D**\dir{-};"CV3"+U;"s1"+D**\dir{-};
				"CV3"+U;"s2"+D**\dir{-};"CV4"+U;"s2"+D**\dir{-};"CV5"+U;"s2"+D**\dir{-};
				"s1"+U;"ft"+D**\dir{-};"s2"+U;"ft"+D**\dir{-};
			\endxy}
			\exa{\label{as:Naetas}\xy
				<3em,4cm>*\as{Ft}="ft",
				<2em,3cm>*\as{σ}="s1",<4em,3cm>*\as{σ}="s2",
				<1em,2cm>*\as{C}="CV1",<2em,2cm>*\as{V}="CV2",<3em,2cm>*\as{C}="CV3",<4em,2cm>*\as{V}="CV4",<5em,2cm>*\as{C}="CV5",
				<1em,1cm>*\as{n}="cv1",<1.7em,1cm>*\as{a}="cv2",<2.3em,1cm>*\as{e}="cv2.5",<3em,1cm>*\as{t}="cv3",<4em,1cm>*\as{a}="cv4",<5em,1cm>*\as{s}="cv5",
				<2em,0cm>*\as{M}="m1",<4.5em,0cm>*\as{M}="m2",
				"m1"+U;"cv1"+D**\dir{-};"m1"+U;"cv2"+D**\dir{-};"m1"+U;"cv2.5"+D**\dir{-};"m1"+U;"cv3"+D**\dir{-};
				"m2"+U;"cv4"+D**\dir{-};"m2"+U;"cv5"+D**\dir{-};
				"cv1"+U;"CV1"+D**\dir{-};"cv2"+U;"CV2"+D**\dir{-};"cv2.5"+U;"CV2"+D**\dir{-};
				"cv3"+U;"CV3"+D**\dir{-};"cv4"+U;"CV4"+D**\dir{-};"cv5"+U;"CV5"+D**\dir{-};
				"CV1"+U;"s1"+D**\dir{-};"CV2"+U;"s1"+D**\dir{-};"CV3"+U;"s1"+D**\dir{-};
				"CV3"+U;"s2"+D**\dir{-};"CV4"+U;"s2"+D**\dir{-};"CV5"+U;"s2"+D**\dir{-};
				"s1"+U;"ft"+D**\dir{-};"s2"+U;"ft"+D**\dir{-};
			\endxy}
		\end{xlist}}
	\end{exe}
\end{multicols}

\subsection{Consonant coalescence}\label{sec:TriCluConDel}
When one of the consonant-final pronouns \ve{iin}
`{\iin}, s/he, it', \ve{siin} `{\siin}, they' or \ve{hiit} `{\hiit}, we'
occurs before a corresponding consonantal
agreement prefix \ve{n-} `\tsc{3sg, 3pl}' or \ve{t-} `\tsc{\t}'
which is in turn attached to a consonant-initial stem,
the final consonant of the pronoun
and the agreement prefix usually coalesce.

In this situation the underlying initial sequence
of two identical consonants is usually degeminated:
giving \ve{nn}C {\ra} [nC] and \ve{tt}C {\ra} [t̪C].
This process is summarised in \qf{ex:ProPreDeg}
with examples of each given in \qf{ex:ProPreDeg2}.

\begin{exe}
	\ex{\begin{xlist}
	\ex{\rule{0pt}{0pt}{} \\[-2.3ex] 
		\begin{tabular}{l@{ }l@{ }l@{ }l@{ }l@{ }l@{ }l}
			\ve{iin} &+& \ve{n}-C & {\ra} & \ve{iin n}C	& {\ra} & [inC] \\
			\ve{siin} &+& \ve{n}-C & {\ra} & \ve{siin n}C	& {\ra} & [sinC] \\
			\ve{hiit} 	&+& \ve{t}-C & {\ra} & \ve{hiit t}C		& {\ra} & [hit̪C] \\
		\end{tabular}\label{ex:ProPreDeg}}
	\ex{\rule{0pt}{0pt}{} \\[-2.3ex] 
		\begin{tabular}{l@{ }l@{ }l@{ }l@{ }l@{ }l@{ }l@{ }l@{ }l} 
			\ve{ii\tbr{n}} &+& \ve{\tbr{n}-muʔi} &\ra& \ve{ii\tbr{n n}muiʔ} &\ra& [ʔɪ\tbr{n}ˈmʊiʔ]
					&{\emb{in-nmuiq.mp3}{\spk{}}{\apl}}& `s/he has' \\
					{\iin} && \n-have &&&&&& \\
			\ve{hii\tbr{t}} &+& \ve{\tbr{t}-mese} &\ra& \ve{hii\tbr{t t}mese}  &\ra& [hɪ\tbr{t̪}ˈmɛsɛ]
					&{\emb{hit-tmese.mp3}{\spk{}}{\apl}}& `we are alone'\\
					{\hiit} && \mc{3}{l}{\t-alone}&&&& \\
		\end{tabular}\label{ex:ProPreDeg2}}
	\end{xlist}}
\end{exe}

Evidence that both consonants \emph{do} survive underlyingly
comes from the fact that the U\=/form of these pronouns
has been attested in this environment.
(See \srf{sec:IrrMfor} for a discussion of functor U\=/forms.)
In my corpus there are 2/45 instances of \ve{sina} `{\siin}' /{\gap}n-C.
Both are examples of the phrase \ve{si\tbr{na} \tbr{n}-\tbr{m}ate=n}.
One of these examples is given in \qf{ex:130825-6, 7-34} below.

\begin{exe}\let\eachwordone=\textnormal \let\eachwordtwo=\itshape
	\ex{\glll	[ʔak \hp{``}ɛj ɔˑ kaːtʊ sɪ\tbr{nə} ˈ\tbr{nm}aːt̪ɛn baj kʊːːs]\\
						\hp{[}ʔ-ak: ``hei hoo kartu si\tbr{na} \tbr{n}-\tbr{m}ate=n, baʔi Kus.\\
						\hp{[}\q-say \hp{``}hey {\hoo} card {\siin} \n-die={\einV} PF Kus\\
			\glt	\lh{[}`I said: ``hey, your cards have expired Kus.' \txrf{130825-6, 7-34}
						{\emb{130825-6-07-34.mp3}{\spk{}}{\apl}}}\label{ex:130825-6, 7-34}
\end{exe}

There are no examples of the U\=/form of the pronoun \ve{iin} `{\iin}'
or \ve{hiit} `{\hiit}' before an agreement prefix in my database,
though \citet[135]{st93} gives examples of the U\=/form of both
these pronouns before their corresponding agreement prefixes in Miomafo.
