\section{Tail-Head linkage}\label{sec:TaiHeaLin}
Another use of U\=/forms in Amarasi discourse is in tail-head linkage.
Tail-head linkage is a repetition structure for slowing down the rate of new information,
``in which the last sentence of one paragraph cross-references to the first sentence 
of the following paragraph'' \citep[9]{lo83}.
Tail-head linkage can also link clauses in sentences.
A simple example of tail-head linkage in English is given in \qf{ex:TaiHeEng} below.

\begin{exe}
	\ex{\begin{xlist}
%		\ex{\gll	2j \tbr{@r2jvd} h@wm\\
%							I \tbr{arrived} home\\}
%		\ex{\gll	wEn 2j \tbr{@r2jvd} 2j wEnt strejt t@ ð@ frI{\j}\\
%							When I \tbr{arrived} I went straight to the fridge \\}
		\ex{\it{I \tbr{arrived} home.}}
		\ex{\it{When I \tbr{arrived}, I went straight to the fridge.}}
	\end{xlist}}\label{ex:TaiHeEng}
\end{exe}

Tail-head linkage in Amarasi typically consists of repetition of a single
verb, with the second instance of the verb introducing an event subsequent
to the event encoded by both verbs,
or introducing extra information about the way in which that event occurred.
One of the repeated verbs is in the U\=/form and the other
repeated verb is in the M\=/form.
The new event introduced resolves the U\=/form half of the tail-head linkage construction.

Tail-head linkage in Amarasi is a kind of dependent
coordination (\srf{sec:DepCoo}) structure with repetition of the first event.
The two typical structures of tail-head linkage in Amarasi
are given in \qf{ex:THL} and \qf{ex:THL2} below.
The first instance of the word encoding event\sub{1} is the tail
and the second instance of this word is the head.

\begin{exe}
	\ex{\xytext{\fbox{event\sub{1}{\M}}\xybarconnect[2][-](D,D){1}&\fbox{event\sub{1}{\U}}\xybarconnect[2][->]{2}&(\ve{=ma}/\ve{=te})&\fbox{event\sub{2}\vp{\U}}}}\label{ex:THL}
	\ex{\xytext{\fbox{event\sub{1}{\U}}\xybarconnect[2][-](D,D){2}\xybarconnect[2][->]{3}&(\ve{=ma}/\ve{=te})&\fbox{event\sub{1}{\M}}&\fbox{event\sub{2}\vp{\U}}}}\label{ex:THL2}
\end{exe}

Except in highly restricted examples it is not
usually grammatical for both verbs encoding the first
event to take the same form of metathesis.\footnote{
		When both verbs are followed by a vowel-initial enclitic, both may be in the M\=/form.}
If the tail is in the U\=/form, the head must be in the \mbox{M\=/form}.
If the tail is in the M\=/form, the head must be in the U\=/form.
The tail and head form a complementary and mutually dependent pair.

There are 72 instances of tail-head linkage with a U\=/form in my corpus
-- 17{\%} (72/423) of all discourse-driven U\=/forms in my corpus.
The M\=/form half of a tail-head
linkage construction is often in the M\=/form due to
a vowel-initial enclitic (Chapters \ref{ch:PhoMet} and \srf{sec:PhoCon})
or as the first member of a serial verb construction (\srf{sec:SVC}, \srf{sec:SynDisDriMet}).