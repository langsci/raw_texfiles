\section{Phonotactic constraints}\label{sec:PhoCon}
There are several phonotactic/phonological environments
which block the use of U\=/forms to signal lack of resolution.
The end result of these different phonotactic constraints
is that only non-nominals which end in CV{\#}
have morphological uses of U\=/forms, and then only
when not followed by a vowel-initial enclitic or
word with an initial consonant cluster.

Firstly, and most obviously, stems which end a vowel
sequence do not have (surface) alternate
U\=/forms and M\=/forms (\srf{sec:NoCha}).
Thus, such stems cannot use U\=/forms to
morphologically signal lack of resolution.

Other situations in which the phonology overrides or blocks
the morphology with the result that U\=/forms do not mark lack of resolution
include when vowel-initial enclitics occur attached to a stem (\srf{sec:PhoCon sec:VowIniEnc}),
when a stem is consonant final (\srf{sec:ConFinVer}),
and when a stem is followed by a consonant cluster (\srf{sec:VerBefCC}).
Each of these different environments is discussed in turn in the following sections.

\subsection{Vowel-initial enclitics}\label{sec:PhoCon sec:VowIniEnc}
As discussed in Chapter \ref{ch:PhoMet},
M\=/forms are obligatory before vowel-initial enclitics.
When a non-nominal occurs with a vowel-initial enclitic,
it obligatorily occurs in the M\=/form and it is
not possible to mark lack of resolution with the clitic host.
However, if the vowel-initial enclitic itself has alternate
U\=/forms and M\=/forms, lack of resolution of the clause can
be signalled by this vowel-initial enclitic taking the U\=/form.
One example is given in \qf{ex:HooMfaanJena} below
-- a typical question asked around the village.

\begin{exe}
	\ex{\glll	hoo {m-faan\j=e\tbr{na} ?}\\
						hoo m-fani=ena \\
						{\hoo} \m-return{\Mv=\een\tbrU}\\
			\glt	`So, you're going back now?'
						\txrf{observation}}\label{ex:HooMfaanJena}
\end{exe}

In \qf{ex:HooMfaanJena} the clitic host \ve{m-faan\j} `return'
occurs in the M\=/form due to the following vowel-initial enclitic.
As a result, it is not possible to mark
that this is a question with this stem
(see \srf{sec:IntUnm} for full discussion of
the use of U\=/forms to mark questions).
Instead, the vowel-initial enclitic occurs in the U\=/form
to signal that this clause is a question.

If the vowel-initial enclitic itself does not have metathesis
alternations, i.e. if the vowel-initial enclitic ends in a vowel sequence,
it is not possible to morphologically mark lack of resolution.

\subsection{Consonant-final U\=/forms}\label{sec:ConFinVer}
Non-nominal stems with a final consonant occur by default in the U\=/form
and do not use U\=/forms to morphologically signal lack of resolution.
Consonant-final non-nominals only occur in the M\=/form
before vowel-initial enclitics.
Consonant-final U\=/forms are glossed {\Uc}
-- U with a \emph{c} for consonant above it --
to mark that these are not discourse-driven U\=/forms.
Two examples of simple declarative sentences with
consonant-final verbs occurring in the U\=/form
are given in \qf{ex2:120715-4, 0.05} and \qf{ex:130823-2, 0.57} below.

\begin{exe}
	\ex{\gll	neno naa paha{\gap}ʔpina-n ia, \hp{3-pile.up{\Uc} {\on} place} a|n-ko\tbr{bub} on bare meseʔ\\
						day {\naa} country{\gap}below-{\N} {\ia} {} {\a\n}-pile.up{\tbrUc} {\on} place one\\
			\glt	`In those days the world was piled up in one place.'
						\txrf{120715-4, 0.05} {\emb{120715-4-00-05.mp3}{\spk{}}{\apl}}}\label{ex2:120715-4, 0.05}
	\ex{\gll	n-ak: ``hiit ta-na\tbr{niʔ} kuan=ii, kaisaʔ Neanpeen. \\
						{\n}-say	\hphantom{``}{\hiit} {\ta}-move{\tbrUc} village={\ii} {\kais} Neanpeen\\
			\glt `They said: ``Let's change the village, it shouldn't be Neanpeen.'
						\txrf{130823-2, 0.57} {\emb{130823-2-00-57.mp3}{\spk{}}{\apl}}}\label{ex:130823-2, 0.57}
\end{exe}

Similarly, the citation form of consonant-final verbs is the U\=/form.
Examples of consonant-final verbs cited in the U\=/form in a recorded wordlist
are given in Table \qf{ex:ConFinVerCitFor} below.

\begin{exe}
	\ex{Consonant-final verb citation forms:}\label{ex:ConFinVerCitFor}
	\sn{\gw\begin{tabular}{llll}
			 Root							&		&Citation								& \\
			 \ve{\rt ʔapuʔ}	&\ra&\ve{na-ʔa\tbr{puʔ}} 	& `is pregnant'\\
			 \ve{\rt manis} 	&\ra&\ve{n-ma\tbr{nis}} 		& `laughs at s.o.'\\
			 \ve{\rt reruʔ}	&\ra&\ve{a|n-re\tbr{ruʔ}}	& `is sleepy'\\
			 \ve{\rt sumak}		&\ra&\ve{a|n-su\tbr{mak}} 	& `dives'\\
		\end{tabular}}
\end{exe}

This behaviour includes verbs whose final consonant is a suffix,
or the consonantal allomorph of the plural enclitic \ve{=n}.
%Suffixes which consist of a single consonant in Amarasis
%include the transitive suffixes \ve{-b} and \ve{-ʔ} (\srf{sec:TraSuf}).
The citation forms of a number of vowel-final verbs
and their corresponding forms with the plural enclitic \ve{=n}
are given in \qf{ex:PluVer} below to illustrate.

\begin{exe}
	\ex{Plural verb citation forms:}\label{ex:PluVer}
	\sn{\gw\begin{tabular}{lllll}
			 Root					& Verb	&			& Verb={\einV} 			& \\
			\ve{\rt nema}	&	\ve{neem}		&\ra&\ve{ne\tbr{ma}=\tbr{n}}	&`come' \\
			\ve{\rt tona}	&	\ve{na-toon}&\ra&\ve{na-to\tbr{na}=\tbr{n}}	&`tell' \\
			\ve{\rt mate}	&	\ve{n-maet}	&\ra&\ve{n-ma\tbr{te}=\tbr{n}}	&`die' \\
			\ve{\rt eki}	&	\ve{n-eik}	&\ra&\ve{n-e\tbr{ki}=\tbr{n}}		&`bring' \\
			\ve{\rt hini}	&	\ve{na-hiin}&\ra&\ve{na-hi\tbr{ni}=\tbr{n}}	&`know' \\
			\ve{\rt mepu}	&	\ve{n-meup}	&\ra&\ve{n-me\tbr{pu}=\tbr{n}}	&`work' \\
			\ve{\rt romi}	&	\ve{n-roim}	&\ra&\ve{n-ro\tbr{mi}=\tbr{n}}	&`like' \\
		\end{tabular}}
\end{exe}

Ro{\Q}is Amarasi behaves differently in this respect
as stems in which the penultimate or final consonant
is /n/ occur in the M\=/form (and with a final consonant cluster) by default.
U\=/forms of such stems are then used to signal lack of resolution.
Two examples of Ro{\Q}is Amarasi sentences with a consonant-final
verb in the M\=/form are given in \qf{ex:08/10/14, p.113}
and \qf{ex:09/10/14, p.114} below.
See \srf{sec:DisDriMetRoqAma} for more discussion of the
use of M\=/forms with a final cluster in Ro{\Q}is Amarasi.

\begin{exe}\let\eachwordtwo=\itshape
	\ex{\glll	\textnormal{\tcb{Ro{\Q}is:}} siin na-sa\tbr{ap}=\tbr{n}.\\
						\textnormal{\tcb{Kotos:}} siin na-sa\tbr{pa}=\tbr{n}.\\
						{} {\siin} {\na}-kick={\einV}\\
			\glt	\lh{Kotos: }`They're playing soccer.'
						\txrf{observation 08/10/14, p.113}}\label{ex:08/10/14, p.113}
	\ex{\glll	\textnormal{\tcb{Ro{\Q}is:}} raump=ein n-ma\tbr{et}=\tbr{n}.\\
						\textnormal{\tcb{Kotos:}} paku=n n-ma\tbr{te}=\tbr{n}.\\
						{} light={\ein} {\n}-die={\einV}\\
			\glt	\lh{Kotos: }`The lights have died.'
						\txrf{observation 09/10/14, p.114}}\label{ex:09/10/14, p.114}
\end{exe}

Consonant-final non-nominals in Kotos
Amarasi take the U\=/form by default.
Such words do not have metathesis alternations
to express discourse functions.

\subsection{U\=/forms before consonant clusters}\label{sec:VerBefCC}
Another phonotactic environment in which non-nominals
do not usually occur in the M\=/form is before consonant clusters.\footnote{
		Functors often occur in the M\=/form before consonant clusters.
		This is connected with the fact that the use of the U\=/form
		with functors is not fully productive.}
Before a consonant cluster the usual form of a non-nominal is
the U\=/form and lack of resolution is not morphologically marked.
This is similar to the fact that certain nouns do
not undergo metathesis before modifiers with an initial consonant cluster,
and thus cannot morphologically mark attributive modification (\srf{sec:CVFinWor}).

Like U\=/forms with a word-final consonant, U\=/forms before a consonant cluster
are glossed {\Uc} to distinguish them from discourse-driven U\=/forms.
In my corpus there are over 300 U\=/forms of verbs before a consonant
cluster and only 21 verbal M\=/forms before a consonant cluster.
Two examples of a U\=/form before a consonant cluster initial root are given in
\qf{130825-6, 10.05} and \qf{ex:130914-1, 0.53} below.

\begin{exe}
	\ex{\glll	uma ʔ-tee =ma, ʔ-ai\tbr{ti} \tbr{br}uuk.	\\
						uma ʔ-tea =ma ʔ-aiti bruuk \\
						{\uma\Uc} \q-arrive =and \q-pick.up{\tbrUc} pants	\\
			\glt	`I arrived (home) and picked up some pants.' \txrf{130825-6, 10.05} {\emb{130825-6-10-05.mp3}{\spk{}}{\apl}}}\label{130825-6, 10.05}
	\ex{\gll	{onai =te}, hoo m-te\tbr{bi} \tbr{ʔt}etaʔ.\\
						like.this  {\hoo} {\m}-turn{\tbrUc} different\\
			\glt	`Like this, you turn (it) differently.'
						\txrf{130914-1, 0.53} {\emb{130914-1-00-53.mp3}{\spk{}}{\apl}}}\label{ex:130914-1, 0.53}
\end{exe}

One of the most frequent kinds of consonant clusters in my corpus
are those created through the addition of a verbal prefix 
to a consonant-initial verb stem (\srf{sec:Pre}).
This is the most common kind of consonant cluster found after verbal U\=/forms.
Two examples are given in \qf{ex:130902-1, 3.41} and \qf{ex:120715-4, 2.26} below.

\begin{exe}
	\ex{\gll	hai m-e\tbr{ki} \tbr{m}-\tbr{s}a\tbr{nu} \tbr{m}-\tbr{b}i reʔ ʔpinan ia =t,\\
						{\hai} {\m}-bring{\tbrUc} {\m}-go.down{\tbrUc} {\m}-{\bi} {\reqt} below {\ia} ={\te}\\
			\glt	`When we went down there,'
						\txrf{130902-1, 3.41} {\emb{130902-1-03-41.mp3}{\spk{}}{\apl}}}\label{ex:130902-1, 3.41}
	\ex{\glll	iin ao-n=ee n-me\tbr{se} \tbr{n}-\tbr{n}ao n-peoʔ aafgw=ii =m,\\
						ini ao-n=ee n-mese n-nao n-peʔo afu=ii =ma\\
						{\iin} body-{\N}={\ee} {\n}-alone{\tbrUc} {\n}-go {\n}-go.by{\M} ground={\ii} =and\\
			\glt	`His body went by itself along the ground.'
						\txrf{120715-4, 2.26} {\emb{120715-4-02-26.mp3}{\spk{}}{\apl}}}\label{ex:120715-4, 2.26}
\end{exe}

While the vast majority of non-nominals are in the U\=/form before a consonant cluster,
there are 15 instances of an M\=/form before such words in my corpus.
Such examples represent only 7{\%} (21/302) of all verbs before a consonant cluster.
Two examples are given in \qf{ex:130907-3, 8.04} and \qf{ex:160326, 10.22} below.

\begin{exe}
	\ex{\glll	surat a|n-poi n-taa\tbr{m} \tbr{n}-\tbr{p}oi n-taam, au ʔ-toup.\\
						surat {\a}n-poi n-tama n-poi n-tama au ʔ-toup\\
						paper{\U} {\a\n}-exit {\n}-enter{\tbrM} {\n}-exit {\n}-enter{\M} {\au} {\q}-receive{\M}\\
			\glt	`Letters would be issued and received, issued and received, I got (one).'
						\txrf{130907-3, 8.04} {\emb{130907-3-08-04.mp3}{\spk{}}{\apl}}}\label{ex:130907-3, 8.04}
	\ex{\glll	n-ei\tbr{k} \tbr{kr}ee\j=ii neem. \\
						n-eki krei=ii nema	\\
						\n-bring{\tbrM} church={\ii} {\nema\M}\\
			\glt `(They) brought the Church here.'
						\txrf{160326, 10.22} {\emb{160326-10-22.mp3}{\spk{}}{\apl}}}\label{ex:160326, 10.22}
%	\ex{\glll	ʔ-istarii\tbr{k} \tbr{br}uuk pasan nima\\
%						ʔ-istarika bruuk pasan nima\\
%						\q-iron{\U} pants set{\U} five{\U}\\
%			\glt	`I ironed five sets of pants.' \txrf{130825-6, 10.50}}\label{ex:130825-6, 10.50}
\end{exe}

%nta'-taa' nbi
%surat anpoi ntaam npoi ntaam, 'toup.
%ntoom nsaen nafani'
%Iim arkit atneen hiit t'aa' tsaeb a'piru' baru
%'istariik bruuk pasan nima.
%'meup 'aan toon bo' haa nbi krei jit, usnaas 'po-'poi neno ia.
%atfoi' 'sui'n ee msa' ate ka bisa fa

\subsubsection{Consonant-final stems before consonant clusters}\label{sec:ConFinaVerBefCC}
There are 44 instances of a consonant-final
stem before a consonant cluster in my corpus.
In 14 instances, epenthesis (\srf{sec:Epe})
occurs to break up the underlying cluster of three consonants.
Two examples are given in \qf{ex:130913-1, 2.30} and \qf{ex:120715-4, 1.52} below.

\begin{exe}
	\ex{\glll	t-pe{\tl}pea mes \sf{baap} \sf{tua} Banus iin na-bara\tbr{b} \tbr{a}|\tbr{n}-\tbr{r}air\\
						t-pe{\tl}peo mes \sf{bapa} \sf{tua} Banus ini na-barab {\a}n-rari\\
						{\t}-{\prd}talk but father old Banus {\iin} {\na}-prepare{\tbrUc} {\a\n}-finish{\M}\\
			\glt	`We talk about it, but father Banus is already prepared.'
						\txrf{130913-1, 2.30} {\emb{130913-1-02-30.mp3}{\spk{}}{\apl}}}\label{ex:130913-1, 2.30}
	\ex{\glll	iin n-mooʔ\j=oo-n on kaunʔ=ii =ma n-nono\tbr{k} \tbr{a}|\tbr{n}-peoʔ aafgw=ii =ma\\
						ini n-moʔe=oo-n on kaunaʔ=ii =ma n-nonok {\a}n-peʔo afu=ii =ma\\
						{\iin} n-do{\Mv}={\oo-\N} {\on} snake={\ii} =and \n-crawl{\tbrUc} \a\n-go.by ground={\ii} =and\\
			\glt	`he did it like the snake and crawled along the ground'
						\xrf{120715-4, 1.52} {\emb{120715-4-01-52.mp3}{\spk{}}{\apl}}}\label{ex:120715-4, 1.52}
\end{exe}

In the remaining 30 examples in my corpus
the cluster of three consonants is not phonemically resolved.
In all but one case, the first consonant (i.e. the final consonant of the stem)
is either the glottal stop /ʔ/ or the alveolar nasal /n/.
That epenthesis is not obligatory after these consonants
is consistent with the data presented in \srf{sec:Epe},
which showed that epenthesis is uncommon between ʔ{\gap}CC,
and only optional between n{\gap}CC.

An example each of final /ʔ/ and /n/ before a consonant cluster
is given in \qf{ex:120923-2, 6.28} and \qf{ex:120715-4, 7.56} below respectively.
In both instances the first consonant of each cluster
is phonetically deleted, or has coalesced with the following consonant.

\begin{exe}\let\eachwordone=\textnormal \let\eachwordtwo=\itshape
	\ex{\glll	[i napɐpɐ \hp{=}mə̆ nsiɾ̞i \tbr{nˑ}aɔ̯̆ ˈpiʉ̞t]\\
						\hp{[}iin na-papaʔ =ma n-siri\tbr{ʔ} \tbr{n}-\tbr{n}ao piut.\\
						\hp{[}{\iin} {\na}-wound{\Uc} =and {\n}-spread{\tbrUc} {\n}-go continue\\
			\glt	\lh{[}`The wound keeps on spreading.'
						\txrf{120923-2, 6.28} {\emb{120923-2-06-28.mp3}{\spk{}}{\apl}}}\label{ex:120923-2, 6.28}
	\ex{\glll	[n͡maˈsenʊn \hp{=}{ɐmːɐ} ʔanmɐˈβanə \tbr{nβ}in \hspace{20mm} \hp{[}ɾɛ̰ʔ nanɐ̰ mɛsɛʔ]\\
						\hp{[}n-ma-senu=n =ama \hp{ʔ}a|n-ma-bana=\tbr{n} \tbr{n}-bi=n {} \hp{[}reʔ nanaʔ meseʔ\\
						\hp{[}{\n}-{\mak}-replace{\Uc}={\einV} =and \hp{ʔ}{\a\n\mak}-hit{\tbrUc}={\einV} {\n}-{\bi}={\einV} {} \hp{[}{\req} inside but\\
			\glt	\lh{[}`They replaced and fought one another inside.'
						\txrf{120715-4, 7.56} {\emb{120715-4-07-56.mp3}{\spk{}}{\apl}}}\label{ex:120715-4, 7.56}
\end{exe}

Verbs nearly always take the U\=/form
before a word which begins with a consonant cluster.
This is because clusters of three consonants are
normally disallowed in Kotos Amarasi.

\subsection{Summary}
There are a number of word classes in Amarasi for which the default form is the M\=/form
and for which the U\=/form is used to signal lack of resolution.
These word classes were listed in \qf{ex:WorClaDisMet} above,
repeated as \qf{ex:WorClaDisMet2} below.

\begin{exe}
	\ex{Word classes with discourse-driven U\=/forms:}\label{ex:WorClaDisMet2}
		\begin{xlist}
			\ex{verbs}
			\ex{numerals}
			\ex{place names}
			\ex{number enclitics (\ve{eni} `{\ein}', \ve{=esa}, `{\es}')}
			\ex{demonstratives (\ve{nana} `{\naan}')}
			\ex{determiners (\ve{=ana} `{\aan}')}
			\ex{pronouns (\ve{ini} `{\iin}', \ve{sini} `{\siin}', \ve{hiti} `{\hiit}')}
			\ex{adverbials (\ve{=ena} `{\een}', \ve{=aha} `just')}
		\end{xlist}
\end{exe}

Given the phonotactic constraints discussed in this section,
it is more accurate to say
that members of these word classes which end in CV{\#}
have discourse-driven U\=/forms when they do not occur
before a vowel-initial enclitic or a word with an initial consonant cluster.

U\=/forms of consonant-final non-nominals and U\=/forms of non-nominals
before a consonant cluster are glossed {\Uc} when it is necessary
to gloss them to distinguish them from U\=/forms which
are used morphologically to mark lack of resolution.
