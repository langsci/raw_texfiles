\section{Discourse structures in Amarasi}\label{sec:DisStrAma}
In this section I discuss the general patterns of Amarasi discourse
by means of a detailed exposition of a single short text.
This section provides the background for properly
understanding the discourse functions of U\=/forms.

The text selected for exposition is \it{Kuareno{\Q}},
a short narrative text about how the village of \it{Kuareno{\Q}}
came to have its current name.\footnote{
	The name \it{Kuareno{\Q}} is historically from
	\ve{kuan} `village' + \ve{ʔrenoʔ} `orange \it{Citrus sinensis}'.}
With sixteen clauses, this text is both short enough to allow detailed exposition,
and still long enough to illustrate a range of discourse structures.
The structure of this text is indicative of other texts.

The outline of this story is given in \trf{tab:SumKuaSto}.
In this table I have given a summary of each clause,
the part of the plot in which it occurs, which conjunctions occur
and the occurrence of U\=/forms and M\=/forms on non-nominals.\footnote{
		Recall that `{\Ucc}' is a consonant-final U\=/form,
		or U\=/form before a consonant cluster.
		As discussed in \srf{sec:DefFor1},
		the default form in such phonotactic environments is the U\=/form.
		`{\Mvv}' is an M\=/form before a vowel-initial enclitic.
		As discussed in Chapter \ref{ch:PhoMet}, M\=/forms
		are obligatory before vowel-initial enclitics.}
I have also tracked repetition between clauses with the use of capital letters,
each of which tracks a unique concept which is repeated.
Thus, the \emph{A} in clauses 1, 2, 13, 14, and 16 indicates
that each of these clauses contains the same concept
(which is not repeated elsewhere), in this case \ve{Kuarenoʔ} `Kuareno{\Q}'.
Similarly, \emph{D} in clauses 5 and 6 indicates that these clauses
contain the same concept, in this case \ve{ʔrenoʔ} `orange tree'.

\begin{table}[h]
	\caption{Summary of Kuareno{\Q} story}\label{tab:SumKuaSto}
	\centering\stl{0.25em}
		\begin{tabular}{rl@{\hspace{0.01em}}rllccccl} \lsptoprule
				&Plot&\mc{1}{l}{Conj.}&Summary		&\tsc{u/m}			&\mc{4}{l}{Repetition}&Index\\ \midrule
				1&\it{Opening}	&& Kuareno{\Q}'s name is K. because	&\tsc{{\Ucc}} \tsc{{\Ucc}}&A&B& & &\qf{ex:130823-2, 0.00}\\ \hline
				2&\it{Setting}	&& at first, its name wasn't K.			&\tsc{u} \tsc{{\Ucc}}			&A&B& & &\qf{ex:130823-2, 0.09}\\
				3&				&& its name was Neanpeen 									&\tsc{m}									& &B&C& &\qf{ex:130823-2, 0.13}\\
				4&				&& there were lots of people							&\tsc{m}									& & & & &\qf{ex:130823-2, 0.17}\\ \hline
				5&\it{Inciting}	&then& they planted an orange tree	&\tsc{m} \tsc{{\Ucc}}			&D& & & &\qf{ex:130823-2, 0.22}\\
				6&\it{incident}	&& a single orange tree 						&													&D& & & &\qf{ex:130823-2, 0.29}\\ \hline
				7&\it{Climax}		&then& it grew two branches 				&													&E& & & &\qf{ex:130823-2, 0.31}\\
				8&				&& it grew two branches 									&													&E& & & &\qf{ex:130823-2, 0.34}\\
				9&				&& one of the branches										&													&F& & & &\qf{ex:130823-2, 0.36}\\
				10&				&& its contents and fruit were red				&													&G&G& & &\qf{ex:130823-2, 0.42}\\
				11&				&& one was white													&													&G& & & &\qf{ex:130823-2, 0.45}\\
				12&				&& one of the branches was white					&													&F& & & &\qf{ex:130823-2, 0.49}\\ \hline
				13&\it{Dénouement}&so& someone called it K.					&\tsc{{\Mvv}} \tsc{{\Ucc}}&A&B& & &\qf{ex:130823-2, 0.51}\\
				14&\mc{2}{r}{so}& they named it K.									&\tsc{{\Mvv}}							&A&B& & &\qf{ex:130823-2, 0.55}\\
				15&				&& let's change it, not Neanpeen					&\tsc{{\Ucc}} \tsc{m}			& & &C&H&\qf{ex:130823-2, 0.56}\\
				16&				&but& let's change its name to K.					&\tsc{{\Mvv}} \tsc{{\Ucc}}&A&B& &H&\qf{ex:130823-2, 0.59}\\ %\hline
%				17&\it{Closure}&& that's it													&								& & & & &\qf{ex:130823-2, 1.03}\\
				\lspbottomrule
		\end{tabular}
\end{table}

I have broken the text up according to the plot structure,
and discuss each chunk in turn.
The identification of different parts of the
plot follows the principles and protocols outlined in \citet{dole01}.
Parts of each chunk which receive special discussion
are indicated in boldface type.

Line \qf{ex:130823-2, 0.00} is the Opening of the story.
After gathering his thoughts,
the narrator provides a short explanation that the text
is about the name of \it{Kuareno{\Q}} village.

\begin{exe}
	\ex{Kuareno{\Q} -- opening: \txrf{130823-2} {\emb{130823-2-00-00.mp3}{\spk{}}{\apl}}}\label{ex:130823-2, 0.00}
		\sn{\glll	ahh, {Kuarenoʔ ahh}, iin kaan-n=ee Kuarenoʔ {na-tuinaʔ ahh}\\
							{} Kuarenoʔ ini kana-n=ee Kuarenoʔ na-tuinaʔ \\
							{}  Kuareno{\Q}{\Uc} {\iin} name-{\N=\ee} Kuareno{\Q}{\Uc} {\na}-because \\
				\glt	`Umm, Kuareno{\Q}, its name is Kuareno{\Q} because,' \txrf{0.00}}
\end{exe}

This opening line is followed by the setting,
given as \qf{ex:130823-2, 0.09-0.17} below.
The Setting is the part of the story in which the narrator provides
background information about the place, time, and participants of the story.
In \qf{ex:130823-2, 0.09-0.17} we learn the time this story
took place (`long ago, at first') and more about the main participant, the village of Kuareno{\Q}.

\begin{exe}
	\ex{Kuareno{\Q} -- Setting: \txrf{130823-2} {\emb{130823-2-00-09-00-17.mp3}{\spk{}}{\apl}}}\label{ex:130823-2, 0.09-0.17}
	\begin{xlist}
		\ex{\glll	na-hu\tbr{nu} =\tbr{t}, iin kaan-n=ee ka= Kuarenoʔ =fa.\\
							na-hunu =te ini kana-n=ee ka= Kuarenoʔ =fa\\
							{\na}-first{\tbrU} {=\tbr{\te}} {\iin} name-{\N=\ee} {\ka}= Kuareno{\Q}{\Uc} ={\fa}\\
				\glt	`Well, at first its name wasn't Kuareno{\Q}.' \txrf{0.09}}\label{ex:130823-2, 0.09}
		\ex{\glll	iin kaan-n=ee ahh Neanpeen.\\
							ini kana-n=ee {} Neanpeen\\
							{\iin} name-{\N=\ee} {} Neanpeen{\M}\\
				\glt	`Its name was Neanpeen.' \txrf{0.13}}\label{ex:130823-2, 0.13}
		\ex{\glll	a|n-nao{\tl}nao =\tbr{te}, a|n-muiʔ toogw=ii na-mfau.\\
							{\a}n-nao{\tl}nao =te, {\a}n-muʔi too=ii na-mfau\\
							\a\n-{\frd}go {=\tbr{\te}} {\a\n}-have{\M} citizen={\ii} {\na}-many\\
				\glt	\lh{\a}`After a while, it had a lot of residents.' \txrf{0.17}}\label{ex:130823-2, 0.17}
	\end{xlist}
\end{exe}

In \qf{ex:130823-2, 0.09} there is a purely discourse
driven U\=/form; \ve{na-hunu} `at first; long ago',
which is resolved by the following two clauses
which describe the situation which held `long ago'.

In \qf{ex:130823-2, 0.09-0.17} there are also two occurrences of
the connector \ve{=te}, glossed as ={\te} `subordinator'.
This particle marks that the preceding clause is temporally
subordinate to the following clause.
This particle is always clause final and provides the background
which sets the scene for the following clause.
The clause preceded by \ve{=te} is the stage
on which the following clause takes place.

In \qf{ex:130823-2, 0.09} the clause \ve{na-hunu} `at first'
is the time of the next clause.
In \qf{ex:130823-2, 0.17}, the clause preceding \ve{=te}
is an event (\ve{a|n-nao{\tl}nao} `it went on') which preceded the clause following \ve{=te}.
Due to the semantics of this connector (background for next clause),
verbs before \ve{=te} obligatorily occur in the U\=/form (resolved by next clause).
The use of this enclitic is discussed in more detail in \srf{sec:Coo=Te}.

After the scene has been set in \qf{ex:130823-2, 0.09-0.17},
the narrator introduces the Inciting Incident,
given as \qf{ex:130823-2, 0.22-0.29} below.
The Inciting Incident of the story is the part of a story
in which something first happens and the storyline gets moving.
In \qf{ex:130823-2, 0.22-0.29} the inciting incident is introduced
by the conjunction \ve{oka =te} `after that, then'.\footnote{
Historically this conjunction is from \ve{okeʔ} `all, finished'
and the subordinating enclitic \ve{=te}.}
It is common for new parts of the plot to be introduced with conjunctions.
Conjunctions which do not introduce new parts of the story,
such as \ve{=ma} `and', are usually clause final.
I call such clause final conjunctions \emph{connectors}.
Connectors are discussed in more detail in \srf{sec:DepCoo}.

\begin{exe}
	\ex{Kuareno{\Q} -- Inciting Incident: \txrf{130823-2} {\emb{130823-2-00-22-00-29.mp3}{\spk{}}{\apl}}}\label{ex:130823-2, 0.22-0.29}
	\begin{xlist}
		\ex{\glll	{\tbr{oka} =\tbr{te}}, siin n-seen n-ana ʔreanʔ=ees, \\
							{okeʔ =te} sini n-sena n-ana ʔrenoʔ=esa \\
							after.that {\siin} {\n}-plant{\M} {\n}-{\ana\Uc} orange={\es}\\
				\glt	`After that, they planted a orange tree,' \txrf{0.22}}\label{ex:130823-2, 0.22}
		\ex{\glll	uʔu meseʔ, ʔreanʔ=ii uʔu meseʔ\\
							uʔu meseʔ  ʔrenoʔ=ii uʔu meseʔ\\
							tree single orange={\ii} tree single\\
				\glt	`A single one, a single orange tree.'\\
							(\emph{lit.} `A single tree, the orange was a single tree.') \txrf{0.29}}\label{ex:130823-2, 0.29}
	\end{xlist}
\end{exe}

Another common feature of Amarasi discourse found in 
\qf{ex:130823-2, 0.22-0.29} is repetition.
%Repetition is common to many speakers in many text genres
%and is not just a feature of this particular text.
The orange tree is repeated twice as is the fact that it was a single tree.
None of these instances of repetition are false starts.
Instead, repetition is a common feature of Amarasi discourse
and is found with all speakers (including eloquent speakers) in many text genres.
%Repetition in the form of semantic parallelism
%is an obligatory part of Amarasi poetry (\S \ref{sec:ParUforMforPoe}).

Repetition has already been seen in
the Opening \qf{ex:130823-2, 0.00}
and Setting \qf{ex:130823-2, 0.09-0.17} of this text,
with three repetitions of \ve{Kuarenoʔ}
and three of \ve{iin kaan-n=ee} `its name'.
Metathesis and repetition interact in Amarasi,
as one use of U\=/forms is to mark
one half of a tail-head linkage construction
with identical verbs (\srf{sec:TaiHeaLin}).

After the Inciting Incident comes the Climax,
the main problem of the story which needs to be solved.
The Climax is given as \qf{ex:130823-2, 0.31-0.49} below.
As with the Inciting Incident, the Climax
is introduced with the conjunction \ve{oke =t} `after that, then'.
As in other parts of the story, the climax also has a large
amount of repetition. %in \qf{ex:130823-2, 0.31-0.49}.
In fact, there is no clause in \qf{ex:130823-2, 0.31-0.49}
which is not repeated in this section.

\begin{exe}
	\ex{Kuareno{\Q} -- Climax: \txrf{130823-2} {\emb{130823-2-00-31-00-49.mp3}{\spk{}}{\apl}}}\label{ex:130823-2, 0.31-0.49}
	\begin{xlist}
		\ex{\glll	{oke =t} iin \tbr{na}-\tbr{tae} \tbr{tae}-\tbr{f} \tbr{nua}. aah\\
							{okeʔ =te} ini na-tae taʔe-f nua\\
							after.that {\iin} {\na}-branch branch-{\f} two\\
				\glt	`After that, it grew two branches. [murmur of satisfaction]' \txrf{0.31}}\label{ex:130823-2, 0.31}
		\ex{\glll	\tbr{na}-\tbr{tae} \tbr{tae}-\tbr{f} \tbr{nua},\\
							na-tae taʔe-f nua\\
							{\na}-branch branch-{\f} two\\
				\glt	`It grew two branches.' \txrf{0.34}}\label{ex:130823-2, 0.34}
		\ex{\glll	ees=ii, iin taaʔʤ=ees=ii ahh, \\
							esa=ii ini taʔe=esa=ii {} \\
							{\es}={\ii} {\iin} branch={\es=\ii} {} \\
				\glt	`One of these, one of its branches,' \txrf{0.36}}\label{ex:130823-2, 0.36}
		\ex{\glll	\tbr{aaf}-\tbr{n}=\tbr{ee} meʔe, \tbr{fua}-\tbr{n}=\tbr{ee} meʔe.\\
							afa-n=ee meʔe fua-n=ee meʔe\\
							content-{\N=\ee} red fruit-{\N=\ee} red\\
				\glt	`Its contents were red, its fruit was red.' \txrf{0.42}}\label{ex:130823-2, 0.42}
		\ex{\glll	ees=ee mutiʔ. \hspace{5mm}mmh\\
							esa=ee mutiʔ\\
							{\es=\ee} white\\
				\glt	`One was white. [murmur of satisfaction]' \txrf{0.45}}\label{ex:130823-2, 0.45}
		\ex{\glll	taaʔʤ=ees=ii mutiʔ.\\
							taʔe=esa=ii mutiʔ\\
							branch={\es\Mv=\ii} white\\
				\glt	`One of the branches was white.' \txrf{0.49}}\label{ex:130823-2, 0.49}
	\end{xlist}
\end{exe}

\largerpage
There are at least three types of repetition in \qf{ex:130823-2, 0.31-0.49}.
Clauses \qf{ex:130823-2, 0.31} and \qf{ex:130823-2, 0.34}
are an instance of verbatim repetition:
part of the clause is simply repeated word for word.
Clause \qf{ex:130823-2, 0.42} contains parallelism,
in which the same or a similar idea is expressed with non-identical words.
In \qf{ex:130823-2, 0.42} the parallelism is between \ve{aaf-n=ee} `its contents'
and \mbox{\ve{fua-n=ee}} `its fruit'.
Clauses \qf{ex:130823-2, 0.42} and \qf{ex:130823-2, 0.45} also
contain parallel concepts, in this case \ve{meʔe} `red' and \ve{mutiʔ} `white'.\footnote{
		The colours of the Indonesian national flag are red and white,
		and a common term for this flag is \it{merah putih} `red white'.
		The similarity between the colour of the fruit in this story
		and the Indonesian flag is probably not a coincidence.}
\clearpage

Parallelism is an important feature of many languages in Timor,
particularly of (but not restricted to) their poetic registers.
\citet{fo88,fo14} and \citet[15ff]{grth97}
discuss the use of parallelism in the languages of this region.
I discuss parallelism in Amarasi and Timor
in more detail in \srf{sec:PoePar} and \srf{sec:MetPar}.

Clauses \qf{ex:130823-2, 0.36}--\qf{ex:130823-2, 0.49}
present a third kind of repetition, namely chiasmus.
Chiasmus typically has the structure ABB′A′,
where the first and final clauses are parallel to one another
and the middle two clauses are parallel to each other.
One use of discourse U\=/forms is to mark the centre of a chiastic structure (\srf{sec:CenChi}).
The chiastic structure of \qf{ex:130823-2, 0.36}--\qf{ex:130823-2, 0.49}
is represented below:

\begin{exe}
	\sn{\xytext{one of the&\fbox{branches}\xybarconnect[4][-](D,D){6}&was&\fbox{red}\xybarconnect[2][-](D,D){2}
							&was&\fbox{white}&one of the&\fbox{branches}}}
\end{exe}

The final part of the story is the Dénouement, the part of the
story where the problem introduced in the climax is solved.
The dénouement of this story is given in \qf{ex:130823-2, 0.51-1.03}.
The Climax and/or the Dénouement of the story is usually the most important
part of the story, and these sections are often referred to collectively as the Peak.

\begin{exe}
	\ex{Kuareno{\Q} -- Dénouement: \txrf{130823-2} {\emb{130823-2-00-51-00-59.mp3}{\spk{}}{\apl}}}\label{ex:130823-2, 0.51-1.03}
	\begin{xlist}
		\ex{\glll	\sf{{\j}adi} esa n-teek=ee =t n-ak Kuarenoʔ. aah \\
							\sf{{\j}adi} esa n-teka=ee =te n-ak Kuarenoʔ {}\\
							so {\es} {\n}-call{\Mv}={\eeV} ={\te} {\n-\ak} Kuareno{\Q}{\Uc}\\
				\glt	`So someone called it Kuareno{\Q}. [murmur of satisfaction]' \txrf{0.51}}\label{ex:130823-2, 0.51}
		\ex{\glll	{onai =m} siin na-kaan-b=ee n-eu: \\
							{onai =ma} sini na-kana-b=ee n-eu \\
							and.so {\siin} {\na}-name{\Mv}-{\b=\eeV} {\n-\eu}\\
				\glt	`and so they named it' \txrf{0.55}}\label{ex:130823-2, 0.55}
		\ex{\glll	n-ak ``hiit ta-naniʔ kuan=ii, kaisaʔ Neanpeen\\
							n-ak \hphantom{``}hiti ta-naniʔ kuan=ii kaisaʔ Neanpeen\\
							{\n}-say \hphantom{``}{\hiit} {\ta}-move{\Uc} village={\ii}  {\kais} Neanpeen{\M}\\
				\glt	`saying ``Let's change the village, it shouldn't be Neanpeen' \txrf{0.56}}\label{ex:130823-2, 0.56}
		\ex{\glll	\sf{tapi} {tanai ahh} ta-nainʔ=ee, iin kaan-n=ee Kuarenoʔ.\\
							\sf{tapi} {} ta-naniʔ=ee ini kana-n=ee Kuarenoʔ\\
							but {} {\ta}-move{\Mv}={\eeV} {\iin} name-{\N=\ee} K.{\Q}{\Uc}\\
				\glt	`but we'll change it, its name will be Kuareno{\Q}.' \txrf{0.59}}\label{ex:130823-2, 0.59}
	\end{xlist}
\end{exe}

\newpage
As in the Inciting Incident and the Climax,
the Dénouement in \qf{ex:130823-2, 0.51-1.03} is also introduced by a conjunction;
in this case the conjunctions used are \ve{{\j}adi}
`so' (from Malay \it{jadi} [\j adi]) and \ve{onai =m} `and so'.
Both these conjunctions have the sense of `so, consequently'
and tend to be used in logical relations, rather than temporal relations.

Again, there is a large amount of repetition in the Dénouement.
Two different verbs for naming occur, \ve{a|n-teek=ee} `called it'
and \ve{na-kaan-b=ee} `named it'.
The verb \ve{ta-naniʔ} `move, change' also occurs twice.
In addition, the final two clauses of the Dénouement
form a high-level chiasmus with the first
two clauses of the setting in \qf{ex:130823-2, 0.09-0.17}.
Such a structure is known as a sandwich structure.

In this short text we see three common features of Amarasi discourse.
Firstly, Amarasi employs a large amount of repetition of different kinds.
Such repetition includes verbatim repetition, parallelism, and chiasmus.
Secondly, new parts of the story are typically introduced with clause initial
conjunctions such as \ve{okeʔ =te} `after that, then' or \ve{onai =m} `and so'.
Thirdly, the particle \ve{=te} is used to background information
which is the setting/background of the following clauses.
In the following sections we will see the way U\=/forms and M\=/forms
interact with repetition as well as the connectors \ve{=ma} and \ve{=te}.
