\section{Default M\=/form}\label{sec:DefFor1}
For the non-nominal word classes given in \qf{ex:WorClaDisMet} above
the default semantic form is the M\=/form
(see \srf{sec:PhoCon} for exceptions).
Even though the M\=/form of these word classes is the semantically default form,
the U\=/form must still be posited as the morphologically underlying form.
This can be shown by the processes of vowel assimilation which occur
in the formation of M\=/forms (\srf{sec:VowAss}).
Two examples of minimal pairs with identical M\=/forms but different U\=/forms are
\ve{{\rt}nene} `press' and \ve{{\rt}nena} `hear' {\ra} \ve{n-neen} `presses'/`hears'
as well as \ve{{\rt}rene} `field' and \ve{{\rt}rena} `force' {\ra}
\ve{na-reen} `makes a field'/`forces'.

For these word classes, the morphologically unmarked form
is the semantically marked form with special discourse uses (unresolved),
and the morphologically marked form is the semantically unmarked form without special discourse uses.
This difference between nominals and other word classes is shown in \trf{tab:NomVerMet}.

\begin{table}[h]
	\caption{Nominal and non-nominal metatheses}\label{tab:NomVerMet}
	\centering
		\begin{tabular}{rcc} \lsptoprule
							& unmarked	& marked\\
							& semantics	& semantics\\ \midrule
			nominal	& U\=/form		& M\=/form \\
			other		& M\=/form		& U\=/form \\ \lspbottomrule
		\end{tabular}
\end{table}

Most of the discussion in this section focusses on verbs
as these are the most well-attested word class with default M\=/forms,
though the statements also hold for the other non-nominal word classes.
The M\=/form of these word classes is the form
used in simple declarative sentences,
the citation form, and the most common form.
Three simple declarative sentences
are given in \qf{ex:130928-1, 0.02}--\qf{ex:120715-3, 0.10} below.
Each of these examples is taken from the very beginning of its text.
The verb in each instance take the M\=/form.

\begin{exe}
	\ex{\glll	neno ia aam Nahor Bani iin n-ma\tbr{et}.\\
						neno ia ama Nahor Bani ini n-mate\\
						day {\ia} father Nahor Bani {\iin} {\n}-die{\tbrM}\\
			\glt	`Today father Nahor Bani died.'
						\txrf{130928-1, 0.02} {\emb{130928-1-00-02.mp3}{\spk{}}{\apl}}}\label{ex:130928-1, 0.02}
	\ex{\glll	krei ia iin naan-n=ii, hai m-re\tbr{es} surat Roma.\\
						krei ia ini nana-n=ii, hai m-resa surat Roma\\
						week {\ia} {\iin} inside-{\N}={\ii} {\hai} {\m}-read{\tbrM} paper Romans\\
			\glt	`During this week we read the book of Romans.'
						\txrf{130920-1, 0.22} {\emb{130920-1-00-22.mp3}{\spk{}}{\apl}}}\label{ex:130920-1, 0.22}
	\ex{\glll	ahh, hai m-baise\tbr{un} fuunn=ee =te, ahh\\
						{} hai m-baisenu funan=ee =te {}\\
						{} {\hai} {\m}-look.up{\tbrM} moon={\ee} ={\te} {}\\
			\glt	`Umm, when we looked up at the moon,'
						\txrf{120715-3, 0.10} {\emb{120715-3-00-10.mp3}{\spk{}}{\apl}}}\label{ex:120715-3, 0.10}
\end{exe}

As the default form, the M\=/form is also the usual citation form.
The citation forms of a number of vowel-final verbs and numerals in one recorded
wordlist are given in \qf{ex:VowFinVerCitFor} below.
Verbs occur with the \tsc{3sg} agreement market \ve{(a|)n-} or \ve{na-}.

\begin{exe}
	\ex{Vowel final verb and numeral citation forms:}\label{ex:VowFinVerCitFor}
	\sn{\gw\begin{tabular}{llll}
				Root					&		&Citation							& \\
				\ve{\rt henu}	&\ra&\ve{na-he\tbr{un}}		&`fill, is full'\\
				\ve{\rt hini}	&\ra&\ve{na-hi\tbr{in}}	&`know'\\
				\ve{\rt ita}	&\ra&\ve{n-i\tbr{it}}			&`look at'\\
				\ve{\rt kisu}	&\ra&\ve{a|n-ki\tbr{us}}	&`see'\\
				\ve{\rt mate}	&\ra&\ve{n-ma\tbr{et}}		&`die'\\
				\ve{\rt nena}	&\ra&\ve{a|n-ne\tbr{en}}	&`hear'\\
				\ve{\rt roʔa}	&\ra&\ve{a|n-ro\tbr{oʔ}}	&`vomit'\\		
				\ve{\rt roro}	&\ra&\ve{a|n-ro\tbr{or}}	&`kill by stabbing'\\
				\ve{\rt tenu}	&\ra&\ve{te\tbr{un}}	&`three'\\
				\ve{\rt nima}	&\ra&\ve{ni\tbr{im}}	&`five'\\
				\ve{\rt hitu}	&\ra&\ve{hi\tbr{ut}}	&`seven'\\
				\ve{\rt fanu}	&\ra&\ve{fa\tbr{un}}	&`eight'\\
		\end{tabular}}
\end{exe}

For nominals, the semantically default form is the U\=/form,
with the M\=/form marking modification (see Chapter \ref{ch:SynMet}).
There are a number of roots in Amarasi
which can occur as either a verb or nominal.
Such roots are cited in U\=/form for the nominal meaning
and the M\=/form for the verbal meaning.
Examples are given in \trf{tab:AmaNomVerPai} below.

\begin{table}[h]
	\caption{Citation forms of noun-verb pairs}\label{tab:AmaNomVerPai}
	\centering
		\begin{tabular}{lllll} \lsptoprule
			Root						&Nom.								&Gloss (N.)		&Verb									& Gloss (V.) \\ \midrule
			\ve{\rt heʔo}		&\ve{he\tbr{ʔo}}		&`(a) saw'		&\ve{n-he\tbr{oʔ}}		&`(to) saw' \\
%			\ve{\rt kinu}		&\ve{ki\tbr{nu}-f}	&`cheek'			&\ve{na-ki\tbr{un}}		&`(to) spit' \\
			\ve{\rt ʔsoko}	&\ve{ʔso\tbr{ko}}		&`sign'				&\ve{na-ʔso\tbr{ok}}	&`make a sign' \\
			\ve{\rt nope}		&\ve{no\tbr{pe}}		&`cloud'			&\ve{n-no\tbr{ep}}		&`be cloudy' \\
			\ve{\rt reko}		&\ve{re\tbr{ko}}		&`good'				&\ve{na-re\tbr{ok}}		&`be good' \\
			\ve{\rt rono}		&\ve{ro\tbr{no}-f}	&`saliva'			&\ve{n-ro\tbr{on}}		&`(to) spit' \\
			\ve{\rt siʔu}		&\ve{si\tbr{ʔu}-f}	&`(an) elbow'	&\ve{n-si\tbr{uʔ}}		&`(to) elbow' \\
			\ve{\rt snasa}	&\ve{sna\tbr{sa}-f}	&`breath'			&\ve{na-sna\tbr{as}}	&`take a break' \\
%			\ve{\rt tika}		&\ve{ti\tbr{ka}-f}	&`heel'				&\ve{na-ti\tbr{ik}}		&`(to) stamp' \\
		%	\ve{\rt }	&\ve{}	&\ve{}	&`'	&`' \\
		\lspbottomrule
		\end{tabular}
\end{table}

The M\=/form is also the most frequent form for non-nominals.
In my corpus M\=/forms comprise 73{\%} (3,858/5,271) of non-nominals.
After excluding M\=/forms which are obligatory before vowel-initial enclitics
(544 instances), U\=/forms which are consonant-final stems (556 instances),
and U\=/forms before consonant clusters (435 instances),
M\=/forms constitute 89{\%} (3,313/3,736) of the relevant word classes.
Put differently, the semantically unmarked form occurs in 89{\%} of instances.
The figures for each word class are detailed in \trf{tab:FreUfoMfoTex}.

\begin{table}[h]
	\caption[Frequency of U\=/forms and M\=/forms in texts]
					{Frequency of U\=/forms and M\=/forms in texts\su{†}}\label{tab:FreUfoMfoTex}
	\centering
		\begin{threeparttable}[b]
			\begin{tabular}{rllll|lll}\lsptoprule
					&	U\=/form	&	/{\gap}CC	&	/C{\#}	&	else.	&	M\=/form	&	/{\gap}=V	&	else.	\\	\midrule
				verbs	&	1,077	&	350	&	441	&	286	&	1,913	&	440	&	1,473	\\	
				numerals	&	41	&	2	&	26	&	13	&	76	&	25	&	51	\\	
				place names	&	99	&	0	&	89	&	10	&	51	&	4	&	47	\\	
				\ve{esa/=esa}	&	79	&	41	&	0	&	38	&	375	&	39	&	336	\\	
				eni	&	21	&	3	&	{--}	&	18	&	171	&	26	&	145	\\	
				pronouns	&	46	&	30	&	{--}	&	16	&	987	&	10	&	977	\\	
				dem./det.	&	24	&	4	&	{--}	&	20	&	144	&	0	&	144	\\	
				\ve{=ena/=aha}	&	27	&	5	&	{--}	&	22	&	140	&	0	&	140	\\	
				total	&	1,414	&	435	&	556	&	423	&	3,857	&	544	&	3,313	\\	
			\lspbottomrule
			\end{tabular}
			\begin{tablenotes}
				\item [†] U\=/form = total U\=/forms,
									/{\gap}CC = U\=/forms before consonant clusters,
									/C{\#} = consonant-final stems in U\=/form,
									else. = U\=/forms elsewhere (discourse-driven U\=/forms),
									M\=/form = total M\=/forms,
									/{\gap}=V = M\=/forms before vowel-initial enclitics,
									else. = other M\=/forms
			\end{tablenotes}
		\end{threeparttable}
\end{table}

M\=/forms are the semantically default form for verbs,
numerals, number enclitics, place names, pronouns, demonstratives,
determiners, and adverbials.
For these word classes U\=/forms normally mark
an unresolved event or situation.
