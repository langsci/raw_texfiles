\section{Introduction}
In this chapter I analyse the morphological use of metathesis
in Amarasi to mark discourse structures.
The final member of a phrase or clause uses
metathesis to mark that the situation/event encoded
by the clause is unresolved and requires another clause to achieve resolution.
A discourse U\=/form (unmetathesised) typically
occurs in a parallel and complementary relationship
with an M\=/form (metathesised), the latter of which resolves the former.

Example \qf{ex:02/08/13, p.20 -2} is a question-answer
pair in which a question with a final U\=/form
is resolved by an answer in the M\=/form.
Example \qf{ex:130909-6, 0.39 -2} contains two events.
The second event is encoded in the M\=/form and is dependent
on the prior clause with a final U\=/form for its realisation.

\begin{exe}
	\ex{\glll \textnormal{\tcb{Q:}} hoo \bxA{mu-be\tbr{ʔi}}? \hspace{10mm} \textnormal{\tcb{A:}} au \bxB{u-be\tbr{iʔ}}! {\lk}\\
						{} hoo {\gp}mu-beʔi {} {} au {\gp}u-beʔi \\
						{} {\hoo} \gp\muu-capable{\tbrU} {} {} {\au} \gp\qu-capable{\tbrM} \\
			\glt	\lh{Q: }`Can you do it?' \hspace{19.75mm}`Yes I can!'
						\txrf{observation 02/08/13, p.20}}\vspace{4pt}\label{ex:02/08/13, p.20 -2} 
	\ex{\glll	m-ak hai nua =kai \bxA{m-taiko\tbr{bi}} =m hai \bxB{m-ma\tbr{et}} okeʔ {\lk}\\
						m-ak hai nua =kai {\gp}m-taikobi =ma hai {\gp}m-mate okeʔ \\
						\m-say {\hai} two {\kai} \gp\m-fall{\tbrU} =and {\hai} \gp\m-die{\tbrM} all \\
			\glt	`So we two will fall down and (then) both die.'
						\txrf{130909-6, 0.39} {\emb{130909-6-00-39.mp3}{\spk{}}{\apl}}}\label{ex:130909-6, 0.39 -2}
\end{exe}

Discourse U\=/forms are used by speakers to signal that
the event or situation is not resolved.
Such a U\=/form represents half of a whole
which requires resolution by another clause.
Discourse-driven U\=/forms leave the audience in a state of  suspense
with the speaker signalling that more information
is required to resolve the situation or event
encoded by a clause with a final U\=/form.

Nearly all word classes which are not canonical nominals
(as defined in \srf{sec:NomWorCla}) have discourse-driven U\=/forms.
These word classes are listed in \qf{ex:WorClaDisMet} below.
I refer to these word classes as \emph{non-nominals} throughout this chapter.
The label `non-nominal' is not ideal as place names,
demonstratives, and pronouns all have many nominal characteristics.
However, there does not seem to be a more appropriate term
which covers all of the word classes in \qf{ex:WorClaDisMet}.

\begin{exe}
	\ex{Word classes with discourse-driven U\=/forms:}\label{ex:WorClaDisMet}
		\begin{xlist}
			\ex{verbs}
			\ex{numerals}
			\ex{place names}
			\ex{number enclitics (\ve{eni} `{\ein}', \ve{=esa}, `{\es}')}
			\ex{demonstratives (\ve{nana} `{\naan}')}
			\ex{determiners (\ve{=ana} `{\aan}')}
			\ex{pronouns (\ve{ini} `{\iin}', \ve{sini} `{\siin}', \ve{hiti} `{\hiit}')}
			\ex{adverbials (\ve{=ena} `{\een}', \ve{=aha} `just')}
		\end{xlist}
\end{exe}

The use of U\=/forms is productive for verbs, numerals, and place names.
For the other word classes listed in \qf{ex:WorClaDisMet},
particularly the number enclitic \ve{=eni} `{\ein}',
the adverbials \ve{=ena} `{\een}' or \ve{=aha} `just',
as well as the demonstratives and determiners,
the use of U\=/forms is less productive, though there
are still many instances in which U\=/forms with
these latter classes signal a lack of resolution.

Discourse U\=/forms typically occur in certain constructions and environments.
These constructions and environments include dependent coordination (\srf{sec:DepCoo}),
tail-head linkage (\srf{sec:TaiHeaLin}), poetic parallelism (\srf{sec:PoePar}),
chiasmus (\srf{sec:CenChi}) and interactions between speakers (\srf{sec:IntUnm}).
These five constructions are summarised in \trf{tab:ConDisUfoTypOcc}
above along with the typical structure of each.
There are 423 discourse-driven U\=/forms in my corpus.
Of these, 406 (96\%) clearly occur in one of the five
constructions/environments given in \trf{tab:ConDisUfoTypOcc}.\footnote{
		With 423 attestations, discourse (un)metathesis is a
		well-attested morphological process.
		For comparison, my corpus has 152 instances of partial reduplication (\srf{sec:Red})
		and 48 of the reciprocal prefix \ve{ma(k)-} (\srf{sec:RecPre}).}

\begin{table}[ht]
	\caption[Constructions in which discourse U\=/forms typically occur]
					{Constructions in which discourse U\=/forms typically occur\su{†}}\label{tab:ConDisUfoTypOcc}
	\centering
		\begin{threeparttable}[b]
		\begin{tabular}{rll}
			\lsptoprule
			Construction & Typical Structure  &\\ \midrule
				&&\\[-6pt]
			Dependent coordination &
				\xytext{\fbox{event\sub{1}{\U}}\xybarconnect[2][->]{2}&(conj.)&\fbox{event\sub{2}({\M})}} &\S\ref{sec:DepCoo}\\
				&&\\[-6pt]\hline
				&&\\[-3pt]
			Tail-head linkage & 
				\xytext{\fbox{event\sub{1}{\M}}\xybarconnect[2][-](D,D){1}&\fbox{event\sub{1}{\U}}\xybarconnect[2][->]{2}&(conj.)&\fbox{event\sub{2}\vp{\M}}} &\S\ref{sec:TaiHeaLin}\\
				&&\\[-6pt]\hline
				&&\\[-6pt]
			Poetic parallelism &
				\xytext{\fbox{synonym\sub{1}{\U}}\xybarconnect[2][-](D,D){2}\xybarconnect[2][->]{2}&conj.&\fbox{synonym\sub{2}{\M}}} &\S\ref{sec:PoePar}\\
				&&\\[-6pt]\hline
				&&\\[-3pt]
			Chiasmus & 
				\xytext{\fbox{information\sub{1}}\xybarconnect[2][-](D,D){2}\xybarconnect[2][<-]{1}&\fbox{U\=/form\vp{\sub{1}}}\xybarconnect[2][->]{1}&\fbox{information\sub{1}}} &\S\ref{sec:CenChi}\\
				&&\\[-6pt]\hline
				&&\\[-6pt]
			Interaction & 
				\xytext{Speaker\sub{1}:&\fbox{U\=/form}\xybarconnect[2][-](D,D){2}\xybarconnect[2][->]{2}&Speaker\sub{2}:&\fbox{M\=/form}} &\S\ref{sec:IntUnm}\\
				&&\\[-6pt] %\hline
				\lspbottomrule
				\end{tabular}
			\begin{tablenotes}
				\item [†] An arrow indicates the form which resolves a U\=/form
												and a line joining two forms indicates forms
												which are semantically identical or parallel.
												Event\sub{1} and event\sub{2} refer to
												two different events or situations,
												with event\sub{1} beginning before event\sub{2}.
			\end{tablenotes}
		\end{threeparttable}
\end{table}

Before I discuss the details of U\=/forms in Amarasi discourse,
I discuss several other facts which provide helpful background information:
the co-occurrence of syntactically driven M\=/forms
and discourse-driven U\=/forms (\srf{sec:SynDisDriMet}),
the fact that the M\=/form is the semantically
unmarked form of non-nominals (\srf{sec:DefFor1}),
phonotactic constraints which block the use
of discourse U\=/forms (\srf{sec:PhoCon}),
and some of the general structures of Amarasi discourse (\srf{sec:DisStrAma}).

\section{Syntactically and discourse-driven metathesis}\label{sec:SynDisDriMet}
Discourse-driven U\=/forms occur with
the final members of phrases, while syntactically driven
M\=/forms occur with the medial members of phrases.
As a result, they are in complementary distribution and 
there is no competition between these two morphological uses of metathesis.
Two illustrative examples are given in \qf{ex2:160326, 5.37}
and \qf{ex2:160326, 18.26} below.

In \qf{ex2:160326, 5.37} the first member of the serial verb
construction \ve{ta-hiin t-ana} `figure out, get to know'
occurs in the M\=/form to mark that
it is modified by the second verb, as discussed in \srf{sec:SVC},
while the final verb occurs in the U\=/form
to signal that the entire verb phrase requires more information to achieve resolution.
See example \qf{ex:160326, 5.37-5.45} on page \pageref{ex:160326, 5.37-5.45}
for more discussion of \qf{ex2:160326, 5.37} and its context.

\begin{exe}
	\ex{\glll	siin neem na-tua Koorʔoot ees reʔ oras mee \hspace{35mm} ka= ta-hi\tbr{in} t-a\tbr{na} =f.\\
						sini neem na-tua Koorʔoto esa reʔ oras mee {} ka= ta-hini t-ana =f.\\
						{\siin} {\nema} \na-settle Koro{\Q}oto{\M} {\esc} {\req} time where {} {\ka}= {\tg}-know{\tbrM} {\tg}-{\ana\tbrU} ={\fa}\\
			\glt	`They came and settled in Koro{\Q}oto, it was at a time which hasn't been figured out.'
						\txrf{160326, 5.37} {\emb{160326-05-37.mp3}{\spk{}}{\apl}}}\label{ex2:160326, 5.37}
\end{exe}

Similarly, in \qf{ex2:160326, 18.26} below the first member
of the noun phrase \ve{kaan auk-k=eni} `praise names' occurs in the M\=/form to signal
that it is modified by the following nominal,
as discussed in Chapter \ref{ch:SynMet}.
This nominal in turn occurs in the M\=/form due to the following
vowel-initial enclitic, as discussed in Chapter \ref{ch:PhoMet}.
The final member of this phrase is the number enclitic \ve{=eni} `{\ein}'
which occurs in the U\=/form to signal that the entire phrase requires
more information to achieve resolution.
See example \qf{ex:160326, 18.26} on page \pageref{ex:160326, 18.26}
for more discussion of this example.

\begin{exe}
	\ex{\glll	siin naiʔ Bain mone \sf{kusus}, \hspace{60mm} siin ka\tbr{an} a\tbr{uk}-k=e\tbr{ni} bisa, Mea aiʔ Tutun.\\
						sini naiʔ Bani mone \sf{kusus} {} sini kana aku-k=eni bisa Mea aiʔ Tutun\\
						{\siin} {\naiq} Bani{\M} male exclusive {} {\siin} name{\tbrM} praise.name{\tbrMv}-{\k=\ein\tbrU} can Mea or Tutun\\
			\glt	`Members of the Bani clan classified as male can
						exclusively have the praise names Mea or Tutun.'
						\txrf{160326, 18.26} {\emb{160326-18-26.mp3}{\spk{}}{\apl}}}\label{ex2:160326, 18.26}
\end{exe}

To summarise, syntactically driven M\=/forms only
occur with medial members of phrases, while discourse
driven U\=/forms only occur with final members of phrases.
As a result, there is no competition between them.
In a single phrase the medial members
can occur in the M\=/form to signal the internal
syntactic structure of the phrase, while the final
member can occur in the U\=/form to signal the discourse
status of the entire phrase.