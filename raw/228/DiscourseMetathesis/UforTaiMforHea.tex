\subsection{U\=/form tail with M\=/form head}\label{sec:UforTaiMforHea}
Tail-head linkage can also involve a U\=/form tail and an M\=/form head.
The structure of this construction is given in \qf{ex:THL U/M}.
In most examples the tail is followed by one of the connectors
\ve{=ma} `and' or \ve{=te} \tsc{set} `when, as' and/or the head is an obligatory
M\=/form due to a following vowel-initial enclitic (Chapters \ref{ch:PhoMet} and \srf{sec:PhoCon})
or as the first member of a serial verb construction (\srf{sec:SVC}, \srf{sec:SynDisDriMet}).

\begin{exe}
	\ex{\xytext{\fbox{event\sub{1}{\U}}\xybarconnect[2][-](D,D){2}\xybarconnect[2][->]{3}&(\ve{=ma}/\ve{=te})&\fbox{event\sub{1}{\M}}&\fbox{event\sub{2}\vp{\U}}}}\label{ex:THL U/M}
\end{exe}

A simple example is given in \qf{ex:120923-1, 6.56-6.59} below.
The tail is the U\=/form \ve{nema} `comes' in \qf{ex:120923-1, 6.56}.
This tail is picked up by the M\=/form head in \qf{ex:120923-1, 6.59 2},
which introduces an event which happens after the subject comes.

\begin{exe}
	\ex{Being healed:  \xytext{\fbox{come{\U}\vp{|}}\xybarconnect[2][-](D,D){2}\xybarconnect[2][->]{3}&when&\fbox{come{\M}\vp{|}}&\fbox{prayed away{\M}\vp{|}}}
		\txrf{120923-1} {\emb{120923-1-06-56-06-59.mp3}{\spk{}}{\apl}}}\label{ex:120923-1, 6.56-6.59}
	\begin{xlist}
		\ex{\glll	natiʔ mu-toon=ee na-hiin he ne\tbr{ma} =t,\\
							natiʔ mu-tona na-hine he nema =te\\
							careful \muu-tell{\Mv}={\eeV} \na-know{\M} {\he} {\nema\tbrU} ={\te} \\
				\glt	`Ensure you tell him so he knows that if when he comes,' \txrf{6.56}}\label{ex:120923-1, 6.56}
		\ex{\glll	ne\tbr{em} he t-ʔonen t-pasat t-aan=ee.\\
							nema he t-ʔonen t-pasat t-ana=ee\\
							{\nema\M} {\he} {\tg}-pray{Uc} {\tg}-whack.away{\Uc} {\t}-{\ana\Mv}={\eeV}\\
				\glt	`He comes to have it prayed away.'
							\txrf{6.59} }\label{ex:120923-1, 6.59 2}
	\end{xlist}
\end{exe}

A similar example is given in \qf{ex:160326, 4.57-5.05} below.
In this example the U\=/form verb \ve{n-romi} `likes' occurs in
\qf{ex:160326, 5.02} with an explanation of what is desired
introduced by the M\=/form version of this verb in \qf{ex:160326, 5.05}.

\newpage
\begin{exe}
	\ex{Naming the village Koro{\Q}oto: 
			\xytext{\fbox{like{\U}\vp{g}}\xybarconnect[2][-](D,D){1}\xybarconnect[2][->]{2}&\fbox{like{\M}\vp{g}}&\fbox{change}}
		\txrf{160326} {\emb{160326-04-57-05-05.mp3}{\spk{}}{\apl}}}\label{ex:160326, 4.57-5.05}
	\begin{xlist}
		\ex{\gll	{oka =te} siin hai beʔi naʔi siin na-bua=n =ama,\\
							after.that {\siin} {\hai} PM PF ={\siinN} \na-gather={\einV} =and\\
				\glt	`Then those ancestors (of ours) gathered and,' \txrf{4.57}}
		\ex{\gll	n-- n-ro\tbr{mi}. \\ %\tcb{\footnotemark}\\
							{} \n-like{\tbrU}\\
				\glt	\lh{n--} `(they) wanted to.' \txrf{5.02}}\label{ex:160326, 5.02}
		\ex{\glll	\hp{n--} n-ro\tbr{im} reʔ kuan=ii kaan-n=ee na-nainʔ=ee \hp{n-- }na-ʔko Haarʔoo n-eu Koorʔoot.\\
							{} n-romi reʔ kuan=ii kana-n=ee na-naniʔ=ee \hp{n-- }na-ʔko Haarʔoo n-eu Koorʔoto\\
							{} \n-like{\tbrM} {\reqt} village={\ii} name-{\N}={\ee} \na-move{\Mv}={\eeV} \hp{n-- }\na-{\qko} Haar{\Q}oo \n-{\eu} Koro{\Q}oto{\M}\\
				\glt	\lh{n--}`Wanted to change village's name from Haar{\Q}oo to Koro{\Q}oto.' \txrf{5.05}}\label{ex:160326, 5.05}
	\end{xlist}
\end{exe}

Another example is given in \qf{ex:130825-8, 1.34-1.38} below.
In this example the U\=/form \ve{u-ʔmate} `kill' is directly followed
by the M\=/form head which introduces an event which follows this action.
In this example the head is obligatorily in the M\=/form
due to a following vowel-initial enclitic.

\begin{exe}
	\ex{Trying taps: \xytext{\fbox{turn.off{\U}}\xybarconnect[2][-](D,D){1}\xybarconnect[2][->]{3}&\fbox{turn.off{\Mv}}&and&\fbox{not flow}}
		\txrf{130825-8} {\emb{130825-8-01-34-01-38.mp3}{\spk{}}{\apl}}}\label{ex:130825-8, 1.34-1.38}
	\begin{xlist}
		\ex{\glll	a|ʔ-toroʔ on reʔ ia =ma, ohh, i{\j}a oe mainikin.\\
							{\a}ʔ-toroʔ on reʔ ia =ma {} i{\j}a oe mainikin\\
							\a\q-catch.liquid{\Uc} like {\reqt} {\ia} =and {} {\ia} water cold\\
				\glt	\lh{\a}`I caught the water like this. Ohh, this one is cold water.' \txrf{1.34}}
		\ex{[audience laughs] \txrf{1.37}}
		\ex{\glll	u-ʔma\tbr{te},\\
							u-ʔmate \\
							{\qu}-kill{\tbrU}\\
				\glt	`I turned (the tap) off,'}
		\ex{\glll	u-ʔma\tbr{at\j}=ee =m ka= na-sai =fa.\\
							u-ʔmate=ee =ma ka= na-sai =fa\\
							{\qu}-kill{\tbrMv}={\eeV} =and {\ka}= {\na}-flow ={\fa}\\
				\glt	`I turned it off and it didn't flow.' \txrf{1.38}}
	\end{xlist}
\end{exe}

Speakers reject instances in which both parts of the
tail-head linkage construction are in the M\=/form.
This is shown in (\ref{ex:130825-8, 1.34-1.38}′) below,
in which the tail-head linkage construction of \qf{ex:130825-8, 1.34-1.38}
has been manipulated to have two M\=/forms.
This provides evidence that the speaker has intuitively
constructed his discourse in \qf{ex:130825-8, 1.34-1.38}
so that the M\=/form which must be an M\=/form (due to
the following vowel-initial enclitic) does not
co-occur with another M\=/form of the same verb.

\begin{exe}
	\exp{ex:130825-8, 1.34-1.38}[*]{\glll	
								u-ʔma\tbr{et}, u-ʔma\tbr{at\j}=ee =m ka= na-sai =fa \\
								u-ʔmate u-ʔmate=ee =ma ka= na-sai =fa\\
								{\qu}-kill{\tbrM} {\qu}-kill{\tbrMv}={\eeV} =and {\ka}= \na-flow ={\fa}\\
				\glt	`(I turned (it) off, I turned it off and it didn't flow.)' \xrf{elicit. 25/02/16 p.30}}
\end{exe}

In \qf{ex:130928-1, 0.30-0.33} below,
tail-head linkage serves not to introduce a subsequent event,
but rather to provide details on the manner in which the event was carried out.
In this case the tail and head are both forms of \ve{n-rame} `plasters'.
The introduced manner adverbial is \ve{reko{\tl}reko} `properly'.
Again, the head is in the M\=/form due to a following vowel-initial enclitic.


\begin{exe}
	\ex{Digging a grave:  \xytext{\fbox{plaster{\U}}\xybarconnect[2][-](D,D){1}\xybarconnect[2][->]{2}&\fbox{plaster{\Mv}}&\fbox{properly}}
		\txrf{130928-1} {\emb{130928-1-00-30-00-33.mp3}{\spk{}}{\apl}}}\label{ex:130928-1, 0.30-0.33}
	\begin{xlist}
		\ex{\glll	iin ka= n-haan\j=ee ruum=aah =fa =te,\\
							ini ka= n-hani=ee ruma=aha =fa =te\\
							{\iin} {\ka}= {\n}-dig{\Mv}={\eeV} empty=just ={\fa} ={\te}\\
				\glt	`He didn't just dig the grave emptily (with plain dirt walls).' \txrf{}}
		\ex{\glll	n-hani n-raar\j=ee =te, n-ra\tbr{me}.\\
							n-hani n-rari=ee =te, n-rame\\
							{\n}-dig{\Uc} {\n}-finish{\Mv}={\eeV} ={\te} {\n}-plaster{\tbrU}\\
				\glt	`When he finished digging it, he plastered (it).' \txrf{0.30}}
		\ex{\glll	\hspace{52.3mm} n-ra\tbr{am\j}=ee reko{\tl}reko.\\
							{} n-rame=ee reko{\tl}reko\\
							{} {\n}-plaster{\tbrMv}={\eeV} {\frd}good\\
				\glt	\hspace{52.5mm} `He plastered it properly.' \txrf{0.33}}
	\end{xlist}
\end{exe}

A version of \qf{ex:130928-1, 0.30-0.33} in which
the first half of the tail-head linkage occurs
in the M\=/form was judged strange
as shown in (\ref{ex:130928-1, 0.30-0.33}′) below.
This provides evidence the speaker has intuitively constructed
the discourse in \qf{ex:130928-1, 0.30-0.33} to achieve a pairing
of a U\=/form with an M\=/form.

The only time two M\=/forms are acceptable is when
both occur with a vowel-initial enclitic attached,
as in (\ref{ex:130928-1, 0.30-0.33}′′) in which case
the metathesis is an automatic response to the 
presence of a vowel-initiall enclitic (see Chapter \ref{ch:PhoMet}).

\begin{exe}
	\sn{Elicitation: \txrf{elicit. 09/02/16 p.9}}
	\exp{ex:130928-1, 0.30-0.33}{
		\begin{xlist}
			\ex[*]{\glll	n-hani n-raar\j=ee =te, n-ra\tbr{em}.\\
								n-hani n-rari=ee =te, n-rame\\
								{\n}-dig{\Uc} {\n}-finish{\Mv}={\eeV} ={\te} {\n}-plaster{\tbrM}\\
					\glt	`(When he finished digging it, he plastered (it).)'}
			\ex[]{\glll	\hspace{52.3mm} n-ra\tbr{am\j}=ee reko{\tl}reko.\\
								{} n-rame=ee reko{\tl}reko\\
								{} {\n}-plaster{\tbrMv}={\eeV} {\frd}good\\
					\glt	\hspace{52.5mm} `He plastered it properly.'}
		\end{xlist}}
	\exi{(\ref{ex:130928-1, 0.30-0.33}′′)}{
		\begin{xlist}
			\ex[✔]{\glll	n-hani n-raar\j=ee =te, n-ra\tbr{am\j}=ee.\\
								n-hani n-rari=ee =te, n-rame=ee\\
								{\n}-dig{\Uc} {\n}-finish{\Mv}={\eeV} ={\te} {\n}-plaster{\tbrMv=\eeV}\\
					\glt	`When he finished digging it, he plastered (it).'}
			\ex[]{\glll	\hspace{52.3mm} n-ra\tbr{am\j}=ee reko{\tl}reko.\\
								{} n-rame=ee reko{\tl}reko\\
								{} {\n}-plaster{\tbrMv}={\eeV} {\frd}good\\
					\glt	\hspace{52.5mm} `He plastered it properly.'}
		\end{xlist}}
\end{exe}

A tail-head linkage construction in Amarasi usually
has two identical verbs which differ in the U\=/form or M\=/form.
A U\=/form tail is complemented by an M\=/form head
and a U\=/form head is paralleled by an M\=/form tail.
Speakers intuitively construct their discourse in such
a way as to achieve a pairing of a U\=/form with an M\=/form.
One way to do this is by forcing the head to be in the M\=/form
with a vowel-initial enclitic and having the tail in the U\=/form.

\subsection{U\=/form tail with U\=/form head}\label{sec:UforTaiUforHea}
There are eleven examples in my corpus of tail-head linkage
in which both the tail and the head occur in the U\=/form.
On the face of it, this is a highly unexpected structure
as U\=/forms canonically require an M\=/form to achieve resolution.
However, a closer look reveals that in each instance one of the verbs is in the U\=/form due to
other factors, such as occurring before a consonant cluster
or being part of another tail-head linkage construction.

One example is given in \qf{ex:130920-1, 3.32-3.34} below.
In this example the tail \ve{m-resa} is in the U\=/form
due to the following consonant cluster, and is thus glossed `{\Uc}' (\srf{sec:VerBefCC}).
The head is also in the U\=/form as it is introducing the next event,
which provides its resolution.

\begin{exe}
	\ex{Proofreading Bible translations: \xytext{\fbox{read{\Uc}}\xybarconnect[2][-](D,D){1}&\fbox{read{\U}\vp{\Uc}}\xybarconnect[2][->]{2}&and&\fbox{ask\vp{\Uc}}}
		\txrf{130920-1} {\emb{130920-1-03-32-03-34.mp3}{\spk{}}{\apl}}}\label{ex:130920-1, 3.32-3.34}
	\begin{xlist}
		\ex{\glll	\sf{bukan} hai m-re\tbr{sa} n-mees.\\
							\sf{bukan} hai m-resa n-mese\\
							\tsc{neg} {\hai} \m-read{\tbrUc} \n-alone\\
				\glt	`We didn't read (it) by itself.' \txrf{3.32}}
		\ex{\glll	hai m-re\tbr{sa} =ma, hai m-mak-tana=n mi-knuutʔ=ee.\\
							hai m-resa =ma hai m-mak-tana=n mi-knutuʔ=ee\\
							{\hai} \m-read{\tbrU} =and {\hai} \m-\mak-ask={\einV} \mi-fine{\Mv}={\eeV}\\
				\glt	`We read and we asked one another (about it) to refine it.' \txrf{3.34}}
	\end{xlist}
\end{exe}

A very similar example is given in \qf{130907-4, 3.21} below.
In this case the tail (\ve{nema} `comes') of the tail-head linkage construction
occurs immediately before a consonant cluster.
This consonant cluster is also the first
verb of the serial verb construction which contains the head
and introduces a new event.

\begin{exe}
	\ex{\xytext{\fbox{come{\Uc}}\xybarconnect[2][-](D,D){1}&\fbox{come{\U}\vp{\Uc}}\xybarconnect[2][->]{2}&when&\fbox{tell{\M}\vp{\Uc}}}}
	\sn{\glll	{onai =m} mes ne\tbr{ma} n-fain ne\tbr{ma} =t, na-toon =kau =ma\\
						{onai =ma} mes nema n-fani nema =te na-tona =kau =ma\\
						and.so but {\nema\tbrUc} \n-back{\M} {\nema\tbrU} ={\te} \na-tell{\M} ={\kau} =and\\
			\glt	`But so (he) came, when he came back he told me:'
						\txrf{130907-4, 3.21} {\emb{130907-4-03-21.mp3}{\spk{}}{\apl}}}\label{130907-4, 3.21}
\end{exe}

Example \qf{ex:130928-1, 0.43-0.55} below is slightly different.
In this example the tail-head linkage construction involves
two parallel verbs (\srf{sec:ParVer}).
The first verb (\ve{t-pafaʔ} `protect') is consonant final,
and thus occurs in the U\=/form and is glossed `{\Uc}' (\srf{sec:ConFinVer}).
The head of the construction then occurs in the U\=/form
to introduce the elaboration; \ve{on reʔ mee} `in which way, how'.

\newpage
\begin{exe}
	\ex{Burying a dead person: \xytext{\fbox{protect{\Uc}}\xybarconnect[2][-](D,D){2}&or&\fbox{bury{\U}\vp{\Uc}}\xybarconnect[2][->]{2}&and&\fbox{how\vp{\Uc}}}
		\txrf{130928-1} {\emb{130928-1-00-43-00-55.mp3}{\spk{}}{\apl}}}\label{ex:130928-1, 0.43-0.55}
	\begin{xlist}
		\ex{\gll	areʔ amahonit, anaʔaprenat, too mfaun=eni \hp{{\nema}} \hp{\na-gather={\einV}} neem na-bua=n =am\\
							every elder official citizen many={\ein} {} {} {\nema} \na-gather={\einV} =and\\
				\glt	`All the clan elders, officials, many people came and gathered' \txrf{0.43}}
		\ex{\gll	he na-ʔuab=ein n-eu reʔ he\\
							{\he} \na-speak={\einV} {\n-\eu} {\reqt} {\he}\\
				\glt	`to talk about' \txrf{0.48}}
		\ex{\gll	a|t-pa\tbr{faʔ} aiʔ t-su\tbr{ba} =ma, on reʔ mee.\\
							\a\t-protect{\tbrUc} or bury{\tbrU} =and like {\reqt} how\\
				\glt	\lh{\a}`the way in which he should be protected or buried.' \txrf{0.55}}
	\end{xlist}
\end{exe}

Most of the remaining examples of tail-head linkage
with both a U\=/form tail and U\=/form head are examples
in which the head is itself a tail for an anaphoric tail-head linkage
construction with the verb \ve{rari} `finish'.
One of these examples has already been given in \qf{ex:130825-8, 1.06-1.13},
the relevant part of which is repeated as \qf{ex:130825-8, 1.06-1.13-2} below.
In \qf{ex:130825-8, 1.06-1.13-2} the initial M\=/form \ve{ʔ-iiŋgw} `drink'
is the tail of a tail-head linkage construction with following \mbox{\ve{ʔ-inu}} `drink'
which is the tail of a tail-head linkage construction with following \ve{u-rari} `finish',
which is resolved by the following clause. %\ve{ʔ-taam ʔ-ai kraan=ii} `entered (and) turned on the tap'.


\begin{exe}
	\ex{Exploring a hotel room: \txrf{130825-8} {\emb{130825-8-01-06-01-10.mp3}{\spk{}}{\apl}}}\label{ex:130825-8, 1.06-1.13-2}
	\sn{\xytext{\fbox{drink{\Mv}\vp{|}}\xybarconnect[2][-](D,D){1}&\fbox{drink{\U}\vp{|}}\xybarconnect[2][->]{2}&and&\fbox{finish{\U}\vp{|}}\xybarconnect[2][->](D,D){2}&when&\fbox{enter turn.on tap\vp{|}}}}
	\begin{xlist}
		\ex{\glll	ʔ-took ʔ-oka bruuk=ii =m ʔ-ait biir \sf{kaleŋ} \hspace{5mm} =siin =m ʔ-i\tbr{iŋgw}=een\\
							ʔ-toko ʔ-oka bruuk=ii =ma ʔ-aiti biir \sf{kaleŋ} {} =sini =ma ʔ-inu=ena\\
							{\q}-sit{\M} {\q}-{\ok\Uc} pants={\ii} =and {\q}-pick.up{\M} beer can {} ={\siinN} =and {\q}-drink{\tbrMv}={\een}\\
				\glt	`I sat down in the pants, picked up some beer cans and drank.' \txrf{1.06}}
		\ex{\glll	ʔ-i\tbr{nu} =m u-ra\tbr{ri} =t, a|ʔ-taam ʔ-ai kraan=ii,\\
							ʔ-inu =ma u-rari =te {\a}ʔ-tama ʔ-ai kraan=ii\\
							{\q}-drink{\tbrU} =and {\qu}-finish{\tbrU} ={\te} \a\q-enter{\M} {\q}-push tap={\ii}\\
				\glt	`I drank and when I finished, I went in and turned on the tap,' \txrf{1.10}}
		%\ex{\glll	ʔ-i\tbr{nu} =m \\
							%ʔ-inu =ma \\
							%{\q}-drink{\tbrU} =and \\
				%\glt	`I drank and' }
		%\ex{\glll	u-ra\tbr{ri} =t, a|ʔ-taam ʔ-ai kraan=ii,\\
							%u-rari =te {\a}ʔ-tama ʔ-ai kraan=ii\\
							%{\qu}-finish{\tbrU} ={\te} \a\q-enter{\M} {\q}-push tap={\ii}\\
				%\glt	`when I finished, I went in and turned on the tap,' \txrf{1.10}}
	\end{xlist}
\end{exe}
