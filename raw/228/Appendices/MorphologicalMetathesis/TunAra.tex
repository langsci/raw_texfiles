
\section{Tunisian Arabic}\label{sec:TunAra}
Metathesis in Tunisian Arabic is described by \cite{kidr86}.
Their discussion begins with the observation that Tunisian Arabic has a process of
phonologically conditioned metathesis (\srf{sec:PhoMet}),
in which the medial CV sequence of a CCVC stem metathesises
before a vowel-initial suffix.
Examples are given in \qf{ex:TunAraMorMet} below.

\begin{exe}
	\ex{CC\sub{2}V\sub{1}C {\ra} CV\sub{1}C\sub{2}C /{\gap}-V \hfill\cite[61]{kidr86}}\label{ex:TunAraMorMet}
	\sn{\gw\begin{tabular}{llcll}
								&Stem							&			&\mc{2}{l}{Suffixed Form}\\
		`palms'			&\it{n\tbr{xa}l}	&{\ra}&\it{n\tbr{ax}l-a}	&`a palm' \\
		`mountain'	&\it{\j\tbr{bə}l}	&{\ra}&\it{\j\tbr{əb}l-i}	&`mountains' \\
		`he wrote'	&\it{k\tbr{tə}b}	&{\ra}&\it{k\tbr{ət}b-u}	&`they wrote' \\
		`month'			&\it{ʃ\tbr{ħa}r}	&{\ra}&\it{ʃ\tbr{aħ}r-iːn}&`two months' \\	
	\end{tabular}}		
\end{exe}

However, there are a number of verbs in which CV {\ra} VC metathesis
alone results in a nominalisation, producing what \citeauthor{kidr86}
call a \it{nomen actionis} (action noun).
Such metathesis only affects words of the shape CCVC.
Examples are given in \qf{ex:TunAraDerMet} below.
%(There is also one example of a metathesis with concurrent apophony of the vowel;
%\it{ħ\tbr{rə}m} `he prohibited' {\ra} \it{ħ\tbr{ar}m} `prohibition'.)

\begin{exe}
	\ex{Nominalising metathesis \hfill\citep[62]{kidr86}}\label{ex:TunAraDerMet}
	\sn{\gw\begin{tabular}{llcll}
										&Verb							&			&Noun&\\
		`he understood'	&\it{f\tbr{hə}m}	&{\ra}&\it{f\tbr{əh}m}		&`understanding'\\
		`he was sick'		&\it{m\tbr{rˁa}ðˁ}	&{\ra}&\it{m\tbr{arˁ}ðˁ}	&`sickness'\\
		`he owned'			&\it{m\tbr{lə}k}		&{\ra}&\it{m\tbr{əl}k}		&`asset'\\
		`he lied'				&\it{k\tbr{ðə}b}		&{\ra}&\it{k\tbr{əð}b}		&`lying'\\
		`he tightened'	&\it{ħ\tbr{sˁa}rˁ}	&{\ra}&\it{ħ\tbr{asˁ}rˁ}	&`act of tightening'\\
		`he blasphemed'	&\it{k\tbr{fɔ}r}		&{\ra}&\it{k\tbr{ɔf}r}		&`blasphemy'\\
		`he prohibited'	&\it{ħ\tbr{rə}m}		&{\ra}&\it{ħ\tbr{ar}m}		&`prohibition'\\
	\end{tabular}}		
\end{exe}

An alternate analysis of the same data would be to identify the nouns as the base
from which verbs are derived by VC {\ra} CV metathesis.
\cite{kidr86} adduce both diachronic evidence
as well as native speaker judgements in favour of their
analysis of metathesis as a nominaliser.

Metathesis is only one of a number of nominalisation strategies in Tunisian Arabic.
Another nominalisation strategy is affixation.
Nominalising affixes include \it{-aːn}, \it{-(j)a},
\it{m(a)-} or a combination of \it{m-{\ldots}-a}.
Examples are given in \qf{ex:TunAraAff} below.
Suffixation with a vowel-initial suffix
also triggers phonologically conditioned metathesis of CCVC roots,
as seen in \qf{ex:TunAraMorMet} above.

\begin{exe}
	\ex{Nominalising affixation \hfill\citep[63f]{kidr86}}\label{ex:TunAraAff}
	\sn{\gw\begin{tabular}{llcll}
										&Verb					&			&Noun										&\\
		`he attached'		&\it{rˁbatˁ}	&{\ra}&\it{rˁabtˁ-\tbr{a}}		&`act of attaching' \\
		`he read'				&\it{qra}			&{\ra}&\it{qraː-\tbr{ja}}			&`reading' \\
		`he blasphemed'	&\it{kfɔr}		&{\ra}&\it{kɔfr-\tbr{aːn}}		&`blasphemy'\footnotemark\\
		`he asked'			&\it{tˁləb}		&{\ra}&\it{\tbr{ma}-tˁləb}		&`request' \\
		`he loved'			&\it{ħabb}		&{\ra}&\it{\tbr{m}-ħabb-\tbr{a}}	&`act of loving' \\
	\end{tabular}}		
\end{exe}
\footnotetext{
Some verbs have multiple nominalising strategies.
The verb \it{kfɔr} `blaspheme' is one such example, either undergoing metathesis,
as shown in \qf{ex:TunAraDerMet}, or suffixation, as shown here in \qf{ex:TunAraAff}.}

Another nominalisation strategy is apophony,
either replacing a short vowel with the equivalent long vowel
or replacing it with a vowel of a different quality.
Examples are given in \qf{ex:TunAraApo} below.

\begin{exe}
	\ex{Nominalising apophony \hfill\citep[64]{kidr86}}\label{ex:TunAraApo}
	\sn{\gw\begin{tabular}{llcll}
									&Verb			&			&Noun				&\\
		`he slept'		&\it{rq\tbr{a}d}&{\ra}&\it{rq\tbr{aː}d}	&`sleep' \\
		`he went mad'	&\it{xb\tbr{ə}l}&{\ra}&\it{xb\tbr{aː}l}	&`going mad' \\
		`he entered'	&\it{dx\tbr{ə}l}&{\ra}&\it{dx\tbr{uː}l}	&`act of entering' \\
		`he swam'			&\it{ʕ\tbr{aː}m}&{\ra}&\it{ʕ\tbr{uː}m}	&`swimming'\\
		`he sold'			&\it{b\tbr{aː}ʕ}&{\ra}&\it{b\tbr{iː}ʕ}	&`(a) sale' \\
	\end{tabular}}		
\end{exe}

The final nominalisation strategy is zero derivation;
that is conversion of a verb into a noun with no phonological change.
Examples are given in \qf{ex:TunAraZerDer} below.

\begin{exe}
	\ex{Zero derivation \hfill\citep[63,65]{kidr86}}\label{ex:TunAraZerDer}
	\sn{\gw\begin{tabular}{lll}
		\it{ʕməl} 	& `he did' {\tl} `deed' \\
		\it{ʕtˁasˁ}	& `sneeze' {\tl} `act of sneezing' \\
		\it{nðˁar} 	& `he saw' {\tl} `seeing' \\
		\it{xbar}		& `he informed' {\tl} `informing' \\
	\end{tabular}}		
\end{exe}

\citet[71]{kidr86} carried out two tests to determine
how productive each of these nominalisation strategies were for CCVC verbs.
In each case metathesis was the most productive nominalisation strategy.

In the first test speakers were presented with ten fictional verbs
and a variety of nominalisations formed according
to each possible process illustrated in \qf{ex:TunAraDerMet}--\qf{ex:TunAraZerDer} above.
Metathesis was the preferred strategy in 8/10 instances
in the first run of this test and was preferred in 9/10 instances in the second run.\footnote{
		The other acceptable nominalisation strategy was suffixation with \it{-aːn}.
		There were also two responses in which either metathesis or suffixation with \it{-aːn}
		were judged acceptable.}

Similar judgements were given for the loan words
\it{nmar} (< French \it{numéroter}) `to number',
\it{mraʃ} (< French \it{marcher}) `to march'
and \it{mrəs} (< French \it{remercier}) `to thank'.
Among loanwords the only exception was \it{bləf} < English \it{bluff},
for which the preferred nominalisation strategy was zero derivation.

The second test \citeauthor{kidr86} carried out
involved choosing either metathesis or zero derivation
as the preferred nominalisation strategy.
In this test 17/18 responses selected metathesis.

In summary, metathesis in Tunisian Arabic is one of several processes
available to nominalise verbs with the structure CCVC.
Metathesis is productive and is the preferred nominalisation strategy.
That metathesis in Tunisian Arabic is associated with other processes
is consistent with the data in Chapter \ref{ch:SynchMet}
in which metathesis is associated with
a large number of additional processes.
%However, in Amarasi most of these processes are consequences of metathesis,
%such as vowel assimilation triggered by metathesis (\srf{sec:VowAss}),
%while in Tunisian Arabic these processes are independent of metathesis,
%with the exception of suffixation triggering metathesis.
