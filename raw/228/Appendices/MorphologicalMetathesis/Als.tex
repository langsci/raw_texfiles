\section{Alsea}\label{sec:Als}
Alsea is a now extinct language of the Oregon coast (see \frf{fig:LanWesAmeMorMet}).
The only consonants which participate in metathesis in Alsea are sonorants.
Metathesis in Alsea is mostly morphemically conditioned (\srf{sec:MorpheConMet}).
One suffix which triggers metathesis is the third person object imperative suffix \it{-t}.
Examples are given in \qf{ex:AlsMorphemicMet} below
in which the metathesised stems on the right
can be compared with unmetathesised counterparts on the left.

\begin{exe}
	\ex{Alsea morphemically conditioned metathesis \hfill\citep[8f]{bu07}}\label{ex:AlsMorphemicMet}
		\sn{\stl{0.3em}\gw\begin{tabular}{lrll}
		`had closed it'				&\it{t\tbr{mú}s-sa-nχ}		&\it{t\tbr{úm}s-t}		&`close it!'\\
		`agreed to it'				&\it{t'\tbr{má}s-sal-tχ}	&\it{t'\tbr{ám}s-t}		&`finish it!'\\
		`had been sliding'		&\it{st\tbr{lá}k-sal-tχ}	&\it{st\tbr{ál}k-t}		&`slide it!'\\
		`is packing'					&\it{ʦu\tbr{lá}q'n-tχ}		&\it{ʦu\tbr{ál}q'n-t}	&`pack it!'\\
		`is close to shore'		&\it{t\tbr{lú}qʷ'-χ}			&\it{t\tbr{úl}qʷ'-t}	&`bring it close to shore!'\\
		`is in act of hiding'	&\it{p\tbr{yá}χ-aw-tχ}		&\it{p\tbr{áy}X-t}		&`hide it!'\\
		`had pierced'					&\it{qɬ\tbr{jú}t-sal}			&\it{qɬ\tbr{új}-t}		&`prick him!'\\
	\end{tabular}}		
\end{exe}

That this metathesis is not conditioned by the phonological shape of the suffix
is shown by the contrast between suffixes with an identical form,
one of which triggers metathesis while the other does not.
The intransitive imperative suffix \it{-χ} triggers metathesis
while the completive realis \it{-χ} suffix does not trigger metathesis.
Examples are given in \qf{ex:AlsMorphemicMet2} below.

\begin{exe}
	\ex{Alsea morphemically conditioned metathesis \hfill\citep[8f]{bu07}}\label{ex:AlsMorphemicMet2}
	\sn{\gw\begin{tabular}{lrll}
											&\tsc{cmpl.rl}			&\tsc{intr.imp}		& \\
		`dances with them'&\it{k\tbr{ná}χ-χ}	&\it{k\tbr{án}χ-χ}&`dance with them!' \\
		`are lying in bed'&\it{ʦ\tbr{nú}s-χ}	&\it{ʦ\tbr{ún}s-χ}&`lie down!' \\
		`is hiding'				&\it{p\tbr{já}χ-χ}	&\it{p\tbr{áj}χ-χ}&`hide!' \\
		`is floating'			&\it{ʦp\tbr{jú}t-χ}	&\it{ʦp\tbr{új}t-χ}&`float!' \\
	\end{tabular}}		
\end{exe}

In addition to such morphemically conditioned metathesis there are also hints that
Alsea had a process of morphological metathesis which signalled aspect.
\cite{bu07} gives three potential examples, given in \qf{ex:AlsMorMet} below.

\begin{exe}
	\ex{Alsea morphological metathesis \hfill\citep[10]{bu07}}\label{ex:AlsMorMet}
	\sn{\stl{0.4em}\gw\begin{tabular}{lrll}
		`keep it shut!'		&\it{t\tbr{mú}s-t}			&\it{t\tbr{úm}s-t}			&`shut it!' \\
		`is stretched out'&\it{ʦɬ\tbr{já}q-tχ}		&\it{ʦɬ\tbr{áj}q-tχ}		&`made it straight' \\
		`was (not) 				&\it{ʦqʷ\tbr{ná}qʷ-ɬn-χ}&\it{ʦqʷ\tbr{án}qʷ-ɬn-χ}&`was being \\ \hhline{~}
		\hp{`}overtaken'	&												&												&\hp{`}overtaken' \\
	\end{tabular}}		
\end{exe}

However, such examples come only from elicitation
with no indication of the context in which they could be used.
Nonetheless, given the (historic) location of Alsea,
bordering on the area in which Salishan languages are spoken (see \frf{fig:LanWesAmeMorMet}),
it would not be surprising if Alsea had 
also developed a morphological process of morphological metathesis to mark aspect.\footnote{
		Alsea is not considered genealogically related to the Salishan languages.}