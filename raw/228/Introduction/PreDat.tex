\section{Presentation of data and notational conventions}\label{sec:PreDat}
Data from Amarasi, or another variety of Meto,
is transcribed phonemically and presented in italic font.\footnote{
		There are only three non-phonemic aspects of my transcription.
		Firstly, foreign proper names are transcribed orthographically
		when they contain non-native phonemes the IPA representation
		of which is not identical to their orthographic, e.g. \ve{Lince} [li{\ny\tS}e].
		Secondly, /ɡw/ is transcribed \it{<g>} before rounded vowels (\srf{sec:VoiObs}).
		Thirdly, /n/ {\ra} [ŋ] is transcribed \it{<ng>} when it occurs before /ɡw/
		without an intervening morpheme break.
		These last two non-phonemic conventions
		can be seen in the word for `teacher',
		which according to my analysis has the form /tunɡwuru/,
		but is transcribed as \ve{tuŋguru}.}
Example sentences are given with up to two gloss lines.
A typical example is given in \qf{ex:120715-4, 0.55 ch:Intr} below.

\begin{exe}
	\ex{\glll	\sf{ahir{\ny}a} ahh, n-aim naan baar\j=esa =m na-maikaʔ n--\\
						\sf{ahir{\ny}a} {} n-ami naan bare=esa =ma na-maikaʔ {}\\
						in.the.end {} 3-look.for{\M} {\naan} place{\Mv}={\es} =and {\na}-settle {}\\
			\glt	`In the end, he looked there for a place and settled.'
						\txrf{120715-4, 0.55} {\emb{120715-4-00-55.mp3}{\spk{}}{\apl}}}\label{ex:120715-4, 0.55 ch:Intr}
\end{exe}

The first line is the phonemic transcription with morpheme breaks indicated.
Affixes are separated by the hyphen -.
Enclitics are separated from their host by the equals sign =.
Vowel initial enclitics which induce morphophonemic processes on their
host (Chapter \ref{ch:PhoMet}), are attached directly to the host,
while other enclitics are offset.
An example of each kind of enclitic can be seen in \qf{ex:120715-4, 0.55 ch:Intr}
with vowel initial \ve{=esa} `one' and consonant initial \ve{=m} `and'.

Word-initial epenthetic /a/ is separated by the vertical line |.
The underscore {\gap} is used to separate two parts of a phrase with
a non-compositional meaning or phrases where
one element does not occur independently.
An example of epenthesis occurs in \ve{a|n-kobub}
`piled up' in \qf{ex:120715-4, 0.05 ch:Int} below,
and an example of a non-compositional phrase is
\ve{paha{\gap}ʔpinan} `country{\gap}below' = `world'
in \qf{ex:120715-4, 0.05 ch:Int}.

Instances of Indonesian/Kupang Malay code-switching or unassimilated loans 
are transcribed in a sans-serif typeface.
Thus, in example \qf{ex:120715-4, 0.55 ch:Intr} the word
\ve{\sf{ahir{\ny}a}} `in the end' is from Kupang Malay \it{ahirnya}.
Phonetic strings which are pauses are indicated by a final \it{<hh>} and are usually unglossed.
In example \qf{ex:120715-4, 0.55 ch:Intr} \ve{ahh} is a pause
with the phonetic quality approximating [aːː],
similarly \ve{nehh} is a pause which sounds like [nɛːː].
False starts are not glossed and indicated by a final en-dash --.
One example is the final \ve{n--} in example \qf{ex:120715-4, 0.55 ch:Intr} above.
Commas indicate pauses and/or intonation breaks
and full stops represent the end of an intonation unit.
Capital letters are only used for proper names.

The second line gives the underlying form
of morphemes before processes of metathesis, consonant insertion, and vowel assimilation occur.
It also gives the underlying forms of enclitics which have multiple forms (\srf{sec:SenEnc}).
The third line gives the morpheme by morpheme gloss.
When a morpheme is ambiguous between several values,
these values are separated by a slash /. An example
is the verbal agreement prefix \ve{m-} `{\m}' which
agrees with first person exclusive,
second person singular, and second person plural.
Glosses mostly follow the Leipzig Glossing Rules
with a full list of glosses used in this book,
including non-standard glosses, given beginning on \prf{ch:Abb}.

\begin{table}[ht]
	\caption{Glosses for U-forms and M-forms}\label{tab:GloUfoMfo}
		\begin{tabular}{ll}
			\lsptoprule
						Gloss & Use \\ \midrule
				\tsc{u}		& U\=/form \\
				%\tsc{u\shiftleft{1.3pt}{\scalebox{2.0}{ͨ}}}		& 1. U\=/form of consonant-final stem\\
				\tsc{u\raisebox{-4pt}{\scalebox{1.75}{ͨ}}}		& 1. U\=/form of consonant-final stem\\
									& 2. U\=/form before consonant cluster\\
				\tsc{m}		& M\=/form\\
				\tsc{m\shiftleft{0.8pt}{̿}}		& M\=/form before vowel-initial enclitic \\
				\tsc{m\shiftleft{0.25pt}{\raisebox{-4pt}{\scalebox{1.75}{ͨ}}}}		& M\=/form before consonant cluster\\
			\lspbottomrule
		\end{tabular}
\end{table}

Glosses indicating U\=/forms and M\=/forms
are usually only given when potentially relevant to the discussion at hand.
Glosses for U\=/forms and M\=/forms in different phonotactic environments are given in \trf{tab:GloUfoMfo},
with a number of examples given in \qf{ex:120715-4, 0.05 ch:Int}--\qf{130821-1, 6.20} below.
See Chapter \ref{ch:StrMetAma} for more discussion of the distribution
of each of these forms.
Glosses for U\=/forms or M\=/forms are not given when a form
does not distinguish between them.

\begin{exe}
	\ex{\glll	neno naa paha{\gap}ʔpina-n ia, a|n-kobub on bare meseʔ\\
						neno naa paha{\gap}ʔpina-n ia	{\a}n-kobub on bare meseʔ\\
						day{\tbrU} {\naa} land{\gap}below{\tbrU}-{\N} {\ia} {\a\n}-pile{\tbrUc} {\on} place{\tbrU} one\\
			\glt	`In those days the world was piled up in one place.'
						\txrf{120715-4, 0.05} {\emb{120715-4-00-05.mp3}{\spk{}}{\apl}}}\label{ex:120715-4, 0.05 ch:Int}
	\ex{\glll	uma ʔ-tee =ma, ʔ-aiti bruuk.	\\
						uma ʔ-tea =ma ʔ-aiti bruuk \\
						{\uma\tbrUc} \q-arrive =and \q-pick.up{\tbrUc} pants{\U}	\\
			\glt	`I arrived (home) and picked up some pants.'
						\txrf{130825-6, 10.05} {\emb{130825-6-10-05.mp3}{\spk{}}{\apl}}}\label{130825-6, 10.05 ch:Int}
	\ex{\glll	hii m-euk siis\j=ii =m	\\
						hii m-eku sisi=ii =ma\\
						{\hii} \m-eat{\tbrM} meat{\tbrMv}={\ii} =and	\\
			\glt	`You ate the meat and' \txrf{120923-1, 6.01} {\emb{120923-1-06-01.mp3}{\spk{}}{\apl}}}\label{120923-1, 6.01}
	\ex{\glll	afi{\gap}naa au ʔ-tae iin sura srainʔ=ii =t	\\
						afi{\gap}naa au ʔ-tae ini surat sraniʔ=ii =te	\\
						yesterday {\au} ʔ-look.down {\iin} paper{\tbrMc} baptism{\tbrMv}={\ii} ={\te}\\
			\glt	`Yesterday when I looked at her baptismal certificate,'
						\txrf{130821-1, 6.20} {\emb{130821-1-06-20.mp3}{\spk{}}{\apl}}}\label{130821-1, 6.20}
\end{exe}

Gloss lines are followed by a free translation into English.
Words not present in the Amarasi example
but supplied in the free translation to increase
its naturalness are enclosed in brackets ().
Important para-linguistic information such as gestures
are described in square brackets [] in the free translation.
Occasionally a literal translation of part or all of the Amarasi example is given.
Literal translations are enclosed in brackets and preceded by the abbreviation `\emph{lit.}'.

The numeric code to the right of the free translation
is a reference to which text the example comes from.
These codes follow the format \emph{yy-mm-dd-no., time in text}.
Thus, the code {\ttfamily 120715-4, 0.55} in example \qf{ex:120715-4, 0.55 ch:Intr} above
indicates that this example begins at about 55 seconds
into the fourth recording made on the 15/07/2012.

Examples with the speaker icon \spk{}
have an accompanying sound file.
These
sound files can be downloaded from TROLLing (The Tromsø Repository of Lan-
guage and Linguistics) at \url{https://doi.org/10.18710/IORWF6}  \citep{ed20}. These
sound files are organised in the repository according to the chapter in which they
occur with additional information on their specific location, such as example or
table number, embedded in the file name. See the ReadMe in the TROLLing repos-
itory for a complete explanation.


In addition to examples from my text collection,
three other kinds of examples occur.
Firstly, data which was encountered during the course
of my fieldwork but not recorded is indicated as \emph{observation}
usually with the date and page reference to my notebook; e.g. {\ttfamily observation 09/10/14, p.113}.
Secondly, data which were collected during elicitation are marked as \emph{elicit.}
with the date and page reference to my notebook; e.g. {\ttfamily elicit. 15/03/2016 p.47}
Finally, data from the Amarasi Bible translation are referenced
by book, chapter, and verse, e.g. {\ttfamily John 3:16}.

When longer examples from a single text are given,
a short description usually precedes the text
(followed by the unique code cross referencing the text).
The data following this title is then labelled alphabetically.
An example is given in \qf{ex:120715-4, 0.43-0.45 ch:Intr} below.
When an example involves more than one speaker,
different speakers are indicated with Greek letters.

\largerpage
\begin{exe}
	\ex{How Moo{\Q}-hitu made the world:
				\txrf{120715-4} {\emb{120715-4-00-43-00-45.mp3}{\spk{}}{\apl}}}\label{ex:120715-4, 0.43-0.45 ch:Intr}
	\begin{xlist}
		\ex{\glll	n-bi{\tl}bi oo\j=ee naan-n=ee {onai =te},\\
							n-bi{\tl}bi oe=ee nana-n=ee {onai =te}\\
							\n-{\prd}{\bi} water={\ee} inside-{\N=\ee} and.then\\
				\glt	`Having been in the water for a while,' \txrf{0.43}}\label{ex:120715-4, 0.43 ch:Intr}
\clearpage
		\ex{\glll	a|n-moʔe =ma n-poo\j=ena n-bi metoʔ.\\
							{\a}n-moʔe =ma n-poi=ena n-bi metoʔ\\
							{\a\n}-make =and \n-exit={\een} {\n}-{\bi} dry\\
				\glt	\lh{a|}`(he) made and went out onto dry land.' \txrf{0.45}}\label{ex:120715-4, 0.45 ch:Intr}
	\end{xlist}
\end{exe}

When data on languages other than Amarasi or Meto is cited,
such data is transcribed in italics phonemically according to IPA conventions.\footnote{
		For the sake of complete clarity, the palatal glide /j/
		is always transcribed \it{<j>} while the 
		palatal affricate /\j/ is always transcribed \it{<\j>}.}
Data from languages with a widely used standard
orthography are usually transcribed orthographically followed
by a phonemic IPA transcription, an example is English \it{mouse} /maʊs/.
