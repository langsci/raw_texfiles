\section{Goals and the use of theory}\label{sec:GoaUseThe}
The main goal of this book is to present an accurate
description of the forms and functions of metathesis in Amarasi
(chapters \ref{ch:StrMetAma}--\ref{ch:DisMet}).
A secondary goal is to propose a clear analysis of the data.
A third goal to situate the Amarasi data within
its typological, geographical, and cultural context
(Chapters \ref{ch:SynchMet} and \ref{ch:ConCon})

Notably, it is \emph{not} the main goal of this book to
present the Amarasi data as an argument in
favour of any particular theoretical model.
While I make frequent use of representations and
tools from different theoretical models,
I do so mainly to illustrate clearly
aspects of the Amarasi data in a helpful way
and as explicit strategy to summarise certain generalisations.

Thus, in Chapters \ref{ch:StrMetAma} and \ref{ch:PhoMet}
I make use of Autosegmental theory as it helpfully
illustrates the processes which occur in the derivation of M\=/forms from U\=/forms.
Similarly, in describing M\=/forms before consonant clusters
\srf{sec:CCIniMod} I make use of Optimality Theory
as the tableaux of this theory illustrate well the large
number of potential outputs a particular string could generate.
Likewise, in Chapter \ref{ch:SynMet} I make use of X-bar theory to analyse
the role of metathesis within the syntax.

In general, different theoretical models
and the analyses these entail are deployed in this book
in an expedient manner according to what seems most
illuminating for the Amarasi data.
The primary use of theory is to present a clear and simple
analysis of Amarasi metathesis,
not a theoretically consistent analysis.
Thus, the observant reader will note, for instance,
that in my account of phonologically conditioned metathesis
in Chapter \ref{ch:PhoMet} I make frequent use of
constraints developed within Optimality Theory
without ever presenting an Optimality Theory tableau.
While I find some Optimality Theory constraints helpful
in understanding the data, an actual account embedded within Optimality Theory
clouds rather than illuminates the description.\footnote{
		This is not to say that Optimality Theory is wrong,
		or that it cannot or should not be used to analyse Amarasi metathesis.
		Instead, I merely do not find a full Optimality Theory account
		of this aspect of Amarasi metathesis a helpful aid.}

The main exception to this approach is in the analysis
of the structure of metathesis in Chapter \ref{ch:StrMetAma}.
In this chapter I explicitly formulate an analysis
using an autosegmental model of phonology \citep{go76}
and a rule-based model of process morphology \citep{ma74,an92}.
I do this because these models allow me to propose a
unified analysis of the form of Amarasi metathesis.

However, my primary commitment is not to any particular theory,
or any particular analysis, but to the Amarasi data itself.
I would welcome criticism of the analyses
proposed in this book so long as any alternate analyses
remain faithful to the primary data upon which any analysis must be based.
Similarly, I would welcome any dialogue with this book
which attempts to provide a unified theoretical account of all of the data.