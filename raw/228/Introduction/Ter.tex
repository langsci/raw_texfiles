\section{Terminology}\label{sec:Ter}
In this section I give definitions of potentially ambiguous linguistic terminology.
The definitions given here should be taken only as a practical guide
to understand how terms are used in this book
and should \emph{not} be taken as strong claims
about the theoretical status of any of the elements defined.

As used in this book, a \emph{word} is the minimal meaningful
phonological string which can occur in isolation.\footnote{
		Two typical environments in which words occur in isolation are
		in response to a question or in collection of a wordlist.
		Likewise, pauses are not usually allowed in the middle of a word.
		If such a pause occurs, the speaker usually repeats the entire word from the beginning.}
A \emph{morpheme} is ``an indivisible stretch of phonetic (or phonological)
material with a unitary meaning'' \citep[49]{an92}.\footnote{
		In many morphological theories the morpheme does not play a central role,
		including \cite{ma74,an92} and \cite{st01}.
		While I am extremely sympathetic to such theories,
		the morpheme is still a useful analytic tool for much of the Amarasi data.}
A \emph{root} is an underlying single morpheme without any affixes attached.

We can furthermore distinguish between \emph{bound} and \emph{free morphemes}.
A free morpheme is a root which can occur as a word without any other morphemes attached.
A typical example is \ve{kaut} `papaya'.
A bound morpheme is a root which cannot occur as a word.
Instead a bound morpheme must surface attached to another morpheme.
A \emph{clitic} is a morpheme which is phonologically bound
to a clitic host, but has a separates syntactic status to the host.
A typical example is the determiner \ve{=ee}, which marks definiteness.
While this determiner must occur attached to a host (e.g. \ve{kaut=ee} `the papaya')
which is the head of a noun phrase, the enclitic itself 
is the head of a separate determiner phrase (\srf{sec:DetPhr}).
My definitions of all these terms when applied to Amarasi or Meto data
are summarised in \qf{ex:TerDef} below, with a number of examples also given.

\begin{exe}
	\ex{Terminological definitions \txrf{}}\label{ex:TerDef}
	\begin{xlist}
		\ex{Morpheme = indivisible phonetic stretch with unitary meaning}
			\sn{\ve{n-} `third person verbal agreement', \ve{kobub} `pile up', \ve{kaut} `papaya', \ve{=ee} `{\ee}, third person determiner'}
		\ex{Word = minimal phonological string which can occur in isolation}
			\sn{\ve{n-kobub} `piles up', \ve{kaut} `papaya', \ve{kaut=ee} `the papaya'}
		\ex{Bound morpheme = morpheme which cannot occur as an independent word}
			\sn{\ve{n-} `third person verbal agreement', \ve{=ee} `{\ee}'}
		\ex{Root = underlying single morpheme}
			\sn{\ve{{\rt}n-} `third person verbal agreement', \ve{{\rt}kobub} `pile up', \ve{kaut} `papaya', \ve{{\rt}=ee} {\ee}}
		\ex{Free morpheme = morpheme which is an eligible word}
			\sn{\ve{kaut} `papaya', \ve{teun} `three'}
		\ex{Affix = bound morpheme with no separate syntactic status to its host}
			\sn{\ve{n-} `third person verbal agreement', \ve{-m} {\mg} `first person exclusive or second person genitive'}
		\ex{Clitic = bound morpheme with different syntactic status to its host}
			\sn{\ve{=ee} `{\ee}', \ve{=ma} `and', \ve{=kau} `{\kau}'}
		\ex{Stem = a word or root to which a bound morpheme attaches}
			\sn{\ve{n-\tbr{kobub}} `piles up', \ve{\tbr{kaut}=ee} `the papaya'}
		\ex{Citation Form = usual form of a word given in wordlist style elicitation}
	\end{xlist}
\end{exe}

I also make a distinction between two kinds of words and roots,
\emph{functors} and \emph{lexical words/roots} \citep[85ff]{zo78,gr91}.
Functors are morphemes which have grammatical uses,
such as relativisers, demonstratives, topic markers, and pronominals,
while lexical words/roots typically refer to events, states, properties, and things.
