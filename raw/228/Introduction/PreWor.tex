\section{Previous work}\label{sec:PreWor}
The earliest description of Meto known to me is \cite{mu57},
which contains a wordlist in what is probably a variety of Molo.
After this, the next earliest work is \cite{kl94}
which contain what appears to be an Amanuban wordlist,
though forms from other varieties are also given.

There are also works by the Dutch linguist
J. C. G. Jonker which contain data on Meto.
This includes \cite[270f]{jo04} which is a one page
glossed Amfo{\Q}an text with notes.
\cite{jo06} discusses word-final consonants in a number of
Austronesian languages including Meto.
The Meto data in \cite{jo06} is mostly Amfo{\Q}an, though data on
other varieties, including Amarasi, is also given.
Much of \citeauthor{jo06}'s Meto data also
occurs in etymological notes in \cite{jo08};
an 805-page dictionary of the Termanu variety of the Rote cluster.

\cite{ca44} provides a wordlist in Meto ``from Dutch sources''.
This appears to be based on Jonker's data
and \cite{jo06} is the source for
the discussion of final consonants in \cite[29]{ca44c}.

The first in depth treatment of Meto is that
of the Dutch missionary Pieter Middelkoop.
\citeauthor{mi39} published a collection of Amarasi texts \citep{mi39}
which had been previously collected by Jonker, a collection
of funeral chants \citep{mi49}, and a sketch grammar of Molo \citep{mi50}.
The other work by Middelkoop is an unpublished 673-page draft dictionary of Molo,
which was still in preparation before his death \citep{mi72}.\footnote{
		Thanks goes to James Fox for giving me his copy of \cite{mi72}.}
\citeauthor{mi39}'s materials on Meto contain much valuable data.
However, the transcription employed by \citeauthor{mi39} is not phonemic
and certain contrasts are under-represented.
%In order to assist future researchers I give
%here the most common transcriptional issues one encounters
%when working with \citeauthor{mi39}'s material.
%
%\begin{itemize}
	%\item The glottal stop /ʔ/ is often written as <{\Q}> between two vowels.
				%Word initially, before consonants, and word finally it is not usually written.
	%\item In \cite{mi39,mi50} the grave accent sometimes marks double vowels,
				%and sometimes phonetic vowel quality and/or stress.
				%In particular the mid-low allophone [ɛ] of /e/ is usually written \it{<è>}
	%\item A final apostrophe in \cite{mi72} marks a double vowel.
	%\item Both /ao/ and /au/ are transcribed \it{<au>}.
	%\item Prefixes consisting of a single consonant are often written
				%with a previous vowel-final word.
	%\item The final vowel of the pronouns \ve{ina} `{\iin}', \ve{sina} `{\siin}', \ve{hita} `{\hiit}'
				%is usually written with a following inflected verb.
%\end{itemize}

There are also a number of papers on Meto by Hein Steinhauer,
who worked on the Nilulat dialect of Miomafo.
This includes a description of verb morphology \citep{st93}
and a series of papers which provide an initial
description of the form of metathesis within the noun phrase
\citep{st96,st96b,st08}.

Other works which I have been able to access
on Meto include a Masters Thesis on Miomafo \citep{ta88},
a grammar produced by the Indonesian Pusat Bahasa \citep{ta89},\footnote{
		Thanks goes to Patrick McConvell for providing me
		with his copies of \citet{ta88} and \citet{ta89}.}
a description of quantification in Amanuban \citep{mebe14},
an Optimality Theory account of the segmental phonology of Miomafo \citep{is13},
a description of consonant insertion in Nai{\Q}bais Amfo{\Q}an \citep{cu18},
and a discussion of serial verb constructions in Amarasi
as being one source of similar constructions in Kupang Malay \citep{jagr11}.
