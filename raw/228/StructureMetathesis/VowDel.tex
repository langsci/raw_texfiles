\subsection{Vowel deletion}\label{sec:VowDel}
The final complication in the formation of the basic M\=/form
involves words which end in VVCV(C){\#}
in the U\=/form; words with a phonetic diphthong.
Such words derive their M\=/form by deletion of the final vowel as well as any final consonant.
The surface relationship between the segments of the U\=/form and M\=/form of
\ve{nautus} [ˈnəwt̪ʊs] {\ra} \ve{naut} [ˈnə.ʊt̪] `beetle' is given in \qf{as:nautus/naut},
with more examples given in \qf{ex:VVCV->VVC} below.

\begin{exe}
	\exa{\xy
		<0em,2.5cm>*\as{`beetle'}="gloss",
		<2.5em,2cm>*\as{n}="u1",<3.5em,2cm>*\as{a}="u2",<4.5em,2cm>*\as{u}="u3",<5.5em,2cm>*\as{t}="u4",<6.5em,2cm>*\as{u}="u5",<7.5em,2cm>*\as{s}="u6",<0em,2cm>*\as{U\=/form:}="u",
		<2.5em,1.5cm>*\as{C}="uC1",<3.5em,1.5cm>*\as{V}="uC2",<4.5em,1.5cm>*\as{V}="uC3",<5.5em,1.5cm>*\as{C}="uC4",<6.5em,1.5cm>*\as{V}="uC5",<7.5em,1.5cm>*\as{C}="uC6",
		<2.5em,0.5cm>*\as{C}="mC1",<3.5em,0.5cm>*\as{V}="mC2",<4.5em,0.5cm>*\as{V}="mC3",<5.5em,0.5cm>*\as{C}="mC4",
		<2.5em,0cm>*\as{n}="m1",<3.5em,0cm>*\as{a}="m2",<4.5em,0cm>*\as{u}="m4",<5.5em,0cm>*\as{t}="m5",<0em,0cm>*\as{M\=/form:}="m",
		{\ar@{->} "uC1"+D;"mC1"+U};{\ar@{->} "uC2"+D;"mC2"+U};{\ar@{->} "uC3"+D;"mC3"+U};{\ar@{->} "uC4"+D;"mC4"+U};
	\endxy}\label{as:nautus/naut} 
	\ex{{\ldots}V\sub{1}V\sub{2}C\sub{1}V\sub{3}(C\sub{2}){\#} {\ra} {\ldots}V\sub{1}V\sub{2}C\sub{1}{\#}}\label{ex:VVCV->VVC}
	\gw\sn{\begin{tabular}{llll|llll}
		 U\=/form						&		&\mc{2}{l|}{M\=/form}		& U\=/form						&		&\mc{2}{l}{M\=/form}		\\
		\ve{aun\tbr{u}}		&\ra&\ve{aun}		&`spear'	&\ve{naut\tbr{us}}	&\ra&\ve{naut}	&`beetle'\\
		\ve{n-ait\tbr{i}}	&\ra&\ve{n-ait}	&`pick up'&\ve{kaun\tbr{aʔ}}	&\ra&\ve{kaun}	&`snake'\\
		\ve{n-aen\tbr{a}}	&\ra&\ve{n-aen}	&`run, flee'		&\ve{aik\tbr{aʔ}}		&\ra&\ve{aik}		&`thorn'\\
		\ve{uab\tbr{aʔ}}	&\ra&\ve{uab}		&`speech'	&\ve{ain\tbr{a}-f}	&\ra&\ve{ain}		&`mother'\\
		\end{tabular}}
\end{exe}

Sequences of three surface vowels do not occur in Amarasi.
Thus, this vowel deletion can be analysed as resulting from phonological constraints of the language.
If consonant-vowel metathesis were to occur, it would result in a disallowed sequence
of three vowels which is resolved by vowel deletion.

\subsection{Irregular M\=/forms and U\=/forms}\label{sec:IrrMfor}
There are a handful of morphemes in my database which have irregular M\=/forms.
Firstly, the plural enclitic \ve{=enu}
has the M\=/form \ve{=uun} with irregular assimilation of the initial vowel.
This plural enclitic is uncommon in my data.
Instead, the form \ve{=eni/=ein} is most frequent (\srf{sec:PluEnc}).

Secondly, \ve{ai{\j}oʔo} `casuarina tree'
and \ve{naisoʔo} `garlic, shallot' have M\=/forms
derived by deleting the final /ʔo/ sequence.
Examples include \ve{ai{\j}o\tbr{ʔo}} + \ve{teas} `heartwood'
{\ra} \ve{ai{\j}o teas} `heartwood of a casuarina tree'
and \ve{naiso\tbr{ʔo}} + \ve{meʔe} `red' {\ra} \ve{naiso meʔe} `shallot'.

There are also a number of functors with final
/a/ in the U\=/form which form their M\=/form by deletion of this vowel.
These functors are given in \qf{ex:LexDel} below.
These functors usually occur in the M\=/form
and only take the U\=/form before consonant clusters
or when the plural enclitic \ve{=n} (\srf{sec:PluEnc}) is attached.

\begin{exe}
	\ex{{\ldots}V\sub{1}Ca{\#} /{\gap}CC {\ra} {\ldots}V\sub{1}C{\#}}\label{ex:CvaCC}
	\gw\sn{\begin{tabular}{lcll}
		U\=/form 		&		&\mc{2}{l}{M\=/form}					\\
		\ve{eta}	&\ra&\ve{et}	&`{\et}; at, in, on'\\
		\ve{ofa}	&\ra&\ve{of}	&`later, surely'	\\
		\ve{fina}	&\ra&\ve{fin}	&`because, so'		\\
		\ve{tara}	&\ra&\ve{tar}	&`until'					\\
		\ve{n-aka}	&\ra&\ve{n-ak}	&`say'						\\
%		\ve{n-oka}	&\ra&\ve{n-ok}	&`with accompany'	\\
	%	\ve{a}		&\ra&\ve{}		&`'	&\tsc{}		\\
		\end{tabular}}
\end{exe}

Finally, while most pronouns are VV{\#} final and thus
do not have distinct U\=/forms and M\=/forms,
those pronouns which do have both forms have multiple U\=/forms;
one with final /a/ and one with final /i/.
These pronouns are given in \qf{ex:LexDel} below.

\begin{exe}
	\ex{{\ldots}V\sub{1}CV\sub{2}{\#} {\ra} {\ldots}V\sub{1}C{\#}}\label{ex:LexDel}
	\gw\sn{\begin{tabular}{lclclll}
		U\=/form\sub{1} 	& 	&U\=/form\sub{2}		&		&M\=/form			&						&										\\
		\ve{\hp{=}ina}	&\ra&\ve{\hp{=}ini}		&\ra&\ve{iin}		&`s/he, it'	&\tsc{3sg.nom}			\\
		\ve{\hp{=}sina}	&\ra&\ve{\hp{=}sini}	&\ra&\ve{siin}	&`they'			&\tsc{3pl}					\\
		\ve{=sina}			&\ra&\ve{=sini}				&\ra&\ve{=siin}	&`them'			&\tsc{3pl}					\\
		\ve{\hp{=}hita}	&\ra&\ve{\hp{=}hiti}	&\ra&\ve{hiit}	&`we'				&\tsc{1pl.incl.nom}	\\
		\ve{=kita}			&\ra&\ve{=kiti}				&\ra&\ve{=kiit}	&`us'				&\tsc{1pl.incl.acc}	\\
		\end{tabular}}
\end{exe}

U\=/forms ending in /a/ are historically conservative.
Thus, PMP *sida > \ve{sina} > \ve{sini} `they',
and PMP *kita > \ve{hita} > \ve{hiti} `we',
as well as *kita > \ve{=kita} > \ve{=kiti} `us'.
The U\=/forms ending in /a/ tend only to
be used before consonant clusters, while the other
U\=/forms tend to be used with a morphological function,
though there are counterexamples in both cases.

While at an abstract level of phonological
organisation the M\=/form of these pronouns
must be analysed as containing a vowel sequence,
these pronouns are usually unstressed and, as a result, the vowel
sequence is usually realised as a single short vowel.
The vowel sequence in the M\=/form of these pronouns
is usually only realised as phonetically long in certain environments,
such as before vowel-initial enclitics (\srf{sec:PosDet}).

Comparative evidence that these pronouns
have an underlying sequence of two vowels comes from Kusa-Manea
in which M\=/forms of these pronouns have a /Va/ sequence:
\ve{ian} `{\iin}, s/he, it', \ve{sian} `{\siin}, they',
and \ve{hiat} `\tsc{1pl.incl}, we'.

%\footnote{
%		That the M\=/form of these functors has only a single vowel
%		was confirmed by an instrumental phonetic study.
%		Although excluded from the data used to measure vowel length in \srf{sec:QuaLenVowSeq},
%		there are 207 instances of these words
%		in the four texts used for this phonetic study.
%		They have an average length of 0.071 seconds,
%		below the average of 0.098 seconds for a single vowel,
%		and well below the average of 0.129 seconds
%		for a sequence of identical vowels.}

\subsection{No change}\label{sec:NoCha}
Words which end in a vowel sequence do not have distinct U\=/forms and M\=/forms.
Some examples are given in \qf{ex1:VV->VV} below.

\begin{exe}
	\ex{VV{\#} {\ra} VV{\#}}\label{ex1:VV->VV}
	\gw\sn{\begin{tabular}{lcll}
		 U\=/form			&		&\mc{2}{l}{M\=/form}				\\
		\ve{hau}		&\ra&\ve{hau}		&`tree, wood'	\\
		\ve{pui}		&\ra&\ve{pui}		&`quail'			\\
		\ve{biʤae}	&\ra&\ve{biʤae}	&`cow'				\\
		\ve{meo}		&\ra&\ve{meo}		&`cat'				\\
		\ve{ai}			&\ra&\ve{ai}		&`fire'				\\
		\ve{kee}		&\ra&\ve{kee}		&`turtle, tortoise'	\\
		\ve{pansoe}	&\ra&\ve{pansoe}&`earthworm'	\\
		\ve{ʔsao}		&\ra&\ve{ʔsao}	&`viper'			\\
		\end{tabular}}
\end{exe}