\section{M\=/forms before consonant clusters}\label{sec:CCIniMod}
In the previous sections I described and analysed the basic M\=/form.
This is the M\=/form taken by vowel-final verbs,
as well as nouns before an attributive modifier
with only a single initial consonant.
In addition to the basic M\=/form,
Amarasi has an additional M\=/form which is used
by nouns before attributive modifiers 
which begin with a consonant cluster.
This M\=/form is derived by deletion of any final consonant with no further changes.

M\=/forms before consonant clusters are glossed with `{\Mc}'
(\tsc{m} with a `c' for consonant above it).
{\MC}-forms are the paradigmatic and morphological equivalents of basic M\=/forms
in a different phonological environment.
Basic M\=/forms occur before attributive modifiers
which begin with a single consonant,
while {\MC}-forms occur before attributive modifiers with an initial cluster.
This process is not predictable consonant deletion to
avoid a cluster of three consonants. 
(Such clusters are avoided in other situations by
epenthesis of /a/ as discussed in \srf{sec:Epe}).

Examples of {\MC}-forms are shown in \qf{ex:C->0/CC} below for each word shape.
The modifiers used to illustrate are \ve{mnasiʔ} `old', \ve{kbubuʔ} `round'
\ve{mnanuʔ} `long' and \ve{mnatuʔ} `ripe, cooked' as semantically appropriate.
Words which end in a vowel in the U\=/form do not have distinct
M\=/forms before modifiers which begin with a consonant cluster.
It is possible at an abstract level to analyse the M\=/form
of such words as being formed by deletion of the final empty C-slot.

\begin{exe}
	\ex{C{\#} {\ra} {\0}/{\gap}CC }\label{ex:C->0/CC}
	\gw\sn{\begin{tabular}{rlll}
			 U\=/form						&			&{\MC}-form						&\\
			\ve{muʔi\tbr{t}}	&{\ra}&\ve{muʔi mnasiʔ}	&`old animal'	\\
			\ve{kau\tbr{t}}		&{\ra}&\ve{kau mnatuʔ}	&`cooked/ripe papaya'	\\
			\ve{nautu\tbr{s}}	&{\ra}&\ve{nautu kbubuʔ}&`round beetle'	\\
			\ve{fafi}					&{\ra}&\ve{fafi mnasiʔ}	&`old pig'	\\
			\ve{ume}					&{\ra}&\ve{ume kbubuʔ}	&`round house'	\\
			\ve{aunu}					&{\ra}&\ve{aunu mnanuʔ}	&`long spear'	\\
			\ve{oo}						&{\ra}&\ve{oo kbubuʔ}		&`round (piece of) bamboo'	\\
		\end{tabular}}
\end{exe}

The relationship between the surface forms of the U\=/form and {\MC}-form
\ve{muʔit} {\ra} \ve{muʔi} `animal' and \ve{fafi} {\ra} \ve{fafi} `pig'
are shown in \qf{as:muqit/muqi} and \qf{as:fatu/fatu} below.

\begin{multicols}{2}
	\begin{exe}
		\exa{\xy
			<0em,2.5cm>*\as{`animal'}="gloss",
			<2.5em,2cm>*\as{m}="u1",<3.5em,2cm>*\as{u}="u2",<4.5em,2cm>*\as{ʔ}="u3",<5.5em,2cm>*\as{i}="u4",<6.5em,2cm>*\as{t}="u5",<0em,2cm>*\as{U\=/form:}="u",
			<2.5em,1.5cm>*\as{C}="uC1",<3.5em,1.5cm>*\as{V}="uC2",<4.5em,1.5cm>*\as{C}="uC3",<5.5em,1.5cm>*\as{V}="uC4",<6.5em,1.5cm>*\as{C}="uC5",
			<2.5em,0.5cm>*\as{C}="mC1",<3.5em,0.5cm>*\as{V}="mC2",<4.5em,0.5cm>*\as{C}="mC3",<5.5em,0.5cm>*\as{V}="mC4",
			<2.5em,0cm>*\as{m}="m1",<3.5em,0cm>*\as{u}="m2",<4.5em,0cm>*\as{ʔ}="m4",<5.5em,0cm>*\as{i}="m3",<0em,0cm>*\as{{\MC}-form:}="m",
			{\ar@{->} "uC1"+D;"mC1"+U};{\ar@{->} "uC2"+D;"mC2"+U};{\ar@{->} "uC3"+D;"mC3"+U};{\ar@{->} "uC4"+D;"mC4"+U};
		\endxy}\label{as:muqit/muqi}
	\end{exe}
	\begin{exe}
		\exa{\xy
			<0em,2.5cm>*\as{`pig'}="gloss",
			<2.5em,2cm>*\as{f}="u1",<3.5em,2cm>*\as{a}="u2",<4.5em,2cm>*\as{f}="u3",<5.5em,2cm>*\as{i}="u4",<0em,2cm>*\as{U\=/form:}="u",%<6.5em,2cm>*\as{t}="u5",
			<2.5em,1.5cm>*\as{C}="uC1",<3.5em,1.5cm>*\as{V}="uC2",<4.5em,1.5cm>*\as{C}="uC3",<5.5em,1.5cm>*\as{V}="uC4",%<6.5em,1.5cm>*\as{C}="uC5",
			<2.5em,0.5cm>*\as{C}="mC1",<3.5em,0.5cm>*\as{V}="mC2",<4.5em,0.5cm>*\as{C}="mC3",<5.5em,0.5cm>*\as{V}="mC4",
			<2.5em,0cm>*\as{f}="m1",<3.5em,0cm>*\as{a}="m2",<4.5em,0cm>*\as{f}="m4",<5.5em,0cm>*\as{i}="m3",<0em,0cm>*\as{{\MC}-form:}="m",
			{\ar@{->} "uC1"+D;"mC1"+U};{\ar@{->} "uC2"+D;"mC2"+U};{\ar@{->} "uC3"+D;"mC3"+U};{\ar@{->} "uC4"+D;"mC4"+U};
		\endxy}\label{as:fatu/fatu}
	\end{exe}
\end{multicols}

VVC{\#} words with a final /n/ form a partial exception to this rule
when they occur before a modifier which begins with two nasals.
In such instances either the final consonant is deleted,
or it is retained and epenthesis occurs.
One example is \ve{kuan} `village' modified by \ve{mnaa{ʔ}} `old, former'
in which case both \ve{kua mnaa{ʔ}} or \ve{kuan a|mnaa{ʔ}} 
occur with an attributive meaning.\footnote{
		Nekmese{\Q} village was founded in the 1970s
		and many people still maintain fields and gardens near the old village
		(see \srf{sec:LanBac} for more details).
		Thus, the phrase \ve{kua(n a)mnaa{ʔ}} is frequently heard.
		The form \ve{kua mnaaʔ} is much more common in my experience.}

Likewise, when asked to translate `old tap' into Amarasi (\ve{kraan} `tap' + \ve{mnaaʔ} `old'),
Roni (my main consultant) produced the string \ve{kraan a|mnaa{ʔ}}.
I immediately then presented him with the string \ve{kraa mnaa{ʔ}}
which he interpreted as being `old glass', from \ve{kraas} + \ve{mnaa{ʔ}}.

In \srf{sec:ConDel/CC} and \srf{sec:CVFinWor} below I sketch
a partial analysis of {\MC}-forms within Optimality Theory.
I do this because the tableaux that this theory employs
illustrate well the large number of potential outputs
the combination of a noun followed by an attributive
modifier could potentially generate.
The purpose of this book is not to give a
complete Optimality Theory account of metathesis in Amarasi.
Indeed, the high level of opacity in the formation of M\=/forms
-- including at least one derived environment effect (\srf{sec:AssOfA}) --
indicates that standard Optimality Theory
would not fare particularly well in Amarasi.
Nonetheless Optimality Theory is still a useful tool
to illuminate certain aspects of the structure of the language.

\subsection{Consonant deletion}\label{sec:ConDel/CC}
When a consonant-final word, such as \ve{muʔit} `animal',
occurs before an attributive modifier
with an initial cluster, such as \ve{mnasiʔ} `old',
the final consonant of the first noun is deleted.
This yields \brac{NP} \ve{muʔi} \ve{mnasiʔ}] `an old animal'.

In such instances, there are a large number of potential
outputs involving combinations of: metathesis, consonant deletion, and/or epenthesis.
Each of these potential outputs is given in the Optimality Theory
tableau in \qf{ex:muqi mnasiq} below, along with the constraints they violate.
The definitions constraints are given in \qf{ex:Constraints}.
Their ranking is according to the order given.

\begin{exe}
	\ex{\begin{xlist}
		\ex{\tsc{*CC{\#}}: No final consonant clusters}
		\ex{\tsc{*-CC-}: No foot medial consonant clusters}
		\ex{\tsc{*CCC}: No clusters of three consonants}
		\ex{\tsc{Dep}: No epenthesis}
		\ex{\tsc{Max}: No deletion}
		\ex{\tsc{\M}: Mark the M\=/form}\label{ex:MarMfo}
		\ex{\tsc{Linearity}: No metathesis}
	\end{xlist}}\label{ex:Constraints}
\end{exe}

Constraint \qf{ex:MarMfo} is equivalent to
\tsc{RealizeMorpheme} in the sense of \citet{ku01}.
This constraint is included as the forms under discussion
are those which are paradigmatically and morphologically
equivalent to the basic M\=/form which occurs before attributive
modifiers with no initial cluster.

\begin{exe}
	\ex{\rule{0pt}{0pt}{} \\[-5ex]\stl{0.41em}
	\begin{tabular}[t]{|rrl||c|c|c|c|c|c|c|} \hline
		\multicolumn{3}{|c||}{\brac{NP} \ve{muʔit} + \ve{mnasiʔ} ]} 
																						&*CC{\#}&*-CC-	&*CCC 		&\tsc{Dep}&\tsc{Max}&{\M}			& \tsc{Lin} \\[0.5ex]\hline
		\hline a. &				& \ve{muiʔt mnasiʔ} 	&*! 		&{\cgr}	&{\cgr}**	&{\cgr}		&{\cgr} 	&{\cgr}		& {\cgr}* \\
		\hline b. &				& \ve{muʔti mnasiʔ}		&				&*!			&{\cgr}		&{\cgr}		&{\cgr}		&{\cgr}		&	{\cgr}*	\\
		\hline c. &				& \ve{muiʔ mnasiʔ}		&				&				&*! 			&{\cgr} 	&{\cgr}* 	&{\cgr}		& {\cgr}* \\
		\hline d. &				& \ve{mui mnasiʔ} 		&				&				&					&					&**! 			&{\cgr}		& {\cgr} 	\\
		\hline e. &				& \ve{muiʔt a|mnasiʔ} &*! 		&{\cgr}	&{\cgr} 	&{\cgr}*	&{\cgr} 	&{\cgr}		& {\cgr}* \\
		\hline f. &				& \ve{muʔit mnasiʔ} 	&				&				&*! 			&{\cgr} 	&{\cgr} 	&{\cgr}*	& {\cgr}	\\
		\hline g. &				& \ve{muʔit a|mnasiʔ} &				&				&					&*! 			&{\cgr} 	&{\cgr}*	& {\cgr}	\\
		\hline h. &{\hand}& \ve{muʔi mnasiʔ} 		&				&				&					&					&* 				&{\cgr}		& {\cgr}	\\
	\hline \end{tabular}}\label{ex:muqi mnasiq}
\end{exe}

Table \qf{ex:muqi mnasiq} shows that the output with deletion of the
final consonant, \ve{muʔi mnasiʔ}, is the best output.
This candidate marks the M\=/form and also
avoids final consonant clusters, foot medial clusters,
clusters of three consonants, and epenthesis.
While it does have consonant deletion,
it only deletes one consonant while the next best candidate
\ve{\tcb{*}mui mnasiʔ} has two consonants deleted.

When a consonant-final word occurs before a predicative modifier
with an initial consonant cluster,
epenthesis usually occurs between the two words.
This is shown in \qf{ex:muqit-amnasiq} and \qf{tr:muqit-amnasiq}
below, which can be contrasted with the attributive phrases
in \qf{ex:muqi-mnasiq} and \qf{tr:muqi-mnasiq}.

\begin{multicols}{2}
	\begin{exe}
		\ex{\gll \brac{NP} {muʔit \bracr{}} \brac{NP} {a|mnasiʔ \bracr{}}\\
							{} animal{\U} {} {\a}old\\
				\glt \lh{\brac{NP} }`Animals are old.'}\label{ex:muqit-amnasiq}
		\ex{\gll \brac{NP} muʔi mnasiʔ \bracr{}\\
							{} animal{\Mc} old {}\\
				\glt \lh{\brac{NP} }`(an) old animal'}\label{ex:muqi-mnasiq}
	\end{exe}
\end{multicols}
%\newpage
\begin{multicols}{2}
	\begin{exe}
		\ex{\begin{forest} %where n children=0{tier=word}{}
			[S,[NP,[N,[\ve{muʔit}\\animal{\U}]]][NP,[N,[\ve{a|mnasiʔ}\\old]]]]
		\end{forest}}\label{tr:muqit-amnasiq}
		\ex{\begin{forest} %where n children=0{tier=word}{}
			[S,[\vp{NP}{\ldots},[,phantom]][NP,[N,[\ve{muʔi}\\animal{\Mc}]][N,[\ve{mnasiʔ}\\old]]]]
		\end{forest}}\label{tr:muqi-mnasiq}
	\end{exe}
\end{multicols}

This can be explained by positing that while epenthesis
is not allowed within a single phrase,
it \emph{is} allowed between two separate phrases.
In the terminology of Optimality Theory,
the constraint \tsc{Dep} is more highly ranked than \tsc{Max} within a single phrase,
while between two phrases \tsc{Max} is more highly ranked than \tsc{Dep}.
A modified version of table \qf{ex:muqi mnasiq}
is given in \qf{ex:muqit amnasiq} below for predicative phrase
with the constraints re-ordered as appropriate.
The constraint \tsc{\M} `mark the M\=/form' has been removed as
this is not a requirement of predicative phrases.

\begin{exe}
	\ex{\rule{0pt}{0pt}{} \\[-5ex]\stl{0.45em}
	\begin{tabular}[t]{|rrl||c|c|c|c|c|c|} \hline
		\multicolumn{3}{|c||}{\brac{NP} \ve{muʔit} ] + \brac{NP} \ve{mnasiʔ} ]}
																						&*CC{\#}&*-CC-	&*CCC 		&\tsc{Max}&\tsc{Dep}& \tsc{Lin} \\[0.5ex]\hline
		\hline a. &				& \ve{muiʔt mnasiʔ} 	&*! 		&{\cgr}	&{\cgr}**	&{\cgr}		&{\cgr} 	& {\cgr}* \\
		\hline b. &				& \ve{muʔti mnasiʔ}		&				&*!			&{\cgr}		&{\cgr}		&{\cgr}		&	{\cgr}*	\\
		\hline c. &				& \ve{muiʔ mnasiʔ}		&				&				&*! 			&{\cgr}* 	&{\cgr} 	& {\cgr}* \\
		\hline d. &				& \ve{mui mnasiʔ} 		&				&				&					&**! 			&{\cgr}		& {\cgr} 	\\
		\hline e. &				& \ve{muiʔt a|mnasiʔ} &*! 		&{\cgr}	&{\cgr} 	&{\cgr} 	&{\cgr}*	& {\cgr}* \\
		\hline f. &				& \ve{muʔit mnasiʔ} 	&				&				&*! 			&{\cgr} 	&{\cgr} 	& {\cgr}	\\
		\hline g. &{\hand}& \ve{muʔit a|mnasiʔ} &				&				&					&					&*	 			& {\cgr}	\\
		\hline h. &				& \ve{muʔi mnasiʔ} 		&				&				&					&*!				& 				& {\cgr}	\\
	\hline \end{tabular}}\label{ex:muqit amnasiq}
\end{exe}

Table \qf{ex:muqit amnasiq} shows that when two separate
noun phrases occur next to one another
a cluster of three consonants is resolved by epenthesis.
It is better to epenthesise between noun phrases
than to have a cluster of three consonants.

\begin{multicols}{2}
	\begin{exe}
		\exa{\label{as:muqit-amnasiq}\xy
			<3em,6cm>*\as{PrWd}="PrWd1",<9.5em,6cm>*\as{PrWd}="PrWd2",
			<3em,5cm>*\as{Ft}="ft1",<11em,5cm>*\as{Ft}="ft2",
			<2em,4cm>*\as{σ}="s1",<4em,4cm>*\as{σ}="s2",<7em,4cm>*\as{σ}="s3",<10em,4cm>*\as{σ}="s4",<12em,4cm>*\as{σ}="s5",
			<1em,3cm>*\as{C}="CV1",<2em,3cm>*\as{V}="CV2",<3em,3cm>*\as{C}="CV3",<4em,3cm>*\as{V}="CV4",<5em,3cm>*\as{C}="CV5",
			<6em,3cm>*\as{C}="CV6",<7em,3cm>*\as{V}="CV7",<8em,3cm>*\as{C}="CV8",
			<9em,3cm>*\as{C}="CV9",<10em,3cm>*\as{V}="CV10",<11em,3cm>*\as{C}="CV11",<12em,3cm>*\as{V}="CV12",<13em,3cm>*\as{C}="CV13",
			<1em,2cm>*\as{m}="cv1",<2em,2cm>*\as{u}="cv2",<3em,2cm>*\as{ʔ}="cv3",<4em,2cm>*\as{i}="cv4",<5em,2cm>*\as{t}="cv5",
			<6em,2cm>*\as{ʔ}="cv6",<7em,2cm>*\as{a}="cv7",<8em,2cm>*\as{m}="cv8",<9em,2cm>*\as{n}="cv9",
			<10em,2cm>*\as{a}="cv10",<11em,2cm>*\as{s}="cv11",<12em,2cm>*\as{i}="cv12",<13em,2cm>*\as{ʔ}="cv13",
			<3em,1cm>*\as{M}="m1",<10.5em,1cm>*\as{M}="m2",
			<3em,0cm>*\as{SynWd}="SynWd1",<10.5em,0cm>*\as{SynWd}="SynWd2",
			"SynWd1"+U;"m1"+D**\dir{-};"SynWd2"+U;"m2"+D**\dir{-};
			"m1"+U;"cv1"+D**\dir{-};"m1"+U;"cv2"+D**\dir{-};"m1"+U;"cv3"+D**\dir{-};"m1"+U;"cv4"+D**\dir{-};"m1"+U;"cv5"+D**\dir{-};
			"m2"+U;"cv8"+D**\dir{-};"m2"+U;"cv9"+D**\dir{-};"m2"+U;"cv10"+D**\dir{-};
			"m2"+U;"cv11"+D**\dir{-};"m2"+U;"cv12"+D**\dir{-};"m2"+U;"cv13"+D**\dir{-};
			"cv1"+U;"CV1"+D**\dir{-};"cv2"+U;"CV2"+D**\dir{-};"cv3"+U;"CV3"+D**\dir{-};"cv4"+U;"CV4"+D**\dir{-};
			"cv5"+U;"CV5"+D**\dir{-};"cv6"+U;"CV6"+D**\dir{-};"cv7"+U;"CV7"+D**\dir{-};
			"cv8"+U;"CV8"+D**\dir{-};"cv9"+U;"CV9"+D**\dir{-};"cv10"+U;"CV10"+D**\dir{-};
			"cv11"+U;"CV11"+D**\dir{-};"cv12"+U;"CV12"+D**\dir{-};"cv13"+U;"CV13"+D**\dir{-};
			"CV1"+U;"s1"+D**\dir{-};"CV2"+U;"s1"+D**\dir{-};"CV3"+U;"s1"+D**\dir{-};
			"CV3"+U;"s2"+D**\dir{-};"CV4"+U;"s2"+D**\dir{-};"CV5"+U;"s2"+D**\dir{-};
			"CV6"+U;"s3"+D**\dir{-};"CV7"+U;"s3"+D**\dir{-};"CV8"+U;"s3"+D**\dir{-};
			"CV9"+U;"s4"+D**\dir{-};"CV10"+U;"s4"+D**\dir{-};"CV11"+U;"s4"+D**\dir{-};
			"CV11"+U;"s5"+D**\dir{-};"CV12"+U;"s5"+D**\dir{-};"CV13"+U;"s5"+D**\dir{-};
			"s1"+U;"ft1"+D**\dir{-};"s2"+U;"ft1"+D**\dir{-};"s4"+U;"ft2"+D**\dir{-};"s5"+U;"ft2"+D**\dir{-};
			"ft1"+U;"PrWd1"+D**\dir{-};"s3"+U;"PrWd2"+D**\dir{-};"ft2"+U;"PrWd2"+D**\dir{-};
		\endxy}
		\exa{\label{as:muqi-mnasiq}\xy
			<3em,6cm>*\as{PrWd}="PrWd1",<8em,6cm>*\as{PrWd}="PrWd2",
			<3em,5cm>*\as{Ft}="ft1",<8em,5cm>*\as{Ft}="ft2",
			<2em,4cm>*\as{σ}="s1",<4em,4cm>*\as{σ}="s2",<7em,4cm>*\as{σ}="s3",<9em,4cm>*\as{σ}="s4",
			<1em,3cm>*\as{C}="CV1",<2em,3cm>*\as{V}="CV2",<3em,3cm>*\as{C}="CV3",
			<4em,3cm>*\as{V}="CV4",<5em,3cm>*\as{C}="CV5",
			<6em,3cm>*\as{C}="CV6",<7em,3cm>*\as{V}="CV7",<8em,3cm>*\as{C}="CV8",
			<9em,3cm>*\as{V}="CV9",<10em,3cm>*\as{C}="CV10",
			<1em,2cm>*\as{m}="cv1",<2em,2cm>*\as{u}="cv2",<3em,2cm>*\as{ʔ}="cv3",<4em,2cm>*\as{i}="cv4",
			<5em,2cm>*\as{m}="cv5",<6em,2cm>*\as{n}="cv6",<7em,2cm>*\as{a}="cv7",<8em,2cm>*\as{s}="cv8",
			<9em,2cm>*\as{i}="cv9",<10em,2cm>*\as{ʔ}="cv10",
			<2.5em,1cm>*\as{M}="m1",<7.5em,1cm>*\as{M}="m2",
			<5em,0cm>*\as{SynWd}="SynWd1","SynWd1"+U;"m1"+D**\dir{-};"SynWd1"+U;"m2"+D**\dir{-};
			"m1"+U;"cv1"+D**\dir{-};"m1"+U;"cv2"+D**\dir{-};"m1"+U;"cv3"+D**\dir{-};"m1"+U;"cv4"+D**\dir{-};
			"m2"+U;"cv5"+D**\dir{-};"m2"+U;"cv6"+D**\dir{-};"m2"+U;"cv7"+D**\dir{-};
			"m2"+U;"cv8"+D**\dir{-};"m2"+U;"cv9"+D**\dir{-};"m2"+U;"cv10"+D**\dir{-};
			"cv1"+U;"CV1"+D**\dir{-};"cv2"+U;"CV2"+D**\dir{-};"cv3"+U;"CV3"+D**\dir{-};"cv4"+U;"CV4"+D**\dir{-};"cv5"+U;"CV5"+D**\dir{-};
			"cv6"+U;"CV6"+D**\dir{-};"cv7"+U;"CV7"+D**\dir{-};"cv8"+U;"CV8"+D**\dir{-};"cv9"+U;"CV9"+D**\dir{-};
			"cv10"+U;"CV10"+D**\dir{-};
			"CV1"+U;"s1"+D**\dir{-};"CV2"+U;"s1"+D**\dir{-};"CV3"+U;"s1"+D**\dir{-};
			"CV3"+U;"s2"+D**\dir{-};"CV4"+U;"s2"+D**\dir{-};"CV5"+U;"s2"+D**\dir{-};
			"CV6"+U;"s3"+D**\dir{-};"CV7"+U;"s3"+D**\dir{-};"CV8"+U;"s3"+D**\dir{-};
			"CV8"+U;"s4"+D**\dir{-};"CV9"+U;"s4"+D**\dir{-};"CV10"+U;"s4"+D**\dir{-};
			"s1"+U;"ft1"+D**\dir{-};"s2"+U;"ft1"+D**\dir{-};"s3"+U;"ft2"+D**\dir{-};"s4"+U;"ft2"+D**\dir{-};
			"ft1"+U;"PrWd1"+D**\dir{-};"ft2"+U;"PrWd2"+D**\dir{-};
		\endxy}
	\end{exe}
\end{multicols}

In an analysis which is considered in \srf{sec:ProsMet},
I propose that members of an attributive phrase
are members of a single category -- the syntactic word --
while each member of a predicative phrase
is a member of a different syntactic word.\footnote{
	This is essentially the same as proposing that
	attributive phrases are a (syntactic) compound,
	even though their members may belong to different
	prosodic categories.}
The relationship between the prosodic structure, morphological structure,
and the Syntactic Word(s) of the attributive phrase
\ve{muʔi mnasiʔ} `(an) old animal',
and the predicative phrase \ve{muʔit a|mnasiʔ} `animals are old'
are shown in \qf{as:muqit-amnasiq} and \qf{as:muqi-mnasiq} above respectively.

When a consonant-final nominal occurs before a 
modifier with an initial consonant cluster,
the cluster of three consonants is usually resolved in Amarasi.
In an attributive phrase, such as that
represented in \qf{as:muqi-mnasiq},
the M\=/form must be realised to mark the presence of this attributive modifier.
Metathesis is blocked as it would result in a cluster of three consonants,
as exemplified in \qf{ex:muqi mnasiq}.
As a result, the final consonant of the first
noun is deleted to express the M\=/form.
This also has the effect of resolving the cluster of three consonants.

However, when the phrase consists of two syntactic words,
such as that represented in \qf{as:muqit-amnasiq},
there is no need to mark the M\=/form.
As a result, the cluster of three consonants remains.
Epenthesis of /a/ (preceded by an automatic glottal stop
--- see \srf{sec:GloStoIns}) then occurs between these two syntactic words,
thus resolving the cluster of three consonants.

\subsection{No change}\label{sec:CVFinWor}
Vowel-final words do not have a distinct M\=/form
before attributive modifiers with an initial cluster.
Metathesis in this environment is blocked as it would
create a cluster of three consonants.
It is more important in Kotos Amarasi to avoid a cluster of three
consonants than it is to mark the M\=/form.

However, there are at least two logical ways in which Amarasi
could avoid a cluster of three consonants and still mark the
M\=/form for CV{\#} final words.
Firstly, metathesis could occur with subsequent epenthesis: i.e.
\ve{fafi} `pig' + \ve{mnasiʔ} `old' {\ra}
\ve{\tcb{*}fai\tbr{f} \tbr{mn}asiʔ} {\ra} \ve{\tcb{*}faif a|mnasiʔ}.
Epenthesis is attested elsewhere in Amarasi
to break up sequences of three consonants (\srf{sec:Epe}).
Secondly, metathesis could take place with subsequent deletion of the final consonant,
\ve{\tcb{*}fai\tbr{f} \tbr{mn}asiʔ} {\ra} \ve{\tcb{*}fai mnasiʔ}.
Consonant deletion is attested elsewhere in
the derivation of M\=/forms (\srf{sec:MetConDel}, \srf{sec:ConDel}, \srf{sec:ConDel/CC}).

We thus have at least four possible outputs when a CV{\#}
final word is modified by a nominal with an initial consonant cluster.
Each of these potential outputs is given in the Optimality Theory
tableau in \qf{ex:fafi mnasiq} below, along with the constraint(s) they violate.
These constraints and their ranking
were given in \qf{ex:Constraints} above.

\newpage
\begin{exe}
	\ex{\rule{0pt}{0pt}{} \\[-5ex]
	\begin{tabular}[t]{|rrl||c|c|c|c|c|} \hline
		\multicolumn{3}{|c||}{\brac{NP} \ve{fafi} + \ve{mnasiʔ} ]} 
																					& *CCC&\tsc{Dep}&\tsc{Max}&\tsc{\M}	&\tsc{Lin} \\[0.5ex]\hline
		\hline a. & 			& \ve{faif mnasiʔ}	& *!	&{\cgr} 	&{\cgr} 	&{\cgr}		&{\cgr}*\\
		\hline b. & 			& \ve{faif a|mnasiʔ}& 		&*!				&{\cgr} 	&{\cgr}		&{\cgr}*\\
		\hline c. & 			& \ve{fai mnasiʔ}		& 		&  				&*! 			&{\cgr}		&{\cgr} \\
		\hline d. &{\hand}& \ve{fafi mnasiʔ} 	& 		& 				& 				&*				&{\cgr} \\
	\hline \end{tabular}}\label{ex:fafi mnasiq}
\end{exe}

Potential output (\ref{ex:fafi mnasiq}a.) \ve{\tcb{*}faif mnasiʔ}
does not occur because it is worse to have a cluster of three consonants
than it is to mark the M\=/form.
Potential output (\ref{ex:fafi mnasiq}b.) \ve{\tcb{*}faif a|mnasiʔ}
does not occur because it is worse to epenthesise
(within a single phrase) than it is to mark the M\=/form.
Potential output (\ref{ex:fafi mnasiq}c.) \ve{\tcb{*}fai mnasiʔ}
does not occur because it is worse to delete a medial consonant
than it is to mark the M\=/form.
This leaves the occurring output \ve{fafi mnasiʔ},
which fails to mark the M\=/form but does not violate
any of the more highly ranked constraints.

\subsection{Ro{\Q}is Amarasi modifiers with an initial cluster}\label{sec:RoqAnaCCIniMod}
In Ro{\Q}is Amarasi metathesis occurs before words
which begin with a consonant cluster.
Examples are given in \trf{tab:RoqMetConClu} on the next page
alongside Kotos Amarasi equivalents (where known) for comparison.

Metathesis of CV(C){\#} final words is the most common
pattern before CC-initial modifiers in my Ro{\Q}is data.
However, two other patterns are also found.\footnote{
		These alternate patterns are most frequent for
		one of my consultants from Tunbaun,
		though do sporadically occur in the speech of others.}
Firstly, there are two examples in my corpus in
which metathesis does not occur:
\ve{smana-f} `spirit' + \ve{kninuʔ} `clean, holy'
{\ra} \ve{smana kninuʔ} `Holy Spirit'
and \ve{hana-f} `voice' + \ve{tbaat} `lies across, in-between'
{\ra} \ve{hana tbaat} `intermediate dialect'.\footnote{
		The same speaker also uses \ve{haan tbaat} `intermediate dialect'
		at another point in the same text.}

Secondly, diphthongisation of the stressed vowel (\srf{sec:RoqAmaDip})
without metathesis can also occur. There are six examples in my data
of which three are: \ve{u\tbr{mi}} `house' + \ve{kbubuʔ} `round' {\ra}
\ve{u͡\tbr{{\i}mi} kbubuʔ} `round house'
(alongside Ro{\Q}is \ve{uim kbubuʔ}, Kotos \ve{umi kbubuʔ}),
\ve{ne\tbr{no}} `day' + \ve{krei} `church' {\ra} \ve{ne͡\tbr{ono} krei} `Sunday'
(alongside Ro{\Q}is \ve{neon krei}, Kotos \ve{neno krei}), and
\ve{ra\tbr{si}} `matter' + \ve{skoor} `school' {\ra} \ve{ra͡\tbr{{\i}si} skoor}
`school matters' (Kotos \ve{rasi skoor}).
Given that diphthongisation of stressed vowels
followed by a closed syllable is an automatic process
in Ro{\Q}is (see \srf{sec:RoqAmaDip}),
it is probably best to analyse such instances as underlyingly
unmetathesised, with an automatic phonological rule applying.
Nonetheless, diachronically, such forms are intermediate
between metathesised and unmetathesised forms,
as was discussed in \srf{sec:OriMetAma}.

\begin{table}[ht]
	\centering\caption{Ro{\Q}is metathesis before consonant clusters}\label{tab:RoqMetConClu}
	\begin{tabular}{l@{\hspace{0.4em}}c@{\hspace{0.4em}}l@{ }llll}\lsptoprule
			 Noun				&	& mod.					&		& Ro{\Q}is 								& Kotos & \\ \midrule
		\ve{kruru-f} 	&+& \ve{tnana-f}	&\ra& \ve{kru\tbr{ur tn}anaf} & \ve{kruru tnanaf} & `middle finger' \\
		`finger'			&+&	`middle'			&&&&	\\
		\ve{umi}			&+& \ve{kbubuʔ}		&\ra& \ve{u\tbr{im kb}ubuʔ} 	& \ve{umi kbubuʔ}		& `round house' \\
		`house'				&+&	`round'				&&&&	\\
		\ve{tenoʔ}		&+& \ve{kmoro-f}	&\ra& \ve{te\tbr{on km}orof}	& \ve{teno kmorof}	& `egg yolk' \\
		`egg'					&+&	`yellow'			&&&&	\\
		\ve{fatu} 		&+& \ve{kruru-f}	&\ra& \ve{fa\tbr{ut}{\gap}\tbr{kr}uruf} 	& \ve{fatu kruruf}	& `soft coral' \\
		`stone'				&+&	`finger'			&&&&	\\
		\ve{ikaʔ} 		&+& \ve{tnopos}		&\ra& \ve{i\tbr{ik}{\gap}\tbr{tn}opos} 	& \ve{} & `silver moony \\
		`fish'				&+&	`silver'			&& \mc{3}{r}{(\it{Monodactylus argentus})'}	\\
		\ve{ikaʔ} 		&+& \ve{kbiti}		&\ra& \ve{i\tbr{ik}{\gap}\tbr{kb}iti} 		& \ve{} & `spinefoot \\
		`fish'				&+&	`scorpion'		&&&&\hp{`}(\it{Siganus spp.})'	\\
%		\ve{} 		&+& \ve{}	&\ra& \ve{} 	& \ve{} & `' \\
%		`'				&+&	`'			&&&&	\\
		\lspbottomrule
	\end{tabular}
\end{table}

Finally, VVC{\#} final words in Ro{\Q}is delete their
final consonant before all modifiers,
including modifiers with an initial cluster,
to derive an M\=/form in the same way as Kotos (\srf{sec:ConDel}).
Two examples from my Ro{\Q}is data are: \ve{knaaʔ}
+ \ve{mnanuʔ} `long' {\ra} \ve{knaa mnanuʔ} `long beans'
and \ve{kniit} `crab' + \ve{snaen} `sand' {\ra} \ve{knii snaen}
`horned ghost crab (\emph{Ocypode ceratophthalma})'.
Given that Ro{\Q}is permits clusters of three consonants,
deletion of the final consonant in these instances
provides additional evidence that the consonant deletion
Kotos Amarasi is indeed a morphological process.

%Buraen
%uim krei
%uim skoor
%uim preent
%niim mnonof
%teon kmorof
%teon kmuitif
%kruur tnanaf
%faut kruruf
%knaam fno'ot (k.o. katydid) knamat
%akbeun bjae/bjakase' (big fly)
%kuum treukus
%teem brahu'
%niin tboorn
%koor bjakase'
%seor bjae te'i
%kroom bjakase' (kroma')
%kbaut bjakase' (kbautus)
%kroom snaen (p. 16)
%iik knaapn (p. 18)
%iik knaes
%iik snaen
%iik knaap fafi
%iik tnopos
%iik kbiti
%iik kmuru
%iik mneas
%
%Tunbaun
%uim krei, uimi krei
%uim skoor, uimi skoor
%kaun smata'
%
%Texts:
%atoin mnasi'
%haan tbaat
%
%neono tnana'
%neono krei
%raisi skoor
%raisi krei
%raisi kninu'
%uiri mnatu'
%
%smana kninu'
%hana tbaat ~ haan tbaat