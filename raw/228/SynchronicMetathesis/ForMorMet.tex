\section{Forms of synchronic metathesis}\label{sec:For ch:SynchMet}
In this section I summarise some common features in the forms
of synchronic metathesis among the languages discussed in this chapter
which also occur with metathesis in Amarasi:
the prevalence of consonant-vowel metathesis among
languages in which metathesis has a morphological function
and the existence of associated phonological processes.

\subsection{Consonant-vowel metathesis}
The only kind of metathesis known to develop
a morphological function is that of adjacent consonants and vowels;
either CV {\ra} VC or VC {\ra} CV.
No instances of vowel-vowel or consonant-consonant
metathesis are known to be morphological,
though the latter does occur as an automatic
phonological process (\srf{sec:PhoMet}).
Furthermore, instances of morphological metathesis
discussed in this chapter can all be located with respect
to a stressed vowel, the word edge, or both.

These two facts arise from the historic development
of metathesis, as summarised in \srf{sec:OriSynMet}.
The non-existence of morphological processes
of consonant- consonant or vowel-vowel metathesis
is also connected with the development of such processes.
There does not currently appear to be any series of
phonetically natural changes whereby such a process could develop.

Under compensatory metathesis (\srf{sec:ComMet}) the crucial step in the development of
true synchronic metathesis is weakening and loss of an unstressed vowel.
This accounts for instances of metathesis which are located with respect to stress.
Similarly, under pseudo-metathesis (epenthesis and deletion, \srf{sec:EpeApo})
the only (known) attested cases which have developed into true metathesis
are those in which an epenthetic vowel has been added to a word edge.
This accounts for instances of metathesis which
are located with respect to a word edge.

Synchronic accounts of metathesis should take into
consideration the history of these processes.
Thus, descriptions of languages in which metathesis developed after stressed syllables
should include this fact in any description of the synchronic process.
A rule such as CV {\ra} VC /\'V{\_} achieves this by including
the stressed syllable as a constraining environment.
Likewise, in languages in which metathesis developed by epenthesis and apocope
should be informed by the fact that metathesis only developed at word edges.
Again, a rule such as VC {\ra} CV /{\_}{\#} achieves this 
by constraining metathesis to the word edge.

\subsection{Associated phonological processes}
Synchronic consonant-vowel metathesis is typically
associated with other processes.
In some cases these processes co-occur with metathesis,
and in others they occur instead of metathesis
for words of a particular phonotactic shape.

There are two reasons why metathesis is usually associated with other processes.
Firstly, in cases such as Leti,
morphological metathesis has developed through the accumulation
of a number of different processes (\srf{sec:OriMorMet}),
with some of these processes still being attested alongside metathesis
in certain phonotactic or phonological environments.

Secondly, in cases such as Mambae and Rotuman,
it is the metathesis itself which triggers other phonological processes.
These processes are a response to the new phonological shape
of the stem created through metathesis, such as
assimilation of newly adjacent vowels.
These are the kinds of processes associated with metathesis in Amarasi.
