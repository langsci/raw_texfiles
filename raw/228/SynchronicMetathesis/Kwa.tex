\subsection{Kwara'ae}\label{sec:Kwa}
Metathesis in Kwara'ae has been described by
\citet{so80} and \citet{he04,he05}.
\citet{blga98} also present previously unpublished data
collected by Andrew Pawley and David Gegeo.
Metathesis in Kwara'ae has been analysed as phonologically conditioned (\srf{sec:PhoMet})
but it is not restricted to a subset of words with specific phonological properties.
Instead nearly every word of the lexicon is affected by metathesis in Kwara'ae.

\subsubsection{Forms}
Metathesis in Kwara'ae is CV {\ra} VC metathesis.
Examples are shown in \qf{ex:KwVCV->VVC} below.
In the literature on Kwara'ae the unmetathesised
form (U\=/form) is called the \emph{citation form} and the metathesised form
(M\=/form) is called the \emph{normal form}.
I refer to them with the more iconic terms \emph{U\=/form} and \emph{M\=/form}.

\begin{exe}
	\ex{V\sub{1}CV\sub{2} {\ra} V\sub{1}V\sub{2}C \hfill\citep[1]{he04}}\label{ex:KwVCV->VVC}
	\sn{\gw\begin{tabular}{rcll}
		 U\=/form 					&			& \mc{2}{l}{M\=/form}  \\
		\it{ˈlo.\tbr{ʔi}} &{\ra}& \it{ˈlo\tbr{i̯ʔ}} & `snake' \\
		\it{ˈbu.\tbr{ri}} &{\ra}& \it{ˈbu̯\tbr{ir}} & `behind' \\
		\it{ˈbo.\tbr{re}} &{\ra}& \it{ˈbo̯\tbr{er}} & `although' \\
	\end{tabular}}
\end{exe}

Depending on the length of the word,
metathesis in Kwara'ae can occur multiple times.
Two examples are given in \qf{ex:MulMetKwa} below.
The difference in stress which is seen in examples such as
\it{da.ˈro.ʔa.ˌni.da} {\ra} \it{ˈdao̯r.ʔa.ˌni̯ɛd} `to share them'
is significant and is the phonological conditioning environment
by which \cite{he04} analyses Kwara'ae metathesis.

\begin{exe}
	\ex{Kwara'ae multiple metatheses: \hfill\citep[2]{he04}}\label{ex:MulMetKwa}
	\sn{\gw\begin{tabular}{rcll}
		 U\=/form														&			& M\=/form &  \\
		\it{ˈke.\tbr{ta}.ˌla.\tbr{ku}} 	&{\ra}& \it{ˈke̯\tbr{at}.ˌla\tbr{u̯k}} & `my height' \\
		\it{da.ˈ\tbr{ro}.ʔa.ˌni.\tbr{da}}&{\ra}& \it{ˈda\tbr{o̯r}.ʔa.ˌni̯\tbr{ɛd}} & `to share them' \\
	\end{tabular}}
\end{exe}

Metathesis in Kwara'ae often triggers other phonological processes
including glide formation, vowel deletion, and umlaut.
The different phonological processes with which metathesis
is associated are described in \srf{sec:KwaGliFor}--\srf{sec:KwaSum} below.

Published descriptions of Kwara'ae report different
details for some of these phonological processes.
In part these differences may stem from researchers working
with different speakers of different ages.
However, another likely source of variation
is that a single speaker can also use
different M\=/forms depending on speech speed
(Patrick Andrews p.c. February 2015).

In addition to the difference in metathesis,
U\=/forms have the labiodental fricative [f]
where M\=/forms have the voiceless glottal fricative [h] \citep[18]{he04}.

\paragraph{Glide formation}\label{sec:KwaGliFor}
As can be seen from the examples in \qf{ex:KwVCV->VVC} and \qf{ex:MulMetKwa},
when a vowel sequence surfaces in the M\=/form,
the higher vowel is realised as a glide. %\citep[22]{he04}.
If the vowels are of equal height,
as in \it{ˈbo.re} {\ra} \it{ˈbo̯er} `although',
the first vowel is realised as a glide.
\citet[319]{so80} likewise states that metathesised forms consist only of one syllable,
though he does not give rules for which of the underlying vowels surfaces as a glide.

When a word ends in a vowel sequence,
the M\=/form is derived from the U\=/form through glide formation alone.
This is shown in \qf{KwaVV} below:

\begin{exe}
\ex{V\sub{1}V\sub{2} {\ra} V̯\sub{1}V\sub{2}\hfill\citep[13]{he04}}\label{KwaVV}
	\sn{\gw\begin{tabular}{rcll}
		 U\=/form								&			& M\=/form &  \\
		\it{ʔo.ˈd\tbr{o.a}} 	&{\ra}& \it{ˈʔo.d\tbr{o̯a}} & `wall' \\
		\it{ˈd\tbr{o.e}} 			&{\ra}& \it{ˈd\tbr{o̯e}} & `great, big' \\
		\it{ˈne.i.ˌr\tbr{i.a}}&{\ra}& \it{ˈnei̯.ˌr\tbr{i̯ɛ}} & `this one' \\
	\end{tabular}}%
\end{exe}

\paragraph{Vowel deletion}
When a word ends in V\sub{1}V\sub{2}CV\sub{3}{\#},
and V\sub{2} and V\sub{3} are of the same quality,
the first two vowels undergo glide formation and
the final vowel is deleted.
This is shown in \qf{KwaVseq1} below.

\begin{exe}
	\ex{V\sub{1}{\sub{α}}V\sub{2}{\sub{β}}CV\sub{3}{\sub{β}} {\ra} V̯\sub{1}{\sub{α}}V\sub{2}{\sub{β}}C \hfill\citep[27-28]{he04}}\label{KwaVseq1}
	\sn{\gw\begin{tabular}{rcll}
			U\=/form							&			& M\=/form &  \\
			\it{fu.ˈi.r\tbr{i}} &{\ra}& \it{ˈhu̯ir} & `that' \\
			\it{bi.ˈa.l\tbr{a}} &{\ra}& \it{ˈbi̯al} & `smoke' \\
	\end{tabular}}%
\end{exe}

\paragraph{Vowel shift}\label{sec:KwaVoShi}
The low central vowel /a/ has a different quality
after metathesis when the preceding vowel is high.
It is described as schwa [ə] by \citet[315]{so80},
while \citet[23]{he04} describes it as varying between
[ɛ] and [ə] after /i/ and as [ʌ] after /u/.
Examples are given in \qf{Kw2} below.

\begin{exe}
	\ex{V\tsc{[+hi]}Ca {\ra} V̯əC: \hfill\citep[23]{he04}}\label{Kw2}
		\sn{\gw\begin{tabular}{rcll}
			 	U\=/form								&			& M\=/form &  \\
				\it{a.ˈsi.\tbr{la}} 	&{\ra}& \it{ˈa.ˌsi̯\tbr{ɛl} {\tl} ˈa.ˌsi̯\tbr{əl}} & `sweet' \\
				\it{fa.ˈʔu.\tbr{ta}}	&{\ra}& \it{ˈha.ˌʔu̯\tbr{ʌt}} & `which, how, why' \\
		\end{tabular}}
\end{exe}

Likewise, certain combinations of vowel ``fuse'' into
a single vowel rather than a sequence of glide and vowel.
\citet[316]{so80} gives a rule in which /oi/ is realised as [øˑ],
/oe/ as [œˑ], /ae/ as [æˑ] and /ai/ is realised as either [ɛi] or [ɛˑ].
This is similar to the processes of umlaut which have
operated in the Germanic languages (\srf{sec:OriUml}).


\begin{exe}
	\ex{V\sub{α}CV\sub{β} {\ra} V\sub{αβ}C \hfill\citep[316]{so80}}\label{KwVfusionS}
		\sn{\gw\begin{tabular}{rcll}
			 U\=/form														&			& M\=/form &  \\
			\it{m\tbr{o}l\tbr{i}} 						&{\ra}& \it{m\tbr{øˑ}l} 							& `lemon' \\
			\it{as\tbr{o}f\tbr{e}} 						&{\ra}& \it{as\tbr{œˑ}f} 						& `rat' \\
			\it{m\tbr{a}ʔ\tbr{e}t\tbr{a}ʔ\tbr{e}elo}&{\ra}& \it{m\tbr{æˑ}ʔ.t\tbr{æˑ}ʔ.eol} & `doorway' \\
			\it{d\tbr{a}m\tbr{i}} 						&{\ra}& \it{d\tbr{ɛi}m {\tl} d\tbr{ɛˑ}m}	& `gum' \\
		\end{tabular}}
\end{exe}

\citet{he04} does not report front rounded vowels,
but he does report a similar process when the first vowel of the sequence is /a/.
He states that ``[{\ldots}] there is some free variation: if V\sub{2} = [e], [i] or [u],
sometimes the vowel combination can be realized as a single vowel.''
He only gives examples of /ae/ {\ra} [æˑ], /ai/ {\ra} [eˑ] and /au/ {\ra} [oˑ].

\begin{exe}
	\ex{V\sub{α}CV\sub{β} {\ra} V\sub{αβ}C \hfill\citep[24]{he04}}\label{KwVfusionH}
		\sn{\gw\begin{tabular}{rcll}
			 U\=/form										&			& M\=/form &  \\
			\it{ˈs\tbr{a}.t\tbr{e}}		&{\ra}& \it{ˈs\tbr{æ}ˑt} {\tl} \it{ˈs\tbr{ae̯}t} & `chin, beard' \\
			\it{ˈm\tbr{a}.ʔ\tbr{i}}		&{\ra}& \it{ˈm\tbr{eˑ}ʔ} {\tl} \it{ˈm\tbr{ai̯}ʔ} & `come' \\
			\it{li.ˈm\tbr{a}.k\tbr{u}}&{\ra}& \it{ˈli.ˌm\tbr{oˑ}k} {\tl} \it{ˈli.m\tbr{au̯}k} & `my hand' \\
		\end{tabular}}%
\end{exe}

\paragraph{Long vowels}\label{sec:KwaLonVow}
When the penultimate and final vowel of the U\=/form are identical,
\citet{so80}, Pawley and Gegeo (cited in \citealt{blga98}) and \citet{he04}
all transcribe the vowel of the M\=/form as half-long, using the symbol [ˑ].
Other descriptions of Kwara'ae, such as,
\citet{si77} and \citet{trha83} do not transcribe such vowels as long.

\begin{exe}
	\ex{V\sub{α}CV\sub{α} {\ra} V\sub{α}ˑC \hfill\citep[25]{he04}}\label{KwlongV}
		\sn{\gw\begin{tabular}{rcll}
			 U\=/form									&			& M\=/form &  \\
			\it{ˈk\tbr{i}.n\tbr{i}} &{\ra}& \it{ˈk\tbr{iˑ}n} & `female' \\
			\it{ˈm\tbr{a}.n\tbr{a}} &{\ra}& \it{ˈm\tbr{aˑ}n} & `her/his eye' \\
			\it{ˈm\tbr{o}.k\tbr{o}} &{\ra}& \it{ˈm\tbr{ɔˑ}k} & `smell' \\
		\end{tabular}}
\end{exe}

However, as noted by \citet[25]{he04},
no author justifies the use of this half-long mark,
with \citeauthor{he04} indicating that this is a point for further research.
An instrumental phonetic study of Kwara'ae vowels
would probably settle the matter.\footnote{
		For Amarasi I carried out an instrumental study of vowel length
		in which I showed that there is a statistically significant
		difference in length between the penultimate vowel of a U\=/form
		with identical penultimate and final vowels and the final
		vowel of the M\=/form of such words.
		I analyse this difference in length as being due to the M\=/forms
		containing a sequence of two identical vowels.
		(see \srf{sec:QuaLenVowSeq} and \srf{sec:QuaMfoEndVVC}).}
It is also possible that such vowels are long in some contexts and short in others,
depending on variables such as phrasal stress and the rate of speech.

\paragraph{Voiceless vowels}\label{sec:KwaVoiVow}
Optional voiceless vowels also occur after certain consonants in the U\=/form.
\citet[19]{he04} reports such vowels after the consonants [ʔ], [h], [l] and [s].
These vowels do not count as vowels for the purposes of stress assignment,
with stress falling on the penultimate vowel, not counting final voiceless vowels.
After word-final stops, voiceless vowels do not occur,
though the final stop is often strongly aspirated.

\begin{exe}
	\ex{V\sub{1}CV\sub{2} {\ra} V\sub{1}V\sub{2}C{\r*V}\sub{2} \hfill\citep[19]{he04}}\label{KwvoicelessVH}
		\sn{\gw\begin{tabular}{rcll}
			 U\=/form			&			& M\=/form &  \\
			\it{ˈma.ʔu} &{\ra}& \it{ˈmau̯ʔ\tbr{u̥}} & `fear' \\
			\it{ˈʔa.fe} &{\ra}& \it{ˈʔae̯h\tbr{e̥}} & `wife' \\
			\it{ˈbu.su} &{\ra}& \it{ˈbuˑs\tbr{u̥}} & `to burst' \\
			\it{ˈro.do} &{\ra}& \it{ˈrɔˑ\tbr{dʰ}} & `night' \\
			\it{ˈnau̯.ku} &{\ra}& \it{ˈnau̯\tbr{kʰ}} & `I' \\
		\end{tabular}}
\end{exe}

Pawley and Gegeo (cited in \citealt{blga98})
describe voiceless vowels in a wider variety of contexts than is described by \citet{he04}.
According to Pawley and Gegeo, a final voiceless vowel is the usual realisation of words in the M\=/form.
Such vowels only do not occur when there is a word-final nasal
or if the resulting diphthong is a sequence of a high vowel followed by a non-high vowel.

\begin{exe}
	\ex{V\sub{1}CV\sub{2} {\ra} V\sub{1}V\sub{2}C{V̥}\sub{2} \hfill(Pawley and Gegeo in \citealt[530]{blga98})}\label{KwvoicelessVBG}
		\sn{\gw\begin{tabular}{rcll}
			 U\=/form			&			& M\=/form &  \\
			\it{ˈfusi}	&{\ra}& \it{huis\tbr{i̥}} & `cat' \\
			\it{ˈkado}	&{\ra}& \it{kaod\tbr{o̥}} & `thin' \\
			\it{ˈoso}		&{\ra}& \it{oˑs\tbr{o̥}} & `lie' \\
	\end{tabular}}
\end{exe}

According to \citet[20]{he04},
the differences between his data and the data cited by \citet{blga98}
likely comes from working with speakers of different generations.
\citeauthor{he04} states: \emph``[{\ldots}] it's reasonable that her [Kwara'ae consultant's]
speech pattern reflects another stage in the decline of the final vowel.''
%The aspiration of final stops reported by \citet[18]{he04} seems also to
%represent the final remnant of the voiceless vowels reported by Pawley and Gegeo.

\begin{table}[h]
	\caption{Kwara'ae metathesis}\label{tab:KwaMet}
	\begin{tabular}{c|ccccccl}
		\lsptoprule
	V\sub{1}{\da}	&i							&e						&a				&o	&u			&	{\la}V\sub{2}\\			
			\midrule
			i	&{iˑ}										&--						&jɛ, jə		&jo	&ju			&\\
			e	&{ej}										&ɛˑ						&e̯a				&e̯o	&ew			&\\
			a	&{aj, ej, eˑ, (ɛj, ɛˑ)}	&{æ;, ae̯}			&aˑ				&ao̯	&aw, oˑ	&\\
			o	&{oj, (øˑ)}							&o̯e, we, (œˑ)	&o̯a				&ɔˑ	&ow			&\\
			u	&{wi}										&wɛ						&wʌ, (wə)	&--	&uˑ			&\\
		\lspbottomrule
	\end{tabular}
\end{table}

\paragraph{Summary}\label{sec:KwaSum}
The processes with which metathesis in Kwara'ae is associated
include glide formation, umlaut, and vowel deletion.
The effects of deriving the M\=/form on the first
and second vowels of the U\=/form in Kwara'ae are given in \trf{tab:KwaMet}.
This table is adapted from \citep[26]{he04} with qualities reported by \citet{so80} included in brackets.
The symbols used by \citeauthor{he04} for the high vowel glides:
[u̯] and [i̯], have been replaced with the symbols [w] and [j].

\subsubsection{Distribution of metathesis}\label{sec:KwaFun}
U\=/forms and M\=/forms in Kwara'ae belong to different speech registers.
In everyday normal speech the M\=/form is used,
while the U\=/form is used in traditional songs,
for clarification \citep[3]{he04}, and when calling out.
\citet[19]{wage86} report that calling out has three main uses in Kwara'ae discourse:

\begin{quote}
First, people call out for practical reasons in running a household,
such as to locate a missing person or to bring a family member home for a meal.
Secondly, a Kwara'ae man or woman working in the bush and hearing someone
working nearby but out of sight will call out to seek identification of the other person.
Thirdly, people call out from house to house, or as someone passes on the path, as a strictly social activity.
They ask polite questions, or joke, tease, and engage in pleasant banter. \hfill\citep{wage86}
\end{quote}

In addition to the use of unmetathesised forms, calling out is marked
by a special intonation contour and certain emphatic particles.
Two examples of Kwara'ae calling out are given in \qf{KwCallingOut} below.
Note also the extra length on the final syllable of the second form of `father' in example
\qf{KwCallingOut1} as well as the particle \emph{ku} in \qf{KwCallingOut2}.
These two features are also distinctive of calling out.

\begin{exe}\let\eachwordone=\itshape
	\ex{Kwara'ae calling out: \hfill\citep[24,21]{wage86}}\label{KwCallingOut}
		\begin{xlist}
			\ex{
				\gll maʔ! ma\tbr{ʔaːː}! \\
				father{\textbackslash}\tsc{m} father{\tbrU} \\
				\glt `Dad! Da-ad!'\label{KwCallingOut1}}
			\ex{
				\gll Sa\tbr{la}! Sal! Sal ku! lae maiʔ tua hain Mo\tbr{sa}! \\
				Sala{\tbrU} Sala{\M} Sala{\M} \tsc{part} go here stay with:\tsc{3sg.poss} Mosa{\tbrU} \\
				\glt `Sala! Sala! Hey, Sala! Come here and babysit Mosa!' \label{KwCallingOut2}}
		\end{xlist}
\end{exe}

The use of different forms in different speech registers is confirmed by Patrick Andrews
(p.c. February 2015)
who reports that (among other uses) the unmetathesised forms
are used when making a point to a child or to emphasise words in a speech.
He compares the use of the metathesised forms to that of English contractions,
such as \it{couldn't} from \it{could not},
with the former being the everyday form and the latter being used in special circumstances.
This difference in distribution suggests that different forms
are used in different (discourse) pragmatic contexts.

\cite{he04} proposes an analysis of Kwara'ae metathesis framed within Optimality Theory
in which metathesis is conditioned by stress.
Under this analysis, metathesis in Kwara'ae is a response
to the need to make stressed syllables heavy,
with a vowel-glide combination counting as a heavy syllable.
This analysis is discussed in more detail in \srf{sec:ProMorKwa}.

Given that different forms are used in different speech registers,
an analysis of Kwara'ae metathesis as being driven by stress
would predict that different registers have different stress rules.
While it is likely that such a hypothesis would be borne out,
to the best of my knowledge this has not yet been demonstrated.

Nearly every word in Kwara'ae is affected by metathesis.
If it is the case that different speech registers have
different stress patterns, which in turn drives the metathesis,
Kwara'ae has (rampant) phonologically conditioned metathesis
though the phonological conditions triggering metathesis
are themselves driven by the discourse.
