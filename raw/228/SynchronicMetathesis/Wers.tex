\subsection{Wersing}\label{sec:Wer}
Wersing (Trans-New Guinea, Alor) has a process of
synchronic consonant-vowel metathesis.
Based on current data, Wersing appears to have phonologically
conditioned metathesis, though there are indications
that it may also have morphemically conditioned metathesis.

\cite{sche14}, describing the Pureman dialect,
report that the final CV sequence
of a stem metathesises to VC before
either the realis suffix \it{-a} or the specific enclitic \it{=a}.
Examples are shown in \qf{ex:WerMet1} and \qf{ex:WerMet2} below,
in which the second line shows the underlying forms.
In each example the corresponding unmetathesised forms \it{*gə-tati-a}
and \it{*saku=a} are ungrammatical

\begin{multicols}{2}
\let\eachwordone=\itshape
	\begin{exe}
		\ex{\glll ganiŋ wetiŋ gǝ-ta\tbr{it}-a \\
							ganin wetin g-ta\tbr{ti}-a \\
							\tsc{3clsf:hum} five 3-stand-\tsc{rl} \\
				\glt `There are five people standing.'}\label{ex:WerMet1}
		\ex{\glll hans sa\tbr{uk}=a \\
							hans sa\tbr{ku}=a \\
							Hans elder=\tsc{spec} \\
				\glt `Mr. Hans' }\label{ex:WerMet2}
	\end{exe}
\end{multicols}

\cite{ba18}, describing the Kolana dialect,
presents a greater range of data for Wersing metathesis.
Based on \citeauthor{ba18}'s description,
final CV {\ra} VC metathesis is obligatory
for most stems of a certain shape before
a morpheme beginning with a vowel.
\citeauthor{ba18} describes metathesis as only affecting
words in which the penultimate and final vowels are identical
or words in which the final vowel is a high vowel.

Thus, in \qf{ex:bolu} below the noun \it{bolu}
`trumpet shell' occurs unmetathesised before a
consonant-initial verb while in \qf{ex:boul}
the same noun occurs metathesised before a vowel-initial verb.

\begin{exe}
\let\eachwordone=\itshape
	\ex{\gll	ne-pa g-wai bo\tbr{lu} lewena\\
						\tsc{1sg}-father 3-go trumpet.shell{\tbrU} look.for\\
			\glt	`My father goes to look for trumpet shell.'}\label{ex:bolu}
	\ex{\gll	neta bo\tbr{ul} usasi\\
						\tsc{1sg} trumpet.shell{\tbrM} blow\\
			\glt	`I blow a trumpet.'}\label{ex:boul}
\end{exe}

Similarly, in \qf{ex:gadi} the third person
pronoun \it{gadi} occurs unmetathesised
before consonant-initial \it{wuiŋ} `catch' but metathesised
in \qf{ex:gaid} before the vowel-initial word \it{areiŋ} `bury'.

\newpage
\begin{exe}
\let\eachwordone=\itshape
	\ex{\gll	pulis ga\tbr{di} wuiŋ\\
						police \tsc{3sg}{\tbrU} catch\\
			\glt	`The police arrested him.'}\label{ex:gadi}
	\ex{\gll	ni-wai lwen a-miŋ ga\tbr{id} areiŋ \\
						\tsc{1px}-go place \tsc{dist-loc} \tsc{3sg}{\tbrM} bury \\
			\glt	`We went to bury him there.'}\label{ex:gaid}
\end{exe}

Unmetathesised forms do not occur before vowel-initial
morphemes, as shown in \qf{ex:naid1} below in which it
is ungrammatical for unmetathesised \it{nadi} `\tsc{1sg}'
to occur before the vowel-initial demonstrative \it{o-ba}.
Instead, the metathesised form must be used as shown in \qf{ex:nadi1}.

\begin{exe}
\let\eachwordone=\itshape
	\ex{\begin{xlist}
		\ex[*]{\gll	na\tbr{di} o-ba Wersiŋ ge-aniŋ\\
								\tsc{1sg}{\tbrU} \tsc{prox-dem} Wersing 3-person \\}\label{ex:naid1}
		\ex[]{\gll	na\tbr{id} o-ba Wersiŋ ge-aniŋ\\
								\tsc{1sg}{\tbrM} \tsc{prox-dem} Wersing 3-person \\
					\glt	`I am a Wersing person.' \txrf{}}\label{ex:nadi1}
	\end{xlist}}\label{ex:naid}
\end{exe}

Similarly, metathesised forms cannot usually be used before
consonant-initial morphemes, as shown in \qf{ex:nadi2}
in which it is ungrammatical for metathesised \it{naid} `\tsc{1sg}'
to occur before the demonstrative \it{ba}.
Instead, the unmetathesised form \it{nadi} must be used,
as shown in \qf{ex:naid2}.

\begin{exe}
\let\eachwordone=\itshape
	\ex{\begin{xlist}
		\ex[*]{\gll	na\tbr{id} ba Wersiŋ ge-anin obo\\
								\tsc{1sg}{\tbrM} \tsc{dem} Wersing 3-person this \\}\label{ex:nadi2}
		\ex[]{\gll	na\tbr{di} ba Wersiŋ ge-anin obo\\
								\tsc{1sg}{\tbrU} \tsc{dem} Wersing 3-person this \\
					\glt	`I am a Wersing person.' \txrf{}}\label{ex:naid2}
	\end{xlist}}
\end{exe}

Metathesis in Wersing apparently does not affect stems
which end in /a/ with a different penultimate vowel.
Thus, the \tsc{1sg} pronoun \it{neta}
and \tsc{1px} pronoun \it{nita} are reported to only
have a single (unmetathesised) form.
However, words in which both the penultimate and final
vowels are /a/ do have metathesised forms.
\cite{ba18} gives the example of \it{kana} {\ra} \it{kaan} `already, \tsc{pfv}'.

Metathesis in Wersing thus appears to be an automatic
process which affects most CV{\#} final words
when they occur before another vowel.
This is similar to Amarasi metathesis before vowel-initial enclitics
which can be analysed as a phonologically
conditioned process (see Chapter \ref{ch:PhoMet}).\footnote{
		If the distribution of metathesised and unmetathesised
		forms in Wersing is predictable and in complementary distribution,
		it would be impossible to determine which of the CV or VC
		final form of metathesising words is underlying.}

Finally, \cite{ba18} also shows that metathesis
occurs in other environments in Wersing.
Thus, the word \it{akumi} `group' 
is obligatorily metathesised before the quantifiers \it{ba} and \it{tme} [təmɛ],
as shown in \qf{ex:akuim1} and \qf{ex:akuim2} respectively.

\begin{multicols}{2}
\let\eachwordone=\itshape
	\begin{exe}
		\ex{\begin{xlist}
			\ex{\gll	aniŋ aku\tbr{im} ba\\
								person group{\tbrM} one\\
					\glt	`The group of people.' \txrf{}}
			\ex[*]{\gll	aniŋ aku\tbr{mi} ba\\
								person group{\tbrU} one\\
					\glt	`(The group of people.)' \txrf{}}
		\end{xlist}}\label{ex:akuim1}
	\end{exe}
\end{multicols}
\begin{multicols}{2}
\let\eachwordone=\itshape
	\begin{exe}		
		\ex{\begin{xlist}
			\ex{\gll	g-niŋ aku\tbr{im} tme\\
								3-person group{\tbrM} some\\
					\glt	`Some group of people.' \txrf{}}
			\ex[*]{\gll	g-niŋ aku\tbr{mi} tme\\
								3-person group{\tbrU} some\\
					\glt	`(Some group of people.)' \txrf{}}
		\end{xlist}}\label{ex:akuim2}
	\end{exe}
\end{multicols}

Similarly, \it{lomu} {\ra} \it{loum} `say' must occur
metathesised before the demonstrative \it{ba}
or the aspectual marker \it{kana} `already, \tsc{pfv}',
as shown in \qf{ex:loum1} and \qf{ex:loum2} below.

\begin{multicols}{2}
\let\eachwordone=\itshape
	\begin{exe}
		\ex{\begin{xlist}
			\ex{\gll	ge-lo\tbr{um} ba lewois obo!\\
								3-say{\tbrM} \tsc{dem} listen this\\
					\glt	`Listen to his saying!' \txrf{}}
			\ex[*]{\gll	ge-lo\tbr{mu} ba lewois obo!\\
								3-say{\tbrU} \tsc{dem} listen this\\
					\glt	`(Listen to his saying!)' \txrf{}}
		\end{xlist}}\label{ex:loum1}
	\end{exe}
\end{multicols}
\begin{multicols}{2}
\let\eachwordone=\itshape
	\begin{exe}		
		\ex{\begin{xlist}
			\ex{\gll	neta looro lo\tbr{um} kana\\
								\tsc{1sg} right say{\tbrM} \tsc{prf}\\
					\glt	`I've said it right.' \txrf{}}
			\ex[*]{\gll	neta looro lo\tbr{mu} kana\\
								\tsc{1sg} right say{\tbrU} \tsc{prf}\\
					\glt	`(I've said it right.)' \txrf{}}
		\end{xlist}}\label{ex:loum2}
	\end{exe}
\end{multicols}

The basis for metathesis in examples such as
\qf{ex:akuim1}--\qf{ex:loum2} is not entirely clear.
This may be a case of morphemically conditioned metathesis,
though more data is needed on Wersing to determine this.