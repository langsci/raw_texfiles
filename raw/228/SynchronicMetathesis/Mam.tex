%Fogaça 2017:121ff
%Metathesis is a phonological process that occurs by simply changing the order of sounds that occurs in a word. In Mambae this change occurs in the last syllable CV that becomes VC in some environments. In Mambae Sul the metathesis can be conditioned phonologically, morphologically or syntactically. In this section we discuss the metathesis as a phonological process and the process of assimilation that occurs on account of this. In the following chapters the process of conditioned metathesis will be approached morphologically and syntactically.
%Table 4.13
%It is observed in the table above that the first column refers to the words with the non-metathesis form (NM\=/form) CV and the second column the metathesis form (M\=/form) occurs by transforming the last syllable into VC. This process is clear from the first five data in the table.
%In the last five data of the table, it is observed that in the -M form it causes a sequence of identical VV vowels to occur. However, this is not a homogeneous process. In the sixth and seventh data, the sequence of identical vowels occurs by the simple occurrence of the process of metathesis. Already in the eighth, ninth and tenth data, this identical VV sequence occurs due to the assimilation process that occurs.
%This process of assimilation is restricted, and only occurs with words ending in /a/ when it undergoes the process of metathesis, in which this vowel /a/ tends to assimilate the traits of the previous vowel, becoming identical to the first vowel of the sequence VV.
%Table 4.14
%It is observed that there are in the language other VV sequences in which the /a/ occurs as a second vowel. The assimilation process described above is restricted to M-shaped words

%Fogaça 2017:126f
%A unique feature of the South Mambae is the process of Metastasis, which does not occur in the Northwest and Northeast Central Mambae. Crowley (1987: 34) defines Metathesis as the "simple change of the order that sounds occur". Among the Mambae languages, what occurs is usually the inversion of the last syllable in the South Mambae, as shown in the above data. The phenomenon of metathesis in the Mambae Sul language is much more complex and will be analyzed in the next chapters of grammatical description.
%In the North-Central Mambae it is evident the addition of the suffix {-a} at the end of common names in metatesificado forms, resilabificando the words. Considering that in the Mambae the words are predominantly paroxítonas, there is the change of the accent of these, that they become trissílabas, but with accentuation in the second syllable. According to Blust (2009, page 633), this morpheme has also been identified in the language Kiandarat by Collins (1982, p.111 apud Blust, 2009, p.663), but has not yet been identified or clarified, a synchronic and diachronic analysis of this phenomenon within the Austronesian languages.

%Fogaça 2017:136
%In Mambae the process of metathesis has different functions. In morphology it has a function in the construction of class of words (BURQUEST, 2006, p 176). The NM-shape defines the names, while the M-shape determines the verbs.
%Table 5.3
%It is observed that in the data of table 5.2, this process occurs with verbs related to abstract names, including a lexical loan of the Portuguese language (through the Tetun language), kuidadu, that goes through the process of metathesis to generate the verb kuidaud 'caring '.
%There are also two occurrences of metathesis with a 'near' adverb and a 'broken' adjective.

%Fogaça 2017:146
%The Mambae Northeast-Central and Northwest varieties mark the inalienable possession with the genitive suffix {-n} postfixed in the thing possessed - a mark of possession commonly found in the Austronesian languages (BLUST, 2005, p.215). However, in the South Mambae, the inalienable possession is marked by the Non-Metátese form of the name possessed. The possessor may be an assigning possessive name or pronoun (see §6.2.2).
%Within the Mambae group the concept of property may be inalienable, such as the case of a house and some trees (especially fruit trees - whoever plants becomes the owner of that tree). The difference in examples 245 and 246 is observed when there is or is no specification of who owns the house.

\subsection{Mambae}\label{sec:Mam} %Fogaça page:121ff, 126f, 136 (n.→v.), 146 (poss.), 
\il{Mambae|(}Mambae is an Austronesian language/dialect
cluster spoken in Timor-Leste (East Timor),
from the north coast around Dili all the way to the south coast (see \frf{fig:CVMetTimReg}).
On the basis of lexical comparison, \citet{fo17}
identifies three main varieties of Mambae:
Northwest Mambae, Central Mambae, and South Mambae.
The forms and functions of metathesis
vary between different varieties of Mambae.

In this section I focus on South Mambae
from the village (\it{suco}) of Letefoho;
the variety which is the focus of the descriptions in \cite{gr14} and \cite{fo17}.
I offer some initial observations on metathesis in other varieties
in \srf{sec:OthVarMam} based on the comparative data in \cite{fo17}.

\subsubsection{Forms}
Metathesis in South Mambae is final CV {\ra} VC metathesis.
Only words ending in CV have distinct U\=/forms and M\=/forms.
Examples of metathesis in South Mambae are given in \qf{ex:VCV->VVC-Mam} below.

\begin{exe}
	\ex{South Mambae metathesis \hfill\citep[122]{fo17}}\label{ex:VCV->VVC-Mam}
	\sn{\gw\begin{tabular}{lcll}
			U\=/form						&		&M\=/form	& \\
			\it{e\tbr{tu}}		&\ra&\it{e\tbr{ut}}		&`rice' \\
			\it{ma\tbr{ne}}		&\ra&\it{ma\tbr{en}}	&`male, man' \\
			\it{da\tbr{to}}		&\ra&\it{da\tbr{ot}}	&`nobleman' \\
			\it{bru\tbr{si}}	&\ra&\it{bru\tbr{is}}	&`hot' \\
			\it{ko\tbr{de}}		&\ra&\it{ko\tbr{ed}}	&`good' \\
			\it{fa\tbr{ta}}		&\ra&\it{fa\tbr{at}}	&`four' \\
			\it{fu\tbr{tu}}		&\ra&\it{fu\tbr{ut}}	&`together' \\
%			\it{\tbr{}}	&\ra&\it{\tbr{}}	&`' \\
%			\it{\tbr{}}	&\ra&\it{\tbr{}}	&`' \\
	\end{tabular}}
\end{exe}

Metathesis is associated with two other phonological processes.
The first is assimilation of final /a/
to the quality of the previous vowel after metathesis.
This is shown in \qf{ex:VCa->VC-Mam} below,
which also gives reconstructed Proto-Malayo-Polynesian (PMP) forms for comparison.
U\=/forms which are not (yet) attested are indicated with an asterisk.
Such assimilation also occurs in Amarasi.

\begin{exe}
	\ex{South Mambae V{\sub{α}}Ca {\ra} V{\sub{α}}V{\sub{α}}C \hfill\citep[122]{fo17}}\label{ex:VCa->VC-Mam}
	\sn{\gw\begin{tabular}{llcll}
			PMP		&U\=/form									&		&M\=/form	& \\
			*ina		&\it{\hp{*}i\tbr{na}}		&\ra&\it{i\tbr{in}}		&`mother' \\
%							&{\hp{*}i\tbr{da}}			&\ra&\it{i\tbr{id}}		&`one' \\
			*binai	&\it{\hp{*}hi\tbr{na}}	&\ra&\it{hi\tbr{in}}	&`woman' \\
%							&\it{*ku\tbr{da}}				&\ra&\it{ku\tbr{ud}}	&`horse' \\
			*Rumaq	&\it{\hp{*}u\tbr{ma}}		&\ra&\it{u\tbr{um}}		&`house' \\
			*quzan	&{*u\tbr{sa}}						&\ra&\it{u\tbr{us}}		&`rain' \\
%			\it{\tbr{a}}	&\ra&\it{\tbr{}}	&`' \\
%			\it{\tbr{a}}	&\ra&\it{\tbr{}}	&`' \\
	\end{tabular}}
\end{exe}

Assimilation of final /a/ only occurs after
metathesis and /a/ freely occurs as the second
member of a vowel sequence in VVC{\#} final words.
Two examples are \it{kiak} `poor' and \it{hean} `rowing'.

Secondly, word-final /i/ usually lowers to /e/ after metathesis
when the penultimate vowel is /a/.
Examples are given in \qf{ex:aCi->aeC-Mam} below.
Where the U\=/form is not (yet) known to occur,
this is indicated with an asterisk.

\begin{exe}
	\ex{South Mambae aCi {\ra} aeC }\label{ex:aCi->aeC-Mam}
	\sn{\gw\begin{tabular}{llcll}
			PMP			&	\mc{2}{l}{\hp{*}U\=/form}			&M\=/form			& \\
			*talih	&\it{\hp{*}ta\tbr{li}}	&{\ra}&\it{ta\tbr{el}}	&`rope' \\
			*kami		&\it{\hp{*}a\tbr{mi}}		&{\ra}&\it{a\tbr{em} {\tl} \it{a\tbr{im}}}	&`we (excl.)' \\
			*babuy	&		{*ha\tbr{hi}}				&{\ra}&\it{ha\tbr{eh}}	&`pig' \\
			*hapuy	&		{*a\tbr{fi}}				&{\ra}&\it{a\tbr{ef}}		&`fire' \\
			*tasik	&		{*ta\tbr{si}}				&{\ra}&\it{ta\tbr{es}}	&`sea' \\
%			*	&\it{\hp{*}}	&{\ra}&\it{}	&`' \\
%			*	&\it{\hp{*}}	&{\ra}&\it{}	&`' \\
	\end{tabular}}%
\end{exe}

Again, lowering of /i/ {\ra} /e/ in South Mambae
is restricted to M\=/forms.
The vowel /i/ freely occurs as the
second member of a vowel sequence in VVC{\#}
final roots, such as \it{araik} `lower, humble',
\it{tais} `no, not' and \it{sabai} {\tl} \it{sabait} `cloud'.\footnote{
		The sequence /ai/ is often realised [əi]
		with the first vowel centralised \citep[6]{gr14}.}

The processes of vowel assimilation which
occur after metathesis in Mambae show
that the M\=/form is derived from the U\=/form and not visa versa.
While in all cases the M\=/form can be predicted
with knowledge of the U\=/form, the reverse is not true.
Thus, given an M\=/form such as \it{hiin} `woman'
we can generate both the correct U\=/form \it{hina}
and incorrect \it{*hini}. Similarly, given the
M\=/form \it{tael} `rope' both correct \it{tali}
and incorrect \it{*tale} are possible U\=/forms

\subsubsection{Functions}\label{sec:MamFun}
Metathesis has three main functions in South Mambae:
derivation, phrase formation, and possession.
Each of these functions is discussed in turn.
All these functions are examples of morphological metathesis (\srf{sec:MorMet}).

Formally, the M\=/form is derived from the U\=/form.
However, the M\=/form appears to be the default semantic
form with the U\=/form having specific functions.
Thus, for instance, most words are cited in the M\=/form
and many words have only been attested in the M\=/form.

\paragraph{Derivation}
Metathesis is also used in derivation in Mambae.
One productive derivational use is noun/verb derivation.
Examples are given in \qf{ex:MamVerDer} below.
If the M\=/form is the basic semantic form
as suggested above, this would be a process of nominalisation.

\newpage
\begin{exe}
	\ex{South Mambae derivation \hfill\cite[136]{fo17}}\label{ex:MamVerDer}
	\sn{\gw\begin{tabular}{llll}
			& M\=/form 	& U\=/form 	&	\\
		`die, be dead'	&\it{ma\tbr{et}}	&\it{ma\tbr{te}}	&`death'	\\
		`love' (v.)	&\it{do\tbr{im}}	&\it{do\tbr{mi}}	&`love' (n.)	\\
		`live'	&\it{mo\tbr{ir}}	&\it{mo\tbr{ri}}	&`life'	\\
		`teach'	&\it{no\tbr{ir}}	&\it{no\tbr{ri}}	&`teaching, lesson'	\\
		`take care'	&\it{kuida\tbr{ud}}	&\it{kuida\tbr{du}}	&`care, caution'	\\
		`approach'	&\it{fede\tbr{is}}	&\it{fede\tbr{si}}	&`near, close'	\\
		`break'	&\it{a\tbr{of}}	&\it{a\tbr{fo}}	&`broken'	\\
	%	`'	&\it{\tbr{}}&\it{\tbr{}}&`'\\
	%	`'	&\it{\tbr{}}&\it{\tbr{}}&`'\\
	\end{tabular}}%
\end{exe}

That the loanword \it{kuidadu} `care, caution'
(from Portuguese \it{cuidado}) also has a
verbal and nominal form derived by metathesis
is evidence that this is
a productive process in South Mambae.

In addition to such verb/noun pairs, there are a number
of U\=/form/M\=/form pairs which are semantically
and/or historically related but for which the M\=/form is not a verb.
Examples include \it{li\tbr{ma}} `hand, arm' \it{li\tbr{im}} `five',
\it{to\tbr{na}} `year, age, birthday' \it{to\tbr{on}} `year',
and \it{mu\tbr{na}} `long ago, previously' \it{mu\tbr{un}} `before'.

Examples of uses of the U\=/form and M\=/form
of \it{mate} {\tl} \it{maet} `die, be dead; death' are given in \qf{ex:MamNar} below,
an excerpt from a narrative about the war for independence in Timor-Leste.
The U\=/forms appear to be used in a more active (process)
sense while the M\=/forms are used in a more stative (result) sense.
This text was collected during a 2012 language documentation workshop (see \srf{sec:Meth}).

\begin{exe}\let\eachwordone=\itshape
	\ex{South Mambae narrative:}\label{ex:MamNar}
		\begin{xlist}
		\ex{\gll	mas ni momentu kidura\\
							but \tsc{loc} time \tsc{distal}\\
				\glt `But at that time,'}
		\ex{\gll	artuub rini fe ma\tbr{te}\\
							person many \tsc{rel} die{\tbrU} \\
				\glt `many people died.'}
		\ex{\gll	man tilu ni ai lala met ma\tbr{te}\\
							like currently \tsc{loc} tree inside also die{\tbrU} \\
				\glt `(It was the) same in the jungle (they) also died.'}
		\ex{\gll	maa rende telo met ma\tbr{te}\\
							come surrender finish also die{\tbrU} \\
				\glt `(They) came and surrendered and also died.'}
		\ex{\gll	i artuub rini fe ma\tbr{te}\\
							then person many \tsc{rel} die{\tbrU} \\
				\glt `And many people died.'}
		\ex{\gll	ni uum seer ma\tbr{et}, ni familia seer ma\tbr{et} met\\
							\tsc{loc} house{\M} several die{\tbrM} \tsc{loc} family several die{\tbrM} also\\
				\glt `Several were dead in a house, several were also dead in a family.'}
		\ex{\gll	ubu kiid fe mori\\
							\tsc{classifier} one \tsc{rel} live\\
				\glt `(Maybe only) one person lived.'}
		\ex{\gll	ma\tbr{et} ba loos deslaa kilat hua\\
							die{\tbrM} \tsc{neg} truly because weapon fruit\\
				\glt `Dead not because of rifle bullets,'}
		\ex{\gll	mas ma\tbr{et} deslaa moras, i namaa ba nei\\
							but die{\tbrM} because sick and food \tsc{neg} \tsc{exist}\\
				\glt `but dead because of sickness and lack of food.'}
	\end{xlist}
\end{exe}

\paragraph{Phrase formation}\label{sec:PhrFor}
Metathesis in South Mambae plays a role in compounding
and other phrase formation processes.
The first element of a phrase tends
to occur in the M\=/form and the final element in the U\=/form.
Examples of unmetathesised words phrase finally
are given in \qf{ex:MamPhrFor} below.
All these words are metathesised in the citation form.

\begin{exe}
	\ex{South Mambae phrase formation \hfill\cite{gr14}}\label{ex:MamPhrFor}
	\sn{\gw\begin{tabular}{rlll}
			citation				&phrase									&gloss						&trans \\
			\it{hi\tbr{in}}	& \it{aan hi\tbr{na}}		& child female		&`girl, daughter'\\
			\it{hi\tbr{in}}	& \it{taes hi\tbr{na}}	& sea female			&`north coast'\\
			\it{ma\tbr{en}}	& \it{taes ma\tbr{ne}}	& sea male				&`south coast'\\
			\it{ha\tbr{ut}}	& \it{ulu ha\tbr{tu}}		& head stone			&`head, skull'\\
			\it{ i\tbr{id}}	& \it{liim nai ni\tbr{da}}	& five and one				&`six'\\
			\it{te\tbr{ul}}	& \it{liim nai te\tbr{lu}}	& five and three			&`eight'\\
			\it{fa\tbr{at}}	& \it{liim nai fa\tbr{ta}}	& five and four				&`nine'\\
		\end{tabular}}
\end{exe}

It is not a strict rule that phrase-final elements are always in the U\=/form.
Thus, in addition to \it{taes hi\tbr{na}} `north coast'
and \it{taes ma\tbr{ne}} `south coast',
we also find \it{taat hi\tbr{in}} `grandmother' and \it{taat ma\tbr{en}} `grandfather'.
The degree of lexicalisation may play a role,
with lexicalised phrases occurring with M\=/form initial elements
and U\=/form final elements.

\paragraph{Direct possession}\label{sec:InaPos}
Metathesis also plays a role in possessive constructions.
South Mambae has two different possessive constructions:
indirect possession and direct possession.
Indirect possession is expressed with the possessive particle \it{ni}.
The order is either possessor-\it{ni}-possessum,
as in \qf{ex:MamAliPos1}, or possessum-possessor-\it{ni},
as in \qf{ex:MamAliPos2}.

\begin{exe}\let\eachwordone=\itshape
	\ex{\gll	au fliik Euriko \tbr{ni} tero ni uri.\\
						\tsc{1sg} hear Euriko \tbr{\tsc{poss}} voice \tsc{loc} here\\
			\glt	`I heard Euriko's voice here'}\label{ex:MamAliPos1}
	\ex{\gll	{\ldots} tradisaun een la Same \tbr{ni}\\
						{} tradition about to Same \tbr{\tsc{poss}} \\
			\glt	\lh{\ldots}`about the traditions of Same'
						\hfill\citep[145]{fo17}}\label{ex:MamAliPos2}
\end{exe}

Direct possession is expressed
by the possessor occurring before a U\=/form possessum.
Compare the examples in \qf{ex:MamPos1a}--\qf{ex:MamPos2b} below.
In \qf{ex:MamPos1a} the noun \it{mane} `male, man, husband'
is not possessed and occurs in the M\=/form while
in \qf{ex:MamPos1b} the same noun is possessed and
thus occurs in the U\=/form.
Similarly, in \qf{ex:MamPos2a} \it{uma} `house'
occurs unpossessed and in the M\=/form
while in \qf{ex:MamPos2b} it occurs possessed and in the U\=/form.

\begin{exe}\let\eachwordone=\itshape
	\ex{\gll	ma\tbr{en} idura la universidadi.\\
						man{\tbrM} \tsc{proximal} to university \\
			\glt	`This man goes to university.' }\label{ex:MamPos1a}
	\ex{\gll	ura ma\tbr{ne} la universidadi.\\
						\tsc{3sg} man{\tbrU} to university \\
			\glt	`Her husband goes to university.'}\label{ex:MamPos1b}
	\ex{\gll	au laa u\tbr{um} \\
						\tsc{1sg} go house{\tbrM}\\
			\glt	`I'm going to a house.' }\label{ex:MamPos2a}
	\ex{\gll	au laa au u\tbr{ma} \\
						\tsc{1sg} go \tsc{1sg} house{\tbrU}\\
			\glt	`I'm going to my house.' \hfill\citep[146f]{fo17}}\label{ex:MamPos2b}
\end{exe}

%\cite{fo17} identifies three uses of inalienable possession:
%ownership, part-whole relations, and kin relations.
Use of the U\=/form for the possessum in direct
possessive constructions is similar
to the use of U\=/forms in phrase formation.
In both constructions the U\=/form is used as the second element of the phrase.

One possible reason for this similarity
is that the phrases with final U\=/forms
are (or were) originally possessive constructions.
This approach is hinted at by \cite[147]{fo17} who
identifies one function of direct possession
as expressing a part-whole relation,
as in \it{haeh sisa} `pig meat' = `pork' and \it{ai tia} `tree skin' = `bark'.

\citet[146]{fo17} reports that direct possession in Northwest Mambae and Central Mambae
is expressed with the suffix \it{-n} on the possessum,
examples of which occur in the citation form of several
body parts and kin terms in \citeauthor{fo17}'s appended wordlists.
Examples from Northwest Mambae (Railaco sub-district) include
\it{gugu-n} `mouth', \it{lima-n} `arm', and \it{ina-n} `mother'.

Given that CVC{\#} final words do not have M\=/forms in Mambae,
the use of U\=/forms in direct possession in South Mambae
appears to have arisen from the possessum originally taking the suffix \it{-n},
thus being consonant final and ineligible to undergo metathesis.
After loss of the suffix the only signal of possession was the U\=/form.

\subsubsection{Other varieties of Mambae}\label{sec:OthVarMam}
\cite{fo17} also presents some survey data
from other varieties of Mambae on the basis of which
it is possible to make some preliminary observations on
differences in metathesis among varieties of Mambae.

The citation forms of a number of common nouns from
seven different varieties of Mambae are given in \trf{tab:ComNouDifMamVar}
for comparison. These forms are taken from the comparative wordlists
in \cite{fo17}, with phonetically long vowels retranscribed
as double according to their phonemic structure
and the putative nominal suffix \it{-a} separated by a hyphen.\footnote{
	Although the difference between the mid-high vowels [e] [o] and mid-low [ɛ] [ɔ]
	is known not to be phonemic in South Mambae, I have maintained \citeauthor{fo17}'s
	distinction between these vowels in \trf{tab:ComNouDifMamVar} as their status
	in other varieties is not known.}

\begin{table}[ht]
	\caption[Common nouns in different Mambae varieties]
	{Common Nouns in Different Mambae Varieties\su{†}}\label{tab:ComNouDifMamVar}
		\begin{threeparttable}[b]
	\stl{0.25em}
	\begin{tabular}{llllllllll}\lsptoprule
PMP	&	*talih	&	*wani	&	*hapuy	&	*babuy	&	*bituqən	&	*batu	&		&\\
P. Mam.	&	*tali	&	*ani	&	*api	&	*hahi	&	*hitu	&	*hatu	&	*gelu	&	*neru\\ \midrule
Letefoho	&	\it{\hp{*}tael}	&	\it{\hp{*}aen}	&	\it{\hp{*}aɛf}	&	\it{\hp{*}haɛh}	&	\it{\hp{*}hiut}	&	\it{\hp{*}haut}	&	\it{\hp{*}keul}	&	\it{\hp{*}neor}\\
Betano	&	\it{\hp{*}taɛl}	&	\it{\hp{*}aɛn}	&	\it{\hp{*}aɛp}	&	\it{\hp{*}haɛh}	&	\it{\hp{*}hiit}	&	\it{\hp{*}haat}	&	\it{\hp{*}geel}	&	\it{\hp{*}neer}\\
Hatu-U.	&	\it{\hp{*}taal}	&	\it{\hp{*}aan}	&	\it{\hp{*}aap}	&	\it{\hp{*}hae}	&	\it{\hp{*}hiit}	&	\it{\hp{*}haat}	&	\it{\hp{*}kɛɛl}	&	\it{\hp{*}neer}\\ \hline
Laulara	&	\it{\hp{*}tail-a}	&	\it{\hp{*}ain-a}	&	\it{\hp{*}aif-a}	&	\it{\hp{*}haih-a}	&	\it{\hp{*}hiut-a}	&	\it{\hp{*}haut-a}	&	\it{\hp{*}keul-a}	&	\it{\hp{*}neur-a}\\
Aileu V.	&	\it{\hp{*}tael-a}	&	\it{\hp{*}aen-a}	&	\it{\hp{*}aif-a}	&	\it{\hp{*}haih-a}	&	\it{\hp{*}hiut-a}	&	\it{\hp{*}haut-a}	&	\it{\hp{*}keul-a}	&	\it{\hp{*}neur-a}\\
Hatu-B.	&	\it{\hp{*}tail-a}	&	\it{\hp{*}bani}	&	\it{\hp{*}aif-a}	&	\it{\hp{*}haih-a}	&	\it{\hp{*}heut-a}	&	\it{\hp{*}haut-a}	&	\it{\hp{*}keol-a}	&	\it{\hp{*}niur-a}\\
Liquidoe	&	\it{\hp{*}tael-a}	&	\it{\hp{*}ain-a}	&	\it{\hp{*}aif-a}	&	\it{\hp{*}haɛh-a}	&	\it{\hp{*}hiut-a}	&	\it{\hp{*}haut-a}	&	\it{\hp{*}kiul-a}	&	\it{\hp{*}niur-a}\\  \hline
Railaco	&	\it{\hp{*}taɛl-a}	&	\it{\hp{*}aɛn-a}	&	\it{\hp{*}aɛp-a}	&	\it{\hp{*}hɛh-a}	&	\it{\hp{*}hiut-a}	&	\it{\hp{*}hato}	&	\it{\hp{*}gelo}	&	\it{\hp{*}nero}\\
Hatulia	&	\it{\hp{*}tail-a}	&	\it{\hp{*}aen-a}	&	\it{\hp{*a}ep-a}	&	\it{\hp{*}heh-a}	&	\it{\hp{*}hito}	&	\it{\hp{*}hato}	&	\it{\hp{*}gelo}	&	\it{\hp{*}nero}\\
Barzatete	&	\it{\hp{*}taɛl-a}	&	\it{\hp{*a}ɛn-a}	&	\it{\hp{*a}ɛp-a}	&	\it{\hp{*}hɛh-a}	&	\it{\hp{*}hito}	&	\it{\hp{*}hato}	&	\it{\hp{*}gelo}	&	\it{\hp{*}nɛro}\\
	&	\hp{*}`rope'	&	\hp{*}`bee'	&	\hp{*}`fire'	&	\hp{*}`pig'	&	\hp{*}`star'	&	\hp{*}`stone'	&	\hp{*}`wind'	&	\hp{*}`knife'\\ \lspbottomrule
	\end{tabular}
			\begin{tablenotes}
				\item [†]	Mambae Varieties are: South Mambae
									from Letefoho and Betano villages (Same sub-district)
									and Hatu-Udo sub-district,
									Central Mambae from the sub-districts of
									Laulara, Ailei Vila, Hatu-Builico and Liquidoe,
									and Northwest Mambae from the sub-districts
									of Railaco, Hatulia, and Barzatete. PMP reconstructions
									are from \citet{bltr}. Proto-Mambae
									reconstructions are my own.
			\end{tablenotes}
		\end{threeparttable}
\end{table}

\trf{tab:ComNouDifMamVar} shows that a suffix \it{-a}
frequently occurs on common nouns in Central Mambae and Northwest Mambae.
This suffix is not synchronically attested in South Mambae.
Before this suffix CV{\#} final words obligatorily undergo metathesis \citep[126]{fo17}.\footnote{
		When the suffix \it{-a} occurs on VV{\#} final words, no change occurs.
		An example is \it{ai-a} `tree'.}
A similar pattern occurs in Amarasi in which metathesis is obligatory before
vowel-initial enclitics (Chapter \ref{ch:PhoMet}).\footnote{
		One vowel-initial enclitic in Amarasi is the nominal determiner
		\ve{=aa}, which may well be cognate with the Mambae nominal suffix \it{-a}.}

The data in \trf{tab:ComNouDifMamVar} also show a number of differences
in the forms of metathesis between different varieties of Mambae.
The main differences are in the kinds of vowel assimilation which occur.
This ranges from no assimilation (apart from final /a/)
in Laulara to complete assimilation of nearly all vowels in Hatu-Udo,
as well as varieties part way between these two such as Letefoho in which
all vowels usually undergo complete assimilation,
apart from /i/ which lowers to /e/ after /a/.\footnote{
		\trf{tab:ComNouDifMamVar} only shows assimilation of high vowels.
		Assimilation of mid vowels also occurs, as seen in Hatu-Udo
		\it{ma\tbr{ne}} {\ra} \it{ma\tbr{an}} `man, male' and
		\it{le\tbr{lo}} {\ra} \it{le\tbr{el}} `sun'.}
Another kind of vowel assimilation is
height assimilation, seen in Hatu-Builico
and Liquidoe in which penultimate mid vowels
(optionally) raise to high before another high vowel
after metathesis, such as in *n\tbr{e}ru {\ra} \it{n\tbr{i}ur-a} `knife'.
Finally, Northwest Mambae has monophthongisation
of the vowel sequence *ai created by metathesis.
One example of is *hahi > \it{h\tbr{ai}h-a} > \it{h\tbr{e}h-a} `pig'.\footnote{
		More data will probably show that the best analysis of
		Northwest Mambae \it{heh-a} `pig' and \it{ep-a} `fire'
		is actually \it{heeh-a} and \it{eep-a} with an underlying sequence of two identical vowels.
		This is also likely for other apparent disyllables ending in /a/,
		such as Northwest Mambae [uta] `louse' or [noa] `coconut'.
		The behaviour of other forms such as \it{tali} {\ra} \it{tail-a} `rope'
		and  \it{ai} {\ra} \it{ai-a} `tree' which unambiguously undergo metathesis and/or
		preserve the original vowel sequence,
		as well as data from other varieties with the forms
		\it{utun}, \it{uut} `louse' and \it{noo} `coconut',
		strongly indicates that such forms
		are underlying \it{uut-a} and \it{noo-a} respectively.}

\subsubsection{Summary}
There are many similarities between metathesis in Mambae and Amarasi.
Of all the cases of metathesis discussed in this chapter,
Mambae metathesis has the most similarities to Amarasi metathesis.
Firstly, in both Mambae and Amarasi final /a/ also assimilates to the quality
of the previous vowel after metathesis.

Secondly, in both Mambae and Amarasi metathesis interacts
with the formation of nominal phrases.
In Mambae non-final members of a nominal phrase tend to occur metathesised
and final members tend to occur unmetathesised.
In Amarasi this is not a tendency but a rule of the grammar.
Amarasi metathesised nominals are a construct form used before attributive modifiers,
and unmetathesised nominals are used phrase finally (Chapter \ref{ch:SynMet}).

Thirdly, in both Mambae and Amarasi U\=/forms are associated
with nouns and M\=/forms with verbs.
In Mambae this can be seen in derivationally
related pairs such as \it{moir} `live' and \it{mori} `life'.
In Amarasi the default form of nominals is the U\=/form
and the default form of verbs the M\=/form (\srf{sec:DefFor1}).

Finally, metathesis in Northwest and Central Mambae
is obligatory before the nominal suffix \it{-a}.
Similarly, in Amarasi metathesis is obligatory before
vowel-initial enclitics.

The only feature of Mambae metathesis which does
not have a close parallel in Amarasi is the use
of U\=/form nouns in direct possession in South Mambae.
However, as discussed in \srf{sec:InaPos} this feature,
may be a recent development arising from loss
of earlier genitive \it{-n}.
\il{Mambae|)}