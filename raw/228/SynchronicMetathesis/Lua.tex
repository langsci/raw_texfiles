\subsection{Luang}\label{sec:Lua}
Luang (Austronesian, Maluku) has synchronic metathesis
which is analysed as being phonologically conditioned by \citet{tata15}.
Metathesis in Luang is one of several processes which occur to
join adjacent morphemes into a single rhythm unit;
that is, a phrase with only one stressed syllable.
A combination of a word and affix always join into a single rhythm unit,
while two conjoined words contrast with two words which form separate rhythm units:

\begin{quote}
However, there is contrast in Luang between separate
words being joined into one rhythm segment and being left apart.
Known information and mainline event information,
especially at peak points of the story,
are said so rapidly that many words join into one rhythm segment.
When information is new to the hearer or if it is brought into prominence the words are said more
slowly, and therefore do not join into one rhythm segment, but remain separate units. \hfill\citep[24]{tata15}
\end{quote}

While \cite{tata15} analyse Luang metathesis as being conditioned by speech speed and/or stress placement,
these phonological environments are discourse driven.
Metathesis in Luang is thus functionally comparable to
discourse-driven metathesis in Amarasi (Chapter \ref{ch:DisMet}),
though in Amarasi such metathesis is a direct marker of a discourse structure
rather than being conditioned by an intermediate phonological structure.

There is a complex set of phonological rules (one of which is metathesis)
which operate to join two morphemes together in Luang.
Which process operates depends on the phonological shape of the two morphemes,
as well as their respective word classes.
In the simplest case, the final vowel of the first word is deleted.
Such reduction is often followed by assimilation of certain consonants;
see \citet[25]{tata15} for details.
Examples are shown in \qf{ex:LuaVowDel} below.

\begin{exe}
	\ex{Luang vowel deletion\footnotemark \hfill\citep[25]{tata15}}\label{ex:LuaVowDel}
		\sn{\gw\begin{tabular}{rllllll}
		\it{ʔam\tbr{a}} 	&+& \it{-ni}&{\ra}& \it{ʔamni}	& [ˈʔamni]	& `his father' \\
		\it{naʔan\tbr{a}} &+& \it{=wa}&{\ra}& \it{naʔanwa}& [naˈʔanwə]& `s/he ate' \\
		\it{rwok\tbr{a}} 	&+& \it{pa}	&{\ra}& \it{rwokpa}	& [r̩ˈwokpə]	& `they meet to' \\
		\end{tabular}}
\end{exe}
\footnotetext{An alternate analysis of the data
in \qf{ex:LuaVowDel} would be to posit epenthesis
of /a/ after phrase-final consonants.
This is the analysis taken by \cite{st91} for similar data in Roma (\srf{sec:Rom})}

When the first word ends in a high vowel
and the second words begins with {\#}CV where the first vowel is not high,
the final high vowel of the first word spreads.
After spreading the final vowel of a VCV{\#} final word is deleted,
resulting in metathesis similar to the process in Selaru
described on \prf{ex:SelGC->CG} above.
When the high back vowel /u/ spreads over
a coronal consonant (except /r/) it
assimilates and becomes a palatal glide [j].
Examples of Luang high vowel spreading
are given in \qf{ex:LuaHigVowSpr} below.\footnote{
		I follow \citet{tata15} in representing glides
		which are a realisation of vowels after
		high vowel spreading as superscript in the phonetic transcription.}

\begin{exe}	%ʲʷ
	\ex{Luang high vowel spreading \hfill\citep[24]{tata15}}\label{ex:LuaHigVowSpr}
		\sn{\gw\stl{0.4em}
	\begin{tabular}{rllllll}
		\it{ʔamma\tbr{i}} &+& \it{la}&{\ra}& \it{ʔamma\tbr{i}l\tbr{j}a}	& [ʔamˈmailʲə]	& `we come to' \\
		\it{rma\tbr{i}} 	&+& \it{pa}&{\ra}& \it{rma\tbr{i}p\tbr{j}a}		& [r̩maipʲə]	& `they come for' \\
		\it{a\tbr{u}} 		&+& \it{maka}&{\ra}& \it{a\tbr{u}m\tbr{w}aka}	& [ˌauˈmʷakə]	& `wood that' \\
		\it{rken\tbr{i}} 	&+& \it{pa}&{\ra}& \it{rkenp\tbr{j}a}					& [r̩ˈkenpʲə]	& `they put it for' \\
		\it{rmat\tbr{i}} 	&+& \it{de}&{\ra}& \it{rmatd\tbr{j}e}					& [r̩ˈmatdʲə]	& `when they died' \\
		\it{nhor\tbr{u}} 	&+& \it{wa}&{\ra}& \it{nhorw\tbr{u}a}					& [ˈnhorʷuə]	& `already finished' \\
		\it{pwo\tbr{u}} 	&+& \it{de}&{\ra}& \it{pwo\tbr{u}d\tbr{j}e}		& [ˌpwouˈdʲe]	& `that sail boat' \\
		\it{wor\tbr{u}} 	&+& \it{la}&{\ra}& \it{worl\tbr{j}a}					& [ˈworlʲə]	& `two in' \\
%		\it{\tbr{}} 	&+& \it{}&{\ra}& \it{}	&{\ra}& []	& `' \\
%		\it{\tbr{}} 	&+& \it{}&{\ra}& \it{}	&{\ra}& []	& `' \\
	\end{tabular}}
\end{exe}

When a CCV{\#} final noun is joined into a single rhythm segment
with a morpheme which is consonant initial,
the final vowel of the noun is deleted followed by epenthesis
of the vowel /a/ to break up the newly created consonant cluster.
Examples are shown in \qf{ex:LuaVowDelEpe} below.

\begin{exe}
	\ex{Luang vowel deletion and epenthesis \hfill\citep[26]{tata15}}\label{ex:LuaVowDelEpe}
		\sn{\gw\stl{0.5em}
	\begin{tabular}{rlllllll}
		\it{likt\tbr{i}} 		&+& \it{-ni}	&{\ra}& \it{lik\tbr{a}tni}	& [ˈlikatni]& `his house' \\
		\it{ʔonn\tbr{i}} 		&+& \it{=wa}	&{\ra}& \it{ʔon\tbr{a}nwa}	& [ˈʔonanwa]& `the end' \\
		\it{nniaʔert\tbr{i}}&+& \it{-ni}	&{\ra}& \it{nniaʔer\tbr{a}tni}	& [nniaʔˈeratni]& `its meaning' \\
		\it{ʔult\tbr{i}} 		&+& \it{pa}		&{\ra}& \it{ʔul\tbr{a}tpa}	&& `skin for' \\
	\end{tabular}}
\end{exe}

However, when the first word ends in CCV{\#} and is a verb,
metathesis of the final CV sequence occurs.
\cite{tata15} state that it is unclear why verbs
have a different behaviour from nouns.
It is, however, regionally common for nouns and verbs to have different
behaviour regarding metathesis.
This is found in Mambae (\srf{sec:Mam}) as well as Amarasi.
Examples of Luang verbal metathesis are shown in \qf{ex:LuaMet} below.

\begin{exe}
	\ex{Luang metathesis (verbs only) \hfill\citep[26]{tata15}}\label{ex:LuaMet}
		\sn{\gw\stl{0.4em}
		\begin{tabular}{rllllll}
		\it{ʔer\tbr{nu}}	&+& \it{la}		&{\ra}& \it{ʔer\tbr{un}la}		& [ˈʔerunlə]& `go down to' \\
		\it{tow\tbr{ru}}	&+& \it{dojni}&{\ra}& \it{tow\tbr{ur}dojni}	& [towurˈdojni]& `spill completely' \\
		\it{hop\tbr{la}}	&+& \it{=wa}	&{\ra}& \it{hop\tbr{al}wa}		& [ˈhopalwə]& `sailed' \\
		\it{hop\tbr{na}}	&+& \it{pa}		&{\ra}& \it{hop\tbr{an}pa}		& [ˈhopanpə]& `order for' \\
		\it{kul\tbr{ti}}	&+& \it{pa}		&{\ra}&	\it{kul\tbr{it}pa}		&& `stick together for' \\
		\end{tabular}}
\end{exe}

To summarise: in Luang metathesis is one of several processes
which operate when two morphemes (including words) form a single phrase
for the purposes of stress assignment.
It is therefore possible to analyse metathesis as being conditioned by the placement of stress.\footnote{
		An alternate analysis would be to propose that words are joined
		into a single word/phrase by the various
		phonological processes (including metathesis),
		and then stress is assigned as appropriate.}

There is also no apparent phonological reason why metathesis affects verbs but not nouns in Luang.
While Luang metathesis is phonologically conditioned,
it is not clearly phonologically motivated.
Metathesis in Luang may be transitioning from phonologically conditioned metathesis
to morphemically conditioned or morphological metathesis.
Indeed, Leti which is culturally considered a Luangic dialect
has developed morphological metathesis (\srf{sec:Let}).\il{Luang|)}