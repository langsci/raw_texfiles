\documentclass[output=paper]{langsci/langscibook}
\ChapterDOI{10.5281/zenodo.1186597}
\author{Manfred Sailer\affiliation{%
%Institute for English and American Studies, 
Goethe University Frankfurt/Main}%
\lastand Stella Markantonatou\affiliation{Institute for Language and Speech Processing, Athena RIC, Greece}}
\title{Multiword Expressions: Insights from a multi-lingual perspective}
\rohead{Multiword Expressions: Insights from a multi-lingual perspective}
\abstract{In this introductory chapter, we present the basic concept of the volume at hand. The central aspects of the individually contributed chapters are sketched and some of the relations among the chapters are pointed out.}

\maketitle

\begin{document}

\section{Introduction}\label{Sec-Introduction} 
Multiword expressions (MWEs)  are not only a challenge for natural language applications, they also present a challenge to linguistic theory. This is so because, for the vast majority of them, their structure can be predicted by the grammar rules of the language to which they belong while the semantics of a substantial subset of MWEs is unpredictable or fixed. 
Therefore, MWEs often defy the application of the machinery developed for free combinations where the default is that the meaning of an utterance can be predicted from its structure.


%Multiword expressions (MWE) are not only a challenge for natural language applications but they also present a challenge to linguistic theory. They are a challenge because, for the vast majority of them, their structure can be predicted by the grammar rules of the language to which they belong but their semantics is unpredictable or fixed in general. Therefore, they cannot be treated with the machinery developed for `canonical' language where the default is that the meaning of an utterance can be predicted from its structure. 

There is a rich body of primarily descriptive work on MWEs for many European languages but there is little comparative work in this area extending on descriptive, theoretical, and computational issues. 
%
This volume brings together MWE experts with individual languages as their background to explore the benefits of a multi-lingual perspective on MWEs, as regards all the dimensions of linguistic research: descriptive coverage, theoretical scrutiny, and computational exploitation. 

We assume a broad concept of MWE in this volume, using MWE as the cover term for any kind of phraseological unit. \is{multiword expression} As such, it comprises idioms, collocations, complex names, phraseological patterns, etc. We chose the term MWE as the default in this volume, but use its competitors interchangeably with it where no confusion arises. Each contribution will specify explicitly within which empirical sub-domain of phraseology it is located.

We hope that this introductory chapter will help the book to gain easier access to a wider audience and will place it within the current state of research in phraseology and on multiword expressions. We thought that two general issues about this book should be addressed here: the variety of linguistic formalisms used and the general research issues discussed. 

The book contains contributions from various linguistic frameworks. Since the individual contributions are relatively short, we consider it useful to provide a brief overview over the frameworks. 

We will identify some general research questions that we see either prominently emerging in the field or as topics that should be addressed in the future and will show how the contributions in this volume address some of these issues. The multi-lingual perspective will serve as a guiding principle in the choice of topics. Of course, our perspective may well be biased due to personal preferences and limitations. 

Wherever it seems useful, we will point out links between the papers in this volume and show in which respect they point in the same direction or seem to reach mutually incompatible conclusions -- a strong proof of the lively ongoing discussions in the MWE field!

It is a privilege for us that this book appears as one of the first volumes in the new Language Science Press series \emph{Phraseology and Multiword Expressions}. We hope that it will pave the way for future books in this series that will take up some of the questions that are addressed here.


\section{Topics in multi-lingual MWE research}
\label{Sec-Topics}

In this section, we will briefly address three aspects that play an important role in the contributions to this volume: MWE classification, methods and issues in multi-lingual MWE research, and aspects of individual MWE types. In each of the following subsections, we will introduce the basic question and sketch how contributions in this volume address it.

\subsection{Classifications of MWEs}
\label{Sec-Classification}

The classification of MWEs is a challenge. Even more so, as there is no general consensus about what counts as an MWE. 
\citet{Burger:15} characterises phraseolo\-gical units by three properties: \isi{polylexicality}, fixedness, \is{multiword expression!fixedness} and \isi{idiomaticity}, where idiomaticity need not be present in all phraseological units. 
\citet{Fleischer:97} views phraseology as a fuzzy concept with polylexicality as the only obligatorily pres\-ent criterion. 
He assumes three further prototypical properties that define the fuzzy concept. 
As the term \emph{prototypical} suggests, these properties can be present or absent to various degrees. These properties are fixedness, idiomaticity, and  \is{lexicalization} lexicalisation. 
Idioms of the type \textit{kick the bucket} `die’ are the core cases of phrasemes, satisfying all three criteria. 
Collocations (\textit{open the door}) may lack idiomaticity, phraseological patterns (\textit{as goes X so goes Y}) may not be fully lexicalised. 

The concept that an expression can be a gradually more or less typical representative of an MWE has been generalised to the extreme in most versions of Construction Grammar \is{Construction Grammar}  \citep{FillmoreEtAl1988}. 
This framework abandons the split between Lexicon and Grammar and replaces them with a Constructicon that consists of more or less general and complex constructions. 
In this view, traditional lexical entries are specific but simple constructions, and classical rules of grammar are general but complex constructions. 
MWEs, such as idioms, are found in a middle position of this continuum, being rather specific and, at the same time, quite complex constructions. Consequently, it is impossible to define MWEs in this framework -- which has, of course, been a conscious design decision in Construction Grammar. \cite{Baldwin2010} come from a different angle. 
For them, MWE-hood is in the eye of the beholder: we need to define what we assume to be the “rule” (at any level of linguistic description or language use), and anything that deviates from the rule in one way or another will be classified as an MWE. 
In this view, the degree of irregularity or idiosyncrasy of an MWE can be observed, but it will be a yes/no split as to what counts as an MWE and what does not. 

So far, we have discussed three attempts to define the boundaries of the domain of MWE research. 
All of them have proven fruitful in research, and we do not see a point in choosing one over the other in abstracto. 
We can, however, understand the differences if we look at the underlying purpose of the definitions. \citet{Fleischer:97} is in the tradition of the Soviet phraseological research. 
There, phraseology is considered the third pillar of linguistics, complementing the Lexicon and Grammar by looking at objects that have both lexical and phrasal properties. 
\citet{FillmoreEtAl1988} developed their theory in opposition to the very abstract universalist ideas in the Chomskyan paradigm. 
Finally, \citet{Baldwin2010}  have concrete computational applications in the back of their minds such as the extraction of MWEs. 
If there were no difference between MWEs and free combinations, it would be impossible (or meaningless) to build a database of MWEs. 
The insight that emerges from these considerations is that we need to clarify in which context and for which purpose a characterisation and, as we will see in a second, a classification of MWEs has been proposed. Rather than adopting or rejecting a proposal in general, we should examine critically how far a proposal is suitable relative to our own current framework and research question.

To be on the most inclusive side, let us assume that the domain of MWE research consists of any expression that contains more than one basic lexical element and that is lexicalised, fixed, idiomatic, or irregular in one way or the other. This results in a highly heterogeneous set of expressions. Consequently, we need to structure this huge empirical domain by imposing a classification on it. Just as before, however, there is no hope of finding a single classification or taxonomy of MWEs that can be used for all purposes. Nonetheless, some proposed classifications are better than others. This evaluation will need to take into account the purpose of the classification. Parsing, MWE extraction, cognitive representation, second language learning, machine translation, and many other purposes can be thought of. In all these domains, MWEs pose highly intriguing challenges, but it is unlikely that the same classification will be useful for all of them.

For illustration, we can look at a number of classificatory criteria that have been proposed in the literature and show that they are essential for some, but, probably, relatively useless for other purposes. \cite{Makai:72} distinguishes between idioms of decoding and idioms of encoding. The first class of idioms contains expressions that can only be understood if they are known to the hearer. This is the case for expressions such as \textit{kick the bucket} ‘die’, but less so for expressions like \textit{answer the phone} or \textit{brush one’s teeth}. Idioms of encoding are expressions that need to be known in order to produce them. All three examples given would count as idioms of encoding, since it is an arbitrary convention that the idea of doing dental hygiene is expressed as brush one’s teeth in English rather than as clean one’s teeth. In German, it is the other way around, with \textit{Z\"ahne putzen} `teeth clean’ rather than \textit{Z\"ahne b\"ursten} `teeth brush’ being conventionalised, even though the instrument to brush your teeth with is called a \textit{Zahnb\"urste} `toothbrush' in German, just as it is in English.

The distinction between a decoding and an encoding perspective is clearly useful for parsing versus generation, but also for designing MWE collections for foreign language learners, who need both types of MWEs, in contrast to MWE collections for native speakers, which usually contain only idioms of decoding. For the purpose of a computational system for automatic MWE extraction, however, this distinction is completely immaterial; actually, it would be misguiding to evaluate an MWE extraction system with respect to its success in categorizing MWEs correctly as decoding or encoding MWEs.

\is{multiword expression!fixedness|(}
Syntactic flexibility  is a classificatory criterion that has been widely relied upon for retrieving, cataloguing, and parsing MWEs. Whether or not an MWE can appear in a number of different constructions, or, from a different point of view, can undergo some transformations, has been a central concern of treatments of idioms in Generative Grammar (see \citealt{Fraser1970}, for example). One of the most cited works in the computationally oriented MWE literature, namely ``Multiword Expressions: A pain in the neck for NLP" \citep{Sag:2002}, is about the classification of MWEs in terms of syntactic flexibility. This criterion also plays a central role in the contributions to this volume by \citeauthor{Kuiper2018tv}, \citeauthor{Laporte2018tv},
\citeauthor{ParraEtal2018tv}, \citeauthor{BargmannSailer2018tv}, and
\citeauthor{MarkantonatouSamaridi2018tv}
%Laporte, Parra Escart{í}n et al., Bargmann \& Sailer, and Markantonatou \& Samaridi 
-- although the last two contributions are rather interested in the ability of MWEs to appear in different constructions and its theoretical ramifications than in classification per se. Syntactic flexibility remains a concern in classifications that are computationally oriented and rely on more criteria, for instance, classifications that draw on the syntactic function of MWEs: 
\citetv{ParraEtal2018tv} classify MWEs in terms of both syntactic flexibility and syntactic function (namely, whether an MWE functions as  noun, verb or adjective/adverb). 
%\nocite{Kuiper2018tv,Laporte2018tv,ParraEtal2018tv,BargmannSailer2018tv,MarkantonatouSamaridi2018tv}

%In many approaches, however, we find a classification that is purely based on one syntactic criterion
Typically at least two degrees of flexibility  are distinguished, telling apart \textit{kick the bucket}-type expressions, which cannot undergo  \is{passive} passivisation, from \textit{spill the beans}-type MWEs, which can. This is a core distinction for formal theories of idioms such as the one in \isi{Generalized Phrase Structure Grammar}  \citep{GKPS} or in \citet{Nunberg1994}. After all, passivisation has the status of a major diagnostic in linguistic theory. For instance, back in the early 80's, newborn \isi{Lexical Functional Grammar}  (LFG) relied on passivisation in order to advocate lexicalism \is{lexicalism} and to define grammatical functions that are important axioms of the particular theory. Passivisation is discussed by several contributors to this volume, and opinions vary widely.  \citetv{MarkantonatouSamaridi2018tv}, who work within the LFG framework, draw on passivisation, as it seems to be able to split Greek MWE data nearly into two.  \citetv{BargmannSailer2018tv}, on the other hand,
%, who advance a universal semantic theory for MWEs, 
argue that, in the right context, most/all \ili{English} MWEs can passivise. Other languages, such as \ili{German}, impose even fewer or no restrictions on MWE passivisation. It is on these grounds that, according to them, passivisation is neutralised as a universal classificatory diagnostic for MWEs, but it may be valid in individual languages.

\newpage 
\citetv{Laporte2018tv} argues explicitly that the flexibility criterion of classification is highly problematic because it actually points to an ensemble of syntactic behaviours and, to this moment, there has been no reliable research on exactly how this collective behaviour of diagnostics defines  flexibility as a measurable property. It must be said, though, that Laporte does not so much claim that it is not possible to classify MWEs in terms of syntactic flexibility; rather, his argument is that, for the classification of MWEs in terms of a multi-dimensional feature such as syntactic flexibility, an important amount of data about different MWEs and the application of classification methods are required. These methods will apply over sets of features that receive binary values ($+$/$-$), that is, over categorical variables. 

The approaches to syntactic flexibility we have discussed so far are categorical in nature. They ask whether an MWE can participate in a phenomenon or not, but they are not interested in the actual usage of the phenomenon. Of course, syntactic flexibility can be seen from the point of usage: an MWE that is frequently used with a structural ``twist", even if it is the same ``twist" most of the time, is it a syntactically flexible one or not? \citetv{HanksEtal2018tv}
%Hanks, El Maarouf \& Oakes (this volume) 
argue for a quantitative definition of syntactic flexibility that takes into account the frequency of structural variations (``twists") of an MWE in a corpus and the reported first results suggest that there is little agreement between the ``theoretically" and the ``frequency" inspired notion of syntactic flexibility. 


\subsection{Multi-lingual studies of MWEs}
\label{Sec-Multiling}
\largerpage[-1]
Every multi-lingual or cross-lingual study of MWEs is confronted with a number of questions. First, in order to be able to compare a phenomenon across languages, some cross-linguistically, i.e. language-independent constant aspect has to be fixed. In this volume, this is achieved in different ways. In most papers, semantic aspects of the considered class of MWEs are kept constant in the comparison, usually together with some basic syntactic assumptions (such as looking at verbal MWEs).  

%Bargmann \& Sailer (this volume) 
\citetv{BargmannSailer2018tv}
concentrate on one particular type of MWEs, the so-called non-de\-com\-posable idioms. 
They identify this domain by semantic criteria that are independent of a particular language. 
Subsequently, they look at the way in which the languages they consider differ in the  syntactic flexibility of these MWEs.

\citetv{FotopoulouGiouli2018tv}
%Fotopoulou \& Giouli (this volume) 
define their domain of study by semantic and syntactic criteria. They look at verbal MWEs that express emotions. They use a semantic classification of emotion expressions with respect to the type of emotion and its intensity. On the formal side, they use a syntactic representation of MWEs that abstracts over some properties that are particular to individual languages. This allows them to identify comparable MWE classes in Modern \ili{Greek} and \ili{French}.  

\citetv{HanksEtal2018tv}
%Hanks et al.\@ (this volume) 
discuss a particular method to extract MWEs from a corpus and to classify them automatically according to their syntactic flexibility. They present a case study of the \ili{English} word \textit{bite} and its primary \ili{French} translation \textit{mordre} by looking at an identical number of hits from standard general corpora of the two languages. They apply a \isi{Corpus Pattern Analysis} (CPA)  on this data set to identify the usage patterns of these two verbs, which include a number of MWEs. Using statistical collocation measures on the extracted patterns, they manage to determine the syntactic flexibility of each of these patterns. They show that their method can be applied to different languages and demonstrate that the extracted patterns of English and French can be used to study the cross-language correspondences as regards the patterns' literal and idiomatic meanings. 

\is{multiword expression!fixedness|)}

\citetv{OsenovaSimov2018tv}
%Osenova \& Simov (this volume) 
study MWEs in parallel corpora of \ili{Bulgarian} and \ili{English}. They discover that MWE translational equivalents, at least for the particular language pair, tend to be either MWEs themselves or just single words; interestingly, translating an MWE with a compositional phrase is a rare phenomenon in their data. In order to encode these correspondences in a way that can be useful to parsing, they employ  caten\ae{} \citep{OGrady:98}, which are argued to offer adequate expressivity for representing the structural and the semantic properties of MWEs.

\citetv{KoevaEtal2018tv}
%Koeva et al.\@ (this volume) 
is the contribution that looks at the highest number of different languages. The authors compare named entities \is{name} in five different languages from four language groups. The category of named entities is defined semantically as the names given to persons, locations, or organisations. The authors find that depending on the kind of named entity, a number of different semantic aspects may be included within a larger name -- such as a title for a person’s name, for example. They use these semantic categories to define language-neutral, abstract patterns. In a second step, they map these to syntactic patterns for individual languages and identify similarities and differences within the sample of languages they consider. As in the case of \citetv{FotopoulouGiouli2018tv}, sticking to a clearly defined and relatively well-studied semantic domain can provide a very good basis for comparing the variation that is found in the morpho-syntax of the MWEs used in this domain.

\newpage 
%Mititelu \& Leseva (this volume) 
\citetv{MititeluLeseva2018tv}
consider a formal process, namely \is{morphology!derivation} derivation of MWE parts in \ili{Romanian} and \ili{Bulgarian}. They use the same method of data sampling for the two languages: they extract MWEs from general dictionaries of idioms and collocations. Subsequently, they extract occurrences of these MWEs in corpora and classify the types of derivational morphology found in their data. The paper establishes that the productivity of MWEs in derivation is a general phenomenon that should be considered more systematically than it usually is. The use of two languages serves primarily two purposes: first, the authors can make a more general point than they could have when looking at just one language; second, they illustrate the fruitful applicability of their method across languages.

The general, cross-lingual insights made by the contributions in this volume comprise at least the following: 
\begin{enumerate}
\item For well-defined and clearly understood semantic domains, it is possible to create a multi-lingual MWE sample. Once this semantically classified sample has been established, formal properties of the MWEs within the samples can be explored, including syntactic structure, flexibility, or morphological aspects. In a next step, we can seek for generalisations relating these language-specific%
\footnote{Throughout this chapter, we use \emph{language-specific} or \emph{language-independent} in the sense of ``specific to one language’' or ``independent of a particular language'’, rather than in the sense of ``specific/independent of language as such’'.} 
formal properties to the language-neutral semantic classification, both within and across the considered languages.
\item If there are comparable resources available (corpora, MWE collections, treebanks, or more advanced natural language processing tools), the methods of data sampling and data classification for MWEs can often be transferred from one language to another. This means that we will be able to use the same tools to study MWEs in one language and a parallel study of MWEs in another language. It does not mean, however, that we perform a comparison of MWEs in the two languages. \\
\end{enumerate}

\subsection{Special types of MWEs}
\label{Sec-SpecialType}

Given the heterogeneity of MWEs, it is necessary to focus on individual types of MWEs. Remember that we defined MWEs here as complex expressions that show some sort of idiosyncrasy. Consequently, MWEs differ in their basic linguistic properties, but also in the types of idiosyncrasy they display. We have already seen in §\ref{Sec-Multiling} that the limitation to a particular type of MWE is a necessary step for many cross-lingual considerations. In the present subsection, we will consider special types of MWEs, based on their morphological or syntactic structure or operations rather than on their semantics.

Focusing on special types of MWEs has been a useful method in any subdiscipline of linguistics. 
Here is a somewhat arbitrary collection of references to illustrate this point. 
To start with a negative example, the early Generative treatment of MWEs in \cite{Chomsky1957} does not distinguish between MWEs of different degree of \is{multiword expression!fixedness} syntactic flexibility. This is the main reason for the validity of the critique of this approach brought forward in \cite{Chafe:68}.

%the major points of criticism of the early Generative treatment of MWEs expressed in \cite{Chafe:68} are primarily valid, because the criticised approach, does not distinguish between MWEs of different degrees of syntactic flexibility. 

The importance of looking at different MWE types separately was illustrated, for example, in \citet{Krenn:99} and \citet{Gibbs1989a}. 
\citet{Krenn:99} shows that automatic MWE extraction from corpora may require different methods for different types of MWEs. 
\citet{Gibbs1989a} provide evidence that MWE types need to be carefully distinguished in psycholinguistic studies. Similarly, special MWE-types can be useful to address particular research questions: 
\citet{Hoeksema:10idiom}, for example, looks at MWEs containing embedded clauses such as (\ref{hoeksema-ex}), to investigate how big a lexicalised linguistic unit can possibly be. 
\cite{MuellerG:98} looks at binomials as in (\ref{muellerg-binom}) to show that general rules of coordination in German interact with idiosyncratic lexical fillings in these constructions -- in the present example, the law of growing members in co-ordination.

 \ea
\label{hoeksema-ex}
\settowidth \jamwidth{(German)} \il{German}
\textit{maken} [\textit{dat} X \textit{weg-kom-}] ‘leave as soon as possible’ \\ 
\gll We moeten maken dat we weg-komen!\\
we must make that we away-get\\ \jambox{(Dutch)} \il{Dutch}
\glt‘We need to leave!’
\z

\ea
\label{muellerg-binom}
\settowidth \jamwidth{(German)}
\gll fix und fertig / *fertig und fix\\ 
fast and ready {} ready and fast\\ \jambox{(German)}
\glt ‘exhausted’
\z

In this volume, three of such special types of MWEs have been addressed in some of the included chapters: 
MWEs and morphological \is{morphology!derivation} derivation, patterns of Named Entities, and Light Verb Constructions. 
We will briefly summarise these contributions.

%Mititelu \&  Leseva (this volume) 
\citetv{MititeluLeseva2018tv}
offer a rare contribution to the discussion about the derivation \is{morphology!derivation} of MWEs from MWEs.  
Of course, there is a lot of work on derivational morphology, but it does not pay extra attention to the productivity of idioms. 
Also, there is important work advocating that morphological derivation and MWEs should be represented with the same machinery, namely that of Constructions \citep{Riehemann:01}. 
However, derivation phenomena  have hardly been explored within the domain of MWEs, although they are wide-spread across languages. 
Below we use material from 
\citetv{MititeluLeseva2018tv}
%Mititelu \& Leseva (this volume) 
and add some Modern Greek and Serbian data to illustrate the variations of the phenomenon. 
In (\ref{ex:3:fashiondesign}), the pairs of noun MWEs in three languages, namely Bulgarian, English and Modern Greek, can be analysed as standing in a derivation relation. In (\ref{ex:4:promisemoon}) and (\ref{ex:5:pinaolikos}), an adjective MWE and a verb MWE can be analysed as standing in a derivation relation (Modern Greek participles function as adjectives). 
Lastly, in (\ref{ex:6:paparouna}) and (\ref{ex:7:bulka}), we have adjective MWEs of the simile type that are derivationally related with verb MWEs (headed by de-adjectival verbs), again of the \isi{simile} type in two languages, namely Modern Greek and Serbian.


\begin{exe}

\ex \label{ex:3:fashiondesign}
\settowidth \jamwidth{(German)}
\begin{xlist}
\ex \textit{moden dizayn} -- \textit{moden dizayner}  \jambox{(Bulgarian)} \il{Bulgarian}
\ex \textit{fashion design} -- \textit{fashion designer}  \jambox{(English)} \il{English}
\ex \textit{sχieδio moδas} -- \textit{sχieδiastis moδas} \jambox{(Modern Greek )} \il{Greek}
\end{xlist}
\end{exe}

\begin{exe}
\settowidth \jamwidth{(German)}
\ex \label{ex:4:promisemoon}
\begin{xlist}
\ex
\gll 
svalyam zvezdi\\
 take.down stars \\  \jambox{(Bulgarian)}
\glt`to promise the moon’
\ex
\textit{svalyach na zvezdi}\\ 
`one who promises the moon’
\end{xlist}
\end{exe}

\begin{exe}
\settowidth \jamwidth{(German)}
\ex \label{ex:5:pinaolikos}
\begin{xlist}
\ex
\gll
pinao sa likos\\ 
       I.am.hungry like wolf \\ \jambox{(Modern Greek)}
      \glt `being very hungry’
      \ex
\gll pinasmenos sa likos\\
hungry like wolf\\
\glt ‘very hungry’
\end{xlist}
\end{exe}

\begin{exe}
\settowidth \jamwidth{(German)}
\ex \label{ex:6:paparouna}
\begin{xlist}
\ex
\gll 
kokinos san paparuna\\
     red as  poppy \\ \jambox{(Modern Greek)}
\glt `red (because of blushing)’
\ex
\gll kokinizo san paparuna\\
      I.become.red as  poppy\\
\glt    `blushing a lot’
\end{xlist}
\end{exe}

\begin{exe}
\settowidth \jamwidth{(German)}
\ex \label{ex:7:bulka}
\begin{xlist}
\ex
\gll 
crven kao bulka\\ 
       red as  poppy \\ \jambox{(Serbian)}  \il{Serbian}
     \glt `red (because of blushing)’
\ex
\gll       pocrveneo kao bulka \\
 I.become.red as poppy\\
 \glt     `blushing a lot’
 \end{xlist}
 \end{exe}
 
 \citetv{MititeluLeseva2018tv}
 %Mititelu \& Leseva (this volume) 
 map and contrast a wide range of derivation types in Romanian and Bulgarian and, eventually, they reveal a rather complicated and promising field of study. 


As already discussed in §\ref{Sec-Multiling},  
\citetv{KoevaEtal2018tv}
%Koeva, Krstev,  Vitas,  Kyriacopoulou,  Martineau \& Dimitrova (this volume) 
offer a strongly cross-linguistic account of the semantic and syntactic contexts where named entities occur. 
Named entities have often been treated as MWEs -- naturally, only the named entities that are formed of more than one word are MWEs (indicatively \citealt{Downey:al:07}; \citealt{vincze2011}). 

Named Entity Recognition \is{name}  is a widely discussed research topic in computational linguistics. 
In this general context, \citeauthor{KoevaEtal2018tv}, building on the fact that named entities come in patterns in all languages, have set the ambitious goal to enumerate the semantic and syntactic contexts in which named entities occur in a set of languages, namely Bulgarian, English, French, Modern Greek, and Serbian. 
The authors study named entities denoting persons, locations and organisations and show that the semantic patterns could be language independent, while the syntactic patterns vary to some degree according to language specificities such as the existence of articles and cases along with word order preferences.


An impressive amount of literature has been dedicated to Light Verb Constructions \is{light verb construction} (LVCs). Some relatively early approaches include \citet{Jespersen:65},  \citet{Gross1988a}, \citet{Butt:95}, \citet{Melchuk:98}. LVCs are structures that contain a verb that combines with another verb or a predicative noun to yield a monoclausal structure in which the event described is not specified by the (first) verb but by the other predicates. In a sense, the (first) verb is considered to have lost some of its semantic weight and to have turned into a ``light'' verb. In the example below, which has been taken from %Laporte (this volume)
\citetv{Laporte2018tv}, two translation equivalent expressions are given in French and English. In these examples, the main verb \textit{avoir}/\textit{have} is not used with its proper (possessive) semantics while the described event is specified by the noun \textit{conflit}/\textit{conflict}. Consequently, the verb \textit{avoir}/\textit{have} is used as a light verb in (\ref{conflit}). 

\begin{exe}
\ex \label{conflit} 
\settowidth \jamwidth{(German)}
\begin{xlist}
\ex{
\gll Il a eu un conflit avec sa famille.\\
he has had a conflict with his family\\}\jambox{(French)}
\ex  \textit{He had a conflict with his family.}\jambox{(English)}
\end{xlist}
\end{exe}

LVCs occur in many languages and pose interesting questions about the theory of syntax and semantics.  Not surprisingly, one question is how LVCs can be delineated from other types of verb MWEs and from compositional structures.  %Laporte (this volume) 
\citetv{Laporte2018tv} offers a thorough discussion of the criteria used to set apart LVCs from other MWEs and from compositional structures.  More on the descriptive side,  
%Fotopoulou \&  Giouli (this volume) 
\citetv{FotopoulouGiouli2018tv}
include LVCs in their contrastive study of emotive MWEs in Modern Greek and French.


The individual types of MWEs considered in this volume constitute a representative subset of options. First, the studies include some frequently  discussed structures, such as LVCs, but also structures that often remain unnoticed, such as derivation. Second, they include the question of what the internal structure of an MWE is (in its unmodified form), but also which types of operations (parts of) it can undergo. The paper on Named Entities by 
%Koeva et al.\@ (this volume) 
\citetv{KoevaEtal2018tv}
clearly addresses the first type of question, whereas the discussion of derivation by 
\citetv{MititeluLeseva2018tv}
%Mititelu \& Leseva (this volume) 
is concerned with the second type of question. Related to these points is the question of whether an MWE instantiates a general pattern of the language, such as an “ordinary” verb-complement relation, or whether we are dealing with a particular pattern that is productively, though exclusively, realised by MWEs, such as, maybe, some of the Named Entity patterns or the LVCs addressed in some of the papers.

We are positive that the inclusion of MWEs in the linguistic discussion of particular structures or phenomena can lead to important insights both in our understanding of these phenomena and our understanding of MWEs. On the other hand, we consider it important to take a closer look at MWE-specific patterns and to identify in which way their properties relate to the more general phenomena of a language. 




\section{MWEs and linguistic theory}
\label{Sec-LinguisticTheory} 

MWEs are situated at the overlap of the lexicon and grammar. This places them both at the centre and at the margins of linguistic theorizing. Theoretical discussions of MWEs typically take one of the following two questions as their starting point: Can the established tools of the lexicon or grammar be used to model MWEs? What insights can we get on the properties of words or grammatical processes from looking at MWEs? The first question starts from a given theory and applies it to MWEs, the second starts from observations on MWEs and uses them to modify the theory.
Some of the papers in this volume are written from a particular theoretical perspective, including Generative Grammar (\citeauthor{Kuiper2018tv}’s contribution), Lexicon-Grammar (\citeauthor{Laporte2018tv}
and \citeauthor{FotopoulouGiouli2018tv}),
%Laporte and Fotopoulou \& Giouli), %Dependency Grammar (Osenova \& Simov), 
Lexical Functional Grammar (\citeauthor{MarkantonatouSamaridi2018tv}),
%Markantonatou \& Samaridi), 
and Head-driven Phrase Structure Grammar (\citeauthor{BargmannSailer2018tv}).
%(Bargmann \& Sailer). 
In the present section, we will give a brief summary of the role MWEs have played in these theories and how the papers in this volume relate to this. There are, of course, important discussions on MWEs in many other frameworks, which we will have to leave aside here.%
\footnote{See the relevant overview chapters in  \cite{Burger:al:07.2} for some more frameworks.}
%\nocite{Kuiper2018tv,Laporte2018tv,FotopoulouGiouli2018tv,MarkantonatouSamaridi2018tv,BargmannSailer2018tv}

\subsection{Generative Grammar} \label{Sec-GG}
\largerpage
\is{Generative Grammar|(}
Generative Grammar is a cover term for a diverse family of theories going back to \cite{Chomsky1957}. Since we will look separately at two ``spin-off" theories, Lexical Functional Grammar and Head-driven Phrase Structure Grammar, we will limit ourselves here to the theoretical strand that could be called Chomskyan Generative Grammar whose current version is referred to as \isi{Minimalism} \citep{Chomsky:95}. In this tradition, the discussion of MWEs is very much focused on idiomatic, verbal MWEs.  \cite{Kuiper:04} provides an overview over the main developments in Generative Grammar and the role MWEs have played therein. \cite{Nunberg1994} give a detailed and critical evaluation of the use of MWEs in Generative syntactic argumentation. 

From the first mentioning of MWEs in  \citet{Chomsky:65} on, the general analytic conception of MWEs has been that an MWE is inserted into the syntactic derivation as a single unit, though a unit with internal structure. An analytical challenge arises once this assumption is combined with the idea that non-canonical syntactic structures are derived from an underlying basic structure that is determined by argument selection, such as \isi{Deep Structure} or the result of \isi{Merge}. \citet{McCawley:81} shows that these assumptions are incompatible with the data in (\ref{ex-stringsMcC}): if the MWE \textit{pull strings} is inserted as a unit, its parts cannot be spread over a relative clause and the noun it attaches to, as in (\ref{ex-stringsMcC1}). If the head of the relative clause is generated inside the relative clause, (\ref{ex-stringsMcC1}) would no longer be a problem, but, then, (\ref{ex-stringsMcC2}) would be problematic, where the idiomatic noun \textit{strings} is the head of a relative clause that does not contain the rest of the idiom.

\begin{exe}
\ex \label{ex-stringsMcC} 
\begin{xlist}
\ex \label{ex-stringsMcC1}
\textit{The strings that Parky pulled to get me the job.} \citep[135]{McCawley:81}
\ex \label{ex-stringsMcC2}
\textit{Parky pulled the strings that got me the job.}  \citep[137]{McCawley:81}
\end{xlist}
\end{exe}

Only recently, the en bloc insertion approach to MWEs has been relaxed in some publications, such as  \citet{Harley:Stone:13} and  \citet{vCraenenbroeck:al:16draft}. Corver et al.   integrate the distinction between decomposable and non-decomposable MWEs from \citet{Nunberg1994}  into a Minimalist approach and assume distinct structural constraints for the two types of MWEs.

In Generative Grammar, MWEs have typically been used to test structural hypotheses, where two aspects of MWEs have received primary attention: first, their restricted yet not fully blocked syntactic flexibility, and second, their internal structure. For example, idioms provided a major piece of empirical evidence for the raising analysis in Government and Binding Theory \citep{Chomsky:86a}. As for the second point, over the years, the size of MWEs has often been taken as support for various syntactic notions: the perceived inexistence of MWEs including subjects was used as support for the existence of a VP in syntax. More recently, the size of MWEs has been claimed to correlate with phrases, i.e. structural domains that are assumed to be closed for a number of syntactic processes \citep{Svenonius:05}.

In the present volume, \citeauthor{Kuiper2018tv} proposes an interesting new way of constructing syntactic arguments based on MWEs. Starting from the assumption that MWEs typically show some kind of irregularity, he formulates the following Law of Exception:

%%\ea \label{LawOfException} 
\begin{quote}
Law of Exception: All formal properties of the grammar of a language are subject to exceptions manifested in idiosyncrasies in the lexical items of that language.
\end{quote}
%%\z

This approach allows him to derive support for a principle of grammar by showing that there are lexical items violating it. 

\is{Generative Grammar|)}

\subsection{Lexical Functional Grammar (LFG)}
\label{Sec-LFG}

\is{Lexical Functional Grammar|(}

The generative, transformation free, phrase structure grammatical formalism of Lexical Functional Grammar (LFG) is:
\begin{itemize}
\item Unification based: information from the different components of an utterance is unified to form the overall linguistic information content; the linear order of the utterance components is not important.
\item Lexicalistic: linguistic operations are divided into lexical and syntactic operations. For instance, valency changing operations are understood as lexical properties, while co-ordination is analysed as a syntactic phenomenon. The syntactic component of the grammar cannot affect the lexical one.
\end{itemize}

LFG develops different levels of analysis that stand in a mutually constraining relation to each other via well-defined mappings. Different formal means may be employed for the representations at the various levels of analysis such as the m-structure, where morphological information is represented, the \isi{c-structure}, where phrasal structure information is represented, with a tree formalism, the \isi{f-structure}, where  functional relation information such as agreement, binding, and control are represented using attribute-value matrices (AVMs),  and the s-structure, which is dedicated to semantic information. In particular, crucial features of the f-structure are the so-called Grammatical Functions (GFs) that stand for things like subject and object. LFG considers them as primitive notions and uses them to represent relations among the phrasal constituents. 
MWEs were first mentioned in the LFG literature when \citet{Bresnan:82}   used {\em to keep tabs on somebody} in order to construct an argument in favour of \is{lexicalism} lexicalism.  In this discussion, \textit{tabs} heads a meaningless NP that instantiates the object of the structure or the OBJ(ect) GF in LFG parlance. When the passivisation  \is{passive}  lexical rule applies, \textit{tabs} becomes the SUBJ(ect) of the passivised form \textit{tabs were kept on somebody}. An idiomatic verb predicate \textit{keep} is defined in the lexicon (\ref{lfg}) that requires a subject, an object and an indirect argument dubbed ON OBJ.  Also an idiomatic noun \textit{tabs} is defined to be ``semantically empty'’, which, in the LFG conception of grammar, entails that the noun does not introduce a predicate in the representation. Therefore, it does not have a PRED(icate) value and it only has a FORM value:

\ea \label{lfg}
keep  V ($\uparrow$ TENSE) = PRESS\\
          ($\uparrow$ PRED) = `observe<SUBJ,ON OBJ>’\\
          ($\uparrow$ ON OBJ FORM ) $=_c$ TABS\\
\z

However, the semantically empty NP \textit{tabs} should be prohibited from turning up as the object of other predicates. \citet{Andrews:82} proposes a syntactic solution that draws on a reformulation of the Coherence Principle of LFG. \citet[67]{kaplan1995} note that in order to face the problem posed by semantically empty NPs ``a separate condition of semantic completeness could easily be added to our grammaticality requirements, but such a restriction would be imposed independently by a semantic translation procedure. A separate syntactic stipulation is therefore unnecessary.’' Furthermore, \citet[158]{Partee:04} points out that semantically empty NPs would be a problem for a \ia{Montague, Richard} Montague-like compositional approach to Semantics where NPs are assumed to contribute a predicate but that a possible solution would require a non-dispensable semantic translation level.
This discussion highlights important aspects of the LFG approach to MWEs. First, according to \citet{Bresnan:82}, MWEs  that contain an idiomatic V NP component and passivise, can project an f-structure that contains an OBJ(ect) GF.  Therefore, it is assumed that the syntax of MWEs is exactly like the syntax of compositional language in this respect, even when the fixed parts of an MWE are considered. The semantic component of the theory is expected to play an important role. Actually, state-of-the-art LFG has put emphasis on semantics and has offered interesting analyses of ``idiomatic'’ constructions, such as the \textit{way}-construction (\ref{ashudeh}) \citep[30]{Asudeh:al:08}.  

\ea \label{ashudeh}
\textit{Sarah elbowed her way quickly through the crowd.}
\z

It seems that the development of a semantic component questions the traditional conception of the syntactic component of the theory, for instance the so-called semantic forms  \citep{Lowe:2015}. Semantic forms have been crucial for defining the coherence and completeness axioms of LFG that form a major part of the mechanism by which the grammar checks the grammaticality of strings.  In a similar vein, several of the defining properties of Grammatical Functions  reflect a set of behaviours that were considered syntactic, but now a lot of this burden may move to semantics, for instance,  the conditions on the replacement of an NP by a clitic may be semantic in nature  to some considerable degree \citep{Arnold:15}.  So far, passivisation has not received a semantic analysis in LFG and remains, so to speak, the identifier of ``syntactic OBJecthoodness'’, therefore a discussion about the ability of MWEs to passivise could still be argued to be a syntactic discussion.

An applied approach to MWEs was offered by \cite{attia2006}   embedded in an implemented LFG grammar of \ili{Arabic} with a wide coverage. Attia has argued that MWEs should be disambiguated in a preprocessing step, i.e. before parsing. In his system, fixed and semi-fixed MWEs are processed by the morphological component that uses regular grammars (as opposed to the syntactic component that uses context-free grammars). 

Such approaches open up theoretical issues, such as which is the part of speech that should be assigned to the fixed parts that are treated as words and, given the problems stemming from passivisation,  how the LFG syntactic theory is affected by these novel words and their syntactic reflexes.  These issues are a potential challenge for the generally accepted view that MWEs and compositional structures use exactly the same syntax.  
\citetv{MarkantonatouSamaridi2018tv}
%Markantonatou \& Samaridi (this volume) 
discuss exactly this question in the framework of LFG drawing on Modern Greek verb MWEs.  

\is{Lexical Functional Grammar|)}

\subsection{Head-driven Phrase Structure Grammar (HPSG)}
\label{Sec-HPSG}

\is{Head-driven Phrase Structure Grammar|(}

Head-driven Phrase Structure Grammar (HPSG) has its origin in phrase structure grammar frameworks such as \isi{Generalized Phrase Structure Grammar} \citep{GKPS}, but has received a fundamentally different formal basis as a constraint-based feature structure grammar \citep{pollard:98,Richter:04}. 
HPSG encodes all levels of linguistic analysis within one representation, a well-articula\-ted notion of a \ia{de Saussure, Ferdinand} Saussurian \isi{linguistic sign}. 
The lexicon is of central importance for the theory, as all idiosyncratic information projects from the lexicon, and valence-alternation processes are expressed as lexical rules. 
The role of syntax is largely restricted to allowing lexical elements to combine in order to satisfy their valence requirements and, at the same time, to build up the phonological and the semantic representations of a sentence. 
All grammar rules are strictly local, i.e., referring only to a mother node and its immediate daughters. Since \citet{Sag:97}, a proliferation of grammar rules can be observed, which has been an attempt to connect HPSG and Construction Grammar\is{Construction Grammar} more closely and, ultimately, led to the development of Sign-Based Construction Grammar\is{Sign-Based Construction Grammar} (SBCG, \citealt{Sag:12}).
 
There is no treatment of idioms or MWEs in \citet{Pollard:Sag:87} or \citet{Pollard:Sag:94}, but at least since \citet{Krenn:Erbach:94}, there have been approaches to encode MWEs in HPSG. A basic obstacle to this task comes with the formalisation of HPSG: every node in a syntactic tree must be licensed by the grammar. 
This blocks every attempt to integrate idiosyncratic phrasal expressions as units. For this reason, HPSG researchers tend to promote lexical analyses of MWEs. 
This has been done in \citet{Krenn:Erbach:94}, who use the highly expressive selection mechanism of HPSG to account for the co-occurrence of idiom parts. 
The sign-based character of HPSG allows selection not only for syntactic category and semantic type, but for fine-grained syntactic and semantic properties as well, including the selection of a single lexeme. 
The discussion on MWEs has motivated a number of innovations in the theory, such as the use of underspecification in semantics, the introduction of lexeme identifiers in syntax, and the accessibility of specifiers of phrases from a higher node.

Many HPSG publications on MWEs have been written in the context of machine translation projects, which includes \citet{Krenn:Erbach:94}, \citet{Copestake:al:95} and \citet{Sag:2002}.  
An important drawback of the HPSG research on idioms is that it is almost exclusively restricted to the discussion of \ili{English} and \ili{German} examples, though there are recent approaches to \ili{Hebrew} \citep{Sheinfux:al:15}  and \ili{Japanese}  \citep{Haugereid:Bond:11}.

\is{multiword expression!flexibility|(}
%\marginpar{Provide a year in the bibtex!}
Recent approaches, such as \citet{kaysagidioms} and 
%Bargmann \& Sailer (this volume), 
\citetv{BargmannSailer2018tv},
propose such a lexical analysis for all idioms that have a regular syntactic shape. This generates a number of research questions: (i) Can the idiomatic reading be derived using the regular mechanism of semantic combinatorics? (ii) Can the attested differences in syntactic flexibility between idioms be captured? (iii) Can the co-occurrence of idiom parts be guaranteed and an idiom-external use of idiom components be blocked? In addition, syntactically irregular expressions still need to be captured by idiosyncratic grammar rules, and it is far from sure that the required rules satisfy HPSG’s locality restriction that idiosyncrasy can only occur in local mother-daughter relations (see \citealt{Sailer:12}).
Given the constraint-based local nature of HPSG, the answer to (i) can only be positive However, mechanisms of semantic combinatorics have been proposed that are not compatible with standard, \ia{Montague, Richard}  Montagovian, assumptions of \is{compositionality} compositionality, including underspecification and redundant semantic marking. 
Answers to question (ii) typically attempt to show that what seem to be idiosyncratic restrictions on the syntactic flexibility of MWEs follow from the general properties of the considered syntactic processes and the lexical properties of the stipulated idiomatic words. \citet{kaysagidioms} and 
%Bargmann \& Sailer (this volume) 
\citetv{BargmannSailer2018tv}
illustrate this strategy, which can be seen as a variant of the above-mentioned hypothesis in \citet{Nunberg1994}  that the decomposability \is{multiword expression!decomposable} of an MWE is directly connected to its  syntactic flexibility. 
As for question (iii), the selection mechanism is still the most popular means of ensuring the co-occurrence of idiom parts, while more flexible collocation mechanisms have been proposed as well (\citealt{Sailer:03, Soehn:09}). The final question of the analysis of syntactically irregular expressions and, in particular, the possible depth of syntactic idiosyncrasy has not been addressed systematically. \citet{Richter:Sailer:14} and \citet{kaysagidioms} look at MWEs with embedded clauses (such as \textit{know on which side one’s bread is buttered}), but all expressions they consider are syntactically regular. 

\is{multiword expression!flexibility|)}
\is{Head-driven Phrase Structure Grammar|)}

\subsection{Lexicon Grammar (LG)}
\label{Sec-LG} 

\is{Lexicon-Grammar|(}
\is{taxonomy|(}

A lot of pioneering and on-going work on MWEs has the Lexicon-Grammar (LG) framework as a reference point. LG is not a generative grammatical framework; rather, it strongly advocates a classification-based approach. LG relies on the classification of a large number of linguistic structures using as a linguistic unit not the word but the simple sentence that consists of a verb, its subject and two objects at maximum \citep{Gross:82}.  Various structures are identified and used as classificatory properties for verbs, for instance the simple transitive active voice phrase (NP V NP for English) and the simple passive phrase (NP \textit{be} V\textit{ed} \textit{by} NP for English) are listed as independent properties (and not as an ordered pair defining a transformation) that verb predicates such as \textit{write} and \textit{die} may or may not have, depending on whether the corresponding structures are attested.  Matrices are developed for each class of the verbs that demonstrate similar behaviour with respect to these properties. The columns in the matrices are named for the properties and the rows for the various verb predicates; the symbols `$+$'/`$-$' are assigned to the cells depending on whether the predicate is found in the respective structure or not. 

Classifications rely on empirically attested phenomena on the morphological and the syntactic level. Still, meaning seems to retain an important role in the definition of verb classes. For instance   \citet[401--402]{Gross:75-Relation} explains that the verb {\em dire} appears in structures that are not available to other verbs of ``saying'' and then notes: 

\begin{quote}
One might describe these restrictions by means of a standard transformational solution: the syntactic properties that have been observed for the verbs of /saying/ would only be attributed to the verb {\em dire}. All other verbs of /saying/, namely all verbs that indicate an emission of sound or of light, would be considered as intransitive verbs. \citep[402]{Gross:75-Relation}
\end{quote}

In other words, because certain syntactic properties are observed with only one verb, namely with the representative verb \textit{dire}, but not with the other verbs of the same class, a transformation is assumed that relates the syntactic properties of the representative verb \textit{dire} with the syntactic properties of other members of the class; therefore, the verbs of ``saying' do not share exactly the same syntactic properties, and they belong to the same class because they share the ``emission'' semantics and some syntactic properties. Things being so, the  class is defined not by the morphosyntactic properties of its members but by their meaning.


\cite{Gross:77} explains that LG assigns extreme importance to taxonomies
because they pertain to the scientific nature of the linguistic quest. Taxonomies
are a standard practice in biology whereby the use of the best representative
of a species in experiments guarantees reproducibility of results. In the case
of linguistics, the acceptability tests of linguistic structures by native speakers
are considered experiments. \cite{Gross:78} discusses the drawbacks of classification
practices, namely that they result in disjoint classes of classified objects while
in linguistic reality few clear-cut separating lines are observed. Still, he argues
that it is worth paying the value of (probably vast) fragmentation into (not necessarily
homogeneous) classes because this is the only known way of obtaining
an organisation of linguistic data that guarantees reproducibility of linguistic
experiments.


Early on, LG applied taxonomies  on MWEs; LG prefers the term {\em fixed expressions} \is{multiword expression!fixed expression}  for MWEs (\citealt{gross1982,Gross1988a,Gross1988b}). The continuum from fully compositional structures to fully fixed expressions is recognised.  The criteria developed set fixed expressions apart from terminology and professional or other sublanguages, from frequently used compositional structures and from ``support constructions'' (in   §\ref{Sec-SpecialType}  we  encountered these constructions under the name  \is{light verb construction} ``LVCs''). In this volume, the paper on Modern \ili{Greek} and \ili{French} emotive MWEs by 
\citeauthor{FotopoulouGiouli2018tv}
%Fotopoulou \&  Giouli 
studies a set of structures, including MWEs and support verb constructions, that denote emotions; these structures illustrate the aforementioned continuum between compositional and fixed language and an interesting cross-lingual result is obtained, namely that the degree of fixedness is related to the intensity of the emotion denoted.

MWE studies owe a lot to work conducted within LG. 
%Laporte (in this volume) 
\citetv{Laporte2018tv} summarizes some of the work done on MWEs within LG, elaborates on its merits and compares the strongly data-based method of LG with the more hypothesis-driven approach of Generative Grammar. 

\is{taxonomy|)}
\is{Lexicon-Grammar|)}

\largerpage
\section{What do we find important from here on?}
\label{Sec-Future}

All frameworks that are represented in this volume take a competence-oriented approach to MWEs, i.e., they attempt to model the possibilities rather than the usage of MWEs. 
However, with MWEs in particular, it is rather difficult to draw the line between what is a grammatically acceptable variation of an MWE and what is a variation that is licensed by some special rhetoric effect such as word play or what \citet{Egan:08} calls “extended” uses of MWEs. 
Related to this, the rich literature on the discourse-constitutive effect of MWEs remains largely unexplored in its insights for the formal study of MWEs. 
This has direct repercussions on the formal modelling. First, most competence-oriented researchers agree that playful use of MWEs falls outside their empirical domain. 
They do not, however, necessarily agree on whether a particular lexical or structural variation of an MWE is an extended use or not. This has an influence on what set of data they aim to explain. 
While this is a general problem of competence-based approaches, it is particularly prominent in the study of MWEs. 
In the present volume, this contrast can be seen most clearly in the differences between the contributions by \citeauthor{Laporte2018tv} and 
\citeauthor{BargmannSailer2018tv}: Laporte’s data are based on simple sentence frames without context, whereas Bargmann \& Sailer consider all MWE variations that they find in attested examples, taking into account their linguistic context though not the question of whether such examples would be considered rather unnatural by native speakers, even in the given context.

A second aspect that is usually left aside in the included frameworks is the inherent ambiguity of many MWEs. Given that  idiomaticity, i.e., the presence of ``compositional” or ``literal” meaning next to an idiomatic meaning, is one of the three defining prototypical properties of MWEs, this is clearly a question that would deserve attention. 
\is{semantics|literal meaning} 
Particularly intriguing  are cases in which the idiomatic and the literal reading seem to be simultaneously present, as in (\ref{bite-tongue-Ernst}), taken from \citet{Ernst:81}. 

\ea \label{bite-tongue-Ernst} 
\textit{He bit his thirst-swollen tongue.}\\
Reading: `He bit his tongue \& his tongue was thirst-swollen.'
\z

Recent attempts to combine compositional and distributional semantics such as  \citet{Gehrke:McNally:16} can be considered a step towards a modelling of such co-existences of a literal and an idiomatic meaning.

While there are the above-mentioned similarities between the frameworks represented in this volume, there are also considerable differences. One major difference is the relative importance that they attribute to theoretical concepts and to data. 
Research in Generative Grammar is typically hypothesis driven. This has led to many hypotheses about MWEs. While most of them have been proven wrong by now, they were still useful in putting the focus of phraseological research on a particular aspect and led to an increase of knowledge in this domain.  
Lexicon Grammar, on the other hand, is rather data driven and has a relatively long tradition of systematic compilation and classification of data. While this led to the creation of rich resources on MWEs, it is less obvious which implications generalisations over the collected data should have for the theory. 
We have also seen how lexicalist theories such as LFG and HPSG attempt to develop tools to account for the more phrasal phenomena that we find in MWEs. 
The research questions in such frameworks are typically very specific and partly data-driven, partly hypothesis-driven.

The variety of analytic and methodological alternatives used in the theoretical descriptions of MWEs over the years is impressive and shows that this empirical domain has a lot to offer for theoretical linguistic research. 
We would be excited if the present volume stimulated more interaction and mutual reception across framework boundaries. 
There are still many types of MWEs that have not been described formally or for which no data have yet been collected systematically. 
Such studies can potentially corroborate or refute essential properties of a framework or at least motivate a small change in perspective.



\section*{Acknowledgements} 
This book reflects a subset of the activities carried out in the ICT COST Action IC1207 \textit{Parsing and multi-word expressions. Towards linguistic precision and computational efficiency in natural language processing (PARSEME)}, 2013--2017, in particular in the working group \emph{Lexicon-Grammar Interface}. We are grateful to COST and to the co-ordinators of the action, in particular Agata Savary, for this opportunity of fruitful scientific exchange. 

We would like to thank the reviewers of this volume:
 
\begin{itemize}
\item Doug Arnold
\item Anastasia Christofidou
\item Voula Gotsoulia
\item Jack Hoeksema
\item Gianina Iord\v{a}chioaia
\item Koenraad Kuiper
\item Cvetana Krstev
\item Timm Lichte
\item Johanna Monti
\item Stefan Müller
\item Petya Osenova
\item Carlos Ramisch
\item Agata Savary
\item Alexandros Tantos
\item Veronika Vincze
\item Shuly Wintner
\end{itemize} 
Thanks for providing detailed comments on the first versions of the chapters and for helping us, the editors, to decide which papers to include in this volume and for giving useful and informed feedback to our contributors.

We are also grateful to \ia{Savary, Agata} Agata Savary for being the member of the editorial board of \emph{Phraseology and Multiword Expressions} who accompanied us from the (almost) final version of the manuscript to the real publication.

In his function as one of the main editors of Language Science Press, \ia {Müller, Stefan} Stefan M\"uller has helped us getting started with this book project. We would like to thank him and Sebastian Nordhoff for their support concerning all practical and technical questions about the publication. We are grateful to the LangSci proofreaders.
\ia{Parmentier, Yannick} Yannick Parmentier has provided us with a version of the LangSci LaTex-style that was adjusted to this series, for which we are very grateful! Finally, we would like to thank \ia{Minos, Panagiotis} Panagiotis Minos, who invested numberless hours in converting the individual chapters into a consistent format. 

\section*{Abbreviations} 


%\begin{table}
%\begin{tabular}{ll}
%\lsptoprule
\noindent\begin{table}[H]
\begin{tabularx}{.44\textwidth}{lQ}
%Full form  & Abbreviation \\
%\midrule
GF & Grammatical Function \\
HPSG & Head-Driven Phrase Structure Grammar \\
LFG & Lexical Functional Grammar 
\end{tabularx}
%no space or empty line here!
\begin{tabularx}{.55\textwidth}{lQ}
LG & Lexicon Grammar \\
LVC & Light Verb Construction \\
MWE & Multiword Expressions \\
SBCG & Sing-based Construction Grammar
\end{tabularx}
\end{table}
%\lspbottomrule
%\end{tabular}
%\caption{Abbreviations}
%\end{table}


%%\section*{References}

\sloppy
\printbibliography[heading=subbibliography,notkeyword=this]











\end{document}
