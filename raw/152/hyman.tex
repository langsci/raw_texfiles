\documentclass[output=paper]{langsci/langscibook} 
\title{Multiple argument marking in Bantoid: From syntheticity to analyticity} 
\author{%
 Larry M. Hyman\affiliation{University of California, Berkeley}  
}
\ChapterDOI{10.5281/zenodo.823230} %will be filled in at production

\abstract{This paper addresses the mechanisms of change that lead from syntheticity to analyticity in the Bantoid languages of the Nigeria-Cameroon borderland area. I address the different strategies that are adopted as these languages lose applicative “verb extensions” found elsewhere in Bantu and Niger-Congo. I show that although historical recipient, benefactive, and instrumental applicative marking on verbs allowed multiple object noun phrases (send-\textsc{appl} chief letter, cook-\textsc{appl} child rice, cut-\textsc{appl} knife meat), they have been replaced by adpositional phrases and/or \isi{serial verb} constructions in all branches of Bantoid. I map out the different analytic strategies that have been adopted and reconstruct the original verbal, nominal and pronominal sources of the different grammaticalization processes. Of particular interest is the development of a recipient/benefactive preposition ‘to, for’ from the word for ‘hand’ and a comitative/instrumental preposition ‘with’ from a third person plural pronoun.}
\epigram{All evidence points to the... hypothesis that such [analytic] languages are the logically extreme analytic developments of more synthetic languages which because of processes of phonetic disintegration have had to reexpress by analytical means combinations of ideas originally expressed within the framework of the single word. \citep[18--19]{Sapir1933[1949]}}
\maketitle
\begin{document}


% \isi{benefactive}, reciprocal, instrumental, \isi{comitative}, adpositions, grammaticalization, serial verbs



\section{Establishing a Proto-Bantoid synthetic system}\label{sec:hyman:1}

The general issue I address in this paper is how to account for the alternative grammaticalization strategies adopted as a highly synthetic (agglutinative) language develops towards analyticity. My focus will be on the multiple pathways that can be observed between the inherited head-marking verb structures of Proto-\ili{Bantoid} and the more analytical structures found in most of the daughter languages spoken today.\footnote{ The term “\ili{Bantoid}” is used in two senses in the literature. First, it refers to a node in the Niger-Congo family tree that includes both \ili{Bantu} and non-\ili{Bantu} languages; second, it refers to these latter non-\ili{Bantu} languages themselves. In most of my discussion I will be citing such \ili{Bantoid} languages which have evolved significantly further than their agglutinative \ili{Bantu} cousins.} As noted by \citet[187--188]{Dimmendaal2000}, among others, extensive head-marking occurs in at least some languages in all four of \citegen{Greenberg1963} macro-stocks: Nilo-Saharan, Afro-Asiatic, “\ili{Khoisan}”, and, as exemplified in \REF{ex:hyman:1}, multiple branches of Niger-Congo.

\ea%1
    \label{ex:hyman:1}
  \ea \label{ex:hyman:1a}   Seereer [“Atlantic” branch; Senegal]\\
   \gll a up-t-ik-t-ir-oox-k-a  apeel \\
    \textsc{3pl}.\textsc{sm} bury-\textsc{rev}-\textsc{goal}-\textsc{inst.appl-rec-refl-fut}-\textsc{infl}  shovels\\
    \glt ‘they’ll go unbury each other with shovels’ (John Merrill, pers.comm.) 

    \ex \label{ex:hyman:1b} Cicipu [Plateau/Central Nigerian branch; Nigeria]\\
    \gll zzá  nnà  ù-tób-ìl-ìs-ìs-u-wò-wò-nò=mu  sháyì~~~\\
    person    \textsc{rel} \textsc{3sg}-cool-\textsc{pl-caus-caus-v-anticaus-appl-perf=1sg}    tea\\
      \glt  ‘the person who has caused tea to become cooled down in a forceful and iterative fashion for me’ (\citealt[209]{McGill2009})
  
  \ex \label{ex:hyman:1c}
  Moro [Kordofanian; Sudan]\\ 
    \gll owːa          g-ubəð-i-tʃ-ən-ə-ŋó \\
     woman   \textsc{sm.cl}-run-\textsc{caus-appl}-\textsc{pass}-\textsc{perf}-\textsc{3sg}.\textsc{om}\\
     \glt ‘the woman was made to run away from him’ \citep[49]{Rose2013}
  
  \ex \label{ex:hyman:1d}
  \ili{Kinande} (\ili{Bantu}) [\ili{Bantoid} subbranch; Democratic Republic of Congo]\\
    \gll tu-né-mu-ndi-syá-tá-sya-ya-ba-king-ul-ir-an-is-i-á=ky-ô\\
    we-\textsc{tns/asp complex}-them-close-\textsc{rev-appl-rec-caus-caus-infl}=it\\
\glt ‘we will make it possible one more time for them to open it for each other’\\
(Philip Mutaka, pers.comm to \citealt[9]{Nurse2003})
\z 
\z 
% \todo{the \ili{Kinande} string is longer than the line width with "\isi{tense}/aspect". With "tns/asp" it fits.}

The example in \REF{ex:hyman:1d} is of most relevance to the present study, as it illustrates several of the most common \ili{Bantu} derivational suffixes known as verb extensions: \isi{causative}, \isi{applicative} etc. As I noted in \citet{Hyman2003}, the following valence-marking verb extensions tend to occur in the order Cau\-sa\-tive-Ap\-pli\-ca\-tive-Reci\-proc\-al-Pas\-sive (C-A-R-P) in what I shall refer to as Canonical \ili{Bantu} (CB):\\


\begin{table}
\caption{Verb Extensions in Bantu}
\label{tab:hyman:ex2} 
\begin{tabularx}{\textwidth}{Qlllll}
\lsptoprule
& \textit{Causative} & \textit{Applicative} & \textit{Reciprocal} & \textit{Passive} & \textit{(C-A-R-P)}  \\
\midrule
Proto-\ili{Bantu} & -ɪc- & -ɪl- & -an- & -ʊ- &   \\
Shona & -is- & -il- & -an- & -w- &   \\
Makua & -ih- & -il- & -an- & -iw- &   \\
\ili{Chichewa} & -its- & -il- & -an- & -idw- &  \\
\lspbottomrule
% \label{tab:hyman:1}
\end{tabularx}
\end{table}

\noindent
Of the above extensions, the \isi{causative} and \isi{applicative} add valence, while the reciprocal and passive decrease valence. In considering what has occurred within the related \ili{Bantoid} languages, I will be most concerned with how these languages compensate for the loss of valence-adding extensions, e.g. the \isi{applicative}, which has multiple functions in CB, illustrated from \ili{Chichewa} in \REF{ex:hyman:3}.

\ea \let\eachwordone=\rmfamily\upshape
\label{ex:hyman:3}
\textit{tum-ir-}  (send+\isi{applicative}) \\
\gll ‘{send for (s.o.),} {send to (s.o.),} {send with (sth.),} {send to (some place),} {send for (some reason)}’\\
\textsc{benefactive} \textsc{recipient } \textsc{ instrument} \textsc{  \isi{locative} } \textsc{ circumstance}\\
\z
 
  While CB languages are highly agglutinative, Northwest (NW) \ili{Bantu}\linebreak languages often have simpler structures, even extreme analyticity, as in \ili{Nzadi}, a “Narrow \ili{Bantu}” language spoken in the Democratic Republic of Congo which has lost valence-related suffixes, replacing them with the following analytic structures \citep{CraneEtAl2011}:

  
\ea
\label{ex:hyman:4}
\ea 
\isi{causative}:  \\
\gll yà  ó    líŋ mwàán kè  líì  \\
\textsc{2sg} \textsc{pst} want child \textsc{sbjv} cry \\
\glt ‘you made the child cry’ (lit. you wanted that the child cry)\\
\newpage 
\ex 
\isi{benefactive}:  \\
\gll bì   ó   súm mwàán òŋkàáŋ \\
\textsc{1pl} \textsc{pst}  buy   child     book  \\
\glt ‘we bought the child a book’  (double object)
\ex 
recipient:  \\
\gll bì   ó    pé   mwàán fùfú \\
\textsc{1pl} \textsc{pst}  give   child    fufu  \\
\glt ‘we gave the child fufu’
\ex 
\gll bì   ó    pé  fùfú  kó mwáàn \\
\textsc{1pl} \textsc{pst} give fufu to/for  child \\
\glt ‘we gave fufu to the child’
\ex 
instrument:  \\
\gll ndé ó  píŋ ntsúr  tí  mby\v{ɛ} \\
	\textsc{3sg} \textsc{pst} cut  meat with knife\\
\glt ‘he cut meat with a knife’
\ex 
circumstance:   \\
\gll ndé  á    sâl     sám {\downstep}é ndzíì\\
\textsc{3sg} \textsc{pres} work reason of money\\
\glt ‘he is working for money’
\z
\z

As can be seen, the above structures represent four different strategies for dealing with the loss of verb extensions: periphrasis (\ref{ex:hyman:4}a), unmarked double objects (\ref{ex:hyman:4}b,c), adpositions (\ref{ex:hyman:4}d,e) and nominal constructions (\ref{ex:hyman:4}f). Missing in \ili{Nzadi} is a fifth strategy, \isi{serial verb} constructions, which will be become central in the discussion of the \ili{Bantoid} developments discussed below.

\begin{quote}
While the historical changes that have taken place in \ili{Nzadi} definitely give it a ‘non-\ili{Bantu}’ feel, it is clear that \ili{Nzadi} derives from a quite canonical \ili{Bantu} type. \ili{Nzadi} ‘feels’ like a simplified \ili{Bantu} language rather than a \ili{Bantu} language which has developed West African Benue-Congo characteristics (e.g. \ili{Nzadi} does not have the ‘\isi{serial verb} constructions’ attested in Cameroon). \citep[3--4]{CraneEtAl2011}
\end{quote}

\noindent 
In this study I will assume that (pre-) Proto-\ili{Bantoid} was like Proto-\ili{Bantu} (PB) in having verb extensions (\isi{causative}, \isi{applicative}, etc.), multiple objects, and very few—perhaps even only one—adposition.\footnote{Only \textit{*na} ‘with, and’ can be confidently reconstructed for PB and early Niger-Congo.} This naturally raises the question of why synthetic head-marking languages like \ili{Kinande} and \ili{Chichewa} become analytic languages like \ili{Nzadi}? That is, why do such languages undergo such a dramatic change of typology? As far as I know, there have been three proposals in the literature:
The first is that the affixal morphology is lost through “processes of phonetic disintegration” (cf. the \citealt{Sapir1933[1949]} quote at the beginning of the paper). Known as “erosion” \citep[21--28]{HeineReh1984} or “phonological attrition” \citep[4]{Lehmann1985} in the grammaticalization literature, the change in typology is an innocent by-product of natural sound changes, particularly phonetic weakening and loss at word edges:
“The opposite historical \isi{directionality} towards analyticity proceeds mostly by way of erosion and loss of phonological and morphological substance”. \citep[129]{Güldemann2011}
  The second explanation attributes the development of analyticity to contact and imperfect learning by L2 speakers, ultimately leading to creolization.

\begin{quote}
 ... we [should] at least consider that these [analytic] languages’ grammars were incompletely acquired at some point in their history. This is a known cause of analyticity, whereas the idea of generations of first-language speakers ‘dropping’ \textit{all} of the affixes used by previous ones is peculiar at best and implausible at worst. \citep[226]{McWhorter2011}
\end{quote}

\begin{table}
\caption{Syllable length of verb stems in Chichewa vs. Nzadi}
\label{tab:hyman:ex5}
\fittable{
\begin{tabular}{lrrrrrr}
\lsptoprule
& {1${\sigma}$} &  {2${\sigma}$}  & {3${\sigma}$}  & {4${\sigma}$}  & {5${\sigma}+$}    & Totals\\
\midrule
\ili{Chichewa} &  30 (1.4\%)  & 650 (31\%)  & 906  (43.2\%) & 477  (22.8\%) & 22 (1.1\%) & 2095\\
\ili{Nzadi}    & 291 (83.9\%) & 51 (14.7\%) &   2  (0.6\%)  &   1  (0.3\%)  & 0   0\%     & 347\\
\lspbottomrule
\end{tabular}
}
\end{table}


In McWhorter’s account, phonetic erosion would have played little, if any role, in the development of the type of “radical analyticity” seen in \ili{Nzadi}.
  The third account proposed in \citet{Hyman2004} and subsequent papers is that morphology was lost as a result of imposing templatic constraints on stems (in this case, verb stems, which consist of a root + suffixes). Whereas PB did not have such limitations, the changes which took place included imposing a strong-weak structure highlighting the stem-initial CV and maximal size constraints on stems, which limited the ability of verb roots to occur with derivational suffixes. As will have been noted in \REF{ex:hyman:4}, words are very short in \ili{Nzadi}. Compare in \tabref{tab:hyman:ex5}, the number of verb stems having one to five syllables in \ili{Chichewa} vs. \ili{Nzadi}.\footnote{The numbers from \ili{Chichewa} are based on a lexical database of 5,862 entries in Filemaker Pro™ based on \citet{Scott1970} and tone-marked by Al Mtenje. The much smaller \ili{Nzadi} lexicon of 1,035 entries can be found in \citep[281--298]{CraneEtAl2011}.}
As seen, the vast majority of \ili{Nzadi} verbs are monosyllabic, with most of the bisyllabic verbs consisting of relic derived forms, e.g. \textit{dɛf} ‘borrow’ → \textit{dɛfsa} ‘lend’ ({\textless} ‘cause to borrow’). That monosyllabicity is the endpoint of a gradual process of limiting stem size can be seen from the following continuum in NW \ili{Bantu}:

\ea
\label{ex:hyman:6}
\ea 
four ({\textasciitilde}five) syllable maximum in   Yaka \citep{Hyman1998}, Bobangi \citep{Whitehead1899}
Punu (\citealt{Fontaney1980}, \citealt{Blanchon1995})
\ex 
three ({\textasciitilde}four) syllable maximum in Koyo \citep{Hyman2004}, Eton \citep{VandeVelde2008}
\ex 
three-syllable maximum  in   \ili{Tiene} \citep{Ellington1977}, Basaa (\citealt{Lemb1973}, \citealt{Hyman2003}), Kukuya \citep{Paulian1975}
\ex 
two ({\textasciitilde}three) syllable maximum   in  Mankon [Grassfields \ili{Bantu} (GB)] \citep{Leroy1982} 
\ex 
one ({\textasciitilde}two) syllable maximum in  \ili{Nzadi} \citep{CraneEtAl2011}
\z 
\z

  However, it is not just maximal stem size that is innovated, but also templatic prosodic constraints. This is most clearly seen in \ili{Tiene}, which allows a maximally trisyllabic stem having the following properties (\citealt{Ellington1977}, \citealt{Hyman2010}):


\ea
\label{ex:hyman:7}
\ea
five stem shapes:   CV, CVV, CVCV, CVVCV, CVCVCV
\ex 
in the case of C\textsubscript{1}V\textsubscript{1}C\textsubscript{2}V\textsubscript{2}C\textsubscript{3}V\textsubscript{3}: \\ 
    i.  C\textsubscript{2} must be coronal   \\ 
    ii.  C\textsubscript{3} must be non-coronal \\
    iii.  C\textsubscript{2} and C\textsubscript{3} must agree in nasality\\
    iv.  V\textsubscript{2} is predictable (with few exceptions)  
    \z 
\z 

\noindent
The effects of prosodic constraints on morphology can be quite dramatic. Thus, the coronal + non-coronal constraint on C\textsubscript{2} and C\textsubscript{3} can result in infixation, as in (\ref{ex:hyman:8}b,c).

\ea
\label{ex:hyman:8}
\ea 
CB \textit{-ik-} ‘\isi{stative}’:  \textit{ból-a}  ‘break’  →  \textit{ból-ek-ɛ}  ‘be broken’
\ex 
CB \textit{-is-} ‘\isi{causative}’:   \textit{láb-a}  ‘walk’  →   \textit{lá\ul{s}ab-a}    \textit{‘cause to walk’}
\ex 
CB \textit{-il-} ‘\isi{applicative}’:  \textit{bák-a}  ‘reach’  →  \textit{bá\ul{l}ak-a}  \textit{‘reach for’}
  \z 
\z 

  {While McWhorter’s and my explanations both state that more needs to be involved than phonetic erosion, it is unlikely that the innovated infixation process in (\ref{ex:hyman:8}b,c) would have resulted from “incomplete acquisition”. Instead, as I argued in \citet{Hyman2004}, Niger-Congo languages become analytic by the stages outlined in \REF{ex:hyman:9}.}

\ea
\label{ex:hyman:9}
\ea 
start with a full set of (stacked) verb extensions (\isi{causative}, \isi{applicative}, etc.) and multiple objects
\ex 
size (and other prosodic) constraints come to be imposed: 4${\sigma}$ {\textgreater} 3${\sigma}$ {\textgreater} 2${\sigma}$ maximum
\ex 
such maximality constraints result in longer verbs not being able to take extensions
\ex 
to accommodate these verbs, analytic alternatives are favored (and created, if not preexistent)
\ex 
these alternatives come to be used even with shorter verbs, with extensions becoming less favored
\ex 
former valence-related extensions take on new, especially aspectual functions (e.g. various pluractional meanings), or drop out
\z 
\z 

Turning to \ili{Bantoid}, as an example of (\ref{ex:hyman:9}f), \isi{causative} \textit{-sə} has become an iterative extension in \ili{Bangwa} [GB, Bamileke; Cameroon] \citep[243]{Nguendjio1989} in \REF{ex:hyman:10}, while several of the inherited verb extensions have taken on pluractional meanings in \ili{Kejom} [GB, Ring subgroup; Cameroon] (\citealt{Jisa1977}, \citealt{Akumbu2008}) in \REF{ex:hyman:11}.


\ea 
\label{ex:hyman:10}
\begin{tabular}[t]{@{}llcll@{}}
  \textit{sò } & ‘laver’    & → & \textit{sò-sə } & ‘laver plusieurs fois’\\
  \textit{fák} & ‘tourner’  & → & \textit{fák-sə} & ‘tourner plusieurs fois’\\
  \textit{cí-} & ‘casser’   & → & \textit{cí-sə } & ‘casser plusieurs fois’\\
  \textit{yàʔ} & ‘couper’   & → & \textit{yàʔ-sə} & ‘couper plusieurs fois’\\
  \textit{ghɛ} & ‘partager’ & → & \textit{ghɛ-sə} & ‘partager plusieurs fois’\\
\end{tabular}
\z
  
\ea
\label{ex:hyman:11}
\ea
\begin{tabbing}
 tsɔʔ-mə~~ \= ‘jump one after the other’\kill 
 \textit{tsɔʔɔ  } \>   ‘jump’  \\
 \textit{tsɔʔ-mə} \>  ‘jump one after the other’\\
 \textit{tsɔʔ-kə} \>  ‘jump time and again’\\
 \textit{tsɔʔ-lə} \>  ‘jump across things’\\
 \textit{tsɔʔ-tə} \>  ‘jump gently’ (= attenuative)
\end{tabbing}
 \ex 
\begin{tabbing}
 tsɔʔ-mə~~ \= ‘jump one after the other’\kill 
 \textit{dì   } \> ‘cry, cackle’ \\ 
 \textit{dì-mə} \>  ‘lots of children crying’\\
 \textit{dì-kə} \>  ‘cry time and again’\\
 \textit{dì-lə} \>  ‘lots of chickens cackling’
\end{tabbing}\newpage
 \ex 
  \begin{tabbing}
 tsɔʔ-mə~~ \= ‘jump one after the other’\kill 
 \textit{zhwí   } \>  ‘kill’  \\
 \textit{zhwí-tə} \>  ‘kill one by one, bit by bit’\\
 \textit{zhwí-lə} \>  ‘kill lots of people, one after the other’
  \end{tabbing}
 \ex 
  \begin{tabbing}
 tsɔʔ-mə~~ \= ‘jump one after the other’\kill 
 \textit{sù   } \> ‘stab’  \\
 \textit{sù-tə} \> ‘stab lots of things one by one, or one thing many times’\\
 \textit{sù-lə} \> ‘stab with lots of things at one time’
 \end{tabbing}
 
 \z 
\z 
  
  
  To summarize, major changes transformed an originally agglutinative proto language into much more analytic daughter languages in some of NW \ili{Bantu} and \ili{Bantoid}. As a result, non-\ili{Bantu} \ili{Bantoid} languages differ considerably from CB, as summarized in \REF{tab:hyman:ex12}.

\begin{table}
\caption{Comparison of Canonical Bantu with Non-Bantu Bantoid}
\label{tab:hyman:ex12}
\begin{tabularx}{\textwidth}{lQQ}
\lsptoprule
 & {Canonical Bantu} & {Non-\ili{Bantu} Bantoid}\\
\midrule
{phonology} & minimum word = 2 syllables & maximum stem =mostly 2{\textasciitilde}3 syllables\\
\tablevspace
{morphology} & highly synthetic, agglutinative & less so, gradual move towards analyticity\\
\tablevspace
{verb extensions} & many, mostly marking valence & few, mostly marking aspect\\
\tablevspace
{unmarked objects} & multiple & at most two, ultimate limitation to one per verb\\
\tablevspace
{object marking} & head marking on verb & various prepositions and/or serial verbs [diversity!]\\
\tablevspace
{di\isi{transitive} verbs} & a few (*pá ‘give’) & few or none\\
\lspbottomrule
\end{tabularx}
\end{table}
\noindent
Having established that Proto-\ili{Bantoid} had a range of verb extensions, I now consider the structures which have come to replace them in the daughter languages.

\section{Analytic replacements of the lost Proto-Bantoid synthetic structure}\label{sec:hyman:2}

In this section I examine what has replaced the verb extension system inherited by languages in the \ili{Bantoid} area of Cameroon. In order to control the study, I focused exclusively on the marking of valence by head marking on the verb, specifically benefactives (‘for someone’), recipients (‘to someone’) and instruments (‘with something’). As will be seen, \ili{Bantoid} languages either innovate adpositional phrases, \isi{serial verb} constructions, or both. This therefore raises two questions. First, where did \ili{Bantoid} languages get their prepositions (or, in a few cases, postpositions)? Recall that the proto language may have only had one \isi{preposition} \textit{*na}, which occurs widely in Niger-Congo.
\begin{quote}
 A feature common to languages that have obligatory applicatives and to languages that have the type of complex \isi{predicates} presented in section 4.3.6 [\isi{serial verb} constructions] is that, in comparison with other languages, they make only a very limited use of adpositions, since adpositions typically encode the semantic role of obliques, and both mechanisms result in giving the status of direct objects to various semantic types of complements that in other languages tend to be treated as obliques. \citep[124]{CreisselsEtAl2008}
\end{quote}
The second question concerns how \ili{Bantoid} languages developed their \isi{serial verb} constructions (SVCs)? In order to investigate these questions, I decided to survey what has replaced the \isi{benefactive} and recipient functions of the CB \isi{applicative extension} \textit{-il-} and the common \textit{-an-} suffix which marks reciprocal in CB, but also instruments in Cameroonian NW \ili{Bantu}:
\ea
\label{ex:hyman:13}
\ea 
Mokpe [A22] \citep{Henson2001}\\
\gll -sos-\\
‘wash’\\
\gll -sos-an-\\
‘{wash with}’\\  
\ex 
\ili{Akoose} [A15C] \citep[90]{Hedinger2008}\\
\gll -kób-\\
‘catch’\\
\gll -kób-ɛn-\\
‘{catch with}’ \\
  \z 
\z 

From the available literature, aid of colleagues over email, and my own work, the goal was to fill out the following questionnaire for as many as possible of the ca. 100 \ili{Bantoid} languages in this small area of Cameroon.\newpage
\begin{enumerate}
\item
How are benefactives expressed? Which of the following are possible for the meaning ‘he cooked rice for the~child’?\\
  a.  \textsc{double object}: “cook~child~rice” \\
  b.  \textsc{\isi{benefactive} preposition}: “cook rice for child” [if yes, what is the \isi{preposition}?]\\
  c.  \textsc{\isi{serial verb} construction}: “cook rice give child” 
\item 
How are recipients expressed? Which of the following are possible for the meanings ‘he gave the child a book’ or ‘he sent/wrote the woman a letter’? [They are not necessarily the same]\\
  a.  \textsc{double object}: “write woman letter”, “give child book”\\
  b.  \textsc{recipient preposition}: “write letter to woman”, “give book to child” [if yes, what is the \isi{preposition}?] \\
  c.  \textsc{\isi{comitative} preposition}: “write woman with letter”, “give child with book” [if yes, what is the \isi{preposition}?]\\
  d.  \textsc{\isi{serial verb} construction}: “write letter give woman”, “take book give child”
\item     
How are instruments expressed? Which of the following are possible for the meaning ‘he cut the meat with a knife’?\\
  a.  \textsc{instrumental preposition}: “cut meat with knife” [if yes, what is the \isi{preposition}?] \\
  b.  \textsc{\isi{serial verb} construction}: “take knife cut meat”
\end{enumerate}

  The table in the Appendix presents findings from 27 languages. Concerning the marking of ditransitives (benefactives, recipients, instruments), the following generalizations were noted:
  \begin{itemize}
\item[(i)] In all subareas there is at least some resistance to multiple objects, which are often restricted to only a few verbs.
\item[(ii)] There is no \isi{applicative} or instrumental valence-marking by verb extensions, whereas there are identifiable, though not necessarily productive \isi{causative} extensions in many \ili{Bantoid} languages.
\item[(iii)] Virtually all of the flagging and \isi{word order} strategies summarized by \citet{Malchukov2010} are found in this small area, e.g. both adpositions and \isi{serial verb} constructions (SVCs), which represent different responses to the change from syntheticity to analyticity.
\end{itemize}

As mentioned, \ili{Bantoid} languages do retain verbs with recognizable \isi{causative} suffixes. However, \isi{causative} \textit{-sə}, which corresponds directly to CB \textit{-is}-, is usually restricted to \isi{intransitive} roots due to the widespread resistance to double object constructions. In the few transitives that have been found with a \isi{causative} extension, the verb does not become ditransitive:
    
\ea
\label{ex:hyman:14}
\ea 
Babungo \\ 
\textit{ŋw\'{ə} fèe zɔ̏                    }  ‘he was afraid of (i.e. feared) a snake’\\
\textit{m\`{ə} fè-s\`{ə} ŋw\'{ə} (n\`ə zɔ̏)}  ‘I frightened him (with a snake)’\\
\citep[211]{Schaub1985}
\ex 
Bafut \\
\textit{má shwìʔì ŋki}  ‘I am pouring water’\\
\textit{má shwìʔì-s\`ə ŋkì}   ‘I am making water to pour’\\
\citep[102]{Bila1986}
\z 
\z 

  While \isi{causative} extensions are attested, reflexes of the CB \isi{applicative} suffix \textit{-il-} are virtually absent in the \ili{Bantoid} area. One possible exception concerns six out of \citegen{Ngum2004} lexicon of 262 verbs in Meta [GB; \ili{Momo} subgroup]:
 
\eabox{
\label{ex:hyman:15}
\begin{tabular}{llll}
\textit{ghàb} &  ‘share’  & \textit{ghàb-rɨ} & ‘share to’  \\
\textit{cob } & ‘donate’  & \textit{cob-rɨ } &  ‘donate for’  \\
\textit{sòm } & ‘cut’  &    \textit{sòm-bɨ } &  ‘cut into’\\
\textit{wí  } &‘refund’  &  \textit{wíí-rɨ } &  ‘reply, refund to’\\
\textit{wub } & ‘crave’  &  \textit{wub-rɨ }  & ‘crave for’\\
\textit{dìì }  &‘pity’  &   \textit{dìì-rɨ } &  ‘pity for’\\
\end{tabular}
} 

\noindent
However, since \textit{-rɨ} has other functions, it is not clear if this suffix is cognate with PB \isi{applicative} \textit{*-ɪl-}. The only other \isi{applicative} I have found in the area comes from \ili{Vute} (Mambiloid), which has innovated a new extension \textit{-nà} from the main verb ‘to give’.
 “\textit{-nà} is added to a verb to indicate that there is an \isi{indirect object} or \isi{benefactive} NP present in the clause. Its function is similar to a \ili{Bantu} \isi{applicative extension} in this way. \textit{-nà} is derived from the verb \textit{nà-nɨ} ‘to give’.” \citep[8]{Thwing2006}
  \tabref{tab:hyman:1} summarizes the different constructions that replace former \isi{applicative} and instrumental verb extensions.

\begin{table} 
\fittable{ 
\begin{tabular}{lllll}
\lsptoprule\il{Bantoid}
{alignment} & {schema} & {Benefactive} & {Recipient} & {Instrument}\\
\midrule
neutral & verb + X + Y & \textit{cook child rice} & \textit{write woman letter} & \\
indirective & verb + X + [prep Y] & \textit{cook rice for child} & \textit{write letter to woman} & \\
secundative & verb + Y + [prep X] & \textit{cook child with rice} & \textit{write woman with letter} & \textit{cut meat with knife}\\
co-verb (Y) & verb + X + [give Y] & \textit{cook rice give child} & \textit{write letter give woman} & \\
co-verb (X) & [take X] + verb + Y &  &  & \textit{take knife cut meat}\\
\lspbottomrule
\end{tabular}
}
\caption[Benefactive, Recipient, and Instrumental Constructions in Bantoid]{\raggedright Benefactive, Recipient, and Instrumental Constructions in Bantoid}
\label{tab:hyman:1}
\end{table}

Although some languages do maintain unmarked double objects, assumed to be inherited, the more pervasive strategies are to replace head marking with adpositions and/or SVCs, with subareal distributions (see below). Let us first consider prepositions, then serial verbs.
  As mentioned, the proto language had perhaps only one \isi{preposition}, \textit{*na} ‘with’ whose various reflexes \textit{nə, nɨ, ni, nɛ} may expand to take on all three functions ‘for’, ‘to’, and ‘with’, as in \ili{Limbum} [Eastern GB] \citep[259]{Fransen1995}:
  
\ea
\label{ex:hyman:16}
\ea 
\gll wìr bí    fàʔ   nì  Tàr\={i} \\
    we \textsc{fut}\textsubscript{0} work for  lord\\
\glt ‘we will serve [work for] the Lord’
\ex 
\gll mȅ   fā      ŋwàʔ  nì   mūū  wȁ \\
I  give-\textsc{perf} book   to    child  my \\
\glt ‘I have given a book to my child’
\ex 
\gll mȅ   gwàr    c\={i}     nì  ndyàà \\
I    cut-\textsc{perf}  tree  with  axe\\
\glt ‘I have cut the tree with an axe’
\z 
\z 

\begin{table}[t]
\caption{Possessive vs. Locative Agreement in Noni}
\label{tab:hyman:ex17}
\fittable{
\begin{tabular}{ll>{\itshape}l>{\itshape}llr}
\lsptoprule
a. & cl.3 & wáy & w-\'ɛm & ‘my market’ &  (‘at market of me’)\\
&  & f\`ɔ-wǎy & f\=ɔ mē & ‘at my market’   & \\
b. & cl.9 & j\`ɔ\`ɔ & y-\`ɔ & ‘your sg. stream’   & (‘in stream of you sg.’)\\
&  & \`ɛ-j\'ɔ\`ɔ & j\=ɔ w\`ɔ & ‘in your sg. stream’   & \\
c. & cl.9 & còn & y-è & ‘his/her hut’ &   (‘in hut of him/her’)\\
&  & cōǹ & dvū wvù & ‘in his/her hut’   & \\
\lspbottomrule
\end{tabular}
}
\end{table}

  In other cases the source of the \isi{preposition} is from a \isi{locative}. \ili{Bantu} languages have \isi{locative} noun classes that condition agreement. These are also present in certain \ili{Bantoid} languages, although not always easy to identify with PB. Thus, \ili{Noni} [Beboid] \textit{fɔ} in \tabref{tab:hyman:ex17}a is cognate with PB \textit{*pa}, while the other two \isi{locative} noun classes in \tabref{tab:hyman:ex17}b,c have no known PB correspondence \citep{Hyman1981}.
A comparison of the \isi{possessor} marking in these examples reveals that independent pronouns are used instead of \isi{possessive} pronouns with the above \isi{locative} classes, indicating that they are prepositions.
  I suggest that the same \isi{locative} source is involved in the development of the widespread \isi{preposition} \textit{á {\textasciitilde} án} which comes to be used as a \isi{benefactive} and/or recipient \isi{preposition}, e.g. in \ili{Noni}, where the synchronic reflex of \textit{*a} is [ɛ] \citep[80]{Hyman1981}:
  
\ea
\label{ex:hyman:18}
\ea 
\gll mē  n\'ɔ\`ɔ     nd\`ɛ\`ɛ   w{\textlowrisea}n  bèŋkfǔ \\
I   \textsc{perf.foc} cook   child   yams\\
\glt ‘I have cooked the child yams’
\ex 
\gll mē  n\'ɔ\`ɔ     nd\`ɛ\`ɛ  bèŋkfǔ  \=ɛ   wān \\
I  \textsc{perf.foc}  cook    yams   for  child\\
\glt ‘I have cooked yams for the child’
\z 
\z 

\noindent
Assuming an earlier NP PREP NP structure explains the unusual verb + X + Y \isi{word order} in \ili{Medumba} [GB; Bamileke], which has lost the \isi{preposition} \textit{*á}, but still uses the independent pronominal forms as “\isi{indirect object} pronouns” \citep[22]{Voorhoeve1976}:

\ea
\label{ex:hyman:19}
\ea 
\gll á  {\downstep}f\'ɑ  é{\downstep}é  bó \\
he gave it them\\
\glt ‘he gave it to them’  (cf. \isi{direct object} pronoun yób ‘them’)
\ex 
\gll  a\textsuperscript{5}  fɔ\textsuperscript{3}  bum\textsuperscript{2}  bu\textsuperscript{3}\\
he give  egg    dog\\
\glt ‘he gives an egg to the dog’  \citep[2]{Caroompas2014}
\z 
\z 

Two other areal developments can be noted from the data in the Appendix and compared with the accompanying map (\figref{fig:hyman:1}). First, in a contiguous area involving two subgroups of Grassfields \ili{Bantu} (Eastern Grassfields and \ili{Momo}), the \isi{benefactive}/recipient \isi{preposition} is reinforced by the noun ‘hand’ (cf. PB \textit{*-bókò}); hence, ‘to the hand(s) of s.o.’ becomes a new, fuller \isi{preposition}.\footnote{Note that \citet[166]{Heine2002} have ‘hand’ {\textgreater} LOCATIVE, but not RECIPIENT.} Elizabeth Magba (pers.comm.) thus points out the following two possibilities in \ili{Mundani} [GB; \ili{Momo}]:
  
\ea
    \label{ex:hyman:20}
\ea 
\gll tà    tsaa   àkate  yu  abua  tò\\
s/he has-sent letter   the    to   him/her \\
\glt ‘s/he has sent the letter to her/him’ 
\ex 
\gll tà    tsaa     a  tò   àkate  yu \\
s/he has-sent  to him/her   letter   the\\
\glt (idem)
\z 
\z 

\begin{quote}
  The difference between examples [\ref{ex:hyman:20}a] and [\ref{ex:hyman:20}b] in terms of marking the recipient role has to do with a difference in focus: in [\ref{ex:hyman:20}a] \textit{abua tò} (the recipient) is in focus, appearing in clause-final position; in [\ref{ex:hyman:20}b], \textit{àkate yu} (the item sent...) is brought into focus by being shifted to the clause-final position.... The origin of \textit{abua} variously translated as ‘to, for, from, with’ is likely to be the noun \textit{àbu} ‘hand, arm’, possibly suffixed by the Class 7 genitive marker \textit{-a}. (Elizabeth Magba, pers.comm.)
\end{quote}

\begin{figure}
\parbox[t]{.1mm}{
\vspace*{12.5cm}
\scriptsize
\vfill
 \begin{forest}
 for tree={grow'=0,folder, s sep=-2mm}
 [Wide grassfields
  [\textit{Ambele} (1)]
  [Western \ili{Momo} (2)]
  [Menchum (3)]
  [Narrow grassfields
    [\ili{Momo} (4)]
    [Ring 
     [South (5)]
     [East (6)]
     [Center (7)]
     [West (8)]
    ]
    [\textit{Ndemli} (9)]
    [Eastern
      [North (10)]
      [Mbam-Nkam
	[Nun (11)]
	[Bamileke (12)]
	[Ngemba (13)]
      ]
    ]
  ]
 ]
\end{forest} 
}
% \includegraphics[width=.45\textwidth]{figures/a4Hyman-img1.png}
\parbox[t]{.9\textwidth}{
\vspace*{0pt}
\includegraphics[width=.9\textwidth]{figures/hymanmap.pdf}
}
\caption{Map of languages surveyed (base map from \citealt[226]{Watters2003})}
\label{fig:hyman:1}
\end{figure} 

% \begin{figure}
%  \begin{forest}
%  for tree={grow'=0,folder, s sep=-2mm}
%  [Wide grassfields
%   [\textit{Ambele} (1)]
%   [Western \ili{Momo} (2)]
%   [Menchum (3)]
%   [Narrow grassfields
%     [\ili{Momo} (4)]
%     [Ring 
%      [South (5)]
%      [East (6)]
%      [Center (7)]
%      [West (8)]
%     ]
%     [\textit{Ndemli} (9)]
%     [Eastern
%       [North (10)]
%       [Mbam-Nkam
% 	[Nun (11)]
% 	[Bamileke (12)]
% 	[Ngemba (13)]
%       ]
%     ]
%   ]
%  ]
% \end{forest}
% \caption{\color{red}No caption provided}
% \end{figure}

\noindent
It is likely that \ili{Isu} [GB; Ring] \textit{áw\`ɔ} ‘for’ (\isi{benefactive}) derives from \textit{á + k\`ə-w\'ɔ} ‘hand’ (with common prefix-deletion and tonal change) and that \textit{áw\`ɔ} subsequently developed into \textit{â} ‘to’ (recipient) (Roland Kießling, pers.comm.):

\ea
\label{ex:hyman:21}
\ea 
\gll ɣú   fàʔà    áw\`ɔ  d\`ɔŋ  k-ìy \\
\textsc{3pl}  work.\textsc{ipf}   for   king  \textsc{7}-\textsc{of}\\
\glt ‘they worked for the king’
\ex 
\gll ú  k\`ɔʔ  y\`ə  wè  dzài  y\`ə  â  wè \\
\textsc{3s}.\textsc{pst3} see \textsc{cfg}  \textsc{3sg}  tell  \textsc{cfg} to  \textsc{3sg}\\
\glt ‘s/he saw him/her and told him/her’
\z 
\z 

\noindent
Locative \textit{á} is implicated in the similar development of the \isi{benefactive} and recipient \isi{preposition} \textit{â} in closely related \ili{Aghem} [GB; Ring group] \citep[152--8]{Watters1979}, but also marks instruments by itself \citep[45]{Hyman1979}:

\ea
\label{ex:hyman:22}
\ea 
\textit{á  f\'ɨghàm         }  ‘on the mat’\\
\textit{á  k\'ɨ{\downstep}tú}  ‘on the head’ 
\ex 
\textit{á  f\'ɨ{\downstep}ñɨ}  ‘with a knife’ \\ 
\textit{á  k\'ɨkɔŋ          }  ‘with a stirring stick’
\z 
\z 

  The second areal development concerns a new instrumental \isi{preposition} \textit{*b\'ɔ} which replaces \textit{*na} ‘with’ in the North (Jukunoid, Yemne-Kimbi, Beboid, Northern subbranch of Eastern Grassfields). As seen in the \ili{Noni} examples in \REF{ex:hyman:23} \textit{b\'ɔ} is used with persons, instruments and secundative ‘give Y with X’:

\ea
\label{ex:hyman:23}
\ea 
\gll me  nt\'ɔ\'ɔ  b\'ɔ  w{\textlowrisea}n\\
I   come with child \\ 
\glt ‘I am bringing the child’ 
\ex 
\gll me n\'ɔ\`ɔ  ns\`ɛ\=ɛ  ñàm  b\'ɔ  fèñ{\texthighriseo} \\ 
I   \textsc{perf}   cut   meat with knife\\
\glt ‘I have cut meat with a knife’ 
\ex 
\gll me cí  ñá    bɔɔm   b\'ɔ  kèŋg\`ɔm\\
I  \textsc{pst}\textsc{\textsubscript{2}} give children with plantains\\
\glt ‘I gave the children plantains’
\z 
\z 

\noindent
Consistent with earlier speculations, the likelihood is that this \isi{preposition} comes from the third person plural pronoun of the same shape: incorporative ‘they-with s.o.’ {\textgreater} associative ‘they-with sth.’ (‘they left they-with load of yams’) {\textgreater} instrumental ‘with’.
“... perhaps \textit{b\'ɔ} ‘with’ comes from \textit{b\'ɔ} ‘they’.” (\citealt{Hyman1981}: 81, re \ili{Noni})
 “This conjunction [\textit{b\'ə} ‘and’] is identical in form to the third person plural pronoun from which it is probably derived.” (\citealt{Hedinger2008}: 72, re \ili{Akoose})\footnote{ Note that ‘and’ and ‘with’ are often expressed with the same morpheme in \ili{Bantu} languages.}
The likely starting point is incorporative pronouns, widespread in this area, e.g. \ili{Akoose} \citep[73]{Hedinger2008}:

\ea
\label{ex:hyman:24}
\ea 
\gll b\'ə   awi  mwaád\\
they his   wife \\ 
\glt ‘he and his wife’  (i.e. they including his wife)
\ex 
\gll b\'ə  María\\
they Mary\\
\glt ‘s/he and Mary’  (i.e. they including Mary)
\ex 
\textit{súm\=ə}\\
\glt ‘s/he and I’   (lit. we-(s)he’)
\z 
\z 

A diachronic development of \isi{comitative} {\textgreater} instrumental is a very common one cross-linguistically \citep[292]{CreisselsVoisin-Nougier2008}. As seen in \REF{ex:hyman:25}, both the new \isi{preposition} {\textless} ‘they’ and inherited \textit{*na} form secundative \textit{verb Y with X} in the North and Ring groups:\largerpage

\ea
\label{ex:hyman:25}
\ea 
\gll m\=ə   fà  w\`ə   b\'ə    ndì \\ 
I    give you  with water\\
\glt ‘I give you some water’\\
(Koshin [Yemne-Kimbi]; \citealt[309]{Ousmanou2014})
\ex 
\gll m\`ə  k\`ɔ  Làmbí  n\`ə   fá\\  
I  give Lambi with thing\\
\glt ‘I give something to Lambi’\\
(Babungo [GB; Ring]; \citealt[60]{Schaub1985})
\z 
\z 

\noindent
It can be noted that no \ili{Bantoid} language has secundative ‘Y with X’ without also having an alternative ‘X to Y’.

Leaving prepositions, another areal development is \isi{serial verb} constructions (SVCs) which have also been innovated to express multiple arguments in \ili{Bantoid}:
  
\ea
\label{ex:hyman:26}
\ea 
Benefactive ‘give’ (\ili{Bamun} [GB; Nun]) \\
\gll nasha           na     malori  mfa  ne  pon  \\
my.mother cook.\textsc{pst}  rice     give   to children\\
\glt ‘my mother cooked rice for the children’ (Abdoulaye Nchare, pers.comm.)
\ex 
Benefactive ‘give’ (\ili{Mundani})\\
\gll tà   lè    la̹a̹  èghɨdzɨ  ŋa  abua  tò \\ 
she \textsc{pst}\textsubscript{3} cook   food  give  to  him\\
\glt ‘she cooked food for him’ (Elizabeth Magba, pers.comm.)  
\ex 
Instrumental ‘take’ (Ngomba [GB; Bamileke]) \\
\gll n   d\v{ɔ}k       níi      \'{ŋ}-kxɰ\=ɤʔ  t\'ɯ \\
I take.\textsc{pst} machete  \textsc{cns}-cut tree\\
\glt ‘I cut the tree with a machete’ \citep[60]{Satre2010}  
\z 
\z 

  From the table in the Appendix, we can make the following observations concerning the distribution of SVCs: (i) ‘give’ and ‘take’ SVCs are definitely in the minority (see the numbers in the bottom row of the table); (ii) except for \ili{Mbembe} [Mambiloid] in the North and \ili{Ejagham} [Ekoid \ili{Bantu}] in the South, SVCs are found throughout the Grassfields area except the Ring group; (iii) although ‘give’ and ‘take’ SVCs are absent, Ring Grassfields \ili{Bantu} exploits SVCs in other functions. This is extensively documented by \citet{Kießling2011}  for \ili{Isu} and can also be seen in the following example from closely related \ili{Aghem} \citep[204]{Hyman1979}:

\ea
\label{ex:hyman:27}
\gll sǒog\`ɔʔ  v\'ʉ  ndùu nùŋò  èk\'ɔ̞ʔ  z\`ɨghà  màʔà  tsùghò  áw\'ɛ,  nùŋò  èndú  ndùu k\`ɔʔ  ndùu nùŋ\`ɔ  v\`ʉ \\
soldier  that   go   leave ascend  leave  throw  descend children leave    go      go    see     go  woman that \\
\glt ‘the soldier went and abandoned his children and went to see the woman’
\z 

\noindent
The absence of valence-related serial verbs in the Ring subgroup is consistent with \citegen{Foley1985} observation that SVCs are expected to be acquired in the specific order: motion/directional verbs  {\textgreater}  postural verbs  {\textgreater}  \isi{stative}/process verbs  {\textgreater}  valence.
 “On the grammatical side, phonological attrition causes gradual loss of the bound morphemes.... As this verbal morphological is lost, a new device for valence adjustment must be found. Verb serialization begins to be used in this function, \textit{provided serial constructions already exist in the language}.” \citep[51, my emphasis]{Foley1985}

Concerning the order in which different valence SVCs are acquired, the pres\-ent survey of \ili{Bantoid} languages suggests two generalizations. First, ‘give’ SVCs are acquired before ‘take’ SVCs. Thus, \ili{Mfumte} [EG; North] uses a ‘give’ SVC for benefactives, but a \isi{preposition} \textit{w\'ə} ‘with, to’ instead of an instrumental ‘take’ SVC (Greg McLean, pers.comm.):

\ea
\label{ex:hyman:28}
\ea
\gll y\'ə   tó    fá  m\`ə  nku \\
\textsc{3sg} call give  \textsc{1sg}  chief\\
\glt ‘s/he called the chief for me’
\ex 
\gll y\'ə   sɨ  ngyaʔ  w\'ə  mbyì \\
\textsc{3sg} cut  meat   with  knife \\
\glt ‘s/he cut meat with a knife’
\z 
\z 

\noindent
Second, \isi{benefactive} ‘give’ SVCs are acquired before recipient ‘give’ SVCs. Evidence for this has already been seen from \ili{Mundani} (\ref{ex:hyman:20}a) ‘send to’ vs. (\ref{ex:hyman:26}b) ‘cook give’, repeated below (Elizabeth Magba, pers.comm.):

\ea
\label{ex:hyman:29}
\ea 
\gll tà    tsaa   àkate  yu  abua  tò  \\
s/he has-sent letter   the    to    her/him \\
\glt ‘s/he has sent the letter to her/him’
\ex 
\gll tà    lè     la̹a̹  èghɨdzɨ  ŋa  abua  tò \\
s/he  \textsc{pst}\textsubscript{3}  cook food  give  to  her/him\\
\glt ‘s/he cooked food for her/him’
\z 
\z 

\noindent
Fe’fe’ [GB; Bamileke] also supports the idea that ‘give’ is initially oriented towards the \isi{benefactive} rather than the recipient (\citealt{Hyman1971}; pers.notes):\footnote{In these examples \textit{náh} is a common simplification of \textit{ndáh}, the consecutivized form of \textit{làh} ‘take’. The RECIP marker \textit{mbú} is derived from the plural ‘hands’.}

\ea
\label{ex:hyman:30}
\ea 
\gll à    k\`ɑ   láh c\`ɑk  náh   ns\`ɑʔ  mbú   à\\
\textsc{3sg} \textsc{pst}\textsubscript{2} take pot \&take \&come to  me\\
\glt ‘s/he brought the pot to me’
\ex 
\gll à    k\`ɑ   láh  c\`ɑk  náh   ns\`ɑʔ   h\=ɑ   ā  \\
\textsc{3sg} \textsc{pst}\textsubscript{2} take pot \&take \&come give me\\
\glt ‘s/he brought the pot for me’
\ex 
\gll à    k\`ɑ   láh  c\`ɑk  náh   ns\`ɑʔ   h\=ɑ   mbú  à  \\
\textsc{3sg} \textsc{pst}\textsubscript{2} take pot \&take \&come give to me\\
\glt ‘s/he brought the pot to me’
\ex 
\gll à    k\`ɑ   láh  c\`ɑk  náh   ns\`ɑʔ  mbú   à  h\=ɑ   ā  \\
\textsc{3sg} \textsc{pst}\textsubscript{2} take pot  \&take \&come to me give me \\
\glt ‘s/he brought the pot to me for me’ (helped get the pot to me)
\z 
\z 

\noindent
The Fe’fe’ data underscore that there are alternatives—and combinations, e.g. ‘verb + give + to’. In addition, there is a \isi{preposition} \textit{mɑ} ‘with’ which has the same functions as \textit{láh} ‘take’ \citep[33--37]{Hyman1971}.

\ea
\label{ex:hyman:31}
\ea 
\gll à    k\`ɑ   f\'ɑʔ     m\`ɑ     žínù  \\
\textsc{3sg} \textsc{pst}\textsubscript{2} work   with intelligence \\
\glt ‘he worked intelligently’ (he worked with intellligence)
\ex 
\gll à    k\`ɑ   láh      žínù       mfáʔ \\
\textsc{3sg} \textsc{pst}\textsubscript{2} take intellligence \&work \\
\glt ‘he worked intelligently’ (he took intelligence \&worked)
\ex 
\gll à    k\`ɑ   láh      žínù       náh   mf\`ɑʔ  \\
\textsc{3sg} \textsc{pst}\textsubscript{2} take intelligence \&take \&work\\ 
\glt ‘he worked intelligently’ (he took intelligence \&took \&worked)
\z 
\z 

This leaves us with the question: Why do \ili{Bantoid} (and other) languages develop multiple strategies in the passage from syntheticity to analyticity? I take this up in the final section.

\section{Conclusion}\label{sec:hyman:3}

In response to why languages might develop alternative analytic structures, first consider the use of serialized ‘take’ as a “linker” in Fe’fe’ in \REF{ex:hyman:32}.

\ea
\label{ex:hyman:32}
\ea 
\gll à´    mf\'ɑʔ      náh   ngh\v{ɯ}  nk\=ɑɑ \\
\textsc{3sg} work.\textsc{pres} \&take \&make money\\
\glt ‘s/he works and thereby earns money’
\ex 
\gll à´    ncēh      náh   nj\={i}ʔs\={i}  wū \\
\textsc{3sg}  read.\textsc{pres} \&take  \&learn thing \\
\glt ‘s/he reads and thereby learns’
\z 
\z 

\noindent
As seen, I have translated ‘\&take’ as ‘thereby’, since it refers back to a proposition, not to a noun phrase. This is something that \textit{m\`ɑ} ‘with’ cannot do. Besides its ability to express a wider range of semantic roles than the \isi{preposition} ‘with’, ‘take’ can also acquire an aspectual function, e.g. marking \isi{completive aspect} in \ili{Gwari}, a \ili{Nupoid} language of Nigeria \citep{Hyman1970}:

\ea
\label{ex:hyman:33}
\ea (present habitual)\\
\gll wo  si  shnamá  \\   
\textsc{3sg} buy   yam \\
\glt ‘s/he buys a yam’ \\
\ex (present progressive)\\
\gll wo  si  shnamá lo\\   
\textsc{3sg} buy   yam  go \\
\glt ‘s/he is buying a yam’\\
\ex (present perfect)\\
\gll wó  lá  shnamá  si \\   
\textsc{3sg} take   yam   buy \\
\glt ‘s/he has bought a yam’ \\
\z 
\z 

However, I don’t think this is why SVCs develop. Rather, they originate as offering something different from the constructions with which they compete—and may ultimately replace. Much of the discussion concerned with defining SVCs has centered around how SVCs represent\- a single “event” (see \citealt{BohnemeyerEtAl2007}, \citealt{Bisang2009} and references cited therein). However, speech communities differ in how much detail of an event they customarily express. Thus consider the function of ‘take’ as a “custody transfer” verb in \ili{Mungbam} [Yemne-Kimbi] \citep{Lovegren2013}:

\ea
\label{ex:34}
\ea 
\gll m\=ə        m{\bassau} \\  
take.\textsc{irr} drink.\textsc{irr} \\
\glt ‘take and drink!’ \\
 {} [cup is within reach and at the level of the listener’s hands, in front of him]
\ex 
\gll m\=ə             j\'ə          à          m{\bassau} \\
take.\textsc{irr} ascend.\textsc{irr} \textsc{2sg.top} drink.\textsc{irr} \\
\glt ‘take and drink!’\\
 {} [cup is on the floor and has to be “ascended”]
\ex 
\gll m{\bassau}  \\
drink.\textsc{irr}\\
\glt ‘drink!’
\z 
\z 


As Lovegren puts it:
\begin{quotation}
In an event description of this type, the absence of a custody transfer coverb usually indicates that no custody transfer took place (because the theme was already in the agent's custody at the outset of the event, because the action was performed without the agent taking custody, because the theme ceased to exist at the end of the event, etc.), and not that the custody transfer event is left unspecified. The only situation where a simple \isi{imperative} \textit{m{\bassau}}  ‘drink!’ is felicitous is a case where the addressee is already holding a drinking cup. \citep[222]{Lovegren2013}
\end{quotation}

This raises the question of whether there could be comparable \isi{distinctions} in expressing multiple arguments, e.g. benefactives and instruments in the following situations, all representing a single event:

\ea
\label{ex:hyman:35}
\ea 
he cooked rice for child  [the rice is still in the pot]\\
he cooked rice give child  [the rice is in the child’s possession]
\ex 
he cut meat with knife  [the knife was in his hand prior to the cutting] \\
he took knife cut meat  [the knife was not in his hand prior to the cutting]
\z 
\z 

\noindent
A quite logical subsequent step would be for the SVCs in \REF{ex:hyman:35} to become the obligatory structure for expressing benefactives and instruments. Thus, in addition to \citegen{Foley1985} demonstration that valence marking SVCs develop last, languages that have developed \isi{benefactive}, recipient and instrumental SVCs may be at different stages: those like Fe’fe’ which have alternate structures are “younger” \isi{serial verb} languages than those like \ili{Mundani} which lack prepositional alternatives.\footnote{This would of course suggest that more westerly Benue-Congo and Kwa languages which only have SVCs have had their serial verbs much longer.} It is however likely that \ili{Bantoid} developed its SVCs fairly recently. As I pointed out in earlier work \citep[139--141]{Hyman1975}, the type of SVCs surveyed above are an areal phenomenon in West Africa. However, the \ili{Bantoid} distribution suggests there are micro-areas, since within the area surveyed, valence-marking SVCs are restricted to Eastern Grassfields \ili{Bantu} and \ili{Momo} languages. Such discontinuities probably hold in other parts of the continent as well.

To conclude, I would like to draw the perhaps obvious moral that some languages care about certain things more than others. That some languages such as \ili{Mungbam} care more about expressing the individual components of an action than English is not a new observation. Consider in this connection what \citet[87]{Pawley1993} notes about \ili{Kalam}, a language of New Guinea:
“\ili{Kalam} speakers are markedly more analytic and explicit than speakers of European languages in their reporting of the action components of events” \citep[87]{Pawley1993}.
\ili{Kalam} speakers thus say “food consume” for ‘eat’ and “water consume” for ‘drink’ (p.107) and have such elaborate SVC constructions as the following, which \citeauthor{Pawley1993} translates with one English verb (p.88):\footnote{Thanks to \citet{Woodbury2015} for bringing \citet{Pawley1993} to my attention. An example closer to home might be the expression of motion events in  “satellite-framed” \ili{Germanic} languages which encode more about manner than “verb-framed” \ili{Romance} languages (\citealt{Talmy1991}, \citealt{Slobin2003}).}


\ea
\label{ex:hyman:36}
\gll pk     wyk  d   ap  tan     d    ap   yap   g- \\
     strike rub hold come ascend hold come descend do\\
\glt  ‘to massage’
\z 

\noindent
It is clear that different speech communities adopt different conventions for expressing similar events. While English has the compact verb “fetch”, other languages require a tripartite SVC “go take come”. Once a speech community starts to move in such an analytic direction the “drift” can on a life of its own. I would like to suggest a change in conversational conventions is not only responsible for the development of SVCs, but also for their areal diffusion: communities in contact borrow the speech styles of others, and thereby their grammar.

\begin{sidewaystable}[H]
	\captionsetup{width=.8\textwidth}
	\caption{Benefactive, dative \& instrumental structures in Cameroonian Bantoid\label{tab:hyman:6}}
	\fittable{
		\begin{tabular}{lllllllllllll}
			\lsptoprule
			&  &  & \multicolumn{3}{l}{ {BENEFACTIVE}} & \multicolumn{4}{l}{ {RECIPIENT}} & \multicolumn{2}{l}{ {INSTRUMENT}} & \\
			& {{Language}} & {Group} & V Y X & X for Y & V X give Y & V Y X & X to Y & Y with X & V X give Y & V X with Y & take Y V X & {Source}\\
			\midrule
			1 & \ili{Mbembe} & {Jukunoid} & + & ké & + (1) & + & ké & wō & + (1) & wō & − & \citet{Richter2015}\\
			\midrule
			2 & \ili{Mungbam} & {Yemne-Kimbi} & − ? & á ... n\'{ə} & − ? & − ? & á ... n\'{ə} & b\={ɛ}, b\={ə} (7) & − ? & b\={ɛ}, b\={ə} & − & \citet{Lovegren2013}\\
			
			3 & Mundabli & {Yemne−Kimbi} & − & \H{i} & − & − & \H{i} ... lā & ā & − & ā & − & Rebecca Voll\\
			
			4 & Koshin & {Yemne-Kimbi} &  & -l\'{ə} &  &  & -l\'{ə} & b\'{ə} &  & b\'{ə} &  & \citet{Ousmanou2014}\\
			\midrule
			5 & \ili{Noni} & {Beboid} & + & \={ɛ}, -lé (8) & − & − & \={ɛ} & b\'{ɔ} & − & b\'{ɔ} & − & \citet{Hyman1981}\\
			\midrule
			6 & \ili{Mfumte} & {EG-North} & + & − & + (1) & + & w\'{ə} & − & + & w\'{ə} & − (2) & Greg McLean\\
			
			7 & Yamba & {EG-North} &  &  &  & + &  &  &  & b\'{ə} &  & \citet{Nassuna2001}\\
			
			8 & \ili{Limbum} & {EG-North} &  & nì &  &  & nì &  &  & nì &  & \citet{Fransen1995}\\
			
			9 & \ili{Bamun} & {EG-Nun} &  & n\`{ə} & + n\`{ə} (3) &  & n\`{ə} &  & + (3) & n\`{ə} & + (4) & Abdoulaye Nchare\\
			
			10 & \ili{Medumba} & {EG-Bamileke} & − (5) & − & − & + (5) & − & − & − & búù & − (2) & Ariane Ngabeu\\
			
			11 & Fe’fe’ & {EG-Bamileke} & + & mbú & + & + & mbú & − & + mbú & m\`ɑ & + & \citet{Hyman1971}\\
			
			12 & Ngomba & {EG-Bamileke} &  &  & + mb\v{ɔ} (3) &  & mb\v{ɔ} &  &  & n\'{ɛ} & + & Satre (2004, 2010)\\
			
			13 & Bambalang & {EG-Bamileke} &  &  &  &  & nì &  &  & nì &  & \citet{Wright2009}\\
			
			14 & Mankon & {EG-Ngemba} & − ? & n\`{ɨ} & + & − ? & n\`{ɨ}, á \`{m}bó & − ? & − & n\`{ɨ} & − ? & \citet{Leroy2007}\\
			\midrule
			15 & \ili{Mundani} & {Momo} &  &  & + abua (3) &  & abua, á & − & − & − & + & Elizabeth Magba\\
			\midrule
			16 & Babungo & {Ring-South} & − & t\'{ɨ} & − ? & − & t\'{ɨ} & n\`{ə} & − ? & n\`{ə} & − ? & \citet{Schaub1985}\\
			
			17 & \ili{Kejom} & {Ring-Central} & − & à & − & − & à & n\`{ə} & − & n\`{ə} & − & Pius Ajumbu\\
			
			18 & Kom & {Ring-Central} & − & s\={ə} & − & − & s\={ə} & n\`ə & − & n\`ə & − & Blasius Chiatoh\\
			
			19 & \ili{Aghem} & {Ring-West} & − & â & − & − & â & − & − & á(n) (6) & − & \citet{Hyman1979}\\
			
			20 & \ili{Isu} & {Ring-West} & − & áw\`ɔ & − & − & â & n\`ə & − & n\`ə & − & Roland Kießling\\
			\midrule
			21 & Esimbi & {Tivoid?} & − & \=ɔh\=ə kV & − & + & \=ɔh\=ə kV & − & − & ót\=ə & − & Brad Koenig\\
			
			22 & Tiv & {Tivoid} &  & shá & + & + &  &  &  & shá, á &  & \citet{Abraham1940}\\
			\midrule
			23 & Kenyang & {Mamfe} &  & ǹtá, t\`ɔk\'ɔ & − ? & + & ǹtá &  &  & n\`ɛ &  & Tanyi Eyongetah\\
			\midrule
			24 & \ili{Ejagham} & {Ekoid} & + & \`{m}bâ & + & + & \`{m}bâ & − & + & nà & + & John Watters\\
			\midrule
			25 & \ili{Akoose} & {\ili{Bantu} A10} & + (9) & áy\=əlè & − & + & wê & − & − & ne (9) & − & \citet{Hedinger2008}\\
			\midrule
			26 & Tikar & {Tikar} & + &  &  & + & l\`ɛ &  &  & l\`ɛ &  & \citet{Stanley1991}\\
			\midrule
			27 & \ili{Vute} & {Mambiloid} &  &  &  & +(10) &  &  &  &  &  & \citet{Thwing2006}\\
			\midrule
			&  & { totals:} & { 7} & { 16} & { 6} & { 11} & { 22} & { 8} & { 5} & { 24} & { 5} & \\
			\lspbottomrule
		\end{tabular}
	}
\end{sidewaystable} 

\subsubsection*{Notes on \tabref{tab:hyman:6}}
\begin{small}
In \tabref{tab:hyman:6}, ``+'' means the language has the construction (which can be general or limited to certain verbs); ``−'' means it doesn’t have it; blank = no info; ``EG'' = Eastern Grassfields

\begin{enumerate}[label=(\arabic*)]
	\item The \ili{Mfumte} and \ili{Mbembe} structure is \textit{V give Y X} + resumptive ‘with’; 
	\item The \ili{Mfumte} and \ili{Medumba} structure is \textit{take X cut Y with(it)}, two events. 
	\item The \ili{Bamun}, Ngomba and \ili{Mundani} structure is \textit{V X give to Y}; 
	\item The \ili{Bamun} structure is \textit{take X cut Y with(it)} = one event. 
	\item The \ili{Medumba} order is V X Y (the Y is from a PP, X, Y pronouns are distinct). 
	\item \ili{Aghem} \textit{á(n)} is the general \isi{locative} \isi{preposition}, used also with instruments (but not comitatives, which use \textit{à}); 
	\item \ili{Mungbam} Y with X also used for BEN. 
	\item In \ili{Noni}, \=ɛ  means ‘to s.o.’ or ‘for s.o.’s benefit’, while the \isi{locative} suffix \textit{-lé} means ‘for s.o.’ (in s.o.’s stead). 
	\item \ili{Akoose} has productive verb extensions: \isi{applicative} \textit{-e} producing V-e Y X and an \textit{-ɛn} instrumental verb extension producing V-ɛn X Y (Y = the instrument NP). 
	\item \ili{Vute} has an \isi{applicative extension} \textit{-ná} from the verb ‘to give’.
\end{enumerate}
\end{small}

\section*{Abbreviations}\largerpage
\begin{multicols}{2}
\begin{tabbing}
\textsc{anticaus}\hspace{.25em} \= {serial verb construction}\kill
\textsc{asp} \> aspect            \\
\textsc{anticaus} \> anticausative \\
\textsc{appl} \> {applicative}       \\ 
\textsc{caus} \> {causative}         \\ 
\textsc{cb} \> Canonical \ili{Bantu}     \\ 
\textsc{cfg} \> centrifugal        \\ 
\textsc{cns} \> consecutive        \\ 
\textsc{foc} \> focus              \\ 
\textsc{fut} \> future             \\ 
\textsc{gb} \> Grassfields \ili{Bantu}   \\ 
\textsc{infl} \> inflection       \\
\textsc{inst} \> instrumental     \\
\textsc{ipf} \> {imperfective}      \\
\textsc{irr} \> irrealis          \\
\textsc{np} \> noun phrase        \\
\textsc{om} \> object marker      \\
\textsc{pass} \> passive          \\
\textsc{pb} \> Proto-\ili{Bantu}        \\
\textsc{perf} \> perfect(ive)     \\
\textsc{pl} \> plural             \\
\textsc{pres} \> present          \\
\textsc{pst} \> past\\
\textsc{rec} \> reciprocal\\
\textsc{refl} \> reflexive\\
\textsc{rel} \> relative\\
\textsc{rev} \> reversive\\
\textsc{sbjv} \> {subjunctive}\\
\textsc{sg} \> singular\\
\textsc{sm} \> subject marker\\
\textsc{svc} \> {serial verb} construction\\
\textsc{tns} \> {tense}\\
\textsc{top} \> topic marker
\end{tabbing} 
\end{multicols} 

{\sloppy
\printbibliography[heading=subbibliography,notkeyword=this]
}
\end{document}