\documentclass[output=paper]{langsci/langscibook} 
\title{Back again to the future: How to account for directionality in grammatical change} 
\shorttitlerunninghead{How to account for directionality in grammatical change}
\author{%
 Bernd Heine\affiliation{University of Cologne}\and 
Tania Kuteva\affiliation{University of Düsseldorf}\lastand 
 Heiko Narrog\affiliation{Tohoku University}
}
\ChapterDOI{10.5281/zenodo.823234} %will be filled in at production
 

\abstract{Grammaticalization is commonly understood as a regular and essentially directional process. This generalization appears to be agreed upon in some form or other across many different schools of linguistics, even if it has not gone unchallenged. But there are different views on \emph{what} exactly is regular. Taking the development from movement-based verbs to future tenses as an example, the present paper argues that neither contextual features nor inferential mechanisms, analogy, or constructional form seem to provide a sufficient basis for explaining the evolution of grammatical categories. The paper is based on the one hand on findings made in \ili{⁠ǃ⁠Xun}, a Southwest African language of the Kx’a family, formerly classified as “Northern Khoisan”, and on the other hand on a comparison of this language with observations made in the Germanic languages English, Dutch, and Swedish.
}

\maketitle
\begin{document}

% \textbf{Keywords:} context-induced reinterpretation, \isi{de-andative future}, grammaticalization, \ili{Khoisan}, unidirectionality, ⁠ǃ⁠Xun

\section{Introduction}\label{sec:heine:1}
Grammaticalization is widely defined as a regular and directional process. This generalization appears to be agreed upon in some form or other across many different schools of linguistics (but see also, e.g., \citealt{Newmeyer1998}, \citealt{Norde2009}, and the contributions in \textit{Language Sciences} 23), and for many it is unidirectionality that what grammaticalization is about.



There are, however, different views on \textit{what} exactly is regular. Taking the grammaticalization from movement-based verbs to future tenses as an example, the present paper will argue that neither contextual features nor \isi{inferential} mechanisms, analogy, or constructional form seem to provide a sufficient basis for explaining \isi{directionality} in the evolution of grammatical categories. The paper is based on the one hand on findings made in \ili{⁠ǃ⁠Xun}, a Southwest African language of the Kx’a family, formerly classified as “Northern \ili{Khoisan}” \citep{Heine2010}, and on the other hand on a comparison of this language with observations made in the \ili{Germanic} languages English, Dutch, and \ili{Swedish}.



The paper is organized as follows. \sectref{sec:heine:2} deals with the grammaticalization of a range of \isi{future tense} categories in the ``\ili{Khoisan}'' language \ili{⁠ǃ⁠Xun}. In \sectref{sec:heine:3}, the observations made in \ili{⁠ǃ⁠Xun} are related to findings made on the reconstruction of similar future tenses in three \ili{Germanic} languages. The implications of this comparison are discussed in \sectref{sec:heine:4}, and some conclusions are drawn in \sectref{sec:heine:5}.



There is at present a plethora of definitions of grammaticalization. For the purposes of this paper, we will define it as the development from lexical to grammatical forms and from grammatical to even more grammatical forms. And since the development of grammatical forms is not independent of the constructions to which they belong, the study of grammaticalization is also concerned with constructions and with even larger discourse segments \citep[2]{Heine2002}. In accordance with this definition, grammatical developments that do not conform to the definition, such as cases of \isi{degrammaticalization}, degrammation, desinflectionalization, or debonding (\citealt{Norde2009,Norde2014}), are not strictly within the scope of \isi{grammaticalization theory} (see also \citealt[330]{Ramat2015}).


\section{Future tenses in ⁠ǃ⁠Xun dialects}\label{sec:heine:2} 
\subsection{Introduction}


The \ili{⁠ǃ⁠Xun} language, also called \ili{Ju}, is a traditional hunter-gatherer language of southwestern Africa. The language, classified by \citet{Greenberg1963} as forming the Northern branch of the “\ili{Khoisan}” family, has recently been re-classified as forming one of the two branches of Kx'a  \citep{Heine2010}, the other branch of this isolate consisting of the ǂ’Amkoe language of Southern Botswana, consisting of the varieties ǂHoan, N⁠ǃ⁠aqriaxe and \ili{Sasi} \citep{Güldemann2014}.


  \ili{⁠ǃ⁠Xun} is spoken by traditional hunter-gatherers in Namibia, Angola, and Botswana \citep{HeineKönigForthc}. It is a highly context-dependent language, showing fairly substantial analytic-isolating morphology; there is only a small pool of items having exclusively grammatical functions \citep{HeineKönig2005}. Typological characteristics include the presence of a noun class system with four classes, distinguished in pronominal agreement but not on the noun, and contiguous \isi{serial verb} constructions.   The basic \isi{word order} is SVO, though there is a minor SOV order, and a modifier-head construction in nominal possession. Sentences in two of its eleven dialects (E3 and W2), though not in others, are divided into two information units separated by a topic marker, where the topical constituent precedes and the non-topical one follows the marker. Phonological features include four click types and four distinct tone levels. The language is divided into eleven dialects, listed in \tabref{tab:heine:1}.


\begin{table}

\begin{tabularx}{\textwidth}{lXl}
\lsptoprule

Branch & Cluster & Dialect (reference form)\\
\midrule
1 Northwestern 

(NW-\ili{⁠ǃ⁠Xun}) & 1.1 Northern & N1\\
&  & N2\\ 
  & 1.2 Western & W1\\
  &  & W2\\
  &  & W3\\
  & 1.3 Kavango & K\\
\tablevspace
2 Central 

(C-\ili{⁠ǃ⁠Xun}) & 2.1 Gaub & C1\\
& 2.2 Neitsas & C2\\
\tablevspace
3 Southeastern 

(SE-\ili{⁠ǃ⁠Xun}) & 3.1 \ili{Ju} ǀ 'hoan & E1\\
& 3.2 Dikundu & E2\\
  & 3.3 ǂx'āō-ǁ'àèn & E3\\
 
\lspbottomrule
\end{tabularx}

\caption{A classification of ⁠ǃ⁠Xun dialects}
\label{tab:heine:1}
\end{table}

In his grammar of E1, the best documented \ili{⁠ǃ⁠Xun} dialect, \citet[25]{Dickens2005} notes: ``In \ili{Ju} ǀ'hoan, the circumstances in which a sentence is spoken often determine its \isi{tense}, and the verb itself, unlike its English equivalent, is never inflected for time.'' The only forms that he finds in the dialect to express \isi{tense} or aspect are the auxiliaries \textit{kȍh} (\textit{koh} in his writing) for \isi{past tense} and \textit{kú} for the \isi{imperfective}, and even these auxiliaries are used only optionally. This does not seem to apply to the other dialects (see \citealt{HeineKönigForthc}). As \tabref{tab:heine:2} shows, we found dedicated future tenses in eight of the eleven dialects, and only in two dialects there is none, namely in C2 and E1; for the K dialect there is no information. 


\begin{table}

\begin{tabularx}{\textwidth}{XXXXXXXXXX}
\lsptoprule

 \textsc{N}1& N2& W1& W2& W3& C1& C2& E1& E2& E3\\
\midrule
 ú, ò-tā& o, ò-tā& gǀè-ā& oā& ōā& o, oga& -& -& ú:& gǀè\\
\lspbottomrule
\end{tabularx}
\caption{Future tense markers in the ⁠ǃ⁠Xun dialects. No information exists on the K dialect of \tabref{tab:heine:1}. Listed in \tabref{tab:heine:2} are only \textit{dedicated} future tense categories, that is, categories whose primary function it is to express future tense.}
\label{tab:heine:2}
\end{table}

There are a number of similarities in the structure of the \isi{future tense} markers listed in \tabref{tab:heine:2}. First, the markers are throughout placed between the subject and the verb and, second, they are free rather than bound forms. But the markers also differ from one another, in that there are a number of different, or partly different forms. 


  There are no historical records of the language, but internal reconstruction work by \citet{HeineKönigForthc} suggests that no conventionalized \isi{future tense} form or construction can be traced back to Proto-\ili{⁠ǃ⁠Xun}, the hypothetical ancestor of the dialects. But there are two verbs, namely *{\textit{ú}} ‘go’ *{\textit{g}}{ǀ}{\textit{è}} ‘come’, which can. The only reasonable hypothesis is that these verbs were there earlier than the \isi{future tense} markers and that the former must have been involved in the historical development from the former to the latter. On this analysis, at least eight of the eleven dialects of the language appear in fact to have developed movement-based future tenses. Four dialects transparently used the verb *{\textit{ú}} ‘go’, developing what following \citet{Dahl2000} we call a \isi{de-andative future}.  Two other dialects apparently used the verb *{\textit{g}}ǀ{\textit{è}} ‘come’, creating a \isi{de-venitive future} in Dahl’s terminology; we will return to this below. 



However the constructions were not the same in the dialects. While all involved a sequence of two verbs, V\textsubscript{1} and V\textsubscript{2}, three different constructions can be distinguished on the basis of their morphosyntactic behavior, which we will refer to with the terms in \REF{ex:heine:1}. 



\ea%1
    \label{ex:heine:1} 
         Morphosyntactic types of future categories
\ea  Complement-based
\ex  Serializing
\ex  Particle-based
\z
\z



In \textit{complement-based} futures, the \isi{future marker} consists of a \isi{movement verb} (V\textsubscript{1}) meaning ‘go’ or ‘come’ plus the \isi{transitive} suffix \textit{-ā} (glossed ‘T’). This suffix, which turns, e.g., \isi{intransitive} verbs into \isi{transitive} ones, serves to add a complement to the valency of the verb.\footnote{The suffix, glossed ‘T’, is called the ``\isi{transitive} suffix'' by \citet[37--38]{Dickens2005}.} Such a complement can be a noun phrase (cf. \ref{ex:heine:3}), an adverbial phrase, or a complement verb, as in \REF{ex:heine:2}, and the second verb (V\textsubscript{2}) behaves structurally like a complement of V\textsubscript{1}.\footnote{Note that verbs in \ili{⁠ǃ⁠Xun} can typically be used in nominal slots, whereas nouns cannot be used as verbs.} Thus, the meaning of \REF{ex:heine:2} can structurally be rendered as ‘(S)he doesn't go to the coming’, where the \isi{movement verb} (V\textsubscript{1}), \textit{ú} ‘go’, is ambiguous in that it has \isi{future tense} as its second reading (unless indicated otherwise, the examples presented below are taken from \citealt{HeineKönigForthc}).    



\ea%2
    \label{ex:heine:2}
N1 dialect (Southeastern Angola)\\
\gll  yà               ǀōā     ú-       á    tcí.\\
	\textsc{n1}              \textsc{neg}    go/\textsc{fut}-  \textsc{t}   come     \\
\glt ‘He will not come.’
\z


In \textit{serializing} futures, the two verbs V\textsubscript{1} and V\textsubscript{2} are simply juxtaposed, (cf. \ref{ex:heine:3} and \ref{ex:heine:4}), as they are in the \isi{serial verb} construction of the language (\citealt{König2010}; cf. \citealt{Bisang1998}; \citeyear{Bisang2010})



\ea%3
    \label{ex:heine:3} 
E2 dialect (Northeastern Namibia)\\
\gll  \={m}i       ú:        gè-     à    Tàmzó.\\
  \textsc{1sg}     go/\textsc{fut}    stay-    \textsc{t}   Tamzo\\
\glt   ‘I am going to stay in Tamzo.’
\z


\ea%4
    \label{ex:heine:4}    
E3 dialect (Eastern Namibia, western Botswana)  \\
\gll   mí \={m}      (kú)      gǀè        kx'{ā}è     k{ā}.\\
  \textsc{1sg}    \textsc{top}    \textsc{prog}    come/\textsc{fut}   get      \textsc{n4}\\
\glt   ‘I'll have it.'
\z



In \textit{particle-based} futures, the \isi{future marker} consists of an element that is seemingly etymologically opaque. Examples are provided by the markers \textit{ò-tā} in \REF{ex:heine:5}, \textit{oga} in \REF{ex:heine:6},\footnote{The only data available on the C1 dialect stem from \citet{Vedder1910-1911}, who has no consistent tone markings and frequently confounds voiceless and voiced consonants. Thus, \textit{oga} presumably is phonetically [\textit{oka}]. Furthermore, he gives the meaning of \textit{g/yee} as ‘go', which is most likely a mistake and we have tentatively changed it to ‘come' on the basis of strong evidence from the other ten dialects.}  and \textit{óá} in \REF{ex:heine:7}. 



\ea%5
    \label{ex:heine:5} 
N1 dialect (Southern Angola)\\
\gll  \={m}       txòm,     à         ò-tā        ǁé {\ob}…{\cb}.\\
   \textsc{1sg}    uncle    \textsc{2sg}     \textsc{fut}    die.\textsc{sg}\\
\glt    ‘My uncle, you are going to die {\ob}…{\cb}.’ 
   (The tale of the lion and the jackal; \citealt{HeineKönigForthc})
\z




\ea%6
    \label{ex:heine:6} 
C1 dialect (North-central Namibia)\\
\gll    na     tí oga    gǀyee.\\
   \textsc{1sg}   \textsc{icpl}   \textsc{fut}     come\\
\glt    ‘I’ll come.’ \citep[20]{Vedder1910-1911}
\z



\ea%7
    \label{ex:heine:7} 
W2 dialect (Northern Namibia) \\
\gll   hȁ      má    nǁȁn   óá    gǀè.\\
  \textsc{n1}     \textsc{top}   later   \textsc{fut}  come\\
\glt   ‘He’ll come later.’ (Own data)
\z



On the basis of the dialect comparisons carried out by \citet{HeineKönigForthc} it is possible to reconstruct these three particles. First, note that there is general vowel lowering in the dialects whereby \textit{u} tends to be lowered to \textit{o} when there is a non-high vowel in the following syllable, hence \textit{u > o}. The particles \textit{ò-tā} and \textit{oga} can be reconstructed back, respectively, to the sequences *{\textit{ú tà}} and *{\textit{ú kà}}, both meaning ‘go and' (see \sectref{sec:heine:2.1}).
% \todo{section numbering seems to be off. What is 2.1, what is 2.2?} 
Second, the particle \textit{óá} can be reconstructed to the combination *{\textit{ú-}{ā}}, that is, ‘go' plus the \isi{transitive} suffix introducing a complement. \tabref{tab:heine:3} lists the various \isi{future tense} markers and their reconstructed forms.



\begin{table}


\begin{tabularx}{\textwidth}{XXXXXXXXXX}
\lsptoprule

N1& N2& W1& W2& W3& C1& C2& E1& E2& E3\\
\midrule
 ú, ò-tā& o, ò-tā& g{ǀ}è-ā& óá& ōā& o, oga& -& -& ú:& g{ǀ}è\\
 *ú, *ú tà& *ú, *ú tà& *g{ǀ}è-ā& *ú-ā& *ú-ā& *ú, *ú kà&  &  & *ú& *g{ǀ}è\\
\lspbottomrule
\end{tabularx}

\caption{Future tense markers in the ⁠ǃ⁠Xun dialects and corresponding reconstructed forms (cf. \citealt{HeineKönigForthc})}
\label{tab:heine:3}
\end{table}

\subsection{Accounting for the future tenses}\label{sec:heine:2.1}


We observed in \REF{ex:heine:1} that the \isi{future tense} constructions in the \ili{⁠ǃ⁠Xun} dialects appear to be built on three different constructions which we referred to, respectively, as the complement-based, the serializing, and the particle-based types. Now, there are three main constructions in the dialects used to connect two verbs or verb phrases, illustrated in \REF{ex:heine:8} with examples from the W2 dialect. In the complementation construction of \REF{ex:heine:8a}, V\textsubscript{1} is the main verb and V\textsubscript{2} is introduced as its complement. If V\textsubscript{1} is an \isi{intransitive verb}, as \textit{ú} ‘go' in \REF{ex:heine:8a} is, it takes the \isi{transitive} suffix \textit{-ā}, otherwise there is no formal marking. In \isi{verb serialization} no formal marking is needed, as \REF{ex:heine:8b} shows: V\textsubscript{1} and V\textsubscript{2} are simply juxtaposed and any complement that V\textsubscript{1} may have follows V\textsubscript{2}. 



  Coordination, by contrast, uses either of the additive conjunctions *{\textit{tà}} (*{\textit{tè}} in the Southeastern dialects) and \textit{*kà} ‘and', as we saw already in \sectref{sec:heine:2.1}, cf. \REF{ex:heine:8c}. The functions of these conjunctions are not exactly the same: Whereas the former conjoins separate events, the latter typically conjoins events that are conceived as wholes \citep[320]{HeineKönigForthc}. 

 
\ea%8
    \label{ex:heine:8} 
W2 dialect (own data)\\
\ea{ \label{ex:heine:8a}
\gll  hȁ      má    kè      ú-    á ḿ ..         \\
  \textsc{3sg}    \textsc{top}    \textsc{past}    go-    T    eat\\}
  \jambox{[Complementation]}
\glt   `He went to eat.'

\ex{ \label{ex:heine:8b}
\gll  hȁ      má    kè      ú ḿ ..             \\
  \textsc{3sg}    \textsc{top}    \textsc{past}   go    eat\\
  }\jambox{[Verb serialization]}
\glt  `He ate while going.'    

\ex{ \label{ex:heine:8c}
\gll   hȁ      má    kè      ú    kā \H{m}.{\rmfnm}         \\
  \textsc{3sg}    \textsc{top}    \textsc{past}    go    and  eat\\
  }\jambox{[Coordination]}
\glt  `He went and ate.'
\z
\z

\footnotetext{Instead of \textit{kā} ‘and', a much more common coordinating conjunction is \textit{tà} and its equivalents in other dialects.}
% \todo{should the double acute be  \H{m} or \'ḿ?}


The three constructions illustrated in \REF{ex:heine:8} do not all express the same meaning, but are available to speakers as different options to connect verbs or verb phrases. And all the \isi{future tense} constructions discussed in \sectref{sec:heine:2.1} can be traced back to them. 


Thus, example \REF{ex:heine:2} above is suggestive of complementation, and so is example \REF{ex:heine:7} from W2, where the future \isi{tense marker} \textit{óá} can be reconstructed back to a combination of *\textit{ú} `go' plus the \isi{transitive} suffix *\textit{-ā}. \REF{ex:heine:3} and \REF{ex:heine:4}, on the other hand, are instances of the \isi{verb serialization} construction, consisting of two verbs following one another without any formal linkage. In fact, both are ambiguous between a serial lexical and a grammatical interpretation: Thus, \textit{ú: gè} in \REF{ex:heine:3} can mean either `go (and) stay' or `will stay' and, similarly, \textit{gǀè   kx'āè} in \REF{ex:heine:4} can be translated variously as `come (and) get' or `will get'. The collocations \textit{ò-tāǁé} `will die' in \REF{ex:heine:5} and \textit{oga gǀyee} `will come' in \REF{ex:heine:6}, by contrast, can be reconstructed back, respectively, to the coordination construction of Proto-\ili{⁠ǃ⁠Xun} (\textit{*ú tā ǁé} `go and die', \textit{*ú kā gǀè} `go and come', respectively).



To conclude, there appear to have been three different highly schematic constructions involving altogether six partially schematic constructions that developed in the same direction towards \isi{future tense} constructions, namely 
[*\textit{gǀè} + V], 
[*\textit{gǀè-ā} + V], 
[*\textit{ú} + V], 
[*\textit{ú-ā} + V],
[*\textit{ú kà} + V], and 
[*\textit{ú tà} + V] (see \tabref{tab:heine:3}). Note further that in some of the dialects (\textsc N1, N2 and C1) there are two different source constructions leading to the same target, namely a \isi{future tense} construction.



To be sure, these constructions could be argued to have involved a general schema [V\textsubscript{1} + V\textsubscript{2}], but their morphosyntax was different, both on a schematic and a more substantive level. The question then is: How is this situation in the \ili{⁠ǃ⁠Xun} dialects to be explained, that is, what was responsible for this diversity in source constructions? Shared genetic origin is unlikely to account for this situation, with one possible exception: The markers \textit{ò-tā} in \textsc N1 and N2 and \textit{óá} and \textit{ōā} in W2 and W3, respectively, may each be due to a shared ancestor within the respective dialect group. But overall, these future constructions cannot be traced back to one common construction in the proto-language.



It would seem that there is only one reasonable answer to this question, name\-ly with reference to the meaning of the source and the target constructions. What they all have in common is that there was a verb expressing \isi{deictic} movement and belonging to the basic vocabulary in the sense of \citet{Swadesh1952}, and that in present-day \ili{⁠ǃ⁠Xun} there is a construction whose main function it is to express \isi{future tense}. The result was, in the terminology of \citet{Dahl2000}, either a \textit{de-andative} or a \textit{de-venitive} future depending on whether the \isi{movement verb} was `go (to)' or `come (to)'.


  It goes without saying that the overall process is more complex. For example, the \isi{source construction} may also give rise to other target constructions, and \isi{future tense} may only be one of the functions expressed. But in accordance with the definition of grammaticalization used here (see \sectref{sec:heine:1}), our interest is exclusively with this one pathway of change, ignoring the wealth of possible alternative constructional histories.

  On this view, which is in accordance with the framework of \citet{HeineEtAl1991}, there is some fixed semantic relation between source concepts for `go' and `come' and the grammatical target concept of \isi{future tense} in specific contexts. What this seems to entail is the following hypothesis: 

\ea \label{ex:heine:9}
Compared to semantic features, other factors that are likely to be involved are of secondary import in the development from lexical to grammatical material. An explanation of this development must therefore be over and above meaning-based. 
\z


Note, however, that \ili{⁠ǃ⁠Xun} is a language for which no historical records are available, thus making detailed diachronic reconstruction impossible and a falsification of the hypothesis difficult. We will now test the hypothesis in \REF{ex:heine:9} with data from \ili{Germanic} movement-based future tenses, for which arguably the best descriptions are available. 

\section{Future tenses in English, Dutch and Swedish}\label{sec:heine:3}

The account presented in this section is by no means meant to do justice to the grammaticalization of the three \ili{Germanic} future tenses based on movement verbs; rather, our interest is restricted to testing the hypothesis in \REF{ex:heine:9}. The account is based on the collostructional, distinctive collexeme analysis by \citet{Hilpert2008}. Unlike what we observed in \ili{⁠ǃ⁠Xun}, the \isi{constructional format} to be found in all three languages is essentially the same (but see \sectref{sec:heine:4}), involving what we referred to in \sectref{sec:heine:2.1} as the complementation construction: The \isi{movement verb} (V\textsubscript{1}) of the \isi{source construction} is the main verb and its complement contains a non-\isi{finite verb} (V\textsubscript{2}), turning via grammaticalization into the new main verb; hence, the constructional change underlying all grammaticalizations to be discussed can be rendered as leading from \REF{ex:heine:10a} to \REF{ex:heine:10b}. 

\ea \label{ex:heine:10}
\ea \label{ex:heine:10a} {[main verb V\textsubscript{1}        -    non-finite complement verb V\textsubscript{2}]}
\ex \label{ex:heine:10b}  {[\isi{future tense} \isi{auxiliary}  -    main verb]}
\z
\z 

Following \citet{Hilpert2008}, our main concern is with the constructional context of the \isi{tense} categories.

\subsection{The de-andative English \emph{be going to}-future}\label{sec:heine:3.1}

The first example concerns the evolution of the English \textit{be going to}-future, a \isi{de-andative future} in the terminology of \citet{Dahl2000}. The grammaticalization of this evolution has been extensively studied (see \citealt{Hopper2003,Mair2004,Hilpert2008} and the references therein; see also \citealt{Disney2009}). It seems to be well established that the construction was fully grammaticalized in the Early Modern English period by the end of the 17th century or the mid 18th century, and that a drastic increase in its text frequency first occurred in the 19th and early 20th centuries. Note that according to Mair (\citeyear[129]{Mair2004}; \citeyear[244--245]{Mair2011}), the increase of frequency is the outcome, not the driving force of the \textit{be going to-}future.


In his corpus-based collostructional study, \citet[118--121]{Hilpert2008} analyzes the following three stages of this \isi{de-andative future}: 1710--1780 (let us call it period 1), 1780--1850 (period 2), and 1850--1920 (period 3). During period 1, the construction strongly harmonized with \isi{telic} and dynamic verbs, and all distinctive \isi{collexemes} select for animate, intentional subject referents.



   During period 2, it is still \isi{telic} and dynamic verbs that the construction harmonizes with, most elements being compatible with an intentional reading, \textit{be} and \textit{have} now are among the most frequently used complement verbs. However, there are now also inanimate subjects that exclude an intentional interpretation but rather signal imminent future events, like in \REF{ex:heine:11}:



\ea%11
    \label{ex:heine:11}
         English (between 1770 and 1820; \citealt[120]{Hilpert2008})\\
    \textit{In the true sleepy tone of a \ili{Scottish} matron when ten o’clock is going to strike.}
\z


During period 3, there appear to be hardly any lexical restrictions. The verb \textit{happen} now belongs to the ten most frequent complement verbs, and unintentional complement verbs are fully acceptable. \citet[121]{Hilpert2008} concludes that “the occurrence of spontaneous, non-intended events is only encoded by \textit{be going to} in later stages of its development”.   



\subsection{The de-andative Dutch \textit{gaan-}future}\label{sec:heine:3.2}



On the basis of the data available, \citet[113]{Hilpert2008} classifies the history of this \isi{de-andative future} into three periods of time: centuries 16--17, 18--19, and 20, let us refer to them as periods 1, 2, and 3, respectively. 



  During period 1, Hilpert found all distinctive \isi{collexemes} of this period to share an ``\isi{atelic} \isi{aspectual character}”. The \isi{collexemes} encode events involving intentional movement of an animate agent. The events expressed commonly involved literal and intentional motion, associated with \isi{atelic} situation types.



  During period 2, most of the distinctive \isi{collexemes} have the ``\isi{telic} aspectual contour of accomplishment verbs”. There are on the one hand also intentional actions of human agents, but on the other hand also unintended processes such as \textit{sterven} ‘die’. The constructional meaning “is now broadening to accommodate events that are not connected to the intentions of human agents”.



  In period 3, the new verbal complements (distinctive \isi{collexemes}) are again mostly \isi{atelic}. The \isi{future meaning} of \textit{gaan} is fully conventionalized, combining also with verbs denoting involuntary human activities. And now, \textit{gaan} can also combine with inanimate subjects, as in \REF{ex:heine:12}:



\ea%12
    \label{ex:heine:12} 

\langinfo{Dutch}{}{\citealt[117]{Hilpert2008}} \\
\gll Wat    gaat    er      dan  gebeuren,   Sander?\\
  what     goes    there    then  happen    Sander\\
\glt  ‘What is going to happen then, Sander?’
\z




\subsection{The de-venitive Swedish \emph{komma att} future}\label{sec:heine:3.3}



De-venitive futures concern source constructions involving ‘come (to)’ as the \isi{matrix verb}. The following is a sketch of the grammaticalization of the \ili{Swedish} \textit{komma att}-\isi{future construction} based on the collostructional, distinctive collexeme analysis by \citet[125--131]{Hilpert2008}. Hilpert distinguishes three diachronic stages in the development of the construction, we will refer to them as 
period 1 (centuries 16--18), 
period 2 (century 19), and 
period 3 (century 20).



In the earliest documented records of period 1, the most distinctive verbs describe non-\isi{agentive} human activities and involuntary reactions. Verbs, such as \textit{förakta} ‘despise’, 
\textit{sova} ‘sleep’, and 
\textit{rodna} ‘blush’, and 
\textit{höra} ‘hear’ describe activities carried out unintentionally, but have animate subject referents, e.g., 
\textit{sova} ‘sleep’, 
\textit{höra} ‘hear’ \citep[128]{Hilpert2008}.  



It is only in period 2 that typically intentional activities can felicitously combine with the \isi{matrix verb} \textit{komma att}, such verbs being, e.g., \textit{klara} ‘manage’ or \textit{skicka} ‘send’, and the frequency of animate subject referents increases, but in this period there are also examples of future events that are beyond the control of the subject referent, thus expressing predictions about future events.



In period 3, a common pattern consists in the use of \isi{atelic} and \isi{stative} verbs, and the \textit{komma att-}construction “can express a plain sense of prediction”, but also “timeless generic truths that are epistemic rather than \isi{modal}” \citep[130]{Hilpert2008}.  



\subsection{The futures compared}\label{sec:heine:3.4}



The following is not meant to be an evaluation of different linguistic models, nor does it aim at a comprehensive treatment of this subject (for which see \citealt{BörjarsVincent2011}); rather, it is restricted to the following questions:



\ea%13
    \label{ex:heine:13}
\ea Does the framework account for the regularities of change in the development of \isi{future tense} categories?
\ex Does the framework propose a reasonable explanation for unidirectionality?
\z
\z



Both English and Dutch have a \isi{de-andative future}, historically derived from an \isi{auxiliary construction} involving a verb for ‘go (to)’, but the evolution of the two futures was clearly different. \citet[122]{Hilpert2008} summarizes the differences thus: “Converse preferences for perfectivity, \isi{transitivity}, and agentivity can be shown to permeate their respective developments. A historical perspective on the shifting collocational preferences of the two constructions reveals that \textit{be going to} had a special affinity towards speech act verbs, while with \textit{gaan}, movement verbs had a special role”. Central to the development of English \textit{be going to} were in fact \isi{perfective} speech act verbs.  



In its early stages, Dutch \textit{gaan} commonly occurred with typically \isi{imperfective} movement verbs, and it expressed intentional movement. In later usage, the construction accommodates verbs without the meaning of movement and intentionality. This contrasting genesis is to quite some extent reflected in the present situation. English \textit{be going to} attracts verbs that are \isi{transitive}, punctual, and highly \isi{agentive} \citep[121--122]{Hilpert2008}. Dutch \textit{gaan}, by contrast, attracts verbal complements that are \isi{intransitive}, temporally extended, and non-\isi{agentive}; intention is not (i.e., no longer) a part of its constructional semantics.



\tabref{tab:heine:4} deals with some lines of semantic development in the movement (‘go’-) verbs, while \tabref{tab:heine:5} summarizes the corresponding developments in the verbal complements of the two future tenses.\footnote{The information on the \ili{Swedish} \textit{komma att}-construction is incomplete and therefore not listed in these tables.} As these data suggest, there is no difference in the former but dramatic differences in the latter developments; we will return to this issue below.



\begin{table}


\begin{tabularx}{\textwidth}{Qll} 
\lsptoprule
& English \textit{be going to} & Dutch \textit{gaan}\\
\midrule
Early usage & Movement, intention & Movement, intention\\
Present usage, earlier phase & − Movement, +/− Intention & − Movement, +/− Intention\\
Present usage, later phase & − Intention & − Intention\\
\lspbottomrule
\end{tabularx}
\caption{Major semantic developments of the matrix (motion) verbs in two de-andative future tenses of Germanic languages (based on \citealt[116--123]{Hilpert2008})}
\label{tab:heine:4}
\end{table}


\begin{table}
\caption{Major semantic developments of the verbal complements in two de-andative future tenses of Germanic languages (based on \citealt[116--123]{Hilpert2008})}
\label{tab:heine:5}
\begin{tabularx}{\textwidth}{QQQ} 
\lsptoprule
& English \textit{be going to} & Dutch \textit{gaan}\\
\midrule
Early usage & Common with \isi{perfective} speech act verbs & Common with \isi{imperfective} movement verbs\\
Present usage, earlier phase & \isi{transitive}, punctual, and highly \isi{agentive} verbs & \isi{intransitive}, temporally extended, and non-\isi{agentive} verbs\\
\lspbottomrule
\end{tabularx}
\end{table}

\section{What is directional in the evolution of future tenses?}\label{sec:heine:4}


That grammaticalization is essentially (though not entirely) unidirectional, or that there is asymmetry between what is and what is not directional (\citealt{BörjarsVincent2011}), is a generalization that appears to be agreed upon in some form or other across different schools of linguistics (but see also e.g. \citealt{Newmeyer1998,Norde2009}, and the contributions in \textit{Language Sciences} 23), and for many, it is unidirectionality that grammaticalization is about.



  The evolution of de-andative and de-venitive futures has been described as one that is in accordance with the unidirectionality hypothesis. No case has so far been reported where a \isi{future tense} gave rise to a lexical verb meaning `go' or `come' while the opposite development is well documented ever since it was first discussed in detail by \citeauthor{Bybee2011} and associates (\citealt{BybeeEtAl1991}; \citeyear{BybeeEtAl1994}). But in the constructional history of such categories there are many linguistic, pragmatic, and sociolinguistic factors involved. The question then is: What is it in this history that is in fact directional?



  In \sectref{sec:heine:4.1} we will look at some factors that have been argued to show \isi{directionality} in grammaticalization but do not seem to be uncontroversial. In \sectref{sec:heine:4.2} then we will endeavor to isolate phenomena that, at least on the basis of the data discussed in Sections \sectref{sec:heine:2} and \sectref{sec:heine:3}, appear to go in one direction. In addition we will then look into the question of how to account for \isi{directionality}.



\subsection{What is not directional?}\label{sec:heine:4.1}
A\largerpage[2] number of factors and theoretical concepts have been invoked to account for the kinds of grammaticalizations discussed in   \sectref{sec:heine:3}, yet which on closer look raise some questions. We will now look at them in turn.



\subsubsection{Constructions}\label{sec:heine:4.1.1}



One of the theoretical concepts that has more recently been discussed in detail concerns the morphosyntactic format of the constructions involved in grammaticalization: Does the grammaticalization of future tenses require a specific \isi{constructional format} to take place?



  It would seem that the answer is in the negative. We noticed that in the dialects of \ili{⁠ǃ⁠Xun} it was not one type of construction that was responsible for the rise of future tenses but rather three. This is different in the case of the three \ili{Germanic} futures dealt with in \sectref{sec:heine:3}. But even here there appear to be striking differences between the languages examined, as \tabref{tab:heine:6} shows. Whereas the English \textit{be going to-} and the \ili{Swedish} \textit{komma att}-constructions introduce the verbal complement by means of a \isi{preposition}, there is no \isi{preposition} in the Dutch construction. And whereas English requires the verb to be constructed in the \isi{progressive aspect}, this is not a requirement in many other languages.



\begin{table}
\caption{The constructional form of source constructions for movement-based futures}
\label{tab:heine:6}

\begin{tabularx}{\textwidth}{lQQQ}
\lsptoprule
Language & `Go' as the \isi{matrix verb}& Use in \isi{progressive aspect}& Prepositional complement\\
\midrule
E. \textit{be going to} & +& +& +\\
D. \textit{gaan} & +& −& −\\
S. \textit{komma att} & −& −& +\\
\lspbottomrule
\end{tabularx}
\end{table}

Furthermore, in a number of other languages there are construction types that differ dramatically from the ones to be found in the languages examined here. For example, rather than an infinitival or other non-finite complement verb there is a \isi{finite verb} that serves as the complement of the \isi{movement verb}, as the following example from the \ili{Pipil} language of Guatemala shows \citep{Campbell1987}.



\ea%14
    \label{ex:14}
\langinfo{Pipil}{Aztecan, Uto-Aztecan}{\citealt[268]{Campbell1987}}\\ 
\gll ti- \textbf{yu}-  t   ti-  yawi-  t           ti-  paːxaːlua- t          neːpa   ka  kuːhtan.\\
     we-  go-  \textsc{pl} we-  go-   \textsc{pl}  we- walk-         \textsc{pl}  there   in  woods\\
\glt ‘We are going to go take a walk there in (the) woods.’
\z



To conclude, which morphosyntactic form a construction takes does not seem to be a factor that determines \isi{directionality}.



\subsubsection{Context}\label{sec:heine:4.1.2}



Another structural feature concerns the context frame: Does the “same” \isi{grammaticalization process} occurring in different languages involve the same kind of context?



  The example of the English and Dutch de-andative futures, or of the \ili{Swedish} de-venitive \textit{komma att-}future suggests that the answer is again in the negative: As we saw in \tabref{tab:heine:5} of \sectref{sec:heine:3.4}, the de-andative futures of English and Dutch drew on highly contrasting kinds of contexts (that is, complement verbs). Nevertheless, the end product was essentially the same, namely the schematic grammatical function ‘future’. Context change can even show a reversal of \isi{directionality}. For example, Dutch \textit{gaan} occurred at the first stage (period 1, 16--17th century) in the context of \isi{atelic} complement verbs, changing to \isi{telic} verbs in period 2 (18--19th century). In period 3 (20th century) finally, there was another move back to \isi{atelic} verbs \citep[116--117]{Hilpert2008}. 



  Thus, there does not appear to be clear evidence that \isi{directionality} is necessarily determined by the nature of the contextual features involved.\footnote{Andrej Malchukov (p.c.) rightly asks whether grammaticalization does not always involve context expansion; note that context extension constitutes one of the four parameters of grammaticalization in the framework of \citet[2]{Heine2002}. According to that framework, context extension is a necessary but not a sufficient condition for grammaticalization to take place; what is required in addition is at least also desemanticization. 
}



\subsubsection{Inferential mechanism}\label{sec:heine:4.1.3}



Much the same appears to apply to a number of semantic features associated with grammaticalization: The analysis of movement-based futures suggests that not all semantic changes in the development of movement-based futures are unidirectional.



One of them concerns the \isi{inferential} mechanism involved. According to one position surfacing implicitly or explicitly in the relevant literature -- one that can be traced back to \citet{BybeeEtAl1994}, it is the nature of the \isi{inferential} pathway leading from source to target concept that is crucial in grammaticalization, rather than a “macro-shift” from source to target. This pathway is said to be not only responsible for regularities in grammatical change but also for \isi{directionality} \citep[268]{BybeeEtAl1994}.



  Depending on which aspect of the pathway one has in mind, this position must remain controversial. Take the example of the English and Dutch de-andative futures that we presented in \sectref{sec:heine:3}. They are suggestive of an \isi{inferential} pathway leading from physical motion via intentional action to prediction (i.e., \isi{future tense}), as sketched in \REF{ex:heine:15a}. But this is not the only pathway that has been identified. There is an alternative pathway for de-venitive futures (involving verbs meaning ‘come (to)’) that does \textit{not} involve intention, leading from directed motion via the aspectual notion inchoative to prediction (\citealt[322]{Dahl2000}; \citealt[126]{Hilpert2008}). Thus, in addition to \REF{ex:heine:15a}, there is also \REF{ex:heine:15b}.\footnote{For volition-based future tenses, \citet[256]{BybeeEtAl1994} propose the following pathway: DESIRE > WILLINGNESS > INTENTION > PREDICTION.
} 



\ea%15
    \label{ex:heine:15}
  Inferential mechanisms in the development of motion-based \ili{Germanic} future tenses (\citealt[126]{Hilpert2008}, 183)
\ea \label{ex:heine:15a} Directed motion > intention > \isi{future tense}    (English, Dutch)
\ex \label{ex:heine:15b} Directed motion > inchoative > \isi{future tense}   (\ili{Swedish})
\z
\z


To conclude, there does not appear to be a regular \isi{inferential} mechanism leading from motion to prediction; rather, there may be different pathways involved. More specifically, intentionality does not appear to be crucial for movement-based future tenses to arise.



\subsubsection{Intentionality}\label{sec:heine:4.1.4}



More specifically, intentionality is a concept that has been invoked in a number of grammaticalization studies to account for regular grammatical change, most of all for changes leading to \isi{future tense} markers. For \citet[254]{BybeeEtAl1994}, “all futures go through a stage of functioning to express the intention, first the speaker, and later the agent of the main verb”, and this hypothesis was adopted by \citeauthor{Heine1995} (\citeyear{Heine1995}; see also \citealt{Ultan1978}).



It would seem, however, that this hypothesis has to be abandoned on the basis of observations such as the following from movement-based future tenses. These observations suggest not only that intentionality is not necessarily involved in movement-based futures, as we just saw. On the contrary, it can also be at variance with the unidirectionality hypothesis. In the earliest documented records of period 1 of the \ili{Swedish} \textit{komma att}-future, the most distinctive verbs describe involuntary reactions. Verbs such as \textit{förakta} ‘despise’, \textit{sova} ‘sleep’, \textit{rodna} ‘blush’, and \textit{höra} ‘hear’ select animate subject referents but describe activities carried out unintentionally. It is only at a second stage, in period 2, that typically intentional activities can felicitously combine with the \isi{matrix verb} \textit{komma att} (\citealt[128, 131]{Hilpert2008}). 



In the development of the de-andative futures of English and Dutch, by contrast, there was an opposite \isi{directionality} from intentional participants to loss of intentionality as a distinctive feature. Thus, in the English \textit{be going to-}future, all distinctive \isi{collexemes} selected animate, intentional subject referents in the earliest period 1 (1710--1780). During period 2 (1780--1850), there are now also inanimate subjects that exclude an intentional interpretation and in period 3 (1850--1920), unintentional complement verbs are now fully acceptable \citep[121]{Hilpert2008}.



Much the same development from intentional to unintentional events can be observed in the Dutch \textit{gaan-}future. During period 1 (16--17th centuries), the \isi{collexemes} encode events involving intentional movement of an animate agent: The events expressed commonly involve literal and intentional motion. During period 2 (18--19th centuries), there are on the one hand also intentional actions of human agents, but on the other hand also typically unintended processes such as \textit{sterven} ‘die’. The constructional meaning “is now broadening to accommodate events that are not connected to the intentions of human agents” \citep[116]{Hilpert2008}. In period 3 (20th century), \textit{gaan} can also combine with inanimate subjects, incapable of intentional actions. 



Intentionality is closely related to agentivity and, in fact, what has been said about the former also applies in some way or other to the latter. For example, it has been argued that in some pathways of grammaticalization, concepts for willful, \isi{agentive} participants are transferred to also denote inanimate concepts and a body of evidence has been presented for this hypothesis (\citealt{HeineEtAl1991,Heine1997}). As the data in \sectref{sec:heine:3} suggest, however, this not a requirement for the development of movement-based future tenses: In the earliest documented records of period 1 of the \ili{Swedish} \textit{komma att}-future, the most distinctive verbs describe non-\isi{agentive} human activities and involuntary reactions.



In sum, neither of the concepts intentionality and agentivity necessarily behaves directionally: There can be a change from intentional to unintentional activities (cf. the English and Dutch de-andative futures) but also from unintentional to intentional activities (cf. the \ili{Swedish} \isi{de-venitive future}). And changes do not necessarily lead from \isi{agentive} to non-\isi{agentive} subjects.



\subsubsection{Telicity}\label{sec:heine:4.1.5}



And much the same as intentionality concerns \isi{telicity} and the aspectual contours of verbs or events. The Dutch \textit{gaan}-future was associated with \isi{atelic} verbs in the 16th and 17th centuries: “all distinctive \isi{collexemes} of this period share an \isi{atelic} \isi{aspectual character}” \citep[116]{Hilpert2008}. This situation changed substantially in the 18th and 19th centuries, when most of the distinctive \isi{collexemes} had the \isi{telic} aspectual contour of accomplishment verbs. Finally, in the 20th century, the distinctive \isi{collexemes} are again mostly \isi{atelic} \citep[117]{Hilpert2008}. Thus, there appears to be a bidirectional development from \isi{atelic} to \isi{telic} on the one hand from \isi{telic} to \isi{atelic} verbal events on the other.



  Assuming that these are not idiosyncratic, exceptional examples, they show that not all semantic changes in grammaticalization are directional. 



\subsubsection{Analogy}\label{sec:heine:4.1.6}



In a recent study, \citet{Fischer2013} proposed an explanatory account for the English \textit{be going to}-\isi{future tense} in terms of analogy. She hypothesizes that it was similarity, or structural analogy on the morpho-syntactic level that played a central role in the development of this \isi{tense} construction.\footnote{Olga Fischer (p.c.) emphasizes that analogy in the sense of the term used by her includes in the same way the meaning, pragmatics, and the form of the construction concerned.} There was a change in \textit{going} from lexical verb to \isi{auxiliary} and the spread of infinitives from expressing concrete movement to also expressing mental activities, and next also to subjects that were inanimate or empty rather than animate and \isi{agentive}. The role played by analogy was that, once there is an \isi{auxiliary construction} that could behave like an [AUX - V] pattern it “will attract constructions (with different kinds of infinitives/subjects that are in use after other, (functionally) similar [AUX - V] patterns, such as \textit{shall/will} + \isi{infinitive}” \citep[522]{Fischer2013}.



\citet{Fischer2013} appears to favor a perspective according to which analogy is less about what speakers do than about what they do \textit{not} do. She argues that in analogy one “treats something like something else because one does not spot any difference, so it is a negative force rather than a positive one” \citep[519]{Fischer2013}. 



Analogy has been invoked in quite different frameworks dealing with grammaticalization, including generative ones (e.g., \citealt{Kiparsky2012}) and functional ones (e.g., \citealt[39--40]{Hopper2003}). For the latter, analogy effects (linguistic or sociolinguistic) rule spread rather than “rule change” - in other words, analogy presupposes “reanalysis” in grammaticalization. For example, the grammaticalization of the Old English noun \textit{had} ‘person, condition, rank’ into a \isi{derivative morpheme} representing an abstract state (e.g. \textit{biscophad} ‘bishophood’ is said to have involved two instances of reanalysis: (a) compounding followed by (b) semantic and morphological change). Thus, the development from nominal to \isi{derivative morpheme} was due to “reanalysis.” Analogy subsequently had the effect that the \isi{derivative morpheme} no longer required association with a word referring to a person but rather could be extended to new contexts, giving rise to Modern English expressions such as \textit{falsehood.} 



  Analogy is a ubiquitous mechanism, influencing all kinds of grammatical processes, including the present as well as others described in detail by \citet{Fischer2013}. But if taken as the main factor to account for the development then this raises questions. With reference to \isi{directionality}, this raises questions such as the following: 

\begin{enumerate}[label=(\alph*)]
\item Why is there a development from lexical to [AUX-V] pattern – why should there not be a development in the opposite direction? In other words, what accounts for the grammaticalization from lexical verb to \isi{auxiliary} (AUX)? 
\item Is there reason to rule out the possibility that analogy may not also work in the opposite direction, namely leading from the pattern [AUX-V] to another pattern [main verb-infinitival complement] – a pattern where English would have offered a plethora of models? 
\item  Finally, and most importantly, why should analogy be directional – would there be any more general motivation? \citet[521]{Fischer2013} proposes processing errors as playing an important role in analogical processes. The question then is why the same kind of grammaticalization from a pattern [main verb ‘go’ + non-finite verbal complement] to [AUX-V], to be observed in many languages across the globe, should have involved the same process, considering that not all of these languages disposed of a pattern such as English [{shall/will} + \isi{infinitive}].
\end{enumerate}


To conclude, analogy is an important factor in all kinds of grammatical change, but it does not seem to account for the kind of \isi{directionality} to be observed cross-linguistically in the grammaticalization from a lexical verb of goal-oriented physical motion to future \isi{tense marker}. Accordingly, rather than a unidirectional process, \citet[153]{Fischer2000} views grammaticalization as “a more or less accidental concurrence” that “may lead one way as well as another.” Note that her interest appears to be not with crosslinguistic typological generalizations but primarily with understanding the history of English and other \ili{Germanic} languages. Thus, analogy in the way proposed by Fischer can be an important trigger but does not seem to be responsible for the \isi{directionality} to be observed in grammaticalization (but see also \citealt{Kiparsky2012} for a different concept of analogy).



\subsubsection{Frequency}\label{sec:heine:4.1.7}



Frequency of use as an explanatory notion is invoked most of all in usage-based approaches (e.g., \citealt{Bybee2011}; \citealt[225]{Torres2011}). For Bybee and associates, high frequency of use of linguistic phenomena appears to be criterial for grammaticalization to happen (\citealt{Bybee2003}; \citeyear{Bybee2006}): “Thus as long as frequency is on the rise, changes will move in a consistent direction” \citep[77]{Bybee2011}.



While frequency is, in fact, an important factor, it would seem that more evidence is needed to establish that frequency by itself can immediately be causally responsible for the presence of new functional categories.\footnote{We are concerned here \textit{exclusively} with frequency in the rise of a new functional category. The situation is different in subsequent developments of such a category. Note further that a distinction must be made between frequency of the element that provides the source of grammaticalization and that of later uses of this element, as well as between type and token frequency (cf. \citealt[244]{Mair2011}).}  Furthermore, one wishes to know what accounts for increased frequency, that is, \textit{why} do interlocutors use certain linguistic expressions more frequently than others -- in other words, frequency may tell us little about why people use their languages the way they do.



  Is frequency really responsible for \isi{directionality} -- e.g., to the effect that the more frequently a linguistic expression is used the more it will be grammaticalized? It would seem that this question cannot be clearly answered in the affirmative. First, there is linguistic material that is used highly frequently but does not appear to be grammaticalized. This is suggested on the one hand by frequency counts of lexical items, some of which occur highly frequently in texts but may show little effects of grammaticalization. Second, that there is no one-to-one relationship between frequency and grammaticalization can be seen in developments where some grammatical element experiences a decrease in its frequency of use but no corresponding decrease in its grammaticalization. And third, there are some research findings suggesting that the contribution of frequency to grammaticalization is not entirely uncontroversial (\citealt{Hoffmann2004,Hoffmann2005,Brems2007,Mair2011};  \citealt[10]{Hilpert2013}). As we saw in \sectref{sec:heine:3.1}, the dramatic increase in text frequency that the English \textit{be going to-}construction experienced in the early 20th century is shown by \citet[129]{Mair2004} to be the outcome rather than the driving force of grammaticalization.



  On account of such observations one may hesitate to hold frequency of use responsible for \isi{directionality} in grammaticalization.  



\newpage The catalog of factors discussed above is far from exhaustive. What it suggests, however, is that many of the hypotheses that have been volunteered must be taken with care. The question then is what is it ultimately that makes grammaticalization an essentially unidirectional process? This is the subject of the next section. 



\subsection{ What is directional?}\label{sec:heine:4.2}



We saw in the preceding section that a number of the factors that characterize the history of the \isi{future tense} categories in the \ili{Germanic} languages surveyed do not seem to be directly responsible for the \isi{directionality} to be observed in the grammaticalization of these categories. Such changes are either not directional in that they may go in both directions of a chain of grammaticalization or else their contribution to the process is not entirely clear. 



  It would seem that there is essentially only one factor that can be identified both in the \ili{⁠ǃ⁠Xun} dialects and in the \ili{Germanic} futures, as proposed in our discussion of \sectref{sec:heine:3.4}, namely the shift from lexical (or less grammatical) to grammatical meaning, entailing a gradual transformation of lexical as grammatical morphosyntax. The latter process has received considerable scholarly attention (e.g., \citealt{Lehmann2015Thoughts,HeineEtAl1991,BybeeEtAl1994}, and subsequent works), being described as one of structural (morphosyntactic and morphophonological) reduction; we will return to this issue below.



  What all cases examined in this paper in fact share is that there appears to be a fixed semantic relation between source concepts for `go' and `come' and the grammatical target concept of \isi{future tense} in the languages concerned, in accordance with our hypothesis in \REF{ex:heine:9}. This relationship implies a ``macro-shift'' of the kind discussed in this paper. Such a shift can, but need not, take place in virtually any language, and it can be arrested at any point in history, that is, it may be, and not seldom is incomplete -- in other words, the \isi{grammaticalization process} need not take its full course. In the latter case there is only a weakly grammaticalized \isi{future tense}.  



  To be sure, in the case of the \ili{⁠ǃ⁠Xun} dialects there may also have been some kind of drift effect in the sense of \citet{Sapir1921} that contributed to the fact that in eight of the ten documented dialects a movement-based \isi{future tense} arose. But this does not account for the hundreds of other languages in Africa and elsewhere where a similar development took place. And, as far as the information available suggests, clearly the most perspicuous common denominator of all these developments is the source-target relationship between \isi{deictic} movement verbs in combination with another verb and the grammatical function of \isi{future tense}. As we saw in \sectref{sec:heine:2}, this combination need not be one of a \isi{matrix verb} and its complement, it can as well be one of coordination or \isi{verb serialization}, or even of clause subordination. 



  Considering that this development tends to require an extended period of time, possibly involving various intermediate stages and constructional changes, this generalization raises the question of what the underlying causal factors are that can be held responsible for this relationship. We have no clear answer to this question, which is in need of much further research. But there are a few suggestions made in works on this subject matter which may be of help in such work. According to \citet[109]{Hilpert2008} there is an implicature inherent in the meaning of directed movement whereby the content of the verbal complement of ‘go (to)’ and ‘come (to)’ implies a situation in time later than reference time, thereby enabling a “presupposition of a future event”. In a similar fashion, \citet[268]{BybeeEtAl1994} suggest that the temporal meaning that comes to dominate the semantics of the construction ``is already present as an inference from the spatial meaning. When one moves along a path toward a goal in space, one also moves in time.'' This is, for example, also in accordance with what is possibly one of the earliest uses of the English \textit{be going to-}future in \REF{ex:heine:16}, which provides a possible context for a future interpretation.



\ea%16
    \label{ex:heine:16} 
         English\\
  \textit{ther passed a theef byfore Alexandre that \textbf{was goying to be hanged} whiche saide …}\\
\glt  ‘a thief who was going to be hanged passed before Alexander and said … 
  (1477, Mubasshir ibn Fatik, Abu al-Wafa’; \textit{Dictes or sayengis of the philosophhres} [LION: EEBO]; from \citealt{Traugott2012})
\z


Note further that according to \citet[84]{TraugottDasher2002}, in the early stages of the English \textit{be going to}-future “the change is primarily abstraction (spatial > temporal)”, as in the following example:



\ea%16
    \label{ex:heine:17}
          \textit{Witwoud: Gad, I have forgot what \textbf{I was going to say}  to you.}\\ 
    (1699; \citealt[84]{TraugottDasher2002})
\z



The interpretation proposed here is in accordance with that described in detail in \citet{HeineEtAl1991} and \citet{Heine1997}, where the implicature or inference is captured in terms of a metaphorical transfer (SPACE > TIME) within the metonymic-metaphorical model proposed there \citep[70, 113]{HeineEtAl1991}.\footnote{We are grateful to Andrej Malchukov (p.c.) for having drawn our attention to this point.} 



  What this interpretation argues for is that grammaticalization processes such as the ones described in this paper are ultimately due to the cognitive-communi\-cative strategies that interlocutors recruit in order to create their discourse contributions. And one major strategy is to use concrete, referential and clearly delineated expressions to also convey more abstract, non-referential and/or less clearly delineated meanings. In doing so they constantly propose new discourse options, and some of these new options may be used regularly and give rise to new patterns of grammar. On this understanding there is not really ``coevolution of form and meaning'' \citep[4]{BybeeEtAl1994}; rather, the evolution of the former is caused by and, hence, is preceded in time by that of the latter. Accordingly, the \isi{directionality} to be observed in structural reduction is derivative of the semantic changes to be observed in grammaticalization processes of the kind examined in this paper.\footnote{To be sure, structural change can also be instrumental to inducing semantic change, as demonstrated, for example, in the work on \isi{degrammaticalization} (see especially \citealt{Norde2009}), but this does not normally appear to apply to the evolution from lexical to grammatical categories.} 



  The \ili{⁠ǃ⁠Xun} examples discussed in \sectref{sec:heine:2} illustrate this temporal asymmetry between form and meaning in the development of future tenses. As we saw in examples \sectref{sec:heine:2}--\sectref{sec:heine:4}, the future tenses in the \textsc N1, E2, and E3 dialects are ambiguous between the lexical meaning of a \isi{movement verb} and the grammatical meaning of \isi{future tense}. What this suggests is that there must have been a semantic shift from verbal to grammatical meaning and now both coexist in the dialects concerned. But this semantic shift does not appear to have been accompanied so far by corresponding structural (morphosyntactic and/or morphophonological) shift. Accordingly, the only reasonable conclusion is that there was semantic but so far no morphosyntactic change -- in other words, structural change lags behind semantic change \citep{Heinefc}.  



\section{Conclusions}\label{sec:heine:5}



Our starting point was the situation in the ``\ili{Khoisan}'' language \ili{⁠ǃ⁠Xun} of southwestern Africa, where speakers of a number of different dialects appear to have moved in the same direction in designing a \isi{future tense} category. In doing so, they appear to have drawn on a crosslinguistically common conceptual pathway whereby a verb for directed spatial movement belonging to the basic vocabulary in the sense of \citet{Swadesh1952} in combination with another verb over time gives rise to a grammatical category expressing prediction, that is, a \isi{future tense}. Thus, the paper was restricted to one specific pathway of grammaticalization, ignoring other pathways that have movement verbs as their source or \isi{future tense} as their target. Whether, or to what extent, the findings made can be generalized beyond this pathway is a question that is beyond the scope of the present paper. 



  According to the findings presented, it is neither the \isi{constructional format} nor the \isi{inferential} mechanisms or analogy that seem to provide a sufficient basis for explaining the ``macro-shift'' from lexical source to the grammatical target of a \isi{future tense} category. What appears to be involved most of all is some fixed asymmetric semantic relation between source concepts for `go (to)' and `come (to)' and the grammatical target concept of \isi{future tense}. The causal nature of this relation is in need of much further research, it is presumably shaped or influenced by discourse functions, e.g. by the fact that the source meaning is functionally useful ``in a discursively secondary role'' \citep[65]{Harder2011}. 



  Thus, the hypothesis in \REF{ex:heine:9}, proposed on the basis of observations made in the ``\ili{Khoisan}'' language \ili{⁠ǃ⁠Xun}, does not appear to be invalidated by the data examined in \sectref{sec:heine:3} on movement-based future tenses in \ili{Germanic} languages. 



  Much of what was discussed in the paper could have been phrased within the framework of Construction Grammar, that is, as an instance of constructional change (or constructionalization). A considerable part of work within this framework has in fact been devoted more recently to issues of grammaticalization (see, e.g., \citealt{Traugott2003,Noël2007,Trousdale2008,Hilpert2008,Hilpert2013,Hilpert2015,Bisang2010,DeSmet2010,GisbornePatten2011,VanBogaert2011,Trousdale2013,Hüning2014,TraugottTrousdale2014}). This work has brought about a wealth of information on the history of the constructions concerned, including the history of constructions that were the topic of this paper. 



  The main reason for not drawing on this framework here is that the goals of Construction Grammar and \isi{grammaticalization theory} are not the same and, hence, entail a different perspective of what grammatical change is about. The former is concerned with how constructions change, and most of all with what happens on the way from source to target construction. The latter, by contrast, is ultimately concerned with the following questions: What induces interlocutors in discourse across the world to draw on much the same lexical resources to create a new functional category for \isi{future tense}, and why is this semantic process essentially regular, e.g., why is it fairly unlikely that there will be a process in the opposite direction? To our knowledge, the only explanatory account that exists so far is one with reference to the cognitive-communicative strategies that speakers and hearers have when they design their discourse contributions \citep{HeineEtAl1991}.


\newpage
  Thus, \isi{grammaticalization theory} is concerned with the ``macro-shift'' from source to target meaning whereas the main concern of Construction Grammar is with the process leading from the former to the latter, that is, with the constructional history of the process. Accordingly, neither the perspective underlying these two frameworks nor the results obtained are the same. Nevertheless, both frameworks are needed for a comprehensive understanding of grammatical change. 



\section*{Acknowledgements}



We wish to thank Olga Fischer, Martin Hilpert, and Andrej Malchukov for valuable comments on an earlier version of the paper. Furthermore, the first-named author expresses his gratitude to Guangdong University of Foreign Studies and Haiping Long, and the University of Cape Town and Matthias Brenzinger for the academic hospitality received as a visiting professor, which gave him time and inspirations to work on this paper.



\section*{Abbreviations}
\begin{tabularx}{.55\textwidth}{@{}lQ@{}} 
1, 2, 3 & first, second, third person\\
 \textsc{fut} & {future tense}\\
 \textsc{icpl} & {incompletive} aspect\\
 \textsc{n1}, \textsc{n2}, \textsc{n3}, \textsc{n4} & noun class 1, 2, 3, or 4\\
 \textsc{past} & past tense marker\\
 \textsc{prog} & {progressive aspect}\\
\end{tabularx}
\begin{tabularx}{.45\textwidth}{@{}lQ@{}} 
 \textsc{sg} & singular\\
 \textsc{t} & {transitive} suffix (valency-increasing marker)\\
 \textsc{top} & topic marker\\
 \textsc{tr} & linker\\
  & \\
\end{tabularx}



{\sloppy
\printbibliography[heading=subbibliography,notkeyword=this]
}
\end{document}