\documentclass[output=paper]{langsci/langscibook} 
\title{On the grammaticalization of demonstratives in Hoocąk and other Siouan languages} 
\shorttitlerunninghead{On the grammaticalization of demonstratives in Hoocąk} 
\author{%
 Johannes Helmbrecht \affiliation{University of Regensburg}
}
\ChapterDOI{10.5281/zenodo.823238} %will be filled in at production

\abstract{%
% It is common knowledge in grammaticalization theory that demonstrative pronouns and adnominal demonstratives are the source of a number of grammatical forms: demonstrative pronouns often develop into anaphoric pronouns and finally to third person pronouns. Adnominal demonstratives often develop into definite articles, specificity markers, gender markers and/or noun markers. What is less well known is how demonstratives emerge historically in the first place. 
\footnotesize The present paper describes the grammaticalization of a paradigm of adnominal demonstratives in Hoocąk out of a set of three verbs\slash aux\-il\-ia\-ries of existence. This process has never been described for Hoocąk nor does the literature on grammaticalization mention this kind of grammaticalization path.
Hoocąk (also known as Winnebago) is a Siouan language still spoken in Wisconsin. Hoocąk has two paradigms of demonstrative pronouns. The first paradigm of demonstratives goes back to Proto-Siouan. The second paradigm is an innovation in Hoocąk. Two bound deictic forms =\textit{re} and =\textit{ga} are systematically combined with the three positional verbs \textit{nąk} `sit', \textit{ąk} `lie' and \textit{jee} `stand' to form a new paradigm of demonstratives. These new demonstratives are grammatically and semantically different from the first paradigm. First, they are always adnominal demonstratives determining the head noun, while the old paradigm can be used as both adnominally as well as pronominally demonstratives. Secondly, they appear only post-nomi\-nally, while the old paradigm is more variable, occurring both pre- and post-nominally. Thirdly, the new demonstratives classify the head noun as proximate or distal as well as according to its spatial position (neutral, horizontal, vertical), while the old demonstratives distinguish only proximal and distal and are used to refer anaphorically to aforementioned participants, whole propositions and episodes in a narration. Finally, the new paradigm of demonstratives can be used as relativizers and subordinators, which is not possible for the old paradigm of demonstratives. The positional verbs themselves, on the other hand, go back to Proto-Siouan. They are grammaticalized in Hoocąk (and other Siouan languages) as classificatory auxiliaries of being `be.sitting/be.lying/be.standing' and as continuative/progressive markers when combined with other verbs. The grammaticalization processes that are observed in Hoocąk are compared to those found in Siouan languages of other sub-branches of Siouan, in particular to the grammaticalization of classificatory definite articles in Omaha-Ponca (Dhegiha). It is shown that the positional verbs underwent a different grammaticalization path in this sub-branch of Siouan.}

\maketitle

\begin{document}
 




% \textbf{Keywords:} grammaticalization, \isi{demonstrative}s, \ili{Hoocąk}, \ili{Siouan}, posture verbs, \isi{positional} verbs
\section{Introduction}\label{sec:helmbrecht:1}
\subsection{Grammaticalization of demonstratives}\label{sec:helmbrecht:1.1}

Demonstratives are the starting point for a remarkable variety of different grammaticalization processes leading to quite different grammatical forms.\footnote{This paper is dedicated to the memory of Robert L. Rankin who passed away in February 2014. He was the leading scholar in comparative \ili{Siouan} linguistics and much of what we know about \ili{Siouan} languages today is based in one way or another on his research.}
Well-attested in many languages and language families is the development of \isi{demonstrative} pronouns into third person pronouns and finally into argument-indexing pronominal affixes. Another well-attested and often described \isi{grammaticalization process} is the development of adnominal \isi{demonstrative}s to definite articles, specificity markers, noun class/gender markers and finally noun markers. A summary of the grammaticalization paths for \isi{demonstrative}s described in the literature is given in \tabref{tab:helmbrecht:1}. 

\begin{table}
\caption{Demonstratives as sources for various grammaticalizations (summarized from \citealt[Chapter 4]{Diessel1999}, and  \citealt{Heine2002})}

\small
\begin{tabularx}{\textwidth}{p{2cm}@{ }l@{ }QQ}
\lsptoprule
Source(s) &  & Target(s) & Some references\\
\midrule
de\-mon\-stra\-tive pronouns & > & \textbf{3}\textbf{\textsuperscript{rd}}\textbf{ person} \textbf{\scshape pron} > \isi{clitic} \textsc{pron} > affix & \citet[353--360]{Givón1984}; \citet[39--42]{Lehmann1995[1982]}; \citet[112]{Heine2002}\\
& > & relative pronouns & \citet{Lehmann1984}\\
\tablevspace & > & complementizers & \citet[287]{HarrisCampbell1995}; \citet[106]{Heine2002}\\
\tablevspace & > & subordinators (adverbial clauses) &  \citet[114]{Heine2002}\\
\tablevspace & > & sentence connectives & \citet[125]{Diessel1999}; \citet[108]{Heine2002}\\
adnominal \isi{demonstrative}s & > & \textbf{definite articles} > specific\slash indefinite > noun class\slash gender markers & \citet{Greenberg1978}; \citet[38, 55]{Lehmann1995[1982]}; \citet[109]{Heine2002}; and many others\\
& > & relative pronouns & \citet[378--383]{Lehmann1984}; \citet[113]{Heine2002}\\
\tablevspace & > & linkers & \citet[172--188]{Himmelmann1997}\\
\tablevspace & > & boundary markers of postnominal relative clauses\slash\textbf{relative particles} & \citet[132]{Diessel1999}\\
\tablevspace & > & determinatives (\isi{demonstrative}s that function as the head of a relative clause) & \citet[217]{QuirkEtAl1972}\\
\tablevspace & > & \textbf{specific indefinite articles} & \citet{GundelEtAl1993}\\
adverbial \isi{demonstrative}s & > & \textbf{temporal adverbs} & \citet[139]{Diessel1999}\\
& > & directional \isi{preverbs} & \citet[97--104]{Lehmann1995[1982]}\\
iden\-ti\-fi\-ca\-tio\-nal \isi{demonstrative}s & > & \textbf{non-verbal copulas} > focus markers & \citet[147--148]{Diessel1999}; \citet[108, 111]{Heine2002}\\
& > & expletives & \citet[216--219]{Traugott1992}\\
\tablevspace
\lspbottomrule
\label{tab:helmbrecht:1}
\end{tabularx}
\end{table}


All targets in \tabref{tab:helmbrecht:1} that are marked \textbf{bold} are grammaticalizations that can be found in \ili{Siouan} languages and will be presented later at different places in the course of this paper. Where do – in turn - \isi{demonstrative}s come from? From which sources do \isi{demonstrative}s grammaticalize? Not much is known about this question. The following answers are given in the literature:

\begin{enumerate}
\item[i.]
According to \citet[154]{Diessel1999}, there is no evidence that \isi{deictic} roots, on which the \isi{demonstrative}s are based, are grammaticalized from lexical sources. Instead they belong to the basic vocabulary of every language and often show an iconic relationship between the phonetic shape and the meaning (with regard to distance relationships). Diessel (and others) claims that the exophoric usage of \isi{demonstrative}s is basic.

\item[ii.]
In \citet[37--38]{Lehmann1995[1982]} it is suggested that \isi{deictic} roots combine with categorial nouns in order to form new \isi{demonstrative}s (illustrated with examples from \ili{Japanese} \textit{ko-re} `this one', \textit{so-re} `that one', and \textit{a-re} `yonder one' \citep[38]{Lehmann1995[1982]}. One may also think of complex \isi{demonstrative} pronouns like the ones found in \ili{Korean}. \ili{Korean} has three \isi{deictic} particles that are used as determiners (cf. the paradigm in \REF{ex:helmbrecht:1}). If they are used as \isi{demonstrative} pronouns, they have to be combined with a defective categorial noun such as \textit{il} `thing' in example \REF{ex:helmbrecht:2}. 


\newpage
\ea\label{ex:helmbrecht:1}\let\eachwordone=\upshape
\ili{Korean} \citep[295]{Sohn1999}\\
\glll {i} = {person/thing near speaker}\\
{ku} =      {person/thing near hearer}\\
{ce} =      {person/thing away from speaker and hearer}\\
\z

\ea \label{ex:helmbrecht:2}
\ili{Korean} \citep[295]{Sohn1999}

\gll {\ob}\textbf{ce} \textbf{il-ul}{\cb}       nwu-ka       mak-keyss-ni.\\
{}[that thing-\textsc{acc}]   who-\textsc{nom}  block-will-\textsc{q}\\
\glt ‘Who would be able to block \textbf{that}?’ 
\z

\item[iii.]
A third answer can be found in \citet[172/294]{Heine2002}. The authors provide a few examples from \ili{Hausa}, Lingala, and \ili{Ngbaka} that show that adverbial \isi{demonstrative}s such as `here' and `there' may become \isi{proximal} and \isi{distal} \isi{demonstrative}s (`this', `that').
\end{enumerate}

\subsection{Goals of the paper}\label{sec:helmbrecht:1.2}

The goal of the present paper is to present an admittedly incomplete overview of the grammaticalization of \isi{demonstrative}s in \ili{Hoocąk} and other \ili{Siouan} languages. A major role in these historical developments is played by posture verbs denoting `sitting', `standing' and `lying'; Siouanists call them ``\isi{positional} verbs'' or just ``positionals''. More specifically, it will be shown that: 

\begin{enumerate}[label=\roman*.]
\item the Proto-\ili{Siouan} posture verbs became aspect-marking auxiliaries in \ili{Hoocąk} and many other \ili{Siouan} languages;
\item the aspect-marking auxiliaries (\isi{continuative aspect}) were combined with subordinating \isi{deictic} particles in \ili{Hoocąk} that grammaticalized to new adnominal \isi{demonstrative}s;
\item these ``new'' \isi{demonstrative}s preserved a noun classifying and aspect-mark\-ing function, if used to subordinate clauses;
\item the advent of these forms caused a shift in the usage of the old Proto-\ili{Siouan} \isi{demonstrative}s in \ili{Hoocąk}; and finally that
\item the same Proto-\ili{Siouan} \isi{positional} verbs underwent a different \isi{grammaticalization path} in \ili{Mandan};
\item the same Proto-\ili{Siouan} \isi{positional} verbs underwent a different \isi{grammaticalization path} in \ili{Omaha-Ponca} and other \ili{Dhegiha} languages; there they became classificatory articles probably without an intermediate step of being \isi{demonstrative}s, and from these classificatory articles new \isi{demonstrative}s were developed.
\end{enumerate}

\subsection{Hoocąk and the Siouan languages}\label{sec:helmbrecht:1.3}

\ili{Hoocąk} is a North American \ili{Indian} language of the \ili{Siouan} \isi{language family} still spoken at various places in Wisconsin. The \ili{Siouan} \isi{language family} consists of about 17 languages that were originally spoken in a large area covering most of the Great Plains expanding from the Southeast of the US to the Northwest into Southern Canada. The genetic sub-classification of the \ili{Siouan} languages is generally considered to be as summarized in \figref{tab:helmbrecht:2} (cf. \citealt{Rood1979}; \citealt[501]{Mithun1999}; \citealt{ParksRankin2001}).


% % \begin{table}
\begin{figure}\footnotesize
\caption{Genetic classification of the Siouan languages (cf. \citealt{Rood1979}; \citealt[501]{Mithun1999}; \citealt{ParksRankin2001})\label{tab:helmbrecht:2}}
\begin{forest} forked edges, for tree={grow'=east,base=top,align=center}
 [\ili{Siouan}\\languages
  [Catawba [\textbf{Catawba}\textbf{\textsuperscript{†}}] [\textbf{Woccon}\textbf{\textsuperscript{†}}]]
  [\ili{Siouan}
    [Missouri River\\ \ili{Siouan} [\textbf{Crow}] [\textbf{Hidatsa}]]
    [\textbf{Mandan},before drawing tree={y+=9mm}]
    [Mississippi Valley\\ \ili{Siouan}\\ (Central \ili{Siouan}),align=center,base=top 
    [Dakotan
      [\textbf{\ili{Sioux} (}dialects:\\\textbf{Teton-\ili{Lakota}{,}} \\\textbf{Santee-Dakota{;}} \\\textbf{\ili{Yankton-Nakoda})},align=center,base=top]
      [\textbf{Assiniboine}]
      [\textbf{Stoney}]
    ]
    [\ili{Dhegiha}
      [\textbf{Omaha-Ponca}]
      [\textbf{Osage}]
      [\textbf{Kansa}\textbf{\textsuperscript{†}}]
      [\textbf{Quapaw}\textbf{\textsuperscript{†}}]
    ]
    [\ili{Chiwere}-\ili{Hoocąk}
      [\textbf{Chiwere}\\\textbf{(}dialects:\\\textbf{ \ili{Iowa}{,} Oto{,}}\\\textbf{Missouria)}\textbf{\textsuperscript{†}}]
      [\textbf{\ili{Hoocąk}{/}}\\\textbf{Winnebago}]
    ]
    ]
    [Ohio Valley \ili{Siouan}\\(Southeastern \ili{Siouan}),base=top,align=center
      [\textbf{Biloxi}\textbf{\textsuperscript{†}}]
      [\textbf{Ofo}\textbf{\textsuperscript{†}}]
      [\textbf{Tutelo}\textbf{\textsuperscript{†}}]    
    ]
  ]
 ] 
\end{forest}
\end{figure}
% % \begin{tabularx}{\textwidth}{lQlQ}
% % \lsptoprule
% % Catawba & \textbf{Catawba}\textbf{\textsuperscript{†}}
% % \textbf{Woccon}\textbf{\textsuperscript{†}} & \multicolumn{2}{l}{}\\
% % \ili{Siouan} & Missouri River \ili{Siouan} & \multicolumn{2}{l}{\textbf{Crow}
% % \textbf{Hidatsa}}\\
% % & \multicolumn{3}{X}{\textbf{Mandan}}\\
% % \tablevspace & Mississippi Valley \ili{Siouan} (Central \ili{Siouan}) & {Dakotan} & \textbf{\ili{Sioux} (}dialects:\textbf{ Teton-\ili{Lakota}, Santee-Dakota; \ili{Yankton-Nakoda})} 
% % \textbf{Assiniboine}
% % \textbf{Stoney}\\
% % \tablevspace &  & {Dhegiha} & \textbf{Omaha-Ponca}
% % \textbf{Osage}
% % \textbf{Kansa}\textbf{\textsuperscript{†}}
% % \textbf{Quapaw}\textbf{\textsuperscript{†}}\\
% % \hhline{~~--} &  & {\ili{Chiwere}-Hoocąk} & \textbf{\ili{Chiwere} (}dialects:\textbf{ \ili{Iowa}, Oto, Missouria)}\textbf{\textsuperscript{ †}}
% % \textbf{\ili{Hoocąk}/ Winnebago}\\
% % \tablevspace & Ohio Valley \ili{Siouan} (Southeastern \ili{Siouan}) & \multicolumn{2}{l}{\textbf{Biloxi}\textbf{\textsuperscript{†}}
% % \textbf{Ofo}\textbf{\textsuperscript{†}}
% % \textbf{Tutelo}\textbf{\textsuperscript{†}}}\\
% % \tablevspace
% % \lspbottomrule
% % \label{tab:helmbrecht:2}
% % \end{tabularx}
% % \end{table}


Although there is some disagreement about the details of this reconstruction it is uncontroversial that \ili{Hoocąk} (also called \ili{Winnebago} in the older literature) and \ili{Chiwere} (also called \ili{Iowa}-Otoe-Missouria or \ili{Báxoje-Jíwere-Ñút’achi}) form a subgroup of the Central \ili{Siouan} or Mississippi Valley \ili{Siouan} languages. All \ili{Siouan} languages, except the Dakotan languages and perhaps Crow, are highly endangered and are on the verge of extinction or already extinct (indicated by little crosses in \figref{tab:helmbrecht:2}). It is estimated that there are less than 200 Native speakers of \ili{Hoocąk} left, who are all older than 60 years of age. 

\subsection{The data}\label{sec:helmbrecht:1.4}
The data for this study are taken from grammatical descriptions and published text sources preferably of the \ili{Siouan} languages that are documented best. Since the languages of the Southeastern \ili{Siouan} branch (Ohio Valley \ili{Siouan}) are extinct for a long time now the descriptive information is not as detailed as for \ili{Lakota} and other Mississippi Valley \ili{Siouan} languages, or from the Missouri River branch of \ili{Siouan}. The data for \ili{Hoocąk} come from fieldnotes and texts that were collected within the \textsc{dobes} project of the documentation of \ili{Hoocąk} (2003--2008). The historical-comparative data are mainly taken from \citet{RankinEtAl2015}.\footnote{See the website of the \textsc{dobes} funding initiative of the Volkswagen Foundation (\url{http://dobes.mpi.nl}). The glossed texts and audio and video files of the \ili{Hoocąk} documentation project are stored in the digital archive of the Max-Planck-Institute for Psycholinguistics called ``The Language Archive''; the corresponding \textsc{url} is: \url{http://dobes.mpi.nl/projects/hocank}. The website of the \textsc{dobes} project ``Documentation of the \ili{Hoocąk} Language'' led by Johannes Helmbrecht and Christian Lehmann at the University of Erfurt, Germany can be found under the following \textsc{url}: \url{http://www2.uni-erfurt.de/sprachwissenschaft/Vgl\_SW/Hocank/index\_frames.html}.} 

\section{From positional verbs to aspect markers in Siouan}\label{sec:helmbrecht:2}

\subsection{Positional verbs of Hoocąk}\label{sec:helmbrecht:2.1}

\ili{Hoocąk} has a set of three so-called \isi{positional} verbs, which belong to the group of verbs of `being/existence'. These positionals denote the bodily posture of a human or animate subject, or the spatial orientation of an inanimate subject. See the forms and their meanings in \tabref{tab:helmbrecht:3}.

\begin{table}
\caption{Hoocąk positionals (cf. \citealt[45]{Lipkind1945}; \citealt[26]{Helmbrecht2010})}


\begin{tabularx}{\textwidth}{lX}
\lsptoprule
\isi{positional}(s) & meaning\\
\midrule
{=nąk} & ‘be (sitting position/neutral position)’\\
{=jee/=jąą} & ‘be (standing position/vertical)’\\
{=(h)ak/=(h)ąk} & ‘be (lying position/horizontal)’\\
\lspbottomrule
\label{tab:helmbrecht:3}
\end{tabularx}
\end{table}


The \isi{positional} verbs may be used as full verbs (as illustrated in \REF{ex:helmbrecht:3}, or as auxiliaries in combination with another full verb that precedes; cf. the subsequent examples in \REF{ex:helmbrecht:4} and \REF{ex:helmbrecht:5}. The positionals in \ili{Hoocąk} have to be analyzed as enclitics, if they are used as auxiliaries.


\ea  \label{ex:helmbrecht:3}
GMA007 \\
\gllll
{\upshape CW:} wa\v{g}į\v{g}į suucra nųųpįį `eegi hanąkwi\\
~ wa\v{g}į\v{g}į šuucra    nųųpi(w)i `eegi hanąkwi\\
~ wa\v{g}į\v{g}į  šuuc=ra  nųųpiwi  `eegi   ha-\textbf{nąk-wi}       \\
~ ball  be.red=\textsc{def}  two      here   \textsc{coll}-\textbf{\textsc{pos.ntl}}-\textsc{pl}\\
\glt CW:   `there are two red balls sitting here'
\z

In \REF{ex:helmbrecht:3}, the \isi{positional} \textit{nąk} `be.sitting' is used as the sole verb in a \isi{predication} of existence/location. In this case, it classifies the referent of the subject NP, `the two red balls', according to its inherent spatial orientation as sitting. In \REF{ex:helmbrecht:4}, \textit{=nąk} `be.sitting' is used as an \isi{auxiliary} to the main verb \textit{wee} `talk' marking \isi{continuative aspect}. In addition, the \isi{positional} marks the spatial orientation of the actor/subject. 

 
\ea\label{ex:helmbrecht:4}
CGF011  \\
\glll  CG: Virgilga waanąkšąną: \\
  CG: Virgil-ga          wee=\textbf{nąk}=šąną:\\
  CG: Virgil-\textsc{prop}    talk(\textsc{sbj}.\textsc{3sg})=\textbf{\textsc{sbj}.\textsc{3sg}.\textsc{pos.ntl}}=\textsc{decl}\\
  
\glll ``Hakerekjane `ee waakšąną.''\\
 ha-kere-kjane  `ee      wee=\textbf{ak}=šąną\\  
  \textsc{1e.a}-go.back.there-\textsc{fut}  \textsc{3emph}    talk=\textbf{\textsc{sbj}.\textsc{3sg}.\textsc{pos}.\textsc{hor}}=\textsc{decl}  \\


\glll  BO: ``Hakerekjane, connection''.\\
  BO:  ha-kere-kjane,                connection\\
     ~    \textsc{1e.a}-go.back.there-\textsc{fut}, connection  \\


\glt  CG: Virgil \textbf{was saying}: ``\textbf{he's saying} I'm going home''. \\
  BO: I'm going home,   connection.
\z

There is a second \isi{positional} in \REF{ex:helmbrecht:4}, \textit{=ak} `be.lying', which is an \isi{auxiliary} to the main verb \textit{wee} `talk'. The \isi{positional} in this construction indicates likewise the spatial orientation of the actor/subject and \isi{continuative aspect}. However, in this example, the actor is not lying, but moving horizontally. The way home is conceptualized as a long line lying on the surface; movement always requires the `be.lying' or horizontal \isi{positional}.
The next example in \REF{ex:helmbrecht:5} represents an instance of \textit{=jee} `be.standing'. This \isi{positional} indicates that the actor/subject of the `telling' is in a vertical/standing position. In addition, \textit{=jee} marks \isi{continuative aspect}. 

 
\ea\label{ex:helmbrecht:5}
MOV024\\
\glll heej\'{ą}ga hopįnįsge  `eeja \\
  heejąga    ho-pįį=nįįsge        `eeja\\
  now         \textsc{appl}.\textsc{iness}-be.good=\textsc{vague}  there\\

\glll hagiregają, Hank Tga\\
  ha-gii-ire=gają  Hank  \textsc{t}-ga\\
  \textsc{coll}-arrive.back.there-\textsc{sbj}.\textsc{3pl}-\textsc{seq}  Hank  \textsc{t}-\textsc{prop}\\

\glll wokarakjeeną, heg\k{u}.\\
 wa-ho<ka->rak=\textbf{jee=}ną,       heg\k{u}\\
  \textsc{obj}.\textsc{3pl}-<\textsc{poss}.\textsc{rfl}->tell=\textbf{\textsc{pos}.\textsc{vert}}=\textsc{decl}  {that.way}\\
\glt `and when they got back to good ground, Hank T \textbf{was telling} about himself:'
\z

It has to be noted that the \isi{positional} auxiliaries are not the only verbs of `being'. There are four others (see \tabref{tab:helmbrecht:4}) and one of these indicates \isi{continuative aspect} in \ili{Hoocąk} likewise. 

\begin{table}
\begin{tabularx}{\textwidth}{llX}
\lsptoprule
verb of being & meaning & comment\\
\midrule
{n\k{i}hé} & `to be/\textsc{cont}' & this verb of being can be used to mark \isi{continuative aspect}; cf. example \REF{ex:helmbrecht:11}\\
\tablevspace
{heré} & `to be' & \isi{copula}/\isi{auxiliary}, never used to indicate aspect\\
\tablevspace
{'\'{\k{u}}ų} & `to be/do' & never used as an \isi{auxiliary} with a full verb in order to mark \isi{continuative aspect},
rather sometimes it marks a slight \isi{causative} meaning,
frequently combined with one of the positionals\\
\tablevspace
{wa'\k{u}} & `to be/do' & similar as \textit{'\'{\k{u}}ų}\\
\lspbottomrule
\end{tabularx}
\caption{Other verbs of being/existence in Hoocąk} 
\label{tab:helmbrecht:4}
\end{table}
 
\subsection{Positionals of other Siouan languages}\label{sec:helmbrecht:2.2}

The positionals of \ili{Hoocąk} discussed in the previous section are from Common \ili{Siouan}. In all \ili{Siouan} languages, at least traces of the \isi{positional} verbs can be found. Compare the cognate forms as reconstructed by \citet{RankinEtAl2015} in the \textit{Comparative \ili{Siouan} dictionary} (cf. also \citealt{Rankin2004a}) in \tabref{tab:helmbrecht:5}.

\begin{table}
\begin{tabularx}{\textwidth}{XXXXX}
\lsptoprule
 & \textsc{sit} & \scshape lie & \scshape stand & \scshape stand\\
\midrule 
Proto-\ili{Siouan} & \textit{*r\'{ą}•-kE} & \textit{*w\'{\k{u}}•kE} & \textit{*rahÉ} & \textit{*h\'{ą}(-kE)}\\
Crow & \textit{da•čí} & \textit{baačí} & \textit{-} & \textit{áahku}\\
\ili{Hidatsa} & \textit{rá•kE} & \textit{wá•kE} & \textit{rahÉ} & \textit{háhku}\\
\ili{Mandan} & \textit{rąk} & \textit{wąk} & \textit{te} & \textit{hąk}\\
\ili{Lakota} & \textit{yąká} & \textit{y\k{u}ká} & \textit{he} & \textit{hą}\\
\ili{Chiwere} & \textit{n\'{ą}ŋe} & \textit{h\'{ą}ŋe} & \textit{ǰe} & \textit{-}\\
\ili{Hoocąk} & \textit{=n\'{ą}k} & \textit{=(h)ąk} & \textit{=jee} & \textit{=jąą}\\
\ili{Biloxi} & \textit{n\'{ą}ki} & \textit{m\'{ą}ki} & \textit{ne} & \textit{h\'{ą}de}\\
\ili{Tutelo} & \textit{nąka} & \textit{-mąki-} & \textit{ne} & \textit{-h\'{ą}k}\\
\lspbottomrule
\end{tabularx}
\caption{Positionals in Siouan (cf. \citealt{Rankin2004a}, \citealt{RankinEtAl2015})}
\label{tab:helmbrecht:5}
\end{table}

In all \ili{Siouan} languages – except the languages of the \ili{Dhegiha} branch – the positionals are used as auxiliaries and often as markers that indicate \isi{continuative aspect}. The \ili{Dhegiha} languages lost the Proto-\ili{Siouan} \isi{positional} verbs. The positionals developed into classificatory definite articles in these languages, see \sectref{sec:helmbrecht:6.1} below.

\subsubsection{Crow positionals}\label{sec:helmbrecht:2.2.1}

In Crow, there is a set of six \isi{auxiliary} verbs/markers that indicate \isi{continuative aspect}. Three of them are descendants of the Proto-\ili{Siouan} forms marked in bold face in \tabref{tab:helmbrecht:6}.

\begin{table}
\begin{tabularx}{\textwidth}{lQQl}
\lsptoprule
Form & Meaning & Degree of \isi{coalescence} with main verb & Proto-\ili{Siouan}\\
\midrule
\textbf{{datchí}} & `continue (by mouth)' & \textsc{[v+aux]} one word & \scshape sit\\
{dawí} & `continue in motion; begin to' & \textsc{[v+aux]} one word & \\
\textbf{(d)ahkú} & `continue in activity, remain, dwell' & also independent verb & \scshape stand\\
{dachí} & `remain voluntarily' & also independent verb & \\
\textbf{{baachí}} & `lie, remain involuntarily' & always independent & \scshape lie\\
{ilúu} & `do repeatedly, continue' & always independent & \\
\lspbottomrule
\end{tabularx}
\caption{Continuative markers in Crow (cf. \citealt[305--309]{Graczyk2007})}
\label{tab:helmbrecht:6}
\end{table}

These auxiliaries are inflected for person/number of the actor/subject. The actor/subject is obligatorily co-referential with the actor\slash subject of the main verb. If the main verb preceding the \isi{auxiliary} does not form a single word with it, it obligatorily has the same subject marker (\textit{-ak} SS). Crow as well as \ili{Hidatsa} have developed a switch reference marking system. Otherwise, there is an additional \isi{continuative marker} (\textit{-a} \textsc{cont}) between the main verb and the \isi{auxiliary}. These auxiliaries behave differently with respect to the closeness of the \isi{coalescence} with the main verb.

\subsubsection{Mandan positionals}\label{sec:helmbrecht:2.2.2}

\ili{Mandan} has four \isi{positional} verbs with a \isi{stative} meaning; all of them are descendants of the Proto-\ili{Siouan} positionals. These \isi{positional} verbs are used as full verbs designating the existence or being of an entity at some place. For instance, the bound form \textit{te-} `stand' is used to indicate the position of a village in the text in Mixco (not reproduced here; see \citealt[66]{Mixco1997}, sentence 1). 
Three of the four positionals are used – in addition – as auxiliaries to indicate \isi{continuative aspect}. They have a \isi{continuative marker} \textit{-æ} (\textsc{cont}) with them and are translated by Mixco as `abide:sitting\slash standing\slash lying'; cf. \tabref{tab:helmbrecht:7}.

\begin{table}
\caption{Mandan positionals (\citealt[48f]{Mixco1997}).\protect\footnote{\citegen{Mixco1997} analysis deviates somewhat from \citegen{Kennard1936}. \citeauthor{Kennard1936} takes the three positionals \textit{–nąk} `be.sitting', \textit{-hąk} `be.standing' and \textit{–mąk} `be.lying' as auxiliaries that indicate \textsc{cont}inuative aspect if they are preceded by the continuative marker \textit{ha-}.  This marker is not mentioned in \citeauthor{Mixco1997}. Instead, \citeauthor{Mixco1997} postulates that the element \textit{-æ} marks continuative aspect. Note also that [n] and [m] are taken as allophones of /r/ and /w/ before nasal vowels in Mandan by \citeauthor{Mixco1997}. There are no nasal consonants in the phoneme inventory of Mandan.}}
\begin{tabular}{lllll}
\lsptoprule
\multicolumn{2}{l}{Stative verbs} & \multicolumn{2}{l}{Continuative auxiliaries} & Proto-\ili{Siouan}\\
\midrule 
{rąk} & `sit' & \textit{rąk-}{æ} & `abide:sitting' & \scshape sit\\
{hąk} & `stand' & \textit{hąk-}{æ} & `abide:standing' & \scshape stand\\
{wąk} & `lie' & \textit{wąk-}{æ} & `abide:lying' & \scshape lie\\
{te-} & `stand' &  \textit{r\k{u}r\k{i}h} & `exist.\textsc{pl}'  & \scshape stand\\
\lspbottomrule
\end{tabular}
\label{tab:helmbrecht:7}
\end{table}

Interestingly, the \isi{continuative} LIE \textit{wąk-}{æ} is by far the most frequently used aspect marker in the \ili{Mandan} text that I examined. The posture meaning is neutralized in most of these usages. In addition, if the subject is plural, only the LIE \isi{continuative} can be used bearing the regular plural marker; cf. the example in \REF{ex:helmbrecht:6}.\largerpage[2]

\ea \label{ex:helmbrecht:6}
\ili{Mandan} 

"{ko-    h}{\'{\k{u}}}{:}{-      æ   ki-rút-r\k{i},} \textbf{{w\'{ą}:k-       æ-        kræ- oʔš}}!'' 

"\textsc{3sg}-mother-\textsc{sv}  MV-eat-\textsc{ss}   \textbf{abide:lie-}\textbf{\textsc{cont}-\textsc{pl}-\textsc{ind}}\textbf{.male}'' 

{é=    he-  ro:wąk-          oʔš}.

PV=say-\textsc{narr}.\textsc{past}-\textsc{ind}.male

`{``}They're eating their mother up!'' he said.' \citep[69]{Mixco1997}
\z


The combination of full verb plus \isi{auxiliary} indicating \isi{continuative aspect} is not as close as the one in Crow; the SS marker is not obligatory, often one finds a simultaneous (SIM) ending on the preceding verb.
In addition to the \isi{aspect marking function} of the positionals, \ili{Mandan} has developed classificatory \isi{demonstrative}s on the basis of these positionals. They are not as firmly grammaticlized as in \ili{Hoocąk} and differ from the \ili{Hoocąk} ones in that the \isi{positional} follows the \isi{deictic particle} (cf. \citealt[42]{Mixco1997}). I will discuss this construction below in \sectref{sec:helmbrecht:5}.  

\subsubsection{Teton-Lakota, Santee-Dakota, Yankton-Nakota positionals}\label{sec:2.2.3}

In \ili{Lakota}, i.e. the Teton dialect of the \ili{Sioux} language, there are likewise at least three verbs of `being/existence' that are descendants of the Proto-\ili{Siouan} positionals (cf. \citealt[126f]{BoasDeloria1941}); cf. \tabref{tab:helmbrecht:8}.

\begin{table}
\begin{tabularx}{\textwidth}{lQl}
\lsptoprule
Form & Meaning & Proto-\ili{Siouan}\\
\midrule
{yąká} & `to sit', `be.sitting' (spherical objects, animals etc.) & \scshape sit\\
{=hą} & `to stand', `be standing' (long upright objects) & \scshape stand\\
{y\k{u}ká} & `to lie', `be.lying' (mostly animate beings) & \scshape lie\\
\lspbottomrule
\end{tabularx}
\caption{Positionals in Lakota (cf.   \citealt[126]{BoasDeloria1941})}
\label{tab:helmbrecht:8}
\end{table}

All three forms can be used as independent verbs of posture and of `being'\slash`existence' in all three dialects of \ili{Siouan} proper (Santee-Dakota, Teton-\ili{Lakota}, and Yankton-Nakota). See an example from \ili{Lakota} in \REF{ex:helmbrecht:7}.\textbf{} Further examples can be found in \citet{Rankin2004a} and \citet{Barron1982}.

\ea \label{ex:helmbrecht:7}
\ili{Lakota} \\
\gll ... kʼeya\.{s}  tʽimá \textbf{yąká-pi} ki ʼátaya\.{s}   wąwícʽayakapi\.{s}ni       ną  ...\\
  ~         but       in.the.tent   \textbf{sit-they} the   entirely  they.did.not.see.them  and\\
\glt `but sitting in the tent they (the twins) did not see them, and...' (\citealt[193ff]{Deloria1932};  \citealt[170]{BoasDeloria1941})
\z 

However, in \ili{Lakota}, \textit{=hą} has become a fully grammaticalized \isi{enclitic} that marks \isi{continuative aspect} (cf. \citealt[60f]{BoasDeloria1941}; also \citealt[31]{Ingham2003}). As such, it can no longer be inflected for person\slash number of the subject\slash actor; it can even be combined with one of the other positionals; see the example from the same text in \REF{ex:helmbrecht:8}. In this usage, \textit{=hą} has lost completely its posture meaning.\newpage

\ea \label{ex:helmbrecht:8}
\ili{Lakota} \\
\gll tʽé\textbf{hą}           yé{\.{s}} tąyą waápʽe \textbf{yąká-hą-pi}          kʼ\k{u}...\\
a.long.time but well wait      \textbf{sit-\textsc{cont}-they}  the.past\\
\glt `but a long time they were waiting....' (\citealt[193ff]{Deloria1932}; \citealt[170]{BoasDeloria1941})
\z 


Note also, that \textit{=hą} has become the basis of a variety of derivations such as time adverbials; compare \textit{t}ʽ{é}\textbf{{hą}}\textbf{} `a long time'\textbf{} in \REF{ex:helmbrecht:8}; cf. \citet[60f]{BoasDeloria1941}. The two other \ili{Siouan} proper dialects, Santee-Dakota and Yankton-Nakota (cf. \figref{tab:helmbrecht:2}) specialized the \isi{positional} \textit{yąká} `be.sitting' as the neutral or general \isi{auxiliary} in order to mark \isi{continuative aspect} in case that the posture of the actor\slash subject is not in focus or is unimportant (see \citealt[165]{Deloria1932}). The same holds for \ili{Hoocąk}: the `be.sitting' \isi{positional} \textit{=nąk} is also used as the neutral unmarked \isi{continuative marker}. This unmarked\slash neutral usage of the `be.sitting' \isi{positional} \textit{=nąk} is nicely reflected in the textual frequencies of this \isi{auxiliary} in the entire \textsc{dobes} corpus of \ili{Hoocąk}; cf. \tabref{tab:helmbrecht:9}.

\begin{table}
\begin{tabular}{llr}
\lsptoprule
Form & Gloss & Frequency \\
\midrule
{=nąk} & \textsc{pos.ntl} & n= 1286\\
{=jee}/{=jąą} & \textsc{pos}.\textsc{vert} & n = 522\\
=({h}){ąk}/=({h}){ak} & \textsc{pos}.\textsc{hor} & n = 167\\
\lspbottomrule
\end{tabular}
\caption{Absolute frequencies of the positionals in the \textsc{dobes} corpus}
\label{tab:helmbrecht:9}
\end{table}

The `be.sitting' \isi{positional} \textit{=nąk} occurs as an \isi{auxiliary}/verb twice as often in the corpus as the two others together.

\subsubsection{Biloxi positionals}\label{sec:helmbrecht:2.2.4}

\ili{Biloxi} is a \ili{Siouan} language of the Southern branch. The cognate \ili{Biloxi} positionals \textit{nąki} `sit', \textit{mąki} `lie', and \textit{ne} `stand' (see \tabref{tab:helmbrecht:5}) are all used as classifiers in \isi{copula} clauses that localize a non-human subject, and in verbal clauses with a complex \isi{predicate} to mark \isi{continuative aspect}; cf. \REF{ex:helmbrecht:9}.

\ea \label{ex:helmbrecht:9}
\ili{Biloxi} 

{Ayá}{\textsuperscript{n}}{ xotká    u-x}{ĕ}{'} \textbf{{ná}}\textbf{{ñ}}\textbf{{k}}\textbf{{̟}}\textbf{{i}}{,       xyih}{ĕ}{'} \textbf{{ná}}\textbf{{ñ}}\textbf{{k}}\textbf{{̟}}\textbf{{i}}{} [{On't}{̟}{i-yándi}]

  tree   hollow in-sit  \textbf{be.sitting}  growl   \textbf{be.sitting}  Bear-\textsc{sbj}

 `Bear was then in a hollow tree where he was growling.' 
 (\citealt[16]{Dorsey1912}; sentence 10)
\z 

There is one peculiarity in \ili{Biloxi} that other \ili{Siouan} languages lack. \ili{Biloxi} developed a gender classification of the possessum in \isi{possessive} predications with positionals. The \isi{positional} \textit{nąki} `sit' is used as a \isi{copula} in \isi{possessive} clauses that express possession of a female kin. The \isi{positional} \textit{mąki} `lie' is used in turn to indicate that the possessum is a male kin; cf. \tabref{tab:helmbrecht:10} and an illustrative example in \REF{ex:helmbrecht:10}. There are numerous examples in \citeauthor{Dorsey1912} (\citeyear[130]{Dorsey1912}; cf. also \citealt{Kaufmann2011}) that illustrate this sex classification.

\ea \label{ex:helmbrecht:10}
\ili{Biloxi} \\
\gll   Ay-ó\textsuperscript{n}ni       é \textbf{nañkí}\\
  Your-mother  he/she   sit(female.possessum)\\
\glt `You have a mother.' (\citealt[130]{Dorsey1912})
\z

\begin{table}
\begin{tabularx}{\textwidth}{llQl}
\lsptoprule
Positional & Possessum & Meaning & Proto-\ili{Siouan}\\
\midrule 
{nąki} & female kin & `be.sitting' & \scshape sit\\
{mąki} & male kin & `be.lying' & \scshape lie\\
({h}){ąde} & singular; no classification & `to be', `be.moving' & \scshape stand\\
{yuk}{ȇ} & plural of ({h}){ąde}; no classification & `to be', `be.moving.\textsc{pl}' & \\
\lspbottomrule
\end{tabularx}
\caption{Biloxi positionals as copula in possessive clauses (cf. \citealt{Kaufmann2011})}
\label{tab:helmbrecht:10}
\end{table}

Interestingly, one of the Proto-\ili{Siouan} positionals in \ili{Biloxi}, \textit{ne} STAND (cf. \tabref{tab:helmbrecht:5}) seems to have developed into a \isi{demonstrative}, cf. the enty in \citeauthor{Kaufmann2011}'s \ili{Biloxi} dictionary (\citeyear[100]{Kaufmann2011}); there are also examples in \citet[117--167]{Dorsey1912}, where \textit{ne} is used alternatively as a \isi{definite article} and a \isi{demonstrative}. 

\section{From positional auxiliaries to classificatory demonstratives}\label{sec:helmbrecht:3}

Synchronically, \ili{Hoocąk} has two paradigms of \isi{demonstrative}s. The first paradigm is called here the ``old'' paradigm, since these forms can be traced back to Proto-\ili{Siouan} as will be shown later. The second paradigm is called here the ``new'' paradigm, since it is a recent innovation in \ili{Hoocąk}. The forms are composed of the \isi{positional} auxiliaries plus a \isi{deictic particle} distinguishing \isi{proximal} (\textit{-re}) and \isi{distal} (\textit{-ga}); (on the grammaticalization of these particles, see \sectref{sec:helmbrecht:4} below).

\begin{table}
\resizebox{\textwidth}{!}{\begin{tabular}{ll}
\lsptoprule
% paradigm 
{Form} & {Meanings}\\
\midrule
\multicolumn{2}{c}{\textbf{``old'' paradigm}}\\
\midrule
\textit{tée\slash te'é} & `this', `here', `now'\\ 
\textit{mée\slash me'é} & `this'\\ 
\textit{žée\slash že'é} & `that', `there'\\ 
?\textit{ga'á} & `that'\\\midrule
\multicolumn{2}{c}{\textbf{``new'' paradigm}} \\
\midrule
\textit{=nąka} (<={nąk-ga}) & ‘that (sitting\slash neutral position; \isi{distal})’\\ 
\textit{=nąąka} (<={nąąk-ga}) & ‘those (sitting\slash neutral position; plural; \isi{distal})’ \\ 
\textit{=nąągre} (<={nąąk-re}) & `these (sitting\slash neutral position; \isi{proximal}; plural)' \\ 
\textit{=nągre} (<={nąk-re}) & `this (sitting\slash neutral position; \isi{proximal})' \\ 
\textit{=jeega} (<={jee-ga}) & ‘that (standing\slash vertical position; \isi{distal})’\\ 
\textit{=jąąne}      (<={jąą-re}) {=jaane} (<={jee-re}) & `this (standing\slash vertical position; \isi{proximal})'\\ 
\textit{=ąka} (<={ąk-ga}) & ‘that (lying\slash horizontal position; \isi{distal})’ \\ 
\textit{=agre} {=ągre} (<={ąk-re}) & `this (lying\slash horizontal position; \isi{proximal})' \\
\lspbottomrule
\end{tabular}}
\caption{Two paradigms of demonstratives in Hoocąk}
\label{tab:helmbrecht:11}
\end{table}

Both paradigms are frequently used in \ili{Hoocąk} texts. In the subsequent sections (\sectref{sec:helmbrecht:3.1}--\sectref{sec:helmbrecht:3.2}) I will present a brief overview of the semantic, pragmatic and distributional properties of the forms of both paradigms. In \sectref{sec:helmbrecht:4} I will present some suggestions on the grammaticalization of the new paradigm and the effects on the usages of the ``old'' paradigm. 

\subsection{{The ``new'' paradigm of adnominal demonstratives}{ in Hoocąk}}\label{sec:helmbrecht:3.1}

\subsubsection{Morphosyntactic and semantic properties}\label{sec:helmbrecht:3.1.1}

The ``new'' \isi{demonstrative}s in \ili{Hoocąk} are used exclusively as adnominal \isi{demonstrative}s. They always follow the head noun and occur in the same structural position as other determiners such as the definite and indefinite articles at the right edge of the NP; cf. the structural template of the lexical NP in \tabref{tab:helmbrecht:12}.

\begin{table}
\resizebox{\textwidth}{!}{\begin{tabular}{llll}
\lsptoprule
(N) & (Lexical modifier) & Determiner & (Quantifier)\\
\midrule
- noun & - adjectival concepts & - \isi{definite article} \textit{=ra}; & - numerals\\
& & - indefinite article ={hižą}; & - etc. \\
& & - \textbf{``new'' adnominal demonstratives}\\
& & - Ø \\
\lspbottomrule
\end{tabular}}
\caption{Structure of the NP in Hoocąk. Elements in parentheses are optional.}
\label{tab:helmbrecht:12}
\end{table} 

The postnominal \isi{demonstrative}s classify the head noun according to the postural position of its referent and according to its distance from the reference point (\isi{proximal} vs. \isi{distal}). If the postural position of the referent is non-salient, the neutral \isi{demonstrative} is chosen; cf. example \ref{ex:helmbrecht:11}. The postural position of the `coal' in this utterance is not salient, hence the be.sitting\slash neutral \isi{demonstrative} has been chosen.

\ea \label{ex:helmbrecht:11}
BOF008\\
\glll Hegų `ųų hanįhaire, hagoreižą `ųųxįnį seepnąka tuusšąną.\\
    hegų        `ųų          ha-\textbf{nįhe-}ire          hagoreižą   \textbf{'ųųxįnį}   \textbf{seep=nąka}                     tuus=šąną\\
    that.way  do/make \textsc{coll}-\textbf{be/}\textbf{\textsc{prog}}-\textsc{sbj}.\textsc{3pl}  sometime    \textbf{charcoal} \textbf{be.black=}\textbf{\textsc{pos.ntl}:\textsc{dist}}   take{\textbackslash}\textsc{1e.a}=\textsc{decl}\\
\glt  `They kept on going that way, \textbf{that coal} at some point I took it.'
\z

In general, the postural classification of the referents is semantically motivated. Larger animals, for instance, usually are standing, hence the `standing\slash vertical' \isi{demonstrative} is chosen in the utterance in \REF{ex:helmbrecht:12}.

 \ea \label{ex:helmbrecht:12}
HOR064\\
\glll Šųųkįgjeega šųųkxetera haracap nąą'į hegų.\\
\textbf{šųųk-įk=jeega}                     šųųkxete=ra  haracap nąą'į               hegų\\
\textbf{dog-\textsc{dim}=\textsc{pos}.\textsc{vert}:\textsc{dist}}  horse=\textsc{def}    taste        try(\textsc{sbj}.\textsc{3sg})  that.way\\
\glt   `That dog tried to bite the horse.'
\z


The adnominal \isi{demonstrative}s are - like the \isi{definite article} (=\textit{ra}) - used to nominalize a clause. This is a general strategy in \ili{Hoocąk} to indicate subordination. Relative clauses, for instance, usually require a nominalizing determiner such as the \isi{definite article} or one of the ``new'' adnominal \isi{demonstrative}s; cf. an elicited example in \REF{ex:helmbrecht:13}. The new \isi{demonstrative} classifies the head noun according to the posture; in addition, it still preserves a \isi{progressive meaning} for the relative clause. 

\ea \label{ex:helmbrecht:13}
(Phil Mike; elicited example)\\
\glll  wan\'{\k{i}}ną tuujágre\\
{\ob}wan\'{\k{i}}=ra [{tuuc      haa=\textbf{{ágre}}}]\textsubscript{relative clause}]\\
  meat=\textsc{def}    cooked  \textsc{1e.a}.CAUSE=\textbf{\textsc{pos}.\textsc{hor}.\textsc{prox}}\\
  \glt `\textbf{this} meat \textbf{(lying/horizontal)} I am cooking now'
\z

Example \REF{ex:helmbrecht:14} illustrates that the ``new'' adnominal \isi{demonstrative}s are used as subordinators in general. 

 
\ea \label{ex:helmbrecht:14}
BOF023\\
\glll `Eejaxjį hegų hąąp hitanįhąįja hegųgają hegų žige hįšjųwąk, hegų hegų `eeja       hamįknąka, žige hanąąňegi, `eeja wažą yaahąte.\\
    `eejaxjį        hegų       hąąp   hi-taanį-hą=hija hegų=gają  hegų        žige     hį-šjųwą='ąk                       hegų  hegų \textbf{'eeja} \textbf{ha-mįįk=nąka}                      žige   ha-nąą=regi  `eeja   wažą         hi<ha>hąte  \\
    about.there that.way  day    \textsc{ord}-three-times-there    that.way=\textsc{seq}    that.way  again  \textsc{1e.u}-get.sleepy=\textsc{pos}.\textsc{hor}  that.way    that.way  \textbf{there}  \textbf{\textsc{1e.a}}\textbf{-lie.down=}\textbf{\textsc{pos.ntl}:\textsc{dist}}  again  \textsc{1e.a}-sleep=\textsc{sim/loc}  there something <\textsc{1e.a}>dream.of\\
\glt  `About on the third day I got sleepy again, \textbf{lying there} I went to sleep again, I     dreamed again.'
\z


From a semantic point of view it is interesting to see that the speaker chose the \isi{demonstrative} of the neutral position, and not the one of the lying position, which one would have expected. 

\subsubsection{{Pragmatics of the adnominal demonstratives}}\label{sec:helmbrecht:3.1.2}

NPs with one of the adnominal \isi{demonstrative}s mostly appear in texts, if the referent had already been introduced at some distance in the previous text; the \isi{demonstrative}s are used to refer back to an old or fainted topic. The following example illustrates this nicely.\largerpage[2]

 \ea \label{ex:helmbrecht:15}
BOF035\\
\glll    Hiraijixjįgają hegų caaxšepjaane žige hižą ha\v{g}epšąną.\\
    hira<gi>ji-xjį=gają  hegų  \textbf{caaxšep=jaane}                  žige    hižą  ha\v{g}ep=šąną\\
    <\textsc{appl}.\textsc{ben}>reach-\textsc{ints}=\textsc{seq}   that.way    \textbf{eagle=}\textbf{\textsc{pos}.\textsc{vert}:\textsc{prox}} again  one   appear=\textsc{decl}\\
\glt `He was getting close and then \textbf{this eagle} appeared again.'
\z

The eagle had been introduced a few clauses before the one in \REF{ex:helmbrecht:15}, and is then reintroduced by means of a NP with a \isi{proximal} \isi{demonstrative}. Since the eagle appeared up in the sky, its position is conceptualized here as vertical. 
There are also textual examples that illustrate that the \isi{proximal} adnominal \isi{demonstrative} can be used as a specific indefinite article like Colloquial English \textit{this}. Compare the following utterance in \REF{ex:helmbrecht:16}. The ``man'' in this story is mentioned the first time; he is specific, but indefinite.

\ea  \label{ex:helmbrecht:16}
TWI003\\
\glll Ciin\'{ą}k kąn\'{ą}kiregi `eeja ciiregi hagoréižą hagoréižą wąąkjaané hin\'{\k{u}}kra       hakaráikižu roog\'{\k{u}}įňe.\\
    ciinąk  kąnąk-ire=gi                             `eeja  cii-ire=gi hagoreižą  hagoreižą \textbf{wąąk=jaane}      hinųk=ra ha<kara-kii>kižu      roogų-ire\\
    village  place(\textsc{obj}.\textsc{3sg})-\textsc{sbj}.\textsc{3pl}=\textsc{top}   there  live-\textsc{sbj}.\textsc{3pl}=\textsc{top} sometime  sometime  \textbf{man=}\textbf{\textsc{pos}.\textsc{vert}:\textsc{prox}}  woman=\textsc{def} <\textsc{poss}.\textsc{rfl}-\textsc{rcp}>be.together    want-\textsc{sbj}.\textsc{3pl}\\
   \glt    `Where they lived, \textbf{a man} and his wife wanted (something).'
    (lit. `They placed a village, there they lived, once upon a time \textbf{this man} together     with his wife, wanted something')
\z

\subsection{{The ``old'' paradigm of Hoocąk demonstratives}}\label{sec:helmbrecht:3.2}

\subsubsection{{Common Siouan origins}}\label{sec:helmbrecht:3.2.1}

The paradigm of ``old'' \isi{demonstrative}s can be shown to be of Common \ili{Siouan} origin; cf. the cognate forms in \tabref{tab:helmbrecht:13}. The forms for Proto-\ili{Siouan} that were reconstructed distinguish three grades of \isi{deictic} distances. 

\begin{itemize}
\item \isi{proximal}\slash close to speaker, 
\item medial\slash close to hearer, and 
\item \isi{distal}\slash away from both speaker and hearer. 
\end{itemize}

\newpage
\vspace*{2\baselineskip}
\begin{table}
\begin{tabularx}{\textwidth}{lQQQQ}
\lsptoprule
 & \textsc{this} (\isi{proximal}) & \textsc{this} (\isi{proximal}) & \textsc{that} (medial) & \textsc{that} (\isi{distal})\\
\midrule
Proto-\ili{Siouan}\footnotemark{} & \textit{*Ree-}\footnotemark{} & \textit{*re-} & \textit{*šee} & \textit{*kaa}\\
Crow &  &  &  & \textit{kaka}\\
\ili{Hidatsa} &  &  & \textit{še-'e} & \textit{kaa}\\
\ili{Mandan} &  & \textit{re} &  & \textit{ká-}\\
\shadecell \ili{Lakota} &\shadecell  \textit{le-} &\shadecell   &\shadecell  \textit{še-} &\shadecell  \textit{ka-}\\
\shadecell \ili{Chiwere} &\shadecell  \textit{ǰe-} &\shadecell   &\shadecell  \textit{šé-'e} &\shadecell  \textit{gá\slash gá'e\slash ká}\\
\shadecell \ili{Hoocąk} &\shadecell  \textit{te-'e, tée} &\shadecell   &\shadecell  \textit{že-'é\slash žée} &\shadecell  \textit{=ga/} \textsuperscript{?}{ga'a}\\
\shadecell \ili{Omaha-Ponca}\footnotemark{} &\shadecell   &\shadecell  \textit{ðe} &\shadecell  \textit{še} &\shadecell  \textit{ka}\\
\shadecell Kansa\footnotemark{} &\shadecell   &\shadecell  \textit{ye, yé-che, yé-khe} &\shadecell  \textit{še} &\shadecell  \textit{ga}\\
\shadecell Osage\footnotemark{} &\shadecell   &\shadecell  \textit{ðe, ðee} &\shadecell  \textit{še}/{ šee} &\shadecell  \textit{ka\slash kaa}\\
\shadecell \ili{Quapaw}\footnotemark{} &\shadecell   &\shadecell  \textit{de} &\shadecell  \textit{še} &\shadecell  \textit{ká-khe}\\
\ili{Biloxi}\footnotemark{} & \textit{ne-tka} & \textit{de} &  & \textit{ká-wa}\\
\ili{Tutelo}\footnotemark{} & \textit{née} & \textit{lèe} &  & \textit{ka}/{ ko}\\
\lspbottomrule
\end{tabularx}
\caption{Cognate sets of the Common Siouan demonstratives \citep{RankinEtAl2015}}
\label{tab:helmbrecht:13}
\end{table}
\addtocounter{footnote}{-8}
\stepcounter{footnote}\footnotetext{Cf. the \textit{Comparative \ili{Siouan} Dictionary} \citep{RankinEtAl2015}.}
\stepcounter{footnote}\footnotetext{The capital \textit{R} in \textit{*Ree} symbolizes a hypothetical cluster of a resonant /{*r}/ plus a laryngeal; cf. \citet{RankinEtAl1998}. According to the autors of the CSD, there are independent reasons to postulate two different /{r}/ sounds.}
\stepcounter{footnote}\footnotetext{Cf. \citet[324--326]{Boas1907}; \citet[138--142]{Koontz1984}.}
\stepcounter{footnote}\footnotetext{Cf. \citet[350f]{CumberlandRankin2012}; the \isi{proximal} form is only attested as an adverbial \isi{demonstrative}, otherwise only in combination with one of the classifying definite articles.}
\stepcounter{footnote}\footnotetext{Cf. \citet[359--368]{Quintero2004}.}
\stepcounter{footnote}\footnotetext{Cf. \citet[ms]{Rankin2002}; the medial and \isi{distal} forms are attested only in combination with one of the classfying and definite articles in \ili{Quapaw}.}
\stepcounter{footnote}\footnotetext{Cf. See \citet[69]{Einaudi1976} for the proximate and medial form; see \citet[77]{Kaufmann2011} for the \isi{distal form}.}
\stepcounter{footnote}\footnotetext{Cf. \citet[155]{Oliverio1996}.}
\clearpage

That the authors of the CSD reconstructed two different \isi{proximal} \isi{demonstrative}s (\textit{*Ree-} and \textit{*re-}) is motivated by independent reasons (cf. \citealt{RankinEtAl1998}).

\begin{itemize}
\item The languages of Mississippi Valley \ili{Siouan} all preserved the whole set of \isi{demonstrative}s; see the shaded lines in \tabref{tab:helmbrecht:13}. 
\item Reflexes of the Proto-\ili{Siouan} \isi{demonstrative}s are lacking in the Northwestern \ili{Siouan} languages (Crow, \ili{Hidatsa} and \ili{Mandan}).
\item The forms of \ili{Biloxi} and \ili{Tutelo} (both Ohio Valley) are less certain; these languages are not well documented.
\item Interestingly, reflexes of the \isi{distal form} can be found in all \ili{Siouan} languages.
\end{itemize}

The \ili{Hoocąk} forms that are of Common \ili{Siouan} origin are given in \tabref{tab:helmbrecht:14} together with their function and meaning in contemporary \ili{Hoocąk}.

\begin{table}
\small 
\begin{tabularx}{\textwidth}{llL{3cm}lQ}
\lsptoprule
\multicolumn{2}{l}{\ili{Hoocąk} forms} & Meaning & Proto-\ili{Siouan} & Meaning\\
\midrule 
\shadecell {te'e}/{tee} & \shadecell < \textit{te-} + \textit{'ee} &\shadecell  `this', `here', `now' & \shadecell \textit{*Ree-} & \shadecell `this' (\isi{proximal})\\
\tablevspace
\shadecell {me'e\slash mee} & \shadecell < \textit{me-+'ee} &\shadecell  `this', `here' &\shadecell  ? & \shadecell ~\\
\tablevspace
\shadecell {že'e}\slash \textit{žee} &\shadecell < \textit{že-} + \textit{'ee} &\shadecell  `that', `there', `then' &\shadecell&\shadecell \\
{=že/=še} &  & \textsc{quot} & \multirow{-2}{*}{\shadecell  \textit{*šee}}  &\multirow{-2}{*}{\shadecell `that' (medial)} \\
\tablevspace
\shadecell {\textsuperscript{?}}{ga'a} & \shadecell < \textit{ga}- + \textit{'ee} & \shadecell `that' &\shadecell & \shadecell \\
{=ga} &  & `that'; 
(bound \isi{enclitic} form) &\shadecell  &\shadecell \\

{=ga} &  & proper name marker (anthroponyms, kinship terms as proper names); & \shadecell & \shadecell\\

{=ga} &  & sentence connector \isi{continuative} ('and then'); &\shadecell  &\shadecell \\

{=ga} & \textsc{pos}+ga & \isi{distal} classfying adnominal \isi{demonstrative} & \multirow{-5}{*}{\shadecell  \textit{*kaa}\vspace*{6cm}~}  & \multirow{-5}{*}{\shadecell `that' (\isi{distal})\vspace*{6cm}~}\\
\tablevspace
{'ee} &  & `thus', `it', `this', `that', `he', `she' (aforementioned);
always \textsc{emph} or in focus constructions (as a 3rd personal pronoun, free form); & \textit{*'ee} & `that' (aforementioned)\\
\lspbottomrule
\end{tabularx}
\caption{Hoocąk demonstratives: the ``old'' paradigm and its grammaticalizations}
\label{tab:helmbrecht:14}
\end{table}


The \isi{proximal} and medial forms (\textit{te'e} `this' and \textit{že'e} `that') are obviously a composition of the Proto-\ili{Siouan} \isi{deictic stem} (\textit{te-} and \textit{že-}) plus a \isi{demonstrative pronoun} \textit{'ee} `that (aforementioned)', which is likewise attested in all \ili{Siouan} languages. This form is variably analyzed as a free pronoun or \isi{demonstrative pronoun} refering back to somthing already mentioned (aforementioned) in discourse. In \ili{Hoocąk}, it is not only used as an anaphoric pronoun, but also in focus constructions in order to express emphasis on a third person participant.


Semantically, both the \isi{proximal} and medial \isi{demonstrative}s seem to have neutralized the \isi{deictic} distance distinction almost completely; only in the adverbial uses the distinction between ``close to speaker'' and ``far from speaker'' is preserved. 


The \isi{distal form} \textit{ga'a} `that' is mentioned in older sources on \ili{Hoocąk} (cf. \citealt[52]{Lipkind1945}); however, there is not a single instance of this form in our \textsc{dobes} corpus (which contains contemporary but also older texts from the beginning of the 20th century); the composition of this form is analog to the one of the \isi{proximal} and medial forms attaching the anaphoric pronoun \textit{'ee} `that, etc.' to the \isi{distal} \isi{demonstrative} stem \textit{ga-}. The vowel in turn is assimilated to the stem vowel (compare also the closely related \ili{Chiwere} form \textit{ga'e} `that', where the vowel did not undergo this assimilation).


However, the \isi{distal} \isi{deictic stem} \textit{=ga} developed different functions in \ili{Hoocąk}: first, this \isi{demonstrative} became an \isi{enclitic} proper name marker that is used obligatorily with anthroponyms and with kinship terms, if they are used in third person reference function.  Secondly, this \isi{distal} \isi{demonstrative} became a clause or sentence connecting element expressing temporal continuation. And thirdly, this form is used as a \isi{distal} \isi{demonstrative}. The latter is certainly not the major function of this \isi{demonstrative}, there are only a handful instances of this usage in our \textsc{dobes} corpus. However, the \isi{distal} \isi{deictic stem} \textit{=ga} plays an important role in the formation of the new adnominal \isi{demonstrative}s with the \isi{positional} auxiliaries; see below \sectref{sec:helmbrecht:4}.


There is a second \isi{proximal} \isi{demonstrative} \textit{me'e} `this', which is mentioned in older sources \citep[52]{Lipkind1945} and occurs occasionally in our corpus. The origins of this /m/ initial form are unclear. This form cannot be traced back to one of the two Proto-\ili{Siouan} \isi{proximal} \isi{demonstrative}s on the basis of the known sound laws. 


\subsubsection{{Morphosyntactic and semantic properties of the Hoocąk forms}}\label{sec:helmbrecht:3.2.2}

The ``old'' \isi{demonstrative}s are used predominantly as \isi{demonstrative} pronouns or as adverbial \isi{demonstrative}s ('here', `there', `now', `then') in our texts. Sometimes they are also used as adnominal \isi{demonstrative}s, but these occurences are not frequent. 
If they function as adnominal \isi{demonstrative}s, they occur always postnominally or more specifically, at the right edge of the NP. This is probably the Common \ili{Siouan} order. The descendents of the Proto-\ili{Siouan} forms in the other \ili{Siouan} languages are all postnominally. However, the \isi{word order} rules with respect to the ``old'' \isi{demonstrative}s have become less strict. Although these instances are rare in our corpus, the forms can also occur prenominally.
If the \isi{demonstrative}s are used pronominally, they are used almost always endophorically, i.e they refer back anaphorically to a previously mentioned discourse participant, or they refer back to a whole proposition or episode of a narration; this is called here discource \isi{deictic} reference. Note that this kind of discourse \isi{deictic} reference is possible also with the \isi{demonstrative pronoun} \textit{'ee} `he/she/it/this/that etc.' 
The ``old'' \isi{demonstrative}s may also be used as identificational \isi{demonstrative}s in non-verbal or \isi{copula} clauses.
Probably the most frequent use of these \isi{demonstrative}s in our text corpus is their use as adverbial \isi{demonstrative}s that are better translated as `here' and `there' or `now' and `then'. In these uses, the adverbial \isi{demonstrative} refers to a previously mentioned situation or exophorically to the actual speech situation (\isi{proximal}).

To summarize the findings:
\begin{itemize}
\item
The major result of the grammaticalization of the ``new'' \isi{demonstrative}s is that there appeared a new paradigmatic distinction between \textbf{\isi{demonstrative} pronouns} and \textbf{adnominal demonstratives}. The ``old'' \isi{demonstrative}s lost their usage as adnominal determiners (not entirely, though). This function has been taken over by the ``new'' paradigm of \isi{demonstrative}s. 
\item
On the other hand, the ``old'' \isi{demonstrative}s are dominantly used pronominally in a variety of constructions and adverbially.
\item
In addition, it can be observed that the Proto-\ili{Siouan} threefold \isi{proximal}\slash medial\slash \isi{distal} distinction has been bleached or even neutralized in the ``old'' \ili{Hoocąk} \isi{demonstrative}s. There is no longer semantically a medial \isi{demonstrative}, and the \isi{distal form} \textit{ga'a} `that (\isi{distal})' has been lost entirely in this paradigm.
\end{itemize}

\section{The grammaticalization of the Hoocąk adnominal demonstratives}\label{sec:helmbrecht:4}

As has been shown in \tabref{tab:helmbrecht:11}, the adnominal \isi{demonstrative}s are historically a combination of the \isi{positional} auxiliaries plus two \isi{deictic} particles; \textit{=re} for \isi{proximal} and \textit{=ga} for \isi{distal} deixis. The different functions\slash disributions of both \isi{deictic} particles are summarized in \tabref{tab:helmbrecht:15}.  



Besides the occurence of these particles in combination with the \isi{positional} auxiliaries, they are still used independently; \textit{=re} (\textsc{dem}.\textsc{prox}) is quite frequent in our texts corpus, \textit{=ga} (\textsc{dem}.\textsc{dist}) rather rare. If they are used independently, they usually nominalize a clause in order to indicate subordination. Recall that nominalization is a major strategy to form subordinate clauses in \ili{Hoocąk}; cf. \REF{ex:helmbrecht:17} for an illustrative example.

\ea \label{ex:helmbrecht:17}
  MAP013\\
\glll jaagu waac `eeja hamįnągre paaxų nąga hegu `eeja waac `eeja nąąjįp nąga nįį `eeja waakįnįpšąną\\
    jaagu \textbf{waac}  \textbf{'eeja} \textbf{ha-mįįnąk=re}             paaxų      nąga  hegu       `eeja waac  `eeja  nąą<ha>jįp                           nąga  nįį  `eeja ho-ha-kįnįp=šąną\\
    what  \textbf{boat}   \textbf{there}  \textbf{\textsc{1e.a}}\textbf{-sit=}\textbf{\textsc{dem}.\textsc{prox}}   pour{\textbackslash}\textsc{1e.a}  and    that.way  there boat  there  <\textsc{1ea}>tilt.with.the.foot   and  water  there \textsc{appl}.\textsc{iness}-\textsc{1e.a}-fall.down=\textsc{decl}\\
\glt `whatever, I sat in the boat, I poured it out, and there I tipped over it (the boat), I fell in the water.' (lit. `whatever, sitting in the boat there, I poured it out, and ....')
\z

\begin{table}
\fittable{
\begin{tabular}{ll}
\lsptoprule
form & target(s)\\
\midrule
={re} (\isi{proximal})\footnotemark{} & \isi{proximal} adnominal \isi{demonstrative} (\isi{positional auxiliary} + ={re})\\
& nominalizer\slash subordinator\\
 & \isi{imperative} marker (IMP)\\
 & derivational means for time adverbials\\
\tablevspace
={ga} (\isi{distal}) & \isi{distal} adnominal \isi{demonstrative} (\isi{positional auxiliary} + ={ga})\\
& nominalizer\slash subordinator\\
 & sentence connector (\isi{continuative})\\
 & proper name marker\\
\lspbottomrule
\end{tabular}
}
\caption{Grammaticalization of the deictic particles/bound forms \textit{=re}  and \textit{=ga}}
\label{tab:helmbrecht:15}
\end{table}

\footnotetext{The historical source of \textit{=re} remains speculative. Perhaps it goes back to Proto-\ili{Siouan} *\textit{ree}. I am grateful to Rory Larson, who indicated to me this possibility. As far as we know, such a historical developement would not violate known \ili{Siouan} sound laws.} 

The \isi{clitic} \isi{deictic particle} \textit{=re} (\textsc{dem}.\textsc{prox}) indicates subordination of the entire clause (in bold face), which otherwise could not be distinguished from a \isi{main clause} with regard to its grammatical marking. Other determiners such as the \isi{definite article} and the ``new'' adnominal \isi{demonstrative}s occur in the same structural slot with the same function, namely indicating subordination. It seems quite likely to me that the grammaticalization of the ``new'' \isi{demonstrative}s was mediated by the subordinating function of these particles. The \isi{deictic} particles as subordinators always appear at the end of the \isi{subordinate clause}, and if this \isi{subordinate clause} contains a \isi{continuative} marking \isi{positional}, this \isi{positional} always appears immediatley before the deicitic particle. At one point in the history of \ili{Hoocąk}, the \isi{positional auxiliary} lost its person\slash number inflection for the subject\slash actor of the \isi{subordinate clause} and got fused with the nominalizing deicitc particle. Finally, this fused form extended its distribution and was generalized as a \isi{demonstrative} determiner that could also occur with plain nouns in a NP. The grammaticalization of the ``new'' adnominal \isi{demonstrative}s, therefore, may have come about in three principal steps, cf. \tabref{tab:helmbrecht:16}.


\begin{table}
\begin{tabularx}{\textwidth}{lQ}
\lsptoprule
Step 1 & relative clauses or subordinated clauses with a \isi{positional auxiliary} (\isi{continuative aspect} marking) are nominalized by \textit{=re}\slash \textit{=ga}\\
\tablevspace
Step 2 & the \isi{positional auxiliary} + \textit{=re}\slash \textit{=ga} are reanalyzed as a subordinating \isi{demonstrative}\\
\tablevspace
Step 3 & extension of the range of usages of the subordinating \isi{demonstrative}s, for instance as a determiner with a plain noun in a NP \\
\lspbottomrule
\end{tabularx}
\caption{Grammaticalization of classifying adnominal demonstratives in Hoocąk}
\label{tab:helmbrecht:16}
\end{table}
 

A construction that represents the developement from step 1 to step 2 in \tabref{tab:helmbrecht:16} could be the following text example:\newpage

 \ea \label{ex:helmbrecht:18}
MOV041\\
\gllll nįge paašihajawiga 'eeja                (hąho) žegu                                                                                           howé hiperes kįįjee(n) ~ ~\\
	nįge paašihajawiga 'eeja                hąho žegų                                                                                            howé hiperes kįįjeeną\\
	nįge \textbf{paaši=ha-jee-wi=ga} 'eeja  hąho žeegų                                                                                           howe hiperes kįį=jee=ną\\
	where \textbf{dance{\textbackslash}}\textbf{\textsc{1e.a}=\textsc{coll}-\textsc{pos}.\textsc{vert}-\textsc{pl}=\textsc{dem}.\textsc{dist}} there \textsc{intj} thus    go.about   know(\textsc{sbj}.\textsc{3sg}) make.self-\textsc{pos}.\textsc{vert}=\textsc{decl}\\
\glt `The place, \textbf{where we were dancing}, there he knows his way around.'
\z

The \isi{positional auxiliary} \textit{=jee} `be.standing' that marks \isi{continuative} in the \isi{subordinate clause} (given in bold face) is still inflected for the person\slash number of the subject\slash actor of the \isi{subordinate clause}. The \isi{distal} \textit{=ga} is a nominalizing form marking subordination; once this inflection disappears, \textit{=jee} and \textit{=ga} are ready to be reanalysed as a single form. As was shown above, the ``new'' \isi{demonstrative} retains the \isi{continuative} \isi{aspect marking function} (the \isi{auxiliary} had) and in relative clauses the classificatory function. It makes also sense to interpret this construction as the starting point for the grammaticalization of \textit{=ga} as a sentence connector indicating continuation.

\section{Grammaticalization of classificatory demonstratives in Mandan}\label{sec:helmbrecht:5}


A different way to create classificatory \isi{demonstrative}s can be found in \ili{Mandan}. As already mentioned, \ili{Mandan} has three \isi{positional} auxiliaries (cf. \tabref{tab:helmbrecht:7} above) that are used as full verbs in existential and \isi{locative} clauses, and that are used as auxiliaries expressing \isi{continuative aspect} when accompanied by a \isi{continuative marker}. In addition these positionals combine with two \isi{demonstrative} pronouns - \textit{d}{ɛ} `this' and \textit{ąt} `that' - in order to form classificatory \isi{demonstrative} pronouns; cf. the forms in \REF{ex:helmbrecht:19}. 

\ea \label{ex:helmbrecht:19}
\ili{Mandan} (\citealt[28f]{Kennard1936})\\
\textit{d}{ɛ}-{nąk}   \\
this-sitting
\glt `this one (be.sitting)'

\textit{d}{ɛ}-{hąk}  
\glt  `this one (be.standing)'

\textit{d}{ɛ}-{mąk}   
\glt `this one (be.lying)'

\textit{ą}{t}-{nąk}   \\
that-sitting
\glt `that one (be.sitting)'

\textit{ą}{t}-{hąk} 
\glt   `that one (be.standing)'

\textit{ąt}-{mąk}  
\glt  `that one (be.lying)'     
\z

Note that the \isi{proximal} \textit{d}{ɛ} `this' in Kennard is represented as \textit{re} `this' in more recent studies (cf. \citealt[42]{Mixco1997}). Of the two \isi{demonstrative}s \textit{d}{ɛ} `this'\slash \textit{ąt} `that', only \textit{d}{ɛ} `this' can be traced back to Proto-\ili{Siouan}. A similar combination of ``old'' \isi{demonstrative}s with the positionals as in \ili{Mandan} does not exist in \ili{Hoocąk} (I found only one example of this composition in the entire \ili{Hoocąk} corpus). What is also interesting is that the order of forms in \ili{Mandan} is different. The \isi{demonstrative} form precedes the \isi{positional auxiliary}. It is particular this property that suggests that a different scenario has to be assumed with respect to the grammaticalization of the classificatory \isi{demonstrative}s (with regard to posture) in \ili{Mandan}. This question needs more research.


Another interesting difference between \ili{Hoocąk} and \ili{Mandan} is that \ili{Mandan}, in addition, grammaticalized the \isi{positional} auxiliaries to adnominal classificatory \isi{demonstrative}s without any combination with \isi{deictic} particles. It is the plain forms of the \isi{positional} auxiliaries that are used as \isi{demonstrative}s in the examples in \REF{ex:helmbrecht:20} from \citeauthor{Kennard1936}. 


\ea  \label{ex:helmbrecht:20}
\ili{Mandan} (\citealt[28f]{Kennard1936})\\


  \textit{óti-hąk}     \\
  lodge-this.standing
\glt `this lodge'


  \textit{máta-mąk}  \\
  river-this.lying
\glt   `this river'


  \textit{hár}{ɛ}{-nąk}   \\
  cloud-this.sitting       
\glt  `this cloud'
\z

It is difficult to think of a gramaticalization process that reanalyzes `be' auxiliaries to \isi{proximal} \isi{demonstrative}s without any support from \isi{deictic} particles, and to the best of my knowledge, such a process never has been described in the literature. This process is attested, however, only for \isi{proximal} deixis. For \isi{distal} deixis, the \textit{ąt} `that' \isi{demonstrative} has to be used. The \isi{positional} auxiliaries do not occur in this function.

\section{Omaha-Ponca (Dhegiha) made it differently}\label{sec:helmbrecht:6}

The grammaticalization of the \isi{positional} verbs\slash auxliaries in the \ili{Dhegiha} subgroup of \ili{Siouan} is remarkably different from that of the other \ili{Siouan} languages and has been extensively investigated by several authors: cf. \citealt{Rankin1977}; \citealt{Barron1982}; \citealt{Rankin2004a}; \citealt{Eschenberg2005}. This section strongly builds on the results of their research. I won't summarize these results \textit{in toto} here for lack of space. Instead, I will select some of the grammaticalizations involving the Proto-\ili{Siouan} positionals, classificatory \isi{demonstrative}s, and \isi{continuative aspect} marking auxiliaries in this sub-branch of \ili{Siouan}, in order to contrast them with \ili{Hoocąk}. The following grammaticalizations of positionals and definite articles in \ili{Omaha-Ponca} (OP) will be presented:\newpage

\begin{itemize}
\item
from positionals to classificatory definite articles (\sectref{sec:helmbrecht:6.1});
\item
from classificatory definite articles to classificatory \isi{demonstrative}s (\sectref{sec:helmbrecht:6.2}); and
\item
from classificatory definite articles\slash copulas to \isi{continuative} marking auxiliaries (\sectref{sec:helmbrecht:6.3}).
\end{itemize}


It will be shown, in particular, that the Proto-\ili{Siouan} positionals developed very differently in OP and the other \ili{Dhegiha} languages compared to what has been discussed so far with regard to \ili{Hoocąk} and some non-\ili{Dhegiha} \ili{Siouan} languages. 

\subsection{{From positional verbs/auxiliaries} {to classificatory definite articles}}\label{sec:helmbrecht:6.1}

All \ili{Dhegiha} languages have developed remarkable paradigms of up to ten definite articles that - among other things - classify their nouns according to semantic features such as: animate vs. inanimate, agent, vs. non-agent, moving, standing, sitting, horizontal, vertical, round, scattered and singular and plural, cf. the summary in \figref{fig:helmbrecht:1} below.


\begin{figure}
\caption{System of definite articles in Omaha-Ponca (cf. \citealt[144]{Koontz1984})}
\label{fig:helmbrecht:1}
%%%\includegraphics[width=\textwidth]{figures/Helmbrecht-tree.png}
\resizebox{\textwidth}{!}{\begin{forest}
	[+definite [+animate,edge={-{Stealth[]}} [+agent [-plural,name=minusplural [-motion [\textit{akʰa}] ] ] [+plural [+motion, name=plusmotion [\textit{ama}] ] ] ] [-agent [moving [\bfseries\textit{thi\textsuperscript{n}}] ] [standing [\bfseries\textit{tʰo\textsuperscript{n}}] ] [sitting [\bfseries\textit{thi\textsuperscript{n}kʰe}] ] [plural [\textit{ma}] ] ] ] [-animate,edge={-{Stealth[]}} [horizontal [\itshape kʰe]] [vertical [\bfseries\itshape t\textsuperscript{n}e]] [round [\bfseries\itshape tho\textsuperscript{n}]] [scattered [\itshape ge]] ] ]
	\draw (minusplural) -- (plusmotion);
\end{forest}}
\end{figure}
       
Some of these definite articles are descendants of the Proto-\ili{Siouan} posture verbs; they are marked bold in \figref{fig:helmbrecht:1}; the others have presumably a different origin or their etymology is unclear (cf. \citealt[209]{Rankin2004a}; \citealt[181ff]{Eschenberg2005}).
All classificatory definite articles in OP (cf. \figref{fig:helmbrecht:1}) are multifunctional and occur in different constructions; for textual evidence and discussion, see \citet[112--176]{Eschenberg2005}:

\begin{itemize}
\item
they can be used as \textbf{relativizer}, some of them also as \textbf{general subordinator} (with a \isi{locative} element attached to it);
\item
almost all classificatory definite articles in OP (except \textit{ge} +def/-anim\slash scat\-tered (\textsc{pl})) can be used as \textbf{auxiliaries indicating progressive aspect};
\item
all definte articles occur as \textbf{copulas in existential and locatives clauses} of the type ``it/this is.standing a dog'';
\item
some of the definte articles in OP are used as sentence-final \textbf{evidential markers} indicating either direct evidence of the event by the speaker, \isi{inferential} evidence, or hearsay;
\item
a few of the classifying definite articles can be used as \textbf{clause linking devices} (this holds for \textit{t}{ʰ}{e} +def./-animate\slash vertical and \textit{ge} +def./-animate\slash scattered).
\end{itemize}

The question arises: how do \isi{positional} auxiliaries become definite articles? \citet[182--206]{Eschenberg2005} argues that the \isi{positional} auxiliaries were reanalyzed as definite articles via their function as copulas in \isi{locative}\slash existential clauses. The definite articles in OP (all the forms in \figref{fig:helmbrecht:1}) -- no matter whether they come from Proto-\ili{Siouan} positionals, or not – are all used as \isi{copula} verbs in \isi{locative} or existential clauses; cf. the illustrating examples in \REF{ex:helmbrecht:21} and \REF{ex:helmbrecht:22}.

 \ea \label{ex:helmbrecht:21}
\ili{Omaha-Ponca} \\
\gll Tizhebe \textbf{tʰe.}                            Uthido\textsuperscript{n}.\\
    Door     be.located.vertical    \textsc{3sg}.lock\\
\glt    ‘The door is positioned vertically. It is locked.’ \citep[189]{Eschenberg2005}
\z

\ea \label{ex:helmbrecht:22}
\gll Tizhebe-\textbf{tʰe}                    uthido\textsuperscript{n}.\\
    Door-\textsc{def}.\textsc{vert}      \textsc{3sg}.lock\\
\glt    ‘The door is locked.’ \citep[189]{Eschenberg2005}
\z

In \REF{ex:helmbrecht:21}, the form \textit{tʰe} `be.located.vertical' is used as a \isi{copula} in a clause of location\slash existence. If the second \isi{predication} ``It is locked'' becomes pragmatically more important than the existence of the ``door'', the \isi{copula} may be reinterpreted as a determiner, which simply presupposes the existence of the ``door''; this is illustrated in \REF{ex:helmbrecht:22}. No matter whether the historical scenario proposed by Eschenberg in her dissertation is correct or not, there is no evidence so far that the classificatory definite articles in OP developped from \isi{demonstrative}s (what one would expect). To the contrary, it is the classifying definite articles that became eventually classificatory \isi{demonstrative}s as will be briefly shown in the next section. 

\subsection{{From classificatory definite articles to classificatory demonstratives}}\label{sec:helmbrecht:6.2}

OP has preserved the ``old'' paradigm of Proto-\ili{Siouan} \isi{demonstrative}s; see \tabref{tab:helmbrecht:13} above and \tabref{tab:helmbrecht:17}.


These \isi{demonstrative}s are used variously as \textbf{\isi{demonstrative} pronouns} and as \textbf{determiners}. In both usages they \textbf{co-occurred with the definite articles}, which led to the merging of the \isi{demonstrative} stems plus a classificatory article. The results are grammaticalized \textbf{classificatory demonstratives} (cf. \citealt[215]{Rankin2004b}).

\ea  \label{ex:helmbrecht:23}
\ili{Omaha-Ponca} (\citealt[26/27]{Dorsey1890})\\
\gll  Éga\textsuperscript{n} ðisan{\textasciitilde}´ga    méga\textsuperscript{n} \textbf{še-kʰe} ðizáiga  hă,  á-biamá.\\
    so      your.younger.brother likewise  \textbf{that-def.inanimate.lying}    take.ye  ?     said.he-they say\\
\glt     ` ``Do you and your younger brother take that?'' he said.'
    \z


\begin{table}
\begin{tabularx}{\textwidth}{lllX}
\lsptoprule
\multicolumn{2}{l}{\isi{demonstrative} pronouns\slash determiners} & \multicolumn{2}{l}{\isi{demonstrative} adverbs}\\
\midrule
{ð}{e} & `this' (close to speaker) & \textit{tu} & here\\
{še} & `that' (close to hearer), & \textit{šu} & there\\
{ka} & `that' (remote, out of sight) & \textit{ku} & yonder\\
\lspbottomrule
\end{tabularx} 
\caption{Omaha-Ponca ``old'' demonstratives}
\label{tab:helmbrecht:17}
\end{table}

In \REF{ex:helmbrecht:23}, the \isi{demonstrative pronoun} \textit{še} `that (close to hearer)' is combined with the inanimate definite classificatory article \textit{k}{ʰ}{e} forming a classificatory \isi{demonstrative}. This combination is phonologically one word. Other examples in the text collection of J. O. \citet{Dorsey1890} illustrate that these classificatory \isi{demonstrative}s may be formed with other definite classificatory articles (sitting\slash lying\slash standing) as well, and that they also may occur as determiners (see also \citealt[215]{Rankin2004b}; \citealt[101f]{Eschenberg2005}). The formation of classificatory \isi{demonstrative}s in OP resembles closely the forms in \ili{Mandan}, see \sectref{sec:helmbrecht:5} above. The difference, however, is that in OP the \isi{deictic} stems are combined with definite articles, while in \ili{Mandan} these stems are combined with \isi{positional} auxiliaries. The order of elements in the classificatory \isi{demonstrative}s in OP is also different to the order of elements in \ili{Hoocąk}, where the \isi{deictic particle} follows the \isi{positional auxiliary}.

\subsection{{From classificatory definite articles to continuative aspect marking auxiliaries}}\label{sec:helmbrecht:6.3}

The last \isi{grammaticalization process} in OP with regard to the \isi{positional} verbs and classificatory definite articles that will be presented here is the development of auxiliaries that indicate \isi{continuative aspect}. The \ili{Dhegiha} languages had lost the Proto-\ili{Siouan} positionals that were used as aspect marking auxiliaries in other \ili{Siouan} languages. Instead, the Proto-\ili{Siouan} positionals developed into classificatory definite articles. The next step in the historical process is that these classificatory definite articles developed into \isi{continuative} marking auxiliaries. This can be concluded from the fact that the classificatory definte articles received a new verbal \isi{conjugation} that differs from the inherited \isi{conjugation} of the positionals. 
It is not fully clear how this process came about. Rankin argues that it was the Proto-\ili{Siouan} sitting \isi{positional} *\textit{r\k{i}k} that inherited the \isi{aspect marking function} in OP. This \isi{positional} developed into different definite articles on the one hand, but was also combined with a bound verb \textit{-he} `to be in a place', which in turn could be conjugated. Later on, the other classificatory definite articles were analogically conjugated according to this model. 
Eschenberg contests this view. According to her, the starting point for the grammaticalization of the definite articles to \isi{continuative} auxiliaries is their use as locational copulas; compare the scenario in \REF{ex:helmbrecht:24} and \REF{ex:helmbrecht:25}.

\ea \label{ex:helmbrecht:24}
\ili{Omaha-Ponca} \citep[190]{Eschenberg2005}\\
\gll {\ob}{\ob}Tizhebe uthido\textsuperscript{n}{\cb}   \textbf{tʰe}{\cb}.\\
      Door      \textsc{3sg}.lock \textbf{be.located.vertical}\\
\glt   `The locked door is vertical.’
\z

\ea \label{ex:helmbrecht:25}
\gll {\ob}{\ob}Tizhebe{\cb} {\ob}uthido\textsuperscript{n}-\textbf{tʰe}{\cb}{\cb}.\\
       Door        \textsc{3sg}.lock-\textbf{be.located.vertical}\\
\glt    `The door is being locked.’
\z
    
In \REF{ex:helmbrecht:24}, the \isi{copula} modifies the whole NP ``the locked door''. This clause represents the original usage of the \isi{definite article} \textbf{{t}}\textbf{{ʰ}}\textbf{{e}} `the.inanimate.vertical', while in \REF{ex:helmbrecht:25}, the same form is used as an \isi{auxiliary} and modifying only the preceding \isi{predicate}. 
No matter, which hypothesis is correct, it remains that the \isi{continuative} \isi{auxiliary} in OP developed from the definite articles perhaps via their use as copulas and NOT from the old Proto-\ili{Siouan} \isi{positional} auxiliaries directly that were lost in \ili{Dhegiha} eventually. 

\section{Conclusions}\label{sec:helmbrecht:7}

It has been shown that the Proto-\ili{Siouan} posture verbs (``positionals'' in Siouanist terminology) grammaticalized to aspect marking auxiliaries in many \ili{Siouan} languages. As auxiliaries, they underwent different degrees of \isi{coalescence} with the full verb up to the point of being an uninflected \isi{continuative aspect} marker in \ili{Lakota}. These aspect marking auxiliaries were ultimately lost in the \ili{Dhegiha} languages. 
In \ili{Hoocąk}, the \isi{positional} auxiliaries grammaticalized to adnominal \isi{demonstrative}s via a subordinating construction and the \isi{coalescence} with a \isi{deictic particle}. The ``new'' adnominal \isi{demonstrative}s preserved a classificatory function; cf. \REF{ex:helmbrecht:26}. 

\ea \label{ex:helmbrecht:26}
Grammaticalization of positionals to \isi{demonstrative}s in \ili{Hoocąk}\\\vspace*{.5\baselineskip}
\begin{tabular}{@{ }l@{ }l@{ }l@{ }l@{ }l@{ }l@{ }l}
{posture verb} & {→} & {positional}     & {→} & {classificatory} & → & nominalizer\\
			   &	 & \isi{auxiliary}\slash	&     &	adnominal & \\\cline{7-7}
			   &     & \isi{continuative}     &     & \isi{demonstrative} & & subordinator\\
			   &     &  aspect marker   &     &               & & \\\cline{7-7} 
		  	   &	 &					&	  &				  &	&relativizer\\ 
\end{tabular}
\z

A different \isi{grammaticalization path} towards \isi{demonstrative}s was found in \ili{Mandan}. The \isi{positional} auxiliaries combine with one of the ``old'' Proto-\ili{Siouan} \isi{demonstrative pronoun} to form a ``new'' classsificatory \isi{demonstrative pronoun}. The other path is the reanalysis of the \isi{positional auxiliary} as an adnominal \isi{demonstrative}; cf. \REF{ex:helmbrecht:27}.


\ea \label{ex:helmbrecht:27}
Grammaticalization of \isi{positional} to \isi{demonstrative}s in \ili{Mandan}\\\vspace*{.5\baselineskip}
\begin{tabular}{@{ }l@{ }l@{ }l@{ }l@{ }l}
{posture verbs} & {→} & \isi{positional auxiliary}\slash & → & classificatory \isi{demonstrative}\\
			    &     & \isi{continuative aspect} & & pronoun of the type\\
			    &     & marker				& & \textit{dɛ-nąk} `this one (be.sitting)'\\
				&	  &						& → & classificatory \isi{demonstrative}\\
				&	  &						&  & determiner of the type\\															
				&     &						&  &  \textit{óti-hąk} `this lodge'\\
\end{tabular}
\z

Even more different are the grammaticalization paths found in \ili{Omaha-Ponca} representing the \ili{Dhegiha} sub-branch of \ili{Siouan}. Here the \isi{positional} auxiliaries grammaticalized to classificatory definite articles perhaps via their use as copulas in existential\slash \isi{locative} clauses. No evidence was presented in the literature that this process was preceeded by a stage in which the positionals were \isi{demonstrative}s. This is remarkable since it is commonly held that \isi{demonstrative}s become definite articles and not vice versa. Instead, classificatory \isi{demonstrative}s were grammticalized in OP from the definite articles by combining them with the ``old'' Proto-\ili{Siouan} \isi{demonstrative}s. In addition, the classificatory definite articles developed into \isi{continuative} marking auxiliaries; cf. \REF{ex:helmbrecht:28}.  

\ea \label{ex:helmbrecht:28}
Grammaticalization of positionals to \isi{demonstrative}s in \ili{Omaha-Ponca}\\\vspace*{.5\baselineskip}
\begin{tabular}{@{ }l@{ }l@{ }l@{ }l@{ }l@{ }l@{ }l}
{posture verb} & {→} & {positional} & {→} & classificatory defi- & → & classificatory \\
			   &     & \isi{auxiliary}    &     & nite article   		 &   &    \isi{demonstrative}     \\
			   &	 &				&     & 				     & → & \isi{continuative}\\
			   &     &				&	  &						 &   & \isi{auxiliary} 
\end{tabular}
\z \largerpage[2]

\section*{Abbreviations}
\begin{tabularx}{.5\textwidth}{@{}>{\scshape}lQ} 
1, 2, 3 & first, second, third person\\
a &   actor\\ 
acc &   accusative\\
appl.iness & inessive {applicative} prefix\\ 
coll &   collective marker\\ 
cont &  {continuative}\\ 
decl &   declarative\\ 
def &   definite\\ 
dem &   {demonstrative}\\ 
dist &   {distal}\\ 
du &   dual\\ 
e &   exclusive\\ 
emph &    emphatic\\ 
fut &   future\\ 
i &   inclusive\\ 
ind.male & indicative with male address\\ 
mv &   middle voice\\ 
n &   noun\\ 
narr.past & narrative {past tense}\\ 
\end{tabularx}
\begin{tabularx}{.5\textwidth}{>{\scshape}lQ@{}} 
nom &   {nominative}\\ 
pl &   plural\\ 
pos.hor & `be (lying\slash horizontal position)'\\ 
pos.ntl & `be (sitting\slash neutral position)'\\ 
pos.vert & `be (standing\slash vertical position)'\\ 
poss.refl & {possessive} reflexive\\ 
pro &   pronoun\\ 
prop &   proper name marker\\ 
prox &   {proximal}\\ 
q &   interrogative particle\\ 
quot &   quotative\\
sbj &   subject\\ 
seq &   sequential\\ 
sg &   singular\\ 
sim/loc & simultaneous/{locative}\\
ss &   same subject marker\\ 
sv &   stem vowel\\ 
u &   undergoer\\ 
v &   verb\\ 
\end{tabularx}  
 
{\sloppy\printbibliography[heading=subbibliography,notkeyword=this]}
\end{document}