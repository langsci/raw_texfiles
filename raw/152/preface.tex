\documentclass[output=paper]{LSP/langsci} 
\title{Preface}
\author{Walter Bisang \lastand Andrej Malchukov\affiliation{Johannes Gutenberg-Universität Mainz}}
\abstract{\vspace*{-2\baselineskip}}
% \ChapterDOI{}

\makeatletter
\renewcommand{\lsCollectionPaperTitle}{{%
  \renewcommand{\newlineTOC}{}
  \renewcommand{\newlineCover}{\\} 
  \\[-1\baselineskip]
% \vspace*{-2\baselineskip}
  \noindent{\LARGE ~}\\
  \bigskip  
  \noindent\@title}}
  
\renewcommand{\lsCollectionPaperTOC}{{%
  \iflsCollectionChapter%
    \protect\numberline{~}\fi
  \@title\newline{\normalfont\@author}}}
 \makeatother
% \setcitation{Klamer, Marian}{Preface}{1--3}
\maketitle

\rohead{}
\ChapterDOI{10.5281/zenodo.1044383}

\begin{document}
% \addchap{Preface}
%  {Walter Bisang \& Andrej Malchukov (Johannes Gutenberg-Universität Mainz) }
% \begin{refsection}
\noindent The present volume originated from the symposium on “Areal patterns of grammaticalization and cross-linguistic variation in grammaticalization scenarios”\linebreak held on 12--14 March 2015 at Johannes Gutenberg-Universität Mainz. The main purpose of the conference was to bring together leading experts on grammaticalization, combining expertise in \isi{grammaticalization theory} with expertise in particular language families, in order to explore cross-linguistic variation in grammaticalization scenarios. The participants together with the organizers of the conference (Walter Bisang \& Andrej Malchukov) aim at a systematic study of grammaticalization scenarios as well as research on their \isi{areal variation}, all of this leading to a planned Comparative Handbook of Grammaticalization Scenarios and an accompanying database. Additionally, certain papers which address some of the main questions raised by the organizers of the conference have been invited to the present volume.

Grammaticalization studies and \isi{grammaticalization theory} have been one of the most successful research paradigms introduced in late 20\textsuperscript{th} century linguistics. The milestone of grammaticalization research includes such works as \citet{Lehmann2015Thoughts} on “Thoughts on Grammaticalization”, \citet{HeineEtAl1991} on “Grammaticalization: A Conceptual Framework”, \citet{BybeeEtAl1994} on “The Evolution of Grammar: Tense, Aspect and Modality in the Languages of the World”, \citet{Heine2002} on “World Lexicon of Grammaticalization” and \citet{Hopper2003} on “Grammaticalization”, to name just a few. Even critiques of \isi{grammaticalization theory} (see e.g., \citealt{Newmeyer1998},  \citealt{Campbell2001}; also see \citealt{Lehmann2004Theory} for a critical response) did not stop this research, which numbers in thousands of publications (see the monumental “The Oxford Handbook of Grammaticalization” by \citealt{Narrog2011} for the state of the art in research on grammaticalization). 

Yet, in spite of its obvious successes, some aspects remain controversial and are in need of further study. One aspect concerns \isi{areal variation} in grammaticalization scenarios. Contrary to the alleged universality of grammaticalization processes and paths, grammaticalization shows \isi{areal variation}, as was most emphatically pointed out by Bisang with particularly telling examples from Southeast Asian languages (\citealt{Bisang1996,Bisang2004tam,Bisang2011,Bisang2015}; also see  \citealt{Ansaldo2004}). Even though there are many grammaticalization paths in these languages, most of them characteristically diverge from such processes by the absence of the co-evolution of meaning and form as it is generally taken for granted in the literature. Thus, the semantic development of a lexical item into a marker of a grammatical category (e.g., verbs meaning ‘give’ > \isi{benefactive} markers) is not necessarily accompanied by phonetic reduction and morphologization (there are phonological properties that operate against the development of bound forms, see \citealt{Ansaldo2004}). This lack of form-meaning coevolution in grammaticalization processes in Southeast Asian languages is just one manifestation of \isi{areal variation} in grammaticalization scenarios which has been underestimated in the literature. Another one is the higher relevance of pragmatic inference as it is manifested in the lack of obligatoriness and in the multifunctionality of grammaticalized markers. Second, the universality of grammaticalization processes has yet to be reconciled with a wide-spread belief that these processes are construction-specific. Given that the constructions in question are language-specific, it is an open question how one should account for cross-linguistic patterns of grammaticalization. While the construction-specific nature of grammaticalization has long been acknowledged in the literature \citep{BybeeEtAl1994}, this aspect came to the fore with the advent of Construction Grammar approaches to grammaticalization (\citealt{GisbornePatten2011}, \citealt{Traugott2013}). Both aspects noted above raise the issue of how to reconcile universal and language-particular aspects of grammaticalization phenomena. The contributions to this volume address this issue in one way or another.

Perhaps the paper by \textbf{Bernd Heine, Tania Kuteva} and \textbf{Heiko Narrog} on “Back again to the future: How to account for \isi{directionality} in grammatical change?\textit{”} addresses this question heads on. Drawing on material from \ili{Khoisan} languages but also on comparative data from \ili{Germanic}, the authors trace the development of future markers. They note that though originally we are dealing with different source constructions including motion verbs, all of them result in a \isi{future meaning}. The answer which the authors give to the puzzle stated above, is that constructional details within or across languages do not preclude universality. They suggest that universality should be formulated in functional (semantic, cognitive) terms as a semantic relation between the source and the target concepts (here the relation between directed spatial movement and the meaning of future). This is a very interesting solution to the problem, even if it is formulated in a rather absolute manner. After all, it is clear that in other cases constructional details would matter, as in the case of ‘give’-verbs that develop into \isi{benefactive} markers in constructions with a verbal host or into a \isi{dative} marker in constructions with a nominal host. It remains to be seen if the notion of ‘host’ (from \citealt{Himmelmann2004}) is sufficient to explain all the divergent paths of grammaticalization. Another solution to the puzzle, which is not at variance with the solution suggested in this paper, is that the constructional details are obliterated as grammaticalization proceeds — a process that instigates convergence between different paths.

The chapter by \textbf{Guillaume Jacques} on “The origin of \isi{comitative} adverbs in \ili{Japhug}” studies an interesting scenario in a \ili{Tibetan} language where a \isi{proprietive} denominal form develops into a \isi{comitative} marker. The path where a denominal \isi{proprietive} verb (with a meaning ‘having N’) in its nonfinite form is reanalyzed as a \isi{comitative} form of a noun (that is: ‘one having branches’ > ‘with branches’) has not been specifically recorded in the literature even though the development from a \isi{possessive} to a \isi{comitative} function at a more general level is well documented (for example, for serial verbs). This then provides an additional example of the importance of functional aspects for the explanation of the commonalities of grammaticalization paths, as suggested by the above paper of Heine et al.

\textbf{Denis Creissels’s} paper on “Copulas originating from ‘see/look’ verbs in Man\-de languages” proposes a new \isi{grammaticalization path} involving the routinization of an \isi{ostensive} use of the \isi{imperative} of ‘see’ or ‘look’. This pathway is not documented in the literature  \citep{Heine2002}. Creissels documents this scenario across \ili{Mande} languages and additionally notes some parallel phenomena in Arabic varieties and in French. French \textit{voici}/\textit{voilà} constitutes a well-known example of how the \isi{imperative} of verbs with the meaning of ‘see’ is grammaticalized into an \isi{ostensive} \isi{predicator}. In some Arabic varieties the development seems to be mediated through the stage of a \isi{modal}/discourse particle: The \isi{grammaticalization path} SEE/LOOK (\isi{imperative}) > MODAL/DISCURSIVE PARTICLE > COPULA is unusual since it goes partly against intersubjectification as one would expect in the development from \isi{copula} to discourse particle. 

\textbf{Larry Hyman} in his paper on “Multiple argument marking in \ili{Bantoid}: From syntheticity to analyticity” shows how to account for the adoption of alternative grammaticalization strategies when a language develops from high syntheticity (agglutination) towards analyticity. The challenge is how to account for the pathway from the inherited head-marking verb structures of Proto-\ili{Bantoid} to the more analytical structures found in many of the daughter languages. After a careful examination of the data involving valency-changing morphology (“valency extensions”) of more conservative \ili{Bantu} languages and their analytic counterparts in more innovative \ili{Bantoid} languages, the author raises the question of an ultimate explanation for the move to more analytic structures. Since conventional scenarios, appealing to “erosion” as a byproduct of natural sound change, or else to \isi{language contact} (in the line of McWhorter’s pidginization scenario) seem to be inapplicable here, Hyman suggests that morphology was lost as a result of maximal-size “templatic” constraints on stems. The idea that the shift to analyticity is due to constraints on the number of syllables is highly interesting, but it also raises the question of the factor that ultimately conditioned the templatic constraints. More generally it shows that little is known about the paths of attrition (phonological reduction), no matter whether it is due to erosion or to templatic constraints.

\textbf{Annie Montaut} in her paper on “Grammaticalization of \isi{participle}s and gerunds in \ili{Indo-Aryan}: Preterite, future, \isi{infinitive}” discusses developments of non-finite forms to finite markers in \ili{Indo-Aryan} languages. One such path from passive past participle\is{past passive participle} to \isi{past tense} is well-known, as it is famously responsible for the rise of ergativity. The author however notes that similar developments are documented in the evolution of the passive future/\isi{obligative} \isi{participle} and the \isi{infinitive} into \isi{future tense} markers in different branches of \ili{Indo-Aryan}. Interestingly, the latter developments have not resulted in ergativity. One of the factors accounting for this difference is competition with other forms. As the author shows the resilience of the old future in Indo Aryan languages inhibited the development of gerunds into future markers in some languages of the Western branch. Another factor is analogical influence from other patterns, among them the responsibility of \isi{dative} subject sentences for the realignment of the \isi{gerund} construction in Western \ili{Indo-Aryan}. Thus, competition with other forms and analogical influence can go a long way in explaining variation in grammaticalization paths as well as the alignment of individual verbal forms. This issue is also highlighted by a comparison of \ili{Indo-Aryan} with \ili{Romance}. From such a broader perspective, one cannot exclude the existence of general functional constraints in this domain as well (cf. \citealt{Malchukov2011} on the TAM-hierarchy for ergativity splits). 

\largerpage\textbf{Christian Lehmann}, one of the founding fathers of grammaticalization research, discusses the topic of “Grammaticalization of \isi{tense}/aspect/mood marking in \ili{Yucatec} \ili{Mayan}”. He shows that the formation of preverbal TAM markers is due to the convergence of different constructions, including adverbial modification, complementation based on aspectual or \isi{modal} verbs, the motion cum purpose construction and the verb-\isi{focus construction}. Yet, to cite the author, “although the four constructions are clearly distinct, they share a \isi{clause-initial position} which becomes the melting-pot for the aspectual and \isi{modal} \isi{formatives} recruited from different sources”. The author’s notion of a “melting pot” seems similar to the concept of “attractor position” in the approach of \citet{Bisang1992}, even though the latter term has been applied to the typologically rather different languages of East and mainland Southeast Asia. More generally, Lehmann’s (and Bisang’s) scenario is again in line with the hypothesis that formal reduction is the ultimate explanation of convergence in grammaticalization paths.

\textbf{Johannes Helmbrecht} discusses the grammaticalization of \isi{demonstrative}s in \ili{Hoocąk} and other \ili{Siouan} languages.\textbf{} As noted by Helmbrecht, while the evolution of \isi{demonstrative}s into anaphoric pronouns and finally to third person pronouns is well documented, the origin of \isi{demonstrative}s themselves is not well studied. On the basis of comparative \ili{Siouan} data Helmbrecht shows that the two bound \isi{deictic} forms -\textit{re} and -\textit{ga} are systematically combined with the three \isi{positional} verbs \textit{nąk} ‘sit', \textit{ąk} ‘lie' and \textit{jee} ‘stand' in order to form a new paradigm of \isi{demonstrative}s. The verbal origin of these new \isi{demonstrative} markers can explain why they classify the head noun according to its spatial position (neutral, horizontal, vertical). Other \ili{Siouan} languages show variation on this theme, but they all have a classificatory \isi{demonstrative} as an output structure even though the source constructions involving a \isi{positional} verb are not identical. This situation provides again good evidence for convergent paths. 

The final paper by \textbf{Björn Wiemer \& Ilja Seržant} on “Diachrony and typology of \ili{Slavic} aspect: What does morphology tell us?”\textbf{} discusses the evolution of aspect in \ili{Slavic} languages. It is a paper which combines typological and historical approaches trying to trace the origin of the \ili{Slavic} \isi{aspectual system} and explain why similar developments have not been attested in other European languages. As a tentative explanation for the renewal of the \isi{perfective}/\isi{imperfective} opposition, which in a way continues an older distinction between \isi{aorist} and imperfect in Proto-Indo-European, the authors implicate the substrate influence from \ili{Uralic}/Altaic in \ili{Slavic}. While this explanation is tentative it gains credibility, since similar areal explanations have been proposed for other grammatical subsystems. Thus, the preservation of a rich case system in \ili{Slavic} has been attributed to an \ili{Uralic}/Altaic substrate (see \citealt{Kulikov2009} for discussion and references). 

\largerpage All the papers presented in this volume provide valuable contributions to the documentation of grammaticalization paths. The authors propose novel grammaticalizations paths not reported in the literature (Helmbrecht, Lehmann, Jacques, Creissels), they offer explanations for the universality and the parametrization of grammaticalization scenarios (Heine et al. from a more general theoretical perspective based on data on the emergence of future markers, Lehmann on TAM marking in \ili{Yucatec} \ili{Maya}, and Montaut on alignment systems in \ili{Indo-Aryan}), they provide in-depth analyses of neglected aspects of grammaticalization (Hyman on paths of phonetic attrition), and they explore the role of areal factors and \isi{language contact} as an explanatory factor of grammaticalization processes (Wiemer \& Seržant).

As far as the question of resolving the tension between the construction-spe\-ci\-fic nature of grammaticalization and the universality of its paths is concerned, there are two answers emerging. One answer, clearly articulated in the article by Heine et al., proposes that universal paths should be formulated in functional\slash conceptual terms, while the details of the input constructions differ. In addition, several contributions point out that constructional differences become partially opaque in the processes of reduction associated with grammaticalization (most clearly illustrated by Lehmann and Helmbrecht). Hence, we would like to suggest that the convergent trajectory of grammaticalization paths can be partially explained by form-related grammaticalization processes of reduction which blur distinctive properties of individual constructions. We see this as a promising perspective to reconcile the differences between the universal approach and the perspective of Construction Grammar. Future research will show the relative impact of these two explanatory factors for different grammaticalization scenarios, but it is expected that both play a role in later, more systematic explanations.\\

\noindent Walter Bisang \& Andrej Malchukov\hfill Mainz, March 2016

\section*{Acknowledgements}

We would like to thank our authors for their contributions and for their useful discussions in the course of our internal reviewing. We are also very grateful to Larry Hyman for his comments on this introduction. Moreover, we would like to thank Iris Rieder and Linlin Sun for the careful copy-editing of the manuscripts. Finally, we would like to express our gratitude to the \ili{German} Research Foundation for its financial support of the project “Cross-linguistic variation in grammaticalization processes and areal patterns of grammaticalization” (Bi 591/12-1).


{\sloppy
\printbibliography[heading=subbibliography,notkeyword=this]
}
% \end{refsection}

\end{document}