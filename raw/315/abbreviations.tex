\addchap{\lsAbbreviationsTitle}
% \addchap{Abbreviations and symbols}

\section*{Glossing}
This book follows the Leipzig Glossing Rules and employs its standard abbreviations.\footnote{\url{https://www.eva.mpg.de/lingua/resources/glossing-rules.php} (Accessed in April 2021)} Other abbreviations employed in this book but not defined by the Leipzig Glossing Rules are listed below.

\bigskip

\begin{tabularx}{.45\textwidth}{lQ}
	\textsc{ade} & Adessive \\
	\textsc{anim} & Animate \\
	\textsc{antc} & Anticausative \\
	\textsc{antp} & Antipassive \\
	\textsc{aor} & Aorist \\
	\textsc{aug} & Augmented \\
	\textsc{asp} & Aspect \\
	\textsc{assoc} & Associative \\
	\textsc{cmpv} & Completive \\
	\textsc{ctr} & Contrastive \\
	\textsc{conj} & Conjunctive \\
	\textsc{conn} & Connector \\
	\textsc{cont} & Continuative \\
	\textsc{dim} & Diminutive \\
	\textsc{disc} & Discursive \\
	\textsc{ep} & Epenthetic \\
	\textsc{evid} & Evidential \\
	\textsc{fin} & Finite \\
	\textsc{frust} & Frustrative \\
	\textsc{inch} & Inchoative \\
	\textsc{int} & Intentional \\
\end{tabularx}
\begin{tabularx}{.45\textwidth}{lQ}
	\textsc{inv} & Inverse \\
	\textsc{link} & Linker \\
	\textsc{med} & Medial \\
	\textsc{mid} & Middle \\
	\textsc{min} & Minimal \\
	\textsc{mir} & Mirative \\
	\textsc{mod} & Modal \\
	\textsc{name} & Personal name \\
	\textsc{nlocut} & Non-locutor \\
	\textsc{nsit} & New situation \\
	\textsc{opt} & Optative \\
	\textsc{part} & Partitive \\
	\textsc{pred} & Predicate \\
	\textsc{prop} & Proprietive \\
	\textsc{quot} & Quotative \\
	\textsc{real} & Realis \\
	\textsc{rdpl} & Reduplication \\
	\textsc{restr} & Restrictor \\
	\textsc{subord} & Subordinate \\
	\textsc{th} & Thematic \\
	\textsc{und} & Undergoer \\
\end{tabularx}

\newpage

\section*{Macroareas}
\begin{tabularx}{.30\textwidth}{lQ}
	\textsc{af} & Africa \\
	\textsc{ea} & Eurasia \\
\end{tabularx}
\begin{tabularx}{.30\textwidth}{lQ}
	\textsc{pn} & Papunesia \\
	\textsc{au} & Australia \\
\end{tabularx}
\begin{tabularx}{.30\textwidth}{lQ}
	\textsc{na} & North America \\
	\textsc{sa} & South America \\
\end{tabularx}

\section*{Symbols}
\begin{tabularx}{.95\textwidth}{cQ}
	\~{} & Indicates reduplication. \\
	† & Indicates a marginally productive voice. \\
	* & Indicates an ungrammatical \textit{or} reconstructed form. \\
	? & Indicates an uncertain form, meaning \textit{or} diachronic development. \\
	↔ & Indicates a comparison between two diatheses (see \sectref{def:principles}).  \\
	→ & Indicates a diachronic development (see Chapter \ref{sec:diachrony}).  \\
	← & Same as above. \\
\end{tabularx}

\section*{Shorthands}
\begin{tabularx}{.95\textwidth}{cQ}
	sb. & somebody \\
	sth. & something \\
	e.o. & each other \\
	self & oneself \\
\end{tabularx}