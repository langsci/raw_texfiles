\chapter{Complex voice syncretism} \label{sec:complex-syncretism}
Given the seven voices of focus in this book (i.e. passive\is{passive voice}, antipassive\is{antipassive voice}, reflexive\is{reflexive voice}, reciprocal\is{reciprocal voice}, anticausative\is{anticausative voice}, causative\is{causative voice}, applicative\is{applicative voice}), 99 patterns of voice syncretism can logically be posited when one considers \textit{more than two} voices sharing the same voice marking. However, only seventeen patterns of complex voice syncretism\is{voice syncretism, complex} have actually been attested in the language sample and these represent the focus of this chapter (\tabref{tab:ch5:complex-patterns}). The patterns are discussed in terms of maximal syncretism\is{voice syncretism, maximal}, meaning that any given voice marking is discussed with regard to its full range of voice functions. Some of the patterns have already been mentioned briefly in terms of minimal syncretism\is{voice syncretism, minimal} in the previous chapter, but receive a more comprehensive treatment in this chapter. The distinction between minimal\is{voice syncretism, minimal} and maximal voice syncretism\is{voice syncretism, maximal} has been explained in Chapter \ref{introduction}.

\begin{table}
	\begin{tabularx}{.92\textwidth}{ccc}
		\lsptoprule
		Middle & Antipassive & Causative \\
		\midrule
		\textsc{refl-recp-antc} & \textsc{antp-refl-recp} & \textsc{caus-refl-antc} \\
		\textsc{pass-refl-recp} & \textsc{antp-refl-antc} & \textsc{caus-pass-recp} \\
		\textsc{pass-refl-antc} & \textsc{pass-antp-antc} & \textsc{caus-pass-antc} \\
		\textsc{pass-recp-antc} & \textsc{appl-antp-recp} & \textsc{caus-appl-pass} \\
		\textsc{pass-refl-recp-antc} & \textsc{antp-refl-recp-antc} & \textsc{caus-pass-refl-recp} \\
		& \textsc{pass-antp-refl-antc} & \\
		\lspbottomrule
	\end{tabularx}
	\caption{Patterns of full complex voice syncretism}
	\label{tab:ch5:complex-patterns}
\end{table} 

Sixteen of the seventeen patterns of complex voice syncretism\is{voice syncretism, complex} covered by this chapter are divided into the three groupings shown in \tabref{tab:ch5:complex-patterns} to facilitate their discussion in a convenient manner. Middle syncretism\is{middle syncretism} refers to complex voice syncretism\is{voice syncretism, complex} involving three or four of the following voices: passive\is{passive voice}, reflexive\is{reflexive voice}, reciprocal\is{reciprocal voice}, anticausative\is{anticausative voice} (\sectref{sec:complex-syncretism:middle}). In turn, antipassive\is{antipassive voice} and causative\is{causative voice} voice syncretism refer to complex voice syncretism\is{voice syncretism, complex} involving the antipassive\is{antipassive voice} voice (\sectref{sec:complex-syncretism:antipassive}) and the causative\is{causative voice} voice (\sectref{sec:complex-syncretism:causative}), respectively. The last pattern not included in \tabref{tab:ch5:complex-patterns} is passive-antipassive-reflexive-reciprocal-anticausative syncretism which is discussed separately from the other groupings due to its rare nature (\sectref{sec:complex-syncretism:multiplex}).

\section{Middle syncretism} \label{sec:complex-syncretism:middle}
As shown in the beginning of this chapter, five patterns of complex \isi{middle syncretism}\is{voice syncretism, complex} are attested in the language, and each of these patterns is illustrated in this section. The most complex pattern\is{voice syncretism, complex} of \isi{middle syncretism}, passive-reflexive-reciprocal-anticausative syncretism, is attested in the Indo-European language Eastern Armenian\il{Armenian, Eastern} (\lang{ea}) and the Uto-Aztecan language Huasteca Nahuatl\il{Nahuatl, Huasteca} (\lang{na}).

\begin{table}
	\setlength{\tabcolsep}{2.3pt}
	\begin{tabularx}{\textwidth}{llllll}
		\lsptoprule
		\multicolumn{6}{l}{Eastern Armenian\il{Armenian, Eastern} \citep[177f., 240, 322, 334, 340ff., 358ff., 610, 661]{dum-tragut:2009}} \\ 
		\midrule
		\textsc{pass} & \example{span-} & ‘to kill sb.’ & ↔ & \example{span-\textbf{v}-} & ‘to be killed [by sb.]’ \\
		\textsc{pass} & \example{merž-} & ‘to reject sth.’ & ↔ & \example{merž-\textbf{v}-} & ‘to be rejected [by sb.]’ \\
		\textsc{refl} & \example{sanr-} & ‘to comb sb.’ & ↔ & \example{sanr-\textbf{v}-} & ‘to comb self’ \\
		\textsc{refl} & \example{paštpan-} & ‘to defend sb.’ & ↔ & \example{paštpan-\textbf{v}-} & ‘to defend self’ \\
		\textsc{recp} & \example{tesn-} & ‘to see sth.’ & ↔ & \example{tesn-\textbf{v}-} & ‘to see e.o.’ \\
		\textsc{recp} & \example{hambur-} & ‘to kiss sb.’ & ↔ & \example{hambur-\textbf{v}-} & ‘to kiss e.o.’ \\
		\textsc{antc} & \example{ǰard-} & ‘to break sth.’ & ↔ & \example{ǰard-\textbf{v}-} & ‘to break’ \\
		\textsc{antc} & \example{šarž-} & ‘to move sth.’ & ↔ & \example{šarž-\textbf{v}-} & ‘to move’ \\
		\midrule\midrule
		\multicolumn{6}{l}{Huasteca Nahuatl\il{Nahuatl, Huasteca} \citep[90ff.]{llanes:al:2017}} \\
		\midrule
		\textsc{pass} & \example{tlali-} & ‘to put sth.’ & ↔ & \example{\textbf{mo}-tlali-} & ‘to be put [by sb.]’ \\
		\textsc{refl} & \example{ilpi-} & ‘to tie sth.’ & ↔ & \example{\textbf{mo}-ilpi-} & ‘to tie self’ \\
		\textsc{recp} & \example{ita-} & ‘to see sth.’ & ↔ & \example{\textbf{mo}-ita-} & ‘to see e.o.’ \\
		\textsc{recp} & \example{wika-to-} & ‘to get along with sb.’ & ↔ & \example{\textbf{mo}-wika-to-} & ‘to get along with e.o.’ \\
		\textsc{antc} & \example{tlan-} & ‘to lift sth.’ & ↔ & \example{\textbf{mo}-tlan-} & ‘to stand up’ \\
		\textsc{antc} & \example{kweso-} & ‘to sadden sb.’ & ↔ & \example{\textbf{mo}-kweso-} & ‘to get sad’ \\
		\lspbottomrule
	\end{tabularx}
	\caption{Passive-reflexive-reciprocal-anticausative syncretism}
	\label{tab:ch5:pass-refl-recp-antc}
\end{table}

As illustrated in \tabref{tab:ch5:pass-refl-recp-antc}, the syncretism is characterised by the suffix \example{-v} in Eastern Armenian\il{Armenian, Eastern}, and by the prefix \example{mo-} in Huasteca Nahuatl\il{Nahuatl, Huasteca}. Observe that the Eastern Armenian verb \example{hambur-} without \example{-v} is not explicitly given in \citeauthor{dum-tragut:2009}’s (\citeyear{dum-tragut:2009}) grammar of the language (see instead, e.g., \citealt[162]{sakayan:2007}). \citet[81f.]{llanes:al:2017} only provide one example each for the passive\is{passive voice} and reflexive\is{reflexive voice} functions of the prefix \example{mo-} in Huasteca Nahuatl\il{Nahuatl, Huasteca}, yet describe the functions as if they were productive\is{productivity} and also explicitly mention the “syncretism between reflexive\is{reflexive voice}, reciprocal\is{reciprocal voice}, middle and passive\is{passive voice} meanings”. Interestingly, \citet[102]{llanes:al:2017} remark that “none anticausative\is{anticausative voice} use has been documented in the corpus for the prefix” (sic), yet at least two of their examples qualify as such in this book (\example{mo-tlan-} and \example{mo-kweso-}). Note that a directional marker \example{-to} is included in the verb \example{mo-wika-to-}. \citet[91]{llanes:al:2017} argue that “[a]lthough the base verb \example{wika} ‘get along’ could be analysed here as an \isi{intransitive} verb since it is suffixed by a directional marker, this verb is still \isi{bivalent} (the second argument would be an oblique argument introduced by the directional marker)”. In other words, when succeeded by the suffix \example{-to} the verb \example{wika-} entails two semantic participants\is{semantic participant}: one who gets along, and another with which one gets along.

The other four patterns of complex \isi{middle syncretism}\is{voice syncretism, complex} attested in the language sample are each attested in at least two languages, and for practical reasons the patterns are therefore illustrated by a single language each in \tabref{tab:ch5:middle}: reflexive-reciprocal-anticausative syncretism in the Torricelli language \ili{Yine} (\lang{pn}), pas\-sive-re\-flex\-ive-re\-ci\-proc\-al syncretism in the Nadahup language \ili{Hup} (\lang{sa}), pas\-sive-re\-flex\-ive-anti\-cau\-sative syncretism in the Tangkic language \ili{Kayardild} (\lang{au}), and pas\-sive-re\-ci\-pro\-cal-anti\-cau\-sative syncretism in the Highland East Cushitic language Si\-daa\-ma\il{Sidaama} (\lang{af}). In \ili{Yeri} the “detransitivizing morpheme”\is{detransitivisation} \example{d-} serves as voice marking in the reflexive\is{reflexive voice}, reciprocal\is{reciprocal voice} and anticausative\is{anticausative voice} voices. \citet[369f.]{wilson:2017} explicitly recognises each of these voice functions, and remarks that the anticausative\is{anticausative voice} function “is particularly common with specific posture-arrange \isi{transitive} verb roots, where its use creates several of the posture verbs”. This particular pattern is not just the most common pattern of \isi{middle syncretism} attested in the language sample, but the most common of all complex patterns\is{voice syncretism, complex} (see \tabref{tab:ch6:voice-syncretism-maximal-complex-macroarea} on page \pageref{tab:ch6:voice-syncretism-maximal-complex-macroarea}). In turn, in \ili{Hup} the prefix \example{hup-} serves as voice marking in the passive\is{passive voice}, reflexive\is{reflexive voice}, and reciprocal\is{reciprocal voice} voices. However, the reciprocal\is{reciprocal voice} function of the prefix is “marginal” and always “interchangeable with the Interactional preform \example{ʔũh-}” (\citealt[473, 485f.]{epps:2008};; cf. \example{ʔũh-nɔʔ-} ‘to give e.o. sth.’). Unlike the affixes \example{-v}, \example{mo-}, and \example{d-} in Eastern Armenian, Huasteca Nahuatl, and Yeri, respectively, the \ili{Hup} prefix \example{hup-} does not have a documented anticausative\is{anticausative voice} function. The diachrony of the prefix is discussed in \sectref{diachrony:refl2recp} and \sectref{diachrony:refl2pass}.

\begin{table}[t]
	\setlength{\tabcolsep}{3.2pt}
	\begin{tabularx}{\textwidth}{llllll}
		\lsptoprule
		\multicolumn{6}{l}{\ili{Yeri} \citep[369f., 385, 451, 461, 692]{wilson:2017}} \\
		\midrule  
		\textsc{refl} & \example{altou} & ‘to cover sth.’ & ↔ & \example{\textbf{d}-altou} & ‘to cover self’ \\
		\textsc{refl} & \example{iesebɨl} & ‘to whip sb.’ & ↔ & \example{\textbf{d}-iesebɨl-} & ‘to whip self’ \\
		\textsc{recp} & \example{okɨrki} & ‘to help sb.’ & ↔ & \example{\textbf{d}-okɨrki} & ‘to help e.o.’ \\
		\textsc{recp} & \example{iekewa} & ‘to be angry at sb.’ & ↔ & \example{\textbf{d}-iekewa} & ‘to be angry at e.o.’ \\
		\textsc{antc} & \example{awɨl} & ‘to hang sth.’ & ↔ & \example{\textbf{d}-awɨl} & ‘to hang’ \\
		\textsc{antc} & \example{awera} & ‘to make sth. lie flat’ & ↔ & \example{\textbf{d}-awera} & ‘to lie flat’ \\
		\midrule\midrule
		\multicolumn{6}{l}{\ili{Hup} \citep[46, 479, 483, 486, 513, 574]{epps:2008}} \\
		\midrule 
		\textsc{pass} & \example{kɨ́t-} & ‘to cut sth.’ & ↔ & \example{\textbf{hup}-kɨ́t-} & ‘to be cut [by sb.]’ \\
		\textsc{pass} & \example{mǽh-} & ‘to kill sb.’ & ↔ & \example{\textbf{hup}-mǽh-} & ‘to be killed [by sb.]’ \\
		\textsc{refl} & \example{kɨ́t-} & ‘to cut sth.’ & ↔ & \example{\textbf{hup}-kɨ́t-} & ‘to cut self’ \\
		\textsc{refl} & \example{cúʔ-} & ‘to grab sth.’ & ↔ & \example{\textbf{hup}-cúʔ-} & ‘to grap self’ \\
		\textsc{recp} & \example{wǽd-} & ‘to eat sth.’ & ↔ & \example{\textbf{hup}-wǽd-} & ‘to eat e.o.’ \\
		\textsc{recp} & \example{nɔʔ-} & ‘to give sb. sth.’ & ↔ & \example{\textbf{hup}-nɔʔ-} & ‘to give e.o. sth.’ \\
		\midrule\midrule
		\multicolumn{6}{l}{\ili{Kayardild} \citep[1f., 79, 212, 352, 427, 532, 490, 696]{evans:1995}} \\
		\midrule
		\textsc{pass} & \example{bala-} & ‘to hit sth.’ & ↔ & \example{bala-\textbf{a}-} & ‘to be hit [by sb.]’ \\
		\textsc{pass} & \example{raa-} & ‘to spear sth.’ & ↔ & \example{ra-\textbf{yii}-} & ‘to be speared [by sb.]’ \\
		\textsc{refl} & \example{mardala-} & ‘to rub sth.’ & ↔ & \example{mardala-\textbf{a}-} & ‘to rub self’ \\
		\textsc{refl} & \example{kala-} & ‘to cut sth.’ & ↔ & \example{kala-\textbf{a}-} & ‘to cut self’ \\
		\textsc{antc} & \example{dara-} & ‘to break sth.’ & ↔ & \example{dara-\textbf{a}-} & ‘to break’ \\
		\textsc{antc} & \example{mirndili-} & ‘to shut sth.’ & ↔ & \example{mirndili-\textbf{i}-} & ‘to shut’ \\
		\midrule\midrule
		\multicolumn{6}{l}{\ili{Sidaama} \citep[117, 186, 220, 225, 315, 334, 342, 545]{kawachi:2007}} \\
		\midrule 
		\textsc{pass} & \example{ɡan-} & ‘to hit sth.’ & ↔ & \example{ɡan-\textbf{am}-} & ‘to be hit [by sb.]’ \\
		\textsc{pass} & \example{haišš-} & ‘to wash sth.’ & ↔ & \example{haišš-\textbf{am}-} & ‘to be washed [by sb.]’ \\
		\textsc{recp} & \example{sunkʼ-} & ‘to kiss sb.’ & ↔ & \example{sunkʼ-\textbf{am}-} & ‘to kiss e.o.’ \\
		\textsc{recp} & \example{tʼaad-} & ‘to meet sb.’ & ↔ & \example{tʼaad-\textbf{am}-} & ‘to meet e.o.’ \\
		\textsc{antc} & \example{hiikkʼ-} & ‘to break sth.’ & ↔ & \example{hiikkʼ-\textbf{am}-} & ‘to break’ \\
		\textsc{antc} & \example{tʼiss-} & ‘to make sb. sick’ & ↔ & \example{tʼiss-\textbf{am}-} & ‘to get sick’ \\
		\lspbottomrule
	\end{tabularx}
	\caption{Four patterns of complex middle syncretism}
	\label{tab:ch5:middle}
\end{table}

In \ili{Kayardild} passive-reflexive-anticausative syncretism is characterised by a so-called “middle suffix” with a range of allomorphs,\is{allomorphy} two of which are relevant to the examples presented in \tabref{tab:ch5:middle}: \example{-yii} found on stems ending in a long vowel which is shortened, and \isi{vowel lengthening} (or \example{-V}) found on stems ending in a short vowel other than /u/ \citep[276f.]{evans:1995}. The verb \example{mardala-} in the table also can have the meaning ‘to paint sth.’ \citep[726]{evans:1995}. Unlike the affixes \example{-v}, \example{mo-}, \example{d-} and \example{hup-} in Eastern Armenian\il{Armenian, Eastern}, Huasteca Nahuatl\il{Nahuatl, Huasteca}, \ili{Yeri}, and \ili{Hup} above, the suffix \example{-yii/-V} in \ili{Kayardild} is not used as voice marking in the reciprocal\is{reciprocal voice} voice which is instead characterised by the suffix \example{-(n)thu/-nju} (e.g. \example{bala-thu-} ‘to hit e.o.’, \citealt[487]{evans:1995}; see also \sectref{diachrony:recp2refl}). Finally, passive-reciprocal-anticausative syncretism in \ili{Sidaama} is characterised by the suffix \example{-am}. \citet[333ff., 342ff.]{kawachi:2007} explicitly recognises the passive\is{passive voice} and reciprocal\is{reciprocal voice} functions of the suffix but does not mention any anticausative\is{anticausative voice} function. However, it is evident from several of the examples found in \citeauthor{kawachi:2007}’s (\citeyear[e.g. 117]{kawachi:2007}) grammar of the language that the suffix also has this function. For instance, in one case \citet[186]{kawachi:2007} translates the verb \example{hiikkʼ-am-} accompanied by an emphatic reflexive pronoun ‘(the mirror) got broken by itself’ highlighting that no other \isi{semantic participant} is involved. The diachrony of the \ili{Sidaama} suffix \example{-am} is discussed in \sectref{diachrony:pass2recp}, in which it is argued that the suffix represents a rare instance of reciprocal\is{reciprocal voice} voice marking developing a passive\is{passive voice} function.

\largerpage
\section{Antipassive syncretism} \label{sec:complex-syncretism:antipassive}
Eleven languages in the language sample feature one of the six patterns of complex antipassive\is{antipassive voice} voice syncretism\is{voice syncretism, complex} presented at the beginning of this chapter (see \tabref{tab:ch5:complex-patterns} on \pageref{tab:ch5:complex-patterns}). Patterns involving both the passive\is{passive voice} and antipassive\is{antipassive voice} voices are discussed in the next section, while patterns involving both the antipassive\is{antipassive voice} and reflexive\is{reflexive voice} voices are treated in \sectref{sec:complex-syncretism:antp-refl} and applicative-antipassive-reciprocal syncretism in \sectref{sec:complex-syncretism:appl-antp-recp}.

\subsection{Passive-antipassive-*} \label{sec:complex-syncretism:pass-antp}


\begin{table}[b]
	\setlength{\tabcolsep}{3.6pt}
	\begin{tabularx}{\textwidth}{llllll}
		\lsptoprule
		\multicolumn{6}{l}{\ili{Tatar} (\citealt[173]{zinnatullina:1993}; \citealt[473, 484f.]{burbiel:2018})} \\
		\midrule 
		\textsc{pass} & \example{sayla-} & ‘to choose sth.’ & ↔ & \example{sayla-\textbf{n}-} & ‘to be chosen [by sb.]’ \\
		\textsc{pass} & \example{alda-} & ‘to deceive sb.’ & ↔ & \example{alda-\textbf{n}-} & ‘to be deceived [by sb.]’ \\
		\textsc{antp} & \example{peşer-} & ‘to cook sth.’ & ↔ & \example{peşer-\textbf{en}-} & ‘to cook [sth.]’ \\
		\textsc{antp} & \example{teg-} & ‘to sew sth.’ & ↔ & \example{teg-\textbf{en}-} & ‘to sew [sth.]’ \\
		\textsc{refl} & \example{tara-} & ‘to comb sb.’ & ↔ & \example{tara-\textbf{n}-} & ‘to comb self’ \\
		\textsc{refl} & \example{sört-} & ‘to dry sth.’ & ↔ & \example{sört-\textbf{en}-} & ‘to dry self’ \\
		\textsc{antc} & \example{karañgıla-} & ‘to darken sth.’ & ↔ & \example{karañgıla-\textbf{n}-} & ‘to darken’ \\
		\textsc{antc} & \example{ütmäslä-} & ‘to dull sth.’ & ↔ & \example{ütmäslä-\textbf{n}-} & ‘to dull’ \\
		\lspbottomrule
	\end{tabularx}
	\caption{Passive-antipassive-reflexive-anticausative syncretism}
	\label{tab:ch5:pass-antp-refl-antc}
\end{table}

Complex voice syncretism\is{voice syncretism, complex} involving both the passive\is{passive voice} and antipassive\is{antipassive voice} voices is only attested in two languages in the language sample, the Turkic language \ili{Tatar} (\lang{ea}) and the language isolate \ili{Mosetén} (\lang{sa}). The former language features passive-antipassive-reflexive-reciprocal syncretism, while the latter language features passive-antipassive-reciprocal syncretism. The syncretism in Tatar is cha\-rac\-te\-ri\-sed by the suffix \example{-n}, as illustrated in \tabref{tab:ch5:pass-antp-refl-antc}. The passive\is{passive voice} suffix \example{-n} appears to be an allomorph\is{allomorphy} of another passive\is{passive voice} suffix \example{-l} which can be traced back to Common Turkic\il{Turkic, Common}. In \ili{Tatar} the allomorph\is{allomorphy} \example{-n} appears on stems ending in /l/ or a consonant cluster involving the phoneme, while the allomorph\is{allomorphy} \example{-l} appears elsewhere \citep[473]{burbiel:2018}. The anticausative\is{anticausative voice} suffix \example{-n} appears to be similar in this respect. By contrast, the suffix \example{-n} in the reflexive\is{reflexive voice} and antipassive\is{antipassive voice} voices has no allomorph\is{allomorphy} \example{-l} and is historically linked to a third person pronoun (\sectref{diachrony:refl2antp}).



\citet[236, 306ff.]{sakel:2004} argues that \ili{Mosetén} has three suffixes with “the same form” \example{-ki}: a “verbal stem marker”, a “middle marker”, and an “antipassive\is{antipassive voice} marker”. \citeauthor{sakel:2004}’s markers are here treated as a single syncretic suffix, \example{-ki}, which qualifies as voice marking in the passive\is{passive voice}, antipassive\is{antipassive voice}, and anticausative\is{anticausative voice} voices, as shown in \tabref{tab:ch5:pass-antp-antc}. Observe that stem-final /i/ becomes /a/ when followed by \example{-ki} and certain other suffixes \citep[47, 308]{sakel:2004}. An “associated motion marker” \example{-ki} is also recognised by \citet[273]{sakel:2004}, but there is a structural difference between this and the passive-antipassive-anticausative marker \example{-ki} that “has to do with the vowel change before the suffix”. Moreover, as associated motion is not directly relevant to this book, the function is ignored. The use of \example{-ki} as a verbal stem marker is not of primary interest here either, as it is “only used with bound verbal roots” to form verbal stems \citep[218, 236]{sakel:2004}. However, from a \isi{language-specific} perspective, it may be worth noting that verbal stems incorporating the suffix in question are “\isi{intransitive} and can have stative or dynamic meanings” \citep[236]{sakel:2004}, qualities often associated with passives\is{passive voice}, antipassives\is{antipassive voice}, and/or anticausative\is{anticausative voice} in the literature. \citet[307, 479]{sakel:2004} only provides one example of the anticausative\is{anticausative voice} use of the suffix \example{-ki}, yet she explicitly states that the suffix can express “spontaneous events” and notes that such events “are sometimes called ‘anticausative\is{anticausative voice}’”, and it is therefore assumed that the function in question is productive\is{productivity} with other verbs as well. Glossed examples of the passive-antipassive syncretism in the language have already been provided in \sectref{sec:simple-syncretism:pass-antp} (see examples \ref{ex:Moseten:eat:a}--\ref{ex:Moseten:work:b} on page \pageref{ex:Moseten:eat:a}).

\begin{table}
	\setlength{\tabcolsep}{4.2pt}
	\begin{tabularx}{\textwidth}{llllll}
		\lsptoprule
		\multicolumn{6}{l}{\ili{Mosetén} \citep[306ff.]{sakel:2004}} \\
		\midrule 
		\textsc{pass} & \example{jeb-i-} & ‘to eat sth.’ & ↔ & \example{jeb-a-\textbf{ki}-} & ‘to be eaten [by sb.]’ \\
		\textsc{pass} & \example{raem’-yi-} & ‘to bite sb.’ & ↔ & \example{raem’-ya-\textbf{ki}-} & ‘to be bitten [by sb.]’ \\
		\textsc{antp} & \example{karij-tyi-} & ‘to work on sth.’ & ↔ & \example{karij-tya-\textbf{ki}-} & ‘to work on [sth.]’ \\
		\textsc{antp} & \example{san-i-} & ‘to write sth.’ & ↔ & \example{san-a-\textbf{ki}-} & ‘to write [sth.]’ \\
		\textsc{antc} & \example{jofor’-yi-} & ‘to open sth.’ & ↔ & \example{jofor’-ya-\textbf{ki}-} & ‘to open’ \\
		\lspbottomrule
	\end{tabularx}
	\caption{Passive-antipassive-anticausative syncretism}
	\label{tab:ch5:pass-antp-antc}
\end{table}

\newpage

Another language in the sample, the Central Salish language \ili{Musqueam} (\lang{na}), features syncretism superficially similar to that described for Mosetén above. In \ili{Musqueam} the suffix \example{-m} serves as voice marking in both the passive\is{passive voice} and anticausative\is{anticausative voice} voices. As discussed in \sectref{def:passives-antipassives}, the suffix even has an “antipassive-like” function shown alongside the passive\is{passive voice} and anticausative\is{anticausative voice} functions in \tabref{tab:ch5:pass-antp-antc-2}. However, as the suffix does not have a proper antipassive\is{antipassive voice} function, passive-antipassive-anticausative syncretism is not acknowledged for \ili{Musqueam} here. In any case, observe that in the passive\is{passive voice} voice the suffix \example{-m} is added onto a verbal stem, but in the antipassive-like and anticausative\is{anticausative voice} voices the suffix is in variation with verbal marking in the contrasting diatheses\is{diathesis} according to which they are defined (\example{-t}). The difference in vowel length between the verbal forms \example{híˑl-} and \example{híl-} is morphophonologically\is{morphophonology} conditioned \citep[147f.]{suttles:2004}.

\begin{table}
	\setlength{\tabcolsep}{3.4pt}
	\begin{tabularx}{\textwidth}{llllll}
		\lsptoprule
		\multicolumn{6}{l}{\ili{Musqueam} \citep[35, 43, 51, 230f., 447f.]{suttles:2004}} \\
		\midrule 
		\textsc{pass} & \example{c̓éw-ɘt} & ‘to help sb.’ & ↔ & \example{c̓éw-ɘt-\textbf{əm}} & ‘to be helped [by sb.]’ \\
		\textsc{pass} & \example{k̓ʷłé-t} & ‘to tip sth. over’ & ↔ & \example{k̓ʷłé-t-\textbf{əm}} & ‘to be tipped over [by sb.]’ \\
		\textsc{“antp”} & \example{kʷə́n-ət} & ‘to get/take sth.’ & ↔ & \example{kʷə́n-\textbf{əm}} & ‘to get [sth.]’ \\
		\textsc{“antp”} & \example{k̓ʷxé-t} & ‘to count sth.’ & ↔ & \example{k̓ʷxé-\textbf{m}} & ‘to count [sth.]’ \\
		\textsc{antc} & \example{híˑl-t} & ‘to roll sth.’ & ↔ & \example{híl-\textbf{əm}} & ‘to roll’ \\
		\textsc{antc} & \example{pk̓ʷə́-t} & ‘to scatter sth.’ & ↔ & \example{pk̓ʷə́-\textbf{m}} & ‘to splash/billow out’ \\
		\lspbottomrule
	\end{tabularx}
	\caption{Passive-“antipassive”-anticausative syncretism}
	\label{tab:ch5:pass-antp-antc-2}
\end{table}

\subsection{Antipassive-reflexive-*} \label{sec:complex-syncretism:antp-refl}
Complex voice syncretism\is{voice syncretism, complex} involving the antipassive\is{antipassive voice} and reflexive\is{reflexive voice} voices is particularly noteworthy in the Oto-Manguean language Acazulco Otomí\il{Otomí, Acazulco}, the Southern Iroquoian language \ili{Cherokee} (both \lang{na}), the Tacanan language \ili{Ese Ejja} (\lang{sa}), and the Northern Chukotko-Kamchatkan language \ili{Chukchi} (\lang{ea}) which all feature antipassive-reflexive-reciprocal-anticausative syncretism. Similar syncretism has been observed for other languages sporadically in the literature (e.g. \citealt[780ff]{letuchiy:2007} on the Northwest Caucasian language \ili{Adyghe}). The syncretism in Acazulco Otomí\il{Otomí, Acazulco}, \ili{Ese Ejja} and \ili{Chukchi} qualifies as type 1a syncretism\is{voice syncretism, full resemblance -- type 1} and is illustrated in \tabref{tab:ch5:antp-refl-recp-antc}. In \ili{Cherokee} the syncretism qualifies as type 1b syncretism\is{voice syncretism, full resemblance -- type 1}. 

\begin{table}[t]
	\setlength{\tabcolsep}{4.6pt}
	\begin{tabularx}{\textwidth}{llllll}
		\lsptoprule
		\multicolumn{6}{l}{Acazulco Otomí\il{Otomí, Acazulco} \citep[294, 513]{hernandez-green:2015}} \\
		\midrule 
		\textsc{antp} & \example{pèni} & ‘to wash sth.’ & ↔ & \example{\textbf{m}-pèni} & ‘to wash [sth.]’ \\
		\textsc{antp} & \example{tà̱i} & ‘to buy sth.’ & ↔ & \example{\textbf{n}-tà̱i} & ‘to buy [sth.]’ \\
		\textsc{refl} & \example{hë́ʼtʼ} & ‘to see sth.’ & ↔ & \example{\textbf{ntx}-hë́ʼtʼ} & ‘to see self’ \\
		\textsc{refl} & \example{hò} & ‘to hit sth.’ & ↔ & \example{\textbf{ntx}-hò} & ‘to hit self’ \\
		\textsc{recp} & \example{hò} & ‘to hit sth.’ & ↔ & \example{\textbf{ntx}-hò} & ‘to hit e.o.’ \\
		\textsc{recp} & \example{tsú̱i} & ‘to scold sb.’ & ↔ & \example{\textbf{n}-tsú̱i} & ‘to scold e.o.’ \\
		\textsc{antc} & \example{kóʼmbi} & ‘to cover sth.’ & ↔ & \example{\textbf{n}-kóʼmbi} & ‘to cover up’ \\
		\textsc{antc} & \example{phà̱gi} & ‘to spill sth.’ & ↔ & \example{\textbf{m}-phà̱gi} & ‘to spill’ \\
		\midrule\midrule
		\multicolumn{6}{l}{\ili{Ese Ejja} \citep[520ff.]{vuillermet:2012}} \\
		\midrule 
		\textsc{antp} & \example{ba-} & ‘to see sth.’ & ↔ & \example{\textbf{xa}-ba-\textbf{ki}-} & ‘to see [sth.]’ \\
		\textsc{antp} & \example{iña-} & ‘to grab sth.’ & ↔ & \example{\textbf{xa}-iña-\textbf{ki}-} & ‘to grab [sth.]’ \\
		\textsc{refl} & \example{jabe-} & ‘to comb sb.’ & ↔ & \example{\textbf{xa}-jabe-\textbf{ki}-} & ‘to comb self’ \\
		\textsc{refl} & \example{paa-} & ‘to cover sth. up’ & ↔ & \example{\textbf{xa}-paa-\textbf{ki}-} & ‘to cover self up’ \\
		\textsc{recp} & \example{nabatoxo-} & ‘to kiss sb.’ & ↔ & \example{\textbf{xa}-nabatoxo-\textbf{ki}-} & ‘to kiss e.o.’ \\
		\textsc{recp} & \example{kwya-} & ‘to hit sth.’ & ↔ & \example{\textbf{xa}-kwya-\textbf{ki}-} & ‘to hit e.o.’ \\
		\textsc{antc} & \example{isa-} & ‘to tear sth.’ & ↔ & \example{\textbf{xa}-isa-\textbf{ki}-} & ‘to tear’ \\
		\textsc{antc} & \example{saja-} & ‘to break sth.’ & ↔ & \example{\textbf{xa}-saja-\textbf{ki}-} & ‘to break’ \\
		\midrule\midrule
		\multicolumn{6}{l}{\ili{Chukchi} (\citealt[220ff.]{nedjalkov:2006}; \citealt[186]{kurebito:2012})} \\
		\midrule 
		\textsc{antp} & \example{juu-} & ‘to bite sb.’ & ↔ & \example{juu-\textbf{tku}-} & ‘to bite [sb.]’ \\
		\textsc{antp} & \example{penrə-} & ‘to fall on sth.’ & ↔ & \example{penrə-\textbf{tko}-} & ‘to fall on [sth.]’ \\
		\textsc{refl} & \example{lpiw-} & ‘to cut sth.’ & ↔ & \example{lpiw-\textbf{tku}-} & ‘to cut self’ \\
		\textsc{refl} & \example{ittil-} & ‘to hit sth.’ & ↔ & \example{ittil-\textbf{tku}-} & ‘to hit self’ \\
		\textsc{recp} & \example{ukwet-} & ‘to kiss sb.’ & ↔ & \example{ukwet-ə-\textbf{tku}-} & ‘to kiss e.o.’ \\
		\textsc{recp} & \example{lʔu-} & ‘to see sth.’ & ↔ & \example{lʔu-\textbf{tku}-} & ‘to see e.o.’ \\
		\textsc{antc} & \example{ejpə-} & ‘to close sth.’ & ↔ & \example{ejpə-\textbf{tku}-} & ‘to close’ \\
		\lspbottomrule
	\end{tabularx}
	\caption{Antipassive-reflexive-reciprocal-anticausative syncretism}
	\label{tab:ch5:antp-refl-recp-antc}
\end{table}

In Acazulco Otomí\il{Otomí, Acazulco} antipassive-reflexive-reciprocal-anticausative syncretism is characterised by the nasal prefix \example{n-} with the allomorphs\is{allomorphy} \example{m-}, \example{nt-} (before /x/), and \example{ntx-} (before a glottal fricative or stop). \citet[512, 525]{hernandez-green:2015} explicitly remarks that the extensive syncretism of this suffix is productive\is{productivity} but does not otherwise discuss the suffix further. In \ili{Ese Ejja} the circumfix \example{xa-…-ki} serves as voice marking in the antipassive\is{antipassive voice}, reflexive\is{reflexive voice}, reciprocal\is{reciprocal voice}, and anticausative\is{anticausative voice} voices. As already noted in \sectref{sec:simple-syncretism:pass-antp}, \citet[519]{vuillermet:2012} even suggests that the circumfix can have a “passive-like” function which, however, does not qualify as proper passive\is{passive voice} in this book, for which reason it is not included in \tabref{tab:ch5:antp-refl-recp-antc}. Next, antipassive-reflexive-reciprocal-anticausative syncretism in \ili{Chukchi} is characterised by the suffix \example{-tku/-tko} conditioned by \isi{vowel harmony} which \citet[221]{nedjalkov:2006} tellingly has been called “the most polysemous\is{polysemy} suffix” in the language. Only one example of its anticausative\is{anticausative voice} use is provided in \tabref{tab:ch5:antp-refl-recp-antc}, yet both \citet[186]{kurebito:2012} and \citet[222]{nedjalkov:2006} explicitly mention that the suffix has such use. Indeed, \citet[222]{nedjalkov:2006} considers the anticausative\is{anticausative voice} function one of the default readings of the suffix. Thus, although only \citeauthor{nedjalkov:2006} provides an explicit anticausative\is{anticausative voice} example of the suffix \example{-tku/-tko}, the anticausative\is{anticausative voice} function is here assumed to be productive\is{productivity} with other verbs as well. The schwa in the verb \example{ukwet-ə-tku-} is simply epenthetic.

Antipassive-reflexive-reciprocal-anticausative type 1b syncretism\is{voice syncretism, full resemblance -- type 1} in \ili{Cherokee} is exemplified in \tabref{tab:ch5:antp-refl-recp-antc-3}. As described in \sectref{resemblance-type1b}, \ili{Cherokee} has what \citet[343, 347]{montgomery-anderson:2008} calls a “reflexive\is{reflexive voice} prefix” \example{at-/ataa(t)-} serving as voice marking in the antipassive\is{antipassive voice}, reflexive\is{reflexive voice}, and reciprocal\is{reciprocal voice} voices; and a “\isi{middle voice} prefix” \example{at-/ataa-/ali-} serving as voice marking in the anticausative\is{anticausative voice} voice. The former prefix has the allomorphs\is{allomorphy} \example{at-} (before the vowel /a/), \example{ataat-} (before all other vowels), and \example{ataa-} (before all consonants); while the latter prefix has the allomorphs\is{allomorphy} \example{at-} (before all vowels), \example{ali-} (before the consonant /h/ and seemingly also before /s/ and /n/), and \example{ataa-} (before all other consonants). Evidently, the allomorphs\is{allomorphy} of the two prefixes overlap under certain phonological conditions, namely before the vowel /a/ and before consonants other than /h/, /s/, and /n/. Furthermore, observe that verbs in \ili{Cherokee} have five stems that “express different grammatical information about the \isi{tense}, \isi{aspect}, and \isi{mood}” \citep[252]{montgomery-anderson:2008}. These different stems are “present continuous”, “incompletive”, “immediate”, “completive”, and “deverbal noun” (for instance used with auxiliary verbs). For example, the five stems of the verb ‘to help sb.’ are \example{-steelíha}, \example{-steeliísk}, \example{-steéla}, \example{-steelvvh}, and \example{-stehlt} \citep[224f.]{montgomery-anderson:2008}. This phenomenon explains the stem-related differences in \tabref{tab:ch5:antp-refl-recp-antc-3} (e.g. the stem \example{-xxjakahl} is completive, while the stem \example{-jakalvyska} is present continuous). Observe also that the digraph ⟨xx⟩ “indicates that the vowel of the prefix that attaches to the stem is lengthened”, while the digraph ⟨x́x⟩ indicates that the prefix “has a high tone” \citep[xii]{montgomery-anderson:2008}. The word-initial grapheme ⟨h⟩ ni the verb \example{-x́xhliisíha}  does not represent the phoneme /h/ but forms part of the digraphs ⟨hl⟩ representing the phoneme /ɬ/.

\begin{table}
	\setlength{\tabcolsep}{3.1pt}
	\begin{tabularx}{\textwidth}{llllll}
		\lsptoprule
		\multicolumn{6}{l}{\ili{Cherokee} \citep[201, 249, 275, 345, 366, 371, 373f., 382]{montgomery-anderson:2008}} \\
		\midrule
		\textsc{antp} & \example{-steelvvh} & ‘to help sb.’ & ↔ & \example{-\textbf{ataa}-stehlt} & ‘to help [sb.]’ \\
		\textsc{antp} & \example{-olihka} & ‘to recognise sb.’ & ↔ & \example{-\textbf{ataat}-olihka} & ‘to recognise [sb.]’ \\
		\textsc{refl} & \example{-olihka} & ‘to recognise sb.’ & ↔ & \example{-\textbf{ataat}-olihka} & ‘to recognise self’ \\
		\textsc{recp} & \example{-steelvvh} & ‘to help sb.’ & ↔ & \example{-\textbf{ataat}-steelvvh} & ‘to help e.o.’ \\
		\textsc{refl} & \example{-kohwthíha} & ‘to see sth.’ & ↔ & \example{-\textbf{ataa}-kohwthíha} & ‘to see self’ \\
		\textsc{recp} & \example{-kooh} & ‘to see sth.’ & ↔ & \example{-\textbf{ataa}-kooh} & ‘to see e.o.’ \\
		\textsc{antc} & \example{-x́xhliisíha} & ‘to gather sth.’ & ↔ & \example{-\textbf{ataa}-x́xhliisíha} & ‘to gather’ \\
		\textsc{antc} & \example{-xxjakahl} & ‘to rip sth.’ & ↔ & \example{-\textbf{ataa}-jakalvyska} & ‘to rip’ \\
		\lspbottomrule
	\end{tabularx}
	\caption{Antipassive-reflexive-reciprocal-anticausative syncretism}
	\label{tab:ch5:antp-refl-recp-antc-3}
\end{table}

Two other patterns of complex antipassive\is{antipassive voice} syncretism\is{voice syncretism, complex} are attested in the language sample, anti\-pas\-sive-re\-flex\-ive-re\-ci\-pro\-cal syncretism and anti\-pas\-sive-re\-flex\-ive-anti\-cau\-sa\-tive syncretism. The former pattern is attested in the Katukinan language \ili{Katukina-Kanamari} (\lang{sa}) and the Mangarrayi-Maran language \ili{Mangarrayi} (\lang{au}), while the latter pattern is attested in the language isolate \ili{Oksapmin} (\lang{pn}). The Gunwinyguan language \ili{Nunggubuyu} (\lang{au}) features both patterns. Anti\-pas\-sive-reflexive-reciprocal syncretism is illustrated for \ili{Katukina-Kanamari}, \ili{Mangarrayi} and \ili{Nunggubuyu} in \tabref{tab:ch5:antp-refl-recp}, whereas antipassive-reflexive-anticaus\-a\-tive syncretism is illustrated for \ili{Nunggubuyu} and \ili{Oksapmin} in \tabref{tab:ch5:antp-refl-antc}. In Ka\-tu\-ki\-na-Ka\-na\-ma\-ri antipassive-reflexive-reciprocal syncretism is characterised by an “intransitiviser” (\textit{intransitivizador}) with the allomorphs\is{allomorphy} \example{-i} (after /k/), \example{-k} (after the vowel /u/), and \example{-hik} (after /ŋ/ and all vowels but /u/, \citealt[121ff.]{dos-anjos:2011}). Considering the notable phonological differences between these allomorphs,\is{allomorphy} for comparative purposes the examples should ideally have featured the same allomorphs.\is{allomorphy} Unfortunately, \citeauthor{dos-anjos:2011} does not provide any clear antipassive\is{antipassive voice} examples involving the allomorphs\is{allomorphy} \example{-k} or \example{-i} nor any clear reflexive\is{reflexive voice} and reciprocal\is{reciprocal voice} examples involving the allomorph\is{allomorphy} \example{-hik}. The verb \example{kɯni-hik} ‘to bite self’ \citep[122]{dos-anjos:2011} does represent a reflexive\is{reflexive voice} voice if it is assumed that a verb \textsuperscript{?}\example{kɯni} with the meaning ‘to bite sth.’ exists in the language (the verb in question is not explicitly given in \citeauthor{dos-anjos:2011}’ grammar). Nevertheless, since the three voices are described as featuring the same voice marking with the same allomorphs,\is{allomorphy} it is assumed that each allomorph\is{allomorphy} can serve productively\is{productivity} as voice marking in the antipassive\is{antipassive voice}, reflexive\is{reflexive voice}, and reciprocal\is{reciprocal voice} voices in the language. Observe that the verb \example{uu} in \tabref{tab:ch5:antp-refl-recp} also appears variously as \example{uɯ} and \example{wu} in \citeauthor{dos-anjos:2011}’ grammar.

\begin{table}[t]
	\setlength{\tabcolsep}{6pt}
	\begin{tabularx}{\textwidth}{llllll}
		\lsptoprule
		\multicolumn{6}{l}{\ili{Katukina-Kanamari} \citep[121ff., 138, 336, 342f., 346f., 381]{dos-anjos:2011}} \\
		\midrule 
		\textsc{antp} & \example{tyaman} & ‘to cut sth.’ & ↔ & \example{tyaman-\textbf{hik}} & ‘to cut [sth.]’ \\
		\textsc{antp} & \example{topohan } & ‘to blow sth.’ & ↔ & \example{topohan -\textbf{hik}} & ‘to blow [sth.]’ \\
		\textsc{refl} & \example{uu} & ‘to like sth.’ & ↔ & \example{uu-\textbf{k}} & ‘to like self’ \\
		\textsc{refl} & \example{hi:k} & ‘to see sth.’ & ↔ & \example{hi:k-\textbf{i}} & ‘to see self’ \\
		\textsc{recp} & \example{pu} & ‘to eat sth.’ & ↔ & \example{pu-\textbf{k}} & ‘to eat e.o.’ \\
		\textsc{recp} & \example{tohi:k} & ‘to look at sth.’ & ↔ & \example{tohi:k-\textbf{i}} & ‘to look at e.o.’ \\
		\midrule\midrule
		\multicolumn{6}{l}{\ili{Mangarrayi} \citep[95f., 135f., 154f., 220]{merlan:1989}} \\
		\midrule
		\textsc{antp} & \example{gurwa-} & ‘to encircle sth.’ & ↔ & \example{gurwa-\textbf{jiyi}-} & ‘to encircle [sth.]’ \\
		\textsc{antp} & \example{miwu-} & ‘to sneak & ↔ & \example{miwu-\textbf{jiyi}-} & ‘to sneak \\
		& & \multicolumn{1}{r}{away from sb.’} & & & \multicolumn{1}{r}{away from [sb.]’}\\
		\textsc{refl} & \example{wa-} & ‘to look at sth.’ & ↔ & \example{wa-\textbf{ñjiyi}-} & ‘to look at self’ \\
		\textsc{refl} & \example{bu-} & ‘to hit sb.’ & ↔ & \example{bu-\textbf{yi}-} & ‘to hit self’ \\
		\textsc{recp} & \example{bu-} & ‘to hit sb.’ & ↔ & \example{bu-\textbf{yi}-} & ‘to hit e.o.’ \\
		\textsc{recp} & \example{ŋaniwu-} & ‘to speak to sb.’ & ↔ & \example{ŋaniwu-\textbf{jiyi}-} & ‘to talk to e.o.’ \\
		\midrule\midrule
		\multicolumn{6}{l}{\ili{Nunggubuyu} \citep[392]{heath:1984}} \\
		\midrule 
		\textsc{antp} & \example{lharma-} & ‘to chase sth.’ & ↔ & \example{lharma-\textbf{nʸji}-} & ‘to chase [sth.]’ \\
		\textsc{antp} & \example{wargura-} & ‘to carry sth.’ & ↔ & \example{warguri-\textbf{nʸji}-} & ‘to carry [sth.]’ \\
		\textsc{refl} & \example{wanᵍa-} & ‘to bite sth.’ & ↔ & \example{wanᵍi-\textbf{nʸji}-} & ‘to bite self’ \\
		\textsc{refl} & \example{ṟa-} & ‘to spear sth.’ & ↔ & \example{ṟi-\textbf{nʸji}-} & ‘to spear self’ \\
		\textsc{recp} & \example{wanᵍa-} & ‘to bite sth.’ & ↔ & \example{wanᵍi-\textbf{nʸji}-} & ‘to bite e.o.’ \\
		\textsc{recp} & \example{ṟa-} & ‘to spear sth.’ & ↔ & \example{ṟi-\textbf{nʸji}-} & ‘to spear e.o.’ \\
		\lspbottomrule
	\end{tabularx}
	\caption{Antipassive-reflexive-reciprocal syncretism}
	\label{tab:ch5:antp-refl-recp}
\end{table}



In \ili{Mangarrayi} the suffix \example{-yi/-(ñ)jiyi} typically serves as voice marking in either the reflexive\is{reflexive voice} or reciprocal\is{reciprocal voice} voice, but "[i]n a few cases" \citep[136]{merlan:1989} the suffix can even have an antipassive\is{antipassive voice} function, as shown in \tabref{tab:ch5:antp-refl-recp}. This table also illustrates antipassive-reflexive-reciprocal syncretism in the other Australian language, \ili{Nunggubuyu}, characterised by the suffix \example{-nʸji}. Observe that “the root-final vowel may change to /i/" before this suffix "depending on \isi{verb class}” \citep[101f., 392]{heath:1984}. Moreover, note that the verb \example{lharma-nʸji-} also can have the meaning ‘to chase e.o.’, but the verb \example{wargu-ri-nʸji-} cannot have the meaning ‘to carry e.o. on shoulder’ as this sense is “semantically awkward since carrying on shoulder is intrinsically nonreciprocal” \citep[392]{heath:1984}. 

\begin{table}
	\setlength{\tabcolsep}{2.6pt}
	\begin{tabularx}{\textwidth}{llllll}
		\lsptoprule
		\multicolumn{6}{l}{\ili{Nunggubuyu} \citep[390, 394]{heath:1984}} \\
		\midrule 
		\textsc{antp} & \example{yaḻgiwa-} & ‘to pass sth.’ & ↔ & \example{yaḻgiw-\textbf{i}-} & ‘to pass [sth.]’ \\
		\textsc{antp} & \example{wuṟama-} & ‘to go around sth.’ & ↔ & \example{wuṟam-\textbf{i}-} & ‘to go around [sth.]’ \\
		\textsc{refl} & \example{na-} & ‘to see sth.’ & ↔ & \example{n-\textbf{i}-} & ‘to see self’ \\
		\textsc{refl} & \example{lhamalhama-} & ‘to praise sth.’ & ↔ & \example{lhamalham-\textbf{i}-} & ‘to praise self’ \\
		\textsc{antc} & \example{ḻaḻaga-} & ‘to raise sth.’ & ↔ & \example{ḻaḻag-\textbf{i}-} & ‘to get up’ \\
		\textsc{antc} & \example{nᵍaṉḏa-} & ‘to sink sth. ’ & ↔ & \example{nᵍaṉḏ-\textbf{i}-} & ‘to sink’ \\
		\midrule\midrule
		\multicolumn{6}{l}{\ili{Oksapmin} \citep[239ff., 301, 369]{loughnane:2009}} \\
		\midrule 
		\textsc{antp} & \example{xtol} & ‘to look at sth.’ & ↔ & \example{\textbf{t}-xtol} & ‘to look at [sth.]’ \\
		\textsc{antp} & \example{aŋ de-/ml-} & ‘to look for sth.’ & ↔ & \example{aŋ \textbf{t}-x-} & ‘to look for [sth.]’ \\
		\textsc{refl} & \example{gəx de-/ml-} & ‘to wash sth.’ & ↔ & \example{gəx \textbf{t}-x-} & ‘to wash self’ \\
		\textsc{antc} & \example{dpəlkwe} & ‘to turn sth. over’ & ↔ & \example{\textbf{t}-dpəlkwe} & ‘to turn over’ \\
		\textsc{antc} & \example{dəlpə} & ‘to beget sth.’ & ↔ & \example{\textbf{t}-dəlpə} & ‘to begin’ \\
		\lspbottomrule
	\end{tabularx}
	\caption{Antipassive-reflexive-anticausative syncretism}
	\label{tab:ch5:antp-refl-antc}
\end{table}

By contrast, antipassive-reflexive-anticausative syncretism in \ili{Nunggubuyu} is characterised by the suffix \example{-i}, as shown in \tabref{tab:ch5:antp-refl-antc}. Observe that the combination of a root-final vowel and this suffix results in the phoneme /i(ː)/ \citep[98ff.]{heath:1984} and that the meaning of the verb \example{nᵍaṉḏa-} is more precisely ‘to throw sth. into water’. It is also worth noting that the antipassive\is{antipassive voice} use of the suffix \example{-i} is only “limited to a few verbs” \citep[390]{heath:1984}. \tabref{tab:ch5:antp-refl-antc} also illustrates anti\-pas\-sive-re\-flex\-ive-anti\-cau\-sative syncretism in \ili{Oksapmin} characterised by the prefix \example{t-}. Note that “[c]omplex predicates consisting of a coverb plus a \isi{light verb} are frequently used in \ili{Oksapmin}” \citep[310]{loughnane:2009} and in such complex predicates the voice marking is found on the \isi{light verb}. The choice between the light verbs\is{light verb} \example{de-} and \example{ml-} “depends on the particular \isi{tense} used”, while the use of the \isi{light verb} \example{x-} “is triggered by the presence of certain prefixes”, including \example{t-} \citep[323]{loughnane:2009}. \citet[238ff.]{loughnane:2009} provides only one example of the reflexive\is{reflexive voice} use of the prefix in question yet treats reflexivity\is{reflexive voice} as one of its three main functions, and it is therefore assumed to be productive.

\subsection{Applicative-antipassive-reciprocal} \label{sec:complex-syncretism:appl-antp-recp}
Applicative-antipassive-reciprocal syncretism has hitherto only been attested in the Eskimo language Central Alaskan Yupik\il{Yupik, Central Alaskan} (\lang{na}) in which the syncretism is characterised by the suffix \example{-ut}, as shown in \tabref{tab:ch5:appl-antp-recp}. See also \citet[96ff.]{mithun:2000} for a discussion and examples of the verb \example{ikayur-}. The final phoneme \example{-r} /ʁ/ is omitted before the suffix \example{-ut} as a result of “intervocalic velar deletion” \citep[211f.]{miyaoka:2012}. The antipassive\is{antipassive voice} function of the suffix \example{-ut} appears to have evolved diachronically from the applicative\is{applicative voice} and reciprocal\is{reciprocal voice} functions (\sectref{diachrony:recp2antp}). 

\begin{table}
	\begin{tabularx}{.97\textwidth}{llllll}
		\lsptoprule
		\multicolumn{6}{l}{Central Alaskan Yupik\il{Yupik, Central Alaskan} \citep[656, 844, 915ff., 929, 953, 1091]{miyaoka:2012}} \\
		\midrule
		\textsc{appl} & \example{ner-} & ‘to eat sth.’ & ↔ & \example{ner-\textbf{ut}-} & ‘to eat sth. with sb.’ \\
		\textsc{appl} & \example{kenir-} & ‘to cook sth.’ & ↔ & \example{keni-\textbf{ut}-} & ‘to cook sth. for sb.’ \\
		\textsc{antp} & \example{nalaq-} & ‘to find sth.’ & ↔ & \example{nalaq-\textbf{ut}-} & ‘to find [sth.]’ \\
		\textsc{antp} & \example{ikayur-} & ‘to help sb.’ & ↔ & \example{ikayu-\textbf{ut}-} & ‘to help [sb.]’ \\
		\textsc{recp} & \example{ikayur-} & ‘to help sb.’ & ↔ & \example{ikayu-\textbf{ut}-} & ‘to help e.o.’ \\
		\textsc{recp} & \example{tangrr-} & ‘to see sth.’ & ↔ & \example{tangrr-\textbf{ut}-} & ‘to see e.o.’ \\
		\lspbottomrule
	\end{tabularx}
	\caption{Applicative-antipassive-reciprocal syncretism}
	\label{tab:ch5:appl-antp-recp}
\end{table}

\section{Causative syncretism} \label{sec:complex-syncretism:causative}
Five patterns of complex causative\is{causative voice} voice syncretism\is{voice syncretism, complex} are attested in the language sample (see \tabref{tab:ch5:complex-patterns} on page \pageref{tab:ch5:complex-patterns}), though only three of the patterns have been attested exclusively as type 1 syncretism\is{voice syncretism, full resemblance -- type 1}: causative-passive-reflexive-reciprocal syncretism, causative-passive-anticausative syncretism, and cau\-sa\-tive-ap\-pli\-ca\-tive-passive syncretism. The remaining two patterns of complex causative\is{causative voice} voice syncretism\is{voice syncretism, complex} involve some partial resemblance and therefore represent type 2 syncretism\is{voice syncretism, partial resemblance -- type 2}: causative-passive-reciprocal syncretism and causative-reflexive-anti\-caus\-a\-tive syncretism. Each of these patterns is discussed in the following sections.

\subsection{Causative-passive-*} \label{sec:complex-syncretism:caus-pass}
Complex causative\is{causative voice} voice syncretism\is{voice syncretism, complex} involving both the causative\is{causative voice} and passive\is{passive voice} voices is attested in three languages in the sample: the North Omotic language \ili{Wolaytta} (\lang{af}), the language isolate \ili{Korean} (\lang{ea}), and the Arawakan language \ili{Yine} (\lang{sa}). As already discussed in \sectref{resemblance-type1b}, the former language features causative-passive-reflexive-reciprocal syncretism characterised by the suffix \example{-ett}, as illustrated in \tabref{tab:ch5:caus-pass-refl-recp}. Unlike in the causative\is{causative voice} voice, this suffix can have a high pitch (\example{-étt}) in the passive\is{passive voice}, reflexive\is{reflexive voice}, and reciprocal\is{reciprocal voice} voices. The pitch of the suffix in the latter voices is dependent on the “tonal prominence” of the stem to which it is attached: the allomorph\is{allomorphy} \example{-ett} is found on stems with tonal prominence, whereas the allomorph\is{allomorphy} \example{-étt} is found on stems without tonal prominence \citep[84ff., 1013]{wakasa:2008}. Observe that “[w]hen a base stem ends in a geminated consonant, it is usually reduced to a single consonant” when the suffix \example{-ett} or \example{-étt} is attached \citep[1014]{wakasa:2008}. The “most salient” use of the suffix \example{-ett} is passive\is{passive voice}, yet its reciprocal\is{reciprocal voice} functions appears to be common as well \citep[1022ff.]{wakasa:2008}. In turn, the reflexive\is{reflexive voice} use is rather marginal although “there are indeed examples” in which the suffix is used to “express reflexive\is{reflexive voice} situations” \citep[1028]{wakasa:2008}. It is unclear how productive\is{productivity} the causative\is{causative voice} suffix is, as \citet[1005ff.]{wakasa:2008} simply mentions it alongside other means of marking causativity\is{causative voice} in the language.

\begin{table}
	\begin{tabularx}{\textwidth}{llllll}
		\lsptoprule
		\multicolumn{6}{l}{\ili{Wolaytta} \citep[734, 988, 1008, 1013f., 1022, 1029]{wakasa:2008}} \\
		\midrule 
		\textsc{caus} & \example{Ceegg-} & ‘to become old’ & ↔ & \example{Ceeg-\textbf{ett}-} & ‘to make sth. old’ \\
		\textsc{caus} & \example{bal-} & ‘to err’ & ↔ & \example{bal-\textbf{ett}-} & ‘to make sb. err’ \\
		\textsc{pass} & \example{dóór-} & ‘to pile sth. up’ & ↔ & \example{dóór-\textbf{ett}-} & ‘to be piled up [by sb.]’ \\
		\textsc{pass} & \example{dog-} & ‘to forget sth.’ & ↔ & \example{dog-\textbf{étt}-} & ‘to be forgotten [by sb.]’ \\
		\textsc{refl} & \example{meeCC-} & ‘to wash sth.’ & ↔ & \example{meeC-\textbf{ett}-} & ‘to wash self’ \\
		\textsc{refl} & \example{bonc-} & ‘to respect sb.’ & ↔ & \example{bonc-\textbf{étt}-} & ‘to respect self’ \\
		\textsc{recp} & \example{gílil-} & ‘to tickle sb.’ & ↔ & \example{gílil-\textbf{ett}-} & ‘to tickle e.o.’ \\
		\textsc{recp} & \example{zor-} & ‘to advise sb.’ & ↔ & \example{zor-\textbf{étt}-} & ‘to advise e.o.’ \\
		\lspbottomrule
	\end{tabularx}
	\caption{Causative-passive-reflexive-reciprocal syncretism}
	\label{tab:ch5:caus-pass-refl-recp}
\end{table}

In \ili{Korean} the suffix \example{-(C)i} can serve as voice marking in the causative\is{causative voice}, passive\is{passive voice}, and anticausative\is{anticausative voice} voices, as shown in \tabref{tab:ch5:caus-pass-antc}. This syncretism is particularly interesting from a diachronic perspective because the passive\is{passive voice} and anticausative\is{anticausative voice} functions both seem to have developed from the causative\is{causative voice} function, as further discussed in \sectref{diachrony:caus2antc} and \sectref{diachrony:caus2pass}. 

\begin{table}
	\begin{tabularx}{.95\textwidth}{llllll}
		\lsptoprule
		\multicolumn{6}{l}{\ili{Korean} (\citealt[82f.]{baek:1997};; \citealt[369, 375]{sohn:h-m:1999})} \\
		\midrule 
		\textsc{caus} & \example{cwul-} & ‘to decrease’ & ↔ & \example{cwul-\textbf{li}-} & ‘to decrease sth.’ \\
		\textsc{caus} & \example{nwup-} & ‘to lie down’ & ↔ & \example{nwup-\textbf{hi}-} & ‘to lay sth.’ \\
		\textsc{pass} & \example{kkul-} & ‘to pull sth.’ & ↔ & \example{kkul-\textbf{li}-} & ‘to be pulled [by sb.]’ \\
		\textsc{pass} & \example{mek-} & ‘to eat sth.’ & ↔ & \example{mek-\textbf{hi}-} & ‘to be eaten [by sb.]’ \\
		\textsc{antc} & \example{yel-} & ‘to open sth.’ & ↔ & \example{yel-\textbf{li}-} & ‘to open’ \\
		\textsc{antc} & \example{mak-} & ‘to block sth.’ & ↔ & \example{mak-\textbf{hi}-} & ‘to block’ \\
		\lspbottomrule
	\end{tabularx}
	\caption{Causative-passive-anticausative syncretism}
	\label{tab:ch5:caus-pass-antc}
\end{table}

Finally, as illustrated in \tabref{tab:ch5:caus-recp-pass}, in \ili{Yine} the suffix \example{-ka} serves as voice marking in the passive\is{passive voice} voice and bears only partial resemblance\is{voice syncretism, partial resemblance -- type 2} with the suffix \example{-kaka} found in the causative\is{causative voice} and reciprocal\is{reciprocal voice} voices. \citet[268f.]{hanson:2010} only provides a single example of the reciprocal\is{reciprocal voice} function of the latter suffix, yet her discussion of this function clearly suggests that it is productive\is{productivity}. Diachronically, the former suffix has been linked to both passivity\is{passive voice} and causativity\is{causative voice} \citep{wise:1990}, and the latter suffix to reciprocity\is{reciprocal voice}, \isi{comitativity}, and causativity\is{causative voice} (\sectref{diachrony:recp2caus}).

\begin{table}
	\setlength{\tabcolsep}{2.2pt}
	\begin{tabularx}{\textwidth}{llllll}
		\lsptoprule
		\multicolumn{6}{l}{\ili{Yine} \citep[191, 211, 265, 269ff.]{hanson:2010}} \\
		\midrule 
		\textsc{caus} & \example{-halna} & ‘to fly’ & ↔ & \example{-halna-\textbf{kaka}} & ‘to make sth. fly’ \\
		\textsc{caus} & \example{-himata} & ‘to know’ & ↔ & \example{-himata-\textbf{kaka}} & ‘to make sb. know’ \\
		\textsc{recp} & \example{-hiylaka} & ‘to hit sth.’ & ↔ & \example{-hiylaka-\textbf{kaka}} & ‘to hit e.o.’ \\
		\textsc{pass} & \example{-hiylata} & ‘to kill sb.’ & ↔ & \example{-hiylata-\textbf{ka}} & ‘to be killed [by sb.]’ \\
		\textsc{pass} & \example{-hiçha} & ‘to search for sth.’ & ↔ & \example{-hiçha-\textbf{ka}} & ‘to be searched for [by sb.]’ \\
		\lspbottomrule
	\end{tabularx}
	\caption{Causative-passive-reciprocal syncretism}
	\label{tab:ch5:caus-recp-pass}
\end{table}

\subsection{Causative-reflexive-anticausative} \label{sec:complex-syncretism:caus-refl-antc}
Causative-reflexive-anticausative syncretism has hitherto only been attested in the Northern Chukotko-Kamchatkan language \ili{Chukchi} (\lang{ea}), and is characterised by both full\is{voice syncretism, full resemblance -- type 1} and partial resemblance\is{voice syncretism, partial resemblance -- type 2} in voice marking. In this language the suffix \example{-et/-at} serves as voice marking in the reflexive\is{reflexive voice} and anticausative\is{anticausative voice} voices, as well as in the causative\is{causative voice} voice accompanied by the prefix \example{r-/n-} forming a circumfix. The allomorphs\is{allomorphy} of the prefix are conditioned by its position on the verb: the prefix \example{r-} appears word-initially while \example{n-} appears elsewhere \citep[51]{dunn:1999}. The syncretism in \ili{Chukchi} is illustrated in \tabref{tab:ch5:caus-refl-antc}. Nevertheless, note that the anticausative\is{anticausative voice} function of the suffix is marginal. Indeed, \citet[21]{dunn:1999} argues that it is “not systematic or productive”\is{productivity} and \citet[187]{kurebito:2012} states that there are only three “anticausative\is{anticausative voice} verbs formed by adding the suffix”. Finally, observe that the schwa in the verb \example{r/n-ə-lw-et} is simply epenthetic.

\begin{table}
	\begin{tabularx}{.92\textwidth}{llllll}
		\lsptoprule
		\multicolumn{6}{l}{\ili{Chukchi} (\citealt[256]{dunn:1999}; \citealt[6]{stenin:2017}; \citealt[186f.]{kurebito:2012})} \\
		\midrule 
		\textsc{caus} & \example{lw} & ‘to burn’ & ↔ & \example{\textbf{r/n}-ə-lw-\textbf{et}} & ‘to burn sth.’ \\
		\textsc{caus} & \example{went} & ‘to open’ & ↔ & \example{\textbf{r/n}-went-\textbf{et}} & ‘to open sth.’ \\
		\textsc{refl} & \example{qetw} & ‘to stab sb.’ & ↔ & \example{qetw-\textbf{et}} & ‘to stab self’ \\
		\textsc{refl} & \example{ejup} & ‘to prick sb.’ & ↔ & \example{ejup-\textbf{et}} & ‘to prick self’ \\
		\textsc{antc} & \example{ejp} & ‘to close sth.’ & ↔ & \example{ejp-\textbf{et}} & ‘to close’ \\
		\textsc{antc} & \example{tejwŋ} & ‘to divide sth.’ & ↔ & \example{tejwŋ-\textbf{et}} & ‘to divide’ \\
		\lspbottomrule
	\end{tabularx}
	\caption{Causative-reflexive-anticausative syncretism}
	\label{tab:ch5:caus-refl-antc}
\end{table}

\subsection{Causative-applicative-passive} \label{sec:complex-syncretism:caus-appl-pass}
Causative-applicative-passive syncretism characterised exclusively by full resemblance in voice marking\is{voice syncretism, full resemblance -- type 1} is attested in the language isolate \ili{Kutenai} (\lang{na}). \citet[300]{morgan:1991} argues that the so-called “Transitive-Ditransitive\is{transitive}\is{ditransitive} Suffix” \example{-(i)ɬ} in this language has two functions: a “simple \isi{transitive} function” and a “\isi{ditransitive} function”, qualifying as causative\is{causative voice} and applicative\is{applicative voice}, respectively. Additionally, \citet[301]{morgan:1991} argues that the language has the “Passive Suffix” \example{-(i)ɬ}. Although he makes “a clear distinction” in writing between the causative-appli\-ca\-tive suffix \example{-(i)ɬ} and the passive\is{passive voice} suffix \example{-(i)ɬ}, he admits that they have “the same form” and “it would appear that these two suffixes are related”. The syncretism is illustrated in \tabref{tab:ch5:caus-appl-pass}. \citeauthor{morgan:1991} only provides one example of the causative\is{causative voice} use of the suffix \example{-(i)ɬ}, yet his description discussion of the suffix in question suggests that the causative\is{causative voice} function is indeed productive.\is{productivity} Causative-applicative-passive syncretism in the Panoan language \ili{Chácobo} (\lang{sa}) is slightly different than that found in Kutenai as it is based on both full\is{voice syncretism, full resemblance -- type 1} and partial\is{voice syncretism, partial resemblance -- type 2} resemblance in voice marking. More specifically, in \ili{Chácobo} the suffix \example{-ʔak} serves as voice marking in both the causative\is{causative voice} and applicative\is{applicative voice} voices, while the suffix \example{-ʔaká} serves as voice marking in the passive\is{passive voice} voice. This syncretism is also illustrated in \tabref{tab:ch5:caus-appl-pass}. \citet[644]{tallman:2018} argues that the passive\is{passive voice} suffix likely is historically composed of the causative-applicative suffix \example{-ʔak} and the plural clitic \example{=kán}, noting that /k/ in coda position is “always deleted” while /n/ in coda position is “deleted in most morphosyntactic contexts”. The suffix \example{-ʔak} itself ultimately “seems to be related diachronically to the \isi{transitive} verb root \example{ak} ‘make, do, hit’” \citep[652]{tallman:2018}.

\begin{table}
	\setlength{\tabcolsep}{2.5pt}
	\begin{tabularx}{\textwidth}{llllll}
		\lsptoprule
		\multicolumn{6}{l}{\ili{Kutenai} \citep[291f., 300, 305f., 363, 377]{morgan:1991}} \\
		\midrule 
		\textsc{caus} & \example{ʔup} & ‘to die’ & ↔ & \example{ʔup-\textbf{iɬ}} & ‘to kill sb.’ \\
		\textsc{appl} & \example{haɬuqɬawut} & ‘to fish’ & ↔ & \example{haɬuqɬawut-\textbf{iɬ}} & ‘to fish for sth.’ \\
		\textsc{appl} & \example{qa-kiʔ} & ‘to say sth.’ & ↔ & \example{qa-ki-\textbf{ɬ}} & ‘to say/tell sb. sth.’ \\
		\textsc{pass} & \example{ʔiktuquʔ} & ‘to wash sth.’ & ↔ & \example{ʔiktuquʔ-\textbf{ɬ}} & ‘to be washed [by sb.]’ \\
		\textsc{pass} & \example{pi¢-quwaʔt-iɬ} & ‘to shear sth.’ & ↔ & \example{pi¢-quwaʔt-iɬ-\textbf{iɬ}} & ‘to get sheared [by sb.]’ \\
		\midrule\midrule 
		\multicolumn{6}{l}{\ili{Chácobo} \citep[620, 629, 636, 651ff., 656f., 675]{tallman:2018}} \\
		\midrule 
		\textsc{caus} & \example{yaho} & ‘to shake’ & ↔ & \example{yaho-\textbf{ʔak}} & ‘to shake sth.’ \\
		\textsc{caus} & \example{baha} & ‘to be bright’ & ↔ & \example{baha-\textbf{ʔak}} & ‘to brighten sth.’ \\
		\textsc{appl} & \example{koʃo} & ‘to spit’ & ↔ & \example{koʃo-\textbf{ʔak}} & ‘to spit on sb.’ \\
		\textsc{appl} & \example{ʂoo} & ‘to breathe’ & ↔ & \example{ʂoo-\textbf{ʔak}} & ‘to breathe on sb.’ \\
		\textsc{pass} & \example{rota} & ‘to hang sth.’ & ↔ & \example{rota-\textbf{ʔaká}} & ‘to be hung [by sb.]’ \\
		\textsc{pass} & \example{pi} & ‘to eat sth.’ & ↔ & \example{pi-\textbf{ʔaká}} & ‘to be eaten [by sb.]’ \\
		\lspbottomrule
	\end{tabularx}
	\caption{Causative-applicative-passive syncretism}
	\label{tab:ch5:caus-appl-pass}
\end{table}

\section{Permic and Slavic voice syncretism} \label{sec:complex-syncretism:multiplex}
The most complex pattern of voice syncretism\is{voice syncretism, complex} attested in the language sample is passive-antipassive-reflexive-reciprocal-anticausative syncretism. This kind of syncretism is rare, not only because it is attested in only one language in the sample, but because no other pattern of complex syncretism\is{voice syncretism, complex} involving five (or more) voices has hitherto been attested. The language in the sample featuring the syncretism in question is the Permic language \ili{Udmurt} (\lang{ea}) which has already been mentioned sporadically in the previous chapter. As illustrated in \tabref{tab:ch5:multiplex}, passive-antipassive-reflexive-reciprocal-anticausative syncretism in this language is characterised by the suffix \example{-śk}. As also shown in the table, the closely related language \ili{Komi} features the same kind of syncretism, characterised by the cognate suffix \example{-ś}. \citet[284]{bartens:2000} notes that the antipassive\is{antipassive voice} function of the suffixes in \ili{Udmurt} and \ili{Komi} often is associated with some degree of habituality,\is{habitual} which is not surprising from a cross-linguistic perspective (\citealt{polinsky:2017}). Additionally, the suffixes can in some contexts have a \isi{resultative}-like function (e.g. \ili{Komi} \example{kyvyz-} ‘to hear/listen’ ↔ \example{kyvyz-yś}- ‘to have heard enough’, \citealt[285]{bartens:2000}), and in \ili{Udmurt} the suffix \example{-śk} even serves as a present \isi{tense} marker (e.g. \ili{Udmurt} \example{myn-iśk-omy} ‘we go’, cf. \ili{Komi} \example{mun-am} ‘we go’, \citealt[179ff.]{bartens:2000}). The suffixes have been reconstructed\is{reconstruction} \example{*-śk} for \ili{Proto-Permic}, but the exact development of the many functions of the suffix remains a topic of debate (for an overview of different theories and hypotheses, see \citealt[168ff.]{kozmacs:2003}).

\begin{table}
	\setlength{\tabcolsep}{2.3pt}
	\begin{tabularx}{\textwidth}{llllll}
		\lsptoprule
		\multicolumn{6}{l}{\ili{Udmurt} (\citealt[226f.]{perevoscikov:1962}; \citealt[573]{kirillova:2008}; \citealt[122]{winkler:2011};} \\
		\multicolumn{6}{r}{\citealt[306f., 310ff.]{tanczos:2014})} \\
		\midrule 
		\textsc{pass} & \example{kvaśt-} & ‘to dry sth.’ & ↔ & \example{kvaśt-\textbf{iśk}-} & ‘to be dried [by sb.]’ \\
		\textsc{pass} & \example{uśt-} & ‘to open sth.’ & ↔ & \example{uśt-\textbf{ïśk}-} & ‘to be opened [by sb.]’ \\
		\textsc{antp} & \example{kopa-} & ‘to hoe sth.’ & ↔ & \example{kopa-\textbf{śk}-} & ‘to hoe [sth.]’ \\
		\textsc{antp} & \example{vur-} & ‘to sew sth.’ & ↔ & \example{vur-\textbf{iśk}-} & ‘to sew [sth.]’ \\
		\textsc{refl} & \example{korma-} & ‘to scratch sth.’ & ↔ & \example{korma-\textbf{śk}-} & ‘to scratch self’ \\
		\textsc{refl} & \example{syna-} & ‘to comb sb.’ & ↔ & \example{syna-\textbf{śk}-} & ‘to comb self’ \\
		\textsc{recp} & \example{ćupa-} & ‘to kiss sb.’ & ↔ & \example{ćupa-\textbf{śk}-} & ‘to kiss e.o.’ \\
		\textsc{recp} & \example{dźygyrja-} & ‘to embrace sb.’ & ↔ & \example{dźygyrja-\textbf{śk}-} & ‘to embrace e.o.’ \\
		\textsc{antc} & \example{pytsa-} & ‘to close sth.’ & ↔ & \example{pytsa-\textbf{śk}-} & ‘to close’ \\
		\textsc{antc} & \example{uśt-} & ‘to open sth.’ & ↔ & \example{uśt-\textbf{ïśk}-} & ‘to open’ \\
		\midrule\midrule
		\multicolumn{6}{l}{\ili{Komi} \citep[284f.]{bartens:2000}} \\
		\midrule 
		\textsc{pass} & \example{k’ośav-} & ‘to tear sth. down’ & ↔ & \example{k’ośav-\textbf{ś}-} & ‘to be torn down [by sb.]’ \\	
		\textsc{pass} & \example{vöć-} & ‘to make/build sth.’ & ↔ & \example{vöć-\textbf{ś}-} & ‘to be made/built [by sb.]’ \\
		\textsc{antp} & \example{kyj-} & ‘to hunt sth.’ & ↔ & \example{kyj-\textbf{ś}-} & ‘to hunt [sth.]’ \\
		\textsc{antp} & \example{dor-} & ‘to forge sth.’ & ↔ & \example{dor-\textbf{ś}-} & ‘to forge [sth.]’ \\
		\textsc{refl} & \example{vi-} & ‘to kill sb.’ & ↔ & \example{vi-\textbf{ś}-} & ‘to kill self’ \\
		\textsc{refl} & \example{lyj-} & ‘to shoot sth.’ & ↔ & \example{lyj-\textbf{ś}-} & ‘to shoot self’ \\
		\textsc{recp} & \example{ad’ʒ́-} & ‘to see sth.’ & ↔ & \example{ad’ʒ́-\textbf{yś}-} & ‘to see e.o.’ \\
		\textsc{recp} & \example{jir-} & ‘to bite sth.’ & ↔ & \example{jir-\textbf{ś}-} & ‘to bite e.o.’ \\
		\textsc{antc} & \example{šond-} & ‘to warm sth.’ & ↔ & \example{šond-\textbf{yś}-} & ‘to warm’ \\
		\textsc{antc} & \example{juk-} & ‘to divide/split sth.’ & ↔ & \example{juk-\textbf{ś}-} & ‘to divide/split’ \\
		\midrule\midrule
		\multicolumn{6}{l}{\ili{Russian} (personal knowledge; cf. \citealt[680f.]{knjazev:2007} and \citealt[7f.]{malchukov:2017})} \\
		\midrule 
		\textsc{pass} & \example{stroit’} & ‘to build sth.’ & ↔ & \example{stroit’-\textbf{sja}} & ‘to be built [by sb.]’ \\
		\textsc{pass} & \example{pisat’} & ‘to write sth.’ & ↔ & \example{pisat’-\textbf{sja}} & ‘to be written [by sb.]’ \\
		\textsc{antp} & \example{kusat’} & ‘to bite sth.’ & ↔ & \example{kusat’-\textbf{sja}} & ‘to bite [sth.]’ \\
		\textsc{antp} & \example{bodat’} & ‘to butt sb.’ & ↔ & \example{bodat’-\textbf{sja}} & ‘to butt [sb.]’ \\
		\textsc{refl} & \example{myt’} & ‘to wash sth.’ & ↔ & \example{myt’-\textbf{sja}} & ‘to wash self’ \\
		\textsc{refl} & \example{odevat’} & ‘to dress sb.’ & ↔ & \example{odevat’-\textbf{sja}} & ‘to dress self’ \\
		\textsc{recp} & \example{vstretit’} & ‘to meet sb.’ & ↔ & \example{vstretit’-\textbf{sja}} & ‘to meet e.o.’ \\
		\textsc{recp} & \example{celovat’} & ‘to kiss sb.’ & ↔ & \example{celovat’-\textbf{sja}} & ‘to kiss e.o.’ \\
		\textsc{antc} & \example{slomat’} & ‘to break sth.’ & ↔ & \example{slomat’-\textbf{sja}} & ‘to break’ \\
		\textsc{antc} & \example{zakryt’} & ‘to close sth.’ & ↔ & \example{zakryt’-\textbf{sja}} & ‘to close’ \\	
		\lspbottomrule
	\end{tabularx}
	\caption{\textsc{pass-antp-refl-recp-antc} syncretism}
	\label{tab:ch5:multiplex}
\end{table}

As noted in \sectref{geniusiene-syncretism}, \citet{geniusiene:1987} lists reflexive\is{reflexive voice}, reciprocal\is{reciprocal voice}, “\isi{decausative}” (anticausative\is{anticausative voice}), passive\is{passive voice}, and “absolute” (antipassive\is{antipassive voice}) functions for “suffixes containing \example{-d-} or \example{-z-}” in the Ugric language \ili{Hungarian} (\lang{ea}), which is distantly related to \ili{Udmurt} and \ili{Komi}. However, \citeauthor{geniusiene:1987} does not actually differentiate between the many suffixes that she refers to (e.g. \example{-od}, \example{-oz}, \example{-kod}, \example{-koz}, inter alia) and in reality it seems that no single suffix can serve as voice marking in each of the five voices (for an overview of the various markers and their individual functions, see \citealt{karoly:1982}). The same is true for the prefixes \example{na-} and \example{nɨɨ-} in the Uto-Aztecan language \ili{Shoshoni} (\lang{na}) also briefly discussed in \sectref{geniusiene-syncretism}. In fact, so far it has only been possible to find one other language featuring passive-antipassive-reflexive-reciprocal-anticausative syncretism, the Slavic language \ili{Russian} (\lang{ea}). As illustrated in \tabref{tab:ch5:multiplex} alongside \ili{Udmurt} and \ili{Komi}, in Russian the syncretism is characterised by the suffix \example{-sja/-s’} (see, e.g., \citealt[40ff.]{nedjalkov:silnickij:1969};; \citealt[11f.]{faltz:1985};; \citealt{gerritsen:1990}; \citealt{israeli:1997}; \citealt[902]{kazenin:2001a}; \citealt[680f.]{knjazev:2007};; \citealt[113f.]{malchukov:2015};; \citeyear[7f.]{malchukov:2017}). The diachrony of the syncretism in \ili{Russian} is better known than in the Permic languages, and in the next chapter it is described how the suffix \example{-sja/-s’} ultimately descends from the \ili{Proto-Indo-European} reflexive pronoun \example{*s(u)e} (\citealt[397]{kulikov:2010}; \citeyear[276]{kulikov:2013}). Finally, it might be worth noting that \ili{Udmurt}, \ili{Komi}, and \ili{Russian} are spoken in close proximity to each other, and it is not unlikely that the languages have influenced each other with regard to the functional scope of the voice marking in the respective languages.

\section{Overview} \label{sec:complex-syncretism:overview}
As demonstrated in this chapter, seventeen patterns of complex voice syncretism\is{voice syncretism, complex} have been attested in the language sample (see \tabref{tab:ch5:complex-patterns} on page \pageref{tab:ch5:complex-patterns}). As 99 patterns of complex voice syncretism\is{voice syncretism, complex} can logically be posited given the seven voices of focus in this book, there are thus 82 patterns that currently remain unattested altogether. Furthermore, it is worth noting that two of the seventeen attested patterns of complex voice syncretism\is{voice syncretism, complex} covered in this chapter feature some partial resemblance in voice marking\is{voice syncretism, partial resemblance -- type 2} (type 1 and type 2 syncretism), while the remaining fifteen patterns are characterised exclusively by full resemblance (type 1 syncretism)\is{voice syncretism, full resemblance -- type 1}. For the sake of easy reference, an overview of the various patterns of complex voice syncretism\is{voice syncretism, complex} is provided in \tabref{tab:ch5:overview}. The languages are listed in the same order as they have been discussed in the previous sections, and page numbers provide references to examples. Parentheses in the table indicate type 1b syncretism\is{voice syncretism, full resemblance -- type 1}, and square brackets indicate type 2 syncretism\is{voice syncretism, partial resemblance -- type 2}. 

As briefly mentioned in \sectref{sec:complex-syncretism:middle}, for practical reasons all patterns of \isi{middle syncretism} have not been illustrated for all languages in which they have been attested. For the sake of transparency, it can here be mentioned that passive-reflexive-anticausative syncretism also is attested in the Germanic language \ili{Danish} (\lang{ea}); passive-reflexive-anticausative syncretism also in the Kordofanian language \ili{Lumun} (\lang{af}), the Sino-Tibetan language \ili{Dhimal} (\lang{ea}), and the Panoan language \ili{Chácobo} (\lang{sa}); and passive-reflexive-reciprocal syncretism also in \ili{Páez}, the Athapaskan language \ili{Tanacross} (both \lang{na}), Yauyos Quechua\il{Quechua, Yauyos}, and the language isolate \ili{Mosetén} (both \lang{sa}). In the latter three languages the syncretism involves some partial resemblance\is{voice syncretism, partial resemblance -- type 2}. Finally, reflexive-reciprocal-anticausative syncretism is also attested in the Semitic language Darfur Arabic\il{Arabic, Darfur} (\lang{af}), the South-Central Dravidian language \ili{Telugu} (\lang{ea}), the language isolate \ili{Gaagudju}, the Mangrida language \ili{Gurr-Goni} (both \lang{au}), the Yuman language \ili{Jamul Tiipay} (\lang{na}), the Cariban language \ili{Panare}, the Caribbean Arawakan language \ili{Garifuna}, the Central Arawakan language \ili{Paresi-Haliti} (all three \lang{sa}), and the North Halmaheran language \ili{Ternate} (\lang{pn}). In the latter two languages the syncretism in question involves some partial resemblance\is{voice syncretism, partial resemblance -- type 2}. The voice marking characterising the syncretism in these seventeen languages can be found in Appendix C. 

\begin{table}
	\setlength{\tabcolsep}{3pt}
	\begin{tabularx}{\textwidth}{lccccccccl}
		\lsptoprule
		& Marking & \textsc{refl} & \textsc{recp} & \textsc{antc} & \textsc{pass} & \textsc{antp} & \textsc{caus} & \textsc{appl} & \\
		\midrule
		Armenian\il{Armenian, Eastern} & \example{-v} & + & + & + & + & & & & (p. \pageref{tab:ch5:pass-refl-recp-antc}) \\
		Nahuatl\il{Nahuatl, Huasteca} & \example{mo-} & + & + & + & + & & & & (p. \pageref{tab:ch5:pass-refl-recp-antc}) \\
		\ili{Yeri} & \example{d-} & + & + & + & & & & & (p. \pageref{tab:ch5:middle}) \\
		\ili{Hup} & \example{hup-} & + & + & & + & & & & (p. \pageref{tab:ch5:middle}) \\
		\ili{Kayardild} & \example{-yii/-V} & + & & + & + & & & & (p. \pageref{tab:ch5:middle}) \\
		\ili{Sidaama} & \example{-am} & & + & + & + & & & & (p. \pageref{tab:ch5:middle}) \\
		\ili{Tatar} & \example{-n} & + & & + & + & + & & & (p. \pageref{tab:ch5:pass-antp-refl-antc}) \\
		\ili{Mosetén} & \example{-ki} & & & + & + & + & & & (p. \pageref{tab:ch5:pass-antp-antc}) \\
		Otomí\il{Otomí, Acazulco} & \example{n-} & + & + & + & & + & & & (p. \pageref{tab:ch5:antp-refl-recp-antc}) \\
		\ili{Ese Ejja} & \example{xa-…-ki} & + & + & + & & + & & & (p. \pageref{tab:ch5:antp-refl-recp-antc}) \\
		\ili{Chukchi} & \example{-tku} & + & + & + & & + & & & (p. \pageref{tab:ch5:antp-refl-recp-antc}) \\
		\ili{Cherokee} & \example{at(aa)(t)-} & + & + & (+) & & + & & & (p. \pageref{tab:ch5:antp-refl-recp-antc-3}) \\
		Katukina\il{Katukina-Kanamari} & \example{-i/-k/-hik} & + & + & & & + & & & (p. \pageref{tab:ch5:antp-refl-recp}) \\
		\ili{Mangarrayi} & \example{-yi/-(ñ)jiyi} & + & + & & & + & & & (p. \pageref{tab:ch5:antp-refl-recp}) \\
		\ili{Nunggubuyu} & \example{-nʸji} & + & + & & & + & & & (p. \pageref{tab:ch5:antp-refl-recp}) \\
		\ili{Nunggubuyu} & \example{-i} & + & & + & & + & & & (p. \pageref{tab:ch5:antp-refl-antc}) \\
		\ili{Oksapmin} & \example{t-} & + & & + & & + & & & (p. \pageref{tab:ch5:antp-refl-antc}) \\
		Yupik\il{Yupik, Central Alaskan} & \example{-ut} & & + & & & + & & + & (p. \pageref{tab:ch5:appl-antp-recp}) \\
		\ili{Wolaytta} & \example{-ett/-étt} & + & + & & + & & (+) & & (p. \pageref{tab:ch5:caus-pass-refl-recp}) \\
		\ili{Korean} & \example{-(C)i} & & & + & + & & + & & (p. \pageref{tab:ch5:caus-pass-antc}) \\
		\ili{Yine} & \example{-kaka}, [\example{-ka}] & & + & & [+] & & + & & (p. \pageref{tab:ch5:caus-recp-pass}) \\
		\ili{Chukchi} & [\example{r-/n-}]\example{…-et} & + & & + & & & [+] & & (p. \pageref{tab:ch5:caus-refl-antc}) \\
		\ili{Kutenai} & \example{-(i)ɬ} & & & & + & & + & + & (p. \pageref{tab:ch5:caus-appl-pass}) \\
		\ili{Chácobo} & \example{-ʔak}[\example{á}] & & & & [+] & & + & + & (p. \pageref{tab:ch5:caus-appl-pass}) \\
		\ili{Udmurt} & \example{-śk} & + & + & + & + & + & & & (p. \pageref{tab:ch5:multiplex})\\
		\lspbottomrule
	\end{tabularx}
	\caption{Overview of maximal complex voice syncretism}
	\label{tab:ch5:overview}
\end{table}
