\chapter{Towards an intonational analysis}
\section{Introduction}
This chapter proposes an abstract formalisation of the phonetic observations discussed in the preceding chapters. The proposed formalisms serve a descriptive purpose in the tradition of structuralism and theoretical phonology and may be an informative departure point for models of how language users represent given patterns cognitively. The observations are formalised within the Autosegmental-Metrical (AM) model to enable the comparison to other intonation systems. As we will show, particular aspects of the phonological analyses either do not yield an unambiguous analysis or cannot be implemented into available formalisms. We will discuss possible consequences of these modelling difficulties accordingly. 

The structure of the chapter is as follows: First, we will recapitulate the phonetic observations (\sectref{sec:7.2}). We will then compare the tonal events under discussion to similar patterns in other languages. Based on these comparisons, the rise-falls in Tashlhiyt turn out to be best described as edge tones with a primary association to a prosodic edge and a secondary association to a specific tone bearing unit. Different possibilities are evaluated as to which prosodic edge the tonal events primarily associate to, which symbolic representation and internal structure is most appropriate to represent the tonal events, and to which tone bearing units the tonal events secondarily associate (\sectref{sec:7.3}). While the resulting analysis captures general trends, there remains unexplained variability, which we will subsequently turn to. We will discuss two types of variability found in the present study, both of which pose a problem to the AM model: discrete variability and gradient variability (\sectref{sec:7.4}). Finally, we will summarise the results of the proposed analysis (\sectref{sec:7.5}).

\section{Recapitulation of observations}\label{sec:7.2}
The previous chapters presented data that yielded a number of observations based on qualitative and quantitative phonetic analyses as well as a perceptual evaluation for a subset of these observations. The relevant findings are summarised here.

Chapter 4 presented an instrumental investigation on word stress. There was no compelling evidence that one syllable within a word is phonetically enhanced in comparison to other syllables. Supported by a number of qualitative arguments, these findings were interpreted as evidence for the absence of lexically determined strength relationships. We concluded that there is no word stress in Tashlhiyt.

Chapter 5 presented qualitative observations revealing that questions were characterised by a rise to a pitch peak followed by a fall usually occurring on the last word of the phrase. This pattern was found for yes-no questions and echo questions. Contrastive statements were characterised by a similar tonal contour, i.e. a rise to a pitch peak followed by a fall, usually occurring on the contrasted word. In both contours, the entire trajectory of the rise-fall was realised on a single syllable either on the penult or on the final syllable. The co-occurrence with a syllable indicated a phonetically enhancing function of the tonal event, i.e. the syllable the rise-fall co-occurred with sounded longer and louder. 

Quantitative analyses of production data showed that compared to statements, questions had a higher pitch level and a greater pitch range. Echo questions were found to have a lower pitch level than y/n questions and a higher pitch level than corresponding statements. Statements and questions differed not only in scaling, but also in the tendency to place the tonal event on the penult or final syllable within the word. The rise-fall was more often realised on the final syllable in questions than in statements. Importantly, this was only a probabilistic trend. Both sentence modalities exhibited rise-falls that occurred on either the penult or the final syllable. Mirroring these probabilistic differences, there was a tendency for the pitch peak in questions to be realised later within the syllable than in statements, independent of which syllable it occurred on. 

A subsequent perceptual experiment revealed that listeners made use of both the pitch scaling cues and the discrete location of the tonal event to distinguish questions from statements. Utterances with a higher pitch level were more likely to be rated as questions than utterances with a lower pitch level. Utterances with a pitch peak on the final syllable were more likely to be rated as questions than utterances with a pitch peak on the penult. Alignment differences within the syllable did not affect ratings, however.

Irrespective of sentence modality, the location of the tonal event was dependent on several interacting factors. Generally, the pitch peak was more likely to occur on the final syllable than on the penult. Additionally, syllables with more sonorous nuclei were more likely to co-occur with the pitch peak than syllables with less sonorous nuclei, and heavy syllables were more likely to co-occur with the pitch peak than light syllables. These factors influenced the distribution of pitch peaks significantly, but could not account for all observations by themselves, leaving a certain amount of unexplained variation. There were a number of cases in which a speaker produced two distinctly different pitch peak locations in two different repetitions of the same target word with the same sentence modality.

Finally, Chapter 6 explored the alignment of  pitch peaks when the target word did not contain a vowel or a sonorant consonant. Results showed three different possible realisation patterns: the pitch peak either remained unrealised (no noticeable pitch movement), occurred on the vowel of the preceding word, or occurred on a non-lexical schwa, word-medially or word-finally. The tone-carrying schwa in the latter pattern was shown to exhibit distributional and phonetic properties that are difficult to reconcile with what has been claimed in the literature. The observed schwa is neither a purely phonetic artifact nor a lexically inserted phonological element. It has been argued that schwa is a postlexical unit that enables the realisation of intonational tones.

\section{Analysis of the rise-falls}\label{sec:7.3}
We first turn to the descriptive category into which the investigated tonal events fall. Tashlhiyt exhibits a rise to a local high pitch target followed by a fall in pitch to flag questions and signal contrastive focus. This resembles patterns found for well-documented languages such as Bengali, Greek, Italian, and Moroccan Arabic, among others. Differences in the location of this tonal event have been analysed in different ways for different languages. In the following, we will review how the patterns in Tashlhiyt compare with similar tonal events in other languages. Subsequently, we will explore possible phonological representations.\il{Bengali}\il{Greek}\il{Italian}\il{Arabic (Moroccan)}

In the Autosegmental-Metrical model, there are different representations for a rise-fall, depending on the regularities of the tonal contour across different contexts with which it co-occurs. In varieties of Italian  that have a rise-fall question contour (cf. \citealt{Grice.etal2005ita,SavinoGrice2011}), the pitch peak is taken to be part of the rising pitch accent (represented as an L+H* tone). The pitch accent associates with the head of the intonation phrase, i.e. the stressed syllable of the most relevant unit. The subsequent fall is attributed to a low edge tone (L\%, resulting in a L+H* L\% sequence). In other languages, such as Standard Hungarian (\citealt{Ladd1983,Varga2002}) and Cypriot Greek (\citealt{Arvaniti1998}), a rise to a pitch peak is analysed as an edge tone and is taken to be associated with the edge of the intonation phrase. Its placement can be represented in two ways, either it is specified as simply aligning with prosodic constituents without an association to a tone bearing unit (e.g. with the right edge of a prosodic phrase, following \citealt{PierrBeck1988}) or it has a secondary association to the penult or final syllable (\citealt{Grice.etal2000}). In Standard Greek and Romanian, the pitch peak is placed even further away from the edge, typically on a post-focal lexically strong syllable. It is analysed as a phrase accent, an edge tone of a prosodic constituent smaller than the intonation phrase, secondarily associated to a TBU (represented as H-, see \citealt{Grice.etal2000}). In this case, the pitch peak is neither on the strongest syllable in the phrase, nor close to the edge of the phrase. It marks the head of a post-focal constituent but not of the entire phrase. This high tone is preceded by a low nuclear pitch accent (L*) and followed by a low intonation phrase edge tone (L\%, resulting in a L* H- L\% sequence).\is{pitch accent}\is{edge tone}\is{secondary association}\is{phrase accent} \il{Italian}\il{Greek}\il{Romanian}\il{Greek (Cypriot)}\il{Hungarian} 

The preceding recapitulation identified three descriptive categories (see also Chapter 2): pitch accents, edge tones, and edge tones with an additional association to a TBU. The choice of which descriptive category represents a tonal event best is mainly informed by the temporal alignment of tonal targets with certain segmental landmarks. Patterns of alignment are often taken as direct evidence for a proposed phonological association (\citealt{Arvaniti.etal2000}). 

Proposing the association of a tone to an edge is justified by the consistent alignment of the tone with that edge. Proposing the association of a tone to a TBU is justified by the consistent alignment of the tone with (or close to) that TBU. The analysis of secondarily associated edge tones is often justified by an alternation between a tone aligned with the edge of the phrase and a tone aligned to a structural unit dislocated from the edge. For example, in Grice’s analysis of Palermo Italian\il{Italian (Palermo)} (\citeyear{Grice1995}), a rise-fall in pitch consists of a rising pitch accent (L+H* in Grice’s annotation) followed by low boundary tone (L-). When the final syllable is accented, the subsequent fall does not fully reach a low target. If the accented syllable is not final, however, there is a complete fall towards a low pitch target. This is analysed as an edge tone that additionally associates to the final syllable.\is{pitch accent}\is{edge tone}\is{phrase accent}\is{tonal association}\is{tonal alignment}\is{tone bearing unit}\is{secondary association} 

\subsection{Rise-falls in Tashlhiyt: primary association to prosodic constituents}
The question arises as to which descriptive category characterises rise-falls in Tashlhiyt best. Generally, the high target of the rise exhibits a rather discrete alignment pattern, i.e. it aligns either with the penult or with the final syllable. Crucially, in the presented data, the peak was seldom found somewhere in-between. This discrete alignment is accompanied by the auditory impression of prominence, i.e. the syllable the tone co-occurs with sounds longer and louder. This impression is further supported by quantitative evidence for durational and articulatory adjustments (\citealt{Diercks2011,GordonNafi2012,Grice.etal2015tash}). Segments exhibit greater duration and higher intensity, vowels are articulated more peripherally, and articulatory gestures exhibit larger displacement difference when co-occurring with tones. The consistency in alignment of the pitch peak and the accompanied phonetic enhancement are analysed as phonological association of tones to tone bearing units (as already assumed by \citealt{DE1985}).\is{tonal association}\is{tonal alignment}\is{tone bearing unit}

The tonal events cannot be pitch accents in the sense of a tone or tonal complex associating to metrically strong 
syllables because Tashlhiyt has been shown to lack lexically determined metrical structure. Moreover, there is systematic variation with regard to the discrete association of the tonal complex under investigation across and within words. This further underlines the conclusion that metrical strength is something that is determined at the postlexical rather than the lexical level. The tonal events can neither be simple edge tones that are only primarily associated to the periphery of a prosodic constituent. While the tonal events are somehow attracted to the right edge of the phrase, they are not aligned at a fixed distance from its edge. The tonal events are edge seeking and, at the same time, associated to a tone bearing unit. This resembles phrase accents, as proposed by \citet{Grice.etal2000}. The analysis of the tonal events as hybrid tones is compatible with findings on Moroccan Arabic, a language in close contact with Tashlhiyt. \citet{Hellmuth.etal2015} discuss a rise-fall contour in y/n questions. The high target of the rise is neither consistently aligned with a stressed syllable nor consistently aligned at a fixed distance from the utterance edge. They analysed this contour as an “edge-aligned pitch accent” which associates to the final foot of the rightmost word in the phrase.\is{pitch accent}\is{word stress}\is{edge tone}\is{tonal association}\is{tonal alignment}\is{tone bearing unit}\is{secondary association}\is{phrase accent} \il{Arabic (Moroccan)}

The following analysis for questions in Tashlhiyt can be tentatively proposed: there is a phonologically specified tonal event, which will be preliminarily referred to as a LHL tonal complex. The LHL sequence can be analysed as an intonation phrase edge complex with a secondary association to the penult or final syllable, similar to Standard Hungarian and Cypriot Greek (\citealt{Grice.etal2000}). Which of these two syllables the tonal complex is associated with is, unlike in other languages, not dependent on lexically determined metrically strong syllables. Similar to questions, contrastive statements exhibit an LHL tonal complex, too. Unlike questions, this tonal complex is located on the contrasted word. Again, this tonal event is edge seeking, i.e. it exhibits a general preference for the final syllable. It can be considered as secondarily associated to either the penult or final syllable.\is{tonal association}\is{tone bearing unit}\is{secondary association}\il{Hungarian}\il{Greek (Cypriot)}

The question as to which prosodic constituents the rise-falls in questions and contrastive statements are primarily associated with, still remains. This is related to the question as to whether the LHL complex has to be decomposed into smaller primitives that are potentially associated to different prosodic constituents. Bengali exhibits tonal placement patterns comparable to what has been found for Tashlhiyt (\citealt{HayesLahiri1991}) and its analysis may be insightful for an evaluation of rise-falls under investigation. Hayes and Lahiri’s discuss two intonation contours in which the combination of pitch accents and edge tones of different prosodic constituents combine to form a rise-fall pattern. Y/n questions and statements with utterance-final narrow focus are both characterised by a rise-fall in pitch. In y/n questions, the pitch peak is achieved later in the syllable and has a greater pitch excursion than in corresponding statements. The authors account for this difference by means of edge tones associated with different constituents in the prosodic hierarchy. Hayes and Lahiri propose two levels of prosodic phrasing above the word: a focused element constitutes a phonological phrase (φ), a phrase larger than the word and smaller than the intonation phrase. In addition to the phonological phrase, they propose an intonation phrase (ɩ) which contains one or multiple phonological phrases. \is{pitch accent}\is{edge tone}\is{tonal association}\is{yes-no question}\is{focus} \il{Bengali} 

  \begin{figure}[h!]
  \centering 
   \includegraphics[width=0.8\textwidth]{figures/Figure_7_1.png}
  \caption{Schematic representation of (a) a y/n question and (b) a statement with intonation phrase-final narrow focus in Bengali (\citealt{HayesLahiri1991}), adapted from \citet[136]{Gussenhoven2002}.}
   \label{fig:7.1}
   \end{figure}

In y/n questions, the fall in pitch is represented as a HL complex associated with the right edge of the intonation phrase (H\% L\%, \figref{fig:7.1}a). In corresponding statements, the H is an edge tone associated to the phonological phrase and the L is an edge tone associated to the intonational phrase (Hφ L\%, \figref{fig:7.1}b). This analysis allows them to capture the phonetic similarity of the tonal events by proposing the same sequence of tones. At the same time, it accounts for the phonetic differences in terms of alignment and scaling by assuming different association patterns with the prosodic structure.

Analogously, a very simple prosodic structure can be proposed to account for tonal placement in Tashlhiyt. We propose an intonation phrase which is characterised by certain edge tones (here L\% in statements and LHL\% in questions), a clear pause following the right edge of the phrase, and lengthening of the segments at the right edge. Similar to Bengali, one may propose that a focused constituent in Tashlhiyt constitutes its own phonological phrase. This would resonate with the preliminary analysis of phrase-medial tones in Tashlhiyt proposed by \citet{Grice.etal2011}. Such an analysis may also account for the anticipated tone pattern discussed in Chapter 6. Recall that when no sonorant consonant or vowel is available in the contrasted word (e.g. /inna tkʃf/ ‘Did he say ‘it dried’?’), the tonal complex can be found on a syllable in the preceding word (final syllable of /inna/), i.e. outside the contrasted word. The tonal complex would be analysed as associating with the rightmost available TBU within the same phrase. However, as opposed to the intonation phrase, there is no independent acoustic evidence for a phrase boundary after focused constituents. There is neither an audible pause nor lengthening of segments at the right edge. In fact, vowel coalescence across the proposed edge can be frequently observed. These findings raise doubts as to whether the proposal of a prosodic boundary is phonetically justified. Moreover, proposing a phonological phrase is not necessary to account for the present observations. \is{yes-no question}\is{edge tone}\is{pitch scaling}\is{tonal alignment}\is{tonal association} 

Alternatively, one could analyse the tonal complex in contrastive statements as an edge tone of the phonological word, which is a well-defined prosodic domain in Tashlhiyt (e.g. \citealt{DE2002}). In light of the anticipated tone pattern, the tonal complex must be permitted to surface outside of its word-domain. This is in line with Pierrehumbert and Beckman’s analysis of Japanese \il{Japanese}(\citeyear{PierrBeck1988}), in which they allow for the association of a right edge tone of the accentual phrase with the first TBU of the following accentual phrase (cf. Chapter 2).\is{tone bearing unit}\is{tonal association}

We proceed by postulating an association of the LHL with the phonological word. Regardless of the constituent the tone is associated with, the proposed analysis assumes that questions and contrastive statements are expressed by the same tonal sequences (LHL), which are associated to different prosodic constituents. Similar to Hayes and Lahiri’s analysis for Bengali (\citeyear{HayesLahiri1991}), the present account allows one to capture the phonetic similarity of the tones by proposing the same sequence of tones. At the same time, it accounts for the phonetic differences in terms of alignment and scaling by proposing different association patterns with the prosodic structure. Recall that the LHL in question reaches its high target later within the syllable than the LHL in statements. This difference is captured by the analysis of the tone as associated with the intonation phrase in questions and with the phonological word in contrastive statements. Crucially, this analysis avoids the introduction of additional phonological primitives to account for subtle alignment differences like ‘alignment features’ (\citealt{Remijsen2013}) or ‘supplementary association of tones’ (\citealt{Prieto.etal2005,FacePrieto2007}).\footnote{\citet{Prieto.etal2005} use the term ‘secondary association’. However, as \citet{Arvaniti.etal2006} pointed out, their mechanism differs significantly from the original usage of the term by \citet{PierrBeck1988} and \citet{Grice.etal2000}. In accordance with \citet{Arvaniti.etal2006}, the term supplementary association is used here to avoid confusion.}\is{tone bearing unit}\is{tonal association}\is{tonal alignment}\is{pitch scaling}\il{Bengali}

\largerpage[-1]
\subsection{The internal structure of the rise-fall}
So far the rise-fall in pitch has been analysed as a LHL tonal complex that is either associated to the right edge of the intonation phrase in questions or to the right edge of the phonological word in contrastive statements. Even though the symbolic representation of the discussed tonal events are descriptively well justified by its form (sharp rise to a high target followed by a sudden drop towards a low target), a parsimonious representation may not need this maximal specification. It may be possible, to reduce and/or decompose the tonal complex into separate tonal events. 

To evaluate this possibility, an additional intonation contour of Tashlhiyt is considered. A recent study by \citet{Bruggeman.etal2017} investigated the intonational marking of question words in Tashlhiyt (henceforth referred to as the ‘question word tune’). Question words in a direct interrogative construction co-occurred with a rise-fall in pitch (represented as a LHL sequence). In these contexts, Bruggeman et al. measured the alignment of the high target. While the high target was consistently produced somewhere on the question word, there was no stable alignment with any specific segmental landmark within the word. The onset of the rise consistently occurred in utterance-initial position and was not necessarily located on the question word. The trailing low target was found after the high target near the right edge of the question word. Bruggeman et al. analysed this as an intonation phrase initial \%L and a HL tonal complex marking narrow focus on the question word. They interpreted the HL to be associated with the question word (cf. \figref{fig:7.2}).\is{tonal alignment}\is{focus}\is{edge tone} 

  \begin{figure}[h!]
  \centering 
   \includegraphics[width=0.6\textwidth]{figures/Figure_7_2.png}
  \caption{Schematic representation of tonal association for the question word tune as proposed by \citet{Bruggeman.etal2017} (a) /mani ʁ izˤra ahuli/ ‘Where does he see the sheep?’. (b) /imma managu rad tbdut lχdmt/ ‘So when will you start working?’. Note that the HL is analysed as loosely associated with the question word exhibiting no further association to a syllable.}
   \label{fig:7.2}
   \end{figure}
   
In contrast to the question word tune, the low leading tone of the LHL marking contrastive statements, echo questions, and y/n questions cannot easily be accounted for by an intonation phrase-initial low boundary tone (\%L). The rise to the pitch peak is a sharp rise with the low target located immediately before the pitch peak realised within the same syllable. 

\largerpage[-1]
While a decomposition of the tonal complex appears to capture distributional observations in the question word tune (\citealt{Bruggeman.etal2017}), it is not an attractive formalism for the rise-falls of questions and contrastive statements discussed in Chapter 5. The rise-fall in the question word tune and rise-fall events investigated in this book differ with respect to their tune-text-association. Bruggeman et al. did not find any evidence for an association of the H tone to a constituent below the word level. Distribution of variance was unimodal and there was no indication of phonetic enhancement correlated with peak position. This stands in sharp contrast with the data discussed in Chapter 5 as well as with the data discussed in \citet{Grice.etal2015tash}, where the pitch peak was reached on either the penult or the final syllable and its location was accompanied by phonetic enhancement. The former phenomenon has been analysed as a tonal event loosely associated to the question word, the latter phenomena have been analysed as tonal events associated with tone bearing units. We can thus conclude that the descriptively most justified representation is a tritonal complex (LHL) for both questions and contrastive statements.\is{tonal alignment}\is{focus}\is{edge tone} 

This analysis might be extended to other focus types exhibiting larger focus domains than just single words as well as neutral statements. It is very likely that Tashlhiyt exhibits mechanisms to express different focus types in addition to the contrastive focus investigated in the present work. Impressionistic observations indicate that non-contrastive elements of the utterance are tonally marked by less prominent rise-fall movements, an observation that resembles high pitch accents in Germanic languages: they do not have a large pitch excursion but sound prominent. Besides these impressionistic observations, there are no further insights on focus marking in Tashlhiyt. Speakers reliably use morphosyntax to express information structure, which involves fronting of the focused constituent (cf. Chapters 2 and 5). This fronting often comes with a prosodic break as well as a very prominent rising pitch movement (\citealt{Sadiqi1997,DE2002,MettouchiFleisch2010}). Speakers were rather resistant to traditional methods of eliciting different focus types in absence of morphosyntactic devices. If tonal events indeed mark less prominent focus types (e.g. broad or narrow focus), the present analysis can be taken as a baseline from which future analyses can depart. Less prominent tonal events may be formalised with less complex tonal representations (e.g. H, LH, HL). This would resemble pitch accent distinctions in well-described intonation systems. For example, the distinction between rising peaks and high rising late peaks in German is communicatively relevant and reflected in their phonological representations (H* and L+H* in GToBi, see \citealt{Grice.etal2005ger}).\is{tonal alignment}\is{focus}\is{pitch scaling}\is{pitch accent}\il{German}  

Ultimately, the most adequate analysis of intonation in Tashlhiyt can only be proposed in light of the full inventory of intonational events. It may turn out that certain analytical decisions need to be revised in light of more available data. Leaving these analytical decisions for future research, all discussed representations conceptually share that tones are attracted to certain positions defined by prosodic constituents and their edges, specific structural units, and specific tonal targets in the tonal sequence. This appears trivial in light of the consistent and stable attraction of tonal events towards prominent positions in other languages. However, Tashlhiyt exhibits an intricate interaction between association requirements. This results in an unusual degree of variability and, in turn, makes a phonological analysis of the tune-text-association challenging.
 
\subsection{Secondary association to tone bearing units} 
While the LHL complexes in questions and statements consistently co-occur with certain prosodic constituents (the final word of the intonation phrase for questions, the contrasted word in contrastive statements), the actual location within these constituents is prone to variability. When there is only one sonorant nucleus in the word (a vowel or a sonorant consonant), the tone is consistently located on this element. If there are multiple sonorants, the tone can be realised either on the penult or on the final syllable. 

These findings are in line with impressionistic observations made by \citet{DE1985}: they provide an analysis of optional syllabification in intonation phrase-final position that goes hand in hand with the placement of intonational tones. For example, when the word /igidr/ ‘eagle’ is at the end of a phrase with question intonation, they report a rising f0 contour occurring on the final /r/. In this case, it is analysed as a syllable nucleus which functions as a tone bearing unit (TBU, resulting in /i.gi.dr̩/). Alternatively, the final consonant can “lose” its syllabic status and, in turn, be “annexed” to the previous syllable (resulting in /i.gidr/). In this case, the pitch peak is aligned with the second vowel /i/. They propose an optional rule of prepausal annexation, turning trisyllabic /i.gi.dr̩/ into disyllabic /i.gidr/ with a final complex coda. \is{tone bearing unit}This analysis is supported by Elmedlaoui’s intuitions about syllable count and acceptability of tonal placement. According to Dell and Elmedlaoui, this alternation is only observable when the word has a final sonorant consonant, which could either form the nucleus of a light syllable or be part of a complex coda. They state that “similar observations can be made with other intonations” (\citealt[119f.]{DE1985}).

Under this assumption, a word such as /tugl/ would be monosyllabic if the pitch peak is on /u/ and disyllabic if  the pitch peak is on /l/. Even though this analysis may reflect Elmedlaoui’s native speaker intuitions about syllable count, it is not sufficient to explain some of the observations made by \citet{Grice.etal2015tash} and the production corpus presented in Chapter 5. First, disyllabic words with two vowels have also been observed to alternate with respect to the pitch peak position. For instance, in /ba.ba/ ‘father’, the peak can occur on either of the vowels, at least in contrastive statements. For native speakers, words with two vowels are disyllabic regardless of the position of the pitch peak. Thus, there appear to be cases in which the position of the peak cannot be accounted for by the syllabification of the word. Second, disyllabic words with a final heavy syllable (e.g. /tu.glt/) showed pitch peak alternations, too. According to \citet[120]{DE1985}, prepausal annexation “requires the prepausal syllable to be an open one”. A closed syllable showing this alternation of pitch peaks would make a super complex syllable coda necessary with /glt/ in the coda of monosyllabic /tuglt/. This level of complexity of syllable structure is not supported by any native speaker intuition reported on in the literature. It is concluded that resyllabification cannot account for the spectrum of evidence coherently. Although Dell and Elmedlaoui’s impressionistic observations reflect strong tendencies, the placement of tones is not as clear-cut as they propose. Moreover, their account does not capture tonal placement in words without sonorants. \is{tone bearing unit} 

When there is no sonorant available, tonal placement for both functions exhibit a high degree of variability, with at least three possible realisation patterns: the pitch peak either remains unrealised (no noticeable pitch movement = no surfacing tone), occurs on the vowel of the preceding word (anticipated tone), or occurs on a non-lexical schwa word-medially or word-finally (tone on schwa). These patterns were only found in target words containing neither lexical vowels nor sonorant consonants. These observations suggest a representational difference between syllables with sonorants and syllables without sonorants. This is achieved either by appealing to Dell and Elmedlaoui’s syllabification algorithm (1985, 1996, 2002) or by appealing to alternative syllabification models, such as the epenthetic vowel account proposed by \citet{Coleman2001}. In the former, the difference is captured by differentiating between syllables with sonorant nuclei and those with obstruent nuclei (/tn̩.dm̩/ vs /tb̩.dg̍/). In this case, the no surfacing tone pattern and the anticipated tone pattern are only permitted when the word contains no sonorant nuclei. In the latter account, the epenthetic vowel account, schwa vowels are assumed to occupy syllables that do not contain lexical vowels. The words /tən.dəm/ and /təb.dəg/ only differ in the identity of the coda consonant. In this case, the no surfacing tone and the anticipated tone patterns are only permitted in words with empty nuclei (or schwa nuclei) and obstruents in the coda. Regardless of the underlying subsyllabic analysis, distinctions between sonorants and obstruents have been made for other intonation systems. In their analysis of Japanese, \citet{PierrBeck1988} propose that sonorants but not obstruents can serve as TBUs.\is{tone bearing unit}

\citet{Grice.etal2015tash} accounted for the tonal placement patterns in Tashlhiyt with the following phonological analysis. They assume that the tone bearing unit is a syllable containing a sonorant (either a vowel or a sonorant consonant). The LHL tonal complex (in their analyses the H tone) under discussion is an edge tone seeking secondary association to a TBU in the edge-final word. If there is no TBU in the edge-final word, the tone does not have a secondary association and is only primarily associated to the edge itself. In the absence of a sonorant, the tone simply aligns with an element with enough voicing or energy to make it audible. In this approach, a target word with no lexical vowel or sonorant would require the LHL to align with voiced material as close to the edge as possible. This may result in the no surfacing tone pattern with the LHL tonal sequence not being realised at all. Alternatively, the tone can align with an element with enough voicing or energy to make it audible. This is achieved by an alignment with the rightmost available sonorant in the preceding word or to a schwa, an element that is phonetically prominent enough to bear a pitch movement. This account captures the distribution of tones and does not have to assign any phonological status to schwa co-occurring with the tone. 

While this is a parsimonious analysis with respect to the available data in \citet{Grice.etal2015tash}, it does not account for three observations presented in this book. 

First, there is evidence that the tone in the anticipated tone pattern phonetically strengthens the syllable it co-occurs with, i.e. it exhibits spatio-temporal enhancement of the segments. \is{tone bearing unit}\is{edge tone}\is{secondary association}\is{tonal alignment}\is{tonal association}

Second, schwas co-occurring with tonal events exhibit distributional properties that are difficult to reconcile with a purely phonetic analysis. The anticipated tone pattern and the presence of schwa in a voiceless target word are mutually exclusive. Either there is a schwa and it carries the tone, or the tone is anticipated and there is no schwa. This observation suggests a structural function of schwa. 

Third, the choice of tonal placement strategy in the absence of sonorants appears to be somewhat independent of the actual phonetic material available to carry the tonal movement. In a substantial number of cases, the tone is anticipated even though the final word contains voiced obstruents. These voiced obstruents often surface with vowel-like elements, that are, phonetically speaking, well suited to carry tonal movements from both a production and a perceptual point of view. Moreover, words like /tb̩.dg̍t/ usually surface with multiple schwas (e.g. [təbədəgət]). If the tone is really aligned with the last phonetic element that is able to carry the tone, one would expect the tone to be aligned with the last available schwa. This is not the case. If the tone occurs on a schwa, there is a strong tendency for the tone to co-occur with the schwa between the onset and nucleus of the final syllable (e.g. between /d/ and /g/ in /tb̩.dg̍t/), irrespective of whether there is another schwa following. The alignment of tones in these cases is systematically dislocated from the edge and not as close to the edge as possible.\is{tonal alignment}\is{tonal association}

Given these three observations, the analysis for tonal placement in cases of obstruent-only words as simple edge tones (\citealt{Grice.etal2015tash}) should be revised. The phonetic enhancement of elements occurring with tones should be taken as evidence for the association of tones to TBUs. In that vein, it is proposed that both the anticipated tone and the tone on schwa are instances of tonal association to TBUs (similar to sonorant syllables). \is{edge tone}\is{tone bearing unit}\is{tonal association}In the absence of a lexical TBU in the word (syllable with a sonorant), the tone either associates with a TBU of the preceding word (anticipated tone), or associates with a postlexically triggered schwa (tone on schwa). Alternatively, the tone can be deleted (no surfacing tone).

To sum up, the rise-falls in questions and contrastive statements are analysed as LHL tonal complexes that are associated to the intonation phrase (questions) and to the phonological word (contrastive statements). The tone is analysed as an edge tone that is secondarily associated to a tone bearing unit in the phrase-final word. The tone bearing unit is identified as a syllable with a sonorant, or a schwa element that is inserted into the consonantal string to bear a functionally relevant tonal movement. Consider \figref{fig:7.3} for a schematic illustration of the analysis.\is{edge tone}\is{tone bearing unit}\is{tonal association}\is{secondary association}

\begin{figure} 
  \centering 
   \includegraphics[width=0.8\textwidth]{figures/Figure_7_3.png}
  \caption{Schematised analysis of prosodic structure and tune-text-association for a simple sentence with the focused constituent at the right edge of the intonation phrase (ı). (a) illustrates a question with a LHL intonation phrase edge tone secondarily associated to either the penult or the final syllable. (b) illustrates a contrastive statement with a LHL word edge tone secondarily associated to either the penult or the final syllable.}
   \label{fig:7.3}
   \end{figure}
   
While the phonological analysis proposed here captures our observations in a parsimonious way, it does not offer an account for the vast amount of variability with regard to the association of the LHL to a specific TBU. It does not account for global pitch scaling distinctions between sentence modalities, either. In the following section, these two remaining issues are discussed.
  
\section{Formalising variability}  \label{sec:7.4}
There are two types of variable form-function mappings that pose some difficulties for formal intonational analyses: ‘discrete variability’ and ‘gradient variability’.

‘Discrete variability’ refers to the variability in the frequency of occurrence of mutually exclusive, categorically definable events. Functions can be expressed by different phonetic events that occur with varying probabilities. The intonation system of Tashlhiyt exhibits discrete variability in tonal placement, i.e. categorically definable tonal events (tone on penult or final syllable) are associated with specific structural positions probabilistically (i.e. in x\% of cases the tonal event is associated to the penult, and in y\% of cases the tonal event is associated to the final syllable).

‘Gradient variability’ refers to gradual modulations of a phonetic parameter that may go hand in hand with a gradual modulation in meaning. For example, in English certain phonetic dimensions such as pitch scaling can be modulated to signal emphasis in that small differences in scaling correspond to small differences in emphasis (\citealt{Bolinger1961,Ladd2014}). In addition to signalling gradual meaning differences, they may also signal more discrete meaning differences, e.g. the distinction between questions and statements (see Chapter 2). Often it remains unclear where to draw the line between gradual and discrete distinctions. Tashlhiyt illustrates gradient variability in terms of pitch scaling.\is{pitch scaling}\il{English} 

\subsection{Discrete variability in intonation}  
The presented data suggests that it is not possible to predict the occurrence of an intonational event deterministically. The occurrence of a particular event can only be stated as a probabilistic distribution affected by multiple interacting factors rather than a deterministic rule (or mapping) that applies across the board.

Most phonological models, including the Autosegmental-Metrical model, assume a rigid mapping of form and function. For example in German\il{German}, a high rising pitch accent with a late peak is described as signalling contrastive focus or new information, while a falling pitch accent with an early peak is described as signalling broad focus or given information (cf. \citealt{Grice.etal2005ger,kohler2006,FeryKuegler2008,RitterGrice2015}). These statements suggest a one-to-one mapping of form and function. Such a one-to-one mapping cannot account for discrete variability as it is found, for example, for pitch accent categories in German. \citet{Baumann.etal2015} investigated the intonational encoding of information status in German. Their results revealed that there are specific pitch accents that are often used to encode specific information status. For example, when in nuclear position, new referents are prototypically realised with a high rising accent, and given referents with a falling accent. Most importantly, these form-function mappings were found to be only probabilistic preferences, in that speakers used a certain pitch accent category in the majority of cases, but not always (see also \citealt{Schafer.etal2000,MueckeGrice2014,CangemiGrice2016,Grice.etal.accepted}).\is{pitch accent}\is{focus}\is{information status}

Variable tonal placement in more general terms is not new to phonologists in general and intonational phonologist in particular. For example, \citet{HayesLahiri1991} explicitly mention the high degree of variability in tonal placement for Bengali:\il{Bengali}\il{Swedish}

\begin{quote} Although the generalization that H\textsubscript{p} links to the right edge of the focused P-phrase seems secure to us, we must note that phonetically, there is some variation: the phonetic location of the H\textsubscript{p} peak often occurs one or two syllables before, and occasionally a syllable or two after the ]\textsubscript{p} boundary to which H\textsubscript{p} is linked phonologically. Similar variation in the placement of H\textsubscript{p} has been noticed for Swedish by Bruce (1977). Where differences of focus are to be made precise, the alignment of H\textsubscript{p} can be controlled carefully to do this. But in less guarded speech, there is variation. (\citealt[65]{HayesLahiri1991})
\end{quote}

The authors abstracted away from the observed variability by proposing a single underlying association. In order to make such a leap to a unifying level of abstraction, convincing arguments need to be made. For example, such a proposal can be put forward if variation is gradual in nature, i.e. the tone is located somewhere around the edge with a unimodal distribution of variance. Alternatively, certain factors like tonal crowding or the segmental context may account for the variation in a predictable way (for a discussion, see Chapter 2).\is{tonal crowding}

In Tashlhiyt, however, distributions of tonal alignment are clearly multimodal. Moreover, there appear to be no clear distributional or contextual parameters accountable for some of the variance, i.e. speakers alternate between tonal events on the penult and the final syllable within exactly the same lexical and pragmatic context. A unifying abstraction away from this variability is not empirically justified. Even though, there are statistically preferred mappings (e.g. LHL on the final syllable in questions, LHL on the penult in statements), there is no a priori reason to formalise these mappings in a way that ignores their probabilistic nature. However, in terms of model-theoretical reasons, a deterministic mapping of form and function enables a parsimonious phonological description. In this case, questions could be described as exhibiting a rise-fall on the final syllable and contrastive statements as exhibiting a rise-fall on the penult. Despite it being economic and simple, this description is empirically not adequate.\is{tonal alignment}

While not prominently discussed in the domain of intonation research, discrete variability has been discussed with regard to phenomena of the segmental domain. A prototypical phenomenon that has been argued to exhibit discrete variability is the /t/ deletion in English.\footnote{For exposition purposes, we ignore here that deletion phenomena are most likely not discrete alternations but should rather be conceived of as continuous modulations of gestural coordination with differing degrees of overlap (\citealt{BrowmanGoldstein1986}).} In English, /t/ is variably deleted in complex codas (e.g. \citealt{Guy1980}). This deletion has been argued to apply probabilistically dependent on multiple linguistic (and non-linguistic) factors such as its segmental context and its prosodic position. \citet{Labov1969} formalised these phonological patterns as ordinary rules that can be marked to be optional and can encode the contextual factors that promote or inhibit their application. While Labov’s model only accounted for relative differences (deletion is more likely than no deletion), subsequent authors offered more elaborate mathematical implementations that allow for quantitative predictions. For example, \citet{Sankoff.etal2005} used multivariate stepwise logistic regression to predict application of a specific rule as a function of multiple independent variables. \il{English}

Alternatively, ‘Optimality Theory’ (OT, \citealt{PrinceSchmolensky1993}) proposes that the observed patterns of language are a result of interacting competing constraints. OT offers a formalism that allows the evaluation of alternatives based on ranked preferences. However, as opposed to traditional rule-based models, it allows for the violation of these preferences. The general OT architecture, which assumes non-probabilistic output, has been extended to enable the modelling of discrete variability (see \citealt{Anttila2012}, for an overview). For example, ‘Stochastic Optimality Theory’ introduces numerically weighted constraints that predict discrete outcomes probabilistically (\citealt{Boersma1997,BoersmaHayes2001}). A different approach is taken by the ‘Multiple Grammars Theory’ (\citealt{Kroch1989,Kiparsky1993}) which proposes that the linguistic system of an individual is defined by multiple competing grammars that themselves refer to different rankings of competing constraints. The interaction of these grammars results in discrete variability. The probability of a certain event to occur (e.g. the /t/ being deleted or the tone being associated to the penult) is defined by the number of grammars predicting this particular event divided by the total number of grammars (see e.g. \citealt{Reynolds1994}, and \citealt{Anttila1997}, for an extension of these ideas).\is{Optimality Theory} 

In sum, several phonological formalisms explicitly address probabilistic form-function mappings. In addition to models that evaluate the contribution of competing factors with traditional mathematical inference, there are variants of OT that can model probabilistic variation. The Autosegmental-Metrical model, however, does not offer any mechanism to model variable form-function mappings.

The observed discrete variability poses a threat to those phonological models that assume a rigid one-to-one mapping of form and function. It has to be stressed that Tashlhiyt is not an exception to the norm. Well-studied languages such as German exhibit similar probabilistic mappings of different tonal events expressing the same function (e.g. H+!H* and L+H* expressing given referents, \citealt{Baumann.etal2015}). What makes Tashlhiyt seemingly different from the German case is that it shows a probabilistic association of arguably the same tonal event to different structural units in the utterance.\is{tonal association}\il{German}

\subsection{Gradient variability in intonation}  
After we have explored possible formalisms to capture discrete variability, we now turn to instances of gradient variability. Tashlhiyt uses raised pitch level and greater pitch range to flag questions. This has been observed for other languages such as Bengali (\citealt{HayesLahiri1991}) and Moroccan Arabic (\citealt{Benkirane1998}) among others (see Chapter 5 for an overview). There are generally two ways global scaling differences have been treated in the past. Either these differences were considered phonological or they were considered paralinguistic, and therefore not incorporated in a phonological analysis. We will discuss these in turn. Building on the AM model, there are two different approaches to this issue, which roughly correspond to what \citet{Ladd2008} identifies as ‘intrinsic’ and ‘extrinsic’ factors affecting pitch scaling.\footnote{\citet{Ladd2008} also discusses a third factor affecting pitch scaling: the ‘metrical’ factor. Since the metrical approach models scaling relations across multiple tonal events within the same utterance, it does not contribute to the present analysis which is based on simple utterances containing one intonational event only.}\is{pitch scaling}\is{paralinguistic}\il{Bengali}\il{Arabic (Moroccan)} 

Intrinsic factors refer to pitch scaling differences encoded in the tonal specifications relative to the tonal space (\citealt{BeckmanPierr1986,Sosa1999}). For example, in his work on Spanish\il{Spanish}, \citet{Sosa1999} argues for incorporating pitch-scaling differences in the tonal string itself. He proposes H+H* for an extra high nuclear accent in questions, as opposed to a L+H* in corresponding statements. Certain local tonal events may trigger the lowering or raising of following tones. For example, Sosa analyses y/n questions as having an initial high boundary tone (e.g. \%H) which triggers ‘up scaling’ of the rest of the tonal string, i.e. the pitch height of all subsequent tones in the utterance is raised.\is{pitch scaling}\is{yes-no question} 

Extrinsic factors refer to overall modifications of the tonal space that can be incorporated by adding representation, for example, a ‘register tier’. This register tier is orthogonal to the tonal tier, i.e. H and L tones can be in an upper or a lower register. \citet{Yip1989} proposed such a register tier to account for contrasts between high falls and low falls in Mandarin and Cantonese. \citet{Snider1999} similarly introduced it to account for up scaling in different tone languages. \citet{InkelasLeben1990} applied a register tier to capture the extra high tone at the end of questions in Hausa\il{Hausa}. Similarly, \citet{Ladd1983} proposed scaling features (e.g. [raised peak]) that are associated to the tones to account for scaling asymmetries in English.\is{pitch scaling}\is{yes-no question} \il{Mandarin}\il{Cantonese}\il{English}

As Ladd discusses (2008: 308f.), intrinsic and extrinsic factors should not be seen as mutually exclusive or applicable across the board. The question is not which analysis is best suited cross-linguistically and across different phenomena. The question is which analysis is most parsimonious and empirically adequate for the linguistic system under investigation. In Tashlhiyt, pitch scaling appears to be a relevant parameter in production. Y/n questions exhibit overall higher pitch values than echo questions which exhibit higher values than statements. In perception, listeners make use of this information to distinguish questions from statements. Pitch scaling should therefore be considered a meaningful phonetic parameter and should be incorporated into any phonological analysis.\is{pitch scaling}\is{yes-no question}\is{echo question} 

An intrinsic account can capture the observed differences between statements and questions by proposing an initial high edge tone that raises following tones (\%H). For the distinction of questions types, an additional level must be assumed. A tripartite distinction could be formalised by two different initial high edge tones that raise following tones to different degrees (\%H and \%\textasciicircum H). The model-theoretical advantage of an intrinsic account is that the phonological specification does not have to be enriched by additional primitives (register tier or features). The description would solely rely on a tonal tier including left edge tones triggering different degrees of pitch scaling. However, this account increases the inventory of tonal events significantly. An extrinsic account, on the other hand, would add either orthogonal register features (‘high’ for echo questions and ‘super high’ for y/n questions) or an additional register tier with two different register specifications for questions. The model-theoretical advantage of an extrinsic account is that the inventory of tonal events remains small and no additional up scaling mechanisms need to be proposed.\is{edge tone}\is{pitch scaling}\is{yes-no question}\is{echo question} 

Given the amount of symbolic representations necessary to account for a simple observation, both accounts appear to be cumbersome. If new data uncovers evidence for even more levels of global pitch scaling, the formalism needs to be adjusted accordingly, inflating the inventory of functional elements.

To circumvent the enrichment of phonological primitives, pragmatic differences between echo and y/n questions could be outsourced and treated as paralinguistic, i.e. not phonological (e.g. \citealt{Pierr1980,Bolinger1989}). Paralinguistic aspects of speech are considered to deal with interpersonal interaction and the speaker’s current emotional state (\citealt{Ladd1983}). The distinction between an echo question and a y/n question could thus be conceived of as a paralinguistic distinction referring to the speaker’s level of surprise towards the proposition (among other things). However, as is discussed in detail by \citet{Ladd2008,Ladd2014} and others, the distinction between paralinguistic and linguistic meaning in intonation is not clear-cut. One argument for a contrast being a linguistic contrast lies in the quantitative nature of its form-function mapping. Paralinguistic contrasts are considered to scale linearly with the function they express. If raising pitch level can signal emphasis, raising the pitch level even higher can signal even more emphasis. Linguistic contrasts, on the other hand, are considered to have a non-linear mapping of form and function with instances of the contrast falling in either one or the other category. This distinctions is, however, empirically difficult to uphold. \citet{GussenhovenRietveld2000} have shown that high rises and low rises in Dutch are non-linearly perceived as discretely different contours. This distinction is used to signal surprise, a function clearly related to aspects of interpersonal interaction and therefore traditionally associated with paralinguistic functions. As opposed to that, unambiguous linguistic contrasts, such as the distinction between questions and statements, have been shown to exhibit linear, non-categorical properties as demonstrated experimentally by e.g. \citet{LaddMorton1997}.\is{yes-no question}\is{echo question}\is{paralinguistic}\il{Dutch}

\largerpage
It can be concluded (e.g. \citealt{Ladd2008}) that there is no clear boundary between language and paralanguage when dealing with intonational phenomena. Shifting the contrast signalled by pitch scaling into the domain of paralanguage is not empirically justified. Such an exclusion can be motivated by model-theoretical consideration and the desire to have a parsimonious model, i.e. a model with a small inventory of symbolic primitives. Based on statistical distributions of acoustic parameters, however, no such choice can be vindicated. 

Similar to discrete variability, gradient variability in terms of pitch scaling is difficult to incorporate into the AM framework in a straightforward way. Even though these types of variability are ubiquitous in intonational systems, their formalisation remains a problem in intonational phonology.

\section{Summary}\label{sec:7.5}
The present chapter discussed the phonological analysis of intonational patterns in Tashlhiyt within the Autosegmental-Metrical model. The tonal events in questions and contrastive statements have been analysed as LHL tonal complexes that are associated to the intonation phrase (questions) and to the phonological word (contrastive statements). The tones have been analysed as edge tones that are secondarily associated to a tone bearing unit in the respective domain. The tone bearing unit has been identified as a syllable with a sonorant nucleus or a schwa that is inserted into the consonantal string to bear a functionally relevant tonal movement. While these analytical choices resemble analyses of comparable phenomena in other languages, certain aspects of it turned out to remain ambiguous and open for alternative interpretations. Some of these aspects of the analysis may become disambiguated by further observations in the future. Other aspects may remain ambiguous since they are mainly determined by model-theoretical considerations of parsimony and simplicity.\is{tonal association}\is{edge tone}\is{secondary association}\is{tone bearing unit} 

Yet other aspects of the analysis could not be accounted for within the AM model. The investigated intonation contours are characterised by discretely definable tonal events that are probabilistically associated to structural units. This type of discrete variability cannot be incorporated within an AM analysis. Orthogonal to that, Tashlhiyt exhibits multiple functionally relevant levels of pitch scaling. Even though the AM model offers different formalisation mechanisms for scaling distinctions, the available formalisms appear rather cumbersome. In these cases, the number of analytical primitives chosen by the analyst is mainly informed by model-theoretical considerations.\is{pitch scaling} 


