\chapter{Hic sunt dracones}

\begin{flushright}
\textit{“Stress and intonation in Tashlhiyt are still terrae incognitae.”} 

François Dell and Mohamed Elmedlaoui (2002)
\end{flushright}
\vspace{5mm}

Urban legends claim that English cartographers placed the phrase “here be dragons” (Latin: hic sunt dracones) at the edges of their known world.\footnote{However, there is only one known map that actually carries this phrase (the Lenox Globe, da Costa, 1879).}Ancient maps are often found to depict fable creatures, sea serpents, and exotic animals in unknown areas. The present book explores the unknown territories referred to in the quote by François Dell and Mohamed Elmedlaoui. Tashlhiyt, a Berber language spoken in South Morocco, can indeed be considered an exotic case on the linguistic-typological map. The language is characterised by exceptional phonotactic patterns, allowing for whole utterances without a single phonological vowel. These structures make an analysis of prosodic aspects in Tashlhiyt, notably those involving pitch, both interesting and challenging. We take on this challenge and shed light on the as of yet under-explored areas of stress and intonation in Tashlhiyt. With the presented analyses, we aim to make three contributions to the literature.

First, this book presents a quantitative exploration of stress and intonation in Tashlhiyt. The results will contribute to a currently small body of instrumental studies on sound patterns in Tashlhiyt in particular and in Berber languages in general. Apart from impressionistic observations, stress and intonation are mostly neglected in available descriptions of Berber languages. For instance, \citet{KossmannStroomer1997} do not even mention the terms “stress” or “intonation” in their concise overview of Berber phonology. 

Second, the findings presented here contribute to intonational typology by supplementing our knowledge of intonation systems with data from a Berber language. Intonation has long been neglected in the study of languages in general. While there has been a noticeable increase in interest within the last three decades, intonation is still under-represented in linguistic descriptions leaving the majority of documented languages under-documented with regard to intonation. While there are a great number of intonational descriptions for most European languages, systematic instrumental investigations of less documented languages are still rare. To understand how intonation systems differ from each other and what generalisations can or cannot be made across languages, thorough descriptions of typologically diverse languages are necessary. The linguistic system of Tashlhiyt exhibits two structural properties that make it an important case study for suprasegmental description and analysis. On the one hand, Tashlhiyt has been described as lacking word stress\is{word stress} and, on the other hand, it allows for phonotactic patterns that are adverse to the production and perception of pitch. 

In well-investigated languages, for instance West Germanic languages, certain tonal events often co-occur with lexically stressed syllables. Investigations of languages that do not exhibit word stress are very rare. The question arises as to how well the generalisations about intonation based on languages with word stress hold for languages without such word-prosodic patterns. Since early description, Tashlhiyt has been argued to lack lexically determined word stress by numerous authors. We will evaluate this claim and investigates the placement of intonational events, contributing to a small body of literature on intonation in languages without word stress.

In addition to the important role of word prosody, intonation cannot be understood in isolation, disregarding its segmental structure. Intonational pitch movements – the pragmatically relevant variation of the rate of vocal fold vibration - is superimposed onto segments. For the realisation of intonational pitch movements, certain articulatory and perceptual requirements need to be met. Most importantly, the segments on which the pitch movements are realised need to be voiced to enable pitch modulation in the first place. This is not a trivial requirement. Most languages are characterised by a systematic occurrence of vowels within words. Each word (or syllable) has at least one element of high intensity and rich harmonic structure enabling the realisation of pitch movements and the perceptual retrieval of pitch. As opposed to that Tashlhiyt exhibits exceptionally rare phonotactic patterns. There are whole utterances without a phonological vowel and words can be comprised of voiceless segments only. In these cases, the phonetic opportunity for the execution of intonational pitch movements is exceptionally limited. The present book explores how these typologically rare phonotactic patterns interact with intonational aspects of linguistic structure. Tashlhiyt turns out to be an intriguing case study of how an intonation system can accommodate to adverse phonological environments.

Third, in addition to descriptive and typological contributions, the intonational patterns in Tashlhiyt turn out to be of great importance for evaluating existing intonation models in particular and phonological theories in general. The observed patterns in Tashlhiyt exhibit an unusual high degree of variability. This variability is probabilistic in nature. Discretely definable tonal events occur in different locations with a certain likelihood, but are never fully predictable in a deterministic way. Current models of intonation do not offer concrete formalisation mechanisms for these probabilistic distributions. We will argue that this variability reflects different ways to resolve functional conflicts between the necessity to express tonal movements in privileged positions and the lack of phonetic material to realise these tonal movements. The discussion of tonal placement patterns in Tashlhiyt will contribute to our understanding of the interaction between intonation and segmental phenomena and will shed new light on the applicability of current intonation models. 

The present book is organised as follows: chapter 2 will introduce relevant concepts and terminology of intonation and prosodic theory within the conceptual framework of the ‘Autosegmental-Metrical’ model, the currently most commonly used formal apparatus to describe intonation.

Chapter 3 will introduce Tashlhiyt Berber. First, we will discuss the historical roots of the Berber people in North Africa and Morocco. Then, we will introduce the speaker community on which the results of this book rest upon: Tashlhiyt speakers living in Agadir. Subsequently, we will sketch the linguistic system of Tashlhiyt including basic word order, morphological categories and, particularly relevant, phonological structures. During the discussion of the latter, we will introduce relevant findings on phonotactic patterns and syllable structure.

Following these introductory chapters, three empirical studies will be presented: chapter 4 will explore the possibility of Tashlhiyt having word stress. We will show that, as opposed to recent claims, there is no empirical evidence for the existence of lexically determined metrical structures within the word. \is{word stress}

Chapter 5 explores the intonational marking of two communicative functions: flagging questions and marking contrastive constituents. We will show that the intonational expression of these functions resemble each other to a certain degree but differ systematically. Both functions are expressed by specific tonal events that differ with respect to pitch scaling and their temporal alignment with the phrase. Moreover, these differences in production are evaluated in a perception experiment. We will show that both scaling and alignment are relevant perceptual cues to distinguish ambiguous syntactic constructions. Aside from systematic interactions between tonal placement and communicative function, the placement of tonal events exhibits an unusual degree of variability. While there are several competing factors affecting tonal placement in a systematic way, there remains a significant amount of unexplained variability. Crucially, the placement of tonal events is sensitive to segmental characteristics such as sonority and syllable weight.\is{pitch scaling}\is{tonal alignment}\is{syllable weight}

Chapter 6 will expand on these findings and explore the realisation of tonal events in light of Tashlhiyt’s exceptional phonotactic flexibility. The investigation will focus on tonal placement in words that are comprised of only voiceless obstruents. In these cases, tonal placement exhibits a high degree of variability and interacts with the segmental level in intricate ways. We will argue that patterns of tonal placement may be informative for an evaluation of the linguistic status of particular segmental structures.

Chapter 7 will summarise the aforementioned instrumental observations and will attempt to account for the results with a phonological analysis within the Autosegmental-Metrical model of intonation. This analysis will allow a parsimonious formalisation of the majority of observations. However, we will show that intonational patterns in Tashlhiyt present certain challenges to the current Autosegmental-Metrical model.

Finally, Chapter 8 will recapitulate the main contributions of this book, situating these in the context of cross-linguistic observations, and will discuss possible theoretical implications. We will conclude by pointing out potential avenues for future research.

