\chapter{Relative clauses} \label{chap:relatives}
\section{Introduction}
This chapter, building on previous work by \citet{jackson06guanxiju, jacksonlin07} on Tshobdun and Situ, and on \citet{jacques16relatives} on Japhug, presents an overview of relativizing constructions in Japhug. 

Three different ways of classifying relative clauses are outlined, respectively based on the form of the verb and the structure of the clause (§\ref{sec:relative.subtypes}), on the position of the relativized element (§\ref{sec:position.head.relative}) and on its function in the relative clause (§\ref{sec:function.relativization}). 

Various morphosyntactic phenomena specific to relative clauses (including resumptive pronouns, case marking etc) are discussed in §\ref{sec:relative.morphosyntactic.specificities}, and some potential ambiguities between relative and complement clauses are analyzed in §\ref{sec:relative.complement.ambiguities}.

Two sections are devoted to the functions of relative clauses other than noun modification: focalization (using pseudo-clefts, §\ref{sec:pseudo.cleft}) and quantification (universal quantification and expression of indefiniteness, §\ref{sec:headless.relatives.quantification}).

Since an important proportion of relative clauses in Japhug are participial clauses, section §\ref{sec:participles} in a previous chapter partially overlaps with some of the sections of this chapter.

\section{Subtypes of relative clauses} \label{sec:relative.subtypes}
\subsection{Participial relative clauses} \label{sec:participial.relatives}
\is{relative clause!participle}
Participles (§\ref{sec:participles}) are the main way of building relative clauses in Japhug. Three types of participles are found: subject \forme{kɯ-} (§\ref{sec:subject.participles}), object \forme{kɤ-} (§\ref{sec:object.participle}) and oblique \forme{sɤ-/z-} (§\ref{sec:oblique.participle}).

The subject participle (§\ref{sec:subject.participle.subject.relative}) is the only available strategy to relativize subjects (whether from intransitive §\ref{sec:intr.subject.relativization} or transitive verbs §\ref{sec:tr.subject.relativization}) and possessor of subjects (§\ref{sec:S.possessor.relativization}).

The object participle is used to relativize objects and semi-objects (§\ref{sec:object.participle.relatives}). It competes in this function with finite relative clauses (§\ref {sec:object.relativization}, §\ref{sec:semi.object.relativization}). It can also marginally relativize locative arguments (§\ref{sec:object.participle.other.relative}, §\ref{sec:locative.relativization}) and possessor of objects (§\ref{sec:O.possessor.relativization}). Unprefixed object participles in \forme{kɤ-} are not easily distinguishable from subject participles of passive forms \forme{kɯ-ɤ\trt}, due to vowel contraction (§\ref{sec:object.participle.ambiguity}).
 
The oblique participle can relativize many non-core functions, including locative adjuncts (§\ref{sec:locative.participle.relatives}), instruments (§\ref{sec:instrumental.participle.relatives}) and other adjuncts (§\ref{sec:other.oblique.participle.relatives}), and compete in these functions with prenominal finite relatives (§\ref{sec:locative.relativization}, §\ref{sec:time.relativization}).
 
Since participial relative clauses are described in detail in chapter \ref{chap:non-finite}, the reader is referred to the relevant sections in that chapter for a focused discussion.
 
\subsection{Finite relative clauses} \label{sec:finite.relatives}
\is{relative clause!finite}
Some relative clauses in Japhug have a verb in finite form, with person indexation and TAME marking. For instance, the verb \forme{pɯ-mto-t-a} in \textsc{1sg}\fl{}3 (§\ref{sec:indexation.mixed}) Aorist form of the headless relative clause in (\ref{ex:aj.pWmtota.nW}) could stand on its own as a complete sentence meaning `I saw/have seen it'. The only clue that this appositive clause is a relative is the presence of the determiner \forme{nɯ}, whose status in this context is discussed in §\ref{sec:relative.determiners.complementizer}.

\begin{exe}
\ex \label{ex:aj.pWmtota.nW}
 \gll ma ɯʑo ɯ-mdoʁ nɯnɯ, [aj pɯ-mto-t-a] nɯ, \\
 \textsc{lnk} \textsc{3sg} \textsc{3sg}.\textsc{poss}-colour \textsc{dem} \textsc{1sg} \textsc{aor}-see-\textsc{pst}:\textsc{tr}-\textsc{1sg} \textsc{dem} \\
 \glt `[The wolves] that I have seen, the colour [of their fur is brown]' (27-spjaNkW)
\japhdoi{0003704\#S21}
\end{exe}

Finite relative clauses are not however completely identical to the corresponding independent clauses, even when the head is internal (§\ref{sec:head-internal.relative}). Four main morphosyntactic differences can be observed.

First, the verb of finite relative clauses can be subjected to totalitative reduplication (§\ref{sec:totalitative.relatives}), a morphological category that is restricted to relative clauses, and not found in main clauses or other types of complement clauses.

Second, possessor prefixes in finite relative clause can undergo possessor neutralization (§\ref{sec:relative.possessor.neutralization}).

Third, some TAME categories are not allowed in relative clauses, including Modal categories such as Irrealis (§\ref{sec:irrealis}) and Imperative (§\ref{sec:imperative}) and Evidential categories such as Inferential (§\ref{sec:ifr}), Sensory (§\ref{sec:sensory}) and Egophoric Present (§\ref{sec:egophoric}). The TAME categories attested in finite relatives are Imperfective (§\ref{sec:imperfective}), Factual Non-Past (§\ref{sec:factual}), Aorist (§\ref{sec:aor}, as in \ref{ex:aj.pWmtota.nW} above) and Past Imperfective (§\ref{sec:pst.ifr.ipfv}).

Fourth, there are some restrictions on person indexation in finite relative claus\-es. Monotransitive verbs are not attested in the corpus in local forms, and in inverse non-local forms (§\ref{sec:monotransitive.object.relativization}). Only \textit{ditransitive} verbs allow inverse and local person forms in relatives.

Finite relative clauses are not compatible with subject relativization, but they can be used when relativizing objects (§\ref{sec:object.relativization}), semi-objects (§\ref{sec:semi.object.relativization}), possessors of objects (§\ref{sec:O.possessor.relativization}), goals (§\ref{sec:locative.relativization.finite}), objects embedded in object complement clauses (§\ref{sec:out.complement.relativization.tr}) and some time adjuncts (§\ref{sec:time.relativization}).

\subsection{Genitival relative} \label{sec:genitival.relatives}
\is{relative clause!genitive} \is{genitive!relative clause}
Both participial and finite prenominal relatives (§\ref{sec:prenominal.relative}) are attested with a genitive marker \forme{ɣɯ} (§\ref{sec:gen.other}) before the head noun, which generally does not take a possessive prefix in this construction. This type of construction is common in texts translated from Chinese, and is certainly due to calquing from Chinese in many cases, for instance in (\ref{ex:kurAZi.GW.relative}).  

\begin{exe}
\ex \label{ex:kurAZi.GW.relative}
 \gll [ɯʑo ku-rɤʑi] ɣɯ \textbf{si} ɯ-pa nɯtɕu jɤ-azɣɯt-nɯ \\
 \textsc{3sg} \textsc{ipfv}-stay \textsc{gen} tree \textsc{3sg}.\textsc{poss}-down \textsc{dem}:\textsc{loc} \textsc{aor}-arrive-\textsc{pl} \\
\glt `They arrived at the bottom of the tree where he was staying.' (140512 alibaba-zh)
\japhdoi{0003965\#S22}
\end{exe}
 
A similar tendency to calque Chinese relative constructions with the genitive is observed in other Gyalrongic languages \citep{lai18genitivization}.  For this reason, examples of genitival relative clauses from translated stories will not be taken into account in this chapter. 

In non-translated texts and in conversations, prenominal genitival clauses are also marginally attested to relativize core arguments, as in (\ref{ex:WkWnWsmAn.GW.smAnba}), though this is uncommon. Finite prenominal genitival clauses are however often used for locative and temporal adjunct relativization (§\ref{sec:locative.relativization.finite}).

\begin{exe}
\ex \label{ex:WkWnWsmAn.GW.smAnba}
 \gll [tɯ-ɕɤrɯ ɣɯ sm..., ɯ-kɯ-nɯsmɤn] ɣɯ \textbf{smɤnba} ŋu. \\
 \textsc{indef}.\textsc{poss}-bone \textsc{gen} \textit{incomplete word} \textsc{3sg}.\textsc{poss}-\textsc{sbj}:\textsc{pcp}-treat \textsc{gen} doctor be:\textsc{fact} \\
 \glt `It is a doctor who treats bone [fractures].'  (140426 laXthAB)
 \japhdoi{0003810\#S34}
 \end{exe}
  
This type of clause should not be confused with superficially similar constructions such as that of com\-ple\-ment-taking nouns (§\ref{sec:complement.taking.IPN}), and with true genitival constructions with a headless relative clause possessor, as in (\ref{ex:WkWGAwu.GW.tWrNa}), where the noun following the genitive (\forme{tɯ-rŋa} `one's face') is not an argument of the relative clause.
 
\begin{exe}
\ex \label{ex:WkWGAwu.GW.tWrNa}
\gll [ɲɯ-kɯ-ɣɤwu] ɣɯ tɯ-rŋa nɯ tsa ɲɯ-fse tɕe \\
 \textsc{ipfv}-\textsc{sbj}:\textsc{pcp}-cry \textsc{gen} \textsc{genr}.\textsc{poss}-face \textsc{dem} a.little \textsc{sens}-be.like \textsc{lnk} \\
 \glt `It looks a bit like the face of someone crying.' (18-qromJoR)
\japhdoi{0003532\#S97}
\end{exe}
  
\subsection{Relator nouns} \label{sec:Wspa.relative}  
\is{relative clause!relator noun} \is{relator noun!relativization}
Three inalienably possessed nouns, \japhug{ɯ-stu}{place},  \japhug{ɯ-sta}{place} and \japhug{ɯ-spa}{material}, are semantically bleached when used as head of prenominal relatives (§\ref{sec:prenominal.relative}), and are in the process of becoming grammaticalized as relativizers. 

These three nouns are fossilized oblique participles (\tabref{tab:spa.sta.stu}, §\ref{sec:lexicalized.oblique.participle}). The former two are used to relativize locative adjuncts, more exceptionally goals (§\ref{sec:Wstu.relativization.subject}), while  \japhug{ɯ-spa}{material} serves to build  instrument (§\ref{sec:instrument.relativization}) and in some cases object relative clauses.
 
The noun \forme{ɯ-spa} still preserves its original meaning in examples such as (\ref{ex:kACphAt.Wspa}), from which its other uses derive.
 
\begin{exe}
\ex \label{ex:kACphAt.Wspa} 
\gll tɯ-jaʁ ʁe nɯ kɯ [kɤ-ɕpʰɤt] \textbf{ɯ-spa} nɯ pjɯ́-wɣ-sɯ-stʰoʁ ŋu \\
\textsc{genr}.\textsc{poss}-hand left \textsc{dem} \textsc{erg} \textsc{obj}:\textsc{pcp}-patch \textsc{3sg}.\textsc{poss}-material \textsc{dem} \textsc{ipfv}-\textsc{inv}-\textsc{caus}-press be:\textsc{fact} \\
\glt `One presses with the left hand on the piece of cloth that is to be patched.' (12-kAtsxWb)
\japhdoi{0003486\#S33}
\end{exe}

From the purposive object relative meaning `(material) that is to be $X$ed' as in (\ref{ex:kACphAt.Wspa}), an instrumental interpretation `(material) that is used to $X$' arose: example (\ref{ex:paR.kAmbi.Wspa}) illustrates the semantic proximity between oblique instrumental relative clauses (§\ref{sec:instrumental.participle.relatives}) and the \forme{ɯ-spa} clauses. Purposive clauses in \forme{ɯ-spa} (§\ref{sec:purposive.clauses}) also derive from this type of constructions.

\begin{exe}
\ex \label{ex:paR.kAmbi.Wspa} 
\gll  [paʁ ɯ-sɤ-χsu] ŋu, [paʁ kɤ-mbi] \textbf{ɯ-spa} ŋu \\
pig \textsc{3sg}.\textsc{poss}-\textsc{obl}:\textsc{pcp}-feed be:\textsc{fact} pig \textsc{obj}:\textsc{pcp}-give \textsc{3sg}.\textsc{poss}-material be:\textsc{fact} \\
\glt `It is pig fodder, it is [something that can] be given to pigs.' (150822 laoye zuoshi zongshi duide-zh)
\japhdoi{0006298\#S170}
\end{exe}  

The relator nouns differ from other nouns serving as heads of relative clause in that they can occur even when an overt non-generic head noun is present, such as \japhug{kʰɯna}{dog} in (\ref{ex:kAtsWm.Wspa.khWna}) and \japhug{kɯspoʁ}{hole} in (\ref{ex:BZW.WsAGi.Wstu}). 

 \begin{exe}
\ex \label{ex:kAtsWm.Wspa.khWna} 
\gll  tɤrʁaʁkɕi nɯ iɕqʰa, kɯ-ɣɤrʁaʁ, [kɤ-tsɯm] \textbf{ɯ-spa} kʰɯna nɯ ɲɯ-ŋu. \\
hunting.dog \textsc{dem} \textsc{filler} \textsc{sbj}:\textsc{pcp}-hunt \textsc{obj}:\textsc{pcp}-take.away \textsc{3sg}.\textsc{poss}-material dog \textsc{dem} \textsc{sens}-be \\
\glt `Hunting dogs are dogs that are taken to hunt.' (140426 liegou he zhonggou-zh)
\japhdoi{0003812\#S1}
\end{exe}  

 \begin{exe}
\ex \label{ex:BZW.WsAGi.Wstu} 
\gll  [βʑɯ ɯ-sɤ-ɣi] \textbf{ɯ-stu} kɯspoʁ nɯtɕu, iɕqʰa nɯnɯ z-ɲɯ-rku-nɯ. \\
mouse \textsc{3sg}.\textsc{poss}-\textsc{obl}:\textsc{pcp}-come \textsc{3sg}.\textsc{poss}-place hole \textsc{dem}:\textsc{loc} the.aforementioned \textsc{dem} \textsc{tral}-\textsc{ipfv}-put.in-\textsc{pl} \\
\glt `People put [the flower of the burdock] in the holes  which mice come out from (as a trap).' (13-tCamu)
\japhdoi{0003498\#S67}
\end{exe}  
  
These examples are very rare, and not unproblematic\footnote{It could be alternatively possible to analyze \forme{kɯ-spoʁ} as a subject participle   (\tabref {tab:lexicalized.S.nmlz}, §\ref{sec:lexicalized.subject.participle}), and argue that \forme{ɯ-stu kɯ-spoʁ} is a participial relative `the place that has a hole'. }  but nevertheless suggest that \forme{ɯ-spa}, \forme{ɯ-stu} and \forme{ɯ-sta} are advancing in the grammaticalization cline.   In the closely related language Khroskyabs \citep[580]{lai17khroskyabs}, the clitic \forme{=spi}, exact cognate of \forme{ɯ-spa}  (\tabref{tab:spa.sta.stu}, §\ref{sec:lexicalized.oblique.participle}), has become the main object relativizer.\footnote{The clitic \forme{=spi} has a function that is still close to that of \forme{ɯ-spa} in examples such as (\ref{ex:paR.kAmbi.Wspa}):   \citet[514]{lai17khroskyabs} explains that it refers to `un objet spécifiquement destiné à subir l’action.'  }
 
  
\subsection{Interrogative pronouns and correlative constructions} \label{sec:interrogative.relative}
\is{relative clause!correlative} \is{pronoun!interrogative}
All interrogative pronouns, including \japhug{tɕʰi}{what}  (§\ref{sec:tChi}), \japhug{ɕɯ}{who} (§\ref{sec:CW.pronoun}),  \japhug{ŋotɕu}{where} (§\ref{sec:NotCu}) and \japhug{tʰɤjtɕu}{when} (§\ref{sec:thAstWG}), can be used in correlative relative constructions as free-choice indefinites `whoever/whatever/whenever' (§\ref{sec:interrogative.indef}). The pronoun can occur on its own or in apposition with an overt head noun as in (\ref{ex:cai.tChi.tandza}).

\begin{exe}
\ex \label{ex:cai.tChi.tandza}
\gll ɯʑo kɯ [\textbf{<cai>} \textbf{tɕʰi} ta-ndza] nɯ ɣɯ ɯ-mdoʁ nɯ ɲɯ-ndɤm ɲɯ-ŋu. \\
\textsc{3sg} \textsc{erg} vegetable what \textsc{aor}:3\flobv{}-eat \textsc{dem} \textsc{gen} \textsc{3sg}.\textsc{poss}-colour \textsc{dem} \textsc{ipfv}-take[III] \textsc{sens}-be \\
\glt  `It takes the colour of whatever vegetable it has eaten.' (25-caiqajW)
\japhdoi{0003634\#S18}
\end{exe}

When a postclausal relator noun head (§\ref{sec:prenominal.relative}, §\ref{sec:Wstu.relativization.subject}) is present as in (\ref{ex:NotCu.jAkAri.Wstu}), the interrogative pronoun remains \textit{in situ}.

\begin{exe}
\ex \label{ex:NotCu.jAkAri.Wstu}
\gll [\textbf{ŋotɕu} jɤ-kɯ-ɤri] \textbf{ɯ-stu} nɯtɕu kɯ-mɤrʑaβ kɯ-ɕe ra tu-ti-nɯ ŋgrɤl. \\
where \textsc{aor}-\textsc{sbj}:\textsc{pcp}-go[II] \textsc{3sg}.\textsc{poss}-place \textsc{dem}:\textsc{loc} \textsc{sbj}:\textsc{pcp}-marry \textsc{genr}:S/O-go:\textsc{fact} be.needed:\textsc{fact} \textsc{ipfv}-say-\textsc{pl} be.usually.the.case:\textsc{fact} \\
\glt `People say that whatever place [the ladybug] flies to, is where one has to go to get married.' (26-kWlAGpopo)
\japhdoi{0003670\#S38}
\end{exe}

Correlative relatives can be participial (\ref{ex:NotCu.jAkAri.Wstu}, \ref{ex:tChi.nWkWGe}) or finite (\ref{ex:cai.tChi.tandza}), depending on the function of the relativized element (§\ref{sec:function.relativization}).

\begin{exe}
\ex \label{ex:tChi.nWkWGe}
\gll [ɯ-jaʁ \textbf{tɕʰi} nɯ-kɯ-ɣe ʑo] tu-ndze \\
\textsc{3sg}.\textsc{poss}-hand what \textsc{aor}:\textsc{west}-\textsc{sbj}:\textsc{pcp}-come[II] \textsc{emph} \textsc{ipfv}-eat[III] \\
\glt `It eats whatever it can get its hands on.' (28-qapar)
\japhdoi{0003700\#S103}
\end{exe}


%kɯ-wxti ra kɯ nɯ-kɤ-ndza, [tɕhi tu-ndza-nɯ kɯ-ŋu] nɯ ʑɯrɯʑɤri tɕe ɲɯ-mbi-nɯ.

 In (\ref{ex:CW.kW.kWmWrkW}), the fact that the interrogative pronoun \japhug{ɕɯ}{who} takes the ergative shows that it belongs to the same clause as the transitive verb \forme{lu-kɤ-tɕɤt}, as the matrix verb \japhug{cʰa}{can} is intransitive (§\ref{sec:semi.transitive}): it is therefore embedded within the complement clause.

\begin{exe}
\ex \label{ex:CW.kW.kWmWrkW}
\gll  [[\textbf{ɕɯ} kɯ [kɯ-mɯrkɯ kɯ-ŋu] lu-kɤ-tɕɤt] pɯ-kɯ-cʰa] nɯ, a-sci rɟɤlpu cʰɯ-ta-sɯ-ndo-nɯ ŋu \\
who \textsc{erg} \textsc{sbj}:\textsc{pcp}-steal  \textsc{sbj}:\textsc{pcp}-be \textsc{ipfv}-\textsc{inf}-take.out \textsc{aor}-\textsc{sbj}:\textsc{pcp}-can \textsc{dem} \textsc{1sg}.\textsc{poss}-instead king \textsc{ipfv}-1\fl{}2-\textsc{caus}-take-\textsc{pl} be:\textsc{fact} \\
\glt `Whoever succeeds in catching (the one who) is the thief, I will make him king in my stead.' (2003 qachGa)
\japhdoi{0003372\#S5}
\end{exe}

Possessors of subjects (§\ref{sec:S.possessor.relativization}) can also undergo correlative maximalizing relativization, as in (§\ref{sec:S.possessor.relativization}).

\begin{exe}
\ex \label{ex:WtWji.kWnAtWG}
\gll [\textbf{ɕɯ} \textbf{ɯ}-tɯ-ji kɯ-nɤtɯɣ] nɯnɯra kɯ tɤ-mtʰɯm ku-sqa-nɯ tɕe, \\
who \textsc{3sg}.\textsc{poss}-\textsc{indef}.\textsc{poss}-field \textsc{sbj}:\textsc{pcp}-happen.to.be \textsc{dem}:\textsc{pl} \textsc{erg} \textsc{indef}.\textsc{poss}-meat \textsc{ipfv}-cook-\textsc{pl} \textsc{lnk} \\
\glt `Whoever$_i$ [it is] whose$_i$ fields happen to be [the ones that are to be ploughed by the whole village$_j$], those people$_i$ cook meat (for the village$_j$).' (150909 kWnWjlAmtshi)
\japhdoi{0006306\#S17}
\end{exe}

Correlative relatives have commonalities with universal concessive conditionals (§\ref{sec:universal.concessive.conditional}), but in the latter the subordinate clause and the main clause do not necessarily share a common element.

\section{Morphosyntactic specificities of relative clauses} \label{sec:relative.morphosyntactic.specificities}
This section presents morphosyntactic phenomena distinguishing relative clauses from independent sentences or complement clauses, including resumptive pronouns (§\ref{sec:resumptive}), case marking (§\ref{sec:relative.genitive.possessor.subject}), possessive prefix neutralization (§\ref{sec:relative.possessor.neutralization}), determiner replication (§\ref{sec:relative.determiners}) and totalitative reduplication (§\ref{sec:totalitative.relatives}).

It also shows that the demonstrative-like elements \forme{nɯ} that follow the relatives 
are not complementizers (§\ref{sec:relative.determiners.complementizer}), and discusses word order within the relative, in particular the presence of postverbal elements in head-internal relative clauses (§\ref{sec:relative.postverbal}).

\subsection{Resumptive pronouns} \label{sec:resumptive}
\is{pronoun!resumptive} \is{relative clause!resumptive pronoun} 
A resumptive third person pronoun \forme{ɯʑo} occurs in conjoined relative clauses with the correlative additive focus marker \forme{ri} (§\ref{sec:ri.additive}). It is obligatory in this construction when the relativized element is the intransitive subject, as in the clause \forme{ɯʑo$_i$ ri kɯ-sna} in (\ref{ex:WZo.ri.kWsna}).

\begin{exe}
\ex \label{ex:WZo.ri.kWsna}
\gll nɯnɯ [\textbf{ɯ}$_i$-pʰɯ ri kɯ-wxti] [\textbf{ɯʑo}$_i$ ri kɯ-sna] ŋu \\
\textsc{dem} \textsc{3sg}.\textsc{poss}-price also \textsc{sbj}:\textsc{pcp}-be.big \textsc{3sg} also \textsc{sbj}:\textsc{pcp}-be.good be:\textsc{fact} \\
\glt `[Silver] is (a metal that is) both expensive (whose price is big) and nice.' (30-Com)
\japhdoi{0003736\#S190}
\end{exe}

This construction occurs in particular in texts from Chinese to translate the construction \ch{又……又……}{yòu X yòu Y}{both X and Y}, as in (\ref{ex:WZo.ri.kWwxti}).\footnote{The Chinese original has \ch{这只癞蛤蟆又大又丑}{zhè zhī làiháma yòu dà yòu chǒu}{the toad was big and ugly}.}

\begin{exe}
\ex \label{ex:WZo.ri.kWwxti}
\gll nɯnɯ qajɯ kɯ-sɤjlɯ\redp{}jloʁ nɯ jo-ɣi tɕe,  [\textbf{ɯʑo} ri kɯ-wxti], [\textbf{ɯʑo} ri kɯ-sɤjlɯ\redp{}jloʁ] ci pjɤ-ŋu. \\
\textsc{dem} bug \textsc{sbj}:\textsc{pcp}-\textsc{emph}\redp{}disgusting \textsc{dem} \textsc{ifr}-come \textsc{lnk} 
\textsc{3sg} also \textsc{sbj}:\textsc{pcp}-be.big \textsc{3sg} also \textsc{sbj}:\textsc{pcp}-\textsc{emph}\redp{}be.disgusting \textsc{indef} \textsc{ifr}.\textsc{ipfv}-be \\
\glt `The disgusting creature (the toad) came, it was big and disgusting.' (150818 muzhi guniang-zh)
\japhdoi{0006334\#S84}
\end{exe}
 
\subsection{Totalitative reduplication} \label{sec:totalitative.relatives}
\is{reduplication!totalitative} \is{totalitative!reduplication} \is{relative clause!totalitative reduplication}
The main verb of relative clauses, whether in participial (\ref{ex:kWkWngo.kWkAkWnAndza}) or in finite form (\ref{ex:pWpaGWtndZi}, \ref{ex:tWtAfsea}), can undergo reduplication of the first syllable (§\ref{sec:verb.initial.redp}) to express universal quantification of the relativized element, whose syntactic function can be object (\ref{ex:pWpaGWtndZi}), semi-object (\ref{ex:tWtAfsea}) or intransitive subject (\ref{ex:kWkWngo.kWkAkWnAndza}), and even transitive subject (\ref{ex:qartshaz.pWpWkWmtshAm}) (however some transitive subject participle cannot undergo initial reduplication for morphological reasons, see §\ref{sec:totalitative.redp}).

\begin{exe}
\ex \label{ex:pWpaGWtndZi}
\gll [\textbf{tɯ-rɣi} pɯ\redp{}pa-ɣɯt-ndʑi ʑo] nɯnɯ lo-ji-ndʑi. \\
\textsc{indef}.\textsc{poss}-seed \textsc{total}\redp{}\textsc{aor}:3\flobv{}:\textsc{down}-bring-\textsc{du} \textsc{emph} \textsc{dem} \textsc{ifr}-plant-\textsc{du} \\
\glt `They planted all the seeds that they had brought (down from heaven).' (31-deluge)
\japhdoi{0004077\#S145}
\end{exe}
  

\begin{exe}
\ex \label{ex:tWtAfsea}
\gll [aʑo tɯ\redp{}tɤ-fse-a] nɯ tɤ-fse tɕe \\
\textsc{1sg} \textsc{total}\redp{}\textsc{aor}-be.like-\textsc{1sg} \textsc{dem} \textsc{imp}-be.like \textsc{lnk} \\
\glt `Do everything like I do.' (140426 jiagou he lang-zh,18)
\japhdoi{0003804\#S17}
\end{exe}

\begin{exe}
\ex \label{ex:kWkWngo.kWkAkWnAndza}
\gll <zhengfu> kɯ [kɯ\redp{}kɯ-ngo], nɯnɯ [kɯ\redp{}kɤ-kɯ-nɤndza ʑo] nɯ, andi comuco tɕetu taʁ ri, nɤki tsʰupa ci tu tɕe, ɯnɯre to-sɯ-ɤwɯwum. \\
government \textsc{erg} \textsc{total}\redp{}\textsc{sbj}:\textsc{pcp}-be.sick \textsc{dem} \textsc{total}\redp{}\textsc{aor}-\textsc{sbj}:\textsc{pcp}-have.leprosy \textsc{emph} \textsc{dem} west  \textsc{topo} up.there up \textsc{loc} \textsc{filler} village \textsc{indef} exist:\textsc{fact} \textsc{lnk} \textsc{dem}:\textsc{loc} \textsc{ifr}-\textsc{caus}-gather \\
\glt `In the west up there in Kyomkyo there is a village. The government gathered all the sick people, all the lepers, up there.' (25-khArWm)
\japhdoi{0003644\#S58}
\end{exe}

\begin{exe}
\ex \label{ex:qartshaz.pWpWkWmtshAm}
\gll  tɕe \textbf{nɯra} \textbf{qartsʰaz} \textbf{mu} \textbf{nɯra},  [pɯ\redp{}pɯ-kɯ-mtsʰɤm] nɯ ɯ-rkɯ nɯtɕu tu-owɯwum-nɯ ŋu \\
\textsc{lnk} \textsc{dem}:\textsc{pl} deer female \textsc{dem}:\textsc{pl} \textsc{total}\redp{}\textsc{aor}-\textsc{sbj}:\textsc{pcp}-hear \textsc{dem} \textsc{3sg}.\textsc{poss}-side \textsc{dem}:\textsc{loc} \textsc{ipfv}-gather-\textsc{pl} \\
\glt `All the does that have heard [the buck] gather around it.' (27-qartshAz)
\japhdoi{0003702\#S131}
\end{exe}


Although initial reduplication occurs with finite verb forms in main clauses with various meanings (§\ref{sec:verb.initial.redp}), totalitative reduplication is exclusively attested in relative clauses. A more detailed discussion of this phenomenon is provided in §\ref{sec:totalitative.redp}.

Totalitative reduplication is specificity to relative clauses, and does not occur in complement clauses. It can be used as a test to distinguish between the two types of clauses in ambiguous cases (§\ref{sec:relative.complement.ambiguities}).

 \subsection{Genitive possessor } \label{sec:relative.genitive.possessor.subject}
 \is{genitive}
In object relative clauses (§\ref{sec:object.relativization}), transitive subjects can sometimes be marked by the genitive instead of the ergative, in particular when the agent can be construed as a possessor of the object, as in the finite headless relative in (\ref{ex:aZWG.nWnWBeta}).  


 \begin{exe}
\ex \label{ex:aZWG.nWnWBeta}
 \gll aʑɯɣ [nɯ-nɯ-βde-t-a] nɯ ɯ́-ŋu \\
 \textsc{1sg}:\textsc{gen} \textsc{aor}-\textsc{auto}-throw-\textsc{pst}:\textsc{tr}-\textsc{1sg} \textsc{dem} \textsc{qu}-be:\textsc{fact} \\
 \glt `Is it the one that I lost?' (140427 bianfu jingji he shuiniao-zh)
 \japhdoi{0003836\#S28}
\end{exe}

Example (\ref{ex:GW.kW.nWkAkho}) illustrates the hesitation between the ergative in the first relative clause, and the genitive in the second one (where the agent is the \textit{former} possessor of the object).

\begin{exe}
\ex \label{ex:GW.kW.nWkAkho}
 \gll tɕe [atu tɯmɯkɤrŋi kɯ ɯ-jaʁ nɯ-kɤ-kʰo] nɯ, smɤnmimitoʁ nɯnɯ ɣɯ [nɯ-kɤ-kʰo] nɯnɯ, ko-ɕtʰɯz\\
\textsc{lnk} up.there heavenly(god) \textsc{erg} \textsc{3sg}.\textsc{poss}-hand \textsc{aor}-\textsc{obj}:\textsc{pcp}-give \textsc{dem}  \textsc{anthr} \textsc{dem} \textsc{gen} \textsc{aor}-\textsc{obj}:\textsc{pcp}-give \textsc{dem} \textsc{ifr}:\textsc{east}-turn.towards\\
\glt `He turned [the magical object] that the god from heaven had given him, that Smanmi Metog had given him, [in the direction of the râkshâsas].' (2011-04-smanmi)
\end{exe}

In the case of participial object relatives with a possessive prefix, the transitive subject/possessor does not require genitive marking, just in the same way as possessive prefixes on the possessum (with optional genitive) are sufficient to mark possession between two nouns (§\ref{ex:prefix.expression.of.possession}, §\ref{sec:gen.possession}). In (\ref{ex:WZo.WkAtshi}) for instance, the pronoun \forme{ɯʑo} `it' (the camel) directly precedes the participle \forme{ɯ-kɤ-tsʰi} `the (water) that it drinks' without either ergative or genitive marker.

\begin{exe}
\ex \label{ex:WZo.WkAtshi}
 \gll ɯ-zgo ɯ-ŋgɯ nɯtɕu ɯʑo [ɯ-kɤ-tsʰi] tu-nɯ-rke ɲɯ-kʰɯ \\
\textsc{3sg}.\textsc{poss}-mountain \textsc{3sg}.\textsc{poss}-inside \textsc{dem}:\textsc{loc} \textsc{3sg} \textsc{3sg}.\textsc{poss}-\textsc{obj}:\textsc{pcp}-drink \textsc{ipfv}-\textsc{auto}-put.in[III] \textsc{sens}-be.possible \\
\glt `[The camel] can store the [water] that it drinks in its hump.' (19-rNamoN)
\japhdoi{0003552\#S48}
\end{exe}

The transitive subject in these constructions is formally a possessor, external to the relative clause. In (\ref{ex:stu.WsAtCha.WkAnWrga}) in particular, the presence of the degree adverb \japhug{stu}{most} (§\ref{sec:stu.superlative}) between the pronoun \forme{ɯʑo} and the head noun \forme{ɯ-sɤtɕʰa} `its place' is a clue that \forme{ɯʑo} is not included in the relative clause.

\begin{exe}
\ex \label{ex:stu.WsAtCha.WkAnWrga}
\gll  tɕeri ɯʑo [stu \textbf{ɯ-sɤtɕʰa} ɯ-kɤ-nɯ-rga] nɯ, \\
\textsc{lnk} \textsc{3sg} most \textsc{3sg}.\textsc{poss}-place \textsc{3sg}.\textsc{poss}-\textsc{obj}:\textsc{pcp}-\textsc{appl}-like \textsc{dem} \\
\glt `The place that it likes most...' (20-xsar 10)
\japhdoi{0003568\#S10}
\end{exe}

\subsection{Possessive prefix neutralization} \label{sec:relative.possessor.neutralization}
\is{prefix!possessive} \is{neutralization!possessive prefix}
In relative clauses, the possessor of inalienably possessed nouns can be neutralized to the indefinite possessor prefix \forme{tɯ-/tɤ-} (§\ref{sec:indef.genr.poss}), even when the possessor is definite. In (\ref{ex:tWNga.nWkAmbi}) for example, the possessor of \forme{-ŋga} `clothes' is the transitive subject of both sentences (the main character of the story), and is marked with the \textsc{3sg} prefix \forme{ɯ-} in the first sentence. In the second sentence, the noun \forme{-ŋga} is located in a head-internal relative clause (§\ref{sec:semi.object.relativization}), and even though the possessor is the same as in the previous sentence, the indefinite prefix \forme{tɯ-} appears.  
 
\begin{exe}
\ex \label{ex:tWNga.nWkAmbi}
\gll tɕendɤre ɯ-ŋga ra ɲɤ-tɕɤt tɕe iɕqʰa,  [rgɤnmɯ kɯ \textbf{tɯ-ŋga} nɯ-kɤ-mbi] nɯra ɲɤ-tɕɤt tɕe   \\
\textsc{lnk} \textsc{3sg}.\textsc{poss}-clothes \textsc{pl} \textsc{ifr}-take.off \textsc{lnk} \textsc{filler} old.woman \textsc{erg} \textsc{indef}.\textsc{poss}-clothes \textsc{aor}-\textsc{obj}:\textsc{pcp}-give \textsc{dem}:\textsc{pl} \textsc{ifr}-take.off \textsc{lnk} \\
\glt `He took off his clothes, he took off the (magical) clothes that the old woman had given him.' (140508 shier ge tiaowu de gongzhu-zh)
\japhdoi{0003937\#S165}
\end{exe}

Some inalienably possessed nouns have constraints on the syntactic function of the referent indexed by the possessive prefix (§\ref{sec:biactantial.ipn}). In particular, \japhug{tɤ-pɤro}{present} indexes the agent (the person giving the present) as possessor, never the recipient, as shown by (\ref{ex:apAro.YWtambi}), where a \textsc{2sg} \forme{nɤ-pɤro} or an indefinite possessor \forme{tɤ-pɤro} would be agrammatical (see also examples \ref{ex:apAro} and \ref{ex:nApAro} in §\ref{sec:biactantial.ipn}).
	
\begin{exe}
\ex \label{ex:apAro.YWtambi}
\gll a-pɤro ɲɯ-ta-mbi ŋu \\
\textsc{1sg}.\textsc{poss}-present \textsc{ipfv}-1\fl{}2-give be:\textsc{fact} \\
\glt `I am giving it to you as a present.' (elicited)
\end{exe} 
	 
In relative clauses (whether participial or finite ones), it is possible to index the agent like in main clauses (\ref{ex:apAro.relative}), but  possessor neutralization is also possible, as in (\ref{ex:tApAro.relative}).

\begin{exe}
\ex \label{ex:apAro.relative}
\gll [a-pɤro nɯ-mbi-t-a] nɯ a-rɟit ŋu  \\
	\textsc{1sg}.\textsc{poss}-present \textsc{aor}-give-\textsc{pst}:\textsc{tr}-\textsc{1sg} \textsc{dem} \textsc{1sg}.\textsc{poss}-child be:\textsc{fact} \\
\glt `The one to whom I gave a present is my child.' (elicited)
\end{exe} 

\begin{exe}
\ex \label{ex:tApAro.relative}
\gll 	[tɤ-pɤro nɯ-mbi-t-a] \textbf{tɤ-rɟit} nɯ a-tɕɯ ŋu \\
\textsc{indef}.\textsc{poss}-present \textsc{aor}-give-\textsc{pst}:\textsc{tr}-\textsc{1sg} 	\textsc{indef}.\textsc{poss}-child \textsc{dem} \textsc{1sg}.\textsc{poss}-son be:\textsc{fact} \\
\glt `The child to whom I gave a present is my son.' (elicited)
\end{exe} 

Relative clauses are not the only syntactic contexts where possessor neutralization is attested: it also occurs when inalienably possessed nouns take prenominal modifiers (§\ref{sec:possessive.prefixes.prenominal}).


\subsection{Determiners} \label{sec:relative.determiners}
\is{relative clause!determiner}
\subsubsection{Determiners on internal head} \label{sec:head-internal.relative.determiners}
In head-internal and postnominal relatives, postnominal determiners such as the indefinte marker \forme{ci} (§\ref{sec:indef.article}) or the demonstrative \forme{nɯ} (§\ref{sec:nW.topic}) can appear both on the internal head noun and repeated following the relative clause. In (\ref{ex:rJAlpu.ci.kWNAn.ci}) for instance, \forme{ci} occurs after the noun \japhug{rɟɤlpu}{king}  and at the end of the relative after the participle \forme{kɯ-ŋɤn} `(the one) who is evil'. 

\begin{exe}
\ex \label{ex:rJAlpu.ci.kWNAn.ci}
\gll [wuma ʑo \textbf{rɟɤlpu} \textbf{ci} kɯ-ŋɤn] ci pjɤ-tu tɕe, \\
really \textsc{emph} king \textsc{indef} \textsc{sbj}:\textsc{pcp}-be.evil \textsc{indef} \textsc{ifr}.\textsc{ipfv}-exist \textsc{lnk} \\
\glt `There was a very evil king.' (140511 1001 yinzi-zh)
\japhdoi{0003963\#S9}
 \end{exe} 

In (\ref{ex:tAkhW.sAZnWlhoR}), the indefinite \forme{ci} is also repeated after the head noun \forme{kɯspoʁ} `hole' (a lexicalized participle, §\ref{sec:lexicalized.subject.participle}) and following the participle \forme{sɤz-nɯ-ɬoʁ} `which (the smoke) comes out from' (with locative §\ref{sec:locative.participle.relatives} or instrumental §\ref{sec:instrument.relativization} relativizing function). In this case the relative can be analyzed as either postnominal or head-internal.

\begin{exe}
\ex \label{ex:tAkhW.sAZnWlhoR}
\gll [\textbf{kɯ-spoʁ} \textbf{ci} tɤ-kʰɯ sɤz-nɯ-ɬoʁ] ci ɲɯ-βze \\
\textsc{sbj}:\textsc{pcp}-have.a.hole \textsc{indef} \textsc{indef}.\textsc{poss}-smoke \textsc{obl}:\textsc{pcp}-\textsc{auto}-come.out \textsc{indef} \textsc{ipfv}-make[III] \\ 
\glt `[The potter] makes a hole which the smoke comes out from.' (30-kWrAfcAr)
\japhdoi{0003740\#S27}
\end{exe}

Determiner repetition is also found with relativized transitive subjects marked with the ergative (see \ref{ex:WkWnWmbrApW} in §\ref{sec:head-internal.relative.postnominal}).


Examples like (\ref{ex:rJAlpu.ci.kWNAn.ci}) and (\ref{ex:tAkhW.sAZnWlhoR}) are however relatively rare, in most cases the determiner either follows the relative (see for example  \ref{ex:mWNi.kAkWGe.XpWn} in §\ref{sec:intr.subject.relativization}) or the head noun (\ref{ex:pGAtCW.ci.kWmpCWmpCAr})
  
\begin{exe}
\ex \label{ex:pGAtCW.ci.kWmpCWmpCAr}
\gll  [\textbf{pɣɤtɕɯ} \textbf{ci} kɯ-mpɕɯ\redp{}mpɕɤr ʑo] jɤ-ɣe \\
bird \textsc{indef} \textsc{sbj}:\textsc{pcp}-\textsc{emph}\redp{}be.beautiful \textsc{emph} \textsc{aor}-come[II] \\
\glt `A beautiful bird came.' (2003 qachGa)
\japhdoi{0003372\#S14}
\end{exe} 

In the case of prenominal relatives, the determiner \forme{ci} can be found between the head noun and the relative clause, as in (\ref{ex:nW.kWfse.ci.qajW}). This usage is not to be confused with that of prenominal \forme{ci} (§\ref{sec:identity.modifier}).

\begin{exe}
\ex \label{ex:nW.kWfse.ci.qajW}
\gll tɕe [nɯ kɯ-fse] ci \textbf{qajɯ} ɣɤʑu. \\
\textsc{lnk} \textsc{dem} \textsc{sbj}:\textsc{pcp}-be.like \textsc{indef} bug exist:\textsc{sens} \\
\glt  `There is a bug (invertebrate animal) that is like that.' (hist180421 haixing)
\japhdoi{0006113\#S27}
\end{exe} 

\subsubsection{Determiner or complementizer} \label{sec:relative.determiners.complementizer}
An important proportion of relative clauses are followed by the forms \forme{nɯ} and \forme{nɯnɯ}, whose functions in other contexts include distal demonstrative pronouns (§\ref{sec:demonstrative.pronouns}), demonstrative determiners (§\ref{sec:demonstrative.determiners}) or topic markers (§\ref{sec:nW.topic}, §\ref{sec:definiteness}), and \forme{ra} (and \forme{nɯra}), a plural marker (§\ref{sec:plural.determiners}). In the case of finite relative clauses (§\ref{sec:finite.relatives}), whose main verb generally has the same form as that of the corresponding independent sentence (in \ref{ex:pWXsuj.nW} for instance), the presence of these words may be the only clue that the clause is a relative.

\begin{exe}
\ex \label{ex:pWXsuj.nW}
\gll [iʑo ndɤre pɯ-χsu-j] nɯ, tɯ-rdoʁ tɕʰi mɯ-pɯ-nnɯ-pe mɤ-xsi ma nɯnɯ nɯfse pjɤ-si \\
\textsc{1pl} \textsc{lnk} \textsc{aor}-feed-\textsc{1pl} \textsc{dem} one-piece what \textsc{neg}-\textsc{pst}.\textsc{ipfv}-\textsc{auto}-be.good \textsc{neg}-\textsc{genr}:know \textsc{lnk} \textsc{dem} like.that \textsc{ifr}-die \\
\glt `Of the [tortoises] that we raised, one [of them] died just like that, I don't know what went wrong.' (140510 wugui)
\japhdoi{0003951\#S51}
\end{exe}  

Given the fact that many languages have complementizers, including relative pronouns, which are homophonous with (and historically related to) demonstratives (for instance `that' in English), it is legitimate to wonder whether such an analysis to possible for the \forme{nɯ} or \forme{nɯra} that follow relative clauses.

However, in prenominal relatives such as (\ref{ex:akAsWz.Cku}), \forme{nɯ} is located \textit{after} the head noun, for instance \japhug{ɕku}{Allium} in (\ref{ex:akAsWz.Cku}), never before it. If it were a subordinator, one would expect \forme{nɯ} to be located between the head noun and the relative.

\begin{exe}
\ex \label{ex:akAsWz.Cku}
\gll   tɕe [aʑo a-kɤ-sɯz] \textbf{ɕku} nɯ nɯra ŋu \\
\textsc{lnk} \textsc{1sg} \textsc{1sg}.\textsc{poss}-\textsc{obj}:\textsc{pcp}-know Allium \textsc{dem} \textsc{dem}:\textsc{pl} be:\textsc{fact} \\
\glt `These are the [plants belonging to the gender] \textit{Allium} that I know about.' (07-Cku)
\japhdoi{0003424\#S160}
\end{exe}  

Another clue that \forme{nɯ}, \forme{nɯnɯ} and \forme{nɯra} in this context are better analyzed as demonstrative determiners (§\ref{sec:demonstrative.determiners}, with or without topicalizing function, §\ref{sec:nW.topic}, §\ref{sec:definiteness}) is that relative clauses can take circumposed demonstratives (exactly like circumnominal demonstratives, §\ref{sec:demonstrative.determiners}), as shown by the pre- and post-clausal \forme{nɯnɯ} in (\ref{ex:nWnW.pGArnoR.kAti.nWnW}).

\begin{exe}
\ex \label{ex:nWnW.pGArnoR.kAti.nWnW}
\gll nɯnɯ [pɣɤrnoʁ kɤ-ti] nɯnɯ tɤjmɤɣ tɤrca ɲɯ-ŋu ɯmɤ-kɯ-ŋu-ci ma \\
\textsc{dem} fungus.sp \textsc{obj}:\textsc{pcp}-say \textsc{dem} mushroom together \textsc{sens}-be \textsc{prob}-\textsc{peg}-be-\textsc{peg} \textsc{lnk} \\
\glt `That [thing that] is called \forme{pɣɤrnoʁ} is probably [to be classified] among the fungi.' (22-BlamajmAG)
\japhdoi{0003584\#S130}
\end{exe}  

%kɯki athi qaɕti kɤ-ntsɣe lu-kɤ-ɣɯt nɯ cho naχtɕɯɣ
\subsection{Postverbal elements} \label{sec:relative.postverbal}
\is{relative clause!postverbal position}
Although Japhug is a strict verb-final language, some words can appear post-verbally in independent sentences, including sentence-final particles (§\ref{sec:interjections}), ideophones (§\ref{sec:ideo:morpho}) and some adverbs (§\ref{sec:postverbal.adv}).

Relative clauses never take sentence-final particles (unlike some complement clauses, §\ref{sec:reported.speech.sfp}), but some postverbal elements are nevertheless possible. 

First, the adverb \japhug{tsa}{a little} can occur postverbally with a semantic scope clearly restricted to the relative clause, as in (\ref{ex:nWlWz.thWkWGe.tsa}) 

\begin{exe}
\ex \label{ex:nWlWz.thWkWGe.tsa}
\gll [nɯ-lɯz tʰɯ-kɯ-ɣe tsa] nɯra tɕe, ʁdɯrɟɤt ɯ-skɤt tu-βze-a ɲɯ-tso-nɯ, \\
\textsc{3pl}.\textsc{poss}-age \textsc{aor}-\textsc{sbj}:\textsc{pcp}-come[II] a.little \textsc{dem}:\textsc{pl} \textsc{lnk}  \textsc{topo} \textsc{3sg}.\textsc{poss}-language \textsc{ipfv}-make[II]-\textsc{1sg} \textsc{sens}-understand-\textsc{pl} \\
\glt `Those who are a little older understand when I speak the [Japhug] language of Gdongbrgyad. (150901 tshuBdWnskAt)
\japhdoi{0006242\#S11}
\end{exe} 

Ideophones are very commonly postverbal even in relative clauses, as in (\ref{ex:kWwGrum.sWNsWN}), though mainly with stative verbs in subject participle form.

\begin{exe}
\ex \label{ex:kWwGrum.sWNsWN}
\gll tɕe nɯ ɯ-βri ɯ-taʁ nɯra qandʐi kɯ-fse, aɣrɤɣrum kɯ-fse  ma [kɯ-wɣrɯ\redp{}wɣrum ʑo sɯŋsɯŋ] maʁ \\
\textsc{lnk} \textsc{3sg}.\textsc{poss}-body \textsc{3sg}.\textsc{poss}-\textsc{top} \textsc{dem}:\textsc{pl} be.dark:\textsc{fact} \textsc{sbj}:\textsc{pcp}-be.like be.whitish:\textsc{fact} \textsc{sbj}:\textsc{pcp}-be.like \textsc{lnk} \textsc{lnk} \textsc{sbj}:\textsc{pcp}-emph\redp{}be.white \textsc{emph} \textsc{idph}(II):pure.white not.be:\textsc{fact} \\
\glt `The top part [of the mushroom called \forme{kuɣrummɤɣ} `white mushroom'] is dark, whitish, it is not pure white.' (21-kuGrummAG)
\japhdoi{0003574\#S2}
\end{exe} 

The emphatic \forme{ʑo} (§\ref{sec:emphatic.Zo}) is also analyzable as a postverbal adverb especially when followed by a demonstrative determiner as in (\ref{ex:stu.kWjaR.Zo}) (see also \ref{ex:pWpaGWtndZi} and \ref{ex:kWkWngo.kWkAkWnAndza}, §\ref{sec:totalitative.relatives}).

\begin{exe}
\ex \label{ex:stu.kWjaR.Zo}
\gll [\textbf{skɤm-ndʐi} kɯ-jɯ\redp{}jaʁ], [stu kɯ-jaʁ ʑo] nɯ tu-qɤr-nɯ tɕe \\
ox-skin \textsc{sbj}:\textsc{pcp}-\textsc{emph}\redp{}be.thick most \textsc{sbj}:\textsc{pcp}-be.thick \textsc{emph} \textsc{dem} \textsc{ipfv}-select-\textsc{pl} \textsc{lnk} \\
\glt `People select very thick ox skin, the thickest.' (24-mbGo) \japhdoi{0003621}
\end{exe} 

However, the indefinite determiner \forme{ci} is also attested before the emphatic \forme{ʑo} as in (\ref{ex:smar.kWwxtWxti.ci}), showing that it is external to the relative in this case, since \forme{ci} itself must be external (otherwise it would be located just after the head noun \japhug{smar}{river}, §\ref{sec:head-internal.relative.determiners}).

\begin{exe}
\ex \label{ex:smar.kWwxtWxti.ci}
\gll [\textbf{smar} kɯ-wxtɯ\redp{}wxti] ci ʑo pɯ-tu ɲɯ-ŋu \\
river \textsc{sbj}:\textsc{pcp}-\textsc{emph}\redp{}be.big \textsc{indef} \textsc{emph} \textsc{pst}.\textsc{ipfv}-exist \textsc{sens}-be \\
\glt `There was a huge river.' (2005 Kunbzang)
\end{exe} 

In the absence of determiner (as in \ref{ex:pWpaGWtndZi}, §\ref{sec:totalitative.relatives}), it is unclear whether postverbal \forme{ʑo} belongs to the relative or not.

\section{Position of the relativized element} \label{sec:position.head.relative}
Pre-, post-nominal and head-internal relative clauses are all attested in Japhug, though the position of the relativized element is not free and depends on various factors, in particular its function in the relative clause.

\subsection{Headless} \label{sec:headless.relative}
\is{relative clause!headless}
An important proportion, if not a large majority of the relatives in the corpus lack an overt head. For instance, in (\ref{ex:pWkWsi.WCa}), minimal relative clause \forme{pɯ-kɯ-si} `(the one(s)) that has/have died' only contains a verb in participial form (§\ref{sec:subject.participle.subject.relative}).\footnote{The noun \forme{ɯ-ɕa} `its meat/flesh' in (\ref{ex:pWkWsi.WCa}) is not the head of the relative, but the possessee of that head (§\ref{sec:genitival.relatives}). }

\begin{exe}
\ex \label{ex:pWkWsi.WCa}
\gll [pɯ-kɯ-si] ɯ-ɕa nɯ tu-ndze ŋu ma. \\
\textsc{aor}-\textsc{sbj}:\textsc{pcp}-die \textsc{3sg}.\textsc{poss}-flesh \textsc{dem} \textsc{ipfv}-eat[III] be:\textsc{fact} \textsc{lnk} \\
\glt `[Crows] eat the flesh of [animals] that have died.' (22-qajdo)
\japhdoi{0003596\#S15}
\end{exe} 

Nominal heads are optional in relative clauses (as they are in the noun phrase in general, §\ref{sec:noun.phrases.word.order}). The only non-optional heads are the resumptive pronouns that occur in a very specific construction (§\ref{sec:resumptive}) and the possessive prefixes on the possessee in relatives whose heads are possessor of subject or object (§\ref{sec:possessor.relativization}).

\subsection{Prenominal} \label{sec:prenominal.relative}
\is{relative clause!prenominal}
Three types of prenominal relatives must be distinguished, depending on the nature of the relationship between the head noun and the relative clause.

First, \textit{genitival} prenominal relatives are those in which the genitive marker \forme{ɣɯ} is inserted between the subordinate clause and the head noun (§\ref{sec:genitival.relatives}). 
 
Second, \textit{relator noun} prenominal relatives take generic inalienably possessed noun such as \japhug{ɯ-spa}{material} (§\ref{sec:instrument.relativization}, §\ref{sec:Wspa.relative}), \japhug{ɯ-stu}{place} (§\ref{sec:Wstu.relativization.subject}) or \forme{ɯ-sŋi} `the day when' (§\ref{sec:time.relativization}). This type of prenominal clauses are common in the case of adjunct relativization. 

Third, \textit{standard} prenominal relative clauses have head nouns without possessive prefixes (unless the head noun is inalienably possessed, in which case possessor neutralization may take place, §\ref{sec:relative.possessor.neutralization}). They are the preferred relativization type in the case of transitive subjects (§\ref{sec:tr.subject.relativization}), and are also available for other core arguments (§\ref{sec:object.relativization}, §\ref{sec:intr.subject.relativization}).

\subsection{Head-internal} \label{sec:head-internal.relative}  
\is{relative clause!head-internal}
When the head noun of the relative is overt, it can occur within the relative at the position that would be expected in the corresponding independent sentence: for instance, the noun \japhug{tɯ-ŋga}{clothes} is located between the instrumental adjunct and the verb both in the head-internal relative in (\ref{ex:tWNga.thWkABzu}) and in the independent clause in (\ref{ex:tWNga.chWBzunW}).

\begin{exe}
\ex \label{ex:tWNga.thWkABzu}
\gll [tɤ-rme kɯ \textbf{tɯ-ŋga} tʰɯ-kɤ-βzu] nɯra ʁɟa tu-ndze ɲɯ-ŋu. \\
\textsc{indef}.\textsc{poss}-hair \textsc{erg} \textsc{indef}.\textsc{poss}-clothes \textsc{aor}-\textsc{obj}:\textsc{pcp}-make \textsc{dem}:\textsc{pl} completely \textsc{ipfv}-eat[III] \textsc{sens}-be \\
\glt `It eats all of the clothes that are made of [animal] hair.' (28-kWpAz)
\japhdoi{0003714\#S104}
\end{exe} 

\begin{exe}
\ex \label{ex:tWNga.chWBzunW}
\gll paʁ-ndʐi kɯ tɯ-ŋga cʰɯ-βzu-nɯ, paʁ-ndʐi kɯ tɯ-xtsa tu-βzu-nɯ ra ɲɯ-ŋgrɤl \\
pig-skin \textsc{erg} \textsc{indef}.\textsc{poss}-clothes \textsc{ipfv}-make-\textsc{pl} pig-skin \textsc{erg} \textsc{indef}.\textsc{poss}-shoes \textsc{ipfv}-make-\textsc{pl} \textsc{pl} \textsc{sens}-be.usually.the.case \\
\glt `(Nowadays, unlike in former times), [people] make clothes and shoes from pig skin.' (05-paR)
\japhdoi{0003400\#S108}
\end{exe} 

The following subsections focus on the conditions where head-internal relatives are selected rather than prenominal ones (§\ref{sec:head-internal.relative.prenominal}), and on the criteria that can be used to distinguish between head-internal and postnominal relatives (§\ref{sec:head-internal.relative.postnominal}).

\subsubsection{Head-internal vs. prenominal relatives} \label{sec:head-internal.relative.prenominal}
 \is{ambiguity!relative clause}
Core argument relatives can be either head-internal or prenominal. When the relativized element is the transitive subject, prenominal position is more common (§\ref{sec:tr.subject.relativization}), but for intransitive subject (§\ref{sec:intr.subject.relativization}), direct object (§\ref{sec:object.relativization}) and quasi-objects (§\ref{sec:semi.object.relativization}), head-internal relatives are by far the most common type. 

When the relativized element is a noun such as \japhug{tɯ-tɕʰa}{information, news} or \japhug{ftɕaka}{method} which can take adnominal complements (§\ref{sec:complement.taking.nouns}), only head-internal relatives are possible, as prenominal clauses are interpreted as complements instead of relatives. For instance, the head noun \forme{tɯ-tɕʰa} in  (\ref{ex:tWtCha.jAkAGWt}) cannot be moved after the participle: $\dagger$\forme{a-tɕɯ kɯ jɤ-kɤ-ɣɯt \textbf{tɯ-tɕʰa} nɯ} is not accepted.

\begin{exe}
\ex \label{ex:tWtCha.jAkAGWt}
\gll [a-tɕɯ kɯ \textbf{tɯ-tɕʰa} jɤ-kɤ-ɣɯt] nɯ ɲɯ-pe \\
\textsc{1sg}.\textsc{poss}-son \textsc{erg} \textsc{indef}.\textsc{poss}-news \textsc{aor}-\textsc{obj}:\textsc{pcp}-bring \textsc{dem} \textsc{sens}-be.good \\
\glt `The information that my son has brought is pleasing.' (elicited)
\end{exe} 
  
\subsubsection{Head-internal vs. postnominal relatives} \label{sec:head-internal.relative.postnominal}
 \is{ambiguity!relative clause}
%tɤŋe ri mɯ-pjɤ-nɯrga, mɯntoʁ kɯ-mpɕɤr nɯra ri iɕqʰa, qaɲi nɯ kɯ mɤ-kɯ-nɯrga kɯ-fse ci pjɤ-ŋu tɕe, 
Since Japhug has strict verb-final order (§\ref{sec:basic.word.order}), in head-internal relative clauses the verb follows the head noun, as it would in a postnominal relative. In many cases it is indeed impossible to ascertain whether the head-noun belongs or not to the relative. For instance, in (\ref{ex:khWtsa.pWkABRum}) \forme{kʰɯtsa pɯ-kɤ-βʁum} `a bowl that has been turned upside down' there is no clear evidence for analyzing the head noun \japhug{kʰɯtsa}{bowl} as internal or external to the relative. 
  
\begin{exe}
\ex  \label{ex:khWtsa.pWkABRum}
\gll \textbf{kʰɯtsa} pɯ-kɤ-βʁum ʑo ɲɯ-fse. \\
bowl \textsc{aor}-\textsc{obj}:\textsc{pcp}-turn.upside.down \textsc{emph} \textsc{sens}-be.like \\
\glt `It looks like a bowl that has been turned upside down.' (23-mbrAZim)
\japhdoi{0003604\#S195}
\end{exe} 

However,  with other types of relative clauses there is sometimes positive evidence that the head noun is internal. When the function of the relativized element is that of transitive subject (§\ref{sec:tr.subject.relativization}), the presence of ergative marking can be a criterion for analyzing it as relative-internal. In (\ref{ex:WkWnWmbrApW}) for instance, the phrase \forme{tɤ-pɤtso ci kɯ} is necessarily internal to the participial relative clause, since the main verb \forme{jɤ-ɣe} is intransitive, and only the transitive verb \forme{ɯ-kɯ-nɯmbrɤpɯ} in the relative clause can have triggered ergative marking.\footnote{In addition, example (\ref{ex:WkWnWmbrApW}) provides an interesting case of double determiner marking (§\ref{sec:head-internal.relative.determiners}). } 
  
\begin{exe}
\ex  \label{ex:WkWnWmbrApW}
\gll  [\textbf{tɤ-pɤtso} \textbf{ci} \textbf{kɯ} <yangma> ɯ-kɯ-nɯmbrɤpɯ] ci jɤ-ɣe tɕe \\
 \textsc{indef}.\textsc{poss}-child \textsc{indef} \textsc{erg} bicycle \textsc{3sg}-\textsc{sbj}.\textsc{pcp}-ride \textsc{indef} \textsc{aor}-come[II] \textsc{lnk} \\
\glt `A boy who was riding a bicycle arrived.' (Pear story, \iai{Tshendzin})
\end{exe}

In (\ref{ex:tArJit.tAkWrqoR}) however, the transitive subject \forme{tɯrme} owes its ergative to the main verb \forme{ɲɯ-jtsʰi} rather than to the participle \forme{tɤ-kɯ-rqoʁ}. The clause \forme{tɤ-rɟit tɤ-kɯ-rqoʁ} `hugging the child' is analyzable either as an appositive postnominal relative clause, or as a participial clause.


\begin{exe}
\ex  \label{ex:tArJit.tAkWrqoR}
\gll  [tɯrme kɯ [tɤ-rɟit tɤ-kɯ-rqoʁ] tɯ-nɯ ɲɯ-jtsʰi] nɯ kɯ-fsɯ\redp{}fse ʑo ɲɯ-jtsʰi ɲɯ-ŋu.\\
person \textsc{erg} \textsc{indef}.\textsc{poss}-child \textsc{aor}-\textsc{sbj}:\textsc{pcp}-hug \textsc{indef}.\textsc{poss}-breast \textsc{ipfv}-give.to.drink \textsc{dem} \textsc{sbj}:\textsc{pcp}-\textsc{emph}\redp{}be.like \textsc{emph} \textsc{ipfv}-give.to.drink \textsc{sens}-be \\
\glt `[The monkey mother] nurses [her baby] in the same way as a human breastfeeds, hugging her child.' (19-GzW)
\japhdoi{0003536\#S26}
\end{exe}

In the case of object relatives,  the presence of the transitive subject (\forme{tɤ-tɕɯ nɯ kɯ} `the boy' in \ref{ex:WtaYi.kathW}), of an adjunct (the instrumental phrase \forme{rŋɯl kɯ} `from silver' in \ref{ex:qaR.thWkAsWBzu}) or of an adverb (\forme{atʰi} \textsc{downstream} in \ref{ex:kAntsGe.lukAGWt}) belonging to the relative before the head noun are sufficient criteria to show that the relative is head-internal, and cannot be analyzed as post-nominal.
 
\begin{exe}
\ex \label{ex:WtaYi.kathW}
\gll  [tɤ-tɕɯ nɯ kɯ \textbf{ɯ-tɤɲi} nɯ ka-tʰɯ] nɯ ɯ-taʁ kɤ-nɯ-ɬoʁ-ndʑi   \\
\textsc{indef}.\textsc{poss}-son \textsc{dem} \textsc{erg} \textsc{3sg}.\textsc{poss}-staff \textsc{dem} \textsc{aor}:3\flobv{}:\textsc{east}-spread \textsc{dem} \textsc{3sg}.\textsc{poss}-on \textsc{aor}:\textsc{east}-\textsc{auto}-come.out-\textsc{du} \\
\glt `They crossed [the river] on the staff that the boy had put across it as a bridge.' (2005 Kunbzang)
 \end{exe}  
 
\begin{exe}
\ex \label{ex:qaR.thWkAsWBzu}
\gll   [rŋɯl kɯ \textbf{qaʁ} tʰɯ-kɤ-sɯ-βzu] nɯra ko-sɯ-ɤʑirja-nɯ. \\
silver \textsc{erg} hoe \textsc{aor}-\textsc{obj}:\textsc{pcp}-\textsc{caus}-make \textsc{dem}:\textsc{pl} \textsc{ifr}-\textsc{caus}-be.aligned-\textsc{pl} \\
\glt `They lined up the hoes that had been made from silver.' (28-qajdoskAt)
\japhdoi{0003718\#S97}
\end{exe}  

\begin{exe}
\ex \label{ex:kAntsGe.lukAGWt}
\gll kɯki [atʰi \textbf{qaɕti} kɤ-ntsɣe lu-kɤ-ɣɯt] nɯ cʰo naχtɕɯɣ \\
\textsc{dem}.\textsc{prox} downstream peach \textsc{inf}-sell \textsc{ipfv}:\textsc{upstream}-\textsc{obj}:\textsc{pcp}-bring \textsc{dem} \textsc{comit} be.the.same:\textsc{fact} \\
\glt `[Wild peaches] are  like those peaches that are brought from the areas downstream (i.e. the Sichuan plains) to be sold.' (08-qaCti)
\japhdoi{0003456\#S45}
\end{exe}  
 
 \subsection{Postnominal} \label{sec:postnominal.relative}
 \is{relative clause!postnominal}
 There are only few unambiguous postnominal relative clauses in Japhug, since most relative clauses comprising a noun and a verb can be analyzed as minimal head-internal relatives. The only clear cases are provided by transitive subject (§\ref{sec:tr.subject.relativization}) participial relative clauses in \forme{kɯ-} (§\ref{sec:subject.participle.subject.relative}). In these cases, the head noun, being a transitive subject,  must take the ergative \forme{kɯ} (§\ref{sec:A.kW}) if embedded in the clause. The presence or absence of ergative flagging can therefore be used as a criterion to distinguish between postnominal and head-internal clauses (§\ref{sec:head-internal.relative.postnominal}). Examples are rather uncommon, since transitive subject relative clauses are most commonly prenominal (§\ref{sec:tr.subject.relativization}). 

Many examples of postnominal relatives are preceded by a pause, as in (\ref{ex:tWsNaR.WkWspa}), and are appositive clauses, rather than truly subordinate clauses. 

 \begin{exe}
\ex \label{ex:tWsNaR.WkWspa}
\gll \textbf{tɤ}-\textbf{mu} ci, [tɯ-sŋaʁ ra ɯ-kɯ-spa] ci pjɤ-tu tɕe \\
\textsc{indef}.\textsc{poss}-mother \textsc{indef} \textsc{nmlz}:\textsc{action}-cast.a.spell \textsc{pl} \textsc{3sg}.\textsc{poss}-\textsc{sbj}:\textsc{pcp}-be.able \textsc{indef} \textsc{ifr}.\textsc{ipfv}-exist \textsc{lnk} \\
\glt `There was an old woman who knew spells.' (150818 muzhi guniang-zh)
\japhdoi{0006334\#S15}
\end{exe}
 
Several examples of postnominal clauses are found with perception verbs with totalitative reduplication, such as  (\ref{ex:tWrme.pWpWkWmto2})  below, (\ref{ex:qartshaz.pWpWkWmtshAm}) in §\ref{sec:totalitative.relatives} above and (\ref{ex:tWrme.pWpWkWmto}) in §\ref{sec:tr.subject.relativization}.
 
 \begin{exe}
\ex \label{ex:tWrme.pWpWkWmto2}
\gll [\textbf{tɯrme} \textbf{ra} [pɯ\redp{}pɯ-kɯ-mto]] kɯ ``wo, nɯ ɯ-tɯ-pe nɯ" ntsɯ to-ti-nɯ. \\
person \textsc{pl} \textsc{total}\redp{}\textsc{aor}-\textsc{sbj}:\textsc{pcp}-see \textsc{erg} \textsc{interj} \textsc{dem} \textsc{3sg}.\textsc{poss}-\textsc{nmlz}:\textsc{deg}-be.good \textsc{sfp} always \textsc{ifr}-say-\textsc{pl} \\
\glt `All the people who saw it said `It is so nice!'.' (150827 mengjiangnv-zh)
\japhdoi{0006290\#S24}
\end{exe}
 
When the relativized element is the intransitive subject, head-internal and postnominal relatives can be distinguished by the relative position of head nouns and adjuncts. It is particularly clear in the case of the comparative construction (§\ref{comparee.relativization}).


\section{Function of the relativized element}\label{sec:function.relativization} 
This section presents a classification of relative clauses based on the syntactic function of the relativized element inside the relative. 

Core arguments, possessor of core arguments as well as a certain number of adjuncts can be relativized using participial or finite clauses. The syntactic functions that are not accessible to relativization are listed in §\ref{sec:accessibility.relativization}.


\subsection{Intransitive subject} \label{sec:intr.subject.relativization} 
\is{relative clause!intransitive subject} \is{subject!relativization}
The only way to relativize intransitive subjects in Japhug is by a subject participial relative in \forme{kɯ-} (§\ref{sec:subject.participle.subject.relative}). 

\subsubsection{Position of the head noun} \label{sec:intr.subj.head.noun.position}
When the head noun of an intransitive subject relative is overt, it is generally located before the participle as in (\ref{ex:tArZaB.jAkWGe}). This example can be analyzed either as a postnominal relative (§\ref{sec:postnominal.relative}) or as minimal head-internal relative (§\ref{sec:head-internal.relative}).

\begin{exe}
\ex \label{ex:tArZaB.jAkWGe}
\gll [\textbf{tɤ-rʑaβ} jɤ-kɯ-ɣe] nɯnɯ kʰro mɯ-pɯ-sna ɲɯ-ŋu.  \\
\textsc{indef}.\textsc{poss}-wife \textsc{aor}-\textsc{sbj}:\textsc{pcp}-come[II] \textsc{dem} much \textsc{neg}-\textsc{pst}.\textsc{ipfv}-be.good \textsc{sens}-be \\
\glt `The wife who had come [to their house] (i.e. that they had married) was not nice.' (meimeidegushi)
\end{exe} 

The presence of adjuncts, such as \forme{kɯ-mɤku} `before' (§\ref{sec:velar.inf.adverb}) before the head noun offers evidence that, in some cases, the head-internal relative analysis is preferable (§\ref{sec:head-internal.relative}). Example (\ref{ex:kWnArWra.jAkAri}) also illustrates that head-internal relatives with embedded purposive complements located between the head noun and the verb are possible.

\begin{exe}
\ex \label{ex:kWnArWra.jAkAri}
\gll iɕqʰa [kɯ-mɤku \textbf{tɕaχpa} [kɯ-nɤrɯra] jɤ-kɯ-ɤri] nɯ ʁnɯz nɯ pjɤ-sat. \\
the.aforementioned \textsc{inf}:\textsc{stat}-be.before thief \textsc{sbj}:\textsc{pcp}-look.around \textsc{aor}-\textsc{sbj}:\textsc{pcp}-go[II]  \textsc{dem} two \textsc{dem} \textsc{ifr}-kill \\
\glt `He killed the two thieves who had gone scouting just before.' (140512 alibaba-zh)
\japhdoi{0003965\#S188}
\end{exe} 

Prenominal clauses (§\ref{sec:prenominal.relative}) are less common than head-internal or postnominal ones in the case of intransitive subject relativization, but still attested, especially in the case of multiple relative clauses sharing the same head noun, as in (\ref{ex:mWNi.kAkWGe.XpWn}).
 
\begin{exe}
\ex \label{ex:mWNi.kAkWGe.XpWn}
\gll 	 [mɯŋi kɤ-kɯ-ɣe] \textbf{χpɯn} [tʰɯ-kɯ-rgɤz] ci pjɤ-tu tɕe \\
\textsc{topo} \textsc{aor}:\textsc{east}-\textsc{sbj}:\textsc{pcp}-come[II] monk \textsc{aor}-\textsc{sbj}:\textsc{pcp}-be.old \textsc{indef} \textsc{ipfv}.\textsc{ifr}-exist \\
\glt `There was an old monk who had come from Mengi.' (08-kWqhi)
\japhdoi{0003454\#S18}
\end{exe} 

Participles of adjectival stative verbs in attributive function are one of the most common type of subject relative clauses (this construction is discussed in more detail in §\ref{ex:attributive.participles.stative.verbs}).

\subsubsection{Comparee} \label{comparee.relativization}
\is{relative clause!comparee}
The compared element of comparative constructions is relativized like a normal intransitive subjects, but with an overt standard. Head-internal (\ref{ex:WZo.sAz.rWdaR.kWxtCi}), postnominal (\ref{ex:rWdaR.WZo.sAz.kWxtCi}) and prenominal (\ref{ex:WZo.sAz.kWxtCi.qajW}) relatives are all attested and equally common, and can be distinguished by the relative position of the head noun and the standard of comparison \forme{ɯʑo sɤz} `than him/her/itself'.

\begin{exe}
\ex 
\begin{xlist}
\ex \label{ex:WZo.sAz.rWdaR.kWxtCi}
\gll [ɯʑo sɤz \textbf{rɯdaʁ} kɯ-xtɕi] nɯra tu-ndze \\
\textsc{3sg} \textsc{comp} animal \textsc{sbj}:\textsc{pcp}-be.small \textsc{dem}:\textsc{pl} \textsc{ipfv}-eat[III] \\
\glt `It eats the animals that are smaller than it is.'(20-sWNgi)
\japhdoi{0003562\#S22}
\ex \label{ex:rWdaR.WZo.sAz.kWxtCi}
\gll \textbf{rɯdaʁ} [ɯʑo sɤz kɯ-xtɕi], \textbf{pɣa} [ɯʑo sɤz kɯ-xtɕi] nɯra tu-ndze ɲɯ-ɕti. \\
animal \textsc{3sg} \textsc{comp} \textsc{sbj}:\textsc{pcp}-be.small bird \textsc{3sg} \textsc{comp} \textsc{sbj}:\textsc{pcp}-be.smal \textsc{dem}:\textsc{pl} \textsc{ipfv}-eat[III] \textsc{sens}-be.\textsc{aff} \\
\glt `It eats the animals and the birds that are smaller than it is.'(24-ZmbrWpGa)
\japhdoi{0003628\#S87}
\ex \label{ex:WZo.sAz.kWxtCi.qajW}
\gll  [ɯʑo sɤz kɯ-xtɕi] \textbf{qajɯ} ra tu-ndze ŋgrɤl mɤ-ŋgrɤl mɤ-xsi ma \\
\textsc{3sg} \textsc{comp} \textsc{sbj}:\textsc{pcp}-be.small  bug \textsc{pl} \textsc{ipfv}-eat[III] be.usually.the.case:\textsc{fact} \textsc{neg}-be.usually.the.case:\textsc{fact} \textsc{neg}-\textsc{genr}:know \textsc{lnk} \\
\glt `I don't know whether it eats the bugs that are smaller than itself.' (26-kWrNukWGndZWr)
\japhdoi{0003672\#S48}
\end{xlist}
\end{exe} 


\subsection{Transitive subject}  \label{sec:tr.subject.relativization}
\is{relative clause!transitive subject} \is{subject!relativization}
Transitive subjects are exclusively relativized using subject participial relative clauses in \forme{kɯ-} (§\ref{sec:subject.participle.subject.relative}) like intransitive subjects (§\ref{sec:intr.subject.relativization}), but take a possessive prefix coreferent with the object (§\ref{sec:subject.participle.possessive}), unless another prefix is present (§\ref{sec:subject.participle.other.prefixes}).

In a minority of cases, the transitive subject head noun can occur before the verb. In this position, it sometimes takes the ergative, as illustrated by the phrase \forme{tɤ-nmaʁ nɯ kɯ} in (\ref{ex:tAnmaR.nW.kW.YWkWnWCar}). 
 \largerpage
\begin{exe}
\ex \label{ex:tAnmaR.nW.kW.YWkWnWCar}
\gll [\textbf{tɤ-nmaʁ} \textbf{nɯ} \textbf{kɯ} ɯ-rʑaβ kɯ-ɤntɕʰɯ ɲɯ-kɯ-nɯ-ɕar], [aʁɤndɯndɤt tɤndɤɣri tu-kɯ-βzu] pjɤ-tu. \\
\textsc{indef}.\textsc{poss}-husband \textsc{dem} \textsc{erg} \textsc{3sg}.\textsc{poss}-wife \textsc{sbj}:\textsc{pcp}-be.several \textsc{ipfv}-\textsc{sbj}:\textsc{pcp}-\textsc{auto}-search everywhere illegitimate.child \textsc{ipfv}-\textsc{sbj}:\textsc{pcp}-make \textsc{ifr}.\textsc{ipfv}-exist \\
\glt `There were men who had several female companions, and had illegitimate children everywhere.' (140427 tAndAGri)
\japhdoi{0003858\#S3}
\end{exe} 

The presence of the ergative here unambiguously indicates that the relative is head-internal and that the transitive subject is embedded inside it, since the main verb \forme{pjɤ-tu} `there used to be' is intransitive (§\ref{sec:head-internal.relative.prenominal}).

Alternatively, the transitive subject can occur in absolutive form (for example \forme{tɯrme} `person' in \ref{ex:tWrme.pWpWkWmto}),\footnote{In (\ref{ex:tWrme.pWpWkWmto}), the ergative \forme{kɯ} follows the relative clause, which has transitive subject function in the main clause. The head noun \forme{tɯrme} itself is not assigned ergative case by the transitive participle \forme{pɯ\redp{}pɯ-kɯ-mto}. } showing that the relative is postnominal (§\ref{sec:postnominal.relative}).


\begin{exe}
\ex \label{ex:tWrme.pWpWkWmto}
\gll [\textbf{tɯrme} [pɯ\redp{}pɯ-kɯ-mto]] kɯ ɲɯ-nɤ-mpɕɤr-nɯ tɕe ``ɲɯ-pe" tu-ti-nɯ pjɤ-ŋu \\
person \textsc{total}\redp{}\textsc{aor}-\textsc{sbj}:\textsc{pcp}-see \textsc{erg} \textsc{sens}-\textsc{trop}-be.beautiful-\textsc{pl} \textsc{lnk} \textsc{sens}-be.good \textsc{ipfv}-say-\textsc{pl} \textsc{ifr}.\textsc{ipfv}-be \\
\glt `All the people who saw it found it beautiful and said it was nice.' (140510 sanpian yumao-zh)
\japhdoi{0003947\#S83}
\end{exe}


There are no examples of double ergative marking, with a head-internal relative  in transitive subject function in the main clause itself, taking the ergative both on the head noun and at the end of the clause, for instance as construction such as ?\forme{[tɯrme kɯ pɯ\redp{}pɯ-kɯ-mto] kɯ} instead of \forme{[tɯrme [pɯ\redp{}pɯ-kɯ-mto]] kɯ} in (\ref{ex:tWrme.pWpWkWmto}).

%not to confuse with:
%ɯ-ɕna kɯ qale ju-kɯ-sɯ-lɤt tɯrme nɯ kɯ /ma/ kɯ-mbɯ-mbat ʑo ɯ-ɕna ɯ-ŋgɯ ci qale jo-sɯɣe qhe, 140505_liuhaohan_zoubian_tianxia
\subsection{Object} \label{sec:object.relativization}
\is{relative clause!object} \is{object!relativization}
Relativization of direct objects allows for a greater variety of constructions than that of subjects: both finite and participial relative clauses are possible. 

In object participial relatives, the participle can be prefixed with an orientation preverb as in (\ref{ex:tCheme.pWkAsat}), or with a possessive prefix coreferent with transitive subject. In the first case (restricted to third person subjects), the relatives are most commonly head-internal (or postnominal) as in (\ref{ex:tCheme.pWkAsat}), but other constructions are possible. Attested types of object participial relatives are described in §\ref{sec:object.participle.relatives}.

\begin{exe}
\ex \label{ex:tCheme.pWkAsat}
\gll  [\textbf{tɕʰeme} pɯ-kɤ-sat] nɯ pɣɤtɕɯ ci to-sci qʰe, \\
girl \textsc{aor}-\textsc{sbj}:\textsc{pcp}-kill \textsc{dem} bird \textsc{indef} \textsc{ifr}:\textsc{up}-be.born \textsc{lnk} \\
\glt `The girl who had been killed was reborn as a bird.' (2014-kWlAG)
\end{exe}

Since object participles cannot take both possessive prefixes and orientation preverbs (§\ref{sec:object.participle.possessive}), finite relativization is the only way to specify both TAME and person in an object relative, as in (\ref{ex:qajW.tutia}) and (\ref{ex:ndzrW.thWtWfset}).

\begin{exe}
\ex \label{ex:qajW.tutia}
\gll  nɯ [\textbf{qajɯ} \textbf{kɯ-ɲaʁ} tu-ti-a] nɯ nɯ kɯ-fse ɲɯ-βze ɲɯ-ŋu. \\
\textsc{dem} worm \textsc{sbj}:\textsc{pcp}-black \textsc{ipfv}-say-\textsc{1sg} \textsc{dem} \textsc{dem} \textsc{sbj}:\textsc{pcp}-be.like \textsc{ipfv}-grow \textsc{sens}-be \\
\glt `The black worm that I am talking about grows like that.' (28-kWpAz)
\japhdoi{0003714\#S30}
\end{exe}

\begin{exe}
\ex \label{ex:ndzrW.thWtWfset}
\gll nɤʑo [iɕqʰa \textbf{ndzrɯ} tʰɯ-tɯ-fse-t] nɯ ci tɤ-nɯ-tsʰɤt tɕe\\
\textsc{2sg} just.before chisel \textsc{aor}-2-whet-\textsc{pst}:\textsc{tr} \textsc{dem} a.little \textsc{imp}-\textsc{auto}-try \textsc{lnk}\\
\glt `Try the chisel that you have just whetted.' (150902 luban-zh)
\japhdoi{0006268\#S116}
\end{exe}


Object finite relatives with overt head noun are mainly head-internal, as (\ref{ex:ndzrW.thWtWfset}) (see also §\ref{sec:head-internal.relative.postnominal}), or ambiguous between postnominal and head-internal as in (\ref{ex:qajW.tutia}). Prenominal finite relatives are attested, but rarer. All four possibilities to relativize objects (finite vs? participial, head-internal vs. prenominal) are illustrated in (\ref{ex:pGa.mto}).

\begin{exe}
\ex \label{ex:pGa.mto}
\begin{xlist}
\ex \label{ex:pGa.pamto}
\gll [a-tɕɯ kɯ \textbf{pɣa} pa-mto] nɯ \\
\textsc{1sg}.\textsc{poss}-son \textsc{erg} bird \textsc{aor}:3\flobv{}-see \textsc{dem} \\
\ex \label{ex:pGa.pWkAmto}
\gll [a-tɕɯ kɯ \textbf{pɣa} pɯ-kɤ-mto] nɯ \\
\textsc{1sg}.\textsc{poss}-son \textsc{erg} bird \textsc{aor}-\textsc{obj}:\textsc{pcp}-see \textsc{dem} \\
\ex \label{ex:pamto.pGa}
\gll [a-tɕɯ kɯ pa-mto] \textbf{pɣa} nɯ \\
\textsc{1sg}.\textsc{poss}-son \textsc{erg}  \textsc{aor}:3\flobv{}-see bird \textsc{dem} \\
\ex \label{ex:pWkAmto.pGa}
\gll [a-tɕɯ kɯ pɯ-kɤ-mto] \textbf{pɣa} nɯ \\
\textsc{1sg}.\textsc{poss}-son \textsc{erg}  \textsc{aor}-\textsc{obj}:\textsc{pcp}-see bird \textsc{dem} \\
\end{xlist}
\glt `The bird that my son saw' (elicited)
\end{exe}

There are some contexts where finite relatives have to be head-internal or postnominal and where the prenominal position is agrammatical (§\ref{sec:head-internal.relative.prenominal}). 


\subsubsection{Monotransitive verbs} \label{sec:monotransitive.object.relativization}
When the object of monotransitive verbs are relativized using a finite relative, the verb must be in direct form, even when the transitive subject of the relative has a possessive prefix coreferent with the object. For instance, in (\ref{ex:Wmu.kW.kanWpoR}), only the 3\flobv{} form \forme{ka-nɯpoʁ} is possible, the inverse 3$'$\fl{}3 configuration \forme{kɤ́-wɣ-nɯpoʁ} is not possible in the context. 

\begin{exe}
\ex \label{ex:Wmu.kW.kanWpoR}
\gll [ɯ$_i$-mu kɯ ka-nɯpoʁ] \textbf{tɤ-pɤtso}$_i$ nɯ \\
\textsc{3sg}.\textsc{poss}-mother \textsc{erg} \textsc{aor}:3\flobv{}-kiss \textsc{indef}.\textsc{poss}-child \textsc{dem} \\
\glt `The child whose mother kissed him' (elicited)
\end{exe}

In addition, only third person object forms (excluding inverse 3\fl{}1/2 and local configurations) of monotransitive verbs can be relativized. This constraint does not apply to triactantial causative (§\ref{sec:object.causative.relativization}) or secundative verbs (§\ref{sec:secundative.theme.relativization}).
%a-laχtɕha ɯ-pɯ tɤ-kɤ-pa nɯra ɲɤ-me.
\subsubsection{Theme of indirective verbs} \label{sec:indirective.relativization}
\is{relative clause!theme}
The theme of indirective verbs such as \japhug{kʰo}{give} or \japhug{ti}{say} (§\ref{sec:ditransitive.indirective}) has the same morphosyntactic status as the direct object of a monotransitive verb, both from the point of view of person indexation and of relativization: it is also relativizable by both finite clauses (\ref{ex:tAkAtWt.tosWRjit}) and object participial clauses (\ref{ex:mbala.kW.tatWt}).

\begin{exe}
\ex \label{ex:tAkAtWt.tosWRjit}
\gll tɤtɕɯpɯ nɯ kɯ [iɕqʰa mbala-do nɯ kɯ tɤ-kɤ-tɯt] nɯ to-sɯʁjit. \\
boy \textsc{dem} \textsc{erg} just.before ox-old \textsc{dem} \textsc{erg} \textsc{aor}-\textsc{obj}:\textsc{pcp}-say[II] \textsc{dem} \textsc{ifr}-remember \\
 \glt `The boy remembered [what] the old ox had told him.' (150828 niulang-zh)
\japhdoi{0006318\#S56}
 \end{exe}
 
 \begin{exe}
\ex \label{ex:mbala.kW.tatWt}
\gll tɕeri [mbala kɯ ta-tɯt] nɯ to-stu \\
\textsc{lnk} ox \textsc{erg} \textsc{aor}:3\flobv{}-say[II] \textsc{dem} \textsc{ifr}-do.like \\
\glt `He did it [the way that] the ox had said.' (150828 niulang-zh)
\japhdoi{0006318\#S133}
\end{exe} 

\subsubsection{Object of causativized transitive verbs} \label{sec:object.causative.relativization}
Causativized transitive verbs are triactantial (causer, causee and object, §\ref{sec:ditransitive.causative}) and differ from monotransitive verbs (§\ref{sec:monotransitive.object.relativization}) in that their object can be relativized using finite relatives with a verb in inverse configuration (\ref{ex:tAwGznAmYo.nWra}) or with a first or second person object (including local configurations, as in \ref{ex:jAkWsWCGaza.nW}).  
\largerpage
\begin{exe}
\ex \label{ex:tAwGznAmYo.nWra}
\gll [tɤ́-wɣ-z-nɤmɲo] nɯra kɯ\redp{}kɯ-fsɯ\redp{}fse ʑo to-βzu pjɤ-cʰa.   \\
\textsc{aor}-\textsc{inv}-\textsc{caus}-watch \textsc{dem}:\textsc{pl} \textsc{total}\redp{}\textsc{sbj}:\textsc{pcp}-\textsc{emph}\redp{}be.like \textsc{emph} \textsc{ifr}-make \textsc{ifr}-can \\
\glt `[Luban]$_i$ succeeded in recreating all [the objects] that [his teacher]$_j$ had shown him$_i$ exactly as they were [before].' (150902 luban-zh)
\japhdoi{0006268\#S152}
\end{exe} 
 
\begin{exe}
\ex \label{ex:jAkWsWCGaza.nW}
\gll [\textbf{mbalɤ-pɯ} jɤ-kɯ-sɯ-ɕɣaz-a] nɯ pa-mto tɕe \\
ox-\textsc{dim} \textsc{aor}-2\fl{}1-\textsc{caus}-take.back-\textsc{1sg} \textsc{dem} \textsc{aor}:3\flobv{}-see \textsc{lnk} \\
\glt `She saw the calf that you had me take back home.' (140512 fushang he yaomo1-zh)
\japhdoi{0003969\#S128}
\end{exe} 

In examples (\ref{ex:tAwGznAmYo.nWra}) and (\ref{ex:jAkWsWCGaza.nW}), the person configuration indexes causer and causee, not the direct object. 

The theme of secundative verbs can be relativized using the same type of constructions (§\ref{sec:secundative.theme.relativization}).

\subsubsection{Only argument of transitive verbs with dummy subjects}  \label{sec:dummy.subj.object.relativization}
The only argument of dummy transitive verbs (§\ref{sec:transitive.dummy}), despite resembling a direct object, is relativized with subject participles, as \forme{tu-kɯ-rku} in (\ref{ex:tCHWwWr.turke}) (§\ref{sec:rku.lv}) and \forme{ku-kɯ-tsʰoʁ} in (\ref{ex:Wmat.kukWtshoR}) (§\ref{sec:tshoR.lv}). 

\begin{exe}
\ex \label{ex:tCHWwWr.turke}
\gll nɯre ri tɕʰɯwɯr tu-rke ŋu tɕe, tɕe [\textbf{tɕʰɯwɯr} tu-kɯ-rku] nɯnɯ cimbɤrom ŋu \\
\textsc{dem}:\textsc{loc} \textsc{loc} blister \textsc{ipfv}-put.in[III] be:\textsc{fact} \textsc{lnk} \textsc{lnk} [\textbf{blister} \textsc{ipfv}-\textsc{sbj}:\textsc{pcp}-put.in] \textsc{dem} phlycten be:\textsc{fact} \\
\glt `(At the place where the skin is burnt), a blister forms there, the blister that forms is a phlycten.' (27-tWfCAl 122)
\japhdoi{0003710\#S115}
\end{exe} 

\begin{exe}
\ex \label{ex:Wmat.kukWtshoR}
\gll tɕe iɕqʰa mbrɤz ɣɯ [\textbf{ɯ-mat} ku-kɯ-tsʰoʁ] nɯ ɯ-tsʰɯɣa ɲɯ-fse \\
\textsc{lnk} the.aforementioned rice \textsc{gen} \textsc{3sg}.\textsc{poss}-fruit \textsc{ipfv}-\textsc{sbj}:\textsc{pcp}-attach \textsc{dem} \textsc{3sg}.\textsc{poss}-shape \textsc{sens}-be.like \\
\glt `Its shape is a bit like that of grains of rice (growing on the stalk).' (19-khWlu)
\japhdoi{0003540\#S85}
\end{exe}

Finite clauses and object participial clauses cannot be used to relativize these arguments. This is one of the clues (§\ref{sec:dental.inf}, §\ref{sec:transitive.dummy}) that the verbs in these constructions are only partially transitive.

\subsection{Quasi-objects} \label{sec:semi.object.relativization}
A certain number of absolutive arguments that are not indexed as direct objects on the verb (§\ref{sec:polypersonal})) have objectal properties: the semi-objects of
semi-transitive verbs (§\ref{sec:semi.transitive}) and the themes of secundative verbs (§\ref{sec:ditransitive.secundative}). They can be
relativized like direct objects of monotransitive verbs, both with finite and participial
relatives \citep{jacques16relatives}.

%{sec:locative.relativization}
 %{sec:object.participle.relatives} 

  \subsubsection{Semi-objects}  \label{sec:semi.tr.relativization}
  \is{relative clause!semi-object} \is{semi-object!relativization}
 Semi transitive verbs like \japhug{rga}{like}, \japhug{βɟɤt}{obtain} or \japhug{aro}{own} are morphologically intransitive and their subject is marked in the absolutive (§\ref{sec:semi.transitive}), but they take in addition an absolutive semi-object (§\ref{sec:semi.object}). The semi-object can be relativized with either object participial relatives (\ref{ex:pGa.ra.nWkArga}) or finite relative clauses as in (\ref{ex:pWBJata.nW}) and (\ref{ex:aroa.nW}).
 
  \begin{exe}
\ex \label{ex:pGa.ra.nWkArga}
\gll   [pɣa ra nɯ-kɤ-rga] nɯ qaj ntsɯ ŋu   \\
bird \textsc{pl} \textsc{3pl}-\textsc{obj}:\textsc{pcp}-like \textsc{dem} wheat always be:\textsc{fact} \\
  \glt `[The food] that birds like is always wheat (not barley).' (23 pGAYaR)
\japhdoi{0003606\#S24}
   \end{exe} 
   
  \begin{exe}
\ex \label{ex:pWBJata.nW}
\gll [\textbf{laχtɕʰa} pɯ-βɟat-a] nɯ kʰɯtsa pɯ-ŋu \\
thing \textsc{aor}-obtain-\textsc{1sg} \textsc{dem} bowl \textsc{pst}.\textsc{ipfv}-be \\
\glt `The thing that I obtained was a bowl.' (elicited)
   \end{exe} 
   
  \begin{exe}
\ex \label{ex:aroa.nW}
\gll [aʑo \textbf{qaʑo} aro-a] nɯ kɯki ŋu  \\
\textsc{1sg} \textsc{sheep} own:\textsc{fact}-\textsc{1sg} \textsc{dem} \textsc{dem}.\textsc{prox} be:\textsc{fact} \\
  \glt `The sheep that I own is this one.' (elicitation)
   \end{exe} 
   
 \subsubsection{Theme of secundative verbs}  \label{sec:secundative.theme.relativization}
Secundative verbs such as \japhug{mbi}{give} or \japhug{sɯxɕɤt}{teach} index the recipient as direct object, but also take a non-indexed absolutive argument referring to the theme (§\ref{sec:ditransitive.secundative}). When both the subject and the direct object are third person, theme can be relativized either with a finite clause (\ref{ex:sla.kW.nWwGmbi}, \ref{ex:pWwGsWxCAt.pjAsWxCAt}) or an object participial clause (example \ref{ex:tWNga.nWkAmbi}, §\ref{sec:relative.possessor.neutralization}).  Head-internal clauses are most common (\ref{ex:sla.kW.nWwGmbi}), but prenominal and even genitival ones (§\ref{sec:genitival.relatives}) are also found, as in (\ref{ex:pWwGsWxCAt.pjAsWxCAt}).
 
\begin{exe}
\ex \label{ex:sla.kW.nWwGmbi}
\gll  [sla kɯ, nɤki, \textbf{kumpɣɤ-ŋgɯm} nɯ́-wɣ-mbi] nɯ pjɤ-qrɯ \\
moon \textsc{erg} \textsc{filler} hen-egg \textsc{aor}-\textsc{inv}-give \textsc{dem} \textsc{ifr}-break \\
\glt `She broke the egg that the moon had given her.' (140506 shizi he huichang de bailingniao-zh)
\japhdoi{0003927\#S252}
\end{exe} 
 
 
\begin{exe}
\ex \label{ex:pWwGsWxCAt.pjAsWxCAt}
\gll [tɤ-tɕɯ nɯ kɯ, saŋrɟɤz ra kɯ pɯ́-wɣ-sɯxɕɤt] ɣɯ \textbf{kʰɤndɯn} nɯ ɯ-rʑaβ χsɯm nɯ pjɤ-sɯxɕɤt \\
\textsc{indef}.\textsc{poss}-son \textsc{dem} \textsc{erg} buddha \textsc{pl} \textsc{erg} \textsc{aor}-\textsc{inv}-teach \textsc{gen} mantra \textsc{dem} \textsc{3sg}.\textsc{poss}-wife three \textsc{dem} \textsc{ifr}-teach \\
\glt `He taught his three wives the mantra that the Buddhas had taught him.' (2012 Norbzang)
\japhdoi{0003768\#S311}
\end{exe} 

In finite relative clauses, the inverse 3$'$\fl{}3 configuration is most often found when the transitive subject of the main clause corresponds to the direct object (recipient) of the relative clause, as in (\ref{ex:sla.kW.nWwGmbi}) and (\ref{ex:pWwGsWxCAt.pjAsWxCAt}), and the relativized element is the theme. Secundative verbs resemble in this regard causativized transitive verbs (§\ref{sec:object.causative.relativization}) and differ from monotransitive verbs, whose objects cannot be relativized using a finite  3$'$\fl{}3 configuration (§\ref{sec:monotransitive.object.relativization}).

The themes of secundative verbs being relativized with the same constructions as their direct objects, ambiguity can arise especially in the case of short relatives. For instance, the participle \forme{a-kɤ-sɯxɕɤt} of \japhug{sɯxɕɤt}{teach} can be understood as relativizing the direct object/recipient (\ref{ex:akAsWxCAt.lhamo}) or the theme (\ref{ex:akAsWxCAt.shuxue}). The antipassive can be used to disambiguate: since this derivation removes the object, the only interpretation left for the antipassive participle \forme{a-kɤ-sɤ-sɯxɕɤt} is theme relativization `the (subject) that I taught' (see also example \ref{ex:nAkArAmbi}, §\ref{sec:object.participle.relatives}).

\begin{exe}
\ex 
\begin{xlist}
\ex \label{ex:akAsWxCAt.lhamo}
\gll  [a-kɤ-sɯxɕɤt] nɯ ɬamu pɯ-ŋu. \\
\textsc{1sg}-\textsc{obj}:\textsc{pcp}-teach \textsc{dem}  \textsc{anthr} \textsc{pst}.\textsc{ipfv}-be \\
\glt `The person whom I taught it to was Lhamo.' (elicited)
\ex \label{ex:akAsWxCAt.shuxue}
\gll  [a-kɤ-sɯxɕɤt] nɯ  <shuxue> pɯ-ŋu. \\
\textsc{2sg}-\textsc{obj}:\textsc{pcp}-teach \textsc{dem} mathematics \textsc{pst}.\textsc{ipfv}-be \\
\glt `The [subject] that I taught them/him/her was maths.' (elicited)
\ex \label{ex:akAsAsWxCAt}
\gll [a-kɤ-sɤ-sɯxɕɤt] <yuwen>  pɯ-ŋu. \\
\textsc{2sg}-\textsc{obj}:\textsc{pcp}-\textsc{antip}-teach Chinese what \textsc{pst}.\textsc{ipfv}-be \\
\glt `The [subject] that I taught them/him/her was Chinese.' (elicited)
\end{xlist} 
\end{exe} 

Finite relatives in direct form can also be used to relativize both the direct object (\ref{ex:nWtWmbit.CW}) or the theme (\ref{ex:nWmbita.kWmtChW}). 

\begin{exe}
\ex 
\begin{xlist}
\ex \label{ex:nWtWmbit.CW}
\gll [rŋɯl nɯ-tɯ-mbi-t] nɯ ɕɯ pɯ-ŋu? \\
silver \textsc{aor}-2-give-\textsc{pst}:\textsc{tr} \textsc{dem} who \textsc{pst}.\textsc{ipfv}-be \\
\glt `To whom did you give money?' (elicited)
\ex \label{ex:nWmbita.kWmtChW}
\gll [a-tɕɯ \textbf{tɤ-pɤro} nɯ-mbi-t-a] nɯ kɯmtɕʰɯ pɯ-ŋu \\
 \textsc{1sg}.\textsc{poss}-son \textsc{indef}.\textsc{poss}-present \textsc{aor}-give-\textsc{pst}:\textsc{tr}-\textsc{1sg} \textsc{dem} toy \textsc{pst}.\textsc{ipfv}-be \\
 \glt `The present I gave my son was a toy.' (elicited)
 \end{xlist} 
\end{exe} 

Finite clauses are needed to specify both TAME and a first or second person subject, as in (\ref{ex:nWtWmbit.CW}) and (\ref{ex:nWmbita.kWmtChW}), as object participles cannot index the subject (\ref{ex:akAsWxCAt.lhamo}) while at the time taking an orientation preverb.

Finite relative clauses are also required to index a first or second person direct object (recipient), for instance in local configurations such as  `the thing that I have (taught/given) you' in (\ref{ex:pWtasWxCAt.nWndWn}). This type of clause is only interpretable as theme relativization. 

\begin{exe}
\ex \label{ex:pWtasWxCAt.nWndWn}
\gll  a-tɕɯ tɯ\redp{}tɯ-ŋu nɤ, [pɯ-ta-sɯxɕɤt] nɯ ci nɯ-ndɯn ra \\
\textsc{1sg}.\textsc{poss}-son \textsc{cond}\redp{}2-be:\textsc{fact} \textsc{add} \textsc{aor}-1\fl{}2-teach \textsc{dem} a.little \textsc{imp}-recite be.needed:\textsc{fact} \\
\glt `If you are my son, recite [the mantra] that I have taught you.' (2012 Norbzang)
\japhdoi{0003768\#S188}
\end{exe} 
%	a-laχtɕha ɯ-pɯ tɤ-kɤ-pa nɯra ɲɤ-me. 


\subsection{Goal and locative} \label{sec:locative.relativization}
\is{relative clause!goal} \is{goal!relativization}
\is{relative clause!locative} \is{locative!relativization}
Relativization of locative/goal adjuncts or arguments with oblique participial clauses  is described in  §\ref{sec:locative.participle.relatives}. This section presents the relativization of locative phrases with finite clauses, object participles and relator nouns.
 

\subsubsection{Finite relativization}  \label{sec:locative.relativization.finite}
Locative marking on goals and locative arguments and adjuncts is optional, and they can occur in absolutive form (§\ref{absolutive.goal}), like semi-objects (§\ref{sec:semi.object}). Another commonality between goal/locative arguments and semi-objects is the ability to be relativized using finite relative clauses (§\ref{sec:semi.tr.relativization}). 

Finite locative relative clauses are most often prenominal, in particular with a genitive marker (§\ref{sec:genitival.relatives}), as in (\ref{ex:YWCenW.GW.tsxu}).

\begin{exe}
\ex \label{ex:YWCenW.GW.tsxu}
\gll [ʑara kɯ-lɤɣ ɲɯ-ɕe-nɯ] ɣɯ \textbf{tʂu} ci tu tɕe  \\
\textsc{3pl} \textsc{sbj}:\textsc{pcp}-herd \textsc{ipfv}:\textsc{west}-go-\textsc{pl} \textsc{gen} path \textsc{indef} exist:\textsc{fact} \textsc{lnk} \\
\glt `(At that place), there is a path which they take to go herd [cattle].' (140522 Kamnyu zgo)
\japhdoi{0004059\#S292}
\end{exe} 

Headless (\ref{ex:kurAZi.khu.Wsta}) and head-internal (\ref{ex:kha.jAwGtsWmnW}) finite locative relatives are also attested. In the latter case, the head noun cannot receive locative case, and must be in absolutive form.

\begin{exe}
\ex \label{ex:kurAZi.khu.Wsta}
\gll [ku-rɤʑi] nɯ kʰu ɯ-sta ɕti ndɤre \\
\textsc{ipfv}-stay \textsc{dem} tiger \textsc{3sg}.\textsc{poss}-place be.\textsc{aff}:\textsc{fact} \textsc{lnk} \\
\glt `The place where he is [now] is a tiger's lair.' (2003kandZislama)
\end{exe} 

\begin{exe}
\ex \label{ex:kha.jAwGtsWmnW}
\gll [\textbf{kʰa} jɤ́-wɣ-tsɯm-nɯ] nɯnɯ,  [lonba ɕom kɯ nɯ-kɤ-sɯ-βzu] kʰa pjɤ-ŋu \\
house \textsc{ifr}-\textsc{inv}-take.away-\textsc{pl} \textsc{dem} all iron \textsc{erg} \textsc{aor}-\textsc{obj}:\textsc{pcp}-\textsc{caus}-make house \textsc{ifr}.\textsc{ipfv}-be \\
\glt `The house where [the king] had taken them, it was a house completely made of iron' (140505 liuhaohan zoubian tianxia-zh)
\japhdoi{0003913\#S149}
\end{exe}

Non-permanent and non-specific location can be relativized with finite relative clauses, as in (\ref{ex:kurAZi.khu.Wsta}) and (\ref{ex:kha.jAwGtsWmnW}). Thus, the finite relative \forme{ku-rɤʑi nɯ} in (\ref{ex:kurAZi.khu.Wsta}) can be translated as `the place where he happens to be' while `the place where he stays (permanently), his staying place' is better expressed with an oblique participle \forme{ɯ-(sɤ)z-rɤʑi} (§\ref{sec:locative.participle.relatives}).
 
\subsubsection{Object participle} \label{sec:locative.relativization.object}
The negative object participle of the perception verbs \japhug{mto}{see} and \japhug{mtsʰɤm}{hear} has a special use: in (\ref{ex:WmAkAmto}) for instance, \forme{ɯ-mɤ-kɤ-mto} means `(somewhere) s/he cannot see him/her', the relativized element being locative rather than object  (see §\ref{sec:object.participle.other.relative} for further discussion). This type of relative clauses are always headless.

\begin{exe}
\ex \label{ex:WmAkAmto}
\gll [iɕqʰa qaɕpa kɯ ɯ-mɤ-kɤ-mto ʑo] jo-ɕe  \\
the.aforementioned frog \textsc{erg} \textsc{3sg}.\textsc{poss}-\textsc{neg}-\textsc{obj}:\textsc{pcp}-see \textsc{emph} \textsc{ifr}-go   \\
\glt `She went to [a place] where the frog would not find her.' (150818 muzhi guniang-zh)
\japhdoi{0006334\#S143}
\end{exe}

\subsubsection{Relator noun} \label{sec:Wstu.relativization.subject}
The two inalienably possessed nouns \japhug{ɯ-stu}{place}  and \japhug{ɯ-sta}{place}, both originating from lexicalized oblique participles (\tabref{tab:spa.sta.stu}, §\ref{sec:lexicalized.oblique.participle}) can be used as relator nouns of prenominal locative relative clauses.

The noun \japhug{ɯ-stu}{place} either selects subject participial clauses, as in (\ref{ex:kWrAZi.Wstu}) and (\ref{ex:tAntAm.lAkWZa.Wstu}), or oblique participial clauses (\ref{ex:nWsAnbaR.Wstu}). The relativized element can be a static location, but also a goal (see \ref{ex:NotCu.jAkAri.Wstu}, §\ref{sec:interrogative.relative}).
 
\begin{exe}
\ex \label{ex:kWrAZi.Wstu}
 \gll [ɯʑo kɯ-rɤʑi] ɯ-stu ʑo nɯ kú-wɣ-sɯ-ɤsɯɣ. \\
\textsc{3sg} \textsc{sbj}:\textsc{pcp}-stay \textsc{3sg}.\textsc{poss}-place \textsc{emph} \textsc{dem} \textsc{ipfv}-\textsc{inv}-\textsc{caus}-be.tight \\
\glt `One presses the place (in the cow's hide) where [the bug] is.' (25-akWzgumba)
\japhdoi{0003666\#S11}
\end{exe}

\begin{exe}
\ex \label{ex:tAntAm.lAkWZa.Wstu}
 \gll [nɯnɯ tɯ-ɤntɤm lɤ-kɯ-ʑa] ɯ-stu nɯ ɯ-mpʰɯsku tu-kɯ-ti ŋu. \\
\textsc{dem} \textsc{inf}:II-be.flat \textsc{aor}:\textsc{upstream}-\textsc{sbj}:\textsc{pcp}-start \textsc{3sg}.\textsc{poss}-place \textsc{dem} \textsc{3sg}.\textsc{poss}-rump \textsc{ipfv}-\textsc{genr}-say be:\textsc{fact} \\
\glt  `(When one goes upstream), the place where [the slope of the mountain] starts getting flatter (i.e. the point of inflection in the slope of the mountain) is called  the `rump' (of the mountain).' (150908 Wmphsku)
\japhdoi{0006302\#S8}
\end{exe}
 
\begin{exe}
\ex \label{ex:nWsAnbaR.Wstu}
 \gll ʑara [nɯ-sɤ-ɤnbaʁ] \textbf{ɯ-stu} nɯtɕu jo-nɯ-ɬoʁ-nɯ tɕe, \\
 \textsc{3pl} \textsc{3pl}.\textsc{poss}-\textsc{obl}:\textsc{pcp}-hide \textsc{3sg}.\textsc{poss}-place \textsc{dem}:\textsc{loc} \textsc{ifr}-\textsc{auto}-come.out-\textsc{pl} \textsc{lnk} \\
 \glt `They came out of their hiding place.' (140426 luozi he qiangdao-zh)
 \japhdoi{0003814\#S23}
\end{exe}

The noun \forme{ɯ-sta}, though also compatible with subject participial clauses, is more often found with finite relative clauses, as illustrated by  (\ref{ex:pWnANkWNke.Wsta}) and (\ref{ex:thWnAqharu.chAnAqharu}).  

\begin{exe}
\ex \label{ex:pWnANkWNke.Wsta}
 \gll  [pɯ-nɤŋkɯŋke] ɯ-sta nɯra rcanɯ, tɯ-ɕnaβ pɯ-kɤ-βde ʑo fse \\
 \textsc{pst}.\textsc{ipfv}-\textsc{distr}:walk \textsc{3sg}.\textsc{poss}-place \textsc{dem}:\textsc{loc} \textsc{unexp}:\textsc{deg} \textsc{indef}.\textsc{poss}-snot \textsc{aor}-\textsc{obj}:\textsc{pcp}-throw \textsc{emph} be.like:\textsc{fact} \\
 \glt `The places where [the slug] has passed look like snot has been spilled  [on them].' (26-qro)
 \japhdoi{0003682\#S128}
 \end{exe}

While \forme{ɯ-stu} and \forme{ɯ-sta} have very close meanings, in the case of verbs taking a goal such as \japhug{ru}{look at} (§\ref{sec:orienting.verbs}), a semantic difference can be observed: the former specifically indicates the goal (\ref{ex:lAru.Wstu}), while the latter is used to indicate the place where the action takes place (\ref{ex:lAru.Wsta}).

\begin{exe}
\ex 
\begin{xlist}
\ex \label{ex:lAru.Wstu}
\gll lɤ-ru ɯ-stu nɯtɕu \\
\textsc{aor}:\textsc{upstream}-look \textsc{3sg}.\textsc{poss}-place \textsc{dem}:\textsc{loc} \\
\glt `The direction (upwards) he looked.' (elicited)
\ex \label{ex:lAru.Wsta}
\gll lɤ-ru ɯ-sta nɯtɕu \\
\textsc{aor}:\textsc{upstream}-look \textsc{3sg}.\textsc{poss}-place \textsc{dem}:\textsc{loc} \\
\glt `The place where/from which he looked upwards.' (elicited)
\end{xlist}
\end{exe}
 
For instance, in (\ref{ex:thWnAqharu.chAnAqharu}), the clause \forme{tʰɯ-nɤqʰaru ɯ-sta} cannot be understood as `the place where he had looked back (in whose direction he looked)'.

\begin{exe}
\ex \label{ex:thWnAqharu.chAnAqharu}
 \gll lo-ɕe tɕe tɕelo [tʰɯ-nɤqʰaru] ɯ-sta lɤ-azɣɯt nɤ li cʰɤ-nɤqʰaru. \\
\textsc{ifr}:\textsc{upstream}-go \textsc{lnk} upstream \textsc{aor}:\textsc{downstream}-look.back \textsc{3sg}.\textsc{poss}-place \textsc{aor}-reach \textsc{add} again \textsc{ifr}:\textsc{downstream}-look.back  \\
\glt `He went up there, and when he arrived at the place up there that he had looked back from, he looked back again.' (2003 kAndzwsqhaj.2)
 \end{exe}
 

 
\subsection{Instrument} \label{sec:instrument.relativization}
\is{relative clause!instrument} \is{instrument!relativization}
Instruments are most commonly relativized using oblique participial relatives (§\ref{sec:instrumental.participle.relatives}). All instrument oblique relatives in the corpus are headless, and often limited to the participle itself. When the nominalized verb is transitive, the participial relative can contain an object  as in (\ref{ex:qandZi.chWsAGnda}).\footnote{Antipassivization is required to demote the direct object even for participial verb forms (§\ref{sec:antipassive.participle}). }

\begin{exe}
\ex \label{ex:qandZi.chWsAGnda}
\gll nɯnɯ [qandʑi cʰɯ-sɤ-ɣnda] nɯ tʰoŋtʰɤr ɲɯ-rmi \\
\textsc{dem} bullet \textsc{ipf}-\textsc{obl}:\textsc{pcp}-ram \textsc{dem} ramrod \textsc{sens}-be.called \\
 \glt `What is used to ram a bullet [into the muzzle of the gun] is called a ramrod.' (28-CAmWGdW)
\japhdoi{0003712\#S52}
\end{exe} 
 
 Instrumental participial relative clauses can take the generic inalienably possessed noun \japhug{ɯ-spa}{material} as overt head, to disambiguate with other types of relatives, in particular locative ones, built with an oblique participle (§\ref{sec:oblique.participle}). For instance, in (\ref{ex:tusANke.Wspa}), the focus is on the use of the path (a path specially made in order to be able to walk inside the field), rather than simply on the location (`the place where one walks').
\largerpage
\begin{exe}
\ex \label{ex:tusANke.Wspa}
\gll tɕe tɯ-ji ɯ-χcɤl tu-kɯ-ŋke mɤ-kʰɯ ma tɤ-rɤku tu tɕe tɕe, nɯ ɣɯ [tu-sɤ-ŋke] \textbf{ɯ-spa}, ɯ-tʂu <zhuanmen> ɯ-rkoz ɲɯ́-wɣ-βzu ŋgrɤl tɕe ɯnɯnɯ tʂu nɯ ftɕɤru tu-kɯ-ti ŋu \\
\textsc{lnk} \textsc{indef}.\textsc{poss}-field \textsc{3sg}.\textsc{poss}-middle \textsc{ipfv}-\textsc{genr}:S/O-walk \textsc{neg}-be.possible:\textsc{fact} \textsc{lnk} \textsc{indef}.\textsc{poss}-crops exist:\textsc{fact} \textsc{lnk} \textsc{lnk} \textsc{dem} \textsc{gen} \textsc{ipfv}-\textsc{obl}:\textsc{pcp}-walk \textsc{3sg}.\textsc{poss}-material \textsc{3sg}.\textsc{poss}-path specially \textsc{3sg}.\textsc{poss}-special \textsc{ipfv}-\textsc{inv}-make be.usually.the.case:\textsc{fact} \textsc{lnk} \textsc{dem} path \textsc{dem} summer.path \textsc{ipfv}-\textsc{genr}-say be:\textsc{fact} \\
\glt `One cannot walk in the middle of the fields, because there are crops. To walk into it, one specially makes a path, and that path is call `summer path'.' (definition, 15-06-05)
\end{exe} 

Alternatively, in the case of verbs with a sigmatic causative prefix in instrumental function (§\ref{sec:sig.caus.instrumental}), the instrument can be relativized as if it were a transitive subject using the \forme{kɯ-} participle (§\ref{sec:subject.participle.subject.relative}, §\ref{sec:tr.subject.relativization}), for example \forme{ɯ-kɯ-sɯ-mpʰɯl} `(the thing) that it reproduces with, (the thing) which makes it reproduce' in (\ref{ex:WkWsWmphWl}).

\begin{exe}
\ex \label{ex:WkWsWmphWl}
\gll tɕeri nɯnɯ [ɯ-kɯ-sɯ-mpʰɯl] nɯ li ɯ-zrɤm ɲɯ-ɕti ma ɯ-rɣi ɲɯ-maʁ.  \\
but \textsc{dem} \textsc{3sg}-\textsc{sbj}:\textsc{pcp}-\textsc{caus}-reproduce \textsc{dem} again \textsc{3sg}.\textsc{poss}-root \textsc{seve}-be.\textsc{aff} \textsc{lnk} \textsc{3sg}.\textsc{poss}-seed \textsc{seve}-not.be \\
\glt `What it reproduces with is its root, not its seeds.' (11-paRzwamWntoR)
\japhdoi{0003476\#S100}
\end{exe}
 
\subsection{Comitative} \label{sec:comitative.relativization}
\is{relative clause!comitative} \is{comitative!relativization}
Comitative arguments marked with the postposition \forme{cʰo} (§\ref{sec:comitative}) can be relativized with the oblique participle (§\ref{sec:other.oblique.participle.relatives}). Such relatives are generally headless, as in (\ref{ex:WsAzrAkrAz}).\footnote{The verb \japhug{rɤkrɤz}{have a discussion} can select a comitative argument (see example \ref{ex:cho.torAkrAzndZi}, §\ref{sec:comitative}). }

\begin{exe}
\ex \label{ex:WsAzrAkrAz}
\gll ɯʑo ɯ-χti ɲɤ-me tɕe, ɲɯ-sɤzdɯxpa tɕe ɯ-sɤz-rɤkrɤz ri maŋe, ɯ-kɯ-qur ri maŋe \\
 \textsc{3sg} \textsc{3sg}.\textsc{poss}-companion \textsc{ifr}-not.exist \textsc{lnk} \textsc{sens}-be.pitiful \textsc{lnk} \textsc{3sg}.\textsc{poss}-\textsc{obl}:\textsc{pcp}-discuss also not.exist:\textsc{sens} \textsc{3sg}.\textsc{poss}-\textsc{sbj}:\textsc{pcp}-help also not.exist:\textsc{sens} \\
\glt `Her husband passed away, poor her, she has nobody to talk with, and nobody to help her.' (12-BzaNsa)
\japhdoi{0003484\#S111}
\end{exe}


If overt, the head does not take the comitative postposition \forme{cʰo} as in (\ref{ex:asAmWmi}), showing that this type of relative cannot be head-internal and is rather postnominal (§\ref{sec:postnominal.relative}).

\begin{exe}
\ex \label{ex:asAmWmi}
\gll \textbf{tɯrme} [a-sɤ-ɤmɯmi] nɯ lɤβzaŋ ɲɯ-rmi. \\
person \textsc{1sg}.\textsc{poss}-\textsc{obl}:\textsc{pcp}-be.on.good.terms \textsc{dem}  \textsc{anthr} \textsc{sens}-be.called \\
\glt `The person I am on good terms with is Lobzang.' (elicited)
\end{exe}


\subsection{Dative} \label{sec:dative.relativization}
\is{relative clause!dative} \is{dative!relativization}
Dative arguments (marked with the relator nouns \forme{ɯ-ɕki} or \forme{ɯ-pʰe}, §\ref{sec:dative}) can only be relativized with oblique participial clauses (§\ref{sec:other.oblique.participle.relatives}). For example, the recipient of the indirective verb \japhug{ti}{say} (§\ref{sec:ditransitive.indirective}) is relativized with the participle \forme{sɤ-ti} `(person) to whom one talks to', as in (\ref{ex:aRi.sAti}).\footnote{The relative clause \forme{tɕi-mu tɕi-wa kɯ-naχtɕɯɣ} is analyzed in §\ref{sec:S.possessor.relativization}.}

\begin{exe}
\ex \label{ex:aRi.sAti}
\gll tɕeri [``a-ʁi" sɤ-ti] dɤn ma aʑo ɣɯ, nɤkinɯ, a-ʁi, [tɕi-mu tɕi-wa kɯ-naχtɕɯɣ] a-ʁi, 
nɯnɯra nɯ-ɕki tɕe ``a-ʁi" tu-ti-a, a-wɤmɯ ɯ-rɟit tɕe ``a-ʁi" tu-ti-a. \\
\textsc{lnk} \textsc{1sg}.\textsc{poss}-younger.sibling \textsc{obl}:\textsc{pcp}-say be.many:\textsc{fact} \textsc{lnk} \textsc{1sg} \textsc{gen} \textsc{filler} \textsc{1sg}.\textsc{poss}-younger.sibling  \textsc{1du}.\textsc{poss}-mother  \textsc{1du}.\textsc{poss}-father \textsc{sbj}:\textsc{pcp}-be.the.same  \textsc{1sg}.\textsc{poss}-younger.sibling \textsc{dem}:\textsc{pl} \textsc{3pl}.\textsc{poss}-\textsc{dat} \textsc{loc}  \textsc{1sg}.\textsc{poss}-younger.sibling \textsc{ipfv}-say-\textsc{1sg} \textsc{1sg}.\textsc{poss}-brother \textsc{3sg}.\textsc{poss}-children \textsc{lnk} \textsc{1sg}.\textsc{poss}-younger.sibling \textsc{ipfv}-say-\textsc{1sg}  \\
\glt `There are many [people] to whom one says `my younger sibling'; those of my ``younger siblings'' whose parents I share, I say `my younger sibling' to them, and I (also) say  `my younger sibling' to my brothers' children.' (140425 kWmdza02) \japhdoi{0003786}
\end{exe}

The subject can be optionally indexed as a possessive prefix on the oblique participle, as in (\ref{ex:azrAthu}).

\begin{exe}
\ex \label{ex:azrAthu}
\gll [a-z-rɤ-tʰu] nɯ tsʰɯndzɯn <laoshi> ŋu \\
\textsc{1sg}.\textsc{poss}-\textsc{obl}:\textsc{pcp} \textsc{dem}  \textsc{anthr} teacher be:\textsc{fact} \\
\glt `The [person] I ask is teacher \iai{Tshendzin}. (elicited)
\end{exe}

There are no examples of head-internal dative relative clauses, with the relativized element taking dative marking.
 
\subsection{Time adjuncts} \label{sec:time.relativization}
\is{relative clause!adjuncts} \is{adjunct!relativization}
Oblique participial clauses can have a temporal interpretation, both with transitive verbs (\ref{ex:WsAphWt.WsNi}) and intransitive stative verbs (\ref{ex:tAjmAG.WsAdAn}). This way of relativizing temporal adjuncts is however quite limited (§\ref{sec:other.oblique.participle.relatives}).

\begin{exe}
\ex \label{ex:WsAphWt.WsNi}
\gll [ɯ-sɤ-pʰɯt] nɯ, ɯ-sŋi nɯ a-mɤ-pɯ-pe tɕe tɕe li tu-kɯ-ɕɯ-ngo ɲɯ-ŋgrɤl \\
\textsc{3sg}.\textsc{poss}-\textsc{obl}:\textsc{pcp}-cut \textsc{dem} \textsc{3sg}.\textsc{poss}-day \textsc{dem} \textsc{irr}-\textsc{neg}-\textsc{ipfv}-be.good \textsc{lnk} \textsc{lnk} again \textsc{ipfv}-\textsc{genr}:S/O-\textsc{caus}-be.sick \textsc{sens}-be.usually.the.case \\
\glt `If the day when [the tree] is cut is not auspicious, it makes people sick.' (24-kWqar)
\japhdoi{0003620\#S9}
\end{exe}

\begin{exe}
\ex \label{ex:tAjmAG.WsAdAn}
\gll [tɤjmɤɣ ɯ-sɤ-dɤn] ʑo ɲɯ-ŋu, tʰamtʰam. \\
mushroom \textsc{3sg}.\textsc{poss}-\textsc{obl}:\textsc{pcp}-be.many \textsc{emph} \textsc{sens}-be now \\
\glt `Now is a [period] when mushrooms are abundant.' (conversation, 16-08-11)
\end{exe}

The more common way of relativizing temporal adjuncts is by using finite clauses followed by a temporal relator noun in \textsc{3sg} possessive form such as  \forme{ɯ-sŋi} `the day when...' (\ref{ex:RmbGWzWn.tu.WsNi}), \forme{ɯ-xpa} `the year when...' (\ref{ex:maozhuxi.nWme}), or \forme{ɯ-raŋ} `the time when...' etc.  
 
\begin{exe}
\ex \label{ex:RmbGWzWn.tu.WsNi}
\gll [[slɤzɯn tu] cʰo] [ʁmbɣɯzɯn tu] ɯ-sŋi nɯnɯ, skɤrma mɤ-sna ra tu-ti-nɯ ŋu \\
lunar.eclipse exist:\textsc{fact} \textsc{comit} solar.eclipse exist:\textsc{fact} \textsc{3sg}.\textsc{poss}-day \textsc{dem} time \textsc{neg}-be.good:\textsc{fact} \textsc{pl} \textsc{ipfv}-say-\textsc{pl} be:\textsc{fact} \\
\glt `They said that the day of a lunar or a solar eclipse is not auspicious.' (29-mWBZi)
\japhdoi{0003728\#S155}
\end{exe}

When these prenominal relatives are used as temporal adjuncts in the main clause, with or without (\ref{ex:maozhuxi.nWme}) locative marking, they serve to express temporal clause linking (§\ref{sec:temporal.reference}).

\begin{exe}
\ex \label{ex:maozhuxi.nWme}
\gll [iʑora <maozhuxi> nɯ-me] ɯ-xpa nɯnɯ, skɤrtɕin maŋe tu-ti-nɯ pɯ-ŋgrɤl. \\
\textsc{1pl} chairman.Mao \textsc{aor}-not.exist \textsc{3sg}.\textsc{poss}-year \textsc{dem} Venus not.exist \textsc{ipfv}-say-\textsc{pl} \textsc{pst}.\textsc{ipfv}-be.usually.the.case \\
\glt `They say that the year when our Chairman Mao passed away, Venus did not appear.' (29-LAntshAm)
\japhdoi{0003726\#S75}
\end{exe}
 

%tɯ-mɤrʑaβ /tɤ/ tɤ-mda ɣɯ nɤ-ŋga tɤ-sɯɣndze ra
%150818_muzhi_guniang
\subsection{Possessor} \label{sec:possessor.relativization}
\is{relative clause!possessor} \is{possessor!relativization}
Possessors of intransitive subjects and direct objects can be relativized using the same relative constructions as their possessees. It is unclear whether possessors of other arguments can be relativized (in particular possessors of transitive subjects).

\subsubsection{Possessor of intransitive subject}  \label{sec:S.possessor.relativization}
\is{possessor} \is{subject!intransitive} \is{relative clause!possessor}
Possessors of intransitive subjects are relativized with subject participial clauses (§\ref{sec:subject.participle.other.relative}). In this construction, the subject of the clause is overt and takes an obligatory possessive prefix, such as the \textsc{3pl} \forme{nɯ-} in (\ref{ex:nWmtChi.kWdAn}).\footnote{This expression may be a nativized calque from Chinese \ch{多嘴}{duōzuǐ}{big mouth}. }

\begin{exe}
\ex \label{ex:nWmtChi.kWdAn}
\gll [\textbf{nɯ}-mtɕʰi kɯ-dɤn] nɯra \\
\textsc{3pl}.\textsc{poss}-mouth \textsc{sbj}:\textsc{pcp}-be.many \textsc{dem}:\textsc{pl} \\
\glt `Those who talk too much' (24-qro)
\japhdoi{0003626\#S112}
\end{exe}

The head noun can be overt  as in (\ref{ex:qapri.WrW.kWtu}), with determiner repetition (§\ref{sec:head-internal.relative.determiners}).


\begin{exe}
\ex \label{ex:qapri.WrW.kWtu}
\gll  akɯ zɯ [\textbf{qapri} \textbf{ci} \textbf{ɯ}-kɤχcɤl \textbf{ɯ}-ʁrɯ kɯ-tu] ci ɣɤʑu tɕe \\
east \textsc{loc} snake \textsc{indef} \textsc{3sg}.\textsc{poss}-top.of.head \textsc{3sg}.\textsc{poss}-horn \textsc{sbj}:\textsc{pcp}-exist \textsc{indef} exist:\textsc{sens} \textsc{lnk} \\
\glt `In the east, there is a snake with a horn on his head.' (2005, divinitation)
\end{exe}

 
Prenominal possessor relatives are also attested, but extremely rare, and the third person possessive prefix on the possessee is required as in (\ref{ex:Wcu.kWtu.rJAGi}).

\begin{exe}
\ex \label{ex:Wcu.kWtu.rJAGi}
\gll  [\textbf{ɯ}-cu kɯ-tu] \textbf{rɟɤɣi} pjɤ-ɕti \\
\textsc{3sg}.\textsc{poss}-additive \textsc{sbj}:\textsc{pcp}-exist tsampa \textsc{ifr}.\textsc{ipfv}-be.\textsc{aff} \\
\glt `It was tsampa mixed with broad beans.' (i.e. poor quality tsampa) (2003-kWBRa)
 \end{exe}

Subject possessor relative can also occur in apposition with another relative clause, as in (\ref{ex:Wmi.kWzWzri}).

\begin{exe}
 \ex \label{ex:Wmi.kWzWzri}
 \gll [ɯ-mi kɯ-zɯ\redp{}zri ʑo] [rkaŋraŋ kɯ-rmi] ci pɯ-tu ɲɯ-ŋu \\
 \textsc{3sg}.\textsc{poss}-leg \textsc{sbj}:\textsc{pcp}-\textsc{emph}\redp{}be.long \textsc{emph}  \textsc{anthr} \textsc{sbj}:\textsc{pcp}-be.called \textsc{indef} \textsc{pst}.\textsc{ipfv}-exist \textsc{sens}-be \\
 \glt `There was someone called Rkangring who had long legs.' (2005 Norbzang)
 \end{exe}

First or second person possessors can also be relativized, as in (\ref{ex:aXti.kWtu}) and (\ref{ex:nAmu.nAwa.kWtshoz}) (see also \ref{ex:amu.kWme}, §\ref{sec:pers.pronouns}).

\begin{exe}
\ex \label{ex:aXti.kWtu}
\gll aʑo [a-χti kɯ-tu] ɲɯ-ŋu-a \\
\textsc{1sg} \textsc{1sg}.\textsc{poss}-companion \textsc{sbj}:\textsc{pcp}-exist \textsc{sens}-be-\textsc{1sg} \\
\glt `I am someone with a husband.' (i.e. whose husband is still alive, unlike hers) (12-BzaNsa)
\japhdoi{0003484\#S110}
\end{exe}

This type of relative can trigger either third person singular indexation, or index the relativized element: both options (\textsc{3sg} \forme{ŋu} `s/he/it is' and \textsc{2sg} \forme{tɯ-ŋu} `you are') have been tested and are possible, as shown in example (\ref{ex:nAmu.nAwa.kWtshoz}).

\begin{exe}
\ex \label{ex:nAmu.nAwa.kWtshoz}
\gll [\textbf{nɤʑo} \textbf{nɤ}-mu \textbf{nɤ}-wa kɯ-tsʰoz] ŋu/tɯ-ŋu  \\  
\textsc{2sg} \textsc{2sg}.\textsc{poss}-mother  \textsc{2sg}.\textsc{poss}-father \textsc{sbj}:\textsc{pcp}-be.complete be:\textsc{fact}/2-be:\textsc{fact}  \\
\glt `You are someone both of whose parents are still alive.' (elicited)
\end{exe}

In example (\ref{ex:tCimu.tCiwa.kWnaXtCWG}),\footnote{The context of the relative clause (\ref{ex:tCimu.tCiwa.kWnaXtCWG}) is found in example (\ref{ex:aRi.sAti}) (§\ref{sec:dative.relativization}).}, the relativized element corresponds to the \textit{third person component} of the first dual (exclusive) possessive prefix \forme{tɕi-} on the noun dyad \forme{tɕi-mu tɕi-wa} `our parents' (§\ref{sec:dyads}).
  

\begin{exe}
\ex \label{ex:tCimu.tCiwa.kWnaXtCWG}
\gll [\textbf{tɕi}-mu \textbf{tɕi}-wa kɯ-naχtɕɯɣ] \\
\textsc{1du}.\textsc{poss}-mother  \textsc{1du}.\textsc{poss}-father \textsc{sbj}:\textsc{pcp}-be.the.same \\
\glt `Those whose parents are the same as mine.' (140425 kWmdza02) \japhdoi{0003786}
\end{exe}

Although the verb \japhug{naχtɕɯɣ}{be the same} can select a comitative argument (§\ref{sec:comitative}), (\ref{ex:tCimu.tCiwa.kWnaXtCWG}) is not an example of comitative relativization (§\ref{sec:comitative.relativization}). Rather, the construction from which (\ref{ex:tCimu.tCiwa.kWnaXtCWG}) has been relativized is the one in (\ref{ex:tCimYitChAz.YWnaXtCWG}), where the equality concerns the possessors of the intransitive subject.

\begin{exe}
\ex \label{ex:tCimYitChAz.YWnaXtCWG}
\gll tɕi-mɲitɕʰɤz ra ɲɯ-naχtɕɯɣ \\
\textsc{1du}.\textsc{poss}-temper \textsc{pl} \textsc{sens}-be.the.same \\
\glt `We have the same temperament = She has the same temperament as I.' (12-BzaNsa)
\japhdoi{0003484\#S53}
\end{exe}

 In the participial relative taking \forme{mɤ-kɯ-sɤ-mto} `the one that is not visible' as its main verb in (\ref{ex:mAkWsAmto.schiz}), the head \japhug{smar}{river} is not the possessor of the subject in the proper sense, but the possessor of a noun (\japhug{ɯ-βzɯr}{its side}) subject of a clause embedded within another clause (headed by the participle \forme{kɯ-fse} `that is like...') serving as the subject of \forme{mɤ-kɯ-sɤ-mto} `the one that is not visible'.  

  \begin{exe}
\ex \label{ex:mAkWsAmto.schiz}
 \gll  [maka tɕekɯ ku-kɯ-ru tɕe tɕendi \textbf{smar} [[ɯ-βzɯr tɕʰi kɯ-fse ŋu] kɯ-fse] mɤ-kɯ-sɤ-mto ʑo] scʰiz nɯ-azɣɯt ɲɯ-ŋu. \\
 at.all east \textsc{ipfv}:\textsc{east}-\textsc{genr}:S/O-look \textsc{lnk} west river \textsc{3sg}.\textsc{poss}-side what \textsc{sbj}:\textsc{pcp}-be.like be:\textsc{fact} \textsc{sbj}:\textsc{pcp}-be.like \textsc{neg}-\textsc{sbj}:\textsc{pcp}-\textsc{prop}-see \textsc{emph} \textsc{approx}.\textsc{loc} \textsc{aor}:\textsc{west}-reach \textsc{sens}-be \\
\glt `He arrived at a river where if one looked from one bank to the other, what was on the other side was not at all visible.' (Divination 2005)
 \end{exe}
 
 This particularly convoluted example is however not representative of what is usually found in the corpus.

In (\ref{ex:WkWXsu.kWme}), we find a prenominal relative containing another verb in subject participle form that can be interpreted a possessor relativization, as in (\ref{ex:kWkWtu.head.internal}): \forme{ɯ-kɯ-χsu kɯ-me}  `having no feeder'.

\begin{exe}
\ex \label{ex:WkWXsu.kWme}
 \gll  [ɯ-pɕi kɯ-rɤʑi] [ɯ-kɯ-χsu kɯ-me] lɯlu ɣɤʑu tɕe nɯnɯ kupa kɯ <yemao> tu-ti ŋu \\
 \textsc{3sg}.\textsc{poss}-outside \textsc{sbj}:\textsc{pcp}-stay \textsc{3sg}.\textsc{poss}-\textsc{sbj}:\textsc{pcp}-feed \textsc{nmzl}:S/A-not.exist cat exist:\textsc{sens} \textsc{lnk} \textsc{dem} Chinese \textsc{erg} wild.cat \textsc{ipfv}-say be:\textsc{fact} \\
 \glt `There are cats that live outside, that nobody feeds, Chinese people call them wild cats.' (21-lWlu)
\japhdoi{0003576\#S2}
 \end{exe}

However, there are cases of relative clauses with the subject participle of the negative existential verb \forme{kɯ-me} and an \textit{intransitive} verb in participial (or finite) form in the preceding complement clause, for instance \forme{tɤ-kɯ-mbri} in (\ref{ex:tAkWmbri.kWme}). It is manifest that here the relativized element is neither the subject of \japhug{me}{not exist} nor a possessor, but rather the subject of the verb of the complement clause \japhug{mbri}{cry, sing} (see §\ref{sec:relativization.complement.type}).

\begin{exe}
\ex \label{ex:tAkWmbri.kWme}
 \gll  pɣɤtɕɯ nɯ kɯnɤ [[tɯ-ɣjɤn cinɤ ʑo tɤ-kɯ-mbri] kɯ-me], nɯ to-ɣɤscɤscɤt ʑo to-mbri ɲɯ-ŋu, \\
bird \textsc{dem} also one-time even.one \textsc{emph} \textsc{aor}-\textsc{sbj}:\textsc{pcp}-make.noise \textsc{sbj}:\textsc{pcp}-not.exist \textsc{dem} \textsc{ifr}-do.quickly \textsc{emph} \textsc{ifr}-make.noise \textsc{sens}-be \\
\glt `Even the bird, which had not even sung once [since it had come to the palace], immediately started singing.' (2012 qachGa)
\japhdoi{0004087\#S165}
 \end{exe}

The clause \forme{tɯ-ɣjɤn cinɤ ʑo tɤ-kɯ-mbri kɯ-me} here is in fact the nominalized version of the postverbal negative construction  (§\ref{sec:negation.existential}). In main clauses, this construction combines a negative existential verb in impersonal (third singular) form with a complement clause in finite form. In (\ref{ex:tAkWmbri.kWme}), we see that when the intransitive subject of a postverbal negative construction is nominalized, both the matrix verb \japhug{me}{not exist} and the verb of the complement clause \forme{tɤ-kɯ-mbri} occur in subject participle form (§\ref{sec:relativization.complement.type}). This construction, though superficially similar to that in (\ref{ex:WkWXsu.kWme}), is therefore different from it.


\subsubsection{Possessor of object}  \label{sec:O.possessor.relativization}
Possessors of objects can be relativized with object participial relatives (\ref{ex:pGAtCW.WsroR.kAkAri}) or finite relatives (\ref{ex:Wkho.mWthasWphWt}) exactly like direct objects (§\ref{sec:object.relativization}), but like subject possessors, they require the possessee to be overt and to bear a possessive prefix (\forme{ɯ-sroʁ} `its life' in \ref{ex:pGAtCW.WsroR.kAkAri} and \forme{ɯ-kʰo} `its house' in \ref{ex:Wkho.mWthasWphWt}).

\begin{exe}
\ex \label{ex:pGAtCW.WsroR.kAkAri}
\gll iɕqʰa [\textbf{pɣɤtɕɯ} \textbf{ɯ}-sroʁ kɤ-kɤ-ri] nɯ to-tɕɤt tɕe \\
the.aforementioned bird  \textsc{3sg}.\textsc{poss}-life \textsc{aor}-\textsc{obj}:\textsc{pcp}-save \textsc{dem} \textsc{ifr}-take.out \textsc{lnk} \\
\glt `He took out (from his bag) the bird whose life he had saved.' (140428 yonggan de xiaocaifeng-zh)
\japhdoi{0003886\#S67}
\end{exe}

\begin{exe}
\ex \label{ex:Wkho.mWthasWphWt}
\gll [\textbf{qro} ta-fsraŋ] nɯnɯ, [\textbf{ɯ}-kʰo mɯ-tʰa-sɯ-pʰɯt] nɯnɯra ɣɯ nɯ-rɟɤlpu (nɯ) kɯ, qro rcanɯ, stoŋtsu kɯmŋu ʑo jo-ɣɯt. \\
ant \textsc{aor}:3\flobv{}-protect \textsc{dem} \textsc{3sg}.\textsc{poss}-hive \textsc{neg}-\textsc{aor}:3\flobv{}-\textsc{caus}-take.off \textsc{dem}:\textsc{pl} \textsc{gen} \textsc{3pl}.\textsc{poss}-king \textsc{dem} \textsc{erg} ant \textsc{unexp}:\textsc{deg} thousand five \textsc{emph} \textsc{ifr}-bring \\
\glt `The queen (king) of the ants that he had saved, whose anthill he had saved from destruction (by his brothers), brought five thousand ants.' (140510 fengwang-zh)
\japhdoi{0003939\#S98}
\end{exe}

  
%kɯmaʁ ci li ci, ʑmbɤr ɲɯ-kɯ-ɬoʁ ci fsapaʁ ɯ-ŋgo ɣɤʑu tɕe, 27-kharwut, 35
%  zrɯɣ nɯ kɯ, ɯ-se, tɤ-se tʰɯ-kɤ-mqlaʁ nɯ, nɯ ɯ-stu nɯ ɲɯ-ɣɯrni. 
 
%tɤ-tɕɯpɯ nɯnɯ taɴɢoʁ ɯ-ŋgɯ kɤ-rku ɕ-thɯ-kɤ-βde nɯnɯ 
%cʰɤ-tsɯm nɤ cʰɤ-tsɯm tɕe, 
%140514_huishuohua_de_niao
 
\subsection{Relativization out of complement clause} \label{sec:out.complement.relativization}
possessor
Relativization of arguments and adjuncts out of complement clauses is possible in certain conditions.\footnote{For an example of arguments  inside complements which \textit{cannot} be relativized, see §\ref{sec:AM.mvc.relativizability}.  }

\subsubsection{Intransitive matrix verbs}  \label{sec:out.complement.relativization.intr}
In the case of modal verbs taking subject complement clauses such as \japhug{ra}{be needed}, `be necessary' and \japhug{kʰɯ}{be possible},  subject participle clauses are attested to relativize not the intransitive subject of these verbs (§\ref{sec:intr.subject.relativization}), but rather arguments of the subject complement clause, including direct objects (\ref{ex:tWthe.kWra}), \ref{ex:kAnWtsW.kWra}) or transitive subjects (\ref{ex:tWnAjonW.kWra}).

\begin{exe}
\ex \label{ex:tWthe.kWra}
\gll tɕe jisŋi [[tɯ-tʰe] kɯ-ra] ɯ́-tu? \\
\textsc{lnk} today 2-ask[III]:\textsc{fact} \textsc{sbj}:\textsc{pcp}-be.needed \textsc{qu}-exist:\textsc{fact} \\
\glt `Is there anything you need to ask today?' (conversation 17-08-21)
\end{exe}

\begin{exe}
\ex \label{ex:kAnWtsW.kWra}
\gll [kɤ-nɯtsɯ kɯ-ra] ra kɯnɤ tu-kɯ-nɯ-ti] nɯnɯra tɕaɣi tu-sɤrmi-nɯ ŋgrɤl. \\
[[[\textsc{inf}-hide] \textsc{sbj}:\textsc{pcp}-be.needed] \textsc{pl} also \textsc{ipfv}-\textsc{sbj}:\textsc{pcp}-\textsc{auto}-say \textsc{dem}:\textsc{pl} parrot \textsc{ipfv}-call-\textsc{pl} be.usually.the.case:\textsc{fact} \\
\glt `People call `parrots' those who tell [everything], including things that should [remain] hidden.' (24-qro)
\japhdoi{0003626\#S121}
\end{exe}

\begin{exe}
\ex \label{ex:tWnAjonW.kWra}
\gll [tɤ-rɤku tu-kɯ-nɯ-ndza], [[tɯrme ntsɯ tɯ-nɤjo-nɯ] mɤ-kɯ-ra], [koŋla tɯrme kɯ-pɯ\redp{}pe ʑo] a-nɯ-tɯ-ɤβzu-nɯ smɯlɤm \\
\textsc{indef}.\textsc{poss}-crop \textsc{ipfv}-\textsc{sbj}:\textsc{pcp}-\textsc{auto}-eat people always 2-wait:\textsc{fact}-\textsc{pl} \textsc{neg}-\textsc{sbj}:\textsc{pcp}-be.needed really people \textsc{sbj}:\textsc{pcp}-emph\redp{}be.good \textsc{emph} \textsc{irr}-\textsc{pfv}-2-become-\textsc{pl} prayer \\
\glt `May you become nice people who eat crops and do not need to wait (i.e. ambush) for humans (to eat them).' (2005 Norbzang)
\end{exe}

In addition to core arguments, subject participial clauses are also used to relativize goals (\ref{ex:Cea.kWra}) and locative adjuncts (\ref{ex:YWnWpea.kWra}).
 
\begin{exe}
\ex \label{ex:Cea.kWra}
\gll [[ɕe-a] kɯ-ra] nɯ alo ŋu \\
go:\textsc{fact}-\textsc{1sg} \textsc{sbj}:\textsc{pcp}-be.needed \textsc{dem} upstream be:\textsc{fact} \\
\glt  `[The place] where I have to go is upstream.' (elicited)
\end{exe}

\begin{exe}
\ex \label{ex:YWnWpea.kWra}
\gll [[aʑo-sɯso ɲɯ-nɯ-pe-a] kɯ-kʰɯ] nɯtɕu nɯ-ɕe-a ŋu \\
\textsc{1sg}-as.wish \textsc{ipfv}-\textsc{auto}-do[III]-\textsc{1sg} \textsc{sbj}:\textsc{pcp}-be.possible \textsc{dem}:\textsc{loc} \textsc{vert}-go:\textsc{fact}-\textsc{1sg} be:\textsc{fact} \\
\glt  `I am going back to a place where I can do as I like.' (140426 jiagou he lang-zh)
\japhdoi{0003804\#S69}
\end{exe}

\subsubsection{Semi-transitive matrix verbs} \label{sec:out.complement.relativization.cha}
The subject participle of the semi-transitive \japhug{cʰa}{can} can be used to relativize the subject of the complement clause, when it is at the same time subject of \forme{cʰa} itself, as in (\ref{ex:tumbro.mAkWcha}) or in (\ref{WkukWndWn.kWcha}) below (§\ref{sec:relativization.complement.type}).
 
\begin{exe}
\ex \label{ex:tumbro.mAkWcha}
\gll [\textbf{si} [wuma tu-mbro] mɤ-kɯ-cʰa] ci ŋu tɕe, \\
tree really \textsc{ipfv}-be.high \textsc{neg}-\textsc{sbj}:\textsc{pcp}-can \textsc{indef} be:\textsc{fact} \textsc{lnk} \\
\glt `It is a tree that cannot grow very high.' (12-Zmbroko)
\japhdoi{0003490\#S83}
\end{exe}


Objects in complement clauses are relativized with the object participle \forme{kɤ-cʰa} as in (\ref{ex:chWnWtsWm.WmAkAcha}), which can in addition take a possessive prefix coreferent with the transitive subject of the complement clause (§\ref{sec:object.participle.possessive}). In (\ref{ex:chWnWtsWm.WmAkAcha}) for instance, the prefix \forme{ɯ-} on \forme{ɯ-mɤ-kɤ-cʰa} is coreferent with \japhug{qaliaʁ}{eagle}.

\begin{exe}
\ex \label{ex:chWnWtsWm.WmAkAcha}
\gll [[qaliaʁ kɯ, nɤkinɯ, nɯɕimɯma kɯ-sɯsu cʰɯ-nɯ-tsɯm] ɯ-mɤ-kɤ-cʰa] nɯnɯra \\
eagle \textsc{erg} \textsc{filler} immediately \textsc{sbj}:\textsc{pcp}-be.alive \textsc{ipfv}:\textsc{downstream}-\textsc{vert}-take.away \textsc{3sg}.\textsc{poss}-\textsc{neg}-\textsc{obj}:\textsc{pcp}-can \textsc{dem}:\textsc{pl} \\
\glt `Those (the animals) that the eagle is not able to carry away while they are still alive.' (150819 RarphAB)
\japhdoi{0006356\#S1}
\end{exe}

\subsubsection{Transitive matrix verbs} \label{sec:out.complement.relativization.tr}
With transitive com\-ple\-ment-taking verbs such as \japhug{rɲo}{experience} (§\ref{sec:rYo.complements}) or \japhug{spa}{be able} (§\ref{sec:spa.verb}), the subject participle occurs if the relativized element is the (transitive or intransitive) subject of the complement clause, as in (\ref{ex:kArWCmi.kWspa}) or (\ref{ex:kACe.pWkWrYo}).

\begin{exe}
\ex \label{ex:kArWCmi.kWspa}
\gll pɣa [[kɤ-rɯɕmi] ɯ-kɯ-spa] ci \\
\textbf{bird} \textsc{inf}-speak \textsc{3sg}.\textsc{poss}-\textsc{sbj}:\textsc{pcp}-be.able \textsc{indef} \\
\glt `A bird that is able to speak' (2005 Norbzang)
\end{exe}

\begin{exe}
\ex \label{ex:kACe.pWkWrYo}
\gll [kɤ-ɕe] pɯ-kɯ-rɲo pjɤ-dɤn-nɯ ri \\
\textsc{inf}-go \textsc{aor}-\textsc{sbj}:\textsc{pcp}-experience \textsc{ifr}.\textsc{ipfv}-be.many-\textsc{pl} \textsc{lnk} \\
\glt `Many people have gone there (those who have gone there were many).' (140514 huishuohua de niao-zh)
\japhdoi{0003992\#S82}
\end{exe}

To relativize the direct object of a transitive verb in the complement clause, the com\-ple\-ment-taking verb can be in finite form (\ref{ex:kAmto.pWrYota}) or in object participle form (\ref{ex:kAmtsWG.pWkArYo}), as if the complement-internal objects were the direct objects of the main verb of the relative clause (§\ref{sec:object.relativization}). 

\begin{exe}
\ex \label{ex:kAmto.pWrYota}
\gll [aʑo [kɤ-mto] pɯ-rɲo-t-a] \textbf{tɕʰemɤpɯ} ci tu tɕe, \\
\textsc{1sg} \textsc{inf}-see \textsc{aor}-experience-\textsc{pst}:\textsc{tr}-\textsc{1sg} girl \textsc{indef} exist:\textsc{fact} \textsc{lnk} \\
\glt `There is a girl whom I saw before and....' (150819 haidenver-zh)
\japhdoi{0006314\#S369}
\end{exe}

\begin{exe}
\ex \label{ex:kAmtsWG.pWkArYo}
\gll [[ʁmɤrɲɯɣ kɯ kɤ-mtsɯɣ] pɯ-kɤ-rɲo] \textbf{tɯrme} nɯ \\
mosquito \textsc{erg} \textsc{inf}-bite \textsc{aor}-\textsc{obj}:\textsc{pcp}-experience person \textsc{dem} \\
\glt `Someone who has been stung by mosquitoes before' (elicited)
\end{exe}

The contrast between \forme{pɯ-kɯ-rɲo} in (\ref{ex:kACe.pWkWrYo}) and  \forme{pɯ-kɤ-rɲo} (\ref{ex:kAmtsWG.pWkArYo}) is not as straightforward as it might seem at first glance. The transitive verb \forme{rɲo} lacks non-generic inverse forms, and treats both subjects and objects of its velar infinitive clauses in the same way as transitive subject in the matrix clause, both in terms of indexation and flagging (§\ref{sec:rYo.complements}). 

%ɯ-fsosoz tɕe ɯ-wa ɯ-phe /ɯkɯ/ ɯ-kɯ-ndza ju-kɯ-ɣi nɯ pɣa ci ɲɯ-ŋu, kɯki ɯ-rme ŋu to-ti ri,
 
\subsubsection{Complement type} \label{sec:relativization.complement.type}
The complements whose arguments or adjuncts are relativized can be either finite (\ref{ex:tWthe.kWra}, \ref{ex:tWnAjonW.kWra}, \ref{ex:Cea.kWra}, \ref{ex:YWnWpea.kWra}, \ref{ex:tumbro.mAkWcha}) or infinitival clauses (\ref{ex:kAnWtsW.kWra}, \ref{ex:kArWCmi.kWspa}, \ref{ex:kAmto.pWrYota}), just like when the com\-ple\-ment-taking verbs are in finite form. In addition, the verb of the complement clause can also bear subject participle form if the relativized element is the subject of the complement clause, as in (\ref{smAnkhaN.kWCe.kWra}) and (\ref{WkukWndWn.kWcha}) (see also §\ref{sec:subject.participle.other.relative}, §\ref{sec:subject.participle.complementation}).

\begin{exe}
\ex \label{smAnkhaN.kWCe.kWra}
\gll  [[smɤnkʰaŋ kɯ-ɕe] kɯ-ra] ɣɤʑu. \\
hospital \textsc{sbj}:\textsc{pcp}-go \textsc{sbj}:\textsc{pcp}-be.needed exist:sens \\
\glt `There are people [with nosebleed] who have to go to the hospital. \\
\end{exe}
 
\begin{exe}
\ex \label{WkukWndWn.kWcha}
\gll  [[kɯki ɯ-ku-kɯ-ndɯn] kɯ-cʰa] ci ŋu \\
\textsc{dem}.\textsc{prox} \textsc{3sg}.\textsc{poss}-\textsc{ipfv}-\textsc{sbj}:\textsc{pcp}-read \textsc{sbj}:\textsc{pcp}-can \textsc{indef} be:\textsc{fact} \\
\glt `She is someone who can read this.' (2003 sras)
\end{exe}

Pairs of verbs in subject participle form should not necessarily be analyzed as complement clauses embedded in relatives. In (\ref{ex:pWkWNGlWt.kWthW}), \forme{pɯ-kɯ-ɴɢlɯt} `(bone) that has been broken, fracture' is not a (subject) complement of \forme{kɯ-tʰɯ}  `the one that is serious' (which is not a com\-ple\-ment-taking stative verb). Rather, \forme{wuma ʑo pɯ-kɯ-ɴɢlɯt kɯ-tʰɯ} is simply a head-internal relative clause (§\ref{sec:head-internal.relative} ) with the subject participle (itself a headless relative clause) \forme{pɯ-kɯ-ɴɢlɯt} as its subject. 

\begin{exe}
\ex \label{ex:pWkWNGlWt.kWthW}
\gll  [wuma ʑo [pɯ-kɯ-ɴɢlɯt] kɯ-tʰɯ] nɯra qʰe ndɤre, tɕʰaχɕaŋ tu-te qʰe tɕe tu-xtɕɤr ŋu.  \\
really \textsc{emph} \textsc{aor}-\textsc{sbj}:\textsc{pcp}-\textsc{acaus}:break \textsc{sbj}:\textsc{pcp}-be.serious \textsc{dem}:\textsc{pl} \textsc{lnk}   \textsc{lnk} splinter \textsc{ipfv}-put[III]   \textsc{lnk}  \textsc{lnk} \textsc{ipfv}-attach be:\textsc{fact} \\
\glt `As for the fractures that are serious, he puts a splinter on them and attaches it.' (140426 laxthab)
\japhdoi{0003810\#S7}
\end{exe}
 
\subsection{Constraints on relativizability} \label{sec:accessibility.relativization}
\is{relative clause!constraints} 
\tabref{tab:relatives.japhug} summarizes the syntactic functions accessible to relativization in main clauses in Japhug, indicating the clause types available for relativizing each function. The relatives can be headless in all cases, except that of possessor relativization, where a possessive prefix on the possessee is required (§\ref{sec:possessor.relativization}). Intransitive subjects and object are preferentially relativized with head-internal clauses, while transitive subjects, goals and adjuncts are more often relativized by prenominal clauses.


\begin{table}[h]
\caption{Summary of relative clauses in Japhug } \label{tab:relatives.japhug}
\begin{tabular}{lcccccc}
\lsptoprule
&\multicolumn{3}{c}{Participial Clause} & \multicolumn{2}{c}{Finite Clause} \\
\cmidrule(lr){2-4}
Function & \forme{kɯ-}  & \forme{kɤ-}  & \forme{sɤ-}  &  \\
\midrule
S	& \Y &&&   \\
possessor of S & \Y &&&   \\
A & \Y & &&  \\
\tablevspace
O & & \Y && \Y &\\
possessor of O & & \Y && \Y &\\
semi-object & & \Y && \Y &\\
theme & & \Y && \Y&\\ 
\tablevspace
goal & & &\Y  & \Y   \\
\tablevspace
dative & &&\Y \\
comitative & &&\Y \\
instrument  &(\Y) &&\Y \\ 
time adjunct  & &&\Y & \Y (prenominal) \\
locative adjunct  &&&\Y & \Y (prenominal) \\ 
\lspbottomrule
\end{tabular} 
\end{table}

Not all adjuncts are relativizable in Japhug: causees (§\ref{sec:causee.kW}), standards of comparative constructions (marked by the postpositions \forme{sɤz} or \forme{staʁ}, §\ref{sec:comparative}) and exceptive phrases in  \japhug{ma}{apart from} (§\ref{sec:exceptive}) apparently cannot be relativized, following a well-known cross-linguistic generalization \citep{keenan77accessibility}.

In addition,  relativization of the direct object of transitive verbs (in subject participle form) in the purposive clauses of motion verbs is not possible, as discussed in §\ref{sec:AM.mvc.relativizability}. Many  arguments or adjuncts from complement clauses can be relativized (§\ref{sec:out.complement.relativization}), but the limits on relativizability in embedded clauses is a topic for further fine-grained research.


\section{Relative clauses and focalization}
\is{relative clause!focalization} \is{focalization!relativization}
\subsection{Pseudo-cleft constructions} \label{sec:pseudo.cleft}
 \is{pseudo-cleft}  
Among the possible means of focalizing noun phrases, most if not all languages use an equative construction with a headless relative clause (or a relative clause with an overt head noun, if this head noun is different from the focalized noun phrase) in the topicalized position and the focalized noun phrase as the nominal predicate, as in English `[What matters] are \textbf{his ideas}' or `[The thing that matters] is \textbf{meaning}'.

Such constructions are generally referred to as pseudo-clefts, by contrast with cleft sentences, a type of construction (such as English `It is \textbf{his ideas} [that matter]') where the focalized noun is the head of the clause defining it, that clause being built like a relative, but sometimes presenting language-specific differences with relative clauses of the same type.\footnote{\citet[123--124]{creissels06sgit2} points out for instance that in French clefting of dative phrases (\textit{C'est à Jean \textbf{que} tu as donné le livre}) differs from the corresponding relativization (\textit{La personne \textbf{à laquelle} tu as donné le livre}).}

Pseudo-cleft constructions in Japhug consist of a headless relative clause (generally with the determiner \forme{nɯ} in topicalizing function, §\ref{sec:nW.topic}) followed by a nominal predicate with an affirmative (\japhug{ŋu}{be}, \japhug{ɕti}{be}) or negative (\japhug{maʁ}{not be}, §\ref{sec:suppletive.negative}) copula. 

Pseudo-clefts in Japhug do occur in intransitive subject (\ref{ex:jAkWGe.nW} and  \ref{ex:pWkWcha.nW}), direct object (\ref{ex:WkAndza.nWnW} and \ref{ex:akAnWrga.nW.nAZo}) or semi-objects (\ref{ex:pGa.ra.nWkArga}, §\ref{sec:semi.tr.relativization}) functions.

\begin{exe}
\ex \label{ex:jAkWGe.nW}
\gll [stu kɯ-mɤku jɤ-kɯ-ɣe] nɯ rɟɤlpu pjɤ-ŋu.\\
most \textsc{sbj}:\textsc{pcp}-be.before \textsc{aor}-\textsc{sbj}:\textsc{pcp}-come[II] \textsc{dem} king \textsc{ifr}.\textsc{ipfv}-be\\
\glt `The one who came first was the king.' (140514 xizajiang he lifashi-zh)
\japhdoi{0003996\#S62}
\end{exe}

\begin{exe}
\ex \label{ex:WkAndza.nWnW}
\gll ɯ-kɤ-ndza nɯnɯ nɯ tɕe, tɯ-ci ɯ-ŋgɯ qajɯ nɯra ɲɯ-ŋu rca ma \\
\textsc{3sg}.\textsc{poss}-\textsc{obj}:\textsc{pcp}-eat \textsc{dem} \textsc{dem} \textsc{lnk} \textsc{indef}.\textsc{poss}-water \textsc{3sg}.\textsc{poss}-inside bugs \textsc{dem}:\textsc{pl} \textsc{sens}-be \textsc{sfp} \textsc{lnk} \\
\glt `What [the otter] eats is aquatic animals, probably.' (28-qapar)
\japhdoi{0003720\#S86}
\end{exe}

When the predicate contains a first or second person pronoun, person indexation on the copula is obligatory (on the indexation rules of copulas, see §\ref{sec:copula.basic}). In  (\ref{ex:akAnWrga.nW.nAZo}) for instance, replacing the copula in the predicate \forme{nɤʑo tɯ-ŋu} by third person form $\dagger$\forme{nɤʑo ŋu} would be completely agrammatical.
 
\begin{exe}
\ex \label{ex:akAnWrga.nW.nAZo}
\gll [aʑo a-kɤ-nɯ-rga] nɯ nɤʑo tɯ-ŋu tɕe \\
\textsc{1sg} \textsc{1sg}.\textsc{poss}-\textsc{obj}:\textsc{pcp}-\textsc{appl}-like \textsc{dem} \textsc{2sg} 2-be:\textsc{fact} \textsc{lnk} \\
\glt `The one whom I love is you.' (160708 riquet5)
\japhdoi{0006185\#S27}
\end{exe}

Pseudo-cleft used to focalize transitive subjects (\ref{ex:tukWrqoR}) are rare, and mainly concerns instruments (§\ref{sec:instrument.relativization}) rather than agents.

\begin{exe}
\ex \label{ex:tukWrqoR}
\gll [ɯ-xtɤpa tu-kɯ-rqoʁ] nɯ pɤɕtʰɤβ ŋu. \\
\textsc{3sg}.\textsc{poss}-belly \textsc{ipfv}-\textsc{sbj}:\textsc{pcp}-hug \textsc{dem} belly.band be:\textsc{fact} \\
\glt `[The thing that] is attached around (`hugs') its belly is the belly band.' (30-tAsno)
\japhdoi{0003758\#S84}
 \end{exe}
 
Oblique participles can be used in pseudo-clefts to focalize locative adjuncts (\ref{ex:WsAdAn}).
 
\begin{exe}
\ex \label{ex:WsAdAn}
\gll  ma [ɯ-sɤ-dɤn] nɯ tɕetu rɯŋgu ŋu. \\
\textsc{lnk} \textsc{2sg}.\textsc{poss}-\textsc{obl}:\textsc{pcp}-be.many \textsc{dem} up.there pasture be:\textsc{fact} \\
\glt `The [place] where they are [most] numerous is up there on the pastures. (17-xCAj)
\japhdoi{0003528\#S81}
\end{exe}

 Most pseudo-clefts are participial relatives as in the examples above, but there are also finite object relatives (§\ref{sec:object.relativization}) as in (\ref{ex:tutWmtshi.nWnW}).
 
\begin{exe}
\ex \label{ex:tutWmtshi.nWnW}
\gll  [tu-tɯ-mtsʰi] nɯnɯ, mbalɤ-pɯ ɲɯ-maʁ \\
\textsc{ipfv}-2-lead \textsc{dem} ox-\textsc{dim} \textsc{sens}-not.be \\
\glt `What you are leading is not a calf, (but rather....)' (140512 fushang he yaomo)
\japhdoi{0003967\#S137}
\end{exe}

An alternative (and much rarer) type of pseudo-cleft has the headless relative as the nominal predicate, as in (\ref{ex:stu.Zo.WkAnWzdWG}).

\begin{exe}
\ex \label{ex:stu.Zo.WkAnWzdWG}
\gll ɲɤ-ɣɤwu matɕi tɕendɤre nɯ [stu ʑo ɯ-kɤ-nɯzdɯɣ] nɯ pjɤ-ɕti tɕe  \\
\textsc{ifr}-cry because \textsc{lnk} \textsc{dem} most \textsc{emph} \textsc{3sg}.\textsc{poss}-\textsc{obj}:\textsc{pcp}-worry \textsc{dem} \textsc{ifr}.\textsc{ipfv}-be.\textsc{add} \textsc{lnk} \\
\glt `He cried because it was what he was most worried about.' (140506 shizi he huichang de bailingniao-zh)
\japhdoi{0003927\#S64}
\end{exe}
 
Although pseudo-cleft constructions are well-attested in the corpus, they are not the main morphosyntactic device to focalize constituents in Japhug. Other focalizing constructions include sentence-final copulas (§\ref{sec:focalization.final.copula}) and focus particles (§\ref{sec:focus}). Some focalized noun phrases are also devoid of any specific marking of focalization, even intonational ones (§\ref{sec:focalization.overt}).

\subsection{Non-equative pseudo-cleft} \label{sec:pseudo.cleft2}
In addition, we find pseudo-cleft constructions whose second member is not a nominal predicate with a copula, but rather a clause which could stand as a complete sentence. The subject participle of \japhug{pe}{be good} is commonly used in these constructions in affirmative (\forme{kɯ-pe} `what is good is that...', `fortunately...') or negative form (\forme{kɯ-pe} `what is bad is that...', `unfortunately...') as in (\ref{ex:mAkWpe.tCe}).

\begin{exe}
\ex \label{ex:mAkWpe.tCe}
\gll   [mɤ-kɯ-pe] tɕe iɕqʰa, [ɯ-mu nɯ, βdaʁmu nɯ ʑatsa ɲɤ-si].\\
\textsc{neg}-\textsc{sbj}:\textsc{pcp}-be.good \textsc{lnk} \textsc{filler} \textsc{3sg}.\textsc{poss}-mother  \textsc{3sg}.\textsc{poss}-mother \textsc{dem} queen \textsc{dem} soon \textsc{ifr}-die \\
\glt `Unfortunately (what is bad is that...), her mother, the queen, died early.' (140504 baixuegongzhu-zh)
\japhdoi{0003907\#S12}
\end{exe}

In such constructions, the (topicalized) participial clause is followed by the linker \forme{tɕe} in topicalizing function (§\ref{sec:tCe.topic}) rather than the determiner \forme{nɯ}.

\begin{exe}
\ex \label{ex:mAkWnaXtCWG.tCe}
\gll tɕeri [mɤ-kɯ-naχtɕɯɣ] tɕe, [ɯ-ndzrɯ ɣɤʑu].  \\
but \textsc{neg}-\textsc{sbj}:\textsc{pcp}-be.similar \textsc{lnk} \textsc{3sg}.\textsc{poss}-claw exist:\textsc{sens} \\
\glt `What is different [between the footprints of the bear and those of a small child] is that [the former] has claws.' (21-pri)
\japhdoi{0003580\#S37}
\end{exe}

 
\section{Quantification} \label{sec:headless.relatives.quantification}
\is{relative clause!quantification} \is{quantification!relativization}
Relative clauses are found in three quantificational constructions. 

First, correlative relative clauses occur with interrogative pronouns used as free-choice indefinites (§\ref{sec:interrogative.relative}). In (\ref{ex:nAkAthu.tChi.kWtu}) for instance, the relative whose main verb is the participle \forme{kɯ-tu} has the pronoun \japhug{tɕʰi}{what} in apposition with the object participle \forme{nɤ-kɤ-tʰu} `(the things) that you ask' as head, with the meaning `any' or `whatever'.

\begin{exe}
\ex \label{ex:nAkAthu.tChi.kWtu}
\gll [[\textbf{nɤ-kɤ-tʰu}] \textbf{tɕʰi} kɯ-tu] tɤ-tʰe jɤɣ-o \\
\textsc{2sg}.\textsc{poss}-\textsc{obj}:\textsc{pcp}-ask what \textsc{sbj}:\textsc{pcp}-exist \textsc{imp}-ask[III] be.possible:\textsc{fact}-\textsc{sfp} \\
\glt `Ask any/whatever question you [may] have.' (conversation, 17-09-06)
\end{exe}

 
Second, totalitative reduplication (§\ref{sec:totalitative.redp}) indicates universal quantification of the relativized element (§\ref{sec:totalitative.relatives}). In particular, the reduplicated form of the \forme{kɯ\redp{}kɯ-tu} `all those that/who exist' of the existential verb \japhug{tu}{exist} can follow nouns (\ref{ex:azda.ra.kWkWtu}), with a meaning similar to the adverbial determiner \japhug{tʰamtɕɤt}{all} (§\ref{sec:universal.quant}). 

\begin{exe}
\ex \label{ex:azda.ra.kWkWtu}
\gll \textbf{a-zda} \textbf{ra} kɯ\redp{}kɯ-tu ʑo a-tɤ-bɯɣ-nɯ smɯlɤm \\
\textsc{1sg}.\textsc{poss}-companion \textsc{pl} \textsc{total}\redp{}\textsc{sbj}:\textsc{pcp}-exist \textsc{emph} \textsc{irr}-\textsc{pfv}-miss.home-\textsc{pl} prayer \\
\glt `May all of my companions miss home!' (2005 Norbzang)
\end{exe}

Constructions such as that in (\ref{ex:azda.ra.kWkWtu}) originate from head-internal (or postnominal) participial relatives (`all my existing companions'). However, it is unclear whether this analysis is still valid synchronically, and even whether \forme{kɯ\redp{}kɯ-tu} is still a participle. It may have been reanalyzed as a postnominal quantifier, as suggested by the fact that it can occur with pronouns, as in (\ref{ex:nWZora.kWkWtu}), where an interpretation in terms of a relative clause, whether restrictive (`all those of you who exist'?) or non-restrictive (`you, all the existing ones'?) is more difficult.
 
\begin{exe}
\ex \label{ex:nWZora.kWkWtu}
\gll  \textbf{nɯʑora} kɯ\redp{}kɯ-tu ʑo lɤ-nɯ-jɣɤt-nɯ \\
\textsc{2pl} \textsc{total}\redp{}\textsc{sbj}:\textsc{pcp}-exist \textsc{emph} \textsc{imp}:\textsc{upstream}-\textsc{vert}-turn.back \\
\glt `Turn back and go to your homes.' (2005 Norbzang)
\end{exe}

Third, headless relative clauses (§\ref{sec:headless.relative}) in existential constructions (§\ref{sec:existential.basic}) are more often used than indefinite pronouns (§\ref{sec:indef.pro}) to express non-specific indefinite entities, whether humans (`someone', \ref{ex:WkWlAT.GAZu}) or inanimate ones  (`something', \ref{ex:WkAndza.GAZu1}).

\begin{exe}
\ex \label{ex:WkWlAT.GAZu}
\gll  [a-<dianhua> ɯ-kɯ-lɤt] ɣɤʑu \\
 \textsc{1sg}.\textsc{poss}-phone \textsc{3sg}.\textsc{poss}-\textsc{sbj}:\textsc{pcp}-release exist:\textsc{sens} \\
\glt `There is \textbf{someone} calling me on the phone!' (conversation, 22-08-2018)
\end{exe}

\begin{exe}
\ex \label{ex:WkAndza.GAZu1}
\gll sɯŋgɯ si ɯ-mat a-pɯ-dɤn tɕe, nɯ ɯ-xpa nɯ ɲɯ-rkɯn ma kʰro kɤ-mto mɯ́j-ɣi ma [ɯ-kɤ-ndza] ɣɤʑu. \\
forest tree \textsc{3sg}.\textsc{poss}-fruit \textsc{irr}-\textsc{ipfv}-be.many \textsc{lnk} \textsc{dem} \textsc{3sg}.\textsc{poss}-year \textsc{dem} \textsc{sens}-be.few \textsc{lnk} much \textsc{obj}:\textsc{pcp}-see \textsc{neg}:\textsc{sens}-come \textsc{lnk} \textsc{3sg}.\textsc{poss}-\textsc{obj}:\textsc{pcp}-eat exist:\textsc{sens} \\
\glt `In years when there are many fruits on the trees in the forest, there are fewer \textit{Garrulax sp.} (\forme{ʁmɯrcɯ}) because they do not come [where they can be] seen, as they have \textbf{things} to eat (in the forest).' (23-pGAYaR)
\japhdoi{0003606\#S77}
\end{exe}

Head-internal relatives (§\ref{sec:head-internal.relative} ) can have a similar meaning if their head noun is a generic noun like \japhug{tɯrme}{person} (§\ref{sec:tWrme.generic}), as in (\ref{ex:tWrme.kWngo.GAZu}).

\begin{exe}
\ex \label{ex:tWrme.kWngo.GAZu}
\gll  iɕqʰa [\textbf{tɯrme} kɯ-ngo] ɣɤʑu tɕe, ɯ-kɯ-rtoʁ jɤ-ari-a wo! \\
just.before person \textsc{sbj}:\textsc{pcp}-be.ill exist:\textsc{sens} \textsc{lnk} \textsc{3sg}.\textsc{poss}-\textsc{sbj}:\textsc{pcp}-see \textsc{aor}-go[II]-\textsc{1sg} \textsc{sfp} \\
\glt `There is \textbf{someone} who is sick, I went to see him/her.' (conversation)
\end{exe}

Headless relative clauses also occur with a partitive meaning (some individuals among a group), as in (\ref{ex:WkAndza.GAZu2}). Note the difference in interpretation of the proposition \forme{ɯ-kɤ-ndza ɣɤʑu} in (\ref{ex:WkAndza.GAZu1}) `it has something to eat' and in (\ref{ex:WkAndza.GAZu2}) `there are some (fishes) that it eats'.

\begin{exe}
\ex \label{ex:WkAndza.GAZu2}
\gll qaɟy cʰo qajɯ ɯ-ŋgɯz kɯnɤ, [ɯ-kɤ-ndza] ɣɤʑu, [ɯ-mɤ-kɤ-ndza] ɣɤʑu. [ɯ-kɤ-nɤ-mɯm], [ɯ-mɤ-kɤ-nɤ-mɯm] ɣɤʑu. \\
fish \textsc{comit} bugs \textsc{3sg}.\textsc{poss}-inside:\textsc{loc} also \textsc{3sg}.\textsc{poss}-\textsc{obj}:\textsc{pcp}-eat exist:\textsc{sens}  \textsc{3sg}.\textsc{poss}-\textsc{neg}-\textsc{obj}:\textsc{pcp}-eat exist:\textsc{sens} \textsc{3sg}.\textsc{poss}-\textsc{obj}:\textsc{pcp}-\textsc{trop}-be.tasty \textsc{3sg}.\textsc{poss}-\textsc{neg}-\textsc{obj}:\textsc{pcp}-\textsc{trop}-be.tasty exist:\textsc{sens}  \\
\glt `Among fishes and other marine animals, there are \textbf{some} that [whales]$_i$ eat and \textbf{some} that they$_i$ don't eat, \textbf{some} that they$_i$ find tasty and \textbf{some} that they$_i$ don't find tasty.' (160703 jingyu)
\japhdoi{0006169\#S21}
\end{exe}

Since Japhug lacks negative pronouns (§\ref{sec:indef.pro}), the only way to express a meaning corresponding to negative pronouns in Japhug is by combining a headless relative clause with a negative existential verbs (§\ref{sec:negative.pronoun}), as in (\ref{ex:nAkWnWGmu.me}).

\begin{exe}
\ex \label{ex:nAkWnWGmu.me}
\gll nɤ-kɯ-nɯɣ-mu me \\
\textsc{2sg}.\textsc{poss}-\textsc{sbj}:\textsc{pcp}-\textsc{appl}-be.afraid not.exist:\textsc{fact} \\
\glt `\textbf{Nobody} is afraid of you!' (2002 qaCpa)
\end{exe}


Headless relative clauses in negative constructions are often embedded in a participial relative with the positive existential verb \japhug{tu}{exist}, as in (\ref{ex:WCWkWphWt.kWtu.me}).

\begin{exe}
\ex \label{ex:WCWkWphWt.kWtu.me}
\gll tɕeri [nɯra [ɯ-ɕɯ-kɯ-pʰɯt] ra kɯ-tu] me ma \\
\textsc{lnk} \textsc{dem}:\textsc{pl} \textsc{3sg}.\textsc{poss}-\textsc{tral}-\textsc{sbj}:\textsc{pcp}-take.off \textsc{pl} \textsc{sbj}:\textsc{pcp}-exist not.exist:\textsc{fact} \textsc{lnk} \\
\glt `But nobody goes and picks [wild strawberries].' (because nobody likes to eat them) (11-paRzwamWntoR)
\japhdoi{0003476\#S81}
\end{exe}

 
\section{Relative vs. complement clauses}  \label{sec:relative.complement.ambiguities}
This section discusses constructions which present real or apparent ambiguity between relative and complement clauses, and proposes a few syntactic tests to disambiguate the two.

To the cases studied below, the adnominal complement clauses (§\ref{sec:complement.taking.nouns}), which resemble prenominal relatives (§\ref{sec:prenominal.relative}), must be added.


\subsection{Ambiguity (finite clauses)} \label{sec:finite.relative.complement.ambiguity}
\is{ambiguity!relative clause} \is{ambiguity!complement clause}
Finite relative clauses (§\ref{sec:finite.relatives}), when they occur as objects or semi-object of verbs of perception (\japhug{mto}{see}, \japhug{mtsʰɤm}{hear} etc) or cognition (\japhug{tso}{know, understand}, `know' etc) may not be easily distinguishable from complement clauses, as this type of verb can take either nominal objects/semi-objects or complement clauses, and both finite relatives and finite complements are found with the same demonstrative determiners \forme{nɯ} and \forme{nɯnɯ} (§\ref{sec:relative.determiners.complementizer}, §\ref{sec:complement.determiner}).

For instance, in (\ref{ex:mbrWtCW.latCAt.nW}), the clause \forme{mbrɯtɕɯ la-tɕɤt} (§\ref{sec:object.relativization}) can either be interpreted as a head-internal finite object relative clause (§\ref{sec:head-internal.relative} , §\ref{sec:object.relativization}) `the knife that (the butcher) had unsheathed' or as a complement clause (§\ref{sec:finite.complement}) `that (the butcher) had unsheathed his knife'. Both interpretations would make sense in the context (§\ref{sec:relative.core.arg}).

\begin{exe}
\ex \label{ex:mbrWtCW.latCAt.nW}
\gll spjaŋkɯ ʁnɯz ni kɯ nɯnɯ nɤki, [\textbf{mbrɯtɕɯ} la-tɕɤt] nɯ pa-mto-ndʑi tɕe, wuma ʑo ɲɤ-mu-ndʑi. \\
wolf two \textsc{du} \textsc{erg} \textsc{dem} \textsc{filler} knife \textsc{aor}:3\flobv{}-take.out \textsc{dem} \textsc{aor}:3\flobv{}-see-\textsc{du} \textsc{lnk} really \textsc{emph} \textsc{ifr}-be.afraid-\textsc{du} \\
\glt `The two wolves, seeing that he had unsheathed his knife (or: seeing the knife that he had unsheathed), were very afraid. (150902 liaozhai lang-zh)
\japhdoi{0006340\#S29}
\end{exe}

Three criteria (already mentioned in §\ref{sec:finite.relatives}) can however help disambiguating between the two analyses in specific contexts. 

First, the presence of totalitative reduplication (§\ref{sec:totalitative.relatives}) indicates that the subordinate clause can only be analyzed as a relative, as in (\ref{ex:tWtAmWtindZi.nW}).\footnote{The relativized element of this headless clause is the semi-object, see §\ref{sec:recip.amW.indirective} on the argument structure of \japhug{amɯti}{say to each other}. }
 
\begin{exe}
\ex \label{ex:tWtAmWtindZi.nW}
\gll  stu kɯ-xtɕi nɯ kɯ nɯra [tɯ\redp{}tɤ-amɯ-ti-ndʑi] nɯ pjɤ-mtsʰɤm. \\
most \textsc{sbj}:\textsc{pcp}-be.small \textsc{dem} \textsc{erg} \textsc{dem}:\textsc{pl} t\textsc{otal}\redp{}\textsc{aor}-\textsc{recip}-say-\textsc{du} \textsc{dem} \textsc{ifr}-hear \\
\glt `The youngest [boy] heard all the things that they had said to each other.' (160630 poucet1)
\japhdoi{0006065\#S32}
\end{exe}

Second, finite subordinate clauses whose verb is in the Inferential, Sensory, Egophoric, Irrealis or Imperative cannot be relative clauses. For instance \forme{mɯ-pjɤ-pe} in Inferential Imperfective must be a complement clause.

\begin{exe}
\ex \label{ex:mWpjApe.nW.kotso}
\gll tɕe ɯʑo si tɤ-mda kóʁmɯz nɤ nɯ [mɯ-pjɤ-pe] nɯ ko-tso ri ɲɤ-maqʰu ɲɯ-ŋu. \\
\textsc{lnk} \textsc{3sg} die:\textsc{fact} \textsc{aor}-be.the.time only.then \textsc{add} \textsc{dem} \textsc{neg}-\textsc{ifr}.\textsc{ipfv}-be.good \textsc{dem} \textsc{ifr}-understand \textsc{lnk} \textsc{ifr}-be.late \textsc{sens}-be \\
\glt  `Just before dying, he understood that [what he had done] was not good, but it was too late.' (aesop nongfu yu she-zh)
\japhdoi{0006268\#S113}
\end{exe}

Third, finite relative clauses cannot relativize subjects, and therefore `plain' intransitive verbs (excluding semi-transitive verbs §\ref{sec:semi.transitive} and verbs with goals §\ref{sec:intr.goal}) cannot occur in finite relative clauses (except in the case of locative and time adjunct relativization, §\ref{sec:time.relativization}). Therefore, clauses such as \forme{mɯ-pjɤ-pe} `it was not good' in (\ref{ex:mWpjApe.nW.kotso}) and \forme{tu-ɣɤwu} `it cries/howls' in (\ref{ex:tuGAwu.nW.WmtshAm}) cannot be interpreted as subject relatives `(the thing) that was not good' or `(the wolf) that howls'.

\begin{exe}
\ex \label{ex:tuGAwu.nW.WmtshAm}
\gll nɯ [tu-ɣɤwu] nɯ ɯ-mtsʰɤm pɯ-rɲo-t-a ma  \\
\textsc{dem} \textsc{ipfv}-cry \textsc{dem} \textsc{3sg}.\textsc{poss}-\textsc{bare}.\textsc{inf}:hear \textsc{aor}-experience-\textsc{pst}:\textsc{tr}-\textsc{1sg} \textsc{lnk} \\
\glt `I did hear [wolves] howl.' (27-spjaNkW)
\japhdoi{0003704\#S27}
\end{exe}

\subsection{Ambiguity (non-finite clauses)} \label{sec:non-finite.relative.complement.ambiguity}
\is{ambiguity!relative clause}
Due to the resemblance between the velar infinitives \forme{kɯ-} and \forme{kɤ-} (§\ref{sec:velar.inf}) on the one hand and the subject (§\ref{sec:subject.participle.ambiguities}) and especially object participles (§\ref{sec:infinitives.participles}), distinguishing between infinitival clauses and participial relatives is not always trivial.

In (\ref{ex:yazi.tAkAmWrkW}), we find a series of non-finite verb forms in \forme{kɯ-} and \forme{kɤ-} in clauses that are object of the com\-ple\-ment-taking verb \japhug{fɕɤt}{tell} (§\ref{sec:velar.infinitives.complement.clauses}). Given the meaning of this example, it could appear to be preferrable to analyze these examples as complement clauses, translating \forme{<yazi> tɤ-kɤ-mɯrkɯ, tɤ-kɤ-ndza} as `(he told him) that he had stolen and eaten a duck' and \forme{ɯ-βri ...nɯ-kɯ-ɬoʁ} as `that his body started itching, and that feathers started growing on it'. 

\begin{exe}
\ex \label{ex:yazi.tAkAmWrkW}
\gll tɕe nɯra [pɯ\redp{}pɯ-kɯ-fse] nɯra, [<\textbf{yazi}> tɤ-kɤ-mɯrkɯ, tɤ-kɤ-ndza], qʰe ɕɤr tɕe [\textbf{ɯ-βri} tɤ-kɯ-rɤʑa] qʰe, [ɯ-βri tɕe iɕqʰa, <yazi> ɣɯ \textbf{ɯ-muj} nɯ-kɯ-ɬoʁ], nɯra pjɤ-fɕɤt. \\
\textsc{lnk} \textsc{dem}:\textsc{pl} \textsc{total}\redp{}\textsc{pst}.\textsc{ipfv}-\textsc{sbj}:\textsc{pcp}-be.like \textsc{dem}:\textsc{pl}  duck \textsc{aor}-\textsc{obj}:\textsc{pcp}-steal \textsc{aor}-\textsc{obj}:\textsc{pcp}-eat \textsc{lnk} evening \textsc{loc} \textsc{3sg}.\textsc{poss}-body \textsc{aor}-\textsc{sbj}:\textsc{pcp}-itch \textsc{lnk}   \textsc{3sg}.\textsc{poss}-body \textsc{loc} the.aforementioned duck \textsc{gen} \textsc{3sg}.\textsc{poss}-feather \textsc{aor}-\textsc{sbj}:\textsc{pcp}-come.out \textsc{dem}:\textsc{pl} \textsc{ifr}-tell \\
\glt `He told [the old man] everything that had happened, about \textbf{the duck that he had stolen and eaten}, and \textbf{his body itching, and the duck feathers that had grown on it}.' (150904 maya-zh)
\japhdoi{0006364\#S26}
 \end{exe}

However, the problem with analyzing these clauses as infinitival complement clauses is that the intransitive  motion verb \japhug{ɬoʁ}{come out}, being dynamic, never takes a \forme{kɤ-} infinitive, and that the form \forme{nɯ-kɯ-ɬoʁ} can only be a subject participle `(something) that has come out'. For this reason, the non-finite clauses in (\ref{ex:yazi.tAkAmWrkW}) have to be analyzed as head-internal and headless (object and subject) participial relative clauses.

An analysis of clauses in \forme{kɤ-} as infinitive complement clauses is restricted to com\-ple\-ment-taking verbs which unambiguously select \forme{kɤ-} infinitives of plain intransitive verbs (§\ref{sec:infinitives.participles}, §\ref{sec:velar.infinitives.complement.clauses}). 
 
\subsection{Participial clauses in core argument function} \label{sec:relative.pretence}
Some verbs such as the verbs of pretence \japhug{ʑɣɤpa}{pretend} and \japhug{nɯɕpɯz}{pretend} `dress up as', `imitate' select subject participle clauses as objects or semi-objects such as \forme{kɯ-ngo} in (\ref{ex:kWngo.toZGApa}), which could appear to be similar to the purposive complement of motion verbs (§\ref{sec:purposive.clause.motion.verbs}).

\begin{exe}
\ex \label{ex:kWngo.toZGApa}
\gll  [kɯ-ngo] to-ʑɣɤpa tɕe \\
\textsc{sbj}:\textsc{pcp}-be.sick \textsc{ifr}-pretend \textsc{lnk} \\
\glt `She pretended to be sick.' (Nyima Wodzer 2002)
 \end{exe}

However, there is clear evidence that these clauses are in fact headless relatives: (\ref{ex:kWngo.toZGApa}) can literally be translated as `she pretended to be a sick person'. The difference with purposive clauses can be shown by three tests. 

First, unlike motion verbs, pretence verbs can take nouns as objects (as shown by \ref{ex:qaɕpa.tonWCpWz} and \ref{ex:tAkWnWCpWz.pjAsWXsAl}) instead of clauses with subject participles.  

\begin{exe}
\ex \label{ex:qaɕpa.tonWCpWz}
 \gll  qaɕpa to-nɯɕpɯz, qaɕpa ɯ-rqʰu to-ŋga, \\
frog \textsc{ifr}-pretend frog \textsc{3sg}.\textsc{poss}-skin \textsc{ifr}-wear \\
\glt `He dressed up as a frog, he wore a frog's skin.' (2002 qaCpa)
\end{exe}
 
 Second, these verbs can occur with a participial clause whose subject is overt and different from the subject of the verb of the matrix clause, as in (\ref{ex:tApAtso.kWGAwu.kAnWCpWz}) where  \japhug{tɤ-pɤtso}{child} is the subject of the verb \japhug{ɣɤwu}{cry} in the participial clause, but not the subject of  \japhug{nɯɕpɯz}{pretend}, `disguise as', `imitate' (§\ref{sec:constr.participial.clause}). Such a subject mismatch would be completely ungrammatical with a purposive clause.
 
\begin{exe}
\ex \label{ex:tApAtso.kWGAwu.kAnWCpWz}
 \gll   [\textbf{tɤ-pɤtso} kɯ-ɣɤwu] ʑo kɤ-nɯɕpɯz mɤ-spe-a ma nɯ mɯma spe-a \\
 \textsc{indef}.\textsc{poss}-child \textsc{sbj}:\textsc{pcp}-cry \textsc{emph} \textsc{inf}-imitate \textsc{neg}-be.able[III]:\textsc{fact}-\textsc{1sg} \textsc{lnk} \textsc{dem} apart.from be.able[III]:\textsc{fact}-\textsc{1sg} \\
\glt `I cannot imitate a child crying, but apart from that I can imitate [anything].' (27-kikakCi)
\japhdoi{0003700\#S134}
\end{exe}

Third, we find examples like (\ref{ex:Wmi.kWmNAm.tonWCpWznW}) where the subject of the verb in the main clause is not coreferent with the subject of the participial clause but with the possessor of the subject. These cases can be accounted for as possessor relative clauses (§\ref{sec:subject.participle.other.relative}, §\ref{sec:S.possessor.relativization}):  (\ref{ex:Wmi.kWmNAm.tonWCpWznW}) could thus be literally translated as `he pretended to be someone whose leg hurt'.

\begin{exe}
\ex \label{ex:Wmi.kWmNAm.tonWCpWznW}
 \gll  tɕe [\textbf{ɯ}-mi kɯ-mŋɤm] to-nɯɕpɯz  \\
 \textsc{lnk} \textsc{3sg}.\textsc{poss}-leg \textsc{sbj}:\textsc{pcp}-hurt \textsc{ifr}-pretend \\
 \glt `He pretended to feel pain in his leg.' (140426 lang yisheng-zh)
 \japhdoi{0003808\#S9}
\end{exe}

 \subsection{Relativized complement clauses}  \label{sec:relativized.complement.clause}
In (\ref{ex:tWtatWt.tostu}),  the clause \forme{sŋaʁspa kɯ tɯ\redp{}ta-tɯt} `all (the things that) the sorcerer had said' is a headless object finite relative (§\ref{sec:object.relativization}); the presence of totalitative reduplication in particular, shows that it cannot be interpreted as a complement clause (§\ref{sec:totalitative.relatives}).

\begin{exe}
\ex \label{ex:tWtatWt.tostu}
\gll tɕeri [sŋaʁspa kɯ tɯ\redp{}ta-tɯt] nɯ to-stu \\
\textsc{lnk} sorcerer \textsc{erg} \textsc{total}\redp{}\textsc{aor}:3\flobv{}-say[II] \textsc{dem} \textsc{ifr}-do.like \\
\glt `He did it everything the way that the sorcerer had said.' (140511 alading-zh)
\japhdoi{0003953\#S85}
\end{exe} 


The syntactic function of this relative clause in the main clause is semi-object of the secundative verb \japhug{stu}{do like}, which encodes the manner of the action as semi-object, and the entity subjected to the action as the direct object  (§\ref{sec:ditransitive.secundative}). 

In view of (\ref{ex:tWtatWt.tostu}), it is tempting to analyze the clause \forme{``nɯ a-tɤ-fse ɲɯ-ra" tɤ-tɯ-tɯt} `(that) you said ``it should be (done) like that'' in (\ref{ex:YWra.tAtWtWt.nW}), which occurs in the same syntactic context, as a finite object relative clause too, differing from that of (\ref{ex:tWtatWt.tostu}) by being head-internal (§\ref{sec:head-internal.relative}) instead of headless.

\begin{exe}
\ex \label{ex:YWra.tAtWtWt.nW}
\gll [[``\textbf{nɯ} \textbf{a-tɤ-fse} \textbf{ɲɯ-ra}"] tɤ-tɯ-tɯt] nɯ tɤ-stu-t-a ŋu \\
\textsc{dem} \textsc{irr}-\textsc{pfv}-be.like \textsc{sens}-be.needed \textsc{aor}-2-say[II] \textsc{dem} \textsc{aor}-do.like-\textsc{pst}:\textsc{tr}-\textsc{1sg} be:\textsc{fact} \\
\glt `I did it [the way that] you said should be done.' (28-smAnmi)
\japhdoi{0004063\#S315}
\end{exe} 

What is remarkable about the construction in (\ref{ex:YWra.tAtWtWt.nW}) is that the relativized element is not a noun, but the (object) complement clause \forme{nɯ a-tɤ-fse ɲɯ-ra} `it should be (done) like that', embedded within the relative. 

A common example of relativized complement clause is found when the transitive perception verb \japhug{mtsʰɤm}{hear} (§\ref{sec:mto.mtshAm.complement}) occurs with the object participle \forme{kɤ-ti} of the verb \japhug{ti}{say} as in (\ref{ex:konWrNu.kAti}). In this example, \forme{tɯrme nɯ ko-nɯrŋu} has a double status: it is the object complement clause of \forme{kɤ-ti} within the relative, and at the same time, it constitutes the relativized element of the head-internal participial clause (§\ref{sec:object.relativization}). 

\begin{exe}
\ex \label{ex:konWrNu.kAti}
\gll `[[`tɯrme nɯ ko-nɯrŋu''] kɤ-ti] mɯ-pɯ-mtsʰam-a \\
person \textsc{dem} \textsc{ifr}-have.pig.disease \textsc{sbj}:\textsc{pcp}-say \textsc{neg}-\textsc{aor}-hear-\textsc{1sg} \\
\glt `I have never heard of people getting the pig disease.' (25-khArWm)
\japhdoi{0003644\#S77}
\end{exe} 

The literal meaning of this construction can be conveyed in English as `I have not heard ``a man got the pig disease'' being said'.
