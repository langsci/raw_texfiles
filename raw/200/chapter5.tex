%versao de 23-MAR-2019

\chapter{Modificação nominal}

Neste capítulo, discutiremos a interpretação de sintagmas nominais modificados por sintagmas adjetivais, sintagmas preposicionais e orações relativas. Do ponto de vista teórico, veremos como relacionar o uso dos
sintagmas adjetivais e preposicionais como predicados principais e
como modificadores nominais. Além disso, veremos como analisar semanticamente
orações relativas a fim de que também possam servir como
modificadores. Voltaremos a utilizar \textsc{atribuições} e
introduziremos um novo princípio composicional, chamado de
\textsc{abstração funcional}.

\section{Modificadores nominais}

No capítulo 2, discutimos o uso de sintagmas nominais, adjetivais
e preposicionais que funcionam como predicados principais em
sentenças como as em (\ref{pri}) a seguir:

%xl
\begin{exe}
\ex\label{pri}
\begin{xlist}
\ex João é médico.\label{pria}
\ex João é feliz.\label{prib}
\ex João é de Campinas.\label{pric}
\end{xlist}
\end{exe}

\n Já no capítulo anterior, analisamos descrições definidas em que
um sintagma nominal aparece junto ao artigo definido como na
sentença em (\ref{dosv}):

\begin{exe}
\ex O cachorro fugiu.\label{dosv}
\end{exe}


\n Vamos agora discutir alguns casos em que dois ou mais desses
elementos se combinam, como nos
exemplos abaixo:

%xl
\begin{exe}
\ex\label{pc}
\begin{xlist}
\ex Sultão é um cachorro branco.\label{pca}
\ex João é um médico de Campinas. \label{pcb}
\ex O cachorro branco de Campinas fugiu.\label{pcc}
\end{xlist}
\end{exe}

\n Nesses casos, diremos que os sintagmas adjetivais e
preposicionais modificam os sintagmas nominais juntos dos quais
aparecem. Em (\ref{pca}), por exemplo, temos o nome \textit{cachorro}
modificado pelo adjetivo \textit{branco} formando o predicado \textit{cachorro branco}. Para que (\ref{pca}) seja verdadeira, não basta apenas que
Sultão seja um cachorro, ou apenas que ele seja branco, mas
sim que ele seja as duas coisas, isto é, um cachorro branco. Da
mesma forma, em (\ref{pcb}), João é descrito como um indivíduo que
é tanto médico quanto proveniente de Campinas. Já em (\ref{pcc}),
quem fugiu foi um animal que tem três características: é um
cachorro, é branco e é de Campinas.

Além dos sintagmas adjetivais e preposicionais, analisaremos
também o uso de orações relativas como modificadores nominais,
como em (\ref{tre}):

\begin{exe}
\ex O cachorro que o João comprou fugiu.\label{tre}
\end{exe}

Nesse caso, o referente em questão satisfaz duas condições: ser um cachorro e ter sido comprado pelo João. Orações relativas exercem um papel particularmente importante na modificação nominal. Essas orações fornecem um estoque potencialmente ilimitado de modificadores, sem que se precise alargar infinitamente o léxico com adjetivos e preposições relacionadas a critérios arbitrariamente específicos, como seria requerido, por exemplo, para replicar o sentido do DP abaixo:

\begin{exe}
	\ex O cachorro que o João comprou de um chinês no início da semana passada durante uma feira de animais na cidade de São Paulo.\label{ret}
\end{exe}

\n Sendo modificadores complexos construídos no nível sentencial, tira-se partido da composicionalidade e da recursividade típicas das línguas naturais, aliviando o papel do léxico. 

Note-se, por fim, que, além de orações relativas, os modificadores nominais podem também ser orações infinitivas (como em \textit{documentos para assinar}) ou gerundivas (como em \textit{caixas contendo medicamentos}). Esses, entretanto, não serão analisados semanticamente neste livro.



\section{Sintagmas adjetivais}

Comecemos pela sentença abaixo:

\begin{exe}
	\ex O cachorro branco fugiu.\label{sul}
\end{exe}

\n Do ponto de vista sintático, vamos considerar o sintagma adjetival
como um adjunto do sintagma nominal, conforme a representação
a seguir:

\begin{figure}[H]
	\centerline{ \Tree [ [ [ o ].D [ [ [ cachorro ].N ].NP [ [ branco ].A ].AP ].NP ].DP [ [ fugiu ].V  ].VP ].S } \caption{Modificação de NP por AP }
\end{figure}



Temos uma descrição definida na posição de sujeito encabeçada pelo artigo \textit{o} e que toma por complemento um NP. Até aqui, nada de novo. A novidade é a natureza complexa desse NP, formado por um AP e outro NP. E, aqui, deparamo-nos com um problema.
Quando analisamos NPs e APs no capítulo 2, tratamos a extensão
desses sintagmas como sendo funções de tipo $\langle e,t\rangle$:

\begin{exe}
	\ex \den{[$_{\text{NP}}$ cachorro]} = $\lambda x_{e}.\ \predica{cachorro}{x}$
\end{exe}

\begin{exe}
	\ex \den{[$_{\text{AP}}$ branco]} = $\lambda x_{e}.\ \predica{branco}{x}$
\end{exe}

\n A questão agora é como obter a extensão do NP \textit{cachorro
branco}. Aplicação funcional não pode nos servir, já que
temos duas funções de tipo $\langle e,t\rangle$, nenhuma podendo
servir como argumento para a outra. Neste ponto, restam-nos duas alternativas. Na primeira, modificamos a extensão atribuída a NPs e/ou APs,
de modo que a extensão de um possa servir de argumento à extensão do outro. Na segunda, criamos um novo princípio composicional que
especifique como obter a extensão de um sintagma cujos
constituintes imediatos possuem, ambos, extensões de tipo $\langle
e,t\rangle$. 


Comecemos com a primeira alternativa. Para não
perdermos tudo o que vimos no capítulo anterior sobre a combinação
de NPs com os determinantes \textit{o}, \textit{a}, vamos manter as
extensões dos NPs como funções de tipo $\langle e,t\rangle$, e
vamos alterar as extensões dos APs. Esqueçamos então,
temporariamente, as extensões de tipo $\langle e,t\rangle$ que
atribuímos a esses sintagmas.

Retornemos ao valor semântico do NP \textit{cachorro
branco}. Como as condições de verdade de (\ref{sul}) nos dizem que
quem fugiu deve ser não apenas um cachorro, mas também ser
da cor branca, já sabemos o que queremos para a extensão do NP:

\begin{exe}
	\ex \den{[$_{\text{NP}}$ cachorro branco]} = $\lambda x_{e}.\ \predica{cachorro}{x}\ \&\ \predica{branco}{x}$
\end{exe}

\n Como tanto a extensão do NP \textit{cachorro} quanto a do NP
\textit{cachorro branco} são de tipo $\langle e,t\rangle$, para
valermo-nos de aplicação funcional, devemos atribuir à extensão de
AP, e, por conseguinte, à extensão do adjetivo \textit{branco}, o
tipo $\langle\langle e,t\rangle,\langle
e,t\rangle\rangle$. Informalmente, \textit{branco} modifica a extensão do NP, retornando uma outra função que, por sua vez, retorna o valor 1 apenas para os indivíduos previamente caracterizados pela extensão do NP que são brancos. Pondo tudo isso em termos formais, temos:

\begin{exe}
	\ex \den{branco} = $\lambda F_{\langle e,t \rangle}.\lambda x_{e}.\ F'(x)\ \&\ \predica{branco}{x}$
\end{exe}

\begin{exe}
	\ex \den{cachorro branco} = \den{branco}(\den{cachorro})
\end{exe}

\n Essa igualdade entre os tipos semânticos da extensão do NP que serve de
argumento à extensão de AP e da extensão do NP
resultante permite derivar a
extensão de NPs modificados por mais de um adjetivo, como no exemplo a seguir, através de múl\-ti\-plas instâncias de aplicação funcional:

\begin{exe}
	\ex $[_{\text{NP}}\ \text{cachorro branco peludo} ] $
\end{exe}

\begin{figure}[H]
	\centerline{ \Tree [ [ [ [ cachorro ].N ].NP [ [ branco ].A ].AP ].NP [ [ peludo ].A ].AP ].NP } \caption{ Iteração de modificadores adjetivas }
\end{figure}


\begin{exe}
	\ex \den{cachorro branco peludo} = \den{peludo}(\den{cachorro branco})
\end{exe}

\n Como já sabemos que:

\begin{exe}
	\ex \den{cachorro branco} = \den{branco}(\den{cachorro})
\end{exe}

\n temos que:

\begin{exe}
	\ex \den{cachorro branco peludo} = \\ \den{peludo}(\den{branco}(\den{cachorro})) = \\
	$\lambda x.\ \predica{cachorro}{x}\ \&\ \predica{branco}{x}\ \&\ \predica{peludo}{x}$
\end{exe}



\n Temos, portanto, como interpretar estruturas em que um ou mais
adjetivos modificam um sintagma nominal. Em todos esses casos, o resultado é uma função de tipo $\langle e,t\rangle$, que em casos como (\ref{sul}) é o tipo de argumento esperado pela extensão do artigo definido, conforme vimos no último capítulo. Resta-nos, agora, retornar aos
casos em que o adjetivo não aparece junto a nenhum nome, como em \textit{Sultão é branco}. Faremos isso na próxima seção, em que introduziremos uma nova ferramenta analítica.

\section{Mudança de tipos}

Consideremos o exemplo a seguir:

\begin{exe}
	\ex Sultão é branco.\label{wer}
\end{exe}

\n Para casos como esse, precisamos ainda de nossa antiga entrada
lexical que atribuía à extensão de adjetivos como \textit{branco}
o tipo $\langle e,t\rangle$. Isto equivale a dizer que precisamos
assumir que esse adjetivo é ambíguo e que no léxico estão listadas
as duas entradas abaixo (vamos usar a marca $'$ para distinguir as
entradas):

\begin{exe}
	\ex
	\begin{xlist}
		\ex \den{branco} = $\lambda x_{e}.\ \predica{branco}{x}$
		\ex \den{branco$'$} = $\lambda F_{\langle e,t\rangle}.\ F'(x)\ \&\ \predica{branco}{x}$
	\end{xlist}
\end{exe}

\n O mesmo valeria pra outros adjetivos, como \textit{peludo}, por exemplo:

\begin{exe}
	\ex
	\begin{xlist}
		\ex \den{peludo} = $\lambda x_{e}.\ \predica{peludo}{x}$
		\ex \den{peludo$'$} = $\lambda F_{\langle e,t\rangle}.\ F'(x)\ \&\ \predica{peludo}{x}$
	\end{xlist}
\end{exe}



\n Duplicando, uma a uma,  as entradas lexicais de todos os
adjetivos, deixamos escapar a generalização de que a ambiguidade
de tipos $\langle e,t\rangle$ e $\langle\langle e,t\rangle,\langle
e,t\rangle\rangle$ não é uma característica acidental encontrada
nesse ou naquele adjetivo, mas um fenômeno geral que parece afetar
toda a classe dos adjetivos. Além disso, nossa intuição parece
clara em reconhecer que a semelhança entre os adjetivos nos pares
de sentenças abaixo não é apenas fonológica, mas também semântica. Ou seja, os usos dos adjetivos em cada um dos membros desses pares
nos parecem claramente relacionados do ponto de vista do
significado. Parece justo, portanto, cobrar de uma teoria semântica a explicitação dessa relação.

Por fim, cumpre notar que não são apenas adjetivos listados no
léxico, mas também sintagmas adjetivais e preposicionais complexos, como
\textit{fiel ao João} ou \textit{de Campinas}, que podem aparecer acompanhando ou não um
sintagma nominal, como atestam os exemplos abaixo:

%xl
\begin{exe}
\ex\label{sas}
\begin{xlist}
\ex Sultão é fiel ao João.\label{sasa}
\ex O cachorro fiel ao João fugiu.\label{sasb}
\end{xlist}
\end{exe}

%xl
\begin{exe}
	\ex\label{sasp}
	\begin{xlist}
		\ex João é de Campinas.\label{sasap}
		\ex O médico de Campinas chegou.\label{sasbp}
	\end{xlist}
\end{exe}

\n Para esses casos, também precisaríamos ter duas possibilidades:

\begin{exe}
	\ex
	\begin{xlist}
		\ex \den{[$_{\text{AP}}$ fiel ao João]} = $\lambda x_{e}.\ \predica{fiel}{x,joão}$
		\ex \den{[$_{\text{AP}}$ fiel ao João]} = $\lambda F_{\langle e,t\rangle}.\ \lambda x_{e}.\ F'(x)\ \&\ \predica{fiel}{x,joão}$
	\end{xlist}
\end{exe}

\begin{exe}
	\ex
	\begin{xlist}
		\ex \den{[$_{\text{PP}}$ de Campinas]} = $\lambda x_{e}.\ \predica{ser\_de}{x,campinas}$ 
		\ex \den{[$_{\text{PP}}$ de Campinas]} = $\lambda F_{\langle e,t\rangle}.\ \lambda x_{e}.\ F'(x)\ \&\ \predica{ser\_de}{x,campinas}$
	\end{xlist}
\end{exe}

\n Para captar essa generalização acerca da possibilidade de uso
de sintagmas adjetivais e preposicionais acompanhados ou não de um sintagma
nominal, vamos formular uma regra que se aplica a todos os
sintagmas adjetivais, relacionando formalmente as duas extensões
que se podem atribuir a eles. Trata-se de uma \textsc{regra de
mudança de tipos}, que assume que o tipo básico das extensões dos
APs é $\langle e,t\rangle$ e explicita como obter uma extensão de
tipo $\langle\langle e,t\rangle,\langle e,t\rangle\rangle$ a
partir da extensão de tipo $\langle e,t\rangle$:

\begin{exe}
	\ex Regra de mudança de tipos para sintagmas adjetivais e preposicionais \\
	Seja \textit{X} um sintagma adjetival ou preposicional cuja extensão $\alpha$ é de tipo $\langle e,t\rangle$. Mude a extensão de \textit{X} de $\alpha$ para $\alpha '$, sendo $\alpha '$ de tipo $\langle\langle e,t\rangle,\langle e,t\rangle\rangle$ e definida da seguinte forma: $\alpha ' = (\lambda F_{\langle e,t\rangle}.\lambda x_{e}.\ F'(x)\ \&\ \alpha(x) = 1)$
\end{exe}



\n A aplicação desta regra fica condicionada ao contexto sintático em que o sintagma aparecer. Assim, para o
adjetivo \textit{branco}, por exemplo, basta listá-lo no léxico
como denotando uma função de tipo $\langle e,t\rangle$. Isso
(junto com o princípio dos nós não ramificados) nos daria a
extensão do AP necessária para a interpretação de
sentenças como (\ref{wer}):

\begin{exe}
	\ex \den{AP} = $\lambda x.\ \predica{branco}{x}$
\end{exe}

\n Para sentenças como (\ref{sul}), em que o AP modifica um NP,
recorreríamos à regra acima, que nos forneceria o seguinte:

\begin{exe}
	\ex \den{AP$'$} = $\lambda F_{\langle e,t\rangle}.\lambda x_{e}.\ F'(x)\ \&\ \llbracket \text{AP} \rrbracket (x) = 1$\\
	\den{AP$'$} = $\lambda F_{\langle e,t\rangle}.\lambda x_{e}.\ F'(x)\ \&\ (\lambda x.\ \predica{branco}{x})(x) = 1$\\
	\den{AP$'$} = $\lambda F_{\langle e,t\rangle}.\lambda x_{e}.\ F'(x)\ \&\ \predica{branco}{x}$
\end{exe}

\n O leitor poderá verificar a adequação dessa regra, aplicando-a
na derivação das condições de verdade das demais sentenças
apresentadas nesta seção (ver exercício I). O recurso a uma regra de mudança de tipos nos proporcionou captar a flexibilidade atrelada ao uso dos sintagmas adjetivais e preposicionais sem apelarmos a ambiguidades idiossincráticas vinculadas separadamente a cada adjetivo listado no léxico. O preço pago foi a postulação de uma nova regra semântica. Na próxima seção, veremos uma alternativa que mantém uma única interpretação para os APs e PPs. 

\section{Conjunção funcional}

Voltemos à tarefa a que nos propusemos no
início do capítulo: interpretar expressões como \textit{cachorro
branco}, sabendo que seus constituintes imediatos tinham como extensões funções de tipo $\langle e,t\rangle$. Desta vez, vamos
analisar um tratamento alternativo, ao qual fizemos alusão mais
acima, em que os tipos semânticos de NPs e APs permanecem
inalterados, mas um novo princípio composicional é introduzido.
Chamemos este novo princípio de \textsc{Conjunção funcional}:

\begin{exe}
	\ex Conjunção funcional \\
	Seja $\alpha$ um nó ramificado, cujos constituintes imediatos são $\beta$ e $\gamma$, tal que \den{$\beta$} e \den{$\gamma$} pertençam a D$_{\langle e,t\rangle}$. Neste caso, \den{$\alpha$} = $\lambda x_{e}.\ $\den{$\beta$}$(\textit{x}) = 1$\ \&\ \den{$\gamma$}$(\textit{x}) = 1$
\end{exe}


\n Conjunção funcional permite que obtenhamos uma extensão de tipo
$\langle e,t\rangle$ para um nó cujos constituintes imediatos
possuem, ambos, extensões do tipo $\langle e,t\rangle$. Note a semelhança com o que vimos no capítulo 3 sobre a interpretação
da conjunção \textit{e} em casos como [$_{\text{VP}}$ fuma e bebe] ou NPs como [$_{\text{NP}}$ médico e psicólogo], inclusive no uso do conectivo lógico \& na metalinguagem. Voltando a
nosso exemplo, utilizamos o nosso novo princípio para obter a
extensão do NP \textit{cachorro branco} da seguinte forma:

\begin{exe}
	\ex \den{cachorro branco} = $\lambda x_{e}.\ \llbracket  \text{cachorro} \rrbracket(x) = 1\ \&\ \llbracket \text{branco} \rrbracket(x) = 1$\\
	\den{cachorro branco} = $\lambda x_{e}.\ \predica{cachorro}{x}\ \&\ \predica{branco}{x}$
\end{exe}

\n Conjunção funcional também nos serve nos casos em que um PP
modifica um NP, como em \textit{médico de Campinas}:



\begin{figure}[H]
	\centerline{ \Tree [ [ [ médico ].N ].NP [ [ de ].P [ [ Campinas ].N ].NP ].PP ].NP } \caption{Modificação de NP por PP }
\end{figure}

\n Note que este uso da preposição \textit{de} indicando origem
geográfica difere do uso que vimos quando tratamos de NPs como
\textit{irmão de Maria} ou APs como \textit{fiel ao João}. Para esses últimos, a
preposição foi tratada como um item semanticamente vácuo. Para o
caso acima, o que queremos é o seguinte:

\begin{exe}
	\ex \den{de} = $\lambda x_{e}.\ \lambda y_{e}.\ \predica{ser\_de}{y,x}$
\end{exe}

\n Como o leitor já pode antecipar, a extensão do PP \textit{de Campinas} é obtida através de Aplicação
Funcional. Já a extensão do NP \textit{médico de Campinas} é obtida utilizando-se Conjunção funcional:

\begin{exe}
	\ex \den{de Campinas} = \den{de}(\den{Campinas}) = $\lambda y.\ \predica{ser\_de}{y,campinas}$
\end{exe}

\begin{exe}
	\ex\den{médico} = $\lambda y.\ \predica{médico}{y}$
\end{exe}

\begin{exe}
	\ex \den{médico de Campinas} = $(\lambda y.\ \llbracket \text{médico} \rrbracket = 1\ \&\ \llbracket \text{de Campinas} \rrbracket = 1)$ \\
	\den{médico de Campinas} = $\lambda y.\ \predica{médico}{y}\ \&\ \predica{ser\_de}{y,campinas}$
\end{exe}

No restante deste capítulo, continuaremos a nos valer da regra de mudança de tipos formulada
mais acima em vez da conjunção funcional. Tal escolha,
entretanto, é de certo modo arbitrária, não estando baseada em
nenhuma consideração teórica ou em\-pí\-ri\-ca. Tanto em um caso como no outro, aplicação funcional precisará ser acompanhada de uma nova regra para que todos os casos relevantes sejam cobertos.

\section{Modificação e classes de comparação}

Considere o argumento abaixo:

\begin{exe}
\ex Sultão é um cachorro branco.\\
Todo cachorro é um animal.\\
Logo, Sultão é um animal branco.
\end{exe}

\n Trata-se de um argumento válido, ou seja, um argumento em que a
verdade das premissas leva necessariamente à verdade da conclusão.
Nosso sistema nos ajuda a entender esse fato. Das condições de
verdade que atribuímos à primeira premissa, sabemos que se ela é
verdadeira, então Sultão é um cachorro e Sultão é branco. A
verdade da segunda premissa (cuja interpretação discutiremos no
próximo capítulo) nos diz que, para todo indivíduo, se esse
indivíduo é um cachorro, então ele é um animal. Como já sabemos
que Sultão é um cachorro, logo concluímos que Sultão é um animal.
Como da verdade da primeira premissa também já sabemos que Sultão
é branco, concluímos que Sultão é um animal e Sultão é branco. E
isso satisfaz as condições de verdade da conclusão
do argumento.

Considere agora o seguinte argumento:

\begin{exe}
\ex\label{nueba} Lili é uma formiga grande.\\
Toda formiga é um animal.\\
Logo, Lili é um animal grande.
\end{exe}

\n Esse argumento não parece válido. Ainda que assumamos a verdade
das premissas, a verdade da conclusão não se segue. A razão é que
uma formiga grande pode muito bem ser um animal pequeno. Ou seja,
quando usamos o adjetivo \textit{grande} na primeira premissa,
estamos muito provavelmente comparando o tamanho de Lili com algo como o tamanho médio
das formigas. Já no caso da conclusão, parecemos estar comparando
o tamanho de Lili com o tamanho  médio dos animais em geral, ou,
pelo menos, de alguns animais mais comuns como formigas,
cachorros, tigres, elefantes, etc. Diferentemente do
significado do adjetivo \textit{branco}, o significado de
adjetivos como \textit{grande} parece fazer menção a uma classe de
comparação, a partir da qual se estabelece um parâmetro de
avaliação para se dizer se o adjetivo se aplica ou não a um
determinado indivíduo. Como vimos acima, a natureza do nome que o
adjetivo modifica influencia a escolha dessa classe: a classe das
formigas, no caso do NP \textit{formiga grande}, ou a classe dos
animais, no caso do NP \textit{animal grande}. Mas não é só o
contexto linguístico que influencia a escolha da classe de
comparação. Pense no uso do adjetivo \textit{alto} na sentença
abaixo:

\begin{exe}
\ex João é alto.\label{alt}
\end{exe}

\n Note primeiro que o adjetivo não aparece adjacente a qualquer substantivo comum. Imagine agora que João meça um metro e oitenta e cinco
centímetros, e que alguém nos pergunte se (\ref{alt}) é verdadeira
ou falsa. Diferentes contextos levariam a diferentes respostas.
Por exemplo, se João é uma criança de dez anos, certamente diremos
que (\ref{alt}) é verdadeira. Se João já é um adulto jogador de
basquete, já não estaríamos tão certos, e provavelmente
acrescentaríamos que \textit{para um jogador de basquete}, João não
é alto. Nota-se, portanto, que a classe de comparação em
questão pode vir também do contexto extrasentencial e que, à
medida que uma conversa segue seu curso, diferentes classes de
comparação podem ser usadas na interpretação de adjetivos como
\textit{grande} ou \textit{alto}.

Para captarmos essa dependência contextual e para bloquearmos
inferências como a do argumento em (\ref{nueba}), precisamos tratar adjetivos
como \textit{grande} ou \textit{alto} de maneira distinta da que
tratamos \textit{branco}. Não iremos
de modo algum propor uma teoria detalhada sobre o estabelecimento
e a natureza de uma classe de comparação, muito menos elucidar
questões profundas sobre dependência contextual. Nosso objetivo
aqui é mais modesto. Queremos apenas indicar na entrada lexical de
certos adjetivos a influência do contexto, ainda que de maneira
rudimentar. Assim, contentemo-nos com o seguinte:

\begin{exe}
	\ex \den{grande} = $\lambda x_{e}.$ o tamanho de $x$ está acima do padrão estabelecido pela classe de comparação fornecida pelo contexto de fala.
\end{exe}

\n A não validade do argumento em (\ref{nueba}) decorre, portanto, do fato de
que o contexto de fala poder mudar entre o momento que falamos/ouvimos a
primeira premissa e o momento que falamos/ouvimos a conclusão,
tornando a extensão da ocorrência do adjetivo \textit{grande} na
primeira premissa diferente da extensão da ocorrência do mesmo na
conclusão.

A despeito das diferenças apontadas acima entre o comportamento lógico-semântico dos adjetivos \textit{branco} e \textit{grande}, a aplicação deles a um NP guarda uma semelhança. Em ambos os casos, sentenças da forma \textit{DP é um NP AP} acarretam sentenças da forma \textit{DP é um NP}. Concretamente, se Sultão é um cahorro branco, então Sultão é um cachorro. Igualmente, se Sultão é um cachorro grande, então Sultão é um cachorro. Gostaríamos, antes de finalizar essa seção, de mencionar que há adjetivos para os quais essas inferências não são válidas e que colocam problemas adicionais ao nosso sistema, seja ele baseado na ideia de modificação como aplicação funcional ou baseado no princípio de conjunção funcional. Pense, por exemplo, em adjetivos como \textit{suposto}. Dizer que alguém é um suposto criminoso não nos leva à conclusão de que esse alguém é um criminoso. Ou pense em adjetivos como \textit{falso}. Seria um diamante falso um diamante? E um relógio falso, seria um relógio? Muito provavelmente, sua reposta aqui seria um inseguro `talvez'. Esses casos, que não discutiremos aqui, desafiam o tratamento que demos à modificação adjetival e mesmo o projeto de uma semântica extensional, já que as extensões de NPs como \textit{suposto criminoso} ou \textit{diamante falso} não parecem sequer depender da extensão dos respectivos substantivos modificados (sobre isso, ver algumas das sugestões de leitura ao final do capítulo).  

\section{Orações relativas}

Como já mencionamos na introdução deste capítulo, assim como sintagmas adjetivais e preposicionais podem modificar
sintagmas nominais, também constituintes sentencias podem servir
de modificadores nominais, na forma de orações relativas, por exemplo. Em
(\ref{rel}) abaixo, estão alguns exemplos contendo variedades
dessas orações encontradas em português:


\begin{exe}
    \ex\label{rel}
    \begin{xlist}
        \ex  O cachorro que João adora fugiu.\label{rela}
        \ex  O cachorro de que João gosta fugiu.\label{relb}
        \ex  O cachorro que João gosta fugiu.\label{relc}
        \ex  O cachorro que João adora ele fugiu.\label{reld}
    \end{xlist}
\end{exe}

\n Em (\ref{rela}), temos uma oração relativa modificando o nome
\textit{cachorro}. A posição relativizada corresponde à posição de
objeto direto do verbo \textit{adorar}. O mesmo se dá em
(\ref{relb})-(\ref{relc}), exceto pelo fato de que o verbo
\textit{gostar} é transitivo indireto, requerendo o uso da
preposição \textit{de}. (\ref{relb}) pertence ao registro formal
da língua, enquanto (\ref{relc}) e (\ref{reld}) pertencem ao
registro falado, usado com enorme frequência no português brasileiro coloquial. Em (\ref{rela})-(\ref{relc}), a posição relativizada
aparece vazia, enquanto em (\ref{reld}), um pronome aparece
nesta posição. Esse pronome --- \textit{ele}, no caso --- é chamado de
\textsc{pronome resumptivo}.

Vamos nos ocupar aqui apenas de (\ref{rela}) e (\ref{reld}), já que
(\ref{relb}) e (\ref{relc}) exigem um refinamento sintático que
está além do que podemos oferecer neste livro. A esse respeito,
cumpre dizer que, mesmo nos casos de  (\ref{rela}) e (\ref{reld}),
vamos nos esquivar de uma série de detalhes morfossintáticos que
têm sido objeto de estudo de muitos sintaticistas e para os quais
diversas análises alternativas têm sido propostas não só para o
português brasileiro, mas também para outras línguas.

Comecemos por (\ref{reld}), cujo sujeito receberá a seguinte
estrutura:

\begin{figure}[H]
	\centerline{ \Tree [.DP [.D o ] [.NP [.NP [.N cachorro ] ] [.S\1 que$_1$ [.S [.DP João ] [.VP [.V adora ] [.DP ele$_1$ ] ] ] ] ] ].DP } \caption{Modificação de NP por oração relativa com pronome resumptivo }
\end{figure}


\n Trataremos a palavra \textit{que} que aparece no início da
oração relativa como um pronome relativo e atribuiremos a ele o
mesmo índice atribuído ao pronome resumptivo \textit{ele} que
aparece na posição relativizada, indicando que esses dois
elementos estão relacionados. A oração relativa como um todo está
adjungida ao NP \textit{cachorro} e o NP resultante
\textit{cachorro que João adora ele} atua como complemento do
determinante \textit{o}, formando um DP.

Passemos então à interpretação desse DP, começando pelo NP
\textit{cachorro que João adora ele}. Façamos um paralelo com o que
vimos anteriormente a respeito do NP \textit{cachorro branco}.
Tratamos esse NP como denotando uma função de tipo $\langle
e,t\rangle$ que leva um indivíduo qualquer no valor de verdade 1,
se, e somente se, esse indivíduo for um cachorro e for também
branco. Parece, então, natural tratar o NP \textit{cachorro
que  João adora ele} como denotando uma função de tipo $\langle
e,t\rangle$ que leva um indivíduo qualquer no valor de verdade 1
se, e somente se, esse indivíduo for um cachorro e for também um
indivíduo que João adora.

\begin{exe}
	\ex \den{cachorro que João adora ele} = $\lambda x.\ \predica{cachorro}{x}\ \&\ \predica{adora}{joão,x}$
\end{exe}

\n Para nossa oração relativa,
queremos associá-la semanticamente a uma função de tipo $\langle e,t\rangle$ que leve um
indivíduo \textit{x} no valor de verdade 1
se, e somente se, João adora \textit{x}. Dessa forma, poderemos estender nossa análise de modificação
envolvendo sintagmas adjetivais e preposicionais para casos
envolvendo orações relativas. A tarefa que nos resta, claro, é
obter tal resultado composicionalmente.

Com esse intuito, vamos primeiro operar uma pequena modificação na estrutura que adotamos acima para a oração relativa. Seguindo uma proposta elaborada em \cite{heikra98}, vamos assumir que, na interface sintaxe-semântica, o índice que atribuímos ao pronome relativo seja transferido do pronome para o constituinte irmão deste pronome, adjungindo-se a ele, como esquematizado a seguir:


\begin{figure}[H]
	\centerline{ \Tree [. que$_i$ $\ \ \ \ \ \alpha\ \ \ \ \ $ ] \hspace{0.3in} $\Longrightarrow$ \hspace{0.3in} \Tree [. {\ \ \ que\ \ \ } [. $i$ $\ \ \ \ \ \alpha\ \ \ \ \ $ ] ] } \caption{Transferência de índice }
\end{figure}

Aplicando esse processo à estrutura que estamos analisando, teríamos o seguinte (para facilitar a referência futura aos constituintes no topo da oração relativa, vamos rotulá-los de S$'$ e S$''$):

\begin{figure}[H]
	\centerline{ \Tree [.DP [.D o ] [.NP [.NP [.N cachorro ] ] [.S\2 que [.S\1 1 [.S [.DP João ] [.VP [.V adora ] [.DP ele$_1$ ] ] ] ] ] ] ] } \caption{Oração relativa após transferência de índice }
\end{figure}


\n Esse processo de transferência de ín\-di\-ces é apenas uma conveniência, mas que será especialmente
útil quando analisarmos estruturas derivadas por movimento e a interpretação de sintagmas quantificadores no pró\-xi\-mo ca\-pí\-tu\-lo.

Vejamos, então, como podemos proceder. Com relação ao constituinte
S, já sabemos o que o nosso sistema deriva. Para uma atribuição $g$ qualquer, temos:

\begin{exe}
	\ex \den{S}$^{g}$ = 1 \textit{sse} João adora g(1).
\end{exe}

\n Como esse constituinte contém um pronome, sua extensão depende
da atribuição \textit{g}. Mas note que, ao passarmos de
\textit{S} para a oração relativa completa, não queremos que esse pronome seja tratado
dessa forma. Ou seja, não queremos que ele se refira
a um indivíduo tornado saliente pelo contexto de fala. Ao
contrário, pela interpretação da oração relativa que descrevemos
acima, esse pronome funciona como uma \textsc{variável ligada} e não se refere a um indivíduo contextualmente saliente.

Formalmente, para derivarmos a extensão que desejamos,
precisamos de um mecanismo que nos permita obter uma função (tipo
$\langle e,t\rangle$, no nosso caso) a partir da extensão de
\textit{S} (que é dependente de uma atribuição), levando-se em
conta o índice do pronome relativo que marca a posição
relativizada no interior de \textit{S}. Para interpretar  constituintes como $S'$, que dominam um
índice numérico, elaboraremos um novo princípio composicional,
que chamaremos de \textsc{abstração funcional}. Vamos enunciá-lo e depois explicá-lo:

\begin{exe}
	\ex Abstração funcional \\
	Seja $\alpha$ um nó ramificado cujos constituintes imediatos são um índice numérico \textit{i} e $\beta$. Então, \den{$\alpha$}$^{g}$ = $\lambda x. \llbracket \beta \rrbracket^{g[i \rightarrow x]}$
\end{exe}



\n Se você se lembrar do que vimos no capítulo 2 sobre a notação lambda, você perceberá que a expressão correspondente à extensão de $\alpha$ na regra acima é uma \textit{abstração}-$\lambda$ obtida a partir da extensão de $\beta$. Abstrações lambda, como já sabemos, representam funções. Daí o nome desse princípio. Quanto à notação $\underline{\llbracket \beta \rrbracket^{g[i
\rightarrow x]}}$, ela deve ser interpretada da seguinte forma: a
extensão de $\beta $ em relação a uma atribuição $g'$,
que é idêntica a $g$ exceto pelo fato de que $g'$
mapeia o número $i$ no indivíduo $x$, ou seja, $g'(i)=x$. Os exemplos
abaixo ilustram a notação $g[{i \rightarrow x}]$:

\begin{exe}
	\ex Exemplos de atribuições modificadas \\\\
	$g:\ \left[%
	\begin{array}{@{}c@{\ }l@{\ }l@{}}%
	1 & \rightarrow& \text{João} \\
	2 & \rightarrow & \text{Pedro} \\
	3 & \rightarrow & \text{Maria} \\
	\end{array}\right]$\ \ \ \ \ \ $g[1\rightarrow \text{Carlos}]:\ \left[%
	\begin{array}{@{}c@{\ }l@{\ }l@{}}%
	1 & \rightarrow& \text{Carlos} \\
	2 & \rightarrow & \text{Pedro} \\
	3 & \rightarrow & \text{Maria} \\
	\end{array}\right]$\\\\

	$g:\ \left[%
	\begin{array}{@{}c@{\ }l@{\ }l@{}}%
	1 & \rightarrow& \text{João} \\
	2 & \rightarrow & \text{Pedro} \\
	3 & \rightarrow & \text{Maria} \\
	\end{array}\right]$\ \ \ \ \ \ $g[3 \rightarrow \text{Paula}]:\ \left[%
	\begin{array}{@{}c@{\ }l@{\ }l@{}}%
	1 & \rightarrow& \text{João} \\
	2 & \rightarrow & \text{Pedro} \\
	3 & \rightarrow & \text{Paula} \\
	\end{array}\right]$\\\\

	$g:\ \left[%
	\begin{array}{@{}c@{\ }l@{\ }l@{}}%
	1 & \rightarrow& \text{João} \\
	2 & \rightarrow & \text{Pedro} \\
	\end{array}\right]$\ \ \ \ \ \ $g[3 \rightarrow \text{Maria}]:\ \left[%
	\begin{array}{@{}c@{\ }l@{\ }l@{}}%
	1 & \rightarrow& \text{João} \\
	2 & \rightarrow & \text{Pedro} \\
	3 & \rightarrow & \text{Maria} \\
	\end{array}\right]$
\end{exe}

\n Note que para qualquer atribuição \textit{g}, o domínio de
$g[i \rightarrow x]$ será idêntico ao domínio de \textit{g} se o
número \textit{i} já pertencer ao domínio de \textit{g}, como nos dois primeiros casos acima.
Entretanto, o domínio de $g[i \rightarrow x]$ incluirá o domínio de \textit{g} se o número \textit{i} não
pertencer ao domínio de \textit{g}, como no terceiro caso acima. Mas o mais importante nesse ponto é notar que, para uma atribuição \textit{g}, um indivíduo \textit{x} e um índice \textit{i} quaisquer, teremos sempre:

\begin{exe}
	\ex $g[i \rightarrow x](i) = x$
\end{exe}

\n Antes de prosseguir, uma última e breve observação notacional. Adotamos aqui para as atribuições modificadas a notação $g[i \rightarrow x]$, que tomamos emprestada de \cite{buring05}, e que nos parece mais transparente e amigável do que a notação mais comum em manuais de lógica, que é $g[i/x]$. De qualquer forma, trata-se apenas de uma questão de gosto, sendo ambas equivalentes semanticamente.

Com essas definições e exemplos em mente, podemos usar
a abstração funcional para a obtenção da extensão do
constituinte $S'$ na estrutura que estamos
interpretando:

\begin{exe}
	\ex \den{$S'$}$^{g}$ = $\lambda x. \llbracket \text{S} \rrbracket^{g[1 \rightarrow x]}$ \label{adbel}
\end{exe}

\n Note que a mesma variável $x$ que aparece junto ao operador lambda aparece na especificação da atribuição em relação à qual a extensão de S está relativizada. É isso que permitirá ao(s) pronome(s) que estivere(m) marcado(s) com o índice $1$ no interior de S se comportarem semanticamente como variáveis ligadas. Vejamos os detalhes. S é uma sentença simples formada por um verbo transitivo e seus dois argumentos. Sua extensão, portanto, é obtida através de aplicação funcional (além, claro, do princípio dos nós não ramificados):

\begin{exe}
	\ex \den{S}$^{g[1 \rightarrow x]}$ = \den{adora}$^{g[1 \rightarrow x]}$(\den{ele$_{1}$}$^{g[1 \rightarrow x]}$)(\den{João}$^{g[1 \rightarrow x]}$)
\end{exe}

\n As extensões do verbo \textit{adorar} e do sujeito \textit{João}, por não conterem pronomes, não apresentam nada de novo:

\begin{exe}
	\ex 
	\begin{xlist}
		\ex \den{adora}$^{g[1 \rightarrow x]}$ = $\lambda x.\lambda y.\ \predica{adora}{y,x}$
		\ex \den{João}$^{g[1 \rightarrow x]}$ = \textit{joão}
	\end{xlist}
\end{exe}

\n Em relação ao pronome \textit{ele}, dado o que sabemos desde o capítulo anterior e o que acabamos de ver acima, temos:

\begin{exe}
	\ex \den{ele$_{1}$}$^{g[1 \rightarrow x]}$ = $g[1 \rightarrow x](1)$ = $x$ 
\end{exe}

\n Juntando tudo isso, temos:

\begin{exe}
	\ex \den{S}$^{g[1 \rightarrow x]}$ = 1 \textit{sse} $\predica{adora}{joão,x}$
\end{exe}

\n Somando isso ao que já tínhamos em (\ref{adbel}), chegamos ao seguinte:

\begin{exe}
	\ex \den{S$'$}$^{g}$ = $\lambda x.\ \predica{adora}{joão,x}$
\end{exe}

\n No fim das contas, o pronome resumptivo $ele_{1}$ acabou sendo interpretado como uma variável ligada pelo operador lambda. Esse, por sua vez, foi introduzido pela regra de abstração funcional ao se deparar com o índice $1$, o mesmo índice do pronome. Isso foi possível pois aplicação funcional e o princípio dos nós não ramificados não interferem na atribuição, apenas a ``passam adiante'', por assim dizer. Isso permite uma conexão semântica entre o índice numérico que aparece junto ao pronome relativo no início da oração e os pronomes marcados com esse mesmo índice em seu interior, mesmo que haja uma distância sintática entre eles.

Voltando à nossa derivação, obtivemos o resultado que desejávamos, com o constituinte S$'$ se assemelhando semanticamente a um predicado de indivíduos. Note ainda que a
extensão de S$'$ não depende da atribuição \textit{g}, sendo
sempre a mesma função que leva um indivíduo no valor 1 se, e
somente se, João adora esse indivíduo. Isto é desejável, já que
a contribuição da oração relativa que estamos analisando não
parece mesmo depender do contexto.

Para obtermos a extensão do NP \textit{cachorro que João adora ele},
temos duas opções, de acordo com o que vimos nas seções anteriores. A primeira delas é assumir que o pronome relativo seja
semanticamente vácuo e nos valer do princípio de conjunção
funcional, já que tanto a extensão
do NP \textit{cachorro} quanto a extensão da oração relativa
\textit{que João adora ele} seriam de tipo $\langle e,t\rangle$. A segunda opção seria atribuir uma extensão ao pronome relativo e
valermo-nos de aplicação funcional. Para tanto, essa
extensão deve ser uma função de tipo $\langle et,\langle et,
et\rangle\rangle$, definida da seguinte forma:

\begin{exe}
	\ex \den{que}$^{g}$ = $\lambda F_{\langle e,t\rangle}.\lambda G_{\langle e,t\rangle}.\lambda x_{e}.\ G'(x)\ \&\ F'(x)$
\end{exe}

\n Note que, de acordo com essa entrada, cabe ao pronome relativo o papel da conjunção dos predicados correspondentes à oração relativa e ao NP que ela modifica. Para a oração relativa S$''$ acima, teríamos:

\begin{exe}
	\ex \den{S$''$}$^{g}$ = \den{que}$^{g}$(\den{S$'$}$^{g}$) \\
	\den{$S''$}$^{g}$ = $\lambda G_{\langle e,t\rangle}.\lambda x_{e}.\ G'(x)\ \&\ \predica{adora}{joão,x}$
\end{exe}

\n Note que essa extensão da oração relativa é de tipo $\langle et,et\rangle$, o mesmo tipo dos sintagmas adjetivais que modificam NPs, refletindo o caráter de modificador nominal destas orações. Como a extensão de \textit{cachorro} é uma função de tipo
$\langle e,t\rangle$, valemo-nos de aplicação funcional mais uma
vez para chegarmos à extensão do NP \textit{cachorro que João
adora ele}:

\begin{exe}
	\ex \den{NP}$^{g}$ = \den{S$''$}$^{g}$(\den{cachorro}$^{g}$)\\
	\den{NP}$^{g}$ = $\lambda x_{e}.\ \predica{cachorro}{x}\ \&\ \predica{adora}{joão,x}$
\end{exe}

\n Por fim, para a obtenção da extensão do DP \textit{o cachorro
que João adora ele}, basta aplicarmos a extensão do artigo definido
à extensão do NP que acabamos de derivar e o resultado, como o
leitor poderá verificar, será o seguinte: o único indivíduo
\textit{x} saliente no contexto, tal que \textit{x} é um cachorro
e o João adora \textit{x}.

\subsection{Vestígios e movimento sintático}

\n A oração relativa que analisamos acima continha um pronome
resumptivo na posição relativizada. Como já mencionamos, esse não
é um traço comum a todas as orações relativas, podendo a posição
relativizada aparecer vazia, como no exemplo (\ref{rela}),
repetido abaixo como (\ref{erl}):

\begin{exe}
    \ex O cachorro que João adora fugiu. \label{erl}
\end{exe}

\n Do ponto de vista sintático, analisaremos esse tipo de oração relativa de acordo com a Teoria da Regência e Ligação (ver referências ao final do capítulo). Dessa perspectiva teórica, o pronome relativo \textit{que} se moveria da posição relativizada (objeto
direto, no exemplo acima) até o início da oração, deixando em sua
posição de partida um vestígio \textit{t} (de \textit{trace}, em inglês) com o mesmo índice e sem realização fonética. Essa
análise está representada abaixo (assumindo o mesmo processo de
transferência de índice discutido anteriormente):

\begin{figure}[H]
	\centerline{ \Tree [.NP [.NP cachorro ] [. que [. 1 \qroof{João adora t$_{1}$}.S ] ] ] } \caption{Estrutura resultante do movimento do pronome relativo e da transferência de índice }
\end{figure}



\n Será essa a estrutura que o componente
semântico interpretará. Para tanto, precisamos atribuir uma
extensão ao vestígio indexado. Assumiremos aqui que orações relativas
com vestígios e com pronomes resumptivos contribuem de maneira idêntica para
as condições de verdade das sentenças que as contêm e analisaremos
os vestígios da mesma forma como analisamos os pronomes
resumptivos, ou seja, como variáveis ligadas. Vestígios, ao
contrário dos pronomes, não admitem interpretações dependentes do contexto. Para
nós, entretanto, isso não será relevante, já que admitiremos
tratar-se de uma restrição puramente sintática. Temos, então, o
seguinte:

\begin{exe}
	\ex Entrada lexical dos vestígios \\
	Para qualquer vestígio \textit{t}, atribuição \textit{g} e número natural \textit{i},\\

	$\llbracket t_{i}\rrbracket^{g} =
	\begin{cases}
    	g(i) & \text{se } i\in \text{Domínio}(g)\\
    	indefinido & \text{se } i\not\in \text{Domínio}(g)
	\end{cases}$
\end{exe}


\n Dada a semelhança na interpretação de pronomes e vestígios, a
derivação de (\ref{erl}) prossegue de maneira idêntica à derivação
de sua contraparte com o pronome resumptivo. Consequentemente, as
mesmas condições de verdade serão derivadas.

Nos exemplos que discutimos acima, os pronomes relativos estavam
coindexados ou com um vestígio ou com um pronome. É possível,
entretanto, que em uma mesma oração relativa haja um vestígio e um
pronome, ambos coindexados com o mesmo pronome relativo. Considere
o exemplo abaixo:

\begin{exe}
    \ex O cachorro que mordeu o dono dele fugiu. \label{fug}
\end{exe}

\n Em uma das leituras possíveis para (\ref{fug}), o pronome
\textit{ele} tem sua referência determinada contextualmente. Imagine, por exemplo,
que estamos apontando para \textit{Sultão}, um cachorro cujo dono
foi mordido por um outro cachorro, que fugiu logo após a mordida.
Essa leitura pode ser captada atribuindo ao pronome \textit{ele}
um índice diferente do índice atribuído ao pronome relativo e ao
vestígio deixado na posição de sujeito:

\begin{exe}
    \ex $[$ cachorro [ que$_{1}$ [ t$_{1}$ mordeu o dono dele$_{2}$ ]]]
    \label{est}
\end{exe}

\n Essa atribuição de índices leva à seguinte extensão para a
oração relativa:

\begin{exe}
	\ex \den{que$_{1}$ t$_{1}$ mordeu o dono dele$_{2}$}$^{g}$ = $\lambda x.\ \predica{mordeu}{x, o dono de g(2)}$
\end{exe}

\n Note que essa extensão depende do valor da atribuição
\textit{g}. Se, por exemplo, \textit{g}(2) = Sultão, então teremos
uma função que leva um indivíduo no valor de verdade 1 se, e
somente se, este indivíduo tiver mordido o dono de Sultão. Vejamos, como exercício, a derivação dessa extensão. Após a transferência do índice do
pronome relativo, temos a seguinte estrutura para a oração relativa:

\begin{exe}
    \ex $[_{\text{S}''}$ que [$_{\text{S}'}$ 1 [$_{\text{S}}$ t$_{1}$ mordeu o dono dele$_{2}$ ]]]   \label{esa}
\end{exe}

\n Apresentamos abaixo a derivação, passo a passo, dessa estrutura para que o leitor possa conferir em detalhe a aplicação de tudo o que vimos até aqui. Procederemos de cima para baixo na estrutura:

\begin{exe}
	\ex Derivação semântica de (\ref{esa})\\
	1. \den{S$''$}$^{g}$ = \den{que}$^{g}$(\den{S$'$}$^{g}$) \hfill (AF)\\
	2. \den{S$'$}$^{g}$ = $\lambda x.$ \den{S}$^{g[1 \rightarrow x]}$ \hfill (AbF)\\
	3. \den{S$''$}$^{g}$ = \den{que}$^{g}$($\lambda x.$ \den{S}$^{g[1 \rightarrow x]}$) \hfill (1,2)\\
	4. \den{S}$^{g[1 \rightarrow x]}$ = \den{mordeu o dono dele$_{2}$}$^{g[1 \rightarrow x]}$(\den{t$_{1}$}$^{g[1 \rightarrow x]}$) \hfill (AF)\\
	5. \den{mordeu o dono dele$_{2}$}$^{g[1 \rightarrow x]}$ = \\
	\hspace*{\fill} \den{mordeu}$^{g[1 \rightarrow x]}$(\den{o dono dele$_{2}$}$^{g[1 \rightarrow x]}$) \ \ \ (AF) \\
	6. \den{mordeu}$^{g[1 \rightarrow x]}$ = $\lambda x.\lambda y.\ \predica{mordeu}{y,x}$ \hfill (L)\\
	7. \den{o dono dele$_{2}$}$^{g[1 \rightarrow x]}$ = \den{o}$^{g[1 \rightarrow x]}$(\den{dono dele$_{2}$}$^{g[1 \rightarrow x]}$) \hfill (AF)\\
	8. \den{o}$^{g[1 \rightarrow x]}$ = $\lambda F.\ \iota z[F'(z)]$ \hfill (L)\\
	9. \den{dono dele$_{2}$}$^{g[1 \rightarrow x]}$ = \den{dono}$^{g[1 \rightarrow x]}$(\den{dele$_{2}$}$^{g[1 \rightarrow x]}$) \hfill (AF)\\
	10. \den{dono}$^{g[1 \rightarrow x]}$ = $\lambda x.\lambda y.\ \predica{dono}{y,x}$ \hfill (L)\\
	11. \den{dele$_{2}$}$^{g[1 \rightarrow x]}$ = \den{ele$_{2}$}$^{g[1 \rightarrow x]}$ \hfill (NNR, \textit{de} é vácua)\\
	12. \den{ele$_{2}$}$^{g[1 \rightarrow x]}$ = $g[1 \rightarrow x](2)$ = $g(2)$ \hfill (L, Atr.Mod)\\
	13. \den{dono dele$_{2}$}$^{g[1 \rightarrow x]}$ = $\lambda y.\ \predica{dono}{y,g(2)}$ \hfill (9,10,11,12)\\
	14. \den{o dono dele$_{2}$}$^{g[1 \rightarrow x]}$ = $\iota z[\predica{dono}{z,g(2)}]$ \hfill (7,8,13)\\
	15. \den{mordeu o dono dele$_{2}$}$^{g[1 \rightarrow x]}$ = \\
	\hspace*{\fill} $\lambda y.\ \textsc{mordeu}(y,\iota z[\textsc{dono}(z,g(2))])$ \ \ \  (5,6,14)\\
	16. \den{t$_{1}$}$^{g[1 \rightarrow x]}$ = $g[1 \rightarrow x](1)$ = $x$ \hfill (L,Atr.Mod)\\
	17. \den{S}$^{g[1 \rightarrow x]}$ = 1 \textit{sse} $\textsc{mordeu}(x,\iota z[\textsc{dono}(z,g(2))])$ \hfill (4,15,16)\\
	18. \den{S$'$}$^{g}$ = $\lambda x.\ \textsc{mordeu}(x,\iota z[\textsc{dono}(z,g(2))])$ \hfill (2,17)\\
	19. \den{S$''$}$^{g}$ = \den{que}$^{g}$($\lambda x.\ \textsc{mordeu}(x,\iota z[\textsc{dono}(z,g(2))])$) \hfill (1,18)\\
	20. \den{que}$^{g}$ = $\lambda F_{\langle e,t\rangle}.\lambda G_{\langle e,t\rangle}.\lambda x_{e}.\ G'(x)\ \&\ F'(x)$ \hfill (L)\\
	21. \den{S$''$}$^{g}$ = $\lambda G_{\langle e,t\rangle}.\lambda x_{e}.\ G'(x)\ \&\ \textsc{mordeu}(x,\iota z[\textsc{dono}(z,g(2))])$ \hfill (19,20)
\end{exe}

\n Combinando o resultado acima com o a extensão do nome
\textit{cachorro}, obteremos o resultado que desejávamos
para o NP \textit{cachorro que mordeu o dono dele}:

\begin{exe}
	\ex \den{NP}$^{g}$ = $\lambda x_{e}.\ \predica{cachorro}{x}\ \&\ \textsc{mordeu}(x,\iota x[\textsc{dono}(x,g(2))])$
\end{exe}

\n Na outra leitura possível para (\ref{fug}), quem fugiu foi um
cachorro que mordeu o próprio dono. Essa leitura pode ser
captada se coindexarmos o pronome relativo, o vestígio na
posição de sujeito e o pronome \textit{ele}:

\begin{exe}
    \ex $[_{\text{NP}}$ cachorro [ que$_{1}$ [$_{\text{S}}$ $t_{1}$ mordeu o dono dele$_{1}$ ] ] ]
    \label{coi}
\end{exe}

\n Nesse caso, dada a coindexação entre o vestígio, o pronome
\textit{ele} e o pronome relativo, obteremos o seguinte:

\begin{exe}
	\ex \den{t$_{1}$}$^{g[1 \rightarrow x]}$ = $x$ 
\end{exe}

\begin{exe}
	\ex \den{dele$_{1}$}$^{g[1 \rightarrow x]}$ = $x$
\end{exe} 

\begin{exe}
	\ex \den{S}$^{g[1 \rightarrow x]}$ = 1 \textit{sse} $\textsc{mordeu}(x,\iota z[\textsc{dono}(z,x)])$
\end{exe}

\n O restante da derivação procede de maneira análoga ao que vimos
acima, resultando na seguinte extensão para o NP \textit{cachorro que mordeu o dono dele}:

\begin{exe}
	\ex \den{NP}$^{g}$ = $\lambda x_{e}.\ \predica{cachorro}{x}\ \&\ \textsc{mordeu}(x,\iota z[\textsc{dono}(z,x)])$
\end{exe}

\n Conforme desejávamos, obtivemos uma extensão que não depende da atribuição.

\bigskip

\begin{tcolorbox}[parbox=false,boxrule=0pt,sharp corners,breakable]

\section*{Sugestões de leitura}
\addcontentsline{toc}{section}{Sugestões de leitura}

\n Para uma cobertura abrangente da modificação (adjetival e adverbial), ver \cite{morzycki15}. Sobre a semântica dos adjetivos graduáveis, como \textit{grande} e \textit{alto}, e das construções em que aparecem, ver \cite{murphy10}, capítulo 11, para uma discussão informal. Para um tratamento formal, ver \cite{kennedy97} e as referências lá citadas. Sobre as diversas estratégias de relativização do português brasileiro, ver \cite{tarallo90}. O tratamento sintático das orações relativas via movimento e ligação de vestígios tem origem nos trabalhos de Noam Chomsky. Para uma introdução, consultar \cite{carnie13}, \cite{haegeman94} e \cite{mioal05}. Para análises em outros quadros teóricos que não se valem de movimento sintático, ver \cite{sagal03} e \cite{jacobson99}. Por fim, notamos que a regra de conjunção funcional que utilizamos neste capítulo corresponde à regra de modificação predicacional (\textit{predicate modification}, em inglês) definida em \cite{heikra98}. 

\end{tcolorbox}

\bigskip

\begin{tcolorbox}[parbox=false,boxrule=0pt,sharp corners,breakable]

\section*{Exercícios}
\addcontentsline{toc}{section}{Exercícios}

\n\textbf{I.} Considere o DP abaixo e derive sua extensão, passo a passo, indicando os princípios composicionais
utilizados.\\


\n (1)  $[_{\text{DP}}\ \text{A escritora que João detestou o livro que ela publicou}]$ \\

\n\textbf{II.} Considere a sentença (2) abaixo e as duas atribuições de índices em (i) e (ii) logo a seguir:\\

\n (2) O cachorro que João comprou mordeu o gato que Maria comprou.\\

\n (i) O cachorro que$_{1}$ João comprou t$_{1}$ mordeu o gato que$_{1}$ Maria comprou t$_{1}$.\\

\n (ii) O cachorro que$_{1}$ João comprou t$_{1}$ mordeu o gato que$_{2}$ Maria comprou t$_{2}$.\\

\n Há alguma diferença semântica entre essas representações? Por quê? Qual a diferença ente o que está acontecendo nesses casos e o que vimos com o exemplo (\ref{fug}) e as representações em (\ref{est}) e (\ref{coi})?


\end{tcolorbox}












%%

%%
