\chapter{Introduction}
\label{sec:1}

The working environment of translators has changed tremendously in recent decades. Typesetters have been replaced by computers, printed sources of information have been replaced by electronic and online sources of information. Instead of translating every single word from scratch, \isi{translation memory systems} store translations and recall them when certain similarities exist between new source text segments and a source text segment that has been translated before. Instead of word lists and printed glossaries, translators use \isi{terminology management systems} to assure consistency. 
\is{machine translation systems|(}Machine translation (MT) systems have been developed for over 70 years now – nonetheless, they only recently started to affect the working environment of most translators.\is{machine translation systems|)} 
\is{post-editing|(}To improve efficiency and cost-effectiveness, organisations increasingly use MT and edit the MT output to create a fluent text that adheres to the relevant text conventions. This procedure is known as post-editing (PE). Although PE has also been around since the 1980s, it remained a rather niche market for decades. This, however, has changed with PE being established on the translation market in recent years – causing mixed feelings among professional translators. The working conditions are changing and some translators are comfortable with this change, while some are not.\is{post-editing|)}
But what changes for the professional translator who disregards external circumstances? What influence does the integration of MT have on the cognitive load of professional translators?



The aim of this study is to investigate different problem solving behaviours in \isi{translation from scratch}\footnote{The term "translation from scratch" was used – instead of „human translation“ for example – because it implies that no further CAT tools were used for the translation, like \isi{translation memory} or terminology management systems.} (TfS) and \isi{post-editing} (PE). I assume that some problems might already be solved by MT output, while, on the other hand, the MT system might also create new translation problems. Hence, participants will exhibit at least some different \isi{problem solving behaviour} in the two tasks. This will be analysed according to research behaviour as well as the syntactic quality of MT output. These analyses will not only include screen-recording data and final translation products, but also keylogging and eyetracking data. Finally, this study will focus on problem identifiers in translation process data. While early \isi{translation process research} (e.g. %\label{ref:ZOTEROITEMCSLCITATIONcitationIDysieRWZwpropertiesformattedCitationKrings1986plainCitationKrings1986dontUpdatetruecitationItemsid1709urishttpzoteroorggroups3587itemsX92T2ZTGurihttpzoteroorggroups3587itemsX92T2ZTGitemDataid1709typebooktitleWasindenKpfenvonbersetzernvorgehtEineempirischeUntersuchungzurStrukturdesbersetzungsprozessesanfortgeschrittenenFranzsischlernernpublisherGunterNarrVerlagpublisherplaceTbingeneventplaceTbingennoteZuglBochumUnivDiss198586authorfamilyKringsgivenHansPeterissueddateparts1986schemahttpsgithubcomcitationstylelanguageschemarawmastercslcitationjsonRNDbOMnOYtWjc}
\citealt{Krings1986}) attempted to identify and classify problems via think-aloud protocols, I will focus on unconscious process data, namely keylogging and eyetracking data, to initially determine which parameters might be interesting for predicting translation problems and to then model an approach to find translation problems in translation process data with the help of mere keylogging data.



Another key aspect of this study will be the theoretical concept of \isi{\textit{translation problems}}. While (theoretical) translation studies have already addressed this issue, the resulting assumptions do not necessarily coincide with assumptions, concepts, and models developed in psychology. Therefore, this study will also introduce the insights on \isi{\textit{problem solving}} generated in both fields, what they have in common and how the differences can be resolved.



This study is structured as follows: \sectref{sec:2} provides a brief overview of MT, while \sectref{sec:3} introduces PE. The next chapter (\sectref{sec:4}) explores how MT and PE are perceived in different areas of translation. The concept of \textit{problem solving} is explored from different angles in \sectref{sec:5}. Next, the research question is implemented (\sectref{sec:6}), the data set and the experiment are specified (\sectref{sec:7}), and the questionnaires used in the experiment are assessed (\sectref{sec:8}). The next three chapters examine the translation process data. First, an analysis is conducted on the research behaviour of the participants (\sectref{sec:9}), then eyetracking and keylogging data are compared in regard to the different syntactic quality of the MT output (\sectref{sec:10}), and finally keylogging parameters are analysed to define the extent to which they help in predicting problematic translation units (\sectref{sec:11}). \sectref{sec:12} introduces an approach to identify translation problems according to keylogging data. A summary of the findings is presented in \sectref{sec:13} and the final chapter (\sectref{sec:14}) deals with aspects that could be examined in the future.


