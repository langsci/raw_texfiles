\chapter{Hidden problem indicators}
\label{sec:11}

Chapter \sectref{sec:9} and \sectref{sec:10} presented examples of explicit problem indicators, i.e. research behaviour and the quality of the syntactic MT output. In this chapter, I want to explore further parameters that are included in the \isi{TPR-DB} which can be used as indicators for problems, and which are not as obviously identifiable as problem indicators, such as research instances and the syntactic quality of the MT output. The aim is to find predictors for problems without consulting screen recording or using think-aloud protocols in the experiments, but to identify problems from mere keylogging data. First, I will look at additional parameters that might reveal problem solving activity. Or in other words, parameters that help identify problematic source text units taking into account the entire data set as well as single \isi{Part-of-Speech} (\isi{PoS}) classes as it can be assumed that the values for the parameters vary a lot between individual PoS classes. Then, those parameters will be matched to the production times and \isi{gaze behaviour} of the participants. Finally, I will check how the results of those parameters behave for the most researched words. Version 2.310 of the database will be used as this newer version includes more parameters than the previous versions due to the fact that the range of available, automatically calculated parameters is sometimes expanded with updates to the database.


\section{Discussion of problem identifying parameters}
\label{sec:11:1}

The \isi{TPR-DB} offers many keylogging and eyetracking parameters that are automatically calculated from the raw keylogging and eyetracking data. I already used parameters concerning production times and gaze data, such as \textit{Dur}, \textit{GazeS} or \textit{GazeT}, in previous chapters to analyse keylogging and eyetracking behaviour. In this chapter, I now want to focus on additional parameters that potentially help identify problematic source text units because they mirror the behaviour of the participants. These parameters are further analysed as to whether they are statistically influenced by the tasks, the status and the experience of the participants. The analysis will be conducted on a word level, referring to the source text word.



The first parameter that might be promising for \isi{problem identification} is \isi{\textit{Munit}}, which states how many micro units were necessary to produce the source text unit, i.e. how often a participant worked on the unit (cf. %\label{ref:ZOTEROITEMCSLCITATIONcitationID8YAXx5rjpropertiesformattedCitationCarlSchaefferandBangalore2016plainCitationCarlSchaefferandBangalore2016citationItemsid145urishttpzoteroorgusers1255332itemsFRU2TXJ9urihttpzoteroorgusers1255332itemsFRU2TXJ9itemDataid145typechaptertitleTheCRITTTranslationProcessResearchDatabasecontainertitleNewDirectionsinEmpiricalTranslationProcessResearchpublisherSpringerpublisherplaceHeidelbergNewYorkaopage1354eventplaceHeidelbergNewYorkaoauthorfamilyCarlgivenMichaelfamilySchaeffergivenMoritzfamilyBangaloregivenSrinivasissueddateparts2016schemahttpsgithubcomcitationstylelanguageschemarawmastercslcitationjsonRNDmTSa93QhKq}
\citealt{CarlEtAl2016critt}: 51). It provides no indication whatsoever of how many characters were changed\slash edited, or when the participant worked on the target text unit. A value of 1 would imply in TfS that the unit was produced and not changed in the remaining session; a value of 3 on the other hand represents that the unit was produced and changed twice during the session. A value of 0 is only possible in TfS if the source text unit was not produced in the target text. However, a value of 0 is common in PE and MPE, because this means that the MT output was accepted and left unedited. Hence, \textit{Munit} is usually higher than 1 in TfS, but this is not necessarily always the case. This is also reflected in the high \isi{standard deviation} of the values: TfS – mean: 1.12, sd: 0.68; PE – mean: 0.55, sd: 0.70; MPE – mean: 0.45, sd: 0.67. This shows that once the translation was produced, less changes were made to the translation draft than to the MT output in the PE and MPE tasks. An analysis for a linear mixed model (only including the tasks conducted as well as the status and experience of the participants) shows that \textit{Munit} is influenced by the task and the difference is significant between all three tasks (TfS and PE: $t=\pm47.01$, $p<0.0001$, TfS and MPE: $t=\pm55.28$, $p<0.0001$, PE and MPE: $t=\pm7.87$, $p<0.0001$). The status of the participant does not add to the model as was tested with an ANOVA ($\chi^2(1)=0.57$, $p=0.4508$) and neither does the experience ($\chi^2(1)=1.74$, $p=0.1865$). In this and the following analysis, the individual participants and the different texts are set as random effects in the calculation.



Next, I will concentrate on the parameter \isi{\textit{InEff}}, which “measures the ratio of the number of produced characters divided by the length of the final translation” (%\label{ref:ZOTEROITEMCSLCITATIONcitationIDCcv6lGBzpropertiesformattedCitationCarlSchaefferandBangalore2016plainCitationCarlSchaefferandBangalore2016citationItemsid145urishttpzoteroorgusers1255332itemsFRU2TXJ9urihttpzoteroorgusers1255332itemsFRU2TXJ9itemDataid145typechaptertitleTheCRITTTranslationProcessResearchDatabasecontainertitleNewDirectionsinEmpiricalTranslationProcessResearchpublisherSpringerpublisherplaceHeidelbergNewYorkaopage1354eventplaceHeidelbergNewYorkaoauthorfamilyCarlgivenMichaelfamilySchaeffergivenMoritzfamilyBangaloregivenSrinivasissueddateparts2016schemahttpsgithubcomcitationstylelanguageschemarawmastercslcitationjsonRNDRUVNMVxvkO}
\citealt[26]{CarlEtAl2016critt}). A value of 1 is added to the token length of the final translation to cover the space that usually follows a word. However, sometimes a word is not followed by a space, e.g. at the end of a sentence or when the word is followed by a comma. Nonetheless, the extra value for the space is added automatically to the length of the final translation. Hence, the \textit{InEff} value can become lower than 1 for a word (cf. ibid.). P05, e.g., edited the MT output \textit{schwer} for the source word \textit{hard} to \textit{schwierig}. According to the recordings, (s)he did not delete any letters, but inserted the letter \textit{i} and the syllable \textit{ig}. Three letters were added, the final word has nine letters plus one space, which equals $\frac{3}{10} = 0.3$ for the \textit{InEff} value. When a word was not edited in PE and MPE, it receives a value of 0. Usually, one would expect a value of at least around 1 for all words that were produced in TfS. However, some words also received an \textit{InEff} value of 0 because they were not realised as single words in the target text. P06, e.g., realised the phrase \textit{tend to have} with the \ili{German} \textit{tendenziell}; all final editing effort for this word was mapped on the source word \textit{have}, which received an \textit{InEff} value of 1.83, while \textit{tend} and \textit{to} received a value of 0. In contrast to \textit{Munit}, \textit{InEff} reflects how many characters were edited, but neither whether those edits were made during target text production or in the editing\slash reviewing phase nor how often the participant went back to the unit and changed\slash edited it. The mean values (0.76, sd: 2.67 for all tasks) of \textit{InEff} are again higher for TfS (1.15, sd: 2.97) compared to PE (0.61, sd: 2.15) and MPE (0.50, sd: 2.78), because the target text first needs to be created in TfS. Again, the parameter is influenced by the task in a linear mixed model, but the differences are only significant between TfS and PE ($t=\pm11.65$, $p<0.0001$) as well as TfS and MPE ($t=\pm7.19$, $p<0.0001$), and between PE and MPE ($t=\pm13.95$, $p=0.0277$). Again, the parameter neither statistically differs between students and professionals ($\chi^2(1)=1.22$, $p=0.2687$), nor does the experience vector have an influence ($\chi^2(1)=0.15$, $p=0.6993$).



Two more values that will be considered in the following analysis are \isi{\textit{HTra}} and \isi{\textit{HCross}}. Both parameters refer to the concept of \textit{entropy} – a term coined by Claude E. Shannon for information \isi{entropy}, which describes the uncertainty of the content of a message. The higher the value, the more uncertain is the information in a message. In translation, a high \isi{entropy} value “represents a set of co-activated translation possibilities that are equally good choices for the translation of a source text item” (%\label{ref:ZOTEROITEMCSLCITATIONcitationIDI26MpVctpropertiesformattedCitationBangaloreetal2016plainCitationBangaloreetal2016citationItemsid152urishttpzoteroorgusers1255332itemsVQKM26ZXurihttpzoteroorgusers1255332itemsVQKM26ZXitemDataid152typechaptertitleSyntacticVarianceandPrimingEffectsinTranslationcontainertitleNewDirectionsinEmpiricalTranslationProcessResearchpublisherSpringerpublisherplaceHeidelbergNewYorkDordrechtLondonpage211238eventplaceHeidelbergNewYorkDordrechtLondonauthorfamilyBangaloregivenSrinivasfamilyBehrensgivenBergljotfamilyCarlgivenMichaelfamilyGhankotgivenMaheshwarfamilyHeilmanngivenArndtfamilyNitzkegivenJeanfamilySchaeffergivenMoritzfamilySturmgivenAnnegreteditorfamilyCarlgivenMichaelfamilyBangaloregivenSrinivasfamilySchaeffergivenMoritzissueddateparts2016schemahttpsgithubcomcitationstylelanguageschemarawmastercslcitationjsonRNDBTMBuvFJad}
\citealt{BangaloreEtAl2016}: 213). The more variance between the individual translations, the higher the \isi{entropy} value becomes. However, \isi{entropy} not only represents the amount of different translation possibilities, but also weights them according to their frequency. “[I]t captures the distribution of probabilities for each translation option, so that more likely choices and less likely choices are weighted accordingly.” (ibid.: 214; see also for more information on calculating \isi{entropy} values) The first parameter (\textit{HTra}) provided information about the word translation \isi{entropy} of a source text unit (mean:~1.70, sd:~1.15). Entropy expresses how many different word translations were used within the data set considering how probable one translation choice is, i.e. it “is the sum over all observed word translation probabilities (i.e. expectations) of a given ST word [...] into TT words […] multiplied with their information content.” (ibid.: 31) The higher \textit{HTra}, the higher the \isi{word entropy}, and the more translation variety can be found in the data set for the particular source word. The second parameter (\textit{HCross}) expresses the \isi{entropy} of the word order (mean: 1.47, sd: 0.94). If it is 0, all participants chose the same word position for the word in the target text (cf. %\label{ref:ZOTEROITEMCSLCITATIONcitationIDb4YVa8bNpropertiesformattedCitationCarlAizawaandYamada2016plainCitationCarlAizawaandYamada2016citationItemsid147urishttpzoteroorgusers1255332itemsC89BEN8Purihttpzoteroorgusers1255332itemsC89BEN8PitemDataid147typepaperconferencetitleEnglishtoJapaneseTranslationvsDictationvsPosteditingcontainertitleLrec2016ProceedingsTenthInternationalConferenceonLanguageResourcesandEvaluationpublisherELRApage40244031eventThe10thInternationalConferenceonLanguageResourcesandEvaluationauthorfamilyCarlgivenMichaelfamilyAizawagivenAkikofamilyYamadagivenMasaruissueddateparts2016schemahttpsgithubcomcitationstylelanguageschemarawmastercslcitationjsonRNDRiMRMzt6Rl}
\citealt{CarlEtAl2016japanese}). As the two parameters are not calculated independently of the task and the participants – one value per word was calculated for the whole data set – they will not be analysed statistically themselves in a linear mixed model.


\section{Problematic part-of-speech categories}
\label{sec:11:2}

In the following chapters, the parameters will be analysed according to \isi{PoS} class to identify if certain \isi{PoS} classes cause more effort than others. Punctuation marks were excluded from the analysis because they are not interesting for the research. In \sectref{sec:11:3}, the parameters and \isi{PoS} classes will then be related to production times and eyetracking data.


\subsection{Indications in \textit{Munit}}
\label{sec:11:2:1}

This chapter will analyse the first indicators of problematic word classes according to their mean values of the parameters introduced in \sectref{sec:11:1} (\textit{Munit}, \textit{InEff}, \textit{HTra}, and \textit{HCross}). An explanation of all part-of-speech abbreviations and an overview about all mean and sd values can be found in Appendix~\ref{sec:Appendix:D}, Tables~\ref{tab:D:1} and \ref{tab:D:2}. 



First, I will consider \isi{\textit{Munit}}, which indicates how often a word was visited during one session. As can be seen in \figref{fig:11:1} (the concrete numbers can be found in \tabref{tab:D:2} in Appendix~\ref{sec:Appendix:D}), the parameter is the highest in TfS for all \isi{PoS} classes. This is expected because, in contrast to the PE tasks, the target text first has to be produced. Hence, \textit{Munit} is usually higher than 1 and for this reason \figref{fig:11:2} shows the same data but with \textit{Munit} \textminus1 for TfS to present the mere editing effort.


\begin{figure}
\caption{\textit{Munit} according to PoS class and task}
\label{fig:11:1}
% % \includegraphics[width=\textwidth]{figures/DissertationNitzkeberarbeitet-img32.jpg}
\resizebox{\textwidth}{!}{\begin{tikzpicture}[trim axis right,trim axis left]
\pgfplotstableread[col sep=tab]{data/Fig11.1.csv}{\table}
    \pgfplotstablegetcolsof{\table}
    \pgfmathtruncatemacro\numberofcols{\pgfplotsretval-1}
        \begin{axis}[
                    ybar,
                    xtick=data,
                    axis lines*=left,
%                     nodes near coords,
                    ymin=0,
                    xticklabels from table={\table}{P},
                    bar width=2pt,
                    width=20cm,
                    height=5cm,
                    xticklabel style={rotate=90, anchor=east},
                    ticklabel style={font=\footnotesize},
                    enlarge x limits={0.05},
                    colormap/Accent,
                    cycle list/Accent,
                    legend style={at={(1.0,0.8)},anchor=east},
                    reverse legend,
                    legend style={font=\footnotesize},
                    ]
            \foreach \i in {1,...,\numberofcols} {
                \addplot+[
                    /pgf/number format/read comma as period, fill,draw=none
                    ] table [x index={1},y index={\i},x expr=\coordindex] {\table};
                \pgfplotstablegetcolumnnamebyindex{\i}\of{\table}\to{\colname} % Adding column headers to legend
                \addlegendentryexpanded{\colname}
            }   
            \end{axis}    
\end{tikzpicture}}
\end{figure}

 

\figref{fig:11:2} provides a very different impression than \figref{fig:11:1}. Here, \textit{Munit} is the highest for PE in most cases, except for RBR and WDT, for which \textit{Munit} is higher in MPE. However, as both word classes occured less than five times in all texts, this can be considered an exception\footnote{\tabref{tab:D:1} in Appendix~\ref{sec:Appendix:D} presents how often which \isi{PoS} class occurred in the total data set (counting each participant) and how often they occurred in which source text. The categories JJR, JJS, POS, RBR, RBS, RP, WDT, WP, and WRB occurred less than five times. For this reason, some of them will be summarised or grouped with another category in the upcoming chapters.
}.
The MT system translated the superlative \textit{the most vulnerable (countries)} with \textit{am meisten gefährdeten (Länder)} (literal translation: \textit{the most endangered (countries)}), which can be considered a lexical mistranslation rather than a problematic translation of the grammatical form. In other words, the problem is not the \isi{PoS} class itself, but the lexical item. Further, some \isi{PoS} categories become negative for TfS when one is subtracted from the initial value, which indicates that some words or phrases were not realised as such in the target text, e.g. comparative adjectives (JJR). P03, for example, translated the phrase \textit{[...] can support population densities 60 to 100 times greater than [...]} as \textit{kann die 60- bis 100-fache Bevölkerungsdichte ermöglichen} (literal translation: \textit{can the 60 to 100 times population density enable}). The comparison that needs the comparative adjective (JJR) in this sentence is paraphrased and hence a comparative adjective is not necessary in the target text any more.

\begin{figure}
\caption{Munit \textminus1 for TfS according to PoS class and task}
\label{fig:11:2}
% % \includegraphics[width=\textwidth]{figures/DissertationNitzkeberarbeitet-img33.jpg}
\resizebox{\textwidth}{!}{\begin{tikzpicture}[trim axis right,trim axis left]
\pgfplotstableread[col sep=tab]{data/Fig11.2.csv}{\table}
    \pgfplotstablegetcolsof{\table}
    \pgfmathtruncatemacro\numberofcols{\pgfplotsretval-1}
        \begin{axis}[
                    ybar,
                    xtick=data,
                    axis lines*=left,
%                     nodes near coords,
                    xticklabels from table={\table}{P},
                    bar width=2pt,
                    width=20cm,
                    height=5cm,
                    xticklabel style={rotate=90, anchor=east},
                    ticklabel style={font=\footnotesize},
                    enlarge x limits={0.05},
                    colormap/Accent,
                    cycle list/Accent,
                    legend style={at={(1.0,0.8)},anchor=east},
                    reverse legend,
                    legend style={font=\footnotesize},
                    ]
            \foreach \i in {1,...,\numberofcols} {
                \addplot+[
                    /pgf/number format/read comma as period, fill,draw=none
                    ] table [x index={1},y index={\i},x expr=\coordindex] {\table};
                \pgfplotstablegetcolumnnamebyindex{\i}\of{\table}\to{\colname} % Adding column headers to legend
                \addlegendentryexpanded{\colname}
            }   
            \end{axis}    
\end{tikzpicture}}
\end{figure}

 


Conclusively, I can see that more changes are made in the PE and MPE task when I compare \figref{fig:11:1} and \figref{fig:11:2}. If I assume that the first translation can also be considered as the first draft that is improved in the revision phase, this first translation draft is much more reliable than the MT output, as far fewer changes are made in the TfS task. This is one additional argument why PE cannot simply be compared to reviewing or proof-reading tasks as it requires many more changes.



When I look at the results in \figref{fig:11:2}, \textit{Munit} is the highest for particles (PR) in all three tasks. This \isi{PoS} class, however, only occurred once in Text~2 and is furthermore a part of a verb construction (\textit{to cough up}). Hence, I cannot conclude that this word class is especially hard to process without any further testing. The \textit{Munit} value is very similar for all other word categories in TfS ranging between \textminus0.17 and 0.38, except for the \textit{wh}{}-adverbs (WRB), which have a negative value of \textminus0.75, but also only occurred once in Text~5. Another eye-catching result in the TfS data is the negative value for VBP (non-3. person singular present verbs). When looking into the data, the reason for this seems to be primarily that this \isi{PoS} class was sometimes realised simultaneously with other classes and, therefore, the production and editing effort was mapped on the other involved \isi{PoS} classes. In contrast to the TfS task, striking differences are visible between PE and MPE. These refer to modal words (MD), nouns (NN and NNS) with the exception of proper nouns (NNP), verbs (VB, VBG, VBN, VBP, VBZ), the preposition \textit{to}, which is in most cases part of a verb construction, too, and finally \textit{wh}{}-determiners and -pronouns. The latter two categories can be disregarded in this analysis as well because they occur only once in Text~2 and in Text~5. Therefore, a larger data set would be necessary to confirm those impressions.



It was tested whether the tasks influence the outcome of \textit{Munit}. Further, it was tested whether status and\slash or experience of the participants contribute to the model, i.e. if they also statistically influence the parameter. The ANOVA test did not suggest that status and\slash or experience add to the model. If only status or experience add to the model, the values of the regression are indicated for the respective parameter. If both potentially add to the model, but only separately and not when they are both integrated in the model, the one that contributes the most is chosen. Finally, the individual participants and the single texts were set as random effects. The analysis was not possible for five \isi{PoS} categories (RBR, RBS, RP, WDT, WRB) because there were not enough data points to be analysed. When the \textit{Munit} value was chosen for which the production of the word is not included (\textit{\isi{Munit}\textminus1}), \textit{Munit} was always significantly higher for PE than for TfS, except for superlative adjectives (JJS). Similarly, the parameter was always higher for MPE than for TfS, except for superlative adjectives (JJS) and for possessive endings (POS). These exceptions might be caused by the few occurrences in the texts or by insufficient quality of the MT output. Further, the difference is significant between PE and MPE for prepositions and subordinate conjunctions (IN), adjectives (JJ), singular or mass nouns (NN), plural nouns (NNS), adverbs (RB), the preposition \textit{to} (TO) and past participle verbs (VBN). These \isi{PoS} classes were modified more often in PE than in MPE. The reasons can either be that the necessary changes were not detectable without the source text or that they were overedited in the PE task. It is hard to explain why the latter would happen in PE. One would expect this phenomenon in MPE as the participants might feel insecure about the meaning of the \isi{translation unit} and hence edit it more often. On the other hand, it is conceivable that the participants improve MT units that would not necessarily need editing because they know the source text and hence are not satisfied with an acceptable target text unit, but rather choose a better translation equivalent.



Conclusively, the \isi{PoS} categories that show statistically different values for \textit{Munit} might indicate problematic MT output that requires changes, but these changes may not be that clearly related to the individual \isi{PoS}. For the categories IN (preposition / subordinate conjunction) and TO (\textit{to}), it is also possible that they themselves are not that problematic, but that they  introduce problematic text units and hence, must be changed as part of the overall \isi{translation unit}. The status of the participants has a significant influence only on adjectives (JJ) and singular proper nouns (NNP),\footnote{Plural proper nouns did not occur in the data set.} the experience coefficient has no statistically significant impact.



The tests were also conducted for the initial \textit{Munit} value, i.e. the parameter originally found in the table, where the production of the translation in TfS is still reflected in the value (remember that I subtracted a value of 1 from \textit{Munit} to exclude the different approaches to the task) to verify that no patterns are missed. As was expected, the values were all statistically significant when comparing TfS and PE as well as TfS and MPE, even for superlative adjectives (JJS). The only exception is the value for \textit{wh}{}-pronouns (WP), where there is no statistical significant difference between TfS and PE. The difference is that \textit{Munit} is statistically higher for TfS in these analyses, while it was statistically smaller in the initial test. As the \textit{Munit} values for PE and MPE are not changed, the results of the statistical tests obviously do not change, either. Further, the status of the participants again has a significant influence on adjectives (JJ) and singular proper nouns (NNP). It is not surprising that the status of the participant is such a negligible indicator in the linear regression models for the single \isi{PoS} categories and that the experience vector has none at all. Especially when I consider that for the overall data, the experience vector had no significant influence as well as the status of the participant. This might point in the direction that not enough data have been gathered in this experiment to distinguish the influence of status or the experience of the participants. Another reason might be that this influence simply cannot be measured on a word class level.


\largerpage
In summary, the \textit{Munit} value is influenced by the different tasks, but usually not by the characteristics of the participants' level of professionalism. While no particular \isi{PoS} class of those I could analyse seems to increase the \textit{Munit} value in the TfS task, the participants had to change the MT output more often for modal words, nouns and verbs than for other \isi{PoS} classes.


\subsection{Indications in \textit{InEff}}
\label{sec:11:2:2}

The parameter \isi{\textit{InEff}} is also influenced by the task because a target text has to first be produced in TfS, while there is already the MT output in PE and MPE. Therefore, similarly as for \textit{\isi{Munit},} I will take the raw values for \textit{InEff} into account (\figref{fig:11:3}) and will then again subtract the value 1 for TfS (\figref{fig:11:4}).


\begin{figure}
\caption{\textit{InEff} for TfS according to PoS class and task}
\label{fig:11:3}
% % \includegraphics[width=\textwidth]{figures/DissertationNitzkeberarbeitet-img34.jpg}
\resizebox{\textwidth}{!}{\begin{tikzpicture}[trim axis right,trim axis left]
\pgfplotstableread[col sep=tab]{data/Fig11.3.csv}{\table}
    \pgfplotstablegetcolsof{\table}
    \pgfmathtruncatemacro\numberofcols{\pgfplotsretval-1}
        \begin{axis}[
                    ybar,
                    xtick=data,
                    axis lines*=left,
%                     nodes near coords,
                    ymin=0,
                    xticklabels from table={\table}{P},
                    bar width=2pt,
                    width=20cm,
                    height=5cm,
                    xticklabel style={rotate=90, anchor=east},
                    enlarge x limits={0.05},
                    colormap/Accent,
                    cycle list/Accent,
                    legend style={at={(1.0,0.8)},anchor=east},
                    reverse legend,
                    ticklabel style={font=\footnotesize},
                    legend style={font=\footnotesize},
                    ]
            \foreach \i in {1,...,\numberofcols} {
                \addplot+[
                    /pgf/number format/read comma as period, fill,draw=none
                    ] table [x index={1},y index={\i},x expr=\coordindex] {\table};
                \pgfplotstablegetcolumnnamebyindex{\i}\of{\table}\to{\colname} % Adding column headers to legend
                \addlegendentryexpanded{\colname}
            }   
            \end{axis}    
\end{tikzpicture}}
\end{figure}

 


The values in \figref{fig:11:3} are usually the highest for TfS, as was expected. However, the values are very similar between TfS and PE or even higher in PE for personal pronouns (PRP), particles (RP), some verb categories (VBG, VBN, and VBP), and \textit{wh}{}-pronouns (WP). However, the occurrences of RPs and WPs is very low and can therefore be disregarded as more data would be necessary to make reasonable assumptions.


\begin{figure}
\caption{\textit{InEff} \textminus1 for TfS according to PoS class and task}
\label{fig:11:4}
% % \includegraphics[width=\textwidth]{figures/DissertationNitzkeberarbeitet-img35.jpg}
\resizebox{\textwidth}{!}{\begin{tikzpicture}[trim axis right,trim axis left]
\pgfplotstableread[col sep=tab]{data/Fig11.4.csv}{\table}
    \pgfplotstablegetcolsof{\table}
    \pgfmathtruncatemacro\numberofcols{\pgfplotsretval-1}
        \begin{axis}[
                    ybar,
                    xtick=data,
                    axis lines*=left,
%                     nodes near coords,
                    xticklabels from table={\table}{P},
                    bar width=2pt,
                    width=20cm,
                    height=5cm,
                    xticklabel style={rotate=90, anchor=east},
                    enlarge x limits={0.05},
                    colormap/Accent,
                    cycle list/Accent,
                    legend style={at={(1.0,0.8)},anchor=east},
                    reverse legend,
                    ticklabel style={font=\footnotesize},
                    legend style={font=\footnotesize},
                    ]
            \foreach \i in {1,...,\numberofcols} {
                \addplot+[
                    /pgf/number format/read comma as period, fill,draw=none
                    ] table [x index={1},y index={\i},x expr=\coordindex] {\table};
                \pgfplotstablegetcolumnnamebyindex{\i}\of{\table}\to{\colname} % Adding column headers to legend
                \addlegendentryexpanded{\colname}
            }   
            \end{axis}    
\end{tikzpicture}}
\end{figure}

 
\largerpage

As I can see in \figref{fig:11:4}, some \textit{InEff} values drop below 0 when the initial production value of 1 is subtracted in TfS. This may indicate two things: First, it is possible that the \isi{PoS} categories are sometimes not realised in the target text. Second, it indicates that for those \isi{PoS} categories, most translators decided to retain their initial idea of the target word. If I assume that an \textit{\isi{InEff} \textminus1} value higher than 0.2\footnote{Assuming that this would represent a correction of one letter (one deletion activity and one extra insertion activity) in a ten-letter word: $\frac{12}{10}-1=0.2$.} expresses high \isi{production effort}, this applies to only nine \isi{PoS} categories (CC, CD, JJS, NNS, POS, PRP, PRP\$, WDT, WP). Further, the \textit{\isi{InEff} \textminus1} value is still the highest for TfS in coordinating conjunctions (CC), superlative adjectives (JJS), possessive endings (POS), and \textit{wh}{}-determiners (WDT).



The figures also show that the \textit{InEff} values are usually very close between PE and MPE with some exceptions (CD, JJR, JJS, NNS, PRP, RBR, RP, VB, VBD, VBG, VBN, WDT, and WP), which are then usually higher for PE, except for CD, RBR, and WDT – the latter two, however, occur seldom in the data. When the values are higher for PE, however, this might indicate that the MT output contained mistakes that could not be corrected without the help of the source text.



In the following, I will again test whether the \isi{task} as well as the \isi{status} and\slash or experience of the participants has a significant influence on the parameter \textit{InEff} with linear mixed models. As was done for \textit{Munit}, the individual participants and the texts are taken as random effects. A dash "–" again marks when there was no significant result, or the \isi{ANOVA} test did not suggest that status and\slash or experience add to the model. In this analysis, I will examine both result tables, one for the regular \textit{InEff} value and one for \textit{\isi{InEff} \textminus1} because \figref{fig:11:3} and \figref{fig:11:4} suggest that the the differences between TfS and PE\slash MPE are not that obviously influenced by the task as for \textit{Munit}. With some exceptions, most data show statistically significant results when comparing TfS with PE and MPE. As can be seen in \tabref{tab:D:2} in Appendix~\ref{sec:Appendix:D}, no significant differences can be found at all for possessive markers (POS), personal pronouns (PRP), and non-3. person singular present verbs (VBP) for \textit{InEff}. Furthermore, the difference is not significant for base form verbs (VB), past participle verbs (VBN), and \textit{wh}{}-pronouns (WP) when comparing TfS and PE, nor for cardinal numbers (CD) when comparing TfS and MPE. When the differences are significant, they are significantly higher for TfS with the exception of particles (RP), which, as mentioned above, only occurred once in one text and hence no reasonable conclusions can be drawn from this. The difference between PE and MPE data is only significant for plural nouns (NNS), base form verbs (VB), and gerund verbs (VBG), which are all higher for PE.



The differences between PE and MPE obviously do not change when I subtract 1 from the \textit{InEff} value for TfS. Although most differences are still significant, but then statistically higher in PE and\slash or MPE, some changes become visible when comparing TfS and PE\slash MPE. The tests show no significant difference at all for coordinating conjunctions (CC), cardinal numbers (CD), superlative adjectives (JJS), possessive endings (POS), possessive pronouns (PRP\$), and \textit{wh}{}-pronouns (WP). This shows that \isi{post-editing} the MT output takes similar typing effort for those \isi{PoS} classes as it does when translating the word from scratch. The differences are only significant for TfS and PE in personal pronouns (PRP) and comparative adjectives (JJR), but not between TfS and MPE.



Status and experience of the participants did not play a significant role in all tests. Hence, the efficiency (or inefficiency) on a word class level of the participants may not be dependent on their professional experience for the same reasons mentioned in \sectref{sec:11:2:1} on the parameter \textit{Munit}.


\subsection{Indications in \textit{HTra} and \textit{HCross}}
\label{sec:11:2:3}

As was already mentioned in \sectref{sec:11:1}, \isi{\textit{HTra}} and \isi{\textit{HCross}} were calculated across the different tasks, including all tasks into the calculations. The values were calculated per source text item, including all sessions, independent of the tasks. Hence, I will first look at the mean values of the parameters (see \figref{fig:11:5}, \figref{fig:11:6}, and also \tabref{tab:D:2} in Appendix~\ref{sec:Appendix:D}).



I assume that higher \textit{HTra} values might indicate longer production and processing times because the participant has more \isi{translation choices} in his\slash her mental lexicon. Similarly, %\label{ref:ZOTEROITEMCSLCITATIONcitationIDN27KTVvLpropertiesformattedCitationSchaefferetal2016plainCitationSchaefferetal2016citationItemsid79urishttpzoteroorgusers1255332itemsQJKG2242urihttpzoteroorgusers1255332itemsQJKG2242itemDataid79typechaptertitleWordtranslationentropyEvidenceofearlytargetlanguageactivationduringreadingfortranslationcontainertitleNewDirectionsinEmpiricalTranslationProcessResearchpublisherSpringerpublisherplaceBerlinpage183210eventplaceBerlinauthorfamilySchaeffergivenMoritzfamilyDragstedgivenBarbarafamilyHvelplundgivenKristianTangsgaardfamilyBallinggivenLauraWintherfamilyCarlgivenMichaelissueddateparts2016schemahttpsgithubcomcitationstylelanguageschemarawmastercslcitationjsonRND0Q0UDU0NEE}
\citet{SchaefferEtAl2016} hypothesise that the more literal units (identical word order, translation items correspond one-to-one, only one translation possible for the ST item) are translated, the easier they are to process and the stronger the priming effects. They therefore analyse whether \textit{Cross} (the absolute numbers with which \textit{HCross} is calculated) and \textit{HTra} have an influence on early \isi{gaze behaviour}, which would also support that reading for translation differs from reading for understanding. The results show that \textit{Cross} and \textit{HTra} have a positive significant influence on the first \isi{fixation duration}, and \textit{Cross} also on first pass \isi{gaze duration} (ibid.). Accordingly, it seems plausible that higher \textit{HTra} values might also indicate problem solving activity. On the other hand, a broad range of possible translations might also just be part of solving a task – deciding on one translation choice rather than solving a problem. In summary, I will include \textit{HTra} in the analysis and will then decide whether it is an indicator for problem solving activity or not according to the results.


\begin{figure}
\caption{Mean values of \textit{HTra} according to PoS class}
\label{fig:11:5}
% % \includegraphics[width=\textwidth]{figures/DissertationNitzkeberarbeitet-img36.jpg}
\resizebox{\textwidth}{!}{\begin{tikzpicture}[trim axis right,trim axis left]
\pgfplotstableread[col sep=tab]{data/Fig11.5.csv}{\table}
    \pgfplotstablegetcolsof{\table}
    \pgfmathtruncatemacro\numberofcols{\pgfplotsretval-1}
        \begin{axis}[
                    ybar,
                    xtick=data,
                    axis lines*=left,
%                     nodes near coords,
                    ymin=0,
                    xticklabels from table={\table}{P},
                    bar width=5pt,
                    width=20cm,
                    height=5cm,
                    xticklabel style={rotate=90, anchor=east},
                    enlarge x limits={0.05},
                    colormap/Greys-9,
                    cycle list/Greys-9,
                    reverse legend,
                    ticklabel style={font=\footnotesize},
                    legend style={font=\footnotesize},
                    ]
            \foreach \i in {1,...,\numberofcols} {
                \addplot+[
                    /pgf/number format/read comma as period, fill=Greys-H,draw=none
                    ] table [x index={1},y index={\i},x expr=\coordindex] {\table};
                \pgfplotstablegetcolumnnamebyindex{\i}\of{\table}\to{\colname} % Adding column headers to legend
                \addlegendentryexpanded{\colname}
            }   
            \legend{}
            \end{axis}    
\end{tikzpicture}}
\end{figure}

 


As I can see in \figref{fig:11:5}, the \textit{HTra} values differ a lot regarding the \isi{PoS} class. While I have only one word class with no \isi{entropy} (comparative adverbs - RBR), superlative adverbs (RBS), gerund verbs (VBG), and past participle verbs (VBN) are on average higher than 2.5 and particles (RP) even have an average \textit{HTra} value of more than 3. An obvious question arises: Why is there no \isi{entropy} for comparative adverbs and a very high \isi{entropy} for superlative adverbs? The only (and convincing) explanation is that both \isi{PoS} classes only occurred once in the texts and hence no overall conclusions can be drawn. The same accounts for particles which only occur once in the texts, too. Most word classes range between 1.5 and 2.5 (DT, IN, JJ, JJR, MD, NN, NNS, POS, PRP\$, RB, TO, VB, VBD, VBN, VBP, VBZ, WDT, WP, WRB) and only coordinating conjunctions (CC), cardinal numbers (CD), superlative adjectives (JJS), proper nouns (NNP), and personal pronouns (PRP) are below 1.5. With the latter word classes, one might wonder how cardinal numbers (CD) and proper nouns (NNP) even received an \isi{entropy} value above 1 at all. While the latter still seems plausible, because one can imagine that proper nouns might be translated in different ways as they have to be explicated or simplified, or have different grammatical properties in \ili{German}, different translations for cardinal numbers cannot be explained easily. The data set was consulted to find examples. The phrase \textit{Beijing Olympics} in text 3 - both words tagged as NNPs – was often translated as \textit{Olympischen Spielen in Peking} (\textit{Olympic Games in Beijing}; the phrase is grammatically adjusted to the sentence), where \textit{Olympischen Spielen} was tagged on \textit{Olympics} and \textit{in Peking} on \textit{Beijing}. One participant, however, decided to translate the unit as \textit{Olympischen Sommerspielen in Peking} (Olympic Summer Games in Beijing), which is also a valid translation. When I look at the data for cardinal numbers, some variation appears as well. The phrase \textit{four life sentences} (\textit{four} being tagged as CD) in Text~1 cannot be translated literally into \ili{German}, but was translated as \textit{vierfach lebenslänglich}, \textit{viermal lebenslänglich}, \textit{vier Mal lebenslänglich} etc., which all basically express the same meaning. Finally, it is striking that the \textit{HTra} values for verb categories in general seem higher than for other word classes, which might indicate that verbs naturally occur with many different translation choices (and therefore might be more difficult to translate).


\begin{figure}
\caption{Mean values of \textit{HCross} according to PoS class}
\label{fig:11:6}
% % \includegraphics[width=\textwidth]{figures/DissertationNitzkeberarbeitet-img37.jpg}
\resizebox{\textwidth}{!}{\begin{tikzpicture}[trim axis right,trim axis left]
\pgfplotstableread[col sep=tab]{data/Fig11.6.csv}{\table}
    \pgfplotstablegetcolsof{\table}
    \pgfmathtruncatemacro\numberofcols{\pgfplotsretval-1}
        \begin{axis}[
                    ybar,
                    xtick=data,
                    axis lines*=left,
%                     nodes near coords,
                    ymin=0,
                    xticklabels from table={\table}{P},
                    bar width=5pt,
                    width=20cm,
                    height=5cm,
                    xticklabel style={rotate=90, anchor=east},
                    enlarge x limits={0.05},
                    colormap/Greys-9,
                    cycle list/Greys-9,
                    reverse legend,
                    ticklabel style={font=\footnotesize},
                    legend style={font=\footnotesize},
                    ]
            \foreach \i in {1,...,\numberofcols} {
                \addplot+[
                    /pgf/number format/read comma as period, fill=Greys-H,draw=none
                    ] table [x index={1},y index={\i},x expr=\coordindex] {\table};
                \pgfplotstablegetcolumnnamebyindex{\i}\of{\table}\to{\colname} % Adding column headers to legend
                \addlegendentryexpanded{\colname}
            }   
            \legend{} % no need for this in one var plot
            \end{axis}    
\end{tikzpicture}}
\end{figure}

 


In general, \textit{HCross} values seem to be relatively high (see \figref{fig:11:6}), which indicates that translations from \ili{English} into \ili{German} require a lot of restructuring. On the other hand, repositioning one word in a sentence also affects the \textit{Cross} value of the remaining words in the sentence. Except for JJS, NNP, RBR, and WDT the values for mean \textit{HCross} are higher than 1. These exceptions, however, can again be explained by the low occurrences in the overall data set. The value is more than 2 for four of the \isi{PoS} classes. It is striking that of those four, three are verb categories, which reflects the different position of the verb in subordinate clauses in \ili{English} and \ili{German} and additionally often seems to be problematic for statistical MT. \textit{HCross} will be integrated in the problem analysis because it is possible that it predicts problematic units. However, it also seems plausible that structural changes can only be categorised as part of the translation task rather than as a translation problem as was already indicated in \sectref{sec:9}.


\section{Influence of problem indicators on keylogging and eyetracking data}
\label{sec:11:3}

\largerpage

In the following analysis, some \isi{PoS} categories are condensed because they hardly appeared in the texts, as was mentioned in the preceding sections. Adjectives (JJ) now also include comparative adjectives (JJR) and superlative adjectives (JJS). The same applies for adverbs (RB), which now also comprise comparative (RBR) and superlative adverbs (RBS). Finally, \textit{wh}{}-determiners (WDT), \textit{wh}{}-pronouns (WP), and \textit{wh}{}-adverbs (WRB) are summarised under the abbreviation WDT. Further, two additional categories are introduced that comprise all nouns (NN + NNP + NNS) and all verbs (VB + VBD + VBG + VBN + VBP + VBZ) to gain more insights into the overall results of these word classes. Linear mixed models were calculated to evaluate the influence of the parameters on the production and processing data, namely production duration, \isi{total fixation duration} and total fixation count on the source and target text separately and combined. Two different approaches were tested in R to find the best fitting models, both including the individual participants and the individual texts as random effects. Finally, the models were created to include all significant parameters and exclude those that do not have a significant influence on the behavioural parameters, and following this formula (if all parameters were included; the example uses \textit{Dur} as the parameter to be predicted):


\ea
\ttfamily summary(lmer(Dur {\textasciitilde} Task + Munit + InEff + HTra + HCross + (1{\textbar}Part) + (1{\textbar}Text), data = dataset))
\z


Colour patterns were created (\figref{fig:11:7}-\figref{fig:11:11}\footnote{T = translation from scratch; P = (bilingual) \isi{post-editing}; M = monolingual post-editing}) that visualise the statistic results. When a square is green, the predictor influences the behavioural parameter significantly, when a square is grey, no significant influence could be detected. When no test could be conducted because of the missing data for MPE, the squares are hatched. First, the results are discussed for the overall data set, independent of \isi{PoS classes} (see pattern in \figref{fig:11:7}). \textit{Dur} is influenced by all parameters, but there is no significant difference between MPE and PE. Gaze behaviour on the source text (\textit{GazeS} and \textit{FixS}) is influenced by the tasks and \textit{HCross}. \isi{Fixation} duration on the target text (\textit{GazeT}) can be predicted by \textit{Munit}, \textit{InEff}, \textit{HTra} and \textit{HCross}, while fixation counts (\textit{FixT}) on the target text are only influenced by \textit{Munit} and \textit{InEff}. It seems reasonable that the tasks do not influence the \isi{gaze behaviour} on the target text because previous chapters have already shown that the \isi{eye movement} behaviour on the source text is similar in TfS and PE. Finally, \isi{total fixation duration} and fixation count on the whole text is influenced by the tasks, \textit{Munit} and \textit{HCross} – for \textit{TGaze} the difference is only significant between both PE tasks and TfS, while for \textit{TFix} it is only significant between MPE and TfS as well as between the PE tasks. Conclusively, \textit{HCross}, \textit{Munit}, and \textit{Task} are the most promising indicators for cognitive effort. \textit{HCross}, however, is negatively directed, meaning that the higher \textit{HCross}, the shorter the production time becomes, which is surprising and contrary to what would be expected. Further, if the task plays an influential role, the production time becomes higher in TfS than in the PE tasks. In \textit{TFix}, the production time is higher in PE than in MPE.


\begin{figure}
\caption{Pattern for statistical influence of predictors on parameters for all data}
\label{fig:11:7}
% % \includegraphics[width=\textwidth]{figures/DissertationNitzkeberarbeitet-img38.jpg}
    \resizebox{.4\textwidth}{!}{\begin{tikzpicture}[
        nodes={minimum height=20pt,minimum width=35pt, inner sep=0pt,outer sep=0pt},
        ]
     \pgfsetmatrixcolumnsep{0mm}
     \pgfsetmatrixrowsep{0mm}
     \matrix [matrix of nodes,nodes in empty cells,
        row 2 column 2/.style={nodes={fill=lsMidGreen}},row 2 column 3/.style={nodes={fill=lsMidGreen}},row 2 column 4/.style={nodes={fill=lsLightGray}},row 2 column 5/.style={nodes={fill=lsMidGreen}},row 2 column 6/.style={nodes={fill=lsMidGreen}},row 2 column 7/.style={nodes={fill=lsMidGreen}},row 2 column 8/.style={nodes={fill=lsMidGreen}},
        row 3 column 2/.style={nodes={fill=lsMidGreen}},row 3 column 3/.style={nodes={pattern=north east lines}},row 3 column 4/.style={nodes={pattern=north east lines}},row 3 column 5/.style={nodes={fill=lsLightGray}},row 3 column 6/.style={nodes={fill=lsLightGray}},row 3 column 7/.style={nodes={fill=lsLightGray}},row 3 column 8/.style={nodes={fill=lsMidGreen}},
        row 4 column 2/.style={nodes={fill=lsMidGreen}},row 4 column 3/.style={nodes={pattern=north east lines}},row 4 column 4/.style={nodes={pattern=north east lines}},row 4 column 5/.style={nodes={fill=lsLightGray}},row 4 column 6/.style={nodes={fill=lsLightGray}},row 4 column 7/.style={nodes={fill=lsLightGray}},row 4 column 8/.style={nodes={fill=lsMidGreen}},
        row 5 column 2/.style={nodes={fill=lsLightGray}},row 5 column 3/.style={nodes={pattern=north east lines}},row 5 column 4/.style={nodes={pattern=north east lines}},row 5 column 5/.style={nodes={fill=lsMidGreen}},row 5 column 6/.style={nodes={fill=lsMidGreen}},row 5 column 7/.style={nodes={fill=lsMidGreen}},row 5 column 8/.style={nodes={fill=lsMidGreen}},
        row 6 column 2/.style={nodes={fill=lsLightGray}},row 6 column 3/.style={nodes={pattern=north east lines}},row 6 column 4/.style={nodes={pattern=north east lines}},row 6 column 5/.style={nodes={fill=lsMidGreen}},row 6 column 6/.style={nodes={fill=lsMidGreen}},row 6 column 7/.style={nodes={fill=lsLightGray}},row 6 column 8/.style={nodes={fill=lsLightGray}},
        row 7 column 2/.style={nodes={fill=lsMidGreen}},row 7 column 3/.style={nodes={fill=lsMidGreen}},row 7 column 4/.style={nodes={fill=lsLightGray}},row 7 column 5/.style={nodes={fill=lsMidGreen}},row 7 column 6/.style={nodes={fill=lsLightGray}},row 7 column 7/.style={nodes={fill=lsLightGray}},row 7 column 8/.style={nodes={fill=lsMidGreen}},
        row 8 column 2/.style={nodes={fill=lsLightGray}},row 8 column 3/.style={nodes={fill=lsMidGreen}},row 8 column 4/.style={nodes={fill=lsMidGreen}},row 8 column 5/.style={nodes={fill=lsMidGreen}},row 8 column 6/.style={nodes={fill=lsLightGray}},row 8 column 7/.style={nodes={fill=lsLightGray}},row 8 column 8/.style={nodes={fill=lsMidGreen}},
         ] {\nitzkematrix};
\end{tikzpicture}}
\end{figure}

 


Coordinating conjunctions (CC) are only influenced by \textit{Task} (all but \isi{eye movement} on the target text), \textit{Munit} (all but \isi{eye movement} on the source text and \isi{total fixation duration} on the target text) and \textit{InEff} (except on \textit{GazeS}), while \textit{HTra} and \textit{HCross} do not predict production time or \isi{gaze behaviour} (see patterns for CC, CD, DT, IN, JJ, and MD in \figref{fig:11:8}). Cardinal numbers (CD) are seldom influenced by the tasks (only in \textit{FixS}, \textit{TGaze}, and \textit{TFix}) or by \textit{Munit} (only in \textit{Dur} and \textit{FixT}). Further, \textit{HCross} had to be excluded from the model, although it often also became significant because the correlation between \textit{HTra} and \textit{HCross} is too strong for this \isi{PoS} category.\footnote{Which was tested for all significant parameters with the vif.mer() function, see \url{https://github.com/aufrank/R-hacks/blob/master/mer-utils.R}, last accessed 11 March 2017.} \textit{InEff} – except for \textit{FixS} – and \textit{HTra} – except for \textit{Dur} – are very predictive. The cognitive effort for determiners (DT) can hardly be predicted by the examined parameters. Only \textit{Dur} is influenced by \textit{Munit} and \textit{HTra} and the \isi{gaze behaviour} on the source text is influenced by \textit{HCross} (with a negative direction). Production duration and \isi{gaze behaviour} on both source and target text can be predicted by \textit{Task, InEff} (except gaze on the source text and \textit{FixT}), and \textit{HTra} (except \textit{Dur} and \textit{FixS}) for prepositions and subordinate conjunctions (IN). \textit{Munit} only played a role in \textit{Dur}, and \textit{HCross} only in \textit{FixS} (though, \textit{HCross} was the only foretelling parameter for \textit{FixS}). Adjectives (JJ) hardly show a consistent pattern. They are influenced by the tasks, especially concerning the difference between PE and TfS, but not when considering the gaze data on the target text. \textit{Dur}, \textit{GazeT}, and \textit{TFix} are influenced by \textit{Munit}, \textit{Dur} and \textit{FixT} by \textit{InEff}, gaze on the target text and on both texts combined is impacted by \textit{HTra}, and, finally, \textit{GazeS} is additionally influenced by \textit{HCross}. The production time of modals (MD) is influenced by \textit{Munit} and \textit{InEff}, the gaze data on the source text by \textit{Task}, \textit{FixT} by \textit{Munit}, and \textit{GazeT}, \textit{TGaze} and \textit{TFix} by \textit{InEff}.

\largerpage
When all noun categories are considered, the production time is influenced by \textit{Munit}, \textit{InEff}, and \textit{HTra} and gaze data on the source text by \textit{Task} and \textit{HTra} (see patterns for all nouns and single noun categories in \figref{fig:11:9}). Total \isi{fixation duration} on the target text can be predicted by \textit{Munit} and \textit{InEff}, while fixation count on the target text is only influenced by \textit{Munit}. However, when combining source and target text, only \textit{InEff} is statistically significant. When I divide nouns into singular (NN) and plural (NNS), the patterns are quite similar. \textit{Task} influences gaze on the source text and both texts, \textit{Munit} has an impact on \textit{Dur} and \isi{eye movement} in the target text, \textit{InEff} additionally on gaze on both texts, and \textit{HTra} on \textit{Dur}, \textit{GazeS} and \textit{TFix} in singular nouns. Plural nouns are significantly impacted by \textit{Task} (\textit{Dur}, gaze on source text), \isi{Munit} (gaze on target text and \textit{TFix}), \textit{InEff} (\textit{GazeT}, \textit{TGaze}), \textit{HTra} (\textit{Dur}, gaze on source and both texts), and \textit{HCross} influences \textit{TGaze}. However, proper nouns (NNP) induce different behaviour and only \textit{Dur} is influenced by \textit{Task} and \textit{InEff}, while none of the parameters has a significant impact on the \isi{gaze behaviour}.

\clearpage 
\begin{figure}
\caption{Patterns for statistical influence of predictors on parameters for CC, CD, DT, IN, JJ, MD}
\label{fig:11:8}
% % \includegraphics[width=\textwidth]{figures/DissertationNitzkeberarbeitet-img39.jpg}
\resizebox{.4\textwidth}{!}{\begin{tikzpicture}[
        nodes={minimum height=20pt,minimum width=35pt, inner sep=0pt,outer sep=0pt},
        ]
     \pgfsetmatrixcolumnsep{0mm}
     \pgfsetmatrixrowsep{0mm}
     \matrix (matrix) [matrix of nodes,nodes in empty cells,
                Greybox={2,...,8}{2,...,8}, 
                Greenbox={2,...,4,7,8}{2}, 
                Greenbox={2}{3},
                Greenbox={7,8}{4},
                Greenbox={2,6,7,8}{5},
                Greenbox={2,4,5,...,8}{6},
                Shadedbox={3,...,6}{3,4}          
              ] {\nitzkematrix};
     \node [below=.1mm of matrix] {\bfseries CC};
     \end{tikzpicture}}
\resizebox{.4\textwidth}{!}{\begin{tikzpicture}[
        nodes={minimum height=20pt,minimum width=35pt, inner sep=0pt,outer sep=0pt},
        ]
     \pgfsetmatrixcolumnsep{0mm}
     \pgfsetmatrixrowsep{0mm}
     \matrix (matrix) [matrix of nodes,nodes in empty cells,
                Greybox={2,...,8}{2,...,8}, 
                Greenbox={4,7,8}{2}, 
                Greenbox={7,8}{3,4},
                Greenbox={2,6}{5},
                Greenbox={2,3,5,6,7,8}{6},
                Greenbox={3,...,8}{7},
                Shadedbox={3,...,6}{3,4}          
              ] {\nitzkematrix};
     \node [below=.1mm of matrix] {\bfseries CD}; 
     \end{tikzpicture}}\newline
\resizebox{.4\textwidth}{!}{\begin{tikzpicture}[
        nodes={minimum height=20pt,minimum width=35pt, inner sep=0pt,outer sep=0pt},
        ]
     \pgfsetmatrixcolumnsep{0mm}
     \pgfsetmatrixrowsep{0mm}
     \matrix (matrix) [matrix of nodes,nodes in empty cells,
                Greybox={2,...,8}{2,...,8}, 
                Greenbox={2}{6,7}, 
                Greenbox={3,4}{8},
                Shadedbox={3,...,6}{3,4}          
              ] {\nitzkematrix};
     \node [below=.1mm of matrix] {\bfseries DT};              
     \end{tikzpicture}}
\resizebox{.4\textwidth}{!}{\begin{tikzpicture}[
        nodes={minimum height=20pt,minimum width=35pt, inner sep=0pt,outer sep=0pt},
        ]
     \pgfsetmatrixcolumnsep{0mm}
     \pgfsetmatrixrowsep{0mm}
     \matrix (matrix) [matrix of nodes,nodes in empty cells,
                Greybox={2,...,8}{2,...,8}, 
                Greenbox={2}{2,3,5,6}, 
                Greenbox={3}{7},
                Greenbox={4}{8},
                Greenbox={5}{6,7},
                Greenbox={6}{7},
                Greenbox={7,8}{3,6,7},
                Shadedbox={3,...,6}{3,4}          
              ] {\nitzkematrix};
     \node [below=.1mm of matrix] {\bfseries IN};              
     \end{tikzpicture}}\newline
     \resizebox{.4\textwidth}{!}{\begin{tikzpicture}[
        nodes={minimum height=20pt,minimum width=35pt, inner sep=0pt,outer sep=0pt},
        ]
     \pgfsetmatrixcolumnsep{0mm}
     \pgfsetmatrixrowsep{0mm}
     \matrix (matrix) [matrix of nodes,nodes in empty cells,
                Greybox={2,...,8}{2,...,8}, 
                Greenbox={2}{2,3,5,6}, 
                Greenbox={3}{2,8},
                Greenbox={4}{2},
                Greenbox={5}{5,7},
                Greenbox={6}{6,7},
                Greenbox={7}{2,3,4,7},
                Greenbox={8}{3,4,5,7},
                Shadedbox={3,...,6}{3,4}          
              ] {\nitzkematrix};
     \node [below=.1mm of matrix] {\bfseries JJ};              
     \end{tikzpicture}}
\resizebox{.4\textwidth}{!}{\begin{tikzpicture}[
        nodes={minimum height=20pt,minimum width=35pt, inner sep=0pt,outer sep=0pt},
        ]
     \pgfsetmatrixcolumnsep{0mm}
     \pgfsetmatrixrowsep{0mm}
     \matrix (matrix) [matrix of nodes,nodes in empty cells,
                Greybox={2,...,8}{2,...,8}, 
                Greenbox={2}{5,6}, 
                Greenbox={3}{2},
                Greenbox={4}{2},
                Greenbox={5}{6},
                Greenbox={6}{5},
                Greenbox={7,8}{6},
                Shadedbox={3,...,6}{3,4}          
              ] {\nitzkematrix};
     \node [below=.1mm of matrix] {\bfseries MD};              
     \end{tikzpicture}}\newline\vspace*{-\baselineskip}
\end{figure}


  

\begin{figure}
\caption{Patterns for statistical influence of predictors on parameters for all nouns}
\label{fig:11:9}
% % \includegraphics[width=\textwidth]{figures/DissertationNitzkeberarbeitet-img40.jpg}
\resizebox{.4\textwidth}{!}{\begin{tikzpicture}[
        nodes={minimum height=20pt,minimum width=35pt, inner sep=0pt,outer sep=0pt},
        ]
     \pgfsetmatrixcolumnsep{0mm}
     \pgfsetmatrixrowsep{0mm}
     \matrix (matrix) [matrix of nodes,nodes in empty cells,
                Greybox={2,...,8}{2,...,8}, 
                Greenbox={3,4,7}{2}, 
                Greenbox={7,8}{3},
                Greenbox={8}{4},
                Greenbox={2,5,6}{5},
                Greenbox={2,5,6,7,8}{6},
                Greenbox={2,3,8}{7},
                Shadedbox={3,...,6}{3,4}          
              ] {\nitzkematrix};
     \node [below=.1mm of matrix] {\bfseries NN};              
     \end{tikzpicture}}
\resizebox{.4\textwidth}{!}{\begin{tikzpicture}[
        nodes={minimum height=20pt,minimum width=35pt, inner sep=0pt,outer sep=0pt},
        ]
     \pgfsetmatrixcolumnsep{0mm}
     \pgfsetmatrixrowsep{0mm}
     \matrix (matrix) [matrix of nodes,nodes in empty cells,
                Greybox={2,...,8}{2,...,8}, 
                Greenbox={2}{2,3,6}, 
                Shadedbox={3,...,6}{3,4}          
              ] {\nitzkematrix};
     \node [below=.1mm of matrix] {\bfseries NNP};              
     \end{tikzpicture}}\newline
     \resizebox{.4\textwidth}{!}{\begin{tikzpicture}[
        nodes={minimum height=20pt,minimum width=35pt, inner sep=0pt,outer sep=0pt},
        ]
     \pgfsetmatrixcolumnsep{0mm}
     \pgfsetmatrixrowsep{0mm}
     \matrix (matrix) [matrix of nodes,nodes in empty cells,
                Greybox={2,...,8}{2,...,8}, 
                Greenbox={2}{2,3,7}, 
                Greenbox={3}{2,7},
                Greenbox={4}{2,7},
                Greenbox={5}{5,6},
                Greenbox={6}{5},
                Greenbox={7}{6,7,8},
                Greenbox={8}{5,7},
                Shadedbox={3,...,6}{3,4}          
              ] {\nitzkematrix};
     \node [below=.1mm of matrix] {\bfseries NNS};              
     \end{tikzpicture}}
\resizebox{.4\textwidth}{!}{\begin{tikzpicture}[
        nodes={minimum height=20pt,minimum width=35pt, inner sep=0pt,outer sep=0pt},
        ]
     \pgfsetmatrixcolumnsep{0mm}
     \pgfsetmatrixrowsep{0mm}
     \matrix (matrix) [matrix of nodes,nodes in empty cells,
                Greybox={2,...,8}{2,...,8}, 
                Greenbox={2}{5,6,7}, 
                Greenbox={3,4}{2,7},
                Greenbox={5}{5,6},
                Greenbox={6}{5},
                Greenbox={7,8}{6},
                Shadedbox={3,...,6}{3,4}          
              ] {\nitzkematrix};
     \node [below=.1mm of matrix] {\bfseries all nouns};              
     \end{tikzpicture}}\newline\vspace*{-\baselineskip}
\end{figure}

  

Possessive endings (POS) rarely occurred in the data set, but still show statistically significant results for \textit{Task} (\textit{GazeS}, \textit{FixS}, \textit{GazeT}), \textit{Munit} (\textit{Dur}, \textit{FixT}, \textit{TGaze}, \textit{TFix}), and \textit{InEff} (\textit{Dur}, \textit{GazeT}, \textit{TGaze}). The production and processing effort of personal pronouns (PRP) is determined by \textit{Task} and \textit{HTra} when considering gaze on the source text, and usually by \textit{Munit}, \textit{InEff}, and\slash or \textit{HTra} for the remaining variables. \textit{TFix} is additionally influenced by \textit{Task}, but only when comparing MPE and PE\slash TfS. \textit{Task} (except for \textit{Dur} and \textit{GazeT}), \textit{Munit} (gaze on target text \& on both texts), \textit{InEff} (\textit{Dur}, \textit{FixS}, gaze on the target text), and \textit{HCross} (gaze on source text and \textit{TGaze}) impact possessive pronouns (PRP\$), while \textit{HTra} is not important. The production times of adverbs (RB) can be predicted by \textit{Munit}, \textit{InEff}, and \textit{HTra} values. Gaze on the source text is influenced by \textit{Task} and \textit{HTra}, while all parameters except \textit{Task} have an impact on \textit{GazeT}. \textit{FixT} is influenced by none of the parameters. The \isi{gaze behaviour} on both text parts is affected by \textit{Task}, and \textit{Munit} has an additional impact on the \isi{total fixation duration}. Problem indicating parameters cannot be analysed for particles (RP), because only one particle occurred in the data set. The participant's behaviour for the word \textit{to} in its different functions can be influenced by all parameters, but very differently: \textit{Task} for all except \textit{Dur} and \textit{FixT}, \textit{Munit} for all except gaze on the source text and \textit{TFix}, \textit{InEff} for \textit{Dur} and \textit{GazeS}, \textit{HTra} for \textit{FixT} and \isi{gaze behaviour} on both text parts, and finally \textit{HCross} for gaze on the source text and \textit{GazeT} (see patterns for POS, PRP, PRP\$, RB, RP, and TO in \figref{fig:11:10}).


\begin{figure}
\caption{Patterns for statistical influence of predictors on parameters for POS, PRP, PRP\$, RB, RP, and TO.}
\label{fig:11:10}
% % \includegraphics[width=\textwidth]{figures/DissertationNitzkeberarbeitet-img41.jpg}
\resizebox{.4\textwidth}{!}{\begin{tikzpicture}[
        nodes={minimum height=20pt,minimum width=35pt, inner sep=0pt,outer sep=0pt},
        ]
     \pgfsetmatrixcolumnsep{0mm}
     \pgfsetmatrixrowsep{0mm}
     \matrix (matrix) [matrix of nodes,nodes in empty cells,
                Greybox={2,...,8}{2,...,8}, 
                Greenbox={3,4,5}{2}, 
                Greenbox={2,6,7,8}{5},
                Greenbox={2,5,7}{6},
                Shadedbox={3,...,6}{3,4}
              ] {\nitzkematrix};
     \node [below=.1mm of matrix] {\bfseries POS};
     \end{tikzpicture}}
\resizebox{.4\textwidth}{!}{\begin{tikzpicture}[
        nodes={minimum height=20pt,minimum width=35pt, inner sep=0pt,outer sep=0pt},
        ]
     \pgfsetmatrixcolumnsep{0mm}
     \pgfsetmatrixrowsep{0mm}
     \matrix (matrix) [matrix of nodes,nodes in empty cells,
                Greybox={2,...,8}{2,...,8}, 
                Greenbox={3,4}{2}, 
                Greenbox={8}{3,4},
                Greenbox={2,5,6,7}{5},
                Greenbox={2,5,6,7,8}{6},
                Greenbox={2,3,4,7,8}{7},
                Shadedbox={3,...,6}{3,4}          
              ] {\nitzkematrix};
     \node [below=.1mm of matrix] {\bfseries RPR}; 
     \end{tikzpicture}}\newline
\resizebox{.4\textwidth}{!}{\begin{tikzpicture}[
        nodes={minimum height=20pt,minimum width=35pt, inner sep=0pt,outer sep=0pt},
        ]
     \pgfsetmatrixcolumnsep{0mm}
     \pgfsetmatrixrowsep{0mm}
     \matrix (matrix) [matrix of nodes,nodes in empty cells,
                Greybox={2,...,8}{2,...,8}, 
                Greenbox={3,4,6}{2}, 
                Greenbox={7,8}{3},
                Greenbox={8}{4},
                Greenbox={5,...,8}{5},
                Greenbox={2,4,5,6}{6},
                Greenbox={3,4,7}{8},
                Shadedbox={3,...,6}{3,4}          
              ] {\nitzkematrix};
     \node [below=.1mm of matrix] {\bfseries RPR\$};              
     \end{tikzpicture}}
\resizebox{.4\textwidth}{!}{\begin{tikzpicture}[
        nodes={minimum height=20pt,minimum width=35pt, inner sep=0pt,outer sep=0pt},
        ]
     \pgfsetmatrixcolumnsep{0mm}
     \pgfsetmatrixrowsep{0mm}
     \matrix (matrix) [matrix of nodes,nodes in empty cells,
                Greybox={2,...,8}{2,...,8}, 
                Greenbox={2}{5,6,7}, 
                Greenbox={3,4}{2,7},
                Greenbox={5}{5,...,8},
                Greenbox={7}{2,3,5},
                Greenbox={8}{2,3,4},
                Shadedbox={3,...,6}{3,4}          
              ] {\nitzkematrix};
     \node [below=.1mm of matrix] {\bfseries RB};              
     \end{tikzpicture}}\newline
     \resizebox{.4\textwidth}{!}{\begin{tikzpicture}[
        nodes={minimum height=20pt,minimum width=35pt, inner sep=0pt,outer sep=0pt},
        ]
     \pgfsetmatrixcolumnsep{0mm}
     \pgfsetmatrixrowsep{0mm}
     \matrix (matrix) [matrix of nodes,nodes in empty cells,
                Greybox={2,...,8}{2,...,8},
                Shadedbox={3,...,6}{3,4}          
              ] {\nitzkematrix};
     \node [below=.1mm of matrix] {\bfseries RP};              
     \end{tikzpicture}}
\resizebox{.4\textwidth}{!}{\begin{tikzpicture}[
        nodes={minimum height=20pt,minimum width=35pt, inner sep=0pt,outer sep=0pt},
        ]
     \pgfsetmatrixcolumnsep{0mm}
     \pgfsetmatrixrowsep{0mm}
     \matrix (matrix) [matrix of nodes,nodes in empty cells,
                Greybox={2,...,8}{2,...,8}, 
                Greenbox={3,4,5,7,8}{2}, 
                Greenbox={7,8}{3},
                Greenbox={8}{4},
                Greenbox={2,5,6,7}{5},
                Greenbox={2,3}{6},
                Greenbox={6,7,8}{7},
                Greenbox={3,4,5}{8},
                Shadedbox={3,...,6}{3,4}          
              ] {\nitzkematrix};
     \node [below=.1mm of matrix] {\bfseries TO};              
     \end{tikzpicture}}\newline\vspace*{-\baselineskip}
\end{figure}

 


When I look at all verb categories combined, most behaviour is influenced by \textit{Munit} (except \textit{FixS}). \textit{Task} are also important except for \textit{Dur}, which is quite surprising, and \textit{FixT}. Production time and gaze on the source text are further influenced by \textit{InEff} (also influential in \textit{TGaze}) and \textit{HTra} (also influential in \textit{TFix}). Finally, \textit{HCross} has an impact on \textit{GazeT} and \textit{TGaze} (see patterns for all verbs and single verb categories in \figref{fig:11:11}). Verb bases (VB) are influenced by \textit{Task}, except for \textit{Dur} and \textit{FixT}, by \textit{Munit}, except for \isi{gaze behaviour} on the source text, and by \textit{HTra}, except for \textit{FixS} and gaze on the target text. \textit{InEff} only has an impact on \textit{Dur}, while \textit{HCross} has an influence on the \isi{gaze behaviour} on the target text. For past tense verbs (VBD), \textit{Task} and \textit{InEff} have an influence on production time and gaze data, excluding \textit{FixT}. \textit{Munit} plays a role for \textit{TFix} and \textit{HCross} for \textit{TGaze} and \textit{TFix}. Gerund verbs (VBG) were only influenced by \textit{Task} (gaze on source text and both texts), \textit{Munit} and \textit{InEff} (gaze on target text and both texts, plus \textit{Dur} for the latter). Past participle verbs (VBN) are influenced by \textit{Task} in \textit{Dur}, \textit{GazeS}, \textit{TGaze}, \textit{TFix}, as well as \textit{Munit} and \textit{InEff} considering gaze on the target text and \textit{TFix}. \textit{TGaze} is also impacted by \textit{InEff} and additionally by \textit{HCross}. Non-3\textsuperscript{rd}{}-person-singular present verbs (VBP) and 3\textsuperscript{rd}{}-person-singular present verbs (VBZ) are influenced by \textit{Task} when regarding gaze on the source text and on both texts. \textit{Dur} for VBP is influenced by \textit{InEff}, while it is in addition influenced by \textit{Munit}, and \textit{HTra}. \textit{Wh}{}-words(WDT) can be predicted everywhere by \textit{InEff}, except for \isi{gaze behaviour} on the source text. Gaze behaviour on both texts is additionally influenced by \textit{Task} and \textit{FixT} by \textit{Munit}. \textit{FixS} is solely impacted by \textit{HTra} (see \figref{fig:11:12}).


\begin{figure}
\caption{Patterns for statistical influence of predictors on parameters for verbs and WDT}
\label{fig:11:11}
% % \includegraphics[width=\textwidth]{figures/DissertationNitzkeberarbeitet-img42.jpg}
\resizebox{.4\textwidth}{!}{\begin{tikzpicture}[
        nodes={minimum height=20pt,minimum width=35pt, inner sep=0pt,outer sep=0pt},
        ]
     \pgfsetmatrixcolumnsep{0mm}
     \pgfsetmatrixrowsep{0mm}
     \matrix (matrix) [matrix of nodes,nodes in empty cells,
                Greybox={2,...,8}{2,...,8}, 
                Greenbox={3,4,5,7,8}{2}, 
                Greenbox={7,8}{3},
                Greenbox={8}{4},
                Greenbox={2,5,6,7,8}{5},
                Greenbox={2}{6},
                Greenbox={2,3,7,8}{7},
                Greenbox={5,6}{8},
                Shadedbox={3,...,6}{3,4}          
              ] {\nitzkematrix};
     \node [below=.1mm of matrix] {\bfseries VB}; 
     \end{tikzpicture}}
\resizebox{.4\textwidth}{!}{\begin{tikzpicture}[
        nodes={minimum height=20pt,minimum width=35pt, inner sep=0pt,outer sep=0pt},
        ]
     \pgfsetmatrixcolumnsep{0mm}
     \pgfsetmatrixrowsep{0mm}
     \matrix (matrix) [matrix of nodes,nodes in empty cells,
                Greybox={2,...,8}{2,...,8}, 
                Greenbox={2,3,4,5,7,8}{2}, 
                Greenbox={2,7,8}{3}, 
                Greenbox={8}{5},
                Greenbox={3,...,8}{6},
                Greenbox={2,...,6}{7},
                Greenbox={7,8}{8},
                Shadedbox={3,...,6}{3,4}          
              ] {\nitzkematrix};
     \node [below=.1mm of matrix] {\bfseries VBD};              
     \end{tikzpicture}}\newline
     \resizebox{.4\textwidth}{!}{\begin{tikzpicture}[
        nodes={minimum height=20pt,minimum width=35pt, inner sep=0pt,outer sep=0pt},
        ]
     \pgfsetmatrixcolumnsep{0mm}
     \pgfsetmatrixrowsep{0mm}
     \matrix (matrix) [matrix of nodes,nodes in empty cells,
                Greybox={2,...,8}{2,...,8}, 
                Greenbox={2}{6}, 
                Greenbox={3,4}{2},
                Greenbox={5,6}{5,6},
                Greenbox={7,8}{2,3,5,6},
                Shadedbox={3,...,6}{3,4}          
              ] {\nitzkematrix};
     \node [below=.1mm of matrix] {\bfseries VBG};              
     \end{tikzpicture}}
\resizebox{.4\textwidth}{!}{\begin{tikzpicture}[
        nodes={minimum height=20pt,minimum width=35pt, inner sep=0pt,outer sep=0pt},
        ]
     \pgfsetmatrixcolumnsep{0mm}
     \pgfsetmatrixrowsep{0mm}
     \matrix (matrix) [matrix of nodes,nodes in empty cells,
                Greybox={2,...,8}{2,...,8}, 
                Greenbox={2,3,7}{2}, 
                Greenbox={3,4}{7},
                Greenbox={5,6,7}{5,6},
                Greenbox={8}{3,6},
                Shadedbox={3,...,6}{3,4}          
              ] {\nitzkematrix};
     \node [below=.1mm of matrix] {\bfseries VBN};              
     \end{tikzpicture}}\newline\vspace*{-\baselineskip}
\resizebox{.4\textwidth}{!}{\begin{tikzpicture}[
        nodes={minimum height=20pt,minimum width=35pt, inner sep=0pt,outer sep=0pt},
        ]
     \pgfsetmatrixcolumnsep{0mm}
     \pgfsetmatrixrowsep{0mm}
     \matrix (matrix) [matrix of nodes,nodes in empty cells,
                Greybox={2,...,8}{2,...,8}, 
                Greenbox={3,4}{2}, 
                Greenbox={7,8}{3,4},
                Greenbox={2}{6},
                Shadedbox={3,...,6}{3,4}          
              ] {\nitzkematrix};
     \node [below=.1mm of matrix] {\bfseries VBP}; 
     \end{tikzpicture}}
\resizebox{.4\textwidth}{!}{\begin{tikzpicture}[
        nodes={minimum height=20pt,minimum width=35pt, inner sep=0pt,outer sep=0pt},
        ]
     \pgfsetmatrixcolumnsep{0mm}
     \pgfsetmatrixrowsep{0mm}
     \matrix (matrix) [matrix of nodes,nodes in empty cells,
                Greybox={2,...,8}{2,...,8}, 
                Greenbox={3,4,7}{2}, 
                Greenbox={7,8}{3}, 
                Greenbox={8}{4},
                Greenbox={2}{6,7},
                Shadedbox={3,...,6}{3,4}          
              ] {\nitzkematrix};
     \node [below=.1mm of matrix] {\bfseries VBZ};              
     \end{tikzpicture}}\newline
\end{figure}

 

\begin{figure}
\caption{Patterns for statistical influence of predictors on parameters for verbs and WDT}
\label{fig:11:12}
% % \includegraphics[width=\textwidth]{figures/DissertationNitzkeberarbeitet-img43.png}
     \resizebox{.4\textwidth}{!}{\begin{tikzpicture}[
        nodes={minimum height=20pt,minimum width=35pt, inner sep=0pt,outer sep=0pt},
        ]
     \pgfsetmatrixcolumnsep{0mm}
     \pgfsetmatrixrowsep{0mm}
     \matrix (matrix) [matrix of nodes,nodes in empty cells,
                Greybox={2,...,8}{2,...,8}, 
                Greenbox={3,4,5,7,8}{2}, 
                Greenbox={7,8}{3},
                Greenbox={8}{4},
                Greenbox={2,3,5,6,7,8}{5},
                Greenbox={2,3,4,7}{6},
                Greenbox={2,3,4,8}{7},
                Greenbox={5,7}{8},
                Shadedbox={3,...,6}{3,4}          
              ] {\nitzkematrix};
     \node [below=.1mm of matrix] {\bfseries all verbs};              
     \end{tikzpicture}}
\resizebox{.4\textwidth}{!}{\begin{tikzpicture}[
        nodes={minimum height=20pt,minimum width=35pt, inner sep=0pt,outer sep=0pt},
        ]
     \pgfsetmatrixcolumnsep{0mm}
     \pgfsetmatrixrowsep{0mm}
     \matrix (matrix) [matrix of nodes,nodes in empty cells,
                Greybox={2,...,8}{2,...,8}, 
                Greenbox={8}{2,4}, 
                Greenbox={7}{3}, 
                Greenbox={6}{5},
                Greenbox={2,5,6,7,8}{6},
                Greenbox={4}{7},
                Shadedbox={3,...,6}{3,4}          
              ] {\nitzkematrix};
     \node [below=.1mm of matrix] {\bfseries WDT};              
     \end{tikzpicture}}\newline\vspace*{-\baselineskip}
\end{figure}

 


In the models that include \textit{Task} as a significant parameter, the production and processing times become higher in TfS than in PE and MPE with three exceptions: \isi{Fixation} counts on the target text in possessive pronouns and \isi{total fixation duration} as well as total fixation count on the whole text in coordinating conjunctions. The differences in coordinating conjunctions became significant only for TfS and PE, not TfS and MPE. Consequently, the times are never significantly higher for MPE than for TfS. The same applies to MPE and TfS. If the difference between the data is significant, it is always significantly higher for TfS than for MPE. The three mentioned exceptions may indicate that the MT output is especially error prone in these \isi{PoS classes}.



The \isi{experience} and\slash or \isi{status} of the participant were added to the model if they were significant. Altogether, one of these characteristics was added to the model 16 times – 13 times the status of the participants had a significant influence, while the experience of the participants had only a significant impact in three cases. Both status and experience have a significant influence only in the production duration of cardinal numbers, and only if one of them is added to the model, not if they are both added, because they are closely related. In most cases, the results were as expected. Every time the status of the participants became influential, the students produced higher production and processing times. Also, the production duration of the cardinal numbers became lower, the more experienced the participants were. However, the \isi{total fixation duration} and \isi{total fixation count} on source and target text in \textit{wh}{}-words became higher the more experienced the participants were. This might indicate that those words are especially delicate to translate and that an experienced translator perceives that they must be translated especially carefully.



In summary, some parameters predict \isi{production time}s and \isi{gaze behaviour} more than others in the 175 models that were tested (see \tabref{tab:11:1}). \textit{Task} (87 times) and \textit{InEff} (81 times) are the most productive as they occurred in almost half of the models (49.7\% and 46.3\% respectively). \textit{Munit} (65 times – 37.1\%) and \textit{HTra} (59 times – 33.7\%) can be considered the second best parameters to predict production and processing times, because they occurred in around one third of the models. \textit{HCross} is, in comparison, not very productive for determining typing and gaze data, because this predictor hardly became significant (25 times – 14.3\%) and was often even directed negatively (10 times). The reason might be that reordering and restructuring is naturally necessary in \ili{English} and \ili{German} translations and hence cannot be considered a problem, but can rather be regarded a task. Furthermore, it hardly seems influential when single \isi{PoS} classes are considered, but it had much influence when the whole data set was considered.



\tabref{tab:11:1} presents how often which parameter became significant for which behaviour measurement. The different tasks have a huge influence on the source text \isi{gaze behaviour} and on the \isi{gaze behaviour} when combining source and target text as, in 68--76\% of the models, \textit{Task} plays a significant role. However, it hardly has an influence on the \isi{gaze behaviour} on the target text with only one significant model for fixation count (4\% of all models) and five for \isi{total fixation duration} (25\% of all models). This confirms earlier results that gaze on the source text is much less intensive in PE than in TfS. However, gaze on the target text is similar in both tasks and hence \isi{gaze behaviour} on the target text is hardly influenced by the tasks. Accordingly, \isi{gaze behaviour} on the target text cannot help in finding problems in the different tasks if I assume that longer production times indicate a problem in the translation of a unit. Finally, \textit{Task} only influences production time in 32\% of all models. One might have expected more influence of the tasks on the production times, but higher production times do not necessarily indicate that the unit was translated from scratch, nor do they help in indicating different problems in the different tasks.

The production time is often influenced by \textit{Munit} (56\% of the models), which was expected, because the total production time becomes longer if the word is changed after the first production (or if the MT must be edited). The values for \textit{Munit} have (almost) no influence on \isi{gaze behaviour} on the source text. One might have expected that the source unit is fixated more often and longer when the text is revised, but this is not the case. This indicates that revisions are often made only within the target text by considering the context and collocations of the translation rather than the source text unit. Another interpretation could be that the attention on the source text is equally distributed, but when problems arise more and longer fixations on the target text are necessary. Consequently, \textit{Munit} has a more important role in target text \isi{gaze behaviour} and becomes significant in over half of the models (52\% for \isi{total fixation duration} models and 64\% for fixation counts). When both source and target text are considered, \textit{Munit} has a significant impact in 44\% and 40\% of the models, respectively.



The values of \textit{InEff} show a similar pattern as the values of \textit{Munit}. However, \textit{InEff} is a better problem indicator than \textit{Munit}, although they are both keylogging values that show that the \isi{translation unit} needed more work than simple production or that the MT output needed editing because \textit{InEff} played more often a significant role in the models. The reason might be that \textit{InEff} is more precise than \textit{Munit}. In 84\% of all models, the \textit{InEff} value has a statistical impact on the production time. Similar to \textit{Munit}, \textit{InEff} has a very low influence on source text \isi{gaze behaviour} (it became significant in 16\% of the models for \textit{TrtS} and \textit{FixS}), while it played a bigger role for the models of the target text \isi{gaze behaviour} (64\% and 44\%). \textit{InEff} had a significant influence on 52\% and 44\% of the models combining source and target text.


\largerpage 
\textit{HTra} is not as influential as the previously analysed parameters. It has a considerable influence on production time and the \isi{gaze behaviour} on the source text (it became significant in 44\% of the models for each measure). However, it played a minor role for predicting \isi{gaze behaviour} on the target text (24\% and 20\% of the models were influenced by \textit{HTra}) and on the combination of source and target text (28\% and 32\%). This might show that the decision process for one translation is settled in the source rather than in the target text. As the \isi{word entropy} represents how many different and equally reasonable translation choices existed, it seems that decision making already takes place when the participants consider the source text, which would also support %\label{ref:ZOTEROITEMCSLCITATIONcitationIDYbpQ4iAPpropertiesformattedCitationSchaefferetal2016plainCitationSchaefferetal2016citationItemsid79urishttpzoteroorgusers1255332itemsQJKG2242urihttpzoteroorgusers1255332itemsQJKG2242itemDataid79typechaptertitleWordtranslationentropyEvidenceofearlytargetlanguageactivationduringreadingfortranslationcontainertitleNewDirectionsinEmpiricalTranslationProcessResearchpublisherSpringerpublisherplaceBerlinpage183210eventplaceBerlinauthorfamilySchaeffergivenMoritzfamilyDragstedgivenBarbarafamilyHvelplundgivenKristianTangsgaardfamilyBallinggivenLauraWintherfamilyCarlgivenMichaelissueddateparts2016schemahttpsgithubcomcitationstylelanguageschemarawmastercslcitationjsonRNDU3yWErpP23}
\citegen{SchaefferEtAl2016} assumption that reading for translation influences the source text reading. Furthermore, this provides evidence that \textit{HTra} indicates decision making behaviour rather than \isi{problem solving behaviour}. Many choices might become a problem, but usually the translators can easily decide on one translation according to context and collocation.



\textit{HCross} is not very productive as a problem indicator, because it only became significant in 4\% of the models for \textit{Dur} and \textit{FixT}, 13\% of the models for \textit{TFix}, and 20\% of the models for \textit{TrtS}, \textit{FixS}, \textit{TrtT}, and \textit{TGaze}.


\begin{table}
\begin{tabular}{lrrrrrrr}
\lsptoprule
 Parameter & Dur & TrtS & FixS & TrtT & FixT & TGaze & TFix\\
 \midrule 
 Task & 8 & 19 & 19 & 5 & 1 & 17 & 18\\
 Munit & 14 & 1 & 0 & 13 & 16 & 11 & 10\\
 InEff & 21 & 4 & 4 & 16 & 11 & 14 & 11\\
 HTra & 11 & 11 & 11 & 6 & 5 & 7 & 8\\
 HCross & 1 & 5 & 5 & 5 & 1 & 5 & 3\\
\lspbottomrule
\end{tabular}
%%please move \begin{table} just above \begin{tabular
\caption{Number of significant results according to parameter per production time or gaze behaviour measure}
\label{tab:11:1}
\end{table}


Conclusively, production time is best predicted by the \textit{InEff} value. The \isi{gaze behaviour} on the source text is highly influenced by the tasks, while the \isi{gaze behaviour} on the target text is best predicted by \textit{InEff} and \textit{Munit}. The pattern is not that clear for source and target text combined. \textit{Task} plays an important role again, but this might be further influenced by the MPE task, which was not included for \isi{gaze behaviour} on the source and on the target text. Additionally, \textit{Munit} and \textit{InEff} might help to predict higher or lower fixation counts or shorter and longer fixation durations with \textit{InEff} being a little more productive. The \isi{word entropy} also has an important impact, but it is not as productive as \textit{Munit} and \textit{InEff}. Finally, a different position of a unit in the target than in the source text does not seem to be a very productive problem indicator (\textit{HCross}) for single \isi{PoS} classes, but when considering all data. Figures~\ref{fig:D:1}--\ref{fig:D:5} in Appendix~\ref{sec:Appendix:D} display visualisations for all predictors and separately for all parameters and all \isi{PoS} classes. They are constructed in the same pattern in which Figures~\ref{fig:11:8}--\ref{fig:11:12} were constructed.


\section{Mapping the parameter with the results of the analysis of the research behaviour}
\rohead{Mapping the parameter with the results of the analysis}
\label{sec:11:4}

Research was already identified as a primary \isi{problem solving indicator} in %\label{ref:ZOTEROITEMCSLCITATIONcitationIDkee3ZJf9propertiesformattedCitationKrings1986plainCitationKrings1986dontUpdatetruecitationItemsid1709urishttpzoteroorggroups3587itemsX92T2ZTGurihttpzoteroorggroups3587itemsX92T2ZTGitemDataid1709typebooktitleWasindenKpfenvonbersetzernvorgehtEineempirischeUntersuchungzurStrukturdesbersetzungsprozessesanfortgeschrittenenFranzsischlernernpublisherGunterNarrVerlagpublisherplaceTbingeneventplaceTbingennoteZuglBochumUnivDiss198586authorfamilyKringsgivenHansPeterissueddateparts1986schemahttpsgithubcomcitationstylelanguageschemarawmastercslcitationjsonRNDVXvtlDGf0J}
\citet{Krings1986}. Hence, the following section will compare the analysed parameters of the most-research words (see \sectref{sec:9:3}) according to the problem solving indicators introduced in the previous section.



For the analysis, I will look at the 18 single words that were researched most often (more than four times). \tabref{tab:D:3} in Appendix~\ref{sec:Appendix:D} lists the mean values of \textit{Munit}, \textit{InEff}, \textit{HTra} and \textit{HCross} for the most researched single words in comparison to the mean value of these parameters according to the word's \isi{PoS} class. One has to bear in mind that all most researched single words are either nouns, verbs, or adjectives. This supports the assumption that content words require more research than function words, at least when translating into the L1. Furthermore, I also report whether the difference between mean value and mean value of the \isi{PoS} class is significant (the results can be found in \tabref{tab:D:4} in Appendix~\ref{sec:Appendix:D}). Most mean values are significantly higher for the most researched words when comparing them to the average value of the \isi{PoS} class. Eleven \textit{Munit} values are significantly higher when the word was often researched. Only the \textit{Munit} value of \textit{bureaucrats} is significantly lower. Ten of the \textit{InEff} values are significantly higher for the most researched words. The value for \textit{HTra} is significantly higher in 14 cases, while it is also significantly lower in three cases. Remember that the \textit{HTra} value is the same for the single words because it is calculated on the basis of all occurrences of the word and its translations. However, the \textit{HTra} values for the \isi{PoS} class vary and therefore a mean value has to be calculated. The same applies to \textit{HCross}. The \textit{HCross} values are significantly higher for the single words in eleven cases and significantly lower four times.



Of the 18 most researched words, 17 had at least one predictor that was significantly higher for the word than for the mean value. However, the word \textit{bureaucrats} was researched often, but the mean values of all four predictors are lower than average for the \isi{PoS} class (NNS), three of them even significantly. Hence, I can assume that the research was caused by a mere lexical problem – a problem which all 18 words have in common.\footnote{From my point of view, it should be open for discussion if the translation of a word can really be categorised as a translation problem and not only as a translation task when a word is simply unknown to the participant (at least in the context) and (s)he looks it up in a dictionary, finds a fitting translation, inserts it in the target text and is done with the translation process.} However, significantly high \textit{HTra} and \textit{HCross} values, like for the adjectives \textit{reluctant} and \textit{vulnerable}, might point to additional semantic or syntactic problems, respectively. High \textit{Munit} and \textit{InEff} values might indicate a general insecurity and\slash or indecisiveness of the participants about how to translate the source item in the context. If most or all values are significantly high, this might point to especially difficult words that may be lexically unknown and hard to integrate in the context.
\rohead{\headmark}
