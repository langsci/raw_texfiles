\chapter{Summary and discussion}
\label{sec:13}

This chapter summarises what we have learned in this study. The first three chapters after the introduction acknowledged MT and PE in general and how those topics are conceived in different fields of translation (theoretical translation studies, translation process research, translation practice, translation communities, and translation didactics), showing that MT and PE are nowadays integral parts of every aspect of translation.



\sectref{sec:5} discussed the concept of \isi{problem solving} in psychology and translation studies. We learned that translation studies do not necessarily differentiate between problem solving and decision making and the terms are not used consistently either. Therefore, the insights from both fields were combined and compared. I argued that translation is in general a task characterised by decisions that need to be made, in which problems occur that need to be solved more or less often. The difference between a translation problem and a translation task is that there is a \isi{hurdle} between the source and the target text in a problem, which the translator has to overcome. In a task, there is no hurdle. The translator knows how to create the desired final text. Most \isi{translation problems} are ill-defined because the steps required to solve the problem were not necessarily learned in advance, experience in different domains is required, personal opinions\slash judgements might be necessary. Further, different solutions and different solution paths are possible (and natural) in translation, which are all characteristics of \isi{ill-defined problems}. Only very few translation problems could be categorised as well-defined, e.g. if a lexical item is unknown to the translator and the concept of the item is so concrete that when he\slash she looks it up in a dictionary, he\slash she will find only one possible solution. However, even in this scenario, different solution paths are imaginable (research in monolingual or bilingual dictionary, corpora, or encyclopedias, or asking a colleague or the client), which would characterise this lexical problem as an ill-defined problem again. While the decisions in a regular translation task can be made in a one step operation, it takes various steps to arrive at the solution to a problem. Increasing professionalism transforms what once was a translation problem into a standard translation task – problem solving operators and strategies become routinised and automatised – and what once might have been an ill-defined problem becomes simple decision making task. Further, professionals tackle problems mostly through reasoning, they know translation rules and know how to explain their problem solving steps, while beginners often have a feeling for a (correct) solution or (correct) way to a solution, but do not know the reason or rules, respectively, for this feeling. However, even the most professional translator encounters translation problems. According to %\label{ref:ZOTEROITEMCSLCITATIONcitationIDso9gEO7zpropertiesformattedCitationFunke2006bplainCitationFunke2006bcitationItemsid126urishttpzoteroorgusers1255332itemsZMSXPPGAurihttpzoteroorgusers1255332itemsZMSXPPGAitemDataid126typechaptertitleKomplexesProblemlsencontainertitleDenkenundProblemlsencollectiontitleEnzyklopdiederPsychologieThemenbereichCTheorieundForschungSer2KognitionpublisherHogrefepublisherplaceGttingenaopage373443eventplaceGttingenaoauthorfamilyFunkegivenJoachimeditorfamilyFunkegivenJoachimissueddateparts2006schemahttpsgithubcomcitationstylelanguageschemarawmastercslcitationjsonRNDnh5MJcC1ul}
\citet{Funke2006} categorisation, translation problems do not qualify as complex problems in my perception. Although they can be categorised as complex and interconnected, and they can pursue multiple aims, they are on the other hand transparent and lack a certain dynamism. Finally, I combined the two \isi{problem solving models} of %\label{ref:ZOTEROITEMCSLCITATIONcitationID5Hn4hbvJpropertiesformattedCitationPretzNaplesandSternberg2003plainCitationPretzNaplesandSternberg2003citationItemsid84urishttpzoteroorgusers1255332itemsU63243KXurihttpzoteroorgusers1255332itemsU63243KXitemDataid84typechaptertitleRecognizingdefiningandrepresentingproblemscontainertitleThepsychologyofproblemsolvingpublisherCambridgeScholarsPublishingpublisherplaceCambridgepage330volume30eventplaceCambridgeauthorfamilyPretzgivenJeanEfamilyNaplesgivenAdamJfamilySternberggivenRobertJissueddateparts2003schemahttpsgithubcomcitationstylelanguageschemarawmastercslcitationjsonRNDndCvUUVWLR}
\citet{PretzEtAl2003} and %\label{ref:ZOTEROITEMCSLCITATIONcitationIDChgWIVBjpropertiesformattedCitationHansPKrings1986aplainCitationHansPKrings1986acitationItemsid104urishttpzoteroorgusers1255332items4PCIW24Nurihttpzoteroorgusers1255332items4PCIW24NitemDataid104typechaptertitleTranslationproblemsandtranslationstrategiesofadvancedGermanlearnersofFrenchL2containertitleInterlingualandinterculturalcommunicationpublisherGunterNarrVerlagpublisherplaceTbingenpage263276eventplaceTbingenauthorfamilyKringsgivenHansPeditorfamilyHousegivenJulianefamilyBlumKulkagivenShoshanaissueddateparts1986schemahttpsgithubcomcitationstylelanguageschemarawmastercslcitationjsonRNDuZ3wvCiPwr}
\citet{Krings1986b} to a translation \isi{problem solving cycle} (see \figref{fig:key:5:1}) that includes eight steps. Of course, not all steps are necessary for every problems and some steps might occur unconsciously. Hence, not all, but many of these steps can be traced in the translation process recordings. In the following, the eight steps are complemented by the way they could possibly be traced in the translation process data: recognise and identify the problem (potentially visible in eye-tracking data), define and present the problems mentally (unconscious), organise the knowledge about the problem (potentially unconscious), develop solution strategies and\slash or choose operators (potentially unconscious), apply those strategies\slash operators (unconscious or visible in screen recording and\slash or eyetracking data), evaluate whether they solve the problem (potentially unconscious), solve the problem (eyetracking and keylogging data), evaluate the solution (unconscious or visible in keylogging and\slash or eyetracking data), and start the process again if necessary for the next or a new problem. In summary, translation requires problem solving abilities every time there is hurdle between source and target text. Finally, the translator engages not only in problem solving, (s)he also helps the reader of the translation to overcome the language hurdle and to solve problems, e.g. (s)he enables the reader to solve troubleshooting problems by providing a well written manual.



After the theoretical groundwork was laid, \sectref{sec:6} introduced the research hypotheses that combine the thoughts on problem solving with the tasks of TfS and PE, which may require different \isi{problem solving behaviour}, as the two tasks demand different approaches from the translator despite their similarities. Hence, I hypothesised that although the problematic source text units are the same for TfS and PE, different \isi{problem solving behaviour} can be encountered in TfS and PE\slash MPE, because the MT output solves or adds problematic units, as well as between students and professionals because of their varying experience. Further, this study wanted to investigate problem solving indicators with the help of keylogging and eyetracking data and presented a model to detect the problems in translation process data so that the time-consuming analysis of screen recording and think aloud data lapse.



Before the data were analysed, the data set was introduced in \sectref{sec:7}. First, methods in translation process research were introduced in \sectref{sec:7:1} with a special focus on the methods utilised in this study (questionnaires, keylogging, eyetracking). \sectref{sec:7:2} described the data set concerning the texts, the participants, and the \isi{PE guidelines}, the technical equipment, and the database from which the data originates. The research hypotheses and methods are placed in the field of \isi{translation process research} in \sectref{sec:7:3} and previous studies that were based on the data set were introduced in \sectref{sec:7:4}. Starting with \sectref{sec:7:5}, the first overall results from the study were presented, considering the time needed for the single tasks (TfS took more time than PE and MPE), the complexity of the texts (\sectref{sec:7:6}), which might be a reason why participants behave differently in different tasks, the participant's \isi{keystroke effort} for modifications in the different tasks (\sectref{sec:7:7}), which was very heterogeneous, and a superficial error analysis of the target texts (\sectref{sec:7:8}), which provides indications about the target text quality. The chapter concluded with a critical assessment of what could have been improved in the setup of the experiments.



\sectref{sec:8} showed the results of the pre- and post-experimental questionnaires. While, on the one hand, metadata were gathered, the questionnaires also asked for personal opinions on MT\slash PE and how satisfied the participants were with the PE task. Although the participants were quite satisfied with their performance in the PE and MPE task, the results outline a (very) negative attitude towards MT and PE which probably had an influence on other questions, too, such as “how often would you have preferred to translate from scratch rather than \isi{post-editing} \isi{machine translation}?” Nonetheless, more participants could imagine integrating MT into their professional work environment after the experiment than before the experiment. This contradictory behaviour might reflect an adverse attitude against MT and PE itself, even though they are not very familiar with the technology (see \figref{fig:key:8:1}) or the tasks (only 1/3 of all participants had post-edited before).



In the data analysis part, the focus was first on the Internet search instances in the analysis of the data (\sectref{sec:9}). According to %\label{ref:ZOTEROITEMCSLCITATIONcitationID4ZuPNPPCpropertiesformattedCitationKrings1986plainCitationKrings1986dontUpdatetruecitationItemsid1709urishttpzoteroorggroups3587itemsX92T2ZTGurihttpzoteroorggroups3587itemsX92T2ZTGitemDataid1709typebooktitleWasindenKpfenvonbersetzernvorgehtEineempirischeUntersuchungzurStrukturdesbersetzungsprozessesanfortgeschrittenenFranzsischlernernpublisherGunterNarrVerlagpublisherplaceTbingeneventplaceTbingennoteZuglBochumUnivDiss198586authorfamilyKringsgivenHansPeterissueddateparts1986schemahttpsgithubcomcitationstylelanguageschemarawmastercslcitationjsonRNDggmltPM9tL}
\citet{Krings1986}, research is a primary \isi{problem solving indicator} and I generally agree. Most research was probably conducted when solving a \isi{lexical problem}. In the majority of the instances, this external problem solving strategy is probably combined with internal knowledge. The participant does not know how to translate a lexical unit, because it is either unknown to him\slash her (in the given context) or (s)he cannot decide on a translation alternative (in the given context) – there is a hurdle between the source text and the target text and research is the chosen solving strategy. First, the screen recordings of all sessions were analysed and very often, a significant difference could be observed in the research behaviour in PE and TfS. Also, the \isi{experience} of the translators seems to influence the research behaviour, which became visible when the participants were separated by status (professional vs. student) and\slash or when the experience coefficient introduced in \sectref{sec:8:1} was taken into consideration. Less experienced translators do more research than more experienced translators. With growing experience lexical problems occur less often because of developing language skills in the L1 and L2 as well as increasing translation skills. Experienced translators may translate more freely and have a better feeling for the semantic meaning in the context. Further, the MT output in the PE task seems to solve lexical problems because less research was conducted than in the TfS task. However, no indications were found that the MT output creates new problems in the PE task because if the participants did not judge the MT output to be acceptable for the target text, they could refer to the source text and could either find a translation with the help of their internal knowledge (or as %\label{ref:ZOTEROITEMCSLCITATIONcitationIDxppjYxGwpropertiesformattedCitationPACTE2005plainCitationPACTE2005citationItemsid223urishttpzoteroorgusers1255332itemsJBVNW3TJurihttpzoteroorgusers1255332itemsJBVNW3TJitemDataid223typearticlejournaltitleInvestigatingtranslationcompetencecontainertitleMetajournaldestraducteurspage0609619volume50issue2authorfamilyPACTEgivenGrupissueddateparts2005schemahttpsgithubcomcitationstylelanguageschemarawmastercslcitationjsonRND333WeWMJ1q}
\citet{Pacte2005} phrased it: their inner support), which can be categorised as a translation task, or they still have to solve the problem, which probably would have occurred in the TfS too, with the help of \isi{external support}. This does not apply to MPE. When the MT output is not suitable for the target text context and no source text is available for the translator, a new problem may arise because the unit may not have been problematic in the TfS task, or the problem might be more difficult to solve because the translator does not know how to use the external resources to find a solution. Accordingly, we can summarise that translation strategies might be transferred to the PE task but not to the MPE task. This is supported by the fact that research behaviour does not seem to be influenced by experience (neither status nor experience coefficient) in MPE. In a next step, the parameters production time (\textit{Dur}), \isi{gaze time} on source text (\textit{GazeS}), and \isi{gaze time} on target text (\textit{GazeT}) were analysed for the 27 most researched words to see whether they are influenced by the research task. When the individual words were analysed, not many statistical difference could be observed, which is probably caused by the few data points. However, if all words are analysed together, both the t-tests and the linear mixed models predict that the parameters are also impacted by the different tasks whether research was conducted or not – with the exception that the task does not influence gaze on the target text, which on the one hand could imply that problem solving eventuates in the source text and on the other hand that gaze on the target text is similar in TfS and PE. Conclusively, research as a problem solving activity is used in another intensity in the different tasks and in relation to the experience of the translator.



\sectref{sec:10} analysed the influence of the \isi{syntactic quality} of the MT output. I hypothesised that syntax is not problematic for the translator in the TfS task, but that the MT output might erect hurdles between the source and the target text, because syntax is often a problem for MT systems in the language pair \ili{English}-\ili{German}. To assess this question, I categorised the syntax of the MT output into acceptable, partly acceptable, and not acceptable and compared the production time and the eyetracking data to those of the TfS task. The analysis was expanded for the parameters \isi{fixation count} on the source (\textit{FixS}) and on the target text (\textit{FixT}). Further, parameters that considered the \isi{total fixation duration} (\textit{TGaze}) and the fixation count (\textit{TFix}) on both source and target text were created. If new hurdles were build by the MT output, when the syntactic MT output was not acceptable, the keylogging and eyetracking data should have been significantly higher in the PE task than in the TfS task. The analysis, however, showed that bad syntactic MT output does prolong the production time of the sentence, however no new problems are created because the \isi{gaze behaviour} of the participant is not impacted by the syntactic MT quality. Hence, creating and correcting the structure of a MT sentences can be categorised as a task, but not as a problem.


\largerpage
As was mentioned above, some of the steps of the \isi{problem solving cycle} might be visible in the translation process data. It is not necessary to label e.g. a very long \isi{fixation duration} on one word as the step “identifying a problem” or “solving a problem” or “evaluating a solution”, but this long fixation potentially shows us that there is a translation problem. Hence, \sectref{sec:11} presented \isi{keylogging parameter}s (\textit{Munit}, \textit{InEff}, \textit{HTra}, and \textit{HCross}) that can be used to predict (hence called predictors) behavioural parameters of the eyetracking and keylogging data. The influence of these \isi{predictors} can vary a lot according to the \isi{PoS} class, though. Further, some predictors are more productive than others on these behavioural parameters. \textit{Task} and \textit{InEff} often have an influence, the influence of \textit{Munit} and \textit{HTra} is moderate, and \textit{HCross} hardly has a significant influence. Interestingly, the status and\slash or the experience of the participants seldom had a statistical influence in the regression models that were created to analyse which predictor influences which parameter in which \isi{PoS} class. The status of the participants was a little more productive than the experience coefficient. After testing the influence of the predictors on the single \isi{PoS} classes, the values of the most researched words, as defined in \sectref{sec:9:3}, were checked against the mean values of the \isi{PoS} class because we can safely assume that these are translation problems. The results show that many predictors are significantly above average for the most researched words and hence this proves that the predictors seem sensible to predict problems.


One major issue when talking about and analysing \isi{translation problems} is to identify the problems in the process data. Think-aloud protocols can help with this issue, but the data are potentially incomplete and subjective, and the task is usually unfamiliar to the participants. Further, the assessment of think-aloud protocols is difficult and time-consuming. Hence, it is desirable to detect translation problems with the help of other translation process data. An approach to identifying problematic units in translation process data was introduced in \sectref{sec:12}, which is mainly based on the findings in \sectref{sec:11}. First, a problem area has to be defined for the \isi{influential predictors}, i.e. \textit{Munit}, \textit{InEff}, \textit{HTra}, and \textit{HCross}. The problem area is calculated by the mean of the respective predictor plus one \isi{standard deviation}. Then, all influential problem areas are combined in a command that also contains the predictor \textit{Task} if required. The calculations can be done either for all words or for single \isi{PoS} categories. According to the research question, additional characteristics can be included in the calculation like participant or text. When this formula is applied, R displays all words that were potentially problematic in the translation process, because they required above average keylogging effort (which were related to the eyetracking data in \sectref{sec:11}). The calculations, however, still need to be tested in real-life translation scenarios, in which ideally, the translators would still be available to verify whether the identified word was really problematic or not. I argued that keylogging data are more reliable than eyetracking data. However, it would still be interesting to expand these calculations with eyetracking parameters because many more steps in the \isi{problem solving cycle} can potentially be identified with eyetracking data than with keylogging data.


\largerpage
The first main hypothesis was that some problems are already solved by the MT output, while some new problems (can) arise from the MT output. This hypothesis was only partly confirmed. The study at hand has shown that pre-trans\-lating the source text with an MT system does not create new problems for the translator. Sometimes the production process might be exceeded or a lexical item needs to be deleted and researched again (the lexical problem was not solved by the MT system), but these instances do not cause problems, they are only a task. Hence, the MT output might support the translator or might not support the translator, but it does not make the translation process more complicated. This, however, only applies to PE (bilingual \isi{post-editing}) where the source text is available to the translator. As we have seen in the chapter on lexical items (\sectref{sec:9}), a mistranslation in the MT output can cause additional problems because it confuses the translator and (s)he has no source text to resolve the problem. In the end, I found different behaviour in TfS and PE. Very often the differences between the different tasks (also when MPE was included) turned out to be (statistically significantly) different in all aspects analysed in this study. Production times are often shorter in PE than in MPE. The source text is neglected much more in PE than in TfS. It is only consulted to check the MT output or when problematic units occur in the MT output in PE, while it is the main source of information in the TfS task. The target text, on the other hand, is consulted roughly equally as often in TfS as in PE. It is the main source of information in the PE task. Further, the tasks also have a vital impact on the research behaviour.



As a side note, this study has proven to me again that MPE is nothing that will be enforced in translation practice any time soon. The MT output is not reliable enough to support a monolingual editing process. Rather, trained translators should be assigned to bilingual \isi{post-editing} task so they can transfer their translation knowledge and expertise to this new task. Additionally, I think, it is essential for translators to get training in the PE task, as well so that they can work efficiently and may enjoy the task rather than considering it a burden.



The second main hypothesis, namely that different \isi{problem solving patterns} develop in relation to the expertise level of the participant, cannot really be confirmed or disproven. While many differences could be found between individual translators in the assessment of the \isi{screen recordings} of the \isi{research behaviour} in regard to their level of expertise, the statistical analysis of the keylogging and eyetracking data hardly proved any influence of the \isi{status} of the participant nor the \isi{experience} according to the experience coefficient. Disappointingly, the latter was even less often a significant influence than the status of the participants. However, in my opinion, we should not disregard experience as an \isi{influential factor} in translation process research. I would rather claim that we have not found the right measure for competence yet. The status of the participants as well as the experience coefficient are (relatively) simple attempts to represent competence, which cannot depict all facets that may contribute. Therefore, it will be necessary in future research to find a way to measure competence in a more fine-grained way that is also applicable for empirical studies. Hence, a numeric value would be favourable. 



It is also possible that the keylogging and eyetracking data do not reveal as many \isi{competency differences} as the screen recording analysis because a certain translation item might be difficult for a more experienced translator as well. However, (s)he draws back on \isi{internal resources} more extensively than the more inexperienced translator. Hence, especially the eyetracking data might not be as expressive. Bear in mind that this is only an hypothesis that needs to be tested in another study.


\newpage 
The calculations introduced in \sectref{sec:12} to identify problematic words in a database could be applied to translation and \isi{post-editing} courses quite easily, because they are only based on keylogging data. \isi{Keylogging} data can be recorded much easier than eyetracking data because some keylogging solutions can be used for free or could be implemented into online applications, while recording eyetracking data requires expensive hard- and software. Further, keylogging programmes do not impact the natural translation situation. In a perfect scenario, the translation trainees would prepare their translation at home, using a software that tracks the keylogging data, and would hand in the finalised translation product together with the keylogging data. The translation trainer, on the other hand, could use the keylogging data to calculate in \isi{R} according to the formulas presented in \sectref{sec:12} which words in the translation assignment could be considered problematic and need discussion in the translation class. This procedure would also help verify, falsify, and assess the calculations presented in \sectref{sec:12}. Of course, these identified problematic words could only be used as an addition, because the calculations presented only consider the word level so far and problems may occur on a sentence or text level (or any other level), too. Also, problems concerning register, text type and domain conventions, etc. cannot exclusively be detected on a word level.


