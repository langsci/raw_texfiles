\chapter{Research hypotheses}
\label{sec:6}

\isi{Problem solving} is an active field in psychology and was also considered in translation studies, where experiments with think-aloud protocols have been conducted. In the following chapters, I want to explore how the problem solving activities of (professional and student) translators change in PE – human editing of machine translated output with available source text – compared to the translation from scratch task. Further, there will be references to monolingual PE tasks – a task where no source text is available – which were part of the experimental settings; however, I assume that MPE is neither comparable with PE nor TfS. Therefore, the main focus will be on the differences between PE and TfS.



In an article on the technologization of the translation process and the growing importance of MT and \isi{translation memory} (TM) systems, %\label{ref:ZOTEROITEMCSLCITATIONcitationIDI6GThTBLpropertiesformattedCitationPym2013plainCitationPym2013citationItemsid206urishttpzoteroorgusers1255332itemsG8UASJ79urihttpzoteroorgusers1255332itemsG8UASJ79itemDataid206typearticlejournaltitleTranslationSkillsetsinaMachinetranslationAgecontainertitleMetaJournaldestraducteursMetaTranslatorsJournalpage487503volume58issue3authorfamilyPymgivenAnthonyissueddateparts2013schemahttpsgithubcomcitationstylelanguageschemarawmastercslcitationjsonRNDyKpt6liSfz}
\citet[493]{Pym2013} suggests that the influence of TM\slash MT systems on translation can be defined easily:


\begin{quote}
[W]hereas much of the translator’s skill-set and effort was previously invested in \textit{identifying} possible solutions to translation problems [...], the vast majority of those skills and efforts are now invested in \textit{selecting} between available solutions, and then adapting the selected solution to target-side purposes [...]. The emphasis has shifted from generation to selection. (emphasis in original text)
\end{quote}


Taking this assumption out of the technology context, it refers to a difference between \isi{problem solving} and \isi{decision making}. While problem solving actively requires the translator to identify or generate solutions, (different) solution(s) are available to the translator (in his\slash her mental lexicon) in decision making and (s)he (simply) has to choose one of the existing solutions in his\slash her mental lexicon. Following this argumentation, PE would be considered a decision making task, and TfS problem solving. However, as we have seen in the last chapters, not all translation units can be considered problem solving in TfS. On the other side, the translator has to choose in PE whether or not the MT is acceptable. When (s)he decides that the MT is not acceptable, (s)he has to start from scratch. Hence, if a translation problem occurs, the MT might suggest a solution. If it does not present a solution, the translator still has to cope with the problem by him\slash herself.



Therefore, I assume that the MT output in the PE task sometimes already solves what might have previously been considered problematic, because the MT output suggests a final state (cf. %\label{ref:ZOTEROITEMCSLCITATIONcitationIDqcSadaQ5propertiesformattedCitationHansPKrings2001plainCitationHansPKrings2001citationItemsid228urishttpzoteroorgusers1255332items4D665XEKurihttpzoteroorgusers1255332items4D665XEKitemDataid228typebooktitleRepairingtextsempiricalinvestigationsofmachinetranslationposteditingprocessespublisherKentStateUniversityPresspublisherplaceKentOhionumberofpages635sourceLibraryofCongressISBNeventplaceKentOhioISBN9780873386715callnumberP309K75132001shortTitleRepairingtextslanguageengauthorfamilyKringsgivenHansPeditorfamilyKobygivenGeoffreySissueddateparts2001schemahttpsgithubcomcitationstylelanguageschemarawmastercslcitationjsonRNDLbV8vumuNk}
\citealt{Krings2001}: 472). However, MT might also introduce new hurdles between source text and target text due to unacceptable MT output. Hence, the translator has to overcome the hurdle. In other words, the defective MT realisation can be considered a hurdle between source text and acceptable target text solution.



Moreover, the participants of the experiment were separated into two groups, namely professional and semi-professional translators. The assumption that translating activity changes with increasing professionalism has been raised before, as mentioned in \sectref{sec:5} and by %\label{ref:ZOTEROITEMCSLCITATIONcitationID5O8aJ44JpropertiesformattedCitationKiraly1995plainCitationKiraly1995citationItemsid21urishttpzoteroorgusers1255332itemsRM3K5NDAurihttpzoteroorgusers1255332itemsRM3K5NDAitemDataid21typebooktitlePathwaystotranslationpedagogyandprocesscollectiontitleTranslationstudiescollectionnumber3publisherKentStateUniversityPresspublisherplaceKentOhionumberofpages175sourceLibraryofCongressISBNeventplaceKentOhioISBN0873385160callnumberP3065K571995shortTitlePathwaystotranslationauthorfamilyKiralygivenDonaldCissueddateparts1995schemahttpsgithubcomcitationstylelanguageschemarawmastercslcitationjsonRNDk1YQuryVob}
\citet[41]{Kiraly1995} "[...] translation processing is probably a mix of conscious and subconscious processes – a mix that may change as translators proceed through their training and become more professional". Further, it was already discussed by %\label{ref:ZOTEROITEMCSLCITATIONcitationIDkVwu08zJpropertiesformattedCitationrtfPavloviuc0u263andJensen2009plainCitationPavloviandJensen2009citationItemsid14urishttpzoteroorgusers1255332itemsB8GBQAKUurihttpzoteroorgusers1255332itemsB8GBQAKUitemDataid14typechaptertitleEyetrackingtranslationdirectionalitycontainertitleTranslationResearchProjects2publisherInterculturalStudiesGrouppublisherplaceTarragonapage93109eventplaceTarragonaauthorfamilyPavlovigivenNataafamilyJensengivenKristianeditorfamilyPymgivenAnthonyfamilyPerekrestenkogivenAlexnderissueddateparts2009schemahttpsgithubcomcitationstylelanguageschemarawmastercslcitationjsonRND9gFdFFMN5K}
\citet{PavlovicPerekrestenko2009} that students invest more cognitive effort into their translations than professionals – evidence was found in this study in regard to \isi{gaze time}, task length, and pupil dilation, but not in average \isi{fixation duration} – which might indicate that students have to handle many more translation problems than professionals.



As was mentioned above, MT output can either deconstruct hurdles, have no impact on the hurdles, or create new hurdles between source and target text. The same problems obviously exist in the source text for both human translation and \isi{post-editing}. Therefore, my main hypothesis is that some problems are already solved by the machine translation (MT) output, while some new problems (can) arise from the MT output. Therefore, the patterns of the applied problem solving strategies change or new strategies – that may not be familiar for TfS – are necessary depending on the task. Further, different patterns develop in relation to the level of expertise of the participant.



Earlier quantitative research on problem solving in translation used think-aloud protocols, e.g. %\label{ref:ZOTEROITEMCSLCITATIONcitationIDNBdhWqiopropertiesformattedCitationKrings1986cplainCitationKrings1986cdontUpdatetruecitationItemsid103urishttpzoteroorgusers1255332itemsVMUMBHRNurihttpzoteroorgusers1255332itemsVMUMBHRNitemDataid103typebooktitleWasindenKpfenvonbersetzernvorgehtEineempirischeUntersuchungzurStrukturdesbersetzungsprozessesanfortgeschrittenenFranzsischlernernpublisherGunterNarrVerlagpublisherplaceTbingenvolume291eventplaceTbingenauthorfamilyKringsgivenHansPissueddateparts1986schemahttpsgithubcomcitationstylelanguageschemarawmastercslcitationjsonRNDImfhVQIWvi}
\citet{Krings1986},
%\label{ref:ZOTEROITEMCSLCITATIONcitationIDJF2thWUkpropertiesformattedCitationrtfLuc0u246rscher1986plainCitationLrscher1986citationItemsid97urishttpzoteroorgusers1255332items57BI7WICurihttpzoteroorgusers1255332items57BI7WICitemDataid97typechaptertitleLinguisticaspectsoftranslationprocessesTowardsananalysisoftranslationperformancecontainertitleInterlingualandinterculturalcommunicationpublisherGunterNarrVerlagpublisherplaceTbingenpage277292eventplaceTbingenauthorfamilyLrschergivenWolfgangissueddateparts1986schemahttpsgithubcomcitationstylelanguageschemarawmastercslcitationjsonRNDNdDVTwrUIn}
 \citet{Lorscher1986}, %\label{ref:ZOTEROITEMCSLCITATIONcitationIDCGQTKrB4propertiesformattedCitationKubiak2009plainCitationKubiak2009dontUpdatetruecitationItemsid102urishttpzoteroorgusers1255332itemsKWMUG2RKurihttpzoteroorgusers1255332itemsKWMUG2RKitemDataid102typethesistitleUbersetzeralsProblemloserEinequalitativeStudiezumProblemloseverhaltenvonsemiprofessionellenUbersetzernpublisherWydawnictwoNaukoweUAMissueddateparts2009schemahttpsgithubcomcitationstylelanguageschemarawmastercslcitationjsonRNDKRLTICtwMd}
\citet{Kubiak2009}, and %\label{ref:ZOTEROITEMCSLCITATIONcitationIDIT725MvXpropertiesformattedCitationAngelone2010plainCitationAngelone2010citationItemsid158urishttpzoteroorgusers1255332itemsBD4NRZGMurihttpzoteroorgusers1255332itemsBD4NRZGMitemDataid158typechaptertitleUncertaintyuncertaintymanagementandmetacognitiveproblemsolvinginthetranslationtaskcontainertitleTranslationandcognitioncollectiontitleAmericanTranslatorsAssociationScholarlyMonographSeriescollectionnumber15publisherJohnBenjaminsPublishingCompanypublisherplaceAmsterdamPhiladelphiapage1740eventplaceAmsterdamPhiladelphiaauthorfamilyAngelonegivenErikeditorfamilyShrevegivenGregoryMfamilyAngelonegivenErikissueddateparts2010schemahttpsgithubcomcitationstylelanguageschemarawmastercslcitationjsonRNDCHfNca5r5w}
\citet{Angelone2010}, as think-aloud protocols were considered useful for the task %\label{ref:ZOTEROITEMCSLCITATIONcitationIDE8Fnc3BdpropertiesformattedCitationWilss1994plainCitationWilss1994citationItemsid236urishttpzoteroorgusers1255332itemsKDSR29MCurihttpzoteroorgusers1255332itemsKDSR29MCitemDataid236typearticlejournaltitleAFrameworkforDecisionmakinginTranslationcontainertitleTargetpage131150volume6issue2authorfamilyWilssgivenWolframissueddateparts1994schemahttpsgithubcomcitationstylelanguageschemarawmastercslcitationjsonRNDZO5yhqY06M}
(cf. \citealt{Wilss1994}: 143). This study, however, attempts to show how to explore problem solving with other methods of translation process research, namely eyetracking and keylogging. Questionnaires were used in the study as well, but not to analyse problem solving, but for meta-data and to analyse subjective attitudes towards MT and PE. I assume that not every translation act is simultaneously a problem solving act, but that most translation activities can be characterised as tasks (according to \citealt{Dorner1987}) for both professional and semi-professional translators. Only in single instances do translators stumble across problematic parts that can be considered a problem and therefore need special attention. Problem identification can either be conscious or subconscious. Therefore, the eyetracking and keylogging data shall reveal subconscious as well as conscious problem solving activity. The first part of the analysis (\sectref{sec:9}) will deal with conscious problem solving, namely \isi{\textit{lexical problem solving}}, where information is retrieved via research – the screen recording software included in Tobii studio allows us to track the Internet research behaviour of the participants and no offline research options were available. The second part of the analysis (\sectref{sec:10}) will focus on \isi{\textit{syntactic structures}}, as those seem to be especially difficult for SMT systems for translations from \ili{English} into \ili{German}. The third part of the analysis (\sectref{sec:11}) is an attempt to characterise \textit{keylogging} and \textit{eyetracking parameters} that indicate (conscious and subconscious) problems or help to identify \isi{problem solving activity} in TfS and PE\slash MPE in general without the help of obvious problem indicators like research or the quality of the MT output. In line with %\label{ref:ZOTEROITEMCSLCITATIONcitationIDa4LGUkXepropertiesformattedCitationrtfOuc0u8217Brien2006plainCitationOBrien2006citationItemsid214urishttpzoteroorgusers1255332itemsAE7QW2HGurihttpzoteroorgusers1255332itemsAE7QW2HGitemDataid214typethesistitleMachinetranslatabilityandposteditingeffortAnempiricalstudyusingTranslogandChoiceNetworkAnalysispublisherDublinCityUniversityauthorfamilyOBriengivenSharonissueddateparts2006schemahttpsgithubcomcitationstylelanguageschemarawmastercslcitationjsonRNDZacxejoTWD}
\citet{OBrien2006}, I will present an attempt to identify problem solving activity in process data with the help of mere keylogging data (as identified in the preceding chapter) in translation process data in the final part (\sectref{sec:12}). Although translation is not in itself seen purely as a problem solving activity in this study, there is a common point of agreement that problem solving might be an indicator for expertise. Hence, the experience of the participants will be taken into consideration in the analysis.



\sectref{sec:7} introduces the data set starting with an overview of translation process research methods. Then, I will present general information on the data set, the placement of the research hypotheses and previous research conducted with the data set, as well as a first overall analysis (session duration, complexity levels of the source texts, general \isi{keystroke effort} and general error analysis of the final target texts, and critics of the methodology), before the empirical analysis is presented (starting in \sectref{sec:8} with the assessment of the questionnaires).


