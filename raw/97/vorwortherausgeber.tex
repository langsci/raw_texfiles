\chapter*{Vorwort der Herausgeber}
\pdfbookmark[0]{Vorwort der Herausgeber}{Vorwort.Herausgeber}

Bei diesem Band handelt es sich um eine kritische Ausgabe des Hauptwerks Georg von der Gabelentz' (1840–1893). Grundlage des Textes sind die erste Auflage von 1891 sowie die posthum erschienene zweite Auflage von 1901, die von Albrecht Graf von der Schulenburg (1865–1902), einem Neffen und Schüler Gabelentz', erheblich überarbeitet und erweitert wurde. 

Beim Vergleich der beiden Auflagen sind drei bzw. vier Fälle zu unterscheiden: Text aus der ersten Auflage, Text aus der zweiten Auflage, sowie gegenüber beiden Auflagen korrigierter und von den Herausgebern erstellter Text. Dazu kommt Text aus Vorpublikationen zweier Abschnitte, die teilweise von beiden Auflagen abweichen:

\begin{enumerate}[1.]
\item \begin{sloppypar}Im dritten Buch, S.~\pageref{III.II.II.8}–\pageref{III.II.II.9}: \textit{Festgruss an Otto von Böhtlingk zum Doktor-Jubiläum 3. Februar 1888 von seinen Freunden}. Stuttgart: Kohlhammer, 1888, S.~26–30.\end{sloppypar}
\item Im vierten Buch, S.~\pageref{IV.III.II.1}–\pageref{IV.III.III}: Über Stoff und Form in der Sprache. \textit{Berichte über die Verhandlungen der königlich-sächsischen Gesellschaft der Wissenschaften zu Leipzig, philologisch-historische Classe}, Band 21, 1889, S.~185-216.
\end{enumerate}

Die Zugehörigkeit eines Textstücks zu einer der vier Instanzen, die für den Text verantwortlich zeichnen (Autor, bearbeitender Autor, die Herausgeber, Vorpublikationen), wird folgendermaßen gekennzeichnet:

Text, der beiden Auflagen gemeinsam ist und die Zustimmung der Herausgeber hat, wird ohne wei­tere Kennzeichnung wiedergegeben. Alles andere wird mit verschiedenfarbigen Mar­kierungen kenntlich gemacht. Text, der nur in der zweiten Auflage und nicht in der ersten Auflage enthalten ist und die Zustimmung der Herausgeber hat, ist \sed{rot markiert}. Text, der aus der ersten Auflage in den Haupttext übernommen wurde und nicht in der zweiten Auflage enthalten ist, ist \fed{dunkelblau mar­kiert}. Von den Herausgebern verantworteter Text, der von beiden Auflagen abweicht, ist \edins{hellblau markiert} worden. Text, der nur in den Vorpublikationen vorkommt, ist \othersrc{grau markiert}.

Text, der von den Herausgebern als fehlerhaft aus dem Haupttext ausgeschlossen worden ist, wird in einer Marginalie wiedergegeben. Die Quelle der Textformen in den Marginalien wird durch das Erscheinungsjahr wiedergegeben: 1891 für die erste Auflage, 1901 für die zweite Auflage, und 1888 resp. 1889 für die beiden Vorpublikationen. An manchen Stellen, z.~B. im Inhaltsverzeichnis, werden aus technischen Gründen keine Marginalien benutzt, sondern die abweichenden Textformen direkt in den fortlaufenden Text gesetzt.

Wenn eine der Vorpublikationen abweichenden Text hat, aber die erste und zweite Auflage übereinstimmen, ist die Stelle dunkelblau markiert. Obwohl im Normalfall die Farbe Dunkelblau sich nur auf die erste Auflage bezieht, bedeutet sie an Stellen, wo drei Textformen im Spiel sind, dass die Variante zuerst in der ersten Auflage erschien und dann in die zweite übernommen wurde. Wenn eine Vorpublikation und die erste Auflage in Einklang sind, ist die Stelle grau markiert.

\begin{sloppypar}In die Marginalien wurden grundsätzlich nur vollständige typographische Wörter aufgenommen. Ein ty­pographisches Wort wird durch Textanfang, Textende oder Wortzwischenräume begrenzt. Satzzeichen sind normalerweise Teil eines typographischen Wortes. Auslassungspunkte (...) in Marginalien gehören nie zum Originaltext. Sie weisen darauf hin, dass von einem zitierten Textstück nur der Anfang und das Ende wiedergegeben werden.\end{sloppypar}

Seitenanfänge der ersten Auflage werden durch dunkelblau markierte Seitenzahlen zwischen senkrechten Strichen angegeben (\fed{{\textbar}123{\textbar}}), die der zweiten Auflage durch rot markierte Seitenzahlen zwischen doppelten senkrechten Strichen (\sed{{\textbar}{\textbar}321{\textbar}{\textbar}}) und die der Vorpublikationen durch grau markierte Seitenzahlen zwischen einfachen senkrechten Strichen (\othersrc{{\textbar}313{\textbar}}).

Die Seitenzahlen im Inhaltsverzeichnis und Register beziehen sich auf die Seitennummerierung dieser Ausgabe. Querverweise im fortlaufenden Text wurden nicht aktualisiert und beziehen sich auf die Seitenzahlen der ersten bzw. zweiten Auflage.

\largerpage[-1]Die Richtigstellung von Sachfehlern (d.~h. von unangemessenen Darstellungen eines behandelten Gegenstandes) ist nicht Aufgabe dieser textkritischen Ausgabe; Fehler, die bei der Textkonstitution berücksichtigt und nach Möglichkeit berichtigt (durch Herausgebertext ersetzt) werden, sind ausschließlich solche, bei denen eine aus dem Textzusammenhang erschließbare Aus­sageintention gestört und nicht realisiert worden ist. Als Subjekte von Aussageintentionen kommen der Autor der ersten Auflage und der Bearbeiter der zweiten Auflage in Frage. Als Verursacher von Störungen kommen sowohl Setzer als auch Korrekturleser der beiden Auflagen in Frage.

Es sind alle textuellen Abweichungen verzeichnet worden, ungeachtet dessen, ob sie beabsichtigt waren oder aus Versehen unterlaufen sind. Rein technische Abweichungen wie unvollständig gedruckte Zeichen (was gelegentlich bei Kommata vorkommt) sind nicht als textuell angesehen worden. Unrichtig eingesetzte, z.~B. kopf­stehende Lettern sind als Grenzfälle gewertet und mit verzeichnet worden.

Der bei weitem überwiegende Teil der Abweichungen geht auf die Tätigkeit des Bearbeiters der zweiten Auflage zurück, entweder positiv, als gezielt veränderter Text, oder negativ, als nicht behobe­ner Druckfehler. Alle, auch die nicht klaren Fälle, werden dem Leser zur Beurteilung unterbreitet. Zugleich ist darauf geachtet worden, dass der durchlaufende Text sich auch ohne Berücksichtigung der Lesarten gut lesen lässt.

\begin{flushright}
-- Die Herausgeber
\end{flushright}