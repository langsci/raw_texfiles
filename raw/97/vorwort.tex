%insert scan of both title pages
\pdfbookmark[0]{Vorwort}{Vorwort}
\chapter*{Vorwort \sed{zur ersten Auflage}.}

\sed{{\textbar}{\textbar}III{\textbar}{\textbar}} \fed{{\textbar}III{\textbar}} Dies Buch verdankt seine Entstehung sehr \update{verschieden\-seitigen}{verschieden\-artigen} Anregungen. Neigung und Beruf haben mich seit Jahren genöthigt, mich mit Sprachen der \update{mannich\-faltigsten}{mannig\-faltigsten} Bauformen zu beschäftigen, manche von ihnen familienweise zu vergleichen, andere lehrend oder schildernd darzustellen. Kathedererfahrungen und häufiger Gedankenaustausch mit befreundeten Fachgenossen über allgemeinere Fragen kamen hinzu; in der einschlägigen Literatur, soweit ich sie kennen lernte, fand ich nur Theile dessen, was ich suchte, Vieles, was mir nicht einleuchtete. Und so wurde es mir zugleich Bedürfniss und Pflicht, mir und Anderen über meinen Standpunkt Rechenschaft zu geben. Schon die Lehrvorträge über vier uns so fremdartige und untereinander so verschiedene Sprachen, wie Chinesisch, Japanisch, Mandschu und Malaisch, nöthigten mich immer wieder, in’s sprachphilosophische Gebiet hinüberzuschweifen. Dabei konnte ich beobachten, wie schwer sich oft die besten Köpfe von den muttersprachlichen Vorurtheilen losringen, wie aber dann, wenn dies gelungen, aus den entlegensten Gebieten herüber auf heimische Spracherscheinungen Licht fallen kann. Darin beruht ja der Werth der Analogie im Organon der inductiven Wissenschaften, dass so oft an weit entfernten Punkten die Thatsachen, ihre Gründe und Wirkungen die nämlichen sind, dass sie aber das eine Mal klarer zu Tage liegen, als das andere. Und gerade die Sprachen unserer Familie leisten in der Verhüllung Erstaunliches, strengen den Fleiss des geschichtlichen Forschers, den Scharfsinn des Denkers da an, wo andere ihren Mechanismus und ihre Geschichte offen zur Schau tragen.

In erster Reihe ist dies Buch für Jene bestimmt, die wir dereinst als Mitarbeiter und Nachfolger zu sehen hoffen. Das möge es entschuldigen, wenn ich den hodegetischen Fragen mehr Raum gegönnt \fed{{\textbar}IV{\textbar}} habe, als dies sonst wohl in Werken verwandten Inhalts üblich ist. Auch das Äusserlichste der Methodik lässt sich ja von Innen heraus erklären. Dagegen habe ich im Interesse der Kürze Manches weggelassen, was man in einem wirklichen Lehr- und Handbuche suchen dürfte: einen Abriss der Phonetik, eine Sammlung von \sed{{\textbar}{\textbar}IV{\textbar}{\textbar}} Definitionen grammatischer Ausdrücke, eine familienweise Übersicht der Sprachen und wohl noch manches Andere.

Unsere Wissenschaft selbst ist noch jung; viele ihrer Gebiete sind kaum erst von Forschern berührt, manche bieten noch den Reiz und die Gefahren eines jungfräulichen Bodens. Dem musste ich vor Allem Rechnung tragen. Der Leser soll sehen, wie schnell ein zielbewusst beharrliches Schaffen tüchtige Früchte gezeitigt hat, und soll an diesen Früchten seinen Theil Mitgenuss haben. Er soll sich aber um Alles nicht einbilden, wir wären schon weiter, als wir sind. Die höchsten und letzten Ziele möge er vor Augen haben. Soweit ich sie zu erkennen glaubte, habe ich auf sie hingewiesen; umschrieben habe ich das ganze Gebiet, soweit ich es ermass, und von dem Rechte des Kartographen, sein Gradnetz auch durch die Terra incognita zu ziehen, habe ich ausgiebigen Gebrauch gemacht. So gut es ging, tastete ich das Reich der Möglichkeiten aus, verfuhr dabei oft apriorisch, suchte aber dann, wenn es meine Erfahrungen erlaubten, an Beispielen das Mögliche als thatsächlich zu erweisen.

Hierin war ich nun besonders schlimm daran. Meine eigenen Erfahrungen und die allgemeineren Schlüsse, die ich daraus zog, habe ich natürlich zumeist in sehr abgelegenen Sprachgebieten geschöpft. Beispiele aber sollen erläutern und müssen möglichst einleuchten, darum möglichst nahe liegen. So musste ich wohl oder übel die meinigen da entnehmen, wo ich von Berufswegen am Wenigsten zu suchen habe: aus der Muttersprache, den bekanntesten Sprachen Europas und der Indogermanistik. Manchmal ging es mir, wie einem Ausländer, der lieber die \retro{Landes\-münze}{Landes\-mütze} thaler- und nickelweise borgt, als sein mitgebrachtes Geld mit Coursverlust ausgiebt; und das unbehagliche Gefühl, das man mit sich schleppt, wenn man bei den Nachbarn wissenschaftliche Anleihen macht, habe ich gründlich kennen gelernt. Da kann ich nun jene Besserberechtigten, deren etwaigen Tadel ich erwarte, nur um \retro{freund\-liche Nachsicht und und um noch freundlichere}{freund\-lichere} Nachhülfe bitten, für den Fall, dass mein Werk eine zweite Auflage erleben sollte. Dies gilt besonders vom dritten Buche.

\fed{{\textbar}V{\textbar}}

Wo es sich dagegen um die Äusserungen der lebendigen Muttersprache handelte, da habe ich geglaubt, meinem Gefühle und Urtheile ebensoviel zutrauen zu dürfen, wie dem Anderer. Umfrage habe ich dann wohl gehalten, um mich zu vergewissern, aber nicht immer bei denen, die gewohnt sind in den Sprachen mit dem Auge des Palimpsestenforschers Doppeltexte zu lesen. Man wird \fed{mich} nicht missverstehen, wenn ich die Gesichtspunkte der einzelsprachlichen und der sprachgeschichtlichen Forschung einander recht schroff \update{entgegen\-setze.}{entgegen\-setzte.} Die Gleichberechtigung Beider erkenne ich ja an, und ich suche zu zeigen, wie die Beiden sich am Ende ineinander verweben müssen. Eben darum aber sehe ich vorerst die Fäden lieber scharf \retro{auseinander\-gehalten,}{aneinander\-gehalten,} als durcheinander gefitzt.

\sed{{\textbar}{\textbar}V{\textbar}{\textbar}}

Mein Buch ist in einer längeren Reihe von Jahren mit grossen Unterbrechungen entstanden, und seine Theile sind keineswegs in der Reihenfolge verfasst, in der sie nun vorliegen. Was mich eben beschäftigte, wurde, sobald es mir reif schien, als Aufsatz niedergeschrieben; mit der Zeit entstand der Plan zum Ganzen, ich hielt Vorlesungen über allgemeine Sprachwissenschaft und füllte je länger je mehr die Lücken meines Manuscriptes aus. Die Spuren einer solchen Entstehung lassen sich kaum verwischen, und ich hoffe, man werde dies entschuldigen. Ein Lehrbuch zu schreiben, etwa ein System, wie es \textsc{Heyse} unternommen, konnte ich nicht wagen. Besser schien es mir, den Leser schildernd und erörternd durch unsere Werkstatt zu führen und natürlich da am Längsten zu verweilen, wo ich selbst mit Vorliebe arbeite. Ich habe hier wenige Genossen, besonders auch unter meinen Landsleuten; und eben dies mag es rechtfertigen, dass ich meinen Standpunkt zur Geltung bringe, nachdem so manche unserer hervorragendsten Indogermanisten ihrerseits das Gleiche gethan. Ich suche Verständigung und thue mein \update{Bestes}{Bestes,} um sie zu finden; ich verlange nichts Besseres, als gegenseitige Anerkennung.

Mit Citaten habe ich einigermassen gekargt. Erstens wollte ich den Umfang des Buches möglichst einschränken, und zweitens mochte ich keinen Anlass zu Prioritätsstreitigkeiten geben, gegen die ich eine gewisse Abneigung hege. In der Geschichte der Wissenschaft kommt es wohl vor, dass Einer so nebenher einen wichtigen, folgenreichen Gedanken ausspricht, den erst viel später ein Anderer ausbeutet. Und dieser Andere kann ebensogut selbständiger Entdecker, als von Jenem angeregt gewesen sein. Erschöpfende Belesenheit masse ich mir nicht an, und sie ist bei \fed{{\textbar}VI{\textbar}} dem Umfange unserer Literatur kaum zu verlangen. Manches, was ich für mein Eigenstes halte, mag sich schon längst in den Werken Anderer vorfinden; und wenn ich es wirklich zum ersten Male zu Papier gebracht habe, so kann, ohne dass ich es mich entsinne, mein verewigter Vater der Urheber gewesen sein.

Auch die Polemik habe ich thunlichst vermieden. Nur wenige Male schien es mir geboten, mich ausdrücklich vor meinen Vorgängern zu verantworten; sonst habe ich mich damit begnügt, meine Meinungen, so gut es anging, für sich reden zu lassen. Manchmal auch mögen mir die abweichenden Ansichten Anderer überhaupt unbekannt geblieben sein. In Wettbewerb da zu treten, wo ich schon von Früheren das Beste geleistet sah, lag am wenigsten in meiner Absicht. Ich hätte aber auf den Zusammenhang des Ganzen und auf die relative Vollständigkeit meines Werkes verzichten, hätte, mit anderen Worten, eine Reihe von Abhandlungen statt eines Buches schreiben müssen, wenn ich in solchen Fällen ganz geschwiegen hätte.

Man wird bemerken, vielleicht missfällig bemerken, dass ich es liebe, meine Sätze auf die Spitze zu treiben. Aus Gefallen am Paradoxen geschieht dies \sed{{\textbar}{\textbar}VI{\textbar}{\textbar}} wahrhaftig nicht. Ich mag nur lieber mir das Argumentum ad absurdum selbst einhalten, als es mir von Anderen entgegenstellen lassen; und am Liebsten möchte ich zeigen, dass meine Gedanken auch bis in ihre letzten Schlussfolgerungen die Probe bestehen. Wo es sich vollends um die Aufstellung von Idealen handelt, da mag ich die Allerweltsweisheit, dass diese doch unerreichbar seien, gar nicht hören. Es gilt ja auch nicht, sie zu erreichen, sondern ihnen näher und immer näher zu kommen. Genug, wenn wir das Endziel und die nächste Wegesstrecke vor uns sehen und die Klüfte kennen, in die uns ein überhastetes Streben stürzen kann.

Nach Kräften habe ich den Bedürfnissen der Philologen und Sprachlehrer Rechnung getragen. Eine verfehlte Unterrichtsmethode kann dem Schüler den Lehrgegenstand für Lebenszeit verleiden; und verfehlt scheint es mir allemal zu sein, wenn bei jungen Köpfen mehr darauf \retro{abgezielt wird,}{abgezielt,} ihnen ein Wissen und Können beizubringen, als die Sehnsucht nach Wissen und Können zu wecken. Denn das Gelernte wird wieder verlernt, das gewonnene Interesse aber wächst und wirkt fort. Meine Klagen über Missgriffe im Sprachunterrichte waren schon gedruckt, ehe die Schulreform für Preussen in Angriff genommen war. Sind sie veraltet \fed{{\textbar}VII{\textbar}} oder verspätet, dann um so besser. Jene Widersacher unserer classischen Bildung, die sich darauf stützen, wie wenig anregend oft die Gymnasien gerade in ihren Hauptfächern wirken, haben leider bisher einen Schein Rechtens für sich. Der muss ihnen genommen werden; der Beweis muss geführt werden, dass die scheinbar trockenste Wissenschaft in Wahrheit eine der lebensvollsten und anregendsten ist. Was das griechisch-römische Alterthum für unsere wissenschaftliche, künstlerische und staatliche Gesittung gewesen, davon können hundert Reformer nicht einen Deut abhandeln. An den Philologen ist es, sie vollends zum Schweigen zu bringen. Gelingt es ihnen, den Sprachunterricht zu einer Schule des Verstandes und Geschmackes zu gestalten, so werden sie auch Geschmack und Verständniss für die Sprachstudien erwecken. Wir leben in einer Zeit der Monographien. Der Einzelne vergräbt sich zu gern in’s Einzelne, verliert den Zusammenhang mit dem Ganzen und klagt dann, wenn er sich vereinsamt sieht. Es ist entweder beschränkter Dünkel oder zimpferliche Scheu vor Dilettanterei, wenn man den Verkehr mit den Nachbarwissenschaften ablehnt und nicht da mitgeniessen will, wo man nicht mitschaffen kann.

Indem ich während des Druckes das Register anfertige, entdecke ich, dass ich mich doch öfter wiederholt habe, als mir lieb ist. Die Art, wie das Buch zu Stande gekommen, möge dies einigermassen entschuldigen. Oft musste ja schon um der Sache willen derselbe Gedanke an verschiedenen Stellen wiederkehren; und dann geschah es wohl, dass mir auch wieder derselbe Ausdruck oder dasselbe Beispiel als besonders bezeichnend vorschwebte und in die Feder floss. \fed{Auch ein paar wirkliche Fehler habe ich entdeckt und verzeichnet.}

\so{Berlin}, im Februar 1891.

\chapter*{\sed{Vorwort zur zweiten Auflage.}}

\begin{sloppypar} {\textbar}{\textbar}VII{\textbar}{\textbar} \sed{Hiermit übergebe ich die „Sprachwissenschaft“ meines verewigten Oheims \textsc{Georg von der Gabelentz} in vermehrter und verbesserter Gestalt den philologischen Fachkreisen und bitte dieselben, dieses Vermächtnis des allzufrüh heimgegangenen Gelehrten in wohlwollender Weise aufnehmen zu wollen. – Leider war es ihm nicht mehr vergönnt, den weiteren Ausbau dieses grossartig angelegten Werkes mit eigener Hand zu unternehmen. Er, der unserer jungen Wissenschaft so wunderbare Wege gezeigt und führend vorangegangen ist, musste gerade auf der Höhe seines Schaffens, im Vollbesitz des gewaltigen Materials auf sprachlichem Gebiet, den Schauplatz seines Wirkens nur zu bald verlassen. – Dem Ueberlebenden erübrigte es, mit schonender Hand das Geschaffene, soweit es irgend anging, zu erhalten, und nur da, wo der Fortschritt der Wissenschaft es dringend verlangte, Änderungen und Erweiterungen vorzunehmen.}
\end{sloppypar}

\sed{\so{Wildenroth} bei München, Mai 1901.
\begin{flushright}
\textsc{Dr. Graf Schulenburg}.
\end{flushright}}
