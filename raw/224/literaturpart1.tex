\WeitereLiteratur{

\begin{sloppypar}

\paragraph*{Grammatik und Linguistik}

Einführungen in die Linguistik wie \citet{MeibauerEa2015} liefern Diskussionen vieler grundlegender Begriffe.
Neben Kapitel~1 aus \citet{Eisenberg2013a} ist \citet{Engel2009} besonders einschlägig, der auch zur Valenz eine kurze, aber aufschlussreiche Diskussion liefert \citep[70--73]{Engel2009}.
Einen Einblick in die Sprachtheorie gibt \citet{Mueller2010}, ebenfalls mit ausführlicher Diskussion des Valenzbegriffs.
Die aktualisierte und stark erweiterte englische Fassung ist \citet{Mueller2018}.
Eine einführende Darstellung, die das Potential von Merkmal-Wert-Kodierungen formal ausschöpft, ist \citet{Mueller2008}.

\paragraph*{Grammatik in Schule und Studium}

Wie Kapitel~\ref{sec:grammatikimlehramtsstudium} vermuten lässt, empfehle ich \citet{Eisenberg2004} und \citet{Eisenberg2013c} zur Frage der Verortung von Grammatik in Schule und Studium.
Ebenso aufschlussreich ist \citet{Portmanntselikas2011}.
Weitere gur lesbare Artikel sind zum Beispiel \citet{Haecker2009} oder \citet{Ossner2007}.
Zum Begriff der Bildungssprache kann \citet{Feilke2012} gelesen werden.
Eine für den Lehrberuf nahezu unverzichtbare Lektüre sind \citet{Bredel2013}, \citet{BredelEa2017} und \citet{BredelPieper2015}.
Für eine praktische Anleitung zu einem schulischen Grammatikunterricht im Sinne dieses Buchs sollten \citet{Menzel2017} und als Hintergrund dazu \citet{EisenbergMenzel1995} konsultiert werden.

\paragraph*{Sprachnorm und Sprachkritik}

Zu Akzeptabiltität und Grammatikalität mit Bezug zum schulischen Deutschunterricht kann \citet{Koepcke2011} gelesen werden.
Sehr wichtige Beiträge zum Thema sind \citet{Eisenberg2008}, Kapitel~1 aus \citet{Eisenberg2013a} und Kapitel~2 aus \citet{Eisenberg2013b}.
Die Betrachtung der Sprachgeschichte hilft, zu verstehen, dass synchrone Sprachnormen niemals Absolutheitsanspruch haben können, und \citet{NueblingEa2010} wird als Einstieg in die relevante Literatur empfohlen.

\paragraph*{Empirie}

Einen kompakten Überblick über empirische Verfahren in der Linguistik bietet \citet{Albert2007}.
Die Korpuslinguistik wird in \citet{PerkuhnEa2012} einführend dargestellt.

\paragraph*{Valenz}

Alles Wesentliche zur klassischen Valenztheorie und verschiedenen Weiterentwicklungen kann der Einleitung von \citet{HelbigSchenkel1991} entnommen werden.
In diesem Buch findet sich auch ein Verzeichnis der Valenzmuster von deutschen Verben.
Ein weiteres Valenzlexikon der deutschen Sprache ist VALBU \citep{SchumacherEa2004}, inklusive einer Online-Fassung.

\end{sloppypar}

}
