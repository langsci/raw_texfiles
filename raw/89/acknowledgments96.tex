\addchap{Acknowledgments (1996)}
\epigraph{\itshape Just as the Navajo weavers purposely make one error in a rug, to let the soul out, so I cannily craft errors into all of my papers.\\[-2\baselineskip]}{\citep{Ross1979}}

\noindent This book is a substantially revised version of my University of Toronto M.A. Forum paper \citep{Schutze1991}. It has benefited enormously in both content and style from the contributions of several people. None of them bears any responsibility for its remaining flaws, cannily crafted or otherwise; they are all the fault of that little person who runs around inside my computer making it work. First and foremost, I would like to thank my supervisors at Toronto, Peter Reich and Graeme Hirst, without whose comments and criticisms a far inferior product would have resulted. Peter enthusiastically supported my academic work for several years, and has supported this project in particular since August 1990, when we both discovered to our surprise and delight that there is a literature on the topic of grammaticality judgments. Graeme was invaluable in pointing out relevant work in fields that I was unaware of, in tracking down current unpublished research, and in his meticulous scrutiny of my prose. He was most generous with his time and energy, despite innumerable other priorities. At MIT, Noam Chomsky provided extensive comments on the penultimate version of the manuscript, adding important historical perspectives, especially for the first two chapters, and helping me to see the big issues in a more critical light. For this I am very grateful. 

\enlargethispage{\baselineskip}
Several other people have commented on part or all of the manuscript at various stages, including Tom Wasow, James McCawley, Wayne Cowart, Tom Bever, and Jila Ghomeshi. At the University of Chicago Press, Geoffrey Huck encouraged me to turn the paper into a book, provided many helpful suggestions on how to do so, and supported me every step of the way. I am indebted to him for this wonderful opportunity. Karen Peterson edited the manuscript, vastly improving its readability and lucidity, and cheerfully answered my incessant questions about the process. Thanks to Karen and Geoff, the publication process has been a pleasure. David Braun, Colin Phillips, Jonathan Bobaljik, and Orin Percus helped with the proofreading. 

Much of the groundwork for this book was laid in the course of an M.A. Forum class at the University of Toronto, and I owe thanks to my fellow participants. Elan Dresher, our forum supervisor, supplied encouragement and skepticism in just the right doses to keep us working steadily. He also read drafts of several portions of the paper, providing a perspective that would otherwise have been lacking. His open-mindedness and sense of humor were a boon to us all. My fellow forum students, Amy Green, Päivi Koskinen, and Ana Palma dos Santos, commented on my work and, more importantly, provided camaraderie as we faced our tasks together.

Several other people have contributed in important ways to this book. I thank Elizabeth Cowper for useful discussions at the beginning of this project, and for her advice on portions of the book that deal with syntactic theory. Susanne Carroll brought to my attention one of the most important sources on this topic, \citealt{Birdsong1989}. Charles Houpt at Cornell shared his thoughts and course papers with me and encouraged me to pursue this project. I would also like to acknowledge various \textsc{usenet} readers who contributed pointers to the literature.


The research reported in this book was financially supported by a postgraduate scholarship from the Natural Sciences and Engineering Research Council of Canada while I was at the University of Toronto. At MIT my research was supported by the Research Training Grant ``Language: Acquisition and Computation'' awarded by the National Science Foundation (US) to the Massachusetts Institute of Technology (DIR 9113607), by a doctoral fellowship from the Social Sciences and Humanities Research Council of Canada, and by an Imperial Oil Fulbright Scholarship. Their support of my research in cognitive science is gratefully acknowledged.