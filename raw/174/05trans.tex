\section{Trans-Himalayan}\label{sec:5.9}
\subsection{Classification of Trans-Himalayan}\label{sec:5.9.1}

This study includes languages from three of the 42 subbranches of \ili{Trans-Himalayan} (\citealt{vanDriem2014}: 10). These branches are \ili{Sinitic}, Bodish, and \ili{Qiangic}. Of Bodish, only the \ili{Tibetic} subbranch will be included here. \textit{Glottolog} (Hammarström et al. 2016) mentions 475 \ili{Trans-Himalayan} languages, which are almost all located to the south of \isi{NEA}. The exact relation of the individual subbranches is somewhat disputed and is of no particular concern here. Despite certain controversies (\citealt{Kurpaska2010}: 25–62), \ili{Sinitic} is usually divided into seven often mutually incomprehensible main dialect areas called Gan (gàn \zh{赣}), Hakka (\textit{kèjiā} \zh{客家}), \ili{Mandarin} (\textit{guān} \zh{官}), Min (\textit{mǐn} \zh{闽}), Wu (\textit{wú} \zh{吴}), Xiang (\textit{xiāng} \zh{湘}), and Yue (\textit{yuè} \zh{粤}). The existence of separate Pinghua (\textit{pínghuà} \zh{评话}), Jin (\textit{jìn} \zh{晋}), and Hui (\textit{hu\={\i}} \zh{徽}) dialects is somewhat disputed. Of these, \ili{Mandarin} is the largest and, if one includes Jin and Hui within it, the only one located in \isi{NEA}. Recent migrations of speakers of other dialects will not be considered. \ili{Mandarin} itself may be classified into several regional varieties, of which the Southeastern area in and around Sichuan, Chongqing, Guizhou, and \isi{Yunnan} as well as the Jianghuai subdialect area around the lower Yangtze are mostly excluded from this study. An exhaustive overview of \isi{questions} in all the remaining \ili{Mandarin} subdialects is impossible to give because of a lack of high quality materials. The focus will lie on a description of Standard \ili{Mandarin}, to which will be added a sample of regional dialects for which good data was available, especially for interrogatives. Special cases to be addressed are \ili{Dungan} (\textit{dōnggān} \zh{东干}, \textit{dunganskij}/\textcyrillic{дунганский}), \ili{Gangou} (\textit{gāngō}\textit{u} \zh{甘沟}), Hezhou (\textit{hézhō}\textit{u} \zh{河州}) or Linxia (\textit{línxià} \zh{临夏}), \ili{Tangwang} (\textit{tángwāng} \zh{唐汪}), and \ili{Wutun} (\textit{wǔtún} \zh{五屯}). With the exception of \ili{Dungan}, spoken in several Central Asian countries, but derived from Northwest \isi{China}, these languages are all spoken in the Qinghai-\isi{Gansu} area and exhibit a strong influence from \ili{Turkic}, \ili{Mongolic}, or \ili{Tibetic}.

\ili{Tibetic} alone encompasses about 200 different varieties, divided into eight different groups or sections (\citealt{Tournadre2005}; 2014). Here only \ili{Amdo Tibetan} (\textit{ānduō} \zh{安多}) dialects and \ili{gSerpa} (\textit{sè’ěrbà} \zh{色尔坝}) from the northeastern, as well as \ili{Baima} (\textit{báimǎ} \zh{白马}), \ili{Cone} (Chone, \textit{zhuō}\textit{ní} \zh{卓尼}), and \ili{Zhongu} from the eastern section will be included. I currently lack sufficient data for some varieties such as Khalong \ili{Tibetan} from the eastern section spoken in northern Sichuan (see \citealt{Sun2007}).\footnote{There seems to be no official \ili{Chinese} name for \ili{Zhongu} or Khalong yet.} Both \ili{Zhongu} and \ili{Baima} are sometimes considered \ili{Qiangic} instead of \ili{Tibetic}. \ili{Amdo Tibetan} is one of the more dominant languages of the \isi{Amdo Sprachbund} (\citealt{SandmanSimon2016}) and is said to have at least 23 different subdialects \citep[43]{Ebihara2011}. Given the limited amount of information available, only a fraction of this variation can be included here. Of the disputed \ili{Qiangic} branch of \ili{Trans-Himalayan}, only the extinct \ili{Tangut} language, the major language of the Xixia empire (1038-1227), was located in \isi{NEA}. The language was only rediscovered and deciphered in the 20\textsuperscript{th} century \citep{Gong2003}.

Descriptions of \ili{Chinese} dialects frequently suffer from a lack of accuracy and \isi{analysis}, the use of characters for transcription, and an incoherent use of characters for etymological and phonological purposes. This unfortunate and confusing situation could only be partly remedied here. \ili{Chinese} dialectal or historical data that was available in \ili{Chinese} characters exclusively will be transcribed with the official romanization system Pinyin without tones in square brackets.

\subsection{Question marking in Trans-Himalayan}\label{sec:5.9.2}
\subsubsection{Question marking in Sinitic}\label{sec:5.9.2.1}

\textit{Old Chinese} had a sentence-final polar \isi{question marker} \zh{乎} [\textit{hu}] (see \ref{ex:trans:1}) that has been reconstructed as *\textit{ɢˤa} by \citet{BaxterSagart2014b}.

\ea%1
    \label{ex:trans:1}
    \ili{Old Chinese}\\
    \zh{賢者亦樂此乎}\\
    \gll [xian    zhe  yi  le  ci \textbf{{hu}}]\\
    virtuous  \textsc{n}  also  enjoy  this  \textsc{q}\\
    \glt ‘Does a man of virtue also enjoy such (things)?’ \citep[139]{Pulleyblank1995}
    \z

\noindent Its form is somewhat reminiscent of interrogatives that will be described in the next section. There are several other question markers that appear to be contractions of [\textit{hu}] \zh{乎} with other elements and will not be discussed any further here (see \citealt{Pulleyblank1995}: 139ff.). Content \isi{questions} were generally unmarked.

\ea%2
    \label{ex:trans:2}
    \ili{Old Chinese}\\
    \zh{何必曰利}\\
    \gll [\textbf{{he}} bi  yue  li]\\
    why  must  say  profit\\
    \glt ‘Why must you say ‘profit’’? \citep[145]{Pulleyblank1995}
    \z

Of course, there is a lot of variation to be found in historical stages of \ili{Chinese}, but a detailed examination of \isi{diachronic} developments that necessarily also includes all modern \ili{Sinitic} languages goes well beyond the possibilities of this study. Instead, the following will address exclusively the modern \ili{Sinitic} languages located in \isi{NEA}.

\textbf{Standard Mandarin} \ili{Chinese} data are partly based on my own knowledge and were partly elicited or confirmed in 2015 and 2016 with the help of a native speaker from Guiyang, fluent in both the dialect and standard \ili{Mandarin}, living in Germany. \ili{Mandarin} \ili{Chinese} has many different \isi{question marking} strategies, but the default form is the sentence-final marker \textit{ma} \zh{吗} that marks \isi{polar question}s.

\ea%3
    \label{ex:trans:3}
    \ili{Mandarin}\\
    \gll ch\={\i}-fàn   le \textbf{{ma}}?\\
    eat-food  \textsc{pfv}  \textsc{q}\\
    \glt ‘Have you eaten yet?’
    \z

Polar \isi{questions} may also simply be marked with rising \isi{intonation}, though this seems less common. Content {questions} are usually unmarked morphosyntactically and have an \isi{intonation} contour similar to a \isi{declarative sentence}, but with additional \isi{emphasis} of the \isi{interrogative}.

\ea%4
    \label{ex:trans:4}
    \ili{Mandarin}\\
    \gll nǐ  jiào \textbf{{shénme}}?\\
    2\textsc{sg}  call  what\\
    \glt ‘\isi{What is your name?}’
    \z

The marker \textit{ba} \zh{吧} has a function similar to a \isi{polar question} and for practical purposes is classified as such. But it has an additional element of supposition by the speaker. In some instances it is best translated as a \isi{tag question} in \ili{English}.

\ea%5
    \label{ex:trans:5}
    \ili{Mandarin}\\
    \gll nǐ  ch\={\i}-fàn    le \textbf{{ba}}?\\
    2\textsc{sg}  eat-food  \textsc{pfv}  \textsc{q}\\
    \glt ‘You must have already eaten (I suppose)?’
    \z

In another function, \textit{ba} \zh{吧} is also an imperative marker. In its function as \isi{question marker}, which is similar to \ili{English} \textit{must} in relatively certain hypotheses, it has been adopted by a great many languages in \isi{China}, perhaps because of its very specific semantic nuances. Both \textit{ma} and \textit{ba} can be the actual \isi{question marker} in complex \isi{tag question}s.

\ea%6
    \label{ex:trans:6}
    \ili{Mandarin}\\
    \ea
    \gll nǐ  ch\={\i}-fàn    le, \textbf{duì} \textbf{ba}?\\
    2\textsc{sg}  eat-food  \textsc{pfv}    right  \textsc{q}\\
    \glt ‘You have already eaten, right?’
    
    \ex
    \gll nǐ  bāng  wǒ  mǎi.dōngx\={\i}, \textbf{hǎo}  \textbf{ba}?\\
    2\textsc{sg}  help  1\textsc{sg}  shop    good  \textsc{q}\\
    \glt ‘You are going shopping for me, right?’
    \z
    \z

\noindent In both cases \textit{ba} may be substituted with the more neutral \textit{ma}, which is accompanied with a slight change in meaning. Several more patterns are possible, e.g. \textit{shì ma?} \zh{是吗} ‘\textsc{cop} \textsc{q}’, \textit{kěyǐ ma?} \zh{可以吗} ‘be.possible \textsc{q}’, \textit{xíng ma?} \zh{行吗} ‘be.possible \textsc{q}’.

\ili{Mandarin} has a further colloquial marker \textit{ne} \zh{呢} that has both an \isi{interrogative} and non-\isi{interrogative} function (\citealt{LiThompson1981}: 300–307). In its \isi{interrogative} function it has an interesting distribution. It can be found in \isi{negative alternative question}s (A-not-A), \isi{content question}s, and “truncated” or elliptical “\isi{questions} consisting of only one noun” (\citealt{LiThompson1981}: 305) such as \textit{tā} \textbf{\textit{ne}}? ‘(And) how about him/her?’.

\ea%7
    \label{ex:trans:7}
    \ili{Mandarin}\\
    \ea
    \gll nǐ  qù  bu  qù  zhōngguó \textbf{{ne}}?\\
    2\textsc{sg}  go  \textsc{neg}  go  \textsc{pn}    \textsc{q}\\
    \glt ‘Are you (actually) going to \isi{China}?’
    
    \ex
    \gll nà  nǐ  qù \textbf{{nǎli}} \textbf{{ne}}?\\
    that  2\textsc{sg}  go  where  \textsc{q}\\
    \glt ‘Well, where are you going then?’
    \z
    \z

\noindent The last function (\ref{ex:trans:7}b) is what might be called a \isi{topic question}, the exact meaning of which depends on the previous discourse. Yet another connection of \textit{ne} \zh{呢} with \isi{questions} is a special type of \isi{alternative question} construction that we will encounter further below.

As for \isi{focus question}s, there are several possible patterns. One is a cleft-like structure with the copula \textit{shì} \zh{是} and occurs before the focused element. Compare the following examples of a polar and two \isi{focus} \isi{questions}. If the copula stands sentence-finally, it functions as a \isi{question tag}.

\ea%8
    \label{ex:trans:8}
    \ili{Mandarin}\\
    \ea
    \gll nǐ  qù  zhōngguó, \textbf{{shì}} \textbf{{ma}}?\\
    2\textsc{sg}  go  \textsc{pn}    \textsc{cop}  \textsc{q}\\
    \glt ‘You are going to \isi{China}, are you?’
    
    \ex
    \gll {nǐ} \textbf{{shì}} qù  zhōngguó \textbf{{ma}}?\\
    2\textsc{sg}  \textsc{cop}  go  \textsc{pn}    \textsc{q}\\
    \glt ‘Is it \textit{China} that you are going to?’
    
    \ex
    \gll \textbf{{shì}} nǐ  qù  zhōngguó \textbf{{ma}}?\\
    \textsc{cop}  2\textsc{sg}  go  \textsc{pn}    \textsc{q}\\
    \glt ‘Is it \textit{you} who is going to \isi{China}?’\z\z

\noindent The same sentences are possible with the marker \textit{ba} \zh{吧}. Another marker for \isi{focus question}s is likewise based on the copula but has itself a structure of an \isi{A-not-A question}, \textit{shì-bu-shì} \zh{是不是} ‘\textsc{cop}-\textsc{neg}-\textsc{cop}’, which is why no additional \isi{question marker} is present. It is treated as a single marker here that stands before the focused element or attaches to the sentence and functions as a \isi{question tag}.

\ea%9
    \label{ex:trans:9}
    \ili{Mandarin}\\
    \ea
    \gll nǐ  qù  zhōngguó, \textbf{{shìbushì}}?\\
    2\textsc{sg}  go  \textsc{pn}    \textsc{q}\\
    \glt ‘You are going to \isi{China}, aren’t you?’
    
    \ex
    \gll {nǐ} \textbf{{shìbushì}} qù  zhōngguó?\\
    2\textsc{sg}  \textsc{q}    go  \textsc{pn}\\
    \glt ‘Is it \textit{China} that you are going to?’
    
    \ex
    \gll \textbf{{shìbushì}} nǐ  qù  zhōngguó?\\
    \textsc{q}    2\textsc{sg}  go  \textsc{pn}\\
    \glt ‘Is it \textit{you} who is going to \isi{China}?’\z\z

\noindent In the latter two examples (\ref{ex:trans:9}b, \ref{ex:trans:9}c), \textit{shì-bu-shì} may be replaced with the more literary \textit{shì-fǒu} \zh{是否}, a \isi{combination} of the copula with an otherwise uncommon negator. This is part of the literary language and cannot function as a \isi{question tag}. But there are several related question tags such as \textit{duì-}\textit{bu-}\textit{duì} \zh{对不对} ‘correct-\textsc{neg}-correct’, \textit{hǎo-bu-hǎo} \zh{好不好} ‘good-\textsc{neg}-good’ \textit{xíng-bu-xíng} \zh{行不行} ‘be.possible-\textsc{neg}-be.possible’, \textit{kě(yǐ)-bu-kěyǐ} \zh{可以不可以} ‘be.possible-\textsc{neg}-be.possible’, or \textit{huì-}\textit{bu-}\textit{huì} \zh{会不会} ‘be.able-\textsc{neg}-be.able’, all of which are of the A-not-A type but do not usually mark \isi{focus question}s. This is evidence that \textit{shì-bu-shì} actually has the status of a sentence (\isi{A-not-A question}) in \isi{tag question}s, but of a \isi{question marker} in polar and \isi{focus} \isi{questions}.

Alternative \isi{questions} have two main construction types, either mere \isi{juxtaposition} or the use of an \isi{interrogative} disjunctive \textit{háishì} \zh{还是} (different from the standard disjunctive \textit{huòzhě} \zh{或者}) (cf. \citealt{Hölzl2016a}: 20).

\ea%10
    \label{ex:trans:10}
    \ili{Mandarin}\\
    \ea
    \gll zh\=en.de  Ø  jiǎ.de?\\
    true    (or)  false\\
    \glt ‘Is that true or false?’

    \ex
    \gll nǐ  qù  zhōngguó \textbf{{háishì}} {rìběn?}\\
    2\textsc{sg}  go  \textsc{pn}    or.\textsc{q}    \isi{Japan}\\
    \glt ‘Are you going to \isi{China} or \isi{Japan}?’
    \z
    \z

One very dominant question category in \ili{Mandarin} \ili{Chinese} are \isi{negative alternative question}s that exhibit the same two marking strategies as plain alternative \isi{questions}. Juxtaposition is much more frequent and productive in negative \isi{alternative question}s (A-not-A questions) than in plain \isi{alternative question}s (\citealt[20]{Hölzl2016a}).

\ea%11
    \label{ex:trans:11}
    \ili{Mandarin}\\
    \gll nǐ  qù  zhōngguó  (\textbf{{háishì}}) \textbf{{bú}} qù  zhōngguó?\\
    2\textsc{sg}  go  \textsc{pn}    or.\textsc{q}    \textsc{neg}    go  \textsc{pn}\\
    \glt ‘Are you going to \isi{China} or are you not going to \isi{China}?’
    \z

In some cases the whole second alternative is deleted except for \isi{negation}. This may be the basis for the \isi{grammaticalization} of \isi{interrogative} particles such as \textit{ma} \zh{吗}. Note that the following two sentences are completely identical in structure. The major difference appears to be the fact that \textit{ma} is restricted to this construction while \textit{bù} \zh{不} is still a productive negative marker otherwise. But note that in this context sometimes it had already lost its tone, which is a sign of \isi{grammaticalization} (i.e., phonetic erosion, \citealt{Hölzl2015b}).

\ea%12
    \label{ex:trans:12}
    \ea
    \ili{Mandarin}\\
    \gll {nǐ} {qù} {zhōngguó} \textbf{{bu}}?\\
    2\textsc{sg}  go  \textsc{pn}    \textsc{neg}\\
    \glt
    
    \ex
    \gll {nǐ} {qù} {zhōngguó} \textbf{{ma}}?\\
    2\textsc{sg}  go  \textsc{pn}    \textsc{q}\\
    \glt ‘Are you going to China?’
    \z
    \z

In the past tense or in sentences that contain the existential \textit{yǒu} \zh{有}, it is also possible to replace the second alternative with the negative existential. In the former, case an additional marker in the first alternative is necessary. Following \cite[64–70]{SunChaofen2006}, these may glossed as experiential (\textsc{exp}) \textit{guo} \zh{过} and perfective (\textsc{pfv}) \textit{le} \zh{了}, respectively.

\ea%13
    \label{ex:trans:13}
    \ili{Mandarin}\\
    \gll nǐ  qù-\textbf{{guò}}\textbf{{/le}} {zhōngguó} \textbf{{méi}}\textbf{.}\textbf{{yǒu}}?\\
    ’2sg  go-\textsc{exp/pfv}  \textsc{pn} \textsc{neg}\\
    \glt ‘Have you been to \isi{China} or not?’
    \z

My informant tells me that, in the first case, the number of times one has been to \isi{China} as well as the exact time is irrelevant, while in the latter the event is thought to have happened once and relatively recently. Note that the second alternative may not consist of the \isi{disjunction} and a negator, exclusively (*\textit{háishì bu?}) (\citealt{LuoTianhua2013}: 186), which represents a difference with respect to MSEA \citep{Clark1985}.

In extreme cases the whole second alternative is deleted and alternativity is indicated with the help of the disjunctive connective \textit{háishì} \zh{还是} ‘or.\textsc{q}’, exclusively. These developments crucially depend on the context of an elliptical (more precisely analiptic) \isi{alternative question}. \ili{Mandarin} also allows \isi{open alternative question}s.

\ea%14
    \label{ex:trans:14}
    \ili{Mandarin}\\
    \gll {nǐ} {qù} {zhōngguó} \textbf{{háishi}} \textbf{{nǎ.r}}?\\
    2\textsc{sg}  go  \textsc{pn}    or.\textsc{q}  where\\
    \glt
    \z

\noindent In this case, my informant tells me, one of the interrogatives \textit{nǎ} \zh{哪}/\textit{nǎ.r} \zh{哪儿}/\textit{nǎ.li} \zh{哪里} ‘where’ would seem more natural than \textit{shénme} \zh{什么} ‘what’, which, however, is possible in other examples.

An element that can often be encountered in \ili{Mandarin} \isi{questions} is the sentence-final marker \textit{a} \zh{啊} {\textasciitilde} \textit{ya} \zh{呀} that I analyze as enclitic \textit{=(y)a}. It is not a \isi{question marker} as such but “has the semantic effect of softening the query” (\citealt{LiThompson1981}: 313). Some examples will be given further below. Following these authors, the enclitic will be glossed as reduced forcefulness (\textsc{rf}).

As we have just seen, \isi{question marking} in \ili{Mandarin} is relatively complex and exhibits many different constructional patterns. The same is probably true for the other \ili{Sinitic} languages surveyed in the following. But in the absence of native speakers and a lack of detailed information, only some limited information can be given here for each language. \ili{Chinese} as spoken by the Hui (\ili{Chinese} speaking Muslims) in \textbf{Urumqi} is relatively close to Standard \ili{Mandarin}. Polar questions have a cognate of \textit{ma} \zh{吗} and \isi{content question}s optionally have a cognate of \textit{ne} \zh{呢}. As expected there are also \isi{negative alternative question}s.

\ea%15
    \label{ex:trans:15}
    \ili{Mandarin} (Urumqi Hui)\\
    \ea
    \gll {tʂəŋ}\textsuperscript{24}{.ti}\textsuperscript{21}  \textbf{{ma}}\textbf{\textsuperscript{21}}?\\
    real    \textsc{q}\\
    \glt ‘Really?’
    
    \ex
    \gll {ȵi}\textsuperscript{52}  {ʂã}\textsuperscript{44}  \textbf{{nɐr}}\textbf{\textsuperscript{24}}  {tɕ‘y}\textsuperscript{44}?\\
    2\textsc{sg}  \textsc{dir}  where  go\\
    \glt ‘Where are you going?’
    
    \ex
    \gll \textbf{{sei}}\textbf{\textsuperscript{24}}  {tɕiɤu}\textsuperscript{44}    {ʋɤ}\textsuperscript{52}  \textbf{{nə}}\textbf{\textsuperscript{21}}?\\
    who  save    1\textsc{sg}  \textsc{q}\\
    \glt ‘Who will save me (then)?’
    
    \ex
    \gll {ȵi}\textsuperscript{52}  \textbf{{nəŋ}}\textbf{\textsuperscript{24}} \textbf{{pu}}\textbf{\textsuperscript{21}} \textbf{{nəŋ}}\textbf{\textsuperscript{24}}    {lɛ}\textsuperscript{24}?\\
    2\textsc{sg}  be.able    \textsc{neg}  be.able    come\\
    \glt ‘Are you able to come?’ (\citealt{LiuLiji1989}: 222, 219, 206, 217)\z\z

Similar to Standard \ili{Mandarin}, \isi{question tag}s may contain a \isi{question marker}, e.g. \textit{xɔ}\textsuperscript{52}\textit{pa}\textsuperscript{21} \zh{好吧} or may have the form of an \isi{A-not-A question}.

\ea%16
    \label{ex:trans:16}
    \ili{Mandarin} (Urumqi Hui)\\
    \gll {tʂ‘ʅ}\textsuperscript{21}  {liɔ}\textsuperscript{24}  {f\~{æ}}\textsuperscript{ 44}  {tsɛ}\textsuperscript{44}  {tɕ‘y}\textsuperscript{44},  \textbf{{xɔ}}\textbf{\textsuperscript{52} }\textbf{{pu}}\textbf{\textsuperscript{21}} \textbf{{xɔ}}\textbf{\textsuperscript{52}}?\\
    eat  \textsc{pfv}  meal  again  go  good  \textsc{neg}  good\\
    \glt ‘Let’s go after having eaten, alright?’ (\citealt{LiuLiji1989}: 221)
    \z

In the following example of an \isi{alternative question} \REF{ex:trans:17}, the first alternative receives the marker \textit{ne} \zh{呢} that combines with the \isi{disjunction} and necessarily is preceded by the copula that precedes the first focused element.

\ea%17
    \label{ex:trans:17}
    \ili{Mandarin} (Urumqi Hui)\\
    \gll {ȵi}\textsuperscript{52}  \textbf{{sɿ}}\textbf{\textsuperscript{21}}  {tʂ‘ɤu}\textsuperscript{24}{j\~{æ}}\textsuperscript{21}  \textbf{{ȵi}}\textbf{\textsuperscript{44}} \textbf{{xɛ}}\textbf{\textsuperscript{24}}\textbf{{sɿ}}\textbf{\textsuperscript{21}}    {xɤ}\textsuperscript{21}  {ts‘a}\textsuperscript{24}?\\
    2\textsc{sg}  \textsc{cop}  smoke    \textsc{q}  or    drink  tea\\
    \glt ‘Do you (want to) smoke or drink tea?’ (\citealt{LiuLiji1989}: 221)
    \z

\noindent This has an exact parallel in Standard \ili{Mandarin} (\textit{nǐ} \textit{shì} \textit{chōuyān} \textit{ne} \textit{háishì} \textit{h\=e} \textit{chá?}).

An idiosyncratic pattern is the presence of a \isi{question marker} on the first alternative in alternative \isi{questions} that may also lack a cognate of \ili{Mandarin} \textit{háishì} \zh{还是}. This pattern was probably influenced by surrounding \ili{Turkic} languages. Especially intriguing is the optional \isi{combination} of two markers, which is impossible in Standard \ili{Mandarin} but can also be found in Hezhou \ili{Chinese} (see \tabref{tab:trans:1} below).

\ea%18
    \label{ex:trans:18}
    \ili{Mandarin} (Urumqi Hui)\\
    \gll {ni}\textsuperscript{52}  {ta}\textsuperscript{44}  \textbf{{ȵi}}\textbf{\textsuperscript{21}}  \textbf{{ma}}\textbf{\textsuperscript{21}}  {ʋɤ}\textsuperscript{52}  {ta}\textsuperscript{44}?\\
    2\textsc{sg}  big  \textsc{q}  \textsc{q}  1\textsc{sg}  big\\
    \glt ‘Are you older or I?’ (\citealt{LiuLiji1989}: 211)
    \z

\textbf{Xining Mandarin} has unmarked content \isi{questions}, though cognates of \ili{Mandarin} \textit{=(y)a} \zh{啊}/\zh{呀} and \textit{ne} \zh{呢} are optionally present. Polar \isi{questions} have a marker \textit{mɔ}\textsuperscript{53} which does not appear to be a cognate of \ili{Mandarin} \textit{ma} \zh{吗}, which is attested as \textit{ma} (\citealt{ZhangChengzai1980}: 300). The difference with Standard \ili{Mandarin} is mostly phonological in nature. The formal \isi{similarity} to the \ili{Tangut} sentence-final \isi{question marker} \textit{mo}\textsuperscript{2} is probably accidental.

\ea%19
    \label{ex:trans:19}
    Xining \ili{Mandarin}\\
    \ea
    \gll \textbf{{fei}}\textbf{\textsuperscript{2421}}  {ia}\textsuperscript{4453}?\\
    who  \textsc{rf}\\
    \glt ‘Who (is it)?’
    
    \ex
    \gll {lɔ}\textsuperscript{53}  {sɿ}\textsuperscript{21321}  \textbf{{lɛ}}\textbf{\textsuperscript{53}}?\\
    \textsc{hon}  four  \textsc{q}\\
    \glt ‘What about Laosi (the fourth brother)?’
    
    \ex
    \gll {tɕia}\textsuperscript{44}  {xa}\textsuperscript{35}  {mɔ}\textsuperscript{35}  {fɔ}\textsuperscript{44}  {uã}\textsuperscript{3521} \textbf{{mɔ}}\textbf{\textsuperscript{53}}?\\
    3\textsc{sg}  yet  \textsc{neg}  speak  finish    \textsc{q}\\
    \glt ‘Has (s)he still not finished speaking?’ (\citealt{ZhangChengzai1980}: 300)\z\z

There is also the development from \isi{negation} to question markers. In the following example Standard \ili{Mandarin} employs the affirmative potential marker (\textit{ná-}\textbf{\textit{de}}\textit{-dòng}).

\ea%20
    \label{ex:trans:20}
    Xining \ili{Mandarin}\\
    \ea
    \gll {na}\textsuperscript{24}  {tuə} \textsuperscript{213}  {lia}\textsuperscript{53}  (\textbf{{pṿ}}\textbf{\textsuperscript{44}})?\\
    take  move  \textsc{?pot}  \textsc{neg}\\
    \glt ‘Can you take it?’
    
    \ex
    \gll {ȵi}\textsuperscript{53}  {fã}\textsuperscript{213}  {tʂ‘ʅ}\textsuperscript{44}  {liɔ}\textsuperscript{1}  \textbf{{mɔ}}\textbf{\textsuperscript{24}}?\\
    2\textsc{sg}  meal  eat  \textsc{pfv}  \textsc{neg}\\
    \glt ‘Have you eaten?’ (\citealt{ZhangChengzai1980}: 301)
    \z
    \z

The negators are cognates of \ili{Mandarin} \textit{bù} \zh{不} (non-past) and \textit{méi} \zh{没} (past). A native speaker living in Germany in January 2017 made me aware of the fact that the use of \textit{mɔ}\textsuperscript{24} in example (\ref{ex:trans:20}a) is not only possible, but perhaps more natural. In April 2017 the following examples of alternative and \isi{focus question}s were recorded. The transcription and \isi{analysis} roughly follow \citet{ZhangChengzai1980}. Given that the speaker appears to show strong influence from Standard \ili{Mandarin}, tones were omitted for simplicity.

\ea%21
    \label{ex:trans:21}
    Xining \ili{Mandarin}\\
    \ea
    \gll ni  tʂuŋku    ts‘ɿ-lia \textbf{{ma}} \textbf{{xai}}\textbf{{ʂ}}\textbf{{ʅ}} ʐɿpən  ts‘ɿ-lia?\\
    2\textsc{sg}  \textsc{pn}    go-?\textsc{pot}  \textsc{q}  or.\textsc{q}  \textsc{pn}  go-?\textsc{pot}\\
    \glt ‘Are you going to \isi{China} or to \isi{Japan}?’
    
    \ex
    \gll tʂuŋku    tɕ‘y  tʂuɔtsɿ {ʂʅ} {ni} \textbf{{sa}}?\\
    \textsc{pn}    go  ?\textsc{n}    \textsc{cop}  2\textsc{sg}  \textsc{q}\\
    \glt ‘Is it you who is going to \isi{China}?’
    \z
    \z

\noindent Example (\ref{ex:trans:21}a) apprently contains cognates of \ili{Mandarin} \textit{ma} \zh{吗} and \textit{háishì} \zh{还是}. The \isi{question marker} \textit{sa} in example (\ref{ex:trans:21}b) appears to also exist in Hezhou \ili{Chinese} (see below).

\textbf{Dungan} \isi{questions} appear to be very close to \ili{Mandarin} as well. In the first two examples the original was in traditional characters that have been changed into simplified characters. In the last example the \ili{Chinese} characters have been added by me. \ili{Dungan} is usually written with Cyrillic letters, however.

\ea%21
    \label{ex:trans:22}
    \ili{Dungan}\\
    \ea
    \zh{你们欧洲东干人多吗}?\\
    \gll [ni-men  ouzhou    donggan  ren  duo \textbf{{ma}}?]\\
    2\textsc{sg}-\textsc{pl}    \textsc{pn}    \textsc{pn}    person  many  \textsc{q}\\
    \glt ‘Are there many Dunggan in \isi{Europe}?’
    
    \ex
    \zh{咋的呢}?\\
    \gll [\textbf{{za}}-de \textbf{{ne}}?]\\
    how-\textsc{attr}  \textsc{q}\\
    \glt ‘How are you?’ (\citealt{Rimsky-Korsakoff1994}: 486, 515)
    
    \ex{}
    [\zh{第三段儿话里头狒的啥}?]\\
    \gll {ti} {san} {tua.r} {xua} li.t’i  fe-ti \textbf{{sa}}?\\
    \textsc{ord}  three  \textsc{clf}  speech    inside  say-\textsc{adv}  what\\
    \glt ‘What is said in the third sentence?’ (\citealt{Rimsky-Korsakoff1967}: 382)\footnotemark
    \footnotetext{My \ili{Mandarin} informant made me aware of the fact that \zh{狒} [\textit{fei}] is usually employed as a character for ‘to say’ in \isi{Gansu} province.}
    \z
    \z

Question marking in \textbf{Hezhou/Linxia} \ili{Chinese} is very complex and deviates strongly from Standard \ili{Mandarin}. \tabref{tab:trans:1} summarizes the specialized description of question markers by \citet{XieZhang1990}. Especially interesting is a functional differentiation of three different question markers for polar and \isi{content question}s each. The marker \textit{mu}\textsuperscript{3} most likely was borrowed from \ili{Uyghur} \textit{=mu} (\sectref{sec:5.11}). The markers \textit{la}\textsuperscript{3} and \textit{ʐa}\textsuperscript{3} apparently found their way into some \ili{Mongolic} languages (\sectref{sec:5.8}). The \isi{combination} of two question markers such as \textit{ȵi}\textsuperscript{3}\textit{mu}\textsuperscript{3} is similar to Urumqi Hui \ili{Chinese}. A \isi{double marking} pattern for \isi{alternative question}s most likely has been adopted from \ili{Turkic} as well.

\begin{table}
\caption{Hezhou/Linxia Chinese question markers (\citealt{XieZhang1990}: passim); notation of vowels slightly adjusted; elements given in characters only are rendered here in Pinyin without tones in square brackets as an approximation}
\label{tab:trans:1}

\begin{tabularx}{\textwidth}{llQ}
\lsptoprule

\textbf{Type} & \textbf{Form} & \textbf{Usage}\\
\midrule
PQ & \textit{ma}\textsuperscript{3} & with \isit{negation}\\
& \textit{la}\textsuperscript{3} & with assertion\\
& \textit{ȵi}\textsuperscript{3}\textit{mu}\textsuperscript{3} & polite, solemn, younger towards older speakers\\
CQ & \textit{ʐa}\textsuperscript{3} & \textsc{person, thing, quantity}\\
& \textit{ȵi}\textsuperscript{3} & \textsc{place, time, manner}, male and young speakers\\
& \textit{ȵi}\textsuperscript{3}\textit{ʐa}\textsuperscript{3} & \textsc{place, time, manner}, female and old speakers, more polite than \textit{ȵi}\textsuperscript{3}\\
AQ & X \textit{ȵi}\textsuperscript{3}\textit{mu}\textsuperscript{3}, Y \textit{ȵi}\textsuperscript{3} & anticipated actions\\
& X \textit{liɔ}\textsuperscript{3}\textit{mu}\textsuperscript{3}, Y \textit{liɔ}\textsuperscript{3} & past actions\\
& X \textit{ȵi}\textsuperscript{3}\textit{mu}\textsuperscript{3}, Y \textit{ȵi}\textsuperscript{3}, [\textit{haishi}] Z & \ilit{Mandarin} \textit{háishì} ‘or.\textsc{q}’\\
& X \textit{ȵi}\textsuperscript{3}, Y, [\textit{haishi}] Z & \\
& X \textit{mu}\textsuperscript{3}, Y \textit{mu}\textsuperscript{3}, [\textit{haishi}] Z & \\
NAQ & X \textit{ȵi}\textsuperscript{3}\textit{mu}\textsuperscript{3}, \textsc{neg} (X) & \\
& X \textit{la}\textsuperscript{3}, \textsc{neg} (X) & \\
?TQ & \textit{ȵi}\textsuperscript{3}\textit{ʂa}\textsuperscript{3} & \\
 & \textit{(tʂ)ɤ}\textsuperscript{3}\textit{ȵi}\textsuperscript{3}\textit{ʂa}\textsuperscript{3} & \\
& [\textit{jiushi}]\textit{la}\textsuperscript{3} & \ilit{Mandarin} \textit{jiùshì} ‘exactly’\\
& [\textit{duizhe}]\textit{la}\textsuperscript{3} & \ilit{Mandarin} \textit{duì} ‘correct’\\
& [\textit{bushi}]\textit{pɛ}\textsuperscript{3} & \ilit{Mandarin} \textit{búshì} ‘isn’t’\\
& [\textit{jiushi}]\textit{la}\textsuperscript{3}[\textit{shi}] & relative certainty\\
& X \textit{la}\textsuperscript{3}[\textit{bu}](X)[\textit{shi}] & \ilit{Mandarin} \textit{bù} ‘\textsc{neg}’\\
& \textit{ʂa}\textsuperscript{3} & \\
\lspbottomrule
\end{tabularx}
\end{table}

The following two examples of a \isi{polar question} with the \ili{Uyghur} \isi{question marker} (\ref{ex:trans:23}a) as well as an unmarked \isi{content question} (\ref{ex:trans:23}b) were given without characters and tones.

\newpage
\ea%22
    \label{ex:trans:23}
    Hezhou/Linxia\\
    \ea
    \gll {fatsɪ-xa}    kɛ-v\~{ɛ}-li=\textbf{{mu}}?\\
    house-\textsc{acc}  build-finish-\textsc{pfv}=\textsc{q}\\
    \glt ‘Is the house built?’
    
    \ex
    \gll ni  tham-xa \textbf{{ʃɪma}}-la  khu\~{ɛ}-tɛ-li?\\
    2\textsc{sg}  2\textsc{pl}-\textsc{acc}  what-\textsc{inst}  wait-?-?\textsc{int}\\
    \glt ‘What would you serve them with?’ (\citealt{Lee-Smith1996b}: 866, 868)
    \z
    \z

\citet[158]{Dwyer1995} claims that the \textbf{Xunhua} subdialect of the Hezhou/Linxia \ili{Chinese} has a \isi{tag question} marker that derives from an \isi{interrogative} meaning ‘what’. The only example given is a \isi{content question}, however, and more likely, it is cognate with \textit{ʐa}\textsuperscript{3} seen in \tabref{tab:trans:1}. Content \isi{questions} may also be unmarked.

  
\ea%23
    \label{ex:trans:24}
    Hezhou/Linxia (Xunhua)\\
    \ea
    \gll {ŋɔ}\textsuperscript{24}{-m}{\~{ɛ}} \textsuperscript{44}  {phai-ʂã}    \textbf{{aaʒi}}\textbf{\textsuperscript{14}}\textbf{{gə}}\textbf{\textsuperscript{23}}  {{\textquotedbl}p}{\~{æ}}\textsuperscript{ 44}    {dʐə}\textsuperscript{23}{-gə ʂʅtʃʰ}{\~{ɪ} }  {ʂʅ}  {xeɯ}  \textbf{{sa}}\textsuperscript{41}?\\
    1\textsc{sg}-\textsc{pl}    send-\textsc{res}  which    accomplish  this-\textsc{clf} matter  \textsc{cop}  good  \textsc{q}\\
    \glt ‘Whom should we send to take care of this?’
    
    \ex
    \gll {ɲi}\textsuperscript{53} \textbf{{ʂə}}\textbf{\textsuperscript{13}}\textbf{{ma}}\textbf{\textsuperscript{41}}\textbf{{kə}}  {mai}\textsuperscript{53}  {liɔ}?\\
    2\textsc{sg}  what    buy  \textsc{pfv}\\
    \glt ‘What did you buy?’ (\citealt{Dwyer1995}: 158, 162)
    \z
    \z

The negator in the following \isi{negative alternative question} (A-not-A) \REF{ex:trans:25} is what has been called a potential (\textsc{pot}) meaning, cf. \ili{Mandarin} \textit{zhǎo-}\textbf{\textit{bu}}\textit{-dào} ‘seek-\textsc{neg}.\textsc{pot}-\textsc{res}’ ‘not be able to find’. But the affirmative counterpart would usually require another marker to substitute for the negator, cf. \ili{Mandarin} \textit{zhǎo-}\textbf{\textit{de}}\textit{-dào} ‘seek-\textsc{neg}.\textsc{pot}-\textsc{res}’ ‘be able to find’ (\citealt{SunChaofen2006}: 60f.).

\ea%24
    \label{ex:trans:25}
    Hezhou/Linxia (Xunhua)\\
    \gll {dʒə}\textsuperscript{24}-gɤ  {{\textquotedbl}dʐ}{ũən}\textsuperscript{41}{-dzɪ}  {x\~{ɛ}}\textsuperscript{42}, {ɲi}\textsuperscript{53}  \textbf{{na}}\textbf{\textsuperscript{14}}\textbf{{-xa}}\textbf{\textsuperscript{41}}\textbf{{-la}} {na}\textsuperscript{14}{-}\textbf{{bu}}{-xa}\textsuperscript{41}?\\
    this\textsc{-clf}  heavy-\textsc{ext}  very 2\textsc{sg}  carry-\textsc{res}-\textsc{q}    carry-\textsc{neg.pot}-\textsc{res}\\
    \glt ‘This is very heavy, can you carry it or not?’ \citep[173]{Dwyer1995}
    \z

\noindent Following \citet{XieZhang1990}, the marker \textit{-la} has been reanalyzed as a \isi{question marker} here. \citet{Dwyer1995} does not give an example of a \isi{polar question}.

\textbf{Wutun} has a polar \isi{question marker} \textit{-a} that has been compared to both \ili{Mandarin} \textit{ma} \zh{吗} as well as the \isi{interrogative} \textit{a-} ‘which’ (\ili{Mandarin} \textit{nǎ} \zh{哪}) (\citealt{Janhunen2008}: 99). Another possible source might be \ili{Mandarin} \textit{=(y)a} \zh{啊}/\zh{呀}. The marker has been called an enclitic particle, but was written attached to the preceding word with a hyphen. It is analyzed as enclitic here and thus written as \textit{=a}. Similar to some \ili{Mongolic} languages in the area, the marker fuses with certain preceding suffixes, which speaks in favor of an \isi{analysis} as a suffix. For instance, the continuative marker \textit{-zhe}, combined with the \isi{question marker} results in the form \textit{-zha}, which is reminiscent of \ili{Mongolic} languages of the region (\sectref{sec:5.8.2}). Apart from \textit{=a}, \ili{Wutun} allegedly has borrowed the suffix \textit{-mu} from a \ili{Mongolic} source, probably \ili{Bonan} \citep[384]{Sandman2012}. But a \ili{Turkic} origin of the \ili{Wutun} form is more likely, e.g. \ili{Salar} \textit{=mu}, \ili{Uyghur} \textit{=mu} (\sectref{sec:5.11}). Consequently, it has been reanalyzed as enclitic here. According to \citet[894]{Lee-SmithWurm1996} it has the form \textit{-mɵ} and is cognate with \ili{Mandarin} \textit{ma}, which seems unlikely but not impossible. As seen above, it has the form \textit{mɔ}\textsuperscript{24} in Xining \ili{Mandarin}. \ili{Wutun} lacks a tonal distinction.

\ea%25
    \label{ex:trans:26}
    \ili{Wutun}\\
    \ea
    \gll je  ni-de    huaiqi    hai-li=\textbf{{a}}?\\
    this  2\textsc{sg}-\textsc{gen}  book    \textsc{equ}-\textsc{sen.}\textsc{inf}=\textsc{q}\\
    \glt ‘Is this your book?’
    
    \ex
    \gll sama    qe-lio=\textbf{{mu}}?\\
    food    eat-\textsc{pfv=q}\\
    \glt ‘Have you eaten the food?’ (\citealt{Janhunen2008}): 99, 100)
    \z
    \z

The functional distribution of the two suffixes remains somewhat unclear, but \textit{=mu} is said to have been encountered less frequently \citep{Janhunen2008}. Erika Sandman (p.c. 2016) informed me that \textit{=a} is used with imperfective as well as progressive, and \textit{=mu} with perfective as well as resultative aspect. Content \isi{questions} remain unmarked.

\ea%26
    \label{ex:trans:27}
    \ili{Wutun}\\
    \gll {ni} \textbf{{ma}}-ge nian-di-yek?\\
    2\textsc{sg}  what-\textsc{clf}  read-\textsc{progr}-\textsc{ego}\\
    \glt ‘What are you reading?’ (\citealt{Janhunen2008}: 98)
    \z

The following examples in \REF{ex:trans:28} were kindly provided by Erika Sandman (p.c. 2016), see also \cite[287–297]{Sandman2016}.

\ea%27
    \label{ex:trans:28}
    \ili{Wutun}\\
    \ea
    \gll ni  xaitangwa  hai-yek=\textbf{{mu}} ggaigan  hai-yek?\\
    2\textsc{sg}  student    \textsc{equ}-\textsc{ego}=\textsc{q}    teacher    \textsc{equ}-\textsc{ego}\\
    \glt ‘Are you a student or a teacher?’
    
    \ex
    \gll zang  jja-la-gu-la {waiqai} {yek,} \textbf{{mi}}{-yek}?\\
    \textsc{pn}  visit-\textsc{incompl}-\textsc{comp}-\textsc{cond}  hardship  \textsc{ex}  \textsc{neg.ex}-?\textsc{ego}\\
    \glt ‘If you visited Tibet, would there be any hardships?’
    
    \ex
    \gll ni  yan {za-de} {yek} \textbf{{ya}}?\\
    \textsc{2sg} tobacco  smoke-\textsc{n} \textsc{ex  q}\\
    \glt ‘You are a smoker, aren’t you?’
    
    \ex
    \gll {nia}\textbf{{-ha}} {dun-li=}\textbf{{a}}?\\
    2\textsc{sg.}\textsc{obl}-\textsc{top}  cold-\textsc{sen.}\textsc{inf}=\textsc{q}\\
    \glt ‘And how about you, are \textit{you} feeling cold?\z\z

Whether this last example (\ref{ex:trans:28}d) can be analyzed as \isi{focus question}s remains somewhat dubious. Similar to \ili{Japanese} (\sectref{sec:5.6.2}) and \ili{Korean} (\sectref{sec:5.7.2}), it is perhaps better analyzed as a \isi{polar question} with an additional topic. Similar to \ili{Tibetic} there seems to be \isi{interaction} with egophoricity (\sectref{sec:5.9.2.2}). The egophoric marker \textit{-yek} (< \textit{yek} = \ili{Mandarin} \textit{yǒu} \zh{有}) is “typically used with the first person in statements when the \isi{action} is volitional and allows the speaker’s control” \citep[209]{Sandman2016}. There are certain other reasons in which the egophoric marker can be used for non-first person \citep[222]{Sandman2016}, but none of them seems to apply in examples \REF{ex:trans:27}, (\ref{ex:trans:28}a), or \REF{ex:trans:29}.

\newpage 
\ea%28
    \label{ex:trans:29}
    \ili{Wutun}\\
    \ea
    \gll \textbf{{ngu}}-de    minze-li  dongzhek  sho-\textbf{yek}.\\
    1\textsc{sg}-\textsc{gen}  name-\textsc{loc}  \textsc{pn}    say-\textsc{ego}\\
    \glt ‘My name is Dongzhou.’
    
    \ex
    \gll \textbf{{ni}}-de    minze-li \textbf{ma} {sho-}\textbf{yek}.\\
    2\textsc{sg}-\textsc{gen}  name-\textsc{loc}  what    say-\textsc{ego}\\
    \glt ‘\isi{What is your name?}’ (\citealt{Sandman2016}: 294f.)
    \z
    \z

\noindent Thus, \ili{Wutun} appears to follow the \isi{anticipation rule} as described in \sectref{sec:4.4} (\citealt{TournadreLaPolla2014}: 245; \citealt{Sandman2016}: 294). \citet[226]{Sandman2016} mentions another interesting \isi{interaction} with \isi{evidentiality}: “The \isi{questions} with factual evidential [\textit{re}] differ from \isi{questions} that are true requests for information. In my data, factual evidential was used in \isi{questions} with an obvious \isi{answer}”.

\textbf{Tangwang} has also adopted the \ili{Uyghur} \isi{polar question} marker \textit{=mu} and has unmarked \isi{content question}s. Examples provided by \citet{Xu2014} were given without tones. \citet{Lee-Smith1996c} does not mention any \isi{questions}.

 
\ea%29
    \label{ex:trans:30}
    \ili{Tangwang}\\
    \ea
    \gll ȵi  nə {tsuə=}\textbf{{mu}}?\\
    2\textsc{sg}  be.able    ?live=\textsc{q}\\
    \glt ‘Can you live?’ (\citealt{Xu2014}: 357)
    
    \ex
    \gll ȵi-m    tɕia.tsu    tɕi  xuə  jəu-lɪ,\textbf{ {mə}}{-jəu?}\\
    2\textsc{sg}-\textsc{pl}    family    \textsc{gen}  life  \textsc{ex}-?\textsc{ipfv}  \textsc{neg-?ex}\\
    \glt ‘Does your family have a life (worth living) or not?’
    
    \ex
    \gll {tʂʅ}\textsuperscript{31}-{ʃie}\textsuperscript{31} {ʂu}\textsuperscript{24}{-xa} \textbf{{a}}\textbf{\textsuperscript{24}}\textbf{{ṃ}} {mɛ}\textsuperscript{31}{-li?}\\
    this-\textsc{pl}    book-?\textsc{top}  how  sell-\textsc{ipfv}\\
    \glt ‘How does one sell these books?’ \citep[36]{Yibulaheimai1985}\z\z

There are also very few recordings of \isi{questions} in \textbf{Gangou} \ili{Chinese}. Content \isi{questions} remain unmarked. Alternative \isi{questions} appear to have just one marker on the first alternative. Polar \isi{questions} most likely have the sentence-final marker \textit{ma}, but no example was found.

\ea%30
    \label{ex:trans:31}
    \ili{Gangou}\\
    \ea{}
    \gll {aijie} \textbf{{amen}} mei-lai    shuo?\\
    3\textsc{sg}  how  \textsc{neg}-come  \textsc{hs}\\
    \glt ‘Why didn’t he come?’
    
    \ex{}
    \gll ni  liang.ge  huaer    yao-li \textbf{{ma}}, \textbf{{bu}}{-yao?}\\
    2\textsc{sg}  two    flower    want-\textsc{fut}  \textsc{q}  neg-want\\
    \glt ‘Do you two want a flower or not?’ (\citealt{ZhuYongzhong1997}: 447, 437)
    \z
    \z

\subsubsection{Question marking in Tibetic and Qiangic}\label{sec:5.9.2.2}

\cite[64f.]{Ebihara2011} mentions several different ways of forming \isi{questions} in \textbf{\isi{Amdo} Tibetan}. Polar and content \isi{questions} may optionally take a sentence-final clitic \textit{=ni}. Its origin eludes me, but one might compare it with Hezhou/Linxia \textit{ȵi}\textsuperscript{3} \zh{呢}.

\ea%31
    \label{ex:trans:32}
    \ili{Amdo Tibetan} (Gonghe; \ili{Tibetic})\\
    \ea
    \gll tɕʰo  wə    joŋ=\textbf{{ni}}?\\
    2\textsc{sg}  go.out.\textsc{prf}  come=\textsc{q}\\
    \glt ‘You came back?’
    
    \ex
    \gll \textbf{{tɕ}}\textbf{{ʰ}}\textbf{{əzəka}} {ma-joŋ=}\textbf{{ni}}?\\
    why \textsc{neg}-come=\textsc{q}\\
    \glt ‘Why did (you) not come?’ \citep[65]{Ebihara2011}
    \z
    \z

Content \isi{questions} may also remain unmarked and polar \isi{questions} have another enclitic \textit{=na}. The distribution of the two markers among polar \isi{questions} remains unclear.

  
\ea%32
    \label{ex:trans:33}
    \ili{Amdo Tibetan} (Gonghe; \ili{Tibetic})\\
    \ea
    \gll {tɕʰ}{o} \textbf{{s}}\textbf{{ʰ}}\textbf{{ə}} {jən?}\\
    2\textsc{sg}  who  \textsc{cop.cj}\\
    \glt ‘Who are you?’
    
    \ex
    \gll {tɕʰ}o  demo  jən=\textbf{{na}}?\\
    2\textsc{sg}  fine  \textsc{cop.cj}=\textsc{q}\\
    \glt ‘Are you alright?’ (a greeting) \citep[65]{Ebihara2011}
    \z
    \z

Polar \isi{questions} alternatively may be marked with \isi{intonation} exclusively. There is yet another marking strategy for polar \isi{questions} that is almost unique within the Northeast Asian area: \ili{Amdo Tibetan} possesses a verbal prefix for marking polar \isi{questions}. Again, there is no comment on the functional distribution of this marking strategy with respect to the others. But this might be the default marking.

\ea%33
    \label{ex:trans:34}
    \ea
    \ili{Amdo Tibetan} (Gonghe; \ili{Tibetic})\\
    \gll {tɕʰ}o  wol \textbf{{ə}}{-jən?}\\
    2\textsc{sg}  \textsc{pn}  \textsc{q}-\textsc{cop.cj}\\
    \glt ‘Are you \ili{Tibetan}?’ (\citealt{Ebihara2011}: 70, 64)
    
    \ex
    \gll norbə  joŋ=dʑi \textbf{{ə}}{-re?}\\
    \textsc{pn}  come=\textsc{n}  \textsc{q}-\textsc{cop.dj}\\
    \glt ‘Will Norbu come?’
    \z
    \z

\ea%34
    \label{ex:trans:35}
    \ili{Amdo Tibetan} (dPa’ris; \ili{Tibetic})\\
    \gll {tɕʰ}o:    $\chi $\textsuperscript{w}{iɕʰ}{a} \textbf{{ə}}{-jol?}\\
    2\textsc{sg.dat}  book    \textsc{q}-\textsc{cop.cj}\\
    \glt ‘Do you have a book?’ \citep[155]{Ebihara2013}
    \z

In the examples provided by Ebihara, the prefix always attaches to a copula, but it is not restricted to this context. Consider two examples from the Themchen dialect \REF{ex:trans:36}.

\ea%35
    \label{ex:trans:36}
    \ili{Amdo Tibetan} (Themchen)\\
    \ea
    \gll {tɕʰ}o  tsʰ{uŋwa} \textbf{{ə}}{-jən?}\\
    2\textsc{sg}  merchant  \textsc{q}-\textsc{cop.cj}\\
    \glt ‘Are you a merchant?’
    
    \ex
    \gll {bdəʂtɕəl} \textbf{{ə}}{-t}{ʰ}a?\\
    \textsc{pn}    \textsc{q}-go.\textsc{pfv.dj}\\
    \glt ‘Has Dekyi gone?’ (\citealt{Haller2004}: 69, 81)
    \z
    \z

\citet[184]{Janhunen2012a} is correct that the prefix represents an important difference of Amdo-\ili{Tibetan} when compared with the other languages of the \isi{Amdo Sprachbund}. But as seen before, \ili{Amdo Tibetan} has sentence-final particles as well, and as we will see further below there are other languages in the region with a similar pattern. Basically, the same pattern as in the dialects mentioned above is also found in other dialects such as that of Tongren/Rebgong \ili{Amdo Tibetan}, e.g. \textit{tɕʰ}\textit{o demō} \textit{jin=}\textbf{\textit{na}}? ‘How are you?’ (see \ref{ex:trans:33}b), and \textbf{\textit{e}}\textit{-jol’} ‘\textsc{q}-\textsc{cop.cj}’ (see \ref{ex:trans:35}) (\citealt{deRoerich1958}: 98, modified transcription). The descriptions of \ili{Tibetic} languages included here usually do not mention alternative, \isi{focus} or \isi{tag question}s. However, Tongren/Rebgong \ili{Amdo Tibetan} has a \isi{tag question} marker \textit{e-den-gʌ} ‘\textsc{q}-truth-?\textsc{gen}’ (\citealt{deRoerich1958}: 131, modified transcription).

\ili{Amdo Tibetan} has a distinction between conjunct and disjunct marking that usually is manifested in the copula system. The distinction has also been adopted from \ili{Tibetic} by several \ili{Mongolic} (\sectref{sec:5.8.2}) and \ili{Sinitic} (see above) languages of the \isi{Amdo Sprachbund} (\sectref{sec:3.5}). According to \citet[471]{Aikhenvald2012} “the alternation between conjunct and disjunct person marking marks new information and surprise, especially in 1st person contexts. The disjunct person marking indicates something out of the speaker’s control, unexpected and thus surprising.” As we have already seen in \sectref{sec:4.4}, there is some \isi{interaction} of conjunct/disjunct marking and \isi{questions}: “In question sentences for the second person, the conjunct forms are generally used according to the point-of-view of the second person” \citep[69]{Ebihara2011}. In Gonghe \ili{Amdo Tibetan}, for example, there are special conjunct (\textit{jən}, \textsc{neg} \textit{mən}) and disjunct (\textit{re(l)}, \textsc{neg} \textit{mare(l)}) copula forms \citep[69]{Ebihara2011}, see also examples (\ref{ex:trans:33},\ref{ex:trans:34}, \ref{ex:trans:35},\ref{ex:trans:36}a) above.

The \isi{intonation} of \ili{Amdo Tibetan} \isi{questions} has been given in quite some detail by \citet{Sun1986} for the dialect spoken in Xəra village in Northern Sichuan:

\begin{quote}
The \isi{interrogative} word is spoken on a high falling pitch, or, if the \isi{interrogative} word has more than one syllable, a high falling pitch on the last syllable and high level pitch on the other syllables. [...] The typical \isi{intonation} of yes-no \isi{questions} is a high level pitch on /ɤ/ followed by a high falling tune realized on the verbal element. (\citealt{Sun1986}: 60f.)
\end{quote}

\noindent What is given here as /ɤ/ corresponds here to the question prefix \textit{ə-} in Gonghe and other dialects (cf. \citealt{Sun1993}: 959). According to \citet[128]{Denwood1999}, Lhasa \ili{Tibetan} <e> /ʔʌ/, apparently a cognate of \textit{ə-}, generally has dubitative meaning but also functions as a polite \isi{question marker} for the second person.

\textbf{Classical Tibetan} has a sentence-final \isi{question marker} \textit{‘am} that also assimilates to the preceding word. Content \isi{questions} remain unmarked.
\ea%36
    \label{ex:trans:37}
    \ili{Tibetan} (Classical; \ili{Tibetic})\\
    \ea
    \gll mi  de  dgra  yin=\textbf{{nam}}?\\
    person  that  enemy  \textsc{cop=q}\\
    \glt ‘Is that man an enemy?’
    
    \ex
    \gll kyod  kyis  glang  brnyas=\textbf{{sam}}?\\
    2\textsc{sg}  \textsc{erg}  ox  borrow=\textsc{q}\\
    \glt ‘Did you borrow (his) ox?’
    
    \ex
    \gll nga-s    khyod  kyi  khyo \textbf{{ci.ltar}} {sbyin?}\\
    1\textsc{sg}-\textsc{erg}  2\textsc{sg}  \textsc{gen}  husband  how  give\\
    \glt ‘How can I give you back your husband?’ (\citealt{DeLancey2003}: 262, 267)\z\z

According to \citet[267]{DeLancey2003}, the \ili{Classical Tibetan} sentence-final polar \isi{question marker} \textit{‘am} “represents a \isi{reduction} of an earlier balanced question construction, probably *V \textit{‘o ma-}V ‘V(or) not-V?’ > V \textit{‘am} ‘V’”. Thus, the development is from a negative \isi{alternative question} construction to a polar \isi{question marker}. \isi{Amdo} \textit{=na} seems to correspond to \ili{Classical Tibetan} \textit{=nam} (\citealt{deRoerich1958}: 98).

\ili{Zhongu} \REF{ex:trans:39}, \ili{Baima} \REF{ex:trans:40}, and \ili{Tangut}  \REF{ex:trans:41} also have a verbal prefix for \isi{polar question}s and unmarked \isi{content question}s. But \ili{Tangut}, like \ili{Amdo Tibetan}, also has a sen\-tence-final \isi{question marker} \textit{mo}\textsuperscript{2}, which formally looks similar to the marker in Xining \ili{Mandarin} \REF{ex:trans:37}. 

\ea%37
    \label{ex:trans:38}
    \ili{Tangut} (\ili{Qiangic})\\
    \gll {nji}\textsuperscript{2} {kã}\textsuperscript{1} {tśja}\textsuperscript{1} {wjɨ}\textsuperscript{2} {dzjo}\textsuperscript{1} \textbf{{mo}}\textsuperscript{2}?\\
    2\textsc{sg}.\textsc{hon}  sugar  cane  \textsc{perf}  eat  \textsc{q}\\
    \glt ‘Do you eat sugar cane?’ \citep[614]{Gong2003}
    \z

The \ili{Baima} \isi{question marker} has been reanalyzed as a prefix here in analogy to the other languages.

\ea%38
    \label{ex:trans:39}
    \ili{Zhongu} (?\ili{Tibetic})\\
    \ea
    \gll \textbf{{ɐ}}{-sə-kə?}\\
    \textsc{q}-sour-\textsc{mir}\\
    \glt ‘Is it sour?’
    
    \ex
    \gll {tsʰ}o(-sɐ)  gomo \textbf{{tʃ}}\textbf{{ʰ}}\textbf{{atsə}} {\textsuperscript{n}}{də}{rə?}\\
    2\textsc{sg}(-\textsc{dat})  money    how.much  \textsc{ex}\\
    \glt ‘How much money do you have?’ (\citealt{Sun2003a}: 826)
    \z
    \z

\newpage     
\ea%39
    \label{ex:trans:40}
    \ili{Baima} (?\ili{Tibetic})\\
    \ea
    \gll {tɕho}\textsuperscript{13}{ko}\textsuperscript{53} {ŋge}\textsuperscript{13}{re}\textsuperscript{35} \textbf{{e}}\textsuperscript{53}{-ndʑi}\textsuperscript{53} {i}\textsuperscript{53}?\\
    2\textsc{pl}    \textsc{pn}    \textsc{q}-go    ?\textsc{progr}\\
    \glt ‘Are you going to \ili{Baima}?’
    
    \ex
    \gll {tɕhø}\textsuperscript{53} \textbf{{su}}\textsuperscript{341} {re}\textsuperscript{13}?\\
    2\textsc{sg}  who  \textsc{cop}\\
    \glt ‘Who are you?’ (\citealt{Sun1996}: 131, 126)
    \z
    \z

\ea%40
    \label{ex:trans:41}
    \ea
    \ili{Tangut} (\ili{Qiangic})\\
    \gll {mə}\textsuperscript{1} {ˑ}{jij}\textsuperscript{1} {ɣu}\textsuperscript{1} \textbf{{ˑ}}\textbf{{ja}}{-tɕhjɨ}\textsuperscript{1}{-dju}\textsuperscript{1}?\\
    sky  \textsc{gen}  head  \textsc{q}-?\textsc{pot}-have\\
    \glt ‘Does the sky have a head?’ \citep[427]{Jacques2011}
    
    \ex
    \gll \textbf{{thjij}}\textbf{\textsuperscript{2}}\textbf{{sjo}}\textbf{\textsuperscript{2}} {thjj}\textsuperscript{2} {dzjwo}\textsuperscript{2} {tjịj}\textsuperscript{1} ˑ{jij}\textsuperscript{1} {rjur}\textsuperscript{1}ˑ{ar}\textsuperscript{2} {mjj}\textsuperscript{1} {njwi}\textsuperscript{2} ˑ{jj}\textsuperscript{2}?\\
    why    this  person  alone  \textsc{acc}  restrain    \textsc{neg}  can \textsc{comp}\\
    \glt ‘Why can’t you restrain this person alone?’ \citep[614]{Gong2003}
    \z
    \z

It seems possible that the preverbal \isi{question marker} is an areal feature. Some other \ili{Qiangic} languages share the same \isi{question marking} strategy as well. Since all \ili{Qiangic} languages today are located outside of \isi{NEA}, some examples should suffice (\ref{ex:trans:42}, \ref{ex:trans:43}, \ref{ex:trans:44}).

\ea%41
    \label{ex:trans:42}
    Caodeng \ili{rGyalrong} (\ili{Qiangic})\\
    \gll {nɐ{ȷ}}{iʔ} {nɐ-lŋaʔ} \textbf{\'{ə}}-toʔ?\\
    2\textsc{sg}  2\textsc{sg.poss}-child  \textsc{q}-exist\\
    \glt ‘Do you have children?’ (\citealt{Sun2003b}: 498)
    \z

\ea%42
    \label{ex:trans:43}
    Jiaomuzu \ili{rGyalrong} (\ili{Qiangic})\\
    \gll {no} {tʃha} {wutə} \textbf{{ə}}-tə-mut-w?\\
    2\textsc{sg}  tea  this  \textsc{q}-2-drink-2\textsc{sg}\\
    \glt ‘Will you drink this tea?’ \citep[593]{Prins2017}
    \z

\ea%43
    \label{ex:trans:44}
    \ili{Guiqiong} (\ili{Qiangic})\\
    \gll {zo} {gutɕhiɐŋ} \textbf{{ɛ}}-{dʐi?}\\
    3\textsc{sg}  \textsc{pn}    \textsc{q}-\textsc{cop}\\
    \glt ‘Is he a \ili{Guiqiong}?’ (\citealt{Jiang2015}: 304)
    \z

It can also be found in further \ili{Amdo Tibetan} dialects and other \ili{Tibetic} languages of the region such as the \ili{gSerpa} variety in northwestern Sichuan.

\ea%44
    \label{ex:trans:45}
    \ili{Amdo Tibetan} (Xəra; \ili{Tibetic})\\
    \gll {tɕʰ}o  xabda \textbf{{ə}}{-s}{ʰ}{oŋ?}\\
    2\textsc{sg}  deer.chase  \textsc{q}-go.\textsc{com}\\
    \glt ‘Did you go deer-hunting?’ (\citealt{Sun1993}: 959)
    \z

\ea%45
    \label{ex:trans:46}
    \ili{gSerpa} \ili{Tibetan} (\ili{Tibetic})\\
    \gll martsɔ    həts{ʰ}o-lə    rdʒəvɯa  $\chi $tsə-ve \textbf{{ə}}\'{-ɣɔ?}\\
    originally  2\textsc{pl}.family-\textsc{dat}  louse    one-except  \textsc{q}-\textsc{cop}\\
    \glt ‘So there was just one single ‘louse’ in your home?’ (\citealt{Sun2006}: 125)
    \z

Possibly, there are areal connections to \ili{Chinese} dialects, too. Consider the following example from the Hefei dialect spoken in central Anhui province.

\ea%46
    \label{ex:trans:47}
    \ili{Chinese} (Hefei)\\
    \gll [{ni} \textbf{{ke-}}{xiang.xin}]\\
    2\textsc{sg}  \textsc{q-}believe\\
    \glt ‘Do you believe (that)?’ (\citealt{Zhu1985}: 12)
    \z

\noindent In this dialect the marker has the form \textit{k‘əʔ}\textsuperscript{1} or \textit{kəʔ}\textsuperscript{1}. An investigation of the extent of this feature towards the south goes beyond the possibilities of this study. But at least \ili{Mandarin} as spoken in \isi{Yunnan} also has this pattern. Independent of that question, it represents a southern border of the \isi{NEA} area as no other language in the sample has a comparable pattern. It may also be noted that \ili{Qiang}, the southern neighbor of \ili{Baima} and \ili{Zhongu} does not share this pattern (\sectref{sec:4.2.3}, \citealt{LaPollaHuang2003}: 180). The marker sometimes can exhibit a rather complex morphosyntactic behavior. For example, in \ili{Prinmi}, a language also spoken in \isi{Yunnan}, the marker has the form \textit{a} and usually, but not always, takes penultimate position in a sentence (\citealt{Ding2014}: 208).

\ea%47
    \label{ex:trans:48}
    \ili{Prinmi} (\ili{Qiangic})\\
    \ea
    \gll põ\textsuperscript{H}{põ}\textsuperscript{L} \textbf{{a}}\textbf{\textsuperscript{H}}{=ʒe}\textsuperscript{L}?\\
    uncle     \textsc{q=}\textsc{ex.an}\\
    \glt ‘Is Uncle home?’
    
    \ex
    \gll {tʃʰe}\textsuperscript{L} {dzɨ}\textsuperscript{H}{=}\textbf{{a}}\textbf{\textsuperscript{L}}{=ʃo}\textsuperscript{L}?\\
    meal  eat=\textsc{q}=\textsc{opt}\\
    \glt ‘Will (you) have a meal?’ (\citealt{Ding2014}: 209)
    \z
    \z

\noindent Consequently, the marker can stand both in front of or after the verb. In \ili{Japhug} (\ili{rGyalrong}, \ili{Qiangic}), to mention yet another language from Sichuan with the feature in question, the prefix apparently invariably has the form \textit{ɯ-} (and usually receives stress) (see \citealt{Jacques2004}: 400f.). However, the form of the marker is sometimes variable. In \ili{Baima} the \isi{question marker} has several different variants shown in \tabref{tab:trans:2} that are determined by the vowel of the following verb, i.e. it exhibits some form of umlaut. However, unlike \ili{Prinmi}, there is no change in tone.

\begin{table}
\caption{Variants of the question marker in Baima (\citealt{Sun1996}: 85).}
\label{tab:trans:2}

\begin{tabularx}{.8\textwidth}{XX}
\lsptoprule

\textbf{Example} & \textbf{Meaning}\\
\midrule
e\textsuperscript{53}-ndʑi\textsuperscript{53} & \textsc{q}-go\\
ɛ\textsuperscript{53}-tʃe\textsuperscript{53} & \textsc{q}-cut(.tree)\\
ɛ\textsuperscript{53}-ȵɛ\textsuperscript{35} & \textsc{q}-sleep\\
a\textsuperscript{53}-dʑa\textsuperscript{341} & \textsc{q}-mend\\
a\textsuperscript{53}-tɔ\textsuperscript{35} & \textsc{q}-wrap\\
ɔ\textsuperscript{53}-kho\textsuperscript{53} & \textsc{q}-back\\
o\textsuperscript{53}-phu\textsuperscript{35} & \textsc{q}-rub\\
ø\textsuperscript{53}-ȵy\textsuperscript{341} & \textsc{q}-smell\\
ə\textsuperscript{53}-khɐ\textsuperscript{53} & \textsc{q}-toast\\
ə\textsuperscript{53}-ndʑø\textsuperscript{53} & \textsc{q}-hit\\
ə\textsuperscript{53}-dzuɛ\textsuperscript{341} & \textsc{q}-pick(.potatoe)\\
\lspbottomrule
\end{tabularx}
\end{table}

Whether the markers in all languages mentioned above actually are cognates of each other could not be settled here but seems likely except for \ili{Chinese}. However, this is irrelevant from an areal and typological perspective.

\subsubsection{Summary}\label{sec:5.9.2.3}

\tabref{tab:trans:3} summarizes the marking of polar and \isi{content question}s in \ili{Trans-Himalayan} languages located in \isi{Northeast Asia}. Clearly, there is a \isi{tendency} for marked \isi{polar question}s and unmarked content \isi{questions}.

\begin{table}
\caption{Polar and content question markers in Trans-Himalayan languages spoken in \isi{NEA}. Tones are often variable and were thus excluded here.}
\label{tab:trans:3}

\begin{tabularx}{\textwidth}{lQl}
\lsptoprule
& \textbf{PQ} & \textbf{CQ}\\
\midrule
\ilit{Old Chinese} & *ɢˤa [hu]\# & -\\
\ilit{Mandarin} & ma\# & -, ne\#\\
Urumqi Hui \ilit{Chinese} & ma\# & -, nə\# {\textasciitilde} ȵi\# {\textasciitilde} ȵiɛ\#\\
Xining \ilit{Mandarin} & mɔ\# & -, lɛ\#\\
\ilit{Gangou} & ?ma\# & -\\
Hezhou/Linxia & ma\#, la\#, (ȵi)mu\# & -, ʐa\#, ȵi\#, ȵiʐa\#\\
\ilit{Wutun} & \textsc{pfv}, \textsc{res} =mu\#, \textsc{ipfv}, \textsc{progr} =a\# & -\\
\ilit{Tangwang} & =mu\# & -\\
\ilit{Amdo Tibetan} & ə-\textsc{v}, =na\#, =ni\# & -, =ni\#\\
\ilit{Zhongu} & ɐ-\textsc{v} & -\\
\ilit{Baima} & e-\textsc{v} (see \tabref{tab:trans:2}) & -\\
\ilit{Tangut} & ˑja-\textsc{v}, mo\# & -\\
\lspbottomrule
\end{tabularx}
\end{table}

\subsection{Interrogatives in Trans-Himalayan}\label{sec:5.9.3}
\subsubsection{Interrogatives in Sinitic}\label{sec:5.9.3.1}

\ili{Sinitic}, and particularly \ili{Mandarin}, interrogatives are special in several regards. First, modern \ili{Mandarin} interrogatives differ strongly from those in \ili{Old Chinese}. Second, many of the forms are \isi{transparent} formations that indicate a recent origin. Let us first address the \ili{Old Chinese} interrogatives. \tabref{tab:trans:4} gives their recent \isi{reconstruction} in the Baxter Sagart system.

\begin{table}
\caption{Interrogatives in Old Chinese according to \citegen{BaxterSagart2014a,BaxterSagart2014b} reconstruction; Middle Chinese is only an approximation; square brackets indicate uncertain sounds; forms marked by a question mark were not actually reconstructed as interrogatives by Baxter \& Sagart (cf. \citealt{Pulleyblank1995}: 91--97)}
\label{tab:trans:4}

\begin{tabularx}{\textwidth}{Qllll}
\lsptoprule

\textbf{Meaning} & \textbf{Character} & \textbf{Modern Reading} & \textbf{Middle Chinese} & \textbf{Old Chinese}\\
\midrule
how & \zh{安} & ān & 'an & *[ʔ]ˤa[n]\\
how & \zh{焉} & yān & 'jen & *ʔa[n]\\
what & \zh{何} & hé & ha & *[g]ˤaj\\
what\newline (dial. for \zh{何}) & \zh{曷} & hé & hat & *[g]ˤat\\
why, what & \zh{奚} & x\={\i} & hej & *[g]ˤe\\
how many & \zh{幾} (\zh{几}) & jǐ & kj+jX & *kəjʔ\\
how (rhetorical) & \zh{豈} & qǐ & khj+jX & *C.qʰəjʔ\\
who & \zh{孰} & shú & dzyuwk & *[d]uk\\
who & \zh{誰} & shuí & dzywij & *[d]uj\\
who & \zh{疇} & chóu & ?drjuw & ?*[d]ru\\
why, how & \zh{胡} & hú & ?hu & ?*[g]ˤa\\
why & \zh{胡為} & húwèi & ?hu + hjwe & ?*[g]ˤa + *ɢ\textsuperscript{w}(r)aj\\
why not\newline (< \zh{何不}?) & \zh{盍} & hé & ?hap & ?*m-[k]ˤap\\
how, where\newline (< \zh{於何}?) & \zh{惡}(\zh{乎}), \zh{烏} & wù(hú), w\=u & ?'uh, 'u & ?*ʔˤaks, *[ʔ]ˤa\\
\lspbottomrule
\end{tabularx}
\end{table}

Thus, \ili{Old Chinese} may have had resonances in *\textit{ʔ{\textasciitilde}}, *\textit{d{\textasciitilde}}, and *\textit{gˤ{\textasciitilde}} as well as several other interrogatives without such a \isi{submorpheme} (cf. \citealt{Pulleyblank1995}: 91). Regardless of whether the pharyngealization hypothesis (indicated with /ˤ/) turns out to be true (\citealt{BaxterSagart2014a}: 68ff.), several forms qualify as \textit{K}-interrogatives. Perhaps, the polar \isi{question marker} *\textit{ɢˤa} \zh{乎} also belongs here. The exact \isi{analysis} of most forms is unclear. But note that at least in some interrogatives analyzable morphological elements may have been present. The difference between *\textit{[d]uk} \zh{孰} and *\textit{[d]uj} \zh{誰} is especially intriguing. According to \citet[92]{Pulleyblank1995} the former belongs to “a group of words in *-k [...], which are confined to preverbal position referring to the subject, and which usually select the subject from a larger group.” \cite[236ff.]{Xu2006} agrees in part with this assessment and argues that a group of words ending in *\textit{-j} (such as *\textit{[d]uj} \zh{誰}) are more flexible in their syntactic behavior than those in *\textit{-k} (such as *\textit{[d]uk} \zh{孰}). \citet[91]{Pulleyblank1995} furthermore assumes that several interrogatives including \textit{ān} \zh{安} and \textit{yān} \zh{焉} are derived from the “coverb” (perhaps better called preposition) \textit{yú} \zh{於} in \isi{combination} with unspecified other elements. In fact, Baxter and Sagart reconstruct the form as *\textit{[ʔ}\textit{]a} \zh{於}, which makes a connection with *\textit{[ʔ}\textit{]ˤa}\textit{[n}] \zh{安} and *\textit{ʔa[}\textit{n]} \zh{焉} seem possible. But if this assumption is true, the preposition must have fused with a following \isi{interrogative}. These approaches are far from offering a clear picture of the etymology or \isi{morphology} of \ili{Old Chinese} interrogatives. To track the development of interrogatives---or of \isi{questions} in general for that matter---goes well beyond the possibilities of this study (but see \citealt{Peyraube2005}).

Colloquial \textbf{\ili{Mandarin} \ili{Chinese}} (\tabref{tab:trans:5}) potentially has only one \isi{interrogative} that is synchronically non-analyzable, namely \textit{shéi} (\textit{shuí}) ‘who’. All other interrogatives are analyzable to different degrees. Some are straightforward combinations of an \isi{interrogative} and a noun such as \textit{shénme dìfang} ‘where’. The second part simply means ‘place’, but the first element \textit{shénme} ‘what’, like \textit{zěnme} ‘how’ possibly contains a suffix \textit{-me} with an \isi{opaque} meaning. In the complex \isi{interrogative} \textit{zěn(}\textit{me)}-y\textit{àng} ‘how, what kind of, in what way’, the element \textit{-me} may be omitted, which speaks in favor of an \isi{analysis} as a suffix. Other interrogatives productively combine with grammatical elements such as the classifier \textit{ge} \zh{个}. A special case is the \isi{interrogative} \textit{gàn.}\textit{má} ‘to do what, why’, which quite clearly is a contraction of the \isi{transparent} formation \textit{gàn shénme} ‘to do what’. The first element \textit{gàn} ‘to do’ remains \isi{transparent}, but the second element \textit{má} is what is usually called a cranberry morph, because it is not attested outside of this word. The interrogatives \textit{jǐ-} ‘how many’ or \textit{nǎ-} ‘which (one)’ do not qualify as “\isi{basic question words}” either, because they necessarily combine with another element such as a classifier. The lack of a strongly developed \isi{resonance} speaks in favor of a relatively new system of interrogatives. In fact, only \textit{shéi} (\textit{shuí}) \zh{谁} and \textit{jǐ}\textit{-} \zh{几} can be traced back to \ili{Old Chinese}.

\begin{table}
\caption{Mandarin Chinese interrogatives and their analysis (mostly based on own knowledge, elicitation, \citealt{Ross2006}: 160f.); not all combinations are shown}
\label{tab:trans:5}

\begin{tabularx}{\textwidth}{lllQ}
\lsptoprule

\textbf{Meaning} & \textbf{Character} & \textbf{Form} & \textbf{Analysis}\\
\midrule
how + \textsc{adj} & \zh{多} & duō + \textsc{adj} & < ‘very’ < ‘much’\\
e.g., how long & \zh{多长} & duō-cháng & \textit{cháng} ‘long’\\
how many/much & \zh{多少} & duō-shǎo & \textit{duō} ‘much’, \textit{shǎo} ‘few’\\
how + \textsc{adj} & \zh{好} & hǎo- + \textsc{adj} & < ‘very’ < ‘good’, dialectal variant of \textit{duō}\\
how many/much & \zh{几} & jǐ- + \textsc{clf} & usually \textit{jǐ-ge} \zh{几个}\\
at what time & \zh{几点}(\zh{钟}) & jǐ-diǎn(zhōng) & \textit{diǎn(zhōng)} ‘o’clock’\\
which (one) & \zh{哪} & nǎ- + \textsc{clf} & usually \textit{nǎ-ge} \zh{哪个}\\
where & \zh{哪里}/\zh{儿}/\zh{边} & nǎ-li/-(e)r/-biān & \textit{-li}/\textit{-(e)r}/\textit{-biān}\\
who & \zh{谁} & shéi (shuí) & \\
what & \zh{什么} & shén-me & \textit{-me} as in \textit{zěn-me}?,\newline  <-n-m-> = [-mm-]\\
what & \zh{啥} & shá & colloquial variant of \textit{shénme}\\
what & \zh{嘛} & mà & colloquial variant of \textit{shénme}\\
where & \zh{什么地方} & shénme dìfang & \textit{dìfang} ‘place’\\
when & \zh{什么时候} & shénme shíhou & \textit{shíhou} ‘time’\\
why & \zh{为什么} & wèi-shénme & \textit{wèi} ‘for’\\
to do what, why & \zh{干什么} & gàn shénme & \textit{gàn} ‘to do’\\
to do what, why & \zh{干嘛} & gàn.má & colloquial variant of \textit{gàn shénme}\\
on what basis & \zh{凭什么} & píng-shénme & \textit{píng} ‘rely on’\\
how, why & \zh{怎么} & zěnme & \textit{-me} as in \textit{shén-me}?,\newline <-n-m-> = [-mm-]\\
how & \zh{咋} & zǎ & colloquial variant of \textit{zěnme}\\
how & \zh{怎}(\zh{么})\zh{样} & zěn(me)-yàng & \textit{yàng} ‘kind’\\
\lspbottomrule
\end{tabularx}
\end{table}

However, apart from the interrogatives mentioned in \tabref{tab:trans:5}, \ili{Mandarin} has about a dozen or so formal interrogatives given in \tabref{tab:trans:6} that are mostly restricted to the literary language and preserves \ili{Old Chinese} *\textit{[g}\textit{]ˤaj} \zh{何}.\footnote{It may be noted that \ili{Japanese} also preserves the character \zh{何} but has an autochthonous pronunciation \textit{nani} \jp{なに} instead.}

\begin{table}
\caption{Formal or literary Mandarin Chinese interrogatives (\citealt{Ross2006}: 161f.; \citealt{Pulleyblank1995}, partly elicited); not all forms listed}
\label{tab:trans:6}

\begin{tabularx}{\textwidth}{lXXXl}
\lsptoprule

\textbf{Meaning} & \textbf{Character} & \textbf{Form} & \textbf{Base} & \textbf{Meaning}\\
\midrule
why & \zh{何必} & hé-bì & bì & necessarily\\
why not & \zh{何不} & hé-bù & bù & not\\
when (rhetorical) & \zh{何曾} & hé-céng & céng & once\\
how could (one) not & \zh{何尝} & hé-cháng & cháng & once\\
why not & \zh{何妨} & hé-fáng & fáng & impede\\
why & \zh{何故} & hé-gù & gù & reason\\
when & \zh{何时} & hé-shí & shí & time\\
what is & \zh{何为} & hé-wéi & wéi & \textsc{cop}\\
how, why & \zh{何以} & hé-yǐ & yǐ & with, use\\
how & \zh{如何} & rú-\textbf{hé} & rú & to be like\\
why & \zh{为何} & wèi-\textbf{hé} & wèi & for\\
\lspbottomrule
\end{tabularx}
\end{table}

To my knowledge, in \isi{NEA} only \ili{Mandarin} has such a marked contrast between two different sets of interrogatives that depend on style.

\ili{Mandarin} interrogatives display strong paradigmatic similarities with the \isi{demonstratives}. \ili{Mandarin} is not usually analyzed as having paradigms, but nevertheless such an \isi{analysis} seems viable (\tabref{tab:trans:7}).

\begin{table}
\caption{Partial demonstrative and interrogative paradigms in Mandarin (my knowledge)}
\label{tab:trans:7}

\begin{tabularx}{\textwidth}{XXXl}
\lsptoprule
& \textbf{this} & \textbf{that} & \textbf{which}\\
\midrule
plain & zhè- & nà- & nǎ-\\
\textsc{X-pl} & zhè-xi\=e & nà-xi\=e & nǎ-xi\=e\\
\textsc{X(-num)-clf} & zhè(-yi)-ge & nà(-yi)-ge & nǎ(-yi)-ge\\
id. (fused) & \textbf{zhèi}-ge & \textbf{nèi}-ge & \textbf{něi}-ge\\
\textsc{X-loc} & zhè-(e)r & nà-(e)r & nǎ-(e)r\\
\textsc{X-loc} & zhè-li & nà-li & nǎ-li\\
\textsc{X-loc} & zhè-biān & nà-biān & nǎ-biān\\
\textsc{dir-X-loc} & wǎng-zhè-(e)r & wǎng-nà-(e)r & wǎng-nǎ-(e)r\\
\textsc{all-X-loc} & dào-zhè-(e)r & dào-nà-(e)r & dào-nǎ-(e)r\\
\textsc{abl-X}\textsc{-loc} & cóng-zhè-(e)r & cóng-nà-(e)r & cóng-nǎ-(e)r\\
\lspbottomrule
\end{tabularx}
\end{table}

\newpage 
The \isi{interrogative} \textit{duō} is usually used with scalar adjectives such as \textit{duō-jiǔ} \zh{多久} ‘how long’. Strangely, as my informant tells me, some dialects such as those of \isi{Guiyang} use the cognate of \ili{Mandarin} \textit{hǎo} \zh{好} instead. Literally, \textit{duō} \zh{多} means ‘much’ and \textit{hǎo} \zh{好} ‘good’, but notably both share an emphatic meaning of ‘very’. Sentences with both \textit{duō} and \textit{hǎo} can have their original meaning and may be regarded as polysemous. The reading depends on the \isi{intonation} as well as the context. The following elicited example has been idealized and standardized to allow a better comparison.

\ea%48
    \label{ex:trans:49}
    \ili{Mandarin}\\
    \gll zhè  tiáo  hé \textbf{{du}}\textbf{{ō}}\textbf{{/}}\textbf{{hǎo}} {kuān?}\\
    this  \textsc{clf}  river  how    broad\\
    \glt ‘How broad is this river?’
    \z

\noindent Guiyang is located outside of \isi{NEA}, but the same phenomenon can also be observed, for example, in the Shiquan dialect in Shaanxi, which has the form \textit{xao}\textsuperscript{55} \zh{好} (e.g., \textit{xao}\textsuperscript{55}\textit{tɕ}\textit{iu}\textsuperscript{0} \zh{好久} ‘how long’).

There are more descriptions of interrogatives in \ili{Chinese} dialects than can possibly be mentioned here. However, the majority simply rely on a transliteration with characters and do not give a phonetic transcription, which makes the data problematic at best. The following gives the interrogatives from a selection of different dialects, namely Suide \zh{绥德} (northern Shaanxi), Shiquan \zh{石泉} (southern Shaanxi), Yanggao \zh{阳高} (northern Shanxi), Lingshi \zh{灵石} (eastern Shanxi), and Xining \zh{西宁} (eastern \isi{Qinghai}). In addition, the interrogatives of Hui \ili{Chinese} spoken in Urumqi \zh{乌鲁木齐} (northern \isi{Xinjiang}) are given. The list is not meant to provide an exhaustive overview, but gives an impression of dialectal variation found in northern \ili{Mandarin} (\tabref{tab:trans:8}).

\begin{table}[p]
\caption{A selection of interrogatives from Suide (northern Shaanxi, \citealt{Hei2013}: 51, slightly corrected), Shiquan (southern Shaanxi, \citealt{ShiFeng2009}: 14), Yanggao (northern Shanxi, \citealt{SunQinglin2015}: 150, vowel transcription slightly corrected), Xining dialects (eastern Qinghai, \citealt{ZhangChengzai1980}: passim), and Urumqi Hui Chinese (\citealt{LiuLiji1989}: 160f., passim)}
\label{tab:trans:8}

\begin{tabularx}{\textwidth}{XXXX XXX}
\lsptoprule
\textbf{Mean.} & \textbf{Suide} & \textbf{Yanggao} & \textbf{Lingshi} & \textbf{Shiquan} & \textbf{Xining} & \textbf{Urumqi}\\
\midrule
who & ȿuei\textsuperscript{33} & suei\textsuperscript{312} & ȿu\textsuperscript{44} &  & fei\textsuperscript{24} & sei\textsuperscript{24}kɤ\textsuperscript{53}\\
which, who &  & na\textsuperscript{53} kəʔ\textsuperscript{33} & ?iaʔ\textsuperscript{535} kəʔ\textsuperscript{44} & la\textsuperscript{55} go\textsuperscript{21} & a\textsuperscript{44} kɔ\textsuperscript{213} & na\textsuperscript{21} kɤ\textsuperscript{52}\\
what & ȿəʔ\textsuperscript{3}ˑma, ȿəŋ\textsuperscript{52} & sa\textsuperscript{31} & səŋ\textsuperscript{53} & ȿa\textsuperscript{213} & sa\textsuperscript{213} & ʂʅ\textsuperscript{24}mɤ\textsuperscript{21}, sa\textsuperscript{24}\\
which & la\textsuperscript{213} & na\textsuperscript{53} & ?iaʔ\textsuperscript{535} & la\textsuperscript{55} &  & \\
where & la\textsuperscript{213} (ˑli) & na\textsuperscript{53} lɛ\textsuperscript{0} &  & la\textsuperscript{55} li\textsuperscript{0} &  & nɐr\textsuperscript{24}\\
where & la\textsuperscript{213} ·tɐr & na\textsuperscript{53} tər\textsuperscript{31} &  & la\textsuperscript{55}taŋ\textsuperscript{21} & a\textsuperscript{4421} tʂa\textsuperscript{53} ɛ\textsuperscript{341} & \\
when & ʂəŋ\textsuperscript{52} ·xur &  & səŋ\textsuperscript{53} xur\textsuperscript{44} &  &  & \\
& & təŋ\textsuperscript{213} ·xur & tuɤ\textsuperscript{31} xuər\textsuperscript{24} &  &  & tuɤ\textsuperscript{21} xuər\textsuperscript{44}\\
& tɕi\textsuperscript{213} sɿ\textsuperscript{33} & tɕi\textsuperscript{53} səʔ\textsuperscript{33} &  & tɕi\textsuperscript{55} sʅ\textsuperscript{21} &  & \\
&  & sa\textsuperscript{31} sɿ\textsuperscript{31} xɤər\textsuperscript{24} &  & ȿa\textsuperscript{213} sʅ\textsuperscript{21} (xou\textsuperscript{0}) & sa\textsuperscript{213} sɿ\textsuperscript{3521} xɯ\textsuperscript{12354} & \\
how & tsua\textsuperscript{213} & tsa\textsuperscript{312}, tsa\textsuperscript{24} & tsa\textsuperscript{44}tɕi\textsuperscript{44} &  & ?a\textsuperscript{4421}- & tsa\textsuperscript{52}-\\
why & vei\textsuperscript{52} ʂəŋ\textsuperscript{52}, vei\textsuperscript{31} ʂəʔ\textsuperscript{3}·ma & (iəŋ\textsuperscript{31}) vei\textsuperscript{24} sa\textsuperscript{31} & iŋ\textsuperscript{535} uei\textsuperscript{53} səŋ\textsuperscript{53} & uei\textsuperscript{21} ȿa\textsuperscript{213} & uei\textsuperscript{213} sa\textsuperscript{213} & \\
how & təŋ\textsuperscript{213} & tuɤ\textsuperscript{31} &  & \textbf{xao\textsuperscript{55}} & tu\textsuperscript{44} & tuɤ\textsuperscript{21}\\
how much & təŋ\textsuperscript{213} ʂɔ\textsuperscript{213} & tuɤ\textsuperscript{31} sɔu\textsuperscript{53} & tei\textsuperscript{535} sɔ\textsuperscript{212} &  & tu\textsuperscript{34} ʂɔ\textsuperscript{53} & tuɤ\textsuperscript{21}ʂɔ\textsuperscript{52}\\
how many & tɕi\textsuperscript{213} & tɕi\textsuperscript{53} & tɕi\textsuperscript{212} & tɕi\textsuperscript{55} &  & \\
\lspbottomrule
\end{tabularx}
\end{table}

There is a bewildering variety of different forms and combinations of forms that is qualitatively different from most other \isi{interrogative systems} observed in \isi{NEA}. Often a specific function may be expressed with a wide variety of different forms. For instance, the Yanggao dialect is said to have twelve locative forms. Only a selection of forms is included here. The Suide, Yanggao, and Lingshi dialects represent the disputed Jin dialect area that is sometimes distinguished from \ili{Mandarin}. An interesting feature shared by these dialects is a final glottal stop such as in the classifier \zh{个} (\ili{Mandarin} \textit{ge}, Shiquan \textit{go}, Xining \textit{kɔ}, Urumqi Hui \ili{Chinese} \textit{kɤ}, but Suide \textit{kuə}\textbf{\textit{ʔ}}, Yanggao \textit{kə}\textbf{\textit{ʔ}}, and Lingshi \textit{kə}\textbf{\textit{ʔ}}, here given without tones). Many forms that were not listed cannot be found in Standard \ili{Mandarin}. For example, \ili{Mandarin} cannot use the \isi{plural} marker \textit{-men} \zh{们} (Lingshi \textit{ȿu}\textsuperscript{44}\textit{məŋ}\textsuperscript{44}, Xining \textit{fei}\textsuperscript{2421}\textit{m\~{ə}} \textsuperscript{2454}, Urumqi \textit{sei}\textsuperscript{24}\textit{məŋ}\textsuperscript{21}, Yanggao \textit{suei}\textsuperscript{312}\textit{məŋ}\textsuperscript{31}) or the classifier \textit{(yi)ge} \zh{一个} ‘one \textsc{clf}’ in \isi{combination} with the personal \isi{interrogative} \textit{shéi} ‘who’ (Urumqi \textit{sei}\textsuperscript{24}\textit{kɤ}\textsuperscript{53} {\textasciitilde} \textit{sei}\textsuperscript{24}\textit{ji}\textsuperscript{21}\textit{kɤ}\textsuperscript{21}).

A special case is Lingshi \textit{uɛ}\textsuperscript{44}\textit{ȿu}\textsuperscript{44} \zh{兀谁} ‘who’, which contains a demonstrative \textit{uɛ}\textsuperscript{44} unknown in \ili{Mandarin}. Whether Xining \textit{a}\textsuperscript{44}\textit{mə} \textsuperscript{2444} ‘why, how’ is related to \ili{Mandarin} \textit{zěnme} \zh{怎么} ‘why, how’ or \textit{nǎ} \zh{哪} ‘which’ remains somewhat unclear. However, it has clear parallels in Hezhou, \ili{Wutun}, \ili{Tangwang}, and \ili{Gangou} \ili{Chinese}.

\newpage  
For \textit{Hezhou} dialect only a few descriptions of interrogatives are available (\tabref{tab:trans:9}). Both varieties of Hezhou listed have lost the initial nasal in the cognate of \ili{Mandarin} \textit{nǎ} \zh{哪} and contain some etymologically \isi{opaque} derivations.


\begin{table}[t]
\caption{Xunhua and Jishishan Hezhou interrogatives (\citealt{ZhongJinwen2007}: 393f.); forms in square brackets from \citet[156]{Dwyer1995}}
\label{tab:trans:9}

\begin{tabularx}{\textwidth}{Qll}
\lsptoprule

\textbf{Meaning} & \textbf{Xunhua} & \textbf{Jishishan}\\
\midrule
who, which & a\textsuperscript{3}ɿ\textsuperscript{3(3)}kə\textsuperscript{55} [a-ʒi\textsuperscript{24}-gə {\textasciitilde} a-ji\textsuperscript{24}-gə] & a\textsuperscript{44}jɿ\textsuperscript{2}kə\textsuperscript{44}\\
where & a\textsuperscript{3}lɿ\textsuperscript{42} & a\textsuperscript{44}lɿ\textsuperscript{44}lɿ\textsuperscript{2}\\
how & a\textsuperscript{21}mu\textsuperscript{55}liəɯ\textsuperscript{3} & a\textsuperscript{44}mu\textsuperscript{44}liao\textsuperscript{02}\\
what & ʂʅ\textsuperscript{2}ma\textsuperscript{55} & ʂʅ\textsuperscript{31}ma\textsuperscript{44}\\
what & [sa\textsuperscript{13}] & \\
what & tʂə\textsuperscript{55}liəɯ & \\
what (emph.) & [ʂə\textsuperscript{13}ma\textsuperscript{41}kə] & \\
how much/many & tɔ\textsuperscript{24}ʂo\textsuperscript{55} & tuo\textsuperscript{31}ʂao\textsuperscript{44}\\
why & tʂa\textsuperscript{55}liəɯ\textsuperscript{3}ʂʅ\textsuperscript{3} & a\textsuperscript{44}mu\textsuperscript{44}\\
where\newline  (?\ili{Turkic} \textit{-DA}) & [a\textsuperscript{14}-dɪ-r] & \\
why & [a\textsuperscript{13}-men\textsuperscript{24}dʐə] & \\
\lspbottomrule
\end{tabularx}
\end{table}


\cite[513, 515]{Rimsky-Korsakoff1994} mentions the two \textbf{Dungan} forms \textit{dza} \zh{怎} and \textit{sa} \zh{啥}. \citet[76]{Hai2002} also only lists \textit{tɕi}\textsuperscript{41}\textit{ʂɩ}\textsuperscript{24} \zh{几时} ‘when’ and \textit{sa}\textsuperscript{44} \zh{啥} ‘what’, but includes a notation of tones. We have already encountered all these forms in several dialects above.

\begin{table}
\caption{Tangwang (\citealt{Xu2014}: 222, passim), Wutun (\citealt{Janhunen2008}: 69f.; Erika Sandman p.c. 2016), and Gangou interrogatives (\citealt{ZhuYongzhong1997}: 447; \citealt{YangYonglong2014}: 244f. with Mandarin transliteration in square brackets)}
\label{tab:trans:10}

\begin{tabularx}{\textwidth}{XXXl}
\lsptoprule

\textbf{Meaning} & \textbf{Tangwang} & \textbf{Wutun} & \textbf{Gangou}\\
\midrule
who & a-ke & a-ge & [age]\\
where & a-lɪ & a-li & \\
where & a-tha & a-ra/la & [a-biao]\\
when & a-xuɪ &  & \\
how, what kind of & am(tʂɛ) &  & \\
why, how & amutʂe & a-menzai & a-men\\
what kind of & amutɕɪkə &  & \\
what & ʂəma & ma(-ge) & [sha]\\
when & ʂəma ʂʅxəu &  & \\
how much & tuəʂɔ & do & \\
when &  & do-xige & [a-hui(e)r] \\
how many &  & jhi-ge & \\
why &  & ma-shema & \\
\lspbottomrule
\end{tabularx}
\end{table}

Despite the fact that some \textbf{Wutun} interrogatives are cognates with \ili{Mandarin}, the overall picture is quite different. There are two new resonances being built up (\textit{a{\textasciitilde}} and \textit{ma{\textasciitilde}}), neither of which exists in \ili{Mandarin}. \ili{Wutun} has only one basic \isi{interrogative} word, \textit{ma} ‘what’, which possibly is a contracted form of \ili{Mandarin} \textit{shénme} \zh{什么} > \textit{mà} \zh{嘛} ‘what’. A \isi{combination} of \textit{shénme} \zh{什么} with \textit{ge} \zh{个} as in \ili{Wutun} \textit{ma-ge} is impossible in \ili{Mandarin}, but not in the Xunhua subdialect of Hezhou, which has \textit{ʂə}\textsuperscript{13}\textit{ma}\textsuperscript{41}\textit{kə} \zh{什么个}. The development of the meaning of \ili{Wutun} \textit{a-ge} from ‘which one’ to ‘who’ has parallels in several \ili{Mandarin} dialects. Interrogatives in \textit{Tangwang} are relatively straightforward. Only the origin of what appears to be a locative suffix \textit{-tha} remains unclear for now. The description does not seem to be very reliable as individual interrogatives are given in different forms throughout the book (\citealt{Xu2014}). Only for some forms tones were given, which is why they have been removed altogether. Some unclear forms were left aside.

There are no good descriptions of \isi{questions} in \textbf{Gangou} \ili{Chinese}. Only in the last two years have there been any studies of the language in \isi{China} at all. But they are all from one and the same scholar Yang Yonglong (e.g., \citeyear{YangYonglong2014}: 244f.), who does not give sufficient information on pronunciation or grammar and for the most part employs \ili{Chinese} characters for transcription and dialect for translation. According to \citet{YangYonglong2014}, the interrogatives can have a \isi{plural} marker [\textit{-men}] \zh{们} that is said to be pronounced \mbox{/mu/}.

\subsubsection{Interrogatives in Tibetic and Qiangic}\label{sec:5.9.3.2}

\tabref{tab:trans:11} and \tabref{tab:trans:12} give some interrogatives from seven different \textbf{\ili{Tibetic}} or \textbf{\ili{Qiangic}} varieties, heuristically classified into those with and without tones.

\begin{table}
\caption{Gonghe \citep[64]{Ebihara2011}, Themchen (\citealt{Haller2004}: 51, 55, 66, passim), rNgawa \citep[82]{Suzuki2006}, and Zhongu interrogatives (\citealt{Sun2003a}: passim)}
\label{tab:trans:11}

\begin{tabularx}{\textwidth}{lQQQQ}
\lsptoprule

\textbf{Form} & \textbf{\isi{Amdo} (Gonghe)} & \textbf{\isi{Amdo} (Themchen)} & \textbf{\isi{Amdo} (rNgawa)} & \textbf{Zhongu}\\
\midrule
who & sʰə & sʰə & sʰɯ & sə\\
which & kaŋ & kaŋ & kɐŋ & kɔ-te\\
where & kaŋ=na & kaŋ-na, tɕʰə-na & kəŋ-tɯ & kɔ-no\\
how many/much & tə,\newline tɕʰə-mo-zek & tə,\newline \mbox{tɕʰə-mu-(-zəç)} & tɕʰi-mə-zə,\newline kəŋtɕʰəʰtə,\newline doqritɕʰə & tʃʰá-tsə\\
what & tɕʰə-zek & tɕʰə(-zəç) & tɕə-zə & tʃʰə(-tsə)\\
why, how & tɕʰə-zek-a & tɕʰə-zəç-a & kəŋ-tɯ & tʃʰá-tsə-jə\\
how & tɕʰə-gi & tɕʰə-ɣi/-ji &  & \\
when & ?nem & nam & nam,\newline kər-tɯ & nɔ\\
\lspbottomrule
\end{tabularx}
\end{table}

For the \ili{gSerpa} dialect from northern Sichuan, \citet{Sun2006} only mentions the \isi{interrogative} \textit{tɕʰ}\textit{ə} ‘what’. This \isi{interrogative} stem, present in all varieties mentioned here, has been reconstructed as (*\textit{tyi} >) *\textit{tɕ(h)i} ‘what’ for \ili{Proto-Tibetic}, showing palatalization characteristic for \ili{Tibetic} \citep[114]{Tournadre2014}. The derived form, e.g. \textit{tʃʰ}\textit{ə-tsə} in \ili{Zhongu}, according to \citet[831]{Sun2003a}, has the underlying Written \ili{Tibetan} form \textit{ci.cig}, in which the second element seems to be the indefinite article \textit{cig}, derived from the numeral \textit{gcig} ‘one’ (\citealt{DeLancey2003}: 263). A parallel can be found in some \ili{Mongolic} languages of the \isi{Amdo Sprachbund} (\sectref{sec:5.7.3}).

\begin{table}
\caption{gSerpa Tibetan (\citealt{Nagano1980}: passim), \ili{Cone} Tibetan (\citealt{Jacques2014}: passim), and Baima interrogatives (\citealt{Sun1996}: 77ff. 348f., passim) (\textsuperscript{H/L/N} = high/low/neutral tone)}
\label{tab:trans:12}
\begin{tabularx}{\textwidth}{lQlQ}
\lsptoprule

\textbf{Form} & \textbf{gSerpa} & \textbf{Cone} & \textbf{Baima}\\
\midrule
who & su\textsuperscript{H} &  & su\textsuperscript{35} {\textasciitilde} su\textsuperscript{341}\\
which & gän\textsuperscript{L}\textsc{n}di\textsuperscript{N} &  & ka\textsuperscript{35}lɛ\textsuperscript{53}\\
where & go\textsuperscript{L}ne\textsuperscript{H} & kaa\textsuperscript{L}-nə\textsuperscript{H} & ka\textsuperscript{13}la\textsuperscript{53}\\
how many/much & ?či\textsuperscript{H}raʔ\textsuperscript{H} ći\textsuperscript{N}reʔ\textsuperscript{N} & tɕʰən\textsuperscript{L}-dʐe\textsuperscript{H} & tʃhɿ\textsuperscript{53}zɔ\textsuperscript{35},\newline
	  ka\textsuperscript{13}kɔ\textsuperscript{35},\newline
	  tʃhɿ\textsuperscript{13}zɔ\textsuperscript{35}\\
what & či\textsuperscript{H}ći\textsuperscript{H} &  & tʃhɿ\textsuperscript{53}\\
how & či\textsuperscript{H}ru\textsuperscript{L} & tɕʰə\textsuperscript{H}-zə\textsuperscript{L}-ɣe\textsuperscript{L} & ka\textsuperscript{13}tȿo\textsuperscript{53}\\
why, how & či\textsuperscript{H}ći\textsuperscript{H}gi\textsuperscript{N}to\textsc{n}\textsuperscript{H}daʔ\textsuperscript{L}, čha\textsuperscript{H}yi\textsc{n}\textsuperscript{N}na\textsc{n}\textsuperscript{H} & tɕʰɔ\textsuperscript{H}-$\chi $tsə\textsuperscript{L} & \\
when & na\textsc{n}\textsuperscript{L} & næ\textsuperscript{L}wõõ\textsuperscript{L} ndzə\textsuperscript{L}ɣe\textsuperscript{L} & nɔ\textsuperscript{35}ndza\textsuperscript{53}\\
\lspbottomrule
\end{tabularx}
\end{table}

The \isi{plural} may be formed by \isi{reduplication}, which has been adopted by some \ili{Mongolic} languages of the region (\sectref{sec:5.8.3}), e.g. Gonghe \ili{Amdo Tibetan} \textit{sʰ}\textit{ə sʰ}\textit{ə} ‘who (\textsc{pl})’ \citep[54]{Ebihara2011} or \ili{Baima} \textit{su}\textsuperscript{35} \textit{su}\textsuperscript{35} ‘who (\textsc{pl})’ (\citealt{Sun1996}: 78). But in Themchen \isi{Amdo}, for example, there are the \isi{plural} forms \textit{sʰ}\textit{ə-tɕ}\textit{ʰ}\textit{u} and \textit{kaŋ-tɕʰ}\textit{u}, instead. The languages share stems for ‘who’, ‘which’, ‘when’, and ‘what’, the last of which is the basis for several derivations. For instance, Gonghe \ili{Amdo Tibetan} \textit{tɕʰ}\textit{ə-zek-a} contains a dative \isi{case} that has the form \textit{-(k)a} following \textit{k} \citep[60]{Ebihara2011}. This seems to be an exact parallel to Themchen \textit{tɕʰ}\textit{ə-zəç-a}, \ili{Zhongu} \textit{tʃʰ}\textit{á-tsə-jə} (\citealt{Sun2003a}: 797, fn. 51), and possibly \ili{Cone} \textit{tɕ}ʰ\textit{ə}\textsuperscript{H}\textit{-}\textit{zə}\textsuperscript{L}\textit{-}\textit{ɣe}\textsuperscript{L}. There are parallel formations in some \ili{Mongolic} languages of the region (\sectref{sec:5.8.3}). The Gonghe \isi{interrogative} \textit{tɕʰ}\textit{ə-gi} ‘how’ as well as its Themchen cognate \textit{tɕʰ}\textit{ə-ɣi/-ji} apparently contain a purposive or causative conjunction \citep[71]{Ebihara2011}. Given the \isi{analyzability} of many of the forms, for example those starting with \textit{tɕʰ}\textit{-} in \ili{Amdo Tibetan}, there is no clear \isi{resonance} phenomenon in any of the languages mentioned. Apart from the stem \textit{kaŋ} ‘where’, \citet[165]{Ebihara2009} also mentions the form \textit{kaŋ-ŋa}, in which the dative takes the regular form \textit{-ŋa} following \textit{ŋ} \citep[60]{Ebihara2011}. \citet[157]{Ebihara2013} also noted an interesting difference in forms meaning ‘whence’ with either the ablative or the genitive in different dialects of \ili{Amdo Tibetan}.

\ea%49
    \label{ex:trans:50}
    \ili{Amdo Tibetan} (rGya ya)\\
    \gll {cʰ}o \textbf{{kaŋ}}=ne  joŋ=ne?\\
    2\textsc{sg}  where=\textsc{abl}  come=\textsc{aux}\\
    \glt ‘Where are you from?’
    \z

\ea%50
    \label{ex:trans:51}
    \ili{Amdo Tibetan} (dPa’ris)\\
    \gll {tɕʰ}{o} \textbf{{kaŋ}}=ngə  ʁi=le?\\
    2\textsc{sg}  where=\textsc{gen}  come=\textsc{aux}\\
    \glt ‘Where are you from?’ \citep[157]{Ebihara2013}
    \z

There are parallel \isi{interrogative} and demonstrative paradigms with a three way contrast similar to \ili{Japonic} and \ili{Koreanic}. \tabref{tab:trans:13} illustrates these with data from the Themchen dialect. Only the distal demonstrative shows an exact parallel.

\begin{table}
\caption{Demonstrative and interrogative paradigms in Themchen Amdo Tibetan (\citealt{Haller2004}: 51, 64, 66)}
\label{tab:trans:13}

\begin{tabularx}{\textwidth}{XlXXl}
\lsptoprule
& \textbf{\textsc{prox}} \textbf{(speaker)} & \textbf{\textsc{prox}} \textbf{(hearer)} & \textbf{\textsc{dist}} & \textbf{which}\\
\midrule
\textsc{dat} & \textbf{\textit{ndə}} & \textbf{\textit{tə}} & \textit{kan-a} & \textit{kaŋ-a}\\
\textsc{loc} & \textit{ndə-na} & \textit{tə-na} & \textit{kan-na} & \textit{kaŋ-na}\\
\textsc{abl} & \textit{ndə-ni} & \textit{tə-ni} & \textit{kan-ni} & \textit{kaŋ-ni}\\
\lspbottomrule
\end{tabularx}
\end{table}

Finally, let us have a brief look at the interrogatives from the extinct language \textbf{Tangut}. \citet[617, passim]{Gong2003} mentions \textit{sjwɨ}\textsuperscript{1}, \textit{sjwɨ}\textsuperscript{2} ‘who’, \textit{ljọ}\textsuperscript{2} ‘where’, \textit{ljị}\textsuperscript{1} ‘which’ , \textit{wa}\textsuperscript{2} ‘what’ , \textit{wa}\textsuperscript{2}\textit{zjịj}\textsuperscript{1} ‘how many/much’, \textit{zjịj}\textsuperscript{1}-\textit{mə}\textsuperscript{2} ‘how many kinds’, and \textit{thjij}\textsuperscript{2} \textit{(sjo}\textsuperscript{2}) ‘why, how’ (1 = level tone, 2 = rising tone). The forms are very different from \ili{Tibetic} and even \ili{Qiang} (\citealt{LaPollaHuang2003}: 53), which indicates a relatively long time of separation (see also \citealt{Chirkova2012}). However, the \ili{Qiang} \isi{interrogative} system is relatively innovative with many forms, as in the \ili{Tibetic} varieties above, being based on \textit{ȵi(ɣ)i} ‘what’. The \ili{Qiangic} personal interrogatives seem to be among the most conservative (e.g., \ili{Qiang} \textit{sə}, \ili{Guiqiong} \textit{su} etc.) and are probably cognates of the \ili{Tangut} and \ili{Tibetic} forms above.

\clearpage %solid chapter boundary