\chapter{The typology of questions}

Before \chapref{sec:5} investigates the grammar of \isi{questions} in the individual language families of \isi{NEA}, this chapter describes the most important parameters of the typology.


\section{Introduction to the typology of questions}\label{sec:4.1}

There is a certain amount of confusion surrounding the terminology employed for what was called the \textit{grammar of questions} in this study. Grammar books usually employ the terms \textit{question} and \textit{interrogative} (nominal or attributive), but quite inconsistently so. In most cases no clear-cut distinction is drawn and the terminology is simply tacitly taken for granted. A few examples should suffice to illustrate the extent of the problem in \ili{English}-language publications. \citet[250]{Schulze2007}, for example, explicitly employs the term \textit{\isi{interrogativity}} for the cognitive side of the phenomenon and \textit{question} for the linguistic form. A related terminology can be found in \citet[103]{Rajasingh2014}: “\textit{Interrogation} is a semantic process of eliciting information by way of \textit{questioning}.” (my \isi{emphasis}) \citet[411]{Huddleston1994}, on the contrary, “explores the relation between \textit{interrogative}, a category of grammatical form, and \textit{question}, a category of meaning.” (my \isi{emphasis}) \citet[376]{Dixon2012} draws a distinction between different speech acts (e.g., \textit{questions}) and grammatical categories (e.g., \textit{interrogative}). Furthermore, for what is traditionally known as an \textit{\isi{interrogative} pronoun} he employs the much more fitting term \textit{\isi{interrogative} word}. In this study, the term \textit{question} refers either to the formal side or is used as a cover term for both the formal and the semantic side taken together. The semantic side of \isi{questions} will only be named \textit{\isi{interrogativity}} or \textit{\isi{interrogation}} if a clear distinction is called for. The so-called \textit{\isi{interrogative} words}, following \citet{Diessel2003}, will simply be called \textit{interrogatives} in order to preserve a connection to the traditional term and to place at the same time an additional \isi{emphasis} on their \isi{similarity} to so-called \textit{\isi{demonstratives}} and on their possible special position in the language.

For all we know, \isi{question-response sequences} (\citealt{EnfieldStiversLevinson2010}) and, more generally, turn-taking in conversation \citep{Stivers2009} provide a \isi{universal} \isi{enchronic} infrastructure that allows a comparison of different languages with each other. Question-response sequences are usually accompanied by non-linguistic cues such as the \isi{gazing behavior} of the questioner (\citealt{RossanoBrownLevinson2009}) or head movements by the addressee such as a \isi{head shake}. For practical purposes, this study concentrates on the first part of such sequences exclusively, and must leave aside non-linguistic elements. While this omission will perhaps cause some eyebrow-raising among experts, such information can only be obtained through prolonged fieldwork and thus is only available in sufficient detail for very few languages worldwide (e.g., \citealt{Levinson2010}).

A full account of the historical development of the typology of \isi{questions} lies beyond the possibilities of this study. In the following, I will only give a rough sketch with a focus on more recent advances. Apart from some isolated and mostly outdated studies (e.g., \citealt{Bolinger1957}), the modern typology of \isi{questions} by and large may be said to have started around 1970 with \citet{Ultan1978}, a cross-linguistic study based on a sample of 79 languages (originally published in 1969), \citegen{Moravcsik1971} investigation of \isi{polar question}s in 85 languages, and \citet{Danielsen1972}, based on a sample of about 60 languages. Since then, the field has made enormous advances that cannot be reviewed here in every detail. During the 1970s and 1980s there were relatively few \textit{important} publications with long-lasting effects, such as a collection of papers on \isi{questions} in seven languages in \citet{Chisholm1984} and the study by \citet{SadockZwicky1985} (written around 1976 and 1977) in the first edition of \textit{Language Typology and Syntactic Description}. The number of works has been steadily increasing at an ever faster pace from the 1990s until today. By now there are several dozen important publications, not including studies on individual languages, the number of which has been growing even more rapidly. But surprisingly, the only investigation that may be said to represent something like a standard typology is \citet{Siemund2001}, which is a mere 19 pages long and by now over 15 years old. A somewhat updated account by \citet{KönigSiemund2007} can be found in the second edition of \textit{Language Typology and Syntactic Description}. Perhaps the best general introduction to the typology of \isi{questions} to date can be found in volume three of \citegen[376–433]{Dixon2012} \textit{Basic Linguistic Theory}. \tabref{tab:4:1} gives a non-exhaustive overview of some important typological studies of \isi{questions} since 1990, excluding investigations of individual languages and generative approaches. Few studies are based on a large sample and almost all are unrepresentative of the languages of the world. Exceptions include \citegen{Idiatov2007} lengthy investigation of 1850 languages and especially a series of high-quality investigations with a sample of about 900 languages by \cite{Dryer2013k,Dryer2013l,Dryer2013m}. Most studies only focus on specific details but do not cover the entire scope of the \isi{grammar of questions}.

There are many possible classifications of \isi{questions}. For instance, \citet[439]{Sanitt2007} draws a distinction between \textit{empirical} (“\isi{questions} whose presuppositions are undoubted or taken as axiomatic”) and \textit{theoretical} \isi{questions} (“all \isi{questions} which are not empirical”). \citet[559]{Sanitt2011} furthermore introduces a distinction between \textit{closed} \isi{questions} that “have definitive answers” (such as a \isi{riddle}, see \sectref{sec:4.4}) and \textit{open-ended} \isi{questions} that “lead to other \isi{questions}”. These distinctions may be useful for the philosophy of science (e.g., \citealt{Meyer1980}), but to the best of my knowledge they are not relevant for a cross-linguistic investigation.

\begin{table}
\caption{Important typological studies of questions since 1990}
\label{tab:4:1}
\footnotesize	
\begin{tabularx}{\textwidth}{Q@{~}Q@{~}Q@{~}l@{}}
\lsptoprule

\textbf{Authors} & \textbf{Focus} & \textbf{Languages} & Number\\
\midrule
\citealt{MuyskenSmith1990} & interrogatives & \isit{pidgins} and \isit{creoles} & ca. 25\\
\citealt{Heine1991} & interrogatives, \isit{hierarchy} & global, unrepresentative & 14\\
\citealt{Lindström1995} & interrogatives & global, unrepresentative & 24\\
\citealt{Mushin1995} & interrogatives & Australian languages & 26\\
\citealt{Nau1999} & \isit{interrogative} paradigms & European, Australian languages & ca. 20?\\
\citealt{Huang1999} & general & languages of \isit{Taiwan} & 7\\
\citealt{Siemund2001} & general & global & ca. 50?\\
\citealt{Bencini2003} & \isit{question marking}, \isit{grammaticalization} & global, unrepresentative & ca. 25?\\
\citealt{Diessel2003} & interrogatives & global, unrepresentative & 100\\
\citealt{Bhat2004} & interrogatives & global, unrepresentative & ca. 80?\\
\citealt{Idiatov2004} & interrogatives & global, representative & ca. 350\\
\citealt{Hackstein2004} & \isi{rhetorical question}, \isit{grammaticalization} & Indoeuropean languages & ?\\
\citealt{Cysouw2005} & interrogatives & global, unrepresentative & ?\\
\citealt{Cysouw2007} & interrogatives, \isit{grammaticalization} & \ilit{Pichis Ashéninca} (\ilit{Arawakan}),
global, unrepresentative & ?\\
\citealt{Idiatov2007} & interrogatives (who, what) & global, representative & 1850\\

\citealt{KönigSiemund2007} & general & global, unrepresentative & ?\\
\citealt{Schulze2007} & general & global, unrepresentative & ?\\
\citealt{Lichtenberk2007} & interrogatives & Oceanic languages & ca. 25\\
\citealt{Hagège2008} & \isit{interrogative} verbs & global, representative? & 28 (217?)\\
\citealt{Mauri2008} & \isit{question marking}, \isit{alternative questions} & global, representative & ca. 60?\\
\citealt{Mackenzie2009} & interrogatives, \isit{hierarchy} & global, unrepresentative & 50\\
\citealt{Rialland2009} & \isit{intonation} & African languages & 145\\
\citealt{Stivers2009} & \isit{turn-taking} & global, unrepresentative & 10\\
\textit{Journal of Pragmatics} 42\newline (e.g., \citealt{Levinson2010}) & general & global, unrepresentative & 10\\
\citealt{Axelsson2011} & \isit{tag questions} & global, unrepresentative & ca. 12?\\
\citealt{Miestamo2011} & polar \isit{question marking} & \ilit{Uralic} languages & ca. 30 (200?)\\
\citealt{Dixon2012} & general & global & ca. 30?\\
\citealt{Hengeveld2012} & interrogatives & languages of Brazil & 24\\
\citealt{Dryer2013k} [2005] & polar \isit{question marking} & global, representative & 884\\
\citealt{Dryer2013l} [2005] & interrogatives, position & global, representative & 902\\
\citealt{Dryer2013m} [2005] & polar \isit{question marking} & global, representative & 955\\
\citealt{LuoTianhua2013} & general & languages of \isit{China} & 138\\
\citealt{Hackstein2013} & \isit{polar question} marking,
\isit{grammaticalization} & \isit{Indoeuropean} languages & ca. 10\\
\citealt{DingemanseTorreiraEnfield2013} & the word \textit{\isi{huh}} & global, unrepresentative & 31\\
\citealt{Haspelmath2013a}& interrogatives, position & \isit{pidgins} and \isit{creoles} & 76\\
\citealt{Haspelmath2013b} & \isit{question marking} & \isit{pidgins} and \isit{creoles} & 76\\
\citealt{Köhler2013} & \isit{question marking} & \ilit{Ometo} languages & 6\\
\citealt{O’Connor2014} & interrogatives, position & south American Indian languages & 44\\
\citealt{Hölzl2015b} & \isit{question marking} & global, unrepresentative & 50\\
\citealt{Bowern2016} & general & \isit{hunter-gatherer} languages & 194\\
\citealt{Hölzl2016a} & \isit{question marking} & global, unrepresentative & ca. 20\\
\citealt{Köhler2016} & general & African languages & 119?\\
\lspbottomrule
\end{tabularx}
\end{table}

The typology proposed in this study differs from many previous typological accounts that usually first drew a distinction between different question types, especially polar and \isi{content question}s. A similar focus on polar and \isi{content question}s can also be found in most grammar books and specialized descriptions of \isi{questions} in individual languages. In contrast, the present study makes a \textit{primary distinction} between (1) \isi{question marking}, (2) interrogatives, and (3) other functional domains such as \isi{focus} that can combine with \isi{question marking} or interrogatives. Only a secondary distinction is made within the domain of \isi{question marking} in different question types (\citealt{Hölzl2016a}). Of course, this is not an altogether new endeavor. Similar ideas have already been formulated, for example, by \citet[248-249]{Bhat2004} for content \isi{questions}.

\begin{quote}
The purpose of using these pronouns [interrogatives {\textasciitilde} indefinites] in such sentences is merely to indicate that the speaker lacks knowledge regarding a particular constituent. There are two other meanings that need to be expressed in constituent \isi{questions} [CQ], namely 
(i) a \isi{request} for information (\isi{interrogation}) and 
(ii) restriction of that \isi{request} to a particular constituent (namely the indefinite pronoun); these meanings are generally expressed, in these languages, with the help of additional devices; for example, devices like the use of question particles or question \isi{intonation} are used for denoting \isi{interrogation}, whereas devices like the use of \isi{focus} particles or \isi{focus} constructions are used for denoting that the \isi{interrogation} is restricted to a particular constituent. (my square brackets)
\end{quote}

\noindent Bhat’s approach was the impetus for a primary separation of \isi{question marking} from \isi{interrogative}s and \isi{focus} in this study. However, while Bhat concentrated exclusively on \isi{content question}s, this study includes further question types such as polar, alternative, and \isi{focus question}s.

This typology excludes echo, rhetorical, and \isi{indirect question}s. For the most part, research commonly known as “\isi{wh-movement}” or “wh-\isi{fronting}” (e.g., \citealt{Cable2007}) will not play a dominant role within this study either. First of all, very few languages in \isi{NEA} exhibit this phenomenon that can by no means be said to be a \isi{universal} property of language. Second, it is, strictly speaking, neither part of the domains of \isi{question marking}, nor of interrogatives, but belongs to the domain of \isi{focus} marking. This study for the most part also excludes the grammatical category of indefinites that are usually derived from interrogatives and have been discussed in detail elsewhere (see \citealt{VanAlsenoyvanderAuwera2015} for a list of references).

\section{Question marking}\label{sec:4.2}

Typological variation within the domain of \isi{question marking} includes (1) different formal marking strategies, (2) different semantic scopes of question markers over question types, (3) \isi{interaction} of \isi{question marking} with other functional domains, and (4) the overall number of question markers. But before I investigate these four aspects one after another, let me introduce the major question categories that can take \isi{question marking}.

The \textit{major question categories} are \isi{polar question}s (PQ) (also called yes-no \isi{questions}), \isi{content question}s (CQ) (also known as constituent or wh-\isi{questions}), \isi{alternative question}s (AQ) (or disjunctive \isi{questions}), and perhaps \isi{focus question}s (FQ). The designations are somewhat problematic but nevertheless will be adopted here because they are the most common and conventional labels. Consider the following examples corresponding to the \isi{declarative sentence} \textit{I want coffee} (cf. \citealt[18]{Hölzl2016a}):

\newpage 
\ea%1
    \label{ex:4:1}
    \ili{English}\\
    \ea
      \textbf{{Do}} you want coffee?\\

    \ex
      \textbf{{Do}} \textbf{you} want coffee?\\

    \ex
      \textbf{{Do}} you want coffee \textbf{{or}} tea?\\

    \ex
      \textbf{{What}} \textbf{{do}} you want (to drink)?\\
    \z
    \z 

\noindent Generally, a different question category can be postulated if it has a specialized marking in at least one but preferably more languages. \ili{English} alone, for example, fails to differentiate polar and \isi{focus question}s in terms of \isi{question marking}. Examples (1a) and (1b) exhibit the same \isi{question marking} but differ with respect to the marking of \isi{focus}. Focus is understood here in a very broad sense of a “restriction of that \isi{request} to a particular constituent” \citep[246]{Bhat2004} and usually has a contrastive function. Alternative \isi{questions}, though exhibiting the same marking, in addition contain a \isi{disjunction} preceding the second alternative, which is not the case in polar and \isi{focus question}s. Now consider the following examples from the Na-Dene language \ili{Slavey}.

\ea%2
    \label{ex:4:2}
    \ili{Slavey} (Hare, \ili{Na-Dene})\\
    \ea
    \gll \textbf{{sú}} duká    ˀanehˀ\k{i}?\\
    \textsc{q}  like.this  2\textsc{sg}.do\\
    \glt ‘Do you do it this way?’ (\citealt{Rice1989}: 1123)
    
    \ex
    \gll duká \textbf{{n\k{i}}} {ˀanehˀ\k{i}?}\\
    like.this  \textsc{q}  2\textsc{sg}.do\\
    \glt ‘Is this the way that you do it?’ (\citealt{Rice1989}: 1133)

    \ex
    \gll łegǫ́hl\k{i}n\k{i}  gots’\'{ę}  ˀawǫdee \textbf{{gúsh\k{i}}} kǫ́é  ˀawǫt’é?\\
    \textsc{pn}    area.to  2\textsc{sg.opt}.go  or  home  2\textsc{sg}.\textsc{opt}.be\\
    \glt ‘Are you going to Norman Wells or staying home?’ (\citealt{Rice1989}: 1139)

    \ex \ili{Slavey} (Mountain, Na-Dene)\\
    \gll \textbf{{ˀamíi}} {ˀay\'{\k{i}}}{lá?}\\
    who  3.affected.4\\
    \glt ‘Who did it?’ \citep[1141]{Rice1989}
    \z
    \z 

\largerpage[-2]
\noindent The examples in (\ref{ex:4:2}a) to (\ref{ex:4:2}d) are instances of polar, \isi{focus}, alternative, and \isi{content question}s, respectively. Unlike \ili{English}, there is a clear distinction between polar and \isi{focus question}s. Polar, content, and \isi{alternative question}s are generally accepted as separate categories, although \isi{alternative question}s are sometimes subsumed under \isi{polar question}s (e.g., \citealt{Siemund2001}). \textit{Focus questions} \citep{Kiefer1980}, on the other hand, are a category that is not usually recognized or included in grammatical descriptions, but nevertheless has some validity as shown in example (\ref{ex:4:2}b) above (see also \citealt{Dixon2012}: 395-396). \citet[2]{Miestamo2011} includes them in his definition of \isi{polar question}s “encompassing all interrogatives eliciting yes/no replies, regardless of whether they are neutral or biased towards a positive or negative \isi{answer}, or whether they have [a] broad sentence \isi{focus} or a more narrow \isi{focus} on a particular constituent.” The prime example for the cross-linguistic relevance of \isi{focus question}s used in \cite{Hölzl2016a} is the \ili{Japonic} language \ili{Yuwan} (see \sectref{sec:5.6.2}), which contains specialized question markers for polar (\textit{-mɨ}), \isi{focus} (\textit{-ui}), and \isi{content question}s (\textit{-u}). \citet[396]{Dixon2012} mentions yet another example with different polar (\textit{-ée}) and \isi{focus} \isi{question marking} (\textit{-áa}), the Cushitic (\ili{Afroasiatic}) language \ili{Tunni}. However, he also includes \isi{focus question}s in the category of \isi{polar question}s. \citet{Miestamo2011} is certainly correct that in many languages \isi{focus question}s exhibit affinities with polar \isi{questions}, but this is not always the case. In the Australian language \ili{Bardi} (\ref{ex:4:3}), for example, there is an affinity between \isi{focus} and alternative, but not \isi{polar question}s, which are marked with sentence-initial \textit{nganyji}, which is derived from an \isi{interrogative}.

\ea%3
    \label{ex:4:3}
    \ea
	\ili{Bardi} (\ili{Nyulnyulan})\\
    \gll gooyarr=\textbf{{arda}} aarli  mi-n-Ø-nya-gal?\\
    two=\textsc{q}      fish  2-\textsc{tr}-[\textsc{pst}]-catch-\textsc{rec.pst}\\
    \glt ‘Was it \textit{two} fish that you caught?’\\

    \ex
    \gll ngay=\textbf{{arda}} nga.n.k.iid.a  broome-ngan, \textbf{{gardi}} joo=\textbf{{warda}}?\\
    1\textsc{min}=\textsc{q}  go    \textsc{pn}-\textsc{all}    or  2\textsc{min}=\textsc{q}\\
    \glt ‘Will I go to Broome today or will it be you?’ \citep[619]{Bowern2012}
    \z
    \z

\noindent There do indeed seem to be relatively few languages with specialized marking strategies for \isi{focus question}s, although this might be a distortion due to the fact that the category itself is not widely known in linguistic circles and therefore did not make it into grammatical descriptions.

Following the methodology sketched above, there are some indications for additional types of \isi{questions}. One such example are \textit{\isi{negative polar question}s} (NPQ). In most languages, including \ili{English}, these are expressed in the same way as plain \isi{polar question}s except for the addition of a negator, e.g. \textbf{\textit{Don’t}} \textit{you want coffee?} But in \ili{Urarina}, a language with no clear affiliations to other languages spoken in Peru, there is a specialized NPQ marker \textit{ta} different from the plain PQ marker \textit{=na}. An NPQ in addition requires a negative marker \textit{=ne}.

\ea%4
    \label{ex:4:4}
    \ea
    \ili{Urarina}\\
    \gll hanone    mẽsahe    auna-i=\textbf{{ɲa}}?\\
    morning  message  hear-2=\textsc{q}\\
    \glt ‘Did you hear the message in the morning?’

    \ex
    \gll \textbf{{ta}} kure  kwitʉkʉ-i=\textbf{{ɲe}}?\\
    \textsc{q}  price  know-2=\textsc{neg}\\
    \glt ‘Don’t you know the price?’ (\citealt{Olawsky2006}: 832, 834)
    \z
    \z

\noindent Note the different syntactic behavior of the plain and the \isi{negative polar question} markers. This category seems to play no important role for \isi{NEA}, however, which is why it has not been taken into account in this study (but see \sectref{sec:5.6.2} on \ili{Shuri}).

\newpage 
Minor subtypes of \isi{questions} include \textit{\isi{negative alternative question}s} (NAQ, \textit{Do you want coffee} \textbf{\textit{or not}}?) and \textit{\isi{open alternative question}s} (OAQ, \textit{Do you want coffee} \textbf{\textit{or what}}?), but their cross-linguistic relevance remains to be investigated. NAQs are mentioned as a separate category because they play a very important role in the \isi{grammaticalization} of polar \isi{question marker}s (\sectref{sec:4.2.3}). The category of open alternative questions has been proposed by \citet{TolskayaTolskaya2008}. NAQs and OAQs, it seems, do not fulfill the requirement as an independent \isi{question type} as they are generally based on the same construction as \isi{alternative question}s. In this study the two are simply taken as useful labels for special subtypes of alternative \isi{questions}.

\tabref{tab:4:2} lists some defining properties of the major question types described above. There may well be additional properties, but these are the most important ones for the purposes of this study. The term \textit{proposition}, which is a very common and useful label, should not be understood in a logical sense, but in terms of \textit{\isi{embodied simulation}} (see \sectref{sec:4.4}). Polar questions and \isi{focus question}s both expect an \isi{answer} in the positive or in the negative. If they are marked with the same marker, this usually attaches to the verb in polar \isi{questions} and to the focal element in \isi{focus question}s. Very often, the same \isi{question marker} can also be found in \isi{alternative question}s, where it attaches once to each alternative. Focus questions and \isi{content question}s share a narrow \isi{focus} on one constituent. The difference lies in the fact that in content \isi{questions} this usually is an \isi{interrogative}, while in \isi{focus} \isi{questions} it is usually a fully specified nominal or verbal phrase or other part of the sentence.

\begin{table}
\caption{Important properties of individual question types (cf. \citealt{Dik1997}: 260; \citealt{Dixon2012}; \citealt{Hölzl2015e,Hölzl2016a}).}
\label{tab:4:2}

\begin{tabularx}{\textwidth}{lQQQQ}
\lsptoprule
& \textbf{Expectancy from hearer} & \textbf{Questioned part of proposition} & \textbf{Level of specificity} & \textbf{Specified alternatives}\\
\midrule
CQ & specific information & part (or whole) & schematic & none\\
PQ & \isit{confirmation} or \isit{negation} & whole & specific & one\\
FQ & id. & part & id. & id.\\
AQ & \isit{selection} from alternatives & part (or whole) & specific (or schematic) & two or more\\
\lspbottomrule
\end{tabularx}
\end{table}

However, \isi{content question}s do not necessarily contain an \isi{interrogative} as can be seen in \ili{Wari’}, a Chapacuran language spoken in Brazil in which \isi{demonstratives} fulfill the function of interrogatives (\citealt{EverettKern1997}). The presence of interrogatives as a defining feature of \isi{content question}s is also problematic otherwise. For instance, there are languages in which an \isi{interrogative} develops into a polar \isi{question marker} but is still identical to the \isi{interrogative}. Content \isi{questions} are better defined as \isi{questions} that have a narrow sentence \isi{focus} on a schematic subpart of the “proposition” and thus inquire about very specific information instead of a \isi{confirmation} (cf. \citealt{Dixon2012}). For instance, \textit{Are you leaving tomorrow}? (FQ) is much more specific than \textit{When are you leaving?} (CQ). The level of \isi{specificity} has been adopted from \citet[238]{Arnheim1969} and \citet[19]{Langacker2008} and will be further described in \sectref{sec:4.4}. Open alternative questions are partly schematic.

Polar \isi{questions} are unique in inquiring about the whole, but there are some alternative \isi{questions} such as \textit{Is it raining or did someone leave the sprinkler on?} (\citealt{SadockZwicky1985}: 179) and content \isi{questions} of the \textsc{reason} type (\textit{why?}) that are similar in this regard. The \isi{answer} to a question such as \textit{Why are you leaving?} may either be a whole (\textit{It is going to rain.}) or a partial clause (\textit{Because of the rain}). In fact, in many languages interrogatives referring to the category of \textsc{reason} are derived from \isi{interrogative} verbs. In many instances in \isi{NEA} these are \isi{converb} forms of an \isi{interrogative verb} meaning ‘to do what’, e.g. \ili{Manchu} \textit{ai-na-me} ‘what-\textsc{v}-\textsc{cvb.ipfv}’. Because verbs usually stand for the whole “proposition”, this is direct evidence that \textsc{reason} interrogatives may also refer to the whole as well. But most alternative \isi{questions} and \isi{content question}s, as well as all \isi{focus question}s, \isi{focus} on a subpart of a given “proposition”. This is directly mirrored in overt \isi{focus} markers that are often found in \isi{focus} and content \isi{questions} as well as the fact that the part of \isi{alternative question}s that is identical in both alternatives (the unfocused part) may fall victim to ellipsis in one of the two alternatives in the majority of languages.

Another difference concerns the number of alternatives that are specified \citep[260]{Dik1997}. Content \isi{questions} imply many possible alternatives but specify none. All expected alternatives, however, have in common the schematic meaning just mentioned. In other words, an \isi{answer} to \textit{When are you leaving?} may be \textit{tomorrow}, \textit{the day after tomorrow}, \textit{July 4} etc. If the expectancy was wrong, the \isi{answer} may also be different (e.g., \textit{No, I am not leaving at all.}), but this is something that all question categories have in common (see \sectref{sec:4.4}). Polar and \isi{focus question}s imply more than one alternative but have only one specified. Alternative questions by definition have more than one alternative and do not usually imply more (but see example \ref{ex:4:22} below from \ili{Mauwake}).

In some cases it is not easy to differentiate between types of \isi{question marking} and types of \isi{questions}. Two of the most difficult examples are \textit{\isi{tag question}s} and so-called \textit{\isi{A-not-A question}s}, both of which in previous studies have been treated as either a \isi{question type} or marking strategy of polar \isi{questions} (see \citealt{Hölzl2016a}: 20). \REF{ex:4:5} and \REF{ex:4:6} are typical examples (based on own knowledge).

\ea%5
    \label{ex:4:5}
    \ea
    \ili{English}\\
      You want coffee, \textbf{{right}}?\\

    \ex
      You want coffee, \textbf{{don’t}} \textbf{{you}}?\\
    \z
    \z

\ea%6
    \label{ex:4:6}
    \ili{Mandarin}\\
    \zh{你要不要喝茶}?\\
    \gll {n\u{\i}} \textbf{{yào}}  \textbf{{bu}}  \textbf{{yào}} h\=e  chá?\\
    2\textsc{sg}  want  \textsc{neg}  want  drink  tea\\
    \glt ‘Do you want to drink tea?’
    \z

\newpage     
\noindent The solution proposed by \cite{Hölzl2016a} is to classify A-not-A as elliptic \isi{negative alternative question}s without overt marking, but \isi{juxtaposition} of the two alternatives. From this point of view they are neither a marking strategy for polar \isi{questions}, nor a \isi{question type} on their own, but are a very special case of alternative \isi{questions} (see also \citealt{Clark1985}). The category of \isi{tag question}s was either not described or are simply non-existent in the majority of languages in \isi{NEA} and therefore do not play a significant role within this study. But their presence in many languages from other parts of the world, such as \isi{Europe}, means that they cannot be neglected. Problematically, many investigations of tag \isi{questions} fail to define them properly. Furthermore, what is usually recognized as a \isi{tag question} has a plethora of different meanings, which makes them extremely difficult to define from a functional point of view (\citealt{Mithun2012}: 2166f.). The fact that tag \isi{questions} can be described in terms of a \textit{\isi{question tag}} in relation to a so-called \textit{anchor} \citep{Axelsson2011} makes them unique among all of the question types mentioned above. In fact, the traditional category of tag \isi{questions} usually consists of two sentences, which is why they do not, actually, qualify as \isi{question type} at all. Perhaps tag \isi{questions} thus have to be described in different terms. For reasons that will become clearer in \sectref{sec:4.4}, tag \isi{questions} will be treated as a construction type located on a different level of \isi{analysis} than the other question types. However, for practical purposes their formal properties will be briefly discussed with the other question types in \sectref{sec:4.2.1}.

The following subsections address marking strategies (\sectref{sec:4.2.1}), the scope of \isi{question marking} over different question types (\sectref{sec:4.2.2}), the \isi{interaction} of \isi{question marking} with other functional domains such as \isi{focus} (\sectref{sec:4.2.3}), and finally, the overall number of question markers in individual languages (\sectref{sec:4.2.4}).

\subsection{Marking strategies}\label{sec:4.2.1}
\largerpage

Previous accounts of \isi{question marking} are often restricted to the marking of \isi{polar question}s (e.g., \citealt{Miestamo2011}; \citealt{Dryer2013k,Dryer2013m}). This section takes a broader perspective and investigates \isi{question marking} strategies in all question types, including, for the sake of completeness, the problematic category of tag \isi{questions}.

\citet{Miestamo2011}, in analogy to his earlier typology of \isi{negation}, investigates the distinction and symmetry between the marking of polar \isi{questions} and declarative sentences. Some of the categories in \citet{Dryer2013m} are likewise based on such a comparison. My own typology builds on these approaches and draws a broad distinction between marked and unmarked \isi{polar question}s and declaratives, which defines the four different types shown in \tabref{tab:4:3}.

\begin{table}
\caption{Marking of polar questions as opposed to declaratives (cf. \citealt{Hölzl2016a}: 21)}
\label{tab:4:3}

\begin{tabularx}{\textwidth}{XXl}
\lsptoprule
& \textbf{unmarked PQ} & \textbf{marked PQ}\\
\midrule
\textbf{marked declarative} & Type 1: \ili{Sanuma}, \ili{Sheko} & Type 3: \ili{Crow}, \ili{Sabanê}\\
\textbf{unmarked declarative} & Type 2: \ili{Yélî Dnye} & Type 4: \ili{English}, \ili{Bengali}\\
\lspbottomrule
\end{tabularx}
\end{table}

\clearpage 
For Types 1 and 2 consider examples from the Ethiopian language \ili{Sheko} \REF{ex:4:7} and \ili{Yélî Dnye} \REF{ex:4:8}, a language without clear affiliation spoken on Rossel Island.

\ea%7
    \label{ex:4:7}
    \ili{Sheko} (Omotic, Afroasiatic)\\
    \ea
    \gll n̩=m\=a\=ak-\=a-\textbf{{m}}.\\
    1\textsc{sg}=tell-put-\textsc{irr}\\
    \glt ‘I will tell.’

    \ex
    \gll n̩=m\=a\=ak-\=a?\\
    1\textsc{sg}=tell-put\\
    \glt ‘Shall I tell?’ \citep[402]{Hellenthal2010}
    \z
    \z

\ea%8
    \label{ex:4:8}
    \ili{Yélî Dnye}\\
    \gll yi kópu dê d:uu./?\\
    that.\textsc{anaph}  thing  3.\textsc{imm.pst.pl}  make.\textsc{pl}\\
    \glt ‘He made it./?’ \citep[2743]{Levinson2010}
    \z

Note that \ili{Yélî Dnye} even lacks a distinction of \isi{intonation}. Types 1 and 2 are exceedingly rare and are altogether absent from \isi{NEA}, which is why they will be neglected in this study (see \citealt{Köhler2013} for Type 1). Type 4 is by far the most frequent cross-linguistically, followed by Type 3.

The \isi{interaction} of overt question markers with \isi{intonation} complicates matters, but this will be ignored for the moment. The South American language \ili{Sabanê} \REF{ex:4:9} and \ili{Bengali} \REF{ex:4:10} illustrate Types 3 and 4, respectively.

\ea%9
    \label{ex:4:9}
    \ili{Sabanê} (Nambikwaran)\\
    \ea
    \gll {iney-}{i-}{ntal-}\textbf{{i}}.\\
    fall-\textsc{v}-\textsc{pret.neut}-\textsc{decl}\\
    \glt ‘(S)he fell.’

    \ex
    \gll {iney-i-}{ntal-}\textbf{{a}}?\\
    fall-\textsc{v}-\textsc{pret.neut}-\textsc{q}\\
    \glt \glt ‘Did (s)he fall?’ (\citealt{Araujo2004}: 205)
    \z
    \z

\ea%10
    \label{ex:4:10}
    \ili{Bengali} (\ili{Indo-Iranian}, Indo-European)\\
    \ea
    \gll tumi    take    cenô.\\
    2\textsc{sg}    3\textsc{sg}.\textsc{obj}  know.2.\textsc{pr.s}\\
    \glt ‘You know him.’
    
    \ex
    \gll tumi \textbf{{ki}} take    cenô.\\
    2\textsc{sg}  \textsc{q}  3\textsc{sg}.\textsc{obj}  know.2.\textsc{pr.s}\\
    \glt ‘Do you know him?’ \citep[200]{Thompson2012}
    \z
\z

\noindent The different morphosyntactic status of the marker is unimportant for this primary distinction.

However, by considering Types 3 and 4 exclusively, there is a variety of different formal types of \textit{\isi{polar question} marking} (e.g., \citealt{Siemund2001}; \citealt{Miestamo2011}; \citealt{Dryer2013k,Dryer2013m}). Spoken language is one-dimensional. In order to signal certain information such as \isi{interrogativity}, there are thus limited means available. We may simply modulate the phonation of the speech stream (\isi{intonation}), change the order of elements in the speech stream (\isi{word order}), or we may add material (morphosyntax). Among the elements that can be added are affixes, clitics, or free elements such as particles. These may stand either before or after another element (prefixes vs. suffixes, proclitics vs. enclitics, preposed vs. postposed particles). The element with respect to which these question markers can be located may either be the whole sentence or a subpart such as the first constituent or the verb. Affixes are less free in their position than clitics and particles, and usually attach to the verb.

Apart from some exceptions, \textit{intonation} is not normally described in detail for languages in \isi{NEA}, if it is mentioned at all. Within this study it was impossible to remedy this unfortunate fact, but where possible some rough outlines are sketched (such as falling or rising \isi{intonation} etc.). Intonation, although not \isi{universal}, is certainly among the most important ways of marking \isi{questions} cross-linguistically. However, in the majority of languages, \isi{intonation} is combined with other markers. In \citegen{Dryer2013m} sample of 955 languages only 173 languages (about 18\%) exclusively made use of \isi{intonation} for polar \isi{question marking}. In \isi{NEA} the number is even lower (\chapref{sec:6}). Concerning the location and contour of question \isi{intonation} there are no absolute universals (see \citealt{Sicoli2014}: 4 and references therein). In fact, generalizations such as final rising \isi{intonation} in \isi{polar question}s are not true for individual languages like \ili{English} (\citealt{Couper-Kuhlen2012}), let alone from a cross-linguistic perspective. For example, \citet[928]{Rialland2009} describes what she calls the \textit{lax question prosody} found in a relatively large area of central and western Africa, which is generally characterized by “a falling pitch contour, a sentence-final low vowel, vowel lengthening, and a breathy utterance termination produced by the gradual opening of the glottis.” Because of the absence of reliable and good information on \isi{intonation}, this study necessarily focuses on the material aspect of \isi{question marking}.

Question marking by \textit{\isi{word order} change} is almost entirely restricted to Western \isi{Europe} and \ili{Indo-European} languages (e.g., \citealt{Hackstein2013}), and is extremely rare from a cross-linguistic perspective \citep{Dryer2013m}. This is a feature of European languages that clearly differentiates them from the rest of Eurasia, including \isi{NEA}. An example can be found in \ili{Finnish} \REF{ex:4:11}.

\ea%11
    \label{ex:4:11}
    \ili{Finnish} (\ili{Uralic})\\
    \ea
    \gll sä {tuu-}{t.}\\
    2\textsc{sg}    come-2\textsc{sg}\\
    \glt ‘You’re coming.’

    \ex
    \gll \textbf{{tuu}}{-}{t} sä?\\
    come-2\textsc{sg}  2\textsc{sg}\\
    \glt ‘Are you coming?’ (\citealt{Miestamo2011}: 7, 12)
    \z
    \z

\noindent This seems to be a pattern that originates in \ili{Germanic} languages from where it spread to some \ili{Uralic}, Romance, and \ili{Slavic} languages. Following \citet[12]{Miestamo2011}, one may assume an original second position enclitic marking \isi{questions} as well as \isi{focus}. Such markers normally attach to the fronted verb in \isi{polar question}s and the loss of this marker quite naturally leaves the \isi{fronting} of the verb to mark polar \isi{questions}. We furthermore know that several \ili{Indo-European} languages had a second position clitic or particle such as \textit{=ne} in \ili{Latin} (cf. \sectref{sec:5.5.2}). According to Miestamo this is also what happened in \ili{Finnish}, which still preserves a second position clitic in other constructions.

\ea%12
    \label{ex:4:12}
    \ili{Finnish} (\ili{Uralic})\\
    \gll tule-t=\textbf{{ko}} sinä?\\
    come-2\textsc{sg}=\textsc{q}    2\textsc{sg}\\
    \glt ‘Are you coming?’ \citep[12]{Miestamo2011}
    \z

\noindent Perhaps, \ili{Germanic} had a second position clitic comparable to \ili{Gothic} \textit{=u} (\citealt{BrauneHeidermanns2004}: 178) that was already lost in other Old \ili{Germanic} languages. While the loss of the \isi{question marker} is not actually attested for \ili{Germanic}, it is for some other European languages such as the \ili{Uralic} language Pite \ili{Saami}. \cite[186-187, 244]{Wilbur2014} notes that there used to be a second position \isi{question marker} \textit{=gu(s)} in Pite \ili{Saami} that attached to a verb in \isi{polar question}s and that almost entirely disappeared during the 20th century. Today, polar \isi{questions} are usually marked by verb-initial \isi{word order} only. Of course, the development in languages such as Pite \ili{Saami} may have been influenced by \isi{language contact} as well.

Many examples of different morphosyntactic markers can be found throughout this section as well as in \chapref{sec:5}, which is why no further examples will be given here. A rare strategy is the use of \textit{infixes} such as in \ili{Koasati}, a \ili{Muskogean} language spoken in the US (cf. \citealt{Dixon2012}: 384). In \ili{Koasati}, \isi{questions} may be “formed by infixing a glottal stop between the penultimate and ultimate syllables.” \citep[301]{Kimball1991} The \ili{Koasati} \isi{question marker} is a true infix \textit{-ʔ-} that can, but does not necessarily, coincide with a mopheme boundary \citep[302]{Kimball1991}. Similarly rare are \textit{auxiliaries} that are restricted to marking \isi{questions} \citep[4]{Miestamo2011}. One example stems from the \ili{Salish} language \ili{Halkomelem}, which has the auxiliary \textit{lí-}. This should not be confused with auxiliaries encountered in, but not restricted to, \isi{questions} such as \ili{English} \textit{to do}, or with \isi{interrogative} verbs such as \textit{to do what} that are interrogatives and not question markers (e.g., \citealt{Hagège2008}). According to \citet[593]{HymanLeben2000}, there are some languages in which \isi{questions} can be marked with \textit{tones}:

\begin{quote}
In \textbf{\ili{Hausa}} \{Chadic, \ili{Afroasiatic}\}, a L is added after the rightmost lexical H in a yes/no question, fusing with any pre-existing lexical L that may have followed the rightmost H (which is raised somewhat, as are any following L \isi{tones} whatever their source). As a result, lexical tonal contrasts are neutralized. In statements, [káì] ‘head’ is tonally distinct from [káí] ‘you [masculine]’. But at the end of a yes/no question, they are identical, consisting of an extra-H gliding down to a raised L. In \textbf{\ili{Nembe}} \{Ijoid, ?\ili{Niger-Congo}\}, a final lexical L becomes H in statements, and a final lexical H becomes L in \isi{questions}. Thus, L-L / LH contrasts such as [dìrì] ‘book’ / [bùrú] ‘yam’ are neutralized as L-H in statements, but as L-L in \isi{questions}. A similar case is found in \textbf{\ili{Isoko}} \{Atlantic-Congo, \ili{Niger-Congo}\}, where a final L marks positive \isi{questions}, while a final H marks negative \isi{questions}. This causes a final lexical L to remain L in a positively expressed question, while this final L becomes a LH rise in a negatively expressed question: [ùbì] ‘book’ / [ùb\u{\i}] ‘book? [negative]’. (my boldface and braces)
\end{quote}

\noindent No example has been found in \isi{NEA} for these last three types of \isi{question marking}.

Generally, it seems, the same \isi{question marking} strategies as in polar \isi{questions} can also be employed in other question types. However, this has not actually been investigated. \cite[292]{KönigSiemund2007}, for instance, argue that “\isi{alternative question}s can be neglected since, at least from our current perspective, they do not seem to show any striking typological variation.” This general negligence of \textit{alternative questions} may be partly due to the fact that in any given language they are known to be much less frequent than polar or \isi{content question}s \citep[2728]{Hoymann2010}. But \cite{Siemund2001} and \cite{KönigSiemund2007} are clearly wrong in their assessment that \textbf{alternative questions} do not exhibit any interesting variation to be discovered. On the contrary, they actually exhibit much more variation than \isi{polar question}s because, in addition to the \isi{question marking} strategies encountered above, they show \isi{interaction} with \isi{coordination}, have two or more possible loci of marking, and display interesting patterns of ellipsis that may affect the \isi{question marking}.

The simplest marking strategy is a mere \isi{juxtaposition} of the two alternatives. However, the two alternatives may still be marked with \isi{intonation} patterns that are not always specified. For instance, in \ili{Amis} \REF{ex:4:13} each alternative is marked with “a leveling-rising-falling \isi{intonation} pattern” (\citealt{Huang1999}: 650).

\ea%13
    \label{ex:4:13}
    \ili{Amis} (Nuclear Austronesian, \ili{Austronesian})\\
    \gll ma-tayal  kísu    ma-fúti?\\
    \textsc{ag.foc}-work  2\textsc{sg.nom}  \textsc{ag.foc}-sleep\\
    \glt ‘Are you going to work or sleep?’ (\citealt{Huang1999}: 651)
    \z

So-called \isi{A-not-A question}s, frequently encountered in MSEA, are perhaps best analyzed as a subtype of this type of \isi{alternative question} marking with an additional negator.

\ea%14
    \label{ex:4:14}
    \ili{Mon} (Monic, \ili{Austroasiatic})\\
    \gll klɜŋ \textbf{{hùˀ}} klɜŋ?\\
    come  \textsc{neg}  come\\
    \glt ‘Are (you) coming (or not)?’ \citep[60]{Clark1985}
    \z

In other cases the two alternatives may be conjoined with the help of a \isi{disjunction}. For example, \ili{Saisiyat} \REF{ex:4:15} makes exclusive use of a \isi{disjunction}, but lacks any further \isi{question marking}, including \isi{intonation}.

\newpage 
\ea%15
    \label{ex:4:15}
    \ili{Saisiyat} (Nuclear Austronesian, \ili{Austronesian})\\
    \gll niʃo    ʔam  ŋyaw \textbf{{a}} ʔam  ʔahœʔ?\\
    2\textsc{sg.gen}  want  cat  or  want  dog\\
    \glt ‘Do you want a cat or do you want a dog?’ (\citealt{Huang1999}: 652)
    \z

Some languages such as \ili{Finnish} have a special \isi{interrogative} \isi{disjunction} (\textit{tai}) that is not identical to the standard \isi{disjunction} (\textit{vai}) (e.g., \citealt{Haspelmath2007}: 25). In other languages there is no \isi{disjunction} but a \isi{question marker}, for example on the first alternative. Consider the following negative \isi{alternative question} \REF{ex:4:16}, which exhibits the same \isi{question marker} found in polar \isi{questions}.

\ea%16
    \label{ex:4:16}
    \ili{Guiqiong} (\ili{Qiangic}, \ili{Trans-Himalayan})\\
    \gll zo gutɕhiɐŋ dʐi \textbf{{lɐ}}, \textbf{{mɛ}}{-dʐi}?\\
    3\textsc{sg}  \textsc{pn}    \textsc{cop}  \textsc{q}   \textsc{neg-cop}\\
    \glt ‘Is (s)he a \ili{Guiqiong} or not?’ (\citealt{Jiang2015}: 305)
    \z

In \ili{English} (as in the translation of 16 above) the polar \isi{question marking} strategy on the first alternative (in this case \isi{word order} change) is combined with a \isi{disjunction}, which appears to be a common European phenomenon. However, this is combined with a special \isi{intonation} contour in \ili{English}, which rises on the first and falls on the second alternative. In other languages, there is a \isi{question marker} attached to the second alternative. The following example \REF{ex:4:17} is also a negative \isi{alternative question}.

\ea%17
    \label{ex:4:17}
    \ili{Palula} (\ili{Dardic}, \ili{Indo-European})\\
    \gll tu    the  phedíl-u, \textbf{{ki}} \textbf{{na}}?\\
    2\textsc{sg.nom}  to  arrive.\textsc{pfv}-\textsc{sg.m}  ?\textsc{q}  \textsc{neg}\\
    \glt ‘Did you receive it or not?’ \citep[404]{Liljegren2016}
    \z

This, again, may be combined with disjunctions. In other languages there are question markers on each alternative, with or without \isi{disjunction}. Examples for these types can be found below such as in \REF{ex:4:21}. \tabref{tab:4:4} schematically shows some possible types of \isi{interaction} of \isi{disjunction} and \isi{question marking}. Of course, it is simplified and does not show all possible marking strategies such as the use of \isi{intonation}, particles, clitics, affixes etc. It merely schematically illustrates \isi{juxtaposition}, \isi{single marking} on the first or second alternative and \isi{double marking}, all of which may combine with disjunctions. It becomes apparent that there are dozens of combinations of these patterns with different marking strategies, which makes it impossible to present them all in this section. Each type, furthermore, can interact with other domains such as \isi{negation}.

\begin{table}
\caption{Schematic interaction of disjunction and question marking.}
\label{tab:4:4}

\begin{tabularx}{\textwidth}{XXl}
\lsptoprule
\textbf{1st alternative} &  & \textbf{2nd alternative}\\
\midrule
X & +/-\textsc{disj} & Y\\
X-\textsc{q} & +/-\textsc{disj} & Y\\
X & +/-\textsc{disj} & Y-\textsc{q}\\
X-\textsc{q} & +/-\textsc{disj} & Y-\textsc{q}\\
\lspbottomrule
\end{tabularx}
\end{table}

What is more, the plethora of different patterns in \tabref{tab:4:4} above does not even cover all \isi{alternative question} marking strategies found in the languages of the world. \ili{Khwarshi}, for example, in addition to \isi{double marking}, contains cases in which the \isi{disjunction} \textit{ya(gi)}, borrowed from Awar, is not employed once but twice \REF{ex:4:18}.

\newpage
\ea%18
    \label{ex:4:18}
    \ili{Khwarshi} (\ili{Tsezic})\\
    \gll me \textbf{{ya}} ło-\textbf{{k}} n-eq-še \textbf{{ya}} lac’á-\textbf{k} l-i-še?\\
    2\textsc{sg.erg}  or  water(IV)-\textsc{q}  IV-bring-\textsc{prs}  or  food(IV)-\textsc{q} IV-do-\textsc{prs}\\
    \glt ‘Will you bring water or make the meal?’ \citep[458]{Khalilova2009}
    \z

The language \ili{Edo} (\ili{Niger-Congo}) has a \isi{disjunction} \textit{rà}, either once between the two alternative, or twice following each alternative \citep[23]{Ọmọruyi1988}. Additionally, the markers on the different alternatives are not necessarily identical as can be illustrated with data from \ili{Tshangla} as spoken in Bhutan \REF{ex:4:19}.

\ea%19
    \label{ex:4:19}
    \ili{Tshangla} (\ili{Trans-Himalayan})\\
    \gll ser-ga rengan tang-pe \textbf{{mo}}, shing-ga rengan tang-pe \textbf{ya}?\\
    gold-\textsc{loc}  ladder  bridge-\textsc{inf}  \textsc{q}  wood-\textsc{loc}  ladder  bridge-\textsc{inf} \textsc{q}\\
    \glt ‘Should I put up a silver ladder or a wooden ladder?’ \citep[193]{Andvik2010}
    \z

\noindent In \ili{Tshangla}, \textit{mo} is also a \isi{polar question} marker and \textit{ya}, which is optional in alternative \isi{questions}, is also found in \isi{content question}s.

In some languages there is a complex expression meaning ‘(and) if not’ \REF{ex:4:20}, which functions more or less like a \isi{disjunction} but should be kept distinct as it is etymologically \isi{transparent}.

\ea%20
    \label{ex:4:20}
    \ili{ǂ\=Akhoe Haiǁom} (\ili{Khoe})\\
    \gll uri  ra  ari-b.a \textbf{tama-s}    \textbf{ga}  \textbf{i-o} !gû  ra  ari-b.a?\\
    jump  \textsc{prog}  dog-3\textsc{sg.m}  \textsc{neg}-3\textsc{sg.f}  \textsc{pot}  \textsc{stat}-if walk  \textsc{prog}  dog-3\textsc{sg.m}\\
    \glt ‘Does the dog jump or does the dog walk?’ \citep[2733]{Hoymann2010}
    \z

Yet another dimension of variation concerns the number of alternatives. While it is true that the most typical \isi{alternative question}s exhibit two alternatives, there are also examples with more than two, such as in \REF{ex:4:21}.

\ea%21
    \label{ex:4:21}
    \ili{Mauwake} (\ili{Trans-New Guinea})\\
    \gll no matukar ikiw-i-nan=\textbf{{i}} dylup=\textbf{{i}} \textbf{{e}} sarang?\\
    2\textsc{sg}  \textsc{pn}    go-\textsc{n.pst}-\textsc{fut}.2\textsc{sg}=\textsc{q}  \textsc{pn}=\textsc{q}    or  \textsc{pn}\\
    \glt ‘Will you go to Matukar, Dylup, or Sarang?’ (\citealt{Berghäll2015}: 310)
    \z

\noindent \ili{Mauwake} usually has an enclitic \textit{=i} at the first alternative and a \isi{disjunction} before the second. When three alternatives are present, the first two take the enclitic. This example also illustrates that the question markers in the individual alternatives do not have to attach at the same place. When the set of possible \isi{answer}s is expected to be open, the construction differs slightly and the second alternative also takes the \isi{question marker} \REF{ex:4:22}.

\ea%22
    \label{ex:4:22}
    \ili{Mauwake} (\ili{Trans-New Guinea})\\
    \gll matukar ikiw-i-nan=\textbf{i} \textbf{e} dylup ikiw-i-nan=\textbf{i}?\\
    \textsc{pn}    go-\textsc{n.pst}-\textsc{fut}.2\textsc{sg}=\textsc{q}  or  \textsc{pn}  go-\textsc{n.pst}-\textsc{fut}.2\textsc{sg}=\textsc{q}\\
    \glt ‘Will you go to Matukar or Dylup (or perhaps neither)?’ (\citealt{Berghäll2015}: 311)
    \z

Some languages do not allow ellipsis of identical parts (e.g., the \ili{Austronesian} language \ili{Rukai}, \citealt{Zeitoun2007}). All other languages allow some form of deletion. A very useful distinction that was introduced by \citet{Huang1999} for Austronesian languages on \isi{Taiwan} is that between forward (analipsis, \ref{ex:4:23}b) and backward deletion (catalipsis, \ref{ex:4:23}c) (see also \citealt{Haspelmath2007}: 39).

\ea%23
    \label{ex:4:23}
    \ili{Mandarin} (\ili{Trans-Himalayan})\\
    \ea
    \gll n\u{\i}  qù  zh\=ongguó  (háishì)  bú   qù  zh\=ongguó?\\
    2\textsc{sg}  go  \isi{China}    or.\textsc{q}    \textsc{neg}  go  \isi{China}\\

    \ex
    \gll n\u{\i}  qù  zh\=ongguó  (háishì)  bú   qù  {\longrule}  ?\\
    2\textsc{sg}  go  \isi{China}    or.\textsc{q}    \textsc{neg}  go  (\isi{China})\\

    \ex
    \gll n\u{\i}  qù  {\longrule}    (háishì)  bú   qù  zh\=ongguó?\\
    2\textsc{sg}  go  (\isi{China})  or.\textsc{q}    \textsc{neg}  go  \isi{China}\\
    \glt ‘Are you going to \isi{China} or are you not going to \isi{China}?’ (elicited, own knowledge, cf. \citealt{Hölzl2015b})
    \z
    \z 

\noindent In \isi{alternative question}s the part that is not focused on may fall victim to \isi{ellipsis}. In other words, (elliptical) alternative \isi{questions} are somewhat similar to \isi{focus question}s. This contrasts with the common assumption of alternative \isi{questions} being related to \isi{polar question}s, exclusively (e.g., \citealt{Siemund2001}).

Content question marking has not been investigated very often. Many languages have morphosyntactically unmarked \isi{content question}s, but these may exhibit special \isi{intonation} patterns that often are not clearly specified in the available descriptions. The remaining languages seem to make use of all the most common \isi{question marking} strategies discussed above for \isi{polar question}s and will thus be excluded here. Many examples can be found in \sectref{sec:5}.

The marking of \textit{\isi{focus question}s} is difficult to investigate because most grammatical descriptions simply do not mention it. Most likely, they can exhibit more or less the same range of marking strategies as polar \isi{questions}. Given their \isi{interaction} with the domain of \isi{focus}, they will be discussed further in \sectref{sec:4.2.3} on the \isi{interaction} of functional domains. Several examples can be found throughout \chapref{sec:5}.

\begin{table}
\caption{A typology of \isi{question tag}s according to \citet[803]{Axelsson2011}}
\label{tab:4:5}
\begin{tabularx}{\textwidth}{lQl}
\lsptoprule
invariant question tags & - neutral & no restrictions\\
& - polarity-biased &\\
& - polarity-dependent & \\
\tablevspace
\tablevspace
variant question tags & - lexically-dependent & ↕\\
& - marginal grammatically- dependent & \\
& - grammatically- dependent & most restrictions\\
\lspbottomrule
\end{tabularx}
\end{table}

\textit{Tag questions} have been excluded from the list of central question types in this study. Nevertheless, some information on their formal properties seem to be in order. Perhaps the best typology of \isi{tag question}s has been given by \citet[803]{Axelsson2011} (\figref{tab:4:5}). A main difference is drawn between invariant and variant tags. Invariant tags appear to be more common, both cross-linguistically and in \isi{NEA}. Each is furthermore divided into three different subtypes.

So-called neutral and polarity-biased \isi{question tag}s are neutral with respect to the polarity of the anchor, although the latter often prefers positive or negative anchors. Polarity-dependent question tags, as the name suggests, are restricted to either positive or negative anchors. Consider the following examples from \ili{English} \REF{ex:4:24}, where the first is a neutral (non-dependent) and the latter a grammatically-dependent \isi{question tag} (own knowledge).

\ea%24
    \label{ex:4:24}
    \ili{English}\\
    \ea
      \textit{You want coffee, \textbf{{right}}?}\\
    
    \ex
      \textit{You want coffee, \textbf{{don’t you}}?}\\
    \z
    \z

Marginal grammatically-dependent question tags, on the other hand, “are cases where the use of a certain \isi{question tag} is dependent on a certain grammatical feature in the anchor (other than polarity), but where there are no variable grammatical features in the tag itself.” \citep[805]{Axelsson2011} In lexically-dependent question tags, a lexical element of the anchor is also found in the tag \citep[805]{Axelsson2011}. There are relatively many languages in \isi{NEA} for which no \isi{tag question}s are attested. While at least in some cases this may be due to the lack of sufficient information, tag \isi{questions} most likely are not a \isi{universal} property of language.

Another useful dimension of question tags that is somewhat less relevant for other question markers concerns its \textit{etymological transparency}. \ili{German}, for example, has a variety of tags, among which we find a form \textit{ge(lle)} that is completely \isi{opaque} from a \isi{synchronic} perspective (\ref{ex:4:25}a). \ili{German} \textit{richtig}, on the other hand, is a common adjective related to \ili{English} \textit{right} (\ref{ex:4:25}b). Both are neutral question tags.

\newpage 
\ea%25
    \label{ex:4:25}
    \ili{German}\\
    \ea
      \textit{Du magst Kaffee, \textbf{ge(lle)}?}\\
    
    \ex
      \textit{Du magst Kaffee, \textbf{richtig}?}\\
    \z
    \z

\noindent The meaning and \isi{word order} are identical to the \ili{English} sentence \textit{You want coffee, right?} above (\ref{ex:4:24}a). In fact, most question markers are \isi{opaque} from a \isi{synchronic} perspective. Question tags, on the other hand, are frequently \isi{transparent}. Question markers furthermore tend to be extremely short (see \sectref{sec:6.1.1}). Question tags certainly can be short as well (e.g., \ili{English} \textit{eh?}), but generally tend to be longer and more complex than usual question markers (e.g., \ili{English} \textit{isn’t it?}, \ili{Mandarin} \textit{duì-bu-duì?}). These properties underline their separate status.

\citet[2167]{Mithun2012} roughly differentiates between epistemic (e.g., informational, confirmatory), and affective (e.g., facilitating, attitudinal, peremptory, aggressive) functions of tags. \citet{Axelsson2011} crucially investigated only confirmation seeking (perhaps better called epistemic) question tags, which reduces the problem of their classification considerably. The typology correctly excludes confirmation seeking constructions that are not formally tag \isi{questions} \citep[796]{Axelsson2011}. \ili{Hadiyya} (\ref{ex:4:26}), for example, has a confirmation seeking suffix \textit{-lla}, which combines with the polar \isi{question marker} \textit{-nni(yye)}.

\ea%26
    \label{ex:4:26}
    \ili{Hadiyya} (Cushitic, \ili{Afroasiatic})\\
    \gll kaa    ii    diinate    mass-i-t-aa-tto-\textbf{{lla}}{-yyo-}\textbf{{nni}}?\\
    2\textsc{sg.voc}  1\textsc{sg.gen}  money.\textsc{acc}  take-\textsc{e}-2\textsc{sg}-\textsc{prs.pfv}-2\textsc{sg}-\textsc{conf}-\textsc{neg}-\textsc{q}\\
    \glt ‘You have taken my money, haven’t you?’ \citep[27]{Sulamo2013}
    \z

\noindent Given the fact that the question is one single sentence, it is better classified as a special kind of \isi{polar question}. \sectref{sec:4.4} elaborates on the classification of \isi{tag question}s. Non-epistemic uses are likewise excluded from this study.

\subsection{The scope of question marking}\label{sec:4.2.2}

While different marking strategies for \isi{questions} are well-known, it is usually not recognized that these differ in their \isi{semantic scope} over different question types (but among others see \citealt{Dixon2012}: 389-390 and especially \citealt{Hölzl2015b,Hölzl2016a}). Given the lack of information for \isi{NEA}, this study will make use of a limited \isi{conceptual space} shown in \figref{fig:4:1} that only includes the most central question types. As can be seen, \isi{polar question}s take a central position while other types---especially \isi{content question}s---have a peripheral position. Solid lines indicate the possibilities that two categories may be marked with the same marker. The \isi{semantic scope} of a given marker may be shown as a closed line that encloses those categories that may be marked by it (i.e. its \isi{semantic scope}).

There is one possible implicational \isi{universal} that needs further testing in other parts of the world but seems reasonably robust for now.

\ea\upshape%27
    \label{ex:4:27}
      Content questions are only marked in the same way as \isi{focus} or \isi{alternative question}s if \isi{polar question}s are also marked in the same way.\\
    \z

\noindent The \isi{universal} is represented on the \isi{conceptual space} as the lack of connecting lines between categories (\figref{fig:4:1}). These presumably impossible connections are given as dashed lines. Note that this is an example for the so-called \textit{\isi{Semantic Map Connectivity Hypothesis}}: “any relevant language-specific and/or construction-specific category should map onto a \textit{connected region} in \isi{conceptual space}” \citep[134]{Croft2003}. A possible counterexample from \ili{Tshangla}, which allows the use of the content \isi{question marker} \textit{ya} in \isi{alternative question}s, can be found in \sectref{sec:4.2.1}.

\begin{figure}
% % % \includegraphics[width=\textwidth]{figures/fig_4_1.jpg}
\begin{tikzpicture}
  \matrix (HoelzlFig41) [matrix of nodes,nodes in empty cells,column sep=1cm, row sep=.75cm]
    {
      & AQ & \\
      & PQ & \\
      FQ & & CQ\\
    };
  \draw[thick,dotted] (HoelzlFig41-1-2) -- (HoelzlFig41-3-1) -- (HoelzlFig41-3-3) -- (HoelzlFig41-1-2);
  \draw[thick,solid] (HoelzlFig41-1-2) -- (HoelzlFig41-2-2);
  \draw[thick,solid] (HoelzlFig41-3-1) -- (HoelzlFig41-2-2);
  \draw[thick,solid] (HoelzlFig41-3-3) -- (HoelzlFig41-2-2);
\end{tikzpicture}
\caption{Limited conceptual space of question marking}
\label{fig:4:1}
\end{figure}

Another possible implicational \isi{universal} concerns the dashed line between \isi{focus} and alternative \isi{questions} (\citealt{Hölzl2015b}).

\ea\upshape%28
    \label{ex:4:28}
      Focus and alternative \isi{questions} can only be marked in the same way if polar \isi{questions} are also marked in the same way.\\
    \z

\noindent Only one possible exception (the \ili{Nyulnyulan} language \ili{Bardi} spoken in Australia, see \ref{ex:4:3} above) was found within the global 50 language sample investigated by \cite{Hölzl2015b}. An obstacle for confirming or disproving the \isi{universal} is severely hampered by the lack of adequate data for the majority of languages. The dashed lines are also meant to indicate that such connections might be possible after all but clearly are dispreferred.

If the \isi{conceptual space} is universally applicable, which should be the long-term goal, then it poses several powerful constraints on how markers can expand their scope. An extension of the \isi{semantic scope} of a given marker, for example, is only possible if there is a connection in \isi{conceptual space}. Every language shows a distinctive \isi{semantic map}, but languages may have similar patterns due to universals, tendencies, chance, \isi{language contact} or common inheritance. Given that question markers are often and freely borrowed from one language to the next, semantic maps easily change their shape.

Content questions, which often remain unmarked morphosyntactically, are a special case. By comparing polar and content \isi{questions} and further differentiating between morphosyntactically marked versus unmarked content \isi{questions}, one gets a matrix of four language types (\tabref{tab:4:6}).

\begin{table}
\caption{Polar and content question marking strategies among 50 languages, based on \cite{Hölzl2015b}; the classification of three languages remained unclear}
\label{tab:4:6}

\begin{tabularx}{\textwidth}{XXl}
\lsptoprule
& CQ marked & CQ unmarked\\
\midrule
PQ=CQ & Type 1: 9 & Type 3: 1?\\
PQ${\neq}$CQ & Type 2: 9 & Type 4: 28\\
\lspbottomrule
\end{tabularx}
\end{table}

Type 4 appears to be the most and Type 3 the least common, cross-linguistically.\footnote{Given the lack of information \cite{Hölzl2015b} omitted \isi{intonation}, which should be included in future studies.} In sum, there is a deep bifurcation between \isi{content question}s on the one hand and polar \isi{questions} on the other. However, as we will see more clearly in \chapref{sec:6}, \isi{polar question}s have closer relations to the other question types.

\subsection{Interaction of functional domains}\label{sec:4.2.3}

The term \textit{\isi{functional domain}} here covers broad \isi{universal} categories such as \isi{negation}, \isi{focus}, or \isi{question marking}, which themselves have many subcategories. \cite[24]{Hölzl2016a} distinguished between four different types of \isi{interaction} between such functional domains shown in \REF{ex:4:29}.

\ea\upshape%29
    \label{ex:4:29}
    \ea
      \isi{grammaticalization} (1)\\

    \ex
      \isi{combination} (2)\\

    \ex
      \isi{fusion} (3)\\

    \ex
      \isi{interaction} (split types) (4)\\

    \z
    \z 

\noindent For practical purposes, the \isi{combination} of \isi{disjunction} with \isi{question marking} was already covered above in \sectref{sec:4.2.1}.

         (1) Grammaticalization in this context is understood as a cover term for the \textit{shift in meaning} of a linguistic element from one \isi{functional domain} to another. Many details, of course, are language- and construction-specific, but here only a cursory overview similar to the \textit{World} \textit{Lexicon of  Grammaticalization} (\citealt{Heine2002}) can be given (cf. \citealt{Hölzl2015b}). Consider the following \isi{polar question} from a language in Nepal \REF{ex:4:30}.

\ea%30
    \label{ex:4:30}
    \ili{Bantawa} (Kiranti, \ili{Trans-Himalayan})\\
    \gll {am-k\textsuperscript{h}}{e} ham-si {tɨ-k\textsuperscript{h}}{ar-a-}\textbf{{ʔo}}?\\
    2\textsc{sg.gen}-lice  swap-\textsc{sup}  2\textsc{as}-\textsc{go}-\textsc{pst}-\textsc{q}\\
    \glt ‘Did you go to swap lice?’ (i.e. ‘Did you go to have sex?’) \citep[205]{Doornenbal2009}
    \z

\noindent The marker \textit{-ʔo} has been glossed as a \isi{question marker}, but it is really a nominalizer, which is presumably the reason why the example has an additional semantic component ‘is it the fact that’. A similar development has also been described for Tucanoan languages in South America, which

\begin{quote}
exhibit a historical and semantic relationship between \isi{nominalization}s and \isi{questions}. We have also tried to demonstrate that formally the latter originate from the former through a process of upgrading a nominalized predication to the status of an independent utterance from an inferential or mirative construction. Semantically, the \isi{interrogative} meaning must have become conventionalized via stages expressing doubt or surprise. (\citealt{vanderAuweraIdiatov2008}: 46)
\end{quote}

\noindent Whether exactly the same developmental path was followed in \ili{Bantawa} or other languages with this phenomenon is not known to me.

Two other well-known examples are the development of disjunctions and negators to \isi{polar question} markers. However, both of these developments usually start within the context of an elliptic \isi{alternative question}. In some languages such as \ili{Edo} \REF{ex:4:31}, the second alternative is fully elliptic and the \isi{disjunction} can take over the function of a \isi{polar question} because no second alternative is specified (cf. \citealt{Dixon2012}: 399-400).

\ea%31
    \label{ex:4:31}
    \ili{Edo} (\ili{Niger-Congo})\\
    \ea
    \gll Òsà{r}ọ́   bọ́   òwá \textbf{{rà}} Òsà{r}ọ́  rhìé  òkhùò?\\
    \textsc{pn}  build  house  or  \textsc{pn}  take  woman\\
    \glt ‘Did Osaro build a house or did he marry a woman?’

    \ex
    \gll Òsà{r}ọ́   bọ́   òwá \textbf{{rà}}?\\
    \textsc{pn}  build  house  \textsc{q}\\
    \glt ‘Did Osaro build a house?’ (\citealt{Ọmọruyi1988}: 22, 23)
    \z
    \z

Similarly, negators can develop into \isi{polar question} markers in \isi{negative alternative question}s when the second alternative only consists of the negator. Examples of this sort can be found in \ili{Mandarin} (\sectref{sec:5.9.2.1}), for instance. A related development seems to start from negative alternative \isi{questions} as well, but in this case the first alternative appears to have been deleted. In \ili{Kham} (\ili{Trans-Himalayan}), for example, the prefix \textit{ma-} can express both \isi{negation} and polar \isi{questions} \citep[96-101]{Watters2002}. Negators such as \ili{German} \textit{nich(t)} ‘not’ can also develop into question tags.

Yet another frequent development is from \isi{interrogative}s to \isi{polar question} markers and \isi{question tag}s. This development is very rare in \isi{NEA} but many examples can be found in Indo-European languages (\sectref{sec:5.5.2}, \citealt{Hackstein2013}: 100). Example \REF{ex:4:10} from \ili{Bengali} above, for example, contains the polar \isi{question marker} \textit{ki}, which is most likely derived from the \isi{interrogative} \textit{ki} ‘what’ (\citealt{Thompson2012}: 200-203), see also \REF{ex:4:17}. In the language \ili{Palula} the \isi{interrogative} \textit{ga} ‘what’ developed into a \isi{question tag} \REF{ex:4:32}.

\ea%32
    \label{ex:4:32}
    \ili{Palula} (\ili{Dardic}, \ili{Indo-European})\\
    \gll so gúum \textbf{{ga}}?\\
    3\textsc{sg.nom}  go.\textsc{pfv}.\textsc{sg.}\textsc{m}  what\\
    \glt ‘He left, didn’t he?’ \citep[404]{Liljegren2016}
    \z

\noindent This development can also be found in other languages of South Asia. For instance, the \ili{Dravidian} language \ili{Kurux} employs the interrogative \textit{ender} ‘what‘ as a question marker in sentence-initial position \citep[241-242]{Kobayashi2017}. Another example mentioned above stems from \ili{Bardi}. \sectref{sec:6.1.3} summarizes the most important \isi{grammaticalization} paths found during this study (see also \citealt{Bencini2003}).

        (2) Question marking is frequently \textit{combined} with interrogatives in content \isi{questions} and disjunctions in \isi{alternative question}s. Interrogatives ({\textasciitilde} indefinites) are almost \isi{universal}, but there are many languages without disjunctions, for example in northern \isi{NEA}. Another special case concerns \isi{focus} markers that are frequently present in \isi{focus} and sometimes other question types (\figref{fig:4:2}). In \ili{English}, for example, \isi{focus question}s are expressed by usual \isi{polar question} marking and additional intonational \isi{focus} or a cleft construction \REF{ex:4:33}.

\ea%33
    \label{ex:4:33}
    \ili{English}\\
    \ea
      \textit{Did \uline{you} go there?}\\
    
    \ex
      \textit{Is it \uline{you} who went there?}\\
    \z
    \z

\noindent In both cases \isi{focus} and \isi{question marking} are merely combined with each other. For practical purposes disjunctions and \isi{focus} marking will be treated together with \isi{question marking} in this study, but one should keep in mind that they really belong to different functional domains that merely overlap with each other.

\begin{figure}
% % % \includegraphics[width=\textwidth]{figures/fig_4_2.jpg}
\begin{tikzpicture}
  \matrix (HoelzlFig41) [matrix of nodes,nodes in empty cells,column sep=1cm, row sep=.75cm]
    {
      & AQ & \\
      & PQ & \\
      FQ & & CQ\\
    };
  \draw[thick,dotted] (HoelzlFig41-1-2) -- (HoelzlFig41-3-1) -- (HoelzlFig41-3-3) -- (HoelzlFig41-1-2);
  \draw[thick,solid] (HoelzlFig41-1-2) -- (HoelzlFig41-2-2);
  \draw[thick,solid] (HoelzlFig41-3-1) -- (HoelzlFig41-2-2);
  \draw[thick,solid] (HoelzlFig41-3-3) -- (HoelzlFig41-2-2);
  \node[above=\baselineskip of HoelzlFig41-1-2.base,anchor=base,color=gray] {disjunctions, focus};  
  \node[below=\baselineskip of HoelzlFig41-3-1.base,anchor=base,color=gray] {focus};
  \node[below=\baselineskip of HoelzlFig41-3-3.base,anchor=base,color=gray,align=right] {interrogatives, focus};
\end{tikzpicture}
\caption{Typical interaction of question marking with other functional domains}
\label{fig:4:2}
\end{figure}

Previous studies of \isi{question marking} have presumably focused on polar \isi{questions}, because these exhibit the least interference with other functional domains.

         (3) In instances of \textit{fusion}, on the other hand, a \isi{question marker} also has additional functions such as \isi{focus} marking. When a \isi{question marker} also functions as a \isi{focus} marker, it usually attaches to the verb in \isi{polar question}s and to the focal element in \isi{focus question}s. Such an example can be found in the South American language \ili{Quechua} as spoken in Cusco \REF{ex:4:34}.

\ea%34
    \label{ex:4:34}
    Cusco \ili{Quechua} (Quechuan)\\
    \ea
    \gll wasi-y-maŋ  hamu-ŋki=\textbf{{chu}}?\\
    house-1-\textsc{dat}  come-2=\textsc{q.foc}\\
    \glt ‘Do you come to my house?’

    \ex
    \gll wasi-y-maŋ=\textbf{{chu}} hamu-ŋki?\\
    house-1-\textsc{dat}=\textsc{q.foc}  come-2\\
    \glt ‘Do you come to \textit{my} house?’ \citep[29]{Ebina2011}
    \z
    \z

\noindent See \sectref{sec:6.1.3} for a list of examples from \isi{NEA}.

         (4) The most complex \isi{question marking} systems are \textit{split systems}. In such languages the choice between different question markers depends on other domains such as person, number, tense, aspect, mood, \isi{evidentiality}, clause type etc. \sectref{sec:6.1.3} lists all instances found in \isi{NEA}. A relatively simple example can be found in the language \ili{Qiang} \REF{ex:4:35}, which has a split based on person.

\ea%35
    \label{ex:4:35}
    \ili{Qiang} (\ili{Qiangic}, \ili{Trans-Himalayan})\\
    \ea
    \gll ʔũ ʐme ŋuə-n-\textbf{a}?\\
    2\textsc{sg}  \textsc{pn}  \textsc{cop}-2\textsc{sg}-\textsc{q}\\
    \glt ‘Are you a \ili{Qiang}?’

    \ex
    \gll the:  ʐme  ŋuə-Ø-\textbf{{ŋua}}?\\
    \textsc{3sg}  \textsc{pn}  \textsc{cop-(3sg)}-\textsc{q}\\
    \glt ‘Is (s)he a \ili{Qiang}?’ (\citealt{LaPollaHuang2003}: 180)
    \z
    \z

\noindent Only second person \isi{singular} forms take the marker \textit{-a} instead of \textit{-ŋua}. Many examples of \isi{split type}s exhibit instances of fusion, but this is not necessarily so, as this example illustrates. An example for a split in \isi{combination} with fusion stems from the Amazonian language \ili{Kulina} \REF{ex:4:36}, which combines \isi{question marking} with \isi{gender}.

\ea%36
    \label{ex:4:36}
    \ili{Kulina} (\ili{Arawan})\\
    \ea
    \gll osonaa=\textbf{{ko}}?\\
    \textsc{pn=q.m}\\
    \glt ‘Is he a Kashinawa?’

    \ex
    \gll osonaa=\textbf{{ki}}?\\
    \textsc{pn=q.f}\\
    \glt ‘Is she a Kashinawa?’ \citep[193]{Dienst2014}
    \z
    \z

The markers appear in both polar and \isi{focus question}s. \ili{Omotic} languages (\ili{Afroasiatic}) exhibit some of the most complex split systems (see \citealt{Amha2007}; 2012; \citealt{Hellenthal2010}: 401ff.; \citealt{Köhler2013}; \citeyear{Köhler2016}; \citealt{Treis2014}; \citealt{Hölzl2016a}: 26 and references therein). Again, see \sectref{sec:6.1.3} for those split types encountered in \isi{NEA}.

\subsection{The number of markers}\label{sec:4.2.4}

A dimension not mentioned in \cite{Hölzl2016a} is the sheer amount of question markers present in a given language. Arguably, this is yet another dimension of the \isi{complexity} of the \isi{grammar of questions}. There is a certain connection with both the scope of \isi{question marking} and the \isi{interaction} with other functional domains. A smaller scope of question markers is usually, but not necessarily, correlated with a higher number of markers. The \isi{question marker} \textit{=Ku} in the \ili{Tungusic} language \ili{Evenki}, for example, has a broad scope that covers polar, \isi{focus}, and alternative \isi{questions}, and, indeed, \ili{Evenki} has only a rather small amount of other question markers that also depend on the dialect, however. If \isi{question marking} interacts with certain other domains such as person marking, there tends to be a higher number of markers. The average number and possible variation among the languages of the world is not entirely certain but presumably most languages have at least one or a few question markers. It is, furthermore, not self-explanatory how question markers should be counted at all. For instance, the \ili{Tungusic} language \ili{Manchu} has a \isi{question marker} \textit{=ni} that fuses with certain words such as the negative existential \textit{ak\=u} to yield a complex form \textit{ak\=un}. Should \textit{=ni} and \textit{-n} be counted as one or two markers? Despite such problems, it is usually unproblematic to establish whether a certain language exhibits a larger or smaller amount of markers relative to other languages. The Nicobarese (\ili{Austroasiatic}) language \ili{Muöt}, for example, according to \citet[114]{Rajasingh2014}, only has one \isi{question marker}, namely final rising \isi{intonation}. The Cushitic (\ili{Afro-Asiatic}) language \ili{Hadiyya}, to give a slightly more complex example, has three main question markers, rising \isi{polar question} \isi{intonation}, the polar and alternative \isi{question marker} \textit{-nni(yye)}, and the confirmation seeking suffix \textit{-lla} that is usually combined with \textit{-nni(yye)} \citep{Sulamo2013}. The majority of languages in \isi{NEA} and worldwide seem to cluster somewhere around this relatively small amount of question markers, but there are some languages with extremely complex \isi{question marking} systems (e.g., \sectref{sec:5.4.2} on \ili{Yupik}, \sectref{sec:5.9.2.1} on \ili{Sinitic}, and \sectref{sec:5.14.2} on \ili{Yukaghiric}). Perhaps the upper end is formed by \ili{Omotic} (\ili{Afroasiatic}) languages, which sometimes exhibit a plethora of several dozen different forms organized in many different paradigms (e.g., \citealt{Amha2007,Amha2012}; \citealt{Hellenthal2010}: 401ff.; \citealt{Köhler2013,Köhler2016}; \citealt{Treis2014}; \citealt{Hölzl2016a}: 26, and references therein).

\section{Interrogatives}\label{sec:4.3}

What will simply be called \textit{interrogatives} here has variously been termed \textit{wh-words}, \textit{\isi{interrogative} pronouns}, \textit{\isi{interrogative} words}, \textit{question words} etc. But these terms are problematic from several perspectives. First, \textit{wh} refers to an \ili{English} language writing convention exclusively (variously pronounced /h/ {\textasciitilde} /w/), even fails to capture \ili{English} forms such as \textit{how}, and has no validity whatsoever from a typological perspective. Interrogatives are, furthermore, not necessarily pronominal. Instead, they represent what has been called “a meta word-class, spanning a number of major classes” \citep[80]{Dixon2002} or “a pan-basic-word-classes \isi{word class}” \citep[409]{Dixon2012}. As we will see during this section, there are \isi{interrogative} nouns, verbs, adjectives, adverbs etc. The terms \textit{question word} or \textit{\isi{interrogative} word} are, therefore, more adequate than the other terms but still problematic. While \textit{pronoun} suggests a connection with grammar, \textit{word} clearly indicates a lexical category. It has been shown by \citet{Diessel2003}, however, that interrogatives (and perhaps \isi{demonstratives}), do not clearly belong to either of these categories. \citet[636]{Diessel2003} is certainly right in his view (also accepted by \citealt{CysouwHackstein2011}) that

\begin{modquote}
while grammatical markers are commonly derived from lexical expressions, \isi{demonstratives} and interrogatives cannot be traced back to lexical items. While both are often reinforced by other lexemes, there is no evidence from any language that a new demonstrative or \isi{interrogative} developed from a lexical source (unless the lexical source first functioned to reinforce a genuine demonstrative or \isi{interrogative}). All this suggests that \isi{demonstratives} and interrogatives have a special status in language and should be kept separate from genuine grammatical markers.
\end{modquote}

\noindent Like lexical items, both \isi{demonstratives} and interrogatives are often the source for several grammatical items. In a brief discussion on Funknet, Heine \& Kuteva (p.c. 2018) made me aware of the fact that there are indeed several examples of \isi{demonstratives} with lexical origins. Nevertheless, interrogatives and perhaps \isi{demonstratives} might still form a class by themselves that is neither, strictly speaking, lexical, nor grammatical. “Grammatical markers organize the information flow in the ongoing discourse, whereas basic \isi{demonstratives} and interrogatives are immediately concerned with the speaker-hearer \isi{interaction}.” \citep[635]{Diessel2003} Interestingly, the two often share paradigmatic similarities (see below) and the only known language without interrogatives, the \ili{Chapacuran} language \ili{Wari’}, uses \isi{demonstratives} instead (\citealt{EverettKern1997}).

There are several imaginable typologies for interrogatives, but many of them do not make too much sense from a cross-linguistic perspective. For example, one might count the \textit{number of forms} that may be encountered in one language. The number of interrogatives among languages is highly variable. There are none in \ili{Wari’} (\citealt{EverettKern1997}) but up to about 30 in \ili{German} according to my count, including derived forms. However, apart from the practical problem that almost no grammatical description mentions more than a handful of forms, it is by no means clear how such forms should actually be counted. \citet[1133]{Mackenzie2009}, for instance, counts “only the simple forms as true \isi{interrogative} forms“. Similarly, \citet[46]{Hengeveld2012} only include “\isi{basic question words}”. The necessary condition for these claims is a clear-cut boundary between forms that can be analyzed and those that cannot. However, the existence of such a boundary is far from clear because \textit{analyzability} is clearly “a \isi{matter of degree}” \citep[352]{Langacker2008}. Let me illustrate this with the help of interrogatives in the \ili{Tungusic} language \ili{Manchu} (\sectref{sec:5.10.3}). There certainly are some non-analyzable “basic” interrogatives such as \textit{we} ‘who’ that even historically are not \isi{transparent}. Then there are forms such as \textit{atanggi} ‘when’, which is not analyzable synchronically but shares a \isi{resonance} (a \isi{submorpheme}) \textit{a{\textasciitilde}} with several other forms. In all likelihood it is ultimately based on the \isi{interrogative} \textit{ai} ‘what’, but the derivation remains unclear, since a word meaning perhaps ‘time’ with this form is not attested. \citet{Mackenzie2009} and \citet{Hengeveld2012} would perhaps include both of these forms into the category of “\isi{basic question words}”, but this is an arbitrary decision. \ili{Manchu} furthermore has a form \textit{aiseme} ‘why’ that clearly is a \isi{combination} of \textit{ai} ‘what’ and the quotative \textit{seme}, which in turn may be analyzed as \textit{se-me} ‘say-\textsc{cvb.ipfv}’. Despite its formal \isi{analyzability}, the semantic side is not fully compositional. Further problems for an \isi{analysis} are so-called cranberry morphs such as the second element in \ili{Manchu} \textit{ai-bi-de} ‘where’, which stands opposed to the fully-analyzable form \textit{ai-ba-de} ‘what-place-\textsc{loc}’. \ili{Manchu} simply has no independent form \textit{bi} that would explain the second element in \textit{ai-bi-de}. It is not the first person \isi{singular} nominative \textit{bi}, nor the copula \textit{bi} which cannot take any \isi{case} markers. The most likely scenario is an idiosyncratic development from \textit{ba} ‘place’. In any case, the point is that there are partly analyzable forms that constitute a scale between non-analyzable and fully-analyzable forms (see also \citealt{Cysouw2005}). This background also means that reconstructions of clear \isi{interrogative} “stems” for any given proto-language in most instances must in principle be considered problematic. While the \isi{analyzability} of interrogatives tends to decrease over the course of time if no new forms are built, there may also be a development in the opposite direction, as witnessed in the reanalysis of \ili{German} \textit{wor.um} ‘around where, about what’ as \textit{wo.}\textit{rum}, which allows a reconnection to the word \textit{wo} ‘where’ that historically lost the final \textit{-r} (PIE *\textit{k\textsuperscript{w}}\textit{ór}) and the creation of a new form \textit{rum}.

There are several more possible dimensions for a typology of interrogatives. Some investigations (\citealt{Heine1991}; \citealt{Peyraube2005}; \citealt{Mackenzie2009}; \citealt{Hengeveld2012}) have combined several of these dimensions (e.g., \isi{analyzability}, polysemy, length) into one typology. However, the results that take the form of a \isi{hierarchy} are simply not valid from a cross-linguistic perspective (\citealt{Hölzl2015c}). A study mostly neglected in later typologies (but see \citealt{Peyraube2005}) has been conducted by \citet{Heine1991}, who investigated what they called “metaphorical relations” and how they related to interrogatives in 14 different languages. Their result is a \isi{hierarchy} that has the following form (\ref{ex:4:37}, slightly adapted):

\ea\upshape%37
    \label{ex:4:37}
    \textsc{person < thing < activity < place < time < manner < purpose/cause}
    \z

\noindent According to their study, the first four categories on the \isi{hierarchy} showed minimal phonological and morphological \isi{complexity} and were often monosyllabic. \textsc{time} and \textsc{quality} were slightly more complex. \textsc{purpose} and \textsc{cause} were found to be much more complex and often had the form “what-\isi{case}”. Furthermore, in the languages investigated, \textsc{thing} and \textsc{activity} were claimed not to be differentiated (e.g., \ili{English} \textit{what}, \textit{(to do) what}). They had several interesting conclusions such as the following:

\begin{quote}
While it remains unclear what the exact correlations between the linguistic and the cognitive structure of pronouns are, a few assumptions may be tentatively formulated. First, the relative degree of morphological \isi{complexity} that a pronoun exhibits is likely to correlate to some extent with the relative degree of its \textbf{cognitive} \textbf{complexity}. [...] Second, formal \isi{similarity} between different pronominal categories may be indicative of some kind of conceptual relation between these categories. (\citealt{Heine1991}: 59, my boldface)
\end{quote}

Let us now address an interesting typology by \citet{Mackenzie2009}, who, strangely, did not mention the study by \citet{Heine1991}. He investigated interrogatives in a sample of 50 languages. More specifically, he concentrated on so-called “cognitive \isi{complexity}”, which may be accessed through an investigation of system \isi{complexity} (extent of polysemy), item \isi{complexity} (extent of \isi{analyzability}), and signal \isi{complexity} (number of phonemes, length), all of which were also included by \citet{Heine1991}. The result of his study also takes the form of a \isi{hierarchy} which has the following form (\ref{ex:4:38}, put into a comparable format):

\ea\upshape%38
    \label{ex:4:38}
    \textsc{person/thing < place < time < manner < quantity < cause}
    \z

\noindent The major difference is that \citet{Heine1991} included \textsc{activity} instead of \textsc{quantity}. In fact, \citet[1150]{Mackenzie2009} himself noted that “none of the central hypotheses has been fully vindicated”. In my eyes, the main problem is the \isi{combination} of different typological dimensions that are not directly connected (such as polysemy and length) and thus simply lead to inconclusive results. \citet{Mackenzie2009} furthermore made some minor but unimportant mistakes such as counting letters instead of phonemes for \ili{Mandarin} and including expressions about the time of day into the category of \textsc{time}.

A follow-up study of \citet{Mackenzie2009} was conducted by \citet{Hengeveld2012}, who proposed a \isi{hierarchy} based on “\isi{basic question words}” (i.e., non-analyzable interrogatives).

\ea\upshape%39
    \label{ex:4:39}
    \textsc{person/thing < place < manner < quantity/time/reason}
    \z

\noindent \citet{Mackenzie2009}, who also investigated this problematic category, found the following slightly deviating \isi{hierarchy}:

\ea\upshape%40
    \label{ex:4:40}
    \textsc{person/thing < place < quantity < manner < time < reason}
    \z

However, the idea of a cross-linguistically valid \isi{hierarchy} of “\isi{basic question words}” has to be refuted, too \citep{Hölzl2015c}. For example, \ili{Tungusic} data result in the \isi{hierarchy} shown in \REF{ex:4:41} (see \sectref{sec:5.10.3}):

\ea\upshape%41
    \label{ex:4:41}
    \textsc{person/manner/quantity < time < thing/reason <} \textbf{\textsc{place}}
    \z

\noindent As can be seen, there are severe problems such as the completely different location of \textsc{place} on the \isi{hierarchy}. In other words, such a \isi{hierarchy} simply does not make sense from a cross-linguistic perspective. There is no reason to assume that a one-dimensional construct is capable of capturing the much more complex phenomenon of interrogatives. There might be some exceptions such as the frequency of certain interrogatives across languages that could converge to a certain degree, but this has not been investigated and turned out to be impossible to investigate for \isi{NEA} due to lack of sufficient data for almost all languages. It is also possible to investigate the mere length of interrogatives (e.g., \ili{German} \textit{wer} ‘who’ is shorter than \textit{warum} ‘why’), but there does not appear to be a \isi{universal} \isi{hierarchy} either (\citealt{Hölzl2015c}). At least there may be a \isi{tendency} for some categories (e.g., ‘who’, ‘what’) to have shorter forms than others \citep[1139]{Mackenzie2009}, but this is not exclusively connected with the overall frequency in texts. For instance, the shortest \isi{interrogative} in the \ili{Tungusic} language \ili{Nanai} is \textit{ui} ‘who’, which is much less frequent than \textit{xooni} ‘how’ \citep[320]{Kazama2007}. Furthermore, there may be some convergence in the order in which interrogatives are learned by children during language \isi{acquisition}. Previous research indicates a \isi{hierarchy} of the following sort (\citealt{Tomasello2003}: 159 and references therein, \ref{ex:4:42}, slightly adapted).

\ea\upshape%42
    \label{ex:4:42}
    \textsc{thing/place} < \textsc{person} < \textsc{manner/reason} < \textsc{time}
    \z

\noindent However, the hierarchy is based on only a handful of languages and there is insufficient data for most languages in \isi{NEA}.

The following will address the (1) \isi{semantic scope} (\sectref{sec:4.3.1}), (2) \isi{word class} membership (\sectref{sec:4.3.2}), (3) diachrony (\sectref{sec:4.3.3}), (4) inflectional properties (\sectref{sec:4.3.4}), and (5) the connection of interrogatives to \isi{demonstratives} (\sectref{sec:4.3.5}), which, for the purposes of this study, seem to be the most important dimensions for a typology.

\subsection{Semantic scope of interrogatives}\label{sec:4.3.1}

For the illustration of differences in the \textit{semantic scope} of interrogatives consider example \ref{ex:4:43} from the language \ili{Kusunda}, a language without clear affiliation spoken in Nepal, and their \ili{English} translations.

\ea%43
    \label{ex:4:43}
    \ili{Kusunda}\\
    \ea
    \gll \textbf{{nəti}} na?\\
    \textsc{int}  this.\textsc{an}\\
    \glt ‘\textit{Who} is this?’

    \ex
    \gll \textbf{{nəti}} ta?\\
    \textsc{int}  this.\textsc{inan}\\
    \glt ‘\textit{What} is this?’ \citep[48]{Watters2006}
    \z
    \z

\noindent The two categories of \textsc{person} and \textsc{thing} are expressed with two different interrogatives (\textit{who} and \textit{what}) in \ili{English} but with one (\textit{nəti}) in \ili{Kusunda}. Thus, there is a difference in \isi{semantic scope} of the interrogatives over different semantic categories. Usually, a narrow \isi{semantic scope} goes along with a larger number of interrogatives and \textit{vice versa}. In these examples, animateness in \ili{Kusunda} is expressed by the \isi{demonstratives} instead. As \cite{Cysouw2005,Cysouw2007} has shown, this particular polysemy (\textsc{person}=\textsc{thing}) is rare worldwide but relatively common in South America. In Eurasia it can also be found in \ili{Baltic} and \ili{Tocharian B}.

The determination of the \isi{semantic scope} of a given \isi{interrogative} presupposes a fixed set of \textit{semantic categories}. However, there is a certain dispute as to how many different categories should be postulated. The comparison in \tabref{tab:4:7} is not exhaustive, but sufficient for our purposes (see also \citealt{Mushin1995} etc.).

\begin{table}
\caption{A selection of different categorizations of interrogatives (\citealt{Hölzl2015c})}
\label{tab:4:7}
\small
\begin{tabularx}{\textwidth}{QQQQQQ}
\lsptoprule
\textbf{\citealt{MuyskenSmith1990}} & \textbf{\citealt{Heine1991}} & \textbf{\citealt{Diessel2003}} & \textbf{\citealt{Cysouw2005}} & \textbf{\citealt{Mackenzie2009}} & \textbf{\citealt{Dixon2012}}\\
\midrule
who & \textsc{person} & person & \textsc{person} & ?individual [+/-hum] & who\\
what & \textsc{object} & thing & \textsc{thing} &  & what\\
- & \textsc{activity} & - & \textsc{-} & - & -\\
which & - & - & \textsc{selection} & - & which\\
- & - & amount & \textsc{quantity} & quantity & how many/ how much\\
why & \textsc{cause} & - & \textsc{reason} & reason & why\\
- & \textsc{purpose} & - & \textsc{-} & - & -\\
how & \textsc{quality} & manner & \textsc{manner} & manner & how\\
where & \textsc{space} & place & \textsc{place} & location & where\\
- & - & direction:to & \textsc{-} & - & -\\
- & - & direction:from & \textsc{-} & - & -\\
when & \textsc{time} & time & \textsc{time} & ?time & when\\
\lspbottomrule
\end{tabularx}
\end{table}

There is no agreement in terminology or number of different categories. This study follows \citegen{Cysouw2005} approach but adds additional categories. Strangely, only \citet{Heine1991} include the categories of \textsc{activity} and \textsc{purpose}, of which at least the first is rather crucial from a cross-linguistic perspective, and only \citet{Diessel2003} mentions spatial interrogatives with an allative or ablative meaning. There are, furthermore, many more categories that are not included in the list, but the most prominent ones are certainly represented. One category that should perhaps be added is \textsc{kind}, which might have been overlooked because \ili{English} \textit{what kind of} and similar forms in other European languages is fully analyzable and thus appears non-basic. Nevertheless, this category has to be distinguished from the category of \textsc{selection}, e.g. \ili{English} \textit{which (one)}, which does not classify but individualizes a given referent. Thus, this study tentatively distinguishes the categories of \textsc{person}, \textsc{thing}, \textsc{selection}, \textsc{activity}, \textsc{cause}, \textsc{manner}, \textsc{quantity}, \textsc{place}, \textsc{time}, and \textsc{kind}. Some of these have secondary subcategories such as \textsc{count} (\textit{how many}) or \textsc{mass} (\textit{how much}) in \textsc{quantity} and \textsc{location} (\textit{where}), \textsc{direction} (\textit{whither}), and \textsc{source} (\textit{whence}) in \textsc{place}. The category \textsc{purpose} will not be distinguished from \textsc{cause} as it does not appear to play a crucial role for languages in \isi{NEA}. The same is true for the difference between \textsc{manner} and \textsc{quality}. There are some additional categories, but including them is not absolutely necessary because only a handful of forms is attested for most languages in \isi{NEA}. There are a number of subcategories that will not be addressed any further. Pite \ili{Saami} (\ili{Uralic}), for example, apart from the selective \isi{interrogative} \textit{mikir-} ‘which’ has a special \isi{interrogative} \textit{gåb-} ‘which one (out of two) (\textsc{sg}), which two (\textsc{pl})’ \citep[123]{Wilbur2014}.

The question ‘\isi{What is your name?}’ (see \citealt{Idiatov2007}) often allows the use of two different interrogatives, ‘who’ and ‘what’. In some languages (e.g., \ref{ex:4:44}) both interrogatives may be used.

\ea%44
    \label{ex:4:44}
    \ili{Abui} (\ili{Timor-Alor-Pantar})\\
    \ea
    \gll a-ne \textbf{{nala}}?\\
    2\textsc{sg.}\textsc{inal}-name  what\\
    \glt ‘What is your name?’

    \ex
    \gll a-ne \textbf{{maa}}?\\
    2\textsc{sg.}\textsc{inal}-name  who\\
    \glt ‘\isi{What is your name?}’ (\citealt{Kratochvíl2007}: 129)
    \z
    \z

\noindent Thus, there is no absolutely clear-cut or at least a language specific boundary between the categories of \textsc{person} and \textsc{thing}. Similar problems exist for other categories such as \textsc{mass} versus \textsc{count}.

Interrogatives have what can be called \textit{schematic} (e.g., \citealt{Langacker2008}) meaning and they express \isi{basic semantic categories} (e.g., \citealt{Schulze2007}). Direct evidence for the basic meaning of interrogatives can be found in many languages that have \textit{transparent} interrogatives (\citealt{MuyskenSmith1990}) such as \ili{English} \textit{what kind of} or \textit{what for}. A list of frequent elements that are combined with interrogatives can be found in \tabref{tab:4:8}. For example, the \ili{Trans-Himalayan} language \ili{Anong} has a rather general \isi{interrogative} \textit{k\textsuperscript{h}}\textit{a}\textsuperscript{55} {\textasciitilde} \textit{k\textsuperscript{h}}\textit{a}\textsuperscript{31} that, if combined with a personal classifier, forms the \isi{interrogative} \textit{k\textsuperscript{h}}\textit{a}\textsuperscript{31}-\textit{io}\textsuperscript{55} ‘who’ (\citealt{SunHongkai2009}: 73-74). In \ili{Sheko} (Omotic, Afroasiatic) the \isi{interrogative} \textit{yírà} ‘what’ can take a “motive” marker; the resulting form \textit{yír-èʃǹtà} has acquired the meaning ‘why’ (\citealt{Hellenthal2010}: 411-412). Useful but much less common alternatives for the designation of interrogatives are \textit{epistememes} \citep{Mushin1995} or \textit{ignoratives} \citep[443-461]{Miyaoka2012}, which both emphasize their relation to knowledge.

\begin{table}
\caption{Examples for semantic connections between interrogatives and basic nouns etc.; see \chapref{sec:5} for many examples}
\label{tab:4:8}

\begin{tabularx}{\textwidth}{QQl}
\lsptoprule

\textbf{Category} & \textbf{English} & \textbf{Basic Elements}\\
\midrule
\textsc{person} & who & man, person, one, \textsc{dem, clf}\\
\textsc{thing} & what & thing\\
\textsc{selection} & which (one) & one, \textsc{clf}\\
\textsc{kind} & what kind of & kind, sort, class\\
\textsc{activity} & to do what & to do, to make\\
\textsc{cause} & why, what for & cause, reason, \textsc{dat}, \textsc{cvb, purp}\\
\textsc{manner} & how & way, fashion, manner\\
\textsc{quantity, mass} & how much & much, few, ?amount\\
\textsc{quantity, count} & how many & many, ?number\\
\textsc{place, location} & where & place, side, \textsc{loc}\\
\textsc{place, direction} & whither, where to & direction, \textsc{all}\\
\textsc{place, source} & whence, where from & ?source, \textsc{abl}\\
\textsc{time} & when & time (+ \textsc{loc})\\
\lspbottomrule
\end{tabularx}
\end{table}

\figref{fig:4:3} is a slightly revised version of \citegen{Cysouw2005} illustration of major pathways of the derivation of interrogatives, and may also be understood as a \isi{conceptual space} for interrogatives (\citealt{Hölzl2015c}). Similar to the \isi{conceptual space} for \isi{question marking} in \sectref{sec:4.2.2}, this \isi{conceptual space} of interrogatives allows a comparison of the \isi{semantic scope} of individual interrogatives within one or across several languages. Connections between categories indicate the possibility that they can be covered by the same \isi{interrogative}. Arrows furthermore show common paths of developments, either merely semantic or by means of derivation and \isi{inflection}.

\begin{figure}
% % \includegraphics[width=\textwidth]{figures/fig_4_3.jpg}
   \vspace*{\baselineskip}
\begin{tikzpicture}
  \path [use as bounding box] node[matrix, matrix of nodes, row sep=.5cm, column sep=.5cm, nodes={font=\scshape}] (HoelzlFig43)
    {
      activity & & reason & & \\
         & thing &  & kind & \\
      person & & manner & & time\\
      & selection & & quantity & \\
      & & place & & \\    
    };
    \draw[thick,-{Triangle[]}] (HoelzlFig43-1-1) -- (HoelzlFig43-1-3);
    \draw[thick,-{Triangle[]}] (HoelzlFig43-2-2) -- (HoelzlFig43-1-1);
    \draw[thick,-{Triangle[]}] (HoelzlFig43-2-2) -- (HoelzlFig43-1-3);
    \draw[thick,-{Triangle[]}] (HoelzlFig43-2-2) -- (HoelzlFig43-2-4);
    \draw[thick,-{Triangle[]}] (HoelzlFig43-2-2) -- (HoelzlFig43-3-3);
    \draw[thick] (HoelzlFig43-3-1.east) -- (HoelzlFig43-2-2.south) -- (HoelzlFig43-4-2.north) -- (HoelzlFig43-3-1.east);
    \draw[thick,-{Triangle[]}] (HoelzlFig43-3-3) -- (HoelzlFig43-1-3);
    \draw[thick,-{Triangle[]}] (HoelzlFig43-3-3) -- (HoelzlFig43-3-5);
    \draw[thick,-{Triangle[]}] (HoelzlFig43-3-3) -- (HoelzlFig43-2-4);
    \draw[thick,-{Triangle[]}] (HoelzlFig43-3-3) -- (HoelzlFig43-4-4);
    \draw[thick,{Triangle[]}-{Triangle[]}] (HoelzlFig43-4-2) -- (HoelzlFig43-5-3);
    \draw[thick,-{Triangle[]}] (HoelzlFig43-4-2) -- (HoelzlFig43-4-4);
    \draw[thick,-{Triangle[]}] (HoelzlFig43-4-2) -- (HoelzlFig43-3-3);
    \draw[thick,-{Triangle[]}] (HoelzlFig43-4-4) -- (HoelzlFig43-3-5);
    \draw[thick,-{Triangle[]}] (HoelzlFig43-5-3) -- (HoelzlFig43-3-3);
    \draw[thick,-{Triangle[]}] (HoelzlFig43-5-3) -- (HoelzlFig43-4-4);
    \draw[thick,-{Triangle[]}] (HoelzlFig43-2-2) to [bend left=80]  (HoelzlFig43-3-5);
    \draw[thick,-{Triangle[]}] (HoelzlFig43-4-2) to [bend right=80] (HoelzlFig43-3-5);
\end{tikzpicture}
    \vspace*{\baselineskip}
\caption{A conceptual space of interrogatives}
\label{fig:4:3}
\end{figure}

Cross-linguistic data suggest that interrogatives meaning ‘what’ or ‘which’ are the unmarked and most basic members of the \isi{interrogative} system and often serve as the basis for the derivation of other interrogatives. The \isi{grammaticalization} of interrogatives to question markers and the use of interrogatives in \isi{open alternative question}s offer additional evidence for this hypothesis; in both cases it is typically an unmarked \isi{interrogative} with the meaning ‘what’ that is employed. The category \textsc{person} occupies a special position as it appears to be less prone to changes and more stable diachronically.

The \isi{conceptual space} was in need of several slight revisions. For \isi{NEA} the category of \textsc{activity} had to be added; it is integrated into the map with the following connections: \textsc{thing→activity→reason} (\citealt{Hölzl2015c}). An example is \ili{Manchu}, which has an \isi{interrogative} \textit{ai} ‘what’. This \isi{interrogative} may take a verbalizer \textit{-na-} to yield \textit{ai-na-} ‘to do what’, which, in turn, may take the imperfective \isi{converb} marker \textit{-me}, resulting in the complex \isi{interrogative} \textit{ai-na-me} ‘why’ (literally ‘doing what’ or ‘in order to do what’). The category of \textsc{kind} has also been tentatively added. For example, \ili{English} \textit{what kind of} suggests a connection \textsc{thing→kind} (see also \citealt{Idiatov2007}: 51ff.) and \ili{Mandarin} \textit{zěnme yàng de} \textsc{manner→kind} (\textit{zěnme} ‘how’, \textit{yàng} ‘kind, type’, \textit{de} ‘\textsc{attr}’). It may be necessary to update further aspects of the \isi{conceptual space} in future studies such as a possible connection \textsc{selection→kind}, but for \isi{NEA} the most important aspects are present.

Apart from these categories, the space also lacks the categories of \textsc{direction} and \textsc{source}, that are clearly related to the category of \textsc{place}. Further categories such as translatives or prolatives will be ignored due to a lack of data for most languages in \isi{NEA}. \figref{fig:4:4} shows these three categories on a small \isi{conceptual space} (\citealt{Hölzl2015c}) that is already known from studies in \isi{case} marking (e.g., \citealt{Creissels2006}). Within Cysouw’s \isi{conceptual space}, the category of \textsc{place} appears to cover not only \textsc{location} but the two categories of \textsc{direction} and \textsc{source} as well. For instance, \ili{Manchu} \textit{absi} ‘how’ derives from a form meaning ‘whither’ and \ili{German} \textit{woher} ‘whence’ may also mean ‘how, why’ in certain contexts (e.g., \textit{woher denn?} ‘why then?’). The \isi{conceptual space} for locative interrogatives may be conceptualized as the result of zooming in on the category of \textsc{place}. A close-up examination of \textsc{quantity} reveals the limited \isi{conceptual space} \textsc{mass}---\textsc{count}.

Within the \isi{conceptual space} for locative interrogatives, languages differ with respect to scope, \isi{markedness}, whether they have \isi{case} marking or special forms, and whether \isi{case} markers are also found on nouns or not. \ili{English} used to have the special forms \textit{whence} and \textit{whither}, but they have been replaced with the \isi{case} marked forms \textit{where to} and \textit{where from}. Within the new system, \textit{where} is unmarked for \isi{case}. While in \ili{English} \textit{to} and \textit{from} are usual \isi{case} markers (or prepositions), \ili{German} \textit{wo-hin} and \textit{wo-her} (derived from \textit{wo} ‘where’) have special suffixes that may otherwise only be found in the \isi{demonstratives} (and as verboids, see \sectref{sec:5.5.3.2}). \ili{English} and \ili{German} have three different forms, but \ili{Italian} \textit{dove} has scope over both \textsc{location} (\textit{Dove sei?} ‘Where are you?’) and \textsc{direction} (\textit{Dove vai?} ‘Where are you going?’), while \textsc{source} is expressed with \textit{di}/\textit{da dove} (\textit{Di dove sei?} ‘Where are you from?’, \textit{Da dove vieni?} ‘Where are you coming from?’). A recent book on spatial interrogatives that appeared after finishing this book could unfortunately not be taken into account \citep{Stolz2017}.

\begin{figure}
% % \includegraphics[width=\textwidth]{figures/fig_4_4.jpg}
 \begin{tikzpicture}
  \node[regular polygon, regular polygon sides=3, minimum size=3cm,draw] (polygon3) {};
  \node[shift=(polygon3.corner 1),above] {\scshape location}; 
  \node[shift=(polygon3.corner 3),below right] {\scshape source}; 
  \node[shift=(polygon3.corner 2),below left] {\scshape direction}; 
 \end{tikzpicture}
\caption{A simplified conceptual space for subcategories of \textsc{place}}
\label{fig:4:4}
\end{figure}

\subsection{Word class membership of interrogatives}\label{sec:4.3.2}

Typical \isi{word class} membership of interrogatives is relatively straightforward (\tabref{tab:4:9}), although there is some cross-linguistic variation. As mentioned before, interrogatives belong to a lot of different word classes. There are several clues for determining the \isi{word class} of a certain \isi{interrogative} such as inflectional properties or open derivations. For instance, \isi{interrogative} verbs in many languages are either combinations of the \isi{interrogative} ‘what’ with a plain verb such as ‘to do’ (\ili{English} \textit{to do what}) or contain a verbalizing element (\ili{Manchu} \textit{ai-na-} ‘what-\textsc{v}-’). To take another example, causal interrogatives are often verbs with a \isi{converb} marker (\ili{Even} \textit{ja-mi} ‘why’) or nouns with a \isi{case} marker such as the dative (\ili{Buryat} \textit{yüün-de} ‘why’). Nevertheless, \isi{converb} and \isi{case} markers are often related with each other diachronically and fulfill similar adverbial functions. See \chapref{sec:5} for many more examples.

Paradigms in the Australian language \ili{Djabugay} (\ili{Pama-Nyungan}), to give but one additional example, show an interesting split between a pronominal accusative pattern on the one hand (\textsc{person}) and a nominal \isi{ergative} marking on the other (\textsc{thing}) (\tabref{tab:4:10}).

\begin{table}
\caption{Typical word class membership of different interrogatives}
\label{tab:4:9}

\begin{tabularx}{\textwidth}{QQl}
\lsptoprule

\textbf{Category} & \textbf{English} & \textbf{Typical word class}\\
\midrule
\textsc{person} & who & pronoun, noun\\
\textsc{thing} & what & noun, pronoun\\
\textsc{selection} & which (one) & adjective, (pro)noun\\
\textsc{kind} & what kind of & adjective\\
\textsc{activity} & to do what & verb\\
\textsc{cause} & why, what for & adverb\\
\textsc{manner} & how & adverb\\
\textsc{quantity} & how many/much & adjective, numeral\\
\textsc{place} & where, whither, whence & adverb\\
\textsc{time} & whither, where to & adverb\\
\textsc{direction}  & whence, where from & adverb\\
\textsc{source} & when & adverb\\
\lspbottomrule
\end{tabularx}
\end{table}

\begin{table}
\caption{Inflection of Djabugay (Pama-Nyungan) interrogatives \citep[135]{Nau1999}}
\label{tab:4:10}

\begin{tabularx}{\textwidth}{XXl}
\lsptoprule
& \textsc{person} & \textsc{thing}\\
\midrule
S & \textbf{dju:} & nyirrangu\\
A & \textbf{dju:} & \textbf{nyi:}\\
O & dju:ny & \textbf{nyi:}\\
\lspbottomrule
\end{tabularx}
\end{table}

\subsection{The diachrony of interrogatives}\label{sec:4.3.3}

The diachrony of interrogatives can be described with a limited set of developmental paths summarized in \tabref{tab:4:11}. (A) Interrogatives may simply be too old to be analyzable at all. To repeat the example from Chapter 1, \ili{English} \textit{where} or \ili{German} \textit{wo(r-)} go back directly to \ili{Proto-Indo-European} *\textit{k\textsuperscript{w}}\textit{ór}. Apart from phonological changes, the form has been preserved over the course of several millennia. A special subtype of this is the loss of the \isi{resonance}, i.e. the existence of the same initial sounds in several interrogatives (\citealt{BickelNichols2007}; \citealt{Mackenzie2009}, Chapter 1). Such a resonance is usually the sign of an old etymological connection between the participating interrogatives. Given the predominance of suffixes over prefixes (e.g., \ili{Manchu} \textit{ai-\textbf{de}} ‘what-\textsc{dat}‘) and the dominant word order IntN (e.g., Manchu \textit{ai-\textbf{ba-}} ‘what-place-‘) etc., this feature might be especially pronounced in NEA. Phonological changes, such as the bonding and fusion of such analyzable forms, lead to the emergence of resonances. In most \ili{Tungusic} languages, an original \isi{resonance} that is preserved in some languages such as \ili{Nanai} \textit{x{\textasciitilde}}, was lost completely (e.g., Nanai \textit{\textbf{x}aɪ} vs. Manchu \textit{ai} ‘what‘). Such a development is unique as it affects all interrogatives that share the changing phonological feature. All other changes affect only one or two interrogatives at once.

\begin{table}
\caption{The diachrony of interrogatives excluding developments from interrogatives to other domains (\citealt{Hölzl2015c}); PT = Proto-Tungusic}
\label{tab:4:11}

\begin{tabularx}{\textwidth}{llQl}
\lsptoprule
& \textbf{Schematic} & \textbf{Details} & \textbf{Example}\\
\midrule
A & INT\textsubscript{1} > INT\textsubscript{1} & phonological changes & PIE *k\textsuperscript{w}ód > OE \textit{hwæt} > NE \textit{what}\\
B & INT\textsubscript{1} > INT\textsubscript{2} & semantic changes & \ilit{Wutun} \textit{age} ‘which (one) > who’\\
C & INT\textsubscript{1}-X\textsubscript{GRAM} > INT\textsubscript{2} & \isit{inflection}\newline (> fusion) & \ilit{English} \textit{where to}\\
D & INT\textsubscript{1}-X\textsubscript{LEX} > INT\textsubscript{2} & derivation, \isit{reinforcement}\newline  (> fusion) & \ilit{English} \textit{how much}\\
E & (INT\textsubscript{1})-X\textsubscript{LEX} > INT\textsubscript{2} & \isit{replacement} & \ilit{Italian} \textit{che} > \textit{che cosa }> \textit{cosa} ‘what’\\
F & INT\textsubscript{1}, INT\textsubscript{2} > INT\textsubscript{3} & \isit{convergence} & PT *ja, *Kai > Kh. \ilit{Evenki} \textit{i(i)}-\\
G & ?X\textsubscript{LEX} > INT & ?\isit{grammaticalization} & ?\ilit{Evenki} \textit{aŋii} ‘thing > \textsc{int}’\\
\lspbottomrule
\end{tabularx}
\end{table}

(B) There may be semantic changes that leave the formal side more or less intact or are at least not directly connected with it. One such change is the development from the meaning ‘which one’ to ‘who’ as it can be found in several languages in \isi{NEA} such as the \ili{Sinitic} language \ili{Wutun} (see also \citealt{Idiatov2007}). Both \isi{demonstratives} and interrogatives are frequently reinforced with the help of other elements, (C) grammatical (e.g., \ili{Manchu} \textit{ai-de} ‘what-\textsc{loc} > where, why, how’) or (D) lexical (\ili{Manchu} \textit{ai-ba-(de)} ‘what-place-(\textsc{loc}) > where’). Over the course of time these two elements normally fuse into one form. Possible developments of these last three types can also be found on \citegen{Cysouw2005} \isi{conceptual space} (\figref{fig:4:4}). (E) In some instances, however, the original \isi{interrogative} may be dropped such as in \ili{Italian} \textit{(che) cosa} ‘thing > what’. This is somewhat reminiscent of one of the well-known Jespersen cycles for \isi{negation} such as the gradual replacement of \textit{ne} by \textit{pas} in \ili{French}. (F) Convergence is very rare and within \isi{NEA} seems to be restricted to \ili{Tungusic} languages. In some languages such as Khamnigan \ili{Evenki}, perhaps due to phonological changes, two different \isi{interrogative} stems merged into one form. This might be treated as a subtype of change (A) but has an impact on both the form an function of several interrogatives. (G) Whether lexical items can directly develop into interrogatives as argued by \citet{Schulze2007}, for instance, is highly disputed. Most scholars deny this possibility altogether (e.g., \citealt{Diessel2003}; \citealt{CysouwHackstein2011}) and I tend to agree. There may be some valid examples, such as in \ili{Evenki} (see \sectref{sec:5.10.3}), but this certainly is much less common than developments (C), (D), and even (E).

Most of these changes have been taken into account by \citet{MuyskenSmith1990}, who developed one of the best typologies of \isi{interrogative systems} (\tabref{tab:4:12}).

\begin{table}
\caption{The typology of interrogatives according to \citet{MuyskenSmith1990}}
\label{tab:4:12}
\fittable{
\begin{tabular}{llllll}
\lsptoprule
& \textbf{Chinese} \textbf{Pidgin English} & \textbf{Sranan} & \textbf{Jamaican} & \textbf{Latin} & \textbf{KiNubi}\\
\midrule
who & who(-man) & (o)s(u)ma & huu-dat & quis & munú\\
what & wat ting & (o)san & wa(t)/we/wara & quid & s(h)unú\\
when & wat-time & o-ten & wen(-taym) & quando & mitéén\\
where & wat-side & (o)pe & we(-paat) & cur & wén\\
type & \isit{transparent} & \isit{atrophied} & mixed \isit{transparent} & fused & \isit{opaque}\\
& simple & → & → & → & complex\\
\lspbottomrule
\end{tabular}
}
\end{table}

\citet{MuyskenSmith1990} differentiated five different types of \isi{interrogative systems}. Analyzable combinations of interrogatives with other elements are called \textit{transparent}. The fusion of such analyzable forms leads to fused systems such as in \ili{Latin}, which in most forms are still related but synchronically not analyzable. The system in \ili{KiNubi} does not even exhibit such a relic and can be called \textit{opaque}, as the interrogatives are synchronically non-analyzable. Jamaican Creole has both analyzable forms such as \textit{huu-dat} (< \ili{English} \textit{who-that}) or \textit{we(-paat)} (< \ili{English} \textit{where-part}), and non-analyzable forms such as \textit{wa(t)} (< \ili{English} \textit{what}) and therefore can be called \textit{mixed-transparent}. Quite rare are \isi{atrophied} \isi{interrogative systems} that used to be \isi{transparent} but subsequently lost the actual \isi{interrogative} marker, as in \ili{Italian} \textit{(che) cosa}. The \isi{analyzability} of forms, of course, does tend to decrease over the course of time, unless new forms are built. But there may also be a development in the opposite direction, as witnessed in the reanalysis of \textit{wor.um} ‘around where, about what’ as \textit{wo.}\textit{rum} in \ili{German} which allows a connection to the word \textit{wo} ‘where’ that historically lost the final \textit{-r} (PIE *\textit{k\textsuperscript{w}}\textit{ór}).

Under extreme \isi{contact} situations an \isi{interrogative} system may be disturbed or innovated. \cite[65-66]{Bickerton2016} and \citet{MuyskenSmith1990} claim that creole and pidgin languages tend to have \isi{transparent} \isi{interrogative systems}. Chapter 1 has argued that this phenomenon might not be restricted to \isi{creoles}, but could be a more general \isi{tendency} of \isi{simplification} due to non-native L2 \isi{acquisition} of a given language (\citealt{McWhorter2007}; \citealt{Trudgill2011}; \citealt{Operstein2015}). Simplification in this case means the \isi{reduction} in the number of actual interrogatives, the “regularization of irregularities”, and the “increase in morphological \isi{transparency}” \citep[62]{Trudgill2011}. For this reason \tabref{tab:4:12} contains a rough scale of \isi{complexity}. In most cases, innovative \isi{interrogative systems} are based on an \isi{interrogative} meaning ‘what’ or ‘which’. An exception to this rule is the language \ili{Pichis Ashéninca} as described by \citet{Cysouw2007}, in which this function is fulfilled by an \isi{interrogative} meaning ‘where’ (see also \sectref{sec:5.5.3.2} on \ili{German}).

\subsection{Inflectional properties of interrogatives}\label{sec:4.3.4}

The inflectional properties of interrogatives are often quite complex and can only be briefly sketched here (see \citealt{Mushin1995}; \citealt{Nau1999}; \citealt{Siemund2001}: 1020–1023 among others). \chapref{sec:5} gives a great many examples of inflected interrogatives.

For the \isi{inflection} of interrogatives all kinds of morphological types and means are attested cross-linguistically. In \ili{Anong} (\ili{Trans-Himalayan}), for instance, the \isi{plural} of \textit{k\textsuperscript{h}}\textit{a}\textsuperscript{31}\textit{io}\textsuperscript{55} ‘who (\textsc{sg})’ is formed by \isi{reduplication}: \textit{k\textsuperscript{h}}\textit{a}\textsuperscript{31}\textit{io}\textsuperscript{55} \textit{k\textsuperscript{h}}\textit{a}\textsuperscript{31}\textit{io}\textsuperscript{55} ‘who (\textsc{pl})’ (\citealt{SunHongkai2009}: 74). As seen in \sectref{sec:4.3.3}, inflected interrogatives often grammaticalize into interrogatives with a different meaning. Locative interrogatives in \ili{Anong}, for example, exhibit a locative marker that is analyzed as suffix here, \textit{k\textsuperscript{h}}\textit{a}\textsuperscript{31}-\textit{a}\textsuperscript{55} ‘which-\textsc{loc}’ (\citealt{SunHongkai2009}: 73ff.).

\begin{table}
\caption{The inflection of interrogatives in Pite Saami (Uralic; \citealt{Wilbur2014}: 120-121)}
\label{tab:4:13}

\begin{tabularx}{.8\textwidth}{XlXll}
\lsptoprule
& who &  & what & \\
\midrule
& \textsc{sg} & \textsc{pl} & \textsc{sg} & \textsc{pl}\\
\midrule
\textsc{nom} & ge & ge & \textbf{mij} & ma(-h)\\
\textsc{gen} & ge-n & ge-j & ma-n & me-j\\
\textsc{acc} & ge-v & ge-jd & ma-v & me-jd {\textasciitilde} ma-jd\\
\textsc{ill} & ge-sa & ge-jda & ma-sa & me-jda\\
\textsc{iness} & ge-nne & ge-jdne & ma-nne & ma-jdne\\
\textsc{elat} & ge-sste & ge-jsste & ma-sste & ma-jsste\\
\textsc{com} & ge-jna & ge-j & ma-jna & me-j\\
\lspbottomrule
\end{tabularx}
\end{table}

Inflection encompasses verbal (e.g., tense, aspect), nominal (e.g., person, number, gender), and other categories. The \isi{inflection} of individual interrogatives usually depends on the \isi{word class} (\sectref{sec:4.3.2}) and often only a subset of the interrogatives takes \isi{inflection}. In \ili{German}, for example, \textit{wer} ‘who’, but not \textit{was} ‘what’, can take morphological \isi{case} marking. Only the interrogative \textit{wie viel-} ‘how many’ can take the ordinal suffix \textit{-te} that is specific to numerals, e.g. \textit{der wie-viel-te} ‘‘the how manieth’’. The most important inflectional categories for \isi{NEA} are perhaps number and \isi{case} that are often organized into paradigms as in \tabref{tab:4:13}.

Interrogatives may express additional nominal categories such as gender (e.g., Icelandic \textit{hver} ‘who\textsc{.sg}.\textsc{m}, who\textsc{.sg}.\textsc{f}’ or \textit{hvað} ‘who\textsc{.sg}.\textsc{n}’, \citealt{Siemund2001}: 1021), but this plays no important role for most of \isi{NEA}.

Inflectional properties of interrogatives can often be related to (pro)nouns or verbs, but not necessarily so. Often there is an overlap with the \isi{inflection} of \isi{demonstratives}. Consider the paradigms of nouns, \isi{demonstratives}, and interrogatives of in Pite \ili{Saami} (\ili{Uralic}) given in \tabref{tab:4:14}.

\begin{table}
\caption{The inflection of nouns, demonstratives, and interrogatives (\textsc{person}, \textsc{thing}) in Pite Saami, excluding abessive and essive markers for nouns (\citealt{Wilbur2014}: 93, 116, 120-121)}
\label{tab:4:14}

\begin{tabularx}{.8\textwidth}{XlXlXll}
\lsptoprule
& \textsc{n} &  & \textsc{dem} &  & \textsc{int} & \\
\midrule
& \textsc{sg} & \textsc{pl} & \textsc{sg} & \textsc{pl} & \textsc{sg} & \textsc{pl}\\
\textsc{nom} & - & (-h) & -t & (-h) & - (mij) & (-h)\\
\textsc{gen} & (-h) & -j & -n & -j & -n & -j\\
\textsc{acc} & -v & -jt & -v & -jt & -v & -jd\\
\textsc{ill} & -j & -jda & -sa & -jda & -sa & -jda\\
\textsc{iness} & -n & -jn & -n & -jtne & -nne & -jdne\\
\textsc{elat} & -st & -jst & -sste & -jste & -sste & -jsste\\
\textsc{com} & -jn(a) & -jn & -jna & -j & -jna & -j\\
\lspbottomrule
\end{tabularx}
\end{table}

In this language there is a strong overlap of the three different paradigms, which nevertheless all have their special properties. Overall the paradigms of the \isi{demonstratives} and interrogatives are particularly similar to each other (e.g., \textsc{gen.sg} \textit{-n} instead of \textit{-h}).

\subsection{Interrogatives and demonstratives}\label{sec:4.3.5}

Of the connections to other categories, it is especially \isi{demonstratives} that will play an important role within this study (\sectref{sec:4.3.4}, \chapref{sec:5}). In fact, many of the typological dimensions mentioned above, such as the \isi{diachronic} developments, seem to hold for both categories. A connection between the two has often been noted (e.g., \citealt{Dixon2012}), but the best \isi{analysis} of this relation has been given by \citet{Diessel2003}. Consider some examples from the Munda (\ili{Austroasiatic}) language \ili{Kharia} spoken in eastern and central India (\citealt{Peterson2011}: 178-179, 183-184). Demonstratives and interrogatives have parallels both in \isi{inflection} (e.g. \textit{a=te} ‘which=\textsc{obl}’, \textit{u=te} ‘this=\textsc{obl}’), and derivation (e.g., \textit{a=tiˀj} ‘which=side’, \textit{u=tiˀj} ‘this side’). Languages differ from each other in how strongly developed they are and how many interrogatives and \isi{demonstratives} take part in the parallel development. \ili{Kharia}, for example, has yet another \isi{interrogative} (e.g., \textit{i=te} ‘what=\textsc{obl}’) as well as two (and formerly three) additional \isi{demonstratives} (e.g., \textit{ho=te} ‘that.\textsc{med=obl}’, \textit{han=te} {\textasciitilde} \textit{hin=te} ‘that=\textsc{obl}’), not counting a loan from a neighboring language. \citet[635]{Diessel2003} has shown that \isi{demonstratives}, like interrogatives, “cross-cut the boundaries of several word classes”, express \isi{basic semantic categories} (e.g., \ili{Kharia} \textit{tiˀj} ‘side’ etc.), have etymologically non-analyzable stems, are not derived from but reinforced by lexical items, and share a similar pragmatic function (\citealt{Diessel2003}\emph{\textup{;}} 2006): “both types of expressions are commonly used as directives that \textbf{instruct the hearer to search} for a specific piece of information outside of discourse (i.e. in the surrounding situation or in the hearer’s knowledge store).” (\citealt{Diessel2003}: 636, my boldface) One difference between the two elements seems to be the fact that, while \isi{demonstratives} are usually accompanied by a \isi{pointing} gesture \citep{Diessel2006}, this does not appear to be the case for most interrogatives. Although there are deictic interrogatives, they have a schematic meaning that contradicts a specific \isi{pointing} gesture. In \ili{German} discourse, however, in some cases a selective \isi{interrogative} can be accompanied with a \isi{pointing} gesture, but this usually goes along with \isi{looking} at the addressee and furling one’s eyebrows or similar indicators of doubt. Whether there are more specific connections between interrogatives and \isi{gestures} remains to be investigated.

\section{Towards an ecological theory of questions}\label{sec:4.4}

One of the \isi{questions} formulated in the Introduction (Chapter 1) concerned the actual meaning of \isi{questions} \citep[561]{Sanitt2011}. Inspired by \citet{Schulze2007} and \citet{vanderAuweraNuyts2015}, this section thus goes beyond traditional typology and explicitly tries to add several theoretical assumptions concerning the \textit{meaning} of \isi{questions} and sketches what might be called an \isi{ecological theory of questions}.

As noted in the Introduction, the fundamental unit for an ecological theory of language 
% necessarily 
forms the so-called \textit{\isi{organism-environment system}} (OES, \citealt{Järvilehto1998}). Many cognitive approaches overemphasize the importance of the organism and especially the \isi{brain}. As Ulric Neisser---the so-called father of Cognitive Psychology---said in an interview in 1997, his 1976 book “\textit{Cognition and Reality} was partly an attempt to recall my information processing colleagues to reality, saying that there is a whole world out there to look at.” \citep[187]{Szokolszky2013} However, Neisser also correctly pointed out that traditional \isi{Ecological Psychology} (e.g., \citealt{Gibson1979}) overemphasized the environmental aspect, but neglected memory and conceptualization. The theory of the \isi{organism-environment system}, in my opinion, should aim at integrating aspects of both fields. The OES exists on several different \isi{time scales} or causal frames \citep{Enfield2014} and contains language as an integral component (e.g., \citealt{Odling-SmeeLaland2009}; \citealt{Sinha2013}). However, in the remainder of this section a focus will lie on the understudied \isi{microgenetic} frame. Some results from the \isi{diachronic} and \isi{synchronic} perspectives will be taken as hints of the basic infrastructure of this frame. This should not lead to the misunderstanding, however, that basic elements of the \textit{human \isi{interaction} engine} \citep{Levinson2006} or the \textit{economics} of \isi{questions} \citep{Levinson2012b}, most of which are located on the \isi{enchronic} frame and in the sociocultural \isi{ecology}, are unimportant. This section merely focuses on some of the less well understood aspects of \isi{questions} and emphasizes the \isi{microgenetic} frame and the cognitive \isi{ecology} of language (\citealt{SteffensenFill2014}: 7). \citet[3]{Graesser1985} was probably right that “a theory of questioning is a special case of a more general theory of conversation”, which is why only some aspects can be addressed here. Given the brackground of this book, this section is written from a linguistic perspective, although insights from other disciplines are consulted whenever feasible (cf. \citealt{Dillon1982}).

Despite its ecological background, the general outline of the theory advocated here nevertheless is strongly based on the newly emerging \textit{simulation semantics} paradigm that places a focus on the \isi{brain}, but can easily be reconciliated with ecological ideas. The fundamental concept of this theory is so-called \textit{embodied simulation}, which has been defined as “the re-enactment of perceptual, motor and introspective states acquired during experience with the world, body and mind” by \citet[1281]{Barsalou2009} or as “the creation of mental experiences of \isi{perception} and \isi{action} in the absence of their external manifestation” by \citet[14]{Bergen2012}. These two definitions are more or less congruent and highlight different aspects of one and the same phenomenon. A definition offered by \citet[527]{Gallese2009} in addition emphasizes the social aspect of simulations:

\begin{quote}
By means of \isi{embodied simulation} we do not just “see” an \isi{action}, an emotion, or a sensation. Side by side with the sensory description of the observed social stimuli, internal representations of the body states associated with these actions, emotions, and sensations are evoked in the observer, “as if” he or she were doing a similar \isi{action} or experiencing a similar emotion or sensation. That enables our social identification with others.
\end{quote}

\noindent Given its neurological background, the theory may be misunderstood as focusing on the \isi{brain}, exclusively. However, \citet{Barsalou2009} has emphasized that simulations are always situated and multi-modal, which is in accordance with the theory of the OES. The theory is broad enough to bring together \isi{conception}, \isi{perception}, and \isi{action} (and thus the organism and the environment) into one coherent theory. According to \citet[1281]{Barsalou2009}

\begin{quote}
the re-enactment process has two phases: (i) storage in long-term memory of multi-modal states that arise across the brain’s systems for \isi{perception}, \isi{action} and introspection (where ‘introspection’ refers to internal states that include affect, motivation, intentions, metacognition, etc.), and (ii) partial re-enactment of these multi-modal states for later representational use, including prediction.
\end{quote}

\noindent Thus, simulations are never complete re-enactments but are \textit{attenuated} to different degrees (\citealt{Langacker2008}: 536-537).

It is especially the last aspect of a \textit{prediction} or an \textit{anticipation} (\citealt{Järvilehto2009}) that plays a crucial role for a theory of \isi{questions}. Every question (rhetorical \isi{questions} etc. aside) contain aspects that are not actually known by the speaker but merely predicted or anticipated to play a role within a certain context. Assuming the hearer is cooperative \citep{Tomasello2014b}, the question may be answered or responded to in an expected way, if the \isi{anticipation} turns out to be appropriate. For example, one of two specified alternatives of an \isi{alternative question} (\ref{ex:4:45}a) may be chosen as adequate and thus (partly) repeated by the hearer (\ref{ex:4:45}a). If, however, the anticipation was inadequate, then the hearer will most likely point this out and give the appropriate alternative (\ref{ex:4:45}c) or try to find out what the misunderstanding is about (\ref{ex:4:45}d).

\ea%45
    \label{ex:4:45}
    \ili{English}\\
    \ea
      \textit{When are you leaving, tomorrow or the day after tomorrow?}\\

    \ex
      \textit{(I’m leaving) tomorrow.}\\

    \ex
      \textit{I’m not leaving, it is Bill who is leaving.}\\

    \ex
      \textit{I’m not leaving at all, what are you talking about?}\\
    \z
    \z 

\noindent This is traditionally known as \textit{presupposition} of a question. The background of these \isi{predictions} has been called the \textit{pattern completion inference mechanism}.

\begin{quote}
On encountering a familiar situation, an \isi{entrenched situated conceptualization} for the situation becomes active. Typically, though only part of the situation is perceived initially. A relevant person, setting, event or introspection may be perceived, which then predicts that a particular situation—represented by a situated conceptualization—is about to unfold. By running the situated conceptualization as a simulation, the perceiver anticipates what will happen next, thereby performing effectively in the situation. The agent draws inferences from the simulation that go beyond the information given \citep[1284]{Barsalou2009}
\end{quote}

\noindent Polar, \isi{focus}, and \isi{alternative question}s all rely on this anticipatory mechanism. The difference among them has to do with the fact that \isi{predictions} may be more or less plausible, with the consequence that the information given may lead to one or more possible outcomes. In addition, the \isi{uncertainty} may only concern a certain subpart of the entire simulation. This is one aspect of what is usually referred to as \textit{\isi{construal}} (e.g., \citealt{Langacker2008}), the ability to “construe the ‘same’ situation quite differently” (\citealt{Ross2014}: 127). Content questions lack any specific \isi{predictions} but still involve inferences in the sense that they rely on the activation of entrenched situated conceptualization. Consider the example of a broken window. We know from our previous experience that windows usually don’t break on their own and that somebody or something must have caused the glass to break. Most likely we would assume that there must be an agent responsible for breaking the window (e.g., one of the children usually playing soccer in front of the house), leading to the question \textit{Who broke the window?} In case we have encountered a similar situation before and know the identity of a potential agent, we may also ask something like \textit{Did Tom break the window again?} Questions are an expression of the human \isi{imaginative capacity} and thus, ironically, of knowledge, memory, and experience.

\cite[84–87]{Tomasello2008} differentiates between three basic \isi{communicative motives}, i.e. \textit{\isi{requesting}}, \textit{\isi{informing}}, and \textit{\isi{sharing}}. Arguably, \isi{questions} can be used for all three motives. Consider the constructed examples in \REF{ex:4:46}.

\ea%46
    \label{ex:4:46}
    \ili{English}\\
    \ea
      \textit{Could you open the window?}\\

    \ex
      \textit{Did you know Sarah is pregnant?}\\

    \ex
      \textit{That’s beautiful, isn’t it?}\\

    \z
    \z 

\noindent Given the overall focus of this study, however, only \isi{prototypical questions} can be covered here, i.e. actual requests for information (e.g., \citealt{Levinson2012b}), which is a special case of the first motive. However, as we have just seen, every question itself necessarily contains some amount of information.

\begin{quote}
Have you ever hesitated to ask a question? Perhaps you feared it might be foolish. Or it might be too near the bone, too probing. Perhaps it might cause offence. Or it might distract us from the business at hand and lead to other things. Or it might open you up to the reciprocal question, which you would not want to \isi{answer}. Introspection suggests a plethora of reasons for suppressing \isi{questions} that might arise in one’s mind. \citep[19]{Levinson2012b}
\end{quote}

\noindent In a certain sense, \isi{questions} are an example of the \textit{\isi{perception-action cycle}} as postulated in \isi{Ecological Psychology}: “animals [including humans] move so that they can perceive, and perceive so that they can move” (\citealt{SwensonTurvey1991}: 319, my brackets). Questions give information in order to obtain additional information necessary for a certain purpose. Nevertheless, prototypical answers are a better example of the second communicative motive. Interestingly, \isi{requesting} appears to precede \isi{informing} both \isi{phylogenetic}ally and \isi{ontogenetic}ally (\citealt{Tomasello2008}: 137, 247) and thus clearly plays a fundamental role for human beings. The third motive is irrelevant for the purpose of this study.

Prototypical questions may furthermore be characterized as a form of \textit{\isi{exploratory behavior}} that results from \textit{\isi{curiosity}}. According to \citet[87]{Loewenstein1994}, \isi{curiosity} in the sense of “an intrinsically motivated desire for specific information” is raised by the focusing of a gap in our knowledge base. Such “an information gap is characterized by two quantities: what one knows and what one wants to know.” All question types may be characterized in the same terms. In \isi{content question}s the \isi{entrenched situated conceptualization} equips us with a schematic knowledge but inquires about a specific piece of information one wants to know. In the case of \textit{who}, we know about an agent but want to know its identity. In polar and \isi{focus question}s we have a specific assumption but do not know whether this is accurate. In \isi{alternative question}s we can imagine two or more possibilities but do not know which one is the most accurate. The underlying pattern can be called a \textit{\isi{hierarchy of specificity}} of question types (\ref{ex:4:47}, cf. \citealt{Levinson2012b}: 23; \citealt{Hölzl2016b}).

\ea\upshape%47
    \label{ex:4:47}
    CQ < PQ < FQ < AQ
    \z

\noindent The term \textit{\isi{specificity}}, which contrasts with \textit{\isi{schematicity}}, has been adopted from \citet[19]{Langacker2008}; see also \citet[238]{Arnheim1969}. Focus \isi{questions} are more specific than polar \isi{questions}, because the \isi{uncertainty} just concerns the focused subpart which is much more specific than in \isi{content question}s. Alternative questions appear to be the most specific, because they openly specify all plausible alternatives. The possible negative \isi{answer} in polar and \isi{focus question}s opens up a plentitude of alternatives. There is direct evidence for this \isi{hierarchy}. One pattern recurring in many languages is a \isi{combination} of a \isi{content question} followed by a polar, \isi{focus}, or \isi{alternative question} that elaborates on the frame set by the \isi{content question} (e.g., \textit{What do you want,} \textit{coffee or} \textit{tea?}). Consider the following examples from \isi{Northeast Asia} (\ref{ex:4:48}--\ref{ex:4:50}) and beyond (\ref{ex:4:51}--\ref{ex:4:53}).

\ea%48
    \label{ex:4:48}
    \ili{Evenki} (\ili{Tungusic})\\
    \gll si \textbf{i:}-le ŋene-d’e-nni, [{d’u-la-vi=}\textbf{gu}, tatkit-tula=\textbf{gu}]?\\
    2\textsc{sg}  which-\textsc{all}  go-\textsc{prs}-2SG  home-\textsc{all}-\textsc{refl.poss}=\textsc{q} school-\textsc{all}=\textsc{q}\\
    \glt ‘Where are you going, [home or to school]?’ \citep[7]{Nedjalkov1997}
    \z

\ea%49
    \label{ex:4:49}
    \ili{Khorchin} \ili{Mongolian} (\ili{Mongolic})\\
    \gll čii \textbf{jaa.x}-sə=\textbf{{ji}}, [{tɔlgɔ=čin’} ubud-ǰææ-n=\textbf{{ʊʊ}}]?\\
    2\textsc{sg}  do.what-\textsc{p.}\textsc{pfv}=\textsc{q}   head=2\textsc{sg.poss}  hurt-\textsc{prog}-\textsc{prs}=\textsc{q}\\
    \glt ‘What’s up, [is your head aching]?’ \citep[296]{Yamakoshi2015}
    \z

\ea%50
    \label{ex:4:50}
    \ili{Udihe} (\ili{Tungusic})\\
    \gll \textbf{i:}-le una-za-i [{amä:}{-za-la=}\textbf{{nu}} zulie-ze-le=\textbf{{nu}}]?\\
    which-\textsc{loc}  travel-\textsc{subj}-2\textsc{sg}  back-\textsc{n}-\textsc{loc}=\textsc{q}    front-\textsc{n}-\textsc{loc}=\textsc{q}\\
    \glt ‘Where will you travel, [in the front or in the back]?’ (\citealt{NikolaevaTolskaya2001}: 812)
    \z

\ea%51
    \label{ex:4:51}
    \ili{Teiwa} (\ili{Timor-Alor-Pantar})\\
    \gll \textbf{{amidan}} la qau, [{tii’} \textbf{{le}} karian]?\\
    what    \textsc{foc}  good  sleep  or  work\\
    \glt ‘What is better, [sleeping or working]?’ \citep[284]{Klamer2010}
    \z

\ea%52
    \label{ex:4:52}
    \ili{Amis} (\ili{Austronesian})\\
    \gll \textbf{{cima}} ku  ta.tayra  namu    i  taypak, [ci  ɬuŋi  ci  áki]?\\
    who  \textsc{nom}  go    2\textsc{pl.gen}  \textsc{prep}  \textsc{pn} \textsc{nom}  \textsc{pn}  \textsc{nom}  \textsc{pn}\\
    \glt ‘Which one of you is going to Taipei, [Lungi or Aki]?’ (\citealt{Huang1999}: 650)
    \z

\ea%53
    \label{ex:4:53}
    Central Alaskan \ili{Yupik} (\ili{Eskaleut})\\
    \gll \textbf{nali}-ak      assik-siu, [{kuuvviaq} \textbf{{wall’u}} saayuq]?\\
    which-\textsc{abs}.3\textsc{du.sg}  like-\textsc{2sg.3sg.q}    coffee.\textsc{abs.sg}  or  tea.\textsc{abs.sg}\\
    \glt ‘Which do you want, [coffee or tea]?’ \citep[170]{Miyaoka2012}
    \z

\noindent It is difficult to determine whether this is a \isi{universal} pattern, because grammar books never explicitly address it as a phenomenon on its own right. Nevertheless, the fact that it can be found in languages from around the world indicates that it is a strong \isi{tendency} at the very least. Future studies have to determine the exact meaning of this pattern, which may differ from instance to instance and from language to language. Additional examples from \ili{Chalkan}, \ili{Chuvash}, \ili{Udihe}, \ili{Uilta}, \ili{Uzbek}, \ili{Kalmyk}, and \ili{Ket} can be found throughout \chapref{sec:5}. See also \sectref{sec:6.3} for examples from the \ili{Timor-Alor-Pantar} language \ili{Abui} and the \ili{Austronesian} language \ili{Balantak}. In general terms the pattern can be described as the iconic linguistic expression of a possible \isi{universal} that starts with the schematic and, by means of \isi{exploration} and \isi{anticipation}, gradually arrives at the more specific (e.g., \citealt{Bar2009}; \citealt{Barsalou2009}). The same phenomenon can be observed in \isi{focus question}s with a focus on generic nouns that are more specific than interrogatives, but are followed by a question with an even more specific or proper noun (e.g., \textit{Do you want tea}\textit{, Earl Grey} \textit{or Pu-Erh} \textit{perhaps?}). In both cases the crucial point is that the first question is located lower on the scale of \isi{specificity} in \REF{ex:4:47} than the second. In a way, epistemic \textit{tag questions} mirror this structure because they start from a rather general statement and arrive at the specific question of whether this statement is appropriate (e.g., \textit{You want tea,} \textit{right?}). A major difference, however, is the fact that the first element in a \isi{tag question} is not a question itself or at least is not overtly marked as such. Another difference is the scope of the second question over the whole proposition in the case of many tag \isi{questions}. Alternative \isi{questions} exhibit some affinity to this pattern as well, but there are major differences. While in all examples above the second sentences elaborate on, or are based on, the first one, alternative \isi{questions} have mutually exclusive alternatives. A \isi{similarity} with alternative \isi{questions} is, however, the fact that in both cases there is the possibility of ellipsis (cf. \textit{Do you want tea} \textbf{\textit{or}} \textit{(do you want) coffee}? and \textit{What do you want, (do you want)} \textit{coffee or} \textit{tea?}). Nevertheless, alternative \isi{questions} are better treated as a \isi{question type} comparable to polar and content \isi{questions} as there are further differences, such as the connection of alternative \isi{questions} with the domain of \isi{coordination}. Tag questions, which are much more similar to this pattern than \isi{alternative question}s, of course, do not repeat the same statement. Instead, \isi{question tag}s may anaphorically refer to the statement (e.g., \textit{isn’t} \textbf{\textit{it}}?).

To borrow a term from \citet{Langacker2008} again, it may be claimed that these combinations of \isi{questions} follow a natural and dynamic path of \textit{\isi{mental access}} that unfolds through time, from the schematic to the specific:

\begin{quote}
Between the moment the organism is confronted with the problem and the moment the final solution is achieved there occur, as a rule, a number of intermediate steps leading, in an hierarchical fashion, \textbf{from general to more specific features} of the sought-after solution. (\citealt{Duncker1939}: 178, \isi{emphasis} modified)
\end{quote}

In \citegen[83]{Langacker2008} terminology, this can also be called a \textit{\isi{reference point} relationship}, in which the second part (the target) is mentally located with respect to the first (the \isi{reference point}). \citegen[102]{Dewey1910} description of the phenomenon is still surprisingly accurate. He differentiates between three different situations, the first two of which define the extremes, i.e. absolute certainty and \isi{uncertainty}:

\begin{quote}
Unless there is something doubtful, the situation is read off at a glance; it is taken in on sight, \textit{i.e.} there is merely apprehension, \isi{perception}, recognition, not judgment. If the matter is wholly doubtful, if it is dark and obscure throughout, there is a blind mystery and again no judgment occurs.
\end{quote}

\noindent The third situation exactly corresponds to the scale of \isi{uncertainty} in between these extremes:

\begin{quote}
But if it suggests, however vaguely, different meanings, rival possible interpretations, there is some \textit{point at issue,} some \textit{matter at stake.} Doubt takes the form of dispute, controversy; different sides compete for a conclusion in their favor. Cases brought to trial before a judge illustrate neatly and unambiguously this strife of alternative interpretations; but any case of trying to clear up intellectually a doubtful situation exemplifies the same traits. A moving blur catches our eye in the distance; we ask ourselves: “What is it? Is it a cloud of whirling dust? a tree waving its branches? a man signaling to us?” Something in the total situation suggests each of these possible meanings. Only one of them can possibly be sound; perhaps none of them is appropriate; yet \textit{some} meaning the thing in question surely has.
\end{quote}

\noindent Not only this \isi{combination} of \isi{questions}, but \isi{questions} in general can be characterized as an expression of \textit{uncertainty} (e.g., \citealt{Schulze2007}). However, \isi{uncertainty} is merely one of several \textit{\isi{collative} variables}, a term coined by \citet[44]{Berlyne1960}.

\begin{quote}
For want of a more satisfactory term, we shall call them \textit{\isi{collative}} \textit{variables} since, in order to evaluate them, it is necessary to examine the similarities and differences, compatibilities and incompatibilities between elements—between a present stimulus and stimuli that have been experienced previously (novelty and change), between one element of a pattern and other elements that accompany it (\isi{complexity}), between simultaneously aroused responses (conflict), between stimuli and expectations (surprisingness), or between simultaneously aroused expectations (\isi{uncertainty}).
\end{quote}

\noindent Given the ecological background of this study, the terms \textit{stimulus} and \textit{response} have to be treated with caution. Instead of passively reacting to the environment, the organism itself may engage in active \isi{exploratory behavior} (e.g., \citealt{Dewey1896}; \citeyear{Dewey1910}: 193; \citealt{Gibson1960}; \citeyear{Gibson1979}: 55ff.; \citealt{Gibson1988}: 5-6). Conceptually, this is a similar distinction as that between \textit{\isi{natural selection}} by the environment and \textit{\isi{niche construction}} by the organism that we have seen in the Introduction (\citealt{Odling-SmeeLaland2009}). In many cases, it is the actions and the movements of the organism itself that lead to the pick-up of \textit{novel}, \textit{changing}, \textit{complex}, \textit{conflicting}, \textit{surprising}, or \textit{uncertain} information. \citet[89]{BaranesaOudeyerGottlieb2015} argue “that \textbf{curiosity} can be viewed as a pro-active process that anticipates, or motivates agents to obtain new information, whereas \textbf{surprise} indicates a reactive process after having processed the information.” (my boldface) This in turn results in further \isi{exploratory behavior}.

Of course, there is also the artificial arousal of \isi{curiosity} such as, for instance, in a \textit{riddle}, which “compares an object to another entirely different object. Its essence consists in the surprise that the solution occasions”. Eventually, “the hearer perceives that he has entirely misunderstood what has been said to him.” \citep[129]{Taylor1943} The \isi{riddle} arouses \isi{curiosity} in the addressee by means of \isi{collative} information and initiates the search for the solution. Take an example from the \ili{Tungusic} language \ili{Uilta}, which starts with the introduction \textit{gaŋ gaŋ gajagoo!} and goes on as follows: \textit{boo toptoŋgoor, naa toptoŋgoor, xai-gəək? toksiik unuu!} ‘In heaven there are spots, on earth there are spots, what are they? Riddle me!’ The \isi{riddle} has several possible answers such as \textit{boo unigərinnii suŋdatta xəsiktənnii} ‘The stars in heaven and the scales of fish.’ The \isi{answer} is followed by the reply \textit{toksiik} ‘Correct.’ \citep[93]{Ikegami1958}, which puts an end to \isi{curiosity}.

A basic typology of different kinds of \isi{curiosity} was also sketched by \citet{Berlyne1954} who differentiates between two dimensions that define four types of \isi{curiosity} (see also \citealt{Dewey1910}: 30ff.). These have been concisely summarized by \citet[77]{Loewenstein1994} as follows.

\begin{quote}
\textbf{Perceptual} \isi{curiosity} referred to “a drive which is aroused by novel stimuli and reduced by continued exposure to these stimuli” [\citealt{Berlyne1954}: 180]. \textbf{Epistemic} \isi{curiosity} referred to a desire for knowledge and applied mainly to humans. \textbf{Specific} \isi{curiosity} referred to the desire for a particular piece of information, as epitomized by the attempt to solve a puzzle. Finally, \textbf{diversive} \isi{curiosity} referred to a more general seeking of stimulation that is closely related to boredom. In the four-way categorization produced by these two dimensions, \textbf{specific perceptual} \isi{curiosity} is exemplified by a monkey’s effort to solve a puzzle, \textbf{diversive perceptual} \isi{curiosity} is exemplified by a rat’s exploration of a maze [...], \textbf{specific epistemic} \isi{curiosity} is exemplified by a scientist’s search for the solution to a problem, and \textbf{diversive epistemic} \isi{curiosity} is exemplified by a bored teenager’s flipping among television channels. (my boldface and square brackets)
\end{quote}

\noindent It is especially \textit{specific epistemic curiosity} that plays a crucial role for the characterization of \isi{questions}. Above, we have already encountered the knowledge gap theory of \isi{curiosity} by \citet{Loewenstein1994}, which is strongly based on \isi{Gestalt Psychology}: “If \isi{curiosity} is like a hunger for knowledge, then a small ‘priming dose’ of information increases the hunger, and the decrease in \isi{curiosity} from knowing a lot is like being satiated by information.” (\citealt{Kang2009}: 963) The first to sketch a \isi{gestalt} approach to \isi{curiosity} was also \citet[181]{Berlyne1954}, proposing “a drive to fill in such gaps in the subject’s experienced representations”. This is based on the well-known \isi{gestalt} principle of \textit{closure}. Fritz \citet[119]{Perls1973}---the father of Gestalt Therapy---put it this way: “The \isi{gestalt} wants to be completed. If the \isi{gestalt} is not completed, we are left with unfinished situations, and these unfinished situations press and press and press and want to be completed.” In a different terminology one could say that an \isi{embodied simulation} wants to be completed. For instance, unanswered \isi{questions} usually lead to an “increased effort in constructing a coherent representation” (\citealt{Hoeks2013}: 8). If there is insufficient information to complete a simulation, \isi{curiosity} and exploration set in. What exactly the evolutionary origins of \isi{curiosity} are is another matter that cannot be addressed here. The point is that \isi{curiosity} is a psychologically real phenomenon and has to be taken into account for a characterization of \isi{questions}. \citegen[219]{Gibson1979} statement that “[t]he visual system \textit{hunts} for comprehension and clarity” can perhaps be generalized to the entire \isi{organism-environment system}. Humans seek comprehension and clarity, and \isi{questions} are one way of achieving this. \citegen[182]{Berlyne1954} description is based on a somewhat outdated terminology but nevertheless remains basically valid:

\begin{quote}
When a question is put, whether by the subject himself or by somebody else, and the \isi{answer} is already known, the appropriate \isi{response} is made as a reaction conditioned by previous learning to the stimulus-pattern, and this relieves the drive immediately, so that the subject can proceed to some other activity. However, when the \isi{answer} is not known, the drive will persist, and some sort of trial-and-error process can be expected to follow as with any other drive-state.
\end{quote}

\noindent He mentions three different possibilities for this “trial-and-error process”, \textit{thinking}, \textit{observation}, and \textit{recourse to authority}. The first refers to processes mostly restricted to the organism such as problem solving or memory, but the latter two roughly correspond to the physical and social environment, respectively (see also \citealt{Lewin1936}: 24ff.; \citealt{SteffensenFill2014}: 7).

Put differently, one may \textit{resolve curiosity} in three different but interrelated ways. First, in most cases one’s own experience and memory are sufficient, although in some cases additional thought processes such as problem solving may be necessary. This is the traditional realm of \isi{Cognitive Science}. In his \textit{Natural history of human thinking}, \citet{Tomasello2014a} defines \textit{thinking} as

\begin{quote}
a single cognitive process, but one that involves several key components, especially (1) the ability to cognitively represent experiences to oneself “offline”; (2) the ability to simulate or make inferences transforming these representations causally, intentionally, and/or logically; and (3) the ability to self-monitor and evaluate how these simulated experiences might lead to specific behavioral outcomes---and so to make a thoughtful behavioral decision.
\end{quote}

\noindent The fundamental mechanism of representation assumed in this study is \isi{embodied simulation} as defined above. Perhaps most instances of \isi{curiosity} are simply resolved by inferences and \isi{predictions} and their subsequent evaluation whether they are plausible or not. But it is wrong to assume, as Tomasello seems to be well aware, that simulations may be completely “offline” or detached from the environment. In fact, as \citet[1]{Glenberg1997} observed, simulations may be said to be basically “driven by the environment”:

\begin{quote}
A significant human skill is learning how to suppress the overriding contribution of the environment to conceptualization, thereby allowing memory to guide conceptualization. The effort used in suppressing input from the environment pays off by allowing prediction, recollective memory, and language comprehension. \end{quote}

\noindent The pay-off is a plausible evolutionary explanation to pay less attention to a potentially dangerous environment. But \citegen{Glenberg1997} inclusion of language comprehension is problematic, as language usually is an aspect of our social environment. Simulations based on language, for example when we listen to somebody asking us a question, are certainly driven by the (social) environment.

Second, in some cases we may encounter situations or objects that we have not encountered before or are otherwise unfamiliar with. In this case we may simply move around, explore, and change our relative perspective and distance in order to perceive previously inaccessible aspects. This is something \isi{Ecological Psychology} has focused on from its very beginnings (see \citealt{Gibson1979}). In this sense, \isi{curiosity} is simply resolved by physically exploring and changing our position relative to the problematic object. Small objects, of course, may be grasped and turned in order to be investigated in a more thorough fashion. A different example of physical exploration based on diversive instead of specific \isi{curiosity} can be illustrated by the wanderlust of the \ili{Tungusic} speaking \ili{Evenki} in \isi{Siberia}.

\begin{quote}
The \ili{Evenki} learn from an early age \textbf{to be interested in, rather than frightened by, risky situations} and the possibility of exploring new territories. For them, seeking out new places offers a wonderful opportunity to experience companionship and, as a result, it is common to go somewhere \textbf{just for the sake of exploration}. (\citealt{SafonovaSántha2013}: 142, my boldface)
\end{quote}

\noindent This is one of several reasons for the extraordinary wide distribution of \ili{Evenki} and their close linguistic relatives such as the \ili{Even} over all of the northern half of \isi{NEA} (\sectref{sec:2.10}).

Third, instead of observing or \isi{thinking} on our own, we may also see whether other people can help us clarify certain aspects of a problematic object or situation. This is mostly accomplished by means of language, of course, and the most prototypical tool for this are \isi{questions}. \citet[636]{Hodges2009}, in analogy to \citegen{Gibson1979} \textit{ambient optic array}, proposed the name \textit{dialogical array},

\begin{quote}
a group of hearer-speakers surrounding a given speaker-hearer, listening and talking in ways that reveal, inevitably, something of their perspectives, their intentions, and their histories relative to the present place and time. Like light, the ordered \isi{gestures} of the array, as well as their disordering and reordering over time, allow a participant in the array to have their own orderings restructured on various scales. It is an array of partners, actual and potential, who provide information, not just about themselves as intentional agents and as objects, but about objects, events, and agencies beyond the physical and temporal horizons of the immediate physical surround. \citep[636]{Hodges2009}
\end{quote}

\noindent The \isi{dialogical array} \textit{\isi{affords}} (\citealt{Lewin1936}; \citealt{Gibson1979}) linguistic \isi{interaction} such as asking \isi{questions}. From one point of view \isi{questions} are a form of bodily \isi{action} that bring about changes in the \isi{dialogical array}, which in turn allows the pick-up of new information (cf. \citealt{SwensonTurvey1991}). This reliance on other people potentially brings with it the danger of deception and misinformation as well as of social costs (e.g., \citealt{Levinson2012b}: 20), but pays off by being faster and requiring less effort, especially if we are dealing with complex problems. This might be the reason why \isi{questions} apparently are a \isi{universal} property of language. While \isi{exploratory behavior} can also be found in other animals, language in general and \isi{questions} in particular crucially depend on the \textit{\isi{ultra-social}} nature of human beings who usually tend to cooperate with each other in ways that are unique \citep{Tomasello2014b}.

Of course, the above distinction is only a heuristic one. In principle, the three means of resolving \isi{curiosity} are interrelated and often combined. They merely highlight different aspects of the \isi{organism-environment system} (\citealt{Lewin1936}: 27; \citealt{SteffensenFill2014}: 7). Questions, for instance, necessarily contain aspects of all three types of \isi{exploration} mentioned above. While the social dimension is the most important, both physical movements (e.g., \isi{eye contact}) as well as \isi{thinking} (e.g., \isi{predictions}) are crucial elements as well. Questions trigger incomplete simulations in the hearer, based on her experience and memory, who then engages in \isi{exploratory behavior} herself. Here basically the same three mechanisms come into play. Either the hearer has sufficient information to fill in the gaps herself, or she engages in other \isi{exploratory behavior} (e.g., \isi{looking} something up), or seeks additional help and asks the same question of somebody else who is likely to know the \isi{answer}.

The discussion thus far has overemphasized the \isi{microgenetic} aspect of \isi{questions}, but this last point has mentioned some \isi{enchronic} properties as well. The social aspect of \isi{questions} can be observed, for example, through a relatively strong obligation on the part of the addressee to respond (e.g., \citealt{Levinson2012b}: 16). The \isi{interaction} of \isi{questions} with \isi{evidentiality} offers additional insights into the social nature of \isi{interrogativity}. In many languages that have perspective marking, there is a shift to the perspective of the addressee in \isi{questions}. Consider some examples from the \ili{Cha’palaa} language spoken in Ecuador that has an egophoric system.

\ea%54
    \label{ex:4:54}
    \ili{Cha’palaa} (\ili{Barbacoan})\\
    \ea
    \gll (i-ya)    pipe-\textbf{{yu}}.\\
    1-\textsc{top}    bathe-\textsc{ego}\\
    \glt ‘I bathed.’

    \ex
    \gll (ñu/ya)    pipe-\textbf{{we}}.\\
    2/3    bathe-\textsc{n.ego}\\
    \glt ‘You/(s)he bathed.’

    \ex
    \gll (ñu-ya)    pipe-\textbf{{yu}}?\\
    2-\textsc{foc}    bathe-\textsc{ego}\\
    \glt ‘Did you bathe?’

    \ex
    \gll (ya-a)    pipe-\textbf{{n}}?\\
    3-\textsc{foc}    bathe-\textsc{n.ego.q}\\
    \glt ‘Did (s)he bathe?’ (\citealt{SanRoqueFloydNorcliffe2016}: 136-137)
    \z
    \z 

\noindent The egophoric marker \textit{-yu} appears in both statements that refer to a first person (\ref{ex:4:54}a) and in \isi{questions} that refer to a second person (\ref{ex:4:54}c). \citet[245]{TournadreLaPolla2014} capture this phenomenon with the \textit{\isi{anticipation rule}}, which they illustrate with \ili{Tibetan}: “whenever the speaker asks a direct question of the hearer, she should anticipate the access/source available to the hearer and select the evidential auxiliary/copula accordingly.” The underlying mechanism can be explained with the help of \isi{embodied simulation}: The questioner asks the question \textit{as if} she was the addressee herself \citep[527]{Gallese2009}, using \isi{predictions} obtained through mentally simulating the situation. See \sectref{sec:5.9.2.1} on \ili{Wutun} and \sectref{sec:5.9.2.2} on \ili{Amdo Tibetan} for additional examples.

According to \citet[248]{Schulze2007}, furthermore, there is a “strong coupling of the first person with assertions and of the second person with modal features, among them \isi{interrogativity}.” This important observation, it seems, can be directly observed in a number of languages that exhibit a \isi{split type} based on person. \ili{Qiang}, for example, which we have encountered above, has a special \isi{question marker} for second person \isi{singular}. Some \ili{Turkic} languages have a split system that is sensitive to second person, too (\sectref{sec:5.11.2}). Regarding \ili{West Greenlandic} (\ili{Eskaleut}), \citet[199]{Sadock1984} observed that

\begin{quote}
in all cases where the subject is second person, there is an \isi{interrogative} form that is distinct from the indicative; in some cases where the subject is third person (nowadays only where there is no object, but formerly also where the object was third person), there are distinct \isi{interrogative} and indicative forms; but in no case where the subject is first person is there a separate \isi{interrogative} form.
\end{quote}

\newpage 
\noindent Sadock proposes the \isi{hierarchy} in \REF{ex:4:55}.

\ea\upshape%55
    \label{ex:4:55}
    2 > 3 > 1
    \z

\noindent Questions are most likely to refer to a second person and least likely to refer to a first person. This should not be interpreted as a strict implicational \isi{hierarchy}, however, which allows no exceptions. Nevertheless, it seems to be a valid \isi{tendency} that had actually already been discovered by \citet[3]{Bolinger1957}: “\textit{You} occurs oftener than not in Qs. It therefore ‘means’ ‘question,’ loosely and insufficiently, but enough so that a locution not otherwise identifiable as a Q becomes one (is reacted to as one) if \textit{you} is present.” This highlights the social aspect of \isi{questions}, which are strongly rooted in communicative \isi{interaction}, and has an analogue in the \isi{gazing behavior} of the questioner. \citet[239]{RossanoBrownLevinson2009}, based on the investigation of the three very different speech communities of \ili{Italian} (Indo-European), \ili{Yélî Dnye} (no affiliation), and Tenejapan \ili{Tzeltal} (Mayan), found that it is especially the questioner who is gazing at the addressee (instead of the other way around), which is in accordance with my subjective impression for conversations in \ili{German}. Recently, \citet[81]{BaranesaOudeyerGottlieb2015} additionally found “that higher \isi{curiosity} was associated with earlier anticipatory orienting of gaze toward the [expected] \isi{answer} location”. These facts are also consistent with the explanation of \isi{questions} as a form of \isi{exploratory behavior} in the \isi{dialogical array} because most other types of exploration involve some kind of active \isi{looking}. As \citet[212]{Gibson1979} put it, “\isi{looking} is always exploring”.