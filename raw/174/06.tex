\chapter{Interrogative constructions in Northeast Asia: A summary}\label{sec:6}

Chapter 5 presented a very detailed description of \isi{questions} in \isi{NEA} based on a classification into language families. This chapter has an areal and typological perspective instead. Unfortunately, the information found for almost all languages is insufficient for an exhaustive typology. Usually, only the elicitation from a native speaker, the existence of a large and modern grammar book or of a specialized description of \isi{questions} offer enough information. Not only is there insufficient information on individual question types and on the \isi{semantic scope} of markers and interrogatives, but most descriptions also lack adequate information on \isi{intonation}. This summary follows the same structure as the previous discussion. \sectref{sec:6.1} gives an overview of \isi{question marking} and \sectref{sec:6.2} of interrogatives. A set of 12 maps in the style of the \textit{\isi{World Atlas of Language Structures}} (WALS), based on a sample of 83 languages for which sufficient information was available, is presented in \sectref{sec:6.4}. However, except for \figref{fig:6:5} (\citealt{Dryer2013k,Dryer2013m}), there is no equivalent for them in the WALS. \sectref{sec:6.3} evaluates the significance of the \isi{grammar of questions} with an \isi{emphasis} on \isi{language contact}.

\section{Question marking}\label{sec:6.1}
% \footnotemark{}

\subsection{Marking strategies}\label{sec:6.1.1}

Chapter 4 introduced a four-way typology based on the \isi{markedness} a comparison of declarative sentences with polar \isi{questions}. \tabref{tab:6:1} and the following discussion is based on the sample of 83 languages. The majority of languages belongs to type 4 while types 1 and 2 are not attested. Type 3 is found in Central Siberian \ili{Yupik}, \ili{Korean}, \ili{Jeju}, and perhaps \ili{Nganasan}, all of which are located in peripheral regions. Tundra \ili{Nenets} and Forest \ili{Enets} show a mixture of types 3 and 4 and were thus excluded.

\begin{table}
\caption{Marking of \isi{polar question}s versus declarative sentences in \isi{NEA}}
\label{tab:6:1}

\begin{tabularx}{\textwidth}{XXl}
\lsptoprule
& \textbf{unmarked PQ} & \textbf{marked PQ}\\
\midrule
\textbf{marked declarative} & Type 1: 0 & Type 3: 4\\
\textbf{unmarked declarative} & Type 2: 0 & Type 4: 77\\
\lspbottomrule
\end{tabularx}
\end{table}

\newpage
Details of the marking of \isi{polar question}s are given in \figref{fig:6:5}. Altogether 45 out of 83 languages (about 54\%) have a sentence-final marker as the major \isi{question marking} strategy. Deviations can mostly be found in peripheral regions such as \isi{Amdo}, \isi{Korea}, the \isi{Ryūkyūan Islands}, \isi{Chukotka}, and the lower \isi{Yenisei}. In comparison, only 314 (about 36\%) of \citegen{Dryer2013k} global sample of 884 languages had a sentence-final \isi{question marker} (particle or clitic). If one considers all the languages that have sentence-final \isi{question marker}s, including those with additional marking strategies (\figref{fig:6:6}), the figure rises to 62 (about 75\%) languages out of 83. This speaks in favor of an extremely strong areal feature of \isi{NEA}. \citegen{Dryer2013k} map indicates that adjacent areas to the west and southwest indeed show less sentence-final question markers. However, there is no such clear boundary with MSEA in the southeast. sentence-final question particles are generally more common in verb-final languages such as in \isi{NEA}, but are also common in SVO languages such as in MSEA (\citealt{Dryer2013a}: 274, 277). Concerning this feature, \cite[Chapter Text]{Dryer2013k} discovered “an area within \isi{Asia} including mainland \isi{Southeast Asia} and extending west into India and north through \isi{China} to \isi{Japan} and eastern \isi{Siberia}”. This study has demonstrated that almost all of \isi{NEA} shares the feature as well. There is a clear area around \isi{Amdo} that extends towards the south and encompasses \ili{Trans-Himalayan} languages from several subbranches; it is characterized by verbal affixes (\sectref{sec:5.9.2.2}). This forms a clear boundary towards the south (see also \citegen{Dryer2013m}). A marked difference also exists between \isi{NEA} and North America (\sectref{sec:5.13.4}).

The marking of \isi{content question}s is shown in \figref{fig:6:7}. Altogether, 41 (about 49\%) out of 83 languages have morphosyntactically unmarked content \isi{questions}. As opposed to \isi{polar question}s only 15 languages (about 17\%) have a sentence-final particle or clitic exclusively, but 13 (about 16\%) have a morphosyntactic marker. However, by counting all languages that have sentence-final markers or affixes among other strategies, the figures rise to 27 (ca. 33\%) and 18 (ca. 22\%), respectively. The first is restricted to the middle part of \isi{NEA}, stretching from \isi{Japan} in the east to \isi{Xinjiang} in the west. Regarding the latter, there are two possible areas: (1) \ili{Koreanic} and northern \ili{Ryūkyūan} in the southeast, (2) \ili{Yupik}, parts of \ili{Samoyedic}, \ili{Yukaghiric}, and perhaps \ili{Negidal} (but not \ili{Turkic}), in the north.

Information for \isi{alternative question}s is unavailable for 34 out of 83 languages (\figref{fig:6:8}). Of the remaining 49 languages, 21 (about 43\%) exhibit the \isi{double marking} type, exclusively. These are mostly located in the northern half, but excluding \isi{Chukotka} and \isi{Kamchatka}. Of the 19 languages with a mixed type, only \ili{Plautdiitsch}, \ili{Yiddish}, and Urumqi Han \ili{Chinese} do not have \isi{double marking} as one of several marking strategies. The remaining 16 languages are mostly located in the southern half. In sum, 37 (about 76\%) out of the 49 languages, at least as one possibility, exhibit the \isi{double marking} strategy. \ili{Selkup} and \ili{Ket} share a unique \isi{double marking} strategy in which the respective markers appear before each alternative. In all other languages the markers follow the alternatives. This indicates areal convergence of \ili{Ket} and \ili{Selkup} as well as a special position of \ili{Ket} among the languages of \isi{NEA} (\sectref{sec:3.5}). Altogether, 17 (about 35\%) out of the 49 languages contain a \isi{disjunction}, which may or may not be accompanied by other \isi{question marker}s (\figref{fig:6:9}). These are mostly located in the southern half of \isi{NEA}, including \ili{Korean}, \ili{Mongolian}, \ili{Chinese}, \ili{Russian}, \ili{Uyghur}, \ili{Kazakh}, and surrounding languages. Focusing only on those languages that have \isi{single marking} in \isi{alternative question}s (\figref{fig:6:10}), there are indications for two areas, (1) a clear area in \isi{Amdo} with possible connections to \isi{Xinjiang} (\ili{Uyghur}) and areas to the south, and (2) \ili{Yiddish} and \ili{Ukrainian} that share an areal background in Eastern \isi{Europe}.

For most languages no relevant data was available for \isi{focus question}s, which is why no map was created. There is sufficient information to conclude that \isi{focus question}s in \ili{Japonic} languages as well as \ili{Mandarin} tend to contain both a \isi{focus} and a \isi{question marker}, while in \ili{Tungusic}, \ili{Amuric}, and \ili{Aleut} as well as \ili{Old Japanese} and \ili{Middle Mongol} the same \isi{question marker} as in polar \isi{questions} is employed, which usually attaches to the verb in \isi{polar question}s and to the element in \isi{focus} in \isi{focus question}s.

Morphosyntactic question markers tend to be extremely short, usually just two or three phonemes long. This indicates their grammatical function as well as relatively high frequency. From genetic and areal perspectives the brevity of the forms represents both an obstacle and a possible pitfall: the shorter a given form, the more likely are chance resemblances. In fact, it is easy to find identical question markers in languages from around the world. But in most cases geographical distance and a lack of \isi{interaction} clearly show that these similarities must be due to chance, e.g. \ili{Amdo Tibetan} \textit{=na}, \ili{Sibe} \textit{=na}, and \ili{Ura} \textit{=na}. Question markers usually have a simple form that may be represented as (C)V(V), i.e. they consist of a minimum of one vowel phoneme (e.g., \ili{Amdo Tibetan} \textit{ə{}-}\textsc{v}) that is optionally preceded by a consonant. In some instances there is a long vowel or a diphthong (e.g., \ili{Hateruma} \textit{=naa}, \ili{Khalkha} \textit{=(y)UU}, \ili{Ulcha} \textit{=nʊʊ}, \ili{Yakut} and \ili{Dolgan} \textit{=duo} \textstyleStrong{{\textit{{\textasciitilde} =duu}}}, Kolyma \ili{Yukaghir} \textstyleStrong{{\textit{=duu}}}). Note that almost all question markers, independent of their morphological status, share this pattern. Only in some rare cases are there question markers that do not conform to this generalization (e.g., \ili{Ket} \textit{bə́}\textit{ndu}, Tundra \ili{Nenets} \textit{=w°h}, \ili{Nganasan} V\textit{{}-sutə}, \ili{Alutor} \textit{matka}, \ili{Koryak} \textit{met’ke}, Xunke \ili{Oroqen} \textit{jɔɔma}). In most cases it may be surmised that the \isi{question marker} is of a relatively young age and will be subject to phonetic erosion during future developments. In some languages the marker may consist of one consonant only. However, while it is true that in \ili{Uyghur}, for instance, the marker may have the form \textit{{}-m}, it still preserves the more conservative variant \textit{=mu} as well. In short, the pattern (C)V(V), although not \isi{universal}, is an extremely strong \isi{tendency} for question markers in \isi{NEA} and perhaps worldwide.

Disjunctions tend to be longer and follow no clear phonotactic pattern (e.g., Atkan \ili{Aleut} \textit{asxuunulax}, Xunke \ili{Oroqen} \textit{aaki}, \ili{Russian} \textit{ili}, \ili{Plautdiitsch} \textit{öuda}, \ili{Yiddish} \textit{odər}, \ili{Mandarin} \textit{háishì}, \ili{Kazakh} \textit{ælde}, \ili{Sarig Yughur} \textit{ta\textsuperscript{h}}\textit{qï}, Tundra \ili{Yukaghir} \textstyleStrong{{\textit{ejk}}}, \textit{uuri}). Question tags can well be very short (e.g., \ili{English} \textit{eh?}, \ili{German} \textit{ge?} etc.), but are usually considerably longer and less homogenous (e.g., \ili{Mandarin} \textit{duì ma?}, \textit{duì-bu-duì?}, \ili{Russian} \textit{ne pravda li?}, \ili{German} \textit{nicht wahr?}, \textit{richtig?}, \ili{English} \textit{don’t you?}, \textit{right?}, \ili{Plautdiitsch} \textit{es nich zöu?}, \ili{Sarikoli} \textit{na sou-d=o?} etc.). This proves the special position of \isi{disjunction}s and \isi{question tag}s with respect to the domain of \isi{question marking}.

\subsection{Semantic scope}\label{sec:6.1.2}

A comparison of polar \isi{questions} with \isi{content question}s reveals that 61 (about 73\%) out of the 83 languages have different marking strategies (\figref{fig:6:11}). In a global sample of 50 languages, \cite{Hölzl2015c} found a comparable figure of 73\%. There is evidence for one clear area including \ili{Japanese}, \ili{Koreanic}, \ili{Ainuic}, Written \ili{Manchu} (not shown on the map), \ili{Kilen}, \ili{Ulcha} (not shown on the map), \ili{Dagur}, \ili{Khorchin} (not shown on the map), and \ili{Ōgami}, which has the same marking in polar and content \isi{questions}. These languages furthermore also tend to mark \isi{focus} and \isi{alternative question}s in the same way. When looking at only those languages in which polar and content \isi{questions} are overtly marked differently (\figref{fig:6:12}), there are three clear areas: (1) parts of \ili{Ryūkyūan}, (2) parts of \ili{Mongolic} and \ili{Turkic}, as well as perhaps Kolyma \ili{Yukaghir}, \ili{Middle Korean} (not shown on the map), and Gyeongsang \ili{Korean} (not shown on the map), as well as (3) \ili{Amuric} and \ili{Uilta}. These results clearly prove \citegen[13]{Levinson2012b} rather dubious implicational \isi{universal} wrong: “For all languages that have clear \isi{interrogative} markers, they mark yes-no \isi{questions} (or \isi{polar question}s) differently from Wh-\isi{questions} (or content \isi{questions}).” A similar implicational \isi{universal} by \citet[1019]{Siemund2001}---“if a language uses a particle to mark constituent interrogatives, then this language will also allow the use of this particle in polar interrogatives”---had already been disproved by Hölzl (\citeyear{Hölzl2015c}; \citeyear[23]{Hölzl2016a}).

The comparison of the \isi{semantic scope} of polar and \isi{alternative question}s is severely hampered by several problems. For instance, \ili{Kazakh} as spoken in \isi{China} has a \isi{question marker} \textit{=MA} that appears in both polar and alternative \isi{questions}. However, the latter additionally exhibit a \isi{disjunction} \textit{ælde}. If the \isi{disjunction} is seen as a \isi{question marker}, then \isi{polar question}s in \ili{Kazakh} are marked differently. If, on the other hand, disjunctions are seen as a different \isi{functional domain} that in some languages combines with \isi{question marking}, the \isi{question marker} of polar and alternative \isi{questions} is the same. This study decided for the latter alternative. However, \ili{Kazakh}, like many other languages of \isi{NEA} employs two of the \isi{polar question} markers in alternative \isi{questions}. Again, we face two mutually exclusive possibilities, but this time no clear solution to the problem is available. For 34 languages no information is available for alternative \isi{questions}. Of the remaining 49 languages, when excluding disjunctions and neglecting the difference between single and \isi{double marking}, 37 (ca. 76\%) exhibit the same marking as in polar \isi{questions}. Seven languages exhibit a mixed type and only five languages (ca. 10\%) exhibit different polar and alternative \isi{question marking} (see \figref{fig:6:8} and \figref{fig:6:13}).

A comparison of the semantic maps of all the \isi{question marking} systems with the help of the \isi{conceptual space} is possible for only a handful of languages. For reasons of space, \figref{fig:6:1} only shows a selection of four languages. These and the data above suggest that there is a strong dividing line between content \isi{questions} and \isi{polar question}s, which in turn show affinities with both \isi{focus} and alternative \isi{questions}. \sectref{sec:4.2.2} introduced a possible \isi{universal} that is shown with dashed lines between content, \isi{focus}, and \isi{alternative question}s (Content questions are only marked in the same way as \isi{focus} or \isi{alternative question}s if \isi{polar question}s are also marked in the same way.)
% \todo[inline]{please change color of "V-mood", "-ii\#", "V-l(a)/-lo", and "ma\#" to gray and avoid overlaps}
\begin{figure}
% % \includegraphics[width=\textwidth]{figures/fig_6_1.jpg}
\begin{tikzpicture}
  \matrix (HoelzlFig61) [matrix of nodes,nodes in empty cells,column sep=1.25cm, row sep=.75cm]
    {
      & AQ & \\
      & PQ & \\
      FQ & & CQ\\
    };
  \draw[thick,dotted] (HoelzlFig61-1-2) -- (HoelzlFig61-3-1) -- (HoelzlFig61-3-3) -- (HoelzlFig61-1-2) node[midway]{\color{gray}\textsc{V-mood}~~~~~~};
  \draw[thick,solid] (HoelzlFig61-1-2) -- (HoelzlFig61-2-2);
  \draw[thick,solid] (HoelzlFig61-3-1) -- (HoelzlFig61-2-2);
  \draw[thick,solid] (HoelzlFig61-3-3) -- (HoelzlFig61-2-2);
  \node[above=1.5\baselineskip of HoelzlFig61-1-1.base,anchor=base,color=gray] {2$\times$\textsc{V-mood}};  
  \node[below=1.5\baselineskip of HoelzlFig61-3-1.base,anchor=base,color=gray,] { \textsc{V-mood} (+ \#\textsc{foc}\slash-\textit{(n)un} \textsc{top})};
  \node[below=1.5\baselineskip of HoelzlFig61-3-3.base,anchor=base,color=gray,align=right] {\textsc{V-mood}}; 
  \coordinate [above=1.5ex of HoelzlFig61-1-2] (A);
  \coordinate [below left=.25ex and 1.5ex of HoelzlFig61-3-1] (B);
  \coordinate [below right=.25ex and 1.5ex of HoelzlFig61-3-3] (C);
  \draw[gray, thick, rounded corners] (A)--(B)--(C) -- cycle;
\end{tikzpicture}
\begin{tikzpicture}
  \matrix (HoelzlFig61b) [matrix of nodes,nodes in empty cells,column sep=1.25cm, row sep=.75cm]
    {
      & AQ & \\
      & PQ & \\
      FQ & & CQ\\
    };
  \draw[thick,dotted] (HoelzlFig61b-1-2) -- (HoelzlFig61b-3-1) -- (HoelzlFig61b-3-3) -- (HoelzlFig61b-1-2);
  \draw[thick,solid] (HoelzlFig61b-1-2) -- (HoelzlFig61b-2-2);
  \draw[thick,solid] (HoelzlFig61b-3-1) -- (HoelzlFig61b-2-2);
  \draw[thick,solid] (HoelzlFig61b-3-3) -- (HoelzlFig61b-2-2);
  \node[above=1.5\baselineskip of HoelzlFig61b-1-1.base,anchor=base,color=gray] {\textit{asxuunulax} `or'};  
  \node[below=1.5\baselineskip of HoelzlFig61b-3-1.base,anchor=base,color=gray] {\textsc{foc} \textit{-ii}};
  \node[below=1.5\baselineskip of HoelzlFig61b-3-3.base,anchor=base,color=gray,align=right] {\textit{Ø}};
  \node[draw, gray, rounded corners, fit=(HoelzlFig61b-1-2)] {};
  \node[draw, gray, rounded corners, fit=(HoelzlFig61b-3-3)] {};
  \coordinate[above=1ex of HoelzlFig61b-2-2] (a);
  \coordinate[right=1ex of HoelzlFig61b-2-2] (b);
  \coordinate[below=1ex of HoelzlFig61b-3-1] (c);
  \coordinate[left=1ex of HoelzlFig61b-3-1] (d);
  \draw[gray, thick, rounded corners, thick] (a) -- (b) -- (c) -- (d) -- cycle;
  \node [above=.1ex of b] {\color{gray} \-\textit{ii\#}};
  \end{tikzpicture} 
  
\vspace*{8mm}\noindent
  \begin{tikzpicture}
  \matrix (HoelzlFig61b) [matrix of nodes,nodes in empty cells,column sep=1.25cm, row sep=.75cm]
    {
      & AQ & \\
      & PQ & \\
      FQ & & CQ\\
    };
  \draw[thick,dotted] (HoelzlFig61b-1-2) -- (HoelzlFig61b-3-1) -- (HoelzlFig61b-3-3) -- (HoelzlFig61b-1-2) node[midway,auto,swap,fill=white]{\color{gray} V-\textit{l(a)\slash-lo}};
  \draw[thick,solid] (HoelzlFig61b-1-2) -- (HoelzlFig61b-2-2);
  \draw[thick,solid] (HoelzlFig61b-3-1) -- (HoelzlFig61b-2-2);
  \draw[thick,solid] (HoelzlFig61b-3-3) -- (HoelzlFig61b-2-2);
  \node[above=1.5\baselineskip of HoelzlFig61b-1-1.base,anchor=base,color=gray] {2$\times$-\textit{l(a)\slash-lo}};  
  \node[below=1.5\baselineskip of HoelzlFig61b-3-1.base,anchor=base,color=gray] {\textsc{foc} -\textit{l(a)\slash-lo}};
  \node[below=1.5\baselineskip of HoelzlFig61b-3-3.base,anchor=base,color=gray,align=right] {\textit{-ŋa, -at(a)}};
  \node[draw, gray, rounded corners, fit=(HoelzlFig61b-3-3)] {};
  \coordinate[above right=1ex and 1ex of HoelzlFig61b-1-2] (a);
  \coordinate[right=1ex of HoelzlFig61b-2-2] (b);
  \coordinate[below=1ex of HoelzlFig61b-3-1] (c);
  \coordinate[left=1ex of HoelzlFig61b-3-1] (d);
  \coordinate[above left=1ex and 1ex of HoelzlFig61b-1-2] (e);
  \draw[gray, thick, rounded corners, thick] (a) -- (b) -- (c) -- (d) -- (e) -- cycle;
  \end{tikzpicture}
\begin{tikzpicture}
  \matrix (HoelzlFig61b) [matrix of nodes,nodes in empty cells,column sep=1.25cm, row sep=.75cm]
    {
      & AQ & \\
      & PQ & \\
      FQ & & CQ\\
    };
  \draw[thick,dotted] (HoelzlFig61b-1-2) -- (HoelzlFig61b-3-1) -- (HoelzlFig61b-3-3) -- (HoelzlFig61b-1-2) node[midway,auto,swap,fill=white]{\color{gray} \textit{ma\#}};
  \draw[thick,solid] (HoelzlFig61b-1-2) -- (HoelzlFig61b-2-2);
  \draw[thick,solid] (HoelzlFig61b-3-1) -- (HoelzlFig61b-2-2);
  \draw[thick,solid] (HoelzlFig61b-3-3) -- (HoelzlFig61b-2-2);
  \node[above=1.5\baselineskip of HoelzlFig61b-1-1.base,anchor=base,color=gray] {\textit{háishi} `or'};  
  \node[below=1.5\baselineskip of HoelzlFig61b-3-1.base,anchor=base,color=gray] {\color{gray} \textit{ma\#} + \textit{shì} \textsc{foc}};
  \node[below=1.5\baselineskip of HoelzlFig61b-3-3.base,anchor=base,color=gray,align=right] {\textit{Ø}};
  \node[draw, gray, rounded corners, fit=(HoelzlFig61b-1-2)] {};
  \node[draw, gray, rounded corners, fit=(HoelzlFig61b-3-3)] {};
  \coordinate[above=1ex of HoelzlFig61b-2-2] (a);
  \coordinate[right=1ex of HoelzlFig61b-2-2] (b);
  \coordinate[below=1ex of HoelzlFig61b-3-1] (c);
  \coordinate[left=1ex of HoelzlFig61b-3-1] (d);
  \draw[gray, thick, rounded corners, thick] (a) -- (b) -- (c) -- (d) -- cycle;
  \end{tikzpicture}
\caption{Semantic scope of question markers in \ili{Korean} (top left), Atkan \ili{Aleut} (top right), \ili{Nivkh} (bottom left), and \ili{Mandarin} (bottom right)}
\label{fig:6:1}
\end{figure}

The only possible exception to this rule found in \isi{NEA} is the \ili{Ryūkyūan} language \ili{Miyara} from the \ili{Japonic} \isi{language family} (\citealt{Davis2015}). In this language, if compared with the \isi{declarative sentence}, both \isi{focus} as well as content \isi{questions} lack the indicative marker. But as further specified in \sectref{sec:5.6.2}, \isi{content question}s, like declaratives, have falling \isi{intonation} while polar and \isi{focus question}s share rising \isi{intonation}. That the indicative marker is missing results from the fact that both types of \isi{questions}, content and \isi{focus}, share a \isi{focus} marker that is incompatible with the indicative. This is a subtype of the phenomenon usually called \textit{\isi{kakari musubi}} (\isi{focus} concord). In the end, \ili{Miyara} thus most likely presents no exception to the \isi{universal}. The second \isi{universal} (Focus and alternative \isi{questions} can only be marked in the same way if polar \isi{questions} are also marked in the same way.) seems to hold for \isi{Northeast Asia} as well. Of course, both universals (or tendencies) can be unified into one form: Focus, alternative, and content \isi{questions} can only be marked in the same way if \isi{polar question}s are also marked in the same way.

\subsection{Interaction of functional domains}\label{sec:6.1.3}

\sectref{sec:4.2.3} identified the following possible interactions of functional domains (see also \citealt{Hölzl2016a}: 24): (1) \isi{grammaticalization}, (2) \isi{combination}, (3) \isi{fusion}, (4) \isi{interaction} (\isi{split type}s). Tables \tabref{tab:6:2} to \tabref{tab:6:5} give a list of all instances of these interactions in \isi{NEA}.

\begin{table}
\caption{Grammaticalization of question markers in \isi{NEA} (1)}
\label{tab:6:2}

\begin{tabularx}{\textwidth}{llQl}
\lsptoprule

\textbf{Language} & \textbf{Type} & \textbf{Form} & \textbf{Source Domain}\\
\midrule
\ilit{Alutor} & PQ & \#matka & ?\textsc{int}\\
\ilit{Koryak} & PQ & \#met’ke & ?\textsc{int}\\
\ilit{Ukrainian} & PQ & \#čy & \textsc{int} ‘how’\\
\ilit{Tocharian A} & ?PQ & aśśi & adposition + \textsc{int}\\
\ilit{Sogdian} & PQ & (ə)ču-t(i) & \textsc{int} ‘what-\textsc{comp}’\\
& AQ & kataar(-əti) & \textsc{int} ‘which-\textsc{comp}’\\
\ilit{German} & TQ & was, wie & \textsc{int} ‘what’, ‘how’\\
& TQ & oder & \textsc{or}\\
\ilit{Selkup} & PQ & qaj V & \textsc{int} ‘what’\\
\ilit{Japanese} & PQ & tte & \textsc{quot}\\
& PQ & no\textsc{\#} & \textsc{nmlz < ?gen}\\
\ilit{Ainu} & PQ & ruwe\textsc{\#}, hawe\textsc{\#}, siri\textsc{\#} & \textsc{nmlz}\\
\ilit{Ōgami} & CQ & {}-ɛɛ & ?\textsc{nmlz}\\
& PQ & =tu\# & \textsc{?foc}\\
& CQ & =ka\# & \textsc{?foc}\\
\ilit{Khorchin} & PQ & ʃii & \textsc{neg}\\
\ilit{Mongolic} & CQ & \ilit{Khalkha} be\#, \ilit{Buryat} be\# {\textasciitilde} =b, \ilit{Khamnigan Mongol} bei\#, \ilit{Oirat} =w {\textasciitilde} =b, \ilit{Shira Yughur} bə\# & \textsc{cop}\\
\ilit{Turkic} & CQ & \ilit{Salar} V-i, \ilit{Tuvan} V-Il, \ilit{Tofa} V-(u)l, \ilit{Dolgan} V-ij, \ilit{Yakut} V\textstyleStrong{{{}-(n)ɪj}} & \textsc{dem} > ?\textsc{cop}\\
 & PQ & e.g., \ilit{Turkish} \textit{=mI} & \textsc{neg} > \textsc{q}\\
\ilit{Mandarin} (dialects) & PQ & ma\# & ?\textsc{neg} (NAQ)\\
& PQ & bu\#, mei\# & \textsc{neg} (NAQ)\\
\ilit{Amdo Tibetan} & PQ & =na & ?\textsc{neg} (NAQ)\\
\ilit{Nganasan} & PQ, CQ, AQ & \textsc{prs} V{}-ŋu/-ŋa, \textsc{pst} V{}-hu/-ha, \textsc{fut} V{}-sutə, \textsc{it} V{}-kəə, \textsc{renarr} V{}-ha & \textsc{tame}\\
\ilit{Chukchi}, \ilit{Kerek} & CQ & \textsc{imp-V} & \textsc{imp}\\
Forest \ilit{Enets} & PQ, CQ, ? & \textsc{pst} V-sa & tense\\
Tundra \ilit{Nenets} & PQ, CQ, AQ & \textsc{pst} V-sa & tense\\
\lspbottomrule
\end{tabularx}
\end{table}

The origin of most \isi{question marker}s is obscure. Several somewhat unclear cases discussed in Chapter 5 such as the \ili{Ryūkyūan} (\textit{=na(a)} and variants) or the \ili{Yakut} and \ili{Dolgan} question markers (\textit{=duo} {\textasciitilde} \textit{=duu}) that could be related to interrogatives meaning ‘what’ were omitted. If \ili{Ōgami} \textit{=ka} and \textit{=tu} indeed derive from \isi{focus} markers, this is most likely also true for several other \ili{Ryūkyūan} languages (e.g., \ili{Shuri}, \ili{Tsuken}, \ili{Tarama}, \ili{Ikema}, \ili{Irabu}, \sectref{sec:5.6.2}). \ili{Mandarin} dialects have also not been listed separately. There are several possible instances of shared \isi{grammaticalization} such as the development from nominalization to question markers in the \ili{Japanese} archipelago (\ili{Ainuic}, \ili{Japonic}) as well as the development of \isi{content question} markers from copulas in several \ili{Mongolic} and \ili{Turkic} languages (see \sectref{sec:5.8.2} and \sectref{sec:5.11.2}).

\begin{table}
\caption{Combination of question markers with other functional domains in \isi{NEA} (2)}
\label{tab:6:3}

\begin{tabularx}{\textwidth}{llQl}
\lsptoprule

\textbf{Language} & \textbf{Type} & \textbf{Form} & \textbf{Meaning}\\
\midrule
\ili{Japanese} & ?FQ & ka\# + wa & \textsc{top}\\
\ili{Korean} & ?FQ & \textsc{V-mood} + -(n)un & \textsc{top}\\
\ili{Wutun} & ?FQ & =mu\#/=a\# + -ha & \textsc{top}\\
\ili{Yuwan} & FQ & V-ui + =du & \textsc{foc}\\
\ili{Shuri} & FQ & =ji\# + =du & \textsc{foc}\\
& FQ & V-ra + =ga & \textsc{foc}\\
\ili{Ikema} & FQ & =na\# + =du & \textsc{foc}\\
\ili{Irabu} & PQ & =ru\# + =ru & \textsc{foc}\\
& CQ & =ga\# + =ga & \textsc{foc}\\
& FQ & 2x =ru + =ru & \textsc{foc}\\
\ili{Miyara} & FQ & {}- + lack of \textsc{ind} -\textsc{n} + =du & \textsc{foc}\\
\ili{Khalkha} & FQ & =(y)UU\# + \isi{intonation} & \textsc{foc}\\
\ili{Mandarin} & FQ & ma\# + shì & \textsc{foc}\\
& AQ & A ne B + háishì & \textsc{or}\\
Urumqi Hui \ili{Mandarin} & AQ & A ȵi\textsuperscript{44} B + xɛ\textsuperscript{24}sɿ\textsuperscript{21} & \textsc{or}\\
Hezhou \ili{Chinese} & AQ & X ȵi\textsuperscript{3}, Y, [haishi] Z etc. & \textsc{or}\\
Xunke \ili{Oroqen} & AQ & 2x =jA + aaki & \textsc{or}\\
\ili{Kilen} & AQ & =a\# + xəɕi & \textsc{or}\\
\ili{Kazakh} & AQ & 2x =MA, 2\textsc{sg} V-MI + ælde & \textsc{or}\\
\ili{Uyghur} & AQ & 2x =mu, \textsc{npst} V-m + yaki & \textsc{or}\\
\ili{Sarig Yughur} & AQ & 2x =Mi, =mu\# {\textasciitilde} V-m + ta\textsuperscript{h}qï & \textsc{or}\\
\lspbottomrule
\end{tabularx}
\end{table}

\begin{table}
\caption{Fusion of question markers with other functional domains in \isi{NEA} (3)}
\label{tab:6:4}

\begin{tabularx}{\textwidth}{QlQl}
\lsptoprule

\textbf{Language} & \textbf{Type} & \textbf{Form} & \textbf{Meaning}\\
\midrule
\ilit{Ainu} & PQ & an & \textsc{cop}\\
\ilit{Shuri} & NPQ & ʔa-ran-i & \textsc{cop}\\
Shirongolic (\ilit{Mongolic}) & PQ & \ilit{Bonan} wu(u), (m)bu, \ilit{Kangjia} vʉ, mbʉ, \ilit{Santa} wu, \ilit{Mangghuer} beinu & \textsc{cop}\\
Shirongolic (\ilit{Mongolic}) & PQ & V-(\textsc{tam}.)\textsc{q} & \textsc{tense}\\
\isi{Amur} \ilit{Nivkh} & FQ & \textsc{foc}=l(a)/=lo & \textsc{foc}\\
East \isi{Sakhalin} \ilit{Nivkh} & FQ & \textsc{foc}=l(a)/=lu & \textsc{foc}\\
Atkan \ilit{Aleut} & FQ & \textsc{foc}=ii & \textsc{foc}\\
\ilit{Russian} & FQ & \#\textsc{foc}=li & \textsc{foc}\\
\ilit{Even}, (Khamnigan) \ilit{Evenki}, \ilit{Udihe} & FQ & \textsc{foc}=KU & \textsc{foc}\\
\ilit{Udihe} & FQ & \textsc{foc}=nA & \textsc{foc}\\
\lspbottomrule
\end{tabularx}
\end{table}

Of course, the exact mechanisms and processes involved in these instances of \isi{grammaticalization} need additional investigation. \tabref{tab:6:3} excludes \isi{negation} and interrogatives. As can be seen, \isi{question marking} most commonly combines with \isi{focus} marking and disjunctions. Three patterns of fusion have been listed in \tabref{tab:6:4}.

\begin{table}
\caption{Split types of polar and content question marking found in \isi{NEA} (4)}
\label{tab:6:5}
\footnotesize
\begin{tabularx}{\textwidth}{llQQ}
\lsptoprule

\textbf{Language} & \textbf{Type} & \textbf{Form} & \textbf{Criterion}\\
\midrule
CSY & PQ, CQ & \textsc{V-mood+agr.q} & person, number\\
NSY & PQ, CQ & \textsc{V-mood+agr.q} & person, number\\
\ilit{Sirenikski} & PQ, CQ & \textsc{V-mood+agr.q} & person, number\\
\ilit{Kazakh} & PQ & =MA\#, 2\textsc{sg} V-MI & person, number\\
\ilit{Hatoma} & CQ & \textsc{pst -, attr} + =wa\#, =ja\# (non-verbal) & clause type\\
\ilit{Sonai} & CQ & =ga, =ja(a) (non-verbal) + =ba \textsc{sel,} + =du \textsc{foc} & clause type\\
Gyeongsang & PQ, CQ & PQ -na, \textsc{cop} -ka, CQ -no, \textsc{cop} -ko & clause type\\
\ilit{Jeju} & PQ, CQ & \textsc{V-mood} & politeness\\
\ilit{Korean} & PQ, CQ & \textsc{V-mood} & politeness\\
Hezhou \ilit{Chinese} & PQ & ma\textsuperscript{3}, la\textsuperscript{3}, ȵi\textsuperscript{3}mu\textsuperscript{3} & politeness, polarity\\
& CQ & ʐa\textsuperscript{3}, ȵi\textsuperscript{3}, ȵi\textsuperscript{3}ʐa\textsuperscript{3} & politeness, semantic category, \isi{gender} of speaker\\
\ilit{Japanese} & PQ & {}-kana(a) vs. -kashira etc. & \isi{gender} of speaker\\
Kolyma \ilit{Yukaghir} & CQ & \textsc{foc.case}, S/O\textsc{agr.q} & grammatical relations\\
Tundra \ilit{Yukaghir} & CQ & \textsc{foc.case}, S/O\textsc{agr.q} & grammatical relations\\
\ilit{Ōgami} & PQ & =ka\#, ?=tu, \textsc{pst}, \textsc{cop}, \textsc{stat} V-ɛɛ & tense, clause type, aktionsart\\
\ilit{Wutun} & PQ & \textsc{pfv}, \textsc{res} =mu\#, \textsc{ipfv}, \textsc{progr} =a\# & tense, aspect\\
\ilit{Nganasan} & PQ, CQ & \textsc{prs} V{}-ŋu/-ŋa, \textsc{pst} V{}-hu/-ha, \textsc{fut} V{}-sutə, \textsc{it} V{}-kəə, \textsc{renarr} V{}-ha & tense, aspect\\
\ilit{Uyghur} & PQ & =mu\#, \textsc{npst} V-m & tense\\
Forest \ilit{Enets} & PQ & \textsc{pst} V-sa & tense\\
\ilit{Negidal} & PQ & =Kʊʊ\# + \textsc{fut} \textsc{V-mood+agr.q}, =i\# & tense, person, ?\\
\ilit{Shuri} & PQ & \textsc{ind} V-mi, \textsc{neg} V-i, \textsc{pst} V-ti, =naa\# & polarity, tense, ?\\
Written \ilit{Manchu} & PQ, CQ & \textsc{cop} =o, =ni, =nio, =nA, =nu & clause type, ?\\
Masana \ilit{Okinoerabu} & PQ & \textsc{ind} -ŋ, -Ø + =nja\# \textsc{+ pst} -ti instead of -ta, =na\# & tense, ?\\
Tundra \ilit{Nenets} & PQ & \textsc{pst} V-sa, \textsc{dub} =w°h & tense, ?\\
\ilit{Amdo Tibetan} & PQ & ə{}-\textsc{v}, =na\#, =ni\# & ?\\
\ilit{Oirat} & PQ & =(y)UU\# {\textasciitilde} =ii\# & ?\\
\ilit{Bonan} & PQ & V-(\textsc{tam}.)\textsc{q}, V-si & ?\\
Hezhou & PQ & ma\#, la\#, (ȵi)mu\# & ?\\
Xunke \ilit{Oroqen} & PQ & =jA\#, =ɔɔ\# & ?\\
\ilit{Udihe} & PQ & =Ku\#, =nA\# & ?\\
\ilit{Kilen} & PQ & =nə\#, =a\#, =ma\# & ?\\
\ilit{Salar} & PQ & =mU\#, =U\# & ?\\
\ilit{Sarig Yughur} & PQ & =Mi\#, =mu\# {\textasciitilde} V-m & ?\\
A \ilit{Nivkh} & PQ & V=l(a)/=lo & ?\\
ES \ilit{Nivkh} & PQ & V=l(a)/=lu & ?\\
A \ilit{Nivkh} & CQ & =ŋa, =at(a) & ?\\
ES \ilit{Nivkh} & CQ & =ŋa, =ŋu, =ara & ?\\
\lspbottomrule
\end{tabularx}
\end{table}

\newpage 
As can be seen from the entries with question markers in \tabref{tab:6:5}, a large number of descriptions fails to mention the criteria for distinguishing between different question markers. At least for \ili{Manchu} it could be shown that it depends in part on clause type (see \sectref{sec:5.7.2}). There is a wide variety of different criteria, but many, like \isi{question marking} itself, are verbal categories (e.g., TAME, agreement, polarity, clause type).

\subsection{Borrowing}

\tabref{tab:6:6} gives a list of borrowed question markers in \isi{NEA}. Some cases are not absolutely clear. See \sectref{sec:3.1} for the methodology of establishing whether a \isi{question marker} has actually been borrowed. Most cases require an additional evaluation and elaboration by experts of the individual languages.

\begin{table}
\caption{Possible instances of borrowing and loan translations of question markers in \isi{NEA}. When several dialects have a given form, only one was mentioned}
\label{tab:6:6}
\small 
\begin{tabularx}{\textwidth}{QllQ}
\lsptoprule
& \textbf{Type} &  & \\
\midrule
\ilit{Yakut}, \ilit{Dolgan} =duo, =duu & PQ, AQ & ?→ & Kolyma \ilit{Yukaghir} =duu\\
\ilit{Mandarin} ba \zh{吧} & PQ & → & \ilit{Mangghuer} ba, \ilit{Bonan} ba, \ilit{Kangjia} ba, \ilit{Santa} ba, \ilit{Kilen} ba, Sanjiazi \ilit{Manchu} ba, \ilit{Sibe} ba, \ilit{Salar} ba, \ilit{Khorchin} ba(a), ?\ilit{Oroqen} baa {\textasciitilde} bəə, Shineken \ilit{Buryat} baa, \ilit{Solon} baa, \ilit{Dagur} baa etc.\\
\ilit{Mandarin} ma \zh{吗} & PQ & → & \ilit{Bonan} ma, \ilit{Kilen} ma, Yibuqi \ilit{Manchu} m\textsc{a}\\
\ilit{Mandarin} háishì \zh{还是} & AQ & → & Yibuqi \ilit{Manchu} xɛʂı, \ilit{Kilen}, \ilit{Oroqen} haʃi\\
\ilit{Russian} ili & AQ & → & \ilit{Ket} ili, \ilit{Chulym} ili, ?\ilit{Kamass} aali, ?Anadyr \ilit{Yukaghir} ali\\
\ilit{Nivkh} -ŋa, =ii & CQ, PQ & → & \ilit{Uilta} =KA, ?=(y)i\\
\ilit{Khamnigan Mongol} bei & CQ & → & Khamnigan \ilit{Evenki} bei\\
\ilit{Russian} li/\textcyrillic{ли} & PQ & → & \ilit{Khamnigan Mongol} =li (AQ)\\
\ilit{Koreanic}, e.g. \ilit{Middle Korean} -ni, -nyo, -(k)o, ?-nja & all & → & \ilit{Jurchenic}, e.g. \ilit{Manchu} =ni, =nio, =o, ?=nA etc.\\
\ilit{Mongolic} ‘\textsc{cop>q}’ (*büi) & CQ & → & \ilit{Salar} -i, \ilit{Tuvan} -Il, \ilit{Dukhan} -Ĭl, \ilit{Yakut} \textstyleStrong{{{}-(n)ɪj, \ilit{Dolgan} -}}ij\\
\ili{Mongolian} =(y)UU {\textasciitilde} y.ii & PQ, AQ & → & \ilit{Salar} -u {\textasciitilde} -o, \ilit{Oroqen} =ɔɔ, Ongkor \ilit{Solon} =uu {\textasciitilde} =ii, ?\ilit{Even} =i, ?\ilit{Negidal} =i, ?\ilit{Uilta} =(y)i\\
\ilit{Dagur} =yee & PQ, AQ & → & \ilit{Oroqen} =j\textsc{e}, ?\ilit{Sibe} =jə etc.\\
\ilit{Udihe} =nu, (?=nA) & PQ, FQ, AQ & → & \ilit{Kilen} =nə\\
\ili{Mongolian} youm=aa & PQ & ?→ & \ilit{Oroqen} =jOOmAA\\
\ilit{Khorchin} jimɛɛ & CQ & ?→ & \ilit{Solon} yeeme\\
\ilit{Buryat} =gü & PQ, AQ & ?→ & \ilit{Solon} =gi\\
\ilit{Ewenic} (\ilit{Tungusic}) *=Ku & PQ, AQ & ?→ & \ilit{Buryat} =gü, \ilit{Khamnigan Mongol} =gu\\
\ilit{Uyghur} =mu & PQ & → & \ilit{Salar}, \ilit{Sarig Yughur}, Hezhou, \ilit{Tangwang}, \ilit{Wutun} =mu\\
\ilit{Uzbek} =mi & PQ & → & \ilit{Tajik} =mi\\
\ilit{Yukaghir} \textsc{agr.q} & CQ & ?→ & \ilit{Negidal} (PQ, CQ, AQ) \textsc{agr.q}\\
\ilit{Ukrainian} čy etc. & PQ, AQ & → & \ilit{Yiddish} ci\\
\ilit{Burushaski} or \ilit{Dardic} =a & PQ & ?→ & \ilit{Sarikoli} =o\\
\ilit{Kamass} =a & PQ & ?→ & \ilit{Kott} â\\
(\ilit{Turkic} ?→) \ilit{Kamass} =bV & AQ & ?→ & \ilit{Kott} =bo\\
Hezhou la\textsuperscript{3} \zh{啦} & PQ & → & \ilit{Santa} {}-la\\
\ilit{Old Japanese} ya & PQ & ?→ & \ilit{Ainu} ya\\
Xining \ilit{Mandarin} lɛ\textsuperscript{53} \zh{呢} & CQ & → & \ilit{Kangjia} le\\
\lspbottomrule
\end{tabularx}
\end{table}

One of the most widespread markers is \ili{Mandarin} \textit{ba} \zh{吧}. Probably due to its special semantics (\sectref{sec:5.9.2.1}), it is far more likely to be borrowed than a more neutral \isi{question marker} such as \ili{Russian} \textit{li}/\textcyrillic{ли} (mostly used in the written language), and in fact it can be found in many languages of \isi{China}, from \isi{Xinjiang} to \isi{Manchuria}. Some very unclear cases were excluded.

In a few cases it is more likely that similarities are due to chance. \ili{Gothic}, for example, has a second position \isi{question marker} \textit{=u} as well as a sentence initial \isi{question marker} \textit{an}. At a first glance, these are surprisingly similar to \ili{Ket} second position \textit{=u} and sentence initial \textit{an} ‘what’, but the large distance in both time and space makes a comparison more than doubtful.

\section{Interrogatives}\label{sec:6.2}
\subsection{Formal properties}\label{sec:6.2.1}

This study has emphasized on formal properties of interrogatives such as the overall shape and the initial sounds. Several language families in \isi{NEA} exhibit a \is{KIN-interrogative}KIN-inter\-ro\-ga\-tive: the \isi{interrogative} meaning ‘who’ in a given language has the form KIN (velar or uvular plosive or fricative, (high) vowel (short or long), (apical) nasal), followed by an optional final vowel, e.g. \ili{Turkish} \textit{kim}, Forest \ili{Nenets} \textit{kim’a}, \ili{Aleut} \textit{kiin} etc. 30 (ca. 36\%) out of the 83 languages exhibit KIN-interrogatives (\figref{fig:6:14}). The phenomenon can be traced back over considerable time-spans to several of the \isi{proto-languages} of \isi{NEA} (\tabref{tab:6:7}). In some instances (indicated with a question mark), the \isi{similarity} most likely is due to pure chance. \ili{Itelmen} \textit{k’e}, for example, superficially resembles the other forms, but most likely derives from PCK *\textit{mikæ} \citep{Fortescue2005}.

\begin{table}
\caption{KIN-interrogatives in \isi{NEA}}
\label{tab:6:7}

\begin{tabularx}{\textwidth}{XXl}
\lsptoprule

\textbf{Language} & \textbf{Form} & \textbf{Source}\\
\midrule
?\ilit{Ainuic} & *gu(n)na & \citealt{Vovin1993}\\
Atkan \ilit{Aleut} & kiin & \citealt{Bergsland1997}\\
\ilit{Eskimo} (\textsc{abs.sg}) & *ki-na & \citealt{FortescueJacobsonKaplan2010}\\
?\ilit{Indo-European} (\textsc{nom.sg}) & *k\textsuperscript{w}í-s {\textasciitilde} *k\textsuperscript{w}ó-s & \citealt{MalloryAdams2006}\\
?\ilit{Itelmen} & k’e < ?*mikæ & \citealt{Fortescue2005}\\
\ilit{Mongolic} & *ke-n & \citealt{Janhunen2003a}\\
\ilit{Samoyedic} & *kim(ɜ) {\textasciitilde} *k{i̮}mä & \citealt{Janhunen1977}\\
\ilit{Turkic} & *kim, \ilit{Oghur} *käm & \citealt{Schönig1999}\\
\ilit{Uralic} & *ki {\textasciitilde} *ke & \citealt{Nikolaeva2006}\\
\ilit{Yukaghiric} & *kin & \citealt{Nikolaeva2006}\\
\lspbottomrule
\end{tabularx}
\end{table}

In \textit{some} of the remaining instances the \isi{similarity} could in fact indicate long distance relationships. However, these limited data cannot, of course, proof any valid genetic unity as was assumed by \citet[217-224]{Greenberg2000}. Nevertheless, their \isi{similarity} as well as the fact that interrogatives meaning ‘who’ appear to be especially conservative, may suggest certain directions that deserve further investigation. The \isi{KIN-interrogative} seems especially promising due to its wide distribution. Related phenomena are often restricted to only a few language families, such as \ili{Turkic} *\textit{qay-} (e.g., \ili{Uyghur} \textit{qay-}, \ili{Khakas} \textit{xay{}-}) and \ili{Tungusic} *\textit{Kai} ‘what, which’ (e.g., \ili{Alchuka} \textit{kai-}, \ili{Nanai} \textit{xaɪ}). Such cases can often be more readily explained by \isi{language contact} or chance.

A phenomenon similar to the well-known \isi{m-T-pronouns} (e.g., \ili{Italian} \textit{mi}, \textit{ti}, \citealt{NicholsPeterson2013}) are so-called \isi{K-interrogatives}: more than two interrogatives in a given language start with a velar or uvular plosive or fricative (\figref{fig:6:15}), e.g. \ili{Nanai} \textbf{\textit{x}}\textit{aɪ} ‘what’, \textbf{\textit{x}}\textit{ado} ‘how many’, \textbf{\textit{x}}\textit{ooni} ‘how’, \ili{Uyghur} \textbf{\textit{q}}\textit{aysi} ‘which’, \textbf{\textit{q}}\textit{ačan} ‘when’, \textbf{\textit{q}}\textit{andaq} ‘how’ etc. The consonant must be identical in the different forms. Altogether 39 (ca. 47\%) out of 83 languages exhibit K-interrogatives, which speaks in favor of a very strong areal feature. While more research is necessary to establish their full geographical extent around the globe, at least parts of Eurasia share the phenomenon, e.g. \ili{Italian} \textit{chi} ‘who’, \textit{che} ‘what’, \textit{quale} ‘which’, all of which start with [k] (my knowledge), or \ili{Bengali} \textit{ke} ‘who’, \textit{ki} ‘what’, \textit{kon} ‘which’ \citep[202]{Thompson2012} (see \sectref{sec:5.5.3.1}). Of course, there are also languages outside of Eurasia with K-interrogatives, but a comprehensive treatment requires a large cross-linguistic sample. \citegen[217-224]{Greenberg2000} investigation of the alleged “\ili{Eurasiatic}” \isi{interrogative} starting with \textit{k-} overlaps with my notions of KIN- and K-interrogatives but is fundamentally different. My categories are first and foremost typological in nature and K-interrogatives are only accepted for a given language if at least three interrogatives share the same initial consonant. \ili{Middle Korean}, for example, which apparently has only one \isi{interrogative} starting with \textit{h-} does not fulfill this criterion and therefore has no \isi{K-interrogatives}. \citet{Greenberg2000}, on the other hand, merely assumes that the individual forms are all related to each other but does not follow any accepted methodology. \citet{Greenberg2000} furthermore does not clearly differentiate between interrogatives with different meanings but treats them as one category. However, as shown in \figref{fig:6:16}, personal interrogatives in 43 (52\%) out of 83 languages do not share the same initial consonant and thus should be kept separate. For instance, in \ili{Yukaghiric} and several \ili{Turkic} languages the personal \isi{interrogative} starts with \textit{k-}, but more peripheral interrogatives start with \textit{q-} instead. This phenomenon is not restricted to NEA, but can also be found in other languages, such as the \ili{Dravidian} language \ili{Kurux} \citep[91]{Kobayashi2017}. In this language, all interrogatives except \textit{neː} ‘who’ begin with an \textit{e{\textasciitilde}}. This result indicates that personal interrogatives (and the category \textsc{person} in general), have a very special position and most likely are more stable than most other interrogatives. \ili{Tungusic}, for example, has the interrogatives *\textit{ŋüi} ‘who’, *\textit{ja-} ‘what’, and a larger group with a \isi{resonance} *\textit{K{\textasciitilde}}. Given that the second and at least some of those interrogatives starting with *\textit{K{\textasciitilde}} have most likely a \ili{Mongolic} origin, the \isi{interrogative} *\textit{ŋüi} could represent an older layer of the \isi{interrogative} system. In general, the distribution of KIN- and \isi{K-interrogatives} overlaps with \isi{m-T-pronouns} and \isi{front rounded vowels}, which could indicate an old dispersal of languages in Eurasia that may have had its origin in southern \isi{NEA} \citep{Nichols2010}.

\subsection{Semantic scope}

Unfortunately, not much can be said about the \isi{semantic scope} of interrogatives in \isi{NEA}. Most descriptions are extremely vague about the exact meaning of interrogatives, which is why no absolute numbers can be given here, but relatively clear cases of polysemous forms have been collected in \tabref{tab:6:8}. Some polysemies are very frequent (e.g., \textsc{quantity mass = count}, \textsc{manner = reason}), while others are extremely rare. \textsc{person = thing} can only be found in \ili{Tocharian B} \textit{k\textsubscript{u}}\textit{se}, \textit{mäksu}, and perhaps \ili{Ainu} \textit{ne-} or \ili{Mongolic} \textit{*ke{}-}.

The \isi{semantic scope} of interrogatives gives a specific pattern for every language. \figref{fig:6:2} illustrates this with the help of several \ili{Mandarin} interrogatives. The largest category encompassing \textsc{thing}, \textsc{activity}, \textsc{reason}, and \textsc{time} is formed by \textit{shénme} ‘what’ and its derivations. In \ili{Mandarin} dialects there are some deviations from this pattern. For example, \textit{nǎ-} ‘which’, in the form \textit{nǎ-yi-ge} ‘which-one-\textsc{clf}’, has expanded its scope to include the category of \textsc{person} as well (\sectref{sec:5.9.2.1}).

\begin{table}
\caption{Polysemous interrogatives in \isi{NEA}}
\label{tab:6:8}

\begin{tabularx}{\textwidth}{XX}
\lsptoprule

\textbf{Scope} & \textbf{Example}\\
\midrule
\textsc{activity = reason} & \ilit{Mandarin} gàn shénme {\textasciitilde} gànmá\\
\textsc{kind = manner} & \ili{Mongolian} yamer\\
\textsc{manner = reason} & \ilit{Mandarin} zěnme\\
\textsc{selection = manner} & \ilit{Tuvan} kandɨg\\
\textsc{selection = person} & \ilit{Mandarin} dialects nǎ-(yi)-ge\\
\textsc{thing = reason} & \ilit{Sogdian} (ə)ču\\
\textsc{thing = selection} & \ilit{Manchu} ya\\
\textsc{place = direction} & \ilit{English} where\\
\textsc{person = thing} & \ilit{Tocharian B} k\textsubscript{u}se, mäksu\\
\textsc{quantity mass = count} & Kolyma \ilit{Yukaghir} qamun\\
\lspbottomrule
\end{tabularx}
\end{table}

As in this example, innovative \isi{interrogative systems} are usually based on the categories of \textsc{thing} and \textsc{selection} (e.g., \citealt{Cysouw2007}). In principle, the pattern can be given for every language for which sufficient information is available. For reasons of space, however, this cannot be accomplished here for all the languages of \isi{NEA}. However, as shown in Chapter 5, the \isi{conceptual space} clearly is able to capture the \isi{semantic scope} and diachrony of most interrogatives in \isi{NEA}.
\begin{figure}
% % % \includegraphics[width=\textwidth]{figures/fig_6_2.jpg}
\vspace*{\baselineskip}
\begin{tikzpicture}
  \path [use as bounding box] node[matrix, matrix of nodes, row sep=.5cm, column sep=.5cm, nodes={font=\scshape}] (HoelzlFig62)
    {
      activity & & reason & & \\
         & thing &  & kind & \\
      person & & manner & & time\\
      & selection & & quantity & \\
      & & place & & \\    
    };
    \draw[thick,-{Triangle[]}] (HoelzlFig62-1-1) -- (HoelzlFig62-1-3) node[near start,above,gray] {\itshape gàn shénme};
    \draw[thick,-{Triangle[]}] (HoelzlFig62-2-2) -- (HoelzlFig62-1-1);
    \draw[thick,-{Triangle[]}] (HoelzlFig62-2-2) -- (HoelzlFig62-1-3);
    \draw[thick,-{Triangle[]}] (HoelzlFig62-2-2) -- (HoelzlFig62-2-4) node[near end,above,gray] {\itshape zěnme};
    \draw[thick,-{Triangle[]}] (HoelzlFig62-2-2) -- (HoelzlFig62-3-3);
    \draw[thick] (HoelzlFig62-3-1.east) -- (HoelzlFig62-2-2.south) -- (HoelzlFig62-4-2.north) -- (HoelzlFig62-3-1.east);
    \draw[thick,-{Triangle[]}] (HoelzlFig62-3-3) -- (HoelzlFig62-1-3);
    \draw[thick,-{Triangle[]}] (HoelzlFig62-3-3) -- (HoelzlFig62-3-5);
    \draw[thick,-{Triangle[]}] (HoelzlFig62-3-3) -- (HoelzlFig62-2-4);
    \draw[thick,-{Triangle[]}] (HoelzlFig62-3-3) -- (HoelzlFig62-4-4);
    \draw[thick,{Triangle[]}-{Triangle[]}] (HoelzlFig62-4-2) -- (HoelzlFig62-5-3);
    \draw[thick,-{Triangle[]}] (HoelzlFig62-4-2) -- (HoelzlFig62-4-4);
    \draw[thick,-{Triangle[]}] (HoelzlFig62-4-2) -- (HoelzlFig62-3-3);
    \draw[thick,-{Triangle[]}] (HoelzlFig62-4-4) -- (HoelzlFig62-3-5);
    \draw[thick,-{Triangle[]}] (HoelzlFig62-5-3) -- (HoelzlFig62-3-3);
    \draw[thick,-{Triangle[]}] (HoelzlFig62-5-3) -- (HoelzlFig62-4-4);
    \draw[thick,-{Triangle[]}] (HoelzlFig62-2-2) to [bend left=130]   node [near end, auto, gray] {\itshape shénme shíhòu} (HoelzlFig62-3-5);
    \draw[thick,-{Triangle[]}] (HoelzlFig62-4-2) to [bend right=130] node [near end, auto, swap, gray] {\textit{duōshǎo}, \textit{jĭ} + \textsc{clf}} (HoelzlFig62-3-5);
    \node[left=3ex of HoelzlFig62-2-2.base west, anchor=base,gray] {\itshape shénme};
    \node[below=\baselineskip of HoelzlFig62-3-1.base, anchor=base,gray] {\itshape shéi};
    \node[left=10ex of HoelzlFig62-5-3.base east, anchor=base ,gray] {\itshape nǎ-li};
    \node[above=\baselineskip of HoelzlFig62-2-4.base,anchor=base, gray, align=left] {\itshape zěnme yàng de};
    \node[draw, gray, rounded corners, fit = (HoelzlFig62-4-2) (HoelzlFig62-5-3), label={[gray]below left:\textit{nǎ-} + \textsc{clf}}] {};
    \node[draw, gray, rounded corners, fit = (HoelzlFig62-1-3) (HoelzlFig62-3-3) (HoelzlFig62-2-4), label={[gray]above:\textit{wèi shénme}}] {};
\end{tikzpicture}
    \vspace*{\baselineskip}
\caption{Semantic scope of several Mandarin interrogatives}
\label{fig:6:2}
\end{figure}

The \isi{semantic scope} of locative interrogatives can be shown with an additional \isi{conceptual space}. \figref{fig:6:3} illustrates this with the help of \ili{Mandarin} data. In \ili{Mandarin}, all three categories are marked with \textit{nǎ-li} or its variants, which is, depending on the construction, combined with verbs or prepositions that derive from verbs. However, there are other systems with either identical forms for more than one category and systems with synchronically \isi{opaque} formations. Some languages for which sufficient information was available are compared in \tabref{tab:6:9}. The individual forms may either be related to each other or not (e.g., \ili{Ukrainian}). There are several interrogatives that have a scope covering two of the categories (e.g., \ili{Japanese}, \ili{English}, \ili{Manchu}, \ili{Mandarin}). \textsc{location} appears to have a \isi{tendency} to be the unmarked member of the group and often serves as a basis for derivations (e.g., \ili{Buryat}, \ili{English}, \ili{German}, \ili{Japanese}, \ili{Mandarin}).

\begin{figure}
% % \includegraphics[width=\textwidth]{figures/fig_6_3.jpg}
 \begin{tikzpicture}
  \node[regular polygon, regular polygon sides=3, minimum size=3cm,draw] (polygon3) {};
  \node[shift=(polygon3.corner 1),above] (a) {\scshape location}; \node[above=\baselineskip of a.base,anchor=base, gray] {\itshape (zài) nǎ-li}; 
  \node[shift=(polygon3.corner 3),below right] (b) {\scshape source}; \node[below=\baselineskip of b.base, anchor=base, gray] {\itshape cóng nǎ-li};
  \node[shift=(polygon3.corner 2),below left] (c) {\scshape direction}; \node[below=\baselineskip of c.base, anchor=base,,gray]  {\itshape (wǎng) nǎ-li};
 \end{tikzpicture}
\caption{Semantic scope of simplified Mandarin locative interrogatives}
\label{fig:6:3}
\end{figure}

\begin{table}
\caption{Some examples for the semantic scope in the category \textsc{place}. Only a selection of forms and languages is listed}
\label{tab:6:9}

\begin{tabularx}{\textwidth}{XXXl}
\lsptoprule

\textbf{Language} & \textbf{\textsc{location}} & \textbf{\textsc{direction}} & \textbf{\textsc{source}}\\
\midrule
\ilit{Amdo Tibetan} & kaŋ-na & kaŋ-a & kaŋ-ni\\
\ilit{Buryat} & xaa-(na) & xai-sha & xaana-haa\\
\ilit{Chukchi} & miŋ-ke & miŋ-kəri & m\textbf{e}ŋ-qo(rə)\\
CS \ilit{Yupik} & na-ni & na-vek & na-ken\\
\ilit{English} & where & where (to) & where from\\
\ilit{Evenki} & ii-du & ii-le & ii-duk\\
\ilit{German} & wo & wo-hin & wo-her\\
\ilit{Japanese} & doko (ni) & doko e/ni & doko kara\\
\ilit{Ket} & bísȅŋ & bíltàn & bílȉl\\
\ilit{Khakas} & xay-da & xay-ɣa & xay-daŋ\\
\ilit{Kolyemal} & ɔdɨ-mæ & ɔdi-ř & ɔdi-sɔ\\
Kolyma \ilit{Yukaghir} & qo-n & qa-ŋide & qo-t\\
\ilit{Manchu} & ai-ba-de & ai-ba-de & ai-ba-ci\\
\ilit{Mandarin} & (zài) nǎ-li & (wǎng) nǎ-li & cóng nǎ-li\\
Tundra \ilit{Nenets} & xə-n’a-na & xə-n’a-h & xə-n’a-d°\\
\ilit{Ukrainian} & de & kudý & zvídky\\
\lspbottomrule
\end{tabularx}
\end{table}

Many languages in \isi{NEA} distinguish the three different categories by means of \isi{case} marking or adpositions (e.g., \ili{Mandarin}, \ili{Khakas}, \ili{Evenki}, \ili{Nenets}, \ili{Kolyemal}, \ili{Amdo Tibetan}, Central Siberian \ili{Yupik}). However, in some instances not all three forms are based on the same stem (e.g., \ili{Buryat}, Kolyma \ili{Yukaghir}).

Information from grammar books is usually insufficient to decide about the \isi{semantic scope} of interrogatives expressing \textsc{quantity (mass}{}---\textsc{count)}. Nevertheless, some clear examples can be given in order to illustrate possible patterns (\tabref{tab:6:10}).

\begin{table}
\caption{Some examples for the semantic scope in the category \textsc{quantity}}
\label{tab:6:10}

\begin{tabularx}{\textwidth}{XXl}
\lsptoprule

\textbf{Language} & \textbf{\textsc{mass}} & \textbf{\textsc{count}}\\
\midrule
\ilit{English} & how much & how many\\
\ilit{German} & wie viel & wie viel-e etc.\\
Kolyma \ilit{Yukaghir} & \textstyleStrong{{qamun}} & \textstyleStrong{{qamun}}\\
\ilit{Mandarin} & du\=oshǎo & du\=oshǎo, j\u{\i}- + \textsc{clf}\\
\ili{Mongolian} & xedii & xedn\\
\lspbottomrule
\end{tabularx}
\end{table}

Some languages have only one (e.g., Kolyma \ili{Yukaghir}), others have two different forms (e.g., \ili{English}, \ili{Mandarin}, \ili{Mongolian}). If there are two different forms, these may either be related etymologically (e.g., \ili{English}, \ili{Mongolian}) or can have a completely different origin (e.g., \ili{Mandarin}). In some cases \textsc{count} is derived from \textsc{mass} (e.g., \ili{German}), which appears to be the unmarked category. In some other cases the semantic scopes of individual forms overlap (e.g., \ili{Mandarin}). The use of any of the forms is usually based on subtle differences and the boundary between mass and count nouns is language-specific. In principle, the distribution of different types, such as \textsc{mass=count} vs. \textsc{mass${\neq}$count}, or \textsc{selection=place} vs. \textsc{selection${\neq}$place} could be shown on geographical maps, but the information for most languages was simply insufficient.

\subsection{Diachrony of interrogatives}

\sectref{sec:4.3} identified seven possible \isi{diachronic} developments of interrogatives. Of these, the convergence of forms is apparently only attested in the northern \ili{Tungusic} languages \ili{Oroqen} and Khamnigan \ili{Evenki}, where the two \ili{Proto-Tungusic} interrogatives *\textit{ja} ‘which, what’ *\textit{Kai} ‘what’ coalesced in a form \textit{i(i)-} (\sectref{sec:5.10.3}). The replacement of interrogatives as in \ili{Italian} \textit{che} > \textit{che cosa} > \textit{cosa} ‘what’ or the development of interrogatives from skratch do not appear to be very widespread. However, some examples can perhaps be found in Tocharian, e.g. PIE *\textit{k\textsuperscript{w}}\textit{i-} ‘\textsc{int}’ + *\textit{so} ‘\textsc{dem}’ > Proto-Tocharian *\textit{k\textsuperscript{w}}\textit{əsë} > \textit{k\textsubscript{u}}\textit{se > se} ‘who’ (\sectref{sec:5.5.3.5}). The remaining four developments (repeated here in \tabref{tab:6:11}) are more frequent.

\begin{table}
\caption{The most important diachronic developments of interrogatives}
\label{tab:6:11}

\begin{tabularx}{\textwidth}{llQl}
\lsptoprule
& \textbf{Schematic} & \textbf{Details} & \textbf{Example}\\
\midrule
1 & INT\textsubscript{1} > INT\textsubscript{1} & phonological changes & PIE *kwód > OE hwæt > NE what\\
2 & INT\textsubscript{1} > INT\textsubscript{2} & semantic changes & \ili{Wutun} age ‘which (one) > who’\\
3 & INT\textsubscript{1}{}-X\textsubscript{GRAM} > INT\textsubscript{2} & \isi{inflection} (> fusion) & \ili{English} where to\\
4 & INT\textsubscript{1}{}-X\textsubscript{LEX} > INT\textsubscript{2} & derivation, \isi{reinforcement} (> fusion) & \ili{English} how much\\
\lspbottomrule
\end{tabularx}
\end{table}

         (1) Most languages have a large number of inherited interrogatives. An exception is \ili{Mandarin}, which apparently preserves only the two \ili{Old Chinese} interrogatives \textit{shéi} (\textit{shuí}) \zh{谁} and \textit{j\u{\i}}\textit{{}-} \zh{几} (\sectref{sec:5.9.3.1}). For details of individual language families, the reader is referred to Chapter 5. The loss of the \isi{resonance} due to phonological changes, which may lead to a very different \isi{interrogative systems}, is only attested for \ili{Tungusic} languages (\sectref{sec:5.10.3}).

         (2) Semantic changes appear to be quite infrequent, but they are often difficult to detect because of the lack of data. Some relatively clear examples have been collected in \tabref{tab:6:12}, which includes only those cases that do not also involve \isi{inflection} or derivation. For example, \ili{Mandarin} \textit{gàn shénme} can mean both ‘to do what’ and ‘why’, without requiring any additional marking.

\begin{table}
\caption{Changes in the semantic scope of interrogatives with some examples}
\label{tab:6:12}

\begin{tabularx}{\textwidth}{lllQ}
\lsptoprule

\textbf{Source} &  & \textbf{Target} & \textbf{Examples}\\
\midrule
\textsc{activity} & → & \textsc{reason} & \ilit{Mandarin} gàn shénme ‘to do what, why’\\
\textsc{manner} & → & \textsc{reason} & \ilit{Mandarin} zěnme ‘how, why’\\
\textsc{place} & → & \textsc{manner} & \ilit{Manchu} absi ‘whither, how’\\
& → & \textsc{reason} & \ilit{German} woher ‘whence, why’\\
\textsc{selection} & → & \textsc{person} & \ilit{Solon} awu\\
\textsc{thing} & → & \textsc{reason} & ?Chitose \ilit{Ainu} hemanta ‘what, why’\\
\lspbottomrule
\end{tabularx}
\end{table}

There are too many instances of \isi{inflection} (3) or derivation (4), which is why \tabref{tab:6:13} lists only general patterns illustrated with some examples. Over time many derived interrogatives fuse, are subject to phonetic erosion, and become unanalyzable (e.g., MHG \textit{w\=ar + umbe} ‘where + around’ > \ili{German} \textit{warum} ‘why’).

\begin{table}
\caption{Some possibilities of inflection and derivation of interrogatives}
\label{tab:6:13}

\begin{tabularx}{\textwidth}{lQ}
\lsptoprule

\textbf{Type} & \textbf{Examples}\\
\midrule
\textsc{int} + \textsc{adj} & \ilit{Japanese} do-no ‘which’ etc.\\
\textsc{int} + \textsc{adp} & \ilit{English} where to, \ilit{Mandarin} wǎng nǎli ‘id’ etc.\\
\textsc{int} + \textsc{case} & \ilit{Evenki} ii-le ‘whither’ etc.\\
\textsc{int} + \textsc{clf} & \ilit{Mandarin} nǎ(-yi)-ge, j\u{\i}-ge etc.\\
\textsc{int} + \textsc{cvb} & \ilit{Manchu} ai.na-me ‘why’, \ilit{Khorchin} jʊʊ gə-ǰ, \ili{Mongolian} yaa-j ‘how’, yaa-gaad ‘why’ etc.\\
\textsc{int} + \textsc{dem} & Tocharian *k\textsuperscript{w}əsë, \ilit{Slavic} *k\textcyrillic{ъ}to, Lingshi \ilit{Mandarin} uɛ\textsuperscript{44}ȿu\textsuperscript{44} ‘who’\\
\textsc{int} + \isi{gender} & \ilit{Ket} bìseŋ-du ‘where-\textsc{3sg.m}’ etc., \ilit{German} welch- ‘which’, \ilit{Ukrainian} kotorýj ‘which’ etc.\\
\textsc{int} + \textsc{n} & \ilit{English} what kind, \ilit{Mandarin} zěnme yàng(de) ‘id’ etc.\\
\textsc{int} + number & \ilit{Aleut} kiin ‘who (\textsc{sg})’, kiin-kux ‘who (\textsc{du})’, Eastern kiin-kun, Atkan kiin-kus ‘who (\textsc{pl})’ etc.\\
\textsc{int +} numeral & \ilit{English} which one, Written \ilit{Tibetan} ci.cig, \ilit{Mandarin} nǎ-yi-\\
\textsc{int} + person/number & \ilit{Ket} bìseŋ-di ‘where-\textsc{1sg}’ etc., \ilit{Nivkh} ja(ŋ).gu- ‘how’, \ilit{German} welch- ‘which’, \ilit{Ukrainian} kotorýj ‘which’ etc.\\
\textsc{int} + \textsc{q} & \ilit{Korean} nwu{}-ku ‘who’, \ilit{Tocharian B} t\=aśśi < t\=a + aśśi ‘where’, \isi{Amur} \ilit{Nivkh} aŋ ?< nar-ŋa ‘who’\\
\textsc{int} + \textsc{v} & \ilit{English} to do what, \ilit{Manchu} ai-na- ‘id’ etc.\\
\lspbottomrule
\end{tabularx}
\end{table}

\clearpage 
\subsection{Borrowing}

\tabref{tab:6:14} gives a list of possibly borrowed interrogatives in \isi{NEA}. Several instances that are marked with a question mark remain somewhat unclear. \ili{Turkic} *\textit{ne} ‘what’, the only autochthonous word starting with an \textit{n-}, is problematic because neither genetic inheritance, nor \isi{borrowing} appear to be plausible explanations for this anomaly, which deserves further research.

\begin{table}
\caption{Possible instances of borrowing of interrogatives in \isi{NEA}}
\label{tab:6:14}

\begin{tabularx}{\textwidth}{QlQ}
\lsptoprule

\textbf{Source} &  & \textbf{Target}\\
\midrule
\ilit{Mongolic} *kedü- ‘how many’, *keli ‘when’, *yaxun {\textasciitilde} *ya- ‘what’ & ?→ & \ilit{Tungusic} *Kadü, *Kaali, *ja(-kun)\\
\ilit{Dagur} joonde ‘why’ & → & Nanmu \ilit{Oroqen} joonde\\
\ilit{Mongolic} *ya- via \ilit{Tungusic} *ja- ‘(to do) what’ & ?→ & \ilit{Nivkh} ja-\\
\isit{Amur} \ilit{Nivkh} \c{r}ag ‘where’, East \isi{Sakhalin} \ilit{Nivkh} nunt/nud ‘what’ & ?→ & \ilit{Uilta} saa ‘where’, sado ‘where’,

nuulu ‘whither’\\
\ilit{Turkic}, e.g. \ilit{Tuvan} kay(ɨ) ‘which’, kažan ‘when’ & → & \ilit{Selkup} qaj, ?quʒan\\
\ilit{Solon} awu ‘who’ & → & Nanmu \ilit{Oroqen} awu\\
\ilit{Udegheic}, e.g. \ilit{Udihe} ni ‘who’, adi, ‘how’, je’-(u), je-me ‘what kind’, \ilit{Oroch} oni ‘how’, \ilit{Udihe} ono-bui ‘which’ & → & \ilit{Kilen} ni, adi, ja, yao, iama-,  oni, ?ɔni{}-biɕi\\
\ilit{Ewenic}, e.g. \ilit{Evenki} ŋi ‘who’, ady ‘how much’,  i-du ‘where’, ee-da ‘why’, \ilit{Oroqen} j\textsc{ee}ma ‘which’, aali ‘when’ & → & \ilit{Kili} ŋii, adii, i-du, ii-daj, e-ma, ali\\
\ilit{Ewenic}, e.g. \ilit{Oroqen} oki ‘how much’ & → & \ilit{Kilen} uki\\
\ilit{Jurchenic}, e.g. \ilit{Manchu} ya ‘which’, atanggi ‘when’ & → & \ilit{Kilen} iətin\\
Hezhou \ilit{Chinese} aʒi\textsuperscript{24}gə {\textasciitilde} aji\textsuperscript{24}gə ‘which one, who’ & → & \ilit{Mangghuer} ayige\\
\ilit{Mandarin} j\u{\i}diǎn ‘what o’clock’ & → & \ilit{Santa} dʑidʑiən(-də)\\
\ilit{Sinitic}, e.g. \ilit{Mandarin} mà ‘what’ & → & \ilit{Santa}, \ilit{Kangjia} ma-\\
\ilit{Mandarin} shénme ‘what’ & → & \ilit{Chinese} Pidgin \ilit{Russian} šýma\\
\ilit{Nganasan} sïlï ‘who’ & → & Taimyr Pidgin \ilit{Russian} syly\\
\ilit{Russian} kokda ‘when’, kakoj, kotoryj ‘which’, kuda ‘where’ & → & Copper Island \ilit{Aleut} ka(g)da, kakuy, katorəye, kuda\\
\ilit{Russian} kakoj ‘which’ & → & \ilit{Chalkan} qaqoy {\textasciitilde} kakoy\\
\ilit{Russian} kak ‘how’ & → & \ilit{Selkup} kak {\textasciitilde} kaŋ\\
\ilit{Iranian}, e.g., \ilit{Sogdian} (ə)ču ‘what’ & ?? & \ilit{Siberian Turkic}, e.g. \ilit{Chalkan} t’u(u) etc.\\
\ilit{Koreanic} *e- & ?? & \ilit{Japonic} *e-\\
\ilit{Turkic} *\textit{qay-}, e.g., \ilit{Uyghur} \textit{qay-}, \ilit{Khakas} \textit{xay{}-} ‘what, which’ & ?? & \ilit{Tungusic} *\textit{Kai}, e.g., \ilit{Alchuka} \textit{kai-}, \ilit{Nanai} \textit{xaɪ}\\
\lspbottomrule
\end{tabularx}
\end{table}

\section{The significance of the grammar of questions}\label{sec:6.3}

What has been called the \textit{\isi{grammar of questions}} in this study is of great significance from a number of different perspectives.

\begin{quote}
Questions are of interest not merely as interrogative sentences or techniques. They are instances of stimuli to which people respond and thus represent a matter of broad intellectual interest beyond grammatical and functional concerns. Questions entail cognitive and expressive processes, social relationships, and interactional discourse. They are also the device by which several enterprises of societal and individual significance characteristically proceed. Apart from any relation to \isi{response}, \isi{questions} alone are of further interest for their function in the thinking of those who ask them—for their motivation of children’s thought and scholars’ inquiry. \citep[162]{Dillon1982}
\end{quote}

  
 
\noindent For example, as seen in \sectref{sec:4.4}, the internal structure of the \isi{grammar of questions} allows some conclusions about the underlying cognitive structure. The frequent \isi{combination} of \isi{content question}s with polar, \isi{focus}, or \isi{alternative question}s allows an inference on the underlying cognitive process that seems to proceed from the schematic to the specific.

\ea%1 
\ili{Abui} (\ili{Timor-Alor-Pantar})\\
    \label{ex:key:1}
    \gll {moku} {kiang} {nu} {he-n-u} \textbf{{nala}}, {moku} {neng} \textbf{{r}}\textbf{{e}} {mayol?}\\
     kid  baby  this  3O.\textsc{loc}{}-be.like.this{}-\textsc{pfv}  what  kid  man  or  woman \\
    \glt ‘What is the baby, a boy or a girl?’ (\citealt{Kratochvíl2007}: 175)
    \z

\noindent As shown in \sectref{sec:4.4}, this pattern can be found in languages around the world. This observation of a recurrent pattern in languages that are unrelated and lack mutual influence suggests a general \isi{tendency}. In fact, the pattern is in accordance with a hypothesis proposed by \citet[1235]{Bar2009}

\begin{quote}
that the human brain is proactive in that it continuously generates predictions that anticipate the relevant future. In this proposal, analogies are derived from elementary information that is extracted rapidly from the input, to link that input with the representations that exist in memory. Finding an analogical link results in the generation of focused predictions via associative activation of representations that are relevant to this analogy, in the given context.
\end{quote}

\noindent Consider the following example from a language spoken in Eastern Sulawesi.

\ea%2
    \ili{Balantak} (Celebic, \ili{Austronesian})\\
    \label{ex:key:2}
    \gll \textbf{{i}}\textbf{{me}} {a} {men} {mae',} {yaku'} \textbf{{kabai}} {i} {koo?}\\
        who  \textsc{art}  \textsc{rel}  go  1\textsc{sg}  or  \textsc{pers.art}  2\textsc{sg}\\
    \glt ‘Who will go, you or I?’ (\citealt{vandenBergBusenitz2012}: 66)
    \z

\noindent The context of the utterance is difficult to reconstruct. But the \isi{content question} contains an \isi{interrogative} that represents an initial categorization of a given referent (in this case \textsc{person}). The following \isi{alternative question} represents possible predictions concerning the identity of that referent. The choice of the \isi{interrogative} thus also offers direct evidence for the most basic categorization and organization of our knowledge.

The Introduction has claimed that the \isi{grammar of questions} can function as yardstick for measuring the intensity of the intensity of \isi{language contact}, areal convergence, unusually strong \isi{language contact}, and \isi{simplification}. The remainder of this section briefly evaluates these claims and argues that in many cases they give valuable and good results. On the identification of long-range relationships see \sectref{sec:6.2.1}.

Regarding the \isi{Amdo Sprachbund}, for example, \citet[6]{Slater2003a} observed the following:

\begin{quote}
It certainly is true that intense two-\isi{language contact} situations have resulted in many instances of localized contact-induced \isi{language change}, and I do not mean to suggest that two-language comparisons should not be made in the Qinghai-\isi{Gansu} region. However, what has often been lacking is an overview of the regional processes of linguistic feature diffusion.
\end{quote}

\noindent In fact, as seen in \sectref{sec:3.5}, many features mentioned by \cite[180ff.]{Janhunen2012a} such as SOV \isi{word order} fail to define the region as a \isi{linguistic area} because they are too frequent worldwide and in adjacent regions. However, the investigation of the \isi{grammar of questions} has potentially revealed two features that could help define the \isi{Amdo Sprachbund}. \citet[384]{Sandman2012}, by comparing two languages, came to the following reasonable conclusion.

\begin{quote}
In \ili{Bonan}, the most common \isi{interrogative} marker is -\textit{u}. The \isi{interrogative} marker {}-\textit{mu} is formed by attaching the \isi{interrogative} marker -\textit{u} to the narrative aspect marker -\textit{m}. The narrative aspect marker indicates stative or habitual aspect in \ili{Bonan}. The \isi{borrowing} of the \isi{interrogative} marker -\textit{mu} is another example of grammatical \isi{borrowing} from \ili{Bonan} to \ili{Wutun}.
\end{quote}

By just looking at these two languages the conclusion is, of course, very plausible because the marker has a clear etymology in \ili{Bonan} but not in \ili{Wutun}. However, there is another possibility that treats the \ili{Wutun} \isi{question marker} as a loan from \ili{Turkic}, e.g. \ili{Uyghur} \textit{=mu}, \ili{Sarig Yughur} \textit{=mu}, or \ili{Salar} \textit{=mU}, perhaps via Hezhou \ili{Chinese} \textit{=mu} or \ili{Tangwang} \textit{=mu}. If this scenario is accurate, the markers in \ili{Wutun} and \ili{Bonan} are only similar by chance. In fact, the \isi{question marker} \textit{{}-mu} in \ili{Bonan} has parallels in other \ili{Mongolic} languages of the area such as \ili{Mongghul} \textit{{}-muu}, \ili{Santa} \textit{{}-mu}, and \ili{Kangjia} \textit{{}-mʉ}. However, even if one excludes the \ili{Mongolic} \isi{question marker}, the presence of a relatively widespread and specific \isi{question marker} \textit{=mu} in \ili{Turkic} and \ili{Sinitic} languages of the region that is absent in the surrounding area is certainly a better defining feature than SOV \isi{word order}. Another example is the presence of \isi{single marking} on the first alternative in \isi{alternative question}s (\figref{fig:6:10}) shared at least by \ili{Gangou} (not shown on the map), Hezhou, \ili{Wutun}, \ili{Santa}, \ili{Bonan}, \ili{Kangjia}, and \ili{Mangghuer}. This feature, again, can also be found in \ili{Uyghur} as well as Urumqi Hui \ili{Chinese}, but not in the surrounding languages in \isi{NEA}. Given the lack of information on alternative \isi{questions}, this feature might well be more widespread in the area. In fact, there is some indication that it can perhaps also be found in the immediate south of the \isi{Amdo Sprachbund} (see \sectref{sec:4.2.1}).

The \isi{Amdo} area, of course, is known to be a region of strong linguistic convergence and even creolization, but \isi{question marking} can also identify \isi{contact} situations that are otherwise hard to detect. It is well-known, for example, that there was \isi{contact} between \ili{Koreanic} and the \ili{Tungusic} language \ili{Manchu}. However, previous studies have been quite unsuccessful in identifying any conclusive linguistic evidence for this historical fact. \cite[224f.]{Vovin2013b} has collected a short but extremely valuable list of 17 \ili{Koreanic} items in \ili{Manchu}, some of which unfortunately are somewhat problematic. In my opinion, the \ili{Manchu} third person pronoun \textit{i}, for example, more likely derives from \ili{Mongolic} *\textit{i} \citep[18]{Janhunen2003a}, because it shares an identical oblique stem formation, e.g. \ili{Manchu} \textit{in-i} ‘3\textsc{sg.obl}{}-\textsc{gen}’, Proto-\ili{Mongolic} *\textit{in-U} > *\textit{in-i} ‘3\textsc{sg.obl}{}-\textsc{gen}’. Pronouns are not easily borrowed and perhaps only the \isi{contact} with \ili{Mongolic} was strong enough (e.g., \citealt{Doerfer1985}). At least some of his correspondences such as \ili{Manchu} \textit{fucihi} ‘Buddha’ (from \ili{Middle Korean} \textit{pwùthyè}) are very plausible. In fact, the form \textit{p‘ut(‘)ihi.n} in the language \ili{Bala} makes this even more likely (\citealt{MuYejun1987}), but cultural loanwords such as this are not necessarily a sign of direct \isi{language contact}. This study has identified a whole list of question markers in \ili{Manchu} (and \ili{Jurchenic}) that appear to systematically derive from a \ili{Koreanic} source. Not only are the \ili{Manchu} question markers very different in form and \isi{semantic scope} from the rest of \ili{Tungusic}, but the forms are strikingly similar to \ili{Koreanic} (see \sectref{sec:5.7.2}, \sectref{sec:5.10.2}). Such markers can only have been borrowed through direct \isi{interaction} of the speakers of these languages.

\ili{Manchu} is perhaps the most aberrant \ili{Tungusic} language and I have previously put forward the possibility that it might even be comparable to languages such as Afrikaans (\citealt{Hölzl2012}; \citeyear[151]{Hölzl2015a}). In fact, not only the \isi{question marking} system, but also the \isi{interrogative} system is rather different from other \ili{Tungusic} languages (\sectref{sec:5.10.3}). While \ili{Manchu} preserves some \ili{Tungusic} interrogatives, there is a large amount of innovative forms that are based on the two stems \textit{ai} ‘what’ and \textit{ya} ‘which’, which speaks in favor of a certain amount of \isi{simplification} (\tabref{tab:6:16}) due to massive non-native \isi{acquisition} in the history of \ili{Manchu} (\citealt{McWhorter2007}; \citealt{Operstein2015}). Of course, this theory should actually include all of \ili{Jurchenic}.

Another striking example is \ili{Mandarin}, which is also known to have experienced a certain amount of \isi{simplification} with respect to Old or Middle \ili{Chinese} and other \ili{Sinitic} languages (\citealt{McWhorter2007}: 104-137) and contains a large amount of analyzable interrogatives as well (\sectref{sec:5.9.3.1}). Notice that this is qualitatively different from the \isi{contact} between \ili{Manchu} and \ili{Koreanic}, which lead to \isi{complexification} instead (\tabref{tab:6:16}). \ili{Manchu} actually exhibits more question markers than other \ili{Tungusic} languages and this must be due to influence from \ili{Koreanic}. Unlike other \ili{Tungusic} languages, but similar to \ili{Korean}, \ili{Manchu} also employs the question markers in \isi{content question}s, which from a certain perspective could be interpreted as a type of redundancy. This must be the result of a different \isi{language contact} scenario that involves longstanding \isi{contact} and perhaps some bi- or multilingualism. See \cite{Hölzl2017a} for an additional discussion of \isi{simplification} and \isi{complexification} of \ili{Tungusic} \isi{interrogative systems}.

\begin{table}
\caption{Complexification and simplification \citep[62]{Trudgill2011}}
\label{tab:6:16}

\begin{tabularx}{\textwidth}{XX}
\lsptoprule

\textbf{Complexification} & \textbf{Simplification}\\
\midrule
irregularization & regularization of irregularities\\
increase in opacity & increase in morphological \isi{transparency}\\
increase in syntagmatic redundancy & \isi{reduction} in syntagmatic redundancy\\
addition of morphosyntactic categories & loss of morphological categories\\
\lspbottomrule
\end{tabularx}
\end{table}

\ili{Tungusic} also offers a good example for yet another type of \isi{language contact} that leads to the mixing of languages. The language \ili{Kilen}, for example, is well-known to be a mixed \ili{Tungusic} language and has been sometimes classified with \ili{Nanai} (e.g., \citealt{Alonso2011}; \citealt{Janhunen2012b}; this study) and sometimes with \ili{Udihe} \citep{Kazama2003}. Influence from \ili{Manchu} has often been overlooked, however (see \citealt{Hölzl2017b}). In fact, \ili{Kilen} exhibits interrogatives that can clearly be shown to derive from \ili{Nanai}, \ili{Udihe}, and \ili{Manchu}, which represent three different branches of the \isi{language family}. Consider the following example.

\ea%3
    \label{ex:key:3}
    \ili{Kilen} (\ili{Tungusic})\\
    \gll\textbf{{ni}} \textbf{{jaɾin}} \textbf{{ja}}{-tulə}    \textbf{ənə}-kiɕiə?\\
      who  when    where-\textsc{all}  go-?\textsc{subj}  \\
    \glt  ‘Who would like to go to where when?’ (\citealt{ZhangPaiyu2013}: 163)
    \z

\noindent The verb most likely derives from \ili{Nanaic}, but every \isi{interrogative} must stem from other branches of the \isi{language family} (\tabref{tab:6:17}). Such a mixed language can only be the result of multilingualism throughout the entire speech community: “Unlike \isi{creoles}, \isi{mixed languages} arise in bilingual settings in which the speakers are equally fluent in the two codes.” \citep[6]{Operstein2015}

\begin{table}
\caption{The etymological brackground of the Kilen elements in \REF{ex:key:3}}
\label{tab:6:17}

\begin{tabularx}{\textwidth}{lXXXXX}
\lsptoprule

\textbf{Language} & \textbf{who} & \textbf{when} & \textbf{whither} & \textbf{\textsc{all}} & \textbf{to go}\\
\midrule
\ilit{Manchu} (\ilit{Jurchenic}) & we & \textbf{ya erin} & \textbf{ya}{}-de & {}-de & gene-\\
\ilit{Nanai} (\ilit{Nanaic}) & ui & xaali & xaosi & {}-dola & \textbf{ənə-}\\
\ilit{Udihe} (\ilit{Udegheic}) & \textbf{ni} & ali & \textbf{j’e}{}-uxi & {}-d\textbf{u}lA & ŋene-\\
\ilit{Evenki} (\ilit{Ewenic}) & ŋi & ookin & i-le & {}-d\textbf{u}lA/-\textbf{tu}lA & ŋene-\\
\lspbottomrule
\end{tabularx}
\end{table}

Another example seen in \isi{NEA} is Copper Island \ili{Aleut}, which exhibits \ili{Russian} and \ili{Aleut} interrogatives (\sectref{sec:5.4.3}). Creoles are both mixed and exhibit “extreme \isi{simplification} on all levels” (\citealt{McWhorter2007}: 254) due to non-native \isi{acquisition}. In principle, they should exhibit both a simplified or \isi{transparent} \isi{interrogative} system as well as interrogatives from different sources and this indeed seems to be the case for at least some of them (\citealt{Bickerton2016}: 65f.; \citealt{MuyskenSmith1990}). There are no true creole languages in \isi{NEA}, but Taimyr Pidgin and \ili{Chinese} Pidgin \ili{Russian} had interrogatives of \ili{Russian} and dialectal \ili{Russian} origin as well as at least one from \ili{Nganasan} and \ili{Chinese}, respectively (\sectref{sec:5.5.3.3}). Taimyr Pidgin furthermore had at least some innovative interrogatives such as \textit{kudy-mera} ‘where’, \textit{kudy-mesto} ‘where’, and \textit{kakoj storona} ‘whither’.

Another type of change we see under creolization is the translation of individual forms such as \ili{Chinese} Pidgin \ili{Russian} \textit{mnogo-malo}, which consists of \ili{Russian} \textit{mnógo}/\textcyrillic{много} ‘much’, \textit{málo}/\textcyrillic{мало} ‘little’ and is a direct translation of \ili{Chinese} \textit{du\=o}\textit{shǎo} \zh{多少} ‘how much’. Calques are not necessarily restricted to creole and pidgin languages, however, but can also be found in instances of bilingual \isi{contact}. Most cases found in \isi{NEA} are partial borrowings and contain an autochthonous \isi{interrogative}, e.g. \ili{Qiang} \textit{ȵa-tian} from \ili{Chinese} \textit{j\u{\i} diǎn} \zh{几点} ‘what hour’ (\citealt{LaPollaHuang2003}: 53f.). A mixture of calque and \isi{borrowing} can also be found in \ili{Santa} \textit{yan shihou} from \ili{Mandarin} \textit{shénme} \textit{shíhou} ‘what time’. Special cases of \isi{borrowing} are, furthermore, \textit{iamə-dʑaka} ‘what thing’ and perhaps \textit{iama-ərin} ‘what time’ in the \ili{Tungusic} language \ili{Kilen}, which derive from two different sources. The actual \isi{interrogative} has been borrowed from \ili{Udihe} \textit{je-me} ‘what kind’, while the second elements derive from \ili{Manchu} \textit{jaka} ‘thing’ and perhaps \textit{erin} ‘time’, both of which are also present in \ili{Manchu} interrogatives. In some cases an \isi{interrogative} has been entirely translated. \ili{Manchu} \textit{ai se-me} and \ili{Khorchin} \ili{Mongolian} \textit{jʊʊ gə-ǰ}, for example, have the same underlying pattern ‘what say-\textsc{cvb.ipfv}’ and both mean ‘why’. \ili{Khalkha} \textit{xer olon} or \ili{Ket} \textit{bìlon} appear to have been formed on the basis of a European pattern also seen in \ili{English} \textit{how many/much}.

These examples illustrate that the \isi{grammar of questions} can indeed function as a preliminary but valuable tool for the identification of different types of \isi{language contact}, but this section has focused on individual instances of diffusion or convergence, exclusively. The following maps of the atlas allow an additional identification of large patterns of areal convergence that is impossible from the study of individual languages alone.


\section{An atlas of the grammar of questions in Northeast Asia}\label{sec:6.4}

The geographical extent of certain features will be demonstrated with the help of a \isi{synchronic} sample of 83 languages (\figref{fig:6:4}, \tabref{tab:6:15}) that covers all 14 language families of \isi{NEA}. Languages with a wide geographical distribution are underlined in \figref{fig:6:4} and shown with bigger symbols in \figref{fig:6:5}-\figref{fig:6:16} below. The maps exclude extinct languages and list only some dialects of a given language. An exception is made for \ili{Ainuic}, which by now is probably completely extinct but has been added for reasons of completeness.

\begin{table}
\caption{The synchronic sample of 83 languages used for the maps}
\label{tab:6:15}
\small
\begin{tabularx}{\textwidth}{XXr}
\lsptoprule

\textbf{Family} & \textbf{Language} & \textbf{Number}\\
\midrule
\textbf{Ainuic} & Chitose \ilit{Ainu} & 1\\
& Saru \ilit{Ainu} & 2\\
& Shizunai \ilit{Ainu} & 3\\
\textbf{Amuric} & \isit{Amur} \ilit{Nivkh} & 4\\
& East \isi{Sakhalin} \ilit{Nivkh} & 5\\
\textbf{Chukotko-Kanchatkan} & \ilit{Chukchi} & 6\\
& \ilit{Alutor} & 7\\
& \ilit{Koryak} & 8\\
& \ilit{Itelmen} & 9\\
\textbf{\ilit{Eskaleut}} & Central Siberian \ilit{Yupik} & 10\\
& \ilit{Aleut}, Atkan & 11\\
& Copper Island \ilit{Aleut} & 12\\
\textbf{\ilit{Indo-European}} & \ilit{Russian} & 13\\
& \ilit{Ukrainian} & 14\\
& \ilit{Plautdiitsch} & 15\\
& \ilit{Yiddish} & 16\\
& \ilit{Sarikoli} & 17\\
\textbf{\ilit{Japonic}} & \ilit{Japanese} & 18\\
& \ilit{Hachij\=o} & 19\\
& \ilit{Yilan Creole} & 20\\
& \ilit{Yuwan} & 21\\
& \ilit{Okinoerabu}, Masana & 22\\
& \ilit{Shuri} & 23\\
& \ilit{Ikema} & 24\\
& \ilit{Ōgami} & 25\\
& \ilit{Irabu} & 26\\
& \ilit{Hateruma} & 27\\
& \ilit{Hatoma} & 28\\
& \ilit{Miyara} & 29\\
& \ilit{Sonai} & 30\\
\textbf{Koreanic} & \ilit{Korean} & 31\\
& \ilit{Jeju} & 32\\
\textbf{Mongolic} & \ilit{Dagur} & 33\\
& \ilit{Buryat} & 34\\
& \ilit{Khamnigan Mongol} & 35\\
& Cyrillic \ilit{Khalkha} & 36\\
& \ilit{Oirat} & 37\\
& \ilit{Shira Yughur} & 38\\
& \ilit{Santa} & 39\\
& \ilit{Bonan} & 40\\
& \ilit{Kangjia} & 41\\
& Huzhu \ilit{Mongghul} & 42\\
& Minhe \ilit{Mangghuer} & 43\\
\midrule
\end{tabularx}
\end{table}

\begin{table}
\begin{tabularx}{\textwidth}{XXX}
\midrule

\textbf{Trans-Himalayan} & Standard \ilit{Mandarin} & 44\\
& Urumqi Hui \ilit{Mandarin} & 45\\
& Hezhou & 46\\
& \ilit{Tangwang} & 47\\
& \ilit{Wutun} & 48\\
& \ilit{Amdo Tibetan}, Gonghe & 49\\
& \ilit{Baima} & 50\\
& \ilit{Zhongu} & 51\\
\textbf{Tungusic} & \ilit{Even} & 52\\
& \ilit{Evenki} & 53\\
& \ilit{Negidal} & 54\\
& Kh. \ilit{Evenki} & 55\\
& \ilit{Oroqen}, Xunke & 56\\
& \ilit{Solon}, Huihe & 57\\
& \ilit{Udihe} & 58\\
& \ilit{Kilen} (An Jun) & 59\\
& \ilit{Nanai} & 60\\
& \ilit{Ulcha} & 61\\
& \ilit{Uilta} & 62\\
& \ilit{Sibe} & 63\\
\textbf{Turkic} & \ilit{Salar} & 64\\
& \ilit{Kazakh} & 65\\
& \ilit{Kyrgyz} & 66\\
& \ilit{Tatar} (Chinese) & 67\\
& \ilit{Uyghur} & 68\\
& \ilit{Tuvan} & 69\\
& \ilit{Tofa} & 70\\
& \ilit{Khakas} & 71\\
& \ilit{Sarig Yughur} & 72\\
& \ilit{Altai Turkic} & 73\\
& \ilit{Chulym}, Middle & 74\\
& \ilit{Dolgan} & 75\\
& \ilit{Yakut} & 76\\
\textbf{Uralic} & \ilit{Nganasan} & 77\\
& Forest \ilit{Enets} & 78\\
& Tundra \ilit{Nenets} & 79\\
& \ilit{Selkup}, Taz & 80\\
\textbf{Yeniseic} & \ilit{Ket} & 81\\
\textbf{Yukaghiric} & Kolyma \ilit{Yukaghir} & 82\\
& Tundra \ilit{Yukaghir} & 83\\
\lspbottomrule
\end{tabularx}
\end{table}

Given somewhat unclear boundaries between languages and dialects in \ili{Japonic}, dialectal variation in this family may be slightly overrepresented. It may be noted that the lack of data for some languages might have led to some distortions. Nevertheless, the general areal patterns seem to be valid. The white line in \figref{fig:6:4} indicates the rough definition of \isi{NEA} adopted in this study. The distribution of the languages clearly shows a large \isi{spread zone} over large parts of \isi{Northeast Asia}, including Northern \isi{China}, \isi{Mongolia}, \isi{Siberia}, \isi{Korea}, and \isi{Japan} (excluding Hokkaid\=o and the \isi{Ryūkyūan Islands}) with few but widespread languages. Residual zones with many local languages are found in northern \isi{Manchuria} (including \isi{Sakhalin} and Hokkaid\=o), the \isi{Ryūkyūan Islands}, the Aleut Islands, \isi{Amdo}, the \isi{Altai}, along the \isi{Yenisei}, and perhaps on \isi{Kamchatka}.

% Map 1
\begin{figure}
\includegraphics[width=\textwidth]{figures/fig_6_4.jpg}
\caption{Approximate geographical location of the 83 languages in the sample (1)}
\label{fig:6:4}
\end{figure}

\clearpage %Map2
\begin{figure}
\includegraphics[height=.5\textheight]{figures/fig_6_5.jpg}
\caption{Polar question marking (2)}
\label{fig:6:5}
\end{figure}

\begin{table}
\begin{tabularx}{\textwidth}{lQr}
\lsptoprule

\textbf{Color}  & \textbf{Type} & \textbf{Languages}\\
\midrule
white & sentence-final marker & 45\\
white with black dots & mixed & 19\\
black with white outline & suffix(es) & 4\\
gray with white outline & no morphosyntactic marker & 6\\
white with vertical black lines & sentence initial marker & 3\\
white with horizontal black lines & prefix(es) & 2\\
white with diagonal black lines & second position marker & 2\\
black & preverbal marker & 1\\
light gray with black outline & \isit{word order} & 1\\
\midrule Total & 9 & 83\\
\lspbottomrule
\end{tabularx}
\end{table}

\clearpage %Map3
\begin{figure}
\includegraphics[height=.5\textheight]{figures/fig_6_6.jpg}
\caption{Sentence-final polar question marker present (3)}
\label{fig:6:6}
\end{figure}

\begin{table}
\begin{tabularx}{\textwidth}{lQr}
\lsptoprule

\textbf{Color} & \textbf{Type} & \textbf{Languages}\\
\midrule
white & sentence-final marker & 62\\
gray with white outline & not present & 21\\
\midrule Total & 2 & 83\\
\lspbottomrule
\end{tabularx} 
\end{table}

\clearpage %Map4
\begin{figure}
\includegraphics[height=.5\textheight]{figures/fig_6_7.jpg}
\caption{Content question marking (4)}
\label{fig:6:7}
\end{figure}

\begin{table}
\begin{tabularx}{\textwidth}{lQr}
\lsptoprule

\textbf{Color} & \textbf{Type} & \textbf{Languages}\\
\midrule
white & no morphosyntactic marker or unclear & 36\\
gray with white outline & sentence-final marker & 15\\
black with white outline & suffix(es) & 13\\
white with black dots & mixed & 14\\
light gray with black outline & mobile enclitic & 2\\
\midrule Total & 5 & 83\\
\lspbottomrule
\end{tabularx}
\end{table}

\clearpage %Map5
\begin{figure}
\includegraphics[height=.5\textheight]{figures/fig_6_8.jpg}
\caption{Alternative question marking (5)}
\label{fig:6:8}
\end{figure}

\begin{table}
\begin{tabularx}{\textwidth}{lQr}
\lsptoprule

\textbf{Color} & \textbf{Type} & \textbf{Languages}\\
\midrule
white & unclear & 34\\
gray with white outline & \isit{double marking} & 21\\
white with black dots & mixed, other & 19\\
light gray with black outline & \isit{single marking} & 5\\
black with white outline & \isit{disjunction} & 4\\
\midrule Total & 5 & 83\\
\lspbottomrule
\end{tabularx}
\end{table}

\clearpage %Map6
\begin{figure}
\includegraphics[height=.5\textheight]{figures/fig_6_9.jpg}
\caption{Presence of disjunction in alternative questions (6)}
\label{fig:6:9}
\end{figure}

\begin{table}
\begin{tabularx}{\textwidth}{lQr}
\lsptoprule

\textbf{Color} & \textbf{Type} & \textbf{Languages}\\
\midrule
white & unclear or not present & 65\\
black with white outline & present & 17\\
\midrule Total & 2 & 83\\
\lspbottomrule
\end{tabularx}
\end{table}

\clearpage %Map7
\begin{figure}
\includegraphics[height=.5\textheight]{figures/fig_6_10.jpg}
\caption{Presence of single marking in alternative questions (7)}
\label{fig:6:10}
\end{figure}

\begin{table}
\begin{tabularx}{\textwidth}{lQr}
\lsptoprule

\textbf{Color} & \textbf{Type} & \textbf{Languages}\\
\midrule
white & unclear or not present & 72\\
light gray & present & 11\\
\midrule Total & 2 & 83\\
\lspbottomrule
\end{tabularx} 
\end{table}

\clearpage %Map8
\begin{figure}
\includegraphics[height=.5\textheight]{figures/fig_6_11.jpg}
\caption{Polar versus content question marking (8)}
\label{fig:6:11}
\end{figure}

\begin{table}
\begin{tabularx}{\textwidth}{lQr}
\lsptoprule

\textbf{Color} & \textbf{Type} & \textbf{Languages}\\
\midrule
white & different & 61\\
gray with white outline & identical & 13\\
white with black dots & mixed & 9\\
\midrule Total & 3 & 83\\
\lspbottomrule
\end{tabularx}
\end{table}

\clearpage %Map9
\begin{figure}
\includegraphics[height=.5\textheight]{figures/fig_6_12.jpg}
\caption{Polar and content questions overtly marked differently (9)}
\label{fig:6:12}
\end{figure}

\begin{table}
\begin{tabularx}{\textwidth}{lQr}
\lsptoprule

\textbf{Color} & \textbf{Type} & \textbf{Languages}\\
\midrule
white & no or unclear & 62\\
gray with white outline & yes & 21\\
\midrule Total & 2 & 83\\
\lspbottomrule
\end{tabularx}
\end{table}

\clearpage %Map10
\begin{figure}
\includegraphics[height=.5\textheight]{figures/fig_6_13.jpg}
\caption{Polar versus alternative question marking (10)}
\label{fig:6:13}
\end{figure}

\begin{table}
\begin{tabularx}{\textwidth}{lQr}
\lsptoprule

\textbf{Color} &   \textbf{Type} & \textbf{Languages}\\
\midrule
gray with white outline & identical & 37\\
black with white outline & different & 5\\
white & mixed or unclear & 41\\
\midrule Total & 3 & 83\\
\lspbottomrule
\end{tabularx}
\end{table}

\clearpage %Map11
\begin{figure}
\includegraphics[height=.5\textheight]{figures/fig_6_14.jpg}
\caption{\isi{KIN-interrogative}s (11): The interrogative meaning ‘who’ in a given language has the form KIN (velar or uvular plosive or fricative, (high) vowel (short or long), (apical) nasal), followed by an optional final vowel, e.g. Turkish \textit{kim}, Forest Nenets \textit{kim’a}, Aleut \textit{kiin} etc.}
\label{fig:6:14}
\end{figure}

\begin{table}
\begin{tabularx}{\textwidth}{lQr}
\lsptoprule

\textbf{Color} & \textbf{Type} & \textbf{Languages}\\
\midrule
white & not present & 53\\
gray with white outline & present & 30\\
\midrule Total & 2 & 83\\
\lspbottomrule
\end{tabularx}
\end{table}

\clearpage %Map12
\begin{figure}
\includegraphics[height=.5\textheight]{figures/fig_6_15.jpg}
\caption{\isi{K-interrogatives} (12): More than two interrogatives in a given language start with the same velar or uvular plosive or fricative, e.g. Nanai \textbf{\textit{x}}\textit{aɪ} ‘what’, \textbf{\textit{x}}\textit{ado} ‘how many’, \textbf{\textit{x}}\textit{ooni} ‘how’, Uyghur \textbf{\textit{q}}\textit{aysi} ‘which’, \textbf{\textit{q}}\textit{ačan} ‘when’, \textbf{\textit{q}}\textit{andaq} ‘how’ etc.}
\label{fig:6:15}
\end{figure}

\begin{table}
\begin{tabularx}{\textwidth}{lQr}
\lsptoprule

\textbf{Color} & \textbf{Type} & \textbf{Languages}\\
\midrule
gray with white outline & not present & 42\\
white & present & 39\\
light gray & unclear & 2\\
\midrule Total & 3 & 83\\
\lspbottomrule
\end{tabularx}
\end{table}

\clearpage %Map13
\begin{figure}
\includegraphics[height=.5\textheight]{figures/fig_6_16.jpg}
\caption{The personal interrogative ‘who’ has a different initial consonant from all other interrogatives (13)}
\label{fig:6:16}
\end{figure}

\begin{table}
\begin{tabularx}{\textwidth}{lQr}
\lsptoprule

\textbf{Color} & \textbf{Type} & \textbf{Languages}\\
\midrule
white & yes & 43\\
gray with white outline & no & 35\\
light gray & unclear & 4\\
white with black dots & mixed & 1\\
\midrule Total & 4 & 83\\
\lspbottomrule
\end{tabularx}
\end{table}