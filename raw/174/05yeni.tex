\section{Yeniseic}\label{sec:5.13}
\subsection{Classification of Yeniseic}\label{sec:5.13.1}

As we have seen in Chapter 3, the \ili{Yeniseic} \isi{language family} differs strongly from most other languages in \isi{NEA} (e.g., \citealt{Comrie1981}: 61-66; \citeyear{Comrie2003}; \citealt{Anderson2003,Anderson2006b}; \citealt{Georg2008}). Today, \ili{Ket} is the only representative of this \isi{language family}, but historically there have been more languages, including \ili{Yugh} (extinct since the 1970s), \ili{Kott} (extinct since 1850), \ili{Assan} (extinct since 1800), \ili{Arin} (extinct since the 1730s), and \ili{Pumpokol} (extinct since the early 1800s) \citep[470]{Vajda2009b}. Several other languages may have existed but these are almost entirely unknown. This chapter will thus be focusing primarily on \ili{Ket}, but where possible comparative data will be included from other languages as well, especially \ili{Yugh} and \ili{Kott}. There are several attempts at a classification. \citet[153]{Georg2008} proposes the following:

\ea\upshape%1
    \label{ex:yeni:1}
\begin{forest}  for tree={grow'=east,delay={where content={}{shape=coordinate}{}}},   forked edges  
[
    [Northern
        [\ili{Yenisei-Ostyakic}
        	[\ili{Ket}]
            [\textsuperscript{†}\ili{Yugh}]
        ] 
    ]
    [?\textsuperscript{†}\ili{Pumpokol}
    ]
    [Southern
    	[\ili{Assanic}
        	[\textsuperscript{†}\ili{Assan}]
            [\textsuperscript{†}\ili{Kott}]
        ]
        [\textsuperscript{†}\ili{Arin}
        ]
    ]
]
\end{forest}   
    \z

According to \citet[470]{Vajda2009b}, \ili{Arin} can perhaps be classified together with {Pumpo\-kol}. Both approaches agree in the number of languages as well as in a close relation of \ili{Ket} and \ili{Yugh} on the one hand and of \ili{Kott} and \ili{Assan} on the other. While Georg classifies \ili{Arin} with \ili{Assan} and \ili{Kott}, Vajda tentatively assumes a connection with \ili{Pumpokol}. Both approaches are well aware of the somewhat unclear position of \ili{Pumpokol}. For lack of sufficient information this chapter will exclude \ili{Assan}, \ili{Arin}, and \ili{Pumpokol}. In addition, \cite{VovinVajdadelaVaissière2016}, and references therein, have, in my eyes, conclusively shown that at least parts of the \isi{Xiongnu} confedertation in what today is northern \isi{China} and \isi{Mongolia} must have spoken a Yeniseic or \ili{Para-Yeniseic} language (cf. \citealt{Shimunek2015}), which indicates that, historically, (Pre-)\ili{Proto-Yeniseic} must have been located much further to the east.

\subsection{Question marking in Yeniseic}\label{sec:5.13.2}

Questions in \ili{Yeniseic} languages have been analyzed by \cite[155–168]{Werner1995}, who based his approach on V. A. Moskovoj. Unfortunately, his account is rather obscure and lacks a proper \isi{analysis} of the examples. Where possible, the \isi{analysis} in this subsection follows \citet{Vajda2004} and \citet{Georg2007}.

Polar questions in \ili{Ket} may take a marker \textit{=u} that usually takes the second position in a sentence, which is a marked difference from most other languages of \isi{NEA}. \citet[159]{Werner1995} claims that \textit{=u} is a particle, but wrote it attached to other words with a hyphen. It is reanalyzed as enclitic here.

\ea%2
    \label{ex:yeni:2}
    \ili{Ket}\\
    \gll toˑk=\textbf{{u}} ɛtam?\\
    axe=\textsc{q}    sharp\\
    \glt ‘Is the axe sharp?’ \citep[159]{Werner1995}
    \z

Another polar \isi{question marker} \textit{t\=am} also converts interrogatives into indefinites, e.g. \textit{t\=am bíla} ‘somehow’ and functions as a \isi{disjunction} if employed twice.

\ea%3
    \label{ex:yeni:3}
    \ili{Ket}\\
    \gll b\=u \textbf{{t\=am}} d[u]-i[k]-n-bes?\\
    3\textsc{sg}  \textsc{q}  3\textsc{m}-here-\textsc{pst}-move\\
    \glt ‘Did he come?’ (\citealt{KotorovaNefedov2015}: 67)
    \z

In \isi{negative polar question}s the enclitic \textit{=u} attaches to the negator \textit{bən’} that in this case has the unexplained form \textit{bən’d}. The \textit{d} might be epenthetic.

\ea%4
    \label{ex:yeni:4}
    \ili{Ket}\\
    \gll toˑk \textbf{{bən’du}} ɛtam?\\
    axe  \textsc{neg.q}    sharp\\
    \glt ‘Is the axe not sharp?’ \citep[159]{Werner1995}
    \z

Note that the enclitic does not take second position here, perhaps because the negator and \isi{question marker} were reanalyzed as one element. Possibly, the form has to be reanalyzed as \textit{bən’-du} in which the second part might be the unexpected third person \isi{singular} masculine predicative marker (Stefan Georg p.c. 2016). However, both \textit{=u} and \textit{bən’du} are said to highlight the following instead of the preceding word (\citealt{KotorovaNefedov2015}: 66).

For \isi{alternative question}s \ili{Ket} has borrowed the \ili{Russian} \isi{disjunction} \textit{ili}/\textcyrillic{или}, used in \isi{interrogative} and non-\isi{interrogative} contexts, but also makes use of \isi{double marking} with the negative polar \isi{question marker} put before each alternative.

\ea%5
    \label{ex:yeni:5}
    \ili{Ket} (Madujka, Kurejka)\\
    \ea
    \gll {t-}{a-[i]}{n-}gij,    túlet  \=on-{am} {us\textsuperscript{j}}{am} \textbf{{ili}} q{ómat}{-}{am?}\\
    \textsc{prev}-\textsc{them}-\textsc{pst}-say  berry  many-\textsc{pr}.3\textsc{sg}  \textsc{ex}  or  few-\textsc{pr}.3\textsc{sg}\\
    \glt ‘Say, are the berries many or few?’ 
    
    \ex
    \gll \textbf{{ána}} qa-du, \textbf{{b}}\textbf{{ə́}}\textbf{{ndu}} hík.biseb, \textbf{{b}}\textbf{{ə́}}\textbf{ndu} qím.biseb?\\
    who  big-\textsc{pr}.3\textsc{sg.m}  \textsc{q}    brother  \textsc{q} sister\\
    \glt ‘Who is bigger, the brother or the sister?’ (\citealt{KotorovaNefedov2015}: 122, 183)\footnote{Many thanks to Stefan Georg (p.c. 2016) for finding these examples and helping with some aspects of their \isi{analysis}.}
    \z
    \z

In example (\ref{ex:yeni:5}b) an \isi{alternative question} follows a \isi{content question} (\sectref{sec:4.4}). Clearly, the morphosyntactic behavior of the \isi{question marker} \textit{qaj} in the \ili{Uralic} language \ili{Selkup} that appears once before each alternative in alternative \isi{questions} has an areal connection to \ili{Ket} \textit{bə́}\textit{ndu} (\sectref{sec:5.12.2}). But while the \ili{Selkup} marker seems to derive from an \isi{interrogative} of perhaps \ili{Turkic} origin (see \sectref{sec:5.13.3}), this is not the case in \ili{Ket}. In another example only one marker is present between the two alternatives. In one case a \isi{negative alternative question} seems to exhibit \isi{juxtaposition}.

\ea%6
    \label{ex:yeni:6}
    \ili{Ket} (Kellog)\\
    \ea
    \gll \=uk    huʔn    h\'ə{nuna-da} \textbf{{b}}\textbf{{\`ə}}\textbf{ndu} \=uk    hïʔp  h\'ə{nuna-du?}\\
    2\textsc{sg.gen}  daughter  small-\textsc{pr}.3\textsc{f} \textsc{q} 2\textsc{sg.gen}  son  small-\textsc{pr}.3\textsc{m}\\
    \glt ‘Is your daughter or your son smaller?’
    
    \ex
    \gll \=u  b{ín}{-k}{u} ké{na}{s-}{keʔ}{t,} bí{n-k}{u} \textbf{ }\textbf{{b}}\textbf{{\=ə}}\textbf{{n}}?\\
    2\textsc{sg}  self-2\textsc{sg}  bright-human  self-2\textsc{sg}  \textsc{neg}\\
    \glt ‘Are you a real (lit. bright) person or not?’ (\citealt{KotorovaNefedov2015}: 199, 228)
    \z
    \z

Content questions in \ili{Ket} are generally unmarked. Interrogatives may be incorporated and thus defocused. Under “object \isi{focus}” the \isi{interrogative} \textit{ákùs} ‘what’ takes the form \textit{aj} and under “subject \isi{focus}” \textit{an} \citep[88]{Vajda2004}.

\ea%7
    \label{ex:yeni:7}
    \ili{Ket}\\
    \ea
    \gll \textbf{{ákùs}} d\'ə{-b-bèt?}\\
    what  3\textsc{f.S}-3\textsc{n}.O-do\\
    \glt ‘Just what is she making?’
    
    \ex
    \gll da-\textbf{{ákùs}}{-[s]-bet?}\\
    3\textsc{f}-what-\textsc{ms}-do\\
    \glt ‘What is she doing?’
    
    \ex
    \gll \textbf{{aj}} d\'ə{-b-bèt?}\\
    what  3\textsc{f.S}-3\textsc{n}.O-do\\
    \glt ‘Just \textit{what is it} she is making?’
    
    \ex
    \gll \textbf{{an}} kú-[i]n-à?\\
    what  2S-\textsc{pst}-active.event\\
    \glt ‘Just \textit{what} happened to you?’ \citep[88]{Vajda2004}\z\z

\noindent Both \textit{an} and \textit{aj} are sometimes called question particles (\citealt{Werner1995}: 156; \citealt{KotorovaNefedov2015}: 66), which clearly must be rejected. For \ili{Yugh} there seems to be the same mistake \citep[214]{Werner1997b}.

\ea%8
    \label{ex:yeni:8}
    \ili{Yugh}\\
    \gll \textbf{{an}} diˑn’e?\\
    what  1S.\textsc{pst.}active.event\\
    \glt ‘Just \textit{what} happened to me?’ \citep[225]{Werner1997b}
    \z

\noindent Perhaps, \textit{an} has the same function as in \ili{Ket} as there are also other forms such as \textit{assa} ‘what’ that may also be incorporated and is thus comparable with \ili{Ket} \textit{ákùs} ‘what’. Historically, \ili{Yugh} \textit{assa} may go back to *\textit{aksa} \citep[157]{Werner2004}, which makes it even more similar to the \ili{Ket} \isi{interrogative}. A cognate of \textit{aj} does not seem to be attested.

\ea%9
    \label{ex:yeni:9}
    \ili{Yugh}\\
    \ea
    \gll u \textbf{{assa}} ku-b-bet’?\\
    2\textsc{sg}  what  2S-3\textsc{n}.O-do\\
    \glt ‘What are you doing?’
    
    \ex
    \gll u  k-\textbf{{assa-}}{iˑget’?}\\
    2sg  2S-what-do\\
    \glt ‘What are you doing?’ (\citealt{Werner1997b}: 225, 226)
    \z
    \z

Apart from \textit{an}, \citet[214]{Werner1997b} claims that there are several more question markers, namely \textit{χala} {\textasciitilde} \textit{χara}, \textit{atá}, and \textit{bən’}. The status of the first could not be settled,\footnote{A connection with a \ili{Mongolian} \isi{question tag} (e.g., \ili{Dukhan} \textit{hala} {\textasciitilde} \textit{harən}) seems too far-fetched (\sectref{sec:5.8.2}).} but \textit{atá} most likely is simply an \isi{interrogative} meaning ‘why’ (\sectref{sec:5.13.3}), while \textit{bən’} is a negator. Werner translates the following sentence with ‘or not’, which seems comparable to example (\ref{ex:yeni:6}b) from \ili{Ket} above.

\ea%10
    \label{ex:yeni:10}
    \ili{Yugh}\\
    \gll dɨlatkat  bɨl’l’a  dɔnaŋd’in, \textbf{{bən’}}?\\
    children  all  ?3.\textsc{pst}.3.come    \textsc{neg}\\
    \glt ‘Have the children all come or not?’ \citep[225]{Werner1997b}
    \z

\noindent For marking \isi{polar question}s, \ili{Yugh} had in addition an unspecified \isi{intonation} pattern \citep[225]{Werner1997b}.

\ili{Even} less information than for \ili{Yugh} \isi{questions} is available for \ili{Kott}. But apparently, \ili{Kott} has a final \isi{question marker} that is analyzed as enclitic here.

\ea%11
    \label{ex:yeni:11}
    \ili{Kott}\\
    \gll huṡ=\textbf{{bo}}?\\
    horse=\textsc{q}\\
    \glt ‘(Is it) a horse?’ (\citealt{Castrén1858}: 153, \citealt{Werner1997c}: 80)
    \z

\noindent For lack of further examples the \isi{semantic scope} of \textit{=bo} remains unclear. Alternative questions seem to take two markers (A=\textit{bo} B=\textit{bo}), although no example was provided by \cite[153]{Castrén1858}. Most likely, \textit{=bo}, like the marker \textit{=bV} in the \ili{Uralic} language \ili{Kamass} (\sectref{sec:5.12.2}), has been borrowed from a \ili{Turkic} source (\sectref{sec:5.11.2}). \cite[154]{Castrén1858} furthermore mentions the \ili{Kott} \isi{question marker} \textit{â}. There is no information on its morphosyntactic behavior or exact function, but it might be connected with the \ili{Kamass} polar \isi{question marker} \textit{=a}. Apparently, \ili{Russian} \textit{li}/\textcyrillic{ли} has also been borrowed. There is no example for a \isi{content question} from \ili{Kott}.

\begin{table}
\caption{Summary of question marking in Yeniseic}
\label{tab:yeni:1}

\begin{tabularx}{\textwidth}{lQlQ}
\lsptoprule
& \textbf{PQ} & \textbf{CQ} & \textbf{AQ}\\
\midrule
\ili{Ket} & \#A=u, \#A b\'əndu, \#A t\=am & - & (b\'əndu) A b\'əndu B, ili ‘or’\\
\ili{Yugh} & - & - & ?\\
\ili{Kott} & =bo\#, ?â & ? & 2x =bo\#\\
\lspbottomrule
\end{tabularx}
\end{table}

\subsection{Interrogatives in Yeniseic}\label{sec:5.13.3}

The \ili{Yeniseic} interrogatives strongly differ from those found in other languages of \isi{Northeast Asia}. Especially the large amount of forms meaning ‘who’ and ‘what’ is exceptional. \ili{Ket} additionally has analyzable forms such as \textit{ásès biˀ} ‘what kind of thing’ and \textit{ásès keˀt} ‘what kind of person’ \citep[32]{Vajda2004}. The existence of special female and male forms of the personal interrogatives is unique but has some typological parallels in the \ili{Indo-European} selective interrogatives (\sectref{sec:5.5.3}). \ili{Ket} and \ili{Yugh} interrogatives usually start with \textit{a{\textasciitilde}} or with \textit{b{\textasciitilde}}, which has no clear parallels in \isi{NEA}, but in \ili{Burushaski}, for example \citep{Yoshioka2012}. It may be remembered that this is first and foremost a typological classification and does not necessarily indicate a genetic connection. The \isi{interrogative systems} in \ili{Ket} and \ili{Yugh} are certainly similar to each other and show some direct cognates (e.g., \ili{Ket} \textit{bísȅŋ}, \ili{Yugh} \textit{bisa\textsuperscript{h}}\textit{:ŋ} ‘where’) and identical categories (e.g., ‘who.\textsc{sg.f’} vs. ‘who.\textsc{sg.m’}). But there are several striking differences (marked with boldface) that suggest a considerable time of divergence.

\begin{table}
\caption{Ket (\citealt{Vajda2004}: 31, 41f., 88) and Yugh interrogatives (\citealt{Werner1997b}: 10, 98f., 103, 211, 214, 226); the Ket forms in square brackets are from \citet[167]{Georg2007}}
\label{tab:yeni:2}

\begin{tabularx}{\textwidth}{XXl}
\lsptoprule

\textbf{Meaning} & \textbf{Ket} & \textbf{Yugh}\\
\midrule
who.\textsc{sg} & \textbf{ánȁ {\textasciitilde} ánȅt} & aneit, anɛt ‘who.\textsc{sg.m}’\\
who.\textsc{pl} & \textbf{ánȅt-aŋ, bílàŋsan} & asein, ase:n\\
what & ák(ù)s, an, \textbf{aj} & assa, an\\
how many & ánùn, [bìlon] & an’ej(a), birejɔ\textsuperscript{h}:n, birɔn\\
which, what kind of & ásès, \textbf{as} & aseis, aš’eiš’(i)\\
when & áskà & aˑškej\\
why & \textbf{áksdìŋt} & \textbf{atá, asɛsaŋ}\\
who.\textsc{sg.f} & \textbf{bésà} & \textbf{asɛra}\\
who.\textsc{sg.m} & \textbf{bítsè} & aneit, anɛt\\
where & bísȅŋ & bisa\textsuperscript{h}:ŋ\\
whither & \textbf{bíltàn}, [bìles] & birɛ\textsuperscript{h}s, birɛ\textsuperscript{h}:š\\
whence & bílȉl, [bili(\textbf{ŋa})l] & birɨ:r, birə:r\\
how & bílȁ, [\textbf{bílunon}] & birej\\
\lspbottomrule
\end{tabularx}
\end{table}

According to \citet[165]{Georg2007}, the form \textit{bílà-ŋ-s-an} ‘who.\textsc{pl}’ can be analyzed as ‘how-\textsc{pl}-\textsc{n}-\textsc{pl}’ with an unexpected \isi{plural} marker \textit{-an}, but a development from ‘how’ to ‘who’ is extremely unlikely. According to \citet[89]{Vajda2013}, \ili{Ket} \textit{bì-l-es} and \ili{Yugh} \textit{bi-r-ɛ\textsuperscript{h}}\textit{:š} ‘whither’ can historically be analyzed as ‘which-\textsc{poss}-open.space’. Diachronically, the actual stem thus may not be \textit{bil-} \citep[167]{Georg2007}, but \textit{bi-}.

\begin{table}
\caption{Paradigm of the Ket locative interrogative \citep[167]{Georg2007}}
\label{tab:yeni:3}

\begin{tabularx}{\textwidth}{XXl}
\lsptoprule
& \textbf{\textsc{sg}} & \textbf{\textsc{pl}}\\
\midrule
\textsc{1} & bìseŋ-di, bìseŋ-am (\textsc{n}) & bìseŋ-daŋ\\
\textsc{2} & bìseŋ-ku & bìseŋ-kaŋ\\
\textsc{3} & bìseŋ-du (\textsc{m}), bìseŋ-da (\textsc{f}) & bìseŋ-aŋ\\
\lspbottomrule
\end{tabularx}
\end{table}

Interestingly, \citet[226]{Werner1997b} also mentions the \ili{Yugh} forms \textit{bi-da} ‘where is it/she?’ and \textit{bi-du} ‘where is he?’ that seem to show a \isi{gender} contrast. This is comparable with the \ili{Ket} forms \textit{bìseŋ-da} and \textit{bìseŋ-du} (\tabref{tab:yeni:3}) that are based on an extended stem (cf. \ili{Yugh} \textit{bisa\textsuperscript{h}}\textit{:ŋ} ‘where’). Perhaps, \ili{Ket} \textit{bìlon} ‘how many’ is based on the European pattern (e.g., \ili{Russian} \textit{kak mnogo}/\textcyrillic{как много}), cf. \ili{Ket} \textit{bílȁ} ‘how’ and \textit{òn} ‘many, much’. Note that \ili{Yugh}, apart from \textit{birɔn}, has a more \isi{transparent} form \textit{birejɔ\textsuperscript{h}}\textit{:n}. \ili{Ket} \textit{ákùs} ‘what’ has an abbreviated variant \textit{ák(ù)s} that “must be quite old and stabilized, since the retention of phonetic [k] in the longer variant can only be understood as a remodelling [sic] after the former.” (\citealt{Georg2007}: 82, fn. 92) This is the basis for \textit{áks-dìŋt(a)} ‘why’, which exhibits an adessive marker \citep[166]{Georg2007}.

For \ili{Kott} there is an extensive description by \cite{Castrén1858} that has been elaborated on by \citet{Werner1997c}. The \ili{Kott} \isi{interrogative} system (\tabref{tab:yeni:4}) also has a \isi{resonance} in \textit{b{\textasciitilde}} but only one form starting with \textit{a-} and also has the form \textit{heɫem} ‘when’ as well as \textit{ṡena} or \textit{ṡina} ‘what’ that deviate from this pattern and are perhaps unrelated to the other forms. They do not appear to have been borrowed from any known language.

\begin{table}
\caption{Kott interrogatives (\citealt{Castrén1858}: 55, 149ff.)}
\label{tab:yeni:4}

\begin{tabularx}{\textwidth}{XX}
\lsptoprule

\textbf{Meaning} & \textbf{Form}\\
\midrule
which, who & aṡix, \textsc{pl} aṡig-ân\\
where & bili\\
whither & biltuŋ\\
whence & bilćaŋ\\
how many/much & bilipei, bilipêi\\
what kind of & biɫäŋ, \textsc{pl} biɫäŋ-an\\
which & bilituiṡe\\
what & ṡena, ṡina, no \textsc{pl}\\
why & ṡena ôjaŋ/uŋô/uŋôjaŋ\\
when & heɫem\\
\lspbottomrule
\end{tabularx}
\end{table}

\largerpage
Reduplication expresses indefinite meaning, e.g. \textit{bili bili} ‘somewhere, everywhere’ (\citealt{Castrén1858}: 150). The complex \isi{interrogative} \textit{ṡena ôjaŋ} ‘why’ is a \isi{transparent} \isi{combination} of \textit{ṡena} ‘what’ and \textit{ôjaŋ} ‘because, for’ (\citealt{Castrén1858}: 202). Following \citet[88]{Vajda2013}, one may identify a stem \textit{bi-} such as in \textit{bi-l-tuŋ}, cognate of \ili{Ket} \textit{bí-l-tàn} ‘whither’, that goes back to \ili{Proto-Yeniseic} *\textit{wi-l-təñ} ‘which-\textsc{poss}-path’. The exact \isi{analysis} of most other forms remains uncertain to me as \ili{Ket}, \ili{Kott}, and \ili{Yugh} have a \isi{tendency} for \isi{opaque} \isi{interrogative systems}. Their historical \isi{analysis} can only be accomplished by an expert of \ili{Yeniseic} languages.

Non-selective Interrogative pronouns in \ili{Yeniseic} have extensive paradigms of \isi{case} marking (\tabref{tab:yeni:5}, see \citealt{Werner1997b}: 98 for \ili{Yugh}; \citealt{Werner1997c}: 79f. for \ili{Kott}). Demonstratives show related paradigms (e.g., \citealt{Werner1997b}: 97, 103).

In sum, \ili{Yeniseic} \isi{interrogative systems} deviate strongly from those in all other languages in \isi{Northeast Asia}. Apart from the formal differences---there are neither KIN- nor K-interrogatives---there are unusual categories such as a \isi{gender} distinction in the personal interrogatives, incorporation, and a large number of different interrogatives with the meaning ‘who’ and ‘what’.

\begin{table}
\caption{Ket singular interrogative paradigms \citep[140]{Werner1997a}}
\label{tab:yeni:5}

\begin{tabularx}{\textwidth}{llllQQ}
\lsptoprule

\textbf{Meaning} & \textbf{who.\textsc{masc}} & \textbf{who.\textsc{fem}} & \textbf{what} & \textbf{who.\textsc{masc}} & \textbf{who.\textsc{fem}}\\
\midrule
\textsc{abs} & an’a & an’a & akus’ & bit’se & bɛs’a\\
\textsc{gen} & an’a-da & an’a-d(i) & akus’-t & bit’se-da & bɛs’a-d(i)\\
\textsc{dat} & an’a-daŋa & an’a-diŋa & akus’-tiŋa & bit’se-daŋa & bɛs’a-diŋa\\
\textsc{ben} & an’a-data & an’a-dita & akus’-tita & bit’se-data & bɛs’a-dita\\
\textsc{abl} & an’a-daŋal’ & an’a-diŋal’ & akus’-tiŋal’ & bit’se-daŋal’ & bɛs’a-diŋal’\\
\textsc{loc} & - & - & akus’-ka & - & -\\
\textsc{pros} & an’a-bes’ & an’a-bes’ & akus’-bes’ & bit’se-bes’ & bɛs’a-bes’\\
\textsc{ades} & an’a-daŋta & an’a-diŋta & akus’-tiŋta & bit’se-daŋta & bɛs’a-diŋta\\
\textsc{com} & an’a-(γ)as’ & an’a-(γ)as’ & akus’-as’ & bit’se-γas, bits’as’ & bɛs’a-γas, bɛs’as\\
\textsc{car} & an’a-(γ)an & an’a-(γ)an & akus’-an’ & bit’se-γan, bits’an’ & bɛs’a-γan, bɛs’an\\
\lspbottomrule
\end{tabularx}
\end{table}


\subsection{Dene-Yeniseian?}\label{sec:5.13.4}

If the \ili{Dene-Yeniseian} hypothesis \citep{Vajda2010} has a basis in actual fact, the common proto-language must be several thousand years older than \ili{Proto-Yeniseic}. It is unlikely that question markers remain stable over such long time spans. Similarities would be expected instead in the \isi{interrogative} system. There have been previous attempts to correlate \ili{Yeniseic} and \ili{Na-Dene} interrogatives, notably by \cite[157ff.]{Werner2004}, but, in the absence of clear cognates and sound laws, any comparison must be preliminary at best. For reasons of space and lack of reliable reconstructions, Na-Dene interrogatives cannot be dealt with here. Nevertheless, it is a possibility that the \textit{type} of \isi{question marking} that is somewhat less prone to changes than the actual question markers shows certain similarities. Given that the \ili{Ket} \isi{question marker} is very different from the surrounding languages---\ili{Selkup} was most likely influenced by \ili{Ket}---chances are high that it represents a relatively old and possibly stable feature. \ili{Na-Dene} consists of \ili{Eyak}, \ili{Tlingit}, and the \ili{Athabaskan} languages. Of course, only a very cursory overview can be given here. According to \citet[267]{Enrico2004}, \ili{Na-Dene} languages have a \isi{tendency} for “clause-initial clitic \isi{interrogative} markers”. Perhaps, what Enrico has in mind are sentence initial question markers such as in \ili{Slavey} (see also \sectref{sec:4.2}).

\ea%12
    \label{ex:yeni:12}
    \ili{Slavey} (Mountain)\\
    \gll \textbf{{h}}\textbf{{\'{\k{i}}}} golǫ  fehk’é?\\
    \textsc{q}  moose  3.shot\\
    \glt ‘Did he shoot a moose?’ \citep[1123]{Rice1989}
    \z

In Western \ili{Apache} there are both sentence initial and final question markers that may be used independently of each other or combined.

\ea%13
    \label{ex:yeni:13}
    \ili{Apache} (Western, San Carlos)\\
    \gll (\textbf{{ya’}})  Katie  nłdzil  (\textbf{{né}})?\\
    \textsc{q}  \textsc{pn}  strong  \textsc{q}\\
    \glt ‘Is Katie strong?’ (\citealt{deReuse2006}: 57)
    \z

However, \ili{Eyak}, \ili{Tlingit}, and \ili{Navajo} all show second position clitics as does \ili{Ket}.

\ea%14
    \label{ex:yeni:14}
    \ili{Eyak}\\
    \gll sAsinL=\textbf{{sh}}?\\
    ?die=\textsc{q}\\
    \glt ‘Did it die?’ (\citealt{Kraussforthcoming}: 648)
    \z

\ea%15
    \label{ex:yeni:15}
    \ili{Tlingit}\\
    \gll lingít{=}\textbf{{gé}} x’eeya.áxch?\\
    \textsc{pn=q}    2\textsc{sg}.understand.it\\
    \glt ‘Do you speak \ili{Tlingit}?’ (\citealt{Cable2007}: 74, fn. 40)
    \z

\ea%16
    \label{ex:yeni:16}
    \ili{Navajo}\\
    \gll dichin=\textbf{{ísh}} níl\k{í}  ?\\
    hunger=\textsc{q}  2\textsc{sg}.\textsc{poss}\\
    \glt ‘Are you hungry?’ (\citealt{Young1987}: 23)
    \z

A difference from \ili{Ket} is the presence of an overt second position \isi{question marker} in \ili{Eyak}, \ili{Tlingit}, and \ili{Navajo} \isi{content question}s as well.

\ea%17
    \label{ex:yeni:17}
    \ili{Eyak}\\
    \gll [\textbf{{dee}} k’u.tse’]{=}\textbf{{d}}?\\
    what  meat=\textsc{q}\\
    \glt ‘What meat?’ (\citealt{Kraussforthcoming}: 586)
    \z

\ea%18
    \label{ex:yeni:18}
    \ili{Tlingit}\\
    \gll \textbf{{d}}\textbf{{aa}}{=}\textbf{{sá}} uwajée wutoo.oowú?\\
    what=\textsc{q}    3\textsc{pl}.think  1\textsc{pl}.bought.it\\
    \glt ‘What did they think we bought?’ \citep[69]{Cable2007}
    \z

\ea%19
    \label{ex:yeni:19}
    \ili{Navajo}\\
    \gll \textbf{{haa}}{=}\textbf{{sh}} yidzaa?\\
    what=\textsc{q}  3\textsc{acc}.3\textsc{nom}.happened\\
    \glt ‘What happened to him?’ \citep[33]{Fountain2008}
    \z

\noindent In \ili{Navajo}, the polar (\textit{=ísh} {\textasciitilde} \textit{=sh}) and \isi{content question} markers (\textit{=shą’} {\textasciitilde} \textit{=sh}) partly overlap in form (\citealt{Young1987}: 23). However, some languages such as Slavey and Western \ili{Apache} have unmarked \isi{content question}s just like \ili{Ket}.

\ea%20
    \label{ex:yeni:20}
    \ili{Slavey} (Hare)\\
    \gll \textbf{{men\k{i}}} ˀahęt’e?\\
    who    3.\textsc{cop}\\
    \glt ‘Who is it?’ \citep[1141]{Rice1989}
    \z

\ea%21
    \label{ex:yeni:21}
    \ili{Apache} (Western, San Carlos)\\
    \gll \textbf{{hadín}} dázhǫ́ nłdzil?\\
    who  very  strong\\
    \glt ‘Who is very strong?’ (\citealt{deReuse2006}: 50)
    \z

Independent of the question of whether \ili{Yeniseic} and Na-Dene are genetically related---which cannot, of course, be proven by typology---, \ili{Na-Dene} shows markedly different \isi{question marking} than most of \isi{NEA}, except \ili{Ket}, parts of \ili{Chukotko-Kamchatkan}, and some \ili{Indo-European} languages. Thus, there is a relatively clear boundary between \isi{NEA} and North America. Furthermore, \citet{Dryer2013k} has shown that polar \isi{question marking} in the Americas in general is much less uniform than that in \isi{NEA}. The frequent sentence initial position of interrogatives likewise differentiates Na-Dene and the Americas in general from \isi{NEA} \citep{Dryer2013l}.

\clearpage %solid chapter boundary