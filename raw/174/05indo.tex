\section{\ili{Indo-European}}\label{sec:5.5}
\subsection{Classification of \ili{Indo-European}}\label{sec:5.5.1}

According to \textit{Glottolog} (\citealt{Hammarström2016}), \ili{Indo-European} encompasses 583 languages. Similar to \sectref{sec:5.9} on \ili{Trans-Himalayan}, this section can only deal with a minor part of the whole \ili{Indo-European} family. The exact internal \isi{phylogenetic} structure of the family is not absolutely clear (see \sectref{sec:2.5}), but one may roughly distinguish 10 different branches as well as a couple of unaffiliated and sparsely attested languages that are excluded here \citep[10]{Fortson2010}: 1. Albanian, 2. \textsuperscript{†}Anatolian, 3. Armenian, 4. Balto-\ili{Slavic} (\ili{Baltic}, \ili{Slavic}), 5. Celtic, 6. \ili{Germanic}, 7. Greek, 8. \ili{Indo-Iranian} (\ili{Indo-Aryan}, \ili{Iranian}, and perhaps \ili{Nuristani}), 9. Italic, and 10. \textsuperscript{†}Tocharian. Only West \ili{Germanic} (\ili{German} dialects, \ili{Yiddish}, \ili{English}), East \ili{Slavic} (\ili{Russian}, \ili{Ukrainian}), East \ili{Iranian} (\ili{Sogdian}, Khotanese, Tumshuquese, \ili{Sarikoli}), and Tocharian (\ili{Tocharian A}, B, and perhaps C) have representatives in \isi{NEA}. For the mixed \ili{Persian}-\ili{Uyghur} language \ili{Eynu} (\textit{àinǔ} \zh{艾努}), spoken in the southeast of the \isi{Tarim} basin, see \sectref{sec:5.11}. \ili{Taimyr Pidgin Russian} (or \ili{Govorka}) and \ili{Chinese Pidgin Russian} (sometimes called \ili{Kyakhta Pidgin}) will be included in this chapter, but the mixed \ili{Russian}-\ili{Aleut} language Mednyj \ili{Aleut} has been treated in \sectref{sec:5.4} on \ili{Eskaleut}.

\subsection{Question marking in \ili{Indo-European}}\label{sec:5.5.2}

\subsubsection{Question marking in Proto-\ili{Indo-European}}\label{sec:5.5.2.1}

PIE presumably had interrogatives in initial position, optionally preceded by a topicalized element \citep[232]{Fortson2006}. Questions in PIE were probably primarily marked with a special \isi{intonation} contour (\citealt{Delbrück1900}: 259–288; \citealt{Lehmann1974}: 101f., 120-123, 179f.; \citealt{Hackstein2013}: 99), although \isi{word order} change is attested in several \ili{Indo-European} branches \citep[102]{Hackstein2013}. Some old \ili{Indo-European} languages had sentence-initial or second position clitics (e.g., \ili{Gothic} \textit{an}, \textit{=u}, \citealt{BrauneHeidermanns2004}). However, the markers in individual branches are not cognates of each other, which is why no such marker can be reconstructed.

\subsubsection{Question marking in Germanic}\label{sec:5.5.2.2}

Modern \ili{Germanic} languages generally have verb-initial \isi{word order} for marking \isi{polar question}s (\ref{ex:indo:1}b). In declarative sentences the verb usually takes second position (\ref{ex:indo:1}a). Consider the following constructed \ili{German} examples as well as their \ili{English} translation. In addition, the \ili{German} \isi{polar question} has a rising \isi{intonation} as opposed to the falling \isi{intonation} in the \isi{declarative sentence}.

\newpage 
\ea%1
    \label{ex:indo:1}
    \ili{German}\footnote{The \isi{glossing} in this chapter is somewhat simplified and relies on the relatively close relationship of \ili{English} with other languages.}
    \ea
    \gll Peter \textbf{{hat}} einen    Hund  gekauft.\\
    \textsc{pn}  has  a.\textsc{acc.m}  dog  bought\\
    \glt ‘Peter bought a dog.’
    
    \ex
    \gll \textbf{{Hat}} Peter  einen    Hund  gekauft?\\
    has  \textsc{pn}  a.\textsc{acc.m}  dog  bought\\
    \glt ‘Did Peter buy a dog?’
    \z
    \z

\noindent If no other auxiliary is present, \ili{English} requires the addition of the auxiliary \textit{to do}. As further explained in Chapter 4, the cross-linguistically untypical phenomenon of \isi{word order} for \isi{question marking} \citep{Dryer2013m} originated in the loss of a second position clitic such as \ili{Gothic} \textit{=u}. Such clitics usually attach to the verb in \isi{polar question}s and to focused elements in \isi{focus question}s. When the \isi{question marker} was lost, the verb-initial \isi{word order} took over its function (e.g., \citealt{Miestamo2011}). \ili{Plautdiitsch} likewise preserves the verb-initial \isi{word order}.

\ea%2
    \label{ex:indo:2}
    \ili{Altai Low German}\\
    \gll \textbf{{väitst}} dyy  va\textsuperscript{u}t  diinə    fryy  feelə    deed?\\
    know.\textsc{prs}.2\textsc{sg}  2\textsc{sg}  what  2\textsc{sg.gen.f}  wife  miss.\textsc{inf}  do.\textsc{prs.}3\textsc{sg}\\
    \glt ‘Do you know what problem your wife has?’\footnote{Cf. non-standard \ili{German} (constructed) \textit{Weißt du, was deiner Frau fehlen tut?}} \citep[170]{Jedig2014}
    \z

An exception among \ili{Germanic} languages is \ili{Yiddish}, which has borrowed the \ili{Polish}, \ili{Ukrainian}, or \ili{Belorussian} initial \isi{question marker}, which will be discussed in \sectref{sec:5.5.3.3}. Nevertheless, there is still a \isi{word order} change as well as final rising \isi{intonation} as opposed to the falling declarative \isi{intonation}.

\ea%3
    \label{ex:indo:3}
    \ili{Yiddish}\\
    \ea
    \gll mojše  hot    gekojft    a  hunt.\\
    \textsc{pn}  has    bought    a  dog\\
    \glt ‘Moses bought a dog.’
    
    \ex
    \gll \textbf{{ci}} \textbf{{hot}} mojše  gekojft    a  hunt?\\
    \textsc{q}  has  \textsc{pn}  bought    a  dog\\
    \glt ‘Did Moses buy a dog?’ (\citealt{SadockZwicky1985}: 181)
    \z
    \z

\noindent The \ili{German} examples in \REF{ex:indo:1} above that were constructed on the basis of these \ili{Yiddish} examples exhibit in addition a slightly different \isi{word order} in that the participle stands sentence-finally. The \ili{Yiddish} \isi{word order} is not usually found in \ili{German} and has an archaic flair to it. The initial \isi{question marker} is optional.

\ea%4
    \label{ex:indo:4}
    \ili{Yiddish}\\
    \gll \textbf{{bist}} du  meshuge?\\
    be.\textsc{prs.ind}.2\textsc{sg}  2\textsc{sg}  crazy\\
    \glt ‘Are you crazy?’ (\citealt{JacobsPrincevanderAuwera1994}: 408)
    \z

\ea%5
    \label{ex:indo:5}
    \ili{German}\\
    \gll \textbf{{Bist}} du  verrückt?\\
    be.\textsc{prs.ind}.2\textsc{sg}  2\textsc{sg}  crazy\\
    \glt ‘Are you crazy?’\footnote{Colloquially, \ili{German} also has the adjective \textit{meschugge} ‘crazy’.}
    \z

Focus questions in \ili{German} and \ili{English} have the same structure as \isi{polar question}s but contain an additional intonational nucleus on the focused element. \ili{English} may also make use of a cleft, e.g. \textit{Is it to school} \textit{that you are going?}

\ea%6
    \label{ex:indo:6}
    \ili{German}\\
    \gll Gehst    du  \ulp{zur}{25} \ule{Schule}?\\
    go\textsc{.prs.}2\textsc{sg}  2\textsc{sg}  to.the.\textsc{f.sg.dat}  school\\
    \glt ‘Are you going \textit{to school}?’
    \z

\noindent I am unaware of any descriptions of \isi{focus question}s in \ili{Yiddish} or \ili{Plautdiitsch} but it is probable that they have a pattern similar to \ili{German}.

Content questions in \ili{German}, \ili{Plautdiitsch} and \ili{Yiddish} do not have a special marking but do have sentence-initial interrogatives. They may be preceded by a conjunction such as \ili{German} \textit{und} ‘and’.

\ea%7
    \label{ex:indo:7}
    \ili{German} (Colloquial)\\
    \ea
    \gll \textbf{{Wo}} is(t)  der    Mensch?\\
    where  is  the.\textsc{m.nom}  person\\
    \glt ‘Where is the person?’
    
    \ex
    \gll Na,  und \textbf{{warum}}?\\
    well  and  why\\
    \glt ‘Well, why then?’
    \z
    \z

\ea%8
    \label{ex:indo:8}
    \ili{Yiddish}\\
    \gll \textbf{{vu}} iz  der    mentsh?\\
    where  is  the.\textsc{m.nom}  person\\
    \glt ‘Where is the person?’ (\citealt{JacobsPrincevanderAuwera1994}: 408)
    \z

\ea%9
    \label{ex:indo:9}
    \ili{Altai Low German}\\
    \gll na,  on \textbf{{vu:rǫm}}?\\
    well  and  why\\
    \glt ‘Well, why then?’ \citep[170]{Jedig2014}
    \z

\ili{Yiddish} has an optional marker \textit{=zhe} that may attach to interrogatives and seems to intensify the sentence (it was translated as ‘on earth’) (\citealt{JacobsPrincevanderAuwera1994}: 413). It is of West \ili{Slavic} origin, e.g. \ili{Czech} \textit{=že} (\citealt{SussexCubberley2006}: 317). \ili{English} differs somewhat from these three languages in that \isi{content question}s usually require an auxiliary or \textit{to do} to follow the \isi{interrogative}.

\ea%10
    \label{ex:indo:10}
    \ili{English}\\
    \ea
    \textit{\textbf{{Where}} \textbf{{is}} the person (going)?}\\
    
    \ex
    \textbf{\textit{Where}} \textbf{\textit{did}} \textit{he go?}\\
    \z
    \z 

Alternative questions in \ili{German} combine usual polar \isi{question marking} (verb first \isi{word order} and \isi{intonation}) with the \isi{disjunction} \textit{oder} [-ɐ] ‘or’. Negative \isi{alternative question}s have the standard negator \textit{nicht} that in colloquial speech often takes the form \textit{nich}. \ili{English} has a similar polar \isi{question marking} in \isi{combination} with \textit{or}.

\ea%11
    \label{ex:indo:11}
    \ili{German}\\
    \ea
    \gll \textbf{{Magst}} du  Tee \textbf{{oder}} Kaffee?\\
    want.\textsc{prs}.\textsc{ind}.2\textsc{sg}  2\textsc{sg}  tea  or  coffee\\
    \glt ‘Do you want tea or coffee?’
    
    \ex
    \gll \textbf{{Magst}} du  Tee \textbf{{oder}} \textbf{{nicht}}?\\
    want.\textsc{prs}.\textsc{ind}.2\textsc{sg}  2\textsc{sg}  tea  or  not\\
    \glt ‘Do you want tea or not?’
    \z
    \z

\noindent Altai Low \ili{German} has verb first \isi{word order} in \isi{combination} with \textit{öuda nich} (= \ili{German} \textit{oder nich(t)}) for negative alternative \isi{questions} and probably \textit{öuda} (= \ili{German} \textit{oder} [-ɐ]) for plain \isi{alternative question}s \citep[177]{Nieuweboer1999}. \ili{Yiddish} also has the verb-in initial position but exhibits alternation between the use of \textit{odər} ‘or’ and \textit{ci}, which has been influenced by \ili{Slavic} \citep[205]{Jacobs2005}.

\ili{German} has two different constructions in which the \isi{disjunction} \textit{oder} takes sentence-final position and acts as a \isi{question marker}. In the first case it is accompanied by a longish and level \isi{intonation}, in the second with a sharp rise in \isi{intonation}. The former is a fully elliptic \isi{alternative question} and the latter a \isi{tag question}.

\ea%12
    \label{ex:indo:12}
    \ili{German}\\
    \ea
    \gll \textbf{{Magst}} du  Kaffee \textbf{{oder} }...?\\
    want.\textsc{prs}.\textsc{ind}.2\textsc{sg}  2\textsc{sg}  coffee  or\\
    \glt ‘Do you want coffee or not?’
    
    \ex
    \gll \textbf{{Du}} magst      Kaffee, \textbf{{oder}}?\\
    2\textsc{sg}  want.\textsc{prs}.\textsc{ind}.2\textsc{sg}  coffee  or\\
    \glt ‘You want coffee, right?’
    \z
    \z

\noindent In the latter type the tag may also take the negative form \textit{oder (etwa) nich(t)} with an optional emphatic marker. \ili{German} has several more tags such as \textit{ja} ‘yes’, \textit{richtig} ‘right’ and \textit{nicht(t)} ‘not’, or dialectal forms such as \textit{wa(t)} (Standard \ili{German} \textit{was} ‘what’) and the synchronically unanalyzable form \textit{ge(lle)}. In all cases a tag seems to be accompanied with a sharp rising \isi{intonation} contour. \ili{English} has related tags such as \textit{right} but is best-known for its tags whose polarity depends on the preceding declarative, e.g. \textit{is it} vs. \textit{isn’t it}, \textit{do you} vs. \textit{don’t you} etc. \ili{Plautdiitsch} has the \isi{tag question} markers \textit{jau?} {\textasciitilde} \textit{jo?} ‘yes’ (\ili{German} \textit{ja}), \textit{nee?} (\ili{German} \textit{nein}, colloquially \textit{ne(e)}), \textit{öuda} ‘or’ (\ili{German} \textit{oder}), and \textit{es nich zöu?} ‘isn’t it’ (\ili{German} \textit{is(t) es nich(t) so?}) attached to the end of a \isi{declarative sentence} (\citealt{Nieuweboer1999}: passim).

Indirect \isi{polar question}s require a special marker, \ili{English} \textit{if} or \textit{whether}, \ili{Plautdiitsch} \textit{ous}, or \ili{German} \textit{ob}. Interestingly, \ili{English} \textit{whether} historically derives from the PIE \isi{interrogative} *\textit{kʷ}\textit{óteros} ‘which of two’ (see \sectref{sec:5.5.3.1}), \ili{German} \textit{ob} and \ili{English} \textit{if} show connections with conditionals, but the etymology of \ili{Plautdiitsch} \textit{ous} is not perfectly clear. \ili{Yiddish} has adopted the use of \textit{ci} in \isi{indirect questions} from \ili{Slavic}.

\ea%13
    \label{ex:indo:13}
    \ili{Altai Low German}\\
    \gll \textbf{{ous}} ät‘  siinə    fryy  ka\textsuperscript{u}n      heilə\\
    if  1\textsc{sg}  3\textsc{sg}.\textsc{gen}.\textsc{f}  wife  can.?\textsc{ind}.\textsc{prs.}1\textsc{sg}  cure.\textsc{inf}\\
    \glt ‘whether I can cure his wife’ \citep[170]{Jedig2014}\footnote{The \isi{word order} of \ili{Plautdiitsch} is also possible in \ili{German} but sounds very archaic. A \ili{German} equivalent would be something like the following: \textit{ob ich seine Frau heilen kann}.}
    \z

In \ili{German} an embedded \isi{polar question} can also stand on its own to form a question usually addressed to oneself and roughly meaning ‘I wonder’.

\ea%14
    \label{ex:indo:14}
    \ili{German}\\
    \gll (Ich  frage      mich,) \textbf{{ob}} es \textbf{wohl} regnen    wird.\\
    1\textsc{sg}  ask.\textsc{prs}.\textsc{ind}.1\textsc{sg}  1\textsc{sg.acc}  if  it  perhaps rain.\textsc{inf}  will.\textsc{prs}.\textsc{ind}.3\textsc{sg}\\
    \glt ‘I wonder whether it will rain.’
    \z

Indirect content \isi{questions} have an \isi{interrogative} instead of the mentioned indirect question marker found in polar, \isi{focus}, or \isi{alternative question}s. Indirect \isi{content question}s in \ili{German} and \ili{English} have the \isi{interrogative} in initial position, but have a different \isi{word order} from plain \isi{content question}s. In \ili{German}, verbs are strictly final in both types of indirect questions.

\ea%15
    \label{ex:indo:15}
    \ili{German}\\
    \gll (Ich  frage      mich,) \textbf{{wer}} das \textbf{{wohl}} \textbf{{ist}}?\\
    1\textsc{sg}  ask.\textsc{prs}.\textsc{ind}.1\textsc{sg}  1\textsc{sg.acc}  who  that  perhaps  is\\
    \glt ‘I wonder who that will be.’
    \z

Indirect content \isi{questions} may also be used on their own for self \isi{questions}. Both types of indirect \isi{questions} are almost obligatorily accompanied with the modal marker \textit{wohl} (cognate with \ili{English} \textit{well}).

\subsubsection{Question marking in Slavic}\label{sec:5.5.2.3}

Most \ili{Slavic} languages have a second position \isi{polar question} clitic \textit{=li} (\citealt{SussexCubberley2006}: 359). In \ili{Russian} it is found especially in the written language. It also marks \isi{focus question}s, in which case the focused element instead of the verb has to take sentence-initial position.

\ea%16
    \label{ex:indo:16}
    \ili{Russian}\\
    \ea
    \gll otvétil=\textbf{{li}} studént    na  vsé  voprósy?\\
    answer.\textsc{m}.\textsc{sg}.\textsc{pst}=\textsc{q}  student    to  all  questions\\
    \glt ‘Did the student answer all the questions?’

    \ex
    \gll studént=\textbf{li} otvétil      na  vsé  voprósy?\\
    student=\textsc{q}  answer.\textsc{m}.\textsc{sg}.\textsc{pst}  to  all  questions\\
    \glt ‘Was it the student who answered all the questions?’ (\citealt{SussexCubberley2006}: 359)
    \z
    \z

\noindent Only some languages lack the clitic but have a sentence-initial particle instead, including \ili{Ukrainian} \textit{čy}/\textcyrillic{чи}, Belorussian \textit{ci}/\textcyrillic{ці}, and Polish \textit{czy}, which has been borrowed by \ili{Yiddish}.

\ea%17
    \label{ex:indo:17}
    \ili{Ukrainian}\\
    \gll \textbf{{čy}} zdoróvyj  tý?\\
    \textsc{q}  healthy    2\textsc{sg}\\
    \glt ‘Are you well?’ (\citealt{SussexCubberley2006}: 359)
    \z

\noindent In \ili{Ukrainian} there is a sharp rise at the end of the sentence in \isi{polar question}s, or over the focused element in \isi{focus question}s \citep[978]{Shevelov1993}. But this is less pronounced if the question is already marked with \textit{cy}. In \isi{focus question}s it is also possible to move the focused element into sentence-initial position.

\ea%18
    \label{ex:indo:18}
    \ili{Ukrainian}\\
    \ea
    \gll \textbf{{cy}} ty tam buv?\\
    \textsc{q}  2\textsc{sg}  there  were\\
    \glt ‘Were you there?’
    
    \ex
    \gll ty buv tam?\\
    2\textsc{sg}  were  there\\
    \glt ‘Were you there?’
    
    \ex
    \gll \textbf{{buv}} ty tam?\\
    were  2\textsc{sg}  there\\
    \glt ‘\textit{Were} you there?’
    
    \ex
    \gll \textbf{{tam}} ty buv?\\
    there  2\textsc{sg}  were\\
    \glt ‘Were you \textit{there}?’ \citep[978]{Shevelov1993}\z\z

Interrogatives are usually fronted in both languages but not necessarily so in \ili{Russian}. Content questions remain unmarked in both languages.

\ea%19
    \label{ex:indo:19}
    \ili{Russian}\\
    \gll \textbf{{čto}} eto?\\
    what  this\\
    \glt ‘What is this?’ \citep[23]{Comrie1984}
    \z

\ea%20
    \label{ex:indo:20}
    \ili{Ukrainian}\\
    \gll \textbf{{de}} ty buv?\\
    where  2\textsc{sg}  were\\
    \glt ‘Where were you?’ \citep[979]{Shevelov1993}
    \z

\noindent In \ili{Russian} the intonational nucleus can most often be found on the \isi{interrogative} itself \citep[24]{Comrie1984}.

Alternative questions in \ili{Russian} and \ili{Ukrainian} are quite different from each other. \ili{Ukrainian} uses the polar \isi{question marker} in between the two alternatives. Given its syntactic behavior in \isi{polar question}s, it may perhaps be said to attach to the beginning of the second alternative. In \ili{Russian}, on the other hand, there is a \isi{disjunction} \textit{ili} ‘or’.

\ea%21
    \label{ex:indo:21}
    \ili{Ukrainian}\\
    \gll ty buv u teatri, \textbf{{čy}} v muzej{i?}\\
    2\textsc{sg}  were  at  theater    \textsc{q}  at  museum\\
    \glt ‘Were you at the theater or in the museum?’ \citep[978]{Shevelov1993}
    \z

\ea%22
    \label{ex:indo:22}
    \ili{Russian}\\
    \gll vy  xotite  čaj, \textbf{{ili}} kofe?\\
    2\textsc{pl}  want  tea    or  coffee\\
    \glt ‘Do you want tea or coffee?’ \citep[23]{Comrie1984}
    \z

\noindent In \ili{Russian} the first alternative takes neutral \isi{polar question} \isi{intonation} with a sharp rise and immediate less sharp fall on \textit{čaj}, the second alternative has falling \isi{intonation} similar to an \isi{interrogative} in \isi{content question}s. \ili{Ukrainian} negative alternative \isi{questions} take \textit{čy ny}/\textcyrillic{чи ни} ‘\textsc{q} \textsc{neg}’ (\citealt{PughPress1999}: 285). In \ili{Russian} the form \textit{ili net}/\textcyrillic{или нет} is used (elicited in June 2017).

\ili{Russian} uses \isi{question tag}s less frequently than \ili{English} or \ili{German}. But one possibility is to attach \textit{ne pravda li} ‘\textsc{neg} truth \textsc{q}’ to a \isi{declarative sentence} \citep[32]{Comrie1984}. \ili{Russian} furthermore has the \isi{question marker} \textit{razve} that may stand sentence-initially and less frequently sentence-finally or be adjacent to the \isi{focus}. \cite[21f.]{Comrie1984} describes the use of \textit{razve} as follows: “the questioner had a certain prior expectation; some piece of new information leads the questioner to believe that his prior expectation may be wrong; therefore he asks the appropriate general question with \textit{razve}”.

\newpage 
\ea%23
    \label{ex:indo:23}
    \ili{Russian}\\
    \gll \textbf{{razve}} ty  uezžaeš?\\
    \textsc{q}  2\textsc{sg}  leave\\
    \glt ‘Are you leaving?’ \citep[22]{Comrie1984}
    \z

\noindent Intonation is again similar to plain \isi{polar question}s. \isi{Topic questions} are introduced with the conjunction \textit{a} ‘and’, e.g. \textit{a viktor?} ‘what about Victor?’ (\citealt{Comrie1984}: 27f.).

Questions in \textit{\ili{Chinese Pidgin Russian}} seem to be generally unmarked. Interrogatives remain \isi{in situ}. Interestingly, the \ili{Chinese} A-not-A pattern is also possible, e.g. \textit{pravda ne pravda?} ‘true \textsc{neg} true’ \citep[39]{Shapiro2010}, cf. \ili{Mandarin} \textit{duì-bu-duì?}.

\ea%24
    \label{ex:indo:24}
    \ili{Chinese Pidgin Russian}\\
    \ea
    \gll za vashe    zh’onusheki  mes’aca  posidi  esa?\\
    \textsc{top}  2\textsc{sg.gen}  wife.?\textsc{pl}  together  sit  \textsc{rel}\\
    \glt ‘Do you sit together with your wives?’
    \footnote{The topic marker \textit{za} stems from \ili{Mongolic} \citep[35]{Shapiro2010}.}

    \ex
    \gll ni-dy \textbf{{šýma}} múr.mur?\\
    2\textsc{sg-gen}  what  say\\
    \glt ‘What are you saying?’ (\citealt{Shapiro2010}: 37, 15)
    \footnote{Cf. \ili{Mandarin} \textit{n\u{\i}-de} \zh{你的} ‘2\textsc{sg-gen}’, \textit{shénme} \zh{什么} ‘what’.}
    \z
    \z

In \textit{\ili{Taimyr Pidgin Russian}} \isi{polar question}s may also be unmarked. But perhaps there is a special \isi{intonation} contour in both \isi{pidgins}.

\ea%25
    \label{ex:indo:25}
    \ili{Taimyr Pidgin Russian}\\
    \gll tebja  urusé-to  jest?\\
    2\textsc{sg}  rifle-\textsc{hl}  \textsc{ex}\\
    \glt ‘Do you have a rifle?’ \citep[312]{Stern2005}
    \z

\noindent The suffix \textit{-to} glossed as “highlighter” usually has a discourse function and is of northern \ili{Russian} origin (see \citealt{Stern2005}: 309; 2012: 439). Content questions are unmarked and interrogatives are either sentence-initial or preverbal \citep[508]{Stern2012}.

\ea%26
    \label{ex:indo:26}
    Taimyr Pidgin\\
    \gll \textbf{{čego}} tebja  nado  menja  čum  mesto?\\
    what  2\textsc{sg}  \textsc{deon}  1\textsc{sg}  tent  place\\
    \glt ‘What do you want in my tent?’ \citep[361]{Stern2012}
    \z

\noindent The polyfunctional \isi{case} marker \textit{mesto}, from a noun meaning ‘place’, is an innovation of the pidgin (\citealt{Stern2012}: 360–382). There is an instance of an open \isi{alternative question} that combines a \isi{disjunction}, an \isi{interrogative}, and a (polar) \isi{question marker} in that order.

\newpage 
\ea%27
    \label{ex:indo:27}
    \ili{Taimyr Pidgin Russian}\\
    \gll ty  govorit,  mama  əmədja    ne  puskat  budem, \textbf{ili}  \textbf{čego=li}?\\
    2\textsc{sg}  say.3\textsc{sg} mother  hither    \textsc{neg}  let.\textsc{inf}  \textsc{aux.fut} or  what=\textsc{q}\\
    \glt ‘Are you saying that we should not let mother come here or what?’ \citep[310]{Stern2005}
    \z

\noindent The use of a \isi{disjunction} may be traced to \ili{Russian} influence, but the presence of the \ili{Russian} polar \isi{question marker} after an \isi{interrogative} might be local influence. Compare \ili{Nganasan} open alternative \isi{questions} in which the second part takes a similar form, e.g. \textit{maa-ŋu-} ‘what-\textsc{aor}.\textsc{q-}’ (\sectref{sec:5.12.2}). However, similar phenomena are also known from \ili{Russian}. For example, the following sentence was recorded in the Allaikhovsky district in the north of the Sakha Republic: \textit{A, zdec’} \textbf{\textit{čto}}\textit{=}\textbf{\textit{li}}? ‘So here or what?’ \citep{Krasovitsky2004}

\subsubsection{Question marking in Iranian}\label{sec:5.5.2.4}

Polar questions in the extinct \ili{Iranian} language \textbf{Saka} are unmarked except for, perhaps, \isi{intonation}. Negative \isi{alternative question}s are marked with a \isi{disjunction} \textit{aa} that later developed into \textit{o}, followed by the negator \textit{ne}. Content questions seem to have remained unmarked and interrogatives were fronted \citep[402]{Emmerick2009}. Optionally a “discourse initiator” (e.g., \textit{tta} ‘thus, so’) could precede an \isi{interrogative} \citep[403]{Emmerick2009}, which is typologically similar to initial \ili{Mandarin} \textit{nà} \zh{那} ‘that’ or \ili{English} \textit{so}.

\ea%28
    \label{ex:indo:28}
    \ili{Saka} (Tumshuqese)\\
    \gll uv\=as\=anu    saṃv\=aru    paitṛyai    p\=atanäya?\\
    laywoman.\textsc{gen.pl}  regulations.\textsc{acc.sg}  can.2\textsc{sg.prs.ind}  keep.\textsc{inf}\\
    \glt ‘Can you keep the regulations of the laywomen?’ (\citealt{Emmerick1985}: 10f., 14)
    \z

\ea%29
    \label{ex:indo:29}
    \ili{Saka} (Khotanese)\\
    \gll \textbf{{cuuḍe}} bremä?\\
    why  weep.2\textsc{sg.ind}\\
    \glt ‘Why are you weeping?’ \citep[402]{Emmerick2009}
    \z

\textbf{Sogdian} has an optional sentence-initial polar \isi{question marker} \textit{(ə)ču-t(i)} ‘what-\textsc{comp}’ (\citealt{Yoshida2009}: 316f.). Negative \isi{alternative question} seem to have the same marker at the beginning of the whole sentence in \isi{combination} with a marker \textit{kataar(-əti)} ‘which(-\textsc{comp)}’ between the two alternatives. \ili{Sogdian} thus has two question markers that derive from interrogatives. \ili{Sogdian} \textit{kataar}, like \ili{English} \textit{whether}, derives from PIE *\textit{kʷ}\textit{ótero-} ‘which of two’ (\sectref{sec:5.5.3}). Content \isi{questions} have no special marking. \isi{Rhetorical questions} have in addition a marker \textit{pnuukar}. Interrogatives remain \textit{in situ}.

\ea%30
    \label{ex:indo:30}
    \ili{Sogdian}\\
    \ea
    \gll \textbf{{ə.ču.ti}} \textbf{{pnuukar}} tawa    waanoo  nee  pat$\gamma $oošti?\\
    \textsc{q}  \textsc{q.rhet}    by.you    thus    \textsc{neg}  heard.\textsc{pret}\\
    \glt ‘Have you never heard this?’
    
    \ex
    \gll \textbf{{ču.ti}} xa  zaakt    ta$\delta $ee$\delta $  a$\gamma $atant \textbf{{kataar.əti}} \textbf{{nee}}?\\
    \textsc{q}  the  children  there  came    \textsc{q}    \textsc{neg}\\
    \glt ‘Have the children arrived at your place or not?’
    
    \ex
    \gll ta$\gamma $u  peernamstar \textbf{{ču}} əktya  k$\theta $aare?\\
    2\textsc{sg}  before    what  deed  do.\textsc{pret}.2\textsc{sg}\\
    \glt ‘What was it that you did before?’ \citep[317]{Yoshida2009}
    \z
    \z 

\noindent There are initial \isi{question marker}s in \ili{Persian} (\textit{aayaa}) and \ili{Tajik} (\textit{oyo}) as well, but these show no connection with interrogatives (\citealt{WindfuhrPerry2009}: 438). \ili{Tajik} may also employ the \ili{Uzbek} sentence-final marker \textit{=mi} (\citealt{WindfuhrPerry2009}: 481, \sectref{sec:5.11.2}).

For \textbf{Sarikoli} there is a relatively old description by \citet[29]{Shaw1876}. According to him, \isi{polar question}s in \ili{Sarikoli} have a sentence-final marker \textit{â}, while \isi{content question}s remain unmarked. This marker, according to \citet{GaoErqiang1985}, has the form \textit{o} and has been reanalyzed here as an enclitic.

\ea%31
    \label{ex:indo:31}
    \ili{Sarikoli}\\
    \gll boʃa=af  tag    wand=\textbf{{o}}?\\
    \textsc{pn}=2\textsc{pl}  actually  see.\textsc{pst}=\textsc{q}\\
    \glt ‘Have (you) actually seen Bosha?’ (\citealt{GaoErqiang1985}: 62)
    \z

\noindent The same marker appears twice in \isi{alternative question}s. The following example contains in addition an element \textit{naji} in between the two alternatives that was glossed as a negator but seems to have the function of a \isi{disjunction} here.

\ea%32
    \label{ex:indo:32}
    \ili{Sarikoli}\\
    \gll maʃ  tu$\chi $ɯ   $\chi $or-an=\textbf{{o}} \textbf{{naji}} wi  budo  $\chi $or-an=\textbf{{o}}?\\
    1\textsc{pl}  chicken  eat-1\textsc{pl}=\textsc{q}  \textsc{neg}  that  beef  eat-1\textsc{pl}=\textsc{q}\\
    \glt ‘Do we eat chicken or do we eat that beef?’ (\citealt{GaoErqiang1985}: 65)
    \z

According to \citet[118]{GaoErqiang1985}, \ili{Wakhi} has a \isi{disjunction} \textit{jo} and \isi{question marking} on the first alternative only (\textit{=a}), which is a construction similar to surrounding languages such as \ili{Uyghur} (\sectref{sec:5.11.2}). In \ili{Sarikoli} there are several \isi{tag question} markers that contain the same \isi{question marker}, e.g. \textit{na sou-d=o?} ‘\textsc{neg} be.possible-3\textsc{sg}=\textsc{q}’ (\citealt{GaoErqiang1985}: 90).

\ea%33
    \label{ex:indo:33}
    \ili{Sarikoli}\\
    \gll tudʒik  ziv    ati  wazon-d, \textbf{{rust=o}}?\\
    \textsc{pn}  language  ?  know.\textsc{n.pst}-3\textsc{sg}  true=\textsc{q}\\
    \glt ‘(She) knows \ili{Sarikoli}, right?’ (\citealt{GaoErqiang1985}: 89)
    \z

\noindent The \isi{question marker} seems to be connected with \ili{Burushaski}, e.g. \textit{bás=}\textbf{\textit{a}}? ‘Is it enough?’, and several surrounding languages \citep[190]{Yoshioka2012}. Consider an example from the \ili{Dardic} language \ili{Palula} spoken in the extreme north of Pakistan where the marker, depending on the dialect, takes the form \textit{=aa} or \textit{=ee}.

\ea%34
    \label{ex:indo:34}
    \ili{Palula} (\ili{Dardic}, \ili{Indo-European})\\
    \gll búd-u=\textbf{{ee}}?\\
    understand.\textsc{pfv}-\textsc{msg}=\textsc{q}\\
    \glt ‘Did you understand?’ \citep[403]{Liljegren2016}
    \z

Content questions in \ili{Sarikoli} are unmarked and interrogatives seem to remain \textit{in situ}.

\ea%35
    \label{ex:indo:35}
    \ili{Sarikoli}\\
    \gll mɯ    vrud    tar \textbf{{ko}}?\\
    1\textsc{sg.gen}  brother    \textsc{ex}  where\\
    \glt ‘Where is my brother?’ (\citealt{GaoErqiang1985}: 86)
    \z

\subsubsection{Question marking in Tocharian}\label{sec:5.5.2.5}

Tocharian has unmarked \isi{polar question}s \REF{ex:indo:36}, but perhaps had a special \isi{intonation} contour that cannot be reconstructed.

\ea%36
    \label{ex:indo:36}
    \ili{Tocharian B}\\
    \gll ate  kamp\=al    yamaṣasta?\\
    away  coat.\textsc{acc}  do.\textsc{pret}.2\textsc{sg}\\
    \glt ‘Have you put (your) coat away?’ \citep[110]{Hackstein2013}
    \z

In \ili{Tocharian A} there are two optional \isi{question marker}s, second position \textit{=te} and sentence-final \textit{aśśi} that may be found together in one sentence.

\ea%37
    \label{ex:indo:37}
    \ili{Tocharian A}\\
    \gll yn\=alek=\textbf{{te}} lo  kälk \textbf{{aśśi}}?\\
    elsewhere=\textsc{q}  away  go.\textsc{prt}.3\textsc{sg}  \textsc{q}\\
    \glt ‘Has (s)he gone somewhere else?’ \citep[111]{Hackstein2013}
    \z

\noindent According to \citet[175]{Hackstein2004} \textit{aśśi} derives from PIE *\textit{h\textsubscript{2}}\textit{et} + *\textit{kʷ}\textit{ih\textsubscript{1}} (cf. \ili{Latin} \textit{atqu\={\i}}), of which the latter part is an instrumental form of an \isi{interrogative} that is also the source of, for example, Polish \textit{czy} (cf. \ili{Latin} \textit{qu\={\i}} ‘how’). The first part *\textit{h\textsubscript{2}}\textit{et}, or *\textit{h\textsubscript{a}}\textit{et} with one of the laryngeals \textit{h\textsubscript{2}} or \textit{h\textsubscript{4}} according to \cite[289ff.]{MalloryAdams2006}, is a preposition meaning ‘away, beyond’ (e.g., \ili{Tocharian B} \textit{at(e)} ‘away’). Hackstein assumes that \textit{aśśi} started out as a \isi{question tag} similar to \ili{German} \textit{wie} ‘how’ or \textit{was} ‘what’. Content questions may be unmarked in both \ili{Tocharian B} (e.g., \citealt{Adams2013}: 157) and \ili{Tocharian A}, e.g. \textbf{\textit{kus}} \textit{täm?} ‘Who is that?’ \citep[156]{Carling2009}. The particle \textit{aśśi} can also be encountered in \isi{content question}s and sometimes fuses with interrogatives, e.g. \textit{t\=a}, \textit{t\=aśśi} ‘where’ (\citealt{SiegSiegling1931}: 182). In some instances the last consonant of the \isi{interrogative} is lengthened, e.g. \textit{kus}, \textit{kuss aśśi} ‘who, what’ (\citealt{SiegSiegling1931}: 190). In \ili{Tocharian A} some interrogatives are also often followed by \textit{pat (nu)}, the exact function of which reamains unclear to me, e.g. \textit{kus pat nu} ‘or what now’ \citep[156]{Carling2009}. In \isi{alternative question}s one marker appears on each alternative in \ili{Tocharian B}.

\ea%38
    \label{ex:indo:38}
    \ili{Tocharian B}\\
    \gll pañäkte=\textbf{{wat}} yopsa, n\=ande{=}\textbf{{wat}}?\\
    \textsc{pn.}\textsc{nom=q}  enter.\textsc{pret}.3\textsc{sg}   \textsc{pn.}\textsc{nom=q}\\
    \glt ‘Has Buddha or Nanda (just) entered?’ \citep[110]{Hackstein2013}
    \z

Interestingly, \citet[111]{Hackstein2013} also has an example of a negative \isi{alternative question} that has a \isi{disjunction}, \ili{Tocharian B} \textit{epe m\=a?} ‘or \textsc{neg}’. The same \isi{disjunction} can optionally also be found in plain alternative \isi{questions} \citep[111]{Hackstein2013}. \ili{Tocharian A} has negative alternative \isi{questions} with the marker \textit{=te} used once on each alternative. In the second alternative it attaches to the negator, \textit{m\=a=te}. In one such example there is an additional element \textit{na} in the second alternative, glossed as a \isi{question marker} by Hackstein.

\ea%39
    \label{ex:indo:39}
    \ili{Tocharian A}\\
    \gll cämpäl=\textbf{{te}} nasan    cesäm    wrasaśśi  waste mäskatsi, \textbf{{m\=a=te}} cämpäl      (\textbf{{na}})  sam?\\
    be.able.\textsc{ger}2.\textsc{nom}=\textsc{q}  \textsc{cop.prs}.1\textsc{sg}  this.\textsc{gen.pl}  being.\textsc{gen.pl}  refuge be.\textsc{inf}    \textsc{neg=q}  be.able.\textsc{ger}2.\textsc{nom}  (\textsc{q})  \textsc{cop.prs.1sg}\\
    \glt ‘Am I able to provide refuge to the beings or am I not able?’ \citep[113]{Hackstein2013}
    \z

\noindent The \isi{double marking} strategy optionally combines with the \isi{disjunction} \textit{epe} ‘or’.

\ea%40
    \label{ex:indo:40}
    \ili{Tocharian A}\\
    \gll \textbf{{m\=a=(t)e}} n\=atäk  cam    br\=a(maṃ) \textbf{{e(pe)}} \textbf{m\=a=(t)e} was?\\
    \textsc{neg=q}    master  this.\textsc{acc.sg}  \textsc{pn}    or  \textsc{neg=q} 1\textsc{pl.acc}\\
    \glt ‘Will the master not keep this Brahman or will he not keep us?’ \citep[113]{Hackstein2013}
    \z


\subsubsection{Summary}\label{sec:5.5.2.6}

Question marking in \ili{Indo-European} is very different from the majority of languages in \isi{NEA}. As expected for a family with such a long history, the marking of \isi{questions} varies strongly from language to language. \ili{Even} within the relatively shallow \ili{Slavic} branch there are marked differences. Generally there is a \isi{tendency} for initial particles or second position clitics and disjunctions. To the best of my knowledge, almost all \ili{Indo-European} languages included here have unmarked content \isi{questions}. Interestingly, at least four languages (\ili{German} \textit{wie}, \textit{was}, \ili{Ukrainian} \textit{čy} (hence \ili{Yiddish} \textit{ci}), \ili{Sogdian} \textit{(ə)ču-t(i)}, \textit{kataar(-əti)}, and \ili{Tocharian A} \textit{aśśi}) show a development from \isi{interrogative} to polar \isi{question marker} and/or \isi{question tag}, which is quite unusual for \isi{NEA} (but see \sectref{sec:5.12.2} on \ili{Selkup}). This is also known from other \ili{Indo-European} languages, such as \ili{Sanskrit} \textit{kad} ‘what’ \citep[100]{Hackstein2013}, \ili{Bengali} \textit{ki} ‘what’ (\citealt{Thompson2012}, see \sectref{sec:4.2.1}), or \ili{Palula} \textit{ga} ‘what’ (\citealt{Liljegren2016}, \sectref{sec:4.2.1}, \sectref{sec:4.2.3}).

\begin{table}
\caption{Summary of question marking in \ili{Indo-European} languages}
\label{tab:indo:1}

\begin{tabularx}{\textwidth}{QQQl}
\lsptoprule
& \textbf{PQ} & \textbf{CQ} & \textbf{AQ}\\
\midrule
\ilit{English} & \#V + (to do) & (to do) & \#V + (to do) + or\\
\ilit{German} & \#V & - & \#V + oder ‘or’\\
\ilit{Plautdiitsch} & \#V & - & ?\#V + öuda ‘or’\\
\ilit{Yiddish} & \#ci V & - & \#V + odər ‘or’, \#V + X ci Y\\
\ilit{Ukrainian} & \#čy & - & X čy Y\\
\ilit{Russian} & -, \#V=li & - & ili ‘or’\\
\ili{Taimyr Pidgin Russian} & - & - & ?ili ‘or’ + =li\\
CPR & - & - & ?\\
\ilit{Sogdian} & \#(ə)ču.t(i) & - & \#(ə)ču.t(i) + X kataar(-əti) Y\\
Khotanese & - & - & aa (> o) ‘or’\\
Tumshuqese & - & - & o ‘or’\\
\ilit{Sarikoli} & =o\# & - & 2x=o\# + ?naji ‘\textsc{neg}’\\
\ilit{Tocharian A} & \#A=te, ?aśśi\# & -, ?aśśi & 2x =te (+ epe ‘or’)\\
\ilit{Tocharian B} & - & - & 2x =wat, epe ‘or’\\
\lspbottomrule
\end{tabularx}
\end{table}

\subsection{Interrogatives in \ili{Indo-European}}\label{sec:5.5.3}

\subsubsection{Interrogatives in Proto-{Indo-European}}\label{sec:5.5.3.1}

A somewhat outdated, but nevertheless useful, typological classification of {Indo-Euro\-pean} is in so-called \textit{centum} and \textit{satəm} languages. The designation follows the \ili{Latin} and \ili{Avestan} words for ‘hundred’, respectively, that represent the two types. The two groups are divided by their reflexes of \ili{Proto-\ili{Indo-European}} velar, palatal and labiovelar consonants. In \textit{centum} languages the palatals and in \textit{satəm} languages the labiovelars became plain velars (\tabref{tab:indo:2}). PIE *\textit{ḱ\d{m}}\textit{tom} ‘hundred’ starts with a palatal and thus remained a palatal in \ili{Iranian} (and later changed to \textit{s} in \ili{Avestan}) but became a plain velar written as <c> in \ili{Latin} (cf. \citealt{Fortson2010}: 146). The languages in this study belong to both the \textit{centum} (\ili{Germanic}, Tocharian) and \textit{satəm} (\ili{Iranian}, \ili{Slavic}) types.

This division is important for the purposes of this study, because PIE interrogatives usually began with the labiovelar *\textit{kʷ} that was preserved in \ili{Germanic} and Tocharian, but changed to plain velars in (Indo-)\ili{Iranian} and (Balto-)\ili{Slavic}. In Tocharian the labiovelars were later mostly lost in favor of plain velars. However, in some instances they show reflexes, e.g. \ili{Tocharian A} \textbf{\textit{ku}}\textit{s}, B \textbf{\textit{k}}\textbf{\textit{\textsubscript{u}}}\textit{se} ‘who’. In \ili{Germanic} *\textit{kʷ} regularly changed to *\textit{hʷ}, e.g. Old \ili{English} \textbf{\textit{hw}}\textit{\=a} or Old High \ili{German} \textbf{\textit{(h)w}}\textit{er} ‘who’. \ili{German} later entirely lost the initial consonant, e.g. \ili{German} \textbf{\textit{w}}\textit{er} ‘who’. In \ili{English} the development is more complicated. The modern spelling preserves the original <hw> with metathesis as <wh>, but the pronunciation varies between /h/ (\textbf{\textit{wh}}\textit{o}) and /w/ (\textbf{\textit{wh}}\textit{at}). In \ili{Slavic} as well as \ili{Iranian}, plain velars were palatalized before front vowels such as \textit{i} but otherwise remained stable, e.g. \ili{Russian} \textbf{\textit{k}}\textit{to} ‘who’ (PIE *\textit{kʷ}\textit{o-} + *\textit{tod}) but \textbf{\textit{č}}\textit{to} ‘what’ (PIE *\textit{kʷ}\textit{i/e-} + *\textit{tod}), or \ili{Avestan} \textbf{\textit{k}}\textit{as(ə)-} etc. ‘who’ (PIE \textit{*kʷ}\textit{ó-s}) but \textbf{\textit{c}}\textit{iš} ‘who’ (PIE \textit{*kʷ}\textit{í-s}) (e.g., \citealt{Fortson2010}: 231f., 421).

\begin{table}
\caption{Developments of PIE velars \citep[58]{Fortson2010}}
\label{tab:indo:2}

\begin{tabularx}{\textwidth}{XXl}
\lsptoprule

 \textbf{centum <} & \textbf{PIE} & \textbf{> satəm}\\
 \midrule 
 *k & *ḱ & *ḱ\\
& *k & *k\\
 *kʷ & \textbf{*kʷ} & \\
\lspbottomrule
\end{tabularx}
\end{table}

PIE interrogatives can be reconstructed as \textit{*kʷ}\textit{o-}, \textit{*kʷ}\textit{e-}, \textit{*kʷ}\textit{i-}, and \textit{*kʷ}\textit{u-} (e.g., \citealt{CysouwHackstein2011}), but a controversy concerns the status of the last of them. \citet[436--441]{Dunkel2014} argues that it must be reconstructed as *\textit{kú-} ‘where’ and thus does not belong to the group of interrogatives starting with *\textit{kʷ}\textit{-}. According to him, the forms beginning with *\textit{kʷ}\textit{-} were actually derived from *\textit{kú} in the first place \citep[436]{Dunkel2014}. In his \isi{analysis}, \textit{*kʷ}\textit{í-} and \textit{*kʷ}\textit{é-} are combinations of *\textit{kú-} with anaphoric stems *\textit{í-} and *\textit{e-}, while the formation of \textit{*kʷ}\textit{ó-} is not solved entirely (see \citealt{Dunkel2014}: 478). Whether this hypothesis is correct cannot be decided here, but it seems possible. If it is accurate, interrogatives in \ili{Indo-European} are ultimately based on locative interrogatives as in \ili{German} (see \sectref{sec:5.5.3.2}).

\tabref{tab:indo:3} gives the list of interrogatives in PIE as reconstructed by \citet{MalloryAdams2006}. A more extensive discussion of interrogatives can be found in \citet[453-479]{Dunkel2014}. It cannot be given here in its entirety, although some of his reconstructions have been integrated into \tabref{tab:indo:3}. In some cases there is a corresponding demonstrative with the same endings, e.g. *\textit{to-deh\textsubscript{a}} ‘then’, *\textit{to-r} ‘there’, *\textit{to-ti} ‘so much/many’ etc. According to the authors, \textit{*kʷ}\textit{om} ‘when’ is a masculine accusative form of \textit{*kʷ}\textit{os} ‘who’, which seems extremely unlikely from a semantic point of view.

\begin{table}
\caption{Selected PIE interrogatives with some cognates according to \cite[419f.]{MalloryAdams2006}; extended with the help of \cite[436-441, 453-479]{Dunkel2014}; accents partly removed}
\label{tab:indo:3}

\begin{tabularx}{\textwidth}{llQ}
\lsptoprule

\textbf{PIE form} & \textbf{PIE meaning} & \textbf{Selected Cognates}\\
\midrule
*kʷós & who \textsc{nom.sg.m} & NE who, Grk toû, Skt kas, Got ƕas\\
*kʷís & who \textsc{nom.sg.an} & Lat quis, Grk tis, Av ciš\\
*kʷód & what \textsc{nom/acc.sg.n} & Lat quod, NE what, Skt kad\\
*kʷíd & what \textsc{nom/acc.sg.n} & Lat quid\\
*kʷóm & ?when \textsc{acc.sg.m} & Got ƕan, OCS ko-gda, (Lat cum)\\
*kʷíh\textsubscript{1} & how \textsc{inst} & Lat qu\={\i}, AE hw\={\i}, \ili{Polish} czy\\
*kʷóteros & which (of two) & Grk poteros, Skt katara-, OCS koteryj\u{\i}\\
*kʷodéh\textsubscript{a} & when & Skt kad\=a\\
*kʷor & where & Lat qu\=or, OHG hw\=ar, Skt karhi\\
*kʷóti & how much/many & Lat quot, Grk posos, Skt kati\\
*kʷéti & how much/many & Av čaiti, \ili{Breton} pet der ‘how many days’\\
*kʷeh\textsubscript{a}k- & what kind of & [North West] OCS kakŭ, \ili{Lithuanian} kok(i)s\\
*kʷoli & ?how much & only OCS kolikŭ ‘how large’, kol\u{\i} ‘how much’\\
*kʷu(ú), ?*kú & where & Lat ubi, Grk pu-, Skt k\=u\\
*kʷu-d\textsuperscript{h}e & where & OCS kŭde, OAv kud\=a, Skt kuha\\
\lspbottomrule
\end{tabularx}
\end{table}

\ili{Proto-Indo-European} thus had K-interrogatives but no \isi{KIN-interrogative} (because of a missing nasal). Perhaps, the relative *\textit{yo-} goes back to an \isi{interrogative} as well but is not attested as such (\citealt{MalloryAdams2006}: 421). There may have been one \isi{interrogative} stem that did not start with *\textit{kʷ}\textit{-} but *\textit{m-}. It has been reconstructed as *\textit{me/o-} ‘who, which’ (\citealt{MalloryAdams2006}: 421\emph{\textup{;}} \citealt{CysouwHackstein2011}\emph{\textup{;}} \citealt{Dunkel2014}: 518-523), e.g. \ili{Tocharian A} \textit{mänt} ‘how’.

\subsubsection{Interrogatives in Germanic}\label{sec:5.5.3.2}

\tabref{tab:indo:4} gives the diachrony of several \ili{Germanic} interrogatives and their modern \ili{German} and \ili{English} cognates. For some additional discussion see also \citet{Dunkel2014}. PIE *\textit{kʷ}\textit{otero-} ‘which of two’ has lost its \isi{interrogative} meaning in \ili{German} \textit{weder} ‘neither’ and in \ili{English} \textit{whether}, used for indirect polar, \isi{focus}, and \isi{alternative question}s.

\ili{German}, \ili{Yiddish}, and \ili{Plautdiitsch} share a single \isi{resonance} in \textit{v{\textasciitilde}}, as \ili{German} <w> is pronounced as [v] as well (\tabref{tab:indo:5}). As mentioned before, \ili{English} has a variation between \textit{w{\textasciitilde}} and two forms starting with \textit{h-}. \isi{Altai} Low \ili{German} \textit{vənäiɐ} ‘when’ is closer to \ili{Dutch} \textit{wanneer} than to \ili{German} \textit{wann}. Similar to \ili{Dutch} \textit{waar} and \ili{English} \textit{where} it also retains a reflex of a final \textit{r} in \textit{vuuɐ}, while \ili{German} only preserves an older form \textit{wor-} in derived forms. But the form \textit{vou-} directly corresponds to \ili{German} \textit{wo-}. Also compare \ili{German} \textit{worau}\textbf{\textit{f}}, \ili{Dutch} \textit{waaro}\textbf{\textit{p}}, and \ili{Plautdiitsch} \textit{vourǫ}\textbf{\textit{p}} ‘on what’ as well as \ili{German} \textit{wa}\textbf{\textit{s}}, \ili{Dutch} \textit{wa}\textbf{\textit{t}}, and \ili{Plautdiitsch} \textit{vau}\textbf{\textit{t}} ‘what’. \ili{Yiddish} \textit{far vos} has direct correspondences in \ili{German} \textit{für was} and \ili{English} \textit{what for}. This is a common European formation, e.g. \ili{Italian} \textit{perché} ‘why’ (cf. \textit{per} ‘for’, \textit{che} ‘what’).

\begin{table}
\caption{Diachrony of selected German and English interrogatives (\citealt{Hackstein2004}: 175; \citealt{Seebold2002}; \citealt{MalloryAdams2006}: 419f.; and \citealt{Kroonen2013}: 261, 264)}
\label{tab:indo:4}
\small
\begin{tabularx}{\textwidth}{llQl}
\lsptoprule
\textbf{PIE} & \textbf{PG} & \textbf{Old Germanic} & \textbf{Modern Germanic}\\
\midrule
*kʷos \textsc{m} & *hʷaz & Got ƕas & -\\
*kʷis \textsc{m} & *hʷiz & OHG (h)wer & NHG wer\\
*kʷeh\textsubscript{2} \textsc{f} & *hʷ\=o & OE hw\=a & NE who\\
&  & ?OE h\=u & NE how\\
*kʷod \textsc{n} & *hʷat & Got ƕat, OHG (h)waz, OE hwæt & NHG was, NE what\\
*kʷotero- & *hʷaþera & Got ƕaþar, OHG (h)wedar, OE hwæðer & (NHG weder ‘neither’, NE whether)\\
*kʷor & *hʷar & Got ƕar, OHG (h)war, OE hw\={æ}r & NHG wo, NE where\\
*kʷom & ? & Got ƕan & -\\
& ? & OHG (h)wanne, (h)wenne, wenno, OE hwanne & NHG wann, (wenn ‘if’), NE when\\
*kʷih\textsubscript{1} & ? & OE hw\={\i}, hw\={y} & NE why\\
? & ? & Got ƕaiwa, OHG (h)wio & NHG wie ‘how’\\
\lspbottomrule
\end{tabularx}
\end{table}

\begin{table}
\caption{English, German (own knowledge), Yiddish (\citealt{Katz1987}: 197; \citealt{JacobsPrincevanderAuwera1994}: 404, 413-414, passim), and Altai Low German interrogatives (\citealt{Jedig2014}: passim); Plautdiitsch forms in square brackets from \citet{Nieuweboer1999}}
\label{tab:indo:5}

\begin{tabularx}{\textwidth}{Qlll}
\lsptoprule
\textbf{English} & \textbf{German} & \textbf{Yiddish} & \textbf{Plautdiitsch}\\
\midrule
who [h-] & wer [-eːɐ] & ver & veeɐ\\
how [h-] & wie [-iː] & vi & [vöu]\\
how much/many & wieviel(e) [-iː-] & vi fil(e) & [vöu fiel]\\
what & was & vos & vaut\\
which & welch-er [-ɐ] & velkher & [vöune-]\\
what kind of & was für [-ʏɐ] ein- [a-] & vos far a & \\
where & wo & vu & vuuɐ\\
wither, where to & wohin & vuhin & vuuɐhaan\\
whence, where from & woher [-eːɐ] &  & vuuɐheeɐ, vouheeɐ\\
when & wann & ven & vənäiɐ\\
why, how come, for what reason & wieso, weshalb, warum &  & vuurǫm\\
what for & wozu, für [-ʏɐ] was & tsu vos, far vos & \\
\lspbottomrule
\end{tabularx}
\end{table}

\ili{German} exhibits an interesting congruence of the two forms \textit{was} ‘what’ and \textit{wie} ‘how’ that in certain circumstances are mutually exchangeable.

\ea%41
    \label{ex:indo:41}
    \ili{German}\\
    \ea
    \gll \textbf{{Wie}}/\textbf{{Was}} ist   dein     Name?\\
    how/what  is  your.\textsc{m.sg}  name\\
    \glt ‘\isi{What is your name?}’
    
    \ex
    \gll Bist   du   fertig \textbf{{oder}} \textbf{{wie}}/\textbf{{was}}?\\
    are  2\textsc{sg}  ready  or  how/what\\
    \glt ‘Are you ready or what?’
    
    \ex
    \gll Du  bist  der    Neue, \textbf{{wie}}/\textbf{{was}}?\\
    you  are  the.\textsc{m.sg}  new.one  how/what\\
    \glt ‘You are the new one, aren’t you?’
    
    \ex
    \gll \textbf{{Wie}}/(\textbf{{Was}}) du  bist  schwanger?\\
    how/what  you  are  pregnant\\
    \glt ‘You are pregnant?’
    \z\z

\noindent \ili{English} cannot employ the \isi{interrogative} \textit{how} in these circumstances. The information on \ili{Yiddish} and \ili{Plautdiitsch} available to me is insufficient for a comparison.

\ili{German} \textit{was für ein-} is a complex \isi{interrogative} similar to \ili{English} \textit{what kind of}. Interestingly it is still separable as witnessed by the following examples.

\ea%42
    \label{ex:indo:42}
    \ili{German}\\
    \ea
    \gll \textbf{{Was\_für\_ein}} Urlaub    ist  das?\\
    what.kind  holiday    is  that\\
    
    \ex
    \gll \textbf{{Was}} ist  das \textbf{{für\_ein}} Urlaub?\\
    what  is  that  kind    holiday\\
    \glt ‘What kind of holiday is this?’
    \z
    \z

\noindent An analogous situation can be seen in \ili{Yiddish}.

\ea%43
    \label{ex:indo:43}
    \ili{Yiddish}\\
    \ea
    \gll \textbf{{vos}}\textbf{{\_}}\textbf{{far}}\textbf{{\_}}\textbf{{a}} yontev iz dos?\\
    what.kind  holiday    is  that\\
    
    \ex
    \gll \textbf{{vos}} iz dos \textbf{{far}}\textbf{{\_}}\textbf{{a}} yontev?\\
    what  is  that  kind    holiday\\
    \glt ‘What kind of holiday is this?’ (\citealt{JacobsPrincevanderAuwera1994}: 413)
    \z
    \z

For \isi{Altai} Low \ili{German} no cognate is attested (but cf. \ili{Dutch} \textit{wat voor een}). The conjugation of \textit{was für ein-} in \ili{German} is highly complex and depends on number, \isi{gender}, and \isi{case} (\tabref{tab:indo:6}). In the \isi{plural} the \isi{interrogative} \textit{welch-} ‘which’ substitutes for \textit{ein-} (cf. \textit{eins} ‘one’). Compare the full paradigm of the \isi{interrogative} \textit{welch-} ‘which’ (\tabref{tab:indo:7}). The genitive forms are rare, but are listed for the sake of completeness. 

\begin{table}
\caption{Conjugation of \textit{was für ein-} ‘what kind of’}
\label{tab:indo:6}

\begin{tabularx}{\textwidth}{XXXXl}
\lsptoprule

\textbf{Case} & \textbf{\textsc{sg.m}} & \textbf{\textsc{sg.f}} & \textbf{\textsc{sg.n}} & \textbf{\textsc{pl}}\\
\midrule
\textsc{nom} & ein-e & ein & ein(s) & Ø/welch-e\\
\textsc{acc} & ein-e & ein-en & ein & Ø/welch-e\\
\textsc{dat} & ein-er & ein-em & ein-em & Ø/welch-en\\
\textsc{gen} & ein-es & ein-er & ein-es & Ø/welch-er\\
\lspbottomrule
\end{tabularx}
\end{table}

\begin{table}
\caption{Conjugation of \textit{welch-} ‘which (one)’}
\label{tab:indo:7}

\begin{tabularx}{\textwidth}{XXXXl}
\lsptoprule

\textbf{Case} & \textbf{\textsc{sg.m}} & \textbf{\textsc{sg.f}} & \textbf{\textsc{sg.n}} & \textbf{\textsc{pl}}\\
\midrule
\textsc{nom} & welch-e & welch-\textbf{er} & welch-es & welch-e\\
\textsc{acc} & welch-e & welch-en & welch-\textbf{es} & welch-e\\
\textsc{dat} & welch-er & welch-em & welch-em & welch-en\\
\textsc{gen} & welch-es & welch-er & welch-es & welch-er\\
\lspbottomrule
\end{tabularx}
\end{table}

Both \textit{was für ein-/welch-} as well as \textit{welch-} may be used either pronominally or attributively. If used attributively and in the \isi{plural}, \textit{was für} may be used on its own. In the \isi{singular} there is the purely pronominal form \textit{was für eins} for the neuter instead of the attibutive form \textit{was für ein}. \ili{German} \textit{wie viel-}, \ili{Yiddish} \textit{vi fil(e)}, and \ili{Plautdiitsch} \textit{vöu fiel} are based on the same underlying pattern as \ili{English} \textit{how many}. The conjugation of \textit{wie viel-} exhibits the same \isi{case} markers as the \isi{plural} forms of \textit{welch-}. While \ili{English} employs \textit{how much} instead of \textit{how many} for mass nouns, \ili{German} \textit{wie viel} simply lacks \isi{inflection}.

In \ili{German}, \ili{Plautdiitsch}, \ili{Yiddish}, and \ili{English} the personal \isi{interrogative} shows a small paradigm. The interrogatives meaning ‘what’ do not show \isi{case} marking.

\begin{table}
\caption{German, Yiddish, Plautdiitsch, and English conjugation of the personal interrogative}
\label{tab:indo:8}

\begin{tabularx}{\textwidth}{XXXXl}
\lsptoprule

\textbf{Case} & \textbf{German} & \textbf{Plautdiitsch} & \textbf{Yiddish} & \textbf{English}\\
\midrule
\textsc{nom} & wer [-eːɐ] & veeɐ & ver & who [h-]\\
\textsc{acc} & wen [-eː-] & ? & vemen & whom\\
\textsc{dat} & wem [-eː-] & veem & vemen & whom\\
\textsc{gen} & wessen & ? & vemens & whose\\
\lspbottomrule
\end{tabularx}
\end{table}

Of these four languages only \ili{German} and perhaps \ili{Plautdiitsch} preserve four distinct forms, although \ili{German} \textit{wessen}, which as has an archaic variant \textit{wes}, is increasingly replaced with \textit{von wem} ‘of whom’. \ili{German} has a parallel paradigm and asymmetry of the definite article or demonstrative \textit{der} ‘that one, the.\textsc{m.sg}’: \textit{der}, \textit{den}, \textit{dem}, \textit{des(sen)}, but \textit{das} ‘that’.

\begin{table}
\caption{German interrogative and demonstrative paradigms}
\label{tab:indo:9}
\small
\begin{tabularx}{\textwidth}{ll@{}ll@{}lQ}
\lsptoprule

\textbf{where} & \textbf{there} &  & \textbf{here} &  & \textbf{Explanation}\\
\midrule
wo(r) & da(r) & hin & hier & her & plain\\
wo-bei & da-bei & - & hier-bei & - & ‘at, by, with’\\
wo-mit & da-mit & - & hier-mit & - & ‘with’\\
wo-nach & da-nach & - & hier-nach & - & ‘to, after’\\
wo-von & da-von & - & hier-von & - & ‘from, of’\\
wo-zwischen & da-zwischen & - & ?hier-zwischen & - & ‘between’\\
wo-hin & da-hin & - & hier-hin & - & ‘there’\\
wo-her & da-her & - & hier-her & - & ‘here’\\
wo-vor & da-vor & - & ?hier-vor & - & ‘in front’\\
wo-durch & da-durch & - & hier-durch & - & ‘through’\\
wo-zu & da-zu & - & hier-zu & - & ‘to’\\
wor.in & d(a)r.in & - & ?hier.in & - & ‘in’\\
wor.auf & d(a)r.auf & (hi)n.auf & hier.auf & (he)r.auf & ‘on, up’\\
wor.unter & d(a)r.unter & (hi)n.unter & hier.unter & (he)r.unter & ‘under’\\
wor.über & d(a)r.über & (hi)n.über & hier.über & (he)r.über & ‘over’\\
wor.aus & d(a)r.aus & (hi)n.aus & hier.aus & (he)r.aus & ‘out’\\
wor.ein & d(a)r.ein & (hi)n.ein & hier.ein & (he)r.ein & ‘in(to)’\\
wor.um, w\textbf{a}r.um & d(a)r.um & hin.um & hier.um & (he)r.um & ‘about, in order to’\\
\lspbottomrule
\end{tabularx}
\end{table}

\ili{Plautdiitsch} \textit{vuurǫm} is comparable to \ili{German} \textit{warum} ‘why’, which is based on MHG \textit{w\=ar + umbe} ‘where + around’. Several more forms in \ili{Plautdiitsch} such as \textit{vou-bii} (\ili{German} \textit{wo-bei}) have a locative basis. Unfortunately, only a few forms from \ili{Plautdiitsch} are attested, which is why \ili{German} forms are given instead (\tabref{tab:indo:9}). \ili{English} shares some of these formations, e.g. \textit{whereby}, \textit{thereby}, \textit{hereby} etc. In one group the \textit{r} is preserved but reanalyzed as belonging to the second element (\textit{wor-}\textit{um} > \textit{wo-}\textit{rum}, while in another group the \textit{r} was lost or at least is not present. This seems also to hold for \ili{Plautdiitsch}, e.g. \textit{vou}\textbf{\textit{r}}\textit{.ǫp} (\ili{German} \textit{wo}\textbf{\textit{r}}\textit{.auf}) and \textit{vou-fǫn} (\ili{German} \textit{wo-von}). Within the first group the vowel following the \textit{r} helped preserve it. Depending on the verb, some of these forms derived from 'where' may also just mean ‘what’, which is highly unusual from a typological perspective \citep{Cysouw2007}. Compare, for instance, \ili{English} \textit{to consist} [\textit{of what}] and \ili{German} [\textit{wor.}\textit{aus}] \textit{bestehen}. The development of the meaning of the individual forms is highly idiosyncratic. For example, \textit{wor.über} may either mean ‘over what place’ but also ‘about what’. The close relationship between \textit{da} and \textit{wo} (\ili{English} \textit{there} and \textit{where}) may be directly traced to \ili{Proto-\ili{Indo-European}} where we find the two forms *\textit{tó-}\textit{r} and *\textit{kʷ}\textit{ó-}\textit{r} (\citealt{MalloryAdams2006}: 419). \ili{German} \textit{hier} [-iː-] (\ili{English} \textit{here}, \ili{Plautdiitsch} \textit{hie}, \ili{Dutch} \textit{hier}) must be a \ili{Germanic} innovation ultimately based on *\textit{h\textsubscript{1}}\textit{ei-} ‘this (one)’ (\citealt{MalloryAdams2006}: 417f.), but it is somewhat obscure (e.g., \citealt{Seebold2002}).

There are also some parallel forms based on \textit{her} ‘here (movement)’ (a variant of \textit{hier}) and \textit{hin} ‘there (movement), towards’ that are also used as preverbs, e.g. \textit{her-kommen} ‘to come here’, \textit{hin.zu-fügen} ‘to add’ etc. In \ili{German} the reanalysis resulted in a few problems such as the fact that there are no separate forms *\textit{he-} (hence \textit{her.um} > \textit{he.rum} > \textit{rum}), *\textit{hi-} (hence \textit{hin.ein} > \textit{hi.nein} > \textit{nein}) etc.

\subsubsection{Interrogatives in Slavic}\label{sec:5.5.3.3}

\tabref{tab:indo:10} shows the development of selected \ili{Slavic} interrogatives over the course of time.

\begin{table}
\caption{Diachrony of Slavic interrogatives with selected cognates according to \cite{Derksen2008}); PS = {Proto-Slavic}, OCS = {Old Church Slavonic}}
\label{tab:indo:10}

\begin{tabularx}{\textwidth}{lQll}
\lsptoprule

\textbf{PIE} & \textbf{PS} & \textbf{OCS} & \textbf{Russian}\\
\midrule
*kʷ\=o-ko- & *kak\textcyrillic{ъ} ‘what (kind of)’ & kak\textcyrillic{ъ} & kakój\\
*kʷo-ter-o- & *koter\textcyrillic{ъ}, *kotor\textcyrillic{ъ} & kotor\textcyrillic{ъi} & kotóryj\\
*kʷo- & *k\textcyrillic{ъ}j\textcyrillic{ъ} ‘who, what, which’ & k\textcyrillic{ъ}i & koj\\
*kʷo- + *tod & *k\textcyrillic{ъ}to ‘who’ & k\textcyrillic{ъ}to & kto\\
*kʷo- + *g\textsuperscript{h}od\textsuperscript{h}-o- & *kog\textcyrillic{ъ}da, *kog\textcyrillic{ъ}do ‘when’ & kogda & kogdá\\
*kʷoli & *koli ‘how much’ & koli & (kóli ‘if’)\\
       & *koliko ‘how much’ & koliko & (Cz arch. koliko)\\
*kʷi/e- + *tod & *č\textcyrillic{ь}to ‘what’ & č\textcyrillic{ь}to & čto\\
*kʷiH & *či ‘\textsc{conj}’ & či ‘because’ & (dial. či ‘if, or’)\\
? & *ak\textcyrillic{ъ} ‘such as’ & jak\textcyrillic{ъ} & (Cz jaký ‘which’)\\
\lspbottomrule
\end{tabularx}
\end{table}

\ili{Russian} \textit{kotóryj} is a direct cognate of \ili{English} \textit{whether}, and \textit{či} of \textit{why}. According to \cite[172, 227]{Derksen2008}, the second part of \ili{Russian} \textit{kogdá} is a dative form of PS *\textit{gôd\textcyrillic{ъ}} ‘right time’ and goes back to PIE *\textit{g\textsuperscript{h}}\textit{od\textsuperscript{h}}\textit{-o-} (\ili{English} \textit{good} goes back to PIE *\textit{g\textsuperscript{h}}\textit{\=od}\textit{\textsuperscript{h}}\textit{-o-}), based on the stem PIE *\textit{g\textsuperscript{h}}\textit{ed\textsuperscript{h}}\textit{-} ‘join, fit together’ (\citealt{MalloryAdams2006}: 381). According to \citet{Derksen2008}, the final \textit{-li} in PS *\textit{koli} ‘how much’ is the \ili{Slavic} \isi{question marker}, but more likely the form simply goes back to PIE *\textit{kʷ}\textit{oli} and is perhaps related to PIE \textit{*kʷ}\textit{eh\textsubscript{a}}\textit{li}, whence \ili{Latin} \textit{qu\=alis} (\citealt{MalloryAdams2006}: 420). It would be unexpected for a further derivational element to attach to the \isi{question marker} as in PS *\textit{koliko}.

\begin{table}
\caption{Selected interrogatives from Russian (\citealt{Wade2011}: passim), Russian as spoken in Inner Mongolia (\citealt{BaiPing2011}: passim), and Ukrainian (\citealt{PughPress1999}: passim)}
\label{tab:indo:11}

\begin{tabularx}{\textwidth}{lQlQ}
\lsptoprule

\textbf{Meaning} & \textbf{Russian} & \textbf{\ili{Russian} (\isi{China})} & \textbf{Ukrainian}\\
\midrule
who & kto/\textcyrillic{кто} & ktɔ/\textcyrillic{кто} & xto/\textcyrillic{хто}\\
whose & čej/\textcyrillic{чей} & tʃjej/\textcyrillic{чей} & čyj/\textcyrillic{чий}\\
what & čto/\textcyrillic{что} & ʃtɔ/\textcyrillic{что} {\textasciitilde} tʃjɔ/\textcyrillic{чо} & ščo/\textcyrillic{що}, výščo/\textcyrillic{вищо}\\
who, which & kotóryj/\textcyrillic{который} &  & kotrýj/\textcyrillic{котрий}\\
whither & kudá/\textcyrillic{куда} & kuˈda/\textcyrillic{куда} & kudý/\textcyrillic{куди}\\
what (kind of), which & kakój/\textcyrillic{какой} & kaˈkɔj/\textcyrillic{какой} & jakýj/\textcyrillic{який}\\
how, in what manner & kak/\textcyrillic{как} & kak/\textcyrillic{как} & jak/\textcyrillic{як}\\
when & kogdá/\textcyrillic{когда} & kagˈda/\textcyrillic{когда} & kolí/\textcyrillic{коли}\\
where & gde/\textcyrillic{где} & gdje/\textcyrillic{где} & de/\textcyrillic{де}\\
how much/many & skól’ko/\textcyrillic{сколько} & ˈskɔlka/\textcyrillic{сколько} & skíl’ky/\textcyrillic{скільки}\\
whence & ot-kúda/\textcyrillic{откуда} &  & zvídky/\textcyrillic{звiдки}, zvidkíl’/\textcyrillic{звiдкiль}, (z)vidkiljá/\textcyrillic{звiдкiля}\\
why & po-čemú/\textcyrillic{почему}, za-čém/\textcyrillic{зачем} & za-ˈtʃjem/\textcyrillic{зачем} & čomú/\textcyrillic{чому}, čohó/\textcyrillic{чого}, pó-ščo/\textcyrillic{пощо}, navý-ščo/\textcyrillic{навищо}\\
\lspbottomrule
\end{tabularx}
\end{table}

\ili{Russian} as spoken in Inner \isi{Mongolia} does not exhibit major differences with respect to Standard \ili{Russian}. Some dialectal forms from \isi{Siberia} can be found in \tabref{tab:indo:15} below. In some instances such as \textbf{\textit{k}}\textit{to} vs. \textbf{\textit{x}}\textit{to} ‘who’ or \textbf{\textit{k}}\textit{ak} vs. \textbf{\textit{j}}\textit{ak} ‘how’ only phonological differences separate \ili{Russian} and \ili{Ukrainian}. In other cases there are different derivations such as in \textit{po-}\textbf{\textit{čemú}} (\textsc{dat}) vs. \textit{pó-}\textbf{\textit{ščo}} (\textsc{nom}) ‘why’. Only in a few instances are there altogether different interrogatives such as \textit{kogdá} vs. \textit{kolí} ‘when’. The \ili{Russian} and \ili{Ukrainian} forms meaning ‘why’ are \isi{case} forms (\textsc{dat}, \textsc{instr}, \textsc{gen}, \textsc{nom}) used with or without prepositions. A preposition can also be found in \ili{Russian} \textit{ot-kúda}/\textcyrillic{откуда} ‘whence’ (< OCS \textit{ot-} < PIE \textit{h\textsubscript{a}}\textit{et} ‘away, beyond’, \citealt{MalloryAdams2006}: 289ff.). \tabref{tab:indo:12} shows the paradigms of the \isi{interrogative} pronouns meaning ‘who’ and ‘what’ in \ili{Proto-Slavic}, \ili{Russian}, and \ili{Ukrainian}. \ili{Russian} \textit{čto}/\textcyrillic{что} ’what’ has the colloqial pronounciations [ʃtɔ] and an informal variant \textit{čo}/\textcyrillic{чо} [tʃjɔ] (e.g., \citealt{BaiPing2011}).

\begin{table}
\caption{Proto-Slavic, Russian, and Ukrainian interrogative paradigms (\citealt{PughPress1999}: 178; \citealt{Shevelov1993}: 961; \citealt{SussexCubberley2006}: 269f.; \citealt{BaiPing2011}: 74, 78)}

\label{tab:indo:12}

\begin{tabularx}{\textwidth}{lllllQQ}
\lsptoprule
& \textbf{PS} &  & \textbf{Russian} &  & \textbf{Ukrainian} & \\
\midrule
\textbf{Case} & \textbf{who} & \textbf{what} & \textbf{who} & \textbf{what} & \textbf{who} & \textbf{what}\\
\textsc{nom} & *k-\textcyrillic{ъ}-to & *č-\textcyrillic{ь}-to & k-to & č-to [ʃtɔ] & x-to & ščo\\
\textsc{gen} & *k-ogo & *č-eso/-\textcyrillic{ь}so & k-ogó [kʌˈvɔ] & č-egó [tʃɪˈvɔ] & k-ohó & č-ohó\\
\textsc{dat} & *k-omu & *č-emu & k-omú & č-emú & k-omú & č-omú\\
\textsc{acc} & = \textsc{nom} & = \textsc{nom} & = \textsc{gen} & = \textsc{nom} & = \textsc{gen} & = \textsc{nom}\\
\textsc{instr} & *k-ěm\textcyrillic{ь} & *č-im\textcyrillic{ь} & k-em & č-em & k-ym & č-ym\\
\textsc{loc} & *k-om\textcyrillic{ь} & *č-em\textcyrillic{ь} & k-om & č-ёm & k-ómu, k-im & č-ómu, č-im\\
\lspbottomrule
\end{tabularx}
\end{table}

The difference between the genitive on the one hand and the nominative on the other, when filling in for the accusative, is connected with the distinction between animate and inanimate meaning \citep[127]{Cubberley2002}. A difference between nominative (e.g. \ili{Ukrainian} \textit{x-to} ‘who, \textit{š-čo} ‘what’) and oblique stems (e.g., \ili{Ukrainian} \textit{k-}, \textit{č-}) is also known from \ili{Iranian} (see below). Similar to \ili{German}, the selective \isi{interrogative} shows extensive paradigms. \ili{Russian} and \ili{Ukrainian} also have a distinction between masculine, feminine, and neuter \isi{gender} but preserve more cases. For reasons of space only some \ili{Ukrainian} \isi{interrogative} paradigms will be given in the following (Tables \ref{tab:indo:13}, \ref{tab:indo:14}).

\begin{table}[p]
\caption{Conjugation of Ukrainian \textit{kotorýj} ‘which’ (\citealt{PughPress1999}: 180)}
\label{tab:indo:13}
\small
\begin{tabularx}{\textwidth}{XXXXl}
\lsptoprule

\textbf{Case} & \textbf{\textsc{m}} & \textbf{\textsc{n}} & \textbf{\textsc{f}} & \textbf{\textsc{pl}}\\
\midrule
\textsc{nom} & kotrýj & kotré & kotrá & kotrí\\
\textsc{gen} & kotróho & id. & kotróji & kotrýx\\
\textsc{dat} & kotrómu & id. & kotríj & kotrým\\
\textsc{acc} & = \textsc{nom}/\textsc{gen} & = \textsc{nom} & kotrú & = \textsc{nom}/\textsc{gen}\\
\textsc{instr} & kotrým & id. & kotróju & kotrýmy\\
\textsc{loc} & = \textsc{dat}, kotrím & id. & = \textsc{dat} & = \textsc{gen}\\
\lspbottomrule
\end{tabularx}
\end{table}

\begin{table}[p]
\caption{Conjugation of Ukrainian \textit{jakýj} ‘what kind of’ (\citealt{PughPress1999}: 180)}
\label{tab:indo:14}

\small
\begin{tabularx}{\textwidth}{XXXXl}
\lsptoprule
\textbf{Case} & \textbf{\textsc{m}} & \textbf{\textsc{n}} & \textbf{\textsc{f}} & \textbf{\textsc{pl}}\\
\midrule
\textsc{nom} & jakýj & jaké & jaká & jakí\\
\textsc{gen} & jakóho & id. & jakóji & jakýx\\
\textsc{dat} & jakómu & id. & jakíj & jakým\\
\textsc{acc} & = \textsc{nom}/\textsc{gen} & = \textsc{nom} & jakú & = \textsc{nom}/\textsc{gen}\\
\textsc{instr} & jakým & id. & jakóju & jakýmy\\
\textsc{loc} & = \textsc{dat}, jakím & id. & = \textsc{dat} & = \textsc{gen}\\
\lspbottomrule
\end{tabularx}
\end{table}



\begin{table}[p]
\caption{Taimyr Pidgin interrogatives (\citealt{Stern2005,Stern2012}: 435ff., 498); some variants were excluded}
\label{tab:indo:15}
\small
\begin{tabularx}{\textwidth}{lXl}
\lsptoprule

\textbf{Form} & \textbf{Meaning} & \textbf{Analysis}\\
\midrule
syly/\textcyrillic{су}\textcyrillic{л}\textcyrillic{у} & who & \ilit{Nganasan} sïlï ‘who’\\
kto/\textcyrillic{кто} & who & = \ilit{Russian}\\
kakoj/\textcyrillic{какой} & which & = \ilit{Russian}\\
gde/\textcyrillic{где} & where & = \ilit{Russian}\\
kuda/\textcyrillic{куда} & whither & = \ilit{Russian}\\
začem/\textcyrillic{зачем} & why & = \ilit{Russian}\\
kak/\textcyrillic{как} & how & = \ilit{Russian}\\
kogda/\textcyrillic{кокда} & when & = \ilit{Russian}\\
kogo/\textcyrillic{кого} & who & \ilit{Russian} \textsc{gen/acc}\\
čego/\textcyrillic{чего} & what & \ilit{Russian} \textsc{gen}\\
kudy-mera/\textcyrillic{куды мера} & where & \ilit{Russian} mera/\textcyrillic{мера} ‘measure’\\
kudy-mesto/\textcyrillic{куды место} & where & \ilit{Russian} mesto/\textcyrillic{место} ‘place’\\
kakoj storona/\textcyrillic{какой сторона} & whither & \ilit{Russian} storona/\textcyrillic{сторона} ‘side’\\
akto/\textcyrillic{акто} & who & = dialectal \ilit{Russian}\\
počto/\textcyrillic{почто} & why & = dialectal \ilit{Russian}\\
čo/\textcyrillic{чо} & what, why & = dialectal \ilit{Russian}\\
\lspbottomrule
\end{tabularx}
\end{table}

\largerpage
As would be expected, most interrogatives in the two pidgin languages are derived from \ili{Russian}. \tabref{tab:indo:15} shows those interrogatives attested for \ili{Taimyr Pidgin Russian}. An interesting fact is the frequent use of the oblique forms \textit{kogo} and \textit{čego}. There are three newly formed complex interrogatives. Mednyj \ili{Aleut} also has some \ili{Russian} interrogatives (\sectref{sec:5.4.3}). Apparently, one \ili{Nganasan} form was borrowed as well.

The phenomenon of one form meaning ‘what’ and ‘why’ is also known from the \ili{Iranian} languages \ili{Sogdian} (\textit{(ə)ču}) and Khotanese (\textit{cu}), see \sectref{sec:5.5.3.4}.

Only a short list of interrogatives in \ili{Chinese Pidgin Russian} can be assembled from the material provided by \citet{Shapiro2010}. Most forms are of \ili{Russian} origin, but at least one is from \ili{Chinese}. The \isi{interrogative} \textit{mnogo-malo} entirely consists of \ili{Russian} material but follows the \ili{Chinese} structure and meaning (\tabref{tab:indo:16}).

\begin{table}
\caption{\ili{Chinese Pidgin Russian} interrogatives mentioned by \cite{Shapiro2010}; the form in < > was given in Chinese Pinyin}
\label{tab:indo:16}

\begin{tabularx}{\textwidth}{llQ}
\lsptoprule 
\textbf{Form} & \textbf{Meaning} & \textbf{Analysis}\\
\midrule
kaka & why & = \ilit{Russian} kak/\textcyrillic{как}\\
kakoj & which & = \ilit{Russian} kakoj/\textcyrillic{какой}\\
<gedao’erli> & which & = \ilit{Russian} kotóryj/\textcyrillic{который}\\
pocheto & why & = dialectal \ilit{Russian}\\
mnogo-malo & how much & \ilit{Russian} mnógo/\textcyrillic{много} ‘much’, málo/\textcyrillic{мало} ‘little’, cf. \ilit{Chinese} du\=oshăo \zh{多少}\\
šýma & what & = \ilit{Chinese} shénme \zh{什么}\\
\lspbottomrule
\end{tabularx}
\end{table}

\subsubsection{Interrogatives in Iranian}\label{sec:5.5.3.4}

\begin{table}[b]
\caption{Sogdian interrogatives (\citealt{Yoshida2009}: passim) in comparison with Yaghnobi (\citealt{Geiger1901}: passim; \citealt{Bielmeier1989}: 482, 484)}
\label{tab:indo:17}

\begin{tabularx}{\textwidth}{lllQ}
\lsptoprule

\textbf{Meaning} & \textbf{Sogdian} & \multicolumn{2}{l}{~~~~~~~~~~\textbf{Yaghnobi}} \\
\midrule
&  & \textbf{Geiger} & \textbf{Bielmeier}\\
\midrule
when & ka$\delta $a & kad & kad\\
which & kataam & kaam & kom, \textsc{obl} komi\newline kuum, \textsc{obl} kuumi\\
where & (ə)kuu & kuu & \\
who & (ə)ke, \textsc{obl} (ə)kya & kax, \textsc{obl} kȧi & kax, \textsc{obl} kay\\
what & (ə)ču & čaa & čo, \textsc{obl} čoy\\
why & (ə)ču & čuu & \\
how many/much & čaaf(ar) & čaaf & čof\\
\lspbottomrule
\end{tabularx}
\end{table}


Most interrogatives in those \ili{Iranian} languages included here are synchronically \isi{opaque} and their etymologies too complex to be given here in their entirety. But consider the interrogatives in \tabref{tab:indo:17}. As can be seen, \ili{Sogdian} interrogatives have some direct correspondences in, or at least similarities to, \ili{Yaghnobi}, the only closely related modern language spoken in Tajikistan. Most up-to-date material on Yaghnobi was published in Tajik and has thus to be excluded.


Some of these forms can directly be traced back to \ili{Indo-European}. For instance, \ili{Yaghnobi} \textit{kad} ‘when’ goes back to PIE *\textit{kʷ}\textit{odéh\textsubscript{a}} and \textit{kuu} ‘where’ to PIE *\textit{kʷ}\textit{u(ú)} (or *\textit{kú-}). The occasional initial vowel in \ili{Sogdian} is perhaps prothetic \citep[286]{Yoshida2009}. Khotanese has several comparable forms such as \textit{kaama-} ‘which’, \textit{ku} ‘where’, \textit{kye} {\textasciitilde} \textit{ce} ‘who’, \textit{cu} ‘what, why’, and some additional forms such as \textit{craama-} ‘what kind of’, \textit{ciiyä} ‘when’, \textit{caalsto} ‘whither’, or \textit{canda}, \textit{cändäka}, \textit{cerä} ‘how much/many’ (\citealt{Emmerick2009}: 387, 389, passim). \tabref{tab:indo:18} lists interrogatives from \ili{Sarikoli}. In order to put them into a proper context, interrogatives from the closely related language \ili{Wakhi} are listed as well. Remember that these two languages are collectively called \ili{Tajik} in \isi{China} but that \ili{Tajik} is really a variety of \ili{Persian}.

\begin{table}
\caption{Selection of Sarikoli, Wakhi (\citealt{GaoErqiang1985}: passim), Tajik, and Persian interrogatives (\citealt{WindfuhrPerry2009}: 438); Sarikoli form in square brackets from \cite[88, 120, 162f.]{Xiren2015}; Wakhi forms in square brackets from \citet[831]{Bashir2009}; not all variants are listed}
\label{tab:indo:18}

\begin{tabularx}{\textwidth}{Xllll}
\lsptoprule

\textbf{Meaning} & \textbf{Sarikoli} & \textbf{Wakhi} & \textbf{Tajik} & \textbf{Persian}\footnotemark\\
\midrule
which & \textbf{tʃ}idum & [kum(d)] & kadom & kodaam\\
where & ko, ku-dʒui & kum-dʒai & ku, ku-jo & ku, ko-jaa\\
when & [\textbf{tʃ}um] &  & key & kay\\
who & \textbf{tʃ}oi, \textsc{obl} -tʃi- & kui & kii & ki\\
what & tseiz, [tsɛiz] & tsiz, [čiz] & čii {\textasciitilde} ča- & če(-)\\
why & [tsarang ‘how’] & [čir] & čaro & čeraa\\
how many/much & tsund, [tsond, \textbf{tʃ}and] & [tsum] & čand & čand\\
\lspbottomrule
\end{tabularx}
\end{table}

\footnotetext{ Several \ili{Persian} interrogatives have been borrowed by \ili{Moghol} (\sectref{sec:5.8.3}).}

According to \citet[163]{Xiren2015}, \ili{Sarikoli} \textit{ts}\textbf{\textit{o}}\textit{nd} ‘how many/much’ stands opposed to \textit{tʃand} ‘how many’, which may have been adopted from \ili{Tajik}. The locative interrogatives contain words meaning ‘place’. All \ili{Iranian} languages included here preserve the split between the two resonances \textit{k{\textasciitilde}} and \textit{č{\textasciitilde}} (or \textit{c{\textasciitilde}}), the latter of which goes back to *\textit{k-} as well. \ili{Sarikoli}, apart from this distinction, has an innovative third type \textit{tʃ{\textasciitilde}} < \textit{k{\textasciitilde}} that cannot be found in \ili{Wakhi} or \ili{Persian}. In \citet{GaoErqiang1985} it thus shows \isi{synchronic} variation between \textit{k-}, \textit{ts-}, and \textit{tʃ-}. Khotanese similarly had free variation between \textit{kye} and the more innovative \textit{ce} ‘who’ \citep[387]{Emmerick2009}. \ili{Sarikoli} \textit{tʃoj} ‘who’ has a special oblique form \textit{tɕi} that is the basis for \isi{case} marking (cf. \ili{Sogdian} and \ili{Yaghnobi} in \tabref{tab:indo:17}).

\ea%44
    \label{ex:indo:44}
    \ili{Sarikoli}\\
    \ea
    \gll \textbf{{tʃoj}} a=ta    ðud?\\
    who.\textsc{nom}    \textsc{acc}-2\textsc{sg.acc}  hit.\textsc{pst}\\
    \glt ‘Who hit you?’
\clearpage %solid chapter boundary    
    \ex
    \gll təw=at      a=\textbf{{tɕi}} ðud?\\
    2\textsc{sg}.\textsc{nom=2sg.pst}  \textsc{acc}-who.\textsc{obl}  hit.\textsc{pst}\\
    \glt ‘Whom did you hit?’
    
    \ex
    \gll təw=at \textbf{{tɕi}}=ri    ðud?\\
    2\textsc{sg}.\textsc{nom=2sg.pst}  who.\textsc{obl-dat}  hit.\textsc{pst}\\
    \glt ‘Whom did you give (it) to?’ (\citealt{Kim2014}: 9)
    \z
    \z

\noindent \citet[35]{GaoErqiang1985} has the full paradigm as follows: \textsc{nom} \textit{tʃoi}, \textsc{gen} \textit{tʃi-an}, \textsc{dat} \textit{tʃi-ri}, \textsc{acc} \textit{a-tʃi}.

\subsubsection{Interrogatives in Tocharian}\label{sec:5.5.3.5}

Despite the fact that the two Tocharian varieties are thought to be relatively closely related, there are nevertheless many differences within the system of interrogatives (\tabref{tab:indo:19}).

\begin{table}
\caption{Selection of Tocharian interrogatives according to \citet{Adams2013} with additional Tocharian A data by \citet[176–191]{SiegSiegling1931} and \citet{Carling2009}}
\label{tab:indo:19}

\begin{tabularx}{\textwidth}{XXX}
\lsptoprule
& \textbf{TA} & \textbf{TB}\\
\midrule
who, what & kus & k\textsubscript{u}se\\
why & k\textsubscript{u}yal & k\=a\\
where & t\=a & k\textsubscript{u}ta-meṃ\\
how, when, if & kupre & kwri, kr\textsubscript{u}i\\
how much/long & kos & kos, kot\\
who, which, what kind of & äntsaṃ, antsaṃ & intsu\\
where & (äntan(n)ene \textsc{conj)} & inte, ente\\
when & (änt\=ane \textsc{conj)} & inte, ente\\
who, what, which &  & mäksu\\
how & mä(n)t & mäkte\\
\lspbottomrule
\end{tabularx}
\end{table}

This might be additional evidence for \citegen[144]{Peyrot2010} assumption that a “Proto-Tocharian may have differed more from its daughter languages than is often suggested by superficial similarities between them”, which could be the result of later convergence. The best etymologies for Tocharian interrogatives have been given by \citet{Adams2013}, but these are too complex and somewhat too uncertain to be given here in full length.

There is a \isi{resonance} in \textit{k(\textsubscript{u}}\textit{){\textasciitilde}} that , as seen before, is a reflex of PIE *\textit{kʷ}\textit{{\textasciitilde}}. There are also interrogatives starting with \textit{m-} such as TA \textit{mänt} that might be based on an in \isi{interrogative} stem PIE *\textit{me/o-} (\citealt{MalloryAdams2006}: 421). However, \citet{Adams2013} assumes that TB \textit{mäksu} ‘who, what, which’ and \textit{mäkte} ‘how’ as a middle part contain the actual PIE \isi{interrogative} stem *\textit{kʷ}\textit{i/u-}, preceded by the PIE particle *\textit{men} and followed by different \isi{demonstratives} or relatives. TB \textit{mäksu} ‘who, what, which’ exhibits a more or less full paradigm based on person, number, and \isi{gender}, e.g. \textit{mäksu} ‘\textsc{nom.sg.m}’, \textit{mäks\=a\textsubscript{u}} ‘\textsc{nom.sg.f}’, \textit{mäktu} ‘\textsc{nom.sg.n}’ (\citealt{Krause1960}: 166). There is some agreement that TA \textit{kus} and TB \textit{k\textsubscript{u}}\textit{se} < Proto-Tocharian *\textit{kʷ}\textit{əsë} ‘who’ are similarly combinations of an \isi{interrogative} with a demonstrative, perhaps PIE *\textit{kʷ}\textit{i-} + *\textit{so} (e.g., \citealt{Kim2012}: 38). This is reminiscent of \ili{Slavic} *\textit{k\textcyrillic{ъ}to} ‘who’ < PIE *\textit{kʷ}\textit{o-} + *\textit{tod} (see \tabref{tab:indo:12} above). \ili{Indo-European} had a distinction between three \isi{demonstratives}, *\textit{so} ‘that one, he’, *\textit{seh\textsubscript{a}} ‘that one, she’, and \textit{tod} ‘that one, it’ (\citealt{MalloryAdams2006}: 417). The difference lies in the fact that Tocharian *\textit{kʷ}\textit{əsë} contains the first of these, and \ili{Slavic} *\textit{k\textcyrillic{ъ}to} the last. TB \textit{k\textsubscript{u}}\textit{se} ‘who’ (TA \textit{kus}) and \textit{k\textsubscript{u}}\textit{ce} ‘whom’ (TA \textit{kuc}) later had the abbreviated forms \textit{se} and \textit{ce}, respectively (\citealt{Kim2012}: 38), which is reminiscent of TA \textit{t\=a} ‘where’ as opposed to TB \textit{k\textsubscript{u}}\textit{ta-}. In \ili{Tocharian B} the meaning of both \textit{k\textsubscript{u}}\textit{se} and \textit{mäksu} encompasses both ‘who’ and ‘what’, which, apart from Baltic languages and \ili{Kusunda}, is quite exceptional in Eurasia.