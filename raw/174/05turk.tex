\section{Turkic}\label{sec:5.11}
\subsection{Classification of Turkic}

The internal diversity of \ili{Turkic} is much more elaborate than that of, say, \ili{Mongolic}, but less so than \ili{Uralic}. Based on Johanson (\citeyear{Johanson1998}: 81f.; \citeyear{Johanson2006a}: 161f.) the languages may roughly be classified as in \figref{exfig:turk:1}.
	An asterisk indicates that a given language is at least partly spoken in \isi{NEA} today. Of course, \ili{Turkic} languages altogether derive from southern \isi{Siberia} and northern \isi{Mongolia} (e.g., \citealt{Yunusbayev2015}), which is why languages such as \ili{Turkish} and \ili{Chuvash} will also briefly be addressed. Most languages included here are from the Northeastern or Siberian branch. The \ili{Turkic} language Yellow \ili{Uyghur} that is also called Western Yughur (\textit{x\={\i}bù yùgù yǔ}  \zh{西部裕固语} in \ili{Chinese}) or \ili{Sarig Yughur}, has to be distinguished from the \ili{Mongolic} language \ili{Shira Yughur} or Eastern Yughur (\textit{d\=o}\textit{ngbù yùgù yǔ}  \zh{东部裕固语} in \ili{Chinese}, \sectref{sec:5.8}). Yellow \ili{Uyghur} has no close relation to \ili{Uyghur}, which belongs to an altogether different branch of \ili{Turkic}. Despite its name, \ili{Fuyu} \ili{Kyrgyz}, spoken in the Heilongjiang province of northeastern \isi{China}, is more closely related to Yellow \ili{Uyghur} and the other Abakan \ili{Turkic} languages than to \ili{Kyrgyz} as such, which belongs to the \ili{Kipchak} branch. Similar to the \ili{Tungusic} language \ili{Sibe} that was partly relocated to \isi{Xinjiang} in 1764, \ili{Fuyu} was brought to \isi{Manchuria} from the \isi{Altai} region in the 1750s under emperor \isi{Qianlong} (see \citealt{HuZhenhua1986,Hu1996}; \citealt{HuImart1987}; \citealt{Janhunen1996}).

% \ea\upshape%1
\begin{figure}
\caption{Classification of Turkic}
    \label{exfig:turk:1}
\scriptsize    
\begin{forest}  for tree={grow'=east,delay={where content={}{shape=coordinate}{}}},   forked edges  
[
    [{\ili{Oghuz}\\(southwestern)}, text width=1.52cm
        [Western
	    [\ili{Turkish}]
            [Gagauz]
            [Azerbaijani]
        ]
        [Southern
        	[dialects in Iran and Afghanistan]
        ]
        [Eastern
	    [\ili{Turkmen}]
            [Khorasan \ili{Turkic}]
        ]
        [?\ili{Salar}]
    ]
    [{\ili{Kipchak}\\ (northwestern)}, text width=1.5cm
        [Western
	    [Kumyk]
            [Karachay-Balkar]
            [Crimean \ili{Tatar}]
            [Karaim]
        ]
        [Northern,l=3cm
	    [\ili{Tatar}*]
            [Bashkir]
        ]
        [Southern
	    [\ili{Kazakh}*]
            [Karakalpak]
            [\ili{Kipchak Uzbek}]
            [Nogai]
            [\ili{Kyrgyz}*]
        ]
    ]
    [{\ili{Uygur-Karluk}/\\\ili{Chagatay} (southeastern)}, text width=2.5cm
    	[Western
        	[\ili{Uzbek}*]
    	]
        [Eastern
	    [\ili{Uyghur}*]
            [?\ili{Eynu}*]
        ]
	[?\ili{Ili Turki}*]
    ]
    [Siberian (northeastern),l=3cm
	  [Southern
		  [Sayan \ili{Turkic}
		  [\ili{Tuvan}*]
		  [\ili{Dukhan}*]
		  [Tofan*]
	      ]
	      [Abakan (\isi{Yenisei}) \ili{Turkic}, l=2cm
		  [\ili{Khakas}*]
		  [\ili{Shor}*]
		  [\ili{Fuyu}*]
		  [Sarig*]
	      ]
	      [\ili{Chulym} \ili{Turkic}*]
	      [\ili{Altai Turkic}*]
	      [\ili{Chalkan}*]            
	  ]
	  [Northern, l=2cm
	      [\ili{Yakut}* (Sakha)]
	      [\ili{Dolgan}*]
	  ] 
    ]
    [\ili{Oghur}, l=2cm
        [\textsuperscript{†}Bulgar]
        [\ili{Chuvash}]
    ]
    [\ili{Khalaj} (Arghu), l=1cm]
]
\end{forest}   
%     \z
\end{figure}

\textbf{Eynu} (\textit{àinǔ} \zh{艾努} in \ili{Chinese}) can be considered a truly mixed language. It is “a language that is structurally and grammatically \ili{Uyghur}, but whose vocabulary is predominantly \ili{Persian} or \ili{Persian}-derived.” (\citealt{Lee-Smith1996a}: 861). Possibly, the origin of \ili{Eynu} lies in its use as a secret language. Because this study is mostly concerned with the \isi{grammar of questions}, it has been classified as basically \ili{Turkic} here (see also \citealt{Hayasi1999} for a discussion). The alternative name \isi{Abdal} has a derogatory meaning and will not be used (see also \citealt{Ladstätter1994}, \citealt{Wurm1997}). Another special case is \textbf{Salar}, which is the result of \isi{language contact} between \ili{Turkic} languages from different branches.

\begin{quote}
\ili{Salar} originated as an \ili{Oghuz} language and during the course of its speakers’ gradual eastward \isi{migration} acquired various influences from southeastern- and north- western-type \ili{Turkic} languages as well as from non-\ili{Turkic} ones. It had been assumed earlier that \ili{Salar} is an isolated dialect of Modern \ili{Uyghur}, mostly on the basis of phonological features such as liquid assimilation and vowel raising. \citep[400]{Hahn1998}
\end{quote}

\ili{Ili Turki} (\textit{tŭ’ĕrkè} \zh{土尔克} in \ili{Chinese}) is a language that is not very well-known and for which only a few descriptions are available. According to \citegen[31]{Hahn1991} classification it shares properties with both the \ili{Kipchak} and the \ili{Uygur-Karluk} branches of \ili{Turkic}, but it has been tentatively classified with the latter in this study.

\subsection{Question marking in Turkic}\label{sec:5.11.2}

The following will briefly describe \isi{question marking} in the \textbf{\ili{Oghuz}} language \textbf{Turkish}, which is actually located outside of the Northeast Asian area but may serve as a \isi{reference point} for the other \ili{Turkic} languages. \ili{Turkish} has a sentence-final particle \textit{=mI} that is usually written detached from its host but is best analyzed as enclitic. It has the vowel harmonic variants listed in \tabref{tab:turk:1}. There is no variation in the consonant that invariably has the nasal shape \textit{m}.

\begin{table}
\caption{Vowel harmonic forms and distribution of the Turkish question marker (\citealt{GökselKerslake2005}: 22, 251; \citealt{Landmann2009}: 4, 24)}
\label{tab:turk:1}

\begin{tabularx}{\textwidth}{XX}
\lsptoprule

\textbf{Preceding vowel} & \textbf{Variant of mI}\\
\midrule
i, e (front unrounded vowel) & mi\\
ü, ö (front rounded vowel) & mü\\
ı, a (back unrounded vowel) & mı\\
u, o (back rounded vowel) & mu\\
\lspbottomrule
\end{tabularx}
\end{table}

The morphosyntactic behavior of the \ili{Turkish} \isi{question marker} is similar to \ili{Proto-Tungusic} and some modern \ili{Tungusic} languages, \ili{Middle Mongol} or \ili{Old Japanese}. It is a mobile particle that also marks \isi{focus} and \isi{alternative question}s and is also part of several \isi{tag question} markers. In \isi{focus question}s it attaches to the element in \isi{focus}, which receives an an additional peak in \isi{intonation} \citep[24]{Landmann2009}. Content questions remain unmarked. Alternative \isi{questions} take two markers and have an optional \isi{disjunction} \textit{yoksa}.

\ea%2
    \label{ex:turk:2}
    \ili{Turkish}
    \ea
    \gll Nermin  okul-a git-miş=\textbf{{mı}}?\\
    \textsc{pn}    school-\textsc{dat}    go-\textsc{pfv.ev=q}\\
    \glt ‘Has Nermin gone to school?’
    
    \ex
    \gll Nermin  okul-a=\textbf{{mı}} git-miş?\\
    \textsc{pn}    school-\textsc{dat=q}    go-\textsc{pfv.ev}\\
    \glt ‘Has Nermin gone \textit{to school}?’
    
    \ex
    \gll Cemal  okul-a    git-ti=\textbf{{mi}},  (\textbf{{yoksa}})    git-\textbf{{me}}-de=\textbf{{mi}}?\\
    \textsc{pn}  school-\textsc{dat}  go-\textsc{pfv=q}  (or)    go-\textsc{neg}-\textsc{pfv=q}\\
    \glt ‘Did Cemal go to school or not?’
    
    \ex
    \gll Ev-e \textbf{{ne}} zaman    gid-ecek-sin?\\
    home-\textsc{dat}  what  time    go-\textsc{fut}-2\textsc{sg}\\
    \glt ‘When will you be going home?’ (\citealt{GökselKerslake2005}: 257, 254, 259)\z\z

\ili{Turkish} has \isi{tag question}s marked with the demonstrative \textit{öyle} ‘like this’ or the negative copula \textit{değil} followed by the regular polar \isi{question marker} \textit{mi} (\citealt{GökselKerslake2005}: 253).

\ea%3
    \label{ex:turk:3}
    \ili{Turkish}
    \ea
    \gll Cemal bugün okul-a git-me-di, \textbf{değil} \textbf{{mi}}?\\
    \textsc{pn}  today  school-\textsc{dat}  go-\textsc{neg}-\textsc{pfv}  \textsc{neg}  \textsc{q}\\
    \glt ‘Cemal didn’t go to school today, did(n’t) he?’
    
    \ex
    \gll Esra Handan-ın abla-sı-ymış, \textbf{{öyle}} \textbf{{mi}}?\\
    \textsc{pn}  \textsc{pn}-\textsc{gen}    elder.sister-3\textsc{sg}.\textsc{poss}-\textsc{ev}.\textsc{cop}  so  \textsc{q}\\
    \glt ‘So Esra is Handan’s elder sister, right?’ (\citealt{GökselKerslake2005}: 253)
    \z
    \z

\noindent Similar \isi{tag question} markers derived from \isi{demonstratives} and negative copulas are also known from \ili{Mongolic} (\sectref{sec:5.8.2}). With this \isi{reference point} in mind we can now address those \ili{Turkic} languages spoken in \isi{Northeast Asia}.

In \textbf{Salar}, the only \ili{Oghuz} language located in \isi{Northeast Asia}, polar \isi{questions} appear to obligatorily take one of several sentence-final question markers. One marker has the vowel harmonic variants \textit{mu}, \textit{mo}, and \textit{mi} and is most likely cognate with with \ili{Turkish} \textit{mI}. Another marker has the two forms \textit{u} and \textit{o} and can be compared with the \ili{Mongolian} polar \isi{question marker} \textit{=(y)UU} that has the two vowel harmonic forms \textit{=(y)uu} and \textit{=(y)oo}. It exhibits a short vowel in some, especially Shirongolic languages that are in close vicinity to \ili{Salar}. The functional difference between the two markers \textit{=mU} and \textit{=U} remains unclear, but they appear to be mutually exchangeable (\citealt{LinLianyun1985}: 91). The vowel harmonic forms of both particles have a somewhat unclear distribution. For example, all three variants, \textit{mu}, \textit{mo}, and \textit{mi}, are apparently possible in example (\ref{ex:turk:4}a). The fact that there are different numbers of vowel harmonic variants can be explained by the different origin of the question markers. Both have been reanalyzed as enclitics here.

\ea%4
    \label{ex:turk:4}
    \ili{Salar}\\
    \ea
    \gll sen  aʃ    iʃ-dʒi=\textbf{{mu}}?\\
    2\textsc{sg}  noodle    drink-\textsc{pst.def=q}\\
    \glt ‘Have you eaten (yet)?’
    
    \ex
    \gll u  ge(l)-miʃ=\textbf{{u}}?\\
    3\textsc{sg}  come-\textsc{pst.indef=q}\\
    \glt ‘Has (s)he come?’\footnote{The meaning of the parentheses in this last example is not entirely clear, but could indicate either optional elements or parts of the verb stem that are lost in \isi{combination} with the suffixes.} (\citealt{LinLianyun1985}: 90, 91)
    \z
    \z

\ili{Salar} \isi{content question}s are almost always marked with the sentence-final \textit{-i}, which is likely to be related to \ili{Tuvan} \textit{-Il}, \ili{Dukhan} \textit{-Ĭl}, \ili{Tofa} \textit{-(u)l}, \ili{Yakut} \textit{-(n)ɪj}, and \ili{Dolgan} -\textit{ij}, all of which are restricted to content \isi{questions} (see below). The forms have all been analyzed as suffixes here, but some might have an enclitic status.

\ea%5
    \label{ex:turk:5}
    \ili{Salar}\\
    \gll ana,  sen \textbf{{ɢala}} va(r)-ʁur-\textbf{{i}}?\\
    girl  2\textsc{sg}  whither  go-\textsc{fut-q}\\
    \glt ‘Miss, where are you going?’ (\citealt{LinLianyun1985}: 116)
    \z

However, in one example \citet[82]{LinLianyun1985} has an example with a copula \textit{deri} that possibly contains the \isi{question marker}, though this was left unanalyzed. The marker \textit{-i} is also absent after verbs with a definite past marking (\citealt{LinLianyun1985}: 71).

\ea%6
    \label{ex:turk:6}
    \ili{Salar}\\
    \gll sen \textbf{{naŋ}}-a    vulə  gel-\textbf{{dʒi}}?\\
    2\textsc{sg}  what-\textsc{dat}  for  come-\textsc{pst}.\textsc{def}\\
    \glt ‘Why have you come?’ (\citealt{LinLianyun1985}: 86)
    \z

Furthermore, \ili{Salar} has a question suffix \textit{-du} {\textasciitilde} \textit{-do} that is directly attached to a verb stem and is said to be connected to the category of \isi{evidentiality}. It is only used if one has observed oneself that the addressee has finished a certain \isi{action}.

\ea%7
    \label{ex:turk:7}
    \ili{Salar}\\
    \gll sen  iʃ-\textbf{{du}}?\\
    2\textsc{sg}  drink-\textsc{q}.\textsc{self}.\textsc{ev}\\
    \glt ‘(I see that) you have finished drinking?’ (\citealt{LinLianyun1985}: 71)
    \z

The very specific meaning as well as the morphosyntactic behavior make it implausible to assume a connection with the \ili{Yakut} and \ili{Dolgan} \isi{question marker} \textit{=duo} {\textasciitilde} \textit{=duu} that we will encounter further below. Like many other languages spoken in \isi{China}, \ili{Salar} has borrowed the \ili{Mandarin} \isi{question marker} \textit{ba} \zh{吧}.

\ea%8
    \label{ex:turk:8}
    \ili{Salar}\\
    \gll u  si-(niɣi)  gaga-ŋ      ira \textbf{{ba}}?\\
    3\textsc{sg}  2\textsc{sg}-(\textsc{gen})  e.brother-?2\textsc{sg.poss}  \textsc{cop}  \textsc{q}\\
    \glt ‘This is your elder brother, right?’ (\citealt{LinLianyun1985}: 84)
    \z

\ili{Tatar}, \ili{Kazakh}, and \ili{Kyrgyz} are the three \textbf{\ili{Kipchak}} languages spoken in \isi{Northeast Asia}. I will address them in turn. \textbf{Tatar} as spoken in \isi{China} has a cognate of the \isi{question marker} in \ili{Turkish} and \ili{Salar} that has the form \textit{=mə} {\textasciitilde} \textit{=mɨ} (\citealt{ChenZongzhenYiLiqian1986}: 30). It has likewise been reanalyzed as enclitic here. As opposed to \ili{Salar}, however, \isi{content question}s remain unmarked.

\ea%9
    \label{ex:turk:9}
    \ili{Tatar} (\isi{China})\\
    \ea
    \gll sɨn \textbf{{qajsə}}{-sə-n    ala-səŋ?}\\
    2\textsc{sg}  which-3\textsc{sg.poss}-\textsc{acc}  want-2\textsc{sg}\\
    \glt ‘Which one do you want?’
    
    \ex
    \gll ismɛʁil    abzɨj  tyrkijɛ-gɛ  bar-ma-ʁan=\textbf{{mə}}?\\
    \textsc{pn}    uncle  \textsc{pn}-\textsc{dat}    go-\textsc{neg}-\textsc{ptcp.pst=q}\\
    \glt ‘Has uncle Ismeril (younger brother of the father) never been to Turkey?’
    
    \ex
    \gll sɨz    ɨlgɛrɨ  χaləq  bartʃasə-ʁa  bar-ʁan i-di-gɨz=\textbf{{mɨ}}?\\
    2\textsc{sg.pol}  before  people  park-\textsc{dat}  go-\textsc{ptcp.pst} \textsc{cop}-\textsc{pst}-2\textsc{sg.pol=q}\\
    \glt ‘Have you been to the People’s Park before?’ \citealt{ChenZongzhenYiLiqian1986}: 147, 111)\z\z

The situation is very similar to \ili{Tatar} as spoken in \isi{Russia}, for which there is more information on \isi{intonation}: In \isi{polar question}s the \isi{question marker} is optional and “the pitch is high on the last accented syllable of the clause” \citep[126]{Poppe1963}. Content \isi{questions} remain unmarked and have either falling \isi{intonation} with two elements (\ref{ex:turk:10}b) or first rising and then falling \isi{intonation} with more elements (\ref{ex:turk:10}c).

\ea%10
    \label{ex:turk:10}
    \ili{Tatar} (\isi{Russia})\\
    \ea
    \gll zäkiyä    kil-de=\textbf{{me}}?\\
    \textsc{pn}    come-\textsc{pst}=\textsc{q}\\
    \glt ‘Did Zäkiyä come?’
    
    \ex
    \gll \textbf{{närsä}} bar?\\
    what  \textsc{cop}\\
    \glt ‘What is it?’
    
    \ex
    \gll kičä    sez-gä \textbf{{kem}} kil-de?\\
    yesterday  2\textsc{pl}-\textsc{dat} who  come-\textsc{pst}\\
    \glt ‘Who came to you yesterday?’ \citep[126]{Poppe1963}\z\z

As compared with the other \ili{Turkic} languages mentioned in this chapter, \isi{questions} in \textbf{Kazakh} are exceptionally well described (e.g., \citealt{GengShiminLiZengxiang1985}; \citealt{Muhamedowa2016}: 17--24). However, only some aspects of \isi{question marking} in \ili{Kazakh} can be included here. For further information, the interested reader is referred to the specialized description of \ili{Kazakh} interrogative constructions by \citet{ZhangDingjing1991}.

Polar \isi{questions} are either marked by rising \isi{intonation} or a particle. Rising \isi{intonation} has an additional semantic component of surprise, e.g. \textit{ol oqəwʃə?} ‘Is (s)he a student?’ (\citealt{ZhangDingjing1991}: 99). The question particle \textit{=MA} is often written detached from its host but probably has the status of an enclitic. But the enclitic is not mobile and thus cannot mark \isi{focus question}s as in \ili{Turkish} \citep[17]{Muhamedowa2016}. It marks polar and \isi{alternative question}s. Alternative \isi{questions} may take an optional \isi{disjunction} \textit{ælde}.

\ea%11
    \label{ex:turk:11}
    \ili{Kazakh} (\isi{China})\\
    \ea
    \gll aʁan    da  bar-də=\textbf{{ma}}?\\
    e.brother  also  go-\textsc{pst=q}\\
    \glt ‘Did your elder brother go as well?’ (\citealt{GengShiminLiZengxiang1985}: 119)
    
    \ex
    \gll ol  oqə.wʃə=\textbf{{ma}},    oqə.t.əwʃə=\textbf{{ma}}?\\
    3\textsc{sg}  student=\textsc{q}    teacher=\textsc{q}\\
    \glt ‘Is (s)he a student or a teacher?’
    
    \ex
    \gll olar  ʃaqər-ma-də=\textbf{{ma}} \textbf{{ælde}} bar-ʁəŋ  kel-me-di=\textbf{{me}}?\\
    3\textsc{pl}  call-\textsc{neg}-\textsc{pst}=\textsc{q}  or.\textsc{q}  go-\textsc{pst.pfv}  come-\textsc{neg}-\textsc{pst}=\textsc{q}\\
    \glt ‘Didn’t they call you or did you (simply) not come?’ (\citealt{ZhangDingjing1991}: 99, 104)\z\z

According to \citet[65]{Muhamedowa2016} standard \ili{Kazakh} \textit{älde} is primarily used in the written language and is restricted to \isi{questions}. Standard \isi{disjunction} is expressed with the help of \textit{nemese} ‘or’. This is a distinction also found in \ili{Mandarin} \ili{Chinese} \textit{háishì} \zh{还是} (\isi{interrogative}) versus \textit{huòzhě} \zh{或者} (standard) (\sectref{sec:5.9.2.1}). In \ili{Chinese} \ili{Kazakh}, \isi{negative alternative question}s either take two markers and an optional \isi{disjunction} (\ref{ex:turk:12}a) or, if the second alternative consists of the negative existential exclusively, only the first alternative is marked (\ref{ex:turk:12}b). In this case, the second alternative simply consists of the negator \textit{dʒoq}. This may have been influenced by \ili{Uyghur} as there are examples with question markers from Kazakhstan (\ref{ex:turk:13}b). Spoken \ili{Kazakh} loses its agreement markers if the \isi{question marker} is present. The sentence \textit{kel-e-di=me?} ‘come-\textsc{prs}-3\textsc{sg}=\textsc{q}’ in written \ili{Kazakh} thus has the spoken equivalent \textit{kel-e=me?} \citep[168]{Muhamedowa2016}. The same phenomenon can be observed in the following \isi{alternative question}.

\ea%12
    \label{ex:turk:12}
    \ili{Kazakh} (\isi{China})\\
    \ea
    \gll ol  kel-e=\textbf{{me}},  kel-\textbf{{me}}{-j=}\textbf{{me}}?\\
    3\textsc{sg}  come-\textsc{prs}=\textsc{q}  come-\textsc{neg-prs}=\textsc{q}\\
    \glt ‘Does (s)he come or not?’
    
    \ex
    \gll sarə-maj-də    dʒaqsə    kør-e-siŋ=\textbf{{be}} \textbf{{dʒoq}}?\\
    yellow-oil-\textsc{dat}    good    see-\textsc{prs}-2\textsc{sg}=\textsc{q}  \textsc{neg}\\
    \glt ‘Do you like butter?’ (\citealt{ZhangDingjing1991}: 104)
    \z
    \z

\ea%13
    \label{ex:turk:13}
    \ili{Kazakh} (Kazakhstan)\\
    \ea
    \gll ornija  orïr-a-sïŋ=\textbf{{ba}},    orïr-\textbf{{ma}}{-y-sïŋ=}\textbf{{ba}}?\\
    place  sit-\textsc{prs}-2\textsc{sg}=\textsc{q}    sit-\textsc{neg}-\textsc{prs}-2\textsc{sg}=\textsc{q}\\
    \glt ‘Are you going to sit down at your place or not?’
    
    \ex
    \gll ornija  orïr-a-sïŋ=\textbf{{ba}}, \textbf{{žoq}}{=}\textbf{{pa}}?\\
    place  sit-\textsc{prs}-2\textsc{sg}=\textsc{q}    \textsc{neg}=\textsc{q}\\
    \glt ‘Are you going to sit down at your place or not?’ \citep[18]{Muhamedowa2016}
    \z
    \z

In sentences with second person \isi{singular} agreement forms, the particle \textit{=MA} often has the appearance of a suffix \textit{-MI} that precedes the agreement suffix. But apparently both constructions are usually possible in such cases.

\ea%14
    \label{ex:turk:14}
    \ili{Kazakh} (\isi{China})\\
    \ea
    \gll nurbek-sɨŋ=\textbf{{ba}}?\\
    \textsc{pn}-2\textsc{sg}=\textsc{q}\\
    \glt ‘Are you Nurbek?’
    
    \ex
    \gll nurbek-\textbf{{pɨ}}{-sɨŋ?}\\
    \textsc{pn}-\textsc{q}-2\textsc{sg}\\
    \glt ‘Are you Nurbek?’ (\citealt{GengShiminLiZengxiang1985}: 120)
    \z
    \z

The enclitic and the suffix both follow \isi{vowel harmony} and depend on the preceding consonant, but have different realizations (Tables \ref{tab:turk:2}, \ref{tab:turk:3}, see also \citealt{Kirchner1998a}: 328).

\begin{table}
\caption{Realizations of the Kazakh enclitic \textit{=MA} (\citealt{GengShiminLiZengxiang1985}: 119, passim, \citealt{ZhangDingjing1991}: passim, and \citealt{Muhamedowa2016}: 17)}
\label{tab:turk:2}

\begin{tabularx}{\textwidth}{XXl}
\lsptoprule
& \textbf{after back vowels} & \textbf{after front vowels}\\
\midrule
after \textit{r}, \textit{j}, V & =ma & =me\\
after voiced C & =ba & =be\\
otherwise & =pa & =pe\\
\lspbottomrule
\end{tabularx}
\end{table}

\begin{table}
\caption{Realizations of the Kazakh suffix \textit{-MI} preceding second person agreement forms (\citealt{GengShiminLiZengxiang1985}: 119, passim; \citealt{Muhamedowa2016}: 17)}
\label{tab:turk:3}

\begin{tabularx}{\textwidth}{XXl}
\lsptoprule
& \textbf{after back vowels} & \textbf{after front vowels}\\
\midrule
after \textit{r}, \textit{j}, V & -mə & -mɨ\\
after voiced C & -bə & -bɨ\\
otherwise & -pə & -pɨ\\
\lspbottomrule
\end{tabularx}
\end{table}

In spoken but not written \ili{Kazakh} there is a \isi{tendency} for the enclitic to lose the \isi{vowel harmony} in favor of the back vowel variants \textit{ma}, \textit{ba}, \textit{pa} \citep[17]{Muhamedowa2016}. Content questions do not have a particle or suffix but exhibit rising \isi{intonation} \citep[20]{Muhamedowa2016}.

\ea%15
    \label{ex:turk:15}
    \ili{Kazakh} (\isi{China})\\
    \gll sɨz \textbf{{qajda}} bar-a-sɨz?\\
    2\textsc{sg.pol}  where  go-\textsc{prs}-2\textsc{sg.pol}\\
    \glt ‘Where do you go?’ (\citealt{GengShiminLiZengxiang1985}: 68)
    \z

A marker specialized to inquire about topics that is restricted to the spoken language takes the form =\textit{ʃI} (i.e. \textit{ʃə {\textasciitilde} ʃi}) or \textit{=ʃe} (\citealt{ZhangDingjing1991}: 103, \citealt{Muhamedowa2016}: 19).

\ea%16
    \label{ex:turk:16}
    \ili{Kazakh} (\isi{China})\\
    \gll omar=\textbf{{ʃe}}?\\
    \textsc{pn=q}\\
    \glt ‘What about Omar?’ (\citealt{GengShiminLiZengxiang1985}: 121)
    \z

A marker with a function similar to \isi{tag question} markers found in \isi{content question}s takes the form \textit{æ} and is thus similar to the marker \textit{a} in Kalmyk in both form and function (\sectref{sec:5.8.2}).

\ea%17
    \label{ex:turk:17}
    \ili{Kazakh} (\isi{China})\\
    \gll munə  saʁan \textbf{{kɨm}} ajtə-p    ber-dɨ, \textbf{{æ}}?\\
    this  2\textsc{sg}  who  tell-\textsc{cvb}  give-\textsc{pst}  \textsc{q}\\
    \glt ‘Who told you that, eh?’ (\citealt{GengShiminLiZengxiang1985}: 131)
    \z

\noindent In this function \textit{æ} is accompanied by rising \isi{intonation} and has the specific meaning “that the speaker remains baffled by the whole situation of a (certain) circumstance even after giving it much thought.” (\zh{它表示说话人对事物的真象百思不得其解}, \citealt{ZhangDingjing1991}: 103). \cite[19]{Muhamedowa2016} simply treats standard \ili{Kazakh} \textit{ä} as a \isi{tag question} marker and translates it as ‘right?’. It is accompanied with rising \isi{intonation}.

The \isi{question marker} in \textit{Ky}\textit{rgy}\textit{z} has the form \textit{=BI} \citep[346]{Kirchner1998b}. It takes the form \textit{=bI} after \textit{z}, \textit{m}, \textit{n}, \textit{ŋ}, \textit{l}, \textit{r}, \textit{w}, and \textit{j} but the form \textit{=pI} after any of the following sounds: \textit{p}, \textit{t}, \textit{k}, \textit{s}, \textit{š}, \textit{č}, and \textit{x} \citep[38]{Kara2003}. As opposed to \ili{Kazakh}, there is no variant with an initial nasal. It can attach to variable word classes, which is why it has been analyzed as an enclitic as in \ili{Kazakh}.

\ea%18
    \label{ex:turk:18}
    \ili{Kyrgyz} (Kyrgyzstan)\\
    \gll al  kel-e=\textbf{{bi}}?\\
    3\textsc{sg}  come-\textsc{prs}=\textsc{q}\\
    \glt ‘Is (s)he coming?’ \citep[39]{Kara2003}
    \z

Basically the same situation as in \ili{Kyrgyz} proper can be observed in \ili{Kyrgyz} as spoken in \isi{China}. Here the enclitic has four (as opposed to two in \ili{Kazakh}) different vowel harmonic variants: \textit{=Bə}, \textit{=Bi}, \textit{=Bu}, and \textit{=By} (\citealt{HuZhenhua1986}: 155).

\ea%19
    \label{ex:turk:19}
    \ili{Kyrgyz} (\isi{China})\\
    \gll bul  kitep=\textbf{{pi}}?\\
    this  book=\textsc{q}\\
    \glt ‘Is this a book?’ (\citealt{Hu1979}: 89)
    \z

In the Teskei dialect of \ili{Kyrgyz} spoken in \isi{Xinjiang} (\textit{tiesikai} \zh{铁斯开} in \ili{Chinese}) the question and agreement markers have the reversed order compared to \ili{Kyrgyz} proper, but only in the second person.

\ea%20
    \label{ex:turk:20}
    \ili{Kyrgyz}\\
    \gll kel-e-sing=\textbf{{bü}}?\\
    go-\textsc{prs}-2\textsc{sg}=\textsc{q}\\
    \glt ‘Are you going?’
    \z

\newpage     
\ea%21
    \label{ex:turk:21}
    \ili{Kyrgyz} (Teskei)\\
    \gll kel-e-\textbf{{bi}}{-sin?}\\
    go-\textsc{prs}-\textsc{q}-2\textsc{sg}\\
    \glt ‘Are you going?’ (\citealt{MakelaikeYumai’erbai1986}: 23)
    \z

\noindent This unusual morphosyntactic alternation suggests influence from \ili{Kazakh} or \ili{Uyghur}. No examples for \isi{focus} or \isi{alternative question}s were found in the literature available to me. Content questions are unmarked.

\ea%22
    \label{ex:turk:22}
    \ili{Kyrgyz} (\isi{China})\\
    \gll \textbf{{qaijda}} bar-a-səŋ?\\
    whither  go-\textsc{prs}-2\textsc{sg}\\
    \glt ‘Where are you going?’ (\citealt{HuZhenhua1986}: 174)
    \z

Apart from the \isi{question marker} mentioned above, there is a whole range of additional constructions with fine semantic differences. \isi{Topic question}s take the marker \textit{=tʃI} (i.e. \textit{tʃə} {\textasciitilde} \textit{=tʃi} {\textasciitilde} \textit{=tʃu} {\textasciitilde} \textit{=tʃy}), a cognate of \ili{Kazakh} =\textit{ʃI} {\textasciitilde} \textit{=ʃe}.

\ea%23
    \label{ex:turk:23}
    \ili{Kyrgyz} (\isi{China})\\
    \gll men  bar-a-mən,  siz=\textbf{{tʃi}}?\\
    1\textsc{sg}  go-\textsc{prs}-1\textsc{sg}  2\textsc{sg}=\textsc{q}\\
    \glt ‘I have to go, what about you?’ (\citealt{HuZhenhua1986}: 155)
    \z

\ili{Kyrgyz} has a marker \textit{beken} that is a contraction of the polar \isi{question marker} with a particle \textit{eken}. Similarly, there is a \isi{question marker} \textit{bejim} that derives from a \isi{combination} of the \isi{question marker} with \textit{dejim}.

\ea%24
    \label{ex:turk:24}
    \ili{Kyrgyz} (\isi{China})\\
    \ea
    \gll al  qərʁəz \textbf{{beken}}?\\
    3\textsc{sg}  \textsc{pn}    \textsc{q}\\
    \glt ‘Isn’t he \ili{Kyrgyz}?’
    
    \ex
    \gll al  tatar \textbf{{bejim}}?\\
    3\textsc{sg}  \textsc{pn}    \textsc{q}\\
    \glt ‘(S)he apparently is \ili{Tatar}, right?’ (\citealt{HuZhenhua1986}: 155, 156)
    \z
    \z

\noindent The exact meaning of these forms remains unclear to me.

There are two \textbf{\ili{Uyghur}-Karluk} languages spoken in \isi{Northeast Asia} as defined here, \ili{Uyghur} and \ili{Uzbek}. \ili{Uzbek} is located for the most part outside of the region but is also spoken by a minority in \isi{Xinjiang}, which is why it has been included here. In addition, there is the \ili{Uyghur}-\ili{Persian} mixed language \ili{Eynu} that is included in this chapter because of the apparent similarities in \isi{question marking} to \ili{Uyghur}. Question marking in \textbf{Uyghur} is rather complex. Fortunately we are in \isi{possession} of descriptions of \ili{Uyghur} from the 19th century. According to the description by \citet[56]{Shaw1878}, \ili{Uyghur} has a marker \textit{=mu}. Similarly to \ili{Kazakh} there is a split that in \ili{Uyghur} depends on the tense affix. The default position of \textit{=mu} is the very end of the sentence. However, if the non-past marker is present, the \isi{question marker} has the shorter form \textit{-m} and precedes the agreement marker \citep[56]{Shaw1878}.

  
\ea%25
    \label{ex:turk:25}
    \ili{Uyghur}\\
    \ea
    \gll qel-ding=\textbf{{mu}}?\\
    do-\textsc{pst}=\textsc{q}\\
    \glt ‘Did you do?’
    
    \ex
    \gll qel-a-\textbf{{m}}{-san?}\\
    do-\textsc{npst}-\textsc{q}-2\textsc{sg}\\
    \glt ‘Do you do?’ \citep[56]{Shaw1878}
    \z
    \z

In fact, Shaw’s explanation is more likely than the one by \citet[366]{TuohutiLitifu2012} and \citet[178]{Abdurehim2014}, who claim that in non-past sentences the marker takes the form \textit{-am} {\textasciitilde} \textit{-äm}. Most likely, the element \textit{-a} {\textasciitilde} \textit{-ä} is a variant of the non-past marker and only \textit{-m} is the \isi{question marker}.

\ea%26
    \label{ex:turk:26}
    \ili{Uyghur}\\
    \gll ätä    kel-äl-ä-\textbf{{m}}{-siz?}\\
    tomorrow  come-\textsc{abil}-\textsc{npst}-\textsc{q}-2\textsc{sg.pol}\\
    \glt ‘Can you come tomorrow?’ (\citealt{TuohutiLitifu2012}: 366)
    \z

\noindent But the descriptions agree that the \isi{question marker} usually has the invariable form \textit{=mu}.

\ea%27
    \label{ex:turk:27}
    \ili{Uyghur}\\
    \gll sän  tapšuruq-ni    išlä-p    bol-duŋ=\textbf{{mu}}?\\
    2\textsc{sg}  homework-\textsc{acc} make-\textsc{cvb}  aux-\textsc{pst}.2\textsc{sg=q}\\
    \glt ‘Have you finished your homework?’ (\citealt{TuohutiLitifu2012}: 219)
    \z

\citet[56]{Shaw1878} states that, colloquially, \textit{=mu} may have the form \textit{=ma}. But more likely the form \textit{=ma} is a \isi{combination} of \textit{=mu} and another marker =\textit{a} that expresses astonishment (\citealt{TuohutiLitifu2012}: 366).

\ea%28
    \label{ex:turk:28}
    \ili{Uyghur}\\
    \gll bayiqi    gäp-ni    aŋli-mi-diŋ=\textbf{{ma}}?\\
    just.now  talk-\textsc{acc}  hear-\textsc{neg}-\textsc{pst}=\textsc{q}\\
    \glt ‘Didn’t you hear what was just said?’ (\citealt{TuohutiLitifu2012}: 367)
    \z

However, the Lopnor dialect of \ili{Uyghur} does not show the fusion with the tense marker “when the second person or \isi{plural} suffixes are attached” and also has a variant \textit{mi}, e.g. \textit{yürü-y-}\textbf{\textit{mi}}\textit{-siz} ‘go-\textsc{npst}-\textsc{q}-2\textsc{sg.pol}’ (\citealt{Abdurehim2014}: 178f.). But the author clearly contradicts himself and also gives the following example: \textit{sat-a-}\textbf{\textit{m}}\textit{-sän} ‘sell-\textsc{npst}-\textsc{q}-2\textsc{sg} \citep[208]{Abdurehim2014}. Polar \isi{questions} in \ili{Uyghur} can also be formed by rising \isi{intonation} alone \citep[208]{Abdurehim2014}. Content \isi{questions} remain unmarked.

\ea%29
    \label{ex:turk:29}
    \ili{Uyghur}\\
    \ea
    \gll \textbf{{nima}} bâr?\\
    what  exist\\
    \glt ‘What is there?’ \citep[81]{Shaw1878}
    
    \ex
    \gll bu \textbf{{nemä}}?\\
    this  what\\
    \glt ‘What is this?’ (\citealt{TuohutiLitifu2012}: 367)
    \z
    \z

Alternative \isi{questions} have an optional \isi{disjunction} \textit{yaki} and two obligatory question markers. Most languages with two question markers have the identical \isi{question marker} on the respective alternatives. The identical part of the two alternatives that is prone to ellipsis usually remains unmarked. Such a situation can also be found in \ili{Uyghur} (see example \ref{ex:turk:32}a below). However, there is a typologically very special situation in \ili{Uyghur} in which the first of the two markers attaches to the identical element of the two alternatives and has a form different from the second marker.

\ea%30
    \label{ex:turk:30}
    \ili{Uyghur}\\
    \gll bügün    kel-a-\textbf{{m}}-siz      (\textbf{{yaki}})  äti=\textbf{{mu}}?\\
    today    come-\textsc{npst}-\textsc{q}-2\textsc{sg.pol}    or  tomorrow=\textsc{q}\\
    \glt ‘Do you come today or tomorrow?’ (\citealt{TuohutiLitifu2012}: 321)
    \z

A similar construction from Kazakhstan has been mentioned by \citet[18]{Muhamedowa2016} but was left unexplained as the \ili{Uyghur} pattern. In this example, the first \isi{question marker} retains its shape, but see above on \ili{Kazakh} for cases in which the agreement marker follows the \isi{question marker} as in \ili{Uyghur}.

\ea%31
    \label{ex:turk:31}
    \ili{Kazakh} (Kazakhstan)\\
    \gll samsung  al-a-sïŋ=\textbf{{ba}},    ayfon=\textbf{{ba}}?\\
    \textsc{pn}    buy-\textsc{prs}-2\textsc{sg}=\textsc{q}  \textsc{pn}=\textsc{q}\\
    \glt ‘Will you buy a Samsung or an iPhone?’ \citep[18]{Muhamedowa2016}
    \z

The form of \isi{negative alternative question}s depends on the clause type that determines the type of negator.

\ea%32
    \label{ex:turk:32}
    \ili{Uyghur}\\
    \ea
    \gll sän  zadi    bar-a-\textbf{{m}}-sän,  bar-\textbf{{ma}}-\textbf{{m}}{-sän?}\\
    2\textsc{sg}  actually  go-\textsc{prs}-\textsc{q}-2\textsc{sg}  go-\textsc{neg}-\textsc{q}-2\textsc{sg}\\
    \glt ‘Will you actually go or not?’
    
    \ex
    \gll saŋa    jan  keräk=\textbf{{mu}} \textbf{{ämäs}}{=}\textbf{{mu}}?\\
    2\textsc{sg}.\textsc{dat} life  need=\textsc{q}  \textsc{neg}=\textsc{q}\\
    \glt ‘Do you want to live?’
    
    \ex
    \gll bašqi-lar-din  al-γan    qärz-ni    qaytur-al-a-\textbf{{m}}{-sän} \textbf{{yoq}}?\\
    other-\textsc{pl}-\textsc{abl}  take-\textsc{p.pfv}  debt-\textsc{acc}  pay-\textsc{abil}-\textsc{q}-2\textsc{sg}  \textsc{neg}\\
    \glt ‘Can you pay the debt you have from the others or not?’ (\citealt{TuohutiLitifu2012}: 368)\z\z

Notice the absence of the second \isi{question marker} in the last example with the negative existential \textit{yoq}, a situation already encountered in \ili{Chinese} \ili{Kazakh}. There are several additional sentence-final question markers, the meaning of which is not absolutely clear, e.g. \textit{ɣu}, \textit{du} \citep[208]{Abdurehim2014}. But \textit{hä} is clearly a \isi{question tag}, and \textit{ču} must be cognate with the topic \isi{question marker} from \ili{Kyrgyz} and \ili{Kazakh} seen above.

There are only a few descriptions of the mixed language \textit{Eynu}. But the same \isi{question marker} \textit{=mu} as in \ili{Uyghur} is present. Consider the following pair of \isi{negative alternative question}s in \ili{Eynu} and \ili{Uyghur}.

\ea%33
    \label{ex:turk:33}
    \ili{Eynu}\\
    \gll xani-da  mike  hes=\textbf{{mu}}, \textbf{{nist}}{=}\textbf{{mu}}?\\
    house-\textsc{loc}  goat  \textsc{cop}=\textsc{q}    \textsc{neg}=\textsc{q}\\
    \z

\ea%34
    \label{ex:turk:34}
    \ili{Uyghur}\\
    \gll öj-de    öʃke  bar=\textbf{{mu}} \textbf{{joq}}{(=}\textbf{{mu}})?\\
    house-\textsc{loc}  goat  \textsc{cop=q}    \textsc{neg(}=\textsc{q)}\\
    \glt ‘Are there any goats in the house or not?’ \citep[245]{Wurm1997}
    \z

\noindent \citet[860]{Lee-Smith1996a}, who has the same example in a slightly different transliteration, does not have the second \isi{question marker} in \ili{Uyghur}. As can be seen, the grammatical structure of the \ili{Eynu} example is nearly identical to \ili{Uyghur}. However, the lexical stems have a \ili{Persian} origin. According to \citet[315]{ZhaoAximu2011}, there is only one \isi{question marker} in \ili{Uyghur} but two in \ili{Eynu}. Content questions in \ili{Eynu} are unmarked as in \ili{Uyghur}.

\ea%35
    \label{ex:turk:35}
    \ili{Eynu}\\
    \gll ma  jɛk  tʃɛʃmɛ  kɛs \textbf{{kim}}?\\
    this  one  eye  person  who\\
    \z

\ea%36
    \label{ex:turk:36}
    \ili{Uyghur}\\
    \gll ma  jɛktʃɛʃmɛ  kiʃi \textbf{{kim}}?\\
    this  one.eyed  person  who\\
    \glt ‘Who is this one-eyed person?’ (\citealt{ZhaoAximu1981}: 46)
    \z

In this example \ili{Eynu} \textit{kɛs} is of \ili{Turkic} origin while \ili{Uyghur} has the same \ili{Persian} loanword for ‘one-eyed’ as \ili{Eynu}. Content \isi{questions} in \ili{Eynu} remain unmarked.

\ea%37
    \label{ex:turk:37}
    \ili{Eynu}\\
    \gll siz \textbf{{nidʒej}}-dɨn  ɦɛs\_bol?\\
    2\textsc{sg.?pol}  where-\textsc{abl}  come.\textsc{pst.2sg}\\
    \glt ‘Where did you come from?’ (\citealt{Hayasi1999}: 31)
    \z

\ili{Uyghur} has the two \isi{tag question} markers \textit{šundaq=mu} and \textit{šundaq=qu} that contain the polar \isi{question marker} or another unidentified marker in \isi{combination} with a demonstrative.

\ea%38
    \label{ex:turk:38}
    \ili{Uyghur}\\
    \gll ular  bügün  käl-mä-ydu, \textbf{{šundaq=qu}}?\\
    3\textsc{pl}  today  come-\textsc{neg}-3\textsc{npst}  just.so=\textsc{q}\\
    \glt ‘You are not coming today, right?’ (\citealt{TuohutiLitifu2012}: 322)
    \z

\isi{Topic question}s such as \textit{sän-}\textbf{\textit{ču}}? ‘what about you?’ exhibit a marker \textit{-ču} (\citealt{TuohutiLitifu2012}: 218) that is probably cognate with the forms seen before in \ili{Kazakh} and \ili{Kyrgyz}. The form dubitative \textit{m’ikin} ‘is it?, may it be?’ \citep[81]{Shaw1878}, is cognate with \ili{Kyrgyz} \textit{beken} and “expresses more of hesitancy between two opinions than the simple \textit{mu}” \citep[56]{Shaw1878}.

One of the least known \ili{Turkic} languages in \isi{NEA} is probably \textbf{Ili Turki} (\citealt{ZhaoAximu1985}). There are only a handful of examples for \isi{questions}. But these are sufficient to illustrate the \isi{question marker} \textit{=MA}.

\ea%39
    \label{ex:turk:39}
    \ili{Ili Turki}\\
    \ea
    \gll bar-dı=\textbf{{ma}}?\\
    go-\textsc{pst}=\textsc{q}\\
    \glt ‘Did (she) go?’
    
    \ex
    \gll bar-dı-q=\textbf{{pa}}?\\
    go-\textsc{pst}-1\textsc{pl}=\textsc{q}\\
    \glt ‘Did we go?’ \citep[31]{Hahn1991}
    \z
    \z

\tabref{tab:turk:4} illustrates that the difference from surrounding \ili{Turkic} languages is mostly phonological in nature.

\begin{table}
\caption{A comparison of three interrogative sentences in six Turkic languages (\citealt{Zhao1989}: 278f., slightly modified)}
\label{tab:turk:4}

\begin{tabularx}{\textwidth}{XXXl}
\lsptoprule
& \textbf{‘Did he give?’} & \textbf{‘Did you see?’} & \textbf{‘Did we go?’}\\
\midrule
\ilit{Ili Turki} & berdi=mä? & k\.{o}rd\.{u}ŋ=bä? & bardıq=pa?\\
\ilit{Kazakh} & berdi=me? & kördiŋ=be? & bärdıq=pa?\\
\ilit{Kyrgyz} & berdi=mi? & kördüŋ=bü? & bärdıq=pı?\\
\ilit{Uzbek} & berdi=mi? & kordiŋ=mi? & bårdiq=mi?\\
\isit{Xinjiang} \ilit{Uzbek} & berdi=mi? & kördiŋ=mi? & bardiq=mi?\\
\ilit{Uyghur} & bärdi=mu? & kördüŋ=mu? & barduq=mu?\\
\lspbottomrule
\end{tabularx}
\end{table}

\largerpage
\textbf{Uzbek} has a polar \isi{question marker} \textit{=mi}, which is “accompanied by a rising pitch in the preceding syllable.” \citep[373]{Boeschoten1998}. The same marker is present in \ili{Uzbek} as spoken in Xijiang (\tabref{tab:turk:4}). As in \ili{Uyghur}, the marker has one form only. It can attach to different word classes.

\newpage 
\ea%40
    \label{ex:turk:40}
    \ili{Uzbek}\\
    \ea
    \gll mümkin=\textbf{{mi}}?\\
    possible=\textsc{q}\\
    \glt ‘Is it possible?’\footnote{This may also be pronounced as \textit{mümkimmi}.}
    
    \ex
    \gll futbol    oyin-g\.{a}  bår-\.{a}ylik=\textbf{{mi}}?\\
    football  match-\textsc{dat}  go-\textsc{vol=q}\\
    \glt ‘Shall we go to the football mtach?’ (\citealt{Boeschoten1998}: 360, 368)
    \z
    \z

\noindent When second person agreement forms are present, the \isi{question marker} may either precede or follow the agreement marker.

\ea%41
    \label{ex:turk:41}
    \ili{Uzbek}\\
    \ea
    \gll såγ-\textbf{{mi}}{-siz?}\\
    good-\textsc{q}-2\textsc{pl}\\
    \glt ‘Are you well?’
    
    \ex
    \gll {k\.el\.{a}-siz=}\textbf{{mi}}?\\
    come-2\textsc{pl}=\textsc{q}\\
    \glt ‘Will you come?’ \citep[373]{Boeschoten1998}
    \z
    \z

By now this phenomenon should be familiar from several languages seen before. As expected, alternative \isi{questions} take two markers and content \isi{questions} remain unmarked. Note the \isi{alternative question} following a \isi{content question} (\sectref{sec:4.4}).

\ea%42
    \label{ex:turk:42}
    \ili{Uzbek}\\
    \ea
    \gll {m\.eŋ\.{a}} \textbf{{qaysi.si}} yaraš-adi,  qizili=\textbf{{mi}},  åqi=\textbf{{mi}}?\\
    1\textsc{sg}.\textsc{dat}  which    suit-?\textsc{cop.pst}  red=\textsc{q}    white=\textsc{q}\\
    \glt ‘Which one suits me, the red one or the white one?’
    
    \ex
    \gll {k\.eč\.{a}} \textbf{{kim}}-lar  k\.el-di?\\
    tonight    who-\textsc{pl}    come-\textsc{pst}\\
    \glt ‘What persons have arrived tonight?’ \citep[371]{Boeschoten1998}
    \z
    \z

From the \textbf{Siberian} branch ten southern and two northern languages are included in this study. Let me first address the southern subbranch. \textbf{\ili{Tuvan}} has the expected polar \isi{question marker} \textit{be}. But there is a marker \textit{-Il} that appears to be developing into a content \isi{question marker}, e.g. \textit{kɨm-ɨl?} ‘Who is it?’ (\citealt{AndersonHarrison1999}: 88f.). The marker is optional as there are also unmarked \isi{content question}s.

\ea%43
    \label{ex:turk:43}
    \ili{Tuvan}\\
    \ea
    \gll xlep  bar=\textbf{{be}}?\\
    bread  \textsc{cop=q}\\
    \glt ‘Is there bread?’
    
    \ex
    \gll \textbf{{kayɨɨn}} kel-gen    siler?\\
    whence  come-\textsc{pst.I}  2\textsc{pl}\\
    \glt ‘Where have you come from?’ (\citealt{AndersonHarrison1999}: 69, 28)
    \z
    \z

The situation in \textbf{Dzungar Tuvan} spoken in \isi{China} is almost identical. But the polar \isi{question marker} is \textit{=BA} with \isi{vowel harmony} as well as variation of the consonant, both of which are absent in \ili{Tuvan} proper.

\ea%44
    \label{ex:turk:44}
    \ili{Tuvan} (Dzungar)\\
    \ea
    \gll ol  gel-di=\textbf{{be}}?\\
    3\textsc{sg}  come-\textsc{pst.II=q}\\
    \glt ‘Has (s)he come?’
    
    \ex
    \gll sen \textbf{{ɢaʒan}} dʒor-ur-sen?\\
    2\textsc{sg}  when  leave-?\textsc{fut}-2\textsc{sg}\\
    \glt ‘When will you leave?’
    
    \ex
    \gll suu-bəs-təŋ    bir.si \textbf{{qajda}}-\textbf{{l}}?\\
    water-1\textsc{pl}.\textsc{poss}-?\textsc{abl}  some  where-\textsc{q}\\
    \glt ‘Where has some of our water gone?’ (\citealt{WuHongwei1999}: 113, 144, 54)\z\z

\noindent \citet[146]{WuHongwei1999} mentions that \isi{questions} may exhibit a certain question \isi{intonation}, but leaves open the details.

Basically the same pattern seen in the two varieties of \ili{Tuvan} above can also be found in \textbf{Dukhan}. Again, the polar \isi{question marker} \textit{=BA} has more varieties than in standard \ili{Tuvan} and the content \isi{question marker} is optional.

\ea%45
    \label{ex:turk:45}
    \ili{Dukhan}\\
    \ea
    \gll batə  gel-gen-\textbf{{be}}?\\
    \textsc{pn}  come-\textsc{post}-\textsc{q}\\
    \glt ‘Has Bat arrived?’
    
    \ex
    \gll \textbf{{gïm}} gel-gen?\\
    who  come-\textsc{post}\\
    \glt ‘Who has come?’
    
    \ex
    \gll am  pis \textbf{{ganǰaar}}-\textbf{{əl}}?\\
    now  1\textsc{pl}  which.\textsc{v.intra.lf}-\textsc{q}\\
    \glt ‘And now what should we do?’ (\citealt{Ragagnin2011}: 193, 131, 188)\z\z

\noindent In \ili{Dukhan}, the content \isi{question marker} \textit{-(Ĭ)l} is said to have an additional intensifying function, which would explain its absence in some sentences in \ili{Tuvan} as well.

A negative \isi{tag question} in \ili{Dukhan} has the form of a negative copula followed by the polar \isi{question marker}.

\clearpage %anchor
\ea%46
    \label{ex:turk:46}
    \ili{Dukhan}\\
    \gll gel-gen \textbf{{emes=pe}}?\\
    come-\textsc{post}  \textsc{neg=q}\\
    \glt ‘(S)he arrived, didn’t (s)he?’ \citep[187]{Ragagnin2011}
    \z

This pattern has the form \textit{eves=be} in \ili{Tuvan} \citep[23]{Harrison2005} and \textit{emes=pe} in Dzungar \ili{Tuvan} \citep[209]{Mawkanuli2005}. In \ili{Sarikoli} there is a similar \isi{question tag} \textit{ɛmɛs hɛˑ} with a \isi{question marker} that is said to also occur in \ili{Uyghur} (\citealt{Hayasi1999}: 31-32). There is an areal connection of this construction to similar constructions in \ili{Mongolic} (e.g., \ili{Mongolian} \textit{bish=uu}, \sectref{sec:5.8.2}), as well as other \ili{Turkic} languages (e.g., \ili{Turkish} \textit{değil=mi}, see above). Tag \isi{questions} in \ili{Tuvan} have the final element \textit{ale}, \ili{Dukhan} has \textit{hala} {\textasciitilde} \textit{harən} \citep[187]{Ragagnin2011}.

\textbf{Tofa} (previously also called \ili{Karagas}), like \ili{Tuvan}, has the invariable marker \textit{=be} (\citealt{Schönig1998}: 414). According to \cite[71]{Castrén1857b}:, there is a variation between two forms \textit{-bè} {\textasciitilde} \textit{-pè} that can attach to different word classes, which is why they can be considered enclitics. Alternative questions, for which no example was given, take the same marker on each alternative. Content questions have a marker \textit{-(u)l} that is cognate with \ili{Dukhan} \textit{-(Ĭ)l} etc.

\ea%47
    \label{ex:turk:47}
    \ili{Tofa}\\
    \ea
    \gll onu    soodap    beer=\textbf{{be}}?\\
    3\textsc{sg}.\textsc{acc} say.\textsc{cvb}  \textsc{obj}.\textsc{vers}.\textsc{p/f}=\textsc{q}\\
    \glt ‘Should I say it again (for you)?’ \citep[260]{Anderson2001}
    
    \ex
    \gll ad-ïŋ \textbf{{qum}}-\textbf{{ul}}?\\
    name-2\textsc{sg}.\textsc{poss}  who-\textsc{q}\\
    \glt ‘\isi{What is your name?}’ (\citealt{Schönig1993}: 199)
    \z
    \z

\noindent Question marking in \ili{Tofa} is thus almost identical to \ili{Tuvan}.

For \textbf{Khakas} (previously also called \ili{Koibal}), \cite[71]{Castrén1857b} mentions a polar \isi{question marker} \textit{-BA} (i.e., \textit{-ba}, \textit{-bä}, \textit{-pa}, \textit{-pä}) that should be considered an enclitic as well. There are additional variants with an initial nasal not mentioned by Castrén (\citealt{Anderson1998}: passim). Alternative questions take two markers and \isi{content question}s are unmarked.

\ea%48
    \label{ex:turk:48}
    \ili{Khakas}\\
    \ea
    \gll sɪrer  xakas-ta-p  čooxta-pča-zar=\textbf{{ba}}?\\
    2\textsc{pl}  \textsc{pn}-\textsc{v}-\textsc{cvb}  speak-\textsc{prs}-2\textsc{pl=q}\\
    \glt ‘Do you (pl.) speak \ili{Khakas}?’
    
    \ex
    \gll \textbf{{kem}} pɪl-er    anɨ?\\
    who  know-\textsc{fut}  3\textsc{sg}.\textsc{acc}\\
    \glt ‘Who knows him/her?’ \citep[87]{Anderson1998}
    \z
    \z

The following polar and \isi{focus question}s were given to me in January 2016 by a native \ili{Khakas} living in Germany with the help of several \ili{Khakas} speakers in \isi{Russia}. The transcription and \isi{analysis} roughly follow \citet{Anderson1998}.

\ea%49
    \label{ex:turk:49}
    \ili{Khakas}\\
    \ea
    \gll sin    taŋ.da,    škola    par-ča-zyŋ?\\
    2\textsc{sg}    tomorrow  school    go-\textsc{prs}-2\textsc{sg}\\
    \glt ‘Are you going to school tomorrow?’
    
    \ex
    \gll \uline{sin}    taŋ.da,    škola    par-ča-zyŋ?\\
    2\textsc{sg}    tomorrow  school    go-\textsc{prs}-2\textsc{sg}\\
    \glt ‘Is it \textit{you} who is going to school tomorrow?’
    
    \ex
    \gll \uline{taŋ.da},    sin    škola    par-ča-zyŋ?\\
    tomorrow  2\textsc{sg}    school    go-\textsc{prs}-2\textsc{sg}\\
    \glt ‘Is it \textit{tomorrow} that you are going to school?’
    
    \ex
    \gll sin    \uline{škola},    taŋ.da    par-ča-zyŋ?\\
    2\textsc{sg}    school    tomorrow  go-\textsc{prs}-2\textsc{sg}\\
    \glt ‘Is it \textit{to school} that you are going tomorrow?’\z\z

\noindent No polar \isi{question marker} was present. Detailed information on \isi{intonation} is not available to me, but \isi{focus} is also clearly marked with \isi{word order}, especially sentence initial position. In one instance the focused element stands in second position.

However, \ili{Khakas} has a special \isi{interrogative verb} ending \textit{-ǯAŋ} found in \isi{content question}s that expresses “semi-rhetorical utterances like ‘how is it possible that...?’, when the speaker believes what they are questioning to in fact not be possible or appropriate” \citep[38]{Anderson1998}. It often combines with interrogatives such as \textit{noɣa} ‘why’, \textit{xaydi} ‘how’ or \textit{xaydar} ‘whither’.

\ea%50
    \label{ex:turk:50}
    \ili{Khakas}\\
    \gll xɨɣɨr-ɣan  čir-zer \textbf{{xaydi}} par-ba-\textbf{{ǯaŋ}}?\\
    invite-\textsc{ppst}  place-\textsc{all}  how  go-\textsc{neg}-\textsc{q}\\
    \glt ‘How is it possible not to go where one was invited?’ \citep[38]{Anderson1998}
    \z

\noindent Given that \textit{-ǯAŋ} originally expressed the habitual past, a connection to \ili{Samoyedic} languages seems possible (\sectref{sec:5.12.2}). In \ili{Enets} and \ili{Nenets}, for example, the suffix \textit{-sa} used to be a past tense marker but acquired an interrogative meaning as well. A difference is that it can be found in both polar and \isi{content question}s.

According to \citet[29]{HuImart1987} \textbf{Fuyu} has a \isi{question marker} \textit{=BA} that has at least eight different realizations, \textit{ba}, \textit{b\u{\i}}, \textit{pa}, \textit{p\u{\i}}, \textit{ma}, \textit{m\u{\i}}, \textit{$\beta $a}, and \textit{$\beta $\u{\i}}. They write the \isi{question marker} attached to the preceding word, but it has been analyzed as enclitic here as it can be attached to a verbal or non-verbal host. Content \isi{questions} remain unmarked, e.g. \textit{ol \textbf{g\u{\i}m}?} ‘Who is (s)he?’ (\citealt{HuImart1987}: 37).

\ea%51
    \label{ex:turk:51}
    \ili{Fuyu}\\
    \gll ol  sin-iŋ    balaŋ=\textbf{{ma}}?\\
    3\textsc{sg}  2\textsc{sg}-\textsc{gen}  son=\textsc{q}\\
    \glt ‘Is he your son?’ (\citealt{HuImart1987}: 29, 37)
    \z

In \textbf{Sarig Yughur} the polar quesion marker has the form \textit{=mi} and an older variant \textit{=pi} after plosives as well as a stressed version \textit{=mʊ} \citep[152]{Roos2000}. It has the reduced form \textit{-m} after the past tense suffix \textit{-(}\textit{\textsuperscript{h}}\textit{)tï} and the evidential \textit{-tï}.

\ea%52
    \label{ex:turk:52}
    \ili{Sarig Yughur}\\
    \ea
    \gll yahȿ=\textbf{{mu}}?\\
    good=\textsc{q}\\
    \glt ‘How are you?’ (a greeting)
    
    \ex
    \gll sen  uzï-\textsuperscript{h}{tï-}\textbf{{m}}?\\
    2\textsc{sg}  sleep-\textsc{pst}-\textsc{q}\\
    \glt ‘Did you sleep?’ \citep[152]{Roos2000}
    \z
    \z

\noindent The variant \textit{=mu} as well as the suffix \textit{-m} indicate strong influence from \ili{Uyghur}. The difference from \ili{Uyghur} only concerns the lack of a person marker following \textit{-m}, but this category is altogether absent from \ili{Sarig Yughur} \citep[100]{Roos2000}. In the first two examples \textit{=mi} is also possible. According to \citet[76]{ChenZongzhen1982} the \isi{question marker} has the slightly different form \textit{=be}, \textit{=me {\textasciitilde} -m}, but no example of the former has been given. The distribution of \textit{=me} and \textit{-m} confirms Roos’ description. (Negative) \isi{alternative question}s take two question makers.

\ea%53
    \label{ex:turk:53}
    \ili{Sarig Yughur}\\
    \gll kʊ  ki-ȿtï-\textbf{{m}} ki-\textbf{{ɣïmstï}}-\textbf{{m}}?\\
    3\textsc{sg}  come-\textsc{fut}.\textsc{ev}-\textsc{q}  come-\textsc{fut}.\textsc{neg}.\textsc{ev}-\textsc{q}\\
    \glt ‘Will (s)he come or not?’ \citep[152]{Roos2000}
    \z

Roos mentions another example of an \isi{alternative question} with a very unusual structure. If there is no mistake, the sentence has a \isi{question marker} at the end of the sentence \textit{in addition} to the two markers already present. Moreover, there is a \isi{disjunction} that \textit{follows} both alternatives.

\ea%54
    \label{ex:turk:54}
    \ili{Sarig Yughur}\\
    \gll sen  puɣïn=\textbf{{mi}},  taɣïn=\textbf{{mi}} \textbf{{ta}}\textbf{{\textsuperscript{h}}}\textbf{{qï}} {c\textsuperscript{h}}unʕaŋ-qa  kuŋcuola-ȿ=\textbf{{mi}}?\\
    2\textsc{sg}  today=\textsc{q}  tomorrow=\textsc{q}  or  chief-\textsc{dat}  work-\textsc{fut}=\textsc{q}\\
    \glt ‘Will you work for the chief today or tomorrow?’ \citep[152]{Roos2000}
    \z

In \textbf{Shor} the \isi{question marker} has the variants \textit{=ba}, \textit{=be}, \textit{=pa}, \textit{=pe}, \textit{=ma}, \textit{=me} \citep[505]{Donidze1997}, which is identical to \ili{Altai Turkic}. Polar \isi{questions} take one \citep[291]{Nevskaja2000}, content \isi{questions} are unmarked, \isi{alternative question}s take two markers.

\ea%55
    \label{ex:turk:55}
    \ili{Shor}\\
    \ea
    \gll \textbf{{kem}} kel-du?\\
    who  come-\textsc{pst}\\
    \glt ‘Who came?’
    
    \ex
    \gll ak  kazyn  özer=\textbf{be} {čok=}\textbf{{pa}}?\\
    white  ?birch  ?grow=\textsc{q}  \textsc{neg=q}\\
    \glt ‘Is the white birch tree growing or not? \citep[505]{Donidze1997}
    \z
    \z

The \isi{question marker} in \textbf{Altai Turkic} has the same variants \textit{=ba}, \textit{=be}, \textit{=pa}, \textit{=pe}, \textit{=ma}, \textit{=me} as in \ili{Shor} \citep[183]{Baskakov1997}. In addition there is an enclitic \textit{=na} {\textasciitilde} \textit{=ne}. The difference in meaning has not been specified, but the latter marker may have dubitative meaning.

\ea%56
    \label{ex:turk:56}
    \ili{Altai Turkic}\\
    \ea
    \gll albyŋ=ba?\\
    ?receive=q\\
    \glt ‘Have you received?’
    
    \ex
    \gll olor  kel-gen=ne?\\
    3\textsc{pl}  come-?\textsc{p.pst}=\textsc{q}\\
    \glt ‘(I wonder whether) they came?’ (\citealt{Baskakov1997}: 183)
    \z
    \z

\noindent Content \isi{questions} appear to be unmarked, e.g. \textbf{\textit{kajda}} \textit{sin?} ‘Where are you?’ \citep[104]{Baskakov1958b}.

The newly identified South \ili{Siberian Turkic} language \textbf{Chalkan} offers a picture that strongly resembles its closely related languages such as Altai Turkic. Polar \isi{questions} take a sentence-final \isi{question marker} and \isi{content question}s remain unmarked.

\ea%57
    \label{ex:turk:57}
    \ili{Chalkan}\\
    \ea
    \gll eme  t’etire  uyuqta-p-sïn=\textbf{{mï}}?\\
    now  till  sleep-\textsc{prs}-2\textsc{sg}=\textsc{q}\\
    \glt ‘Have you been sleeping till now?’
    
    \ex
    \gll t’e, \textbf{{qančïn}}-dan    kečirsin?\\
    well  how.much-\textsc{abl}  ferry.\textsc{prs}.2\textsc{sg}\\
    \glt ‘Well, how much is it for you to ferry us (over the river)?’ \citep[78]{Nevskaja2014}
    
    \ex
    \gll \textbf{{kem}}-niŋ  tiž-i      apaaš’    aa-ŋ=\textbf{{ma}} meeŋ=\textbf{{me}}?\\
    who-\textsc{gen}  tooth-3\textsc{sg.poss} rather.white  3\textsc{sg}-\textsc{gen}=\textsc{q} 1\textsc{sg.gen}=\textsc{q}\\
    \glt ‘Whose teeth are especially white, his/hers or mine?’ (\citealt{Erdal2013}: 98)\z\z

\noindent The exact variants of the \isi{question marker} remain unclear.

Little material is also available for \isi{questions} in \textbf{Chulym} \textbf{Turkic} (also called Ös), but some material based on fieldwork has been published by \citet{Harrison2003} and \citet{AndersonHarrison2006}. Unfortunately, their data do not contain an example of a \isi{polar question}, but there are several \isi{content question}s as well as one open \isi{alternative question} in which the second part is \textit{ili qajdɯɣ?} ‘or what?’ (\citealt{AndersonHarrison2006}: 57). The \isi{disjunction} \textit{ili} as well as the accompanying construction has been borrowed from \ili{Russian}. Content \isi{questions} are unmarked.

\ea%58
    \label{ex:turk:58}
    \ili{Chulym}\\
    \gll \textbf{{kajnaar}} bar-eydi-ŋ?\\
    whither  go-\textsc{prs}-2\textsc{sg}\\
    \glt ‘Where are you going?’ (\citealt{Harrison2003}: 250)
    \z

There is no information on \isi{intonation} either. \citet[184]{AndersonHarrison2004} mention an example of a \isi{polar question} provided by a Middle \ili{Chulym} speaker, i.e. \textit{uluɣ=}\textbf{\textit{be}}? ‘Was it big?’. But apparently, the sentence represents an example of code mixing with \ili{Tuvan}. The same \isi{question marker} also exists in \ili{Chulym}, but has the form \textit{=ba} according to \citet{Birjukovich1997}.

\ea%59
    \label{ex:turk:59}
    \ili{Chulym}\\
    \gll män  par-ad-ym=\textbf{{ba}}?\\
    1\textsc{sg}  go-\textsc{prs}-1\textsc{sg=q}\\
    \glt ‘Do I go?’ \citep[496]{Birjukovich1997}
    \z

\noindent Perhaps the form \textit{=be} was simply borrowed from \ili{Tuvan}. However, according to the data by \cite[97]{LiYong-Sŏng2008} in Middle \ili{Chulym}, apart from \textit{=ba}, there is a second variant \textit{=bä} that follows front vowels and could be identical to \textit{=be} seen above.

\ea%60
    \label{ex:turk:60}
    \ili{Chulym} (Middle)\\
    \gll seeŋ    eer-iŋ      äp-te=\textbf{bä}?\\
    2\textsc{sg}.\textsc{gen}  husband-2\textsc{sg}.\textsc{poss}  house-\textsc{loc}=\textsc{q}\\
    \glt ‘Is your husband at home?’ (\citealt{LiYong-Sŏng2008}: 98)
    \z

But interestingly, the \isi{question marker} does not necessarily stand sentence-final, but may be followed by other elements. The description by \cite{LiYong-Sŏng2008} is insufficient for a clear \isi{analysis}. The example is not a \isi{focus question} in which the mobile \isi{question marker} attaches to the focused element. Instead, one element from the sentence simply follows the predicate, which is the host for the \isi{question marker}. Perhaps this is a \isi{focus question} in which \isi{focus} is expressed with the help of \isi{word order}. The following sentence was translated as ‘Do you have good news?’, which seems to be inadequate.

\ea%61
    \label{ex:turk:61}
    \ili{Chulym} (Middle)\\
    \gll siler-niŋ  čaqšï=\textbf{ba} vest’?\\
    2\textsc{pl}-\textsc{gen} good=\textsc{q}  news\\
    \glt ‘Is your news good?’ (\citealt{LiYong-Sŏng2008}: 210)
    \z

In \textbf{Yakut} and \textbf{Dolgan}, the two northern \ili{Siberian Turkic} languages, there is an enclitic \textit{=duo} {\textasciitilde} \textit{=duu} (\citealt{Stachowski1993}: 83; \citealt{Ebata2011}: 197) that marks polar and \isi{alternative question}s and another marker \textit{-(n)ɪj} \citep[197]{Ebata2011} or -\textit{Iy} (\citealt{Stachowski1998}: 423), showing \isi{vowel harmony}, that is found in \isi{content question}s. The first marker is unique among \ili{Turkic} languages. It can also be found in Kolyma \ili{Yukaghir} \citep[245]{Nagasaki2011} but does not exist in Tundra \ili{Yukaghir} (see \sectref{sec:5.14.2}).

\ea%62
    \label{ex:turk:62}
    \ili{Yakut}\\
    \ea
    \gll kyœrəʁej  ɯrɯa-tɯ-n    ist-e-ʁin=\textbf{{duo}}?\\
    lark    song-3\textsc{sg}.\textsc{poss}-\textsc{acc}  hear-\textsc{prs}-2\textsc{sg}=\textsc{q}\\
    \glt ‘Do you hear the song of larks?’
    
    \ex
    \gll oʁo-ŋ      kɯɯs=\textbf{{duu}} uol=\textbf{{duu}}?\\
    child-2\textsc{sg}.\textsc{poss}  daughter.\textsc{cop}.3\textsc{sg}=\textsc{q}  son.\textsc{cop}.3\textsc{sg}=\textsc{q}\\
    \glt ‘Is your child a daughter or a son?’
    
    \ex
    \gll \textbf{{xanna}} baar-gɯn-\textbf{{ɯj}}?\\
    where exist-\textsc{cop}.2\textsc{sg}-\textsc{q}\\
    \glt ‘Where are you?’ \citep[197]{Ebata2011}\z\z

\ea%63
    \label{ex:turk:63}
    \ili{Dolgan}\\
    \ea
    \gll taahïk-ka    uu  baar=\textbf{{duo}}?\\
    washbasin-\textsc{dat}  water  \textsc{ex}=\textsc{q}\\
    \glt ‘Is there water in the washbasin?’
    
    \ex
    \gll \textbf{{tugunan}} bar-ïa-ŋ-\textbf{{ïy}}?\\
    what.\textsc{inst}  go-\textsc{fut}-2\textsc{sg}-\textsc{q}\\
    \glt ‘What will you go with?’ (\citealt{LiYong-Sŏng2011}: 62, 90)
    
    \ex
    \gll ikki=\textbf{{duo}} üs=\textbf{{duo}} čahy?\\
    two=\textsc{q}    four=\textsc{q}    hour\\
    \glt ‘for two or four hours?’ \citep[83]{Stachowski1993}\z\z

\tabref{tab:turk:5} Summarizes \isi{question marking} in \ili{Turkic} languages in polar and \isi{content question}s. Except for \ili{Salar}, \ili{Tuvan}, \ili{Dukhan}, \ili{Yakut}, and \ili{Dolgan}, the \ili{Turkic} languages in the sample have unmarked content \isi{questions}. Apart from phonological differences, there is very little variation in \isi{polar question} markers. Only \ili{Yakut} and \ili{Dolgan} deviate from the rest of the languages, possibly due to \ili{Yukaghiric} influence. The content \isi{question marker} in these \ili{Turkic} languages has an areal connection to the so-called \ili{Mongolic} “corrogative” particle *\textit{büi} (\sectref{sec:5.8.2}). Similar to \ili{Mongolic}, the \ili{Turkic} markers may have their origin in a copula form. For \ili{Tuvan}, \citet[88]{AndersonHarrison1999} speculate that \textit{-Il} is a quasi-copula that derived from the demonstrative \textit{ol} ‘that’, which is also the source of the third person \isi{singular} pronoun (see also \citealt{Ragagnin2011}: 188). In fact, a development from demonstrative to copula is widely attested, for example for \ili{Chinese} \textit{shì} \zh{是} or \ili{Russian} \textit{eto}/\textcyrillic{это}. No \ili{Turkic} language from the sample marks polar and content \isi{questions} in the same way. Those languages for which sufficient data are available indicate that alternative \isi{questions} are usually marked in the same way as polar \isi{questions} but often exhibit an optional \isi{disjunction} as well. Altogether there is little variation in \isi{polar question} markers among \ili{Turkic} languages. Except for \ili{Yakut} and \ili{Dolgan} \textit{=duo} \textit{{\textasciitilde} =duu} and maybe \ili{Salar} \textit{=U}, all \isi{polar question} markers listed in \tabref{tab:turk:5} appear to be cognate with \ili{Turkish} \textit{=mI}. The major difference lies in how many variants the \isi{question marker} has in a given language. As has often been observed, a very likely origin of the \isi{question marker} lies in \isi{negation}, e.g. \ili{Turkish} \textit{-mA} or Old \ili{Turkic} \textit{-mA} \citep[151]{Erdal1998}. The polar \isi{question marker} was already present in Old \ili{Turkic} and in \ili{Chagatay} as \textit{=mU}, with content \isi{questions} unmarked (\citealt{Erdal1998}: 152; \citealt{BoeschotenVandamme1998}: 171, 175).

\begin{table}
\caption{Summary of question marking in Turkic}
\label{tab:turk:5}

\begin{tabularx}{\textwidth}{XXX}
\lsptoprule
& \textbf{PQ} & \textbf{CQ}\\
\midrule
\isit{Altai} & =MA\# & -\\
\ilit{Chagatay} & =mU & -\\
\ilit{Chalkan} & =MI\# & -\\
\ilit{Chulym} (Middle) & =bA\# & -\\
\ilit{Dolgan} & \textbf{=duo}\# & V-ij\\
\ilit{Dukhan} & =BA\# & V-(Ĭ)l\\
\ilit{Eynu} & =mu\# & -\\
\ilit{Fuyu} & =BA\# & -\\
\ilit{Ili Turki} & =MA\# & ?-\\
\ilit{Kazakh} & =MA\#, V-MI & -\\
\ilit{Khakas} & =BA\# & -\\
\ilit{Kyrgyz} & =BI\# & -\\
Old \ilit{Turkic} & =mU & -\\
\ilit{Salar} & =mU\#, \textbf{=U}\# & V-i\\
\ilit{Sarig Yughur} & =Mi\#, =mu\# {\textasciitilde} V\textbf{-m} & -\\
\ilit{Shor} & =MA\# & -\\
\ilit{Tatar} (\ilit{Chinese}) & =mI\# & -\\
\ilit{Tofa} & =be\# & V-(u)l\\
\ilit{Tuvan} & =be\# & V-Il\\
\ilit{Tuvan} (Dzungar) & =BA\# & V-l\\
\ilit{Uyghur} & =mu\# {\textasciitilde} V\textbf{-m} & -\\
\ilit{Uzbek} & =mi\# & -\\
\ilit{Yakut} & \textbf{=duo}\# \textasciitilde \textbf{=duu}\# & V-(n)ɪj\\
\lspbottomrule
\end{tabularx}
\end{table}

However, in \textbf{Chuvash}, the only extant \textbf{Oghur} language and the most aberrant \ili{Turkic} language, there are three question markers that are usually written attached to the preceding word, which takes an intonational peak.

\ea%64
    \label{ex:turk:64}
    \ili{Chuvash}\\
    \ea
    \gll văl  χula-na  kay-nă-\textbf{{i}}?\\
    3\textsc{sg}  city-\textsc{dat}  go-\textsc{pst}-\textsc{q}\\
    \glt ‘Did (s)he go to the city?’
    
    \ex
    \gll văl  χula-na  kay-nă-\textbf{{im}}?\\
    3\textsc{sg}  city-\textsc{dat}  go-\textsc{pst}-\textsc{q}\\
    \glt ‘Did (s)he really go to the city?’
    
    \ex
    \gll văl  χula-na  kay-nă-\textbf{{ši}}?\\
    3\textsc{sg}  city-\textsc{dat}  go-\textsc{pst}-\textsc{q}\\
    \glt ‘You mean (s)he went to the city?’ \citep[450]{Clark1998}\z\z

While \textit{-i} marks plain polar \isi{questions}, \textit{-im} is said to express \isi{uncertainty} or surprise, and sentences marked with \textit{-ši} express doubt and are in need for further confirmation \citep[450]{Clark1998}. Alternative \isi{questions} take two plain \isi{polar question} markers and an optional \isi{disjunction} \textit{e}. Content \isi{questions} are usually unmarked but may also take the marker \textit{-ši}. In example (\ref{ex:turk:65}a) an \isi{alternative question} follows a \isi{content question} (\sectref{sec:4.4}).

\ea%65
    \label{ex:turk:65}
    \ili{Chuvash}\\
    \ea
    \gll esir \textbf{{ăśta}} purăn-a-tăr,    xula-ra-\textbf{{i}} (\textbf{{e}}) yal-ta-\textbf{{i}}?\\
    2\textsc{sg.pol}  where  live-\textsc{prs}-2\textsc{sg.pol}  city-\textsc{loc}-\textsc{q}  or village-\textsc{loc}-\textsc{q}\\
    \glt ‘Where do you live, in the city or in the village?’
    
    \ex
    \gll muzey \textbf{{ăśta}}-\textbf{{ši}}?\\
    museum  where-\textsc{q}\\
    \glt ‘Where is the museum?’
    
    \ex
    \gll esir \textbf{{ăśta}} śural-nă?\\
    2\textsc{sg.pol}  where  be.born-\textsc{pf}\\
    \glt ‘Where have you been born?’ (\citealt{Landmann2014}: 34, 62)\z\z

These data show a pattern markedly different from those \ili{Turkic} languages located in \isi{NEA}. While \isi{double marking} and disjunctions are attested in other \ili{Turkic} languages as well, there is no formal match to \ili{Chuvash}. Furthermore, no other \ili{Turkic} language investigated here has a \isi{question marker} that can appear in both polar and content \isi{questions}. However, there was an additional dubitative marker (Old \ili{Turkic} \textit{erki}, \ili{Chagatay} \textit{\.e(r)ken} {\textasciitilde} \textit{\.e(r)kin}) that could also appear in \isi{content question}s.

Descriptions of \ili{Turkic} languages rarely explicitly state the syntactic behavior of interrogatives. Often, for instance in \ili{Uzbek} and \ili{Kazakh}, interrogatives are found in \isi{focus} position in front of the verb (\citealt{Boeschoten1998}: 373; \citealt{Muhamedowa2016}: 20).

\subsection{Interrogatives in Turkic}\label{sec:5.11.3}
\largerpage
\ili{Turkic} languages have both KIN- and K-interrogatives. The \ili{Proto-Turkic} \isi{interrogative} meaning ‘who’ has been reconstructed as *\textit{kem} by \citet[74]{Róna-Tas1998} or as *\textit{käm} for \ili{Oghur} (\ili{Chuvash} \textit{kam}), and *\textit{kim} otherwise by \cite[64, 69]{Schönig1999}. \ili{Oghuz} languages, but not \ili{Salar}, differ from other \ili{Turkic} languages in having derivations from the \isi{interrogative} \textit{ne} ‘what’ instead of *\textit{qay-} for locative forms (\citealt{Schönig1999}: 66). In fact, in modern \ili{Turkish} most interrogatives start with an \textit{n{\textasciitilde}} (\citealt{GökselKerslake2005}: 251). In \ili{Turkic}, however, the \isi{interrogative} meaning ‘what’ (\ili{Turkish} \textit{ne}) was originally the only native word starting with an \textit{n-} (e.g., \citealt{Róna-Tas1998}: 74). This phonotactic anomaly might indicate \isi{borrowing} from another language. Comparable forms in \isi{Northeast Asia} exist in \ili{Ainuic}, \ili{Eskaleut}, \ili{Japonic}, \ili{Sinitic}, and \ili{Yukaghiric} languages, but none is a very likely source of the \ili{Turkic} \isi{interrogative}. \citegen{Stachowski2015} claim that *\textit{ne} derives from a \ili{Uralic} demonstrative is not very convincing and deserves further evidence. The \ili{Uralic} language \ili{Selkup} has several interrogatives that have a \ili{Turkic} appearance and were probably borrowed (\sectref{sec:5.12.3}).

Many \ili{Turkic} languages have a difference between a plain velar plosive in ‘who’ and an uvular plosive in other interrogatives. A similar phenomenon is known from some \ili{Mongolic} languages (\sectref{sec:5.8.3}). However, the distribution of the resonances \textit{k{\textasciitilde}}, \textit{n{\textasciitilde}}, and \textit{q{\textasciitilde}} more closely resembles \ili{Yukaghiric} languages (\sectref{sec:5.14.3}). \tabref{tab:turk:6} gives an overview of cognates of five \ili{Turkic} interrogatives. In most languages the interrogatives ‘which’ and ‘where’ are analyzable synchronically and thus do not qualify as so-called \textit{basic-level} interrogatives. \ili{Siberian Turkic} languages contain an innovative form combining the meanings ‘how’ and ‘which’ that exhibits variation between \textit{-n-} (e.g., \ili{Tuvan} \textit{kan-dɨg}) and \textit{-y-} (e.g., \ili{Yakut} \textit{χay-}\textit{daχ}). This variation is already attested in Old \ili{Turkic} \textit{kañu} {\textasciitilde} \textit{kayu}. There is some overlap with another group of languages that exhibit an older derivation with a suffix \textit{-si} that is even attested in \ili{Chuvash} \textit{xă-š(ĕ)} and \ili{Khalaj} \textit{q\=ani(-si)}. Apart from the latter, these are usually based on the variant with \textit{-y-} (e.g., \ili{Tuvan} \textit{kay(ɨ)-zɨ}).

\begin{table}
\caption{Cognates of five Turkic interrogatives; Chaghatay taken from \cite[171, 173]{BoeschotenVandamme1998}, Chuvash from \cite[32f.]{Landmann2014}, Dukhan from \citet[94]{Ragagnin2011}, Khalaj from \cite[107f.]{Doerfer1988}, and South Siberian Turkic partly from \cite[410]{Schönig1998}; see the rest of this chapter for additional variants}
\label{tab:turk:6}
\small
\begin{tabularx}{\textwidth}{>{\raggedright}p{15mm}lQQp{10mm}Q}
\lsptoprule
& \textbf{who} & \textbf{what} & \textbf{which} & \textbf{when} & \textbf{where}\\
\midrule
\ilit{Turkish} & kim & ne & \textbf{hangi} & \mbox{\textbf{ne zaman}} & \textbf{nere}, \textbf{hani}\\
\ilit{Chuvash} & k\textbf{a}m & \textbf{mĕn} & xă-š(ĕ) & xăśan & \textbf{ăśta}\\
\ilit{Khalaj} & kim, küm & näy & q\=ani(-si) & qa´č\=an & \textbf{n\={\i}\textsuperscript{e}}\textbf{rä}\\
 
% \lspbottomrule
% \end{tabularx}
% \end{table}
% 
% \begin{table}
% \small
% \caption{Continued}
% \label{tab:turk:7}
% 
% \begin{tabularx}{\textwidth}{llQQlQ}
% \lsptoprule

Chaghatay & kim & ne & qay(u)(-sï) & qačan & \textbf{qanï}, qan-da, qay-da\\
Old \ilit{Turkic} & k\textbf{ä}m, kim & nä & kañu, kayu & kačan & kañu-da, kayu-da\\
\ilit{Salar} & k\textbf{a}m & nang & ga-si & gaqiang & ga-da\\
\ilit{Tatar} & kɨm & ni & qaj(-sə) & qajtʃan & qaj-da\\
\ilit{Kazakh} & kɨm & ne, nemene & qaj-sə & qaʃan & qaj-da\\
\ilit{Kyrgyz} & kim & \textbf{emne} & qaj-sə & qatʃan & qaj-da\\
\ilit{Uyghur} & kim & nemä & qay-si & qačan & \textbf{nä}\\
\ilit{Ili Turki} & kim & nemä & qay-sı & qačan & \\
Kashgar \ilit{Uyghur} & kim {\textasciitilde} tʃim & nimɛ & qa(j)-si & qatʃan & \textbf{nɛɛ}\\
\ilit{Eynu} & kim & nimɛ & qaj-si & qatʃan & \textbf{nɛ}\\
\ilit{Uzbek} & kim & nim\.{a} & qay-si & qačån & \textbf{qani}, \textbf{qay\.er-d\.{a}}, qayå-t\.{a}\\
\ilit{Tuvan} & kɨm & \textbf{čüü} & kay(ɨ)(-zɨ),

kan-\textbf{dɨg} & kažan & kay-da\\
Dzungar \ilit{Tuvan} & ɢəm & \textbf{dʒy-dʒi-mɛ} & ɢaj(ə)-sə,

{}ɢan-\textbf{dəɣ} & ɢaʒan & ɢaj-da\\
\ilit{Tofa} & q\textbf{u}m & \textbf{čü} & qaj-sy,

qan-\textbf{dyɣ} &  & qaj-da\\
\ilit{Karagas} (\ilit{Tofa}) & kèm, kum & \textbf{ŧü} & kan-\textbf{deg} & kaśan, kähän & kai-da\\
\ilit{Dukhan} & gïm & \textbf{ǰü(ü)} & gae &  & \\
\ilit{Khakas} & kem & nime & xay(-zɨ), xay-\textbf{daɣ} & xayǯan & xay-da\\
\ilit{Koibal} (\ilit{Khakas}) & kem, kim & nô, nêmä, nime & kai-ze,

kai-\textbf{dak} & ka{d̴}en & kai-da\\
\ilit{Fuyu} & g\u{\i}m & n\textsuperscript{y}em & ɢay-z\u{\i}, ɢa-\textbf{dah} & ɢajan & ɢay-da\\
\ilit{Shor} & kem & noo & kaj(y) & qačan & kaj-da\\
\mbox{\ilit{Sarig Yughur}} & k\textsuperscript{h}ïm & ni & qay-sï & qa\textsuperscript{h}ʈan & qay-ta\\
\isit{Altai} & kem & \textbf{d’u-γ} {\textasciitilde} \textbf{ču-γ}, ne & kan-\textbf{dyj} & kačan & kaj-da\\
\ilit{Chalkan} & kem & \textbf{t’u(u)}, \textbf{t’ü-γ} {\textasciitilde} \textbf{t’u-γ/g}, ne & qan-\textbf{dïy}, qan-\textbf{du(γ)} & qažan & qay-da\\
\ilit{Chulym} & kim & \textbf{tʃ\textsuperscript{i}}\textbf{o}, nöömä & qay-\textbf{dïɣ} & qačan {\textasciitilde} qaǰan & qay-da\\
\ilit{Yakut} & kim & \textbf{tuo-}\textbf{χ} & χaya, χay-\textbf{daχ} & χa\textbf{h}an & χa\textbf{nn}a\\
\ilit{Dolgan} & ki(i)m & \textbf{tuo-k {\textasciitilde} tuo-gu} & kaja, kaj-\textbf{da(a)k} & ka\textbf{h}an {\textasciitilde} ka\textbf{g}an & ka\textbf{nn}a\\
\lspbottomrule
\end{tabularx}
\end{table}

Stachowski (\citeyear{Stachowski1990,Stachowski2015}; p.c. 2016) has quite convincingly argued that \ili{Yakut} \textit{tuoχ} ‘what’, apart from \ili{Dolgan} \textit{tuok} {\textasciitilde} \textit{tuogu}, seems to have no cognates in other \ili{Turkic} languages. However, while there are certain phonological problems, there might actually be cognates in other \ili{Siberian Turkic} languages (\tabref{tab:turk:8}). Note, first of all, that there are additional forms that must be connected with \textit{tuoχ} in \ili{Yakut}. These are the forms meaning ‘why’ (\textit{toγo}) and ‘how many’ (\textit{töhö}). Despite their apparent differences, they share a \isi{resonance} in \textit{t{\textasciitilde}}, which suggests a common origin. Second, in both \ili{Yakut} and \ili{Dolgan}, as well as the other \ili{Siberian Turkic} languages, these forms are by and large restricted to these three functions. Third, apart from the question of whether there is a sound law that connects the \ili{Yakut} and \ili{Dolgan} forms with the interrogatives from the other languages, they certainly have a similar overall form. Fourth, the fact that all languages are from the same branch of \ili{Turkic} makes it very plausible to seek a common origin for this anomaly. Fifth, the forms meaning ‘what’ and ‘why’ in all languages have a different vowel quality than the form meaning ‘how many’.

\ili{Tuvan} \textit{čü-den} and \ili{Karagas} (\ili{Tofa}) \textit{ŧü-dän} ‘why’ contain an ablative instead of a dative (\citealt[28]{AndersonHarrison1999}; \citealt[163]{Castrén1857b}). The derivation of some forms such as \ili{Chalkan} \textit{t’üg(g)erek}, \textit{t’ugerek}, \textit{t’urïq} ‘what’ remain unclear to me. There is also an \isi{interrogative verb} ‘to do what’ in \ili{Tofa} (\textit{čoon-}) and \ili{Chalkan} (\textit{t’uvet-}).

\begin{table}
\caption{A tentative list of cognates of a possible interrogative stem in Siberian Turkic (except for Abakan); not all forms are shown}
\label{tab:turk:8}

\begin{tabularx}{\textwidth}{XlXl}
\lsptoprule
& \textbf{what} & \textbf{why} & \textbf{how many}\\
\midrule
\isit{Altai} & d’u-γ {\textasciitilde} ču-γ &  & \\
\ilit{Dukhan} & ǰü(ü) &  & \\
\ilit{Chalkan} & t’u(u), t’ü-γ {\textasciitilde} t’u-γ/g & & \\
\ilit{Chulym} & tʃ\textsuperscript{i}o &  & \\
\ilit{Karagas} (\ilit{Tofa}) & ŧü & ŧü-gä, ŧü-dän & ŧeśe, ŧehe\\
\ilit{Tofa} & čü &  & čehe\\
\ilit{Tuvan} & čüü & čü-ge, čü-den & čeže\\
\ilit{Tuvan}, Dzungar & dʒy-dʒimɛ & dʒy-ge, dʒy-nen & dʒeʒe\\
\midrule
\ilit{Dolgan} & tuo-k {\textasciitilde} tuo-gu & to-go {\textasciitilde} tuo-go & töhö\\
\ilit{Yakut} & tuo-χ & to-γo & töhö\\
\lspbottomrule
\end{tabularx}
\end{table}

\largerpage
\citet{Stachowski2015} tried to connect the \ili{Yakut} form with an \ili{Uralic} demonstrative stem, which is possible but unlikely from a typological perspective. Interrogatives and \isi{demonstratives} may share paradigmatic similarities and may also grammaticalize into similar categories such as relatives, but \isi{demonstratives} do not usually develop into interrogatives. That this change occurred during \isi{borrowing}, which in itself is not the most likely scenario, is not very plausible either. The only similar form in terms of both meaning and form that I was able to find in \isi{NEA} can be found in \ili{Iranian} languages (e.g., \ili{Sogdian} \textit{(ə)ču} ‘what’). However, a connection in terms of \isi{borrowing} seems too far-fetched. According to \citet[85]{Stachowski2015}, \textit{tuoχ} goes back to *\textit{to-ok}, in which the suffix is an intensifier. Possibly, the suffix can be compared with \isi{Altai} \textit{d’u-γ} {\textasciitilde} \textit{ču-γ} and \ili{Chalkan} \textit{t’ü-γ {\textasciitilde} t’u-γ/g}. The other \ili{Turkic} languages with the unusual \isi{interrogative} show a palatalized consonant instead of \textit{t{\textasciitilde}}. Most likely this is the result of the following high vowels. The reason for the apparent irregular development in \ili{Yakut} \textit{tuoχ} and \ili{Dolgan} \textit{tuok} is not perfectly clear, but one possibility would be an analogy to the negative existentials, \ili{Yakut} \textit{suoχ} and \ili{Dolgan} \textit{huok}, that have a regular development. Possibly, the \isi{question marker} \textit{=duu} {\textasciitilde} \textit{=duo} in \ili{Yakut} and \ili{Dolgan} derives from the same source (\textsc{int} > \textsc{q}), but this likewise remains somewhat speculative. More research by experts of these languages will be necessary to clarify these points.
 
 \largerpage
For \textbf{Old Turkic}, several partial \isi{interrogative} paradigms are attested. \tabref{tab:turk:9}. compares some of them with \ili{Sarig Yughur}, for which paradigms were given by \citet{Roos2000}. Apart from phonological changes there are only minor differences between the two languages, which illustrates the relatively young age of \ili{Turkic}.



\begin{table}[b]
\caption{Old Turkic interrogative paradigms \citep[211]{Erdal2004} in comparison with Sarig Yughur \citep[87]{Roos2000}}
\label{tab:turk:9}

\begin{tabularx}{\textwidth}{Xllll}
\lsptoprule
& \textbf{who} &  & \textbf{which} & \\
\midrule
& \textbf{Old Turkic} & \textbf{Sarig Yughur} & \textbf{Old Turkic} & \textbf{Sarig Yughur}\\
\textsc{nom} & \textit{kim}, \textbf{\textit{käm}} & \textit{k\textsuperscript{h}}\textit{ïm} & \textit{kayu}, \textit{ka}\textbf{\textit{ñ}}\textit{u} & \textit{qay-}\textbf{\textit{sï}}\\
\textsc{acc} & \textit{kim-ni} & \textit{k\textsuperscript{h}}\textit{ïm-nï} & \textit{kayu-nï} & \textit{qay-}\textbf{\textit{sï}}\textit{-n}\\
\textsc{gen} & \textit{kim-(n)iŋ} & \textit{k\textsuperscript{h}}\textit{ïm-nïŋ} & \textit{kayu-nuŋ} & -\\
\textsc{dat} & \textit{kim-kä}, \textbf{\textit{käm}}\textit{-kä} & \textit{k\textsuperscript{h}}\textit{ïm-ki} & \textit{kayu-ka} & \textit{qa-ɣa}\\
\textsc{abl} & - & \textit{k\textsuperscript{h}}\textit{ïm-tin} & \textit{kayu-dïn} & \textit{qay-tan}\\
\textsc{loc} & \textit{kim-}\textbf{\textit{tädä}} & - & \textit{kayu-da}, \textit{kañu-da} & \textit{qay-ta {\textasciitilde} qan-ta}\\
\lspbottomrule
\end{tabularx}
\end{table}

For reasons of space, paradigms will not be given in detail for other languages throughout this section.

\clearpage  
Let us now consider interrogatives from individual modern languages. The order will be roughly the same as in \sectref{sec:5.11.2}, starting with the only \textbf{Oghuz} language \textbf{Salar} (\tabref{tab:turk:10}). If available, several descriptions are contrasted for any given language. In some cases only a selection of forms is given.

\begin{table}
\caption{Salar interrogatives (\citealt{Ma1993}: passim; \citealt{LinLianyun1985}: 52, 109, 136, passim); some questionable variants were excluded}
\label{tab:turk:10}

\begin{tabularx}{\textwidth}{XXl}
\lsptoprule
& \textbf{Ma Quanlin et al.} & \textbf{Lin Lianyun}\\
\midrule
who & kam & kem\\
where & gada & ɢada\\
whither &  & ɢala\\
whence &  & ɢaden\\
when & gaqiang, gahao & ɢadʒaŋ, ɢahal\\
which & gasi & ɢajsi\\
for what & naima & \\
what, how etc. & naiqi & \\
how & naiqiu & \\
what & nang & naŋ\\
why & neigei & neɣe\\
how (long) & neisi & \\
how much/many & neisiqiu & nehdʒe\\
\lspbottomrule
\end{tabularx}
\end{table}

\noindent Similarly to \ili{Turkish} and \ili{Tatar}, relatively many of the interrogatives start with \textit{n{\textasciitilde}}. Only \textit{kam} {\textasciitilde} \textit{kem} ‘who’ has an initial \textit{k}, while all other forms start with \textit{g} {\textasciitilde} \textit{ɢ}. \citet[76]{LinLianyun1985} in addition mentions an \isi{interrogative verb} \textit{naxɢur} ‘to do what’ that is claimed to be a contraction of \textit{naŋ et-gur} with the definite future marker. With other verb endings the periphrastic construction is still present.

\ea%66
    \label{ex:turk:66}
    \ili{Salar}\\
    \gll sen \textbf{{naŋ}} \textbf{{et}}{-bər-}\textbf{{i}}?\\
    2\textsc{sg}  what  do-\textsc{progr}.\textsc{indef}-\textsc{q}\\
    \glt ‘What are you doing?’ (\citealt{LinLianyun1985}: 86)
    \z

Interrogatives from the \textbf{Kipchak} languages \ili{Tatar}, \ili{Kazakh}, and \ili{Kyrgyz} are listed in \tabref{tab:turk:11}. The languages all have a similar \isi{resonance} pattern with the \isi{interrogative} ‘who’ being the only one that does not show \textit{q{\textasciitilde}} or \textit{n{\textasciitilde}}. \ili{Tatar} is probably no exception, although in Cyrillic transcription the forms all start with \textcyrillic{к}{\textasciitilde}. This has been transliterated with \textit{q{\textasciitilde}} before an \textit{a}. Only \ili{Kyrgyz} \textit{emne} is an exception from the resonances. Most likely it is an \textit{allegro} form that developed from a form similar to \ili{Kazakh} \textit{nemene}. For comparative purposes, \tabref{tab:turk:11} also contains forms from \ili{Tatar} proper, transliterated from Cyrillic.\footnote{Some forms seem to be pronounced slightly differently, e.g. \textit{nindi}/\textcyrillic{нинди} was given as /nindey/.} \ili{Tatar} \textit{nɛrsɛ} derives from \textit{ni ersɛ} (\citealt{ChenZongzhenYiLiqian1986}: 80).

\begin{table}[p]
\caption{Interrogatives from Chinese Tatar (\citealt{ChenZongzhenYiLiqian1986}: 34, 79f., 185), Tatar (\citealt{Poppe1963}: 81f., 219, 234f. passim), Kazakh (\citealt{GengShiminLiZengxiang1985}: 53, 103, 172, 238), and Kyrgyz  as spoken in China (\citealt{HuZhenhua1986}: 58, 251)}
\label{tab:turk:11}

\fittable{
\begin{tabular}{lllll}
\lsptoprule
& \textbf{Ch. Tatar} & \textbf{Tatar} & \textbf{Kazakh} & \textbf{Kyrgyz}\\
\midrule
who & kɨm & kem & kɨm & kim\\
what & ni, nɛrsɛ & nii, närsä & ne, nemene & emne\\
why & nik, nigɛ & nik, nigä & nege, ne yʃɨn & emne ytʃyn\\
how many/much & nitʃɛ, qantʃa {\textasciitilde} qantʃɛ & ničä & neʃe(w),
qanʃa & netʃe(n), qantʃa\\
how & nitʃɨk & niček & qalaj {\textasciitilde} qandaj & qandaj\\
what (kind of) & nɨndɨj & nindi &  & \\
which & qaj, qajsə & qaj, qajsy & qajsə & qajsə\\
where (to) & qajda, qaj dʒer & qajda & qajda & qajda\\
whence & qajdan & qajdan & qajdan & qajdan\\
when & qajtʃan & qajčan & qaʃan & qatʃan\\
\lspbottomrule
\end{tabular}
}
\end{table}

\begin{table}[p]
\caption{Uyghur (\citealt{TuohutiLitifu2012}: 367; \citealt{Mi1997}: 83), and Uzbek interrogatives (\citealt{Boeschoten1998}: 373; \citealt{Landmann2010}: 24)}
\label{tab:turk:12}

\begin{tabularx}{\textwidth}{lQQQQ}
\lsptoprule
& \textbf{Uyghur} & \textbf{Kashgar Uyghur} & \textbf{\ili{Uzbek} (Boeschoten)} & \textbf{\ili{Uzbek} (Landmann)}\\
\midrule
who & kim & kim {\textasciitilde} tʃim & kim & kim\\
where & nä & nɛɛ & - & -\\
how much & näččä & nɛtʃtʃɛ & n\.eč\.{a} {\textasciitilde} n\.eč(\.{a})t\.{a} & necha, nechta (attr.)\\
what & nemä & nimɛ & nim\.{a} & ne, nima\\
why & nemišqa {\textasciitilde} nemiškä & nimiʃqa & n\.{a}g\.{a}, nim\.{a}g\.{a}, n\.{a}čük, nim\.{a} üčün & nega, nimaga, nima uchun\\
when & qačan & qatʃan & qačån & qachon\\
how & qandaq & qajdaq {\textasciitilde} qandaq & qal\.{a}y, qand\.{a}y, qanaqa & qani\\
how much/many & qančä & qantʃɛ & qanč\.{a} & qancha\\
where & qeni, qäyär & qeni, qɛjɛ(r) & qani, qay\.erd\.{a}, qayåt\.{a} & qayerda, qatta\\
which & qaysi & qa(j)si & qaysi & qay(si)\\
whither &  &  & qay\.erg\.{a}, qayåqqa & qayerga\\
whence & qäyärdin &  & qay\.erd\.{a}, qayåqd\.{a}n & qayerdan\\
\lspbottomrule
\end{tabularx}
\end{table}

\ili{Kazakh} \textit{ne yʃɨn} and \ili{Kyrgyz} \textit{emne ytʃyn}, like \ili{Uzbek} \textit{nim\.{a} üčün}, literally mean ‘what for’. Plural forms in \ili{Kazakh} are formed by \isi{reduplication}, e.g. \textit{kɨm kɨm} ‘who (\isi{plural})’ (\citealt{GengShiminLiZengxiang1985}: 54). This pattern that is also found in \ili{Uzbek}, for example, has parallels in the \isi{Amdo Sprachbund}.

\ili{Uyghur} \textit{nä} ‘where’, or its dialectal Kashgar form \textit{nɛɛ}, is an innovation also found in \ili{Eynu} \textit{nɛ} that might be connected to \ili{Turkish} \textit{nere} and \ili{Khalaj} \textit{n\={\i}\textsuperscript{e}}\textit{rä} ‘where’. \ili{Uzbek} \textit{n\.{a}g\.{a}} ‘why’ has cognates in \ili{Tatar} \textit{nigä} and \ili{Kazakh} \textit{nege} and in some Siberian languages such as \ili{Khakas} \textit{noɣa} or \ili{Fuyu} \textit{noʁo} and is an old dative form. The \ili{Uzbek} \isi{interrogative} \textit{nim\.{a}g\.{a}} ‘why’ has the same basis but is more readily analyzable as the form \textit{nim\.{a}} ‘what’ still exists. The dative can also be found in \textit{qay\.erg\.{a}} ‘whither’. Both \ili{Eynu} and \ili{Ili Turki} forms are almost completely identical to \ili{Uyghur} (\tabref{tab:turk:13}).


Interrogatives from the \textbf{Sayan} subbranch of Southern \ili{Siberian Turkic} languages have been collected in \tabref{tab:turk:14}. \citet[94]{Ragagnin2011} only mentions the three \ili{Dukhan} interrogatives \textit{gïm} ‘who’, \textit{ǰü(ü)} ‘what’, and \textit{gae} ‘which’.

\textbf{Abakan} is the only subbranch of \ili{Siberian Turkic} that lacks the special \isi{interrogative} that might be cognate with \ili{Yakut} \textit{tuoχ}. \tabref{tab:turk:15} summarizes all forms available for \ili{Khakas}, \ili{Fuyu}, as well as \ili{Shor} and compares them with \ili{Sarig Yughur}. \ili{Sarig Yughur} interrogatives are rather different from other Abakan languages. Altogether there are more forms starting with an \textit{n{\textasciitilde}}.



\tabref{tab:turk:16} presents data from \ili{Chulym} and \ili{Altai Turkic} languages. \isi{Altai} has a dialectal difference between southern \textit{ne} ‘what’ and northern \textit{d’uγ} {\textasciitilde} \textit{čuγ} ‘what’ \citep[15]{Baskakov1958a}. For \ili{Chulym}, \cite{AndersonHarrison2006} and \cite{Harrison2003} have the form \textit{tʃ\textsuperscript{i}}\textit{o} for Middle \ili{Chulym}, but \citet[493]{Birjukovich1997} mentions \textit{nömä} instead, which was given as \textit{nöömä} by \cite{LiYong-Sŏng2008}. \ili{Chalkan} furthermore has a verb \textit{t’uvet-} ‘to do what’ and the \ili{Russian} \isi{interrogative} \textit{qaqoy} {\textasciitilde} \textit{kakoy} ‘what kind of’.


Northern \ili{Siberian Turkic} interrogatives (\tabref{tab:turk:17}) have two resonances, \textit{t{\textasciitilde}} and \textit{k{\textasciitilde}}. The latter changed to \textit{χ{\textasciitilde}} in \ili{Yakut}. There is no \isi{resonance} in \textit{n{\textasciitilde}}, which stands in stark contrast even with several Southern \ili{Siberian Turkic} languages. \ili{Yakut} \textit{χanna} and \ili{Dolgan} \textit{kanna} ‘where’ are amalgamated forms that go back to a locative form with an \textit{-n-} instead of a \textit{-y-}, cf. \ili{Sarig Yughur} \textit{qa}\textbf{\textit{y}}\textit{-ta {\textasciitilde} qa}\textbf{\textit{n}}\textit{-ta}. The \isi{similarity} to \ili{Mongolic} languages such as \ili{Khamnigan Mongol} \textit{kaana} or \ili{Buryat} \textit{xaana} ‘where’ is thus due to chance. \ili{Yakut} \textit{χas} and \ili{Dolgan} \textit{kas} ‘how much’ seem to have a cognate in \ili{Tuvan} \textit{qaš}.



\begin{table}[p]
\caption{Ili Turki (\citealt{Hahn1991}: passim) and Eynu interrogatives (\citealt{Lee-Smith1996a}: 857; \citealt[79f., 316, 338]{ZhaoAximu2011}).}
\label{tab:turk:13}

\begin{tabularx}{\textwidth}{XXXl}
\lsptoprule
& \textbf{Ili Turki} & \textbf{\ili{Eynu} (LS)} & \textbf{\ili{Eynu} (ZA)}\\
\midrule
who & kim & kim & ki\\
how many & näččä &  & nɛtʃtʃɛ\\
what & nemä & nimɛ & nemɛ, qaj\\
where &  & nɛ & nɛ\\
when & qačan & qatʃan & qatʃan\\
how much/many & qanča &  & qantʃɛ\\
how & qandaq & qandaq & qandaq\\
which & qaysı & qajsi & qajsi\\
\lspbottomrule
\end{tabularx}
\end{table}

\begin{table}[p] 
\small
\caption{Russian Tuvan (\citealt{AndersonHarrison1999}), Dzungar Tuvan (\citealt{WuHongwei1999}: 42, 231), Tofa \citep[381]{Rassadin1997}, and Karagas (Tofa) interrogatives (\citealt{Castrén1857b}: 23, 163ff.); according to \cite[410]{Schönig1998}, the Tuvan form for ‘who’ is \textit{qïm}; some variants were excluded}
\label{tab:turk:14}

\begin{tabularx}{\textwidth}{XQlQQ}
\lsptoprule
& \textbf{Tuvan} & \textbf{Dzungar Tuvan} & \textbf{Tofa} & \textbf{Karagas}\\
\midrule
who & kɨm & ɢəm & kum & kèm, kum\\
what & čüü & dʒy-dʒimɛ & čü & ŧü\\
to do what &  &  & čoon- & \\
why & čü-ge, čü-den & dʒy-ge, dʒy-nen & qančža ‘how’ & ŧü-gä, ŧü-dän, ŧüneŋ uśun\\
how many & čeže & dʒeʒe & čehe, čü hure, qaš & ŧeśe, ŧehe\\
how, which & kandɨg & ɢandəɣ & qandyɣ & kandeg\\
which & kayɨ {\textasciitilde} kay(ɨ)zɨ & ɢaj(ə)sə & qajsy & kaja\\
where & kayda & ɢajda & qajda & kaida\\
whither & kayaa {\textasciitilde} kaynaar &  &  & kainar\\
whence & kayɨɨn {\textasciitilde} kayɨɨrtan & ɢajlap & qajdan & kajen\\
when & kažan & ɢaʒan &  & kaśan, kähän\\
\lspbottomrule
\end{tabularx}
\end{table}

\begin{table}
\small
\caption{Khakas \citep[21]{Anderson1998}, Koibal (Khakas) (\citealt{Castrén1857b}: 23, 163ff.), Fuyu (\citealt{HuImart1987}: 31), Shor \citep[505]{Donidze1997}, and Sarig Yughur interrogatives (\citealt{Roos2000}: 87, modified transcription); Shor forms in brackets are from \citet[294]{Nevskaja2000}}
\label{tab:turk:15}

\begin{tabularx}{\textwidth}{QlQQQQ}
\lsptoprule
& \textbf{Khakas} & \textbf{Koibal} & \textbf{Fuyu} & \textbf{Shor} & \textbf{Sarig}\\
\midrule
who & kem & kem, kim & g\u{\i}m & kem & k\textsuperscript{h}ïm\\
what & nime & nô, nêmä, nime & n\textsuperscript{y}em & noo & ni\\
why & noɣa & nôdaŋ,

nô kerektäŋ, nimedäŋ, nôderga & noʁo {\textasciitilde} noo &  & naɣʊ, \textbf{nati}\\
when & xayǯan & ka{d̴}en & ɢajan & [qačan] & qa\textsuperscript{h}ʈan\\
how & xaydi & kaidi {\textasciitilde} kai{d̴}i &  &  & \textbf{niyor}\\
how much/many & ninǯe, xanǯa & nem{d̴}e & ninji &  & niɕi, \textbf{niɕor}\\
where & xayda & kaida & ɢayda & kajda, [qayde] & q\textsuperscript{h}an\\
which, how & xay, xayzɨ & kaize, kaizeder & ɢayz\u{\i} & kaj(y) & qaysï\\
whither & xayɣa, xaydar & kaidar &  & [qayaγa] & qay-ta {\textasciitilde} qan-ta\\
whence & xaydaŋ & kaidaŋ &  &  & qay-tan\\
what kind of & xaydaɣ & kaidak & ɢadah {\textasciitilde} ɢad\u{\i}h &  & \textbf{niɕik}\\
\lspbottomrule
\end{tabularx}
\end{table}

  \begin{table}
\caption{Middle Chulym (\citealt{LiYong-Sŏng2008}: 44; \citealt{AndersonHarrison2006}; \citealt{Harrison2003}), Altai \citep[183]{Baskakov1997}, and Chalkan interrogatives (\citealt{Erdal2013}: passim). Not all variants listed.}
\label{tab:turk:16}

\begin{tabularx}{\textwidth}{llllQ}
\lsptoprule
& \textbf{\ili{Chulym} (Li)} & \textbf{Chulym} & \textbf{Altai} & \textbf{Chalkan}\\
\midrule
who & kim &  & kem & kem\\
what & nöömä & tʃ\textsuperscript{i}o & ne & t’u(u), t’ü-γ {\textasciitilde} t’u-γ/g, ne\\
what kind of, how &  &  & kažy & qaydat, qayde, qaydeet, qayt(a)\\
which, how & qaydïɣ & qajdɯɣ & kandyj & qandïy, qandu(γ), qanduu\\
why & qaya & qaja &  & qay ‘how’\\
when & qačan {\textasciitilde} qaǰan & ?qajɣa & kačan & qažan\\
where & qayda &  & kajda & qaya, qayda\\
whither & qaynar & kajnaar &  & qana(a), qayda\\
whence & qaydïn & kajdɨn &  & \\
how many/much & köpä &  & kanča & qanža, qanži, qanča, qančï\\
\lspbottomrule
\end{tabularx}
\end{table}



\clearpage %solid chapter boundary
\begin{table}[t]
\caption{Selected Yakut and Dolgan interrogatives (\citealt{Stachowski1998}: 423; \citealt{Stachowski1993}: passim)}
\label{tab:turk:17}

\begin{tabularx}{\textwidth}{XXl}
\lsptoprule
& \textbf{Yakut} & \textbf{Dolgan}\\
\midrule
who & kim & ki(i)m\\
what & tuoχ & tuok {\textasciitilde} tuogu\\
why (-\textsc{dat}) & toγo & togo {\textasciitilde} tuogo\\
how much & töhö & töhö\\
when (-\textsc{dat}) & χahan & kahan {\textasciitilde} kagan\\
how & χaydaχ & kajda(a)k {\textasciitilde} kajtak\\
how much & χas & kas\\
which & χaya & kaja\\
where, whither & χanna & kanna\\
whence & χantan & kantan\\
along where &  & kanan\\
how much & χahya & kahya(n)\\
what kind of & χannϊk & kannyk\\
\lspbottomrule
\end{tabularx}
\end{table}

  