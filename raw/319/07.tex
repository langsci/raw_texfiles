\chapter{Post-editing risks and data security -- which pitfalls can arise?}\label{sec:7}



    \objectives{
        You will learn...
        \begin{itemize}
            \item what risks can arise in translation and post-editing,
            \item what to keep in mind concerning data security when using MT systems.
        \end{itemize}
        }

\vspace{\baselineskip}


When we talk about PE, we also have to think about possible risks and security concerns. In this chapter, we want to outline the most important considerations, so you know what you have to keep in mind when you start working as a professional post-editor.

\section{Post-editing risks assessment}\label{sec:7:1}

Translating texts generates risks for all actors involved in the translation process (\citealt{canfora2015risikomanagement} or \citealt{canfora2018ostriches}). Although translation contains specific creative and cognitive aspects that alone can be the research focus of many scientific studies \citep{pym2018risk}, decisions made during the entire translation process are underpinned by the same principles as the decisions on any other business level. Therefore, these decisions should be made in an economic framework. One instrument to develop decision criteria in economics is risk management. Generally, business decision criteria can be differentiated between strategic (long-term), tactical (medium-term), and operative (short-term) decisions \citep{hofmann2012prozessgestutztes}. When considering risk management for a PE situation, the following business decisions are of special importance: 

\begin{itemize}
    \item strategic, e.g. if the organisation wants to use MT at all 
    \item operative, e.g. what PE guidelines – full vs. light – are necessary for the specific text or the respective text type in regard to the organisation’s general strategic decisions
\end{itemize}

The international standard ISO \citet{iso2009international} “Risk management – Principles and guidelines” can be used for the translation process in all contexts, because it is a horizontal standard. Risk management is considered an integral part of all processes in an organisation including translation processes (either in-house or as part of a supply chain risk management). In addition to the risks that emerge from translation in general, the use of MT and PE generates risk factors in particular, such as:

\begin{itemize}
    \item data breach: Confidential information are fed into a web-based MT system and end up on the web (as in the case of Statoil, cf. \href{https://www.csoonline.com/article/3236348/data-breached-in-translation.html}{CSO Online}, last accessed 20 August 2021).
    \item loss of control of processes: The clients cannot control whether the translator uses MT or the functionality of the MT system is not transparent to the user of the MT system at all, especially with neural MT.
    \item uncertain liability modalities: In cases of translation errors or problems, the responsibilities concerning liability are not clearly defined. This especially affects the use of MT and PE for high risk texts. In cases of claims for compensation where translation mistakes cause danger to life and limb, the client might partially be blamed.
    \item attitude towards MT: Clients might have difficulties in finding qualified translators and post-editors, because professional translators might still have prejudices against MT and PE (e.g. \citealt{cadwell2018resistance} ; \citealt{guerberof2013professional}; \citealt{laubli2017google}).
    \item quality issues: The quality of the post-edited text might not be sufficient for the purposes of the client or target group.
\end{itemize}
	
Basically, the client should consider whether the benefits outweigh the risks before using MT and PE. Or, in other words, the client has to decide whether the risks are tolerable in a given situation. This includes general considerations arising from the client’s own “risk management policy” (cf. \href{https://www.theirm.org/media/4709/arms_2002_irm.pdf}{IRM 2002}\footnote{last accessed 20 April 2021}). In the context of ISO 31000, the term risk management policy describes a “statement of the overall intentions and direction of an organization related to risk management” (\href{https://www.iso.org/standard/44651.html}{ISO Guide 73:2009}\footnote{last accessed 20 April 2021}). The general risk attitude of the organisation is an important factor in creating the risk management policy, which describes the “organization’s approach to assess and eventually pursue, retain, take or turn away from risk” (\href{https://www.iso.org/standard/44651.html}{ISO Guide 73}, term 3.7.1.1). Accordingly, an organisation can be more or less willing to take risks, and this so called “risk appetite” influences strategic decisions. An organisation with a higher appetite for risks is more willing to take the risks mentioned above than a risk adverse organisation. These decisions are usually made on a long-term basis and therefore usually concern the strategic part of business management (cf. \href{https://www.theirm.org/media/4709/arms_2002_irm.pdf}{IRM 2002}).

On the operative level, risk management can provide decision criteria for or against the use of MT and PE for certain text types. Therefore, the approach to risk management for translations can also be used for decision processes in MT and PE \citep{canfora2018ostriches}. This means that the potential risks must be identified before the actual translation process to foresee problems that might affect different actors involved in the translation process such as the translator, the TSP (Translation Service Provider), the client, the end user or any other agent. This initial analysis should consider the negative consequences of failures in the translation, such as impaired communication, loss of reputation, property damage, lawsuits or other legal consequences, injuries, which could even amount to danger to life and limb, etc. Afterwards, the likelihood that these risks could occur in each case and the priorities of the client regarding the translation risks need to be analysed in compliance with the strategic risk management policy. This means that the client or the project manager has to decide which translation risks must be avoided and which can be tolerated. Therefore, it is sensible to create different categories (e.g. very high-risk, high-risk, and low risk documents) and to categorise the source text documents according to the risk analysis and risk evaluation \citep{canfora2018ostriches}. In line with these categories, different processes can be shaped for the use of MT and PE. Hence, low-risk texts, for example, could be machine-translated with subsequent light PE or even without PE. High-risk texts require full PE so that a balance is created between risk considerations and the advantages of MT and PE. For very high-risk texts, the client has to evaluate whether a combination of MT, full PE, and additional quality control measures like revision ensure the necessary quality. If this is not the case, MT might have to be entirely disregarded for those text types because the risks are too high. Furthermore, it has to be assessed whether it is still more efficient to combine MT, full PE and additional quality measures. Maybe a translation workflow with only human translation would provide more security and higher productivity and reduce costs in the end.

For more details on risks in PE and sustainable workflows for NMT read \citet{canfora2020risks}, who isolate three main risks for NMT: possible damages to clients and customers, liability issues, and cyber risks.

\section{Post-editing and data security}\label{sec:7:2}

Data security is a very important issue when using MT, because not all systems protect your texts and data. If an in-house MT system is used, security considerations are less problematic because the texts that are fed into the system are safely stored on an internal system or server. Still, it might be reasonable to assess who can access the server and the MT system. Further, users of the MT system should be informed about confidentiality issues, especially when working with externals and freelancers. The same holds true for cloud-based systems, which typically use secure encoding. However, if an external system without secure encoding and/or a free online system is chosen, the source text is often saved on the provider’s server and hence might become accessible to third parties. This can be unproblematic, e.g. if we are dealing with the translation of a website and this text will be publicly available anyway. However, if the data are sensitive, MT systems that do not provide a safe environment must be avoided (cf. e.g. \citealt{kamocki2015all}).

The German company DeepL, which provides MT systems, clearly differentiates between the free and the paid versions of their service. Regarding free use, they state
\begin{quote}
    When you use our translation service, only enter texts that you are willing to transfer to our server. Transferring the texts is necessary to offer our service and conduct the translation. We process your texts and their translations for a limited amount of time to train and improve our neural networks and translation algorithms.
    
    If you edit our translation suggestions, these edits will also be transferred to our servers to check the edit for correctness and possibly update the translated text according to your corrections. We also save your edits for a limited amount of time to train and improve our translation algorithm.
    
    Please note that you must not use our translation services for texts containing any kind of personal data.
    \footnote{\url{https://www.deepl.com/privacy}, last accessed 20 April 2021

    Original text:``Wenn Sie unseren Übersetzungsservice nutzen, geben Sie nur Texte ein, die Sie auf unsere Server übertragen wollen. Die Übermittlung dieser Texte ist notwendig, damit wir die Übersetzung durchführen und Ihnen unseren Service anbieten können. Wir verarbeiten Ihre Texte und die Übersetzung für einen begrenzten Zeitraum, um unsere neuronalen Netze und Übersetzungsalgorithmen zu trainieren und zu verbessern.

    Wenn Sie Korrekturen an unseren Übersetzungsvorschlägen vornehmen, werden diese Korrekturen auch an unseren Server weitergeleitet, um die Korrekturen auf Richtigkeit zu überprüfen und gegebenenfalls den übersetzten Text entsprechend Ihren Änderungen zu aktualisieren. Wir speichern auch Ihre Korrekturen für einen begrenzten Zeitraum, um unseren Übersetzungsalgorithmus zu trainieren und zu verbessern.

    Bitte beachten Sie, dass Sie unseren Übersetzungsservice nicht für Texte mit personenbezogenen Daten jeglicher Art nutzen dürfen."}
\end{quote}

When it comes to the paid cloud version, DeepL has a much securer policy
\begin{quote}
    When you use DeepL Pro, your submitted texts or documents will not be stored permanently, but only as long as it takes to create and transmit the translation. After transmitting the translation to you, both the texts or documents you submitted as well as their translations will be deleted. When you use DeepL Pro, we do not use your texts to improve the quality of our service. [...]
    
    Please note that you can use DeepL Pro only for texts containing any kind of personal data if you have a job processing agreement with us [...].\footnote{\url{https://www.deepl.com/privacy}, last accessed 20 April 2021

    Original text: ``Bei der Verwendung von DeepL Pro werden die von Ihnen eingereichten Texte oder Dokumente nicht dauerhaft gespeichert und nur vorübergehend vorgehalten, soweit dies für die Erstellung und Übertragung der Übersetzung notwendig ist. Nach der Übertragung der Übersetzung an Sie werden sowohl die eingereichten Texte oder Dokumente als auch deren Übersetzungen gelöscht. Bei der Verwendung von DeepL Pro verwenden wir Ihre Texte nicht, um die Qualität unserer Dienstleistungen zu verbessern. [...]

    Bitte beachten Sie, dass Sie DeepL Pro grundsätzlich nur für Texte nutzen dürfen, die personenbezogene Daten jeglicher Art enthalten, wenn Sie mit uns eine Auftragsverarbeitungsvereinbarung abgeschlossen haben [...]."}
\end{quote}

This also means that you should never machine translate the text you get from your clients without their permission, especially if you want to use an online MT system as the data will probably be stored by the MT system. Or as \citet[15]{kamocki2015all} summarise for general use
\begin{quote}
    Private users should consider translating only those bits of texts that do not contain any information relating to third parties (which in practice may limit them to translating text into, and not from, a different language to their own). Businesses in particular may find such a limitation rather constricting and to protect their own data and the data of their clients, may prefer to opt for a payable offline MT tool instead of a ‘free’ online service.
\end{quote}

As is the case in many other AI and computational linguistic features, using MT has become so common, especially as it is often implemented on webpages that we might become a little careless. So, we can only advise you to think carefully before you decide to use MT systems, especially in a professional context. 

\newpage

\section*{Crossword puzzle -- chapter 7}

\begin{Puzzle}{10}{12}
|{}	|{}	|{}	|{}	|{}	|{}	|{}	|{}	|[3]C	|.
|{}	|{}	|{}	|{}	|{}	|{}	|{}	|{}	|O	|.
|{}	|{}	|{}	|{}	|{}	|{}	|[4]L	|{}	|N	|.
|[5]R	|I	|S	|K	|[2]S	|{}	|I	|{}	|F	|.
|{}	|{}	|{}	|{}	|T	|{}	|A	|{}	|I	|.
|{}	|{}	|{}	|{}	|R	|{}	|B	|{}	|D	|.
|[1]O	|P	|E	|R	|A	|T	|I	|V	|E	|.
|{}	|{}	|{}	|{}	|T	|{}	|L	|{}	|N	|.
|{}	|{}	|{}	|{}	|E	|{}	|I	|{}	|T	|.
|{}	|{}	|{}	|{}	|G	|{}	|T	|{}	|I	|.
|{}	|{}	|{}	|{}	|I	|{}	|Y	|{}	|A	|.
|{}	|{}	|{}	|{}	|C	|{}	|{}	|{}	|L	|.
\end{Puzzle}

\begin{PuzzleClues}{\textbf{Across}}
\Clue{1}{OPERATIVE}{On what business decision level must the kind of PE guidelines be decided?}
\Clue{5}{RISKS}{Before a document is post-edited, the potential ... have to be analysed and evaluated.}
\end{PuzzleClues}

\begin{PuzzleClues}{\textbf{Down}}
\Clue{2}{STRATEGIC}{On what business decision level must organisations decide whether or not to use MT?}
\Clue{3}{CONFIDENTIAL}{What kind of information should not be fed into a web-based MT system?}
\Clue{4}{LIABILITY}{The responsibilities concerning... are not clearly defined for PE, yet.}
\end{PuzzleClues}
