\chapter{Post-editing and tools -- how do they interact?}\label{sec:6}


    \objectives{
        You will learn...
        \begin{itemize}
            \item what a translation memory system is (if you don't already know), 
            \item how to post-edit in translation memory systems,
            \item what adaptive and interactive MT is.
        \end{itemize}
        }

\vspace{\baselineskip}

This chapter will present how MT output can be integrated in professional translation practice, especially concerning the integration in translation memory systems. Translation memory (TM) systems are essential tools in professional translation workflows, often including translation memory, project management, and terminology management components. First, we will give a short introduction to TM systems \sectref{sec:6:1}. However, as most of you will already have worked with TM systems, this introduction will be very basic and you might consider skipping it. In \sectref{sec:6:2}, we will discuss working in a translation memory system with MT output. And finally, in \sectref{sec:6:3}, we will talk about the latest technological developments for using MT for PE processes. As the field is constantly evolving, new tools and functionalities are being developed to increase productivity and user-friendliness. Hence, we will discuss some examples, e.g. adaptive and interactive MT.

\section{Introduction to translation memory systems}\label{sec:6:1}

This will be a short introduction to translation memories and their basic functionality. Most TM system vendors provide online tutorials nowadays if you want to learn about a new tool. Further, most TM systems work on the same principles, which usually makes it quite easy to use a new tool if you are already familiar with another one.

So, what is a translation memory system? Basically, it saves the translations of texts, i.e., it is a database of previous translations. Usually, the source texts are segmented on a sentence basis. You can translate the text segment by segment and every segment is saved. Further, every source segment is compared to the segments that have already been translated. If the same (100\% or full matches) or a similar (fuzzy matches) segment appears, the previous translation is presented to you and you can decide whether you want to use the same translation, how much you need to edit, or whether you want to translate the segment from scratch. Translation memory systems can often be combined with terminology management systems, dictionaries, MT systems, and/or other helpful tools.

There are different ways to build a translation memory. Of course, you can build a TM with your own translations – either in one big database or you can save your translations thematically according to text type, domain, client, etc. You can import existing databases, e.g. clients often deliver a translation memory containing the translations of other translators. Finally, you can align existing translations and add them to your translation memory. The parallel texts should be available electronically. You can upload the source text and the target text, which are then automatically segmented on a sentence level and aligned. However, the resulting alignments are sometimes error-prone and need some manual corrections, e.g. if one source sentence is translated with two target sentences.

There are many reasons why translators should use translation memory systems. The translations are stored efficiently and it is much easier to recall what has been translated before. In other words, the work that has been done before can be reused easily. Accordingly, the matching process is efficient, too. We not only retrieve what appeared in exactly the same phrasing, but also what is similar to the current sentence or phrase. To put it in a nutshell, redundant work is reduced, which saves time and makes the translation process more efficient. Due to special search functions, linguistic features can be used more consistently, which increases the quality in most text domains. It is unlikely that you will skip a segment during the translation as the texts are preprocessed. Further, the systems already extract the texts that need translation. Hence, it is unlikely to damage the file, e.g. in HTML or XML files. 

Working with translation memory systems can also have some disadvantages. It might not be advisable for every text type to have a consistent, repetitive style, especially if the texts are very creative. Another disadvantage is that you probably only read the text segment by segment. If you have a lot of exact and fuzzy matches, you might not understand the context of the individual sentences correctly and minor or major errors can occur. All in all, the advantages usually outweigh the disadvantages, especially for domain-specific text types.

There are many translation memory suppliers and their systems might have different pros and cons, but their basic functionality is the same. At this point, we do not want to recommend any system in particular. For most systems, you have to buy a license or the software, but there are also some free tools, which usually do not have many advanced functionalities. Nonetheless, they do the job. You might also find a free test version or free test days for some systems. You will probably have to decide which is the best tool for you in the long run -- or your clients, who might use a particular system themselves. Hence, it often makes sense to be flexible and be able to use different systems.

\section{Machine translation in translation memory systems}\label{sec:6:2}

This chapter will present how MT is integrated and presented in different tools. Of course, every system is (at least) a little different, but usually the concepts are the same or similar across tools. Thus, you can transfer what you learn here to the tools you use in your everyday life or will use in future.

First, you have to know how MT is integrated or can be activated and deactivated in the respective translation memory system. The MT can be activated right away -- in this scenario you should check what kind of MT system is activated and if it is suitable for your project (concerning quality, data security, etc.) -- or you have to activate the MT component manually. The systems often offer different MT implementations in the standard settings. However, it is usually also possible to download or purchase other MT systems for the respective TM system. 

After you have selected an MT system for your project, the MT output will often be automatically inserted into empty segments, i.e. segments without full or fuzzy matches from the TM storage. For the latter, you can even define the threshold at which MT suggestions might replace fuzzy matches. However, the MT suggestions might also be presented in addition to the different TM matches. You can, accordingly, use the MT output as an additional option, insert it into the segment (which often happens automatically), and post-edit it. When the post-edited segment is confirmed, it is added to the translation memory.

So, the basic process is very similar to what we already know from translation memory use in translation from scratch. Some tools also provide additional functionalities to measure, for example, PE effort. The plug-in ``Qualitivity" for Trados Studio helps the user to 
\begin{quote}
track the time spent on translating, reviewing \& post-editing the segments from documents, but additionally [it] includes functionality to track every single change made to the segments at a granular level and a means to generate reports based on that data in structured and readable format. (\href{https://community.sdl.com/product-groups/translationproductivity/w/customer-experience/2251/qualitivity}{community.sdl.com}, last accessed 31 May 2021)
\end{quote}

In the next section, we will talk about new and more advanced approaches to integrating MT.

\section{New approaches}\label{sec:6:3}
 
Integrating MT output into TM environments has not been the only advancement in recent years. To make the PE task less repetitive, approaches towards \textit{interactive} and \textit{adaptive} MT have been developed. 
\begin{quote}
    An interactive system tries to autocomplete the text the user is going to type; it either predicts the text the user is going to type or changes the MT suggestion on the basis of what is typed, whereas an adaptive system is an MT system that learns from corrections on the fly and is continuously trained. \citep[118]{daems2019interactive}
\end{quote} 
In other words, interactive MT systems change their suggestions while the segment is post-edited, whereas an adaptive system learns in the background and adapts to the post-editors changes.

\citet{daems2019interactive} investigated how these different modes influence the PE process. They used the commercial tool LILT \footnote{\href{https://lilt.com/}{Lilt Website}, last accessed 8 March 2021} which integrates both interactive and adaptive MT, but also presents TM matches. Compared to a traditional TM system, the use of MT output is a strong focus in the LILT environment. The study was conducted in two rounds - the first using statistical MT, the second using neural MT. Eight professional translators (four per round) participated in the experiment. They worked as Dutch-English translators, which was also the language pair investigated in the study. 
The study found that there was hardly a difference between SMT and NMT concerning post-editing time and effort (measured via keystrokes and mouse clicks), although the initial SMT output produced more errors. Deams and Macken argue that this might be caused by the interactivity and adaptivity of the MT systems, the kind of errors produced by the different systems, and individual behaviour of the post-editors. Further, they studied the whole translation process, which also included fuzzy and full matches from the TM component. 

The CasMaCat project (\citealt{alabau2014casmacat}) provided a TM environment specialised on PE processes with an integrated interactive SMT system. \citet{sanchis2014interactive} tested how the interactivity influenced the PE process. They asked nine freelance translators to full post-edit nine newspaper texts from English into Spanish (the latter was their native language). PE was done in three modes, three texts were conventionally post-edited, i.e. without interactivity, and interactive systems (basic vs. advanced) were provided for the other six texts. All participants were introduced to the CasMaCat environment and the different modes before the experiment. Finally, four reviewers were asked to revise one dataset (consisting of one text from each mode by all participants). The results show that the participants did not become faster in the interactive modes, but in the basic interactive systems, it took fewer keystrokes to post-edit the texts and the post-editors were only a little slower. The quality of the final output also did not significantly change.
After the sessions, the users were asked to give feedback on how satisfied they were with their PE results, how much they liked the tools, whether they would rather have translated from scratch, and whether they would have preferred to work without interactive MT. The results are very mixed and do not present a clear picture.
Taking into perspective that CasMaCat used SMT systems, and the results might be different for modern NMT systems (the results of \citealt{peris2019online} underline this hypothesis), we can take from this study that interactivity might not be the perfect solution for every post-editor and that it might take some time until you are used to the interactivity and can use it properly. This disadvantage does not occur in adaptive systems as the learning process runs in the background.
 

\citet{moorkens2017assessing} conducted a questionnaire study to survey what PE functionalities users expected from their TM environment. In total, 81\% of the participants who replied would like to have a confidence score indicating the expected quality of the MT output.\footnote{Around 70\% of the participants also declared their interest in interactive and adaptive MT.} Some TM environments already offer this kind of evaluation\footnote{e.g. memsource, \url{https://help.memsource.com/hc/en-us/articles/360012527380-Machine-Translation-Quality-Estimation}, last accessed 11 March 2021}. Generally, these scores can help you accelerate the decision about how useful the MT output is and whether to translate from scratch or use the MT output. However, you have to keep in mind that those numbers are calculated automatically and that the process is more complex than the fuzzy match calculation. Hence, we would advise that you do not blindly trust a good quality estimation, but -- similarly as with most full matches -- recheck the MT output.

As PE has become an established task on the translation market, the tools and TM environments will further develop, and the TM environments will potentially increasingly adapt to the PE tasks. Hence, we can only advise that you try to keep up to date and inform yourself about new developments.

\newpage

\section*{Crossword puzzle -- chapter 6}

\begin{Puzzle}{13}{11}
|{}	|[3]T	|{}	|{}	|{}	|{}	|{}	|{}	|[7]A	|{}	|{}	|{}	|[4]F	|.
|{}	|E	|{}	|{}	|{}	|{}	|{}	|{}	|D	|{}	|{}	|{}	|Z |.
|{}	|R	|{}	|{}	|{}	|{}	|{}	|{}	|A	|{}	|{}	|{}	|U |.
|{}	|M	|{}	|{}	|{}	|{}	|[5]E	|M	|P	|T	|Y	|{}	|Z |.
|{}	|I	|{}	|[1]S	|{}	|{}	|{}	|{}	|T	|{}	|{}	|{}	|Y |.
|[6]I	|N	|T	|E	|R	|A	|C	|T	|I	|V	|E	|{}	|M |.
|{}	|O	|{}	|G	|{}	|{}	|{}	|{}	|V	|{}	|{}	|{}	|A |.
|{}	|L	|{}	|[2]M	|A	|N	|A	|G	|E	|M	|E	|N	|T |.
|{}	|O	|{}	|E	|{}	|{}	|{}	|{}	|{}	|{}	|{}	|{}	|C |.
|{}	|G	|{}	|N	|{}	|{}	|{}	|{}	|{}	|{}	|{}	|{}	|H |.
|{}	|Y	|{}	|T	|{}	|{}	|{}	|{}	|{}	|{}	|{}	|{}	|{} |.
\end{Puzzle}

\begin{PuzzleClues}{\textbf{Across}}
\Clue{2}{MANAGEMENT}{Translation Memory Systems often contain a component for project ...}
\Clue{5}{EMPTY}{When an MT system is activated, the MT output is often automatically inserted into ... segments, i.e. segments without matches from the translation memory.}
\Clue{6}{INTERACTIVE}{What do we call an MT system that changes the MT suggestion according to what is being typed while the translator starts post-editing?}
\end{PuzzleClues}

\begin{PuzzleClues}{\textbf{Down}}
\Clue{1}{SEGMENT}{On what level do translation memory systems usually store translations?}
\Clue{3}{TERMINOLOGY}{What else can we manage on a word-level with translation memory systems?}
\Clue{4}{FUZZYMATCH}{What do we call results from the translation memory that are not 100\% equal to the current segment?}
\Clue{7}{ADAPTIVE}{What do we call an MT system that learns from the post-edited segments?}
\end{PuzzleClues}
