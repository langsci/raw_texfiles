\chapter{MT and text types -- which influence do they have?}\label{sec:5}


    \objectives{
        You will learn...
        \begin{itemize}
            \item to assess which text types are more and which are less suitable for machine translation,
            \item what a controlled language is and how it influences MT output.
        \end{itemize}
        }

\vspace{\baselineskip}

This chapter deals with the special characteristics of text types and their effect on MT and PE. This chapter will deal with text types only. The next chapter will focus on PE risks associated with each text that influence the decision whether to use MT and PE. Although considerations on text types are very important for decisions concerning the PE project, the basic principles and assumptions can be explained quickly.


The text type is a very important factor to assess whether MT will be useful and effective for a given source text. Very creative texts are seldom considered suitable for MT because they require flexible translation solutions, variability and creativity. Poems would probably be one of the most demanding text types as content and form usually both play a very important role. They generally rely on rhythm and rhyme, which regular MT systems do not consider. When human translators translate poetry, they might have to translate the content very freely, which is usually also not possible in MT. Similarly, you will probably get unsuitable translations for slogans or certain advertising texts because they often rely on figurative speech or word plays. The concept \textit{transcreation} has become increasingly popular to describe texts that need to be translated very freely (see e.g. \citealt{pedersen2014exploring}).

Very restrictive, highly standardised, redundant and less creative texts on the other hand, are well-suited for MT. These are often domain-specific and adhere to pre-defined and strict text type conventions. Often, repetitions of style and terminology are desired, and the authors of the texts also have to follow certain rules when writing the text. Especially suitable for MT would hence be technical manuals or instructions that may even have been written in a controlled language. 

A controlled language (CL) is a reduced version of a natural language that follows certain rules and guidelines. As the name already implies, the language use is controlled. Often, these controlled languages use restricted vocabulary, where each word has only one meaning and each meaning is only represented by one word. The sentences are usually short and do not include complex syntactic structures. The passive voice is often avoided and sometimes even the use of tenses is restricted. These are only some examples for CL rules. Finally, controlled languages are often used in technical documentation, but also in other domain-specific communication. ``Simplified Technical English" is one quite famous example of a controlled language (e.g. \citealt{knezevic2015improving}). If you want to know more about controlled languages, you can find more information in \citet{kamprath1998controlled} or \citet{kittredge2003sublanguages}.

Using a CL in a MT workflow has proven to yield better MT quality. \citet{aikawa2007impact}, amongst others, show that the CL had a positive influence on PE productivity in three of four tested languages. In general, these improvements are especially measurable for rule-based and statistical MT systems. \citet{marzouk2019evaluation} published a first study on the influence of CL on different MT systems, including neural MT. The study shows that CL improves rule-based, statistical and hybrid MT; in contrast, it has little to no effect on neural MT since its quality is already exceptionally high for technical documentation, which was the text type under investigation. The study only tested a very limited set of rules. If the results can be applied to a more holistic rule-set, the study would imply that using a CL is obsolete when using neural MT. This will be the subject of future studies.

Apart from using a CL, the assumptions for text types also presume that the MT systems were trained on the respective text types and domain. Logically, a system that was trained on legal texts and then has to translate medical texts would probably also produce a lower quality output.

Finally, the use of MT and PE is also becoming more and more widespread within translation tasks which require creative translations. \citet{toral2018post}, for instance, showed that neural MT also provides valuable results for literary translation. Another example is the CompAsS (Computer-Assisted Subtitling)\footnote{\url{https://www.compass-subtitling.com}, last accessed 19/06/2021} project, which aimed at researching and optimising the overall multilingual subtitling process for public TV programs by developing a multi-modal subtitling platform combining automatic-speech-recognition, neural MT and translation management tools. The project results are promising since they prove significant gains in productivity while maintaining acceptable quality standards (\citealt{tardel2020effort}). We should however keep in mind that subtitles are regarded as creative texts, but they are also restricted and controlled since they follow certain rules and standards.

\hspace*{-2.5pt}Nonetheless, MT is still considered more suitable for domain-specific, restricted texts than for creative texts. One general rule of thumb would be that if texts are suitable for translation memory (TM) systems, they might also be suitable for MT systems. In state-of-the-art translation work benches, they are even combined -- i.e. MT candidates are suggested when there are no matches or when the TM fuzzy matches fall under a pre-defined threshold. So, if you are not considering using a TM -- a possible case might be the translation of a novel or an advertisement -- you should not use an MT system either. We will talk about the interaction of MT, PE and other tools in the next section (\sectref{sec:6}).

\newpage

\section*{Crossword puzzle -- chapter 5}

\begin{Puzzle}{11}{9}
|{}	|{}	|{}	|[4]S	|{}	|{}	|{}	|{}	|{}	|{}	|{}	|.
|{}	|{}	|{}	|U	|{}	|{}	|{}	|{}	|{}	|{}	|{}	|.
|{}	|{}	|{}	|B	|{}	|{}	|{}	|{}	|{}	|{}	|{}	|.
|[3]R	|E	|S	|T	|R	|I	|C	|T	|I	|V	|E	|.
|{}	|{}	|{}	|I	|{}	|{}	|{}	|{}	|{}	|{}	|{}	|.
|[1]C	|O	|N	|T	|R	|O	|L	|L	|E	|D	|{}	|.
|{}	|{}	|{}	|L	|{}	|{}	|{}	|{}	|{}	|{}	|{}	|.
|{}	|[2]C	|R	|E	|A	|T	|I	|V	|I	|T	|Y	|.
|{}	|{}	|{}	|S	|{}	|{}	|{}	|{}	|{}	|{}	|{}	|.
\end{Puzzle}

\begin{PuzzleClues}{\textbf{Across}}
\Clue{1}{CONTROLLED}{A ... language is a reduced version of a natural language that follows certain rules and guidelines.}
\Clue{2}{CREATIVITY}{What characteristic makes source texts less suitable for MT?}
\Clue{3}{RESTRICTIVE}{Very ..., standardised, and redundant texts are better suited for machine translations.}
\end{PuzzleClues}

\begin{PuzzleClues}{\textbf{Down}}
\Clue{4}{SUBTITLES}{For what kind of audio-visual translation have fist studies tested the use of MT and PE?}
\Clue{}{}{}
\Clue{}{}{}
\end{PuzzleClues}
