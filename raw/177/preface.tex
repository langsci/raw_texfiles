\chapter{前言}

%\largerpage[4]
\largerpage[-1]
本书是我的《语法理论》(\emph{Grammatiktheorie}) \citep{MuellerGTBuch2}这本德语书的扩展版与修订版。\dotfootnote{
本译著是对《语法理论⸺从转换语法到基于约束的理论》(\emph{Grammatical Theory: From Transformational Grammar to Constraint-Based Approaches})2016年版的翻译(其中第10章参考2018年版)。我们修改了第\ref{chapter-minimalism}章中的错误,并处理了其他一些小问题。⸺译者}
书中介绍了在当代理论学界发挥重要作用,或在过去做出重要贡献的如今仍占据主流地位的各种语法理论。我将这些理论的基本观点进行了解释,并将这些理论应用到德语的“核心语法”中。本书的目标语言仍然采用其德语版中所使用的目标语言,因为很多需要分析的现象不能用英语当作目标语言来解释。而且,很多理论都是在英语的基础上发展起来的,因为这些研究者的母语是英语,将这些理论应用到其他语言的研究中会很有启发性。我将说明这些理论是如何处理论元、附加语、主动/被动转换、局部语序重列(所谓的杂列)、动词位置以及长距离的短语前置(日耳曼语族的语言(除了英语)中动词位于第二位的属性)等现象。
%This book is an extended and revised version of my German book \emph{Grammatiktheorie}
%\citep{MuellerGTBuch2}. It introduces various grammatical theories that play a role in current
%theorizing or have made contributions in the past which are still relevant today. I explain some foundational
%assumptions and then apply the respective theories to what can be called the ``core grammar'' of
%German. I have decided to stick to the object language that I used in the German version of this
%book since many of the phenomena that will be dealt with cannot be explained with English as the object
%language. Furthermore, many theories have been developed by researchers with English as their native
%language and it is illuminative to see these theories applied to another language.
%I show how the theories under consideration deal with arguments and adjuncts, active/passive
%alternations, local reorderings (so-called scrambling), verb position, and fronting of phrases over
%larger distances (the verb second property of the Germanic languages without English).

第二部分探讨对理论发展很重要的根本性问题。这包括我们是否具有语言的内在知识的讨论、人类处理语言的心理语言学的证据的讨论、空语类的地位的讨论,以及我们是整体性地还是组合性地构造和获取话语的问题,也就是,我们是使用短语结构还是词汇结构。
%The second part deals with foundational questions that are important for developing theories.
%This includes a discussion of the question of whether we have innate domain specific knowledge of
%language (UG), the discussion of psycholinguistic evidence concerning the processing of language by
%humans, a discussion of the status of empty elements and of the question whether we construct and perceive utterances 
%holistically or rather compositionally, that is, whether we use phrasal or lexical constructions.

考虑到语言学这一科学领域中有大量的术语是混乱不清的,我在导言部分专门介绍了本书后面章节中将会运用到的术语。第二章介绍短语结构语法,该语法在本书介绍的许多理论中都发挥了重要的作用。我在德语专业本科生的导论课中讲解这两章的内容(除了\ref{sec-PSG-Semantik}有关短语结构语法和语义之间的关系)。高级读者可以略过这些导论性质的章节。后续的章节安排也适用于没有前期知识的读者来理解这些理论的基本内容。有关最新的理论发展的内容更有挑战性:这些内容参见后续将要介绍的章节,以及在现今的理论讨论中相关的其他文献,我们不会在本书中重复表示这些文献,或者对其进行总结。这部分内容可供高水平的学生与学者参考。我将这本书作为高年级本科生研讨课的教材,用它来讲解各种理论的句法方面。这些课件可在我的网页上下载。本书的第二部分更有挑战性,它包括对难点问题和当下研究文献的讨论。
%Unfortunately, linguistics is a scientific field 
%with a considerable amount of terminological chaos. I therefore wrote an introductory
%chapter that introduces terminology in the way it is used later on in the book. The second chapter
%introduces phrase structure grammars, which plays a role for many of the theories that are covered
%in this book. I use these two chapters (excluding the Section~\ref{sec-PSG-Semantik} on interleaving
%phrase structure grammars and semantics) in introductory courses of our BA curriculum for German
%studies. Advanced readers may skip these introductory chapters. The following chapters are
%structured in a way that should make it possible to understand the introduction of the theories
%without any prior knowledge. The sections regarding new developments and classification are more
%ambitious: they refer to chapters still to come and also point to other publications that are
%relevant in the current theoretical discussion but cannot be repeated or summarized in this
%book. These parts of the book address advanced students and researchers. I use this book for teaching
%the syntactic aspects of the theories in a seminar for advanced students in our BA. The slides are
%available on my web page. The second part of the book, the general discussion, is more ambitious and contains the discussion
%of advanced topics and current research literature.

本书只介绍相对较近的理论发展。对于历史文献的回顾,可参见,比如说 \citet{Robins97a-u}和\citet{JL2006a-u}。本书并不包括整
合语言学\isce{整合语言学}{Integrational Linguistics} \citep{Lieb83a-u,Eisenberg2004a,Nolda2007a-u}、
优选论\indexot(\citealp{PS93a-u};\citealp{Grimshaw97a-u};G.\ \citealp{GMueller2000a-u})、角色与参照语法\isce{角色与参照语法}{Role and Reference Grammar} \citep{vanValin93a-ed}以及关系语法
\isce{关系语法}{Relational Grammar} \citep{Perlmutter83a-ed,Perlmutter84b-ed}的内容。我将这些内容留到以后的版本中。
%This book only deals with relatively recent developments. For a historical overview, see for instance
% \citew{Robins97a-u,JL2006a-u}. I am aware of the fact that chapters on
%Integrational Linguistics\is{Integrational Linguistics}
%\citep{Lieb83a-u,Eisenberg2004a,Nolda2007a-u}, Optimality Theory\indexot (\citealp{PS93a-u};
%\citealp{Grimshaw97a-u}; G.\ \citealp{GMueller2000a-u}), Role and Reference Grammar\is{Role and
%  Reference Grammar} \citep{vanValin93a-ed} and Relational Grammar\is{Relational Grammar}
%\citep{Perlmutter83a-ed,Perlmutter84b-ed} are missing. I will leave these theories for later editions.

德语书的最初版本只计划写400页,但是最后的成书规模超出了这一计划:德语教材的第一版有529页,第二版有564页。我在英语版中加入了依存语法和英语的最简语法这两章内容,现在本书有\pageref{LastPage}页。我尽最大努力将所选的理论表述清楚,并列出所有重要的文献。尽管参考文献的列表超过了85页,我也有可能没有列出全部的文献。我对此和其他问题表示歉意。
%The original German book was planned to have 400 pages, but it finally was much bigger: the first
%German edition has 525 pages and the second German edition has 564 pages. I
%added a chapter on Dependency Grammar and one on Minimalism to the English version and now the
%book has \pageref{LastPage} pages. I tried to represent the chosen theories appropriately and to cite all important work. Although the list of
%references is over 85 pages long, I was probably not successful.
%I apologize for this and any other shortcomings.

% There is only one book.
%%% -*- coding:utf-8 -*-

\section*{Available versions of this book}

The canonical version of this book is the PDF document available from the Language Science Press
webpage of this book\footnote{%
\url{\lsURL}
}. This page also links to a Print on Demand version. Since the book is very long, we decided to
split the book into two volumes. The first volume contains the description of all theories and the
second volume contains the general discussion. Both volumes contain the complete list of references
and the indices. The second volume starts with page~\pageref{part-discussion}. The printed volumes
are therefore identical to the parts of the PDF document.







%      <!-- Local IspellDict: en_US-w_accents -->


\addchap{Acknowledgments} 
%content goes here
The help and support of Martin Haspelmath and Sebastian Nordhoff in the preparation of this volume is gratefully acknowledged. 

We would also like to thank the authors of the chapters in this volume for their cooperation during the editing process and especially for their input to the reviewing of chapters by their peers. 

We especially thank the following additional external reviewers, %individuals, 
who contributed their time and expertise to provide independent peer review for the papers in this collection: Lisa Bonnici, Jason Brown, Elisabet Engdahl, Marieke Hoetjes, Beth Hume, Anne O'Keefe, Adam Schembri, Thomas Stolz, Andy Wedel and Shuly Wintner.
 

\section*{本书的出版过程}
%\section*{On the way this book is published}

我从1994年开始写我的毕业论文,并在1997年成功通过答辩。这一阶段的手稿可以在我的网页上获取。在答辩之后,我必须要找到出版商。我很高兴收到了Niemeyer的“语言学研究”系列丛书的邀请,但是同时我对价格感到震惊不已,当时每本书需要186德国马克,这还是在我没有出版商的任何帮助的情况下,自己写书和排版的价格(这个价格是纸版小说的二十倍)。\dotfootnote{%
与此同时,Niemeyer被de Gruyter收购,并停止营业了。这本书的价格现在是139.95欧元 / 196.00美元。欧元的价格相当于273.72德国马克。
} 这基本上意味着我的书是没有出版的:直到1998年,才能在我的网站上看到这本书,并随后在图书馆可以查询到。我的教授转正著作由CSLI出版社出版,价格相对来说合理多了。在我开始写教科书的时候,我就寻找不同的出版渠道,并跟无名印刷需求的出版社协商。Brigitte Narr负责管理Stauffenburg出版集团,她说服我在他们的出版社出版HPSG的教材。这本书的德语版属于我,这样我就可以在我的主页上出版。这一合作是成功的,由此我还可以跟Stauffenburg出版我的第二本关于语法理论的教科书。我想这本书具有更为广泛的相关性,并且可以供非德语的读者阅读。由此,我决定将它翻译为英语。不过,Stauffenburg重点出版德语书籍,我必须找到另一家出版社。幸运的是,出版界的情况与1997年相比发生了戏剧性的翻天覆地的变化:我们现在有高水平的出版社,不仅有严格的同行评审,还有着完全公开的途径。我很高兴Brigitte Narr将本书版权卖回给我,我现在就可以在CC-BY版权下由语言科学出版社出版这本英文版教材了。
%I started to work on my dissertation in 1994 and defended it in 1997. During the whole time the
%manuscript was available on my web page. After the defense, I had to look for a publisher. I was
%quite happy to be accepted to the series \emph{Linguistische Arbeiten} by Niemeyer, but at the same time I
%was shocked about the price, which was 186.00 DM for a paperback book that was written and typeset
%by me without any help by the publisher (twenty times the price of a paperback novel).\footnote{%
 % As a side remark: in the meantime Niemeyer was bought by de Gruyter and closed down. The price of the book is now
 % 139.95 \euro / \$ 196.00. The price in Euro corresponds to 273.72 DM. 
%%This is a price increase of 47\,\%.
%} This
%basically meant that my book was depublished: until 1998 it was available from my web page and after
%%this it was available in libraries only. My Habilitationsschrift was published by CSLI Publications
%for a much more reasonable price. When I started writing textbooks, I was looking for alternative
%distribution channels and started to negotiate with no-name print on demand publishers. Brigitte Narr,
%who runs the Stauffenburg publishing house, convinced me to publish my HPSG textbook with her. The
%\textsc{cop}yrights for the German version of the book remained with me so that I could publish it on my web page. The collaboration was successful so that I also published my second textbook about
%grammatical theory with Stauffenburg. I think that this book has a broader relevance and should be
%accessible for non-German-speaking readers as well. I therefore decided to have it translated into
%English. Since Stauffenburg is focused on books in German, I had to look for another publisher. Fortunately the situation in the publishing sector changed quite dramatically in comparison
%to 1997: we now have high profile publishers with strict peer review that are entirely open access. I am very
%glad about the fact that Brigitte Narr sold the rights of my book back to me and that I can now 
%publish the English version with Language Science Press under a CC-BY license.

%      <!-- Local IspellDict: en_US-w_accents -->

\section*{Language Science Press: scholar-owned high quality linguistic books}

In 2012 a group of people found the situation in the publishing business so unbearable that they
agreed that it would be worthwhile to start a bigger initiative for publishing linguistics books in
platinum open access, that is, free for both readers and authors. I set up a web page and collected
supporters, very prominent linguists from all over the world and all subdisciplines and Martin
Haspelmath and I then founded Language Science Press. At about the same time the DFG had announced
a program for open access monographs and we applied \citep{MH2013a} and got funded (two out of 18 applications got
funding). The money was used for a coordinator (Dr.\ Sebastian Nordhoff) and an economist (Debora
Siller), two programmers (Carola Fanselow and Dr.\ Mathias Schenner), who worked on the publishing
plattform Open Monograph Press (OMP) and on conversion software that produces various formats (ePub, XML,
HTML) from our \LaTeX{} code. Svantje Lilienthal worked on the documentation of OMP, produced
screencasts and did user support for authors, readers and series editors.

OMP was extended by open review facilities and community-building gamification tools
\citep{MuellerOA,MH2013a}. All Language Science Press books are reviewed by at least two external
reviewers. Reviewers and authors may agree to publish these reviews and thereby make the whole
process more transparent (see also \citew{Pullum84a} for the suggestion of open reviewing of journal
articles). In addition there is an optional second review phase: the open
review (see the blog posts by Sebastian Nordhoff about the reviewing options at Language Science
Press\footnote{%
\url{https://userblogs.fu-berlin.de/langsci-press/2015/05/27/axes-of-open-review/}, 2020-09-03.
}). This second optional reviewing phase is completely open to everybody. The whole community may comment on the document
that is published by Language Science Press. After this second review phase, which usually lasts for
two months, authors may revise their publication and an improved version will be published. The
English version of this book was the first book to go through this open review phase. The Chinese
translation was also open for comments on Paperhive. Readers left more than 2500 comments\footnote{%
\url{https://paperhive.org/documents/items/Zf2Qf47i6nf2}, 2020-09-03.}, which were automatically fed into the version control
and bug tracking system used by Language Science Press\footnote{%
\url{https://github.com/langsci/177/}, 2020-09-03.
}.

Currently, Language Science Press has 26 series on various subfields of linguistics with high
profile series editors from all continents. There are 437 members in the respective editorial boards
coming from 49 countries. We have 134 published books with more than 1 Mio downloads.\footnote{%
Downloads by robots excluded, the English version of this textbook was downloaded over 40,000 times
since 2016.
} 1196 authors from 53 countries have published books or chapters with Language Science Press as of March 2020 and there are
572 expressions of interest. 
%Two multi-volume handbooks, one on HPSG and one on LFG, are in
%preparation \citep{HPSGHandbook,LFGhandbook}.


Series editors are responsible for delivering manuscripts that are typeset in \LaTeX{}, but they are
supported by a web-based typesetting infrastructure that was set up by Language Science Press and
there is also conversion software converting Word manuscripts into \LaTeX{}. Proofreading is
community-based. Until now 224 people helped improve our books. Their work is documented in the
Hall of Fame: \url{http://langsci-press.org/hallOfFame}.


Language Science Press is a community"=based publisher, but apart from the press managers Martin
Haspelmath and me, there are two people who are employed for the central organization and
typesetting: Sebastian Nordhoff, who is also a press manager, and Felix Kopecky, who does 
typesetting. Both have 50\,\% positions. In the period of 2018--2020, these two positions got payed
with the help of financial support by 115 academic institutions including  
Harvard, the MIT, and Berkeley and by societies like EuroSLA.\footnote{%
  A full list of supporting institutions is available at:
  \url{http://langsci-press.org/knowledgeunlatched}.
} The Language Science Press approach is endorsed by the leading scholars Noam Chomsky, Adele
Goldberg, and Steven Pinker, who sent letters of support in 2017.\footnote{%
``Very pleased to learn about this fine initiative, a most valuable way to
bring to the general public the results of scholarly work.  It's a
cliché, but true, that we all stand on the shoulders of giants, and rely
on the cultural wealth provided to everyone by past generations.  It is
only proper that the public should gain access to whatever contemporary
scholarship can contribute, and the ideas outlined here seem to be a
very promising way to realize this ideal.'' Noam Chomsky, 2017-02-01.
 
``Language Science Press is setting a standard for freely accessible
articles and books that are carefully reviewed.'' Adele Goldberg, 2017-05-02. 

``Sharing data and methods is one of the pillars of scholarly inquiry. The knowledge created by
scholars belongs to everyone, and open access publications are a major pathway to realizing that
ideal. Language Science Press, together with Knowledge Unlatched, provides an excellent way for us
to make our findings available to the global public.'' Steven Pinker, 2017-01-22. 
} The fundraising for the period 2021--2023 is ongoing.

If you think that textbooks like this one should be freely available to whoever wants to read them
and that publishing scientific results should not be left to profit-oriented publishers, then you
can join the Language Science Press community and support us in various ways: you can register with Language Science Press and have your name
listed on our supporter page with more than 1000 other enthusiasts, you may devote your time and help
with proofreading. We are also looking for institutional supporters like foundations,
societies, linguistics departments or university libraries. Detailed information on how to support
us is provided at the following webpage: \url{http://langsci-press.org/supportUs}.
In case of questions, please contact me or the Language Science Press coordinator at \href{mailto:contact@langsci-press.org}{contact@langsci-press.org}.


~\medskip

\noindent
Berlin, September 04, 2020\hfill Stefan Müller


%      <!-- Local IspellDict: en_US-w_accents -->


%\section*{Preface to the Chinese version}
\section*{中译本前言}

%\largerpage
\enlargethispage{8pt}
我非常高兴看到这本语法理论教材翻译成中文。我希望它对很多语言学专业的学生有所帮助。我衷心感谢王璐璐为这本书所付出的努力。我不知道她在建议翻译本书的时候是否知道需要做多少工作。我是不知道的。不管怎样,我很欣慰本书已翻译完成,我们也完成了这个项目。我还要感谢黄思思和孙薇薇。黄思思负责翻译第\ref{Kapitel-CxG}章、第\ref{Abschnitt-Generativ-Modelltheoretisch}章至第\ref{Abschnitt-UG-mit-Hierarchie}章。孙薇薇负责翻译第\ref{Kapitel-LFG}章、第\ref{Kapitel-CG}章和第\ref{Kapitel-TAG}章。王璐璐负责翻译其余各章及全书统稿。\dotfootnote{
王璐璐在中国传媒大学人文学院工作;黄思思在华侨大学华文教育研究院工作;孙薇薇在剑桥大学计算机科学技术系工作,本项翻译任务系其在北京大学王选所工作时完成。⸺译者
}
%、第\ref{Abschnitt-Diskussion-Performanz}章、第\ref{chap-acquisition}章、第\ref{sec-generative-capacity}章、第\ref{Kapitel-Binarybranching-locality-recursion}章、第\ref{Abschnitt-Diskussion-leere-Elemente}章、第\ref{chap-scrambling-extraction-passive}章、第\ref{Abschnitt-Phrasal-Lexikalisch}章、第\ref{Abschnitt-UG-mit-Hierarchie}章。
%Seeing this Chinese version of the grammar theory textbook makes me very happy. I hope it will be useful
%for many students of linguistics. I wholeheartedly want to thank Wang Lulu (王璐璐) for all the efforts she
%put into the book. I am not sure that she knew how much work this was when she suggested to
%translate the book. I did not. In any case I am really grateful that this book is translated and
%that we finished this project. I want to also thank XX (孙薇薇) and XX (黄思思) for translating Chapter~ and Chapter~,
%respectively.

我要感谢在早期的公开评审阶段提出宝贵意见的曹晓玉、刘海涛和王佳骏。我还要感谢14位校对者(%
%\makeatletter\@proofreader\makeatother
陈榕、霍安頔(Andreas Hölzl)、林巧莉、刘畅、李姝姝、练斐、牛若晨、史红改、佟和龙 、屠爱萍、万姝君、王小溪、杨丰榕 、曾巧仪),他们对本译本的一章或几个章节提出了校对意见,他们的工作切实地提高了本译本的质量。我从他们每一个人那里得到的评论都比我从商业出版社获得的评论多得多。我并没有阅读所有的评论,因为它们大部分都是中文,而我并不会汉语。但是王璐璐告诉我,这些评论跟我在英语版所得到的评论一样:很多评论是关于内容的,而不是仅限于错别字和排版的问题。没有一家商业出版机构的校对员能够发现这些小的错误和缺陷,因为商业出版机构的雇员并不懂得本书所涵盖的所有理论知识。这里还需要特别感谢的有冯志伟、刘晓、卢达威、梅德明、夏军、詹卫东,以及Jeroen van de Weijer,他们都为本书提出了许多中肯的意见。
%I also want to thank the XX proofreaders (\makeatletter\@proofreader\makeatother) that each worked on one or more chapters and
%really improved this book. We got more comments from every one of them than I ever got for a book
%done with a commercial publisher. I could not read all of the comments since most of them were in
%Mandarin Chinese, which I do not speak. But Wang Lulu told me that the comments were comparable in
%kind to the comments I got for the English version: some comments were on content rather than on typos and layout
%issues. No proofreader employed by a commercial publisher would have spotted these little mistakes and
%shortcomings since commercial publishers do not have staff that knows all the grammatical
%theories that are covered in this book. 

%~\medskip
%~\\
%\noindent
\begin{flushright}
\begin{tabular}{c@{}}
%斯特凡 $\cdot$ 米勒\\
Stefan Müller\\
柏林\\
\mytodayc%2019年5月9日\\
\end{tabular}
\end{flushright}
%Berlin, \today\hfill Stefan Müller

% lulu/wsun/sisi: DONE
%      <!-- Local IspellDict: en_US-w_accents -->
