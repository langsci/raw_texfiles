%% -*- coding:utf-8 -*-

\chapter{二叉、局部性和递归性}
%\chapter{Binary branching, locality, and recursion}
\label{Kapitel-Binarybranching-locality-recursion}

本章讨论三个问题:\ref{sec-branching}分析的问题是,是否所有的语言结构都应该是二叉的。\ref{sec-locality}讨论什么信息可以用于选择,即支配中心语是否可以选择被选择成分的内部结构,或是否所有的成分都要限于局部选择。最后,\ref{sec-recursion}讨论了递归性和本书所讲的不同语法理论是否实现了递归性,以及怎样实现递归性的。
%This chapter discusses three points: section~\ref{sec-branching} deals with the question of whether all
%linguistic structures should be binary branching or not. Section~\ref{sec-locality} discusses the
%question what information should be available for selection, that is, whether governing heads can
%access the internal structure of selected elements or whether everything should be restricted to
%local selection. Finally, Section~\ref{sec-recursion} discusses recursion and how/whether it is
%captured in the different grammar theories that are discussed in this book.

%% -*- coding:utf-8 -*-

\section{二叉}
%\section{Binary branching}
\label{sec-branching}

我们\iscesub[|(]{分支}{branching}{二叉}{binary}已经看到这类分支问题在不同理论中的处理方式不同。经典\xbartc 认为一个动词可以与其所有补足语组合。在GB后来的变体中,所有结构都被严格限制为二叉的。其余理论框架在处理分支这一问题时采取的方式有所不同:有的理论坚持二叉结构,而其他理论框架选择平铺结构。
%We\is{branching!binary|(} have seen that the question of the kind of branching structures assumed has received differing treatments in various theories.
%Classical \xbart assumes that a verb is combined with all its complements. In later variants of GB, all structures are strictly binary branching.
%Other frameworks do not treat the question of branching in a uniform way: there are proposals that assume binary branching structures and others
%that opt for flat structures.

\citet[第2.5节]{Haegeman94a-u}以学习力作为论据\isce{语言习得}{language acquisition} (习得等级,见\ref{Abschnitt-Geschwindigkeit-Spracherwerb}对于这一问题的论述)。她讨论了例(\mex{1})中的句子并且表示如果自然语言中允许平铺结构的话,语言学习者必须从八种结构中选择其中一种。另一方面,如果语言中只有二叉结构,那么首先例(\mex{1})中的句子就不会有图\vref{Abbildung-Haegmann-flach}中的结构,所以学习者就不必排除对应的假设。
% \citet[Section~2.5]{Haegeman94a-u} uses learnability arguments\is{language acquisition} (rate of acquisition, see Section~\ref{Abschnitt-Geschwindigkeit-Spracherwerb}
%on this point).
%She discusses the example in (\mex{1}) and claims that language learners have to choose one of eight structures if flat-branching structures can occur in natural
%language. If, on the other hand, there are only binary-branching structures, then the sentence in (\mex{1}) cannot have the structures in
%Figure~\vref{Abbildung-Haegmann-flach} to start with, and therefore a learner would not have to rule out the corresponding hypotheses.
\ea 
\gll Mummy must leave now.\\
     妈妈 必须 离开 现在\\
\mytrans{妈妈必须现在离开。}
\z
\begin{figure}
\begin{forest}
sm edges, empty nodes
[{},phantom    
[{},tier=flat
 [{} [Mummy;妈妈]]
 [{} [must;必须]]
 [{} [leave;离开]]
 [{} [now;现在]]]
%%%%%%%%%%%%%%%%%%%%%%%%%%%%%%
[{}
 [{},tier=flat 
     [{} [Mummy;妈妈]]
     [{} [must;必须]]
     [{} [leave;离开]]]
 [{} [now;现在]]]
%%%%%%%%%%%%%%%%%%%%%%%%%%%%%%%
[{} 
 [{} [Mummy;妈妈]]
 [{},tier=flat 
     [{} [must;必须]]
     [{} [leave;离开]]
     [{} [now;现在]]]]
]
\end{forest}
%\caption{\label{Abbildung-Haegmann-flach}Structures with partial flat-branching}
\caption{\label{Abbildung-Haegmann-flach}部分平铺结构}
\end{figure}%

\noindent
但是, \citet[\page 88]{Haegeman94a-u}提供了证据证明例(\mex{0})的结构如(\mex{1})所示:
%However,  \citet[\page 88]{Haegeman94a-u} provides evidence for the fact that (\mex{0}) has the structure in (\mex{1}):
\ea
\gll {}[Mummy [must [leave now]]]\\
     \spacebr{}妈妈 \spacebr{}必须 \spacebr{}离开 现在\\
     \mytrans{妈妈必须现在离开}
\z
证明这一点的相关测试包括省略构式\isce{省略}{ellipsis},换句话说,可以用代词指称(\mex{0})中的成分。这意味着确实有证据支持语言学家假设的例(\mex{-1})的结构,因此不必假设在我们大脑中,只有二叉结构是被允准的。 \citet[\page 143]{Haegeman94a-u}提到了二叉假说的后果:如果所有的结果都是二叉的,那么在\xbartc 中不可能直接解释包含双及物动词的句子。在\xbartc 中,假设一个中心语与其所有补足语同时组合(见\ref{sec-xbar})。所以,为在\xbartc 中解释双及物动词,就必须假设一个空成分\isce{空成分}{empty element}(\littlevc)(见\ref{sec-little-v})。
%The relevant tests showing this include elliptical constructions\is{ellipsis}, that is, the fact that it is possible to
%refer to the constituents in (\mex{0}) with pronouns. This means that there is actually evidence for
%the structure of (\mex{-1}) that is assumed by linguists and we therefore do not have to assume that
%it is just hard-wired in our brains that only binary-branching structures are allowed.  \citet[\page
%  143]{Haegeman94a-u} mentions a consequence of the binary branching hypothesis: if all structures are
%binary-branching, then it is not possible to straightforwardly account for sentences with
%ditransitive verbs in \xbart. In \xbart, it is assumed that a head is combined with all its
%complements at once (see Section~\ref{sec-xbar}). So in order to account for ditransitive verbs in
%\xbart, an empty element\is{empty element} (\littlev) has to be assumed (see Section~\ref{sec-little-v}).

在\ref{Abschnitt-PSA}讨论刺激贫乏论的过程中我们就应该清楚,只允许二叉结构是我们天赋语言知识的一部分的假设只是一种猜想。Haegeman没有为这一假设提供任何证据。在我们所见的各种理论的讨论中,可以用平铺结构来描述数据。例如,可以假设,在英语中动词与其论元用一个平铺结构来组合\citep[\page 39]{ps2}。有时候有一些理论内部的原因使得选择其中一种分支或另外一种,但是对于其他理论并非总是可行。例如,\gbtc 中的约束理论\isce{约束理论}{Binding Theory}是通过句法树中的统制关系来实现的 \citep[\page 188]{Chomsky81a}。如果假设句法结构对于代词约束有重要作用的话(见第~\pageref{Seite-Bindungstheorie}页),那么就可以根据可见的约束关系来就句法结构做出假设(也可以参见\ref{sec-little-v})。但是,约束现象在不同理论中受到了不同对待。在LFG\indexlfg 中,对于f-结构\isce{f-结构}{f-structure}的限制用于约束理论\citep{Dalrymple93a},但是在HPSG\indexhpsg 理论中约束理论用论元结构列表\isfeat{arg-st}(以一种特定顺序排列的价信息,见\ref{Abschnitt-Arg-St})来操作。
%It should have become clear in the discussion of the arguments for the Poverty of the Stimulus in Section~\ref{Abschnitt-PSA} that
%the assumption that only binary-branching structures are possible is part of our innate linguistic knowledge is nothing more than pure
%speculation. Haegeman offers no kind of evidence for this assumption. As shown in the discussions of the various theories we have seen,  
%it is possible to capture the data with flat structures. For example, it is possible to assume that, in English, the verb
%is combined with its complements in a flat structure \citep[\page 39]{ps2}. There are sometimes theory-internal reasons for
%deciding for one kind of branching or another, but these are not always applicable to other theories. For example, Binding Theory\is{Binding Theory}
%in \gbt is formulated with reference to dominance relations in trees \citep[\page 188]{Chomsky81a}. If one assumes that syntactic structure plays
%a crucial role for the binding of pronouns (see page~\pageref{Seite-Bindungstheorie}), then it is possible to make assumptions about syntactic
%structure based on the observable binding relations  (so also Section~\ref{sec-little-v}). Binding data have, however, received a very different treatment in various theories.
%In LFG\indexlfg, constraints on f-structure\is{f-structure} are used for Binding Theory \citep{Dalrymple93a}, whereas Binding Theory
%in HPSG\indexhpsg operates on argument structure lists\isfeat{arg-st} (valence information that are ordered in a particular way,
%see Section~\ref{Abschnitt-Arg-St}).
 
与Haegeman观点相反的是\citet[第1.6.2节]{Croft2001a}提出了支持平铺结构。在其\indexcxg 激进构式语法(Radical Construction Grammar)FAQ中,Croft注意到像(\mex{1}a) 中所示的短语构式可以被转换成(\mex{1}b)所指的范畴语法的词项\indexcg。
%The opposite of Haegeman's position is the argumentation for flat structures put forward by Croft
%\citeyearpar[Section~1.6.2]{Croft2001a}. In his\indexcxg Radical Construction Grammar FAQ, Croft observes that
%a phrasal construction such as the one in (\mex{1}a) can be translated into a Categorial Grammar
%lexical entry\indexcg like (\mex{1}b).
\eal
\ex {}[\sub{VP} V NP ]
\ex VP/NP
\zl
他认为范畴语法的一个劣势在于它只允许二叉结构,而确实存在包含多于两个部分的构式(第49页)。但是他没有揭示这个问题的准确原因。他甚至自己也承认在范畴语法中可以用多于两个论元的方式来表示构式。对于一个双及物动词,英语范畴语法中的词项应该如 (\mex{1})所示:
%He claims that a disadvantage of Categorial Grammar is that it only allows for binary-branching structures and yet there exist constructions
%with more than two parts (p.\,49). The exact reason why this is a problem is not explained, however. He even acknowledges himself that
%it is possible to represent constructions with more than two arguments in Categorial Grammar. For a ditransitive verb, the entry in Categorial
%Grammar of English would take the form of (\mex{1}):
\ea
((s\bs np)/np)/np
\z
\largerpage
如果我们考察图\vref{Abbildung-TAG-flach-binaer}所示的TAG初级树,就清楚向平铺树和二叉树中融入语义信息都是可行的。二叉树对应范畴语法中的派生树。
%If we consider the elementary trees for TAG in Figure~\vref{Abbildung-TAG-flach-binaer}, it becomes clear that it is equally possible
%to incorporate semantic information into a flat tree and a binary-branching tree.
\begin{figure}
\hfill
\adjustbox{valign=c}{%
\begin{forest}
tag
[S
	[NP$\downarrow$]
	[VP
		[V
			[gives;给]]
		[NP$\downarrow$]
		[NP$\downarrow$]]]
\end{forest}
}
\hfill
\adjustbox{valign=c}{%
\begin{forest}
tag
[S
	[NP$\downarrow$]
	[VP
		[\hspaceThis{$'$}V$'$
			[V
				[gives;给]]
			[NP$\downarrow$]]
		[NP$\downarrow$]]]
\end{forest}}
\hfill\mbox{}
\caption{\label{Abbildung-TAG-flach-binaer}平铺和二叉的基本树}
%	\caption{\label{Abbildung-TAG-flach-binaer}Flat and binary-branching elementary trees}
\end{figure}%
在图\ref{Abbildung-TAG-flach-binaer}的两种分析中,都要赋予带有多个论元的中心语一个意义。归根结底,所需的确切结构取决于人们希望构成的结构的各种限制。本书没有论及这类限制,但是正如上面所论述的,有些理论借助树结构来建立约束关系\isce{约束理论}{Binding Theory}的模型。反身代词\iscesub{代词}{pronoun}{反身}{reflexive}必须限制在树的一个特定局域中。在LFG\indexlfg 和HPSG\indexhpsg 等理论中,这些约束限制没有借助树来刻画。这意味着来自于图\ref{Abbildung-TAG-flach-binaer}某一结构(或其他树结构)的约束现象的证据只是一种理论内部的证据。
%The binary-branching tree corresponds to a Categorial Grammar derivation. In both analyses in 
%Figure~\ref{Abbildung-TAG-flach-binaer}, a meaning is assigned to a head that occurs with a certain
%number of arguments. Ultimately, the exact structure required depends on the kinds of restrictions on structures
%that one wishes to formulate.
%In this book, such restrictions are not discussed, but as explained above some theories model binding relations\is{Binding Theory}
%with reference to tree structures. Reflexive pronouns\is{pronoun!reflexive} must be bound within a particular local domain inside the
%tree. In theories such as LFG\indexlfg and HPSG\indexhpsg, these binding restrictions are formulated
%without any reference to trees.
%\todostefan{Das stand irgendwie schon oben. Vielleicht ist aber ein
%  bisschen Redundanz auch OK.}
%This means that evidence from binding data for one of the structures in Figure~\ref{Abbildung-TAG-flach-binaer} (or for
%other tree structures) constitutes nothing more than theory-internal evidence.

假设句法树有多种结构的另一个动因是可以在任意结点插入附加语\isce{附加语}{adjunct}。在第\ref{Kapitel-HPSG}章中,给出了一个基于HPSG的假设双分支结构的分析。有了这一分析,就可能将一个附加语附加到任意结点,并借此解释附加语在中间区域的自由排列:
%Another reason to assume trees with more structure is the possibility to insert adjuncts\is{adjunct} on any node.
%In Chapter~\ref{Kapitel-HPSG}, an HPSG analysis for German that assumes binary-branching structures was proposed.
%With this analysis, it is possible to attach an adjunct to any node and thereby explain the free ordering of adjuncts
%in the middle field:
\eal
\ex 
\gll {}[weil] der Mann der Frau das Buch \emph{gestern} gab\\
	 {}\spacebr{}因为 \defart{} 男人 \defart{} 女人 \defart{} 书 昨天 给\\
\mytrans{因为这个男人昨天给这个女人这本书}	 
%	 {}\spacebr{}because the man the woman the book yesterday gave\\
%\mytrans{because the man gave the woman the book yesterday}
\ex 
\gll {}[weil] der Mann der Frau \emph{gestern} das Buch gab\\
	 {}\spacebr{}因为 \defart{} 男人 \defart{} 女人 昨天 \defart{} 书 给\\
%	 {}\spacebr{}because the man the woman yesterday the book gave\\
\mytrans{因为这个男人昨天给这个女人这本书}	
\ex 
\gll {}[weil] der Mann \emph{gestern} der Frau das Buch gab\\
	 {}\spacebr{}因为 \defart{} 男人 昨天 \defart{} 女人 \defart{} 书 给\\
%	 {}\spacebr{}because the man yesterday the woman the book gave\\
\mytrans{因为这个男人昨天给这个女人这本书}	
\ex 
\gll {}[weil] \emph{gestern} der Mann der Frau das Buch gab\\
	 {}\spacebr{}因为 昨天 \defart{} 男人 \defart{} 女人 \defart{} 书 给\\
%	 {}\spacebr{}because yesterday the man the woman the book gave\\
\mytrans{因为这个男人昨天给这个女人这本书}	
\zl
但是这个分析并不是唯一的可能。还可以假设一个完全平铺的结构,在这一结构中论元和附加语由一个结点统制。 \citet{Kasper94a}在\hpsgc 理论框架中给出了这样一种分析(也可以参见\ref{Abschnitt-Adjunkte-GPSG}中GPSG使用元规则\isce{元规则}{metarule}来引入附加语的分析)。Kapser需要复杂的关系约束来产生句法树中元素之间的句法关系,并且使用动词和附加语来计算整个成分的语义贡献。使用二叉结构的分析比使用复杂的关系约束的方法更加简单并且⸺鉴于平铺结构缺少理论外部的证据⸺应该选用平铺结构的分析。关于这一点,有人可能会反对说英语的附加语不能出现在论元之间的所有位置上,所以借助二叉的范畴语法分析和图\ref{Abbildung-TAG-flach-binaer}中的TAG分析都是错误的。但是,这是不对的,因为指定附加语的附加位置在范畴语法中是非常重要的。一个副词有范畴 (s\bs np)\bs (s\bs np) 或 (s\bs np)/(s\bs np) ,所以只能与图\ref{Abbildung-TAG-flach-binaer}所示的VP结点对应的成分组合。以同样的方式,在TAG中一个副词的初级树也只能附加到VP结点上(见第\pageref{abb-Adjunktion}页的图\ref{abb-Adjunktion})。所以,就英语附加语的处理而言,二叉结构因此不会做出任何错误的预测。
\iscesub[|)]{分支}{branching}{二叉}{binary}
%This analysis is not the only one possible, however. One could also assume an entirely flat structure where arguments
%and adjuncts are dominated by one node.  \citet{Kasper94a} suggests this kind of analysis in
%\hpsg (see also Section~\ref{Abschnitt-Adjunkte-GPSG} for GPSG analyses that make use of metarules\is{metarule} for the introduction of adjuncts). Kasper requires complex relational constraints\is{relation} that
%create syntactic relations between elements in the tree and also compute the semantic contribution of the entire constituent using the meaning
%of both the verb and the adjuncts. The analysis with binary-branching structures is simpler than those with complex relational constraints and --
%in the absence of theory-external evidence for flat structures -- should be preferred to the analysis with flat structures.
%At this point, one could object that adjuncts in English cannot occur in all positions between arguments and therefore the binary-branching
%Categorial Grammar analysis and the TAG analysis in Figure~\ref{Abbildung-TAG-flach-binaer} are wrong. This is not correct, however, as it is
%the specification of adjuncts with regard to the adjunction site that is crucial in Categorial Grammar.
%An adverb has the category (s\bs np)\bs (s\bs np) or (s\bs np)/(s\bs np) and can therefore only be combined with constituents that correspond to the VP node in
%Figure~\ref{Abbildung-TAG-flach-binaer}. In the same way, an elementary tree for an adverb in TAG
%can only attach to the VP node (see Figure~\ref{abb-Adjunktion} on
%page~\pageref{abb-Adjunktion}). For the treatment of adjuncts in English, binary-branching
%structures therefore do not make any incorrect predictions.
%\is{branching!binary|)}


%      <!-- Local IspellDict: en_US-w_accents -->

%% -*- coding:utf-8 -*-

\section{Locality}
\label{Abschnitt-Diskussion-Lokalitaet}\label{sec-locality}

The\is{locality|(} question of local accessibility of information has been treated in various ways by the theories
discussed in this book. In the majority of theories, one tries to make information about the inner workings of phrases inaccessible
for adjacent or higher heads, that is, \emph{glaubt} `believe' in (\mex{1}) selects a sentential argument but it cannot ``look
inside'' this sentential argument.
\eal
\ex 
\gll Karl glaubt, dass morgen seine Schwester kommt.\\
	 Karl believes that tomorrow his sister comes\\
\glt `Karl believes that his sister is coming tomorrow.'
\ex 
\gll Karl glaubt, dass seine Schwester morgen kommt.\\
	 Karl believes that his sister tomorrow comes\\
\zl
Thus for example, \emph{glauben} cannot enforce that the subject of the verb has to begin with a consonant or that the complementizer
has to be combined with a verbal projection starting with an adjunct.
In Section~\ref{Abschnitt-Kopf}, we saw that it is a good idea to classify constituents in terms of their distribution and
independent of their internal structure. If we are  talking about an NP box, then it is not important what this NP box actually
contains. It is only of importance that a given head wants to be combined with an NP with a
particular case marking. This is called \emph{locality of selection}.

Various linguistic theories have tried to implement locality of selection. The simplest form of this implementation is shown by
phrase structure grammars of the kind discussed in Chapter~\ref{Kapitel-PSG}. The rule in (\ref{ditrans-schema}) on
page~\pageref{ditrans-schema}, repeated here as (\mex{1}), states that a ditransitive verb can occur with three noun phrases, each
with the relevant case:
\ea
\begin{tabular}[t]{@{}l@{ }l@{ }l}
S  & $\to$ & NP({Per1},{Num1},{nom}) \\
   &       & NP(Per2,Num2,{dat})\\
   &       & NP(Per3,Num3,{acc})\\
   &       & V({Per1},{Num1},ditransitive)\\
\end{tabular}
\z
Since the symbols for NPs do not have any further internal structure, the verb cannot require that there has to be a relative
clause in an NP, for example. The internal properties of the NP are not visible to the outside.
We have already seen in the discussion in Chapter~\ref{Kapitel-PSG}, however, that certain properties of phrases have to be outwardly
visible. This was the information that was written on the boxes themselves. For noun phrases, at least information about
person, number and case are required in order to correctly capture their relation to a head.
The gender value is important in German as well, since adverbial phrases such as \emph{einer nach
  dem anderen} `one after the other' have to agree in gender\is{gender} with the noun they refer to
(see example (\ref{Beispiel-einer-nach-dem-anderen}) on
page~\pageref{Beispiel-einer-nach-dem-anderen}). Apart from that, information about the length of
the noun phrases is required, in order to determine their order in a clause. Heavy constituents are normally ordered after lighter ones, and are also often
extraposed (cf.\ Behaghel's \textit{Gesetz der wachsenden Glieder}\is{Gesetz der wachsenden Glieder} `Law of increasing constituents' (\citeyear[\page
139]{Behaghel09a}; \citeyear[\page 86]{Behaghel30})). 

%\largerpage[-1]
Theories that strive to be as restrictive as possible with respect to locality therefore have to develop mechanisms that allow one to only
access information that is required to explain the distribution of constituents.
This is often achieved by projecting certain properties to the mother node of a phrase. In \xbart, the part of speech a head belongs to is
passed up to the maximal projection: if the head is an N, for example, then the maximal projection is an NP. In GPSG, HPSG and variants
of CxG, there are Head Feature Principles responsible for the projection of features. Head Feature Principles ensure that an entire group of
features, so"=called head features, are present on the maximal projection of a head.
Furthermore, every theory has to be capable of representing the fact that a constituent can lack one of its parts and this part is then realized via a long"=distance
dependency in another position in the clause.
As previously discussed on page~\pageref{page-Irish-complementizers}, there are languages in which complementizers inflect depending on whether their
complement is missing a constituent or not. This means that this property must be somehow accessible. In GPSG, HPSG and variants of CxG, there are additional
groups of features that are present at every node between a filler and a gap in a long"=distance dependency.
In LFG, there is f"=structure\is{f"=structure} instead. Using Functional Uncertainty, one can look for the position in the f"=structure where a particular
constituent is missing. In \gbt, movement proceeds cyclically\is{cycle!transformational}, that is, an element is moved into the specifier of CP and can
be moved from there into the next highest CP. It is assumed in \gbt that heads can look inside their arguments, at least they can see the elements in the
specifier position. If complementizers can access the relevant specifier positions, then they can
determine whether something is missing from an embedded phrase or not. In \gbt, there was also an analysis of case
assignment in infinitive constructions in which the case"=assigning verb governs into the embedded
phrase and assigns case to the element in \mbox{SpecIP}. Figure~\ref{Abbildung-ECM} shows the
relevant structure taken from \citew[\page 170]{Haegeman94a-u}.
\begin{figure}[t]
\centering
\scalebox{.98}{%
\begin{forest}
sm edges
[IP
	[NP
		[John]]
	[\hspaceThis{$'$}I$'$
		[I
			[-s]]
		[VP
			[\hspaceThis{$'$}V$'$
				[V
					[believe]]
				[IP
					[NP
						[him]]
					[\hspaceThis{$'$}I$'$
						[I
							[to]]
						[VP
							[\hspaceThis{$'$}V$'$
								[V
									[be]]
								[NP
									[a liar,roof]]]]]]]]]]
\end{forest}
}
\caption{\label{Abbildung-ECM}Analysis of the AcI construction with \emph{Exceptional Case Marking}}
\end{figure}%
Since the Case Principle is formulated in such a way that only finite I can assign case to the subject
(cf.\ page~\pageref{Kasusprinzip-GB}), \emph{him} does not receive case from I. Instead, it is assumed that
the verb \emph{believe} assigns case to the subject of the embedded infinitive.

\addlines[2]
Verbs that can assign case across phrase boundaries are referred to as ECM verbs, where ECM stands for
\emph{Exceptional Case Marking}\is{Exceptional Case Marking (ECM)}. As the name suggests, this instance of case assignment into a phrase was viewed as an
exception. In newer versions of the theory (\eg \citealp[\page 120--123]{Kratzer96a}), all case assignment
is to specifier positions. For example, the Voice\is{category!functional!Voice} head in Figure~\vref{Abbildung-Kratzer} assigns
accusative to the DP in the specifier of VP.
\begin{figure}
\centering
\begin{forest}
sm edges
[VoiceP
	[DP
		[Mittie]]
	[\hspaceThis{$'$}Voice$'$
		[Voice
			[agent]]
		[VP
			[DP
				[the dog,roof]]
			[\hspaceThis{$'$}V$'$
				[V
					[feed]]]]]]
\end{forest}
\caption{\label{Abbildung-Kratzer}Analysis of structures with a transitive verb following Kratzer}
\end{figure}%
Since the Voice head governs into the VP, case assignment to a run"=of"=the"=mill object in this theory
is an instance of exceptional case assignment as well. The same is true in Adger's version of
Minimalism, which was discussed in Chapter~\ref{chap-mp}: \citet{Adger2010a} argues that
his theory is more restrictive than LFG or HPSG since it is only one feature that can be selected by
a head, whereas in LFG and HPSG complex feature bundles are selected. However, the strength of
this kind of locality constraint is weakened by the operation Agree\is{Agree}, which allows for
nonlocal feature checking. As in Kratzer's proposal, case is assigned nonlocally by \littlev to
the object inside the VP (see Section~\ref{sec-case-mp}). 

Adger discusses PP arguments of verbs like \emph{depend} and notes that these verbs need specific
PPs, that is, the form of the preposition in the PP has to be selectable. While this is trivial in
Dependency Grammar, where the preposition is selected right away, the respective information is
projected in theories like HPSG and is then selectable at the PP node. However, this requires that
the governing verb can determine at least two properties of the selected element: its part of speech
and the form of the preposition. This is not possible in Adger's system and he left this for further
research. Of course it would be possible to assume an onP (a phrasal projection of \emph{on} that
has the category `on'). Similar solutions have been proposed in Minimalist theories (see
Section~\ref{sec-functional-projections-minimalism} on functional projections), but such a solution would obviously
miss the generalization that all prepositional phrases have something in common, which would not be
covered in a system with atomic categories that are word specific.


In theories such as LFG\indexlfg and HPSG\indexhpsg, case assignment takes place locally in constructions
such as those in (\mex{1}):

\eal
\ex John believes him to be a liar.
\ex 
\gll Ich halte ihn für einen Lügner.\\
	 I hold him for a.\acc{} liar\\
\glt `I take him to be a liar.'
\ex 
\gll Er scheint ein Lügner zu sein.\\
	 he seems a.\nom{} liar to be\\
\glt `He seems to be a liar.'	 
\ex 
\gll Er fischt den Teich leer.\\
	 he fishes the.\acc{} pond empty\\
\glt `He fishes (in) the pond (until it is) empty.'
\zl
Although \emph{him}, \emph{ihn} `him', \emph{er} `he' and \emph{den Teich} `the pond' are not semantic arguments of the finite verbs, they are syntactic arguments
(they are raised\is{raising}) and can therefore be assigned case locally. See \citew[\page 348--349 and Section~8.2]{Bresnan82c} and \citew[Section~3.5]{ps2}
for an analysis of raising in LFG\indexlfg and HPSG\indexhpsg respectively. See \citew{Meurers99b}, \citew{Prze99}, and
\citew[Section~17.4]{MuellerLehrbuch1} for case assignment in HPSG and for its interaction with raising.

There are various phenomena that are incompatible with strict locality and require the projection of at least some information.
For example, there are question tags\is{question tag} in English\il{English} that must match the subject of the clause with which
they are combined:
\eal
\ex She is very smart, isn't she / * he?
\ex They are very smart, aren't they?
\zl
\citet{BF99a}, \citet{FB2003a} therefore propose making information about agreement or the referential index of the subject\is{subject}
available on the sentence node.\footnote{%
  See also \citew[\page 89]{SP91a-u}.
}
In \citet{Sag2007a}, all information about phonology, syntax and semantics of the subject is represented as the value of a feature \textsc{xarg}\isfeat{xarg} (\textsc{external argument}).
Here, \emph{external argument} does not stand for what it does in \gbt, but should be understood in a more general sense. For example, it makes the possessive pronoun
accessible on the node of the entire NP. \citet{Sag2007a} argues that this is needed to force coreference in English\il{English} idioms\is{idiom|(}:
\eal
\ex He$_i$ lost [his$_i$ / *her$_j$ marbles].
\ex They$_i$ kept/lost [their$_i$ / *our$_j$ cool].
\zl
%\largerpage
\addlines
The use of the \textsc{xarg} feature looks like an exact parallel to accessing the specifier position as we saw in the discussion of GB. However, Sag proposes that complements of prepositions
in Polish\is{Polish} are also made accessible by \textsc{xarg} since there are data suggesting that higher heads can access elements inside PPs \citep[Section~5.4.1.2]{Prze99b}.

In Section~\ref{sec-SbCxG} about Sign-based Construction Grammar, we already saw that a theory that only makes the reference to one
argument available on the highest node of a projection cannot provide an analysis for idioms of the
kind given in (\mex{1}). This is because the subject is made available with verbal heads, however,
it is the object that needs to be accessed in sentences such as (\mex{1}). This means that one has
to be able to formulate constraints affecting larger portions of syntactic structure.
\eal
\ex[]{\label{ex-ich-glaube-mich-tritt-ein-Pferd}
\gll Ich glaube, mich / \# dich tritt ein Pferd.\footnotemark\\
     I   believes me   {} {} you kicks a horse\\
\footnotetext{%
  \citew[\page 311]{RS2009a}.
}
\glt `I am utterly surprised.'
}
\ex[]{
\gll Jonas glaubt, ihn  tritt ein Pferd.\footnotemark\\
     Jonas believes him kicks a horse\\
\footnotetext{%
  \url{http://www.machandel-verlag.de/der-katzenschatz.html}, 2015-07-06.
}
\glt `Jonas is utterly surprised.'
}
\ex[\#]{
\gll Jonas glaubt, dich  tritt ein Pferd.\\
     Jonas believes you kicks a horse\\
\glt `Jonas believes that a horse kicks you.'
}
\zl
Theories of grammar with extended locality domains do not have any problems with this kind of data.\footnote{%
Or more carefully put: they do not have any serious problems since the treatment of idioms in all their
many aspects is by no means trivial \citep{Sailer2000a}.
} An example for this kind of theory is TAG. In TAG, one can specify trees of exactly the right size \citep{Abeille88a,AS89a}.
All the material that is fixed in an idiom is simply determined in the elementary
tree. Figure~\vref{Abbildung-kick-the-bucket-TAG} shows the tree for \emph{kick the bucket} as it is used in (\mex{1}a).
\eal
\ex The cowboys kicked the bucket.
\ex Cowboys often kick the bucket.
\ex He kicked the proverbial bucket.
\zl
\begin{figure}
\centering
\begin{forest}
tag
[S
	[NP$\downarrow$]
	[VP
		[V
			[kicked]]
		[NP
			[D
				[the]]
			[N
				[bucket]]]]]
\end{forest}
\caption{\label{Abbildung-kick-the-bucket-TAG}Elementary tree for \emph{kick the bucket}}
\end{figure}%
%\largerpage[3]
Since TAG trees can be split up by adjunction, it is possible to insert elements between the parts of an idiom as in (\mex{0}b,c) and thus
explain the flexibility of idioms with regard to adjunction and embedding.\footnote{%
	Interestingly, variants of Embodied CxG are strikingly similar to TAG. The Ditransitive
        Construction that was discussed	on page~\pageref{CxG-Active-Ditransitive} allows for additional material to occur between the subject and the verb.
	
	The problems that arise for the semantics construction are also similar. \citet[\page
9]{AS89a} assume that the semantics of \emph{John kicked the proverbial bucket} is computed
from the parts \relation{John}, \relation{kick-the-bucket} and \relation{proverbial}, that is, the added modifiers
always have scope over the entire idiom. This is not adequate for all idioms \citep{FK96a}:
\ea
\gll Er band ihr einen großen Bären auf.\\
	 he tied her a big bear on\\
\glt `He pulled (a lot of) wool over her eyes.'
\z
In the idiom in (i), \emph{Bär} `bear' actually means `lie' and the adjective has to be interpreted accordingly.
The relevant tree should therefore contain nodes that contribute semantic information and also say something
about the composition of these features.

In the same way, when computing the semantics of noun phrases in TAG and Embodied Construction Grammar, one should bear in mind that the adjective
that is combined with a discontinuous NP Construction (see page~\pageref{CxG-DetNoun}) or an NP tree can have narrow scope over the noun
(\emph{all alleged murderers}).
} Depending on whether the lexical rules for the passive\is{passive} and long"=distance dependencies can be applied, the idiom can occur
in the relevant variants.

In cases where the entire idiom or parts of the idiom are fixed, it is possible to rule out adjunction to the nodes of the idiom
tree. Figure~\vref{Abbildung-take-into-account-TAG} shows a pertinent example from
\citet[\page 7]{AS89a}. The ban on adjunction\is{adjunction!ban} is marked by a subscript NA.
\begin{figure}
\centering
\begin{forest}
tag
[S
	[NP$\downarrow$]
	[VP
		[V
			[takes]]
		[NP$\downarrow$]
		[PP$_{{\mathrm{NA}}}$
			[P
				[into]]
			[NP$_{\mathrm{NA}}$
				[N$_{\mathrm{NA}}$
					[account]]]]]]
\end{forest}
\caption{\label{Abbildung-take-into-account-TAG}Elementary tree for \emph{take into account}}
\end{figure}%

The question that also arises for other theories is whether the efforts that have been made to enforce locality should be abandoned altogether.
In our box model in Section~\ref{Abschnitt-Kopf}, this would mean that all boxes were transparent. Since plastic boxes do not allow
all of the light through, objects contained in multiple boxes cannot be seen as clearly as those in the topmost box (the path
of Functional Uncertainty\is{functional uncertainty} is longer). This is parallel to a suggestion made by
\citet{KF99a} in CxG\indexcxg. Kay and Fillmore explicitly represent all the information about the internal structure of a phrase on the mother
node and therefore have no locality restrictions at all in their theory. In principle, one can
motivate this kind of theory in parallel to the argumentation in Chapter~\ref{sec-generative-capacity}. The argument
there made reference to the complexity of the grammatical formalism: the kind of complexity that the
language of description has is unimportant, it is only important what one does with it. In the same way, one can say that regardless of what kind of information
is  in principle accessible, it is not accessed if this is not permitted. This was the approach taken by \citet[\page
143--145]{ps}.

It\label{page-Bender-Wambaya-two} is also possible to assume a world in which all the boxes contain transparent areas where it is possible to see parts of their contents.
This is more or less the LFG world\indexlfg: the information about all levels of embedding contained in the f"=structure\is{f"=structure} is
visible to both the inside and the outside. We have already discussed Nordlinger's \citeyearpar{Nordlinger98a-u} LFG analysis of Wambaya\il{Wambaya} 
on page~\pageref{Seite-Bender-Wambaya}.
In Wambaya, words that form part of a noun phrase can be distributed throughout the clause. For example, an adjective that refers to a noun
can occur in a separate position from it. Nordlinger models this by assuming that an adjective can make reference to an argument in the f"=structure
and then agrees with it in terms of case, number and gender. \citet{Bender2008a} has shown that this analysis can be transferred to HPSG\indexhpsg:
instead of no longer representing an argument on the mother node after it has been combined with a head, simply marking the argument as realized
allows us to keep it in the representation (\citealp{Meurers99b}; \citealp{Prze99};
\citealp[Section~17.4]{MuellerLehrbuch1}). Detmar Meurers\ia{Meurers, Walt Detmar} compares both of these HPSG approaches to different ways of working through a shopping list: in the standard approach taken by \citet{ps2},
one tears away parts of the shopping list once the relevant item has been found. In the other case, the relevant item on the list is crossed out.
At the end of the shopping trip, one ends up with a list of what has been bought as well as the items themselves.
  
I have proposed the crossing"=out analysis for depictive predicates\is{depictive predicate|(} in German and English 
\citep{Mueller2004c,Mueller2008a}. Depictive predicates say something about the state of a person or object during the event
expressed by a verb:
\eal
\ex 
\gll Er sah sie nackt.\footnotemark\\
	 he saw her naked\\
\footnotetext{%
  \citew[\page 94]{Haider85b}.
}
\ex He saw her naked.
\zl
In (\mex{0}), the depictive adjective can either refer to the subject or the object. However, there is a strong preference for readings where
the antecedent noun precedes the depictive predicate
\citep[\page 208]{Loetscher85a}. Figure~\vref{anal-er-die-frau-nackt-sieht} shows analyses for the
sentences in (\mex{1}):
\eal
\ex 
\gll dass er$_i$ die Äpfel$_j$ ungewaschen$_{i/j}$ isst\\
	 that he the apples unwashed eats\\
\glt `that he eats the apples unwashed'
\ex 
\gll dass er$_i$ ungewaschen$_{i/*j}$ die Äpfel$_j$ isst\\
	 that he unwashed the apples eats\\
\glt `that he eats the apples (while he is) unwashed'
\zl
\begin{figure}
%\hfill
\includegraphics[width=\textwidth]{Figures/depictives-lsp-crop.pdf}
%%  \resizebox{\linewidth}{!}{%
%% \begin{forest}
%% sm edges, for tree={l sep= 6ex}
%% [V{[\comps \sliste{ \spirit{1}, \spirit{2} }]}
%% 	[\ibox{1} NP{[\textit{nom}]}
%% 		[er;he]]
%% 	[V{[\comps \sliste{ \ibox{1}, \spirit{2} } ]}
%% 		[\ibox{2} NP{[\textit{acc}]}
%% 			[die Äpfel;the apples,roof]]
%% 		[V{[\comps \sliste{ \ibox{1}, \ibox{2} } ]}
%% 			[\subnode{ap1}{AP}
%% 				[ungewaschen;unwashed]]
%% 			[V{[\comps \sliste{ \subnode{arg11}{\ibox{1}}, \subnode{arg12}{\ibox{2}} }]}
%% 				[isst;eats]]]]]
%% \end{forest}
%% \hspace{1em}
%% \begin{forest}
%% sm edges, for tree={l sep= 6ex}
%% [V{[\comps \sliste{ \spirit{1}, \spirit{2} } ]}
%% 	[\ibox{1} NP{[\textit{nom}]}
%% 		[er;he]]
%% 	[V{[\comps \sliste{ \ibox{1}, \spirit{2} } ]}
%% 		[\subnode{ap2}{AP}
%% 			[ungewaschen;unwashed]]
%% 		[V{[\comps \sliste{ \subnode{arg21}{\ibox{1}}, \spirit{2} } ]}
%% 			[\ibox{2} NP{[\textit{acc}]}
%% 				[die Äpfel;the apples,roof]]
%% 			[V{[\comps \sliste{ \ibox{1}, \ibox{2} } ]}
%% 				[isst;eats]]]]]
%% \end{forest}
%% % This has to be inside of the scaling
%%     %% \begin{tikzpicture}[overlay,remember picture]
%%     %% %% this works with tikzmark
%%     %% \draw[->, bend angle=40, bend left] ($(pic cs:ap1)+(1ex,2ex)$) to($(pic cs:arg11)+(1ex,2.5ex)$);
%%     %% \draw[->, bend angle=40, bend left] ($(pic cs:ap1)+(1ex,2ex)$) to($(pic cs:arg12)+(1ex,2.5ex)$); % 1ex links, 2ex hoch
%%     %% %
%%     %% \draw[->, bend angle=40, bend left] ($(pic cs:ap2)+(1ex,2ex)$) to($(pic cs:arg21)+(1ex,2.5ex)$);
%%     %% \end{tikzpicture}
%% % somehow it stopped working
%% %% this used to work with subnode in texlive 2013 but is broken now
%% \begin{tikzpicture}[overlay,remember picture] 
%% \draw[->, bend angle=40, bend left] (ap1.north) to (arg11.north);
%% \draw[->, bend angle=40, bend left] (ap1.north) to (arg12.north); 
%% %
%% \draw[->, bend angle=40, bend left] (ap2.north) to (arg21.north);
%% \end{tikzpicture}
%}
%\hfill\mbox{}
\caption{Analysis of \emph{dass er die Äpfel ungewaschen isst} `that he the apples unwashed eats' and \emph{dass er ungewaschen die
    Äpfel isst} `that he unwashed the apples eat'}\label{anal-er-die-frau-nackt-sieht}
\end{figure}%
Arguments that have been realized are still represented on the upper nodes, however, they are crossed"=out and thereby marked as ``realized''.
In German, this preference for the antecedent noun can be captured by assuming a restriction that states that the antecedent noun must not yet have been
realized.

It is commonly assumed for English\il{English} that adjuncts are combined with a VP.
\eal
\ex John [[\sub{VP} ate the apples$_i$] unwashed$_i$].
\ex You can't [[\sub{VP} give them$_i$ injections] unconscious$_i$].\footnote{%
\citew[\page 17]{Simpson2003a}.
}
\zl
In approaches where the arguments of the verb are accessible at the VP node, it is possible to establish a relation between
the depictive predicate and an argument although the antecedent noun is inside the VP.
English differs from German in that depictives can refer to both realized (\emph{them} in (\mex{0}b))
and unrealized (\emph{you} in (\mex{0}b)) arguments.

\citet[\page 560]{Higginbotham85a} and \citet{Winkler97a} have proposed corresponding non"=cancellation approaches in \gbt.
There are also parallel suggestions in Minimalist theories: checked features are not deleted, but instead marked as already
checked \citep[\page 14]{Stabler2010b}. However, these features are still viewed as inaccessible.\label{page-non-cancellation-end}

Depending on how detailed the projected information is, it can be possible to see adjuncts and argument in embedded structures as well as their
phonological, syntactic and semantic properties. In the CxG variant proposed by Kay and Fillmore, all information is available. In LFG,
information about grammatical function, case and similar properties is accessible. However, the part of speech is not contained in the f"=structure.
If the part of speech does not stand in a one"=to"=one relation to grammatical function, it cannot be restricted using selection via f"=structure.
Nor is phonological information represented completely in the f"=structure. If the analysis of idioms requires nonlocal access to phonological
information or part of speech, then this has to be explicitly encoded in the f"=structure (see \citew[\page 46--50]{Bresnan82a} for more on idioms). 

In the HPSG variant that I adopt, only information about arguments is projected. Since arguments are always represented by descriptions of type
\type{synsem}, no information about their phonological realization is present. However, there are daughters in the structure so that it is still
possible to formulate restrictions for idioms as in TAG or Construction Grammar (see \citew{RS2009a}
for an analysis of the `horse' example in (\ref{ex-ich-glaube-mich-tritt-ein-Pferd})).
This may seem somewhat like overkill: although we already have the tree structure, we are still projecting information about arguments
that have already been realized (unfortunately these also contain information about their arguments and so on). At this point, one could be inclined
to prefer TAG or LFG since these theories only make use of one extension of locality: TAG uses trees
of arbitrary or rather exactly the necessary size and LFG makes reference to a complete
f"=structure. However, things are not quite that simple: if one wants to create a relation to an
argument when adjoining a depictive predicate in TAG, then one requires a list of
possible antecedents. Syntactic factors (\eg reference to dative vs.\ accusative noun phrases, to argument vs.\ adjuncts,
coordination of verbs vs.\ nouns) play a role in determining the referent noun, this cannot be reduced to semantic relations.
Similarly, there are considerably different restrictions for different kinds of idioms and these cannot all be formulated in terms of restrictions
on f"=structure since f"=structure does not contain information about parts of speech.\is{depictive predicate|)}

One should bear in mind that some phenomena require reference to larger portions of structure. The majority of phenomena can be treated in terms of head
domains and extended head domains, however, there are idioms that go beyond the sentence level. Every theory has to account for this somehow.
\is{locality|)}\is{idiom|)}



%      <!-- Local IspellDict: en_US-w_accents -->

%% -*- coding:utf-8 -*-
\section{递归性}
%\section{Recursion}
\label{sec-recursion}

正如\isce[|(]{递归}{recursion}本书第\pageref{ex-that-max-thinks-that-recursion}页所述,本书中所有的理论都可以解决语言中的自我嵌套问题。例(\ref{ex-that-max-thinks-that-recursion})在这里重复为例(\mex{1}):
%Every\is{recursion|(} theory in this book can deal with self-embedding in language as it was
%discussed on page~\pageref{ex-that-max-thinks-that-recursion}. The example
%(\ref{ex-that-max-thinks-that-recursion}) is repeated here as (\mex{1}):
\ea
\label{ex-that-max-thinks-that-recursion-two}
\gll that Max thinks [that Julia knows [that Otto claims [that Karl suspects [that Richard confirms [that Friederike is laughing]]]]]\\
	\textsc{comp} Max 认为 \spacebr\textsc{comp} Julia 知道 \spacebr\textsc{comp} Otto 声称 \spacebr\textsc{comp} Karl 怀疑 \spacebr\textsc{comp} Richard 确认 \spacebr\textsc{comp} Friederike \textsc{aux} 笑\\
\mytrans{Max认为Julia知道Otto声称Karl怀疑Richard确认Friederike正在笑}
%that Max thinks [that Julia knows [that Otto claims [that Karl
%suspects [that Richard confirms [that Friederike is laughing]]]]]
\z
大部分理论通过嵌套短语结构规则或者统制图式来直接描述这一递归性。但是TAG\indextag 在处理递归性方面是特殊的,因为递归性被排除出了句法树。对应的效应是通过附加操作完成的,这种附加操作允许任意数量的成分插入到句法树中。有时会说构式语法\indexcxg 不能描述自然语言中存在的递归性(\egc \citealp[\page 269]{Leiss2009a})。对构式语法有这样的印象是可以理解的,因为很多分析都是表层导向的。例如,有人会经常谈到[Sbj TrVerb Obj]构式。但是,我们正谈论的构式只要包含句子嵌套或关系小句构式就会变得可以描述递归性了。一个句子嵌套构式可以有以下形式[Sbj that-Verb that-S],其中that-动词可以带句子型补足语,that-S代表相应的补语。that-小句就可以插入到that-S槽中。因为这个that小句也可以是使用这一构式的结果,所以语法也可以产生例(\mex{1})所示的句子:
%Most theories
%capture this directly with recursive phrase structure rules or dominance schemata. TAG\indextag is
%special with regard to recursion since recursion is factored out of the trees. The corresponding
%effects are created by an adjunction operation that allows any amount of material to be inserted
%into trees.  It is sometimes claimed that Construction Grammar\indexcxg cannot capture the existence
%of recursive structure in natural language (\eg \citealp[\page 269]{Leiss2009a}).  This impression
%is understandable since many analyses are extremely surface-oriented. For example, one often talks
%of a [Sbj TrVerb Obj] construction. However, the grammars in question also become recursive as soon
%as they contain a sentence embedding or relative clause construction. A sentence embedding
%construction could have the form [Sbj that-Verb that-S], where a that-Verb is one that can take
%a sentential complement and that-S stands for the respective complement. A \emph{that}-clause can then be inserted
%into the that-S slot. Since this \emph{that}-clause can also be the result of the application of
%this construction, the grammar is able to produce recursive structures such as those in (\mex{1}):

\ea
\gll Otto claims [\sub{that-S} that Karl suspects [\sub{that-S} that Richard sleeps]].\\
	Otto 声称 {} \textsc{comp} Karl 怀疑 {} \textsc{comp} Richard 睡觉\\
\mytrans{Otto声称Karl怀疑Richard睡觉。}
%Otto claims [\sub{that-S} that Karl suspects [\sub{that-S} that Richard sleeps]].
\z
在(\mex{0})中,Karl suspects that Richard sleeps和整个句子都是[Sbj that-Verb that-S]构式的实例。整个句子因此包含一个嵌套的子部分,这一子部分也被同样的构式允准。例 (\mex{0})也包含一个that-S范畴的成分,该成分嵌套在that-S中。关于构式语法中递归和自嵌套\isce{自嵌套}{self-embedding}的更多信息,可以参见 \citew{Verhagen2010a}。
%In (\mex{0}), both \emph{Karl suspects that Richard sleeps} and the entire clause are instances of the [Sbj
%that-Verb that-S] construction. The entire clause therefore contains an embedded subpart that is licensed by
%the same construction as the clause itself. (\mex{0}) also contains a constituent of the category
%\emph{that}-S that is embedded inside of \emph{that}-S. For more on recursion and self-embedding\is{self-embedding} in Construction Grammar, see  \citew{Verhagen2010a}.

与之相似,每一个允许名词与一个属格\iscesub{格}{case}{属格}{genitive}名词短语组合的构式语法也允许递归结构。相关构式可以有[DetNNP[gen]]或[NNP[gen]]形式。[DetNNP[gen]]构式允准例(\mex{1})所示的例子:
%Similarly, every Construction Grammar that allows a noun to combine with a genitive\is{genitive} noun phrase also allows
%for recursive structures. The construction in question could have the form [Det N
%NP[gen] ] or [ N NP[gen] ]. The [Det N NP[gen] ] construction licenses structures such as (\mex{1}):
\ea
\gll [\sub{NP} des Kragens [\sub{NP} des Mantels [\sub{NP} der Vorsitzenden]]]\\
	{} \defart{} 衣领 {} \defart{} 大衣 {} \defart{} 女主席\\
\mytrans{这位女主席的大衣的衣领}
%\gll [\sub{NP} des Kragens [\sub{NP} des Mantels [\sub{NP} der Vorsitzenden]]]\\
%	{} the collar {} of.the coat {} of.the chairwoman\\
%\mytrans{the collar of the coat of the chairwoman}
\z
 \citet{Jurafsky96a}和 \citet*{BLT2009a}使用概率上下文无关文法\iscesub{上下文无关文法}{context-free grammar}{概率上下文无关文法(PCFG)}{probabilistic (PCFG)} (PCFG)来构建一个聚焦于心理语言学可行性和习得模拟的构式语法分析器。上下文无关文法处理例(\mex{0}) 所示的自我嵌套\isce{自嵌套}{self-embedding}结构时没有问题,因此这类构式语法在处理自我嵌套时不会遇到任何问题。
% \citet{Jurafsky96a} and  \citet*{BLT2009a} use probabilistic context-free grammars\is{context-free grammar!probabilistic (PCFG)} (PCFG) for a Construction Grammar parser
%with a focus on psycholinguistic plausibility and modeling of acquisition. Context-free grammars
%have no problems with self-embedding\is{self-embedding} structures like those in (\mex{0}) and thus this kind
%of Construction Grammar itself does not encounter any problems with self-embedding.

 \citet[\page 192]{Goldberg95a}认为英语\ilce{英语}{English}的动结构式\iscesub{构式}{construction}{动结}{resultative}有以下形式:
% \citet[\page 192]{Goldberg95a} assumes that the resultative construction\is{construction!resultative} for English\il{English} has the following
%form:
\ea
{}[SUBJ [V OBJ OBL]] 
\z
这对应着TAG中基本树的复杂结构。LTAG与Goldberg的方法的差异在于每一个结构都需要一个词汇锚位,也就是说,例 (\mex{0})在LTAG中动词应该是固定的。但是在Goldberg的分析中,动词可以独立插入存在的构式中(见\ref{Abschnitt-Stoepselei})。在TAG的相关文献中,经常会强调初级树不包括任何递归。但是整个语法是递归的,因为其他成分可以通过附加插入到句法树中⸺正如例(\mex{-2})和(\mex{-1}) 所示⸺插入到替换项结点也可以产生递归结构。
\isce[|)]{递归}{recursion}
%This corresponds to a complex structure as assumed for elementary trees in TAG. LTAG differs from Goldberg's approach in that every structure requires a lexical
%anchor, that is, for example (\mex{0}), the verb would have to be fixed in LTAG. But in Goldberg's analysis, verbs can be inserted into independently
%existing constructions (see Section~\ref{Abschnitt-Stoepselei}). In TAG publications, it is often emphasized that elementary trees do not contain any recursion.
%The entire grammar is recursive however, since additional elements can be added to the tree using adjunction and -- as (\mex{-2}) and
%(\mex{-1}) show -- insertion into substitution nodes can also create recursive structures.
%\is{recursion|)}



%      <!-- Local IspellDict: en_US-w_accents -->

