\section*{语言科学出版社:归学者所有的高质量语言学出版物}
%\section*{Language Science Press: scholar-owned high quality linguistic books}

%在2012年,有一群人发现出版界的情况令人难以容忍,他们一致认为有必要在公开平台上出版语言学书籍。也就是说,需要一个针对所有读者和作者公开的平台。我建立了一个网页,并征集了支持者,他们是来自%全世界各地的著名语言学家,Martin Haspelmath和我随后就成立了语言科学出版社。几乎同时,DFG公布了一项公开专著的项目,我们申请\citep{MH2013a}并获得了资助(18个申请中只有两家获得了资助)。这%笔钱支付给一位主任(Dr.\ Sebastian Nordhoff)、一位经济学家(Debora Siller)和两位程序员(Carola Fanselow和Dr.\ Mathias Schenner)。他们在公开专著出版社(OMP)出版平台工作,并应用转换软件来%从我们的\LaTeX{}编码中生成不同的格式(ePub、XML、HTML)。Svantje Lilienthal负责OMP的文档,制作屏幕录像,并为作者、读者和编辑提供用户支持。
%In 2012 a group of people found the situation in the publishing business so unbearable that they
%agreed that it would be worthwhile to start a bigger initiative for publishing linguistics books in
%platinum open access, that is, free for both readers and authors. I set up a web page and collected
%supporters, very prominent linguists from all over the world and all subdisciplines and Martin
%Haspelmath and I then founded Language Science Press. At about the same time the DFG had announced
%a program for open access monographs and we applied \citep{MH2013a} and got funded (two out of 18 applications got
%funding). The money is used for a coordinator (Dr.\ Sebastian Nordhoff) and an economist (Debora
%Siller), two programmers (Carola Fanselow and Dr.\ Mathias Schenner), who work on the publishing
%plattform Open Monograph Press (OMP) and on conversion software that produces various formats (ePub, XML,
%HTML) from our \LaTeX{} code. Svantje Lilienthal works on the documentation of OMP, produces
%screencasts and does user support for authors, readers and series editors.

%OMP在公开评论方面和社区建设的游戏化工具方面进行了扩展。所有语言科学出版社出版的图书都至少由两位外部审稿人审稿。审稿人和作者可同意出版这些审稿意见,并使得整个过程更为透明(也可以参见 %\citew{Pullum84a}关于期刊文章的公开评论的建议)。另外,还有可选的第二轮评审过程:公开评审。这一阶段对所有人都是公开的。整个社团都可以评论语言科学出版社出版的书籍。在第二轮评审阶段后,这通常需要%持续两个月的时间,作者会进行修订,进而出版出改进的版本。这本书是经历了这个公开评审阶段的第一本书。标注了公开评审意见的版本可以通过\url{http://langsci-press.org/catalog/book/177}获得。距离本书第一%版发表的两年时间中,该书大约有15000次的下载,并在全世界范围内用于教学与研究。这是每一位作者,也是每一位教师的愿景:将知识传播给每一个人。
%OMP is extended by open review facilities and community-building gamification tools
%\citep{MuellerOA,MH2013a}. All Language Science Press books are reviewed by at least two external
%reviewers. Reviewers and authors may agree to publish these reviews and thereby make the whole
%process more transparent (see also  \citew{Pullum84a} for the suggestion of open reviewing of journal
%articles). In addition there is an optional second review phase: the open
%review. This review is completely open to everybody. The whole community may comment on the document
%that is published by Language Science Press. After this second review phase, which usually lasts for
%two months, authors may revise their publication and an improved version will be published. This
%book was the first book to go through this open review phase. The annotated open review version of this book is still available via
%the \href{\lsURL}{web page of this book}. 

%距离本书第一版发表的两年时间中,该书大约有15000次的下载,并在全世界范围内用于教学与研究。%这是每一位作者,也是每一位教师的愿景:将知识传播给每一个人。语言科学出版社也有了长足的发%展,最新情况是\footnote{详细信息和图表请参考http://userblogs.fu-berlin.de/langsci-press/2018/01/18/achievements-2017/
%}:我们有324本书的作者对出版社表示了极大的兴趣,并且有58本书已经出版。这些书按照20个系列%陆续出版,编委由二百六十三名来自六大洲四十四个国家的学者组成。这些书共有175000次的下载。来%自全世界的一百三十八位学者参与了审校工作。目前,共有296名审校者在语言科学出版社的网站上%注册。%语言科学出版社是一个基于社会团体的出版社,但是仍有专人管理,他是Sebatian Nordhoff。他的职位是有偿的。
%我们已经成功得到了将近一百家学术组织的资助,包括哈佛大学、麻省理工学院和伯克利\footnote{资%助信息列表详见http://langsci-press.org/knowledgeunlatched
%}。如果你想支持我们,可以通过注册、出版、审校或者说服图书馆等组织来资助等方式帮助我们,详%情请见\url{http://langsci-press.org/supportUs}。
%The  first edition of this book was published almost exactly two years ago.  e book has app. 15,000 %downloads and is used for teaching and in research all over the world.  is is what every author and %every teacher dreams of: distribution of knowledge and accessibility for everybody.  e foreword of the  %rst edition ends with a description of Language Science Press in 2016.  is is the situation now:3 We %have 324 expressions of interest and 58 published books. Books are published in 20 book series with %263 members of editorial boards from 44 di erent countries from six continents. We have a total of %175,000 downloads. 138 linguists from all over the world have participated in proofreading.  ere are %currently 296 proofreaders registered with Language Science Press. Language Science Press is a %community-based publisher, but there is one person who manages everything: Sebastian Nordhoff. His position has to be paid. We were successful in acquiring  nancial support by almost 100 academic %institutions including Harvard, the MIT, and Berkeley.4 If you want to support us by just signing the list %of supporters, by publishing with us, by helping as proofreader or by convincing your librarian/institution to support Language Science Press financially, please refer to http: //langsci-press.org/supportUs.

%如今,语言科学出版社拥有20个语言学不同领域的系列书籍,这些高水平的编委成员由263名来自6大洲44个国家的学者组成。我们有58本已经出版的书籍,还有324本书的作者对出版社表示出了极大的兴趣。这些书共%有175000次的下载。系列书籍的编委和作者主要用\LaTeX{}编辑的手稿,但是他们也有由语言科学出版社建立的基于网络的格式模版以及社区里的志愿者的支持。审校也是基于社群的。来自全世界的138位学者参与%了审校工作。目前,共有296名审校者在语言科学出版社的网站上注册。他们的工作被记录在名人堂中:\url{https://langsci-press.org/hallOfFame}。

%Currently, Language Science Press has 26 series on various subfields of linguistics with high
%profile series editors from all continents. We have 127 published and 25 forthcoming books and 572
%expressions of interest. The number of downloads (excluding access by robots) will reach 1 Mio within the next days. Series editors  and authors are responsible for
%delivering manuscripts that are typeset in \LaTeX{}, but they are supported by a web-based typesetting
%infrastructure that was set up by Language Science Press and by volunteer typesetters from the
%community. Proofreading is also community-based. Until now 224 people helped improving our
%books. Their work is documented in the Hall of Fame: \url{http://langsci-press.org/about/hallOfFame}.

%如果你认为想阅读这类教科书的人都应得以免费拥有这些书,而且科学研究的出版不应落入利益驱动的出版社手中,那么你就应该加入语言科学出版社的群%体,并在以下几个方面支持我们:可以在语言科学出版社上注册,将你的名字列在其他将近600名热心学者之中,还可以帮助校对或者修改格式,或者可%以给某本书或者向语言科学出版社捐钱。我们也在寻求机构的支持,像基金会、社团、语言学系或大学图书馆。有关资助的详细信息请参见网页:%\url{https://langsci-press.org/supporters}。
%我们已经成功得到了将近一百家学术组织的资助,包括哈佛大学、麻省理工学院和加州大学伯克利分校\footnote{资助信息列表详见http://langsci-press.org/%knowledgeunlatched
%}。如果你想支持我们,可以通过注册、出版、审校或者说服图书馆等机构来资助等方式帮助我们,详情请见\url{http://langsci-press.org/supportUs}。
%如有问题,请联系我或者语言科学出版社的主任 \href{mailto:contact@langsci-press.org}{contact@langsci-press.org}。
%If you think that textbooks like this one should be freely available to whoever wants to read them
%and that publishing scientific results should not be left to profit-oriented publishers, then you
%can join the Language Science Press community and support us in various ways: you can register with Language Science Press and have your name
%listed on our supporter page with almost 600 other enthusiasts, you may devote your time and help
%with proofreading and/or typesetting, or you may donate money for specific books or for Language
%Science Press in general. We are also looking for institutional supporters like foundations,
%societies, linguistics departments or university libraries. Detailed information on how to support
%us is provided at the following webpage: \url{http://langsci-press.org/about/support}.
%In case of questions, please contact me or the Language Science Press coordinator at \href{mailto:contact@langsci-press.org}{contact@langsci-press.org}.


% ~\medskip

% %\noindent
% \begin{flushright}
% \begin{tabular}{c}
% %斯特凡 $\cdot$ 米勒\\
% Stefan Müller\\
% 柏林\\
% 2018年3月28日\\
% \end{tabular}
% \end{flushright}
% %Berlin, \today\hfill Stefan Müller

%      <!-- Local IspellDict: en_US-w_accents -->


%New text 2020:

在2012年,有一群人发现出版界的情况令人难以容忍,他们一致认为有必要在公开平台上出版语言学书籍。也就是说,需要一个针对所有读者和作者公开的平台。我建立了一个网页,并征集了支持者,他们是来自全世界各地的著名语言学家,Martin Haspelmath和我随后就成立了语言科学出版社。几乎同时,DFG公布了一项公开专著的项目,我们申请\citep{MH2013a}并获得了资助(18个申请中只有两家获得了资助)。这笔钱支付给一位主任(Dr.\ Sebastian Nordhoff)、一位经济学家(Debora Siller)和两位程序员(Carola Fanselow和Dr.\ Mathias Schenner)。他们在公开专著出版社(OMP)出版平台工作,并应用转换软件来从我们的\LaTeX{}编码中生成不同的格式(ePub、XML、HTML)。Svantje Lilienthal负责OMP的文档,制作屏幕录像,并为作者、读者和编辑提供用户支持。
%In 2012 a group of people found the situation in the publishing business so unbearable that they
%agreed that it would be worthwhile to start a bigger initiative for publishing linguistics books in
%platinum open access, that is, free for both readers and authors. I set up a web page and collected
%supporters, very prominent linguists from all over the world and all subdisciplines and Martin
%Haspelmath and I then founded Language Science Press. At about the same time the DFG had announced
%a program for open access monographs and we applied \citep{MH2013a} and got funded (two out of 18 applications got
%funding). The money was used for a coordinator (Dr.\ Sebastian Nordhoff) and an economist (Debora
%Siller), two programmers (Carola Fanselow and Dr.\ Mathias Schenner), who worked on the publishing
%plattform Open Monograph Press (OMP) and on conversion software that produces various formats (ePub, XML,
%HTML) from our \LaTeX{} code. Svantje Lilienthal worked on the documentation of OMP, produced
%screencasts and did user support for authors, readers and series editors.

\addlines
OMP在公开评论方面和社区建设的游戏化工具方面进行了扩展\citep{MuellerOA,MH2013a}。所有语言科学出版社出版的图书都至少由两位外部审稿人审稿。审稿人和作者可同意出版这些审稿意见,并使得整个过程更为透明(也可以参见 \citew{Pullum84a}关于期刊文章的公开评论的建议)。另外,还有可选的第二轮评审过程:公开评审(参见Sebastian Nordhoff发布的有关语言科学出版社的评审观点的博客\footnote{%
\url{https://userblogs.fu-berlin.de/langsci-press/2015/05/27/axes-of-open-review/},2020年9月3日。
})。这一阶段对所有人都是公开的。整个社团都可以评论语言科学出版社出版的书籍。在第二轮评审阶段后,这通常需要持续两个月的时间,作者会进行修订,然后出版改进的版本。这本书的英译版是经历了这个公开评审阶段的第一本书。本书中文版也在Paperhive上经过了公开评审阶段。他们提出了2500多条意见\footnote{%
\url{https://paperhive.org/documents/items/Zf2Qf47i6nf2},2020年9月3日。},这些意见都被自动地输入到语言科学出版社的版本控制和错误跟踪系统中\footnote{%
\url{https://github.com/langsci/177/},2020年9月3日。
}。
%OMP was extended by open review facilities and community-building gamification tools
%\citep{MuellerOA,MH2013a}. All Language Science Press books are reviewed by at least two external
%reviewers. Reviewers and authors may agree to publish these reviews and thereby make the whole
%process more transparent (see also \citew{Pullum84a} for the suggestion of open reviewing of journal
%articles). In addition there is an optional second review phase: the open
%review (see the blog posts by Sebastian Nordhoff about the reviewing options at Language Science
%Press\footnote{%
%\url{https://userblogs.fu-berlin.de/langsci-press/2015/05/27/axes-of-open-review/}, 2020-09-03.
%}). This second optional reviewing phase is completely open to everybody. The whole community may comment on the document
%that is published by Language Science Press. After this second review phase, which usually lasts for
%two months, authors may revise their publication and an improved version will be published. The
%English version of this book was the first book to go through this open review phase. The Chinese
%translation was also open for comments on Paperhive. Readers left more than 2500 comments\footnote{%
%\url{https://paperhive.org/documents/items/Zf2Qf47i6nf2}, 2020-09-03.}, which were automatically fed into the version control
%and bug tracking system used by Language Science Press\footnote{%
%\url{https://github.com/langsci/177/}, 2020-09-03.
%}.

目前,语言科学出版社拥有26个语言学不同领域的系列书籍,这些高水平的编委成员由437名来自49个国家的学者组成。我们有134本已经出版的书籍,这些书已经有超过一百万次的下载。\dotfootnote{%
排除了机器人下载的情况,本书的英文版自2016年以来已经有超过4万次的下载。
} 截至2020年3月,有来自53个国家的1196位作者在语言科学出版社出版了书籍或章节,还有572位作者表达了意向。
%Currently, Language Science Press has 26 series on various subfields of linguistics with high
%profile series editors from all continents. There are 437 members in the respective editorial boards
%coming from 49 countries. We have 134 published books with more than 1 Mio downloads.\footnote{%
%Downloads by robots excluded, the English version of this textbook was downloaded over 40,000 times
%since 2016.
%} 1196 authors from 53 countries have published books or chapters with Language Science Press as of March 2020 and there are
%572 expressions of interest. 
%%Two multi-volume handbooks, one on HPSG and one on LFG, are in
%%preparation \citep{HPSGHandbook,LFGHandbook}.


系列书籍的编辑负责提交用\LaTeX{}排版的手稿,但是他们也有由语言科学出版社建立的基于网络的排版基础设备的支持,也有将Word手稿转换成\LaTeX{}的软件。审校也是基于社群的。截至目前,共有224位人士帮助我们改进了书稿。他们的工作被记录在名人堂中:\url{https://langsci-press.org/hallOfFame}。
%Series editors are responsible for delivering manuscripts that are typeset in \LaTeX{}, but they are
%supported by a web-based typesetting infrastructure that was set up by Language Science Press and
%there is also conversion software converting Word manuscripts into \LaTeX{}. Proofreading is
%community-based. Until now 224 people helped improve our books. Their work is documented in the
%Hall of Fame: \url{http://langsci-press.org/hallOfFame}.


语言科学出版社是一家基于社群的出版社,但是除了出版经理Martin Haspelmath和我,还有两位员工受雇于统筹和排版工作:Sebastian Nordhoff,他也是出版经理,还有Felix Kopecky,他负责排版工作。两人各司其职。在2018至2020年间,这两个职位得到了115家学术机构的资助,包括哈佛、麻省理工和伯克利,还有学会,如欧洲二语习得学会。\dotfootnote{%
 资助机构的完整列表请见: 
  \url{http://langsci-press.org/knowledgeunlatched}.
} 语言科学出版社的方法也得到著名学者Noam Chomsky、Adele Goldberg和Steven Pinker的认可,他们在2017年来信表示支持。\dotfootnote{%
“很高兴知道这么好的建议,这是将学者的研究成果见诸于世的一条极为有价值的途径。老生常谈的是,也是事实,我们都是站在巨人的肩膀上,并且依靠着前辈们为每一个人积累的文化财富。公众应该获得当代学术界所能贡献的一切,而且这里列出的想法看起来是实现这一理想的一条非常有希望的途径。”Noam Chomsky,2017年2月1日。

“语言科学出版社正在为经过仔细审查的文章和书籍制定一个可以自由获取的标准。”Adele Goldberg,2017年5月2日。

“共享数据和方法是学术研究的支柱之一。由学者创造的知识属于每一个人,公开出版是实现这个理念的康庄大道。语言科学出版社,跟知识解锁(Knowledge Unlatched)一起,为我们向全球公众共享我们的发现提供了一个很好的途径。”Steven Pinker,2017年1月22日。
} 2023年的筹款活动正在进行中。
%Language Science Press is a community"=based publisher, but apart from the press managers Martin
%Haspelmath and me, there are two people who are employed for the central organization and
%typesetting: Sebastian Nordhoff, who is also a press manager, and Felix Kopecky, who does 
%typesetting. Both have 50\,\% positions. In the period of 2018--2020, these two positions got payed
%with the help of financial support by 115 academic institutions including  
%Harvard, the MIT, and Berkeley and by societies like EuroSLA.\footnote{%
%  A full list of supporting institutions is available at:
%  \url{http://langsci-press.org/knowledgeunlatched}.
%} The Language Science Press approach is endorsed by the leading scholars Noam Chomsky, Adele
%Goldberg, and Steven Pinker, who sent letters of support in 2017.\footnote{%
%``Very pleased to learn about this fine initiative, a most valuable way to
%bring to the general public the results of scholarly work.  It's a
%cliché, but true, that we all stand on the shoulders of giants, and rely
%on the cultural wealth provided to everyone by past generations.  It is
%only proper that the public should gain access to whatever contemporary
%scholarship can contribute, and the ideas outlined here seem to be a
%very promising way to realize this ideal.'' Noam Chomsky, 2017-02-01.
 %
%``Language Science Press is setting a standard for freely accessible
%articles and books that are carefully reviewed.'' Adele Goldberg, 2017-05-02. 
%
%``Sharing data and methods is one of the pillars of scholarly inquiry. The knowledge created by
%scholars belongs to everyone, and open access publications are a major pathway to realizing that
%ideal. Language Science Press, together with Knowledge Unlatched, provides an excellent way for us
%to make our findings available to the global public.'' Steven Pinker, 2017-01-22. 
%} The fundraising for the period 2021--2023 is ongoing.

如果你认为想阅读这类教科书的人都应得以免费拥有这些书,而且科学研究的出版不应落入利益驱动的出版社手中,那么你就应该加入语言科学出版社的群体,并在以下几个方面支持我们:可以在语言科学出版社上注册,将你的名字列在其他1000余名热心学者之中,还可以帮助校对或者修改格式。我们也在寻求机构的支持,像基金会、学会、语言学系或大学图书馆。有关如何资助的详细信息请参见网页:\url{https://langsci-press.org/supporters}。如有问题,请联系我或者语言科学出版社主任 \href{mailto:contact@langsci-press.org}{contact@langsci-press.org}。
%If you think that textbooks like this one should be freely available to whoever wants to read them
%and that publishing scientific results should not be left to profit-oriented publishers, then you
%can join the Language Science Press community and support us in various ways: you can register with Language Science Press and have your name
%listed on our supporter page with more than 1000 other enthusiasts, you may devote your time and help
%with proofreading. We are also looking for institutional supporters like foundations,
%societies, linguistics departments or university libraries. Detailed information on how to support
%us is provided at the following webpage: \url{http://langsci-press.org/supportUs}.
%In case of questions, please contact me or the Language Science Press coordinator at \href{mailto:contact@langsci-press.org}{contact@langsci-%press.org}.
