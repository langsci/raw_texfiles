\documentclass[output=paper]{langscibook}
\ChapterDOI{10.5281/zenodo.2654347}
\author{James McElvenny \affiliation{University of Edinburgh}}
\title{Preface}
\label{ch:preface}
\abstract{\noabstract}
\begin{document}
\maketitle
Notions of ``form'' have a long history in Western thought on language. When linguistics emerged as an institutionalized discipline in the early decades of the nineteenth century, its practitioners could look back on a multitude of senses and uses of ``form'', embedded in a variety of conceptual schemes. Even though many nineteenth-century linguists sought to emphasize the novelty of their work and imagined a radical break with the ``pre-scientific'' past \citep[see][chap. 1]{MorpurgoDavies1998}, both their everyday practice and their theoretical views were permeated by an intellectual inheritance stretching back over centuries, in which ``form'' occupied a central place.

On a practical level, ``form'' has long been employed in a general sense to refer to the perceptible outer appearances of linguistic expressions, especially in connection with the inflectional variants of words. On a deeper theoretical level, there has often been an effort to find underlying motivations for these appearances and so conceive of ``form'' in senses loaded with metaphysical and epistemological significance. This was the path taken by such movements as the medieval Scholastics and the Enlightenment-era General Grammarians \citep[see][chaps. 8 and 11]{Law2003}, whose successors in the ninetenth century -- despite often disavowing their predecessors -- were similarly engaged in a search for the cognitive, biological or aesthetic bases of linguistic form. 

A particularly prominent figure in nineteenth-century discussions of form in language was Wilhelm von {\Humboldt} (1767--1835), whose writings served as the point of departure for many later scholars. For {\Humboldt} and his followers, there is a sense in which all language is form and nothing else, in that language is the representation we make of the world which, in Kantian fashion, we shape according to our perceptive faculties. ``The essence of language'', writes \citet[17]{Humboldt19051820}, ``consists in pouring the material of the phenomenal world into the form of thoughts.'' (\emph{Das Wesen der Sprache besteht darin, die Materie der Erscheinungswelt in die Form der Gedanken zu giessen.}) A commonplace among the Humboldtians was to claim that each language has its own characteristic form of representation discernible in the form of its expressions. The task of the linguist is to capture these forms and analyse them for what they reveal about the mental, cultural and physical life of language speakers (see \citealt[chap. 5]{MorpurgoDavies1998}; \citealt{Trabant1986}; \citealt{McElvenny2016}). 

The centrality of form to linguistic scholarship continued into the \isi{structuralist} era. The \emph{Cours de linguistique générale} of Ferdinand de {\Saussure} (1857--1913) famously contains the assertion that ``language is a form and not a substance'' (\emph{la langue est une forme et non une substance}) \citep[169]{Saussure19221916}. Following on from the earlier Humboldtian position, a fundamental tenet of structuralism is to conceive of languages as self-contained structures imposed on the material substrate of the world. In describing phonological, grammatical and semantic apparatuses of languages, the \isi{structuralist} is engaged in an investigation of linguistic form \citep[for a classical \isi{structuralist} account couched in these terms, see][54--70]{Lyons1968}.

In the \isi{generativist} era, Noam Chomsky's (b. 1928) efforts to construct an intellectual genealogy for his work involved an attempted appropriation of Humboldtian ``form'', which  rekindled awareness of these ideas in mainstream linguistics. In his \emph{Cartesian linguistics}, \citet[69--77]{Chomsky20091966} sought to assimilate Humboldtian form to his own innovation of generative rules as the underlying system that allows for the creative use of finite means to produce an infinite array of expressions.

The fecundity of ``form'' is visible not only in its polysemy, but also in the family of derivatives it has brought into the world, including such terms as ``formal'', ``formalized'' and ``formalist/formalism''. Like their parent, these terms defy concise definition, although when applied as labels to directions in linguistic research they generally imply concentration on internal systematicity to the exclusion of external explanatory factors alongside an inclination to abstraction and axiomatization -- two tendencies that may in fact manifest independently of one another \citep[cf.][]{Newmeyer1998}. As is explored in several contributions to this volume, formalism as a research mindset is at home in many fields -- such as logic, mathematics, aesthetics and literary studies -- and represents an area of rich historical cross-pollination between linguistics and other disciplines. 

\largerpage[1]\label{p:pref:devices}In a separate but related sense, ``formalism'' as a count noun refers to the devices employed in the representation and analysis of phenomena. Various formalisms in this sense, along with the theoretical views to which they are tied, are also examined in the following chapters.

In composing this volume, we have come together as historians of science and philosophers of language and linguistics to take a critical look at notions of form and their derivatives, and the role they have played in the study of language over the past two centuries. We investigate how these notions have been understood and used, and what this reveals about the way of thinking, temperament and daily practice of linguists.

The first contribution to our volume is Judith Kaplan's examination in \hyperref[chap:kaplan]{Chapter 1} of the role of visual formalisms in representing genealogical relationships between languages. Engaging with some of the latest literature on material culture in the history of science, Kaplan explores how visual diagrams and metaphors helped in grasping relationships between languages in comparative-historical grammar, from the nineteenth century up to the present day. She finds that the tensions between the dominant models of language relationship -- ``tree'' versus ``wave'' models -- were typically conceived in a visual mode, whether this was explicitly represented in a diagram or initially described only as a visual metaphor. She observes shifting commitments to the realism of representations and mutual influences between linguists and those working in neighbouring sciences.

In \hyperref[chap:mcelvenny]{Chapter 2}, James McElvenny compares competing nineteenth-century accounts of ``alternating sounds'' -- a cover term for the apparent unstable phonological variation found in ``exotic'' languages -- for the different attitudes towards linguistic form that they reveal. The traditional view took alternating sounds to be a feature of ``primitive'' languages, which were assumed to have not attained the levels of formal arbitrariness characteristic of European languages. Franz {\Boas}~(1858--1942) famously refuted this view by insisting that all languages have fully developed phonologies and ascribed alternating sounds to perceptual error on the part of outside observers. Georg von der {\Gabelentz} (1840--1893), on the other hand, embraced the phenomenon and wielded it against {\Neogrammarian} doctrine, the leading formal theory of his day. Both {\Boas}' and {\Gabelentz}' positions can claim a measure of theoretical sophistication and at the same time contain obvious faults. McElvenny places these positions in their historical context and considers why {\Boas}' view was so well received in linguistics while {\Gabelentz}' was not.

\hyperref[chap:fortis]{Chapter 3} turns to the links between linguistic, psychological and, above all, aesthetic theory in the work of Edward {\Sapir} (1884--1939). In this chapter, Jean-Michel Fortis provides a detailed exposition of {\Sapir}'s writings on form in language, concentrating in particular on {\Sapir}'s notion of ``form-feeling'' and following the trail -- in some places explicitly marked by {\Sapir} himself and in others reconstructed by Fortis through terminological and conceptual detective work -- to identify his sources of inspiration. Fortis places {\Sapir} in a finely interlaced intellectual network spanning across contemporary \emph{Gestalt} psychology and German art theory, with a heritage extending at least as far back as the Romantic period around the turn of the eighteenth to the nineteenth century.

The focus on {\Sapir} continues in \hyperref[chap:elffers]{Chapter 4}, where Els Elffers critically compares {\Sapir}'s philosophy of science to that of Jerry {\Fodor} (1935--2017) and examines the implications of their views for the treatment of linguistic form. Looking at {\Sapir}'s arguments against the ``superorganic'' in language scholarship and {\Fodor}'s proposal for ``\isi{token physicalism}'', she finds striking similarities between the two, despite their very different intellectual contexts: {\Sapir} was responding to ideas in anthropology emerging from debates about the nature of the \emph{Geisteswissenschaften} in contrast to the \emph{Naturwissenschaften}, whereas {\Fodor} was responding to \isi{logical positivism}. Both scholars, however, concerned themselves with how best to demarcate the individual sciences, with the specific example of linguistics in mind, and settled on the principle of demarcating the sciences not according to their subject matter but the way in which that subject matter is conceived.

In \hyperref[chap:karstens]{Chapter 5}, Bart Karstens undertakes a re-examination of the genesis of \isi{linguistic structuralism} and its early interaction with Russian Formalism, a school of literary analysis from the early twentieth century. Karstens engages in a detailed investigation of the scholarly network around Roman {\Jakobson} (1896--1982) and his role as a vector for the transmission of Russian Formalism first to the {\PragueSchool} of structuralism in the 1920s and then later to the United States. While formalist doctrine was often heavily criticized by the early structuralists, Karstens shows that various formalist views informed elements of early structuralism.

A similar story of ``resistant embrace'' is told in \hyperref[chap:joseph]{Chapter 6}, where John Joseph reconsiders the place of structuralism in French linguistics of the mid-twentieth century, before the onset of the ``post-\isi{structuralist}'' period. Focusing on such figures as Émile {\Benveniste} (1902--1976), Henri {\Meschonnic} (1932--2000), Aurélien {\Sauvageot} (1897--1988) and their closest contemporaries, Joseph demonstrates that each of these figures has a complex relationship to structuralism: at times criticizing the apparent premises of the approach while employing recognizably \isi{structuralist} forms of analysis, or publicly avowing structuralism while straying away from its principles in their own work.

In \hyperref[chap:nefdt]{Chapter 7}, Ryan Nefdt surveys some of the radical changes in theory that generative linguistics has undergone in its short history and derives from them positive lessons for the philosophy of science. Amid the turbulence and instability that has characterized \isi{generative theory}, he identifies one constant: the formal structures in language that generative linguists describe. With the durability of this constant in mind, he advocates for a position of \isi{structural realism} in the philosophy of linguistics. Such a position, he argues, would allow linguists to escape pessimistic meta-induction -- that is, the notion that we must necessarily expect our theories to one day be refuted and superseded -- and allows them to step away from the ontology of natural languages, thereby securing the epistemological basis of the formal approach to language.

The gaze of the last two chapters in our volume is largely directed towards current questions in the philosophy of linguistics, specifically the role of \isi{normativity} and authority in language description. After first tracing the origins of \isi{generative grammar} in formalist approaches to logic, Geoffrey Pullum, in \hyperref[chap:pullum]{Chapter 8}, develops a new perspective on the classical distinction between prescriptivism and descriptivism. He contends that the value of a grammatical description lies in the precise, formalized account it provides of a particular set of linguistic practices, which can guide those who may wish to participate in those practices. In serving as a guide, every grammar has \isi{normative} force, but is not necessarily prescriptive: the grammar-reader may follow its advice but is not compelled to do so.

In \hyperref[chap:riemer]{Chapter 9}, Nick Riemer identifies the ideologies of language he sees embodied in the ``\isi{unique form hypothesis}'', the assumption that every linguistic expression can be reduced to a single, universally agreed underlying representation. While linguists might seek to distance themselves from this hypothesis and its implications, it is, argues Riemer, a recurring motif in linguistics, especially prominent in the teaching of the discipline. Its effects in education are particularly pernicious, since teachers, due to the exigencies of pedagogy, can usually offer no justification for the unique forms they present to their students other than arbitrary authority, a practice that reinforces unreflective submission to authority of all kinds, both at university and in life. Acknowledging that most linguists would shudder at such consequences, Riemer pleads for greater open-mindedness among linguists towards critique of the discipline's foundations.

Although dealing with a broad range of topics from diverse perspectives and in different styles, this volume is the product of concerted collective effort. Each of us came to this project with existing ideas about form and formalism in linguistics. These ideas we set out in draft chapters, which we discussed in person at a meeting in Edinburgh in August 2018. After our meeting, we revised the chapters to reflect the insights gained through our discussion. It is these revised chapters, shaped and harmonized by our dialogue, that are contained in this volume. 

\sloppy
\printbibliography[heading=subbibliography]
\end{document} 