\documentclass[output=paper]{langsci/langscibook} 
\title{Focus and prosody in Tagalog} 
\author{Naonori Nagaya\affiliation{Tokyo University of Foreign Studies}\lastand Hyun Kyung Hwang\affiliation{RIKEN Brain Science Institute}}                                         

\ChapterDOI{10.5281/zenodo.1402557}
\abstract{In this paper, we look into the interaction between focus and prosody in Tagalog. In this language, for most focus conditions regular correspondences between syntax and information structure are observed: canonical constructions are used for sentence focus and predicate focus conditions, while pseudocleft constructions are used for argument focus conditions. However, some wh-questions, in particular targeting non-agent arguments, can be answered by means of canonical constructions as well as pseudocleft constructions. In this experimental study, we examine production data in order to test how Tagalog speakers prosodically distinguish canonical sentences associated with different focus structures.  The results reveal that F0 cues and intensity consistently differentiate focused conditions from all-old utterances throughout the entire sentence. However, the distinct focus conditions are not prosodically differentiated. As for the argument focus condition, there may be durational effects applying to the phrase in narrow focus, but this needs further confirmation.}

% \keywords{Tagalog, focus, information structure, prosody}


\begin{document}

\maketitle

\section{\label{s:nagaya:1}Mismatch between syntactic and focus structure}

\ili{Tagalog}, an \ili{Austronesian} language of the Philippines, has VSO \isi{word order}, displaying VO \isi{word order} correlates in a relatively consistent manner. Thus, in typical transitive clauses as in (\ref{e:nagaya:1}) a predicative verb appears in the clause-initial position, followed by arguments and adjuncts. Arguments and adjuncts are marked by either determiner-like case-markers or prepositions. In this paper, we refer to this type of verb-predicate \isi{clause} as the canonical construction.

\begin{exe}
		\ex\label{e:nagaya:1}\textit{Kumain si Mama ng mami sa kusina.}\\
		\gll K{\USSmaller}um{\USGreater}ain si=Mama {nang=mami\footnotemark} sa=  kusina.\\
		eat<\textsc{av}>  \textsc{p.nom}=Mama  \textsc{gen}=noodles  \textsc{loc}=  kitchen\\
		\glt ‘Mama ate noodles in the kitchen.’
\end{exe}

\footnotetext{In the commonly-used \ili{Tagalog} orthography, the diagraph \textit{ng} represents a velar nasal /ŋ/. An exception is the genitive case-marker for common nouns, which is pronounced as [naŋ] but spelled as \textit{ng}. In this paper, however, it is presented as \textit{nang} instead of \textit{ng} for the sake of convenience.}

\noindent
\ili{Tagalog} also has another construction type, where one of the arguments appears in the clause-initial position. We call this construction type a \isi{pseudocleft construction} in the sense that it involves an equational \isi{clause} structure with a \textit{wh}-relative clause-like structure.\footnote{See 
  \citeauthor{Kaufman2009} (\citeyear*{Kaufman2009,Kaufman2018} [this volume])  for another view of this construction type.}
To illustrate, consider (\ref{e:nagaya:2}) and (\ref{e:nagaya:3}). 

\begin{exe}
	\ex\label{e:nagaya:2}\textit{Si Mama ang kumain ng mami.}\\
	\gll Si=    Mama    ang={\ob}k{\USSmaller}um{\USGreater}ain  nang=  mami{\cb}.\\
	\textsc{p.nom}=  Mama    \textsc{nom}=eat<\textsc{av}>  \textsc{gen}=  noodles\\
	\glt ‘The one who ate noodles is Mama.’\\
	\glt ‘Mama is the one who ate noodles.’
\end{exe}

\begin{exe}
	\ex\label{e:nagaya:3}\textit{Ang mami ang kinain ni Mama.}\\
	\gll {\USOParen}Ang={\USCParen}mami    ang={\ob}k{\USSmaller}in{\USGreater}ain  ni=    Mama{\cb}.\\
	\textsc{nom}=noodles  \textsc{nom}=eat<\textsc{pv.pfv}>  \textsc{p.gen}=  Mama\\
	\glt ‘What Mama ate is noodles.’
\end{exe}

\noindent
As seen in these examples, canonical and \isi{pseudocleft} constructions share the same pro-\linebreak positional content. A contrast between the two construction types lies in the focus assignment patterns with which they are associated (\citealt{Kaufman2005}, \citealt{Nagaya2007}; see \citealt{Lambrecht1994} for the notion of \isi{focus structure} used here). On the one hand, canonical constructions are employed for either \isi{sentence focus} or \isi{predicate focus} structures, see (\ref{e:nagaya:4}) and (\ref{e:nagaya:5}), respectively.

\begin{exe}
	\ex\label{e:nagaya:4}
	\begin{xlist}
		\exi{Q:} \textit{Anong nangyari?}\\
		\gll Ano  =’ng  nang-yari{\USQMark}\\
		what  \textsc{=nom}  \textsc{av:pfv}-happen\\
		\glt ‘What happened?’
		\exi{A:} \textit{Kumain si Mama ng mami.}\\
		\gll K{\USSmaller}um{\USGreater}ain  si=Mama    nang=  mami.\\
		eat<\textsc{av}>  \textsc{p.nom}=Mama  \textsc{gen}=  noodles\\
		\glt ‘Mama ate noodles.’
	\end{xlist}
\end{exe}

\begin{exe}
	\ex\label{e:nagaya:5}
	\begin{xlist}
		\exi{Q:} \textit{Anong ginawa ni Mama?}\\
		\gll Ano  =’ng  g{\USSmaller}in{\USGreater}awa  ni=    Mama{\USQMark}\\
		what  =\textsc{nom}  do<\textsc{pv:pfv}>  \textsc{p.gen}=  Mama\\
		\glt ‘What did Mama do?’
		\exi{A:} \label{e:nagaya:5a} \textit{Kumain siya ng mami.}\\
		\gll K{\USSmaller}um{\USGreater}ain=siya  nang=mami.\\
		eat<\textsc{av}>=\textsc{3sg.nom}  \textsc{gen}=noodles\\
		\glt ‘She ate noodles.’
	\end{xlist}
\end{exe}

\noindent
On the other hand, \isi{pseudocleft} constructions are employed for \isi{narrow focus} or \isi{argument focus}, where the initial constituent of a \isi{clause} is exclusively focused. In particular, this construction type is the only option in \isi{contrastive focus} contexts. Example (\ref{e:nagaya:6}) illustrates an explicit contrast.

\begin{exe}
	\ex\label{e:nagaya:6}
	\begin{xlist}
		\exi{A:}  
		\gll K{\USSmaller}um{\USGreater}ain=daw  si=Maria    nang=  mami.\\
		eat<\textsc{av}>=hearsay  \textsc{p.nom}=Maria  \textsc{gen}=  noodles\\
		\glt ‘(They say) Maria ate noodles.’
		\exi{B:}
		\gll Hindi. Si=Mama  ang={\ob}k{\USSmaller}um{\USGreater}ain  nang=  mami{\cb}.\\
		\textsc{neg}  \textsc{p.nom}=Mama \textsc{nom}=eat<\textsc{av}>  \textsc{gen}= noodles\\
		\glt ‘No. It is Mama (not Maria) who ate noodles.’
	\end{xlist}
\end{exe}

\noindent
Not surprisingly, \textit{wh}-questions must take the form of \isi{pseudocleft} constructions, as in (\ref{e:nagaya:7}) and (\ref{e:nagaya:8}). Attention should be paid to the structural parallelism between (\ref{e:nagaya:2})/(\ref{e:nagaya:3}) and (\ref{e:nagaya:7})/(\ref{e:nagaya:8}).

\begin{exe}
	\ex\label{e:nagaya:7}
	\gll Sino ang={\ob}k{\USSmaller}um{\USGreater}ain nang= mami{\cb}{\USQMark}  {\ob}cf. {\USOParen}\ref{e:nagaya:2}{\USCParen}{\cb}\\
	who.\textsc{nom}  \textsc{nom}=eat<\textsc{av}>  \textsc{gen}=  noodles\\
	\glt ‘Who is the one who ate noodles?’\\
	\glt ‘Who ate noodles?’
\end{exe}

\begin{exe}
	\ex\label{e:nagaya:8}
	\gll Ano  ang={\ob}k{\USSmaller}in{\USGreater}ain ni= Mama{\cb}{\USQMark}  {\ob}cf. {\USOParen}\ref{e:nagaya:3}{\USCParen}{\cb}\\
	what \textsc{nom}=eat<\textsc{pv:pfv}>  \textsc{p.gen}=  Mama\\
	\glt ‘What is it that Mama ate?’\\
	\glt ‘What did Mama eat?’
\end{exe}

\noindent
To summarize, in \ili{Tagalog}, canonical constructions are used for \isi{predicate focus} (henceforth PF) and \isi{sentence focus} (henceforth SF), while \isi{pseudocleft} constructions are employed for \isi{argument focus} (henceforth AF). See \tabref{tab:nagaya:1} for a summary of these observations.

\begin{table}
\begin{tabularx}{\textwidth}{llQ}
	\lsptoprule
	\textbf{Construction type} & \textbf{Focus structure} & \textbf{Contexts}\\
\midrule 
	{{Canonical construction}} & {Predicate Focus (PF) } & {‘What happened to X?’}\newline {‘What did X do?’}\\
	
\tablevspace 	
	& Sentence Focus (SF) & ‘What happened?’\\ 
	
	{{Pseudocleft construction}} & {Argument Focus} (AF)& {‘only’}\\
	{} & {} & {focus of negation/correction}\\
	{}&{}& {\textit{wh}-question}\\
\lspbottomrule
\end{tabularx}
\caption{Construction types and focus structures in Tagalog}
\label{tab:nagaya:1}
\end{table}

\noindent
However, the summary in \tabref{tab:nagaya:1} slightly overstates the regularity of the correspondence between syntactic and \isi{focus structure} because questions targeting an argument do not require a \isi{pseudocleft construction} as the answer. Rather, a question such as ‘What did Mama eat?’ allows for three types of answers, as seen in (\ref{e:nagaya:9}).

\begin{exe}
\ex\label{e:nagaya:9}
	\begin{xlist}
	\exi{Q:}  \textit{Ano ang kinain ni Mama?} [=(\ref{e:nagaya:8})]\\
	\gll \uline{Ano}  ang={\ob}k{\USSmaller}in{\USGreater}ain  ni=    Mama{\cb}{\USQMark}  \\
	what  \textsc{nom}=eat<\textsc{pv:pfv}>  \textsc{p.gen}=  Mama\\
	\glt ‘\uline{What} is it that Mama ate?’
	\glt ‘\uline{What} did Mama eat?’
	\exi{A0:} \label{e9a0}
	\gll \textit{\uline{Mami}.}\\
	noodles\\
	\glt ‘\uline{Noodles}.’
	\exi{A1:} \textit{Kumain siya ng mami.} {[Canonical]} \label{e9a1}\\
	\gll K{\USSmaller}um{\USGreater}ain=siya nang=  mami\llap{\uline{\phantom{\textsc{gen}=  nmami}}}. \\
	eat<\textsc{av}>=3\textsc{sg.nom}  \textsc{gen}=  noodles\\
	\glt ‘She ate  \uline{noodles}.’
	\exi{A2:} \textit{Mami ang kinain niya.} {[Pseudocleft]} \label{e9a2}\\
	\gll \uline{Mami} ang=k{\USSmaller}in{\USGreater}ain=niya. \\
	noodles  \textsc{nom}=eat<\textsc{pv:pfv}>=3\textsc{sg.gen}\\
	\glt ‘What she ate is \uline{noodles}.’
\end{xlist}
\end{exe}

\noindent
That is, the question \textit{Ano ang kinain ni Mama?} ‘What did Mama eat?’ can be answered with a \isi{pseudocleft construction} in (\hyperref[e9a2]{9A2}) as well as with a canonical construction in (\hyperref[e9a1]{9A1}), despite the fact that here only one argument is in focus. In (\hyperref[e9a1]{9A1}), then, we see a mismatch between syntactic and \isi{focus structure} deviating from the regularities stated in \tabref{tab:nagaya:1}.

Note that such a mismatch is not possible when the agent NP is the target of a wh-question. Consider (\ref{e:nagaya:10}).

\begin{exe}
\ex\label{e:nagaya:10}
\begin{xlist}
	\exi{Q:}
	\textit{Sino ang bumili ng mami?}\\
	\gll Sino  ang=  b{\USSmaller}um{\USGreater}ili    nang=    mami{\USQMark}\\
	who  \textsc{nom}=  buy<\textsc{av}>    \textsc{gen}=    noodles\\
	\glt ‘Who bought noodles?’
	\exi{A1:} \label{e10a1}
	\textit{Si Mama.}
	\exi{A2:} \label{e10a2}
	??\textit{Bumili si Mama ng mami.}  [canonical]\\
	\gll B{\USSmaller}um{\USGreater}ili  si=  Mama    nang=  mami.\\
	buy<\textsc{av}>  \textsc{p.nom}= Mama  \textsc{gen}=  noodles\\
	\glt ‘Mama bought noodles.’
	\exi{A3:} \label{e10a3}
	\textit{Si Mama ang bumili.} [\isi{pseudocleft}]\\
	\gll Si=  Mama    ang=  b{\USSmaller}um{\USGreater}ili.\\
	\textsc{p.nom} Mama    \textsc{nom}=  buy<\textsc{av}>\\
	\glt ‘It is Mama who bought noodles.’
\end{xlist}
\end{exe}

\noindent
To answer a question targeting the agent, one can employ an agent NP by itself as in (\hyperref[e10a1]{10A1}) or a \isi{pseudocleft construction} as in (\hyperref[e10a3]{10A3}). However, the use of a canonical construction in (\hyperref[e10a2]{10A2}) is not felicitous. So, canonical constructions are only legitimate answers to argument questions if the argument asked for does not bear the agent role.
	
With regard to the constructions where syntactic and \isi{focus structure} do not properly match the generalizations captured in \tabref{tab:nagaya:1}, the question arises whether in such constructions the narrowly focused constituents differ prosodically from non-focused constituents. That is, do \ili{Tagalog} speakers prosodically distinguish \isi{argument focus} (\hyperref[e9a1]{9A1}) from \isi{predicate focus} (\hyperref[e:nagaya:5a]{5A}) in the canonical construction?

In order to answer this question, we carried out a phonetic experiment. Our working hypothesis is that canonical constructions with different focus structures display the same syntax but with different prosodic cues, such as MaxF0 and duration. To the best of our knowledge, the interaction between focus and \isi{prosody} in \ili{Tagalog} has not been well explored in experimental studies (cp. \citealt{Kaufman2005}). Our study will be the first experimental research on this matter.

The rest of this paper is organized as follows: in \sectref{s:nagaya:2}, we give a detailed description of the method employed for this experimental study. In \sectref{s:nagaya:3}, the results of the experiment and analyses of them are provided. \sectref{s:nagaya:4} concludes this paper.

\section{\label{s:nagaya:2}Method}

In this experimental study, we look into the question of whether \ili{Tagalog} speakers proso-\linebreak dically distinguish canonical sentences associated with different focus structures. To investigate this question, we make an acoustic comparison of the target sentence \textit{Bumili siya nang mami} ‘She bought noodles’ in four different focus contexts: SF, PF, AF, and All-Old contexts (henceforth AO). See (\ref{e:nagaya:11}), (\ref{e:nagaya:12}), (\ref{e:nagaya:13}), and (\ref{e:nagaya:14}), respectively.

\begin{exe}
	\ex\label{e:nagaya:11}
	\begin{xlist}
		\exi{Q:}
		\gll Ano  =’ng  nang-yari{\USQMark}\\
		what  =\textsc{nom}  \textsc{av:pfv}-happen\\
		\glt ‘What happened?’
		\exi{A:}
		\gll B{\USSmaller}um{\USGreater}ili =siya nang= mami\llap{\uline{\phantom{nbuy{\USSmaller}\textsc{av}{\USGreater}  =3\textsc{sg.nom}  \textsc{gen}=  noodle}}}.\\
		buy<\textsc{av}>  =3\textsc{sg.nom}  \textsc{gen}=  noodles\\
		\glt ‘\uline{She bought noodles}.’
	\end{xlist}
\end{exe}

\begin{exe}
	\ex\label{e:nagaya:12}
	\begin{xlist}
		\exi{Q:}  
		\gll Ano  =’ng  g{\USSmaller}in{\USGreater}awa  ni=    Mama{\USQMark}\\
		what  =\textsc{nom}  do<\textsc{pv:pfv}>  \textsc{p.gen}=  Mama\\
		\glt ‘What did Mama do?’
		\exi{A:}
		\gll \uline{B{\USSmaller}um{\USGreater}ili}  =siya nang=  mami\llap{\uline{\phantom{nang=  mami}}}.\\
		buy<\textsc{av}>  =3\textsc{sg.nom}  \textsc{gen}=  noodles\\
		\glt ‘She \uline{bought noodles}.’
	\end{xlist}
\end{exe}

\begin{exe}
	\ex\label{e:nagaya:13}
	\begin{xlist}
		\exi{Q:}  
		\gll \uline{Ano}  =’ng  {\ob}b{\USSmaller}in{\USGreater}ili    ni=    Mama{\cb}{\USQMark}\\
		what  =\textsc{nom}  \phantom{[}buy<\textsc{pv:pfv}>  \textsc{p.gen}=  Mama\\
		\glt ‘\uline{What} did Mama buy?’
		\exi{A:}
		\gll B{\USSmaller}um{\USGreater}ili=siya  nang=  mami\llap{\uline{\phantom{nang=  mami}}}.\\
		buy<\textsc{av}>=\textsc{3sg.nom}  \textsc{gen}=  noodles\\
		\glt ‘She bought \uline{noodles}.’
	\end{xlist}
\end{exe}

\begin{exe}
	\ex\label{e:nagaya:14}
	\begin{xlist}
		\exi{Q:}  
		\gll B{\USSmaller}um{\USGreater}ili  =ba  si=    Mama  nang=  mami{\USQMark}\\
		buy<\textsc{av}>  \textsc{=q}  \textsc{p.nom}=  Mama  \textsc{gen}=  noodles\\
		\glt ‘Did Mama buy noodles?’
		\exi{A:}
		\gll Oo,  b{\USSmaller}um{\USGreater}ili  =siya    nang=  mami.\\
		yes  buy<\textsc{av}>  =\textsc{3sg.nom}  \textsc{gen}=  noodles\\
		\glt ‘Yes, she bought noodles.’
	\end{xlist}
\end{exe}

\noindent
For this experiment, five male participants were recorded. See \tabref{tab:nagaya:2}. All of them are college students in their twenties. They are native speakers of \ili{Tagalog} but from different dialectal backgrounds: Quezon City (3), Rizal (1), Laguna (1). They also speak \ili{English} as a second language. The recordings were made at the University of the Philippines, Diliman. All recording sessions were organized and supervised by the first author. A portable recorder (Zoom H5) with a head-mounted microphone (Shure Beta 54) was employed for the recordings. 

\begin{table}
\begin{tabularx}{\textwidth}{XXXl}
	\lsptoprule
	Participant & Hometown & Gender & Age\\
	\midrule 
	{Speaker 1} & {Laguna} & {male} & {21}\\
	\tablevspace
	Speaker 2 & Quezon City & male & 20\\
	\tablevspace
	{Speaker 3} & {Quezon City} & {male} & {21}\\
	\tablevspace
	Speaker 4 & Rizal & male & 23\\
	\tablevspace
	{Speaker 5} & {Quezon City} & {male} & {25}\\
	\lspbottomrule
\end{tabularx}
	\caption{List of participants}
	\label{tab:nagaya:2}
\end{table}

\noindent
During the recording sessions, participants were asked to read the answers in a list of question-answer pairs. The four target pairs (SF, PF, AF and AO contexts) were randomly dispersed together with nine dummy pairs. See the Appendix for the complete list of question-answer pairs used for this experiment. Each participant repeated the whole list ten times.

At the recording, each participant was instructed to exchange a conversation with another participant. More precisely, one participant asked the questions, and another participant answered them.\footnote{We thank one of the reviewers who hinted at possible effects of convergence between two speakers (see \citealt{Garrod2009, Kim2011, Gorisch2012}) in this setting. However, it seems that such effects were not seriously large in our data because two speakers who exchanged conversations in the recording session exhibited quite different prosodic patterns.}  Speaker 1 was paired with Speakers 2 and 3. Speakers 4 and 5 were paired. Only answers were recorded. Before the actual recording session, participants were asked to practice by reading the two sets of sentences.

\section{\label{s:nagaya:3}Results and discussion}
\subsection{\label{s:nagaya:3.1}Impressionistic comparison of pitch contours}

A total of 200 utterances (4 \isi{information status} x 5 speakers x 10 repetitions) were analyzed. In analyzing the data, \isi{prosodic word} boundaries were manually marked on each utterance. The target sentence \textit{Bumili siya nang mami} ‘She bought noodles’ was divided into three prosodic words\footnote{“P”, “N”, and “A” are labels for prosodic words. They are abbreviations of “predicate”, “\isi{nominative}”, and “accusative”. But this does not imply that \ili{Tagalog} has a nominative-accusative case system.}:\\

\begin{tabular}{l}
$\bullet$ \textit{bumili} ‘bought’ (P)\\
$\bullet$ \textit{siya} ‘she’ (N)\\
$\bullet$ \textit{nang mami} ‘noodles’ (A)\\
\end{tabular}\\

\noindent
For impressionistic comparison of the pitch contours as a function of \isi{information status}, time-normalized pitch tracks in semitone are plotted in \figref{fig:nagaya:1}, averaging across all renditions by each speaker. Overall, the AO condition yielded lower F0s compared to all focused conditions across all speakers. In comparing different focus types, however, speakers exhibited slightly distinct patterns. As shown in the top-left panel of \figref{fig:nagaya:1}, Speaker 1 produced the SF condition (dark solid line) with a slightly higher pitch than the other focus conditions, but no substantial difference was observed between PF (dotted line) and AF (dashed line) in terms of F0. On the other hand, Speaker 2 (top-right panel) exhibited somewhat higher F0 peaks of P and A in the PF condition (dotted line) than in the other focus conditions. The prosodic manifestation of \isi{information status} of this particular speaker seems to be different from the other speakers in that the overall shapes of contours are quite distinct. Specifically, the contours of Speaker 2 in the PF and AO conditions show a different overall pattern from the ones found for AF or SF whereas those of the other speakers exhibit more or less similar overall contour shapes in all information conditions. Speaker 3 (mid-left panel) seems to be quite sensitive to the presence or absence of focus, but does not distinguish different types of focus; PF, SF and AF yielded nearly the same F0 contours. Speaker 4 (mid-right panel) and Speaker 5 (bottom panel) produced SF and PF with a somewhat higher F0 than AF but no remarkable difference was found between information conditions.

\begin{figure}[H]
	\includegraphics[width=\textwidth]{figures/NagayaHwangFINALKorrektur-img2.png}
	\caption{F0 contours of each speaker in semitone: SF, AF, PF and AO are represented by dark solid lines, dashed lines, dotted lines and light solid lines, respectively.}
	\label{fig:nagaya:1}
\end{figure}

\subsection{\label{s:nagaya:3.2}Statistical analyses}

In order to compare prosodic characteristics of different information conditions, maximum F0 (MaxF0), minimum F0 (MinF0), mean F0, mean intensity, and duration values of each \isi{prosodic word} were extracted using the Praat script \sloppy{ProsodyPro} \citep{Xu2013}.

For statistical analysis, linear mixed-effects analyses were conducted using JMP 9, with the speaker as random effects and \isi{information status} as fixed effects. MaxF0, MinF0, meanF0, mean intensity, and duration were used as dependent measures. The analyses were performed separately for each phrase. All reported effects were significant at the p < 0.05 level. The results of our analyses are summarized in \tabref{tab:nagaya:3}.

\begin{table}
	\begin{tabular}{llll}
		\lsptoprule
		& P (\textit{bumili}) & N (\textit{siya}) & A (\textit{nang mami})\\
		\midrule
		MaxF0 & PF=SF=AF>AO & SF=PF=AF>AO & SF=PF=AF>AO\\
		MinF0 & AF=PF=SF>AO & SF=PF=AF>AO & AF=PF=SF>AO\\
		mean F0 & PF=SF=AF>AO & SF=PF=AF>AO & SF=PF=AF>AO\\
		intensity & PF=AF=SF>AO & PF=AF=SF>AO & PF=AF=SF>AO\\
		duration & PF=AF=AO=SF & AF=PF=AO=SF & AF=PF=SF>AO\\
		\lspbottomrule
	\end{tabular}
	\caption{Results of statistical analyses}
	\label{tab:nagaya:3}
\end{table}

\newpage 
Prosodically, all the conditions show the same patterns for P (\textit{bumili}) and N (\textit{siya}); in these parts of the sentence, all the focus conditions were realized with significantly higher F0 and greater intensity than the AO condition while different types of focus were not prosodically differentiated. Interestingly, duration was not significantly different among the four information conditions. 

Similar results are observed in the A phrase (\textit{nang mami}). Focus conditions yielded highest F0, greater intensity, and longer duration compared to the AO condition. However, the four conditions did not differ significantly with respect to the acoustic measurements. Unlike the P phrase (\textit{bumili}) and the N phrase (\textit{siya}), this phrase was realized with longer duration when it received focus. It is conceivable this is an effect of \isi{narrow focus}. Yet, further investigation involving more speakers and material would be necessary to confirm this effect.

\subsection{\label{3.3}Discussion}

The results of our analyses reveal two important facts about the interaction between focus and \isi{prosody} in \ili{Tagalog}. First, it was observed that F0 and intensity consistently differentiated focused conditions from AO. This observation was also confirmed by the statistical analyses. Second, no significant prosodic differences were observed between the distinct focus constructions.  

\largerpage
A general problem for these conclusions, however, pertains to the fact that the intonational contours of the target sentences vary from speaker to speaker to an extent that needs further explanation. The five speakers did utter the same sentence but with quite different contours, as demonstrated in \figref{fig:nagaya:1}. The pitch contours of Speaker 4 may appear to be reasonably similar to that of Speaker 5 to be considered minor variants of the same overall pattern. But the similarities between the remaining contours are less easily amenable to a single underlying melody. It is not clear yet how to account for this variation among \ili{Tagalog} speakers. Dialectal differences could be one factor to consider.  However, the prosodic characteristics of different dialects in \ili{Tagalog} are next to unknown, so this has to remain a speculation at this point. Further, Speakers 2, 3, and 5 produced noticeably different patterns though they are from the same region. Thus, it seems that this large between-speaker variation cannot be attributed solely to dialectal differences.

\section{\label{s:nagaya:4}Conclusion}

In this paper, we presented a preliminary experimental phonetic analysis of the interaction between focus and \isi{prosody} in \ili{Tagalog}. In particular, we highlighted mismatching patterns between syntax and information structure found in question-answer pairs. Some \textit{wh}-questions, specifically ones targeting non-agent arguments, can be answered by means of canonical constructions as well as \isi{pseudocleft} constructions, despite the fact that for most focus conditions \ili{Tagalog} displays regular correspondences between syntax and information structure: canonical constructions are used for SF and PF conditions, while \isi{pseudocleft} constructions are used for AF conditions.

Our working hypothesis was that there might be prosodic cues to distinguish canonical constructions associated with different focus structures. The results of our production study reveal that F0 cues and intensity consistently differentiate focused conditions from all-old utterances throughout the entire sentence. As for the \isi{argument focus} condition, there may be durational effects applying to the phrase in \isi{narrow focus}, but this needs further confirmation.

\section*{Acknowledgements}

An earlier version of this paper was presented at the third International Workshop on Information Structure in \ili{Austronesian} Languages held in ILCAA, Tokyo University of Foreign Studies, on February 18--20, 2016. We are thankful to the audience for the valuable comments and criticism that have helped in improving the manuscript. We are also grateful to three anonymous reviewers for valuable suggestions. Of course, any errors that remain are our responsibility. This work was supported by the Japan Society for the Promotion of Science (Grants-in-Aid \#15K16734, \#15H03206, \#17H02331, and \#17H02333).

\section*{Abbreviations}

\begin{multicols}{2}
	\begin{tabbing}
		glossgloss \= \kill
		\textsc{av} \> \isi{actor voice}\\
		\textsc{cv} \> circumstantial \isi{voice}\\
		\textsc{dup} \> reduplication\\
		\textsc{gen} \> genitive\\
		\textsc{ger} \> gerund\\
		\textsc{ipfv} \> imperfective\\
		\textsc{lk} \> linker\\
		\textsc{loc} \> locative\\
		\textsc{lv} \> \isi{locative voice}\\
		\textsc{neg} \> negator\\
		\textsc{nom} \> \isi{nominative}\\
		\textsc{p} \> personal name\\
		\textsc{pfv} \> perfective\\
		\textsc{pros} \> prospective\\
		\textsc{pv} \> \isi{patient voice}\\
		\textsc{sg} \> singular\\
		\textsc{pl} \> plural\\
		1 \> first person\\
		2 \> second person\\
		3 \> third person\\
		“<>” \> infix\\
		“=” \> cliticization\\
		“{\textasciitilde}” \> reduplication
	\end{tabbing}
\end{multicols}

\section*{Appendix: Target sentences}

Four target sentences and nine dummy sentences were employed in this experiment. Below is the list of the target and filler sentences: sentences (\ref{b}), (\ref{g}), (\ref{i}), and (\ref{m}) are targets (highlighted in bold so that they can be spotted more easily), while the others function as fillers. In the recording sessions, the entire list was repeated ten times. The participants were asked to read these sentences in this order. Only the parts in italics were presented to the participants (i.e., no morphological analyses, interlinear glossing, or translations).

% \appendix
% \setcounter{exx}{0}

\begin{exe}
	\ex\label{a}
	\begin{xlist}
		\exi{Q:} \textit{Saan ka pupunta?}\\
		\gll Saan  =ka    pu{\textasciitilde}punta{\USQMark}\\
		where  =\textsc{2sg}.\textsc{nom}  \textsc{av}:\textsc{pros}:go\\
		\glt ‘Where are you going?’
		\exi{A:}  {\textit{Sa Ministop ako pupunta}.}\\
		\gll Sa=  Ministop  =ako    pu{\textasciitilde}punta.\\
		\textsc{loc}=  Ministop  =\textsc{1sg}.\textsc{nom}  \textsc{av}:\textsc{pros}:go\\
		\glt ‘I am going to a Ministop.’
	\end{xlist}
\end{exe}

\begin{exe}
	\ex\label{b}
	\begin{xlist}
		\exi{\textbf{Q:}} \textit{\textbf{Anong binili ni Mama?}}\\
		\gll Ano  =’ng  b{\USSmaller}in{\USGreater}ili  ni=  Mama{\USQMark}\\
		what  =\textsc{nom} \textsc{pv}:\textsc{pfv}:buy  \textsc{p}.\textsc{gen}= Mama\\
		\glt ‘What did Mama buy?’
		\exi{\textbf{A:}} \textit{\textbf{Bumili siya ng mami.}}\\
		\gll B{\USSmaller}um{\USGreater}ili  =siya    nang=  mami.\\
		\textsc{av}:buy  =\textsc{3sg}.\textsc{nom} \textsc{gen}=  noodles\\
		\glt ‘She bought noodles.’
	\end{xlist}
\end{exe}

\begin{exe}
	\ex\label{c}
	\begin{xlist}
		\exi{Q:} \textit{Sino ang bumili ng mami?}\\
		\gll Sino  ang=  b{\USSmaller}um{\USGreater}ili  nang=  mami{\USQMark}\\
		who  \textsc{nom}=  \textsc{av}:buy    \textsc{gen}=  noodles\\
		\glt ‘Who bought noodles?’
		\exi{A:}  \textit{Si Mama ang bumili.}\\
		\gll Si=  Mama  ang=  b{\USSmaller}um{\USGreater}ili.\\
		\textsc{p}.\textsc{nom}  Mama  \textsc{nom}=  \textsc{av}:buy\\
		\glt ‘It is Mama who bought noodles.’
	\end{xlist}
\end{exe}

\begin{exe}
	\ex\label{d}
	\begin{xlist}
		\exi{Q:} \textit{Ano pa binili ni Mama?}\\
		\gll Ano  =pa  {\USOParen}=ang{\USCParen}  b{\USSmaller}in{\USGreater}ili  ni=  Mama{\USQMark}\\
		what  =else  =\textsc{nom}  \textsc{pv}:\textsc{pfv}:buy  \textsc{p}.\textsc{gen}=  Mama\\
		\glt ‘What else did Mama buy?’
		
		\newpage 
		\exi{A:}  \textit{Mami lang ang binili niya.}\\
		\gll Mami    =lang  ang=  b{\USSmaller}in{\USGreater}ili  =niya.\\
		noodles  =only  \textsc{nom}=  \textsc{pv}:\textsc{pfv}:buy  =\textsc{3sg}.\textsc{gen}\\
		\glt ‘She bought only noodles.’
	\end{xlist}
\end{exe}

\begin{exe}
	\ex\label{e}
	\begin{xlist}
		\exi{Q:} \textit{Saan ka pumunta?}\\
		\gll Saan  =ka    p{\USSmaller}um{\USGreater}unta{\USQMark}\\
		where  =\textsc{2sg}.\textsc{nom}  \textsc{av}:go\\
		\glt ‘Where did you go?’
		\exi{A:}  \textit{Pumunta ako sa Ministop.}\\
		\gll P{\USSmaller}um{\USGreater}unta  =ako    sa=  Ministop\\
		\textsc{av}:\textsc{pfv}:go  =\textsc{1sg}.\textsc{nom}  \textsc{loc}=  Ministop\\
		\glt ‘I went to Ministop.’
	\end{xlist}
\end{exe}

\begin{exe}
	\ex\label{f}
	\begin{xlist}
		\exi{Q:} \textit{Mani ba ang kinain niya?}\\
		\gll Mani    =ba  ang=  k{\USSmaller}in{\USGreater}ain  =niya{\USQMark}\\
		peanuts  =\textsc{q}  \textsc{nom}=  \textsc{pv}:\textsc{pfv}:eat  =\textsc{3sg}.\textsc{gen}\\
		\glt ‘Did she eat peanuts?’
		\exi{A:}  \textit{Hindi.  Mami ang kinain niya.}\\
		\gll Hindi.  Mami    ang=  k{\USSmaller}in{\USGreater}ain  =niya.\\
		\textsc{neg}  noodles  \textsc{nom}=  \textsc{pv}:\textsc{pfv}:eat  =\textsc{3sg}.\textsc{gen}\\
		\glt ‘No. She ate noodles.’
	\end{xlist}
\end{exe}

\begin{exe}
	\ex\label{g}
	\begin{xlist}
		\exi{\textbf{Q:}} \textit{\textbf{Anong ginawa ni Mama doon?}}\\
		\gll Ano  =’ng  g{\USSmaller}in{\USGreater}awa  ni=  Mama  doon{\USQMark}\\
		what  =\textsc{nom}  \textsc{pv}:\textsc{pfv}:do  \textsc{p.gen} Mama  there\\
		\glt ‘What did Mama do there?’
		\exi{\textbf{A:}}  \textit{\textbf{Bumili siya ng mami.}}\\
		\gll B{\USSmaller}um{\USGreater}ili  =siya    nang=  mami.\\
		\textsc{av}:buy  =\textsc{3sg}.\textsc{nom} \textsc{gen}=  noodles\\
		\glt ‘She bought noodles.’
	\end{xlist}
\end{exe}

\begin{exe}
	\ex\label{h}
	\begin{xlist}
		\exi{Q:} \textit{Anong paborito mong pagkain?}\\
		\gll Ano  =ng  paborito  mo=ng  pagkian{\USQMark}\\
		what  =\textsc{nom}  favorite  \textsc{2sg}.\textsc{gen}=\textsc{lk}  food\\
		\glt ‘What is your favorite food?’
		\exi{A:}  \textit{Paborito ko ang mami.}\\
		\gll Paborito  =ko    ang=  mami\\
		favorite  =\textsc{1sg}.\textsc{gen}  \textsc{nom}=  noodles\\
		\glt ‘Noodles are my favorite.’
	\end{xlist}
\end{exe}

\begin{exe}
	\ex\label{i}
	\begin{xlist}
		\exi{\textbf{Q:}} \textit{\textbf{Anong nangyari?}}\\
		\gll Ano  =’ng  nang-yari{\USQMark}\\
		what  =\textsc{nom}  \textsc{av}:\textsc{pfv}:happen\\
		\glt ‘What happened?’
		\exi{\textbf{A:}}  \textit{\textbf{Bumili siya ng mami.}}\\
		\gll B{\USSmaller}um{\USGreater}ili  =siya    nang=  mami.  \\
		\textsc{av}:buy    =\textsc{3sg}.\textsc{nom} \textsc{gen}=  noodles\\
		\glt ‘She bought noodles.’
	\end{xlist}
\end{exe}

\begin{exe}
	\ex\label{j}
	\begin{xlist}
		\exi{Q:} \textit{Sino ang bumili ng mami?}\\
		\gll Sino  ang=  b{\USSmaller}um{\USGreater}ili  nang=  mami{\USQMark}\\
		who  \textsc{nom}=  \textsc{av}:buy    \textsc{gen}=  noodles\\
		\glt ‘Who bought noodles?’
		\exi{A:}  \textit{Bumili si Mama ng mami.}\\
		\gll B{\USSmaller}um{\USGreater}ili  si=    Mama  nang=  mami{\USQMark}\\
		\textsc{av}:buy  \textsc{p}.\textsc{nom}=  Mama  \textsc{gen}=  noodles\\
		\glt ‘Mama bought noodles.’
	\end{xlist}
\end{exe}

\begin{exe}
	\ex\label{k}
	\begin{xlist}
		\exi{Q:} \textit{Anong binili ni Mama?}\\
		\gll Ano  =’ng  b{\USSmaller}in{\USGreater}ili  ni=  Mama{\USQMark}\\
		what  =\textsc{nom}  \textsc{pv}:\textsc{pfv}:buy  \textsc{p}.\textsc{gen}=  Mama\\
		\glt ‘What did Mama buy?’
		\exi{A:}  \textit{Mami ang binili niya.}\\
		\gll Mami    ang=  b{\USSmaller}in{\USGreater}ili  =niya.\\
		noodles  \textsc{nom}=  \textsc{pv}:\textsc{pfv}:buy  =\textsc{3sg}.\textsc{gen}\\
		\glt ‘She bought noodles.’
	\end{xlist}
\end{exe}

\begin{exe}
	\ex\label{l}
	\begin{xlist}
		\exi{Q:} \textit{Masarap ba ang mami nila?}\\
		\gll Ma-sarap  =ba  ang=  mami    =nila{\USQMark}\\
		\textsc{adj}-delicious  =\textsc{q}  \textsc{nom}=  noodles  =\textsc{3pl}.\textsc{gen}\\
		\glt ‘Are their noodles delicious?’
		\exi{A:}  \textit{Oo. Masarap ang mami nila.}\\
		\gll Oo  Ma-sarap  ang=  mami    =nila.\\
		yes  \textsc{adj}-delicous  \textsc{nom}=  noodles  =\textsc{3pl}.\textsc{gen}\\
		\glt ‘Yes, their noodles are delicious.’
	\end{xlist}
\end{exe}

\begin{exe}
	\ex\label{m}
	\begin{xlist}
		\exi{\textbf{Q:}} \textit{\textbf{Bumili ba si Mama ng mami?}}\\
		\gll B{\USSmaller}um{\USGreater}ili  =ba  si=    Mama  nang=  mami{\USQMark}\\
		\textsc{av}:buy  =\textsc{q} \textsc{p}.\textsc{nom}=  Mama  \textsc{gen}=  noodles\\
		\glt ‘Did Mama buy noodles?’
		
		\newpage 
		\exi{\textbf{A:}}  \textit{\textbf{Oo, bumili siya ng mami.}}\\
		\gll Oo,  b{\USSmaller}um{\USGreater}ili  =siya    nang=  mami.\\
		yes  \textsc{av}:buy  =\textsc{3sg}.\textsc{nom} \textsc{gen}=  noodles\\
		\glt ‘Yes, she bought noodles.’
	\end{xlist}
\end{exe}

\printbibliography[heading=subbibliography,notkeyword=this]

\end{document}
