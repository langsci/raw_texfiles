\documentclass[output=paper
,modfonts
,nonflat]{langsci/langscibook} 

\ChapterDOI{10.5281/zenodo.1402541}

\title{Two definite markers in Manado Malay} 
\author{Asako Shiohara\affiliation{Tokyo University of Foreign Studies}\lastand Anthony Jukes\affiliation{La Trobe University}}
% \chapterDOI{} %will be filled in at production

% \epigram{}

\abstract{This chapter discusses referential strategies in Manado Malay (MM), a variety of trade Malay spoken in North Sulawesi, with special focus on how a lexical NP is marked according to the information status of the referent. Like some other Malay varieties, MM uses two strategies to indicate definiteness: articles and the third person singular possessive. The articles are derived from demonstratives and used for direct situational and anaphoric reference, while the possessive is used for reference in which some kind of association is required for identification. An article and a possessive may co-occur in one NP. The semantic domain each form covers is not exclusive to the other but rather belongs to intrinsically different semantic dimensions. Thus, the MM system enables speakers to mark that the referent is textual-situationally accessible and, at the same time, associable to the larger shared situation.}

\begin{document}

\maketitle

\section{\label{s:shiohara:1}Introduction}

This paper discusses referential strategies employed in lexical NPs in Manado \ili{Malay} (hereafter MM). There, forms functionally similar to what is called the “definite marker” in other languages are grammaticalizing from two distinct sources: one is from the third person singular possessive marker \textit{depe} and the other is from the demonstratives.

MM is a variety of trade \ili{Malay} spoken in Indonesia by upwards of 2 million people in North Sulawesi, the Sangir and \ili{Talaud} archipelagos to the north, and Gorontalo to the west. It seems to have developed from North Moluccan \ili{Malay}, but it has developed independently since the 17th century \citep[43--44]{Paauw2008}. Until relatively recently, first language speakers were mainly found in the city of Manado, while elsewhere MM was used as a second language by speakers of the indigenous \ili{Minahasan} and \ili{Sangiric} languages. In recent decades MM has become the first language of virtually the entire population of the region. Although most of the \ili{Minahasan} and \ili{Sangiric} languages are still spoken, even elderly people grew up with MM and it could be considered a “joint” first language, while for many people of all ages, it is their first or even their only language. 

The notion of monolingual MM speakers requires some clarification. The education system, media, and government administration largely use standard Bahasa Indonesia (BI), and so everyone is exposed to this variety and code switching and mixing are pervasive. Some speakers are clear about the significant grammatical and lexical differences between BI and MM, and they call MM “Melayu Manado” or “Bahasa Manado”, recognizing that it is not the same as BI. Others do not have this meta-awareness and believe that the language that they speak is BI. As noted by Paauw, “Manado \ili{Malay} and \ili{Indonesian} (and, in particular, colloquial \ili{Indonesian}) have been converging to the point that speakers of Manado \ili{Malay}, to varying extents and often subconsciously, employ \ili{Indonesian} vocabulary and constructions when using Manado \ili{Malay}, and it is often difficult to draw a line between the two languages” \citep[44]{Paauw2008}.

The data sources of this study are (i) translation/elicitation from standard \ili{Indonesian} sentences, (ii) semi-spontaneous monologue that was obtained using a procedural video as stimulus, and (iii) an unpublished MM-BI dictionary compiled by the Pusat Penerjemahan Bahasa (PPB, Translation Centre) in Tomohon. The last item was particularly useful and the authors would like to thank Albert Polii for making it available to us.

The structure of this chapter is as follows: \sectref{s:shiohara:2} provides a brief overview of the NP structure of MM. In \sectref{s:shiohara:3}, we will examine the semantic function of the two definite marking devices, that is, articles and the third person possessive \textit{depe} based on elicited and published data, and provide a brief comparison to the other \ili{Malay} varieties. In \sectref{s:shiohara:4}, we will see larger texts elicited using a procedural video and confirm the usage of the two devices discussed in \sectref{s:shiohara:3}. In \sectref{s:shiohara:5}, we look at the MM definite marking strategy from a cross-linguistic perspective.

\section{\label{s:shiohara:2}NP structure in MM}

Before discussing the referential strategy of MM, we will show the NP structure in MM, largely based on \citet[424--429]{Prentice1994}. (1) is the structure that Prentice suggests. Note that Prentice calls the \isi{demonstrative} “\isi{deictic}”.

\begin{exe}
	\ex\label{e:shiohara:1} (article) (\textsc{possessor}+\textit{pe}) \textsc{N\textsubscript{head}} (attributive \textsc{n}/\textsc{v})\footnote{Quantifiers may precede or follow the head noun according to its pragmatic status, which we will not go into further in this research.}
\end{exe}

\noindent
Two articles \textit{tu} and \textit{ni}, “both translatable by \textit{the}” \citep[424]{Prentice1994}, are derived from the distal \isi{demonstrative} \textit{itu} and proximal \isi{demonstrative} \textit{ini}, respectively. “The articles both mark the \isi{referent} of the following noun as being known to both speaker and addressee, while \textit{ni} has the added function of indicating geographical temporal and\slash or psychological proximity to the speaker” \citep[424]{Prentice1994}. Examples \hyperref[e:shiohara:2a]{(2a--b)} are examples from \citet[424]{Prentice1994}.

\begin{exe}
	\ex\label{e:shiohara:2}
	\begin{xlist}
		\ex\label{e:shiohara:2a}
		\gll \textit{tu} \textit{anging}\\
		\textsc{art} wind \\
		\glt ‘the wind (e.g. which blew down my coconut palms.)’
		\ex\label{e:shiohara:2b}
		\gll \textit{ni} \textit{anging}\\
		\textsc{art} wind\\
		\glt ‘the wind (e.g. which is blowing now.)’
	\end{xlist}
\end{exe}

\noindent
Prentice suggests that the demonstratives may either precede the head-noun alone or follow the combination of article + noun, as shown in Example \hyperref[e:shiohara:3a]{(3a--d)}.

\begin{exe}
	\ex{‘that island’ or ‘those islands’/ ‘this island’ or ‘these islands’}\label{e:shiohara:3}
	\begin{xlist}
		\ex\label{e:shiohara:3a}
		\gll \textit{itu} \textit{pulo}\\
		that  island\\
		\ex\label{e:shiohara:3b}
		\gll  \textit{ini} \textit{pulo}\\
		this  island\\
		\ex\label{e:shiohara:3c}
		\gll \textit{tu} \textit{pulo} \textit{itu}\\
		\textsc{art}  island  that\\
		\ex\label{e3d}
		\gll \textit{ni} \textit{pulo} \textit{ini}\\
		\textsc{art}  island  this\\
	\end{xlist}
\end{exe}

\noindent
We assume that Prentice’s data was collected in the 1980s and 1990s. More recent MM data shows that the pre-predicate slot is more frequently, though not exclusively, filled by the article. Thus, phrases like \hyperref[e:shiohara:2a]{(2a--b)} or \hyperref[e:shiohara:3c]{(3c--d)} are more common than ones like \hyperref[e:shiohara:3a]{(3a--b)}.

In more recent MM data, the form \textit{tu} may co-occur with either the \isi{demonstrative} \textit{itu} or \textit{ini}, as seen in \textit{tu ruma itu} in example (\ref{e:shiohara:4}) and \textit{tu parkara ini} ‘this problem’ in example (\ref{e:shiohara:5}).

\begin{exe}
	\ex\label{e:shiohara:4}
	\gll \textit{Tu} \textit{ruma} \textit{itu} \textit{ancor} \textit{lantaran} \textit{da} \textit{kena} \textit{bom} \textit{waktu} \textit{prang}.\\
	\textsc{art.d} house that broken because \textsc{pst} affected bomb time war\\
	\glt ‘That house is broken because it was bombed in the war.’ \hfill{(PPB:2)}
\end{exe}

\begin{exe}
	\ex\label{e:shiohara:5}
	\gll \textit{Tu} \textit{parkara} \textit{ini} \textit{so} \textit{lama} \textit{nyanda} \textit{klar}-\textit{klar}.\\
	\textsc{art.d} issue \textsc{dem.p} \textsc{pfv} long \textsc{neg} solved\\
	\glt ‘This issue has not been solved (lit. finished) for a long time.’ \hfill{(PPB:62)}
\end{exe}

\noindent
Example (\ref{e:shiohara:4}) and (\ref{e:shiohara:5}) suggest that the form \textit{tu} has undergone semantic bleaching, as it is neutral regarding the distance to the reference point.

The occurrence of the determiners \textit{tu} and \textit{ni} exhibits a syntactic restriction in that they only occur with S, A, and P but not with an oblique. Consider examples (\ref{e:shiohara:6}) and (\ref{e:shiohara:7}) below, which \citet[430]{Prentice1994} provides to show \isi{word order} variation in the MM transitive \isi{clause}. Examples (\ref{e:shiohara:6}) and (\ref{e:shiohara:7}) both denote almost the same proposition in which “I” is the actor, the basket is the location, and the rice is the theme; and the non-agent NP occurs with the determiner \textit{tu} only when it is P.

\begin{exe}
	\ex\label{e:shiohara:6}
	\gll \textit{Kita} \textit{so}  \textit{isi}  \textit{tu}  \textit{loto}  \textit{deng}  \textit{padi}.\\
	\textsc{1sg}  \textsc{pfv}  fill  \textsc{art.d}  basket  with  rice\\
	\glt ‘I have already filled the basket with rice.’
\end{exe}

\begin{exe}
	\ex\label{e:shiohara:7}
	\gll \textit{Kita}  \textit{so}  \textit{isi}  \textit{tu}  \textit{padi}  \textit{di}  \textit{loto}.\\
	\textsc{1sg}  \textsc{pfv}  fill  \textsc{art.d}  rice  at  basket\\
	\glt ‘I have already filled the rice with a basket.’
\end{exe}

\noindent
In possessive structures, the possessor noun or \isi{pronoun} precedes the head (the possessed item) being followed by the possessive marker \textit{pe}, the short form of \textit{punya} ‘have’ in standard \ili{Malay}. \tabref{tab:1} contains the paradigm of possessives with personal pronouns and a lexical noun.

\begin{table}
	\begin{tabularx}{\textwidth}{Xll}
		\lsptoprule
		1\textsc{sg} & \textit{kita / ta pe}  & \textit{kita pe anak} ‘my child’\\
		1\textsc{pl} & \textit{torang / tong pe} & \textit{torang pe anak} ‘our child’\\
		2\textsc{sg} & \textit{ngana pe} & \textit{ngana pe anak} ‘your (\textsc{sg}.) child’\\
		2\textsc{pl} & \textit{ngoni pe} & \textit{ngoni pe anak} ‘your (\textsc{pl}.) child’\\
		3\textsc{sg}\footnotemark & \textit{dia pe / depe}  & \textit{depe anak} ‘his/ her/ its child’\\
		3\textsc{pl} & \textit{dorang / dong pe} & \textit{dorang pe anak} ‘their child’\\
		lexical noun & \textsc{noun} \textit{pe kamar}  & \textit{anak pe} \textit{kamar} ‘a child’s room’\\
		\lspbottomrule
	\end{tabularx}
	\caption{Possessives}
	\label{tab:1}
\end{table}
\footnotetext{The third person pronouns \textit{dia} (\textsc{sg}) and \textit{dorang} (\textsc{pl}) may refer to both animate and inanimate referents, and so may the possessives, as seen in sentence (\ref{e:shiohara:8}) and (\ref{e:shiohara:9}) among others.}

Among the forms presented in \tabref{tab:1}, the long form of the first-person possessive (\textit{kita pe}) and the short form of the third person (\textit{depe}) are not shown in \citet[424]{Prentice1994}. However these forms, especially \textit{depe}, are much more frequently observed in current MM than their alternatives. 

In MM, the possessor is obligatorily marked when the \isi{referent} of the matrix NP is possessed by, or has a part-whole relation to, a \isi{referent} whose \isi{identity} is clear from the previous utterance – thus, in sentences (\ref{e:shiohara:8}) and (\ref{e:shiohara:9}), the possessive obligatorily occurs.

\begin{exe}
	\ex\label{e:shiohara:8}
	\gll \textit{Tu}  \textit{anak}  \textit{pe}  \textit{gaga}. \textit{depe}    \textit{mata}  \textit{basar}  \textit{deng}  \textit{depe}    \textit{mulu}  \textit{kacili}.\\
	\textsc{art.d}  child  \textsc{poss}  beautiful \textsc{3sg.poss}  eyes  big  and  3\textsc{sg.poss}  mouth  small\\
	\glt ‘How beautiful the child is. Her eyes are big, and her eyes are big, and her mouth is small.’ \hfill{(elicited)}
\end{exe}

\begin{exe}
	\ex\label{e:shiohara:9}
	\gll \textit{Sayang}  \textit{ini}  \textit{pohon},  \textit{depe}    \textit{ujung}  \textit{so}  \textit{potong}.\\
	pity  \textsc{dem.p}  tree  3\textsc{sg.poss}  tip  \textsc{pfv}  cut\\
	\glt ‘(This) poor tree, the (its) top has been chopped off.’ \hfill{(elicited)}
\end{exe}

\noindent
The article and possessive may co-occur in pre-head noun position, as seen in examples (\ref{e:shiohara:10}) and (\ref{e:shiohara:11}), suggesting they are assigned to separate syntactic positions.\footnote{As for the status of possessives, \citet[130--134]{Lyons1999} proposed a typological distinction of DG language and AG language; in the former, the possessive is assigned to the determiner position and, in the latter, to the adjectival or some other position. The compatibility of the article and possessive, seen in sentences (\ref{e:shiohara:10}) and (\ref{e:shiohara:11}), suggests that MM belongs to the latter (AG) type.}

\begin{exe}
	\ex\label{e:shiohara:10}
	\gll \textit{Serta} \textit{tu} \textit{depe} \textit{ubi} \textit{milu} \textit{deng} \textit{sambiki} \textit{so} \textit{mandidi}.\\
	after  \textsc{art.d 3sg.poss} potato  corn  and  pumpkin \textsc{pfv}  boil\\
	\glt ‘after the potato, corn and pumpkin are boiled.’ {(elicited narrative, speaker D: 45)}
\end{exe}

\begin{exe}
	\ex\label{e:shiohara:11}
	\gll \textit{Dia} \textit{no} \textit{tu} {\USSlash} \textit{ni} \textit{kita} \textit{pe} \textit{papa}.\\
	\textsc{3sg}  \textsc{ptc}  \textsc{art.d} / \textsc{art.p}  \textsc{1sg}  \textsc{poss}  father\\
	\glt He is my father. (lit. the my father) \hfill{(PPB dictionary:89)}
\end{exe}

\noindent
This co-occurrence also suggests that they each have semantic functions independent of each other. We will return to this point in \sectref{s:shiohara:4}.

As will be seen in the section that follows, the use of \textit{depe} partially overlaps with that of \ili{English} definite article \textit{the}, but not all \textit{depe}-marked NPs refer to a so-called definite \isi{referent}.

In example (\ref{e:shiohara:12}), neither the possessor \textit{ayang} ‘chicken’ or \textit{de} ‘\textsc{3sg}’ in the possessive is referential, but used attributively.\footnote{Note that the antecedent of \textit{depe} in example (\ref{e:shiohara:12}) is the expression \textit{ayang} ‘chicken’, not the \isi{referent} of the expression \textit{ayang} ‘chicken’. (See \citealt[23]{Krifka2012} on the distinction of expression \isi{givenness} and denotation \isi{givenness}.) This type of anaphorical usage is not observed in the third person possessive \isi{pronoun} in many other languages, such as \ili{English} \textit{its} or \textit{nya} in standard \ili{Indonesian}. Thus, the sentence ‘*I like chicken leg meat, but not its breast meat’ cannot be accepted as the \ili{English} counterpart of example (\ref{e:shiohara:12}).}

\begin{exe}
	\ex\label{e:shiohara:12}
	\gll \textit{Kita} \textit{suka} \textbf{\textit{ayang}} \textit{pe} \textit{kaki}, \textit{mar} \textit{nyanda} \textit{suka} \textbf{\textit{depe}} \textit{dada}.\\
	\textsc{1sg} like  chicken  \textsc{poss}  leg  but \textsc{neg}. like \textsc{3sg.poss} breast\\
	\glt ‘I like chicken leg meat, but not chicken breast meat.’ \hfill{(elicited)}
\end{exe}

\noindent
The development of the articles and possessives that we have seen in this section have been observed in other varieties of \ili{Malay}, to a lesser or greater extent. We will give a brief comparison in \sectref{s:shiohara:3}. The variation of the position of the demonstratives and the long and short forms of the third person possessives mentioned above illustrate the transitional status of the two strategies.

\section{\label{s:shiohara:3}Semantic functions of the articles and the possessive construction}

As mentioned in the introduction, MM has developed two types of definite markers, the sources of which are the demonstratives and possessives. Their compatibility in one NP (e.g. \textit{tu depe ruma} ‘the house of him/her/it) implies that each device has a function independent of each other. In this section, we will examine the semantic function of each strategy, mainly based on MM sentences obtained as translations of target sentences from standard \ili{Indonesian} and utterances observed in every day conversation.

\citet[Chapter 3]{Hawkins2015} makes a distinction between four major usage types of the definite article \textit{the}: \isi{anaphoric}, immediate situational, larger situational, and associative \isi{anaphoric} uses. 

The MM articles \textit{ni} and \textit{tu} are used in cases similar to the first two types, that is, \isi{anaphoric} use and immediate situational use. In sentence (\ref{e:shiohara:13}), the two forms are used for making reference to the entity in the speech situation, in sentence (\ref{e:shiohara:14}), one of the two forms \textit{tu} is used for making reference to the entity or situation mentioned in the previous \isi{discourse}.\footnote{(\ref{e:shiohara:14}) is a sentence obtained as a rough translation of sentence (i) below; an example of \isi{anaphoric} use of the \ili{English} definite article is given in \citet[3]{Lyons1999}.\\\ea An elegant, dark-haired woman, a well-dressed man with dark glasses, and two children entered the compartment. I immediately recognized \textbf{the woman}….\z}

\begin{exe}
	\ex\label{e:shiohara:13}
	\gll \textit{Bole}  \textit{pinjam}  \textit{tu} { } {{\USSlash}} \textit{ni}  \textit{pulpen}{\USQMark}\\
	may  borrow  \textsc{art.d} {/} \textsc{art.p}  ballpoint.pen\\
	\glt ‘May I borrow that ballpoint pen?’ \hfill{(elicited)}
\end{exe}

\begin{exe}
	\ex\label{e:shiohara:14}
	\gll \textit{Ada} \textit{parampuang} \textit{gaga} \textit{deng} \textit{dua} \textit{anak} \textit{da} \textit{masuk} \textit{ke} \textit{satu} \textit{ruangan}. \textit{kita} \textit{langsung} \textit{tahu} \textit{sapa} \textit{tu} \textit{parampuan} \textit{itu}.\\
	exist woman beautiful  and two child \textsc{pst} enter to one room \textsc{1sg} directly know who \textsc{art.d} woman \textsc{dem.d}\\
	\glt ‘An elegant lady and two children came in the room. I immediately knew who the woman was.’
\end{exe}

\noindent
These two uses correspond with what \citet[166, 198]{Lyons1999} calls “textual-situational ostension”. According to Lyons, “what these have in common is that the \isi{referent} is immediately accessible.” Lyons suggested that a primary distinction of definiteness should be made between textual-situational ostension and other usages. The former functionally overlaps with demonstrativeness, and the others do not. A similar view is presented in many previous studies, such as \citet[Chapter 3] {Hawkins2015}, \citet{Himmelmann1996}, and \citet[528]{deMulder2011}. 

Demonstratives are a well-known source of definite markers in many languages, as suggested by \citet{Heine2002} and \citet{Lyons1999} among others. \citet[528]{deMulder2011} suggest that the crucial semantic shift from demonstratives to the definite article is seen from \textit{direct reference} that corresponds to the direct situational use and \isi{anaphoric} use of Hawkins, to \textit{indirect reference}, which corresponds to \isi{anaphoric} associative use and larger situational use. 

Notwithstanding the distinct syntactic position in NP from the demonstratives, the uses of the articles in MM have not undergone a semantic shift and have not extended beyond direct reference. Instead, indirect uses are covered by the third person possessive \textit{depe} ‘\textsc{3sg.poss’} in MM. In the \isi{anaphoric} associative use of \textit{the}, the NP refers to something associable to the \isi{referent} of a previously mentioned NP, while in the larger situational use, the NP refers to something associable to the situation of the utterance itself. In both uses, the hearer is supposed to use shared general knowledge for identification; the hearer and the speaker need to know the \isi{referent} is associable to the antecedent or the utterance situation in question.

Sentences (\ref{e:shiohara:15}--\ref{e:shiohara:16}) are examples of \isi{anaphoric} associative uses.\footnote{Example (\ref{e:shiohara:16}) is obtained as a rough MM equivalent of sentence (ii) below; an example of associative \isi{anaphoric} use of \ili{English} \textit{the} given in \citet[3]{Lyons1999}.(ii) ‘I have just come back from a wedding party. The bride wore blue.’}

\begin{exe}
	\ex\label{e:shiohara:15}
	\gll \textit{Kita}  \textit{lebe}  \textit{suka}  \textit{Australia}  \textit{daripada}   \textit{Jepang} \textit{karna}   \textit{depe}     \textit{sayur}-\textit{sayur}   \textit{lebe}   \textit{sadap}  \textit{deng}  \textit{murah}.\\
	\textsc{1sg}  more  like  Australia  from    Japan  because \textsc{3sg.poss} vegetable.\textsc{red} more  tasty  and   cheap\\
	\glt ‘I like Australia more than Japan, because vegetables there are tastier and cheaper.’ \hfill{(elicited)}
\end{exe}

\begin{exe}
	\ex\label{e:shiohara:16}
	\gll \textit{Kita} \textit{baru} \textit{pulang} \textit{dari} \textit{pesta} \textit{kaweng}. \textit{Depe} \textit{broid} \textit{ta} \textit{pe} \textit{tamang}.\\
	\textsc{1sg} just come.back from party wedding. \textsc{3sg}.\textsc{poss} bride \textsc{1sg} \textsc{poss} friend\\
	\glt ‘I have just come back from a wedding party. The bride was a friend of mine.’ \hfill{(elicited)}
\end{exe}

\noindent
Employment of the third person possessive \textit{depe} for this use can be easily explained by its original meaning; the possessive \textit{depe} includes \textit{de}, the shortened form of the third person \isi{pronoun} \textit{dia} ‘\textsc{3sg}’. The \isi{pronoun} \textit{dia} may be used as an anaphor, and in the possessive, it indicates the presence of a whole to part relation between the \isi{referent} of the \isi{pronoun} and the matrix NP.

From sentences (\ref{e:shiohara:15}) and (\ref{e:shiohara:16}) above, we can see that the semantic relation between the possessor and the head noun is not limited to the simple whole to part relation that is exemplified in sentence (\ref{e:shiohara:8}) and (\ref{e:shiohara:9}) shown in \sectref{s:shiohara:2}. There may be various relations, such as location, as seen in example (\ref{e:shiohara:15}), and occasion, as in example (\ref{e:shiohara:16}).

However, the semantic range the possessive covers does not seem to perfectly overlap with that of \isi{anaphoric} associative \textit{the}. Consider example (\ref{e:shiohara:17}), which \citet[3]{Lyons1999} gives as one of the examples of associative use of the \ili{English} definite article.

\begin{exe}
	\ex\label{e:shiohara:17} I had to get a taxi from the station. On the way, \textbf{the driver} told me there was a bus strike. 
\end{exe}

\noindent
In sentence (\ref{e:shiohara:18}), a rough MM equivalent of sentence (\ref{e:shiohara:17}), the counterpart of the \ili{English} definite NP does not receive any explicit marking, as seen in sentence (\ref{e:shiohara:18}).

\begin{exe}
	\ex\label{e:shiohara:18}
	\gll \textit{Ni} \textit{hari} \textit{kita} \textit{da} \textit{nae} \textit{taksi} \textit{dari} \textit{stasion}. \textit{Di} \textit{tenga} \textit{jalang} {\USOParen}{\USStar}\textit{depe} {\USSlash} {\USStar}\textit{tu}{\USCParen} \textbf{\textit{sopir}} \textit{se} \textit{tau} \textit{tadi} \textit{ada} \textit{cilaka} \textit{brat}.\\
	\textsc{art.p} day \textsc{1sg} \textsc{pft} ride taxi from station at middle way  \phantom{(*}\textsc{3sg.poss} { } \phantom{*}\textsc{art.d} driver \textsc{caus} know before exist accident heavy\\
	\glt ‘I had to get a taxi from the station. On the way \textbf{the driver} told me there was a serious accident.’ \hfill{(elicited)}
\end{exe}

\noindent
In this situation, we can reasonably associate the \isi{referent} of \textit{sopir} ‘the driver’ to \textit{taksi} ‘a taxi’, and that is the reason the NP undergoes the definite marking in \ili{English} sentence (\ref{e:shiohara:17}), but that is not the case in MM. The reason may be that the semantic relation between \textit{sopir} ‘the driver’ and the associated \textit{taksi} ‘a taxi’ cannot be taken as a possessor-possessed, or whole to part relation, to the MM speakers; one of the MM speakers suggested that he could not use the possessive \textit{depe} here because the driver possessed the taxi, not the reverse. This example may show the difference between the \ili{English} definite article in associative use and MM possessives; the former may indicate any type of association, while the latter exhibits some limitations which presumably are attributed to the original possessive meaning. At the present stage of our research, however, we do not have enough data to provide the precise condition where the possessive may or may not occur.\footnote{We might be able to infer that if the ‘possessed’ NP is animate and the ‘possessor’ NP is inanimate, the marking with \textit{depe} may not be permitted, as it contradicts the concept of possession we intuitively would have.}

The use of \textit{depe} in example (\ref{e:shiohara:19}) and (\ref{e:shiohara:20}) overlaps with the “larger situational use” of \textit{the} in Hawkins’s classification, where the \isi{referent} of the \textit{depe} NP is associable to the utterance situation. Note that there is no clear antecedent of the possessive in these examples.

In sentence (\ref{e:shiohara:19}), the NP \textit{depe} \textit{cuaca} refers to the weather of the place the speaker and hearer are located in.

\begin{exe}
	\ex\label{e:shiohara:19}
	\gll \textit{Depe}    \textit{cuaca}    \textit{bae}.\\
	\textsc{3sg.poss}  weather  good\\
	\glt ‘The weather is nice (today).’ (spontaneous utterance obtained from daily conversation)
\end{exe}

\noindent
The sentences in (\ref{e:shiohara:20}) are from a Facebook post. Example \hyperref[e:shiohara:20a]{(20a)} is the original Facebook post made with a picture of yams, and \hyperref[e:shiohara:20b]{(20b)} and \hyperref[e:shiohara:20c]{(20c)} are comments posted by two friends of the poster. In both comments, \textit{ubi} ‘yam’ mentioned in the original post is marked by \textit{tu} and \textit{depe}, and the antecedent of \textit{depe} is not explicitly mentioned.

\begin{exe}
	\ex\label{e:shiohara:20}
	\begin{xlist}
		\ex\label{e:shiohara:20a}
		\gll {\textit{Slamat} \textit{pagi}},  \textit{panen}  \textit{ubi}  \textit{jalar}  \textit{serta}  \textit{menanam}  \textit{ulang}.\\
		good.morning  harvest  yam  spread  after  plant  again\\
		\glt ‘Good morning, harvesting yams and then planting them again.’
		\ex\label{e:shiohara:20b}
		\gll \textit{Mantaap} \textit{Beng} \textit{pe} \textit{besar}-\textit{besar} \textit{kang}   \textit{tu} \textbf{\textit{depe}}  \textbf{\textit{ubi}}{\USQMark}\footnotemark \\
		great  Beng  very  big.\textsc{red}  \textsc{itr}  \textsc{art.d}  \textsc{3sg}.\textsc{poss}  yam\\
		\glt ‘Great Beng the (lit. the its) potatoes are very big, aren’t they?’
		\ex\label{e:shiohara:20c}
		\gll \textit{Banyak}  \textit{tu} \textbf{\textit{depe}}    \textbf{\textit{batata}} \textit{ada}  \textit{panen}.\\
		many  \textsc{art.d}  \textsc{3sg}.\textsc{poss}  sweet.potato  \textsc{pst}  harvest\\
		\glt ‘Lots of the (lit. the its) sweet potatoes were harvested.’
	\end{xlist}
\end{exe}
\footnotetext{The commentator uses the spelling of \textit{bsr}2 and \textit{dp} for \textit{besar-besar} and \textit{depe}, respectively, in her original post.}

\noindent
The lack of a clear antecedent\footnote{We asked the commenter to identify the antecedent of \textit{depe} in sentence \hyperref[e:shiohara:20c]{(20)c} several times, but her answers were not consistent. Her response may show that the \isi{referent} of the antecedent is not a concrete entity that can be clearly mentioned. We might be able to insist that the third person \isi{pronoun} \textit{de} refers to the implied “shared situation”, but the claim may not be accepted, because the third person \isi{pronoun} \textit{dia}, from the long form of \textit{de} in \textit{depe}, may not refer to the situation or proposition. Consider the three pairs of sentence (i). A situation or proposition can be referred to only by the \isi{demonstrative} \textit{begitu}, not by the third person \isi{pronoun} \textit{dia}.
\begin{exe}
	\ex\label{e:shiohara:footnote9}
	\begin{xlist}
		\ex\label{e:shiohara:footnote9a}
		\gll Albert  so  nya mo pusing deng orang laeng pe emosi. So  bagitu Albert  pe  kalakuan.\\
		Albert  \textsc{pfv}  \textsc{neg}  want  bothered  with  person  other  \textsc{poss}  emotion \textsc{pfv}  like.that	Albert  \textsc{poss}  behavior\\
		\glt ‘Albert doesn’t want to be bothered with other people's feelings.~The character of Albert is like that.’ \hfill (elicited)
		\ex\label{e:shiohara:footnote9b}
		\gll Albert  so  nya mo pusing deng orang laeng pe emosi. {\USStar}Dia  Albert  pe  kalakuan.\\
		Albert  \textsc{pfv}  \textsc{neg}  want  bothered  with  person  other  \textsc{poss}  emotion \phantom{*}3\textsc{sg}  Alert  \textsc{poss}  behavior\\
		‘(Intended meaning) Albert doesn’t want to be bothered with other people’s feelings. That’s the character of Albert.’
	\end{xlist}
\end{exe}} in sentences (\ref{e:shiohara:19}) and (\ref{e:shiohara:20}) shows that the form \textit{depe} does not function as the possessive marker. Instead, we can claim that the form \textit{depe} plays a similar semantic role to the larger situational use of \textit{the}, whichever label we give to it in MM grammar. In this use, the \isi{referent} is identified by two processes: one is identifying the nature of the “shared” larger situation intended by the speaker, and the other is identifying the \isi{referent} using the “shared” knowledge that presupposes the existence of the \isi{referent} in the situation \citep{Hawkins2015}.

A similar type of development from the possessive to the definite marker is observed in other languages that are not genetically related, such as \ili{Amharic} \citep{Rubin2010} and \ili{Yucatec} Maya \citep[86--88]{Lehmann1998}, as well as colloquial \ili{Indonesian}, as mentioned in \sectref{s:shiohara:3}. This development can be explained by an affinity between the association and indication of the part-whole relation. \citet[123--124]{Hawkins2015}, in discussing the similarity of associative \isi{anaphoric} and larger situational use, claims that “(T)he notion ‘part-of’ seems to play an important role in defining the number of possible associates. The trigger (of the association) must conjure up a set of objects which are generally known to be part of some larger object or situation.” (For a more recent and precise discussion of the development from possessive to definite marker, see \citealt{Fraund2001}; \citealt{Gerland2014, Gerland2015}).

Other varieties of \ili{Malay} exhibit similar developments to a greater or lesser extent. \citet[675]{Adelaar1996} suggest that the use of the short form of the demonstratives \textit{ni} and \textit{tu} as well as forms such as \textit{pu} or \textit{pun} (derived from \textit{punya} ‘have’ as possessive marker) are among several morphosyntactic features shared among trade \ili{Malay} varieties, which \citet[675]{Adelaar1996} call \ili{Pidgin}-Derived \ili{Malay} (PDM) varieties. Regarding the development of demonstratives into the definite markers, \citet[212--217]{Adelaar2005} points out the \isi{anaphoric} use of the short forms of demonstratives \textit{tu} and \textit{ni} in \ili{Ambon} \ili{Malay} and Cocos \ili{Malay}; they also underwent semantic bleaching similar to that of MM. Similar types of development are reported in both Papuan \ili{Malay} \citep[384--388]{Kluge2017} and Ternate \ili{Malay} \citep[263, 277]{Litamahuputty2012}.

The development of the possessive into a definite marker is also observed in colloquial \ili{Indonesian}, in which the third person possessive enclitic =\textit{nya} is used to indicate identifiability, exhibiting functions similar to MM \textit{depe} in associative \isi{anaphoric} use and larger situational use \citep[161--168]{Englebretson2003}. A rather different distribution was observed in \ili{Baba Malay}, spoken in Malaka and Singapore by “Strait-born” \ili{Chinese}. In \ili{Baba Malay}, the articles \textit{ini} and \textit{itu} cover larger semantic domains, including associative \isi{anaphoric} use and larger situational use \citep[477--480]{Thurgood2001}, although the third person possessive suffix -\textit{nya} also has similar functions to the articles \citep[132--135]{Thurgood1998}.

\section{\label{s:shiohara:4}Determiners and possessives in elicited procedural text}

\subsection{\label{s4.1}Method}

In this section, we will see larger texts elicited by a short cooking video as stimulus to confirm the syntactic and semantic functions of the two strategies outlined in the previous sections. The advantage of employing this method for elicitation is that (i) this type of non-linguistic stimulus enables us to collect more naturalistic data without the influence of a medium language, and (ii) the reference tends to be clear in the text obtained through this method when compared to purely spontaneous utterance in which the \isi{referent} of each NP may not always be easily identified (see \citealt{Majid2012} for details of elicitation methods using stimulus materials.)

The video employed as stimulus here is titled \textit{Tinutuan} ‘Manadonese porridge’. The video was shot by one of the authors and is available from \url{https://youtu.be/cyJanYZjXoo}. We asked four speakers of MM (H, I, D and A) to watch the video and give a commentary in MM. In the video, the main dish \textit{tinutuan} ‘Manadonese porridge’ and the side dishes \textit{tahu} \textit{goreng} ‘fried tofu’ and \textit{dabu-dabu} ‘chili sauce’ are cooked. The outline of the cooking process is shown in \tabref{tab:cookingprocess3}.

\begin{figure}[p]
		\includegraphics[width=\textwidth]{figures/ShioharaJukesFINALkorrektur-img1}
		\caption{\textit{Tinutuan} ‘Manadonese porridge’, \textit{tahu goreng} ‘fried tofu’ and \textit{dabu-dabu} ‘chili sauce’.}
		\label{fig:shiohara:1}
\end{figure}

\begin{table}[p]
\begin{tabularx}{.8\textwidth}{lcll}
	\lsptoprule
	{Name} & {Age} & {From} & {Mother tongue}\\
	\midrule
	H & 65 & Beo, \ili{Talaud} & \ili{Talaud}\\
	I & 36 & Beo, \ili{Talaud} & Manado \ili{Malay}\\
	D & 34 & Sonder, Minahasa & Manado \ili{Malay}\\
	A & 55 & Tomohon, Minahasa & Tombulu\\
	\lspbottomrule
\end{tabularx}
	\caption{MM speakers who provided the narrative}
	\label{tab:2}
\end{table}

\begin{table}[p]
\caption{Outline of the cooking process\label{tab:cookingprocess3}}
\begin{tabularx}{.8\textwidth}{lX}
\lsptoprule
	Scene 1: & showing ingredients\\	
	Scene 2: & cut and peel hard vegetables such as yam and pumpkin\\
	Scene 3: & put the vegetables and rice into a pan and heat them\\
	Scene 4: & cut and wash the leafy vegetables\\
	Scene 5: & mash the pumpkin in the pan, put the leafy vegetables in the pan and mix all the ingredients\\	
	Scene 6: &prepare the side dish \textit{tahu goreng} (fried tofu)\\	
	Scene 7: & prepare \textit{dabu-dabu} (chili sauce)\\
	Scene 8: & serve the dish\\
    \lspbottomrule
\end{tabularx}
\end{table}

\subsection{\label{s4.2}Results}

\subsubsection{\label{s4.2.1}Referent and general referential strategies observed}

There are 45 entities mentioned in the narrations of the four speakers; the range of entities that each speaker mentioned varies depending on the speaker, and the term for the same entity may vary among speakers, too. The referents can be grouped into the semantic categories below.

\begin{itemize}
	\item The speaker (1 type): \textit{Isye}
	\item The name of dishes (3 types): \textit{tinutuan} ‘porridge’, \textit{tahu goreng} ‘fried tahu’, and \textit{dabu-dabu} or \textit{laburan} ‘chili sauce’
	\item Ingredients (1 type): \textit{bahan-bahan} ‘ingredients’
	\item Base ingredients, i.e. root vegetables and rice (6 types): \textit{ubi} ‘potato’, \textit{batata} or \textit{ubi manis} ‘sweet potato’, \textit{ubi kayu} ‘cassava’, \textit{sambiki} ‘pumpkin’, \textit{milu} ‘corn’, \textit{beras merah}, \textit{beras} ‘rice’, \textit{aer} ‘water’
	\item Leafy vegetables (6 types): \textit{sayor} ‘leafy vegetables’, \textit{bayam} ‘amaranth vegetable’, \textit{kangkung} ‘water spinach’, \textit{gedi} ‘aibika leaf’, \textit{kukuru}, \textit{balakama} ‘basil’, \textit{sarimbata}, \textit{baramakusu}, \textit{goramakusu} ‘lemongrass’
	\item Ingredients for side dishes (8 types): \textit{tahu} ‘soybean curd, tofu’, \textit{bawang merah} ‘shallot’, \textit{bawang putih} ‘garlic’, \textit{garam} ‘salt’, \textit{tomat} ‘tomato’, \textit{rica} ‘chili’, \textit{ikan roa} ‘dried fish’, \textit{minyak kelapa} ‘coconut oil’
	\item An attribute or a part of ingredients (4 types): \textit{kuli} ‘skin’, \textit{daong} ‘leaf’, \textit{isi} ‘contents, edible part of vegetable’, \textit{warna} (\textit{kuning}) ‘(yellow) color’
	\item Cooking tools and so on (6 types): \textit{blanga}/\textit{panci} ‘pan’, \textit{kompor} ‘stove’, \textit{mangko} ‘bowl’, \textit{pantumbu} ‘pestle’, \textit{cobe-cobekan} ‘mortar’, \textit{piso} ‘knife’
	\item Body parts of the cook (2 types): \textit{tangan} ‘hand’, \textit{jare} ‘finger’
	\item Others (8 types): \textit{cacing} ‘worm’, \textit{vitamin} ‘vitamin’, \textit{kelihatan} ‘appearance’, \textit{nama} ‘name’, \textit{priksaan} ‘test’, \textit{hasil} ‘result, \textit{pedis} ‘spicy (n)’, \textit{orang} ‘person’
\end{itemize}

\noindent
The text length and number and varieties of the referents mentioned vary among the speakers. \tabref{tab:3} shows the number of words and referents included in each narrative.

\begin{table}
	\begin{tabularx}{.8\textwidth}{Xrr}
		\lsptoprule
		Speaker & Words included & Types of \isi{referent} mentioned\\
		\midrule
		I & 444 & 41\\
		H & 525 & 38\\
		D & 336 & 27\\
		A & 478 & 32\\
		\lspbottomrule
	\end{tabularx}
	\caption{The number of words and referents included in each narrative}
	\label{tab:3}
\end{table}

\noindent
Each \isi{referent} can be expressed by either a personal \isi{pronoun}, a \isi{demonstrative} \isi{pronoun}, or a lexical NP. \tabref{tab:4} counts the occurrences of each strategy.

It should be noted that the argument of the predicate is not expressed when it is salient in \isi{discourse}; category zero counts such arguments. 

\begin{table}
	\begin{tabularx}{.8\textwidth}{Xrrrr}
		\lsptoprule
		Speaker & Zero & Personal  & Demonstrative & Lexical NP\\
		&& \isi{pronoun} &\isi{pronoun} &\\
		\midrule
		I & 131 & 8  &  18 & 105\\
		H & 111 & 5  &  12 & 95\\
		D & 112 & 1  &  4 & 78\\
		A & 117 & 6  &  8 & 89\\
		\lspbottomrule
	\end{tabularx}
	\caption{Occurrence of each strategy}
	\label{tab:4}
\end{table}

\noindent
The actor (the cook) is not mentioned at all in three of the four narratives and is mentioned only once (by the third person singular \isi{pronoun} \textit{dia}) in the remaining narrative. Other non-agent arguments are also often not expressed; a series of cooking processes is expressed by a co-ordinate \isi{clause}, and the entity mentioned in the first \isi{clause} is not expressed in the clauses that follow it. Consider sentence (\ref{e:shiohara:21}), which consists of coordinate clauses expressing a series of actions processing garlic. Here, the actor does not occur throughout the sentence, and the patient, \textit{bawang putih} ‘garlic’ occurs only once in the first \isi{clause}, but not in the three clauses that follow.

\begin{exe}
	\ex\label{e:shiohara:21}
	\gll \textit{Aa}   \textit{kase}  \textit{ancor} \textit{bawang} {{\USSmaller}\textit{me}-{\USGreater}\footnotemark}  \textit{bawang}  \textit{putih} \textit{so}  \textit{kase}   \textit{ancor}   $\phi$  \textit{lagi} \textit{iris-iris} $\phi$  \textit{lagi}     \textit{hh} \textit{campur} $\phi$  \textit{di}   \textit{tahu}.\\
	\textsc{itj}  \textsc{caus}  crush  onion  <re-> onion white \textsc{pfv}  \textsc{caus} crush  { } again  slice { } again \textsc{itj} mix { } at tofu\\
	\glt ‘Aa…(she) crushes the onion…the garlic, after crushing, (she) will slice (it) and mix (it) with tofu.’ \hfill{(speaker H 37--38)}
\end{exe}
\footnotetext{In this utterance, the speaker started to say bawang merah 'shallot', and then corrected herself saying bawang putih ‘garlic’.}

\noindent
In what follows, we focus on how lexical NPs are marked with the articles and/or the possessive. A lexical NP may occur (i) in unmarked form, that is, a bare NP, (ii) with the article \textit{tu} or \textit{ni}, (iii) with possessives \textit{depe} or \textit{dia pe}, (iv) with both the article \textit{tu} and the possessive, (v) with a postposed \isi{demonstrative}, or (vi) with =\textit{nya}, the third person singular possessive enclitic used in standard \ili{Indonesian}. 

Most of the possessives are that of the third person singular \textit{depe} in the text; the text includes only one example of the lexical noun possessor, \textit{sambiki le pe kuli} [pumpkin also \textsc{poss} skin] ‘pumpkin’s skin’.

\tabref{tab:5} shows the occurrence of the article and the possessive construction.

\begin{table}
	\resizebox{\textwidth}{!}{\begin{tabular}{lrrcrrccrr} 
		\lsptoprule
		& Sum of the  & Unmarked & \textsc{art} & POSS & \textit{tu} & \multicolumn{2}{c}{\textsc{dem}} & =\textit{nya} & Others \\\cmidrule(lr){7-8}
		&  lexical NPs & & (\textit{tu}/\textit{ni}) & & + POSS & Pre-posed  & Post-posed \\
		\midrule
		I & 105 & 62 & 32
		(31/1) & 8 & 2 & 0 & 0~ & 0 & 0\\
		H & 95 & 45 & 3 (1/2) & 41 & 0 & 0 & 2\footnote{(\textit{ni}+\textit{ini})} & 3 & 1\\
		D & 79 & 47 & 19
		(14/2) & 9 & 2 & 3 & 1~ & 0 & 1\\
		A & 88 & 77 & 6
		(6/0) & 3 & 0 & 0 & 1~ & 0 & 1\\
		\lspbottomrule
	\end{tabular}}
	\caption{Occurrence of the determiner and the possessive construction}
	\label{tab:5}
\end{table}

\noindent
As observed in \sectref{s:shiohara:2}, in current MM the pre-head noun position is much more frequently filled by the article than by a \isi{demonstrative}. This data confirm the observation; we can see only 3 instances of pre-head noun demonstratives compared to 65 instances of articles. We also mentioned the variation in form of the third person singular possessive. The short form \textit{depe} occurs much more frequently (66 examples) than the long form \textit{dia pe} (3 examples).

The individual narratives exhibit considerable variation in the frequency with which each speaker uses the two strategies – the determiner and the possessive. For example, speaker I prefers to use the article, while speaker H prefers the possessive \textit{depe}. Speaker D uses both in similar frequencies, while speaker A rarely uses either of the markers.

Notwithstanding the difference in preference in using each device, the use in the text maintains the basic semantic function of the determiners and the possessive, which we have shown in \sectref{s:shiohara:3}; the articles mark a textual-situationally given \isi{referent}, while the possessive \textit{depe} or \textit{dia pe} marks a \isi{referent} associable to a given \isi{referent} or utterance situation.

\tabref{tab:6} shows the distribution of NPs with an article and the possessive \textit{depe} in a textually accessible environment.

\begin{table}
	\begin{tabularx}{.8\textwidth}{Xll}
		\lsptoprule
		& Articles & Possessive \textit{depe}\\
		\midrule
		Total & 59 & 61\\
		Not textually accessible & 3 & 30\\
		Textually accessible & 56 & 31\\
		\lspbottomrule
	\end{tabularx}
	\caption{The distribution of the articles and possessive \textit{depe}}
	\label{tab:6}
\end{table}

\noindent
Because of the nature of the text, most of the referents are visible to both the speaker and addressee.\footnote{The addressee in any given narrative is whichever of the authors was present at the time of recording.} That makes it difficult to verify how direct situational accessibility affects both devices. The fact that a considerable number of NPs were not marked by either of the devices, however, suggests that situational accessibility is not a crucial factor for either of the markings.\largerpage

Regarding textual-accessibility, we can see a clear difference of frequency between the articles and possessives. In the 59 occurrences of the NP marked with articles in total, 56 refer to a textually accessible – in other words, previously mentioned, entity.

In contrast to the articles, as expected by the observation of \sectref{s:shiohara:3}, textual-accessibility does not influence the use of the possessive \textit{depe}. 

In the following sections, we will see the details of how each strategy works in the text.

\subsubsection{\label{s4.2.2}Textually accessible use of the article}

As mentioned above, in almost all the occurrences the NPs marked with an article refer to a textually accessible \isi{referent}. The frequency of \textit{tu} is far higher than that of \textit{ni}, as seen in \tabref{tab:5}, which supports Prentice’s view that \textit{ni} is semantically marked (see \sectref{s:shiohara:3}). From the text obtained by the experiment, though, we could not clearly see the functional difference between the two articles.

As mentioned in \sectref{s:shiohara:2}, the determiner \textit{tu} occurs with core arguments (S, A, and P). However, not all textually given S and P referents are marked by the determiner. \tabref{tab:7} shows the frequency of use of the determiner for textually given S and P referents.\footnote{No given A occurs in the four texts.}

\begin{table}
	\begin{tabularx}{.8\textwidth}{Xcc}
		\lsptoprule
		Speaker & Textually accessible ASP & Marked by \textsc{art}\\
		\midrule
		I & 55 & 38 (69\%)\\
		H & 30 & 3 (10\%)\\
		D & 35 & 21 (60\%)\\
		A & 35 & 6 (20\%)\\
		\lspbottomrule
	\end{tabularx}
	\caption{The frequency of the form \textit{tu} and \textit{depe} marking for a given S and P referent}
	\label{tab:7}
\end{table}

\noindent
The preference varies among the speakers. Speaker I and D more frequently used \textit{tu} than the other two speakers. They are younger than the other speakers, and so this may represent a change in progress.

\subsubsection{\label{s4.2.3}The use of the possessive pronoun}

As shown in \sectref{s:shiohara:2} and \sectref{s:shiohara:3}, the possessive covers \isi{anaphoric} associative use as a part of its possessive meaning and also covers the larger situational use of \citet{Hawkins2015} as a result of semantic change.

The obligatory marking of the possessor mentioned in \sectref{s:shiohara:2} is attested by the narratives. Sentence (\ref{e:shiohara:22}) is a typical example.

\begin{exe}
	\ex\label{e:shiohara:22}
	\gll \textit{Serta}  \textit{so}  \textit{ta}-\textit{kaluar}   \textit{depe}     \textit{kuli}, \textit{mo}   \textit{kupas}   \textit{lei}   \textit{tu}   \textit{sambiki}.\\
	after  \textsc{pfv}  \textsc{pass}-peel  \textsc{3sg.poss}   skin \textsc{fut}  peel  again  \textsc{art.d}  pumpkin\\
	\glt ‘After all the peel has been removed, (then she) will peel the pumpkin, too.’ \hfill{(H 13)}
\end{exe}

\noindent
The form \textit{depe} in (\ref{e:shiohara:22}) retains its possessive meaning and indicates that the \isi{referent} of the whole NP is associable to the \isi{referent} of previously mentioned NP. In actual sentences, the associative use and larger situation use cannot always be separated clearly. 

Consider sentence (\ref{e:shiohara:23}). This is the first sentence in scene 6 (preparation of a tofu dish), and the antecedent of \textit{depe} in the NP \textit{depe tahu} ‘the tofu’, is not clear, or is at least unavailable in clauses that directly precede sentence (\ref{e:shiohara:23}).

\begin{exe}
	\ex\label{e:shiohara:23}
	\gll \textit{Skarang}  \textit{mo}  \textit{bekeng} \textbf{\textit{depe}} \textit{tahu}.   \textit{tahu} \textit{taro} \textit{di} \textit{panci}.\\
	now    \textsc{fut}  do \textsc{3sg.poss}   tofu   tofu put at pan\\
	\glt ‘Now (we) want to make the tofu. Put the tofu in the pan.’ \hfill{(I 052)}
\end{exe}

\noindent
Sentence (\ref{e:shiohara:24}) provides a similar example. This is the first sentence in scene 7 (preparation of chili sauce), and the antecedent of \textit{depe} in the NP \textit{depe laburan} ‘the sauce’, is not clear, or at least is unavailable in the clauses that directly precede it.

\begin{exe}
	\ex\label{e:shiohara:24}
	\gll \textit{Itu}   \textit{mo}   \textit{bekeng}   \textit{depe}  \textit{laburan}.\\
	that  \textsc{fut}  make    \textsc{3sg.poss}  sauce\\
	\glt ‘There (she) is going to cook the (its) sauce’. \hfill{(D 80)}
\end{exe}

\noindent
According to the speaker, in both cases, the possessor is the main topic of the whole text: \textit{tinutuan} ‘Manado porridge’, fried tofu and chili sauce always come together with the porridge as a side dish and can be considered a part of the dish.

The dish \textit{tinutuan} does have prior mention and we could therefore say that sentences (\ref{e:shiohara:23}) and (\ref{e:shiohara:24}) are examples of \isi{anaphoric} associative use. But the prior mention of \textit{tinutuan} is made in the very beginning of the whole narrative — far from sentences (\ref{e:shiohara:23}) and (\ref{e:shiohara:24}) (51 and 78 clauses away from each \textit{depe} NP, respectively). It is therefore difficult to consider the NP \textit{tinutuan} to be antecedent of the possessive \textit{depe}. It may be more plausible to think that the \isi{referent} of \textit{depe} NP is associable with the larger situation in which the utterance was made, that is, watching, and talking about, the cooking process of \textit{tinutuan}.

\tabref{tab:8} shows the frequency with which each speaker uses \textit{depe}; each use is classified into those that have an antecedent available in directly preceding clauses – in other words, associative \isi{anaphoric} use and larger situational use.

\begin{table}
	\begin{tabularx}{\textwidth}{Xrrrr} 
		\lsptoprule
		& Lexical NP & \multicolumn{3}{c}{Possessive}\\\cmidrule{3-5}
		&  & Sum & Associative \isi{anaphoric} use & Larger situational use\\
		\midrule
		I & 105 & 10 & 6 & 4\\
		H & 96 & 43 & 13 & 30\\
		D & 78 & 12 & 3 & 9\\
		A & 89 & 5 & 4 & 1\\
		\lspbottomrule
	\end{tabularx}
	\caption{Frequency of the form \textit{depe}}
	\label{tab:8}
\end{table}

\noindent
Differences among speakers are observed in their use of larger situational \textit{depe}. 

As seen in \tabref{tab:8}, one of the four speakers (Speaker H) showed a marked preference for wider topic \textit{depe}, while Speaker I did that to a lesser extent. Speaker H’s distinct use of \textit{depe} is clearly seen in the beginning of his narrative, where he introduces ingredients immediately after the title \textit{tinutuan} ‘Manado Porridge’ is shown. Sentence (\ref{e:shiohara:25}) shows that part; here, speaker H marked the NP expressing ingredients with \textit{depe} ‘\textsc{3sg.poss}’.

\begin{exe}
	\ex\label{e:shiohara:25}
	\begin{xlist}
		\ex\label{e:shiohara:25a}
		\gll \textit{Mo} \textit{bekeng} \textit{masakan} \textit{nama}-\textit{nya} \textit{tinutuan}.\\
		want make food name-\textsc{3sg}.\textsc{poss(BI)} tinutuan\\
		\glt ‘(She) wants to cook food named tinutuan.’ \hfill{(Speaker H: 01)}
		\ex\label{e:shiohara:25b}
		\gll \textit{Ado} \textit{e} \textit{pe} \textit{sadap} \textit{skali} \textit{ini}, \textit{aah} \textit{ini} \textit{batata}, \textit{depe} \textit{batata}, \textit{depe} \textit{ubi}.\\
		\textsc{itj} \textsc{itj} \textsc{itj} delicious very this \textsc{itj} this sweet.potato, \textsc{3sg.poss} sweet.potato, \textsc{3sg.poss} yam\\
		\glt ‘Oh, it is very delicious, this is sweet potato, the sweet potato, the yam.’
	\end{xlist}
\end{exe}

\noindent
Unlike H, the other three speakers introduce the ingredients without any marking. In sentence (\ref{e:shiohara:26}), speaker I describes the same scene.

\begin{exe}
	\ex\label{e:shiohara:26}
	\gll \textit{Bahan-bahan},  \textit{bete},  {\textit{ubi} \textit{kayu}}  \textit{sambiki}  \textit{milu}…\\
	ingredients    taro  sweet.potato  pumpkin  corn\\
	\glt ‘Ingredients…taro, sweet potato, pumpkin, and corn…’ \hfill{(Speaker I: 02)}
\end{exe}

\noindent
Differences among speakers are also seen in the description that follows (\ref{e:shiohara:25}) and (\ref{e:shiohara:26}), respectively, which explains the cooking procedure. Sentence (\ref{e:shiohara:27}) is a description that follows sentence (\ref{e:shiohara:25}). Speaker H keeps employing \textit{depe} for referring to the ingredients given in the previous part of his utterance; here, one of the ingredients \textit{batata} ‘sweet potato’ is marked with \textit{depe}.

\begin{exe}
	\ex\label{e:shiohara:27}
	\gll \textit{Aah} \textit{sekarang} \textbf{\textit{depe}} \textit{batata} \textit{mo} \textit{di}-\textit{kupas} \textit{kase} \textit{kaluar} \textit{depe} \textit{kuli}.\\
	\textsc{itj} now \textsc{3sg.poss} sweet.potato \textsc{fut} \textsc{pass}-peel give go.out \textsc{3sg.poss} skin \\
	\glt ‘Ah, now (she) is going to peel the potato, peel off the skin.’ \hfill{(Speaker H: 11)}
\end{exe}

\noindent
In contrast to that, speaker I employs \textit{tu} to mark all the ingredients that were given in the preceding part of the utterance. Sentence (\ref{e:shiohara:28}) is a part of the description that follows sentence (\ref{e:shiohara:26}).

\begin{exe}
	\ex\label{e:shiohara:28}
	\gll \textit{Pertama}  \textit{kase}   \textit{bersi}   \textit{tu}   \textit{bete},   \textit{kupas} \textit{depe}   \textit{kuli}.\\
	first    \textsc{caus}  clean  \textsc{art.d}  taro  peel  \textsc{3sg.poss}  skin\\
	\glt ‘First, clean the taro, and peel its skin.’ \hfill{(Speaker I: 18)}
\end{exe}

\noindent
It should be noted that all the speakers use both strategies to a greater or lesser extent. Speaker H, who very frequently uses \textit{depe}, also uses \textit{tu} twice to mark a textually accessible \isi{referent}, as in sentence (\ref{e:shiohara:22}), while speaker I, who uses \textit{tu} for most of the textually given referents, also employs larger situational \textit{depe}, as seen in sentence (\ref{e:shiohara:23}) above.

The variation observed in the frequency of each device among speakers, therefore, is not caused by differences in the referential system each of them employs, but by which strategy they prefer to code an \isi{anaphoric} relation of a \isi{referent} in the \isi{discourse} and \isi{discourse} situation. Speaker I prefers to code a relation of a \isi{referent} in the previous \isi{discourse} and therefore uses \isi{anaphoric} articles more frequently, while speaker H prefers to relate a \isi{referent} to a shared situation told by the whole \isi{discourse} and therefore uses the possessive more frequently.
	
As mentioned in \sectref{s:shiohara:2}, the article and possessive may co-occur in one NP. The elicited text includes three examples of such a co-occurrence. Example (\ref{e:shiohara:29}) below and example (\ref{e:shiohara:10}) above from the elicited text and \hyperref[e:shiohara:20b]{(20)b} and \hyperref[e:shiohara:20c]{(20)c} above, which are spontaneous utterances, show this compatibility. In sentence (\ref{e:shiohara:29}), the article \textit{tu} indicates a textual-situational accessibility and the possessive \textit{depe} indicates that the \isi{referent} can be associated with the shared larger situation.

\begin{exe}
	\ex\label{e:shiohara:29}
	\gll \textit{Kase}  \textit{ancor}  \textbf{\textit{tu}} \textbf{\textit{depe}} \textit{sambiki} \textit{supaya} \textit{dapa} \textit{lia} \textit{warna} \textit{kuning}.\\
	cause  smash  \textsc{art.d}  \textsc{3sg.poss} pumpkin {so.that} get see color yellow\\
	\glt ‘(We) smashed the pumpkin, so that we could see the yellow color.’ \hfill{(I 42)}
\end{exe}

\noindent
This suggests that the semantic domain each device covers is not exclusive to the other and belongs to intrinsically different semantic dimensions; one may mark the \isi{referent} as textual-situationally accessible and, at the same time, as identifiable through association with the larger situation shared between the interlocutors.

\section{\label{s:shiohara:5}Summary and discussion}
We have shown referential strategies of MM, with special focus on how a lexical NP is marked according to the \isi{information status} of the \isi{referent}. MM has two strategies to mark so-called “definiteness”: articles and the third person singular possessive \textit{depe}. The articles are derived from demonstratives and are used for direct situational reference and \isi{anaphoric} reference, while the possessive is used for references in which some kind of association is required for identification, which corresponds to \isi{anaphoric} associative use and larger situation use of \ili{English} in the classification of \citet{Hawkins2015}.\largerpage

Both devices still retain their original semantic functions. The semantic domain of the articles does not extend beyond textual-situational accessibility, a direct semantic extension of the demonstratives; while the possessive does not cover all the “associative” relations that would be expressed by the definite NP in \ili{English}, as seen in \sectref{s:shiohara:3}.

Demonstratives are a well-known source of definite markers in many languages. MM articles have established a syntactic position in NPs separated from the postposed demonstratives, and especially \textit{tu} (derived from the distal \isi{demonstrative}) has undergone semantic bleaching. We could expect that the use of the articles might be extended further to indirect reference, such as \isi{anaphoric} associative or larger situational use \citep{Hawkins2015}. This cross-linguistically plausible scenario, however, seems to be blocked by the semantic extension of the possessive \textit{depe}, at least in the present stage.

The article \textit{tu} and possessive \textit{depe} may co-occur in one NP. This fact suggests that the semantic domain which each form covers is not exclusive to the other and belongs to intrinsically different semantic dimensions; one may mark the \isi{referent} as textual-situationally accessible and, at the same time, as identifiable through association.

A very similar type of referential system with demonstratives and possessives is observed in \ili{Cirebon Javanese}, a genetically related language (\citealt{Ewing1995}; \citealt{Ewing2005}). In \ili{Cirebon Javanese}, as in MM, the determiners derived from the demonstratives mark directly shared identifiability, and textual-situational accessibility, while the possessive suffix -\textit{é}, marks identifiability through indirect association. The two devices can frequently co-occur in one NP, because they “are not in some sort of complementary distribution” \citep[80]{Ewing1995}.

Similar, but apparently more grammaticalized patterns of marking are observed in Fehring, a dialect of North Frisian. In Fehring, according to \citet[161ff]{Lyons1999}, which is based on the description of \citet{Ebert1971a, Ebert1971b}, and \citet{deMulder2011}, the strong, less grammaticalized, article is used for textual-situational accessibility, while the weak, more grammaticalized, article is used to indicate \isi{anaphoric} association, unique entity, and generic entity \citep[529]{deMulder2011}. The two articles exhibit complementary distribution in the pre-head noun determiner slot. The result of definite marking development in MM may be the pattern observed in Fehring.

Another possible development may be that one of the two strategies becomes more dominant than the other. As shown in \sectref{s:shiohara:4}, among the four speakers who have provided narrative data, one elder speaker prefers to use the possessive, while the two younger speakers prefer to use the articles. From this generational difference, we might predict that the article will become dominant and extend its semantic domain to indirect reference in the future.

MM is rapidly obtaining native speakers. As it goes in this direction, processes of standardization or homogenization could be expected to affect the marking of definiteness. The process should be monitored through ongoing research.\largerpage[2]

\section*{Abbreviations}
	\begin{tabular}{@{}ll@{\hspace{5em}}ll@{}}
		\textsc{1, 2, 3} & the 1st, 2nd, 3rd person      &\textsc{neg} & negation\\
		\textsc{art.d} & distal article                  &		\textsc{pass} & passive\\
		\textsc{art.p} & proximal article                &		\textsc{pft} & perfect\\
		\textsc{caus} & causative                        &		\textsc{pfv} & perfective\\
		\textsc{dem.d} & distal \isi{demonstrative}      &		\textsc{pl} & plural\\
		\textsc{dem.p} & proximal \isi{demonstrative}    &		\textsc{poss} & possessive\\
		\textsc{excl} & exclusive                        &		\textsc{pst} & past\\
		\textsc{fut} & future                            &		\textsc{ptc} & \isi{discourse} particle\\
		\textsc{itj} & interjection                      &		\textsc{red} & reduplication\\
		\textsc{itr} & \isi{interrogative}               &		\textsc{sg} & singular\\
	\end{tabular}

\sloppy
\printbibliography[heading=subbibliography,notkeyword=this]

\end{document}
