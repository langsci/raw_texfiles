\documentclass[output=paper
,modfonts
,nonflat]{langsci/langscibook} 

\ChapterDOI{10.5281/zenodo.1402537}

\title{Stance, categorisation, and information structure in Malay} 

\author{František Kratochvíl\affiliation{Palack\'{y} University, Olomouc, Czech Republic}\and Nur Izdihar Binte Ismail\affiliation{Nanyang Technological University, Singapore}\lastand Diyana Hamzah\affiliation{Temasek Laboratories, Singapore}}

% \chapterDOI{} %will be filled in at production
% \epigram{}

\abstract{This chapter describes the expression of referents in Singaporean Malay using a parallel corpus of elicited narratives. We demonstrate that the speaker's epistemic stance affects how discourse is constructed. The speaker's epistemic stance is apparent in referent categorisation: referents can be categorised either as ``familiar", when taking a strong epistemic stance, or as ``unfamiliar", when taking a neutral stance. We show that referent categorisation is more fundamental than the information structure notions of \emph{new}, \emph{old}, or \emph{given}. Familiar human or animate referents are expressed with proper names and are pivotal for organising the narrative plot: by constructing other discourse-persistent referents in relation to the familiar referent, their description and tracking simplifies. Human and animate referents categorised as unfamiliar are expressed with nominals. Their descriptions and tracking are more elaborate, involving demonstratives and discourse particles, whose function lies in the coordination of joint attention. Inanimate referents are rarely subject of strong epistemic stance and are therefore expressed with nominals. Their discourse-persistence is the best predictor of how elaborate their description and tracking are.}

\begin{document}\maketitle

\section{Introduction}\label{section:Introduction}\largerpage[2]
One of the fundamental functions of human language is balancing the information disparity between the speaker and the hearer. It has been argued that the speaker and hearer both operate under the assumption that the world presents itself in the same way to their interlocutor. Under such an assumption, the speaker can ``trade places" with the hearer, and can predict and mitigate obvious disparities \citep{Rommetveit1976, Zlatev2008, Duranti2009, Duranti2010}. 

The above information-disparity problem is examined through the study of information structure, i.e. the structural arrangement of various types of information, such as \emph{new}, \emph{old}, \emph{given}, \emph{topic}, \emph{focus}, etc. \citep[cf.][]{Prince1981, Gundel1993, Lambrecht1994, Gundel2004}.

In a broader perspective, however, the complexity of expression of the \emph{new}, \emph{old} and \emph{given} reflects the speaker's \isi{stance} towards the utterance, reality, and the hearer. This \emph{stance}, or \emph{alignment} is manifested in the amount of information disclosed in order to mitigate disparity \citep[cf.][]{DuBois2007}.\footnote{In \ili{Malay}/\ili{Indonesian}, an important work on this aspect of language is \cite{Englebretson2007b}, which primarily deals with the choice of pronouns and its consequences.}

Du Bois' framework conceptualises \isi{stance} as the process of evaluation and positioning towards the object of \isi{stance} and the mutual alignment between subjects emerging from the interaction \citep[171]{DuBois2007}. Stance is achieved through overt communicative means towards any salient dimension of the sociocultural field \citep[163]{DuBois2007}. This process is visualised in Du Bois' original \emph{\isi{stance} triangle}, reproduced here in \figref{fig:StanceTriangle}. 

\begin{figure}
% % \includegraphics[width=6cm]{figures/StanceTriangle.png}
\begin{tikzpicture}[slopenode/.style={sloped,midway,font=\sffamily\footnotesize},corners/.style={font=\sffamily\bfseries}]
\node at (0,4) (st-s1) [corners] {Subject {1}};
\node at (4,2) (st-o)  [corners] {Object};
\node at (0,0) (st-s2) [corners] {Subject {2}};
\draw (st-s1.south) -- node[slopenode,above] {evaluates ►} node [slopenode,below] {◄ positions} (st-o.west);
\draw (st-s2.north) -- node[slopenode,above] {evaluates ►} node [slopenode,below] {◄ positions} (st-o.west);
\draw (st-s2.north) -- node[slopenode,above] {◄ aligns ►} (st-s1.south);
\end{tikzpicture}
\caption{Du Bois' \emph{stance triangle} \citep[161]{DuBois2007}\label{fig:StanceTriangle}}
% % 	\todo[inline]{redo in tikz}
\end{figure}

\noindent
We demonstrate that the compositional vectors of \emph{stance}, namely \emph{evaluation}, \emph{positioning}, and \emph{alignment} can be applied to the study of information structure and \isi{referent} expression. We expand the understanding of Du Bois' \emph{evaluation} to include the choice in identifying a \isi{referent} and \emph{\isi{referent} categorisation}, a term borrowed from \cite{Stivers2007}. The \isi{categorisation} \emph{positions} the speakers towards the object differently in terms of their \isi{epistemic stance}. The choice has consequences for the construction of subsequent \isi{discourse}, as will be documented in \sectref{StanceAndReference}. 

%%%%%%%%%%%%%%%%%%%%%%%%%%%%%%%%%%%%%%%%%%%%%%%%

\section{Methodology, participants, and language situation} 
The data for this paper consists of a set of elicited narratives in Singapore \ili{Malay}. These narratives were collected using four stimuli sets: (i) Getting the Story Straight \citep{SanRoqueEtAl2012}, (ii) Pear Story \citep{Chafe1980}, (iii) Frog Story \citep{Mayer1969}, and (iv) Jackal and Crow \citep{Carroll2011}. The stories allow us to make a systematic comparison of how our subjects categorise a variety of referents (human, animate, inanimate, singular, plural, etc.). By comparing how referents are introduced and tracked, we reveal the consequences of the \isi{categorisation} for \isi{discourse} construction. We rely on the annotation guidelines of the \emph{\isi{RefLex} Scheme} to distinguish various types of referents and their expressions \citep{RiesterBaumann2017}. 

In this section, we describe the \ili{Malay} spoken in Singapore (\ref{Methodology:SgMalay}), our participants (\ref{Methodology:Participants}), and the stimuli used here (\ref{Methodology:GTSS}--\ref{Methodology:Jackal}). The instances where we consulted our Singapore \ili{Malay} Corpus are distinguished with corpus text identifiers.\footnote{Our Singapore \ili{Malay} Corpus consists of about 100 conversations and narratives (spontaneous, planned and elicited), counting about 62,000 words.}

%========================%========================%========================
\subsection{Malay in Singapore}\label{Methodology:SgMalay}

The \ili{Malay} language connects diverse varieties that form the \ili{Malayic} subgroup of\linebreak \ili{Austronesian}. In all probability from Southern Sumatra, \ili{Malay} varieties are now spoken throughout Indonesia, Brunei, Malaysia, Singapore and southern Thailand \citep{Adelaar2004}. In Singapore, \ili{Malay} has always had a special status, given its former role as the administrative language and lingua franca and for its political value in the region \citep{Alsagoff2008}. 

Apart from the symbolic status of the national language of Singapore, \ili{Malay} is one of the four official languages of Singapore, alongside \ili{English}, \ili{Mandarin} \ili{Chinese} and \ili{Tamil}. \ili{Malay} is the assigned \emph{mother tongue} of the ethnic `Malays' in Singapore, a label comprising people of \ili{Malay}, \ili{Javanese}, Boyanese, and \ili{Sundanese} descent as well as other smaller groups from the peninsula and archipelago, which make up 13.3\% of the resident population \citep{KuoJernudd1993, GenHouseholdSurv2015}.\footnote{The 2015 census reveals that \ili{English}-\ili{Malay} bilinguals make up 86.2\% and 14.0\% of the \ili{Malay} and Indian resident population, respectively, and that \ili{Malay} remains the dominant home language of the \ili{Malay} resident population aged 5 years and over (78.4\%) \citep{GenHouseholdSurv2015}.} 

Standard Singapore \ili{Malay} is the formal written and spoken variety taught in schools and used in formal contexts (government and media). It is similar to the standard variety used in Malaysia, with the addition of certain lexical items relevant to the local context. Colloquial Singapore \ili{Malay} is the informal spoken variety. In the past, a number of contact varieties emerged, with distinct syntactic, grammatical and phonological features \citep{Rekha2007}. The best studied among them include: (i) Singapore \ili{Baba Malay}, a \ili{Malay} creole influenced by \ili{Hokkien}, which is typically spoken by the Peranakan population in Singapore \citep{Lee2014}, (ii) Singapore \ili{Bazaar Malay}, a \ili{Malay}-lexified pidgin influenced by \ili{Hokkien} which was the traditional lingua franca for interethnic communication (prior to the rise of \ili{English}) and is typically spoken by \ili{Singaporean} \ili{Chinese}, and (iii) Singapore \ili{Indian Malay}, a \ili{Malay}-lexified pidgin influenced by \ili{Bazaar Malay} and Indian languages which is typically spoken by \ili{Singaporean} Indians \citep{Prentice1996, Daw2005, Rekha2007}. Rising levels of bilingualism with \ili{English} introduce contact features such as code-switching, borrowing of lexicon and structural convergence with \ili{Singlish}.

%========================%========================%========================
\subsection{Participants}\label{Methodology:Participants}

Our participants are all Singapore Malays from diverse linguistic backgrounds. JUR, ISM, and ISH grew up in monolingual \ili{Malay} families, only beginning their \ili{English} studies when they entered primary school at the age of seven. AM, YAN, and SI grew up in  \ili{Malay}-dominant bilingual families. While their exposure to \ili{English} was earlier, all three attended \ili{Malay}-speaking kindergartens, and YAN and SI went on to private religious schools, where the medium of instruction was \ili{Malay} and \ili{English}. MIZ grew up in an \ili{English}-dominant bilingual family, while LQ, HZ and NZ came from families where bilingualism was more balanced. Their formal education in \ili{English} and \ili{Malay} also began in kindergarten. After kindergarten, AM, MIZ, LQ, HZ, and NZ went through the mainstream Singapore education system, where the medium of instruction was \ili{English}.

%========================%========================%========================
\subsection{Getting the Story Straight \citep{SanRoqueEtAl2012}}\label{Methodology:GTSS}
The first stimuli collection is a graphic mini-novel depicting in 16 pictures the transformation of a man, through a descent into jail caused a change of heart, from someone who drinks and beats his wife into a loving father and husband, as shown in \figref{fig:StraightStory}. 

\begin{figure}
\includegraphics[width=\textwidth]{figures/GettingTheStoryStraightPictureRoll.jpg}
\caption{Getting the Story Straight storyline \citep{SanRoqueEtAl2012}\label{fig:StraightStory}}
\end{figure}

\noindent
In the original set-up (see text 1 in \tabref{tab:UsedTextsGTSS}), the pictures were presented in a stipulated sequence to two participants who negotiated and constructed the narrative. When finished, they presented it to a newly arrived third participant. The entire experiment lasted about 20 minutes. The word counts offer a measure of the verbal effort with the second set-up, when the correct picture sequence is presented to a speaker who narrates it. No negotiation took place, since the second participant was instructed to take on the role of the listener. The task lasted only about five minutes on average, and required much less verbal effort (see texts 2--11).

%========================%========================%========================
\subsection{Pear Story \citep{Chafe1980}}\label{Methodology:Pear}
The second stimuli set is the \emph{Pear Story}, a six-minute film. Set in the countryside, it depicts a loose sequence of events happening around an orchard, where a farmer is picking pears. A man walks by with a goat, and a boy on the bicycle comes to collect the fruit. When he later falls and the load of pears spills on the road, three other boys come to his help, who each receive a pear in return. We recorded two versions.

\begin{table}[p]
\caption{Collected versions of \emph{Getting the Story Straight} \citep{SanRoqueEtAl2012}}
\label{tab:UsedTextsGTSS}
 \begin{tabularx}{.7\textwidth}{Xll} 
  \lsptoprule
& text name & words \\
  \midrule
1. & 2014.MLZ.GettingTheStoryStraight & 2235 \\
2. & 2017.NI.GettingTheStoryStraight.JUR & 247 \\
3. & 2017.NI.GettingTheStoryStraight.ISM & 249 \\
4. & 2017.NI.GettingTheStoryStraight.MIZ & 572 \\
5. & 2017.NI.GettingTheStoryStraight.AM & 355 \\
6. & 2017.NI.GettingTheStoryStraight.YAN & 437 \\
7. & 2017.NI.GettingTheStoryStraight.SI & 512 \\
8. & 2017.NI.GettingTheStoryStraight.ISH & 239 \\
9. & 2017.NI.GettingTheStoryStraight.LQ & 214 \\
10. & 2017.NI.GettingTheStoryStraight.HZ & 380 \\
11. & 2017.NI.GettingTheStoryStraight.NZ & 227 \\

  \lspbottomrule
 \end{tabularx}
\end{table}

\begin{table}[p]
\caption{Collected versions of \emph{Pear Story} \citep{Chafe1980}}
\label{tab:UsedTextsPear}
 \begin{tabularx}{.7\textwidth}{Xll} 
  \lsptoprule
& text name & words \\
  \midrule
1. & 2013.CA.PearStory & 169 \\
2. & 2013.LN.PearStory & 442 \\
  \lspbottomrule
 \end{tabularx}
\end{table}

%========================%========================%========================
\subsection{Frog Story \citep{Mayer1969}}\label{Methodology:Frog}
\emph{Frog Story} is a picture book for children (see \figref{fig:FrogStory}) widely used for language comparison. It is the story of a boy whose pet frog escaped from its jar, so he sets out with his dog to find it. We recorded two versions of this story, listed in \tabref{tab:UsedTextsFrog}.

\begin{figure}
\includegraphics[width=\textwidth]{figures/FrogStoryStoryBoard.jpg}
\caption{Frog Story storyline \citep{Mayer1969}\label{fig:FrogStory}}
\end{figure}

\begin{table}[p]
\caption{Collected versions of \emph{Frog Story} \citep{Mayer1969}}
\label{tab:UsedTextsFrog}
 \begin{tabularx}{.7\textwidth}{Xll} 
  \lsptoprule
& text name & words \\
  \midrule
3. & 2013.OG.FrogStory & 963 \\
4. & 2013.SS.FrogStory & 387 \\
  \lspbottomrule
 \end{tabularx}
\end{table}

%========================%========================%========================
\subsection{Jackal and Crow \citep{Carroll2011}}\label{Methodology:Jackal}
\emph{Jackal and Crow} consists of nine pictures presenting a version of the famous Aesop fable of The Fox and the Crow. The fox is drawn to be identifiable as a jackal, wolf, or dog, and the crow holds a fish, instead of cheese.

\begin{figure}
\includegraphics[width=\textwidth]{figures/Jackal.jpg}
\caption{Jackal and Crow storyline \citep{Carroll2011}\label{fig:Jackal}}
\end{figure}

\noindent
We again used two set-ups. The 2013 version is a narration of the picture sequence by a single speaker, while the 2014 version follows the original guidelines of \cite{Carroll2011} and is a negotiation of two speakers, who construct the narrative for a third participant.

\begin{table}[p]
\caption{Collected versions of \emph{Jackal and Crow} \citep{Carroll2011}}
\label{tab:UsedTextsJackal}
 \begin{tabularx}{.7\textwidth}{Xll} 
  \lsptoprule
& text name & words \\
  \midrule
1. & 2013.OG.JackalAndCrow & 212\\
2. & 2014.MLZ.JackalAndCrow & 660 \\
  \lspbottomrule
 \end{tabularx}
\end{table}


%%%%%%%%%%%%%%%%%%%%%%%%%%%%%%%%%%%%%%%%%%%%%%%

\section{Stance and referent categorisation}\label{StanceAndReference}
In \sectref{section:Introduction}, we linked \isi{stance} to the notion of \emph{\isi{referent} categorisation}. Referent \isi{categorisation} refers to the choice a speaker makes by identifying the referents for the hearer. A fundamental dichotomy exists between \emph{proper names} and \emph{descriptions} (nominal expressions). 

Categorisation with proper names positions the speaker as familiar with the object of \isi{stance}. According to \citet{SacksSchegloff2007}, proper names satisfy two discourse-organisational preferences: (i) recognitional preference and (ii) minimised reference. For the first, it is easier to work out the reference to something familiar, even if familiarity is only constructed. The second is a preference for a stable, and perhaps a single, reference form, so that the expression-\isi{referent} pair is stable. Proper names meet both requirements, but nominal expressions require more recognitional effort on the part of the hearer. In addition to speaker's stance, the choice of a proper name reveals  aspects of speaker's identity, such as their relation to the topic and their self-positioning within the community \citep[cf.][13]{Baresova2016}. In a constructed narrative, the identity is symbolic.

Categorisation with  \emph{descriptions} reveals the speaker's neutral \isi{epistemic stance}.\footnote{The term \emph{descriptions} is synonymous with \emph{nominal expressions}.} The hearer's effort is greater in both recognition and maintaining reference, as will be apparent in \sectref{ReferentTracking}.

The effect of \isi{referent} \isi{categorisation} on \isi{discourse structure} is most obvious among the eleven versions of \emph{Getting the Story Straight} (see \sectref{CategorisationHumanGSS}), and is also detected in the \emph{Frog Story} (see \sectref{CategorisationHumanFS}). In contrast, the \isi{referent} \isi{categorisation} in the \emph{Pear Story} is uniform. For \isi{categorisation} of inanimate referents, their \isi{discourse} role is the most important factor. Discourse-persistent referents require more elaborate descriptions than ``props" (see \sectref{CategorisationNonHuman}).

%%%%%%%%%%%%%%%%%%%%%%%%%%%%%%%%%%%%%%%%%%%%%%%

\subsection{Human referent categorisation in \emph{Getting the Story Straight}}\label{CategorisationHumanGSS}
The main participants in \emph{Getting the Story Straight} are: a farmer, his wife, child, and friends. The farmer is present in all frames, while the other characters play a less central role, sometimes restricted to a single frame.

More than half of our subjects categorise the farmer with a proper name (usually a common \ili{Malay} name such as \emph{Adam}, \emph{Halim}, \emph{Samad}, or \emph{Zamri}), making it a referential \isi{pivot} for other human referents (farmer's family and friends). This strategy is in line with the \emph{preferences for person reference} formulated in \cite[24]{SacksSchegloff2007}. Proper names are prototypical and ideal recognitional devices \citep[25]{SacksSchegloff2007} and their use is therefore referentially effective. The \isi{RefLex} scheme classifies the first use of proper names as \emph{r-unused-unknown} \citep[10]{RiesterBaumann2017}. Example (\ref{AbuKedengaran}) illustrates that to track the given \isi{referent} (\isi{RefLex} \emph{r-given}) proper names can be repeated.

\ea\label{AbuKedengaran} 
\langinfo{Singapore Malay}{}{2017.SI.12--14}\\
\gll {\ob}Abu{\cb}         kedengaran me-racau-racau     ber-tanda   dia sudah   mabuk.  {\ob}Abu{\cb}         mula  ber-cerita yang bukan-bukan. Ini  lazim  ber-laku apabila  {\ob}Abu{\cb}         mabuk kerana  minum {minuman keras} itu.\\
\textsc{pn} audible       \textsc{av-}talk.incoherently \textsc{av-}sign  \textsc{3sg} already drunk \textsc{pn} start \textsc{av-}tell      \textsc{rel}  nonsense \textsc{prox} common \textsc{av-}happen  when   \textsc{pn}  drunk because consume alcohol     \textsc{dist}\\
\glt `Abu was heard to rave, which was a sign that he was drunk. Abu started to tell untrue stories. This habitually happened when Abu was drunk from drinking alcohol.'
\z

\noindent
The neutral \isi{epistemic stance} leads to the \isi{categorisation} of the farmer with a nominal expression as a \emph{petani} `farmer' (\isi{RefLex} \emph{r-new}).\footnote{A wealth of literature is dedicated to various aspects of the \ili{Malay} noun phrase. The relevant devices are (i) classifiers \citep{Hopper1986, Chung2000, Chung2008, ClearyKemp2007, Chung2010, SalehuddinWinskel2012}, (ii) demonstratives \citep{Himmelmann1996, Williams2009}, (iii) relative clauses \citep{ColeHermon2005}, (iv) the linker \emph{yang} \citep{vanMinde2008}, and (v) the definite \emph{-nya} \citep{Rubin2010}.} Because the \isi{referent} will persist in \isi{discourse}, it is typically introduced with a \isi{classifier phrase} \citep[cf.][317]{Hopper1986}. We will return to this point in the discussion of example (\ref{KebetulanKejadian}) and again in \sectref{CategorisationNonHuman}.

The farmer's family and friends are always introduced through expressions of their relationship to the farmer, such as \emph{isteri=nya} `his wife' in (\ref{KeduaduaZamri}). The \isi{RefLex} scheme characterises such expressions as \emph{r-bridging-contained} \citep[9]{RiesterBaumann2017}. The bridging containment is realised by  possessive constructions available in \ili{Malay}.  It is interesting that when the farmer is given a name, his wife is usually given one too (e.g. \emph{Alia}, \emph{Hawa}, \emph{Huda}, \emph{Laila}), as in (\ref{KeduaduaZamri}).\footnote{The bridging anaphor between the possessive \emph{-nya} and its target \emph{Zamri} is highlighted using the \isi{RefLex} scheme convention, i.e.  the target of the anaphora is underlined and the referential expression is in square brackets.}

\ea\label{KeduaduaZamri} 
\langinfo{Singapore Malay}{}{2017.MIZ.01}\\
	\gll kedua-dua \ule{Zamri}       dan  {\ob}isteri\ule{-nya}     Alina{\cb}       ber-kerja   seperti pekebun.\\
		both      \textsc{pn} and wife-\textsc{3poss} \textsc{pn} \textsc{av}-work  as      farmer\\
\glt `Both Zamri and his wife Alina work as farmers.'
\z

\noindent
The \isi{categorisation} of the child seems independent of the speaker's \isi{stance} towards the farmer and his wife. In our data, the child is rarely categorised with a proper name. In several versions, although depicted in Frame 2 as held by her mother,  the child is  introduced only in the domestic violence scene, as in (\ref{SambilMendukung}).


\ea\label{SambilMendukung} 
\langinfo{Singapore Malay}{}{2017.YAN.12}\\
	\gll sambil         men-dukung  {\ob}anak\ule{-nya}{\cb},     Huda        mem-beritahu \ule{Halim}       bahawa dia tidak mem-punyai apa-apa  hubungan     {sama sekali} dengan Khalid      {\ldots} dengan Leyman.\\
while \textsc{av-}hold      child-\textsc{3poss} \textsc{pn} \textsc{av-}tell     \textsc{pn} \textsc{comp}   \textsc{3sg} not   \textsc{av-}have any relationship at.all      with   \textsc{pn}       {}  with \textsc{pn}\\
\glt `While carrying her child, Huda told Halim that she did not have any relationship with Khalid [sic], \ldots with Leyman.'
\z

\noindent
Plurality is an important feature of human referents: the farmer's friends are always introduced in a reduplicated form as \emph{kawan-kawan} `friends'. The possessive \emph{=nya} may associate them with the topical farmer. In some versions, the gossiping friend is named (e.g. \emph{Rashid}, \emph{Wahid}). Both strategies are combined in AM's version, where the friends are first introduced as a group in an earlier sentence, and then the gossiper is named as \emph{Rashid}, as shown in (\ref{RashidMenceritakan}).


\ea\label{RashidMenceritakan}  
\langinfo{Singapore Malay}{}{2017.AM.08}\\
	\gll  {\ob}Rashid{\cb}      menceritakan bahawa dia pernah ter-nampak isteri Pak Samad       telah   meng-gatal dengan Encik Romi       semasa dia sedang mem-beli barang rumah di pasar.\\
	\textsc{pn} \textsc{av}.tell         \textsc{comp}   \textsc{3sg} once   \textsc{invol-}see       wife   Mr  \textsc{pn} already \textsc{av-}chat.up   with   sir   \textsc{pn} when   \textsc{3sg} \textsc{prog}   \textsc{av-}buy     item   home  in market\\
	\glt `Rashid told everyone that he had seen Pak Samad's wife flirting with Mr.  Romi while she was buying household items at the market.' 
\z

\noindent
The old man, who sees the fight between the farmer and his wife (\figref{fig:StraightStory}, frame 5), is usually categorised as a relative (usually as the father of the spouse) using a possessed noun, as in (\ref{PapaLailaMelaporkan}). This is a type of \emph{r-bridging-contained}, where the possessor is already known from the context. The introduction is abrupt, because the old man calls the police right away, so there is no time or need to provide any other details.

\ea\label{PapaLailaMelaporkan} 
\langinfo{Singapore Malay}{}{2017.AM.14}\\
	\gll  {\ob}Papa   \ule{Laila}{\cb}       ter-nampak perkara ini  lalu me-lapor-nya,     lalu me-lapor-kan-nya      ke   polis.\\
	father \textsc{pn} \textsc{invol-}see       event   \textsc{prox} then \textsc{av-}report-3      then \textsc{av-}report-\textsc{appl}-3      to   police\\
	\glt `Laila's father saw the incident and reported it, reported it to the police.'
\z

\noindent
The neutral \isi{stance} leads to a nominal \isi{categorisation} of the old man as a neighbour, using an enumerated \isi{classifier phrase}, as in (\ref{KebetulanKejadian}).\footnote{\cite{Hopper1986} described the role and use of classifiers and provided parameters conducive to the use of classifiers based on written nineteenth-century \ili{Malay} (pp. 313--314). According to Hopper, the primary function of classifiers is to grant discourse-new nouns \isi{prominence} and the ability to become topics, whose referents are ``individuated" and ``persistent in \isi{discourse}" (p. 319).} In the \isi{RefLex} Scheme, such a \isi{referent} is classified as \emph{r-new} \citep[11]{RiesterBaumann2017}. It should be noted that the neutral \isi{stance} to the old man does not exclude a strong \isi{stance} to the farmer and his wife, whom SI categorises with proper names. 

\ea\label{KebetulanKejadian} 
\langinfo{Singapore Malay}{}{2017.SI.23}\\
	\gll kebetulan      kejadian tersebut  di-lihat      oleh  {\ob}se-orang    jiran{\cb}.\\
		coincidentally event    mentioned \textsc{pv}-see   by   one-\textsc{cl.human} neighbor\\
\glt `Coincidentally, the incident was seen by a neighbour.'
\z

\noindent
Categorisation of the policemen and court officials is fairly uniform, using various types of nominals. Bare nouns such as \emph{polis} `police', or a group compound \emph{pihak polis} `police force' are the most common.\footnote{The root \emph{pihak} is used in other group compounds such as \emph{pihak berkuasa} `authority, agency', \emph{pihak lawan} `opposition', \emph{pihak musuh} `enemy, enemies', and \emph{pihak pengurusan} `management'.} Within the \isi{RefLex} scheme, the \isi{referent} is classified as \emph{r-unused-known}, because we assume that it is generally known and that appeal can be made to the local security force to stop violence. The only case where an indefinite description (quantified \isi{classifier phrase}) is used is shown in (\ref{TidakLamaSelepas}). This may be a consequence of enumeration, which in \ili{Malay} requires a \isi{classifier phrase}.

\ea\label{TidakLamaSelepas} 
\langinfo{Singapore Malay}{}{2017.SI.24}\\
	\gll tidak lama {selepas itu},  {\ob}dua orang polis{\cb}     datang dan  mem-berkas Adam.\\
		not   long thereafter   two \textsc{cl.human}    police come   and  \textsc{av-}arrest    \textsc{pn}\\
\glt `Not long afterwards, two policemen came and arrested Adam.'
\z

\noindent
Proper names open up a referent-internal perspective: for example, the abuse by the farmer can be presented from the perspective of his wife or the court, and their \isi{stance} can be constructed. This is shown in (\ref{SemasaHawaDipanggil}), where the farmer, introduced as \emph{Adam}, is referred to as \emph{suami-nya} `her husband', embedding him in a kinship relation with expected norms of behavior. HZ's version uses the same strategy to mark the wife's perspective in the same point of the narrative (see \tabref{tab:GTSS:merged}). The speaker can establish and/or maintain differential perspective to the same \isi{referent} in this way \citep[cf.][107]{Enfield2007}.

\ea\label{SemasaHawaDipanggil} 
\langinfo{Singapore Malay}{}{2017.SI.25}\\
\gll  Semasa Hawa di-panggil untuk mem-buat kenyataan di balai   polis, Hawa kelihatan teruk di-cederakan            oleh  {\ob}suami-nya{\cb}       sehingga mata, kepala dan  leher-nya     perlu di-balut.\\
when   \textsc{pn}  \textsc{pv-}call      to    \textsc{av-}make    statement in station police \textsc{pn}  appearance   dreadful \textsc{pv-}injure.\textsc{caus}    by   husband-\textsc{3poss} so.that  eye   head   and  neck-\textsc{3poss} need \textsc{pv-}dress.wound\\
\glt `While Hawa was called in to make a statement at the police station, she seemed badly hurt by her husband, to the point where her eyes, head, and neck had to be bandaged.'
\z

%%%%%%%%%%%%%%%%%%%%%%%%%%%%%%%%%%%%%%%%%%%%%%
\subsection{Categorisation of human referents in \emph{Frog Story} and \emph{Pear Story}}\label{CategorisationHumanFS}
The two versions of \emph{Frog Story} show a similar pattern as \emph{Getting the Story Straight}. A stronger \isi{epistemic stance} leads to \isi{categorisation} of the boy with a proper name. The stronger \isi{stance} allows the speaker to fabulate the boy's character, emotions, and habits, as in (\ref{PadaSuatuMalam}), where the dog is described as the boy's \emph{anjing kesayangan} `beloved dog', and the frog is expressed with a possessive phrase (\emph{r-bridging-contained}).

\ea\label{PadaSuatuMalam} 
\langinfo{Singapore Malay}{}{2013.SS.FrogStory.01}\\
	\gll Pada suatu      malam sebelum tidur {\ob}Abu{\cb}      dan   {\ob}anjing kesayangan \ule{dia}{\cb} sedang  me-renung  {\ob}katak\ule{-nya}{\cb}.\\
on   one        night before  sleep \textsc{pn} and  dog    beloved    \textsc{3sg} \textsc{prog} \textsc{av-}study     frog\textsc{-3poss}\\
\glt `One night before sleeping Abu and his beloved dog were watching his frog.'
\z

\noindent
In an inverted manner, the speaker's neutral \isi{epistemic stance} is reflected in a \isi{categorisation} with descriptions. In (\ref{PadaSuatuHari}), both the boy and his dog are categorised  with indefinite nominals (\isi{RefLex} Scheme:  \emph{r-new}). The friendship between the dog and the boy is constructed later, and is not included in the first description of the dog.

\ea\label{PadaSuatuHari} 
\langinfo{Singapore Malay}{}{2013.OG.FrogStory.01}\\
	\gll Pada suatu  hari, ada   {\ob}se-orang anak kecil{\cb}, {budak lelaki}, yang mem-punyai  {\ob}se-ekor anjing{\cb} sebagai teman-nya.\\
on   one    day   exist one-\textsc{cl.human}  child  small     boy    \textsc{rel}  \textsc{av-}own
one-\textsc{cl.animal}   dog    as      friend-\textsc{3poss}\\
\glt `Once, there was a little boy, who had a dog as his friend.'
\z

\noindent
Both available versions of the \emph{Pear Story} contain no proper names. New human referents (\emph{r-new}) are categorised with enumerated classifier phrases and inanimates with bare nouns.\footnote{Our findings agree with those reported by \cite{Sukamto2013}, who studied written narratives of the \emph{Pear Story} in \ili{Indonesian}.} 

\subsection{Categorisation of non-human referents}\label{CategorisationNonHuman}
Let us now turn to the \emph{Jackal and Crow} texts, which describe a simple plot without human referents.\footnote{As mentioned in \sectref{Methodology:Jackal}, two set-ups were used to collect the two texts. For the analysis of the MLZ version, we are only concerned with the final summary of the story given to the third participant.} Neither of the texts uses proper names; instead, participants are introduced into the \isi{discourse} with enumerated classifier phrases (\isi{RefLex} \emph{r-new}), as in (\ref{PadaZamanDulu}). The fragment also contains two presentational clauses, headed by the verb \emph{terdapat} `exist, be attested in the world, be found'. Vague quantification with \emph{beberapa} `several, few',  or  with reduplicated plural forms such as \emph{ikan-ikan} `(a variety of) fish' does not require a classifier. 

\ea\label{PadaZamanDulu} 
\langinfo{Singapore Malay}{}{2013.MLZ.JackalCrow.140--141}\\
	\gll  Pada {zaman dahulu}, terdapat  {\ob}se-ekor {burung gagak}{\cb}. Dah     beliau  ternampak beberapa bakul  yang terdapat ikan-ikan. \\
in past               exist    one-\textsc{cl.animal}   crow already \textsc{3sg.hon} \textsc{invol}-see       few      basket \textsc{rel}  exist    \textsc{red}-fish\\
\glt `Once upon a time, there was a crow. And it saw several baskets filled with fish.'
\z

\noindent
The second text shows the same pattern. Animate non-human referents are categorised as descriptions, expressed with a \isi{classifier phrase}, if the \isi{referent} will become a topic. In (\ref{PadaSatuHariAda}), the \isi{referent}  \emph{burung gagak} `crow' is introduced as the subject of an inverted existential \isi{clause} headed by \emph{ada} `exist'. The inversion puts the focus on the predicate \citep[270]{Sneddon2012}. The subject is quantified (the numeral \emph{se-} + the classifier \emph{ekor} (animate)), as well as the object of the \isi{relative clause} (\emph{beberapa} `several'). A similar use of classifiers and quantification in introducing new referents is reported in \cite[319]{Hopper1986} for the nineteenth-century written autobiography known as \emph{Hikayat Abdullah}.

\ea\label{PadaSatuHariAda} 
\langinfo{Singapore Malay}{}{2013.OG.JackalCrow.02}\\
	\gll 	Pada satu hari ada    {\ob}se-ekor {burung gagak}{\cb} yang men-jumpai beberapa bakul  ikan.\\
		on   one  day  exist one-\textsc{cl.animal} crow         \textsc{rel}  \textsc{av-}discover  several     basket fish\\
	\glt `Once, there was a crow that found several baskets of fish.'
\z

\noindent
Although the fish is already mentioned as the content of the basket, as introduced in (\ref{PadaSatuHariAda}), this does not grant the fish the status of given information. It requires an upgrade from being a `prop' to become a discourse-persistent \isi{referent} \citep[cf.][319]{Hopper1986}. Analogous to other discourse-persistent referents, the single fish, which is to be picked up by the crow, is introduced with a \isi{classifier phrase}, as in  (\ref{JadiBurungGagakItu}). 

\ea\label{JadiBurungGagakItu} 
\langinfo{Singapore Malay}{}{2013.OG.JackalCrow.03}\\
\gll 	Jadi {burung gagak} itu meng-ambil  {\ob}se-ekor ikan{\cb} untuk jadi   bahan  makan-nya   untuk hari itu.\\
so   crow         \textsc{dst} \textsc{av-}pick      one-\textsc{cl.animal}   fish to    become matter food-\textsc{3poss} for   day \textsc{dist}\\
	\glt `So the crow took a fish (OR one fish) as its meal for the day.'
\z

\noindent
It is interesting to note that the tree, on which the crow lands, is not mentioned at all in the second version. In the first version, its expression is unusual, requiring a placeholder, suggesting retrieval problems, as in (\ref{DiaLandKat}). After the correct label is retrieved, it is realised as an \textsc{n-dem} structure, with the reduced proximate \emph{ni}, requiring resolution in the physical context. The \isi{RefLex} scheme classifies such referents as \emph{r-environment}. This is, however, a non-standard solution in the context of the narrative.

\ea\label{DiaLandKat} 
\langinfo{Singapore Malay}{}{2013.MLZ.JackalCrow.147}\\
\gll Dia LAND    kat   ker\ldots, apa  ni\ldots,  pokok ni.\\
\textsc{3sg} \textsc{cs}.land at    \textsc{part}   what \textsc{prox}  tree  \textsc{prox}\\
\glt `It landed on what\ldots, what's this\ldots, on this tree.'
\z

\noindent
The above examples illustrate what \cite[313]{Hopper1986} refers to as \emph{props}. Event settings are described with bare nouns, which are occasionally enumerated, or reduplicated (\emph{beberapa bakul ikan}, \emph{ikan-ikan}). Props are easily omitted where the context and world knowledge enable the hearer to construct them regardless. 

To summarise, the speaker \isi{epistemic stance} is most apparent in the \isi{categorisation} of humans. A stronger \isi{epistemic stance} leads to the use of proper names for the key characters. We will show in \sectref{ReferentTracking} that the tracking of such characters is simpler than of those humans categorised as nominals. For non-human participants, the speaker's \isi{epistemic stance} is less relevant than what \cite[319]{Hopper1986} termed as \emph{persistence in the discourse}. Future topics are introduced in a more elaborate way (typically with a \isi{classifier phrase}) than \emph{props}. Incidental props have only short persistence and require no tracking \citep[cf.][320]{Hopper1986}. \tabref{tab:Stance:Discourse:Effect} summarises the effects of \isi{stance} and \isi{discourse} role on the \isi{categorisation} and expression of referents in our \ili{Malay} corpus. We should keep in mind that elaborate descriptions can combine a nominal and a proper name, as in (\ref{KeduaduaZamri}). For the sake of our hierarchy postulated here, we consider the proper name to be an indication of the speaker's stronger \isi{epistemic stance}.

\begin{table}
\caption{Effect of stance and discourse role on referent categorisation}
\label{tab:Stance:Discourse:Effect}
 \begin{tabularx}{\textwidth}{Xcc} 
  \lsptoprule
  & \multicolumn{2}{c}{\textsc{epistemic stance}} \\
\textsc{\isi{referent} category} & \textsc{strong} & \textsc{neutral}\\  
\midrule
+human & proper name & \isi{classifier phrase} \\
+animate & ? & \isi{classifier phrase} \\
\textminus animate, +discourse-persistent & \multicolumn{2}{c}{classifier phrase}  \\
\textminus animate, \textminus discourse-persistent &\multicolumn{2}{c}{bare noun} \\
\lspbottomrule
 \end{tabularx}
\end{table}


%%%%%%%%%%%%%%%%%%%%%%%%%%%%%%%%%%%%%%%%%%%%%%%

\section{Categorisation and referent tracking}\label{ReferentTracking}
Many referents persist in \isi{discourse} for some time \citep[317]{Hopper1986} and dedicated constructions indicate their status as given (\isi{RefLex} \emph{r-given}). In this section, we show that the initial \isi{stance} and \isi{categorisation} have global consequences for \isi{referent} tracking.

Human referents categorised with proper names, discussed in \sectref{StanceAndReference}, are tracked with proper names and pronouns, usually \emph{dia} and \emph{ia}. Particles, demonstratives and other markers are used rarely. In contrast, human referents categorised with nominals are tracked in a more elaborate way, requiring a greater effort from the hearer. A range of devices are used, including repetition, and synecdoche; marking with demonstratives, particles, or relative clauses are all common ways of tracking.

\figref{fig:GTSSTimelineSI} and \figref{fig:GTSSTimelineHZ} visualise the \isi{categorisation} and tracking of referents in two quite distinct versions of \emph{Getting the Story Straight}. The expressions are time-aligned as they appear in the story.\footnote{The following abbreviations are used in \figref{fig:GTSSTimelineSI} and \ref{fig:GTSSTimelineHZ} and the tables in the remainder of this chapter: \textsc{cl} classifier, \textsc{n} noun, \textsc{num} numeral, \textsc{pn} proper name, \textsc{poss} possessor, \textsc{pro} \isi{pronoun}, and \textsc{red} reduplication.} Continuous lines mean that the \isi{referent} is not only discourse-persistent but also topical. Whenever the line is interrupted, another topical \isi{referent} appears. Two lines coincide when a reference is made to more than one \isi{referent}, either with plural pronouns (\emph{mereka} `they'), or with possessive constructions (indexing both the possessed and the possessor). 

\figref{fig:GTSSTimelineSI} illustrates the  minimisation of reference: proper names are systematically followed by pronouns \citep[cf.][260]{Heritage2007}, but other devices are not used. As the narrative shifts, a proper name is used to activate the \isi{referent} and the \isi{pronoun} tracking it within the local macro-event, usually corresponding to a single picture. In two places, the speaker used synecdoche (\textsc{n[n]}), which corresponds to the blue line dropping to the bottom of the chart.  

\begin{figure}
\includegraphics[width=\textwidth]{figures/Timeline_Siti_Annotated.png}
\caption{Storyline visualisation of the referential devices in SI version of \emph{Getting the Story Straight}}\label{fig:GTSSTimelineSI}
\end{figure}

\noindent
\figref{fig:GTSSTimelineHZ} shows that the referents are introduced with a \isi{classifier phrase} (\textsc{num-cl-n}) or with a possessor phrase (\textsc{n-poss}), and are tracked almost without exception with pronouns. Particles \emph{pun}, \emph{pula} and the \isi{demonstrative} \emph{tersebut} are used to reactivate a given \isi{referent} as a topic.

\begin{figure}
\includegraphics[width=\textwidth]{figures/Timeline_Hazwani_Annotated.png}
\caption{Storyline visualisation of the referential devices in HZ version of \emph{Getting the Story Straight}}\label{fig:GTSSTimelineHZ}
\end{figure}

Detailed discussion of the patterns visualised in \figref{fig:GTSSTimelineSI} and \figref{fig:GTSSTimelineHZ} and those attested in other texts follow. Demonstratives and the particles \emph{pun}, \emph{pula}, and \emph{lagi} will be treated in \sectref{MarkingStanceJointAttention}. We are only concerned with the nominal expression of referents; zero anaphora, \isi{word order} alternations, and verbal morphology will be discussed elsewhere. 

%%%%%%%%%%%%%%%%%%%%%%%%%%%%%%%%%%%%%%
\subsection{Tracking of human referents}\label{TrackHuman}
The most common way to track human referents is with pronouns, followed by repetition and synecdoche. 
The main characters of \emph{Getting the Story Straight} (the farmer and his wife), regardless of their \isi{categorisation} as familiar or unfamiliar, can be tracked by pronouns. Repetition of the proper name or the nominal expression is also common. In a complex sentence, proper names are restricted to the first mention and tracked with personal pronouns in subsequent positions, such as \emph{dia} and \emph{-nya} in (\ref{ZamriSangatMarah}).

\ea\label{ZamriSangatMarah} 
\langinfo{Singapore Malay}{}{2017.AM.09}\\
\gll {er\ldots}     {\ob}Zamri{\cb}       sangat marah dengan pengetahuan ini, dan  {\ob}dia{\cb}, dan   {\ob}dia{\cb} {telus\ldots} terus       balik  rumah untuk marah isteri {\ob}-nya{\cb}.\\
	\textsc{hesit} \textsc{pn} very angry        by     knowledge   \textsc{prox} and  \textsc{3sg} and  \textsc{3sg} {} immediately return home  to    scold wife-\textsc{3poss}\\
\glt `Zamri was very angry upon receiving this information, and he immediately went back home to scold his wife.'
\z

\noindent
Multiple named referents are tracked with the plural \emph{mereka}, as in (\ref{SepertiHariHariBiasa}).

\ea\label{SepertiHariHariBiasa} 
\langinfo{Singapore Malay}{}{2017.YAN.01}\\
\gll  Seperti hari-hari biasa,     Halim       bersama  isteri-nya     Huda        akan ke   kebun  mereka untuk memetik buah-buah      labu    yang telah pun       masak.\\
as      everyday  accustomed \textsc{pn} together wife-\textsc{3poss} \textsc{pn} will to   garden \textsc{3pl} to   \textsc{av.}pick    \textsc{red}-fruit pumpkin \textsc{rel}  already \textsc{add}  ripe\\
\glt `As on a normal day, Halim and his wife Huda would go to their garden to pick pumpkins that had ripened.'
\z

\noindent
Referents categorised with proper names are tracked with pronouns, even where another topic is present. This is the case in (\ref{SeleranyaJuga}), where the speaker comments on the loss of appetite experienced by the farmer in jail. The farmer, called \emph{Halim} in this version (see (\ref{SepertiHariHariBiasa})), is tracked with the possessive \emph{-nya} to background his  experiencer role and to highlight his experience.

\ea\label{SeleranyaJuga} 
\langinfo{Singapore Malay}{}{2017.YAN.22}\\
\gll  Selera {\ob}-nya{\cb}      juga ter-ganggu dan   {\ob}dia{\cb} tidak dapat meng-habiskan makanan yang di-berikan.\\
appetite-\textsc{3poss} also \textsc{aff}-upset     and  \textsc{3sg} not manage       \textsc{av-}finish       food    \textsc{rel}  \textsc{pv-}give\\
\glt `His appetite was affected and he couldn't finish the food he was given.'
\z

\noindent
Topical kinship terms, such as \emph{isteri-nya} `his wife' in (\ref{IsterinyaBerkata}), are tracked with pronouns. Interestingly, the proper name \emph{Jack} in the complement \isi{clause} cannot become the antecedent of \emph{dia}. This suggests that \ili{Malay} \isi{anaphoric} pronouns target the local topic, or that embedded proper names are not felicitous as antecedents for pronouns.

\ea\label{IsterinyaBerkata} 
\langinfo{Singapore Malay}{}{2017.LQ.07}\\
\gll \ule{Isteri-nya}     ber-kata yang Jack        {salah faham}   dan   {\ob}dia{\cb} setakat beli barang-barang pasaran sahaja.\\
wife-\textsc{3poss} \textsc{av-}say     \textsc{comp} \textsc{pn} misunderstand and  \textsc{3sg} so.far  buy  things        market  only\\
\glt `His wife said that Jack misunderstood and that she only bought goods from 
the market.'
\z

\noindent
As mentioned in the discussion of \figref{fig:GTSSTimelineSI}, proper names are tracked with pronouns where one description is a paraphrase of an earlier one, or follows from it in a logical way, as in (\ref{HalimMemberitahu}).

\ea\label{HalimMemberitahu} 
\langinfo{Singapore Malay}{}{2017.YAN.30--31}\\
\gll  {\ob}Halim{\cb}       mem-beritahu berapa   seksa-nya       hidup {di dalam} penjara.  {\ob}Dia{\cb} me-luahkan rasa    kesal {\ob}-nya{\cb}       {di atas} perbuatan ganas {\ob}-nya{\cb}        {tempoh hari} akibat  mabuk me-minum {minuman keras}.\\
\textsc{pn} \textsc{av-}tell        how.much torturous-\textsc{intens} life  inside   prison  \textsc{3sg} \textsc{av-}express   feeling repent-\textsc{3poss}      at   action    violence-\textsc{3poss} past.time   in.result drunk
\textsc{av-}drink   alcohol\\
\glt `Halim told them what a torment life in prison was. He expressed his feelings of regret over his brutal actions a few days ago, the result of being drunk from drinking liquor.\\
\z

\noindent
In (\ref{SewaktuDiDalam}), a single macro-event in three sentences characterises the farmer's ordeal in jail. The proper name \emph{Adam} is used only in the first sentence, and tracked with  \emph{dia} `\textsc{3sg}' subsequently.

\ea\label{SewaktuDiDalam} 
\langinfo{Singapore Malay}{}{2017.SI.29--31}\\
\gll  Sewaktu     {di dalam} penjara, {\ob}Adam{\cb} tidak henti-henti    menangis.  {\ob}Dia{\cb} tidak lalu untuk makan.  {\ob}Dia{\cb} hanya duduk menangis {di dalam} penjara yang gelap lagi  berbau itu.\\
while  inside   jail   \textsc{pn} not   \textsc{red}-stop  \textsc{av.}cry \textsc{3sg} not   happen to   eat \textsc{3sg} only  sit  \textsc{av.}cry      inside   jail  \textsc{rel}  dark  \textsc{add}  smell  \textsc{dist}\\
\glt `While in jail, Adam did not stop crying. He could not eat. He only sat crying in the jail that was dark and smelly.'
\z

\noindent
We have shown that proper names are tracked with pronouns, and that their repetition creates a rhythm of sub-events. This strategy applies to the main character, the farmer. A somewhat different strategy is used to track the farmer's wife and child. Apart from repetition, particularly common is synecdoche (\isi{RefLex} \emph{r-given}); the farmer's wife and child are referred to as \emph{keluarga} `family', as in (\ref{PakSamadCuba}). In another version, the farmer's status is characterised as \emph{berumah-tangga} `married, having a family', or the couple is referred to with \emph{suami-isteri}.

\ea\label{PakSamadCuba} 
\langinfo{Singapore Malay}{}{2017.AM.26}\\
\gll Pak Samad       cuba mengeratkan hubungan-nya         dengan  {\ob}keluarga-nya{\cb}.\\
Mr  \textsc{pn} try  \textsc{av.}strengthen  relationship-\textsc{3poss} with   family-\textsc{3poss}\\
\glt `Pak Samad tried to improve his relationship with his family.'
\z

\noindent
In one version of \emph{Frog Story}, the boy is categorised with a proper name and tracked by repetition and pronouns (see \tabref{tab:Frog:given}). His dog and frog are categorised and tracked with possessives, highlighting their relationship to the boy.

\ea\label{KemudianAbuDan} 
\langinfo{Singapore Malay}{}{2013.SS.FrogStory.11}\\
\gll 	Kemudian     \ule{Abu}      dan   {\ob}anjing\ule{-nya}{\cb}        memanggil-manggil  {\ob}katak\ule{-nya}{\cb}        lalu tingkap  {\ob}bilik\ule{-nya}{\cb}.\\
subsequently \textsc{pn} and  dog-\textsc{3poss}    \textsc{red-}call  frog-\textsc{3poss}    through window  room-\textsc{3poss}\\
\glt `Then Abu and his dog called repeatedly for his frog through the window of his room.'
\z

\noindent
Categorisation with descriptions reflects the speaker's neutral \isi{epistemic stance} (see also \sectref{StanceAndReference}). Tracking of such referents is more elaborate and besides repetition, pronouns and zero anaphora also include demonstratives and particles (see also \sectref{MarkingStanceJointAttention}). 

Another tracking strategy involves relativisation. Unlike demonstratives or particles, however, relativisation can embed another perspective. In the final scenes of \emph{Pear Story}, the farmer is puzzled by seeing three boys walking by with his pears. The farmer is not aware that the boy who took one of his baskets shared the fruit with these boys when they helped him to pick up the scattered fruit. LN resolved this by constructing the three boys as new, taking up the farmer's perspective, as in (\ref{KemudianDiaTernampak}). Notice that the noun \emph{pear} combines with \emph{tadi}, conforming to the all-knowing perspective of the speaker-storyteller.

\ea\label{KemudianDiaTernampak} 
\langinfo{Singapore Malay}{}{2013.LN.PearStory.31}\\
\gll Kemudian     dia ter-nampak tiga,  {\ob}tiga  orang  budak yang sudah me-makan pear tadi{\cb}.\\
subsequently \textsc{3sg} \textsc{invol-}notice    three three \textsc{cl.human} child \textsc{rel}  already \textsc{av-}eat pear previous\\
\glt `Then he saw three\ldots, three boys who were eating the pears from earlier on.'
\z

\noindent
In CA's version in (\ref{DatangTigaBudak}), the passing boys are presented as the subject of an inverted \isi{intransitive clause} and modified by a \isi{relative clause} referring to the pears received for their help.

\ea\label{DatangTigaBudak} 
\langinfo{Singapore Malay}{}{2013.CA.PearStory.13}\\
\gll {\USSmaller}unclear{\USGreater} datang  {\ob}tiga  budak{\cb} yang mem-bantu budak yang {\USSmaller}unclear{\USGreater} makan buah yang di-petik.\\
{} arrive three boy   \textsc{rel}  \textsc{av-}help     boy   \textsc{rel} {}   eat   fruit \textsc{rel}  \textsc{pv-}pick\\
\glt `[While he was thinking], coming there were three boys who helped the [fallen] boy, eating the harvested fruit.'
\z

\noindent
Relative clauses are utilised in \emph{Pear Story} to distinguish between the children (boy, girl, and the three boys). In the case of the boy, reference is made to his fall, as  in (\ref{BudakYangChildren}). The false start with code-switching may reveal the decision-making of the speaker as to how to most effectively categorise the boy, i.e. with reference to boys who helped, or with reference to the fall. In general, the more elaborate descriptions distinguishing the children confirm our point about the simplifying effect of strong \isi{epistemic stance} on \isi{referent} \isi{categorisation} and tracking.

\ea\label{BudakYangChildren} 
\langinfo{Singapore Malay}{}{2013.LN.PearStory.25}\\
\gll Budak yang {CHILDRE\ldots}  {\ob}budak yang ter-jatuh    tadi     itu pun{\cb}  terus menunggang basikal-nya     kembali.\\
boy   \textsc{rel}  \textsc{cs.}children       boy   \textsc{rel}  \textsc{invol-}fall  \textsc{recent} \textsc{dist} \textsc{event} direct \textsc{av.}ride       bicycle-\textsc{3poss} back\\
\glt `The boy that children\ldots, the boy who fell just now continued riding his bicycle.'
\z

\noindent
\tabref{tab:GTSS:merged} summarises the expressions of the participants in \emph{Getting the Story Straight}, in the order in which they appear in the narrative (the first mention is underlined). The horizontal line divides the texts into two groups according to the \isi{epistemic stance}, taking the \isi{categorisation} of the farmer as a criterion. Above the line are seven texts where the farmer (and usually also his wife) is categorised with a proper name. In the remaining four texts, the speaker's neutral \isi{epistemic stance} is apparent in the \isi{categorisation} of the farmer with a description. In contrast, \isi{categorisation} of the police and friends is more uniform.

%\afterpage
%{
    %\clearpage% Flush earlier floats (otherwise order might not be correct)
%    \thispagestyle{empty}% empty page style (?)
        %\begin{landscape}% Landscape page
 
\begin{sidewaystable}
\caption{Categorisation and tracking of prominent human referents in \emph{Getting the Story Straight}}
\label{tab:GTSS:merged}\scriptsize
 \begin{tabularx}{\textwidth}{Xlllll}
  \lsptoprule
version & farmer & wife & child & friends & police\\
  \midrule
MLZ & \ule{\textsc{pn}}\footnote{\tiny \textsc{pn}: proper name}, \textsc{pro}\footnote{\tiny \textsc{pro}: pronominal element, including \emph{dia}, \emph{mereka}, and \emph{-nya}}, \textsc{n}(+\emph{itu}+\emph{pun}) &  \ule{\textsc{pn}}, \textsc{pro}, \textsc{n-poss}(+\emph{itu}), \textsc{n[n]} & \ule{\textsc{n-poss}}\footnote{\tiny \textsc{n-poss}: possessed noun, e.g. \emph{anak-nya} `their child', \emph{isterinya} `his wife', \emph{bapak Halim} `Halim's father', etc.} & \ule{\textsc{red-n-poss}}\footnote{\tiny \textsc{red-n}: reduplicated noun with an optional possessive \emph{-nya}, e.g. \emph{kawan-kawan(-nya)} or \emph{teman-teman(-nya)} `(his) friends'} & \textsc{n[\ule{n}]}\footnote{\tiny\textsc{n[n]}: both \textsc{n} repetition and synecdoche, e.g. \emph{pihak berkuasa} `authorities' or \emph{keluarga} `family, i.e. wife and child'}\\
MIZ & \ule{\textsc{pn}}, \textsc{pro} & \ule{\textsc{n-poss} + \textsc{pn}},  \textsc{pro}\footnote{\tiny includes also \emph{kedua-dua mereka} `both of them (i.e. wife and child)'},  \textsc{n-poss}, \textsc{n[n]} & \ule{\textsc{n-poss}}, \textsc{pro}, \textsc{n[n]} & \ule{\textsc{red-n-poss}}, \textsc{pro} & \textsc{n[\ule{n}]}, \textsc{pro}\\
AM & \textsc{\ule{pn}}(+\emph{pun}), \textsc{pro} & \ule{\textsc{n-poss} + \textsc{pn}}, \textsc{pro}, \textsc{n[n]} & \textsc{n-poss}\footnote{\tiny not refered to in the first picture frame where it occurs, but later}, \textsc{n[n]} & \textsc{\ule{red-n-poss}}, \textsc{pro} & \textsc{\ule{n}}, \textsc{n}+\emph{pun}\\
YAN &  \textsc{\ule{pn}}, \textsc{pro}&  \ule{\textsc{n-poss} + \textsc{pn}}, \textsc{pro}, \textsc{n[n]} &  \textsc{n-poss}, \textsc{n[n]}&  \textsc{\ule{pn}}, \textsc{pro}, \textsc{n[n]} &  \textsc{n[\ule{n}]}\\
SI &  \textsc{\ule{pn}}, \textsc{pro}, \textsc{n-poss}\footnote{\tiny the farmer, named here \emph{Adam} is referred to as \emph{suami-nya} `her husband', when his wife is reporting to the police} & \ule{\textsc{n-poss} + \textsc{pn}}, \textsc{pro},  \textsc{n-poss}, \textsc{n[n]}  & \textsc{\ule{n}-poss}, \textsc{n[n]}& \textsc{\ule{pn + n-poss}}, \textsc{pro},  & \textsc{\ule{num-cl-n}}, \\
 &  & &  & \textsc{red-n-poss} & \textsc{n[n]}\\
ISH & \textsc{\ule{pn}}, \textsc{pro}, \textsc{[n]}\footnote{\tiny \textsc{[n]}: synecdoche \emph{suami-isteri} `couple, husband and wife'} & \textsc{\ule{pn}}, \textsc{pro}, \textsc{n-poss}, \textsc{n[n]}\footnote{\tiny\textsc{n[n]}: both \textsc{n} repetition and synecdoche \emph{suami-isteri} `couple, husband and wife'} & \textsc{(red-)\ule{n-poss}} &  \textsc{\ule{red-n}-poss} &  \textsc{n[\ule{n}]}\\
LQ & \textsc{\ule{num-cl-n + pn}}, \textsc{pro}, \textsc{n[n]} & \textsc{\ule{num-cl-n}}, \textsc{pro}, \textsc{n[n]-poss}\footnote{\tiny \textsc{n[n]-poss}: repetition and synecdoche, including possessive marking} &  \textsc{\ule{num-cl-n + pn}}, \textsc{pro}, \textsc{[n]-poss}\footnote{\tiny \textsc{[n]-poss}: synecdoche \emph{keluarga-nya} `his family'} &  \ule{\textsc{red-n-poss}}, \textsc{pro} & \textsc{n[\ule{n}]}\\
\midrule
JUR & \ule{\textsc{n}}, \textsc{pro}, \textsc{n}, \textsc{n}+\emph{itu}, \textsc{n}+\emph{pun} & \ule{\textsc{n-poss}}, \textsc{n}, \textsc{pro} & \ule{\textsc{n-poss}}, \textsc{pro+num}\footnote{\tiny \textsc{pro+num}: \isi{pronoun} combined with a numeral, to refer to both the child and her mother, e.g. \emph{mereka dua-dua} `both of them'}, \textsc{n[n]} & \ule{\textsc{red-n-poss}}, \textsc{pro} & \ule{\textsc{n}}\footnote{\tiny \textsc{n}: simple repetition only}\\
ISM  & \ule{\textsc{n}}, \textsc{pro}, \textsc{n}+\emph{pun} & \ule{\textsc{n-poss}}, \textsc{pro} & \ule{\textsc{n-poss}} &  \textsc{\ule{red-n}-poss} &  \ule{\textsc{n}}\\
HZ & \textsc{\ule{num-cl-n}}, \textsc{pro}(+\emph{pun}), \textsc{n-poss}\footnote{\tiny \textsc{n-poss}: possessed noun \emph{suami-nya} `her husband' used when wife's perspective is given}, & \textsc{n\ule{[n]-poss}}, \textsc{pro}, \textsc{n}+\emph{pula} & \textsc{n\ule{[n]-poss}} & \ule{\textsc{red-n-poss}} & \textsc{\ule{n}}\\
 & \textsc{n}+\emph{tu}+\emph{pun}, \textsc{n}+\emph{pula},  \textsc{n}+\emph{tersebut} &   &  &  & \\
NZ & \textsc{\ule{num-cl-n}}, \textsc{pro}(+\emph{pun}), \textsc{n-poss}+\emph{pun}, & \textsc{n\ule{[n]-poss}}(+\emph{pun}), & \textsc{[n]-poss} & \textsc{\ule{n-poss}}, \textsc{red-n}+\emph{tersebut}, & \textsc{\ule{n}}+\emph{tersebut},\\
 & \textsc{n[n]}+\emph{(i)tu}+\emph{pun}, \textsc{n}+\emph{tersebut} &  &  &\textsc{n-poss}  & \textsc{[n]}+\emph{pula} \\
  \lspbottomrule
 \end{tabularx}
\end{sidewaystable} 

    %\end{landscape}
    %\clearpage% Flush page
%}

\noindent
The most significant patterns, discernible in the table, are the following: (i) the speaker's strong \isi{epistemic stance} is reflected in the \isi{categorisation} of prominent referents with proper names; (ii) repetition and pronouns are used to track them; (iii) neutral \isi{stance} leads to \isi{categorisation} with descriptions, typically an enumerated \isi{classifier phrase}; (iv) various types of NPs (including demonstratives and particles) and pronouns are used for tracking; and (v) the farmer's wife, child and friends are referred to with possessive phrases.

\subsection{Tracking animate and inanimate referents}\label{TrackAnimate}

Tracking of non-human referents (animate and inanimate) shows a split pattern, a more diverse one to track animates, especially when fabulated as capable of inner speech (thoughts, plans, or emotions). Inanimate referents, on the other hand, are rarely tracked beyond their first introduction. Discourse-persistent inanimate referents are tracked with various types of NPs, but never with pronouns. \tabref{tab:Jackal:given} and \tabref{tab:Frog:given} summarise the tracking devices in \emph{Frog Story} and \emph{Jackal and Crow}.


\begin{table}
\caption{Categorisation and tracking of referents in \emph{Jackal and Crow}}
\label{tab:Jackal:given}
 \begin{tabularx}{\textwidth}{Xlllll} 
  \lsptoprule
version & crow  & basket & fish & jackal  & tree \\
  \midrule
 MLZ & \ule{\textsc{num-cl-n}}  & \ule{\textsc{quant-n-rc}} & \ule{\textsc{red-n}} & \ule{\textsc{n}}\footnote{bare noun is followed by the additive focus particle \emph{pula} in (\ref{DahLandDia}), but we analyse it as scoping over the entire clause}  & \ule{\textsc{n-dem}} \\
   & \textsc{pro}\footnote{both the honorific \emph{beliau} and default \emph{dia} are used},  & &  & \textsc{pro}(+\emph{pun})\footnote{a range of 1st, 2nd and 3rd person pronouns are used}  &  \\
 & \textsc{n}(+\emph{pun}),  &  & \textsc{n}, &   &  \\
&  \emph{the}+\textsc{n}\footnote{codeswitching is used: \emph{the gagak} `the crow'} &  & \textsc{n}+\emph{itu} &   &  \\
  \midrule
OG & \ule{\textsc{num-cl-n}}   & \ule{\textsc{quant-n}}  & \ule{\textsc{n}}  & \ule{\textsc{num-cl-n}}   & n.a. \\
 & \textsc{pro},  &  &  & \textsc{pro},  &  \\
 & \textsc{n[n]}\footnote{\textsc{n[n]}: the crow is referred to as \emph{burung gagak} `crow', or as \emph{burung} `bird'}  &  & \textsc{n} & \textsc{n}  &  \\
 & \textsc{n[n]}+\emph{itu}  &  & \textsc{n}+\emph{itu} & \textsc{n}+\emph{itu}(+\emph{pun})  &  \\
  \lspbottomrule
 \end{tabularx}
\end{table}

\noindent
\tabref{tab:Jackal:given} shows that \isi{referent} \isi{categorisation} and tracking in \emph{Jackal and Crow} is quite uniform. A single \isi{referent}, the tree in which the crow perches, is completely omitted by OG. In contrast, the two versions of \emph{Frog Story} display the same \isi{epistemic stance} variation as the \emph{Getting the Story Straight} texts, as shown in \tabref{tab:Frog:given}.

%\afterpage{%
    %\clearpage% Flush earlier floats (otherwise order might not be correct)
%    \thispagestyle{empty}% empty page style (?)
    %\begin{landscape}% Landscape page
 
\begin{table}
\caption{Categorisation and tracking of referents in \emph{Frog Story}}
\label{tab:Frog:given}
 \resizebox{\textwidth}{!}{\begin{tabular}{@{}lllllllll@{}} 
  \lsptoprule 
version & boy & dog & frog & jar & forest & bees & rodent & hole\footnote{the cavity occupied by the owl}\\
  \midrule
OG & \ule{\textsc{num-cl-n}} & \ule{\textsc{num-cl-n}}  & \ule{\textsc{num-cl-n}}  & \ule{\textsc{num-cl-n}}  & \ule{\textsc{num-cl-n}}  & \ule{\textsc{n-rc}}  & \ule{\textsc{num-cl-n}}  & \ule{\textsc{num-cl-n}} \\

 & \textsc{pro}, & \textsc{pro}, & \textsc{pro},  & \textsc{pro},  &  &  &  & \\
 & \textsc{n-poss} & \textsc{n-poss}, & \textsc{n-poss} &  &  &  &  & \textsc{n-poss} \\
 & \textsc{n[n]}(+\emph{itu})  & \textsc{n}+\emph{itu}(+\emph{pun}) & \textsc{n}+\emph{itu} & \textsc{n[n]}+(\emph{itu}) & \textsc{n}(+\emph{itu}) & \textsc{n}(+\emph{itu}) & \textsc{n[n]}(+\emph{itu}) & \textsc{pp}\footnote{\textsc{pp}: prepositional phrase: the hole referred to as \emph{di dalam} `inside'}\\
  \midrule
  SS & \ule{\textsc{pn}}  & \ule{\textsc{n-poss}}  & \ule{\textsc{n-poss}} & \ule{\textsc{n-poss}} & \ule{\textsc{n}} & \ule{\textsc{n-rc}} & n.a. & \ule{\textsc{num-n-pp}}\\
 & \textsc{pn}, \textsc{pro}(+\emph{pun}), & \textsc{pro}(+\emph{pun}) &  &  &  &  &  & \\
& \textsc{(red-)n-poss} & \textsc{(red-)n-poss} & \textsc{n-poss} & \textsc{n-poss} &  & \textsc{red-n} &  & \\
 &  &  & \textsc{n}+\emph{itu} & \textsc{n}+\emph{itu} &  &  &  & \textsc{n}+\emph{itu}\\

  \midrule
    \midrule
 & owl &  rock & branch & antlers & deer & water & log & frogs   \\
  \midrule
OG & \ule{\textsc{num-cl-n}} &  \ule{\textsc{num-n}} & \ule{\textsc{num-n-rc}} & \ule{\textsc{n-poss}} & \ule{\textsc{num-cl-n}} & \ule{\textsc{n-rc}} & \ule{\textsc{num-n}}\footnote{the constituent could be interpreted as a compound \emph{satu akar pokok} `a tree root' or a possessive construction \emph{satu akar pokok} `a root of the tree'} & \ule{\textsc{n-poss}}\footnote{realised as a possessor in \emph{bunyi-bunyi katak `frog sounds' and treated as given thereafter}}   \\
 & \textsc{pro}, &   &  &  & \textsc{pro}, &  &  & \textsc{pro}, \textsc{n[n]},  \\
 & \textsc{n}(+\emph{itu}) & \textsc{n}+\emph{itu}   & \textsc{n}+\emph{itu} &  & \textsc{n}+\emph{itu} & \textsc{n} & \textsc{n}+\emph{itu} & \textsc{num-cl-n}+\emph{itu}   \\
 \midrule
SS & \ule{\textsc{num-cl-n}} &  \ule{\textsc{num-n}} & \ule{\textsc{num-n-rc}}\footnote{realised as \emph{satu ranting pokok yang\ldots} `a tree branch that \ldots'} & \ule{\textsc{n-poss}} & \ule{\textsc{n}} & \ule{\textsc{n}}\footnote{realised together with \emph{antlears} as \emph{tanduk rusa} `deer antlers'} & \ule{\textsc{n-rc}} & \ule{\textsc{n-poss}}   \\
&  &   &  &  &  &  &  & \textsc{num-cl-n}   \\
 & \textsc{n}+\emph{itu} &  \textsc{n}+\emph{itu} & \textsc{n}+\emph{itu} &  & \textsc{n}+\emph{itu} & \textsc{n}+\emph{itu} & \textsc{n} & \textsc{n[n]}(+\emph{itu})   \\
  \lspbottomrule
 \end{tabular}}
\end{table} 


    %\end{landscape}
    %\clearpage% Flush page
%}

\noindent
We now turn to the tracking of non-human referents. For animate referents, repetition and pronouns are common. In (\ref{NampaknyaKatakAbu}), the frog (\emph{katak Abu} `Abu's frog') is tracked with the possessive \emph{-nya}, partly because the frog is fabulated as an experiencer (capable of emotion), and thus the description of the boy and dog is consistent with the frog's perspective.

\ea\label{NampaknyaKatakAbu} 
\langinfo{Singapore Malay}{}{2013.SS.FrogStory.54}\\
\gll Nampaknya   {\ob}katak Abu{\cb}         sangat merindui kawan-kawan\ule{-nya}.\\
apparently frog  \textsc{pn} very   miss      \textsc{red}-friend-\textsc{3poss}\\
\glt `Apparently Abu's frog missed his friends.'
\z

\noindent
In (\ref{DahBeliauTernampak}), personal pronouns \emph{dia} and  \emph{beliau} (honorific) refer to the crow, where the honorific is a clue of speaker's sarcasm.

\ea\label{DahBeliauTernampak} 
\langinfo{Singapore Malay}{}{2013.MLZ.JackalAndCrow.141--2}\\
\gll 	Dah      {\ob}beliau{\cb}  ter-nampak beberapa bakul  yang terdapat ikan-ikan. SO   {\ob}dia{\cb} pergi ah,    terbang, terbang, terbang,  {\ob}dia{\cb} pergi dekat ikan tu, zoop!\\
already \textsc{3sg.hon} \textsc{invol-}see       few      basket \textsc{rel}  exist    \textsc{red}-fish \textsc{cs.}so \textsc{3sg} go    \textsc{part} fly      fly      fly      \textsc{3sg} go    near  fish \textsc{dist} \textsc{inter}\\
\glt `And he (the crow) saw several baskets filled with fish. So he went, flew, he came to the fish and went \emph{zoop}.'
\z

\noindent
The same version contains a mini-dialogue, shown in (\ref{SayaNakDengar}), where the first and second person pronouns refer to the jackal and crow, respectively. The follow-up comment where the speaker praises his own story-telling performance is another clue of the speaker's sarcasm.

\ea\label{SayaNakDengar} 
\langinfo{Singapore Malay}{}{2013.MLZ.JackalAndCrow.164}\\
\gll 	 {\ob}Saya{\cb} nak  dengar suara  {\ob}awak{\cb} yang merdu lah. Chey, macam betul   aja.\\
\textsc{1sg}  want hear   \isi{voice} \textsc{2sg}  \textsc{rel}  sweet \textsc{part} \textsc{inter}, kind  genuine \textsc{part}\\
\glt `I want to hear your sweet \isi{voice}, hey that kind of sounds just right.'
\z

\noindent
A summary of the \isi{categorisation} and tracking of non-human given referents in \emph{Jackal and Crow} and \emph{Frog Story} is given in \tabref{tab:Jackal:given} and \tabref{tab:Frog:given}. The common pattern is the limited variation in the description of inanimates, and the lack of tracking thereof. When tracked, the default is to include the \isi{demonstrative} \emph{itu}, which will be discussed in \sectref{GivenDem}.

Finally, \tabref{tab:Pear:merged} shows that the \isi{categorisation} and tracking of referents in \emph{Pear Story} is fairly uniform. Neither speaker takes a strong \isi{epistemic stance}, with the result that the tracking of the children is quite elaborate to distinguish the boy with the basket, from the girl on the bike and the boys who help him pick up spilled fruit.

%\afterpage{%
    %\clearpage% Flush earlier floats (otherwise order might not be correct)
%    \thispagestyle{empty}% empty page style (?)
    %\begin{landscape}% Landscape page
 
\begin{table}
\caption{Categorisation and tracking of referents in \emph{Pear Story}}
\label{tab:Pear:merged}
\resizebox{\textwidth}{!}{\begin{tabular}{@{}lllllllll@{}} 
  \lsptoprule
version & farmer & fruit & baskets & shepherd & boy & bicycle & girl & 3 boys  \\
  \midrule
CA & \ule{\textsc{num-cl-n}}, & \textsc{n} &  \textsc{num-n}\footnote{\textsc{num-n}: quantified noun, e.g. \emph{tiga bakul} `three baskets'} & n.a. & \ule{\textsc{num-cl-n}}, & \textsc{n}  & n.a. & \ule{\textsc{num-n}}\footnote{\textsc{num-n}: enumerated noun phrase \emph{tiga lagi budak kanak-kanak} `three small boys'} \\
 & \textsc{pro}, \textsc{n}+\emph{tadi} &  &  &  & \textsc{pro}, \textsc{n}+\emph{itu} &  &  & \textsc{pro}, \textsc{n(+num)}  \\
  & \emph{si}+\textsc{n}+\emph{itu} &  &  &  &  &  &  &   \\
\midrule
LN & \ule{\textsc{num-cl-n}}, & \textsc{n} & \textsc{n-pp}\footnote{\textsc{n-pp}: noun with a locative prepositional phrase locating the baskets in relation to the tree} & \ule{\textsc{n-poss}}, & \textsc{num-cl-n}, \textsc{pro}, & \textsc{n} & \textsc{num-cl-n}& \ule{\textsc{num-cl-n}}\footnote{realised as \emph{tiga orang dak laki} `three boys'}, \textsc{pro}+\emph{pun} \\
 & \textsc{pro}, \textsc{n}+\emph{itu},  &  &  & \textsc{pro} & \textsc{n}+\emph{lagi}+\emph{tadi}+\emph{itu} &  &  & \textsc{pro}+\emph{tadi}, \textsc{num-cl-n}  \\
 & \textsc{n}+\emph{tadi} &  &  &  &  &  &  & \textsc{red-n-poss}  \\
  \lspbottomrule
 \end{tabular}}
\end{table} 

    %\end{landscape}
    %\clearpage% Flush page
%}

\noindent
The effect of \isi{stance} and \isi{discourse} role on the tracking of referents is summarised in \tabref{tab:Stance:Discourse:Track}, whose structure parallels that of \tabref{tab:Stance:Discourse:Effect} above. 

\begin{table}
\caption{Effect of stance and discourse role on referent tracking}
\label{tab:Stance:Discourse:Track}
 \begin{tabularx}{\textwidth}{Xcc} 
  \lsptoprule
  & \multicolumn{2}{c}{\textsc{epistemic stance}} \\
\textsc{\isi{referent} category} & \textsc{strong} & \textsc{neutral}\\  
\midrule
+human & \textsc{pn}, \textsc{pro} & \textsc{pro}, \textsc{n}(+\emph{itu}/\textsc{part}) \\
+animate & \textsc{pro}(+\textsc{part}), \textsc{n}-\textsc{poss}/+\emph{itu} & \textsc{pro}, \textsc{n}(+\emph{itu}/\textsc{part}) \\
\textminus animate, +discourse-persistent &  \textsc{pro}, \textsc{n}-\textsc{poss}/+\emph{itu} &\textsc{n}(+\emph{itu}) \\
\textminus animate, \textminus discourse-persistent &\multicolumn{2}{c}{n.a.} \\
\lspbottomrule
 \end{tabularx}
\end{table}



%%%%%%%%%%%%%%%%%%%%%%%%%%%%%%%%%%%%%%%%%%%%
\section{Maintaining joint attention}\label{MarkingStanceJointAttention}
In the previous two sections we discussed the role of \isi{stance} for \isi{referent} \isi{categorisation} and tracking. This section focuses on another aspect of interaction and balancing of information disparity. This interactive aspect is part of Du Bois' \isi{stance} model, conceptualised as the alignment between the interlocutors \citep[cf.][171]{DuBois2007}. As Du Bois puts it: \emph{I evaluate something, and thereby position myself, and thereby align with you} (\citeyear[163]{DuBois2007}). Du Bois' \emph{alignment} falls within a larger notion of \emph{joint attention}, which is a type of social cognition \citep[cf.][]{tomasello1995joint}. \citet{Diessel2006demonstratives} applies the notion of \isi{joint attention} to demonstratives, whose primary roles he identifies in (i) locating referents relative to the \isi{deictic} centre, and (ii) coordinating the interlocutors' \isi{joint attention} \citep[469]{Diessel2006demonstratives}. 

While demonstratives are certainly the most prominent joint-attention coordinating devices in \ili{Malay} \citep[cf.][]{Himmelmann1996, Williams2009}, \ili{Malay} possesses a number of adnominal markers with a similar function, most importantly \emph{pun} and \emph{pula}.  In this section, we analyse the use of \ili{Malay} demonstratives and other adnominal markers in relation to coordination of interlocutors' \isi{joint attention} and show how the use of these devices is related to \isi{epistemic stance} and \isi{referent} \isi{categorisation}. The data suggests that neutral \isi{stance} and nominal expression of referents correlate with the use of demonstratives. By taking a neutral \isi{stance}, the speaker expects greater recognitional effort on the side of the hearer and compensates by providing more clues so that \isi{joint attention} can be maintained. We will demonstrate that these clues are demonstratives and particles. While key characters of the story do not require the use of these clues frequently, they are used whenever a more peripheral participant becomes a topic. 

%%%%%%%% Demonstrative %%%%%%%%%%%%%%%%%%%%%
\subsection{Malay demonstratives}\label{GivenDem}
\ili{Malay} demonstratives (both long and short forms) may introduce new information and track ``persistent" referents throughout \isi{discourse} \citep[241]{Himmelmann1996}. The use of demonstratives has implications for how the perspective of the hearer is constructed in interaction, as either having or lacking access to the intended \isi{referent} \citep{Williams2009}, and indicates the speaker's \isi{stance}. Our discussion of the data again follows the referential hierarchy, starting with human referents.

In the following fragment, the farmer is tracked with a \isi{demonstrative} phrase. Such use is common in texts where the \isi{referent} was categorised with a description.\footnote{An overview of all the expressions of key referents in \emph{Getting the Story Straight} can be found in \tabref{tab:GTSS:merged}.} The speaker prevents a possible misalignment with the hearer by using the \isi{demonstrative} and putting focus on the farmer, as affected by \emph{polis} `police', the local topical agent, which moves the plot.

\ea\label{PolisTibaDan} 
\langinfo{Singapore Malay}{}{2017.JUR.18}\\
\gll Polis  tiba   dan  menangkap  {\ob}petani itu{\cb}.\\
police arrive and  \textsc{av.}seize     farmer \textsc{dist}\\
\glt `The police arrived and caught the farmer.'
\z

\noindent
Another example is given in (\ref{AnaknyaTidakKenal}), which immediately follows  (\ref{DiaSangatGembira}). The distal \emph{itu} puts the gardener in focus, and constructs the child's  perspective as not recognising her father. The distal form does not have any spatial meaning here, because the man has just arrived in the scene. Instead, it creates an emotional distance, and marks the \isi{stance} of the child. It locates the responsibility for the non-recognition within the child, and ultimately in the abusive behaviour of her father.\footnote{\cite{Williams2009} describes a similar use of demonstratives in \ili{Indonesian} conversation. \cite{Djenar2014} shows that \emph{nih} and \emph{tuh} have presentative, directive and expressive functions, and explains why \emph{tuh} is used for recognitional and \isi{discourse} deixis.}

\ea\label{AnaknyaTidakKenal} 
\langinfo{Singapore Malay}{}{2017.ISM.21}\\
\gll Anak-nya      tidak kenal kepada  {\ob}pekebun  \ule{itu}{\cb}.\\
child-\textsc{3poss} not   know  to     gardener \textsc{dist}\\
\glt `His child did not recognise the gardener.'
\z

\noindent
The most common way to track \isi{discourse} persistent non-human referents is with \textsc{n}+\emph{itu}. The \isi{demonstrative} has a similar function as the \ili{English} definite article, marking the given referential status of the \isi{referent}. The distal form does not imply any contrast or any spatial relation, and its function is purely referential in marking the given \isi{referent} and perhaps aids the hearer in identifying the \isi{referent}. We do maintain the gloss \textsc{dist} in (\ref{DanMerekaBermain}), but a gloss \textsc{giv} for \emph{given} would be equally plausible.

\ea\label{DanMerekaBermain} 
\langinfo{Singapore Malay}{}{2013.OG.FrogStory.03}\\
\gll 	Umm.  Dan  mereka ber-main bersama-sama tiap-tiap malam, {di mana}  {\ob}anjing itu{\cb} akan tidur {di bawah}  {\ob}katil  {\ob}{budak lelaki} itu{\cb}{\cb}, sementara  {\ob}katak itu{\cb} akan tidur {di dalam} {peti gelas-nya}.\\
\textsc{hesit} and  \textsc{3pl}    \textsc{av-}play    \textsc{red-}together     \textsc{red}-each   night       where dog    \textsc{dist} will sleep below    bed boy   \textsc{dist}  while     frog  \textsc{dist} will sleep inside   glass.jar-\textsc{3poss}\\
\glt `Mmm. They played together every night and the dog would sleep under the boy's bed while the frog slept in its jar.'
\z

\noindent
The above characterisation of \emph{itu} as a definite marker is further supported by the code-switching patterns. Speakers of Colloquial \ili{Singaporean} \ili{Malay} frequently code-switch in \ili{English} across genres. Consider now (\ref{ThenTheGagak}), where the NP contains the \ili{English} definite article \emph{the}, where one would expect \emph{itu}. The \ili{English} \emph{then} corresponds to the eventive \emph{pun}, which will be discussed in \sectref{GivenPun}.

\ea\label{ThenTheGagak} 
\langinfo{Singapore Malay}{}{2013.MLZ.JackalAndCrow.167}\\
\gll 	THEN THE    gagak, THEN THE    gagak nyanyi\\
\textsc{cs.}then \textsc{cs.def} crow  \textsc{cs.then} \textsc{cs.def} crow sing\\
\glt `Then the raven sang.'\\
\z

\noindent
The following two examples from \emph{Frog Story} are a pair, where (\ref{SelepasItu}) shows the \isi{categorisation} of a pair of adult frogs in the final episode of the story. A description consisting of a possessive construction presents the frogs indirectly as ``emitters" of the sound.

\ea\label{SelepasItu} 
\langinfo{Singapore Malay}{}{2013.OG.FrogStory.30}\\
\gll {Selepas itu}, mereka um,       jalan ke   satu ah,   lagi  satu uh,  akar uh,   pokok ya, dan {budak lelaki} itu suruh anjing-nya     diam,  kerana  dia men-dengar ah,    {\ob}bunyi-bunyi    \ule{katak}{\cb} {di belakang} mm,       dahan  pokok itu ya.\\
thereafter   \textsc{3pl}    \textsc{part} move  to   one  \textsc{hesit} other one  \textsc{part} root \textsc{part} tree  yes and boy  \textsc{dist} ask   dog-\textsc{3poss} silent because \textsc{3sg} \textsc{av-}hear    \textsc{hesit} \textsc{red}-noise frog  behind \textsc{part} branch trunk \textsc{dist} yes\\
\glt `After that, they walked to another tree root, and the boy instructed his dog to be quiet because he heard frog noises behind the tree trunk.'
\z

\noindent
Subsequently, the frogs are tracked with \textsc{n}+\emph{itu} (\isi{RefLex} \emph{r-given}).

\ea\label{JadiDenganSenyap} 
\langinfo{Singapore Malay}{}{2013.OG.FrogStory.31}\\
	\gll Jadi dengan senyap, mereka dekat berhampiran dengan um,        {\ob}katak itu{\cb}, dan, akhirnya mereka jumpa dua ekor katak {di belakang} um,       pokok itu, ya.\\
so   with   silence \textsc{3pl}    near  adjacent    with   \textsc{part} frog  \textsc{dist}  and  finally  \textsc{3pl}    find two \textsc{cl.animal} frog  behind     \textsc{part} tree  \textsc{dist}  yes\\
\glt `So with silence, they approached close to the frog and finally they met two frogs behind the tree. '
\z

\noindent
The proximal \emph{ini} is used less frequently and does entail that the \isi{referent} is spatially proximate. The viewpoint from which the proximity is constructed can shift and be located within the participants. In (\ref{JadiMMSelelpas}), the boy's perspective is taken to refer to the frogs, as well as to the relative temporal \emph{ini} `now', located within the story.

\ea\label{JadiMMSelelpas} 
\langinfo{Singapore Malay}{}{2013.OG.FrogStory.33}\\
\gll 	Jadi mm,       selepas      {budak lelaki} itu, ber-cerita-kan      kepada kedua, uh,    {\ob}ibu    dan  bapa katak \ule{ini}{\cb}, bahawa ia     mahu mem-bawa balik, uh    katak yang sebelum \ule{ini} berada di rumah-nya.\\
so   \textsc{part} subsequently boy   \textsc{dist}  \textsc{av-}tell-\textsc{appl}    to     couple \textsc{part} mother and  father frog  \textsc{prox} \textsc{comp}   \textsc{3sg} wish \textsc{av-}carry   return \textsc{part} frog  \textsc{rel}  previously \textsc{prox}  be     in house-\textsc{3poss}\\
\glt `So after the boy explained to both the father and mother frog that he wanted to bring back that frog that before this was in his house.'
\z

\noindent
Apart from the spatial \emph{ini} and \emph{itu}, there are three more \isi{deictic} forms which do not have spatial uses, but are common in \isi{discourse}: \emph{tadi} `recently mentioned', \emph{tersebut} `aforementioned', and \emph{si} `familiar', which will be described below. Their use correlates with a neutral \isi{epistemic stance} and \isi{categorisation} with descriptions, except for \emph{si}, which expresses familiarity and therefore marks a stronger \isi{epistemic stance}.

The \isi{demonstrative} \emph{tadi} `recently mentioned' is a dedicated \isi{anaphoric} form derived from an adverbial meaning `earlier' \citep[133]{Sneddon2012}. It is likely grammaticalised to the adnominal position through a \emph{yang} modifier construction: \emph{N yang tadi} > \emph{N tadi}. In one version of the \emph{Pear Story}, it tracks the farmer picking fruit. The example given in (\ref{JadiBilaKita}) is beautiful, because it verbalises the intention behind using \emph{tadi} in the preceding phrase \emph{kita patah balik\ldots} `let us return back'.\largerpage

\ea\label{JadiBilaKita} 
\langinfo{Singapore Malay}{}{2013.CA.PearStory.11}\\
\gll Jadi bila, kita {patah balik} kepada  {\ob}perkebun \ule{tadi}{\cb},    masa dia turun daripada pokok dia nampak tadi,    dia nampak agak      aneh  kerana  sebab masa dia naik  ada      tiga  bakul.\\
so   when  \textsc{1pl.incl}  turn.back   to     farmer   \textsc{recent} time \textsc{3sg} descend from     tree  \textsc{3sg} see    recently \textsc{3sg} see    slightly weird because reason time \textsc{3sg} climb be three basket\\
\glt `So back to the farmer from earlier, the time he came down from the tree he found it weird as he last saw three baskets.'
\z

\noindent
The \isi{anaphoric} \isi{demonstrative} \emph{tersebut} `aforementioned, that' is used with expressions referring to the farmer in \emph{Getting the Story Straight}. Singapore \ili{Malay} speakers base some of their stylistic preferences on their formal education; the use of particles and of the \isi{demonstrative} \emph{tersebut} strikes native speakers as formal and rote-like. In NZ version, where \emph{tersebut} is used more than in all the other texts combined, the particle is used to track the farmer, his friends and the police (see \tabref{tab:GTSS:merged}). Apart from tracking, \emph{tersebut} puts the focus on the given \isi{referent}. We gloss it as \textsc{given.foc} and translate it with the \ili{English} \emph{that}, which can also have a focusing role. Its use correlates with the neutral \isi{epistemic stance} and \isi{categorisation} of referents with descriptions. Its extensive use by NZ is illustrated in (\ref{DalamKemarahannyaItuDia}). We believe that the frequent use is a personal characteristic of NZ, rather than a general pattern.\footnote{Note that the NP \emph{seorang tua yang\ldots pun} combines with \emph{pun}, while newly introduced into \isi{discourse}, but immediately cast as topic. The particle \emph{pun} seems to work in tandem with \emph{tersebut}, where one marks the \isi{new topic} and the other links explicitly the relevant given \isi{referent}.}

\ea\label{DalamKemarahannyaItuDia} 
\langinfo{Singapore Malay}{}{2017.NZ.05--08}\\
\gll Dalam kemarahannya     itu dia pun  menumbuk isterinya. Se-orang    {orang tua} yang ter-lihat   {\ob}kejadian \ule{tersebut}{\cb}  pun, uh,   mmm,  memanggil polis  dan   {\ob}polis \ule{tersebut}  pun{\cb}  uhhh, menangkap  {\ob}lelaki \ule{tersebut}{\cb}. Di balai   polis  pula, uh,   isteri-nya     pun  mem-beritahu keterangan tentang  {\ob}kejadian \ule{tersebut}{\cb}  kenapa ia  terjadi. Suami-nya       pun  takut  dengan, aah,      apa  yang akan menjadi terhadapnya.\\
in anger-\textsc{3poss} \textsc{dist} \textsc{3sg} \textsc{event} \textsc{av.}punch wife-\textsc{3poss} one-\textsc{cl.human} old.man   \textsc{rel}  \textsc{invol-}see event   \textsc{given.foc} \textsc{top2} \textsc{hesit} \textsc{hesit} \textsc{av.}call      police and  police \textsc{given.foc} \textsc{event} \textsc{hesit} \textsc{av.}catch     man    \textsc{given.foc} in station police then  \textsc{hesit} wife-\textsc{3poss} \textsc{top2} \textsc{av-}report      testimony  about   event    \textsc{given.foc} why    \textsc{3sg} happen husband-\textsc{3poss} \textsc{top2} afraid with    \textsc{hesit} what \textsc{rel}  will happen  about-3\\
\glt `In his anger he then punched his wife. An old man who saw that incident then called the police and those police then caught that man. Then at the police station, his wife explained that incident and why it had happened. Her husband then got frightened over what would happen to him.'
\z

\noindent
In our corpus, the \isi{demonstrative} \emph{si} is used sparsely. Traditional grammars attribute \emph{si} a diminutive function \citep[146]{Sneddon2012} and report that it is never used in address terms, but only in reference to somebody who is not the hearer \citep[374]{Sneddon2012}. The \emph{Wiktionary} entry for \emph{si} contains an accurate characterisation; in addition to  `friendly connotation', `diminutive', and `friendly \isi{categorisation}', it also lists `generic \isi{categorisation}', exemplified in (\ref{SiAyahHarus}).

\ea\label{SiAyahHarus} 
\langinfo{Indonesian}{}{Wiktionary.si$\#$Indonesian}\\
\gll  {\ob}Si ayah{\cb} harus belajar mengenal  {\ob}si anak{\cb}.\\
\textsc{} father must learn \textsc{av.}know \textsc{} child\\
\glt `The father has to learn to know the child.'
\z

\noindent
We propose that \emph{si} is a marker of familiarity, restricted to human referents. It is an expression of a strong \isi{epistemic stance}.\footnote{The notion of familiarity is defined by \cite[278]{Gundel1993} as a special cognitive status where the hearer already has a representation in memory, either in long-term memory, absence of recent mention, or in short-term memory, if he has.} \emph{Si} draws interlocutors' attention to a \isi{referent} by presenting it as familiar, i.e. identifiable within one's knowledge, or recent \isi{discourse}. Tracking proper names with a \emph{si} phrase follows the \emph{triangular} pattern of person reference identified in \citep[229--230]{Haviland2007}, where a new \isi{referent} is anchored in relation to both the speaker and the hearer. The \emph{si} phrase is an explicit anchoring effort in relation to the familiar knowledge of the hearer. Within the \isi{stance} framework proposed by  \cite{DuBois2007}, it is also an alignment device which explicitly interacts with the interlocutors' perspective.

It is not relevant that the familiarity is only constructed as such, because existing familiar referents are identified in exactly the same way, as we will show in (\ref{AbehTakBelajar}). In (\ref{SemasaDiaMemetik}), a discourse-recent \isi{referent} marked with \emph{si} is presented. \cite{Sukamto2013} observes a similar pattern in written \ili{Indonesian} accounts of \emph{Pear Story}. Expressions re-activating the given participants tend to be highly specified, and combine with both \emph{si} and \emph{sang} in the \ili{Indonesian} texts. 

\ea\label{SemasaDiaMemetik} 
\langinfo{Singapore Malay}{}{2013.CA.PearStory.05}\\
\gll Semasa dia memetik buah  dia atas, ada      se-orang budak me-naiki basikal dan  dia ter-nampak buah  {di dalam} bakul  itu lalu    dia memikir harus-kah      dia meng-ambil tetapi memandangkan  {\ob}si   perkebun itu{\cb} begitu perihatin dengan memetik buah  {di atas} pokok lalu    dia meng-ambil satu bakul  tampa   izin.\\
when   \textsc{3sg} \textsc{av.}pick   fruit \textsc{3sg} above exist one-\textsc{cl.human}  boy   \textsc{av-}travel.by    bicycle and  \textsc{3sg} \textsc{invol-}notice    fruit inside   basket \textsc{dist} then \textsc{3sg} \textsc{av.}think.about need-\textsc{q.part} \textsc{3sg} \textsc{av-}take      but considering         \textsc{familiar} farmer   \textsc{dist} so concerned with   \textsc{av}.pick   fruit on.top tree  then \textsc{3sg} \textsc{av-}take      one  basket without approval\\
\glt `When he picked the fruits above, a boy riding a bicycle saw fruits in the basket. Then he thought whether he should take some, but considering that our farmer was so concerned with picking fruits, he actually took one whole basket without permission.'
\z

\noindent
In our Singapore \ili{Malay} corpus, \emph{si} is used invariably to refer to relatives, partners or friends who do not take part in the interaction. The fragment in (\ref{KauSukaTak}) is taken from an interview with an elderly speaker of Singapore \ili{Malay}, who describes here how she got engaged. Her future father-in-law used to ask her, whether she had yet found a \emph{mata-air} `beloved' and whether she liked his son (absent during the exchange).

\ea\label{KauSukaTak} 
\langinfo{Singapore Malay}{}{2016.BandarGirls.202}\\
\gll Kau suka tak, dengan {si\ldots} Arsyad{\USQMark}\\
\textsc{2sg}  like not  with   \textsc{familiar}   \textsc{pn}\\
\glt `Do you like [our] Arsyad, don't you?'
\z

\noindent
In (\ref{AbehTakBelajar}), a mother asks whether her son, who is preparing for a math exam,  is finished with his tutor (absent during the exchange). This is the first mention of the tutor in the conversation, and later in that same conversation, the tutor is tracked with \emph{dia}.

\ea\label{AbehTakBelajar} 
\langinfo{Singapore Malay}{}{2013.SNS.Exam.17}\\
\gll Abeh tak belajar eh{\USQMark} Dah     habis   {\ob}si    dia tu{\cb}  ajar  dah     habis{\USQMark}\\
then not  study   \textsc{q.part} already finish \textsc{familiar}   \textsc{3sg} \textsc{dist} teach already finish\\
\glt `Why don't you prepare anymore? It is done what he [the tutor] taught you?'
\z

\noindent
In our narrative corpus, human referents categorised with proper names may be accompanied by an appositive \emph{si} phrase. In (\ref{SewaktuSedang}), the vegetable seller is constructed as familiar to the farmer, amplifying the effect of the accusation and explaining the rage that follows.\footnote{The man introduced in the drunk gossip (see \figref{fig:StraightStory}, frame 4), is sometimes given a name, such as \emph{Encik Romi} in (\ref{RashidMenceritakan}), or is referred to with a proper name followed by a nominal marked with \emph{si}, such as \emph{Leyman, si penjual surat khabar} `Leyman, the news agent'.} 

\ea\label{SewaktuSedang}  
\langinfo{Singapore Malay}{}{2017.SI.15}\\
	\gll sewaktu     sedang mabuk Abu         mem-beritahu Adam bahawa dia ter-nampak Hawa, isteri Adam, sedang ber-mesra-mesra      bersama   {\ob}\textbf{Sani},       \textbf{si}  \textbf{penjual} \textbf{sayur}{\cb}     di pasar.\\
while  \textsc{prog}   drunk \textsc{pn} \textsc{av-}tell        \textsc{pn} \textsc{comp}   \textsc{3sg} \textsc{invol-}see       \textsc{pn}  wife   \textsc{pn} \textsc{prog}   \textsc{av-}\textsc{red}-cozy  together \textsc{pn} \textsc{art} seller  vegetable in market\\
\glt `While he was drunk, Abu told Adam that he saw Hawa, Adam's wife, behaving  in a friendly way with Sani, [you know] the vegetable seller at the market.'
\z

\noindent
In summary, \ili{Malay} \emph{si} interacts with a specific layer of the hearer's memory: either with the recent memory, or with personal knowledge and stereotypes. Marking unknown and unfamiliar referents with \emph{si} is a request for cooperation to either fill out the speaker's intention, and accept the \isi{referent} in a common ground (in statements), or to supply the relevant knowledge in the next turn (in questions). It is the ultimate device forcing \isi{joint attention}.\footnote{There are some interesting parallels with other markers of familiarity, such as the New Zealand \emph{y'know} \citep[69]{Stubbe1995}, the \ili{Abui} hearer-oriented forms \citep{KraDel2015Definiteness}, or the more grammaticalised systems of engagement \citep{Evans2017a, Evans2017b}.}

We will now proceed to discuss the \ili{Malay} particles \emph{pula}, \emph{lagi}, and \emph{pun}, whose function in manipulation of \isi{joint attention} is even more complex than that of the demonstratives discussed here.

%%%%%%%% pula %%%%%%%%%%%%%%%%%%%%%
\subsection{Particle \emph{pula}}\label{GivenPula}
The \ili{Malay} particle \emph{pula} (colloquial \emph{pulak}) is traditionally characterised as an additive focus particle \citep[236]{Sneddon2012}. \cite[pula(k)]{Nomoto2017} distinguishes between two  functions of the \ili{Malay} \emph{pula}: (i) when placed after the predicate, the particle indexes the speaker's \isi{epistemic stance} --- the situation is marked as not conforming to the speaker's expectation, as surprising, or as evoking doubts; (ii) when combined with nominals, \emph{pula} encodes contrast, but also interacts with expectation. 

Both (\ref{DahLandDia}) and (\ref{DiSatuLadangPula}) employ the additive \emph{pula} when the jackal is categorised with a description.\footnote{Note that there are several additive markers in \ili{Malay}. \cite[91]{Forker2016} discusses only \emph{pun} as additive, while \cite[27]{Goddard2001} calls both \emph{pun} and \emph{pula} emphatic. \cite[236]{Sneddon2012} considers both \emph{juga} and \emph{pula} additive markers, which indicate that ``the focused part is an addition".} In (\ref{DahLandDia}), the additive \emph{pulak} marks the existence of the newly introduced jackal as a somewhat unexpected addition to the \isi{discourse}. The speaker perhaps contradicts the reasonable anticipation of the bird eating the fish, so the appearance of a hungry jackal presents an unexpected twist in the story.\footnote{\ili{Malay} speakers in Singapore are taught in \ili{Malay} language composition classes that the particles \emph{pula} and \emph{pun} make their style ``more interesting" or ``engaging", and mark the ``climax". We believe that at least in some cases, \ili{Malay} speakers may be using these particles for such ``aesthetic" reasons. The aesthetic function of \emph{pun}, as a marker of a particular style is also discussed by \cite[107]{Cumming1991}.}  After all, the fable is well-known, and it is reasonable to expect that the hearer is familiar with the plot. 

\ea\label{DahLandDia} 
\langinfo{Singapore Malay}{}{2013.MLZ.JackalCrow.149}\\
\gll 	Dah     LAND,   dia alih-alih ni   ada    {\ob}musang{\cb} \textbf{pulak} dia nampak.\\
	already \textsc{cs}.land \textsc{3sg} suddenly  \textsc{prox} exist jackal    \textsc{add}  \textsc{3sg} see\\
\glt `And as it landed, the crow suddenly saw that there was also a fox (there).'
\z

\noindent
In the second text, the jackal is introduced as the subject of an inverted existential \isi{clause} with an enumerated classifier structure in (\ref{DiSatuLadangPula}). The jackal is linked to the already known crow with the \isi{relative clause}, where the crow is the object of the involuntary action verb \emph{terlihat} `happen to see'. The additive \emph{pula} marks the newly introduced location, effectively extending the space in which the narrative is constructed. In terms of \isi{joint attention}, the particle forces an update. It constructs the extension of the space in which the story takes place as unexpected or surprising.

\ea\label{DiSatuLadangPula} 
\langinfo{Singapore Malay}{}{2013.OG.JackalCrow.04}\\
\gll 	Di satu ladang \textbf{pula}  ada    {\ob}se-ekor serigala{\cb} yang terlihat   {burung gagak} itu terbang bersama  ikan dalam mulut burung itu.\\
		in one  field  \textsc{add}  exist one-\textsc{cl.animal}   jackal    \textsc{rel}  spot crow         \textsc{dst} fly     together fish inside beak  bird   \textsc{dst}\\
	\glt `In a field, there was a jackal that saw the crow flying with the fish in its mouth.'
\z

\noindent
Example (\ref{DenganItuDiaHarus}) shows the \isi{contrastive} function of \emph{pula}, where the benefit of the police action for the farmer's wife has to be considered in parallel with the punishment of her husband.

\ea\label{DenganItuDiaHarus} 
\langinfo{Singapore Malay}{}{2017.HZ.09--10}\\
\gll {Dengan itu}, dia harus pergi ke,  uh,   {pihak   polis}  dan  beritahu tentang apa  yang terjadi. Um,    {\ob}suami-nya       \ule{pula}{\cb} berasa amat menyesal  akan  apa  terjadi, dan  beliau  amat risau  tentang, um,   apa  yang akan terjadi kepada-nya      iaitu, um,   beliau  harus {di- di-} di-letakkan {di dalam} lockup dan  di-belasah      oleh {pihak polis}.\\
therefore   \textsc{3sg} must  go    to   \textsc{hesit} police and  inform   about   what \textsc{rel}  happen \textsc{hesit} husband-\textsc{3poss} \textsc{con.foc} feel   very \textsc{av.}regretful about what happen   and  \textsc{3sg.hon} very uneasy about \textsc{hesit} what \textsc{rel}  will happen  to-3      namely \textsc{hesit} \textsc{3sg.hon} must  {}  \textsc{pv-}place         inside jail   and  \textsc{pv-}beat.up by    police\\
\glt `So now she had to go to the police and tell them what happened. Her husband (on the other hand) felt very regretful about what had happened, and he was very worried about what would happen to him, that is, he had to be detained in a jail cell and beaten by the police.'
\z

\noindent
The particle \emph{pula} also occurs with left-dislocated locative elements. Its function appears to be to move the narrative along to another location. We have seen one example of this use in  (\ref{DiSatuLadangPula}) and give another in (\ref{DiBalaiPolisPula}) below.

\ea\label{DiBalaiPolisPula} 
\langinfo{Singapore Malay}{}{2017.NZ.07}\\
\gll  {\ob}Di balai   polis  \ule{pula}{\cb}, uh,   isteri-nya     pun  mem-beritahu keterangan tentang kejadian \ule{tersebut}  kenapa ia  terjadi.\\
in station police \textsc{add}  \textsc{hesit} wife-\textsc{3poss} \textsc{top2} \textsc{av-}report      testimony  about   event    \textsc{given.foc} why    \textsc{3sg} happen\\
\glt `Then at the police station, his wife explained that incident and why it had happened.'
\z

\noindent
By using \emph{pula}, the speaker proposes a broadening or update of \isi{joint attention}. In this function \emph{pula} is similar to the demonstratives discussed in \sectref{GivenDem}, because the ``field" of \isi{joint attention} remains essentially the same. 
In the next section we will discuss the use of \emph{lagi}, another additive particle, whose use seems to be more restricted, but allows for scope manipulation.

\subsection{Particle \emph{lagi}}\label{ParticleLagi}
The particle \emph{lagi} indicates repetition with predicates, but with adnominal quantifiers, it has an additive function. The additive function is illustrated in  (\ref{DalamPerjalananPulang}), where the particle highlights that the reference is made to all members of the group \citep[84--85]{Forker2016}. 

\ea\label{DalamPerjalananPulang} 
\langinfo{Singapore Malay}{}{2013.CA.PearStory.08}\\
	\gll Dalam perjalanan pulang \ule{budak} \ule{itu} dengan {tidak sengaja} ter-langgar    batu lalu dibantu      oleh  {\ob}tiga \ule{lagi}  budak kanak-kanak{\cb} untuk mengumpulkan, mem-bangunkan basikal-nya     dan  buah-buahan yang ter-jatuh.\\
while drive      home   child \textsc{dist} with   accident      \textsc{invol}-hit     rock then \textsc{pv}-help  by   three more  boy   child       to    \textsc{av.}collect       \textsc{av-}put.upright  bicycle\textsc{-3poss} and  fruit       \textsc{rel} \textsc{invol-}fall\\
\glt `On the way home, the boy accidentally bumped into stones and is assisted by three other young children to collect the bicycle and fallen fruits.'
\z

\noindent
The additive \emph{lagi} also creates a relationship with the boy, who is the topic of the sentence. Within the \isi{RefLex} scheme, this \isi{referent} is classified as \emph{r-new}, but the presence of the additive marker suggests that this may be a referential type, not distinguished by the \isi{RefLex} Scheme. In terms of \isi{joint attention} manipulation, \emph{lagi} emphasises the existence of another \isi{referent} which should be included in the focus. Additives are known to be scope sensitive \citep[72]{Forker2016}. In (\ref{DalamPerjalananPulang}), the additive marker follows the quantifier, highlighting the precise quantity of the added referents. In the next section we will discuss the use of \emph{pun}, which essentially marks a proposal for a \isi{joint attention} shift.

\subsection{Particle \emph{pun}}\label{GivenPun}
%%%%%%%% pun %%%%%%%%%%%%%%%%%%%%%
The particle \emph{pun} is more frequent than other particles and demonstratives in our \emph{Getting the Story Straight} corpus. This particle has received much attention in the literature, and is treated in the greatest detail in \cite{Goddard2001}, who provides an exhaustive overview of earlier studies (p. 29--30). In our discussion, we adhere to Goddard's analysis. The most common use in our data, is the ``second-position \emph{pun}" which highlights the sentence topic \citep[31]{Goddard2001}. \cite[107]{Cumming1991} suggests that \emph{pun} is a resumptive topic marker attached to left-dislocated noun phrases. Its distribution is further affected by \emph{individuation} of the \isi{referent}, its \emph{semantic role}, its \emph{introduction} into the \isi{discourse}, and the \emph{eventiveness} of the description. The first function is well attested in our narratives; \emph{pun} frequently marks a switch in topic as participants take over the agency in moving the plot forward. One such sequence is given in (\ref{BilaPolisTiba}).

\ea\label{BilaPolisTiba} 
\langinfo{Singapore Malay}{}{2017.ISM.11--13}\\
\gll Bila polis  tiba,   {\ob}pekebun  \ule{pun}{\cb}  di-tangkap. Di mahkamah, isteri-nya     memberi, ah,   tahu hakim apa  yang telah   terjadi.  {\ob}Pekebun  \ule{pun}{\cb}  di-jatuhkan      {hukuman penjara}.\\
when police arrive gardener \textsc{top2} \textsc{pv-}catch in court     wife-\textsc{3poss} give     \textsc{hesit} know judge what \textsc{rel}  already happen gardener \textsc{top2} \textsc{pv-}hand.down     jail.sentence\\
\glt `When the police came, the farmer was arrested. In court, his wife told the judge what had happened. The farmer was then sentenced to a jail term.'
\z

\noindent
In (\ref{PakSamadPunMenceritakan}), which follows directly from (\ref{PakSamadCuba}), \emph{pun} amplifies the \isi{eventiveness} of the sequence (i.e. the progress of the plot). Note that the translation attempts to capture this with the \ili{English} adverb \emph{then} in both sentences.\footnote{Note also the use of the active \isi{voice} in both clauses, highlighting their \isi{eventiveness} (cf. \citealt[this volume]{Djenar2018}).}

\ea\label{PakSamadPunMenceritakan} 
\langinfo{Singapore Malay}{}{2017.AM.27--28}\\
\gll  {\ob}Pak Samad      \ule{pun}{\cb}  men-ceritakan pengalaman-nya     {di dalam} penjara dan  men-jelaskan bahawa dia menyesal dengan tindak-laku-nya.  {\ob}Pak Samad       \ule{pun}{\cb}  ber-janji dengan anak-nya      bahawa dia akan mem-bawa anak-nya      ke jalan-jalan {keesokan          hari}.\\
Mr  \textsc{pn} \textsc{event} \textsc{av-}tell         experience-\textsc{3poss} inside   prison  and  \textsc{av-}explain     \textsc{comp}   \textsc{3sg} \textsc{av.}regret   with   actions-\textsc{3poss} Mr  \textsc{pn} \textsc{event} \textsc{av-}promise  with     child-\textsc{3poss} \textsc{comp}   \textsc{3sg} will \textsc{av-}take    child-\textsc{3poss} to walk.around the.following.day\\
\glt `Pak Samad then told the story of his experiences in jail and made it clear that he regretted his actions. Pak Samad then promised his child that he would take him for a walk the next day.'
\z

\noindent
A similar instance of \emph{pun} amplifying the progress of the plot (i.e. \isi{eventiveness}) is shown in (\ref{DiaSangatGembira}). In colloquial Singapore \ili{English}, the particle \emph{pun} is often translated with \emph{then}, which has the same function in marking the previous event as completed and a new one as commencing.\footnote{Hiroki Nomoto has suggested to us that perhaps the core function of the particle is to indicate a \isi{clause} relationship between two clauses which are told in their order of occurrence, but the particle has to be placed after the subject of the second \isi{clause} \citep[pun]{Nomoto2017}.}

\ea\label{DiaSangatGembira} 
\langinfo{Singapore Malay}{}{2017.ISM.19--20}\\
\gll Dia sangat gembira dapat me-nikmati {cahaya matahari}.  {\ob}Pekebun  \ule{pun}{\cb}  pulang ke   rumah-nya.\\
\textsc{3sg} very elated         get   \textsc{av-}enjoy     sunlight gardener \textsc{event} return to   house-\textsc{3poss}\\
\glt `He was very happy that he got to enjoy the sunshine. The farmer then returned to his house.'
\z

\noindent
\cite[54]{Goddard2001} reports that the \emph{topic focus} function is the most common in his written \ili{Malay} corpus. In our narrative data, the \emph{event sequence} function is more common. An instance of topic focus is given in (\ref{AkuPunLapar}), where the jackal, upon spotting the crow with the fish, is constructed as talking to itself.

\ea\label{AkuPunLapar} 
\langinfo{Singapore Malay}{}{2013.MLZ.JackalAndCrow.151}\\
\gll 	 {\ob}Aku \ule{pun}{\cb}  lapar  ah.\\
	\textsc{1sg} \textsc{top2} hungry \textsc{part}\\
\glt `I am also hungry.'
\z

\noindent
The presence of \emph{resumptive topic} resets the reference of the third person \isi{pronoun} \emph{dia} and \emph{ia}. In (\ref{JadiSerigalaItuPunIngin}), the jackal is referred to as \emph{ia}, while the fish and crow require nominal expressions. The minimisation of the expression of the topic after it has been focused with \emph{pun} resembles the general tendency for minimisation of reference \citep{Heritage2007, SacksSchegloff2007}. We take this as a signal that \emph{pun} indicates a shift of \isi{joint attention} to a new ``field", which is accompanied by a reset in the scope of \isi{anaphoric} devices. 

\ea\label{JadiSerigalaItuPunIngin} 
\langinfo{Singapore Malay}{}{2013.OG.JackalAndCrow.05--6}\\
\gll 	Jadi {\ob}serigala itu \ule{pun}{\cb}  ingin me-makan ikan itu kerana   {\ob}ia{\cb}  sangat lapar. Jadi  {\ob}ia{\cb}  fikir  {\ob}ia{\cb}  mahu ikan yang {di dalam} mulut {burung gagak} itu.\\
		so   jackal    \textsc{dist} \textsc{top2} wish  \textsc{av-}consume fish \textsc{dst} because \textsc{3sg} very   hungry so   \textsc{3sg} think \textsc{3sg} want fish \textsc{rel}  inside   mouth crow         \textsc{dst}\\
	\glt `The jackal wanted to eat the fish because it was so hungry. And it thought, it wanted the fish in the crow's mouth.'
\z

\noindent
Example (\ref{GagakPunSedih}) summarises the outcome for the crow and clearly illustrates the \emph{event sequence} focus function of \emph{pun} \citep[cf.][38]{Goddard2001}.

\ea\label{GagakPunSedih} 
\langinfo{Singapore Malay}{}{2013.MLZ.JackalAndCrow.173}\\
\gll 	 {\ob}Gagak pun{\cb}  sedih sebab   dia kena tipu, bosan.\\
crow \textsc{event} sad   because \textsc{3sg} \textsc{pass} cheat      disgusting\\
\glt `The crow was sad because it got cheated, disgusting.'
\z

\noindent
The particle \emph{pun} does not occur in our texts with inanimates, but this is just a consequence of the construction of the plot in the narratives which we focus on here. There are instances of its use in our Singapore \ili{Malay} corpus, such as (\ref{MemangTakdeJumpa}), which describes the shortage of rice during WWII. \emph{Pun} here highlights the food shortage as a local topic and brings the focus on \emph{porridge}, lexically tracking the topic \emph{beras} (\isi{RefLex} \emph{r-given}, \emph{l-accessible-sub}).

\ea\label{MemangTakdeJumpa} 
\langinfo{Singapore Malay}{}{2016.BIZ.45}\\
\gll 	Memang    takde      jumpa beras, lah, nanti masak, ah,   bikin bubur    ke,  bikin, kalau dapat bubur    pun  dah     bagus lah, sekali-sekali, pun  nak  taruk keledek,     taruk ubi.\\
indeed not find  rice   \textsc{part}  later cook   \textsc{top} make  porridge or   make   if    get porridge \textsc{top2} already good  \textsc{part}  occasionally   \textsc{event} \textsc{mod} put   sweet potato put   tapioca\\
\glt `We couldn't of course find rice, when we cooked porridge for instance, if we got porridge it was already
very good, once in a while, still we had to add sweet potato and tapioca.'
\z

%%%%%%%% DEM pun %%%%%%%%%%%%%%%%%%%%%
\subsection{Demonstratives and particles}\label{GivenDemPart}
Demonstratives may be followed by the particle \emph{pun}. An eventive \emph{pun} can be seen in (\ref{DalamKemarahan}). The speaker confuses the plot, and refers to the wife where the husband is meant.

\ea\label{DalamKemarahan} 
\langinfo{Singapore Malay}{}{2017.NZ.04}\\
\gll Dalam kemarahan, uh,    {\ob}lelaki \ule{itu} \ule{pun}{\cb}  pergi, uh,   pergi ke   suami-nya,      eh,   ke   isteri-nya dan  marah, dan  marah suami-nya      kenapa dia berbual  dengan lelaki lain.\\
in    anger      \textsc{hesit} male   \textsc{dist} \textsc{top2} go     \textsc{hesit} go    to   husband-\textsc{3poss} \textsc{hesit} to   wife-\textsc{3poss} and  angry  and  angry husband-\textsc{3poss} why    \textsc{3sg} converse with   male   other\\
\glt `In anger, that man then went to his wife and scolded his husband [sic] for talking to other men.'
\z

\noindent
In (\ref{DiaMemberitahuTentang}), the farmer is described as \emph{suami tu pun}. The particle \emph{pun} prompts the hearer to attend to the temporal sequence, while the \isi{demonstrative} \emph{tu} places the focus on the same \isi{referent}. The distal may encode the wife's apprehensive \isi{stance} towards her husband.

\ea\label{DiaMemberitahuTentang} 
\langinfo{Singapore Malay}{}{2017.NZ.14}\\
\gll Dia mem-beritahu tentang, aah,      keadaan-nya       {di situ} dan  bagaimana dia insaf    dan  rasa kesan       terhadap kejadian-nya     tersebut.  Dari hari itu,  {\ob}suami  tu  pun{\cb}  tidak me-minum arak lagi  dan  tidak ber-campur dengan kawan-kawan     tersebut.\\
\textsc{3sg} \textsc{av-}tell        about    \textsc{hesit} situation-\textsc{3poss} there   and  how       \textsc{3sg} penitent and  feel consequence about    incident-\textsc{3poss} \textsc{given.foc} from day   \textsc{dist}   spouse \textsc{dist} \textsc{event} not   \textsc{av-}drink   alcohol again and  not   \textsc{av-}mix       with   \textsc{red}-friend \textsc{given.foc}\\
\glt `He told them about the conditions there and how he regretted and felt the  effects of that incident. From that day on, the husband did  not drink alcohol any more and did not mix with those friends.'
\z

\noindent
Multiple demonstratives can combine within a single description, as in (\ref{BudakLagiTadiYang}), where the noun \emph{budak laki} `boy' is followed by the recent mention \emph{tadi} and \emph{itu}. 

\ea\label{BudakLagiTadiYang} 
\langinfo{Singapore Malay}{}{2013.LN.PearStory.18}\\
\gll  {\ob}{Budak lelaki}  tadi     yang menunggang       basikal itu{\cb} te-nampak se-orang {\USOParen}1s{\USCParen} budak perempuan yang juga menaiki       basikal yang bertentangan, yang {\USOParen}1s{\USCParen}   berjalan bertentangan dengan-nya.\\
boy  \textsc{recent} \textsc{rel}  \textsc{av.}ride bicycle \textsc{dist} \textsc{invol-}spot  one-\textsc{cl.human} {}   child female    \textsc{rel}  also \textsc{av.}travel.by   bicycle \textsc{rel}  opposite     \textsc{rel} {} travel  opposite     with-3\\
\glt `The boy who was riding the bicycle saw a girl who was also riding a bicycle in the opposite direction.'
\z

\noindent
\tabref{tab:Stance:JointAttention:Dems} sketches the functions of \ili{Malay} demonstratives and particles in manipulating and directing the interlocutors' \isi{joint attention}. The effect is captured with simple verb phrases --- a conventionalised terminology remains lacking.\footnote{\cite{tomasello1995joint} offers a lucid account of the development of social cognition and the ability to manipulate \isi{joint attention} in children.} This representation also draws on the idea of cognitive states developed in \cite{Gundel1993} but takes the attention asymmetry between the interlocutors as a starting point. The effects fall into two groups, depending on whether the ``field" of \isi{joint attention} remains the same or shifts. 

Within the same field, the proximal \emph{ini} requires a symmetrical manipulation of \isi{joint attention}, while the remaining \isi{deictic} forms indicate a reorientation of attention on the side of the speaker and require a manipulation of the focus on the side of the hearer so that \isi{joint attention} can be renewed. The most forceful  reorientation within the same field is encoded with the epistemic particle \emph{pula(k)}, which indicates a surprise or novelty on the side of the speaker (captured here as ``update"). Finally, the particle \emph{pun} encodes a shift of \isi{joint attention} and entails a reset of anaphora, exemplified in (\ref{JadiSerigalaItuPunIngin}).

\begin{table}
\caption{Joint attention manipulating functions of Malay demonstratives and particles}
\label{tab:Stance:JointAttention:Dems}
 \begin{tabularx}{.7\textwidth}{Xll} 
  \lsptoprule
\textsc{demonstrative}  & \multicolumn{2}{l}{\textsc{\isi{joint attention} manipulation}} \\
 & \textsc{speaker} & \textsc{hearer}\\
\midrule
\emph{si} &		bring in focus	&	activate familiar \\
\emph{ini} &		keep in focus	&	keep in focus\\
\emph{itu}	 &	bring in focus	&	access\\
\emph{tadi}	&	bring in focus	&	recall recent\\
\emph{tersebut} & 	bring in focus	&	recall known\\
\emph{pula(k)} & update/broaden  & update/broaden  \\ 
\emph{lagi} & add in focus & add in focus  \\ 
\midrule
\emph{pun} & shift  & shift  \\ 
\lspbottomrule
 \end{tabularx}
\end{table}

%%%%%%%%%%%%%%%%%%%%%%%%%%%%%%%%%%%%%%%%%%%%%%%

\section{Conclusions}
A systematic comparison of elicited narrative texts organised in a parallel corpus enables us to make several points about \ili{Malay} \isi{discourse} and information structure:

\begin{itemize}
\item The speaker's \isi{stance} is reflected in \isi{referent} \isi{categorisation} and has consequences for \isi{referent} tracking.
\item The stronger \isi{epistemic stance} simplifies expression of referents and their tracking, confirming the claim by \cite{SacksSchegloff2007} that \isi{categorisation} of humans with proper names require less recognitional effort, as shown in \sectref{ReferentTracking}.
\item The neutral \isi{epistemic stance} generally motivates \isi{referent} \isi{categorisation} with descriptions, which need to be more elaborate to track referents effectively.
\item Taking a stronger \isi{epistemic stance}, the speaker can construct and maintain differential perspectives on the referents through their \isi{categorisation}, such as \emph{Adam} vs. \emph{her husband}, or \emph{her father} \citep{Stivers2007}.
\item The variation of expression correlates with the referential status of the \isi{referent} as well. The high referential status allows for tracking with pronouns, but the low status disfavours enumeration and classifiers.
\item The topic focus particle \emph{pun} is preferred with more complete expressions of a third person \isi{referent}, after which the reference can be minimised (zero, \emph{dia}, \emph{ia}, etc.), as argued by \citet{Heritage2007, SacksSchegloff2007}.
\item Both topical and focused participants are eligible for minimisation, but the remaining referents require a fuller expression, for non-humans typically a \textsc{n}+\emph{itu} phrase.
\end{itemize}

\noindent
Future work will focus on the role of \isi{word order} and verbal morphology as well as on the effect of downgrading the role of the hearer to a silent listener, unable to interact where \isi{joint attention} is not achieved \citep{DeLancey1997}. Our parallel corpus contains such information in the negotiations preceding the presentation of the agreed narrative. 

\section*{Acknowledgements}
We would like to thank the editors of this volume and three anonymous reviewers for their encouragement and comments on earlier versions of this paper. We would also like to thank the members of the \ili{Malay} Research Group at the Nanyang Technological University in Singapore for their valuable comments and suggestions: David Moeljadi, Kadek Ratih Dwi Oktarini, Nur Atiqah binte Othman, Hannah Choi Jun Yung, Nur Amirah Binte Khairul Anuar, and Hiroki Nomoto. All the authors gratefully acknowledge the generous support of the \ili{Singaporean} \ili{Malay} community. The authors also acknowledge the generous support of the ILCAA Joint Research Project `Cross-linguistic Perspective on the Information Structure of the \ili{Austronesian} Languages' (PI Dr Atsuko Utsumi), funded by The Research Institute for Languages and Cultures of Asia and Africa (ICLAA) of Tokyo University of Foreign Studies, as well as the research funding through a Tier 1 project `Development of Intonational Models for \ili{Malay} and Singapore \ili{English}' awarded by the Singapore Ministry of Education Tier 1 Grant MOE2013-T1-002-169 (PI Dr Tan Ying Ying). 

\newpage
\section*{Abbreviations}

\begin{multicols}{2}
	\begin{tabbing}
		glossgloss \= \kill
		1, 2, 3 \> person markers\\
		\textsc{aff} \> affected\\
		\textsc{appl} \> applicative\\
		\textsc{av} \> active \isi{voice}\\
		\textsc{comp} \> complementizer \emph{bahawa},\\ \> \emph{yang}\\
		\textsc{cs} \> code-switching\\
		\textsc{dem} \> \isi{demonstrative}\\
		\textsc{dist} \> distal\\
		\textsc{foc} \> focus\\
		\textsc{intens} \> intensifier\\
		\textsc{invol} \> involuntary agent \emph{ter-}\\
		\textsc{hesit} \> hesitation marker\\
		\textsc{mod} \> modal auxiliary\\
		\textsc{pass} \> passive auxiliary \emph{kena}\\
		\textsc{poss} \> possessive\\
		\textsc{prox} \> proximate\\
		\textsc{pv} \> passive \isi{voice}\\
		\textsc{top} \> topic\\
		\textsc{top2} \> switched topic \\ \> (Goddard's \emph{topic focus})
	\end{tabbing}
\end{multicols}

\sloppy
\printbibliography[heading=subbibliography,notkeyword=this]

\end{document}
