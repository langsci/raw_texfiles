\documentclass[output=paper
,modfonts
,nonflat]{langsci/langscibook}  
 
\ChapterDOI{10.5281/zenodo.1402543}
\title{Information structure in Sembiran Balinese} 
\author{I Wayan Arka\affiliation{Australian National University / Universitas Udayana}\lastand I Nyoman Sedeng\affiliation{Universitas Udayana}}                                         

\abstract{This paper discusses the information structure in Sembiran Balinese, an endangered, conservative mountain dialect of Balinese. It presents the first detailed description of the ways topic, focus and frame setter in this language interact with each other and with other elements in grammar. It is demonstrated that Sembiran Balinese employs combined strategies that exploit structural positions, morpho-lexical and syntactic resources in grammar. The description is based on a well-defined set of categories of information structure using three semantic-discourse/pragmatic features of [+/\textminus salient], [+/\textminus given] and [+/\textminus contrast]. This novel approach allows for the in-depth exploration of the information structure space in Sembiran Balinese. The paper also highlights the empirical-theoretical contributions of the findings in terms of the significance of local socio-cultural context, and the conception of information structural prominence in grammatical theory.}
%\Keywords{information structure, voice systems, topic, focus, frame setter, left periphery position, prominence}

\begin{document}

\maketitle

\section{\label{s:arka:1}Introduction}
\largerpage
Sembiran \ili{Balinese} is one of the endangered conservative dialects of \ili{Balinese} (i.e. \textit{\ili{Bali} Aga}, or Mountain \ili{Balinese}). It is spoken by around 4,500 speakers in the mountainous village of Sembiran in northern \ili{Bali}.\footnote{SBD should be distinguished from the Plains \ili{Balinese} dialect, which lexically has been influenced by many other languages, namely, \ili{Javanese}, Sanskrit, \ili{English}, \ili{Arabic} and \ili{Indonesian}. Morphologically, both dialects have slight differences in prefix and suffix systems, but syntactically, both dialects have the same syntactic marking typology.}  Sembiran \ili{Balinese} has a similar morphosyntax to Plains, or Dataran \ili{Balinese}, but it is different in that it lacks the speech level system characteristics of Plains \ili{Balinese}.\footnote{Sembiran \ili{Balinese} lacks the elaborate speech level system of Plains \ili{Balinese}; however, the data suggests that there has been considerable contact with Plains \ili{Balinese}, with the speakers being bilingual and fully aware of the politeness and speech level system. For example, the use of code-switching with the polite \isi{pronoun} \textit{tiyang} in addition to code-switching with \ili{Indonesian} words was found.} The noticeable difference is therefore related to the lexical stock, including the pronominal system, which is discussed in \sectref{s:arka:2}. 

Sembiran \ili{Balinese} is relatively underdocumented compared to Plains \ili{Balinese}. Previous studies on Sembiran \ili{Balinese} include studies by \citet{Astini1996} on consonant gemination and by \citet{Sedeng2007} on morphosyntax. A more comprehensive documentation of Sembiran \ili{Balinese} and other \textit{\ili{Bali} Aga} varieties is needed. 

The present paper on information structure in Sembiran \ili{Balinese} primarily builds on \posscitet{Sedeng2007} work. Our paper is the first thorough description of the information structure in Sembiran \ili{Balinese}, based on a well-defined set of categories of information structure using three features ([+/\textminus salient], [+/\textminus given] and [+/\textminus contrast]). The adopted novel approach makes it possible to map out the information structure in Sembiran \ili{Balinese} in considerable detail and depth, revealing the intricacies of the semantics, syntax and pragmatics involved. The data provides fresh empirical evidence not only for the distinction of the known major categories of \textsc{focus} vs. \textsc{topic}, but also for the subtle distinction of \textsc{frame setter} vs. (\textsc{\isi{contrastive}) topic}. Typologically, certain aspects of the information structure patterns observed in Sembiran \ili{Balinese} are consistent with the patterns found in Plains \ili{Balinese} \citep{Pastika2006} and other \ili{Austronesian} languages with \isi{voice} morphology, such as \ili{Pendau}, a language in central Sulawesi \citep{Quick2005, Quick2007}.  

The paper is structured as follows. Grammatical relations and related salient features of Sembiran \ili{Balinese} are discussed in \sectref{s:arka:2}. This is followed by an overview of information structure and the proposal of decomposing topic and focus into three features ([salient], [given] and [contrast]) in \sectref{s:arka:3}. The main discussions with the presentation of the data and analysis are presented in \sectref{s:arka:4} for topic, \sectref{s:arka:5} for focus and \sectref{s:arka:6} for frame setting and left-periphery positions. The conclusion and further remarks are provided in the final section. 

\section{\label{s:arka:2}Grammatical relations in Sembiran Balinese in brief}

Sembiran \ili{Balinese} is a conservative dialect of \ili{Balinese}. The conservative nature is first evident by the retention of an archaic \ili{Austronesian} feature already lost in Plains \ili{Balinese}, namely the pronouns (\textit{a)ku}/-\textit{ku} and \textit{engko}/-\textit{mu}, as shown in \tabref{tab:arka:1}. 

\begin{table}
\begin{tabular}{lll}
	\lsptoprule
	\textsc{Person} & \textsc{Free Pronoun} & \textsc{Bound Genitive Pronoun}\\
	\midrule
	\textsc{1} & \textit{aku, oké, kaka,} \textbf{\textit{icang}} & -\textit{ku}\\
	\textsc{2} & \textit{engko, cahi,} \textbf{\textit{nyahi}} & -\textit{mu}\\
	\textsc{3} &  \textbf{\textit{iya}} & -\textbf{\textit{a}}\\
	\lspbottomrule
\end{tabular}
\caption{Pronominal systems in Sembiran Balinese}
\label{tab:arka:1}
\end{table}


\noindent
The pronouns in bold in \tabref{tab:arka:1} are those that are also shared with Plains \ili{Balinese}. The bound pronouns in their genitive function in the nominal typically appear with the nasal ligature -\textit{n} and the definite suffix -\textit{e}, leading to the morphologically complex bound forms of -\textit{kune}, -\textit{mune} and -\textit{ane} for the first, second and third persons, respectively. An example is given in (\ref{e:arka:1}):

\begin{exe}
	\ex\label{e:arka:1}
	\gll Engko sa mutang sa engko kén panak-mu-n-e.\\
	2 \textsc{part} \textsc{mid}.debt \textsc{part} 2 to child-2-\textsc{lig}-\textsc{def}\\
	\glt ‘You still owe your son/daughter (a ritual).’ \citep{Sedeng2007}
\end{exe}

\noindent
There is also an intriguing difference in which certain intransitive verbs in Sembiran \ili{Balinese} use the Actor Voice (\isi{AV}) prefix \textit{N}-  with a prenasalised segment retention rather than the middle (MID) \isi{voice} \textit{ma}- used in Plains \ili{Balinese},  as shown in (\ref{e:arka:2}). This further indicates the conservative nature of Sembiran \ili{Balinese} given the fact that a prenasalised segment is an ancient and widespread feature of \ili{Austronesian} languages \citep[224]{Blust2013}; however, it should be noted that this prenasal segment retention only applies for intransitive verbs. In transitive verbs, the nasal property of the \isi{AV} prefix \textit{N}- assimilates with the initial consonant, resulting in no prenasal segment, e.g. \textit{teguh} ‘bite’ $\rightarrow$ \textit{neguh} (<N-teguh) ‘\textsc{av}-bite’. 

\begin{exe}
\ex\label{e:arka:2}
\begin{tabular}[t]{llll}
	& Sembiran & Plains & \\
	Root & \ili{Balinese} & \ili{Balinese}& Gloss\\
	a. \textit{besen} &  \textit{mbesen} & \textit{mabesen} & ‘send message’ \\
	b. \textit{pupur} &  \textit{mpupur} & \textit{mapupur} & ‘make up with powder’\\
	b. \textit{salin} & \textit{nsalin} & \textit{masalin} & ‘change dress’ \\
	c. \textit{kisid} & \textit{ngkisid} & \textit{makasid} & ‘move’ 
\end{tabular}
\end{exe}

\noindent
The morphosyntax of Sembiran \ili{Balinese} is exactly like Plains \ili{Balinese}. Following the conventions of language typology (\citealt{Comrie1978,Dixon1979,Croft2003,Haspelmath2007,Comrie2005,Bickel2011}), grammatical relations are represented using the abbreviated labels, as described in (\ref{e:arka:3}). These labels, particularly A vs. P and G vs. T, are distinguishable by certain semantic entailment properties (\citealt{Dowty1991,Bickel2011,WitzlackMakarevich2011}; among others).  The same labels are used in this paper when the arguments alternate, e.g. the same label A is used for the most actor-like argument in the active structure (i.e. core A argument) and in its passive counterpart, which is grammatically an oblique A. When necessary, a specific semantic role is specified for clarity, e.g. P: goal, meaning a goal of a three-place predicate that is treated as P as it enters a transitive structure. 

 
\begin{exe}
	\ex{Grammatical functions: default generalised semantic relations}\\\label{e:arka:3}
	\begin{tabular}[t]{cll}
		S & = & sole core argument of an intransitive predicate\\
		A & = & most actor-like argument of a bivalent transitive predicate\\
		P & = & most patient-like argument of a bivalent transitive predicate\\
		G & = & most goal/recipient-like argument of a trivalent predicate\footnotemark \\
		T & = & theme of a trivalent predicate 
	\end{tabular}
\end{exe}
\footnotetext{Based on applicativisation in \ili{Balinese}, G is also the generalised role for a source/locative-like argument.}

\noindent
Sembiran \ili{Balinese} also exhibits a grammatical \textsc{subject} or \textsc{pivot}, which plays a role in complex \isi{clause} formations, such as control and relativisation. The \isi{voice system} regulates the selection of a particular role as a \isi{pivot}, which may also bear a particular \isi{discourse} function of topic or focus. In the following examples, which show syntactic control, the verb \textit{mati-ang} ‘dead-\textsc{cause}=kill’ is in the \isi{UV} form in (\ref{e:arka:4a}). P is selected as the \isi{pivot} and is therefore controlled by (i.e. understood as the same as) the matrix subject \textit{engko}. In contrast, in (\ref{e:arka:4b}), because the verb is in the \isi{AV} form, the A argument is selected as the \isi{pivot} and understood as the same as the matrix subject. 

\begin{exe}
	\ex\label{e:arka:4}
	\begin{xlist}
		\ex\label{e:arka:4a}
		\gll Saking  engko  dot  {\ob}\_\_   mati-ang    oké{\cb}{\USQMark}\\
		really  2\textsc{sg}  want {} \textsc{uv}.dead-\textsc{caus}  1\textsc{sg}\\
		\glt ‘You really want me to kill you?’
		\ex\label{e:arka:4b}
		\gll Glema-néné  nagih  {\ob}\_\_  ngmati-ang   i  rangsasa{\cb}.\\
		person-\textsc{def}  want {} \textsc{av}.dead-\textsc{caus}  \textsc{art}  giant  \\
		\glt ‘This person wants to kill the giant.’ \citep[135]{Sedeng2007}
	\end{xlist}
\end{exe}

\noindent
In terms of \isi{word order}, Sembiran \ili{Balinese} is an A/S-V-P (or SVO) language, with an alternative order reflecting different information structure. Sembiran \ili{Balinese} is like Plains \ili{Balinese} in its phrase structure, which are schematised informally in (\ref{e:arka:5}).

\begin{exe}
	\ex\label{e:arka:5}~\vspace*{-\baselineskip}\\
	\begin{minipage}[t]{\linewidth}
	\Tree [.{CP (Extended Clause)} [.XP\\(Frame/ContrTopic) ] [.{CP (Extended Clause)} [.XP\\(Focus) ]  [.C$'$ [.C ]  [.{IP (Finite Core Clause)} [.Pivot ] [.I' [.I ] [.{XP:Predicate} ]]]]]]
	\end{minipage}
\end{exe}

\noindent
The predicate is not necessarily verbal; hence, XP (with X being any lexical category). The grammatical subject/\isi{pivot} is part of the finite core \isi{clause} structure, precisely in Spec, with the IP not shown in (\ref{e:arka:5}).

A sentence can have units placed sentence-initially. This type of sentence is analysed as having an extended \isi{clause} structure. Formally, in terms of X-bar syntax, this extended structure has a unit that is left-adjoined to the \isi{clause} structure. The adjoined element bears the pragmatically salient \isi{discourse} functions (DFs), frame/\isi{contrastive topic} and \isi{contrastive focus}, typically in that order. The left-most sentence’s initial DF position is often called a left dislocated or detached position. The \isi{focus position} is closer to the core \isi{clause} (IP) structure. It is called the Pre-Core Slot in Role and Reference Grammar (RRG) (\citealt{VanValin2005,VanValin1997}). It is a position in [Spec, CP] in terms of the X-bar syntax adopted here. 

Evidence for the structure shown in (\ref{e:arka:5}) is based on the following facts. First, there is evidence associated with interrogatives with question words (QW) in free clauses. The QW focus can appear in situ or can be fronted. When fronted, it must come in [Spec, CP] linearly before the core \isi{clause} structure (IP). This is exemplified in (\ref{e:arka:6a}). In this sentence, the subject \textit{cening} appears in its position within IP; however, the subject can be fronted marked with a topicaliser \textit{en} (\textit{buat}) ‘as for’ as in (\ref{e:arka:6b}), where it appears before the QW focus expression \textit{buwin pidan}. Note that the fronted constituent \textit{(en) cening} is precisely a \isi{contrastive topic}. Crucially, this \isi{contrastive topic} with the explicit marking with \textit{en} cannot come after the QW focus, as can be observed from the unacceptability of (\ref{e:arka:6c}). 

\begin{exe}
	\ex\label{e:arka:6}
	\begin{xlist}
		\ex\label{e:arka:6a}
		\gll {\ob}{\ob}Buwin pidan{\cb}\textsubscript{Foc} {\ob}cening lakar mlali{\cb}\textsubscript{IP}{\cb}\textsubscript{CP}{\USQMark}\\
		\phantom{[[}again  when  {\ob}kid  \textsc{fut}  \textsc{mid}.go.sightseeing\\
		\glt ‘When again will you go (there)?’
		\ex\label{e:arka:6b}
		\gll {\ob}{\ob}{\USOParen}En{\USCParen} cening{\cb}\textsubscript{Top}, {\ob}buwin   pidan{\cb}\textsubscript{Foc}  {\ob}\_ lakar  mlali{\cb}\textsubscript{IP}{\cb}\textsubscript{CP}{\USQMark}\\
		\phantom{[[}as.for  kid  \phantom{[}again  when {} \textsc{fut} \textsc{mid}.sightseeing\\
		\glt ‘As for you, kid, when again will you go (there)?’
		\ex\label{e:arka:6c}
		*[[buwin pidan]\textsubscript{Foc}   [en cening]\textsubscript{Top},  [\_\_ lakar mlali]\textsubscript{IP}]\textsubscript{CP}?
	\end{xlist}
\end{exe}

\noindent
Additional evidence is based on finite complement clauses. Complement clauses are structurally CP with QWs like \textit{pidan} ‘when’, \textit{apa} ‘what/if’, \textit{ken} ‘which’ and \textit{nyen} ‘who’ that can function like complementisers, appearing as part of the CP taking the finite \isi{clause}, as in example (\ref{e:arka:7a}). An important point to note from (\ref{e:arka:7a}) is that the adverbial phrase \textit{buwin mani} ‘tomorrow’ is part of the complement \isi{clause} CP, as in the partial phrase structure tree shown in (\ref{e:arka:7b}). While appearing before \textit{apa}, it is an adjunct of the embedded \isi{clause}, not of the matrix \isi{clause}. The matrix \isi{clause} has its own temporal adjunct, namely \textit{ibi} ‘yesterday’. Also note that the adverbial \textit{buwin mani} is fronted, resulting in a focus interpretation that is indicated by capital letters in the free translation.  

\begin{exe}
	\ex\label{e:arka:7}
	\begin{xlist}
		\ex\label{e:arka:7a}
		\gll Meme   ibi   ntakon    {\ob}buwin   mani   apa   {\ob}ia   teka   mai{\cb}\textsubscript{IP}{\cb}\textsubscript{CP}.\\
		mother  yesterday  \textsc{av}.ask  \phantom{[}again  tomorrow  if  \phantom{[}3\textsc{sg}  come  here\\
		\glt ‘Mother yesterday asked whether TOMORROW he would come here.’
		\ex\label{e:arka:7b}
		\begin{minipage}[t]{\linewidth}
		\Tree [.VP [.V\\{\textit{ntakon}} ] [.CP \qroof{\textit{buwin mani}}.AdvP  [.C$'$ [.C\\\textit{apa} ] \qroof{\textit{ia teka mai}}.IP ]]]
		\end{minipage}
	\end{xlist}

\end{exe}

\noindent
The evidence that there is a \isi{focus position} associated with CP positions before IP is based on the exclusive \isi{focus marker} \textit{ane}, which is also a relativiser. The syntactic constraint is that it must also be associated with the \isi{pivot}. Structurally, this means that the presence of (\textit{a)ne} requires that the position [Spec, IP] and the positions before it (i.e. [Spec, CP] and/or C) must be associated with the \isi{pivot}. Hence, sentence (\ref{e:arka:8b}) is correct when \textit{ane} is used to augment the focus expression \textit{icang ba} in (\ref{e:arka:8b}). In both, the A argument \textit{icang} is the focus and the \isi{pivot}, as evidenced by the form of the verb, which is in the \isi{AV} form. 

\begin{exe}
	\ex\label{e:arka:8}
	\begin{xlist}
		\ex\label{e:arka:8a}
		\gll {\ob}Iya{\cb}\textsubscript{ContrTop}, {\ob}{\ob}icang  ba{\cb}\textsubscript{ContrFoc\_i}  {\ob}\_\_i  mehang   nasi{\cb}\textsubscript{IP}{\cb}\textsubscript{CP}.\\
		\phantom{[}3\textsc{sg} \phantom{[[]}1\textsc{sg} \textsc{part} \phantom{[}\textsc{piv}  \textsc{av}.give rice\\
		\glt ‘As for him, I am the one who provided meals.’
		\ex\label{e:arka:8b}
		\gll {\ob}Iya{\cb}\textsubscript{ContrTop}, {\ob}icang   ba{\cb}\textsubscript{ContrFoc\_i}  {\ob}\textbf{ane}\textsubscript{\_i}  {\ob}\_i mehang nasi{\cb}\textsubscript{IP}{\cb}\textsubscript{CP}.\\
		\phantom{[}3\textsc{sg}      \phantom{[}1\textsc{sg}  \textsc{part}     \phantom{[}\textsc{rel}   \phantom{[}\textsc{piv}  \textsc{av}.give   rice\\
		\glt ‘As for him, I am the one who provided meals.’
		\ex\label{e:arka:8c}
		\gll {{\USStar}{\ob}Iya{\cb}\textsubscript{ContrTop}}  \textbf{ane}\textsubscript{\_i} {\ob}icang   ba{\cb}\textsubscript{ContrFoc\_i}  {\ob}\_i mehang  nasi{\cb}\textsubscript{IP}{\cb}\textsubscript{CP}.\\
		\phantom{*[}3\textsc{sg}      \textsc{rel}  1\textsc{sg}  \textsc{part}        \textsc{piv}  \textsc{av}.give  rice\\
		\glt ‘As for him, I am the one who provided meals.’
	\end{xlist}
\end{exe}

\noindent
An attempt to mark the left-most NP (\textit{iya}) with \textit{ane}, as in (\ref{e:arka:8c}), is ungrammatical. This is because such marking results in a structure with an intervening argument \textit{icang}, which is referentially distinct from the \textit{ane}-marked NP (\textit{iya}). Given the requirement of \textit{ane}, it causes two referentially different NPs to compete for the \isi{pivot} argument. 

Finally, it should be noted that there may be units in the \isi{left periphery} positions that are not necessarily arguments of the main verb. They are represented by XP in (\ref{e:arka:5}). For example, the verb \textit{ngamah} ‘\isi{AV}.having meals’ in (\ref{e:arka:9}) appear in the \isi{left periphery} position. Syntactically, it is not a dependent unit of the predicate \textit{mehang} ‘\isi{AV}.give’. It functions as a frame or presentational topic (cf. \citealt[177--181]{Lambrecht1994}); that is, it introduces an event of ‘eating’, evoking and delimiting certain referents in the \isi{discourse}, including \textit{nasi} ‘rice’, which is the \isi{new focus}.

\begin{exe}
	\ex\label{e:arka:9}
	\gll {\ob}En  buat  ngamah{\cb}\textsubscript{Frame}, {\ob}icang   ba{\cb}\textsubscript{ContrFoc}   mehang  iya  nasi.\\
	\phantom{[}if   about   \textsc{av}.eat \phantom{[}1\textsc{sg}  \textsc{part}  \textsc{av}.give  3\textsc{sg}  rice \\
	\glt ‘As for eating needs, I am the person who gave him meals.’
\end{exe}

\noindent
Additional details for marked \isi{discourse} functions in \isi{left periphery} positions will be provided in \sectref{s:arka:6}. There can also be a right-dislocated position in Sembiran \ili{Balinese}. This is the case for a re-introduced or after-thought topic, which is discussed in \sectref{s5.3}. 

\section{\label{s:arka:3}Information structure: an overview}

An information structure (i-str) is a structure by which meanings are packaged to accommodate speaker-hearer needs for effective communication in a given \isi{discourse} context (cf. \citealt[5]{Lambrecht1994}; \citealt[460]{Vallduví1996}; among others). Examples of the units of i-str that are widely discussed in the literature are different types of topic and focus. Topic and focus are expressed by different formal mechanisms in the grammar of a given language, depending on the available morphosyntactic, prosodic and/or lexical resources. In Sembiran \ili{Balinese}, \isi{voice system}, structural positions, nominal expressions and \isi{prosody} are important resources for i-str.\footnote{Prosody is not discussed in this paper. While we are aware of the role of \isi{prosody} in information structure in (Sembiran) \ili{Balinese}, we have not conducted specific research on this topic. This is one of the areas that needs further research not only in Sembiran \ili{Balinese} but also in Plains \ili{Balinese}.}  

The precise mechanism that underpins the various ways in which information is packaged within and across languages has been subject to intense study (\citealt{Vallduví1996,Erteschick-Shir2007,Dalrymple2011}; among others). For the analysis of i-str in Sembiran \ili{Balinese} in this paper, a parallel model with a LFG-like framework was used, which separates different layers of structures to distinguish predicate argument structure from linear order constituent structure (c-str), surface grammatical relations and information structure. The grammatical relations are represented using labels commonly used by typologists, such as A and P, as in (\ref{e:arka:3}). It was also recognised that the \isi{pivot} is part of the surface grammatical relations. 

In the adopted framework, there is no one-to-one relation between these layers of structures. Thus, different sentences in (\ref{e:arka:10}) are driven by different forces in information structure. They all have the same (underlying) predicate argument structure, but their argument roles are mapped onto different surface grammatical relations and different \isi{discourse} functions. For example, the Actor \textit{John} in (\ref{e:arka:10}) is the grammatical subject-topic in (\ref{e:arka:10a}), but it is a \isi{contrastive focus} (while still being a \isi{pivot}) in (\ref{e:arka:10b}) and a \isi{completive focus} and grammatically oblique in (\ref{e:arka:10c}). Explicit information about i-str is given as necessary, and this is represented by means of annotations, e.g. as in (\ref{e:arka:11}) for sentence (\ref{e:arka:10b}). 

\begin{exe}
	\ex\label{e:arka:10}
	\begin{xlist}
		\ex\label{e:arka:10a}
		\textit{John killed the robber.} 
		\ex\label{e:arka:10b}
		\textit{It’s John who killed the robber.}  
		\ex\label{e:arka:10c}
		\textit{The robber was killed by John.}
	\end{xlist}
\end{exe}

\begin{exe}
	\ex\label{e:arka:11}
	\textit{It is} [\textit{John}]\textsubscript{A:Contr.Foc} [\textit{who killed}  \textit{the robber}]\textsubscript{VP:Comment{\textbar}G}\textsubscript{iven}.
\end{exe}

\noindent
The information structure itself could be considered to consist of different layers with different possible associations of clausal constituents. For example, the i-str system at the broadest level may operate with two layers showing topic–comment and given (presup\-posed)–\isi{new focus} distinctions as seen in \ili{Russian}, where the topic precedes the comment and the given precedes the focus (\citealt[405]{Foley2007info,Comrie1987}). Thus, the focus expression always comes later, and not all of the units of the comment belong to the focus. Consider the mini dialogue in (\ref{e:arka:12}) from \ili{Russian} (\ref{e:arka:13}). 

\begin{exe}
	\ex\label{e:arka:12}
	\begin{xlist}
		\exi{S:}\label{e12s}
		\gll \ili{Maksím}   ubivájet   Aleksé´j-a.     {\USOParen}\ili{Russian}{\USCParen}\\
		Maxim.\textsc{nom}   kills   Alex-\textsc{acc}\\
		\glt ‘Maxim kills Alexei.’
		\exi{Q:}\label{e12q}
		\gll A   Víktor-a{\USQMark}\\
		and   Victor-\textsc{acc}\\
		\glt ‘And Victor?’
		\exi{A:}\label{e:arka:12a}
		\gll Víktor-a  Máksim   zaščǐščajet.\\
		Victor-\textsc{acc}   Maxim.\textsc{nom}  defends\\
		\glt ‘Maxim defends Victor.’      \citep[96]{Comrie1987}
	\end{xlist}
\end{exe}

\begin{exe}
	\ex\label{e:arka:13}
		\begin{tabular}[t]{lll}
			Víktor-a & Máksim & zaščǐščajet\\
			\multicolumn{1}{c}{[- topic -]} & \multicolumn{2}{r}{[- - - - comment - - - -]}\\
			\multicolumn{2}{c}{[- - - - - given - - - - -]} & \multicolumn{1}{r}{[- focus -]}\\
		\end{tabular}
\end{exe}

\noindent
The answer in (\hyperref[e:arka:12a]{12.A}) has the same information structure represented in (\ref{e:arka:13}), where Viktor is the topic and part of the comment (i.e. Maxim) is given. Additional complications in \ili{Balinese} Sembiran will be discussed in \sectref{s:arka:5}, where part of the comment is fronted and gains a \isi{contrastive focus}. 

Building on earlier studies on information structure (\citealt{Vallduví1996}; 
Erte\-schick-Shir 2007 \citeyear{Erteschick-Shir2007};  
\citealt{Dalrymple2011,Krifka2012}; among others), todefined as a prototypical unmarkedpic and focus were conceptualised as two broad categories forming the information structure space where pragmatic and semantic notions of contrast, salience and \isi{givenness} are essential. Following \citet[133]{Choi1999}, topic and focus were analysed as non-primitive notions. It is proposed that they are decomposable into features that capture the three independent but intertwined cognitive-\isi{discourse} properties just mentioned: contrast, salience and \isi{givenness}.

The features [salience] and [\isi{givenness}] are typically topic-related. They encompass important semantic-pragmatic properties in communicative events, such as the particular frame/entity within/about which new information should be understood (i.e. the “aboutness” of the topic), and the degree of \sloppy{importance/salience/prominence} of one piece of information relative to other bits of information in a given context. The latter is related to the “emphatic” element of communication. It reflects the speaker’s subjective choice of highlighting one element and making it stand out for communicative purposes.

While often closely linked, salience and \isi{givenness} are distinct. The two do not always go together. New information (i.e. [\textminus given]), for example, can be [+salient]. This is a situation in which indefinite/generic referents are assigned \isi{emphatic focus}, which is further discussed in \sectref{arka:s4.2}.

The feature [contrast] captures the explicit choice of one alternative with the strong exclusion of the others in a given contrast set. It can be associated with both topic and focus, which is further discussed in \sectref{s5.5} and \sectref{arka:s4.2}. 

  
The three features with their values result in eight possible combinations in the i-str space, as shown in (\ref{e:arka:14}). The features can be used to characterise fine-grained (sub)categories of topic and focus and to explore how they interact. 

  
\begin{exe}
	\ex{Grammatical functions: default generalised semantic relations}\label{e:arka:14}	    
	    \ea{} [+salient,   +given,   +contrast]  =  \isi{contrastive} frame/TOP\\
	    \ex{} [+salient,   \textminus given,    +contrast]  =  \isi{contrastive} (often fronted) FOC\\
	    \ex{} [+salient,   \textminus given,    \textminus contrast]  =  new (i.e. first mentioned  indefinite) TOP\\
	    \ex{} [+salient,   +given,   \textminus contrast]  =  default/continuing/reintroduced TOP\\
	    \ex{} [\textminus salient,   +given,   \textminus contrast]  =  secondary TOP\\
	    \ex{} [\textminus salient,   \textminus given,   \textminus contrast]  =  new (completive/gap) FOC\\
	    \ex{} [\textminus salient,   \textminus given,   +contrast]  =  \isi{contrastive} new FOC\\
	    \ex{} [\textminus salient,   +given,   +contrast]  =  \isi{contrastive} secondary TOP
	\z
\end{exe}

\noindent
To clarify the complexity of information structure involved in Sembiran \ili{Balinese}, the conception of common ground (CG) was adopted (\citealt{Krifka2012}, and the references therein). Two related aspects of CG should be distinguished: the CG contents and CG management. The CG contents refer to the set of information mutually shared by speech participants in a given context. This information can be a set of presupposed propositions in the immediate/current CG and a set of entities introduced earlier in the \isi{discourse} or general shared information.

CG is dynamic. It is continuously modified and adapted for communicative purposes throughout a speech event, e.g. by the addition of new information to the CG contents. The speaker generally has control over how to proceed in a speech event depending on his/her communicative interests/goals, possibly also considering the addressee’s interests/goals. The way the communicative moves are handled to update and develop the CG is part of CG management. 

CG management reflects the speaker’s perspective and attention characterised by the properties shown in (\ref{e:arka:14}), e.g. what is assumed/presupposed, singled out and contrasted, emphasised or new in a given communicative episode. Thus, in example (\ref{e:arka:10}), where the A and P referents ‘John’ and ‘the robber’ are both [+given] (i.e. already shared in the CG), the speaker has more than one option to restructure the information depending on his/her interest or attention for effective communication. The use of the passive structure (\ref{e:arka:10c}), for instance, reflects the choice that P \textit{robber} is of interest and is considered salient about which the new information, the event of killing, should be understood.

In the following sections, the interactions among the properties shown in (\ref{e:arka:14}) in Sembiran \ili{Balinese} are illustrated in more detail.

\section{\label{s:arka:4}Focus}

Focus has been characterised in the literature in terms of two properties: informational newness and the presence of alternatives. In terms of the features outlined in (\ref{e:arka:14}), the first property is captured by the feature [\textminus given]. That is, the focus is the informative, new and non-presupposed part of the proposition (\citealt{Lambrecht1994,Vallduví1996,Dalrymple2014}; among others). It is the information added to the CG by (part of) comment expressions in statements, question-answers in dialogues or actions required in commands. 

Focus can also be characterised in terms of the presence of alternatives (\citealt{Krifka2008,Krifka2012}). This is particularly clear in the case of the \isi{contrastive focus} and the \isi{contrastive topic}, which embeds the focus, as discussed in the previous section. The concept of “alternative” is part of the conception of the set, and according to Krifka, the presence of alternatives is in fact central to the definition of focus. There is a presence of strong alternatives with an overt focus marking in Sembiran \ili{Balinese} that carry the [+contrast] feature. This is to distinguish it from an unmarked \isi{new focus} discussed in this section, which is [\textminus contrast], i.e. carrying no overt contrast in the expressions and no (clear) \isi{contrastive} sets in the current CG other than alternatives due to a general knowledge of things in the world. In \sectref{arka:s4.1}, this general local knowledge in Sembiran \ili{Balinese} is exemplified to illustrate the point that focus indeed shows the presence of potential alternatives in a given shared local socio-cultural setting. The understanding of the choice of one alternative instead of another in a set is implicit and therefore requires a good understanding of broader information in the CG. It is argued that a focus related to this type of alternative has a weak or implicit ‘contrast-like’ meaning and is therefore categorised as having a [\textminus contrast]. New focus in Sembiran \ili{Balinese} is discussed in \sectref{arka:s4.1}, followed by the \isi{contrastive} and emphatic focusses in \sectref{arka:s4.2}. 

\subsection{\label{arka:s4.1}New focus}

New/\isi{completive focus} is defined as a prototypical unmarked focus ([\textminus salient, \textminus given, \textminus contrast]): it has negative values for the relevant features and contains no strong or embedded \isi{contrastive} element. It is [\textminus given], meaning that the information is not part of the CG (e.g. being asked) or is newly added to the current CG either by the speaker or the addressee as the speech event progresses. 

New focus is unmarked in the sense that its expression is assigned no specific tagging to signal a contrast or any other salience, such as structural fronting and/or the use of emphatic markers. New focus is therefore [\textminus contrast, \textminus salient]. An expression with a \isi{new focus} is a constituent unit that appears in its canonical position. While implying the presence of alternatives (due to general world knowledge), \isi{new focus} was analysed as [\textminus contrast], as there is no entity present in the current CG with which it is being contrasted. 

The clearest instance of a \isi{new focus} that shows non-presupposed information with the presence of alternatives is related to question-answer pairs. This is exemplified by polarity interrogatives, as in example (\ref{e:arka:15}). Polarity interrogatives, as the name suggests, are associated with either ‘yes’ or ‘no’ answers. This type of question clearly illustrates the presence of alternatives in a \isi{new focus}. In (\ref{e:arka:15}), \textit{ngara} ‘no’, instead of the other alternative \textit{ae} ‘yes’, is the information being asked; hence, it is new.

\begin{exe}
	\ex{Context: Men Dora told a story about how she gave her money to somebody but did not get her land certificate, and consequently, she lost her land.}\label{e:arka:15}\\
	\begin{xlist}
		\exi{Question:}\label{e15q}
		\gll Bakat   tanah-e{\USQMark}\\
		\textsc{uv}.obtain   land-\textsc{def}\\
		\glt ‘Did you get the land?’
		\exi{Men Dora:}\label{e:arka:15a}
		\gll {\ob}Ngara{\cb}\textsubscript{Foc},   {\ob}ngara{\cb}\textsubscript{Foc}   bakat.\\
		\phantom{[}\textsc{neg}  \phantom{[}\textsc{neg}   \textsc{uv}.obtain\\
		\glt ‘No, (I) didn't get (it).’
	\end{xlist}
\end{exe}

\noindent
A \isi{new focus} in relation to content questions such as \textit{nyén} ‘who’ and \textit{apa} ‘what’ also implies the presence of alternatives. Thus, the \isi{interrogative} in (\ref{e:arka:16a}) can be analysed as having an information structure with a \isi{pragmatic presupposition}, as shown in (\ref{e:arka:16b}) in the CG. The shared \isi{presupposition} is that every traditional garden definitely contains certain plants such as oranges, coconuts, and mangoes. The alternatives within this presupposed set of plants are part of the general local semantic field or knowledge in CG. The entity questioned and the answer given (i.e. X = focus) are among the alternatives in the set classified as PLANTS commonly cultivated in the garden, which is \textit{poh} ‘mango’ in this case. 

\begin{exe}
	\ex\label{e:arka:16}
	\begin{xlist}
		\ex\label{e:arka:16a}
		\gll {Miasa:} Tanah-anne  Patra  m-isi    apa   dowang{\USQMark}\\
		\phantom{Miasa:} land-\textsc{3sg.poss}  Patra  \textsc{av}-contain   what   \textsc{part}\\
		\glt‘What is Patra’s land planted with?’\\
		\gll Dora: {\ob}Poh   cenik-cenik{\cb}\textsubscript{Foc}.\\
		\phantom{Dora:} \phantom{[}mango   small-\textsc{red}\\
		\gll \phantom{Dora:} Mara   jani  m-isi poh.\\
		\phantom{Dora:} new   now  \textsc{av}-content mango\\
		\glt‘Young mango trees. It has just been planted with mangos.’
		\ex\label{e:arka:16b}{Presupposition:} \textit{Patra’s land is planted with X}, where \textit{X} ${\in}$ PLANTS\\
		COMMONLY CULTIVATED IN THE GARDEN\\
		Question:   \textit{What is X?}\\
		Answer:   \textit{X} = ‘mango’ (one of the plants commonly cultivated in the garden)  
	\end{xlist}
\end{exe}

\noindent
A \isi{new focus} in monologue types of genres, such as narratives and descriptions, may also imply alternatives. The speaker in this type of genre, being the sole participant responsible for additional new information to update the CG, often provides new information piece by piece for easy and comprehensible communication with his/her addressee. Crucially, \isi{new focus} expressions often come with modifiers of some type, flagging one piece of information in the current CG signalling one alternative, possibly in anticipation of more (alternative) information later in the \isi{discourse}. For example, the speaker (\textit{Men Dora}) tells her story about herself and discusses her children. In the first sentence in her autobiography (\ref{e:arka:17a}), the modifier \textit{mara besik}, ‘still one’ signifies that \textit{panak} ‘child’ (focus) is just one of the set of children that she will discuss. Later in her story, she discusses other children, including sentence (\ref{e:arka:17b}) about her second child. Here, the adverb \textit{buwin} is used to signify similar new information (‘giving birth’, ‘baby[-boy]’). Note that focus expressions include one of the alternatives that is true, e.g. \textit{buwin ninnya} ‘male again’ is used in (\ref{e:arka:17b}) instead of the other alternative ‘baby-girl’ in accordance with the truth condition of the (updated) CG contents. 

\begin{exe}
	\ex\label{e:arka:17}
	\begin{xlist}
		\ex\label{e:arka:17a}
		\gll {\ob}\textit{Ngelah} {\ob}\textit{panak}{\cb}\textsubscript{P}{\cb}\textsubscript{Emph.Foc} {\ob}\textit{meme}{\cb}\textsubscript{A.Top} {\ob}\textit{mara} \textit{besik}{\cb}\textsubscript{P.NewFoc} \textit{madan} \textit{Butuh} \textit{Dora}.\\
		\phantom{[}\textsc{av}.have  \phantom{[}child   \phantom{[}mother  \phantom{[[}still  one \textsc{mid}.name Butuh Dora\\
		\glt ‘I (mother) gave birth to the first (lit. still one) child, named Butuh Dora.’ 
		\ex\label{e:arka:17b}
		\gll Ba   kento   {\ob}${\varnothing}${\cb}\textsubscript{A.Top}  buwin   sa    ngelah   {\ob}panak{\cb}\textsubscript{Foc}, {\ob}${\varnothing}${\cb}\textsubscript{S.Top}   buwin   {\ob}ninnya{\cb}\textsubscript{Foc}.\\
		after  like.that  {}  again  \textsc{part}  \textsc{av}.have  \phantom{[}child {} again  \phantom{[}male\\
		\glt ‘After that, I again gave birth to a child, (and) (he’s) again a baby boy.’
	\end{xlist}
\end{exe}

\noindent
In terms of its structural expression, a \isi{new focus} is typically part of the comment constituent and distinct from the topic; however, the comment constituent may be split, as in (\ref{e:arka:17a}) where the comment VP is fronted, leaving the numeral phrase modifier in the object position. The whole predicate in sentence (\ref{e:arka:17a}) (i.e. ‘giving birth to one child’) is actually new in the \isi{discourse}. The split with fronting the VP can be considered the speaker’s way of assigning some type of emphatic \isi{prominence} to her phases of motherhood with a series of childbirths. At this point of the story, it is about the first baby. Based on the characterisation of the \isi{new focus} adopted in this paper, \textit{mara besik} ‘first child’ is a proper \isi{new focus} in (\ref{e:arka:17a}). The fronted VP, which gains emphatic meaning by fronting, can be precisely labelled as the emphatic \isi{new focus}. It is a marked focus, which is discussed in next section. 

\subsection{\label{arka:s4.2}Marked focus: Contrastive and emphatic}

A “marked focus” refers to a focus whose information structure contents in the CG are complex, typically characterised by [+contrast, +salient, \textminus given]. The \isi{contrastive} meaning ranges from a strong one to a subtle (emphatic) one, with complex encoding at the formal expression level. The complexity can be structural, involving the use of an extended \isi{clause} structure with unit fronting to the left-periphery. It may also be accompanied by specific focus markers. At the content level, the complexity is indicated by the presence of an embedded element of a contrast set.

The presence of a contrast set constrains the contextual interpretation of the focussed element. In Sembiran \ili{Balinese}, it may range from a focus with a strong \isi{contrastive} meaning (i.e. the choice of one with a clear exclusion of the other alternative(s) in the understood set) to a focus with an emphatic meaning. Emphatic focus, as the term suggests, encodes a class of salient pragmatic nuances, such as emphasis, affirmation and counter-expectation, which the speaker wants the addressee to pay attention to during communication. 

There is no clearly defined difference between \isi{contrastive} and emphatic meanings, as both are associated with the contextual presence of contrasting alternatives. For simplicity, they are discussed under the broad category of \isi{contrastive focus}. In Sembiran \ili{Balinese}, they both make use of the same linguistic resources. The difference, if any, appears to be a matter of degree, involving how explicit the contrasting entities are present in the contrast set and how strong other pragmatic nuances, such as emphasis and affirmation, are expressed in a given context. The degree of the strengths of these nuances in Sembiran \ili{Balinese} can often be seen from the extent of marking present, e.g. whether fronting is also accompanied by an overt \isi{focus marker}. Cases with strong contrast sets are discussed first, followed by cases with subtler emphatic nuances. 

Cases for clear \isi{contrastive focus} can be informally represented, as in (\ref{e:arka:18}). The representation shows that X is the \isi{contrastive focus} if it is the selected member of a contrast set in the current CG, with the other contrasting entity, Y, excluded from the set. The presence of a contrast set is often strong and expressed by some type of structural and/or particle marking. This is exemplified by the question-answer pair from \ili{English} shown in (\ref{e:arka:19}). The question (Q) sets the contrast between two alternatives, which are overtly marked by the disjunction \textit{or} in the question. The answer (A) in (\ref{e:arka:19}) selects one (‘the white’) and excludes the other. The \isi{contrastive focus} of the answer (\hyperref[e:arka:19a]{19A}) can be represented as in (\ref{e:arka:20}). Note that what is new in the answer is not the two entities in the set (as they are both present in the CG), rather, it is the selection of one of them.

\begin{exe}
	\ex\label{e:arka:18}
	\begin{tabular}[t]{ll}
	Contrastive focus X:  & \{[X]\textsubscript{ContrFoc}, [Y] …\}\textsubscript{Foc}\\
	&[where the contrasting entity, Y, is clearly \\
	&established in the immediate/current CG.]
	\end{tabular}
\end{exe}

\begin{exe}
	\ex\label{e:arka:19}
	\begin{xlist}
		\exi{Q: Which laundry did John wash, the white or the coloured?}\label{e19q}
		\exi{A: He \textit{washed} the WHITE laundry.} \citep[48]{Erteschick-Shir2007}\label{e:arka:19a}
	\end{xlist}
\end{exe}

\begin{exe}
	\ex\label{e:arka:20} \{[‘the white’]\textsubscript{ContrFoc}, [‘the coloured’]\}\textsubscript{Foc}
\end{exe}

\noindent
In Sembiran \ili{Balinese}, strategies used to express contrast include the following: fronting to the left-periphery position, structural parallelism, lexical items (e.g. antonymous words), polarity particles, focus markers and \isi{prosodic prominence} (i.e. stress). Fronting is the most common strategy, which is often combined with one or more of the other strategies.  

Consider the context of \isi{contrastive focus} in the second \isi{clause} in (\ref{e:arka:21}). The presence of polarity \textit{ngara} in the first sentence sets one (negative) option in the bipolar contrast set (X in the representation in (\ref{e:arka:22})). The second \isi{clause} adds new specific information to this negative option by stating that the money was actually corrupted (lit. ‘taken and eaten’). Crucially, the speaker provides extra emphasis on this by preposing the VP clause-initially. It therefore bears a \isi{contrastive focus}, as it indicates that the speaker strongly highlights the negative option, excluding the positive option (Y).

\begin{exe}
	\ex{Context: The secretary officer in the village was trusted to collect the money needed to cover the costs for the issuance of the land certificates for a group of people in the village, including the speaker; however, the money was corrupted by the secretary and the village head.  }\label{e:arka:21}\\
	\gll Tau-tau    {\ob}pipis-e{\cb}\textsc{\textsubscript{P.Top}}  {\ob}ngara  setor=a   ke   kantor{\cb}\textsc{\textsubscript{Foc}}, {\ob}{\ob}juwang=a   amah=a   di jalan{\cb}\textsc{\textsubscript{ContrFoc}}{\cb}\textsc{\textsubscript{NewFoc}}   {\ob}ento{\cb}\textsc{\textsubscript{P.Piv.Top}} {\ob}ajak=a     wakil   prebekel-e{\cb}\textsc{\textsubscript{NewFoc}}.\\
	know-\textsc{red}   \phantom{[}money-\textsc{def}   \phantom{[}\textsc{neg}   transfer=3  to   office  \phantom{[[}\textsc{uv}.take=3   \textsc{uv}.eat=3  on   road  \phantom{[}that  \phantom{[}accompany=3    deputy  head.villate-\textsc{def}\\
	\glt ‘Surprisingly, the money was [not transferred to the (Land) office]\textsubscript{Foc},  that (money) was [taken and eaten along the way]\textsubscript{ContrFoc} by him and the village head.’
\end{exe}

\begin{exe}
	\ex\label{e:arka:22}\{[X: ‘not transferred’,‘taken and eaten’]\textsubscript{ContrFoc}, [Y: ‘transferred’]\}\textsubscript{NewFoc}
\end{exe}

\noindent
The example in (\ref{e:arka:23}) further illustrates \isi{contrastive focus}, achieved by using passivisation, whereby the new information is made the subject. Linking an argument with new information (i.e. \isi{new focus}) to the \isi{pivot} is rather unusual as far as information structure are concerned; however this is done for a good communicative purpose, namely to achieve an element of surprise associated with the new information. That is, the focus item needs to be fronted to gain the contrast meaning. Note that sentence (\ref{e:arka:23}) is a reported speech where the first \isi{clause} is the reported question asking for water. The second \isi{clause} is the reported answer. The entity ‘blood’ is the \isi{new focus}, as it is the answer to the question. It is also \isi{contrastive} because it is being contrasted with the expected answer (‘water’). This is a folktale, a work of fiction with a giant as the main character. It is full of surprises, e.g. the giant eats human beings and drinks human blood.  

\begin{exe}
	\ex\label{e:arka:23}
	\begin{xlist}
		\ex{Context: a story about a girl called \textit{bawang}, who is asked to cook by a giant who eats humans.}\label{e:arka:23a}\\
		\gll Mara   iya   nakon-ang   {\ob}yéh{\cb}\textsubscript{P.Foc}, {\ob}getih   kanya{\cb}\textsubscript{P.Top}   tuduh-ang=a.\\
		just.time  3\textsc{sg}   \textsc{av}.ask-\textsc{appl}    \phantom{[}water, \phantom{[}blood   \textsc{part}   \textsc{uv}.point-\textsc{appl}=3\\
		\textbf{‘}When she asked for water, it was blood that he pointed at.’
		\ex{\{[‘blood’]\textsubscript{ContrFoc}, [‘water(asked)’]\}\textsubscript{NewFocus}}\label{e:arka:23b}
	\end{xlist}
\end{exe}

\noindent
In this example, the contrast set is clearly established through (reported) question and answer pairing. However, in other cases, the elements in the contrast set might be fully understood only in relation to a complex locally/culturally specific CG, not simply the truth condition involved. For example, the element of [+contrast] associated with a counter expectation in the following sentence can be fully understood only in the local cultural setting as described in (\ref{e:arka:24}):  

\begin{exe}
	\ex{Socio-cultural context: the speaker is a poor woman involved in the so-called \textit{ngadas} practice: she was given a female piglet to look after as capital by somebody else. The agreement was that when the pig grew and became a mother-pig with its own piglets, the speaker should pay the owner back using the offspring.}\label{e:arka:24}\\
	\gll {\ob}Ba   ada   {\ob}ukuran   lima   bulan{\cb}\textsubscript{NewFoc} {\ob}ubuwin{\cb}{\cb}\textsubscript{Comment} {\ob}kucit-e{\cb}\textsubscript{P.Top}, {\ob}mara   ukuran   setengah{\cb}\textsubscript{ContrFoc}  {\ob}gede-n   kucit-e{\cb}\textsubscript{S.Top}, {\ob}mati{\cb}\textsubscript{ContrFoc}   {\ob}kucit-e{\cb}\textsubscript{S.Top}.\\
	\phantom{[}already   exist  \phantom{[}about      five  month \phantom{[}\textsc{uv}.look.after  \phantom{[}piglet-\textsc{def} \phantom{[}just    about  half  \phantom{[}size-\textsc{nml}  piglet-\textsc{def} \phantom{[}die      \phantom{[}piglet-\textsc{def}\\
	\glt ‘Presumably, for about five months, I looked after the pig. The size of the pig’s body was about half the size of a mature one’s, (but) it died unexpectedly.’
\end{exe}

\noindent
Note that the \isi{contrastive focus} of the last sentence in (\ref{e:arka:24}) arises from the fronting of the predicate \textit{mati} ‘died’. It is also marked by a prominent prosodic stress, resulting in the speaker's subtle complex meaning of ‘surprise, unexpectedness, unwantedness’. This is understood in the socio-cultural context described in (\ref{e:arka:24}), where the piglet is not hers but a type of loan capital. The \isi{contrastive focus} also expresses the speaker’s strong feelings of disappointment in contrast to her expectation that it would grow and eventually give birth to offspring. The piglet’s death was premature (at around five months of age). It was still relatively small, at half the size of a full-grown pig.  

The contrast element in the focus can often be augmented by the use of focus markers in addition to constituent fronting. Here, the use of focus particles \textit{ba} and \textit{jeg} are exemplified in Sembiran \ili{Balinese}. Sentence (\ref{e:arka:25}) illustrates the use of \textit{ba}. This particle appears to have originated from \textit{suba}, the adverb/auxiliary meaning ‘already/perfective’, which is also often abbreviated as \textit{ba}.\footnote{It also has a prepositional-like meaning ‘after’ as in (\textit{su})\textit{ba kento} ‘after (like-)that’. The grammaticalisation of forms with these meanings has been reported in other languages (cf. \citealt[17, 134]{Heine2002}).} Both appear in example (\ref{e:arka:25}). The free translation is given here to show the \isi{emphatic focus} involved, namely the long-awaited and good news about the completion of the making of the shirt material. The focus particle \textit{ba} carries a sense of relief, or of no more thinking/concern on the part of the speaker. Note that it would take days, or even weeks, to complete the weaving. This again points to the fact that the \isi{emphatic focus} has a subtle meaning that would only be understood in a given local cultural setting. 

\begin{exe}
		\ex\label{e:arka:25}{Context, barter-based economy: the speaker promised to give the person (addressed below as \textit{Nang}\footnote{Nang is the vocative use of \textit{nanang} ‘father’. This kin term is used to address the brother of one’s father or any male of the same age as one’s father.}) a hand-woven material to create a shirt in return for six-hundred ears of corn that the person had previously given her.}\\
		\gll Nang,   {\ob}ba   tepud   ba{\cb}\textsubscript{ContrFoc}  {\ob}lakar   baju-ne{\cb}\textsubscript{S.Top}.\\
		father  \phantom{[}\textsc{perf}  done  \textsc{part}  \phantom{[}material  shirt-\textsc{def}\\
		\glt ‘Father, it’s done, the material for the shirt.’
\end{exe}

\noindent
The focus particle \textit{jeg} carries a selection of one option instead of the other(s) with negative nuances, such as something unwanted or no other alternative. This is often associated with an event/action carried out against the speaker’s/addressee’s wishes. This is exemplified in (\ref{e:arka:26}) and (\ref{e:arka:27}). In (\ref{e:arka:26}), the \isi{focus marker} \textit{jeg} appears with the negative sentence, highlighting the absence of any possession whatsoever on the part of the speaker. In (\ref{e:arka:27}), it is an imperative sentence, and the instruction to the addressee is that he must find a doctor, nobody else (e.g. not a shaman), for the mother. Note that in this second sentence, the focus is doubly marked by \textit{jeg} and \textit{ba}.

\begin{exe}
	\ex\label{e:arka:26}{Context: The speaker had a hard life with many children to raise. This sentence is part of the most difficult time in her life when one of her children was seriously ill: }\\
	\gll Kene   ojog=a  masan   parah=sen  jeg   apa  ngara  ngelah.\\
	like.this  visit=3  time   hard=very  \textsc{part}  what   \textsc{neg}  \textsc{av}.have\\
	\glt ‘The worst time hit hard at this time, I had NOTHING whatsoever.’
\end{exe}

\begin{exe}
	\ex\label{e:arka:27}
	\gll Tut Sik,  meme   \textbf{jeg} \textbf{dokter}  ba alih-ang    di  Jakula   nto.\\
	Tut Sik  mother   PART  doctor  \textsc{part}  \textsc{uv}.find-\textsc{appl}  \textsc{prep}  place  that\\
	\glt ‘Tut Gasik, as for me (Mother), you just find a DOCTOR (nobody else) for me in Tejakula.’
\end{exe}

\noindent
The example in (\ref{e:arka:28}) illustrates an \isi{emphatic focus}. It is also achieved by predicate fronting. The speaker describes her husband by placing emphasis on his negative characteristic of being lazy in contrast to an otherwise more positive alternative commonly expected for a good husband. This characterisation of laziness is the first mention of this in the text; hence, it is an emphatic \isi{new focus}. The \isi{contrastive} element has no overt expression in the preceding context. It should be understood based on the good values assumed in the community as part of the general CG.

\begin{exe}
	\ex\label{e:arka:28}
	\gll {\ob}Gelema  kalud  ng-luyur  ngara   nyak  pati   me-gaé{\cb}\textsubscript{Pred.ContrFoc},  {\ob}sommah   meme-né     ento{\cb}\textsubscript{S.Top}.\\
	\phantom{[}person  \textsc{part}   \textsc{av-}wander  \textsc{neg}   want  ever \textsc{av}-work \phantom{[}husband mother-\textsc{poss}  \textsc{dem}\\
	\glt ‘A person who’s wandering around, not wanting to work, my husband is.’ 
\end{exe}

\section{\label{s:arka:5}Topic}

Topic is defined prototypically in terms of the file-card metaphor (\citealt{Reinhart1981,Erteschick-Shir2007,Krifka2012}) in relation to the comment part of a sentence and the i-str features given in (14). The following is the definition of topic, adapted from \citet[28]{Krifka2012}:

\begin{exe}
	\ex\label{e:arka:29}
	Definition of topic:\\
	The prototypical topic constituent of a \isi{clause} is the one referring to a [+salient] entity in the CG, under which the information of the comment constituent is stored or added.
\end{exe}

\noindent
As mentioned, the [+salient] feature is meant to capture the most important cognitive property of an entity (or a set of entities) in a given context about which attention and additional information is given to increase the addressee’s knowledge (in statement), is requested from the addressee (in question), or when an action is requested (in command). This definition is consistent with the traditional concept of “aboutness” topic (\citealt{Reinhart1981}; \citealt[210]{Gundel1988,Lambrecht1994}, among others) and ‘attention’ \citep[44]{Erteschick-Shir2007}. 

The concept of prototype (\citealt{Rosch1978,Taylor1991}) was used in the definition in (\ref{e:arka:29}) to capture different types of topics, particularly because there may be a less canonical topic, called a \isi{secondary topic}. While its \isi{referent} is present in the CG, this topic is not as salient as the \isi{default topic}, which now becomes the \textsc{primary topic}.\footnote{The term \textsc{primary topic} is the \isi{default topic} in the presence of the \isi{secondary topic}. They refer to the same kind of topic and are used interchangeably in this paper.}  The \isi{secondary topic} gains its salience in relation to the \isi{primary topic}; see further discussions in \sectref{s5.4}.

Next, the most common type of topic is discussed first, namely the \isi{default topic} of a \isi{clause}, which is also grammatically a \isi{pivot}.

\subsection{\label{s5.1}Default topic} 

The term \textsc{default topic} is used to refer to the only topic in the basic (i.e. non-extended) \isi{clause} structure characterised by [+salient, +given, \textminus contrast] properties. Its \isi{referent} has been established and shared in the CG (i.e. cognitively/pragmatically salient and given). Crucially, it is not contrasted. That is, as far as CG management is concerned, it is a unit without an embedded element of contrast. Grammatically, it is the \isi{pivot} of the \isi{clause}, occupying a unique \isi{pivot} position in the \isi{clause} structure. As mentioned in \sectref{s:arka:2}, the \isi{pivot} selection and therefore \isi{default topic} selection is signalled by verbal \isi{voice} morphology.

The material realisation of the \isi{default topic} varies for discourse-pragmatic reasons. It can be an overt noun (NP), a free overt \isi{pronoun} or a zero \isi{pronoun}. The data suggests that this is determined by the activation and relative adjacency of the relevant entity in the current CG. An overt common noun topic is typically a definite topic, possibly a re-introduced \isi{default topic}; a \isi{pronoun} or a zero \isi{pronoun} is typically a continuing \isi{default topic}. Each is discussed and exemplified in the next sections.

\subsubsection{{Common nouns and proper names}}

Common noun and proper name topics constitute only 18\% of the total default topics in Sembiran \ili{Balinese}. The majority of default topics are pronominals, unexpressed/zero subjects (67\%) and overt pronominal subjects (15\%).\footnote{These statistics are based on a limited text corpus of 66,677 words, consisting of traditional folktales and a recording of the personal story of Men Dora.  The recording was first transcribed in ELAN, and then the appropriate tagging reflecting grammatical relations and information structure status was done in ELAN, before a simple statistical calculation was undertaken.}  Default topics are typically definite, i.e. having a [+given] property. Nouns gain definiteness in different ways. The most common way in text is a second (or later) mention after it is introduced as the \isi{new focus} in a previous sentence.  For example, in the following excerpt, the noun \textit{jagung} ‘corn’ is introduced in the first sentence and becomes a definite topic later, flagged by a definite marking (-\textit{e, ento}). Likewise, the NP \textit{lakar baju} ‘shirt material’ becomes the topic after the second mention, referred to by the definite determiner \textit{ento}. 

\begin{exe}
	\ex\label{e:arka:30}
	\gll Ba kento,   behang=a   meme   \textbf{jagung}   tigang   atak, nunun   ntas   meme   \textbf{lakar} \textbf{baju},    ba   tepud   \textbf{ento} … ba   kento,  \textbf{jagung-e}   \textbf{ento}   ada   a=bulan   tengah   dahar…\\
	after that  \textsc{uv}.give=3  mother  corn  three  two.hundred \textsc{av}.waive  then  mother  material shirt  \textsc{perf}  finish  that {} \textsc{perf}  that  corn-\textsc{def}  that  exist  one=month  half  \textsc{uv}.eat\\
	\glt ‘After that, I (mother) was given 600 ears of corns…I then waived shirt material; it was then done…after that, the 600 ears of corn were consumed in about one month and a half…’
\end{exe}

\noindent
A common noun can gain its \isi{topicality} (i.e. [+given]) even when first mentioned in the \isi{discourse} through a vocative use. It exophorically refers to the speaker or the addressee in a given context. The noun types that possibly function in this way are typically kin-term nouns. For example, the \isi{default topic} in (\ref{e:arka:31}) is \textit{meme} ‘mother’, used vocatively to refer to the speaker. This is the first sentence in the autobiography text. It is a topic NP because it is the entity/\isi{referent} (i.e. the speaker, \textit{Men Dora}) about whom ‘giving birth to the first child’ is being told. 

\begin{exe}
	\ex\label{e:arka:31}{Context: the speaker Men Dora told her son (i.e. the addressee) about her life.}\\
	\gll \textit{Ngelah}   \textit{panak}    {\ob}\textit{meme}{\cb}\textsubscript{A.Top}    \textit{mara}   \textit{besik}, \textit{madan} \textit{Butuh} \textit{Dora}.\\
	\textsc{av}.have  child.\textsc{p}  \phantom{[}mother  still   one   \textsc{mid}.name  B. Dora\\
	\glt ‘I (mother) gave birth to my first child, named Butuh Dora.’
\end{exe}

\noindent
A common noun can also gain its [+given] property through a possessive relation with the addressee. In example (\ref{e:arka:32}), the NP \textit{nanang caine}, ‘your father’ is the \isi{default topic} of the second sentence. It has not been mentioned in previous sentences. Its \isi{referent} is part of the shared CG information, as it is the father of the addressee who is also the husband of the speaker. 

\begin{exe}
	\ex\label{e:arka:32}{Context: the speaker told her son (i.e. the addressee) about his father.}\\
	\gll \textit{Buina} \textit{ngelah} \textit{panak} \textit{mara} \textit{patpat} \textit{kento} \textit{teh}, {\ob}\textit{nanang}{\cb}\textsc{\textsubscript{a.piv.top}} \textit{cahi}-\textit{ne} \textit{ngalih} \textit{somah} \textit{buwin}.\\
	moreover \textsc{av}.have child.\textsc{p} still four like.that  \textsc{part} \phantom{[}father 2\textsc{sg}-\textsc{def}   \textsc{av}.take wife.\textsc{p} again\\
	\glt ‘In addition, when I had given birth to four children, your father took another wife.’
\end{exe}

\subsubsection{{Pronominal topic: Reintroduced and continuing}}

When a \isi{referent} is highly active in the CG and established as the \isi{default topic}, there is often no need to express it overtly; however, if expressed overtly, it is often realised as a \isi{pronoun}, as in (\ref{e:arka:33a}). In this example, the \isi{default topic} \textit{tiyang}\footnote{Note that \textit{tiyang} is a Plains \ili{Balinese} \isi{pronoun} (h.r.). Speakers of Sembiran \ili{Balinese} are also typically fluent in Plains \ili{Balinese} as well, and code switching is common.}  (the speaker \textit{Men Dora}) is already salient in the general CG. In the two sentences immediately preceding it, the \isi{pronoun} is not the topic. It is a topic in an earlier sentence. In this case, it can be classified as a reintroduced topic (see \sectref{s5.3}); however, in the sentences immediately following (\ref{e:arka:33a}), \textit{tiyang} (or the speaker, index \textit{i}) is maintained as the \isi{default topic}. It is realised as a zero \isi{pronoun}, represented as ${\varnothing}$. The \isi{default topic} is a \isi{continuing topic} in these instances. In short, a \isi{continuing topic} is, like a reintroduced topic, a discourse-level topic, but the \isi{referent} of a \isi{continuing topic} is already present in the immediately preceding sentence.

\begin{exe}
	\ex{Context: the TOP in the two immediately previous sentences is about the speaker’s two last children out of nine children:}\label{e:arka:33}\\
	\begin{xlist}
		\ex\label{e:arka:33a}
		\gll \textit{Ba} \textit{keto} \textit{mara} {\ob}\textit{tiyang}{\cb}\textbf{\textsubscript{A.Piv.Top\_\textit{i}}} \textit{ngelah} \textit{panak} \textit{siya}.\\
		after  that   just  \phantom{[}1\textsc{sg}  \textsc{av.}have  child  nine.\\
		\glt ‘Then, after, I gave birth to a total of nine children.’ 
		\ex\label{e:arka:33b}
		\gll {\ob}${\varnothing}${\cb}\textsubscript{A.\textbf{Piv.Top}}  \textit{Ba}   \textit{nau}   \textit{ne}   \textit{ba}.\\
		\phantom{[}\textit{i}  \textsc{perf}  happy  this  \textsc{perf}\\
		\glt ‘I was already happy.’		
		\ex\label{e:arka:33c}
		\gll {\ob}${\varnothing}${\cb}\textsubscript{A.\textbf{Piv.Top}}  \textit{Ba}   \textit{lupa} {\ob}${\varnothing}${\cb}\textsubscript{A.\textbf{Piv.Top}} \textit{ba} \textit{ngucapang} …\\
		\phantom{[}\textit{i}  \textsc{perf}  forget  \phantom{[}\textit{i}  \textsc{perf}  \textsc{av}.mention\\
		\glt ‘I forgot to mention (something).’
	\end{xlist}
\end{exe}

\noindent
These data from Sembiran \ili{Balinese} represent a common pattern where an entity that is highly salient in a series of immediate states of CG is selected as the \isi{continuing topic}. It is typically formally reduced in its expressions, either as a zero \isi{pronoun} (67\%) or a (\isi{clitic}) \isi{pronoun} (15\%). This fits well with \posscitet[917]{Givón1990} observation that referents that are already active in the CG require minimal coding. This is also consistent with the findings in Plains \ili{Balinese} \citep{Pastika2006} and in other \ili{Austronesian} languages of Indonesia with verbal \isi{voice} morphology, such as \ili{Pendau} \citep{Quick2005}; however, in other \ili{Austronesian} languages with diminishing verbal \isi{voice} morphology, e.g. in certain dialects of Sasak, the use of pronominal clitics is widespread, and the discourse distribution of nominal and voice types is expected to be different (cf. \citealt{Wouk1999}).


\subsection{\label{s5.2}New topic}

While topic is typically [+given], it can be [\textminus given]. This is the topic whose \isi{referent} is firstly mentioned in the \isi{discourse} (i.e. newly introduced in the CG). It is typically introduced by the verb \textit{ada} in Sembiran \ili{Balinese}. Consider (\ref{e:arka:34}), which is the first line of a story. The subject of the first \isi{clause} is an indefinite NP \textit{tuturan satua}. This is a \isi{new topic}. Then, the second \isi{clause} provides more specific information about the \isi{new topic}.

\begin{exe}
	\ex\label{e:arka:34}
	\gll Ada   {\ob}tutur-an   satua{\cb}\textsubscript{NewTop},   madan   I   Bapa   Sedok.\\
	exist  \phantom{[}tell-\textsc{nml}  story  \textsc{mid}.call  \textsc{art}   Father   Sedok\\
	\glt ‘There is a story often talked about called \textit{I Bapa Sedok}.’
\end{exe}

\noindent
The sentence in (\ref{e:arka:35}) is in the middle of a story, but it is the first time the \isi{referent} \textit{nak} ‘person’ is introduced in the context described in (\ref{e:arka:35}). While indefinite and new, \textit{nak} is the topic here, as the information that follows is about this NP, \textit{nak}.

\begin{exe}
	\ex{Context: The sentences are about the speaker’s bad experience. She was deceived by the village official and lost her money in the process of the issuance of land certificates. Somebody else, called Sumarwi, stepped in to replace her money:}\label{e:arka:35}\\
	\gll Ada    nak   nimbalin,   Sumarwi   adan-anne neh   nimbalin.\\
	exist  person  \textsc{av}.replace  Sumarwi  name-\textsc{poss} \textsc{rel}  \textsc{av}.replace\\
	\glt ‘There’s a person reimbursing (the money); Sumarwi is the name of the person who reimbursed the money.’
\end{exe}

\subsection{\label{s5.3}Reintroduced topic}

The term \textsc{reintroduced topic} is used to refer to a topic expression associated with a salient entity already selected as a topic earlier but that is picked up again as a topic in a \isi{clause} (cf. \citealt[760]{Givón1990}); hence [+salient, + given]. It is not associated with a \isi{contrastive} set in the CG, however. The reintroduced topic has been exemplified in (\ref{e:arka:33}). In this example, the \isi{pronoun} exophorically refers to the speaker, so there is no ambiguity issue in its identification. 

When there is more than one entity in the CG that the third-person \isi{pronoun} can refer to, as the \isi{default topic}, a \isi{pronoun} may need further specific information provided by a full NP expression. In Sembiran \ili{Balinese}, this full topic NP may come later in the \isi{clause} in the right dislocated position. This is exemplified in (\ref{e:arka:36}). The \isi{pronoun} \textit{iya} in the last sentence is potentially ambiguous, as there are other participants in the CG indicated by the indices \textit{i} and \textit{j}. To avoid ambiguity, the speaker provides additional information: ‘that (first) co-wife’ (underlined) in the right detached position, index \textit{k}. Thus, there are two topic expressions referring to the same entity in this sentence; \textit{iya} is the \isi{default topic} expression, and the full NP \textit{madu-né ento} is the reintroduced topic. 

\begin{exe}
	\ex{Context: the speaker is the second of three co-wives reporting what the first co-wife (index k) has said about Sapin, the third co-wife (index i).}\label{e:arka:36}\\
	\gll ‘\textit{Tawah} {\ob}\textit{I} \textit{Sapin}-\textit{e} \textit{ento}{\cb}\textsubscript{\_i}. {\ob}{\ob}\textit{panak}{\ob}-a{\cb}\textsubscript{\_i} {\cb}\textsubscript{\_j} \textit{ngara} \textit{gaen-ang}{\ob}=a{\cb}\textsubscript{A\_i} \textit{banten} \textit{behan} ${\varnothing}$\textsubscript{S\_i} \textit{ngara} \textit{ngelah}, orahhang{\ob}=a{\cb}\textsubscript{A\_i} aget=se …’ \textit{kento} {\ob}\textit{iya}{\cb}\textsubscript{S.Top\_k} \textit{m-peta,} {\ob}\textit{madu}-\textit{n}-\textit{né} \textit{ento}{\cb}\textsubscript{Top\_k}.\\
	\phantom{‘}strange \phantom{[}\textsc{art} Sapin-\textsc{def} that \phantom{[[}child-3 {} \textsc{neg} 
	make-\textsc{appl}=3 ritual because {} \textsc{neg} \textsc{av}.have
	say=3 lucky=\textsc{part} {} like.that \phantom{[}3\textsc{sg} \textsc{av}-mention \phantom{[}co-wife-\textsc{lig}-3\textsc{poss} \textsc{det}\\
	\glt ‘Sapin\textsubscript{\_i} was strange. [Her\textsubscript{\_i} child]\textsubscript{\_j}. She\textsubscript{\_i} didn’t make any ritual 
	due to her\textsubscript{\_j} lack of money (when the child unexpectedly died prematurely), and she\textsubscript{\_i} said she was lucky’, she\textsubscript{\_k} said, the (first) co-wife\textsubscript{\_k}.’
\end{exe}

\subsection{\label{s5.4}Secondary topic}

The \textsc{secondary topic} is defined in relation to the default or \isi{primary topic}. It is defined as ‘an entity such that the utterance is construed to be \textsc{about} the relationship between it and the \isi{primary topic} (\citealt[55]{Dalrymple2014}). The \isi{secondary topic} is like the \isi{primary topic} in that it is pragmatically [+given]: It is present in the (immediate) CG; however, it is less salient than the \isi{primary topic}. Saliency reflects some type of \isi{prominence}, which can be assessed based on certain properties related to how it is marked in a given language, e.g. linear order (with the earlier sentence-initial position being more salient than later), linking (with the subject/\isi{pivot} topic being more prominent than the object topic) and explicit marking (with the focus marked by the \isi{contrastive} \isi{focus marker}, which is more prominent than the [unmarked] \isi{new focus}). 

In the \ili{English} example from \citet[148]{Lambrecht1994}, sentence (\ref{e:arka:37c}) contains two topic expressions, both expressed by the pronouns \textit{he} and \textit{her}. Their referents are already present and salient in the CG due to the preceding (\ref{e:arka:37a}) and (\ref{e:arka:37b}) sentences. Sentence (\ref{e:arka:37c}) is about \textit{John}, referred to by the subject \textit{he}; hence, this is the \isi{primary topic}. The \isi{secondary topic}, the object \textit{her}, is part of the comment constituent. The communicative intent of (\ref{e:arka:37c}) – its new information, the focus constituent of the Comment – is the \isi{assertion} of the “love-relation” in which Rosa was not loved by John. 

\begin{exe}
	\ex\label{e:arka:37}
	\begin{xlist}
	\exi{a.}\label{e:arka:37a} \textit{Whatever became of John?}
	\exi{b.}\label{e:arka:37b} \textit{He married Rosa},
	\exi{c.}\label{e:arka:37c} \textit{but} [\textit{he}]\textbf{\textsubscript{PrimaryTop}}\textsubscript{}  [[\textit{didn’t really love}]\textsubscript{Foc} [\textit{her}]\textbf{\textsubscript{SecondaryTop}}]\textsubscript{Comment} 
	\end{xlist}
\end{exe}

\noindent
The \isi{secondary topic}, like the \isi{primary topic}, can also be a \isi{continuing topic} when its \isi{referent} is already salient and present in the general CG. It is often the case that two salient entities in the CG alternate between the primary and \isi{secondary topic}, depending on the focussed predicate involved. Consider the following excerpt from the text in (\ref{e:arka:38}), which comes after example (\ref{e:arka:32}). The speaker repeats the same message with some new information about her status as one of three co-wives of her husband. Both the speaker and her husband are highly salient in the immediate CG. The speaker (index \textit{i}) is the \isi{continuing topic} in the three sentences, becoming the \isi{primary topic} in (\ref{e:arka:38b}) and (\ref{e:arka:38c}). Her husband, realised as the \isi{clitic} \textit{=a} (index \textit{j}), is also the continuing but \isi{secondary topic} in (\ref{e:arka:38b}) and (\ref{e:arka:38c}). Note that in (\ref{e:arka:38b}), the husband becomes the \isi{primary topic} of the adverbial \isi{clause}, realised as a zero \isi{pronoun}. 

\begin{exe}
	\ex\label{e:arka:38}
	\begin{xlist}
		\ex\label{e:arka:38a}
		\gll Ba kento mara {\ob}meme{\cb}\textsubscript{A.Top\_i} ngelah panak patpat,\\
		after that just.after \phantom{[}mother \textsc{av}.have child four\\
		\glt ‘Then, after, I (mother) gave birth to four children,’
		\ex\label{e:arka:38b}
		\gll ${\varnothing}$\textsubscript{P.Top1\_i} Kalahin{\ob}=a{\cb}\textsubscript{A.TOP2\_j} {\ob}${\varnothing}$\textsubscript{ P.Top1\_j}  ngallih  somah buwin{\cb}.\\
		{} \textsc{uv}.left=3 {} \textsc{av}.take  wife again\\
		\glt ‘I was left by him to take a new wife again.’
		\ex\label{e:arka:38c}
		\gll Madu-telu-ang{\ob}=a{\cb}\textsubscript{A.Top2\_j} {\ob}meme{\cb}\textsubscript{P.Top1}.\\
		co-wife-\textsc{caus}=3 \phantom{[}mother\\
		\glt ‘I was made one of his three wives.’
	\end{xlist}
\end{exe}

\noindent
In Sembiran \ili{Balinese}, instances of the \isi{secondary topic} are typically the A of the \isi{UV} verbs expressed as (\isi{clitic}) pronouns. In this case, P is the \isi{primary topic}, also highly topical and selected as the \isi{pivot}. Sembiran \ili{Balinese} is like Plain \ili{Balinese}, in that in both \isi{UV} and \isi{AV} clauses, the A argument is highly topical and even more topical than the U argument. \citet{Pastika2006} presented statistical evidence from a referential distance measure (cf. \citealt{Givón1994}), which showed that the significant factor for the selection of \isi{voice} type, \isi{AV} vs. \isi{UV}, in \ili{Balinese} is the \isi{topicality} of U rather than that of A. 

\subsection{\label{s5.5}Contrastive topic}

The \isi{contrastive topic} expression is defined as being associated with [+salient, +given, +contrast] features. That is, like the types of topic discussed thus far, it refers to a \isi{referent} already present in the CG and is highly salient (e.g. about which comment information is added); however, it differs in that it carries an element of contrast (i.e. [+contrast]). On the expression side, the [+contrast] feature has an explicit marking of some type. On the CG side, it is associated with an established contrast set of referents. 

A contrast is marked in different ways. In the \ili{English} question-answer example in (\ref{e:arka:39}), the contrast set is restricted and established by the nominal \textit{siblings} in question (A) and also by structural parallelism through the coordination accompanied by a parallel \isi{prosody} in the answer. In this pair of questions and answers, the subject NPs in B are instances of a \isi{contrastive topic}, analysed as having a focus embedded in the topic (\citealt{Erteschick-Shir2007}; \citealt[30]{Krifka2012}). Focus carries the presence of alternatives (\citealt{Krifka2012}), an element also shared with [+contrast]; however, the focus may be simply [\textminus contrast]. That is, new information is added to the common ground without an overt \isi{contrastive} reference to other entities in the CG, which has been discussed in detail in \sectref{arka:s4.1}. The \isi{contrastive topic} is represented as a topic with an embedded \isi{contrastive focus} (ContrFoc). For example, (\ref{e:arka:40}): (\hyperref[e40i]{40i}) and (\hyperref[e40ii]{40ii}) are the representations of the \isi{contrastive} topics of clauses \hyperref[e39bi]{B.i} and \hyperref[e39bii]{B.ii}, respectively. The topics consist of a set of two salient referents in the CG that are commented on by means of their different occupations, which are not shown by the representation in (\ref{e:arka:40}).

\begin{exe}
	\ex\label{e:arka:39} \begin{xlist}
		\exi{A:} \textit{What do your siblings do?}
		\exi{B:}
		\begin{xlist}
			\ex\label{e39bi}{[[\textit{My SISter}]\textsubscript{Foc}]\textsubscript{Top}   [\textit{studies  MEDicine}]\textsubscript{Foc}, \textit{and}}
			\ex\label{e39bii}{[[\textit{my BROther}]\textsubscript{Foc}]\textsubscript{Top} \textit{is} 
			[\textit{working on FREIGHT ship}]\textsubscript{Foc}.
			\citep[30]{Krifka2012}}
		\end{xlist}
	\end{xlist}
\end{exe}

\begin{exe}
	\ex\label{e:arka:40}
	\begin{xlist}
		\exi{i.}\label{e40i} \{[‘my sister’]\textsubscript{ContrFoc}, [‘my brother’]\}\textsubscript{Top}
		\exi{ii.}\label{e40ii} \{[‘my sister’], [‘my brother’]\textsubscript{ContrFoc} \}\textsubscript{Top}
	\end{xlist}
\end{exe}

\noindent
Parallelism by means of coordination, as shown in the \ili{English} example above, is common cross-linguistically. Parallelism that encodes a \isi{contrastive} set membership is often achieved by using the same or synonymous lexical items in structurally marked constructions, such as a left-dislocated position.\footnote{Parallelism is a prominent feature of the languages of central and eastern Indonesia, particularly in the domain of ritual language (\citealt{Fox1988,Grimes1997,Kuipers1998,Arka2010,Sumitri2016}). The information structure in a ritual language require further research.}  A \isi{contrastive topic} expressed in this way is found in Sembiran \ili{Balinese}. This is exemplified by the topic expression \textit{iya ba} in the second \isi{clause} in example (\ref{e:arka:41a}). The partial information structure representation is given in (\ref{e:arka:41b}).

\begin{exe}
	\ex\label{e:arka:41}
	\begin{xlist}
		\ex\label{e:arka:41a}
		\gll {\ob}Meme{\cb}\textsubscript{P.Top\_i}  ngara   ajak=a  ng-{\USOParen}g{\USCParen}ellah-ang tanah   warisan   di Pramboan  mapan {\ob}iya   ba{\cb}\textsubscript{Top\_j},  {\ob}somah-anne  senikan   ento{\cb}\textsubscript{P\_j}  ajak=a.\\
		\phantom{[}mother  \textsc{neg}   invite=3  \textsc{av}{}-own-\textsc{appl} land   inheritance   in  Pramboan  since \phantom{[}3\textsc{sg}   \textsc{emph}  \phantom{[}wife-3\textsc{sg.poss}  younger  that   \textsc{uv}.invite=3\\
		\glt ‘[I (mother)] was not invited to share the inherited land in
		Pramboan because it’s [she]\textsubscript{\_j} [his younger wife]\textsubscript{\_j} whom he invited.’
		\ex{CG: \{[‘younger wife’]\textsubscript{\_j} \textsubscript{ContrFoc}, [‘mother’]\textsubscript{\_i}\} \textsubscript{Top}}\label{e:arka:41b}
	\end{xlist}
\end{exe}

\noindent
As seen in (\ref{e:arka:41b}), the ‘co-younger wife’ is the topic, i.e. the salient participant about/to whom a land-sharing invitation was discussed/offered. It is a \isi{contrastive topic}, with the contrast achieved by means of the contrasting element of negation associated with the same verb \textit{ajak}. The younger co-wife is referred to by \textit{iya}, which appears in the left-dislocated topic position and whose pragmatic effect is augmented by the use of the emphatic particle \textit{ba}. The full NP \textit{somah-anne} \textit{senikan ento}, which appears in the \isi{pivot} position, provides additional specific information about \textit{iya}.  

Example (\ref{e:arka:42}) also illustrates a \isi{contrastive topic}. The P object \textit{ne}, ‘this’ refers to the land being discussed. It is topicalised through fronting to the left-most sentence-initial position. This way, it gains its \isi{contrastive} effect; hence, it is a \isi{contrastive topic}. Note that the verb is in the \isi{AV} form with the subject/\isi{pivot} being the A argument \textit{cahi}. The A argument is also pragmatically prominent, appearing with the \isi{focus marker} \textit{ba}. Both of the referents of the A and P arguments are present in the preceding sentences, as described in the context description in (\ref{e:arka:42a}). The information structure is informally represented in (\ref{e:arka:42b}). 

\begin{exe}
	\ex\label{e:arka:42}
	\begin{xlist}
		\ex{Context: Bapak, the officer from the Agrarian Office, measured a piece of state-owned land to be granted to Butuh Dora. Two salient entities are involved in the first \isi{clause}: the addressee \textit{cahi} and the land.}\label{e:arka:42a}\\
		\gll {\ob}Ne{\cb}\textsubscript{Top} {\ob}cahi{\cb}\textsubscript{A.Piv.Foc} ba ngelahang. Ne ngara ukur Bapak ne.\\
		this 2\textsc{sg} \textsc{part} \textsc{av}.have.\textsc{appl} this \textsc{neg} \textsc{uv}.measure father  this\\
		\glt ‘As for this piece of land, YOU are the one who owns (it). This one, I didn’t measure it.’
		\ex{CG:   \{[‘this land’]\textsubscript{ContrFoc}, [‘the other land’],\}\textsubscript{Top} \{[‘you’]\textsubscript{ContrFoc}, [‘the others’], …\}\textsubscript{Foc}}\label{e:arka:42b}
	\end{xlist}
\end{exe}

\noindent
The type of structure given in (\ref{e:arka:42}) is of particular interest, as both A and P are equally \isi{contrastive}, highly salient and already present in the CG. This reflects the interaction between the topic and focus and creates complications regarding the distinction between the primary and secondary topics; however, it appears that in a given structure, only one is selected as the most prominent topic. This is the left-most unit, \textit{ne} ‘this (land)’, because the rest of the predication is about this \isi{referent}. This topic functions as the \isi{frame setter}, which delimits the interpretation of the other parts of the sentence. In terms of CG management, and in line with the definition of topic presented in (\ref{e:arka:29}), based on the free translation, it is this topicalised P/object that is closer to the ‘about topichood’ than the subject \textit{cahi}. The object is more prominent than the subject as far as the information structure is concerned; however, grammatically, there is good cross-linguistic evidence (\citealt{Keenan1977,Bresnan2001}; among others) as well as language specific evidence, e.g. from reflexive binding in (Sembiran) \ili{Balinese} (\citealt{Arka2003,Sedeng2007}) that the object is less prominent than the subject/\isi{pivot}. 

\section{\label{s:arka:6}Frame setting and left-periphery positions}

\subsection{\label{s6.1}Frame setting and topicalisation}

Frame setting, which is exemplified in \ili{English} in (\ref{e:arka:43}), is part of the so-called \textit{delimitation} in information structure \citep{Krifka2012}. The \isi{frame setter} \textit{healthwise/as for his health} in this example restricts the predication: the new/gap focus FINE must be understood within the frame of ‘(his) health’. Note that the topic here is \textit{John}, as the predication of being ‘fine’ is about him.

\begin{exe}
	\ex\label{e:arka:43}
	\begin{xlist}
		\exi{Q:} How is John?\label{e43q}
		\exi{A:} \{Healthwise/As for his health\}, he is FINE. \citep[31]{Krifka2012}
	\end{xlist}
\end{exe}

\noindent
The \isi{frame setter} carries the presence of alternatives within a particular specific CG domain set out, or assumed, by the speaker. This specificity property of the CG overlaps with definiteness characterising the topic, captured by the [+given] property in (\ref{e:arka:14}). The \isi{frame setter} therefore resembles a \isi{contrastive topic}, e.g. the frame setters \textit{healthwise/as for his health} means ‘in terms of/talking about his health instead of his other situations’; ‘his health’ is the specific CG domain within which ‘fine’ must be understood. However, a \isi{frame setter} is not exactly the same as a \isi{contrastive topic} as it might carry only some degree of domain specificity, not the really strong properties of definiteness and saliency captured by [+given] and [+salient] features exhibited in (\ref{e:arka:14}). We argue that the \isi{frame setter} should be characterised as [+salient, ±given, +contrast], where ±given captures the idea of specificity and a low degree of \isi{givenness}; this is further discussed in \sectref{s6.3}.

In Sembiran \ili{Balinese}, like in \ili{English}, the \isi{frame setter} occupies a clause-external \isi{left periphery} position. This is a position left-adjoined to the maximal sentence structure of CP in terms of a version of the X-bar syntax in LFG adopted here; see \sectref{s:arka:2}, also \citep{Arka2003}. Sentence (\ref{e:arka:44a}) is an example of frame setting from Sembiran \ili{Balinese}. The phrase structure of this sentence is given in (\ref{e:arka:44b}).

\begin{exe}
	\ex\label{e:arka:44}
	\begin{xlist}
		\ex\label{e:arka:44a}
		\gll {\ob}En  buat  ngamah{\cb}\textsubscript{Frame}, {\ob}icang   ba{\cb}\textsubscript{ContrFoc}   mehang   iya  nasi.\\
		\phantom{[}if   about \textsc{av}.eat   \phantom{[}1\textsc{sg}  \textsc{part}  \textsc{av}.give  3\textsc{sg}  rice\\
		\glt ‘As for eating needs, I am the person who gave him meals.’
		\ex\label{e:arka:44b}
		\begin{minipage}[t]{\linewidth}
		\Tree [.CP \qroof{\textit{en buat ngamah,}}.{PP\\(FrSetter)} [.CP \qroof{\textit{icang  ba}}.NP \qroof{\textit{mehang iya nasi}}.IP ]]
		\end{minipage}
	\end{xlist}
\end{exe}

\noindent
The predication of ‘giving him rice’ in (\ref{e:arka:44}) must be interpreted in the context of the \isi{frame setter} of ‘eating needs’ instead of other needs. There is no co-referential or argument-dependency relation between the \isi{frame setter} \textit{en buat ngamah} and any element in the predication. The element \textit{nasi} ‘rice’ is related to the \isi{frame setter} in a sense through its semantic field, e.g. ‘food-related’ in this case. 

However, there are cases where the \isi{frame setter} expression can be understood as the syntactic dependent of the predicate. These are cases that are traditionally known as left-dislocation and \isi{topicalisation}, exemplified in \ili{English} in (\ref{e:arka:45}) and (\ref{e:arka:46}), respectively \citep{Foley2007info}. 

\begin{exe}
	\ex\label{e:arka:45}
	\begin{xlist}
		\ex\label{e:arka:45a} Turtles, they make the greatest pets. (Left dislocation)
		\ex\label{e:arka:45b} Mary, I went to university with her.
	\end{xlist}
\end{exe}

\begin{exe}
	\ex\label{e:arka:46}
	\begin{xlist}
		\ex That dish, I haven’t tried. (Topicalisation)
		\ex For Egbert, I would do anything. 
	\end{xlist}
\end{exe}

\noindent
Left-dislocation and \isi{topicalisation} are similar but different types of constructions. In left-dislocation, the \isi{frame setter} and a syntactic dependent in the predication are related by means of a pronominal copy. In \isi{topicalisation}, they are related through a filler-gap relation. In languages such as \ili{English}, left-dislocation is only available for a \isi{pivot}/subject. A \isi{topicalisation} of a subject is ungrammatical, as it would leave the subject position unoccupied, e.g. * \textit{Turtles, -- make the greatest pets} \citep{Foley2007info}. 

In terms of information structure, these topicalised/left-dislocated units are \isi{contrastive} topics, as they carry the presence of contrasting alternatives, e.g. ‘turtles in contrast to other animals as pets’ in (\ref{e:arka:45a}).

Example (\ref{e:arka:47}) illustrates left-dislocation in Sembiran \ili{Balinese}. The left-dislocated topic (index \textit{i}) is anaphorically referenced by the \isi{pronoun} \textit{iya} in the object position. 

\begin{exe}
	\ex\label{e:arka:47}
	\gll {\ob}Beli Dora{\cb}\textsubscript{\_\textit{i}},   nang   tua   jua   ng-adep-in  iya\textsubscript{\_\textit{i}}  nyuh {pluk kutus}   puhun.\\
	\phantom{[}Beli Dora  father   old   \textsc{part}  \textsc{av}-sell-\textsc{appl}  3\textsc{sg}   coconut eighteen   trees\\
	\glt ‘As for Brother Dora, it was uncle who sold eighteen coconut trees to him.’
\end{exe}

\noindent
There is no clear difference between left-dislocation and \isi{topicalisation} in (Sembiran) \ili{Balinese}. The pronominal copy involved in left-dislocation can be dropped. Sentence (\ref{e:arka:47}) is still acceptable when \textit{iya} is elided, making left-dislocation and \isi{topicalisation} indistinguishable in \ili{Balinese}. In addition, the overt third-person \isi{pronoun} \textit{iya} in (Sembiran) \ili{Balinese} only refers to animate beings, typically human referents. A definite non-human inanimate \isi{referent} is expressed by a zero \isi{pronoun} in \ili{Balinese}. Thus, when the fronted topic NP is associated with an inanimate entity, the structure never appears with an overt pronominal copy \textit{iya}.  This is exemplified in (\ref{e:arka:48}), where the fronted topic \textit{nyuh nanange nto} ‘father’s coconut tree’ is in the \isi{left periphery} position. It is the P argument of the verb \textit{nebus} ‘\textsc{av}.redeem’. In its object argument position, the P argument has no overt realisation, indicated by a Ø. Note that it cannot be overtly realised by \textit{iya}, indicated by (*\textit{iya}). 

\begin{exe}
	\ex\label{e:arka:48}
	\gll {\ob}Nyuh  nanang-é  ento{\cb}\textsubscript{Top}  nagih  beli Mudiasir  buwin   nebus   {\USEmptySet}  {\USSlash}   {\USOParen}{\USStar}iya{\USCParen}, nagih   bayah=a  ny-{\USOParen}c{\USCParen}icil {\USEmptySet}, ngara  behang-a   kén   Man Jantuk.\\
	\phantom{[}coconut father-\textsc{def}  that   \textsc{av}.intend  brother Mudiasir again  \textsc{av}.redeem {} {} \phantom{(*}3\textsc{sg} intend  pay=3  \textsc{av}-pay {} \textsc{neg}  give-\textsc{pass}   by  Man Jantuk\\
	\glt ‘As for Father’s coconut tree, (brother) Mudiasi wanted to redeem (it); (he) wanted to pay it in instalments, but it was not accepted by Man Jantuk.’
\end{exe}

\noindent
The fact associated with the definite inanimate \isi{referent}, such as in (\ref{e:arka:48}) and other cases with optional \textit{iya} for a human \isi{referent}, shows that it is unclear whether the unexpressed argument is a gap (i.e. \isi{topicalisation}) or a zero \isi{pronoun} (i.e. left-dislocation). For these reasons, the term \isi{topicalisation} was used for fronted topic NPs in the left peripheral position for both cases with or without an overt pronominal copy; however, if necessary, the empty position was represented by Ø to make the original position of the fronted NP explicit.

\subsection{\label{s6.2}Ordering of marked topic and focus}

When both the topic and focus are fronted to \isi{left periphery} positions, there are immediate questions. First, what is the constraint, if any, in terms of their order? Second, what does the constraint mean in terms of information structure and the broader grammatical system? The empirical issue in relation to the first question is addressed in this section. The second issue is briefly discussed in the conclusion in \sectref{s:arka:7}. 

When two marked \isi{discourse} functions are in the \isi{left periphery} positions, the order is the \isi{frame setter}/\isi{contrastive topic} first, followed by \isi{contrastive focus}, as informally formulated in (\ref{e:arka:49}). This is illustrated in (\ref{e:arka:44}). Another example is given in (\ref{e:arka:50}); however, when the focus is a question word (QW), the fronted focus can precede the \isi{frame setter}/\isi{contrastive topic} (see (\ref{e:arka:54})).

\begin{exe}
	\ex\label{e:arka:49}
	[XP]\textsubscript{Frame}/\textsubscript{ConstrTopic}  [XP]\textsubscript{ConstFoc}   [IP]core \isi{clause}
\end{exe}

\begin{exe}
	\ex\label{e:arka:50}
	\gll {\ob}{\USOParen}En   buat{\USCParen}  behas{\cb}\textsubscript{P.ContrTop},   {\ob}iya{\cb}\textsubscript{A.Piv.ContrFoc}  lakar  meli-ang {\USEmptySet}; {\USOParen}en  buat{\USCParen}  céléng{\cb}\textsubscript{P.ContrTop},   {\ob}cahi{\cb}\textsubscript{ A.Piv.ContrFoc}  ngurus-ang   {\USEmptySet}.\\
	\phantom{[}(If   about)   rice  \phantom{[}3\textsc{sg} \textsc{fut} \textsc{av}.buy-\textsc{appl} {} \phantom{(}if   about   pig     \phantom{[}2\textsc{sg} \textsc{av}.arrange-\textsc{appl}\\
	\glt ‘As for rice, he'll buy it; as for pigs, you have to arrange them.’
\end{exe}

\noindent
In example (\ref{e:arka:50}), there are parallel clausal structures with their P arguments \textit{behas}, ‘rice’ and \textit{celeng}, ‘pig’ functioning as frame setters, which are also \isi{contrastive} topics. Note that the structure is in the \isi{AV} \isi{voice} with the A being a \isi{pivot}. The unmarked position of the object P is postverbal, indicated by Ø. 

The evidence that these sentence-initial expressions in (\ref{e:arka:50}) are topics is that they can be marked by the topic phrasal marker \textit{en buat} ‘as for’. They are also frame setters, as they delimit the interpretation of the predication. That is, the action of buying is about/in relation to rice, whereas the other arrangement is in relation to pigs. 

Another important point to note from example (\ref{e:arka:50}) is that the topic expressions are not definite. They are indefinite/generic, referring to a class of entities called \textit{behas} ‘rice’ versus another class called \textit{celeng} ‘pig’. No particular rice or pig is referred to: any rice or pig would do. In short, this provides evidence from Sembiran \ili{Balinese} that the topic is not necessarily definite.

There is evidence that the actor \isi{pivot} arguments, \textit{iya} and \textit{cahi}, in (\ref{e:arka:50}) are a \isi{contrastive focus} because these are units that can be marked by the \isi{contrastive focus} relativiser (\textit{a})\textit{ne} – an exclusive property of a \isi{pivot}. The fact that only a \isi{pivot} can be relativised is a well-known characteristic of \ili{Austronesian} languages. Thus, the \isi{pivot} \textit{iya} can receive \textit{ane}, marking the contrast (\ref{e:arka:51a}). In contrast, marking the topicalised P results in an ungrammatical structure, as shown in (\ref{e:arka:51b}). 

\begin{exe}
	\ex\label{e:arka:51}
	\begin{xlist}
		\ex\label{e:arka:51a}
		\gll {\ob}Behas{\cb}\textsubscript{P.ContrTop},  {\ob}iya{\cb}\textsubscript{A.Piv.ContrFoc}  ane  lakar  meli-ang.\\
		\phantom{[}rice  \phantom{[}3\textsc{sg}     \textsc{rel}    \textsc{fut}  \textsc{av}.buy-\textsc{appl}\\
		\glt ‘As for rice, he is the one who will buy it.’
		\ex\label{e:arka:51b}
		\gll {\USStar} {\ob}Behas{\cb}\textsubscript{P.ContrTop} ane,   {\ob}iya{\cb}\textsubscript{A.Piv.ContrFoc}  lakar  meli-ang.\\
		\phantom{*} \phantom{[}rice       \textsc{rel}  \phantom{[}3\textsc{sg}  \textsc{fut}    \textsc{av}.buy-\textsc{appl}\\
		NOT FOR ‘As for rice, he will buy it’ or ‘It is rice that he will buy.’
	\end{xlist}
\end{exe}

\noindent
Reversing the order with the \isi{contrastive focus} first and the \isi{frame setter}/\isi{contrastive topic} after compromises the acceptability. In contrast to (\ref{e:arka:50}), for example, the following is unacceptable:

\begin{exe}
	\ex\label{e:arka:52}
	\gll {\USQMark}{\USStar}   {\ob}iya{\cb}\textsubscript{A.Piv.ContrFoc}   {\ob}{\USOParen}en   buat{\USCParen}  behas{\cb}\textsubscript{P.ContrTop}, lakar  meli-ang  {\USEmptySet};\\
	\phantom{?*} \phantom{[}3\textsc{sg}   \phantom{[}(if   about)  rice \textsc{fut}  \textsc{av}.buy-\textsc{appl}\\
	FOR ‘As for rice, he is the one who will buy it.’
\end{exe}

\noindent
Unlike the previous example in (\ref{e:arka:50}), the \isi{frame setter}/topic in example (\ref{e:arka:53a}) is definite: 

\begin{exe}
	\ex\label{e:arka:53}
	\begin{xlist}
		\ex\label{e:arka:53a}
		\gll {\ob}Ne{\cb}\textsubscript{Top}  {\ob}cahi{\cb}\textsubscript{A.Piv.Foc}   ba   ane  ngelahang. \\
		\phantom{[}this   \phantom{[}2\textsc{sg}  \textsc{part}  \textsc{rel}  \textsc{av}.have.\textsc{appl}\\
		\glt ‘This (land), YOU are the one who owns (it).’
		\ex\label{e:arka:53b}
		\gll {\ob}Ne{\cb}\textsubscript{Top}  ba  {\ob}cahi{\cb}\textsubscript{A.Piv}   {\USOParen}ba{\USCParen}  ane   ngelahang. \\
		\phantom{[}this   \textsc{part}  \phantom{[}2\textsc{sg}  \phantom{(}\textsc{part}  \textsc{rel}  \textsc{av}.have.\textsc{appl}\\
		\glt ‘THIS (LAND), YOU are the one who owns (it).’
		\ex\label{e:arka:53c}
		\gll {\USStar} {\ob}Ne{\cb}\textsubscript{Top}  ba  ane   {\ob}cahi{\cb}\textsubscript{A.Piv}   {\USOParen}ba{\USCParen}  ngelahang. \\
		\phantom{*} \phantom{[}this   \textsc{part}  \textsc{rel}   \phantom{[}2\textsc{sg}  \phantom{(}\textsc{part}  \textsc{av}.have.\textsc{appl}\\
		\glt ‘NOT FOR ‘THIS (land), YOU are the one who owns (it)’ or	\glt ‘It is THIS (LAND) that YOU are the one who owns it.’ 
	\end{xlist}
\end{exe}

\noindent
Note that the focus appears with a particle \textit{ba} marking [+contrast]. This particle can be associated with either a topic or a focus; hence, (\ref{e:arka:53b}) with both a topic and a focus marked by \textit{ba} is possible. The topic in (\ref{e:arka:53b}) has a stronger contrast (indicated by placing its translation of THIS in capital letters) than its counterpart without \textit{ba} in (\ref{e:arka:53a}), e.g. with additional affirmation in response to the addressee’s question/hesitation. 

The unacceptability of (\ref{e:arka:53c}) provides further evidence that a \isi{contrastive focus} cannot precede a \isi{contrastive topic} in \isi{left periphery} positions. In this example, an attempt is made to make the first NP a \isi{contrastive focus} by a \textit{ba} and \textit{ane} marking.

Still, it is possible to have a fronted focus before a topicalised NP in (Sembiran) \ili{Balinese}. This is the case when the focus is a question word (QW).  This possibility stems from the constraint that a fronted QW must be associated with a \isi{pivot}, an exclusive property of the \isi{pivot} argument in \ili{Balinese} \citep{Arka2003}. Consider a transitive predicate such as \textit{alih} ‘\textsc{av}.search’ in a declarative sentence, as in (\ref{e:arka:54a}). When the A \isi{pivot} is questioned, the QW can appear in situ, as in (\ref{e:arka:54b}). The P object can be topicalised, as in (\ref{e:arka:54c}). The QW can be fronted as well, as in (\ref{e54d}). Note that the fronted QW must be associated with the PIV, which is in this case the A argument (index \textit{j}) because the verb is in the \isi{AV}. While the NP \textit{Men Tiwas} is closer to the verb than the fronted QW, the fronted QW must be associated with the \isi{pivot} (index j; reading [i]), not the object (reading ii).

\largerpage	
\begin{exe}
	\ex\label{e:arka:54}
	\begin{xlist}
		\ex\label{e:arka:54a}
		\gll Men Sugih   ngalih   Men Tiwas.\\
		Men Sugih   \textsc{av}.look.for  Men Tiwas\\
		\glt ‘Men Sugih looked for Men Tiwas.’
		\ex\label{e:arka:54b}
		\gll Nyen   ngalih   Men Tiwas{\USQMark}\\
		who  \textsc{av}.look.for  Men Tiwas\\
		\glt ‘Who looked for Men Tiwas?’
		\ex\label{e:arka:54c}
		\gll {\ob}Men Tiwas{\cb}\textsubscript{Top}\_\textsubscript{i}   {\ob}{\ob}nyen{\cb}\textsubscript{Foc}    {\ob}ngalih   \_\_\textsubscript{i} {\cb}{\cb}\textsubscript{IP}{\USQMark}\\
		\phantom{[}Men Tiwas  \phantom{[[}who  \phantom{[}\textsc{av}.look.for   \\
		\glt ‘(As for) Men Tiwas, who looked for her?’
		\ex\label{e54d}
		\gll {\ob}Nyen{\cb}\textsubscript{Foc\_j{\USSlash}{\USStar}i}  {\ob}Men Tiwas{\cb}\textsubscript{Top}  {\ob}\_\textsubscript{j}  ngalih    \_\_\textsubscript{i}{\cb}{\USQMark}\\
		\phantom{[}who  \phantom{[}Men Tiwas  \phantom{[}\textsc{a}.\textsc{piv}  \textsc{av}.look.for  \textsc{p}.\textsc{object}\\
		\glt
		i)  ‘Who was looking for Men Tiwas?’\\
		ii)  * ‘Whom Men Tiwas was looking for?’
	\end{xlist}
\end{exe}

\noindent
An adjunct can appear as a \isi{frame setter}. It may also carry an emphatic or \isi{contrastive} meaning. Consider the following example in (\ref{e:arka:55}), where the adverb \textit{ditu} ‘there’ in the \isi{left periphery} position is referentially the same as the sentence-final adjunct PP ‘at Butuh Catra’s place’. The contrastive-emphatic meaning resulting from the appearance of \textit{ditu} ‘there’ in the \isi{left periphery} position is captured by the rather long free translation given the locative adjunct in (\ref{e:arka:55}). 

\begin{exe}
	\ex\label{e:arka:55}
	\gll Ba kento   {\ob}ditu   sa{\cb}\textsubscript{Frame\_i}   {\ob}meme{\cb}\textsubscript{S.Top}   bareng   megae {\ob}di   Butuh Catra-ne   ento{\cb}\textsubscript{\_i}.\\
	after like.that  \phantom{[}there   \textsc{part}  \phantom{[}mother  together  \textsc{mid}.work \phantom{[}at  Butuh Catra-\textsc{def}  that\\
	\glt ‘After that, THERE at Butuh Catra’s place (i.e. not at other places), I (mother) work together.’
\end{exe}

\subsection{\label{s6.3}Scope, contrast, and negation}

In this final subsection, we address the issue of the scope of focus/contrast and related complexity due to the interaction of information with negation, \isi{topicalisation} and pragmatic-contextual implication where local socio-cultural information might also be important. We begin with the different sizes focus can apply.  

Units of different sizes can be put into contrast, bearing a \isi{new focus}, from a broad \isi{new focus} covering the whole sentence (even a string of sentences) to a narrow(er) \isi{new focus} involving smaller/lower clausal constituents, such as VPs with their object NPs, just the object NP or oblique PP of a VP or possibly even just the modifier part of the object NP. A (wide) sentence \isi{new focus} (cf. \citealt{Lambrecht1994}) is exemplified by the answer sentence in example (\ref{e:arka:56}) from \ili{English} \citep{Foley1994}.\footnote{Foley distinguishes between focus and new information. He argues that not all the new information is a focus, e.g. a high falling pitch in the answer sentence in (\ref{e:arka:56}) would indicate that the last NP Los Angeles is the focus (i.e. the earthquake happened just in Los Angeles not in other cities), while the whole sentence carries new information. Focus in Foley’s sense here is equivalent to the emphatic/\isi{contrastive focus} type captured by the [+contrast] feature.}  The same sentence in different \isi{discourse} contexts would have a different information structure involving different units of \isi{new focus}. If the context of the dialogue in (\ref{e:arka:56}) already included Los Angeles (LA) as part of the CG information, then LA would not be part of the \isi{new focus} unit.

\begin{exe}
	\ex\label{e:arka:56}
	\begin{xlist}
		\exi{Q:} \label{e:arka:56a} What happened?
		\exi{A:} \label{e:arka:56b} An earthquake just hit Los Angeles.
	\end{xlist}
\end{exe}

\noindent
Of particular interest are the intricacies of the different sizes of unit being focussed and contrasted through negation. Negation is of particular interest because it illustrates the complexity of a semantic-pragmatic-syntax interface wherein there can be a mismatch between scope in semantics and pragmatic information structure. Consider the yes/no question-answer pair in (\ref{e:arka:57}) and its context.

\begin{exe}
	\ex{Context: Men Dora told a story about how she gave her money to somebody but did not get her land certificate, and consequently, she lost her land.}\label{e:arka:57}\\
	\begin{xlist}
		\exi{Question:}\label{e:arka:57a}
		\gll Bakat   tanah-e{\USQMark}  {\USOParen}={\USOParen}\ref{e:arka:15}{\USCParen}{\USCParen}\\
		\textsc{uv}.obtain   land-\textsc{def}\\
		\glt ‘Did you get the land?’
		\exi{Men Dora:}\label{e:arka:57b}
		\gll {\ob}ngara{\cb}\textsubscript{Foc},   {\ob}ngara{\cb}\textsubscript{Foc}   bakat.\\
		\phantom{[}\textsc{neg}  \phantom{[}\textsc{neg}    \textsc{uv}.obtain\\
		\glt‘No, (I) didn't get (it).’
	\end{xlist}
\end{exe}

\noindent
Semantically, the negation is wide in scope, as it negates the whole sentence/proposition; however, in terms of information structure, the new information (i.e. \isi{new focus}) does not cover the whole sentence/proposition. The \isi{new focus} here is the negative choice itself. A yes/no question offers closed alternatives, and in this instance, the ‘no’ option is chosen/true. The information conveyed by the other parts of the sentence is not new. The subject topic ‘the land’ and the A argument (‘you’) are already understood (i.e. part of the CG) and are therefore elided. The predication encoded by the \isi{UV} verb ‘obtain’ is also presupposed information. 

A negator can often result in different scopes, possibly with ambiguity, typically when the negation is of the type of normal sentential negation, as exemplified in (\ref{e:arka:58a}). Note that, even though the negator \textit{ngara} appears in its preverbal position in this sentence, reading (ii) is possible: it does not negate the predicate \textit{mati}, but only the adjunct \textit{ulihan nyai} following the verb. This reflects the narrow scope of the negation.

\begin{exe}
	\ex\label{e:arka:58}
	\begin{xlist}
		\ex\label{e:arka:58a}
		\gll Kucit-e    ngara   mati   ulihan   nyai.\\
		piglet-\textsc{def}  \textsc{neg}  die  because.of  2\textsc{sg.f}\\
		\glt i.   ‘The piglet was not dead because of you.’\\
		\glt ii.   ‘The piglet was dead not because of you.’    
		\ex\label{e:arka:58b}
		\gll Kucit-e    mati   ngara   ulihan   nyai.\\
		piglet-\textsc{def}  \textsc{neg}  die  because.of  2\textsc{sg.f}\\
		\glt * i.   ‘The piglet was not dead because of you.’\\
		\glt ii.   ‘The piglet was dead not because of you.’ 
	\end{xlist}
\end{exe}

\noindent
Sentence (\ref{e:arka:58b}) exemplifies a constituent negation where the negator immediately precedes the PP. The predicate \textit{mati} is not in the negation scope; hence there is no ambiguity.  

\largerpage 
Then, the negated unit together with its negator can appear in the left-periphery position to express a \isi{contrastive focus}, as illustrated by the excerpt from the corpus in (\ref{e:arka:59}). In this example, the context provides the contrasting reason for the piglet's death. However, it should be noted that, even without an explicit contrasting element as in (\ref{e:arka:58b}), the constituent negation presupposes that something else has caused the death of the piglet; hence the negated adjunct is \isi{contrastive focus}.

\begin{exe}
	\ex\label{e:arka:59}
	\gll Ento   kucit   mati   kinnya   behan   sakiten; {\ob}ngara   {\ob}ulihan   nyai{\cb}{\cb}\textsubscript{Foc}   kada   mati.\\
	that   piglet   dead  because   by  sickness \phantom{[}\textsc{neg}  \phantom{[}because.of  2\textsc{sg.f}  make  dead\\
	\glt ‘The piglet died because of its sickness; not because of you, (the thing that) caused its death.’
\end{exe}

\noindent
Fronting may give rise to \isi{topicalisation}, selecting a narrow scope for negation, and the fronted unit appears to behave like a topic. Its status as contrastive topic (or focus) is, however, not immediately clear. 

Consider (\ref{e:arka:60}), where the negator \textit{ngara} appears in its position preverbally, but its semantic scope is narrow, due to the fronting for the object. That is, the predicate \textit{ngelah}, ‘have’ is not within its scope of negation. Based on the context, it is understood that the speaker might have other types of produce, but she had no coconuts—the relevant produce needed for the barter.

\begin{exe}
	\ex\label{e:arka:60}
	\gll Budi   luwas   ke   gunung   ngalih   sela, {\ob}nyuh{\cb}\textsubscript{ContrFoc}  ngara   ngelah.\\
	want   go  to  mountain  \textsc{av}.get  sweet.potato \phantom{[}coconuts  \textsc{neg}  \textsc{av}.have\\
	\glt ‘I was going to go to the mountain to trade for sweet potatoes, but, as for COCONUTS, I didn’t have any.’
\end{exe}

\noindent
The fronted object P \textit{nyuh} to the left-dislocated position is assigned [+contrast]. The question is whether the fronted unit is a \isi{contrastive focus}, as seen in the fronted adjunct in the preceding example in (\ref{e:arka:59}), or a \isi{contrastive topic}. 

We contend that it is neither; it is a \isi{frame setter}. It is not really the focus, as the focus is in fact the narrow negation with respect to possession of ‘coconuts’.  It is not really the topic, as it is not definite, and it is actually new as far as the immediate CG context is concerned; hence [-given]. 

The \isi{referent} of ‘coconut’, however, could be thought of as [+salient], as evidenced by the fronting (a common strategy for expressing some kind of salience). It is also salient as far as the local socio-cultural context is concerned. That is, the fronting should also be understood as the speaker’s intention to express not only the bipolar sense of contrast in relation to the negation of coconuts, but also in relation to the socio-cultural salience of coconuts vs. other items of farm produce for bartering. The two senses of contrast are represented in (\ref{e:arka:61}), with \hyperref[e:arka:61b]{(b)} showing the sociocultural-economic contrast set.

\begin{exe}
	\ex\label{e:arka:61}
	\begin{xlist}
		\ex\label{e:arka:61a}
		Truth value contrast:\\
		\{[‘have no coconuts’]\textsubscript{NegFocus}, [‘having coconuts’]\}\textsubscript{Frame} 
		\ex\label{e:arka:61b}
		Local cultural-economic contrast:\\
		\{[‘no coconuts for barter’]\textsubscript{NegFoc}  [‘other items for barter]\}\textsubscript{Frame}
	\end{xlist}
\end{exe}

\noindent
The data of the types shown above raise an important issue in the analysis of (\isi{contrastive}) topic and \isi{frame setter} in terms of the feature space outlined in (\ref{e:arka:14}). The challenge is how to capture the different degrees of specificity and salience characterising the CG. The CG may simply be specific in the larger context of a particular domain due to our understanding of the world or due to certain local socio-cultural knowledge. Such referential specificity may not render the \isi{referent} of an entity definite (i.e. [+given]) but the \isi{referent} is not totally [-given] either. We propose that such a \isi{referent} bears a weak given property, represented by [±given] in our proposed feature space; meaning it is ‘specific’, neither indefinite nor definite; see the discussion of the relationship between (in)definiteness and specificity \citep{Enç1991}. This issue highlights the gradient nature of the information structure categories. Thus, the other features [contrast] and [salient] can also be thought of as gradient in nature, represented in the same way, [±contrast] and [±salient] respectively. Exploring the precise implication of adding this weak dimension as another value to the analysis of information structure is beyond the scope of the present paper.

\section{\label{s:arka:7}Conclusion and final remarks}

New data on information structure from Sembiran \ili{Balinese}, an endangered conservative mountain dialect of \ili{Balinese}, has been presented. This is the first detailed study on information structure in this language that outlines the ways the pragmatic functions of the topic and focus interact with each other and with the grammatical functions in the grammar of this language. In a descriptive-empirical context, it has been shown that Sembiran \ili{Balinese} employs combined strategies, exploiting the available structural positions (e.g. left/right periphery, parallel clausal structures) and morpho-lexical and syntactic resources in grammar (e.g. \isi{voice} systems and particles) and general local knowledge. Prosody has been identified to play a role in Sembiran \ili{Balinese}, but its precise role in information structure in this language requires further research.

In an analytical context, the novel approach of the analysis presented is the conceptions of the topic and focus as complex notions, decomposed into three semantic-\isi{discourse}/pragmatic features of [+/\textminus salient], [+/\textminus given] and [+/\textminus contrast]. Based on these features, the information structure space in Sembiran \ili{Balinese} was explored. The investigation revealed that the three features of topic and focus interact in complex ways, allowing for different possibilities to characterise different subtypes of the topic and focus in Sembiran \ili{Balinese}, such as default/\isi{primary topic}, \isi{secondary topic}, \isi{contrastive topic}/focus and \isi{new topic}/\isi{new focus}. Throughout the paper, language-specific characterisations and supporting data for these sub-types of topic and focus in Sembiran \ili{Balinese} have been provided. Thus, this study has contributed to the typology of information structure and the framework by which such a typological study can be conducted.

On a theoretical level, the analysis assumes a modular parallel model of grammar, as in LFG (\citealt{Bresnan2015}, among others, \citealt{Dalrymple2001}) and RRG (\citealt{VanValin2005}, \citealt{VanValin1997}). There have been proposals regarding how i-str units can be formally and precisely mapped onto other layers of structures in grammar (\citealt{King1997}, \citealt{Mycock2013}, \citealt{Butt2014}). The comprehensive classification of the \isi{discourse} functions of the topic and focus provided in this paper can be formally utilised in existing frameworks.

One theoretical point worth highlighting is the concept of \isi{prominence} in linguistic theories, and to certain extent, in language typology. Prominence in LFG, for example, has played a key role in the linking/mapping theory to account for cross-linguistic predictability and variations in semantics-syntax interfaces. The basic principle of any theory of linking is harmonious \isi{prominence} matching: most prominent items across layers tend to require being mapped onto each other. Thus, given the widely agreed cross-linguistic generalisation of the \isi{prominence} of A>P in semantic-argument structure and the \isi{pivot}/subject>object in syntactic argument structure, there is a cross-linguistic tendency of A and the Subject to be mapped to each other. In this context, the \isi{prominence} of the i-str space is included based on the proposed conception, as discussed in \sectref{s:arka:3}. The decomposition of the topic and focus into features with values allows for representing the gradient nature of the types of topic and focus thus far identified. Based on the analysis, the presence of the properties (i.e. with + value) contributes to the \isi{prominence}, which results in the gradience shown in (\ref{e:arka:62}).

Some discussion is needed for the gradience of information structure \isi{prominence} captured by (\ref{e:arka:62}). First, the gradience comes with two opposing ends in which the \isi{contrastive topic}/\isi{frame setter} is the most prominent category (with all features having plusses) and new/\isi{completive focus} is the least prominent (with all features having minuses). The plus value should be understood as the presence of the relevant information structure property, possibly with its overt marking. Thus, from the speaker’s perspective, a \isi{contrastive topic}/\isi{frame setter} encodes an information unit singled out as having some kind of importance, which has been amply demonstrated in the previous discussion as having salient or marked structural and prosodic properties; e.g. fronted, stressed and/or marked by particles; see \sectref{s5.5} and \sectref{s6.1}. Structurally, it is high in the phrase structure tree; see (\ref{e:arka:5}). In contrast, the new/\isi{completive focus} is the least prominent, as evidenced from its structural and prosodic properties in Sembiran \ili{Balinese}; e.g. expressed later in the \isi{clause} (formally low in the phrase structure tree) and linked to a less prominent grammatical function (e.g. P as object, rather than the \isi{pivot}).

Second, the information structure \isi{prominence} captured in (\ref{e:arka:62}) should be understood as reflecting the general pattern, with typical structural and prosodic correlates as just mentioned. However, there is no one-to-one correlation, and there may be a case where focus structurally comes first and the \isi{frame setter}/\isi{contrastive topic} comes second. This is when the focus is linked to the \isi{pivot}, which is the most prominent grammatical function, as seen in the case with a fronted QW, exemplified in (\ref{e:arka:54}). This indicates that grammatical \isi{prominence} outweighs information structure \isi{prominence}, at least in (Sembiran) \ili{Balinese}.



\begin{exe}
\ex\label{e:arka:62}
\begin{tabular}[t]{lllllllll}
	\lsptoprule
	& \multicolumn{8}{c}{$\xleftrightarrow{\hspace{5cm}}$}\\
	& 1 & \cellcolor[gray]{0.8}{2} & \cellcolor[gray]{0.8}{3} & \cellcolor[gray]{0.8}{4} & \cellcolor[gray]{0.7}{5} & \cellcolor[gray]{0.7}{6} & \cellcolor[gray]{0.7}{7} & \cellcolor[gray]{0.6}{8}\\
	\raggedleft salience & + & \cellcolor[gray]{0.8}{+} & \cellcolor[gray]{0.8}{+} & \cellcolor[gray]{0.8}{–} & \cellcolor[gray]{0.7}{+} & \cellcolor[gray]{0.7}{–} & \cellcolor[gray]{0.7}{–} & \cellcolor[gray]{0.6}{–}\\
	\raggedleft given & +\textsuperscript{a}/±\textsuperscript{b} & \cellcolor[gray]{0.8}{–} & \cellcolor[gray]{0.8}{+} & \cellcolor[gray]{0.8}{+} & \cellcolor[gray]{0.7}{–} & \cellcolor[gray]{0.7}{+} & \cellcolor[gray]{0.7}{–} & \cellcolor[gray]{0.6}{–}\\
	\raggedleft contrast & + & \cellcolor[gray]{0.8}{+} & \cellcolor[gray]{0.8}{–} & \cellcolor[gray]{0.8}{+} & \cellcolor[gray]{0.7}{–} & \cellcolor[gray]{0.7}{–} & \cellcolor[gray]{0.7}{+} & \cellcolor[gray]{0.6}{–}\\
	\lspbottomrule
\end{tabular}\\
\begin{tabular}{ll}
1=a.ContrTopic; b.FrameSetter & 5=NewTopic  \\
2=ContrFocus (fronted) & 6=SecondTopic\\
3=Default Topic    & 7=ContrNewFoc    \\
4=ContrSecondTopic & 8=New/Completive Focus
\end{tabular}
\end{exe}

\noindent
Third, the \isi{prominence} is gradient in the sense that there are no discrete or clear boundaries in the ranking of the categories in between the two ends, even though a pattern indicating the ranking of two subclasses is observable in (\ref{e:arka:62}): subgroups 2--4 vs. 5--7. Subgroup 2--4 appears to be more prominent than subgroup 5--7. Evidence for this in Sembiran \ili{Balinese}, for example, comes from the positive values of the information structure features correlating with the structural marking properties. Thus, the \isi{default topic} comes structurally higher, before the verb, whereas the \isi{secondary topic} comes later, after the verb.  

However, the \isi{prominence} of member categories within each group is not always clear. For this reason, no boundaries are represented separating them within their own group (\ref{e:arka:62}). The labelling and ordering of the members of the second group, 5--7, are for convenience only. More research is needed to determine how \isi{new topic} (\ref{e:arka:5}), \isi{secondary topic} (\ref{e:arka:6}) and non-fronted (\isi{contrastive}) \isi{new focus} should be ranked with one another.

Finally, another question regarding \isi{prominence} is whether the three features are also ranked against each other. While a definite answer to this question requires further verification and investigation (as the element of contrast can be achieved by means of more than one strategy), it appears that [+contrast] outweighs [+\isi{givenness}]. Evidence for this can be found in the \isi{contrastive} \isi{new focus} in (Sembiran) \ili{Balinese}, which triggers the linking to the \isi{pivot}, as seen in example (\ref{e:arka:23}). That is, when the A is [+given] and P is [+contrast], even though it is new, it triggers the linking to \isi{pivot} and can claim a position earlier or higher (i.e. more prominent) (phrase-) structurally. 

Prominence is a broad and important concept in typological and theoretical linguistics, and this paper has contributed to the empirical basis of this area of research. The notion of contextual CG is central in the information structure analysis, and this paper has also contributed to the empirical basis in this \isi{discourse} pragmatic research by highlighting the significance of the local socio-cultural information in understanding information structure in Sembiran \ili{Balinese}. Languages vary in terms of coding resources, and this paper has contributed to descriptive linguistics, showing how linear order and constituency, \isi{voice system}, and other grammatical-lexical resources interact to convey complex and subtle communicative meanings. More research is required to uncover whether similar patterns and complexities are encountered in the neighbouring languages, and beyond.\largerpage[2]

\section*{Acknowledgements}

We thank our Sembiran \ili{Balinese} consultants for their help with the data. We also thank two anonymous reviewers for their detailed comments and criticism of the earlier version of this paper. We found their feedback very useful for the revision, and the quality of the published paper has been significantly improved by their feedback. All remaining errors are ours.
  
\section*{Abbreviations}

\begin{multicols}{2}
	\begin{tabbing}
		glossgloss \= \kill
		1, 2, 3 \> first, second and \\ \> third person pronouns\\
		\textsc{a}\> actor\\
		\textsc{acc} \> accusative\\
		\textsc{appl} \> applicative\\
		\textsc{art} \> article\\
		\textsc{av} \> \isi{actor voice}\\
		\textsc{caus} \> causative\\
		CG \> Common Ground\\
		\textsc{def} \> definite\\
		Foc \> Focus\\
		\textsc{fut} \> future\\
		h.r. \> high register\\
		\textsc{lig} \> ligature\\
		\textsc{mid} \> middle \isi{voice}\\
		\textsc{neg} \> negator\\
		\textsc{nml} \> nominaliser\\
		\textsc{nom} \> \isi{nominative}\\
		\textsc{p} \> most patient-like argument \\ \> of transitive verbs\\
		\textsc{part} \> particle\\
		\textsc{perf} \> perfect\\
		Piv \> Pivot\\
		\textsc{poss} \> possessive\\
		\textsc{red} \> reduplication\\
		\textsc{rel} \> relativiser\\
		\textsc{s} \> subject of intransitive verb\\
		\textsc{sg} \> singular\\
		Top \> Topic\\
		\textsc{uv} \> \isi{undergoer voice}
	\end{tabbing}
\end{multicols}

\printbibliography[heading=subbibliography,notkeyword=this]
\end{document}
