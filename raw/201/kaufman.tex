\documentclass[output=paper]{langsci/langscibook} 
\title{Austronesian predication and the emergence of biclausal clefts in Indonesian languages} 
\shorttitlerunninghead{Austronesian predication and biclausal clefts in Indonesian languages } 
\author{Daniel Kaufman\affiliation{Queens College \& ELA}}                                         

\ChapterDOI{10.5281/zenodo.1402547}
\abstract{Information structure is tied up closely with predication in predicate-initial \sloppy{Philippine-type} languages. In these languages, subjects are presupposed and the predicate position operates as a default focus position. The present paper argues that there is no need for biclausal focus constructions in these languages due to the nominal properties of their event-denoting predicates. Non-Philippine-type Austronesian languages develop a stronger noun/verb distinction that I argue ultimately gives rise to biclausal focus constructions. The building blocks of biclausal clefts in Indonesian languages are analyzed as well as the nature of predication in Philippine-type languages. Finally, I discuss a paradox in the syntax of definite predicates in Philippine-type languages. In a canonical predication, the less referential portion (the predicate) precedes the more referential portion (the subject). However, when both portions are definite the relation is reversed such that the more referential portion must be initial. I tie this to animacy effects found in other Austronesian languages in which a referent higher on the animacy scale must linearly precede one that is lower.}

\begin{document}

\maketitle

\section{\label{s:kaufman:1}Introduction}

Languages vary widely in their strategies for indicating pragmatic relations such as \textsc{focus} and  \textsc{topic}. In the simplest case, a language may employ dedicated morphological markers which combine directly with focused or topical constituents. More common perhaps is the use of dedicated syntactic positions, typically on the \isi{left periphery} of the \isi{clause}, which host focused or topicalized constituents. Finally, all languages are thought to express basic information structure via prosodic means, although the actual implementation differs significantly from language to language. Parallel to pragmatic relations such as \textsc{topic} and \textsc{focus}, all languages possess a basic \textsc{subject-predicate} relation which is partly independent of pragmatics but which also intersects with the phenomena of topic, focus and \isi{presupposition}. While there has been notable success in the definition and analysis of pragmatic relations over the last several decades, the true nature of the subject-predicate relation remains one of the most fraught topics in the history of linguistics, with its beginnings in the work of Aristotle and Plato. Indeed, as \citet[83]{Davidson:2005} states with regard to predication, ``It is a mark of Plato’s extraordinary philosophical power that he introduced a problem that remained unresolved for more than two millennia." As could be surmised solely from the disagreement among syntactic analyses of \ili{English}, a robust \emph{cross-linguistic} definition of predicate and subject remains even more elusive. Copular clauses and cleft structures are of special interest here as notional predicates can occupy the canonical syntactic position of the subject (and vice versa) in these sentence types. In the present work, I am concerned with the interplay between the subject-predicate relation and information structure in \ili{Austronesian} languages. Specifically, I would like to answer the following three questions: 

\begin{itemize}
\item[(i)] What is the evidence for biclausal cleft structures in Philippine-type\footnote{``Philippine-type languages" refer to a typological grouping rather than a geographic or phylogenetic one. It is used here to refer to \ili{Austronesian} languages with a historically conservative set of (3 or 4) voices (following \citealt{Blust:2002b}). Crucially, in Philippine-type languages, these voices are symmetrical in that a predicate can only bear one \isi{voice} at a time and the agent argument is not demoted in the non-actor voices, as it would be in a canonical passive.} versus \ili{Indonesian} languages? (\sectref{s:kaufman:3})
\item[(ii)] How and why do biclausal cleft structures come into being outside of Philippine-type languages? (\sectref{s:kaufman:4})
\item[(iii)] What does it mean to be a predicate in Philippine-type languages? (\sectref{s:kaufman:5})
\end{itemize}

\noindent
We can briefly preview the answers put forth below. With regard to (i), I argue that a genuine cleft structure in \ili{Indonesian} languages emerges from a more symmetric Philippine-type system where true biclausal clefts do not exist. In regard to (ii), I show how a distinction between plain modification and modification by \isi{relative clause} develops in \ili{Indonesian} languages and how old functional morphology is recruited to mark relative clauses. Finally, regarding (iii), I argue that the predicate-subject relation in Philippine-type languages is determined by the \emph{relative referentiality} of the two basic parts of a proposition similar to copular clauses in more familiar languages. The more referential half of the predication (i.e. the subject) follows the less referential half (i.e. the predicate). An interesting complication is that when both the predicate and the subject are referential, the part of the predication higher on the \isi{referentiality}/animateness scale precedes the one lower on the scale. That is, the principle which derives the normal predicate-initial order in Philippine-type languages appears to be reversed in these cases.

Languages that sit on the border of the Philippine-type and non-Philippine-type are especially interesting in regard to information structure. In \sectref{s:kaufman:4}, I examine \ili{Balantak} as a language that appears to be transitioning from monoclausal to \isi{biclausal focus} constructions. This in turn sheds light on the development of biclausal constructions in languages that have diverged even further from the Philippine-type, such as \ili{Malay}/\ili{Indonesian}. 

In \sectref{s:kaufman:2}, I attempt to define all the relevant categories in terms that are as theory-neutral as possible. The relevant notions for our purposes are subject and predicate (\sectref{s:kaufman:2.1}), pragmatic relations such as topic, focus and \isi{presupposition} (\sectref{s:kaufman:2.2}), and the various types of clefts together with their component parts (\sectref{s:kaufman:2.3}).

\section{\label{s:kaufman:2}Defining the terms}
\subsection{\label{s:kaufman:2.1}Subject and predicate}
Several streams in philosophy of language, semantics and even formal syntax have taken a purely taxonomic approach to the notion of predicate with the goal of seeking a unifying trait in these types. The philosophical literature, in particular, is replete with claims such as ``predicates ascribe" and ``predicates designate". One of Frege's most important contributions to our current understanding of predicate involved viewing it as an element with \emph{unsaturated} arguments; in his words, ``not all parts of a thought can be complete; at least one must be unsaturated or predicative; otherwise they would not hold together" \citep[193]{Frege:1892}.\footnote{As a reviewer notes, an important aspect of Frege's contribution was to assimilate all types of predication to verbal predication. In Fregean semantics, predicates can be defined simply as categories with unsaturated terms.} For Frege, linguistic elements such as names and definite descriptions could not be classified as predicates as they cannot be naturally thought of as having unsaturated arguments in the way that ``runs" has a single unsaturated argument and ``punches" has two unsaturated arguments. But the fact that languages routinely place definite descriptions, names and even pronouns in the predicate position of clauses that bear all the morphosyntactic hallmarks of canonical predication poses an immediate empirical challenge to Frege's view of predicates as a natural class of \emph{linguistic} elements.\footnote{As noted by \citet{Modrak:1985}, among others, philosophers have chiefly attempted to explain metaphysical predication, which only bears an incidental relation to linguistic predication. Patterns of linguistic predication across human language have thus not played a major role in philosophical investigations.} Under the direct or indirect influence of Frege, a large body of work in syntax has treated such sentences as something other than pure predications. This has led, for instance, to a taxonomic tradition in the study of copular sentences \citep{Mikkelsen:2011a}, in which copular clauses can come in specificational, equational and identificational flavors which largely correlate to the \isi{referentiality} of the ``predicate". 

For present purposes, the notions \textsc{subject} and \textsc{predicate} can be defined following type-theoretic predicate logic. Namely, the subject and predicate are the two constituents that combine to yield a \textsc{truth value}. It is not always a trivial matter to determine what types of strings have a truth value and which do not. In the simple case, we can compare the modification relation in (\ref{e:kaufman:1}) with the predication in (\ref{e:kaufman:2}).

\begin{exe}
	\ex\label{e:kaufman:1}{Tagalog}\\
	\gll ang matangkad na dalaga\\
	\textsc{nom} tall \textsc{lnk} girl\\
	\glt `the tall girl'
\end{exe}

\begin{exe}
	\ex\label{e:kaufman:2}{Tagalog}\\
	\gll Matangkad ang dalaga.\\
	tall \textsc{nom} girl\\
	\glt `The girl is tall.'
\end{exe}

\noindent
\ili{Tagalog} speakers understand (\ref{e:kaufman:1}) as having a potential reference in the world but not a truth value, whereas the opposite intuition holds for (\ref{e:kaufman:2}) (which nonetheless contains the referring expression \textit{ang dalaga} `the girl'). In (\ref{e:kaufman:1}), the entire string consists of a single Determiner Phrase (DP) marked with the \isi{nominative} case determiner \textit{ang}. In (\ref{e:kaufman:2}), the string contains two major phrases, a predicate, followed by the \isi{nominative} marked DP.

\citet{Himmelmann:1986} takes the key feature of predications to be ``challengeability": ``A predicative structure always allows for -- or even demands -- a yes-no reaction" \citep[26]{Himmelmann:1986}. This view, correctly, I believe, draws a strong line between predication on the one hand and modification, secondary predication and even subordinate predication on the other hand, a distinction which not all theories abide by. I also agree here with Himmelmann in understanding predicates to be crucially a relational concept rather than an inherent property of certain types of linguistic elements.\footnote{I prefer though to rely on \emph{the potential for a truth value} rather than the notion of challengeability as the latter cannot cleanly apply to imperatives and interrogatives. While some views on questions and imperatives take them to lack truth values, questions and imperatives seem to me best understood as other (non-assertive) things we do with truth values. Declaratives assert that a proposition is true or false; yes-no questions request the hearer to posit a true or false judgement on the proposition; content questions request a value to make a proposition with a variable true; imperatives demand that the addressee make the proposition true. None of these acts could be executed if the proposition itself had no truth value. Thus, just as \textit{Dog} is not a predication, neither is \textit{Dog?} a possible yes-no question, nor \textit{Dog!} a command. I believe these facts can be unified in any approach that sees speech acts as \emph{operations on truth values} rather than restricting truth values to assertions.}

A particularly vexed question in Philippine linguistics regards which of the two arguments of a basic transitive \isi{clause} should be considered the ``subject", with all possible answers having been posited by different analysts (including ``none of the above", see \citealt{Schachter:1976}). In (\ref{e:kaufman:3}), we see three variations on a \isi{patient voice} \isi{clause} and in (\ref{e:kaufman:4}) we see the same kind of variations in an \isi{actor voice} \isi{clause}.\footnote{In the non-actor voices, the agent is expressed in the genitive case, treated by some as ergative case, while the argument selected by the \isi{voice} morphology is expressed in the \isi{nominative}/absolutive case.} Following a type-theoretic approach, we can see that there is an important difference between the (b) and (c) sentences. In an out-of-the-blue setting, (\ref{e:kaufman:3b}) and (\ref{e:kaufman:4c}) are judged to have truth values but (\ref{e:kaufman:3c}) and (\ref{e:kaufman:4b}) are not. The latter two sentences are not ungrammatical, but they must depend on the preceding \isi{discourse} to obtain a truth value. That is, as long as anyone ate the tofu, (\ref{e:kaufman:3b}) will be judged true but (\ref{e:kaufman:3c}) cannot be judged as true or false even if we know that Juan ate something. Similarly, for just anyone to have eaten tofu does not make the \isi{actor voice} sentence in (\ref{e:kaufman:4b}) true. In order for it to be evaluated as true or false, the preceding \isi{discourse} has to provide the reference for the elided \isi{nominative} argument. 

\newpage 
\begin{exe}
	\ex\label{e:kaufman:3}
	\begin{xlist}
		\exi{\textsc{Tagalog}}
		\ex\label{e:kaufman:3a}
        \gll K{\USSmaller}in{\USGreater}áin-∅ ni Juan ang tokwa.\\
		<\textsc{beg}>eat-\textsc{pv} \textsc{gen} Juan \textsc{nom} tofu\\
		\glt `Juan ate the tofu.'
		\ex\label{e:kaufman:3b}
        \gll K{\USSmaller}in{\USGreater}áin-∅ ang tokwa.\\
		<\textsc{beg}>eat-\textsc{pv} \textsc{nom} tofu\\
		\glt `The tofu was eaten.'
		\ex\label{e:kaufman:3c}
        \gll \%K{\USSmaller}in{\USGreater}áin-∅ ni Juan.\\
		<\textsc{beg}>eat-\textsc{pv} \textsc{gen} Juan\\
		\glt `Juan ate (it).'
	\end{xlist}
\end{exe}

\begin{exe}
	\ex\label{e:kaufman:4}
	\begin{xlist}
		\exi{\textsc{Tagalog}}
		\ex\label{e:kaufman:4a}
        \gll K{\USSmaller}um{\USGreater}áin ng tokwa si Juan.\\
		<\textsc{av.beg}>eat \textsc{gen} tofu \textsc{nom} Juan\\
		\glt `Juan ate tofu.'
		\ex\label{e:kaufman:4b}
        \gll \%K{\USSmaller}um{\USGreater}áin ng tokwa.\\
		<\textsc{av.beg}>eat \textsc{gen} tofu\\
		\glt `(S/he) ate tofu.'
		\ex\label{e:kaufman:4c}
        \gll K{\USSmaller}um{\USGreater}áin si Juan.\\
		<\textsc{av.beg}>eat \textsc{nom} Juan\\
		\glt `Juan ate.'
	\end{xlist}
\end{exe}

\noindent
The key generalization then is that a predicate must combine with a \isi{nominative}/absolutive (in \ili{Tagalog}, \textit{ang} marked) argument to obtain a truth value. On this basis, we can refer to the \textit{ang} phrase as the subject and the clause-initial phrases in the above examples as the predicate. Precisely parallel facts have been observed for several \ili{Polynesian} languages, such as East Futunan and \ili{Tongan} \citep{Dukes:1998, Tchekhoff:1981, Biggs:1974}. \citet{Dukes:1998} sums up the \ili{Tongan} situation in a way that describes Philippine-type languages equally well:

\begin{quote}\small An omitted ergative argument need not presuppose any particular \isi{referent} in the \isi{discourse} and may in fact be interpreted existentially in much the same way an omitted agent in an \ili{English} passive is interpreted. When an absolutive is omitted however, it must be interpreted referentially as a null \isi{pronoun} picking out some previously mentioned individual. As Biggs puts it, native speakers of these languages consider sentences which are missing an absolutive to be `incomplete', whereas sentences missing ergatives are not.\end{quote}

\noindent
Note that this definition of subject is completely independent from the ``subject" as identified by classic syntactic diagnostics like those posited by \citet{Keenan:1976}, e.g.\ binding relations, raising and control, many of which converge on the more agentive argument of a transitive \isi{clause}, as in \ili{English}, and only some of which seem to pick out the \textit{ang} phrase.\footnote{The \textit{ang} phrase relation of \ili{Tagalog} which I refer to here as subject has, in fact, been given so many names over the years to distinguish it from the subject of a hierarchical argument structure that it is hard to keep track. Among others, we find, ``predicate base", ``\isi{pivot}", ``topic", and the neutral \textit{ang} phrase. (See \citealt{Blust:2002b}, \citealt{Ross:2006}, \citealt{Kroeger:2007} and \citealt{Blust:2015} for good summaries of the terminological and conceptual challenges presented by \ili{Austronesian} \isi{voice} and case.) I maintain the term subject here because of the familiarity of the subject-predicate relation, which is specifically relevant here. Moreover, the hierarchical relations between co-arguments of a predicate (e.g.\ the relation between subjects and objects) is not relevant for our purposes and so we can avoid the usual confusion. These are, however, very different relations that should be kept separate terminologically and theoretically, as for instance in \citet{Foley:1984}.} The concept of the subject-predicate relation as posited above is inherently symmetrical; the subject and predicate are simply the two (topmost) constituents which are combined to yield a truth value. However, few if any human languages treat these constituents symmetrically. There are clearly distinct positions for subject and predicate in the vast majority of described languages in the world.\footnote{Diverging from most generative syntacticians, \citet{Dikken:2006a} does argue for a symmetrical view of predication but applies the term very broadly to include phenomena that are generally analyzed as modification. \posscitet{Heycock:2013} overview of predication in generative syntax shows how far this line of work has diverged from the traditional Aristotelian concept of a predicate as part of a bipartite proposition that yields a truth value.} Certain types of copular clauses, however, are apparently reversible in many languages but with subtly different entailments. \citet{Jespersen:1937, Jespersen:1965} notes the distinct meanings of `my brother' in the following passage:

\begin{quote}\small
	\glt Now, take the two sentences: 
	\begin{quote}
	\textit{My brother} was captain of the vessel, and\\
	\textit{The captain} of the vessel was my brother.
	\end{quote}
	In the former the words \textit{my brother} are more definite (my only brother, or the brother whom we were just talking about) than in the second (one of my brothers, or leaving the question open whether I have more than one). \citep[:153]{Jespersen:1965}
\end{quote}

\noindent
Based on a family of similar observations, Jespersen develops the idea that choice of subject and predicate is based on relative familiarity. We can therefore conceptualize predication as an inherently \emph{symmetrical} relation but one whose syntactic expression is highly sensitive to \isi{referentiality}. That is, the more referential of the two elements in a predication relation will be mapped to a position which we can, following tradition, refer to as ``subject" and the less referential of the two will be mapped to a position we call the ``predicate".\footnote{This is also very much in line with the view of predication developed by \citet[331]{Williams:1997a}, who treats referential NPs as potential predicates: ``It is sometimes thought that a predicate cannot be `referential'. It seems to me though that the best we can say is that the predicate is less ``referential" than its subject, and that what we really mean is something having to do with directness of acquaintance." Concomitantly, for \citet[323]{Williams:1997a}, ``A phrase is not predicational in any absolute sense, but only in relation to its subject."} In \ili{English}, a subject initial language, there is only one way to make a predication between \textit{Mary} and \textit{a linguist}, that is, by treating \textit{Mary} as the subject and mapping \textit{a linguist} to the predicate, as in (\ref{e:kaufman:5}).  We say that Philippine-type languages are ``predicate-initial" because the less referential component of the predication relation is clause-initial while the more referential component follows it, as shown in (\ref{e:kaufman:6}). Just as in \ili{English}, the relative \isi{referentiality} of the two parts of the predication determines their position in the \isi{clause}. Reversing the order, as in (\ref{e:kaufman:6b}), results in ungrammaticality.


\begin{exe}
\ex\label{e:kaufman:5}
	\begin{xlist}
	\exi{English}
	\ex\label{e:kaufman:5a} Mary is a linguist.
	\ex\label{e:kaufman:5b} {*A linguist is Mary.}
\end{xlist}
\end{exe}

\begin{exe}
\ex\label{e:kaufman:6}
	\begin{xlist}
	\exi{Tagalog}
	\ex\label{e:kaufman:6a}
    \gll Abogado si Jojo.\\
	lawyer \textsc{nom} Jojo\\
	\glt `Jojo is a lawyer.'
	\ex\label{e:kaufman:6b}
    \gll {\USStar}Si Jojo abogado.\\
	\phantom{*}\textsc{nom} Jojo lawyer\\
\end{xlist}
\end{exe}

\noindent
I would thus like to take a Jespersonian approach here, which does not rely on purported universal properties of subjects (e.g.\ ``referring") or predicates (e.g.\ ``unsaturated"). Having established this still informal, but hopefully workable, concept of predication, we turn to the impressive flexibility of Philippine languages in mapping lexical categories to the predicate and subject positions, as exemplified in (\ref{e:kaufman:7}). This was noted by \citet{Bloomfield:1917} for \ili{Tagalog} and discussed extensively in the subsequent literature \citep{Gil:1993, Himmelmann:1987, Himmelmann:1991, Foley:2008, Schachter:1982, Kaufman:2009cons}.

\begin{exe}
	\ex\label{e:kaufman:7}
	\begin{xlist}
		\exi{Tagalog}
		\ex\label{e:kaufman:7a}
        \gll K{\USSmaller}um{\USGreater}a∼káin ang laláki.\\
		\textsc{<av>imprf}∼eat  \textsc{nom} man\\
		\glt `The man is eating.’
		\ex\label{e:kaufman:7b}
        \gll Laláki ang k{\USSmaller}um{\USGreater}a∼káin.\\
		man \textsc{nom} \textsc{<av>imprf}∼eat\\
		\glt `The eating one is a man.’
	\end{xlist}
\end{exe}

\noindent
In a very simplistic schema, we can conceive of \ili{English} and \ili{Tagalog} differing as in (\ref{e:kaufman:8}) and (\ref{e:kaufman:9}). Whereas the \ili{English} clausal spine makes crucial reference to lexical categories such as NP and VP, Philippine-type phrase structure seems to refer primarily to the functional categories PredP and SubjP which can in turn host phrases of any lexical category (XP and YP below).\footnote{Proponents of this view, in one form or another, include \citet{Scheerer:1924, Lopez:1937, Capell:1964, Starosta:1982, Egerod:1988, DeWolf:1988, Himmelmann:1991, Lemarechal:1991, Naylor:1975, Naylor:1980, Naylor:1995, Kaufman:2009cons}. \citet{Byma:1986}, on the other hand, and most subsequent syntacticians \citep{Richards:2000, Rackowski:2002, Aldridge:2004}, have defended (or assumed) an analysis in which \ili{Tagalog} is also structured like (\ref{e:kaufman:8}). \citet{Richards:2009, Richards:2009a} explicitly argues that all predications in \ili{Tagalog} are mediated by a verbal element but that this element is null in most copular clauses. For reasons of space, I refer the reader to \citet{Kaufman:2015} for a rebuttal of this view.}

%\noindent\parbox[t]{0.5\textwidth}
\begin{exe}
\ex \label{e:kaufman:8}
\textsc{\ili{English}-type phrase structure}\\\Tree [.S NP\\(Subj) VP\\(Pred) ]  
\end{exe}

%\parbox[t]{0.5\textwidth}
\begin{exe}
\ex \label{e:kaufman:9}
\textsc{Philippine-type phrase structure}\\\Tree [.S [.PredP XP ] [.SubjP \textsc{nom} YP ] ] 
\end{exe}

\noindent
In order to understand how phrases are mapped to the predicate and subject position in Philippine-type languages we must first define the crucial pragmatic concepts that come into play. We turn to this in the following subsection. 

\subsection{\label{s:kaufman:2.2}Pragmatic relations}

Presupposition and focus are often described in shorthand as old information and new information, respectively. While this evokes the right idea, \citet{Lambrecht1994} argues against such oversimplified definitions. I follow Lambrecht's definitions for the relevant categories, as given below.\footnote{\citet{Abbott:2000} offers an alternative view of presuppositions as \emph{non-assertions} rather than information known to the hearer. This may well fit the \ili{Austronesian} data better but I leave this question to further work. A good overview of the issues and literature surrounding presuppositions is found in \citet{Kadmon:2000}.}

\begin{exe}
	\ex
	 \label{e:kaufman:10}
	\begin{xlist}
		\exi{\textbf{Presupposition, \isi{assertion}, focus and topic}} \citep[52, 213, 131]{Lambrecht1994}
		\ex\label{e:kaufman:10a} \textsc{\isi{pragmatic presupposition}:} The set of propositions \sloppy{lexico-grammatically} evoked in a sentence which the speaker assumes the hearer already knows or is ready to take for granted \emph{at the time the sentence is uttered}.
		\ex\label{e:kaufman:10b} {\textsc{\isi{pragmatic assertion}:} The proposition expressed by a sentence which the hearer is expected to know or take for granted \emph{as a result of hearing the sentence uttered}.}
		\ex\label{e:kaufman:10c} {\textsc{focus:} The semantic component of a pragmatically structured proposition whereby the \isi{assertion} differs from the \isi{presupposition}.}
		\ex\label{e10d} {\textsc{topic:} A \isi{referent} is interpreted as the topic of a proposition if in a given situation the proposition is construed as being about this \isi{referent}, i.e.\ as expressing information which is relevant to and which increases the addressee's knowledge of this \isi{referent}. }
	\end{xlist}
\end{exe}

\noindent
While the \ili{Tagalog} \textit{ang} phrase is often referred to as ``topic" in different analytic traditions, it has been shown clearly by \citet{Naylor:1975}, \citet{Kroeger:1993} and \citet{Kaufman:2005} to have no inherent pragmatic status beyond its definiteness or \isi{referentiality}. \ili{Tagalog} and, it would seem, all other Philippine languages have a bona fide topic position on the \isi{left periphery}. In \ili{Tagalog}, the fronted pragmatic topic, is followed either by the topic marker \textit{ay} or a pause. In \ili{Tagalog}, but not all Philippine languages, there is also a dedicated \isi{focus position} on the \isi{left periphery} of the \isi{clause} which hosts oblique phrases, exemplified in (\ref{e:kaufman:11}). 

\begin{exe}
	\ex\label{e:kaufman:11}{Tagalog}\\
	\gll {\ob}Sa Manila{\cb}\textsubscript{FOC}=ka=ba p{\USSmaller}um{\USGreater}unta{\USQMark}\\
	\phantom{[}\textsc{obl} Manila=\textsc{2s.nom=qm} <\textsc{av.prf}>go\\
	\glt `Did you go to \textsc{Manila}?' (`Was it to Manila that you went?') \citep[182]{Kaufman:2005}
\end{exe}

\noindent
The presence of an oblique phrase in the \isi{focus position} in the \isi{left periphery} bifurcates the sentence into a focus and a \isi{presupposition}. In the above, `to Manila' is the focus and it is presupposed that the addressee had gone somewhere. The question would be inappropriate if this information was not already part of the \isi{discourse} in the same way that `Was it to Manila that you went?' would be inappropriate in an out-of-the-blue context. On the other hand, a phrase in the left peripheral topic position followed either by the topic marker or a pause, needs to either be in the \isi{discourse} already or contrasted with a similar argument that belongs to the same set. In (\ref{e:kaufman:12}), the fronted oblique phrase can serve as a \isi{contrastive topic}, pragmatically akin to \ili{English}, `What about Naga, have you gone there?'. Note that the topic is further outside the \isi{clause} and thus does not host second position clitics. Furthermore, it does not trigger a \isi{presupposition}. The question in (\ref{e:kaufman:12}) is still felicitous without it being in the \isi{discourse} that the addressee went somewhere.

\begin{exe}
	\ex\label{e:kaufman:12}{Tagalog}\\
	\gll {\ob}Sa Naga{\cb}, p{\USSmaller}um{\USGreater}unta=ka=ba{\USQMark}\\
	\phantom{[}\textsc{obl} Naga <\textsc{av.prf}>go=\textsc{2s.nom=qm}\\
	\glt `To Naga, did you go (there) already?'
\end{exe}

\noindent
With this brief introduction we are now prepared to turn to the mapping of these pragmatic categories on to phrase structure across several \ili{Austronesian} languages. 

\subsection{\label{s:kaufman:2.3}Clefts}

Cleft structures make use of the subject-predicate relation to satisfy requirements of information structure. Subjects are canonically (but not necessarily) topic-like and predicates canonically (but not necessarily) align with the focused constituent of a \isi{clause}. Thus, focused subjects and presupposed predicates constitute non-canonical alignments which languages may either tolerate or avoid by means of various syntactic mechanisms. Languages with a high tolerance for focused subjects, such as \ili{English}, will allow such a subject to be merely marked by \isi{intonation}, as in (\ref{e:kaufman:13a}). However, an option also exists to shift such a focused subject into the predicate position, as in the \textit{it}-cleft sentence in (\ref{e:kaufman:13b}) \citep{Lambrecht1994}. 

\begin{exe}
	\ex\label{e:kaufman:13}
	\begin{xlist}
		\ex\label{e:kaufman:13a} Only {[John]}\textsubscript{FOC} knows Jane.
		\ex\label{e:kaufman:13b} It's only {[John]}\textsubscript{FOC} who knows Jane.
	\end{xlist}
\end{exe}

\noindent
At the same time, the \isi{presupposition} of an \ili{English} \textit{it}-cleft is demarcated syntactically by means of a \isi{relative clause}. Thus, while both (\ref{e:kaufman:13a}) and (\ref{e:kaufman:13b}) contain a \isi{presupposition} `X knows Jane', its pragmatic status is only reflected syntactically in (\ref{e:kaufman:13b}), by means of the \isi{relative clause} \textit{who knows Jane}. The \ili{English} \textit{it}-cleft can thus be said to create a more transparent mapping between the syntactic and pragmatic structure of the \isi{clause}.

Other languages do not tolerate non-canonical mappings such as that in (\ref{e:kaufman:13a}). For instance, the \ili{Malay}/\ili{Indonesian} adverb \textit{saja} `only', which must combine with a focused constituent preceding it, cannot associate with a subject in a simple declarative \isi{clause}, as shown in (\ref{e:kaufman:14a}). Instead, a cleft structure is required in which the presupposed predicate is packaged as a \isi{relative clause}, as shown in (\ref{e:kaufman:14b}).  

\begin{exe}
	\ex\label{e:kaufman:14}
	\begin{xlist}
		\exi{Indonesian}
		\ex\label{e:kaufman:14a}
        \gll Presiden {\USOParen}{\USStar}saja{\USCParen} bisa menilai kinerja menteri.\\
		president \phantom{(*}only can \textsc{av:}evaluate output minister\\
		\glt `A president can evaluate a minister's output.'
		\ex\label{e:kaufman:14b}
        \gll Presiden saja yang bisa menilai kinerja menteri.\\
		president only \textsc{relt} can \textsc{av:}evaluate output minister\\
		\glt `Only a president can evaluate a minister's output.'
	\end{xlist}
\end{exe}

\noindent
As seen in (\ref{e:kaufman:15}), no special manipulation is required in order to narrowly focus the predicate or a part thereof, as this respects the canonical mapping between predicate and focus.

\begin{exe}
	\ex\label{e:kaufman:15}{Indonesian}\\
	\gll Presiden bisa menilai kinerja menteri saja.\\
	president can \textsc{av:}evaluate output minister only\\
	\glt `A president can only evaluate a minister's output.'
\end{exe}

\noindent
Most interestingly, we find that the \ili{Austronesian} tendency to express presuppositions syntactically manifests itself in Philippine \ili{English}, as well. Whereas \ili{English} can employ prosodic focus alone in a sentence like (\ref{e:kaufman:16}), Philippine \ili{English} will invariably employ a TH-cleft (to be introduced below) in the same function, as seen in (\ref{e:kaufman:16}) and (\ref{e:kaufman:17}). The \ili{Tagalog} equivalent is given in (\ref{e:kaufman:18}).

%\noindent\parbox{3.2in}{
\begin{exe}
	\ex\label{e:kaufman:16}{\textsc{US English}}\\
	\textsc{John} will carry your bag.
\end{exe}

%}\parbox{3.5in}{
\begin{exe}
	\ex\label{e:kaufman:17}{\textsc{Philippine English}}\\
	John will be the one to carry your bag.
\end{exe}

\begin{exe}
	\ex\label{e:kaufman:18}{Tagalog}\\
	\gll Si Juan \textbf{ang} mag-da∼dala ng bag mo.\\
	\textsc{nom} Juan \textsc{nom} \textsc{av-imprf∼}carry \textsc{gen} bag \textsc{2sg.gen}\\
	\glt `\textsc{Juan} will carry your bag.' (Lit. `Juan will be the one to carry your bag.')
\end{exe}

\noindent
Clefts thus function to transparently bifurcate the sentence into a focus and a \isi{presupposition}. As seen in the difference between \ili{English} and \ili{Indonesian} above, languages differ as to the extent to which they require such transparency. In terms of the syntactic hallmarks of cleft sentences,  \citet{Lambrecht:2001} offers the following definition:

\begin{exe}
	\ex\label{e:kaufman:19} {\textsc{cleft construction} \citep[467]{Lambrecht:2001}} \\
	A \isi{cleft construction} is a complex sentence structure consisting of a matrix \isi{clause} headed by a copula and a relative or relative-like \isi{clause} whose relativized argument is coindexed with the predicative argument of the copula. Taken together, the matrix and the relative express a logically simple proposition, which can also be expressed in the form of a single \isi{clause} without a change in truth conditions.
\end{exe}

\noindent
There are two components in (\ref{e:kaufman:19}). The structural component defines clefts as a biclausal structure containing a \isi{relative clause} in a larger copular sentence. The semantic component of the definition relates clefts to simpler monoclausal sentences by virtue of their similar meaning. The cleft and the monoclausal structure differ in information structure and implicature but not in their basic truth conditions. 

Lambrecht advocates for a taxonomy of \ili{English} clefts as in (\ref{e:kaufman:20}), where caps indicates focus:

% \begin{exe}
% \ex\label{e:kaufman:20} \textsc{cleft types} \citep[467]{Lambrecht:2001}\\
% \begin{tabular}[t]{lll}
% 	a.\label{e:kaufman:20a} &I like CHAMPAGNE.& Canonical sentence\\
% 	b.\label{e:kaufman:20b} &It is CHAMPAGNE (that) I like.& IT cleft\\
% 	c.\label{e:kaufman:20c} &What I like is CHAMPAGNE.& WH cleft (pseudo-cleft)\\
% 	d.\label{e20d} &CHAMPAGNE is what I like.& Reverse WH cleft\\
% 	&&  (reverse/inverted pseudo-cleft)
% \end{tabular}
% \end{exe}

\begin{exe}
	\ex\label{e:kaufman:20} \textsc{cleft types} \citep[467]{Lambrecht:2001}
	\begin{xlist}
		\ex\label{e:kaufman:20a} 
		\begin{tabular}[t]{L{4.5cm}l}
		I like CHAMPAGNE.& Canonical sentence
		\end{tabular}
		\ex\label{e:kaufman:20b} 
		\begin{tabular}[t]{L{4.5cm}l}
		It is CHAMPAGNE (that) I like.& IT cleft
		\end{tabular}
		\ex\label{e:kaufman:20c} 
		\begin{tabular}[t]{L{4.5cm}l}
		What I like is CHAMPAGNE.& WH cleft (pseudo-cleft)
		\end{tabular}
		\ex\label{e:kaufman:20d} 
		\begin{tabular}[t]{L{4.5cm}l}
		CHAMPAGNE is what I like.& Reverse WH cleft\\
		&(reverse/inverted pseudo-cleft)
		\end{tabular}
	\end{xlist}
\end{exe}

\noindent
In (\ref{e:kaufman:20a}) we find the canonical monoclausal sentence which is roughly equivalent in its truth conditions to the following clefts. The \textit{it}-cleft places the focused phrase in the predicate position of a copular \isi{clause} in which a dummy \isi{pronoun} is the subject. The \isi{presupposition} is packaged as a \isi{relative clause}.\footnote{The precise relation between the two clauses in the \ili{English} \textit{it}-cleft has been subject to rather intense scrutiny, summed up recently by \citet{Reeve:2010}.} The types in (\ref{e:kaufman:20c}) and (\ref{e:kaufman:20d}) are termed alternatively \textit{wh}-clefts or pseudo-clefts (although I will use only the latter term in the following). Here, there is no dummy subject. A \isi{relative clause} headed by an \isi{interrogative} \isi{pronoun} like `what' is in a copular construction with a focused phrase. In \ili{English}, this type is reversible so that the \isi{presupposition} can be the subject of the matrix \isi{clause}, as in the standard pseudo-cleft exemplified in (\ref{e:kaufman:20c}), or the predicate of the matrix \isi{clause}, as in the inverse pseudo-cleft exemplified in (\ref{e:kaufman:20d}). To Lambrecht's taxonomy, we can add the ``TH-cleft", in (\ref{e:kaufman:21}), first identified as a separate type by \citet{Ball:1977a}. Here, the \isi{relative clause} modifies a definitely determined semantically bleached noun phrase, e.g. `the one', `the thing', etc. 

\begin{exe}
	\ex\label{e:kaufman:21}
\begin{tabular}[t]{lll}
	a. & The one/thing I like is CHAMPAGNE.& TH-cleft\\
	b. & CHAMPAGNE is the one/thing I like.& Reverse/inverse TH-cleft
\end{tabular}
\end{exe}

\noindent
The syntax of clefts accommodates information structure in several ways. In structures like the \ili{English} \textit{it}-cleft (\textit{It's \textsc{John} who bit me}), part of the focus semantics derives from mapping a phrase to the object position of a copular structure, a more hospitable position for focused material than the subject position. In all types of cleft sentences, the \isi{presupposition} is clearly demarcated by packaging it as a \isi{relative clause} of some type. 

It is clearly not the case that relative clauses always contain a \isi{presupposition} outside of cleft sentences. For instance, the \isi{relative clause} subject in (\ref{e:kaufman:22}) does not presuppose that someone will come after closing time. The sentence in (\ref{e:kaufman:23}), however, does entail such a \isi{presupposition}, and this shows that it is the determiner or \isi{demonstrative} that gives rise to the \isi{presupposition} rather than anything inherent to the \isi{relative clause} itself.\footnote{See \citet{Kroeger:2009} for a similar point with regard to mistaken assumptions about headless relative clauses. As Kroeger shows for \ili{Tagalog}, the presuppositions in such constructions are triggered by the determiners rather than the relative structure itself. Note that \citet{Kroeger:1993} claims that the subject actually precedes the predicate in \ili{Tagalog} translational equivalents to \ili{English} cleft sentences. On this view, \textit{sino} `who' would be the subject in a sentence like (\ref{e:kaufman:27}). \citet[fn.3]{Kroeger:2009}, however, recants this position and views (apparent) headless relatives like \textit{ang dumating} `the one who arrived' in (\ref{e:kaufman:27}) as being in subject position. This change in perspective was prompted by \ili{Tagalog}'s similar behavior to \ili{Malagasy} as analyzed by \citet{Paul:2008} and \citet{Potsdam:2009}, who offer several pieces of evidence for the predicatehood of (non-adjunct) \isi{interrogative} phrases.}

\begin{exe}
	\ex\label{e:kaufman:22}{Any customer who arrives after closing time will not be served.}
\end{exe}

\begin{exe}
	\ex\label{e:kaufman:23}{I will take care of those customers who arrive after closing time.}
\end{exe}

\noindent
In addition to determiners of a relativized noun, a phrase headed by an \isi{interrogative} \isi{pronoun} can be at least partly responsible for projecting a \isi{presupposition}. A certain chess hustler in Greenwich Village used to rile his opponents with the following rhetorical question during the heat of a match:

\begin{exe}
	\ex\label{e:kaufman:24}{Do you know what I like about your game? Absolutely nothing!}
\end{exe}

\noindent\largerpage[2]
The jarring quality of the answer is the effect of canceling the \isi{presupposition} in the question. The infelicity of the following cleft sentences in (\ref{e:kaufman:25}) makes this clear. That the \isi{presupposition} does not come directly from the relative structure can again be seen in the felicitous (\ref{e:kaufman:26}), which contains a \isi{relative clause} headed by the quantifier `nothing', and which carries no \isi{presupposition}. 

\begin{exe}
	\ex\label{e:kaufman:25}
	\begin{xlist}
		\ex\label{e:kaufman:25a} \%Nothing is what I like about your game.
		\ex\label{e:kaufman:25b} \%What I like about your game is nothing.
	\end{xlist}
\end{exe}

\begin{exe}
	\ex\label{e:kaufman:26}{There is nothing that I like about your game.}
\end{exe}

\noindent
We can say then that relative clauses pave the road for presuppositions without necessarily triggering them directly. As we will see in the following section, what triggers the presuppositional reading in putative Philippine-type clefts is the definite semantics of the \isi{nominative} case marking determiner (e.g.\ \ili{Tagalog} \textit{ang}). Unlike \ili{English}, \isi{interrogative} pronouns are never employed for this purpose in Philippine-type languages. 

\section{\label{s:kaufman:3}The syntactic structure of Austronesian clefts}

A key point of variation between Philippine-type and non-Philippine-type \ili{Austronesian} languages can be exemplified with the following example form \ili{Tagalog} (\ref{e:kaufman:27}) and formal \ili{Indonesian}/\ili{Malay} (\ref{e:kaufman:28}).

\begin{exe}
	\ex\label{e:kaufman:27}
	\begin{xlist}
		\exi{Tagalog}
		\ex\label{e:kaufman:27a}
        \gll Sino \textbf{ang} d{\USSmaller}um{\USGreater}ating{\USQMark}\\
		who \textsc{nom} <\textsc{av.beg}>arrive\\
		\glt `Who arrived?'
		\ex\label{e:kaufman:27b}
        \gll D{\USSmaller}um{\USGreater}ating \textbf{ang} guro.\\
		<\textsc{av.beg}>arrive \textsc{nom} teacher\\
		\glt `The teacher arrived.'
	\end{xlist}
\end{exe}

\begin{exe}
	\ex\label{e:kaufman:28}
	\begin{xlist}
		\exi{Formal \ili{Indonesian}/Malay}
		\ex\label{e:kaufman:28a}
        \gll Siapa \textbf{yang} datang{\USQMark}\\
		who \textsc{relt} arrive\\
		\glt `Who arrived?'
		\ex\label{e:kaufman:28b}
        \gll Datang abang-nya...\\
		arrive elder.brother-\textsc{3s.gen}\\
		\glt `His brother arrived...' 	{(Hikayat Pahang 128:9)}
	\end{xlist}
\end{exe}

\noindent\largerpage
Nearly all Philippine-type languages require some form of case marking on clausal arguments. In (\ref{e:kaufman:27}), this can be seen for \ili{Tagalog} in the case marking determiner \textit{ang}, which can be glossed \isi{nominative} or absolutive (see \citealt{Kaufman:2015a} for discussion), but whose function is also tightly bound up with a definite/specific reading of the following NP \citep{Himmelmann:1997}.\footnote{The precise semantics of \textit{ang} has been debated in the literature. A non-definite reading can be obtained in \ili{Tagalog} with \textit{ang isang...} \textsc{nom} one:\textsc{lnk}. However, without the presence of the numeral \textit{isa} `one', felicitous use of \textit{ang} requires familiarity on the part of the hearer. For this reason, I maintain that definiteness, rather than specificity, best captures the pragmatic function of \textit{ang}.} There are two crucial things to note in this comparison. First, while Philippine-type languages require such case markers,  only few \ili{Austronesian} languages of Indonesia employ case marking on arguments \citep[see][]{Himmelmann:2005}. The relativizer \textit{yang} is strongly preferred in the subject question in (\ref{e:kaufman:28a}) but would not be acceptable in (\ref{e:kaufman:28b}) and can thus be easily distinguished from \textit{ang} in \ili{Tagalog} and the equivalents in other Philippine languages. Second, the case markers of Philippine languages do not discriminate between apparent verbal and nominal complements. \citet{Constantino:1965} was the first to show that this is a far reaching characteristic of Philippine languages with the comparisons in \tabref{tab:kaufman:1} and  \tabref{tab:kaufman:2}. In no Philippine language do putative pseudo-clefts contain an overt relative marker, a \textit{wh}-element, a dummy head noun, or any extra sign of nominalization. The \isi{voice} marked words that serve as predicates in  \tabref{tab:kaufman:1} are simply bare complements to the determiner in  \tabref{tab:kaufman:2}. 

\begin{table}[t]
\begin{tabularx}{\textwidth}{Xllllll}\lsptoprule
		 {Tagalog} & kinaːʔin & naŋ &baːtaʔ & \textbf{aŋ} & maŋga\\
		 {Bikolano} & kinakan& kan& aːkiʔ &\textbf{aŋ} &maŋga\\
		 {Cebuano} & ginkaːʔun& han& bataʔ& \textbf{aŋ} &maŋga\\
		 {Hiligaynon} & kinaʔun& saŋ& baːta & \textbf{aŋ}& pahuʔ\\
		 {Tausug} & kyaʔun& sin& bataʔ& \textbf{in}&mampallam\\
		 {Ilokano} & kinnan& dyay&ubiŋ& \textbf{ti} &maŋga\\
		 {Ibanag}& kinan& na& abbiŋ& \textbf{ik}&maŋga\\
		 {Pangasinan}& kina& =y & ugaw& \textbf{su}& maŋga\\
		 {Kapampangan}& peːŋa=na& niŋ& anak& \textbf{iŋ}&maŋga\\\midrule
		& eat:\textsc{pv.prf} & \textsc{gen} & child& \textsc{nom}& mango \\
		&\multicolumn{5}{l}{`The child ate the mango.'\quad}\\\lspbottomrule
\end{tabularx}\caption{\label{tab:kaufman:1}Philippine sentence patterns following \citet{Constantino:1965}}\bigskip
\end{table}

\begin{table}[t]
\begin{tabularx}{\textwidth}{Xllllll}\lsptoprule
		 {Tagalog} & maŋga & \textbf{aŋ} & kinaːʔin & naŋ &baːtaʔ\\
		 {Bikolano} &maŋga &\textbf{aŋ}& kinakan& kan& aːkiʔ\\
		 {Cebuano} &maŋga& \textbf{aŋ}& ginkaːʔun& han& bataʔ\\
		 {Hiligaynon} & pahuʔ& \textbf{aŋ}& kinaʔun& saŋ& baːta\\
		 {Tausug} &mampallam& \textbf{in}& kyaʔun& sin& bataʔ\\
		 {Ilokano} &maŋga& \textbf{ti}& kinnan& dyay& ubiŋ\\
		 {Ibanag}&maŋga& \textbf{ik}& kinan& na& abbiŋ\\
		 {Pangasinan}& maŋga& \textbf{su}& kina& =y& ugaw\\
		 {Kapampangan}&maŋga& \textbf{iŋ}& peːŋa=na& niŋ& anak\\\midrule
		& mango & \textsc{nom} & eat:\textsc{pv.prf} & \textsc{gen} & child\\
		&\multicolumn{5}{l}{`It was the mango that the child ate.'\quad}\\\lspbottomrule
\end{tabularx}\caption{\label{tab:kaufman:2}Philippine sentence patterns following \citet{Constantino:1965}}
\end{table}

\noindent 
It turns out there is good reason for this symmetry. \citet{Starosta:1982} first noted that the well-known \ili{Austronesian} ``\isi{voice}" paradigm appeared to involve a reanalysis of nominalizations as \isi{voice} markers, as shown in (\ref{e:kaufman:29}). \footnote{Unlike the rest of the forms in (\ref{e:kaufman:29}), the \isi{actor voice} morpheme *\textit{<um>} is not thought to have developed from a nominalizer, as it can be reconstructed to the proto-\ili{Austronesian} verbal paradigm \citep{Ross:2002, Ross:2009, Ross:2015a}.}

\clearpage 

\begin{exe}
	\ex\label{e:kaufman:29}{\ili{Austronesian} \isi{voice} morphology}\\
	\begin{tabular}{l}
		{*\textit{-en} \textsc{patient nominalizer} > \textsc{patient voice}}\\
		{*\textit{-an} \textsc{locative nominalizer} > \textsc{locative voice}}\\
		{*\textit{Si-} \textsc{instrumental nominalizer} > \textsc{instrumental voice}}\\
		{*\textit{<um>} \textsc{agent \isi{voice}/nominalizer}}
	\end{tabular}
\end{exe}

\noindent
I argue in \citet{Kaufman:2009cons} that the large number of morphological and syntactic similarities between nouns and verbs can be attributed to the reanalysis in (\ref{e:kaufman:29}). This receives further support from \citet{Ross:2009}, who shows that \ili{Puyuma}, a \ili{Formosan} language, maintains a division of labor where the verbs of Philippine-type main clauses are restricted to relative clauses. Another set of verbal morphology, now only used in a subset of Philippine-type languages for imperatives and subjunctives, is used to mark main \isi{clause} verbal predicates in \ili{Puyuma}.\footnote{Forms which take this set of morphology, referred to as the dependent paradigm by \citet{Wolff:1973} and the non-indicative paradigm by \citet{Ross:2002}, cannot serve as the complement of a determiner or case marker in Philippine languages \citep[25]{Kaufman:2009cons}.} It seems then that the reanalysis of relative clauses as main \isi{clause} predicates in an earlier \ili{Austronesian} proto-language had the effect of erasing any significant differences between relative clauses and main clauses in the daughter languages. If words formed with  the morphology in (\ref{e:kaufman:29}) are nominalizations, it is no surprise that they can serve as direct complements of determiners such as seen above in \tabref{tab:kaufman:2}. There is no need to relativize the verb phrase in sentences such as those in \tabref{tab:kaufman:2} if the verb is already akin to a thematic nominalization. To make this concrete, we could compare the \isi{patient voice} morpheme in (\ref{e:kaufman:29}) to \ili{English} \textit{-ee} in \textit{employee}. \ili{English} allows for the two semantically similar sentence in (\ref{e:kaufman:30}). 

\begin{exe}
	\ex\label{e:kaufman:30}
	\begin{xlist}
		\ex\label{e:kaufman:30a} George is the one Jane employs.
		\ex\label{e:kaufman:30b} George is Jane's employee.
	\end{xlist}
\end{exe}

\noindent
Clearly, direct relativization from a finite \isi{clause} is far more common and productive in \ili{English} than thematic nominalization. But in \ili{Austronesian}, thematic nominalization, as in (\ref{e:kaufman:30b}), was developed to an unusual degree for the purpose of forming relativizations and these then spread to main clauses.\footnote{While this is unusual, it is not unique. \citet{Shibatani:2009} notes typological similarities between the use of thematic nominalizations in \ili{Austronesian} and similar phenomena in \ili{Qiang}, Yaqui, Turkish and \ili{Quechua}. With regard to terminology, I refer below to these historically nominalized predicates, i.e.\ ``Philippine-type verbs", as participles rather than verbs or nouns. The term participle is preferable because these forms have an intermediate status between plain nouns and the historical verbs of \ili{Austronesian}, the latter which have been relegated to non-indicative contexts in Philippine languages.} A consequence of this, particularly important for focus constructions, is that apparent clefts in Philippine-type languages are monoclausal, just as \ili{English} (\ref{e:kaufman:30b}) is monoclausal. The key facts are reviewed below. 

\subsection{\label{s:kaufman:3.1}Apparent  Philippine-type clefts: monoclausal or biclausal?}

A reasonable analysis of the \ili{English} pseudo-cleft is shown in (\ref{e:kaufman:32}), which can be compared to the canonical monoclausal declarative sentence in (\ref{e:kaufman:31}). 

\begin{exe}
	\ex\label{e:kaufman:31}
		\Tree [.TP \qroof{The child}.DP [.T$'$ T [.VP V\\ate \qroof{the mango}.DP ]]] 
\end{exe}

\begin{exe}
	\ex\label{e:kaufman:32}
	\Tree [.TP \qroof{The mango}.DP [.T$'$ T\\is [.VP V [.CP what\textsubscript{i} [.C$'$ C [.TP \qroof{the child}.DP [.VP V\\ate \textit{t}\textsubscript{i} ]]]]]]]
\end{exe}

\noindent
The \ili{English} pseudo-cleft is considered biclausal because it contains two separate extended projections of a verb phrase, headed by \textit{is} in the matrix \isi{clause} and \textit{ate} in the \isi{relative clause} in (\ref{e:kaufman:32}). Each domain can mark categories like tense, negation and agreement independently. In contrast, the monoclausal (\ref{e:kaufman:31}) only contains a single domain for tense, negation and agreement. There is little reason to believe such a distinction exists in Philippine-type languages. The translational equivalents of (\ref{e:kaufman:31}) and (\ref{e:kaufman:32}) in \ili{Tagalog} both appear monoclausal, as suggested by the analysis in (\ref{e:kaufman:33}) and (\ref{e:kaufman:34}). The only difference is that the participle (descended historically from a nominalization) is in the predicate position in (\ref{e:kaufman:33}) and in the subject position in (\ref{e:kaufman:34}). We can treat both cases, however, as copular clauses, indicated by the (null) Cop in both structures.

\begin{exe}
	\ex\label{e:kaufman:33}
	\Tree [.TP [.PredP [.PartP Part\\\textit{kinain}\\eat.\textsc{pv.prf} \qroof{\textit{naŋ bata}\\\textsc{gen} child}.~~DP\textsubscript{\rm GEN} ]] [.T$'$ T\\Cop [.~~DP\textsubscript{\rm NOM} D\\\textit{ang}\\\textsc{nom} \qroof{\textit{maŋga}\\mango}.NP ]]]
\end{exe}

\begin{exe}
	\ex\label{e:kaufman:34}
	\Tree [.TP [.PredP \qroof{\textit{maŋga}\\mango}.NP ] [.T$'$ T\\Cop [.~~DP\textsubscript{\rm NOM} D\\\textit{ang}\\\textsc{nom} [.PartP Part\\\textit{kinain}\\eat.\textsc{pv.prf} \qroof{\textit{naŋ bata}\\\textsc{gen} child}.~~DP\textsubscript{\rm GEN} ]]]]
\end{exe}

\noindent
One of the few arguments that has been adduced in favor of a biclausal structure for sentences such as (\ref{e:kaufman:34}) is a putatively asymmetric pattern of \isi{clitic} placement.\footnote{The notion that apparent clefts in Philippine languages are biclausal is widespread although often not explicitly argued for. \citet[348]{Nagaya:2007}, for instance, analyzing \ili{Tagalog} information structure in an RRG framework, states  ``A \isi{cleft construction} in \ili{Tagalog} is an \isi{intransitive clause} where its single core argument is a headless \isi{relative clause}, and its nucleus is a noun phrase coreferential with the gap in the headless \isi{relative clause}." as illustrated in his (\ref{e:kaufman:34.5}), where \textit{S} represents a gap in the \isi{relative clause}. 

\begin{exe}
	\ex\label{e:kaufman:34.5}
	\gll Si Boyet\textsubscript{i} ang {\ob}p{\USSmaller}um{\USGreater}atay {\ob}S\textsubscript{i}{\cb} kay Juan{\cb}.\\
	\textsc{nom} Boyet \textsc{nom} <\textsc{av}>kill {} \textsc{obl} Juan\\
	\trans`The who killed Juan is Boyet.' \citep[348]{Nagaya:2007}
\end{exe}

Even in non-derivational frameworks such as RRG, the gap strategy employed commonly for relative clauses in Indo-\ili{European} languages is typically applied to the analysis of \ili{Tagalog} without consideration of alternative analyses.}As discussed in detail in \citet{Kaufman:2010a}, \ili{Tagalog} pronominal clitics are positioned after the first \isi{prosodic word} in their syntactic domain. \citet[320]{Aldridge:2004}, assuming that such clitics strictly take the \isi{clause} as their domain, presents the data in (\ref{e:kaufman:35}) as an argument for the biclausal structure of apparent clefts. If such sentences were monoclausal, it would stand to reason that clitics could follow the \isi{interrogative} directly as in (\ref{e:kaufman:35b}), but such a pattern is ungrammatical. 

%\label{EdieCl}
\begin{exe}
	\ex\label{e:kaufman:35}
	\begin{xlist}
		\exi{Tagalog} \citep[319]{Aldridge:2004}
		\ex\label{e:kaufman:35a}
        \gll Ano ang g{\USSmaller}in{\USGreater}a∼gawa=mo{\USQMark}\\
		what \textsc{nom} <\textsc{beg}>\textsc{imprf}∼do.\textsc{pv}=\textsc{2s.gen}\\
		\glt `What are you doing?'
		\ex\label{e:kaufman:35b}
        \gll {\USStar}Ano=mo ang g{\USSmaller}in{\USGreater}a∼gawa{\USQMark}\\
		\phantom{*}what=\textsc{2s.gen} \textsc{nom} <\textsc{beg}>\textsc{imprf}∼do.\textsc{pv}\\
	\end{xlist}
\end{exe}

\noindent
Second position clitics, however, are not only clause-bound; they are also bound within the DP, as can be seen in the following comparison with the possessive \isi{clitic} \textit{=ko} \textsc{1sg.gen}. With a bare predicate like \textit{kaibigan} `friend', as in (\ref{e:kaufman:36a}), the possessive \isi{clitic} attaches to the first element in the \isi{clause}, in this case, negation. When the predicate is a case marked DP, as in (\ref{e:kaufman:36b}), the associated genitive \isi{clitic} cannot take second position in the \isi{clause} and must attach after the first \isi{prosodic word} within the DP.

%\label{friend}
\begin{exe}
	\ex\label{e:kaufman:36}
	\begin{xlist}
		\exi{Tagalog}
		\ex\label{e:kaufman:36a}
        \gll Hindi{\ob}\textbf{=ko}{\cb}=siya kaibigan{\ob}{\USQMark}\textbf{=ko}{\cb}.\\
		\textsc{neg=1s.gen=3s.nom} friend\textsc{=1s.gen}\\
		\glt `He is not a friend of mine.'
		\ex\label{e:kaufman:36b}
        \gll Hindi{\ob}{\USStar}\textbf{=ko}{\cb} siya ang kaibigan{\ob}\textbf{=ko}{\cb}.\\
		\textsc{neg=1s.gen} \textsc{3s.nom} \textsc{nom} friend\textsc{=1s.gen}\\
		\glt `He is not the friend of mine.'
	\end{xlist}
\end{exe}

\noindent
Similarly, in an event-denoting predication such as (\ref{e:kaufman:37}), second position clitics cannot follow the predicate when they originate within a case-marked DP.

\begin{exe}
	\ex\label{e:kaufman:37}{Tagalog}\\
	\gll Na-dapa{\ob}{\USStar}\textbf{=ko}{\cb} ang kapatid{\ob}\textbf{=ko}{\cb}.\\
	\textsc{beg-}fall=\textsc{1s.gen} \textsc{nom} sibling=\textsc{1s.gen}\\
	\glt `My sibling fell.'
\end{exe}

\noindent
\citet{Aldridge:2004}, citing data similar to (\ref{e:kaufman:36}), essentially comes to the same conclusion.\footnote{\citet[262]{Aldridge:2004}: ``I assume that DP is a strong phase, not permitting movement from it. However, a predicate nominal is not, so the \isi{clitic} would be able to move."} But if this generalization is correct, then the earlier \isi{clitic} argument from (\ref{e:kaufman:35}) for a biclausal cleft structure is neutralized. Clitics are unable to escape from a DP and thus the genitive \isi{clitic} in (\ref{e:kaufman:35}), representing an agent embedded in a \isi{nominative} phrase, cannot follow the \isi{interrogative}. 

\subsection{\label{s:kaufman:3.2}True biclausal clefts in Austronesian languages}

The nominal properties of ``Philippine-type verbs" is largely lost south of the Philippines \citep{Kaufman:2009typo}. Consequently, \ili{Malay}, even in its earliest attested stages, does distinguish relative clauses syntactically through the use of \textit{yang}. As shown earlier in (\ref{e:kaufman:28}), \ili{Indonesian}-type relativizers like \textit{yang} are functionally distinct from Philippine-type case marking determiners. We can further see in (\ref{e:kaufman:38}) and (\ref{e:kaufman:39}) how \ili{Indonesian}-type relativizers are distinguished syntactically from the ``linker" or ``ligature" found in most Philippine-type languages. First, \textit{yang} is not required to mediate adjectival modification, as seen in (\ref{e:kaufman:38a}). Second, it can head a phrase without a preceding noun, as seen in (\ref{e:kaufman:38b}). The Philippine linker/ligature differs on both of these counts. It must mediate all instances of modification, as shown in (\ref{e:kaufman:39a}) and cannot surface without a preceding phrase.\footnote{See \citet{Yap:2011} for a further discussion of \textit{yang} and its expanding functions in the history of \ili{Malay}.}

  
\begin{exe}
	\ex\label{e:kaufman:38}
	\begin{xlist}
		\exi{Indonesian}
		\ex\label{e:kaufman:38a}
        \gll rumah {\USOParen}yang{\USCParen} besar\\
		house \phantom{(}\textsc{relt} big \\
		\glt `big house'
		\ex\label{e:kaufman:38b}
        \gll {\USOParen}yang{\USCParen} ini\\
		\phantom{(}\textsc{relt} this \\
		\glt `this one'
	\end{xlist}
\end{exe}

\begin{exe}
	\ex\label{e:kaufman:39}
	\begin{xlist}
		\exi{Tagalog}
		\ex\label{e:kaufman:39a}
        \gll bahay {\USStar}{\USOParen}na{\USCParen} malaki\\
		house \phantom{*(}\textsc{lnk} big\\
		\glt `big house'
		\ex\label{e:kaufman:39b}
        \gll {\USOParen}{\USStar}na{\USCParen} ito\\
		\phantom{(*}\textsc{lnk} this\\
		\glt `this'
	\end{xlist}
\end{exe}

\noindent
A \isi{relative clause} referring to the agent is built on an \isi{actor voice} VP with the addition of the relativizer \textit{yang}, as in (\ref{e:kaufman:40a}). As can be seen in (\ref{e:kaufman:40b}), the plain VP cannot stand in subject position with the same function.\footnote{Verb phrases can also stand in subject position, typically with the help of a \isi{demonstrative}, when functioning as event nominalizations, as in (\ref{e:kaufman:40.5}).  

\begin{exe}
	\ex\label{e:kaufman:40.5}
	\gll \textsubscript{TP}{\ob}\textsubscript{DP}{\ob}\textsubscript{VP}{\ob}Menilai kinerja mentri{\cb} itu{\cb} susah{\cb}.\\
	\phantom{\textsubscript{TP}[\textsubscript{DP}[\textsubscript{VP}[}\textsc{av:}evaluate output minister that difficult\\
	\trans`Evaluating the output of ministers is difficult.' 
\end{exe}}

The presence of a dedicated relativizer is one crucial piece of evidence for the biclausal nature of the construction. An additional piece of evidence is the optional presence of the copular element \textit{adalah}.

%\label{presiden}
\begin{exe}
	\ex\label{e:kaufman:40}
	\begin{xlist}
		\exi{Indonesian}
		\ex\label{e:kaufman:40a}
        \gll Yang menilai kinerja menteri adalah Presiden.\\
		\textsc{relt} \textsc{av:}evaluate output minister \textsc{cop} president\\
		\glt `(The one) who evaluates the output of a minister is the president.'\footnote{\url{http://nasional.republika.co.id/berita/nasional/politik/16/01/06/o0iwuo354-jokowi-yang-menilai-kinerja-menteri-adalah-presiden}}
		\ex\label{e:kaufman:40b}
        \gll {\USStar}Menilai kinerja menteri adalah Presiden.\\
		\phantom{*}\textsc{av:}evaluate output minister \textsc{cop} president\\
	\end{xlist}
\end{exe}

\noindent
The innovation of a copula in \ili{Indonesian} languages has yet to be studied systematically. The copula \textit{adalah} was innovated in the attested history of \ili{Malay} from a presentative use of the existential \textit{ada} in combination with the emphatic \textit{lah}. Although \ili{English}-like cleft constructions employing both the copula and a \isi{relative clause} can be found in modern \ili{Indonesian}, there remain restrictions on the use of the copula that are not well understood. Specifically, we find that the copula is rejected in questions like (\ref{e:kaufman:41b}), a constructed minimal pair with the attested (\ref{e:kaufman:41a}). 

%\label{dituakan}
\begin{exe}
	\ex\label{e:kaufman:41}
	\begin{xlist}
		\exi{Indonesian}
		\ex\label{e:kaufman:41a}
        \gll Dia adalah yang di-tua-kan di antara sesamanya.\\
		\textsc{3s} \textsc{cop} \textsc{relt} \textsc{pv-}old-\textsc{appl} \textsc{prep} among colleague\\
		\glt `It's him who is treated as an elder among colleagues.'\footnote{\url{http://nasional.kompas.com/read/2014/09/19/06431611/Artidjo.Korupsi.Kanker.yang.Gerogoti.Negara}}
		\ex\label{e:kaufman:41b}
        \gll Siapa {\USOParen}{\USStar}adalah{\USCParen} yang di-tua-kan{\USQMark}\\ %\label{dituakanb}
		who \phantom{(*}\textsc{cop} \textsc{relt} \textsc{pv-}old-\textsc{appl}\\
		\glt `Who is treated as an elder?'
	\end{xlist}
\end{exe}

\noindent
In line with the historical development of \textit{adalah}, it is likely that it selects for a focused complement or at least avoids a presupposed one. This is supported by the fact that the copula is again possible when the \isi{interrogative} is in-situ, as in (\ref{e:kaufman:42}).\footnote{The ungrammaticality of post-\isi{interrogative} copulas and copula stranding is not ​addressed ​by \citet{Cole:2000}.}

%\label{iswho}
\begin{exe}
	\ex\label{e:kaufman:42}{Indonesian}\\
	\gll Yang di-tua-kan {\USOParen}adalah{\USCParen} siapa{\USQMark}\\
	\textsc{relt} \textsc{pv-}old-\textsc{appl} \phantom{(}\textsc{cop} who\\
	\glt `The one treated as an elder is who?'
\end{exe}

\noindent
The use of a dedicated relativizer and copula in \ili{Indonesian} (non-adjunct) content questions and focus constructions shows that this language has developed bona fide biclausal cleft sentences where Philippine-type languages still employ an equational monoclausal structure. Unfortunately, the difference between \sloppy{Philippine-type} and \sloppy{non-Philippine-type} \ili{Austronesian} languages in this regard has not been given much attention by syntacticians. The default hypothesis has treated languages like \ili{Indonesian} as simply having overt markers for what are null functional elements in Philippine-type languages.\footnote{\citet{Potsdam:2009} enumerates the pseudo-cleft analyses proposed for wh-questions across a diverse set of \ili{Austronesian} (including both Philippine-type and non-Philippine-type) languages: Palauan \citep{Georgopoulos:1991a}, \ili{Malay} \citep{Cole:2014}, \ili{Indonesian} \citep{Cole:2005b}, Tsou \citep{Chang:2000}, \ili{Tagalog} \citep{Kroeger:1993, Richards:1998a, Aldridge:2004, Aldridge:2002}, \ili{Seediq} \citep{Aldridge:2004, Aldridge:2002}, \ili{Malagasy} \citep{Paul:2001, Paul:2000a, Potsdam:2006, Potsdam:2006a}, Maori  \citep{Bauer:1991, Bauer:1993}, Niuean \citep{Seiter:1980}, Tuvaluan \citep{Besnier:2000}, \ili{Tongan} \citep{Otsuka:2000, Custis:2004}. \citet{Chung:2010} specifically traces the analysis of content questions in Philippine-type languages as pseudo-clefts to \citet{Seiter:1975}. While such analyses appear well supported for many non-Philippine-type languages, it does not seem justified to assume the same structure for Philippine-type languages.} In the next section, we explore how the morphological glue of biclausal constructions is recruited from existing lexical and functional elements as part of a larger argument that such constructions are relatively recent innovations in the history of \ili{Austronesian}. 

\section{\label{s:kaufman:4}How to jerry-rig an Austronesian biclausal cleft}

We can posit a structure such as the one in (\ref{e:kaufman:43}) for an \ili{English} TH-cleft (introduced earlier in \sectref{s:kaufman:2.3}). The focus here lies in the subject position while the \isi{presupposition} is packaged as a DP containing a \isi{relative clause}. Note that there are multiple elements within the \isi{presupposition} that are special to this construction. 

\begin{exe}
	\ex\label{e:kaufman:43}
	\Tree [.TP [.DP \textit{That} ] [.T$'$ T\\\textit{is} [.VP V [.DP [.DP [.D \fbox{\textit{the}} ]  \qroof{\fbox{\textit{one}}}.NP ] [.CP \fbox{Op\textsubscript{\rm i}}  [.C$'$ [.C \fbox{\textit{that}} ] \qroof{\textit{I saw} {\rm t\textsubscript{\rm i}}}.TP ]]]]]]
\end{exe}

\noindent
The DP proper contains a determiner and a semantically bleached noun, in this case \textit{one}. The modifying CP contains a complementizer \textit{that} and, ostensibly, a null operator in the position otherwise reserved for \isi{interrogative} elements. That these layers are distinct is seen both in historical stages of \ili{English}, as exemplified in (\ref{e:kaufman:44}), and in non-standard modern \ili{English}, (\ref{e:kaufman:45}). 

\begin{exe}
	\ex\label{e:kaufman:44}{Middle English} (Chaucer's \textit{Prolog} 836, cited in \citealt{Curme:1912})\\
	\gll He which that hath the shortest shall beginne.\\
	he \textsc{relt.pron} \textsc{comp} has the shortest shall begin\\
	\glt `He who has the shortest shall begin.'
\end{exe}

\begin{exe}
	\ex{Here I am, in this room, because of an organization whose work that I deeply, deeply admire.\footnote{Ellen Page, "Time to Thrive" Conference, 14 February 2014, Human Rights Campaign video, 0:26, posted and accessed 15 February 2014. Cited from Beatrice Santorini's doubly filled comp example webpage: \url{http://www.ling.upenn.edu/~beatrice/examples/doublyFilledCompExamples.html}.}} \label{e:kaufman:45}
\end{exe}

In (\ref{e:kaufman:46}), the DP `an organization', is modified by a CP which contains both an \isi{interrogative} phase, `whose work', moved to its \isi{left periphery} and the complementizer `that'. 

%\label{orgtree}
\begin{exe}
	\ex\label{e:kaufman:46}
	\Tree [.DP [.DP [.D {\textit{an}} ]  \qroof{{\textit{organization}}}.NP ] [.CP \qroof{{\textit{whose work}}}.DP\textsubscript{\rm i} [.C$'$ [.C {\textit{that}} ] \qroof{\textit{I deeply, deeply admire} {\rm t\textsubscript{\rm i}}}.TP ]]]
\end{exe}

\noindent
Given the distinct roles and positions of the determiner, dummy noun, \isi{interrogative} \isi{pronoun} and complementizer in the above \ili{English} structures, we can now ask where the functionally equivalent morphology of \ili{Austronesian} languages fits in, if at all. 


\citet{Adelaar:1992} argues convincingly that the \textit{ya} element in \textit{yang} is cognate with the third person singular \isi{pronoun} \textit{ia} and that the following velar nasal is cognate with the Philippine linker, which we can treat as a type of complementizer.\footnote{\citet{Reid:2002a} argues for a similar analysis of Philippine case markers, in which they consist of a nominal head plus a linker. Translating Reid's proposal to the current framework, a case marker like \ili{Tagalog} \textit{ang} would have an extremely similar structure to \ili{Indonesian} \textit{yang}. This opens up a possibility whose implications I cannot fully address here, namely, that every Philippine-type DP is akin to a \isi{relative clause} headed by a dummy nominal. There is some evidence to recommend such a view. Philippine-type DPs can contain a larger range of syntactic material than might naively be expected from an Indo-\ili{European} perspective. For example, a case marking determiner can have as part of its complement negation and an independent tense domain, as shown in (\ref{e:kaufman:46.5}).

\noindent\parbox[t]{\linewidth}{\begin{exe}
	\ex\label{e:kaufman:46.5}{Tagalog}\\
	\gll Ku∼kun-in ko bukas ang hindì mo k{\USSmaller}in{\USGreater}ain-∅ kahapon.\\
	\textsc{imprf}∼take-\textsc{pv} \textsc{1s.gen} tomorrow \textsc{nom} \textsc{neg} \textsc{2s.gen} <\textsc{beg}>eat-\textsc{pv} yesterday\\
	\glt `I will take tomorrow what you didn't eat yesterday.'
\end{exe}}

Evidence against treating all DPs as full clauses in Philippine-type languages includes the impossibility of dependent form imperatives in DPs (\sectref{s:kaufman:3} above) as well as the marked nature of topicalization within DPs. The latter argument, however, is weakened by the fact that relative clauses can also plausibly exclude a position for fronted topics. If a \isi{relative clause} analysis is justified for Philippine-type DPs, then the typological division between languages like \ili{Tagalog} and \ili{Indonesian} would have to be characterized not as Philippine-type languages lacking relative clauses but rather lacking bare noun phrase arguments. Historically speaking, bare noun phrase arguments and dedicated \isi{relative clause} markers clearly appear to be innovations in languages south of the Philippine area.}
The \isi{pronoun} \textit{ia} can furthermore be broken down into a person marking determiner element \textit{i} \citep{Ross:2006}, plus \textit{a}, a nominal head, as argued for by \citet{Reid:2002a}. The parts of the \ili{Malay}/\ili{Indonesian} relativizer thus fit cleanly into the earlier template motivated by \ili{English}, as shown in (\ref{e:kaufman:47}). 

\begin{exe}
	\ex\label{e:kaufman:47}
	\Tree [.DP [.DP D\\\textit{i-} NP\\\textit{a} ] [.CP \phantom{hi}  [.C$'$ [.C \textit{-ŋ} ] \qroof{~~~~~~~~}.TP ]]]
\end{exe}

\noindent
Similarly, \citet{Kahler:1974} shows that \ili{Ngaju Dayak} and Old \ili{Javanese} recruit demonstratives to play the role of relativizer, which can be located in the same DP occupied by \textit{ia} above. 

Other languages  make use \isi{interrogative} elements, which we locate on the left branch of CP. In two pioneering investigations of relative clauses in \ili{Indonesian} languages, \citet{Gonda:1943} and \citet{Kahler:1974} note the frequency with which \textit{*anu} is used as a relativizer, as in (\ref{e:kaufman:48}).\footnote{\citet{Kahler:1974} further notes that it appears impossible to reconstruct a dedicated relativizer with any real time depth. I attribute this here to the fact that such elements are not necessary in Philippine-type languages whose event-denoting predicates are already noun-like and can thus serve as direct complements of determiners.}

\begin{exe}
	\ex\label{e:kaufman:48}{Sundanese}\\
	\gll Moal aja deui hajam \textbf{{\USOParen}a{\USCParen}nu} bisa hibar lapas.\\
	\textsc{neg} \textsc{ext} anymore chicken \textsc{relt} can fly fast\\
	\glt `In no case are there anymore chickens which can fly fast.' \citep[264]{Kahler:1974}
\end{exe}

\noindent
In \ili{Sundanese}, the relativizer is \textit{anu}, a cognate of what \citet{Blust:2010b} reconstruct as PMP *\textit{a-nu} ``thing whose name is unknown, avoided, or cannot be remembered: what?" Sangirese \textit{apa(n)}, on the other hand, is cognate with \citeauthor{Blust:2010b}'s reconstruction of PMP *\textit{apa} `what?' and shows evidence for a following nasal linker. The Sangirese relativizer thus fits into our schema as shown in (\ref{e:kaufman:50}). 

\begin{exe}
	\ex\label{e:kaufman:49}{Sundanese}\\
	\gll I sire \textbf{apan} məm-pangasi' su səngkamisa naḷiu e, niuntung bue.\\
	\textsc{pm} \textsc{3p} what:\textsc{lnk} \textsc{av-}plant.rice in one.week past \textsc{det} lucky \textsc{emph}\\
	\glt `They who planted rice last week, are lucky.'
	(\citealt[269]{Kahler:1974} citing
	\citealt{AL:main})
\end{exe}

\begin{exe}
	\ex\label{e:kaufman:50}
	\Tree [.DP [.DP D NP ] [.CP \textit{apa}  [.C$'$ [.C \textit{-n} ] \qroof{~~~~~~~~}.TP ]]]
\end{exe}

\noindent
Yet other languages make use of a bleached noun alone. This strategy is extremely common in Sulawesi where we find various derivations of PMP *\textit{tau} `person', most often in the reduced form \textit{to}, as in \ili{Kulawi} (\ref{e:kaufman:51}). The presupposed portion of the \isi{clause} can be analyzed simply as (\ref{e:kaufman:52}), where all the functional positions are left empty except for the bleached noun. 

\begin{exe}
	\ex\label{e:kaufman:51}{Kulawi} \citep[30]{Adriani:1939}\\
	\gll Ba bangkele \textbf{to} na-mate{\USQMark}\\
	\textsc{qm} woman \textsc{relt} \textsc{prf-}die\\
	\glt `Was it a woman who died?'
\end{exe}

\begin{exe}
	\ex\label{e:kaufman:52}
	\Tree [.DP [.DP D NP\\\textit{to} ] [.CP \textit{~~~}  [.C$'$ C   \qroof{\textit{namate}}.TP ]]]
\end{exe}

\noindent
\ili{Balantak}, a language of the Saluan-Banggai subgroup spoken in the eastern side of Central Sulawesi and recently described by \citet{Van-den-Berg:2012}, displays a fascinating combination of features that put it squarely between Philippine and \ili{Indonesian} typologies. Like Philippine-type languages, it has a largely intact \isi{voice system} and the remnants of a case marking system for NPs.  The case marker \textit{a} indicates the subject (i.e.\ the patient of \isi{patient voice}, agent of agent \isi{voice}, etc.) when it is post-verbal, as shown in (\ref{e:kaufman:53}). Just as in \ili{Tagalog} and other Philippine languages, this marker also functions as a definite determiner.

\begin{exe}
	\ex\label{e:kaufman:53}
	\begin{xlist}
		\exi{Balantak} \citep[47--48]{Van-den-Berg:2012}
		\ex\label{e:kaufman:53a}
        \gll Ma-polos tuu' \textbf{a} sengke'-ku.\\
		\textsc{intr.i}-hurt very \textsc{art} back-\textsc{1s}\\
		\glt `My back really hurts.'
		\ex\label{e:kaufman:53b}
        \gll Boit-i-on \textbf{a} piso'-muu kabai sobii{\USQMark}\\
		sharpen-\textsc{app-pv.i} \textsc{art} knife-\textsc{2p} or let.it.be\\
		\glt `Should your knife be sharpened or shall we just leave it?' 
	\end{xlist}
\end{exe}

\noindent
As in Philippine-type languages, the case marker still allows for complements of all lexical categories, as seen in (\ref{e:kaufman:54}). \citet{Van-den-Berg:2012} term such constructions ``semi-clefts''.

\begin{exe}
	\ex\label{e:kaufman:54}
    \gll ...raaya'a \textbf{a} mam-bayar.\\
	\phantom{...}\textsc{3p} \textsc{art} \textsc{av.i}-pay\\
	\glt `...they were the ones who paid.' \citep[50]{Van-den-Berg:2012}
\end{exe}

\noindent
Remarkably, \ili{Balantak} has also developed a relative marker \textit{men} from the bleached noun \textit{mian} `person' (adding further support to the etymology \textit{to} \textsc{relt} < *\textit{tau} `person' in other languages of Sulawesi). This is seen in (\ref{e:kaufman:55}), where both the case marking determiner \textit{a} and the relativizer \textit{men} co-occur in the \isi{presupposition} of question. 

\begin{exe}
	\ex\label{e:kaufman:55}
    \gll Ai ime \textbf{a} men mae'{\USQMark}\\
	\textsc{emph.art} who \textsc{art} \textsc{relt} go\\
	\glt `Who is going?' \citep[50]{Van-den-Berg:2012}
\end{exe}

\noindent
The functional structure of clefts in \ili{Balantak}, shown in (\ref{e:kaufman:56}), would thus look not very different from \ili{Malay}/\ili{Indonesian}. 

\begin{exe}
	\ex\label{e:kaufman:56}
	\Tree [.DP [.DP D\\\textit{a} NP\\\textit{men} ] [.CP \phantom{hi}  [.C$'$ [.C  ] \qroof{~~~~~~~~}.TP ]]]
\end{exe}

\noindent
What is unique in \ili{Balantak} is that the determiner in this structure maintains a robust NP case-marking function and can attach directly to verbs in many contexts. \ili{Balantak} thus offers us a live view of what must have happened throughout Indonesia. The loss of case marking proceeds hand-in-hand with the rise of relativizers. In \ili{Balantak} questions, it is still the case marker which is obligatory, not the relativizer, as \citet{Van-den-Berg:2012} show explicitly in (\ref{e:kaufman:57}). 

\begin{exe}
	\ex\label{e:kaufman:57}
	\begin{xlist}
		\exi{Balantak}
		\ex\label{e:kaufman:57a}
        \gll Pi-pii takalan a {\USOParen}men{\USCParen} ala-on-muu{\USQMark}\\
		\textsc{red}-how.many liter \textsc{art} \textsc{relt} take-\textsc{pv.i-2p}\\
		\glt `How many liters will you take?' 
		\ex\label{e:kaufman:57b}
        \gll {\USStar}Pi-pii takalan {\USOParen}men{\USCParen} ala-on-muu{\USQMark}\\
		\phantom{*}\textsc{red}-how.many liter \textsc{relt} take-\textsc{pv.i-2p}\\
		\glt `How many liters will you take?' 
	\end{xlist}
\end{exe}

\noindent
But the functional scope of the case marker has also clearly shrunk in comparison to typical Philippine-type languages. Specifically, the \isi{nominative} determiner \textit{a} only occurs post-verbally in \ili{Balantak} whereas in Philippine languages we find no such restriction. The loss of this domain would have given rise to the need for a relativizer \textit{men} in positions where \textit{a} was no longer licensed.\footnote{Like \ili{Balantak}, \ili{Malagasy} also instantiates an intermediate position between canonical Philippine-type languages and \ili{Indonesian}. It is more complex than the languages considered here in possessing a distinct (i) focus complementizer \textit{no}, (ii) relativizer \textit{izay} and (iii) NP marker \textit{ny}. \posscitet{Keenan:2008} analysis of \ili{Malagasy} is close to the one advocated here for Philippine-type languages although \citet{Paul:2001, Law:2007, Kalin:2009, Pearson:2009, Potsdam:2006a} show that the syntax of \ili{Malagasy} cleft constructions is clearly non-equational. \posscitet[220--225]{Potsdam:2006a} examination of the \ili{Malagasy} CP in clefts is especially relevant here but space considerations preclude a full comparison.}

\section{\label{s:kaufman:5}Referentiality and predication}

An idea was put forth earlier that the less referential half of a predication is assigned to the predicate position of a \isi{clause} while the more referential half is packaged as the subject, in line with work on copular clauses in \ili{English} and other well studied languages. We have also seen in the above how the predicate position in \ili{Austronesian} languages functions as a kind of de facto \isi{focus position} by virtue of \ili{Austronesian} languages tending to package presuppositions as subjects. The mechanics of this turn out to contain some surprises. 

First, note that a bare predicate phrase in \ili{Tagalog}, whether it is headed by an entity-denoting, property-denoting or event-denoting word, must precede the subject, as exemplified in (\ref{e:kaufman:58}) and (\ref{e:kaufman:59}).\footnote{Topicalization is possible to achieve the subject initial orders here but it is marked either by the topic marker \textit{ay} or a very clear intonational break after the subject. In short sentences like (\ref{e:kaufman:62}), the intonational break may be more difficult to hear. Speakers seem to agree however that for the order in (b) to be licit, there must be distinct phonological phrases while this is not true for the (a) sentences. It is in fact possible to make the judgments completely unambiguous through the use of clitics. Specifically, we can compare sentences like the following where the second position \isi{pronoun} has two forms, a long form, \textit{ikaw}, used in predicate position, and a \isi{clitic} form \textit{=ka}, used for arguments. When the second person is in a predication with a \isi{demonstrative}, the \isi{clitic} form is ungrammatical: \textit{Ikaw iyan} \textsc{2s.nom} that.\textsc{nom} `That's you' versus \textit{*Iyan=ka} `that.\textsc{nom=2sg.nom}'. When the \isi{demonstrative} is topicalized, the second person \isi{pronoun} retains its predicate form, \textit{Iyan, ikaw} that.\textsc{nom 2sg.nom} `That, is you'.} The basic \isi{word order} in \ili{Tagalog} and the vast majority of Philippine-type languages is thus regularly described as predicate-initial on this basis. 

\begin{exe}
	\ex\label{e:kaufman:58}
	\begin{xlist}
		\ex\label{e:kaufman:58a}
        \gll Guro ako.\\
		teacher \textsc{1s.nom}\\
		\glt `I am a teacher.'
		\ex\label{e:kaufman:58b}
        \gll {\USStar}Ako guro.\\
		\phantom{*}\textsc{1s.nom} teacher\\
		Cf. *A teacher is me.
	\end{xlist}
\end{exe}

\begin{exe}
	\ex\label{e:kaufman:59}
	\begin{xlist}
		\ex\label{e:kaufman:59a}
        \gll Matangkad ako.\\
		tall \textsc{1s.nom}\\
		\glt `I am tall.'
		\ex\label{e:kaufman:59b}
        \gll {\USStar}Ako matangkad.\\
		\phantom{*}\textsc{1s.nom} tall\\
		Cf. *Tall is me.
	\end{xlist}
\end{exe}

\noindent
A paradox surfaces, however, when both parts of the predication are referential or definite. In such cases, it appears that the \emph{more} referential portion of the predication must be located in the clause-initial  predicate position. In a neutral context, \textit{that} fills the subject position in the \ili{English} translation of (\ref{e:kaufman:60}). In \ili{Tagalog}, the \isi{demonstrative} must be positioned in the clause-initial predicate position. In an \ili{English} copular \isi{clause} with a pronominal argument and a definite NP, the pronominal argument will be selected as the subject. In \ili{Tagalog}, the pronominal argument must always be in predicate position when the other argument is definite, as seen in (\ref{e:kaufman:61}).\footnote{As noted in fn.13, \citet[148--149]{Kroeger:1993} analyzes such constructions as inversions where the first constituent is the subject and the latter constituent is the predicate. All evidence, however, points to the initial constituents in such sentences behaving as predicates, leading \citet{Kroeger:2009} to revise his original analysis.}

\begin{exe}
	\ex\label{e:kaufman:60}
	\begin{xlist}
		\ex\label{e:kaufman:60a}
        \gll Iyan ang problema.\\
		that.\textsc{nom} \textsc{nom} problem\\
		\glt `That's the problem.' (Lit. `The problem is that.')
		\ex\label{e:kaufman:60b}
        \gll {\USStar}Ang problema iyan.\\
		\phantom{*}\textsc{nom} problem that.\textsc{nom}\\
	\end{xlist}
\end{exe}

\begin{exe}
	\ex\label{e:kaufman:61}
	\begin{xlist}
		\ex\label{e:kaufman:61a}
        \gll Ako ang guro.\\
		\textsc{1s} \textsc{nom} teacher\\
		\glt `I am the teacher.'
		\ex\label{e:kaufman:61b}
        \gll {\USStar}Ang guro ako.\\
		\phantom{*}\textsc{nom} teacher \textsc{1s}\\
		Cf. \textsuperscript{*}The teacher is me.
	\end{xlist}
\end{exe}

\noindent
In (\ref{e:kaufman:62}), \ili{English} and \ili{Tagalog} again agree in placing the \isi{demonstrative} in the subject position and the first singular \isi{pronoun} in predicate position. 

\begin{exe}
	\ex\label{e:kaufman:62}
	\begin{xlist}
		\ex\label{e:kaufman:62a}
        \gll Ako iyan.\\
		\textsc{1s.nom} that.\textsc{nom}\\
		\glt `That's me.'
		\ex\label{e:kaufman:62b}
        \gll {\USQMark}{\USStar}Iyan ako.\\
		\phantom{?*}that.\textsc{nom} \textsc{1s.nom}\\
	\end{xlist}
\end{exe}

\noindent
Based on the above data, we can no longer say that \ili{Tagalog} merely displays the mirror image of the \ili{English} subject-predicate order. While both \ili{Austronesian} languages and \ili{English} enforce a familiarity condition on subjects (see \citealt[][chap.8]{Mikkelsen:2005}, for a summary of the \ili{English} facts), there appears to be an additional role for an extended definiteness or animacy hierarchy in \ili{Tagalog} and other Philippine languages. The involvement of an animacy hierarchy is clear from the following facts. Just like demonstratives, a third person \isi{pronoun} must be in predicate position if the other half of the predication is definitely determined, as seen in (\ref{e:kaufman:63}). But when a third person \isi{pronoun} is in competition with a first person \isi{pronoun} for predicate position, it is the first person which wins, as shown in (\ref{e:kaufman:64}).

\begin{exe}
	\ex\label{e:kaufman:63}
	\begin{xlist}
		\ex\label{e:kaufman:63a}
        \gll Siya ang problema.\\
		\textsc{3s.nom} \textsc{nom} problem\\
		\glt `S/he's the problem.' (Lit. `The problem is s/he.')
		\ex\label{e:kaufman:63b}
        \gll {\USStar}Ang problema siya.\\
		\phantom{*}\textsc{nom} problem \textsc{3s.nom}\\
	\end{xlist}
\end{exe}

\begin{exe}
	\ex\label{e:kaufman:64}
	\begin{xlist}
		\ex\label{e:kaufman:64a}
        \gll Ako siya.\\
		\textsc{1s.nom} \textsc{3s.nom}\\
		\glt `S/he's me.'
		\ex\label{e:kaufman:64b}
        \gll {\USStar}Siya ako.\\
		\phantom{*}\textsc{3s.nom} \textsc{1s.nom}\\
	\end{xlist}
\end{exe}

\noindent
Although space does not permit a full demonstration of all the possible interactions between NP types, the rules follow a slightly modified version of \posscitet[437]{Aissen:2003} definiteness hierarchy, shown in (\ref{e:kaufman:65}). When both halves of a predication are referential, the portion higher on the scale in (\ref{e:kaufman:65}) will be selected as predicate. 

\begin{exe}
	\ex\label{e:kaufman:65}{\textsc{definiteness/animateness hierarchy}} \citep{Silverstein:1976, Aissen:1999, Aissen:2003}\\
	local [1/2] person > third person pronouns > demonstratives/proper name > Definite NP > Indefinite Specific NP > Non-Specific
\end{exe}

\noindent
The only real optionality, as indicated by the lack of ranking above, is found with demonstratives and proper names. When these two types are in a predication relation, either order is acceptable, as seen in (\ref{e:kaufman:66}). This can potentially be linked to the ability of proper names in \ili{Tagalog} to behave like pronominal clitics \citep{Billings:2005}.

\begin{exe}
	\ex\label{e:kaufman:66}
	\begin{xlist}
		\ex\label{e:kaufman:66a}
        \gll Iyan si Boboy.\\
		that.\textsc{nom} \textsc{nom} Boboy\\
		\glt `That's Boboy.'
		\ex\label{e:kaufman:66b}
        \gll Si Boboy iyan.\\
		\textsc{nom} Boboy that.\textsc{nom}\\
		\glt `That's Boboy.'
	\end{xlist}
\end{exe}

\noindent
In predications where the order is fixed by virtue of the definiteness hierarchy, information structure is flexible. For example, the sentence \textit{ako ang guro} `I am the teacher' can answer both the question in (\ref{e:kaufman:67}) as well as that in (\ref{e:kaufman:68}). This is unusual in Philippine languages, as the clause-initial predicate position is otherwise reserved for the focus of the sentence rather than the \isi{presupposition}.\footnote{\citet{Aldridge:2013} claims that in predications with two definite DPs (two \textit{ang} phrases), the first is always the focus, exemplified with (\ref{e:kaufman:68.5}). I am not convinced that a focus reading is necessary or even unmarked on the first \textit{ang} phrase in (\ref{e:kaufman:68.5b}). Previous authors have disagreed on the pragmatic status of double \textit{ang} phrase predications in \ili{Tagalog}. Aldridge argues that predicate fronting in \ili{Tagalog} (to derive the basic \isi{word order}) is movement to a \isi{focus position}. My feeling is rather that the focus interpretation of the predicate is a result of packaging presuppositions as definitely determined subjects. Once the \isi{presupposition} is subtracted, the left-overs in clause-initial position canonically align with the focus. Examples such as (\ref{e:kaufman:68.5}) are critical to adjudicating between these two analyses but this must be left to further work.

\begin{exe}
	\ex \label{e:kaufman:68.5}
	\begin{xlist}
		\ex\label{e:kaufman:68.5a}
        \gll {\ob}Ang lalaki{\cb} ang na-kita ng babae.\\
		\textsc{nom} man \textsc{nom} \textsc{nvol.pv}-see \textsc{gen} woman\\
		\glt `The man is who the woman saw.'
		\ex\label{e:kaufman:68.5b}
        \gll {\ob}Ang na-kita ng babae{\cb} ang lalaki.\\
		\textsc{nom} \textsc{nvol.pv}-see \textsc{gen} woman \textsc{nom} man\\
		\glt `The man is who the woman saw.'
	\end{xlist}
\end{exe}}

\begin{exe}
\label{e:kaufman:67}
		\ex\label{e:kaufman:67a}
        \gll A: Sino ang guro{\USQMark}\\
		\phantom{A:} who.\textsc{nom} \textsc{nom} teacher\\
		\glt `Who is the teacher?'
		
		\label{e:kaufman:67b}\gll B: Ako ang guro.\\
		\phantom{B:} \textsc{1s.nom} \textsc{nom} teacher\\
		\glt `\uline{I} am the teacher.'
\end{exe}

\begin{exe}
\label{e:kaufman:68}
		\ex\label{e:kaufman:68a}\gll A: Sino=ka{\USQMark}\\
		\phantom{A:} who.\textsc{nom=2s.nom}\\
		\glt `Who are you?'
		
		\label{e:kaufman:68b}\gll B: Ako ang guro.\\
		\phantom{B:} \textsc{1s.nom} \textsc{nom} teacher\\
		\glt `I am \uline{the teacher}.'
\end{exe}

\noindent
I would like to offer a potential solution to the paradox of why it is the more definite or referential element that becomes the predicate when both elements are referential, in stark contrast to the canonical packaging of new information as predicate. The pattern can be accounted for by viewing it as the product of two potentially conflicting constraints. On one hand, presuppositions are packaged as \textit{ang} phrases and what is left in the clause-initial position is the de facto focus. The only principle that predicate selection in the strict sense takes into account is whether an element is definite or not. If one element is definite and the other is not the story ends there; the definite element is packaged as subject while the remainder is placed in predicate position. If both elements are definite, another principle comes into play which only relates secondarily to the subject-predicate relation. This principle demands that elements higher on the definiteness/animacy hierarchy \emph{linearly precede} those which are lower on the hierarchy. The clause-initial predicate position is then pressed into service to make the more animate element precede the less animate one.

Several pieces of evidence from other \ili{Austronesian} languages support this analysis. First of all, as discussed in \citet{Kaufman:2014a}, many \ili{Indonesian} languages have independently arrived at a split proclitic/enclitic system for agent marking.\footnote{Split proclitic/enclitic patterns in the languages of Sulawesi are argued by \citet{Berg:1996} to have developed from a  full proclitic pattern and by \citet{Himmelmann:1996} from a full enclitic paradigm. The history and typology of pronominal proclisis is further discussed by \citet{Wolff:1996, Mead:2002, Kikusawa:2003, Billings:2004}. I believe the comparative evidence points very clearly to split-paradigms resulting from partial accretion rather than loss, besides the obvious preference of Occam's razor for such an account, but the details do not concern us here.} In all attested examples, third person markers procliticize only if the local persons [1/2] have procliticized. First person furthermore tends to procliticize before second person. This can be seen clearly in the languages of Sumatra, as shown in \tabref{tab:kaufman:3} and equally compelling evidence is found in the languages of Sulawesi. On one end of the spectrum, all pronominal agents were enclitic in Old \ili{Malay}. On the other side of the spectrum, Minangkabau, all such agents are expressed as proclitics. In between, Karo \ili{Batak}, Gayo and \ili{Classical Malay} which show a development that respects the animacy hierarchy such that the agents higher on the hierarchy must precede those which are lower. 

%\label{Sumatra}
\begin{table*}
	\begin{tabular}{llllll}\lsptoprule
		& \textbf{Old Malay} & \textbf{Karo Batak}&\textbf{Gayo}&\textbf{Clas. Malay}&\textbf{Minangkabau}\\
		\midrule
		\textsc{1sg.}		& ni-V-\textbf{(ŋ)ku} 		& \cellcolor[gray]{0.8}{\textbf{ku}-}V		& \cellcolor[gray]{0.8}{\textbf{ku}-V} 		& \cellcolor[gray]{0.8}{\textbf{ku}-V}		& \cellcolor[gray]{0.8}{\textbf{den}-V}\\
		\textsc{2sg.}		& (ni-V-\textbf{māmu}) 		& i-V\textbf{-әŋkō} 		& i-V\textbf{-kō} 		& \cellcolor[gray]{0.8}{\textbf{kau}-V}	& \cellcolor[gray]{0.8}{\textbf{aŋ}-V}\\
		\textsc{3sg.}		& ni-V-\textbf{ña} 		& i-V-\textbf{na} 		& i-V\textbf{-é} 		& di-V-\textbf{ña}		& \cellcolor[gray]{0.8}{\textbf{iño}-V}\\ 
		\textsc{1pl.excl} 	& ?	 			& i-V-\textbf{kami} 		& \cellcolor[gray]{0.8}{\textbf{kami-}V} 	& \cellcolor[gray]{0.8}{\textbf{kami}-V} 	& \cellcolor[gray]{0.8}{\textbf{kami}-V}\\
		\textsc{1pl.incl}	& ni-V-\textbf{(n)ta} 		& \cellcolor[gray]{0.8}{\textbf{si}-}V 		& \cellcolor[gray]{0.8}{\textbf{kit\"o-}V} 	&  \cellcolor[gray]{0.8}{\textbf{kita}-V} 	& \cellcolor[gray]{0.8}{\textbf{kito}-V}\\
		\textsc{2pl.}		& ni-V-\textbf{māmu} 		& i-V\textbf{-kam} 		& i-V\textbf{-kam} 		&\cellcolor[gray]{0.8}{\textbf{kamu}-V}	& \cellcolor[gray]{0.8}{\textbf{kau}-V}\\
		\textsc{3pl.}		& ni-V-\textbf{(n)da} 		& i-V\textbf{-na} 		& i-V\textbf{-é} 		& di-V-\textbf{mereka} 	& \cellcolor[gray]{0.8}{\textbf{iño}-V}\\
		\lspbottomrule
	\end{tabular}\caption{\label{tab:kaufman:3}Person marking in the patient voice \citep{Kaufman:2014a}}\bigskip
\end{table*}

In an independent development in several languages of Mindanao in the Philippines, the animacy hierarchy also determines the order of clitics within a \isi{clitic} cluster \citep{Billings:2004, Kaufman:2010}. For instance, in \ili{Maranao}, a first person \isi{clitic} always precedes a second person \isi{clitic} and both first and second person clitics precede third person clitics, as seen in (\ref{e:kaufman:69}). 

\begin{exe}
	\ex\label{e:kaufman:69}{Maranao} \citep{Kaufman:2010}\\
	\begin{xlist}
		\ex\label{e:kaufman:69a}
        \gll M{\USSmaller}iy{\USGreater}a-ilay=ako=ngka.\\
		\textsc{<prf>pv.nvol}-see=\textsc{1s.nom}=\textsc{2s.gen}\\
		\glt `You saw me.'
		\ex\label{e:kaufman:69b}
        \gll M{\USSmaller}iy{\USGreater}a-ilay=ngka=siran.\\
		\textsc{<prf>pv.nvol}-see=\textsc{2s.gen}=\textsc{3p.nom}\\
		\glt `You saw them.'
	\end{xlist}
\end{exe}

\noindent
Both of these phenomena offer support for the idea that there is an earliness principle at play which makes use of the definiteness/animacy hierarchy. A prediction of this analysis, which is driven by linear precedence, is that no subject-predicate paradox of the type found in \ili{Tagalog} should exist in \ili{Austronesian} languages with basic SVO \isi{word order}. This is because the argument which is higher on the definiteness/animacy hierarchy will both make for a more natural subject and naturally precede the predicate in such languages. This prediction is at least borne out in \ili{Indonesian}. As seen in (\ref{e:kaufman:70}), even a subject low on the animacy/definiteness hierarchy precedes the predicate in the unmarked \isi{word order}. In a copular sentence such as that in (\ref{e:kaufman:71}), where a first person \isi{pronoun} is in a predication relation with a definite NP, the pronouns still takes the canonical subject position. Unlike \ili{Tagalog}, it cannot felicitously be positioned in predicate position without special topic-focus \isi{intonation}.
\largerpage[2]

\begin{exe}
		\ex\label{e:kaufman:70}{Indonesian}\\
		\gll Serigala bisa membunuh orang.\\
		wolf can \textsc{av}:kill person\\
		\glt `Wolves can kill people'
\end{exe}

\begin{exe}
	\ex\label{e:kaufman:71}{Indonesian} 
	\begin{xlist}
		\ex\label{e:kaufman:71a}
        \gll Aku guru-nya.\\
		\textsc{1s} teacher-\textsc{def}\\
		\glt `I'm the teacher.'
		\ex\label{e:kaufman:71b}
        \gll \#Guru-nya aku.\\
		\phantom{\#}teacher-\textsc{def} \textsc{1s}\\
	\end{xlist}
\end{exe}

\noindent
Unfortunately, this topic has been left almost completely unexplored for other languages of Indonesia and so it is not yet possible to compare SVO languages of Indonesia with predicate-initial ones more broadly. The predictions of the current analysis are clear though that the unexpected inversions found in \ili{Tagalog} should only occur in predicate-initial languages. 

\section{Conclusion}

I have explored here several related aspects of predication and information structure in \ili{Austronesian} languages. I began by arguing for a monoclausal analysis of apparent clefts in Philippine-type languages and tying this to the nominal nature of Philippine-type verbs. I then showed how true biclausal clefts emerge in \ili{Indonesian} languages where the noun-verb contrast is more robust. In such languages, presupposed verbal material must be relativized before it can occupy subject position. While \ili{Indonesian} relativizers come from varied sources (bleached nouns, interrogatives, pronouns in combination with the linker), it was shown that all patterns under examination fit neatly into a common syntactic template. Finally, I made an attempt at solving a paradox in the subject-predicate relation of Philippine-type languages. I argued that in addition to a canonical familiarity condition on subjects, there exists a linearity condition which requires that the part of a predication which is higher on the definiteness hierarchy precede the part which is lower. The prediction, which requires further exploration, is that SVO languages should not display these unexpected inversions.

It perhaps deserves emphasizing here that syntacticians have been too hasty in positing \ili{English}-like constituency structures and lexical categories in the analysis of \ili{Austronesian} languages. Consequently, important differences between \sloppy{Philippine-type} and \sloppy{non-Philippine-type} \ili{Austronesian} languages have been masked. By stepping back from these assumptions, we can begin to explore fundamental problems in the relation between predication and information structure. Although the present work has only scratched the surface, it has hopefully opened a path for further research in how this relation varies across \ili{Austronesian} languages. The resolution of this problem in \ili{Austronesian} may very well contribute to answering the philosophical questions around predication first put forth by Plato and Aristotle over two millennia ago and debated today.

\section*{Abbreviations}

\begin{multicols}{2}
	\begin{tabbing}
		glossgloss \= \kill
		\textsc{appl} \> applicative\\
		\textsc{art} \> article\\
		\textsc{av} \> \isi{actor voice}\\
		\textsc{beg} \> begun aspect\\
		\textsc{comp} \> complementizer\\
		\textsc{cop} \> copula\\
		\textsc{det} \> determiner\\
		\textsc{emph} \> emphatic\\
		\textsc{ext} \> existential\\
		\textsc{gen} \> genitive case\\
		\textsc{imprf} \> imperfective aspect\\
		\textsc{intr} \> intransitive\\
		\textsc{lnk} \> linker\\
		\textsc{neg} \> negation\\
		\textsc{nom} \> \isi{nominative} case\\
		\textsc{nvol} \> non-voluntary mood\\
		\textsc{obl} \> oblique case\\
		\textsc{pm} \> personal marker\\
		\textsc{prf} \> perfective aspect\\
		\textsc{pst} \> past tense\\
		\textsc{pv} \> \isi{patient voice}\\
		\textsc{qm} \> question marker\\
		\textsc{red} \> reduplication\\
		\textsc{relt} \> relative marker
	\end{tabbing}
\end{multicols}

\sloppy
\printbibliography[heading=subbibliography,notkeyword=this]

\end{document}
