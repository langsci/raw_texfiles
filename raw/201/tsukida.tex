\documentclass[output=paper
,modfonts
,nonflat]{langsci/langscibook} 

\ChapterDOI{10.5281/zenodo.1402553}

\title{Preposed NPs in Seediq} 
\author{Naomi Tsukida\affiliation{Aichi Prefectural University}}
% \chapterDOI{} %will be filled in at production

% \epigram{}

\abstract{Seediq is an Austronesian language spoken in northeastern Taiwan. Its word order is VXS in general (where X stands for adjuncts or arguments other than the subject), but an NP that has some semantic relation to the matrix clause can precede the matrix clause. This NP is followed by the particle \textit{'u} and a non-final pause. I will call such NPs preposed NPs. This paper will investigate the nature of these preposed NPs. What is their semantics and function? What type of NPs can be preposed? What are their anaphoricity and topic-persistence properties? Preposed NPs are often coreferential with the matrix subject, which may or may not be overt. When a preposed NP is coreferential with the main clause subject, and the matrix subject is not overt, then the seeming word order is SVX. When the preposed NP is coreferential with the matrix subject, and that subject is overt, then the seeming word order is SVXS. How do SVX and SVXS differ? What is the function of this double reference in SVXS? In addition, how do these word orders differ from simple VXS order?}

\begin{document}

\maketitle

\section{\label{s:tsukida:1}Introduction}

\ili{Seediq} is an \ili{Austronesian} language, spoken in northeastern Taiwan and belonging to the Atayalic subgroup. There are three \ili{Seediq} dialects: \ili{Teruku}, Tekedaya, and \ili{Tuuda}. The research reported here is based on the \ili{Teruku} dialect, which is mainly spoken in Hualien County. The population of the \ili{Teruku} subgroup is about 30,000, but the younger generations do not speak the language.

In what follows \sectref{s:tsukida:1.1} explains \ili{Seediq} \isi{word order}. Subject and \isi{voice} are covered in \sectref{s:tsukida:1.2}, non-subject arguments in \sectref{s:tsukida:1.3}, and the \textit{'u} particle in \sectref{s:tsukida:1.4}.

\subsection{\label{s:tsukida:1.1}Word order}

The basic \isi{constituent order} in \ili{Seediq} is VXS (X stands for adjuncts or arguments other than the subject). In example (\ref{e:tsukida:1}), \textit{k<em>erut}\footnote{The phoneme inventory of \ili{Teruku} \ili{Seediq} is as follows: p, t, k, q, ', b, d, s, x, h, g (voiced velar fricative), c, l (voiced lateral fricative), r, m, n, ng (velar nasal), w, y, a, i, u, and e (schwa).} ‘<\textsc{av}>cut’ is the V, \textit{bunga} ‘sweet potato’ is a non-subject argument, and \textit{payi} ‘old woman’ is the subject. \textit{Ka} is multi-functional. Its function here is to mark the subject. 

\begin{exe}
	\ex{AV}\label{e:tsukida:1}\\
	\gll \textit{K}{\USSmaller}\textit{em}{\USGreater}\textit{erut} \textit{bunga} \textit{ka} \textit{payi}.\\
	\textsc{<av>}cut sweet.potato \textsc{sbj} old.woman\\
	\glt ‘The old woman cuts sweet potato.’
\end{exe}

\noindent
In NPs, modifiers other than numbers typically follow the head. See \sectref{s:tsukida:3.1} for details. 

\ili{Seediq} has two sets of \isi{clitic} pronouns, \textsc{Nominative} and \textsc{Genitive}, and two sets of independent pronouns, \textsc{Neutral} and \textsc{Oblique}. While all the pronouns have a \textsc{Genitive} \isi{clitic}, the \textsc{Nominative} \isi{clitic} is limited to the first and second persons. Clitic pronouns are second-position clitics, following the first element of the predicate. Some of the examples are =\textit{ku} in example (\ref{e:tsukida:2}) and =\textit{na} in examples (\ref{e:tsukida:6}), (\ref{e:tsukida:7}), and (\ref{e:tsukida:8}). 

\begin{exe}
	\ex{Clitic pronoun}\label{e:tsukida:2}\\
	\gll \textit{K}{\USSmaller}\textit{em}{\USGreater}\textit{erut}=\textit{ku} \textit{bunga} {\USOParen}\textit{ka}  \textit{yaku}{\USCParen}.\\
	\textsc{<av>}cut\textsc{=1sg.nom} sweet.potato \phantom{(}\textsc{sbj}  \textsc{1sg}\\
	\glt ‘I cut sweet potato.’
\end{exe}

\noindent
One can omit adjuncts and arguments including the subject if they are recoverable from the context. Clitic pronouns are enough to indicate arguments, so when there is a \isi{nominative} \isi{clitic} in the sentence, it does not need to be referred to by an independent \isi{pronoun}. 

In addition to the basic VXS order, \ili{Seediq} can have an NP preceding the matrix \isi{clause}, followed by the particle \textit{'u} and a non-final pause. In example (\ref{e:tsukida:3}), \textit{payi} ‘old woman’ is preposed and followed by the \textit{'u} particle and a non-final pause. The matrix subject \textit{ka payi} may be omitted, because, even if it is omitted, it is identifiable as coreferential with the preposed NP.

\begin{exe}
	\ex{Preposed NP}\label{e:tsukida:3}\\
	\gll \textit{Payi} \textit{'u}, \textit{k}{\USSmaller}\textit{em}{\USGreater}\textit{erut} \textit{bunga} {\USOParen}\textit{ka}  \textit{hiya}{\USCParen}.\\
	old.woman \textsc{prt} \textsc{<av>}cut sweet.potato \phantom{(}\textsc{sbj} \textsc{3sg}\\
	\glt ‘The old woman, she cuts sweet potato.’
\end{exe}

\noindent
As a result, SVX order (more precisely, \textit{S 'u}, \textit{V} \textit{X} (\textit{ka} \textit{S})) is realized, though I do not regard the preposed NP as a subject. This paper will investigate the nature of such preposed NPs.

\subsection{\label{s:tsukida:1.2}Subject and voice}

The subject appears in the Neutral case, preceded by the particle \textit{ka}, in clause-final position. \textit{Ka} is multi-functional; some of its functions are subject-marker, linker, and complementizer. Its usage as a linker is yet to be investigated. 

In \ili{Seediq}, thematic roles fall into three groups, according to the verb form they trigger when they are chosen as the subject. The first group, the A group, includes \textsc{Agent}, \textsc{Theme}, and \textsc{Experiencer}; these trigger \isi{AV} forms, as in (\ref{e:tsukida:1}) in \sectref{s:tsukida:1.1}. The second group, the G group, includes \textsc{Patient}, \textsc{Goal}, \textsc{Location}, and \textsc{Recipient}; these trigger GV forms, as in (\ref{e:tsukida:4}). The third group, the C group, includes \textsc{Conveyed theme}, \textsc{Instrument}, and \textsc{Beneficiary}; these trigger CV forms, as in (\ref{e:tsukida:5}). In (\ref{e:tsukida:4}) and (\ref{e:tsukida:5}), the patient \textit{bunga} ‘sweet potato’ and the instrument \textit{yayu} ‘knife’ is the subject, respectively, and the Agent \textit{payi} ‘old woman’ is not the subject anymore.

\begin{exe}
	\ex{GV}\label{e:tsukida:4}\\
	\gll \textit{Kerut}-\textit{un} \textit{payi} \textit{ka} \textit{bunga}.\\
	cut-\textsc{gv1} old.woman \textsc{sbj} sweet.potato\\
	\glt ‘The/An old woman will cut the sweet potato.’
\end{exe}

\begin{exe}
	\ex{CV}\label{e:tsukida:5}\\
	\gll \textit{Se}-\textit{kerut} \textit{bunga} \textit{payi} \textit{ka} \textit{yayu}.\\
	\textsc{cv}-cut sweet.potato old.woman \textsc{sbj} knife\\
	\glt ‘The/An old woman cut the/a sweet potato with the knife.’
\end{exe}

\noindent
Patient and location are grouped together; forms suffixed with -\textit{un} are GV1, and those with -\textit{an} are GV2 forms. Corresponding forms in other languages are often regarded as a \isi{patient voice} and a location \isi{voice}, respectively (\citealt[78]{Huang2016Ata}, for example), but in \ili{Teruku}-\ili{Seediq}, their usage is not so straightforward. Even when the thematic role of the subject is location, if one wants to express a future event, the -\textit{un} form is used, for example. Moreover, even when the thematic role of the subject is patient, the -\textit{an} form may be used if one wants to express a progressive or habitual event. See \citet{Tsukida2012} for more details. 

When the subject is a \isi{pronoun}, it triggers a \isi{nominative} \isi{clitic pronoun} after the first element of the predicate (see example (\ref{e:tsukida:2})). 

There are cases where the thematic role of a \textit{ka}-marked NP and the verb form do not correspond. For example, the verb is in \isi{AV} form but the thematic role of the \textit{ka}-marked NP is patient. In such cases the \textit{ka}-marked NP cannot be regarded as the subject. Such cases exist, though they are rare.

In addition to triggering the \isi{nominative} \isi{clitic pronoun}, subjects show several distinctive morpho-syntactic properties (see \citealt{Tsukida2009} for details). 

\subsection{\label{s:tsukida:1.3}Non-subject arguments}

An NP belonging to the A group (the group of NPs with thematic roles that trigger \isi{AV}) is realized by the genitive \isi{clitic} if it is a \isi{pronoun}, when it is not the subject. The A arguments in (\ref{e:tsukida:4}) and (\ref{e:tsukida:5}) are not subjects. If one replaces them with pronouns, they would be as (\ref{e:tsukida:6}) and (\ref{e:tsukida:7}), respectively.

\begin{exe}
	\ex{GV, with pronominal A}\label{e:tsukida:6}\\
	\gll \textit{Kerut}-\textit{un}=\textit{na} \textit{ka} \textit{bunga}.\\
	cut-\textsc{gv1}=\textsc{3sg.gen}  \textsc{sbj} sweet.potato\\
	\glt ‘He/She will cut the sweet potato.’
\end{exe}

\begin{exe}
	\ex{CV, with pronominal A}\label{e:tsukida:7}\\
	\gll \textit{Se}-\textit{kerut}=\textit{na} \textit{bunga} \textit{ka} \textit{yayu}.\\
	\textsc{cv}-cut=\textsc{3sg.gen} sweet.potato \textsc{sbj} knife\\
	\glt ‘The/An old woman cut the/a sweet potato with the knife.’
\end{exe}

\noindent
As for nouns, they do not have a distinct genitive form; the genitive form is the same as the neutral form. We can regard \textit{payi} ‘old woman’ in (\ref{e:tsukida:4}) and (\ref{e:tsukida:5}) as a genitive form, though it is formally the same as a neutral form. 

When it is not the subject, an NP belonging to the G or C group (the group of NPs with thematic roles that trigger GV or CV) is realized as \textsc{oblique}. The oblique form, however, is not distinct from the neutral form, except for pronouns and nouns with high animacy. For pronouns and NPs with high animacy, the oblique form usually involves the suffix -\textit{an}. The occurrences of \textit{bunga} ‘sweet potato’ in (\ref{e:tsukida:1}), (\ref{e:tsukida:2}), (\ref{e:tsukida:3}), (\ref{e:tsukida:5}) and (\ref{e:tsukida:7}) are regarded as oblique forms, though they are homophonous with neutral forms. \textit{Sediq}-\textit{an} ‘person-\textsc{obl}’ in (\ref{e:tsukida:8}) is an example of a distinct oblique form.

\begin{exe}
	\ex{CV}\label{e:tsukida:8}\\
	\gll \textit{Se}-\textit{kerut}=\textit{na}  \textit{sediq}-\textit{an} \textit{ka} \textit{yayu} \textit{niyi}.\\
	\textsc{cv}-cut=\textsc{3sg.gen} person-\textsc{obl} \textsc{sbj} knife \textsc{prox}\\
	\glt ‘He/She cuts the/a person with this knife.’
\end{exe}

\subsection{\label{s:tsukida:1.4}The \textit{'u} particle}

As mentioned in \sectref{s:tsukida:1.1}, an NP may appear preceding a matrix \isi{clause}, followed by the particle \textit{'u} and a non-final pause. The particle may be \textit{ga}, \textit{de'u}, or \textit{dega}, in addition to \textit{'u}. \textit{Ga} is interchangeable with \textit{'u}. I will use only \textit{'u} in the remainder of the paper. \textit{De'u} and \textit{dega} are interchangeable, as they are derived from \textit{'u} and \textit{ga} by affixation of \textit{de}-. \textit{De'u} and \textit{dega} are used differently from \textit{'u} and \textit{ga}. I will not treat the use of \textit{de'u} and \textit{dega} in this paper. 

What can precede \textit{'u} is not only an NP but also a \isi{clause}. In example (\ref{e:tsukida:9}), two clauses are connected by \textit{‘u}.

\begin{exe}
	\ex{Clause-A 'u, clause-B.}\label{e:tsukida:9}\\
	\gll \textit{M}-\textit{iyah}=\textit{su} \textit{hini} \textit{'u}, \textit{me}-\textit{qaras}=\textit{ku}.\\
	\textsc{av}-come=\textsc{2sg.nom} here \textsc{prt}  \textsc{av}-be.glad=\textsc{1sg.nom}\\
	\glt ‘If you come, I will be glad.’
\end{exe}

\noindent
\textit{'u} is multifunctional, and the two clauses connected by it may have several kinds of relationships. In this sentence, the preposed \isi{clause} is a conditional for the event denoted by the matrix \isi{clause}. One can add an appropriate adverb to express the relationship overtly. In example (\ref{e:tsukida:10}), for example, \textit{nasi} ‘if’ is added to express the conditional meaning overtly.

\begin{exe}
	\ex{With \textit{nasi} ‘if’}\label{e:tsukida:10}\\
	\gll \textit{Nasi}=\textit{su}  \textit{m}-\textit{iyah} \textit{hini} \textit{'u}, \textit{me}-\textit{qaras}=\textit{ku}.\\
	if=\textsc{2sg.nom}  \textsc{av}-come  here  \textsc{prt}  \textsc{av}-be.glad=\textsc{1sg}.\textsc{nom}\\
	\glt ‘If you come here, I will be glad.’
\end{exe}

\subsection{\label{s:tsukida:1.5}Texts}

In the next section, I will investigate the nature of the preposed NPs mentioned in \sectref{s:tsukida:1.1} from the point of view of semantics/function, NP types, \isi{anaphoricity}, and topic persistence, analyzing \ili{Seediq} texts. 

For my analysis, I used parts of \textit{Sufferings of the \ili{Teruku} church} written by Yudaw Pisaw (Tien Shin-de). In these texts, matrix \isi{clause} subjects occur five times more frequently than preposed NPs. I started checking from the beginning and when I reached 3728 words, 262 matrix \isi{clause} subjects and 55 preposed NPs had been found. I therefore stopped counting matrix \isi{clause} subjects but continued counting preposed NPs, up to 7315 words, and found 38 more instances of preposed NPs. In total, I have 93 instances of preposed NPs.

\begin{table}
	\begin{tabularx}{\textwidth}{XXl}
		\lsptoprule
		Words & Matrix subject & Preposed NP\\
		\midrule
		1--3728 & 262 &  55\\
		3729--7315 & -- (stopped counting) & 38\\
		\midrule
		Total & 262 &  93\\
		\lspbottomrule
	\end{tabularx}
	\caption{Matrix subjects and preposed NPs in texts}
	\label{tab:tsukida:1}
\end{table}

\section{\label{s:tsukida:2}Semantics/function of preposed NPs}

It is not the case that one can freely choose an NP to precede the matrix \isi{clause}. Preposing is usually restricted to the following NPs:

\begin{itemize}
\item An NP that is coreferential with the matrix subject (preposed subject)
\item An NP that is coreferential with the A-argument in the matrix \isi{clause} (preposed A)
\item An NP that expresses the place of existence in an existential construction, or the possessor in a possessive construction, or a possessor that is left-dislocated from the matrix subject (preposed possessor, see \sectref{s:tsukida:2.3} for details)
\item An NP that expresses time (preposed time)
\item Choices in alternative questions are preposed even when they are not subjects, A-arguments, times, places, or possessors (preposed alternatives)
\end{itemize}

\noindent
Some preposed NPs apparently do not fit the above criteria. They are not coreferential with a subject, an A argument, or a possessor in the matrix \isi{clause}; nor are they frames or alternatives. These will be discussed in \sectref{s:tsukida:2.6}. 

Among 93 instances of preposed NPs, 66 instances have a preposed NP that is coreferential with the subject of the \isi{clause} (preposed subject); in four instances it is coreferential with an A-argument (preposed A), in one instance it is a possessor (preposed possessor), and in 16 instances it is time (preposed time). There were six instances where it was difficult to judge the exact function of the preposed NP. These 93 instances are summarized in \tabref{tab:tsukida:2}.

\begin{table}
	\begin{tabularx}{\textwidth}{Xll} 
		\lsptoprule
		& Token & \%\\
		\midrule
		Preposed Subject & 66 & 70\%\\
		Preposed A & 4 & 6\%\\
		Preposed Possessor & 1 & 1\%\\
		Preposed Time & 16 & 17\%\\
		Preposed alternative & 0 & 0\%\\
		None of the above & 6 & 6\%\\
		\midrule
		Total & 93 & 100\%\\
		\lspbottomrule
	\end{tabularx}
	\caption{Semantics/function of preposed NP}
	\label{tab:tsukida:2}
\end{table}

\noindent
I will explain each of the above cases in turn.

\subsection{\label{s:tsukida:2.1}Preposed subjects}

One can prepose an NP that is coreferential with the matrix subject, as in (\ref{e:tsukida:3}) and (\ref{e:tsukida:11}). I will call such cases \textsc{preposed subjects}, though preposed NPs are not subjects, actually.

\begin{exe}
	\ex{Preposed subject}\label{e:tsukida:11}\\
	\gll \textit{Niyi\textsubscript{i}} \textit{'u}, \textit{'adi} \textit{'utux}=\textit{ta} {\USOParen}\textit{ka}  \textit{kiya}\textsubscript{i}{\USCParen}.\\
	\textsc{prox} \textsc{prt} \textsc{neg} God=\textsc{1plin.gen} \phantom{(}\textsc{sbj} it\\
	\glt ‘As for this, it is not our God.’
\end{exe}

\noindent
One can omit the matrix subject, as shown in (\ref{e:tsukida:11}).   

When an NP that is coreferential with the subject appears in pre-clausal position, the subject in the regular clause-final position is often omitted. In the text, 13 of 66 preposed subjects had overt subjects, and 53 instances had covert subjects. I will investigate in \sectref{s:tsukida:3.3}, \sectref{s:tsukida:4.3}, and \sectref{s:tsukida:5.3} whether there is any difference between these two cases.

With a verbal predicate, “subject” means the NP that triggers the verb form, appearing in clause-final position preceded by the subject particle \textit{ka}. \textit{Payi} ‘old woman’ in (\ref{e:tsukida:1}), \textit{bunga} ‘sweet potato’ in (\ref{e:tsukida:4}), and \textit{yayu} ‘knife’ in (\ref{e:tsukida:5}) are examples of subjects. For sentence (\ref{e:tsukida:1}), one can put \textit{payi} ‘old woman’ in front of the matrix \isi{clause}, as in (\ref{e:tsukida:12}), but one cannot do the same for \textit{bunga} ‘sweet potato,’ as in (\ref{e:tsukida:13}).

\begin{exe}
	\ex{One can prepose a subject argument}\label{e:tsukida:12}\\
	\gll \textit{Payi} \textit{'u},  \textit{k}{\USSmaller}\textit{em}{\USGreater}\textit{erut} \textit{bunga} {\USOParen}\textit{ka}  \textit{hiya}{\USCParen}.\\
	old.woman \textsc{prt}  <\textsc{av}>cut sweet.potato \phantom{(}\textsc{sbj} \textsc{3sg}\\
	\glt ‘The old woman, she cuts sweet potato.’
\end{exe}

\begin{exe}
	\ex{One cannot prepose a non-subject argument}\label{e:tsukida:13}\\
	\gll {\USStar}\textit{Bunga} \textit{'u}, \textit{k}{\USSmaller}\textit{em}{\USGreater}\textit{erut} \textit{ka} \textit{payi}.\\
	\phantom{*}sweet.potato \textsc{prt} <\textsc{av}>cut \textsc{sbj} old.woman\\
	\glt ‘The old woman cuts sweet potato.’
\end{exe}

\noindent
For sentence (\ref{e:tsukida:4}), one can prepose \textit{bunga} ‘sweet potato’, as in (\ref{e:tsukida:14}).

\begin{exe}
	\ex{GV}\label{e:tsukida:14}\\
	\gll \textit{Bunga} \textit{'u} \textit{kerut}-\textit{un} \textit{payi} {\USOParen}\textit{ka} \textit{bunga}{\USCParen}.\\
	sweet.potato \textsc{prt} cut-\textsc{gv1} old.woman \phantom{(}\textsc{sbj} sweet.potato\\
	\glt ‘The/An old woman will cut the sweet potato.’
\end{exe}

\noindent
A preposed subject is not an actual subject, as I have indicated, but only coreferential with the matrix subject. It seems to share some of the subject properties mentioned in \sectref{s:tsukida:1.2}, however. A \isi{nominative} \isi{clitic} that corresponds to the preposed NP appears in the main \isi{clause}, for example.

\begin{exe}
	\ex{One can prepose a subject argument}\label{e:tsukida:15}\\
	\gll \textit{Yaku} \textit{'u}, \textit{k}{\USSmaller}\textit{em}{\USGreater}\textit{erut}=\textit{ku} \textit{bunga}  {\USOParen}\textit{ka}  \textit{yaku}{\USCParen}.\\
	\textsc{1sg} \textsc{prt} <\textsc{av}>cut=\textsc{1sg.nom} sweet.potato \phantom{(}\textsc{sbj} \textsc{1sg}\\
	\glt ‘As for me, I cut sweet potato.’
\end{exe}

\noindent
We cannot tell, however, whether it is the preposed NP \textit{yaku} or the subject of the matrix \isi{clause} that triggers the \isi{clitic}. 

When the preposed NP is coreferential with the matrix subject, the matrix subject may either appear again or be omitted. For 13 of 66 preposed subject, matrix \isi{clause} subject is overt and for 53 it was not overt. When it is overt, it is realized either by \textit{hiya}, the 3\textsuperscript{rd} person singular \isi{pronoun}, by \textit{dehiya}, the 3\textsuperscript{rd} person plural \isi{pronoun}, or by \textit{kiya} ‘it, so’. In example (\ref{e:tsukida:16}), \textit{Karaw Wacih} (proper name) appears in preposed position and at the end of the matrix \isi{clause}, it is referred to again by a \isi{pronoun} \textit{ka hiya}.

\begin{exe}
	\ex{Preposed subject with overt matrix subject}\label{e:tsukida:16}\\
	\gll \textit{Si'ida} \textit{ka} \textit{Karaw} \textit{Watih} \textit{niyi} \textit{'u} \textit{adi} \textit{ka} \textit{balay}=\textit{bi} \textit{senehiyi} \textit{Yisu} \textit{Kiristu} \textit{ka} \textit{hiya} \textit{niyana}.\\
	then \textsc{lnk} Karaw Wacih \textsc{prox} \textsc{prt} \textsc{neg} \textsc{lnk} really=really \textsc{av}.believe Jesus Christ \textsc{sbj} \textsc{3sg} yet\\
	\glt ‘At that time Karaw Wacih, he did not really believed in Gospel yet.’
\end{exe}

\noindent
I will call this pattern \textit{S 'u, VX ka hiya}. In this denotation, \textit{hiya} represents \textit{hiya}, \textit{dehiya} and \textit{kiya}. The pattern where the matrix subject is not overt, on the other hand, will be labeled \textit{S 'u, VX}.

\subsection{\label{s:tsukida:2.2}Preposed A}

Here by “A” I mean the group of NPs which are not the subject but whose thematic roles trigger \isi{AV} when they become the subject. A \isi{pronoun} is in the genitive case when it appears as non-subject, as in (\ref{e:tsukida:6}) and (\ref{e:tsukida:7}). 

An A argument can be preposed to the matrix \isi{clause}. In a sentence like (\ref{e:tsukida:5}), for example, one can prepose \textit{payi} ‘old woman’, as in (\ref{e:tsukida:17}), because its thematic role is agent. The genitive \isi{clitic pronoun} =\textit{na}, which is coreferential with \textit{payi}  appears in the matrix \isi{clause}.

\begin{exe}
	\ex{CV}\label{e:tsukida:17}\\
	\gll \textit{Payi} \textit{'u}, \textit{se}-\textit{kerut}=\textit{na} \textit{bunga} \textit{ka} \textit{yayu}.\\
	old.woman \textsc{prt}  \textsc{cv}-cut=\textsc{3sg.gen} sweet.potato \textsc{sbj} knife\\
	\glt ‘As for the old woman, she will cut the/a sweet potato with the knife.’
\end{exe}

\noindent
There are four such instances among the 95 preposed NPs.

Preposing A is not allowed if there is no X (non-subject argument). In (\ref{e:tsukida:18}) there is only the A argument \textit{Kumu} (a person’s name) as a non-subject argument. From this sentence, one cannot prepose A, as shown in (\ref{e:tsukida:19}).

\begin{exe}
	\ex{A as non-subject argument}\label{e:tsukida:18}\\
	\gll \textit{B}{\USSmaller}\textit{en}{\USGreater}\textit{arig} \textit{Kumu} \textit{ka} \textit{patas} \textit{niyi}.\\
	<\textsc{cv.prf}>buy  Kumu  \textsc{sbj} book \textsc{prox} \\
	\glt ‘Kumu bought this book.’
\end{exe}

\begin{exe}
	\ex{One cannot prepose A}\label{e:tsukida:19}\\
	\gll {\USStar}\textit{Kumu} \textit{'u},  \textit{b}{\USSmaller}\textit{en}{\USGreater}\textit{arig}=\textit{na} \textit{ka}    \textit{patas}   \textit{niyi}.\\
	\phantom{*}kumu  \textsc{prt} <\textsc{cv.prf}>buy=\textsc{3sg.gen} \textsc{sbj} book \textsc{prox}\\
	\glt ‘As for Kumu, she bought this book./As for Kumu, this book was bought by her.’
\end{exe}

\noindent
A time expression is not enough to enable preposing of A. Even if \textit{sehiga} ‘yesterday’ is X, one cannot prepose A, as shown in (\ref{e:tsukida:20}).

\begin{exe}
	\ex{One cannot prepose A}\label{e:tsukida:20}\\
	\gll {\USStar}\textit{Kumu} \textit{'u}, \textit{b}{\USSmaller}\textit{en}{\USGreater}\textit{arig}=\textit{na} \textit{sehiga}  \textit{ka} \textit{patas} \textit{niyi}.\\
	\phantom{*}Kumu  \textsc{prt} <\textsc{cv.prf}>buy=\textsc{3sg.gen} yesterday \textsc{sbj} book  this\\
	\glt ‘As for Kumu, she bought this book yesterday.’
\end{exe}

\subsection{\label{s:tsukida:2.3}Preposed possessor}

One of the morphosyntactic properties of subjects is that they license possessor left-dislocation. From a subject NP consisting of a noun and a possessor, one can move out the possessor and put it in front of the \isi{clause}. Examples (\ref{e:tsukida:21}) and (\ref{e:tsukida:22}) exemplify this. In (\ref{e:tsukida:21}) \textit{Masaw} (the name of a male person) is part of the NP \textit{tederuy Masaw} ‘Masaw’s car’, which is the subject of the sentence. In (\ref{e:tsukida:22}), on the other hand, \textit{Masaw} is preposed, and in the subject a genitive \isi{clitic pronoun} is present which is coreferential with the preposed \textit{Masaw}.

\begin{exe}
	\ex\label{e:tsukida:21} 
	\gll \textit{Me}-\textit{gerung}  \textit{ka}   \textit{tederuy}   \textit{Masaw}.\\
	\textsc{av}-be.broken   \textsc{sbj} car Masaw\\
	\glt ‘Masaw’s car is broken.’
\end{exe}

\begin{exe}
	\ex\label{e:tsukida:22} 
	\gll \textit{Masaw}  \textit{'u}, \textit{me}-\textit{gerung} \textit{ka} \textit{tederuy}=\textit{na}.\\
	Masaw  \textsc{prt} \textsc{av}-be.broken  \textsc{sbj} car=\textsc{3sg.gen}\\
	\glt ‘As for Masaw, his car is broken.’
\end{exe}

\noindent
There is one such instance among the 93 preposed NPs.

\subsection{\label{s:tsukida:2.4}Preposed time}

Preposed time sets the temporal frame in which the following expression should be interpreted. An example is (\ref{e:tsukida:23}).

\begin{exe}
	\ex{Preposed time}\label{e:tsukida:23}\\
	\gll \textit{Ya'asa} \textit{diyan} \textit{'u}, \textit{m}-\textit{iyah} \textit{r}{\USSmaller}\textit{em}{\USGreater}\textit{igaw} \textit{s}{\USSmaller}\textit{em}{\USGreater}\textit{bu} \textit{ka} \textit{'asu} \textit{sekiya} \textit{'Amirika} \textit{heki}.\\
	because day:time \textsc{prt} \textsc{av}-come  <\textsc{av}>hang:about	<\textsc{av}>hit  \textsc{sbj}  ship sky America  \textsc{reason}\\
	\glt ‘Because in the daytime, American airplanes come to make an air raid.’
\end{exe}

\noindent
There are 16 such instances among the 93 preposed NPs.

\subsection{\label{s:tsukida:2.5}Preposed alternatives}

For alternative questions, alternatives are preposed, as in (\ref{e:tsukida:24}).

\begin{exe}
	\ex{Preposed Alternatives}\label{e:tsukida:24}\\
	\gll \textit{Deha} \textit{niyi} \textit{'u}, \textit{'ima}  \textit{ka} \textit{sewayi}=\textit{su}{\USQMark}\\
	two \textsc{prox} \textsc{prt} who \textsc{sbj} younger:sibling=\textsc{2sg}.\textsc{gen}\\
	\glt ‘Between these two, who is your younger sibling?’
\end{exe}

\noindent
There was no such example in the text I used in this investigation.

\subsection{\label{s:tsukida:2.6}None of the above}

Some preposed NPs apparently do not belong to any of the classifications listed above. Such NPs rarely become subjects, but examples do exist. In (\ref{e:tsukida:25}), \textit{saw niyi} ‘such things’ is a patient, not A, of the verb \textit{me}-\textit{kela} ‘to know’, notionally. It is not a subject, either, because the predicate verb is \isi{AV}, not GV, which would have a patient as its subject. Nor is it a possessor, time or expression of alternatives.

\begin{exe}
	\ex{None of the above}\label{e:tsukida:25}\\
	\gll \textit{Saw}   \textit{niyi}  \textit{'u},  \textit{dima}  \textit{me}-\textit{kela}     \textit{ka}  \textit{nihung}.\\
	like   \textsc{prox} \textsc{prt}   already \textsc{av}-know   \textsc{sbj} Japan\\
	\glt ‘As for such things, Japan already knew.’
\end{exe}

\noindent
There were six such instances in the text I analyzed.

\subsection{\label{s:tsukida:2.7}Discussion}

Let us examine here whether these preposed NPs are “topics” of some sort. 

Preposed subjects, preposed As, and preposed possessors, shown in \sectref{s:tsukida:2.1}, \sectref{s:tsukida:2.2} and \sectref{s:tsukida:2.3}, seem to serve as aboutness topics \citep[40--41]{Krifka2007}. They indicate what the information denoted by the matrix \isi{clause} (=comment) is about. Examples (\ref{e:tsukida:12}) and (\ref{e:tsukida:17}), for example, are about \textit{payi} ‘old woman’. The matrix \isi{clause} tells the information about the preposed NP. 

Preposed time and preposed alternative, on the other hand, seem to serve as frame setters. The \isi{frame setter} indicates that the information actually provided is restricted to the particular dimension specified \citep[47]{Krifka2007}. In example (\ref{e:tsukida:23}), the information provided by the matrix \isi{clause} ‘American airplanes come to make an air raid’ is restricted to the particular dimension specified by the preposed NP \textit{diyan} ‘daytime’, for example. 

For those elements that are neither preposed subjects, preposed As, preposed possessors, preposed time or preposed alternatives, still more research is necessary, but as for example (\ref{e:tsukida:25}), the preposed NP, \textit{saw niyi} ‘such things’ seems to serve to set frame for the information provided by the matrix \isi{clause}, ‘Japan already knew’.

\section{\label{s:tsukida:3}NP types}

In this section, I will investigate the composition of preposed NPs.

\subsection{\label{s:tsukida:3.1}Preliminaries}

In \ili{Seediq} NPs, a noun (a common noun or a proper noun) or a \isi{pronoun} usually functions as the head, but demonstratives and VPs can also function as referential expressions, without any affix or particle. \ili{Seediq} is quite flexible, in the sense of \citet{vanLier2013}.

\begin{exe}
	\ex{A noun as a referential expression}\label{e:tsukida:26}\\
	\gll \textit{Q}{\USSmaller}\textit{em}{\USGreater}\textit{ita}=\textit{ku} \textit{kuyuh}  \textit{ka} \textit{yaku}.\\
	<\textsc{av}>see=\textsc{1sg.nom} woman \textsc{sbj} \textsc{1sg}\\
	\glt ‘I saw a/the woman.’
\end{exe}

\begin{exe}
	\ex{A \isi{demonstrative} as a referential expression}\label{e:tsukida:27}\\
	\gll \textit{Me}-\textit{gerung} \textit{ka}  \textit{gaga}.\\
	\textsc{av}-be:broken \textsc{sbj} \textsc{dist}\\
	\glt ‘That is broken.’
\end{exe}

\noindent
VPs also are used as referential expressions without any additional affixation or particle. \isi{AV} forms mean ‘one who does the action denoted by the verb’/‘one who bears the state denoted by the verb’; GV forms mean ‘the object to which the action denoted by the verb is done’/‘the place where the action denoted by the verb takes place’/‘the recipient who recieves something’/‘the goal toward which the action denoted by the verb aims’/etc.; CV forms mean ‘the instrument by which the action is done’/‘the beneficiary for whom the action is done’/‘the object that is transferred’/etc.” A CV-perfect form may mean ‘what has been done as the action denoted by the verb’ (see (\ref{e:tsukida:48})). \textit{Mpe-tegesa kari} ‘the one who teaches language’ in (\ref{e:tsukida:28}) is an example of an \isi{AV} form used as a referential expression.

\begin{exe}
	\ex{A VP as a referential expression}\label{e:tsukida:28}\\
	\gll \textit{Q}{\USSmaller}\textit{em}{\USGreater}\textit{ita}=\textit{ku}  \textit{mpe}-\textit{tegesa} \textit{kari} \textit{ka}  \textit{yaku}.\\
	<\textsc{av}>see=\textsc{1sg.nom} \textsc{av}.\textsc{fut}-teach language \textsc{sbj} \textsc{1sg}\\
	\glt ‘I saw the language teacher.’
\end{exe}

\noindent
When the head is a noun, it can be modified by demonstratives, numbers, genitive pronouns, another noun, or relative clauses. Modifiers, except for numbers, typically follow the head. 

\begin{exe}
	\ex{A \isi{demonstrative} modifying a noun}\label{e:tsukida:29}\\
	\gll \textit{Q}{\USSmaller}\textit{em}{\USGreater}\textit{ita}=\textit{ku}  \textit{kuyuh} \textit{gaga} \textit{ka} \textit{yaku}.\\
	<\textsc{av}>see=\textsc{1sg.nom} woman \textsc{dist} \textsc{sbj} \textsc{1sg}\\
	\glt ‘I saw that woman.’
\end{exe}

\begin{exe}
	\ex{A number modifying a noun}\label{e:tsukida:30}\\
	\gll \textit{Me}-\textit{gerung} \textit{ka}  \textit{teru} \textit{tederuy}.\\
	\textsc{av}-be:broken \textsc{sbj} three car\\
	\glt ‘The three cars are broken.’
\end{exe}

\begin{exe}
	\ex{A genitive \isi{pronoun} modifying a noun}\label{e:tsukida:31}\\
	\gll \textit{Q}{\USSmaller}\textit{em}{\USGreater}\textit{ita}=\textit{ku} \textit{kuyuh}=\textit{na} \textit{ka} \textit{yaku}.\\
	<\textsc{av}>see=\textsc{1sg.nom} wife=\textsc{3sg}.\textsc{gen} \textsc{sbj} \textsc{1sg}\\
	\glt ‘I saw his wife.’
\end{exe}

\begin{exe}
	\ex{Another noun modifying a noun}\label{e:tsukida:32}\\
	\gll \textit{Me}-\textit{gerung} \textit{ka}  \textit{sapah}   \textit{qehuni}.\\
	\textsc{av}-be:broken \textsc{sbj} house  wood\\
	\glt ‘The wooden house is broken.’
\end{exe}

\begin{exe}
	\ex{A \isi{relative clause} modifying a noun}\label{e:tsukida:33}\\
	\gll \textit{Q}{\USSmaller}\textit{em}{\USGreater}\textit{ita}=\textit{ku} \textit{kuyuh} \textbf{\textit{mpe}}-\textbf{\textit{tegesa}} \textit{kari} \textit{ka}  \textit{yaku}.\\
	<\textsc{av}>see=\textsc{1sg.nom} woman \textsc{av}.\textsc{fut}-teach  language \textsc{sbj} \textsc{1sg}\\
	\glt ‘I saw the woman who will teach language.’
\end{exe}

\noindent
I classified NPs into the following types, and investigated how they appear in preposed position. I also investigated their distribution as subjects and made comparisons.

\begin{itemize}
	\item Personal \isi{pronoun} (PP)
	\item Demonstrative (D)
	\item Proper noun + \textit{niyi} (this) (PD)
	\item Common noun + \textit{niyi} (this) (CD)
	\item Bare proper noun (BPN)
	\item Common noun + Genitive (CG)
	\item Common noun + Relative \isi{clause}/Attributive (CR)
	\item Common noun (CN)
	\item \textit{Saw} nominal (\textit{Saw}) ‘Things like …’
	\item VP
\end{itemize}

\noindent
This classification and ordering are based in part on \sloppy{\posscitet{Gundel1993}} classification which claims that the form of a given NP reflects the \isi{givenness} of the NP and show the Givenness Hierarchy \citep[275]{Gundel1993}.  

\begin{figure}
	\tabcolsep=0.05cm
	\begin{tabular}{ccccccccccc}
		&&&&&& uniquely &&&& type\\
		in focus & > & activated & > & familiar & > &identifiable& > & referential & > &identifiable\\
		\{\textit{it}\} && $\begin{array}{@{}l@{}c@{}}
		\quad&\left\{
		\begin{tabular}{@{}l@{}}
		\textit{that} \\
		\textit{this} \\
		\textit{this} N
		\end{tabular}
		\right\}
		\end{array}$ && \{\textit{that} N\}  && \{\textit{the} N\}  && \{indefinite \textit{this} N\}  && \{\textit{a} N\}\\
	\end{tabular}
	\caption{Givenness hierarchy \citep[275]{Gundel1993}}
	\label{fig:tsukida:1}
\end{figure}

\noindent
In \ili{Seediq}, too, a personal \isi{pronoun} (PP) may reflect \textit{in focus} status, which is at the left-most position in the hierarchy, \isi{demonstrative} (D) may reflect \textit{activated} status, common noun + \isi{demonstrative} (CD) may reflect \textit{activated} or \textit{referential} status, and so on. As \ili{Seediq} does not have a definite article or an indefinite article, it is not easy to identify each NP as uniquely identifiable or type identifiable

At this stage, it is an open question whether and how the different composition in \ili{Seediq} NPs reflects \isi{givenness}. This is still under investigation, and the classification above is only a preliminary step. Though still to be examined more thoroughly, I tried ordering them so that it would correspond, at least partially, to the \isi{givenness} hierarchy by \citet{Gundel1993}. 

I will explain my classification and ordering below.

First is an NP consisting of a personal \isi{pronoun}. In \citet{Gundel1993}, this is an indication of \textit{in focus} status. An example of a personal \isi{pronoun} preposed to a matrix \isi{clause}:

\begin{exe}
	\ex{Personal \isi{pronoun} (PP)}\label{e:tsukida:34}\\
	\gll \textit{Hiya}  \textit{'u},  \textit{ya'a}=\textit{bi}       \textit{teru} \textit{'idas}  \textit{ka}  \textit{ne}-\textit{niq}-\textit{un}=\textit{na}    \textit{hiya}   \textit{ni}\\
	\textsc{3sg} \textsc{prt} \textsc{uncertn}=really   three  month \textsc{sbj} \textsc{fut}-live-\textsc{gv1}=\textsc{3sg}.\textsc{gen} there  and\\
	\glt ‘Then she, it was about three months that she would live there, and ….’
\end{exe}

\noindent
Next is an NP consisting of a \isi{demonstrative}. In \citet{Gundel1993}, this is an indication of \textit{activated} status. An example of a \isi{demonstrative} preposed to a matrix \isi{clause}:

\begin{exe}
	\ex{Demonstrative (D)}\label{e:tsukida:35}\\
	\gll \textit{Niyi}  \textit{'u},   \textit{'adi}   \textit{'utux}=\textit{ta} \textit{ka} \textit{niyi}.\\
	\textsc{prox}   \textsc{prt}   \textsc{neg} God=\textsc{1pi.gen}  \textsc{sbj}  \textsc{prox}\\
	\glt ‘This, this is not our God.’
\end{exe}

\noindent
I also treated the time expression \textit{si'ida} ‘then, at that time’ as \isi{demonstrative}.

\begin{exe}
	\ex{Demonstrative (D)}\label{e:tsukida:36}\\
	\gll \textit{Si'ida} \textit{'u},   \textit{me-seseli}   \textit{ngangut}   \textit{sapah}   \textit{'Umih}   \textit{Yadu}.\\
	then \textsc{prt}{\textbackslash} \textsc{av}-gather outside house \textsc{pn} \textsc{pn}\\
	\glt ‘At that time, they gathered outside of the house of Umih Yadu.’
\end{exe}

\noindent
Next is an NP consisting of a proper name and a \isi{demonstrative} (PD). \citet{Gundel1993} do not include such a form. An example of a proper noun modified by a \isi{demonstrative} appearing in front of the matrix \isi{clause}:

\begin{exe}
	\ex{Proper noun + \textit{niyi} (PD)}\label{e:tsukida:37}\\
	\gll \textit{Ya'asa} \textit{Tiwang} \textit{niyi}  \textit{'u},   \textit{m}-\textit{en}-\textit{niq}  \textit{degiyaq}   \textit{qawgan}.\\
	because Ciwang \textsc{prox} \textsc{prt} \textsc{av}-\textsc{prf}-live  mountain Qawgan\\
	\glt ‘Because this Ciwang, she was a \ili{Teruku}, and she had lived at Mt. Qawgan.’
\end{exe}

\noindent
Proper nouns modified by demonstratives seem to be \textit{activated} or \textit{familiar}.

Next is an NP consisting of a common noun and a \isi{demonstrative}. \citet{Gundel1993} point out the functional difference between \textit{this} and \textit{that} from the point of view of information structure. According to them, \textit{this N} is an indication of \textit{activated} status, but \textit{that N} indicates \textit{familiarity}. I examined the corpus to determine whether the corresponding expressions in \ili{Seediq}, \textit{N niyi} (\textit{this N}) and \textit{N gaga} (\textit{that N}), show a similar difference, but I found only instances of \textit{N niyi}. There are no instances of \textit{N gaga} in the text, but \textit{N gaga} is often found in elicitation. An example of a preposed CD (common noun modified by a \isi{demonstrative}):

\begin{exe}
	\ex{Common noun + \textit{niyi} (CD)}\label{e:tsukida:38}\\
	\gll \textit{Budi} \textit{pe}-\textit{patas}=\textit{mu} \textit{niyi} \textit{'u}, \textit{biq}-\textit{i}=\textit{ku} \textit{haya}   \textit{kuyuh}=\textit{mu}   \textit{rubiq}   \textit{wilang}  \textit{ha}.\\
	bamboo \textsc{fut}-\textsc{cv}.write=\textsc{1sg}.\textsc{gen} \textsc{prox} \textsc{prt} give-\textsc{gv.nfin}=\textsc{1sg}.\textsc{nom} \textsc{ben} wife=\textsc{1sg}.\textsc{gen} \textsc{pn} \textsc{pn} \textsc{gentle}\\
	\glt ‘As for this fountain pen, please give it to my wife Rubiq Wilang.’
\end{exe}

\noindent
Common nouns modified by demonstratives (CDs) are supposed to be \textit{activated}, \textit{familiar} or \textit{uniquely identifiable}. 

Next is an NP consisting of a proper name. An example:

\begin{exe}
	\ex{Bare proper noun (BPN)}\label{e:tsukida:39}\\
	\gll \textit{Kiya} \textit{de'u},  \textit{nihung}  \textit{'u}, \textit{hengkawas}  \textit{19,14}   \textit{m}-\textit{adas}  \textit{hebaraw}=\textit{bi} \textit{merata}  \textit{ni}  \textit{kensat}  \textit{m}-\textit{iyah}    \textit{k}{\USSmaller}\textit{em}{\USGreater}\textit{eremux}  \textit{dexegal}  \textit{Teruku}.\\
	so \textsc{prt} Japan \textsc{prt} year 1914 \textsc{av}-bring many=really soldier and police \textsc{av}-come  <\textsc{av}>invade land  \ili{Teruku}\\
	\glt ‘Then, Japan, in 1914, brought soldiers and police and came to invade the Truku territory.’
\end{exe}

\noindent
Things or persons denoted by proper names are at least \textit{uniquely} \textit{identifiable}. 

Next is an NP consisting of a common noun and a genitive \isi{pronoun}. The genitive expresses a possessor, as illustrated in (\ref{e:tsukida:40}). 

\begin{exe}
	\ex{Common noun + Genitive \isi{pronoun} (CG)}\label{e:tsukida:40}\\
	\gll \textit{Bubu=na} \textit{'u}, \textit{wada} \textit{m}-\textit{arig}  \textit{bawa}   \textit{da}.\\
	mother=\textsc{3sg}.\textsc{gen} \textsc{prt} be:gone \textsc{av}-buy  bun \textsc{ns}\\
	\glt ‘His/Her mother went to buy a bun.’
\end{exe}

\noindent
Common nouns modified by genitive pronouns (CGs) are supposed to be \textit{uniquely identifiable} or \textit{referential}. 

Next is an NP consisting of a common noun and a \isi{relative clause} or attributive expression. In \ili{Seediq}, a noun may be modified by another noun, and the relationship between them is possessor-possessed, material-thing (e.g., paper napkin), purpose-thing (e.g., traveling shoes), whole-part (e.g., face tattoo), product-producer (e.g., his book, meaning ‘the book he wrote’), etc. I collectively call those meanings that are expressed by the modifiers illustrated “attributive”. “Relative \isi{clause}” is actually a VP modifying a noun (see (\ref{e:tsukida:42})). This type, therefore, is a class of NPs consisting of a noun and another noun or a VP modifying it. Examples are: 

\begin{exe}
	\ex{Common noun + Other noun (CR)}\label{e:tsukida:41}\\
	\gll \textit{Kana} \textit{ka} \textit{mensewayi} \textit{de}-\textit{senehiyi} \textit{'uri} \textit{'u}, \textit{tetegeli'ing}   \textit{m}-\textit{usa}   \textit{me}-\textit{seseli} \textit{tehemuku}.\\
	all \textsc{lnk} brother \textsc{pl}-believer  also \textsc{prt} \textsc{av}.hide \textsc{av}-go \textsc{av}-gather \textsc{av}.worship\\
	\glt ‘The brothers and sisters of believers went together hiding to worship as well.’
\end{exe}

\begin{exe}
	\ex{Common noun + VP (CR)}\label{e:tsukida:42}\\
	\gll \textit{Kensat} \textit{nihung} \textit{m}-\textit{eniq}  \textit{Sekadang}  \textit{'u},  \textit{Matsudo}-\textit{sang}   \textit{ni}  \textit{Motoyoshi}-\textit{singsi}  \textit{2}   \textit{hiyi}.\\
	police Japan \textsc{av}-live Sekadang \textsc{prt} Matsudo-Mr. and Motoyoshi-teacher  two  body\\
	\glt ‘The \ili{Japanese} police who were at Skadang, they were Mr. Matsudo and Teacher Motoyoshi, two people.’
\end{exe}

\noindent
Common nouns modified by another noun or VP (CRs) are also supposed to be \textit{uniquely identifiable}, \textit{referential} or type \textit{identifiable}. 

Next is an NP consisting of a common noun alone, as illustrated in (\ref{e:tsukida:43}): 

\begin{exe}
	\ex{Common noun (CN)}\label{e:tsukida:43}\\
	\gll \textit{Ke'man} \textit{'u}, \textit{pengkeku'ung} \textit{sengkekingal} \textit{m}-\textit{usa}  \textit{'ayug}  \textit{daya}  \textit{'alang}=\textit{deha}  \textit{ka}  \textit{se'diq}  \textit{senehiyi}  \textit{Kiristu}.\\
	night \textsc{prt} in:the:darkness  one:by:one \textsc{av}-go  stream  upwards  village=\textsc{3pl}.\textsc{gen} \textsc{sbj} person \textsc{av}-believe  Christ\\
	\glt ‘At night, people who believed in Christ went to the upper stream of their village, in the darkness, one by one.’
\end{exe}

\noindent
Bare common nouns may be \textit{uniquely identifiable}, \textit{referential}, or \textit{type identifiable}. \ili{Seediq} does not have any definite article or indefinite article, so it is often hard to tell.

Next is \textit{saw + noun}. An NP is preceded by \textit{saw}, a preposition meaning ‘like …’; \textit{saw NP} means ‘things like an NP’ or ‘things concerning NP.’ Sometimes \textit{ka} appears between \textit{saw} and \textit{NP}. An example:

\begin{exe}
	\ex{\textit{Saw} nominals}\label{e:tsukida:44}\\
	\gll \textit{Saw} \textit{ka}  \textit{qaya}  \textit{samat} \textit{'u}, \textit{des}-\textit{un}=\textit{deha}  \textit{be'nux},   \textit{seberig}-\textit{an}   \textit{kelemukan}.\\
	like \textsc{lnk} thing wild:animal \textsc{prt} bring-\textsc{gv1}=\textsc{3pl}.\textsc{gen} plain:land  sell-\textsc{gv2} \ili{Taiwanese}\\
	\glt ‘As for things concerning wild animals (animal skins, bones, horns, and the like), they brought them to the plains and sold them to \ili{Taiwanese}.’
\end{exe}

\noindent
\textit{Saw} + \textit{noun} would correspond to \textit{type identifiable} in \isi{givenness} hierarchy by \citet{Gundel1993}.

The last is VP. A VP can function as a referential expression, as in example (\ref{e:tsukida:28}). This type of expression may appear as the subject or as a non-subject argument, but rarely as a preposed element. (\ref{e:tsukida:45}) is another example of a VP functioning as a subject phrase. In this example, a VP \textit{n}-\textit{arig}=\textit{mu} \textit{sehiga} ‘<\textsc{cv}.\textsc{prf}>buy=\textsc{1sg}.\textsc{gen} yesterday’ means ‘what I bought yesterday.’ 

\begin{exe}
	\ex{VP as matrix subject}\label{e:tsukida:45}\\
	\gll \textit{Me}-\textit{gerung}   \textit{ka}   \textit{n}-\textit{arig}=\textit{mu} \textit{sehiga}.\\
	\textsc{av}-be:broken \textsc{sbj} \textsc{cv}.\textsc{prf}-buy=\textsc{1sg}.\textsc{gen} yesterday\\
	\glt ‘What I bought yesterday is broken.’
\end{exe}

\noindent
Such VPs used as referential expression may be \textit{uniquely identifiable}, \textit{referential} or \textit{type identifiable}, depending on the adjuncts contained in them.

\subsection{\label{s:tsukida:3.2}Results}

I investigated what type of NP appears for each type of preposed NP classified in \sectref{s:tsukida:2}. The results are shown in \tabref{tab:tsukida:3}.

\begin{table}[h]
\begin{tabularx}{\textwidth}{Xrrrrrr}
	\lsptoprule
	& \multicolumn{4}{c}{Preposed} &  Others &  Sum\\ \cmidrule{2-5}
	&  Subject &  Time &  A &  Po &  & \\ 
	\midrule
	 PP &  2 &  0 &  0 &  0 &  0 &  2\\
	 D &  6 &  10 &  0 &  0 &  2 &  18\\
	 PD &  12 &  0 &  0 &  0 &  0 &  12\\
	 CD &  4 &  0 &  0 &  0 &  0 &  4\\
	 BPN &  20 &  0 &  2 &  0 &  0 &  22\\
	 CG &  6 &  0 &  1 &  1 &  0 &  8\\
	 CR &  4 &  0 &  0 &  0 &  1 &  5\\
	 CN &  4 &  6 &  1 &  0 &  0 &  11\\
	 \textit{Saw}+ noun &  7 &  0 &  0 &  0 &  3 &  12\\
	 VP &  1 &  0 &  0 &  0 &  0 &  1\\
	 Interrogative &  0 &  0 &  0 &  0 &  0 &  0\\
	 \midrule
	 Total &  66 &  16 &  4 &  1 &  6 & \\
	\lspbottomrule
\end{tabularx}
	\caption{Types of preposed NPs, in terms of coreference}
	\label{tab:tsukida:3}
\end{table}

There are several things I can point out from the results above. 

Personal pronouns rarely appear preposed. There are only two instances of them.

For preposed subjects, person names, accompanied by a \isi{demonstrative} (PD) or not (BPN), appear more often than the other types of NPs. 

For preposed time, there are 10 instances of \textit{si'ida} ‘then, at that time.’ These are classified as \isi{demonstrative}. The other six instances are bare common nouns (CN). 

Among those six instances that were not S, A, Po, or Time, three are of the type \textit{saw} + noun ‘things concerning \textit{noun}.’ It seems easier for speakers to ignore the semantic relationship between the preposed NP and the \isi{voice} of the predicate verb of the matrix \isi{clause} when the preposed NP is \textit{saw} + noun. Let us examine here the three cases more precisely. One is example (\ref{e:tsukida:25}). The other two are (\ref{e:tsukida:46}) and (\ref{e:tsukida:47}).

\begin{exe}
	\ex{Saw + noun}\label{e:tsukida:46}\\
	\gll \textit{Saw} \textit{niyi} \textit{'u}, \textit{berah} \textit{'ini} \textit{'iyah} \textit{dexegal} \textit{Taywan}   \textit{hini}   \textit{ka}   \textit{Nihung}   \textit{han}.\\
	like \textsc{prox} \textsc{prt} before \textsc{neg} \textsc{av}.\textsc{nfin}.come  land  Taiwan  here \textsc{sbj} Japan  temporarily\\
	\glt ‘As for such things, it was before Japan came here to the land of Taiwan.’
\end{exe}

\begin{exe}
	\ex{Saw + noun}\label{e:tsukida:47}\\
	\gll \textit{Saw}   \textit{niyi}   \textit{'u},   \textit{'adi}=\textit{nami}   \textit{t}{\USSmaller}\textit{em}{\USGreater}\textit{egesa}   \textit{ka}   \textit{yami}   \textit{kensat}   \textit{Nihung}  \textit{'uri}   \textit{pini},   \textit{mesa}.\\
	like \textsc{prox} \textsc{prt} \textsc{neg}=\textsc{1plex} <\textsc{av}>teach \textsc{sbj}  \textsc{1plex} police  Japan  also \textsc{anger}  \textsc{hearsay}\\
	\glt ‘As for such things, we \ili{Japanese} police do not teach, they say.’
\end{exe}

\noindent
In these examples, \textit{saw niyi} ‘such things’ seems to indicate rather abstract situations, not concrete entities. Such lack of concreteness seems to lead to inconsistency of the \isi{voice} form of the matrix verb. 

Below, I will investigate whether the situation differs for preposed subjects if the matrix subject is overt or not. I will also compare the preposed subject situation with the situation of the matrix subject.

\subsection{\label{s:tsukida:3.3}Comparison}

Among sentences with preposed subjects, some have an overt matrix subject, and some do not. When the matrix subject is overt, the same \isi{referent} is referred to twice, both in the topic position and in the regular subject position (S \textit{'u}, VX \textit{ka} S). What is the function of this double reference? 

The preposed subject is coreferential with the matrix subject. How does the appearance of an NP in preposed position differ from an NP that is the matrix subject, in internal constituency? I will compare the NPs in these cases. 

\subsubsection{\label{s:tsukida:3.3.1}When the matrix subject is overt and when it is not}

Among 66 instances of preposed subjects, 13 instances are \textit{S 'u, XV ka hiya} pattern (with overt matrix \isi{clause} subject), and 53 instances are \textit{S 'u, VX} pattern (without overt matrix subject). The types of preposed NPs are shown in \tabref{tab:tsukida:4}. 

\begin{table}
\begin{tabularx}{\textwidth}{Xrrrr} 
	\lsptoprule
	& \multicolumn{4}{c}{Preposed subjects}\\
	
	& \multicolumn{2}{c}{S 'u, VX ka hiya.} & \multicolumn{2}{l}{S 'u, VX.}\\
	\midrule
	PP &  0 &  0\% &  2 &  4\%\\
	\midrule
	D &  1 &  8\% &  5 &  9\%\\
	PD &  4 &  31\% &  8 &  15\%\\
	CD &  1 &  8\% &  3 &  6\%\\
	BPN &  6 &  46\% &  14 &  26\%\\
	\midrule
	CG &  0 &  0\% &  6 &  11\%\\
	CR &  0 &  0\% &  4 &  8\%\\
	CN &  0 &  0\% &  4 &  8\%\\
	\midrule
	\textit{Saw}+ noun &  1 &  8\% &  6 &  12\%\\
	\midrule
	VP &  0 &  0\% &  1 &  2\%\\
	\midrule
	Total &  13 &  100\% &  53 &  100\%\\
	\lspbottomrule
\end{tabularx}
	\caption{NP types of preposed subjects with overt matrix subjects and those without.} 
	\label{tab:tsukida:4}
\end{table}

There are considerable differences in internal constituency between \textit{S 'u, VX ka hiya} pattern and \textit{S 'u, VX} pattern. 

We can first point out that what can appear as S of the \textit{S 'u, VX ka hiya} pattern is very limited. Most of them (10 of 13) are PD or BPN. There is one example each of D, CD and \textit{saw} + \textit{noun}. There are no examples of PP, CR, CG, CN or VP. 

As for the PP (personal \isi{pronoun}) type, for all preposed subjects there are very few instances, as we saw in \sectref{s:tsukida:3.2}. The two that occur lack an overt subject. Clauses with a personal \isi{pronoun} as preposed subject and a matrix \isi{clause} with an overt subject seem to be avoided. 

Among the 13 instances of preposed subjects with an overt regular-position subject (\textit{S 'u, VX ka hiya} type), 10 (76\%) are instances of PD or BPN; that is, a proper name, with or without a modifying \isi{demonstrative}, is introduced as the preposed NP and then referred to again in the matrix subject position by a \isi{pronoun}, as in (\ref{e:tsukida:48}).

\begin{exe}
	\ex\label{e:tsukida:48} 
	\gll \textit{Tiwang}  \textit{niyi}  \textit{'u},  \textit{Teruku}   \textit{ka}   \textit{hiya}.\\
	Ciwang  \textsc{prox}  \textsc{prt} \ili{Teruku} \textsc{sbj}  \textsc{3sg}\\
	\glt ‘This Ciwang, she was a \ili{Teruku} person.’
\end{exe}

\noindent
In the \textit{S 'u, VX} pattern, on the other hand, PD and BPN represent only 41 percent of the cases. This is lower than the number of cases with the \textit{S 'u, VX ka hiya} pattern (76\%). 

As for the CR, CG, and CN types (common nouns modified by a VP or another noun, those modified by a genitive \isi{pronoun}, and bare common nouns), there is no instance of the \textit{S 'u, VX ka hiya} pattern. There is only one instance of the CD type (common noun modified by a \isi{demonstrative}). We can thus say that common nouns rarely appear as preposed subjects of the \textit{S 'u, VX ka hiya} pattern. 

For the \textit{S 'u, VX} type, the proportion of common nouns as preposed NPs is higher (17 among 53) than for the \textit{S 'u, VX ka hiya} type. 

As for the \textit{saw} + \textit{noun} type, 6 of 7 instances of them occurred in the \textit{S 'u, VX} type. 

We see in \tabref{tab:tsukida:4} that most instances of \textit{saw} NP \textit{'u} (6 out of 7) occurred in the \textit{S 'u, VX} pattern. 

There is one example of VP in preposed position (see the next section for this example).

\subsubsection{\label{s:tsukida:3.3.2}Preposed subjects and matrix subjects}

Now let us compare preposed subjects and matrix subjects. There are 262 overt matrix subject occurrences in the text. A comparison of NP types in preposed subjects and matrix subjects is given in \tabref{tab:tsukida:5} and visualized in \figref{fig:tsukida:2}.

\begin{table}
\begin{tabularx}{\textwidth}{Xrrrrrr} 
	\lsptoprule
	& \multicolumn{4}{c}{Preposed-Subject} & \multicolumn{2}{c}{Matrix \isi{clause} subjects}\\ \cmidrule{2-5}
	& \multicolumn{2}{l}{S 'u, VX ka hiya.} & \multicolumn{2}{l}{S 'u, VX.} & \multicolumn{2}{l}{ XV ka S}\\
	\midrule
	 PP &  0 &  0\% &  2 &  4\% &  52 &  20\%\\
	 \midrule
	 D &  1 &  8\% &  5 &  9\% &  4 &  2\%\\
	 PD &  4 &  31\% &  8 &  15\% &  10 &  4\%\\
	 CD &  1 &  8\% &  3 &  6\% &  2 &  1\%\\
	 BPN &  6 &  46\% &  14 &  26\% &  64 &  24\%\\
	 \midrule
	 CG &  0 &  0\% &  6 &  11\% &  49 &  19\%\\
	 CR &  0 &  0\% &  4 &  8\% &  10 &  4\%\\
	 CN &  0 &  0\% &  4 &  8\% &  48 &  18\%\\
	 \midrule
	 Saw+noun &  1 &  8\% &  6 &  12\% &  2 &  1\%\\
	 \midrule
	 VP &  0 &  0\% &  1 &  2\% &  21 &  8\%\\
	 \midrule
	 Total &  13 &  100\% &  53 &  100\% &  262 &  100\%\\
	\lspbottomrule
\end{tabularx}
	\caption{NP types of preposed subjects with and without overt regular-position subjects and matrix subjects.}
	\label{tab:tsukida:5}
\end{table}

\begin{figure}
	\includegraphics[width=0.8\textwidth]{figures/tsukida-figure2.png}
	\caption{NP types of preposed subjects with and without overt regular-position subjects and matrix subjects.}
	\label{fig:tsukida:2}
\end{figure}

\noindent
The PP type (personal pronouns) scarcely appears as a preposed subject, but for matrix subjects, the PP type is observed very often (20\%). 

Demonstratives (D), proper names modified by demonstratives (PD), and common nouns modified by demonstratives (CD) appear as preposed subjects (22 out of 66, 33\%), but not as often in regular subject position (16 out of 262, 6\%). 

CD, CR, CG (common nouns modified by demonstratives, another noun, a VP, or a genitive \isi{pronoun}) and CN (bare common nouns) are very common matrix subjects (109 out of 262, 41\%). For preposed subjects, they are much fewer (1 out of 14, 7\%, for S of the \textit{S 'u, VX ka hiya} pattern and 17 out of 52, 32\%, for the S of \textit{S 'u, VX} pattern). 

Personal names, bare (BPN) or with modification by demonstratives (PD), are less common for matrix subjects (74 out of 262, 28\%) than for preposed subjects (10 out of 13, 77\%, for S of the \textit{S 'u, VX ka hiya} pattern and 22 out of 53, 42\%, for S of the \textit{S 'u, VX} pattern). 

VPs can function as referential expressions, as we saw at the beginning of \sectref{s:tsukida:3}. Such expressions appear in regular subject position, but rarely in preposed position. There were 21 instances in which a VP appeared as a matrix subject, but only one of a prepositional phrase in pre-clausal position. Example (\ref{e:tsukida:49}) shows a VP functioning as subject.

\begin{exe}
	\ex{VP as the subject}\label{e:tsukida:49}\\
	\gll \textit{Laqi}=\textit{na} \textit{'Ipay}   \textit{Yudaw}  \textit{ka}   \textit{q}{\USSmaller}\textit{em}{\USGreater}{\USSmaller}\textit{en}{\USGreater}\textit{ita}.\\
	child=\textsc{3sg}.\textsc{gen} Ipay Yudaw \textsc{sbj} <\textsc{av}><\textsc{prf}>see\\
	\glt ‘The one who saw it was her child Ipay Yudaw.’
\end{exe}

\noindent
We can say that NP \textit{ka} VP is similar to a cleft or \isi{pseudocleft construction}. Similar constructions are reported for Philippine languages (\citealt[140]{Himmelmann2005typo}, \citealt[604]{Nagaya2011}), and for many other \ili{Formosan} languages as well. 

A preposed VP is usually interpreted as an adverbial-\isi{clause} predicate with the subject omitted, and not as a referential phrase. This is because \textit{'u} is multifunctional and can mark adverbial clauses as well as preposed subjects, as we saw in \sectref{s:tsukida:1.4}. In the texts, there was one example of a preposed VP that could be interpreted as a referential expression (see example (\ref{e:tsukida:50})). 

\begin{exe}
	\ex{VP preceding the matrix clause}\label{e:tsukida:50}\\
	\gll \textit{P}{\USSmaller}\textit{en}{\USGreater}\textit{le'alay}=\textit{bi}   \textit{senehiyi}   \textit{kari}   \textit{Kiristu}  \textit{ka}  \textit{'alang}  \textit{Besuring}  \textit{'u},  \textit{Talug}   \textit{Payan}.\\
	<\textsc{av.prf}>first=really \textsc{av}.believe   word   Christ \textsc{lnk} village Besuring \textsc{prt} Talug Payab\\
	\glt ‘As for the first one to believe the gospel of Christ in the village of Besuring, it was Talug Payan.’
\end{exe}

\noindent
The VP preceding \textit{'u} in (\ref{e:tsukida:50}), \textit{p}<\textit{en}>\textit{le'alay}=\textit{bi} \textit{senehiyi kari Kiristu ka alang Besuring} cannot be interpreted as an adverbial, meaning ‘as/because/if one was the first to believe the gospel of Christ in the Besuring village’, but only as a referential phrase, meaning ‘the first one to believe the gospel of Christ in the Besuring village’. 

Instances of \textit{saw} + noun ‘such things as \textit{noun}, such things related to \textit{noun}’ are very few for matrix subjects (1 out of 262). We can see more examples in preposed position (1 out of 13 for S of the \textit{S 'u, VX ka hiya} pattern, 6 out 53 for S of the \textit{S 'u, VX} pattern). This distribution seems to be the opposite of that with VPs.

\section{\label{s:tsukida:4}Anaphoricity}

In this section, I will investigate \isi{anaphoricity} in preposed subjects. 

\subsection{\label{s:tsukida:4.1}Preliminaries}

Anaphoricity is an index of the degree to which a \isi{referent} can be said to have a \isi{discourse} antecedent. Following \citet[1687]{Gregory2001}, I apply the label “\isi{anaphoricity}” to an attribute with three possible values:

\begin{itemize}
	\item 0: Tokens containing pre-clausal NPs whose referents have not been mentioned in the preceding \isi{discourse}.
	\item 1: Tokens containing pre-clausal NPs whose referents are members of a set that was previously evoked.
	\item 2: Tokens containing pre-clausal NPs that denote entities that have been mentioned previously in the \isi{discourse}.
\end{itemize}

\noindent
Examples (\ref{e:tsukida:51})–(\ref{e:tsukida:53}) illustrate the three possible \isi{anaphoricity} scores, with referring expressions and their antecedents co-indexed in (\ref{e:tsukida:52}) and (\ref{e:tsukida:53}). 

(\ref{e:tsukida:51}) is an example of an \isi{anaphoricity} score of 0. The preposed NP in this example is \textit{saw ka qaya samat} ‘those things concerning wild animals.’ In the previous text, it is not mentioned at all, so its \isi{anaphoricity} score is 0.

\begin{exe}
	\ex{Anaphoricity score of 0 (not mentioned before)}\label{e:tsukida:51}\\
	\gll \textbf{\textit{Saw}} \textbf{\textit{ka}} \textbf{\textit{qaya}} \textbf{\textit{samat}} \textit{'u},  \textit{des}-\textit{un}=\textit{deha} \textit{be'nux},   \textit{seberig}-\textit{an}   \textit{kelemukan}.\\
	like \textsc{lnk} thing wild:animal \textsc{prt} bring-\textsc{gv1}=\textsc{3pl}.\textsc{gen} plain:land  sell-\textsc{gv2} \ili{Taiwanese}\\
	\glt ‘As for those things concerning wild animals (animal skins, bones, horns, and the like), they brought them to the plain and sold them to \ili{Taiwanese}.’
\end{exe}

\noindent
(\ref{e:tsukida:52}) has an \isi{anaphoricity} score of 1. The preposed NP in example (\ref{e:tsukida:52b}) is \textit{duma} ‘some, others.’ It is a member of \textit{dehiyaan} ‘\textsc{3pl}.\textsc{obl},’ which was mentioned in (\ref{e:tsukida:52a}). 

\begin{exe}
	\ex{Anaphoricity score of 1 (members of a set that was previously evoked)}\label{e:tsukida:52}
	\begin{xlist}
		\ex\label{e:tsukida:52a}
		\gll  \textit{Bitaq} \textit{saw}    \textit{m}-\textit{ahu}     \textit{quyu}  \textit{peqeraqil}  \textit{dehiya'an}\textit{\textsubscript{i}}.\\
		until   like \textsc{av}-wash   snake \textsc{av}.torture \textsc{3pl}.\textsc{obl}\\
		\glt ‘They tortured them up to hitting snake (=expression of harshness).’
		\ex\label{e:tsukida:52b}
		\gll  \textbf{\textit{Duma}}\textit{\textsubscript{i}} \textit{'u},   \textit{pesa}-\textit{'un}=\textit{deha}    \textit{kulu}  \textit{m}-\textit{banah}.\\
		some \textsc{prt} put-\textsc{gv1}=\textsc{3pl}.\textsc{gen} box \textsc{av}-red\\
		\glt ‘Some were put into prison (=red box).’
	\end{xlist}
\end{exe}

\noindent
(\ref{e:tsukida:53}) has an \isi{anaphoricity} score of 2. The preposed NP in example (\ref{e:tsukida:53b}) is \textit{dexegal Taiwan} ‘land of Taiwan.’ It is mentioned in the previous \isi{clause} (\ref{e:tsukida:53a}). 

\begin{exe}
	\ex{Anaphoricity score of 2 (Mentioned before)}\label{e:tsukida:53}
	\begin{xlist}
		\ex\label{e:tsukida:53a}
		\gll  \textit{Saw}   \textit{niyi}  \textit{'u},   \textit{berah}  \textit{'ini}   \textit{'iyah} \textbf{\textit{dexegal}}  \textbf{\textit{Taywan}}\textit{\textsubscript{i}} \textit{hini}  \textit{ka}  \textit{Nihung}  \textit{han}.\\
		like   this \textsc{prt} before \textsc{neg} \textsc{av}.\textsc{nfin}.come  land Taiwan  here \textsc{sbj} Japan temporarily\\
		\glt ‘Things like these, it was before Japan came here to Taiwan.’
		\ex\label{e:tsukida:53b}
		\gll  \textit{Kiya}  \textit{ni}  \textit{pa'ah}  \textit{hengkawas}  \textit{1895}  \textit{siida}, \textbf{\textit{dexegal}}   \textbf{\textit{Taywan}}\textit{\textsubscript{i}} \textit{'u}     \textit{diy}-\textit{un}      \textit{k}{\USSmaller}\textit{em}{\USGreater}\textit{elawa}  \textit{Nihung}.\\
		so     and   from    year       1895   then    land       Taiwan \textsc{prt} have-\textsc{gv1} <\textsc{av}>govern   Japan\\
		\glt ‘Then, from 1895, the land of Taiwan was owned and governed by Japan.’
	\end{xlist}
\end{exe}

\subsection{\label{s:tsukida:4.2}Results}

For preposed subjects, 35 out of 66 had \isi{anaphoricity} scores of 2, 1 out of 7 had a score of 1, and 24 of 66 had scores of 0. The results are summarized in \tabref{tab:tsukida:6}.

\begin{table}
\begin{tabularx}{\textwidth}{Xrr} 
	\lsptoprule
	&  Preposed Subjects &  \%\\
	\midrule
	0 (=no prior mention) &  24 &  36\%\\
	1 (=member of an activated set) &  7 &  11\%\\
	2 (=prior mention) &  35 &  53\%\\
	\midrule
	 Total &  66 &  100\%\\
	\lspbottomrule
\end{tabularx}
	\caption{Anaphoricity of preposed subjects}
	\label{tab:tsukida:6}
\end{table}

\noindent
In nearly half of the instances the score was 2.

\newpage 
\subsection{\label{s:tsukida:4.3}Comparison}

In this section I will compare the \isi{anaphoricity} scores of preposed NPs with and without overt matrix subjects and also with matrix subjects.

\subsubsection{\label{s:tsukida:4.3.1}When the matrix subject is overt and when it is not}

A comparison of the \isi{anaphoricity} of preposed subjects with and without overt matrix subjects is shown in \tabref{tab:tsukida:7}.

\begin{table}
\begin{tabularx}{\textwidth}{Xrrrr} 
	\lsptoprule
	& \multicolumn{4}{c}{Preposed subject}\\ \cmidrule{2-5}
	& \multicolumn{2}{l}{S 'u, VX ka hiya.} & \multicolumn{2}{l}{S 'u, VX.}\\
	\midrule
	0 (=no prior mention) &  7 &  54\% &  17 &  32\%\\
	1 (=member of an activated set) &  0 &  0\% &  7 &  13\%\\
	2 (=prior mention) &  6 &  46\% &  29 &  55\%\\
	\midrule
	Total &  13 &  100\% &  53 &  100\%\\
	\lspbottomrule
\end{tabularx}
	\caption{Anaphoricity of preposed subjects with and without overt  matrix subjects}
	\label{tab:tsukida:7}
\end{table}

\noindent
For S of the \textit{S 'u, XV ka hiya} pattern, the \isi{anaphoricity} score is either 0 or 2; there is no instance of an \isi{anaphoricity} score of 1. Seven out of 13 are examples where the score is 0, the other 6 are examples where the score is 2. For S of the \textit{S 'u, XV ka hiya} pattern, about a half of the instances (29 of 53) had scores of 2. We can say that the proportion of scores of 2 is similar when the matrix subject is present or absent. For scores of 0 and 1, two cases show a difference: for S of the \textit{S 'u, XV} pattern, the \isi{anaphoricity} score is 0 (17 out of 53) or 1 (7 out of 53), whereas there are no instances of \isi{anaphoricity} scores of 1 for S of the \textit{S 'u, XV ka hiya} pattern. 

\subsubsection{\label{s:tsukida:4.3.2}Preposed subjects and matrix subjects}

A comparison of the \isi{anaphoricity} of preposed subjects and matrix subjects is given in \tabref{tab:tsukida:8}, and illustrated in \figref{fig:tsukida:3}. 

\begin{table}
	\begin{tabularx}{\textwidth}{Xrrrrrr} 
		\lsptoprule
		& \multicolumn{4}{l}{Preposed subjects} & \multicolumn{2}{l}{Matrix subjects}\\ \cmidrule{2-5}
		& \multicolumn{2}{l}{S 'u, VX ka hiya.} & \multicolumn{2}{l}{S 'u, VX.} & \multicolumn{2}{l}{VX ka S.}\\
		\midrule
		0 (=no prior mention) &  7 &  54\% &  17 &  32\% &  98 &  37\%\\
		1 (=member of an activated set) &  0 &  0\% &  7 &  13\% &  15 &  6\%\\
		2 (=prior mention) &  6 &  46\% &  29 &  55\% &  149 &  57\%\\
		\midrule
		Total &  13 &  100\% &  53 &  100\% &  262 &  100\%\\
		\lspbottomrule
	\end{tabularx}
	\caption{Anaphoricity of preposed subjects with and without overt  matrix subjects, and matrix subjects.}
	\label{tab:tsukida:8}
\end{table}

\begin{figure}[H]
	\includegraphics[width=0.8\textwidth]{figures/tsukida-figure3.png}
	\caption{Anaphoricity of preposed subjects with and without overt matrix subjects, and matrix subjects.}
	\label{fig:tsukida:3}
\end{figure}

\noindent
For matrix subjects, the proportion of scores of 0, 1, or 2 is somewhat similar to the S of the \textit{S 'u, XV} pattern. For both, more than half of the referents (57\% for matrix subjects and 55\% for S of the \textit{S 'u, XV} pattern) had \isi{anaphoricity} scores of 2 (mentioned in the previous \isi{discourse}), and about one third (37\% for matrix subjects and 33\% for preposed subjects without overt matrix subjects) had scores of 0 (not mentioned at all). The percentage of NPs with scores of 1 is somewhat lower for matrix subjects (6\%) than for S of the \textit{S 'u, XV} pattern (13\%). 

For S of the \textit{S 'u, XV ka hiya} pattern, the rate of each score differs from the other two types. There is no instance of score 1, and the percentage for score 0 is somewhat higher than for the other two. It may be a coincidence which originates from scarcity of the data.

\subsection{\label{s:tsukida:4.4}Correlation between NP types and anaphoricity}

Let us see the correlation between NP types and \isi{anaphoricity}. It is summarized in \tabref{tab:tsukida:9}.

\begin{table}
\begin{tabularx}{\textwidth}{XrYY}
	\lsptoprule
	& \multicolumn{3}{c}{Anaphoricity score}\\ \cmidrule{2-4}
	&  0 &  1 &  2\\
	\midrule
	PP &  0 &  0 &  2\\
	D &  0 &  0 &  6\\
	PD &  1 &  0 &  11\\
	CD &  1 &  0 &  3\\
	BPN &  12 &  0 &  8\\
	CG &  4 &  0 &  2\\
	CR &  2 &  2 &  0\\
	CN &  1 &  3 &  0\\
	\textit{Saw}+ noun &  2 &  2 &  3\\
	VP &  1 &  0 &  0\\
	\midrule
	Total &  24 &  7 &  35\\
	\lspbottomrule
\end{tabularx}
	\caption{Type of NP and anaphoricity}
	\label{tab:tsukida:9}
\end{table}

\noindent
It is interesting to see the \isi{anaphoricity} difference between PD and BPN. For most of the PD (11 of 12), the \isi{anaphoricity} score is 2, while more than half of BPN (12 of 20) had a score of 0. When something already referred to is mentioned again, it appears accompanied by demonstratives. This seems to support the idea that PDs are \textit{activated} or \textit{familiar} while BPN are at least \textit{uniquely identifiable}. For all PP and D also, the \isi{anaphoricity} score is 2. This seems to support the idea that PP and D correspond to \textit{in focus} and \textit{activated}, respectively. 

For those that would correspond to lower \isi{givenness} (CR, CN, \textit{saw} + noun or VP), we can see that score of 2 is not observed very much. 

\section{\label{s:tsukida:5}Topic persistence}

Lastly, I will examine the topic persistence of preposed subjects.

\subsection{\label{s:tsukida:5.1}Preliminaries}

I applied the label of persistence to an attribute with four possible values in the following way:

\begin{itemize}
	\item 0: The pre-clausal NP denotatum is not referred to at all within five subsequent clauses.
	\item 1: The pre-clausal NP denotatum is referred to in subsequent clauses by means of a lexically headed NP rather than a \isi{pronoun}.
	\item 2: The pre-clausal NP denotatum is expressed pronominally within the five following clauses.
	\item 3: The pre-clausal NP denotatum is referred to in subsequent clauses by means of a zero \isi{pronoun}.
\end{itemize}

\noindent
I added a fourth value to the classification in \citet[1689]{Gregory2001}, because in \ili{Seediq} one can omit NPs that are recoverable from the context. I wanted to distinguish covert and overt pronouns, so I limited the score of 2 to those cases with overt pronouns and applied a score of 3 to those cases with covert pronouns. The examples in (\ref{e:tsukida:54}-\ref{e:tsukida:57}) illustrate the four possible persistence scores, with referring expressions and their antecedents co-indexed.

The preposed NP of (\ref{e:tsukida:54a}), \textit{Tiwang niyi} ‘this Ciwang,’ is not referred to at all in at least the following five clauses (though the example below only shows the following two clauses), so the score is 0. From sentence (\ref{e:tsukida:54b}), the topic of the text is changed to \textit{Bakan Hagay} and \textit{Karaw Wacih}.

\begin{exe}
	\ex{Lack of persistence; score of 0}\label{e:tsukida:54}
	\begin{xlist}
		\ex\label{e:tsukida:54a}
		\gll \textit{Tiwang}   \textit{niyi}  \textit{'u}, \textit{dima} \textit{sedu'uy} \textit{kari} \textit{Kiristu} \textit{ka} \textit{hiya} \textit{da}.\\
		Ciwang \textsc{prox} \textsc{prt} already \textsc{av}.have words Christ \textsc{sbj} \textsc{3sg} \textsc{ns}\\
		\glt ‘This Ciwang, she was already a Christian.’
		\ex\label{e:tsukida:54b}
		\gll \textit{Pa'ah} \textit{d}{\USSmaller}\textit{en}{\USGreater}\textit{ehuq}-\textit{an}=\textit{na} \textit{Ekedusan} \textit{ka} \textit{Bakan} \textit{Hagay} \textit{'u}\\
		from <\textsc{prf}>arive-\textsc{gv}=\textsc{3sg}.\textsc{gen} Ekedusan \textsc{sbj} Bakan Hagay \textsc{prt}\\
		\glt ‘Since Bakan Hagay arrived at Ekedusan,’
		\ex\label{e:tsukida:54c}
		\gll \textit{'ida} \textit{s}{\USSmaller}\textit{em}{\USGreater}\textit{ekuxul}=\textit{bi} \textit{m}-\textit{uyas} \textit{'uyas} \textit{'Utux} \textit{Baraw} \textit{deha}     \textit{Karaw} \textit{Watih} \textit{senaw}=\textit{na}.\\
		surely <\textsc{av}>like=really \textsc{av}-sing song  God   Heaven  with Karaw  Wacih husband=\textsc{3sg}.\textsc{gen}\\
		\glt ‘she liked singing songs about God in heaven with her husband Karaw Wacih.’
	\end{xlist}
\end{exe}

\noindent
The preposed NP of (\ref{e:tsukida:55a}), \textit{Lebak Yudaw} ‘Lebak Yudaw,’ is referred to by a lexically headed NP \textit{Lebak niyi} ‘this Lebak’ in sentence (\ref{e:tsukida:55b}), the following \isi{clause}. The score is therefore 1.

\begin{exe}
	\ex{Repeated NP; score of 1}\label{e:tsukida:55}
	\begin{xlist}
		\ex\label{e:tsukida:55a}
		\gll \textit{Si'ida}  \textit{ka} \textbf{\textit{Lebak}}  \textbf{\textit{Yudaw}}\textit{\textsubscript{i}} \textit{'u},  \textit{me}-\textit{'ayung}  \textit{sapah}  \textit{kensat}  \textit{ka}  \textit{hiya}   \textit{ni}\\
		then \textsc{lnk} Lebak Yudaw \textsc{prt} \textsc{av}-assistant  house  police \textsc{sbj} \textsc{3sg} and\\
		\glt ‘At that time, Lebak Yudaw was an assistant at the police station and,’
		\ex\label{e:tsukida:55b}
		\gll \textit{m}-\textit{bahang}   \textit{kari}  \textit{quri}  \textit{t}{\USSmaller}\textit{en}{\USGreater}\textit{egesa}       \textit{Kiristu}  \textit{ka} \textbf{\textit{Lebak}}  \textbf{\textit{niyi}}\textit{\textsubscript{i}} \textit{'uri},\\
		\textsc{av}-listen story about <\textsc{cv}>\textsc{prf}.teach Christ \textsc{sbj}   Lebak  \textsc{prox} also\\
		\glt ‘this Lebak also heard the story about what Christ taught.’
	\end{xlist}
\end{exe}

\noindent
The preposed NP in (\ref{e:tsukida:56a}), \textit{niyi} ‘this,’ is referred to by the \isi{pronoun} \textit{kiya} ‘it, so’ in sentence (\ref{e:tsukida:56b}), the following sentence. The score is therefore 2.

\begin{exe}
	\ex{Pronominal use; score of 2}\label{e:tsukida:56}
	\begin{xlist}
		\ex\label{e:tsukida:56a}
		\gll \textbf{\textit{Niyi}}\textit{\textsubscript{i}} \textit{'u},  \textit{'adi}  \textit{'utux}=\textit{ta}       \textit{ka}  \textit{kiya}\textit{\textsubscript{i}},\\
		\textsc{prox} \textsc{prt} \textsc{neg} God=\textsc{1pi.gen} \textsc{sbj} it\\
		\glt ‘This, it is not our God,’
		\ex\label{e:tsukida:56b}
		\gll \textit{'utux}  \textit{'amirika}       \textit{ka} \textbf{\textit{kiya}}\textit{\textsubscript{i}}.\\
		God    America \textsc{sbj} it\\
		\glt ‘it is the American God.’
	\end{xlist}
\end{exe}

\noindent
The preposed NP of (\ref{e:tsukida:57a}), \textit{kari niyi} ‘this story’ is the patient of the predicate verb of \isi{clause} (\ref{e:tsukida:57b}), \textit{m-iyah t<em>egesa} ‘come to teach,’ but it is not overt. The score is therefore 3.

\begin{exe}
	\ex{Zero pronoun use; score of 3}\label{e:tsukida:57}
	\begin{xlist}
		\ex\label{e:tsukida:57a}
		\gll \textbf{\textit{Kari}}    \textbf{\textit{niyi}}\textit{\textsubscript{i}} \textit{'u},   \textit{n}-\textit{eyah}-\textit{an}=\textit{na}         \textit{m}-\textit{angal}  \textit{pa'ah}  \textit{'alang}   \textit{Besuring}.\\
		story \textsc{prox} \textsc{prt} \textsc{prf}-come-\textsc{gv}=\textsc{3sg.gen} \textsc{av}-take   from   village Besuring\\
		\glt ‘As for this story [=the gospel], it was taken from Besuring village.’
		\ex\label{e:tsukida:57b}
		\gll \textit{Pekelug}   \textit{m}-\textit{iyah}    \textit{t}{\USSmaller}\textit{em}{\USGreater}\textit{gesa} \_\_\textit{\textsubscript{i}} \textit{hiya}  \textit{ka}  \textit{Tiwang}  \textit{'Iwal}.\\
		just \textsc{av}-come   <\textsc{av}>teach {} there \textsc{sbj} Ciwang Iwal\\
		\glt ‘Ciwang Iwal came to teach it just then.’
	\end{xlist}
\end{exe}

\subsection{\label{s:tsukida:5.2}Results}

For preposed subjects, the topic persistence score is 0 for 35\%, 1 for 23\%, 2 for 33\%, and 3 for 9\%. The incidence of score 3 is rather low. It is summarized in \tabref{tab:tsukida:10}.

\begin{table}
\begin{tabularx}{\textwidth}{XYY} 
	\lsptoprule
	&  Preposed subject &  \%\\
	\midrule
	0 (=no persistence) &  23 &  35\%\\
	1 (=repeated NP) &  15 &  23\%\\
	2 (=pronominal use) &  22 &  33\%\\
	3 (=zero \isi{pronoun}) &  6 &  9\%\\
	\midrule
	Total &  66 &  100\%\\
	\lspbottomrule
\end{tabularx}
	\caption{Persistence of preposed subjects}
	\label{tab:tsukida:10}
\end{table}

\subsection{\label{s:tsukida:5.3}Comparison}

In this section I will compare the topic persistence scores of preposed NPs with and without overt matrix subjects, and also with matrix subjects.

\subsubsection{\label{s:tsukida:5.3.1}When the matrix clause subject is and is not overt}

A comparison of topic persistence of preposed subjects depending on the presence or absence of overt matrix subjects is given in \tabref{tab:tsukida:11}.

\begin{table}
\begin{tabularx}{\textwidth}{Xrrrr} 
	\lsptoprule
	& \multicolumn{4}{c}{Preposed subjects}\\ \cmidrule{2-5}
	& \multicolumn{2}{l}{S 'u, VX ka hiya.} & \multicolumn{2}{l}{S 'u, VX.}\\
	\midrule
	0 (=no persistence) &  2 &  15\% &  21 &  40\%\\
	1 (=repeated NP) &  3 &  23\% &  12 &  23\%\\
	2 (=pronominal use) &  8 &  62\% &  14 &  26\%\\
	3 (=zero \isi{pronoun}) &  0 &  0\% &  6 &  11\%\\
	\midrule
	Total &  13 &  100\% &  53 &  100\%\\
	\lspbottomrule
\end{tabularx}
	\caption{Persistence of preposed subjects with and without overt  matrix subjects}
	\label{tab:tsukida:11}
\end{table}

Topic persistence differs considerably depending on the presence or absence of an overt matrix subject. 

62\% (8 out of 13) of the instances of the S of the \textit{S 'u, VX ka hiya} pattern showed a score of 2, but only 26\% (14 out of 53) of the S of the \textit{S 'u, VX} pattern. In contrast, 15\% of the instances (2 out of 13) of the S of the \textit{S 'u, VX ka hiya} pattern showed a score of 0, while 40\% of the S of the \textit{S 'u, VX} (21 out of 53) showed a score of 0. There was no instance of an overt matrix subject (\textit{S 'u, VX ka hiya}) that showed a score of 3; the six instances of score 3 (11\%) are all S of \textit{S 'u, VX}. For S of \textit{S 'u, VX ka hiya}, the proportion of score 2 (62\%) is the highest. For S of \textit{S 'u, VX}, the proportion of score 0 (40\%) is the highest. 

\subsubsection{\label{s:tsukida:5.3.2}Preposed subject and matrix subject}

Now let us compare preposed subjects with matrix subjects. This is shown in \tabref{tab:tsukida:12} and \figref{fig:tsukida:4}.

\begin{table}
\begin{tabularx}{\textwidth}{Xrrrrrr} 
	\lsptoprule
	& \multicolumn{4}{c}{Preposed subjects} & \multicolumn{2}{l}{Matrix subjects}\\ \cmidrule{2-5}
	& \multicolumn{2}{l}{S 'u, VX ka hiya.} & \multicolumn{2}{l}{ S 'u, VX.} & \multicolumn{2}{l}{ VX ka S.}\\
	\midrule
	0 (=no persistence) &  2 &  15\% &  21 &  40\% &  117 &  45\%\\
	1 (=repeated NP) &  3 &  23\% &  12 &  23\% &  51 &  19\%\\
	2 (=pronominal use) &  8 &  62\% &  14 &  26\% &  65 &  25\%\\
	3 (=zero \isi{pronoun}) &  0 &  0\% &  6 &  11\% &  29 &  11\%\\
	\midrule
	Total &  13 &  100\% &  53 &  100\% &  262 &  100\%\\
	\lspbottomrule
\end{tabularx}
	\caption{Persistence of preposed subjects with and without overt matrix subjects}
	\label{tab:tsukida:12}
\end{table}

\begin{figure}
	\includegraphics[width=0.8\textwidth]{figures/tsukida-figure4.png}
	\caption{Persistence of preposed subjects with and without overt regular-position subjects}
	\label{fig:tsukida:4}
\end{figure}

We can see that the matrix subject shows a tendency quite similar to S of \textit{S '}\textit{u, VX}. In many instances of both types (45\% and 40\%, respectively), the \isi{referent} is referred to only once in a single \isi{clause} (for a score of 0). In contrast, many instances of S of \textit{S '}\textit{u, VX} \textit{ka hiya} (double reference) show a score of 2 (referred to again by a \isi{pronoun} within the following five clauses). The doubly referenced entity somehow tends to persist longer.

\subsection{\label{s:tsukida:5.4}Correlation between NP types and topic persistence}

Let us see the correlation between NP types and topic persistence, summarized in \tabref{tab:tsukida:13}.

\begin{table}
\begin{tabularx}{\textwidth}{Xrrrr} 
	\lsptoprule
	& \multicolumn{4}{c}{Topic persistence}\\
	\midrule
	&  0 &  1 &  2 &  3\\
	PP &  0 &  0 &  2 &  0\\
	D &  2 &  1 &  2 &  1\\
	PD &  2 &  2 &  6 &  2\\
	CD &  1 &  1 &  1 &  1\\
	BPN &  5 &  8 &  6 &  0\\
	CG &  4 &  0 &  1 &  1\\
	CR &  2 &  1 &  2 &  0\\
	CN &  2 &  2 &  0 &  0\\
	\textit{Saw}+ noun &  4 &  0 &  2 &  1\\
	VP &  1 &  0 &  0 &  0\\
	\midrule
	Total &  23 &  15 &  22 &  6\\
	\lspbottomrule
\end{tabularx}
	\caption{NP types and Topic pessistence}
	\label{tab:tsukida:13}
\end{table}

\newpage 
As for topic persistence, the tendency is not as clear as in the case of \isi{anaphoricity}. We can say that CR, CG, CN and \textit{saw} + noun, which are supposed to be at the lower position in the \isi{givenness} hierarchy of \citet{Gundel1993}, the topic persistence score tends to be low (score of 0).

\section{\label{s:tsukida:6}Summary}

I examined the semantics/function of NP type, \isi{anaphoricity}, and topic persistence in preposed NPs. 

Semantically, more than two thirds of the preposed NPs are coreferential with the matrix subject. One sixth of them are preposed Time. 

Preposed subjects, As, and possessors can be regarded to indicate aboutness topics, and the matrix \isi{clause} denotes the comment about them. Preposed time, alternative, and those preposed NPs that are not classified as any of the above seem to set a frame to the information provided by the matrix \isi{clause}. 

As for NP types, this paper pointed out several characteristics of preposed NPs. We can say that preposed subjects tend to be proper names. About half of the preposed subjects are bare proper names (BPN) or proper names modified by a \isi{demonstrative} (PD). Only 28\% of the matrix subjects are BPN or PD, so BPN or PD is a characteristic of preposed subjects. 

Another characteristic of the NP types of preposed subjects is that personal pronouns rarely appear preposed. Only 3\% of preposed subjects are personal pronouns, but 20\% of matrix subjects are personal pronouns. 
\newpage
We can point out one characteristic of S of the \textit{S 'u, VX ka hiya} pattern: common nouns rarely appear. When the matrix subject is overt, there are very few instances (7\%) of common nouns (bare CN), modified by a \isi{demonstrative} (CD), by another noun or a VP (CR), or by a genitive \isi{pronoun} (CG). For S of the \textit{S 'u, VX} pattern, one third of preposed subjects are CN, CD, CR, or CG. For matrix subjects, the rate of CN, CD, CR, and CG is higher; it is about 40\%. 

As for \isi{anaphoricity}, there is no drastic difference between S of the \textit{S 'u, VX ka hiya} pattern, S of the \textit{S 'u, VX} pattern, or matrix subjects. Matrix subjects showed slightly more (57\%) high scores (2) than preposed subjects (53\%). Half of the instances of S of the \textit{S 'u, VX ka hiya} pattern (50\%) showed a score of 0, which is the highest proportion among the three. 

This paper also examined coreference between types of NPs and \isi{anaphoricity}. Difference in \isi{anaphoricity} between PD and BPN seems worth noting. Most of the PD are already mentioned in the previous \isi{discourse}, while BPN are not necessarily so. Also, PP and D are all mentioned in the previous \isi{discourse}. 

As for topic persistence, preposed S of the \textit{S 'u, VX} pattern and matrix subjects showed similar tendencies (\tabref{tab:tsukida:12} in \sectref{s:tsukida:5.3.2}). Totally, 40\% or 45\% showed a score of 0, 23\% or 19\% showed a score of 1, 25\% showed a score of 2, and 12\% or 11\% showed a score of 3. S of the \textit{S 'u, VX ka hiya} pattern, on the other hand, showed a different tendency: only 14\% showed a score of 0, 21\% showed a score of 1, and 64\% showed a score of 2. There are no instances of preposed subjects with overt matrix subjects that showed a score of 3.

To summarize, we can say that S of the \textit{S 'u, VX} pattern and matrix subjects showed similar tendencies except for NP types. What type of NP it is determines whether it appears in preposed position or in regular subject position. S of the \textit{S 'u, VX ka hiya} pattern, on the other hand, is used to give further information about a proper name. It tends to persist longer in the following clauses. As for \isi{anaphoricity}, there is no drastic difference among the three.

\section*{Acknowledgments}

I thank two anonymous reviewers for useful comments. Also I would like to thank Editage (\href{http://www.editage.jp}{www.editage.jp}) for \ili{English} language editing. For any mistakes in the present paper, however, solely the author is responsible.

\section*{Abbreviations}

\begin{multicols}{2}
	\begin{tabbing}
		glossgloss \= \kill
		\textsc{anger} \> anger\\
		\textsc{av} \> \isi{actor voice}\\
		\textsc{bpn} \> bare person name\\
		\textsc{cg} \> common noun modified\\ \> with Genitive \isi{pronoun}\\
		\textsc{cn} \> common noun\\
		\textsc{cnj} \> conjunction\\
		\textsc{cv} \> conveyance \isi{voice}\\
		\textsc{cr} \> common noun modified\\ \>by another noun or VP\\
		\textsc{d} \> \isi{demonstrative}\\
		\textsc{dist} \> distant\\
		\textsc{ex} \> exclusive\\
		\textsc{fut} \> future\\
		\textsc{gen} \> genitive\\
		\textsc{gentle} \> gentle\\
		\textsc{gv} \> goal \isi{voice}\\
		\textsc{gv1} \> goal \isi{voice} 1\\
		\textsc{gv2} \> goal \isi{voice} 2\\
		\textsc{hearsay} \> hearsay\\
		\textsc{in} \> inclusive\\
		\textsc{lnk} \> linker\\
		\textsc{neg} \> negative\\
		\textsc{nfin} \> non-finite\\
		\textsc{nom} \> \isi{nominative}\\
		\textsc{rdpl} \> reduplication\\
		\textsc{pd} \> person name modified\\ \> by \isi{demonstrative}\\
		\textsc{pl} \> plural\\
		\textsc{po} \> Possessor\\
		\textsc{pp} \> personal \isi{pronoun}\\
		\textsc{prf} \> perfective\\
		\textsc{prox} \> proximant\\
		\textsc{prt} \> particle\\
		\textsc{reason} \> reason\\
		\textsc{sbj} \> subject\\
		\textsc{sg} \> singular\\
		\textsc{uncertn} \> uncertain\\
		1 \> first\\
		2 \> second\\
		3 \> third
	\end{tabbing}
\end{multicols}

\sloppy
\printbibliography[heading=subbibliography,notkeyword=this]

\end{document}
