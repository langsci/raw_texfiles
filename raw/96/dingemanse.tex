\documentclass[output=paper]{langsci/langscibook}
\ChapterDOI{10.5281/zenodo.573781}

\author{Mark Dingemanse\affiliation{MPI for Psycholinguistics, Nijmegen}}
\title{On the margins of language: Ideophones, interjections and dependencies in linguistic theory} 
\shorttitlerunninghead{On the margins of language}
\begin{document}
\maketitle

\is{ideophones} \is{interjections} 
\noindent In this chapter I explore some dependencies between form and function in ideophones and interjections, two word classes traditionally considered marginal in linguistics. It is as much about dependencies in language--how different aspects of linguistic structure causally relate to each other--as about dependencies in linguistics: how our theorising may be contingent on preconceived notions of what language is like. 


Ideas about language influence how we carry out the scientific tasks of observation and explanation. Observation is the discovery of rules and regularities in language structure. It raises the question of methods. How do we design linguistic inquiry so as to facilitate accurate and meaningful observations? Explanation is the description of observations in causal terms. It raises the question of mechanisms: what entities and processes do we posit to account for the observations? The tools we use for observation and explanation are our methods and theories, which act like optical instruments. They enhance our powers of observation at one level of granularity (at the expense of others), and they bring certain phenomena in focus (defocusing others). Our views of language, including what we consider central and marginal, are shaped and constrained by these tools — and sometimes they may need recalibration.



\is{rara} \is{marginalia} There are several ways to characterise the margins of language. Here I distinguish between rara and marginalia. \textsc{Rara} are typologically exceptional phenomena which illustrate the fringes of linguistic diversity. Examples are nominal tense or affixation by place of articulation \citep{Wohlgemuth2010} \textsc{Marginalia} are typologically unexceptional phenomena that many linguists think can be ignored without harm to linguistic inquiry. They are not rare, but linguistic practice assigns them to the margin by consensus \citep{Joseph1997}. Whereas rara can be objectively described as exceptional, marginalia are viewpoint-depen\-dent. One goal of this chapter is to critically examine received notions of marginality by inspecting two supposed marginalia: ideophones and interjections. \is{ideophones} \is{interjections} 


\section{Ideophones: morphosyntax can depend on mode of representation}

Ideophones are words like \textit{gorogoro} ‘rolling’ and \textit{kibikibi} ‘energetic’ in \ili{Japanese}, or \textit{kɛlɛŋkɛlɛŋ} ‘glittery’ and \textit{saaa} ‘cool sensation’ in \ili{Siwu}, a \ili{Kwa} language of Ghana. They can be defined as marked words that depict sensory imagery: words whose marked forms invite iconic interpretations and evoke sensory meanings. They appear to be uncommon in standard average European languages, \is{European languages} which has led some scholars to assume that “the number of pictorial, imitative, or onomatopoetic nonderived words in any language is vanishingly small” \citep[758]{Newmeyer1992}. Typological evidence shows that these words are in fact common across the world’s languages and that they number well into the thousands in many of them \citep{Dingemanse2012}. 


\is{ideophones} Much research on ideophones has focused on their striking forms, with deviant phonotactics and distinctive prosody vying for attention. Their morphosyntactic behaviour has received less consideration, as a common view is that ideophones by definition have no \is{syntax} syntax \citep{Childs1994}. However, that simple statement conceals an interesting puzzle. A basic insight of linguistic typology is that lexical classes and their morphosyntactic realisation are best described in language-specific terms \citep{Croft2001}. There is little reason to assume that what we call a ``noun" for comparative purposes will show the same morphosyntactic behaviour in unrelated languages. Indeed, precisely because the structural facts can be so different across languages, comparative concepts tend to have a semantic basis \citep{Haspelmath2010}. Ideophones are different. Important aspects of their form and function appear to be predictable across languages.


\largerpage
\is{ideophones} \is{syntax} Ideophones typically display a great degree of syntactic independence. They tend to occur at the edge of the utterance, unburdened by morphology and not deeply embedded in the syntactic structure of the clause. In the \ili{Siwu} example below, the ideophone \textit{pɔkɔsɔɔ} ‘carefully’ appears in utterance-final position and is syntactically optional: the utterance would be well-formed without it.  

\ea
\gll   iyɔ  nɛ  ɔti  kere  a-à-\textup{{\textbar}}ti  ↑pɔkɔsɔɔɔɔɔ↑\textup{\textsubscript{\{falsetto\}}}\textup{ {\textbar}}\textsc{\textsubscript{gesture}}\\
  {{so}}  TP  {{sieving}}  {{just}}  {you-\textsc{fut}-sieve}  {\textsc{idph}}{{slow/easy}}\\
\glt ‘Then you’ll just be sieving ↑\textit{pɔkɔsɔɔɔɔɔ}↑ [carefully]’ \\
((\textsc{gesture}: two-handed demonstration of gently jiggling a sieve))
\z



\is{ideophones} \is{syntax} Constructions like this are found in many languages of the world. Why would ideophones show similar patterns of morphosyntactic independence across unrelated languages? A promising explanation is that ideophones in such cases are an instance of showing rather than saying, depictions \is{depiction} rather than descriptions. Just as white space separates images from text on a page, so the syntactic freedom of ideophones helps us to see them as depictive performances in otherwise mostly descriptive utterances \citep{Kunene1965}. What we see here is the encounter of two distinct and partly incommensurable methods of communication: the discrete, arbitrary, descriptive system represented by ordinary words, and the gradient, iconic, \is{iconicity} depictive system represented by ideophones. These two systems place different requirements on the material use of speech, yet both are part of one linearly unfolding speech stream. The morphosyntactic independence of ideophones may be a solution to this linearisation problem. 



What kind of evidence could support this proposal? One clue for the depictive nature of ideophones is that they tend to be produced with prosodic foregrounding: features of delivery that make the ideophone stand out from the surrounding material. Thus in the \ili{Siwu} example above, the ideophone \textit{pɔkɔsɔɔ} ‘carefully’ is prosodically foregrounded by means of markedly higher pitch (↑) and falsetto phonation. Further underlining their depictive nature, ideophones are also more susceptible to expressive modification than ordinary words, often showing iconic resemblances between form and meaning. Additionally, they are often--as in the example above--produced together with iconic gestures \citep{Nuckolls1996}. \is{iconicity} \is{gesture} 



\is{corpus data} Corpus data can provide a natural laboratory to test the dependency more directly. In many languages, ideophones do in fact participate in sentential syntax to varying degrees. A common enough response is to ignore this: we know that ideophones are supposed to have no syntax, most data appear to confirm this, so we discount the few remaining exceptions. To do so is to accept a preconceived notion of ideophones as marginal. A more interesting question is what happens when ideophones do show greater morphosyntactic integration. \is{ideophones}



\is{prosody} \is{syntax} \is{ideophones} What happens is that we find an inverse relation between prosodic foregrounding and morphosyntactic integration. Ideophones that are more deeply integrated in the structure of the clause lose their prosodic foregrounding. In example \ref{ex:dingemanse:2} from \ili{Siwu}, the same ideophone \textit{pɔkɔsɔɔ} appears as an adjectival modifier in a noun phrase \textit{ìra pɔkɔsɔ-à} ‘easy thing’. It carries the adjectival suffix -\textit{à} and is not foregrounded or expressively modified in any way.

\ea \label{ex:dingemanse:2}
 \gll a-bu  sɔ  ìra  pɔkɔsɔ-à  i-de  ngbe:\\
 {{you-think}}  {{that}}  {{thing}}  \textsc{idph}.easy/slow-\textsc{adj}  {{it-be}}  {{this:}}{\textsc{q}}\\
\glt ‘You think this here is an easy thing?’
\z



\is{ideophones} \is{syntax} \is{prosody} \is{depiction} Examples like this can be multiplied, and all show the same interaction: syntactic freedom and prosodic foregrounding go hand in hand, and the more integrated the ideophone is, the less likely it is to undergo foregrounding. The interaction works out essentially the same way for ideophones across a wide range of languages \citep{Dingemanse2012}. The tell-tale signs of depiction that occur when ideophones are morphosyntactically independent all disappear when ideophones lose their freedom and are assimilated to become more like normal words. So the dependency looks like this:


\ea\textit{Morphosyntax can depend on mode of representation.} 

\textup{The morphosyntactic freedom of ideophones across languages is causally dependent on the fact that ideophones inhabit a depictive mode of representation.}
\z

\is{marginalia} \is{ideophones} \is{syntax} The marked morphosyntactic profile of ideophones receives a unified explanation. Discovering the causal mechanism requires abandoning the assumption that ideophones are always marginal, and accepting that explanations of morphosyntactic behaviour can come from outside morphosyntax. A semiotic account provides the most likely cause, and close attention to corpus data helps solidify it.


\section{Interjections: form can depend on interactional ecology}

\is{interjections} Interjections are words like \textit{Ouch!}, \textit{Oh.} and \textit{Huh?} in \ili{English}, or \textit{Adjei!} ‘Ouch!’ \textit{Ah} ‘Oh.’ and \textit{Ã?} ‘Huh?’ in \ili{Siwu}. They can be defined as conventional lexical forms which are monomorphemic and typically constitute an utterance of their own \citep{Wilkins1992}. To the extent that interjections constitute their own utterances, they have little to do with other elements of sentences or with inflectional or derivational morphology, so they could be justifiably called marginal. If we follow scholarly traditions that take the sentence as the main unit of analysis, that might be all there is to say. 


\is{conversational analysis} \is{interactional linguistics} Yet utterances, whether they consist of simple interjections or complex sentences, virtually never occur in isolation. They are responsive to prior utterances or elicit responses in turn; and as decades of work in conversation analysis and interactional linguistics have shown, they do so in highly ordered, normatively regulated ways \citep{Schegloff2007,Selting2001}. As every bit of language is ultimately socially transmitted, \is{transmission!social} the structure of conversation forms the evolutionary landscape for linguistic items. How does language adapt to this landscape? What are the constraints and selective pressures it imposes? To make these questions tractable, it is useful to take one bit of conversational structure and consider its properties in detail. 



\is{interjections} Consider the interjection English \textit{Huh?}, used when one has not caught what someone just said. This interjection, along with other practices for initiating repair, \is{repair} fulfills an important role in maintaining mutual understanding in the incessant flow of interaction that is at the heart of human social life. At this level of granularity, the interjection is far from marginal — in fact it is right where the action is. Here are two simplified transcripts from conversations recorded in Ghana and Laos. A word equivalent in form and function to English \textit{Huh?} is the central pivot in the sequence, signaling a problem in a prior turn and inviting a redoing in the next. This may seem a trivial operation, especially since we do it so often---but therein lies the crux: without items like this, our conversations would be constantly derailed.

\ea
\ea
\langinfo{Siwu}{\ili{Kwa}, Ghana}{}\\
  \begin{tabular}{ll}
    A   & \textit{Mama sɔ ba.}\\
	&‘Mama says “come”!’ \\
    B 	& \textbf{\textit{ã:}}\\
	&\textbf{‘Huh?’}\\
    A   &\textit{Mama sɔ ba.}\\
	&‘Mama says “come”!’  
    \end{tabular}
  \ex
  \langinfo{Lao}{\ili{Tai-Kadai}, Laos}{courtesy of Nick Enfield}\\
  \begin{tabular}{ll}
    A 	& \textit{nòòj4 bòò1 mii2 sùak4 vaa3 nòòj4}\\
	& ‘Noi, don’t you have any rope, Noi?’\\
    B 	& \textbf{\textit{haa2}}\\
	&  \textbf{‘Huh?’}\\
      A & \textit{bòò1 mii2 sùak4 vaa3}\\
	&  ‘Don’t you have any rope?’\\
    \end{tabular}\\
%   \lspbottomrule
\z
\z

%{\todo[inline]{source missing}}\\
%SEBASTIAN: I contacted the author and this example is currently unpublished. He gave me the corpus identifier: Gunpowder_2767800 for (1) and Palmoil_1295987 for (2).
% \midrule

\largerpage
 
Comparative work on communicative repair in dozens of spoken languages reveals a striking fact. The interjection occuring in this interactional environment always has a very similar shape: a monosyllable with questioning prosody and all articulators in near-neutral position \citep{Dingemanse2013}. And this is not the only interjection of this kind. In language after language, a highly effective set of streamlined interjections contributes to the smooth running of the interactional machinery. Other examples of interjections that fulfil important interactional functions and that appear to be strongly similar across languages include \textit{oh} and \textit{ah }(signaling a change in state of knowledge), \textit{mm }(signaling a pass on claiming the conversational floor), and \textit{um}/\textit{uh} (signaling an upcoming delay in speaking). 
 


\is{cultural adaptation} \is{biological adaptation} It may be tempting to posit that these words are simply instinctive grunts like sighs or sneezes, explaining their cross-linguistic similarity at one blow. However, this proposal merely shifts the question and wrongly assumes that biological adaptation offers a simpler explanation than cultural adaptation. (The survival value of sighs and sneezes is fairly straightforward; much less so for this range of interjections.) A more parsimonious proposal, worked out in detail for \textit{Huh?} in \citet{Dingemanse2013}, is that the interactional environment in which these items occur may provide, for each of them, a distinct set of selective pressures--for minimality, salience, contrast, or other adaptive properties--that squeezes them into their most optimal shape. The resulting paradigm of words may come to have certain universal properties by means of a mechanism of convergent cultural evolution. So the dependency is as follows:


\ea
\textit{Form can depend on interactional ecology.} 
\is{interactional ecology} \is{interjections}
\textup{Strong and unexpected similarities in basic discourse interjections across unrelated languages are causally dependent on their appearance in common interactional environments where they are shaped by the same selective pressures.}
\z

\is{interjections} Interjections are often cast as the blunt monosyllabic fragments of the most primitive and emotional forms of language. Comparative research on social interaction is fast undoing this view, and shows how at least some interjections may be adaptive communicative tools, culturally evolved for the job of keeping our social interactional machinery in good repair. 


\section{Discussion}

About 150 years ago, influential Oxford linguist Max Müller proclaimed of imitative words that “they are the playthings, not the tools, of language”, and almost in the same breath pooh-poohed interjections with the slogan “language begins where interjections end” \citep[346, 352]{Müller1861}. Such statements helped shape a scholarly climate in which it is easy to take for granted that we already know where the most important questions about language lie. Yet with linguistics and neighbouring fields constantly finding new sources of data, methods and insights, it is natural every once in a while to take a step back and question received wisdom. 


\is{ideophones} \is{interjections} \is{syntax} \is{marginalia} Ideophones and interjections are similar in that they share a degree of syntactic independence, one basis for portraying them as marginal. However, as we have seen, beneath this superficial similarity lie different semiotic functions and distinct causal forces. Ideophones are syntactically independent because they inhabit a mode of representation that is different from the remainder of the speech signal. Their freedom helps foreground their special status as depictive signs. From ideophones we learn that the morphosyntax of linguistic items may depend at least in part on mode of representation. Interjections are syntactically independent because their main business is not carried out within utterances but at other levels of linguistic structure. Their patterning is best analysed in relation to their discursive and interactional context. From interjections we learn that the form of linguistic items may depend at least in part on interactional ecology.



Linguistic discovery is viewpoint-depen\-dent, as are our ideas about what is marginal and what is central in language. The challenges posed by the supposed marginalia discussed here provide some useful pointers for widening our field of view. Ideophones challenge us to take a fresh look at language and consider how it is that our communication system combines multiple modes of representation. Interjections challenge us to extend linguistic inquiry beyond sentence level, and remind us that language is social-interactive at core. Marginalia are not obscure, exotic phenomena that can be safely ignored. They represent opportunities for innovation and invite us to keep pushing the edges of the science of language.


\section*{Acknowledgements}

I thank \ili{Kimi} Akita, Nick Enfield and Felix Ameka for being interlocutors on these topics over the last few years. I gratefully acknowledge funding from the Max Planck Society for the Advancement of Science and an NWO Veni grant. 


{\sloppy
\printbibliography[heading=subbibliography,notkeyword=this]
}

\end{document}%%remove this line and move all lines below to localbibliography.bib
