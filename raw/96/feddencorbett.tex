\documentclass[output=paper]{langsci/langscibook}
\ChapterDOI{10.5281/zenodo.573782}

\author{Sebastian Fedden\affiliation{{The University of Sydney}}\lastand Greville G. Corbett\affiliation{{University of Surrey}}}
\title{Understanding intra-system dependencies: Classifiers in Lao} 
\begin{document} 
\maketitle

\section{Introduction}

\is{number} \is{gender} \is{concurrent systems} We are fascinated by the significant but understudied analytic issue of when different linguistic systems (particularly morphosyntactic features) should be recognized in a given language. In the most straightforward instances we can see that two systems are orthogonal (logically independent of each other), and so each should be postulated in an adequate analysis. Thus traditional accounts of languages like \ili{Italian}, which recognize a number system and a gender system, are fully justified. There are instances which are a little less straightforward. There may be dependencies between different features, for example in \ili{German} there is neutralization of gender in the plural, but we would still have good grounds for recognizing two systems. 

\is{nominal classifications} \is{classifiers} Turning to the specific area of nominal classification, we see that it is certainly an interesting and challenging area of linguistics, but that after a long research tradition we still do not have a clear picture of the different types of classification device that languages employ, much less of their interaction with and dependencies on each other in individual languages. In order to make progress we should undertake analyses of key languages. In some languages we find, arguably, a gender system together with a classifier system, and the interest of the analysis is to determine whether indeed there are two systems of nominal classification or whether the two candidate systems are in fact inter-dependent. In this chapter, however, we undertake a case study which allows us to explore the more difficult yet intriguing issue of dependencies between systems of the same type, that is, between two possible classifier systems. \is{classifiers} Basing ourselves on \citeauthor{Enfield2004}  (\citedate{Enfield2004,Enfield2007}), we examine the \ili{Tai-Kadai} language \ili{Lao}. There are two sets of classifiers, which appear in different constructions. First there is a set of numeral classifiers which are used in contexts of quantification following the numeral. \is{numeral classifiers} Second, \ili{Lao} has a set of classifiers consisting of phonologically reduced forms of the numeral classifiers, and appearing as a proclitic before a range of modifiers. Within the broad question of nominal classification, and the even more general issue of recognizing concurrent systems, we are interested in possible dependencies between these two sets of classifiers. \is{concurrent systems}

\section{Lao}

\ili{Lao} (\citeauthor{Enfield2004} \citeyear{Enfield2004,Enfield2007}), a \ili{Tai-Kadai} language spoken by about 15 million people in Laos and Thailand, has two sets of classifiers. The first set consists of more than 80 numeral classifiers (\textsc{num\_cl}) \citep[xxi-xxiii]{Kerr1972}, which appear in contexts of quantification in a construction where the noun comes first, followed by a numeral (or quantifying expression such as how many?, every or each), followed by a classifier. Two typical examples illustrating the use of the numeral classifiers \textit{too3 }‘\textsc{num\_cl}:\textsc{animate}’ and \textit{khan2} ‘\textsc{num\_cl}:\textsc{vehicle}’, respectively, are given in examples (1) and (2) \citep[120,124]{Enfield2007}. The numbers after \ili{Lao} words indicate tones.

\ea%1
    \label{ex:fc:1}

    \gll   kuu3    sùù4    paa3  sòòng3  too3\\
	 1\textsc{sg.b}    buy    fish  two    \textsc{num\_cl}:\textsc{animate}\\
    \glt ‘I bought two fish.’
    \z
 

\ea%2
    \label{ex:fc:2}

    \gll kuu3    lak1  lot1    sòòng3  khan2\\
	 1\textsc{sg.b}    steal  vehicle  two    \textsc{num\_cl}:\textsc{vehicle}\\
    \glt ‘I stole two cars.’
    \z


The first singular pronoun \textit{kuu3} is glossed ‘B’ here to indicate ‘bare’, that is, semantically unmarked for politeness, as opposed to ‘P’ (‘polite’). When the referent is retrievable from the context, the head noun is often omitted, as in example (3) \citep[139]{Enfield2007}:

\ea%3
    \label{ex:fc:3}

    \gll   kuu3    sùù4    sòòng3  too3\\
	    1\textsc{sg.b}    buy    two    \textsc{num\_cl}:\textsc{animate}\\
    \glt ‘I bought two (e.g., fish).’
    \z
    

\is{numeral classifiers} Numeral classifiers are virtually obligatory and are only very rarely omitted. Semantically \ili{Lao} numeral classifiers express distinctions of shape, size, material, texture, measure and social value. Some numeral classifiers have relatively broad semantics, e.g. \textit{too3} ‘\textsc{num\_cl}:\textsc{animate}’ or \textit{phùùn3} ‘\textsc{num\_cl}:\textsc{cloth}’, whereas others are rather specific, e.g. \textit{qong3} ‘\textsc{num\_cl}:\textsc{monks}’. For nouns which do not have a numeral classifier conventionally assigned to them, the noun is used in this construction to classify itself, giving rise to a set of repeaters which is in principle open. 

Most numeral classifiers double as nouns in the language, e.g. the numeral classifier \textit{khon2} for people (excluding monks) means ‘person’ as a noun, and \textit{sên5} for ribbon-shaped things, such as cables and roads, means ‘line’ as a noun. As is typical of classifier systems, \is{classifiers} the meaning of the classifier is more general than the meaning of the noun from which the classifier is derived. Another characteristic, also common in numeral classifier languages, is that only a relatively small subset of these 80 classifiers is commonly used in discourse. The most frequent ones in \ili{Lao} are \textit{khon2} ‘person’ for humans, \textit{too3} ‘body’ for animals, but also for trousers and shirts, and \textit{qan3}, which does not double as a noun, for small things.

The second set of classifiers appears in a different construction: first comes the noun, followed by a classifier, followed by a modifier. The set of modifiers includes the general demonstrative \textit{nii4}, the non-proximal demonstrative \textit{nan4}, the numeral \textit{nùng1} ‘one’, relative clauses and adjectives. \citet[137]{Enfield2007} calls these modifier classifiers \is{modifier classifiers} (\textsc{mod\_cl}). In principle, all numeral classifiers can appear as modifier classifiers, but in a phonologically reduced proclitic form, which is typically unstressed and shows no tonal contrasts. The following examples illustrate the use of modifier classifiers, with a demonstrative in (4), an adjective in (5) and a relative clause in (6) \citep[139,143]{Enfield2007}. Modifier classifiers are not obligatory with adjectives and relative clauses. 

\ea%4
    \label{ex:fc:4}
    \gll   kuu3    siø=kin3    paa3  toø=nii4\\
	   1\textsc{sg.b}    \textsc{irr=}eat    fish  \textsc{mod\_cl}:\textsc{non.human}=\textsc{dem}\\
    \glt    ‘I’m going to eat this fish.’

    \z

	 

\ea%5
    \label{ex:fc:5}

    \gll   kuu3    siø=kin3    paa3  (toø=)ñaaw2\\
	    1\textsc{sg.b}    \textsc{irr=}eat    fish  (\textsc{mod\_cl}:\textsc{non.human}=)long\\
    \glt ‘I’m going to eat the long fish.’
    \z


\ea%6
    \label{ex:fc:6}

    \gll   khòòj5  kin3  paa3  (toø=)caw4                sùù4\\
	 \textsc{1sg.p} eat    fish  (\textsc{mod\_cl}:\textsc{non.human}=)\textsc{2sg.p}    buy\\
    \glt  ‘I ate the fish (the one which) you bought.’
    \z

\is{modifier classifiers} \is{numeral classifiers} In practice, however, almost all modifier classifiers used in discourse come from the following set of three: \textit{phuø}, which has no corresponding numeral classifier, for humans, \textit{toø} ({\textless} \textit{too3}) for non-humans and \textit{qanø} ({\textless} \textit{qan3}) for inanimates (\textit{ø} indicates neutralization of tone). Although \textit{toø} ‘\textsc{mod\_cl}:\textsc{non.human}’ and \textit{qanø} ‘\textsc{mod\_cl}:\textsc{inanimate}’ are clearly related to the numeral classifiers \textit{too3} and \textit{qan3}, respectively, their semantics is much more general. The modifier classifier \textit{toø} can in fact be used with any noun with an animal or inanimate referent and \textit{qanø} can be used with any noun with an inanimate referent. Therefore, for inanimates, either modifier classifier is fine. This is illustrated in examples (7) and (8) for the noun \textit{sin5} ‘\ili{Lao} skirt’ (which in the numeral classifier system takes \textit{phùùn3} ‘\textsc{num\_cl:cloth}’). Semantically, (7) and (8) are equivalent \citep[141]{Enfield2007}. See \citet{Carpenter1986,Carpenter1991} for the same phenomenon in \ili{Thai}.

\ea%7
    \label{ex:fc:7}
    \gll    khòòj5  mak1    sin5      toø=nii4\\
	    \textsc{1sg.p} like\textsc{  }  \ili{Lao}.skirt  \textsc{mod\_cl}:\textsc{non.human}=\textsc{dem}\\
    \glt ‘I like this skirt.’
    \z

\ea%8
    \label{ex:fc:8}
    \gll    khòòj5  mak1    sin5      qanø=nii4\\
    \textsc{1sg.p} like\textsc{  }  \ili{Lao}.skirt  \textsc{mod\_cl}:\textsc{inanimate}=\textsc{dem}	\\
    \glt  ‘I like this skirt.’
    \z

Although the use of a modifier classifier has a unitizing function and strongly implies singular, its use with a numeral other than ‘one’ is possible; in fact if a noun is modified by both a numeral and a demonstrative the modifier classifier construction is used. This is shown in example (9) \citep[140]{Enfield2007}.

\ea%9
    \label{ex:fc:9}

    \gll  kuu3    siø=kin3    paa3  sòòng3  toø=nii4\\
	    \textsc{1sg.b}  \textsc{irr}=eat    fish  two    \textsc{mod\_cl}:\textsc{non.human}=\textsc{dem}\\
    \glt  ‘I’m going to eat these two fish.’
    \z

\ili{Lao} provides a particularly interesting instance of what we are looking for, namely a set of data where we might reasonably consider postulating two systems of the same general type (two systems of classifiers). It is therefore natural to want to compare the two systems. In \tabref{tab:1} we draw up a matrix which integrates the numeral and modifier classifiers of \ili{Lao}. The leftmost column gives the classes of nouns and the second column lists the appropriate classifier in the numeral classifier construction. Then, for each numeral classifier, the table specifies which modifier classifiers are possible. For reasons of space we have to restrict the number of numeral classifiers, but this is not a problem since all classifiers not covered in \tabref{tab:1} are for inanimates, which means that \textit{toø}, \textit{qanø} or the phonologically reduced form of the numeral classifier can be used. They all follow the pattern given in the row labelled “etc.”.
\definecolor{LightGray}{gray}{0.85}
\definecolor{DarkGray}{gray}{0.7}
\begin{table}
\begin{tabularx}{\textwidth}{lp{1.5cm}p{1cm}p{1.4cm}p{1.5cm}X}
\lsptoprule 
 &  & \multicolumn{4}{c}{Modifier classifiers}\\
 \hhline{~~----}\\[-.9em]
Assignment& \parbox[t]{1cm}{Numeral\\classifiers}  & \textit{phuø} \mbox{‘human’} & \textit{toø} {‘non-human’} & \textit{qanø} \mbox{‘inanimate’} & \mbox{Reduced form of} \mbox{numeral classifier}\\
\midrule
human & \textit{khon2} & yes & no & no & yes\\
monk & \textit{qong3} & yes & no & no & yes\\
animal & \textit{too3} & no & \cellcolor{DarkGray} yes & no & yes [= \textit{toø}]\\
small thing & \textit{qan3} & no & \cellcolor{DarkGray} yes & \cellcolor{LightGray} yes & yes [= \textit{qanø}]\\
line & \textit{sên5} & no & \cellcolor{DarkGray} yes & \cellcolor{LightGray} yes & yes\\
lump & \textit{kòòn4} & no & \cellcolor{DarkGray} yes & \cellcolor{LightGray} yes & yes\\
cloth & \textit{phùùn3} & no & \cellcolor{DarkGray} yes & \cellcolor{LightGray} yes & yes\\
etc. & etc. & no & \cellcolor{DarkGray} yes & \cellcolor{LightGray} yes & yes\\
\lspbottomrule
\end{tabularx}
\caption{\ili{Lao} numeral and modifier classifiers}
\label{tab:1}
\end{table}

\is{modifier classifiers} The phonologically reduced form is always an option in the modifier classifier construction. For humans, either \textit{phuø} or \textit{khonø} can be used in the modifier classifier construction. For monks, these are possible but considered disrespectful. For animals, \textit{toø} is used. For inanimates, either \textit{toø} or \textit{qanø} are possible. 

\is{numeral classifiers} In \tabref{tab:1} we see that for each numeral classifier we can fully predict which modifier classifiers are possible. Given that the modifier classifier system is small and based on general semantic divisions, it is not surprising that it can be predicted from the system with the larger inventory (and hence smaller divisions), i.e. the numeral classifier system.

A good test case which indicates the dependency between the two systems is situations in which different classifiers can be used depending on properties of the referent; here we can examine whether one system is still predictable from the other. This investigation is not intended as a contribution to the semantics of classifier systems, rather our focus is on the dependency or lack of dependency between systems.

We start with the relatively straightforward case of regular polysemy \is{polysemy} \citep{Apresjan1974,Nunberg1996}. As we would expect, the noun \textit{mèèw2} ‘cat’ takes the numeral classifier \textit{too3} ‘\textsc{num\_cl}:\textsc{animate}’ and the modifier classifier \textit{toø } ‘\textsc{mod\_cl}:\textsc{non.human}’. The same classification is possible, if the referent is not a real cat but a toy cat. This is an instance of a regular polysemous relation between an animal and a representation of that animal. With an inanimate toy as the referent, the numeral classifier \textit{qan3} ‘\textsc{num\_cl}:\textsc{small.object}’ is also possible, as in (10), which would not be acceptable for living cats. This general fact related to polysemy needs to be specified only once; the fact that it holds true equally of the modifier classifier is fully predictable, as in (11).

\ea%10
    \label{ex:fc:10}
    \gll    mèèw2  saam3  qan3\\
    cat      three    \textsc{num\_cl}:\textsc{small.object}	\\
    \glt ‘three toy cats’
    \z

\ea%11
    \label{ex:fc:11}
    \gll   mèèw2  qanø=nii4\\
    cat      \textsc{mod\_cl}:\textsc{inanimate}=\textsc{dem}	\\
    \glt  ‘this toy cat’
    \z

\is{polysemy} Regular polysemy is the straightforward situation, it does not provide strong support for our case, because we could argue that there are two systems of classifiers, numeral classifiers and modifier classifiers, and regular polysemy is available to each of them; assignment to each of them could operate independently, and the same result would be reached. Thus regular polysemy provides an argument, but hardly a strong argument, for the claim that the systems are in fact inter-dependent. 

We therefore move on to cases where a referent has been manipulated out of its normal shape. Even in these situations the systems are parallel. For example, paper normally comes in sheets. In \ili{Lao}, the noun \textit{cia4} ‘paper’ takes the numeral classifier \textit{phèèn1} ‘\textsc{num\_cl}:\textsc{flat}’. As expected, the modifier classifier for paper is \textit{phèènø} ‘\textsc{mod\_cl}:\textsc{flat}’ (or \textit{toø} ‘\textsc{mod\_cl}:\textsc{non.human}’ or \textit{qanø} ‘\textsc{mod\_cl}:\textsc{inanimate}’, as is possible for all inanimates). While we can use the same classification if the referent is a crumpled sheet of paper, now the numeral classifier \textit{kòòn4} ‘\textsc{num\_cl}:\textsc{lump}’ is also possible, as in (12), and so is the modifier classifier \textit{kòònø} ‘\textsc{mod\_cl}:\textsc{lump}’, as in (13). 

\ea%12
    \label{ex:fc:12}
    \gll  cia4     saam3   kòòn4\\
    paper    three    \textsc{num\_cl}:\textsc{lump}	\\
    \glt    ‘three crumpled pieces of paper’
    \z


\ea%13
    \label{ex:fc:13}
    \gll	  cia4     kòònø=nii4\\
    paper    \textsc{mod\_cl}:\textsc{lump}=\textsc{dem}	\\
    \glt ‘this crumpled piece of paper’
    \z

\is{modifier classifiers} \is{numeral classifiers} Going further, we shall see that there is predictability of the modifier classifier from the numeral classifier even if the referent does not have a normal or expected shape. Take, for example, pieces of putty, which is designed to be modelled into all sorts of shapes, but does not have an inherent shape of its own. In this situation the referent determines classifier use and there is no falling back on an inherent shape when the referent has been manipulated out of that shape. If the referent is a lump of putty the classifiers \textit{kòòn4} ‘\textsc{num\_cl}:\textsc{lump}’ or \textit{kòònø} ‘\textsc{mod\_cl}:\textsc{lump}’ are used, but not \textit{phèèn1} ‘\textsc{num\_cl}:\textsc{flat}’ or \textit{phèènø} ‘\textsc{mod\_cl}:\textsc{flat}’. If the piece of putty is flat the classifiers \textit{phèèn1} ‘\textsc{num\_cl}:\textsc{flat}’ or \textit{phèènø} ‘\textsc{mod\_cl}:\textsc{flat}’ are used, but not \textit{kòòn4} ‘\textsc{num\_cl}:\textsc{lump}’ or \textit{kòònø} ‘\textsc{mod\_cl}:\textsc{lump}’.

\section{Conclusion}
\largerpage
Recall that our concern is the analytical issue of recognizing systems with interesting dependencies as opposed to independent concurrent systems. This is a general issue. For instance, turning to a different feature, we note that languages of Australia \il{Australian} were frequently analysed as having two different case systems. \citet{Goddard1982} argues convincingly for integrated single systems. This fits these languages more readily into broader typological \is{typology} patterns, and also simplifies the analysis of verbal government in the particular languages. Similarly, \ili{Lao} has provided a fascinating study. At one level, we might say that there are two systems of classifiers, numeral and modifier classifiers, which appear in different constructions. In terms of the assignment of particular classifiers within those systems, however, we find an interesting dependency. Given the choice of numeral classifier, the appropriate modifier classifier is predictable. This is an argument for dependency between the systems. However, it appears not to be a strong argument. For ordinary uses of nouns, it might be objected that the lexical semantics of the noun are available equally for assigning both types of classifier. Yet this objection (in favour of two concurrent systems) is, perhaps, not fully convincing in those instances where the appropriate classifier cannot be assigned straightforwardly from the lexical semantics of the noun. Then, in cases of regular polysemy, \is{polysemy} the fact that the choice of modifier classifier seems to “follow” the choice of numeral classifier is also indicative, but again not fully convincing. When finally we look at manipulations of the referent, natural or less so, the fact that even here the choice of modifier classifier follows that of the numeral classifier confirms the interesting dependency between the two systems. Thus the use of the smaller set of forms is predictable given the larger set of forms; this fact prompts us to conclude that \ili{Lao} has a single integrated system of classifiers. \is{concurrent systems}

More generally, where there are potentially two systems in play, as we find in \ili{Lao}, we need to argue carefully for and against analyses which rest on a dependency between the systems. This is important for typological purposes, and it may also lead to a clearer view of the particular language being investigated. 

\section*{Acknowledgements} 
We would like to thank Nick Enfield for providing us with examples and commenting on an earlier version of this chapter. The support of the AHRC (grant: Combining Gender and Classifiers in Natural Language, grant AH/K003194/1) is gratefully acknowledged.

{\sloppy
\printbibliography[heading=subbibliography,notkeyword=this]
}

\end{document} 