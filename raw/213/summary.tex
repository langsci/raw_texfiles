\addchap{Summary and guide to the book}

This book is the revised version of my doctoral dissertation, which I finished in early 2018. As dissertations go, I was more concerned with getting the details right than with creating an elegant textbook account of the topic, so that eventually I ended up writing more than 400 pages on multi-verb constructions - a result that was neither intended nor particularly encouraging to potential readers. Yet, while working on the phenomenon, it became more and more obvious to me that multi-verb strings represent such a complex and challenging issue that they are best illustrated with numerous examples from as many different languages as possible. Therefore, during revision of the chapters and sections, I made only moderate cuts so as to provide the reader with a rich set of data and abundant discussion. As it was felt that the sheer volume might put off those readers who merely wish to read an introduction to the topic, or who are interested in specific information, I have added a short summary to the book, explaining the hypotheses, research questions and decisions, and summing up the main arguments on which the conclusions are based. The following sections are intended to guide the reader through the book, and point out the relevant sections which he or she might wish to look at more closely. 

\section*{Introduction}

Strings of two or more verbs within linguistic units are a well-known phenomenon from many languages of the world (a collection of examples is found in §\ref{underspecified}). But despite the fact that linguists are well aware of their presence, the knowledge of exactly why such strings exist, or how they are formed, is still rather limited. Languages not only differ in the type and frequency of strings used, but also in the way such strings are integrated into their grammatical systems (that is, the way they are subject to syntactic, morphological, and even phonological constraints). This book is an attempt to study variation in verb combinations from an areal perspective. Eastern Indonesia is a region in which most, if not all, languages show signs of using multi-verb strings, making it particularly well-suited to cross-linguistic analysis. A further advantage is that the region has already been studied within the verb serialisation framework. There are serial verb accounts available for individual languages, as well as for language groups (see §\ref{previouswork} for references and discussion). For the present study, a total of 2146 multi-verb strings from 32 languages and two affiliations, Austronesian and Papuan,\footnote{To be precise, Papuan is not a genealogical expression but rather a cover term for a set of mutually unrelated language families in the Australasian region, conveniently used to address the non-Austronesian languages. See §\ref{sec:geneticlineage} for discussion.} were analysed in terms of grammatical and semantic features, both by reviewing the available literature as well as by searching two extensive language documentation corpora (see §\ref{sec:data} for an overview of the data sample).

The book is divided into seven chapters. The first three chapters are devoted to introducing the topic (\chapref{ch:introduction}), the linguistic area (\chapref{ch:area}), and the body of literature on multi-verb constructions and related concepts (\chapref{ch:theory}). The following two chapters form the analytical core of the study. \chapref{ch:gram} is concerned with the grammatical behaviour of multi-verb constructions in Eastern Indonesia, while \chapref{ch:sem} presents an account of their semantic properties. The results obtained from these chapters then feed into a typology of multi-verb constructions as laid out in \chapref{ch:constructions}. \chapref{ch:discussion} wraps up the findings, and provides a discussion of potential directions for further research. 

A short summary of each chapter is given in the following sections. Before doing so, however, I will start with a short introduction to the terminology used.

\section*{Notes on terminology}

This book is not written under a specific linguistic theory, but draws from ideas and concepts from quite different and partially unrelated linguistic fields. To be more precise, it combines insights from serialisation typology with semantic approaches such as predicate decomposition and Davidsonian semantics. Alongside these terms and concepts, I will also introduce some new terminology in order to capture and name the patterns that I think can be made visible from combining grammatical typology and verb semantics. Key terms and concepts that are used throughout the book are written in small capitals the first time they appear in each chapter. The reader will find a short definition of these terms in the glossary.

\largerpage[-1]
In this study, I will assume that multi-verb strings are tokens of actual linguistic constructions, in the constructionist sense of the word (see §\ref{sec:construction} for discussion). That is, an occurrence of, say, a \textsc{go} verb\footnote{Under the assumption that verbs from different languages share comparable semantic components, reference to verb classes in a cross-linguistic sense is made by printing the gloss in small capitals. For further background and discussion, see §\ref{sec:decomposition} on lexical decomposition in verbs.} and an activity verb is not simply an ad-hoc formation, but can be traced to a constructional schema which informs the actual output, and sets limits to its pragmatic use (but see the discussion in §\ref{sec:discourse} for an alternative approach). Whenever I speak of \textit{construction} (in particular in \chapref{ch:constructions}), I refer to a constructional schema, and not to an \emph{instance} or \emph{case} of a construction. Unfortunately, the use of the acronyms SVC (serial verb construction) and MVC (multi-verb construction) in the literature is in most cases ambiguous. For instance, SVC may either mean a certain construction, or a specific \emph{instance} of a construction. The use of the short form MVC in this book is intended to avoid this ambiguity, and only refers to \emph{instances} of constructions (although the acronym is of course rather misleading). To give an example, a data point from the sample may be referred to as a \textsc{motion-to-action} MVC, but the motion-to-action \emph{construction} as such is referred to by writing out the term ``construction".\footnote{Speaking of constructions in cross-linguistic analysis raises the question whether these constructions only exist on the single-language level or whether there are shared \emph{areal} constructions that form by way of convergence or other mechanisms in scenarios of prolonged language contact. As it is possible to sort MVCs from different languages and language families into what looks like a recurring constructional schema, as I do in \chapref{ch:constructions}, one may assume that areal constructions indeed exist. A motion-to-action construction, for instance, would then in Sulawesi languages as well as in languages of Western Papua involve a comparable constructional template.} 

Another assumption on which this book is based is that multi-verb constructions in Eastern Indonesia do not just form a random set, but are related by some kind of family resemblance. I will, in this book, distinguish between four \textsc{types}, or families, of constructions: \textsc{component-relating construction}s, \textsc{stage-rela\-ting constructions}, \textsc{modifying constructions}, and \textsc{free juxtaposition constructions} (for a definition, see below or read §\ref{sec:levels-event}). Each of these MVC \emph{types} subsumes a set of multi-verb \emph{constructions}, which in turn are represented by the \emph{instances} (MVCs) in the data sample. The term \textit{type} is only used for these higher-order constructional groups. Grouping constructions into constructional types is not informed by morphosyntactic evidence, as I argue throughout the book, but by semantic mechanisms of complex event formation. Such mechanisms will be called \textit{semantic techniques} (see \chapref{ch:sem}, especially §\ref{sec:levels-event}).

Scholars that are new to the field of multi-verb strings may easily be confused by the wealth of concepts available from the literature: serialisation or SVCs, complex predicates, MVCs, verb chaining, medial verbs, and coverb constructions, among others, all seem to describe ways of conceptualising events by using two or more verb-like elements in close succession. While these concepts cannot be used interchangeably, they do seem to overlap. In this book, I will only discuss two of these concepts in more detail: SVCs and MVCs. Serialisation is the traditional concept for multi-verb strings which seem to neither involve different kinds of verbs (as opposed to, say, coverb constructions where there are two, clearly different classes of verbs), nor differential formal marking on the verbs (unlike, for instance, medial verbs with their specific morphology). The term MVC is, in this regard, similar to the concept SVC, yet its use is less fraught with scientific tradition and therefore, as I will argue in §\ref{section:multi-verbconstructions}, better suited to an explorative analysis of multi-verb strings in Eastern Indonesia. Note that the term MVC is used throughout the book, unless the examples referred to are taken from the serialisation literature, and are not included in the data sample. In such cases, multi-verb strings are still called SVCs, as they are in the source publication.

\section*{Grammatical versus semantic properties of MVCs}

As multi-verb strings consist of verbs, i.e., of grammatical elements, it seems logical to regard the whole phenomenon as a grammatical one. This is basically what the linguistic community has done so far. Most research on serialisation focused on \emph{grammatical} properties of SVCs, resulting in a range of classificatory approaches sorting multi-verb strings into formal categories. While this may have seemed promising at first, the results so far look rather disappointing. The established categories lack explanatory power. Take \citet{vanstaden2008serial} as an example: In their study on SVCs in Eastern Indonesia, constructions in which both verbs receive inflectional marking are dubbed ``independent serialisation" (as opposed to ``dependent serialisation", in which only one of the verbs receives formal marking). However, what seems intuitive and easy to define at the formal level lacks clear correlates at the functional level. That is, independent serialisation, as defined by van Staden and Reesink, can be associated with a wide range of functions in different languages: Motion, direction, instrument, comitative, manner, and aspect may all be encoded by independent serialisation in their data sample. At the same time, languages are not consistent in their choice: the same function may, in one language, occur as independent serialisation, in another language as dependent serialisation, and so on. 

In §\ref{previouswork}, three (mainly grammatical) approaches from the Australasian area will be discussed and critically examined with respect to the properties they possess, and whether they may be applied to the data sample. I will show that there are various problems associated with these approaches. One is that some of the proposed categories consist of more than a single grammatical property. This can be demonstrated by the above example. Independent serialisation in \citet{vanstaden2008serial} not only requires two (or more) verbs with identical inflection, but the verbs' arguments also need to keep their syntactic function unchanged. Van Staden and Reesink's definition thus excludes examples where the object of the first verb is reanalysed as the subject of the second verb (such as, for instance, in the \textit{hit pig die}-construction). Such cases of functional switch belong to yet another SVC category in van Staden and Reesink's typology. However, the data show that both properties, namely verb inflection pattern and syntactic functions, can, in principle, occur in different combinations, and are thus independent from each other. Lumping them together into a single category, as in van Staden and Reesink's ``independent serialisation", seems to complicate the picture rather than providing useful insights. 

For this reason, I take a step back and identify three morphosyntactic properties from these studies that were both independently used in languages, and accessible through a survey of published data: (i) argument sharing between the verbs in multi-verb strings, (ii) head marking on the verbs, and (iii) contiguity between the verbs (i.e., how many constituents may come to stand between the verbs). All multi-verb strings in the data sample were annotated for these properties. In \chapref{ch:gram}, I present the results of this analysis. I show that the patterns emerging from such a morphosyntactic analysis will not help understand how multi-verb strings come into being in the first place, nor when they are used in the languages of Eastern Indonesia.

\chapref{ch:sem} thus shifts our perspective to the semantics of multi-verb strings. Looking into how verbs and verb combinations shape linguistic event expressions is, I argue, crucial for gaining an understanding of the diversity of multi-verb patterns in the data sample. The chapter consists of three parts. The first part provides the conceptual basis by discussing the relationship between real world events and linguistic event concepts (see §\ref{sec:real-world-linguistic-events}). I demonstrate that linguistic event expressions exist both at the lexeme level as well as at higher (predicate, clause, discourse) levels. Each of these conceptual levels is then assigned a specific kind of event. §\ref{sec:event-typology} introduces \textsc{lexeme-level events} (LLEs), \textsc{predicate-level events} (PLEs), and \textsc{clause-level events} (CLEs). I show that verbal interaction in MVC formation can, in fact, take place at any of these levels (see discussion in §\ref{sec:levels-event}). 

\newpage
In the second part of \chapref{ch:sem}, I turn to semantic frameworks that help explain what happens during the formation of multi-verb strings: event arguments and lexical decomposition. Davidson introduced a further component to the argument frame of verbs: the event argument. Looking at anaphor use in \ili{English}, he observed that \textit{it}-constructions often refer back to a previous predication \citep{davidson1967logical}. Familiar as this may seem at first, he then made a further intriguing observation: If \textit{it}, such as in \textit{it happened at midnight}, refers to some event, why is it singular? Clearly, there must be a single, specific antecedent available for \textit{it} to appear in singular number. The explanation, he argued, is that event construals in language are conceptualised as specific identifiable entities. To model an event antecedent, Donaldson assumed that certain verbs possess a hidden event argument, which may then be targeted by \textit{it} and other anaphoric expressions. In §\ref{sec:davidsonian}, I have a closer look at Donaldson's event arguments, and how they might relate to MVC formation in Eastern Indonesia. 

Lexical decomposition is another semantic approach that proves useful in MVC analysis. Dissecting verbs into smaller sublexical predicates such as \textbf{move'} or \textbf{do'} provides a set of semantic constants to which MVC formation appears to be sensitive. In §\ref{sec:decomposition}, I introduce different approaches to predicate decomposition and discuss their applicability to MVC analysis. The insights from these sections form, in part three, the foundation for a theory of semantic interaction in MVC formation (see §\ref{sec:levels-event}, and below).

\section*{Semantic techniques and MVC types}

One of the defining properties of events in the Davidsonian framework is that each event represents an identifiable chunk of something going on at a particular place and time. This something going on is formally encoded in the event argument, and thus part of the argument frame of a verb (or a verb combination). Not all predicates, however, seem to have an event argument. While, say, \textit{Jones buttered the toast at midnight} is just fine, \textit{Jones was (being) fast at midnight} or \textit{Jones possessed the toast at midnight} sound odd (at least without additional context). Statives are among the most prototypical examples of verbs (or, rather, predicates) that may lack an event argument (the opposite, however, has been argued for under the Neo-Davidsonian paradigm; see for instance \citealt{higginbotham2000events}). Thus, we may distinguish between those verbs that license a spatiotemporally definable event (following \citealt{carlson1977reference} and the stage-level/individual-level distinction, I call this an \textsc{event stage}), and others that do not. This assumption is crucial to my theory of different kinds of semantic interaction in MVCs as presented in §\ref{sec:levels-event}. The idea behind this is quite simple: a combination of verbs, each bringing in its own event stage, will quite naturally yield a two-stage event construal, such as the \textsc{go-do}-construction that is discussed under the label motion-to-action in §\ref{sec:motion-to-action}. A combination of verbs, of which only one brings in an event stage (think of a stative verb like \textsc{fast} entering a multi-verb construction), will instead lead to a single-stage event construal.

Two-stage event construals come in two forms: a tight construal, and a loose one. The tight construal is called \textsc{stage-relating construction} (SREL is my shorthand) because two event stages come to stand in close conceptual relation to each other (I argue that this is closer than in simple clause coordination). §\ref{sec:stage-relating} presents an overview of the stage-relating constructions that can be found in the Eastern \ili{Indonesian} data sample. The second kind of two-stage event construal is named \textsc{free juxtaposition} (abbreviated FJUX), and appears to lack some of the constructional restrictions present in stage-relating proper (see §\ref{sec:criteria_mvcs} for criteria that help distinguish between the different construction types). Free juxtaposition rather resembles a kind of asyndetic coordination and probably forms only a peripheral MVC type in Eastern Indonesia. §\ref{sec:fjux} presents examples from the data sample and discusses different constructions.

If, on the other hand, one of the verbs in a MVC does not contribute its own event stage, the outcome will be a single-event construal. This is the case in \textsc{modifying construction}s (or short: MOD) with combinations of dynamic and stative verbs, but also with certain other stageless verbs, as I argue in §\ref{sec:modification}. The data sample attests to a wide range of modifying constructions in Eastern \ili{Indonesian} languages, as we will see in §\ref{sec:modifying}. A further group of MVCs does consist of verbs that can project an event stage when in simplex predicate function, but fail to do so in certain verb combinations. This is the point where insights from verb decomposition theories come into play. I argue in §\ref{sec:merging} that verbs with identical sublexical predicates merge their lexical structure, rather than each projecting its own event stage. The result is a second kind of single-stage event expression in which each verb contributes some of the lexical components to the resulting construction. This formation technique yields what I call \textsc{component-relating construction}s (CREL is a shorthand). A discussion of the component-relating constructions from the Eastern \ili{Indonesian} data sample can be found in §\ref{sec:crel}.

Prototypical component-relating constructions consist of two (or more) motion verbs each contributing part of the information. For instance, take a combination of manner of motion plus path, as in \textit{She ran up that hill}. In \ili{English} (as in satellite-framed languages in general) information concerning the path of a motion event are mostly encoded by particles and appositions. This is in contrast to so-called verb-framed languages which tend to lexicalise path information in verbs (yielding something like \textit{She upped that hill running}). Still other languages seem to use neither strategy, but employ multi-verb strings instead where both manner and path (or related motion information) are contributed by verbs (see also \citealt{Ameka2013}). This then looks like \textit{She ran ascended that hill} in many multi-verb languages. Such cases, I argue, make up their own specific category in MVC typology because one of the verbs' event stages is suppressed (or, rather, merged with the other one) under the formation process. This \textsc{merging} would be hard to explain unless there is some kind of driver inside the verbs' lexical structure. This is exactly where lexical decomposition provides new insights by assuming that certain verb classes possess identical sublexical features (for instance, by sharing the feature \textit{motion verb}). Such shared semantic constants then trigger, under certain circumstances, a merging interpretation, i.e., the event stages of both verbs are understood to be identical rather than following up on each other.

\section*{Further issues}

There are some further issues covered in this book, two of which I briefly want to draw attention to here, as they may prove vital for understanding multi-verb strings. These are hierarchy in MVCs, and the role of discourse in MVC formation.

Multi-verb strings not only come in binary structures, but sometimes can host three or even more verbs in a series. Such strings are typically analysed as flat structures of concatenated verbs. Yet, there is nothing that would prevent such verb series to be analysed as hierarchical structures, i.e., multi-verb strings that contain in one of their slots not just a verb, but another embedded multi-verb string. This is, in fact, what I assume throughout this book. MVC embedding, I want to argue, is not only possible, but actually quite common in Eastern Indonesia (though not all languages will allow it to the same degree). I metaphorically speak of \textsc{stacked MVC}s, and give examples of such hidden hierarchies at various points in the course of the book (see also §\ref{sec:stackedmvcs} for discussion).

\largerpage
A final issue I want to emphasise here is the role of discourse development in MVC formation. I have mentioned above that I regard multi-verb strings as constructions that are based on shared linguistic templates. This conforms well to the type of data that can be derived from published sources. However, a close look at corpus data, or at otherwise unmodified ``messy" natural speech data reveals that there are striking mechanisms of MVC formation rooted in discourse planning and development. Some such mechanisms have been described under labels such as tail-head linkage, or summarizing constructions (though mostly without  reference to multi-verb strings). Unfortunately, most of the published data on multi-verb strings present isolated SVCs without any trace of the discourse environment, so that it was not possible within this study to focus more on the question of discourse. Yet, from the data I present and discuss briefly in §\ref{sec:discourse}, it does seem clear that discourse requirements have a share in shaping multi-verb strings by way of repetition and compression.
