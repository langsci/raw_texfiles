\documentclass[output=paper]{langsci/langscibook}  
\ChapterDOI{10.5281/zenodo.1314337}
\title{East Benue-Congo} 
\author{%
John R. Watters  \affiliation{SIL International}  
} 
 

\abstract{Chapter one introduces this volume on East Benue-Congo (EBC) and the chapters addressing issues of nouns, pronouns, and verbs within specific branches and EBC as a whole. The chapter identifies the location of EBC and its branches as well as the external and internal classification of EBC. It situates EBC’s likely original homeland and the geography of its probable expansion routes that led to the current location of its branches. It then provides a context for the chapters focused on noun classes in EBC in general and nominal affixes in Kainji and Plateau in particular, as well as the reconstruction issues they raise. It also notes certain issues related to Bantoid and to the presence of the Bantu languages within Bantoid, especially its dominance within Bantoid that has the potential of skewing historical analyses. 
}  

\maketitle
\begin{document}
\label{sec:1}
\section{East Benue-Congo (EBC): its location}\label{sec:watters:1}
 
The category label ‘\ili{East Benue-Congo}’ (or ‘\ili{Eastern Benue-Congo}’) is a relatively recent one. It is widely known from  \citet[30-36]{WilliamsonBlench2000} in their introduction to the language family ‘\ili{Benue-Congo}’. They report that \citet{Blench1989} had actually proposed it a decade earlier in response to the reassignment of what had been \ili{Eastern Kwa} languages into a “New” \ili{Benue-Congo}, a reassignment proposed by  \citet{BennettSterk1977}. Blench proposed that the \ili{Eastern Kwa} languages, now assigned to \ili{Benue-Congo}, be given the title ‘\ili{West Benue-Congo}’. That left the original ‘\ili{Benue-Congo}’ languages with the complementary title of ‘\ili{East Benue-Congo}’. This label represents the result of a process dating back to \citet{Greenberg1963} and even earlier to \citet{Westermann1927} and \citet{Johnston191922}. Westermann had given a set of West African languages the title ‘\ili{Benue-Cross}’. \citet{Greenberg1963} then added the \ili{Bantu} languages to Westermann’s \ili{Benue-Cross}, expanding the set of related languages and assigning it the new name ‘\ili{Benue-Congo}’. These details and more on the historical process of categorization from Greenberg’s proposed \ili{Benue-Congo} to today’s \ili{Benue-Congo} are provided in \citet[247-274]{Williamson1989} and \citet[30-36]{WilliamsonBlench2000}. 

A few points are worth highlighting and reiterating from this history about \ili{Benue-Congo} and its relationship to the EBC of this volume. First, the content of the category ‘\ili{East Benue-Congo}’ has not changed since \citet{Greenberg1963} proposed it as ‘\ili{Benue-Congo}’. In fact, the category label referred to in much of the literature from Greenberg in 1963 until \citeauthor{WilliamsonBlench2000} in 2000 was simply ‘\ili{Benue-Congo}’ or ‘\ili{Eastern South-Central Niger-Congo}’ from \citet{BennettSterk1977}. For example, \citegen{deWolf1971} study \textit{The noun class system of \ili{Proto-Benue-Congo}} concerned the languages that are now being referred to as ‘\ili{East Benue-Congo}’, a subset of the new, current \ili{Benue-Congo} family.    

Second, Greenberg made the decision, a radical one for its time, yet a reasonable one, that all the Narrow \ili{Bantu} languages formed a subgroup within a subgroup of \ili{Benue-Congo}. Greenberg’s proposal is now generally accepted. This inclusion of the \ili{Bantu} languages has not changed with the adoption of the label ‘\ili{East Benue-Congo}’. All \ili{Bantu} languages are a subgroup of the Bantoid branch within EBC.

Third, Greenberg identified four branches within his \ili{Benue-Congo}, namely, \ili{Plateau}, \ili{Jukunoid}, \ili{Cross River}, and \ili{Bantoid} \citep[8-9]{Greenberg1966}. Plateau is sometimes referred to as \ili{Platoid} \citep{Gerhardt1989}. However, more recently    \citet[31]{WilliamsonBlench2000} identified the \ili{Kainji} languages as forming a fifth branch. The \ili{Kainji} languages in Greenberg’s and previous classifications was positioned as a \ili{Plateau} subgroup, specifically formerly \ili{Plateau} 1a, b. It now forms a fifth branch of the new EBC.

Fourth, \citet[31-32]{WilliamsonBlench2000} note that \citet{Shimizu1975} and \citet{Gerhardt1989} proposed that \ili{Jukunoid} be included within \ili{Platoid}. Another way to state their proposal is that \ili{Jukunoid} is more closely related to \ili{Platoid} than it is to \ili{Cross River} or \ili{Bantoid}. Williamson and Blench indicate this conclusion in their figure \figref{fig:watters:2}.11 \citet[31]{WilliamsonBlench2000} by including \ili{Jukunoid} as a branch of a larger genetic unit that includes the parallel branches of \ili{Kainji}, three \ili{Platoid} groupings, \ili{Beromic}, and \ili{Tarok}. This proposed grouping provides some internal structure to  EBC, namely, a two-way division of the five EBC branches into what Williamson \& Blench label ‘\ili{Central Nigerian}’ (i.e. \ili{Kainji}, \ili{Plateau} with further elaboration, and \ili{Jukunoid}) and ‘\ili{Bantoid-Cross}’ (i.e. \ili{Cross River} and \ili{Bantoid}).

\begin{figure}[b!]
\includegraphics[width=\textwidth]{figures/africa.pdf}
\caption{\label{fig:watters:1} The locations of the five branches of EBC}
\end{figure}


\newpage 
The simplified map in \figref{fig:watters:1} identifies the current general location of each branch of EBC. Two branches, 
\ili{Kainji} (1) and 
\ili{Platoid} (2) are found entirely within Nigeria. The other three branches, 
\ili{Jukunoid} (3), 
\ili{Cross River} (4), and 
\ili{Bantoid} (5) 
are represented in both Nigeria and Cameroon, but the representation of 
\ili{Jukunoid} (3) and 
\ili{Cross  River} (4) in Cameroon is minimal. 
\ili{Bantoid} (5) 
in Nigeria and Cameroon, however, includes the following groups in both countries: \ili{Jarawan}\footnote{  \citet{SimonsFennig2018} report two \ili{Jarawan} languages in Cameroon: Mboa is listed with 1,490 speakers in 2000, and Nagumi is listed as extinct.} , \ili{Dakoid}\footnote{\citet[182-183]{Boyd1989} was not convinced that Daka (Dakoid) was closer to \ili{Bantoid} (represented by \ili{Vute}, \ili{Mambiloid}, \ili{Bantoid}) than it was to some \ili{Gur} languages. However, eleven years later    \citet[27]{WilliamsonBlench2000} state that the inclusion of \ili{Dakoid} within \ili{Benue-Congo} “is now widely accepted”. The most recent consideration of \ili{Dakoid} being \ili{Bantoid} is found in \citet{Blench2012} in which the use of nominal suffixes is pointed out as a trait that \ili{Dakoid} shares with \ili{Mambiloid}.}, \ili{Mambiloid}, \ili{Tivoid}, \ili{Beboid}, \ili{Grassfields}\footnote{Of the 67 \ili{Wide Grassfields} languages only two or three are also spoken in Nigeria.}, and \ili{Ekoid}. \ili{Nyang} and \ili{Tikar} are only found in Cameroon. Meanwhile, the \ili{Bantu} group (6) within \ili{Bantoid} is not found in Nigeria, but is found in Cameroon and multiple countries across central, eastern, and southern Africa, as the map shows. The \ili{Bantu} languages are found between the dotted lines in \figref{fig:watters:1} that run across this central, eastern, and southern region of Africa. The \ili{Bantu} group is the dominant group within \ili{Bantoid} and even within EBC in terms of its geographic spread, the number of languages included, and the number of speakers involved. However, the map provides a helpful reminder that the size of a branch or a group or subgroup is not determinant in the process of comparison and reconstruction. The smaller branches must also be considered as being as potentially significant as a dominant group like the \ili{Bantu} subgroup in reconstructing proto-\ili{Bantoid}, proto-\ili{Bantoid-Cross}, and proto-EBC. 




The distribution of EBC branches strongly suggests that EBC originated in Nigeria. (See \sectref{sec:watters:3} for more details and references.) This conclusion derives from the assumption that where a language family is more fragmented and shows greater diversity, that is where the given language family likely originated. Diversification develops over time and so greater linguistic diversity in one region generally represents greater historical time depth than a more homogeneous region. \citet{Henrici1973} and \citet{Heine1973} demonstrated that the most diverse region in \ili{Bantu} is its northwest region that borders on the other Bantoid groups in Cameroon. Building on that observation, the other EBC branches outside \ili{Bantoid} represent even greater diversity, with \ili{Kainji} and \ili{Platoid} indicating significant time depth. This is seen in the modifications and reconfigurations of their noun class systems as shown by Blench (\chapref{sec:3} \& \chapref{sec:4}) in this volume.    

\section{EBC: its classification}\label{sec:watters:2}

Turning from the geographic location of the EBC branches and their possible relative time depths, \figref{fig:watters:2} summarizes the current understanding of the external and internal classification of EBC. Externally, EBC is a sister subfamily of the subfamily \ili{West Benue-Congo} within the larger family of \ili{Benue-Congo} languages. Internally, the five branches of EBC divide into two major units: \ili{Central Nigerian} (\ili{Kainji}, \ili{Plateau}, and \ili{Jukunoid}) and \ili{Bantoid-Cross} (\ili{Cross River} and \ili{Bantoid}). 

 
\begin{figure}
% \includegraphics[height=.3\textheight]{figures/wattersEBCclassification.png}
\begin{forest}
  for tree={ 
    grow'=0,
    child anchor=west,
    parent anchor=south,
    anchor=west,
    calign=first,
    edge path={
      \noexpand\path [draw, \forestoption{edge}]
      (!u.south west) +(7.5pt,0) |- node[inner sep=1.25pt] {} (.child anchor)\forestoption{edge label};
    },
    before typesetting nodes={
      if n=1
        {insert before={[,phantom]}}
        {}
    },
    fit=band,
    before computing xy={l=15pt},
  }
[Benue-Congo (976)
  [West Benue-Congo (83)]
  [East Benue-Congo (893)
    [Central-Nigerian (133)
      [Kainji (59)]
      [Plateau (54)]
      [Jukunoid (20)]
    ]
    [Bantoid-Cross (760)
      [Cross-River (68)]
      [Bantoid (692)
	[Wider Bantoid (152)]
	[Bantu (540)]
      ]
    ]
  ]
]  
\end{forest}

\caption{The external and internal classification of East Benue-Congo}
\label{fig:watters:2}
\end{figure}



To gain a sense of the number of languages involved in EBC, a proposed number of languages associated with the given unit in \figref{fig:watters:2} is provided from \citet{SimonsFennig2018}. The \ili{Niger-Congo} macrofamily is listed as the largest language family in the world in that it has the greatest number of listed living languages: 1,539. \ili{Benue-Congo} is the largest family within \ili{Niger-Congo}, listed with 978 languages or 63\% of all \ili{Niger-Congo} languages. Of those 978 \ili{Benue-Congo} languages, EBC is listed with 893, or 58\% of all \ili{Niger-Congo} languages and 91\% of all \ili{Benue-Congo} languages. Within EBC, Bantoid has 692 languages or 45\% of all \ili{Niger-Congo} and 71\% of all \ili{Benue-Congo} languages and is clearly the dominant grouping. Within \ili{Bantoid}, the \ili{Bantu} languages account for 78\% of all Bantoid languages and more than one-third of all \ili{Niger-Congo} languages. That leaves 153 \ili{Bantoid} languages in the nine other \ili{Bantoid} groups.

The EBC languages are distributed over an extraordinary land mass. They cover much of Nigeria from the northwest and north to the center and the east and southeast; all of southern Cameroon; and multiple nations of central, eastern, and southern Africa, as shown in \figref{fig:watters:1}. The speakers of these languages number in the hundreds of millions. 

It should be noted at this point that the classification within EBC, at the level of its branches and their internal groups, is still not fully settled. This is also true at the macro level of \ili{Niger-Congo}. Various proposed groups have indeterminate boundaries with those that are considered most closely related to them. Both \citet[109-122]{Blench2006} and Good (to appear) make this point emphatically. Many groups have a certain coherency, but it is still a matter of further research as to where the actual boundaries between groups lie and what linguistic features identify those boundaries. This includes the boundary between \ili{Bantu} and the other \ili{Bantoid} groups along the northwest boundary of \ili{Bantu} Zone A. The use of trees and references to groups by name does not mean that the status of the group relative to other groups is well defined. What defines the boundaries is often unclear in part due to a lack of reconstructions of phonologies, morphologies, and lexicons. Given this uncertainty in classification, it may be more helpful in some cases to identify a core set (or sets) of languages within a given group that appear to bear a close genetic relationship to one another. Reconstruction of the phonology, morphology, and lexicon of such core sets could then be compared to other core sets, hopefully assisting in the comparative process and reconstruction of larger groupings and potentially identifying relevant boundary markers. However, for now, the impact of the imprecise nature of boundaries is that it will not always be easy to identify what is an innovation or what is a shared inheritance. Also, it may have to be accepted that the imprecise nature of classification of these languages will remain with us due to incomplete data sets, the methods used, and ultimately the linguistic histories of these languages.  


\section{EBC: likely origins and expansion} \label{sec:watters:3}

\citet[269-272]{Williamson1989} and \citet[134]{Blench2006} follow \citet{Armstrong1981}. They propose that the ancestral center of the \ili{Benue-Congo} languages is likely located in the region of the confluence of the Niger and Benue Rivers. This location is indicated in \figref{fig:watters:3} as the “Benue-Congo Homeland.” The subsequent expansion from that location is mapped out in \figref{fig:watters:3}.

\begin{figure}
\caption{Benue-Congo expansion from homeland to current locations}
\label{fig:watters:3}
\includegraphics[height=.45\textheight]{figures/watters-img2.pdf}
\end{figure}

The proposal that the confluence of the Niger and Benue rivers was the likely point of origin of \ili{East Benue-Congo} is the most reasonable one despite the extraordinary current geographical distribution of the \ili{Benue-Congo} languages (\figref{fig:watters:1}).  It is reasonable based on two assumptions.


First, it is the location that most easily allows for a shared origin of both the \ili{West Benue-Congo} and EBC languages, providing a plausible point of origin. Whether there is a clear linguistic demarcation between the West and East sectors of \ili{Benue-Congo} or not, the region around the Niger-Benue confluence provides the simpler explanation of their distribution in the absence of evidence to the contrary. 

\newpage 
Second, the greatest linguistic diversity is found in the western region of EBC, that is, in Nigeria and Cameroon, whereas the \ili{Bantu} languages further east do not display anything close to the same linguistic diversity even though they cover an exceptionally larger geographical expanse within Africa. Such diversity would indicate that speakers of \ili{Benue-Congo} languages had been resident in the region of Nigeria and Cameroon well before the \ili{Bantu} expansion began.

   
 



\figref{fig:watters:3} suggests the probable expansion routes of EBC people from the Niger-Benue confluence to their current locations. This multi-directional expansion was likely due to agricultural, ecological, economic, and social factors. It recognizes the two-way division of \ili{Benue-Congo} into western and eastern areas. The ancestors of the \ili{West Benue-Congo} largely migrated southwest of the confluence except for the \ili{Igboid}, who crossed to the eastern side of the Niger, while the ancestors of the \ili{East Benue-Congo} languages migrated northwest, north, and east of the confluence.  The \ili{Kainji} are distributed primarily northwest of the Niger-Benue confluence; the \ili{Plateau} are essentially north of the confluence; and the \ili{Jukunoid} are to the east, up the Benue River basin. The \ili{Bantoid-Cross} likely also migrated east up the Benue  River basin, but probably south of the river and the \ili{Jukunoid}, settling in a region marked out by modern-day Makurdi, Wukari, and Gboko. Later the \ili{Cross River} peoples migrated south into to the Cross River basin and expanded along its western banks to the \ili{Atlantic} coast, later crossing over to the eastern banks of the Cross River. Some of the \ili{Bantoid} peoples stayed in the \ili{Bantoid-Cross} homeland or spread out along what is now the Nigeria-Cameroon border. Others migrated further to the east into the mountains of Cameroon and then across the Cameroon Volcanic Line to the eastern slopes of the mountains of western Cameroon and eventually into the Sanaga River valley. From this last region \ili{Bantu} began its expansion into central, eastern and southern Africa. 

For some temporal perspective, \citet[126-138]{Blench2006} discusses models of the \ili{Niger-Congo} expansion. He proposes the beginning of \ili{Benue-Congo} to be around 5500 BP, \ili{Bantoid} to be around 4500 BP, and the \ili{Proto-Bantu} period to 4000 BP. \citet[106-116]{Ehret2016} dates \ili{Proto-Bantu} to 3000 BCE, and provides further elaboration of the \ili{Proto-Bantu} communities and their continuing expansion.


\section{EBC: nouns, pronouns, verbs}\label{sec:watters:4}
This volume is the first in what will hopefully be a growing set of edited volumes and monographs concerning \ili{Niger-Congo} comparative studies. This first volume addresses matters that are relevant to the entire EBC family as well as the particular branches of \ili{Kainji}, \ili{Plateau}, and \ili{Bantoid}. The \ili{Jukunoid} and \ili{Cross River} branches are not the subject of these chapters, but they will be addressed in the next volume concerning EBC. In the case of \ili{Bantoid}, the particular focus is on \ili{Grassfields} and \ili{Bantu} though other \ili{Bantoid} subgroups are referenced. The potential topics for comparative studies among these languages are numerous, but this volume is dedicated to the specific issues of nominal affixes, third person pronouns, and verbal extensions.

In terms of comparative studies, these chapters fall under various topics. Three chapters concern the wider issue of comparative morphology. In particular, they concern the morphology of noun class systems and the possibility of reconstructing the nominal affixes and concord elements of the proto-classes. Good’s chapter addresses the issue of identifying the systemic attributes that make up \ili{Niger-Congo} and EBC noun class systems. Blench’s chapters on nominal affixes in \ili{Kainji} and \ili{Plateau} demonstrate the significant challenges that exist in reconstructing the nominal systems of these two EBC branches.

Three other chapters concern wider issues of reconstructing Bantoid. One of these issues involves the dominance of \ili{Bantu} in relation to the nine other identified \ili{Bantoid} subgroups. It is generally assumed that \ili{Bantu} is the most conservative group within \ili{Bantoid} as well as EBC. Yet, at the same time, \ili{Bantu} certainly has innovated. So, to what extent can one assume that \ili{Proto-Bantu} equals \ili{Proto-EBC}, \ili{Proto-Bantoid-Cross}, let alone \ili{Proto-Bantoid} that most narrowly includes \ili{Bantu} within its grouping? This is a tempting assumption to make, but it is a process of attribution that can be suspect. The relationships within \ili{Bantoid} probably involve layering of units which involve both historical processes of retention and innovation as well as language contact and areal processes. The challenge is to know if a given phenomenon reconstructed at one level can automatically be attributed to the higher level available. This issue presents itself in Hyman's chapters on verbal extensions and nasal nominal prefixes. Finally, Hyman’s  other chapter on third person pronouns in \ili{Grassfields} provides an excellent example of internal reconstruction within a subgroup in which the divergences are identified and validated as historical retentions in one case and innovations in the other. 


\section{Reconstructing nominal affixes of Proto-EBC: Kainji and Plateau}\label{sec:watters:5}

  Noun classes, with their system of nominal affixes and associated concord markers, are perhaps the major distinguishing feature of the \ili{Niger-Congo} macrofamily as well as its branches like the EBC family. In order to reconstruct the noun class system of \ili{Proto-EBC} and each of its branches, reconstruction will need to start at the lowest levels within each branch, using the comparative method. As   \citet[162]{CampbellPoser2008} write: “The comparative method has always been the primary tool for establishing these relationships.” It has served \ili{Indo-European} studies well over the past century. As \citet{Hall1950} notes for studying \ili{Proto-Romance}, referencing Trager\footnote{\citet[463]{Trager1946} wrote concerning the change of emphasis in the study of historical linguistics: “It seems to me that historical linguists must now restate their tasks much more precisely. When we have really good descriptive grammars of all existing \ili{French} dialects, we can reconstruct \ili{Proto-Francian}, \ili{Proto-Burgundian}, \ili{Proto-Norman-Picard}, etc. Then we can reconstruct \ili{Proto-French}; then, with a similarly acquired statement of \ili{Proto-Provencal}, we can formulate \ili{Proto-Gallo-Romaic};~next, with similar accurately developed reconstructions of \ili{Proto-Ibero-Romaic}, \ili{Proto-Italian}, etc., we can work out \ili{Proto-Romaic} as a whole.”} for support, the comparative method is the best method in reconstructing \ili{Proto-Romance}. Research began at the dialect levels of the Romance languages and was built up into larger and larger units until the forms of \ili{Proto-Romance} were determined. Relative to the languages of EBC outside of \ili{Bantu}, however, this method has been difficult to use in the past because of the lack of data. Access to each dialect level of most of these languages is simply not available, so using mass comparisons has been the common method.  Yet, more language data is available today than forty years ago when \citet{deWolf1971} proposed a reconstruction of the noun classes of Proto-EBC (“\ili{Proto-Benue-Congo}” at that time). 

In this context, Blench provides valuable overviews of noun class systems in the \ili{Kainji} languages in \chapref{sec:3} and the \ili{Plateau} languages in \chapref{sec:4} of this volume. These branches are further away from \ili{Proto-Bantu} and \ili{Bantoid}, where our understanding of what may have been included in the \ili{Proto-EBC} noun class system is clearer. They demonstrate how opaque a noun class system can become over time relative to more conservative contexts such as the \ili{Bantu} and \ili{Bantoid} ones. Along with the overview of noun class systems Blench provides an updated proposal for the comprehensive classification of these major subgroups. He also provides with each chapter a significant set of references, important material for future researchers. 

In the case of \ili{Kainji} (\chapref{sec:3}), a challenge to a straightforward comparative reconstruction of the \ili{Proto-Kainji} noun class system presents itself. Blench points out that the \ili{Kainji} languages and its subgroups are marked by significant diversity in noun class systems. This diversity suggests systems that have undergone various cycles involving analogical change, mergers, loss, and affix renewal. This means that it is highly unlikely that the full system for \ili{Proto-Kainji} can be reconstructed. On the other hand, subunits of \ili{Kainji} might lend themselves to some reconstruction and so provide possible insights when these are compared to the larger set of EBC languages. It would be important to do as much reconstruction as possible at the lower levels in order to provide as much comparative data as possible from \ili{Kainji}.

Encouragingly, Blench notes that there is sufficient evidence for \ili{Proto-Kainji} having classes 1/2 for persons, class 6a for liquids and some mass nouns, and a diminutive affix \textit{*kV-}. The class pair 1/ 2 \textit{*u-/*ba-} is cognate with the \ili{Proto-Bantu} \textit{*mu-/*ba-.} The class 6a prefix \textit{*mV-} is cognate with a class prefix \textit{*ma-} found throughout \ili{Niger-Congo}. The diminutive prefix \textit{*kV-} is likely cognate with the diminutive prefix \textit{ke-} that is attested in \ili{Plateau} languages (Blench p.c.) and with \textit{kɛ-} in the \ili{Bantoid}, \ili{Ekoid} language \ili{Mbe} (personal notes), suggesting it is likely a \ili{Proto-EBC} diminutive prefix.

On the other hand, Blench is uncertain about the possibility of reconstructing a homorganic nasal prefix for \ili{Proto-Kainji}. Such a prefix shows up in \ili{Bantoid} languages as the prefix for noun classes 9 and 10.

He also notes that the vowels of CV- prefixes are often underspecified. A similar process is found elsewhere in EBC where the phonological or even phoneti c quality of the prefix vowel harmonizes with the quality of the first vowel of the root.

An unusual proposal for \ili{Proto-Kainji} is that it might have had class trios rather than class pairs. The three-way distinction would involve distinguishing singular, countable plural, and non-countable plural.

The major conclusion is that \ili{Kainji} must have inherited a significant noun class system from \ili{Proto-EBC}. At the same time, the \ili{Kainji} languages appear to have experimented with that inheritance more vigorously than other major EBC subgroups and perhaps had more time to do so if they were the first group to separate from EBC. This diversity makes the reconstruction of the exponents of these classes, i.e. their nominal affixes and concord affixes, for the \ili{Proto-Kainji} noun class system a challenge and will likely result in a limited, partial view.

In the case of \ili{Plateau} (\chapref{sec:4}), the situation may be even bleaker for reconstructing \ili{Proto-Plateau} noun class exponents than in \ili{Kainji}. Blench notes in his concluding notes that “the connection with \ili{Niger-Congo} noun classes remains tenuous.” 

 
He does note evidence for a possible class pair referring to persons, the prefixes being \textit{*V-/*bV-}, as well as a nasal class used with “liquids, mass nouns, and abstracts”. Both of these are relevant to \ili{Proto-EBC} and the larger \ili{Niger-Congo} macrofamily. The form of this nasal prefix in \ili{Proto-Plateau} is uncertain, though \textit{*ma-} may be a possibility even if it is not common synchronically. \ili{Proto-Plateau} may also have had homorganic nasal prefixes, but their possible relationship to \ili{Proto-EBC} is not clear because their likely semantic relationship is unknown.

The conclusion in the case of the Plateau languages is that noun classes were a definite feature of \ili{Proto-Plateau}. However, what can be reconstructed as exponents of those classes is limited. As with \ili{Kainji}, detailed reconstruction of some subunits of Plateau may be productive and serve as a substitute for identifying the exponents of \ili{Proto-Plateau}. 

Therefore, it appears likely that the results from further research on these two major subgroups of EBC will not make a determinative contribution to the reconstruction of \ili{Proto-EBC} noun classes, but could play an important supportive role in confirming hypotheses about \ili{Proto-EBC} as they develop. This challenge to detailed reconstruction of EBC noun class exponents raises the question as to whether there might be another way to gain insight into the EBC noun class system. This other way would be to look at the noun class system from a systemic perspective as opposed to the micro level of morphemes. Blench (p.c.) reminds me that the data available to \citet{deWolf1971} could not justify his reconstruction of the exponents of the noun classes of EBC but instead he was influenced by knowledge of \ili{Bantu}. I will return to the influence of \ili{Bantu} studies below in \sectref{sec:watters:7}.


\section{Noun class systemic topics in EBC}\label{sec:watters:6}
 
Good (\chapref{sec:2}) offers a perspective of EBC noun class systems that focuses on their morphological properties. These properties will be noted two paragraphs below.  Some might contest this perspective, contending that it is merely typological, with no relevance to the reconstruction of the \ili{Proto-EBC} noun class exponents. However, I would suggest that a careful consideration of the points Good makes offers insights into reconstructing various features of the \ili{Proto-EBC} noun class system. They can provide frames for understanding the architecture of the subsystems that may have been operating in the larger system.

Good notes that in the reconstruction of EBC noun classes, the focus is on discrete exponents of the noun classes that form pairings to mark number on nouns. However, these exponents are elements in a larger system involving the classification of nouns that is associated with a variety of morphosyntactic properties. The identification of these properties (see below) as they obtain to \ili{Proto-EBC} is a valid and crucial research area in expanding our knowledge of \ili{Proto-EBC} noun classes and their systems. The research on the properties of the \ili{Proto-EBC} noun class system is not in opposition to detailed reconstruction, but is complementary. Given the less than sanguine conclusion about reconstructing exponents in \ili{Proto-Kainji} and \ili{Proto-Plateau} above (see \sectref{sec:watters:3}), system-based analysis could be helpful in expanding our understanding of \ili{Proto-EBC} noun classes.  

So what are some of these properties? In the case of \ili{Proto-EBC} and its subgroups, the reduction of noun classes in individual languages or subgroups must consider areal influence and not simply language-internal structural processes. Context matters. 

Within that context, there is the issue of kinds of affixes. Nominal prefixes are predominant, but there are EBC languages that have suffixes as exponents of a noun class as well. Even circumfixal elements are found. The possibility of prefixing, suffixing, and even circumfixal affixation needs to be accounted for in any full history of EBC noun classes. This includes the interplay of prefixes and suffixes according to the morphosyntactic context of the noun as seen in the language \ili{C’lela} in \ili{Kainji}. A given noun will have a prefix in one grammatical context but a suffix in another.

In terms of concord markers, several questions must be resolved. What are the domains of concord that are relevant to reconstructing \ili{Proto-EBC} noun classes? What is the minimal set of domains within the concord system for \ili{Proto-EBC}? How is concord with a given noun class indicated within the noun phrase, sentence, and discourse? Furthermore, how many series of noun class concord markers might there have been? Two seems to be the minimum, but there could have been more. 

Finally, there is also the need to determine noun class identity and class pairing. Humans in classes 1/2 seems stable for many languages, but many of the other pairings are not so stable. To what extent did the \ili{Proto-EBC} noun class systems have a non-canonical pairing structure for some classes? So while the past is viewed through the lens of the synchronic realities of current EBC languages, how much can be accounted for by the reconstruction of the \ili{Proto-EBC} noun class system and how much can be accounted for by losses and innovations through time, remembering all the while that the \ili{Proto-EBC} system is unlikely to have been a fully elegant, symmetrical, transparent system? 


\section{The long shadow of Bantu on Bantoid and potentially EBC}\label{sec:watters:7}
\subsection{Brief historical review}


Comparative and historical studies in EBC benefit from and are challenged by the coherence of the \ili{Bantu} languages. They form one subgroup within Bantoid, but it must be remembered that means they are also part of the larger EBC family. When \citet{Greenberg1963} proposed that \ili{Bantu} was actually a subgroup of \ili{Bantoid}, he stepped into an existing division among scholars as to the relationship between the \ili{Bantu} languages and the languages of West Africa. Some viewed the similarities between the two groups of languages as the result of accident while others viewed them as the result of a genealogical relationship, a shared origin.

\citet{Guthrie1962} attempted to explain the {\textquotedbl}Bantuisms{\textquotedbl} of the West Sudanic languages by claiming that speakers of a language or languages related to \ili{Proto-Bantu} had been absorbed into certain communities of West Sudanic speakers.  This absorption (i.e. {\textquotedbl}contamination{\textquotedbl} or {\textquotedbl}mixed language{\textquotedbl}) theory supposedly gave a sufficient account for the Bantuisms found in these languages.  Guthrie specifically claimed that languages such as the \ili{Ekoid} languages had only false reflexes of the \ili{Proto-Bantu} forms of the noun class prefixes and concord elements (cf. \citealt{Guthrie1962}.20 footnote 3). These languages were like \ili{Bantu} but not \ili{Bantu}, so he called them “Bantoid”. 

However, by \citeyear{Guthrie1971} Guthrie had slightly modified his position concerning the Bantoid subgroups such as \ili{Ekoid}. His modification, however, was put in the most tentative, non-committal terms possible:

\begin{quote}
It may therefore be tentatively inferred that the \ili{Ekoid} languages may to some extent share an origin with some of the Zone A languages [namely, \ili{Bobe} and \ili{Yambassa}], but that they seem to have undergone considerable perturbations. (\citealt{Guthrie1967}/1971.v.2.15 – brackets are mine)
\end{quote}

This statement indicates that Guthrie was never able to shake himself free from his \ili{Bantu}-centric point of view and see that the likely relationship between other \ili{Bantoid} subgroups and Narrow \ili{Bantu} involved a shared origin. In fact, he does not clarify for us how the genetic relationship could ever be {\textquotedbl}to some extent{\textquotedbl}.  In what way can one have a partial genetic relationship between two languages?  This possibility would imply that the Bantoid subgroups had multiple genetic origins, an implausible state of affairs until demonstrated. 

A different position was taken by \citet{Johnston191922} and \citet{Westermann1927} and   \citet{WestermannBryan1952}, who viewed the shared “Bantuisms” as deriving from a common origin. To make his point, Johnston referred to them as “\ili{Semi-Bantu}” languages. So when \citet{Greenberg1963} classified \ili{Bantu} languages with a multitude of other subgroups within the \ili{Benue-Congo} family, he was motivated by genetic considerations and, as noted by \citet{Winston1966}, this limitation to genetic considerations was Greenberg’s major contribution to the debate in African language classification.  Guthrie’s classification by contrast was as dependent on typological considerations as on genetic ones (\citealt{Williamson1971}.249). 

\subsection{Responses to Greenberg’s proposal}


A common response to Greenberg’s proposal that the \ili{Bantu} languages actually formed a subgroup within a subgroup of the EBC family was, for a number of researchers, to seek to validate this proposal. This involved research particularly in the 1960s to the 1980s. 

Studies by \citet{Crabb1965,Voorhoeve1971,Hyman1972,Hyman1980borderland,Hyman1980nasalclasses}, and    \citet{HymanVoorhoeve1980}, reviewed by \citet{Watters1982}, all made claims about specific language groups and their relation to \ili{Bantu}.  Voorhoeve and Hyman argued for a genetic relationship between the \ili{Mbam-Nkam} languages of Cameroon and \ili{Bantu} based on sound correspondences, cognate roots, and noun class correspondences.  Crabb argued for the same relationship between the \ili{Ekoid} languages and \ili{Bantu} on the basis of 1) a high degree of common vocabulary with the better known \ili{Bantu} languages, and 2) certain suppletive forms which appear to bear a relationship to \ili{Bantu} roots and noun class prefixes which would be resistant to borrowing.  Others pursued lexicostatistical studies that included at least some \ili{Bantu} languages along with languages from the region to the northwest of \ili{Bantu}: see \citet{Henrici1973,Heine1973}, and  \citet{CoupezEtAl1975}. Their results supported the likelihood of a genetic relationship between \ili{Bantu} and its northwest neighbors. 

These studies were instrumental in further affirming Greenberg’s proposal. In  addition, many other studies and dissertations have been published that demonstrate a variety of proposed genetic relationships between a given \ili{Bantoid} language or subgroup outside of  \ili{Bantu} and the \ili{Bantu} subgroup itself, whether represented by an individual \ili{Bantu} language or the \ili{Common Bantu} of Guthrie or the \ili{Proto-Bantu} of \citet{Meeussen1967}. Such studies continue to have their place of importance in the continuing discovery of relationships among the Bantoid subgroups and \ili{Bantu}, but also the other EBC subgroups of \ili{Kainji}, \ili{Plateau}, \ili{Jukunoid}, and \ili{Cross River} and their relationships with \ili{Bantu} and Bantoid.


\subsection{Challenges in building an integrated view of Bantoid} 


The significant amount of research on \ili{Bantu} languages over the past century has been an extraordinary benefit in researching the lesser known \ili{Bantoid} languages. The proposed reconstructions by \citet{Guthrie1967,Guthrie1971,Meeussen1967}, and \citet{BastinEtAl2002} of \ili{Proto-Bantu} or \ili{Common Bantu} forms have provided multiple suggestions as to the meaning and the role of forms in other \ili{Bantoid} languages, both morphological and lexical. 

  
In the midst of these benefits there is also a challenge. It is tempting, whether conscious or subconscious, to take a \ili{Bantu}-centric view and begin conceiving \ili{Proto-Bantoid} as being equivalent to \ili{Proto-Bantu}, and even perhaps extending the temptation and conceiving \ili{Proto-EBC} as being equivalent to \ili{Proto-Bantu}. \ili{Bantu} has received the attention of a multitude of linguists for more than a century and \ili{Proto-Bantu} has been reconstructed in ways to which no other \ili{Bantoid} subgroup can compare. Also, by comparison, \ili{Bantu} languages are rich in verbal and nominal morphology in ways that are frequently minimal or non-existent in other \ili{Bantoid} subgroups. They are also more numerous by far than the number of languages in other Bantoid subgroups. In fact, my impression is that the number of \ili{Bantu} languages (more than 500) and the enormous amount of research done on \ili{Bantu} languages over the past century set them apart from all language families of Africa. 

It can be easy to treat \ili{Bantu} statically and forget that \ili{Proto-Bantu} and its own subgroups and individual languages have their own history of retentions, innovations and borrowings. So, in reconstructing \ili{Bantoid} and EBC, caution has to be taken. Just because \ili{Bantu} has a given feature does not mean it was also present in \ili{Proto-Bantoid} or in \ili{Proto-EBC}. It may have originated in \ili{Proto-Bantu}. Within EBC and within \ili{Bantoid} in particular, there likely is a layering of relationships that we still do not understand well. But let me offer a few examples of how this layering may be present and effect our claims about where a given feature was innovated. Care is needed not to attribute everything found in \ili{Proto-Bantu} to \ili{Proto-Bantoid}, and in \ili{Proto-Bantoid} to \ili{Proto-EBC}. The same holds in studying the subgroups of \ili{Bantoid} and not inferring from one subgroup that a given phenomenon must be \ili{Proto-Bantoid}.  Here are some examples.

\subsubsection{Tense in Bantu}

One example involves tense in \ili{Bantu}. \ili{Bantu} languages are rich in tense categories. Most \ili{Bantu} languages have multiple past categories and multiple future categories. Among the other \ili{Bantoid} subgroups in which tense is found, the more widely publicized are the \ili{Grassfields} languages. At the same time, other \ili{Bantoid} languages do not mark tense as a morphological verbal category. They are aspect-prominent like most languages in West Africa. This includes \ili{Bantoid} subgroups such as \ili{Ekoid}, \ili{Tivoid}, and \ili{Nyang} (\ili{Mamfe}).  

Nurse recognized that tense within \ili{Bantoid} was not limited to \ili{Bantu} but overlapped with some of the other \ili{Bantoid} subgroups when he wrote: 
\newpage 

\begin{quote}
[…]it would seem most likely in the present state of knowledge that tense was innovated within the community ancestral to today’s \ili{Bantu} languages (2.10.2(iv, vii)) \citep[282-283]{Nurse2008}. 
\end{quote}

It was unclear whether it had been innovated within \ili{Bantoid} or perhaps “at some level of \ili{Bantoid-Cross} tree” \citep[282]{Nurse2008}. A future volume in the \ili{Niger-Congo} Comparative Series is in preparation to address this very topic.

However, the point I want to make here is that if \ili{Bantu} as well as two or more adjacent subgroups in \ili{Bantoid} also mark tense, it is easy to assume that tense was a \ili{Proto-Bantoid} phenomenon. The explanation for those subgroups without tense is simply to claim that they lost their tense marking. However, one would expect to find residual forms pointing to antiquated tense markers, but these are not present. 

For nearly forty years I assumed that historically the \ili{Ejagham} language within \ili{Ekoid} would have had marked tense categories even though there were no pres\-ent-day marked tense categories \citep[364-365]{Watters1981}. At the same time, I could not find any residual or fossilized forms to support this assumption, but the fact that \ili{Bantu} marked tense and was closely related to \ili{Bantoid} languages was sufficient for me to make the assumption. It was Nurse’s excellent work on \textit{Tense and Aspect in Bantu} (\citeyear{Nurse2008}) that alerted me to the \ili{Bantu} verbal realities and their contrast with the wider \ili{Niger-Congo} verbal realities. It led me to reverse my assumption in 2012. This was spelled out in 2012 in what will appear as \citet{Watters2018}.  

The fact is that some of the \ili{Bantu} phenomena may be restricted to \ili{Bantu}, some of them may be shared with some other \ili{Bantoid} subgroups, and some may be inherited from \ili{Proto-Bantoid}, \ili{Proto-Bantoid-Cross}, or \ili{Proto-EBC}. Because of the extraordinary amount of research that has been published on \ili{Bantu} languages and because of their morphologically complex forms, it can be tempting to assume that \ili{Bantu} has conserved what was once \ili{Proto-Bantoid} and the rest of \ili{Bantoid} has moved from an earlier synthetic mode to a more analytic one. 

However, as is being noted and reiterated here, if what is found in \ili{Proto-Bantu} traces back to \ili{Proto-Bantoid}, does that mean that it also traces back to \ili{Proto-Bantoid-Cross} and \ili{Proto-EBC} and \ili{Proto-Niger-Congo}? As we seek to better understand \ili{Bantoid}, I would encourage caution in making strong claims for \ili{Proto-Bantoid}, for example, until sufficient coverage on a given phenomenon has been achieved involving all or most all of the \ili{Bantoid} subgroups. I would suggest we look for layering among the \ili{Bantoid} subgroups as expansions proceeded from west to east and innovations were made along the way within sub-regions of \ili{Bantoid} and not necessarily shared with those they left behind. 

\newpage 
\citet[406-407]{Watters1989} notes the contrastive hypotheses about \ili{Bantoid}. \citet{Williamson1971} and \citet{Greenberg1974} accept a clear two-way split within \ili{Bantoid}. However, \citet{Meeussen1974} countered that it was too early to determine the internal structure of \ili{Bantoid} and preferred to remain with a multibranch hypothesis since too little was still known as to the internal \ili{Bantoid} relationships. Meeussen’s suggestion resembles \citet{Blench2015} noted above in \sectref{sec:watters:4}. Up to the present, most of our judgments about the internal structure of Bantoid are based on lexicostatistics, and that will remain the case until more research on morphological and lexical reconstructions is achieved.


\subsubsection{Synthetic and analytic structures: the verb}

Turning to another example, \citet[183--187]{Güldemann2003} raises the issue of {Ban\-tu} word forms, morphology and their grammaticalization history.” Considering the verbal word in \ili{Bantu}, the most complex word form in \ili{Bantu}, in \ili{Bantoid}, and in even EBC, the question that could be asked is: Did \ili{Proto-Bantoid}, or \ili{Proto-EBC} for that matter, originally have a fully synthesized verb much like that in \ili{Bantu}, so that what most \ili{Bantoid} groups present today is the result of a process they went through of isolating many or all of the morphemes, thus becoming analytic in structure? Or were the earlier forms more like those in most \ili{Bantoid} groups, some verbal affixes but mostly analytic with isolated morphemes or clitics that were then synthesized in early \ili{Bantu} or pre-\ili{Proto-Bantu}? G\"{u}ldemann argues that much of the \ili{Bantu} verbal morphology can be shown to have likely derived historically from a more analytic structure with isolated morphemes. 

An important interaction about these matters at the levels of \ili{Bantu}, \ili{Bantoid}, EBC, and \ili{Niger-Congo} is that between \citet{Güldemann2011} and \citet{Hyman2011}. G\"{u}ldemann proposes that \ili{Bantu} synthetic forms derive from more analytic forms found elsewhere in EBC. Hyman’s response is instructive in his comments about possible historical recycling of morphosyntax, and the likely areal diffusion of more recent innovations along G\"{u}ldemann’s proposed “\ili{Macro-Sudan} belt”. It is a sobering interaction that underscores the importance of \textit{local} comparative research. G\"{u}ldemann’s hypothesis can provide a framework for further research, but it can also generate a healthy skepticism about macro-claims that do not have the benefit of systematic reconstructions of the given phenomenon at lower levels.

At the same time, G\"{u}ldemann’s proposal exemplifies the need to give the imagination freedom to look beyond \ili{Bantu} and the related \ili{Bantoid} groups to EBC and all its branches and even \ili{Benue-Congo} at an even  higher level, and ask questions such as: Where do the morphologically complex verb forms of \ili{Bantu} best fit, as a \ili{Bantu} innovation or as \ili{Bantu} retention, but if a retention, a retention of what historical level?   


\subsubsection{Verbal extensions in Bantoid}

Another example involves verbal extensions. Hyman (\chapref{sec:5}) provides a valuable, detailed overview of verbal extensions in \ili{Grassfields} and \ili{Bantoid}. There are challenges in relating \ili{Proto-Bantu} Zone A verbal extensions to verbal extensions in the other \ili{Bantoid} subgroups. In \ili{Bantu}, extensions such as causative, applicative, passive etc. mark the valency of the given verb. By contrast, in \ili{Bantoid} languages they may mark either valence values or aspectual values. Hyman provides an excellent panorama comparing particular verbal extensions found in \ili{Grassfields} with those in \ili{Bantu} Zone A. He notes the semantic innovation of the \ili{Grassfields} in reassigning extensions more aspectual values than the valence ones while next door valence values are commonly found in the \ili{Bantu} Zone A languages. This overview serves as an excellent foundation for future comparative studies of verbal extensions in all \ili{Bantoid} subgroups as well as languages of \ili{Cross River}, \ili{Jukunoid}, \ili{Plateau}, and \ili{Kainji}, in order to better understand how they may have been present at the level of \ili{Proto-EBC} and each of its major subgroups. It also points to the difficulty of defining a clear boundary between \ili{Bantu} and its \ili{Bantoid} neighbors.

The questions I have raised above about the layering of evidence for innovation and retention relate to Hyman’s article as follows: Just as it can be tempting to project \ili{Proto-Bantu} onto \ili{Proto-Bantoid}, it might be tempting to project \ili{Proto-Bantu} plus \ili{Proto-Grassfields} and other eastern \ili{Bantoid} subgroups (e.g. \ili{Beboid}, \ili{Mambiloid}, \ili{Tikar}) onto \ili{Proto-Bantoid}. The region within \ili{Grassfields} where the largest number of contrastive verbal extensions are found outside of \ili{Bantu} could be a region of innovation rather than retention, and those Bantoid groups to the west of Grassfields may instead better represent \ili{Proto-Bantoid} with their reduced number of extensions and their –CV shape. However, Hyman notes that the direction of change for extensions is to begin as valency marking morphemes. They then change to primarily marking aspect with some residual valence functions that become lexicalized. Finally, they change to having only aspectual values. This suggests that these verbal extensions are \ili{Proto-Bantoid} extensions and likely much older, having undergone this transition from valency to aspect marking. So the extensions are not a case of inappropriate projections of \ili{Proto-Bantu} categories onto \ili{Proto-Bantoid}. But this line of questioning may need to be used with each \ili{Bantu} extension individually. 

\newpage
Turning to another topic raised by verbal extensions, Hyman’s study provides a possible answer to the boundary issue between \ili{Bantu} and the other \ili{Bantoid} groups. His chart of extensions for \ili{Bantu} Zone A languages and selected \ili{Bantoid} languages gives evidence to support the claim that the presence and absence of the passive is a likely boundary marker (see \citealt{Watters1989}: 416). The Sanaga River valley (or \ili{Bantu} Zone A) serves as a boundary between those languages with a passive extension (i.e. Narrow \ili{Bantu} languages) and those without a passive extension (i.e. the remainder of the Bantoid languages). These other Bantoid languages commonly use the third person plural verbal prefix but with non-specific reference to mark the passive notion. Another possible boundary may be the applicative, being present in Narrow \ili{Bantu} but absent in the remainder of Bantoid. Hyman (p.c.) also notes the possible role of the applicative in this matter. For the passive and applicative in Bantoid other than \ili{Bantu}, see \citet[360]{Watters1981} for \ili{Ejagham} in the \ili{Ekoid} group and \citet[252]{Watters2003} for the multiple languages in the \ili{Grassfields} group. 

\subsubsection{Nasal nominal prefixes in Bantoid \& EBC}

To continue the topic of how \ili{Bantu} can be an influence in analyzing other Bantoid subgroups and Bantoid as a whole, Hyman (\chapref{sec:6}) presents the matters of \ili{Bantu} nasal nominal prefixes. He provides an important overview of the questions revolving around the presence and absence of nasal prefixes in \ili{Bantu} noun classes 1, 3, 4, 6a, 9, 10, and their cognates. Class 6a generally occurs throughout \ili{Niger-Congo} displaying a form cognate with \textit{*ma-} as the prefix, so this class is not the major focus. \citet{Hyman1980borderland} covers similar details but using data that was available more than thirty years ago. More is known today, as demonstrated in Hyman (\chapref{sec:6} of this volume) and \citet{Blench2015}.

  The questions Hyman raises are numerous and complex. He provides the possible answers and their competing assumptions to these questions. In terms of research on Bantoid and, more widely, all EBC, it appears likely that \ili{Proto-EBC} used oral vowels for these prefixes while Proto-\ili{Bantu} used nasal consonants in a CV- structure: \textit{*mʊ}\textit{{}-}, \textit{*mɪ}\textit{{}-}, \textit{*ma-} (classes 1, 3, 4); or a homorganic nasal \textit{*N-} (classes 9, 10). Whatever may have existed in \ili{Proto-Niger-Congo} or whatever may have happened across the \ili{Niger-Congo} macrofamily in terms of having a full set of nasal nominal prefixes for cognates to \ili{Proto-Bantu} noun classes 1, 3, 4, 6a, 9, and 10, it might advance our understanding if we could unravel the layers within Bantoid first, reconstructing the noun classes for each Bantoid subgroup, and then for \ili{Cross River} and \ili{Jukunoid}, and possibly then from possible insights from reconstructions of various subunits within \ili{Kainji} and \ili{Plateau}. A place to start would be to reconstruct the nominal prefixes and concord affixes for each \ili{Bantoid} subgroup. Even at this level it is not always straightforward.   \citet{GoodLovegren2017} demonstrate that reconstructing nasal classes can be complicated even within what is clearly a dialect cluster.

  Indeed, within \ili{Bantoid}, subgroups vary relative to the presence of nasal and oral prefixes. For example, \ili{Grassfields} is divided in this matter \citep[55]{Stallcup1980geo}. \ili{Western Grassfields} has oral prefixes in classes 1 or 3, and nasal prefixes on only some nouns in classes 9 and 10. This contrasts with \ili{Eastern Grassfields} which has nasal prefixes in classes 1 and 3, and homorganic nasal prefixes on all nouns in classes 9 and 10. Leaving the \ili{Grassfields} and going farther west, Hyman points to \ili{Tiv} that does not have nasal prefixes in classes 1, 3, 4, 9, or 10 (\citealt{VoorhoeveDeWolf1969}: 52). Contrastively, also to the west, \ili{Proto-Ekoid} likely had nasal prefixes in classes 1, 3, 4, 9, and 10 (\citealt{Watters1981,Watters1980ejagham,Watters2016}). This uneven distribution of nasal prefixes in \ili{Bantoid} subgroups does not clearly point to \ili{Proto-Bantoid} having a full set of nasal prefixes. The layering of their presence suggests the possibility that the innovation started with some subgroups but not in others, and in the case of \ili{Grassfields}, with its two-way division, it may involve different waves of migrations into the \ili{Grassfields}. A first wave that became \ili{Eastern Grassfields} possessed (or innovated?) the set of nasal prefixes while a later wave (or waves) that became \ili{Momo} and \ili{Ring} languages did not arrive with nasal prefixes. Only over the centuries of contact with \ili{Eastern Grassfields} language they have begun marking some nouns in classes 9 and 10 with homorganic nasal prefixes.


One hypothesis put forward some forty years ago was that \ili{Bantoid} could be divided into two groups, the \ili{Bane} group and the \ili{Bantu} group. In testing this hypothesis, Voorhoeve (\citeyear*{Voorhoeve1980bane}, see also \citealt{Watters1982}: 89) found that grammatical criteria and lexical criteria gave contradictory conclusions. He also discussed nasal prefixes in noun classes 1, 3, and 6, raising significant questions for any kind of definitive criteria for distinguishing \ili{Bantu} and the other subgroups of \ili{Bantoid}. Areal spreading of various features seems to have been involved.


\subsubsection{Third person pronouns in Grassfields}

Finally, Hyman (\chapref{sec:7}) provides a fascinating presentation of third person pronouns in \ili{Eastern Grassfields}, \ili{Momo}, and \ili{Ring} (the two together form \ili{Western Grassfields}), and their relation to \ili{Proto-Bantu} forms. It is clear that \ili{Momo} and \ili{Ring} have innovated new forms for third person pronouns by using demonstratives and the noun ‘body’ as the sources for the innovations. In contrast, \ili{Eastern Grassfields} maintains the original pronominal forms and these are closely related to \ili{Proto-Bantu} forms. 

This is the kind of comparative study needed for each subgroup or closely related subgroups on various topics. The goals in each case would be to determine the earliest forms and identify any innovations and what the sources of those innovations might be.  Such studies would provide an excellent database for comparing \ili{Bantoid} subgroups and assist in reconstructing the history of \ili{Bantoid}.

Our understanding of the relationships between the groups of languages beyond the \ili{Bantu} boundary is still at a rudimentary level. It is hoped that these six chapters will alert others to the challenges and motivate them to join the process of clarifying their history. 

\section*{Acknowledgements}

I am grateful to Roger Blench, Jeff Good, Larry Hyman, and Valentin Vydrin for fruitful comments on earlier drafts of this chapter. 
 
{\sloppy
\printbibliography[heading=subbibliography,notkeyword=this]
}
\end{document}