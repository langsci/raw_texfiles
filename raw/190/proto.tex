\documentclass[output=paper]{langsci/langscibook} 
\ChapterDOI{10.5281/zenodo.1314327}
\author{Larry M. Hyman\affiliation{University of California, Berkeley}}
\title{Common Bantoid verb extensions} 
\abstract{In this paper I survey verb extensions within different Bantoid languages and subgroups, comparing them to Cameroonian Bantu zone A. Extending my survey of Niger-Congo extensions \citep{Hyman2007}, I show that there is a band of contiguous languages in the Grassfields area where a number of contrastive verb extensions have relative productivity (cf. the studies in \citealt{IdiataMba2003}). Interestingly, the languages in question belong to several subgroups: Limbum (NE Eastern Grassfields Bantu), Noni (Beboid), Kom and Babanki (Ring Western Grassfields Bantu), Bafut and Mankon (Ngemba Eastern Grassfields Bantu). Other languages in these same subgroups are not in this geographical band and have very few extensions. The above-mentioned languages allow a possible reconstruction of *CV extensions with \textit{*s, *t, *n, *l, *k,} and \textit{*m}. A major property of Bantoid extensions is the relative frequency of aspectual-type extensions, especially marking different types of pluractionality (iterative, frequentative, distributive, repetitive), diminutive (attenuation of action), and intensive (augmentation of action) semantics. In many languages the same suffix form covers two or more of these functions. The hypothesis is that the original system was more like Proto-Bantu, with extensions being more valence-related, but over time these very same extensions became reinterpreted as aspectual. However, the great variety of extensions in and outside of Bantoid suggests that there may have been more extensions at a pre-Proto-Bantu stage.}
\maketitle
\begin{document}
\label{sec:5}
\section{Introduction}
While the presence and identity of \ili{Proto-Bantu} verb extensions has long been established, with relatively little controversy \citep{Meeussen1967,Schadeberg2003}, we do not have a clear sense of the verb extension system(s) that existed at pre-\ili{Proto-Bantu} stages.\footnote{This paper was first presented at the Workshop on \ili{Bantu} and its Closest Relatives, Berlin Bantu Conference (B4ntu), April 6-9, 2011.} My goal in this paper is to consider some of the issues arising in \ili{NW Bantu}, \ili{Grassfields Bantu}, and some of their closest \ili{Bantoid} relatives. The questions I shall be concerned with are: 
\begin{itemize}
\item[(i)]  What are \ili{Bantoid} verb extensions like? 
\item[(ii)] What can be reconstructed at a \ili{Bantoid} Pre-\ili{Proto-Bantu} level? 
\item[(iii)] What, if anything, do they tell us about \ili{Proto-Bantu}? 
\end{itemize}

\noindent
My goal in this chapter is to evaluate our current knowledge to determine what the \ili{Common Bantoid} verb extensions are that might be considered for such reconstruction.\footnote{For a recent overview of the languages considered to be \ili{Bantoid}, see \citet{Blench2015}.}

\section{{Grassfields Bantu}}

It is often remarked that the comparative study of \ili{Bantu} (and \ili{Niger-Congo}) verb extensions has been neglected in favor of noun classes. The same has been true in Bantoid studies. As a case in point, let us consider \ili{Grassfields Bantu}. In \REF{ex:proto:1} I present two subclassifications of what we might identify as “\ili{Narrow Grassfields Bantu}”, i.e. ignoring \ili{Ndemli} (cf. \citealt{Stallcup1980geo,WattersLeroy1989,Piron1995,Watters2003}):

\eabox{\label{ex:proto:1}
  \small
  \begin{xlist}
  \parbox[t]{.35\textwidth}{
    \ex
    \vspace{.1mm}\hspace*{-5mm}
    \begin{forest}
    [\ili{Grassfields Bantu} 
      [\ili{WGB} 
	[\ili{Ring}] 
	[\ili{Momo}] 
	[\parbox{2.5cm}{\centering \ili{EGB}\\(\ili{Mbam-Nkam})}] 
	]
    ]
    \end{forest}
    \hspace*{-1cm}
    }
    \parbox[t]{.25\textwidth}{
    \ex
    \vspace{.1mm}
    \begin{forest}
    [\ili{Grassfields Bantu} [~~~\ili{Ring}] [\ili{Momo}] [\ili{EGB}] [\parbox{2.3cm}{\ili{Western~Momo}\\\citep{Blench2010}} ] ] 
    \end{forest}
    }
  \end{xlist}
}




As seen, the older subclassification in (\ref{ex:proto:1}a) recognizes a binary split between \ili{Western vs. Eastern Grassfields Bantu} (\ili{WGB}, \ili{EGB}), while (\ref{ex:proto:1}b) presents all of the subbranches as coordinate. Identification of some of the languages are as follows (\citealt{HymanVoorhoeve1980,Watters2003}):

\ea%2
    \label{ex:proto:2}
    \ea \ili{Ring}:  \ili{Aghem}, \ili{Isu}, \ili{Weh}, \ili{Bum}, \ili{Bafmeng}, \ili{Kom}, \ili{Oku}, \ili{Babanki}, \ili{Lamnso’}, \ili{Babungo}, \ili{Babessi}
    \ex  \ili{Momo}:  \ili{Moghamo}, \ili{Metta}, \ili{Menemo}, \ili{Ngembu}, \ili{Ngamambo}, \ili{Ngie}, \ili{Oshie}, \ili{Ngwo}, \ili{Mundani}, \ili{Njen}
    \ex  \ili{EGB}:  \ili{Ngemba} (e.g. \ili{Mankon}, \ili{Bafut}), \ili{Bamileke} (e.g. \ili{Yemba}, \ili{Ghomala}, \ili{Medumba}, \ili{Fe’fe’}), \ili{Nun} (e.g. \ili{Bamun}, \ili{Bali}), \ili{North} (e.g. \ili{Limbum}, \ili{Adere})
    \z
\z

Early on, the \ili{Grassfields Bantu} Working Group discovered significant differences which seemed to motivate the division between \ili{EGB} and \ili{WGB}. As seen in \tabref{extab:proto:3}a-g, most of the criteria for such a split concerned noun class marking \citep[55]{Stallcup1980geo}.

\begin{table}
\begin{tabularx}{\textwidth}{lllXlX}
\lsptoprule
% \ea%3
  &  &  &  {\ilit{Eastern Grassfields Bantu}} &  &  {\ilit{Western Grassfields Bantu}}\\
\midrule
% \multirow{5}{*}{\resizebox{1mm}{47mm}{\Huge\{}}
& \twikzmark{a}{a.}  &  & nasal prefix in class 1 and class 3 nouns &  & absence of the nasal\\
& b. &  & no distinction between class 6 and class 6a &  & distinction between class 6 a- and class 6a \textit{mə-}\\
& c. &  & nasal prefix on all 9/10 nouns &  & nasal prefix only on some 9/10 nouns\\
& d. &  & absence of classes 4 and 13; class 19 rare &  & presence of classes 4 and 13; class 19 frequent\\
& e. &  & noun prefixes all carry a /L/ tone &  & most noun prefixes carry a /H/ tone\\
& f. &  & no noun suffixes &  & many noun suffixes, e.g. plural \textit{-tí}, \textit{-sí}\\
& \twikzmark{g}{g.} &  & class 2 or 6a generalizes to mark plural &  & class 10 or 13 generalizes to mark plural\\
& h. &  & innovation of \textit{síŋə́} ‘bird’, -\textit{kìə́} ‘water’ &  & maintenance of \textit{*-nɔ̀ní} ‘bird’, \textit{*-díbá} ‘water’\\
& i. &  & maintenance of \textit{*-úmà} ‘thing’ &  & \textit{*-úmà} is lost, other roots come in\\
\multicolumn{3}{l}{\raggedleft {Plus:}} & maintenance of inherited 3\textsuperscript{rd} person pronouns &  & introduction of new 3\textsuperscript{rd} person pronouns\\
\multicolumn{3}{l}{} &  &  & \citepv{Hymangrassfieldstv}\\
\lspbottomrule
\end{tabularx}

\begin{tikzpicture}[overlay,remember picture]
\draw [decorate,decoration={brace,amplitude=10pt,mirror},xshift=-4mm] (a.north west) -- (g.south west) node {};
\end{tikzpicture}
 
% \connect{a}{g}
% \todo[inline]{needs brace from a-g}
\caption{Criteria for distinguishing Eastern from Western Grassfields}
\label{extab:proto:3}
\end{table}

A major question we continue to face is the extent to which \ili{Proto-Bantu} (PB) is representative of pre-PB, e.g. “\ili{Proto-Bantoid}”, which includes \ili{Proto-Grassfields Bantu}. It is commonly assumed that PB is conservative, preserving many features of \ili{Proto-Niger-Congo} (PNC). Concerning the criteria in (\ref{extab:proto:3}a-c), it was once generally accepted that \ili{Narrow Bantu} innovated nasals in the noun prefixes for classes 1, 3, 4, 6, 9 and 10. However \citet{Miehe1991} argued that the nasals are archaic, which \citet[43-44]{Williamson1993} accepts, but which is still somewhat unsettled.\footnote{Cf. the recent workshop “Nasal Noun Class Prefixes in \ili{Bantu}: Innovated or Inherited?” which I co-organized with Gudrun Miehe at the Paris Bantu Conference (Bantu5) on June 12, 2013. See \citetv{Hymannasaltv}.} All this to say that attention has largely been on noun classes, which have often served not only as the major criterion for inclusion within \ili{Niger-Congo}, but also subgrouping.

Concerning verb extensions, the PB system has been reconstructed as in \REF{ex:proto:4}, where it is useful to distinguish three sets (\citealt{Meeussen1967,Schadeberg2003}):

\ea%4
    \label{ex:proto:4}
    \ea productive extensions
      \ea  \textit{*-i-}  ‘causative’  iv.  \textit{*-ɪk-}  ‘neuter/stative’
      \ex  \textit{*-ɪc-i- } [\textit{-ɪs-}]  ‘causative’  v.  \textit{*-an-}  ‘reciprocal/associative’
      \ex  \textit{*-ɪd-}  [\textit{-ɪl-}]  ‘applicative’  vi.  *(\textit{-ɪC-})-ʊ-  ‘passive’ (\textit{-ɪbw-}, \textit{-ɪgw-})
    \z
    \ex  unproductive extensions often restricted to post-radical position or specific combinations
      \ea  \textit{*-ɪk-}  ‘impositive’  iv.  \textit{*-ad-} (\textit{-al-})  ‘extensive’
      \ex  \textit{*-am-}  ‘positional’  v.  \textit{*-at-}  ‘tentive’ (contactive)
      \ex  \textit{*-a(n)g-}  ‘repetitive’  vi.  \textit{*-ʊk-}/\textit{*-ʊd-} (\textit{-ʊl-})  ‘reversive/separative’ (intr./trans.)
    \z
    \ex  frozen, mostly unidentifiable -VC- expansions
      \ea  \textit{*-u-, *-im-, *-un-, *-ing-  iii.  *-ɪm-, *-ɔm-, *-ɔng-}  (but only after CV-)
      \ex  \textit{*-ang-, *-ab-, *-ag-, *-ak-  iv.  *-ʊt-}
    \z
  \z
\z

Attempts to reconstruct extensions in PNC and certain other branches of NC have been few, but typically produce forms resembling PB (\tabref{extab:proto:5}).

\begin{table}[p]
\begin{tabularx}{\textwidth}{llQQQ}
\lsptoprule
% \ea%5
 & & \ilit{Proto-Niger-Congo} \citep{Voeltz1977} & \ilit{Proto-Bantu} \citep{Schadeberg2003} & \ilit{Proto-Atlantic} \citep{Doneux1975}\\
\midrule
a. & applicative & \textit{*-de} & \textit{*-ɪd-} & \textit{*-ed}\\
b. & causative & \textit{*-ci,} \textit{*-ti} & \textit{*-ic-i-} & (\textit{*-an})\\
c. & contactive & \textit{*-ta} & \textit{*-at-} & \\
d. & passive & \textit{*-o} & \textit{*-ɪb-ʊ-} & \textit{*-V} [+back]\\
e. & reciprocal & \textit{*-na} & \textit{*-an-} & \textit{*-ad}\\
f. & reversive (tr.) & \textit{*-to} & \textit{*-ʊd-} & \textit{*-t}\\
g. & reversive (intr.) & \textit{*-ko} & \textit{*-ʊk-} & \\
h. & stative/neuter & \textit{*-ke} & \textit{*-ɪk-} & \\
i. & stative/positional & \textit{*-ma} & \textit{*-am-} & \\
\lspbottomrule
\end{tabularx}
\caption{Proposed reconstructions of verb extensions }
\label{extab:proto:5}
\end{table}

It is however possible that PB may have lost (merged) earlier distinctions. When one compares \ili{Bantu} with some of the \ili{Atlantic} languages, for instance, one observes that the latter often distinguish more than one applicative extension where \ili{Bantu} typically has only \textit{*-ɪd-} \citep[157]{Hyman2007}.\footnote{Nuba mountain languages also typically distinguish benefactive vs. locative applicatives, in addition to other extensions not distinguished in \ili{Bantu}. An overview of \ili{Nuba} mountain verb extensions \citep{Hyman2014} is available upon request.}

\begin{table}[p]
\begin{tabularx}{\textwidth}{llQQQ}
\lsptoprule
% \ea%6
& & 
{ \ilit{Chichewa} (\citealt{HymanMchombo1992})} &
{ \ilit{Temne} (\citealt{Wilson1961,Kanu2004})} &
{ \ilit{Fula} \citep{Arnott1970}}\\
\midrule
	       & causative & \textit{-is-} & \textit{-s} & \textit{-n-}\\
\hhline{~~---} & allative & \textit{-ir-} & \textit{-r} & \textit{-r-} ?\\
	       & locative & \textit{-ir-} & \textit{-r} & \textit{-r-}\\
\hhline{~~~~-} & recipient & \textit{-ir-} & \textit{-r} & \textit{-an-}\\
\hhline{~~~-~} & benefactive & \textit{-ir-} & \textit{-a̘} & \textit{-an-}\\
	       & circumstance & \textit{-ir-} & \textit{-a̘} & \textit{-an-}\\
\hhline{~~~~-} & manner & \textit{-ir-} & \textit{-a̘} & \textit{-r-}\\
\hhline{~~~-~} & instrument & \textit{-ir-} & \textit{-a̘-nɛ} & \textit{-r-}\\
\hhline{~~---}
\lspbottomrule
\end{tabularx}
\caption{Comparing Chichewa (Bantu) with two Atlantic languages}
\label{extab:proto:6}
\end{table}

However, \ili{Grassfields Bantu} and even \ili{Bantu zone A} may diverge from PB, as seen in \tabref{extab:proto:7} \citep[160]{Hyman2007}:\footnote{The capitals I and U indicate harmonizing vowels in \ili{Gunu}.}

\begin{table}
\small
\fittable{
\begin{tabular}{l@{~}l l@{~}l l@{~}l l@{~}p{2cm}@{}}
\lsptoprule
%  (7) 
    \multicolumn{2}{p{3cm}}{\raggedright\ilit{Mokpe}   A22   \citep{Connell1997,Henson2001}} & 
    \multicolumn{2}{p{3cm}}{\raggedright\ilit{Gunu}   A62\newline     \citep{Orwig1989}} & 
    \multicolumn{2}{p{3cm}}{\raggedright\ilit{Mankon} (\ilit{EGB}) \citep{Leroy1982}} &
    \multicolumn{2}{p{3.3cm}}{\raggedright\ilit{Bafut} (\ilit{EGB})\newline  \citep{TamanjiMba2003}}\\
   \midrule
\textit{-an-ɛ} & reciprocal & \textit{-anIn} & réciproque & \textit{-nə} & réciproque & \textit{-nə} & reciprocal\\
\tablevspace 
\textit{-an-a} & instrumental & \textit{-an} & pluriel, iteratif & \textit{-nə} & stative, réfl & \textit{-nə} & stative/intr\\
\tablevspace 
\textit{-o-a} & reversive & \textit{-Ug} & réversif intr. & \textit{-kə} & intransitif & \textit{-kə} & stative/intr\\
 &  & \textit{-Ig} & intensif & \textit{-kə} & itératif & \textit{-kə} & iterative\\
\tablevspace 
\textit{-is-ɛ} & causative & \textit{-i} & causatif & \textit{-sə} & causatif & \textit{-sə} & causative\\
 &  & \textit{-Id} & diminutif & \textit{-tə} & diminutif & \textit{-tə} & attenuative/ iterative\\
\tablevspace 
\textit{-e-a, -ɛl-ɛ} & applicative & \textit{-In} & applicatif &  &  & \textit{-lə} & random\\
\tablevspace 
\textit{-am-a} & positional & \textit{-Im} & statif &  &  &  & \\
\tablevspace 
\textit{-av-ɛ} & passive & =\textit{VlÚ} & passif &  &  &  & \\
\tablevspace 
\textit{á- ... -ɛ} & reflexive & \textit{bá-} & réfléchi &  &  &  & \\
\lspbottomrule
\end{tabular}
}
\caption{Comparing Bantu Zone A (Mokpe \& Gunu) with Grassfields (Mankon \& Bafut)}
\label{extab:proto:7}
\end{table}

Within \ili{NW Bantu} it is not uncommon for certain notions to be expressed by a sequence of verb suffixes, sometimes with a specific final vowel (FV). This is especially the case with the reciprocal and the instrumental, the latter not having a distinct form in most \ili{Bantu}.\footnote{As 
 seen in \tabref{extab:proto:6}, some \ili{Bantu} languages use the applicative suffix for instruments; others may use the causative extension, while still others require a preposition to express an instrument.}  
Note that the \textit{-an-} of the ‘instrumental’ might better be identified as ‘associative’ which, in its reciprocal use, is found in the sequence \textit{-ang-an-} sporadically throughout the \ili{Bantu} zone. See also \citet{BostoenNzangBie2010} for the development of such “double suffixes” in {A70}. Thus, in addition to the above, \ili{Kwasio (A81)} distinguishes \textit{-al-a} ‘recip.’ vs. -\textit{ɛl-ɛ} ‘instr.’ (\citealt{NgueUm2002}:  (< \textit{*-an-a, *-an-ɛ}), while \ili{Mpompon} (A86c) has instrumental \textit{-ɛ́l-ɛ̀} vs. reciprocal \textit{tí-...-là} (\citealt[38]{NgantchoLebika2003}). In addition, the \textit{-ɪd} and \textit{-tə} diminutive extensions in zone A and Bantoid do not have an obvious cognate in what I will refer to as a central or canonical \ili{Bantu} (CB). They are, for instance, not obviously related to the unproductive tentive or extensive extensions in (\ref{ex:proto:4}b).

\newpage 
In fact, once we move out into \ili{Bantu}’s closest relatives, we run into a number of problems:

\begin{itemize}
\item[(i)]  Many Bantoid languages have few extensions, often limited to one per verb root. In addition, although \ili{Bantu} languages typically allow more than one extension in sequence, many of the Bantoid languages allow only one extension per verb root.

\item[(ii)] The forms or functions of the extensions may not correspond to those in \ili{Narrow Bantu}, as I have already noted concerning the diminutive extension.

\item[(iii)] The forms may be polysemous, the semantics difficult to characterize, and the functions contradictory. I will give several examples of this below.

\item[(iv)] The roots of “formally” extended verbs often do not occur unextended. While this is sometimes the case even with productive extensions in CB, the problem is exacerbated in \ili{Bantoid}, where the extensions are less productive (and their function harder to characterize).

\item[(v)] Such “formal extensions” pose problems of segmentation. It is often hard, if not impossible to tell if a \textit{CVte} verb stem should be segmented as \textit{CV-te} or \textit{CVt-e}.

\item[(vi)] There is considerable, rather impressive variation vs. the relative stability of CB extensions.
\end{itemize}

These problems will become further evident from the data presented in the following section.

\section{Survey of Bantoid verb extensions with focus on Grassfields}

As a result of all of the above, \citet[1]{Blench2011} quite accurately appraises our current understanding: “In contrast to \ili{Bantu}, verbal extensions in \ili{Bantoid} languages remain very poorly known.” For the purpose of attempting a reconstruction, I therefore had to first conduct a reasonably comprehensive survey of Bantoid verb extensions based on available literature. I present the forms that were found in the following composite table—which should be considered a “first pass”, with some amalgamations (\tabref{extab:proto:8}).

\begin{table}
\fittable{
\begin{tabular}{l@{~}p{10mm}p{10mm}p{5mm}p{8mm}p{5mm}p{5mm}p{5mm}p{8mm}p{5mm}p{5mm}p{5mm}p{5mm}}
\lsptoprule
 & &  {\textsc{plur}} &  {\textsc{dim}} &  {\textsc{intens}} &  {\textsc{caus}} &  {\textsc{appl}} & {\textsc{rec}} & {\textsc{assoc}} & {\textsc{sep}} & {\textsc{intr}} & {\textsc{pass}} & {\textsc{stat}}\\
\midrule
Kenyang\il{Kenyang} &
Btd & \textit{ti, ka} &   &  \textit{ka} &  \textit{si, ti} &   &   &   &   &   &   &  \textit{ɛ}\\
% \tablevspace
Mbe\il{Mbe} & 
Btd &  &  &  &  &  &  &  & \textit{li,ri} &  \textit{li} &  & \\
% \tablevspace
\ilit{Tikar} &
Btd & \textit{k/ga’} &  &  & \textit{si, li} &  &  &  &  & \textit{li} &  & \\
% \tablevspace
Vute\il{Vute} &
Btd &  &  &  & \textit{tɨ, hɨ, lɨ} & \textit{na} &  \textit{an} &   &   &  \textit{lɨ} &  & \\
% \tablevspace
\ilit{Kemezung} &
Btd &  &  &  & \textit{sə} &   &  \textit{nə} &  &  &  &  & \\
% \tablevspace
Noni\il{Noni} & 
Btd & \textit{yɛ kɛn} &  \textit{cɛ} &   &  \textit{se, ke} &   &  \textit{ɛn, nɛn,  sɛn, yɛn} &   &  \textit{tEn} &   &   &  \textit{m}\\
% \tablevspace
Babanki\il{Babanki (Kejom)} & 
Ring\il{Ring} & \textit{tə, kə, lə,  m´} &  \textit{tə, nə} &   &  \textit{sə} &   &   &  \textit{(nə)} &   &  \textit{(mə)} &  & \\
% \tablevspace
Kom\il{Kom} & 
\il{Ring}Ring & \textit{tə, lə, nə} &  \textit{tə, lə} &   &  \textit{sə} &   &  \textit{nə} &  &  &  &  & \\
% \tablevspace
Lamnso'\il{Lamnso'}\footnote{For a slightly different, fuller identification of the \ilit{Lamnso’} verb extensions along with examples, see \citet{Blench2016}.} &
{Ring} & \textit{kir, ti(n), ri} &  \textit{ti} &  \textit{si(n), ti(n)} &  \textit{si, ir} &   &  \textit{nen} &   &   &  \textit{in} &   &  \textit{(im)}\\
% \tablevspace
Babungo\il{Babungo} & 
{Ring} &  &  &  & \textit{sə, (tə)} &   &  \textit{nə} &   &  \textit{nə} &  \textit{nə} &  & \\
% \tablevspace
\ilit{Isu} & 
Ring\il{Ring} & \textit{i, lə} &   &  \textit{i, lə} &  \textit{i} &  &  &  &  &  &  & \\
% \tablevspace
Meta\il{Meta} & 
Mo & \textit{ri, ni} &  \textit{ri, ni} &   &  \textit{ri, ni} &  \textit{ri} &  \textit{ri, ni} &  \textit{ri} &  \textit{ri} &  &  & \\
% \tablevspace
\ilit{Mundani} & 
Mo & \textit{t} &  \textit{t} &   &   &   &   &   &   &  \textit{t} &  & \\
% \tablevspace
\ilit{Baba I} &
Nun & \textit{tə} &  &  &  &  &  &  &  &  &  & \\
% \tablevspace
\ilit{Limbum} & 
NE & \textit{ni, shi, se, te, nger} &  \textit{ri} &   &  \textit{si} &   &  \textit{ni} &   &  \textit{ni} &  \textit{ti, té}  &   &  \\
\lspbottomrule                         
\end{tabular}
}
\caption{Survey of Bantoid verb extensions}
\label{extab:proto:8}
\parbox{\textwidth}{\raggedright\scriptsize
\textsc{Plur} = pluractional (multiplicity), iterative, repetitive, frequentative; \textsc{Dim} = diminutive, attenuative; \textsc{Intens} = intensive, quantity, effort, completely; \textsc{Caus} = causative, transitive; \textsc{Appl} = applicative, benefactive; \textsc{Assoc} = associative (together) with, manner or instrumental, simultaneity; \textsc{Sep} = separative, ablative, reversive, bifurcative; \textsc{Intr} = detransitivizing, spontaneous (‘by itself’), \textsc{Pass} = Passive; \textsc{Stat} = stative, positional. Btd = {Bantoid}, Mo = {Momo}, NE = NE {Grassfields}, Ng = {Ngemba} ({EGB}), Bk = {Bamileke} ({EGB}), Ada = {Adamawa}.
}
\end{table}

\begin{table} 
\fittable{
\begin{tabular}{l@{~}p{10mm}p{10mm}p{5mm}p{8mm}p{5mm}p{5mm}p{5mm}p{8mm}p{5mm}p{5mm}p{5mm}p{5mm}}
\lsptoprule
 & & {\textsc{plur}} & {\textsc{dim}} & {\textsc{intens}} & {\textsc{caus}} & {\textsc{appl}} & {\textsc{rec}} & {\textsc{assoc}} & {\textsc{sep}} & {\textsc{intr}} & {\textsc{pass}} & {\textsc{stat}}\\
\midrule
\ilit{Yamba} & NE &  &  &  & \textit{sə} &  &  &  &  &  &  & \\
{\ilit{Bafut}} & 		\textit{Ng} &  \textit{tə, kə} &  \textit{tə} &  \textit{kə} &  \textit{sə} &   &  \textit{nə} &  \textit{nə} &   &  \textit{nə, kə}  &   & \\
{\ilit{Mankon}} & Ng &		 \textit{kə} &  \textit{tə} &   &  \textit{sə} &   &  \textit{nə} &   &   &  \textit{kə} &   &  \textit{nə}\\
\ilit{Ngombale} & Bk &		 \textit{té} &   &   &   &   &   &   &   &   &   &  \textit{e}\\
\ilit{Ngwe} & Bk &		 \textit{te} &   &   &   &   &  \textit{(ŋe)} &   &   &   &   & \\     
\ilit{Yemba} & Bk & 		\textit{ti} &   &   &  \textit{ni} &   &  \textit{ni} &   &   &  \textit{ti} &   &  \textit{ni}\\
\ilit{Ngiemboon} & Bk &	 \textit{tɛ} &   &   &   &   &  \textit{tɛ} &   &   &   &   &  \textit{e}?\\
{\ilit{Bangwa}} & Bk & 	\textit{sə} &   &   &   &   &   &   &   &   &   & \\
{\ilit{Shingu}} & Bk & 	\textit{sə} &   &   &  \textit{ni} &   &  \textit{ni} &   &   &  \textit{ti} &   &  \textit{ni}\\
\ilit{Balong} & A10 &  &  &  & \textit{il} &  \textit{il} &   &   &   &  \textit{il} &   & \\
\ilit{Mokpe} & A20 &  &  &  &	 \textit{isɛ} &  \textit{ea, ɛlɛ} &  \textit{anɛ} &  \textit{ana} &  \textit{oa} &   &  \textit{avE} &  \textit{ama}\\                          
\ilit{Bakoko} & A40 &  &  &  && \textit{le} &  \textit{lán} &  \textit{lán} &   &   &  \textit{bE\$, lE\$}  & \\
\ilit{Basaa}\footnote{The umlaut in the \ilit{Basaa} extensions indicate that the given suffix causes vowel height to rise.} & A40 &  &  &  &	 \textit{¨s, ¨ha} &  \textit{¨l, nɛ} &  \textit{na} &   &   &   &  \textit{¨(b)a} &  \textit{í}\\
{\ilit{Tunen}} & A40 &  &	 \textit{Vl} &  \textit{In, on} &  \textit{i, si} &  \textit{In} &  \textit{Inan} &   &  \textit{Un} &   &   &  \textit{Im}\\
\ilit{Nomaante} & A40 &  & 	\textit{It, ItIt} &  \textit{ak} &  \textit{i, si} &  \textit{In} &  \textit{an} &   &  \textit{Vl} &   &   &  \textit{Im}\\
{\ilit{Bafia}} & A50 & 	\textit{tɨ, kə} &   &  \textit{tɨ} &  \textit{sɨ} &   &  \textit{Cɛn} &   &   &   &   &  \textit{ɨ, ɛn}\\
{\ilit{Gunu}} & A60 &		 \textit{an} &  \textit{Id} &  \textit{Ig} &  \textit{i} &  \textit{In} &  \textit{Inan} &   &  \textit{Ug} &   &  \textit{VlU} &  \textit{Im}\\
{\ilit{Tuki}} & A60 &  &  &  &	 \textit{iy} &  \textit{en} &  \textit{an} &   &   &   &  \textit{érí} & \\
 & A70 &  &  & 		\textit{lá} &  \textit{(l)a} &   &  \textit{(a)ni} &   &   &   &  \textit{ba}  & \\
 & A80 &  &  &  &	 \textit{gù} &   &  \textit{àlà} &  \textit{na, ElE} &   &   &  \textit{a} & \\
 & A80 &  &  & 		\textit{ug, ula} &  \textit{al} &   &  \textit{la, ya} &   &   &   &  \textit{ow} &  \textit{ya}\\
 & A80 &  &  &  &	 \textit{ə̀zə̀} &  \textit{éà} &  \textit{ə̀là} &   &  \textit{bà?} &   &  \textit{ówà} & \\
\ilit{Mpompon} & A80 &  &  &  & \textit{sɛ̀l} &   &  \textit{là} &   &   &   &  \textit{ì...yâ}  & \\
\ilit{Kako} & A90 &  &  &  &     \textit{s, iɗy} &   &  \textit{in} &   &   &  \textit{in} &   & \\
\lspbottomrule
\end{tabular}
}
\label{extab:proto:8ctd}
\end{table}



\newpage 
From the above we can make the following observations:

\begin{itemize}
\item[(i)]  Verb extensions are most widespread in a contiguous area including \ili{Limbum} (\ili{EGB}), \ili{Noni} (\ili{Beboid}), Central \ili{Ring} (\ili{WGB}), and \ili{Ngemba} (\ili{EGB}), indicated by the ellipse I have drawn over the following map.
\item[(ii)] Areas outside the oval area on the map have undergone considerable reduction in their extensions, sometimes dramatically. \ili{Ejagham}, for example, only has a stative suffix \textit{-am} and a few frozen relics of causative \textit{-i,} e.g. Western \ili{Ejagham} \textit{-ríg} ‘to be burn’, \textit{-ríg-í} ‘to burn something’ (\citealt[444, fn. 1]{Watters1981}).
\item[(iii)] Related Grassfields languages outside the oval have fewer extensions, e.g. Western \ili{Ring} \citep{Kiessling2004}, \ili{Momo}, and \ili{Bamileke}—often few forms with considerable polysemy and unpredictability.
\end{itemize}

One language which shows a wide range of verb extensions is \ili{Babanki (Kejom)}. Out of 434 verbs, 324 from \citet{Jisa1977} and 122 from \citet{Akumbu2008}, I have counted the following number of entries for each of six extensions, whose meanings are also identified (\tabref{extab:proto:9}).

\begin{table}
% \fittable{

\begin{tabularx}{\textwidth}{QSSSSSr}
\lsptoprule
& {\textit{-tə}} & { \textit{-sə}} & { \textit{-mə}} & { \textit{-lə}} & {\textit{-kə}} & {\textit{-nə}}\\
\midrule
{Total number} & 203 & 142 & 56 & 37 & 33 & 19\\
% \tablevspace
{independent root} & 150 & 100 & 30 & 22 & 24 & 8\\
% \tablevspace
{“formal”} & 53 & 42 & 26 & 15 & 9 & 11\\
% \tablevspace
  {Primary} \mbox{meaning}&\scriptsize\centering attenuative ‘a little’  &\scriptsize causative {‘cause~to~V’}  &\scriptsize associative \mbox{‘with,~together’} &\scriptsize augmentative ‘a lot’&\scriptsize repetitive  \mbox{‘time and again’}&\scriptsize (varies)\\ 
\lspbottomrule                          
\end{tabularx}
% }
\caption{Babanki (Kejom) verb extension}
\label{extab:proto:9}
\end{table}

\largerpage[2]
\noindent
I have also separately indicated those entries for which an independent root exists vs. those which are “formal” extensions without a corresponding independent root. As seen, attenuative \textit{-tə} is the most attested, followed by causative \textit{-sə}.\footnote{\textit{-tə} has another common meaning: iterative ‘one after another’.} \citet[39-44]{Bila1986} identifies the following extensions in \ili{Bafut} (\ili{EGB}) which occur with independent roots, often with different, overlapping meanings (\tabref{extab:proto:10}).\footnote{Bila uses the term “spontaneous” to refer to what I have identified as “middle” (voice), which he says “indicates that the action suggested by the verb is capable of going on without the assistance of an external agentive force.” (p.42)}\vspace{-5mm}

\begin{figure}[p]
\includegraphics[width=\textwidth]{figures/hymanproto.pdf}
\caption{Map adapted from Jean-Marie Hombert from \citet[xii]{Hyman1979ed}}
\label{fig:hyman:map}
\end{figure}






\begin{table}[p]
\fittable{
\begin{tabular}{rrrrrr}
\lsptoprule
\textit{-kə} &  \textit{-tə} &  \textit{-nə} &  \textit{-sə} & { \textit{-lə}}\\
\midrule
388 & 338 & 171 & 117 & 112\\
distributive (320) & diminutive (314) & reciprocal (52) & causative & randomness (66)\\
repetitive (28) & repetitive (16) & simultaneous (72) &  & roughness (22)\\
middle (16) & distributive (8) & middle (47) &  & on several parts (7)\\
quantitative (20) &  &  &  & \\
\lspbottomrule
\end{tabular}
}
\caption{Bafut verb extensions}
\label{extab:proto:10}
\end{table}

\clearpage 



     
Another extension that Bila mentions is perfective \textit{-mə}, which being inflectional can occur on all 600 verbs in his corpus. Even ignoring this suffix, the comparison between \ili{Bafut} and \ili{Babanki} in \tabref{extab:proto:11} shows that the lexical frequency of the extensions can vary considerably in different languages.

\begin{table}
\begin{tabularx}{\textwidth}{ll p{0mm} Q p{1mm} Q p{1mm} Q p{1mm} Q p{1mm} Q p{1mm} Q}
\lsptoprule
% \ea%11 
Bafut\il{Bafut} & n=600 &  & \textit{mə }(600) & > & \textit{kə} (388) & > & \textit{tə} (338) & > & \textit{nə} (171) & > & \textit{sə} (117) & > & \textit{lə} (112)\\
\tablevspace
Babanki\il{Babanki} & n=434 &  & \textit{tə }(203) & > & \textit{sə} (142) & > & \textit{mə} (56) & > & \textit{lə} (37) & > & \textit{kə} (33) & > & \textit{nə} (19)\\
\lspbottomrule
\end{tabularx}
\caption{Lexical frequency of extensions in Bafut and Babanki}
\label{extab:proto:11}
\end{table}

Returning to \ili{Babanki}, the examples in (\ref{ex:proto:12}a) are representative of the 80+ verbs found to have a very clear causative meaning:

\newcommand{\fourbox}[4]{\parbox{2cm}{#1}\parbox{3cm}{#2}\parbox{2cm}{#3}\parbox{3cm}{#4}}
\newlength{\boxone}
\newlength{\boxtwo}
\newlength{\boxthree}
\newlength{\boxfour}
\setlength{\boxone}{2cm}
\setlength{\boxtwo}{2cm}
\setlength{\boxthree}{2cm}
\setlength{\boxfour}{2cm}

\setlength{\boxone}{3cm}
\setlength{\boxtwo}{5cm}

\newcommand{\twobox}[2]{\parbox[t]{\boxone}{#1}\parbox[t]{\boxtwo}{#2}\\\medskip}
\ea%12
    \label{ex:proto:12}    
    \ea 
    \twobox{\textit{vì}  ‘come’}{\textit{vì-sə̀}  ‘bring near}
    \twobox{\textit{fɛ́n}  ‘be black’}{\textit{fɛ́n-sə́}  ‘make black’} 
    \twobox{\textit{dhú}  ‘go’}{\textit{dhú-sə́}  ‘carry away’}
    \twobox{\textit{zhɨ́ } ‘eat’}{\textit{zhɨ́-sə́}  ‘feed’} 
    \twobox{\textit{búŋ}  ‘melt’}{\textit{búŋ-sə́}  ‘cause to melt’}
    \twobox{\textit{lyɔ́m}  ‘hurt self’}{\textit{lyɔ́m-sə́}  ‘hurt s.o.’} 
    \ex
    \twobox{\textit{vì}  ‘come’}{\textit{vì-nə̀}  ‘come with’}
    \twobox{\textit{tsí}  ‘spend night’}{\textit{tsí-nə́}  ‘... with a woman’}   
    \ex  
    \twobox{\textit{cò}  ‘pass’}{\textit{cò-mə̀}  ‘meet and pass’}
    \twobox{\textit{kwèʔè}  ‘think’}{\textit{kwèʔ-mə̀} ‘think together’} 
    \twobox{\textit{gè}  ‘share’}{\textit{gè-mə̀}  ‘share equally’}
    \twobox{\textit{táŋ}  ‘count’}{\textit{táŋ-mə́}  ‘quarrel’} 
    \twobox{\textit{shɨ̀ʔ}  ‘measure’}{\textit{shɨ̀ʔ-mə̀}  ‘compare measures’}   
    \z
\z

\noindent
From (\ref{ex:proto:12}a) there is no question, then, that \ili{Babanki} \textit{-sə} is related to \ili{PB} *-\textit{ɪc-i-}. Of the 19 verb roots which take \textit{-nə} only the two examples in (\ref{ex:proto:12}b) show a clear comitative meaning, suggesting cognacy with \ili{PB} \textit{*-an-}. Finally, (\ref{ex:proto:12}c) illustrates the ‘associative’ meaning of \textit{-mə}, which may ultimately be related to \ili{PB} positional \textit{*-am-}, which has a passive function in zone C (cf. the stative function of  \ili{Gunu} \textit{-Im-} in \tabref{extab:proto:7}).

While the above and other specific meanings can be identified for individual extensions, any of the six suffixes can be used with varying pluractional meanings:

 
\setlength{\boxone}{3.5cm}
\setlength{\boxtwo}{7cm}
\ea%13
    \label{ex:proto:13}
  \ea \textit{-tə} (23) \\ 
  \twobox{\textit{  bɛ́n}  ‘dance’}{  \textit{bɛ́n-tə́}  ‘dance time and again’} 
      \twobox{\textit{bɔ́ŋ}  ‘pick up’}{  \textit{bɔ́ŋ-tə́}  ‘pick up many things one by one’} 
      \twobox{\textit{bwìɛ̀ʔɛ̀}  ‘carry’}{  \textit{bwìɛ̀ʔ-tə̀}  ‘carry (lots of people, lots of things)’} 
      \twobox{\textit{cɔ́ʔ}  ‘borrow, lend’}{  \textit{cɔ́ʔtə́}  ‘lend continuously to lots of people, borrow from lots of sources’}
      \twobox{\textit{gè}  ‘share’ }{ \textit{gè-tə̀}  ‘share one by one’}
      \twobox{\textit{shù}  ‘stab’}{  \textit{shù-tə̀}  ‘stab lots of things one by one or one thing many times’}
  \ex \textit{-lə}  (12)  \\
      \twobox{\textit{zhwí}  ‘kill’}{  \textit{zhwí-lə́} ‘kill one after the other, lots of people’}
      \twobox{\textit{mì}  ‘swallow’}{  \textit{mì-lə̀}  ‘swallow fast, gulping, too much in mouth’}
      \twobox{\textit{té}  ‘abuse’}{  \textit{té-lə́} ‘abuse lots of people or abuse one person with lots of abuse’}
      \twobox{\textit{bwìʔì}  ‘hit’}{  \textit{bwìʔ-lə̀}  ‘give blows a lot’}
 \ex \textit{-kə}  (7)  
      \twobox{\textit{dì}  ‘cry’ }{ \textit{dì-kə̀}  ‘cry time and again’ }  
      \twobox{\textit{fʌ́ŋ}  ‘fall’}{  \textit{fʌ́ŋ-kə́}  ‘fall time and again’}
      \twobox{\textit{pfɨ́ } ‘die’}{  \textit{pfɨ́-kə́}  ‘die one after the other’}
      \twobox{\textit{tsɔ́ʔɔ́}  ‘jump’ }{ \textit{tsɔ́ʔ-kə́}  ‘jump time and again’}
 \ex  \textit{-mə} (4)
      \twobox{\textit{ lám}  ‘marry’}{ \textit{lám-mə́}  ‘marry a lot’}
      \twobox{\textit{shwíé}  ‘sink’}{  \textit{shwíé-mə́ } ‘sink \& surface and sink \& surface’}
      \twobox{\textit{tsɔ́ʔɔ́}  ‘jump’}{  \textit{tsɔ́ʔ-mə́}  ‘jump time and again’}
 \ex  \textit{-sə} (2)  
      \twobox{\textit{bvù}  ‘grind’}{  \textit{bvù-sə̀ } ‘grind \& mix lots of things’}
      \twobox{\textit{gè}  ‘divide, share’}{  \textit{gè-sə̀}  ‘separate into more parts’  }
 \ex  \textit{-nə}
      \twobox{\textit{lém}  ‘bite’}{  \textit{lém-nə́ } ‘bite and leave and bite another spot’  (= the only example)  }
  \z
\z

Several verbs have two or three different pluractional forms:
 
\ea%14
    \label{ex:proto:14}
    \ea 
	\twobox{\textit{tsɔ́ʔɔ́}  ‘jump’}{  \textit{tsɔ́ʔ-mə́}  ‘jump one after the other’} 
        \twobox{~}{\textit{tsɔ́ʔ-kə́}  ‘jump time and again’} 
        \twobox{~}{\textit{tsɔ́ʔ-lə́}  ‘jump across things’} 
        \twobox{cf. }{ \textit{tsɔ́ʔ-tə́}  ‘jump gently’ (= attenuative—see below)} 
    \ex  
	\twobox{\textit{dì} ‘cry, cackle’ }{ \textit{dì-mə̀}  ‘lots of children crying’}
        \twobox{~}{\textit{dì-kə̀}  ‘cry time and again’}
        \twobox{~}{\textit{dì-lə̀}  ‘lots of chickens cackling’}
    \ex  
	\twobox{\textit{zhwí}  ‘kill’ }{ \textit{zhwí-tə́}  ‘kill one by one, bit by bit’}
        \twobox{~}{\textit{zhwí-lə́}  ‘kill lots of people, one after the other’}
    \ex 
	\twobox{\textit{sù}  ‘stab’}{  \textit{sù-tə̀}  ‘stab lots of things one by one, or one thing many times’}
        \twobox{~}{\textit{sù-lə̀}  ‘stab with lots of things at one time’}
    \z
\z

For an understanding of the possible meanings, compare \citegen{Wood2007} distinction between “event-internal” vs. “event-external” pluractionality, e.g. in \il{Yurok}Yu\-rok:
\begin{modquote}
  ...the Repetitive (event-internal) prefix refers to repetitions which are closely-spaced in time on a single occasion, which may indicate plurality of a transitive object or an intransitive subject... and which commonly have an implied completion or result. The Iterative (event-external) pluractional, in contrast, can refer to repetition on one or more occasions, including habitual repetition, and can indicate distributive plurality of any argument. An interesting additional property of the Iterative is that it has an apparent intensification meaning in certain cases. I have suggested that such uses be analysed as instances in which a standard of comparison or lower bound of a gradable predicate is pluralised by the pluraction, rather than an event argument. \citep[255]{Wood2007}  
\end{modquote}


In \ili{Babanki} also there is thus a relatedness and potential overlap of the different notions of pluractionality (many participants, many actions, over and over, one after the other, bit by bit, etc.). These meanings spill over into others. Pluractionality may straightforwardly lead to augmentative or intensive interpretations, e.g. with \textit{-lə}: 
 
\ea%15
    \label{ex:proto:15}
  \ea 
  \twobox{\textit{sáʔá}  ‘spring forward’ }{\textit{sáʔ-lə́}  ‘spring, jump for joy’}
 \ex 
 \twobox{\textit{sù}  ‘stab’ }{\textit{sù-lə̀}  ‘stab with lots of things at one time’}
 \ex  
 \twobox{\textit{fósé}  ‘force’ }{\textit{fó-lə́}  ‘be too tight (space), crowded, congested’}
 \ex  
 \twobox{\textit{gàʔà}  ‘speak’ }{\textit{gàʔ-lə̀}  ‘talk as if crazy’}
 \ex  
 \twobox{\textit{mì}  ‘swallow’}{\textit{mì-lə̀}  ‘swallow fast, gulping, too much in mouth’}
  \z	
\z

There also is a potential relatedness between pluractionality and attenuation (“diminutivizing”), e.g. with \textit{-tə}:\footnote{Some of these meanings are reminiscent of \ili{Bantu} frequentative/distributive verb stem reduplication which typically has the meaning ‘do something a little here and there’.}
 
\ea%16
    \label{ex:proto:16}
  \ea 
  \twobox{\textit{cíʔ}  ‘close, shut’ }{\textit{cíʔ-tə́}  ‘shut, close lots of things one after the other or a bit’}
 \ex 
 \twobox{\textit{ló}  ‘lick’ }{\textit{ló-tə́}  ‘lick time and again or little by little, slowly’}
 \ex 
 \twobox{\textit{tyɛ́f}  ‘advise’}{\textit{tyɛ́f-tə́}  ‘advise one by one or a little’}
 \ex 
 \twobox{\textit{sh\`{ü}}  ‘wash’ }{\textit{sh\`{ü}-tə̀}  ‘wash lots of things or a little, part(s) of body’}
 \ex 
 \twobox{\textit{kwíʔ}  ‘tie’ }{\textit{kwíʔ-tə́}  ‘tie in different bundles or gently’}
 \ex 
 \twobox{\textit{ny\'{ü}}  ‘drink’ }{\textit{ny\'{ü}-tə́}  ‘drink bit by bit or a little bit’}
 \ex 
 \twobox{\textit{ghɔ́ʔ}  ‘become fat’ }{\textit{ghɔ́ʔ-tə́}  ‘get fat little by little’}
  \ex 
  \twobox{\textit{fwìè}  ‘rot (intr.)’  }{\textit{fwìè-tə̀}  ‘rot in bits’  \textit{vs.} fwìè-kè ‘rot time and again’}
  \z
\z

Such semantic relatedness may lead to massive conflation/merger, as in \ili{Meta} (in the \ili{Momo} subgroup), which has two different extensions, /\textit{-dɨ}/ (${\rightarrow}$ \textit{-rɨ}) and /\textit{-nɨ}/. The following examples drawn from the 262 verbs from \citet{Ngum2004} show the realizations of the two extensions (attenuatives are given where possible):

\ea%17
    \label{ex:proto:17}
\ea
     \twobox{\textit{kwí}  ‘grow’ }{\textit{kwí-rɨ}  ‘grow a bit’}
     \twobox{\textit{sob}  ‘cut’ }{\textit{sob-rɨ}  ‘cut small’}
     \twobox{\textit{mèd}  ‘swallow’}{\textit{me-rɨ}  ‘swallow in small quantities’\\  (d ${\rightarrow}$ Ø / {\longrule}r)}
     \twobox{\textit{mìg}  ‘measure’ }{\textit{mìg-rɨ}  ‘measure with, comparatively’}
     \twobox{\textit{kɔʔ}  ‘climb’ }{\textit{kɔʔ-rɨ}  ‘climb a bit’}
 \ex 
     \twobox{\textit{nyə̀m}  ‘push down’}{\textit{nyə̀m-bɨ}  ‘press down gently’  (d ${\rightarrow}$ b / m {\longrule})}
     \twobox{\textit{tàn}  ‘delay’ }{\textit{tàn-dɨ}  ‘delay a bit’}
     \twobox{\textit{fàŋ}  ‘be fat’ }{\textit{fàŋ-gɨ}  ‘be a bit fat’  (d ${\rightarrow}$ g / ŋ {\longrule})}
 \ex 
     \twobox{\textit{cə̀b}  ‘pinch’}{\textit{cə̀p-ɨ}  ‘pinch a bit’      }
     \twobox{\textit{ghàd}  ‘pour’}{\textit{ghàt-ɨ}  ‘pour a bit’}
     \twobox{\textit{jɨ́g}  ‘eat’ }{\textit{jɨ́k-ɨ}  ‘eat a bit’}
 \ex 
     \twobox{\textit{wà}  ‘be rough’ }{\textit{wàà-nɨ}  ‘be a bit rough’}
     \twobox{\textit{cɔʔ}  ‘borrow’ }{\textit{cɔʔ-nɨ}  ‘borrow from’}
     \twobox{\textit{wèm}  ‘tie’ }{\textit{wèm-nɨ}  ‘tie loosely’}
     \twobox{\textit{bin}  ‘dance’ }{\textit{bi-nɨ}  ‘dance a bit’  (n ${\rightarrow}$ Ø / {\longrule} n)}
     \twobox{\textit{màŋ}  ‘seize’ }{\textit{màŋ-nɨ}  ‘seize several objects’}
  \z
\z

From the above examples it appears that there are two different attenuative extensions, each with two allomorphs whose distribution can be predicted. This is confirmed in the following distributions of \ili{Meta} extensions, where T = a voiceless stop, D = a voiced stop or glottal stop, and N = a nasal consonant).

\begin{table}
% \ea%23
%     \label{ex:proto:23}
\caption{Distribution of Meta extensions and their functions based on \citet{Ngum2004}}
\label{extab:proto:23} 
\begin{tabularx}{\textwidth}{Xrrrrr} 
\lsptoprule
 & {{CV(D)-\textit{rɨ}}} & {{CVN-\textit{Dɨ}}} & {{CVT-\textit{ɨ}}} & {{CV(N)-\textit{nɨ}}} & totals\\
\midrule
 {attenuative} & 12 & 10 & 6 & 3 & 31\\
 {repetitive/completely} & 10 & 10 & 6 & 1 & 27\\
 {random/roughly} & 0 & 1 & 0 & 5 & 6\\
 {reciprocal/reflexive} & 5 & 2 & 2 & 4 & 13\\
 {associative} & 6 & 2 & 1 & 2 & 11\\
 {instrumental} & 2 & 3 & 1 & 1 & 7\\
 {ablative/separative} & 4 & 5 & 1 & 1 & 11\\
 {causative} & 4 & 6 & 1 & 7 & 18\\
 {applicative (‘to, for’)} & 4 & 1 & 1 & 0 & 6\\
\midrule
 {      totals} & {47} & {40} & {19} & {24} & \\
\lspbottomrule
\end{tabularx}
% \z
\end{table}

\noindent
The first two columns show that \textit{-rɨ} appears after CV and /CVD/ roots, while \textit{-Dɨ} appears after /CVN/ roots. We can assume underlying /-\textit{r}\textit{ɨ}/ with a rule that converts the /r/ into a homorganic voiced stop after a nasal. The third and fourth columns show that \textit{-ɨ} appears after /CVD/ roots (whose D becomes devoiced), while \textit{-nɨ} appears after /CV/ and /CVN/ roots. Assuming /-\textit{n}\textit{ɨ}/, we would have to say that /b, d, g/  ${\rightarrow}$ p, t, k and the nasal drops out. (Perhaps the language once had geminate stops which devoiced and then degeminated.) In any case this odd allomorphy likely results from an earlier stage where there were more extensions, e.g. \textit{*-tɨ}, \textit{*-dɨ}, \textit{*-nɨ}, \textit{*-sɨ} etc., as in \ili{Babanki} and \ili{Bafut}.

\section{Significance of Grassfields extensions for Bantoid and Bantu}
\largerpage
At this point I would like to raise two questions. First, what does the above mean for \ili{Proto-Bantoid}? In response I would venture the following: (i) Given the rather large set of suffixes in \ili{Limbum}, \ili{Ring}, and \ili{Ngemba}, it is likely that we can reconstruct at least six extensions at the \ili{Proto-Grassfields} level, e.g. \textit{*-s}, \textit{*-t}, \textit{*-n}, \textit{*-l}, \textit{*-k},  and \textit{*-m}. (ii) \ili{Noni} and \ili{Vute} suggest that most or all of these existed at an earlier \ili{Bantoid} stage as well. (iii) The functions that clearly can be reconstructed are \textit{*-s} ‘causative’ and \textit{*-n} ‘reciprocal/associative’. (iv) Pluractional meanings are extremely widespread, hence tempting to reconstruct, but they have clearly spread areally throughout the Nigeria-Cameroon area, including \ili{Chadic} \citep{Newman1990}. (v) The attenuative/diminutive function is quite widespread, even spilling over into A40 and A60!

The second question is: What is the relation of these reconstructions to PB? It is tempting to reconstruct parallel functions and forms of each of the PB extensions in \REF{ex:proto:4} above. However, I have been able to document the applicative only in two languages (\ili{Meta} and \ili{Vute}). In \ili{Meta} I have found only six examples of \textit{-rɨ} having various applicative-like functions (recipient, circumstance, directional): 
 
\ea%24
    \label{ex:proto:24}
  \twobox{\textit{ghàb}  ‘share’ }{ \textit{ghàb-rɨ}  ‘share to’}
 \twobox{ \textit{wí } ‘refund’  }{\textit{wíí-rɨ } ‘reply, refund to’}
 \twobox{ \textit{cob}  ‘donate’ }{ \textit{cob-rɨ}  ‘donate for’} 
 \twobox{\textit{wub}  ‘crave’  }{\textit{wub-rɨ}  ‘crave for’}
  \twobox{\textit{sòm } ‘cut’  }{\textit{sòm-bɨ}  ‘cut into’} 
 \twobox{ \textit{dìì}  ‘pity’ }{ \textit{dìì-rɨ}  ‘pity for’}  
\z

\noindent
Since \textit{-rɨ} has other functions, it is not clear if this suffix is cognate with PB applicative \textit{*-ɪd-}. The situation is much less ambiguous in \ili{Vute}, where applicative \textit{-nà} is innovative:

\begin{quote}
\textit{-nà} is added to a verb to indicate that there is an indirect object or benefactive NP present in the clause. Its function is similar to a \ili{Bantu} applicative extension in this way. \textit{-nà} is derived from the verb \textit{nà-nɨ} ‘to give’. \citep[8]{Thwing2006}
\end{quote}


\section{The shift from valence to aspectual extensions}

It cannot have escaped notice that most of the extensions which have (possibly lexicalized) aspectual meanings such as pluractional, attenuative/diminutive etc., resemble the valence extensions of \ili{Bantu} (and other \ili{Niger-Congo}). Thus consider the following examples from \ili{Bangwa} (\ili{Bamileke}) which show that the repetitive suffix \textit{-sɨ}, clearly cognate with the causative extension found throughout \ili{Bantu}, marks “une action ou une situation qui se répète plusieurs fois” in this language \citep[243]{Nguendjio1989}:

\ea%25
    \label{ex:proto:25}
  \parbox{3cm}{\textit{ghɛ̀ } ‘partager’} ${\rightarrow}$~~  \textit{ghɛ̀-sə̀}  ‘partager plusieurs fois’  \\
  \parbox{3cm}{\textit{sò}  ‘laver’}     ${\rightarrow}$~~  \textit{sò-sə̀ } ‘laver plusieurs fois’\\
  \parbox{3cm}{\textit{cí}-  ‘casser’}   ${\rightarrow}$~~  \textit{cí-sə́}  ‘casser plusieurs fois’\\
  \parbox{3cm}{\textit{fák } ‘tourner’}  ${\rightarrow}$~~  \textit{fák-sə́}  ‘tourner plusieurs fois’\\
  \parbox{3cm}{\textit{yàʔ } ‘couper’}   ${\rightarrow}$~~  \textit{yàʔ-sə̀ } ‘couper plusieurs fois’\\
\z

The same could be seen from \textit{-sə} in closely related \ili{Shingu} \citep[88]{Ndawouo1990} and \textit{-si} in \ili{Fe’fe’} \citep{Ngangoum1970} and the phenomenon extends into \ili{Bantu} and other \ili{Benue-Congo} languages in Nigeria. 
As \citet[5]{Gerhardt1988} puts it, “What is remarkable about these [verb extensions in \ili{Jarawan Bantu}] is that those with syntactic functions have been lost, while aspect-like VEs are still present.” I would differ only in not assuming that the current meanings are proto. Instead, I would like to propose that valence extensions, i.e. those that have to do with argument structure, generally become pluractional, attenuative etc. by a three-stage process:

\eabox{
%25
\begin{tabularx}{\textwidth}{XlXlX}
\textit{Stage I} &  & \textit{Stage II} &  & \textit{Stage III}\\
valence ${\supset}$ aspect & > & aspect ${\supset}$ valence & > & aspect\\
\end{tabularx}
}

First, valence marking affixes start to acquire aspectual meanings, which have spread areally. Then the aspectual meanings become primary, with gradually lexicalized, residual valence functions. The final stage is for the extensions to have only an aspectual function. According to this model, \ili{PB} is at stage I, \ili{zone A Bantu} is somewhere between stage I and stage II, and \ili{Bantoid} is somewhere between stage II and stage III.

The evidence for such a valence > aspect realignment is considerable. First, there is the phonetic similarity already alluded to. Second, aspectual extensions may correlate with valence: In \ili{Bafut} the iterative/repetitive extension \textit{-kə} is used with intransitives, while the “contextual variant” \textit{-tə} is used with transitives (\citealt[22]{TamanjiMba2003}; \citealt[99]{Bila1986}). This is strikingly reminiscent of the \ili{PB} reversive (“separative”) suffixes \textit{*-ʊk-} (intr.) vs. \textit{*-ʊd-} (tr.). Other forms also suggest that transitivizing extensions tend to be coronal, while detransitivizing ones tend to be velar (cf. PB applicative \textit{*-ɪd-} vs. stative \textit{*-ɪk-}). Finally, there are natural semantic pathways for these developments, e.g. causative > intentional/intensive \citep{Kiessling2004}.

The final question is: Why does this happen? The reason can be seen in the fact that the change of valence to aspect suffixes correlates with phonological, morphological and syntactic changes within Bantoid. In \tabref{extab:proto:26} I contrast the situation in Canonical \ili{Bantu} vs. \ili{Bantoid}.

\begin{table}
\begin{tabularx}{\textwidth}{lQQ}
\lsptoprule
% \ea%26
%     \label{ex:proto:26}    
          & {Canonical Bantu} & {Bantoid}\\
\midrule
{phonology} & minimum word = 2 syllables & maximum stem = 2 or 3 syllables\\
\tablevspace
{morphology} & highly synthetic, agglutinative & less so, gradual move towards analyticity\\
\tablevspace
{unmarked objects} & multiple & one per verb\\
\tablevspace
{marked objects} & head marking on verb & prepositions, serial verbs\\
\lspbottomrule
\end{tabularx}
\caption{Canonical Bantu contrasted with Bantoid}
\label{extab:proto:26}
\end{table}

If we assume that \ili{Proto-Bantoid} was also head-marking in the sense of \citet{Nichols1986}, where valence operations are indicated by verb suffixes, the driving force behind the change likely was phonological: While CB languages have no upper limit on word size (in fact, many require words to have at least two syllables), many \ili{NW Bantu} and \ili{Bantoid} languages place an upper limit on the number of syllables that a word (and especially a verb stem) can have. By so doing, this limits the availability of suffixes, since a verb that already has exhausted, say, a three-syllable maximum size will not be able to take a causative or applicative extension. Instead, some other, specifically analytical marking will be required: a periphrastic causative (‘make that S’), prepositions ‘for’ and ‘with’. and so forth. The newly introduced mechanisms then come to be the preferred structures. We already see some of this happening in languages that still have some valence-related extensions, e.g. a causative. While the causative can also be added to transitive and ditransitive verbs in Canonical \ili{Bantu}, what happens in many Bantoid languages is that \textit{*s} is mostly restricted to intransitive verbs. That is, while it can make an intransitive transitive, it cannot make a transitive verb ditransitive. Those relatively few transitive verbs that can take a causative extension restructure their arguments, as in the following \ili{Babungo} (\ili{Ring}) and \ili{Bafut} (\ili{Ngemba}) examples.
 
\ea%30
\ea 
\ili{Babungo} \\
    \twobox{\textit{ŋwə́ fèe zɔ}̏ }{ ‘he was afraid of (i.e. feared) a snake’  \citep[211]{Schaub1985}}
    \twobox{\textit{mə̀ fè-sə̀ ŋwə́ (nə̀ zɔ̏) }}{ ‘I frightened him (with a snake)’}
 \ex 
 \ili{Bafut}\\
      \twobox{\textit{má shwìʔì ŋki} }{ ‘I am pouring water’}
      \twobox{\textit{má shwìʔì-sə̀ ŋkì }  }{‘I am making water to pour’   \citep[102]{Bila1986}  }
\z
\z

\noindent
A causativized transitive verb cannot take two objects. Thus, after pointing out that the causative adds a valence to intransitive verbs, \citet[102]{Bila1986} notes: “This suffix however does not add to the valency of the [transitive] verb but it rather modifies the meaning of the verb by adding the causative meaning to the basic meaning of the verb.” 

\section{Summary and conclusion}

In the preceding sections we have seen the following: 
\begin{itemize}
\item[(i)]  \ili{Proto-Bantoid} definitely had multiple verb extensions, probably at least \textit{*-s}, \textit{*-t}, \textit{*-n}, \textit{*-l}, \textit{*-k}, \textit{*-m}. 
\item[(ii)] Languages within the “oval” on the map provided above show the greatest number of contrasts. 
\item[(iii)] \ili{Bantoid} languages outside the oval have simplified the situation considerably. 
\item[(iv)] There is an unmistakable tendency for valence-related extensions to become aspectual. 
\item[(v)] Contributing factors to the change and loss of extensions (and their ability to combine) are phonological (maximal size constraints), morphological (drift towards analyticity) and syntactic (change from head- to dependent-marking of arguments).
\end{itemize}

I conclude with some final observations and speculations:   
\begin{itemize}
\item[(i)]  Most of the \ili{Bantoid} languages restrict verbs to one extension, mostly of the shape *-CV. 
\item[(ii)] -CVC shapes such as \ili{Noni} \textit{-kɛn, -nɛn, -sɛn, -yɛn} and \ili{Lamnso'} \textit{-sin}, \textit{-tin}, \textit{-kir} suggest that these were originally two extensions which fused. 
\item[(iii)] Such a process is particularly common in \ili{Bantu} when \ili{PB} reciprocal \textit{*-an-} is involved (see \citealt{BostoenNzangBie2010} for documentation in A70). 
\item[(iv)] Interestingly, different shapes of \textit{-(C)ɛn} can mark reciprocals in \ili{Noni} \citep[39-40]{Hyman1981}, suggesting \textit{-C-ɛn-} (cf. \citealt[62ff]{Guarisma2000} re \textit{-Cɛn} in \ili{Bafia} (A53)). 
\item[(v)] Many Bantoid languages have productive aspectual suffixes not mentioned above, e.g. perfective \textit{-mV}, probably related to PB \textit{*-mad} ‘finish’, realized with various vowels in \ili{Grassfields Bantu}.
\end{itemize}

The most puzzling question to me is where diminutive/attenuative \textit{*t} is from, especially the \textit{-It-}/\textit{-Id-} in A40 and A60. This has no obvious source in \ili{PB} or cognates in CB and can therefore be an areal innovation. The phonetic similarity to plural diminutive noun class 13 \textit{ti-} is intriguing, as it becomes suffixal \textit{-tí} in some \ili{WGB}. About this class in the \ili{Momo} languages, \citet[209]{Stallcup1980momo} notes: “19/13 was originally a diminutive gender.... This gender also contains a number of items [in \ili{Moghamo}] which occur generally in profusion — ‘star, fly, bird’ etc.” Here again we have the relation between plural and diminutive! While nominalization often results in a verb extension appearing on a noun, the reverse has not been established. Could sound symbolism be involved somehow? Cf. \ili{Basaa} (A43) \textit{títígí} ‘small’.

I opened this paper by posing three questions of which I have addressed the first two: 
\begin{itemize}
\item[(i)]  What are \ili{Bantoid} verb extensions like? 
\item[(ii)] What can be reconstructed at the \ili{Proto-Bantoid} level? 
\item[(iii)] What, if anything, do they tell us about \ili{Proto-Bantu}?
\end{itemize}
 The third question presents a more different problem of interpretation. As we have seen, there is a considerable number of cognate extensions between the two and, indeed, further out at least within \ili{Benue-Congo} and \ili{Gur}. It is thus likely that some of what has been hypothesized for \ili{Proto-Bantoid} is considerably older, as \citet{Voeltz1977} originally supposed. Unfortunately, because of the kinds of the functional changes, mergers, and losses that have occurred within \ili{Bantoid}, we can only speculate as to what the system looked like at the earliest stages. The great variety of extensions that we see outside of \ili{Bantu} does, however, raise the possibility that \ili{Bantu} itself may have lost earlier contrasts (e.g. between different types of applicatives which merged to \textit{*-ɪd-}), not just that \ili{Proto-Bantu} is conservative from the point of view of \ili{Niger-Congo}.

\section*{Acknowledgement}
My thanks to John Watters and Roger Blench for valuable comments on an earlier version of this paper.
  

{\sloppy
\printbibliography[heading=subbibliography,notkeyword=this]
}
 

\end{document}