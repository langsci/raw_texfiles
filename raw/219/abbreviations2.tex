\addchap{Orthographic conventions}
% \addchap{Abbreviations and symbols}




Throughout, I use the standard written orthography of ZAI \citep{alfabeto1956}, which generally follows the orthographic conventions of Mexican \ili{Spanish}, for example: 

 

\begin{tabular}{lp{4.5cm}} 


\textit{ch} & /t\textipa{S}/ \\


\textit{g} and \textit{gu} & /g/ \\


\textit{hu} & /w/  \\

\textit{g\"{u}} & /gw/  \\
 

\textit{dx} & /d\textipa{Z}/ \\


\textit{xh} & /\textipa{S}/  \\

 

\textit{x}* & /\textipa{Z}/  \\
    
\end{tabular}
 
 
 \noindent *Note, however, that \textit{x} before voiceless consonants is pronounced [\textipa{S}]; often used as \textsc{poss} morpheme.\bigskip

 

Although ZAI is a tonal language, \isi{tone} is not marked in the ZAI orthography. I note the underlying tonal information in the gloss (the superficial tones can be straightforwardly derived from the underlying tones -- although this requires more investigation (P\'{e}rez B\'{a}ez, p.c.) -- and use the following notation for tones: 


\begin{tabular}{lp{4.5cm}} 


rising (LH) \isi{tone} & [\textsuperscript{LH}] \\



high (H) \isi{tone} & [\textsuperscript{H}] \\



low (L) \isi{tone} & unmarked \\



Glottalized vowels & apostrophe [{'}] immediately after the vowel  \\



 Laryngealized vowels & two consecutive vowels, [VV] (still within a single syllable)  \\


\end{tabular}


%\end{multicols}




