\documentclass[output=paper,colorlinks,citecolor=brown,]{langsci/langscibook}
\ChapterDOI{10.5281/zenodo.3764851}
%\bibliography{localbibliography}
%\usepackage{langsci-optional}
\usepackage{langsci-gb4e}
\usepackage{langsci-lgr}

\usepackage{listings}
\lstset{basicstyle=\ttfamily,tabsize=2,breaklines=true}

%added by author
% \usepackage{tipa}
\usepackage{multirow}
\graphicspath{{figures/}}
\usepackage{langsci-branding}

%
\newcommand{\sent}{\enumsentence}
\newcommand{\sents}{\eenumsentence}
\let\citeasnoun\citet

\renewcommand{\lsCoverTitleFont}[1]{\sffamily\addfontfeatures{Scale=MatchUppercase}\fontsize{44pt}{16mm}\selectfont #1}
  

\title{Czech modal complement ellipsis from a comparative perspective}

\author{Hana Gruet-Skrabalova\affiliation{Université Clermont Auvergne, \\Laboratoire de recherche sur le langage}}


\abstract{    This paper deals with modal complement ellipsis in Czech from a comparative perspective. I show that Czech modal complement ellipsis displays a mixed behaviour in comparison with languages like English, Dutch and French. Like English, it allows for various extractions from the ellipsis site and for different subjects in antecedent-contained deletion constructions. Like French and Dutch, it does not allow for intervening elements between the modal verb and the ellipsis site and it requires voice identity of the elided VP and its antecedent. Adopting a deletion approach to ellipsis, I propose to account for this behaviour by parametrizing the syntactic properties of a presumably universal ellipsis feature [E], initially proposed by \cite{Lobeck1995}. In my proposal, the syntax of [E] includes the head-licensing ellipsis and the ellipsis site. I argue that the type of licensing head (T, V or Mod) and the type of ellipsis site (VP, TP or VoiceP) induce the properties of modal complement ellipsis that I observe at the surface.

\keywords{modal complement ellipsis, VP ellipsis, modal verbs, auxiliary verbs, ellipsis parameter}
}

\IfFileExists{../localcommands.tex}{
  \usepackage{langsci-optional}
\usepackage{langsci-gb4e}
\usepackage{langsci-lgr}

\usepackage{listings}
\lstset{basicstyle=\ttfamily,tabsize=2,breaklines=true}

%added by author
% \usepackage{tipa}
\usepackage{multirow}
\graphicspath{{figures/}}
\usepackage{langsci-branding}

  
\newcommand{\sent}{\enumsentence}
\newcommand{\sents}{\eenumsentence}
\let\citeasnoun\citet

\renewcommand{\lsCoverTitleFont}[1]{\sffamily\addfontfeatures{Scale=MatchUppercase}\fontsize{44pt}{16mm}\selectfont #1}
  
  %% hyphenation points for line breaks
%% Normally, automatic hyphenation in LaTeX is very good
%% If a word is mis-hyphenated, add it to this file
%%
%% add information to TeX file before \begin{document} with:
%% %% hyphenation points for line breaks
%% Normally, automatic hyphenation in LaTeX is very good
%% If a word is mis-hyphenated, add it to this file
%%
%% add information to TeX file before \begin{document} with:
%% %% hyphenation points for line breaks
%% Normally, automatic hyphenation in LaTeX is very good
%% If a word is mis-hyphenated, add it to this file
%%
%% add information to TeX file before \begin{document} with:
%% \include{localhyphenation}
\hyphenation{
affri-ca-te
affri-ca-tes
an-no-tated
com-ple-ments
com-po-si-tio-na-li-ty
non-com-po-si-tio-na-li-ty
Gon-zá-lez
out-side
Ri-chárd
se-man-tics
STREU-SLE
Tie-de-mann
}
\hyphenation{
affri-ca-te
affri-ca-tes
an-no-tated
com-ple-ments
com-po-si-tio-na-li-ty
non-com-po-si-tio-na-li-ty
Gon-zá-lez
out-side
Ri-chárd
se-man-tics
STREU-SLE
Tie-de-mann
}
\hyphenation{
affri-ca-te
affri-ca-tes
an-no-tated
com-ple-ments
com-po-si-tio-na-li-ty
non-com-po-si-tio-na-li-ty
Gon-zá-lez
out-side
Ri-chárd
se-man-tics
STREU-SLE
Tie-de-mann
}
  \togglepaper[5]%%chapternumber
}{}
%\togglepaper[5]

\begin{document}
\maketitle

\il{Czech|(}
\section{Introduction} \label{sec:1}
This paper deals with verb-phrase \isi{ellipsis} that occurs in context of \isi{modal} verbs, as in \REF{1a}. Following \cite{Aelbrecht2008}, I will refer to this phenomenon as \textsc{\isi{modal complement} ellipsis} (\isi{MCE}) in order to distinguish it from a well-known phenomenon of VP \isi{ellipsis} in \REF{1b} (\citealt{Ross1969,Sag1976,Merchant2001}; among others). When it comes to other languages than \ili{English}, it seems relevant to set apart \isi{ellipsis} after auxiliary verbs and \isi{ellipsis} after \isi{modal} verbs, because the former is not always available in languages that display the latter, as e.g. \ili{Romance} languages (\citealt{busquets2001ellipse,Depiante2001,Dagnac2008,Dagnac2010}, a.o.) and \ili{Germanic} languages like \ili{German} or \ili{Dutch} \citep{Lobeck1995,Aelbrecht2008}, see \REF{2} and \REF{3} respectively.\footnote{VP \isi{ellipsis} after auxiliary verbs is not available in \ili{Romance} languages, except for \ili{Portuguese}, see \cite{Cyrino-Matos2002}.} The contrast between \ili{English} on the one hand and \ili{Romance} and other \ili{Germanic} languages on the other has been argued to follow from the properties of \isi{modal} verbs (see  \sectref{sec:2}).

\begin{exe}
\ex \label{english}
\begin{xlist}
\ex\label{1a} John helped them, but Mary could not.
\ex \label{1b}John helped them, but I did not.
\end{xlist}

\ex
\begin{xlist}\label{2}
\ex[]{\gll Jean les a aidés, mais Marie n'a pas  pu.\\
J. them has helped.\textsc{pl} but  M. \textsc{neg}.has \textsc{neg} could\\
\glt `Jean helped them, but Marie could not.' \hfill (\ili{French})}
\ex[*]{\gll Jean les a aidés, mais je n'ai pas. \\
J. them has helped.\textsc{pl} but I \textsc{neg}.have \textsc{neg} \\
\glt Intended: `Jean helped them, but I did not.' \hfill (\ili{French})}
\end{xlist}

\ex
\begin{xlist}\label{3}
\ex[]{ \gll Jan heeft ze geholpen, maar Maria mocht niet. \\
J. has them helped but M. could  not \\
\glt `Jan helped them, but Maria could not.' \hfill (\ili{Dutch})}
\ex[*]{ \gll Jan heeft ze geholpen, maar Ik heb niet. \\
J. has   them helped but  I  have not\\
\glt Intended:  `Jan helped them, but I did not.' \hfill (\ili{Dutch})}
\end{xlist}
\end{exe}

\largerpage
\noindent Interestingly, \ili{Czech} behaves differently from both \ili{English}-like and \ili{Romance}-like languages in that: (i) \isi{ellipsis} is only partially available after auxiliary verbs, compare \REF{czech4b} with the past auxiliary and \REF{czech4c} with the \isi{future auxiliary}, and (ii) \isi{ellipsis} after \isi{modal} verbs in \REF{czech4a} does not behave entirely like either VP \isi{ellipsis} in \ili{English} or like \isi{MCE} in \ili{French} or \ili{Dutch}, as we will see in detail in \sectref{sec:4}.\footnote{See \sectref{sec:3.1} for different forms of the auxiliary \isi{verb} \textit{být} `be' in \ili{Czech}.} While the possibility of \isi{ellipsis} after auxiliary verbs in \ili{Czech} can be claimed to depend on their morphosyntactic status (see \sectref{sec:3.1}), I will argue that the mixed properties of \ili{Czech} \isi{MCE} follow from both the properties of \isi{modal} verbs (see  \sectref{sec:3.2}) and the structure targeted by \isi{ellipsis}. Adopting a deletion approach to \isi{ellipsis}, I will propose that we can account for (not only) \ili{Czech} \isi{MCE} by parametrizing the syntactic properties of a presumably universal \isi{ellipsis feature} [E], initially proposed by \cite{Lobeck1995} and formalized in \cite{Merchant2001}, which determines the \isi{licensing head} and the selection of the \isi{ellipsis site} in each language.

\newpage
\begin{exe}
\ex
\begin{xlist}
\ex[]{\label{czech4a} \gll Jan jim   pomohl, ale Marie bohužel nemohla.  \\
 J.    them.\textsc{dat} helped   but Marie unfortunately \textsc{neg}.could\\
\glt    `Jan helped them, but unfortunately Marie could not.' \hfill (\ili{Czech})}
\ex[*]{\label{czech4b} \gll  Jan jim  pomohl, ale já bohužel nejsem. \\
 J.   them.\textsc{dat} helped   but I  unfortunately \textsc{neg}.\textsc{aux}.\textsc{1sg} \\
\glt Intended: `Jan helped them, but unfortunately I did not.' \hfill (\ili{Czech})}
\ex[]{\label{czech4c} \gll	Jan jim  bude  pomáhat, ale  Marie bohužel  nebude. \\
J. them.\textsc{dat} \textsc{aux}.\textsc{3sg} help  but Marie unfortunately \textsc{neg}.\textsc{aux}.\textsc{3sg}\\
\glt   `Jan will help them, but unfortunately Marie won't.' \hfill (\ili{Czech})}
\end{xlist}
\end{exe}

\noindent The paper is organized as follows. \sectref{sec:2} briefly presents the main evidence for a deletion approach to \isi{MCE}. \sectref{sec:3} presents the properties of auxiliary and \isi{modal} verbs in \ili{Czech}. I discuss here \isi{ellipsis} after auxiliary verbs and argue that \isi{modal} verbs behave syntactically neither like T nor like V heads. \sectref{sec:4} focuses on the properties of \isi{MCE} in \ili{Czech} in comparison with \ili{English}, \ili{French} and \ili{Dutch}. I show that \ili{Czech} \isi{MCE} resembles \ili{English} VP \isi{ellipsis} in that it allows for various extractions from the \isi{ellipsis site} and for different subjects in \textsc{antecedent-contained deletion} (\isi{ACD}) constructions. By contrast, \ili{Czech} \isi{MCE} is similar to \ili{French} and \ili{Dutch} in that it does not allow for intervening elements between the \isi{modal} \isi{verb} and the \isi{ellipsis site} and it requires \isi{voice identity} of the elided phrase and its antecedent. \sectref{sec:5} proposes a syntactic analysis of this variation based on the mechanism of \isi{ellipsis} and the syntax of the feature [E], as developed in \cite{Aelbrecht2010}. \sectref{sec:6} sums up the paper.


\section{Assumptions about the syntax of ellipsis}\label{sec:2}
\largerpage
There are two main approaches to \isi{ellipsis} in the literature, the deletion approach and the null-proform approach, both of which have been applied to VP \isi{ellipsis} and to \isi{MCE}. Within the first approach, \isi{ellipsis} is considered to be deletion or not spelling-out of a fully specified verbal phrase. This analysis is generally assumed for VP \isi{ellipsis} in \ili{English} after both auxiliary and \isi{modal} verbs (\citealt{ross1967constraints,Sag1976,Hankamer-Sag1976,Merchant2001,Merchant2008a}, a.o.), see \REF{5a}, but it has also been recently argued for \ili{Dutch} \citep{Aelbrecht2008,Aelbrecht2010} and \ili{Romance} \citep{Dagnac2010}. The second type of analysis sees \isi{ellipsis} as involving a null verbal proform, so-called null-complement anaphora, represented by \textit{e} in \REF{5b}. This analysis has been in particular proposed by \cite{Depiante2001} for \ili{Spanish} and \ili{Italian}, and by \cite{Lobeck1995} for \ili{Dutch}.

\begin{exe}
\ex
\begin{xlist}
\ex\label{5a} You can help me, but she can't [\textsubscript{VP} \sout{help me}].
\ex \label{5b}
\gll Je    kan me wel  helpen, maar ze kan niet [\textsubscript{VP} \textit{e}].  \\
 you can me \textsc{prt}  help  but  she can not \\
\glt `You can help me, but she cannot.' \hfill (\ili{Dutch})
\end{xlist}
\end{exe}

\noindent The main argument in favour of a deletion approach that I will adopt here consists in the possibility of extraction from the \isi{ellipsis site}. Extraction of the wh-object from the elided VP is possible in \ili{English}, while it seems impossible in \ili{Dutch} and \ili{Spanish}, compare \REF{6a} and \REF{6b}--\REF{6c}. However, even if \ili{Dutch} does not behave exactly like \ili{English}, \cite{Aelbrecht2008,Aelbrecht2010} shows that at least subject extraction from the elided VP in \ili{Dutch} in \REF{7} is possible, contrary to object extraction. She argues that this is because \isi{MCE} in \ili{Dutch} targets a larger string than VP \isi{ellipsis} in \ili{English}, namely \isi{VoiceP}, which constitutes a phase blocking the object extraction (i.e. when the \isi{licensing head} is merged, the \isi{ellipsis site} is sent to \textsc{phonetic form} (\isi{PF}) and the site is frozen for extraction).

\begin{exe}
\ex
\begin{xlist}
\ex[]{\label{6a} I don't know who Sue invited, but I know who she couldn't invite.}
\ex[*]{\label{6b}
\gll Ik weet  wie   Katrien moet uitnodigen maar ik weet niet wie  ze  moet niet.\\
I know who K. must invite but I know not who she must not \\
\glt Intended: `I know whom Katrien must invite, but I don't know whom she must not.' \hfill (\ili{Dutch}; \citealt{Aelbrecht2008})}
\ex[*]{\label{6c}
\gll José no  sabe qué libro  Maria quiere leer, pero Pedro sabe qué revisto Anna no pudo.\\
J. not knows which book M. wants read but P. knows which revue A. not can\\
\glt Intended: `José doesn't know which book Maria wants to read, but Pedro knows which revue Anna can't.' \hfill (\ili{Spanish}; \citealt{Depiante2001})}
\end{xlist}
\ex  \label{7}
\begin{xlist}
\ex
\gll Deze broek moet niet gewassen worden, maar die rok moet wel.\\
this pants must not  washed become but that skirt must \textsc{prt} \\
\glt  `These pants don't need to be washed, but that skirt does.' \\ \hfill (\ili{Dutch}; \citealt{Aelbrecht2010})
\ex
{\ldots} maar die rok$_{i}$ moet wel [\textsubscript{TP} t$_{i}$ [\textsubscript{VoiceP} t$_{i}$ \sout{gewassen worden}]]
\end{xlist}
\end{exe}

\noindent Likewise, \cite{Dagnac2010} shows that even object extraction is possible in \ili{Romance} if subjects in both clauses are identical, as in \REF{8} and \REF{9}. She calls this constraint The Same Subject Constraint. Assuming that \isi{modal} verbs in \ili{Romance} are raising verbs selecting a TP, Dagnac argues that the \isi{ellipsis} after \isi{modal} verbs in \ili{Romance} is not a VP deletion but a TP deletion. This allows to explain e.g. why \isi{ellipsis} in \isi{ACD} constructions requires subjects of both elided TP and its TP antecedent to be identical.

\begin{exe}
\ex \label{8}
\begin{xlist}
\ex \label{8a}\gll Maintenant, je sais à qui je peux confier mon fils, mais je ne sais toujours pas  à qui je ne peux pas. \\
now I know to who I can entrust my son but I \textsc{neg} know still \textsc{neg} to who I \textsc{neg} can \textsc{neg} \\
\glt `Now I know to whom I can entrust my son, but I still don't know to whom I can't.'	\hfill (\ili{French}; \citealt{Dagnac2010})
\ex {\ldots}{} à qui$_{i}$ je$_{j}$  ne peux pas [\textsubscript{TP} t$_{j}$ [\textsubscript{\textit{v}P} t$_{j}$  \sout{confier mon fils} t$_{i}$]]
\end{xlist}

\ex \label{9}
\gll Ahora, ya sé a quién puedo confiar mi hijo, pero todavía no sé a quién no puedo.\\
now already know.\textsc{1sg} to who can.\textsc{1sg} confide my son but still \textsc{neg} know.\textsc{1sg} to who not can.\textsc{1sg}\\
\glt `Now I know to whom I can confide my son, but I still don't know to whom I can't.' \hfill (\ili{Spanish}; \citealt{Dagnac2010})
\end{exe}

\noindent In addition, an overt \isi{pronoun} and the verbal anaphor le faire `do it' are ungrammatical with wh-extraction and \isi{relativization} from the VP in \ili{French}, see \REF{10a} and \REF{10b} respectively \citep{Dagnac2008}.\footnote{\ili{French} \isi{modal} verbs may combine with an overt \isi{pronoun} in contexts without extraction. Here, the \isi{pronoun} can be analysed as a pronominalization of the overt TP complement of the \isi{modal} \isi{verb}. These contexts thus constitute arguments neither for deletion, nor for null anaphora. See also \sectref{sec:4.1}.
\ea \gll Jean peut te répondre, mais moi, je ne (\hspace{-2pt} le) peux pas.\\
J. can  you answer  but me  I  \textsc{neg} {} it can  \textsc{neg} \\
\glt `Jean can answer you, but I can not.' \hfill (\ili{French})\z
\ea \gll Je vais résister aussi longtemps que je (\hspace{-2pt} le) peux.\\
I   will resist as   long that  I   {} it  can\\
\glt `I will resist as long as I can.' \hfill (\ili{French})\z\label{fn:3}} This also supports the claim that there is a movement from an elided structure, which cannot be reduced to a null \isi{pronoun}.

\begin{exe}
\ex \label{10}
\begin{xlist}
\ex \label{10a}
\gll Il embrasse \minsp{\{} qui il peut / *\hspace{-2pt} qui  il  peut  le faire / *\hspace{-2pt} qui  il   le peut\}.\\
he kisses {} who he can  {}  {} who he can it do {}  {} who he it can \\
\glt `He kisses who he can.' \hfill (\ili{French}; \citealt{Dagnac2008})
\ex \label{10b}
\gll  Léa lit tous les livres  \minsp{\{} qu'elle peut / *\hspace{-2pt} qu'elle  peut le faire\hspace{20pt}  / *\hspace{-2pt} qu'elle le peut\}. \\
 Lea reads all the books {} {that.she} can {} {} {that.she} can it do {} {} {that.she} it can {}\\
\glt `Léa reads all the books she can.' \hfill (\ili{French}; \citealt{Dagnac2008})
\end{xlist}
\end{exe}


\noindent Dagnac's and Aelbrecht's arguments thus make it very reasonable to assume that there is an underlying syntactic structure in contexts involving \isi{MCE}, but they also suggest that we need to specify for each language:
\begin{enumerate}
    \item the type of head \isi{licensing ellipsis},
    \item the \isi{ellipsis site}.
\end{enumerate}
I will propose in \sectref{sec:5} that these two micro-parameters can be encoded in the syntax of the \isi{ellipsis feature} [E] responsible for the distribution of \isi{ellipsis} throughout languages.


\section{Auxiliary and modal verbs in Czech}\label{sec:3}


\subsection{The auxiliary \textit{být}}\label{sec:3.1}

\ili{Czech} is a West-Slavic language with a rich morphology in both its nominal system (number, gender, case) and its verbal system (\isi{tense}, voice, aspect). It also differs from \ili{English}, \ili{French} and \ili{Dutch} in that (i) it is a subject pro-drop language, (ii) it has \isi{second position} clitics (\isi{2PCl}) including \isi{pronominal} and verbal (auxiliary) clitics, and (iii) it has -- despite its basic SVO order -- a relatively free word order that reflects the information structure of the clause. Like many other languages, it shows various elliptical constructions, such as gapping and sluicing \citep{Gruet-Skrabalova2016}.

\ili{Czech} uses only the auxiliary \isi{verb} \textit{být} `be', in three series of forms: (i) forms in \textit{js-} in the past \isi{tense}, (ii) forms in \textit{by-} in the conditional mood, (iii) forms in \textit{bud-} in the future \isi{tense}. Past and conditional forms are \isi{2PCl}, which combine with lexical \textit{-l} participles in the \isi{active voice} and with the (non-\isi{clitic}) \isi{passive auxiliary} \textit{byl} `been' and lexical \textit{-n} participles in the \isi{passive voice}, see \REF{11a}--\REF{11b}.\footnote{\ili{Czech} morphologically distinguishes active \textit{-l} participles and passive \textit{-n} participles. The former are considered as tensed forms (see \citealt{Veselovská1995,Veselovská2008}), which may bear \isi{sentential negation} \textit{ne-}.} The future forms are autonomous words and combine with non-finite \isi{imperfective} verbs in the \isi{active voice} and with \textit{-n} participles in the \isi{passive voice}, see \REF{11c}--\REF{11d}. \cite{Veselovská1995,Veselovská2008} argues that \ili{Czech} \isi{clitic} auxiliaries bear agreement, but not \isi{tense} features, and thus that they are generated above T (cf. \citealt{Roberts2010}, who places \ili{Slavic} 2PCL in the C-system). By contrast, the non \isi{clitic} \isi{future auxiliary} should be merged within the extended VP since it is sensitive to aspect on the lexical \isi{verb} (cf. \citealt{Kyncl2008}), as shown in \REF{11c}. For the purpose of this paper, I assume that \isi{2PCl} auxiliaries are generated in a low C-head, while non \isi{clitic} future and passive auxiliaries are generated below T (Aspect and \isi{Voice} respectively, see \citealt{Cinque2004}), as indicated in \REF{12}. I also assume the finite auxiliaries move to T to check their T-feature.

\begin{exe}
\ex	\label{11}
\begin{xlist}
\ex\label{11a} \gll Já jsem \minsp{\{} (\hspace{-2pt} ne-) četl / (\hspace{-2pt} ne-) přečetl\} všechny tyhle knihy.\\
I  \textsc{past}.\textsc{1sg} {} {} \textsc{neg} read.\textsc{impf}  {} {} \textsc{neg} \textsc{pf}.read all these books\\
\glt  `I (have) (not) read all these books.' \hfill (\ili{Czech})
\ex\label{11b} \gll Já jsem  (\hspace{-2pt} ne-) byl pozván.\\
I  \textsc{past}.\textsc{1sg} {} \textsc{neg} been invited.\textsc{pass} \\
\glt `I was (not) invited.' \hfill (\ili{Czech})
\ex\label{11c}\gll Já budu \minsp{\{} číst / *\hspace{-2pt} přečíst\} všechny tyhle knihy.\\
I \textsc{fut}.\textsc{1sg} {} read.\textsc{impf} {} {} \textsc{pf}.read all these books\\
\glt   `I will read all these books.'  \hfill (\ili{Czech})
\ex\label{11d}\gll Já (\hspace{-2pt} ne-) budu  pozván.\\
I  {} \textsc{neg} \textsc{fut}.\textsc{1sg} invited.\textsc{pass}\\
\glt `I will (not) be invited.'  \hfill (\ili{Czech})
\end{xlist}

\ex \label{12} {[\textsubscript{CP} {\ldots} CL [\textsubscript{TP} {\ldots} [\textsubscript{AspP} bud- [\textsubscript{VoiceP}  byl [\textsubscript{VP} {\ldots}]]]]]}
\end{exe}

\noindent As has already been shown in \sectref{sec:1}, \isi{ellipsis} is not available with \isi{clitic} auxiliaries, see \REF{13b}. \cite{Gruet-Skrabalova2012} argues that this follows precisely from their \isi{clitic} status: 2P clitics cannot license VP \isi{ellipsis} because they appear too high in the structure with respect to the VP domain. I will return to the analysis of \isi{ellipsis} after the \isi{future auxiliary} as in \REF{13a} in \sectref{sec:5.2}.

\ea\label{13}
\ea[]{\label{13a} \gll Já  budu  číst  nahlas,  a ty budeš   taky.\\
 I \textsc{fut}.\textsc{1sg} read aloud and you \textsc{fut}.\textsc{2sg} too \\
\glt `I will read aloud, and you will too.'\hfill (\ili{Czech})}
\ex[*]{\label{13b}\gll Já \minsp{\{} jsem  / bych\} četl  nahlas,  a ty \minsp{\{} jsi  / bys\} taky. \\
I {} \textsc{past}.\textsc{1sg} {} \textsc{cond}.\textsc{1sg} read aloud  and you {} \textsc{past}.\textsc{2sg} {} \textsc{cond}.\textsc{2sg} too \\
\glt Intended: `I read aloud, and you did too.' / `I would read aloud, and you would too.'\hfill (\ili{Czech})}
\z\z


\subsection{Modal verbs} \label{sec:3.2}
There are five strictly \isi{modal} verbs in \ili{Czech}: \textit{moci/moct} `can/be able to', \textit{smět} `may/be allowed to', \textit{muset} `must/have to', \textit{nemuset} `need not', and \textit{mít} `have to'. These verbs have mixed morphosyntactic properties, as shown in \cite{Kyncl2008}: like functional verbs, they have no imperative (\textit{*mus}) and no passive (\textit{*musen, *mocen}) and they do not combine with \isi{aspectual} affixes (*\textit{domuset, *musívat}). They are not sensitive to the \isi{aspectual} makeup of the lexical verbs either, contrary to the \isi{future auxiliary} \textit{budu} requiring \isi{imperfective} verbs in \REF{11c} above. Like lexical verbs, \isi{modal} verbs combine with the auxiliary \textit{být} `be', see \REF{14b}--\REF{14c}, and bear the prefix \textit{ne-} expressing \isi{sentential negation} when they are finite, as we can see in \REF{14a}--\REF{14b}. They can be followed by active or passive infinitival verbs, see \REF{14d}.

\begin{exe}
\ex \label{14}
\begin{xlist}
\ex \label{14a}\gll Nemůžu přece \minsp{\{} přečíst  / číst\} všechny tyhle knihy. \\
\textsc{neg}.can.\textsc{1sg} though {} \textsc{pf}.read {}  read.\textsc{impf} all these books \\
\glt `I cannot read all these books.'  \hfill (\ili{Czech})
\ex \label{14b}\gll Nemohl jsem přece \minsp{\{} přečíst / číst\} všechny tyhle knihy.\\
\textsc{neg}.could  \textsc{past}.\textsc{1sg} though {} \textsc{pf}.read {}  read.\textsc{impf} all  these books\\
\glt `I could not read all these books.'  \hfill (\ili{Czech})
\ex \label{14c}\gll Nikdy  nebudu moci  \minsp{\{} přečíst  / číst\} všechny tyhle  knihy. \\
never \textsc{neg}.\textsc{fut}.\textsc{1sg} can {} \textsc{pf}.read {} read.\textsc{impf} all these books \\
\glt `I will not be able to read all these books.'   \hfill (\ili{Czech})
\ex \label{14d} \gll Já budu muset být pozván.\\
I   \textsc{fut}.\textsc{1sg}  must  be   invited.\textsc{pass} \\
\glt  `I will have to be invited.'   \hfill (\ili{Czech})
\end{xlist}
\end{exe}

\noindent Although \isi{modal} verbs can occur with auxiliary verbs, \isi{modal} verbs cannot co-occur, just like in \ili{English}, and contrary to \ili{Romance} or \ili{Dutch}, see \REF{15}. The co-occurrence of \ili{French} and \ili{Dutch} \isi{modal} verbs in \REF{15c} and \REF{15d} respectively can be explained if we assume, as has been argued in the literature \citep{Ruwet1972,Wurmbrand1999,Wurmbrand2001}, that they are raising verbs selecting not a VP, but a TP complement.\footnote{Traditionally (e.g. \citealt{Ross1969a}), \isi{deontic} verbs have been claimed to be control predicates and \isi{epistemic} verbs to be raising predicates. For \cite{Wurmbrand1999}, however, this semantic difference is not represented in syntax.}

\ea\label{15}
\ea[*]{\label{15a}
\gll Já musím  moci  přečíst  ty      knihy.\\
 I   must     can    \textsc{pf}.read  these books\\
\glt Intended: `I must be able to read these books.'\hfill (\ili{Czech})}
\ex[*]{You must can read these books.}
\ex[]{\label{15c}
\gll Vous devez pouvoir lire  ces    livres.\\
 you   must   can       read these books\\
 \glt `You must be able to read these books.'  \hfill (\ili{French})}
\ex[]{\label{15d}
\gll  Hij moet goed kunnen koken.\\
he  must well  can        cook \\
\glt  `He must be able to cook well.'\hfill (\ili{Dutch}; \citealt{Lobeck1995})}
\z\z

\noindent \ili{Czech} \isi{modal} verbs also systematically require climbing of \isi{pronominal} \isi{clitic} complements of the lexical \isi{verb}, and thus behave obligatorily like restructuring verbs \citep{Medová2000}. Although \isi{clitic} climbing is typical for \ili{Romance} languages (e.g. \citealt{Rizzi1978,Roberts1991}), it is no longer true for \isi{modal} verbs in \ili{French}.

\begin{exe}
\ex \label{16}
\begin{xlist}
\ex\label{16a}
\gll Petr musí přečíst  ty  knihy.  / Petr (\hspace{-2pt} je) musí (*\hspace{-2pt} je)   přečíst. \\
P.    must \textsc{pf}.read these books   {} P.  {} \textsc{cl}:them must {} \textsc{cl}:them \textsc{pf}.read\\
\glt   `Petr must read these books / them.' \hfill (\ili{Czech})
\ex \label{16b}
\gll  Pierre doit  lire   ces livres.  /  Pierre (*\hspace{-2pt} les)  doit  (\hspace{-2pt} les)  lire.\\
P.  must read these books  {} P.   {} \textsc{cl}:them must {} \textsc{cl}:them read\\
\glt `Pierre must read these books / them.' \hfill (\ili{French})
\end{xlist}
\end{exe}

\noindent The properties of \ili{Czech} \isi{modal} verbs discussed above suggest that they behave neither like T heads, as in \ili{English} (\citealt{Sag1976} a.o.), nor like V heads, as in \ili{French} or \ili{Dutch} \citep{Aelbrecht2008,Dagnac2010}. Given their restructuring properties, I assume that \isi{modal} verbs are heads of a specific \isi{functional projection} ModP between V and T (cf. \citealt{Cinque2004}) selecting an extended VP projection as complement.

As already said, \isi{ellipsis} is available after \isi{modal} verbs, as shown in \REF{17a}. Interestingly,  \isi{ellipsis} may occur even if the \isi{modal} \isi{verb} follows the \isi{future auxiliary} and is therefore non-finite, as in \REF{17b}. This suggests that \isi{MCE} is not licensed by the head T. Note that the \isi{modal} \isi{verb} does not appear in the first sentence and thus constitutes new information in the \isi{elliptical clause}.

\begin{exe}
\ex \label{17}
\begin{xlist}
\ex \label{17a}\gll  I když  já čtu nahlas, ty    nemusíš.\\
 even if I read.\textsc{1sg} aloud   you \textsc{neg}.must.\textsc{2sg} \\
\glt `Although I read aloud, you do not have to.' \hfill (\ili{Czech})
\ex \label{17b} \gll I když   já budu číst nahlas,  ty    nebudeš  muset.\\
even if  I   \textsc{fut}.\textsc{1sg} read aloud you \textsc{neg}.\textsc{fut}.\textsc{2sg} must\\
\glt `Although I will have to read aloud, you will not have to.' \hfill (\ili{Czech})
\end{xlist}
\end{exe}

\section{Modal complement ellipsis in Czech} \label{sec:4}

\subsection{English-like properties}\label{sec:4.1}

\sloppy This section focuses on \ili{Czech} \isi{MCE} in comparison with \ili{English}, \ili{Dutch} and \ili{French}. We will see that \ili{Czech} \isi{MCE} looks like \ili{English} VP \isi{ellipsis} with respect to extraction and subjects in \isi{ACD} constructions, and like \ili{French} and \ili{Dutch} \isi{MCE} with respect to the size of the elided string and voice properties of the elided VP and its antecedent.

Starting with non-elliptical constructions, we can see that the verbs `can' and `must' in the languages under discussion can have two readings, a \isi{deontic} (root) reading and an \isi{epistemic} reading. \ili{Czech} is similar to \ili{English} in that both readings are also acceptable in elliptical constructions, although \isi{ellipsis} appears most frequently with the \isi{deontic reading}. In contrast, it has been observed for \ili{Romance} and \ili{Dutch} that \isi{MCE} is only available with \isi{deontic reading} of these \isi{modal} verbs:\footnote{Cf. \cite{Barbiers1995,Lobeck1995,Aelbrecht2008}. For my informants, the \isi{verb} \textit{moeten} ‘must’ would be ruled out in \REF{18d}, the \isi{verb} \textit{hoefen} ‘should’ being more acceptable.}

\begin{exe}
\ex	 {Deontic reading}\label{18}
\begin{xlist}
\ex\label{18a} John couldn't come to the party, and Peter was not allowed.
\ex \label{18b}
\gll  Jan na večírek přijít  nemohl    a    Petr  nesměl.\\
J.    to  party    come \textsc{neg}.could  and P.     \textsc{neg}.could\\
\glt `Jan was not able to come to the party, and Petr was not allowed.' \\ { } \hfill (\ili{Czech})
\ex \label{18c}
\gll  Jean a   pu venir  à  la soirée,  mais Pierre n'a  pas  pu. \\
J.  has could come to the party  but   P. {\textsc{neg}.has} \textsc{neg} could\\
\glt `Jean could come to the party, but Pierre couldn't.' \hfill  (\ili{French})
\ex\label{18d}
\gll  Jan kon    naar het feest kommen, maar Piet mocht niet. \\
 J.   could to     the party come      but     P.    could   not\\
 \glt `Jean could come to the party, but Piet was not allowed.' \hfill (\ili{Dutch})
\end{xlist}


\ex  {Epistemic reading}\label{19}
\begin{xlist}
\ex[]{\label{19a}	It can be true, but it doesn't have to.}
\ex[]{\label{19b}
\gll  Může to být pravda, ale nemusí. \\
can    it  be true        but \textsc{neg}.must	\\
\glt `It can be true, but it doesn't have to.'  \hfill (\ili{Czech})}
\ex[*]{\label{19c}
\gll Cela peut être vrai, mais cela ne    doit  pas. \\
this   can  be    true  but   this \textsc{neg} must \textsc{neg}\\
\glt Intended: `It can be true, but it doesn't have to.'  \hfill (\ili{French})}
\ex[?]{\label{19d}
\gll Het zou       waar kunnen zijn, maar het hoeft niet. \\
it  should  true   can be    but     it   should not\\
\glt `It should be true, but it shouldn't have to.' \hfill (\ili{Dutch})}
\end{xlist}
\end{exe}

\noindent Another property \ili{Czech} shares with \ili{English} concerns the possibility of pronominalizing the elided string. Actually, missing material after \isi{modal} verbs in \ili{Czech} is not in complementary distribution with an overt \isi{pronoun}, as shown in \REF{20}. \ili{French} and \ili{Dutch} behave differently except for contexts with extraction like in \REF{21}; see \sectref{sec:2}, footnote \ref{fn:3}. This might be not completely surprising if both pronominalization and \isi{MCE} in these languages target a TP, as proposed by \citet{Dagnac2010}; see \sectref{sec:5.1}.


\begin{exe}
\ex \label{20}
\begin{xlist}
\ex \label{20a} John will answer you, but I can't (*it).
\ex \label{20b}
\gll Jan  ti    odpoví, ale  já (*\hspace{-2pt} to) nemůžu.\\
J.    you.\textsc{dat} answers but I       {} it   \textsc{neg}.can\\
\glt `Jan will answer you, but I can't.' \hfill (\ili{Czech})
\ex \label{20c}
\gll   Jean te    répondra,  mais moi, je ne  (\hspace{-2pt} le) peux pas. \\
J.  you answer.\textsc{fut}.\textsc{3sg} but me I \textsc{neg} {} it can \textsc{neg}\\
\glt `Jean will answer you, but I can't.' \hfill (\ili{French})
\ex \label{20d}
\gll  Jan zal je antwoorden, maar ik kan (\hspace{-2pt} het) niet.\\
J. will you answer but I can {} it not\\
\glt `Jan will answer you, but I can't.' \hfill (\ili{Dutch})
\end{xlist}
\ex \label{21}
\begin{xlist}
\ex \label{21a}
\gll  Jean  lit tous les livres  qu'il   (*\hspace{-2pt} le) peut.\\
J.  reads all  the books {that he}   {} it   can \\
\glt `Jean reads all the books he can.' \hfill (\ili{French})
\ex \label{21b}
\gll  Joris leest  elk  boek dat hij (*\hspace{-2pt} het) kan.\\
J.   reads every book that he     {} it    can \\
\glt `Joris reads every book he can.' \hfill (\ili{Dutch})
\end{xlist}
\end{exe}

\largerpage[-3]
\noindent Elliptical \isi{relative} clauses, so-called antecedent contained deletion (\isi{ACD}), display another property in which languages may differ. In \ili{Czech} and \ili{English}, a \isi{relative} clause containing \isi{ellipsis} and its matrix clause may have different subjects. In contrast, \ili{Romance} and \ili{Dutch} require both subjects to be identical (the Same Subject Constraint, see \sectref{sec:2}):\footnote{Here is an attested and more natural example for \REF{22b}:
\ea
\gll Helenku  bolelo břicho a tak  Elizabetka snědla všechny bonbóny který Helenka nemohla.\\
Helenka.\textsc{acc} ached stomach and so Elizabetka ate all sweets that   Helenka \textsc{neg}.could \\
\glt ‘Helenka had a stomach ache, so Elizabetka ate all sweets that Helenka could not.’\\\xspace\hfill (\ili{Czech})\z}

\begin{exe}
\ex \label{22}
\begin{xlist}
\ex[]{\label{22a} John reads all the books that Mary can't.}
\ex[]{\label{22b}
\gll  Jan čte    všechny knihy, které Marie nesmí.\\
  J.    reads all        books   that   M.     \textsc{neg}.can\\
\glt `Jan reads all books that Marie is not allowed to read.' \hfill (\ili{Czech})}
\ex[*]{\label{22c}
\gll Jean lit  tous les livres  que  Marie ne    peut pas.\\
 J.  reads all   the books that  M.      \textsc{neg} can  \textsc{neg}\\
\glt Intended: `Jean reads all the books that Marie can't.' \\ \hfill (\ili{French}; \citealt{Dagnac2008})}
\ex[*]{\label{22d}
\gll Joris leest  elk     boek  dat Monika moet niet. \\
  J. reads every book that M.  must not\\
\glt Intended: `Joris reads every book that Monika doesn't have to.'\\ \hfill (\ili{Dutch}; \citealt{Aelbrecht2008})}
\end{xlist}
\end{exe}

\noindent Finally, recall that the main argument for the deletion approach of \isi{MCE} is based on extraction from the elided string. In \ili{Czech}, several types of extraction are possible, to both A- and A$'$- positions. Extraction to \isi{subject position} can be seen with the inaccusative \isi{verb} přijít `come' in the example \REF{16a} above, repeated in \REF{23}.

\begin{exe}
\ex \label{23}\textit{Extraction to \isi{subject position}:}\smallskip\\
\gll Jan na večírek přijít  nemohl a Petr nesměl.\\
J.    to  party    come \textsc{neg}.could  and P.    \textsc{neg}.could\\
\glt `Jan could't come to the party and Petr was not allowed.' \hfill (\ili{Czech})
\end{exe}

\noindent The example \REF{24} shows regular wh-object extraction of the \isi{dative} wh-word \textit{komu} `to whom' to the CP domain (cf. \sectref{sec:2}, ex. \REF{7}), and the example \REF{25} shows topicalization of an \isi{accusative} DP-object. It must, however, be noted that extraction from VP in \ili{English}, like in \REF{25a}, usually requires a specific contrastive \isi{focus} \citep{Schuyler2001,Lasnik2001,Merchant2008b}. This is also true for \ili{French}, where topicalization is acceptable provided the context is contrastive enough. In \ili{Czech}, no specific \isi{prosody} is required to accompany extraction in the examples below. I assume that this is because the word order in \ili{Czech} is much is much freer than in \ili{English} and \ili{French} and serves to mark a specific information status of a phrase. In \ili{English}, the information \isi{focus} or topic status is generally marked by \isi{prosody}. With respect to extraction, \ili{Czech} may thus seem even more permissive than \ili{English}.

\begin{exe}
\ex \label{24} \textit{Wh-object extraction:}\smallskip\\
\gll Vím komu můžu  děti svěřit a komu nemůžu.\\
know.\textsc{1sg} who.\textsc{dat} can.\textsc{1sg} children confide and who.\textsc{dat} \textsc{neg}.can.\textsc{1sg}\\
\glt `I know to whom I can confide my children and to whom I can't.' \hfill (\ili{Czech})

\ex \label{25} \textit{Object topicalization:}\\
\begin{xlist}
\ex[]{\label{25a} The toy she had already bought, but the book she really couldn't.}
\ex[]{\label{25b}
\gll Hračky už jsem koupila, ale  knížky  jsem  ještě nemohla. \\
toys.\textsc{acc} already \textsc{past}.\textsc{1sg} bought but books.\textsc{acc} \textsc{past}.\textsc{1sg} yet \textsc{neg}.could\\
\glt `I have already bought the/some toys, but I was not yet able to buy the/some books.' \hfill (\ili{Czech})}
\ex[]{\label{25c}
\gll Les jouets, je les  ai déjà  achetés, mais les livres, je n'ai  pas encore pu.\\
the toys I them have already bought  but   the books I \textsc{neg}.have \textsc{neg} yet could\\
\glt `I have already bought the toys, but I haven't yet been able to buy the books.' \hfill (\ili{French})}
\ex[*]{\label{25d}
\gll Het speelgoed had ik al gekocht, maar het boekje kon ik niet.\\
the toy  have I  already bought  but    the book    could I not\\
\glt Intended: `I have already bought the toy, but I haven't yet been able to buy the book.' \hfill (\ili{Dutch})}
\end{xlist}
\end{exe}

\noindent The last example in this section presents a more questionable extraction: in \ili{English}, \REF{26a} is generally considered as a case of pseudogapping, but some also analyse it as involving movement \citep{Aelbrecht2008,Gengel2013}. Whether or not we assume that \REF{26b} involves an object extraction, \ili{Czech} would again be similar to \ili{English} rather than to \ili{French} or \ili{Dutch}.

\begin{exe}
\ex \label{26}
\begin{xlist}
\ex[]{\label{26a}I will vote for him, and you can for her.}
\ex[]{\label{26b}
\gll Já budu  volit pro něj  a ty  můžeš pro ni.\\
 I  \textsc{fut}.\textsc{1sg} vote for  him and you can for her	\\
\glt `I will vote for him and you can for her.' \hfill (\ili{Czech})}
\ex[*]{\label{26c}
\gll  Je voterai pour lui, et   tu peux   pour elle.\\
I  vote.\textsc{fut}.\textsc{1sg} for him and you can for her\\
\glt Intended: `I will vote for him and you can for her.' \\ \hfill (\ili{French}; \citealt{Dagnac2010})}
\ex[*]{\label{26d}
\gll Ik zal voor hem stemmen, en  je  kan voor haar.\\
I  will for   him vote and you can for  her\\
\glt Intended: `I will vote for him and you can for her.' \\ \hfill (\ili{Dutch}; \citealt{Aelbrecht2008})}
\end{xlist}
\end{exe}

\subsection{Differences from English} \label{sec:4.2}

Despite several similarities with \ili{English} VP \isi{ellipsis} presented in the previous subsection, \ili{Czech} \isi{MCE} is also (at least to some extent) similar to \ili{French} and \ili{Dutch} \isi{MCE}.

First, we observe that verbal elements intervening between modals and the VP must be elided, see the \isi{passive auxiliary} `be' in \REF{27}. In \ili{French} and \ili{Dutch}, these elements also include \isi{negation} and aspect morphemes, which in \ili{Czech} appear directly on the \isi{verb} stem.\footnote{See also these two examples:
\ea[*]{\gll Paul peut avoir fini en juin, et Luc peut aussi avoir [\sout{fini { } { } { } { }en juin}].\\
Paul can  have finished in June and Luc can too have {{ }finished in June}\\
\glt Intended: `Paul can have finished in June and Luc can have.' \hfill (\ili{French}; \citealt{Dagnac2008})}\z
\ea[*]{\gll Paul peut repasser LM01 et Luc  peut {ne pas} [\sout{repasser LM01}].\\
Paul can  take LM01 and Luc can not {{ }take { } { } { } LM01}\\
\glt Intended: `Paul is allowed to take LM01 and Luc is allowed not to.'\\\xspace\hfill (\ili{French}; \citealt{Dagnac2008})}\z}

\begin{exe}
\ex \label{27}
\begin{xlist}
\ex \label{27a} This text can be read aloud but this poetry really cannot (be).
\ex \label{27b}
\gll Tento text může být čten nahlas, ale tato báseň  skutečně nemůže (*\hspace{-2pt} být).\\
this text can be read.\textsc{pass} aloud but this poetry really \textsc{neg}.can {} be\\
\glt `This text can be read aloud but this poetry really cannot be.' \hfill (\ili{Czech})
\ex \label{27c}
\gll Ce texte peut être lu à voix haute, mais ce poème ne peut vraiment pas (*\hspace{-2pt} être).\\
this text can be read in voice high but this poetry \textsc{neg} can really \textsc{neg} {} be\\
\glt `This text can be read aloud but this poetry really cannot be.' \\ \hfill (\ili{French}; \citealt{Dagnac2010})
\ex \label{27d}
\gll Deze tekst kan hardop gelezen worden, maar deze poëzie kan echt niet (*\hspace{-2pt} zijn).\\
this text can aloud read be but this  poetry can  really not {} be\\
\glt `This text can be read aloud but this poetry cannot be.' \hfill (\ili{Dutch})
\end{xlist}
\end{exe}

\noindent By contrast, \isi{second position} \isi{clitic} auxiliaries are obligatory with \isi{MCE} in \ili{Czech}, see \REF{28a}, which is not surprising since they occur very high in the clause (cf. \sectref{sec:3.1}).\footnote{Contrary to \isi{MCE}, auxiliary clitics cannot escape sluicing, as has been noted by \cite{Merchant2001} in his Sluicing-Comp Generalization. This follows if \isi{MCE} targets a smaller structure than sluicing does.} Note also that \isi{second position} \isi{pronominal} clitics that are complements of the lexical \isi{verb} are excluded. These clitics normally appear on \isi{modal} verbs, see \REF{16a} above and \REF{28b}. This suggests that \isi{pronominal} clitics must be elided before \isi{clitic} climbing.\footnote{For \cite{Roberts2010}, \isi{pronominal} clitics consistently escape the interior of the low \textit{v-}cycle. If this is true, \isi{MCE} in \ili{Czech} targets a larger structure than the low \textit{v-}cycle, which is compatible with my analysis in  \sectref{sec:5}.}

\begin{exe}
\ex \label{28}
\begin{xlist}
\ex \label{28a} \gll Já jsem to musela podepsat, ale ty *(\hspace{-2pt} jsi) (*\hspace{-2pt} to) nemusel.\\
I \textsc{past}.\textsc{1sg} it.\textsc{cl} had.to sign but you {} \textsc{past}.\textsc{2sg} {} it.\textsc{cl} \textsc{neg}.had.to\\
\glt `I had to sign it, but you didn't have to.' \hfill (\ili{Czech})
\ex \label{28b} \gll Já jsem to musela  (*\hspace{-2pt} to)   podepsat (*\hspace{-2pt} to).\\
I \textsc{past}.\textsc{1sg} it.\textsc{cl} had.to  {} it.\textsc{cl}  sign  {} it.\textsc{cl} \\
\glt `I had to sign it.' \hfill (\ili{Czech})
\end{xlist}
\end{exe}


\noindent Finally, it has been pointed out \citep{Hardt1993,Merchant2008a,Merchant2013} that the voice of the elided VP and that of its VP antecedent in \ili{English} may differ. In \REF{29a}, the \isi{elliptical clause} is in the \isi{active voice}, while the clause with the VP antecedent is in the \isi{passive voice}. The example \REF{29b} shows the opposite distribution: passive \isi{elliptical clause} and active clause containing the antecedent.\footnote{Apparent counter-examples in \ili{English} are reanalysed by \cite{Merchant2013} as cases of pseudogapping.} Assuming that voice is encoded on the head \isi{Voice} and that \isi{Voice} is distinct from the head \textit{v}, \cite{Merchant2008a} argues that VP \isi{ellipsis} in \ili{English} targets a verbal phrase (\textit{v}P/VP) below \isi{Voice} head. Ellipsis therefore does not include \isi{VoiceP}.

\begin{exe}
\ex \label{29}
\begin{xlist}
\ex \label{29a}This problem had to be solved long ago, but obviously nobody could (solve it). \hfill \citep{Merchant2008a}
\ex \label{29b}The janitor must remove the trash whenever it is apparent that it should be (removed). \hfill \citep{Merchant2008a}
\end{xlist}
\end{exe}


\noindent Contrary to \ili{English}, we can see in \REF{30} that \ili{Czech}, \ili{French} and \ili{Dutch} require the same active or passive morphosyntax in both the \isi{elliptical clause} and the clause containing the VP antecedent.\footnote{\isi{Voice} identity also applies to \isi{ellipsis} after the auxiliary future \isi{verb}, see \sectref{sec:5.2}.} Ellipsis is excluded here because the \isi{elliptical clause} is presumably in the \isi{active voice}, while the clause containing the antecedent is in the \isi{passive voice}. This suggests that \isi{VoiceP} is included in the \isi{ellipsis site}.

\begin{exe}
\ex \label{30}
\begin{xlist}
\ex \label{30a}
\gll Ten problem měl být dávno vyřešen ale nikdo zřejmě nemohl *(\hspace{-2pt} ho vyřešit).\\
this problem had.to be longtime solved.\textsc{pass} but nobody obviously \textsc{neg}.could  {} it solve\\
\glt `This problem had to be solved long ago, but nobody could solve it.' \\ \hfill (\ili{Czech})
\ex \label{30b}
\gll Ce problème aurait déjà dû être résolu, mais personne n'a pu *(\hspace{-2pt} le résoudre).\\
this problem has.\textsc{cond} yet had.to be solved but nobody \textsc{neg}.has could  {} it solve\\
\glt `This problem had to be solved long ago, but nobody could solve it.' \\ \hfill (\ili{French}; \citealt{Dagnac2010})
\ex \label{30c}
\gll  Dit probleem had al lang geleden opgelost moeten worden maar niemand  kon *(\hspace{-2pt} het opgelessen). \\
 this problem  had already long ago solved must be but nobody could {} it solve\\
\glt `This problem had to be solved long ago, but nobody could solve it.' \\  \hfill (\ili{Dutch}) \\
\end{xlist}
\end{exe}

\noindent Given the properties discussed above, I conclude that \ili{Czech} \isi{MCE} seems to target a larger structure that VP \isi{ellipsis} in \ili{English}, but probably a smaller structure than \isi{MCE} in \ili{French} or \ili{Dutch}.

\subsection{Summary} \label{sec:4.3}

\tabref{tab:1:properties} summarizes the properties discussed in this section. Rows 1 to 3 indicate for each language whether it allows \isi{ellipsis} after auxiliary verbs (Aux + \isi{ellipsis}), co-occurrence of an auxiliary and a \isi{modal} \isi{verb} (T + \isi{Mod}), and co-occurrence of two \isi{modal} verbs (\isi{Mod} + \isi{Mod}). With respect to \isi{MCE}, rows 4 to 8 indicate whether it is compatible with \isi{deontic} and \isi{epistemic} reading of \isi{modal} verbs (Deont/\isi{Epist} reading), with subject extraction, object topicalization, wh-object extraction and object scrambling (if there is any). Finally, rows 9 to 12 show whether \isi{MCE} requires identical subjects in \isi{ACD} constructions (Same Subject Constraint) and identical voice on the \isi{verb} (\isi{Voice} identity) and whether it allows a \isi{passive auxiliary} to occur after the \isi{modal} \isi{verb} (Passive Aux). We can see that \ili{Czech} shares most but not all the examined properties with \ili{English}.

\begin{table}
\caption{Properties related to MCE}
\label{tab:1:properties}
 \begin{tabularx}{\textwidth}{rlXXXX}\lsptoprule
    & &  {English} &  {Czech} &  {French} &  {Dutch}\\ \midrule
1 & Aux + \isi{ellipsis} & Yes & Yes/No (\isi{2PCl}) & No & No \\
2 & T + \isi{Mod} & No & Yes & Yes & Yes\\
3 & \isi{Mod} + \isi{Mod} & No & No & Yes & Yes\\
4 & Deont/\isi{Epist} reading & Yes/Yes & Yes/Yes & Yes/No & Yes/No?\\
5 & Subject extraction & Yes & Yes & Yes & Yes\\
6 & Object topicalization & Yes & Yes & Yes & No\\
7 & Wh-object extraction & Yes & Yes & Yes & No\\
8 & Object scrambling (?) & Yes & Yes & No & No\\
9 & Same Subject Constraint & No & No & Yes & Yes\\
10 & Overt \isi{pronoun} & No & No & Yes/No & Yes/No\\
11 & \isi{Voice} identity & No & Yes & Yes & Yes\\
12 & Passive Aux & Yes & No & No & No\\ \lspbottomrule
\end{tabularx}
\end{table}

\section{Proposal} \label{sec:5}

Before proposing an analysis of \isi{MCE} in \ili{Czech}, I present here my assumptions about the general mechanism \isi{licensing ellipsis}, following in particular \cite{Aelbrecht2010}, and to some extent \cite{Lobeck1995}, \cite{Merchant2001} and \cite{Craenenbroeck-Lipták2013}. I thus assume that (i) \isi{ellipsis} is triggered in a checking relation (Agree) between the \isi{licensing head} X and the \isi{ellipsis site} YP, (ii) there is a feature [E] that occurs on the head of Y and indicates that YP will not be spelled out (non-pronunciation at \isi{PF}) once the feature is checked out by the head X.\footnote{This assumption is at variance with \cite{Merchant2001}, for whom it is the complement of the head bearing [E] that is elided.}

The feature [E] has a specific syntax consisting of two properties:
(i) selection of the head on which the feature may occur, i.e. the head of the constituent that will be elided (\textsc{sel} X), (ii) uninterpretable features that must be checked against the features of the head \isi{licensing ellipsis} (uY). I will propose that parametrizing these two properties accounts for the behaviour of (not only) \ili{Czech} \isi{MCE}. The elided YP must be given \citep{Barbiers1995,Lobeck1995,Merchant2001} and syntactically/structurally isomorphic with its antecedent (cf. \citealt{fiengo1994indices}).

\subsection{Specifying the properties of the [E]-feature in MCE} \label{sec:5.1}

We have seen in \sectref{sec:3} that the behaviour of \isi{modal} verbs in \ili{Czech} suggests that they are neither T nor V head, but rather head of a specific \isi{functional projection} between T and V, which I call \isi{Mod}. They can therefore co-occur with T but not with other \isi{modal} verbs.

To account for \isi{ellipsis} licensed in the contexts of \isi{modal} verbs in \ili{Czech}, I propose in \REF{31a} that the feature [E] is merged on the head \isi{Voice}, i.e. that it selects as \isi{ellipsis site} the phrase headed by \isi{Voice} (\textsc{sel} \isi{Voice}). Moreover, the feature [E] must have its uninterpretable features (uMod) checked out by the head \isi{Mod}, i.e. it is licensed by \isi{Mod}. The properties of the [E] feature in \ili{Czech} would differ from the properties of the [E] feature in \ili{English}, \ili{French} and \ili{Dutch} respectively, as shown in \REF{31b}--\REF{31d}. In \ili{English}, \isi{ellipsis} targets \textit{v}P and is licensed by T (see \citealt{Merchant2008a}). In \ili{French}, \isi{ellipsis} targets TP and is licensed by \isi{modal} V selecting a TP (see \citealt{Dagnac2008}). In \ili{Dutch}, \isi{ellipsis} targets \isi{VoiceP} but it is licensed by a \isi{deontic} V (see \citealt{Aelbrecht2008}). Contrary to \ili{Dutch}, however, we do not need to postulate that the \isi{VoiceP} in \ili{Czech} constitutes a phase blocking object extraction, since both subject and object extractions may take place before \isi{VoiceP} is sent to \isi{PF}.


\ea \label{31} The syntax of [E] feature in \isi{MCE}:
\hfill{(cf. \citealt{Merchant2008a})}
\ea \label{31a} \ili{Czech}: 	E\textsubscript{MCE} [\textsc{infl} [uMod], \textsc{sel} [\isi{Voice}]]
\ex \label{31b} \ili{English}: E\textsubscript{MCE} [\textsc{infl} [uT], \textsc{sel} [\textit{v}]] \hfill (see \citealt{Merchant2008a})	%\jambox{(cf. \citealt{Merchant2008a})}
\ex \label{31c} \ili{French}:	E\textsubscript{MCE} [\textsc{infl} [uV], \textsc{sel} [T\textsubscript{nf}]] \hfill (see \citealt{Dagnac2008}) %\jambox{(cf. \citealt{Dagnac2008})}
\ex \label{31d} \ili{Dutch}:	E\textsubscript{MCE} [\textsc{infl} [uV\textsubscript{deon}], \textsc{sel} [\isi{Voice}]] \hfill (see \citealt{Aelbrecht2008}) %\jambox{(cf. \citealt{Aelbrecht2008})}
\z
\z
%\vspace{2cm}
\ea \label{32}
\ea \label{32a} \ili{Czech}:	[\textsubscript{ModP} může \sout{[\textsubscript{VoiceP} [\textsubscript{\textit{v}P} t\textsubscript{subj} [\textsubscript{VP} {\ldots} ]]]}]\\

\ex \label{32b} \ili{English}: [\textsubscript{TP} can [\textsubscript{AspP} (have) [\textsubscript{VoiceP} (been) [\textsubscript{\textit{v}P} t\textsubscript{subj} \sout{[\textsubscript{VP} {\ldots} ]}]]]]
\ex \label{32c} \ili{French}:	[\textsubscript{VP} peut \sout{[\textsubscript{TP} t\textsubscript{subj} [\textsubscript{AspP} [\textsubscript{\textit{v}P} t\textsubscript{subj} [\textsubscript{VP} {\ldots} ]]]]}]
\ex \label{32d} \ili{Dutch}: [\textsubscript{VP} kan [\textsubscript{TP} t\textsubscript{subj} \sout{[\textsubscript{VoiceP} [\textsubscript{\textit{v}P} t\textsubscript{subj} [\textsubscript{VP} {\ldots} ]]]}]]
\z
\z

\begin{figure}
    \centering
    \begin{forest}
sn edges
 [ModP
   [\isi{Mod} [může,name=Mod head]
    ]
    [\isi{VoiceP}
    [\isi{Voice}\textsubscript{E},name=voice head]
     [\textit{v}P [{ } { } {\ldots} { } { } { },roof]]
    ]]
\draw[->] (voice head.south) to[out=south,in=south] node[below]{E\textsubscript{[\textsc{sel}, \isi{Voice}, uMod]}} (Mod head.south);
\draw[thick, - ] (1.6,-0.4) arc (90:180:2);
\end{forest}
    \caption{Syntactic structure for \REF{32a}}
    \label{fig:32a}
\end{figure}

\noindent The proposed analysis can account for the properties of \ili{Czech} \isi{MCE} as follows. First, \isi{MCE} requires \isi{voice identity} in both the \isi{elliptical clause} and its antecedent, i.e. both clauses must be either active or passive. Assuming that the parallelism requirement on \isi{ellipsis} includes voice features, postulating that the feature [E] targets \isi{VoiceP} guarantees that \isi{ellipsis} takes place only if elided and antecedent \isi{VoiceP} are identical. Furthermore, since \isi{VoiceP} is neither a nominal nor a clausal phrase, it follows that it cannot be pronominalized by an overt \isi{pronoun}.

Second, \isi{MCE} does not target the \isi{clitic} and the future auxiliaries, but it cannot leave aside the \isi{passive auxiliary}. Since \isi{clitic} auxiliaries are generated high in the structure, the analysis predicts that they will not be included in the \isi{ellipsis site}. Likewise, the \isi{future auxiliary} generated above \isi{modal} verbs will not be elided, see \sectref{sec:3.2}, ex. \REF{17b}. In contrast, the \isi{passive auxiliary} located in the \isi{VoiceP} will be elided along with the \isi{VoiceP}.

Third, \isi{MCE} allows extraction of focused (wh-object) and contrastively focused XPs (contrastive topics). Since elided elements are informationally given, it follows that only focused XPs can escape \isi{ellipsis} and undergo extraction. This is especially visible in the case of \isi{pronominal} object clitics, which cannot be focused, and will thus never be allowed to escape the \isi{ellipsis site}. Extraction of non-identical XPs from the \isi{ellipsis site} could, however, be viewed as problematic for parallelism constraints assumed for deletion, although these constraints do not mean full morphophonological identity. I thus propose to assume with Merchant that \isi{focus} overrides ``identity condition'' in deletion \citep{Merchant2001}. In the case of subject extraction from \textit{v}P to TP, for instance, the identity required for deletion reduces to the type of argument (referential DP), but it does not concern the meaning or the reference of the DP subjects themselves. In the case of \isi{ACD}, we can consider that the subject of the \isi{relative} clause must escape deletion precisely because it is contrasted with the subject on the main clause.

\ea
\ea \label{33a}
\gll Jan čte všechny knihy, které$_{i}$ Eva nesmí (\hspace{-2pt} číst t$_{i}$). \\
Jan reads all  books that    Eva \textsc{neg}.can  {} read\\
\glt `Jan reads all the book that Eva can't.' \hfill (\ili{Czech})
\ex \label{33b}  Jan čte [\textsubscript{VoiceP} [\textsubscript{VP} t\textsubscript{sub} t$_{v}$ všechny knihy [\textsubscript{CP} které$_{i}$ [\textsubscript{TP} Eva [\textsubscript{ModP} nesmí \sout{[\textsubscript{VoiceP} [\textsubscript{\textit{v}P} t\textsubscript{sub} [\textsubscript{VP} {\ldots} t$_{i}$ ]]]}]]]]]
\z
\z

\noindent The observation that extraction is relatively easy in \ili{Czech} can be related to the monoclausal structure of sentences with \isi{modal} verbs. In the case of an intermediate extraction (if we assume objects scrambling instead of pseudogapping), like in \REF{34}, we can suppose that extracted elements are hosted by a TP-internal Focus position (following \citealt{belletti2004aspects}) between the \isi{modal} \isi{verb} and the elided \isi{VoiceP}. This kind of extraction would not be available in \ili{French} or \ili{Dutch}, where \isi{ellipsis} targets the TP complement of the \isi{modal} V.

\begin{exe}
\ex \label{34}
\gll Já třeba   napíšu Heleně  básničku, a ty zas  můžeš písničku.\\
I  maybe \textsc{pf}.write  H.\textsc{dat}   poem.\textsc{acc} and you then can     song.\textsc{acc} \\
\glt `I might write a little poem for Helen, and you, you can write a little song for her.' \hfill (\ili{Czech})
\end{exe}
\begin{figure}
\label{fig:34b}
\caption{Syntactic structure for \REF{34}}
\begin{forest}
sn edges
 [ModP
   [\isi{Mod} [můžeš,name=Mod head]
    ]
   [FocP
     [DP$_{i}$
      [písničku]
      ]
     [Foc$'$
      [Foc]
      [\isi{VoiceP}
       [\isi{Voice}$_{E}$,name=voice head]
       [\textit{v}P [{ } { } napsat { } { } { },roof]]
      ]]]]
\draw[->] (voice head) to[out=south west,in=south] node[near start,below]{E\textsubscript{[\textsc{sel}, \isi{Voice}, uMod]}} (Mod head);
\draw[thick, - ] (3.8,-2.5) arc (90:165:2);
\end{forest}
\end{figure}

\subsection{Extending the analysis to ellipsis after the future auxiliary} \label{sec:5.2}


Since \ili{English} \isi{modal} and auxiliary verbs represent the same kind of head \citep{ross1967constraints}, it is not surprising that they behave alike with respect to \isi{ellipsis}. We can thus reduce the analysis of \isi{MCE} to the analysis of VP \isi{ellipsis}, by having the same [E] feature for both. In \ili{French} and \ili{Dutch}, \isi{modal} and auxiliary verbs are syntactically different heads and behave differently with respect to \isi{ellipsis}. If these languages only possess the [E] feature with the syntax given above, \isi{ellipsis} after auxiliary verbs will be excluded since auxiliary verbs are not V heads and do not have a non-finite TP complement.

As for \ili{Czech}, I propose that the analysis in terms of VoiceP-\isi{ellipsis} can be extended the \isi{future auxiliary} because: (i) the \isi{future auxiliary} is a functional verbal head between V and T, (ii) its complement is an extended VP, and (iii) the \isi{ellipsis} also requires \isi{voice identity}:

\ea \label{35}
\ea \label{35a}
\gll Udělali to, kdykoliv   \minsp{\{} museli  (\hspace{-2pt} \sout{to udělat}) / *\hspace{-2pt} to muselo \hspace{2cm} (\hspace{-2pt} \sout{být} \sout{uděláno})\}.\\
 did.\textsc{pl}  it   whenever {} had.to.\textsc{pl} {} {it do} {} {} it  had.to  {} {} be done.\textsc{pass}\\
\glt `They did it whenever they did have to (do it).'  \hfill (\ili{Czech})

\ex \label{35b}
\gll Měl být operován, ale \minsp{\{} nebude (\hspace{-2pt} \sout{operován}) { } / \hspace{1.3cm} *\hspace{-2pt} nebudou (\hspace{-2pt} \sout{ho { } operovat})\}.\\
had.to be operated.\textsc{pass} but {} \textsc{neg}.\textsc{fut}.\textsc{3sg}  {} operated.\textsc{pass} {} {} {} {} \textsc{neg}.\textsc{fut}.3\textsc{pl} {} {him operate}\\
\glt  `He had to be operated but he will not (be operated).'  \hfill (\ili{Czech})
\z \z

\noindent To allow both the \isi{future auxiliary} and \isi{modal} verbs to license \isi{ellipsis}, I suggest defining the [E] feature as follows:
\begin{enumerate}
    \item the \isi{licensing head} is a functional verbal head F$_{v}$, that can be realized as \isi{Mod} and Asp,
    \item the uninterpretable features of [E] are uF$_{v}$, see \figref{fig:36b}.
\end{enumerate}

\begin{exe}
\ex \label{36}
\gll Měl být  operován, ale pro$_{i}$ [\textsubscript{AspP} nebude \sout{[\textsubscript{VoiceP} \textsc{pass} [\textsubscript{VP} operován t$_{i}$]]}].\\
had.to be   operated.\textsc{pass} but {} {} \textsc{neg}.\textsc{fut}.\textsc{3sg} {{ } { } { } { } { } { } { } { } { } { } { } { } { }operated.\textsc{pass}} \\
\glt `He had to be operated but he will not (be operated).'  \hfill (\ili{Czech})
\end{exe}

\begin{figure}
    \centering
    \caption{Syntactic structure for \REF{36}}
\label{fig:36b}
\begin{forest}
sn edges
 [F\textsubscript{v}P
   [F\textsubscript{v} [nebude,name=F head]
    ]
   [\isi{VoiceP}
    [\isi{Voice}\textsubscript{E},name=voice head]
    [\textit{v}P [{ } operován { },roof]]
      ]]
\draw[->] (voice head.south) to[out=south,in=south] node[below]{E\textsubscript{[\textsc{sel}, \isi{Voice}, uFv]}} (F head.south);
\draw[thick, - ] (1.6,-0.5) arc (90:165:2);
\end{forest}
\end{figure}

\section{Conclusion} \label{sec:6}

In this paper, I have dealt with \isi{MCE} in \ili{Czech} from a \isi{comparative} perspective. I have shown that \ili{Czech} \isi{modal} verbs and \ili{Czech} \isi{MCE} exhibit a mixed behaviour with respect to languages like \ili{English}, \ili{French}, and \ili{Dutch}. Like \ili{English}, it allows for various extractions from the \isi{ellipsis site} and for different subjects in \isi{ACD} constructions. Like \ili{French} and \ili{Dutch}, it does not allow for intervening elements between \isi{modal} \isi{verb} and \isi{ellipsis site} and it requires \isi{voice identity} of the elided VP and its antecedent.

I have argued that the properties of \isi{MCE} that we observe at the surface are induced by the head \isi{licensing ellipsis} (F$_{v}$, V, or T) and the \isi{ellipsis site} (\isi{VoiceP}, TP, or VP). Adopting a deletion account of \isi{MCE} based on a presumably universal \isi{ellipsis feature} [E], I have undertaken to parametrize this feature [E], whose properties include precisely the \isi{licensing head} and the \isi{ellipsis site}. In addition, the properties of this [E] feature also imply whether \isi{ellipsis} is available with auxiliary verbs. As for \ili{Czech}, I have proposed that [E] is licensed by a functional verbal head F$_{v}$, which can be realized by \isi{modal} verbs or the \isi{future auxiliary} and targets \isi{VoiceP}.

The question remains whether we can relate the parametrization of the [E] feature to other language properties. One hypothesis to explore can be found in \cite{Cyrino-Matos2002}, who claim that there is a correlation between the possibility of verbal \isi{ellipsis} (after both auxiliary and \isi{modal} verbs) and the structure of the extended verbal projection, in particular the realization of aspect. This issue is however outside the scope of this paper and must be left to further investigation.

\section*{Abbreviations}
\begin{tabularx}{.58\textwidth}{@{}lX@{}}
	1, 2, 3 & first, second, third person\\
	2P{Cl}&{second position} clitics\\
	\textsc{acc} & {accusative}\\
	{ACD}&antecedent-contained deletion\\
	\textsc{aux} & auxiliary\\
	\textsc{cl} & {clitic}\\
    \textsc{cond}&conditional\\
	\textsc{dat} & {dative}\\
    {deont}/{epist}& {deontic}/{epistemic}\\
    \textsc{fut}&future {tense}\\
	\textsc{impf}&{imperfective}\\
\end{tabularx}
\begin{tabularx}{.38\textwidth}{@{}lX@{}}
    {Mod} & {modal} {verb}\\
	{MCE}&{modal complement ellipsis}\\
	\textsc{neg} & {negation}\\
    \textsc{pass}&passive\\
	\textsc{past}&past {tense}\\
	{PF}& phonetic form\\
    \textsc{pf}&{perfective}\\
	\textsc{pl} & plural\\
	\textsc{prt} & particle\\
	\textsc{sg} & singular\\
\end{tabularx}

\section*{Acknowledgements}
I would like to thank two anonymous reviewers for their helpful comments and suggestions.

\sloppy
\printbibliography[heading=subbibliography,notkeyword=this]

\il{Czech|)}
\end{document}
