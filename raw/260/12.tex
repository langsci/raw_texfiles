\documentclass[output=paper, colorlinks, citecolor=brown, newtxmath]{langsci/langscibook}
\ChapterDOI{10.5281/zenodo.3764865}
%\bibliography{localbibliography}
%\usepackage{langsci-optional}
\usepackage{langsci-gb4e}
\usepackage{langsci-lgr}

\usepackage{listings}
\lstset{basicstyle=\ttfamily,tabsize=2,breaklines=true}

%added by author
% \usepackage{tipa}
\usepackage{multirow}
\graphicspath{{figures/}}
\usepackage{langsci-branding}

%
\newcommand{\sent}{\enumsentence}
\newcommand{\sents}{\eenumsentence}
\let\citeasnoun\citet

\renewcommand{\lsCoverTitleFont}[1]{\sffamily\addfontfeatures{Scale=MatchUppercase}\fontsize{44pt}{16mm}\selectfont #1}
  
\IfFileExists{../localcommands.tex}{
  \usepackage{langsci-optional}
\usepackage{langsci-gb4e}
\usepackage{langsci-lgr}

\usepackage{listings}
\lstset{basicstyle=\ttfamily,tabsize=2,breaklines=true}

%added by author
% \usepackage{tipa}
\usepackage{multirow}
\graphicspath{{figures/}}
\usepackage{langsci-branding}

  
\newcommand{\sent}{\enumsentence}
\newcommand{\sents}{\eenumsentence}
\let\citeasnoun\citet

\renewcommand{\lsCoverTitleFont}[1]{\sffamily\addfontfeatures{Scale=MatchUppercase}\fontsize{44pt}{16mm}\selectfont #1}
  
  %% hyphenation points for line breaks
%% Normally, automatic hyphenation in LaTeX is very good
%% If a word is mis-hyphenated, add it to this file
%%
%% add information to TeX file before \begin{document} with:
%% %% hyphenation points for line breaks
%% Normally, automatic hyphenation in LaTeX is very good
%% If a word is mis-hyphenated, add it to this file
%%
%% add information to TeX file before \begin{document} with:
%% %% hyphenation points for line breaks
%% Normally, automatic hyphenation in LaTeX is very good
%% If a word is mis-hyphenated, add it to this file
%%
%% add information to TeX file before \begin{document} with:
%% \include{localhyphenation}
\hyphenation{
affri-ca-te
affri-ca-tes
an-no-tated
com-ple-ments
com-po-si-tio-na-li-ty
non-com-po-si-tio-na-li-ty
Gon-zá-lez
out-side
Ri-chárd
se-man-tics
STREU-SLE
Tie-de-mann
}
\hyphenation{
affri-ca-te
affri-ca-tes
an-no-tated
com-ple-ments
com-po-si-tio-na-li-ty
non-com-po-si-tio-na-li-ty
Gon-zá-lez
out-side
Ri-chárd
se-man-tics
STREU-SLE
Tie-de-mann
}
\hyphenation{
affri-ca-te
affri-ca-tes
an-no-tated
com-ple-ments
com-po-si-tio-na-li-ty
non-com-po-si-tio-na-li-ty
Gon-zá-lez
out-side
Ri-chárd
se-man-tics
STREU-SLE
Tie-de-mann
}
  \togglepaper[12]%%chapternumber
}{}
%\togglepaper[12]

\author{Marko Simonović\affiliation{University of Nova Gorica}\orcid{0000-0002-9651-6399}\lastand  Boban Arsenijević\affiliation{University of Graz}\orcid{0000-0002-1124-6319}}
 \title{Syntax predicts prosody: Multi-purpose morphemes in Serbo-Croatian}


\abstract{We consider four Serbo-Croatian suffixes which appear in various structural positions and display different prosodic behaviour in these positions. Such suffixes allow us to establish the effects of the structural context on prosody by constructing minimal pairs between e.g.  derivation and inflection. All four suffixes are shown to fit the generalization that derivational morphology is more accented than inflectional morphology. We propose a formal explanation and discuss the functional benefits of a surface differentiation between the two uses.

    \keywords{prosody, derivation, inflection, multi-purpose morphemes, Serbo-Cro\-atian, Optimality Theory}
    }
% \usepackage{langsci-optional}
\usepackage{langsci-gb4e}
\usepackage{langsci-lgr}

\usepackage{listings}
\lstset{basicstyle=\ttfamily,tabsize=2,breaklines=true}

%added by author
% \usepackage{tipa}
\usepackage{multirow}
\graphicspath{{figures/}}
\usepackage{langsci-branding}

% 
\newcommand{\sent}{\enumsentence}
\newcommand{\sents}{\eenumsentence}
\let\citeasnoun\citet

\renewcommand{\lsCoverTitleFont}[1]{\sffamily\addfontfeatures{Scale=MatchUppercase}\fontsize{44pt}{16mm}\selectfont #1}
  



\begin{document}%
\maketitle%
\il{Serbo-Croatian|(}

\section{Introduction}\label{sec:simonovic:1}

Morphemes in different structural positions have different phonological properties. This insight has been formalised within various frameworks in both phonology and syntax. In phonology, roots have long been recognised as allowing more phonological contrast than affixes, an observation which has been formalised within Optimality Theory as the constraint family Root Faithfulness (see e.g. \citealt{Mcc1993} and  \citealt{Bec1997} for a discussion of roots as one of the privileged positions in phonology). In a related model, \citet{Revithiadou1999} presents evidence for the prosodic dominance of syntactic heads (stems and \isi{derivational} affixes can be heads, but \isi{inflectional} affixes cannot). Several accounts couched in Distributed Morphology \citep{Hal1993,Hal1994} deal with pro\-so\-dic asymmetries of this kind. \citet{Don2017} argues that \ili{Spanish} suffixes that express phi-features behave as prosodic adjuncts, which excludes them from the domain of stress assignment under certain circumstances. \citet{Mar2002} presents evidence for prosodic behaviour that is a function of syntactic phasehood.

To our knowledge, few analyses along these lines have been proposed of cases where the same morpheme appears in different environments, and virtually no analysis targets the same affix in its \isi{derivational} and \isi{inflectional} uses. Data involving the same affix are crucial because they constitute minimal pairs in which the only difference is the structural position of the affix. Such minimal pairs are the only type of evidence immune to alternative accounts that explain asymmetries between different structural positions as results of accident or functional pressures on lexicalisation, which do not need to be recorded in the grammar. In other words, while there are formal accounts of phonological asymmetries between inflection and derivation \citep[e.g.][]{Mcc1993, Bec1997, Revithiadou1999} the fact that the \isi{derivational} suffix X is accented whereas the \isi{inflectional} suffix Y is not does not immediately strike researchers as a fact in need of a grammatical explanation. However, in cases where the same affix gets different prosodic treatment in different structural positions, we can be sure to see a grammatical mechanism at work.

The first analysis of the same morpheme in different structural environments we are aware of is a cursory discussion of nominalising \textit{-ost} in \ili{Slovenian} in \citet{Mar2002}. This suffix seems to combine with the \isi{adjective} \textit{mlad} `young' in two different nominalisations: in \textit{mlad-óst} `youth' and in \textit{mlád-ost} `youngness' (several other pairs of -\textit{ost}-nominalisations are listed). For Marvin, the relevant distinction is that between root nominalisations and deadjectival nominalisations. In the root \isi{nominalisation} \textit{mlad-}\textit{óst}, there is no \isi{adjectival} head between the root and the nominal head -\textit{ost}, so the root and the suffix are in the same syntactic phase. As a consequence, the suffix imposes its \isi{prosody} (\textit{mlad-óst}). In the deadjectival \isi{nominalisation} \textit{mlád-ost}, there is a (silent) \isi{adjectival} head between the root and the nominal head -\textit{ost}, which causes a separate spell-out of \textit{mlád}. The suffix arrives “too late” to affect the \isi{stress pattern} of the whole, so the resulting \isi{stress pattern} is that in \textit{mlád-}\textit{ost}. \citet{Arsim2013} analyse the \ili{Serbo-Croatian} cognate of the same suffix using the lexicalist mechanism of \textsc{Lexical Conservatism} \citep[first proposed in][]{Ste1997}, a constraint which enforces copying the \isi{prosody} of the base in all paradigm members. Lexical Conservatism has no influence on non-paradigm members. To stay with the same \ili{Slovenian} example, \textit{mlád-}\textit{ost} counts as part of the paradigm of  the \isi{adjective} \textit{mlád} (based on semantic transparency and the fact that the pattern is productive). As such, \textit{mlád-}\textit{ost} copies the \isi{prosody} of \textit{mlád} due to the pressure of Lexical Conservatism. \textit{Mlad-}\textit{óst}, on the other hand, is a separate lexical item and its stress only depends on the general constraints (which in this case seem to enforce the faithfulness to the stress specification of the affix \textit{-ost}). \citet{Sim2014} present an analysis along the same lines of the \ili{Serbo-Croatian} deverbal nominalisations, to which we will turn in \sectref{sec:simonovic:41}.

\citet{Mar2002} and \citet{Arsim2013} use different formal tools, but both account for the targeted data sets. Given this background, our main goal is to expand the data set. We achieve this by discussing the influence of the structural position on \isi{prosody} in cases of maximal \textsc{multi-functionality}: those cases where one of the structures in which the affix surfaces is clearly inflection whereas the other one is derivation. Furthermore, we observe various cases of such multi-functionality within the same language in order to establish generalisations which hold across morphemes.

The main \isi{focus} of this paper lies on the prosodic behaviour of affixes which occur both in inflection and derivation and have different prosodic effects in the two domains. To the best of our knowledge, while multi-functional affixes are relatively frequent, so far, this kind of systematic dichotomy at the level of \isi{prosody} has only been attested in \ili{Serbo-Croatian}. This is why the empirical \isi{focus} of this paper will be on data from this language.\largerpage[1]

The rest of this paper is organised as follows. \sectref{sec:simonovic:2} addresses the issue of identifying affixes which can surface in both inflection and derivation as well as predictions concerning the prosodic effects of such affixes. Based on the existing literature, the prediction is put forward that the \isi{derivational} uses of the affixes should go hand in hand with more accentedness, whereas the \isi{inflectional} uses should be characterised by less accentedness. In \sectref{sec:simonovic:3} we present the key features of \ili{Serbo-Croatian} \isi{prosody} and its notation. We then list four ways in which the \isi{prosodic pattern} of the base can be influenced by an affix in \ili{Serbo-Croatian}. \sectref{sec:simonovic:4} presents a detailed overview of four \ili{Serbo-Croatian} affixes which appear in both inflection and derivation. We keep track of their \textsc{surface distinguishability} and  \textsc{accentedness asymmetries} in the two contexts. In \sectref{sec:simonovic:5} we identify the common patterns in the data presented in \sectref{sec:simonovic:4}, observing that the \isi{prosodic pattern} in the derivations seems to be the same at least across the suffixes performing the same function. In \sectref{sec:simonovic:6} we consider the theoretical consequences of the observed asymmetries. \sectref{sec:simonovic:7} concludes this paper.

\section{Multi-functional affixes}\label{sec:simonovic:2}

Cases of the same affix appearing in both derivation and inflection are not hard to come by. Below we quote examples from \ili{English} and \ili{Italian}. Both \ili{English} \textit{-ed}\textsubscript{A} and \ili{Italian} \textit{-uto}\textsubscript{A} appear as regular past/\isi{passive participle} endings when combined with verbs, but also as adjectivisers when combined with nouns.

\begin{exe} \label{ex:simonovic:1}
	\ex \ili{English} \textit{-ed}\textsubscript{A}
	\begin{xlist}
		\ex fear fear-ed \hfill{(Inflection)}
		\ex beard beard-ed \hfill{(Derivation)}
	\end{xlist}
\end{exe}
\begin{exe}\label{ex:simonovic:2}
	\ex \ili{Italian} \textit{-uto}\textsubscript{A}
	\begin{xlist}
	    \ex \gll tem-ere tem-uto \\
        fear-\textsc{inf} fear-\textsc{pass.ptcp} \\ \hfill{(Inflection)}
		\glt `to fear' `feared'
		\ex \gll barba \hspace{0.15cm} barb-uto \\
        beard {} beard-uto\\\hfill{(Derivation)}	\end{xlist}
	\end{exe}


\noindent \ili{Serbo-Croatian} has several suffixes which behave in a similar way. Moreover, \ili{Serbo-Croatian} pairs of this type are often characterised by surface distinguishability by means of \isi{prosody}. In \REF{ex:simonovic:3} and \REF{ex:simonovic:4} we show how \ili{Serbo-Croatian} \textit{-at}\textsubscript{A} and \textit{-an}\textsubscript{A} appear in different constellations.

\ea \ili{Serbo-Croatian} \textit{-at}\textsubscript{A} \label{ex:simonovic:3} \begin{xlist} \ex \gll prìzna-ti prȉzna-at
\\
recognise-\textsc{inf} recognise-\textsc{pass.ptcp} \\ \hfill{(Inflection)}
		\glt `recognise' \hspace{0.4cm} `recognised'
		\ex \gll pȑs-a { } pr̀s-at \\
bust {} bust-at\\ \hfill{(Derivation)}
		\glt `bust'  \hspace{0.06cm} `busty'
	\end{xlist}
\z


\ea \ili{Serbo-Croatian} \textit{-an}\textsubscript{A}\label{ex:simonovic:4}
\begin{xlist}
\ex \gll pòsla-ti pȍsla-an\\
send-\textsc{inf} send-\textsc{pass.ptcp}\\ \hfill{(Inflection)}
		\glt `to send'  `sent'
		\ex \gll gȉps-a gìps-an \\
plaster-\textsc{gen} plaster-an\\ \hfill{(Derivation)}
 		\glt `plaster' \hspace{0.45cm} `made of plaster'

	\end{xlist}
\z

\noindent We will discuss the details of the prosodic representation of \ili{Serbo-Croatian} in  \sectref{sec:simonovic:3}. For the moment, suffice it to say that both suffixes systematically display different prosodic patterns in the two uses and, previewing our findings, that the \isi{derivational} endings are more accented. This asymmetry is in the same direction as those observed in \citet{Mar2002} and \citet{Arsim2013}, and it also matches the cross-linguistic tendencies that will be discussed below.

Before moving on, we briefly address one potential objection to this presentation of the data. There is an obvious alternative: accidental homonymy between unrelated affixes. So \ili{English} \textit{-ed}, \ili{Italian} \textit{-uto}, as well as \ili{Serbo-Croatian} \textit{-at} and \textit{-an} may be not a single affix but pairs of unrelated affixes which happen to have the same form or, more precisely, the same segmental content. Our arguments against this view can be summarised as follows:
\begin{itemize}
  \item Whether inflection or derivation, the category of the word resulting from affixation is the same (\isi{adjective} in all the cases discussed above). On the accidental homonymy analysis, this would be another accident.
  \item Most of these suffixes are old in both uses, without a diachronic tendency to phonologically split into two different suffixes – which is what would be expected were they different items.
  \item In \ili{Serbo-Croatian}, the two uses of the same affix are systematically distinguished by different prosodic patterns, as discussed in this paper.
  \item Finally, it would be quite surprising for a \ili{Germanic}, a \ili{Romance} and a \ili{Slavic} language to have an accidental homonymy between the suffixes with exactly the same purposes: the \isi{passive participle} and an \isi{adjectival} suffix.
  \end{itemize}

Once we accept that the \isi{derivational} and \isi{inflectional} affixes in question are indeed the same affix, prosodic asymmetry is predicted to exist in some language. This follows from the cross-linguistic generalisation that \isi{derivational} affixes are phonologically more prominent than \isi{inflectional} affixes (see e.g. \citealt{Bec1997, Revithiadou1999}). Based on facts from several languages from different language families, \citet{Revithiadou1999} proposes two constraints that favour \isi{prosodic prominence} of \isi{derivational} affixes. \textsc{HeadFaith} is a \isi{faithfulness constraint} which protects \isi{lexical prominence} of syntactic heads (\isi{derivational} affixes and stems are argued to be syntactic heads, unlike \isi{inflectional} affixes). \textsc{HeadStress} is a \isi{markedness constraint} that militates against stress on non-heads. This constraint is violated whenever \isi{inflectional} affixes are stressed.

The observed asymmetry predicts that there should exist two types of languages:
\begin{itemize}
  \item languages such as \ili{Italian} in which both \isi{derivational} and \isi{inflectional} affixes can be stressed (or otherwise prosodically strong) and
\item languages such as \ili{English} or \ili{Dutch} in which \isi{derivational} affixes can be prosodically prominent but \isi{inflectional} affixes cannot.\end{itemize}
To our knowledge, there is no language which is a mirror image of \ili{English} and \ili{Dutch}. In such a language, \isi{inflectional} suffixes would be more stressed than \isi{derivational} affixes.

In \ili{Serbo-Croatian}, both \isi{derivational} and \isi{inflectional} affixes can be either accented or accentless (but accented \isi{inflectional} suffixes are becoming rare, see \citealtv{chapters/13}). Prosodic prominence in \ili{Serbo-Croatian} involves stress, tone and \isi{vowel length} and the lexical sponsor of the surface \isi{prosody} is not easily determined. This is why we first turn to \ili{Serbo-Croatian} \isi{prosody} and its representation in the following section.

\section{Standard Serbo-Croatian prosody
} \label{sec:simonovic:3}

\ili{Serbo-Croatian} is traditionally classified as a \textsc{pitch-accent} system in which the distribution of stress is predictable from that of \textsc{high tone} (\citealt{Zec1988, Zec1999}). Every prosodic word is characterised by a single \textsc{tonal accent} headed by the single stressed syllable of the word. Classical descriptions distinguish between falling and rising tonal accents. In the \textsc{falling accents}, stress and \isi{high tone} co-occur on the same syllable, which is typically the first syllable of the word. Depending on the length of the stressed syllable, there are long falling and short falling accents (in \textit{lȃđa} `boat' and \textit{krȁđa} `theft', respectively). The \textsc{rising accents} are traditionally analysed as spans of two adjacent syllables which both have a \isi{high tone}, whereas only the first syllable also carries stress (but see \citealt{Zsiga-Zec2013} for arguments that in some varieties the first syllable of the rising accents only carries stress but no \isi{high tone}). The rising accents can also be long or short, depending on the length of the stressed syllable (as in \textit{báka} `grandmother' and \textit{màča} `sword.\textsc{gen}', respectively).

Most accounts of \ili{Serbo-Croatian} \isi{prosody} share some central assumptions. The rising accents are generally assumed to have a lexical sponsor in the rightmost syllable of the span, which automatically spreads onto the preceding syllable: the \isi{rising accent} in \textit{màča} `sword.\textsc{gen}' then derives from /maʈ͡ʂa\textsubscript{H}/. This spreading accounts for the fact that rising accents do not occur in monosyllables, where only falling accents are possible. Falling accents are the realisation of high tones that cannot spread to the left: those on the initial syllables. This also accounts for the fact that falling accents are restricted to initial syllables. The falling accents are assumed to get assigned to the initial syllable as a default in the absence of lexically specified \isi{prosody}: /novine/ will become \textit{nȍvine} `newspaper' if no \isi{high tone} gets assigned in the derivation.

An obvious disadvantage of the traditional diacritics is that they do not mark the second part of the \isi{rising accent}, so the reader needs to memorise the diacritics and `imagine' a \isi{high tone} after every rising-\isi{accent} diacritic. A disadvantage of using an IPA notation is that it has some overlap with the traditional notation, which confuses those \ili{Serbo-Croatian} speakers who are used to the traditional notation. This is why we use a different, more transparent notation in this paper: \isi{high tone} is marked by capitalisation and \isi{vowel length} by \isi{doubling}. Stress is not marked, as it predictably falls on the leftmost high-toned (= capitalised) syllable.

\begin{table}
\caption{Traditional diacritics and the notation used here}
\label{tab1_p}
 \begin{tabular}{ l l l }
\lsptoprule & monosyllables & polysyllables\\
\midrule
long falling  &  grȃd = grAAd  `town' & lȃđa = lAAđa `boat'  \\
short falling  & grȁd = grAd  `hail' & krȁđa = krAđa `theft'
 \\
long rising  &   /  &    báka = bAAkA `grandmother'\\
short rising  &  / &  màča = mAčA `sword.\textsc{gen}'\\
 \lspbottomrule
 \end{tabular}
\end{table}

\ili{Serbo-Croatian} suffixes display varying behaviour with respect to \isi{prosody}. \tabref{tab2} illustrates four suffixes interacting with bases which have a \isi{rising accent}, i.e. they have an underlying \isi{high tone} in their representation. The behaviour of these four affixes can be described as:\begin{itemize}
  \item \textsc{accent-bearing} (\textit{-ana} deletes the \isi{accent} of the base and imposes its own),
  \item \textsc{accent-attracting} (\textit{-iji} moves the \isi{accent} closer to itself),
  \item \textsc{accent-neutral} (\textit{-oost} does not change the \isi{prosody} of the base) and
  \item \textsc{accent-erasing} (with \textit{-aaj}, the \isi{high tone} of the base is deleted, but no other \isi{high tone} is added by the suffix).
  \end{itemize}
Note that in the fourth example, the accent-erasing suffix leads to default \isi{prosody} i.e. to the initial short \isi{falling accent}.

\begin{table}
\small
\caption{Prosodic effect of suffixes in Serbo-Croatian}
\label{tab2}
 \begin{tabular}{l l l l l }
\lsptoprule
Base  & šEćEr  & rAnjIv  & rAnjIv  & pOkUš-ati  \\
& `sugar' & `vulnerable' & `vulnerable' & `try.\textsc{inf}'\smallskip \\
Derivation & šećer-AnA   & ranjIv-Ijii  & rAnjIv-oost  & pOkuš-aaj   \\
 &  `sugar factory' & `more vulnerable' & `vulnerability' & `attempt'\smallskip  \\
Behaviour   & accent-bearing & accent-attracting & accent-neutral & accent-erasing \\
 \lspbottomrule
 \end{tabular}
\end{table}

\section{Case studies} \label{sec:simonovic:4}

In this section we present a detailed overview of the prosodic behaviour of four \ili{Serbo-Croatian} affixes which appear both in inflection and derivation: the nominal \textit{-VVje}\textsubscript{N}  and the \isi{adjectival} \textit{-en}\textsubscript{A}, \textit{-an}\textsubscript{A}, \textit{-at}\textsubscript{A}.

Throughout the discussion of the affixes, we will keep track of two aspects of their behaviour. One is \textsc{surface distinguishability}: to what extent are the two uses of the suffix distinguishable in the surface form? The other concerns the formal status of the relevant uses of the suffix in relation to protypical derivation or inflection. In each case we consider the \textsc{prosodic dominance} (accentedness), \textsc{productivity} and \textsc{semantic transparency} of the suffixes. In line with the typological generalisations and what we found in \ili{Serbo-Croatian} in \citet{Arsim2013} and \citet{Sim2014}, the general expectation is that \isi{inflectional} uses should be prosodically non-dominant, productive and semantically transparent, whereas the \isi{derivational} uses should be more prosodically dominant, less productive and less semantically transparent.

In the literature on \ili{Serbo-Croatian}, the affixes under scrutiny here, especially the \isi{adjectival} \textit{-en}\textsubscript{A}, \textit{-an}\textsubscript{A} and  \textit{-at}\textsubscript{A}, are analysed as different morphemes in their \isi{derivational} and \isi{inflectional} uses. Since we are proposing a new unified analysis of the suffixes in question, we limit our attention to those cases where the presence of the suffix is unquestionable. We therefore restrict our corpus to the cases of concatenation of morphemes without any irregular over- or underapplication of phonological or morphological processes e.g. unexpected intervening segments or consonant mutation. We do make an exception regarding one process, as it is fully productive in at least one of the contexts, that of so-called \textit{iotation}. Iotation is a consonant palatalisation, typically before a \textit{j}, e.g. in /sxʋati+en/ which surfaces as \textit{sxʋat͡ɕen} `understood' via the intermediate /sxʋatjen/, as \textit{tj} is palatialised to \textit{t͡ɕ}.

\largerpage
In the rest of the paper, we distinguish between \isi{derivational} and \isi{inflectional} versions using the following notation: \textit{affix}\textsubscript{DERIV} will be used for the \isi{derivational} versions, whereas \textit{affix}\textsubscript{INFLECT} will be used for the \isi{inflectional} versions of the same affixes.
In \sectref{sec:simonovic:41} the nominal suffix \textit{-VVje}\textsubscript{N} is discussed, whereas in  \sectref{sec:simonovic:42} the \isi{adjectival} suffixes \textit{{-en}}\textsubscript{A}, \textit{{-an}}\textsubscript{A} and \textit{{-at}}\textsubscript{A} are in \isi{focus}.
\subsection{Case study 1: \textit{-VVje}\textsubscript{N}}\label{sec:simonovic:41}

This case study summarises some of the findings presented in \citet{Sim2014}, placing them in the context of this paper. The suffix \textit{-VVje}\textsubscript{N} consists of a \isi{vowel length} that gets realised on the last syllable of the base and the segmental content \textit{-je}. It combines with verbal bases, as well as with phrasal units, mainly VPs, N(um)Ps and PPs. In this paper, we only \isi{focus} on its application in the verbal domain, where it derives neat minimal pairs depending on the \isi{aspectual} properties of the base.

When combined with \isi{imperfective} verbs, the suffix \textit{-VVje}\textsubscript{N} derives event-de\-noting deverbal nominalizations. This pattern is fully productive (the suffix combines with all \isi{imperfective} verbs), typically semantically transparent (fully compositional), and the suffixation does not affect the \isi{prosody} of the base. This type of derivation hence shows a number of properties typical for inflection and the suffix can therefore be represented as \textit{-VVje}\textsubscript{N.INFLECT} (\tabref{tab3}).

\begin{table}
\caption{\textit{-VVje}\textsubscript{N.INFLECT}}
\label{tab3}
 \begin{tabular}{l l|l l}
\lsptoprule
unapređIIvAti & `promote.\textsc{\isi{ipfv}.inf}' & unapređIIvAAnje & `promoting.\textsc{ipfv}' \\
sAAdIti & `plant.\textsc{\isi{ipfv}.inf}' & sAAđEEnje & `planting.\textsc{ipfv}' \\
čEkati & `wait.\textsc{\isi{ipfv}.inf}' & čEkaanje   &`waiting.\textsc{ipfv}' \\
psOvAti &  `swear.\textsc{\isi{ipfv}.inf}' & psOvAAnje & `swearing.\textsc{ipfv}' \\
 \lspbottomrule
 \end{tabular}
\end{table}

When combined with \isi{perfective} verbs, suffix \textit{-VVje}\textsubscript{N} derives factitive nominalizations. The derivation is idiosyncratic, barely productive, frequently lexicalized and imposes its own prosodic shape. It is hence much closer to prototypical derivation and the suffix can be represented as \textit{-VVje}\textsubscript{N.DERIV} (\tabref{tab4}, see also \citealt{Sim2018}).

\begin{table}
\caption{\textit{-VVje}\textsubscript{N.DERIV}}
\label{tab4}
 \begin{tabular}{ l l|ll}
\lsptoprule
unaprEEdIti &`promote.\textsc{\isi{pfv}.inf}'  & unapređEEnjE &`promotion' \\
zasAAdIti &`plant.\textsc{\isi{pfv}.inf}'	&	*zasađEEnjE &\\
sAčEkati  &`wait.\textsc{\isi{pfv}.inf}' 	& *sačekAAnjE &\\
opsOvAti &`swear.\textsc{\isi{pfv}.inf}' & *opsovAAnje'& \\
 \lspbottomrule
 \end{tabular}
\end{table}

Summarising the prosodic bahaviour of \textit{-VVje}\textsubscript{N} when combined with vebal participles, we can establish two patterns with a clear divide between them: \mbox{\textit{-VVje}\textsubscript{N.INFLECT}} behaves as accent-neutral as illustrated in \tabref{tab3} above, whereas the nominalisations with \textit{-VVje}\textsubscript{N.DERIV} predictably have penult stress. In other words, the \textit{-VVje}\textsubscript{N.DERIV} behaves as accent-attracting.

\largerpage
There is full surface distinguishability between the productive and transparent pattern, which shows properties of \isi{inflectional} morphology, on the one hand and the idiosyncratic pattern, which acts like prototypical derivation, on the other hand. This asymmetry fits the typological generalisation that the properties of inflection are more likely to coincide with prosodic inactivity, and the properties of derivation go hand in hand with prosodically active behavior.


\subsection{Case study 2:  \textit{-en}\textsubscript{A}, \textit{-an}\textsubscript{A}, \textit{-at}\textsubscript{A}}\label{sec:simonovic:42}

\ili{Serbo-Croatian} has a set of three different suffixes which are equivalent in relevant respects to the \ili{English} \textit{-ed} and the \ili{Italian} \textit{-uto }discussed in \sectref{sec:simonovic:2}, i.e. which are used both for the \isi{passive participle} and for the derivation of adjectives. These suffixes are \textit{-en}\textsubscript{A}, \textit{-an}\textsubscript{A} and \textit{-at}\textsubscript{A}.
We applied the following selection criteria in assembling our data set of forms which contain \textit{{-en}}\textsubscript{A}, \textit{{-an}}\textsubscript{A} and \textit{{-at}}\textsubscript{A}. Regarding the \isi{passive participle} use, we only included in our corpus verbs in which \textit{{-en}}\textsubscript{A}, \textit{{-an}}\textsubscript{A} and \textit{{-at}}\textsubscript{A} can clearly be reconstructed as the \textsc{pass.ptcp} morphemes (app. $90\%$  of all the verbs). A detailed overview of verbal paradigms with prosodic information can be found in \citet{Klaic2013}.

Our corpus of non-participial adjectives derived by  \textit{{-en}}\textsubscript{A}, \textit{{-an}}\textsubscript{A} and\textit{ {-at}}\textsubscript{A} was assembled using the Reverse dictionary of the \ili{Serbian} language (\citealt{Nik2000}) and various available descriptions of \ili{Serbo-Croatian} \citep{Bab2002, Ste1979, Bar1995}.
As explained at the beginning of this section, we restricted our corpus to the clearest cases of the use of the suffixes in question. Only words which have a clear structure stem+\textit{en}\textsubscript{A}/\textit{an}\textsubscript{A}/\textit{at}\textsubscript{A} were included. Specifically, words with a more complex suffix structure (e.g. \textit{papir}\textsubscript{N}+\textit{n}\textsubscript{A}+\textit{at}\textsubscript{A} `made of paper'), and words with stems synchronically lacking a semantic relation to the derivation (e.g. \textit{iskr}\textsubscript{N}+\textit{en}\textsubscript{A} `honest', synchronically not related to \textit{iskra} `sparkle') were excluded from the corpus. Additionally, words with stem modifications other than iotation (e.g. \textit{stamben} `residential' which is clearly related to \textit{stan} `apartment') were excluded as well.

\largerpage
For the prosodic specification of the bases and results of suffixation, we have consulted the prosodic intuitions of modern \ili{Serbo-Croatian} speakers. The full set of words with the \isi{derivational} versions of \textit{-en}\textsubscript{A}, \textit{-an}\textsubscript{A} and \textit{-at}\textsubscript{A} can be found in the Appendix. The \isi{inflectional} versions of these suffixes are productive (especially for the first two), which is why we worked with \isi{verb} classes rather than a corpus.
We discuss each of the three suffixes in a separate subsection, including a quantitative overview of their prosodic behaviour.

\subsubsection{\textit{-an}}\label{sec:simonovic:421}
Adjectives in \textit{-an}\textsubscript{DERIV} are mostly \isi{denominal} and have the interpretation of being made of the material denoted by the base noun, or having a property related to its semantics to a large extent. Their distribution in the corpus is shown in \tabref{tab5}.

\begin{table}
\caption{Adjectives in \textit{-an}\textsubscript{DERIV} in the corpus}
\label{tab5}
\begin{tabular}{l l l l l}
\lsptoprule
 Base type & N & V & A & Phrase
\\
\midrule
Example  &  Ulj-An & pIj-An & mEk-an  & /   \\
   & `made of oil' & `drunk' & `soft' & /  \\
& (UUlj-e `oil') & (pI-ti `drink') & (mEk `soft') & /	\smallskip\\
Prosodic behaviour & Attracting & Attracting & Attracting & / \\
& & & Neutral \smallskip& \\
Number of items & 33 & 1 & 2 & 0 \\
 \lspbottomrule
 \end{tabular}
\end{table}

\textit{-an}\textsubscript{DERIV} virtually always surfaces as the second part of the \isi{rising accent} (the only exception being the \isi{adjective} \textit{mEk-an} illustrated in \tabref{tab5}). It is therefore prosodically active and overwrites the \isi{prosody} of the base. \textit{-an}\textsubscript{DERIV} is only attested with monosyllabic bases.

\textit{-an}\textsubscript{INFLECT} shows a prosodically inactive behavior. Without exceptions, passive participles in \textit{-an}\textsubscript{INFLECT} have a \isi{prosodic pattern} that exists elsewhere in the verbal paradigm, as illustrated in \tabref{tab6}, where the other form with the same \isi{prosodic pattern} is represented in bold. In other words, it does not affect the \isi{prosody} of the base, and therefore we classify it as unaccented (i.e. neutral).

\begin{table}
\caption{Participles in \textit{-an}\textsubscript{INFLECT}}
\label{tab6}
 \begin{tabular}{ l l l l l}
\lsptoprule
\textsc{inf}&\textsc{pres.1sg}&\textsc{pst.ptcp}&\textsc{pass.ptcp}& Gloss\\
\midrule
pIItAti&\textbf{pIItaam}&pIItAo&\textbf{pIItaan}&	`ask.\textsc{ipfv}' \\
zapIItiAti&\textbf{zApIItaam}&zapIItAo&\textbf{zApIItaan}&`ask.\textsc{pfv}'\\
čItAti&čItAAm&\textbf{čItao}&\textbf{čItaan}&`read.\textsc{ipfv}'\\
pročItAti&pročItAAm&\textbf{prOčitao}&\textbf{prOčitaan}&`read.\textsc{pfv}'\\
 \lspbottomrule
 \end{tabular}
\end{table}

\largerpage
Even though they are segmentally identical, the two uses of \textit{-an}, the participial \textit{-an}\textsubscript{INFLECT} and the \isi{denominal} adjectivizer \textit{-an}\textsubscript{DERIV}, are surface-distinguishable:
\begin{itemize}
  \item
\textit{-an}\textsubscript{DERIV} is always part of a rising span (\textit{UljAn} `made of oil').
\item
\textit{-an}\textsubscript{INFLECT} is never part of a rising span (\textit{pIItaan }`ask.\textsc{\isi{ipfv}.pass.ptcp}').
\item \textit{-an}\textsubscript{DERIV} never surfaces in a long syllable ({UljAn} `made of oil').
\item  \textit{-an}\textsubscript{INFLECT} always surfaces in a long syllable (\textit{pIItaan }`ask.\textsc{\isi{ipfv}.pass.ptcp}').
\end{itemize}

The prosodic behaviour of the two uses of the suffix \textit{-an} thus fully complies with the generalizations in \sectref{sec:simonovic:2}. Suffix \textit{-an}\textsubscript{DERIV} is prosodically active: it always imposes the same pattern, overwriting the \isi{prosody} (including the \isi{vowel length}) of the stem (\textit{UljAn} `made of oil' vs \textit{UUlje} `oil'). Suffix \textit{-an}\textsubscript{INFLECT} is accentless, i.e. neutral: the result of suffixation bears a \isi{prosodic pattern} which has already been present in the paradigm of the base.

\subsubsection{\textit{-at}}\label{sec:simonovic:422}

\textit{-at}\textsubscript{DERIV} derives \isi{denominal} adjectives with the structure base\textsubscript{N}\textit{-at}, and the interpretation of having the denotation of the base noun to a large extent. It has the following quantitative distribution in the corpus.

\begin{table}
\caption{Adjectives in \textit{-at}\textsubscript{DERIV} in the corpus}
\label{tab7}
 \begin{tabular}{ l l l l l}
\lsptoprule
 Base type & N & V & A & Phrase\\
\midrule
Example  &  zUb-At & / & /  & /   \\
  & `toothy' & / & / & / \\
& (zUUb `tooth') & / & / & / \smallskip	\\
Prosodic behaviour & Attracting & / & / & /\smallskip \\
Number of items & 17 & 0 & 0 & 0 \\
 \lspbottomrule
 \end{tabular}
\end{table}

The use of \textit{-at}\textsubscript{DERIV} is additionally constrained by one phonotactic and one semantic restriction on bases: the bases are strictly monosyllabic, and all denote body parts.
The participial \textit{-at}\textsubscript{INFLECT} always has a \isi{prosodic pattern} that exists elsewhere in the verbal paradigm (typically in the \isi{past participle}).
\begin{table}
\caption{Participles in \textit{-at}\textsubscript{INFLECT} }
\label{tab8}
 \begin{tabular}{ l l l l l}
\lsptoprule
\textsc{inf} & \textsc{pres.1sg} & \textsc{pst.ptcp} & \textsc{pass.ptcp} &  Gloss\\
\midrule
prepOznAti & prepOznAAm & \textbf{prEpoznao} & \textbf{prEpoznaat} &`recognise'\\
prOdAti&prOdAAm & \textbf{prOdao} & \textbf{prOdaat} &	`sell'\\
porAvnAti & porAvnAAm & \textbf{pOravnao} & \textbf{pOravnaat} & `flatten'\\
 \lspbottomrule
 \end{tabular}
\end{table}

This situation leads to the same generalisation as with the suffix \textit{-an}. The two uses of the same suffix, \textit{{-at}}\textsubscript{DERIV}  and \textit{{-at}}\textsubscript{INFLECT}  are surface-distinguishable:
\begin{itemize}
\item \textit{-at}\textsubscript{DERIV} is always part of a rising span (\textit{zUbAt} `toothy'), while \textit{-at}\textsubscript{INFLECT}  never is (\textit{prOdaat} `sell.\textsc{pass.ptcp}');
\item \textit{-at}\textsubscript{INFLECT} is always part of a long syllable ({prOdaat} `sell.\textsc{pass.ptcp}'), whe\-re\-as \textit{-at}\textsubscript{DERIV} always surfaces in a short syllable (\textit{zUbAt} `toothy').
\end{itemize}


All in all, the prosodic behaviour of \textit{-at } is as expected: {\textit{{-at}}}\textsubscript{INFLECT} is accentless and {\textit{{-at}}}\textsubscript{DERIV} always imposes the same \isi{prosodic pattern}, deleting the \isi{prosody} of the stem (e.g. removing the \isi{vowel length} of \textit{zUUb} `tooth' in \textit{zUbAt} `toothy').

\subsubsection{\textit{-en}}\label{sec:simonovic:423}
\largerpage%longdistance
The two \isi{adjectival} suffixes that we have considered so far display prosodic behaviour that neatly fits the tendencies outlined in \sectref{sec:simonovic:2}. The situation is somewhat less black-and-white with the suffix \textit{-en}, which shows relatively higher productivity in derivation.

\textit{-en}\textsubscript{DERIV} derives adjectives from bases of different categories and yields four different prosodic patterns. With phrasal bases, the stress falls on the final syllable of the first member of the phrasal base (\textit{jednO-cIfr-en} `one-digit'), which indicates that the initial syllable of the second member, which heads the construction, bears a \isi{high tone}. We only found one example with an \isi{adjectival} base, and it is a rather unique form that is in a suppletion relation with its own base (\textit{mAl-En} `little' cf. the definite form \textit{mAAl-ii} `little'). This one example, as well as a vast majority of \isi{denominal} adjectives derived by the suffix \textit{{-en}}\textsubscript{DERIV}, show a stress-attracting behavior of the suffix similar to that of {\textit{{-an}}}\textsubscript{DERIV} and {\textit{{-at}}}\textsubscript{DERIV}. All such cases involve a monosyllabic base. In six cases – all with polysyllabic nouns as bases – {\textit{{-en}}}\textsubscript{DERIV} shows a neutral behavior (the derived \isi{adjective} has the \isi{accent} pattern of the base). Tellingly, in all such cases, the \isi{stress pattern} of the base is not stem-final (e.g. \textit{IzlOžb-a} `exhibition', \textit{IzlOžb-en} `related to an exhibition'), so that the accent-attracting property of the suffix would have caused a \isi{stress shift} (*\textit{izlOžb-En}). Finally, in two cases {\textit{{-en}}}\textsubscript{DERIV} erases the lexical specification of the \isi{prosody} of the base – hence the derived \isi{adjective} receives the default \isi{prosody} (short falling initial \isi{accent}). In sum, {\textit{{-en}}}\textsubscript{DERIV} displays several patterns, out of which the most frequent one is the same as that of {\textit{{-an}}}\textsubscript{DERIV} and {\textit{{-at}}}\textsubscript{DERIV}.

\begin{table}
\caption{Adjectives in \textit{-en}\textsubscript{DERIV} in the corpus}
\label{tab1_d}
 \begin{tabular}{ l l l l l}
\lsptoprule
 Base type & N & V & A & Phrase\\
\midrule
Example  &  rAž-En & / & mAl-En &  dvOsmIsl-en   \\
& `made of rye' & / & `little' & `ambiguous'  \\
& (rAAž  & / & (mAAlii  & (dvAA smIIsla \\
& `rye') & / & `little') & `two senses')\smallskip\\
Prosodic behaviour & Attracting (59) & / & Attracting & Pre-stressing\\
& Neutral (6)&&&\\
& Erasing (2)\smallskip&&&\\
 Number of items & 67 & 0 & 1 & 10\\
 \lspbottomrule
 \end{tabular}
\end{table}

This behavior suggests an interplay between syntactic and phonological factors in the assignment of \isi{prosody}. On the syntactic side, there seems to exist a sensitivity to complexity (phrasal vs. simplex bases) and to categorial specifications (nouns vs. adjectives). On the phonological side, the length of the base seems to play a role. Taking a more detailed look reveals another generalisation: \textit{{-en}}\textsubscript{DERIV} never shifts the stress of the base to another syllable (but it can delete the H and the \isi{vowel length} of the base). This is also true of all cases of \textit{-an}\textsubscript{DERIV} and
\textit{-at}\textsubscript{DERIV}, simply due to the fact that these two always combine with monosyllabic stems.


\largerpage%longdistance
Passive participles derived using \textit{{-en}}\textsubscript{INFLECT} always have a \isi{prosodic pattern} that exists elsewhere in the verbal paradigm, as illustrated in \tabref{tab10}. The only exception is formed by four `rising' classes, where the suffix seems accent-erasing, yet without affecting \isi{vowel length}. These are illustrated in \tabref{tab11}.

\begin{table}
\caption{Participles in \textit{-en}\textsubscript{INFLECT}}
\label{tab10}
 \begin{tabular}{ l l l l l}
\lsptoprule
\textsc{inf}&\textsc{pres.1sg}&\textsc{pst.ptcp}&\textsc{pass.ptcp}& Gloss
\\
\midrule
\textbf{vAditi} & \textbf{vAdiim} & \textbf{vAdio} & \textbf{vAđen} & `take out' \\
stvOrIti & \textbf{stvOriim} & stvOrIo & \textbf{stvOren} & `create' \\
otvOrIti & \textbf{OtvOriim} & otvOrIo & \textbf{OtvOren} & `open' \\
odlUUčIti & \textbf{OdlUUčiim} & odlUUčIo & \textbf{OdlUUčen} & `decide'
\\
\lspbottomrule
 \end{tabular}
\end{table}


\begin{table}
\caption{Participles in \textit{-en}\textsubscript{INFLECT} in `rising' classes}
\label{tab11}
 \begin{tabular}{ l l l l l}
\lsptoprule
\textsc{inf}&\textsc{pres.1sg}&\textsc{pst.ptcp}&\textsc{pass.ptcp}& Gloss
\\
\midrule
lOmIti & lOmIIm & lOmIo & \textbf{lOmljen} & `break' \\
žElEti & žElIIm & žElEo & \textbf{žEljen} & `want'\\
trUUbIti & trUUbIIm & trUUbIo & \textbf{trUUbljen} & `honk'  \\
žIIvEti & žIIvIIm & žIIvEo & \textbf{žIIvljen} & `live'
\\
\lspbottomrule
 \end{tabular}
\end{table}

In accounting for this pattern, we should take into account that in most analyses \textit{-en}\textsubscript{INFLECT} attaches to verbal bases that include the theme vowel \textit{-i-} which becomes consonantal and causes iotation of the stem-final consonant. The standard analysis is that \textit{lomljen} corresponds to the underlying /lomi+en/ which first becomes /lomjen/. As in these verbs the underlying H seems to originate on the theme vowel \textit{-i-}, it seems plausible for the H to disappear together with the syllabicity feature of the vowel. As a result, the form remains without an underlying H and therefore surfaces with a short \isi{falling accent}: \textit{lOmljen}. A functional gain of such a change is that distinguishability is improved: participles are kept different from \isi{denominal} forms with \textit{{-en}}\textsubscript{DERIV}, such as \textit{rAžEn} `made of rye'.


\largerpage
\sloppy
Summarising the picture, \textit{{-en}}\textsubscript{INFLECT}  and \textit{{-en}}\textsubscript{DERIV}  are not surface-distin\-guish\-able either in polysyllables (\textit{OtvOr-en} `open.\textsc{pass.ptcp}' vs. \textit{OpOrb-en}  `related to opposition') or in monosyllables (\textit{smIšljen} `conjecture.\textsc{pass.ptcp}' vs. \textit{smIslen} `meaningful'). At the same time, in short \textit{en-}participles there seems to exist an active process that enforces distinguishability between them and the main pattern in \isi{denominal} derivations.
The prosodic behaviour of \textit{-en}\textsubscript{INFLECT} and \textit{-en}\textsubscript{DERIV} shows a partial overlap. \textit{-en}\textsubscript{INFLECT} is neutral or erasing, whereas \textit{-en}\textsubscript{DERIV} is neutral, erasing or attracting. The observed pattern still exhibits an asymmetry and still in the expected direction since  the \isi{derivational} suffix \textit{-en}\textsubscript{DERIV} is more accented than the \isi{inflectional} \textit{-en}\textsubscript{INFLECT}.\footnote{Accent-attracting and accent-bearing \textit{-en}\textsubscript{INFLECT} are attested in some of the inherited verbal classes which lack a theme vowel, which were excluded from our corpus due to the fact that the morphological structure of the \isi{participle} is opaque. The peculiar pattern which we report without analysing it here is that, at least for some speakers, \isi{perfective} verbs display accent-bearing \textit{-en}\textsubscript{INFLECT}, whereas their \isi{imperfective} counterparts display an accent-attracting version of {\textit{-en}}\textsubscript{INFLECT}. In the example below, we show the feminine version of the \isi{passive participle} in order to illustrate the contrast.
\fussy
\ea
		\begin{xlist} \ex `grind.\textsc{pfv}': \gll
		sAmlEti	sAmEljeem	sAmlEo samlev-En-A \\
        \textsc{inf} \textsc{pres.1sg} \textsc{pst.ptcp} \textsc{pass.ptcp}
\\
\ex `grind.\textsc{ipfv}': \gll  mlEti	mEljeem	mlEo mlEv-En-a
\\ \textsc{inf} \textsc{pres.1sg} \textsc{pst.ptcp} \textsc{pass.ptcp} \\
		\glt
	\end{xlist}
	\zlast}

\section{Common pattern}\label{sec:simonovic:5}

The prosodic behaviour of the four affixes analysed in \sectref{sec:simonovic:4} fits the generalisation that \isi{derivational} affixes are more prosodically dominant than \isi{inflectional} affixes. However, the commonalities seem to go even further. The dominant prosodic behavior is essentially the same for the four observed suffixes: it can be modelled by assuming an underlying representation with a \isi{high tone} and the capacity to erase (parts of) the \isi{prosody} of the stem. As mentioned above, this ability is somewhat more limited for the \isi{adjectival} suffixes \textit{-an}\textsubscript{DERIV}, \textit{-at}\textsubscript{DERIV} and \textit{-en}\textsubscript{DERIV}, which can delete the length and tone of the base, but cannot cause a \isi{stress shift}. On the other hand, \textit{-VVje}\textsubscript{DERIV}  seems to leave no traces of the base \isi{prosody} whatsoever, also shifting the stress position of the base. The representation of the four suffixes would then be along the following lines: \begin{itemize}
\item /-VVje\textsubscript{H}/ + \isi{deletion of base tone}, \isi{vowel length} and stress,
\item /-an\textsubscript{H}/ + \isi{deletion of base tone} and \isi{vowel length},
\item /-en\textsubscript{H}/ + \isi{deletion of base tone} and \isi{vowel length},
\item /-at\textsubscript{H}/ + \isi{deletion of base tone} and \isi{vowel length}.
\end{itemize}

Implementing such representations would account for the fact that e.g. the base \textit{unaprEEdIti }/unapreedi\textsubscript{H}ti/ `promote.\textsc{\isi{pfv}.inf}' loses both its \isi{vowel length} and its H in \textit{unapređEEnjE} /unapređeenje\textsubscript{H}/ `promotion'. Such a solution would be similar to what \citet{Mar2002} proposes as the underlying representation of the \ili{Slovenian} nominalising suffix \textit{-ost}, which we repeat in \REF{ex:simonovic:5}. Note that this bracket insertion amounts to overwriting the stress of the base.

\ea \label{ex:simonovic:5} \gll -ost \\ Delete stress on the stem, insert a bracket at the right edge of the stem: \\...* * *(
		\\
\z

\noindent In addition to these very elaborate underlying representations, we would need another mechanism that prevents these lexical prosodic specifications from surfacing in inflection. In \citet{Mar2002}, this is spell-out which proceeds in phases, in \citet{Arsim2013}, this is Lexical Conservatism which enforces the preservation of the base \isi{prosody} in paradigm members. However, what both approaches seem to leave unaccounted for is the fact that all the \isi{derivational} uses of different suffixes cause the same pattern: in \ili{Slovenian} it is stem-final stress, in \ili{Serbo-Croatian} it is stem-final H. Given the strikingly similar prosodic behaviour in the four suffixes, a prefered explanation would be that the prosodic behavior of the suffix depends entirely on the structure in which this suffix occurs, not only in the case of \isi{inflectional} uses, but also in the \isi{derivational} ones. According to such an explanation, all the suffixes we consider here would underlyingly be without any \isi{prosodic prominence} and they would behave as accented (\ili{Slovenian}) and accent-attracting (\ili{Serbo-Croatian}) due to prominence which they receive when occurring in a particular structure. In the next section, we consider what the implementation of such a solution would entail.

\section {Theoretical implementation}\label{sec:simonovic:6}
Distributed Morphology offers an insightful way to distinguish between morphological structures. As outlined in \cite{Mar2002}, the difference between \textit{nóvost} `newness' and \textit{novóst} `novelty' would be in their structural complexity, as shown in \figref{fig1}. The difference in \isi{prosody} would then naturally follow due to the phasal spell-out.


\begin{figure}
\centering
    \begin{forest}
    [nP
    [n, [ost, name=ost]]
   [aP
        [a, [∅, name=∅]]
        [√P, [\textsc{nov}]]
        ]
        ]
    \end{forest}
    \begin{forest}
    [nP
    [n, [ost, name=ost]]
   [√P, [\textsc{nov}]]]
    \end{forest}
    \caption{\textit{nóvost} `newness' and \textit{novóst} `novelty'}
    \label{fig1}
\end{figure}


Two important predictions that this approach makes are: \begin{itemize}
\item In root nominalisations (e.g. \textit{novóst}), the suffix can impose (idiosyncratic) selectional requirements on bases with which it combines and the pattern therefore has limited productivity.
\item The meaning of the root nominalisations cannot be compositionally derived from the meanings of their parts.
\end{itemize}

Both of these predictions seem to be born out. However, the same model makes some less desirable predictions:
\begin{itemize}
\item The nouns derived by means of the unstressed \textit{-ost} are expected to have a compositional interpretation. There are, however, clear exceptions, e.g. \textit{znán-ost} `science' is clearly related to \textit{znán} `known' but its meaning cannot be derived from that of \textit{znán} compositionally.
\item The root \isi{nominalisation} analysis predicts \isi{relative} freedom of the stressed \textit{-óst} in combining with roots which otherwise surface as verbs, nouns or do not surface independently. This is unfortunately not born out. Out of the very few \textit{-ost}/\textit{-óst} nouns which have roots which do not surface as independent adjectives, some have the stressed \textit{-óst} (e.g. \textit{krepóst} `virtue') but others have the unstressed \textit{-ost} (e.g. \textit{kakóvost} `quality').
\end{itemize}

The same problems of the root-derivation analysis extend to our Serbo-Cro\-a\-tian data. The three \isi{adjectival} suffixes whose base category we examined show a clear tendency to select nominal bases. Finally, and most importantly, a phasal-spellout account seems unable to model the assignment of prominence by the structure and leaves us with several suspiciously similar underlying representations of different affixes.

What is necessary, then, is an alternative which would allow for the \isi{prosody} of the \isi{derivational} versions of the affixes to be assigned by the structure.  We believe that a viable alternative can be offered by \cite{Revithiadou1999}. If the distinction between \isi{derivational} and \isi{inflectional} affixes is in headhood (only \isi{derivational} affixes being heads), then \textsc{HeadFaith} (a \isi{faithfulness constraint} which protects \isi{lexical prominence} of syntactic heads) is already sufficient to produce the asymmetry between \isi{inflectional} and \isi{derivational} uses. This would still mean that we have to stick to all the affixes having an underlyingly specified H, and the account would be as strong as those presented by \cite{Mar2002} or \citet{Arsim2013}. However, \textsc{HeadStress} (a \isi{markedness constraint} that militates against prominence on non-heads) can get us further. In stress systems, this constraint can enforce adding epenthetic stress to a head that has no lexically sponsored prominence (e.g. in the \ili{Slovenian} \textit{nov-óst} `novelty'). In \ili{Serbo-Croatian}, this constraint can enforce the epenthesis of a \isi{high tone}, in e.g. \textit{unapređEEnjE} `promotion'.

A final piece of the puzzle is the fact that at least in our \ili{Serbo-Croatian} data set, the nominal affix\textit{-VVje}\textsubscript{N} overrides the prosodic specification of the base more radically than the \isi{adjectival} affixes \textit{-en}\textsubscript{A}, \textit{-an}\textsubscript{A} and \textit{-at}\textsubscript{A}: the former is able to cause stress shifts with respect to the surface \isi{prosody} of the base. We believe that this is a consequence of a cross-linguistic tendency for nominal content to receive more prominence than other categories, which has been discussed in the literature under the rubric of Noun privilege (see \citealt{Smi2011} for a discussion). While constraints enforcing Noun privilege have been proposed for roots which surface as nouns, there is no reason not to extend them to nominalising affixes.

\section{Conclusions}\label{sec:simonovic:7}
We have analysed four \ili{Serbo-Croatian} affixes which occur both as \isi{derivational} and as \isi{inflectional}. We provided an account that mutually relates their most prominent semantic, structural and prosodic properties in a systematic way, thus supporting the view that these cases indeed manifest different uses of the same suffix rather than pairs of homonymous suffixes. In each of the four cases, we compared the \isi{inflectional} and the \isi{derivational} uses of the suffix, sharing the same target category, yet with differences in interpretation that can be derived from the different contexts.
The prosodic patterns of the derived words confirm the initial generalization that \isi{derivational} affixes are more prosodically prominent than \isi{inflectional} affixes. We speculated about both functional and formal mechanisms behind this regularity. Our tentative analysis lends support to the interface model presented in \cite{Revithiadou1999}.


\section*{Abbreviations}


\begin{tabularx}{.45\textwidth}{@{}lX@{}}
\textsc{\textsc{√}}&root\\
\textsc{1}&1st person\\
\textsc{2}&2nd person\\
\textsc{3}&3rd person\\
\textsc{a}&{adjective}\\
\textit{\textit{a}}&{adjectival} category\\
\textsc{deriv}&{derivational}\\
\textsc{gen}&genitive\\
\textsc{H}&{high tone}\\
\textsc{inf}&{infinitive}\\
\textsc{inflect}&{inflectional}\\
\end{tabularx}
\begin{tabularx}{.45\textwidth}{@{}lX@{}}
\textsc{\textsc{ipfv}}&{imperfective}\\
\textsc{n}&noun\\
\textit{\textit{n}}&nominal category\\
\textsc{nom}&nominative\\
\textsc{\textsc{pass}}&passive\\
\textsc{\textsc{pcpt}}&{participle}\\
\textsc{\textsc{pfv}}&{perfective}\\
\textsc{\textsc{pres}}&present\\
\textsc{\textsc{pst}}&past\\
\textsc{pl}&plural\\
\textsc{sg}&singular\\
\end{tabularx}

\section*{Acknowledgements}

We are thankful to the audience of FDSL 12.5 and the two anonymous reviewers for the useful comments and suggestions. We acknowledge financial support from the Slovenian Research Agency (program No. P6-0382).


\sloppy
\printbibliography[heading=subbibliography,notkeyword=this]



\newpage
\section*{Appendix}\label{secapp}
This appendix contains the annotated corpus material for the observed \isi{adjectival} suffixes.

\begin{table}[h]
\caption{\textit{an}-adjectives with nominal bases and no segmental change}
\label{tabapp1}
 \begin{tabularx}{\textwidth}{ l X l l}
\lsptoprule
\textit{an}-\isi{adjective} &  & Base &
\\
\midrule
brOnz-An & `made of bronze' & brOOnz-a & `bronze'\\
zEmlj-An & `made of soil' & zEmlj-A & `soil'\\
grOžđ-An & `made of grape' & grOOžđ-e & `grape' \\
Ulj-An & `made of oil' & UUlj-e & `oil' \\
cIgl-An & `made of brick' & cIIgl-A & `brick' \\
tAft-An & `made of taffeta' & tAft & `taffeta'\\
plEh-An & `made of tin' & plEh & `tin'\\
gIps-An & `made of plaster' & gIps & `plaster'\\
cIc-An & `made of textile' & cIc & `textile' \\
plIš-An & `made of velvet' & plIš & `velvet'\\
rAž-An & `made of rye' & rAAž & `rye'\\
štOf-An & `made of cloth' & štOf & `cloth'\\
zvjEzd-An & `starry' & zvijEEzd-A & `star'\\
 \lspbottomrule
 \end{tabularx}
\end{table}

\begin{table}
\caption{\textit{an}-adjectives with nominal bases and iotation}
\label{tabapp2}
\begin{tabularx}{\textwidth}{ l X l l}
\lsptoprule
 \textit{an}-\isi{adjective} &  & Base &
\\
\midrule
sUnč-An & `sunny' & sUUnc-e & `sun'\\
žIvč-An & `nervous' & žIIvAc & `nerve'\\
dAšč-An & `made of bars' & dAsk-A & `bar' \\
kOnč-An & `made of thread' & kOnAc & `thread' \\
brOnč-An & `made of bronze' & brOOnc-a & `bronze' \\
lAnč-An & `chain-like' & lAAnAc & `chain'\\
nOvč-An & `related to money' & nOvAc & `money'\\
pUpč-An & `umbilical' & pUpak & `belly button'\\
pUšč-An & `related to rifle' & pUšk-a & `rifle' \\
tRšč-An & `made of cane' & tRsk-a & `cane'\\
nEpč-An & `palatal' & nEpc-E & `palate'\\
brOjč-An & `made of numbers' & brOOjk-a & `number'\\
sRč-An & `brave' & sRc-e & `heart'\\
mOžd-An & `brain-related' & m(O)Ozak & `brain'\\
žIč-An & `made of wire' & žIc-a & `wire'\\
svEč-An & `celebrative' & svEEtAk & `holiday'\\
vOšt-An & `made of wax' & vOsak & `wax'\\
kOšt-An & `related to bones' & kOOst & `bone'\\
zUpč-An & `geary' & zUUbAc & `gear'\\
pjEšč-An & `made of sand' & pijEEsAk & `sand'\\
 \lspbottomrule
 \end{tabularx}
\end{table}

\begin{table}
\caption{\textit{an}-adjective with a verbal base}
\label{tabapp3}
\begin{tabularx}{\textwidth}{ l X l l}
\lsptoprule
\textit{an}-\isi{adjective} &  & Base & \\
\midrule
pIj-An & `drunk' & pI-ti & `drink' \\
 \lspbottomrule
 \end{tabularx}
\end{table}

\begin{table}
\caption{\textit{an}-adjective with an adjectival base and no segmental modifications}
\label{tabapp4}
\begin{tabularx}{\textwidth}{ l X l l}
\lsptoprule
\textit{an}-\isi{adjective} &  & Base & \\
\midrule
mEk-an & `soft' & mEk & `soft' \\
 \lspbottomrule
 \end{tabularx}
\end{table}

\begin{table}
\caption{\textit{an}-adjectives with an adjectival base and iotation}
\label{tabapp5}
\begin{tabularx}{\textwidth}{ l X l l}
\lsptoprule
\textit{an}-\isi{adjective} &  & Base & \\
\midrule
mlAđ-An & `young' & mlAAd & `young' \\
 \lspbottomrule
 \end{tabularx}
\end{table}

\begin{table}
\caption{\textit{at}-adjectives with nominal bases and no segmental changes}
\label{tabapp6}
\begin{tabularx}{\textwidth}{ l X l l}
\lsptoprule
\textit{at}-\isi{adjective} &  & Base &    \\
\midrule
nOg-At & `who has big feet' & nOgA & `foot'
\\  krAk-At & `who has big limbs' & krAAk & `limb'
\\  Uh-At & `who has big ears' & Uho & `ear'
\\  brAd-At & `who has a big beard' & brAAdA & `beard'
\\  bRk-At & `who has a big mustache' & bRRk & `mustache'
\\  rOg-At & `who has big horns' & rOOg & `horn'
\\  glAv-At & `who has a big head' & glAAvA & `head'
\\  gUz-At & `who has a big bottom' & gUUz & `bottom'
\\  lEđ-At & `who has a big back' & lEEđA & `back'
\\  plEć-At & `who has big shoulders' & plEćA & `shoulder'
\\  nOs-At & `who has a big nose' & nOOs & `nose'
\\  pRs-At & `who has a big chest' & pRsa & `chest'
\\  sIs-At & `who has big tits' & sIsa & `tit'
\\  zUb-At & `who has big teeth' & zUUb & `tooth'
\\  krIl-At & `who has big wings' & krIIlO & `wing'
\\  rEp-At & `who has a big tail' & rEEp & `tail' \\
 \lspbottomrule
 \end{tabularx}
\end{table}

\begin{table}
\caption{\textit{at}-adjectives with iotized nominal bases}
\label{tabapp7}
\begin{tabularx}{\textwidth}{ l X l l}
\lsptoprule
 \textit{at}-\isi{adjective} &  & Base & \\
\midrule
kOšč-At & `who has big bones' & kOska & `bones' \\
 \lspbottomrule
 \end{tabularx}
\end{table}

\begin{table}
\caption{\textit{en}-adjectives with monosyllabic nominal bases and no segmental changes}
\label{tabapp8}
\begin{tabularx}{\textwidth}{ l X l l}
\lsptoprule
 \textit{en}-\isi{adjective} &  & Base & \\
\midrule
bAkr-En & `made of copper' & bAkAr & `copper'
\\  bOrb-En & `related to fight' & bOrb-A & `fight'
\\  tvOrb-En & `related to making' & tvOrb-A & `making'
\\  drUštv-En & `sociable' & drUUštv-O & `society'
\\  dvOjb-En & `related to dilemma' & dvOjb-A & `dilemma'
\\  glAzb-En & `related to music' & glAzb-A & `music'
\\  hImb-En & `pretentious' & hImb-A & `pretending'
\\  kRzn-En & `made of fur' & kRRzn-O & `fur'
\\  kOpn-En & `related to soil' & kOpn-o & `soil'
\\  pApr-En & `related to pepper' & pApAr & `pepper'
\\  jEčm-En & `made of barley' & jEčAm & `barley'
\\  Ovs-En & `related to oat' & OvAs & `oat'
\\  Ognj-En & `made of fire' & OgAnj & `fire'
\\  plAtn-En & `made of canvas' & plAAtn-O & `canvas'
\\  slUžb-En & `official' & slUžb-A & `service'
\\  stAkl-En & `made of glass' & stAkl-O & `glass'
\\  sUkn-En & `made of cloth' & sUUkn-O & `cloth'
\\  svOjstvE-n  & `characteristic' & svOOjstv-O & `property'
\\  vApn-En & `made of limestone' & vAApn-O & `limestone'
\\  vAtr-En & `made of fire' & vAtr-a & `fire'
\\  dRv-En & `made of wood' & dRv-o & `wood'
\\  glln-En & `made of clay' & glIIn-A & `clay'
\\  gUm-En & `made of rubber' & gUm-a & `rubber'
\\  lAn-En & `made of flax' & lAn & `flax'
\\  lEd-En & `related to ice' & lEEd & `ice'
\\  slAm-En & `made of straw' & slAm-a & `straw'
\\  lIm-En & `made of tin' & lIm & `tin'
\\  mEd-En & `made of honey' & mEEd & `honey'
\\  svIl-En & `made of silk' & svIIl-A & `silk'
\\  vOd-En & `made of water' & vOd-A & `water'
\\  vUn-En & `made of wool' & vUn-a & `wool'
\\  pUt-En & `fleshy' & pUUt & `flesh'
\\
 \lspbottomrule
 \end{tabularx}
\end{table}

\begin{table}
\caption{\textit{en}-adjectives with monosyllabic nominal bases and no segmental changes (cont'd)}
\label{tabapp9}
\begin{tabularx}{\textwidth}{ l X l l}
\lsptoprule
\textit{en}-\isi{adjective} &  & Base &   \\
\midrule
sAn-En & `related to dream' & sAn & `dream' \\
  mIsl-En & `related to thought' & mIIsao & `thought' \\
  pAkl-En & `related to hell' & pAkAo & `hell' \\
  cRkv-En & `related to church' & cRRkv-a & `church'\\
  gRl-En & `related to throat' & gRl-o & `throat' \\
  Igl-En & `related to needle' & Igl-A & `needle'\\
  jEtr-En & `made of liver' & jEtr-a & `liver' \\
  rAž-En & `made of rye' & rAAž & `rye' \\
  kAzn-En & `related to punishment' & kAzn-a & `punishment' \\
  kIčm-En  & `related to spine' & kI(I)čm-a & `spine' \\
  svAdb-En & `related to wedding' & svAdb-A & `wedding' \\
  Usn-En & `related to lips' & Usn-a & `lip' \\
  zdrAvstv-En & `related to health' & zdrAvstv-O & `health' \\
  žAlb-En & `related to complaint' & žAlb-A & `complaint' \\
  žRtv-En & `related to sacrifice' & žRRtv-a & `sacrifice' \\
  kAv-En-ii/kAf-En-ii & `related to coffee' & kAv-A/kAf-A & `coffee' \\
  zOb-En & `made of oat' & zOOb & `oat' \\
  mArv-En & `related to cattle' & mAArv-a & `cattle' \\
  pIsm-en & `literate' & pIIsm-O & `letter' \\
smIsl-en & `sensible' & smIIsao & `sense' \\
 \lspbottomrule
 \end{tabularx}
\end{table}

\begin{table}
\caption{\textit{en}-adjectives with polysyllabic nominal bases and no segmental changes}
\label{tabapp10}
\begin{tabularx}{\textwidth}{l X l l}
\lsptoprule
\textit{en}-\isi{adjective} &  & Base & \\
\midrule
božAnstv-En & `wonderful' & božAnstv-O & `deity' \\
  jedInstv-En & `unique' & jedIInstv-O & `unity' \\
  dostojAnstv-En & `with dignity' & dostojAnstv-O & `dignity' \\
  veličAnstv-En & `great' & veličAnstv-O & `greatness' \\
  knjigovOdstv-En-ii & `related to bookkeeping' & knjigovOdstv-O & `bookkeeping' \\
  prvEnstv-En & `primary' & prvEnstv-O & `priority' \\
  ubIstv-En & `related to murder' & ubIIstv-O & `murder'\\
  kOsItr-en & `made of tin' & kOsItar & `tin' \\
  mOlItv-en & `related to prayer' & mOlItv-a & `prayer' \\
  OdrEdb-en & `specificational' & OdrEdb-a & `specification' \\
  pOrEdb-en & `\isi{comparative}' & pOrEdb-a & `comparison' \\
  IzlOžbe-n & `exhibitional' & IzlOžb-a & `exhibition' \\
  OpOrb-en & `oppositional' & OpOrb-a & `opposition' \\
 \lspbottomrule
 \end{tabularx}
\end{table}

\begin{table}
\caption{\textit{en}-adjective with a nominal bases and iotation of the base}
\label{tabapp11}
\begin{tabularx}{\textwidth}{ l X l l}
\lsptoprule
 \textit{en}-\isi{adjective} &  & Base &      \\
\midrule
gvOzd-En & `made of iron' & gvOOžđ-e & `iron' \\
 \lspbottomrule
 \end{tabularx}
\end{table}

\begin{table}
\caption{\textit{en}-adjective with an adjectival base}
\label{tabapp12}
\begin{tabularx}{\textwidth}{ l X l l}
\lsptoprule
\textit{en}-\isi{adjective} &  & Base &  \\
\midrule
mAl-En & `small' & mAAl-ii & `small'\\
 \lspbottomrule
 \end{tabularx}
\end{table}

\begin{table}
\caption{\textit{en}-adjectives with phrasal bases}
\label{tabapp14}
\begin{tabularx}{\textwidth}{ l X l l}
\lsptoprule
\textit{en}-\isi{adjective} &  & Base &    \\
\midrule
bezAzl-En & `harmless' & bez zlA & `without evil' \\
dvOsmIsl-en & `ambiguous' & dvAA smIIsla & `two meanings'\\
bEskIčm-en & `spineless' & bez kI(I)čmee & `without spine'\\
lakOmIsl-en & `impetuous' & lAka mIIsao & `light thought'\\
bEsmIsl-en & `senseless' & bez smIIsla & `without sense'  \\
bEspOsl-en & `idle' & bez pOslA & `without job' \\
zApOsl-en & `employed' & za pOslOm & `for job' \\
UpOsl-en  & `busy' & u pOslU & `in job' \\
jednOcIfr-en & `single-digit' & jEdnA cIfra & `one digit' \\
dvOcIfr-en & `double-digit' & dvEE cIfre & `two digits' \\
 \lspbottomrule
 \end{tabularx}
\end{table}
\il{Serbo-Croatian|(}
\end{document}
