\documentclass[output=paper]{langsci/langscibook} 
\ChapterDOI{10.5281/zenodo.1181789}
\author{
Luke Galea\affiliation{University of Malta}\lastand
Adam Ussishkin\affiliation{University of Arizona}
}
\title{Onset clusters, syllable structure and syllabification in Maltese}
%\shorttitlerunninghead{}
%\ChapterDOI{} %will be filled in at production
\abstract{This chapter aims to describe syllable structure and the phonotactic constraints on onset consonants in Standard Maltese. The current work is based on the phonetic and phonological description of Maltese in \citet{azzopardi1981phonetics}
% \footnote{(azzopardi1981phonetics) is an unpublished dissertation.}
and \citet{maltese_book}. The phonological account provided here, however, is grounded in an Onset-Rhyme model. Furthermore, the phonotactics of Maltese are described in terms of sonority. After establishing the nature of onset consonants in Maltese, we address the process of syllabification in Maltese. }
\maketitle

%Forest
\usepackage{forest}
%For long down arrows
\newcommand{\xdownarrow}[1]{%
  {\left\downarrow\vbox to #1{}\right.\kern-\nulldelimiterspace}
}

\begin{document}

\section{Maltese syllable structure} 

Before describing the possible \isi{syllable} structures in Maltese, it is important to highlight that Maltese monosyllables are restricted by complementary quantity. This means that in \isi{monosyllabic} words (cf. \tabref{tab:galea:1}), short vowels are either followed by a \isi{geminate} (G) or by a \isi{consonant cluster} (CC), and long vowels are followed by a single \isi{consonant} but never by a \isi{geminate} \citep{azzopardi2002vowel}. This does not mean that open syllables in Maltese do not occur; however, they are not restricted by quantity. 

\afterpage{
\begin{table}
\begin{tabularx}{\textwidth}{XXXX}
\lsptoprule
CVG & [hɐpp] &\textit{ħabb}& ‘he loved’\\
CVCC & [tɐlp] &\textit{talb}& ‘prayer’\\
CV:C & [kɐ:p]& \textit{kap}& ‘boss’\\
\lspbottomrule
\end{tabularx}
\caption[Complementary quantity in Maltese]{Complementary quantity in Maltese\protect\footnotemark}
\label{tab:galea:1}
\end{table}
\footnotetext{As noted by a reviewer, some non-standard varieties might have different forms. Furthermore, traditionally the digraph ‘għ’ in Maltese is linked to \isi{vowel} lengthening \citep[as discussed in][] {azzopardi1981phonetics}. However, a thorough phonetic/\isi{phonological} study on this has not been carried out.}
}%END AFTERPAGE

Therefore, in Maltese the \isi{syllable} types V:G and V:CC do not occur due to this complimentary quantity restriction, and as a result are not found in \isi{syllable} structures with added onsets or codas.

\citet{azzopardi1981phonetics} and \citet{maltese_book} present the possible \isi{syllable} types in Maltese. They argue that the minimal \isi{syllable} requirement is a \isi{vowel}. The maximum number of onset consonants is three and the maximum number of \isi{coda} consonants is two. Thus, a maximal Maltese \isi{syllable} would have the shape (C)(C)(C)V(C)(C). 

A clearer picture of the possible \isi{syllable structure} of Maltese is presented in \citep[48]{camilleri2014stem}, who discusses \isi{syllable} structures that occur as monosyllables and within word forms. We extend \citet{camilleri2014stem}’s list of possible \isi{syllable} structures, adding additional structures to that list, in order to provide an exhaustive list (cf. \tabref{tab:galea:2}) of the possible \isi{syllable} structures (both as monosyllables and within word forms). Therefore, the possible \isi{syllable} structures listed in \tabref{tab:galea:2} are based on \citet{azzopardi1981phonetics}, \citet{maltese_book} and \citet{camilleri2014stem}. What is presented in this chapter is a first attempt at fully capturing the possible syllabic structures (both onsets and codas) in Maltese \citep[some of this work appears in][]{lukediss}. However, our focus in this paper is on onsets, and the description in \tabref{tab:galea:2} is split into four categories: 1) \isi{vowel-initial syllable} structures: \textit{V-initial}, 2) one-\isi{consonant onset} \isi{syllable} structures: \textit{C-initial}, 3) two \isi{consonant onset} \isi{syllable} structures: \textit{CC-initial} and 4) three \isi{consonant onset} \isi{syllable} structures: \textit{CCC-initial} to show the syllabic nature of onsets in Maltese. A --- in \tabref{tab:galea:2} refers to forms that do not occur as either monosyllables or within-word forms.

\afterpage{
\begin{table}
\caption{Possible syllable types and onset distribution in Maltese} 
\label{tab:galea:2}
\begin{tabularx}{\textwidth}{lQQp{3cm}c}
\lsptoprule

\bfseries Initial & \bfseries Syllable type & \bfseries Monosyllable & \multicolumn{2}{c}{\bfseries Within-word forms}\\
\midrule
V & V\footnotemark{} & [ʊ]  

\textit{hu} ‘he’ & \multicolumn{2}{p{3cm}}{[ʊ.hut] 
\newline
\textit{uħud}\newline ‘some’}\\
\tablevspace
& VG & [ɔmm] 
\newline
\textit{omm} \newline‘mother’ & {} --- & \\
\tablevspace
& VCC & [ɛlf] 
\newline
\textit{elf} \newline‘thousand’ & {} --- & \\
\tablevspace
& VCCC\footnotemark{} & [ɪntʃ] 
\newline
\textit{int=x}
\newline
you.2sg/2pl=neg

‘aren’t you’ & {} --- & \\
\tablevspace
& V:C & [ɐ:f] 
\newline
\textit{af}
\newline
know.3sg
\newline
‘know’ & \multicolumn{2}{p{3cm}}{[ɐ:f.sɐ] 
\newline
\textit{għafsa}\newline ‘a squeeze’}\\
\tablevspace
& V:

(e.g., V:CVC) & {}--- & \multicolumn{2}{p{3cm}}{[ɛ:.mɛs] 
\newline
\textit{għemeż}
\newline
wink.3sgm.perf
\newline
‘he winked’}\\
\tablevspace
& VC 

(e.g., VC.CVC) & {}--- & \multicolumn{2}{p{3cm}}{[ɔr.bɔt]\newline
\textit{orbot}\newline ‘tie (imp.)’}\\
\lspbottomrule
\end{tabularx}
\end{table}
\addtocounter{footnote}{-2}
\stepcounter{footnote}\footnotetext{ This category (V) is problematic as it is not clear whether such a \isi{monosyllable} exists as an autonomous stress bearing unit or not. Furthermore, the language does not provide many examples of this type, which might add to its questionable status.} 
\stepcounter{footnote}\footnotetext{ There is a lack of morpheme-internal triconsonantal codas and the cluster spans two morphemes.} 
}%END AFTERPAGE

\afterpage{
\noindent
\begin{tabularx}{\textwidth}{lXXX}
\lsptoprule
C & CV & [lɛ] 
\newline
\textit{le} 
\newline‘no’ & [lɛ.fɐʔ] 
\newline
\textit{lefaq} 
\newline‘he sobbed’\\
\tablevspace
& CV: & [dʒɪ:] 
\newline
\textit{\.gie} 
\newline‘he came’ & [dʒɪ:.li]~
\newline
\textit{\.gieli} 
\newline‘sometimes’\\
\tablevspace
& CVW & [rɐw]\footnotemark{} 
\newline
\textit{raw}
\newline ‘they saw’ & [rɐw.kɔm] 
\newline
\textit{rawkom}
\newline ‘they saw you’\\
\tablevspace
& CVC & {} --- & [hɐz.bɛt] 
\newline
\textit{ħasbet} 
\newline‘she thought’\\
\tablevspace
& CV:C & [tɐ:f] 
\newline
\textit{taf}
\newline ‘she knows’ & [tɐ:f.nɐ] 
\newline
\textit{tafna} 
\newline‘she knows us’\\
\tablevspace
& CVG & [hɐpp] 
\newline
\textit{ħabb} ‘he loved’ & [tɪn.hɐpp] 
\newline
\textit{tinħabb}
\newline ‘to be loved’\\
\tablevspace
& CVGC & [zɐmmʃ] 
\newline
\textit{żammx}
\newline ‘he didn’t hold’ & [ɪn.zɐmmʃ] 
\newline
\textit{inżammx}
\newline ‘it wasn’t held’\\
\tablevspace
& CVCC & [bɐrt] 
\newline
\textit{bard} 
\newline ‘cold’ & [kɐz.bɐrt] 
\newline
\textit{kasbart}
\newline ‘I disgraced’\\
\tablevspace
& CVCCC & [mɔrtʃ] 
\newline
\textit{mortx} 
\newline‘didn’t go’ & \textit{{}---}\\
\lspbottomrule
\end{tabularx}
\footnotetext{There is disagreement in the literature on whether this \isi{vowel} is a long \isi{vowel} or not (see \citealt{borg1978historical}:231; and \citealt{camilleri2014stem}:48). However, neither of these studies investigated this issue empirically and we suggest that this would be the best way of resolving the issue.} 
}%END AFTERPAGE

\afterpage{
\noindent
\begin{tabularx}{\textwidth}{lXXX}
\lsptoprule
CC & CCV & [blɐ] 
\newline
\textit{bla}
\newline ‘without’ & [stɐ.hɐ] 
\newline
\textit{staħa}
\newline ‘he was shy’\\
\tablevspace
& CCV: & [kju:] 
\newline
\textit{kju} 
\newline‘queue’ & [kpɪ:.pɛl] 
\newline
\textit{kpiepel} ‘hats’\\
\tablevspace
& CCVW & [tfɛw] 
\newline
\textit{tfew} 
\newline‘they switched sth off’ & [tfɛw.kɔm]
\newline
\textit{tfewkom} 
\newline ‘they outshone you’\\
\tablevspace
& CCVC & {}--- & [ftɐh.tʊ]
\newline
\textit{ftaħtu }
\newline ‘I opened it’\\
\tablevspace
& CCV:C & [frɐ:k] 
\newline
\textit{frak}
\newline ‘crumbs’ & [kni:s.jɐ] 
\newline
\textit{knisja }
\newline‘church’\\
\tablevspace
& CCVG & [frɔtt] 
\newline
\textit{frott} 
\newline‘fruit’ & [ʊ.zu:.frʊtt] 
\newline
\textit{użufrutt }
\newline‘usufruct’\\
\tablevspace
& CCVGʃ\footnotemark{} & [ʔbɐttʃ] 
\newline
\textit{qbatx} 
\newline‘didn’t catch’ & [ɪn.ʔbɐttʃ] 
\newline
\textit{inqbadtx }
\newline‘I didn’t get caught’\\
\tablevspace
& CCVCC & [frɪsk] 
\newline
\textit{frisk}
\newline ‘fresh’ & {}---\\
\tablevspace
& CCVCCC & [hsɪltʃ] 
\newline
\textit{ħsiltx}
\newline ‘didn’t wash’ & [ɪn.hsɪltʃ] 
\newline
\textit{inħsiltx} 
\newline ‘I didn’t shower’\\
\lspbottomrule
\end{tabularx}
\footnotetext{ This \isi{syllable structure} can only occur through \isi{morphological inflection}, through the addition of the negative \isi{suffix} /-ʃ/.} 
}%END AFTERPAGE

\noindent
\begin{tabularx}{\textwidth}{lXXX}
\lsptoprule
CCC & CCCV: & [strɔ:] 
\newline
\textit{straw} 
\newline‘straw’ & [zbrɐ:.nɐ] 
\newline
\textit{żbrana}
\newline ‘he exploded’\\
\tablevspace
& CCCVW & [ʃtrɐw] 
\newline
\textit{xtraw} 
\newline‘they bought’ & [ʃtrɐw.nɐ] 
\newline
\textit{xtrawna}
\newline ‘they bought us’\\
\tablevspace
& CCCVC & {}--- & [strɐm.bɐ] 
\newline
\textit{stramba }
\newline ‘odd (fem.)’\\
\tablevspace
& CCCV:C & [sptɐ:r] 
\newline
\textit{sptar}
\newline ‘hospital’ & {}---\\
\tablevspace
& CCCVCC & [strɐmp] 
\newline
\textit{stramb} 
\newline‘odd (m)’ & {}---\\
\tablevspace
& CCCVG & [ftrɐkk] 
\newline
\textit{f’trakk} 
\newline‘in a truck’ & {}---\\
\lspbottomrule
\end{tabularx}

Focusing on the structures CVW, CCVW and CCCVW, \citet{camilleri2014stem} claims that the \isi{vowel} before syllable- or word-final glides (/w, j/) is always a short \isi{vowel}. Therefore, following \citeauthor{camilleri2014stem}’s (\citeyear{camilleri2014stem}) description, this creates the possible \isi{syllable} structures CVC, CCVC, CCCVC, where the \isi{coda} \isi{consonant} is always a \isi{glide}. We do not fully commit to \citeauthor{camilleri2014stem}’s (\citeyear{camilleri2014stem}) claim because sequences such as [ɐw], [ɛw], [ɐj] and so on are what \citet{azzopardi1981phonetics} and \citet{maltese_book} consider to be diphthongs. Therefore, the rhyme of the \isi{syllable} is a \isi{vowel} plus a transition to another \isi{vowel} or a \isi{glide} \citep[cf.][]{azzopardi1981phonetics}. Bearing this in mind, it is not clear whether the \isi{vowel} before is short or not. Since there are no empirical studies that show the phonetic realizations of diphthongs in Maltese, we consider these structures to be of the type C(C)(C)VW, where W stands for the glides /w, j/. A \isi{glide} is part of the nucleus, because if it were a separate \isi{consonant} we would predict \isi{vowel} lengthening since a short \isi{vowel} plus a \isi{coda} \isi{consonant} would violate the bimoraic minimum on \isi{syllable} nuclei (e.g., compare [tɐw] \textit{taw} ‘they gave’ and [rɐ:t] \textit{rat} ‘she saw’). 

\newpage
The list of possible \isi{syllable} structures presented in \tabref{tab:galea:2} differ from those proposed by \citet{camilleri2014stem}. \citet{camilleri2014stem} lists the \isi{syllable structure} CCV: as occurring only as a within-word form but not as a \isi{monosyllable}. \citet{camilleri2014stem} illustrated this type through the word /kni:sja/ \textit{knisja} ‘church’. We disagree with this description as following the \isi{syllabification} process in Maltese (which is discussed in §2), the /s/ serves as a \isi{coda} to the previous \isi{syllable} (and not as an onset to the following \isi{syllable}); therefore, the \isi{syllable structure} of the word /kni:sja/ \textit{knisja} is not CCV:.CCV but CCV:C.CV. In the list in \tabref{tab:galea:2}, we provide the example /kju:/ \textit{kju} ‘queue’ (another possible example is /blu:/ \textit{blu} ‘blue’), which show that that the structure CCV: can also occur as a \isi{monosyllable}.\footnote{Nonetheless, these are open empirical questions, which should be investigated in production studies.}  

Two structures are not reported by \citet{camilleri2014stem}. First, the structure CCCV: in /strɔ:/ \textit{straw} ‘straw’ occurs both as a \isi{monosyllable} and within words. Secondly, a long \isi{vowel}, V:, can occur as a \isi{syllable} within words, e.g., /ɛ:/ in /\textbf{ɛ:}.mɛs/ \textit{għemeż} ‘he winked’ or /ɐ:/ in /\textbf{ɐ:}.fɐs/ \textit{\.għafas} ‘he pressed’. In (C)CVGC, the C following the \isi{geminate} is restricted to the occurrence of the \isi{morpheme} /-ʃ/ used for negation as in the examples: [ɪn.\textbf{zɐmmʃ}] \textit{inżammx} ‘it was not held’ and [ɪn\textbf{.ʔbɐttʃ}] \textit{inqbadtx} ‘I didn’t get caught’, or /-s, -z/ as a \isi{suffix} for English-origin plurals; e.g., /klɐpps/ \textit{clubs} ‘clubs’. Furthermore, the \isi{syllable type} C(C)VCCC as in the examples (from \tabref{tab:galea:2}) [mɔrtʃ] \textit{mortx} ‘I didn’t go’ and [hsɪltʃ] \textit{ħsiltx} ‘I didn’t wash’ (and other words which include these syllables) are limited to the 1\textsuperscript{st} person and 2\textsuperscript{nd} person negative inflected forms.

In the following subsections, we describe the phonotactic constraints of each \isi{syllable structure} type from \tabref{tab:galea:2} in detail. Specifically, we address both phonetic and \isi{phonological} issues of each \isi{syllable structure} type. The description of the \isi{permissible onset} consonants is achieved through the principles of \isi{sonority} \citep[for codas cf.][]{lukediss}. In this work, we adopt the \isi{sonority scale} below. Furthermore, we also adopt Selkirk’s Sonority Sequencing Principle \citep{selkirk1984major}, which requires a \isi{sonority} rise between a left-margin constituent and the \isi{syllable} peak: 

\eabox{
\begin{tabular}{llll}
Vowels & > glides & > sonorants & > obstruents\\
High Sonority & & & Low Sonority \\
\end{tabular}
}


\subsection{The nucleus}


All vowels in Maltese can serve as a \isi{syllable nucleus}. As a matter of fact, the language allows vowels on their own to occur as a permissible \isi{syllable}. This is restricted to a few words, typically function words and often \isi{unstressed}, such as /ɪ/ \textit{hi} ‘she’, /ʊ/ \textit{hu} ‘he’ or \textit{u} ‘and’, some exclamations such as /ɔ:/ ‘oh’, but also, less frequently, content words such as /ɐ:/ ‘confusion’. 


\subsection{Vowel-initial syllable structures} 

It is debatable whether Maltese allows onsetless syllables. The phonetic realization of onsetless syllables shows that vowels are variably preceded by an epenthetic \isi{glottal} stop, which constitutes a \isi{syllable onset}; e.g., /ʊ/ $\rightarrow$ [ʔʊ] \textit{hu} ‘he’ \citep{azzopardi1981phonetics}. As a matter of fact, \citet{maltese_book} claim that this insertion is more likely to happen in utterance-initial or in post-pause position.\footnote{\citet{lukediss} provides similar results to this claim. Some speakers seem to insert a \isi{glottal} stop before the epenthetic \isi{vowel} before word-initial geminates. This might suggest that, at least for some speakers, \isi{glottal} stops before vowels are required by their \isi{phonology}.}  This might suggest that the preferred \isi{syllable structure} in Maltese requires onsets (i.e., .CV…), and this type of \isi{epenthesis} occurs in spoken \ili{Arabic} dialects and dialects of English and \ili{German}. To illustrate, syllables in \ili{Arabic} always require an onset. If syllables lack an onset, a \isi{glottal} stop is inserted (cf. Standard \ili{Arabic}, \ili{Egyptian} \ili{Arabic}: \citet{gadoua2000consonant}; Cairene \ili{Arabic}: \citet{wiltshire1998extending,youssef2013place}). The preceding context triggers the insertion of a \isi{glottal} stop; \citet{wiltshire1998extending} argued that when the \isi{definite article} is in phrase-initial position an epenthetic \isi{glottal} stop is always inserted, as in [ʔil.mu.dar.ris] ‘the teacher’. This observation is also put forward by \citet{youssef2013place}, who claimed that in Cairene \ili{Arabic}, the \isi{definite article} /il/ is always preceded by an epenthetic \isi{glottal} stop: [ʔil]. 

Historically, Maltese had a \isi{voiced} pharyngeal approximant [ʕ], which is no longer present in current Maltese though it is represented in the orthography by the digraph <għ>. \citet{borg1997maltese} and \citet{brame1972abstractness} argue that vowels adjacent to \isi{orthographic} <għ> are lengthened, whereas \citet{puech1979parlers} argues that this \isi{vowel duration} is context-dependent. \citet{hume2009vowel} investigated this observation by recording two native speakers of Maltese. They investigated whether the vowels adjacent to <għ> are lengthened in a variety of positions within the word. Focusing on absolute phrase-initial position, \citet{hume2009vowel} argued that there is increased \isi{vowel duration} in the <għ> context in \isi{monosyllabic} words; e.g., in a minimal pair such as [ɐ:tt] \textit{għadd} ‘he counted’ and [ɐtt] \textit{att} ‘act’, they showed that the duration of the \isi{vowel} /ɐ/ is longer in the <għ> context. Nonetheless, even though they had a number of vowel-initial syllables in their corpus, \citet{hume2009vowel} did not report whether there were any \isi{glottal} stop insertions before the \isi{vowel}. 

To sum up, potentially underlyingly vowel-initial syllables in Maltese might actually be phonetically realized as .CV…, where the C is an epenthetic \isi{glottal} stop. If this is true, there are no truly vowel-initial syllables in the language, because the epenthetic \isi{glottal} stop serves as an onset to a \isi{vowel-initial syllable}. Words that have \isi{orthographic} <għ> or <h> in absolute initial position tend to have longer adjacent vowels. However, only the durations of vowels adjacent to <għ> have been investigated empirically \citep{hume2009vowel}. Furthermore, <għ>-initial words would be preceded by a \isi{glottal} stop. Related evidence to this can be found in orthography where <għ>-initial words are occasionally misspelled by literate native speakers with the letter <q> (the grapheme used to represent \isi{glottal} stop); e.g., <qandi> instead of <għandi> ‘I have’. This evidence shows that some speakers consider that the \isi{glottal} stop is part of the \isi{phonology} of these words. However, production studies need to be carried out to fully understand this phenomenon. 


\subsection{Permissible onsets in Maltese}

Almost all consonants in the inventory of Maltese constitute permissible single onsets; examples are listed in \tabref{tab:galea:3} below. The status of the phone /ʒ/ in Maltese is unclear \citep[cf.][]{maltese_book}. It occurs in some loan words such as [tɛlɛvɪʒɪn] \textit{televixin} ‘television’, where the \isi{voiced} post-alveolar \isi{fricative} constitutes an onset to the final \isi{syllable}. Furthermore, it can occur as part of onset clusters such as [ʒbi:ɐ]\footnote{This [ʒ] is only \isi{voiced} because it is C\textsubscript{1} in a CC onset in which C\textsubscript{2} is \isi{voiced}, thus triggering regressive \isi{voicing assimilation}, which operates in Maltese onset clusters and is discussed later on this section.}  \textit{xbiha} ‘image’; however, there are no \isi{monosyllabic} words which have [ʒ] as a single \isi{onset consonant}. In all of the examples presented in \tabref{tab:galea:3}, there are no \isi{sonority} violations in the \isi{onset consonant}. The structure conforms to the SSP, since a single \isi{consonant} is always less sonorous than a \isi{vowel} as the nucleus.

\afterpage{
\begin{table}
\caption{Simple onsets in Maltese}
\label{tab:galea:3}
\begin{tabularx}{.8\textwidth}{Ql}
\lsptoprule

\bfseries \multirow{7}{*}{Stops} & [pɐ:ɹ] \textit{par} ‘pair’ \\
& [bɐ:ɹ] \textit{bar} ‘bar’\\
& [tɐ:ɹ] \textit{tar} ‘he flew’\\
& [dɐ:ɹ] \textit{dar/dahar} ‘back/house’\\
& [kɐ:p] \textit{kap} ‘head of an institution’\\
& [gɔst] \textit{gost} ‘fun’\\
& [ʔɐ:m] \textit{qam} ‘he woke up’\\
\tablevspace

\bfseries \multirow{6}{*}{Fricatives} & [fɐ:ɹ] \textit{far} ‘it overflowed’\\
& [vɐ:ɹɐ] \textit{vara} ‘statue’\\
& [sɐ:ɹ] \textit{sar} ‘it became’\\
& [zɐ:ɹ] \textit{żar} ‘he visited’\\
& [ʃɐ:ɹ] \textit{xahar/xagħar} ‘month/hair’\\
& [hɐll] \textit{ħall ‘}vinegar/ he undid (a knot)’\\

\tablevspace
\bfseries \multirow{4}{*}{Affricates} & [tʃɐ:ɹ] \textit{\.car} ‘clear’\\
& [dʒɐ:ɹ] \textit{\.gar} ‘neighbour’\\
& [tsɔkk] \textit{zokk} ‘branch’\\
& [dzɔ:nɐ] \textit{żona} ‘zone’\protect\footnotemark{}\\

\tablevspace

\bfseries \multirow{2}{*}{Nasals} & [mɐ:ɹ] \textit{mar} ‘he went’\\
& [nɐ:ɹ] \textit{nar} ‘fire’\\

\tablevspace

\bfseries \multirow{2}{*}{Glides} & [wɐʔt] \textit{waqt} ‘during’\\
& [ju:m] \textit{jum} ‘day’\\

\tablevspace

\bfseries \multirow{2}{*}{Liquids} & [lɐ:t] \textit{lat} ‘point of view’\\
& [rɐ:t] \textit{rat} ‘she saw’\\
\lspbottomrule
\end{tabularx}
\end{table}
\footnotetext{For some speakers this is pronounced as [zɔ:nɐ].} 
}% END AFTERPAGE

\subsection{Permissible onset clusters in Maltese}

It is generally claimed that the larger the distance in \isi{sonority} between the \isi{first consonant} (C\textsubscript{1}) and the second \isi{consonant} (C\textsubscript{2}) in a \isi{consonant cluster}, the more well-formed the \isi{onset cluster} is \citep{topintzi2011}. Nonetheless, clusters having the same or similar \isi{sonority} are allowed to occur in sequence in a number of languages, such as \ili{Russian} and Bulgarian. This is referred to as the \textit{Minimum Sonority Distance} principle \citep[cf.][]{selkirk1984major,levin1985metrical, parker2011}. Maltese is one of the languages that allows clusters with minimum \isi{sonority distance}. To compare, \ili{Spanish}, for example, only allows onset clusters which are made up of an \isi{obstruent} and liquid; e.g., /kr/ in /krus/ ‘cross’ \citep{baertsch2002}, which means that onset clusters in \ili{Spanish} have a larger distance in \isi{sonority} between C\textsubscript{1} (e.g., /k/) and C\textsubscript{2} (e.g., /r/). On the other hand, languages such as \ili{Russian}, Bulgarian and Leti allow onset clusters containing consonants which are closer on the \isi{sonority scale}; e.g., /kn/ in \ili{Russian} /kniga/ ‘book’. However, \citet[1168]{parker2011} claims that “if a language permits clusters with a lower \isi{sonority distance}, it allows clusters of all higher distances as well” but not the other way around, which is the case in Maltese. Clusters that have minimum \isi{sonority distance} give rise to plateaus. Sonority plateaus arise when there is no difference in \isi{sonority} between the members of a \isi{consonant cluster} (such as in Maltese /tp/ in /tpɛjjɛp/ ‘you/she smokes’ or /sf/ in /sfɔrts/ ‘effort’). The SSP states that there must be one peak from the onset to the \isi{syllable nucleus}; thus, plateaus in the onset violate the SSP. \citet{blevins1995}, following \citet{otto1904}’s version of the SSP, accounts for such plateaus, whereas other versions of the SSP do not \citep[e.g.][]{selkirk1984major,clements1990role, zec2007syllable}. A \isi{syllable} with an \isi{onset cluster} such as /kl/ in /klɪ:m/ \textit{kliem} ‘kliem’ or /pr/ in /prɛtsts/ \textit{prezz} ‘price’ has a higher \isi{sonority distance}, and this leads to a rising \isi{sonority} profile from the onset to the \isi{syllable nucleus}. In comparison, \isi{consonant} clusters such /kt/ in /ktɪ:b/ \textit{ktieb} ‘book’ or /dv/ in /dvɐljɐ/ \textit{dvalja} ‘table cloth’ lead to a \isi{sonority plateau} and, thus, a possible violation of the SSP.  

In addition to allowing \isi{onset consonant} clusters with very ‘flat \isi{sonority}’ \citep{zec2007syllable}, Maltese also places a constraint on word-initial \isi{tautosyllabic} \isi{consonant} clusters: they are restricted by a \isi{voicing assimilation rule} which operates regressively. Therefore, \isi{consonant} clusters are both \isi{voiced} or both voiceless: e.g., [bdɛw] \textit{bdew} ‘we started’; [pkɪ:t] \textit{bkiet} ‘she cried’. 

To give an example of the range of possible clusters from low \isi{sonority distance} to high \isi{sonority distance}, we show the spectrum of possible \isi{consonant} clusters beginning with /p/ in \tabref{tab:galea:4}. The permissible clusters start from those that have a minimum \isi{sonority distance} (e.g., /pt/, /pk/), which lead to a \isi{sonority plateau}, which are followed by clusters that have a higher \isi{sonority distance} (e.g., /pr/ and /pj/). 

\afterpage{
\begin{table}
\caption{Permissible /p/-initial clusters} 
\label{tab:galea:4}
\begin{tabularx}{\textwidth}{llXl}
\lsptoprule
\bfseries MSD	&	\bfseries Cluster	&	\bfseries Example	&	\bfseries Sonority	\\
\midrule
\bfseries Low	&	/pt/	&	[ptɐ:.lɐ] btala ‘holiday’	&	\multirow{3}{*}{\bfseries Plateau}	\\
\multirow{12}{*}{$\xdownarrow{3cm}$}	&	/pk/	&	[pkɛw] bkew ‘they cried’	&		\\
	&	/pʔ/	&	[pʔɐjt] bqajt ‘I stayed’	&		\\
%\tablevspace	
	&	/ptʃ/	&	[ptʃɛj.jɛtʃ] bċejjeċ ‘pieces’	&	\\
	&	/pts/	&	[ptsɪ:.tsɛn] bżieżen ‘bread rolls’	&		\\
	&	/ps/	&	[psɐrt] bsart ‘I guessed’	&		\\
	&	/pf/\protect\footnotemark	&	[pfɔr.mɐ] b’forma ‘with a shape’	&		\\
	&	/pʃ/	&	[pʃɐ:.rɐ] bxara ‘announcement’	&		\\
	&	/ph/	&	[phɐ:l] bħal ‘like’	&		\\
	&	/pn/	&	[pnɪ:.tsɛl] pniezel ‘brushes’	&		\\
	&	/pl/	&	[plɐt:] platt ‘plate’	&		\\
	&	/pr/	&	[prɛts:] prezz ‘price’	&		\multirow{3}{*}{\bfseries Increase} \\
	&	/pw/	&	[pwɪ:.nɪ] pwieni ‘pains’	&		\\
\bfseries High	&	/pj/	&	[pjɐ:n] pjan ‘plan’	&		\\
\lspbottomrule
\end{tabularx}
\end{table}
\footnotetext{Cluster /pf/ appears only in the case of the \isi{preposition} \textit{b’} before /f/. As one reviewer noted, this type of \isi{sonority} profile is limited to morphologically complex examples (e.g., /fp/ in /fprɔtʃɛss/ \textit{f’pro\.cess} ‘in process’}
} %END AFTERPAGE

\REF{ex:galea:2} lists some examples of minimum distance \isi{sonority} clusters of \isi{voiced} \isi{consonant} clusters:

\ea\label{ex:galea:2}
{Voiced \isi{consonant} clusters}
\ea /bd/ in /bdi:l/ bdil ‘change’\\
\ex /dg/ in /dgɔrr/ tgorr ‘you complain’\\
\ex /zb/ in /zbi:p/ żbib ‘raisins’\\
\z
\z

In the case of higher \isi{sonority distance} onset clusters, Maltese allows: Obstruent + Nasal, Obstruent + Liquid, Obstruent + Glide, as in \REF{ex:galea:3}. 

\newpage 
\ea\label{ex:galea:3}
{Examples of higher \isi{sonority distance} clusters}\\
\ea
{Obstruent + Nasal} \\
/tn/ in /tnɛjn/ tnejn ‘two’\\
/zm/ in /zmɪ:n/ żmien ‘time’\\
\ex
{Obstruent + Liquid} \\
/dl/ in /dlɐ:m/ dlam ‘darkness’\\
  /fr/ in /frɐ:r/ Frar ‘February’\\
\ex
{Obstruent + Glide}\\
/ʔw/ in /ʔwɪ:l/ qwiel ‘idioms’ \\
 /vj/ in /vjɐtʃtʃ/ vja\.g\.g ‘journey’ \\
\z
\z
 
The \isi{voicing assimilation rule} is not strictly respected in clusters beginning with /ʔ/ and /h/. When these consonants occur as C\textsubscript{1} in a CC \isi{consonant cluster}, \isi{voicing assimilation} is violated when C\textsubscript{2} is a \isi{voiced} \isi{obstruent} e.g., /ʔb/ in /ʔbi:l/ \textit{qbil} ‘agreement’ and /hd/ in /hdu:t/ \textit{ħdud} ‘Sundays’. Even though the voicing harmony is violated, the SSP is not; instead, this leads to a \isi{sonority plateau}. In the opposite case, when a \isi{voiced} \isi{obstruent} is in C\textsubscript{1} position and /ʔ/or /h/ is in C\textsubscript{2} (e.g., /bʔ/ in /bʔɐjt/ \textit{bqajt} ‘I stayed’, and /dh/ in /dhu:l/ \textit{dħul} ‘entrances’), such clusters lead to a \isi{sonority} reversal. \citet{maltese_book} claim that the frequency of \isi{consonant cluster} onsets with /ʔ/ and /h/ + \isi{voiced} \isi{obstruent} (e.g., [hd]) is lower than that of CC onsets of /ʔ/ and /h/ + voiceless \isi{obstruent} (e.g., [ht]). Furthermore, /ʔ/ and /h/ also cluster with consonants further up in the \isi{sonority scale} as in \REF{ex:galea:4a}:

\ea\label{ex:galea:4a}
{Consonant clusters with /ʔ/ and /h/ as C\textsubscript{1}}\\
\ea
/ʔl/ in /ʔlu:p/ qlub ‘hearts’
\ex
/ʔr/ in /ʔrɐ:r/ qrar ‘confession’
\ex
/hm/ in /hmɐ:r/ ħmar ‘donkey’
\ex
/hl/ in /hlɐ:s/ ħlas ‘payment’ 
\z
\z

\subsection{Sibilant onset clusters}
Maltese allows sibilant-initial onset clusters. To start with, Maltese permits sibi\-lant-initial clusters which have a high \isi{sonority distance} and do not violate the SSP as in \REF{ex:galea:4}.


\ea\label{ex:galea:4}
{Sibilant onset clusters: high \isi{sonority} distance}\\
\ea
/sr/ in /srɪ:p/ \textit{sriep} ‘snakes’
\ex
/zr/ in /zrɐ:r/ \textit{żrar} ‘coarse aggregate used in concrete’ 
\ex
/ʃm/ in /ʃmu:n/ \textit{Xmun} ‘Simon’
\ex
/ʃl/ in /ʃlɔkk/ \textit{Xlokk} ‘south east’ 
\ex
/zm/ in /zmɛrtʃ/ \textit{żmer\.c} ‘awry’
\z
\z

In \isi{sibilant} \isi{obstruent} clusters, the \isi{voicing assimilation rule} still applies in \isi{sibilant} clusters as in \REF{ex:galea:5}.

\largerpage
\ea\label{ex:galea:5}
{Sibilant onset clusters: Voicing assimilation}\\
\ea
/sk/ in /sku:r/ skur ‘dark’
\ex
/sp/ in /spɪss/ spiss ‘often’
\ex
/ʃt/ in  /ʃtɐ:ʔ/  xtaq ‘he wished’
\ex
/ʃk/ in /ʃkɪ:l/ xkiel ‘obstacle’
\ex
/zb/ in /zbɐll/ żball ‘mistake’
\ex
/zv/ in /zvɔ:k/ zvog ‘vent’ 
\z
\z
\newpage

Clusters such as /sk sp st zb/ in \REF{ex:galea:5}, just like in English and \ili{Italian}, pose a challenge to the Sonority Sequencing Principle since the \isi{sibilant} is more sonorous than the stop (in the first five examples in \REF{ex:galea:5}) and leads to a \isi{sonority plateau} in /zv/.  

The \isi{syllabification} of \isi{sibilant} initial clusters has been a long-standing debate in \isi{phonology}. Numerous approaches have been proposed: approaches which span from the strictly \isi{phonological}, such as \citet{kaye1992you}, to more experimental approaches such as \citet{browman1992}. Experimental evidence suggests that there is not a universal solution to \isi{syllabification}: in some languages, like English \citep{marin2010}, \isi{sibilant} clusters pattern like non-\isi{sibilant} clusters and are considered to be \isi{tautosyllabic}, but in other cases such as \ili{Italian}, sibilant-\isi{obstruent} clusters, unlike obstruent-liquid clusters, are heterosyllabic \citep{hermes2013}. In languages such as \ili{Moroccan} \ili{Arabic}, Tashlhiyt \ili{Berber} and possibly Maltese, sibilant-initial clusters and obstruent-initial clusters are heterosyllabic \citep[see][for a preliminary articulatory study]{hermesetal}.

\subsection{Sonorant-initial clusters}
Maltese has consonantal sequences that have a \isi{sonorant} (/l m n r/) as C\textsubscript{1}. Maltese has combinations of \isi{sonorant} + stop (e.g., /lp/, /md/, /nt/, /rk/), \isi{sonorant} + \isi{fricative} (e.g., /ls/, /ms/, /nz/, /rv/), \isi{sonorant} + \isi{glottal} (e.g., /mʔ/ and /nh/) sequences. However, such sequences violate the SSP, as C\textsubscript{1} is more sonorous than C\textsubscript{2}. Also, such clusters are examples of \isi{sonority} reversals, where C\textsubscript{1} is more \isi{sonorant} than C\textsubscript{2.} In order to avoid this \isi{sonority} reversal one of two strategies can be employed in Maltese. First, \citet{azzopardi1981phonetics} proposes that the realization of sonorants as C\textsubscript{1} in a \isi{consonant} sequence could be syllabic. Thus, /mʔɐ:r/ surfaces as [m̩ .ʔɐ:r] \textit{mqar} ‘at least’. This realization does not violate the SSP because a syllabic \isi{consonant} constitutes its own \isi{syllable nucleus}. The other strategy is to insert a \isi{vocalic element} of [ɪ]-like quality before the \isi{sonorant} \isi{consonant}: [ɪm.ʔɐ:r]. In this case, the \isi{vowel} [ɪ] serves as a \isi{syllable nucleus}, which is followed by the \isi{sonorant} [m], which serves as \isi{coda} to the first \isi{syllable}. In addition, it is possible for a prothetic \isi{glottal} stop to be inserted before the \isi{vocalic element}. If this \isi{glottal} stop were represented in the \isi{phonological} structure, then this would constitute a \isi{syllable onset}. More examples of sonorant-initial clusters are presented in \REF{ex:galea:6}:

\ea\label{ex:galea:6}
{Realization of sonorant-initial clusters}\\
\ea /lp/ $\rightarrow$ [ɪl.pu:p] or [ʔɪl.pu:p] or [l̩.pu:p] lpup ‘wolves’
\ex /md/ $\rightarrow$ [ɪm.di:.nɐ] or [ʔɪm.di:.nɐ] or [m̩.di:.nɐ] Mdina ‘Mdina (name of town)’
\ex /nz/ $\rightarrow$ [ɪn.zi:t] or [ʔɪn.zi:t] or [n̩.zi:t] nżied ‘I add’
\ex /rv/ $\rightarrow$ [ɪr.vɛll] or [ʔɪr.vɛll] or [r̩.vɛll] rvell ‘rebellion’
\ex /mh/ $\rightarrow$ [ɪm.hɐ:r] or [ʔɪm.hɐ:r] or [m̩.hɐ:r] mħar ‘clams’ 
\z
\z

Regardless of which strategy is employed, sonorant-initial clusters in Maltese are never \isi{tautosyllabic}, but rather are always heterosyllabic. 

\subsection{CCC-initial clusters} 
As shown in \tabref{tab:galea:2}, Maltese also allows for tri-consonantal word-initial clusters (abbreviated to CCC-initial). \citet{maltese_book} show that the premitted combinations of consnants are very restricted. C\textsubscript{1} is usually a \isi{fricative} (/s, ʃ, z/) or a bilabial stop (i.e., /p, b/). C\textsubscript{2} can be either an oral stop (i.e., /p, b, t, d, k, g/) or the \isi{fricative} /f/. C\textsubscript{3} tends to be occupied by a \isi{sonorant} but can be filled by any other consonants. It is important to note that \isi{voicing assimilation} still applies in CCC-initial clusters. Furthermore, the prefixes /b-, p-/ ‘with’, /ʃ-/ ‘what’ and /f-/ ‘in’ can contribute to the creation of CCC-initial onsets. In \tabref{tab:galea:2}, we provide the example [ftrɐkk] \textit{f’trakk} ‘in a truck’, where the \isi{first consonant} [f] is a prefix, leading to the triconsonantal cluster [.ftr…]. Additional examples of triconsonantal clusters in Maltese are provided in \REF{ex:galea:7}.

\ea\label{ex:galea:7}
{CCC-initial}\\
\ea 
{[stʔɐrr] stqarr ‘he confessed’}
\ex 
{[zbrɔffɐ] żbroffa ‘he exploded’}
\ex 
{[sptɐ:r] sptar ‘hospital’}
\z
\z

\section{Syllabification in Maltese}%2
According to \citet{maltese_book}, \isi{polysyllabic} words which have one \isi{consonant} in \isi{medial position}, such as CVCVC, are syllabified as CV.CVC, where the \isi{medial consonant} constitutes a \isi{syllable onset} to the following \isi{syllable}, as in \REF{ex:galea:8}. This follows the Maximum Onset Principle (MOP) that a \isi{consonant} flanked between two vowels is more likely to \isi{syllabify} as an onset rather than a \isi{coda} \citep[cf.][]{kahn1976}.

\ea\label{ex:galea:8}
{Syllable division of one medial consonant}\\
\ea {[kɪ.sɛr] kiser ‘he broke’}
\ex {[mɪ:.tʊ] mietu ‘they died’}
\ex {[lɐ:.pɛs] lapes ‘pencil’}
\ex {[tɪ.fɛl] tifel ‘a boy’}
\z
\z

In \isi{polysyllabic} words with structures like CVCCV or CVCCVC, \isi{medial consonant} sequences are not treated as \isi{consonant} clusters as they tend to be syllabified as the \isi{coda} to the preceding \isi{syllable} and the onset of the following \isi{syllable} \citep[cf.][]{azzopardi1981phonetics}. Therefore, the medial cluster in a CVCCV word split across the two syllables (CVC.CV); see \REF{ex:galea:9} for examples. 

\ea\label{ex:galea:9}
{Syllable division of \isi{medial consonant} sequences} \\
\ea {[hɔl.mɐ] ħolma ‘dream’}
\ex {[tɐh.fɛr] taħfer ‘forgiveness’}
\ex {[ʃɔr.tɐ] xorta ‘sameness’}
\ex {[tɔʔ.bɐ] toqba ‘hole’}
\z
\z

The same \isi{syllable} division applies to word-medial geminates as shown in \REF{ex:galea:10}.

\ea\label{ex:galea:10}
{Syllable division of word-medial geminates}\\
\ea {[hɐf.fɛr] ħaffer ‘he dug’}
\ex {[rɐt.tɐp] rattab ‘he softened’}
\ex {[tɛl.lɛf] tellef ‘he disrupted’}
\ex {[ʔɐtʃ.tʃɐt] qa\.c\.cat ‘he removed’}
\z
\z

Word-initial geminates occur due to morphophonological processes; however, they are disallowed phonologically. Word-initial geminates tend to be preceded by an epenthetic \isi{vowel}, which in Maltese is a \isi{vowel} of [ɪ]-like quality (see \citealt{lukediss} for results on the production of the epenthetic \isi{vowel} in different conditions across a number of speakers). For this reason, we assume that word-initial geminates in Maltese, like word-medial geminates, are ambisyllabic, where the first part of the \isi{geminate} serves as a \isi{coda} to the previous \isi{syllable} and the second part of the \isi{geminate} serves as an onset to the following \isi{syllable}. Therefore, underlying word-initial geminates surface as word-medial geminates and are syllabified in the same way as word-medial geminates; see \REF{ex:galea:11}. 

\largerpage
\ea\label{ex:galea:11}
{Syllable division for word-initial geminates}\\
\ea {/ppɐkkja/ $\rightarrow$ [ɪp.pɐk.kjɐ] ippakkja ‘he packed’}
\ex {/ddɛffɛs/ $\rightarrow$ [ɪd.dɛf.fɛs] iddeffes ‘he poked his nose in s.o. else’s affairs’}
\ex {/ssɛbbɐh/ $\rightarrow$ [ɪs.sɛb.bɐh] issebbaħ ‘he was beautified’}
\z
\z

We argue that \isi{vowel} \isi{epenthesis} before word-initial geminates allows the \isi{syllabification} of stray consonants \citep{ito1986,ito1989}. 

In the case of three-\isi{consonant} sequences in word-\isi{medial position}, \citet{azzopardi1981phonetics} proposed that the preferred \isi{syllabification} of such sequences is as a \isi{consonant} syllabified as a \isi{coda} to the preceding \isi{syllable} followed by a two-\isi{consonant} \isi{onset cluster} to the following \isi{syllable}, as in \REF{ex:galea:12}.

\ea\label{ex:galea:12}
{Syllabification of medial clusters} \\
\ea {[mɐ\textbf{h.fr}ɐ] maħfra ‘forgiveness’}
\ex {[mɪ.nɪ\textbf{s.tr}ʊ] ministru ‘minister’}
\z
\z

It is also possible for such clusters to be syllabified in such a way that the first two consonants constitute a \isi{consonant cluster} in \isi{coda} position, and the third \isi{consonant} constitutes a simple onset in \isi{coda} position, as in \REF{ex:galea:13}.\footnote{We acknowledge that this is highly speculative and the implications of our intuitions need to be emperically investigated.}  

\ea\label{ex:galea:13}
{Syllabification of medial clusters}\\
\ea {[jɐ\textbf{ʔs.m}ʊ] jaqsmu ‘they divide/share’}
\ex {[hlɪ\textbf{st.k}ɔm] ħlistkom ‘I freed you (pl.)’}
\z
\z

There might be a correlation between \isi{syllable} boundary and \isi{morpheme} boundary in examples like [hlɪst.kɔm] \textit{ħlistkom} ‘I freed you (pl.)’, where the \isi{coda} \isi{consonant cluster} [st] belongs to the \isi{verb} and the initial [k] is part of the \isi{clitic}. Yet, this is not the case in [jɐʔs.mʊ]\footnote{A counterexample of this is the 3F \isi{clitic} [ɐ], as in [jɐʔ.sɐm.ɐ] \textit{jaqsamha} ‘he breaks her’, where the \isi{morpheme} constitutes a \isi{syllable} on its own.}  \textit{jaqsmu} ‘they divide/share’, where the \isi{suffix} -ʊ is not placed in a \isi{syllable} of it own. It is possible that in cases where the \isi{morpheme} has a CVC structure (such as /\textbf{k}ɔm/ ‘you (pl.)’, such morphemes could constitute separate syllables. This suggests that \isi{morpheme} boundaries are respected more than \isi{syllable} boundaries, and as a result, this would lead to a division of a sequence of three consonants to CC.C. 

\subsection{Syllabification of sonorant-initial clusters and word-initial geminates} 
As previously described, sonorant-initial clusters and word-initial geminates in Maltese trigger \isi{vowel} \isi{epenthesis} in syllable-initial position \citep{azzopardi1981phonetics,maltese_book}, as in \REF{ex:galea:14}. 

\ea\label{ex:galea:14}
{Insertion before sonorant-initial clusters and word-initial geminates}\\
\ea {/mhɐ:r/ $\rightarrow$ [ɪmhɐ:r] imħar ‘clams’}
\ex {/ʃʃɛjjɛr/ $\rightarrow$ [ɪʃʃɛjjɛr] ixxejjer ‘you/she wave(s)’}
\z
\z

Here, we discuss the role of the epenthetic \isi{vowel} in the \isi{syllabification} of sonorant-initial clusters and word-initial geminates. There seems to be a cross-linguistic consensus on the function of epenthetic vowels: they serve to repair input forms which do not meet a language’s structural requirements \citep{hall2011}. \citet{hall2011} describes three ways in which epenthetic vowels surface. First, following \citet{ito1986,ito1989} \isi{epenthesis} allows the \isi{syllabification} of stray consonants. Second, following \citet{broselow1982}, \isi{epenthesis} is triggered by a particular sequence of consonants. Finally, following \citet{cote2000}, \isi{epenthesis} is triggered by the need to make consonants perceptible. The case of \isi{epenthesis} in word-initial position in Maltese falls into all three categories. Here, we describe how the epenthetic \isi{vowel} in Maltese syllabifies stray consonants. 

First, the location of the epenthetic \isi{vowel} before sonorant-initial and word-initial geminates in Maltese is fixed: the epenthetic \isi{vowel} always precedes a \isi{sonorant}–\isi{initial consonant} cluster (e.g., /nt/, /lt/, /ms/)\footnote{Unless such the sonorants are treated as syllabic.}  or word-initial \isi{geminate} (e.g., /dd/, /vv/, /ss/). As the examples in \tabref{tab:galea:5} show, the epenthetic \isi{vowel} is fixed both in position and also in quality as it always surfaces as a \isi{vowel} of /ɪ/-like quality. 

\begin{table}
\caption{Epenthetic vowel before sonorant-initial consonant clusters and word-initial geminates}
\label{tab:galea:5}
\fittable{
\begin{tabular}{l@{}l@{}lll r@{}l@{}lll}
\lsptoprule
\multicolumn{5}{l}{\bfseries Sonorant \isi{initial consonant} clusters} & 
\multicolumn{5}{l}{\bfseries Word-initial geminates}\\
\midrule 
/\textbf{nf}ɐʔt/ &$\rightarrow$& [ɪ\textbf{n.f}ɐʔt]& \textit{infaqt}& ‘I spent’ & /\textbf{dd}ɐhhal/& $\rightarrow$& [ɪ\textbf{d.d}ɐh.hal]& \textit{iddaħħal}& ‘to be inserted’\\
/\textbf{rb}ɐht/ &$\rightarrow$& [ɪ\textbf{r.b}ɐht]& \textit{irbaħt}& ‘I won’   & /\textbf{vv}ɔtɐ/  &$\rightarrow$ &[ɪ\textbf{v.v}ɔ:.ta]  &  \textit{ivvota} &‘to vote’\\
\lspbottomrule
\end{tabular}
}
\end{table}

Unlike word-initial geminates and sonorant-initial clusters, obstruent-initial clusters do not trigger \isi{epenthesis}. Obstruent + \isi{obstruent} (e.g., /pt, bd, sf/) or \isi{obstruent} + \isi{sonorant} (e.g., /tl, km/) do not trigger \isi{epenthesis} before the \isi{first consonant} or between the two consonants. This is in contrast to other varieties of \ili{Arabic}, which break up word-initial clusters by inserting an epenthetic \isi{vowel} between C\textsubscript{1} and C\textsubscript{2} in the cluster \citep[cf.][]{watson2007,kiparsky2003}. In addition there are other dialects in which the epenthetic \isi{vowel} is before C\textsubscript{1}, e.g., [ismiʕt] ‘I heard’ in Cyrenaic \ili{Arabic} \citep[cf.][]{Mitchell1960}.\footnote{We would like to thank one of our reviewers for pointing out this reference.} 

Following the principle of Prosodic Licensing, which “requires all \isi{phonological} units [to] belong to higher prosodic structure” \citep[3]{ito1986}, \isi{epenthesis} allows the \isi{syllabification} of otherwise unsyllabifiable consonants. Furthermore, the principle of Prosodic Licensing ensures that each segment in the \isi{phonological string} is syllabified. Therefore, for \isi{syllabification} to take place, segments must belong to higher prosodic structures such as syllables. Any segments that are not linked to syllables must be dealt with in order to satisfy Prosodic Licensing. Epenthesis can be explained through the \isi{syllabification} of stray consonants as posited by \citealt{ito1986,ito1989}. Following Itô’s directionality of \isi{syllabification}, we postulate that \isi{syllabification} takes places from right to left. The process of \isi{syllabification} in Maltese allows for Stray Epenthesis \citep{ito1986}, where stray consonants are syllabified precisely because a \isi{vowel} is inserted, providing a new \isi{syllable} for such consonants to be parsed by. Maltese, unlike \ili{Korean} or Attic \ili{Greek}, does not allow for Stray Erasure, where stray consonants are deleted from the \isi{phonological string}. Evidence for this comes from production studies of word-initial geminates in Maltese, which shows that the duration of the \isi{geminate} is longer than that of singletons \citep[cf.][]{galeacues}.

Therefore, the \isi{sonorant} in sonorant-\isi{initial consonant} clusters and the first part of the geminates in word-initial geminates trigger Stray Epenthesis \citep{ito1986}. These segments are not deleted but trigger \isi{epenthesis} as all segments in a \isi{phonological string} have to be syllabified. Following Stray Epenthesis, the \isi{sonorant} in the \isi{consonant} clusters (e.g., /lt/ in \REF{ex:galea:15}) and the first part of the \isi{geminate} (e.g., /ff/ in \REF{ex:galea:16}) become the \isi{coda} of a preceding \isi{syllable}. The epenthetic \isi{vowel} fills in the nucleus of the preceding \isi{syllable} (cf. \REF{ex:galea:16} below). 

\ea\label{ex:galea:15}
{Right-to-left \isi{syllabification} of sonorant-initial clusters}\\
\begin{tabular}{rl}
[ltɪ:m] & ltiem ‘orphan’\\
.tɪ:m & \\
*l.tɪ:m & \\
ɪl.tɪ:m & \\
\end{tabular}
\z

\ea\label{ex:galea:16}
{Right-to-left \isi{syllabification} of word-initial geminates} \\
\begin{tabular}{rl}
[ɪffɪrmɐ] & ffirma ‘to \isi{sign}’\\
.mɐ & \\
fɪr.mɐ & \\
*f.fɪr.mɐ &\\
ɪf.fɪr.mɐ & \\
\end{tabular}
\z

In addition, any of the prefixes that can be added to a \isi{verb} serve as an onset to this added \isi{syllable} (cf. \figref{fig:galea:1}). For instance, the first person \isi{imperfect prefix} /n-/ can only be added before the epenthetic \isi{vowel}, thus a form like \textit{*nffirma} is banned (cf. \REF{ex:galea:17}). The result is a \isi{syllable} with an epenthetic \isi{vowel} as its nucleus and the prefix as an onset. 

\begin{figure}[t]
%\includegraphics[width=.5\textwidth]{figures/a3GaleaUssishkin-img1.png} 
\begin{forest}
 [ ,phantom, s sep=1cm
    [σ
      [O [n,tier=word]]
      [R [N [ɪ,tier=word]]
         [C [f, tier=word,name=galeFig1f]]]
    ]
    [σ
      [O,name=galeFig1O]
      [R [N [ɪ,tier=word]]
	 [C [r,tier=word]]
      ]
    ]
    [σ
      [O [m,tier=word]]
      [R [N [ɐ,tier=word]]]
    ]
 ]
\draw (galeFig1O.south) -- (galeFig1f.north);
\end{forest}
\caption{Syllabification of the inflected verb form [nɪffɪrmɐ] ‘I sign’\\
In the representation of geminates, geminates are associated to the coda and onset slots; and it is assumed that these double associations represent the geminates. Such a representation is widespread within the literature on geminates, and we follow \citet{davis2011geminates} with respect to conventions for geminate representations with respect to syllable structure.}
\label{fig:galea:1}
\end{figure}

\ea\label{ex:galea:17}
{Syllabification of \isi{imperfect prefix} /n-/ ‘n’}\\
\begin{tabular}{rl}
[n-ffɪrmɐ] & niffirma ‘I \isi{sign}’\\
.mɐ & \\
fɪr.mɐ & \\
n\textbf{ɪ}f.fɪr.mɐ & 
\end{tabular}
\z

A reviewer points out that this rule does not explain why \textit{*inffirma} is ruled out given that in Maltese there is a comparable form \textit{nfired} ‘to be separated’. However, Maltese \isi{syllable structure} does not allow for a cluster made up of a \isi{morphological prefix} and a word-initial \isi{geminate} (i.e., such as *i\textbf{nff}irma); on the other hand, it allows for a cluster made up of a \isi{morphological prefix} and a singleton (such as \textit{nfired} ‘to be separate’).

\largerpage
Following \citet{nesporvogel} we take this to be the domain of the \isi{prosodic word} as it consists of a stem (i.e., the \isi{verb}) and a prefix which is added as a result of \isi{morphological inflection} (as in the case of \textit{niffirma} in (20)) or derivation. This is also reinforced by \citet{selkirk1996prosodic}'s proposal that the left and the right edges of words coincide with the left and right edges of the \isi{prosodic word}, which was subsequently adopted for Maltese by \citet{kiparsky2011} and \citet{wolf2011}. Therefore, word-initial geminates which result due to a morphophonological process are part of a single \isi{prosodic word}, as in \REF{ex:galea:18}. 

\let\eachwordone=\upshape
\ea\label{ex:galea:18}
{Prosodic Word (PWd)}\\
\ea
\gll [ɪffɪrmɐ]\textbf{\textsubscript{PWd} }\\ 
ffirma \\
\glt ‘to \isi{sign}’
\ex 
\gll [nɪffɪrmɐ]\textbf{\textsubscript{PWd}} \\
niffirma\\
\glt ‘I \isi{sign}’
\z
\z

Furthermore, the application of Stray Epenthesis applies in phonological-ini\-tial position and when the previous word ends in a \isi{consonant} (as in \REF{ex:galea:19}):

\ea\label{ex:galea:19}
{Syllabification of word-initial geminates}\\
\gll [lu:k.ɪv.vɔ:.tɐ] \\
{Luke~ivvota}\\
\glt  ‘Luke voted’
 \z
 
 In cases where the word before sonorant-initial and word-initial geminates ends in a \isi{vowel}, a number of strategies can be invoked. \citet{hoberman2003verbal} claim that the prothetic \isi{vowel} before word-initial geminates does not occur when the preceding word ends in a \isi{vowel}. We claim that in such cases, we find cross-\isi{morpheme} and cross-word boundary \isi{syllabification}. When a previous word ends in a \isi{vowel}, the stray \isi{consonant} in the following word serves as a \isi{coda} to that \isi{syllable}, which results in cross-word \isi{syllabification}, as in \REF{ex:galea:20}.

\ea\label{ex:galea:20}
{Cross-word \isi{syllabification}: word-initial geminates}\\
\gll [.(ʔ)ɐn.dɐ\textbf{d}.dɐh.hɐl.] \\
{għandha ddaħħal} \\
\glt ‘she has to enter’
\z

Another strategy is Stray Epenthesis, resulting in an inserted \isi{vowel} before the word-initial \isi{geminate}, as in \REF{ex:galea:21}.

\ea\label{ex:galea:21}
{Across word \isi{syllabification}: word-initial geminates}\\
\gll [.(ʔ)ɐn.dɐ.ɪ\textbf{d}.dɐh.hɐl.] \\
{għandha ddaħħal} \\
\glt ‘she has to enter’
\z

On the other hand, unlike sonorant-initial clusters or word-initial geminates, Stray Epenthesis does not operate with \isi{obstruent-initial consonant} clusters. Ob\-stru\-ent-initial consonant clusters are \isi{tautosyllabic} and the \isi{first consonant} is not syllabified as the \isi{coda} of a previous vowel-final word, as in \REF{ex:galea:22}.  

\ea\label{ex:galea:22}
{Onset clusters}\\
\gll [hɐf.\textbf{nɐ.ptɪ:}.hɪ] \\
{ħafna btieħi} \\
\glt ‘a lot of inner courtyards’
 \z
 
 \subsection{Summary} %2.5
In this chapter, we have presented an overview of some of the key phenomena related to the \isi{phonetics} and \isi{phonology} of Maltese syllables. More concretely, we outlined the possible \isi{syllable} structures that can occur as monosyllables and within words in Maltese. As a matter of fact, this can be directly compared with the possible \isi{syllable} structures of some varieites of \ili{Arabic}, \ili{Italian}, and English (the languages from which Maltese originates). Therefore, we propose that a fruitful future study would involve comparing descriptions of \isi{syllable} structures in Maltese and of the languages Maltese originates from. 

This chapter also showed that the possibilities of onset clusters in Maltese are not very heavily restricted. Specifically, Maltese allows for both low \isi{sonority distance} (e.g., /.pt…/) and high \isi{sonority} onset clusters (e.g., /.tl…/). Moreover, in the low \isi{sonority distance} onset clusters, Maltese permits \isi{sonority} reversals and \isi{sonority} plateaus. Therefore, even though the \isi{sonority} framework was used to describe the possible clusters in Maltese, some problems remain. A thorough phonetic analysis using experimental techniques such as an articulography can shed light on the \isi{syllable} affiliation and possible \isi{syllabification} of such different clusters by looking at the \isi{gestural} overlap and the timing of the gestures. 

In comparing onset clusters and word-initial geminates, we have shown that word-initial geminates (e.g., /.pp…/) behave similarly to sonorant-initial clusters (e.g., /.lt…/), where they tend to be preceded by an epenthetic \isi{vowel}. We argued that sonorant-initial clusters and word-initial geminates in Maltese are banned in the \isi{phonology} and the presence of a preceding vocalic insertions leads to a process of resyllabification. 

%\section*{Abbreviations}
\section*{Acknowledgements}
We would like to thank two anonymous reviewers for their comments. Thanks are also due to Marie Alexander, Albert Gatt, Michael Spagnol and Alexandra Vella for reading earlier versions of this paper.

\let\eachwordone=\itshape
\sloppy
\printbibliography[heading=subbibliography,notkeyword=this] 
\end{document}