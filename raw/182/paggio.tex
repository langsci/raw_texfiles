\documentclass[output=paper]{LSP/langsci}
\ChapterDOI{10.5281/zenodo.1181805}
\author{Patrizia Paggio\affiliation{University of Copenhagen and University of Malta}\and
Luke Galea\affiliation{University of Malta}\lastand 
Alexandra Vella\affiliation{University of Malta}}
\title{Prosodic and gestural marking of  complement fronting in Maltese}

\abstract{This paper deals with the use of complement fronting in a
  corpus of Maltese conversations. Four different kinds of
  constructions are distinguished based on the discourse status of the
  fronted complement: focus movement, topicalisation and two types of
  left dislocation. A discussion is carried out of the ways in which
  suprasegmental features, both in terms of prosody and gestures,
  underpin the discourse functions of the four construction types. Our
  findings show that a falling pitch accent is
  %present nearly overall
  nearly always present
  on the fronted complement, and that there is a tendency
  for gestures to accompany this same complement. We also show that
  the four construction types can be ordered on the basis of
  suprasegmental complexity with focus movement as the least complex,
  followed by topicalisation, and finally both types of left
  dislocation as the most complex.}
% Rev PP: replaced last sentence
  %Gestures, however,
  %are more likely to occur in examples involving left dislocation or
  %topicalisation than those displaying focus movement.}

  \maketitle

\usepackage{lipsum}
\begin{document}


\section{Introduction}
%Maltese is a configurational language, i.e. a language in which word
%order is largely determined by information structure rather than
%grammar constraints.
% Rev PP
Maltese is often characterised as a language in which \isi{word order} is
relatively free, and largely determined by information structure
rather than grammar constraints.  The option of placing a sentence
complement sentence-initially, in other words fronting it, is one of
the possibilities available to Maltese speakers to mark this
complement with respect to its discourse and information structure
status.

In this paper, we investigate the use of complement fronting in a
corpus of Maltese conversations. Based on the different types of
\isi{discourse status} carried by the fronted complement in context, we
posit four different kinds of constructions. We then analyse the
prosodic contours of the examples as well as the gestures produced by
the speakers in conjunction with the fronted complement. Our aim is to
%show how suprasegmental features, both in the \isi{prosody} and the
%gestures, underpin the discourse functions of the four construction
%types.
% Rev PP
show how suprasegmental features, such as prosodic and 
\isi{gestural} features, underpin the discourse functions of the four construction
types.


\largerpage
To our knowledge, this is the first study of complement fronting in
% Maltese building on empirical multimodal data.
% Rev PP: added whole paragraph
Maltese building on empirical multimodal data, in other words the
first study using non-constructed data which allow us to study this
phenomenon as it occurs in real conversations, and to include \isi{gestural}
features in the analysis.

It was in fact the availability of the conversational multimodal data,
which will be described below, and the initial observation that
gestures seemed to be very prominent in conjunction with fronted
constituents in those data, which provided the motivation for this
study. It is a generally accepted generalisation that hand gestures,
when they occur, are temporally aligned with the main \isi{sentence accent}
\citep{Kendon80,McNeill92,Loehr04,AlahverdzhievaLascarides10}, which
is in turn associated with sentence focus
\citep{Lambrecht1994,EV_EE95}. %But
However, we are not aware of any previous
attempt at enriching this body of work with knowledge of how gestures
may be used in conjunction with complement fronting, and their
relation to prosodic features in these constructions.


The structure of the paper is as follows. In
\sectref{section:fronting} we define complement fronting and
give an overview of the literature on relevant constructions mostly
based on a discussion of English examples.
% Rev PP: added next sentence
Based on the literature, we distinguish a number of different
constructions all involving complement fronting, i.e. \isi{topicalisation},
\isi{focus movement}, and two types of \isi{left dislocation}.  In
\sectref{section:maltese} we give an account of previous studies
%of this phenomenon in Maltese. We then describe our data in
% Rev PP: added more explanation
of this phenomenon in Maltese, and explain how the examples discussed
in these studies fit the different constructions we are considering.  We
then describe our data in \sectref{section:data}, in particular
%how the data have been analysed from the point of view of \isi{prosody},
% Rev PP: changed and added
how the data have been annotated from the point of view of \isi{prosody},
gestures, and \isi{discourse status}. We also provide some counts of the
annotated categories for each annotation level.
\sectref{section:results}
presents the results, both in terms of quantitative analyses and
qualitative discussions of chosen examples.
% Rev PP
The two different analysis methods serve different purposes. While
frequency counts are presented to make generalisations about how
different features are represented in the different constructions,
qualitative descriptions and discussions of a choice of
representative examples are intended to offer a more detailed
understanding of the data.
Finally, \sectref{section:conclusion} contains the conclusion.

\newpage 
\section{Complement fronting}
\label{section:fronting}

\largerpage
%% Rev PP: note added
{\em Complement fronting} is a \isi{syntactic} mechanism whereby a
non-subject constit\-u\-ent\footnote{The \isi{subject} of a sentence can also be
  fronted in conjunction with \isi{left dislocation}, as will be discussed
  further on. The same is also possible with \isi{subject} extraction as in
  `This I hope will never happen'. The focus of this paper is,
  however, on complement fronting.} is placed {\em sentence-initially}
out of its canonical position, and thereby acquires a special status
in terms of the information structure of the sentence. An example from
English is the song title in example~(\ref{ex:song}a), where the
fronted object is enclosed in square brackets, and the canonical
object position is indicated by an underscore. The non-fronted
counterpart of the same sentence is shown in~(\ref{ex:song}b).


\begin{exe}  
  \ex{}
  \begin{xlist}
  \ex[]{\em $[$This one thing$]$ I know {\longrule}.}
  \ex[]{\em I know this one thing.}
  \end{xlist}
  \label{ex:song}
\end{exe}

The term {\em topicalisation} has often been used to refer to this
construction at least in English, see e.g.  \cite{Lambrecht1994},
based on the fact that the initial position in a sentence is often
occupied by the sentence topic.\footnote{We follow here
  \citet{Lambrecht1994} and many others in understanding {\em topic}
  as that part of the sentence-presupposed information which the rest
  of the sentence predicates something about. %Still
  According to the
  same framework, {\em focus} is defined as the non-presupposed, new
  part of the sentence.}  However, in terms of information packaging
this \isi{syntactic} structure corresponds to at least two different
constructions. One is \isi{topicalisation} proper, in which the fronted
complement indeed corresponds to the sentence topic, while the rest of
the sentence predicates new information about the complement. The
other is a different construction in which the fronted complement
corresponds to the focus of the sentence rather than its topic. The
latter construction has been called {\em focus topicalisation}
\citep{Gundel1974}, {\em focus movement} \citep{Prince1981a}, {\em
  focus preposing} \citep{Vallduvi1992, Ward1996}, and {\em
  linksrhematisierung} \citep{Stempel1981}. In addition to being
different from the point of view of information packaging, in English
the two constructions are also associated with different prosodic
contours \citep{Chafe1976}, in that \isi{topicalisation} exhibits two focal
accents, and \isi{focus movement}\footnote{From here on, we will use the
  term {\em focus movement} to refer to the construction in which the
  fronted complement corresponds to the focus of the
  sentence. However, we are not hereby assuming a transformational
  approach, according to which the complement would be base-generated
  in one position and moved to the front.} only one. Compare
sentences~(\ref{ex:top1}a) and~(\ref{ex:top1}b) below, where small
caps have been added to the phrases that receive focal
\isi{accent}\footnote{In~(\ref{ex:top1}b), small caps are exactly as in the
  original source. In~(\ref{ex:top1}a), on the contrary, they were
  added. Prince uses a graphical notation showing the FALL FALL
  contour characteristic of \isi{topicalisation} constructions in English.}.

\begin{exe}
  \ex{}
  \begin{xlist}
    \ex[]{{\citep[251]{Prince1981a}}\\
      \em {\sc Stardust memories} I saw {\sc yesterday}.}
    \ex[]{{\citep[295]{Lambrecht1994}}\\
      \em {\sc Fifty-six hundred dollars} we raised yesterday.}
  \end{xlist}
  \label{ex:top1}
\end{exe}

The two constructions are also different in terms of their pragmatic
function. The main pragmatic function of topicalised constructions in
English is to mark a partially-ordered set relation, or {\em poset}
relation, between the denotation of the topicalised complement and a
previously evoked discourse entity \citep{Prince1981a}. New information
about this entity is predicated in the open proposition corresponding
to the rest of the sentence. In~(\ref{ex:top1}a), for example, {\em
  Stardust memories} is contrasted with other films and {\em
  yesterday} contributes the new, focal information. In focus
movement, on the other hand, the denotation of the fronted complement
is discourse-new information, and it corresponds in fact to a new
attribute assigned to an otherwise salient \isi{referent} (here, the
amount of money raised).

In addition to \isi{topicalisation} and \isi{focus movement}, a third construction
type needs to be mentioned because it will be relevant to our
discussion of complement fronting in Maltese. This is {\em left
  dislocation}, which in English and other languages is distinguished
from \isi{topicalisation} and \isi{focus movement} both in \isi{syntactic} and pragmatic
terms. Syntactically, the difference consists in the fact that the
fronted constituent (often co-referential with the \isi{subject} of the
sentence), is resumed by a \isi{pronoun} that occurs in the canonical
position this constituent would have in the non-dislocated counterpart
of the sentence. Even though \isi{left dislocation} often involves the
detachment of a \isi{subject}, complement dislocation is also possible, as
shown by the following example (coindexation indices are ours):

\begin{exe}  
  \ex{}
  {\citealt[27]{GregoryMichaelis2001}}\\
           {\em $[$Smiley Burnette$]_i$, I don't remember if you were old enough to remember $[$him$]_i$.}
  \label{ex:LD11}
\end{exe}

According to some authors \citep{Lambrecht2001, GregoryMichaelis2001},
the main pragmatic function of \isi{left dislocation} is to promote a
discourse-new \isi{referent} to topic status. Since the initial position in
a sentence, however, is `reserved' for topical information, the
expression denoting the discourse-new \isi{referent} is detached from the
rest of the sentence by means of \isi{syntactic} as well as prosodic
means. The rest of the sentence contains a \isi{pronoun} that is
coreferential with the dislocated constituent, and in fact if this
constituent is dropped, the sentence is still
well-formed. \citet{geluykens1992} describes \isi{left dislocation} as an
interactional device for introducing referents. In his analysis, the
left dislocated expression is a complete move which calls for
acknowledgement from the listener, as shown by the fact that it is
often associated with a falling tone, and followed by a prosodic
boundary and a pause. However, another type of \isi{left dislocation} has also been
described \citep{Prince1997, geluykens1992, GregoryMichaelis2001}
where the dislocated object involves a {\em poset} relation, similarly
to what happens in \isi{topicalisation} constructions.

\citet{Lambrecht2001} notes that a dislocated constituent may also be
%coindexed with an \isi{inflectional} \isi{affix} in \ili{Romance}, \ili{Bantu} and,
% Rev PP
coindexed with an \isi{affix} in \ili{Romance}, \ili{Bantu} and,
interestingly for the present study, \ili{Semitic} languages. He quotes the
following example from Classical \ili{Arabic}, in which the \isi{clitic} \isi{pronoun}
{\em hu} refers back to {\em Halid} in the
%initial sentence position (`A' in the gloss stands for accusative):
initial sentence position 
%(the gloss of the example is reproduced from
%the original source and follows slightly different conventions
%compared to the Maltese examples)
(the glossing of the example -- including the separation into morphemes -- is our adaptation of the original to the conventions used here):

\ea\label{ex:LD:ara}
\langinfo{Classical Arabic}{}{\citealt[109]{Moutaouakil1989}}\\
\gll Halid-un, q\~{a}bal-tu-hu l-yawm-a\\
     Halid.{\sc nom} met.{\sc 1sg}$>${\sc 3sg} the-day.{\sc acc}\\
\glt `Halid, I met him today.'
\z


%\ea\label{ex:LD:ara}
%\langinfo{Classical Arabic}{}{\citealt[109]{Moutaouakil1989}}\\
%\gll Halidun, q\~{a}baltuhu l-yawma\\
%     Halid$_{nom}$ met-1sg-3sgA the-day$_{acc}$\\
%\glt `Halid, I met him today.'
%\z

In Moutaouakil's original account, the fronted complement is
categorised as being the {\em theme}, which the author describes as a
predication-external pragmatic function, to be distinguished from
topic, %that
which is predication-internal.

\section{Complement fronting in Maltese}
\label{section:maltese}

The literature on complement fronting in Maltese is relatively
sparse. \citet{BorgAlexander2009} give an account of \isi{topicalisation},
which they describe as a process whereby constituents are moved to the
leftmost initial position in the sentence, away from their canonical
position. One of the examples they give is
in~(\ref{ex:cat1}),\footnote{Maltese examples are glossed following the
  Leipzig glossing rules (\url{https://www.eva.mpg.de/lingua/resources/glossing-rules.php}). Thus,
  `-' separates segmentable morphemes, but is also used in Maltese
  writing, and therefore in the examples, to attach the \isi{definite article}
  to the relevant \isi{noun}, `=' separates a \isi{clitic}, including the definite
  article in the gloss ({\sc def}), and `.' is used to list
  non-segmentable meta-linguistic elements. A list of the abbreviations
  used is provided at the end of this paper.} where {\em il-ġurdien} `the mouse'
is fronted, as opposed to what the same authors call ``an unmarked
reporting of the same situation'' (p.72) in~(\ref{ex:cat2}). The
fronted version of this example also shows the use of the
\isi{pronominal} \isi{clitic} {\em u} attached to the main \isi{verb}, which agrees in
number and gender with the fronted object.


\ea\label{ex:cat1}
{\citep[71]{BorgAlexander2009}}\\
\gll Il-ġurdien, il-qattus-a qabd-it=u.\\
     {\sc def}=mouse.{\sc sg.m} {\sc def}=cat.{\sc sg}-{\sc f} caught.3.{\sc prf}-3.{\sc sg.f}=3.{\sc sg.m}\\
\glt `As for the mouse, the cat caught it.'
\z

\ea\label{ex:cat2}
{\citep[72]{BorgAlexander2009}}\\
\gll Il-qattus-a qabd-et il-ġurdien.\\
     {\sc def}=cat.{\sc sg.f}  caught.3.{\sc prf}-3.{\sc sg.f} {\sc def}=mouse.{\sc sg.m}\\
\glt `The cat caught the mouse.'
\z

%     {\def}=mouse.{\sg}.{\m} {\def}=cat.{\sg}.{\f} caught.3.{\prf}-3.{\sg}.{\f}=3.{\sg}.{\m}\\

A number of examples are given in this work to illustrate that under
certain conditions not only object complements, but also adverbials,
prepositional complements, and even subjects can be fronted, and that
chains of fronted constituents are also possible, as
in~(\ref{ex:sister}).

\ea\label{ex:sister}
{\citealt[76]{BorgAlexander2009}}\\
\gll Jien, oħt=i, l-ittra, ktib-t=hie=l=ha lbieraħ.\\
     I sister.{\sc sg.f}=1.{\sc sg.poss} {\sc def}=letter.{\sc sg.f} wrote.{\sc prf}-1.{\sc sg}=3.{\sc sg.f}={\sc indr}=3.{\sc sg.f} yesterday\\
\glt `I, my sister, the letter, I wrote it to her yesterday.'
\z

Crucially, the authors claim that this type of construction, which
they call \isi{topicalisation}, is characterised in Maltese by a specific
prosodic contour, in that i) the fronted constituent constitutes
its own tone group starting on a High pitch on the first stressed
\isi{syllable} and moving to a Low pitch on the last stressed \isi{syllable}; ii)
the rest of the sentence can receive an unmarked intonation pattern
with nuclear stress on the last stressed \isi{syllable}, or a \isi{contrastive}
intonation pattern with a nuclear stress placed elsewhere; iii) a
pause may be observable between the two tone groups. In the case of
multiple topicalisations, each topicalised constituent involves its own 
separate tone group.

In example~(\ref{ex:cat1}), thus, it is argued that there are two
distinct tone groups, and that as a consequence, the fronted
object is separated from the remaining part of the sentence. In the
second tone group, nuclear stress would either fall on the final \isi{verb}
in the unmarked case, or on {\em il-qattusa} `the cat' in a
\isi{contrastive} focus reading of the \isi{subject}.

An additional piece of evidence is given to support the idea that the
fronted constituent is somehow detached, or, as the authors put
it, ``not strictly speaking in a \isi{grammatical} relation to the rest
  of the sentence'' \citep[73]{BorgAlexander2009}, namely the fact that
the object marker {\em lil} `to', which is normally obligatory with
person names functioning as objects, is no longer obligatory if a
person name is fronted. Finally, the authors claim that, when an
object is fronted, the main \isi{verb} has to bear a \isi{pronominal} \isi{clitic}
co-referential with this object. 

In other words, the definition of \isi{topicalisation} they propose is based
on \isi{syntactic} and prosodic characteristics all pointing to the fact
that the fronted constituent does not belong to the main sentence
%predication. These characteristics, however, seem indeed to correspond
% Rev PP
predication. These characteristics, however, rather seem to correspond
to those mentioned earlier in our account of \isi{left dislocation}. As far
as the \isi{discourse status} of the fronted constituent is concerned, the
authors seem to assume that it always expresses given information,
while the rest of the sentence predicates something new about the
fronted element. In other words, a fronted constituent in Maltese, in
this account, always seems to correspond to a topic, and fronting of
one or more constituents thus seems never to involve \isi{focus movement}.

In an earlier work on \isi{word order} in Maltese, \citet{FabriBorg2002} 
investigate which order combinations of S, V, and O are
grammatically possible in Maltese in contexts where each of the three
constituents is either the focus, the topic, or a \isi{contrastive}
focus. In general, it is not clear whether, according to Fabri and
Borg, one can assume a canonical, or unmarked \isi{word order} for
Maltese. Clearly, however, not all word orders are possible in all
discourse contexts. For our purposes, the two orders OSV and OVS, both
involving object fronting, are interesting. Unfortunately, the authors
do not provide naturally occurring examples to illustrate the
different contexts, but from the tables in which their claims are
summarised, it would seem that in both OSV and OVS the object can be
focus or topic depending on the \isi{prosody}.

\citet{Vella1995} also examines the different \isi{word order} possibilities
in Maltese with respect to their \isi{prosody}. In this early work, and in
contrast to \citet{FabriBorg2002}, she restricts her analysis to
structures not involving cliticisation, attempting, in so doing, to
come up with a \isi{phonological} explanation for the \isi{word order}
possibilities in Maltese. Vella invokes the notion of {\em focus} and
the related assignment of [±focus] \citep{Vella1995, Vella2009}
suggesting that the latter results from speakers’ manipulation of
semantic material in different discourse contexts. She follows
\citet{Gussenhoven83}'s use of the term {\em variable} to refer to the
material to which speakers obligatorily assign [+focus], and the term
{\em background} to refer to that stretch of speech assigned [-focus].
\citet[283]{Gussenhoven83} provides the following formulation:
``[+focus] makes the \isi{speaker}’s declared contribution to the
conversation whilst [-focus] constitutes his cognitive starting
point''. Apart from a brief reference to \isi{left dislocation} in \citet{Vella1995}
%her 1995 work,
Vella does not attempt to distinguish between different types of
complement fronting (\isi{topicalisation}, \isi{focus movement} or left
dislocation) as elaborated in the literature. Nevertheless her
examples, especially the constructed ones, appear to fit better into
the category involving \isi{focus movement} than into either of the two
other categories. The Map Task data examples in \citet{Vella2003, Vella2009}
%her 2003 and 2009 papers
are similarly used to illustrate different instances involving
\isi{focus movement} resulting from a variety of conditions such as changes
in \isi{word order}, cliticisation, negation and the presence of indefinite
pronouns, all of which appear to trigger the assignment of [+focus] to the
variable.  In terms of \isi{prosody}, the clear conclusion of all of Vella’s
work is that statements\footnote{A parallel construction has been
  described to occur in questions having an early [+focus]. In this
  case, a rising \isi{pitch accent} is followed by an upstepping \isi{phrase}
  \isi{accent} linked to a \isi{secondary prominence} and a \isi{boundary tone} which
  continues on a level high to the edge of the \isi{phrase}.}  involving
\isi{focus movement} and therefore an early [+focus], are characterised
prosodically by a falling \isi{pitch accent}. This falling \isi{pitch accent} is
followed by a movement involving a slight rise, which she analyses as a
sequence consisting of a \isi{phrase accent} linked to a secondary
prominence and a \isi{boundary tone} rising to the edge of the \isi{phrase}.

Left dislocation in Maltese is discussed in \citet{Bezzina2015}, who
examines the different properties of \isi{left dislocation} examples in spoken
data. Interestingly for our discussion, Bezzina refers to the examples
in \citet{BorgAlexander2009} as examples of \isi{left dislocation}, even
though the authors use the term \isi{topicalisation}. She claims that the
general purpose of the construction is that of promoting new referents
to topic status, and notes that the dislocated constituent is
perceived as detached from the rest of the sentence. Her main interest
is in the way the degree of formality of the data affects the
construction. She shows, in fact, that a formal style may allow for
syntactically rather complex dislocated elements. 

At least two of the characteristics noted by \citet{BorgAlexander2009}
with respect to the constructions they refer to as instances of
\isi{topicalisation} – which \citet{Bezzina2015} refers to as examples of
\isi{left dislocation} – do not seem to occur in the \isi{focus movement} examples
which feature in Vella’s work. These are separation into different
tone groups by means of a pause and the accompanying, also separate,
falling intonational movements.\footnote{It is worth noting that the
  `chains of fronted constituents' noted by Borg and
  Azzopardi-Alexander (2009) in their examples are mirrored by a
  similar effect noted in particular in Vella’s (2003) work. This is
  the possibility of `tone copying' as described by Grice et
  al. (2000) in the case of \isi{phrase} accents. This phenomenon involves a
  \isi{pitch accent} assigned to an early [+focus] element being followed by
  not one, but many, \isi{phrase accent} and \isi{boundary tone} sequences (see
  examples in 2003: 1778).} The \isi{focus movement} examples described by
Vella, by contrast, involve a falling \isi{pitch accent} only on the [+focus] 
element; any post-focal elements usually involve a slight
rise consisting of the \isi{phrase} and boundary \isi{tone sequence} mentioned
earlier.

To sum up, previous studies of complement fronting in Maltese provide
evidence for the fact that any of the constructions described in
the previous section, i.e. \isi{topicalisation}, \isi{left dislocation}, and focus
movement, may be at play when a complement is fronted. However, to our
knowledge no systematic data-driven account has been given so far of
what distinguishes these constructions in terms of their syntax,
the \isi{discourse status} of the fronted constituent, and the
suprasegmental features associated with them. It is the aim of this
article to fill this gap by proposing such an account based on
multimodal data, in other words spoken language data and accompanying
\isi{gestural} behaviour. We will be concerned with complement fronting as
exemplified in~(\ref{ex:top:m1}), to be compared with the non-fronted
counterpart in~(\ref{ex:nontop:m1}). We will, on the other hand, not
be concerned with examples involving fronting of adverbials, or
%\isi{subject} dislocation.
% Rev PP
\isi{subject} fronting.

\ea\label{ex:top:m1}
{MAMCO: 19\_g\_148}\\
\gll il-Baileys in-ħobb ukoll\\
     {\sc def}=Baileys-{\sc sg.m} 1-love.{\sc ipfv.sg} as.well\\
\glt `Baileys I like as well.'
\z


\ea\label{ex:nontop:m1}
%\langinfo{Maltese}{}{}\\
\gll in-ħobb il-Baileys ukoll\\
     1-love.{\sc ipfv.sg} {\sc def}=Baileys-{\sc sg.m} as.well\\
\glt `I like Baileys as well.'
\z


\section{Corpus data}
\label{section:data}

The data described in this paper were taken from the multimodal corpus
of Maltese MAMCO \citep{PaggioVella14}. This corpus is made up of
twelve video-recorded first encounter conversations. Twelve speakers
(six males, six females) participated in two sets of recordings, all
of which were made in \isi{Malta}. At the time of recording, all
speakers were students at the University of \isi{Malta}. All speakers were
Maltese dominant speakers and had not met prior to the
experiment. They were instructed to get to know each other. The set up
for the collection of this corpus was the same as was previously used
for the Nordic multimodal corpus of first meeting dialogues NOMCO
\citep{nomco-lrec10}, and involves pairs of speakers standing in a
studio conversing freely for about 5 minutes.

In this study, our focus is on constructions displaying complement
fronting. In particular, we investigate what prosodic contours are
associated with the constructions, whether the fronted complement in
these examples is accompanied by hand gestures, and what the discourse
status of the fronted complement is. A total of 36 examples involving
complement fronting were selected manually from the 24 dialogue
recordings. Some of the examples contain a \isi{clitic} \isi{pronoun}
coreferential with the fronted complement, others don't.
In~(\ref{ex:tequila}) and~(\ref{ex:second}) we show two examples: in
the former, the \isi{clitic} {\em h} refers to and agrees with the fronted
complement.\footnote{Note that in example~(\ref{ex:tequila}) the
  \isi{speaker} treats {\em tequila} as a masculine \isi{noun}, probably
  associating it with the masculine {\em drink}, even though the
  `correct' \isi{grammatical} gender is feminine.}  In the latter, the
fronted element {\em second year} involves a code-switch into English
of a structure which, in Maltese, would have been a prepositional
\isi{phrase} {\em fit-tieni sena} `in the second year’: no \isi{clitic} is involved
(and none would have been involved had there been no code-switch).

\ea\label{ex:tequila}
{MAMCO: 20\_g\_165}\\
\gll it-tequila j-rid j-koll-i burdata għali=h\\
     {\sc def}=tequila.{\sc sg.m} 1-want  1-have-{\sc 1sg} mood for={\sc 3.sg.m}\\
\glt `Tequila I need to be in the mood for it.'
\z


\ea\label{ex:second}
{MAMCO: 10\_f\_31}\\
\gll second year għad-ni\\
     second year still-{\sc 1sg}\\
\glt `In my second year, I am.’ 
\z

 

%\ea\label{ex:twenty}
%%\langinfo{Maltese}{}{}\\
%\gll Twenty     għalaq-na\\
%     Twenty     close.{\sc 3sg.pfv-1pl}\\
%\glt ‘Twenty, we turned’ 
%\z

A first summary of the data showing the distribution of clitics and
gestures is provided in \tabref{tab:paggio:corpus_stats}.

\begin{table}
  \begin{tabular}{lll}
    \lsptoprule
 fronted complement     & with \isi{gesture}  & without \isi{gesture} \\
    \midrule
with \isi{clitic} & 11 & 0    \\
without \isi{clitic} & 16 & 9    \\
    \midrule
total & 27 & 9    \\
\lspbottomrule
  \end{tabular}
  \caption{Corpus data statistics: gestures and clitics (absolute counts)}
  \label{tab:paggio:corpus_stats}
\end{table}

The sound files were transcribed and annotated in PRAAT
\citep{Praat2009}. Gestures, where present, were annotated using the ANVIL
tool for multimodal annotation \citep{Kipp2004}. In addition, the
examples were also coded in a separate text file with categories
referring to the \isi{discourse status} of the various
referents. Transcriptions and annotations are described in detail in
what follows.

%The annotated data will be made available through a web site.
% Rev PP
The annotated data can be obtained through the authors. A complete
list of the examples from the corpus is included at the end of this paper
together with their semi-literal translation.

  \subsection{Annotation of prosody}


%The main purpose of the prosodic analysis was to verify the claims
% Rev PP
The main purpose of the annotation and subsequent analysis of the
\isi{prosody} of the selected structures was to test the claims advanced in
\citet{BorgAlexander2009} about the prosodic characteristics of
fronted complements in Maltese, and at the same time to explore the
question whether different constructions might be distinguished in
Maltese based on their different intonation patterns, as is the case
for \isi{topicalisation} vs \isi{focus movement} in English.

The annotation was carried out following \citet{Vella1995, Vella2003,
  Vella2009}. It is couched in the Autosegmental-Metrical framework of
Intonational Phonology, see e.g.  \citet{Pierrehumbert80} and
  \citet{Ladd2008}. It involved the identification of tunes consisting of
sequences of pitch or \isi{phrase accent} and boundary tones.  Tones can be
\textbf{H}(igh) or \textbf{L}(ow). Pitch \isi{accent} tones are those associated
with prominent syllables having nuclear status and are marked by means
of an asterisk, \textbf{*}\footnote{Tones can also be associated with
  prominent syllables which are prenuclear, hence H*. An instance of
  this can be found on {\textit{\textsc {nies}}} in the second,
  w(eak)-branching of the two \isi{phonological} phrases in
  example~(\ref{ex:malti}): nuclear prominence in this example falls
  on {\it In}*{\textit {\textsc {gliż}}} in the first \isi{phonological}
  \isi{phrase} within the intonational \isi{phrase}.}. Phrase \isi{accent} tones are
those having a secondary association of the sort described by
\citet[180]{grice2000place} as tones which ``resemble ordinary pitch
accents, but do not signal focus or prominence in the same way [as
  ordinary pitch accents] reflecting their essentially peripheral
nature''. These are marked by means of a hyphen, \textbf{-}, following
the relevant tone. Boundary tones are marked as \textbf{p} or \textbf{i}
depending on whether they are associated with a \isi{phonological} \isi{phrase}
boundary or an intonational \isi{phrase boundary}.




  
In the \isi{prosodic annotation} of example~(\ref{ex:test2}), for instance,
corresponding to~(\ref{ex:second}) discussed earlier, we see a falling
tune H*+L starting on the accented \isi{syllable} *{\textsc{\textit {se}}}
of the fronted complement *{\textsc{\textit se}}{\it cond year}, and
falling to the edge of the \isi{phonological} \isi{phrase} (Lp). In instances where a
boundary target might be expected but where its realisation may be
difficult to determine or tease out as a separate tonal target
(separate in this case from the following L \isi{phrase accent}),
parentheses are used. This is the case here.  The fall is followed by
a \isi{phrase accent} L- on the accented \isi{syllable} of {\textsc{\textit
    {għad}}}{\it ni} rising slightly to the boundary at the edge of
the intonational \isi{phrase} Hi. In the textual rendering of this and the
succeeding examples, the \isi{syllable} carrying the \isi{sentence accent} is
shown in small caps and preceded by an asterisk, whilst any syllables
carrying a secondary \isi{accent} in postnuclear position are shown in small
caps without additional marking. Boundaries are shown by means of a
bar, ‘|’, and are indicated even in the absence of a physical break. A
list of the symbols used in the annotation is provided at the end of this paper.

%\begin{exe}
%  \ex{}
%\begin{tabular}{llllllll}
%  il-\textbf{\textsc{bai}} leys & $|$ & in & {\sc ħobb} & $|$ & u &{\sc koll} & $|$ \\
%  H*+L &    Lp & & L- & Hi & & L- & Hi\\
%\end{tabular}
%\label{ex:wrong}
%\end{exe}
 
 
\eabox{
\begin{tabular}{llll}
%   *\textsc{se}cond year & $|$ & {\sc għad} ni & $|$\\
  *\textsc{se}cond year & $|$ & {\sc g{\Ħ}ad} ni & $|$\\
                  H*+L           & (Lp)  &      L-       & Hi\\
\end{tabular}
\label{ex:test2}
}

\figref{fig:praatex1} displays the PRAAT screen dump
corresponding to the same example.

\begin{figure}
\includegraphics[height=.3\textheight]{figures/praat_secyear.pdf}
\caption{PRAAT screen dump showing the prosodic annotation of the example {\em Second year għadni}.}
\label{fig:praatex1}
\end{figure}

%\begin{table}
%  \begin{tabularx}{\textwidth}{Qrrrr}
%    \lsptoprule
%    Nuclear \isi{pitch accent} type & \multicolumn{4}{Z{5cm}}{Post-nuclear \isi{phrase accent} +\newline
%                              boundary sequence type}\\
%    \midrule
%    {Fall}\\
%    &    -- & L-Hi & L-Hi L-Hi & L-Hi L-Hi L-Hi \\
%    \midrule
%    H*+L (Lp)                     &   & 15 & 10 & 1\\
%    H*+L (Lp) H*+L (Lp)           & 1 & 5 & 2 & \\
%    H*+L (Lp) H*+L (Lp) H*+L (Lp) & & 1 & & \\
%   \midrule
%   {Total Fall} & \multicolumn{4}{r}{35}\\
%   \tablevspace
%   {Rise}\\
%   &  \multicolumn{4}{r}{L+H-Hi} \\
%   \midrule  
%   L* H & \multicolumn{4}{r}{1}\\
%   \midrule
%   {Total Rise} & \multicolumn{4}{r}{1}\\
%   \midrule\midrule
%   {Grand total} & \multicolumn{4}{r}{36}\\
%\lspbottomrule
%  \end{tabularx}
%  \caption{Frequency counts of different combinations of one or more nuclear pitch accent (fall or rise) and post-nuclear phrase accent + boundary sequences}
%  \label{tab:paggio:patterns}
%\end{table}

\begin{table}
  \begin{tabularx}{\textwidth}{lrrrr}
    \lsptoprule
    Nuclear \isi{pitch accent} type & \multicolumn{4}{Z{5cm}}{Post-nuclear \isi{phrase accent} +\newline
                              boundary sequence type}\\
    \midrule
    {Fall}
    &    -- & L-Hi & L-Hi L-Hi & L-Hi L-Hi L-Hi \\
    \midrule
    H*+L (Lp)                     &   & 15 & 10 & 1\\
    H*+L (Lp) H*+L (Lp)           & 1 & 5 & 2 & \\
    H*+L (Lp) H*+L (Lp) H*+L (Lp) & & 1 & & \\
   \midrule
   {Total Fall} & \multicolumn{4}{r}{35}\\
   \tablevspace
   {Rise}
   &  \multicolumn{4}{r}{L+H-Hi} \\
   \midrule  
   L* H & \multicolumn{4}{r}{1}\\
   \midrule
   {Total Rise} & \multicolumn{4}{r}{1}\\
   \midrule
   {Grand total} & \multicolumn{4}{r}{36}\\
\lspbottomrule
  \end{tabularx}
  \caption{Frequency counts of different combinations of one or more nuclear pitch accent (fall or rise) and post-nuclear phrase accent + boundary sequences}
  \label{tab:paggio:patterns}
\end{table}

Counts of the various prosodic patterns found in the corpus are shown
in \tabref{tab:paggio:patterns}. The majority of our examples
(i.e. 27/36, or 75\%) have one nuclear \isi{pitch accent} on the fronted
complement. The remaining examples (i.e. 9/36, or 25\%) have two or
three \isi{nuclear pitch} accents, the first of which is also on the fronted
complement. The second nuclear \isi{pitch accent} (and the third in the one
example involving three consecutive pitch accents) is on a following
element in the rest of the utterance, either within the same
intonational \isi{phrase} (although a separate \isi{phonological} \isi{phrase}), or in a
separate intonational \isi{phrase}. The nuclear \isi{pitch accent} on the fronted
complement in all except one example is followed by the \isi{phrase accent}
and boundary \isi{tone sequence}, L- Hi.  Such a pattern is described by
\citet[51]{Vella2009}, who states that a nuclear \isi{pitch accent} is
``followed by a L \isi{phrase accent} linked to the stressed \isi{syllable}
closest to the edge of the intonational \isi{phrase} and a final Hi boundary
tone''.
%In only one case of the 37 examples reported on here,
%a yes-no question is involved. The question is shown
%in~(\ref{ex:question}).
% Rev PP
A yes-no question is involved in one of the examples, shown in~(\ref{ex:question}).

\ea\label{ex:question}
{MAMCO: 23\_f\_22}\\
\gll l-universita’ qiegħed inti? \\
   {\sc def}=university.{\sc sg.f} stay{\sc 3.sg}	you\\
\glt `The university do you attend (it)?'
\z

Yes-no questions in Maltese have a different tonal structure as
compared to statements, see \citet[51]{Vella1995, Vella2009}. The
fronted complement in the question carries a nuclear \isi{pitch accent}
(just as statements do). However, the nuclear \isi{pitch accent} in this
case is rising (i.e. L* Hp) rather than falling (i.e. H*+L). In
postnuclear position, the \isi{phrase accent} and boundary \isi{tone sequence} is
L+H- Hi. The \isi{prosodic annotation} of the example is shown
in~(\ref{ex:question_pros}).

\eabox{
  \begin{tabular}{lllll}
    l-universi&*\textsc{ta'} & $|$ & qiegħed & \textsc{in}ti?\\
              &    L*           & Hp  &      L+H-     & Hi\\
\end{tabular}
\label{ex:question_pros}
}

% Change in second revision
%To sum up, in our data the tendency for fronted complements to carry
%their own nuclear falling \isi{pitch accent} is clearly very strong. The
To sum up, there is a clear tendency in our data for fronted complements to carry
their own nuclear falling \isi{pitch accent}. The
tendency for the intonation of elements which follow the fronted
complement to carry the \isi{phrase accent} and boundary \isi{tone sequence} L- Hi
%described for example~(\ref{ex:test2}) is also very strong. Only a very
described for example~(\ref{ex:test2}) is also clear. Only a very
small number of examples in the data analysed, in fact, involve more
than one falling \isi{pitch accent}.


\subsection{Annotation of hand gestures}

In this study, hand gestures are considered to be suprasegmental features
on a par with prosodic features. There are good reasons for this
assumption.
% Rev PP: what follows has slightly changed, I forgot to keep the original.
There is large agreement in the literature that \isi{hand gesture} strokes
are temporally aligned (or slightly precede) the main \isi{sentence accent}
\citep{Kendon80,Bolinger86,McNeill92,AlahverdzhievaLascarides10}, and
it has been observed and verified on annotated multimodal data
\citep{Loehr04, Loehr07} that \isi{gesture} phrases are temporally
coordinated with intermediate phrases in the sense of
\citet{Pierrehumbert80}. In an empirical study of \ili{German} data (276
examples), \citet{Ebertetal11} find that \isi{gesture} strokes tend to
precede \isi{sentence accent} by 0.36s on average, in other words they
confirm what seems to be generally acknowledged in the literature. However,
the authors of this study make the claim that whatever alignment is
observed between \isi{gesture} phrases and intonationally motivated
stuctures is a by-product of an interdependence between gestures and
focus phrases, which in turn is motivated by information
structure. They do find evidence to confirm this claim, since they
observe that the onsets of \isi{gesture} phrases in their data align with
new-information foci with a time lag of only 0.31s on average (and a
small standard deviation). The same kind of temporal
%inter-dependency
interdependence is not found, on the other hand, between \isi{gesture}
phrases and \isi{contrastive} focus phrases.

To our knowledge, no one has investigated whether hand gestures play a
role in conjunction with complement fronting. Since we have seen that
fronted complements in Maltese are accompanied by pitch accents, we
would expect that hand gestures, if present, would be likely to align
with them. 
% Rev PP: added
However, finding that hand gestures are coordinated with fronted
complements would seem to contradict \citet{Ebertetal11}'s claim that
\isi{gesture} phrases align with focus phrases in that fronted complements,
as we have seen, do not necessarily correspond to sentence foci.  In
fact, a first look at the data gave us the impression that there was a
tendency for fronted complements to be accompanied by gestures.  The
goal of the \isi{gesture} annotation was to verify this expectation in a
systematic way and to provide a new perspective from which to look at
the relation between gestures and discourse structure.

\begin{figure}
  \includegraphics[height=.25\textheight]{figures/second_year.jpg}
   \includegraphics[height=.25\textheight]{figures/second_year_link.pdf}
   \caption{Annotation of a hand gesture in ANVIL: gesture element with link to corresponding words.}
\label{fig:anvil1}
\end{figure}

For each of the examples under discussion, if a \isi{hand gesture} by the
\isi{speaker} overlaps the fronted complement, this \isi{hand gesture} was
annotated as a temporal element associated with the corresponding
video frames.
% Rev PP: added part on MUMIN coding scheme and \isi{gesture} annotation
%Labels were then added to the defined \isi{gesture} element to
%indicate which hand was used ({\em BothHandsAsymmetric, BothHandsSymmetric,
%RightSingleHand}, or {\em LeftSingleHand}), and the semiotic type of the \isi{gesture}
%({\em Symbolic, Iconic, Deictic}, or {\em IndexicalNonDeictic}).
The annotation procedure and the labels used to annotate gestures are
taken from the MUMIN coding scheme \citep{Allwoodetal07}, an annotation
scheme for multimodal behaviour which provides attributes for the
annotation of shape, dynamics and function of head movements, facial
expressions, hand gestures, and body posture. The scheme has been
successfully used to code multimodal behaviour in several languages,
e.g. in the NOMCO project, which has developed annotated
conversational data for \ili{Danish}, Swedish, Finnish and \ili{Estonian} \citep{nomco-lrec10, PaggioNavarretta2016}.

\largerpage
According to what the MUMIN scheme prescribes, we do not explicitly
mark \isi{gesture} strokes, which we understand as the most dynamic parts of
the gestures, nor do we mark the internal structure of a \isi{gesture} in
terms of its preparation, prestroke hold, stroke, and retraction (see
e.g. \citealt{McNeill92}). Instead, we create temporal elements in the
annotation that correspond to the whole duration of the \isi{gesture} from
the beginning of the movement to its completion. In a series of
gestures, we follow \cite{Kipp2004}'s recipe to distinguish the various
gestures: essentially, we draw a boundary every time a \isi{gesture} changes
direction and velocity, and a new stroke is visible.

Only two types of attributes were selected from the MUMIN scheme and
annotated in our data. There are attributes that indicate which hand
was used as well as whether the hands in a two-handed \isi{gesture} are used
symmetrically, and others that specify the semiotic type of the
\isi{gesture}. They are shown in \tabref{tab:paggio:attributes}.

\begin{table}
  \begin{tabularx}{\textwidth}{lQ}
    \lsptoprule
 Attribute     & Values \\
    \midrule
    {Handedness} &  {BothHandsAsymmetric, BothHandsSymmetric}  {RightSingleHand, LeftSingleHand} \\
    \midrule
{Semiotic type} & {Symbolic, Iconic, Deictic, IndexicalNonDeictic}\\
\lspbottomrule
  \end{tabularx}
  \caption{Hand gesture annotation attributes}
  \label{tab:paggio:attributes}
\end{table}


Whilst the handedness features should be self-explanatory, the semiotic ones
deserve some comment. {\em Symbolic} is used to annotate conventional
emblematic gestures; {\em iconic} is used for gestures that express
the content of their object by similarity -- either in a concrete or
an abstract way; {\em deictic} is used for hand gestures that identify
an object spatially; finally {\em IndexicalNonDeictic} is used for
batonic gestures, or beats.  We have not yet analysed how the two sets
of attributes are used in the data: in future, we intend to
investigate whether semiotic type interacts in systematic ways
with discourse features of the associated referents.

The \isi{gesture} annotation of an example discussed previously,
see~(\ref{ex:test2}), is illustrated in \figref{fig:anvil1}. The
video frame shows the point of maximal extension of the \isi{hand gesture}
performed by the \isi{speaker} on the right.
% Rev PP: added following sentence.
% Changed in second revision.
% We call this point the apex of the \isi{gesture}.
Below the frame is a section of
the ANVIL annotation board displaying the word transcription, the
\isi{prosodic annotation}, the English translation, and the \isi{hand gesture}
element, which is linked to the words {\em second year}. The \isi{gesture}
is categorised as a {\em LeftSingleHand} one, and the annotation also
contains the semiotic feature {\em Symbolic} (not visible in the
figure), which is reserved for conventionalised, emblematic gestures
like the `two' \isi{gesture} in question. The annotation also shows
additional tracks (syllables, FacialExpressions, HeadMovements, and
BodyPosture) that were not used for this study and are therefore left
empty.

A total of 30 hand gestures are present in the fronted complement
example dataset. Of these, 27 (90\%), occur in conjunction with the
fronted complement.
%Although we
%don't have a baseline against which to compare this proportion (for
%example a random dataset of utterances with co-occurring gestures
%displaying no complement fronting), the tendency seems very strong. In
%other words, our expectation that gestures, where present, would align
%with accented fronted complement, is clearly confirmed by the data.
% Rev PP
This looks like a pattern, indicating a strong tendency for fronted
complements to be accompanied by gestures. To check that this is a real
tendency, we also analysed all the hand gestures produced by two of
the MAMCO speakers in two different conversations. Both speakers
produce 80 hand gestures for which the whole extension from the
beginning of the movement to its end has been annotated as described
earlier. Of the 80 gestures, only 17 (21\%) in the case of one
\isi{speaker}, and 13 (16\%) in the case of the other, are aligned with the
initial sentence constituent. Six of these cases (2 and 4,
respectively) involve fronting. The remaining gestures occur in the
middle of the sentence, towards the end, or span the whole
sentence. The last type makes up a large portion of the gestures (63
and 67, respectively). These gestures have a long duration, either
because they are repeated or because they have a long prestroke hold,
and their extension spans the duration of the whole sentence.

These numbers seem to provide a more complex picture than the one
described by \citet{Ebertetal11} for \ili{German}, and call for a detailed
analysis of the alignment between \isi{gesture} strokes and pitch accents in
Maltese. For the present study, however, it suffices to note that in
general, the probability for a \isi{gesture} to align with the initial
sentence constituent in our data (without spanning the rest of the
sentence at the same time) is relatively low. This probability
increases in sentences where the initial constituent is
a fronted complement.


\subsection{Annotation of discourse status}

The purpose of annotating the fronted complement with respect to the
\isi{discourse status} of the corresponding \isi{referent} was to use discourse
status to distinguish between the constructions discussed
previously.


The discourse \isi{referent} corresponding to the fronted complement was
annotated using one of the three categories {\em new, poset}, or {\em
  old}. {\em New} means that the \isi{referent} has not been mentioned
%earlier and is not implied; {\em poset} that it has not been
% Rev PP: added footnote
earlier and is not implied, in other words that it is referentially new;\footnote{For a discussion of the difference between referentially and relationally new, see e.g. \cite{GundelFretheim08}.} {\em poset} that it has not been
mentioned, but stands in what \citet{Prince1981a} calls a
partially-ordered set relation with an already mentioned or implied
\isi{referent} (for instance by expressing contrast or by referring to a
more specific but related concept); finally {\em old} means that the
\isi{referent} has already been mentioned. The distribution of the three
categories is shown in \tabref{tab:paggio:discourse_stats}.


\begin{table}
  \begin{tabular}{lr}
    \lsptoprule
 Discourse status     & Counts \\
    \midrule
new & 15   \\
poset & 7    \\
old & 14    \\
    \midrule
total & 36    \\
\lspbottomrule
  \end{tabular}
  \caption{Corpus data statistics: discourse status of the fronted complement (absolute counts)}
  \label{tab:paggio:discourse_stats}
\end{table}


\newpage 
\section{Results}
\label{section:results}

% Rev PP: added following paragraph
In this section we analyse the way in which the different contructions
involving fronting which we described earlier are realised in the
corpus data. We start by providing some corpus statistics intended to
give a quantitative view of different properties of these
constructions in our data, and we then analyse examples which we
consider typical of these tendencies in a qualitative fashion.

%Rev PP: added subsections
\subsection{Corpus statistics}

%Based on the overview of the literature, we make the following
%assumptions on how \isi{discourse status} of the fronted complement can be
%mapped onto different constructions:
% Rev PP
Based on the overview of the literature, we distinguish four different
constructions based on the \isi{discourse status} of the fronted
complement. In addition, the presence or absence of a \isi{clitic} or a
\isi{pronoun} coreferential with the fronted complement is used as a diagnostic to keep
\isi{topicalisation} and \isi{left dislocation} apart.

\begin{itemize}
\item
By definition, in \isi{focus movement} (FM) constructions the fronted
complement is {\em new}. Following \citet{Prince1981a}'s analysis, we
expect it often to be an attribute that is added as new information to
an otherwise presupposed \isi{referent}.
\item
In \isi{left dislocation} constructions, there are two possibilities, as we
saw earlier. The fronted complement can be {\em new}, and introduced
as a new topic for subsequent reference. It can, however, also be {\em
  old}. Following \citet{geluykens1992}, we will call the two types of
\isi{left dislocation} LD1 and LD2, respectively. In either case, there is
always a \isi{clitic} or a \isi{pronoun} in the rest of the sentence which has the
same \isi{referent} as the fronted complement and syntactically agrees
with it.
\item
Finally in \isi{topicalisation} constructions (TOP), the fronted complement is
either {\em old} or it stands in a {\em poset} relation with an already
introduced \isi{referent}. There is no \isi{clitic} or \isi{pronoun} in the rest of the
sentence that agrees with the topicalised complement.
% Rev PP: added paragraph
Note that examples of \isi{topicalisation} without a following \isi{clitic} in our
data also include PP fronting. This seems to confirm that
cliticisation is linked to a specific construction rather than to
\isi{syntactic} properties of the fronted constituent.
\end{itemize}


%\tabref{tab:paggio:const_stats} shows the occurrence of the
%different \isi{discourse status} labels in the four constructions.
% Rev PP

\tabref{tab:paggio:const_stats} shows counts of the four constructions in
the corpus together with a specification of the discourse label of the
fronted complement, which was used for the construction
classification.
%An interesting question given this taxonomy is whether the
%suprasegmental characteristics provided by \isi{prosody} and gestures to
%some extent reflect these distinctions.
% Rev PP
Given this taxonomy and the distribution of the data shown in the
table, the question we ask in this section is whether the
suprasegmental characteristics provided by \isi{prosody} and gestures to
some extent differ depending on the construction type.


\begin{table}
  \begin{tabular}{lrrr}
    \lsptoprule
 Construction type     & new & poset & old \\
    \midrule
FM & 11 & 0 & 0   \\
LD1 & 4 & 0 & 0   \\
LD2 & 0  & 0 & 7  \\
TOP & 0 & 7 & 7   \\
\midrule
total & 15 & 7 & 14    \\
\lspbottomrule
  \end{tabular}
  \caption{Constructions and discourse status of the fronted complement (absolute counts)}
  \label{tab:paggio:const_stats}
\end{table}


We saw earlier that the majority of our examples (27) are
characterised by the occurrence of a single \isi{pitch accent} on the
fronted complement, whilst the remaining 9 examples display two pitch
accents (three in one single case). If we look at how the two prosodic
patterns map onto the different construction types
(\tabref{tab:paggio:const_pitch}), an interesting tendency seems to
emerge.

\begin{table}
  \begin{tabular}{lrr}
    \lsptoprule
 Construction type     & One \isi{accent} & Two accents \\
    \midrule
FM & 10 (.91) & 1 (.09)  \\
LD1 & 1 (.25) & 3 (.75)  \\
LD2 & 3 (.43) & 4 (.57)  \\
TOP & 13 (.93) & 1 (.07)  \\
\midrule
total & 27 (.75) & 9 (.25) \\
\lspbottomrule
  \end{tabular}
  \caption{Constructions and pitch accent (counts and proportions)}
  \label{tab:paggio:const_pitch}
\end{table}

The numbers show that the tendency for \isi{topicalisation} and
\isi{focus movement} constructions to be accompanied by only one pitch
\isi{accent} is inverted in the case of \isi{left dislocation}, where we see a
slight preponderance of the two-\isi{accent} pattern (7 vs 4). The
differences are statistically significant (Fisher's exact test,
p-value = 0.004918). The different pattern displayed by left
dislocation reflects the fact that the fronted complement in this
construction is somehow detached from the rest of the construction, as
also indicated by the presence of a \isi{clitic} or \isi{pronominal} reference.
The length of the utterance (in the sense of the number of words
% used), may also, however, contribute to the presence of an additional
used), may also, however, in itself contribute to the presence of an additional
\isi{pitch accent}. In fact, most of the cases in which two pitch accents
occur, but also most of the \isi{left dislocation} constructions, are
relatively long.
% Rev PP: added following sentences.
This makes sense in terms of discourse strategy. Left dislocation
constructions introduce the \isi{referent} in a more elaborate way, and
therefore often have more substantial material in the \isi{clause}.

\begin{table}[b]
  \begin{tabular}{lrlrl}
    \lsptoprule
 Construction type     & \multicolumn{2}{c}{Gesture yes} & \multicolumn{2}{c}{Gesture no} \\
    \midrule
FM & 5 & (.46) & 6 & (.54)  \\
LD1 & 4 & (1) & 0 & (0)  \\
LD2 & 7 & (1) & 0 & (0)  \\
TOP & 11 & (.75) & 3 & (.21)  \\
\midrule
total & 27 & (.75) & 9 & (.25) \\
\lspbottomrule
  \end{tabular}
  \caption{Constructions and pitch accent (counts and proportions)}
  \label{tab:paggio:const_gesture}
\end{table}

Turning now to gestures (\tabref{tab:paggio:const_gesture}), we see here
that \isi{left dislocation} and \isi{topicalisation} constructions seem to fall into
a different category in that they are always or nearly always
characterised by the presence of a \isi{gesture} (100\% of the LD1 and LD2
cases, and 75\% of the TOP ones), against a more or less 50/50
distribution in the case of \isi{focus movement}. The differences, %after
%having collapsed
once the two LD types are collapsed, are significant (Fisher's exact
test, p-value = 0.01135).
%We can advance the tentative explanation
%that gestures are instrumental in marking the topical nature of the
%fronted complement in \isi{left dislocation} and \isi{topicalisation}
%constructions.
% Rev PP: rephrased and added
It is tempting to advance the tentative explanation that gestures are
instrumental in marking the topical nature of the fronted complement
in \isi{left dislocation} and \isi{topicalisation} constructions. 

\subsection{Analysis of four examples}

%To conclude, below we give examples of what we regard as prototypical
%examples of the four construction types from our corpus.

To provide a more detailed analysis of the tendencies identified in
the statistical analysis, we give below what we consider particularly
illustrative examples of the four construction types from our
corpus. Given our focus on both \isi{prosody} and gestures, we have chosen
examples where gestures are always produced in conjunction with the
fronted complement, even though about half of the examples of focus
movements do not contain a \isi{gesture}. For each example we describe the
way prosodic and \isi{gestural} characteristics have been annotated.

\ea\label{ex:end-of-month}
{MAMCO: 18\_g\_116}\\
\gll sa 	l-aħħar 	ta-x-xahar 	għand=hom\\
    till	{\sc def}=end 	of-{\sc def}=month	have={\sc 3.pl}\\
\glt `Till the end of the month they have.'
\z


Example~(\ref{ex:end-of-month}) is a \isi{focus movement} construction. The
two speakers are talking about how much time students have left to prepare
for their exams. The fronted complement {\em sa l-aħħar tax-xahar}
`till the end of the month' is a temporal expression that provides a
new attribute to the presupposed timeframe of the action, and is thus
annotated as {\em new}.


\begin{figure} 
\includegraphics[height=.3\textheight]{figures/praat_lahhar.pdf}
\caption{PRAAT screen dump showing the prosodic annotation of example~(\ref{ex:end-of-month}) {\em  sa l-aħħar 
tax-xahar għandhom} `Till the end of the month they have'.}
\label{fig:praat_lahhar}
\end{figure}

\begin{figure}
  \includegraphics[height=.25\textheight]{figures/end-of-month.jpg}
   \includegraphics[height=.25\textheight]{figures/end-of-month-link.pdf}
   \caption{Focus movement and gesturing in  example~(\ref{ex:end-of-month})}
\label{fig:end-of-month}
\end{figure}


The \isi{prosody} is characterised by a falling \isi{pitch accent}, H*+L, on the
nuclear accented \isi{syllable} of the fronted complement, *{\textit
  {\textsc {xahar}}}.  Pitch continues to fall to a Low \isi{phrase}
\isi{accent}. L-, associated with the secondary \isi{accent} on {\textit {\textsc
    {għand}}} in {\it għandhom}, followed by a slight rise to a Hi
boundary at the end of the \isi{phrase}. There is no clear intermediate
target for a Low boundary, Lp, following the H*+L \isi{pitch accent}, in
this example. \figref{fig:praat_lahhar} displays the
PRAAT screen dump showing the prosodic contour and annotation. The
\isi{gesture} performed by the \isi{speaker} on the left, and shown in
\figref{fig:end-of-month}, is a batonic \isi{gesture} ({\em
  IndexicalNonDeictic}) performed with the left hand.
%and aligned with the fronted focus.
% Rev PP: added what follows.
The arrows in the figure are intended to show the trajectory of the
\isi{gesture}: the hand starts from a resting position close to the body, is
lifted forward and brought back to its initial position. The segment
corresponding to the \isi{gesture} in the annotation board shows the entire
% Changed in second revision
%extension of the movement, the apex of which, however, is aligned with
%the \isi{pitch accent} on the fronted focus.
extension of the movement, which overlaps with
the fronted focus carrying the \isi{pitch accent}.


\ea\label{ex:malti}
{MAMCO: 36\_k\_105}\\
\gll Malti u Ingliż ħafna nies 	ikoll=hom\\
Maltese and English   many  people have=3.PL\\
\glt `Maltese and English many people have them'\\
\z



Example~(\ref{ex:malti}) is a left dislocated construction of the LD1
type. The two speakers are discussing course requirements, and one of
them mentions Maltese and English as being subjects that a lot of
people meet the requirements for. Maltese and English have not been
mentioned previously and are not contrasted with other subjects or
requirements. They have therefore been labelled as {\em new}. The
verbal \isi{affix} {\em -hom} agrees in number with the fronted
complement.\footnote{It can also be argued, however, that {\em hom} in
  this example agrees with the \isi{plural} \isi{subject}. Were such an analysis
  to be chosen, the example would have to be re-categorised as a
  \isi{focus movement} example rather than a case of LD1.} The discourse
function of LD1 is, as we saw earlier, to promote a new \isi{referent} to
being the topic of the sentence. Interestingly, the other \isi{speaker}
acknowledges the introduction of the new \isi{referent} by nodding, thus
making this example neatly conform with \citet{geluykens1992}'s view
of \isi{left dislocation} as an interactional device.


\begin{figure}[b]
\includegraphics[width=\textwidth]{figures/praat_ld1.pdf}
\caption{PRAAT screen dump showing the prosodic annotation of example~(\ref{ex:malti}) {\em  Malti u Ingliż ħafna nies ikollhom} `Maltese and English many people have them'.}
\label{fig:praat_malti}
\end{figure}


The \isi{prosody} in this case, see \figref{fig:praat_malti} is again
characterised by a falling \isi{pitch accent}, H*+L, on the fronted element,
{\it In}*{\textit {\textsc {gliż}}}. In this case the fall is not
visible (although it is auditorily perceptible) due to the presence of
the \isi{obstruent} (/z/ in \isi{word-final position} in Maltese is devoiced to a
[s]). The \isi{phonological} \isi{phrase} containing the fronted complement in
this case is followed by another \isi{phonological} \isi{phrase} having a H tone,
H*, on the accented \isi{syllable} {\textit{\textsc {nies}}}, followed by a
\isi{phrase accent}, L-, on the \isi{syllable} carrying \isi{secondary prominence}
{\textit{\textsc{kol}}} of {\it ikollhom} and a slight rise to a High
\isi{boundary tone}, Hi, at the end of the \isi{phrase}. The main difference here
is that the fronted element gets its own separate \isi{pitch accent}, which
is not the case for the \isi{focus movement} case illustrated earlier.


\begin{figure}[b]
   \includegraphics[height=.25\textheight]{figures/malti1.jpg}
   \includegraphics[height=.25\textheight]{figures/malti2.jpg}
   \includegraphics[height=.23\textheight]{figures/malti-link.pdf}
   \caption{Left dislocation (LD1) and gesturing in example~(\ref{ex:malti})}
\label{fig:malti}
\end{figure}

\newpage 
As for the \isi{gestural} behaviour, the \isi{speaker} actually produces two hand
gestures, one for each of the nouns in the fronted complement. Both
%are symmetrical two-handed gestures, where the hands move first to the
% Rev PP
are symmetrical two-handed gestures, where the hands move together
first to the left, and then to the right, as can be seen in
\figref{fig:malti}. In the annotation board, the red vertical
line corresponding to the mouse position highlights the second
% Changed in second revision
%\isi{gesture}, the apex of which temporally coincides with the \isi{pitch accent}
\isi{gesture}, which overlaps with the fronted complement
that carries the \isi{pitch accent}. The first \isi{gesture}, in turn, aligns
temporally with the unaccented \textit{Malti} `Maltese', the other \isi{noun} in
the fronted complement. In other words, we see here an example where
gestures accompany the fronted complement, but where there isn't a
complete correspondence between the \isi{gestural} and the prosodic
features.
\newpage 

%\ea\label{ex:tequila2}
%{MAMCO: 20\_g\_165}\\
%\gll It-tequila j-rid j-koll-i burdata għali=h\\
%     {\sc def}=tequila.{\sc sg.m} 1-want  1-have-{\sc 1.sg} mood for={\sc 3.sg.m}\\
%\glt `Tequila, I need to be in the mood for it'
%\z

\ea\label{ex:tequila2}
{MAMCO: 20\_g\_165}\\
\gll it-tequila j-rid j-koll-i burdata għali=h\\
     {\sc def}=tequila.{\sc sg.f} 1-want  1-have-{\sc 1.sg} mood for={\sc 3.sg.m}\\
\glt `Tequila I need to be in the mood for it.'
\z



Example~(\ref{ex:tequila2}), which was also mentioned earlier as
example~(\ref{ex:tequila}), is a left dislocated construction of the
%LD2 type. The \isi{referent} of the fronted complement, {\em sa it-tequila}
% Rev PP
LD2 type. The \isi{referent} of the fronted complement, {\em it-tequila}
`tequila' has just been mentioned by the other \isi{speaker} in the context
of a discussion of various alcoholic drinks. The \isi{discourse status}
label used is therefore {\em old}. The current \isi{speaker}, on the left in
\figref{fig:tequila}, makes this \isi{referent} the topic of her
utterance and states her attitude towards it.
% Rev PP: added what follows
Note that there is a lack of agreement between the fronted complement
{\em it-tequila}, which is feminine, and the masculine \isi{clitic} in {\em
  għalih}.\footnote{One of the reviewers of this paper considered the error in this example a slip on the part of the \isi{speaker}. Another, however, noted that tequila is often considered masculine in Maltese speech, on a par with {\em wiski}, {\em vodka} etc.}
%We consider this a slip on the part of the \isi{speaker}.


\begin{figure}[b]
  \includegraphics[height=.25\textheight]{figures/tequila.jpg}
   \includegraphics[height=.25\textheight]{figures/tequila-link.pdf}
   \caption{Left dislocation (LD2) and gesturing in example~(\ref{ex:tequila2})}
\label{fig:tequila}
\end{figure}

\clearpage
The \isi{prosody} of the example, shown in \figref{fig:praat_tequila},
is characterised by a falling \isi{pitch accent}, H*+L, on the fronted
element {\it te}*{\textit{\textsc{qui}}}{\it la}, with a clear Low
\isi{phrase boundary}, Lp at the end of this element. Pitch continues to
fall to a Low \isi{phrase accent}, L-, on the secondary \isi{accent} on {\it
  għa}{\textit{\textsc{lih}}}, and there is a final slight rise to a
High \isi{boundary tone}, Hi. In this case, although there is no clear pause
following the fronted complement, a \isi{phonological} \isi{phrase boundary}, Lp,
does seem to be present.


\begin{figure}
\includegraphics[width=\textwidth]{figures/praat_tequila.pdf}
\caption{PRAAT screen dump showing the prosodic annotation of example~(\ref{ex:tequila2}) {\em it-tequila jrid jkolli burdata għalih} `Tequila I need to be in the mood for it'.}
\label{fig:praat_tequila}
\end{figure}


On the \isi{gestural} level the \isi{speaker} (on the left) performs what looks
like a deictic \isi{gesture}, as if pointing at an imaginary tequila in the
air.
% Rev PP: added what follows
The dynamic of this \isi{gesture} corresponds to the upward arrow in the
figure, and the first \isi{gestural} element in the annotation board. The
% Changed in second revision a couple of lines below
point of maximal extension of the \isi{gesture} (which is not, however,
explicitly annotated) coincides very clearly with the \isi{pitch accent} on
the fronted complement. The hand is then lowered with the index still
extended in two subsequent, shorter movements performed after the
\isi{phonological} \isi{phrase boundary}.  Interestingly, the other \isi{speaker} (on
the right) also gestures at the same time, as if acknowledging the
joint topic. Again, we see the interactional nature of left
dislocation realised in the gestures.


\ea\label{ex:recordings}
{MAMCO: 4\_b\_155}\\
\gll recordings għand-i\\
     recordings 1-have-{\sc 1.sg}\\
\glt `Recordings I have.'
\z


Finally, an example of a topicalised construction is shown
in~(\ref{ex:recordings}). The speakers are discussing the methods they
used in their dissertations. The male \isi{speaker} explains that he
conducted interviews. The female \isi{speaker} then says that she does not
have data from interviews, but that instead she has some
recordings. The \isi{referent} corresponding to the fronted object, {\em
  %  recordings}, stands in a {\em poset} relation to {\em interviews},
  % Rev PP
  recordings}, stands in a {\em poset} relation to {\em interviews}
which both speakers have just mentioned: more specifically, it marks a
contrast between the two referents.

The \isi{prosody} is characterised by a falling \isi{pitch accent}, H*+L, on the
nuclear accented \isi{syllable} of the fronted complement {\it re}*{\textit
  {\textsc{cor}}}{\it dings}, as shown in
\figref{fig:praat_recordings}. It is difficult to ascertain
whether there is an L \isi{boundary tone}, Lp, separating the \isi{phrase}
containing the fronted complement from the \isi{phrase accent} and boundary
\isi{tone sequence}, L-Hi, on {\textit{\textsc{g{\Ħ}an}}}
of {\it għandi}. 


\begin{figure}
\includegraphics[height=.3\textheight]{figures/praat_recordings.pdf}
\caption{PRAAT screen dump showing the prosodic annotation of
  example~(\ref{ex:recordings}) {\em recordings għandi} `recordings I have'.}
\label{fig:praat_recordings}
\end{figure}

As for the gestures, the \isi{speaker} (on the right) accompanies the
topicalised object (and the corresponding \isi{pitch accent}) with a batonic
\isi{gesture} performed with the right hand, as can be seen in
\figref{fig:recordings}.
% Rev PP: added what follows
From the annotation board in the same
figure it can also be seen that this \isi{gesture} is immediately preceded
by another one in correspondence with the negated `interviews' in the
preceding sentence.


\begin{figure}
  \includegraphics[height=.25\textheight]{figures/recordings.jpg}
   \includegraphics[height=.25\textheight]{figures/recordings_link.pdf}
   \caption{Topicalisation and gesturing in example~(\ref{ex:recordings})}
\label{fig:recordings}
\end{figure}

% REv PP: added
To sum up, the examples discussed above show what seems to be a rather
fundamental difference between \isi{left dislocation} constructions on the
one hand, and \isi{topicalisation} and \isi{focus movement} on the other, a
difference which is also indicated by the quantitative analysis of the
prosodic features. Left dislocation examples display a more complex
suprasegmental structure, more often characterised by two pitch
accents and the presence of multiple gestures, sometimes on the part
of both speakers. There are, however, more initial gestures in
\isi{topicalisation} than in \isi{focus movement} constructions.

\section{Conclusions}
\label{section:conclusion}

This paper deals with complement fronting in Maltese, and examines the
interface between syntax, \isi{prosody}, discourse and gestures by discussing the
temporal alignment of pitch accents and gestures with the fronted
complement, as well as the \isi{discourse status} of the \isi{referent} denoted by
this same complement in different contexts. This study is the first of
its kind in that it uses data taken from a corpus of spoken Maltese
(MAMCO). Our results contribute to what previous research has shown,
but also give a more detailed analysis by providing an account of four
different constructions all involving complement fronting: focus
movement, \isi{topicalisation} and two types of \isi{left dislocation}.

\newpage 
Overall, the results show that, unless the example is a question, the
fronted complement has a falling nuclear \isi{pitch accent}, annotated as H*+L 
(Lp). However, there is a tendency for \isi{left dislocation} to have
two falling \isi{nuclear pitch} accents, one on the fronted complement and
the other on another complement following it. In the majority of the
examples, the nuclear \isi{pitch accent} on the fronted complement was
followed by a low boundary \isi{phrase accent}, L- Hi. As for the
realisation of gestures, our results show that \isi{left dislocation} and
\isi{topicalisation} constructions have a clear tendency (75-100\%) to
be accompanied by a \isi{hand gesture} on the fronted complement. In the
case of \isi{focus movement}, on the other hand, the likelihood of a \isi{gesture}
occurring is much less (about 50\%).
% Rev PP
These figures contrast with the much lower probability of
sentence-initial gestures (10-21\%) in a baseline of 160 non-fronted
examples from the same corpus.

Keeping in mind that this was a corpus-based investigation using
limited spoken data and, therefore, the number of examples was small,
we make the following tentative conclusions. Firstly, the \isi{prosody} on fronted
complements is similar across the four types of construction (unless
the fronted complement involves a question which in turn has a
different prosodic structure than statements); however, the presence
of an additional \isi{pitch accent} in \isi{left dislocation} examples seems to
strengthen the detached nature of the fronted complement, which is
also signalled in some cases by the presence of verbal or \isi{gestural}
feedback by the interlocutor. Secondly, the occurrence of gestures
partitions the constructions in a slightly different way, with left
dislocation and \isi{topicalisation} on the one hand, and \isi{focus movement} on
the other.
In this connection, it is noteworthy that gestures align
more readily with topics than foci in constructions involving
fronting.



% Rev PP: I'm not so sure about other languages, so I would leave out what follows.
%contrary to what has been found in other \isi{syntactic} contexts
%and for other languages.
The two sets of findings seem to point to the
%fact that the four construction types can be placed on a continuum,
fact that the four construction types can be placed on a continuum as
regards the complexity of the suprasegmental structure, with focus
movement and both types of \isi{left dislocation} on the two ends of the
scale, and \isi{topicalisation} in the middle, sharing some features with
\isi{focus movement} (prosodic structure) and others with \isi{left dislocation}
(presence of gestures). This continuum is illustrated in
\figref{fig:continuum}.

\begin{figure}
\fbox{
\parbox{6.5cm}{
{\LARGE
$-\xleftrightarrow[]{\text{suprasegmental~complexity}}+$\\}
 FM \hfill TOP \hfill  LD1/2
 }
 }
%   \includegraphics[height=.10\textheight]{figures/continuum.jpg}
   \caption{Maltese fronted complement constructions ordered on a continuum of suprasegmental complexity}
\label{fig:continuum}
\end{figure}


\newpage 
An aspect which has not been analysed in depth, and which could
constitute a direction for future work, relates to the transition
between the fronted complement and the rest of the sentence. In left
dislocated constructions, in contrast to \isi{focus movement} ones, the
transition seems to be characterised by some sort of
discontinuity. Such discontinuities are often perceptually noticeable
but not necessarily easy to identify acoustically, thus rendering
\isi{phonological interpretation} difficult.

% Rev PP: added
In addition, a more thorough analysis of the temporal coordination
between \isi{gesture} phrases and speech in the entire corpus would provide
a more solid basis to understand the relation between gestures and
discourse in more general terms.

\section*{Acknowledgements}

We would like to thank the students from the Institute of Linguistics
at the University of \isi{Malta} who helped with the transcription and
annotation of the MAMCO corpus. We also thank Marie
Azzopardi-Alexander, Elisabet Engdahl and the external reviewer for
their comments on the first version of this article.




\section*{Abbreviations}
\subsection*{Abbreviations used in the glosses}
\begin{tabularx}{.45\textwidth}{lQ}
1 & First person \\
3 & Third person \\
{\sc def} & Definite \\ 
{\sc f}  & Feminine \\
{\sc indr}  & Indirect Object \\
\end{tabularx}
\begin{tabularx}{.45\textwidth}{lQ}
{\sc ipfv}  & Imperfect Verb \\
{\sc m}  & Masculine \\
{\sc poss}  & Possessive \\ 
{\sc prf}  & Perfect Verb \\
{\sc sg}  & Singular \\
\end{tabularx}
  
% \section*{Appendix II: symbols used in the prosodic annotation}

\subsection*{Individual symbols used in the prosodic annotation}

\begin{tabularx}{\textwidth}{lQ}
H &	High tone\\
L &	Low tone\\
* &	prominence marker,  e.g. H* represents a High tone associated with a prominent    (accented) \isi{syllable} usually in nuclear position, but possibly     also in prenuclear position.\\
- &     secondary prominence marker,  e.g. L-  represents a Low tone associated with a \isi{syllable} having a \isi{secondary prominence} in post-nuclear position.\\ 
\end{tabularx}

\noindent
\begin{tabularx}{\textwidth}{lQ}
p &	phonological {phrase boundary} marker,\\
  &     e.g. Lp is a {phonological}  {phrase boundary} Low tone.\\
i &	intonational {phrase boundary} marker,\\
  &     e.g. Hi is an intonational  {phrase boundary} High tone.\\
( ) &	marker of a phonologically expected tonal target\\
    & which does not seem to be realised phonetically.\\
\end{tabularx}

\subsection*{Patterns used in the prosodic annotation}

Examples of patterns combining the symbols above are the following:

\vspace{0.3cm}
\noindent
\begin{tabularx}{\textwidth}{lQ}
  H* + L (Lp)  &	Falling \isi{pitch accent} with a Low \isi{boundary tone}   \isi{phonological} target which may or may not be realised. \\
  
L- Hi &	Low \isi{phrase accent} and slight rise to an H boundary  tone associated with  a \isi{secondary prominence} in postnuclear position.\\
\end{tabularx}
 

\section*{Fronted complement examples corpus}

\begin{enumerate}
\item {\em bl-interviews għamiltha}\\
`with interviews I do it' 
\item {\em  Ħaż-Żabbar     għandi     kuġin     minn hemmhekk     jien}\\
`Ħaż-Żabbar I have [a] cousin from there' 
\item {\em  Wied il-Għajn ija immur ta}\\
`Wied il-Għajn  yes do I go [there]' 
\item {\em  jiena mhux interviews recordings għandi differenti 'iġifieri }\\
`I don't have interviews recordings I have I mean' 
\item {\em  emozzjonijiet     qiegħda     nagħmel     infatti}\\
`emotions I'm doing in fact' 
\item {\em  linguistics  jiena}\\
`linguistics I do' 
\item {\em  imma dil-water fight  qatt ma mort}\\
`this water fight never I went' 
\item {\em  proċedura     u     hekk     tal-qorti     għa'na     m'għamilnihomx}\\
`procedures and so on of the courts still we haven’t done them' 
\item {\em  id-dar     ta'     ħdejha     toqgħod     iz-zija     tiegħi     fiha  }\\
`the house next to it lives my aunt in it' 
\item {\em  second year għadni}\\
`second year I'm still [in]' 
\item {\em  opra ma taraħħiex bil-wiefqa taraha bil-qiegħda}\\
`an opera you don't see it standing you see it sitting down'
\item {\em  tipo mużika tal-parties ma nħobbhiex}\\
`as in  music  for parties I don’t like it' 
\item {\em  twenty two ħa nagħlaq}\\
`twenty two I'm going to be' 
\item {\em  picnic u hekk ħa nitħajjar immur }\\
`picnic and such I’m going to be tempted to go [to]' 
\item {\em  practicals u hekk għadna għaddejjin s'issa}\\
`practicals and such still we are carrying on till now' 
\item {\em  u n-nagħġu ilni ma nara’}\\
`and the goat [nickname] for a while I haven’t seen' 
\item {\em  Martini per eżempju     jogħġobni}\\
`Martini for example I like' 
\item {\em  sa     l-aħħar     tax-xahar     għandhom}\\
`till the end of the month they have' 
\item {\em  il-Baileys inħobb  ukoll}\\
`Baileys I like as well' 
\item {\em  it-tequila jrid jkolli burdata għalih}\\
`tequila I need to be [in the] mood for it' 
\item {\em  i...     i... ije  l-università  qiegħed}\\
`ye... ye... yes [at] the University I am [there]' 
\item {\em  ħafna nies  it-tequila     jdejjaqhom ħafna}\\
`a lot of people tequila they dislike [it] a lot' 
\item {\em  l-università     qiegħed     inti?}\\
`the university do you attend [it]?' 
\item {\em  twenty għalaqna}\\
`twenty we turned' 
\item {\em  sentej' iżgħar     minnek     jien     kont}\\
`two years younger than you I was'
\item {\em  il-Fabian anka jien ili ma narahom ta}\\
`Fabian also I in a while haven’t seen them too' 
\item {\em  il-tagħkom naħseb  il-ħadd     m'għadni     nara     jie'a}\\
`your class-mates I think none of them I see' 
\item {\em  outskirts ħafna noqgħod}\\
`outskirts a lot I live [there]' 
\item {\em  ee Antonia jisimni jien}\\
`uh Antonia my name is me' 
\item {\em  l-filosofija kelli intermediate}\\
`philosophy I had [at] intermediate'
\item {\em  u Chetcuti tgħidx k'm konna nittnejku bih miskin}\\
`and Chetcuti you don’t say how much we used to make fun of him poor [him]'
\item {\em  sal-erba u nofs għandna}\\
`until 4:30 we have'
\item {\em  ma ma Dr. Moses kont}\\
  `oh dear with Dr Moses I was'
\item {\em  dak il-hassle m'ghandix aptit jien}\\
`that hassle I don’t fancy it'
\item {\em  l-għadam ta’  Novembru qatt  m'għamilthom u qatt ma doqthom 'iġifieri onestament}\\
`the bones of November never I made them and never I tasted them I mean honestly'
\item {\em  Malti u Ingliż ħafna nies ikollhom}\\
`Maltese and English many people have them' 
\end{enumerate}

\sloppy
\printbibliography[heading=subbibliography,notkeyword=this] 
\end{document}

