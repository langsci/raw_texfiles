\documentclass[output=paper]{langsci/langscibook} 
\ChapterDOI{10.5281/zenodo.1181803}
\author{Marie Azzopardi-Alexander\affiliation{Institute of Linguistics, University of Malta}}
\title{Maltese Sign Language: Parallel interwoven journeys of the Deaf community and the researchers}
\shorttitlerunninghead{Maltese Sign Language}
%\ChapterDOI{} %will be filled in at production
\abstract{This article traces the rapid development of Maltese Sign Language (LSM) from a language that was reportedly restricted to informal day-to-day communication by the Deaf community, to one that is now widely used in both informal and formal settings, including in the context of academic subjects such as the sciences, and in the context of professional activities.  The article gives an account of LSM from a historical perspective, paying particular attention to its roots within the Deaf community, culminating in its recent recognition as an official language of Malta.}
\maketitle

\begin{document}

 
\section{Introduction}
This article traces the rapid development of \ili{Maltese Sign Language}
(LSM) from a language that is reported to have been used only for
informal day-to-day communication by the \isi{Deaf}\footnote{The use of
  lowercase \textit{deaf} will refer to any or all hearing-impaired persons
  whereas uppercase \textit{Deaf} will be restricted to those who use \isi{sign}
  language and consider themselves members of the \isi{Deaf} community.}
community \citep[7]{lj86} to one that is used in
both informal and formal settings and for a variety of academic
subjects such as science as well as in applications such as
professional hairdressing and automotive civil engineering
\citep[55 ff]{aa15}.  A noteworthy vocabulary
explosion occurred rapidly to meet the demands of signers,
particularly since access to secondary and post-secondary education
was enabled through \isi{sign language} interpreters since 2001.

This development parallels the research interest in LSM originating in
the 1980s (see Section 3) and culminating around ten years later with
the setting up of the \ili{Maltese Sign Language} Research Project at the
University of \isi{Malta}’s Institute of
Linguistics.\footnote{\url{http://www.um.edu.mt/linguistics/research/maltesesignlang}}
This led to the start of courses in \ili{Maltese Sign Language} taught by
young \isi{Deaf} adults and the compilation of the \ili{Maltese Sign Language}
Dictionary (see \citealt{a03} and \citealt{a04}), work that
is now continuing on the online version. Nevertheless, LSM is a
minority language in a tiny island and the \isi{Deaf} community faces the
enormous challenge of surviving within the already \isi{bilingual setting}
of Maltese and English \citep[52]{aa15}.

\section{Looking back} 

\subsection{Undocumented beginnings}

Very little is known about the hearing-impaired population in \isi{Malta}
(henceforth the \isi{deaf}) beyond that recorded in its educational history.
Looking around one can still see those, now elderly, \isi{deaf} persons who
did not benefit from the educational system and who still managed to
survive.  It is impossible to gauge their quality of life.  No attempt
has been made to ask for their stories probably because research has
so far been limited to the more easy-to-access younger members of the
\isi{Deaf} community.  The older \isi{Deaf} who did not access education at all
must have been limited in their communication to matters of every day
life with those who have lived with them or who are in their close
vicinity (e.g. local shopkeepers).  The only source of information on
this is hearsay – people who remember “il-mutu” or “il-muta” (the dumb
man or woman) who stuck out in the locality.  Their vocal
communication amounted to unintelligible vocalisations to the outsider
but was sufficient to get by with family and other acquaintances who
understood and presumably used home-made signs.  They were sheltered
by strong family structures.  A few are still identifiable in various
towns and villages.  They did not usually work beyond occasional odd
jobs given by family members or friends and their social interactions
were limited.  It would be desirable to enable them to record their
own perspectives before their world disappears.  Similar stories are
known elsewhere because hearing family members have told them
(e.g. various migrants to the US reported by \citealt{t09} and
many others) or because others who were themselves \isi{Deaf} could present
the \isi{Deaf} perspective more reliably (e.g. \citealt{c96}).

%\todo{Check capitalization of section titles everywhere}
\subsection{Seedlings of the Deaf community}\label{sec:azzopardi:2.2}
\largerpage
  
\isi{Deaf} children in \isi{Malta} were given the opportunity to attend school for
the first time in 1956 though some whose hearing loss was not too
severe were sent to their local school prior to that, in spite of the
school’s inability to cater specifically for them in any way.  They
were expected to fit in.  Nevertheless, unlike in other places such as
the UK, the USA and mainland Europe, there were never any boarding
schools for the \isi{Deaf} in \isi{Malta} – probably because the size of the
island combined with the small number of \isi{deaf} children does not
warrant residential education.  Hence the \isi{Deaf} community did not
flourish beyond what interaction could be fitted into the school day.

Although \isi{deaf} and \isi{Deaf} children were educated together in the one
school, the educational system was intensively oral-aural, “with
auditory training, lip-reading and speech lessons taking a good slice
off the time-table.” \citep[36]{g91} The time for
the rest of the curriculum was significantly reduced. Signing was not
presented as a means of full access at the school for the \isi{deaf} since
few teachers could use more than a few signs though they gradually
moved towards a more total-communicative approach in the 1980s.  This
meant that access to the ordinary curriculum was very limited.
“Besides the three Rs\footnote{The three Rs are reading, writing, and arithmetic.} these children were trained in carpentry,
printing, lace making, needlework and home economics, thus preparing
them for a better future.” \citep[38]{g91}
English was not taught in the special unit apart from “a few common
words and phrases” if they were going to the Trade School.  This was
not challenged by the educators although, as the Head of the \isi{Deaf} Unit
of the time admitted, “We have always found that by teaching only
Maltese in our schools we are condemning our \isi{deaf} children to be
second class citizens in a country where Secondary Education, public
examinations etc. have a predominant English background.” 
\citep[39]{g91}

Parents started to consider mainstreaming\footnote{Mainstreaming
  refers to education within regular schools.  Mainstreaming \isi{deaf}
  children in \isi{Malta} preceded the Inclusive Education movement in the
  1990s where all children with disability were welcomed into regular
  schools and usually granted the help of a Learning Support
  Assistant.} their \isi{deaf} children motivated by the knowledge that
their children would not be missing much in the mainstream that they
would have accomplished at the \isi{Deaf} Unit where basic literacy and
numeracy formed the bulk of the curriculum.  “The method of aural-oral
teaching {\dots} at times has been enforced even with children who could
not follow it, with the result that the latter could neither
communicate orally or in an officially recognised \isi{sign}
language{\dots}. and have had to resort to a primitive environmental \isi{sign}
language understood only amongst themselves.” 
\citep[50-51]{ba91}

Thus, at the request of one or two of the parents, \isi{deaf} children
started to be mainstreamed in the 1970s over the next few years as a
result of the parents’ growing awareness that special education for
\isi{deaf} children was far from being academically at par with what they
would be exposed to in the mainstream.  Where intensive
parental/family support could be given, some children did very well in
the mainstream.  We are told that a \isi{deaf} child “absorbed and is
absorbing a lot of our attention and time {\dots} interpreting for her
most of the time” \citep[45]{bezz91}.  Others did not thrive
within the mainstream school system \citep[38]{g91}.  This is no surprise particularly because at the time mainstream
primary school classes tended to be much larger, often 30\% larger
than the current average of 17.6 in State schools, 25.4 in Church
schools and 20.2 in Independent schools 
\citep{o16}.  Moreover, \isi{deaf} children had to have extensive
parental academic support at home to enable them to cope with the
learning of their hearing peers.

\subsection{Mainstreaming – dissolution of the deaf-deaf contact}

Professionals such as psychologists, social workers and even priests
were unable to communicate with \isi{deaf} youngsters or adults and this was
felt throughout. Teachers often took on the role of interpreters
where ex-students turned to them for help of all kinds.  Families –
usually one particular hearing member of the family – often acted as
interpreter but in some situations this did not happen.  Even those
who completed their secondary education successfully and continued
into post-secondary level did not feel completely at ease in the
hearing world. One of the most academically successful youngsters
states publicly at the 1991 conference \textit{Partnership between
  \isi{Deaf} People and Professionals} that using signs with \isi{deaf} people
made communication quicker and easier but he would always speak to
hearing people.  Unfortunately he felt left out when his work-mates
“do not always tell me what has happened, because deafness is a hidden
handicap, so they forget to explain to me.  This also happens to other
people like my swimming coach and also my teachers.”  He also
anticipated problems were he to have a hearing girlfriend because
“hearing people do not know enough about the \isi{deaf}.”

\begin{quote}
I think I am different from hearing people.  They can communicate
quickly.  I communicate slowly.  Hearing people can communicate
easily. Sometimes I communicate with difficulty.

Like myself, \isi{deaf} people in \isi{Malta} have difficulties at home, at work,
and at other places.  I am very lucky that I have little or no
problems with my family at home but I know that many \isi{deaf} people have
a lot of problems with their family \citep[41]{bk91}.
\end{quote}

He concludes with a “wish that in the future, \isi{deaf} people in \isi{Malta}
would have more opportunities to improve the quality of their life”
\citep[43]{bk91}. Sign language made life much easier but
with the size of the \isi{Deaf} \isi{Maltese community}, there are inevitable
disadvantages if \isi{Deaf} youngsters and adults are to work and socialise
within the dominantly hearing community, greater disadvantages than
those of larger populations with larger \isi{Deaf} communities.

Mainstreaming separates the \isi{deaf} from each other completely.  Often,
there is only one \isi{deaf} child in a school. I have occasionally been
present when a \isi{deaf} child is introduced to other \isi{deaf} persons and s/he
is surprised and then exhilerated to realise that s/he is not alone,
not the only \isi{deaf} one any more.

The small group of girls and the small group of boys who were educated
together in mainstream schools continued to form a miniature
community. These two groups were separate from the \isi{Deaf} Unit and were
even freer to foster \isi{sign language}.  By that time, in the early 1990s,
a qualified teacher of the \isi{deaf} who was a fluent \isi{signer}\footnote{She
  had qualified as a teacher of the \isi{deaf} in the UK in 1991 and her
  ability to use British Sign Language led her to progress quickly to
  becoming fluent in \ili{Maltese Sign Language}.} facilitated their access
to some of the secondary \isi{school curriculum}.  The teacher challenged
the children to develop signs they required for the subjects they
followed in the mainstream and to discuss the different signs they
came up with in order to agree on usable signs.  The children’s
friendship blossomed, particularly because they shared more than they
could share with hearing peers with whom they often felt left out
since communicating was an effort. Retelling jokes and stories to \isi{deaf}
peers can become frustrating for hearing youngsters, slowing down
spontaneous conversation.  Summaries of everyday conversations filters
out jokes and other important titbits that are technically not really
informative, even at home within the family.  The fact that the
children in these groups managed to continue into post-secondary
education may point to the fact that this kind of semi-mainstreaming
may reap benefits and should be considered as a way forward.

\section{The emergence of Maltese Sign Language}

It is commonly acknowledged that “Very little is known about the
history of \isi{sign} languages; most evidence is anecdotal.  It is likely
that in the past, as in the present, there has been some contact
between signers from different countries~{\dots}” \citep[81]{woll84} It
could be said that many of the \isi{deaf} children at the hearing-impaired
unit in Pietà formed the first \isi{Maltese community} of \isi{Deaf} people along
with the two small groups of children taught together in the
mainstream (see \sectref{sec:azzopardi:2.2}).  Of course, they were very young and did not
include \isi{Deaf} adults so they did not have the advantages of exposure to
adult \isi{sign language} except in the case of one particular child whose
parents were also \isi{Deaf}.  This reaped some benefits to the others as
well who were exposed to the adult \isi{Deaf} community more extensively in
their late teens through the \isi{Deaf} club.  However, on the whole they
were deprived of the continuity of \isi{sign language} users which is
important to all \isi{Deaf} communities and they were left to their own
devices in constructing signs.  Later, teachers used signs from
British Sign Language (BSL) and from Gestuno\footnote{Gestuno was the
  name given to the first pulication of internationally-agreed on \isi{sign}
  vocabulary useful at international meetings.  However, this soon
  developed into International Sign to enable more \isi{Deaf} people to
  understand each other in international settings (e.g. The World
  Federation of the \isi{Deaf} congresses).}
\citep{lj86}, though most of these were not
retained in the long term.

\isi{Deaf} communities emerge naturally when profoundly \isi{deaf} people meet on
a regular basis.  This has been known to happen in schools for the
\isi{deaf} across continents \citep{rr05}. In spite of
the lack of adult to child \isi{sign language} exposure, and in spite of the
mainly oral educational setting, Maltese \isi{Deaf} youngsters are captured
signing by Peter Llewellyn-Jones during visits to the \isi{Deaf} Unit.  One
of the teachers of the \isi{deaf} who taught the children at the Pietà \isi{Deaf}
Unit at the time observes five years later: “It is fascinating {\dots} to
see how resourceful the hearing impaired can be, even in the most
difficult situations.  Also fascinating is their ability to find or,
better still, invent signs adapted from their local environment”
\citep[50]{ba91}. Alex Borg also observes how the \isi{deaf}
turn to “natural gestures in a kind of basic \isi{sign language}” at the
\isi{Deaf} Unit.

The \isi{Deaf} youngsters had started to develop signs distinct from those
of BSL and Gestuno imported by teachers of the \isi{deaf} since the
vocabulary was published in 1975.  Some of the signs developed as all
the \isi{deaf} children started to come together at the \isi{Deaf} Club and
reflected more of the Maltese reality and culture as time went by.
The \isi{sign} for DAR (HOUSE) reflects the flat roofs although most Maltese
children would still draw the typical sloping roofed house; the \isi{sign}
for RA\.GEL (MAN) reflects the cap worn by mainly elderly Maltese
men. The term \ili{Maltese Sign Language} was used first by
\citet[7]{lj86} and subsequently by researchers in
their discussions with members of the \isi{Deaf} community in the mid-1990s
and in the first publication of the \ili{Maltese Sign Language} Project, the
first volume of the Dictionary \citep{a03} as
well as in subsequent publications \citep{a09,a05}.  The acronym for \ili{Maltese Sign Language} was established
internationally as LSM in accordance with the Maltese name
\textit{Lingwa tas-Sinjali Maltija.}

\section{The emergence of the Deaf community}

A number of people and events led to the \isi{Deaf} moving beyond the
‘control’ of the hearing teachers and parents who led the Association.
Nevertheless some hearing individuals recognised the need for the \isi{deaf}
to be masters of their own destiny, to move away from what can be
considered kind-hearted but nevertheless paternalistic attitudes of
the hearing.  This was important for them to develop their own
identity and belong as first-class citizens to a decidedly \isi{Deaf}
community which was Deaf-led.

\citet{ba91} mentions the setting up of a Maltese Sign
Language Project.  However, attempts to follow up the reference
pointed to the Bristol University \isi{Deaf} Studies-led research project,
which involved collecting data of \ili{Maltese Sign Language} along with
data of other European and Middle Eastern \isi{sign} languages and was not a
Malta-initiated project – at least the author could not trace any
references to it.  It seems to point, instead, to the intention of the
Special Education Department to look into the use of \isi{sign language} in
\isi{deaf education} with the help of the UK agencies mentioned in the
\citet{lj86} report.  Nothing appears to have come
out of the project in terms of \isi{deaf education}, \isi{sign language}
interpreting or even other professionals specialised with the \isi{deaf}
which were listed in the ‘General Comments and Suggestions’ section of
the report.

\citet{bezz91} reports that he established and coordinated a
self-help group of parents of \isi{deaf} children who met regularly and
organised educational and social events for their children that
included the whole family, enabling them and eventually \isi{deaf} adults to
meet on a regular basis \citep[45]{bezz91}.  Moreover, Bezzina
was very concerned about the lack of use of \ili{Maltese Sign Language} in
education and wanted to expose \isi{deaf} children and their families to
\isi{sign language} since many \isi{deaf} children “are leaving school unable to
speak, read, write or communicate manually except with close relatives
and/or friends” \citep[46]{bezz91}. Out of context this reflects
that in the past most decisions concerning the \isi{Deaf} were made in
hearing-led settings.  However, more recent events indicate the \isi{Deaf}
are now in a position to determine what happens to their community.
In fact many \isi{Deaf} activists were involved in discussions during the
phase where the \ili{Maltese Sign Language} Law was being discussed (see
\sectref{sec:azzoparid:9.1}).

\newpage 
Bezzina was the mind (and spirit) behind the opening of the \isi{Deaf} Club
in 1981 at Lascaris Wharf in Valletta. This enabled the \isi{Deaf} to come
together with the expected results that \ili{Maltese Sign Language} was used
much more extensively, it was passed on to the younger \isi{deaf} who became
primary users and hence they contributed to its development by
extending its vocabulary to meet their needs, and \isi{Deaf} adults started
marrying and continuing to visit the \isi{Deaf} Club with their mainly
hearing children.  Bezzina expressed two important thoughts publicly:

\begin{quote}
{\dots} we have to give the \isi{deaf} adults more space.  We have to believe
in their capabilities.  The Maltese \isi{deaf} adults should gradually lead
their own community {\dots} Maybe this conference will be the start of a
\isi{Deaf} Pride movement based on the Maltese \isi{Deaf} Culture with the Maltese
Sign Language as the Unifying force between the members of this
community \citep[48-49]{bezz91}.
\end{quote}

\section{Deaf culture and identity}

One important milestone was reached when classes of Maltese Sign
Language started to be taught at the University of \isi{Malta} Institute of
Linguistics and were later offered also as evening classes by the
Education Department and the \isi{Malta} College of Arts, Science and
Technology (MCAST) where they are still a popular addition to the
evening course programmes offered by the two institutions.

Recently, changes were made to the \ili{Maltese Sign Language} courses in
order to make \isi{deaf} culture part of the course design rather than in
answer to incidental questions asked by interested hearing adults.
This reflects the greater confidence of the \isi{Deaf} tutors in presenting
themselves as members of a minority group identifiable by their
language but fitting into the hearing world.

The ‘voice’ of the \isi{Deaf} can be seen in their language pride and their
conscious ownership reflected in the active and conscious formation of
new signs whenever the need arises.  They are aware that they cannot
work independently of each other because consensus is required for
signs to thrive.  It is hoped that more research will focus on the
process of \isi{sign} formation from the initial makeshift iconic \isi{sign} to
the more subtle signs \citep{a09}, from the
first use instigated by an immediate need to the time when it becomes
accepted by the larger group of signers.

\section{Deaf education}

Although over the years \isi{deaf} children have been supported, they still
share the dilemma of American (and probably many other) \isi{deaf} children
receiving a little service from a lot of professionals and still
“falling through the cracks.” \citep[198]{ol14}. \citet{ol14} recommend, on the basis of the
research done particularly in the VL2 Labs\footnote{The Brain and
  Language Lab for Neuroimaging developed by Laura-Ann Petitto in
  2012.} by Petitto and her team, “ongoing support from an individual
who has been schooled in all the issues they face” to enable their
Individual Educational Plan\footnote{Every \isi{deaf} child has an
  Individual Educational Plan (IEP) in \isi{Malta}.  However, there is lack
  of monitoring how and by whom the plan is to be realised beyond what
  is said at the IEP biannual meetings.} to be fulfilled \citep[198]{ol14}, in particular the development of bimodal
\isi{bilingual} skills.  The advantages of being bimodal \isi{bilingual} can be
attested both in the cognitive as well as in linguistic, educational
and socio-emotional domains, particularly in identity formation.
Previous concerns about the learning of \isi{sign language} having a
negative impact on the \isi{deaf} child, especially educationally and
specifically on learning spoken language, can now be shelved as
archaic.  Indeed, early learning of \isi{sign language} provides the \isi{Deaf}
child with support in learning the spoken and written language:

\begin{quote}
Does the knowledge of a natural \isi{sign language} facilitate \isi{Deaf}
children’s learning to read and write? The data collected in this
study seem to lead to a positive answer to this question, by showing a
strong relationship between LSF (\ili{French} Sign Language) and written
\ili{French} skills developed by \isi{bilingual} \isi{Deaf}
children \citep[45]{n08}.
\end{quote}

Although a great deal of work still needs to be done in this area,
Niederberger asserts the strong positive relationship between early
exposure to a \isi{sign language}, particularly to abundant narrative
exposure and literacy in the language spoken around the \isi{Deaf} child.
Moreover, the research points to the use of metalinguistic skills in
\isi{sign language} that positively impacts the child’s development of the
written language.

\citet[43]{p07} considers the lack of a language policy for \isi{deaf}
children as an `area of concern' which ``continues to hamper a clear
understanding of the linguistic, socio-emotional and cultural needs of
\isi{deaf} children'' and which reflects on ``the contribution of \isi{deaf} adults
in the education of \isi{deaf} children{\dots}(and) the development of suitable
assessment protocols for LSM, Maltese and \ili{Maltese English}{\dots}.''. The
role of \isi{Deaf} adults in supporting \isi{sign language} within the home was a
recommendation made in 1998 along with several others by the
Kummissjoni Ministerjali dwar l-Edukazzjoni tal-Persuni Neqsin
mis-Smigħ (Ministerial Commission on the Education of the Hearing-Impaired).  This would enable \ili{Maltese Sign Language} to develop with
continuity.  So far, the \isi{Deaf} themselves have no way of working for
this continuity and research cannot contribute to more than
establishing what the different varieties consist of and how they
differ from each other.

\section{Maltese Sign Language – From basic to refined}

It must be assumed that Maltese \isi{deaf} individuals probably made use of
signs at home that were iconic or which extended from local non-\isi{deaf}
signs used by others in the community.  If \isi{deaf} persons did not
actually meet anywhere except by coincidence, then one can assume that
they used signs we now call ‘home signs’ that shared the usual
features of any basic \isi{sign language} used for day to day activities
with family and close friends: iconicity, the mirroring to different
degrees of the physical or other identifiable features of
referents. More abstract concepts were most likely expressed through
association with more iconic signs with which the abstract concepts
are associated. Thus, for example, the signs for days of the week were
expressed through the signs for the major activity of the day such as
the \isi{sign} for doing the laundry. Thus, as soon as Maltese \isi{deaf} children
started to meet on a more regular basis at school they communicated
using these signs, each adjusting to signs of other members of the
group where these seemed to them to be ‘better’ signs i.e. ones that
were faster to produce, or which shared more elements of the group’s
different signs for the same object or concept.

Since Maltese \isi{deaf} adults did not usually get married \citep{g91}
until around the late 1970s, probably because they did not usually
meet except by coincidence, there did not seem to be any
generation-to-generation transfer of signs.  It is the first community
of signers who attended the school for the \isi{deaf} who must have formed
the first \ili{Maltese Sign Language}, however basic.  From then on, it is
likely that every other group who came together at the school would
have learnt and possibly contributed to the then relatively slow
development of the language since their lives were still very
restricted in educational terms.  The first recording of signs was
carried out for the comparative study of signs across around 20 \isi{sign}
languages in Europe and the Middle East led by the Bristol University
European Centre for Sign Language Research.  \citet{ba91}
refers to the local part of the study as “a feasibility study on
Maltese signs currently used at that time” \citep[52]{ba91}\footnote{However, no information about this is available in the
  public domain and information from the Education
  Department is currently unavailable.} As mainstreaming replaced the
\isi{deaf} unit, youngsters were again separated off in their district
schools.  However, once the \isi{Deaf} Club was opened they had recourse to
the other \isi{Deaf} members and hence to \isi{sign language}.  Nevertheless,
because their education went well beyond that of the older \isi{deaf}, their
need for signs beyond the every-day signs enabled \isi{sign language} to
flourish.  The result was also a discontinuity such that the older
\isi{Deaf} currently use different signs from those of the younger
\isi{Deaf}. Contacts with \isi{Deaf} communities overseas, facilitated by the
social media, and sometimes leading to lasting relationships involving
commitment, is now visible through signs borrowed from such contacts.
Some of the adult \isi{Deaf} are able to point to different members to
indicate “heavy borrowers”. Whether the borrowed signs replace the
local ones in the long term needs to be seen. Many \isi{Deaf} youngsters
resist using the borrowed signs possibly because of their language
pride.

The more recent rapid increase in \isi{sign} vocabulary (see Section 3) is a
response to the very rapid changes in the lives of the \isi{Deaf}.  The most
noteworthy \isi{vocabulary explosion} occurred rapidly to meet the demands
of signers who had a \isi{sign language} interpreter at
school\footnote{Sign Language interpreters have, to date, only had
  informal training with substantial input by the \isi{Deaf} community prior
  to their acceptance as competent for the task.  Formal training is
  currently being planned.} and followed classes in science and in
various other subjects.  Signs had to be created to cover the
vocabulary for the subjects for which \isi{sign language} interpreting
service was made available, starting with Mathematics, Home Economics
and proceeding to Physics, Biology and much more.  Since these signs
are still being inputted for the forthcoming online Maltese Sign
Language dictionary we are still unable to specify the size of the
vocabulary.  The creation of new signs led to their discussion with
peers and a growing consciousness of what they were involved in when
they needed to create new signs.  They discussed how they signed
different concepts and whether they liked or disliked what they had
come up with. They analysed what aspects they liked and what they did
not like and this growing consciousness and refined meta-linguistic
skills constitute a much-used resource though some would insist on keeping the
more iconic signs.

\isi{Deaf} signers have been interpreting daily news bulletins on TV since
2012, and the \isi{school curriculum} has been made more accessible to \isi{Deaf}
children first at secondary school especially since the first \isi{sign}
language interpreter was appointed by the \isi{Deaf} Association in
2001. Access across educational levels, including University, through
\isi{sign language} interpreting is currently provided on request, \isi{subject}
to availability.  Adolescents and young adults have started to follow
part-time evening courses after full-time work.  Growing confidence in
their abilities once they are ensured of access is changing the
\isi{Deaf} lifestyle even though they still lag behind their hearing peers
academically.

\section{The contribution of the research community}

This development of \ili{Maltese Sign Language} parallels the research
interest in LSM originating around 1994 and leading to the setting up
of the \ili{Maltese Sign Language} Research Project at the University of
\isi{Malta} Institute of Linguistics.  The project to some extent triggered
the \isi{Deaf} community’s heightened pride and interest in their \isi{sign}
language.

The first main aim of the project was the compilation of the Maltese
Sign Language Dictionary which resulted in two published volumes
(\citealt{a03} and \citealt{a04}), and two completed but
unpublished volumes.  It was not financially viable to publish the
hard copies and, since then, work progressed in view of having an
online version.  Work on this met with some difficulties which slowed
down progress, but it is currently hoped that the online dictionary
will be available in a relatively short time.  It is also hoped that
arrangements can be made to enable its regular update. The work on the
dictionary brought together small groups of \isi{Deaf} youngsters and adults
for periods of time discussing among themselves the signs they used.
They were all volunteers.  The hearing researchers soon recognised the
fact that signing changed in their presence and so the initial data
collection was not used.  Furthermore, there was no other study –
linguistic or otherwise - on which to base the data collection apart
from what was reported by \citet{lj86}.

Very soon it was possible to engage two \isi{Deaf} researchers to work on a
part-time basis as sponsorship of the project by their
employers.\footnote{The Bank of Valletta and the Works Department
  sponsored the project by allowing one of their employees to join the
  research team for the equivalent of a 1 day a week basis for several
  years.}  This enabled the signing for the data collection to be
more natural since no hearing researchers were involved.  A fresh
start was made by asking the \isi{Deaf} participants to take full charge of
the data collection.  One of the participants was in a position to
present signs used by the older generation and this enriched the
project unexpectedly. Occasional meetings took place to point out gaps
and ask questions about usage but it was considered unnecessary to
interfere beyond this.

The \isi{Deaf} did not adjust their signing but simply worked together.
They often disagreed of course, a healthy step towards more
representative data.  This meant that they were becoming more
sophisticated meta-linguistically.  They were intrigued by the fact
that their language was of interest to University academics.  This
helped them \isi{sign} more openly in most settings and they became
conscious of many things they had not previously thought about in
terms of themselves as communicators using \isi{sign language}.

Entries in the \ili{Maltese Sign Language} dictionaries
(\citealt{a03} and \citealt{a04}) include a description of
the signs in Maltese and English, frames from video clips showing from one to
three components of each \isi{sign} as well as the \isi{signwritten form} of the
\isi{sign} following the Valerie Sutton SignWriting system
\citep{s95}. \figref{fig:azzopardi:azz1} shows a page from the first
volume of the dictionary, {\em Animals}, which is the only volume that
includes illustrations.  \figref{fig:azzopardi:azz2} shows a page from the
second volume of the dictionary, {\em Places}.  The image on the top right
of each entry is the \isi{signwritten form}. This is being updated for the online dictionary on the basis of \citet{g14}. 

The first volume, Animals, has just over 100 entries. The second
volume, Places, has over 360 entries.  The online dictionary that
should be launched in March 2018 will contain all volumes
including those that have not been published each of which contains
around 350 entries.  There are around 3,000 entries in total so far.

  
\begin{figure}[p]  
\includegraphics[height=.8\textheight]{figures/a11AzzopardiAlexander-img1.png}
%\includegraphics[height=.8\textheight]{figures/a11AzzopardiAlexander-img1.jpg}
 \caption{\label{fig:azzopardi:azz1} Extracts from the Maltese Sign Language Dictionary Volume 1 \citep{a03} for the entry BEBBUXU / SNAIL}

\end{figure}

\begin{figure}[p]
\includegraphics[height=.8\textheight]{figures/a11AzzopardiAlexander-img2.png}
%\includegraphics[height=.8\textheight]{figures/a11AzzopardiAlexander-img2.jpg}
 \caption{\label{fig:azzopardi:azz2} Extract from the Maltese Sign Language Dictionary Volume  2 \citep{a04} for the entries AMERICA and SOUTH AMERICA}
 
\end{figure}


The online dictionary will have the advantage of video recordings for
the full \isi{sign} and hence provide a better teaching and learning
resource than static video clips.  Eventually signs will need to be
placed in proper contexts as illustrations of the various entries.
More financial input is required to enable the maintenance and the
development of the dictionary.  The \isi{Deaf} community can be engaged
directly to ensure that this is activated.

\section{Academic research on Maltese Sign Language}

Work on the dictionary entries generated a great deal of linguistic
information most of which still needs to be investigated in depth.
However, one can see the strands within the language tapestry within
which \ili{Maltese Sign Language} flourishes.  It is possible to trace some
interesting contact phenomena (see \citealt{aa15},
especially pp. 57 ff) in studying the dictionary entries.

In the short history of the study of \ili{Maltese Sign Language} there are
only a few pieces of work that derive from academic study.  However,
some of the works completed so far on LSM are significant and should
constitute a bridge to more extensive studies.

\newpage 
Early interest resulted in two undergraduate theses.  These include a
study of the communicative competence of a young \isi{Deaf} boy who used
very little speech and who signed to his family.  The thesis includes
a compilation of signs used by the child \citep{d88}.  At
the time there was no contact between the child and other \isi{Deaf}
persons, old or young.  So it would be interesting to compare the
lexicon compiled with that of current LSM. Another study seven years
later focussed on two \isi{Deaf} adults who used extensive signing in their
communication.  They were educated at the \isi{Deaf} Unit
\citep{p95}. Another 6 years later we find a study of the \isi{sign}
language used by two children one of whom had used \isi{sign language} all
his life with his signing \isi{Deaf} parents \citep{azz2001} whereas
another %studied
study focussed on the narrative skills of Maltese youngsters
\citep{f02}.

The first Master’s thesis is a comparative study of the communication
skills of 3 \isi{deaf} children, a cochlear-implanted child with
post-lingual hearing loss, one \isi{Deaf} child who used both speech and
\isi{sign} as she had been exposed to \isi{sign} soon after diagnosis and was
brought up by a signing speaking family and a \isi{Deaf} child who signed
but spoke very little \citep{a05}.

A huge milestone was reached with the Master’s thesis that focussed on
LSM classifier constructions \citep{g06}, as well as with the
study of how the Maltese \isi{Deaf} construct signs at different levels of
abstraction in different lexical fields \citep{m10}. These
works could be considered as initiating \isi{sign} linguistics research on
LSM. \citet{g06} is a detailed study of the way classifiers are
constructed in the LSM and how they behave. Galea analyses the
internal structure of LSM classifier handshapes as well as their
orientation and movement.  She considers the 3-way notional
classification of classifier handshapes in the literature – that of
Whole Entity, Size and Shape Specifier and Handle Handshapes
classifiers and discusses its limitations.  Different movements of
classifiers are discussed in detail as is the function of holds
(stationary classifier \isi{handshape}) in combination with the other
elements such as the articulating (moving) hand in creating
prepositional meaning, maintaining reference and differing contact
resulting in different lexical meaning.  Galea also discusses how
movement can be meaningful within the signed construction but can also
form part of the lexical meaning of the \isi{sign} itself. She concludes
that the distinction between these constructions is signalled by
non-manuals such as eye-gaze rather than by hand movement and hence
that the verbal versus nominal distinction in LSM involves these
non-manuals.  She thus opens up a whole new area of research that
calls for immediate attention.  In the course of the study, she
questions whether \isi{sign} linguists, internationally, were unduly
concerned about establishing parallels with spoken language research
and thus moved their attention away from important considerations
stemming exclusively from the manual modality.

\citet{m10} showed how LSM enables its users to distinguish
between different levels of abstraction through structural means.  She
found the use of simultaneous \isi{morpheme} compositions, reduced morphemes
resulting from extensive assimilation of \isi{handshape}, location,
orientation and movement to form a unitary whole as well as
\isi{compounding} with reduced movements.  She identifies the different
features involved in the compression of superordinate signs as
including loss of movement within constituent parts, loss of
morphemes, faster transitions of handshapes, \isi{handshape} differences and
durational compression \citep[150]{m10}.  These structural
characteristics resemble the \isi{sign} formations reported in other \isi{sign}
langauges such as ASL \citep[225 ff]{KB79}.

Different linguistic aspects are tackled by \citet{g14} within
the context of adapting the Valerie Sutton SignWriting System as a
standardised way of writing \ili{Maltese Sign Language}.  In this work Galea
investigates the way pronominals work in LSM in great detail.  She
then looks into how agreement verbs are used in relation to these
pronominals. The study presents a very interesting linguistic analysis
and is the start, it is hoped, of further in-depth linguistic research
into this relatively new \isi{sign language}.  Naturally, there is a very
long way to go.  Some of that mileage will hopefully be covered by
\isi{Deaf} researchers themselves in the not too distant future.
\citet{g14} in fact involves the \isi{Deaf} perspective to reach the
decisions expounded in the work.

\section{Maltese Sign Language officially recognised in Malta}

\subsection{Recognition}\label{sec:azzoparid:9.1}

When this article was started, the Maltese Parliament was expected to
put the \ili{Maltese Sign Language} Bill through its third reading in
November 2015.  The Bill was put through Parliament on March
16th 2016 and became an 
Act\footnote{\url{http://www.justiceservices.gov.mt/DownloadDocument.aspx?app=lp&itemid=26704&l=1}} signed by the President of \isi{Malta}, Marie
Louise Coleiro Preca, on March 24th 2016.  The Sign Language Law ACT
No. XVII of 2016 (ATT Nru XVII tal-2016) provides for the setting up a
Sign Language Council similar to that set up within the \ili{Danish} Sign
Language
Law.\footnote{http://dsn.dk/tegnsprog/about-the-danish-sign-language-council}
The \isi{Deaf} community currently has 3 representatives on the newly
appointed Sign Language Council. The law could make a great change to
their lives and, particularly, to the future of the \isi{Deaf} through
fuller access all round. Since LSM has become an official language, resources should be created to benefit the \isi{Deaf} community, particularly
within education.  In a few years, it is hoped that \isi{Deaf} children will
access all of the primary and secondary \isi{school curriculum} as well as
higher education through \isi{sign language} interpreters.  The interpreting
service goes hand in hand with the recognition of LSM in \isi{Malta} both
because the \isi{interpreting service} is being requested more extensively
by the \isi{Deaf} community and because more full-time interpreters will be
employed particularly once \isi{sign language} interpreter training is
offered.

A great deal of work needs to be done in order to build the resources
necessary for all this.  Official recognition is just the gate being
opened.  The community is tiny and human resources in the field are
limited.  It is hoped that the motivation and hard work of the Council
members will enable the \isi{Deaf} to achieve better and lead fuller lives.

The law will only be as strong as the Council, empowered by the human
resources who constitute it and by the financial resources it will
have to enable them to recommend the appropriate measures and monitor
their delivery.

\subsection{Great expectations}

The current number of full-time \isi{sign language} interpreters, five,
means that in practice the present number of interpreters can cope
with only a very limited part of \isi{Deaf} children’s school day.  The
expected growth of the service goes hand in hand with the recognition
of LSM in \isi{Malta}. Once the importance of the early exposure to \isi{sign}
language is recognised and the number of \isi{sign language} interpreters
increases to give full educational access to each \isi{Deaf} child, the
achievements of \isi{Deaf} children will improve and there will be a good
number who would be able to access higher education. This will
parallel the development in other countries. So far no profoundly
prelingual \isi{deaf} or \isi{Deaf} Maltese youngster has followed a degree course
at university.  It is partly because very often profoundly \isi{deaf}
children do not achieve the same academic results as their hearing
peers and partly because, even when they do, they have not, so far,
been able to access the lectures delivered in spoken language and
available to them by very limited auditory means alone and
corresponding lip-reading which equates to visual guesswork.

Naturally, it also depends on how demanding parents are.  Parents have
always been a major force to contend with.  It is hoped they will
continue to be.  Unfortunately, there is limited understanding
resulting from lack of readily available information.

Little is known about \isi{bimodal bilingual education} for the \isi{Deaf}
locally. Bimodal \isi{bilingual} education is known to facilitate \isi{Deaf}
children’s development of “positive self-esteem and a strong sense of
identity” and to show “evidence of improved pupil attainment”
\citep[19]{sg07}.  In both the mainstream and
schools for the \isi{Deaf} settings “good practice exists where \isi{deaf} adults
have a specific responsibility as role models and also potentially as
mentors for the \isi{deaf} pupils as they develop their identities, esteem
and confidence. The papers presented at the Multimodal Multilingual
Outcomes workshop in
Stockholm\footnote{\url{http://www.ling.su.se/om-oss/evenemang/workshops-och-konferenser/multimodal-multilingual-outcomes-in-deaf-and-hard-of-hearing-children-1.285132}}
in June 2016 pointed to the advantages of \isi{sign} bilingualism
for all children with hearing loss, even those with cochlear implants.

A review of local \isi{deaf education} must take place with the involvement
of all stakeholders, including those who are intent on excluding \isi{sign}
language from their \isi{deaf} children’s lives.  Evidence-based information
must be available to help parents and young people make the right
choices for the right reasons.

\section{Conclusion}

This attempt to work out the history of \ili{Maltese Sign Language} and of
the community that uses it is still not as complete as one would like.
What is needed is to engage the older \isi{Deaf} to narrate their own
perspective of what happened.  They report that they have only
recently started to use LSM with pride instead of restricting it to
settings with \isi{Deaf} participants with no hearing onlookers.  There are
still some who would say that they are stared at but this does not
stop them using it because they recognise it as a worthy means of
communication and of interest to academics.  They act as participants
for data that forms the basis of academic research.  Gradually they
will themselves be the researchers.  \citet{g14} has shown that
they are able to contribute to metalinguistic thought and discussion
and so there is an urgent need to enable some of them to work on
academic research.  As the Maltese \isi{Deaf} continue to use and develop
their language and contribute to academic research through their
collaboration with hearing researchers, their meta-linguistic skills
are likely to become more sophisticated than those of language users
in only the spoken modality.  This is an inevitable result as the
research becomes meaningful to them and so they could be given
opportunities to become protagonists in academic research.

 
 
 
 
%\section*{Abbreviations}
%\section*{Acknowledgements}

\sloppy
\printbibliography[heading=subbibliography,notkeyword=this] 
\end{document}
