\documentclass[output=paper]{LSP/langsci} 
\ChapterDOI{10.5281/zenodo.1181785}
\author{Patrizia Paggio\affiliation{Institute of Linguistics and Language Technology, University of Malta; CST, University of Copenhagen} 
 \lastand Albert Gatt\affiliation{Institute of Linguistics and Language Technology, University of Malta} 
}

\title{Introduction}
\abstract{}
\maketitle

\begin{document}

\noindent The purpose of this publication is to present a snapshot of the state
of the art of research on the languages
%recent cutting edge research contributions on languages used on 
of the Maltese islands, which include standard Maltese,
%AG: No, nothing on dialects. Commented out refs to dialects
%%, but also Maltese dialects, ***I don't think
%we have anything on dialects, do we? PP*** 
\ili{Maltese English} and \ili{Maltese Sign Language}.

\isi{Malta} is a tiny but densely populated country, with over 422,000
inhabitants spread over only 316 square kilometers.
%122 square miles. 
It is a \isi{bilingual}
country, with Maltese and English as official languages. Maltese is
%\ili{Semitic}
a descendant of \ili{Arabic}, but due to the history of the island, it has borrowed
extensively from \ili{Sicilian}, \ili{Italian} and English. Furthermore, local
dialects still coexist alongside the official \isi{standard variety}.
%
%%The point on dialects seems irrelevant. PP
%%AG: True, but it doesn't harm the intro to just mention them here.
The status of English as a second language dates back to British
colonial rule, and just as in other former British colonies, a
characteristic Maltese variety of English has developed. To these
languages must be added \ili{Maltese Sign Language} ({\em Lingwa tas-Sinjali Maltija}; LSM), which is the language
of the Maltese \isi{Deaf} community. LSM was recenty recognised as \isi{Malta}’s
third official language by an Act of Parliament in 2016.

%%AG: Tried to flesh out the below a little bit more, though I don't think it needed much work
%
%Without pretending to give a complete account of the languages of \isi{Malta}, the present volume is an attempt at describing various interesting aspects of this unique linguistic diversity 
While a volume such as the present one can hardly do justice to all
aspects of a diverse and complex linguistic situation, even in a small
community like that of \isi{Malta}, our aim in editing this book was to
shed light on the main strands of research being undertaken in the
Maltese linguistic context.

\section{Overview of the volume}
Of the three languages (or, in the case of \ili{Maltese English}, varieties)
represented in this collection, Maltese is perhaps the best-studied,
with a rich tradition of descriptive and theoretical work and, more
recently, experimental and computational studies. Maltese is the focus
of six of the contributions in this book.

Puech's paper on ``Loss of \isi{emphatic} and \isi{guttural} consonants'' traces
the development of \isi{emphatic} obstruents and gutturals that Maltese
inherited from \ili{Arabic}, but which underwent substantial change in the
transition from Medieval to Contemporary Maltese. Puech's argument
centres on evidence from documentary and other sources in the history
of Maltese which, while written, nevertheless contain valuable
insights and observations into ongoing changes in the Maltese sound
system, enabling the contemporary linguist to map such changes over
the long term.

By contrast, Galea and Ussishkin's paper on ``Onset clusters, \isi{syllable structure} and \isi{syllabification} in Maltese'' 
contributes to an already sizeable body of work on the description of Maltese phonotactic constraints and \isi{syllable structure}, 
here couched within an Onset-Rhyme model and stressing the role of \isi{sonority} 
in determining possible onset clusters in Maltese syllables, yielding an exhaustive and fine-grained 
description of possible clusters that will provide solid grounds for future work on Maltese 
\isi{syllabification} strategies and phonotactics.

The contribution by Paggio, Galea and Vella, entitled ``Prosodic and
\isi{gestural} marking of complement fronting in Maltese", is also concerned
with \isi{phonological} processes, but focusses on their interaction with
\isi{gesture} in spoken Maltese, a topic which has received comparatively
little attention. The authors rely on a sample of annotated,
spontaneous conversations in Maltese, identifying a subset of
utterances that evince complement fronting, which is further broken
down into subtypes (\isi{topicalisation}, \isi{focus movement} and left
dislocation). These instances are further analysed according to
\isi{gestural} and prosodic characteristics, showing that fronted
complements have a strong tendency to be accompanied by gestures and a
falling \isi{pitch accent}. At the same time, the \isi{phonological} complexity
and the tendency to co-occur with gestures is also dependent on the
type of complement fronting in question. To date, this study is one of
only a handful of studies on \isi{gesture} and its interaction with other
levels of linguistic analysis in Maltese.

Of the remaining three contributions on Maltese, two papers, one by
Lucas and Spagnol and another by Gatt and Fabri, focus on
morphology. Like the work of Paggio et al, both have a strong
empirical orientation.

Lucas and Spagnol's paper %on 
``Conditions on %%PP 27.8 Conditions on %%AG part of title was commented out
/t/-insertion in Maltese \isi{numeral} phrases: A reassessment''
investigates the factors which determine the insertion of a /t/ in
cardinal numerals preceding a \isi{plural noun}. The main puzzle here is the
apparent optionality of /t/-insertion. This motivates the question
whether the distribution of /t/-insertion is due to \isi{phonological}
and/or morphological constraints. Lucas and Spagnol present an
exhaustive analysis of data collected from a production experiment in
which \isi{numeral} phrases were elicited orally, using nouns with complex
word-initial clusters consisting of two consonants. Their conclusion
is that the primary influence on /t/-insertion is %added 'a' PP 27.8 
a morphological pattern, though this also interacts with \isi{phonological}
properties. According to %the 
these new findings, certain
morphological patterns determining the arrangement of root consonants
and vowels in \isi{plural} nouns are strongly resistant to /t/-insertion. At
the same time, the findings do not support a strict separation along
the lines drawn in previous descriptive work, for example, between
whole and broken plurals (the former do allow /t/-insertion, albeit
less frequently). Finally, the authors also shed light on potential
\isi{sociolinguistic} variables, especially gender, that could influence the
inter-\isi{speaker} variation in /t/-insertion.

The paper ``Borrowed affixes and \isi{morphological productivity}: A case
study of two Maltese nominalisations'' by Gatt and Fabri deals with
\isi{derivational} processes in Maltese. In particular, it focusses on two
non-\ili{Semitic} \isi{derivational} suffixes, {\em -Vr} and {\em -(z)zjoni}, and
asks the question how productive they are. The paper gives an outline
of morphological derivation in Maltese, and explains both \ili{Semitic} and
\ili{Romance} \isi{derivational} processes before describing the two
nominalisations of interest. It then presents a careful and detailed
%%AG: Change below based on proof reader's remark
corpus analysis based on data from the {\em
  Korpus Malti}, an online corpus of Maltese. Several different measures of productivity are
estimated, with tests of the degree to which
%and it is also tested to what degree 
the two affixes can be
considered indirectly borrowed, that is
%in other words 
first borrowed from
another language and %gradually becoming likely to be used on native
%stems to form novel derivations. 
then gradually becoming likely to form novel derivations in combination
with native stems.  The various statistical measures nicely
converge towards a view of {\em -Vr} as the more productive of the two
deverbal suffixes, and the more likely to be used with both \ili{Semitic}
and \ili{Romance} stems, in spite of {\em -(z)zjoni} being the most
frequently used.

The final paper on Maltese is Camilleri's contribution ``On \isi{raising}
and copy \isi{raising} in Maltese''. Here, Camilleri seeks to give, first, a
descriptive account and a typology of types of \isi{raising} phenomena in
Maltese%; 
; and second, a formalisation couched within the framework of
Lexical-Functional Grammar (\isi{LFG}). Consistent with this lexicalist
orientation, Camilleri first seeks to identify the properties of
\isi{raising} predicates and gives a precise characterisation of their
lexical entries, before proposing a twofold account of \isi{raising},
whereby some \isi{raising} phenomena are accounted for in terms of
structure-sharing, determined via constraints stipulated at the level
of {\em f(unctional)}-structure, while others are better explained in
terms of \isi{anaphoric binding}. Camilleri's work, while an important
contribution to \isi{LFG} in its own right, is also strongly empirical in
flavour, with conclusions %being driven by attention to
based on
naturally-occurring examples obtained from corpora, among other
sources.

The study of \ili{Maltese English}, especially with the purpose of
establishing the defining characteristics of this variety of English,
is a relatively new area of research. Three of the contributions
included in this volume deal with \ili{Maltese English}, which is explored
from the different perspectives of rhythm, the syntax of nominal
phrases and lexical choice.

The paper by Grech and Vella, ``Rhythm in \ili{Maltese English}'', studies
variability in \isi{vowel duration} in six \ili{Maltese English} speakers. An
average \isi{durational variability} measure is calculated for each \isi{speaker}
in terms of a normalised Pairwise Variability Index (nPVI), which is
based on the differences in duration between all successive \isi{vowel}
pairs. The six speakers were rated in a previous study for the degree
to which they could be identified as speakers of \ili{Maltese English}. In
the present paper, the authors find a negative correlation between the
speakers' nPVI and their degree of identifiability as \ili{Maltese English}
speakers. In other words, the less variability in \isi{vowel duration} they
display, the more they are perceived as speaking \ili{Maltese English}. This
correlation indicates that rhythm, measured in terms of \isi{vowel}
duration, is a significant feature in listeners' perception of a
specific Maltese variety of English.

The paper by Schembri ``On the characterisation of \ili{Maltese English}''
applies \isi{error analysis} to identify fossilised transfer errors that
have acquired status as stable features of \ili{Maltese English}. A
theoretical distinction is made between developmental errors on the
one hand, which are due to simplification of \isi{target language}
structures%; 
, and transfer errors on the other. The latter are caused
by \isi{native language} interference. When transfer errors still appear at
\isi{advanced learner} level, and occur systematically in a community of
speakers in a \isi{bilingual} context, they can be said to mark a regional
variant of the language. The empirical data studied in the paper
consist of a corpus of 7,500 \isi{noun} phrases extracted from English
examination scripts by Maltese university students. Schembri discusses
errors in the use of prepositions, nominal \isi{affixation} and \isi{compounding},
and concludes that the feature most likely to become a stable marker of 
\ili{Maltese English} is the overuse of the \isi{preposition} {\em of}.
%the most likely to become a stable feature of \ili{Maltese English} is the
%overuse of the \isi{preposition} {\em of}.

The third paper on \ili{Maltese English}, ``Language change in Maltese
English: The influence of age and parental languages'' by Krug and
S{\"o}nning, deals with lexical choice in \ili{Maltese English} between
British and American variants. The paper
presents data from a questionnaire in which 424 Maltese
informants were asked about their preferences concerning 
%about 60 pairs of
lexical variants. The results are described, and specific words
are discussed in detail. A mixed-effects model of the data is
then run with age and the parents' \isi{native language} as factors,
and it is found that age has the strongest effect on the informants' preferences. 
Interestingly, the pattern created by age shows an
increasingly stronger trend towards less British usage in the youngest
generations. The authors take this as evidence of an ongoing change,
probably due to globalisation. The model also shows that the mother's
language has a stronger influence on informants' choices than the
father's, probably due to the different roles of the two parents in
Maltese families.

The last contribution to this volume, ``\ili{Maltese Sign Language}:
Parallel interwoven journeys of the \isi{Deaf} community and the
researchers'' by Marie Azzopardi-Alexander, discusses the way in which
LSM has evolved in parallel with the
development of LSM research. The author explains how \isi{sign} languages
emerge naturally when communities of profound \isi{deaf} people are
formed. The origins of LSM can probably be traced back to the 70's,
when young Maltese signers started to develop the first signs distinct
from British Sign Language, and which reflected specific traits of Maltese
society. Initial iconic gestures used for every day purposes changed
gradually into conventionalised signs, and the vocabulary of LSM grew
rapidly to include the abstract signs necessary to cover
the vocabulary of school subjects for which a \isi{sign} interpreting
service had become available. The author argues that LSM research has
played a crucial role in empowering \isi{deaf} signers and directly
contributed to the LSM \isi{vocabulary growth} by involving \isi{Deaf} users in
the \ili{Maltese Sign Language} Research Project at the University of \isi{Malta}
Institute of Linguistics. The material made available through the
project also stimulated important studies on several aspects of LSM,
which are briefly summarised in the paper.

\section{Summary}
In summary, we believe the present volume has the potential to present
a unique snapshot of a complex linguistic situation in a
geographically restricted area. Given the nature and range of topics
proposed, the volume will likely be of interest to researchers in both
theoretical and comparative linguistics, as well as those working with
 experimental and corpus-based methodologies. Our hope is that the 
studies presented here will also serve to pave the way for further research
on the languages of \isi{Malta}, encouraging researchers to also take new directions, including
the exploration of variation and \isi{sociolinguistic} factors which, while often raised as 
explanatory constructs in the papers presented here, remain under-researched.

\end{document}
