\chapter{Introduction}
\section{The language and its speakers}\label{sec-language}

Ruruuli-Lunyala (JE103; Glottocode: ruul1235, ISO 639-3: ruc) is known under a number of  spelling variants including Ruuli, Ruli, Luruuri-Lu\-nya\-ra, Ruruli-Lunya\-ra, Ruruuli-Runyala, as well Luduuli. 
Ruruuli-Lu\-nyala is the native language of the Baruuli\footnote{As in other Bantu languages, the prefix \emph{ba-} in Ruruuli-Lu\-nyala is used with human nouns in plural and thus appears on names of various ethnic groups. 
Individual members of the two ethnic groups are endonymically referred to as \emph{Muruuli} and \emph{Munyala}, respectively. 
The names of the areas associated with the two kingdoms have the prefix \emph{bu-} and are called \emph{Buruuli} and \emph{Bunyala}.} and Banyala ethnic groups which mainly reside in the districts of Kayunga and Nakasongola, as well as in Kiryandongo, Amolator, Buyende, Masindi, Hoima, and Luweero of Uganda (see Figure~\ref{fig-Districts} for the map of the first two districts). 
According to \citet[71]{Uganda2016National}, there are 190,122  ethnic Baruuli and 47,699 Banyala. 


\begin{figure}[!htb]
\begin{center}
  \includegraphics[width=0.9\linewidth]{figures/Districts.jpg}
  \caption{Nakasongola and Kayunga \citep{OpenStreetMap}}\label{fig-Districts}
\end{center}
\end{figure}

%\begin{figure}[!htb]
%\begin{center}
%  \includegraphics[width=0.5\linewidth]{figures/Kayunga.jpg}
%  \caption{The district of Kayunga\\ \citep{wikiKayunga}}\label{fig-Kayunga}
%\end{center}
%\end{figure}

%\begin{figure}[!htb]
%\begin{center}
%  \includegraphics[width=0.5\linewidth]{figures/Nakasongola.jpg}
%  \caption{The district of Nakasongola\\\citep{wikiNakasongola}}\label{fig-Nakasongola}
%\end{center}
%\end{figure}

\largerpage[-1]
Following \citet{Ladefogedetal1972Language} and \citeauthor{Schoenbrun1994Great} (\citeyear{Schoenbrun1994Great,Schoenbrun1997Historical}),\footnote{Schoenbrun's genealogical study did not include any data from Ru\-ruu\-li\hyp{}Lu\-nya\-la, presumably,  as there were not available, however, ``RuRuli" is listed as part of the Rutara group \citet[118–119]{Schoenbrun1994Great}.
It is followed by a list of languages and dialects omitted from Map 1 and this list includes ``RuRuli" in the Rutara group.}  
\citet{Hammarstrometal2017Glottolog32} classify Ru\-ruu\-li\hyp{}Lu\-nya\-la as a North Rutara < Rutara < West Nyanza language of the Great Lakes Bantu group of languages. 
The degree of mutual intelligibility between the two major varieties viz.\,Ruruuli and Lunyala is high. 
On the basis of a word list of over 90 items, \citet[74]{Ladefogedetal1972Language} identified that Lunyala and Ruruuli have 91\% of the words in common.  
Similarly, \citet{VanderWaletal2005Luruuri} report 90\% mutual intelligibility between these two language varieties. 
They furthermore mention that these two communities share a common heritage such as a similar clan system, names and nomenclature systems and strategies, traditional cuisine, as well as marriage ceremonies among others. 
The  Baruuli and Banyala are listed as two independent ethnic groups among the 65 indigenous communities by the Constitution of the Republic of Uganda (as amended in 2005, 2009, and 2018).\footnote{Banyala (spelled \emph{Banyara}) as item 22 and Baruuli (spelled \emph{Baruli}) as item 27, \url{https://ulii.org/akn/ug/act/statute/1995/constitution/eng@2018-01-05}, accessed 7 March 2021.} 
According to \citet{VanderWaletal2005Luruuri} and \citet[179]{Nakayiza2013Sociolinguistics}, Ru\-ruu\-li\hyp{}Lu\-nya\-la has three dialects, including Western Ruruuli spoken in Masindi district, Eastern Ruruuli spoken in Nakasongola district, and Lunyala mainly spoken in Kayunga district. 
Further research into the dialectal variation of Ru\-ruu\-li\hyp{}Lu\-nya\-la is needed to either confirm or modify this claim.  
%(see also \citealt{NamyaloSubmitted} for further details). 
The four varieties mentioned in the dictionary entries are currently Ruruuli, Lunyala, Buyende, and Kiryandongo (see Section~\ref{sec-userguide-dialect}).

\section{A short history of Buruuli and Bunyala }
Before their annexation to Bunyoro Kingdom in the 15th c., Buruuli and Bunyala were independent chieftaincies. 
Buruuli had its capital in Kamunina in Kwigeri, while Bunyala’s capital was at Ibbaale (now, in the Bbaale County) (see \citealt{Semakula1972History} and \citealt{Musoke2005Buruuli}). 
After prolonged wars between the Buganda and Bunyoro Kingdoms, Buruuli and Bunyala were annexed to Buganda in a deal that was sealed by the 1900 Buganda Agreement, a treaty that formalized the relationship between the Kingdom of Uganda and the British Uganda Protectorate (see e.g.\,\citealt{Green2008Understanding} and \citealt{Stonehouse2012Peripheral} on the history of Uganda's “Lost Counties”). 

After the annexation of Buruuli and Bunyala to Buganda, the Baganda colonial agents took over full administration and management of these chieftaincies. 
Many of the interviewed elders report that the Baganda agents’ administration brutality forced many Baruuli and Banyala to flee to different parts of the country, leaving behind a small population of the Ru\-ruu\-li\hyp{}Lu\-nya\-la speakers. Those who remained were deprived of their traditional rights on land and became squatters. 

The political situation had sociolinguistic consequences. 
Thus, \citet[: 42]{Mwogezi2004History} reports that, in addition to the forced labour which led to forced migration, “in their push for assimilation, or to Bugandanise the lost counties, under threat of jail or other punishments, all indigenous languages spoken in these counties were banned. Luganda was made compulsory in all the counties that were colonised by the Buganda Kingdom”. Thus, Baganda colonial agents made it a policy to use Luganda in all public domains such as in churches, mosques, schools, courts of law as well as in administrative/government offices. Ru\-ruu\-li\hyp{}Lu\-nya\-la remained only a “home” language as it was illegal to speak Ru\-ruu\-li\hyp{}Lu\-nya\-la in public and anyone found speaking it would be severely punished. One of the interviewees who preferred anonymity recalls that “one day, my father and I were heard speaking Lunyala in a market. We did not know the chief was near. A day later, he was summoned to go to the court for trial. I was also told to go with my father. We were both found guilty of speaking Lunyala in a public place. My dad was sentenced to two months in jail. As a child, I was told to draw water and fill a drum every day for seven days” (Field Interview, January 2018). The Baruuli and Banyala who failed to master Luganda were denied access to services like medical care, education and attending churches. Thus, the reduction in the number of Ru\-ruu\-li\hyp{}Lu\-nya\-la speakers, the poor economic status and the language policy that was introduced by the Baganda forced many Banyala and Baruuli to adopt Luganda as a language for survival and a language of wider communication reducing Ru\-ruu\-li\hyp{}Lu\-nya\-la to a home language. 


\section{Data gathering techniques and the corpus}\label{sec-Data}

The data for this grammar sketch and dictionary were collected during multiple field trips in 2012 and in 2016–2020 carried out by the authors of this book. 
The present dictionary and the grammar were meant to be representative of the Ruruuli and Lunyala varieties and if possible to capture further varieties. 
To reach this goal, a large number of speakers from different locations were recorded and interviewed (see below). 

The two major field trip locations were Kayunga and Nakasongola. These two towns also served as the location of several community workshops dedicated to lexicographic topics and the compilation of the corpus.
In addition to these two major sites, further trips were undertaken to a number of other locations in Kayunga district, Nakasongola district, and Buyende district. 
Furthermore, speakers from a large number of locations participated in the dictionary workshops and corpus compilation workshops. 
Furthermore, several language consultants regularly visited Kampala to work with the project members.

The grammar sketch and the dictionary are based on four major sources of data, viz.\,a corpus of naturalistic texts, elicitations of specific grammatical forms and constructions and grammaticality judgments, elicitations for phonetic analysis, as well as sociolinguistic questionnaires.
The first two data sources are presented in some detail below.

The corpus of naturalistically produced texts and traditional narratives was collected during these trips, as well as during a field trip in 2012 by Saudah Namyalo with the purpose of collecting traditional narratives. 
Most of the sessions were video and audio recorded. 
On many occasions the linguist in charge of a recording session withdrew from the conversation and left speakers to themselves to encourage a more natural conversation. 
To encourage the use of specialised vocabulary, on many occasions the speakers were offered a range of topics and questions to discuss during the recording sessions (e.g.\,traditional naming practices, fishing techniques, preparation of local dishes and drinks, etc.).

As of 2020, the corpus contains speech of 105 speakers (79 male and 26 female speakers) of various age groups and backgrounds. 
The total length of recordings exceeds 40 hours. 
The size of the transcribed corpus of spoken language available to the team exceeded 205,000 words.
The majority of the naturalistically produced texts were recorded in Kayunga and Nakasongola during the workshops organised by the project. 
Further recordings were made in the following locations: 
Kibbaale and Ki\-de\-ra (Buyende District),
Katitiza (Nakasongola district),
Bbaale, Gweero, Ka\-goye, Kikota, Kitatyana, Kyerima, Mugongo, and Sokoso (Kayunga district). 
Each recording session was supplemented with metadata: we kept track of the data and place of recordings, participants, their age, provenance, linguistic competence, and education.

The corpus of naturalistically produced texts was supplemented with available written data (collections of proverbs produced by the communities, readers, books about the history of the kingdom, the translation of the portions of the Bible produced by SIL in Uganda). 
The size of the corpus of written language exceeded 45,000 words.

In addition to the corpus of naturalistically produced speech, multiple elicitations sessions were organised by the project members to study individual phonological and morpho-syntactic topics, as well as to work on senses and grammatical properties of lexical items in the dictionary, and to collect examples sentences. 
The elicitation sessions were conducted in English and Luganda. 
All the consultants who participated in the elicitation sessions were proficient in Ru\-ruu\-li\hyp{}Lu\-nya\-la, Luganda, and English.
