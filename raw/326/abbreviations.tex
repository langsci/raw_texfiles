\addchap{Abbreviations and conventions}
% \addchap{Abbreviations and symbols}

This section outlines the conventions used in the grammar sketch. 
The conventions used in the dictionary are presented separately in the dictionary user's guide.

In  the examples of individual word, e.g.\,in~(\ref{ex-phon-consonants}), <…> angle brackets are used represent the orthographic letters (or graphemes). 
[…] square brackets are used for the phonetic surface representation, whereas slashes /…/ are used for the underlying phonological representation, as well as to refer to allomorphs of a morpheme. 
We use single quotes `…' for translations of Ru\-ruu\-li\hyp{}Lu\-nya\-la words and clauses into English. 

We represent Ru\-ruu\-li\hyp{}Lu\-nya\-la cited within the text and examples without glosses in italics. 
No italics are used in the examples with glosses, such as e.g.\,(\ref{ex-NC-locative-all}). 
In glossed examples, the first line represents the orthographic transcription, the second line represents the underlying form of individual morphemes, the third line represents the glosses, and finally, in the fourth line, an idiomatic translation into English is provided and, if necessary, supplemented with a more literal translation.

The glosses follow the Leipzig Glossing Rules.\footnote{See \url{https://www.eva.mpg.de/lingua/resources/glossing-rules.php}.} The abbreviations used in the grammar sketch and in the dictionary part are listed below. 
Note that the numbers 1 and 2 are used to indicate both to the first and second person and the noun class 1 and 2. 
They two uses are easily distinguished: the first and second person are always accompanied by the indication of number (singular and plural, e.g.\,1\textsc{sg}), whereas the noun classes 1 and 2 are never accompanied by the indication of number, as nouns of the noun class 1 are singular, and nouns of noun class 2 are plural (see Section~\ref{sec-morh-noun-class} for details).

\begin{tabularx}{.55\textwidth}{lQ}
1\textsc{sg} &  first person singular\\
1\textsc{pl} & first person plural\\
2\textsc{sg} &  second person singular\\
2\textsc{pl} & second person plural\\
1, 2, 3, etc. & noun class (the number is not followed by either \textsc{sg} or \textsc{pl})\\
\textsc{add} & additive\\
adj. & adjective\\
adv. & adverb\\
\textsc{appl} & applicative\\
Ar. & Arabic\\
\textsc{assoc} & associative\\
\textsc{aug} & augment\\
\textsc{aux} & auxiliary\\
C & consonant\\
\textsc{caus} & causative\\
\textsc{com} & comitative\\
\textsc{conj} & conjunction\\
conj. & conjunction\\
\textsc{cop} & copula\\
\textsc{dist} & distal\\
En. & English\\
\textsc{ex} & existential\\
\textsc{foc} & focus\\
\textsc{fut} & future\\
\textsc{fv} & final vowel\\
G & glide\\
\textsc{gen} & genitive\\
H & high tone\\
\textsc{hab} & habitual\\
ideo.& ideophone\\
\textsc{inf} & infinitive\\
interj.& interjection\\
interrog.& interrogative pronoun\\
intr. & intransitive\\
lit.  & literally\\
L & low tone\\
\textsc{loc} & locative\\
Lug. & Luganda\\
\end{tabularx}
\begin{tabularx}{.44\textwidth}{lQ}
\textsc{med} & medial\\
n. & noun\\
N & nasal\\
\textsc{nar} & narrative\\
NC & noun class\\
\textsc{neg} & negation\\
\textsc{num} & numeral\\
\textsc{obj} & object\\
part. & particle\\
\textsc{pass} & passive\\
\textsc{pers} & persistive\\
pf. & prefix\\
\textsc{pfv} & perfective\\
\textsc{pl} & plural\\
Port. & Portuguese\\
\textsc{poss} & possessive\\
prep. & preposition\\
pro. & pronoun\\
\textsc{prog} & progressive\\
\textsc{prox} & proximate\\
\textsc{prs} & present\\
\textsc{pst} & past\\
\textsc{rec} & recent\\
\textsc{recp} & reciprocal\\
\textsc{red} & reduplication\\
\textsc{refl} & reflexive\\
\textsc{rel} & relativiser\\
\textsc{rem } & remote\\
\textsc{rfut} & remote future\\
sf. & suffix\\
\textsc{sbj} & subject\\
\textsc{sbjv} & subjunctive\\
\textsc{sg} & singular\\
Sw. & Swahili\\
\textsc{ta} & tense, aspect\\
tr. & transitive\\
\textsc{tam} & tense, aspect, mood\\
UgE & Ugandan English\\
V & vowel\\
v. & verb\\
\end{tabularx}
