\chapter{Syntax}

This section provides an overview of some aspects of Ruruuli-Lu\-nya\-la syntax. 
After an introduction to the clause structure in Section~\ref{sec-clause}, a range of syntactic topics related to the noun phrase structure are discussed in Section~\ref{sec-syntax-NP}. 
The syntax of adverbs is addressed in Section~\ref{sec-syntax-adverbs}.  
Various diathesis-related phenomena (including extensions) are discussed in Section~\ref{sec-syntax-extensions}. 
A range of further syntactic topics covered in this grammar sketch reflects the present state of description Ruruuli-Lunyala. 
They include the syntax of interrogatives (Section~\ref{sec-interrogatives}), non-verbal predication (Section~\ref{sec-non-verbal-predication}), modal verbs (Section \ref{sec-modal-verbs}), as well as complementation strategies (Section~\ref{sec-syntax-complementation}). 

\section{Basic verbal clause and constituent order}\label{sec-clause}

The basic constituent order of the declarative independent clause in  Ruruuli-Lunyala  is SVO, i.e. the subject of a transitive or an intransitive verb precedes the verb, as in~(\ref{ex-syntax-clause-intr}).
If there is an object, it most frequently immediately follows the verb, as in~(\ref{ex-syntax-clause-tr}). 
However, in some circumstances, objects can be fronted. 
In this case they are indexed on the verbs by means of the respective object prefix, as e.g.\,using the prefix \emph{mu-} `1O' in~(\ref{ex-syntax-clause-fronted}), see also Section~\ref{sec-verb-indexing} on argument indexing.
	
\ea \label{ex-syntax-clause}
\begin{xlist}	
\ex   \label{ex-syntax-clause-intr}
    \glll    Omuyembe gunula muno.\\
     o-muyembe gu-nul-a muno\\
	\textsc{aug}-3.mango \textsc{3sbj}-taste.well-\textsc{fv} much\\
    \glt  `A mango is very tasty.’

\ex   \label{ex-syntax-clause-tr}
    \glll   Omusuma aibbire obwita bwange.\\
      o-musuma a-ibb-ire o-bwita bu-a-ange\\
	\textsc{aug}-1.thief \textsc{1sbj}-steal-\textsc{pfv} \textsc{aug}-14.millet 14-\textsc{assoc}-\textsc{1sg}\\
    \glt  `A thief has stolen my millet.’
    
 \ex   \label{ex-syntax-clause-fronted}
    \glll  Omwana amugingiire ku mabega.\\
      o-mwana a-mu-gingiire ku mabega\\
	\textsc{aug}-1.child \textsc{1sbj}-\textsc{1obj}-carry:\textsc{appl:pfv} 17.\textsc{loc} 6.shoulder\\
    \glt  `She is carrying the baby on the shoulders.’
\end{xlist}	
\z

Adjuncts which have the form of locative phrases appear in two positions. 
They either follow the verb and the object, if there is one, such as e.g.\,the locative adjunct \emph{oku kyegeseryo} `at the college' in~(\ref{ex-syntax-clause-adj2}) or the temporal adjunct \emph{oku Paasika} `on Easter Sunday' in~(\ref{ex-syntax-clause-adj3}). 
Less commonly they appear clause initially, as in~(\ref{ex-syntax-clause-adj-initial}). 
In case of locative phrases indicating location, the verb can agree with the locative phrase, as in~(\ref{ex-syntax-clause-adjinitial4}). 
This construction is common with copulas but not with other verbs, and even with copulas it is not obligatory, as~(\ref{ex-syntax-clause-adjinitial2}) suggests.\footnote{Bantu languages are known to vary with respect to the properties and restrictions on the locative inversion, see e.g.\,\citet{Salzmann2011Towards} for a recent overview. Further research is needed to describe these characteristics in Ruruuli\hyp{}-Lunyala.}
The position of adverbs in a clause is discussed in Section~\ref{sec-syntax-adverbs}.

\ea \label{ex-syntax-clauseadj}
\begin{xlist}	
\ex   \label{ex-syntax-clause-adj2}
    \glll    Nkusoma saayansi oku kyegeseryo.\\
     n-ku-som-a saayansi o-ku kyegeseryo\\
\textsc{1sg.sbj}-\textsc{prog}-study-\textsc{fv} 9.science \textsc{aug}-\textsc{17.loc} 17.college\\
    \glt  `I am studying science at the college.’

\ex   \label{ex-syntax-clause-adj3}
    \glll   Naizayo oku Paasika.\\
     n-a-iz-a=yo o-ku Paasika\\
	\textsc{1sg.sbj}-\textsc{fut}-come-\textsc{fv}=23.\textsc{loc} \textsc{aug}-17.\textsc{loc} 9.Easter\_Sunday\\
    \glt  `I will come there on Easter Sunday.’
\end{xlist}	
\z

\largerpage
\ea \label{ex-syntax-clause-adj-initial}
\begin{xlist}	
\ex   \label{ex-syntax-clause-adjinitial1}
    \glll    E Buganda omusaiza talya bifuka.\\
     e Buganda o-musaiza ti-a-li-a bifuka\\
\textsc{23.loc} 14.Buganda \textsc{aug}-1.man \textsc{neg}-\textsc{1sbj}-eat-\textsc{fv} 8.leftovers\\
    \glt   `In Buganda a man does not eat leftovers.’

\ex   \label{ex-syntax-clause-adjinitial2}
    \glll    Oku isana ente zirya bisagazi.\\
    o-ku isana e-nte zi-li-a bisagazi\\
\textsc{aug}-\textsc{17.loc} 5.drought  \textsc{aug}-10.cow 10\textsc{sbj}-eat-\textsc{fv} 8.elephant\_grass\\
    \glt `During a drought cows eat elephant grass.’	
 
\ex   \label{ex-syntax-clause-adjinitial4}
    \glll   Oku kyalo kini kuliku oMubiito?\\
     o-ku kyalo ki-ni ku-li=ku o-Mubiito\\
	\textsc{aug}-\textsc{17.loc} 7.village 7-\textsc{prox} 17.\textsc{sbj}-\textsc{cop}=\textsc{17.loc} \textsc{aug}-1.Mubiito\\
    \glt  `Is there any Mubiito (a person from the bushbuck clan)  in this village?’

\ex   \label{ex-syntax-clause-adjinitial3}
    \glll   Oku nzira ya Jinja bukanga ku kagera Waligga.\\
   o-ku nzira ya Jinja bu-kang-a ku kagera Waligga\\
	\textsc{aug}-\textsc{17.loc} 9.road \textsc{9.gen} 9.Jinja 14-end-\textsc{fv} \textsc{17.loc} 12.river 9.Waligga\\
    \glt  `On Jinja road it (the river Sezibwa) ends at a small river Waligga.’
        	
\end{xlist}	
\z

\section{Clauses with non-verbal predicates}\label{sec-non-verbal-predication}

Four copulas are used to express non-verbal predication in Ruruuli-Lu\-nya\-la.\footnote{This section benefited from the account of non-verbal predication in \citet{Lorenzen2018Nonverbale}.} These are 
\textit{ni}, \textit{li},\footnote{Note that \textit{li} is glossed as a copula (\textsc{cop}) in this section  whereas it some examples earlier it was glossed as an auxiliary (\textsc{aux}), e.g.\,in (\ref{ex-ruulipers}). 
The different glosses are chosen to indicate the respective function of \textit{li} in Ru\-ruu\-li\hyp{}Lu\-nya\-la.} \textit{ta} and \textit{ti}. 
Additionally, the semi-copula \textit{bba} can be used in some of the contexts where the real copulas occur, i.e.\,in locational clauses and predicative possession. 
The indexing of arguments on the copulas is summarized in Section~\ref{sec-morpho-copulas}.

With proper inclusion, equative and attributive clauses, the usage of a copula is not required with a non-pronominal subject in the present tense, as in~(\ref{ex-ruulinocopula1})--(\ref{ex-ruulinocopula3}). 
Note the absence of the augment on nominal predicates.

\ea \label{ex-ruulinonverbal}
\begin{xlist}
	\ex \label{ex-ruulinocopula1}
	\glll Omukali musomesya.\\
		  o-mukali musomesya\\
		\textsc{aug}-1.woman 1.teacher\\
	\glt ‘The woman is a teacher.’ 

\ex\label{ex-ruulinocopula2}
	\glll Omukali ma wange.\\
	  o-mukali ma wa-a-nge\\
		\textsc{aug}-1.woman 1.mother 1-\textsc{assoc}-\textsc{1sg}\\
	\glt ‘The woman is my mother.’ 	

\ex \label{ex-ruulinocopula3}
	\glll Omukali mukooto.\\
		  o-mukali mu-kooto\\
		\textsc{aug}-1.woman 1-big\\
\glt ‘The woman is big.’
\end{xlist}
\z

In non-present tenses, as in~(\ref{ex-ruulinocopula4})--(\ref{ex-ruulinocopula6}), as well as with pronominal subjects, as in~(\ref{ex-ruulinocopula7}), predication of proper inclusion, equation and attribution requires the copula \textit{li}. 

\ea \label{ex-ruulinonverbal2}
\begin{xlist}
\ex \label{ex-ruulinocopula4}
	\glll Omukali yali musomesya.\\
	  o-mukali a-a-li musomesya\\
		\textsc{aug}-1.woman \textsc{1sbj}-\textsc{pst}-\textsc{cop} 1.teacher\\
	\glt ‘The woman was a teacher.’ 
	
\ex \label{ex-ruulinocopula5}
	\glll Omukali yali ma wange.\\
	  o-mukali a-a-li ma wa-a-nge\\
		\textsc{aug}-1.woman 1-\textsc{pst}-\textsc{cop} 1.mother 1-\textsc{assoc}-\textsc{1sg}\\
	\glt ‘The woman was my mother.’ 
	
\ex \label{ex-ruulinocopula6}
	\glll Omukali yali mukooto.\\
		   o-mukali a-a-li mu-kooto\\
		\textsc{aug}-1.woman 1-\textsc{pst}-\textsc{cop} 1-big\\
	\glt ‘The woman was big.’

\ex\label{ex-ruulinocopula7}
	\glll Ati oli musaiza mukulu.\\
	  ati o-li musaiza mu-kulu\\
		now 2\textsc{sg.sbj}-\textsc{cop} 1.man 1-mature\\
	\glt ‘Now you are a mature man.’ 

\end{xlist}
\z

The copulas \textit{li}, \textit{ta} and the semi-copula \textit{bba} are used in locative predication. 
The speakers claim that \textit{ta} in most cases is used interchangeably with \textit{li}. 
Furthermore, \textit{ta} is said to be an archaic form that is still in use in some dialects. 
On the other hand, in some contexts \textit{ta} implies an element of exclusiveness that is not covered by \textit{li}, as in~(\ref{ex-ruuliexclusive}). 
Further research is needed to determine the semantic and other differences in the use of these two copulas.

\ea \label{ex-ruuliexclusive}
	\glll Olusozi Rwenzori lu-ta mu Uganda.\\
	  o-lusozi Rwenzori lu-ta 			mu Uganda\\
		\textsc{aug}-11.mountain 11.Rwenzori 11-\textsc{cop} 18.\textsc{loc} 1.Uganda\\
\glt ‘Moutain Rwenzori is (found) in Uganda (implied: only in Uganda).’ 
\z

Following \citet[6]{Pustet2003Copulas}, we classify \textit{bba} as a semi-copula since it is not semantically empty but adds meaning to the predicate phrase. 
The semi-copula can be translated as `to stay, remain, become' and additionally conveys temporal meaning which is not the case for the true copulas \textit{li} or \textit{ta}, 
compare~(\ref{ex-ruulibba}) with (\ref{ex-ruulitaa})–(\ref{ex-ruulili}).

\ea \label{ex-ruulinonverbal3}
\begin{xlist}
	\ex \label{ex-ruulibba}
	\glll Omukali abba mu Kampala.\\
	  o-mukali a-bba mu Kampala.\\
		\textsc{aug}-1.woman \textsc{1sbj}-\textsc{cop} 18.\textsc{loc} 1.Kampala\\
\glt ‘The woman lives/stays in Kampala.’

	\ex \label{ex-ruulitaa}
	\glll Omukali ata mu Kampala.\\
	  o-mukali a-ta mu Kampala.\\
		\textsc{aug}-1.woman \textsc{1sbj}-\textsc{cop} 18.\textsc{loc} 1.Kampala\\
\glt ‘The woman is in Kampala.’ 

	\ex \label{ex-ruulili}
	\glll Omukali ali mu Kampala.\\
		  o-mukali a-li mu Kampala\\
		\textsc{aug}-1.woman \textsc{1sbj}-\textsc{cop} 18.\textsc{loc} 	1.Kampala\\
\glt ‘The woman is in Kampala.’ 	
\end{xlist}
\z

Bantu languages often encode predicative possession by means of a copula and the comitative preposition/conjunction \textit{na} `and, with'. 
In Ru\-ruu\-li\hyp{}Lu\-nya\-la, typically, the copula \textit{li} is used, as in~(\ref{ex-ruulilina}).\footnote{In Ru\-ruu\-li\hyp{}Lu\-nya\-la, the copula \textit{li} and the preposition \textit{na} seem to be merging into the verb \textit{lina} `to have' and the dictionary section contains the respective entry (see Part~\ref{sec-Dictionary}). 
The merge of a copula with the comitative preposition has also been observed in other Bantu languages, e.g.\,in Rangi (\citealt[74]{Gibson2012Rangi}).\label{footnote-lina}}
Predicative possession can also be occasionally expressed by means of the less common copula \textit{ta} followed by the preposition \textit{na}, as in~(\ref{ex-ruulitaana}), as well as the semi-copula \textit{bba} followed by \textit{na}, as in~(\ref{ex-ruulibbana}). 
The speakers claim that \textit{li} and \textit{ta} can be used interchangeably and that the preference for one or the other is dialectal, the second option is however very infrequent in the corpus.
There is no difference between temporal and permanent predicative possession. 
The expression of predicative possession by means of the copula \textit{bba} might imply a modal component of `should' or `must', as indicated in the translation of (\ref{ex-ruulibbana}). 

\ea \label{ex-ruulipredicativeposs}
\begin{xlist}
	\ex \label{ex-ruulilina}
	\glll Omukali ali n' abaana.\\
	  o-mukali a-li na a-baana\\
		\textsc{aug}-1.woman \textsc{1sbj}-\textsc{cop} \textsc{com} \textsc{aug}-2.child\\
	\glt ‘The woman has children.’ 

\ex \label{ex-ruulitaana}
	\glll Omukali ata n' abaana.\\
	  o-mukali a-ta na a-baana\\
		\textsc{aug}-1.woman 1-\textsc{cop} \textsc{com} \textsc{aug}-2.child\\
	\glt ‘The woman has children.’ 	

\ex \label{ex-ruulibbana}
	\glll O-mukali abba n' abaana.\\
	 o-mukali a-bba na a-baana\\
		\textsc{aug}-1.woman 1-\textsc{cop} \textsc{com} \textsc{aug}-2.child\\
\glt ‘The woman is supposed to have children./The woman stays with the children.’
\end{xlist}
\z

Existential predication is expressed by a construction with the copula \textit{li}. 
The copula carries both a locative prefix and a locative enclitic of the noun classes 16, 17, 18 and 23, see Table~\ref{tab:locprefixes} for an overview.  
The prefix and the enclitic typically belong to the same locative noun class, as in~(\ref{ex-ruuli16.loc})--(\ref{ex-ruuli18.loc}). 
However, this is not obligatory, see e.g.\, (\ref{ex-ruuli16.23}). 
The use of \textit{waliwo} is by far the most common way to express existential predication in the corpus, 
but other combinations of the copula \textit{li} with a locative prefix and a locative enclitic are attested as well.
Further investigation is needed to identify the conditions of the choice between the various combinations like the one illustrated in~(\ref{ex-ruulimanyloc}).

\begin{table}

\caption{The locative prefixes and enclitics}
\label{tab:locprefixes}
	\begin{tabular}{lll}
\lsptoprule
	class&prefix&enclitic\\
\midrule
	16&\textit{wa-}&\textit{=wo}\\
	17&\textit{ku-}&\textit{=ku}\\
	18&\textit{mu-}&\textit{=mu}\\
	23&\textit{e-}&\textit{=yo}\\
\lspbottomrule
	\end{tabular}
\end{table}


\ea \label{ex-ruuliexistential}
\begin{xlist}
\ex \label{ex-ruuli16.loc}
	\glll Waliwo omuzumu.\\
		  wa-li=wo o-muzumu\\
		1\textsc{6sbj}-\textsc{cop}=16.\textsc{loc} \textsc{aug}-1.devil\\
	\glt ‘There is a devil.’ 
\ex \label{ex-ruuli17.loc}
	\glll Kuliku omwojo.\\
	  ku-li=ku o-mwojo\\
		1\textsc{7sbj}-\textsc{cop}=17.\textsc{loc} \textsc{aug}-1.boy\\
	\glt ‘There is a boy.’ 	
\ex \label{ex-ruuli18.loc}
	\glll Mulimu omuntu. \\
	 mu-li=mu o-muntu\\
		1\textsc{8sbj}-\textsc{cop}=18.\textsc{loc} \textsc{aug}-1.person\\
	\glt ‘There is a person.’	

\ex \label{ex-ruuli16.23}
	\glll Waliyo ekintu ekindi. \\
	wa-li=yo e-kintu e-kindi\\
		1\textsc{6sbj}-\textsc{cop}=23.\textsc{loc} \textsc{aug}-7.thing \textsc{aug}-7.other\\
	\glt ‘There is another thing.’ 
	
\ex\label{ex-ruulimanyloc}
  \glll Mulimu iswe aBalanda, mulimu { } aBayinda, waliwo aBakurubyo, { } { } { } { } waliwo aBapapali.\\
   mu-li=mu iswe a-Balanda mu-li=mu { }  a-Bayinda wa-li=wo a-Bakurubyo { } { } { } { } wa-li=wo a-Bapapali\\
	1\textsc{8sbj}-\textsc{cop}=18.\textsc{loc} 1\textsc{pl} \textsc{aug}-2.Mulanda 1\textsc{8sbj}-\textsc{cop}=18.\textsc{loc} { } \textsc{aug}-2.Muyinda 1\textsc{6sbj}-\textsc{cop}=16.\textsc{loc} \textsc{aug}-2.Mukurubyo { } { } { } { }  \textsc{16sbj}-\textsc{cop}=16.\textsc{loc} \textsc{aug}-2.Mupapali\\
\glt `There are us the Balanda, there are the Bayinda, there are the Bakurubyo, there are the Bapapali.'
	
\end{xlist}
\z

The copula \textit{ni} is used when the subject of the clause is focused. 
In contrast to the other copulas, \textit{ni} takes a subject agreement suffix, the respective word-forms are given above in Table~\ref{tab-cop-agreement} (Section~\ref{sec-morpho-copulas}). 
It can stand alone in predications where no copula is required, i.e.\,in proper inclusion, equative and attributive clauses in present tense with a nominal subject, as in~(\ref{ex-ruulini1})--(\ref{ex-ruulini3}). 
In cases where a copula is required, \textit{ni} cannot replace another copula but can only be added to the clause, as in~(\ref{ex-ruulini4})--(\ref{ex-ruulini5}).

\ea \label{ex-ruulini}
\begin{xlist}
	\ex \label{ex-ruulini1}
	\glll Omukali niye musomesya.\\
	  o-mukali ni-ye musomesya\\
		\textsc{aug}-1.woman \textsc{cop}-\textsc{1sbj} 1.teacher\\
	\glt ‘The woman is the one who is a teacher.’ 

\ex \label{ex-ruulini2}
	\glll Omukali niye ma wange.\\
	  o-mukali ni-ye ma wa-a-nge\\
		\textsc{aug}-1.woman \textsc{cop}-\textsc{1sbj} 1.mother 1-\textsc{assoc}-\textsc{1sg}\\
	\glt ‘The woman is the one who is my mother.’ 	

\ex \label{ex-ruulini3}
	\glll Omukali niye mukooto.\\
	  o-mukali ni-ye mu-kooto\\
		\textsc{aug}-1.woman \textsc{cop}-\textsc{1sbj} 1-big\\
\glt ‘The woman is the one who is big.’

\ex \label{ex-ruulini4}
	\glll *Omukali niye mu Kampala.\\
		o-mukali ni-ye mu Kampala\\
		\textsc{aug}-1.woman \textsc{cop}-\textsc{1sbj} 18.\textsc{loc} 1.Kampala\\
\glt Intended: ‘It is the woman who is in Kampala.’

\ex \label{ex-ruulini5}
	\glll Omukali niye ali mu Kampala.\\
	 o-mukali ni-ye a-li mu Kampala\\
		\textsc{aug}-1.woman \textsc{cop}-\textsc{1sbj} \textsc{2sbj}-\textsc{cop} 18.\textsc{loc} 1.Kampala\\
\glt ‘It is the woman who is in Kampala.’
\end{xlist}
\z

In the negation of proper inclusion, equative and attributive clauses in present tense and with a non-pronominal subject, the invariable negative copula \textit{ti} is used, as in~(\ref{ex-ruulinonverbalneg}). 

\ea \label{ex-ruulinonverbalneg}
\begin{xlist}
\ex \label{ex-ruuliti1}
	\glll Omukali ti musomesya.\\
		  o-mukali ti musomesya\\
		\textsc{aug}-1.woman \textsc{neg}.\textsc{cop} 1.teacher\\
	\glt ‘The woman is not a teacher.’ 
\ex \label{ex-ruuliti2}
	\glll Omukali ti ma wange.\\
		  o-mukali ti ma wa-a-nge\\
		\textsc{aug}-1.woman \textsc{neg}.\textsc{cop} 1.mother 1-\textsc{assoc}-\textsc{1sg}\\
	\glt ‘The woman is not my mother.’ 	

\ex \label{ex-ruuliti3}
	\glll Omukali ti mukooto.\\
		  o-mukali ti mu-kooto\\
		\textsc{aug}-1.woman \textsc{neg}.\textsc{cop} 1-big\\
	\glt ‘The woman is not big.’
\end{xlist}
\z

\noindent In non-present, as in~(\ref{ex-ruuliti6}), as well as with pronominal subjects, as in~(\ref{ex-ruuliti4}), however, the standard negation marker \textit{ti-} is prefixed to the copula \textit{li} (see Section~\ref{sec-negation} on the use of \textit{ti-} with verbs). 
The same marker is prefixed to the copula \textit{ni} when the subject of the negative statement is focused, as in~(\ref{ex-ruuliti5}).

\ea \label{ex-ruulinonverbalneg2}
\begin{xlist}
\ex \label{ex-ruuliti6}
	\glll Tikyali kyangu.\\
		  ti-ki-a-li ki-angu\\
		\textsc{neg}-\textsc{7sbj}-\textsc{pst}-\textsc{cop} 7-easy\\
\glt ‘It was not easy.’

\ex \label{ex-ruuliti4}
	\glll Kubba toli Muganda.\\
	  kubba ti-o-li Muganda\\
		because \textsc{neg}-2\textsc{sg.sbj}-\textsc{cop} 1.Muganda\\
\glt ‘Because you are not a Muganda.’ 

\ex \label{ex-ruuliti5}
	\glll Omukali tiniye mukooto.\\
		  o-mukali ti-ni-ye mu-kooto\\
		\textsc{aug}-1.woman \textsc{neg}-\textsc{cop}-\textsc{1sbj} 1-big\\
	\glt ‘It is not the woman who is big.’ 	
\end{xlist}
\z

As for the negation of locational predication, as in~(\ref{ex-ruulilocli})--(\ref{ex-ruulilocbbaneg}), and predicative possession, as in~(\ref{ex-ruulipredicativeposs}),~(\ref{ex-ruuliposslineg})--(\ref{ex-ruulipossbbaneg}), the copulas \textit{li} and \textit{ta}, as well as the semi-copula \textit{bba} are negated by means of the standard negation marker \textit{ti-}. 

\ea \label{ex-ruulinonverbalneg3}
\begin{xlist}
\ex \label{ex-ruulilocli}
	\glll Omusaale guli mu kibira.\\
		  o-musaale gu-li mu kibira\\
		\textsc{aug}-3.tree \textsc{3sbj}-\textsc{cop} 18.\textsc{loc} 7.forest\\
	\glt ‘The tree is in the forest.’ 
	
\ex \label{ex-ruuliloclineg}
	\glll Omusaale tiguli mu kibira.\\
		  o-musaale ti-gu-li mu kibira\\
		\textsc{aug}-3.tree \textsc{neg}-\textsc{3sbj}-\textsc{cop} 18.\textsc{loc} 7.forest\\
	\glt ‘The tree is not in the forest.’ 	
	
\ex \label{ex-ruuliloctaa}
	\glll Omusaale guta mu kibira.\\
		  o-musaale gu-ta mu kibira\\
		\textsc{aug}-3.tree \textsc{3sbj}-\textsc{cop} 18.\textsc{loc} 7.forest\\
	\glt ‘The tree is in the forest.’ 
	
\ex \label{ex-ruuliloctaaneg}
	\glll Omusaale tiguta mu kibira.\\
		  o-musaale ti-gu-ta mu kibira\\
		\textsc{aug}-3.tree \textsc{neg}-\textsc{3sbj}-\textsc{cop} 18.\textsc{loc} 7.forest\\
\glt ‘The tree is not in the forest.’ 
	
	\ex \label{ex-ruulilocbba}
	\glll Omusaale gubba mu kibira.\\
		  o-musaale gu-bba mu kibira\\
		\textsc{aug}-3.tree \textsc{3sbj}-\textsc{aux} 18.\textsc{loc} 7.forest\\
	\glt `The tree stays in the forest/is supposed to be in the forest.’ 

\ex \label{ex-ruulilocbbaneg}
	\glll Omusaale  tigubba mu kibira.\\
	  o-musaale  ti-gu-bba mu kibira\\
		\textsc{aug}-3.tree \textsc{neg}-\textsc{3sbj}-\textsc{cop} 18.\textsc{loc} 7.forest\\
	\glt ‘The tree does not stay in the forest/is not supposed to be in the forest.’ 	

\ex \label{ex-ruuliposslineg}
	\glll Omukali tali na baana. \\
	  o-mukali ti-a-li na baana\\
		\textsc{aug}-1.woman \textsc{neg}-\textsc{1sbj}-\textsc{cop} \textsc{com} 2.child\\
\glt ‘The woman does not have children.’ 
	
	\ex \label{ex-ruuliposstaaneg}
	\glll Omukali tata na baana.\\
	  o-mukali ti-a-ta na baana\\
		\textsc{aug}-1.woman \textsc{neg}-\textsc{1sbj}-\textsc{cop} \textsc{com} 2.child\\
\glt ‘The woman does not have children.’

	\ex \label{ex-ruulipossbbaneg}
	\glll Omukali tabba na baana. \\
	  o-mukali ti-a-bba na baana\\
		\textsc{aug}-1.woman \textsc{neg}-\textsc{1sbj}-\textsc{cop} \textsc{com} 2.child\\
\glt ‘The woman is not supposed to have children/does not stay with the children.’ 	
\end{xlist}
\z

Existential clauses in Ru\-ruu\-li\hyp{}Lu\-nya\-la may be negated by means of the standard negator \textit{ti-}. This strategy  seems to be used only in non-present tense, as in~(\ref{ex-tiwaliwo}). 
It is infrequently found in the corpus and was never produced by any of the speakers in elicitation sessions. 
The more common strategy includes the dedicated negative existential \textit{ndoo} and the locative enclitics that are also used in the affirmative. 
This construction  is used for both present and non-present, as in~(\ref{ex-ndoopres})--(\ref{ex-ndoononpres}). 
The origin of the negative existential is unclear and a similar form is not available in closely related languages. 
However, the usage of a special negation construction in at least one type of non-verbal predication is common cross-linguistically \citep{Miestamo2017Negation}. 

\ea \label{ex-ruuliexistentialneg}
\begin{xlist}
	\ex \label{ex-tiwaliwo}
	\glll  Eirai budi tiwaaliwo sente.\\
	e-irai budi 			ti-wa-a-li=wo sente\\
		\textsc{aug}-long\_ago last\_time \textsc{neg}-1\textsc{6sbj}-\textsc{pst}-\textsc{cop}=16.\textsc{loc} 9.money\\
	\glt ‘Long ago there was no money.’ 	
	
\ex \label{ex-ndoopres}
	\glll Naye ndoowo mugaso.\\
		  naye ndoo=wo mugaso\\ 
		but \textsc{neg}.EX=16.\textsc{loc} 3.importance\\
	\glt ‘But there is no importance.’
	
\ex \label{ex-ndoononpres}
	\glll Ndoowo Munyala.\\
	  ndoo=wo Munyala\\
		\textsc{neg}.EX=16.\textsc{loc} 1.Munyala\\
	\glt ‘There was no Munyala.’
\end{xlist}
\z


\section{Noun phrase}\label{sec-syntax-NP}

A Ru\-ruu\-li\hyp{}Lu\-nya\-la NP consists of a head and optional dependents, which always follow it.
The head is either a noun or a personal pronoun. 
Pronouns do not normally take dependents. 
The classes of dependents discussed in this chapter include determiners, pronominal and nominal possessors (Sections \ref{sec-NP-possessive-pronoun} and~\ref{sec-NP-genitive}), 
attributive adjectives (Section~\ref{sec-NP-adjective}), numerals (Section~\ref{sec-NP-numerals}), as well as relative clauses (Section~\ref{sec-NP-relative}).

\subsection{Pronominal possessors}\label{sec-NP-possessive-pronoun}

The form of possessive pronouns is discussed in Section~\ref{sec-morph-posspro}. 
Possessive pronouns follow the head noun and do not have augments, as in~(\ref{ex-NP-possessive-pronoun}). 

\ea \label{ex-NP-possessive-pronoun}
\begin{xlist}	
\ex
    \glll    Omukoli wange mugarambi.\\
     o-mukoli wa-a-nge mu-garambi\\
	\textsc{aug}-1.worker 1-\textsc{assoc}-\textsc{1sg} 1-lazy\\
    \glt  `My worker is lazy.’

\ex     \glll   Amembe ga nte wange gasongoli.\\
	  a-membe ga nte wa-a-nge ga-songoli\\
	\textsc{aug}-6.horn 6.\textsc{gen} 1.cow 1-\textsc{assoc}-\textsc{1sg} 6-straight\\
    \glt  `The horns of my cow are straight.’
\end{xlist}	
\z

It is possible to use possessive pronouns without a head noun. In this case the augment attaches to the possessive pronoun, as in~(\ref{ex-NP-possessive-pronoun-headless}).

\ea \label{ex-NP-possessive-pronoun-headless}
    \glll   Batume abaabwe, owange tiyaba kola ebyo.\\
      ba-tum-e a-ba-a-bwe o-wa-a-nge ti-ab-a kol-a eby-o\\
	\textsc{2sbj}-send-\textsc{sbjv} \textsc{aug}-2-\textsc{assoc}-2 \textsc{aug}-1-\textsc{assoc}-\textsc{1sg} \textsc{neg}-\textsc{aux}-\textsc{fv} do-\textsc{fv} 8-\textsc{med}\\
    \glt  `Let them send theirs (=their children, noun class 2) because mine (=my child, noun class 1) will not do that.’
\z

	
\subsection{Nominal possessors}\label{sec-NP-genitive}

Apart from the constructions with possessive pronouns discussed in Section~\ref{sec-morph-posspro}, Ru\-ruu\-li\hyp{}Lu\-nya\-la also has a genitive construction with a nominal head and a nominal dependent. 
As we will discuss below, the head can be semantically a possessee and the dependent can be a possessor, but other semantic relations can hold between the two items.
In the Ru\-ruu\-li\hyp{}Lu\-nya\-la genitive construction, the dependent is marked by a genitive marker that consists of a pronominal prefix, which shows agreement with the head noun, and the element \emph{a}. The terms used for the marker \emph{a} include \emph{connective}, as well as \emph{connexive} or \emph{associative marker} (see \citealt{Vandeveldeetal2019Nominal}). 
In the examples in this grammar sketch, we do not segment the associative marker and gloss the combination of the agreement prefix and the associative marker as \textsc{gen}, as e.g.\,in~(\ref{ex-poss-NC}). (The associative marker is segmented and glossed with \textsc{assoc} in cases of pronominal possessors, e.g.\,in Section~\ref{sec-morph-posspro}.)
 
The agreeing forms of the genitive marker are given in Table~\ref{tab-genitive}. 
The table does not contain the locative noun classes, as the form of the genitive marker depends on the noun class of the respective head noun in a locative phrase and not on the locative marker (see Section~\ref{sec-morh-locative}). 
For instance, in the locative phrase \emph{omu kooti za Uganda} `in Uganda's courts of law' the genitive \emph{za} carries the agreement prefix of the word \emph{kooti} `court of law' of noun class 9 and not of class 18 for \emph{(o)mu}.

\begin{table}
\caption{Genitive markers}
 
\begin{tabular}{l l l}
\hline
 
Class	& Genitive & Examples\\
\hline
1 & \emph{wa} & (\ref{ex-poss-NC1a}), (\ref{ex-poss-NC1b})\\
2 & \emph{ba} & (\ref{ex-poss-NC2b})\\
3 & \emph{gwa} & (\ref{ex-poss-NC3a})\\
4 & \emph{gya}/\emph{ya} (dial.\,\emph{za}) &\\
5 & \emph{lya} (dial.\,\emph{rya})& \\
6 & \emph{ga} & (\ref{ex-poss-NC6b})\\
7 & \emph{kya} &\\
8 & \emph{bya} &\\
9 & \emph{ya} &\\
10 & \emph{za} &\\
11 & \emph{lwa} &\\
12 & \emph{ka} &\\
13 & \emph{twa}  &\\
14 & \emph{bwa} & (\ref{ex-poss-NC14})\\
15 & \emph{kwa} & (\ref{ex-poss-NC15})\\
20 & \emph{gwa} &\\
22 & \emph{ga} &\\
\hline
\end{tabular}
\label{tab-genitive}
\end{table}

\largerpage
The head of the genitive construction – if there is one – precedes the dependent, as the examples in~(\ref{ex-poss-NC}) illustrate for some of the noun classes. 
The dependent does not have an augment, whereas the presence of the augment on the head noun is conditioned by the wider syntactic environment. 
In headless noun phrases, the genitive marker can be preceded by the augment.

As in other Bantu languages, the genitive construction is not dedicated to the expression of a specific semantic relation between the head and the dependent. 
Apart from the semantic relation of possession in a broad sense, including ownership, as in (\ref{ex-poss-NC5b}), kinship relations, as in (\ref{ex-poss-NC1b}), part-whole relations, as in (\ref{ex-poss-NC3b}) and  (\ref{ex-poss-NC6b}), the construction is used for various other abstract relations. 
Thus, the dependent noun can have argument relation with the head nouns, as in (\ref{ex-poss-NC15}), it can qualify the head noun, as in (\ref{ex-poss-NC1a}), classify it, as (\ref{ex-poss-NC3a}) and (\ref{ex-poss-NC9a}), locate it in time and space, as in (\ref{ex-poss-NC2b}), and enter a number of other semantic relations, as the examples below illustrate.

\ea \label{ex-poss-NC}
\begin{xlist}	
\ex \label{ex-poss-NC1a}
    \glll 	omusaiza	wa	maani\\
    		o-musaiza	wa	maani\\
	    	\textsc{aug}-1.man	1.\textsc{gen}	6.strength\\
    \glt	‘a strong man' \\(lit.\,`a man of strength')

    \ex \label{ex-poss-NC1b}
    \glll 	omwana	wa	Bob\\
    		 o-mwana	wa	Bob\\
   		\textsc{aug}-1.child	1.\textsc{gen}	1.Bob\\
    \glt 	‘Bob’s child’
    
   \ex \label{ex-poss-NC2b}
  \glll omwana wa leero\\
  	o-mwana wa leero\\
   \textsc{aug}-1.child	1.\textsc{gen}		today\\
    \glt ‘today’s child’
    
    \ex \label{ex-poss-NC3a}
\glll		omukubi		gwa		maido\\
		o-mukubi	 	gwa		maido\\
   		\textsc{aug}-3.sauce 	3.\textsc{gen}	6.groundnut\\
    \glt	‘the groundnut sauce’

\ex \label{ex-poss-NC3b}
    \glll	omukonda   		gwa		kikopo\\
    		o-mukonda   		gwa		kikopo\\
		\textsc{aug}-3.handle       	3.\textsc{gen}	7.cup\\
    \glt	‘the handle of the cup’

\ex \label{ex-poss-NC5b}
    \glll 	eryato		lya		Musoke\\
    		e-ryato		lya		Musoke\\
  		 \textsc{aug}-5.boat	5.\textsc{gen}	1.Musoke\\
    \glt 	‘Musoke's boat’
    
    
\ex \label{ex-poss-NC6b}
    \glll	amatwi		ga		mbwene\\
    		a-matwi		ga		mbwene\\
		\textsc{aug}-6.ear		6.\textsc{gen}	1.dog\\
    \glt ‘the dog's ears’

\ex \label{ex-poss-NC9a}
    \glll ekooti		ya		byoya\\
    e-kooti		ya		byoya\\
  \textsc{aug}-9.coat	9.\textsc{gen}		8.wool\\
    \glt  `a wool coat'

\ex \label{ex-poss-NC14}
    \glll obusanso bwa musaale\\
    o-busanso bwa musaale\\
 \textsc{aug}-14.twig 14.\textsc{gen} 3.tree\\
    \glt  ‘tree twigs'
    
\ex \label{ex-poss-NC15}
    \glll okufa kwa Walusimbi\\
    o-kufa kwa Walusimbi\\
	\textsc{aug}-15.death 15.\textsc{gen} 1.Walusimbi\\
    \glt  ‘Walusimbi's death’
\end{xlist}
\z

In addition to the more common cases illustrated above, the genitive construction is also found with dependent nouns in the locative noun classes, as in~(\ref{ex-poss-loc}).

\ea \label{ex-poss-loc}
    \begin{xlist}	
    \ex \label{ex-poss-loca}
    \glll abaana ba mu kyalo\\
    a-baana		ba	mu		kyalo\\
 	\textsc{aug}-2.child		2.\textsc{gen}	18.\textsc{loc}	7.village\\
    \glt  ‘children in the countryside’
        
    \ex \label{ex-poss-locb}
    \glll	 abantu	ba	mu		Kampala\\
    a-bantu	ba	mu		Kampala\\
	 \textsc{aug}-2.man	2.\textsc{gen}	18.\textsc{loc}	1.Kampala\\
    \glt ‘people in Kampala’
     \end{xlist}	
\z


\subsection{Attributive numerals}\label{sec-NP-numerals}

Attributive numerals follow the head noun.
The numerals 1 to 5 (\emph{mwe(i)}, `one', \emph{biri} `two', \emph{satu} `three', \emph{inai} `four', and \emph{taanu} `five') agree with the head noun, as in~(\ref{ex-NP-num}), whereas higher numerals do not agree with the head noun, as in~(\ref{ex-NP-num-high}).

\ea \label{ex-NP-num}
\begin{xlist}
\ex 	\label{ex-NP-num1}
	\glll	Bulimu ati ebika bisatu.\\
		bu-li=mu ati e-bika bi-satu\\
		8-\textsc{cop}=18.\textsc{loc} now \textsc{aug}-8.type 8-three\\
	\glt ‘They are now of three types.’

\ex 	\label{ex-NP-num2}
	\glll  Cwa ekisaale ekyo ebicweka bibiri.\\
	cw-a e-kisaale eky-o e-bicweka bi-biri\\
		break-\textsc{fv} \textsc{aug}-7.stick 7-\textsc{med} \textsc{aug}-8.piece 8-two\\
\glt ‘Break that stick into two pieces.’
\end{xlist}
\z

\ea \label{ex-NP-num-high}
\begin{xlist}
	
\ex 	\label{ex-NP-num7}
	\glll	Yandekeire abaana musanju.\\
	a-a-n-lek-ir-ire				 a-baana 	musanju\\
	\textsc{2sbj}-\textsc{pst}-\textsc{1sg.obj}-leave-\textsc{appl}-\textsc{fv} 	\textsc{aug}-2.child 	seven\\
	\glt ‘He left seven children to me.’

\ex 	\label{ex-NP-num8}
	\glll  Tulimu ebigurupu munaanai.\\
	tu-li=mu e-bigurupu munaanai\\
 	1\textsc{pl.sbj}-\textsc{cop}=18.\textsc{loc} \textsc{aug}-8.group eight\\
\glt ‘We are in eight groups.’
\end{xlist}
\z


\subsection{Attributive adjectives} \label{sec-NP-adjective}

The morphological properties of adjectives are discussed in Section~\ref{sec-adjectives}. 
Below in~(\ref{ex-NP-adj}) are a few examples of attributive adjectives used in a sentence. 
Attributive adjectives regularly follow the head noun, as in~(\ref{ex-NP-adj}). 
They are mostly used with an augment when the head noun has an augment, as in~(\ref{ex-NP-adj1}). 
They are used without an augment when the head noun does not have one, as in~(\ref{ex-NP-adj2}).

\ea \label{ex-NP-adj}
\begin{xlist}
	
\ex 	\label{ex-NP-adj1}
	\glll	Omuyembe ogubbisi ti gusai kulya.\\
	o-muyembe o-gu-bbisi ti gu-sai ku-li-a\\
	\textsc{aug}-3.mango \textsc{aug}-3-unripe \textsc{neg}.\textsc{cop} 3-good \textsc{inf}-eat-\textsc{fv}\\
	\glt ‘An unripe mango is not good for eating.’

\ex 	\label{ex-NP-adj2}
	\glll  Timwalyanga		binage		bibbisi.\\
	ti-mu-a-li-a-nga		binage		bi-bbisi\\
		\textsc{neg}-2\textsc{sg.sbj}-\textsc{pst}-eat-\textsc{fv}-\textsc{hab}	8.tilapia	8-raw\\
\glt ‘You used to not eat raw fish.’

\end{xlist}
\z

Many concepts translated into English with adjectives are not adjectives in Ru\-ruu\-li\hyp{}Lu\-nya\-la. 
Instead, various alternative strategies of nominal qualification are used. 
Genitive constructions (see Section~\ref{sec-NP-genitive}) are a common strategy, as illustrated in~(\ref{ex-NP-adjB}).

\ea \label{ex-NP-adjB}
\begin{xlist}
\ex \label{ex-NP-adjB1}
	\glll  Tuli 		bantu 	ba		idembe.\\
	tu-li 		bantu 	ba		idembe\\
		1\textsc{pl.sbj}-\textsc{cop}	2.person	2.\textsc{gen}		5.peace\\
	\glt `We are peaceful people.’ (lit. ‘We are people of peace.’)
		
\ex \label{ex-NP-adjB2}
	\glll	Yaali		musaiza	wa	maani.\\
	a-a-li		musaiza	wa	maani\\
	\textsc{1sbj}-\textsc{pst}-\textsc{cop}	1.man		1.\textsc{gen}	6.power\\
	\glt ‘He was a powerful man.’ (lit. ‘He was a man of power.’)

\end{xlist}
\z


\subsection{Relative clauses} \label{sec-NP-relative}

There are two main relativisation strategies in Ru\-ruu\-li\hyp{}Lu\-nya\-la.  
The first construction makes use of the relative prefixes, which have the same form as the relativised noun's augment, viz.\,\textit{a-, e-} or \textit{o-} glossed as \textsc{sbj.rel}, as in~(\ref{ex-NP-relative-sbj}). 
This strategy is used for subject relativisation only. 

\ea \label{ex-NP-relative-sbj}
\begin{xlist}
\ex
\label{ex-NP-relative-sbja}
	\glll Neekunwire { } { } { } { } okusanga { }  { } omusaiza { } eyatakanga { }  okunswera.\\
	 n-eekunwire { } { } { } { }  o-ku-sang-a { }  { }  o-musaiza { } e-a-a-tak-a-nga  { }  o-ku-n-swer-a\\
		\textsc{1sg.sbj}-be\_shy:\textsc{pfv} { } { } { } { }  \textsc{aug}-\textsc{inf}-meet-\textsc{fv}  { }  { }  \textsc{aug}-1.man { } \textsc{sbj.rel}-\textsc{1sbj}-\textsc{pst}-want-\textsc{fv}-\textsc{hab}  { } \textsc{aug}-\textsc{inf}-\textsc{1sg.obj}-marry-\textsc{fv}\\
    \glt ‘I was shy to meet the man who wanted to marry me.’
	
\ex 	\label{ex-NP-relative-sbjb}
	\glll 	Tulina ebintu ebitusisira ebintu.\\
		tu-lin-a 		e-bintu 		e-bi-tu-siis-ir-a e-bintu\\
		1\textsc{pl.sbj}-have-\textsc{fv} 	\textsc{aug}-8.thing 	\textsc{sbj.rel}-\textsc{8sbj}-\textsc{1pl.obj}-spoil-\textsc{appl}-\textsc{fv} \textsc{aug}-8.thing\\
    \glt `We have things (i.e.\,birds) which spoil for us things (i.e.\,crops).'
	
\end{xlist}
\z

The relativisation of non-subjects is achieved through the use of the relativisers given in Table~\ref{tab-relativiser} and is illustrated in~(\ref{ex-NP-relative-nonsubject}).
For the relativisation of a locative adjunct the noun class 16 relativiser \emph{we} is used, as in~(\ref{ex-NP-relative-nonsubject3}).

\begin{table}[!h]

\caption{Non-subject relativisers}
\label{tab-relativiser}
	\begin{tabular}{lll}
\lsptoprule
Class & Relativiser\\
\midrule
1 & \emph{gwe}\\
2 & \emph{be}\\
3 & \emph{gwe}\\
4 & \emph{gye}\\
5 & \emph{lye}\\
6 & \emph{ge}\\
7 & \emph{kye}\\
8 & \emph{bye}\\
9 & \emph{gye}\\
10 & \emph{ze}\\
11 & \emph{lwe}\\
12 & \emph{ke}\\
13 & \emph{twe}\\
14 & \emph{bwe}\\
16 & \emph{we}\\
20 & \emph{gwe}\\
22 & \emph{ge}\\
23 & \emph{gye} &\\
\lspbottomrule
	\end{tabular}
\end{table}

\ea \label{ex-NP-relative-nonsubject}
\begin{xlist}
\ex 
	\glll  Abamagye bakwata obwato bwe basanga oku nyanja.\\
	a-bamagye ba-kwat-a o-bwato bwe ba-sang-a o-ku nyanja\\
		\textsc{aug}-2.soldier	\textsc{2sbj}-confiscate-\textsc{fv}	\textsc{aug}-14.boat 14.\textsc{rel} \textsc{2sbj}-find-\textsc{fv}  \textsc{aug}-17.\textsc{loc} 9.lake\\
	\glt ‘The soldiers confiscate boats they find on the lake.'

\ex 	\label{ex-NP-relative-nonsubject2}
	\glll Tulina onyonyi gwe bayeta okisyo.\\
		tu-lin-a o-nyonyi [gwe ba-et-a o-kisyo]\\
		1\textsc{pl.sbj}-have-\textsc{fv} \textsc{aug}-1.bird 1.\textsc{rel} \textsc{2sbj}-call-\textsc{fv} \textsc{aug}-1.weaver.bird\\
	\glt `We have a bird they call weaver bird.'

\ex 	\label{ex-NP-relative-nonsubject3}
	\glll Tukwiza okufuna we twabba.\\
		tu-ku-iz-a o-ku-fun-a we tu-a-bb-a\\
		1\textsc{pl.sbj}-\textsc{prog}-\textsc{aux}-\textsc{fv} \textsc{aug}-\textsc{inf}-get-\textsc{fv} 16.\textsc{rel} 1\textsc{pl.sbj}-\textsc{fut}-stay-\textsc{fv}\\
	\glt `We will get (a place) where we will stay.'
\end{xlist}
\z

Headless relative clauses are also possible and very common with the relativiser of noun class 7 and 8, as in~(\ref{ex-NP-relative-headless}).

\ea \label{ex-NP-relative-headless}
\begin{xlist}
\ex	\label{ex-ruulirelativesbj}
	\glll Okubona kye nkukoba?\\
	 o-ku-bon-a [kye n-ku-kob-a]\\
		2\textsc{sg.sbj}-\textsc{prog}-see-\textsc{fv} 7.\textsc{rel} \textsc{1sg.sbj}-\textsc{prog}-say-\textsc{fv}\\
	\glt ‘Do you understand what I am saying?'
	
\ex	
	\glll  Yaba omukayule tubone kye yakola.\\
		ab-a o-mu-kayul-e tu-bon-e [kye a-a-kol-a]\\
		go-\textsc{fv} 2\textsc{sg.sbj}-\textsc{1obj}-annoy-\textsc{sbjv} 1\textsc{pl.sbj}-see-\textsc{sbjv} 7.\textsc{rel} \textsc{1sbj}-\textsc{fut}-do-\textsc{fv}\\
	\glt ‘Go and annoy her, let's see what she will do.'
\end{xlist}
\z

With both relativisation strategies, the relative clause is negated by means of the post-initital negative prefix \textit{ta-}, as in~(\ref{ex-ruulirelativesbjneg})--(\ref{ex-ruulirelativeobjneg}).

\ea \label{ex-ruulirelative-neg}
\begin{xlist}

\ex\label{ex-ruulirelativesbjneg}
\glll Naye abataabbaanga na kifo nibayaba.\\
	naye [a-ba-ta-a-bba-a-nga na kifo] ni-ba-ab-a\\
		but \textsc{rel}-\textsc{2sbj}-\textsc{neg}-\textsc{pst}-\textsc{cop}-\textsc{fv}-\textsc{hab} \textsc{com} 7.place \textsc{nar}-\textsc{2sbj}-go-\textsc{fv}\\
	\glt ‘But those who didn't have a place (to sleep) went.’

\ex \label{ex-ruulirelativeobjneg}
	\glll Amaani ga kanca gwe otasobola  kubonaku…\\
	a-maani ga kanca [gwe o-ta-sobol-a  ku-bon-a=ku]\\ 
		\textsc{aug}-6.power 6.\textsc{gen} 1.God 1.\textsc{rel} 2\textsc{sg.sbj}-\textsc{neg}-can-\textsc{fv} \textsc{inf}-see-\textsc{fv}=17.\textsc{loc}\\
	\glt `The power of God whom you cannot see…'

\end{xlist}
\z


\section{Extensions, diathetical operations, and grammatical voice}\label{sec-syntax-extensions}

In Bantu linguistics the term \emph{extension} refers to certain suffixes which follow the root of the verb. 
Bantu extensions include applicative, causative, passive and reciprocal, as well as a range of others, such as impositive, neuter or tentive.
The inventory of extensions, as well as their functions and properties vary from language to language. 
In the typologically-oriented grammatical descriptions some of the extensions can be termed \emph{diathetical operations}, that is, strategies which modify the basic diathesis of a predicate (see e.g.\,\citet[4]{Zunigaetal2019Grammatical}. 
Another strategy, namely the reflexive prefix is not part of the Bantu system of extensions, however, it is a diathetical operation according to this definition. 
The term \emph{(grammatical) voice} refers to the grammatical category whose values correspond to particular diathesis marked on the form of predicates. 
Diathetical operations involve two types of processes. 
On the one hand, there are modifications of valency in the sense of the number of arguments in the semantic argument structure, these processes are referred to as \emph{argument installment}, when new arguments are added  to the semantic structure of the clause as syntactic core argument, or \emph{removal}, when arguments are removed to the semantic structure of the clause. 
On the other hand, there are changes in the morpho-syntactic properties of the arguments or syntactic transitivity, these processes are argument \emph{promotion} and \emph{demotion} (including \emph{suppression}) \citep[4]{Zunigaetal2019Grammatical}. Promoted arguments acquire morpho-syntactic properties, such as the ability to be indexed on the verb or participate in certain syntactic operations, whereas demoted arguments loose certain morpho-syntactic properties. 

All Bantu extensions are occasionally indiscriminately treated as de\-ri\-vational by some Bantu scholars (e.g.\,\citealt{Schadebergetal2019Bantu}). 
However, the productivity of individual extensions as well as the transparency of the meaning of the resulting form varies, so that some extensions, such as passive, are on the inflectional side of the inflection-derivation spectrum, whereas others are purely derivational. 

Most of the Ru\-ruu\-li\hyp{}Lu\-nya\-la extensions discussed in this section can be categorised as various categories of grammatical voice according to the definition above. 
As the sections below show, one or more extensions may be suffixed to the root. 
We first discuss the grammatical voices and the respective extensions which increase the semantic valency of the respective verb, these are the applicative (Section~\ref{sec-applicative}) and the causative (Section~\ref{sec-causative}). 
We will next consider the grammatical voices which only modify the transitivity of the verb and thus the morpho-syntactic structure of a clause, but do not alter the argument structure of the verb (i.e.\,the number and semantic argument roles of the involved participant remains the same). 
These are the passive voice discussed in Section~\ref{sec-passive}, as well as the reflexive (Section~\ref{sec-reflexive}) and reciprocal (Section~\ref{sec-reciprocal}) voices. 
In this section we also discuss the impersonal passive, as its functions are similar to the regular passive, though it is not an extension and not a grammatical voice technically (Section~\ref{sec-impersonal}).


\subsection{Applicative}\label{sec-applicative}

The distribution of the allomorphs of the  Ru\-ruu\-li\hyp{}Lu\-nya\-la applicative suffix \emph{-ir} is discussed in  Section~\ref{sec-extension-applicative}. 
This section presents some of the major functions of the applicative construction. 
As in some other Bantu languages, the causative extension in Ru\-ruu\-li\hyp{}Lu\-nya\-la has acquired an instrumental applicative function, it is discussed in Section~\ref{sec-causative}. 

The major characteristic of an applicative construction cross-lingui\-sti\-cally is to allow the coding of an  adjunct as an object (\citealt[1]{Peterson2006Applicative}). 
One of the major usages of the applicative construction in Ruruuli-Lu\-nya\-la is to add a beneficiary to a clause, which is then coded as an object, i.e.\,it is not overtly flagged, as in~(\ref{ex-applictative-benef-nominal}), and is indexed on the verb under the same conditions under which regular objects are indexed on the verb, as in~(\ref{ex-applictative-benef-pronominal}) (see Section~\ref{sec-verb-indexing}).\footnote{It remains a topic for future research to consider the syntactic properties of applied object and compare them to the properties of regular objects.}

\ea \label{ex-applictative-benef}
\begin{xlist}
	\ex \label{ex-applictative-benef-nominal}
	\glll  Asumbiire omwana ebiyaata.\\
		a-sumb-ir-ire o-mwana e-biyaata\\
		\textsc{1sbj}-cook-\textsc{appl}-\textsc{pfv} \textsc{aug}-1.child \textsc{aug}-8.potato\\
	\glt ‘She has cooked potatoes for the child.’	

	\ex \label{ex-applictative-benef-pronominal}
	\glll  Nkoorayo omusokoto gwa taaba.\\
		n-kol-ir-a=yo o-musokoto gwa taaba\\
		\textsc{1sg.obj}-make-\textsc{appl}-\textsc{fv}=23.\textsc{loc} \textsc{aug}-3.hand-rolled\_cigarette 3.\textsc{gen} 1.tobacco\\
	\glt ‘Make for me a hand-rolled cigarette.’	
\end{xlist}
\z

The second most common context for the use of the applicative is somewhat unusual: It is commonly used when there is an adjunct marked by one of the locative classes (see Section~\ref{sec-morh-locative}). 
The applicative seems to be particularly common with the phrases marked by the noun class 18 \emph{(o)mu}.  
In this case, the locative marking is preserved and no promotion to object takes place. 
The respective locative-marked noun phrase most frequently encodes a location, as in (\ref{ex-applictative-locaug}), or a manner of action, as in~(\ref{ex-applictative-manner}). 

\ea \label{ex-applictative-locaug}
\begin{xlist}
	\ex \label{ex-applictative-locaug-a}
	\glll  E Nalufeenya { }  baakoleireyo ebikolobero bingi.\\
	e Nalufeenya { }  ba-a-kol-ir-ire=yo e-bikolobero bi-ingi\\
		23.\textsc{loc}	9.Nalufeenya  { }  \textsc{2sbj}-\textsc{pst}-do-\textsc{appl}-\textsc{pfv}=23.\textsc{loc} \textsc{aug}-8.crime  8-numerous\\
	\glt ‘They committed a lot of crimes in Nalufeenya.'

	\ex \label{ex-applictative-a}
	\glll Omubumbi { } { }   abumbira { } { } { }   mu { } ibumbiro.\\
	  o-mubumbi  { } { }   a-bumb-ir-a { }  { } { }   mu { }  ibumbiro\\
		\textsc{aug}-1.potter  { }  { }  \textsc{aug}-do\_pottery-\textsc{appl}-\textsc{fv} { }  { }  { } 18.\textsc{loc}  { }  9.pottery\_workshop\\
	\glt ‘The potter works in a pottery workshop.'
\end{xlist}
\z

\ea \label{ex-applictative-manner}
\begin{xlist}
	\ex \label{ex-applictative-manner1}
	\glll  Enkangu zitambuura mu nyiriri.\\
	e-nkangu zi-tambul-ir-a mu nyiriri\\
		\textsc{aug}-10.stink\_ant \textsc{10s}-walk-\textsc{appl}-\textsc{fv} 18.\textsc{loc} 10.line\\
	\glt ‘Stink ants walk in lines.’	

	\ex \label{ex-applictative-manner2}
	\glll Obote yabaziranga mu ngero.\\
	  Obote		a-a-baz-ir-a-nga mu ngero\\
		1.Obote \textsc{1sbj}-\textsc{pst}-speak-\textsc{appl}-\textsc{fv}-\textsc{hab} 18.\textsc{loc} 10.parable\\
	\glt ‘Obote used to speak in parables.’	
\end{xlist}
\z

\noindent Furthermore, the presence of certain manner adverbs in a clause strongly correlates  with the use of the applicative.
They include \emph{bwe\-reere} `for no reason' and \emph{kyarumwe} `for good, forever, definitively', as in~(\ref{ex-applictative-adv}) (see also \citealt{Atuhairwe2019Applicative}).

\ea \label{ex-applictative-adv}
\begin{xlist}
	\ex \label{ex-applictative-adv1}
	\glll 	Yayabiire kyarumwe.\\
		  a-a-ab-ir-ire kyarumwe\\
		\textsc{1sbj}-\textsc{pst}-go-\textsc{appl}-\textsc{pfv}-\textsc{fv} for\_good\\
	\glt ‘He left for good.’	

	\ex \label{ex-applictative-adv2}
	\glll Omutaani waamwe yamukubbiire bwereere.\\
	  o-mutaani wa-a-mwe a-a-mu-kubb-ir-ire bwereere\\
		\textsc{aug}-1.som 1-\textsc{assoc}-1 \textsc{1sbj}-\textsc{pst}-\textsc{1obj}-beat-\textsc{appl}-\textsc{pfv} for\_no\_reason\\
	\glt ‘He has beaten his son for no reason.’	
\end{xlist}
\z


\subsection{Causative}\label{sec-causative}

Ruruuli-Lunyala has two causative suffixes viz.\,\emph{-isy} (with the allomorphs [isj] and [esj]) and \emph{-y}. 
The distribution of the allomorphs is discussed in Section~\ref{sec-extension-causative}. 
Causatives increase the semantic valency of predicates by introducing a new agent (a causer) into the argument structure (see e.g.\,\citealt[15]{Zunigaetal2019Grammatical} for a recent overview). 
Causatives also increase the transitivity of predicates by one and the added argument functions as the syntactic subject. 

Compare the three examples in~(\ref{ex-caus-bina}). Whereas (\ref{ex-caus-binaA}) contains the underived form of the intransitive verb \emph{bina} `dance', (\ref{ex-caus-binaB}) and~(\ref{ex-caus-binaC}) have transitive verbs with a new agent (the person who causes another person to dance), which functions as the subject of the predicate. 
The agent of the corresponding non-causative form functions as the object of the causative and can be expressed with an object prefix on the verb, as in~(\ref{ex-caus-binaC}). 
Further examples are given in~(\ref{ex-caus-band})–(\ref{ex-caus-gait}).

\ea \label{ex-caus-bina}
\begin{xlist}

\ex \label{ex-caus-binaA}
	\glll Omaama akubina.\\
	  o-maama a-ku-bin-a\\
		\textsc{aug}-1.mother \textsc{1sbj}-\textsc{prog}-dance-\textsc{fv}\\
	\glt ‘Mother is dancing.'

\ex \label{ex-caus-binaB}
	\glll Obinisya omaama.\\
	  o-bin-isy-a o-maama\\
		2\textsc{sg.sbj}-dance-\textsc{caus}-\textsc{fv} \textsc{aug}-1.mother\\
	\glt ‘You make (my) mother dance.'

\ex \label{ex-caus-binaC}
	\glll Omubinisya.\\
	  o-mu-bin-isy-a\\
		2\textsc{sg.sbj}-\textsc{1obj}-dance-\textsc{caus}-\textsc{fv}\\
	\glt ‘You make her dance.'
\end{xlist}
\z

\ea \label{ex-caus-band}
\begin{xlist}

	\ex \label{ex-caus-bazaA}
	\glll Baabire kubandwa mbandwa Katigo.\\
	  ba-ab-ire ku-bandw-a mbandwa Katigo\\
		\textsc{2sbj}-go-\textsc{pfv}	\textsc{inf}-worship-\textsc{fv} 1.spirit 1.Katigo\\
	\glt ‘They have gone to worship the spirit called Katigo.' 
	
	\ex \label{ex-caus-bazaB}
	\glll Nibataka  { }   { }  okutubandwisya { }  { }  oKawumpuli.\\
	  ni-ba-tak-a  { }   { }  o-ku-tu-bandw-isy-a  { }  { }  o-Kawumpuli\\
		\textsc{nar}-\textsc{2sbj}-want-\textsc{fv}  { }   { }  \textsc{aug}-\textsc{inf}-\textsc{1pl.obj}-worship-\textsc{caus}-\textsc{fv}  { } { }  \textsc{aug}-1.Kawumpuli\\
	\glt ‘They wanted to make us worship Kawumpuli.’
	

\end{xlist}
\z
	
\ea \label{ex-caus-gait}
\begin{xlist}
	\ex \label{ex-caus-gaitA}
	\glll	Omukali wange mmugaitire olugoye luni.\\
		o-mukali wa-a-nge n-mu-gait-ire o-lugoye lu-ni\\
		\textsc{aug}-1.wife 1-\textsc{assoc}-\textsc{1sg} \textsc{1sg.sbj}-\textsc{1obj}-pay\_fine-\textsc{pfv} \textsc{aug}-11.garment 11-\textsc{prox}\\
	\glt	‘I paid this garment as a fine to my wife.'
	
	\ex \label{ex-caus-gaitB}
	\glll	Tumugaitisya ontaama.\\
		tu-mu-gait-isy-a o-ntaama\\
		1\textsc{pl.sbj}-\textsc{1obj}-pay-\textsc{caus}-\textsc{fv} \textsc{aug}-1.sheep\\
	\glt	‘We make him pay a sheep (as a fine).' 
\end{xlist}
\z

In addition to productive causatives discussed above, dozens of lexicalised causatives are listed in the dictionary.  Most of them are built with the causative suffix \emph{-y}. Consider the pair of examples in~(\ref{ex-caus-lexicalised}).

\ea \label{ex-caus-lexicalised}
\begin{xlist}
	\ex \label{ex-caus-lexicaliseda}
	\glll	Airukire.\\
		a-iruk-ire\\
	 	\textsc{1sbj}-run-\textsc{pfv}\\
	\glt	‘He ran.’ 

	\ex \label{ex-caus-lexicalisedB}
	\glll	Batwirukya.\\
		ba-tu-iruky-a\\
		\textsc{2sbj}-\textsc{1pl.obj}-chase-\textsc{fv}\\
	\glt	‘They chase us away.’ (lit.\,`They make us run.')
\end{xlist}
\z

As in some other Bantu languages (see e.g.\,\citealt[175–176]{Schadebergetal2019Bantu}),\footnote{See also \citet[64–65]{Peterson2006Applicative} on the causative-applicative isomorphism more generally.} the causative suffix in Ru\-ruu\-li\hyp{}Lu\-nya\-la has acquired another function and is used to introduce an instrument (in a broad sense) into a clause akin to what applicative does in other languages. 
This instrument does not become a subject of the predicate and does not replace the original agent in the subject position, instead it is added as an object, consider (\ref{ex-caus-instr1}). 

\ea \label{ex-caus-instr1}
\begin{xlist}
	\ex \label{ex-caus-instr1A}
	\glll	Yomboka enyumba ya nkoko.\\
		ombok-a e-nyumba ya nkoko.\\
		build-\textsc{fv} \textsc{aug}-9.house 9.\textsc{gen} 10.chicken\\
	\glt ‘Build a chicken shed.’ 

	\ex \label{ex-caus-instr1B}
	\glll	Omutoma ogwo bagwombokesyanga amayumba.\\
		o-mutoma ogw-o ba-gu-ombok-esy-a-nga a-mayumba\\
		\textsc{aug}-3.Ficus\_natalensis 3-\textsc{med} \textsc{2sbj}-\textsc{3obj}-build-\textsc{caus}-\textsc{fv}-\textsc{hab} \textsc{aug}-6.house\\
	\glt ‘That bark cloth tree (Lat.\,\emph{Ficus natalensis}), it is used it to build houses.'
\end{xlist}
\z

\noindent The nature of the introduced instrument varies and includes tools (\ref{ex-caus-instr2A}), body parts (\ref{ex-caus-instr1C}), materials (\ref{ex-caus-instr1B}), as well as people who assist in an activity (\ref{ex-caus-instr2A}).

\ea \label{ex-caus-instr2}
\begin{xlist}
	\ex \label{ex-caus-instr2B}
	\glll Nadala abantu ababbanga bakwetaaga emiyo gini okukekesya ebiteere…\\
	  nadala a-bantu a-ba-bb-a-nga ba-ku-eetaag-a e-miyo gi-ni o-ku-kek-esy-a e-biteere\\
		especially \textsc{aug}-2.people \textsc{rel}-\textsc{2sbj}-\textsc{aux}-\textsc{fv}-\textsc{hab} \textsc{2sbj}-\textsc{prog}-need-\textsc{fv} \textsc{aug}-4.knife 4-\textsc{prox} \textsc{aug}-\textsc{inf}-slice-\textsc{caus}-\textsc{fv} \textsc{aug}-8.sweet\_potatoe\\
	\glt ‘Especially people who needed these knives to slice sweet potatoes, …'

\ex \label{ex-caus-instr1C}
	\glll	Onenesya amaino.\\
		o-nen-esy-a a-maino\\
	  2\textsc{sg.sbj}-bite-\textsc{caus}-\textsc{fv} \textsc{aug}-6.tooth\\
	\glt ‘You bite using the teeth.' 
	
\ex \label{ex-caus-instr2A}
	\glll	Eirai abantu beyambisyanga muno abantu abakulu.\\
		e-irai a-bantu ba-ee-yamb-isy-a-nga muno a-bantu a-ba-kulu\\
	 	\textsc{aug}-in\_the\_past \textsc{aug}-2.man \textsc{2sbj}-\textsc{refl}-help-\textsc{caus}-\textsc{fv}-\textsc{hab} much \textsc{aug}-2.man \textsc{aug}-2-elderly\\
	\glt ‘In the past, people used to help themselves a lot with (the advice from) elders.’ 
\end{xlist}
\z


\subsection{Passive}\label{sec-passive}
\largerpage
The passive voice is characterised by the reduced morpho-syn\-tac\-tic transitivity of the verb (e.g.\,the passive verb is intransitive when its active counterpart is transitive). 
Furthermore, the morpho-syntactic properties of the agent argument and the patient argument are affected. 
In comparison to an active clause, the agent argument is syntactically demoted and often suppressed, whereas the patient argument is promoted and acquires the morpho-syntactic properties of the subject 
(see \citealt[83–84]{Zunigaetal2019Grammatical} on the prototypical passive). 

Ruruuli-Lunyala has a productive passive construction. 
It is marked by the post-radical/pre-final suffix \emph{-ibw} (with the allomorphs [ibw] and [ebw]) or \emph{-w} (see Section~\ref{sec-extension-passive} on the distribution of the allomorphs). 
The patient argument of a transitive verb is promoted to the subject in the passive clause, 
for instance, it triggers the subject agreement on the verb, compare the two examples in (\ref{ex-passive}). 
In the active clause in~(\ref{ex-passive-act}) the agent argument of the verb \emph{nena} `bite' triggers the subject agreement on the verb; whereas in the passive counterpart with the same verb in~(\ref{ex-passive-pass}) the promoted patient argument triggers the subject agreement. 
The demoted agent argument can be realised in the same clause and does not receive any special marking, as e.g.\,\emph{emida} `lice' and \emph{enjunzai} `jiggers' in~(\ref{ex-passive-pass}).

\ea \label{ex-passive}
\begin{xlist}
	\ex \label{ex-passive-act}
	\glll Ebirumba bini tibinena bantu.\\
	  e-birumba bi-ni ti-bi-nen-a bantu\\
		\textsc{aug}-8.wasp 8-\textsc{prox} \textsc{neg}-\textsc{8sbj}-bite-\textsc{fv} 2.person\\
	\glt ‘These wasps don't bite people.’ 

	\ex \label{ex-passive-pass}
	\glll Twanenibwe emida enjunzai.\\
	  tu-a-nen-eibwe e-mida e-njunzai\\
		1\textsc{pl.sbj}-\textsc{pst}-bite-\textsc{pass}:\textsc{pfv} \textsc{aug}-4.louse \textsc{aug}-10.jigger\\
	\glt ‘We were bitten by lice, jiggers.’ 
\end{xlist}
\z


\subsection{Impersonal passive}\label{sec-impersonal}

In addition to the passive voice construction discussed in Section~\ref{sec-passive} above, Ru\-ruu\-li\hyp{}Lu\-nya\-la regularly employs what might be called an impersonal passive construction (or detailed studies of this construction in other Bantu languages, see e.g.\,\citealt[74–75, 96–97]{Givon1995Functionalism}, \citealt{Kawasha2007Passivization} and \citealt{Kulaetal2010Argument}). 
In this construction illustrated in~(\ref{ex-imperspassive}), the verb has the class 2 subject prefix \emph{ba-}, which is interpreted as non-referential: 
There is never an overt subject in the same clause and no subject referent can be reconstructed from the discourse. 
Morphologically, this construction is unmarked, that is, it is still an active construction. 
If expressed by a noun phrase, the patient argument is frequently fronted, as in~(\ref{ex-imperspassive-a}). 
The patient argument still triggers the object agreement on the verb, that is, it is not promoted to the subject (at least not with respect to agreement). 
Whether it acquires any morpho-syntactic properties of the subject in Ru\-ruu\-li\hyp{}Lu\-nya\-la, as e.g.\,is claimed for the \emph{ba-}passive construction in Bemba \citep{Kulaetal2010Argument}, remains a topic for further research.

\ea \label{ex-imperspassive}
\begin{xlist}

	\ex \label{ex-imperspassive-b}
	\glll Eitakali lya Nantaba balisendere.\\
	  e-itakali lya Nantaba ba-li-send-ire\\
		\textsc{aug}-5.land 5.\textsc{gen} 1.Nantaba \textsc{2sbj}-\textsc{5obj}-grade-\textsc{pfv}\\
	\glt ‘Nantaba's land was graded.’ 
	
	\ex \label{ex-imperspassive-a}
	\glll  Omwojo oyo baamusiramwire.\\
		o-mwojo oyo ba-a-mu-siramw-ire\\
		\textsc{aug}-1.boy 1.\textsc{med} \textsc{2sbj}-\textsc{pst}-\textsc{1obj}-circumcise-\textsc{pfv}\\
	\glt ‘That boy was circumcised.’ 


\end{xlist}
\z

\subsection{Reciprocal}\label{sec-reciprocal}
The term `reciprocal' is used to refer to situations with a meaning of the type ‘(to/of/against/from/with/on/at/etc.) each other’, i.e.\,situations which involve at least two participants which are in the identical reverse relation to each other. 
That is, these participants  all perform two identical semantic roles (e.g.\,of an agent and a patient) (see \citealt[6–7]{Nedjalkov2007Overview}). 
For instance, in~(\ref{ex-recipr-plain}), both Nakato and Babirye perform two semantic roles, viz.\,the agent and the patient at the same time and the example contains two subevents presented as one: 
(a) Nakato hits Babirye and (b) Babirye hits Nakato. 

In the Ru\-ruu\-li\hyp{}Lu\-nya\-la reciprocal construction the involved participants are expressed either as a comitative noun phrase, as in~(\ref{ex-recipr-plain}), or as a noun phrase in the plural, as in~(\ref{ex-recipr-pl}). 
The construction is marked by the suffix \emph{-angan}.
The morphology of the reciprocal marking is discussed in Section~\ref{sec-extension-reciprocal}.

\ea \label{ex-recipr-plain}
	\glll ONakato n’ oBabirye  bakubbangana.\\
	  o-Nakato na o-Babirye ba-kubb-angan-a\\
		\textsc{aug}-1.Nakato \textsc{com} \textsc{aug}-1.Babirye \textsc{2sbj}-hit-\textsc{recp}-\textsc{fv}\\
\glt ‘Nakato and Babirye hit each other.'
\z

\ea \label{ex-recipr-pl}
	\glll Abafumbo banwegerangana.\\
	  a-bafumbo ba-nweger-angan-a.\\
		\textsc{aug}-2.lover \textsc{2sbj}-kiss-\textsc{recp}-\textsc{fv}\\
\glt ‘Lovers kiss each other.'
\z

\subsection{Reflexive}\label{sec-reflexive}

The reflexive in Ru\-ruu\-li\hyp{}Lu\-nya\-la is marked by the prefix \mbox{\emph{ee-}}.\footnote{For orthographic considerations, the reflexive verbs are spelled with one <e> when listed as headwords in the dictionary.} 
The reflexive prefix immediately precedes the verb stem and derives from the Proto-Bantu \emph{*i-} \citep[109–110]{Meeussen1967Bantu}. 
This is the only diathetical operation in Ru\-ruu\-li\hyp{}Lu\-nya\-la marked by a prefix and not by a suffix and is thus not considered an extension in Bantu studies. 

The reflexive prefix \emph{ee-} is obligatorily used when the patient argument of a transitive verb is referentially identical with its agent argument in the subject function. 
Compare the two clause with the transitive verb \emph{swaka} `to cover' in~(\ref{ex-refl-tr}). 
In the transitive construction in~(\ref{ex-refl-tr1}) the agent and the patient arguments are referentially not identical, whereas in the intransitive reflexive construction  in~(\ref{ex-refl-tr2}) the agent and the patient are referentially identical.

\ea \label{ex-refl-tr}
\begin{xlist}
\ex \label{ex-refl-tr1}
	\glll Ebbandeegi yakusweka mukono.\\
	 e-bbandeegi 	ya-ku-swek-a mukono\\
		\textsc{aug}-9.bandage	\textsc{9sbj}-\textsc{prog}-cover-\textsc{fv} 3.hand\\
	\glt `The bandage is covering the hand.'

\ex \label{ex-refl-tr2}
	\glll Weesweke ebbulangiti.\\
	  o-ee-swek-e				e-bbulangiti\\
		2\textsc{sg.sbj}-\textsc{refl}-cover-\textsc{sbjv}	\textsc{aug}-9.blanket\\
	\glt `Cover yourself with the blanket.'
\end{xlist}	
\z

With ditransitive verbs the prefix \emph{ee-} is used when the recipient (or goal) argument is referentially identical with the A argument, compare the two examples in~(\ref{ex-refl-ditr}) with the stem \emph{alika} ‘to name, give name’.

\ea \label{ex-refl-ditr}
\begin{xlist}
\ex \label{ex-refl-ditr1}
	\glll Tukwaba    kwalika	abaana amabara.\\
		  tu-ku-ab-a    			ku-alik-a		a-baana 	a-mabara\\
		1\textsc{pl.sbj}-\textsc{prog}-\textsc{aux}-\textsc{fv}	\textsc{inf}-name-\textsc{fv}	\textsc{aug}-2.child	\textsc{aug}-6.name\\
	\glt ‘We are going to give the children names.’
\ex \label{ex-refl-ditr2}
	\glll Weeyaalika ibara Nampiina.\\
		  o-ee-a-alik-a 				ibara		Nampiina\\
		2\textsc{sg.sbj}-\textsc{refl}-\textsc{pst}-name-\textsc{fv}		5.name	1.Nampiina\\
	\glt ‘You named yourself Nampiina.’
\end{xlist}
\z

Furthermore, the prefix \emph{ee-} is used to indicate coreference between the agent argument in the subject roles and an applied object (mostly, a beneficiary, see Section~\ref{sec-applicative}). 
Compare the two examples in~(\ref{ex-refl-appl}). In (\ref{ex-refl-appl1}), the transitive verb \emph{luma} `to cultivate' takes an agent and a patient argument. 
In~(\ref{ex-refl-appl2}), the beneficiary is introduced with the applicative suffix \emph{-ir} and the reflexive prefix \emph{ee-} indicates that the beneficiary is referentially identical with the subject argument.

\ea \label{ex-refl-appl}
\begin{xlist}
\ex \label{ex-refl-appl1}
	\glll Nduma 	pampa.\\
	  	n-lum-a 				pampa\\
		\textsc{1sg.sbj}-cultivate-\textsc{fv}		9.cotton\\
	\glt 	‘I cultivate cotton.’
\ex \label{ex-refl-appl2}
	\glll Njeena	neelumira						eyo.\\
		nje-ena	n-ee-lum-ir-a					eyo\\
		1\textsc{sg}-also	\textsc{1sg.sbj}-\textsc{refl}-cultivate-\textsc{appl}-\textsc{fv}	there\\
	\glt ‘I also cultivate (crops) for myself there.’
\end{xlist}
\z

In addition to the productive reflexive construction discussed above, Ru\-ruu\-li\hyp{}Lu\-nya\-la has dozens of reflexiva tantum verbs, i.e.\,verbs which do not form a derivative opposition with any verb in modern Ru\-ruu\-li\hyp{}Lu\-nya\-la. 
Some of them are listed in~(\ref{ex-refl-tantum}).\footnote{Reflexiva tantum and semantic non-reversible reflexive verbs discussed below are widespread in the Bantu languages (see e.g.\,\citealt{Marlo2015Exceptional} for examples from a range of Bantu languages).}\textsuperscript{,}\footnote{Reflexiva tantum can be distinguished from verb roots which happen to begin with an /ee/, e.g.\,\emph{eya} `to sweep', as only the latter can form the imperative construction with a bare stem and the final vowel. The imperative of the reflexiva tantum can be built only using the form with the subject prefix and the subjunctive suffix, as in~(\ref{ex-refl-tantumex}), see the discussion of the examples in~(\ref{ex-aspect-sbjv-refl}) earlier in the book.}
These verbs can be intransitive (\ref{ex-refl-tantum1intr}) and transitive (\ref{ex-refl-tantum1tr}). 
An example sentence is provided in~(\ref{ex-refl-tantumex}).

\ea \label{ex-refl-tantum}
\begin{xlist}
\ex \label{ex-refl-tantum1intr}
	Intransitive: \emph{enga} ‘to ripen, be ripe’, \emph{esaragula} `to boil', \emph{epaakuula} `to be arrogant', \emph{ebaizagala} `to belch'

\ex \label{ex-refl-tantum1tr}	
	Transitive: \emph{ega} 'to learn', \emph{eteja} ‘to understand, comprehend, grasp’,  \emph{ebutuka} ‘to respond, answer’, \emph{esamba} `to avoid', \emph{eyayira} `to threaten'

\end{xlist}
\z
	
\ea \label{ex-refl-tantumex}
	\glll Mwesambe            entalo		za	basaiza	abo.\\
		mu-eesamb-e           e-ntalo	za	basaiza	ab-o\\
		2\textsc{pl.sbj}-avoid-\textsc{sbjv}	\textsc{aug}-10.fight	10.\textsc{gen}	2.man 		2-\textsc{med}\\
	\glt ‘Avoid the fights of those men.’
\z

Furthermore, a large number of reflexive verbs have non-reflexive counterparts but have undergone semantic derivation and have semantic components absent in the non-reflexive counterparts. 
Such lexicalised reflexive verbs have sometimes also undergone idiosyncratic changes to their valency and/or transitivity.  
For instance, the lexicalised intransitive reflexive verb \emph{esobola} `to be capable', as in (\ref{ex-refl-nonreversible2}), derives from the modal verb \emph{sobola} `can, to be able to' which regularly takes an infinitive complement, as in (\ref{ex-refl-nonreversible1}) (see Section \ref{sec-modal-verbs} on modal verbs).

\ea \label{ex-refl-nonreversible}
\begin{xlist}
\ex \label{ex-refl-nonreversible1}
	\glll Ati nsobola okwira.	\\
	 	ati 	n-sobol-a o-ku-ir-a\\
		now	\textsc{1sg.sbj}-can-\textsc{fv}	\textsc{aug}-\textsc{inf}-return-\textsc{fv}\\
	\glt 	`Now I can come back.'
\ex \label{ex-refl-nonreversible2}
	\glll	Baleetanga mbuli n' onte akwesobola.\\
	ba-leet-a-nga 		mbuli 	ni	o-nte a-ku-esobol-a\\
		\textsc{2sbj}-bring-\textsc{fv}-\textsc{hab}	10.goat	even	\textsc{aug}-1.cow \textsc{1sbj}-\textsc{prog}-be\_capable(\textsc{refl})-\textsc{fv}\\
	\glt ‘They could bring goats, even a cow in case one was capable.’
\end{xlist}
\z


\section{Interrogatives}\label{sec-interrogatives}

There are two main types of interrogative clauses in Ru\-ruu\-li\hyp{}Lu\-nya\-la: polar or yes-no questions are discussed in Section~\ref{sec-interrogatives-polar}, whereas content questions are discussed in Section~\ref{sec-interrogatives-content}.

\subsection{Polar interrogatives}\label{sec-interrogatives-polar}

Polar questions in Ru\-ruu\-li\hyp{}Lu\-nya\-la are not morphologically marked and the constituent order is identical to that of declarative sentences, as the example with a question-answer pair in~(\ref{ex-q-polar}) illustrates.
A distinct intonation pattern, as well as the context are the sole indications of polar questions.

\ea \label{ex-q-polar}
\begin{xlist}
\ex \label{ex-q-polar-q}
	\glll Omusyetete ogumaite?\\
	 o-musyetete o-gu-maite\\
		\textsc{aug}-3.mousebird 2\textsc{sg.sbj}-\textsc{3obj}-know:\textsc{pfv}\\
	\glt ‘Do you know the mousebird?’

\ex \label{ex-q-polar-a}
	\glll Omusyetete ngumaite.\\
		o-musyetete n-gu-maite\\
	\textsc{aug}-3.mousebird \textsc{1sg.sbj}-\textsc{3obj}-know:\textsc{pfv}\\
	\glt ‘I know the mousebird.’
\end{xlist}
\z


\subsection{Content interrogatives}\label{sec-interrogatives-content}

Ruruuli-Lunyala makes use of a number of question words which can be divided into two groups: 
(i) non-agreeing invariable interrogatives and (ii) agreeing interrogatives (see Table~\ref{tab-question-words}). 
The basic word order of constituent questions is mostly identical to the one in declarative clauses.  However, the position of the question words varies: 
The question word \emph{lwaki} `why' is aways clause-initial. 
\emph{ayi} `where', \emph{di} `when', and \emph{tayi} `how' either exclusively or primarily follow the verb. 
\emph{naani} `who', \emph{mekai} and \emph{ngai} `how many, how much' occur in situ.
Finally, the position of \emph{kiki}/\emph{ki} `what' and other agreeing forms of `what' varies.

\begin{table}
\caption{Question words}
\label{tab-question-words}
	\begin{tabular}{lll}
\lsptoprule

Type  & Question word & Position\\
\midrule
non-agreeing & \emph{lwaki} `why' & clause-initial\\
 & \emph{ayi} `where' & mostly post-verbal\\
 & \emph{di}  `when' & post-verbal\\
 \midrule
occasionally   & \emph{naani} `who, whom, whose' & in situ\\
agreeing & \emph{(ki)ki} `what, which' &  variable\\
 \midrule
agreeing  & \emph{ntyai} `how' & post-verbal\\
 & \emph{mekai} `how much, how many' & in situ\\
 & \emph{ngai}  `how much, how many' & in situ\\
 \lspbottomrule
 	\end{tabular}
 \end{table}

\subsubsection{\emph{lwaki} `why'} 
The question word \emph{lwaki} `why' asks about reason. 
It is the only question word which always occurs clause-initially, as in (\ref{ex-q-lwaki1}).
The question word can be preceded by clausal conjunctions if there are any, as in~(\ref{ex-q-lwaki2}), as well as by topicalised objects, as in~(\ref{ex-q-lwaki3}).

\ea \label{ex-q-lwaki}
\begin{xlist}
\ex \label{ex-q-lwaki1}
	\glll Lwaki ombwene akubwoigola?\\
	 lwaki o-mbwene a-ku-bwoigol-a\\
		why \textsc{aug}-1.dog \textsc{1sbj}-\textsc{prog}-bark-\textsc{fv}\\
	\glt ‘Why is the dog barking?'

\ex	\label{ex-q-lwaki2}
	\glll Naye lwaki tobyala?\\
		naye lwaki ti-o-byal-a\\
	but why \textsc{neg}-2\textsc{sg.sbj}-give\_birth-\textsc{fv}\\
	\glt ‘But why don’t you bear children?'

\ex	\label{ex-q-lwaki3}
	\glll Ebintu { }  byamu { } lwaki { } obikola { } biralebirale?\\
	  e-bintu { }  bi-a-mu { } lwaki { } o-bi-kol-a { }  biralebirale\\
		\textsc{aug}-8.thing { }  8-\textsc{assoc}-2\textsc{sg} { } why { } 2\textsc{sg.sbj}-\textsc{8obj}-do-\textsc{fv} { }  carelessly\\
\glt ‘Why do you do your things carelessly?'
\end{xlist}
\z

\subsubsection{\emph{ayi} `where'} 

To ask for a location or direction, the question word \emph{ayi} ‘where’ is used. 
Usually it occurs clause-finally in situ, as in~(\ref{ex-q-ayi-1}).
However, the position after the verb seems to be also possible when there is an object, as in the elicited example in~(\ref{ex-q-ayi-1}).
In non-verbal clauses (see Section~\ref{sec-non-verbal-predication} for details) the question word \emph{ayi} ‘where’ seems to fuse with the copula \emph{li}, as in~(\ref{ex-q-ayi-3}).

\ea \label{ex-q-ayi}
\begin{xlist}
\ex	\label{ex-q-ayi-1}
	\glll Omwana ayabire ayi?\\
	 o-mwana a-ab-ire ayi\\
		\textsc{aug}-1.child \textsc{1sbj}-go-\textsc{pfv} where\\
	\glt `Where has the child gone?’

\ex	\label{ex-q-ayi-2}
	\glll Watoire ayi obuyinza okuvuga mmotoka yange?\\
	 o-a-tol-ire 		ayi 		o-buyinza	o-ku-vug-a mmotoka ya-a-nge\\
	2\textsc{sg.sbj}-\textsc{pst}-get-\textsc{pfv}	where	\textsc{aug}-14.permission \textsc{aug}-\textsc{inf}-drive-\textsc{fv} 9.car 9-\textsc{assoc}-\textsc{1sg}\\
	\glt `Where did you get the permission to drive my car?'
	
\ex	\label{ex-q-ayi-3}
	\glll Ati owaamu alayi?\\
		Ati o-wa-a-mu a-li-ayi\\
		now \textsc{aug}-1-\textsc{assoc}-2\textsc{sg} \textsc{1sbj}-\textsc{cop}-where\\
	\glt	`Now where is yours?'\end{xlist}
\z


\subsubsection{\emph{di} `when'}  

The question word \emph{di} `when' is used to ask about the time of an event or state and is illustrated in~(\ref{ex-q-di}).
It occurs immediately after the verb, whereas the position of temporal adverbials (adverbs and noun phrases) varies.  

\ea \label{ex-q-di}
\begin{xlist}
\ex	\label{ex-q-di-1}
	\glll Omwana alyega di okuluka ekiibbo?\\
		 o-mwana 		a-li-eg-a di o-ku-luk-a e-kiibbo\\
		\textsc{aug}-1.child \textsc{1sbj}-\textsc{fut}-learn-\textsc{fv} when \textsc{aug}-\textsc{inf}-weave-\textsc{fv} \textsc{aug}-7.basket\\
	\glt `When will the child learn how to weave a basket?'

\ex	\label{ex-q-di-2}
	\glll Yayabire di?\\
		a-a-ab-ire di\\
		\textsc{1sbj}-\textsc{pst}-go-\textsc{pfv} when\\
	\glt	`When did he go?'
\end{xlist}
\z

\subsubsection{\emph{naani} `who'}  
The question word \emph{naani} `who, whom, whose' asks about animate participants. 
It occurs in situ and is most commonly attested in the subject position. 
The following examples illustrate this question word used to ask about various clause participants, as well as the respectively variable position within the clause. 
The question is about the subject in~(\ref{ex-q-naani-subj}), whereas it is about the object in~(\ref{ex-q-naani-obj}), and about the possessor in~(\ref{ex-q-naani-poss}). 
In~(\ref{ex-q-naani-subj}) the question is about the subject, it is about the object in~(\ref{ex-q-naani-obj}), and about the possessor in~(\ref{ex-q-naani-poss}). 
In the last case, the question word  \emph{naani} follows the genitive \emph{wa}, see Section~\ref{sec-NP-genitive}.

\ea \label{ex-q-naani}
\begin{xlist}
\ex 	\label{ex-q-naani-subj}
	\glll Naani amaite okubumba ebinaga?\\
		 naani a-maite o-ku-bumb-a e-binaga\\
	who \textsc{1sbj}-know:\textsc{pfv} \textsc{aug}-\textsc{inf}-mould-\textsc{fv} \textsc{aug}-8.pot\\
	\glt `Who knows how to mould pots?'

\ex 	\label{ex-q-naani-obj}
	\glll Okudomawalya naani?\\
		o-ku-domawaly-a naani\\
	2\textsc{sg.sbj}-\textsc{prog}-fool-\textsc{fv} who\\
	\glt ‘Whom are you fooling?'

\ex	\label{ex-q-naani-poss}
	\glll Mwala wa naani?\\
	  mwala wa naani\\
	1.daughter  1.\textsc{gen} who\\
\glt ‘Whose daughter is she?'
\end{xlist}
\z

The interrogative pronoun \emph{naani} does not bear a noun class prefix unless it unambiguously has a plural referent, then it takes the prefix \emph{ba-}, as in~(\ref{ex-q-baani}).

\ea \label{ex-q-baani}
\begin{xlist}
\ex 
	\glll Kayisi banaani?\\
	 kaisi ba-naani\\
		by.the.way 2-who\\
	\glt `By the way who are they?'

\ex 
	\glll Bakusasula banaani?\\
	 ba-ku-sasul-a ba-naani\\
		\textsc{2sbj}-\textsc{prog}-pay-\textsc{fv} 2-who\\
	\glt `To whom are they paying'
\end{xlist}
\z


\subsubsection{\emph{ki, kiki, biki, bikiki}, etc.\,`what'}  

To ask the `what' question either the non-agreeing question word \emph{ki}, as in~(\ref{ex-q-ki-object}), or its agreeing variants are used.
Among the agreeing forms, the word-form \emph{kiki} with the class 7 prefix \emph{ki-}, as in~(\ref{ex-q-kiki-obl}), is the most common one. 
The plural agreeing forms of class 8 \emph{bi-ki} `8-what', as in~(\ref{ex-q-biki}) and \emph{bi-kiki} `8-what' are also common,\footnote{The form \emph{bi-kiki} suggests that the agreeing form \emph{ki-ki} `7-what' is reanalysed as a non-agreeing stem \emph{kiki} `what'.}
Occasionally other agreeing can be used if the expected answer is a noun or other word of a specific noun class, as e.g.\,\emph{mu-ki} `1-what' in~(\ref{ex-q-rhetoric-muki}). 

\ea \label{ex-q-kiki}
\begin{xlist}
\ex	\label{ex-q-ki-object}
	\glll Abageni obagenywire ki?\\
		 a-bageni o-ba-genyul-ire ki\\
		\textsc{aug}-2.visitor 2\textsc{sg.sbj}-\textsc{2obj}-serve-\textsc{pfv} what\\
	\glt `What have you served to the visitors?'

\ex	\label{ex-q-kiki-obl}
	\glll Bagabiika mu kiki?\\
		 ba-ga-biik-a mu kiki\\
		\textsc{2sbj}-\textsc{6obj}-keep-\textsc{fv} 18.\textsc{loc} what\\
	\glt `Where do they keep it?' (lit.\,`They keep it in what?')
		
\ex	\label{ex-q-biki}
	\glll Mukubaza bikiki?\\
		 mu-ku-baz-a bi-kiki\\
		\textsc{2pl.sbj}-\textsc{prog}-talk-\textsc{fv}  8-what\\
	\glt `What are you talking?'

\ex	\label{ex-q-bikiki}
	\glll Mukukola biki?\\
		 mu-ku-kol-a bi-ki\\
		\textsc{2pl.sbj}-\textsc{prog}-do-\textsc{fv}  8-what\\
	\glt `What are you doing?'
\end{xlist}
\z

Two major syntactic strategies to ask a `what' question are available: 
In addition to the less strategy illustrated in~(\ref{ex-q-kiki}), where the question word is placed clause-finally, another common strategy is to use a construction with a relative clause and a question word in situ, as in~(\ref{ex-q-rel}) (see Section~\ref{sec-NP-relative}, for the description of the relative clause structure). 
Further research is needed to identify what determines the preferences for the various constructions and forms.

\ea \label{ex-q-rel}
\begin{xlist}

\ex	\label{ex-q-rel1}
	\glll Kiki ekyo ekikoore eiraka erikooto?\\
		 kiki eky-o e-ki-kol-ire e-iraka e-ri-kooto\\
		what 7-\textsc{med} \textsc{rel}-\textsc{7sbj}-make-\textsc{pfv} \textsc{aug}-5.sound \textsc{aug}-5-loud\\
	\glt `What has made the loud sound?’

\ex	\label{ex-q-rel2}
	\glll OMuseveni kiki ky' alekerewo?\\
		 o-Museveni ki-ki kye a-lek-ere=wo\\
		\textsc{aug}-1.Museveni 7-what \textsc{7.rel} 1-leave-\textsc{pfv}=\textsc{16.loc}\\
	\glt `What has Museveni left?’
\end{xlist}
\z

The question words discussed above are also frequently used clause-finally primarily in rhetorical questions to which the speaker himself immediately provides an answer, as in~(\ref{ex-q-rhetoric}). 

\ea \label{ex-q-rhetoric}
\begin{xlist}
\ex	\label{ex-q-rhetoric-a}
	\glll Nibazikola ki? Nibazirya.\\
		 ni-ba-zi-kol-a ki ni-ba-zi-li-a\\
		\textsc{nar}-\textsc{2sbj}-\textsc{10obj}-do-\textsc{fv} what \textsc{nar}-\textsc{2sbj}-\textsc{10obj}-eat-\textsc{fv}\\
	\glt `And they did what to it? They ate it.'

\ex	\label{ex-q-rhetoric-muki}
	\glll Nayenga Amin tiyali muki? Muzibu muno.\\
		 nayenga Amin ti-a-a-li mu-ki mu-zibu muno\\
		but 1.Amin \textsc{neg}-\textsc{1sbj}-\textsc{pst}-\textsc{cop} 1-what 1-difficult very\\
	\glt `But Amin wasn't what? Very difficult.'
	
\ex	\label{ex-q-rhetoric-c}
	\glll Byona byona biriga mu kiki? Mu nte.\\
		 by-ona by-ona bi-lig-a mu ki-ki mu nte\\
		8-all 8-all 		8-come\_from-\textsc{fv} 18.\textsc{loc} 7-what 18.\textsc{loc} 10.cattle\\
	\glt `All that come from what? From the cattle.'
\end{xlist}
\z


\subsubsection{\emph{tyai} `how'}  

The question word  \emph{tyai} `how' is used to ask about the manner of an action or a state and is illustrated in~(\ref{ex-q-tyai}). 
It immediately follows the verb. 
It always shows agreement with the subject of the clause. 

\ea \label{ex-q-tyai}
\begin{xlist}
\ex \label{ex-q-tyai-1}
	\glll Omanya otyai?\\
		 o-many-a o-tyai?\\
		2\textsc{sg.sbj}-know-\textsc{fv} 2\textsc{sg}-how\\
	\glt `How do you know?'

\ex 	\label{ex-q-tyai-2}
	\glll Agenda eri etyai?\\
		agenda e-li 	e-tyai\\
		9.agenda \textsc{1sbj}-\textsc{cop}	9-how\\
	\glt	`How is the agenda?'
	
\ex 	\label{ex-q-tyai-3}
	\glll Bawangwire batyai akalulu?\\
		ba-wangul-ire ba-tyai a-kalulu\\
		\textsc{2sbj}-succeed\_in-\textsc{pfv} 2-how \textsc{aug}-12.election\\
	\glt	`How did they win the elections?'
\end{xlist}
\z


\subsubsection{\emph{ingai/mekai} `how much/how many'} 

The interrogative stems \emph{ingai} and \emph{mekai} ‘how much/how many' are used to ask about the quantity of something. 
They seem to be used interchangeably. 
Both forms always follow and show agreement with the noun they modify, e.g.\,the form \emph{zingai} in~(\ref{ex-q-zingai}) shows agreement with the noun \emph{gologolo} `container  (10)', \emph{emekai} in~(\ref{ex-q-mekai}) agrees with the noun \emph{miruka} `parish (4)', whereas \emph{bamekai} in~(\ref{ex-q-mekai-ba}) shows agreement in class 2 with a human plural referent, which not overtly expressed in this clause.

\ea \label{ex-q-ingai}
\begin{xlist}

\ex \label{ex-q-zingai}
	\glll Okwendya egologolo zingai eza biyimba?\\
		o-ku-endy-a 	e-gologolo zi-ingai e-za biyimba\\
		2\textsc{sg.sbj}-\textsc{prog}-want-\textsc{fv} \textsc{aug}-10.container 10-how\_many \textsc{aug}-10.\textsc{gen} 8.bean\\
	\glt	`How many containers of beans do you want?'
	
\ex \label{ex-q-mekai}
	\glll Nakasongola erimu emiruka imekai? \\
	 Nakasongola e-li=mu e-miruka i-mekai\\
		9.Nakasongola \textsc{1sbj}-\textsc{cop}=18.\textsc{loc} \textsc{aug}-4.parish 4-how\_many\\
	\glt `How many parishes are in Nakasongola?’
	
\ex \label{ex-q-mekai-ba}
	\glll Bamekai abayabire eigolo?\\
		ba-mekai a-ba-ab-ire e-igolo\\
		\textsc{2sbj}-how\_many \textsc{rel}-\textsc{2sbj}-go-\textsc{pfv} \textsc{aug}-yesterday\\
	\glt	`How many (people) went yesterday?'
\end{xlist}
\z


\section{Other illocutions}\label{sec-syntax-illocutions}

The category of illocution is concerned with identifying sentences as instances of specific types of speech act \citet[1190]{Hengeveld2004Illocution}. 
In addition to declaratives and interrogatives discussed earlier (Sections~\ref{sec-clause} and \ref{sec-interrogatives} respectively), this section describes the expression of other most frequent illocutions, viz.\, the imperative (Section \ref{sec-imperative}), as well as the less common ones, such as the prohibitive (Section~\ref{sec-prohibitive}), as well as the hortative and jussive (Section~\ref{sec-hortative}). 
As the sections below make clear, they are organized according to types of speech act and not according to their formal properties.


\subsection{Imperative construction}\label{sec-imperative}

There are two imperative constructions in Ru\-ruu\-li\hyp{}Lu\-nya\-la. 
The first one is used for singular addressees. 
It is the only independent verb form in Ru\-ruu\-li\hyp{}Lu\-nya\-la that lacks the subject prefix and is marked with the final vowel.
It is formed by the verb stem with optional extensions followed by the final vowel \emph{-a}, as in~(\ref{ex-aspect-imper1}). 
The second construction is used for plural addressees: 
it is formed by prefixing the regular subject prefix to the verb stem and suffixing the subjunctive suffix \emph{-e}, as in~(\ref{ex-aspect-imper2}).
These two examples illustrate that there is no one-to-one relation between illocutionary functions and formal mood categories: 
for the same illocutionary function of imperative two mood  suffixes – indicative in (\ref{ex-aspect-imper1}) and subjunctive in~(\ref{ex-aspect-imper2}) – are used. 

\ea \label{ex-aspect-imper}
\begin{xlist}
\ex	\label{ex-aspect-imper1}
	\glll Iza.\\
	iz-a\\
		come-\textsc{fv}\\
	\glt ‘Come.’ 
\ex 	\label{ex-aspect-imper2}
	\glll Mwize.\\
	  mu-iz-e\\
		2\textsc{pl.sbj}-come-\textsc{sbjv}\\
	\glt  ‘You (pl.), come.'
\end{xlist}
\z

Whereas the indicative construction illustrated in~(\ref{ex-aspect-imper1}) is used to form the second person singular imperative with the majority of verbs, there are deviations from this pattern: all the reflexive verbs – including a substantial number of reflexiva tantum – form the second person singular imperative using the regular subject prefix and the subjunctive suffix \emph{-e}, as in~(\ref{ex-aspect-sbjv-refl-sg}). 
The same construction is used with the second person plural imperative of reflexive verbs, as is illustrated with a reflexive tantum verb in~(\ref{ex-aspect-sbjv-refl-pl}).

\ea \label{ex-aspect-sbjv-refl}
\begin{xlist}
\ex	\label{ex-aspect-sbjv-refl-sg}
	\glll Wekuutye 		oku			mwomo.\\
	  o-ee-kuuty-e 		o-ku			mwomo\\
		2\textsc{sg.sbj}-\textsc{refl}-rub-\textsc{sbjv} \textsc{aug}-17.\textsc{loc} 3.wall\\
\glt  ‘Rub yourself against the wall!’

\ex 	\label{ex-aspect-sbjv-refl-pl}
	\glll Mwesambe	entalo.\\
	  mu-eesamb-e            		entalo\\
		2\textsc{pl.sbj}-avoid-\textsc{sbjv} 10.fight\\
\glt ‘Avoid fights!’ (plural addressee) 

\end{xlist}
\z

When the imperative construction consists of two verbs, e.g.\,a verb of motion followed by another verb, the second verb takes the regular second person singular or plural subject prefix and the subjunctive suffix, as in~(\ref{ex-ruuli2verbimpsing}). 
If the subject is a second person plural addressee, as expected, also the first verb takes the subject prefix and the subjunctive suffix, as in~(\ref{ex-ruuli2verbimppl}). 

\ea 
\begin{xlist}
\ex \label{ex-ruuli2verbimpsing}
	\gll Iz-a o-zingulul-e o-lukosi!\\
		come-\textsc{fv} 2\textsc{sg.sbj}-uncoil-\textsc{sbjv} \textsc{aug}-11.thread\\
\glt ‘Come and uncoil the thread!’ 

	\ex \label{ex-ruuli2verbimppl}
	\gll Mu-iz-e mw-eroor-e 			o-ku musyoli!\\
		2\textsc{pl.sbj}-come-\textsc{sbjv} 2\textsc{pl.sbj}-see-\textsc{sbjv}	\textsc{aug}-17.\textsc{loc} 1.magician\\
\glt ‘Come and see the magician!’ 
\end{xlist}
\z

Finally, imperatives with an object indexing deviate from the two major patterns discussed above: 
They take the subjunctive suffix both with a second person singular and with a plural addressee, see~ (\ref{ex-ruuliimpobj2})--(\ref{ex-ruuliimpobj3}). 
The use of the final vowel \emph{-a} in an imperative construction with object agreement prefixes is ungrammatical, as in~(\ref{ex-ruuliimpobj2wrong})--(\ref{ex-ruuliimpobj3wrong}). 
However, they do not take any subject agreement prefixes, compare~(\ref{ex-ruuliimpobj3}) with~(\ref{ex-ruuliimpobj3wrong}).

\ea \label{ex-ruuliimpobj}
\begin{xlist}
\ex 	\label{ex-ruuliimpobj1}
	\glll Kubba enkaito zaamu!\\	
	    kubb-a enkaito za-a-mu\\
		polish-\textsc{fv} 10.shoe 10-\textsc{assoc}-2\textsc{sg}\\
	\glt ‘Polish your shoes!’ 
	
\ex	\label{ex-ruuliimpobj2}
	\glll   Zi-kubb-e!\\ 
	        zi-kubb-e!\\
		\textsc{10obj}-polish-\textsc{sbjv}\\
	\glt ‘Polish them (e.g.\,shoes)!’
	
\ex 	\label{ex-ruuliimpobj3}
	\glll   Mu-zi-kubb-e!\\
	        mu-zi-kubb-e\\
		2\textsc{pl.sbj}-\textsc{10obj}-polish-\textsc{sbjv}\\
	\glt ‘You (pl.), polish them (e.g.\,shoes)!’

\ex \label{ex-ruuliimpobj2wrong}
	\gll *zi-kubb-a\\
		\textsc{10obj}-polish-\textsc{fv}\\
	\glt Intended: ‘Polish them (e.g.\,shoes)!’

\ex \label{ex-ruuliimpobj3wrong}
	\gll *mu-zi-kubb-a/-e\\
		2\textsc{pl.sbj}-\textsc{10obj}-polish-\textsc{fv}/-\textsc{sbjv}\\
	\glt Intended: ‘You (pl.), polish them (e.g.\,shoes)!’
\end{xlist}
\z

\subsection{Prohibitive construction}\label{sec-prohibitive}
The prohibitive construction is used to express a prohibition addressed to a single or plural addressee. 
In the prohibitive construction, the postinitial negation marker \emph{ta-} or its allomorph \emph{t-} and the final vowel \emph{-a} are used, as in~(\ref{ex-ruuliprohibnormal}). 
Sometimes, the regular negative form with the prefix \emph{ti-} is used to express negative imperative semantics. 
This seems to be the case when the statement  is rather an instruction than a command. 
In contrast to the imperative discussed above, there is no deviation from this rule when it comes to reflexive verbs, as in~(\ref{ex-ruuliprohibrefl}) with a reflexiva tantum, second person plural subjects, as in~(\ref{ex-ruuliprohib2pl}), or with verb forms with object indices, as in~(\ref{ex-ruuliprohibobj}). 

\ea \begin{xlist}
\ex	\label{ex-ruuliprohibnormal}
	\glll	Otakubb-a nkaito zaamu.\\
		 o-ta-kubb-a nkaito za-a-mu\\
		2\textsc{sg.sbj}-\textsc{neg}-polish-\textsc{fv} 10.shoe 10-\textsc{assoc}-2\textsc{sg}\\
	\glt ‘Don't polish your shoes.’ 

\ex 	\label{ex-ruuliprohibrefl}
	\glll  Oteebulyabulya.\\
		o-ta-eebulyabuly-a\\
		2\textsc{sg.sbj}-\textsc{neg}-be\_evasive-\textsc{fv}\\
	\glt ‘Don't be evasive.’ 

\ex 	\label{ex-ruuliprohib2pl}
	\glll Mutakubba nkaito.\\
		mu-ta-kubb-a nkaito\\
		2\textsc{pl.sbj}-\textsc{neg}-polish-\textsc{fv} 10.shoes\\
	\glt ‘You (pl.), don't polish the shoes.’ 

\ex 	\label{ex-ruuliprohibobj}
	\glll Otazikubba.\\
		o-ta-zi-kubb-a\\
		2\textsc{sg.sbj}-\textsc{neg}-\textsc{10obj}-polish-\textsc{fv}\\
	\glt ‘Don't polish them (e.g.\,the shoes).’ 	
\end{xlist}
\z

The prohibitive construction used when there are two verbs is slightly different from the imperative construction with one verb. 
Whereas the second verb carries the subject prefix and subjunctive suffix in the affirmative, as in~(\ref{ex-ruuli2verbimpsing})--(\ref{ex-ruuli2verbimppl}), in the negative the infinitive is used instead both with a singular and plural addressee, as in~(\ref{ex-ruuliprohib2verbsing}) and~(\ref{ex-ruuliprohib2verbpl}) respectively. 

\ea \label{ex-ruuliprohib2verb}
\begin{xlist}
	\ex \label{ex-ruuliprohib2verbsing}
	\glll  Otayaba kusyoma maizi.\\
		o-ta-ab-a ku-syom-a maizi\\
		2\textsc{sg.sbj}-\textsc{neg}-go\_away-\textsc{fv} \textsc{inf}-fetch-\textsc{fv} 6.water\\
\glt ‘Don't go to fetch water.’ 

	\ex \label{ex-ruuliprohib2verbpl}
	\glll	Mutayaba kusyoma maizi.\\
		mu-ta-ab-a ku-syom-a maizi\\
		2\textsc{pl.sbj}-\textsc{neg}-go\_away-\textsc{fv} \textsc{inf}-fetch-\textsc{fv} 6.water\\
\glt ‘You (pl.), don't go and bring water.’ 
\end{xlist}
\z

\subsection{Hortative and jussive construction}\label{sec-hortative}
Hortative refers to the function of an expression to encourage or incite someone to action, specifically to the first person plural forms (i.e.\,to the constructions of the type ‘let us VERB').
The Ru\-ruu\-li\hyp{}Lu\-nya\-la hortative is constructed by affixing the first person plural subject agreement marker to the verb and the subjunctive suffix \emph{-e}, which replaces the final vowel \emph{-a}. 
This construction is illustrated in~(\ref{ex-ruulihortative}). 
In the negative hortative construction, the postinitial negation marker \emph{ta-} and final vowel \emph{-a} are used, as in~(\ref{ex-ruulihortativeneg}). 

\ea \label{ex-ruulimoods}
\begin{xlist}
\ex \label{ex-ruulihortative}
	\glll  Twigale olwige.\\
		tu-igal-e o-lwige\\
		1\textsc{pl.sbj}-close-\textsc{sbjv} \textsc{aug}-11.door\\
	\glt ‘Let's close the door.’ 
	
\ex \label{ex-ruulihortativeneg}
	\glll	Tutaigala lwige.\\
		  tu-ta-igal-a lwige\\
		1\textsc{pl.sbj}-\textsc{neg}-close-\textsc{fv} 11.door\\
	\glt ‘Let's not close the door.’ 
\end{xlist}
\z

In order to form a jussive, the same construction as in hortative is used but with a non-first person plural subject, as in  (\ref{ex-ruulijussiv}). 

\ea \label{ex-ruulijussiv}
\begin{xlist}
\ex \label{ex-ruulijussivepos}
	\glll Baigale amadirisa.\\
		 ba-igal-e a-madirisa\\
		\textsc{2sbj}-close-\textsc{sbjv} \textsc{aug}-10.window\\
	\glt ‘Let them close the windows.’ 
\ex \label{ex-ruulijussiveneg}
	\glll	Bataigala madirisa.\\
		  ba-ta-igal-a madirisa\\
		\textsc{2sbj}-\textsc{neg}-close-\textsc{fv} 10.window\\
	\glt ‘Let them not close the windows.’
\end{xlist}
\z


\section{Modal verbs}\label{sec-modal-verbs}

Modal predicates have meanings such as ‘may’ and ‘can’ and describe likelihood, possibility, ability, permission and obligation (see e.g.\, \citealt[33]{Palmer2001Mood}). We follow \citet{vanderAuweraetal1998Modality} and distinguish two major types of modal expression, viz. possibility and necessity, as well as four modality domains viz.\, participant-internal modality, par\-ti\-ci\-pant-external modality with a subtype of deontic modality, as well as epistemic modality. The various types are briefly summarised below following \citet[80–81]{vanderAuweraetal1998Modality}. 
Participant-internal modality refers to a kind of possibility or necessity internal to a participant engaged in the state of affairs, i.e.\,to a participant’s ability or capacity in the first case and to his internal need in the second case. 
By contrast, par\-ti\-ci\-pant-external modality refers to circumstances that are external to the participant engaged in the state of affairs, these are the circumstances that make this state of affairs either possible or necessary. 
Deontic modality identifies the enabling or obliging circumstances external to the participant as someone (often the speaker of the utterance) or some social or ethical norms permit or oblige the participant to engage in the state of affairs. 
Finally, epistemic modality refers to a judgment of the speaker: the speaker judges a proposition as either uncertain or probable.

In Ru\-ruu\-li\hyp{}Lu\-nya\-la at least six different verbs are regularly used as modal predicates. 
Some of these verbs are attested in other West Nyanza languages with similar meanings (see e.g.\,\citealt[313]{Nabirye2016Corpus-based}, \citealt{Kawalyaetal2018Reconstructing}). Below we illustrate these modal verbs and discuss the types of modality they express.

Possibility is expressed in Ru\-ruu\-li\hyp{}Lu\-nya\-la by the modal verbs \emph{sobola} and \emph{yinza} both translated as ‘can, may, be able’. 
These two verbs can express all possibility domains: parti\-ci\-pant-internal possibility, as in (\ref{ex-modal-verbs-sobol1}) and (\ref{ex-modal-verbs-yinz1}), non-deontic parti\-ci\-pant-external possibility, deontic parti\-ci\-pant-external possibility, as in (\ref{ex-modal-verbs-sobol2}), as well as epistemic possibility, as in (\ref{ex-modal-verbs-yinz2}).

\ea \label{ex-modal-verbs-sobol}
\begin{xlist}
\ex \label{ex-modal-verbs-sobol1}
	\glll Nsobola 	okusosoitoora	omuntu		ekiibulo.\\
	  n-sobola 	o-ku-sosoitoor-a	o-muntu		e-kiibulo\\
		\textsc{1sg.sbj}-can-\textsc{fv}	\textsc{aug}-\textsc{inf}-serve-\textsc{fv}	\textsc{aug}-1.person		\textsc{aug}-7.meal\\
	\glt ‘I can serve a person a meal’
	
\ex \label{ex-modal-verbs-sobol2}
	\glll Osobola	okwaba	omu		kanisa.\\
		 o-sobol-a	o-ku-ab-a	o-mu		kanisa\\
		2\textsc{sg.sbj}-can-\textsc{fv}	\textsc{aug}-\textsc{inf}-go-\textsc{fv}	\textsc{aug}-18.\textsc{loc}	9.church\\
	\glt ‘You can go to church.’
\end{xlist}
\z

\ea \label{ex-modal-verbs-yinz}
\begin{xlist}
\ex \label{ex-modal-verbs-yinz1}
	\glll Oyinza	kunfunirayo		onkowu	 omwei?\\
		  o-yinz-a	ku-n-fun-ir-a=yo				o-nkowu		o-mwei\\
		2\textsc{sg.sbj}-can-\textsc{fv}	\textsc{inf}-\textsc{1sg.obj}-get-\textsc{appl}-\textsc{fv}=23.\textsc{loc}		\textsc{aug}-1.guinea\_fowl	1-one\\
	\glt ‘Can you get me one guinea fowl?’

	
\ex \label{ex-modal-verbs-yinz2}
	\glll E		Kidera	eyo	omu 		katale		gayinza { } { } { } okubbayoku.\\
		 e		Kidera	eyo	o-mu 		katale		ga-yinz-a { } { } { } o-ku-bb-a=yo=ku\\
			23.\textsc{loc}	Kidera	there	 \textsc{aug}-18.\textsc{loc}	12.market	\textsc{6sbj}-can-\textsc{fv} { } { } { } \textsc{aug}-\textsc{inf}-be-\textsc{fv}=23.\textsc{loc}=17.\textsc{loc}\\
	\glt ‘There may be some (spears to buy) there at Kidera in the market.’
\end{xlist}
\z

Necessity is expressed by the modal verbs \emph{lina} and \emph{teekwa} both translated as ‘must, have to’. 
Both of these verbs are used to express deontic (\ref{ex-modal-verbs-lin}) and non-deontic participant-external necessity. 
Only \emph{teekwa} is found in our corpus to express epistemic necessity, as in (\ref{ex-modal-verbs-tekw}).

\ea \label{ex-modal-verbs-other}
\begin{xlist}
\ex \label{ex-modal-verbs-lin}
	\glll Olina			kusala		musaayi.\\
	  o-lin-a			ku-sal-a		musaayi\\
		2\textsc{sg.sbj}-have.to-\textsc{fv}	\textsc{inf}-sacrifice-\textsc{fv}	3.blood\\
	\glt ‘You have to sacrifice blood.’

\ex \label{ex-modal-verbs-tekw}
	\glll Kateeka 	okubbamu		omu		kidoodolo.\\
	 ka-teek-a 	o-ku-bba=mu		o-mu		kidoodolo\\
		1\textsc{2sbj}-must-\textsc{fv}	\textsc{aug}-\textsc{inf}-be=18.\textsc{loc}	\textsc{aug}-18.\textsc{loc}	7.granary\\
	\glt ‘It must be there in the granary.’
	\end{xlist}
\z

All the modal verbs discussed above occur exclusively with infinitive complements. 
To express participant-internal necessity the verb \emph{endya} and occasionally \emph{taka} are used. 
In most cases they are used with the meaning ‘want’ and in this meaning they take both infinitive and subjunctive complements primarily conditioned by whether the subject of the matrix clause and the complement verb are identical. 
In their modal meaning, they also allow for these two complementation strategies under the same conditions. 
With the same subject in the two clauses we find an infinitive complement, as in~(\ref{ex-modal-verbs-yendya}).
In addition to these two verbs, the corpus contains a few tokens of the verb \emph{etaaga} ‘need’, which is also used to express participant-internal necessity. 

\ea \label{ex-modal-verbs-yendya}
	\glll Nkwendya kutanaka.\\
	  n-ku-yendy-a			ku-tanak-a\\
		\textsc{1sg.sbj}-\textsc{prog}-need-\textsc{fv}		\textsc{inf}-vomit-\textsc{fv}\\
	\glt ‘I need to vomit.’
\z

Of the verbs discussed above, only some are also found as lexical verbs taking nominal objects. 
\emph{sobola} has the meaning ‘manage smth./smb.’ when used as the main verb, 
\emph{etaaga} is used as ‘need smth.’, \emph{lina} is common as ‘have smth./smb.’ (see also Footnote~\ref{footnote-lina}),  \emph{taka} and \emph{yendya} both meaning ‘want smth./smb.’.



\section{Complement clauses}\label{sec-syntax-complementation}

There are three main complement types in Ru\-ruu\-li\hyp{}Lu\-nya\-la depending on the complement verb form: indicative complements with the final vowel suffix \textit{-a} on the verb, subjunctive complements with the verb suffix \textit{-e}, as well and infinitive complements, which are marked with the class 15 prefix \textit{ku-} glossed here as \textsc{inf}. 
The most frequent and versatile complement type in Ru\-ruu\-li\hyp{}Lu\-nya\-la is the infinitive complement, which occurs with all classes of complement-taking predicates except for perception predicates  \citep{Sorensenetal2020Clausal}. 

In indicative complements, the complement's verb form does not differ from an independent declarative clause's verb form: 
There is the obligatory subject agreement and optional object agreement on the verb. 
Furthermore, the verb can occur with the same TA-categories as a verb in an independent declarative clause. 
When the verb in an indicative complement is to be negated, the standard negator \textit{ti-} is used, as in~(\ref{ex-Ruruuli-Lunyalacompindicative}) (see also \citealt{Sorensenetal2020Clausal}).

\ea 	\label{ex-Ruruuli-Lunyalacompindicative} 
	\glll Babonire tiekyali ya mugaso.\\
	  babon-ire [ti-e-kya-li ya mugaso]\\
		\textsc{2sbj}-see-\textsc{pfv} \textsc{neg}-\textsc{1sbj}-\textsc{pers}-\textsc{cop} 9.\textsc{gen} 3.importance\\
\glt ‘They have seen they are no longer of importance.’ 
\z

As we discussed in Section~\ref{sec-mood}, the Ru\-ruu\-li\hyp{}Lu\-nya\-la subjunctive is used in independent clauses to express positive jussive and hortative, as in~(\ref{ex-subjunctiveindependent}), as well as optative and modal meanings. 
The negation of a subjunctive complement seems highly unnatural and is not found in the corpus.

\ea \label{ex-ruulisubjunctivecompl}
\begin{xlist}
\ex \label{ex-subjunctiveindependent}
	\glll Tusomesye baana.\\
  tu-somesy-e baana\\
		1\textsc{pl.sbj}-teach-\textsc{sbjv} 2.child\\
	\glt ‘Let us educate children.’ 	
\ex \label{ex-subjunctivedependent}
	\glll Otaka  oteewo olukonko.\\ 
	 o-tak-a  [o-ta-e=wo o-lukonko]\\ 
		2\textsc{sg.sbj}-want-\textsc{fv}  2\textsc{sg.sbj}-put-\textsc{sbjv}=16.\textsc{loc} \textsc{aug}-11.rift\\
	\glt ‘You want to cause a rift.' 
\end{xlist}
\z

The third type of complement clauses contains the infinitive verb form. 
In order to form the infinitive, class 15 prefix \textit{ku-} is added to the verb stem that is followed by the final vowel \textit{-a}. 
The class prefix is often preceded by the corresponding augment \textit{o-}. 
The distribution of the augment is conditioned by a range of syntactic and semantic factors and its description goes beyond the scope of the present study. 
Suffice it to say that the augment \textit{o-} cannot be used on the infinitive when the complement-taking predicate is negated, as in~(\ref{ex-ruuliku}) (cf.\,\citealt{Sorensenetal2020Clausal}).

\ea 	\label{ex-ruuliku} 
	\glll Tintaka kunywa isara.\\
	  ti-n-tak-a ku-nyw-a isara\\
		\textsc{neg}-\textsc{1sg.sbj}-like-\textsc{fv} \textsc{inf}-drink 5.stale.brew\\
\glt ‘I don’t like drinking stale brew.’ 
\z

Infinitives themselves are negated by means of the postinitial negation marker \textit{ta-}. 
When negated, the infinitive carries the class 14 prefix \textit{bu-} instead of the regular class 15 prefix \textit{ku-}. 
The class prefix can be preceded by the respective augment \textit{o-}, as in~(\ref{ex-ruulibu}). 
It is not yet clear under which circumstances the speakers prefer to negate the infinitive directly instead of negating the matrix verb. 
In any case, the patter with  \textit{bu-} is used to negate infinitives also when they do not form a complement.

\ea 	\label{ex-ruulibu}
	\glll Omusaiza ayinza obutaleetawo mukali.\\
	  o-musaiza a-yinz-a [o-bu-ta-leet-a=wo mukali]\\
		\textsc{aug}-1.man \textsc{2sbj}-be\_able-\textsc{fv} \textsc{aug}-16-\textsc{neg}-bring-\textsc{fv}=16.\textsc{loc} 1.wife\\
\glt ‘The man may not bring there a wife.’ 
\z


\section{Adverbials}\label{sec-syntax-adverbs}

This section discusses several types of adverbials. We use the term “adverbial” to refer to the elements that serve to specify further the circumstances of the verbal or sentential referent (see \citealt{Maienbornetal2011Adverbs}). 
We start with a discussion of the use of most common semantic classes of adverbs:  manner adverbs (Section~\ref{sec-syntax-adverbs-manner}), temporal adverbs (Section~\ref{sec-syntax-adverbs-temporal}), and degree adverbs (Section~\ref{sec-syntax-adverbs-degree}). 
We also touch upon adverbial clauses in Section~\ref{sec-syntax-adverbial clauses}.

\subsection{Manner adverbs} \label{sec-syntax-adverbs-manner}

Manner adverbs are used to specify the manner in which an eventuality or an action unfolds. 
This is a relatively large class in Ruuruli-Lunyala showing various derivation patterns, some examples are given in~(\ref{ex-adv-derivation}) above. 
The following examples illustrate how manner adverbs are used. 
(\ref{ex-adv-manner-sentence1})–(\ref{ex-adv-manner-sentence2}) are typical examples with the adverb immediately following the verb, whereas (\ref{ex-adv-manner-sentence3}) shows that it is also possible to have the adverb clause-finally after the object. 

\ea \label{ex-adv-manner-sentence}
\begin{xlist}
\ex 	\label{ex-adv-manner-sentence1}
	\glll Omuganda nayindula mangu eibara lyamu.\\
  	o-muganda 		ni-a-yindul-a 			mangu e-ibara 	lya-a-mu\\
		\textsc{aug}-1.Muganda 	\textsc{nar}-\textsc{1sbj}-interpret-\textsc{fv} 	quickly \textsc{aug}-5.name 	5-\textsc{assoc}-2\textsc{sg}\\
	\glt ‘The Muganda quickly interpreted your name.’ 

\ex 	\label{ex-adv-manner-sentence2}
 	\glll Ebintu byamu lwaki obikola biralebirale?\\
		 e-bintu by-a-mu lwaki o-bi-kol-a biralebirale\\
		\textsc{aug}-8.thing 8-\textsc{assoc}-2\textsc{sg} why 2\textsc{sg.sbj}-\textsc{8obj}-make-\textsc{fv} carelessly\\
\glt ‘Why do you do your things carelessly?'

\ex 	\label{ex-adv-manner-sentence3}
	\glll Noluka obuguwa bwamu mpolampola.\\
	ni-o-luk-a 		o-buguwa 		bu-a-mu 	mpola-mpola\\
	\textsc{nar}-2\textsc{sg.sbj}-weave-\textsc{fv} 	\textsc{aug}-14.string 	14-\textsc{assoc}-2\textsc{sg} 	slowly-<\textsc{red}>\\
	\glt ‘You weaved your strings very slowly.' 	
\end{xlist}
\z


\subsection{Temporal adverbs} \label{sec-syntax-adverbs-temporal}
Temporal adverbs form a large class of adverbs. They are morphologically different from other adverbs as they can occur with the augment, as e.g.\,\emph{e-mambya} (\textsc{aug}-tomorrow) `tomorrow' in~(\ref{ex-adverbs-temp1}). 
They occur either clause-initially, as in~(\ref{ex-adverbs-temp1}), or clause-finally, as in~(\ref{ex-adverbs-temp2}).

\ea \label{ex-adverbs-temp}
\begin{xlist}
\ex \label{ex-adverbs-temp1}
	\glll  Emambya twalya kaita ka mweryai.\\
	e-mambya tu-a-li-a kaita ka mweryai\\
	\textsc{aug}-tomorrow	1\textsc{pl.sbj}-\textsc{fut}-eat-\textsc{fv} 12.millet 12.\textsc{gen}  3.fresh\_millet\\
	\glt `Tomorrow we will eat a meal of freshly harvested millet.'

\ex \label{ex-adverbs-temp2}
	\glll  Yaliire enyaawo eigolo.\\
a-a-liire e-nyaawo e-igolo\\
\textsc{1sbj}-\textsc{pst}-eat:\textsc{pfv} \textsc{aug}-9.udder\_meat \textsc{aug}-yesterday\_evening\\
	\glt `He ate udder meat yesterday evening.'
\end{xlist}
\z

\subsection{Degree adverbs}\label{sec-syntax-adverbs-degree}
A few adverbs of degree are used both in positive clauses as well as in negative ones to reinforce negation. 
They include \textit{kimwe(i)} `totally, completely', \textit{kaaki\-mwe(i)} `thoroughly, totally', and \textit{kakyarumwe(i)} `totally, completely'.
These adverbs usually directly follow the verb they modify, as in~(\ref{ex-ruulireinforcing}):

\ea \label{ex-ruulireinforcing}
\begin{xlist}

\ex \label{ex-ruuliinflittle}
	\glll  Abakagiri abo mbamanyire kimwei, aBakagiri abo bantu batuufu.\\
		a-Bakagiri a-bo n-ba-many-ire kimwei a-Bakagiri a-bo bantu ba-tuufu\\
		\textsc{aug}-2.Bakagiri 2-\textsc{med} \textsc{1sg.sbj}-\textsc{2obj}-know-\textsc{pfv}  totally \textsc{aug}-2.Bakagiri 2-\textsc{med} 2.person 2-correct\\
	\glt ‘I totally know those Bakagiri, those Bakigiri are correct people.’ 
	
\ex \label{ex-ruuliinfmore}
	\glll	Yamukubbiire kaakimwe.\\
		a-a-mu-kubb-iire kaakimwe\\
		\textsc{1sbj}-\textsc{pst}-\textsc{1obj}-beat-\textsc{appl:pfv}  thoroughly\\
	\glt ‘He beat him thoroughly.’ 
\end{xlist}
\z

When used with negative constructions, these adverbs are often translated as `at all'. 
However, they express different degrees of reinforcement. 
Specifically, \textit{kimwei} adds the least fortification to a negative statement, as in~(\ref{ex-synt-reinforce-little}). 
The reinforcement is strengthened when \textit{kaakimwei} is used, as in~(\ref{ex-synt-reinforce-more}). 
The strongest fortification of negation is achieved by the use of \textit{kakyarumwei}, as in~(\ref{ex-synt-reinforce-strongest}) (see also \citealt{Ruppertetal2021Negation}). 
Note that with these adverbs, the verb takes an applicative suffix both in negative and in positive forms, compare (\ref{ex-synt-reinforce-none}) with~(\ref{ex-synt-reinforce-little})–(\ref{ex-synt-reinforce-more}).

\ea \label{ex-synt-reinforce}
\begin{xlist}
\ex \label{ex-synt-reinforce-none}
	\glll  Tinlya muceere.\\
		ti-n-li-a muceere\\
		\textsc{neg}-\textsc{1sg.sbj}-eat-\textsc{fv} 3.rice\\
	\glt ‘I don't eat rice.’ 
	
\ex \label{ex-synt-reinforce-little}
	\glll  Tinlyaira kimwe muceere.\\
		ti-n-lyair-a kimwe muceere\\
		\textsc{neg}-\textsc{1sg.sbj}-eat:\textsc{appl}-\textsc{fv} totally 3.rice\\
	\glt ‘I don't eat rice at all.’ 
	
\ex \label{ex-synt-reinforce-more}
	\glll Tinlyaira kaakimwe muceere.\\
	 ti-n-lyair-a kaakimwe muceere\\
		\textsc{neg}-\textsc{1sg.sbj}-eat:\textsc{appl}-\textsc{fv} thoroughly 3.rice\\
	\glt ‘I don't eat rice at all.’ 
	
\ex \label{ex-synt-reinforce-strongest}
	\glll  Tinlyaira kakyarumwe muceere.\\
	ti-n-lyair-a kakyarumwe muceere\\
		\textsc{neg}-\textsc{1sg.sbj}-eat:\textsc{appl}-\textsc{fv} completely 3.rice\\
	\glt ‘I don't eat rice at all.’  
\end{xlist}
\z


\subsection{Adverbial clauses}\label{sec-syntax-adverbial clauses}

Adverbial clauses modify other clauses in a sentence in a way similar to the way in which an adverb modifies a proposition \citet[:237]{Thompsonetal2007Adverbial}. 
Various types of adverbial clauses can be distinguished on the basis of to the semantic roles they play. 
The two most frequent types of adverbial clauses in Ru\-ruu\-li\hyp{}Lu\-nya\-la are adverbial clauses introduced by the conjunctions \emph{nga} and \emph{ni}.

The most frequent type of adverbial clauses are temporal clauses and conditional clauses marked by the conjunction \textit{nga} `while, when, if', as in~(\ref{ex-adverbial-nga}).\footnote{Ruuli has further, less common types of adverbial clauses, which are currently being investigated, e.g.\,the concessive adverbial clause introduced by the conjunction \emph{wade} `though, even if, although'.}
The conjunction \textit{nga} is always the first item in the respective adverbial clause and can e.g.\,preceded the subject, as in~(\ref{ex-adverbial-nga-N}).
These clauses are negated by means of the standard negator \emph{ti-}, as in~(\ref{ex-adverbial-nga-neg}) (see Section~\ref{sec-negation}).

\ea \label{ex-adverbial-nga}
\begin{xlist}

\ex \label{ex-adverbial-nga-N}
    \glll   Naiza nga eisana ligwire.\\
        n-a-iz-a nga e-isana li-gu-ire\\
        1\textsc{sg.sbj}-\textsc{fut}-come-\textsc{fv}	when	\textsc{aug}-5.sun 	\textsc{5sbj}-set-\textsc{fv}\\
    \glt `I will come at dusk.' (lit.\,`I will come, when the sun sets.')
	
\ex
    \glll   Nkumanyisyemu 				nga 	twireyo 			oku 		bbaibbuli?\\
        n-ku-many-isy-e=mu			nga	tu-ire=yo			o-ku		bbaibbuli\\
        1\textsc{sg.sbj}-2\textsc{sg.obj}-know-\textsc{caus}-\textsc{sbjv}=18.\textsc{loc}	when	1\textsc{pl.sbj}-return.\textsc{pvf}=23.\textsc{loc}	\textsc{aug}-17.\textsc{loc}	9.Bible\\
    \glt `Should I let you know when we return to the Bible?'

\ex \label{ex-adverbial-nga-neg}
    \glll Toyabanga mu kisiko nga tokwaite mwigo.\\
         ti-o-yaba-nga mu kisiko nga ti-o-kwaite 	        	mwigo\\
	   \textsc{neg}-2\textsc{sg.sbj}-go-\textsc{hab} 18.\textsc{loc}	7.bush	while \textsc{neg}-2\textsc{sg.sbj}-hold:\textsc{pfv} 3.stick\\
	\glt ‘Never go to the bush without a stick.’
\end{xlist}
\z

A clause introduced by the conjunction \textit{nga} `while, when' is often used to express the caritive meaning of \textit{without + noun} and \textit{without + verb}.
For the meaning of \textit{without + noun} the predicative possession construction including the copula \textit{li} merged with the comitative preposition \textit{na} (see Section~\ref{sec-non-verbal-predication}) is used in the \textit{nga}-clause, as in~(\ref{ex-adverbial-nga-neg}) and (\ref{ex-withoutnoun2-lina}). 
\emph{Nga}-clauses with the dedicated negative existential copula \emph{ndoo} can also be used for the expression of caritive semantics, as in~(\ref{ex-withoutnoun2-ndoo}).

\ea \label{ex-withoutnoun2-lina}
\begin{xlist}
\ex \glll   Yaizireyo eyo nga taalina motoka.\\
	        a-a-iz-ire=yo eyo nga ti-a-a-lin-a motoka\\
	    	\textsc{1sbj}-\textsc{pst}-come-\textsc{pfv}=23.\textsc{loc} there \textsc{conj} \textsc{neg}-\textsc{1sbj}-\textsc{pst}-have-\textsc{fv} 9.car\\
	\glt ‘He came there without a car.' (lit.\,`He came there while he did not have a car.')

\ex \label{ex-withoutnoun}
	\glll   Yayabire omu irwaliro nga talina mugendya.\\
	        a-a-ab-ire		omu	irwaliro nga	ti-a-lin-a		mugendya\\ 
	    	\textsc{1sbj}-\textsc{pst}-go-\textsc{pfv} 	18\textsc{.loc}	5.hospital while	\textsc{neg}-1\textsc{sbj}-have-\textsc{fv}	1.caretaker\\
    \glt ‘He went to the hospital without a caretaker.’ (lit.\,‘He went to the hospital while he did not have a caretaker.’) 
\end{xlist}
\z

\ea \label{ex-withoutnoun2-ndoo}
    \glll Baamubbwere nga ndoowo bujuurwa.\\
    ba-a-mu-bbwere nga ndoo=wo bujuurwa\\
	2\textsc{sbj}-\textsc{pst}-1\textsc{obj}-detain:\textsc{pfv}	while	\textsc{neg.ex}=16.\textsc{loc}	14.evidence\\
	\glt ‘He was imprisoned without any evidence.’
\z

\noindent In order to express \textit{without + verb}, the \textit{nga}-clause has the negated form of the respective verb, as in~(\ref{ex-withoutverb}).

\ea \label{ex-withoutverb}
	\glll   Nga taliisirye nkoko, yatambwirembe.\\
	        nga ti-a-liisirye nkoko a-a-tambwire=mbe\\
	    	\textsc{conj} \textsc{neg}-\textsc{1sbj}-feed:\textsc{pfv} 10.chicken \textsc{1sbj}-\textsc{pst}-walk:\textsc{pfv}=\textsc{foc}\\
	\glt ‘He just walked away without feeding the chicken.' (lit.\,`While he had not fed the chicken, he just walked away.')
\z

In contrast to the clause-initial conjunction \emph{nga}, the conjunction \emph{ni} `if, when' occupies the immediately preverbal position. 
Some examples are provided in~(\ref{ex-adverbial-ni}). 
With conditional clauses it seems to be more common than the conjunction \emph{nga}, though further research is needed to identify whether this is a case of free variation or the two conjunctions have a complementary distribution.

\ea \label{ex-adverbial-ni}
\begin{xlist}
\ex \glll   Ni bakukoba afiire, togaana.\\
	        ni ba-ku-kob-a a-fiire  ta-o-gaan-a\\
	    	if \textsc{2sbj}-\textsc{2sg.obj}-say-\textsc{fv} \textsc{1sbj}-die:\textsc{pfv} \textsc{neg}-\textsc{2sg.sbj}-refuse-\textsc{fv}\\
	\glt ‘If you are told, that he has died, don't refuse.'
\ex
	\glll   Oikendi n'	 agutoonyaku gujunda.\\
		o-ikendi ni	a-gu-toony-a=ku gu-jund-a\\ 
	    	\textsc{aug}-1.rain	when \textsc{1sbj}-\textsc{3obj}-rain-\textsc{fv}	\textsc{3sbj}-decompose-\textsc{fv}\\
    \glt `When it rains on it, it decomposes.'
\end{xlist}
\z
