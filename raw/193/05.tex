\chapter{Lower copies and movement chains} \label{ch:5}
\section{Introduction} \label{sec:5introduction}
In \chapref{ch:3}, I argued that \isi{Comparative Deletion} is an epiphenomenon that is primarily related to the \isi{Overtness Requirement} on left-peripheral elements, which states that overt \isi{lexical material} is licensed in an \isi{operator position} only if the \isi{operator} itself is overt. As was shown, there are four logical possibilities, depending on whether the \isi{operator} moves on its own, and whether the \isi{operator} is overt or not. If the \isi{operator} is able to strand a lexical \isi{AP} or \isi{NP} (or there is no lexical XP base-generated together with the \isi{operator} at all), the lexical XP is spelt out in its \isi{base position}, and the \isi{overtness} of the \isi{operator} is immaterial, as is the information structural status of the lexical XP. If an \isi{overt operator} takes the lexical XP along to the [Spec,\isi{CP}] position, the lexical XP is licensed irrespective of its information-structural status. However, if a phonologically \isi{zero operator} takes the lexical XP to the \isi{clausal} \isi{left periphery}, the entire phrase in [Spec,\isi{CP}] has to be deleted in order to avoid a violation of the \isi{Overtness Requirement}. In this case, the lower \isi{copy} of the \isi{movement chain} (in the \isi{base position}) is realised overtly if it is contrastive. This leads to an asymmetry between contrastive and non-contrastive XPs: in the case of the latter, the absence of any overt \isi{copy} results in the surface phenomenon traditionally referred to as \isi{Comparative Deletion}. The realisation of contrastive XPs, on the other hand, appears to be straightforward. Using data mainly from \ili{Slavic}, this chapter will demonstrate that the availability of the lower \isi{copy} for overt realisation is not universal. Further, I will argue that the difference between \ili{English} and \ili{Slavic} in this respect lies chiefly in the availability of multiple \textit{wh}-fronting in \ili{Slavic}. In order to gain a better understanding of multiple \isi{operator movement} and \isi{movement} chains, I will start by reviewing the analysis of \citet{boskovic2002}.

\section{Multiple operator movement -- \citet{boskovic2002}} \label{sec:5multipleoperator}
When there are multiple \textit{wh}-elements in a single clause, different languages show different behaviour with respect to where the individual \textit{wh}-elements appear. As \citet[352--353]{boskovic2002} shows, traditionally four types are assumed: languages where only one \textit{wh}-element is fronted, while the others remain in situ (e.g. \ili{English}); languages where none of the \textit{wh}-elements is fronted, hence \textit{wh}-in-situ languages (e.g. \ili{Chinese}); languages that show both of these options (e.g. \ili{French}); and languages where the fronting of multiple (and in fact all) \textit{wh}-elements occurs (e.g. \ili{Bulgarian}, based on \citealt{rudin1988}). The last option is illustrated by example in (\ref{bulgarian5}) from \ili{Bulgarian} (\citealt[352, ex. 5]{boskovic2002}):

\ea \gll \textbf{Na} \textbf{kogo} \textbf{kakvo} dade Ivan? \label{bulgarian5}\\
to who what gave Ivan\\
\glt `What did Ivan give to who?'
\z

As can be seen, both \textit{na kogo} `to whom' and \textit{kakvo} `what' are fronted. While the traditional assumption is that multiple fronting languages constitute a separate type, \citet{boskovic2002} claims that multiple \textit{wh}-fronting languages fall into the three other types, and they exhibit a special type of \textit{wh}-in-situ elements.

One key argument comes from superiority effects in multiple \textit{wh}-fronting languages, which show a parallel distribution to \textit{wh}-fronting in non-multiple-\textit{wh}-fronting languages (\citealt[353--357]{boskovic2002}). That is, there are languages that always exhibit superiority effects (e.g. \ili{Bulgarian}, hence parallel to \ili{English}), there are languages that exhibit superiority in contexts where \ili{French}-type languages require fronting, but not in others (e.g. \ili{Serbo-Croatian}, hence parallel to \ili{French}); and there are languages that exhibit no superiority effects at all (e.g. \ili{Russian}, thus parallel to \ili{Chinese}); see \citet[355]{boskovic2002}. Consider the examples in (\ref{scl5}) from \ili{Serbo-Croatian} (\citealt[353, exx. 6 and 7]{boskovic2002}):

\largerpage[-2]
\ea \label{scl5}
\ea [] {\gll Ko koga voli? \label{sckokoga}\\
who whom loves\\
\glt `Who loves whom?'}
\ex [] {\gll Koga ko voli? \label{sckogako}\\
whom who loves\\
\glt `Who loves whom?'}
\ex [] {\gll [Ko koga voli], taj o njemu i govori. \label{sclonggramm}\\
\phantom{[}who whom loves that-one about him even talks\\
\glt `Everyone talks about the person they love.'}
\ex [?*] {\gll [Koga ko voli], taj o njemu / o njemu taj i govori. \label{sclongungramm}\\
\phantom{[}whom who loves that-one about him {} about him that-one even talks\\
\glt `Everyone talks about the person they love.'}
\z
\z

As can be seen in (\ref{sckokoga}) and (\ref{sckogako}), there is no superiority effect in short-distance matrix questions with a null C: the order of the fronted subject and object is free. By \isi{contrast}, in the embedded context demonstrated in (\ref{sclonggramm})--(\ref{sclongungramm}), only the configuration where the subject precedes the object is grammatical, as in (\ref{sclonggramm}): the reverse order, given in (\ref{sclongungramm}), is ungrammatical. The same applies to pattern involving long-distance \isi{movement} or an overt question \isi{particle} \textit{li} in C (\citealt[354]{boskovic2002}). \ili{Bulgarian} shows superiority effects in all of these contexts (\citealt[354]{boskovic2002}), while \ili{Russian} demonstrates no superiority effects in any of these contexts (\citealt[354--355]{boskovic2002}, following \citealt{stepanov1998}).

According to \citet[355]{boskovic2002}, superiority effects are attested in the languages under scrutiny whenever \textit{wh}-\isi{movement} is obligatory, that is, whenever there is a strong [+wh] \isi{feature} on C. \ili{Serbo-Croatian} is similar to \ili{French} in that it does not require \textit{wh}-\isi{movement} in all contexts, and thus superiority effects cannot be observed in short-distance matrix questions with a null C. At the same time, all \textit{wh}-phrases undergo fronting in \ili{Serbo-Croatian}, as well as in \ili{Bulgarian} and \ili{Russian}, hence no \textit{wh}-phrase is licensed in situ: \citet[355]{boskovic2002} argues that this is independent from a [+wh] \isi{feature} on C, since that can be checked off only once. Obligatory fronting is illustrated in (\ref{scfronting}) below for \ili{Serbo-Croatian}:

\ea \label{scfronting}
\ea []{\gll Ko \v{s}ta kupuje?\\
who what buys\\
\glt `Who buys what?'}
\ex [?*]{\gll Ko kupuje \v{s}ta?\\
who buys what\\
\glt `Who buys what?'}
\z
\z

The fronting requirement applies to echo questions as well (\citealt[356]{boskovic2012}): this applies not only to \ili{Serbo-Croatian}, \ili{Bulgarian} and \ili{Russian} but also to \ili{Polish} (see \citealt{wachowicz1974}) and \ili{Hungarian} (see \citealt{ekiss1987}). Consider the example in (\ref{ivan5}) from \ili{Serbo-Croatian} (\citealt[356, ex. 16a]{boskovic2012}):

\ea [?*]{\gll Ivan kupuje \v{s}ta? \label{ivan5}\\
Ivan buys what\\
\glt `Ivan buys what?}
\z

Following \citet{stjepanovic1999diss} and the original idea of \citet{horvath1986}, \citet[356--357]{boskovic2012} argues that the driving force underlying this kind of fronting is focus, which can be observed in the case of non-\textit{wh} elements as well. This is illustrated by the examples in (\ref{serbocroatian5}) from \ili{Serbo-Croatian} (\citealt[357, ex. 17]{boskovic2012}):

\ea \label{serbocroatian5}
\ea []{\gll JOVANA savjetuje.\\
Jovan.\textsc{acc} advises\\
\glt `She/He advises Jovan.'}
\ex [?*]{\gll Savjetuje JOVANA.\\
advises Jovan.\textsc{acc}\\
\glt `She/He advises Jovan.'}
\z
\z

Crucially, multiple \textit{wh}-fronting languages demonstrate predictable behaviour with respect to the interpretation of multiple questions (\citealt[357--359]{boskovic2012}). In languages like \ili{English} and \ili{German}, multiple \textit{wh}-questions are compatible with a pair-list answer only (and not with single-pair answers), and the same holds for \ili{Bulgarian} and \ili{Romanian}. \ili{Serbo-Croatian}, \ili{Russian} and \ili{Polish}, however, just like \ili{French}, allow single-pair answers as well.

As noted by \citet[359--379]{boskovic2012}, there are certain exceptions to the obligatoriness of fronting all \textit{wh}-phrases in multiple \textit{wh}-fronting languages. There are three types of exceptions: semantic, phonological, and syntactic.

Regarding semantic exceptions, D-linked \textit{wh}-phrases and particular echo \textit{wh}-phrases are allowed to remain in situ (\citealt[359--364]{boskovic2012}). The exception with a D-linked \textit{wh}-phrase is illustrated in (\ref{scdlink}) below for \ili{Serbo-Croatian} (\citealt[360, ex. 26a]{boskovic2012}):

\ea \gll Ko je kupio koju knijgu? \label{scdlink}\\
who is bought which book\\
\glt `Who bought which book?'
\z

As noted by \citet[359]{boskovic2012}, the phenomenon has been observed in the literature for various languages, such as by \citet{wachowicz1974} for \ili{Polish}, and by \citet{pesetsky1987, pesetsky1989} for \ili{Polish}, \ili{Czech}, \ili{Russian} and \ili{Romanian}. \citet[360]{boskovic2012} argues that this is so because the ``range of reference of D-linked \textit{wh}-phrases is [\ldots] discourse-given'', and they are therefore expected not to undergo focus fronting. The same applies to \textit{wh}-phrases in echo questions, especially with a surprise reading (rather than a clarification reading) because the \textit{wh}-phrase in these cases is known to the speaker as well, hence it represents \textsc{given} information (\citealt[362--364]{boskovic2012}).

Regarding phonological exceptions, \citet[364--376]{boskovic2012} shows that two homophonous \textit{wh}-phrases are not allowed to be fronted at the same time if they are adjacent to each other. Consider the example in (\ref{stauslovljava5}) from \ili{Serbo-Croatian} (\citealt[364, ex. 37]{boskovic2012}):

\ea \gll \v{S}ta uslovljava \v{s}ta? \label{stauslovljava5}\\
what conditions what\\
\glt `What conditions what?'
\z

As can be seen, in this case the second \textit{wh}-element is licensed to appear in situ. The configuration involving adjacent, phonologically identical \textit{wh}-phrases is ruled out only if the two \textit{wh}-words are adjacent: if an \isi{adverb} appears between the two, the structure is licensed (\citealt[364]{boskovic2012}), which indicates that the rule applies at \isi{PF} rather than in syntax. In addition, it cannot be a result of violating a superiority constraint, since the rule applies to any two \textit{wh}-elements, not just the highest one and the one immediately below it, while superiority effects are relevant only for the highest \textit{wh}-element (\citealt[365--367]{boskovic2012}). The ban on identical \textit{wh}-phrases is strictly phonological in nature: for instance, the sequence \textit{kogo na kogo} `whom to whom' is licensed in \ili{Bulgarian}, while *\textit{na kogo kogo} `to whom whom' is not (\citealt[365--367]{boskovic2002}, following \citealt{billingsrudin1996}). At the same time, a \textit{wh}-phrase is licensed in situ only in the second case, indicating that the in-situ option is only a last resort (\citealt[367]{boskovic2002}).

To account for the phenomenon, \citet[367--376]{boskovic2002} argues that while \isi{PF} normally spells out the highest \isi{copy} of a \isi{movement chain}, in these cases a lower \isi{copy} is spelt out. As pointed out by \citet[367--368]{boskovic2002}, the availability of spelling out lower copies at \isi{PF} has been proposed by a number of authors in the literature (for instance by \citealt{bobaljik1995diss}, \citealt{runner1998}, \citealt{pesetsky1997, pesetsky1998}, \citealt{richards1997diss}, \citealt{roberts1997}, and \citealt{nunes1999}), and the idea is essentially similar to spelling out non-trivial copies at \isi{LF}, as argued by \citet{chomsky1995}. In this sense, spelling out the lower \isi{copy} of a \textit{wh}-element in \ili{Slavic} is a last resort option to avoid PF-violation, which would result from the spelling out of two consecutive, phonologically identical \textit{wh}-elements.

Regarding syntactic exceptions, \citet[376--379]{boskovic2002} shows that it is possible for \textit{wh}-phrases to stay in situ when they are extracted out of non-\textit{wh}-islands, as noted by \citet{comorovski1996} for \ili{Romanian}. Consider the examples in (\ref{romanian}) from \ili{Romanian} (\citealt[377, ex. 65]{boskovic2002}):

\ea \label{romanian}
\ea []{\gll Ion a auzit zvonul c\u{a} Petru a cump\u{a}rat CE?\\
Ion has heard the.rumour that Petru has bought what\\
\glt `Ion heard the rumour that Petru has bought what?'}
\ex [*]{\gll Ce a auzit Ion zvonul c\u{a} Petru a cump\u{a}rat?\\
what has heard Ion the.rumour that Petru has bought\\
\glt `Ion heard the rumour that Petru has bought what?'}
\z
\z

As can be seen, in this case the \textit{wh}-element \textit{ce} `what' has to remain in situ. While islands are generally assumed to be syntactic in nature, \citet[377--379]{boskovic2002} argues that islandhood is a \isi{PF} property to some extent, and that realising a \isi{copy} within the island may save the construction from \isi{island violation} which would arise when spelling out the higher \isi{copy} of the same \isi{movement chain}.

 \citet{boskovic2002} argues convincingly that multiple \textit{wh}-fronting languages do not behave in a uniform fashion, but he also shows that exceptions to the \isi{movement} of \textit{wh}-phrases are restricted and can either be explained by the focussed nature of fronted \textit{wh}-elements or by the \isi{copy} theory of \isi{movement}. These exceptions are therefore essentially predictable.

\section{Predicative comparatives in Czech and Polish} \label{sec:5predicative}
As was discussed in \chapref{ch:3} in detail, \ili{English} allows the realisation of contrastive lower copies in \isi{subcomparative} structures, while non-contrastive lower copies must be eliminated. This is demonstrated in (\ref{ralphpeter5}):

\ea \label{ralphpeter5}
\ea [*]{Ralph is taller than Peter is \textbf{tall}.} \label{predcomp}
\ex []{The table is longer than the office is \textbf{wide}.} \label{predcompcontrast}
\z
\z

I argued that the higher \isi{copy} of the quantified expression, landing in the lower [Spec,\isi{CP}] via \isi{operator movement}, is deleted in Standard \ili{English} due to the \isi{Overtness Requirement}, while the lower \isi{copy} is regularly eliminated, unless it is contrastive. Consider the examples in (\ref{english5}):

\ea  \label{english5}
\ea Ralph is taller than \sout{\textbf{x-tall}} Peter is \sout{\textbf{x-tall}}.
\ex The table is longer than \sout{\textbf{x-wide}} the office is \textbf{wide}.
\z
\z

As should be obvious, the \isi{contrast} between (\ref{predcomp}) and (\ref{predcompcontrast}) is dependent on certain factors. First, the \isi{operator} has to be covert: the higher \isi{copy} would be fully realised with an \isi{overt operator} (since it would obey the \isi{Overtness Requirement}), irrespectively of whether the \isi{AP} is contrastive or not (as is the case for non-standard \ili{English} \textit{how}). Second, the \isi{operator} has to be a Deg head and thus not extractable from the \isi{QP}: non-contrastive lower copies are not ruled out in languages where the \isi{zero operator} can be extracted (as was shown for \ili{German}). Third, the realisation of the contrastive lower \isi{copy} must be allowed.

When considering cross-linguistic data, it is obvious that the \ili{English} pattern cannot be universal and that even the third condition is not always met. The following examples show that the lower \isi{copy} cannot be realised in \ili{Czech} (cf. \citealt{bacskaiatkari2015fdsl}):

\ea \label{czechpred}
\ea	[*]{\gll Marie	je vyšší, než	je \textbf{vysoký}	Karel.\\
Mary is	taller than	is tall Charles\\
\glt `Mary is taller than Charles.'}
\ex	[*]{\gll Ten stůl je delší, než je ta kancelář \textbf{široká}. \label{czechcontrast}\\
that desk	is longer	than is	that office	wide\\
\glt `The desk is longer than the office is wide.'}
\z
\z

The question arises why \ili{Czech} does not allow the realisation of the contrastive lower \isi{copy} in (\ref{czechcontrast}). In principle, one may think this is because \ili{Czech} has no \isi{zero operator} at all, and indeed, we saw in \chapref{ch:3} that \ili{Czech} does in fact have an \isi{overt operator}, \textit{jak} `how'. If this is indeed the reason, then \ili{Czech} is essentially similar to \ili{Hungarian} (see \chapref{ch:3}). The relevant examples are repeated in (\ref{czechpattern5}):

\ea \label{czechpattern5}
\ea [??]{\gll Marie	je vyšší,	než	\textbf{jak} \textbf{vysoký} je Karel.\\
Mary	is taller	than	how	tall is Charles\\
\glt `Marie is taller than Charles.'}
\ex	[?]{\gll Marie je	vyšší, než	\textbf{jak} je	\textbf{vysoký}	Karel.\\
Mary	is taller	than how is	tall Charles\\
\glt `Marie is taller than Charles.'}
\ex [??]{\gll Ten stůl je delší, než	\textbf{jak} \textbf{široká} je	ta kancelář.\\
that desk	is longer	than how wide is that	office\\
\glt `The desk is longer than the office is wide.'}
\ex	[]{\gll Ten	stůl je	delší, než \textbf{jak} je ta kancelář \textbf{široká}.\\
that desk	is longer	than how	is that	office wide\\
\glt `The desk is longer than the office is wide.'}
\z
\z

As can be seen, the \isi{operator} \textit{jak} may either appear together with the \isi{AP} in [Spec,\isi{CP}] or the \isi{AP} may be stranded; in either case, it does not make any significant difference whether the \isi{AP} is contrastive or not. This behaviour is expected on the basis of cross-linguistic data for comparatives with overt operators.

However, the fact that \ili{Czech} has an overt \isi{comparative operator} does not actually explain the ungrammaticality of  (\ref{czechpred}). Namely, the same ungrammaticality can be observed in \ili{Polish}:

\ea \label{polishpred}
\ea [*]{\gll Maria jest wyższa niż Karol jest \textbf{wysoki}.\\
Mary is taller than Charles is tall\\
\glt `Mary is taller than Charles.'}
\ex [*/??]{\gll Stół jest dłuższy niż biuro jest \textbf{szerokie}. \label{polishpredcontr}\\
desk is longer than office is wide\\
\glt `The desk is longer than the office is wide.'}
\z
\z

In \isi{contrast} to \ili{Czech}, \ili{Polish} has no overt comparative operators either. A possible candidate would be the \isi{degree} \isi{operator} \textit{jak} `how', which is available in interrogatives (cf. the data of \citealt[81]{borsleyjaworska1981}), as demonstrated in (\ref{polishjak5}):

\ea \label{polishjak5}
\ea []{\gll \textbf{Jak} \textbf{wysoki} jest	Karol?\\
how	tall is	Charles\\
\glt `How tall is Charles?'}
\ex	[*/??]{\gll \textbf{Jak} jest	Karol	\textbf{wysoki}?\\
how	is Charles tall\\
\glt `How tall is Charles?'}
\z
\z

However, \textit{jak} is not available in comparative subclauses (cf. \citealt{bacskaiatkari2015fdsl}):

\ea \label{polishjakcomp5}
\ea [*]{\gll Maria jest wyższa niż \textbf{jak} \textbf{wysoki} jest Karol.\\
Mary is taller than how tall is Charles\\
\glt `Mary is taller than Charles.'}
\ex [*/??]{\gll Stół jest dłuższy niż \textbf{jak} \textbf{szerokie} jest biuro .\\
desk is longer than how wide is office\\
\glt `The desk is longer than the office is wide.'}
\z
\z

The data in (\ref{polishjakcomp5}) clearly show that the reason why both sentences in (\ref{polishpred}) are ungrammatical cannot be the availability of an \isi{overt operator} because \ili{Polish} does not allow the \isi{operator} to be overt at all. This also implies that the \isi{operator} (required by \isi{degree} semantics) has to be zero.

In other words, the problem with \ili{Polish} is essentially the following. First, the ungrammaticality of (\ref{polishpred}) cannot be attributed to the availability of an \isi{overt operator}, hence \ili{Polish} is different from languages like \ili{Hungarian} in this respect. Second, while \ili{Polish} has a \isi{zero operator}, it cannot be extracted on its own and moved to the [Spec,\isi{CP}], as was shown to be the case in \ili{German} and \ili{Dutch} in \chapref{ch:3}, since in that case both sentences in (\ref{polishpred}) should be grammatical. Third, the \isi{zero operator} in \ili{Polish} should then be similar to the \ili{English} one, which is a non-extractable Deg head (see \chapref{ch:3}); however, in that case one would expect the realisation of a contrastive lower \isi{copy} to be possible, which is again not met, since (\ref{polishpredcontr}) is ungrammatical. It seems, then, that the sentences in (\ref{polishpred}) are ungrammatical because even contrastive lower copies of a \isi{movement chain} are not licensed to be realised in comparatives in \ili{Polish}.

Before turning to the issue of why this should be so, let me first review some properties of attributive comparatives in \ili{Czech} and \ili{Polish}. In particular, I will argue that there is a \isi{zero operator} in \ili{Czech} as well, and that the ungrammaticality of (\ref{czechpred}) goes back to the same reasons as that of its \ili{Polish} counterpart, and can be explained in a principled way. In this way, \ili{Czech} will be shown to be similar to \ili{Polish} rather than to \ili{Hungarian}.

\section{Attributive comparatives in Czech and Polish} \label{sec:5attributive}
As was shown in \chapref{ch:4}, based on the analysis given by \citet{kennedymerchant2000}, the \isi{QP} is extractable from the \isi{nominal expression} in \ili{Czech} and \ili{Polish}, and this property is not restricted to the comparative \isi{subclause} but it can be observed in interrogatives as well, where the \isi{QP} is visible. Observe the examples in (\ref{czechinterrogatives5}) from \ili{Czech} (\citealt[104, ex. 30]{kennedymerchant2000}):

\ea \label{czechinterrogatives5}
\ea \gll \textbf{Jak} \textbf{velké} \textbf{auto}	Václav koupil? \label{czechintwhole}\\
how	big	car	Václav bought\\
\glt `How big a car did Václav buy?'
\ex \gll \textbf{Jak} \textbf{velké} Václav	koupil \textbf{auto}?\\
how	big Václav bought	car\\
\glt `How big a car did Václav buy?'
\z
\z

As can be seen, it is possible to move the entire \isi{nominal expression} containing the \isi{QP}, as in (\ref{czechintwhole}), but it is also possible that the \isi{QP} \textit{jak velké} `how big' moves out on its own, and the \isi{noun} is stranded. The same can be observed in \ili{Polish}, as shown by the example in (\ref{howlongpolish5}), taken from \citet[104, ex. 29]{kennedymerchant2000}:

\ea \label{howlongpolish5}
\ea \gll \textbf{Jak} \textbf{d\l{}ug\k{a}} \textbf{sztuk\k{e}} napisa\l{} Pawe\l{}?\\ 
how long play wrote Pawel\\
\glt `How long a play did Pawel write?'
\ex \gll \textbf{Jak} \textbf{d\l{}ug\k{a}} napisa\l{} Pawe\l{} \textbf{sztuk\k{e}}?\\
how long wrote Pawel play\\
\glt `How long a play did Pawel write?'
\z
\z

In comparative subclauses, it is possible to have an overt \isi{lexical verb} and a remnant \isi{NP}, showing that the \isi{QP} has moved out on its own. This is illustrated in (\ref{czechacd}) for \ili{Czech} (\citealt[105, ex. 32b]{kennedymerchant2000}):

\ea \gll Václav	koupil větší auto	než	Tomáš	ztratil	loď. \label{czechacd}\\
Václav bought	bigger car than	Tomáš	lost boat\\
\glt `Václav bought a bigger car than the boat that Tomáš lost.'
\z

The same is true for \ili{Polish}, as shown by (\ref{polishacdrepeat}) below (\citealt[104, ex. 31a]{kennedymerchant2000}):

\ea \gll Jan napisa\l{} d\l{}u\.{z}szy list, ni\.{z} Pawe\l{} napisa\l{} sztuk\k{e}. \label{polishacdrepeat}\\
Jan wrote longer letter than Pawel wrote play\\
\glt `Jan wrote a longer letter than Pawel wrote a play.'
\z

In these cases, the higher \isi{copy} of the \isi{QP} is deleted in a [Spec,\isi{CP}] position due to the \isi{Overtness Requirement}. The remnant \isi{NP} is not affected because it is not a lower \isi{copy} itself, and hence its overt realisation does not require enforcing the pronunciation of a lower \isi{copy}. The point is that there is a \isi{zero operator} in \ili{Czech} and \ili{Polish} that can combine with lexical APs. If so, it is expected that the same \isi{zero operator} can combine with APs if the \isi{AP} is in a \isi{predicative position}, too.

\section{Movement chains} \label{sec:5movement}
Based on what was said above, it seems that in \ili{Czech} and \ili{Polish}, the \isi{zero operator} taking lexical APs is non-extractable, just as in \ili{English}. This predicts that lower copies of non-contrastive APs are unacceptable just as they are in \ili{English}; the relevant examples are repeated in (\ref{givenrepeat}), where (\ref{czechgivenrepeat}) is from \ili{Czech} and (\ref{polishgivenrepeat}) is from \ili{Polish}:

\ea \label{givenrepeat}
\ea	[*]{\gll Marie je vyšší, než	je \textbf{vysoký} Karel. \label{czechgivenrepeat}\\
Mary is	taller than	is tall Charles\\
\glt `Mary is taller than Charles.'}
\ex [*]{\gll Maria jest wyższa niż Karol jest \textbf{wysoki}. \label{polishgivenrepeat}\\
Mary is taller than Charles is tall\\
\glt `Mary is taller than Charles.'}
\z
\z

The higher \isi{copy} of the \isi{QP} is deleted in [Spec,\isi{CP}] due to the \isi{Overtness Requirement}, and the lower \isi{copy} should be eliminated regularly as a lower \isi{copy}; in (\ref{givenrepeat}), the lower \isi{copy} is not F-marked either and there is thus no reason for it to stay overt, just like in the \ili{English} counterpart of the sentences.

However, as was seen earlier, the lower \isi{copy} is not licensed with F-marked APs either in \ili{Czech} and \ili{Polish}, as shown by (\ref{czechcontrastrepeat}) for \ili{Czech} and by (\ref{polishcontrastrepeat}) for \ili{Polish}:

\ea \label{czechpolish}
\ea	[*]{\gll Ten stůl je delší, než je ta kancelář \textbf{široká}. \label{czechcontrastrepeat}\\
that desk	is longer	than is	that office	wide\\
\glt `The desk is longer than the office is wide.'}
\ex [*/??]{\gll Stół jest dłuższy niż biuro jest \textbf{szerokie}. \label{polishcontrastrepeat}\\
desk is longer than office is wide\\
\glt `The desk is longer than the office is wide.'}
\z
\z

I assume that this is because \ili{Czech} and \ili{Polish} generally do not license the realisation of lower copies of a \isi{movement chain}, in line with the analysis given by \citet{boskovic2002}. In order to capture the cross-linguistic differences, consider the abstract representations in (\ref{abstract5}), using \textit{tall} as the \isi{adjective}, \textsc{than} for the comparative \isi{complementiser} and \textsc{how} for an overt \isi{comparative operator} (not to be taken as the \ili{English} \isi{operator}), as well as $\emptyset$ for a \isi{zero operator}:

\ea \label{abstract5}
\ea []{\textsc{than} \textsc{how} tall \ldots{} \sout{\textsc{how} tall}} \label{confighowtall}
\ex []{\textsc{than} \textsc{how} \ldots{} \sout{\textsc{how}} tall} \label{confighowgaptall}
\ex [*]{\textsc{than} $\emptyset$ tall \ldots{} \sout{$\emptyset$ tall}} \label{configzerotall}
\ex []{\textsc{than} \sout{$\emptyset$ tall} \ldots{} \sout{$\emptyset$ tall}} \label{configzerotalllower}
\ex [*]{\textsc{than} \sout{$\emptyset$ tall} \ldots{} $\emptyset$ tall} \label{configzerotallnolower}
\ex []{\textsc{than} $\emptyset$ \ldots{} \sout{$\emptyset$} tall} \label{configzero}
\ex []{\textsc{than} \textsc{how} tall\textsubscript{F} \ldots{} \sout{\textsc{how} tall\textsubscript{F}}} \label{confighowtallcontrast}
\ex []{\textsc{than} \textsc{how} \ldots{} \sout{\textsc{how}} tall\textsubscript{F}} \label{confighowgaptallcontrast}
\ex [*]{\textsc{than} $\emptyset$ tall\textsubscript{F} \ldots{} \sout{$\emptyset$ tall\textsubscript{F}}} \label{configzerotallcontrast}
\ex []{\textsc{than} \sout{$\emptyset$ tall\textsubscript{F}} \ldots{} $\emptyset$ tall\textsubscript{F}} \label{configzerotalllowercontrast}
\ex [*]{\textsc{than} \sout{$\emptyset$ tall\textsubscript{F}} \ldots{} \sout{$\emptyset$ tall\textsubscript{F}}} \label{configzerotallnolowercontrast}
\ex []{\textsc{than} $\emptyset$ \ldots{} \sout{$\emptyset$} tall\textsubscript{F}} \label{configzerocontrast}
\z
\z

In examples (\ref{confighowtall})--(\ref{configzero}), the \isi{adjective} is not contrastive, while in examples (\ref{confighowtallcontrast})--(\ref{configzerocontrast}) it is. In (\ref{confighowtall}) and (\ref{confighowtallcontrast}), the \isi{operator} is overt and it takes the \isi{adjective} to the [Spec,\isi{CP}] position, and both elements remain overt: it does not matter whether the \isi{adjective} is contrastive or not, since the higher \isi{copy} can be regularly spelt out. This can be observed in the case of \ili{English} (substandard) \textit{how} and \ili{Czech} \textit{jak} `how'; note that this option is available for all overt comparative operators taking an \isi{AP}. In (\ref{confighowgaptall}) and (\ref{confighowgaptallcontrast}), the \isi{operator} is overt and it does not take the \isi{adjective} to the [Spec,\isi{CP}]; this option (stranding) is not available for all overt operators, but this can be observed in the case of \ili{Czech} \textit{jak}. Again, it does not matter whether the \isi{AP} is contrastive or not, since it does not take part in \isi{movement} at all.

In (\ref{configzerotall}) and (\ref{configzerotallcontrast}), the \isi{operator} is zero and it takes the lexical \isi{AP} to [Spec,\isi{CP}]: the configuration is illicit because the higher \isi{copy} is not eliminated, even though it violates the \isi{Overtness Requirement}, according to which overt \isi{lexical material} is licensed in an \isi{operator position} only if the \isi{operator} itself is overt. The information structural status of the \isi{AP} is irrelevant in this respect. If the \isi{operator} can be extracted on its own, as in (\ref{configzero}) and (\ref{configzerocontrast}), the \isi{Overtness Requirement} is satisfied since no \isi{AP} moves to the [Spec,\isi{CP}] at all, and the \isi{AP} can be realised in its \isi{base position}, irrespective of whether it is contrastive or not: this can be observed in \ili{German} and \ili{Dutch} with the zero comparative operators.

Nevertheless, an \isi{AP} moving to [Spec,\isi{CP}] with a \isi{zero operator} does not mean that the structure is ruled out: the grammatical possibilities are dependent on certain properties of \isi{movement} chains. In (\ref{configzerotalllower}), the lower \isi{copy} of the \isi{AP} is realised: the configuration is ruled out because a lower \isi{copy} of a \isi{movement chain} is licensed only under special circumstances, and since the \isi{AP} is not contrastive, there is no reason to enforce the realisation of its lower \isi{copy}. Hence, regular \isi{deletion} should take place, as in (\ref{configzerotallnolower}), which is a grammatical configuration, as is known from Standard \ili{English}. The ungrammaticality of (\ref{configzerotalllower}) and the grammaticality of (\ref{configzerotallnolower}) are not language-specific, though: they follow from universal principles of grammar. Similarly, the configuration in (\ref{configzerotallnolowercontrast}) is ruled out universally as in this case both copies of a contrastive element are deleted, and the \isi{AP} is not recoverable. However, the configuration in (\ref{configzerotalllowercontrast}) is subject to cross-linguistic variation: in this case, the higher \isi{copy} is regularly eliminated by the \isi{Overtness Requirement}, and the lower \isi{copy} is realised because the \isi{AP} is F-marked. For this, it is necessary for the language to allow the realisation of lower copies of a \isi{movement chain}, in case the pronunciation of the higher \isi{copy} would cause a violation at \isi{PF}. This is possible in \ili{English}, but not in \ili{Czech} and \ili{Polish}.

Recall from \sectref{sec:5multipleoperator} that the realisation of lower copies is very restricted in \ili{Slavic}, as shown by \citet{boskovic2002}. There are three major kinds of exceptions. First, D-linked \textit{wh}-phrases may be realised in situ because the range of reference is actually discourse-given, and the phrase is not expected to undergo focus fronting. This obviously does not apply to (\ref{czechpolish}), where the APs are not discourse-given, as opposed to D-linked nominal expressions (such as \textit{which book}), where the \isi{NP} is taken to be discourse-given. Moreover, as far as D-linked \textit{wh}-phrases are concerned, \citet{boskovic2002} argues that they do not undergo \isi{movement}: this is possible because the [wh] \isi{feature} on C has already been checked by the first \textit{wh}-constituent moving there, and all other \textit{wh}-phrases are subject to some kind of focus fronting. In the comparative \isi{subclause}, however, the [rel] \isi{feature} on the C head can be checked off only if the only \isi{relative operator} of the clause moves there, which is the \isi{comparative operator} itself, and as this cannot be separated from the \isi{AP}, the \isi{AP} takes part in \isi{movement} as well. Apart from D-linked \textit{wh}-phrases, certain echo \textit{wh}-elements may also remain in situ: again, this is not applicable to the case of comparatives, and comparative \isi{operator movement} is not compatible with the assumption of the \isi{operator} element remaining in situ.

Second, the lower \isi{copy} of certain \textit{wh}-phrases may be realised if the \textit{wh}-element is phonologically adjacent to another fronted \textit{wh}-element that is immediately adjacent to it. This clearly cannot be the reason for the ungrammaticality of (\ref{czechpolish}), as there is no second \isi{operator} element in the clause and the \isi{operator} is not even overt in the first place.

Third, the realisation of the lower \isi{copy} is possible if the pronunciation of the higher \isi{copy} would be an instance of \isi{island violation} (non-\textit{wh}-islands). This is again not the case in (\ref{czechpolish}), where the \isi{operator} + \isi{AP} combination moves regularly to [Spec,\isi{CP}] from a \isi{predicative position}, and there is no island at all, hence no \isi{island violation} can occur either.

\largerpage[1]
In sum, the ungrammaticality of (\ref{czechpolish}) lies in the unavailability of the lower \isi{copy} of the \isi{QP} in \ili{Czech} and \ili{Polish}, as the conditions under which lower copies can be realised are not satisfied here. Note also that the contrastive \isi{AP} in (\ref{czechpolish}) in its \isi{base position} is ungrammatical as a lower \isi{copy} but there is nothing ruling out the realisation of contrastive phrases here: on the contrary, as was shown in \chapref{ch:3} in connection with \ili{Czech}, this is precisely the position where contrastive elements are preferably located in the clause in these languages (see also \citealt{simikwierzba2012}). 

\section{More on cross-linguistic differences} \label{sec:5moreon}
So far, we have seen that there is evidently an important connection between multiple \textit{wh}-fronting languages and the availability of predicative subcomparatives. Namely, if a multiple \textit{wh}-fronting language has a zero Deg \isi{operator}, then the lower \isi{copy} of the entire \isi{degree expression} cannot be realised even if the \isi{AP} is contrastive, and predicative subcomparatives are thus not derivable with the operators in question. However, this does not imply that all multiple \textit{wh}-fronting languages lack predicative subcomparatives: it is predicted that these structures will be absent if the \isi{operator} is a Deg head and it is zero, but in all other cases the \isi{AP} either appears in [Spec,\isi{CP}] or does not move at all, and thus the question of realising a lower \isi{copy} does not arise in the first place.

We have already seen that \ili{Czech} allows the \isi{overt operator} \textit{jak} `how', which is extractable. But even \ili{Slavic} languages may allow an \isi{extractable operator} (as in \ili{German} and \ili{Dutch}); consider the examples in (\ref{scpattern5}) from \ili{Serbo-Croatian}:\footnote{I owe many thanks to Boban Arsenijević for his help with the \ili{Serbo-Croatian} data.}

\ea \label{scpattern5}
\ea \gll Pavao je viši nego (što) je \textbf{visok} Petar. \label{scgiveninternal}\\
Paul is taller than \phantom{(}what is tall Peter\\
\glt `Paul is taller than Peter.'
\ex \gll Pavao je viši nego (što) je Petar \textbf{visok}. \label{scgivenfinal}\\
Paul is taller than \phantom{(}what is Peter tall\\
\glt `Paul is taller than Peter.'
\ex \gll Sto je duži nego (što) je \textbf{visok} ured. \label{sccontrastinternal}\\
table is longer than \phantom{(}what is wide office\\
\glt `The table is longer than the office is wide.'
\ex \gll Sto je duži nego (što) je ured \textbf{visok}. \label{sccontrastfinal}\\
table is longer than \phantom{(}what is office wide\\
\glt `The table is longer than the office is wide.'
\z
\z

Note that individual speakers may differ regarding their judgements and preferences concerning the presence/absence of \textit{što} `what'; since this kind of variation is not immediately relevant to our present discussion, I will not investigate the issue here. Suffice it to say that \textit{što} is a lower C head, similarly to \ili{English} \textit{what}, as discussed in \chapref{ch:3} (see \citealt{bacskaiatkari2016alh} on the role of lower complementisers and the status of \textit{što}).

The point is that \ili{Serbo-Croatian} allows the realisation of the \isi{AP} in the \isi{subclause}, not only when it is contrastive, as in (\ref{sccontrastinternal}) and (\ref{sccontrastfinal}), but also when it is not, as in (\ref{scgiveninternal}) and (\ref{scgivenfinal}). The \isi{AP} can appear either clause-finally or clause-internally in both constellations, there being no information-structural constraints on its preferred position. The possibility of (\ref{sccontrastinternal}) and (\ref{sccontrastfinal}) contrasts with the data from \ili{Czech} and \ili{Polish}, while all the three languages are multiple \textit{wh}-fronting languages. However, the grammaticality of (\ref{scgiveninternal}) and (\ref{scgivenfinal}) indicates that the \ili{Serbo-Croatian} zero \isi{comparative operator} differs from the ones in \ili{Czech} and \ili{Polish}: it is a \isi{QP} \isi{modifier}, which can be extracted on its own, just like in \ili{German} and \ili{Dutch}, whereas the \isi{zero operator} in \ili{Czech} and \ili{Polish} is a Deg head, just like in \ili{English}. Hence, the \isi{AP} in (\ref{sccontrastinternal}) and (\ref{sccontrastfinal}) is not the realisation of a contrastive lower \isi{copy} but a stranded \isi{AP}, just like (\ref{scgiveninternal}) and (\ref{scgivenfinal}) contain a stranded \isi{AP}, too. It can thus be concluded that the ban on realising lower copies of a \isi{movement chain} is relevant in the derivation of comparatives only if the \isi{operator} is a zero Deg head but not otherwise.

It seems that the Standard \ili{English} pattern is highly unusual: while \ili{Czech} and \ili{Polish} also have a zero, non-extractable \isi{comparative operator}, they do not allow the realisation of contrastive lower copies either, and thus they do not show the asymmetric pattern attested in \ili{English}. However, \ili{English} is not completely unique: \ili{Norwegian} shows the same asymmetry. Consider the examples in (\ref{norwegian5}):\footnote{The \ili{Norwegian} data stem from the cross-Germanic survey I conducted as part of my project ``The syntax of functional left peripheries and its relation to \isi{information structure}'' in 2016/2017. Both informants marked (\ref{norwegiancd}) as ungrammatical; (\ref{norwegiansubcomp}) was marked with two question marks by my informant from Rogaland county, while my informant from Vest-Agder county marked it as perfectly grammatical. The \isi{markedness} of (\ref{norwegiansubcomp}) can be attributed to the fact that subcomparatives are generally far more difficult to parse than ordinary comparatives, since they involve more than a single dimension of comparison, rather than to true dialectal differences.}

\ea \label{norwegian5}
\ea [*]{\gll Mary er eldre enn Peter er \textbf{gamal}. \label{norwegiancd}\\
Mary is older than Peter is old\\
\glt `Mary is older than Peter.'}
\ex [?]{\gll Katten er feitere enn kattedøra er \textbf{vid}. \label{norwegiansubcomp}\\
the.cat is fatter than the.cat.flap is wide\\
\glt `The cat is fatter than the cat flap is wide.'}
\z
\z

The same applies to \ili{Icelandic}, as shown by (\ref{icelandic5}):\footnote{The \ili{Icelandic} data stem from the cross-Germanic survey mentioned above. My two informants, one from Reykjavík and the other from Austurland (Eastern Region), had the same judgements.}

\ea \label{icelandic5}
\ea [*]{\gll María er eldri en það sem Pétur er \textbf{gamall}.\\
Mary is older than what that Peter is old\\
\glt `Mary is older than Peter.'}
\ex []{\gll Kötturinn er feita en kattahurðin er \textbf{breið}.\\
the.cat is fatter than cat.flap is wide\\
\glt `The cat is fatter than the cat flap is wide.'}
\z
\z

Let me sum up the cross-linguistic differences in predicative comparatives, based on the findings presented in \chapref{ch:3} and \chapref{ch:5}. There are three major factors determining the overt realisation of the quantified expression: whether the \isi{operator} is overt, whether it is extractable, and whether lower copies of a \isi{movement chain} can be realised if the pronunciation of the higher \isi{copy} would cause the derivation to crash at \isi{PF}. The possibilities are summarised in (\ref{treeoperators5}):

%\newpage
\ea \upshape \label{treeoperators5}
\scalebox{0.8}{
\begin{forest} baseline, qtree, for tree={align=center} 
[\textbf{\isi{operator} overt?}
	[YES
		[\textbf{\isi{operator} extractable?}
			[YES [\ili{Czech}\\(\textit{jak})] [\ili{Hungarian}\\(\textit{amennyire})]]
			[NO [\ili{English}\\(\textit{how})] [\ili{Dutch}\\(\textit{hoe})] [\ili{Hungarian}\\(\textit{amilyen})]]
		]
	]
	[NO, name=no [{} [{} [{} [{}
		[\textbf{\isi{operator} extractable?}, name=op
			[YES [\ili{German}\\($\emptyset$)] [\ili{Dutch}\\($\emptyset$)] [\ili{Estonian}\\($\emptyset$)] [\ili{Serbo-Croatian}\\($\emptyset$)]]
			[NO, name=no2 [{} [{} [\textbf{lower copies available?}, name=op2 [YES [\ili{English}\\($\emptyset$)] [\ili{Norwegian}\\($\emptyset$)] [\ili{Icelandic}\\($\emptyset$)]] [NO [\ili{Czech}\\($\emptyset$)] [\ili{Polish}\\($\emptyset$)]]]]]]
		]
	]]]]]
]
\path [draw] (no.south) -- (op.north);
\path [draw] (no2.south) -- (op2.north);
\end{forest}}
\z

As can be seen, the first question is whether the \isi{operator} is overt or not. This determines whether the information-structural properties of the \isi{AP} taken by the \isi{operator} matter inasmuch as, with overt operators, the pattern is essentially symmetric and both types of APs are available, while there is variation if the \isi{operator} is covert. With overt operators, the \isi{AP} can (and sometimes must) be realised in [Spec,\isi{CP}] together with the \isi{operator}, while this option is excluded with covert operators, which may only allow the realisation of the \isi{AP} in its \isi{base position}.

If the \isi{operator} is overt, the next question is whether it is extractable. This decides on the possible positions of the \isi{AP} in the \isi{subclause}, that is, whether it is restricted to appear in the [Spec,\isi{CP}] with the given \isi{operator} or whether it may be stranded in a lower position, while the \isi{operator} still has to move to [Spec,\isi{CP}]. If the \isi{operator} is extractable, the \isi{AP} can move up together with the \isi{operator} or it may be stranded. If it moves up to [Spec,\isi{CP}], its information-structural properties are not relevant, and both contrastive and non-contrastive APs are licensed here equally. If the \isi{AP} is stranded, its preferred position in the clause largely depends on the information-structural requirements of the given language, and contrastive and non-contrastive APs may differ in terms of their preferred positions. This can be observed in \ili{Czech} and \ili{Hungarian} (see \chapref{ch:3}).

If the \isi{operator} is overt and not extractable, the \isi{AP} always moves up with the \isi{operator} to [Spec,\isi{CP}], and the information structural status of the \isi{AP} is not relevant. This was observed in \ili{English}, \ili{Dutch} and \ili{Hungarian} (see \chapref{ch:3}).

If the \isi{operator} is not overt, then the next question is again whether it is extractable or not. Contrary to what we saw in the case of overt operators, this question here not only decides on the possible positions of the \isi{AP} in the \isi{subclause} but it crucially decides whether non-contrastive APs can be realised or not: namely, if the \isi{AP} moves up to [Spec,\isi{CP}] together with the \isi{covert operator}, it must be eliminated because it violates the \isi{Overtness Requirement}. If the \isi{covert operator} is extractable, the \isi{AP} may be stranded, irrespective of its information-structural status, and so even non-contrastive APs can be realised overtly. This can be observed in \ili{German}, \ili{Dutch} and \ili{Estonian} (see \chapref{ch:3}).

If the \isi{operator} is not overt and is not extractable either, then the next question is whether the language allows the realisation of lower copies of a \isi{movement chain} in cases where the pronunciation of the higher \isi{copy} would cause the derivation to crash at \isi{PF}. This decides whether contrastive APs can be realised (and hence whether subcomparatives are possible): the higher \isi{copy} is deleted in any case due to the \isi{Overtness Requirement} and non-contrastive lower copies are regularly eliminated as lower copies of a \isi{movement chain}, thus the only question is whether contrastive lower copies can overwrite the general rule of deleting lower copies. This is possible in \ili{English}, resulting in the classical ``\isi{Comparative Deletion}'' pattern with non-contrastive APs, as opposed to subcomparatives with contrastive APs. However, the realisation of lower copies is not possible in languages like \ili{Czech} and \ili{Polish}, and subcomparatives are thus not derivable: a contrastive lower \isi{copy} cannot be realised, yet the complete elimination of non-given elements is universally prohibited.

The importance of this is that the \ili{English} pattern, where \isi{Comparative Deletion} refers to the obligatory elimination of a non-contrastive \isi{AP} from the comparative \isi{subclause}, is not universal: in fact, it is highly language-specific, and it can only be regarded as a result of several factors. Thus, \isi{Comparative Deletion} cannot be regarded as a universal phenomenon or a parameter either, and the analysis of the particular \ili{English} pattern cannot be solely based on Standard \ili{English} data but must take other languages and non-standard varieties into consideration.

