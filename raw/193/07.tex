\chapter{Conclusion} \label{ch:7}
The aim of this book was to provide an analysis for the syntactic structure of comparatives, with special attention paid to the derivation of the \isi{subclause}. Naturally, the analysis given here is not broad enough to cover all issues connected to comparatives; still, the ones that have been dealt with are of crucial importance and the proposed account explains how the comparative \isi{subclause} is connected to the \isi{matrix clause}, how the \isi{subclause} is formed in the syntax and what additional processes contribute to its final structure. In addition, the main interest of my research was to cast light upon these problems in cross-linguistic terms and to provide a model that allows for variation. This also enables one to give a more adequate explanation for the phenomena found in \ili{English} comparatives, since the properties of \ili{English} structures can then be linked to general settings of the language, and hence need no longer be considered as idiosyncratic features of the grammar of \ili{English}.

In \chapref{ch:2}, I provided a unified analysis of \isi{degree} expressions, with the aim of relating the structure of comparatives to that of other (the absolute and the superlative) degrees. Building on results of previous analyses such as \citet{bresnan1973}, \citet{izvorski1995}, \citet{corver1997} and \citet{lechner1999diss, lechner2004}, I proposed a feature-based account that may explain various differences both with respect to the \isi{degree morpheme} and the lexical \isi{adjective} itself, either in \ili{English} or cross-linguistically. It was shown that gradable adjectives are located within a \isi{degree} phrase (DegP), which in turn projects a \isi{quantifier} phrase (QP), and that these two functional layers are always present for gradable adjectives, irrespective of whether there is a phonologically visible element in these layers. The difference can be captured by considering (\ref{talltaller}):

\ea \label{talltaller}
\ea	Mary is \textbf{tall}. \label{tallconclusion}
\ex	Mary is \textbf{taller} than John. \label{taller}
\z
\z

While in (\ref{tallconclusion}) only a bare \isi{adjective} (\textit{tall}) is visible, in (\ref{taller}) the comparative \isi{degree morpheme} (-\textit{er}) and the comparative \isi{subclause} (\textit{than John}) are also overt. Nevertheless, building on \isi{degree} semantics, I argued that the \isi{DegP} and the \isi{QP} are necessary also in the case of (\ref{tallconclusion}) since the \isi{degree} interpretation has to be present syntactically as well; in addition, modifiers also provide arguments for the existence of the \isi{QP} layer. One of the strongest arguments comes from structures like (\ref{periphrasticconclusion}):

\ea Mary is \textbf{more intelligent} than John. \label{periphrasticconclusion}
\z

In this case, the \isi{degree morpheme} -\textit{er} appears as part of \textit{more} and not as a suffix on the lexical \isi{adjective} itself; as was shown, \textit{more} is in fact a composite of the Q head \textit{much} and the Deg head -\textit{er}. The proposed structure is given in (\ref{treeqp7}):

\ea \upshape \label{treeqp7}
\begin{forest} baseline, qtree, for tree={align=center}
[QP
	[QP [(far),roof]]
	[Q$'$
		[Q
			[-er\textsubscript{i} + much,name=more]
		]
		[DegP
			[AP [intelligent,roof]]
			[Deg$'$ [Deg [t\textsubscript{i},name=trace]] [CP [than John,roof]]]
		]
	]
]
\draw[->] (trace) to[out=south west,in=south,looseness=1.5] (more);
\end{forest}
\z

As can be seen, the lexical \isi{AP} and the XP expressing the standard value (here the \isi{CP} \textit{than John}) are both arguments of the \isi{degree head}, in line with \citet{lechner1999diss, lechner2004}. In addition, there is a \isi{QP} layer projected on top of the \isi{DegP}, such that the Deg head moves up to the Q head; the specifier of the \isi{QP} may in turn host other (QP) modifiers. The Deg head is zero in (\ref{tallconclusion}) and is filled by -\textit{er} in (\ref{taller}) and in (\ref{periphrasticconclusion}). The \isi{movement} of the Deg head up to the Q head accounts for the formation of \textit{more}; in all the other cases, the morpheme -\textit{er} is attached to the lexical \isi{adjective} following it at \isi{PF}.

\largerpage[-1]
Since the comparative \isi{subclause} is the complement of the \isi{degree head}, the Deg head can impose selectional restrictions on it, which explains the difference between the complement of the Deg head -\textit{er} and that of \textit{as}. Relevant examples are given in (\ref{erasdiff7}):

\ea \label{erasdiff7}
\ea []{Mary is \textbf{taller} [than John].}
\ex	[*]{Mary is \textbf{taller} [as John].}
\ex	[*]{Mary is \textbf{as tall} [than John].}
\ex	[]{Mary is \textbf{as tall} [as John].}
\z
\z

Though all of these restrictions are associated with the Deg head rather than with the \isi{adjective}, I also considered cases where the \isi{adjective} has arguments of its own, as in (\ref{proudofpp}):

\ea	Mary is proud [of her husband]. \label{proudofpp}
\z

In (\ref{proudofpp}), the \isi{adjective} \textit{proud} takes the \isi{PP} as its complement, and thus the \isi{PP} is base-generated as the complement of the A head. Problems arise when such complements appear in comparative structures, as in (\ref{prouderof7}):

\ea	Mary is prouder [of her husband] than Susan is. \label{prouderof7}
\z

Since the \isi{PP} appears after the \isi{degree head}, it is obvious that it undergoes \isi{extraposition} of some sort. I argued that this \isi{extraposition} is not syntactic in nature but follows from the fact that the \isi{PP} can be spelt out on its own as a phase (just like the comparative \isi{subclause}) and hence appears in the \isi{PF} string later than the \isi{adjective}.

The analysis can account for differences between gradable and non-gradable adjectives; in addition, it was shown that the distinction between predicative and attributive adjectives can also be captured in that predicative-only adjectives are equipped with a [--nom] \isi{feature} while attributive-only adjectives are equipped with a [+nom] \isi{feature}, all other adjectives allowing both options. By way of \isi{agreement} with the \isi{degree head}, this \isi{feature} percolates up to the entire \isi{QP} and defines whether it can, may or must agree with a \isi{nominal expression}. On the other hand, certain Deg heads may also be inherently marked as either [+nom] or [--nom], which accounts for why superlatives are invariably attributive.

In \chapref{ch:3}, I presented a novel analysis of \isi{Comparative Deletion} by reducing it to an \isi{overtness} constraint holding on operators. In this way, \isi{Comparative Deletion} can be reduced to morphological differences, and cross-linguistic variation is not conditioned by way of postulating an arbitrary parameter that defines whether a certain language has \isi{Comparative Deletion} or not. This account is strongly feature-based in the sense that differences are ultimately dependent on whether a certain language has overt operators equipped with the relevant -- [+compr] and [+rel] -- features.

As was seen, the phenomenon of \isi{Comparative Deletion} traditionally denotes the absence of an adjectival or \isi{nominal expression} from the comparative \isi{subclause}:

\ea \label{cdconclusion}
\ea	Ralph is more qualified than Jason is \sout{\textbf{x-qualified}}. \label{xqualifiedconclusion}
\ex	Ralph has more qualifications than Jason has \sout{\textbf{x-many qualifications}}. \label{xmanyqualificationsconclusion}
\ex	Ralph has better qualifications than Jason has \sout{\textbf{x-good qualifications}}. \label{xgoodqualificationsconclusion}
\z
\z
	
In all of the examples above in (\ref{cdconclusion}), \textit{x} denotes a certain \isi{degree} or quantity to which a certain entity is qualified, good etc. As far as Standard \ili{English} is concerned, this is an \isi{operator} with no phonological content. Earlier analyses of \isi{Comparative Deletion} simply acknowledged that in predicative comparatives such as (\ref{xqualifiedconclusion}) an \isi{adjectival expression} is deleted. By \isi{contrast}, in nominal comparatives such as (\ref{xmanyqualificationsconclusion}) and in attributive comparatives such as (\ref{xgoodqualificationsconclusion}) a \isi{nominal expression} is deleted.

I rejected the possibility of \isi{Comparative Deletion} taking place at the base-generation site and therefore the representations in (\ref{cdconclusion}) are only descriptively adequate. One of the greatest problems regarding the claim that \isi{Comparative Deletion} takes place at the base-generation site is that it should target different constituents obligatorily, since the overt presence of the quantified expressions in (\ref{cdconclusion}) would lead to ungrammatical constructions. I argued that such an operation could not be conditioned and that \isi{Comparative Deletion} must be the result of more general processes.

Another problem concerning \isi{Comparative Deletion} and the \isi{deletion} site concerns information structural properties. In \isi{subcomparative} structures, an adjectival or nominal element may be left overt in the \isi{subclause}; contrary to the examples in (\ref{cdconclusion}), these elements are not logically identical to an antecedent in the \isi{matrix clause}:

\ea \label{subcompconclusion}
\ea	The table is longer than the desk is \textbf{wide}.
\ex	Ralph has more books than Jason has \textbf{manuscripts}.
\ex	Ralph wrote a longer book than Jason did \textbf{a manuscript}.
\z
\z

One of the central questions often discussed in the relevant literature is whether constructions like the ones in (\ref{subcompconclusion}) are exempt from \isi{Comparative Deletion} and are hence essentially different from the ones in (\ref{cdconclusion}), or whether \isi{Comparative Deletion} applies in both types.

I argued that \isi{Comparative Deletion} takes place at the \isi{left periphery} in the \isi{subclause} in a [Spec,\isi{CP}] position in all cases given in (\ref{cdconclusion}) and (\ref{subcompconclusion}) due to an \isi{Overtness Requirement} that requires the presence of an \isi{overt operator} if there is \isi{lexical material} (an \isi{AP} or an \isi{NP}) located in an \isi{operator position}. Since Standard \ili{English} has no overt operators, the \isi{deletion} of the higher \isi{copy} always takes place in [Spec,\isi{CP}]. As was shown, the lower \isi{copy} may then be realised overtly, but this happens only if it is contrastive: this condition is satisfied in (\ref{subcompconclusion}) but not in (\ref{cdconclusion}), and lower copies are therefore not pronounced in cases like (\ref{cdconclusion}).

Given that \isi{deletion} takes place in a [Spec,\isi{CP}] position if the \isi{Overtness Requirement} is not satisfied, it is not surprising that a visible \isi{operator} can appear in this position, which is possible for certain dialects of \ili{English} that accept, for instance, \textit{how} as a \isi{comparative operator}:

\ea	\% Ralph is more qualified than \textbf{how qualified} Jason is. \label{howqualifiedconclusion}
\z

As I argued, structures like (\ref{howqualifiedconclusion}) involve \isi{operator movement} in the same way the ones in (\ref{cdconclusion}) and (\ref{subcompconclusion}) do; the difference is that \textit{how} can appear overtly in the [Spec,\isi{CP}] position because it does not violate the \isi{Overtness Requirement}.

In addition to instances like (\ref{howqualifiedconclusion}), \chapref{ch:3} also showed that there are languages and language varieties that allow the \isi{degree element} to be combined with a lexical \isi{AP}/\isi{NP} to appear overtly in the [Spec,\isi{CP}] position, as is the case in \ili{Hungarian} (cf. \citealt{kenesei1992}):

\ea \label{hungfull}
\ea \gll Mari magasabb, mint \textbf{amilyen} \textbf{magas} P\'eter. \label{hungpredfullconclusion}\\
Mary taller than how tall Peter\\
\glt `Mary is taller than Peter.'
\ex \gll Marinak t\"{o}bb macsk\'aja van, mint \textbf{ah\'any} \textbf{macsk\'aja} P\'eternek van.\\
Mary.\textsc{dat} more cat.\textsc{poss.3sg} is than how.many cat.\textsc{poss.3sg} Peter.\textsc{dat} is\\
\glt `Mary has more cats than Peter has.'
\ex \gll Marinak nagyobb macsk\'aja van, mint \textbf{amilyen} \textbf{nagy} \textbf{macsk\'aja} P\'eternek van.\\
Mary.\textsc{dat} bigger cat.\textsc{poss.3sg} is than how big	cat.\textsc{poss.3sg} Peter.\textsc{dat} is\\
\glt `Mary has a bigger cat than Peter has.'
\z
\z

As was seen, \ili{Hungarian} allows the overt presence of the \isi{degree} elements; again, this was shown to be so because the \isi{Overtness Requirement} is satisfied in cases like (\ref{hungfull}). Since the \isi{Overtness Requirement} is not specifically related to comparatives, the parametric variation attested across languages can also be linked to more general properties instead of treating \isi{Comparative Deletion} as a parameter.

Strongly related to the status of operators, \chapref{ch:3} also examined the question of how the internal structure of \isi{degree} expressions plays a role in the different behaviour of individual comparative operators. In \ili{Hungarian}, for instance, the \isi{operator} \textit{amilyen} `how' may appear together with the \isi{adjective}, as in (\ref{hungpredfullconclusion}), though the \isi{adjective} may not be stranded, as illustrated in (\ref{nostrand7}):

\ea [*]{\gll Mari	magasabb,	mint \textbf{amilyen} Péter	\textbf{magas}. \label{nostrand7}\\
Mary taller	than how	Peter	tall\\
\glt `Mary is taller than Peter.'}
\z

\chapref{ch:3} argued that the reason behind this is that the \isi{operator} \textit{amilyen} is a Deg head and as such it cannot be extracted from the \isi{degree expression} that it is the head of. Adopting the general structure for \isi{degree} expressions given in \chapref{ch:2}, I made the claim that there are two possible \isi{operator} positions:

\ea \label{treeopconclusion} \upshape 
\begin{forest} baseline, qtree, for tree={align=center}
[QP
	[QP
		[Op.,roof]
	]
	[Q$'$
		[Q
			[Op.\textsubscript{i},name=er]
		]
		[DegP
			[AP [\phantom{clever},roof]]
			[Deg$'$ [Deg [t\textsubscript{i},name=trace1]] [G]]
		]
	]
]
\draw[->] (trace1) to[out=south west,in=south,looseness=1.5] (er);
\end{forest}
\z

As can be seen, one \isi{operator position} is the Deg head, and operators of this type ultimately undergo \isi{movement} to the Q head position. These heads then cannot be extracted from within the entire \isi{QP} projection, and the lexical \isi{AP} (if they take any) necessarily moves together with them. Some of these operators were also shown to be able to act as proforms, hence standing for the \isi{DegP} without a visible lexical \isi{AP} there. On the other hand, there are operators that are \isi{QP} modifiers located in the [Spec,\isi{QP}] position: these cannot be proforms but since they are phrase-sized, they are able to move out on their own, at least if the entire \isi{QP} functions as a \isi{predicate} in the clause. This can be observed in the case of the \ili{Hungarian} \isi{operator} \textit{amennyire} `how much', illustrated in (\ref{amennyire7}):

\ea \label{amennyire7}
\ea \gll Mari	magasabb,	mint \textbf{amennyire} \textbf{magas} Péter.\\
Mary taller	than how.much	tall Peter\\
\glt `Mary is taller than Peter.'
\ex \gll Mari	magasabb,	mint \textbf{amennyire} Péter \textbf{magas}.\\
Mary taller	than	how.much Peter tall\\
\glt `Mary is taller than Peter.'
\z
\z

The extractability of operators is thus responsible for whether the \isi{AP} may be stranded or not; in other words, extractability is not directly linked to \isi{Comparative Deletion}, which is ultimately a surface reflex of the \isi{Overtness Requirement} that holds for copies in a [Spec,\isi{CP}] position, but it depends on the position of the \isi{operator} in the functionally extended \isi{degree expression}.

As far as \ili{Hungarian} is concerned, \chapref{ch:3} also showed that if the \isi{adjective} is overt, the \isi{operator} has to be overt as well; this is due to the fact that \ili{Hungarian} does not have zero comparative operators, as shown by (\ref{nozero7}):

\ea \label{nozero7}
\ea \gll Mari magasabb, mint \textbf{(*magas)} P\'eter.\\
Mary taller than \phantom{\textbf{(*}}tall Peter\\
\glt `Mary is taller than Peter.'
\ex \gll Mari magasabb, mint P\'eter \textbf{(*magas)}.\\
Mary taller than Peter \phantom{\textbf{(*}}tall\\
\glt `Mary is taller than Peter.'
\z
\z

My analysis of \isi{Comparative Deletion} takes into account that languages differ with respect to the presence/absence of the \isi{operator} in a more intricate way than one that could be formulated on a +/-- basis. The factors responsible for cross-linguistic variation are related to the internal structure of \isi{degree} expressions, the \isi{overtness} of \isi{degree} operators and also to information structural properties. However, \isi{Comparative Deletion} is not a direct reflex of the information structural status of lexical projections associated with the \isi{degree} elements but it is a factor that plays a role as far as the realisation of lower copies in a \isi{movement chain} is concerned and may also be linked to the preferred position of a lexical \isi{AP} in the comparative \isi{subclause} if the \isi{AP} can be stranded.

\chapref{ch:4} aimed at providing an adequate explanation for the phenomenon of \isi{Attributive Comparative Deletion}, as attested in \ili{English}, by way of relating it to the regular mechanisms underlying \isi{Comparative Deletion}, as described in \chapref{ch:3}. I showed that \isi{Attributive Comparative Deletion} can only be understood as a descriptive term referring to a phenomenon that is a result of the interaction of more general syntactic processes, since there is no reason to postulate any special mechanism underlying it in the grammar. The elimination of such a mechanism allows one to achieve a unified analysis of all types of comparatives. In addition, \chapref{ch:4} argued that \isi{Attributive Comparative Deletion} is not a universal phenomenon, and its presence in \ili{English} can be conditioned by independent, more general rules, while the absence of such restrictions leads to the absence of \isi{Attributive Comparative Deletion} in other languages.

\isi{Attributive Comparative Deletion} is a phenomenon that involves the obligatory \isi{deletion} of the quantified \isi{AP} and the \isi{lexical verb} from the comparative \isi{subclause}, if the quantified \isi{AP} functions as an attribute within a \isi{nominal expression}. Consider the examples given in (\ref{acdexamples7}):

\ea \label{acdexamples7}
\ea [] {Ralph bought a bigger cat than George did \sout{buy} a \sout{big} cat flap.} \label{conclusionacd}
\ex [] {Ralph bought a bigger cat than George \sout{bought} a \sout{big} cat flap.} \label{conclusioacdlex}
\ex [*] {Ralph bought a bigger cat than George bought a \sout{big} cat flap.} \label{conclusioadjdel}
\ex [*] {Ralph bought a bigger cat than George bought a big cat flap.}
\ex [*] {Ralph bought a bigger cat than George \sout{bought} a big cat flap.}
\ex [*] {Ralph bought a bigger cat than George did \sout{buy} a big cat flap.} \label{conclusioverbdel}
\z
\z

Both the \isi{adjective} (\textit{big}) and the \isi{lexical verb} (\textit{buy}) have to be eliminated from the comparative \isi{subclause}; this is possible either by eliminating the tensed \isi{lexical verb}, as in (\ref{conclusioacdlex}), or by deleting the \isi{lexical verb} and leaving the auxiliary \textit{do} bearing the \isi{tense} morpheme intact, as in (\ref{conclusionacd}). Since the \isi{verb} and the \isi{adjective} both have to be deleted, the examples in (\ref{conclusioadjdel})–(\ref{conclusioverbdel}) are ungrammatical.

As \chapref{ch:4} argued, this is because the \isi{degree expression} in the \isi{subclause} is not licensed to appear in a particular position within the extended \isi{nominal expression}. In other words, the obligatory elimination of the \isi{adjective} is not due to the fact that it is \textsc{given}; the overt presence of the \isi{attributive adjective} is ungrammatical even if it is different from its matrix \isi{clausal} counterpart, as shown in (\ref{nocontrastverbacd7}):

\ea \label{nocontrastverbacd7}
\ea	[*]{Ralph bought a bigger cat than George \sout{bought} a wide cat flap.}
\ex [*]{Ralph bought a bigger cat than George did \sout{buy} a wide cat flap.}
\z
\z

\largerpage[1]
On the other hand, the \isi{deletion} of the \isi{lexical verb} was shown to be required only if part of the \isi{DP} is overt; in case the entire \isi{DP} is eliminated, the \isi{lexical verb} can stay overt, as shown in (\ref{dpovert7}):

\ea	Ralph bought a bigger cat than George bought \sout{a big cat}. \label{dpovert7}
\z

These phenomena raise a number of questions that were answered in \chapref{ch:4}. The major questions are why the \isi{adjective} is not allowed to remain overt even if it is contrastive, why the \isi{verb} is also affected and how the \isi{lexical verb} and the \isi{adjective} can be deleted, as they do not seem to be adjacent in (\ref{conclusionacd}) and (\ref{conclusioacdlex}). I adopted the proposal made by \citet{kennedymerchant2000} regarding the syntactic position of the quantified \isi{AP} in the \isi{nominal expression} in structures like (\ref{conclusionacd}) and (\ref{conclusioacdlex}). According to this, the quantified \isi{AP} moves to the \isi{left edge} of the extended nominal projection and is hence adjacent to the \isi{lexical verb} at \isi{PF}. I also made the claim that the inversion option is available because in nominal expressions such as \textit{a cat} there is no \isi{DP} layer and the quantified expression may move to [Spec,\isi{FP}], while in structures containing a \isi{DP} layer the \isi{DP} is a boundary to such \isi{movement} operations. The structure for the quantified expression in the \isi{subclause} of attributive comparative constructions is as follows:

\ea \label{treefpconclusion} \upshape
\begin{forest} baseline, qtree, for tree={align=center}
[FP
	[QP\textsubscript{i}
		[how big,roof, name=qp]
	]
	[F$'$
		[F
			[(of)]
		]
		[NumP
			[Num$'$ [Num [a]] [NP [t\textsubscript{i}, name=trace] [N$'$ [N [dog]]]]]
		]
	]
]
\draw[->] (trace) to[out=south west,in=south,looseness=1.5] (qp);
\end{forest}
\z

I argued that the quantified \isi{AP} has to be eliminated because of the \isi{Overtness Requirement}: the quantified \isi{AP} moves to an \isi{operator position} (the specifier of the \isi{FP} projection) and, just as in the [Spec,\isi{CP}] position, \isi{lexical material} is licensed to appear here only if the \isi{operator} is overt. Since this condition is not met in the case of the \isi{comparative operator} in \ili{English}, the \isi{AP} has to be deleted. However, there is no separate mechanism that could carry it out and so a more general process has to apply, which is VP-\isi{ellipsis}. Given that VP-\isi{ellipsis} inevitably affects the \isi{lexical verb}, it is explained why the \isi{verb} has to be deleted.

In addition, \chapref{ch:4} aimed at addressing the relation between \isi{Attributive Comparative Deletion} and ordinary \isi{Comparative Deletion}. I showed that the higher \isi{copy} of the quantified \isi{DP} is deleted in a [Spec,\isi{CP}] position in attributive comparatives as well, and attributive comparatives are thus not exceptional in this respect. On the other hand, the reason for the ungrammaticality of the quantified \isi{AP} in the [Spec,\isi{FP}] position of the extended \isi{nominal expression} is due to the same \isi{Overtness Requirement} that was claimed to be responsible for the obligatory elimination of the higher \isi{copy} in the [Spec,\isi{CP}] position.

Furthermore, I took cross-linguistic differences into consideration, and it was shown that in \ili{Hungarian} the full structure may be visible in the \isi{subclause}:

\ea \label{acdhungconclusion}
\gll Rudolf nagyobb macsk\'at vett, mint amilyen sz\'eles macskaajt\'ot Mikl\'os vett.\\
Rudolph bigger cat.\textsc{acc}	bought.\textsc{3sg} than how wide cat.flap.\textsc{acc}	Mike	bought.\textsc{3sg}\\
\glt `Rudolph bought a bigger cat then Mike did a cat flap.'
\z

This is so because the \isi{comparative operator} is visible in \ili{Hungarian} and hence the entire quantified \isi{nominal expression} can be overt, as in (\ref{acdhungconclusion}).

\chapref{ch:4} also pointed out that \ili{German} does not allow \isi{Attributive Comparative Deletion} either:

\ea [*]{\gll  Ralf hat eine gr\"o{\ss}ere Wohnung als Michael ein Haus. \label{germanacdconclusion}\\
Ralph has a.\textsc{acc.f} bigger.\textsc{acc.f} flat than Michael a.\textsc{acc.n} house\\
\glt `Ralph has a bigger flat than Michael a house.'}
\z

The reason for this is that \ili{German} does not have the kind of inversion that \ili{English} has within the extended \isi{nominal expression}, and the \isi{adjective} is never located in a position that would cause ungrammaticality; in addition, it is not adjacent to the \isi{verb} either. The non-adjacency of the \isi{adjective} and the \isi{verb} is also due to the fact that the \isi{VP} is head-final in \ili{German}: thus, VP-\isi{ellipsis} cannot apply in the way it does in \ili{English}. The analysis presented in \chapref{ch:4} is hence able to account for cross-linguistic differences as well, since these are in fact reducible to more general properties of the respective languages.

Regarding the mechanisms underlying the phenomenon of \isi{Comparative Deletion} and that of \isi{Attributive Comparative Deletion}, I argued that the \isi{Overtness Requirement} regulates the realisation of the higher \isi{copy}, while the realisation of the lower \isi{copy} is essentially tied to the lexical XP being contrastive. In \chapref{ch:5}, I examined some languages that cannot realise contrastive lower copies either.

As far as the higher \isi{copy} is concerned, the \isi{Overtness Requirement} on left-peripheral elements is decisive, since this states that overt \isi{lexical material} is licensed in an \isi{operator position} only if the \isi{operator} itself is overt. I argued that there are four major logical possibilities, depending on whether the \isi{operator} moves on its own, and whether the \isi{operator} is overt or not. If the \isi{operator} is able to strand a lexical \isi{AP} or \isi{NP} (or there is no lexical XP base-generated together with the \isi{operator} at all), the lexical XP is spelt out in its \isi{base position}, and the \isi{overtness} of the \isi{operator} is immaterial, as is the information structural status of the lexical XP. If an \isi{overt operator} takes the lexical XP along to the [Spec,\isi{CP}], the lexical XP is licensed irrespective of its information structural status. However, if a phonologically \isi{zero operator} takes the lexical XP to the \isi{clausal} \isi{left periphery}, the entire phrase in [Spec,\isi{CP}] must be deleted in order to avoid a violation of the \isi{Overtness Requirement}. In this case, the lower \isi{copy} of the \isi{movement chain} (in the \isi{base position}) is realised overtly if it is contrastive. This leads to an asymmetry between contrastive and non-contrastive XPs: if the XP is contrastive, the absence of any overt \isi{copy} results in the surface phenomenon traditionally referred to as \isi{Comparative Deletion}. The realisation of contrastive XPs, on the other hand, appears to be straightforward, at least in \ili{English}.

Using data from \ili{Slavic}, \chapref{ch:5} demonstrated that the availability of the lower \isi{copy} for overt realisation is not universal. Consider the \ili{Polish} in (\ref{polish7}):

\ea \label{polish7}
\ea [*]{\gll Maria jest wyższa niż Karol jest \textbf{wysoki}. \label{polishcdconclusion}\\
Mary is taller than Charles is tall\\
\glt `Mary is taller than Charles.'}
\ex [*/??]{\gll Stół jest dłuższy niż biuro jest \textbf{szerokie}. \label{polishsubcompconclusion}\\
desk is longer than office is wide\\
\glt `The desk is longer than the office is wide.'}
\z
\z

While the ungrammaticality of (\ref{polishcdconclusion}) is expected on the basis of the \ili{English} pattern, the fact that \ili{Polish} apparently lacks predicative subcomparatives in the \ili{English} way, that is, the ungrammaticality of (\ref{polishsubcompconclusion}), is not expected. As was shown in \chapref{ch:5}, \ili{Polish} is not unique in this respect: \ili{Czech} shows the same distribution, too. I argued that the realisation of the lower \isi{copy} is dependent on more general properties of \isi{movement} chains in a certain language, which results in a difference between \ili{English} and \ili{Polish}/\ili{Czech}. In particular, as demonstrated by \citet{boskovic2002}, \textit{wh}-elements have to undergo fronting in multiple \textit{wh}-fronting languages such as \ili{Polish} independently of an active [wh] \isi{feature} on C: that is, while the first moved \textit{wh}-constituent checks off the [wh] \isi{feature} on C and thus it undergoes ordinary \textit{wh}-\isi{movement}, the further \textit{wh}-elements merely undergo obligatory fronting. This is presumably because these elements are equipped with an \textsc{edge} \isi{feature}. \citet{boskovic2002} shows that apparent exceptions to the fronting requirement are relatively rare and they are subject to certain conditions; further, these instances do not involve the lack of fronting but rather the realisation of a lower \isi{copy} of a \isi{movement chain}. I argued that since these requirements are absent from comparative constructions, the realisation of a contrastive lower \isi{copy} is not possible in these languages.

I argued that there are thus three major factors determining the overt realisation of the quantified expression: whether the \isi{operator} is overt, whether it is extractable, and whether lower copies of a \isi{movement chain} can be realised if the pronunciation of the higher \isi{copy} would cause the derivation to crash at \isi{PF}. The possibilities are summarised in the representation given in (\ref{summary7}):

%\newpage
\ea \upshape \label{summary7}
\scalebox{0.8}{
\begin{forest} baseline, qtree, for tree={align=center} 
[\textbf{\isi{operator} overt?}
	[YES
		[\textbf{\isi{operator} extractable?}
			[YES [\ili{Czech}\\(\textit{jak})] [\ili{Hungarian}\\(\textit{amennyire})]]
			[NO [\ili{English}\\(\textit{how})] [\ili{Dutch}\\(\textit{hoe})] [\ili{Hungarian}\\(\textit{amilyen})]]
		]
	]
	[NO, name=no [{} [{} [{} [{}
		[\textbf{\isi{operator} extractable?}, name=op
			[YES [\ili{German}\\($\emptyset$)] [\ili{Dutch}\\($\emptyset$)] [\ili{Estonian}\\($\emptyset$)] [\ili{Serbo-Croatian}\\($\emptyset$)]]
			[NO, name=no2 [{} [{} [\textbf{lower copies available?}, name=op2 [YES [\ili{English}\\($\emptyset$)] [\ili{Norwegian}\\($\emptyset$)] [\ili{Icelandic}\\($\emptyset$)]] [NO [\ili{Czech}\\($\emptyset$] [\ili{Polish}\\($\emptyset$)]]]]]]
		]
	]]]]]
]
\path [draw] (no.south) -- (op.north);
\path [draw] (no2.south) -- (op2.north);
\end{forest}}
\z

The most important finding in this respect is that the \ili{English} pattern, where \isi{Comparative Deletion} refers to the obligatory elimination of a non-contrastive \isi{AP} from the comparative \isi{subclause}, is not universal at all: in fact, it is highly language-specific, and it can only be regarded as a result of several factors. Thus, \isi{Comparative Deletion} cannot be regarded as a universal phenomenon or a parameter either, and the analysis of the particular \ili{English} pattern cannot be solely based on Standard \ili{English} data but must take other languages and non-standard varieties into consideration.

Finally, \chapref{ch:6} aimed at accounting for optional \isi{ellipsis} processes that play a crucial role in the derivation of typical comparative subclauses. These processes are not directly related to the structure of \isi{degree} expressions and the elimination of the quantified expression from the \isi{subclause}; nevertheless, they were shown to interact with the mechanisms underlying \isi{Comparative Deletion} or the absence thereof.

In \ili{English} predicative structures, shown in (\ref{predicative7}), this involves the elimination of the \isi{copula} from subclauses such as the one given in (\ref{predellipticalconclusion}), as opposed to the one given in (\ref{predfullconclusion}):

\ea \label{predicative7}
\ea	Ralph is more enthusiastic than Jason is. \label{predfullconclusion}
\ex	Ralph is more enthusiastic than Jason. \label{predellipticalconclusion}
\z
\z

In nominal comparatives, as shown in (\ref{nominal7}), the \isi{lexical verb} may be deleted:

\ea \label{nominal7}
\ea	Ralph bought more houses than Michael bought flats. \label{nomfullconclusion}
\ex	Ralph bought more houses than Michael did flats. \label{nomgapconclusion}
\ex	Ralph bought more houses than Michael did. \label{nomvpellipsisconclusion}
\ex	Ralph bought more houses than Michael. \label{nomellipsisconclusion}
\z
\z

Verb \isi{deletion} may result either in a \isi{subclause} without any verbal element, as in (\ref{nomellipsisconclusion}), or the \isi{tense} morpheme may be carried by the \isi{dummy auxiliary}, as in (\ref{nomgapconclusion}) and (\ref{nomvpellipsisconclusion}). In addition, depending on whether the object contains a contrastive \isi{noun} or not, the object \isi{nominal expression} remains overt, as in (\ref{nomfullconclusion}) and (\ref{nomgapconclusion}), or does not appear overtly, as in (\ref{nomvpellipsisconclusion}) and (\ref{nomellipsisconclusion}). A very similar pattern arises in attributive comparatives, as shown in (\ref{attributive7}):

\ea \label{attributive7}
\ea	Ralph bought a bigger house than Michael did a flat.
\ex	Ralph bought a bigger house than Michael did.
\ex	Ralph bought a bigger house than Michael.	
\z
\z

The main question was whether the \isi{deletion} of the \isi{lexical verb} is merely the \isi{deletion} of the verbal head or whether there is VP-\isi{ellipsis} at hand; in the latter case, the possibility of having overt objects (or parts of objects) must be accounted for. Using the analysis given in \chapref{ch:4}, \chapref{ch:6} argued that \isi{gapping} is an instance of VP-\isi{ellipsis}, which proceeds from a left-to-right fashion at \isi{PF}, and the starting point of it is an [E] \isi{feature} on a functional v head, in line with \citet{merchant2001}. The endpoint of \isi{ellipsis} is a contrastive phrase, if there is any. I also showed that since the [E] \isi{feature} can be present on a C head as well, the derivation of comparative subclauses at \isi{PF} may involve \isi{ellipsis} starting from an [E] \isi{feature} either on a C or a v head. Since the final string may be ambiguous, one of the central questions is whether a uniform kind of \isi{ellipsis} mechanism may account for these ambiguities; this was indeed shown to be possible.

On the other hand, the fact that reduced comparative subclauses also exist in \ili{Hungarian} raises the question of how languages that have overt comparative operators exclusively may show the elimination of the entire \isi{degree expression}, given that there is no \isi{Comparative Deletion} in these languages. For instance, predicative comparatives in \ili{Hungarian} show the variation given in (\ref{hungpredicative7}):

\ea \label{hungpredicative7}
\ea \gll	Mari	magasabb volt, mint \textbf{amilyen} \textbf{magas} Péter \textbf{volt}. \label{hungpredfullellipsisconclusion}\\
Mary taller	was.\textsc{3sg} than	how	tall	Peter was.\textsc{3sg}\\
\glt `Mary was taller than Peter.'
\ex	\gll Mari	magasabb volt, mint	Péter. \label{hungpredellipticalconclusion}\\
Mary taller	was.\textsc{3sg} than	Peter\\
\glt `Mary was taller than Peter.'
\z
\z

In (\ref{hungpredfullellipsisconclusion}) the \isi{subclause} contains all the elements overtly, while the \isi{degree expression} and the \isi{verb} are absent from (\ref{hungpredellipticalconclusion}). The same can be observed in nominal comparatives, as illustrated in (\ref{hungnominal7}):

\ea \label{hungnominal7}
\ea \gll Mari	több macskát vett, mint	\textbf{ahány} \textbf{macskát} Péter \textbf{vett}.\\
Mary more	cat.\textsc{acc} bought.\textsc{3sg} than	how.many cat.\textsc{acc} Peter bought.\textsc{3sg}\\
\glt `Mary bought more cats than Peter did.'
\ex \gll Mari	több macskát vett, mint	Péter. \label{hungnomellipticalconclusion}\\
Mary more	cat.\textsc{acc} bought.\textsc{3sg} than Peter\\
\glt `Mary bought more cats than Peter did.'
\z
\z

Finally, attributive comparatives also show this pattern, as illustrated in (\ref{hungattributive7}):

\ea \label{hungattributive7}
\ea	\gll Mari nagyobb macskát vett, mint \textbf{amilyen} \textbf{nagy} \textbf{macskát} Péter \textbf{vett}.\\
Mary bigger	cat.\textsc{acc} bought.\textsc{3sg} than how big cat.\textsc{acc} Peter bought.\textsc{3sg}\\
\glt `Mary bought a bigger cat than Peter did.'
\ex \gll Mari nagyobb	macskát	vett,	mint Péter. \label{hungattrellipticalconclusion}\\
Mary bigger	cat.\textsc{acc} bought.\textsc{3sg} than	Peter\\
\glt `Mary bought a bigger cat than Peter did.'
\z
\z

In all of these cases the sentences of a given pair have the same meaning. The main research question was whether the \isi{deletion} of the \isi{degree expression} is independent from that of the \isi{verb} or not. As \chapref{ch:6} showed, these are not two independent processes, since the \isi{verb} cannot be overt in the absence of an overt \isi{degree expression}. I argued that this is so because it is ungrammatical to have an \isi{operator} in its \isi{base position} in \ili{Hungarian}, but since there is no separate mechanism that would eliminate the \isi{degree expression}, a more general \isi{ellipsis} process has to apply, which is essentially VP-\isi{ellipsis}. The \isi{ellipsis} mechanism is fairly similar to the one attested in \ili{English} and the differences were linked to the slightly different internal structure of the functional layers in the two languages. Otherwise \isi{ellipsis} is carried out by an [E] \isi{feature} on the highest possible \isi{functional head} in \ili{Hungarian} too; the \isi{ellipsis domain} for the subclauses in (\ref{hungpredellipticalconclusion}), (\ref{hungnomellipticalconclusion}) and (\ref{hungattrellipticalconclusion}) is given in (\ref{treehungariantpellipsisconclusion}):

\ea \label{treehungariantpellipsisconclusion} \upshape 
\begin{forest} baseline, qtree, for tree={align=center}
[FP
	[DP\textsubscript{i}
		[P\'eter,roof]
	]
	[F$'$
		[F
			[$\emptyset$\\{[}E{]}]
		]
		[TP, name=tp
			[t\textsubscript{i} volt {[}\textsubscript{QP} amilyen magas{]}\\t\textsubscript{i} vett {[}\textsubscript{DP} ahány macskát{]}\\t\textsubscript{i} vett {[}\textsubscript{DP} amilyen nagy macskát{]},roof]
		]
	]
]
\path [draw, thick] (tp.north west) -- (tp.south east);
\path [draw, thick] (tp.north east) -- (tp.south west);
\end{forest}
\z

As can be seen, both the \isi{verb} (the \isi{copula} or a \isi{lexical verb}) and the quantified expression (either a \isi{QP} or a \isi{DP} containing a \isi{QP}) are located in the \isi{ellipsis domain}, which is the complement of the F head: the \isi{FP} itself is the leftmost projection that can host the [E] \isi{feature}.

I argued that in case the \isi{verb} is contrastive, \isi{ellipsis} is slightly different. If the contrastive \isi{verb} is a \isi{copula}, then the [E] \isi{feature} can be located on a lower \isi{functional head} (the head of the \isi{PredP}, predicative phrase) and hence the \isi{ellipsis} site is located lower. The \isi{copula} itself is base-generated in Pred but moves regularly further up to T and to the F head. However, if the contrastive \isi{verb} is a \isi{lexical verb}, this is not base-generated in Pred but moves there only from a lower (V) head: this involves a \isi{movement} step that normally does not take place, since the \isi{movement} of the \isi{verb} is not triggered to a \isi{functional head} containing [E]. The resulting configurations are thus marked, even though they are not ungrammatical.

I showed that the difference between \ili{English} and \ili{Hungarian} in terms of \isi{gapping} effects is chiefly a result of the different prosody in the two languages: while the \isi{Intonational Phrase} is right-headed in \ili{English}, it is left-headed in \ili{Hungarian}. Therefore, while contrastive elements are located at the \isi{right edge} of the \isi{ellipsis domain} in \ili{English}, in \ili{Hungarian} they are to the left of the \isi{functional head} hosting the [E] \isi{feature} and consequently not part of the \isi{ellipsis domain}. \chapref{ch:6} also showed that since there is strong directionality in terms of \isi{ellipsis}, in that it proceeds in a strict left-to-right fashion, this kind of \isi{ellipsis} works only in head-initial phrases since the \isi{ellipsis domain} (the complement) has to follow the head hosting the [E] \isi{feature}. This accounts for why \ili{German} does not have VP-\isi{ellipsis} in the way \ili{English} does: the \ili{German} \isi{VP} and all \isi{vP} layers are head-final while in \ili{English} all \isi{VP} projections are head-initial. Cross-linguistic differences concerning optional \isi{ellipsis} processes can thus be reduced to more general properties that hold in individual languages, and hence \isi{ellipsis} processes are not construction-specific.

