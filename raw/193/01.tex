\chapter{Introduction} \label{ch:1}
\section{Aims and scope} \label{sec:1introduction}
The core problem to be dealt with in this book is the syntax of comparatives, that is, the structure of sentences that express comparison. As far as the notion of syntactic structure is concerned, I will basically adopt a minimalist framework (cf. for instance \citealt{chomsky2001, chomsky2004, chomsky2008}) and, in line with the principles of mainstream generative grammar, I assume that the derivation of structures is constrained by economy, and the number of structural layers, derivational steps and additional mechanisms is as small as possible. This means that although I adopt the view that various functional layers and mechanisms can be associated with these layers, I will keep them to a minimum and will not venture to introduce new ones unless there seems to be ample reason to do so.

Regarding the focus on comparative structures in particular, even though comparatives seem to be a very specific domain of research within syntax, the derivation of their structure raises questions of far more general interest, and providing meaningful answers to these questions may also have a bearing on our understanding of syntactic mechanisms, regarding, for instance, the functional \isi{left periphery} of clauses, clause-typing, or various \isi{ellipsis} processes.

It is very probably this diversity of problems that led to a significant interest in comparatives in generative frameworks already in the 1970s, most notably in \citet{bresnan1973, bresnan1975}, followed by various analyses with more or less shared concerns: for example, \citet{corver1993, corver1997}, \citet{izvorski1995}, \citet{lechner1999diss, lechner2004}, \citet{kennedy1997diss, kennedy1999, kennedy2002}, \citet{kennedymerchant1997, kennedymerchant2000}, and more recently \citet{reglero2006}. I will strongly rely on these previous findings and especially the questions raised by them. While many questions have been answered by previous accounts, there are several others that have remained unresolved and have not received an adequate explanation which would hold cross-linguistically as well. Moreover, any proposal should follow from general principles of the grammar rather than by applying construction-specific mechanisms. The aim of this book is to provide such an analysis and to enable a better understanding of \isi{comparative clause} formation.

In the following, I will briefly provide an overview of the structure of comparatives, to be followed by the concise outline of the problems to be dealt with in this book.

\section{The structure of comparatives} \label{sect:1structure}
In any human language, there are various means of expressing comparison between entities (or properties), and structures traditionally referred to as comparatives constitute only a subset of these possibilities. Consider the examples in (\ref{implicitcomp1}):

\ea \label{implicitcomp1}
\ea Mary was indeed furious when she saw that you had broken her vase. But you should have seen her mother! \label{implicitcomparison}
\ex	Mary is tall but Susan is very tall. \label{verycomparison}
\ex	Mary is faster than Susan. \label{thancomparison}
\z
\z

In (\ref{implicitcomparison}), comparison is only implied: the first sentence makes it explicit that Mary was furious to a certain \isi{degree} but the second sentence contains no explicit reference to a \isi{degree}, yet it implies that the \isi{degree} to which Mary's mother was furious exceeds the \isi{degree} to which Mary was furious. In (\ref{verycomparison}), both the \isi{degree} to which Mary is tall and the \isi{degree} to which Susan is tall are explicitly referred to: without any further specification, it is understood that on a scale of height, the \isi{degree} to which Mary is tall is greater than what is contextually taken to be average and that the \isi{degree} to which Susan is tall is considerably greater than the average. Hence, the degrees of tallness are explicitly referred to, even if they remain vague; however, the comparison between the two degrees is not made explicit, but the relation of the two degrees can be inferred. Finally, (\ref{thancomparison}) exhibits a canonical comparative structure, which expresses that the \isi{degree} to which Mary is fast exceeds the \isi{degree} to which Susan is fast.

The present book aims at analysing syntactic comparative constructions, that is, the type represented by (\ref{thancomparison}) above. The sentence in (\ref{thancomparison}) shows the most important elements of comparative constructions: in this case, the degrees of speed of two entities are compared. The reference value of comparison is expressed by \textit{faster} in the \isi{matrix clause} (\textit{Mary is faster}) and it consists of a \isi{gradable predicate} (\textit{fast}) and a comparative \isi{degree marker} (-\textit{er}). The standard value of comparison (that is, to which something else is compared) is expressed by the \isi{subordinate clause} (\textit{than Susan}) and is introduced by the \isi{complementiser} \textit{than}, which also serves as the standard marker.

There are some important remarks to be made here. First, in (\ref{thancomparison}), the comparative \isi{degree marker} is a bound morpheme attached to the \isi{gradable predicate}; however, this is not an available option for all adjectives in \ili{English} and very often a periphrastic structure is used, when -\textit{er} is present in the form of \textit{more}, as in (\ref{periphrastic1}):

\ea	Mary is more pretentious than Susan. \label{periphrastic1}
\z

Languages differ in terms of whether they allow both kinds of \isi{comparative degree} marking and some languages (such as \ili{German}) allow only the morphological way of comparative \isi{adjective} formation, while others (such as \ili{Italian}) have the periphrastic way by default.

Second, in (\ref{thancomparison}) the standard value of comparison is introduced by the \isi{complementiser} \textit{than} and the string \textit{than Susan} is underlyingly a clause. This is explicitly shown by examples like (\ref{maryfaster1}) that contain a \isi{finite verb} as well:

\ea	Mary is faster than Susan is. \label{maryfaster1}
\z

Since the clause can be recovered, comparatives formed with \textit{than} are invariably \isi{clausal}. However, languages also differ with respect to the distribution of whether they have \isi{clausal} and/or phrasal comparison. For instance, \ili{Hungarian} has both \isi{clausal} comparatives, introduced by \textit{mint} `than/as' and phrasal comparatives, see (\ref{maritaller1}), where the standard value is expressed by an inherently Case-marked \isi{DP}:

\ea	\label{maritaller1} \gll Mari magasabb	Zsuzsánál.\\
Mary taller	Susan.\textsc{ade}\\
\glt `Mary is taller than Susan.'
\z

In this case, the \isi{DP} \textit{Zsuzsánál} is inherently marked for adessive case and there is no clause that could be recovered. As my primary concern in this book is the structure of comparative subclauses, I will not be dealing with instances of phrasal comparison more than necessary: that is, I will briefly include them in the discussion when the arguments of the \isi{degree morpheme} are considered and will relate them to subordinate clauses in this respect, but apart from this, they fall outside the scope of the present investigation.

It is also important to mention that \isi{degree} constructions denote a larger set of structures than comparatives, within which one can distinguish between two major types, see (\ref{asmoreless1}): comparatives expressing \isi{equality}, as shown in (\ref{asas}), and comparatives expressing \isi{inequality}, as in (\ref{morethan}) and (\ref{lessthan}):

\ea \label{asmoreless1}
\ea Mary is as diligent as Susan. \label{asas}
\ex	Mary is more diligent than Susan. \label{morethan}
\ex	Mary is less diligent than Susan. \label{lessthan}
\z
\z

In (\ref{asas}), the \isi{degree} to which Mary is diligent is the same as the \isi{degree} to which Susan is diligent; by \isi{contrast}, in (\ref{morethan}) and (\ref{lessthan}) the degrees are different, such that the \isi{degree} to which Mary is diligent is higher in (\ref{morethan}) and lower in (\ref{lessthan}). As can be seen, the comparative \isi{subclause} is introduced by \textit{as} in (\ref{asas}) and by \textit{than} in both (\ref{morethan}) and (\ref{lessthan}). The present book aims at providing an analysis for comparatives expressing \isi{inequality} and more precisely for ones of the type given in (\ref{morethan}); nevertheless, the analysis has relevant conclusions for all types but I will not venture to discuss further differences here. The choice regarding (\ref{morethan}) is not arbitrary, though: this is the type that encompasses all comparative-related issues to some extent and the relevant literature has also mostly discussed this type.

\section{The problems to be discussed} \label{sec:1problems}
To start with, \chapref{ch:2} will discuss the structure of \isi{degree} expressions, with the aim of providing a unified analysis that relates the structure of comparatives to that of other (absolute and superlative) degrees. Naturally, a number of questions arise concerning the general structure of \isi{degree} phrases, of which I will select only the ones that are relevant for the present book. The importance of comparatives in this respect is that they tend to contain a number of elements overtly that clearly indicate the presence of various functional layers, presenting a challenge for previous analyses, but at the same time indicating certain ways in which the syntactic structure of \isi{degree} adjectives can best be captured.

One such problem is the presence of the \isi{degree morpheme} itself, which becomes obvious when comparing the sentences in (\ref{asmoreless1}):

\ea \label{degreeparadigm1}
\ea Mary is \textbf{tall}. \label{nodegreemorpheme}
\ex	Mary is \textbf{taller} than Peter. \label{degreemorpheme}
\z
\z

The \isi{contrast} between (\ref{nodegreemorpheme}) and (\ref{degreemorpheme}) is that while the very same lexical \isi{adjective} (\textit{tall}) appears in both cases, in (\ref{degreemorpheme}) there is an additional \isi{degree morpheme} (-\textit{er}). The fact that the \isi{degree marker} is syntactically separate from the \isi{adjective} is more clearly indicated by periphrastic comparatives such as (\ref{periphrastic}):

\ea Mary is \textbf{more intelligent} than John. \label{periphrastic}
\z

In (\ref{periphrastic}), the \isi{comparative degree} is marked by \textit{more}; \chapref{ch:2} will account for the difference and the relatedness of structures like (\ref{degreemorpheme}) and (\ref{periphrastic}), showing that the same functional layers are present and the head element in the \isi{degree expression} is -\textit{er} in both cases.

Second, the relation between the comparative \isi{degree marker} and the comparative \isi{subclause} must also be explained as the type of the \isi{subclause} seems to be defined by the comparative marker in the \isi{matrix clause}:

\ea \label{selection}
\ea []{Mary is \textbf{taller} [than John].}
\ex	[*]{Mary is \textbf{taller} [as John].}
\ex	[*]{Mary is \textbf{as tall} [than John].}
\ex	[]{Mary is \textbf{as tall} [as John].}
\z
\z

As shown by the examples in (\ref{selection}), if the \isi{degree expression} in the \isi{matrix clause} contains the morpheme -\textit{er}, then the \isi{subclause} must be introduced by \textit{than}; conversely, a \isi{degree expression} with \textit{as} in the \isi{matrix clause} requires a \isi{subclause} introduced by \textit{as}. These selectional restrictions are obviously not dependent on the lexical \isi{adjective}, which is invariably \textit{tall}. I will show in \chapref{ch:2} that the comparative \isi{subclause} is one argument of the \isi{degree head}, the other being the lexical \isi{AP} itself; consequently, there are restrictions that hold between the \isi{degree head} and the \isi{subclause} but there are none that would hold between the \isi{AP} and the \isi{subclause}.

Even though my main concern is not the argument structure of adjectives, it should be mentioned that adjectives may have arguments of their own:

\ea	Mary is proud [of her husband]. \label{apargument}
\z

In cases like (\ref{apargument}), the \isi{adjective} (\textit{proud}) takes a \isi{PP} (\textit{of her husband}) as its complement; this must also be accounted for, especially in relation to the subclauses indicated in (\ref{selection}), which are not directly introduced by the \isi{adjective} itself but are nevertheless obligatory. \chapref{ch:2} will argue that \isi{PP} complements of adjectives are indeed complements of the \isi{adjective} head but may appear in a right-dislocated position due to the nature of cyclic spellout to \isi{PF}.

The structure adopted for \isi{degree} expressions will be used when accounting for \isi{Comparative Deletion} in \chapref{ch:3}, which constitutes the core part of the book. My aim here is to reduce the cross-linguistic differences attested in connection with \isi{Comparative Deletion} to minimal differences in the relevant operators. I intend to show that \isi{Comparative Deletion} is merely a surface phenomenon and hence does not have to be treated as a parameter distinguishing between languages; instead, I will adopt a feature-based account that can handle language-internal variation as well. I will argue that the difference is ultimately not between individual languages but rather between overt operators that do and covert operators that do not trigger \isi{Comparative Deletion}. To my knowledge, this claim is radically new in the literature and hopefully it may account for several phenomena that have been unexplained so far. This chapter will also present data that has not been discussed in the literature, including non-standard \ili{English}, \ili{German} and \ili{Dutch} patterns, as well as \ili{Hungarian} and \ili{Slavic} (mostly \ili{Czech}) data.

The phenomenon of \isi{Comparative Deletion} traditionally denotes the absence of an adjectival or \isi{nominal expression} from the comparative \isi{subclause}, as indicated in the following examples:

\ea \label{cd}
\ea	Ralph is more qualified than Jason is \sout{\textbf{x-qualified}}. \label{xqualified}
\ex	Ralph has more qualifications than Jason has \sout{\textbf{x-many qualifications}}. \label{xmanyqualifications}
\ex	Ralph has better qualifications than Jason has \sout{\textbf{x-good qualifications}}. \label{xgoodqualifications}
\z
\z

In the sentences above, \textit{x} denotes a certain \isi{degree} or quantity as to which a certain entity is qualified, good, etc. This is an \isi{operator} that has no phonological content. In (\ref{xqualified}), an \isi{adjectival expression} is deleted: this type is referred to as the predicative comparative since the quantified \isi{adjectival expression} functions as a \isi{predicate} in the \isi{subclause}. By \isi{contrast}, in both (\ref{xmanyqualifications}) and (\ref{xgoodqualifications}) a \isi{nominal expression} is deleted; structures like (\ref{xmanyqualifications}) are nominal comparatives, where a \isi{nominal expression} bears quantification, while (\ref{xgoodqualifications}) is an attributive comparative, where the quantified \isi{adjectival expression} is an attributive \isi{modifier} within a \isi{nominal expression}.

Therefore, one of the most important questions to be answered in connection with \isi{Comparative Deletion} is how to account for the fact that different constituents seem to be deleted by \isi{Comparative Deletion}. Moreover, this \isi{deletion} process seems to be obligatory to the extent that the presence of the quantified expressions in (\ref{cd}) would lead to ungrammatical constructions; therefore, a proper analysis of \isi{Comparative Deletion} must also address the issue why this process seems to be obligatory. I will argue that the site of \isi{Comparative Deletion} is not the one indicated in (\ref{cd}) but a left-peripheral, [Spec,\isi{CP}] position. The reason why the strings indicated as deleted elements in (\ref{cd}) cannot be overt is that they are lower copies of a moved constituent and are regularly eliminated.

The role of \isi{information structure} underlying \isi{Comparative Deletion} has to be taken into consideration as well. In \isi{subcomparative} structures, an adjectival or nominal element may be left overt in the \isi{subclause}; as opposed to the examples in (\ref{cd}), these elements are not logically identical to an antecedent in the \isi{matrix clause}:

\ea \label{subcomp}
\ea	The table is longer than the desk is \textbf{wide}.
\ex	Ralph has more books than Jason has \textbf{manuscripts}.
\ex	Ralph wrote a longer book than Jason did \textbf{a manuscript}.
\z
\z

I will show in \chapref{ch:3} that \isi{movement} takes place even in these cases, and hence the higher \isi{copy} is regularly eliminated; the reason why the lower copies are realised overtly is that they are contrastive. My analysis will thus crucially differ from those (for example \citealt{kennedy2002}) that try to capture the surface dissimilarity between (\ref{cd}) and (\ref{subcomp}) on the basis of whether \textit{wh}-\isi{movement} takes place overtly, as in (\ref{cd}), or covertly, as in (\ref{subcomp}). I assume that syntactic \isi{movement} triggered by a [+wh] or a [+rel] \isi{feature} cannot be sensitive to the information structural properties of the lexical XP (\isi{AP}/\isi{NP}) that moves together with the \isi{operator} for independent reasons (that is, the non-extractability of the \isi{operator} from the functional projections containing these lexical elements).

Given that \isi{deletion} in the [Spec,\isi{CP}] position takes place if the \isi{operator} is zero, it can be expected that visible operators can remain overt in this position. Though this option is not available in Standard \ili{English}, substandard dialects may allow configurations such as (\ref{howqualified}) below:

\ea	\% Ralph is more qualified than \textbf{how qualified} Jason is. \label{howqualified}
\z

Naturally, an analysis of \isi{Comparative Deletion} must also address the question of how examples such as (\ref{howqualified}) relate to the ones given in (\ref{cd}) or (\ref{subcomp}); I will argue that all of these constructions involve the \isi{movement} of the quantified expression, but the higher \isi{copy} is not elided in (\ref{howqualified}) since the \isi{overtness} requirement on left-peripheral elements is satisfied.

Apart from varieties of \ili{English} that allow instances like (\ref{howqualified}), in some languages full \isi{degree} expressions can be regularly attested at the \isi{left periphery} of the \isi{subclause}, as in the examples in (\ref{hungcomp1}) from \ili{Hungarian} (cf. \citealt{kenesei1992}):

\largerpage[-2]
\ea \label{hungcomp1}
\ea \gll Mari magasabb, mint \textbf{amilyen} \textbf{magas} P\'eter. \label{hungpredfull}\\
Mary taller than how tall Peter\\
\glt `Mary is taller than Peter.'
\ex \gll Marinak t\"{o}bb macsk\'aja van, mint \textbf{ah\'any} \textbf{macsk\'aja} P\'eternek van.\\
Mary.\textsc{dat} more cat.\textsc{poss.3sg} is than how.many cat.\textsc{poss.3sg} Peter.\textsc{dat} is\\
\glt `Mary has more cats than Peter has.'
\ex \gll Marinak nagyobb macsk\'aja van, mint \textbf{amilyen} \textbf{nagy} \textbf{macsk\'aja} P\'eternek van.\\
Mary.\textsc{dat} bigger cat.\textsc{poss.3sg} is than how big	cat.\textsc{poss.3sg} Peter.\textsc{dat} is\\
\glt `Mary has a bigger cat than Peter has.'
\z
\z

As can be seen, \ili{Hungarian} allows the overt presence of the \isi{degree} elements, which again shows that \isi{Comparative Deletion} must be subject to (parametric) variation. I will argue that this variation can be accounted for by the \isi{Overtness Requirement}: \ili{Hungarian} has overt operators while Standard \ili{English} does not, and therefore the overt presence of lexical elements in a [Spec,\isi{CP}] position is available in \ili{Hungarian}, just as in the case of non-standard varieties of \ili{English}.

Strongly related to this, the question arises to what extent the internal structure of the \isi{degree expression} plays a role and whether individual operators exhibit different behaviour in this respect. In \ili{Hungarian}, there are two comparative operators, \textit{amilyen} `how' and \textit{amennyire} `how much'. The \isi{operator} \textit{amilyen} may appear together with the \isi{adjective}, as in (\ref{hungpredfull}), but it does not allow the stranding of the \isi{adjective}, as shown in (\ref{maritallerhow1}):

\ea [*]{\gll Mari	magasabb,	mint \textbf{amilyen} Péter	\textbf{magas}. \label{maritallerhow1}\\
Mary taller	than how	Peter	tall\\
\glt `Mary is taller than Peter.'}
\z

On the other hand, \ili{Hungarian} has another \isi{operator}, \textit{amennyire} `how much', which allows both options for the \isi{adjective}, as shown in (\ref{maripeter1}):

\ea \label{maripeter1}
\ea \gll Mari	magasabb,	mint \textbf{amennyire} \textbf{magas} Péter.\\
Mary taller	than how.much	tall Peter\\
\glt `Mary is taller than Peter.'
\ex \gll Mari	magasabb,	mint \textbf{amennyire} Péter \textbf{magas}.\\
Mary taller	than	how.much Peter tall\\
\glt `Mary is taller than Peter.'
\z
\z

In addition, as shown in (\ref{hungzero1}), \ili{Hungarian} also seems to require the presence of some \isi{operator} if the \isi{adjective} is overt (note, however, that it is allowed for the \isi{adjective} and the \isi{operator} to be non-overt at the same time):

\ea \label{hungzero1}
\ea \gll Mari	magasabb,	mint \textbf{(*magas)} Péter.\\
Mary taller	than \phantom{\textbf{(*}}tall Peter\\
\glt `Mary is taller than Peter.'
\ex \gll Mari	magasabb,	mint Péter \textbf{(*magas)}.\\
Mary taller	than Peter \phantom{\textbf{(*}}tall\\
\glt `Mary is taller than Peter.'
\z
\z

I will show in \chapref{ch:3} that \ili{Hungarian} lacks a \isi{covert operator}, and that the difference between \textit{amilyen} and \textit{amennyire} is due to the fact that they occupy different positions in the extended \isi{degree expression}, based on the findings concerning the structure of \isi{degree} expressions in \chapref{ch:2}. Hence, my analysis of \isi{Comparative Deletion} is based on the assumption that languages differ with respect to the presence/absence of the \isi{operator} in a more intricate way than one that could be formulated on a +/-- basis.

Following these lines of thought, \chapref{ch:4} will address a special instance of \isi{Comparative Deletion}, which is traditionally referred to in the literature as \isi{Attributive Comparative Deletion}. I will show that \isi{Attributive Comparative Deletion} can only be understood as a descriptive term indicating a phenomenon that is a result of the interaction of more general syntactic processes, and therefore there is no reason to postulate any special mechanism underlying \isi{Attributive Comparative Deletion} in the grammar. By eliminating such a mechanism, it is possible to achieve a unified analysis of all types of comparatives. \chapref{ch:4} will also show that \isi{Attributive Comparative Deletion} is not a universal phenomenon: its appearance in \ili{English} can be conditioned by independent, more general rules and the absence of such restrictions may lead to the absence of \isi{Attributive Comparative Deletion} in other languages. In this respect, novel data from \ili{German} and \ili{Hungarian} will be presented and discussed.

\isi{Attributive Comparative Deletion} refers to a peculiar phenomenon that involves the obligatory \isi{deletion} of the quantified \isi{AP} and the \isi{lexical verb} from the comparative \isi{subclause}, if the quantified \isi{AP} functions as an attribute within a \isi{nominal expression}. Consider the examples in (\ref{acd1}):

\ea \label{acd1}
\ea	[]{Ralph bought a bigger cat than George did \sout{buy} a \sout{big} cat flap.} \label{acddid}
\ex []{Ralph bought a bigger cat than George \sout{bought} a \sout{big} cat flap.} \label{acdtensed}
\ex	[*]{Ralph bought a bigger cat than George bought a \sout{big} cat flap.} \label{acdap}
\ex	[*]{Ralph bought a bigger cat than George bought a big cat flap.}
\ex	[*]{Ralph bought a bigger cat than George \sout{bought} a big cat flap.}
\ex	[*]{Ralph bought a bigger cat than George did \sout{buy} a big cat flap.} \label{acdbuy}
\z
\z

As can be seen, both the \isi{adjective} (\textit{big}) and the \isi{lexical verb} (\textit{buy}) have to be eliminated from the comparative \isi{subclause}: this is possible either by eliminating the tensed \isi{lexical verb}, as in (\ref{acdtensed}) or by deleting the \isi{lexical verb} and leaving the tense-bearing auxiliary \textit{do} intact, as in (\ref{acddid}). Note that both the \isi{verb} and the \isi{adjective} have to be deleted, as indicated by the ungrammaticality of the sentences in (\ref{acdap})–(\ref{acdbuy}).

The obligatory elimination of the \isi{adjective} is not directly related to the fact that it is \textsc{given}; the overt presence of the \isi{attributive adjective} is ungrammatical even if it is different from its matrix \isi{clausal} counterpart, as shown in (\ref{ralphgeorge1}):

\ea \label{ralphgeorge1}
\ea	*Ralph bought a bigger cat than George \sout{bought} a wide cat flap.
\ex	*Ralph bought a bigger cat than George did \sout{buy} a wide cat flap.
\z
\z

It seems that the elimination of the \isi{adjective} from the particular position is obligatory. On the other hand, note that the \isi{deletion} of the \isi{lexical verb} is required only if part of the \isi{DP} is overt; if the entire \isi{DP} is eliminated, as in (\ref{bigcat1}), the \isi{lexical verb} can remain:

\ea	Ralph bought a bigger cat than George bought \sout{a big cat}. \label{bigcat1}
\z

There are a number of questions that arise in connection with these phenomena. First, it has to be explained why the \isi{adjective} has to be deleted and cannot appear overtly even if it is contrastive. Second, one has to account for the fact that the \isi{deletion} of the \isi{adjective} happens alongside with the \isi{deletion} of the \isi{lexical verb}: this is interesting especially because in structures like (\ref{acddid}) and (\ref{acdtensed}) the \isi{verb} and the \isi{lexical verb} do not even seem to be adjacent.

In line with \citet{kennedymerchant2000}, \chapref{ch:4} will show that the quantified adjectival phrase moves to a left-peripheral position within the extended \isi{nominal expression} and hence appears as the leftmost element within that \isi{nominal expression}, which results in its adjacency to the \isi{lexical verb} at \isi{PF}. I will argue that the unacceptability of the lexical \isi{AP} in this position is due to a violation of the \isi{Overtness Requirement}: this position within the \isi{nominal expression} is essentially an \isi{operator position}, and therefore \isi{lexical material} is licensed to appear there only if the \isi{operator} is visible, the condition of which is not met in the case of the \isi{comparative operator}. The \isi{ellipsis} mechanism effectively eliminating the \isi{AP} is VP-\isi{ellipsis}, which necessarily affects the \isi{lexical verb}; contrary to \citet{kennedymerchant2000}, who claim that the rest of the \isi{nominal expression} undergoes \isi{rightward movement}, I will argue that the \isi{overtness} of the F-marked \isi{DP} (\textit{a cat flap}) in (\ref{acddid}) and (\ref{acdtensed}) is possible because \isi{ellipsis} proceeds in a strict left to right fashion at \isi{PF} and F-marked constituents may stop \isi{ellipsis}.

In this way, \isi{Attributive Comparative Deletion} will be sufficiently linked to \isi{Comparative Deletion}, as the \isi{deletion} of the higher \isi{copy} takes place even in cases like (\ref{acddid}) and (\ref{acdtensed}); furthermore, the PF-\isi{uninterpretability} underlying both phenomena follows from the same kind of constraint, that is, the \isi{overtness} requirement. On the other hand, VP-\isi{ellipsis} is not a construction-specific mechanism either, and there is no reason to suppose a special process underlying \isi{Attributive Comparative Deletion}.

The analysis of \isi{Attributive Comparative Deletion} will also take cross-linguistic differences into consideration. For instance, in languages like \ili{Hungarian} the full string may be visible in the \isi{subclause}:

\ea \label{acdhung}
\gll Rudolf nagyobb macsk\'at vett, mint amilyen sz\'eles macskaajt\'ot Mikl\'os vett.\\
Rudolph bigger cat.\textsc{acc}	bought.\textsc{3sg} than how wide cat.flap.\textsc{acc}	Mike	bought.\textsc{3sg}\\
\glt `Rudolph bought a bigger cat then Mike did a cat flap.'
\z

I will show that the acceptability of (\ref{acdhung}) in \ili{Hungarian} follows from the fact that the \isi{comparative operator} is overt in \ili{Hungarian} and hence no \isi{Comparative Deletion} is attested at all; on the other hand, the quantified \isi{adjective} does not undergo \isi{movement} to the \isi{left periphery} within the \isi{nominal expression} either.

On the other hand, there are languages, such as \ili{German}, that do not permit \isi{Attributive Comparative Deletion}, even if they have zero comparative operators:

\ea [*] {\gll  Ralf hat eine gr\"o{\ss}ere Wohnung als Michael ein Haus. \label{germanacd}\\
Ralph has a.\textsc{acc.f} bigger.\textsc{acc.f} flat than Michael a.\textsc{acc.n} house\\
\glt `Ralph has a bigger flat than Michael a house.'}
\z

I will show that the unacceptability of (\ref{germanacd}) stems chiefly from the fact that the \isi{VP} (as all \isi{vP} layers) is head-final in \ili{German} and therefore VP-\isi{ellipsis} is not attested; furthermore, the \ili{German} \isi{nominal expression} does not allow the kind of inversion (that is, the \isi{movement} of the quantified \isi{AP} to a left-peripheral position) that can be observed in \ili{English}. In this way, my analysis of \isi{Attributive Comparative Deletion} accounts for cross-linguistic variation, apart from providing an explanation for the \ili{English} data.

Regarding the mechanisms underlying the phenomenon of \isi{Comparative Deletion} and that of \isi{Attributive Comparative Deletion}, it seems that the \isi{Overtness Requirement} regulates the realisation of the higher \isi{copy}, while the realisation of the lower \isi{copy} is essentially tied to the lexical XP being contrastive. In \chapref{ch:5}, I will address the question why some languages cannot realise contrastive lower copies either.

As far as the higher \isi{copy} is concerned, the \isi{Overtness Requirement} on left-peripheral elements is crucial, since this states that overt \isi{lexical material} is licensed in an \isi{operator position} only if the \isi{operator} itself is overt. Hence, there are four logical possibilities, depending on whether the \isi{operator} moves on its own, and whether the \isi{operator} is overt or not. If the \isi{operator} is able to strand a lexical \isi{AP} or \isi{NP} (or there is no lexical XP base-generated together with the \isi{operator} at all), the lexical XP is spelt out in its \isi{base position}, and the \isi{overtness} of the \isi{operator} is immaterial, as is the information structural status of the lexical XP. If an \isi{overt operator} takes the lexical XP along to the [Spec,\isi{CP}] position, the lexical XP is licensed irrespective of its information structural status. However, if a phonologically \isi{zero operator} takes the lexical XP to the \isi{clausal} \isi{left periphery}, the entire phrase in [Spec,\isi{CP}] has to be deleted in order to avoid a violation of the \isi{Overtness Requirement}. In this case, the lower \isi{copy} of the \isi{movement chain} (in the \isi{base position}) is realised overtly if it is contrastive. This leads to an asymmetry between contrastive and non-contrastive XPs: in the case of the latter, the absence of any overt \isi{copy} results in the surface phenomenon traditionally referred to as \isi{Comparative Deletion}. The realisation of contrastive XPs, on the other hand, appears to be straightforward. 

Using data mainly from \ili{Slavic}, \chapref{ch:5} will demonstrate that the availability of the lower \isi{copy} for overt realisation is not universal. Again, the discussion relies on new data that have not been discussed so far but the existence of which is crucial in understanding the idiosyncratic properties of the Standard \ili{English} pattern. Consider the data in (\ref{polish1}) from \ili{Polish}:

\ea \label{polish1}
\ea [*]{\gll Maria jest wyższa niż Karol jest \textbf{wysoki}. \label{polishcd}\\
Mary is taller than Charles is tall\\
\glt `Mary is taller than Charles.'}
\ex [*/??]{\gll Stół jest dłuższy niż biuro jest \textbf{szerokie}. \label{polishsubcomp}\\
desk is longer than office is wide\\
\glt `The desk is longer than the office is wide.'}
\z
\z

While the ungrammaticality of (\ref{polishcd}) is expected on the basis of the \ili{English} pattern, the question arises why \ili{Polish} lacks predicative subcomparatives in the \ili{English} way, that is, why (\ref{polishsubcomp}) is ungrammatical. As will be shown, \ili{Polish} is not unique in this respect: \ili{Czech} shows the same distribution. I will argue that the realisation of the lower \isi{copy} is dependent on more general properties of \isi{movement} chains in a certain language, which results in a difference between \ili{English} and \ili{Polish}. In particular, I will show that the difference between \ili{English} and \ili{Polish} in this respect lies chiefly in the availability of multiple \textit{wh}-fronting in \ili{Polish}. As demonstrated by \citet{boskovic2002}, \textit{wh}-elements have to undergo fronting in multiple \textit{wh}-fronting languages independently of an active [wh] \isi{feature} on C: that is, while the first moved \textit{wh}-constituent checks off the [wh] \isi{feature} on C and thus undergoes ordinary \textit{wh}-\isi{movement}, the further \textit{wh}-elements merely undergo obligatory fronting. I assume that this is because these elements are equipped with an \textsc{edge} \isi{feature}. \citet{boskovic2002} shows that apparent exceptions to the fronting requirement are relatively rare and they are subject to certain conditions; further, these instances do not involve the lack of fronting but rather the realisation of a lower \isi{copy} of a \isi{movement chain}. I argue that since these requirements are absent from comparative constructions, it follows naturally that the realisation of a contrastive lower \isi{copy} is not possible in these languages.

Finally, apart from issues directly related to the structure of \isi{degree} expressions and the functional \isi{left periphery} of comparative subclauses, the present book also aims at accounting for optional \isi{ellipsis} processes that play a crucial role in the derivation of typical comparative subclauses. These issues will be discussed in \chapref{ch:6}.

In \ili{English} predicative structures, see (\ref{ralphjason1}), this involves the elimination of the \isi{copula} from structures such as (\ref{predelliptical}), as opposed to the one given in (\ref{predfull}):

\ea \label{ralphjason1}
\ea	Ralph is more enthusiastic than Jason is. \label{predfull}
\ex	Ralph is more enthusiastic than Jason. \label{predelliptical}
\z
\z

In nominal comparatives, see (\ref{ralphhouses1}), the \isi{lexical verb} may be deleted:

\ea \label{ralphhouses1}
\ea	Ralph bought more houses than Michael bought flats. \label{nomfull}
\ex	Ralph bought more houses than Michael did flats. \label{nomgap}
\ex	Ralph bought more houses than Michael did. \label{nomvpellipsis}
\ex	Ralph bought more houses than Michael. \label{nomellipsis}
\z
\z

Verb \isi{deletion} may result in a \isi{subclause} without any verbal element, as in (\ref{nomellipsis}), or the \isi{tense} morpheme may be carried by the \isi{dummy auxiliary}, as in (\ref{nomgap}) and (\ref{nomvpellipsis}). In addition, depending on whether the object contains a contrastive \isi{noun} or not, the object \isi{nominal expression} remains overt, as in (\ref{nomfull}) and (\ref{nomgap}), or it does not appear overtly, as in (\ref{nomvpellipsis}) and (\ref{nomellipsis}). A very similar pattern arises in attributive comparatives, as shown in (\ref{ralphbighouse1}):

\ea \label{ralphbighouse1}
\ea	Ralph bought a bigger house than Michael did a flat.
\ex	Ralph bought a bigger house than Michael did.
\ex	Ralph bought a bigger house than Michael.	
\z
\z

The main question here is whether the \isi{deletion} of the \isi{lexical verb} is merely the \isi{deletion} of the verbal head or whether this involves VP-\isi{ellipsis}; in the latter case, the possibility of having overt objects (or parts of objects) must be accounted for. Developing the analysis given in \chapref{ch:4}, in \chapref{ch:6} I will argue that \isi{gapping} is an instance of VP-\isi{ellipsis}, which proceeds in a left-to-right fashion at \isi{PF} and the starting point of it is an [E] \isi{feature} on a functional v head, in line with \citet{merchant2001}, and the endpoint of \isi{ellipsis} is a contrastive phrase, if there is any. I will also show that since the [E] \isi{feature} can be present on a C head as well, the derivation of comparative subclauses at \isi{PF} may involve \isi{ellipsis} starting from an [E] \isi{feature} either on a C or a v head. Since the final string may be ambiguous, one of the central questions is whether a uniform kind of \isi{ellipsis} mechanism may account for these ambiguities; this will be shown to be possible.

On the other hand, the fact that reduced comparative subclauses also exist in \ili{Hungarian} raises yet another question, which is how languages with exclusively overt comparative operators may show the elimination of the entire \isi{degree expression}, given that there is no \isi{Comparative Deletion} in these languages. For instance, predicative comparatives in \ili{Hungarian} show the pattern in (\ref{maripred1}):

\ea \label{maripred1}
\ea \gll	Mari	magasabb volt, mint \textbf{amilyen} \textbf{magas} Péter \textbf{volt}. \label{hungpredfullellipsis}\\
Mary taller	was.\textsc{3sg} than	how	tall	Peter was.\textsc{3sg}\\
\glt `Mary was taller than Peter.'
\ex	\gll Mari	magasabb volt, mint	Péter. \label{hungpredelliptical}\\
Mary taller	was.\textsc{3sg} than	Peter\\
\glt `Mary was taller than Peter.'
\z
\z

As can be seen, in (\ref{hungpredfullellipsis}) the \isi{subclause} contains all the elements overtly, while the \isi{degree expression} and the \isi{verb} are absent from (\ref{hungpredelliptical}). The same phenomenon can be observed in nominal comparatives, see (\ref{maricats1}):

\largerpage[-2]
\ea \label{maricats1}
\ea \gll Mari	több macskát vett, mint	\textbf{ahány} \textbf{macskát} Péter \textbf{vett}.\\
Mary more	cat.\textsc{acc} bought.\textsc{3sg} than	how.many cat.\textsc{acc} Peter bought.\textsc{3sg}\\
\glt `Mary bought more cats than Peter did.'
\ex \gll Mari	több macskát vett, mint	Péter.\\
Mary more	cat.\textsc{acc} bought.\textsc{3sg} than Peter\\
\glt `Mary bought more cats than Peter did.'
\z
\z

Finally, the same is true for attributive comparatives, as shown in (\ref{maribigcat1}):

\ea \label{maribigcat1}
\ea	\gll Mari nagyobb macskát vett, mint \textbf{amilyen} \textbf{nagy} \textbf{macskát} Péter \textbf{vett}.\\
Mary bigger	cat.\textsc{acc} bought.\textsc{3sg} than how big cat.\textsc{acc} Peter bought.\textsc{3sg}\\
\glt `Mary bought a bigger cat than Peter did.'
\ex \gll Mari nagyobb	macskát	vett,	mint Péter.\\
Mary bigger	cat.\textsc{acc} bought.\textsc{3sg} than	Peter\\
\glt `Mary bought a bigger cat than Peter did.'
\z
\z

In all of these cases it is true that the sentences of a given pair have the same meaning. The question is whether the \isi{deletion} of the \isi{degree expression} is independent from that of the \isi{verb} or not. As I will show in \chapref{ch:6}, using novel and systematically tested data, these are not two independent processes since the \isi{verb} cannot be overt in the absence of an overt \isi{degree expression}. I will argue that this is the case because it is ungrammatical to have an \isi{operator} in its \isi{base position} in \ili{Hungarian}, but since there is no separate mechanism that would eliminate the \isi{degree expression}, a more general \isi{ellipsis} process has to apply, which is essentially VP-\isi{ellipsis}. The \isi{ellipsis} mechanism is fairly similar to the one attested in \ili{English} and the differences will be linked to the slightly different internal structure of the functional layers in the two languages. Otherwise \isi{ellipsis} is carried out by an [E] \isi{feature} on the leftmost \isi{functional head} in \ili{Hungarian} too.

I will argue that the difference between \ili{English} and \ili{Hungarian} in terms of \isi{gapping} effects is chiefly a result of the different prosody in the two languages: while the \isi{Intonational Phrase} is right-headed in \ili{English}, it is left-headed in \ili{Hungarian}. As a consequence, while contrastive elements are located at the \isi{right edge} of the \isi{ellipsis domain} in \ili{English}, in \ili{Hungarian} they are to the left of the \isi{functional head} hosting the [E] \isi{feature} or are themselves located in that head and consequently not part of the \isi{ellipsis domain} either. \chapref{ch:6} will show that since there is strong directionality in terms of \isi{ellipsis}, in that it proceeds in a strict left-to-right fashion, this kind of \isi{ellipsis} works only in head-initial phrases since the \isi{ellipsis domain} (the complement) has to follow the head hosting the [E] \isi{feature}. This accounts for why \ili{German} does not have VP-\isi{ellipsis} the way \ili{English} has it: the \ili{German} \isi{VP} and all \isi{vP} layers are head-final, while in \ili{English} all \isi{VP} projections are head-initial. Cross-linguistic differences concerning optional \isi{ellipsis} processes can thus also be reduced to more general properties that hold in individual languages, and hence \isi{ellipsis} processes in comparatives are not construction-specific.

