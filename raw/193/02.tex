\chapter{The structure of degree expressions} \label{ch:2}
\section{Introduction} \label{sec:2introduction}
In this chapter, I provide a unified analysis of \isi{degree} expressions that relates the structure of comparatives to that of other -- absolute and superlative -- degrees. Naturally, there arise a number of questions concerning the general structure of \isi{degree} phrases, of which I will select only the ones that are relevant for the present work. Since my analysis is strongly built on the results of previous accounts, I will first give a short overview of the relevant literature, showing the problematic points thereof that I intend to eliminate in my approach. Again, the literature concerning the syntax of \isi{degree} expressions is far greater than the selected examples presented here but I restrict myself to discussing those analyses that bear crucial significance for the understanding of comparatives.

\section{Earlier accounts} \label{sec:2earlier}
\subsection{The problems to be discussed} \label{sec:2theproblems}
When considering the general structure of \isi{degree} expressions, comparatives are especially interesting to consider because they contain a number of overt elements that clearly indicate the presence of various functional layers, presenting a challenge for previous analyses.

The very first problem is the appearance of the \isi{degree morpheme} itself. Consider the examples in (\ref{marydeg2}):

\ea \label{marydeg2}
\ea Mary is \textbf{tall}.\label{marytall}
\ex Mary is \textbf{taller} than John.\label{marytaller}
\z
\z

By comparing (\ref{marytall}) and (\ref{marytaller}), it should be obvious that while it is the very same lexical \isi{adjective} (\textit{tall}) that appears in both cases, in (\ref{marytaller}) there is an additional \isi{degree morpheme} (that is, -\textit{er}). The fact that the \isi{degree marker} is syntactically separate from the \isi{adjective} is more clearly indicated by periphrastic comparatives such as (\ref{compmore}):

\ea Mary is \textbf{more intelligent} than John.\label{compmore}
\z

In (\ref{compmore}), the \isi{comparative degree} is marked by \textit{more}; a sound analysis for the structure of \isi{degree} expressions must also account for the difference and the relatedness of structures like (\ref{marytaller}) and (\ref{compmore}).

Moreover, the relation between the comparative \isi{degree marker} and the comparative \isi{subclause} must also be explained as the type of the \isi{subclause} seems to be defined by the comparative marker in the \isi{matrix clause}:

\ea \label{degreemarker}
\ea []{Mary is \textbf{taller} [than John].}
\ex [*]{Mary is \textbf{taller} [as John].}
\ex [*]{Mary is \textbf{as tall} [than John].}
\ex []{Mary is \textbf{as tall} [as John].}
\z
\z

As can be seen, if the \isi{degree expression} in the \isi{matrix clause} contains the morpheme -\textit{er}, then the \isi{subclause} must be introduced by \textit{than}; conversely, a \isi{degree expression} with \textit{as} in the \isi{matrix clause} requires a \isi{subclause} introduced by \textit{as}. These selectional restrictions are obviously not dependent on the lexical \isi{adjective}, which is \textit{tall} in all of the examples in (\ref{degreemarker}).

Last but not least, adjectives may have arguments of their own. Consider:

\ea Mary is proud [of her husband].\label{ppargument}
\z

The \isi{adjective} \textit{proud} takes the \isi{PP} as its complement; this must also be accounted for, especially in relation to the subclauses indicated in (\ref{degreemarker}), which are not directly introduced by the \isi{adjective} itself but are nevertheless obligatory.

\subsection{\textit{Much}-deletion -- \citet{bresnan1973}} \label{sec:2muchdeletion}
I will start the overview with Bresnan's landmark paper, which opened the discussion on comparative constructions by taking into account a large number of phenomena not even considered before. The most important contribution of \citet{bresnan1973} is probably the separation of functional heads (Det and Q in her analysis, Deg and Q in later analyses), which makes it possible to explain why certain degree-like elements behave differently; moreover, the role of \textit{much} is also addressed, which is crucial in terms of the structure of comparatives.

One of the most important observations is that \textit{more} is a composite of \textit{much} and the \isi{degree morpheme} -\textit{er}, hence in a way the comparative form of \textit{much}. This is immediately shown by the paradigm of \isi{degree} expressions. Consider the following examples (taken from \citealt[277, exx. 4 and 5]{bresnan1973}):

\ea \label{paradigm}
\ea	as / too / that / so \textbf{much} bread
\ex	as / too / that / so \textbf{little} bread
\ex	as / too / that / so \textbf{many} people
\ex	as / too / that / so \textbf{few} people
\z
\z

As can be seen, all \isi{degree} elements (i.e., \textit{as}, \textit{too}, \textit{that} and \textit{so}) combine with either \textit{much}, \textit{little}, \textit{many} or \textit{few}. Likewise with -\textit{er}, we find all four forms as shown in (\ref{moreless}), cf. \citet[277, exx. 4, 5 and 7]{bresnan1973}:

\ea \label{moreless}
\ea -er \textbf{much} bread $\rightarrow$ \textbf{more} bread
\ex	-er \textbf{little} bread $\rightarrow$ \textbf{less} bread
\ex	-er \textbf{many} people $\rightarrow$ \textbf{more} people
\ex	-er \textbf{few} people $\rightarrow$ \textbf{fewer} people
\z
\z

Naturally, there must be rules in the grammar for the changes from combinations such as -\textit{er much} into \textit{more}: these are partly syntactic rules and partly suppletion rules that belong to the level of morphology (\citealt[279]{bresnan1973}).

The structure of \isi{degree} expressions can be drawn up as given in (\ref{treebresnan2}), according to \citet[277, ex. 6]{bresnan1973}:

\ea \label{treebresnan2} \upshape 
\begin{forest} baseline, qtree, for tree={align=center} 
[QP
	[\phantom{xxxx}Det\phantom{xxxx} 
		[as\\too\\that\\so\\-er, name=node 1]
	] 
	[Q
		[much\\many\\little\\few, name=node 2]
	]
]
\path [draw, decorate, decoration={brace, amplitude=10pt, mirror}] (node 1.north west) -- (node 1.south west);
\path [draw, decorate, decoration={brace, amplitude=10pt}] (node 1.north east) -- (node 1.south east);
\path [draw, decorate, decoration={brace, amplitude=10pt, mirror}] (node 2.north west) -- (node 2.south west);
\path [draw, decorate, decoration={brace, amplitude=10pt}] (node 2.north east) -- (node 2.south east);
\end{forest}
\z

Degree expressions like \textit{as much} are QPs, though \citet[277]{bresnan1973} admits that the label ``is merely a temporary convenience''. The head of the \isi{QP} is occupied by the elements \textit{much}, \textit{many}, \textit{little} and \textit{few}, while the \isi{degree} elements -- including the comparative -\textit{er} -- are determiners in the specifier. Admittedly, the analysis has the advantage of ruling out certain impossible configurations such as *\textit{too more}: the Det position cannot be filled by \textit{too} and -\textit{er} at the same time (\citealt[277]{bresnan1973}), which would not be predicted by an analysis taking elements like \textit{more} as atomic.

Let us now turn to cases where \isi{degree} elements are followed by a lexical \isi{adjective} (or \isi{adverb}) and not a \isi{noun}. The paradigm given in (\ref{paradigm}) does not seem to hold there, as the data in (\ref{bresnanmary2}) indicate (see \citealt[278, exx. 8 and 9]{bresnan1973}):

\ea \label{bresnanmary2}
\ea	[]{Mary is \textbf{more} intelligent.} \label{moreadj}
\ex	[*]{Mary is \textbf{so much} intelligent.} \label{somuchadj}
\ex []{Mary speaks \textbf{more} cogently.} \label{moreadv}
\ex [*]{Mary speaks \textbf{so much} cogently.} \label{somuchadv}
\z
\z

The data above show the following problem: apparently, the sequence of a \isi{degree element} (e.g., \textit{so}) and \textit{much} before an \isi{adjective} or an \isi{adverb} is not permitted, as indicated by the ungrammaticality of (\ref{somuchadj}) and (\ref{somuchadv}). However, \textit{more} is acceptable in that position, as in (\ref{moreadj}) and (\ref{moreadv}). Therefore, if one maintains the idea that \textit{more} is made up of -\textit{er} and \textit{much} in the same way as, for example, \textit{so much} is constructed, then there are obviously conflicting requirements here.

\citet[278]{bresnan1973} mentions two logical possibilities that may account for this: either \textit{more} does not derive from -\textit{er} + \textit{much} when preceding adjectives and adverbs, or it is deleted if it directly precedes an \isi{adjective} or an \isi{adverb}. Arguing for the latter, she provides an additional rule in the form of \textit{Much}-\isi{deletion}, given below (\citealt[278, ex. 10]{bresnan1973}):

\ea \upshape much $\rightarrow$ $\emptyset$ / [\ldots \underline{\hspace{1cm}} A]\textsubscript{AP}\\where A(P) = Adjective or Adverb (Phrase) \label{muchdeletion}
\z

The fact that -\textit{er much} becomes \textit{more} is not merely a morphological matter: the syntax accounts for the \isi{word order} change from the initial -\textit{er much} into \textit{much -er}, and morphology substitutes this latter form with \textit{more}. According to \citet[279, ex. 20]{bresnan1973}, the syntactic derivation is the following:

\setlength\columnsep{-6cm}
\begin{multicols}{2}
\ea \label{muchertree}
\ea \upshape
\begin{forest} baseline, qtree, for tree={align=center} 
[QP
	[Det
		[-er]
	]
	[Q
		[much]
	]
]
\end{forest}
$\Rightarrow$
\ex \upshape
\begin{forest} baseline, qtree, for tree={align=center} 
[QP
	[Det
		[$\emptyset$]
	]
	[Q
		[much+er,roof]
	]
]
\end{forest}
\z
\z

\end{multicols}

\setlength\columnsep{0cm}

By way of cliticisation, -\textit{er} is attached to \textit{much}, ultimately resulting in \textit{more}. The point is that \textit{much} will not be adjacent to the \isi{adjective} following the original string -\textit{er much}. The item -\textit{er} will act as an intervener and consequently the rule given in (\ref{muchdeletion}) does not -- and could not -- apply. This is straightforward in the case of analytic comparatives (such as \textit{more intelligent}) but requires extra rules for accommodating morphological comparatives (such as \textit{taller}). \citet[279]{bresnan1973} assumes that \textit{taller} is in fact underlyingly \textit{more tall}, and is derived by separate rules for simple comparatives: first, \textit{much-er tall} becomes \textit{much-er taller}, and subsequently \textit{much-er} is deleted, leaving \textit{taller} as the final result. As far as the exact mechanism behind this is concerned, it is crucially missing from the analysis.

Turning back to the syntax of \isi{degree} expressions, (\ref{muchertree}) shows that the core idea is to treat \textit{much} or \textit{many} as a Q head, which takes a Det \isi{degree} item as a specifier. If a \isi{degree expression} is modified by another one, this is achieved via adjunction. Consider the following example (\citealt[290, ex. 132a]{bresnan1973}):

\ea I have \textbf{as many too many} marbles as you. \label{asmanytoomany}
\z

Here the \isi{degree expression} \textit{too many} is modified by \textit{as many}. As shown in (\ref{treebresnanqp2}), the latter is left-adjoined to the former (\citealt[290, ex. 131]{bresnan1973}):

\ea \label{treebresnanqp2} \upshape
\begin{forest} baseline, qtree, for tree={align=center}
[QP
	[QP
		[(QP)]
		[QP
			[Det [as]]
			[Q [many]]
		]
	]
	[QP
		[Det
			[too]
		]
		[Q
			[many]
		]
	]
]
\end{forest}
\z

A \isi{QP} can be modified by another \isi{QP} in a recursive way: additional QPs are adjoined in the same fashion. If there is also a lexical \isi{adjective} (or \isi{adverb}), the \isi{QP} is left-adjoined to it (\citealt[294, ex. 147]{bresnan1973}):

\ea \upshape \label{asmuchclear}
\begin{forest} baseline, qtree, for tree={align=center}
[AP
	[QP
		[QP
			[Det [as]]
			[Q [much]]
		]
	]
	[AP
		[A
			[clear]
		]
	]
]
\end{forest}
\z

The representation in (\ref{asmuchclear}) shows the underlying structure: \textit{much}-\isi{deletion} will later eliminate \textit{much}, which is immediately followed by an \isi{adjective}, ultimately giving the grammatical string \textit{as clear}. The same would be true if the \isi{adjective} had an adverbial \isi{modifier} (e.g., \textit{as much utterly stupid} $\rightarrow$ \textit{as utterly stupid}, see \citealt[294]{bresnan1973}).

As for the comparative \isi{subclause}, \citet[318--319]{bresnan1973} notes that it may originate in the Det (dominating the -\textit{er} or the \textit{as} head); however, how this is precisely achieved is not described. In the final structure, the comparative \isi{subclause} ends in an extraposed position, as shown in (\ref{compsubclausetree}) for the string \textit{taller than my mother is tall}, see \citet[319, ex. 251]{bresnan1973}:

\ea \upshape \label{compsubclausetree}
\begin{forest} baseline, qtree, for tree={align=center}
[AP
	[AP
		[QP
			[QP [Det [-er]] [Q [much]]]
		]
		[AP
			[A [tall]]
		]
	]
	[S
		[COMP
			[than]
		]
		[S
			[NP [my mother,roof]]
			[VP [Cop [is]] [Pred [QP [QP [Det [x]] [Q [much]]]] [AP [A [tall]]]]]
		]
	]
]
\end{forest}
\z

As can be seen, the \isi{subclause} is ultimately an adjunct to the entire \isi{AP}, though it should be base-generated where the Det is located.

Though the analysis admittedly has advantages, it raises a number of problems as well. First of all, the structural representation can obviously not be maintained in a minimalist framework, especially as far as the Det is concerned. If elements like -\textit{er} are indeed to be treated as heads and not as phrase-sized constituents, they should not be located in a specifier. This immediately raises the question of where \isi{degree} items are located with respect to the \isi{AP} and the \isi{QP}; that is, which projection dominates which. If the \isi{degree} item is indeed a head, rather than a phrase, it is highly unlikely that it would be dominated by the \isi{AP}, unless extra \isi{movement} processes are involved.

It is likewise problematic to relate QPs to each other by way of adjunction. It is true that \isi{QP} modifiers are to a large extent recursive but certain restrictions seem to hold on their order, for instance, while \textit{as many too many (marbles)} is grammatical, *\textit{too many as many (marbles)} is not.

Moreover, the very mechanism of \textit{much}-\isi{deletion} is highly questionable. It is credible that the formation of \textit{more} before adjectives and adverbs should not differ from how it is formed before nouns. However, by merely considering the logical possibilities, this leaves us two alternative options and not just one, as \citet{bresnan1973} would imply. The first option is \textit{much}-\isi{deletion} before adjectives and adverbs. The second option is \textit{much}-insertion elsewhere. The former option has two main problems: first, it is not clear why \textit{much} should be inserted even when it lacks the syntactic function of a dummy and does not bear any semantic role. Second, the rule of \textit{much}-\isi{deletion} is highly arbitrary (cf. also \citealt{corver1997, jackendoff1977, brame1986}) and does not follow from any general constraint. It is therefore a rather circuitous way of defining the morphological difference between adjectives that form their comparative degrees with \textit{much} and those that do not.

Last but not least, the position of the comparative \isi{subclause} also raises at least two major questions. On the one hand, it remains unexplained how it is base-generated under the Det node. On the other hand, the \isi{extraposition} of the clause to the right is also dubious, primarily because it seems to be obligatory \isi{rightward movement}, in addition to the fact that \isi{rightward movement} in itself is problematic. As there is very little said about the position of the comparative \isi{subclause}, it is not surprising that the issue is not discussed in relation to \isi{PP} arguments of adjectives, which should also be accommodated in the structure.

\largerpage[1]
\subsection{A DP-shell for comparatives -- \citet{izvorski1995}} \label{sec:2dpshell}
Let us now turn to the analysis of \citet{izvorski1995}, who markedly builds on the semantics of comparative structures with respect to the formation of the syntactic structure. The importance of this study lies fundamentally in the fact that it aims at providing a unified syntactic representation for \isi{degree} expressions, which will also play a crucial part in later analyses. By way of adopting a DP-shell analysis, \citet{izvorski1995} intends to provide a unified structure for predicative and nominal structures, which is desirable in the sense that the \isi{degree expression} itself should not be different depending on whether it is a \isi{predicate} or base-generated within a \isi{nominal expression}.

According to \citet[107--118]{izvorski1995}, the elements \textit{more}, \textit{less} and \textit{as} are of the category Det, and they are heads of the \isi{DP} they introduce. In this way, DPs have in fact two \isi{DP} layers (hence the DP-shell), in the same way as double object constructions have VP-shells, cf. \citet{larson1988}. It has to be mentioned that the label D for \isi{degree} items is fundamentally used as a convenient syntactic notation and is therefore not intended to imply that all \isi{degree} expressions would be nominal (\citealt[111--119]{izvorski1995}): they could also be of the category Deg, as for \citet{abney1987diss} and \citet{corver1990diss}.

According to \citet[107--118, see especially ex. 23]{izvorski1995}, the general structure of comparatives is the following:

\ea \upshape \label{izvorskitree}
\begin{forest} baseline, qtree, for tree={align=center}
[DP
	[D$'$
		[D
			[more/less/as\textsubscript{i}]
		]
		[DP
			[XP]
			[D$'$ [D [t\textsubscript{i}]] [PP [than/as \ldots,roof]]]
		]
	]
]
\end{forest}
\z

The XP stands for the lexical projection -- a bare \isi{AP} or \isi{NP} -- in the structure; in this way, there is no syntactic difference between predicative (e.g. \textit{more intelligent than \ldots} and nominal (e.g. \textit{more cats than \ldots}) comparatives, other than the category of the XP itself.

As \citet[109--119]{izvorski1995} points out, the analysis has the advantage of both directly relating the \isi{degree element} -- that is, \textit{more}, \textit{less} or \textit{as} -- to the \isi{comparative complement} (here: the \isi{PP}) and at the same time accounting for their discontinuity in the surface structure. Yet, this immediately raises the problem of distributional differences, as \isi{degree} expressions containing an \isi{AP} and those containing an \isi{NP} clearly do not behave in the same way syntactically. \citet[111--120]{izvorski1995} overcomes this by saying that D is underspecified for the relevant (nominal or adjectival/adverbial) features; hence, it can take either of them into its (lower) specifier. Via specifier–head \isi{agreement}, the XP is in turn responsible for specifying these features on the D head; finally, the \isi{movement} of the D to the higher D position causes the features to be present on the entire \isi{DP}.

This analysis clearly eliminates some of the problems that I mentioned in connection with \citet{bresnan1973}, such as the treatment of Det as a specifier, the mechanism of \textit{much}-\isi{deletion} or the connection between the comparative \isi{subclause} and the \isi{degree head}. However, new ones arise as well, in particular the treatment of \textit{more} and \textit{less} as atomic: apart from the fact that there seems to be ample evidence in favour of analysing \textit{more} as \textit{much} + -\textit{er}, Izvorski's proposal crucially leaves unexplained how simple morphological comparatives (e.g. \textit{taller}) are formed.

In addition, the way to overcome distributional differences is ad hoc and does not take into account that there might be differences in terms of \isi{modification},  too. As a matter of fact, the issue of \isi{modification} is altogether missing from Izvorski's analysis (consider examples such as (\ref{asmanytoomany}) above). The same applies to the position of arguments, especially the \isi{PP} arguments of adjectives.

Moreover, while the account in \citet{izvorski1995} is general enough in the sense that it covers (or intends to cover) the structure of both predicative and nominal comparatives, it fails to say anything about attributive comparatives (e.g., \textit{a more intelligent dog than \ldots}). As has been said, the XP is either a bare \isi{AP} or a bare \isi{NP}. It is not clear how an \isi{NP} containing an attribute could be accommodated in the structure, especially because in these cases the \isi{comparative degree} is associated primarily with the lexical \isi{AP} and not with the entire \isi{NP}, which becomes even more evident when considering attributive comparatives containing a morphological \isi{degree} form (e.g. \textit{a bigger dog than \ldots}), where the \isi{degree morpheme} -\textit{er} is clearly marked on the \isi{adjective}.

Last but not least, the treatment of the \isi{subclause} is highly questionable: apart from the fact that \citet{izvorski1995} analyses it as a \isi{PP}, an issue I intend to address later on, there seems to be a problem in terms of \isi{extraposition}, too. At first glance, the kind of \isi{extraposition} proposed by \citet{bresnan1973} seems to be fortunately eliminated by \citet{izvorski1995}: it is the \isi{degree element} that moves away from the \isi{subclause}. However, it has to be noted that the position given in (\ref{izvorskitree}) cannot be the final one. Consider the examples in (\ref{brenda2}):

\ea \label{brenda2}
\ea Brenda is \textbf{more enthusiastic} now [than she used to be].
\ex \textbf{More students} like Brenda's classes [than George's].
\z
\z

As can be seen, the bracketed comparative subclauses are separated by intervening material not only from the \isi{degree element} \textit{more} but also from the lexical \isi{AP} (\textit{enthusiastic}) or \isi{NP} (\textit{students}). Therefore, its final position cannot be within the \isi{degree expression}, that is, the \isi{DP} in Izvorski's analysis.

\subsection{\textit{Much}-support -- \citet{corver1997}} \label{sec:2muchsupport}
Let us now turn to the analysis presented by \citet{corver1997}, which is a landmark paper in terms of functional projections in the extended \isi{AP}, primarily because it makes an important distinction between determiner-like and quantifier-like \isi{degree} items in a more explicit way than \citet{bresnan1973} did. In addition, \citet{corver1997} adopts a \isi{functional head} approach instead of a lexical head approach, which conforms to the general assumption that it is functional projections that dominate lexical ones and not vice versa. Last but not least, by claiming that the presence of \textit{much} is due to insertion, \citet{corver1997} presents a theoretically more adequate treatment of \textit{much} than the one given by \citet{bresnan1973}, which included an extra \isi{deletion} operation from the structures without a visible \textit{much}.

Relying on \citet{bresnan1973}, \citet[120--123]{corver1997} starts from the split \isi{degree} hypothesis; that is, the idea that there should be a difference between quantifier-like \isi{degree} items (QPs) and determiner-like \isi{degree} items (DegPs). According to this, the general structure of \isi{degree} expressions should be the following:

\ea \upshape \label{corvertree}
\begin{forest} baseline, qtree, for tree={align=center}
[DegP
	[Deg$'$
		[Deg]
		[QP
			[Q$'$ [Q] [AP [A$'$ [A]]]]
		]
	]
]
\end{forest}
\z

Contrary to \citet{bresnan1973}, however, \citet[122--123]{corver1997} treats the items \textit{more} and \textit{less} as atomic, in the sense that they are claimed to be base-generated as such -- similarly to \textit{enough} or the dummy \isi{quantifier} \textit{much} -- and not as the results of syntactic derivation.

Note that the structure proposed by \citet{bresnan1973}, as given in (\ref{asmuchclear}), is crucially different from the one shown in (\ref{corvertree}). The former is a lexical head approach, in that the entire \isi{degree expression} is headed by the lexical A head, whereas the latter is a \isi{functional head} approach, where the \isi{AP} is dominated by functional layers in the \isi{degree expression}.

There are reasons to believe that this is indeed the case. First, as pointed out by \citet[124--125]{corver1997}, the syntactic derivation of morphological comparatives (e.g. \textit{taller}) would be problematic if the bound -\textit{er} morpheme were located in the specifier of the \isi{AP}. In order to derive \textit{taller}, either -\textit{er} would have to move rightward or the \isi{adjective} would have to move to its own specifier -- in both cases, general constraints on \isi{movement} would be violated. By \isi{contrast}, under the \isi{functional head} approach the \isi{adjective} head can move up to the \isi{functional head} -\textit{er}. Note that this is a problem only if one assumes that the derivation of the final string \textit{taller} from the underlying -\textit{er tall} is carried out in syntax; as will be shown later on, this is not necessarily the case.

Second, the lexical head approach would face severe problems in connection with differences like (\ref{corverextr}), see \citet[125, exx. 16c and 17c]{corver1997}:

\ea \label{corverextr}
\ea *\textbf{How\textsubscript{i}} do you think he is [\emph{t}\textsubscript{i} dependent on his sister]?
\ex	\textbf{How heavily\textsubscript{i}} do you think he is [\emph{t}\textsubscript{i} dependent on his sister]?
\z
\z

As can be seen, it is grammatical to extract a phrase such as \textit{how heavily} from within the \isi{degree expression}, while the \isi{extraction} of \textit{how} is banned. The difference could not be explained under the lexical head approach, where \textit{how} and \textit{how heavily} would both be phrase-sized specifiers (QPs) within an \isi{AP}. In Corver's approach, however, only the latter qualifies as a phrase-sized constituent: \textit{how} in itself is a \isi{functional head} above the \isi{AP} and therefore it is straightforward that it cannot move out on its own.

Third, \citet[125, ex. 18]{corver1997} also calls attention to an interesting \isi{extraction} paradigm, given in (\ref{corverparadigm2}):

\ea \label{corverparadigm2}
\ea [?]{\textbf{How many IQ-points\textsubscript{i}} is John [\emph{t}\textsubscript{i} less smart (than Bill)]?} \label{howmany}
\ex	[*]{\textbf{How many IQ-points less\textsubscript{i}} is John [\emph{t}\textsubscript{i} smart (than Bill)]?} \label{howmanyless}
\ex []{[\textbf{How many IQ-points less smart} (than Bill)] is John?} \label{howmanylessap}
\z
\z

As pointed out by \citet[125--126]{corver1997}, the lexical head approach would have to face the problem of extracting phrases from a specifier position both in (\ref{howmany}) and (\ref{howmanyless}), though the latter case is clearly ungrammatical. The \isi{functional head} approach can handle this too: in (\ref{howmany}), a \isi{degree expression} (\textit{how many IQ-points}) is moved out of a specifier position from within the \isi{degree expression} headed by \textit{less}; by \isi{contrast}, (\ref{howmanyless}) exhibits the \isi{movement} of non-constituents, that is, of a phrase-sized specifier and the \isi{functional head}. Naturally, the \isi{movement} of the entire \isi{degree expression} headed by \textit{less} is again grammatical, see (\ref{howmanylessap}).

Returning to the problem concerning the status of \textit{much}, it has to be mentioned that \citet[123]{corver1997} makes a crucial distinction between the lexical \isi{quantifier} \textit{much} and the functional dummy \isi{quantifier} \textit{much}. An example of the first one is given in (\ref{toomuchtootall2}) below (based on \citealt[121, ex. 5]{corver1997}):

\ea She is \textbf{too much too tall}. \label{toomuchtootall2}
\z

In this case, the element \textit{much} is claimed to be located in a specifier position of the extended \isi{AP} projection (\citealt[123]{corver1997}). By \isi{contrast}, dummy \textit{much} is a Q head in the extended \isi{AP} and is found in examples such as (\ref{toomuchso}) below, see \citet[123, ex. 11]{corver1997}:

\ea John is fond of Sue. Maybe \textbf{too much so}. \label{toomuchso}
\z

The appearance of dummy \textit{much} is, according to \citet[123]{corver1997}, due to last resort insertion as the \isi{adjective} in these cases does not move up to the Q head position. In other words, syntax crucially derives the structure without \textit{much} and insertion happens only if necessary: this is exactly the opposite of what \citet{bresnan1973} claimed; that is, that the syntactic derivation by default contains \textit{much} and a later rule may delete it. As was mentioned at the end of \sectref{sec:2muchdeletion}, the possibility of inserting dummy \textit{much} is in fact logically plausible, even though \citet{bresnan1973} does not take it into consideration. In a way, \citet{corver1997} seems to answer one of the most compelling questions that arise in connection with the analysis given by \citet{bresnan1973}.

Moreover, \citet[126--128]{corver1997} provides evidence for the existence of the QP-layer, which was only rather intuitively proposed by \citet{bresnan1973}. Consider the following examples in (\ref{johnfond2}) below (\citealt[126, exx. 20a and 21a]{corver1997}):

\ea \label{johnfond2}
\ea John seems fond of Mary, and Bill seems so too.
\ex	John is fond of Mary. Bill seems [much less so]. \label{lessso}
\z
\z

Both cases are instances of \textit{so}-pronominalisation: \textit{so} replaces the entire \isi{AP} \textit{fond of Mary} and, as \citet[126]{corver1997} argues, not merely the \isi{adjective} \textit{fond} and not the entire \isi{degree expression} either, as indicated by the fact that in (\ref{lessso}) \textit{so} appears in a \isi{degree expression} headed by \textit{less}. This could still be accommodated in a system using only a \isi{DegP} above the \isi{AP}; but consider the data given in (\ref{johnfondmary2}), taken from \citet[127, exx. 23a and 24a]{corver1997}:

\ea \label{johnfondmary2}
\ea John is fond of Mary. *Maybe he is [too so].
\ex	John is fond of Mary. Maybe he is [too \textbf{much} so].
\z
\z

As can be seen, the string \textit{too so} is not grammatical: \textit{much} has to be inserted into the structure. This can be handled relatively well if one assumes a structure like (\ref{corvertree}), where the Deg head would be \textit{too}, the Q head \textit{much} and the element \textit{so} would occupy the position of the \isi{AP}, see \citet[127--128]{corver1997}.

Contrary to \citet{bresnan1973}, \citet[128--129]{corver1997} argues that the Q head position is underlyingly empty and the insertion of \textit{much} is only a last resort option: the insertion of \textit{much} in all cases would violate general principles of economy. In this way, \textit{much}-support is similar to \textit{do}-support in the extended verbal domain, as described by \citet{chomsky1991}; see \citet[129]{corver1997}.

As for the position of modifiers, \citet[154--161]{corver1997} argues that they are located in the specifier position of the \isi{QP}. Consider:

\ea {[}\textsubscript{QP} extremely \emph{e} [\textsubscript{AP} poisonous]] \label{extremelyap}
\z

Under this approach, modifiers such as \textit{extremely} are located in the [Spec,\isi{QP}]; the Q head is empty. By \isi{contrast}, though modifiers like \textit{well} or \textit{far} are likewise located in [Spec,\isi{QP}], they attract the \isi{adjective} head to move up to the Q head, see \citet[160]{corver1997}:

\ea {[}\textsubscript{QP} far different\textsubscript{i} [\textsubscript{AP} \emph{t\textsubscript{i}} from the others]] \label{farap}
\z

\citet[160]{corver1997}, in line with \citet{larson1987}, assumes that the morpheme -\textit{ly} is a case-marking element and that the \isi{AP} needs to be assigned Case. Hence, while in (\ref{extremelyap}) the morpheme -\textit{ly} can assign Case to the \isi{AP} in situ, in (\ref{farap}) there is no -\textit{ly} morpheme and the \isi{AP} can get Case only via \isi{movement} to the specifier of the \isi{QP}.

Although Corver's analysis is in many respects attractive, it still raises certain problems. The most evident one is perhaps the treatment of modifiers. It is not clear why the \isi{AP} should be assigned Case at all, and how case assignment can be linked to the -\textit{ly} morpheme. More importantly, the distinction between elements like \textit{far} and ones like \textit{extremely} is not as simple as it may seem on the basis of \citet{corver1997}. Consider the examples in (\ref{maryagatha2}):

\ea \label{maryagatha2}
\ea	[*]{Mary is \textbf{far tall}.}
\ex	[]{Mary is \textbf{far taller} (than Agatha).} \label{fartaller}
\ex	[]{Mary is \textbf{very/extremely tall}.}
\ex	[*]{Mary is \textbf{very/extremely taller} (than Agatha).}
\z
\z

The data above show that the modifiers \textit{far} and \textit{extremely} do not appear in the same constructions. While \textit{extremely} appears regularly with the \isi{absolute degree} (e.g. \textit{tall}), and therefore patterns with \textit{very}, \textit{far} normally occurs when the \isi{degree expression} is comparative (e.g. \textit{taller}). The exceptional case is actually the one that \citet{corver1997} uses for his analysis, namely the possibility of \textit{far different}; I will return to the question of why \textit{different} patterns with \isi{comparative degree} expressions rather than absolute ones later, but the basic claim will be that \textit{different} is inherently comparative.

At any rate, there seems to be a crucial distinction among modifiers in terms of which \isi{degree} they co-occur with. This difference remains unobserved and hence unexplained by \citet{corver1997}. On the other hand, the fact that modifiers cannot be classified on the basis of whether they have the -\textit{ly} ending or not is reinforced by the example of \textit{very}, which behaves like \textit{extremely} but could hardly be treated as a -\textit{ly} \isi{adverb}.

Furthermore, there is also a structural problem in connection with the status of modifiers in the analysis of \citet{corver1997}. As shown in (\ref{extremelyap}) and (\ref{farap}), the modifiers in question are located in the specifier of the \isi{QP}, which -- on the basis of the structure given in (\ref{corvertree}) -- correctly predicts that these elements have to precede the \isi{AP} and, if applicable, dummy \textit{much}. However, the same structure in (\ref{corvertree}) would require Deg heads to precede these modifiers, which is clearly not the case, as shown by \textit{far taller} in (\ref{fartaller}) and by \textit{far more intelligent} in (\ref{farintelligent}):

\ea Mary is \textbf{far more intelligent} than Agatha. \label{farintelligent}
\z

These data explicitly show that the structure of \isi{degree} expressions cannot be the one given in (\ref{corvertree}) or at least additional mechanisms would have to be taken into consideration.

Apart from the problem of how modifiers are treated by \citet{corver1997}, the position of the comparative \isi{subclause} itself is not even addressed, with respect to the matrix \isi{clausal} \isi{degree expression} and, possibly, arguments of adjectives. Assuming that the \isi{subclause} is closely related to the Deg head, it is not clear how it ultimately appears in a clause-final position and how it is base-generated next to the Deg head in the first place. The specifier of the \isi{DegP} seems to be a possible position but as \citet{corver1997} himself does not mention this possibility, I will refrain from speculating about it here.

\subsection{The QP--DegP analysis -- \citet{lechner1999diss, lechner2004}} \label{sec:2qpdegp}
Before turning to my proposal, let me briefly discuss the analysis provided by \citet{lechner2004}, a revised version of \citet{lechner1999diss}, which answers some of the questions that emerged in connection with the previous accounts mentioned here and which provides important insights concerning the actual relations between the various functional projections. This study is important first and foremost because it reconsiders the syntactic relationship between the \isi{AP} and the Deg head, in that it reflects the semantics of the Deg head much better than previous analyses.

\citet[22]{lechner2004} partially adopts the functional AP-hypothesis; that is, that the \isi{AP} is embedded under a \isi{functional projection}, the \isi{DegP}, cf. \citet{abney1987diss}, \citet{bresnan1973}, \citet{corver1990diss, corver1993, corver1997}, and \citet{kennedy1999}. However, \citet[22--23]{lechner2004} assigns a different structure to the \isi{DegP}, in that he proposes that the \isi{AP} is base-generated in the specifier position of the \isi{DegP} and not as a complement, in this respect recalling the proposal made by \citet{izvorski1995}. At the same time, the complement position serves to accommodate the comparative \isi{subclause}.

The structure -- using the \isi{DegP} in a string such as \textit{Mary is younger than Peter is} -- is shown below (see \citealt[22, ex. 45]{lechner2004}):

\ea \upshape \label{lechnertree}
\begin{forest} baseline, qtree, for tree={align=center}
[DegP
	[AP
		[younger,roof]
	]
	[Deg$'$
		[Deg\\\textsubscript{{[}+comparative{]}}]
		[\textit{than}-XP
			[than Peter is,roof]
		]
	]
]
\end{forest}
\z

An advantage of assuming that the \isi{AP} is in the specifier of the \isi{DegP} is that in this way, they can enter into a specifier--head relationship, and the [+comparative] Deg head can check off the features of the \isi{AP}. Note that \citet[23]{lechner2004} claims that comparative morphology is base-generated directly on the A head, and therefore a string like \textit{younger} cannot be syntactically decomposed into \textit{young} and the \isi{degree morpheme} -\textit{er}, contrary to \citet{bresnan1973}, but in line with \citet{izvorski1995} and \citet{corver1997}. As a matter of fact, \citet[23]{lechner2004} assumes that -\textit{er} morphology manifests a reflex of \isi{feature} checking: this, however, selectively surfaces only on certain A heads, namely ones that are monosyllabic or bisyllabic. Hence, in the case of periphrastic forms (e.g., \textit{more intelligent}), the \isi{feature} is claimed to be spelt out on Deg, resulting in the string \textit{more} + A.

This raises a rather compelling question in connection with periphrastic structures, namely that if the comparative \isi{feature} is spelt out on Deg in the form of \textit{more}, then, according to the representation in (\ref{lechnertree}), the string should actually be A + \textit{more}, e.g. *\textit{intelligent more}, which is clearly not the case. \citet{lechner2004} leaves the derivation of the grammatical order unexplained. However, \citet[25]{lechner1999diss} originally proposed that in periphrastic comparatives the \isi{DegP} is embedded under a \isi{QP}. Thus, for a string like \textit{more intelligent than Peter is}, the structure in (\ref{lechnertree}) should be modified in the way given in (\ref{treeqplechner2}):

\ea \upshape \label{treeqplechner2}
\begin{forest} baseline, qtree, for tree={align=center}
[QP
	[Q$'$
		[Q
			[more\textsubscript{i},name=more]
		]
		[DegP
			[AP [intelligent,roof]]
			[Deg$'$ [Deg [t\textsubscript{i},name=trace]] [\textit{than}-XP [than Peter is,roof]]]
		]
	]
]
\draw[->] (trace) to[out=south west,in=south,looseness=1.5] (more);
\end{forest}
\z

As can be seen, if there is a \isi{QP} layer above \isi{DegP}, \textit{more} can move up to the Q head position, thus resulting in the grammatical \isi{word order}.

One advantage of the analysis given in (\ref{lechnertree}), as \citet[23]{lechner2004} argues, is ``the dissociation of the surface position of -\textit{er} from the location of its interpretation''. The problem of not separating these two becomes obvious when considering the \textit{unhappier} Bracketing Paradox, see \citet{beard1991}, \citet{pesetsky1985} and \citet{sproat1992}. This paradox lies in the observation that \textit{unhappier} seems to be subject to two conflicting requirements. On the one hand, morpho-phonological rules would assign the following bracketing to the string (see \citealt[23, ex. 47a]{lechner2004}):

\ea	{[}un [happier]] \label{paradox}
\z

The reason behind this is that -\textit{er} may only be attached to an A head that maximally consists of two syllables, hence it must be attached prior to \textit{un}-. However, this seems to produce the interpretation `not happier' instead of `more unhappy'. On the other hand, in order to derive the correct interpretation, the bracketing should be the one given in (\ref{lechnerbracketing}), see \citet[23, ex. 47b]{lechner2004}:

\ea	{[}[unhappy] er] \label{lechnerbracketing}
\z

Note that in this case the morpho-phonological rules mentioned in connection with (\ref{paradox}) are violated.

In order to overcome this problem, \citet[23]{lechner2004} proposes that the correct bracketing is the one in (\ref{paradox}), but the interpretation of -\textit{er} is not directly associated with its \isi{base position}: it is a manifest of feature-checking, which involves the entire \isi{AP} (\textit{unhappy}).

With respect to the location of adjectival arguments, \citet[26]{lechner2004} makes use of some \ili{German} data exhibiting such constructions to provide additional evidence for the structure he attributes to nominal comparatives. According to his analysis, the \isi{PP} argument of an \isi{adjective} is a complement of the adjectival head and it may be subject to right dislocation. Consider (\citealt[26, ex. 51]{lechner2004}):

\ea \label{ppstolzorder}
\ea	\gll weil Hans [\textsubscript{PP} auf seinen Hund] stolz ist \label{ppstolz}\\
since Hans \phantom{}	of his.\textsc{m.acc} dog proud is\\
\glt `since Hans is proud of his dog'
\ex	\gll weil Hans stolz ist [\textsubscript{PP} auf seinen Hund] \label{stolzpp}\\
since Hans proud is \phantom{} of his.\textsc{m.acc} dog\\
\glt `since Hans is proud of his dog'
\z
\z

According to \citet[26]{lechner2004}, the underlying order is the one indicated in (\ref{ppstolz}), building on the assumption that the \isi{AP} is head-final; for such views, see for instance \citet{haiderrosengren1998}.  As will be discussed later, taking such a stance is problematic not only in terms of maintaining a universal directionality of headedness (cf. \citealt{kayne1994}) but also because it may rather be the case that the \ili{German} \isi{AP} is in fact head-initial. Nevertheless, taking up the argumentation of \citet{lechner2004}, (\ref{stolzpp}) is claimed to exhibit right dislocation of the \isi{PP} argument. However, if the \isi{AP} is an attribute in a \isi{nominal expression}, see (\ref{dislocation2}),dislocation is not possible (\citealt[26, ex. 54]{lechner2004}):

\ea \label{dislocation2}
\ea [] {\gll weil Hans eine [\textsubscript{PP} auf ihren Hund] stolze Frau getroffen hat\\
since Hans a.\textsc{f.acc} \phantom{} of her.\textsc{m.acc} dog proud.\textsc{f.acc} woman meet.\textsc{ptcp} has\\
\glt `since Hans met a woman proud of her dog'}
\ex	[*] {\gll weil Hans eine stolze Frau getroffen hat [\textsubscript{PP} auf ihren Hund] \label{ppattribute}\\
since Hans a.\textsc{f.acc} proud.\textsc{f.acc} woman meet.\textsc{ptcp} has \phantom{} of her.\textsc{m.acc} dog\\
\glt `since Hans met a woman proud of her dog'}
\z
\z

As can be seen, the \isi{extraposition} of the \isi{PP} is ungrammatical; this leads \citet[27]{lechner2004} to conclude that \isi{extraposition} is not permitted from a \isi{DegP} that is an attribute within a \isi{nominal expression}. The same is not true for the comparative \isi{subclause}: this can apparently be extraposed. \citet{lechner1999diss, lechner2004} introduces a special mechanism for it, by way of which the (original) comparative \isi{subclause} ends in such a position that it is coordinated with the (original) \isi{matrix clause}. Since this is clearly a kind of syntactic process that would go against standard minimalist assumptions and also a problematic proposal inasmuch as comparatives can hardly be considered coordinated structures (see \citealt{bacskaiatkari2010odd}), I will not present this part of Lechner's analysis here.

Even if one disregards the problems related to the \isi{movement} of the comparative \isi{subclause}, further ones arise in connection with the analysis given by \citet{lechner1999diss, lechner2004}. First, the treatment of \textit{more} is highly disputable as it does not take into consideration that it is built up of \textit{much} and the \isi{degree morpheme}. It is therefore also not straightforward how strings like \textit{as many (books)} should be analysed, where \textit{as many} obviously cannot be considered atomic.

Second, the status of the \isi{QP} is not clear either. Though on the basis of \citet{lechner1999diss} it ought to be generated in periphrastic structures, neither \citet{lechner1999diss} nor \citet{lechner2004} assume its presence in morphological comparatives. It appears that these contain merely \isi{DegP} projections. On the one hand, this is a problem for a unified analysis of \isi{degree} expressions as the maximal projections would be different, that is, either a \isi{QP} or a \isi{DegP}, without even implying any syntactic difference. More importantly, the absence of a \isi{QP} layer leaves the question of where modifiers are located unanswered.

Last but not least, the treatment of \isi{PP} arguments is far from being uncontroversial, especially because \citet{lechner2004} takes it for granted that the \isi{AP} is head-final and the \isi{PP} underlyingly precedes the A head. The opposing view is quite substantially present in the literature; see for instance \citet{webelhuth1992}. However, there are serious problems with Lechner's examples as well in the sense that the data as such are misleading. Consider:

\ea \label{stolzauf}
\ea [] {\gll [\textsubscript{PP}	Auf	seinen	Hund]	sollte	Hans stolz sein. \label{stolz1}\\
\phantom{} of his.\textsc{m.acc} dog should.\textsc{cond.3sg} Hans proud be\\
\glt `Hans should be proud of his dog.'}
\ex [\%] {\gll [\textsubscript{PP}	Auf	seinen	Hund]	stolz sollte Hans sein. \label{stolz2}\\
\phantom{} of his.\textsc{m.acc}	dog	proud should.\textsc{cond.3sg}	Hans be\\
\glt `Hans should be proud of his dog.'}
\ex [] {\gll Stolz	[\textsubscript{PP}	auf	seinen	Hund]	sollte Hans sein. \label{stolz3}\\
proud \phantom{} of his.\textsc{m.acc}	dog	should.\textsc{cond.3sg} Hans be\\
\glt `Hans should be proud of his dog.'}
\z
\z

The data show the possible \isi{movement} patterns of APs containing \isi{PP} complements in main clauses. The most typical order is the one in (\ref{stolz1}), where only the \isi{PP} moves to a position preceding the \isi{verb} \textit{sollte}. However, it is also possible to move the entire \isi{degree expression}. In that case, the natural order is A + \isi{PP}, as in (\ref{stolz3}). If the \isi{PP} precedes the A head, as in (\ref{stolz2}), the clause is not accepted by all speakers, and speakers who allow it remarked that the \isi{adjective} \textit{stolz} `proud' must be stressed, which indicates that the position of the \isi{adjective} on the right is most probably due to information structural requirements (and is therefore not a neutral order). This is already problematic for \citet[26]{lechner2004}, but the problem only increases with the speakers who do not accept (\ref{stolz2}) at all, while \citet{lechner2004} would predict (\ref{stolz2}) to be the unmarked case.

The apparent contradiction between (\ref{ppstolzorder}) and (\ref{stolzauf}) can be explained if we consider some basic facts about \ili{German} clause structure. In simple terms, subclauses show the underlying \isi{word order} SOV, the \isi{VP} (and the \isi{TP}) being head-final (\citealt[34]{haider1985}), whereas in main clauses the inflected \isi{verb} moves to the topmost C (see \citealt[30]{fanselow2004}, following \citealt{denbesten1989}, \citealt[133--134]{richtersailer1998}). The moved \isi{verb} comes second in the clause; it tolerates only one preceding constituent. This condition is satisfied in (\ref{stolz3}), where the A head precedes the \isi{PP} complement; however, in (\ref{stolz2}) the \isi{word order} is either the result of moving two constituents before the \isi{verb} (ungrammatical) or of the \isi{PP} moving into a position above the \isi{AP} (speaker-dependent), which is tolerated normally (by all speakers) only if the \isi{AP} is contained within a \isi{nominal expression}, as in (\ref{ppattribute}). 

I will return to the question of why \isi{degree} expressions differ in predicative and in attributive structures -- for now, suffice it to say that the core problem concerning the data provided by \citet{lechner2004} is that they only seemingly support his claim, but the desired word orders arise merely because he uses subclauses.

Apart from (\ref{stolzauf}), the possibility of intervening modifiers also indicates that the order \isi{PP} + A head cannot be the underlying one. Consider the examples in (\ref{lisamannstolz2}):

\ea \label{lisamannstolz2}
\ea \gll Lisa ist (wirklich) stolz [\textsubscript{PP} auf ihren Mann]. \label{wirklichpp}\\
Liz	is  \phantom{(}really proud \phantom{}	of her.\textsc{m.acc} husband\\
\glt `Liz is (really) proud of her husband.'
\ex \gll Lisa	ist	[\textsubscript{PP}	auf	ihren	Mann] (wirklich) stolz. \label{ppwirklich}\\
Liz	is \phantom{}	of her.\textsc{m.acc}	husband \phantom{(}really proud\\
\glt `Liz is (really) proud of her husband.'
\z
\z

In (\ref{wirklichpp}), the \isi{adjective} \textit{stolz} takes a \isi{PP} complement and may optionally be modified by an \isi{adverb} such as \textit{wirklich} `really'. In (\ref{ppwirklich}), the \isi{adjective} and the \isi{PP} complement appear in the reverse order; since the \isi{adverb} \textit{wirklich} can intervene between the two, it is obviously not the underlying order. This raises the question of where modifiers could be located in the analysis provided by \citet{lechner2004}, indicating that his structural representation is far from complete.

\largerpage[2]
\section{Towards the analysis} \label{sec:2towardstheanalysis}
In this section, I will present my analysis for \isi{degree} expressions, which may provide a better explanation for the problems mentioned above. I will chiefly concentrate on the \isi{comparative degree}, but absolute and superlative constructions will also be shown to fit into the representation. I adopt the proposal of \citet{lechner2004} that the \isi{AP} and the \isi{CP} are arguments of the \isi{degree head}. The \isi{AP} and the \isi{CP} establish a predicative relationship within the \isi{DegP}, which is in this respect similar to the Relator Phrase of \citet{dendikken2006} in its function. Consider the representation for the string \textit{far more interesting than the first one}:

\ea \upshape \label{qptree}
\begin{forest} baseline, qtree, for tree={align=center}
[QP
	[QP
		[far,roof]
	]
	[Q$'$
		[Q
			[-er\textsubscript{i} + much,name=er]
		]
		[DegP
			[AP [interesting,roof]]
			[Deg$'$ [Deg [t\textsubscript{i},name=trace1]] [CP [than the first one,roof]]]
		]
	]
]
\draw[->] (trace1) to[out=south west,in=south,looseness=1.5] (er);
\end{forest}
\z

As can be seen, there are two major layers that constitute a \isi{degree expression}: the \isi{DegP} and the \isi{QP}. Since arguments for the \isi{DegP} and the \isi{QP} have already been put forward in the literature, as described in the previous section, I will mention only additional arguments that support the analysis given in (\ref{qptree}).

First, let us consider the \isi{DegP}. The Deg head imposes selectional restrictions on its complement: in absolute constructions -- in \ili{English} -- it can be expressed by a \isi{PP} headed by \textit{for}, in comparatives it is a \isi{CP} headed by \textit{than} and in superlatives it is a \isi{PP} headed by \textit{of}:

\ea \label{degarg}
\ea	Mary is tall [\textsubscript{PP} for a schoolgirl]. \label{absarg}
\ex	Mary is taller [\textsubscript{CP} than her classmates].
\ex	Mary is the tallest [\textsubscript{PP} of the girls]. \label{suparg}
\z
\z

The structure is invariably the one given in (\ref{qptree}); in absolute constructions like (\ref{absarg}), the Deg head takes a \isi{PP} complement (\textit{for a schoolgirl}) and the Deg head itself is a zero; in superlatives like (\ref{suparg}), the Deg head takes a \isi{PP} complement (\textit{of the girls}) and is filled by -\textit{est}.

It is important to note that selectional restrictions concern the relevant \isi{degree} features rather than the syntactic category of the complements. For instance, a superlative \isi{degree morpheme} selects a complement with a superlative \isi{feature} -- since the P head \textit{of} may be equipped with this \isi{feature}, it is an \textit{of}-\isi{PP} that ultimately appears as the superlative complement. However, there are languages that allow the realisation of one \isi{degree} complement by categorically different XPs. Consider the data in (\ref{raulo2}) from \ili{Italian}:

\ea \label{raulo2}
\ea \gll Raulo è più alto [\textsubscript{CP} che Alessandro].\\
Ralph	is more tall.\textsc{m} {} that Alexander\\
\glt `Ralph is taller than Alexander.'
\ex \gll Raulo è più alto [\textsubscript{PP} di Alessandro].\\
Ralph is more tall.\textsc{m} {} of Alexander\\
\glt `Ralph is taller than Alexander.'
\z
\z

As can be seen, in \ili{Italian} the \isi{comparative complement} can either be a clause introduced by \textit{che} `that' or a \isi{PP} headed by \textit{di} `of'; in both cases, the comparative \isi{degree head} is \textit{più} `more'. In \ili{Hungarian}, as shown in (\ref{lujza2}), there is a choice between a \isi{CP} and a \isi{DP} with inherent (adessive) case (cf. \citealt{wunderlich2001}):

\ea \label{lujza2}
\ea \gll Lujza magasabb volt, [\textsubscript{CP} mint Mari].\\
Louise taller was.\textsc{3sg} \phantom{} than Mary\\
\glt `Louise was taller than Mary.'
\ex \gll Lujza magasabb volt [\textsubscript{DP}	Marinál].\\
Louise taller was.\textsc{3sg} {} Mary.\textsc{ade}\\
\glt `Louise was taller than Mary.'
\z
\z

In both constructions, the \isi{DegP} is headed by the morpheme -\textit{bb} `-er'. Apparently, \ili{Russian}\footnote{I owe many thanks to Maria Shkapa for her indispensable help with the \ili{Russian} data.} displays the same kind of variation:

\ea \label{russiancomp}
\ea \gll Ona vyše [\textsubscript{NP} svoix	odnoklassnikov]. \label{russiangen}\\
she taller {}	her.\textsc{gen.pl} classmates.\textsc{gen}\\
\glt `She is taller than her classmates.'
\ex \gll Ona vyše	[\textsubscript{CP} čem eë odnoklassniki].\\
she	taller {} than her classmates.\textsc{nom}\\
\glt `She is taller than her classmates.'
\z
\z

In the \isi{matrix clause}, the \isi{degree expression} is \textit{vyše} `taller', which contains the \isi{comparative morpheme} -\textit{eje} `-er'; the \isi{comparative complement} is either a \isi{CP} or a \isi{nominal expression} marked for the genitive case.

In \ili{Russian}, adjectives can regularly appear both in morphological and periphrastic constructions. However, only the \isi{clausal} \isi{comparative complement} is allowed with periphrastic comparatives:

\ea \label{russianselection}
\ea [*] {\gll Ona boleje vysokaja [\textsubscript{NP} svoix odnoklassnikov]. \label{russiandp}\\
she more tall.\textsc{f} {} her.\textsc{gen.pl} classmates.\textsc{gen}\\
\glt `She is taller than her classmates.'}
\ex [] {\gll Ona	boleje vysokaja	[\textsubscript{CP}	čem	eë odnoklassniki].\\
she	more tall.\textsc{f} {} than	her	classmates\\
\glt `She is taller than her classmates.'}
\z
\z

As should be obvious, the ungrammaticality of (\ref{russiandp}) cannot be the result of the mere fact that the \isi{degree expression} is comparative since in that case (\ref{russiangen}) should also be ruled out. There are two crucial differences between the \isi{degree} expressions in (\ref{russiancomp}) and the ones in (\ref{russianselection}): the \isi{degree head} itself, which is -\textit{eje} in (\ref{russiancomp}) and \textit{boleje} in (\ref{russianselection}), and the form of the \isi{adjective}, that is, while \textit{vyše} is not inflected for gender, \textit{vysokaja} is. This latter difference has the prediction that, since attributes have to agree with their nouns in gender in \ili{Russian}, morphological \isi{comparative degree} expressions will never be attributes and, consequently, the inherently case-marked \isi{NP} \isi{comparative complement} will not appear in attributive comparatives either. This prediction is in fact borne out.

On the other hand, it should be obvious that the Deg head imposes restrictions on both its specifier and its complement. It selects for complements headed by certain elements and it agrees with the \isi{AP}, which may be manifest in diverse features; for instance, it may select exclusively for APs that are in a predicative form. I will address this issue later on; for the time being, suffice it to say that the way the \isi{degree head} imposes restrictions on its arguments suggests that features independent from the \isi{degree} property are also involved.

\largerpage[1]
Returning now to the examples given in (\ref{degarg}), it is worth mentioning that although the \isi{DegP} proposed here does bear some resemblance to the Relator Phrase of \citet{dendikken2006}, the treatment of the \isi{PP} in (\ref{absarg}) highlights a crucial difference from his analysis. According to \citet[63]{dendikken2006}, in structures such as \textit{big for a butterfly}, the \isi{AP} big is located in the specifier position of an RP, the \isi{DP} \textit{a butterfly} is the complement and the R head itself is \textit{for} -- using the \isi{DegP} analysis, this would translate \textit{for} as a Deg head. By \isi{contrast}, I propose that the complement position of the Deg head is occupied by the \isi{PP} \textit{for a butterfly}, which leaves the Deg headed by a zero relator, that is, the \isi{absolute degree} morpheme. The advantage is that this way, the complement may act as one constituent, irrespectively of whether it is a \isi{PP} (like \textit{for a butterfly}) or a \isi{CP} (like \textit{than the first one}). Separating the \isi{CP} from its \isi{complementiser} head \textit{than} would clearly be problematic; the same is true for the \isi{PP}, as shown in (\ref{fronting2}). The PP may actually be moved on its own, as shown in (\ref{frontedabsarg}), in the same way as the \isi{PP} argument in superlatives, as in (\ref{frontedsuparg}):

\ea \label{fronting2}
\ea	{}[\textsubscript{PP} For an adult], he is tiny. \label{frontedabsarg}
\ex	{}[\textsubscript{PP} Of all the girls], she is the most beautiful. \label{frontedsuparg}
\z
\z

Of course, there are further restrictions on which phrases may actually undergo \isi{movement}; that is, while the fact that a given string may undergo \isi{movement} on its own seems indicative of that string being a phrase, it is not true that all phrases may undergo \isi{movement}. This largely has to do with whether the complements of the Deg head are phrasal (smaller than a clause) or \isi{clausal}. As for \ili{English}, the \isi{PP} complements in absolute and superlative constructions may move, while the \isi{CP} in comparatives cannot. In \ili{Hungarian}, there are two types of comparative complements: CPs and case-marked DPs. While CPs cannot move out, case-marked DPs can:

\ea \label{hungfront}
\ea [*]{\gll [\textsubscript{CP}	Mint Péter]\textsubscript{i} magasabb voltam \emph{t}\textsubscript{i}. \label{mintfront}\\
{} than Peter taller was.\textsc{1sg}\\
\glt `I was taller than Peter.'}
\ex	[]{\gll [\textsubscript{DP}	Péternél]\textsubscript{i} magasabb	voltam \emph{t}\textsubscript{i}. \label{adefront}\\
{} Peter.\textsc{ade} taller	was.\textsc{1sg}\\
\glt `I was taller than Peter.'}
\z
\z

Since both (\ref{mintfront}) and (\ref{adefront}) are comparative structures, the difference with respect to \isi{extraposition} is the result of having different syntactic categories, and not of having different degrees.

Note that the difference indicated in (\ref{hungfront}) is truly a result of a difference in the syntactic categories and is independent from the fact that (\ref{mintfront}) contains \isi{ellipsis}: the non-elided counterpart of (\ref{mintfront}) would equally be ungrammatical. Consider the examples in (\ref{mintpeter2}):

\ea [*]{\gll [\textsubscript{CP} Mint Péter volt]\textsubscript{i} magasabb voltam \emph{t}\textsubscript{i}. \label{mintpeter2}\\
{} than Peter was.\textsc{3sg} taller	was.\textsc{1sg}\\
\glt `I was taller than Peter.'}
\z

On the other hand, there are languages that tolerate the fronting of an elliptical \isi{clausal} \isi{comparative complement}. Consider the example in (\ref{alspeter2}) from \ili{German} (see \citealt{bacskaiatkari2014alh} on \textit{als}-clauses being elliptical and not phrasal in \ili{German}):

\ea \label{alspeter2}
\gll {[}\textsubscript{CP} Als	Peter] war ich größer.\\
{} than Peter was.\textsc{1sg} I	taller\\
\glt `I was taller than Peter.'
\z

Returning now to the structure in (\ref{qptree}), it can be seen that the \isi{AP} moves up to the specifier of the \isi{DegP} in order to agree with the \isi{degree head}. One argument in favour of such an \isi{agreement} is that in this way certain illicit configurations may be ruled out. Consider:

\ea \label{nongradable}
\ea	[*]{Liz is more pregnant than Mary.} \label{pregnantgrad}
\ex [*]{This instalment is more impossible than the previous one.}
\z
\z

The ill-formedness of the constructions in (\ref{nongradable}) stems from the comparative use of \textit{pregnant} and \textit{impossible}: pregnant and \textit{impossible} are non-gradable adjectives, and hence cannot agree with a comparative \isi{degree head}. I assume that with non-gradable adjectives, a \isi{DegP} (and \isi{QP}) layer is regularly not projected, since a Deg head cannot license a non-gradable \isi{AP} in its specifier. Exceptions are non-gradable adjectives used as gradable ones in a given context that licenses a gradable interpretation: a sentence like (\ref{pregnantgrad}) may be licensed exceptionally in a context to mean that Liz is more visibly pregnant than Mary, or that her pregnancy is more advanced than that of Mary.\footnote{Note that since these adjectives are non-gradable, they do not tolerate \isi{degree} modifiers either, e.g. *\textit{very pregnant}, *\textit{quite impossible}. This is in line with the claim that the Deg head has to agree both with the \isi{AP} and with \isi{QP} modifiers, if any.}

Another case where there is clearly \isi{agreement} involved between the \isi{AP} and the Deg head is \ili{Icelandic}. In \ili{Icelandic}, as in other \ili{Scandinavian} languages, the \isi{adjective} has to agree in gender with the \isi{noun} it qualifies, both, when the \isi{adjective} is a \isi{predicate} and when it is an attribute. In addition, in \ili{Icelandic} there is gender \isi{agreement} between the \isi{AP} and the Deg head, as demonstrated by the examples in (\ref{icelandic2}):

\ea \label{icelandic2}
\ea []{\gll	rík-ur \label{rikur}\\
rich-\textsc{m}\\
\glt `rich'}
\ex	[]{\gll rík-ast-ur \label{rikastur}\\
rich-\textsc{superlative.m-m}\\
\glt `richest'}
\ex [*]{\gll rík-ust-ur \label{rikustur}\\
rich-\textsc{superlative.f-m}\\
\glt `richest'}
\z
\z

In (\ref{rikur}), the \isi{adjective} \textit{rík} `rich' takes a masculine ending -\textit{ur}, forming the absolute \isi{adjective} \textit{ríkur}. In (\ref{rikastur}), in order to form the superlative, the superlative masculine morpheme -\textit{ast} is added to the stem \textit{rík}, and is followed by the regular masculine ending -\textit{ur}, resulting in the final form \textit{ríkastur}. The reason why \textit{ríkustur} in (\ref{rikustur}) is ungrammatical is that it contains the superlative feminine morphemes -\textit{ust} instead of the masculine one. Thus, in \ili{Icelandic} there is not only \isi{agreement} between the full \isi{QP} and the \isi{noun} but also within the \isi{DegP}. Note that in other \ili{Scandinavian} languages, such as \ili{Danish}, there is \isi{agreement} only between the \isi{QP} and the \isi{noun}; however, the \ili{Danish} \isi{comparative morpheme} will invariably remain -\textit{(e)re}, irrespectively of gender.

Let us now examine the \isi{QP} layer, which is invariably present on top of a \isi{degree expression}. In (\ref{qptree}), it is headed by \textit{much}, and the specifier may accommodate a \isi{QP} \isi{modifier} such as \textit{far}; the \isi{QP} is obviously necessary for accommodating both of these elements, as was argued for in the previous section. I also adopt the view expressed by \citet{corver1997} that \textit{much} is present in periphrastic structures in a similar way as other dummy elements (e.g., \textit{do}) enter the derivation, hence there is no need to stipulate any additional process such as \textit{much}-\isi{deletion}.

Periphrastic comparatives and superlatives are formed in the way given in (\ref{qptree}): the Deg head -\textit{er}/-\textit{est} moves up to the Q head filled by \textit{much} and the merge of \textit{much} and -\textit{er}/-\textit{est} gives \textit{more}/\textit{most} (cf. \citealt{bresnan1973, corver1997, beck2011, kantor2008}). Head adjunction results in the order -\textit{er much} (or -\textit{est much}) in syntax, due to Kayne's Linear Correspondence Axiom (\citealt{kayne1994}), see also the Mirror Principle of \citet{baker1985, baker1988}. It is the result of morphological merge at the \isi{PF} interface that -\textit{er}/-\textit{est} is attached to \textit{much}, which follows it.

\largerpage[1]
This becomes even more important in the case of morphological comparatives, where there is no \textit{much}. Here the Deg head -\textit{er}/-\textit{est} is still moved to a zero Q head in syntax and the \isi{degree morpheme} undergoes morphological merge with the \isi{AP} at \isi{PF}, as argued for by \citet[45--51]{kantor2010diss}. This is reinforced by the existence of irregular (suppletive) forms, such as \textit{better}, which are formed by merging -\textit{er} with \textit{good}: this form obviously cannot be the result of simple syntactic merge of the two elements. Moreover, as \citet[49--51]{kantor2010diss} argues, the variation in possible forms in the case of complex adjectives can only be explained by attributing the mechanism to \isi{PF}. For instance, an \isi{adjective} like \textit{good-looking} may have its comparative form either as \textit{better-looking} or as \textit{more good-looking}, the former being a clear indication of the fact that the -\textit{er} is not attached to the \isi{AP} (\textit{good-looking}) in syntax but to the \isi{adjective} itself in \isi{PF}.\footnote{Note that it is the idiosyncratic property of (compound) adjectives whether they count as morphologically transparent or not. Whereas morphologically transparent ones (e.g. \textit{well-paid} or \textit{long-lasting}) tend to have both forms (e.g. \textit{better-paid} and \textit{more well-paid}, or \textit{longer-lasting} and \textit{more long-lasting}), the ones that are not transparent (e.g. \textit{easy-going} or \textit{hard-working}) can only be formed with \textit{more} (e.g. \textit{more easy-going} and not *\textit{easier-going}, or \textit{more hard-working} but not *\textit{harder-working}).}

Last but not least, the \isi{QP} \isi{modifier} is located in the specifier because it has to agree with the Q head (which is here \textit{much}). As was mentioned in connection with the analysis given by \citet{corver1997}, there are selectional restrictions as to which \isi{modifier} can appear together with which \isi{degree}, as illustrated in (\ref{veryfar2}):

\ea \label{veryfar2}
\ea	Mary is \textbf{very tall} / *\textbf{far tall}.
\ex	Mary is *\textbf{very taller} / \textbf{far taller}.
\z
\z

As can be seen, the \isi{QP} \textit{very} can appear in absolute constructions but not in comparatives. By \isi{contrast}, \textit{far} is compatible with the \isi{comparative degree} but not with the absolute. Due to this, the \isi{QP} modifiers are clearly not adjuncts; therefore, the analysis based on specifier--head \isi{agreement} can explain the restrictions better than one treating them as adjuncts, as was done for instance in \citet{bresnan1973} and \citet{corver1997}.

Since the possibility of certain modifiers is merely dependent on the relevant \isi{degree} features (and not, for instance, on the presence or absence of the ending -\textit{ly}, as was proposed by \citealt{corver1997}), it is not inexplicable that strings such as \textit{far different} should exist, even though there is no overt -\textit{er} morpheme present. Since, as was mentioned before, \textit{far} normally co-occurs with the \isi{comparative degree}, the way to overcome this problem is to say that the \isi{adjective} \textit{different} inherently expresses comparison and therefore may be equipped with an inherent [+compr] \isi{feature}, which agrees with a comparative Deg head. In fact, this is supported by the fact that in certain (American) dialects \isi{degree} expressions with \textsl{different} typically take a \textit{than}-clause instead of a \isi{PP}, as in (\ref{differentthan2}):

\ea	University life is \textbf{different} than I expected. \label{differentthan2}
\z

It is therefore preferable to analyse the relationship between \isi{QP} modifiers and \isi{degree} expressions as one determined by \isi{agreement} between the \isi{modifier} and the \isi{degree head} moving to Q, rather than one depending on the -\textit{ly} morpheme.

The analysis presented so far also has the advantage of treating morphological and periphrastic comparatives in a unified way, by assuming that the appearance of \textit{much} in periphrastic comparatives is due to regular dummy insertion and not the lack of a stipulated \isi{deletion} rule. The two remaining questions are therefore the role of the \isi{DegP} other than marking the \isi{degree} itself and how it may account for phenomena related to \isi{PP} arguments of adjectives, and the mechanisms behind the \isi{extraposition} of the comparative \isi{subclause}.

\section{Predicative and attributive adjectives} \label{sec:2predicativeattributive}
One important question regarding the analysis presented above is how it can account for the differences between predicative and attributive adjectives. There are adjectives that are inherently predicative; consider the examples in (\ref{predadj2}):

\ea \label{predadj2}
\ea	[]{The girl was \textbf{afraid}.} \label{afraid}
\ex	[*]{I saw an \textbf{afraid} girl.} \label{afraidgirl}
\z
\z

As can be seen, the \isi{adjective} \textit{afraid}, which is inherently predicative, can appear as a sentential \isi{predicate}, as in (\ref{afraid}), but cannot be an attribute within a \isi{nominal expression}, as demonstrated by the ungrammaticality of (\ref{afraidgirl}). Similar adjectives include \textit{alive}, \textit{asleep} or \textit{ill} in \ili{English}.

On the other hand, there are inherently non-predicative adjectives too, such as \textit{main} in (\ref{attradj2}):

\ea \label{attradj2}
\ea	[*]{The reason is \textbf{main}.} \label{main}
\ex	[]{That is the \textbf{main} reason.} \label{mainreason}
\z
\z

Contrary to \textit{afraid}, the \isi{adjective} \textit{main} cannot function as a sentential \isi{predicate}, as shown in (\ref{main}), but may be an attribute, as in (\ref{mainreason}). It is interesting to note that most attributive-only adjectives are also non-gradable, e.g. \textit{main}, \textit{northern}, \textit{mere}, \textit{previous} or \textit{utter}. However, this is by no means a necessity, as demonstrated by the examples in (\ref{moreattr2}):

\ea \label{moreattr2}
\ea	It is a \textbf{more recent} theory than the traditional transmission model.
\ex	As he drinks, he gets into a \textbf{more drunken} state.
\z
\z

In fact, \isi{gradability} and the choice between predicative and attributive uses are two independent properties which allow for six logical combinations: \isi{gradability} is clearly binary and independently from this, adjectives may be predicative-only, attributive-only and may allow both options. Examples are given in Table \ref{tableadjectives}.

\begin{table}
\begin{tabular}{cccc}
\lsptoprule
{} & predicative-only & attributive-only & both\\
\midrule
gradable & \textit{afraid} & \textit{drunken} & \textit{tall} \\
non-gradable & \textit{alive} & \textit{main} & \textit{pregnant} \\
\lspbottomrule
\end{tabular}
\caption{The classification of adjectives.}
\label{tableadjectives}
\end{table}

This strongly suggests that apart from the fact that a Deg head may appear only with a \isi{gradable adjective}, there are also further features to be considered.

It has to be mentioned that there are considerable cross-linguistic differences as to which adjectives qualify as predicative-only or attributive-only. In \ili{Russian}, for instance, all the adjectives mentioned above can be both predicative and attributive. Consider the examples in (\ref{russian}).

\ea \label{russian}
\ea	\gll Eto \textbf{glavnyj} vokzal.\\
this.\textsc{n} main railway.station\\
\glt `This is the main railway station.'
\ex \gll Etot vokzal	\textbf{glavnyj}.\\
this.\textsc{m} railway.station main\\
\glt `This railway station is the main one.'
\z
\z

As can be seen, the \isi{adjective} \textit{glavnyj} `main' can appear both as a \isi{predicate} and as an attribute, contrary to \ili{English} \textit{main}. This shows that although there are general syntactic and semantic properties that play a crucial role in determining whether a given \isi{adjective} can be predicative and/or attributive, and which hold across languages, there are also important cross-linguistic differences, and individual lexical items may be idiosyncratic, too.

The fact that there are idiosyncratic properties to be considered as well is indicated by the existence of synonyms that behave differently, in spite of there being no differences in their morphological structure. Such a pair is \textit{ill} and \textit{sick} in \ili{English}: while \textit{sick} may act both as a \isi{predicate} and as an attribute, \textit{ill} is licensed only in a \isi{predicative position}. Apart from such unpredictable properties, however, there are of course certain semantic and syntactic properties that make restrictions predictable. As pointed out by \citet{kenesei2014}, relational adjectives tend not to occur in attributive positions (cf. \citealt{bally1944}, \citealt{mcnallyboleda2004} and \citealt{fradin2007}, among others). This has an interesting morphosyntactic correlation in \ili{Hungarian}, where the denominal adjective-forming suffix -\textit{i} produces relational adjectives; therefore, as can be expected, adjectives formed with this suffix tend not to be allowed in attributive positions. Still, although most relational adjectives are attributive-only, there are ones that can function as predicates, e.g., \textit{English}. Another semantic class that is known to be attributive-only is that of evaluative adjectives such as \textit{damned}. I cannot examine these issues in detail here; however, it is important to bear in mind that there are several factors that determine whether a given \isi{adjective} is predicative or attributive, including both semantic and syntactic features and cross-linguistic differences.

I propose that the difference between predicative-only and attributive-only adjectives can be formalised with the help of features that are independent from \isi{gradability}.\footnote{Since the aim of the present discussion is to provide an adequate analysis for the structure of \isi{degree} expressions, I do not venture into a detailed examination of a feature-based categorisation of adjectives and will restrict myself to a basic distinction between predicative and non-predicative adjectives, which should suffice for the purpose of analysing \isi{degree} expressions. As discussed by \citet{fradin2007}, there are several criteria that have to be considered when examining the distribution of adjectives, including the syntactic position that they can take, \isi{gradability}, and whether the \isi{adjective} is denominal. These factors also interact with one another. Moreover, \citet{fradin2007} points out that, at least in languages that allow or even prefer postnominal \isi{modification}, such as \ili{French}, the distinction between a prenominal and a postnominal adjectival \isi{modifier} is also crucial. Again, this is a problem I am not going to address, especially because the postnominal appearance of adjectives in the languages under scrutiny is rather due to the presence of a reduced \isi{relative clause} and not to true rightward attachment of the \isi{QP} \isi{modifier}. Finally, I am not going to deal with the issue of category shift, that is, when an \isi{adjective} can be assigned two different \isi{feature} matrices depending on the \isi{noun} it modifies, e.g., \textit{osseux} `bony' is gradable in constructions such as \textit{visage osseux} `bony face' but not in ones such as \textit{tuberculose osseuse} `bone tuberculosis', see \citet[84--85]{fradin2007}.} First, let us consider the following examples:

\ea \label{adjectivesparadigm}
\ea	Mary is pregnant.
\ex	Mary is a pregnant woman.
\ex	Mary is tall.
\ex	Mary is a tall woman.
\z
\z

The respective semantic representations of the adjectives in (\ref{adjectivesparadigm}) are given in (\ref{semanticsadjectives}) below:

\ea \label{semanticsadjectives}
\ea	\upshape \textsc{pregnant}(x) \label{pregnant}
\ex	\upshape $\exists$x[\textsc{woman}(x)\&\textsc{pregnant}(x)]] \label{girlpregnant}
\ex	\upshape $\exists$d[\textsc{tall}(x,d)] \label{tall}
\ex	\upshape $\exists$x[\textsc{woman}(x)\&$\exists$d[\textsc{tall}(x,d)]] \label{girltall}
\z
\z

As can be seen, the difference between non-gradable adjectives like \textit{pregnant} in (\ref{pregnant}) and (\ref{girlpregnant}) and gradable adjectives like \textit{tall} in (\ref{tall}) and (\ref{girltall}) is that the former simply denote sets of entities (x) that have a certain property, while gradable adjectives denote an ordered set of entities along the degrees (d) of an implied scale (see \citealt{kennedymcnally2005, cresswell1976, heim2000, kennedy1999}). A \isi{gradable adjective} equipped with a relevant syntactic \isi{feature}, call it [+deg], contains information in its semantics with respect to a \isi{degree} variable that quantifies over it; this is translated into syntax in such a way that the [+deg] \isi{feature} must be checked against a Deg head. Non-gradable adjectives, on the other hand, are [--deg] and cannot enter into an \isi{agreement} relationship with a Deg head; consequently, these adjectives are not supposed to be located within a \isi{DegP} (and hence a \isi{QP}).

On the other hand, there is a distinction between predicative and attributive adjectives: the latter do not take an individual but a variable (\textit{x}), which is in turn taken by another \isi{predicate}, both predicates being in the scope of the existential \isi{quantifier}. Syntactically, this difference should be the presence of a \isi{feature} that can be checked off against a \isi{noun}, call it [+nom]. Attributive-only adjectives are inherently [+nom] and if they do not appear in an \isi{attributive position}, this \isi{feature} cannot be checked off by \isi{agreement}. By \isi{contrast}, predicative-only adjectives are inherently [--nom] and if they appear as attributes, there is a \isi{feature} mismatch with the \isi{noun} head, which causes ungrammaticality.

Adjectives that can be both predicates and attributes allow for both [+nom] and [--nom]. This may be manifest in distinct forms between the two uses; that is, in certain languages (such as \ili{German}) there are inflected forms in the \isi{attributive position} that overtly show \isi{agreement}, while this is not so in predicative uses. Since non-gradable adjectives can also be both [+nom] and [--nom], it should be clear that the choice is primarily not encoded in the Deg head, but rather on the \isi{AP} itself: in the case of APs without a \isi{DegP} projection, this syntactic information cannot be introduced elsewhere.

Naturally, in the case of gradable adjectives these features percolate up within the \isi{QP}. First, the [\textpm nom] \isi{feature} of the \isi{AP} percolates up to the \isi{DegP} via specifier--head \isi{agreement} (cf. \citealt{yoon2001, ortizdeurbina1993, horvath1997}). Second, the \isi{movement} of the Deg head to the Q head assures the percolation of the \isi{feature} to the Q head. Hence a [+nom] \isi{QP} can and must enter into a further \isi{agreement} relationship with a nominal head.

Feature percolation is summarised in the diagram in (\ref{treeqpdegp2}):

\ea \upshape \label{treeqpdegp2}
\begin{forest} baseline, qtree, for tree={align=center}
[QP
	[Q$'$
		[Q,name=Q
			[X\textsubscript{i}]
		]
		[DegP
			[AP\textsubscript{{[}+nom{]}},name=AP [\phantom{xxxx},roof]]
			[Deg$'$ [Deg,name=Deg [t\textsubscript{i}]] [\ldots]]
		]
	]
]
\draw[dashed,->] (AP) to[out=south west,in=south west,looseness=1.5] node[pos=0.4,right] {\textsubscript{[+nom]}} (Deg);
\draw[dashed,->] (Deg) to[out=south west,in=west,looseness=2] node[pos=0.5,right] {\textsubscript{[+nom]}} (Q);
\end{forest}
\z

Predicative QPs can function as predicates in the clause or as postnominal modifiers and the Deg head is equipped with a [--nom] \isi{feature}. Attributive QPs, by \isi{contrast}, are modifiers of NPs and the Deg head is equipped with a [+nom] \isi{feature}.

In the cases I have looked at so far, it was invariably the \isi{AP} that defined the [\textpm nom] nature of the \isi{degree expression}, the Deg head itself being underspecified for this \isi{feature}. However, it is possible that certain Deg heads are inherently [+nom] or [--nom]. This is the case of the \ili{Russian} comparative head -\textit{eje} as given in (\ref{russianselection}), which appears exclusively as a \isi{predicate}: it takes a [--nom] \isi{AP} in its specifier, uninflected for gender, and can never appear as an attribute. On the other hand, superlative constructions seem to be universally attributive-only (cf. \citealt{matushansky2008}, based on \citealt{heim1999}) and therefore it is justifiable that superlative Deg heads are inherently [+nom].

I do not wish to elaborate on the syntax and the semantics of superlatives here and to present an account for why superlatives are inherently [+nom] in particular. Note that the obligatory presence\footnote{The picture is in fact somewhat more complex in this respect, and the article may be omitted in certain cases, see \citet{heim1999} and \citet[423--426]{croitorgiurgea2016} for discussion. Note, however, that definite DPs do not always require an overt definite article either, and the fact that the article does not always occur when a superlative is present reinforces the assumption that the article is not part of the degree phrase.} of the \isi{definite article} in superlatives is due to the presence of a nominal projection and is not required by the \isi{QP} itself, as indicated by (\ref{thebest2}):

\ea This hypothesis is \textbf{*\textsuperscript{/??}(the) best}. \label{thebest2}
\z

As indicated, the \isi{definite article} \textit{the} cannot be left out without affecting the grammaticality of the clause; still, there is no overt \isi{noun} required. This is not the case for absolute and comparative adjectives such as (\ref{goodone}) and (\ref{betterone}):

\ea \label{goodbetterone2}
\ea This hypothesis is \textbf{the good *(one)}. \label{goodone}
\ex This hypothesis is \textbf{the better *(one)}. \label{betterone}
\z
\z

In (\ref{goodbetterone2}), there has to be either an overt lexical \isi{noun} or at least the \isi{proform} \textit{one}, otherwise the structure is ungrammatical. By \isi{contrast}, in simple predicative structures the article is absent and so is the \isi{noun} (or \textit{one}):

\ea
\ea	This hypothesis is \textbf{good}.
\ex	This hypothesis is \textbf{better}.
\z
\z

Such constructions are not readily available for superlatives, however, as shown in (\ref{best2}):

\ea	*\textsuperscript{/??}This hypothesis is \textbf{best}. \label{best2}
\z

Note also that different languages may behave differently with respect to the obligatory \isi{overtness} of the \isi{noun} head. In \ili{Hungarian}, for instance, no such requirement is attested. Consider:

\ea \label{hungovertnoun}
\ea \gll Ez	az	elmélet	\textbf{a}	\textbf{jó}.\\
this the	theory the	good\\
\glt `This theory is the good one.'
\ex	\gll Ez	az	elmélet	\textbf{a}	\textbf{jobb}.\\
this the theory	the	better\\
\glt `This theory is the better one.'
\ex	\gll Ez az	elmélet	\textbf{a}	\textbf{legjobb}.\\
this the theory	the	best\\
\glt `This theory is the best one.'
\z
\z

As can be seen, all the cases in (\ref{hungovertnoun}) involve the sequence of an overt \isi{definite article} and an \isi{adjective} but there is no phonologically visible \isi{noun} head. I will not examine why this option is available for the absolute and the comparative degrees in \ili{Hungarian} but not in \ili{English}. What is important to note is that in the absence of a nominal projection, the superlative is not possible in \ili{Hungarian} either:

\ea \label{hungsuperlativenoun}
\ea []{\gll Ez az	elmélet	\textbf{jó}. \label{hungsuperlativenounabs}\\
this the theory	good\\
\glt `This theory is good.'}
\ex	[]{\gll Ez az	elmélet	\textbf{jobb}. \label{hungsuperlativenouncomp}\\
this the theory	better\\
\glt `This theory is better.'}
\ex [*]{\gll Ez az	elmélet	\textbf{legjobb}. \label{hungsuperlativenounung}\\
this the theory	best\\
\glt `This theory is the best.'}
\z
\z

In structures such as (\ref{hungsuperlativenoun}), there is no covert \isi{noun} head and the \isi{QP} functions as a \isi{predicate} in the clause. This is possible with the absolute and the \isi{comparative degree}, as in (\ref{hungsuperlativenounabs}) and (\ref{hungsuperlativenouncomp}), respectively, but since the \isi{superlative degree} is licensed only if there is a \isi{noun} head in the structure, (\ref{hungsuperlativenounung}) is not grammatical.

One of the obvious advantages of the analysis presented so far is that it provides a unified approach that covers both predicative and attributive structures. Recall that this was precisely one of the chief concerns expressed by \citet{izvorski1995}. However, her analysis was shown to be problematic for several reasons. Contrary to her assumptions, I claim that the inner syntactic structure of \isi{degree} expressions is the same in both cases, but the features determining whether the entire \isi{QP} may function as a \isi{predicate} or an attribute are indeed QP-internal.

\section{Arguments of adjectives} \label{sec:2arguments}
Providing a formal account for the differences between predicative and attributive adjectives becomes especially important when considering arguments of adjectives. Recall that certain adjectives are known to have arguments of their own, as shown in (\ref{lizmary2}):

\ea \label{lizmary2}
\ea	Liz is proud [\textsubscript{PP} of her husband].
\ex	Mary is afraid [\textsubscript{PP} of snakes].
\z
\z

In the examples above, the adjectives \textit{proud} and \textit{afraid} take the bracketed PPs as their arguments. However, adjectives with \isi{PP} complements are not allowed in an \isi{attributive position}, as shown in (\ref{app2}):

\largerpage[1]
\ea \label{app2}
\ea	[*]{Liz is a proud [\textsubscript{PP} of her husband] woman.} \label{proudofwoman}
\ex	[]{Liz is a proud woman.} \label{proudwoman}
\ex	[]{Liz is a woman proud [\textsubscript{PP} of her husband].} \label{womanproudof}
\z
\z

As demonstrated by the data above, the appearance of \textit{proud} with its \isi{PP} complement is ungrammatical in the \isi{attributive position}, as shown in (\ref{proudofwoman}), despite the fact that \textit{proud} can otherwise appear in this position, as shown in (\ref{proudwoman}). It is of course possible to have the \isi{adjective} together with its \isi{PP} argument in a postnominal position, as in (\ref{womanproudof}).

The same pattern can be observed in the case of inherently predicative-only adjectives, see (\ref{marysnakes2}):

\ea \label{marysnakes2}
\ea	[*]{Mary is an afraid [\textsubscript{PP} of snakes] girl.} \label{afraidofgirl}
\ex	[]{Mary is a girl afraid [\textsubscript{PP} of snakes].} \label{girlafraidof}
\z
\z

The ungrammaticality of (\ref{afraidofgirl}) is expected since the appearance of the \isi{adjective} \textit{afraid} in an \isi{attributive position} would be ungrammatical anyway; again, the postnominal position leads to an acceptable construction, as in (\ref{girlafraidof}). It seems that the ungrammaticality of (\ref{afraidofgirl}) is truly due to a problem with the particular position.

The explanation for this relies on the observation that PPs are invariably \mbox{[--nom]} in \ili{English}. This is straightforward as they cannot be attributes. Consider the examples in (\ref{ladder2}):

\ea \label{ladder2}
\ea	[]{The ladder is [\textsubscript{PP} behind the house].}
\ex	[*]{The [\textsubscript{PP} behind the house] ladder is green.} \label{behindthehouseladder}
\ex	[]{The ladder [\textsubscript{PP} behind the house] is green.} \label{postnompp}
\z
\z

As can be seen, the \isi{PP} \textit{behind the house} can naturally appear in a \isi{predicative position} but is excluded as the attribute of the \isi{noun} \textit{ladder}, as shown in (\ref{behindthehouseladder}). However, it is grammatical for the \isi{PP} to appear post-nominally, as in (\ref{postnompp}).

One apparent counterexample is the case of \textit{inside}, which can appear as an attribute, see (\ref{insideattr2}):

\ea \label{insideattr2}
\ea	The robbery was an inside job.
\ex	He was keen to get an inside look.
\z
\z

However, \textit{inside} in these cases is an \isi{adjective} and not a \isi{preposition}. The availability of \textit{inside} as an \isi{adjective} is demonstrated by the possibility of comparative and superlative forms, see (\ref{moreinside2}):

\ea \label{moreinside2}
\ea	The trip gave us a more inside look at the area.
\ex	The guide promised to give us the most inside look at the area.
\z
\z

The question arises whether PPs could function as attributes at all. Interestingly, \ili{Hungarian} postpositional phrases seem to allow this. Consider the examples in (\ref{letra2}):

\ea \label{letra2}
\ea	[]{\gll A	létra	[\textsubscript{PP}	a	ház	mögött]	van. \label{hungpppred}\\
the	ladder {} the	house	behind	is.\\
\glt `The ladder is behind the house.'}
\ex [*]{\gll A	[\textsubscript{PP}	ház	mögött]	létra	zöld. \label{hungppattr}\\
the {} house behind	ladder green\\
\glt `The ladder behind the house is green.'}
\z
\z

In (\ref{hungpppred}), the \isi{PP} \textit{a ház mögött} `behind the house', headed by the postposition \textit{mögött} `behind', is in a \isi{predicative position}. By \isi{contrast}, in (\ref{hungppattr}) it appears as an attribute within the \isi{nominal expression}, and the result is ungrammatical. The only possibility for the \isi{PP} to appear in an \isi{attributive position} is when it is embedded in a phrase headed by the suffix -\textit{i}:\footnote{Note that the suffix -\textit{i} is attached to the entire \isi{PP}, not only to the P head. As pointed out by \citet[163]{kenesei1995}, the -\textit{i} suffix derives an \isi{AP} from the \isi{PP} but the attachment of this suffix to a bare P head would be ungrammatical, as shown in (\ref{behindladder2}):

\ea	[*]{\gll a	mögött-i	létra \label{behindladder2}\\
the	behind-\textsc{aff}	ladder\\
\glt `the ladder behind'}
\z

The reason for this is that the P head must have a complement and cannot stand on its own. If the -\textit{i} suffix were attached to the P head directly, however, then the string \textit{mögötti} would be an \isi{adjective} as such and should be allowed to appear as a \isi{modifier}. Since this is not the case, it should be clear that the suffix -\textit{i} is attached to the entire \isi{PP}.}

\ea \gll	A	[\textsubscript{XP} [\textsubscript{PP}	ház	mögött]	-i]	létra	zöld. \label{ppaffix}\\
the	{} {} house	behind	\textsc{aff}	ladder	green\\
\glt `The ladder behind the house is green.'
\z

I will not venture to examine the exact status of the suffix -\textit{i} here; suffice it to say that PPs in themselves cannot function as attributes in \ili{Hungarian} either. In any case, the point is that in \ili{English}, there is no construction such as (\ref{ppaffix}) available for PPs either, and therefore PPs in \ili{English} are never attributive in nature.

The problem regarding the position of attributive APs taking \isi{PP} complements is also indicated by \ili{German} \isi{word order} differences (cf. \citealt[202]{haider1985}), as was partly discussed in connection with \citet{lechner1999diss, lechner2004}. Consider the examples in (\ref{lisastolz2}):

\ea \label{lisastolz2}
\ea []{\gll Lisa	ist	(wirklich) stolz	[\textsubscript{PP}	auf	ihren Mann.] \label{predstolzpp}\\
Liz	is really	proud	{} of	her.\textsc{m.acc} husband\\
\glt `Liz is (really) proud of her husband.'}
\ex []{\gll Lisa	ist	[\textsubscript{PP}	auf	ihren	Mann]	(wirklich) stolz. \label{predppstolz}\\
Liz	is {}	of	her.\textsc{m.acc}	husband	really proud\\
\glt `Liz is (really) proud of her husband.'}
\ex	[]{\gll Die	[\textsubscript{PP}	auf	ihren	Mann]	stolze Frau ist Lisa. \label{attrppstolz}\\
the.\textsc{f} {}	of her.\textsc{m.acc}	husband	proud.\textsc{f} woman is Liz\\
\glt `The woman proud of her husband is Liz.'}
\ex	[*]{\gll Die stolze	[\textsubscript{PP}	auf	ihren	Mann] Frau	ist	Lisa. \label{attrstolzpp}\\
the.\textsc{f}	proud.\textsc{f} {} of	her.\textsc{m.acc} husband woman is Liz\\
\glt `The woman proud of her husband is Liz.'}
\z
\z

In (\ref{predstolzpp}), the \isi{adjective} \textit{stolz} `proud' takes a \isi{PP} complement and may optionally be modified by an \isi{adverb} such as \textit{wirklich} `really'. In (\ref{predppstolz}), the \isi{adjective} and the \isi{PP} complement appear in the reverse order. Recall that since the \isi{adverb} \textit{wirklich} can intervene between the two, it is obviously not the underlying order. This is crucial because while in predicative structures both orders converge, in the case of attributive adjectives only the inverse order, that is, where the \isi{PP} has moved to the left, is grammatical, as in (\ref{attrppstolz}), and the \isi{adjective} taking its \isi{PP} complement in its \isi{base position} leads to ungrammaticality, as in (\ref{attrstolzpp}).

The reason for all this is that head-complement \isi{agreement} between the \isi{adjective} and its \isi{PP} complement rules out a \isi{feature} mismatch between the head and the \isi{PP}. This makes two important predictions. First, inherently [+nom] adjectives do not take \isi{PP} complements. Second, adjectives that otherwise allow both for [+nom] and [--nom] may take a \isi{PP} complement, but if the \isi{QP} functions as an attribute, the \isi{PP} has to escape from this position prior to \isi{PF} transfer. This is possible in \ili{German}, where the \isi{PP} can be moved to the left. Therefore, the lower \isi{copy} (the complement of the \isi{adjective} head) can be deleted. In \ili{English}, by \isi{contrast}, there is no such \isi{movement} available; as a consequence, PPs cannot be taken by attributive adjectives.

The fact that the behaviour of \isi{PP} arguments is directly linked to the structure of \isi{degree} expressions by way of applying the same features renders an optimal explanation for the interrelated phenomena considered here.

\section{Phases and deletion} \label{sec:2phases}
It seems that \isi{PP} arguments, while not available as complements of adjectives in attributive constructions, may appear together with adjectives in predicative positions without causing further problems for the analysis. However, this is not exactly the case, as the \isi{PP} complement is apparently not adjacent to the \isi{adjective} head:

\ea \label{extrapos}
\ea	[]{Liz is proud enough [\textsubscript{PP} of her husband]. \label{proudenoughpp}}
\ex	[*]{Liz is proud [\textsubscript{PP} of her husband] enough. \label{proudppenough}}
\z
\z

Although the \isi{PP} \textit{of her husband} is clearly the argument of the \isi{adjective} \textit{proud}, it is ungrammatical for it to remain adjacent to the head, as shown by (\ref{proudppenough}). The only grammatical configuration is the one shown in (\ref{proudenoughpp}), where \textit{enough} seems to intervene between the two. Note that the same would be true for a Deg head such as -\textit{er}:

\ea	Liz is prouder [\textsubscript{PP} of her husband] than Mary is. \label{lizprouder2}
\z

In (\ref{lizprouder2}), the \isi{adjective} \textit{proud}, which moves up to the specifier of the \isi{DegP}, is again not adjacent to its original complement \isi{PP}.

Though it may be tempting to analyse constructions like (\ref{proudenoughpp}) as the results of \isi{rightward movement} of the \isi{PP}, the phenomenon can actually be explained by phase theory. Phases are derived syntactic objects, which are transferred to the interfaces as such (\citealt[9]{chomsky2008}). Therefore, phases may be spelt out separately. However, there are two important rules to be observed here. First, the phases spelt out the earliest will appear last in the \isi{PF} order; and second, phases that are already spelt out become opaque, that is, invisible for syntax (\citealt{chomsky2001, chomsky2004, chomsky2008}; \citealt{nissenbaum2000diss}; \citealt{svenonius2004}; \citealt{kantor2008}).

To illustrate this, let us take the example of the \isi{CP} complement in comparatives, as described by \citet{kantor2008}. Consider the examples in (\ref{tallercp2}):

\ea \label{tallercp2}
\ea	[*]{I saw a taller [\textsubscript{CP} than John] man.} \label{tallerthandp}
\ex	[]{I saw a taller man [\textsubscript{CP} than John].} \label{tallerdpthan}
\ex	[]{I saw [\textsubscript{DP} a [\textsubscript{QP} taller [\textsubscript{CP} {\normalfont opaque}]] man] [\textsubscript{CP} than John].} \label{tallerdpthanpostnom}
\z
\z

In (\ref{tallerthandp}), the \isi{CP} appears adjacent to \textit{taller}, that is, in its \isi{base position} as a complement within the \isi{QP} modifying the \isi{NP} \textit{man}. The result is, however, ungrammatical: the well-formed configuration is shown in (\ref{tallerdpthan}), where the \isi{CP} appears as the rightmost element. \isi{PF} ordering is shown in (\ref{tallerdpthanpostnom}). The \isi{CP} \textit{than John}, as a phase, is spelt out first: hence its rightmost position in the linear structure. Since it is spelt out, it will appear as opaque in the syntactic structure in its \isi{base position} (and will of course not be overt at \isi{PF} either).

There are two observations to be made here. First, the order of \isi{spell-out} is not completely independent from the order of merge. If a phase-sized XP is merged into the structure earlier than a phase-sized YP, and if the XP can be spelt out earlier than YP is merged, then XP will naturally be spelt out earlier than YP. Second, any XP can be spelt out only if it has checked off its uninterpretable features. This is crucial when dealing with cross-linguistic data. In \ili{Hungarian}, for instance, relative clauses are embedded within a \isi{DP} headed by a matrix pronominal element that is responsible for introducing the \isi{relative clause} into the structure (cf. \citealt[243--248]{ekiss2002}). It is a possible configuration that the \isi{CP} is spelt out earlier but the \isi{DP}, which can for instance be a focus, has features to be checked and cannot be spelt out. This is demonstrated in (\ref{focus2}):

\ea \gll [\textsubscript{DP} Azt [\textsubscript{CP} \emph{opaque}]] felejtsd el, [\textsubscript{CP}	amiről beszéltünk]! \label{focus2}\\
{} that.\textsc{acc} {} {} forget.\textsc{imp.2sg}	off	{} what.\textsc{rel.del} talked.\textsc{1pl}\\
\glt `Forget what we talked about!'
\z

Since the \isi{pronoun} \textit{azt} is focussed but the \isi{subclause} itself is not, they naturally appear as disjoint elements in the linear structure. However, if the \isi{subclause} is interpreted as a \isi{topic}, see (\ref{topic2}), it may move together with the rest of the \isi{DP} (example from \citealt[244, ex. 40a]{ekiss2002}):

\ea \gll	[\textsubscript{DP}	Azt, [\textsubscript{CP} amiről	beszéltünk,]] felejtsd el! \label{topic2}\\
{} that.\textsc{acc} {} what.\textsc{rel.del}	talked.\textsc{1pl} forget.\textsc{imp.2sg}	off\\
\glt `Forget what we talked about!'
\z

I will not examine here the conditions on why and how subordinate CPs may not appear sentence-finally, as it would require a separate and thorough investigation on its own, especially because in several languages, such as \ili{Japanese}, \ili{Korean}, \ili{Chinese} and \ili{Turkish}, there are pre-nominal relative clauses, see \citet{larsontakahashi2007}.

Turning back to the seemingly extraposed PPs in structures like (\ref{extrapos}), the explanation relies on the assumption that PPs can be considered phases too (\citealt{leeschoenfeld2007}; \citealt{drummondhornsteinlasnik2010}; \citealt{gallego2010}; \citealt{fowlie2010}); consequently, they can be spelt out separately. Therefore, what happens in the case of (\ref{extrapos}) can be demonstrated as given in (\ref{lizprouderdeletion2}):

\ea	Liz is proud [\textsubscript{PP} \emph{opaque}] enough [\textsubscript{PP} of her husband]. \label{lizprouderdeletion2}
\z

The \isi{PP} \textit{of her husband}, being a phase, is spelt out first and it appears as the last element in the \isi{PF} ordering. At the same time, it becomes opaque in its \isi{base position} in the syntax.

It has to be stressed that this does not happen in an unrestricted way. The \isi{PP} can be spelt out only if its features are checked off. As should be obvious, [--nom] features cannot be present in an attributive construction, hence a structure like (\ref{ppunchecked}) is ruled out:

\ea *Liz is a proud woman [\textsubscript{PP} of her husband]. \label{ppunchecked}
\z

From this, it follows that separate \isi{spell-out} is not an escape hatch for ungrammatical configurations to converge, but is instead very strictly rule-governed.

A further restriction concerns ordering: the phase spelt out first appears last. This predicts that the order of a comparative \isi{subclause} and a \isi{PP} argument of an \isi{adjective} is fixed, as shown in (\ref{lizcppp2}):

\ea \label{lizcppp2}
\ea	[]{Liz is prouder [\textsubscript{PP} of her husband] [\textsubscript{CP} than Mary is].}
\ex	[*]{Liz is prouder [\textsubscript{CP} than Mary is] [\textsubscript{PP} of her husband].}
\z
\z

Only the order in which the \isi{CP} appears last converges. This is so because the \isi{CP} is merged into the construction earlier than the \isi{PP}, and therefore the \isi{CP} has to be spelt out first.

This shows that though the ordering of various elements largely depends on the order of \isi{PF} transfers, \isi{PF} ordering is ultimately defined by syntax. The present analysis is fairly advantageous to previous ones that neither considered the difference between the base and the surface position of the comparative \isi{subclause}, nor did they apply some kind of \isi{rightward movement}. On the other hand, the apparent \isi{extraposition} of comparative subclauses and \isi{PP} arguments of adjectives can be handled in a similar way, without assuming that they would have the same or even similar positions in the syntax.
