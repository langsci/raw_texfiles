\addchap{Acknowledgements}
\begin{refsection}

This book grew from my dissertation, published by Universitätsverlag Potsdam under the title \textit{The syntax of comparative constructions: Operators, ellipsis phenomena and functional left peripheries} in 2014. Chapters 2--4 and 6 have been revised and updated from the original, while Chapter 5 is entirely new. I owe many thanks to the reviewer and the proofreaders for their constructive comments and suggestions, as well as Sebastian Nordhoff for technical assistance.

The research behind this book was funded by the German Research Fund (DFG), as part of my research project ``The syntax of functional left peripheries and its relation to information structure'' (BA 5201/1-1) and formerly as part of the SFB-632 ``Information structure: The linguistic means for structuring utterances, sentences and texts''. I would like to thank my fellow German taxpayers for making this enterprise possible.

I definitely owe many thanks to many people for assisting me in various ways during the time I was working on this book. I would especially like to thank Gisbert Fanselow and Malte Zimmermann for encouraging me in this endeavour and for helpful discussions during all these years. Extra thanks go to Gisbert for supervising my dissertation back then and for his continued support of my research in my postdoctoral life.

My work, especially regarding ellipsis, benefited a lot from discussions with my late colleague Luis Vicente. It is therefore particularly tragic that I cannot show him the book any more -- I can only hope that the result is something he would have appreciated.

I am indebted to Lisa Baudisch, my research assistant in the project, whose help in evaluating the data from surveys and corpora prevented the valuable empirical basis of the whole enterprise from turning into a nightmare.

Further, I am highly grateful to many others of my present and former colleagues at the University of Potsdam, whose suggestions, ideas and generally inspiring research have truly fostered my own work. In particular, I would like to thank Marta Wierzba for her help in improving my questionnaires and Doreen Georgi for lots of discussions concerning relative clauses. Many thanks are due to Martin Salzmann, Radek Šimík, Boban Arsenijević, Craig Thiersch, Andreas Schmidt, Mira Grubic, Anne Mucha, Agata Renans, Flavia Adani, and Joseph DeVeaugh-Geiss for all their questions and suggestions during these years. I would also like to thank Claudius Klose, Jana Häussler, Frank Kügler, Anja Gollrad, Suse Genzel, Maria Balbach,
Maja Stegenwallner-Schütz, 
Nikos Engonopoulos, 
Thuan Tran, Henry Zamchang Fominyam and Mary Amaechi for their various roles in making this time truly motivating and fun. I also owe many thanks to Jutta Boethke and Elke Pigorsch for their excellent support in administration during my time in the SFB. And, of course, I am truly indebted to Ines Mauer, who had the task of administering my own project.

I would like to thank my students of the seminars ``The syntax of comparative constructions'' and ``Diachronic syntax'', whose original and intelligent questions have been inspirational for my research.

Outside of Potsdam, I owe many thanks to my project cooperation partners for inspiring discussions and their useful suggestions concerning various parts of my research: Ellen Brandner (Konstanz/Stuttgart), 
Katalin É. Kiss (Budapest),
Marco Coniglio (Berlin/Göt\-tin\-gen), 
Agnes Jäger (Köln), Marlies Kluck (Groningen), Svetlana Petrova (Wuppertal) and Helmut Weiß (Frankfurt). I am highly grateful to Jason Merchant for an inspiriting discussion on comparatives at ZAS.

My work has benefited substantially from the comments of anonymous reviewers I received for my papers and conference abstracts submitted during my doctoral project. I also owe many thanks to the audiences of various conferences I attended while I was working on this book, of which I would like to mention the following: ``Linguistic Evidence'' in 2014 in Tübingen (and Alexander Dröge and Ankelie Schippers in particular), ``Third Cambridge Comparative Syntax Conference'' in 2014 in Cambridge (and Ian Roberts and Georg Höhn in particular), ``16th Diachronic Generative Syntax Conference'' in 2014 in Budapest (and Bea\-trice Santorini and Istv\'an Kenesei in particular), ``12th International Conference on the Structure of Hungarian'' in 2015 in Leiden (and Marcel den Dikken in particular), 17th Diachronic Generative Syntax Conference in 2015 in Reykjavík (and Roland Hinterhölzl, Jim Wood and Anthony Kroch in particular), ``Budapest Linguistics Conference'' in 2015 in Budapest (and Moreno Mitrovi\'c, Mojm\'ir Do\v{c}ekal and Kerstin Hoge in particular), ``Categories in Grammar -- Criteria and Limitations'' in 2015 in Berlin (and Horst Simon and Christian Forche in particular), ``12th International Conference on Greek Linguistics'' in 2015 in Berlin, ``SaRDiS 2015: Saarbrücker Runder Tisch für Dialektsyntax'' in 2015 in Saarbrücken (and Augustin Speyer and Oliver Schallert in particular), ``11th European Conference on the Formal Description of Slavic Languages'' in 2015 in Potsdam (and \v{Z}eljko Bo\v{s}kovi\'c and Sergey Avrutin in particular), the workshop ``The Grammatical Realization of Polarity. Theoretical and Experimental Approaches'' of the ``38th Annual Conference of the German Linguistics Society'' in 2016 in Konstanz, the ``39th Generative Linguistics in the Old World'' in 2016 in Göttingen (and Gereon Müller and Jeroen van Craenenbroeck in particular), ``Budapest--Potsdam--Lund Linguistics Colloquium'' in 2016 in Budapest (and Gunlög Josefsson and Lars-Olof Delsing in particular), ``Potsdam Summer School in Historical Linguistics 2016: Word Order Variation and Change: Diachronic Insights into Germanic Diversity'' in 2016 in Potsdam (and Theresa Biberauer in particular), ``Generative Grammatik des Südens 42'' in 2016 in Leipzig (and Hubert Haider and Philipp Weisser in particular), ``SaRDiS 2016: Saarbrücker Runder Tisch für Dialektsyntax'' in 2016 in Saarbrücken (and Ellen Brandner and Göz Kaufmann in particular), ``12th European Conference on the Formal Description of Slavic Languages'' in 2016 in Berlin (and Jiri Kaspar, Teodora Radeva-Bork and Roland Meyer in particular), ``Equative Constructions'' in 2016 in Cologne (and Doris Penka in particular).

I owe many thanks to my informants for their valuable judgements. Many of them have been mentioned above; here I would like to add that I am highly grateful for \L{}ukasz J\c{e}drzejowski and Marta Ruda for their help with the Polish data. I am highly grateful to all my informants for completing my cross-Germanic survey: here I would also like to thank Ida Larsson for her help in finding Norwegian informants and J\'ohannes G\'isli J\'onsson for his help in finding Icelandic informants.

Finally, I would like to thank my friends and my family for their support. In particular, many thanks go to my parents for their love and encouragement ever since I was born and for being the fun people they are. And of course lots of thanks are due to Ralf for everything, such as sharing my passion for photographing Berlin, and generally for making each and every day of my life wonderful. This book is dedicated to him.



\printbibliography[heading=subbibliography]
\end{refsection}

