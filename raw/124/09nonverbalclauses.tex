\chapter[Nonverbal and copular clauses]{Nonverbal and copular clauses}\label{ch:9}
\is{Clause!nominal|(}\section{Introduction}\label{sec:9.1}

This chapter deals with clauses which do not have a lexical verb as predicate. These clauses contain either no verb, an existential verb, or a copula verb. 

The following types can be distinguished and will be discussed in turn: 

\begin{itemize}
\item 
NP NP clauses, i.e. clauses in which both the subject and the predicate are noun phrases (\sectref{sec:9.2});

\item 
existential clauses\is{Clause!existential}, both verbal and non-verbal (\sectref{sec:9.3});

\item 
clauses with a prepositional predicate (\sectref{sec:9.4});

\item 
numerical clauses (\sectref{sec:9.5});

\item 
clauses containing a copula verb (\sectref{sec:9.6}).

\end{itemize}
\section{NP NP clauses}\label{sec:9.2}

When a nominal clause\is{Clause!nominal} consists of two noun phrases, one of them is the subject; for the other noun phrase, there are two possibilities: it may either be referential\is{Referentiality} or non-referential. When the noun phrase is non-referential, it is a true predicate, which gives new information about the subject, expressing that the subject belongs to a certain class. When the non-subject noun phrase is a referential noun phrase, the clause establishes a relation of identity between the two noun phrases, expressing that both are descriptions of the same referent. In this grammar, these two constructions are labelled \textsc{classifying} and \textsc{identifying} clauses\is{Clause!identifying}, respectively.\footnote{\label{fn:459}Various terms are used in the literature. \citet[233]{Dryer2007Clause} distinguishes between “equational clauses” and “true nominal predicate clauses”. The distinction is fundamental in some Polynesian languages; terms used in Pol\is{Clause!equative, equational}ynesian linguistics include: classifying and equative predicates \citep[78]{Bauer1993}, predicational and identificational NPs (\citealt[430]{ChungMason1995}), predicate nominals and equatives (\citealt{DeLacy1999}), class-inclusion and equational sentences \citep[45]{Cook1999}.}

In Rapa Nui, these two types of clauses are distinguished by the use of the predicate marker \textit{he} in classifying clauses\is{Clause!classifying} and the preposition \textit{ko} in identifying clauses\is{Clause!identifying}.

A third type of NP NP clauses, attributive clauses\is{Clause!attributive}, is characterised by the absence of any prenominal marker and the presence of an adjective in the predicate NP.

\subsection{Classifying clauses}\label{sec:9.2.1}
\is{Clause!classifying|(}
In classifying clauses\is{Clause!classifying}, a nominal predicate provides information about the subject by expressing that the subject belongs to a certain class of entities. The predicate is introduced with \textit{he}\is{he (nominal predicate marker)}, which indicates non-referentiality\is{Referentiality} (\sectref{sec:5.3.4}).

The unmarked order\is{Constituent order} in these clauses is Subject–Predicate.

\ea\label{ex:9.1}
\gll A Thor Heyerdahl he científico e tahi.\\
\textsc{prop} Thor Heyerdahl \textsc{ntr} scientist \textsc{num} one\\

\glt 
‘Thor Heyerdahl was a scientist.’ \textstyleExampleref{[R376.007]} 
\z

\ea\label{ex:9.2}
\gll Te toromiro he tumu hauha{\ꞌ}a e tahi. \\
\textsc{art} toromiro \textsc{pred} tree importan \textsc{num} one \\

\glt 
‘The toromiro is an important tree.’ \textstyleExampleref{[R478.053]} 
\z

\ea\label{ex:9.3}
\gll Rā me{\ꞌ}e era pē he tiare he mōrī. \\
\textsc{dist} thing \textsc{dist} like \textsc{pred} flower \textsc{pred} light \\

\glt
‘Those things (that look) like flowers are lights.’ \textstyleExampleref{[R210.199]} 
\z

\is{Constituent order}The predicate may also come first. This happens only when the subject is well-es\-tab\-lished, i.e. topical\is{Topic, topicality} in discourse; it tends to be expressed by a pronoun as in \REF{ex:9.4}, or a generic noun phrase as in \REF{ex:9.5}. In this construction, the predicate is prominent. In \REF{ex:9.5}, for example, the predicate conveys unexpected, surprising information.

\ea\label{ex:9.4}
\gll E ai rō {\ꞌ}ā e tahi taŋata tire, \textbf{he} \textbf{piroto} \textbf{{\ꞌ}avione} a ia. \\
\textsc{ipfv} exist \textsc{emph} \textsc{ident} \textsc{num} one person Chile \textsc{pred} pilot airplane \textsc{prop} \textsc{3sg} \\

\glt 
‘There was one Chilean, he was an airplane pilot.’ \textstyleExampleref{[R378.013]} 
\z

\ea\label{ex:9.5}
\gll \textbf{He} \textbf{taŋata} tau manu era, \textbf{he} \textbf{poki} \textbf{{\ꞌ}a} \textbf{Uho} tau manu era. \\
\textsc{pred} person \textsc{dem} bird \textsc{dist} \textsc{pred} child of\textsc{.a} Uho \textsc{dem} bird \textsc{dist} \\

\glt
‘That bird was a human being, that bird was Uho’s child.’ \textstyleExampleref{[Mtx-7-12.069]}
\z

In \REF{ex:9.6}, Tangaroa (who has transformed himself into a seal, and is mistaken for a seal by the people) wants to emphasise that he is the king, not a real seal as the people think. The predicate \textit{he {\ꞌ}ariki} is counterexpectative and occurs before the subject.\footnote{\label{fn:460}Notice that \textit{ko Taŋaroa}, which is an apposition\is{Apposition} to the predicate, is not fronted but remains in its post-subject position; see sec. \sectref{sec:9.2.5} for more examples of split predicates.} 

\ea\label{ex:9.6}
\gll He raŋi mai te re{\ꞌ}o o te pakia: ‘\textbf{He} \textbf{{\ꞌ}ariki} au ko Taŋaroa’. \\
\textsc{ntr} call hither \textsc{art} voice of \textsc{art} seal \textsc{pred} king \textsc{1sg} \textsc{prom} Tangaroa \\

\glt 
‘The voice of the seal cried: I am king Tangaroa.’ \textstyleExampleref{[Mtx-1-05.008]}
\z

Just as in verbal clauses, the subject of classifying clauses\is{Clause!classifying} may be left out:

\ea\label{ex:9.7}
\gll He aŋi mau {\ꞌ}ā pē nei ē: \textbf{he} \textbf{{\ꞌ}ariki}. \\
\textsc{ntr} true really \textsc{ident} like \textsc{prox} thus \textsc{pred} king \\

\glt 
‘It is true: he is a king.’ \textstyleExampleref{[Fel-46.053]}
\z

\ea\label{ex:9.8}
\gll \textbf{Ta{\ꞌ}e} \textbf{he} \textbf{taŋata}, \textbf{he} \textbf{{\ꞌ}aku{\ꞌ}aku}, pē ira {\ꞌ}ā au. \\
\textsc{conneg} \textsc{pred} person \textsc{pred} spirit like \textsc{ana} \textsc{ident} \textsc{1sg} \\

\glt 
‘That is not a man, it is a spirit, and so am I.’ \textstyleExampleref{[Mtx-7-04.058]}
\z
\is{Clause!classifying|)}

\subsection{Identifying clauses}\label{sec:9.2.2}
\is{ko (prominence marker)!in identifying clauses|(}\is{Clause!identifying|(}
Identifying clauses serve to identify the referent of one noun phrase with the referent of the other noun phrase in the clause. Both NPs are preceded by a \textit{t}{}-determiner (\sectref{sec:5.3.2}) such as the article \textit{te}, indicating that they are referential. In all identifying clauses, one noun phrase is preceded by the prominence marker \textit{ko} (\sectref{sec:4.7.11}). 

A few examples:

\ea\label{ex:9.9}
\gll Te me{\ꞌ}e ena o te pā{\ꞌ}eŋa {\ꞌ}uta ko tō{\ꞌ}oku māmā era. \\
\textsc{art} thing \textsc{med} of \textsc{art} side inland \textsc{prom} \textsc{poss.1sg.o} mother \textsc{dist} \\

\glt 
‘That (person) on the inland side is my mother.’ \textstyleExampleref{[R411.057]} 
\z

\ea\label{ex:9.10}
\gll Pero ko au te suerekao o te hora nei. \\
but \textsc{prom} \textsc{1sg} \textsc{art} governor of \textsc{art} time \textsc{prox} \\

\glt 
‘But I am the governor now (or: the governor now is me).’ \textstyleExampleref{[R201.007]} 
\z

\ea\label{ex:9.11}
\gll Te ŋāŋata mātāmu{\ꞌ}a o Rapa Nui ko te {\ꞌ}ariki era ko Hotu Matu{\ꞌ}a  ananake ko tō{\ꞌ}ona hua{\ꞌ}ai.\\
\textsc{art} men first of Rapa Nui \textsc{prom} \textsc{art} king \textsc{dist} \textsc{prom} Hotu Matu’a  together \textsc{prom} \textsc{poss.3sg.o} family\\

\glt
‘The first people of Rapa Nui were king Hotu Matu’a with his family.’ \textstyleExampleref{[R350.015]} 
\z

Notice that the \textit{ko}{}-marked NP, in the case of a common noun\is{Noun!common}, is always followed by a postnominal demonstrative\is{Demonstrative!postnominal} \textit{nei}, \textit{ena} or \textit{era}; the combination of the article \textit{te} with one of these demonstratives\is{Demonstrative} indicates definiteness\is{Definiteness} (\sectref{sec:4.6.3.1}).

As both noun phrases are referential\is{Referentiality} and definite\is{Definiteness}, and both refer to the same entity, it is not always clear which NP is subject and which is predicate. Constituent order\is{Constituent order} cannot be used as the sole criterion, as both subject and predicate of a nominal clause\is{Clause!nominal} may come first.\footnote{\label{fn:461}See examples (\ref{ex:9.1}–\ref{ex:9.6}) in classifying clauses\is{Clause!classifying}; the same is true in other types of nominal clauses\is{Clause!nominal}, e.g. locative clauses (\sectref{sec:9.4.1}).} It is even questionable whether the term \textit{predicate} is appropriate at all in identifying clauses\is{Clause!identifying} (see \citealt[440]{Anderson2004}): as both noun phrases are referential expressions, they are fundamentally different from predicates, which designate properties or events rather than referring to entities. 

Even so, the terms \textit{subject} and \textit{predicate} may be used in identifying clauses\is{Clause!identifying} in a loose way, in the sense that the subject is the entity to be identified, and the predicate is the identifying expression. In some cases it is clear which NP is the subject, as this NP functions as discourse topic\is{Topic, topicality}. In other cases, however, it is difficult to identify subject and predicate – unless we adopt a simple syntactic definition. As indicated above, in every identifying clause one noun phrase is marked with \textit{ko}, while the other is an unmarked NP. Taking the \textit{ko}{}-marked NP as predicate provides a simple criterion. Moreover, this analysis coincides with the intuitive assignment of subject and predicate in those cases where the distinction is clear: in examples like \REF{ex:9.11}, it is clear that the unmarked NP is subject, while the \textit{ko}-marked NP serves to identify this subject.

In the examples so far, the identifying clause consists of two common noun phrases. When the clause contains a pronoun or proper noun, the use of \textit{ko} is described by the following two rules:

%\setcounter{listWWviiiNumcileveli}{0}
\begin{enumerate}
\item 
If the clause contains a proper noun\is{Noun!proper}, this is always \textit{ko}{}-marked.

\item 
If the clause contains a pronoun\is{Pronoun!personal}, this is usually \textit{ko}{}-marked,\footnote{\label{fn:462}I have not found any exceptions to this rule in the text corpus, though there are a few exceptions in the New Testament translation.} unless the other constituent is a proper noun.

\end{enumerate}

This is illustrated in the following examples.

Common NP\is{Noun!common} + proper noun:

\ea\label{ex:9.12}
\gll Te kona hope{\ꞌ}a o te nehenehe \textbf{ko} \textbf{{\ꞌ}Anakena}. \\
\textsc{art} place last of \textsc{art} beautiful \textsc{prom} Anakena \\

\glt
‘The most beautiful place (of the island) is Anakena.’ \textstyleExampleref{[R350.013]} 
\z

Pronoun + common NP:

\ea\label{ex:9.13}
\gll Pero \textbf{ko} \textbf{au} te suerekao o te hora nei. \\
but \textsc{prom} \textsc{1sg} \textsc{art} governor of \textsc{art} time \textsc{prox} \\

\glt
‘But I am the governor now (or: the governor now is me).’ \textstyleExampleref{[R201.007]} 
\z

Pronoun + proper noun:

\ea\label{ex:9.14}
\gll A au \textbf{ko} \textbf{Omoaŋa}. \\
\textsc{prop} \textsc{1sg} \textsc{prom} Omoanga \\

\glt
‘I am Omoanga.’ \textstyleExampleref{[R314.101]} 
\z

These patterns make sense if we assume that \textit{ko} always marks the predicate. Proper names are inherently highly identifiable\is{Identifiability} (their reference is always unique and unambiguous in a given context), so it is not surprising that they serve as an identifying expression (predicate) rather than as a referent to be identified (subject). The same is true for pronouns. Between proper nouns\is{Noun!proper} and pronouns, the former are identifiable to a higher degree\is{Identifiability}: within a given context, a proper noun has unambiguous unique reference; for a pronoun, more contextual clues may be needed to establish its reference. This can be represented in a \textit{hierarchy of identifiability}\is{Identifiability}:

\ea\label{ex:9.14a}
  proper nouns\is{Noun!proper} {\textgreater} pronouns {\textgreater} common nouns\is{Noun!common}
\z

The idea that \textit{ko} marks the predicate is also confirmed by the fact that an identifying clause may consist of a \textit{ko}-phrase only; this follows from the general rule in Rapa Nui that the predicate is obligatory, while the subject can be omitted:

\ea\label{ex:9.15}
\gll —¿Ko ai koe? —\textbf{Ko} \textbf{au} \textbf{{\ꞌ}ana}. \\
~~~~~\textsc{prom} who \textsc{2sg} ~~~\textsc{prom} \textsc{1sg} \textsc{ident} \\

\glt 
‘—Who are you? —It’s me.’ \textstyleExampleref{[Mtx-7-04.071–072]}
\z

\ea\label{ex:9.16}
\gll —Me{\ꞌ}e era ko Tito. —{\ꞌ}Ēē. \textbf{Ko} \textbf{ia}. \\
~~~thing~ \textsc{dist} \textsc{prom} Tito ~~~yes \textsc{prom} \textsc{3sg} \\

\glt 
‘—That one (in the picture) is Tito. —Yes. It’s him.’ \textstyleExampleref{[R414.163–165]}
\z

In \REF{ex:9.14} above, the pronoun is not marked with \textit{ko} when the other constituent is a proper noun. There are also a few cases in the corpus where a pronoun and a proper noun are both \textit{ko}{}-marked. Two examples are provided below:

\ea\label{ex:9.17}
\gll Ko au ko Totimo. \\
\textsc{prom} \textsc{1sg} \textsc{prom} Totimo \\

\glt 
‘I am Totimo.’ \textstyleExampleref{[R399.193]} 
\z

\ea\label{ex:9.18}
\gll —¿Ko ai koe? —Ko au ko Huri {\ꞌ}Avai. \\
~~~~~\textsc{prom} who \textsc{2sg} ~~~\textsc{prom} \textsc{1sg} \textsc{prom} Huri Avai \\

\glt
‘—Who are you? —I am Huri Avai.’ \textstyleExampleref{[Mtx-3-01.127–128]}
\z

If the pronoun is taken as the subject, these clauses are counterexamples to the claim that only the predicate is marked with \textit{ko}. However, a different analysis is also possible: the pronoun can be analysed as the predicate (with implicit subject), with the proper noun added as apposition\is{Apposition}, ‘It’s me, Totimo’. In both examples above this analysis is plausible. In \REF{ex:9.17}, for example, the situation is as follows: there is a blind girl, Mahina Tea, who knows a boy called Totimo. Totimo walks up to her, embraces her and utters the clause quoted here. An analysis as predicate + apposition\is{Apposition} is appropriate here.\footnote{\label{fn:463}This analysis is reinforced by the fact that in some cases the two constituents are separated by a comma:
\ea \gll 
Ko au, ko Hotu {\ꞌ}Iti te Mata{\ꞌ}iti {\ꞌ}a Hotu Matu{\ꞌ}a.\\
  \textsc{prom} \textsc{1sg} \textsc{prom} Hotu Iti te Mata’iti of\textsc{.a} Hotu Matu’a\\
  \glt 
  ‘It’s me, Hotu Iti te Mata’iti, son of Hotu Matu’a.’ (Ley-2-08.025)\z } 

In other cases this analysis is less plausible, as in the following exchange:

\ea\label{ex:9.19}
\gll —¿Ko ai koe? ... —¡\textbf{Ko} \textbf{au} nei \textbf{ko} \textbf{Vaha} ko to{\ꞌ}o i a Huri {\ꞌ}a Vai! ... —¡{\ꞌ}E \textbf{ko} \textbf{au} nei \textbf{ko} \textbf{Kaiŋa} ko to{\ꞌ}o i a Vaha! \\
~~~~~\textsc{prom} who \textsc{2sg}  ~ ~~~\textsc{prom} \textsc{1sg} \textsc{prox} \textsc{prom} Vaha \textsc{prom} take \textsc{acc} \textsc{prop} huri a Vai  ~ ~~~~and \textsc{prom} \textsc{1sg} \textsc{prox} \textsc{prom} Kainga \textsc{prom} take \textsc{acc} \textsc{prop} Vaha \\

\glt
‘—Who are you? —I am Vaha, who takes (=kills) Huri a Vai! —And I am Kainga, who takes Vaha!’ \textstyleExampleref{[R304.97-101]}
\z

Especially in the last clause, an appositional analysis doesn’t appear to be appropriate. Possibly, these constructions can be analysed as topic + comment constructions (\sectref{sec:8.6.1.3}): ‘(As for) me, I’m Kainga.’\footnote{\label{fn:464}\citet[47]{DeLacy1999} discusses cases in \ili{Māori} where both constituents are \textit{ko}-marked; these are different in that both constituents are a (long) common noun phrase. This enables De Lacy to analyse these as clefts, i.e. biclausal constructions.} 

\subsection{Comparing classifying and identifying clauses}\label{sec:9.2.3}

In the examples of classifying clauses in \sectref{sec:9.2.1} above, the predicate NP clearly indicates that the subject belongs to a certain class of entities; the subject is part of a category described by the predicate. 

In some cases however, the class of entities described by the predicate has only one member, i.e. this class coincides with the referent of the subject. This is illustrated in the following examples:

\ea\label{ex:9.20}
\gll A Tiki \textbf{he} \textbf{poki} o te ra{\ꞌ}ā {\ꞌ}e \textbf{he} \textbf{{\ꞌ}atua} \textbf{rahi} o rāua. \\
\textsc{prop} Tiki \textsc{pred} child of \textsc{art} sun and \textsc{pred} god great of \textsc{3pl} \\

\glt 
‘Tiki was the son of the sun and their high God.’ \textstyleExampleref{[R376.027]} 
\z

\ea\label{ex:9.21}
\gll A au \textbf{he} \textbf{pū{\ꞌ}oko} \textbf{o} \textbf{Rapa} \textbf{Nui} {\ꞌ}i te ao ta{\ꞌ}ato{\ꞌ}a. \\
\textsc{prop} \textsc{1sg} \textsc{pred} head of Rapa Nui at \textsc{art} world all \\

\glt 
‘I am the head (leader) of Rapa Nui in the whole world.’ \textstyleExampleref{[R648.290]} 
\z

\ea\label{ex:9.22}
\gll A ia \textbf{he} \textbf{matu{\ꞌ}a} \textbf{tane} o tō{\ꞌ}oku matu{\ꞌ}a vahine. \\
\textsc{prop} \textsc{3sg} \textsc{pred} parent male of \textsc{poss.1sg.o} parent female \\

\glt
‘He is the father of my mother.’ \textstyleExampleref{[R487.040]} 
\z

These clauses are very similar in sense to identifying clauses\is{Clause!identifying}, which express that two noun phrases have identical reference (\sectref{sec:9.2.2}). In fact, in most examples above, the predicate is translated with a definite noun phrase in \ili{English}, which is characteristic of an identifying clause. Some examples of identifying clauses\is{Clause!identifying} are very similar to the classifying clauses above:

\ea\label{ex:9.23}
\gll He {\ꞌ}ite ia Tu{\ꞌ}u Koihu {\ꞌ}i tū hora era tū ŋā nu{\ꞌ}u era  \textbf{ko} \textbf{tū} \textbf{ŋā} \textbf{tahutahu} \textbf{era}.\\
\textsc{ntr} know then Tu’u Koihu at \textsc{dem} time \textsc{dist} \textsc{dem} \textsc{pl} people \textsc{dist}  \textsc{prom} \textsc{dem} \textsc{pl} witch \textsc{dist}\\

\glt 
‘At that moment Tu’u Koihu knew that those people (whom he saw) were those witches.’ \textstyleExampleref{[R233.023]} 
\z

\ea\label{ex:9.24}
\gll Te hau era, ho{\ꞌ}i, e hī era \textbf{ko} \textbf{te} \textbf{ŋā} \textbf{hau} \textbf{hiro} \textbf{era} e toru kave. \\
\textsc{art} cord \textsc{dist} indeed \textsc{ipfv} to\_fish \textsc{dist} \textsc{prom} \textsc{art} \textsc{pl} cord braid \textsc{dist} \textsc{num} three fibre \\

\glt
‘(in a description of fishing techniques:) The lines they fished with, were those lines braided with three strands.’ \textstyleExampleref{[R539-1.218]}
\z

These examples show that the choice between the two constructions in Rapa Nui is not determined by the criterion of uniqueness, that is, whether or not the predicate defines a single unique entity.\footnote{\label{fn:465}\citet{Lyons1999} mentions uniqueness as one of the necessary conditions for definiteness\is{Definiteness}. Uniqueness is defined as: “there is only one entity satisfying the description used, relative to the context.”} Rather, classifying constructions serve to \textit{describe} the subject by giving new information about it, while identifying clauses\is{Clause!identifying} serve to \textit{identify} a referent with an entity already known to the hearer. The referent of the identifying noun phrase must be accessible\is{Accessibility} to the hearer, otherwise a classifying construction with \textit{he} is used.

To give an example, in the context preceding \REF{ex:9.20} above, there has been no mention of the son of the sun and the high God, so the hearer does not necessarily know that there is such a person as the child of the sun, or that the people in the story had a high God at all. Therefore, this person is not accessible to the hearer. By contrast, in \REF{ex:9.23}, ‘those witches’ refers to witches who have been mentioned earlier in the story; the identifying clause enables the hearer to identify this known entity with the subject ‘those people’. Similarly, in \REF{ex:9.24} the speaker refers to a type of fishing line which he assumes to be known by the hearer (even though it has not been mentioned in the text itself).

The referent of a noun phrase in an identifying clause must not only be unique and accessible\is{Accessibility}, it also needs to be a specific\is{Specific reference}, bounded entity. In the following two examples, the predicate noun phrase could be considered as unique and accessible; nevertheless, it is marked with \textit{he}\is{he (nominal predicate marker)}, i.e. the construction is a classifying clause\is{Clause!classifying}. In \REF{ex:9.25}, the noun phrase refers to priests in general, not to any specific priest. Likewise, in \REF{ex:9.26}, the reference is to adults in general.\footnote{\label{fn:466}These examples are cleft\is{Cleft} constructions, which are discussed in more detail in sec. \sectref{sec:9.2.6} below.}

\ea\label{ex:9.25}
\gll \textbf{He} \textbf{ivi{\ꞌ}atua} \textbf{nō} te kope era e puā era {\ꞌ}i te ta{\ꞌ}u era i a ia te {\ꞌ}ao. \\
\textsc{pred} priest just \textsc{art} person \textsc{dist} \textsc{ipfv} touch \textsc{dist} at \textsc{art} year \textsc{dist} at \textsc{prop} \textsc{3sg} \textsc{art} reign \\

\glt 
‘The priest was the only person (lit. only the priest was the person) who would touch him (te bird man) in the year in which he reigned.’ \textstyleExampleref{[R641.008]} 
\z

\ea\label{ex:9.26}
\gll Te ŋā poki ko {\ꞌ}ite {\ꞌ}ana o ruŋa i te me{\ꞌ}e ta{\ꞌ}ato{\ꞌ}a o te naonao {\ꞌ}e \textbf{he} \textbf{pa{\ꞌ}ari} te me{\ꞌ}e i ta{\ꞌ}e {\ꞌ}ite. \\
\textsc{art} \textsc{pl} child \textsc{prf} know \textsc{cont} of above at \textsc{art} thing all of \textsc{art} mosquito and \textsc{pred} adult \textsc{art} thing \textsc{pfv} \textsc{conneg} know \\

\glt 
‘The children know everything about the mosquitoes, and the adults are the ones who don’t know.’ \textstyleExampleref{[R535.159]} 
\z

We may conclude that a nominal predicate designating an accessible\is{Accessibility}, individuated, bounded entity is marked with \textit{ko}; in all other cases, a classifying construction with \textit{he}\is{he (nominal predicate marker)} is used. A similar difference between \textit{ko} and \textit{he} can be observed with topicalisation in verbal clauses (\sectref{sec:8.6.2}).

As the examples above show, classifying predicates always consist of a common noun phrase. Proper nouns\is{Noun!proper} and pronouns never serve as a classifying predicate; in fact, they are never preceded by \textit{he}. This is to be expected, as proper nouns\is{Noun!proper} and pronouns by definition qualify as identifying predicates: they have unique reference, are accessible in the context, and refer to a specific, bounded entity. On the other hand, they do not designate a class of entities, hence are not suitable as classifying predicates.

\subsection{Constituent order in identifying clauses}\label{sec:9.2.4}
\is{Constituent order}\is{Clause!identifying}
The order of constituents in an identifying clause depends to some extent on the type of noun phrases involved. When both constituents are common noun phrases, the predicate usually occurs after the subject, as illustrated in (\ref{ex:9.23}–\ref{ex:9.24}) and \REF{ex:9.11} above. The predicate may come first when it conveys significant and possibly surprising information as in \REF{ex:9.27}, or when it is a discourse topic established in the preceding context as in \REF{ex:9.28}.\footnote{\label{fn:467}Cf. \citet{Levinsohn2007}: prominence may involve both new information (focal prominence) and established information (thematic\is{Thematicity} prominence).}

\ea\label{ex:9.27}
\gll Ta{\ꞌ}e he {\ꞌ}atua tau me{\ꞌ}e era, he taŋata; \textbf{ko} \textbf{te} \textbf{ŋā} \textbf{io} \textbf{era} \textbf{{\ꞌ}Āmai} tau me{\ꞌ}e era.\\
\textsc{conneg} \textsc{pred} god \textsc{dem} thing \textsc{dist} \textsc{pred} person \textsc{prom} \textsc{art} \textsc{pl} youngster \textsc{dist} Amai \textsc{dem} thing \textsc{dist}\\

\glt 
‘These beings are not gods, they are men; these beings are the Amai guys.’ \textstyleExampleref{[Mtx-7-37.029]}
\z

\ea\label{ex:9.28}
\gll \textbf{Ko} \textbf{te} \textbf{me{\ꞌ}e} \textbf{nei} te me{\ꞌ}e u{\ꞌ}i rahi o te mu{\ꞌ}a {\ꞌ}ā. \\
\textsc{prom} \textsc{art} thing \textsc{prox} \textsc{art} thing see much of \textsc{art} front \textsc{ident} \\

\glt
‘This (= the difficulties mentioned in the previous clause) was something often seen in the past.’ \textstyleExampleref{[R107.009]} 
\z

When the identifying clause contains a pronoun (whether subject or predicate), this is always in an initial position, as illustrated in (\ref{ex:9.13}–\ref{ex:9.14}) above. 

When the clause consists of a proper noun and a common noun phrase, they may occur in either order, as the following examples show. Putting the predicate before the subject gives it more prominence\is{Prominence}. In \REF{ex:9.31}, for example, the predicate \textit{ko Korikē} is contrasted with other persons. In \REF{ex:9.32}, Anakena is singled out between other places on the island.

\ea\label{ex:9.29}
\gll Te kona hope{\ꞌ}a o te nehenehe \textbf{ko} \textbf{{\ꞌ}Anakena}. \\
\textsc{art} place last of \textsc{art} beautiful \textsc{prom} Anakena \\

\glt 
‘The most beautiful place is Anakena.’ \textstyleExampleref{[R350.013]} 
\z

\ea\label{ex:9.30}
\gll Te matu{\ꞌ}a o Hotu Matu{\ꞌ}a \textbf{ko} \textbf{Ta{\ꞌ}ane} \textbf{Arai}. \\
\textsc{art} parent of Hotu Matu’a \textsc{prom} Ta’ane Arai \\

\glt 
‘The father of Hotu Matu’a was Ta’ane Arai.’ \textstyleExampleref{[Ley-2-01.003]}
\z

\ea\label{ex:9.31}
\gll ¿\textbf{Ko} \textbf{Korikē} te me{\ꞌ}e nei {\ꞌ}o ko Titata? ... ¿\textbf{Ko} \textbf{Titata} te me{\ꞌ}e nei? \\
~\textsc{prom} Korike \textsc{art} thing \textsc{prox} or \textsc{prom} Titata  ~ ~\textsc{prom} Titata \textsc{art} thing \textsc{prox} \\

\glt 
‘(pointing at someone in a picture:) Is this Korike or Titata? Is it Titata?’ \textstyleExampleref{[R415.568–572]}
\z

\ea\label{ex:9.32}
\gll \textbf{Ko} \textbf{{\ꞌ}Anakena} \textbf{mau} \textbf{nō} te kona kai māuiui {\ꞌ}ā e noho mai ena. \\
\textsc{prom} Anakena really just \textsc{art} place \textsc{neg.pfv} sick \textsc{cont} \textsc{ipfv} stay hither \textsc{med} \\

\glt 
‘Anakena was the only place where the people who lived there did not get sick.’ \textstyleExampleref{[R231.098]}\textstyleExampleref{} 
\z

\subsection{Split predicates}\label{sec:9.2.5}
\is{Constituent order}\is{Split predicate|(}
When a clause has a pronominal subject and the predicate comes first, certain postnominal modifiers of the predicate are placed after the subject. In \REF{ex:9.33}, \textit{\mbox{ō{\ꞌ}ou}} is a postnominal possessive modifying the predicate \textit{repahoa}; it is separated from the nucleus by the subject \textit{au}. 

\ea\label{ex:9.33}
\gll He repahoa nō \textbf{au} ō{\ꞌ}ou. \\
\textsc{pred} friend just \textsc{1sg} \textsc{poss.2sg.o} \\

\glt
‘I am just your friend.’ \textstyleExampleref{[R308.032]} 
\z

This predicate split is obligatory; clauses like the following do not occur:

\ea\label{ex:9.34}
\textit{*He repahoa nō ō{\ꞌ}ou au.}
\z

As discussed in \sectref{sec:4.6.6}, this process also takes place when the subject is a demonstrative pronoun; this is illustrated in (\ref{ex:9.35}–\ref{ex:9.36}) below. The stranded\is{Stranding} element is often a possessive as in \REF{ex:9.33}; it may also be a numeral as in \REF{ex:9.35}, or a relative clause\is{Clause!relative} as in \REF{ex:9.36}. While in \REF{ex:9.36} the relative clause as a whole is separated from the head noun \textit{famiria}, in \REF{ex:9.37} the relative clause itself is split up: the verb phrase\is{Clause!relative} (\textit{aŋa mau {\ꞌ}ā}) moves along with the head noun, while the direct object is stranded after the subject (\sectref{sec:11.4.5} on raising of relative clause\is{Clause!relative} verbs). 

\ea\label{ex:9.35}
\gll He {\ꞌ}a{\ꞌ}amu nō \textbf{nei} e tahi...\\
\textsc{pred} story just \textsc{prox} \textsc{num} one\\

\glt 
‘This (what follows) is a story...’ \textstyleExampleref{[Luke 11:5]}
\z

\ea\label{ex:9.36}
\gll Famiria hope{\ꞌ}a \textbf{rā} oho mai mai kampō, mai {\ꞌ}Anakena. \\
family last \textsc{dist} go hither from countryside from Anakena \\

\glt 
‘This was the last family who came from the countryside, from Anakena.’ \textstyleExampleref{[R413.889]} 
\z

\ea\label{ex:9.37}
\gll He vi{\ꞌ}e {\ob}aŋa mau {\ꞌ}ā\,{\cb} \textbf{a} \textbf{ia} {\ob}i te me{\ꞌ}e haŋa o te {\ꞌ}Atua\,{\cb}. \\
\textsc{pred} woman {\db}do really \textsc{ident} \textsc{prop} \textsc{3sg} {\db}\textsc{acc} \textsc{art} thing want of \textsc{art} God \\

\glt
‘She is a woman who really does the things God wants.’ \textstyleExampleref{[1 Tim. 5:10]}
\z

Split constituents also occur around the particle \textit{ia}\is{ia ‘then’} ‘then’ (\sectref{sec:4.5.4.1}), which occurs after the first constituent of the clause; postnuclear elements occur after \textit{ia}:

\ea\label{ex:9.38}
\gll Te matu{\ꞌ}a vahine \textbf{ia} o Hēmi he ha{\ꞌ}amata he mana{\ꞌ}u.... \\
\textsc{art} parent female then of Hemi \textsc{ntr} begin \textsc{ntr} think \\

\glt 
‘Then Hemi’s mother started to think...’ \textstyleExampleref{[R476.042]} 
\z

\ea\label{ex:9.39}
\gll {\ꞌ}I tu{\ꞌ}a \textbf{ia} o Kālia e tahi io {\ꞌ}ā{\ꞌ}ana i ohu atu...\\
at back then of Kalia \textsc{num} one young\_man \textsc{poss.3sg.a} \textsc{pfv} shout away\\

\glt 
‘Behind Kalia, one young man of her (family) shouted...’ \textstyleExampleref{[R345.084]} 
\z

\citet[119–120]{Clark1976} analyses this process as extraposition of the second constituent of the predicate over the subject. Alternatively, the split can be described as movement of the predicate with stranding\is{Stranding} of the postnominal modifier; fronting of a constituent is a common process (both crosslinguistically and in Rapa Nui), while it is difficult to see why a modifier would be moved to the right.
\is{Split predicate|)}
\subsection{Clefts}\label{sec:9.2.6}
\is{Cleft|(}
A cleft\is{Cleft} construction consists of two noun phrases, one of which is a simple noun phrase, while the other contains a relative clause\is{Clause!relative}, often without head noun \citep[278]{Payne1997}. Clefts are formally identifying clauses\is{Clause!identifying} – their main constituents are two coreferential NPs – but they express an event or action; the latter is relegated to the relative clause\is{Clause!relative}. The effect of a cleft\is{Cleft} construction is to put the simple NP in focus\is{Focus}.

In Rapa Nui cleft\is{Cleft} constructions, the simple NP comes first and is marked with \textit{ko}, as is expected with predicates of identifying clauses (\sectref{sec:9.2.2}). The second noun phrase contains an anchor noun functioning as head of the relative clause\is{Clause!relative}; this is either a repetition of the noun in focus, or a generic noun like \textit{me{\ꞌ}e}\is{mee ‘thing’@me{\ꞌ}e ‘thing’} ‘thing’. The cleft\is{Cleft} construction is thus similar to the \ili{English} construction ‘Mary was the one who won’,\footnote{\label{fn:468}Cleft constructions of the type ‘X was the one who...’ are often called pseudo-clefts (\citealt[279]{Payne1997}; \citealt[9]{Bauer1991} for \ili{Māori}). On the question whether Rapa Nui also has “real” clefts, i.e. without anchor noun, see sec. \sectref{sec:8.6.2.1}.} though a noun is used instead of ‘one’ and there is no copula verb. As in all relative clauses\is{Clause!relative}, the verb is usually marked with \textit{i}, \textit{e} or unmarked (\sectref{sec:11.4.3}).

A few examples:

\ea\label{ex:9.40}
\gll Ko te nūna{\ꞌ}a era {\ꞌ}a {\ꞌ}Ōrare {\ob}te nūna{\ꞌ}a i rē\,{\cb}. \\
\textsc{prom} \textsc{art} group \textsc{dist} of\textsc{.a} Orare {\db}\textsc{art} group \textsc{pfv} win \\

\glt 
‘(in a report about a music contest:) Orare’s group was the group that won.’ \textstyleExampleref{[R539-3.313]}
\z

\ea\label{ex:9.41}
\gll Ko te ŋā me{\ꞌ}e nei {\ob}te me{\ꞌ}e mo ai o te taŋata  mo oho mo ruku o te hora nei\,{\cb}.\\
\textsc{prom} \textsc{art} \textsc{pl} thing \textsc{prox} {\db}\textsc{art} thing for exist of \textsc{art} person  for go for dive of \textsc{art} time \textsc{prox}\\

\glt 
‘These things (which have just been listed) are the things that people need to go diving nowadays.’ \textstyleExampleref{[R360.002]} 
\z

\ea\label{ex:9.42}
\gll Ko mātou nō {\ob}te me{\ꞌ}e noho o nei\,{\cb}. \\
\textsc{prom} \textsc{1pl.excl} only {\db}\textsc{art} thing stay of \textsc{prox} \\

\glt 
‘(in the description of a house:) We are the only ones living here.’ \textstyleExampleref{[R404.050]} 
\z

\ea\label{ex:9.43}
\gll Ko Timo {\ob}te me{\ꞌ}e {\ꞌ}ori tako{\ꞌ}a o roto nei\,{\cb}. \\
\textsc{prom} Timo {\db}\textsc{art} thing dance also of inside \textsc{prox} \\

\glt
‘Timo is the one who is also dancing inside (= in this picture).’ \textstyleExampleref{[R414.129]} 
\z

The effect of relegating the verb to a relative clause\is{Clause!relative} is that the initial noun phrase is in focus\is{Focus}, while the event or action is backgrounded. Clefts are used when the event or action as such is presupposed; it has already been mentioned as in \REF{ex:9.41}, or can be inferred from the context: in \REF{ex:9.40}, the context of a musical contest presupposes that there is a winner, while the important new information is the identity of the winner. The act of winning is therefore backgrounded, while the noun phrase referring to the winner is put in focus.

The examples so far represent the most common construction, in which clefts are constructed as identifying clauses\is{Clause!identifying} with a \textit{ko}{}-marked predicate. Clefts may also be classifying clauses\is{Clause!classifying}, with a \textit{he}{}-marked predicate. As discussed in \sectref{sec:9.2.1}, identifying clauses\is{Clause!identifying} are used when the predicate refers to a unique individual which is accessible to the hearer; in other cases, classifying clauses\is{Clause!classifying} are used. This happens for example when the noun phrase is generic:

\ea\label{ex:9.44}
\gll Te ŋā poki ko {\ꞌ}ite {\ꞌ}ana o ruŋa i te me{\ꞌ}e ta{\ꞌ}ato{\ꞌ}a o te naonao {\ꞌ}e \textbf{he} \textbf{pa{\ꞌ}ari} {\ob}te me{\ꞌ}e i ta{\ꞌ}e {\ꞌ}ite\,{\cb}. \\
\textsc{art} \textsc{pl} child \textsc{prf} know \textsc{cont} of above at \textsc{art} thing all of \textsc{art} mosquito and \textsc{pred} adult {\db}\textsc{art} thing \textsc{pfv} \textsc{conneg} know \\

\glt 
‘The children know everything about the mosquitoes, and the adults are the ones who don’t know.’ \textstyleExampleref{[R535.159]} 
\z

Classifying\is{Clause!classifying} cleft\is{Cleft} constructions are especially common with the verb \textit{haŋa} ‘want’ and other expressions of volition\is{Verb!volition}/desire (\sectref{sec:3.2.3.1.1} on the nominal tendency of volition verbs). With these verbs, the noun phrase does not contain a full relative clause\is{Clause!relative}, but a bare modifying verb, such as \textit{haŋa} in \REF{ex:9.45}; if the subject of this verb is expressed, it is a possessive pronoun (\textit{tā{\ꞌ}aku} in \REF{ex:9.45}) or a genitive phrase (\sectref{sec:11.4.4}):

\ea\label{ex:9.45}
\gll \textbf{He} \textbf{kāpē} tā{\ꞌ}aku me{\ꞌ}e haŋa. \\
\textsc{pred} coffee \textsc{poss.1sg.a} thing want \\

\glt 
‘Coffee is what I want (lit. my thing want).’ \textstyleExampleref{[R221.024]} 
\z

\ea\label{ex:9.46}
\gll Mō{\ꞌ}ona te me{\ꞌ}e manava mate \textbf{he} \textbf{hoi} \textbf{eke}... \\
\textsc{ben.3sg.o} \textsc{art} thing stomach die{\rmfnm} \textsc{pred} horse climb \\

\glt 
‘For him, the thing (he) liked most was climbing his horse (and going around the island).’ \textstyleExampleref{[R439.008]} 
\z
\footnotetext{\textit{Manava mate} is an idiom expressing love or endearment.}

Clefts also occur in questions, when a verb argument is questioned: identifying clefts with \textit{ko ai}\is{ai ‘who’} ‘who’ (\sectref{sec:10.3.2.1}), classifying clefts with \textit{he aha}\is{aha ‘what’} ‘what’ (\sectref{sec:10.3.2.2}).

As discussed in \sectref{sec:8.6.3}, the actor-emphatic\is{Actor-emphatic construction} (AE) construction also serves to put a noun phrase in focus\is{Focus}. It is not entirely clear which conditions determine the choice between an AE construction and a cleft\is{Cleft}. However, AE’s are only used to put agentive subjects in focus; in order to put non-agentive subjects in focus as in \REF{ex:9.41} or non-subjects as in \REF{ex:9.45}, only clefts can be used.
\is{ko (prominence marker)!in identifying clauses|)}\is{Clause!identifying|)}\is{Cleft|)}

\subsection{Attributive clauses}\label{sec:9.2.7}
\is{Clause!attributive|(}
In an attributive clause, an inherent – and usually permanent – property is attributed to the subject.\footnote{\label{fn:470}Non-permanent properties are expressed as verbal predicates, see sec. \sectref{sec:3.5.1.5}.} This property is in most cases expressed as an adjective. Now an adjective as such cannot serve as a nominal predicate in Rapa Nui, and therefore an anchor noun is needed to fit the adjective into the syntactic structure. This anchor noun is either identical to the subject noun or a generic noun like \textit{me{\ꞌ}e}\is{mee ‘thing’@me{\ꞌ}e ‘thing’} ‘thing’.\footnote{\label{fn:471}In related languages, cognates of \textit{me{\ꞌ}e} also serve as anchor noun for adjectival or verbal predicates; see e.g. \citet[38]{LazardPeltzer2000} on \ili{Tahitian}.} 

The predicate may be marked with \textit{he}\is{he (nominal predicate marker)} as in \REF{ex:9.47}, in which case the clause is a classifying clause\is{Clause!classifying} (\sectref{sec:9.2.1}). This is rare, though; usually the predicate is a bare noun phrase, lacking any determiner.

Below are some examples, with the anchor noun emphasised.

With repetition of the subject noun:

\ea\label{ex:9.47}
\gll Te {\ꞌ}ati ena o te kahu {\ꞌ}i rā noho iŋa \textbf{he} \textbf{{\ꞌ}ati} nuinui e tahi. \\
\textsc{art} problem \textsc{med} of \textsc{art} clothes at \textsc{dist} stay \textsc{nmlz} \textsc{pred} problem big:\textsc{red} \textsc{num} one \\

\glt 
‘The problem of clothing at the time was a big one.’ \textstyleExampleref{[R380.093]} 
\z

\ea\label{ex:9.48}
\gll \textbf{Taŋata} {\ꞌ}uri{\ꞌ}uri te taŋata nei {\ꞌ}e \textbf{taŋata} rakerake. \\
person black:\textsc{red} \textsc{art} person \textsc{prox}\textsc{} and person bad:\textsc{red} \\

\glt
‘This man is dark and ugly.’ \textstyleExampleref{[R372.133]} 
\z

With a generic noun:

\ea\label{ex:9.49}
\gll Māuiui nei \textbf{me{\ꞌ}e} rakerake, me{\ꞌ}e pe{\ꞌ}e. \\
sick \textsc{prox} thing bad:\textsc{red} thing infect \\

\glt 
‘This disease was serious, it was contagious.’ \textstyleExampleref{[R231.318]} 
\z

\ea\label{ex:9.50}
\gll \textbf{Me{\ꞌ}e} {\ꞌ}iti{\ꞌ}iti koe {\ꞌ}i roto i te vaikava. \textbf{Me{\ꞌ}e} nuinui koe mō{\ꞌ}oku... \\
thing small:\textsc{red} \textsc{2sg} at inside at \textsc{art} ocean thing big:\textsc{red} \textsc{2sg} \textsc{ben.1sg.o} \\

\glt
‘You are a little thing in the ocean. You are big to me...’ \textstyleExampleref{[R474.007]} 
\z

These examples show that, as in other nominal clauses\is{Clause!nominal}, either the subject may come first as in \REF{ex:9.47} and \REF{ex:9.49}, or the predicate as in \REF{ex:9.48} and \REF{ex:9.50}.

In the examples above, the property is an adjective. It may also be another type of noun modifier: a verbal clause as in (\ref{ex:9.51}–\ref{ex:9.52}), or a modifying noun as in \REF{ex:9.53}.

\ea\label{ex:9.51}
\gll Me{\ꞌ}e ta{\ꞌ}e kai kōkoma moa māua. \\
thing \textsc{conneg} eat intestines chicken \textsc{1du.excl} \\

\glt 
‘We (are people who) don’t eat chicken intestines.’ \textstyleExampleref{[Ley-8-53.008]}
\z

\ea\label{ex:9.52}
\gll Toko{\ꞌ}a, a Manutara, me{\ꞌ}e vara unu i te {\ꞌ}ava. \\
also \textsc{prop} Manutara thing usually drink \textsc{acc} \textsc{art} liquor \\

\glt 
‘Also, Manutara was (someone who was) given to drinking liquor.’ \textstyleExampleref{[R309.055]} 
\z

\ea\label{ex:9.53}
\gll {\ꞌ}E henua nei, henua ma{\ꞌ}uŋa rahi. \\
and land \textsc{prox} land mountain many \\

\glt
‘And this land is a land of many mountains.’ \textstyleExampleref{[R348.004]} 
\z

As (\ref{ex:9.51}–\ref{ex:9.52}) show, the modifying verb may be preceded by preverbal particles, including the negator \textit{ta{\ꞌ}e}\is{tae (negator)@ta{\ꞌ}e (negator)}.

As in other clause types, the subject of attributive clauses\is{Clause!attributive} may be omitted:

\ea\label{ex:9.54}
\gll {\ꞌ}I nei te {\ꞌ}ariki ana noho, \textbf{kona} rivariva. \\
at \textsc{prox} \textsc{art} king \textsc{irr} stay place good:\textsc{red} \\

\glt 
‘Here the king would live, it was a good place.’ \textstyleExampleref{[Mtx-2-01.031]}
\z

\ea\label{ex:9.55}
\gll \textbf{Kai} ta{\ꞌ}e piropiro, \textbf{kai} rivariva. \\
food \textsc{conneg} rotten:\textsc{red} food good:\textsc{red} \\

\glt 
‘It is not rotten food, it is good food.’ \textstyleExampleref{[R310.382]} 
\z

Finally, Rapa Nui has a somewhat peculiar construction consisting of a bare noun phrase headed by \is{mee ‘thing’@me{\ꞌ}e ‘thing’}\textit{me{\ꞌ}e} or another generic noun, followed by a \textit{he}{}-marked NP. This construction is not very common, but entirely grammatical. It is especially used to express general truths.

\ea\label{ex:9.56}
\gll Me{\ꞌ}e mate he taŋata. \\
thing die \textsc{pred} person \\

\glt 
‘Man is mortal.’ \textstyleExampleref{[R210.073]} 
\z

\ea\label{ex:9.57}
\gll Me{\ꞌ}e rakerake he taŋi ŋā matu{\ꞌ}a. \\
thing bad:\textsc{red} \textsc{pred} cry \textsc{pl} parent \\

\glt 
‘It’s a bad thing, crying for one’s parents.’ \textstyleExampleref{[Ley-9-55.073]}
\z

\ea\label{ex:9.58}
\gll Kona hī kahi pa{\ꞌ}i he hakanonoŋa.\\
place to\_fish tuna in\_fact \textsc{pred} fishing\_zone\\

\glt
‘The \textit{hakanonoŋa} (= certain zones of the sea) are places to fish for tuna.’ \textstyleExampleref{[R200.030]} 
\z

This construction is unusual in that both noun phrases seem to be marked as a nominal predicate. However, a more plausible analysis is also possible: the construction may be a subjectless attributive clause, in which the predicate \textit{\mbox{me{\ꞌ}e} X} is followed by an apposition\is{Apposition} introduced by \textit{he}\is{he (nominal predicate marker)}. \REF{ex:9.56} could be paraphrased as ‘It’s (a) mortal (thing), man is.’ This appositional analysis is suggested by the use of \textit{he} (see \sectref{sec:5.12.1} for the use of \textit{he} in appositions\is{Apposition}), and by the fact that the \textit{he}{}-marked NP always occurs after the \is{mee ‘thing’@me{\ꞌ}e ‘thing’}\textit{me{\ꞌ}e} phrase. 
\is{Clause!attributive|)}

\section{Existential clauses}\label{sec:9.3}
\is{Clause!existential|(}
Existential clauses state the existence of a person or thing. In Rapa Nui, they are either constructed as a verbless clause or with the existential verb \textit{ai}\is{ai ‘to exist’}.\footnote{\label{fn:472}In this respect, Rapa Nui shows characteristics of both \is{Eastern Polynesian}EP languages (where existential clauses are verbless, with a \textit{he}{}-marked Existee as in Rapa Nui), and non-EP languages (where existential clauses\is{Clause!existential} are constructed with the verb \textit{ai}/\textit{iai} (\citealt[101]{Clark1976,Clark1997}.} 

\subsection{Verbless and verbal existential clauses}\label{sec:9.3.1}
\is{Clause!existential}
Verbless existential clauses\is{Clause!existential} contain only one core consituent, which is introduced by \textit{he}; the use of \textit{he}\is{he (nominal predicate marker)} shows that this constituent is predicate rather than subject.\footnote{\label{fn:473}According to \citet[241]{Dryer2007Clause}, it is in many languages unclear whether the theme of an existential clause\is{Clause!existential} should be considered a subject. In many languages, it is clear that the theme is not subject, e.g. in European languages like \ili{Dutch} (‘Er is een hond in de tuin’ = there is a dog in the garden, rather than ‘Een hond is in de tuin’) and \ili{French} (‘Il y a un chien dans le jardin’ = there is a dog in the garden).} This means that existential clauses\is{Clause!existential} conform to the general rule that the predicate is the only obligatory constituent.

\ea\label{ex:9.59}
\gll He taŋata ko Eŋo. \\
\textsc{pred} man \textsc{prom} Engo \\

\glt 
‘There was a man (called) Engo.’ \textstyleExampleref{[Mtx-7-28.001]}
\z

\ea\label{ex:9.60}
\gll \textbf{He} \textbf{repa} e rua te {\ꞌ}īŋoa ko Makita ko Roke{\ꞌ}aua. \\
\textsc{ntr} young\_man \textsc{num} two \textsc{art} name \textsc{prom} Makita \textsc{prom} Roke’aua \\

\glt
‘There were two young men, named Makita and Roke’aua.’ \textstyleExampleref{[R243.001]} 
\z

The noun phrase may contain a prenominal numeral; as discussed in \sectref{sec:5.3.5}, prenominal numerals are in determiner position, hence they replace the predicate marker \textit{he}:

\ea\label{ex:9.61}
\gll \textbf{E} \textbf{tahi} \textbf{poki} te {\ꞌ}īŋoa ko Eva ka ho{\ꞌ}e {\ꞌ}ahuru matahiti. \\
\textsc{num} one child \textsc{art} name \textsc{prom} Eva \textsc{cntg} one ten year \\

\glt 
‘There was a child called Eva, ten years old.’ \textstyleExampleref{[R210.001]} 
\z

Existential clauses can also be expressed with the verb \textit{ai}\is{ai ‘to exist’} ‘to exist’, with the Theme or \textsc{Existee} as subject of the clause. This construction is rare in older texts, but in modern Rapa Nui it is more common than the verbless construction.

Usually \textit{ai} has continuous aspect marking \textit{e V {\ꞌ}ā}\is{e (imperfective)!e V {\ꞌ}ā}\textit{/{\ꞌ}ana} (\sectref{sec:7.2.5.4}), while the verb phrase also has the emphatic particle \textit{rō}\is{ro (emphatic marker)@rō (emphatic marker)}. \textit{E~ai rō {\ꞌ}ā/{\ꞌ}ana} is such a common combination that it almost seems to be a frozen expression. 

\ea\label{ex:9.62}
\gll E ai rō {\ꞌ}ā e tahi poki nei te {\ꞌ}īŋoa ko Mariki. \\
\textsc{ipfv} exist \textsc{emph} \textsc{cont} \textsc{num} one child \textsc{prox} \textsc{art} name \textsc{prom} Mariki \\

\glt 
‘There was a child called Mariki.’ \textstyleExampleref{[R380.001]} 
\z

\ea\label{ex:9.63}
\gll ¿E ai rō {\ꞌ}ā te ika o roto? \\
~\textsc{ipfv} exist \textsc{emph} \textsc{cont} \textsc{art} fish of inside \\

\glt
‘Are there fish inside (the net)?’ \textstyleExampleref{[R241.058]} 
\z

However, \textit{ai}\is{ai ‘to exist’} is used with other aspectuals as well, for example neutral \textit{he} \REF{ex:9.64} and exhortative\is{Exhortative} \textit{e} \REF{ex:9.65}:

\ea\label{ex:9.64}
\gll {\ꞌ}I tō{\ꞌ}ona mahana \textbf{he} \textbf{ai} mai te aŋa he {\ꞌ}āua titi, {\ꞌ}o he rau kato...\\
at \textsc{poss.3sg.o} day \textsc{ntr} exist hither \textsc{art} work \textsc{pred} fence build or \textsc{pred} leaf pick\\

\glt 
‘On some days there was work: building fences or picking leaves...’ \textstyleExampleref{[R380.084]} 
\z

\ea\label{ex:9.65}
\gll Mo oho e tahi taŋata ki tai, \textbf{e} \textbf{ai} te me{\ꞌ}e ta{\ꞌ}ato{\ꞌ}a o te hī. \\
if go \textsc{num} one person to sea \textsc{exh} exist \textsc{art} thing all of \textsc{art} to\_fish \\

\glt 
‘If someone goes to the sea, he needs all the fishing gear (lit. there should be all the things of fishing).’ \textstyleExampleref{[R354.002]} 
\z

\newpage 
\subsection{Existential-locative clauses}\label{sec:9.3.2}
\is{Clause!existential-locative|(}
Many existential clauses\is{Clause!existential} do not just state the existence of something, but rather its existence in a certain place: ‘There is water here’. These clauses can be labelled ‘existential-locative’.\footnote{\label{fn:474}These are different from locative clauses, which predicate the location of a certain referent: ‘The water is here.’ Rapa Nui, like many other languages, employs different constructions for these two clause types. See \citet[241]{Dryer2007Clause} for general discussion.} 

Just like plain existential clauses\is{Clause!existential}, existential-locative clauses may be either verbless as in (\ref{ex:9.66}–\ref{ex:9.67}) or verbal as in (\ref{ex:9.68}–\ref{ex:9.69}). In older texts, they are always verbless.

\ea\label{ex:9.66}
\gll He taote e tahi \textbf{{\ꞌ}i} \textbf{muri} \textbf{i} \textbf{a} \textbf{ia}. \\
\textsc{pred} doctor \textsc{num} one at near at \textsc{prop} \textsc{3sg} \\

\glt 
‘There was a doctor with her.’ \textstyleExampleref{[R210.090]} 
\z

\ea\label{ex:9.67}
\gll He taŋata \textbf{to} \textbf{nei}... Ŋata Vake te {\ꞌ}īŋoa.\\
\textsc{pred} person \textsc{art}:of \textsc{prox} Ngata Vake \textsc{art} name\\

\glt 
‘There was a man here, called Ngata Vake.’ \textstyleExampleref{[Ley-3-02.002]}
\z

\ea\label{ex:9.68}
\gll ¿E ai rō {\ꞌ}ā te ika \textbf{o} \textbf{roto}? \\
~\textsc{ipfv} exist \textsc{emph} \textsc{cont} \textsc{art} fish of inside \\

\glt 
‘Are there fish inside (the net)?’ \textstyleExampleref{[R241.058]} 
\z

\ea\label{ex:9.69}
\gll ¡{\ꞌ}Āhani {\ꞌ}ō e ai rō {\ꞌ}ā te hare hāpī mā{\ꞌ}ohi \textbf{o} \textbf{nei}! \\
~if\_only really \textsc{ipfv} exist \textsc{emph} \textsc{cont} \textsc{art} house school indigenous of \textsc{prox} \\

\glt
‘If only there were an indigenous school here!’ \textstyleExampleref{[R242.061]} 
\z

As the examples above show, the locative adjunct in these constructions is often introduced by \textit{to} (in older Rapa Nui) or \textit{o} (in modern Rapa Nui)\is{to (possessive prep.)}.\footnote{\label{fn:475}\textit{To} is a contraction of the article \textit{te} + the genitive preposition \textit{o} (\sectref{sec:6.2}).} The possessive preposition\is{o (possessive prep.)} \textit{o}, when used in a locative construction, often indicates that a referent belongs to a certain place, i.e. comes from that place or is located there permanently. It may, however, also indicate the location of a referent at a given moment, and therefore be similar in sense to \textit{{\ꞌ}i} (see (\ref{ex:6.44}–\ref{ex:6.46} in \sectref{sec:6.3.1}).
\is{Clause!existential-locative|)}

\subsection{Possessive clauses}\label{sec:9.3.3}
\is{Clause!possessive|(}
Possessive clauses establish a relationship of possession between two entities:\footnote{\label{fn:476}Possessive clauses (‘John has a book’) are different from proprietary clauses\is{Clause!proprietary} (‘The book is John’s’, \sectref{sec:9.4.2}). See \citet{Clark1969}.}  ‘John has a book’ expresses that John is the possessor\is{Possession} of a book. In Rapa Nui, this relation is expressed by an existential clause\is{Clause!existential},\footnote{\label{fn:477}This is common in many languages, see \citet[244]{Dryer2007Clause}.} in which the possessee noun phrase is modified by a possessor\is{Possession}; the construction can be paraphrased as ‘John’s book exists’ or ‘There is John’s book.’

In modern Rapa Nui, possessive clauses\is{Clause!possessive} are constructed as verbal existential clauses\is{Clause!existential}, in which the existential verb \textit{ai}\is{ai ‘to exist’} takes the possessee as subject. \REF{ex:9.70} is literally ‘His house in Hanga Roa existed’, \REF{ex:9.71} is ‘Two their children existed’.

\ea\label{ex:9.70}
\gll E ai rō {\ꞌ}ā tō{\ꞌ}ona hare {\ꞌ}i Haŋa Roa. \\
\textsc{ipfv} exist \textsc{emph} \textsc{cont} \textsc{poss.3sg.o} house at Hanga Roa \\

\glt 
‘He had a house in Hanga Roa.’ \textstyleExampleref{[R250.249]} 
\z

\ea\label{ex:9.71}
\gll He ai e rua rāua ŋā poki. \\
\textsc{ntr} exist \textsc{num} two \textsc{3pl} \textsc{pl} child \\

\glt 
‘They had two children.’ \textstyleExampleref{[R211.002]} 
\z

\ea\label{ex:9.72}
\gll E ai rō {\ꞌ}ā te kona {\ꞌ}oka mahute {\ꞌ}a Kekepuē ko tetu.\\
\textsc{ipfv} exist \textsc{emph} \textsc{cont} \textsc{art} place plant mulberry of\textsc{.a} Kekepue \textsc{prom} huge\\

\glt
‘Kekepue had a huge plantation of mulberries.’ \textstyleExampleref{[Fel-1978.008]}
\z

As these examples show, the possessor\is{Possession} is expressed in the subject noun phrase: it is either a possessive pronoun as in (\ref{ex:9.70}–\ref{ex:9.71}), or a possessive noun phrase as in \REF{ex:9.72}. (For more details, see \sectref{sec:6.2.1} on possessives in the noun phrase, \sectref{sec:6.3.1} on the semantic range of possessive constructions, and \sectref{sec:6.3.2} on the choice between \textit{o} and \textit{{\ꞌ}a}.)

The clause may be preceded by a noun phrase coreferential to the possessor\is{Possession}; this happens especially when the possessor\is{Possession} is a full noun phrase. This noun phrase is left-dislocated\is{Dislocation!left} and is syntactically not a constituent of the clause that follows; the clause as a whole is a topic-comment construction (\sectref{sec:8.6.1.3}). \REF{ex:9.73} can be translated literally as ‘All the tribes, their leaders existed.’

\ea\label{ex:9.73}
\gll {\ob}Ta{\ꞌ}ato{\ꞌ}a mata\,{\cb}\textsubscript{\textup{i}} e ai rō {\ꞌ}ana te rāua\textsubscript{\textup{i}} taŋata pū{\ꞌ}oko. \\
{\db}all tribe \textsc{ipfv} exist \textsc{emph} \textsc{cont} \textsc{art} \textsc{3pl} person head \\

\glt 
‘All the tribes had their leaders.’ \textstyleExampleref{[R371.006]} 
\z

\ea\label{ex:9.74}
\gll {\ob}E tahi vaka {\ꞌ}āpī\,{\cb}\textsubscript{\textup{i}} e ai tako{\ꞌ}a tō{\ꞌ}ona\textsubscript{\textup{i}} taura. \\
{\db}\textsc{num} one boat new \textsc{exh} exist also \textsc{poss.3sg.o} rope \\

\glt
‘A new boat also needs its ropes.’ \textstyleExampleref{[R200.083]} 
\z

In these topic-comment\is{Topic-comment construction} constructions, the possessor\is{Possession} is often not expressed again in the subject NP. \REF{ex:9.75} is literally: ‘We, money exists’; \REF{ex:9.76} is ‘This woman, there were two daughters.’

\ea\label{ex:9.75}
\gll {\ob}A mātou\,{\cb} e ai nei te moni. \\
{\db}\textsc{prop} \textsc{1pl.excl} \textsc{ipfv} exist \textsc{prox} \textsc{art} money \\

\glt 
‘We have money.’ \textstyleExampleref{[R621.027]} 
\z

\ea\label{ex:9.76}
\gll {\ob}Vi{\ꞌ}e nei\,{\cb} e ai rō {\ꞌ}ā e rua poki vahine. \\
{\db}woman \textsc{prox} \textsc{ipfv} exist \textsc{emph} \textsc{cont} \textsc{num} two child female \\

\glt 
‘This woman had two daughters.’ \textstyleExampleref{[R491.008]} 
\z

In older texts, possessive clauses\is{Clause!possessive} may also be constructed as a verbless existential clause\is{Clause!existential}. Instead of the verb \textit{ai} with its subject, these have a \textit{he}\is{he (nominal predicate marker)}{}-marked nominal predicate. The possessor\is{Possession} is expressed as \textit{to}\is{to (possessive prep.)} + NP or a \textit{t-}possessive\is{Pronoun!possessive!t-class} pronoun. 

\ea\label{ex:9.77}
\gll He {\ꞌ}oka nō \textbf{to} \textbf{te} \textbf{hare}. \\
\textsc{pred} pole just \textsc{art}:of \textsc{art} house \\

\glt 
‘The house had only rafters (no supporting poles).’ \textstyleExampleref{[Ley-2-12.007]}
\z

\ea\label{ex:9.78}
\gll He poki \textbf{tā{\ꞌ}ana} e tahi, poki tamāroa. \\
\textsc{pred} child \textsc{poss.3sg.a} \textsc{num} one child male \\

\glt
‘He had a child, a boy.’ \textstyleExampleref{[Ley-9-56.002]}
\z

In modern Rapa Nui, verbless possessive clauses\is{Clause!possessive} only occur in the following circumstances:

When the predicate noun phrase contains a numeral:

\ea\label{ex:9.79}
\gll \textbf{E} \textbf{tahi} ō{\ꞌ}oku hoa repa ko Hoahine te {\ꞌ}īŋoa.\\
\textsc{num} one \textsc{poss.1sg.o} friend friend \textsc{prom} Hoahine \textsc{art} name\\

\glt
‘I have a friend whose name is Hoahine.’ \textstyleExampleref{[R213.014]} 
\z

When the clause is negated, using \textit{{\ꞌ}ina}\is{ina (negator)@{\ꞌ}ina (negator)} (\sectref{sec:10.5.1}):

\ea\label{ex:9.80}
\gll {\ꞌ}\textbf{Ina} pa{\ꞌ}i o māua kona mo noho. \\
\textsc{neg} in\_fact of \textsc{1du.excl} place for stay \\

\glt
‘For we do not have a place to live.’ \textstyleExampleref{[R229.210]} 
\z

As these examples show, in these cases the possessor\is{Possession} is a Ø-possessive\is{Pronoun!possessive!Ø-class} pronoun within the predicate noun phrase. These clauses are different from the old constructions illustrated in (\ref{ex:9.77}–\ref{ex:9.78}), where the possessor\is{Possession} is a separate constituent.\footnote{\label{fn:478}If the possessives in (\ref{ex:9.77}–\ref{ex:9.78}) were part of the predicate noun phrase, the possessor\is{Possession} would be marked with the preposition \textit{o} in \REF{ex:9.77}, and a Ø-possessive\is{Pronoun!possessive!Ø-class} pronoun\is{Pronoun!possessive} in \REF{ex:9.78}.} 
\is{Clause!possessive|)}
\subsection{Conclusion}\label{sec:9.3.4}

Whether an existential clause is verbless or verbal, depends on the type of clause: simple existential, existential-locative, or possessive. However, there is a general development over time in which verbless constructions are replaced by verbal ones. This is summarised in \tabref{tab:63}:

\newpage
\begin{table}
\fittable{
\begin{tabular}{lcccc}
\lsptoprule
  &
  \multicolumn{2}{c}{old texts} & 
  \multicolumn{2}{c}{new texts}
\\
  & 
  verbless  (\textit{he N})&
  verbal  (\textit{ai})& 
  verbless (\textit{he N})& 
  verbal (\textit{ai})\\
\midrule
existential & x& (x)& x& x\\
existential-locative & x& –& x& x\\
possessive & x& –& (x)& x\\
\lspbottomrule
\end{tabular}
}
\caption{Types of existential clauses}
\label{tab:63}
\end{table}

\is{Clause!existential|)}

\section{Prepositional predicates}\label{sec:9.4}

Various types of prepositional phrases may serve as predicate of a nonverbal clause.

\subsection{Locative clauses}\label{sec:9.4.1}
\is{Clause!locative|(}
Locative clauses consist of a subject noun phrase and a prepositional phrase with locative sense as predicate. Either phrase may come first. The locative phrase is often introduced by \textit{{\ꞌ}i}\is{i ‘in, at’@{\ꞌ}i ‘in, at’}, marking stationary location, possibly followed by a locational\is{Locational} as in \REF{ex:9.81}. Other prepositions may also be used, as \REF{ex:9.83} shows.

\ea\label{ex:9.81}
\gll A nua \textbf{{\ꞌ}i} \textbf{roto} i te hare. \\
\textsc{prop} Mum at inside at \textsc{art} house \\

\glt 
‘Mum is in the house.’ \textstyleExampleref{[R333.284]} 
\z

\ea\label{ex:9.82}
\gll \textbf{{\ꞌ}I} {\ꞌ}Anakena te hare noho o Matakaroa... \\
at Anakena \textsc{art} house stay of Matakaroa \\

\glt 
‘In Anakena was the house where Matakaroa lived...’ \textstyleExampleref{[Mtx-3-09.003]}
\z

\ea\label{ex:9.83}
\gll —¿\textbf{Mai} hē rā koe? —\textbf{Mai} tai nei.\\
~~~~from \textsc{cq} \textsc{intens} \textsc{2sg} ~~~from sea \textsc{prox}\\

\glt 
‘—Where are you (coming) from? —From the seaside.’ \textstyleExampleref{[R245.084]} 
\z
\is{Clause!locative|)}

\subsection{Proprietary clauses}\label{sec:9.4.2}
\is{Clause!proprietary|(}
Proprietary clauses (also known as “genitive predicates”, \citealt[248]{Dryer2007Clause}) consist of a subject noun phrase and a predicate expressing a possessor\is{Possession}. In Rapa Nui, the latter is either a noun phrase marked with genitive \textit{o} or \textit{{\ꞌ}a}, or a Ø-possessive\is{Pronoun!possessive!Ø-class} pronoun. (\sectref{sec:6.3.1} on the semantic range of possessive constructions, \sectref{sec:6.3.2} on the choice between \textit{o} and \textit{{\ꞌ}a}.) 

\ea\label{ex:9.84}
\gll Te hare nei, ta{\ꞌ}e ō{\ꞌ}oku; o tā{\ꞌ}aku mā{\ꞌ}aŋa ena ko Puakiva. \\
\textsc{art} house \textsc{prox} \textsc{conneg} \textsc{poss.1sg.o} of \textsc{poss.1sg.a} adopted\_child \textsc{med} \textsc{prom} Puakiva \\

\glt 
‘This house is not mine; it belongs to my adopted child Puakiva.’ \textstyleExampleref{[R229.268]} 
\z

\ea\label{ex:9.85}
\gll A {\ꞌ}Ārahu o te mata era o te Tūpāhotu. \\
\textsc{prop} Arahu of \textsc{art} tribe \textsc{dist} of \textsc{art} Tupahotu \\

\glt 
‘Arahu was of the Tupahotu tribe.’ \textstyleExampleref{[R432.002]} 
\z

\ea\label{ex:9.86}
\gll {\ꞌ}Ā{\ꞌ}ana ho{\ꞌ}i te uka era, {\ꞌ}a Métraux. \\
\textsc{poss.3sg.a} indeed \textsc{art} girl \textsc{dist} of\textsc{.a} Métraux \\

\glt 
‘That girl belongs to him, Métraux.’ \textstyleExampleref{[R416.813]} 
\z

\ea\label{ex:9.87}
\gll Ō{\ꞌ}oku mau {\ꞌ}ana te hape. \\
\textsc{poss.1sg.o} really \textsc{ident} \textsc{art} fault \\

\glt
‘The fault is really mine.’ \textstyleExampleref{[R236.095]} 
\z

As these examples show, the predicate may come after the subject as in (\ref{ex:9.84}–\ref{ex:9.85}), or before the subject as in (\ref{ex:9.86}–\ref{ex:9.87}).

Occasionally, proprietary clauses\is{Clause!proprietary} are constructed with the locative preposition \textit{i}\is{i (preposition)}, which may have a possessive sense (\sectref{sec:4.7.2.3}). \textit{I} in proprietary clauses\is{Clause!proprietary} tends to indicate possession in an abstract sense, e.g. possession of qualities or attributes; however, as \REF{ex:9.89} shows, it is also used with concrete entities.

\ea\label{ex:9.88}
\gll I a tātou mau {\ꞌ}ā te pūai mo haka ma{\ꞌ}itaki i te kāiŋa. \\
at \textsc{prop} \textsc{1pl.incl} really \textsc{ident} \textsc{art} power for \textsc{caus} clean \textsc{acc} \textsc{art} homeland \\

\glt 
‘Ours is the power to clean the island.’ \textstyleExampleref{[R535.240]} 
\z

\ea\label{ex:9.89}
\gll I a mātou te kai ko piropiro {\ꞌ}ā. \\
at \textsc{prop} \textsc{1pl.excl} \textsc{art} food \textsc{prf} rotten:\textsc{red} \textsc{cont} \\

\glt 
‘Ours is the rotten food.’ \textstyleExampleref{[R310.263]} 
\z

The proprietary clause construction also serves to form nominalised actor-emphatic\is{Actor-emphatic construction} clauses (\sectref{sec:8.6.3})\is{Clause!proprietary}.
\is{Clause!proprietary|)}

\subsection{Other prepositional predicates}\label{sec:9.4.3}

Any prepositional phrase may serve as the predicate of a nominal clause\is{Clause!nominal}. This results in clauses that could be labelled “benefactive” \REF{ex:9.90}, “instrumental” \REF{ex:9.91} or “comparative\is{Comparative}” \REF{ex:9.92}; however, these labels should not obscure the fact that these clauses simply follow the general pattern of a NP PP clause. 

\ea\label{ex:9.90}
\gll Te haŋa o te hānau {\ꞌ}e{\ꞌ}epe \textbf{mō{\ꞌ}ona} te kāiŋa nei. \\
\textsc{art} want of \textsc{art} race corpulent \textsc{ben.3sg.o} \textsc{art} homeland \textsc{prox} \\

\glt 
‘What the ‘corpulent race’ wanted was, that the island should be for them.’ \textstyleExampleref{[Ley-3-06.011]}
\z

\ea\label{ex:9.91}
\gll Tō{\ꞌ}ona orara{\ꞌ}a \textbf{hai} pura pere Tomatō. \\
\textsc{poss.3sg.o} living \textsc{ins} only play \textit{toma\_todo} \\

\glt 
‘His living was (=he earned his living) merely by playing \textit{toma todo} (a card game).’ \textstyleExampleref{[R250.145]} 
\z

\ea\label{ex:9.92}
\gll A kōrua ta{\ꞌ}e mau {\ꞌ}ana \textbf{pē} Kava. \\
\textsc{prop} \textsc{2pl} \textsc{conneg} really \textsc{ident} like Kava \\

\glt
‘You are not really like Kava.’ \textstyleExampleref{[R229.488]} 
\z

As with all types of nominal clauses\is{Clause!nominal}, the constituent order is not fixed, though the subject tends to come first, as (\ref{ex:9.90}–\ref{ex:9.92}) show.

\section{Numerical clauses}\label{sec:9.5}
\is{Clause!numerical|(}\is{Numeral}
In numerical clauses,\footnote{\label{fn:479}See \citet[108]{Clark1969} on this term.} the predicate is a numeral phrase, consisting of a numeral with preceding particle (\sectref{sec:4.3.2}). The numeral predicate comes first; it is followed by the subject noun phrase.

\ea\label{ex:9.93}
\gll {\ob}E tahi\,{\cb} {\ob}te rāua poki vahine nehenehe\,{\cb}. \\
{\db}\textsc{num} one {\db}\textsc{art} \textsc{3pl} child female beautiful \\

\glt
‘They had one beautiful daughter (lit. one [was] their beautiful daughter).’ \textstyleExampleref{[R338.001]} 
\z

In this example, the numeral phrase \textit{e tahi} is predicated of the subject \textit{te rāua poki vahine nehenehe}. \textit{E tahi} is not part of the noun phrase that follows, as is indicated by the determiner introducing that noun phrase; numerals within a noun phrase are never followed by a determiner (\sectref{sec:5.4.1}). 

In the following example, the numeral is followed by a \textit{t}{}-possessive\is{Pronoun!possessive!t-class} pronoun, which occupies the determiner position in the noun phrase (\sectref{sec:6.2.1}); again, this indica\is{Numeral}tes that the numeral is not part of the subject NP, but a separate constituent.

\ea\label{ex:9.94}
\gll He tu{\ꞌ}u mai... e tahi paiheŋa, \textbf{e} \textbf{rua} tō{\ꞌ}ona pū{\ꞌ}oko. \\
\textsc{ntr} arrive hither \textsc{num} one dog \textsc{num} two \textsc{poss.3sg.o} head \\

\glt 
‘One day a dog came, which had two heads (lit. two its heads).’ \textstyleExampleref{[R435.003]} 
\z

The following sentence, which is superficially almost identical to \REF{ex:9.93}, has a fundamentally different structure.

\ea\label{ex:9.95}
\gll E tahi rāua poki vahine nehenehe. \\
\textsc{num} one \textsc{3pl} child female beautiful \\

\glt
‘They had one beautiful daughter (lit. one their beautiful daughter).’
\z

This is an existential clause\is{Clause!existential}, which consists of a single NP containing the numeral \textit{e tahi}; the absence of a determiner after \textit{tahi} indicates that the numeral is part of the noun phrase. This is confirmed by the fact that the noun phrase as a whole can be used as constituent of a larger clause, for example as subject of an existential verb:

\ea\label{ex:9.96}
\gll E ai rō {\ꞌ}ana {\ob}e tahi rāua poki vahine\,{\cb}. \\
\textsc{ipfv} exist \textsc{emph} \textsc{cont} {\db}\textsc{num} one \textsc{3pl} child female \\

\glt 
‘They had one daughter (lit. there was one their daughter).’ \textstyleExampleref{[R338.001 revised]} 
\z

Numerical clauses are not very common. It is more common for a numeral to be embedded within a noun phrase, as in \REF{ex:9.95} above. This is also illustrated in (\ref{ex:9.60}–\ref{ex:9.62}) in \sectref{sec:9.3.1}. 
\is{Clause!numerical|)}\is{Numeral}

\section{Copula verbs}\label{sec:9.6}
\is{Verb!copula}\is{Verb!copula|(}
Copula verbs\is{Verb!copula} serve to link a nominal subject to a nominal or otherwise non-verbal predicate. While copula verbs\is{Verb!copula} may have all the morphosyntactic trappings of a verb, they are semantically empty \citep[115]{Payne1997} or nearly empty. 
\is{ai ‘to exist’!as copula verb|(}
Copula verbs\is{Verb!copula} are unusual in Polynesian languages; the only example I am aware of concerns the contact-induced development of verbs ‘have’ and ‘be’ in Mele-Fila and Emae in Vanuatu (\citealt[337]{Clark1986}; \citealt[119]{Clark1994}), though there is a possible example in \ili{Hawaiian} (see Footnote \ref{fn:482} on p.~\pageref{fn:482}).\footnote{\label{fn:480}\citet[154]{Harlow2007Maori} mentions \textit{ai} as a copula verb in older \ili{Māori}; however, as this verb only takes a single argument, it seems to be an existential verb like Rapa Nui \textit{ai} in existential clauses\is{Clause!existential}, rather than a copula. (The example \textit{Kia ai he moenga...} is translated ‘Let there be a bed...’) As \citet[160]{Dixon2010-2} points out, “a defining feature for a copula verb is that it \textit{must} be able to occur in a construction with two core arguments.”}  In Rapa Nui, the existential verb \is{ai ‘to exist’!as copula verb}\textit{ai} is used as a copula verb in some constructions. This use is absent in older texts; possibly it is developing under influence of \ili{Spanish}, where copular clauses have \textit{ser} or \textit{estar} ‘to be’. Another recent introduction is \textit{riro} ‘become’, which equally functions as a copula verb. In the following sections, these verbs will be discussed in turn.

\subsection{\textit{Ai} ‘to exist’ as a copula verb}\label{sec:9.6.1}

\textit{Ai} usually functions as an existential verb ‘to be, exist’ (\sectref{sec:9.3.1}). Existential constructions with \textit{ai} can be analysed as intransitive\is{Verb!intransitive} verbal clause with the Existee as subject. However, \textit{ai} is also used in a construction involving both a subject and a nonverbal predicate. This construction is uncommon, but it does occur. Examples in the text corpus are scarce; more examples are found in the Bible translation, probably due to the higher frequency of subordinate clause constructions in Biblical texts.

At first sight, the following two examples involve a copula verb construction. The verb \textit{ai} (preceded by the subordinators \textit{mo} ‘if’ and \textit{ana} ‘irrealis’, respectively) is followed by two noun phrases: a subject and a \textit{he}\is{he (nominal predicate marker)}{}-marked noun phrase. In both cases, \textit{ai} appears to be a copula verb in a classifying clause\is{Clause!classifying}.

\ea\label{ex:9.97}
\gll Mo \textbf{ai} koe he Kiritō... \\
if exist \textsc{2sg} \textsc{pred} Christ \\

\glt 
‘If you are the Christ...’ \textstyleExampleref{[Mat. 26:63]}
\z

\ea\label{ex:9.98}
\gll {\ꞌ}Ina te {\ꞌ}Atua he tapa atu ana \textbf{ai} koe he hūrio {\ꞌ}o ta{\ꞌ}e he hūrio. \\
\textsc{neg} \textsc{art} God \textsc{pred} consider away \textsc{irr} exist \textsc{2sg} \textsc{pred} Jew or \textsc{conneg} \textsc{pred} Jew \\

\glt
‘God does not consider whether you are a Jew or not a Jew.’ \textstyleExampleref{[Colossians, introduction]}
\z

However, on a closer look, \textit{ai} may not be a copula verb here. As it turns out, \textit{ai} in subordinate clauses can be followed by a complete verbal clause; the latter is no different in structure from a main clause. Below are two examples, again introduced by \textit{mo} and \textit{ana}:

\ea\label{ex:9.99}
\gll Mo \textbf{ai} {\ob}kai oho {\ꞌ}ā koe ki te kona roaroa...\,{\cb} \\
if exist {\db}\textsc{neg.pfv} go \textsc{cont} \textsc{2sg} to \textsc{art} place far:\textsc{red} \\

\glt 
‘If you haven’t been to distant places (lit. if it is you haven’t gone)...’ \textstyleExampleref{[R615.519]} 
\z

\ea\label{ex:9.100}
\gll ¡E u{\ꞌ}i he ra{\ꞌ}e ana \textbf{ai} {\ob}e haŋa rō te taŋata\,{\cb}! \\
~\textsc{exh} look \textsc{ntr} first \textsc{irr} exist {\db}\textsc{ipfv} want \textsc{emph} \textsc{art} person \\

\glt
‘First you must see whether the people want it (lit. whether it is the people want).’ \textstyleExampleref{[R647.248]} 
\z

In (\ref{ex:9.99}–\ref{ex:9.100}) it is clear that \textit{ai} is not the predicate of the clause between brackets. Rather, \textit{ai} is an (existential) verb followed by a complete (independent) clause.\footnote{\label{fn:481}See further \sectref{sec:11.5.1.1} (\textit{mo}) \sectref{sec:11.5.2.2} (\textit{ana}) on the use of \textit{ai} with subordinating markers.} The same analysis is possible for \REF{ex:9.97} above; in that case \textit{koe he Kiritō} is a complete (nominal) clause, in which \textit{ai} does not play a role. The same is true for \REF{ex:9.98}. If this analysis is correct, \textit{ai} in (\ref{ex:9.97}–\ref{ex:9.98}) is not a copula verb. A compelling reason to adopt this analysis of \REF{ex:9.97} is, that the subject of a verb marked with \textit{mo} is normally expressed as a possessive (\sectref{sec:11.5.1.2}). The fact that the subject in \REF{ex:9.97} is nominative \textit{koe}, makes it an unlikely candidate for the subject position of the \textit{mo-}clause. 

In other cases, however, the analysis above is implausible. First, the subject after \textit{mo ai} may be expressed as a possessive, strongly suggesting that it is indeed the subject of the \textit{mo}{}-clause, hence an argument of \textit{ai}. This suggests that \textit{ai} in \REF{ex:9.101} is bivalent (hence copular), taking two arguments just like the transitive\is{Verb!transitive} verb \textit{{\ꞌ}ui} in \REF{ex:9.102}.

\ea\label{ex:9.101}
\gll Mo \textbf{ai} {\ob}ō{\ꞌ}ou\,{\cb} {\ob}he Kiritō\,{\cb}, ka kī mai. \\
if exist {\db}\textsc{poss.2sg.o} {\db}\textsc{pred} Christ \textsc{imp} say hither \\

\glt 
‘If you are the Christ, say so.’ \textstyleExampleref{[Luk. 22:67]}
\z

\ea\label{ex:9.102}
\gll he kona mo \textbf{{\ꞌ}ui} {\ob}ō{\ꞌ}ou\,{\cb} {\ob}i ta{\ꞌ}a me{\ꞌ}e ta{\ꞌ}e {\ꞌ}ite\,{\cb}\\
\textsc{pred} place for ask {\db}\textsc{poss.2sg.o} {\db}\textsc{acc} \textsc{poss.2sg.a} thing \textsc{conneg} know\\

\glt
‘a place for you to ask the things you don’t know’ \textstyleExampleref{[R239.049]} 
\z

Second, a copular analysis of \textit{ai} is plausible when it occurs in a main clause. Although \REF{ex:9.103} below could be interpreted as existential \textit{ai}, this is not very plausible, as there are no unambiguous examples of \textit{ai} in main clauses followed by an independent clause expressing the Existee. A monovalent analysis is even less likely when the two noun phrases occur on either side of the verb, as in \REF{ex:9.104}. 

\ea\label{ex:9.103}
\gll E ai {\ob}kōrua\,{\cb} {\ob}he nu{\ꞌ}u {\ꞌ}ina e tahi hape\,{\cb}. \\
\textsc{exh} exist {\db}\textsc{2pl} {\db}\textsc{pred} people \textsc{neg} \textsc{num} one fault \\

\glt 
‘You should be people without fault.’ \textstyleExampleref{[Mat. 5:48]}
\z

\ea\label{ex:9.104}
\gll {\ob}Tu{\ꞌ}u nu{\ꞌ}u ena\,{\cb} he ai {\ob}he nu{\ꞌ}u ō{\ꞌ}oku\,{\cb}. \\
{\db}\textsc{poss.2sg.o} people \textsc{med} \textsc{ntr} exist {\db}\textsc{pred} people \textsc{poss.1sg.o} \\

\glt
‘Your people will be my people.’ \textstyleExampleref{[Ruth 1:16]}
\z

We may conclude that \textit{ai} is occasionally used as a copula verb. Using \textit{ai} enables a speaker to embed nominal clauses\is{Clause!nominal} into constructions which only allow verbal clauses, for example subordinate clauses as in \REF{ex:9.101}, and exhortations as in \REF{ex:9.103}.

While all examples so far concern classifying clauses\is{Clause!classifying}, other types of verbless clauses may have the copula as well. Here is an example of a locative clause\is{Clause!locative}. Again, the subject is possessive, as the verb \textit{ai} is nominalised. 

\ea\label{ex:9.105}
\gll He koa tō{\ꞌ}ona matu{\ꞌ}a {\ꞌ}o te \textbf{ai} haka{\ꞌ}ou mai {\ob}ō{\ꞌ}ona\,{\cb} {\ob}{\ꞌ}i nei\,{\cb}.\\
\textsc{pred} happy \textsc{poss.3sg.o} parent because\_of \textsc{art} exist again hither {\db}\textsc{poss.3sg.o} {\db}at \textsc{prox}\\

\glt 
‘Her parents were happy because she was here again.’ \textstyleExampleref{[R441.018]}
\is{ai ‘to exist’!as copula verb|)}\z

\subsection{\textit{Riro} ‘to become’}\label{sec:9.6.2}
\is{riro ‘become’|(}
\textit{Riro} ‘to become’ expresses the transformation of an entity into something else. It was borrowed from \ili{Tahitian}\is{Tahitian influence} relatively recently: \textit{riro} is not found in older texts, the oldest occurrences are in the stories collected in the early 1970s by Felbermayer (\citealt{Felbermayer1971,Felbermayer1973,Felbermayer1978}). 

\textit{Riro} occurs in a few stories in which a person turns into an animal. In older versions of these stories, the process of transformation is implicit and the new identity is expressed by a non-verbal clause; in new versions, \textit{riro} is used. The following examples are from two versions of the same story, which tells about a child turning into a fish. In the old version in \REF{ex:9.106}, no verb is used to describe the transformation; the new version in \REF{ex:9.107} employs the verb \textit{riro}.

\ea\label{ex:9.106}
\gll He uru mai te e{\ꞌ}a, he to{\ꞌ}o i tau poki era. \textbf{He} \textbf{ika} tau poki era. \\
\textsc{ntr} enter hither \textsc{art} wave \textsc{ntr} take \textsc{acc} \textsc{dem} child \textsc{dist} \textsc{pred} fish \textsc{dem} child \textsc{dist} \\

\glt 
‘A wave came in and took the child. The child (became) a fish.’ \textstyleExampleref{[Mtx-7-10.019]}
\z

\ea\label{ex:9.107}
\gll He \textbf{riro} rō atu {\ꞌ}ai tū poki era he ika. \\
\textsc{ntr} become \textsc{emph} away \textsc{subs} \textsc{dem} child \textsc{dist} \textsc{pred} fish \\

\glt
‘The child became a fish.’ \textstyleExampleref{[R338.006]} 
\z

As \REF{ex:9.107} shows, the verb \textit{riro} has two arguments: the subject \textit{tū poki era} and a \textit{he}\is{he (nominal predicate marker)}{}-marked noun phrase expressing the class to which the subject belongs after the transformation. Apart from the verb, the clause has the same structure as the verbless classifying clause\is{Clause!classifying} in \REF{ex:9.106}. This shows that \textit{riro} is a true copula verb, linking two noun phrases with an identity relation. Two more examples of the same construction:

\ea\label{ex:9.108}
\gll He riro te rima o Kāiŋa he toto. \\
\textsc{ntr} become \textsc{art} hand of Kainga \textsc{pred} blood \\

\glt 
‘Kainga’s hand became (all) blood(y).’ \textstyleExampleref{[R243.074]} 
\z

\ea\label{ex:9.109}
\gll I pa{\ꞌ}ari era i pohe rō a ia mo riro he oromatu{\ꞌ}a. \\
\textsc{pfv} adult \textsc{dist} \textsc{pfv} desire \textsc{emph} \textsc{prop} \textsc{3sg} for become \textsc{pred} priest \\

\glt 
‘When he was grown up, he desired to become a priest.’ \textstyleExampleref{[R231.004]} 
\z

While the form and meaning of \textit{riro} were borrowed from \ili{Tahitian}\is{Tahitian influence}, its status as a copula verb is unique to Rapa Nui.\footnote{\label{fn:482}There is one possible exception: for \ili{Hawaiian}, \citet[63]{Cook1999} gives an example from an old text (1918) where \textit{he} (which is a nominal predicate marker, as in Rapa Nui) marks the resulting entity after the verb \textit{lilo}, an argument normally marked with \textit{i} (related to \ili{Tahitian} \textit{{\ꞌ}ei} in \REF{ex:9.110}?). Apparently, this construction, which corresponds exactly to the Rapa Nui construction \textit{riro he}, is unknown nowadays.} In \ili{Tahitian}, the resulting entity after \textit{riro} is marked with the preposition \textit{{\ꞌ}ei}:\footnote{\label{fn:483}\ili{Tahitian} \textit{{\ꞌ}ei} has various uses, all of which have to do with a state not yet realised; see \citet[364–365]{AcadémieTahitienne1986}.}

\ea\label{ex:9.110}
\gll {\ꞌ}Ua riro tō {\ꞌ}oe tuahine {\ꞌ}ei pōti{\ꞌ}i purotu. ~ \textup{(\ili{Tahitian})}\\
\textsc{prf} become \textsc{art}:of \textsc{2sg} sister to girl pretty \\

\glt 
‘Your sister has become a beautiful girl.’ \textstyleExampleref{(\citealt[272]{AcadémieTahitienne1986})} 
\z
\is{Verb!copula|)}\is{riro ‘become’|)}

\section{Conclusions}\label{sec:9.7}

This chapter has dealt with various types of clauses, all of which do not have a lexical verb as predicate. Many of these are verbless; others have either the existential verb \textit{ai} or – occasionally – a copula verb.

Regarding clauses with a noun phrase predicate, two types can be distinguished. Classifying clauses contain a true predicate providing information about the subject by including it in a certain class; identifying clauses express an identity relation between two referents. In classifying clauses the predicate has the predicate marker \textit{he}; in identifying clauses, it has the prominence marker \textit{ko}. The identifying construction is only used if the predicate is already known to the hearer as an individual entity.

Rapa Nui has a cleft construction, which consists of an identifying or classifying predicate followed by a subject noun phrase containing a relative clause. Unlike other Polynesian languages, Rapa Nui requires the relative clause to contain a head noun, resulting in the construction sometimes called “pseudo-cleft”.

Like clefts, attributive clauses (those with an adjectival predicate expressing an inherent property) need a head noun in the predicate; in other words, rather than ‘This tomato [is] yellow’, Rapa Nui has ‘This tomato [is] a yellow tomato’. This makes attributive clauses very similar in structure to classifying clauses, but while the predicate marker is obligatory in classifying clauses, in attributive clauses it is usually omitted.

Existential clauses may be verbless (with the Existee as nominal predicate) or verbal (using the verb \textit{ai}, with the Existee as subject). They may be expanded with a possessor to form possessive clauses; these are usually constructed with a verb: ‘His house existed’ = ‘He had a house’. Possession may also be expressed in a topic-comment construction: ‘As for him, there was a house.’

In recent years, Rapa Nui has seen the emergence of two copula verbs: \textit{ai} ‘to be’ and \textit{riro} ‘to become’. This development becomes clear by comparing old and new versions of stories in which a person transforms into an animal: in old versions the transformation is expressed in a nominal clause, in new versions \textit{riro} is used. In copula constructions, the nominal predicate is marked with \textit{he}, just as in nonverbal clauses. \textit{Riro} was borrowed from \ili{Tahitian}, but only in Rapa Nui did it develop into a copula verb.
\is{Clause!nominal|)}
