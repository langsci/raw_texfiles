\chapter[Possession]{Possession}\label{ch:6}
\section{Introduction}\label{sec:6.1}
\is{Possession|(}
This chapter describes the syntax and semantics of possessive constructions.\footnote{\label{fn:282}The term \textit{possessive} is used as a technical term here including not only relationships of possession, but any relationship expressed by possessive pronouns\is{Pronoun!possessive}, possessive prepositions, or benefactive pronouns\is{Pronoun!benefactive} or noun phrases (cf. \citealt[94]{Lichtenberk1985}). For relationships within noun phrases, the term \textit{possessee} is used for the head, \textit{possessor}\is{Possession} for the possessive modifier.} Possessive constructions in Rapa Nui are defined by the use of the possessive prepositions \textit{o}\is{o (possessive prep.)} and \textit{{\ꞌ}a}\is{a (possessive prep.)@{\ꞌ}a (possessive prep.)}. They occur in a wide variety of syntactic environments: as modifiers in the noun phrase, as predicates of nominal clauses, and in several other constructions. This range of possessive constructions is discussed in \sectref{sec:6.2}. \sectref{sec:6.3} deals with the semantics of possessives. Possessives express a wide range of relationships besides the idea of possession as such; these are described in \sectref{sec:6.3.1}. 

Whether a possessive construction is marked with \textit{o} or \textit{a} depends on the semantic relationship between possessor and possessee. The distinction between the two classes of possession is discussed in \sectref{sec:6.3.2}–\ref{sec:6.3.4}.

The \textit{o/a} distinction also applies to the benefactive prepositions \textit{mo/ma} (\sectref{sec:4.7.7}),\footnote{\label{fn:283}Possessive constructions marked with prepositions or possessive pronouns\is{Pronoun!possessive} are the common pattern in Polynesian. In this respect, Polynesian is unusual within the wider family of Oceanic languages: the latter are generally characterised by a distinction between direct and indirect possession (\citealt{Lynch1997}; \citealt{Lichtenberk1985}). Direct possession is marked by suffixes on the noun; in Polynesian, traces of this system survive in a number of kinship terms\is{Kinship term} ending in \textit{-na} (\citealt{Lynch1997}; \citealt{Marck1996Kin}); in Rapa Nui: \textit{tuakana} ‘older sibling’; \textit{taina} ‘sibling’, \textit{makupuna} ‘grandchild’, \textit{tupuna} ‘ancestor’; possibly also \textit{ha{\ꞌ}ana} ‘woman’s sister’s husband (obsolete)’ (\citealt[99]{Métraux1971}), cf. \is{Proto-Polynesian}PPN \textit{*sa{\ꞌ}a} ‘family, clan’.} which in turn form the basis for benefactive pronouns\is{Pronoun!benefactive} (\sectref{sec:4.2.3}). The semantic relationships discussed in the sections below are also valid for benefactives.

\section{Possessive constructions}\label{sec:6.2}
\is{Possession}
As mentioned above, all possessive constructions contain a possessive preposition \textit{o} or \textit{{\ꞌ}a}. In certain constructions, \textit{o} and \textit{{\ꞌ}}\textit{a} coalesce with the determiner \textit{te} into a marker \textit{to} or \textit{ta}. These four forms (\textit{o} and \textit{{\ꞌ}a}, \textit{to} and \textit{ta}) in turn form the basis for possessive pronouns\is{Pronoun!possessive} (\sectref{sec:4.2.2}). The \textit{o/a} forms are labelled Ø-possessives, the \textit{to/ta} forms \textit{t-}possessives.

In this section, the range of possessive constructions is discussed. \sectref{sec:6.2.1} deals with the use of possessives in the noun phrase. A peculiar noun phrase construction is the partitive; this is discussed in \sectref{sec:6.2.2}. Other possessive constructions (such as possessive clauses) are listed in \sectref{sec:6.2.3}; they are discussed in more detail elsewhere in this grammar. \sectref{sec:6.2.4} summarises the use of the different possessive forms.

\subsection{Possessives in the noun phrase}\label{sec:6.2.1}

Possessive \textsc{noun phrases} (i.e. those involving a common or proper noun) occur at the end of the noun phrase, after other particles. They are introduced by \textit{o}\is{o (possessive prep.)} or \textit{{\ꞌ}a}\is{a (possessive prep.)@{\ꞌ}a (possessive prep.)}:\textstyleFootnoteSymbol{} \footnote{\label{fn:284}In other Eastern Polynesian languages, possessive noun phrases may also occur in the determiner position, introduced by a \textit{t-}form \textit{to} or \textit{ta}. See for example the following example from \ili{Tahitian} (pers.obs.):
\ea
\gll
E mea maita{\ꞌ}i {\ob}tā terā ta{\ꞌ}ata\,{\cb} tipi.    ~ ~ \textup{(\ili{Tahitian})}\\
  \textsc{pred} thing good {\db}\textsc{art}:of\textsc{.a} \textsc{dist} man knife\\
\glt 
‘That man’s knife is good.’ 
\z
The head noun \textit{tipi} is preceded by a genitive noun phrase \textit{tā terā ta{\ꞌ}ata}; the possessive preposition \textit{ta} is based on the article \textit{te} + possessive \textit{a}. In Rapa Nui this construction occurs only in headless\is{Noun phrase!headless} noun phrases (\sectref{sec:5.6}). \citet[336]{Fischer2000Rapanui} gives an example of a prenominal noun phrase possessor\is{Possession} (\textit{t}\textit{ā te taŋata poki} ‘the man’s children’), but such a construction does not occur anywhere in my corpus.}

\ea\label{ex:6.1}
\gll ko te vaka tama{\ꞌ}i era \textbf{o} \textbf{te} \textbf{{\ꞌ}ariki} \\
\textsc{prom} \textsc{art} canoe fight \textsc{dist} of \textsc{art} king \\

\glt 
‘the battle canoe of the king’ \textstyleExampleref{[R345.013]} 
\z

\ea\label{ex:6.2}
\gll te poki \textbf{{\ꞌ}a} \textbf{Taka} \\
\textsc{art} child of\textsc{.a} Taka \\

\glt 
‘Taka’s child’ \textstyleExampleref{[Mtx-3-03.024]}
\z

\ea\label{ex:6.3}
\gll mai tū {\ꞌ}ōpata nei {\ꞌ}ā \textbf{o} \textbf{te} \textbf{karikari} \\
from \textsc{dem} cliff \textsc{prox} \textsc{ident} of \textsc{art} narrow\_part \\

\glt
‘from this same cliff of the narrow rock’ \textstyleExampleref{[R112.011]} 
\z

Possessive noun phrases also appear in headless\is{Noun phrase!headless} noun phrases\is{Noun phrase!headless}, in which case \textit{o}/\textit{{\ꞌ}a} coalesces with the determiner to \textit{to}/\textit{ta} (examples (\ref{ex:5.102}–\ref{ex:5.103}) in \sectref{sec:5.6}).

When the possessor\is{Possession} is \textsc{pronominal}, i.e. a possessive pronoun, it may occur in three different positions, as \tabref{tab:36} and \tabref{tab:37} in \sectref{sec:5.1} show: in determiner position; before the noun but not in determiner position; after the noun.

As explained in \sectref{sec:4.2.2}, Rapa Nui has two sets of possessive pronouns\is{Pronoun!possessive}: \textit{t-}possessives (\textit{tā{\ꞌ}ana, te mātou}) and Ø-possessives (\textit{{\ꞌ}ā{\ꞌ}ana, o~mātou})\is{Pronoun!possessive!Ø-class}. Which form is used, depends on the position of the pronoun in the noun phrase. \is{Genitive}

When the noun phrase needs a determiner (\sectref{sec:5.3.2.1}), possessive pronouns\is{Pronoun!possessive} usually occur in the determiner position. In this position, \textit{t}{}-possessives\is{Pronoun!possessive!t-class} are used. This happens for example when the noun phrase is subject, or occurs after a preposition:

\ea\label{ex:6.4}
\gll ¿He aha \textbf{tō{\ꞌ}ona} rua {\ꞌ}īŋoa? \\
~\textsc{pred} what \textsc{poss.3sg.o} two name \\

\glt 
‘What is his second name?’ \textstyleExampleref{[R412.079]} 
\z

\ea\label{ex:6.5}
\gll Ko haŋa {\ꞌ}ana a au mo u{\ꞌ}i i \textbf{tā{\ꞌ}aku} vi{\ꞌ}e mo hāipoipo. \\
\textsc{prf} want \textsc{cont} \textsc{prop} \textsc{1sg} for look \textsc{acc} \textsc{poss.1sg.a} woman for marry \\

\glt 
‘I want to find a wife to marry’ \textstyleExampleref{[R491.005]} 
\z

\ea\label{ex:6.6}
\gll hai matavai {\ꞌ}i \textbf{tō{\ꞌ}oku} mata \\
\textsc{ins} tear at \textsc{poss.1sg.o} eye \\

\glt
‘with tears in my eyes’ \textstyleExampleref{[R221.009]} 
\z

That the possessive pronoun is in determiner position, is also confirmed by the fact that prenominal quantifiers\is{Quantifier} occur after the possessor\is{Possession}, as in \REF{ex:6.4}.

Possessive pronouns may also occur before the noun in noun phrases not containing a determiner\is{Quantifier}. In that case, Ø\nobreakdash-possessives\is{Pronoun!possessive!Ø-class} are used.\footnote{\label{fn:285}When a \textit{t}{}-possessive\is{Pronoun!possessive!t-class} occurs in this position, the numeral is not part of the noun phrase, but predicate of a so-called ‘numerical clause’\is{Clause!numerical} (\sectref{sec:9.5}).} This happens especially when the noun phrase contains a prenominal numeral, but also after the negator \textit{{\ꞌ}ina}\is{ina (negator)@{\ꞌ}ina (negator)}. Prenominal numerals preclude the use of all determiners (\sectref{sec:5.3.2.2}), while \textit{{\ꞌ}ina} precludes the use of \textit{t-}determiners (\sectref{sec:10.5.1}):

\ea\label{ex:6.7}
\gll He ai e tahi \textbf{{\ꞌ}ā{\ꞌ}ana} poki {\ꞌ}i roto o te vi{\ꞌ}e ko Rurita. \\
\textsc{ntr} exist \textsc{num} one \textsc{poss.3sg.a} child at inside of \textsc{art} woman \textsc{prom} Rurita \\

\glt 
‘He had one (lit. there was one his) child by the woman Rurita.’ \textstyleExampleref{[R309.027]} 
\z

\ea\label{ex:6.8}
\gll Te nu{\ꞌ}u nei e ai rō {\ꞌ}ā e rua \textbf{rāua} ŋā poki \\
\textsc{art} people \textsc{prox} \textsc{ipfv} exist \textsc{emph} \textsc{cont} \textsc{num} two \textsc{3pl} \textsc{pl} child \\

\glt 
‘These people had two children (lit. there were two their children).’ \textstyleExampleref{[R481.005]} 
\z

\ea\label{ex:6.9}
\gll {\ꞌ}Ina \textbf{{\ꞌ}ā{\ꞌ}aku} nanue~para era o nei. \\
\textsc{neg} \textsc{poss.1sg.a} kind\_of\_fish \textsc{dist} of \textsc{prox} \\

\glt
‘My \textit{nanue para} fish is not here.’ \textstyleExampleref{[R301.272]} 
\z

When plural Ø-pronouns\is{Pronoun!possessive!Ø-class} occur before the noun, the \textit{o} is omitted.\footnote{\label{fn:286}\citet[106]{Wilson1985} gives an example from Rapa Nui in which the genitive preposition is not omitted (modified spelling \& gloss):
\ea
\gll 
E rua \textbf{o} \textbf{mātou} hare.\\
 \textsc{num} two of \textsc{1pl.excl} house\\
\glt   ‘We have two houses.’ \z
Unfortunately, no source is given for this example; it may well be erroneous, as no examples of this construction occur in my corpus.} This means that they have the same form as the corresponding personal pronouns\is{Pronoun!personal}; only their position identifies them as possessive pronouns\is{Pronoun!possessive}.

\ea\label{ex:6.10}
\gll E ai rō {\ꞌ}ā e tahi \textbf{rāua} poki tane te {\ꞌ}īŋoa ko Iovani. \\
\textsc{ipfv} exist \textsc{emph} \textsc{cont} \textsc{num} one \textsc{3pl} child male \textsc{art} name \textsc{prom} Iovani \\

\glt 
‘They had one son (lit. there was one their son) named Iovani.’ \textstyleExampleref{[R238.002]} 
\z

\ea\label{ex:6.11}
\gll {\ꞌ}Ina \textbf{tātou} haŋu mo kai ko kuā nua. \\
\textsc{neg} \textsc{1pl.incl} sustenance for eat \textsc{prom} \textsc{coll} Mum \\

\glt
‘We don’t have any food left to eat, me and Mum.’ \textstyleExampleref{[R372.047]} 
\z

After the noun, these pronouns do have the \textit{o}, as illustrated in \REF{ex:6.12} below.

Finally, possessive pronouns\is{Pronoun!possessive} may occur at the end of the noun phrase, in the same position as possessive noun phrases (see (\ref{ex:6.1}–\ref{ex:6.3}) above). In this position, Ø-possessives\is{Pronoun!possessive!Ø-class} are used:

\ea\label{ex:6.12}
\gll He hiŋa {\ꞌ}i tū kori haŋa rahi era \textbf{o} \textbf{rāua} he haka nininini {\ꞌ}i ruŋa  o te ma{\ꞌ}uŋa.\\
\textsc{ntr} fall at \textsc{dem} play love much \textsc{dist} of \textsc{3pl} \textsc{pred} \textsc{caus} spin:\textsc{red} at above  of \textsc{art} mountain\\

\glt 
‘He fell during that much-loved game of theirs, (which was) sliding down the hill.’ \textstyleExampleref{[R313.103]} 
\z

\ea\label{ex:6.13}
\gll pē tū vārua moe era {\ꞌ}ā e tū poki taina era \textbf{ō{\ꞌ}ona} \\
like \textsc{dem} spirit lie\_down \textsc{dist} \textsc{cont} \textsc{ag} \textsc{dem} child sibling \textsc{dist} \textsc{poss.3sg.o} \\

\glt
‘like that dream dreamt by her sister (lit. that sister of hers)’ \textstyleExampleref{[R347.131]} 
\z

These postnominal possessives occur when the determiner slot is occupied by another element. As these examples show, this especially happens when the noun phrase contains a demonstrative determiner such as \textit{tū}\is{tu (demonstrative determiner)@tū (demonstrative determiner)}. \textit{Tū} fulfills the requirement for the noun phrase to have a determiner, but it precludes the use of a prenominal possessive, hence the possessive is placed after the noun.

Sometimes a \textit{t}-possessive\is{Pronoun!possessive!t-class} pronoun before the noun occurs together with a Ø-pos\-ses\-sive\is{Pronoun!possessive!Ø-class} after the noun. In this double possessive construction, the two pronouns reinforce each other:

\ea\label{ex:6.14}
\gll Ka turu era \textbf{tu{\ꞌ}u} rima \textbf{ō{\ꞌ}ou} ki te kai era mo to{\ꞌ}o mai. \\
\textsc{cntg} go\_down \textsc{dist} \textsc{poss.2sg.o} hand \textsc{poss.2sg.o} to \textsc{art} food \textsc{dist} for take hither \\

\glt 
‘When your hand goes down to take the food...’ \textstyleExampleref{[R310.088]} 
\z

\ea\label{ex:6.15}
\gll Ki \textbf{ta{\ꞌ}a} u{\ꞌ}i \textbf{{\ꞌ}ā{\ꞌ}au}, ¿e hau rā hora {\ꞌ}i te rivariva ki te hora nei? \\
to \textsc{poss.2sg.a} look \textsc{poss.2sg.a} ~\textsc{ipfv} exceed \textsc{dist} time at \textsc{art} good:\textsc{red} to \textsc{art} time \textsc{prox} \\

\glt
‘In your view, was that time better than the present time?’ \textstyleExampleref{[R380.106]} 
\z

Possessive doubling only happens in the second person. The \textit{t}{}-possessive\is{Pronoun!possessive!t-class} before the noun is always one of the shortened forms \textit{tu{\ꞌ}u} or \textit{ta{\ꞌ}a} (\sectref{sec:4.2.2.1.1}).

\subsection{The partitive construction}\label{sec:6.2.2}
\is{Partitive}\is{Partitive|(}
\is{Noun phrase!headless}Besides the common construction “\textit{t}{}-possessive\is{Pronoun!possessive!t-class} N” discussed above, Rapa Nui has a construction “\textit{t}{}-possessive\is{Pronoun!possessive!t-class} \textit{o te} N”. In this construction, the possessee has been demoted from the head noun position to a possessive phrase \textit{o te N}. The construction has a partitive\is{Partitive} sense, indicating someone’s share, portion: \textit{\mbox{tā{\ꞌ}aku} o te vai} = ‘my portion of the water, the part of the water that is mine’. Some examples: 

\ea\label{ex:6.16}
\gll Mai \textbf{tā{\ꞌ}aku} \textbf{o} \textbf{te} \textbf{vai}. \\
hither \textsc{poss.1sg.a} of \textsc{art} water \\

\glt 
‘Give me some water.’ \textstyleExampleref{[Notes]}
\z

\ea\label{ex:6.17}
\gll Mo {\ꞌ}avai atu i \textbf{tō{\ꞌ}ou} \textbf{o} \textbf{te} \textbf{parehe}... \\
for give away \textsc{acc} \textsc{poss.2sg.o} of \textsc{art} piece \\

\glt 
‘(I want) to give a piece to you...’ \textstyleExampleref{[R219.021]} 
\z

\ea\label{ex:6.18}
\gll He ta{\ꞌ}o tako{\ꞌ}a \textbf{to} \textbf{rāua} \textbf{o} \textbf{te} \textbf{taŋata} mo kai.\\
\textsc{ntr} cook\_in\_earth\_oven also \textsc{art}:of \textsc{3pl} of \textsc{art} man for eat\\

\glt
‘They also cooked people for them(selves) to eat.’ \textstyleExampleref{[Mtx-3-01.282]}
\z

As the examples above show, the sense of ‘share, portion’ often implies that the item is not yet in the hands of the possessor\is{Possession}, but destined for him or her. 

This construction may be emphatic: ‘yours, nobody else’s’:

\ea\label{ex:6.19}
\gll ...{\ꞌ}e a koe ka ha{\ꞌ}amata ka kimi \textbf{tā{\ꞌ}au} \textbf{o} \textbf{te} \textbf{repa}. \\
~~~and \textsc{prop} \textsc{2sg} \textsc{cntg} begin \textsc{cntg} search \textsc{poss.2sg.a} of \textsc{art} young\_man \\

\glt 
‘...and you should start looking for your own boyfriend.’ \textstyleExampleref{[R315.258]} 
\z

\ea\label{ex:6.20}
\gll He haka eke i te poki nei, he eke ko ia i \textbf{tō{\ꞌ}ona} \textbf{o} \textbf{te} \textbf{hoi}. \\
\textsc{ntr} \textsc{caus} go\_up \textsc{acc} \textsc{art} child \textsc{prox} \textsc{ntr} go\_up \textsc{prom} \textsc{3sg} \textsc{acc} \textsc{poss.3sg.o} of \textsc{art} horse \\

\glt 
‘He lifted the boy on the horse, and he mounted on his own horse.’ \textstyleExampleref{[R105.028]} 
\z

As these examples show, in this construction the long second-person pronouns \textit{tō{\ꞌ}ou} and \textit{tā{\ꞌ}au} are used, even though prenominal possessive pronouns\is{Pronoun!possessive} usually have one of the short forms \textit{ta{\ꞌ}a, tu{\ꞌ}u} etc (\sectref{sec:4.2.2.1.1}). There is another difference between prenominal possessives and partitives. While prenominal possessives can only be pronouns (\sectref{sec:6.2.1} above), the possessive in a partitive\is{Partitive} construction may also be a full noun phrase. This noun phrase is constructed with a possessive preposition \textit{to} or \textit{ta}, following the \textit{o}/\textit{a} distinction (\sectref{sec:6.3.2}). In the following examples, just as in some of the examples above, the construction expresses something destined for the possessor\is{Possession}.

\ea\label{ex:6.21}
\gll {\ꞌ}Ī au he ha{\ꞌ}ata{\ꞌ}a i \textbf{to} \textbf{Vaha} o te kahu. \\
\textsc{imm} \textsc{1sg} \textsc{ntr} separate \textsc{acc} \textsc{art}:of Vaha of \textsc{art} cloth(es) \\

\glt 
‘I will put apart some clothes for Vaha.’ \textstyleExampleref{[R229.194]} 
\z

\ea\label{ex:6.22}
\gll {\ꞌ}Ī au he ha{\ꞌ}ata{\ꞌ}a i \textbf{ta} \textbf{Māria} o te kai. \\
\textsc{imm} \textsc{1sg} \textsc{ntr} separate \textsc{acc} \textsc{art}:of\textsc{.a} Maria of \textsc{art} food \\

\glt 
‘I will put apart some food for Maria.’ \textstyleExampleref{[Notes]}
\z

A similar but simpler construction – which can be labelled “pseudo-partitive” – is \textit{to~te~N}. In this construction, the noun phrase \textit{te N} is introduced by \textit{to} in a possessive/partitive sense: 

\ea\label{ex:6.23}
\gll Te taŋata e ai rō {\ꞌ}ā tā{\ꞌ}ana kai, ka va{\ꞌ}ai \textbf{to} \textbf{te} \textbf{taŋata}  {\ꞌ}ina {\ꞌ}ā{\ꞌ}ana kai.\\
\textsc{art} man \textsc{ipfv} exist \textsc{emph} \textsc{cont} \textsc{poss.3sg.a} food \textsc{imp} give \textsc{art}:of \textsc{art} man  \textsc{neg} \textsc{poss.3sg.a} food\\

\glt 
‘The man who has food, should give some to the man who does not have food.’ \textstyleExampleref{[Luke 3:11]}
\z

\ea\label{ex:6.24}
\gll Te ŋā kai {\ꞌ}āpī ra{\ꞌ}e era... e ma{\ꞌ}u \textbf{to} \textbf{te} \textbf{hare} \textbf{pure} {\ꞌ}i ra{\ꞌ}e. \\
\textsc{art} \textsc{pl} food new first \textsc{dist} \textsc{ipfv} carry \textsc{art}:of \textsc{art} house prayer at first \\

\glt
‘The first food... they first had to take some to the church (lit. carry those of the church)’ \textstyleExampleref{[R539-3.150]}
\z

This construction is reminiscent, syntactically speaking, of the headless\is{Noun phrase!headless} possessive construction (\sectref{sec:5.6}), of which an example is given here: 

\ea\label{ex:6.25}
\gll Ko Koka te {\ꞌ}īŋoa o tō{\ꞌ}ona hoi... ko Parasa  \textbf{to} \textbf{te} \textbf{rū{\ꞌ}au} era {\ꞌ}ā{\ꞌ}ana.\\
\textsc{prom} Koka \textsc{art} name of \textsc{poss.3sg.o} horse \textsc{prom} Parasa  \textsc{art}:of \textsc{art} old\_woman \textsc{dist} \textsc{poss.3sg.a}\\

\glt
‘Koka was the name of the horse he went on, Parasa the (name) of his old wife.’ \textstyleExampleref{[R539-1.420]}
\z

There is an important difference though: while in \REF{ex:6.25} \textit{to te rū{\ꞌ}au} has a straightforward possessive sense (parallel to the possessive phrase \textit{o tō{\ꞌ}ona hoi}), in (\ref{ex:6.23}–\ref{ex:6.24}) the possessive phrase occurs in a context where normally the dative preposition \textit{ki} would be used.

Semantically, (\ref{ex:6.23}–\ref{ex:6.24}) are similar to the partitive\is{Partitive} construction discussed above. In both cases, the noun phrase refers to something which is destined for the person referred to; moreover, the sense is partitive\is{Partitive}: ‘some of the food, some of the clothes’. Also, in both cases the \textit{to}{}-phrase is independent: there is no head noun to which it is attached. The difference is that in the partitive\is{Partitive} construction in (\ref{ex:6.16}–\ref{ex:6.18}) above the possessee is expressed by a genitive phrase \textit{o te kahu} which is semantically the head of the phrase (the noun phrase as a whole refers to ‘clothes’, not to ‘Vaha’), while in (\ref{ex:6.23}–\ref{ex:6.24}) it is not expressed at all.
\is{Partitive|)}
\subsection{Other possessive constructions}\label{sec:6.2.3}

Possessive constructions occur not only as modifiers in the noun phrase, but in a range of other constructions as well. This section gives a concise listing; all of these constructions (with the exception of the elliptic construction in \REF{ex:6.34} below) are discussed in more detail elsewhere in this grammar.\footnote{\label{fn:287}Not included here are possessives expressing the subject of a relative clause in the “possessive-relative construction”; as argued in \sectref{sec:11.4.4}, these should be considered as normal noun-phrase possessors which are syntactically separate from the relative clause.}
\is{Pronoun!possessive|(}
\subparagraph{\ref{sec:6.2.3}.1} \textit{o}{}-class Ø-possessives\is{Pronoun!possessive!Ø-class} are used to mark the S/A argument of a clause introduced by preverbal \textit{mo}\is{mo (preverbal)} ‘if; in order to’ (\sectref{sec:11.5.1.2}):

\ea\label{ex:6.26}
\gll Mo kī \textbf{ō{\ꞌ}oku} he teatea, he rere a ruŋa he kī atu he {\ꞌ}uri{\ꞌ}uri. \\
if say \textsc{poss.1sg.o} \textsc{pred} white:\textsc{red} \textsc{ntr} jump by above \textsc{ntr} say away \textsc{pred} black:\textsc{red} \\

\glt
‘If I say it’s white, he jumps up and says it’s black.’ \textstyleExampleref{[R480.003]} 
\z

Occasionally, \textit{O}-class Ø-possessives are used to mark the S/A argument of a main clause (\sectref{sec:8.6.4.1}):

\ea\label{ex:6.27}
\gll He u{\ꞌ}i atu \textbf{ō{\ꞌ}oku} i tō{\ꞌ}oku pāpā era...\\
\textsc{ntr} look away \textsc{poss.1sg.o} \textsc{acc} \textsc{poss.1sg.o} father \textsc{dist}\\

\glt 
‘Then I saw my father...’ \textstyleExampleref{[R101.012]} 
\z

\subparagraph{\ref{sec:6.2.3}.2} \textit{a}{}-class Ø-possessives\footnote{\label{fn:288}\textit{A}-forms only occur with singular pronouns and with proper nouns (\sectref{sec:6.3.2}); with plural pronouns and with common nouns, only the default \textit{o}{}-forms are available.}\is{Pronoun!possessive!Ø-class} serve to express the Agent in the actor-emphatic\is{Actor-emphatic construction} construction (\sectref{sec:8.6.3}):

\ea\label{ex:6.28}
\gll ¡\textbf{{\ꞌ}Ā{\ꞌ}au} rō ta{\ꞌ}a moeŋa nei o māua i toke! \\
~\textsc{poss.2sg.a} \textsc{emph} \textsc{poss.2sg.a} mat \textsc{prox} of \textsc{1du.excl} \textsc{pfv} steal \\

\glt 
‘It was you who stole that mat of ours!’ \textstyleExampleref{[R310.428]} 
\z

\subparagraph{\ref{sec:6.2.3}.3} Ø-possessives (both \textit{a}{}- and \textit{o}{}-class) served as the predicate of proprietary clauses\is{Clause!proprietary} (\sectref{sec:9.4.2}).

\ea\label{ex:6.29}
\gll \textbf{Ō{\ꞌ}ona} ho{\ꞌ}i te {\ꞌ}āua era. \\
\textsc{poss.3sg.o} indeed \textsc{art} field \textsc{dist} \\

\glt 
‘That field is his.’ \textstyleExampleref{[R413.228]} 
\z

\subparagraph{\ref{sec:6.2.3}.4} In older Rapa Nui, the \textit{t}{}-possessives\is{Pronoun!possessive!t-class} serve as the predicate of possessive clauses (\sectref{sec:9.3.3})\is{Clause!possessive}. In modern Rapa Nui, this construction is no longer in use.\footnote{\label{fn:289}In modern Rapa Nui, possessive clauses are constructed as verbal or verbless existential clauses; the possessor is expressed not as a predicate, but as a noun phrase modifier (\sectref{sec:9.3.3}):}

\ea\label{ex:6.30}
\gll He poki \textbf{tā{\ꞌ}ana} e tahi, poki tamāroa. \\
\textsc{ntr} child \textsc{poss.3sg.a} \textsc{num} one child male \\

\glt 
‘He had a child, a boy.’ \textstyleExampleref{[Ley-9-57.002]}
\z

\subparagraph{\ref{sec:6.2.3}.5} Possessives of the \textit{o}{}-class may serve as the predicate of existential-locative clauses (\sectref{sec:9.3.2}); see \sectref{sec:6.3.1}.8 below on the locative use of possessives. In modern Rapa Nui, Ø-possessives are used as in \REF{ex:6.31}; in older Rapa Nui, \textit{t}{}-possessives were used as in (\ref{ex:6.32}–\ref{ex:6.33}). 

\ea\label{ex:6.31}
\gll ¿E ai rō {\ꞌ}ā te ika \textbf{o} \textbf{roto}? \\
~\textsc{ipfv} exist \textsc{emph} \textsc{cont} \textsc{art} fish of inside \\

\glt 
‘Are there fish inside (the net)?’ \textstyleExampleref{[R241.058]} 
\z

\ea\label{ex:6.32}
\gll He taŋata \textbf{to} \textbf{nei}... Ŋata Vake te {\ꞌ}īŋoa.\\
\textsc{pred} person \textsc{art}:of \textsc{prox} Ngata Vake \textsc{art} name\\

\glt 
‘There was a man here, called Ngata Vake.’ \textstyleExampleref{[Ley-3-02.002]}
\z

\ea\label{ex:6.33}
\gll He taŋata \textbf{to} \textbf{ruŋa} to te motu, ko Motu Tapu te {\ꞌ}īŋoa. \\
\textsc{pred} man \textsc{art}:of above \textsc{art}:of \textsc{art} islet \textsc{prom} Motu Tapu \textsc{art} name \\

\glt 
‘There was a man on the islet which was called Motu Tapu’ \textstyleExampleref{[Ley-8-52.024]}
\z

\subparagraph{\ref{sec:6.2.3}.6} Finally, \textit{to} (i.e. the \textit{t}{}-possessive \textit{o}{}-form) + locational is sometimes used as an elliptic noun phrase\is{Locational}. This use is especially found in older Rapa Nui. \textit{To} + noun refers to a group of people situated in the location indicated by the locational: ‘those ones inside/outside/over there...’ (cf. the headless\is{Noun phrase!headless} construction 2 above). In the following example, \textit{to haho} is a short way to refer to ‘the people outside’.

\ea\label{ex:6.34}
\gll He toe e toru te {\ꞌ}aŋahuru nō toe, ku oti {\ꞌ}ā \textbf{to} \textbf{haho}.  \\
\textsc{ntr} remain \textsc{num} three \textsc{art} ten just remain \textsc{prf} finish \textsc{cont} \textsc{art}:of outside  \\

\glt 
‘Only thirty (men) were left, those outside were finished.’ \textstyleExampleref{[Mtx-3-01.092]}
\z

\subsection{Summary: use of possessive forms}\label{sec:6.2.4}

The occurrence of the different possessive forms is summarised in \tabref{tab:42}.

\begin{table}[t]
% \resizebox{\textwidth}{!}{
\footnotesize{
\begin{tabularx}{\textwidth}{L{20mm}Xcccccccc}
\lsptoprule
&  & \multicolumn{4}{c}{\textit{t-}possessive} & \multicolumn{4}{c}{Ø-possessive}\\
&  & \multicolumn{2}{c}{pronoun} & \multicolumn{2}{c}{noun} & \multicolumn{2}{c}{pronoun} & \multicolumn{2}{c}{noun}\\
&  & \textit{o}& \textit{a}& \textit{o}& \textit{a}& \textit{o}& \textit{a}& \textit{o}& \textit{a}\\
 & § & \textit{tō{\ꞌ}oku}& \textit{tā{\ꞌ}aku}& \textit{to NP}& \textit{ta NP}& \textit{ō{\ꞌ}oku}& \textit{{\ꞌ}ā{\ꞌ}aku}& \textit{o NP}& \textit{{\ꞌ}a NP}\\
\midrule
prenominal, determiner position &  \ref{sec:6.2.1}& x& x&  &  &  &  &  & \\
\tablevspace
prenominal, otherwise &  \ref{sec:6.2.1} &  &  &  &  & x& x&  & \\
\tablevspace
postnominal &  \ref{sec:6.2.1} &  &  &  &  & x& x& x& x\\
\tablevspace
partitive &  \ref{sec:6.2.2}& x& x& x& x&  &  &  & \\
\tablevspace
headless noun phrase &  \ref{sec:5.6}&  &  & x& x&  &  &  & \\
\tablevspace
S/A of \textit{mo}{}-clause &  \ref{sec:11.5.1.2}&  &  &  &  & x&  & x& \\
\tablevspace
S/A of main clause &  \ref{sec:8.6.4.1}&  &  &  &  & x&  & x& \\
\tablevspace
actor-emphatic Agent &  \ref{sec:8.6.3}&  &  &  &  &  & x&  & x\\
\tablevspace
proprietary predicate &  \ref{sec:9.4.2}&  &  &  &  & x& x& x& x\\
\tablevspace
possessive clause pred. (old RN) &  \ref{sec:9.3.3}& x& x& x& x&  &  &  & \\
\tablevspace
neg./num. possessive clause &  \ref{sec:9.3.3}&  &  &  &  & x& x& x& x\\
\tablevspace
existential-locative pred. (old RN) &  \ref{sec:9.3.2}&  &  & x&  &  &  &  & \\
\tablevspace
existential-locative pred. (modern RN) &  \ref{sec:9.3.2}&  &  &  &  &  &  & x& \\
\tablevspace
elliptic noun phrase &  \ref{sec:6.2.3} &  &  & x&  &  &  &  & \\
\lspbottomrule
\end{tabularx}
% }
}
\caption{Possessive constructions}
\label{tab:42}
\end{table}

\is{Pronoun!possessive|)}

The discussion in the previous sections has shown that various possessive forms are used, depending on the construction. Three parameters play a role, which are reflected in \tabref{tab:42}:

\begin{itemize}
\item 
the choice between \textit{t}{}- and Ø-possessives;

\item 
pronominal versus noun phrases possessors. In most constructions, both are possible, but in prenominal positions only pronominal possessors occur; 

\item 
\textit{o}{}- and \textit{a}{}-class. In most constructions both occur, depending on the semantic relationship between possessor and possessee; in some constructions, only one class is used. Regardless of the construction, \textit{a-}class forms are only used with singular pronouns and proper nouns (\sectref{sec:6.3.2}).

\end{itemize}

Summarising these data: \textit{t}{}-possessives are used in the following environments:

\begin{itemize}
\item 
in the noun phrase, in determiner position (this includes partitive constructions, headless and elliptic noun phrases);

\item 
in old Rapa Nui as the predicate of possessive clauses and existential-locative clauses.

\end{itemize}

Ø-possessives\is{Pronoun!possessive!Ø-class} are used:\footnote{\label{fn:290}The range of use of the Ø-possessive\is{Pronoun!possessive!Ø-class}s is remarkably similar to the use of \textit{n-}possessives which occur in most other \is{Eastern Polynesian}EP languages, e.g. \ili{Māori} \textit{nāku} ‘mine’, \textit{nōna} ‘his’ (cf. \citealt[316]{Wilson2012}). For example, in \ili{Māori} and \ili{Tahitian} \textit{n}{}-possessors are used in the actor-emphatic construction\is{Actor-emphatic construction} and in proprietary clauses\is{Clause!proprietary}; in \ili{Tahitian} and \ili{Hawaiian}, they also occur in the noun phrase (\citealt[208–209]{Bauer1993}; \citealt{Harlow2000}; \citealt[175–176]{LazardPeltzer2000}, 189; \citealt[349]{Cook2000}). \ili{Māori} also has Ø-possessive\is{Pronoun!possessive!Ø-class} pronouns\is{Pronoun!possessive}, which occur in the noun phrase and in negated possessive clauses\is{Clause!possessive} (\citealt[202, 381]{Bauer1993}; \citealt[359]{Harlow2000}). This suggests that the Rapa Nui Ø-possessive\is{Pronoun!possessive!Ø-class} pronouns\is{Pronoun!possessive} are cognates of both the \textit{n}{}-possessives and Ø-possessives\is{Pronoun!possessive!Ø-class} in other \is{Eastern Polynesian}EP languages: PEP had a set of Ø-possessives\is{Pronoun!possessive!Ø-class} and a set of \textit{n-}possessives; in Rapa Nui, the initial \textit{n-} was lost, so that both sets coincided; in \is{Central-Eastern Polynesian}CE languages, \textit{n-} was retained. 

The reverse scenario, in which \is{Eastern Polynesian}PEP only had the Ø forms and \textit{n-} was added in \is{Central-Eastern Polynesian}PCE, is unlikely. First, it would leave the co-existence of Ø- and \textit{n-}possessives in \ili{Māori} unexplained: if \is{Eastern Polynesian}PEP only had Ø-possessives\is{Pronoun!possessive!Ø-class}, it would be hard to explain why \textit{n-} was added in some contexts, while in other contexts the Ø-forms were retained. Second, there is no ready explanation for the addition of \textit{n-} within Central-Eastern Polynesian, while on the other hand the existence of \textit{n-} in \is{Eastern Polynesian}PEP can be explained either from the past tense marker \textit{ne}, or – more likely – from the possessive \textit{ni} which occurs in various Outliers (see \citealt[50]{Wilson1982}; \citealt[101]{Wilson1985}; \citealt[263]{Clark2000Possessive}).

We may conclude that the \textit{n}{}-possessives are not a \is{Central-Eastern Polynesian}PCE innovation as suggested by \citet[12]{Green1985}, but already present in \is{Eastern Polynesian}PEP (\sectref{sec:1.2.2}).}

\begin{itemize}
\item 
in the noun phrase, in non-determiner positions;

\item 
in actor-emphatic\is{Actor-emphatic construction} constructions;

\item 
as S/A arguments of \textit{mo}{}-clauses and – occasionally – main clauses;

\item 
as the predicate in proprietary and existential-locative clauses;

\item 
in possessive clauses\is{Clause!possessive} containing a numeral, and negative possessive clauses\is{Clause!possessive}.\is{Pronoun!possessive}

\end{itemize}
\section{The semantics of possessives}\label{sec:6.3}

As in many languages, possessive constructions express a wide range of relationships between two entities. These are listed in \sectref{sec:6.3.1}.

When the possessor is a singular pronoun or a proper noun, it can be expressed in two ways, using either \textit{o} or \textit{a}. The distinction between \textit{o} and \textit{a} is discussed in \sectref{sec:6.3.2}–\ref{sec:6.3.4}; in those sections, the range of relationships expressed by the possessive will be discussed and illustrated in more detail. 

\subsection{Relationships expressed by possessives}\label{sec:6.3.1}

Possessive constructions serve to express a wide range of relationships between two entities.

\subparagraph{\ref{sec:6.3.1}.1} Ownership:

\ea\label{ex:6.35}
\gll He haka hopu i \textbf{tā{\ꞌ}ana} paiheŋa.\\
\textsc{ntr} \textsc{caus} wash \textsc{acc} \textsc{poss.3sg.a} dog\\

\glt 
‘She washed her dog.’ \textstyleExampleref{[R168.012]} 
\z

\subparagraph{\ref{sec:6.3.1}.2} Whole/part relations:

\ea\label{ex:6.36}
\gll He puru i te papae \textbf{o} \textbf{te} \textbf{hare}. \\
\textsc{ntr} close \textsc{acc} \textsc{art} door of \textsc{art} house \\

\glt 
‘He closed the door of the house.’ \textstyleExampleref{[R310.144]} 
\z

\subparagraph{\ref{sec:6.3.1}.3} Interhuman relationships, such as kinship and friendship:

\ea\label{ex:6.37}
\gll He hokorua a au i \textbf{tō{\ꞌ}oku} repahoa.\\
\textsc{ntr} accompany \textsc{prop} \textsc{1sg} \textsc{acc} \textsc{poss.1sg.o} friend\\

\glt 
‘I accompany my friend.’ \textstyleExampleref{[R208.138]} 
\z

\subparagraph{\ref{sec:6.3.1}.4} Attributes:

\ea\label{ex:6.38}
\gll ¿He aha \textbf{to{\ꞌ}u} tau \textbf{tu{\ꞌ}u} rivariva? \\
~\textsc{ntr} what \textsc{poss.2sg.o} pretty \textsc{poss.2sg.o} good:\textsc{red} \\

\glt 
‘What (use) is your beauty, your goodness?’ \textstyleExampleref{[R372.045]} 
\z

\subparagraph{\ref{sec:6.3.1}.5} Specification (epexegetical use), where the possessive has the same referent as the head noun:

\ea\label{ex:6.39}
\gll He eke ki ruŋa ki te {\ꞌ}ana \textbf{o} \textbf{{\ꞌ}Ana} \textbf{Havea}. \\
\textsc{ntr} go\_up to above to \textsc{art} cave of Ana Havea \\

\glt 
‘He climbed above the cave (of) Ana Havea.’ \textstyleExampleref{[Mtx-7-18.010]}
\z

\subparagraph{\ref{sec:6.3.1}.6} Actions, feelings and emotions, when these are expressed as nouns or nominalised verbs\is{Verb!nominalised}:

\ea\label{ex:6.40}
\gll Me{\ꞌ}e rahi \textbf{tō{\ꞌ}oku} māuruuru ki a koe, e koro ē. \\
thing much \textsc{poss.1sg.o} thank to \textsc{prop} \textsc{2sg} \textsc{voc} Dad \textsc{voc} \\

\glt 
‘I am very grateful (lit. much is my gratitude) to you, Dad.’ \textstyleExampleref{[R363.112]} 
\z

\ea\label{ex:6.41}
\gll ...{\ꞌ}o \textbf{tō{\ꞌ}oku} kī nō mo to{\ꞌ}o mai i a Puakiva mai i a ia. \\
~~~~because\_of \textsc{poss.1sg.o} say just for take hither \textsc{acc} \textsc{art} Puakiva from at \textsc{prop} \textsc{3sg} \\

\glt 
‘(Kava is crying) because I said (lit. because of my saying) I would take Puakiva away from her.’ \textstyleExampleref{[R229.017]} 
\z

\subparagraph{\ref{sec:6.3.1}.7} Any kind of association, for example between a person and a group, or a person and a location:

\ea\label{ex:6.42}
\gll He raŋi te {\ꞌ}ariki ki \textbf{tō{\ꞌ}ona} taŋata... \\
\textsc{ntr} call \textsc{art} king to \textsc{poss.3sg.o} man \\

\glt 
‘The king called out to his people...’ \textstyleExampleref{[MsE-055.005]}
\z

\ea\label{ex:6.43}
\gll Ka haka hoki ki \textbf{tō{\ꞌ}ona} henua.\\
\textsc{cntg} \textsc{caus} return to \textsc{poss.3sg.o} land\\

\glt 
‘Let (him) return to his country.’ \textstyleExampleref{[Ley-9-63.065]}
\z

\subparagraph{\ref{sec:6.3.1}.8} The possessor may be a location to which the possessee belongs as in \REF{ex:6.44}, or a place where the possessee is located at a given time, as in (\ref{ex:6.45}–\ref{ex:6.46}).

\ea\label{ex:6.44}
\gll Te me{\ꞌ}e nei he heke, he {\ꞌ}animare e tahi \textbf{o} \textbf{rote} \textbf{vaikava}. \\
\textsc{art} thing \textsc{prox} \textsc{pred} octopus \textsc{pred} animal \textsc{num} one of inside\_the ocean \\

\glt 
‘The octopus is an animal of (lit. of inside) the ocean.’ \textstyleExampleref{[R356.029]} 
\z

\ea\label{ex:6.45}
\gll {\ꞌ}Ina he ika \textbf{o} \textbf{{\ꞌ}Apina}. \\
\textsc{neg} \textsc{prom} fish of Apina \\

\glt 
‘There are no fish at Apina.’ \textstyleExampleref{[R301.292]} 
\z

\ea\label{ex:6.46}
\gll Ko Alfredo te me{\ꞌ}e era \textbf{o} \textbf{mu{\ꞌ}a} i te microfono. \\
\textsc{prom} Alfredo \textsc{art} thing \textsc{dist} of front at \textsc{art} microphone \\

\glt
‘Alfredo is the one in front of the microphone.’ \textstyleExampleref{[R415.600]} 
\z

In (\ref{ex:6.45}–\ref{ex:6.46}), \textit{o} is close in meaning to the locative \textit{{\ꞌ}i} ‘in, at’. As these examples show, the possessive is used especially in negative\is{Negation} or interrogative\is{Question} clauses. In those sentences, \textit{{\ꞌ}i}\is{i ‘in, at’@{\ꞌ}i ‘in, at’} is considered awkward.

\subparagraph{\ref{sec:6.3.1}.9} When the head noun expresses time, the possessive may express an event with respect to which this time applies. In \REF{ex:6.47} the event is punctual, and ‘three days’ is the time elapsed after the event. In \REF{ex:6.48} the event is durative\is{Aspect!durative}, and ‘eight days’ is the time elapsed since the beginning of this event. (In both cases, \textit{ka} indicates that a certain moment in time has been reached, see \sectref{sec:4.3.2.2}.)

\ea\label{ex:6.47}
\gll Ka toru mahana \textbf{o} \textbf{te} \textbf{tanu} o Kava, he {\ꞌ}ui e Puakiva ki a Pipi... \\
\textsc{cntg} three day of \textsc{art} bury of Kava \textsc{ntr} ask \textsc{ag} Puakiva to \textsc{prop} Pipi \\

\glt 
‘Three days after (lit. of) the burial of Kava, Puakivi asked Pipi...’ \textstyleExampleref{[R229.358]} 
\z

\ea\label{ex:6.48}
\gll Ka va{\ꞌ}u mahana \textbf{o} \textbf{te} \textbf{noho} o Eugenio o te hāpī {\ꞌ}i Vaihū... \\
\textsc{cntg} eight day of \textsc{art} stay of Eugenio of \textsc{art} teach at Vaihu \\

\glt
‘When he had stayed and taught for eight days in Vaihu...’ \textstyleExampleref{[R231.203]} 
\z

The possessive after the temporal noun may also refer to somebody or something which was involved in a certain event at the time specified. The event itself is expressed as a relative clause\is{Clause!relative} following this noun. \REF{ex:6.49} can be translated literally ‘Three days of the rain which fell’.

\ea\label{ex:6.49}
\gll E toru mahana \textbf{o} \textbf{te} \textbf{{\ꞌ}ua} i hoa ai, ko reherehe atu {\ꞌ}ā te {\ꞌ}ō{\ꞌ}one. \\
\textsc{num} three day of \textsc{art} rain \textsc{pfv} throw \textsc{pvp} \textsc{prf} soft:\textsc{red} away \textsc{cont} \textsc{art} soil \\

\glt 
‘When it had been raining for three days, the ground was quite soft.’ \textstyleExampleref{[R378.040]} 
\z

\ea\label{ex:6.50}
\gll E tahi nō mahana \textbf{o} \textbf{te} \textbf{pahī} \textbf{holandese} \textbf{nei} i noho mai {\ꞌ}i nei  {\ꞌ}i Rapa Nui.\\
\textsc{num} one just day of \textsc{art} ship \ili{Dutch} \textsc{prox} \textsc{pfv} stay hither at \textsc{prox}  at Rapa Nui\\

\glt
‘The \ili{Dutch} ship only stayed one day here on Rapa Nui (lit. Just one day of the \ili{Dutch} ship that stayed).’ \textstyleExampleref{[R373.005]} 
\z

\subparagraph{\ref{sec:6.3.1}.10} Regardless of the semantic relationship, the possessor may express something which does not yet belong to the possessor\is{Possession}, but which the possessor\is{Possession} intends to have:\footnote{\label{fn:291}Cf. \citet{Lichtenberk2002}, who gives examples of “prospective possessive relationships” in several Oceanic languages.}

\ea\label{ex:6.51}
\gll Ki iri tāua ki \textbf{te} \textbf{tāua} māmari vīvī kimi. \\
\textsc{hort} ascend \textsc{1du.incl} to \textsc{art} \textsc{1du.incl} egg partridge search \\

\glt 
‘Let’s go up to look for partridge eggs (lit. to search our partridge eggs).’ \textstyleExampleref{[R245.192]} 
\z

\ea\label{ex:6.52}
\gll Mo pohe ō{\ꞌ}oku mo oho mo hī, he oho au ki \textbf{tā{\ꞌ}aku} \textbf{ika}. \\
if desire \textsc{poss.1sg.o} for go for to\_fish \textsc{ntr} go \textsc{1sg} to \textsc{poss.1sg.a} fish \\

\glt 
‘If I desire to go fishing, I go fishing (lit. to my fish).’ \textstyleExampleref{[R647.061]} 
\z

\ea\label{ex:6.53}
\gll Ko haŋa {\ꞌ}ana a au mo u{\ꞌ}i i \textbf{tā{\ꞌ}aku} \textbf{vi{\ꞌ}e} mo hāipoipo. \\
\textsc{prf} want \textsc{cont} \textsc{prop} \textsc{1sg} for look \textsc{acc} \textsc{poss.1sg.a} woman for marry \\

\glt 
‘I want to find a wife for me (lit. my wife) to marry.’ \textstyleExampleref{[R491.005]} 
\z

\subsection{\textit{A}{}- and \textit{o}{}-possessives}\label{sec:6.3.2}
\is{Possession!o/a distinction|(}
Like most Polynesian languages, Rapa Nui makes a distinction between two types of possessive marking, which are characterised by the vowels \textit{o} and \textit{a}, respectively.\footnote{\label{fn:292}Besides the grammars of individual languages, see \citet[42-44]{Clark1976}; \citet{Capell1931}; \citet{Biggs2000}. On the \textit{o/a} distinction in Rapa Nui, see especially \citet{MulloyRapu1977}.
In Rapa Nui, \textit{{\ꞌ}a} (both as a preposition and at the start of possessive pronouns\is{Pronoun!possessive}) is written with a glottal\is{Glottal plosive}, while \textit{o} is not. The main reason is, that \textit{{\ꞌ}a} happens to occur initially more often than \textit{o}. It is used, for example, in the active-emphatic construction (\sectref{sec:8.6.3}). See sec. \sectref{sec:2.2.5} on the (non\nobreakdash-)occurrence of initial glottals\is{Glottal plosive} in particles. It is not impossible that the glottal-non/glottal distinction in these particles was inherited from an earlier stage. \citet[259]{Clark2000Possessive} points out that in \ili{Tongan}, certain \textit{a}{}-forms have a glottal\is{Glottal plosive}, while the corresponding \textit{o}{}-forms do not (e.g. \textit{\mbox{he{\ꞌ}eku}} ‘my\textsc{.a}’ vs. \textit{hoku} ‘my.O’). See also \citet[48]{Wilson1982}.

On the other hand, the pervasive presence of the glottal\is{Glottal plosive} in sg. possessive pronouns\is{Pronoun!possessive} (\mbox{\textit{tā{\ꞌ}aku}}, \mbox{\textit{{\ꞌ}ā{\ꞌ}aku}}, \mbox{\textit{mō{\ꞌ}oku}}, \textit{tō{\ꞌ}oku} etc.) suggests that originally the glottal\is{Glottal plosive} preceded both \textit{a} and \textit{o} (cf. \citealt[232]{Lynch1997}; \citealt[50]{Wilson1982}).} In most languages this distinction is pervasive, affecting all possessive noun phrases and pronouns. In Rapa Nui, the \textit{o}/\textit{a} distinction is only made with the following nominal elements:\footnote{\label{fn:293}Neutralisation of the \textit{o}/\textit{a}{}-distinction is not uncommon in Polynesian languages. In \ili{Niuean} \citep[34]{Seiter1980}, the distinction is completely lost. The same is true in a group of Outliers: \ili{Nukeria}, \ili{Takuu}, \ili{Nukumanu} and \ili{Luangiua} (\citealt[11]{Wilson1982}; \citealt[267]{Clark2000Possessive}), while in \ili{Rennell}, the distinction is lost in third person pronouns (Nico Daams, p.c.\ia{Daams, Nico}).}

\begin{enumerate}
\item 
Singular \is{Pronoun!possessive}pronouns:\footnote{\label{fn:294}For the forms of possessive pronouns\is{Pronoun!possessive}, see sec. \sectref{sec:4.2.1}.} 
\end{enumerate}

\ea\label{ex:6.54}
\gll
\textbf{tā{\ꞌ}ana} poki; \textbf{tō{\ꞌ}ona} matu{\ꞌ}a; e tahi \textbf{{\ꞌ}ā{\ꞌ}ana} poki; e tahi \textbf{ō{\ꞌ}ona} matu{\ꞌ}a\\
\textsc{poss.3sg.a} child \textsc{poss.3sg.o} parent \textsc{num} one \textsc{poss.3sg.a} child \textsc{num} one \textsc{poss.3sg.o} parent\\

\glt
‘his/her child; his/her parent; one child of his/hers; one parent of his/hers’ 
\z

\begin{enumerate}
\setcounter{enumi}{1}
\item 
Names and other proper nouns\is{Noun!proper}:
\end{enumerate}

\ea\label{ex:6.55}
\gll
te poki \textbf{{\ꞌ}a} Tiare; te matu{\ꞌ}a \textbf{o} Tiare\\
\textsc{art} child of\textsc{.a} Tiare \textsc{art} parent of Tiare\\

\glt
‘Tiare’s child; Tiare’s parent’  
\z

With common nouns\is{Noun!common} and plural pronouns, only \textit{o}{}-forms are used:\footnote{\label{fn:295}A peculiar exception, in which a common noun phrase is an \textit{a}{}-possessor\is{Possession}, is the expression \textit{{\ꞌ}a te hau} ‘Chilean, from the mainland’, as in \textit{va{\ꞌ}ehau \textbf{{\ꞌ}a te hau}} ‘Chilean soldiers’ (R539-1.616). \textit{Te hau} seems to be used as a name here, meaning something like ‘the State’; proper names in Rapa Nui may contain the article \textit{te}.}

\ea\label{ex:6.56}
\gll
te poki \textbf{o} te {\ꞌ}ariki; te matu{\ꞌ}a \textbf{o} te {\ꞌ}ariki\\
\textsc{art} child of \textsc{art} chief \textsc{art} parent of \textsc{art} chief\\

\glt 
‘the chief’s child; the chief’s parent’\textstyleExampleref{} 
\z

\ea\label{ex:6.57}
\gll
tū poki era \textbf{o} \textbf{rāua}\\
\textsc{dem} child \textsc{dist} of \textsc{3pl}\\

\glt
‘that child of theirs’\textstyleExampleref{} 
\z

The two types of possessive constructions will be referred to as \textit{a}{}-possession and \textit{o}{}-possession. The choice between the two can often be predicted from the head noun (the possessee): \textit{matu{\ꞌ}a} is \textit{o}{}-possessed, \textit{poki} is \textit{a}{}-possessed. However, many words can be possessed with either \textit{o} or \textit{{\ꞌ}a}; \citet[43]{Englert1978} gives the following pair of examples (for more examples, see \sectref{sec:6.3.4.1}):

\ea\label{ex:6.58}
\gll He to{\ꞌ}o \textbf{tō{\ꞌ}ona} \textbf{kahu} mo tata. \\
\textsc{ntr} take \textsc{poss.3sg.o} clothes for wash \\

\glt 
‘She took her (own) clothes to wash’.\textstyleExampleref{} 
\z

\ea\label{ex:6.59}
\gll He to{\ꞌ}o \textbf{tā{\ꞌ}ana} \textbf{kahu} mo tata. \\
\textsc{ntr} take \textsc{poss.3sg.a} clothes for wash \\

\glt
‘She took her clothes (the clothes that had been given to her as a laundress) to wash.’\textstyleExampleref{} 
\z

The choice for \textit{{\ꞌ}a-} or \textit{o}{}-possession, then, is not an inherent property of the noun; it is determined by the \textbf{relation} between the possessor\is{Possession} and the possessee, not by the nature of the possessee as such.\footnote{\label{fn:296}See also \citet[151]{Chapin1978}.} If many nouns are always \textit{a-}possessed or always \textit{o}{}-possessed, this is because they always stand in the same relationship to the possessor\is{Possession}. For example, when \textit{poki} ‘child’ is possessed, i.e. ‘A is \textit{poki} of B’, this usually means that A stands in a child-parent relationship to B, a relationship which is expressed by \textit{a-}possession.

The \textit{o/a} distinction does not only affect possessive pronouns\is{Pronoun!possessive} and genitive constituents in the noun phrase (including partitives, see \sectref{sec:6.2.2}), but benefactives as well: the latter are constructed with either \textit{mā} or \textit{mo} when followed by a singular pronoun or proper noun, depending on the nature of the relationship between the two referents involved (\sectref{sec:4.7.7}).

\subsection{Possessive relations marked with \textit{a} and \textit{o}}\label{sec:6.3.3}

In \sectref{sec:6.3.1}, a general overview was given of relationships expressed by possessive constructions. The present section provides a detailed discussion of these relationships, categorised by \textit{a}{}- and \textit{o}{}-marking.

\sectref{sec:6.3.3.1} deals with relationships between people, while \sectref{sec:6.3.3.2} and \sectref{sec:6.3.3.3} discuss relationships involving non-human possessees. \sectref{sec:6.3.3.4} deals with nominalised verbs\is{Verb!nominalised} and their arguments. \sectref{sec:6.3.4} addresses the question whether a general characterisation of \textit{{\ꞌ}a-} and \textit{o}{}-possession is possible.

\subsubsection[Human possessees]{Human possessees}\label{sec:6.3.3.1}

When both possessor\is{Possession} and possessee are human, the situation is relatively straightforward in the case of kinship relations. These will be discussed in \sectref{sec:6.3.3.1.1}. Other interhuman relationships are discussd in \sectref{sec:6.3.3.1.2}.

\paragraph[Kinship relations]{Kinship relations}\label{sec:6.3.3.1.1}
\is{Kinship term|(}
\textit{{\ꞌ}A}-possession is used to express the following kinship relations:

%\setcounter{listWWviiiNumlxxxivleveli}{0}
\begin{enumerate}
\item 
children of the possessor\is{Possession}, including adoptive children: \textit{tā{\ꞌ}au poki/{\ꞌ}atariki/vovo} ‘your child/firstborn/daughter’.

\item 
spouses: \textit{tā{\ꞌ}aku vi{\ꞌ}e} ‘my wife’, \textit{tā{\ꞌ}ana korohu{\ꞌ}a} ‘her old man’.

\end{enumerate}

All other kinship relationships are expressed with \textit{o}{}-possession:

%\setcounter{listWWviiiNumxileveli}{0}
\begin{enumerate}
\item 
parents, including adoptive parents and godparents: \textit{tō{\ꞌ}oku matu{\ꞌ}a/māmā/comadre} ‘my parent\textit{/}Mum\textit{/}godmother’.

\item 
siblings: \textit{tō{\ꞌ}ou ŋā taina} ‘your brothers and/or sisters’.\footnote{\label{fn:297}\citet[22]{MulloyRapu1977} quote one example from Métraux’\ia{Métraux, Alfred} published stories where \textit{taina} is \textit{a}{}-possessed:
\ea
\gll  He tomo Poie ki te motu ananake ko \textbf{tā{\ꞌ}ana} ŋā taina.\\
  \textsc{ntr} go\_ashore Poie to \textsc{art} islet together \textsc{prom} \textsc{poss.3sg.a} \textsc{pl} sibling \\
  \glt 
  ‘Poie landed on the island, together with his brothers.’ (Mtx-3-01.311) \z
According to Mulloy \& Rapu, this suggests that in the past younger brothers were \textit{a}{}-possessed, a situation which was changing to \textit{o}{}-possession in the 1930s, when this story was recorded. However, \textit{tā{\ꞌ}ana} turns out to be a faulty transcription in the printed text: the text in Métraux’\ia{Métraux, Alfred} notebook (notebook 4, p. 170) has the regular \textit{tō{\ꞌ}ona}. 
Note, however, that Mtx’s texts do show some other irregularities in the use of \textit{{\ꞌ}a} and \textit{o} possession, without a clear reason: \textit{te matu{\ꞌ}a {\ꞌ}a Ure} ‘Ure’s father’ (Mtx-7-03.108); \textit{ta{\ꞌ}u ha{\ꞌ}ana} ‘your brother-in-law’ (Mtx-7-30.062); in both cases, \textit{a}{}-possession is used where one would expect \textit{o}.}

\item 
grandparents and grandchildren: \textit{tō{\ꞌ}ona makupuna} ‘his grandchild’; \textit{tō{\ꞌ}oku māmā\-rū{\ꞌ}au} ‘my grandmother’.\\
However, grandchildren may also be \textit{a}{}-possessed, whereby the grandchild is in fact treated in the same way as one’s own child:\footnote{\label{fn:298}As with siblings, \citet[22]{MulloyRapu1977} suggest that a shift has been taking place in the possession class of grandchildren; the text corpus gives no evidence of such a shift, however.}

\end{enumerate}

\ea\label{ex:6.60}
\gll ...e {\ꞌ}a{\ꞌ}amu nō {\ꞌ}ana e \textbf{tā{\ꞌ}ana} \textbf{ŋā} \textbf{makupuna} era... \\
~~~\textsc{ipfv} tell just \textsc{cont} \textsc{ag} \textsc{poss.3sg.a}\textsc{} \textsc{pl} grandchild \textsc{dist} \\

\glt 
‘...her grandchildren told...’ \textstyleExampleref{[R380.007]} 
\z

\begin{enumerate}
\setcounter{enumi}{3}
\item 
further offspring and offspring in general: \textit{tō{\ꞌ}ona hinarere} ‘his great-grandchild’; \textit{tō{\ꞌ}ona hakaara} ‘his descendants’.

\item 
uncles/aunts and nephews/nieces: \textit{tō{\ꞌ}oku pāpātio} ‘my uncle’; \textit{tō{\ꞌ}ou} \textit{sobirino} ‘your nephew’.\\
When nephews/nieces are indicated with \textit{poki} ‘child’, i.e. placed on a par with one’s own children, they are \textit{a}{}-possessed. The following example is said by an uncle to his nephew:

\end{enumerate}

\ea\label{ex:6.61}
\gll ¿He aha \textbf{tā{\ꞌ}aku} \textbf{poki} ka mana{\ꞌ}u rō ki te pāpā? \\
~\textsc{pred} what \textsc{poss.1sg.a} child \textsc{cntg} think \textsc{emph} to \textsc{art} father \\

\glt 
‘Why does my child think of his father?’ \textstyleExampleref{[R230.026]} 
\z

\begin{enumerate}
\setcounter{enumi}{5}
\item 
all in-law relationships: \textit{tō{\ꞌ}ou hunoŋa} ‘your son/daughter-in-law’; \textit{tō{\ꞌ}ou huŋavai} ‘your father/mother-in-law’; \textit{tō{\ꞌ}ou ta{\ꞌ}okete} ‘your brother/sister-in-law’.

\item 
the family as such:

\end{enumerate}

\ea\label{ex:6.62}
\gll He haka ma{\ꞌ}u rā moni ki \textbf{tō{\ꞌ}ona} \textbf{hua{\ꞌ}ai} {\ꞌ}i Harani. \\
\textsc{ntr} \textsc{caus} carry \textsc{dist} money to \textsc{poss.3sg.o} family at France \\

\glt
‘He sent that money to his family in France.’ \textstyleExampleref{[R231.013]} 
\z

\begin{itemize}
\item[]
However, in the sense of a nuclear family (people living together in one house), family may also be \textit{a}{}-possessed:
\end{itemize}
%\todo[inline]{this par. needs to be indented, as sequel of numbered item 7}

\ea\label{ex:6.63}
\gll E noho era a Manutara ananake ko \textbf{tā{\ꞌ}ana} \textbf{hua{\ꞌ}ai}. \\
\textsc{ipfv} stay \textsc{dist} \textsc{prop} Manutara together \textsc{prom} \textsc{poss.3sg.a} family \\

\glt 
‘Manutara lived with his family.’ \textstyleExampleref{[R309.039]} 
\z
\is{Kinship term|)}

\paragraph[Other human relationships]{Other human relationships}\label{sec:6.3.3.1.2}
%\setcounter{listWWviiiNumlxxxviiileveli}{0}
\begin{enumerate}
\item 
Friends, companions or colleagues are \textit{o}{}-possessed: \textit{tō{\ꞌ}oku hoa/hokorua} ‘my friend/ companion’.

\item 
When the possessee is higher in status or authority, or in charge of the possessor\is{Possession}, \textit{o} is used.

\end{enumerate}

\ea\label{ex:6.64}
\gll He e{\ꞌ}a mai he kimi i \textbf{tō{\ꞌ}ona} \textbf{kape}. \\
\textsc{ntr} go\_out hither \textsc{ntr} search \textsc{acc} \textsc{poss.3sg.o} boss \\

\glt
‘He went out and searched for his boss.’ \textstyleExampleref{[R237.008]} 
\z

\begin{enumerate}
\setcounter{enumi}{2}
\item 
When the \textit{possessor}\is{Possession} is higher in status or authority, or in charge of the possessee (e.g. as employer or teacher), \textit{{\ꞌ}a} is used.

\end{enumerate}

\ea\label{ex:6.65}
\gll Te \textbf{ma{\ꞌ}ori} \textbf{aŋa} \textbf{hare} \textbf{{\ꞌ}a} \textbf{Hotu} \textbf{Matu{\ꞌ}a} ko Nuku Kehu tō{\ꞌ}ona {\ꞌ}īŋoa. \\
\textsc{ntr} expert make house of\textsc{.a} Hotu Matu’a \textsc{prom} Nuku Kehu \textsc{poss.3sg.o} name \\

\glt 
‘Hotu Matu’a’s house builder (who was in his service) was called Nuku Kehu.’ \textstyleExampleref{[Ley-2-12.002]}
\z

\ea\label{ex:6.66}
\gll He uŋa ia e Ietū e rua o \textbf{tā{\ꞌ}ana} \textbf{nu{\ꞌ}u} \textbf{hāpī}. \\
\textsc{ntr} send then \textsc{ag} Jesus \textsc{num} two of \textsc{poss.3sg.a} people learn \\

\glt
‘Then Jesus sent out two of his disciples.’ \textstyleExampleref{[Mrk. 14:13]}
\z

\begin{itemize}
\item[]
This also means that \textit{{\ꞌ}a} is used for a group of people over which the possessor\is{Possession} is in charge: 
\end{itemize}

\ea\label{ex:6.67}
\gll Ko arma {\ꞌ}ā a au i \textbf{tā{\ꞌ}aku} \textbf{ekipo} mai i a marzo {\ꞌ}ā. \\
\textsc{ntr} assemble \textsc{cont} \textsc{prop} \textsc{1sg} \textsc{acc} \textsc{poss.1sg.a} group from at \textsc{prop} March \textsc{ident} \\

\glt
‘From March on, I have put together my group.’ \textstyleExampleref{[R625.082]} 
\z

\begin{itemize}
\item[]
On the other hand, for a group of people to which the possessor\is{Possession} belongs, \textit{o} is used. 
\end{itemize}

\ea\label{ex:6.68}
\gll He aŋa tau kope era i te koro kumi, ananake ko \textbf{tō{\ꞌ}ona} \textbf{taŋata} i aŋa ai.\\
\textsc{ntr} make \textsc{dem} person \textsc{dist} \textsc{acc} \textsc{art} feast\_house long together \textsc{prom} \textsc{poss.3sg.o}  person \textsc{pfv} make \textsc{pvp}\\

\glt
‘That man made a large feast house, together with his people he made it.’ \textstyleExampleref{[Mtx-4-03.003]}
\z

\begin{enumerate}
\setcounter{enumi}{3}
\item 
Somewhat unexpectedly, when the possessee is a subordinate, \textit{o} tends to be used: \textit{tō{\ꞌ}oku rarova{\ꞌ}e/tāvini} ‘my subordinate/servant’.

\end{enumerate}
\subsubsection[Non{}-human possessees with {\ꞌ}a]{Non-human possessees with \textit{{\ꞌ}a}}\label{sec:6.3.3.2}

With non-human possessees, \textit{{\ꞌ}a} is used in the following situations:

%\setcounter{listWWviiiNumxxixleveli}{0}
\begin{enumerate}
\item 
The possessee is an instrument handled by the possessor\is{Possession}. This includes a wide variety of objects: tools, bags and other containers, musical instruments, objects used as parts to make something, et cetera.

\end{enumerate}

\ea\label{ex:6.69}
\gll He hoa i \textbf{tā{\ꞌ}ana} \textbf{hau}. \\
\textsc{ntr} throw \textsc{acc} \textsc{poss.3sg.a} cord \\

\glt 
‘He threw out his fishing line.’ \textstyleExampleref{[R338.024]} 
\z

\ea\label{ex:6.70}
\gll {\ꞌ}Ina e ko haha{\ꞌ}o te {\ꞌ}ature ki roto ki \textbf{tā{\ꞌ}ana} \textbf{kete}. \\
\textsc{neg} \textsc{ipfv} \textsc{neg.ipfv} insert \textsc{art} kind\_of\_fish to inside to \textsc{poss.3sg.a} basket \\

\glt
‘He did not put the \textit{ature} fish in his basket’ \textstyleExampleref{[Ley-5-27.011]}
\z
\begin{itemize}
\item[]
This category includes furniture, except furniture supporting the body (see 6c in the next section).
\end{itemize}
%\todo[inline]{this par. needs to be indented}

\begin{enumerate}
\setcounter{enumi}{1}
\item 
The possessee is something produced or caused by the possessor\is{Possession}.

\end{enumerate}

\ea\label{ex:6.71}
\gll Mai hai tiare mo tui o \textbf{tā{\ꞌ}aku} \textbf{karone}. \\
hither \textsc{ins} flower for string of \textsc{poss.1sg.a} necklace \\

\glt 
‘Give me some flowers to make my necklace.’ \textstyleExampleref{[R175.006]} 
\z

\ea\label{ex:6.72}
\gll ...i pāpa{\ꞌ}i ai i \textbf{tā{\ꞌ}ana} \textbf{puka} ra{\ꞌ}e era.\\
~~~\textsc{pfv} write \textsc{pvp} \textsc{acc} \textsc{poss.3sg.a} book first \textsc{dist}\\

\glt
‘(In the year 1948) he wrote his first book.’ \textstyleExampleref{[R539-1.080]}
\z

\begin{enumerate}
\setcounter{enumi}{2}
\item 
The possessee is a dream of the possessor\is{Possession} (‘to dream’ is \textit{moe i te vārua}, lit. ‘lie down a spirit’).

\end{enumerate}

\ea\label{ex:6.73}
\gll Ko moe {\ꞌ}ana au i \textbf{tā{\ꞌ}aku} \textbf{vārua}. \\
\textsc{prf} lie\_down \textsc{cont} \textsc{1sg} \textsc{acc} \textsc{poss.1sg.a} spirit \\

\glt
‘I have had a (lit. my) dream.’ \textstyleExampleref{[R167.045]} 
\z

\begin{itemize}
\item[]
However, dreams can be \textit{o}{}-possessed as well:\footnote{\label{fn:299}The same variability is seen in \ili{Māori}, where \textit{moemoe}\textit{ā} ‘dream’ is \textit{o}{}-possessed for some speakers and \textit{a}{}-possessed for others \citep[170]{Harlow2007Maori}.}
\end{itemize}
%\todo[inline]{this par. needs to be indented}

\ea\label{ex:6.74}
\gll Ka vānaŋa tahi rō i \textbf{to{\ꞌ}u} \textbf{moe} \textbf{vārua.}\\
\textsc{imp} talk all \textsc{emph} \textsc{acc} \textsc{poss.2sg.o} lie spirit\\

\glt
‘Tell your dream completely.’ \textstyleExampleref{[R105.075]} 
\z

\begin{enumerate}
\setcounter{enumi}{3}
\item 
The possessee is land worked by the possessor\is{Possession}.

\end{enumerate}

\ea\label{ex:6.75}
\gll E hakaheu {\ꞌ}ana tū rū{\ꞌ}au era i \textbf{tā{\ꞌ}ana} \textbf{kona} \textbf{{\ꞌ}oka} \textbf{tiare}. \\
\textsc{ipfv} weed \textsc{cont} \textsc{dem} old\_woman \textsc{dist} \textsc{acc} \textsc{poss.3sg.a} place plant flower \\

\glt
‘The old woman was weeding her flower garden.’ \textstyleExampleref{[R301.103]} 
\z

\begin{enumerate}
\setcounter{enumi}{4}
\item 
The possessee is food. This can be food grown, caught or otherwise obtained by the possessor\is{Possession} as in \REF{ex:6.76} and \REF{ex:6.77}, or food/drink consumed – or destined to be consumed – by the possessor\is{Possession} as in \REF{ex:6.78}.

\end{enumerate}

\ea\label{ex:6.76}
\gll He to{\ꞌ}o i \textbf{tā{\ꞌ}ana} \textbf{kūmara} kerikeri era. \\
\textsc{ntr} take \textsc{acc} \textsc{poss.3sg.a} sweet\_potato dig:\textsc{red} \textsc{dist} \\

\glt 
‘He took his sweet potato that he had dug up.’ \textstyleExampleref{[Mtx-7-25.022]}
\z

\ea\label{ex:6.77}
\gll {\ꞌ}Ina kai rava{\ꞌ}a rahi \textbf{tā{\ꞌ}ana} \textbf{ika}. \\
\textsc{neg} \textsc{neg.pfv} obtain much \textsc{poss.3sg.a} fish \\

\glt
‘He did not catch much fish (lit. his fish)’ \textstyleExampleref{[R312.004]} 
\z


\ea\label{ex:6.78}
\gll Ko hiko {\ꞌ}ā \textbf{tā{\ꞌ}aku} \textbf{haraoa} e Te Manu. \\
\textsc{prf} snatch \textsc{cont} \textsc{poss.1sg.a} bread \textsc{ag} Te Manu \\

\glt
‘Te Manu has snatched away my bread.’ \textstyleExampleref{[R245.039]} 
\z

\begin{enumerate}
\setcounter{enumi}{5} 
\item 
The possessee is an animal or plant owned by the possessor\is{Possession}. 

\end{enumerate}

\ea\label{ex:6.79}
\gll He hāŋai i \textbf{tā{\ꞌ}ana} \textbf{oru}. \\
\textsc{ntr} feed \textsc{acc} \textsc{poss.3sg.a} pig \\

\glt 
‘He raised pigs (lit. his pigs).’ \textstyleExampleref{[R423.019]} 
\z

\ea\label{ex:6.80}
\gll He pa{\ꞌ}o mai i \textbf{tā{\ꞌ}ana} \textbf{mahute} i \textbf{tā{\ꞌ}ana} \textbf{hauhau}. \\
\textsc{ntr} chop hither \textsc{acc} \textsc{poss.3sg.a} mulberry \textsc{acc} \textsc{poss.3sg.a} kind\_of\_tree \\

\glt
‘He chopped down his mulberry and \textit{hauhau} trees.’ \textstyleExampleref{[R352.030]} 
\z
\begin{itemize}
\item[]
Horses, however, are \textit{o}{}-possessed, as they are animals of transport (see (\ref{ex:6.86}–\ref{ex:6.87}) in the next section).
\end{itemize}
%\todo[inline]{this par. needs to be indented}

\subsubsection[Non{}-human possessees with o]{Non-human possessees with \textit{o}}\label{sec:6.3.3.3}

With non-human possessees, \textit{o} is used in the following situations:

%\setcounter{listWWviiiNumlxileveli}{0}
\begin{enumerate}
\item 
The possessee is something inherently belonging to the possessor\is{Possession}: \textit{tō{\ꞌ}oku hakari/{\ꞌ}īŋoa/ora/vārua} ‘my body/name/life/spirit’, \textit{tō{\ꞌ}ona matahiti} ‘her years = her age’

\item 
The possessee is a part of the possessor\is{Possession}: \textit{tō{\ꞌ}ona raupā} ‘its leaves (of a tree)’; \textit{tō{\ꞌ}ona taha tai} ‘its coast (of the island)’. This includes body parts: \textit{tō{\ꞌ}oku mata}/\textit{tariŋa}/ \mbox{\textit{pū{\ꞌ}oko}}/\textit{kōkoma} ‘my eye/ear/head/intestine’.

\item 
The possessee is produced naturally by the possessor\is{Possession}. This includes body secretions, eggs of an animal, breathing and the voice: \textit{tō{\ꞌ}ona {\ꞌ}ā{\ꞌ}anu} ‘his saliva’, \textit{tō{\ꞌ}oku matavai} ‘my tears’, \textit{tō{\ꞌ}ona māmari} ‘its eggs (of a hen)’.\\
Young of animals, on the other hand, are \textit{a}{}-possessed (like human children): \textit{\mbox{tā{\ꞌ}ana} mā{\ꞌ}aŋa} ‘its chicks (of a hen)’.\\
Fruits and flowers of plants can be included in this category, although these may also be \textit{o}\nobreakdash-possessed by virtue of being part of a whole (see 2 above).

\end{enumerate}

\ea\label{ex:6.81}
\gll {\ꞌ}E {\ꞌ}i rā kona he tupu te pua, {\ꞌ}e he {\ꞌ}ūa{\ꞌ}a \textbf{tō{\ꞌ}ona} \textbf{tiare}. \\
and at \textsc{dist} place \textsc{ntr} grow \textsc{art} kind\_of\_plant and \textsc{ntr} blossom \textsc{poss.3sg.o} flower \\

\glt
‘And in that place the \textit{pua} grew and its flowers blossomed’ \textstyleExampleref{[R532-07.081]}
\z

\begin{enumerate}
\setcounter{enumi}{3} 
\item 
The possessee is an attribute, a quality or a status of the possessor\is{Possession}: \textit{tō{\ꞌ}ona rivariva/ pūai/māramarama} ‘his/her goodness/\textit{}strength/\textit{}wisdom’; \textit{tō{\ꞌ}ona kōrore/{\ꞌ}eo/tau} ‘its colour/smell/beauty’.

\end{enumerate}

\ea\label{ex:6.82}
\gll {\ꞌ}Ai, ho{\ꞌ}i, tū pū era {\ꞌ}ai, \textbf{tō{\ꞌ}ona} \textbf{raro~nui} {\ꞌ}e \textbf{tō{\ꞌ}ona} \textbf{{\ꞌ}a{\ꞌ}ano}. \\
there indeed \textsc{dem} hole \textsc{dist} there \textsc{poss.3sg.o} deep and \textsc{poss.3sg.o} wide \\

\glt 
‘That there is the hole, its depth and its width.’ \textstyleExampleref{[R620.095]} 
\z

\ea\label{ex:6.83}
\gll ...{\ꞌ}o hakame{\ꞌ}eme{\ꞌ}e mai i \textbf{tō{\ꞌ}oku} \textbf{veve} e Mako{\ꞌ}i. \\
~~~lest mock hither \textsc{acc} \textsc{poss.1sg.o} poor \textsc{ag} Mako’i \\

\glt
‘...so that Mako’i would not mock my poverty.’ \textstyleExampleref{[R214.050]} 
\z

\begin{itemize}
\item[]
This also includes sicknesses: \textit{tō{\ꞌ}ona māuiui/renkē/kokoŋo} ‘his sickness/\textit{}dengue/ cold’.
\end{itemize}

\begin{enumerate}
\setcounter{enumi}{4} 
\item 
The possessee is an attitude or feeling of the possessor\is{Possession}: \textit{\mbox{tō{\ꞌ}oku} heva/koromaki/ma\-mae} ‘my mourning/sadness/pain’; \textit{\mbox{tō{\ꞌ}ou} haŋa/haka {\ꞌ}aroha/māuruuru} ‘your love/ compassion/gratitude’. This includes error and sin: \textit{tō{\ꞌ}oku hape} ‘my fault’, as well as thoughts and opinions: \textit{tu{\ꞌ}u mana{\ꞌ}u} ‘your thought/opinion’.

\item 
The possessee is something containing, covering, supporting, carrying or transporting the possessor\is{Possession}. This includes:

%\setcounter{listWWviiiNumlxilevelii}{0}
\begin{enumerate}
\item 
clothing and footwear worn by the possessor\is{Possession}: \textit{tō{\ꞌ}oku kahu/kamita/\mbox{kiriva{\ꞌ}e}/ kete} ‘my clothes/shirt/shoes/pocket’.\footnote{\label{fn:300}\textit{Kete} means ‘pocket’ in modern Rapa Nui. In the past, \textit{kete} used to mean ‘basket’ and was \textit{a}{}-possessed, like any container.}\\
Clothing is \textit{a-}possessed when it does not refer to clothing to be worn, but functions just as a possession or an object to be handled:

\end{enumerate}
\end{enumerate}

\ea\label{ex:6.84}
\gll He tu{\ꞌ}u a au, he tata i \textbf{tā{\ꞌ}aku} \textbf{kahu}. \\
\textsc{ntr} arrive \textsc{prop} \textsc{1sg} \textsc{ntr} wash \textsc{acc} \textsc{poss.1sg.a} clothes \\

\glt
‘I arrived (at the crater lake) and washed my clothes.’ \textstyleExampleref{[R623.011]} 
\z

\begin{enumerate}
\setcounter{enumi}{5}
\item[]
\begin{enumerate}
\setcounter{enumii}{1}
\item 
other things covering or adorning the body, such as jewellery, eyeglasses, tattoos and body paint: \textit{tō{\ꞌ}ona karone/hei/tāpe{\ꞌ}a/hi{\ꞌ}o} ‘her necklace\textit{/} headdress\textit{/}ring\textit{/}glasses’. Watches, however, are \textit{a}{}-possessed; presumably, they are not classified with jewellery, but with tools and instruments (see 1 in the previous section):
%\todo[inline]{NB number 6 is repeated in front of b, c and d; the same is true for 8b. Can this be avoided? No big issue though.}

\end{enumerate}
\end{enumerate}

\ea\label{ex:6.85}
\gll {\ꞌ}Ina {\ꞌ}ā{\ꞌ}aku hora. \\
\textsc{neg} \textsc{poss.1sg.a} time \\

\glt
‘I don’t have a watch.’ \textstyleExampleref{(\citealt[17]{MulloyRapu1977})} 
\z

\begin{enumerate}
\setcounter{enumi}{5}
\item[]
\begin{enumerate}
\setcounter{enumii}{2}
\item 
objects supporting or containing the body:

\end{enumerate}
\end{enumerate}

\ea\label{ex:6.86}
\gll He ha{\ꞌ}amata he aŋa i \textbf{tō{\ꞌ}ona} \textbf{pē{\ꞌ}ue}. \\
\textsc{ntr} begin \textsc{ntr} make \textsc{acc} \textsc{poss.3sg.o} mat \\

\glt 
‘He began to make his mat.’ \textstyleExampleref{[R344.030]} 
\z

\ea\label{ex:6.87}
\gll te pu{\ꞌ}a e pu{\ꞌ}a era \textbf{te} \textbf{rua} \textbf{o} \textbf{Eugenio} \\
\textsc{art} cover \textsc{ipfv} cover \textsc{dist} \textsc{art} hole of Eugenio \\

\glt
‘the lid that covered Eugenio’s grave’ \textstyleExampleref{[R231.353]} 
\z

\begin{itemize}
\item[]
\begin{itemize}
\item[]
Other furniture is \textit{a-}possessed, like tools and instruments (see 1 in the previous section): \textit{tā{\ꞌ}aku {\ꞌ}amurama{\ꞌ}a} ‘my table’.
\end{itemize}
\end{itemize}

\begin{enumerate}
\setcounter{enumi}{5}
\item[]
\begin{enumerate}
\setcounter{enumii}{3}
\item 
dwelling places: \textit{tō{\ꞌ}ona hare/karapā} ‘his house/tent’.

\item 
buildings and rooms in general: \textit{tō{\ꞌ}ona oficina/piha hāpī/piha moe} ‘her office/classroom/bedroom’. However, buildings not for sheltering humans are \textit{a}{}-possessed: \textit{tā{\ꞌ}aku hare moa} ‘my chicken house’.

\item 
means of transport, including horses: \textit{tō{\ꞌ}ou {\ꞌ}auto/vaka/hoi} ‘your car\textit{/}boat\textit{/} horse’. Other animals are \textit{a}{}-possessed, see 6 in the previous section.\\
In the following example, a banana trunk is used to slide down a hill, i.e. as a means of transport; hence it is \textit{o}{}-possessed, even though plants are normally \textit{a}{}-possessed (6 in the previous section):

\end{enumerate}
\end{enumerate}

\ea\label{ex:6.88}
\gll He eke te kope ra{\ꞌ}e ki ruŋa \textbf{tō{\ꞌ}ona} \textbf{huri}. \\
\textsc{ntr} go\_up \textsc{art} person first to above \textsc{poss.3sg.o} banana\_trunk \\

\glt
‘The first person mounted his banana trunk.’ \textstyleExampleref{[R313.028]} 
\z

\begin{enumerate}
\setcounter{enumi}{6}
\item 
The possessee is the country, territory or place to which the possessor\is{Possession} belongs.

\end{enumerate}

\ea\label{ex:6.89}
\gll Kai hoki hoko{\ꞌ}ou ki \textbf{tō{\ꞌ}ona} \textbf{kāiŋa}, ki Ma{\ꞌ}ori. \\
\textsc{neg.pfv} return again to \textsc{poss.3sg.o} homeland to Ma’ori \\

\glt 
‘He did not return to his homeland Ma’ori anymore.’ \textstyleExampleref{[MsE-005.004]}
\z

\ea\label{ex:6.90}
\gll He oho a {\ꞌ}Orohe ki roto i tō{\ꞌ}ona \textbf{piha} \textbf{hāpī}. \\
\textsc{ntr} go \textsc{prop} Orohe to inside to \textsc{poss.3sg.o} room learn \\

\glt
‘Orohe goes into his classroom.’ \textstyleExampleref{[R334.027]} 
\z

\begin{enumerate}
\setcounter{enumi}{7}
\item 
The possessee is property owned by the possessor\is{Possession}. This includes:

\begin{enumerate}
\item 
land, for example, a plantation or garden:

\end{enumerate}
\end{enumerate}

\ea\label{ex:6.91}
\gll te ŋā {\ꞌ}āua {\ꞌ}oka tarake era \textbf{o} \textbf{Te} \textbf{Mōai} \\
\textsc{art} \textsc{pl} field to\_plant corn \textsc{dist} of Te Moai \\

\glt
‘the corn fields of Te Moai’ \textstyleExampleref{[R539-2.154]}
\z

\begin{itemize}
\item[]
\begin{itemize}
\item[]
This means that fields and gardens can be either \textit{a-} or \textit{o-}possessed, depending on whether the focus is on possession (\textit{o}) or labour (\textit{{\ꞌ}a}); cf. \REF{ex:6.75} in the previous section. 
\end{itemize}
\end{itemize}

\begin{enumerate}
\setcounter{enumi}{7}
\item[]
\begin{enumerate}
\setcounter{enumii}{1}
\item 
money: \textit{tō{\ꞌ}oku moni} ‘my money’.

\item 
property in general: \textit{tō{\ꞌ}ou me{\ꞌ}e} ‘your belongings (lit. things)’; \textit{tō{\ꞌ}ona hauha{\ꞌ}a} ‘his riches, possessions’.

\end{enumerate}
\setcounter{enumi}{8}
\item 
The possessee is an event, and the possessor\is{Possession} is the person concerning whom, with respect to whom, this event happens.

\end{enumerate}

\ea\label{ex:6.92}
\gll He oho te taŋata ta{\ꞌ}ato{\ꞌ}a ki \textbf{tō{\ꞌ}ona} \textbf{pure}. \\
\textsc{ntr} go \textsc{art} man all to \textsc{poss.3sg.o} prayer \\

\glt 
‘All the people went to his (funeral) mass.’ \textstyleExampleref{[R309.141]} 
\z

\ea\label{ex:6.93}
\gll He ma{\ꞌ}u... i te uka ki \textbf{tō{\ꞌ}ona} \textbf{ŋoŋoro}. \\
\textsc{ntr} carry \textsc{acc} \textsc{art} girl to \textsc{poss.3sg.o} feast \\

\glt
‘They carried the bride (lit. girl) ... to her wedding (lit. feast).’ \textstyleExampleref{[R539-3.033]}
\z
\begin{itemize}
\item[]
This includes stories, songs, pictures and other work of art with the possessor\is{Possession} as theme: \textit{tō{\ꞌ}oku {\ꞌ}a{\ꞌ}amu} ‘the story about me’; \textit{te hoho{\ꞌ}a o Tiare} ‘the picture of Tiare, showing Tiare’.
\end{itemize}
%\todo[inline]{this par. needs to be indented}

\begin{enumerate}
\setcounter{enumi}{9}
\item 
The possessor\is{Possession} is a place where the possessee lives, stays, or originates from:

\end{enumerate}

\ea\label{ex:6.94}
\gll He e{\ꞌ}a mai te taŋata \textbf{o} \textbf{{\ꞌ}Ana} \textbf{te} \textbf{Ava} \textbf{Nui}. \\
\textsc{ntr} go\_out hither \textsc{art} man of Ana te Ava Nui \\

\glt 
‘The people of Ana te Ava Nui went out.’ \textstyleExampleref{[Mtx-3-01.283]}
\z

\ea\label{ex:6.95}
\gll Rano Aroi... koia ko \textbf{tō{\ꞌ}ona} \textbf{ŋā{\ꞌ}atu} \\
Rano Aroi \textsc{com} \textsc{prom} \textsc{poss.3sg.o} bulrush \\

\glt
‘Rano Aroi with its bulrush’ \textstyleExampleref{[R112.051]} 
\z

\begin{enumerate}
\setcounter{enumi}{10}
\item 
The possessee is a noun referring to time: \textit{t}\textit{ō{\ꞌ}ona mahan poreko} ‘his birthday’.

\end{enumerate}

\ea\label{ex:6.96}
\gll {\ꞌ}Ina ō{\ꞌ}oku hora. \\
\textsc{neg} \textsc{poss.1sg.o} time \\

\glt
‘I don’t have time.’ \textstyleExampleref{(\citealt[17]{MulloyRapu1977})} 
\z

\begin{enumerate}
\setcounter{enumi}{11}
\item 
The possessor\is{Possession} specifies the reference of the possessee, it is a specific instance of the possessee (epexegetical use).

\end{enumerate}

\ea\label{ex:6.97}
\gll {\ꞌ}i \textbf{te} \textbf{{\ꞌ}āva{\ꞌ}e} \textbf{era} \textbf{o} \textbf{{\ꞌ}Ātete} \\
at \textsc{art} month \textsc{dist} of August \\

\glt 
‘in the month of August’ \textstyleExampleref{[R250.063]} 
\z

\ea\label{ex:6.98}
\gll Te pīkano nei {\ꞌ}i \textbf{te} \textbf{kona} \textbf{era} \textbf{o} \textbf{Roiho}. \\
\textsc{art} eucalyptus \textsc{prox} at \textsc{art} place \textsc{dist} of Roiho \\

\glt
‘These eucalyptus trees are in that place (called) Roiho.’ \textstyleExampleref{[R130.008]} 
\z

\begin{enumerate}
\setcounter{enumi}{12}
\item 
\textit{O-}possessive pronouns\is{Pronoun!possessive} are used in what could be called a distributive\is{Distributive} sense:

\end{enumerate}

\ea\label{ex:6.99}
\gll {\ꞌ}I rā noho iŋa te me{\ꞌ}e ena he pua{\ꞌ}a ka {\ꞌ}aŋahuru {\ꞌ}o ka hānere  atu i \textbf{tō{\ꞌ}ona} \textbf{kope} ka tahi.\\
at \textsc{dist} stay \textsc{nmlz} \textsc{art} thing \textsc{med} \textsc{pred} cow \textsc{cntg} ten or \textsc{cntg} hundred  away at \textsc{poss.3sg.o} person \textsc{cntg} one\\

\glt 
‘In that time each person (lit. his person one) had tens or hundreds of cows.’ \textstyleExampleref{[R107.035]} 
\z

\ea\label{ex:6.100}
\gll E ho{\ꞌ}e {\ꞌ}ahuru mā ho{\ꞌ}e huru kē, huru kē, huru kē  \textbf{tō{\ꞌ}ona} \textbf{puka} {\ꞌ}o \textbf{tō{\ꞌ}ona} \textbf{{\ꞌ}a{\ꞌ}amu}.\\
\textsc{num} on ten plus one manner different manner different manner different  \textsc{poss.3sg.o} book or \textsc{poss.3sg.o} story\\

\glt 
‘There are eleven different books and different stories.’ \textstyleExampleref{[R206.019]} 
\z

\subsubsection[Possession with nominalised verbs]{Possession with nominalised verbs}\label{sec:6.3.3.4}

The arguments of nominalised verbs\is{Verb!nominalised} are often expressed as a possessor (\sectref{sec:8.7}). 

When the possessor is Patient, i.e. undergoes the action, \textit{o}{}-possession is used:

\ea\label{ex:6.101}
\gll He taŋi {\ꞌ}o te \textbf{mate} \textbf{o} \textbf{Huri} \textbf{{\ꞌ}a} \textbf{Vai}. \\
\textsc{ntr} cry because\_of \textsc{art} die of Huri a Vai \\

\glt 
‘He cried because of the death of Huri a Vai.’ \textstyleExampleref{[R304.104]} 
\z

\ea\label{ex:6.102}
\gll E rua matahiti toe mo oti o \textbf{tō{\ꞌ}oku} \textbf{hāpī}. \\
\textsc{num} two year remain for finish of \textsc{poss.1sg.o} learn \\

\glt 
‘There are two years left to finish my schooling.’ \textstyleExampleref{[R399.070]} 
\z

\ea\label{ex:6.103}
\gll {\ꞌ}I te mahana era o \textbf{tō{\ꞌ}ona} \textbf{tanu,} he nehenehe nō.\\
at \textsc{art} day \textsc{dist} of \textsc{poss.3sg.o} bury \textsc{ntr} beautiful just\\

\glt 
‘On the day of his funeral (‘his being buried’), it was beautiful.’ \textstyleExampleref{[R309.140]} 
\z

When the possessee is Agent, i.e. performs the action, the situation is more complicated. Actions as such tend to be \textit{o}{}-possessed:\footnote{\label{fn:301}This is different from the situation in other Polynesian languages, where subjects of transitive\is{Verb!transitive} verbs (and often intransitive\is{Verb!intransitive} agentive verbs as well) tend to be marked with \textit{a}, while objects and non-agentive subjects are marked with \textit{o} (See e.g. \citealt{Chung1973}; \citealt[69]{Clark1981}; \citealt[197–201]{LazardPeltzer2000}; \citealt[173–174]{Cablitz2006}; \citealt[540–541]{MoselHovdhaugen1992}; \citealt[503–505]{Besnier2000}; \citealt[140–142]{ElbertPukui1979}). For \ili{Hawaiian}, \citet{Baker2012} shows that the choice between \textit{a} and \textit{o} for subjects is pragmatically motivated: \textit{a}{}-marked subjects are agentive and/or volitional and/or individuated.}

\ea\label{ex:6.104}
\gll ...{\ꞌ}i \textbf{tō{\ꞌ}oku} \textbf{hiko} mai i te poki mai tu{\ꞌ}u hua{\ꞌ}ai. \\
~~~at \textsc{poss.1sg.o} snatch hither \textsc{acc} \textsc{art} child from \textsc{poss.2sg.o} family \\

\glt 
‘...because I took (lit. in my taking) the child away from your family.’ \textstyleExampleref{[R229.027]} 
\z

\ea\label{ex:6.105}
\gll Ko koa {\ꞌ}ā a au {\ꞌ}i te hora nei {\ꞌ}o \textbf{tō{\ꞌ}ona} \textbf{tute} mai  i a au.\\
\textsc{prf} happy \textsc{cont} \textsc{prop} \textsc{1sg} at \textsc{art} time \textsc{prox} because\_of \textsc{poss.3sg.o} chase hither  \textsc{acc} \textsc{prop} \textsc{1sg}\\

\glt 
‘I am now happy because of his chasing me.’ \textstyleExampleref{[R214.053]} 
\z

\ea\label{ex:6.106}
\gll He {\ꞌ}ui e tū tahutahu era i te tumu o \textbf{tō{\ꞌ}ona} \textbf{tere}. \\
\textsc{ntr} ask \textsc{ag} \textsc{dem} witch \textsc{dist} \textsc{acc} \textsc{art} reason of \textsc{poss.3sg.o} travel \\

\glt
‘The witch asked about the reason for his trip.’ \textstyleExampleref{[R532-07.043]}
\z

When the noun refers to the product or result of an action rather than the action itself, it is \textit{a}{}-possessed:

\ea\label{ex:6.107}
\gll E hakaroŋo rivariva \textbf{tā{\ꞌ}aku} \textbf{hāpī}. \\
\textsc{exh} listen good:\textsc{red} \textsc{poss.1sg.a} teach \\

\glt 
‘Listen well to my teaching.’ \textstyleExampleref{[Luke 8:18]}
\z

\ea\label{ex:6.108}
\gll He koa ia te {\ꞌ}Atua {\ꞌ}i te \textbf{tutia} era \textbf{{\ꞌ}a} \textbf{{\ꞌ}Avere}. \\
\textsc{ntr} happy then \textsc{art} God at \textsc{art} sacrifice \textsc{dist} of\textsc{.a} Abel \\

\glt
‘God was happy with Abel’s sacrifice.’ \textstyleExampleref{[Gen. 4:4]}
\z

The following pair of examples show the contrast between the action as such as in \REF{ex:6.109} and the product of an action as in \REF{ex:6.110}:

\ea\label{ex:6.109}
\gll He riro he taŋata rivariva hai \textbf{{\ꞌ}aiua} \textbf{o} \textbf{Eugenio}. \\
\textsc{ntr} become \textsc{pred} man good:\textsc{red} \textsc{ins} help of Eugenio \\

\glt 
‘He became a good man with Eugenio’s help.’ \textstyleExampleref{[R231.316]} 
\z

\ea\label{ex:6.110}
\gll \textbf{Tā{\ꞌ}ana} \textbf{{\ꞌ}aiua} he pua{\ꞌ}a e tahi. \\
\textsc{poss.3sg.a} help \textsc{pred} cow \textsc{num} one \\

\glt
‘His help/contribution (for the feast) was a cow.’ \textstyleExampleref{[Notes]}
\z

Verbs expressing verbal utterances (‘say’, ‘tell’, ‘sing’) show the same distinction between the product of an action and the action itself. Utterances made by the possessor\is{Possession} – words, stories, songs, et cetera – are \textit{a}{}-possessed, as in (\ref{ex:6.111}–\ref{ex:6.112}). On the other hand, when the act of uttering itself is in focus, the possessor\is{Possession} is \textit{o}{}-marked, as in (\ref{ex:6.113}–\ref{ex:6.114}):

\ea\label{ex:6.111}
\gll I oti era te \textbf{{\ꞌ}a{\ꞌ}amu} \textbf{{\ꞌ}a} \textbf{{\ꞌ}Orohe}... \\
\textsc{pfv} finish \textsc{dist} \textsc{art} story of\textsc{.a} Orohe \\

\glt 
‘When Orohe’s story was finished...’ \textstyleExampleref{[R334.249]} 
\z

\ea\label{ex:6.112}
\gll He katikati i \textbf{tā{\ꞌ}ana} \textbf{hīmene} a Kava. \\
\textsc{ntr} sing \textsc{acc} \textsc{poss.3sg.a} song \textsc{prop} Kava \\

\glt 
‘Kava sang his song.’ \textstyleExampleref{[R229.158]} 
\z

\ea\label{ex:6.113}
\gll Nōatu \textbf{tō{\ꞌ}ona} \textbf{ture} mai. \\
no\_matter \textsc{poss.3sg.o} scold hither \\

\glt 
‘Don’t mind his scolding.’ \textstyleExampleref{[Egt-02.184]}
\z

\ea\label{ex:6.114}
\gll Ko huru kē {\ꞌ}ā {\ꞌ}o \textbf{tō{\ꞌ}oku} \textbf{ta{\ꞌ}e} \textbf{pāhono} i te  vānaŋa {\ꞌ}ui mai.\\
\textsc{prf} manner different \textsc{cont} because\_of \textsc{poss.1sg.o} \textsc{conneg} answer \textsc{acc} \textsc{art}  word ask hither\\

\glt 
‘\is{Verb!nominalised}He feels strange because I didn’t answer (lit. my not answering) his question.’ \textstyleExampleref{[R363.108]} 
\z

Finally, in the actor-emphatic\is{Actor-emphatic construction} construction (\sectref{sec:8.6.3}), Agents are \textit{a}{}-possessed.

\subsection{General discussion}\label{sec:6.3.4}
\subsubsection[Summary]{Summary}\label{sec:6.3.4.1}

The examples in the previous sections show that the choice between \textit{{\ꞌ}a} and \textit{o} depends on the semantic relation between the two referents, not on the actual noun used. A given noun can be \textit{a}{}- or \textit{o}{}-possessed, depending on the relation to the possessor\is{Possession}. \tabref{tab:43} gives a few examples.

\begin{table}
\begin{tabularx}{115mm}{L{32mm}L{45mm}L{30mm}}
\lsptoprule
 & {use with \textit{o}} & {use with \textit{{\ꞌ}a}}\\
\midrule
\textit{māmari} ‘egg’ & egg of a chicken & egg as food\\
{\textit{kahu} ‘clothes’} & clothes worn & clothes handled\\
{\textit{korohu{\ꞌ}a} ‘old man’} & old father, father\nobreakdash-in\nobreakdash-law etc. & old husband\\
{\textit{{\ꞌ}a{\ꞌ}amu} ‘story’} & story about & story by\\
{\textit{karone} ‘necklace’} & necklace worn & necklace made by\\
\lspbottomrule
\end{tabularx}
\caption{Some \textit{a}- and \textit{o}-possessed words}
\label{tab:43}
\end{table}

The fact that the \textit{o}/\textit{a} distinction has a semantic basis, also means that new words (usually \ili{Spanish} borrowings) are integrated into the system on the basis of the semantic relation they bear to their possessors. For example, \textit{kōrore} ‘colour’, \textit{{\ꞌ}auto} ‘car’ and \textit{sobirino} ‘nephew’ are \textit{o}{}-possessed, while \textit{koneta} ‘trumpet’ and \textit{ekipo} ‘group’ are \textit{a}{}-possessed.\footnote{\label{fn:302}See \citet[203]{Makihara2001Adaptation} for more examples.}

In fact, apart from lexical changes, the system shows a remarkable stability over time, as far as the sources show. None of the semantic categories described in the previous sections shows shifts in possessive marking between older texts and modern Rapa Nui. (It is only with younger speakers who master the language imperfectly that the \textit{o}/\textit{a} distinction is starting to break down.)

The findings from \sectref{sec:6.3.3} can be summarised as follows:

\begin{itemize}
\item 
\textit{O}-possession applies to inherent properties, parts, things produced without effort, qualities, attitudes, actions undergone or (sometimes) done, nominalised actions, body covering and transport, countries, land owned, money, subjects of discourse or art, epexegetical constructions, family relations except spouse and children, friendship, persons of higher status, and servants.

\item 
\textit{A}-possession applies to the product of actions, utterances, dreams, land that is worked, instruments, products, food, animals/plants, spouses, children, and persons of lower status.

\end{itemize}

The next section deals with the question whether the \textit{o}/\textit{a} distinction can be explained by a general rule.

\subsubsection{A general rule?}\label{sec:6.3.4.2}

The \textit{o}/\textit{a} distinction occurs more or less along the same lines in almost all Polynesian languages,\footnote{\label{fn:303}There are minor differences between languages. In \ili{Tahitian}, for example, horses are classified as domesticated animals (\textit{a}{}-possessed) rather than means of transport (\textit{o}{}-possessed). Money is \textit{a}{}-possessed, buildings (except dwellings) are \textit{a}{}-possessed. Children are \textit{a}{}-possessed, but young of animals are \textit{o}{}-possessed (\citealt[86–92]{AcadémieTahitienne1986}). In \ili{Māori}, grandchildren are \textit{a-}possessed, and so are servants. Food is \textit{a}{}-possessed, but drinking water is \textit{o}{}-possessed \citep[44]{Biggs1973}.} and it has been described in various ways. 

In general linguistic literature, the distinction between two classes of possession, one of which is more permanent and/or closer to the possessor\is{Possession}, is usually labelled alienable/inalienable, and this terminology is followed by \citet[102]{DuFeu1996}: \textit{o}{}-possession is inalienable, \textit{a}{}-possession is alienable. \citet{PukuiElbert1957} use the same terms for \ili{Hawaiian}. \citet[42]{Englert1978} makes a similar distinction when he states that \textit{o} is used with objects which, in the idea of the speaker, are closer to the possessor\is{Possession}. \citet{Hohepa1967} characterises the distinction as one between inherited and acquired possession. According to \citet[145]{Capell1931}, “\textit{o} forms indicate a passive relation to the possessor\is{Possession}, the \textit{a} forms an active relationship”. \citet[43]{Biggs1973} extends this further: \textit{a} is used “when the possessor\is{Possession} is active, dominant or superior to that which is possessed”; \textit{o} is used “when the possessor\is{Possession} is passive, subordinate or inferior to that which is possessed”. Finally, \citet{MulloyRapu1977} propose a distinction between dependence and responsibility.\footnote{\label{fn:304}This explanation is already suggested – though not accepted – for \ili{East Futunan} by \citet[146]{Capell1931}: “A native explanation of the use of \textit{tiaku} with \textit{tafine}, wife, and \textit{tapakasi}, pig, is that they are ‘objects of special care’!”

Other approaches have been suggested. \citet{Bennardo2000Conceptual,Bennardo2000Possessive} proposes a dichotomy in terms of opposing directionality: for \textit{a}{}-possession the origin is specified, for \textit{o}{}-possession the direction/recipient is specified. Finally, \citet{Elbert1969} refrains from a general characterisation, suggesting that the labels “\textit{o}{}-class” versus “\textit{a}{}-class” may be the easiest for students.} 

What, then, is the most appropriate way to characterise the \textit{{\ꞌ}a/o} distinction in general terms?

First of all, the distinction between \textsc{alienable} and \textsc{inalienable} is not very accurate in describing which items are \textit{o}{}- and \textit{a}{}-possessed. Inalienable possession refers to inherent and/or permanent relationships, such as kinship and part-whole \citep[185]{Dryer2007Noun}. While it is true that the \textit{o}{}-possessive indicates inherent and/or permanent possessions like body and soul, body parts and land, its use is much broader, including categories like attitudes and feelings, clothing, jewellery, means of transport and actions undergone. The alienable/inalienable distinction is therefore inadequate as a general characterisation. The same is true for the distinction between \textsc{inherent} and \textsc{acquired} possession.

The distinction between \textsc{dominant} and \textsc{subordinate} makes a number of correct predictions: some possessors that are dominant with respect to their possessees, are \textit{a}{}-marked, while some possessors that are subordinate with respect to their possessees, are \textit{o}{}-marked. The leader or organiser of a group has a dominant role, while the subjects of a king have a subordinate role. I am dominant with respect to the tools and instruments I handle, the products I make, and the animals and plants I possess.

For other categories, however, this distinction does not work very well. Can a person said to be subordinate with respect to his/her body, voice, feelings and attitudes, or with respect to his/her house, clothing, and vehicle? The subordinate category is inaccurate in certain interhuman relationships as well: spouses are mutually \textit{{\ꞌ}a}{}-marked, yet not mutually dominant; siblings are mutually \textit{o-}marked, yet not mutually subordinate. 

\citet{MulloyRapu1977} suggest an alternative: \textsc{responsibility} versus \textsc{dependence}. A possessor\is{Possession} who is responsible towards the possessee is expressed with \textit{{\ꞌ}a}, a possessor\is{Possession} who is dependent versus the possessor\is{Possession} is expressed with \textit{o}. From the perspective of the possessee, \textit{{\ꞌ}a} is used when it depends on the possessor\is{Possession}, \textit{o} is used when it is responsible for the possessor\is{Possession}.\footnote{\label{fn:305}Cf. also \citet{Thornton1998} for an analysis of the \textit{o}/\textit{a} distinction in \ili{Māori} in cultural terms (“mind set and spirituality”, 381), i.e. in terms of \textit{tapu} (sacredness) and \textit{mana} (power).}

This idea enables us, for example, to explain the use of \textit{{\ꞌ}a} and \textit{o} with respect to interpersonal relationships. A person is responsible with respect to his or her spouse and children, hence \textit{a}{}-possession. A person depends on his or her parents and extended family, hence \textit{o}{}-possession. A child is dependent on its parents, hence \textit{o}{}-possession. A person is responsible for his/her nuclear family \mbox{(\textit{{\ꞌ}a}),} but depends on the wider family as a support system (\textit{o}).

For non-human referents, things which “care for, protect, and shelter the possessor\is{Possession}” (\citealt[23]{MulloyRapu1977}) are \textit{o}{}-possessed, as the possessor\is{Possession} depends on them. On the other hand, possessions which the possessor\is{Possession} cares for, shelters and protects, are \textit{a}{}-possessed. 

However, for other categories the responsibility/dependence dichotomy is less satisfactory. In a certain sense, a person is dependent on inherent attributes like body and soul. It is even conceivable that someone is dependent on qualities like size, beauty and poverty, as these attributes define a person. It is a bit of a stretch, however, to qualify attitudes like love, compassion, error and sin under the heading of dependency. The same applies for actions and events undergone, like ‘problem, punishment, imprisonment’, and even more so for actions performed by the possessor\is{Possession}. Further, can a person said to be dependent on his saliva or tears, or a chicken on its eggs? Categories like these are defined by neither dependency nor responsibility.

The dichotomy of \textsc{active} versus \textsc{passive} is more promising as a general explanation. In many cases when \textit{{\ꞌ}a} is used, the possessor\is{Possession} has an active role towards the possessee. A person is active when performing an act or making an utterance; people are active with respect to the land they work, the instruments they use, the products they make, the animals they care for and the food they eat. They are passive with respect to their spirit, life, age and body parts, with respect to buildings and means of transport (although here passivity is expressed more appropriately as dependence, see above), and with respect to feelings, thoughts, and actions they undergo.

In describing interhuman relationships, the terms “active” and “passive” are somewhat less clear, unless “passive” is explained in terms of dependence or subordinance: a child is “passive” with respect to its parents insofar as it depends on its parents for its needs; a worker is “passive” with respect to his boss, insofar as the latter takes the initiative in telling him what to do. In the same way, “active” in these relationships can be explained in terms of responsibility, being in charge: a king is “active” with respect to his subordinates in the sense that he is responsible of caring for them.

However, like the other dichotomies, the active/passive opposition does not explain why \textit{o} possession applies to actions performed. Nor does it explain well why so many interhuman relationships are mutually \textit{o-}possessed. Biggs’ conclusion seems justified, that “efforts to generalise in terms of a binary opposition have not met with general acceptance. There are always many examples where the opposition doesn’t fit well, if at all.” (\citealt{Biggs2000}) In the next section, a different solution will be proposed.

\subsubsection[o as unmarked possession]{\textit{o} as unmarked possession}\label{sec:6.3.4.3}

\citet[44]{Clark1976} suggests that the relationship between \textit{a} and \textit{o} in Polynesian is not symmetrical: “\textit{*a} [...] indicates a relation of control or authority of the adjunct over the head. The relation indicated by \textit{*o} can best be characterised as covering all relations not included in \textit{a.}” This idea is presented again in \citet{Biggs2000}: \textit{a} marks an active or dominant possessor\is{Possession}; \textit{o} is the unmarked form, used in all other cases. \citet[16]{Wilson1982} characterises \textit{a-}possession as indicating relationships initiated by the possessor\is{Possession}, while \textit{o} is used for everything else.

There are indeed indications that the relation between \textit{a} and \textit{o} in Rapa Nui is not symmetrical. One such indication is the large number of family relationships which are mutually \textit{o}{}-possessed. Concepts like “dependence” do not explain these well. A child depends on its parent, a person depends on his family. But does an uncle depend on his nephew, or a mother-in-law on her daughter-in-law, to warrant the use of \textit{o}? 

Another indication is suggested by those categories of \textit{o}{}-possession not explained by any of the dichotomies discussed above, e.g. \textit{o}{}-possessed actions, time words (‘your birthday’), distributive\is{Distributive} constructions (‘his day’ = ‘a certain day’), and epexegetical constructions (‘the town of Hanga Roa’). 

A third indication is the asymmetry displayed within some categories: people under a leader can be either \textit{{\ꞌ}a} or \textit{o}{}-possessed, while on the other hand the leader is always \textit{o}{}-possessed.

These facts can be explained by stating that \textit{o} is the \textsc{unmarked} possessive marker. \textit{{\ꞌ}A} is used to express that the possessor\is{Possession} has an \textsc{active} role, which includes being in charge, responsible, or dominant with respect to the possessor\is{Possession}; in all other cases, \textit{o} is used. This rule correctly explains why tools and instruments (things to be used) are \textit{a}{}-possessed, just like animals and plants (things to be cared for), while possessions in general are \textit{o}{}-possessed.

It also explains why certain categories normally \textit{a-}possessed may in certain cases take \textit{o}{}-possession: \textit{o-}possession does not imply a passive or dependent possessor\is{Possession}, but only refrains from marking the possessor\is{Possession} as active or dominant. 

Thirdly, this rule explains why \textit{o} is used in constructions where the distinction between active and passive does not play a role, such as distributives\is{Distributive}, epexegetical possessives and time words. In all these cases, \textit{o} is used as the default marker.

Lastly, this rule goes some way to explaining the use of \textit{a} and \textit{o} possession for actions. A possessor\is{Possession} is active with respect the product of his action (e.g. a feast organised, a saying uttered, a teaching performed); on the other hand, it is less clear whether a person can be said to be active with respect to the action as such; and indeed, here Rapa Nui tends to have \textit{o}{}-possession.

\subsubsection[The o/a distinction and the nominal hierarchy]{The \textit{o}/\textit{a} distinction and the nominal hierarchy}\label{sec:6.3.4.4}
\is{Nominal hierarchy|(}
In Rapa Nui there is one more indication that \textit{o} is the unmarked form: as discussed in \sectref{sec:6.3.2}, common noun phrases and plural pronouns are \textit{o}{}-possessors in all contexts, regardless their semantic relationship to the possessee. The marked form \textit{{\ꞌ}a} is used only with a subset of nominal constituents: singular pronouns and proper nouns\is{Noun!proper}. 

This subset coincides with a subset of the “nominal hierarchy”. Certain referents are inherently more likely to function as topics of discourse, or to be agents of a verb, than others. Pronouns are more likely agents than common nouns\is{Noun!common}, human referents are more likely agents than inanimates. This has led linguists to propose a nominal hierarchy – a.k.a. “animacy hierarchy” or “topic-worthiness hierarchy” – along the following lines (see \citealt[150]{Payne1997}; cf. \citealt[413]{Foley2007}):

\ea\label{ex:6.115a}
1{\rmfnm} {\textgreater} 2 {\textgreater} 3 {\textgreater} proper names {\textgreater} humans {\textgreater} non\nobreakdash-human {\textgreater} inanimates \\
\z
\footnotetext{\label{fn:306}The numbers refer to first, second, and third person respectively. The complete hierarchy also includes 1\textsuperscript{st}, 2\textsuperscript{nd} and 3\textsuperscript{rd} person agreement, a category not relevant for Rapa Nui.}
Another distinction cuts partly across the hierarchy above:
\ea\label{ex:6.115b}
  definite {\textgreater} indefinite
\z
Languages may grammaticalise any part of this hierarchy, for example in case marking.\footnote{\label{fn:307}In some languages, only constituents high on this hierarchy get accusative case-marking (i.e. are case-marked when used as Patient), while only elements lower on the hierarchy get ergative\is{Ergativity} case-marking (i.e. are case-marked when used as Agent). (See \citealt{Dixon1994}.)} Rapa Nui has grammaticalised this hierarchy with respect to possessive marking: only pronouns and proper names, which are high on the hierarchy, may take the “active” possessive marking with \textit{{\ꞌ}a}; elements lower on the hierarchy always get the default marking with \textit{o}.

This leaves the question why only singular possessive pronouns\is{Pronoun!possessive} have the option of taking active marking. Why do plural pronouns only get default marking, even though they are higher on the scale than proper names? 

This lack of distinction in the plural cannot be explained from the nominal hierarchy as given above, but may have to do with the behaviour of singular and plural in general. \citet{Dixon1994} observes that languages sometimes have more distinctions in the singular than in the plural. Distinctions that exist in the singular, may be neutralised in the plural. 

This fact itself may have something to do with the nominal hierarchy. Just like proper names are more topic-worthy than common nouns\is{Noun!common}, and definite nouns more topic-wor\-thy than indefinite nouns, it is conceivable that singular referents are more topic-worthy than plural referents. In all cases a highly individuated referent is more topic-worthy than a less individuated one; highly individuated (singular, definite) referents tend to be topics of discourse.

We may therefore tentatively add another dimension which cuts across the nominal hierarchy: 
\ea\label{ex:6.115c}
singular {\textgreater} plural
\z

Under this hypothesis, Rapa Nui makes the \textit{{\ꞌ}a}/\textit{o} distinction for a subset of nominal referents which is high on the nominal hierarchy. Items lower on the hierarchy always take the default \textit{o} marking.
\is{Nominal hierarchy|)}

\section{Conclusions}\label{sec:6.4}

Possessive constructions are widely used. They occur as noun phrase modifiers and as nominal predicates, but may also be used to mark arguments in a verbal clause; the latter happens in the actor-emphatic construction, in clauses introduced by \textit{mo} ‘in order to’, and occasionally in main clauses.

Possessives are united by the use of a possessive preposition; they are distinguished along three parameters:

\begin{itemize}
\item 
the form of this preposition: \textit{o} versus \textit{{\ꞌ}a}; 

\item 
a bare preposition \textit{o/{\ꞌ}a} (Ø-possessives) versus coalescence of the preposition with the article \textit{te} to the forms \textit{to}/\textit{ta} (\textit{t}{}-possessives); 

\item 
pronominal versus full noun phrase possessors.

\end{itemize}

Forms with \textit{to} and \textit{ta} are used when the possessor is in determiner position; in older Rapa Nui, they are also found as possessive clause predicates. In all other contexts, Ø-forms are used.

Possessive constructions express a wide range of semantic relationships, including attributes, parts, verb arguments, and various kinds of associations. They may express prospective possessive relationships, relationships which do not yet hold but are expected to come into being: ‘I am looking for my wife to marry’; ‘let’s search our eggs in the field’.

As in other Polynesian languages, certain relationships are marked with \textit{o}, others with \textit{{\ꞌ}a}. Various proposals havs been made in the past to characterise the \textit{o/a} distinction, but the only way to account for the wide range of \textit{o-}marked relationships is to view \textit{o} as default marker; \textit{{\ꞌ}a} is only used when the possessor is dominant and/or active in relation to the possessee.

The idea that \textit{o} is the default marker is confirmed by the fact that for plural pronouns and common nouns, \textit{o} is the only marker used, while \textit{{\ꞌ}}\textit{a} is limited to singular pronouns and proper nouns. This can be explained by an expanded version of the nominal hierarchy which has been shown to play a role in various grammatical areas cross-linguistically: only nominal constituents high in this hierarchy exhibit the \textit{o/a} distinction.
\is{Possession|)}\is{Possession!o/a distinction|)}
