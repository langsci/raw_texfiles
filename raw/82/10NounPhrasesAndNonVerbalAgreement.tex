\chapter{Noun phrases and non"=verbal agreement}
\label{chap:10}


\section{Noun phrase properties}
\label{sec:10-1}

\subsection{Types of noun phrases}
\label{subsec:10-1-1}


A noun phrase can consist of: 


A pronoun, as \textit{taním} and \textit{ɡa} in (\ref{ex:10-1}).

\begin{exe}
\ex
\label{ex:10-1}
%modified
%modified
\gll \textbf{taním} \textbf{ɡa} na láad-u  \\
\textsc{3pl.erg} anything \textsc{neg} find.\textsc{pfv"=msg} \\
\glt `They didn't find anything.' (A:DRA003)
\end{exe}

A nominalised clause, as the one headed by the Verbal Noun \textit{thainií} in (\ref{ex:10-2}).

\begin{exe}
\ex
\label{ex:10-2}
%modified
\gll \textbf{putríi} \textbf{ǰhaní} \textbf{thainií} zarurí bháanu \\
son-\textsc{gen} marriage do.\textsc{vn} necessity become.\textsc{prs"=msg}  \\
\glt `It is necessary to get one's son married [lit: Making a~son's marriage becomes a~necessity].' (A:MAR009)
\end{exe}

Or a~noun head, with or without preceding modifiers, as can be seen in (\ref{ex:10-3}).

\begin{exe}
\ex
\label{ex:10-3}
%modified
%modified
\gll \textbf{ṣiṣ-íi} \textbf{so} \textbf{tuúš} \textbf{tukṛá} \textbf{mhaás} aṭ-í \textbf{qábur} the aṭ-í tíi wée ḍhanɡóol-u \\
head-\textsc{gen} \textsc{def.msg.nom} some piece flesh bring-\textsc{cv} grave  to bring-\textsc{cv} \textsc{3sg.obl} in bury.\textsc{pfv"=msg}  \\
\glt `[He] brought some piece of the flesh to the grave and buried it there.' (A:BER014)
\end{exe}

It is the latter type that will be the main focus of this chapter. Pronouns (see \chapref{chap:5}), in the sense of pro"=noun phrases, are not treated here, other than very briefly in connection with apposition; and nominalised clauses, as described in \sectref{sec:13-4} and \sectref{sec:13-5}, have the internal syntax of clauses and are thus not examples of noun phrase syntax proper. 


A number of modifiers can precede (and exceptionally follow) a~noun head within the noun phrase, some of them single words, others phrases or even clauses in themselves: adjectives/adjective phrases, genitive phrases, quantifiers/{\allowbreak}quantifier phrases, determiners and relative clauses. As will be clear in the discussion below (\sectref{subsec:10-1-2}) some words can fill more than one such modifier function, and in some cases even extend into adverbial modification. The differentiation between the classes of modifiers is not always clearcut.


It is also possible for a (substantivised) modifier alone to function as the head of a~noun phrase. This is particularly common with adjectives, and to a lesser extent with cardinal numerals. The ability of demonstratives to function as modifiers as well as pro"=NPs is of course another example of the same tendency.


\subsection{Modifiers in noun phrases}
\label{subsec:10-1-2}


\textbf{Adjectives or adjective phrases} are descriptive in nature, often capturing inherent properties or qualities of the entity referred to by the head noun. They may consist of a~single adjective, as \textit{ɡáaḍu} in (\ref{ex:10-4}).

\begin{exe}
\ex
\label{ex:10-4}
%modified
\gll so ba \textbf{ɡáaḍ-u} \textbf{maidóon} \\
\textsc{3msg.nom} \textsc{top} big-\textsc{msg} field \\
\glt `It was a~big field.' (A:JAN030)
\end{exe}

An adjective phrase can also be complex. It can consist of two or more adjectives, symmetrically related to each other, such as \textit{dhríɡi} and \textit{bhakúli} in (\ref{ex:10-5}), both pointing to certain characteristics of the noun they modify. 

\begin{exe}
\ex
\label{ex:10-5}
%modified
\gll se insaan-á \textbf{áa} \textbf{dhríɡ-i} \textbf{bhakúl-i} \textbf{lhaléemi} ɡhin-í \\
\textsc{def} human-\textsc{obl} \textsc{idef} long-\textsc{f} thick-\textsc{f} stick  take-\textsc{cv} \\
\glt `The man took a~long, thick stick{\ldots}' (A:KIN024)
\end{exe}

The adjective head can also itself be modified, usually by a~scalar quantifier, such as \textit{bíiḍu} modifying \textit{mhoóru} and \textit{šišáwu} in (\ref{ex:10-6}), assuming that the noun can have more or less of the quality implied by the modifying adjective phrase. 

\begin{exe}
\ex
\label{ex:10-6}
%modified
%modified
\gll \textbf{bíiḍ-u} \textbf{mhoór-u} \textbf{bíiḍ-u} \textbf{šišáw-u} \textbf{šay} bh-áan-u\\
very-\textsc{msg} sweet-\textsc{msg} very-\textsc{msg} good-\textsc{msg} thing become-\textsc{prs"=msg} \\
\glt `It is a very sweet and a very good thing.' (A:KEE052)
\end{exe}

A repeated adjective, on the other hand, such as \textit{muxtalíf} in (\ref{ex:10-7}), usually signifies a~distributive meaning.\footnote{Although the adjective phrase in the example is predicative, it still illustrates the point.}

\begin{exe}
\ex
\label{ex:10-7}
%modified
\gll tasíi xaaneé \textbf{muxtalíf} \textbf{muxtalíf} yháand-a \\
\textsc{3sg.gen} shelf.\textsc{pl} different different come.\textsc{prs"=mpl} \\
\glt `There are different kinds of shelves.' (A:HOW050)
\end{exe}

Finally, Perfective Participial clauses with the external syntax of APs, can also function as modifiers of a~head noun (see \sectref{subsec:13-6-6}). 


\textbf{Genitive phrases} are very common as modifiers. Although this is also a~matter of specifying quality in some
sense, it is usually a~specification of origin, relatedness or material, as illustrated in
(\ref{ex:10-8})--(\ref{ex:10-11}). They therefore
also share some of the identifying characteristics of demonstratives and relative clauses (see
below). A modifying genitive phrase can also be a~nominalised clause with a~Verbal Noun in the genitive case as its head (see \sectref{subsec:13-6-6}).

\begin{exe}
\ex
\label{ex:10-8}
%modified
\gll tarkaáṇ \textbf{maṇḍaw-íi} \textbf{kráam} široó th-áan-u  \\
carpenter veranda-\textsc{gen} work start do-\textsc{prs"=msg} \\
\glt `The carpenter starts making the veranda [lit: starts the work of the veranda].' (A:HOW072)

\ex
\label{ex:10-9}
%modified
\gll ni xu \textbf{ux-íi} \textbf{rhaíi} hín-i \\
\textsc{3pl.prox.nom} but camel-\textsc{gen} footprint be.\textsc{prs-f}  \\
\glt `These must be the footprints of a~camel.' (A:HUA061)

\ex
\label{ex:10-10}
%modified
\gll \textbf{phaíi} \textbf{báabu} ǰhaamatreé díi xarčá bi dawa-áan-u \\
girl.\textsc{gen} father son.in.law.\textsc{obl} from costs also ask.for-\textsc{prs"=msg} \\
\glt `The girl's father also asks the son"=in"=law to cover the expenses.' (A:MAR032)

\ex
\label{ex:10-11}
%modified
\gll \textbf{díiš-e} \textbf{xálaka} ǰamá bhe \\
village-\textsc{gen} people collected become.\textsc{cv} \\
\glt `The village people used to gather{\ldots}' (B:AVA198)
\end{exe}


\textbf{Quantifiers or quantifier phrases} are as a~modifier category rather heterogeneous and comprise a~few subclasses. One easily distinguishable subclass is cardinal numerals. Used alone, they simply specify quantity, as in (\ref{ex:10-12}). They can also, as seen in (\ref{ex:10-13}), be modified themselves by e.g., \textit{taqriibán} (often pronounced \textit{qariibán}) `about, approximately', a~Perso"=Arabic loan, and thereby become a~bit more relative.

\begin{exe}
\ex
\label{ex:10-12}
%modified
\gll so ta bač bhil-u \textbf{trúu} \textbf{ǰáan-a} ba hiimeel-í híṛ-a  \\
\textsc{3msg.nom} \textsc{cntr} saved become.\textsc{pfv"=msg} three  person-\textsc{pl} \textsc{top} glacier-\textsc{obl} take.away.\textsc{pfv"=mpl} \\
\glt `He was saved, but three persons were taken by the avalanche.' (B:AVA214)

\ex
\label{ex:10-13}
%modified
\gll \textbf{qaribán} \textbf{bhiíš} \textbf{kaal-á} maxadúši  \\
about twenty year-\textsc{pl} before \\
\glt `About twenty years ago...' (A:GHA048)
\end{exe}

While cardinal numerals can be used only to modify count nouns, scalar quantifiers, such as \textit{tuúš} `some' (\ref{ex:10-14}) and \textit{úča} `a few' (\ref{ex:10-15}) are used to quantify mass nouns as well as count nouns in plural. Some of these are also freely used in adverbial constructions, for instance to modify adjectives, adverbs or entire clauses. One particularly frequent multipurpose modifier is \textit{bíiḍu} `much, many, very'.

\ea
\label{ex:10-14}
%modified
\gll \textbf{tuúš} \textbf{čhoót} míi se yaar-íi rumeel-í maǰí ɡhaṇḍ-í wíi-a keé-na ɡal-úum\\
some cheese \textsc{1sg.gen} \textsc{def} friend-\textsc{gen} handkerchief-\textsc{obl}  in tie-\textsc{cv} water-\textsc{obl} why-\textsc{neg} throw-\textsc{1sg} \\
\glt `Why don't I put some cheese in my friend's handkerchief and throw it into the water?' (A:SHY043)

\ex
\label{ex:10-15}
%modified
\gll atshareet-á \textbf{úč-a} \textbf{xálak} de \\
Ashret-\textsc{obl} few-\textsc{mpl} people be.\textsc{pst}\\
\glt `There were few people in Ashret.' (A:JAN001)
\z


Another strategy for quantification, seen in (\ref{ex:10-16}) and (\ref{ex:10-17}), is by means of a partitive noun phrase. It specifies the quantity of the head noun, often itself preceded by or modified by a~cardinal numeral. Typically, but not exclusively, the nouns used in such partitive phrases denote containers or measuring terms of various kinds. In many ways it would make sense to describe higher numerals (such as 20, 100, 1000) as heads of partitive phrases, modified by the cardinal numerals 1--19 to express the numbers 21--39, etc. (see \sectref{sec:6-4}).

\begin{exe}
\ex
\label{ex:10-16}
%modified
\gll \textbf{panǰ} \textbf{phuṭ-í} \textbf{ṣo} \textbf{phuṭ-í} \textbf{kir} dít-u síinta \\
five foot-\textsc{pl} six foot-\textsc{pl} snow fall.\textsc{pfv"=msg} \textsc{condh} \\
\glt `When five or six feet snow had fallen...' (B:AVA198)

\ex
\label{ex:10-17}
%modified
\gll máa=the \textbf{dúu} \textbf{ɡilees-í} \textbf{c̣hiír} da \\
\textsc{1sg}.\textsc{nom}=to two glass-\textsc{pl} milk give.\textsc{imp.sg} \\
\glt `Give me two glasses of milk!' (A:HLE2298)
\end{exe}


\textbf{Determiners} and their use in noun phrases have been described elsewhere (see \sectref{sec:5-2} and \sectref{subsec:5-2-6}). This class includes all words or phrases that have the function of singling out a~referent in contrast to other referents. It primarily identifies the referent of the noun head among a~number of potential referents. Whereas the genitive phrase modifier defines or introduces a~referent, the determiner points out a~particular referent, `this house' in (\ref{ex:10-18}), and `the man', `another day', and `other people' in (\ref{ex:10-19}), whether already defined or not.

\begin{exe}
\ex
\label{ex:10-18}
%modified
\gll \textbf{eení} \textbf{ɡhooṣṭ-á} šíiṭi ma seé hín-u\\
\textsc{prox} house-\textsc{obl} inside \textsc{1sg.nom} fall.asleep.\textsc{cv} be.\textsc{prs"=msg} \\
\glt `I was asleep inside this house.' (A:HUA014-5)

\ex
\label{ex:10-19}
%modified
%modified
%modified
\gll \textbf{se} \textbf{míiš-a} \textbf{dúi} \textbf{dees-á} baačaá wazíir o \textbf{dúi} \textbf{xálak-a} samaṭ-í ilaán thíil-i \\
\textsc{def} man-\textsc{obl} other day-\textsc{obl} king minister and other  people-\textsc{pl} gather-\textsc{cv} announcement do.\textsc{pfv-f}  \\
\glt `Next day the man called the king, the prime minister and other people together and made an~announcement.' (A:UXW060)
\end{exe}

\textbf{Relative clauses} (see \sectref{sec:13-6}) serve a~very similar purpose of identifying the referent, as the `brutes' in (\ref{ex:10-20}), and the `stories' in (\ref{ex:10-21}). For many relative clauses, however, it is unclear to what extent they should be considered part of the noun phrase at all or rather be analysed as entirely paratactic constructions. The latter is especially true of the so"=called co"=relative or relative"=correlative constructions common in IA languages. 


\begin{exe}
\ex
\label{ex:10-20}
%modified
\gll \textbf{tas} \textbf{mheer-í} \textbf{ɡal-í} \textbf{zaalim"=aan-óom} dhút-a pharé ɡúuli bi de ɡíia de \\
\textsc{3sg.acc} kill-\textsc{cv} throw-\textsc{cv} brute-\textsc{pl"=obl} mouth-\textsc{obl} toward  bread also put.\textsc{cv} go.\textsc{pfv.pl} \textsc{pst} \\
\glt `The brutes, who had killed him, had also put bread in his mouth and left.' (A:GHA076-7)

\ex
\label{ex:10-21}
%modified
\gll xalk-íim ṣaawaá \textbf{teér} \textbf{bhíl-a} \textbf{qiseé} thawóol-a\\
people-\textsc{pl.obl} \textsc{manip} passed become.\textsc{pptc"=mpl} story.\textsc{pl} make.do.\textsc{pfv"=mpl}\\
\glt `He had the people tell him what had happened [lit: stories which had passed].' (A:UXW057)
\end{exe}


\subsection{Apposition}
\label{subsec:10-1-3}

Apposition is another phenomenon with seemingly fuzzy borders, and it is not always obvious what is to be analysed as noun phrase syntax and what as noun derivation. The following are a~few different types of noun phrases consisting of a~final noun head and one or more modifying (or further specifying) noun phrases (or what would at least constitute noun phrases if used independently). The head and the modifier have the same referent.


The head can be a~title or a~designation and the preceding apposition is usually a~proper name. Only the head, in this case \textit{hakím} `ruler' in (\ref{ex:10-22}), and \textit{ṣoó} `king' in (\ref{ex:10-23}), is inflected.

\ea
\label{ex:10-22}
%modified
\gll \textbf{ɣeirat"=xaán} \textbf{hakim-í} ɡhooṣṭ-á panaahí dawéel-i \\
Ghairat.Khan ruler-\textsc{gen} house-\textsc{obl} shelter ask.for.\textsc{pfv-f} \\
\glt `He asked for shelter in the ruler Ghairat Khan's house.' (B:ATI025)
\ex
\label{ex:10-23}
%modified
\gll \textbf{šuǰaaulmúlk} \textbf{ṣoo-íi} waxt-íi ba lo daarulxalaafá c̣hatróol-a the ɡúum \\
Shuja.ul.Mulk king-\textsc{gen} time-\textsc{gen} \textsc{top} \textsc{dist.msg.nom}  capital Chitral-\textsc{obl} to go.\textsc{pfv.msg}  \\
\glt `During the reign of king Shuja"=ul"=Mulk the capital was moved to Chitral.' (A:MAH005)
\z

A complex proper name (i.e., many names referring to the same person), as in (\ref{ex:10-24}) or in (\ref{ex:10-25}), functions in much the same way, as its last component is more prominent. Quite often this component is (or has been) a~title of some kind. For short complexes, it is probably a~matter of word formation rather than apposition. The included names constitute a~single phonological word, with the last syllable of the final component receiving the main accent (or the suffix according to shift"=accent rules applied), but the preceding parts are entirely or partly deaccented.

\begin{exe}
\ex
\label{ex:10-24}
%modified
\gll bhiooṛkúi$\sim$ ba \textbf{pir=saahíb} ǰáand-u de\\
Biori.valley \textsc{top} Pir=Lord alive-\textsc{msg} be.\textsc{pst}\\
\glt `In Biori vally Pir Sahib was [still] alive.' (B:ATI047)

\ex
\label{ex:10-25}
%modified
\gll \textbf{mulaa=mhaamad=seed-á} the maalúm heensíl-u hín-u ki\\
Mullah=Mahmad=Said-\textsc{obl} to knowledge stay.\textsc{pfv"=msg} be-\textsc{prs"=msg} \textsc{comp}\\
\glt `Mullah Mahmad Said knew that...' (A:MAH011)
\end{exe}

A special case is when the obligatory string \textit{aleehisalaám} `peace be upon him' (PBUH) is added after the name of a~prophet has been uttered, according to Islamic tradition. Then the final syllable of that standard string receives the accent, as is seen in (\ref{ex:10-26}), and any inflections are attached to the end of the entire phrase.

\begin{exe}
\ex
\label{ex:10-26}
%modified
\gll xu \textbf{eesé} \textbf{waqt-íi} \textbf{peeɣambár} \textbf{hazrát} \textbf{iliaás} \textbf{aleehisalaam-íi} beet-í káaṇ na th-íi de \\
but \textsc{dist.obl} time-\textsc{gen} prophet lord  Elijah peace.be.upon.him-\textsc{gen} 
word-\textsc{pl} \textsc{host} \textsc{neg} do-\textsc{3sg} \textsc{pst} \\
\glt `But he was not listening to Lord Ilyas [Elijah], PBUH, the prophet of the time.' (A:ABO011)
\end{exe}

The head as well as the preceding apposition can also be common nouns. Although the apposition functions as the modifier, and as such takes the place of an~adjective phrase, both nouns together contribute almost equally to the identification or specification of the referent, `female person' and `women' in (\ref{ex:10-27}), and `shepherd' and `boy' in (\ref{ex:10-28}). Each of them can separately and independently function as a~referring noun phrase. 

\ea
\label{ex:10-27}
%modified
%modified
\gll tas sanɡí \textbf{čúur} \textbf{páanǰ} \textbf{ǰéeni} \textbf{kuṛíina} áa ba míiš \textbf{{\ldots}} phray-áan-a\\
\textsc{3sg.acc} with four five female.person woman.\textsc{pl} one \textsc{top} man {} send-\textsc{prs"=mpl}\\
\glt `They send with her four or five women and one man.' (A:MAR082-3)

\ex
\label{ex:10-28}
%modified
\gll se \textbf{áak} \textbf{bakaraál} \textbf{phoó} the ašáx de  \\
\textsc{3fsg.nom} \textsc{idef} shepherd boy to in.love be.\textsc{pst}  \\
\glt `She was in love with a~shepherd boy.' (A:SHY002)
\z

In other cases, the head is a~proper name and the preceding apposition is a~kinship term. In (\ref{ex:10-29}), both the noun head, `Mullah Mahmad Said', and the head of the apposition, `grandfather', take inflections, thus displaying a~higher degree of independence (and symmetry) than the aforementioned types. 

\ea
\label{ex:10-29}
%modified
\gll \textbf{míi} \textbf{se} \textbf{dóod-a} \textbf{mulaa=mhaamad=seed-á} the ba eesó paalawaáṇ maalúm heensíl-u hín-u\\
\textsc{1sg.gen} \textsc{def} grandfather-\textsc{obl} Mullah=Mahmad=Said-\textsc{obl} to  \textsc{top} \textsc{rem.msg.nom} strong.man knowledge  stay.\textsc{pfv"=msg} be.\textsc{prs"=msg}\\
\glt `My grandfather Mullah Mahmad Said knew this strong man.' (A:MAH027)
\z


The head can also be a~common noun and the preceding apposition a~pronoun. The role of the pronoun, `we' in (\ref{ex:10-30}) and `you' in (\ref{ex:10-31}), is not much different from that of a~determiner. 

\begin{exe}
\ex
\label{ex:10-30}
%modified
\gll \textbf{be} \textbf{páanǰ} \textbf{ǰáan-a} ba ɡíia \\
\textsc{1pl.nom} five person-\textsc{pl} \textsc{top} go.\textsc{pfv.pl}  \\
\glt `The five of us left.' (A:GHA007)

\ex
\label{ex:10-31}
%modified
\gll \textbf{tu} \textbf{ateeṇ-ú} \textbf{takṛá} \textbf{íṇc̣-a} díi ma kanáa bhe uḍhíiw"=um? \\
\textsc{2sg.nom} such-\textsc{msg} strong bear-\textsc{obl} from \textsc{1sg.nom}  like.what become.\textsc{cv} flee-\textsc{1sg} \\
\glt `How can I flee from such as strong bear as you?' (A:KAT136)
\end{exe}

\section{Word order in the noun phrase}
\label{sec:10-2}


Although it is possible to use more than two modifiers in the same noun phrase, it is not too common in natural speech, and therefore the following description of the relative word order should be regarded as a~presentation of tendencies more than as hard and fast rules without exceptions. The clearest tendencies can be seen in the order of genitive phrases, determiners, quantifiers and adjective phrases with respect to each other. The following order seems to be more or less fixed: Genitive NP + Determiner + Quantifier + AP + Head. The determiner, \textit{se} in (\ref{ex:10-32}) and \textit{aní} in (\ref{ex:10-33}), precedes the quantifier.

\begin{exe}
\ex
\label{ex:10-32}
\gll se dúu mítr-a \\
\textsc{def} two friend-\textsc{pl} \\
\glt `the two friends' (A:MIT025)

\ex
\label{ex:10-33}
\gll aní páanǰ bhraawú \\
\textsc{prox} five brother.\textsc{pl}  \\
\glt `these five brothers' (A:ASC003)
\end{exe}

The determiner, \textit{se} in (\ref{ex:10-34}) and \textit{ɡóo} in (\ref{ex:10-35}), also precedes the adjective phrase.

\begin{exe}
\ex
\label{ex:10-34}
\gll se búuḍ-i kúṛi \\
\textsc{def} old-\textsc{f} woman \\
\glt `the old woman' (A:WOM462)

\ex
\label{ex:10-35}
\gll ɡóo saxt xálak-a \\
some tough people-\textsc{pl}  \\
\glt `some tough people' (A:KEE043)
\end{exe}


The quantifier, \textit{dúu} in (\ref{ex:10-36}), precedes the adjective phrase \textit{ɡéeḍi}. Although numerals and adjective phrases only rarely co"=occur in the same noun phrase, there is a~strong feeling about the grammaticality of this order vis-à-vis the opposite one.

\begin{exe}
\ex
\label{ex:10-36}
\gll dúu ɡéeḍ-i durbaṭ-í (*ɡéeḍi dúu durbaṭí) \\
two big-\textsc{f} pot-\textsc{pl} \\
\glt `two big pots' (A:HLE2474)
\end{exe}


The genitive NP, regardless of its internal complexity, generally precedes all other modifiers
whenever they occur in the same noun phrase: adjective phrases, as in (\ref{ex:10-37}), quantifiers, as in (\ref{ex:10-38}), indefinite determiners, as in (\ref{ex:10-39}), and definite determiners, as in
(\ref{ex:10-40}).

\begin{exe}
\ex
\label{ex:10-37}
%modified
\gll \textbf{kaṭamuš-íi} lhéṇḍ-i kakaríi \\
Katamosh-\textsc{gen} bald-\textsc{f} skull \\
\glt `the bald scalp of Katamosh' (A:KAT152)

\ex
\label{ex:10-38}
%modified
\gll \textbf{míi} dúu kučúr-a \\
\textsc{1sg.gen} two dog-\textsc{pl}  \\
\glt `my two dogs' (A:HUA017)

\ex
\label{ex:10-39}
%modified
\gll \textbf{díiš-ii} \textbf{yaá} \textbf{teeṇíi} \textbf{qóom-ii} áa ɡhaḍeeró  \\
village-\textsc{gen} or \textsc{refl} tribe-\textsc{gen} \textsc{idef} elder \\
\glt `an elder of the village or of one's own tribe' (A:MAR060)

\ex
\label{ex:10-40}
%modified
\gll \textbf{míi} \textbf{se} \textbf{preṣ-íi} se bhraawú \\
\textsc{1sg.gen} \textsc{def} mother.in.law-\textsc{gen} \textsc{def} brother.\textsc{pl}  \\
\glt `the brothers of that mother"=in"=law of mine' (A:HUA122)
\end{exe}


Another modifier preceding the head of the genitive NP, the definite \textit{se} in (\ref{ex:10-41}) and the indefinite \textit{ak} in (\ref{ex:10-42}), is normally interpreted as part of the genitive NP and as such a~modifier of the genitive noun rather than the head of the main noun phrase.

\begin{exe}
\ex
\label{ex:10-41}
%modified
\gll \textbf{se} \textbf{kuṇaak-íi} paaṇṭí \\
\textsc{def} child-\textsc{gen} clothes  \\
\glt `the/that child's clothes' (A:BER012)

\ex
\label{ex:10-42}
%modified
\gll \textbf{ak} \textbf{táapeṛ-e} ṭék-a \\
\textsc{idef} hill-\textsc{gen} top-\textsc{obl}  \\
\glt `on the top of a~hill' (B:BEL301)
\end{exe}


However, the order between the genitive NP and other modifiers is not entirely fixed, and although the preferred order is genitive NP first, there are a~few cases, such as (\ref{ex:10-43}) and (\ref{ex:10-44}), where the preceding modifier most likely is a~direct modifier of the main noun head, `owner' and `fort' respectively, rather than being a~part of the genitive NP. This interpretation is based partly on agreement features, partly on context, and it opens the construction up to a~certain degree of syntactic ambiguity but seldom with any real risk of semantic misinterpretation.

\begin{exe}
\ex
\label{ex:10-43}
%modified
\gll so \textbf{ɡhooṣṭ-íi} khaamaád \\
\textsc{def.msg.nom} house-\textsc{gen} owner  \\
\glt `the house owner' (A:MIT020)

\ex
\label{ex:10-44}
%modified
\gll áak \textbf{ḍaaku"=aan-óom"=ii} qilaá  \\
\textsc{idef} robber-\textsc{pl"=obl"=gen} fort  \\
\glt `a den of thieves' (A:PIR008)
\end{exe}


Occasionally a~genitive modifier is heavy"=shifted to a~position after the noun head, as \textit{míišii práačamii} in (\ref{ex:10-45}). This may alternatively be interpreted as an~example of afterthought.

\begin{exe}
\ex
\label{ex:10-45}
%modified
\gll áak ɡáaḍ-u haál \textbf{míiš-ii} \textbf{práač-am"=ii} \\
\textsc{idef} big-\textsc{msg} hall man-\textsc{gen} guest-\textsc{pl.obl"=gen}  \\
\glt `a big hall for the man's guests' (A:SMO021)
\end{exe}


It is harder to make any generalisations about the position of relative clauses vis-à-vis other modifiers, partly due to their questionable status as an~integral part of the noun phrase (as already mentioned), and partly because of the lack of any clear evidence as far as co"=occurrence of relative clauses and other modifiers is concerned.


\section{Agreement patterns}
\label{sec:10-3}

There are two main types of agreement within the noun phrase, a~lower"=differentiating determiner agreement and a~higher"=differentiating adjectival agreement. Agreement between a~predicate adjective phrase and the subject noun phrase basically follows the same principles as attributive adjectival agreement. In addition to those patterns, there is an extended partial agreement between a~scalar quantifier used adverbially and the head of a~noun phrase.


\subsection{Determiner agreement}
\label{subsec:10-3-1}


For the most common determiners, there is one form agreeing with a~nominative"=masculine singular head, within each determiner set, and another form used with all other heads, as far as case, number and gender are concerned. This kind of agreement is displayed in \tabref{tab:10-1}.


\begin{table}[ht]
\caption{Determiner agreement (the definite article \textit{so/se})}
\begin{tabularx}{.75\textwidth}{ l Q Q Q }
\lsptoprule
&
\multicolumn{2}{l}{\textbf{Masculine}} &
\textbf{Feminine} \\
&
Singular &
Plural &
\\\midrule
\textsc{nom} &
\textit{so} &
\textit{se} &
\textit{se}\\
\textsc{nnom} &
\textit{se} &
\textit{se} &
\textit{se}\\\lspbottomrule
\end{tabularx}
\label{tab:10-1}
\end{table}


All determiners with more than one form have: a) one that agrees with the nominative"=masculine singular head, ending in an~accented \textit{ó} or \textit{ú}, such as \textit{so} agreeing with the nominative singular masculine head in (\ref{ex:10-46}), and: b) another form ending in an~accented \textit{é} or \textit{í}, for example, \textit{se} agreeing with the non"=masculine head in (\ref{ex:10-47}), \textit{se} agreeing with the non"=nominative head in (\ref{ex:10-48}), and \textit{se} agreeing with the non"=singular head in (\ref{ex:10-49}).

\largerpage

\ea
\label{ex:10-46}
%modified
\gll eesé zanɡal-í áa baṭ-á ǰhulí harí \textbf{so} \textbf{kuṇaák} bheešóol-u\\
\textsc{rem} forest-\textsc{obl} \textsc{idef} stone-\textsc{obl} on take.away-\textsc{cv}  \textsc{def.nom.msg} child[\textsc{nom.msg}] seat.\textsc{pfv"=msg}\\
\glt `In that forest he took the child to a~stone and seated him.' (A:BER005)

\ex
\label{ex:10-47}
%modified
\gll ɡhaḍeerá phed-í laṣ čax kaṭéeri ɡhin-í \textbf{se} \textbf{taáǰ} čhiníl-i \\
elder.\textsc{obl} arrive-\textsc{cv} completely swiftly knife take-\textsc{cv}  \textsc{def} crown[\textsc{nom.fsg}] cut.\textsc{pfv-f}\\
\glt `The older [brother] came, took a~knife, and cut off the crown.' (A:DRA016)
\ex
\label{ex:10-48}
%modified
\gll aǰdahaá katoolíi-a wée ač-aníi sanɡi-eé lhooméea \textbf{se} \textbf{míiš-a} ḍáḍi išaará thíil-u, thanaáu dhrak-é\\
dragon fodder.sack-\textsc{obl} in enter-\textsc{vn} with-\textsc{incl} fox.\textsc{obl} \textsc{def} man\textsc{[msg]}-\textsc{obl} toward hint do.\textsc{pfv"=msg} string pull-\textsc{imp.sg}\\
\glt `Just as the dragon went into the sack, the fox signaled to the man to pull the string.' (B:DRB036)

\ex
\label{ex:10-49}
%modified
\gll \textbf{se} \textbf{xálaka} qalánɡ na d-áa bhaá ba uḍheew-í ɡíia \\
\textsc{def} people[\textsc{nom.mpl}] tax \textsc{neg} give-\textsc{inf} be.able.to.\textsc{cv}  \textsc{top} flee-\textsc{cv} go.\textsc{pfv.mpl}  \\
\glt `The people were not able to pay the taxes, so they left, fleeing.' (A:MAB030)
\z

\subsection{Adjectival agreement}
\label{subsec:10-3-2}


Also, for most cases of adjectival agreement, there is a~unique nominative masculine singular form ending in \textit{u}, but there is an~additional differentiation between masculine and feminine, as shown in \tabref{tab:10-2}.


\begin{table}[ht]
\caption{Adjectival agreement (\textit{dhríɡ-} `tall, long')}

\begin{tabularx}{.75\textwidth}{ l Q Q Q }
\lsptoprule
&
\multicolumn{2}{l}{\textbf{Masculine}} & \textbf{Feminine} \\
&
Singular &
\multicolumn{1}{Q}{Plural} &
\\\midrule
\textsc{nom} &
\textit{dhríɡ-u} &
\textit{dhríɡ-a} &
\textit{dhríɡ-i}\\
\textsc{nnom} &
\textit{dhríɡ-a} &
\textit{dhríɡ-a} &
\textit{dhríɡ-i}
\\\lspbottomrule
\end{tabularx}
\label{tab:10-2}
\end{table}


The great majority of inflecting adjectives occur in the three forms ending \textit{-u, -a} and \textit{-i} (and an~additional but marginally used feminine plural \textit{-im} in predicative sentences, see \sectref{subsec:10-3-3} below). There is thus agreement with the noun head in gender, number and case. Whereas agreement in gender is consistently maintained, compare (\ref{ex:10-50}) with (\ref{ex:10-51}), number agreement (compare (\ref{ex:10-52}) with (\ref{ex:10-53})) and case agreement (compare (\ref{ex:10-54}) with (\ref{ex:10-55})) is mostly neutralised. 

\begin{exe}
\ex
\label{ex:10-50}
%modified
\gll ma ba \textbf{ɡáaḍ-u} \textbf{zuaán} \textbf{míiš} de \\
\textsc{1sg.nom} \textsc{top} grown-\textsc{msg} young man[\textsc{nom}.\textsc{msg}] \textsc{pst} \\
\glt `I was in the prime of my youth.' (A:PAS004)

\ex
\label{ex:10-51}
%modified
\gll eesé dáwur"=ii eeṛé keéṇ \textbf{ɡéeḍ-i} \textbf{keéṇ} de \\
\textsc{dist} age-\textsc{gen} \textsc{rem} cave big-\textsc{f} cave[\textsc{nom}.\textsc{fsg}] be.\textsc{pst} \\
\glt `In those times that cave was a~big cave.' (A:CAV008)

\ex
\label{ex:10-52}
%modified
\gll míi ɡhooṣṭ-á \textbf{lhoók-a} \textbf{lhoók-a} \textbf{maasuumaán} \textbf{kuṇaak-á} hín-a \\
\textsc{1sg.gen} house-\textsc{obl} small-\textsc{mpl} small-\textsc{mpl}  innocent child[\textsc{nom.m}]-\textsc{pl} be.\textsc{prs"=mpl}\\
\glt `There are small innocent children in my house.' (A:KIN017)

\ex
\label{ex:10-53}
%modified
\gll \textbf{se} \textbf{lhoók-a} \textbf{kuṇaak-á} ba siɡréṭ uc̣h-í ba áak dúu tróo kaš the ba \\
\textsc{def} small-\textsc{obl} child[\textsc{msg}]-\textsc{obl} \textsc{top} cigarette lift-\textsc{cv} \textsc{top } one two three drag do.\textsc{cv} \textsc{top}  \\
\glt `The little child lifted the cigarette and started smoking.' (A:SMO007)

\ex
\label{ex:10-54}
%modified
\gll \textbf{puróoṇ-a} \textbf{xálak} asíi díiš-a wée hín-a \\
old-\textsc{mpl} people[\textsc{nom.mpl}] \textsc{1pl.gen} village-\textsc{obl} in be.\textsc{prs"=mpl} \\
\glt `There are old people in our village.' (A:MAR127)

\ex
\label{ex:10-55}
%modified
\gll \textbf{puróoṇ-a} \textbf{xalkíim} the patá \\
old-\textsc{obl} people[\textsc{m}].\textsc{pl.obl} to known \\
\glt `The old people know [lit: It is known to old people].' (A:MAR126)
\end{exe}


However, as was pointed out in \chapref{chap:6}, there is a~category of invariable (non"=inflecting) adjectives, not showing any kind of overt agreement at all with the head of the noun phrase, such as \textit{muxtalíf} in (\ref{ex:10-56}) and \textit{taaqatwár} in (\ref{ex:10-57}). It should be noted that although such adjectives do occur attributively, they are more readily used predicatively and some of them exclusively so.

\begin{exe}
\ex
\label{ex:10-56}
%modified
\gll aalmaaríi bi \textbf{muxtalíf} \textbf{ḍizeen-í} yhéend-i \\
cupboard.\textsc{gen} also different design-\textsc{pl} come.\textsc{prs-f} \\
\glt `Cupboards come in many different designs.' (A:HOW049)

\ex
\label{ex:10-57}
%modified
\gll insaán xu \textbf{bíiḍ-u} \textbf{taaqatwár} \textbf{šay} \\
human.being but much-\textsc{msg} powerful thing \\
\glt `Man, however, is a~very powerful being.' (A:KIN006)
\end{exe}


\subsection{Predicate agreement}
\label{subsec:10-3-3}

Just as adjectival attributes agree with their heads, the heads of predicate phrases agree in gender and number with the head of the subject noun phrase, as displayed in \tabref{tab:10-3}. That is in as far as the particular adjective belongs to the inflecting class. It should be noted, however, that the large majority of adjectives used in the predicative function are of the invariable type. 


\begin{table}[ht]
\caption{Predicate agreement}
\begin{tabularx}{.75\textwidth}{ l Q Q Q Q }
\lsptoprule
& \multicolumn{2}{l}{\textbf{Masculine}} & \multicolumn{2}{l}{\textbf{Feminine}} \\
&
Singular &
\multicolumn{1}{ l}{Plural} &
Singular &
Plural\\\midrule
\textsc{nom} &
\textit{-u} &
\textit{-a} &
\textit{-i} &
\textit{-i/-im}\\
\textsc{nnom} &
\textit{-a} &
\textit{-a} &
\textit{-i} &
\textit{-i/-im}
\\\lspbottomrule
\end{tabularx}
\label{tab:10-3}
\end{table}


As with attributively used adjectives, there is a~non"=optional inflectional contrast between adjectives agreeing with a~masculine"=singular head, as in (\ref{ex:10-58}), and adjectives agreeing with a~masculine"=plural head, as in (\ref{ex:10-59}).

\begin{exe}
\ex
\label{ex:10-58}
\gll {\ob}\textbf{so}{\cb} {\ob}\textbf{bíiḍ-u} \textbf{trók-u}{\cb} de \\
\textsc{3msg.nom} much-\textsc{msg} thin-\textsc{msg} be.\textsc{pst} \\
\glt `He was very thin.' (A:KAT003)

\ex
\label{ex:10-59}
\gll aksár {\ob}\textbf{ǰandeé}{\cb} {\ob}\textbf{trók-a}{\cb} bh-áan-a \\
often he.goat[\textsc{m}].\textsc{pl} thin-\textsc{mpl} become.\textsc{prs"=mpl} \\
\glt `Often the he"=goats become thin.' (A:MAR069)
\end{exe}


A secondary (optional) feminine singular"=plural inflectional differentiation seems to be permitted in at least copula"=less sentences, as can be seen when comparing the adjectives in (\ref{ex:10-60}) and (\ref{ex:10-61}).\footnote{The status of agreement with \textit{-im} requires further investigation. The form as relating to plurality has probably originated (relatively recently) in the class of feminine nouns (see \chapref{chap:4}) ending in \textit{-i}, and has subsequently spread analogically to the verbal and adjectival paradigms, although not yet consistently applied in the latter two.}

\begin{exe}
\ex
\label{ex:10-60}
\gll {\ob}\textbf{aní} \textbf{kaṭéeri}{\cb} {\ob}\textbf{búk-i}{\cb} máa=the dúi da \\
\textsc{prox} knife[\textsc{f}] dull-\textsc{f} \textsc{1sg.nom=}to another give.\textsc{imp.sg} \\
\glt `This knife is dull. Give me another one!' (A:Q9.0160)

\ex
\label{ex:10-61}
\gll {\ob}\textbf{ṭíinčuk"=am-i} \textbf{laméeṭi-m}{\cb} {\ob}\textbf{tíiṇ-im}{\cb} \\ 
scorpion-\textsc{obl"=gen} tail[\textsc{f}]-\textsc{pl } sharp-\textsc{fpl} \\
\glt `The tails of scorpions are sharp.' (A:PHS2118.06)
\end{exe}

Apart from a~larger variety of adjectives and adjective phrases allowed predicatively rather than attributively, even quantifiers may occur in a~predicative function. Quantifiers with the ability to inflect agree with the subject noun phrase in gender/number, as can be seen with \textit{bíiḍ-} `much, many' in (\ref{ex:10-62}) and (\ref{ex:10-63}), and \textit{biǰóol-} `several' in (\ref{ex:10-64}).

\begin{exe}
\ex
\label{ex:10-62}
\gll {\ob}\textbf{eeṛé} \textbf{šay-á}{\cb} {\ob}\textbf{bíiḍ-a}{\cb} \\
\textsc{dist} thing[\textsc{m}]-\textsc{pl} much-\textsc{mpl} \\
\glt `There are lots of those things.' (A:HUA047)

\ex
\label{ex:10-63}
\gll {\ob}\textbf{tasíi} \textbf{duṣmaán}{\cb} {\ob}\textbf{bíiḍ-u}{\cb} \\
\textsc{3sg.gen} enemy[\textsc{msg}] much-\textsc{msg} \\
\glt `She has many enemies.' (A:KEE008)

\ex
\label{ex:10-64}
\gll atshareet-á wée {\ob}\textbf{xálak}{\cb} {\ob}\textbf{biǰóol-a}{\cb} bhíl-a \\
Ashret-\textsc{obl} in people[\textsc{mpl}] several-\textsc{mpl} become.\textsc{pfv"=mpl} \\
\glt `In Ashret people became numerous.' (A:GHA001)
\end{exe}

\subsection{Extended agreement}
\label{subsec:10-3-4}


The adjectival agreement is also extended or copied to inflecting adjectives used as adjuncts of other adjectives (primarily \textit{bíiḍ-} `much'), i.e., as adverbs. That means that the gender and number of the noun that the adjective head agrees with (whether attributively or predicatively) is also lended in agreement to the adjective adjunct, this regardless of the ability of the adjective head itself to inflect (compare with the agreement patterns of adverbs in Gujarati, \citealt{hookjoshi1991}). This is seen in the forms of \textit{bíiḍ-} in (\ref{ex:10-65})--(\ref{ex:10-69}) and \textit{šóo} in (\ref{ex:10-70}).

\ea
\label{ex:10-65}
\gll {\ob}\textbf{čhéel"=ii} \textbf{phaaidá}{\cb} bi {\ob}\textbf{bíiḍ-u} \textbf{ɡáaḍ-u}{\cb} {\ob}\textbf{kaṛaáu}{\cb} bi {\ob}\textbf{bíiḍ-u} \textbf{ziaát}{\cb} \\
she.goat-\textsc{gen} benefit[\textsc{msg}] also much-\textsc{msg} big-\textsc{msg}  effort[\textsc{msg}] also much-\textsc{msg} great \\
\glt `There are many benefits of the goat, but they also require a lot of work.' (A:KEE078)

\ex
\label{ex:10-66}
\gll {\ob}\textbf{lasíi} \textbf{phaaideé}{\cb} {\ob}\textbf{bíiḍ-a} \textbf{ziaát}{\cb} hín-a \\
\textsc{3sg}.\textsc{dist.gen} benefit[\textsc{m}].\textsc{pl} much-\textsc{mpl} great  be.\textsc{prs"=mpl} \\
\glt `She provides many benefits.' (A:KEE021)

\ex
\label{ex:10-67}
\gll {\ob}\textbf{tasíi} \textbf{yéei}{\cb} /{\ldots}/ {\ob}\textbf{bíiḍ-i} \textbf{xafá}{\cb} bhíl-i hín-i\\
\textsc{3sg.gen} mother[\textsc{fsg}] {} much-\textsc{f} upset become.\textsc{pfv-f} be.\textsc{prs"=f} \\
\glt `His mother became very upset.' (A:KAT004)

\ex
\label{ex:10-68}
\gll {\ob}\textbf{kaṭamúš}{\cb} /{\ldots}/ {\ob}\textbf{bíiḍ-u} \textbf{xušaán}{\cb} bhíl-u hín-u \\
Katamosh[\textsc{msg}] {} much-\textsc{msg} happy become.\textsc{pfv"=msg} be.\textsc{prs"=msg}\\
\glt `Katamosh became very happy.' (A:KAT078)

\ex
\label{ex:10-69}
\gll {\ob}\textbf{se}{\cb} heewand-á {\ob}\textbf{bíiḍ-a} \textbf{xušaán}{\cb} hóons"=an de \\
\textsc{3pl.nom} winter-\textsc{obl} much-\textsc{mpl} happy stay-\textsc{3pl} \textsc{pst} \\
\glt `They were very happy during the winter.' (A:SHY008)

\ex
\label{ex:10-70}
\gll {\ob}\textbf{kaṭamúš}{\cb} {\ob}\textbf{šóo} \textbf{čaáx}{\cb} bhíl-u hín-u \\
Katamosh[\textsc{msg}] good.\textsc{msg} fat become.\textsc{pfv"=msg}  be.\textsc{prs"=msg} \\
\glt `Katamosh became really fat.' (A:KAT082)
\end{exe}

This is also the case with adjuncts in adverbial phrases used predicatively in which the adjunct agrees in gender and number with the noun head of the subject. Again, it is primarily the scalar modifier \textit{bíiḍ-} `much' that is being used. In (\ref{ex:10-71}), \textit{bíiḍ-} agrees in feminine gender with \textit{iskuúl} `school', and in (\ref{ex:10-72}) it agrees in masculine singular with \textit{ɡhoóṣṭ} `house'. 

\begin{exe}
\ex
\label{ex:10-71}
\gll {\ob}\textbf{asíi} \textbf{iskuúl}{\cb} bi asaám the {\ob}\textbf{bíiḍ-i} \textbf{dhúura}{\cb} hín-i \\
\textsc{1pl.gen} school[\textsc{fsg}] also \textsc{1pl.acc} to much-\textsc{f} distant be.\textsc{prs-f}  \\
\glt `Our school is also very far away for us.' (A:OUR016)

\ex
\label{ex:10-72}
\gll {\ob}\textbf{míi} \textbf{ɡhoóṣṭ}{\cb} {\ob}\textbf{bíiḍ-u} \textbf{dhúura}{\cb} hín-u \\
\textsc{1sg.gen} house[\textsc{msg}] much-\textsc{msg} distant be.\textsc{prs"=msg}  \\
\glt `My house is very far away.' (A:DHE3174)
\end{exe}

A form of incomplete agreement pattern is seen between an~argument in a~nominalised complement clause and the adjunct of the complement"=taking adjective. In example (\ref{ex:10-73}), the adjunct \textit{bíiḍ-} agrees with the feminine direct object of the nominalised verb \textit{xat} `letter', whereas in (\ref{ex:10-74}) and (\ref{ex:10-75}) the choice of the masculine singular seems to be due to its default value rather than one triggered by agreement with any particular argument. This certainly deserves more in"=depth study.\footnote{Note that \textit{askóon} `easy' is a~non"=inflecting adjective.}

\begin{exe}
\ex
\label{ex:10-73}
\gll {\ob}\textbf{urdú} \textbf{xat}{\cb} čooṇṭainií {\ob}\textbf{bíiḍ-i} \textbf{askóon}{\cb} \\
Urdu letter[\textsc{fsg}] write.\textsc{vn} much-\textsc{f} easy  \\
\glt `It's very easy to write a~letter in Urdu.' (A:HLE3131)

\ex
\label{ex:10-74}
\gll lab utrap"=ainií {\ob}\textbf{bíiḍ-u} \textbf{askóon}{\cb} \\
fast run-\textsc{vn} much-\textsc{msg} easy \\
\glt `It's very easy to run fast.' (A:HLE3130)

\ex
\label{ex:10-75}
\gll kuṇaak-á saatainií {\ob}\textbf{bíiḍ-u} \textbf{askóon}{\cb} \\
children-\textsc{pl} take.care.\textsc{vn} much-\textsc{msg} easy \\
\glt `It's very easy to take care of children.' (A:HLE3129)
\end{exe}

Adjectival agreement of this kind may be extended even to clause"=level adverbial modification, as can be seen in examples (\ref{ex:10-76}) and (\ref{ex:10-77}).

\begin{exe}
\ex
\label{ex:10-76}
%modified
\gll aṛó \textbf{bíiḍ-u} bhakulíil-u hín-u\\
 \textsc{dist.msg.nom} much-\textsc{msg} fatten.\textsc{pfv"=msg} be.\textsc{prs-f} \\
\glt `He has fattened a~lot.' (A:DHE3162)

\ex
\label{ex:10-77}
%modified
\gll šumaalí húuši \textbf{bíiḍ-i} ziaát teéz bhe nikhéet-i\\
northern wind[\textsc{fsg}] much-\textsc{f} excessively strong become.\textsc{cv} come.out.\textsc{pfv"=f}\\
\glt `The North Wind blew as hard as she could.' (A:NOR005)
\end{exe}