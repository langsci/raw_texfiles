\chapter{{Conclusions and implications}}\label{sec:8}

This final chapter will consider the findings of the preceding chapters on sailors’ demography, speech communities and linguistic features to respond to the central claim that there is a distinct Ship English that was spoken by British sailors in the early Atlantic \isi{colonial period}. The first section, Conclusions, will clarify my own position on the distinctiveness, stability and spread of Ship English and address the potential typology of the variety. The second section, Implications, will consider the newly presented baseline data on the linguistic features of Ship English in terms of how this might be integrated into theories and research in dialectology and contact linguistics. Throughout this chapter, I acknowledge the scope of the work that still needs to be done and attempt to clarify some specific areas of study that may hopefully motivate future studies.

\section{{Conclusions}}%8.1

\subsection{{A distinct and stable variety}}%8.1.1

The central claim of this book is that there is a distinct “Ship English” that was spoken by British sailors in the early colonial context (determined as roughly the period between 1620 and 1750). This claim is substantiated by two chapters dedicated to the socio-historical context of the variety which identify speakers’ demographics (\chapref{sec:3}) and the characteristics of the \isi{speech community} (\chapref{sec:4}) in addition to three chapters dedicated to linguistic findings which identify salient and repeated markers of sailors’ speech during this period in terms of \isi{noun} phrases (\chapref{sec:5}), \isi{verb} phrases (\chapref{sec:6}), and clause, sentence, and \isi{discourse level} phenomena (\chapref{sec:7}). The socio-historical data demonstrate that Ship English of the early Atlantic \isi{colonial period} developed within a unique maritime demographic and socio-linguistic context which promoted the development of new forms and enabled \isi{feature transfer} via oral traditions. The linguistic data demonstrate that Ship English has distinctive charactersitics with regard to morphosyntactic, syntactic and \isi{discourse level} variation.

It is pertinent at this point to stress that my claim throughout this book is not that Ship English is a \isi{dialect} with \textit{unique characteristics} but rather with unique patterns of frequency and distribution for particular features that marked speakers as members of an extended maritime community. As such, the characteristic features of Ship English here identified are also documented in other varieties of Early Modern English not associated with the maritime community and found in the Early Modern English of authors like Milton, Bunyan, Dryden, Evelyn and Pepys. However, the patterns of frequency in their work are unlikely to correspond consistently with Ship English (unless the author’s intention was to represent \isi{maritime speech}). Dialectologist and specialist in the creation of World Englishes, Schneider, explains that, “frequency shifts are a core property of [the] diffusion process” and that it is not the creation of new forms that indicate the emergence of new World Englishes but changed frequencies that become strengthened by feedback loops and lead to quantitative difference \citep{Schneider2018}. Hence, this book presents morphosyntactic features that were salient among sailors and that were likely to have shifted to a higher frequency of distribution than other contemporary dialects. Thus, when these features are considered together, they mark the emergence of a recognizably distinct variety. 

The facts that there are recognizable patterns of frequency among sailors, and that a high distribution of certain morphosyntactic features mark the variety as distinctive to other contemporary varieties (in ways that can be recognized and reproduced), appear to support Bailey and Ross’s claim that there is a distinct type of English that was spoken by sailors during the period of early English \isi{colonial expansion} (1988: 194) upon which my own central hypothesis is based. As such, the data and discussion presented in the linguistic chapters aim to support the central hypothesis of a distinct maritime \isi{dialect} by identifying the features with the highest frequency that served as salient markers of Ship English, not to document all the features that either do or do not occur in the variety nor to present a comparative analysis between feature frequencies in Ship English and counterparts in other contemporary varieties. Yet, now that descriptive work is available on the most salient features of Ship English, albeit by no means definitive or necessarily complete, future studies might focus on comparative approaches. Hoewever, such work is beyond the scope of this book. 

\largerpage[-1]
This study builds on and extends the few prior studies into variation in sailors’ speech and here briefly re-presents this literature to substantiate the claim that Ship English was a comprehensive variety. The unique lexical characteristics of Ship English have been amply documented in the many dictionaries and word\-lists published since Captain John Smith’s \textit{Sea Grammar} in 1627; indeed, most work on the subject has comprised lexical items that constitute the \isi{professional jargon} of the \isi{crew} intended for circulation among maritime professionals (see \chapref{sec:2}: Review of the Literature). Investigations that provide evidence of the unique phonological and morphological characteristics of Ship English are \citeapo{Matthews1935} monograph on sailors’ pronunciation in the second half of the \isi{seventeenth century} and \citegen{BaileyRoss1988} article on the morphosyntactic features of Ship English.\footnote{It should be recognized that Matthews states that his work “should be regarded as a cross-section in the history of pronunciation…[and] It is not pretended that it describes the ‘\isi{seaman}’s \isi{dialect}’ of the period” (1935: 196, see full citation in \sectref{sec:2.1.2}). Hence, although I discuss Matthews’ work as one of the two previously published studies on Ship English, it is important to note that he himself did not present his research as a study on any distinct “\isi{seaman}’s \isi{dialect}”.}  This book now offers the first extended compilation of evidence on the sociolinguistic, syntactic and \isi{discourse level} features that characterize the variety.\footnote{Given the limited scope of research to date, I do not propose that this study is definitive or that it will not be subject to subsequent revision, modification, and potential correction as we learn more about Ship English. Yet, the baseline data provided can serve as an entry point for scholars to integrate this variety into their work on language history and change; it will also hopefully prompt continued research.}  Hence, we now have evidence that Ship English of the early English \isi{colonial period} around the Atlantic had distinctive lexical, morphological, syntactic, and discourse features that support its status as a distinct historical variety of English. 

The evidence supports the view that Ship English was stable and diffused in ways comparable to other varieties of English. Sea shanties record language features of Ship English over time and sea-music historian Stan Hugill explains that this demonstrates that the variety had stability. His analysis of two \isi{shanty} verses published in 1549 concludes: “the form and language of these early shanties, apart from the fact that the English is Chaucerian, are very much like what our sailors of the sail sang three hundred years later” (\citealt{Hugill1969}: 3). Hugill’s reference to “Chaucerian” English is most probably an allusion to archaic vocabulary but the “form and language” that he refers to, described as very much like the speech of sailors three hundred years later, shows that Ship English was established enough to be recognized as a distinct variety for three centuries.\footnote{It is possible that the stability Hugill observes from the sixteenth to the nineteenth centuries extends even further back with common origins in Mediterranean Lingua Franca (or Sabir) reportedly used since the thirteenth century among sailors traversing the Mediterranean basin (\citealt{Parkvall2005}).}  It is possible that the “language” Hugill refers to incorporates (and potentially derives from) the distinct jargon of the maritime profession, yet his reference to a distinctive “form” suggests a variety with syntactic and discourse-level variation that was stable enough to spread and be identified over time. As discussed in \chapref{sec:3}, this stability was, in part, motivated by the members of a \isi{speech community} who worked in real and pseudo-kinship groups that regularly included partners, wives and children whose communication was predominantly oral owing, in part, to the high levels of illiteracy among sailors. Shipboard oral cultures were furthermore closely connected and expressed a common identity through oral and performance mediums such as storytelling, music, gaming, and dramatic play as examined in \chapref{sec:4}. Owing to the composition of these speech communities, the language features that marked the variety as distinctive, once established, were likely able to spread quickly among those within the community and they were additionally reinforced internally by the oral speech practices and strong affiliations among groups of sailors.\footnote{It is possible that there were distinctive sub-community features shared among specific groups of sailors speaking Ship English that marked differences in status (e.g., experienced seamen compared to those appointed through family connections, financial necessity, or impressment). Although the research at this stage is not sufficient to support such a claim, this might be a direction for future study.}  

Characteristic features of Ship English were not only diffused and reinforced among the sailors of any one vessel, fleet, or shipping route; they were also transferred to communities on land that were in contact with sailors, e.g., communities of traders in ports and groups of family members and friends in each \isi{sailor}’s country of origin. The effects of \isi{language transfer} and change might not have been circular as suggested by \citegen{Schmidt1872} Wave Model (discussed in \sectref{sec:2.2.1}: Dialect change and \isi{new dialect formation}),  instead, transmission would have likely occurred in coastal and estuary locations where recruitment, trade and leisure brought local populations and sailors into close contact.\footnote{It is likely that \isi{feature transfer} was also occurring from the port community into the ships thus giving rise to the possibility of regional variations among the fleets that sourced their crews from specific locations. For example because London{}-born sailors were most heavily represented in naval vessels, coastal northerners in the \isi{merchant service}, and coastal southerners and westerners in privateering and piracy, it may be that those maritime communities incorporated more linguistic features from the home-regions of their \isi{crew} majority. Furthermore, these potential sub-distinctions may have served to identify sailors as belonging to a particular fleet or service.}\textsuperscript{,}\footnote{To illustrate some evidence of such transfer, a joint letter drafted by the residents of one harbor town (presumed to be Edinburgh) in testimony against Lt. Lilburne [SP 42/6 c.1700] includes zero inflection on \isi{third person} indicative verbs, \isi{emphatic} use of the auxiliary “do”, prepositional “for” with an \isi{infinitive} complement, \isi{possessive} pronominal forms without anterior nominal markers, and accusative-case pronouns in \isi{genitive} phrases—all of which are identified as characteristic features of Ship English.} It is also interesting that many linguistic features associated with Ship English occur in wives’ letters to mariner-husbands.\footnote{Examples of Ship English features in wives’ letters include the omission of articles [HCA 1/9/22], the idiomatic post-nominal phrase “last past” as a marker of temporal sequence [HCA 1/14/76], and the use of “that” at the head of an independent clause as a formal \isi{discourse marker} to index status with no apparent \isi{syntactic function} [HCA 1/98/118].} Thus, evidence suggests that salient linguistic features associated with Ship English were transferred among port communities and family networks in contact with sailors and these features may have been consciously expressed in order to identify with the sailors or/and subconsciously acquired through regular contact with speakers of the variety.  

\subsection{{The typology of Ship English} }%8.1.2

Having provided detailed evidence for Ship English as a distinct variety in Chapters \ref{sec:5}, \ref{sec:6} and \ref{sec:7}, the problem remains as to how we intend to classify this variety. Some historical and non-academic publications use the word “language” to refer to Ship English, e.g., Russell’s descriptions of “the language of the sea” (1883: viii), and “Sailor’s language… compounded of the terms referring to the various parts of ships …[with] a mass of rough sayings into the forecastle, many of which are sanctified by touches of rude poetry” (\citealt{Russell1883}: ix) and the claim in the introductory comments to Smith’s \textit{Sea Grammar} that the work gives explanations and translations for “the language both of ships and Seas” (\citealt{Smith1627}). \citeauthor{Choundas2007} likewise claims that there is a sub-category of this “language of the sea” that could be classified as “a freestanding \isi{pirate} language” (2007: 2). Furthermore, it appears that writers who describe the “language” of sailors do not intend to imply that their system of communication is distinct enough to be unintelligible to speakers of other varieties of English, for example, but rather to stress the differences in their speech in which “meaning is really so subtle as utterly to defy translation” (\citealt{Russell1883}: xv) and highlight the internal systemic coherence that provides a “uniform way of talking” (\citealt{Choundas2007}: 2). Indeed, the term “Ship English” (coined by \citealt{Hancock1976}: 33) already classifies the variety as a type of English. However, the question remains as to what sub-classification of the English language Ship English constitutes. The internal systemic cohesion among speakers disqualifies it as a manifestation of any one individual’s idiolect and the phonological, morphological, syntactic and discourse features attested in the variety show that it is more than \isi{professional jargon}. The remaining potential typological classifications with which we might classify this variety are as a \isi{dialect}, a \isi{sociolect}, or a koine, each of which are discussed in the following paragraphs. 

Ship English is a \isi{dialect}, commonly defined as a manner of speech characteristic of a group of people (\textit{Oxford Eng. Dict.} 1989, Vol 4: 599) and more specifically defined by linguists in terms of three classifications: regional dialects, social dialects (sociolects), and ethnic dialects (ethnolects) (\citealt{WolframSchilling2016}). In addition to describing sailors’ speech as a language (discussed above), \citet{Russell1883} also uses the word “\isi{dialect}” to describe Ship English: “\isi{sailor}’s talk is a \isi{dialect} [...] [in which] English words are used, but their signification is utterly remote from the meaning they have in shore parlance” (\citealt{Russell1883}: ix).\footnote{This citation additionally suggests the possibility of creolization through the process of relexification, yet given the wider context of the comments, it is unlikely that this was what the author intended to propose.}  The suggestion that there is a “sea \isi{dialect}” (\citealt{Russell1883}: xiii) is entirely logical given the general definition of the word as characteristic to group of people and the fact that the variety is characteristic to sailors. Yet Russell’s classification was also a product of trends in nineteenth-century London among scholars with an increased interest in \isi{dialect} studies. Interest in regional and socially stigmatized varieties in the late nineteenth century gave rise to a host of \isi{dialect} glossaries {and motivated comparative studies and \isi{dialect} theory (\citealt{Petyt1980}: 35–38). Indeed, it was at that time that the term “\isi{dialect}” came to be explicitly associated with a regional sub-standard variety in comparison to the variety that was championed as the national standard.}\footnote{{Although}, by modern standards, regional dialects might be classified as “regiolects” {comparable to “sociolects” and “ethnolects”, the general term “\isi{dialect}” was originally} associated with regional speech patterns as sociolinguistics and ethnolinguistics had yet to become established fields of study. Furthermore,{ it is worth noting that modern} linguists universally recognize that everyone speaks a \isi{dialect} and that dialects are not inherently sub-standard although the word “\isi{dialect}” was traditionally associated with social stigmatization and popular notions of inferiority (\citealt{WolframSchilling2016}: 2–17).}  \citeauthor{Görlach1999} describes how{ the} speech of sailors was stigmatized in a period of increasing standardization, as “the speech of those who cannot do any better” at best (1999: 484), or “the gibberish of the uneducated” at worst (p.532).{ Such commonplace beliefs about the lack of social value associated with sailors and their speech explains how Ship English might have been classified as a \isi{dialect} in contemporary studies and in popular opinion as this term would have explicitly marked the variety as substandard and stigmatized.}

{Scholars who have written on \isi{sailor}s’ speech accept that each individual \isi{sailor} would have entered the community with a \isi{dialect} reflective of the region in which that person was raised.} Even though the possibility of babies being born aboard ships is covered in \sectref{sec:4.2.4},  it is unlikely that the vessel was a permanent home to infants learning their native language. Instead, \sectref{sec:3.4} presents data attesting to the fact that most sailors went to sea between the ages of 12 and 16 and would therefore already have established the characteristic speech patterns of their home region. \citeauthor{Matthews1935} makes this point explicit in his monograph on sailors’ pronunciation in the second half of the \isi{seventeenth century}. His final comments in the introductory section stress the relevance of regional influence with the assertion that the writers of the logbooks “must have come from almost every shire’s end of England” (p.195) and thus the findings represent many local dialects. He states “This study, therefore, should be regarded as a cross-section in the history of pronunciation, an account of the various pronunciations in use among the tarpaulin seamen” (p.196). He goes on to state that “It is not pretended that it describes the ‘\isi{seaman}’s \isi{dialect}’ of the period” (p.196), thus explicitly foregrounding the variation of dialects aboard ships in contrast to a potential \isi{seaman}’s \isi{dialect}.\footnote{Although whether Matthews’ comments imply that there was a “\isi{seaman}’s \isi{dialect}” of the period (but that his work did not aim to represent it) or that he does not believe that a “\isi{seaman}’s \isi{dialect}” existed is uncertain.}  The fact that each \isi{sailor} would have entered the community with an existing \isi{regional dialect} makes the classification of any potential “\isi{seaman}’s \isi{dialect}” problematic not only as it implies competition between dialects,\footnote{A modern understanding of diglossia might help to mitigate the suggested conflict between dialects in this context, but this term still carries with it the suggestion that one of the dialects is a standard and the other is a \isi{regional dialect} or \isi{sociolect}, whereas in the context of sailors, neither of the two dialects of English were necessarily the standard (indeed, Standard English might have been additionally acquired in a context of pluri-dialectalism).}  but also because it suggests a process of acquiring native language forms at a young age in a way that did not apply to maritime recruits. Moreover, the historical association of the word “\isi{dialect}” with regional distinctness is problematic as ships’ communities were transient and potentially overlapped with existing geographical \isi{dialect} areas (but without the suggestion that language forms were necessarily transferred). Further research may even attest to regional variations of Ship English such as Mediterranean, Atlantic, and Pacific Ship English, influenced by specific regional variations of the languages their crews encountered. In short, Ship English is understandably classified as a \isi{dialect} in general terms given the broad scope of the term to describe a variety characteristic to one group of people coupled with its working-class stigma of inferiority with respect to the standard. However, for linguists, the sub-classification of the variety demands more specific attention. 

Of the three possible \isi{dialect} classifications available to modern linguists (i.e., \isi{regional dialect}, \isi{sociolect}, and ethnolect) the classification of Ship English as an ethnolect is the least probable given the data presented in \chapref{sec:2} on recruitment practices and sailors’ places of origin and associated language abilities. Some linguists might argue that the \isi{regional dialect} classification is valid given that the data reflects an Atlantic variety of Ship English, yet the transient nature of ships’ speech communities at sea coupled with the notion that maritime communities in ports span five continents might frustrate efforts to define a unified region that underpins the variety. Furthermore, the \isi{regional dialect} classification might also be perceived as problematic {given the assumption that regional dialects are} learned during native \isi{language acquisition}{ (\citealt{ChambersTrudgill1998}: 5). This leaves only one other potential classification.} 

The classification of Ship English as a \isi{sociolect}, characteristic of a group that shares a social identity rather than an ethnic or regional identity \citep[122]{Trudgill2003}, eliminates the problematic associations of ethnic homogeneity and regional unity that are associated with regional dialects and ethnolects and instead promotes focus on the social factors that unified diverse crews. In addition, the problematic suggestion of native \isi{language acquisition} that a \isi{dialect} classification implies is also mitigated as sociolects are often acquired after native fluency through conscious choice to demonstrate group affiliation and passive acquisition of group-specific language features (\citealt{Durrell2004}: 200–205).\footnote{It is important to recognize that although certain sociolects can be acquired as additional varieties later in life (e.g., Instant Messenger and Internet varieties), they can also be coded into native \isi{language acquisition}, particularly when they are associated with social status derived from economic class divisions or regional variation (e.g., \citegen{Labov1966} research based on the Lower East Side of New York City that showed systemic social stratification).}  Furthermore, these socially-conditioned varieties of speech often consciously demonstrate social and professional identification, socioeconomic class, age group and/or ethnic and political affiliation. The term “\isi{sociolect}” is therefore well suited to the classification of Ship English, a variety that is acquired after native fluency among predominantly young working-class men who share a professional context, maritime folklore, and solidarity in the face of hardships at sea. Hancock appears to support a \isi{sociolect} classification in his definition of “a situation-specific register of English, which I call \textit{Ship English}” (1986: 85, author’s italics). \citet{Schultz2010} also takes this position, talking about Ship English as a “\isi{sociolect}” in his unpublished Master’s thesis.\footnote{Hancock served as the academic supervisor on Shultz’s committee at the University of Texas at Austin in 2010 and so may have influenced Shultz’s position in this respect.}  Yet, one caveat remains in that the classification of Ship English as a \isi{sociolect} may suggest social homogeneity among its speakers.\footnote{Although I accept that some definitions of “\isi{sociolect}” do not demand social homogeneity as a defining factor of the \isi{speech community}, most specify that speakers belong to the same social stratum and have comparable ages, incomes, and experiences in addition to frequent contact with one another.}  This book explicitly aims to dispel the popular stereotype that all British mariners were monolingual, lower-class, Caucasian men in their mid-twenties who performed manual labor and led a profligate single life. Indeed, the data in \chapref{sec:3} demonstrates the diversity of maritime communities and this diversity is in danger of being overlooked if their variety of language is classified in such a way as to suggest uniformity in social grouping. Thus, Ship English might reasonably be classified as a \isi{sociolect} given the general scope of the term to apply to a variety characteristic of a group of people connected by social factors such as age, profession, and ideology, yet this classification should not suggest that all speakers shared social homogeneity. 

The problems of assuming social homogeneity if we define Ship English as a \isi{sociolect} or implying regional origins if we define Ship English as a \isi{regional dialect} are resolved if we classify Ship English by its process of formation and not the characteristics of its speakers. As such, Ship English might be classified as a koine, defined as a stabilized contact variety that develops when mutually intelligible varieties (either regional dialects or sociolects) come into contact. After a period of interaction or integration among the speakers of these contact varieties, variant linguistic features mix and become levelled among the group (\citealt{Siegel2001}: 175).\footnote{The term “koine” (or “koiné”) is the Greek word for “common”, and was originally applied to a common Greek \isi{dialect} that developed among the regionally-diverse armies of Alexander the Great in the 4\textsuperscript{th} century BC (\citealt{Andriotis1995}). The term has been more recently applied in contact linguistics as a levelled variety when mutually intelligible dialects come into contact, it has also been used more specifically in terms of \isi{creole genesis theory} \citep{Siegel2001} and the phenomena of new regional and immigrant varieties (\citealt{Trudgill1986}, \citealt{Kerswill2004}).} Trudgill’s theory of \isi{new dialect formation} through koineization in colonial territories (discussed in more detail in \chapref{sec:2}) specifies a three-stage process of mixing, \isi{leveling} and simplification that results in “a single unitary variety” (\citealt{Trudgill1986}: 27). Le Page’s theory on linguistic focusing explains how such a process can result from social conformity and group identification; “the emergence of a closeknit group, a sense of solidarity and a feeling of shared territory are all conditions favouring focusing” (\citealt{Milroy1986}: 378). In \chapref{sec:2}, I propose that the mixing, \isi{leveling}, and simplification process of koineization was potentially happening on board international sailing vessels of the Atlantic because of social conformity and group identification among sailors. \citeauthor{Schultz2010}’s Master’s thesis on the sociolinguistic context of Ship English similarly claims that the development of a unitary ship’s variety of English was made possible by maritime communities of practice, in which “linguistically, strong networks act as a norm enforcement mechanism” (2010: 7–8). The data presented in Chapters \ref{sec:3} and \ref{sec:4} serve to support the claim that linguistic focusing, leading to koineization, was likely happening in maritime communities as a result of the practices that sailors used to reinforce \isi{group identity}. In \chapref{sec:3}, the data on sailors illustrates the hardships that likely increased social dependence, including harsh recruitment measures, lack of pay, high mortality rates, and the dangers of shipboard work, disciplinary action and conflict. This chapter also explains how many lower-ranking sailors and unpaid workers were denied shore leave for fear of desertion, thus forcing their speech communities to become even more insular. The data presented in \chapref{sec:4} regarding speech communities shows how \isi{social cohesion} among crews was facilitated by kinship bonds in mess groups and that collective agency and resistance were often a form of protection against brutal discipline by tyrannous commanding officers and perpetual \isi{subordination}. Among sailors, collective activism was a necessity for survival and the ritual consumption of alcohol bound sailors’ insular communities in trade, conflict, and times of spiritual distress. Group identity was reinforced through shared beliefs in ancient maritime folklore and expressed orally in these communities, thus providing the context necessary to focus linguistic diversity and derive a single unitary variety that expressed shared experience. 


The few modern scholars who have published research on Ship English or features of sailors’ speech appear to support the interpretation that koineization was happening in maritime communities. \citet{Matthews1935} and \citet{BaileyRoss1988} describe common linguistic features among divergent crews, and both acknowledge the role of professional or social identification although neither use the term koineization or classify the resulting variety as a koine or \isi{new dialect}. Matthews presents the \isi{phonology} of sailors as a shared paradigm motivated by the technical jargon of professional association. He explains in his introductory notes, “a craft imposes certain traditional pronunciations upon those who engage in it. For sailors, whatever their early \isi{dialect} and education, there must have been certain conventions of pronunciation for words used exclusively in the sea-trade” (1935: 193). Matthews falls short of claiming that his findings show a leveled pronunciation system, yet his explanation of how sailors, regardless of their own \isi{dialect}, were obliged by their craft to observe certain conventions of pronunciation certainly suggests the stages of mixing and \isi{leveling} that occur in koineization with the motivational force of professional association serving to provide linguistic focusing. \citegen{BaileyRoss1988} article on morphosyntactic variation of Ship English similarly describes stages of mixing and \isi{leveling}. In the conclusion to their paper they explain: 


\begin{quotation}
variation and change were certainly no greater than in most nonstandard varieties of English, and in the process of its formation Ship English seems to have eliminated the most abberant features of British dialects, as \citet{Hancock1976} suggested. As a result, it shares many of its morpho-syntactic structures with other British regional and social dialects; in fact, it is not at all clear that \textit{grammatically} Ship English is a unique \isi{sociolect}, although its lexical uniqueness is apparent. At least in its morpho-syntax, Ship English represents a kind of ‘levelled’ variety similar to those discussed by Trudgill (1986), with the most widespread nonstandard features preserved and the most restricted ones apparently lost. (\citealt{BaileyRoss1988}: 207, authors’ italics)
\end{quotation}

\largerpage
The wording of Bailey and Ross’s paper appears to echo Matthews’ suggestion that “whatever their early \isi{dialect} and education, there must have been certain conventions…used exclusively in the sea-trade” (1935: 193) and might thus be seen as an elaboration of Matthews earlier position that Ship English is a variety resulting from mixing dialects and deriving leveled features as a result of professional association. Thus, both \citet{Matthews1935} and \citet{BaileyRoss1988} support the interpretation of mixing and \isi{leveling} with linguistic focusing. My own position may be interpreted as taking their claims one step further by suggesting that the result of the mixing and \isi{leveling} led to default paradigms (i.e. simplification) and resulted in a shared unitary koine that follows all the stages of \citegen{Trudgill1986} theory of koineization. However, although some linguistic features identified in Chapters \ref{sec:5}, \ref{sec:6} and \ref{sec:7} attest to linguistic simplification (e.g., zero inflection, default pronominal forms, \isi{negative concord}, \isi{leveling} of the present and \isi{past tense} \isi{copula} forms, overt auxiliaries in all modalities, multiple functionality of specific lexemes, and the use of coordinating conjunctions to build super-structures regardless of parallelism requirements) other features suggest complexity (e.g., \isi{linguistic conditioning} of feature omission and placement, \isi{subordination} marked by present-\isi{participle} phrases, [\isi{non-specific verb} + \isi{specifying nominal complement}] constructions, fronting of intransitive verbs before a \isi{noun phrase} subject, and swearing to mark \isi{realis} and imperative modalities).\footnote{It is worth noting that some features analysed can be presented as evidence of simplification \textbf{and} as evidence of linguistic complexity, e.g., the \isi{verb} “to do”. On one hand, “do support” may have been a simplified default in all affirmative \isi{verb} phrases to aid the process of acquisition for language learners, yet it also potentially functions to mark \isi{aspectual} and/or subordinating meaning in affirmative clauses in the \isi{indicative mood} when used in conjunction with \isi{preterit} forms (see full discussion in \chapref{sec:6} on the auxiliary “to do”). I have purposely presented both interpretations as I think that they are not mutually exclusive. Instead, the use of “do” may have changed with any individual speaker’s fluency in the language. Learners might have defaulted to a universal use of “do support” without \isi{aspectual} or subordinating meaning, and native/fluent speakers might have used the \isi{verb} “do” to mark subtle distinctions in meaning between \isi{verb} phrases.  } Yet it is important to recognize that the term “simplification” in Trudgill’s framework does not necessarily suggest linguistic simplification, but a determinism of the \isi{new dialect formation} that is marked by the manifestation of a “final, stable, relatively uniform outcome…[when] the \isi{new dialect} appears as a stable, crystallised variety” (\citealt{Trudgill2004}: 113). The “simplification” is represented in the emergence of default paradigms and not a suggested linguistic simplification that we might associated with \isi{pidgin} varieties. The data presented on Ship English indicates that there was still significant \isi{free variation} with respect to certain features (e.g., the omission of articles, the selection of regular weak or unmarked \isi{preterit} forms, and the placement of adverbial constituents) yet there were also clear default paradigms that marked the variety (e.g., marking \isi{genitive case} with prepositional phrases rather than nominal inflection, post-nominal placement of \isi{present participle} phrases, overt use of the auxiliary “to do” in constructions with affirmative \isi{indicative modality}, fronting verbs if they can function as auxiliaries, and the use of a \isi{stative} \isi{present participle} in progressive structures). Therefore, Ship English might be reasonably classified as a koine owing to its formation through a process of mixing and \isi{leveling} regional dialects and its emergence as a stable uniform variety with default linguistic paradigms determined from a variety of input features, however, this interpretation necessarily also classifies the variety as a new world \isi{dialect} and therefore alludes back to the problematic assumptions of geographical containment and native \isi{language acquisition} once the variety achieves a uniform outcome. 

In summary, the various potential classifications with which we might categorize Ship English all have their merits, yet none are without problematic assumptions about the nature of the language variety or the people who spoke it. I believe that although the variety was undoubtedly influenced by the idiolects, regional dialects and \isi{professional jargon} of those within the maritime community, it constitutes a unitary system of morpho-syntactic and discourse variation that extends well beyond individual speakers, occupational lexicon, or the influence of any one \isi{regional dialect}. Taking into consideration the analysis and opinions of previous researchers working on Ship English and the socio-historical and linguistic evidence presented in these chapters, I believe that although sailors’ speech has been historically associated with a \isi{professional jargon} and more recently conceptualized as an occupational \isi{sociolect}, the most appropriate classification we can assign is that Ship English of the early \isi{colonial period} was a \isi{new dialect} of English that was formed through the mixing, \isi{leveling} and simplification processes of koineization. This conclusion is presented in the full knowledge that research into Ship English is in its early stages and further work on the shared linguistic systems of sailors may lead to modifications and amendments of this claim as we learn more about the process of \isi{language transfer} and the extent of linguistic simplification in the variety. We should also bear in mind that like any linguistic system, Ship English manifests itself on a continuum of localized and individual variation and there are also likely sub-categories of Ship English according to sailing region, \isi{crew} composition, and the type of vessel or \isi{voyage}. Furthermore, I do not propose that there was one type of Ship English as a monolithic variety, but instead that there were core linguistic features that identified speakers as sailors. These linguistic features, like any others, were “transmitted by normal social and cultural forces” \citep[129]{McDavid1979} and were therefore prone to idiolectal and systemic change in addition to adaption and replacement over time.

\section{{Implications}}%8.2

One of the aims of this study was to generate a baseline of linguistic data that describe common features of Ship English with the hope that this variety might be integrated into the discourse, the theories and the research on \isi{dialect} variation and \isi{language contact} in the early \isi{colonial period}. As such, the implications of findings presented in the chapters of this book are discussed in terms of how they relate to our understanding of dialectology in the discipline of historical linguistics and the theories of \isi{pidgin} and \isi{creole genesis} in the discipline of contact linguistics. This section on the implications of Ship English also includes some general observations on recovering the agency of sailors and advocating for future work that integrates data on sociolinguistics and emulates the multidisciplinary approaches of Atlantic Studies.

\subsection{{Relevance for dialectology}}%8.2.1

The recognition of Ship English as a distinct historical \isi{dialect} of British English prompts a reconceptualization of \isi{dialect} change and \isi{feature transfer} in the British Isles. Despite the traditional models of \isi{dialect diffusion} that emphasize central geographical focal points with concentric waves of influence, this study focuses on a \isi{dialect} that was formed without a common center and drew influences from various locations that were not regionally adjacent. Rather than being defined by a central geographical focal point with rivers and seas serving only to limit the extent of \isi{dialect} expansion, the genesis of Ship English refutes assumptions about the obstruent nature of waterways in \isi{dialect diffusion} and instead encourages scholars to envision rivers as conduits of communication and seas as fertile spaces where new forms of speech could incubate and stabilize. The reconceptualization of waterways as potential spaces for \isi{dialect} change and expansion is perhaps particularly relevant to scholars of \isi{historical dialectology}, and although maritime communities may have served as agents of \isi{language change} with respect to English as far back as Anglo-Saxon times, the agency of sailors in \isi{dialect} change might particularly interest scholars whose research interests coincide with the era of expanding maritime technology, trade and exploration that converted the seas in the early modern period into busy spaces of transit and imperial regulation. Scholars interested in the Early Modern English period of \isi{dialect} change in the late sixteenth and seventeenth centuries might particularly focus on the potential internal changes that were driven by increasing coastal transportation services of wheat, cloth, and coal (\citealt{Willan1967}). The sailors who worked in these trades were likely to have acquired and transported \isi{dialect} features from port to port, particularly as their profession may have carried the type of covert prestige that is often associated with working-class regional speech and occupations associated with masculinity and toughness (\citealt{Petyt1980}: 160). Indeed, their agency may have derived the kind of “pan-variety parallelism” that \citeauthor{Tagliamonte2013} claims occurred among northern British regions and crossed the Irish Sea in which “all communities share the same (variable) system in each case and it is only in the subtle weights and constraint of variation that the differences emerge” (2013: 192). If, indeed, maritime workers had an influence in transmitting and \isi{leveling} \isi{dialect} features among ports of the British Isles, then this process was also potentially happening on a larger scale given that “[t]he combination of ocean and river routes defined the shape of the Atlantic zone” (\citealt{Thornton2000}: 56). Consequently, it is possible that \isi{maritime speech} communities also leveled linguistic features around the Atlantic and established the type of supra-regional varieties that are still evident in pan-Caribbean English usage (\citealt{Allsopp2003}).  

The data presented here on Ship English and the idea that leveled features composed a stable variety may help to refine what “\isi{dialect}” implies for the continued effective use of this term in the new information age. To echo a statement from the introduction, we live in a world so interconnected by air travel, media and online networks that we rarely consider the importance of maritime travel or those who depended upon it in an age before we physically and digitally took to the skies. Yet, studying the speech communities of people who sailed the waters that connected the edges of the known world may ironically give us some insight into how supra-regional varieties are formed in a modern world where global networks connect distant places and incubate varieties of language that neither adhere to any standard regional norms nor diffuse via regional adjacency. As \citeauthor{Darvin2016} explains in his paper on language and identity in the digital age, technology has revolutionized the way we communicate; increased travel, mobile communication and online connectivity have blurred the boundaries of space “leading to new identifications, allegiances, and relations” (2016: 523). As a result, new digital technologies necessitate concepts of \isi{dialect formation} and research methodologies that can trace what \citeauthor{StornaiuoloHall2014} describe as the “echoing of ideas across spaces, people and texts” (2014: 28). 

This book likewise attempts to trace the echoing of sailors’ language features across the space, people and texts of the early \isi{colonial period}. It presents Ship English as a language variety with no regional origin that is defined by its medium of transmission; this definition creates parallels between non-regional, techn\-o\-logy-mediated varieties such as Instant Messaging (IM) and texting varieties of English (\citealt{WarschauerMatuchniak2010}) in addition to translingual varieties that evolve online (\citealt{Canagarajah2013}).  Like many of these new varieties emerging via technology, Ship English is not a traditional \isi{dialect} defined by regional parameters nor is it a \isi{sociolect} shared by a single stratum of society, instead, it is a poorly-understood variety derived from the mixing and \isi{leveling} of distinct regional features with influence from other languages. It furthermore demonstrates a type of simplification that facilitates learner acquisition but also permits the complex syntactic variation that enables fluent speakers to express subtlety and complexity in meaning. Thus, perhaps a variety such as Ship English could provide an impetus to refine models of \isi{dialect} genesis for the digital age in which mediums of communication and high-levels of non-native acquisition are more important than identifying a single geographical origin, social class or ethnic group in which native speakers acquire fluency. In short, if we can re-conceptualize the term “\isi{dialect}” in a way that especially de-emphasizes its regional restrictions and instead highlights its potential range through the medium by which it is transmitted, then Ship English might help us understand the processes through which newly emerging global varieties are developing. 

If we accept that Ship English was formed through a process of \isi{dialect} mixing, \isi{leveling}, and simplification in the same way that \citet{Trudgill1986} describes the formation of immigrant koines, then the data on Ship English also serve to expand our understanding of the contexts in which koineization can take place. Since he first proposed his three-stage theory of \isi{new dialect formation} in 1986, Trudgill envisioned the process of koineization as one connected with language developments in colonial territories where immigrants with different varieties of mutually intelligible regional dialects gathered and their language features blended. \citeauthor{Trudgill2004}’s more recent book published in 2004, \textit{New Dialect Formation: The Inevitability of Colonial Englishes,} explicitly connects the process of koineization with immigrant and settler forms of English. However, his evidence on Southern Hemisphere varieties are prefaced by the claim that all colonial varieties derive from a combination of comparable factors: adaptation to a new physical environment, different linguistic changes in the mother country and the colony, \isi{language contact} with indigenous languages and with other European languages, and internal \isi{dialect} contact (2004: 1–7). The relevance of these factors become clear when we envision the ship itself as a microcosm of the colonial state, as \citeauthor{LinebaughRediker2000} propose in their history of the revolutionary Atlantic (2000). Their chapter entitled “Hydrarchy: Sailors, Pirates, and the Maritime State” explores how the ship itself represented one critical process in which capitalists organized and united the exploitation of human labor. They explain:

\begin{quotation}
The consolidation of the maritime state took place in the 1690s, by which time the Royal Navy had become England’s greatest employer of labor, its greatest consumer of material, and its greatest industrial enterprise…Here were Braithwaite’s “walls of the State”, an enclosure built around a new field of property whose value and appreciation were expressed in a congeries of changes in the 1690s. (\citealt{LinebaughRediker2000}: 148)
\end{quotation}

Rediker continues to explore the idea of the maritime nation state, and specifically the renegade colony of the \isi{pirate ship} in \textit{Villains of all Nations} (2004). His analogy is made explicit in the chapter entitled “The New Government of the Ship” that explores how pirates established a new social order: “It’s hallmark was a rough, improvised, but effective egalitarianism” (2004: 61). Thus, not only can we envision the ship itself as a microcosm of the colonial state, but also a space in which revolution against colonial control was expressed prior to any American colonial declaration of independence from British control. Given such circumstances, the idea that processes of koineization occurred comparable to the developments in the immigrant communities of colonial territories and concurrent with the establishment of new governments is reasonable. \citeauthor{LinebaughRediker2000} acknowledge the likelihood of \isi{language change} aboard ships in their observation “European imperialism also created the conditions for the circulation of experience within the huge masses of labor that it had set in motion…[and] The circulation of experience depended in part on the fashioning of new languages” (2000: 152). I propose that these “new languages” (i.e., new dialects of English) were forged through the same three-stage process of kionization that Trudgill explains happened in colonial spaces where immigrants worked and lived together, giving their mutually-intelligible language features the opportunity to mix, level and simplify into a new variety specific to that space. In short, evaluating Ship English as a koine permits scholars in dialectology to expand the scope of koineization and apply Trudgill’s theories to varieties of English that develop in transient colonial spaces that are not necessarily defined by geographical parameters or international treaties. Indeed, this interpretation of the term reclaims the transient context of its origins referring to the common Greek \isi{dialect} spoken among the mobilized armies of Alexander the Great in the 4\textsuperscript{th} century BC. 

Ship English, here presented as a newly-recognized \isi{dialect} of English, impacts the field of World Englishes and \isi{historical dialectology} in the sense that the theories and approaches specific to these disciplines must now incorporate and account for this variety and its speakers in its theories of \isi{dialect formation} and usage. The fact that Ship English does not fit neatly into any one geographical territory prompts a revision of how we understand internal \isi{dialect} change in and around the British Isles and compels us to reconsider the scope of what constitutes a global variety of English. It also encourages scholars to advocate for the type of interdisciplinary perspective central to scholars of Atlantic Studies, expressed by the editors of the journal \textit{Atlantic Studies} as a discipline that:

\begin{quotation}
explores transnational, transhistorical, and transdisciplinary intersections, but also addresses global flows and perspectives beyond the Atlantic as a closed or self-contained space…[and] considers the Atlantic as part of wider networks, a space of exchange, and an expanding paradigm beyond the limits of its own geography, moving beyond national, regional, and continental divides by examining entangled histories and cultures…[and therefore] challenges critical orthodoxies that have drawn sharp lines between the experiences and representations of the Atlantic world and its wider global context. (\citealt{TaylorandFrancisGroup2016}, \textit{Atlantic Studies} §Aims and Scope)
\end{quotation}

Embracing interdisciplinary approaches necessarily means embracing complexity in \isi{dialect} studies, and this only becomes more complex when we aim to centralize the human agency integral to \isi{dialect} contact. Recovering the human stories that explain \isi{dialect} change is particularly important to \isi{dialect} research methodologies in order to challenge trends which produce sterile and monolithic explanations of \isi{dialect} change in which human stories are either ignored completely or relegated to a footnote. The focus of this book, by foregrounding demographic data, aims to recover the agency of ordinary people who motivated extraordinary change. There is no doubt that ships’ communities bred unique language practices because of the unique composition of those who worked within them. Therefore, throughout this book, I have attempted to acknowledge and respect the complex realities and the linguistic agency of all the people who lived and worked in maritime communities. Furthermore, I offer the findings presented here as testimony to the undocumented, undervalued, and often unnamed majority of workers aboard sailing vessels of the early \isi{colonial period} and I encourage other scholars of dialectology and historical linguistics to likewise consider the agency of marginalized people who may have only left behind ambiguous traces on the palimpsest of the official historical record. These marginalized peoples may not have played a major role in imperial history; yet, they potentially helped shape and direct the incremental changes that characterized their oral cultures. As Daniels expressed so eloquently in the abstract for his paper on the Atlantic marketplace at a conference on the emergence of the maritime nation in England:

\begin{quotation}
If we abandon the idea of centrality of the mother countries, their rulers, and their institutions, we might imagine a grittier and more organic Atlantic world constructed from the strands of individual lives and the repercussions of their actions rather than an Atlantic world engineered from above the heads of constitute parts. (\citealt{Daniels2015})
\end{quotation}

Complex, contradictory, and confusing data in \isi{dialect} studies reflects the realities of the individuals who motivated \isi{dialect} change and who were driven by self-preservation and pulled by the local and inter-imperial regulation that shaped the spaces in which they lived. This research and the complexities of the data it presents in order to advocate for the recognition of Ship English as a comprehensive \isi{dialect} is anticipated to contribute to a growing movement in modern scholarship that challenges traditional models of \isi{dialect diffusion}, embraces interdisciplinary perspectives, and foregrounds the humanity of \isi{dialect} change in methodological approaches. 

\subsection{{Relevance for contact linguistics}}%8.2.2

One of the aims of this study was to outline the socio-demographics of the maritime communities and examine how variant linguistic features may have developed and spread among these communities. As I explained in the introduction, my principal focus is to present baseline data that substantiates the fundamental claim that Ship English of the early \isi{colonial period} was a distinct variety. It is not intended to support of any one school of \isi{creole} linguistics, nor is it intended to present Ship English as a formative variety in \isi{creole genesis}, however, the findings relate to contact linguistics and particularly to theories of how pidgins and creoles emerged in colonial regions around the Atlantic. More specifically, the findings substantiate claims by \citet{Reinecke1938} and  \citealt{Hancock1972} \citealt{Hancock1976} \citealt{Hancock1986} and \citealt{Hancock1988} that a potential type of language spoken on ships may have influenced \isi{creole} development.\footnote{Because of this connection between my own research and Ian Hancock’s formative theories on \isi{creole genesis}, we collaborated on a paper for the Society of \isi{Pidgin} and \isi{Creole} Linguistics’ conference in January {2017}. This paper makes the connections between Hancock’s work and my own explicit in a theory of how maritime workers established a coastal English on the Upper Guinea coast that was later transferred to the Caribbean (\citealt{DelgadoHancock2017}, January 7).} The central premise of integrating Ship English into \isi{creole} studies is based on the proposition that seamen are the most logical connection between ports of the Atlantic and the creoles that demonstrate common origins but are spoken over 12,000 nautical miles apart in places such as the Caribbean, Suriname, and the Guinea Coast (\citealt{Hancock1976,FaraclasEtAl2012}). It is reasonable to suggest that Ship English was the variety of English used for coastal trade and therefore the most logical language in contact, particularly considering sailors’ roles in settling and maintaining many of the colonies where \isi{creole} languages arose (\citealt{BaileyRoss1988,Holm1988}). Another possible theory to explain the influence of Ship English during the early \isi{colonial period} is that the \isi{dialect} was introduced along the coast of Africa and fed into emerging local varieties which were more functional for coastal trade than any one of the regional African languages. When sailors established domestic relationships with African women in multilingual enclaves of settlement, these local varieties of coastal English then became influenced by African features and conditioned by language universals to form a coastal \isi{pidgin} continuum. This \isi{pidgin} was subsequently creolized by the generations born to these multilingual communities who then transmitted their varieties of speech to the slaves they were employed to manage and the regions they were sent to labor in. Future studies that apply such theories can now compare the features of English-lexifier Atlantic Creoles with the features of Ship English to determine if significant similarities support these theoretical claims. 

The suggestion that a coastal \isi{pidgin} developed off the coasts of West Africa has echoes in \citegen{Reinecke1938} claim that \isi{pidgin} English was developed on the colonial plantations from the trade jargon of the ports. Reinecke, like many of his contemporaries, belittled trading pidgins as “makeshift language…[and] mangled little dialects” \citep[107]{Reinecke1938} whilst at the same time realizing the importance of mariners’ contributions to intercultural communication. More recently, scholars have recognized the importance of \isi{pidgin} varieties and placed more value on the role of maritime communities in the contentious pidgin-\isi{creole} continuum debate \citep[7]{Holm1988}, for example: \citet{Hancock1976} has long championed the influence of maritime vocabulary on Krio, a \isi{creole} of Sierra Leone; \citet{Holm1981} identifies the importance of sailors as the agents of \isi{language change} in his extensive work on Nicaragua’s Miskito Coast \isi{Creole}; and \citet{Dillard1992} has proposed that the most authentic American varieties of English derived from a \isi{pidgin} formed from sailors’ language. Although scholars such as these have often alluded to the role of a potential \isi{pidgin} derived from sailors’ speech, none have supported a claim that such pidgins were already in use throughout maritime communities. Yet this was also potentially the case, particularly given the extensive international \isi{language contact} that certain crews regularly experienced, not only as a natural consequence of their trading activities but also because of their international recruitment practices that created multinational crews with high levels of linguistic diversity. Consequently, and given that we now have a more comprehensive idea of what Ship English was owing to the findings presented here, future work can now engage with the question of whether Ship English gave rise to a distinct nautical \isi{pidgin} on the high seas in addition to potential coastal and plantation pidgins after \isi{language contact} occurred in colonial contexts. 

\largerpage
The new data on Ship English and the emerging possibility that we can linguistically support a claim that sailors spoke a nautical \isi{pidgin} owing to their trade and recruitment practices around the Atlantic might have a major impact in theories of \isi{creole genesis}. Scholars who advocate for a theory of \isi{creole genesis} in which the new stabilized creoles developed when earlier \isi{pidgin} forms were expanded for the use of native speakers (\citealt{Holm1988}) could now consider the possibility of a potential Ship \isi{Pidgin} as one of the pre-existing forms that were available to the transported workers of plantation and port speech communities. Scholars who support a monogenesis theory, in which — in its most radical interpretation — all creoles have a single proto-\isi{pidgin} that developed out of Portuguese contact in the West African gold trade in the 15\textsuperscript{th} century might now consider the possibility that this proto-\isi{pidgin} developed in a nautical context. This idea is additionally plausible if we consider the role of the \isi{slave trade} as an industry that motivated \isi{language contact} situations and remember the key role of maritime traffic in this trade and thus the potential of mariners as agents of \isi{language change}. Scholars who argue for founder theories and processes of competition and selection \citep{Mufwene1996} could now consider Ship English (or a related Ship \isi{Pidgin}) as potential sources of features that competed for selection in new regional forms. The likelihood of Ship English influencing regional language developments would be significant in a theory in which “structural features of creoles have been predetermined to a large extent (but not exclusively) by characteristics of the vernaculars spoken by the populations that founded the colonies in which they developed” (\citealt{Mufwene1996}: 84). If we consider sailors as one of the “populations that founded the colonies” then this theory of language ecology not only incorporates Ship English, but also prompts debate about the very nature of what scholars perceive to be the prestigious \isi{superstrate} varieties available in the colonial setting. In short, scholars who subscribe to theories of pidgin-\isi{creole genesis}, monogenesis theory, and founder principles of competition and selection might begin to work in new directions given this new linguistic data on Ship English that provides a baseline for comparison with regional creoles and motivates the potential for future research on Ship Pidgins. 

The suggestion that the existence of Ship English might prompt debate about the nature of the \isi{superstrate} is explicit in \citegen{BaileyRoss1988} article on “The Shape of the Superstrate: Morphosyntactic Features of Ship English”.\footnote{Bailey and Ross propose that Ship English should be recognized as the proto-typical variety of the \isi{superstrate}, illustrated in their description of “what seems to be the earliest component of the \isi{superstrate}—the ‘Ship English’ spoken by British sailors during the 16\textsuperscript{th}, 17\textsuperscript{th}, and 18\textsuperscript{th} centuries” (1988: 194–195).}  Nearly thirty years after their article was published, this book now presents substantial data on the syntactic features of Ship English and permits scholars to revisit and revise concepts of the \isi{superstrate} with the advantage of access to new empirical data. If the syntactic features of certain creoles compare with Ship English, then the data presented here may prompt revision of the simplified binary model of African substrates (with phonological, syntactic, and semantic influences) opposed to a unitary European \isi{superstrate} (with lexical influence).\footnote{The lesser-used term “adstrate” refers to a language that is considered (reflecting the status of its speakers) to be neither superior nor subordinate, but present in the contact situation as a potential source of \isi{feature transfer} or borrowing.}  Furthermore, the over-simplification that all varieties of English were prestigious in dichotomy with West African languages fails to acknowledge the stigmatized varieties spoken by people from low socio-economic backgrounds who were exported to work in the early colonial system, living in outcast communities, or escaping political hegemony in the British Isles. Dismissive of such complexity, the use of “substrate” and “\isi{superstrate}” alludes to the origins of the terms in Romance linguistics as referents of power yet also suggests their linguistic contribution to creoles as a direct result of the imbalance of power in \isi{language contact} situations. For example, Holm explains how superstrates contribute lexical features because, “usually those with less power (speakers of \textit{substrate} languages) are more accommodating and use words from the language of those with more power (the \textit{superstrate})” (1988: 5, author’s italics). The substrate-\isi{superstrate} binary model is perpetuated in \isi{creole} theories even though “the concepts are still largely used intuitively” \citep[55]{Selbach2008} and “there has been no satisfactory answer to the perennial question of the degree to which the structure of \isi{superstrate} and substrate languages influence that of creoles” (\citealt{Holm2009}: 218). Yet equating the status of the speaker to the influence of their language in an emerging \isi{creole} context might prove to be overly deterministic as we learn more about the subtleties of \isi{language contact} and the complexities of \isi{feature transfer}. This determinism is compounded by the term “lexifier” which is used synonymously with “\isi{superstrate}” and has no equivalent form that pairs with the term “substrate” to signify phonological, syntactic, and semantic contributions. Scholars who cross-match the social referent “substrate” with the functional referent “lexifier” as though they refer to the same paradigm compound this confusion. For example, \citeauthor{McWhorter2005}’s \textit{Defining Creole} (2005) that explains the “traditional emphasis in \isi{creole} studies on transfer from \textit{substrate} languages rather than \textit{lexifier}” (p.85, my italics).\footnote{Although here criticized for his equivalency of the terms “substrate” and “lexifier”, McWhorter also acknowledges that simplified interpretations of a unitary and prestigious \isi{superstrate} are flawed. He states, “\textit{superstratists} have rightly criticized the tendency to compare creoles with standard varieties of their lexifiers, calling attention to the models for \isi{creole} constructions in now-obscure regional dialects spoken by the white colonists” (\citealt{McWhorter2005}: 143).}  The effect is to suggest that a substrate language cannot \textit{be} the lexifier although it has been demonstrated that substrate lexical transfer has played a key role in the configuration of many \isi{creole} grammars (\citealt{Kihm1989,Migge1998}). The parallel assumption that prestigious superstrates determine the lexical features of \isi{creole} languages has been challenged by scholars such as \citeauthor{Selbach2008} in her paper: “The Superstrate is not always the Lexifier” (2008). Yet these presumptions will continue if the terms “\isi{superstrate}” and “lexifier” continue to be treated as synonyms referring to both social status and lexical contribution. Challenging these presumptions in a way that permits us to approach the processes of \isi{language change} in contact situations without prejudgment will happen as more scholars appreciate the potential contributions (beyond lexicon) of non-standard varieties such as Ship English. In short, the syntactic focus of this study offers a baseline of data that might be used in comparative studies to investigate the potential syntactic influence of Ship English in the creoles that developed around the colonial territories of the Atlantic region and thus challenge any restrictive assumptions that prestigious superstrates contributed only lexical features and syntactic variation can be entirely explained by substrate influence. 

\section{{Summary}}%8.3

The data presented in the chapters of this book support the central claim that there is a distinct Ship English that was spoken by British sailors in the early colonial context with a unique socio-historical context and characteristic linguistic features. The sociolinguistic, syntactic and discourse-level features presented in this study add to the existing data on lexicon, \isi{phonology} and morphosyntax attested to in prior scholarship to provide a comprehensive baseline of data for this newly-recognized historical variety of English. Furthermore, we can assert that Ship English stabilized and spread in maritime communities through predominantly oral speech practices and strong affiliations among groups of sailors. The variety was also transferred to port communities and sailors’ home regions through regular contact with speakers, a process that was also potentially intensified by covert prestige. The variety was not monolithic, however, and its features likely existed on a continuum of localized and individual variation and were prone to idiolectal and systemic change in addition to adaption and replacement over time just like the features of any other variety. 

Ship English was a historical \isi{dialect} of English and although varied linguistic sub-classifications of this \isi{dialect} are possible, none are without problematic assumptions about the nature of the variety or its speakers. The classification of the \isi{dialect} as an ethnolect is the least probable given the data available on global recruitment practices. The classification of Ship English as a \isi{regional dialect} has validity in that it indexes contemporary assumptions about {substandard and stigmatized usage and acknowledges that the variety was used in specific geographical regions, specifically Atlantic trade routes} and the trading ports of their coastal zones. {Yet, this classification} is problematic{ as it suggests} geographical unity and transmission through either geographical adjacency or generational \isi{language acquisition}. The most convincing classification of Ship English is that it was a \isi{sociolect} that was formed through the mixing, \isi{leveling} and simplification processes of koineization. This classification is furthermore supported by the implicit recognition of mixing, \isi{leveling}, and simplification processes in the limited scholarship on the variety.

The recognition of Ship English as a distinct historical \isi{dialect} of British English challenges traditional models of \isi{dialect diffusion} and prompts a reconceptualization of \isi{dialect} change and \isi{feature transfer} in the British Isles by specifically recognizing the potential of supra-regional varieties formed by the \isi{leveling} of mutually intelligible linguistic features. It also serves to expand our understanding of the contexts in which koineization can take place. Consequently, it prompts us to re-conceptualize the term “\isi{dialect}” in a way that de-emphasizes its traditional regional restrictions and instead highlights its potential range through the medium by which it is transmitted. Such approaches might influence thinking on \isi{language change} in the new information age and prompt us to reevaluate what we consider a native, a secondary and a world variety of English. My own methodology and approach also advocate for interdisciplinary perspectives and the recovery of the complex human stories critical to understanding linguistic change. 

Data on Ship English is necessarily relevant to contact linguistics and specifically theories on pidgins and creoles because maritime workers connected the diverse ports around the Atlantic and helped to found and settle the regions that developed creoles. New data on Ship English provide a baseline for comparison with regional creoles and motivate the potential for future research on Ship Pidgins which, in turn, impact theories of pidgin-\isi{creole genesis}, monogenesis theory, and founder principles of competition and selection in regions that saw the emergence of \isi{creole} languages. Furthermore, the recognition of a stigmatized variety of English prevalent in the contact situation of the early colonial Atlantic prompts discourse about the nature of the \isi{superstrate} and challenges our assumptions relating to the lexical, phonological, syntactic, and semantic influences of different languages and varieties in the contact situation. Although the data presented have clear implications in the field of contact linguistics, there is still considerable work to be done to verify its significance in \isi{historical dialectology}. It is my hope that future research might compare the Ship English features identified here to the features of contemporary varieties used in the Early Modern English period to clarify the extent to which these features were marked in sailors’ speech. Future studies might also explore the possibilities of parallel developments in the languages of other sea-going nations to determine if levelled nautical varieties commonly developed in situations comparable to the speech communities that derived Ship English. Studies might also explore the potential variation between Mediterranean, Atlantic, and Pacific varieties of Ship English. Such work might help us achieve a more comprehensive understanding of the linguistic processes that occurred during times of rapid \isi{colonial expansion} and may also shed light on how levelled varieties develop among transient communities.  


