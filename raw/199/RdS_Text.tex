%%% -*- Mode: LaTeX -*-

% \newgeometry{hmargin=2cm,vmargin=1in}
\chapter*{Principes logiques de la formation des mots -- Logical principles of the formation of words}
\addcontentsline{toc}{chapter}{Principes logiques de la formation des mots -- Logical principles of the formation of words}
\label{ch.1911text}

\begin{refsection}
\lettrine[loversize=0.1, nindent=0em]{T}{he text of René de
  Saussure's first little book} follows here together\linebreak with an English
translation.  In the translation, French words cited as examples have
been preserved as such and italicized, with the first instance of a given
word on a page provided with an English gloss in the early pages.
Since the range of French examples cited by de Saussure is quite
limited, however, glosses are dispensed with in later portions of the
work for words that should be familiar.  French words cited as
concepts or ideas, in contrast, have generally been translated except
where this would impair the sense of the text (in which case they have
been treated in the same manner as examples). Words from other
languages (in particular, German) presented without glosses by de
Saussure have been left in that form.

The translation has attempted to follow the original as closely as
possible: Our goal is to make the French original accessible to the
English reader, rather than to recreate the work as René de Saussure
might have written it in English. The pagination of the original text
has been preserved and indicated at the top of each page, although no
attempt has been made to maintain the division of pages into lines. We
have retained the original typography to the extent possible. Inserted
material (e.g. opening or closing quotes missing in the original) is
enclosed in square brackets; we trust no confusion will result from
confusion with the use of such brackets in the text.\largerpage[2]

The volume is dedicated to ``M. le Professeur Th. Flournoy'', without
further elaboration, and some remarks on this scholar are in order
here. Théodore Flournoy was born in Geneva in 1854 and died there in
1920.  He studied philosophy and medicine before turning to
psychology, and held a chair in Experimental Psychology at the
University of Geneva from 1891 until his death. A member of another of
Geneva's socially prominent protestant families (the Claparèdes), he
would naturally have come into contact with the de Saussures, and in
particular with René's brother Ferdinand. The two attended the same
schools, and both eventually held chairs at the University, although
since Flournoy was three years older, they were not particularly close
in their youth.\newpage

\begin{figure}[htb]
  \begin{center}
    \includegraphics*[0,0][1.705in,2.5in]{./Photos/Theodore_Flournoy.eps}
  \end{center}
  \caption{Théodore Flournoy (1854--1920 )}
\end{figure}

Flournoy was a significant figure in the early development of
psychology in Europe, and his best known work, \emph{From India to the
  Planet Mars} \citep{flournoy1900:india-to-mars} was a major
influence on Carl Jung. This book involved a detailed recounting and
analysis of a series of séances with a Geneva medium Cathérine-Élise
Müller (identified in the book by the pseudonym Mlle. Hélène
Smith). Mlle. ``Smith'' in a series of trances over a five year period
recounted a series of supposed experiences in past lives, including a
life on Mars, life as Marie Antoinette, and a life in India. Flournoy
takes her experience quite seriously and does not treat it as
fraudulent, but rather works out in detail the ways in which what she
describes originates in her own early experience and reflects the
operations of a subconscious mental life. All of this was quite
congenial to those such as Jung (and William James, with whom he was
also in contact) developing similar views of the mind
\citep{witzig82:flournoy}.

Important to Flournoy's connection with the Saussures, however, is the
fact that he involved Ferdinand with the analysis of the series of the
medium's ``Hindoo Cycle'' séances, several of which he attended
\pgcitep{joseph12:saussure}{426ff.}. Ferdinand was consulted
especially with regard to the idea that some of Mlle. ``Smith's''
utterances on these occasions were in (some form of) Sanskrit, since
Ferdinand was an authority on that language. Flournoy had also
consulted Ferdinand earlier in connection with his ideas about
syn{\ae}sthesia, and indeed they maintained cordially collegial
relations for much of their joint careers at the University of Geneva.
Apart from this, however, and the connections between the families
(Flournoy's daughter Ariane married Ferdinand's son Raymond in 1919),
there is little evidence for a close association specifically between
Flournoy and René, apart from one point: in 1909, Flournoy hosted the
International Congress of Psychology at the University of Geneva, and
René was one of the plenary speakers, speaking on the advantages of
Esperanto \pgcitep{joseph12:saussure}{561}.  René's dedication of the
1911 book, therefore, appears to reflect more in the way of general
respect for a notable figure in the science of the mind than a more
specific and more personal link.
{\sloppy\addsecmark{}
\printbibliography[heading=subbibliography,notkeyword=this]}
\end{refsection}
\fontsize{10pt}{12pt}\selectfont

% \rohead{}
% \lehead{}
\newpage\addchapmark{René de Saussure}\addsecmark{The 1911 Text}

\vspace*{2in}
\begin{center}
{\large PRINCIPES LOGIQUES}\\[2ex] 
DE LA\\[2ex] {\Large FORMATION DES MOTS}\\[2ex]
par\\[1ex]
  René DE SAUSSURE\\
  \emph{Privat-docent à l'Université de Genève}
\end{center}
%\raggedright
\begin{sloppypar}


\newpage
\vspace*{2in}
\begin{center}
  GENÈVE\\
  IMPRIMERIE ALBERT KÜNDIG
\end{center}
\vspace*{1in}
\begin{center}
  À M. le Professeur Th. Flournoy
\end{center}

\TextPage[no] {\begin{center}
    {\large PRINCIPES LOGIQUES}\\[2ex]
    {\footnotesize DE LA}\\[2ex]
    {\Large FORMATION DES MOTS}\vspace*{4ex} \rule{1in}{0.4pt}
  \end{center}
  \vspace*{4ex}

  Un mot est le symbole d’une idée. Les idées simples, ou considérées
  comme telles, sont représentées généralement par des \emph{mots
    simples}, tels que: «homme», «grand», «table», etc., c’est-à-dire
  par des mots indécomposables en plusieurs parties. Les idées plus
  complexes sont représentées par des \emph{mots composés}, tels que:
  «porte-plume» en français, «Dampfschiff» en allemand, etc., ou par
  des \emph{mots dérivés}, tels que: «grandeur», «humanité», etc.,
  décomposables en plusieurs parties («hum-an-ité»)

  On est donc naturellement conduit à examiner les deux questions
  suivantes:

  1° Etant donné un mot composé, quelle est l’idée complexe
  représentée par ce mot? C’est là le problème de \emph{l’analyse des
    mots composés}.} %
{\begin{center}
    {\large LOGICAL PRINCIPLES}\\[2ex]
    {\footnotesize OF THE}\\[2ex]
    {\Large FORMATION OF WORDS}\vspace*{4ex} \rule{1in}{0.4pt}
  \end{center}
  \vspace*{4ex}

  A word is the symbol of an idea. Ideas that are simple, or regarded
  as such, are generally represented by \emph{simple words,} such as
  \emph{homme} `man', \emph{grand} `tall', \emph{table} `table',
  etc. --- that is, by words that cannot be decomposed into several
  parts. More complex ideas are represented by \emph{compound words,}
  such as \emph{porte-plume} `penholder' in French, \emph{Dampfschiff}
  `steamship' in German, etc., or by \emph{derived words} such as
  \emph{grandeur} `size, height', \emph{humanité} `humanity',
  etc. which can be decomposed into several parts
  (\emph{hum-an-ité}).

  We are thus naturally led to examine the following two questions:

  1° Given a compound word, what is the complex idea that this word
  represents? This is the problem of \emph{the analysis of compound
    words}.}

\TextPage{2° Etant donnée une idée complexe, quel est le mot composé
  qui la représente? C’est le problème de la \emph{synthèse des mots}
  composés.

  Pour résoudre ce double problème, il faut des données; ces données
  sont les mots simples. Peu importe, du reste, la forme de ces mots
  simples: que l’on dise «homme», comme en français, ou «Mensch»,
  comme en allemand, pour symboliser l’idée «homme», cela ne modifie
  en rien les lois qui régissent la formation des mots. Peu importe
  aussi l’étymologie des mots simples; ces questions peuvent
  intéresser le linguiste, mais pour le logicien les mots simples sont
  des données conventionnelles analogues aux symboles mathématiques,
  et ce qui importe, c’est la définition de chaque symbole,
  c'est-à-dire l’\emph{idée} représentée par chaque mot simple.

  Les principes logiques de la formation des mots sont donc les mêmes
  pour toutes les langues, du moins pour toutes celles qui partent des
  mêmes éléments primitifs. Ainsi, dans nos langues européennes (les
  seules dont je m’occuperai), il y a deux sortes d’éléments
  primitifs: les \emph{mots-radicaux}, tels que: «homme», «grand»,
  etc., et les \emph{affixes}, tels que: «iste» (dans «violoniste»),
  «pré» (dans «prévenir»), etc. Au point de vue logique, il n’y a pas
  de différence essentielle entre un radical et un affixe; ceux-ci
  sont, du reste, souvent} {2° Given a complex idea, what is the
  compound word that represents it?  This is the problem of \emph{the
    synthesis of compound words}.

  To resolve this double problem, data are required: these data are
  the simple words. Apart from that, the form of the simple words does
  not matter: whether one says \emph{homme} `man' as in French or
  \emph{Mensch} as in German to symbolize the idea ``man'', it does
  not at all change the laws that govern the formation of words. The
  etymology of simple words also does not matter; those questions may
  interest the linguist, but for the logician simple words are
  conventional givens analogous to mathematical symbols, and what
  matters is the definition of each symbol, that is the \emph{idea}
  represented by each simple word.

  The logical principles of the formation of words are thus the same
  for all languages, or at least for all those that begin from the
  same primitive elements. Thus, in our European languages (the only
  ones with which I will be concerned), there are two kinds of
  primitive element: \emph{root words}, such as \emph{homme} `man',
  \emph{grand} `tall', etc., and \emph{affixes,} such as \emph{-iste}
  (in \emph{violoniste} `violinist'), \emph{pré} (in \emph{prévenir}
  `forwarn' [literally `precede']), etc. From the logical point of
  view, there is no essential difference between a root and an affix:
  these are often}


\TextPage{\protect\noindent d'anciens radicaux. Il est vrai que la
  soudure entre un affixe et un radical n’est pas, en général, de meme
  nature que la soudure entre deux radicaux, mais cela ne tient pas à
  une différence spécifique entre les affixes et les radicaux; cela
  tient à d’autres causes que nous examinerons plus loin.

  On peut donc considérer les affixes comme des mots simples, et les
  mots dérivés au moyen d’affixes, comme de véritables mots
  composés. Il n’y a plus alors que deux sortes de mots: les
  \emph{mots simples} (radicaux, préfixes, suffixes), et les
  \emph{mots composés} par combinaison de mots simples.

  On peut comparer un mot composé à une molécule construite au moyen
  de trois sortes d’atomes (radicaux, préfixes, suffixes); l’analyse
  et la synthèse logique des mots est alors comparable à l’étude d’une
  molécule dont les atomes sont connus, et le double problème que nous
  cherchons à résoudre peut s'énoncer: «Trouver l’idée exprimée par
  une molécule donnée», ou réciproquement «construire la molécule
  représentant une idée donnée».

  Or, la condition essentielle pour que ce problème soit susceptible
  d'une solution logique et précise est que \emph{les atomes}
  (radicaux, préfixes et suffixes) \emph{qui représentent les
    matériaux primitifs de la formation des mots soient des éléments
    absolument invariables et indépendants}, dont on connaît
  exactement le contenu individuel, c’est-à-dire qu’il faut}
{\protect\noindent former roots. It is true that the juncture between
  an affix and a root is not in general of the same type as the
  juncture between two roots, but that has nothing to do with a
  specific difference between affixes and roots; it has other causes
  that we will examine below.

  We can therefore consider affixes as simple words, and words derived
  by means of an affix as real compound words. There are then only two
  sorts of word: \emph{simple words} (roots, prefixes, suffixes) and
  \emph{compound words} formed by combining simple words.

  A compound word can be compared to a molecule built by means of
  three sorts of atoms (roots, prefixes, suffixes); the analysis of
  the logical synthesis of words is thus comparable to the study of a
  molecule of which the atoms are known, and the double problem which
  we are trying to solve can be formulated as ``Find the idea that a
  given molecule expresses'' or inversely ``construct the molecule
  that represents a given idea.''

  Now the essential condition for this problem to be subject to a
  logical and precise solution is that \emph{the atoms} (roots,
  prefixes, suffixes) \emph{that represent the basic material for word
    formation should be absolutely invariant and independent
    elements,} whose individual content is known exactly.  That is, it
  is necessary }

\TextPage{\protect\noindent que le sens et le contenu de chaque
  radical ou affixe reste toujours le même, quelles que soient les
  circonstances particulières où il se trouve. Cela signifie que dans
  une molécule comme, par exemple, «grandeur», composée de plusieurs
  atomes (radical «grand», suffixe «eur»), l’atome «grand» est
  exactement le même mot que l’adjectif «grand» considéré
  isolément\footnote{II est bien entendu que le principe de
    l’invariabilité des atomes se rapporte non à la forme extérieure,
    mais au sens de ces atomes. Ainsi, dans les mots \emph {homme,
      humain, humanité}, l’atome \emph{homme} se transforme en
    \emph{hum} et l’atome \emph{ain} devient \emph{an}; mais ces
    atomes, variables de forme, sont invariables de sens, c’est-à-dire
    que dans le mot \emph{hum-an-ité}, l’atome \emph{hum} est
    exactement le même mot que le substantif \emph{homme} considéré
    isolément. Les causes qui ont ici transformé les atomes réguliers
    \emph{homme, ain} en \emph{hum} et \emph{an} sont d’ordre purement
    physiologique et peuvent intéresser le philologue, non le
    logicien. Du reste, cette variation de forme des atomes ne se
    produit guère que dans les langues latines. Dans les langues
    germaniques, slaves, etc., les atomes restent presque toujours
    invariables de sens et de forme. Ex.: \emph{Mensch},
    \emph{mensch-lich}, \emph{Mensch-lich-keit}.}.


  Le but de la présente étude est précisément de montrer qu’à part
  quelques exceptions qui, du reste, ne sont qu’apparentes, il en est
  bien ainsi dans les langues naturelles et que, par conséquent, il
  est possible d’établir une théorie logique et précise du mécanisme
  de la formation des mots.}  %
{\protect\noindent that the sense and content of each root or affix
  should always remain the same, whatever the particular circumstances
  in which it is found.  This means that in a molecule such as, for
  example, \emph{grandeur} `size, height', composed of multiple atoms
  (root \emph{grand} `large, tall', suffix \emph{eur} `-ness'), the
  atom \emph{grand} is exactly the same word as the adjective
  \emph{grand} considered in isolation.\footnote{It is to be
    understood that the principle of the invariability of atoms
    relates not to the exterior form, but to the sense of these
    atoms. Thus, in the words \emph{homme} `man', \emph{humain}
    `human', \emph{humanité} `humanity', the atom \emph{homme} is
    transformed into \emph{hum} and the atom \emph{ain} becomes
    \emph{an}; but these atoms, while variable in form, are invariant
    in sense.  That is, in the word \emph{hum-an-ité}, the atom
    \emph{hum} is exactly the same word as the noun \emph{homme}
    considered by itself. The causes that have transformed the regular
    atoms \emph{homme, ain} into \emph{hum} and \emph{an} are of a
    purely physiological order, and may interest the philologist but
    not the logician. On the other hand, this variation in form of
    atoms is almost exclusive to the Romance languages.  In the
    Germanic, Slavic, etc. languages, the atoms are almost always
    invariable in sense and in form. E.g. \emph{Mensch} `man',
    \emph{mensch-lich} `human', \emph{Mensch-lich-keit} `humanity'.}

  The aim of the present work is precisely to show that apart from
  some exceptions which are, however, only apparent, this is indeed
  the case in natural languages, and that consequently it is possible
  to establish a logical and precise theory of the mechanism of word
  formation.  }

\TextPage [no] {\begin{center} CHAPITRE PREMIER
  \end{center}
  
\begin{center}
  \textbf{ANALYSE DES MOTS}
\end{center}\addcontentsline{toc}{section}{1 Analyse des mots}

Le problème principal à résoudre est le suivant: \emph{Etant donne un
  mot composé} (c'est-à-dire une combinaison de radicaux, de préfixes
et de suffixes), \emph{trouver l’idée totale représentée par ce mot}.

\textsc{Invariabilité des éléments}. — De même qu’un tout est
l’ensemble de ses parties, \emph{l’idée totale représentée par un mot
  composé est l’ensemble ou, si l’on veut, la résultante des idées
  partielles représentées par les différentes parties de ce
  mot}. Cette vérité semble évidente, mais il ne faut pas oublier
qu’elle présuppose \emph{l’invariabilité} de sens et
\emph{l’indépendance} des divers éléments ou atomes qui entrent dans
la composition du mot à analyser. L’analyse logique des mots n’est
possible que si les symboles sur lesquels on opère sont des éléments
invariables; ainsi le sens, la valeur d’un atome, ne doit dépendre que
de lui-même et nullement du sens ou de la valeur des atomes qui
l’environnent. On peut dire } %
{\begin{center}{FIRST CHAPTER}
  \end{center}

  \begin{center}{\textbf{THE ANALYSIS OF WORDS}}
  \end{center}

  The principal problem to be solved is the following: \emph{Given a
    compound word} (that is, a combination of roots, prefixes and
  suffixes), \emph{find the total idea this word represents.}

  \textsc{Invariability of the elements.} --- Just as a whole is the
  totality of its parts, \emph{the entire idea represented by a
    compound word is the totality, or if you will, the resultant of
    the partial ideas represented by the different parts of the word.}
  This truth seems obvious, but it is necessary not to forget that it
  presupposes the \emph{invariability} of sense and the
  \emph{independence} of the various elements that enter into the
  composition of the word to be analyzed. The logical analysis of
  words is only possible if the symbols with which we work are
  invariant elements; thus the sense, the value of an atom, must
  depend only on itself and not at all on the sense or the value of
  the atoms that surround it. It can be said }

\TextPage{\protect\noindent alors que \emph{le sens d’un mot composé
    ne dépend que de son propre contenu et de tout son contenu},
  c’est- à-dire du contenu de ses différentes parties considérées
  isolément.

  \textsc{Règles de dérivation}. — Il n’est donc pas besoin d’établir
  des \emph{règles de dérivation} reliant l’un à l’autre le sens des
  mots d’une même famille (comme «homme», «humain», «humanité»;
  «couronne», «couronner», «couronnement»), car on crée ainsi des
  liens artificiels entre des atomes qui devraient rester indépendants
  et interchangeables comme les différentes pièces d’une machine.

  Il faut chercher le sens logique d’un mot quelconque dans le mot
  lui-même et non pas dans la manière dont ce mot semble dérivé d’un
  autre mot. Dériver un mot d’un autre, c’est simplement ajouter un ou
  plusieurs atomes au mot primitif; par exemple, substantifier un
  adjectif, c’est ajouter à cet adjectif un atome contenant l’idée
  substantive; ainsi, en ajoutant au mot «homme» les atomes «ain» et
  «ité», on obtient le mot «humanité», dont le sens est connu dès que
  l’on connaît le sens et la valeur des trois atomes qui composent ce
  mot et sans que l’on ait à se préoccuper d’autre chose.  }
{\protect\noindent then that \emph{the sense of a compound word
    depends only on its own content and on all of its content,} that
  is on the content of its different parts considered in isolation.

  \textsc{Rules of derivation.} --- There is thus no need to establish
  \emph{rules of derivation} linking to each other the senses of words
  belonging to the same family (such as \emph{homme} `man',
  \emph{humain} `human', \emph{humanité} `humanity'; \emph{couronne}
  `crown (n.)', \emph{couronner} `(to) crown', \emph{couronnement}
  `coronation'), because that would create artificial links between
  atoms that must remain independent and interchangeable like the
  different parts of a machine.

  It is necessary to look for the logical sense of any word whatsoever
  in the word itself, and not in the way the word seems to be derived
  from another word. To derive one word from another is simply to add
  one or more atoms to the basic word; for example, to nominalize an
  adjective is to add to the adjective an atom that contains a nominal
  idea; thus, in adding to the word \emph{homme} the atoms \emph{ain}
  and \emph{ité}, we get the word \emph{humanité}, whose sense is
  known once we know the sense and the value of the three atoms which
  compose this word and without having to be concerned with anything
  else. }

\TextPage{\begin{center}§ 1. — Etude des atomes.\end{center}\addcontentsline{toc}{subsection}{1 Etude des atomes}

  Les \emph{atomes} sont les mots simples (radicaux, préfixes ou
  suffixes) qui constituent les éléments invariables au moyen desquels
  on construit les mots composés.

  Chaque mot simple représente une idée. Cette idée est plus ou moins
  particulière, plus ou moins générale\footnote{On peut dire en gros
    que les affixes représentent des idées plus générales que les
    radicaux. En effet, plus une idée est générale, plus elle est
    fréquente dans le discours. Les mots qui représentent les idées
    générales tendent donc à se transformer en suffixes ou préfixes,
    précisément à cause de leur fréquente répétition.}, mais les
  différentes idées ne sont pas indépendantes les unes des autres;
  elles ne sont pas juxtaposées comme des noix sur un bâton; elles
  forment des hiérarchies ou, plus exactement, elles s’emboîtent les
  unes dans les autres en allant du particulier au général,
  c’est-à-dire que toute idée particulière contient
  \emph{implicitement} en elle-même une série d’idées de plus en plus
  générales qu’elle entraîne à sa suite dès qu’on la touche. Un atome
  ne représente donc pas simplement une idée particulière isolée, mais
  toute une série d’idées plus générales, quoique celles-ci ne soient
  pas exprimées explicitement. Cette remarque est importante; c’est
  elle qui permet de considérer le sens d’un mot} %
{\begin{center}§ 1. — The study of atoms.\end{center}

  \emph{Atoms} are the simple words (roots, prefixes and suffixes)
  that constitute the invariable elements by means of which compound
  words are built.

  Every simple word represents an idea. This idea is more or less
  specific, more or less general,\footnote{We can say roughly that
    affixes represent more general ideas than roots. Effectively, the
    more general an idea, the more frequent it is in discourse. Words
    that represent general ideas thus tend to be transformed into
    suffixes or prefixes, precisely because of their frequent
    repetition.} but the different ideas are not independent of one
  another; they are not juxtaposed like beads on a string; they form
  hierarchies, or more precisely, they fit together with one another
  in passing from the specific to the general.  That is, every
  specific idea contains \emph{implicitly} in itself a series of more
  and more general ideas that it leads to as soon as one apprehends
  it. An atom thus does not represent simply an isolated specific
  idea, but an entire series of more general ideas, even though these
  are not explicitly expressed. This remark is important: it is this
  that makes it possible to consider the sense of a compound word }

\TextPage{\protect\noindent composé comme ne dépendant que du sens
  individuel de ses différents éléments, car nous verrons que les
  idées générales sous-entendues jouent, dans l’analyse des mots, un
  rôle aussi important que les idées particulières exprimées par les
  mots simples. Il est donc nécessaire de bien se rendre compte de
  tout ce que contient un atome, soit extérieurement, soit
  intérieurement.

  Prenons, par exemple, le mot radical «cheval»: ce mot représente une
  idée particulière; c’est la partie \emph{apparente} de l’atome. Mais
  cette idée particulière contient en elle-même d’autres idées plus
  générales. Ainsi, si nous nous plaçons, par exemple, au point de vue
  zoologique, l’idée «cheval» contient celle de «animal mammifère»,
  qui contient elle- même celle de «animal vertébré», qui contient à
  son tour celle de «animal», qui contient celle de «un être réel»
  (personne ou chose), qui contient enfin l’idée de «un être» tout
  court, de «quelque chose qui est, qui existe», soit réellement, soit
  idéellement. L’idée d’«un être» est tellement générale qu’elle n’en
  contient plus d’autres; c’est ce qu'on appelle \emph{l’idée
    substantive} .

  Ainsi, dire que le mot «cheval» est un substantif, c’est dire
  simplement que l’idée la plus générale sous-entendue dans le mot
  «cheval» est l’idée de «un être», de «quelque chose qui est». Mais
  il faut remarquer, à ce propos, que cette idée comprend }
{\protect\noindent as depending only on the individual sense of its
  different elements, because we will see that the understood general
  ideas play a role in the analysis of words that is as important as
  the specific ideas expressed by the simple words. It is thus
  necessary to take account of everything that is contained in an
  atom, either externally or internally.

  Let us take, for example, the root word \emph{cheval} `horse': this
  word represents a specific idea, which is the \emph{evident} part of
  the atom. But that specific idea contains other more general ideas
  within it. Thus, if we take for example the zoological point of
  view, the idea \emph{cheval} contains that of \emph{animal
    mammifère} `mammalian animal', which itself contains that of
  \emph{animal vertébré} `vertebrate animal', which in turn
  contains the that of \emph{animal} `animal', which contains that of
  \emph{un être réel} `an actual being' (person or thing), which
  finally contains the idea simply of \emph{un être} `a being',
  \emph{quelque chose qui est, qui existe} `something that is, that
  exists' either in reality or ideally. The idea of \emph{un être}
  is so general that it does not contain anything further: it is what
  we call \emph{the nominal idea}.

  Thus, to say that the word \emph{cheval} is a noun is simply to say
  that the most general idea understood in the word \emph{cheval} is
  the idea of \emph{un être, quelque chose qui est}. But it is
  necessary to note in that connection that this idea includes }

\TextPage{\protect\noindent non seulement les êtres réels, ou concrets
  de nature, (comme «homme», «table», etc.), mais aussi les êtres
  idéels, ou abstraits de nature\footnote{ J’emploie les mots «réel»
    et «idéel», car on ne peut pas classer les substantifs d’après les
    notions de «concret» et d’«abstrait», ces notions n’ayant qu’une
    valeur relative (comme les notions «particulier» et «général»),
    puisque le même mot peut toujours être pris au sens concret et au
    sens abstrait. On peut dire, toutefois, que les êtres \emph{réels}
    (personne ou chose) sont concrets de nature; leur sens primitif
    est concret et le sens abstrait n’est que dérivé; au contraire,
    pour les êtres \emph{idéels}, le sens primitif est abstrait et le
    sens concret n’est que dérivé.}, (comme, par exemple, «théorie»,
  «genre», «science»), c’est-à- dire les êtres de raison créés par
  l’homme, qui les a abstraits de la réalité en vue du langage. En
  d’autres mots, le substantif ne correspond pas seulement aux êtres
  qui forment la substance du Cosmos, mais aussi à ceux qui forment
  celle du \mbox{langage.\footnotemark}\footnotetext{L’idée substantive peut elle-même être
    généralisée, comme toute idée particulière; l’idée d’«un être»
    particulier, réel ou idéel, contient encore en elle-même l’idée
    plus générale de «l’être» en général, «l’être abstrait»,
    «l’exister», tout comme l’idée particulière «homme» contient
    l’idée plus générale de «l’homme en général», «l’homme» au sens
    abstrait. On peut donc distinguer l’idée substantive particulière
    «un être» et l’idée substantive générale «l’être»; la première
    contient la seconde; donc la dernière idée qui, à cause de sa
    généralité, n’en contient plus aucune autre, est l’idée
    substantive générale de «l’être en général», «l’être abstrait»} }
{\protect\noindent not only entities that are real, or concrete by
  nature (such as \emph{homme} `man', \emph{table} `table', etc.) but
  also entities that are ideal, or abstract in their nature\footnote{I
    use the words ``real'' and ``ideal'' because we cannot classify
    nouns by the notions of ``concrete'' and ``abstract'', notions
    that have only relative value (like the notions ``specific'' and
    ``general''), since the same word can always be taken in a
    concrete sense and in an abstract sense. We can say, however, that
    \emph{real} beings (persons or things) are concrete in their
    nature; their basic sense is concrete and the abstract sense is
    only derived. On the other hand, for \emph{ideal} beings the basic
    sense is abstract and the concrete sense is only derived.} (such
  as for instance \emph{théorie} `theory', \emph{genre} `type',
  \emph{science} `science'), that is mental entities created by man,
  who has abstracted them from reality for the purposes of
  language. In other words, nouns correspond not only to the entities
  that form the substance of the Cosmos, but also to those that form
  that of language.\footnote{The nominal idea can itself be
    generalized, like any specific idea: the idea of ``an entity'',
    real or ideal, contains in itself the more general idea of ``an
    entity'' in general, ``abstract entity, existant'', just as the
    specific idea ``man'' contains the more general idea of ``man in
    general'', ``man'' in the abstract sense. We can thus distinguish
    the specific nominal idea ``an entity'' from the general nominal
    idea ``entity''; the first contains the second, and thus the final
    idea which by reason of its generality contains no other is the
    general nominal idea of ``entity in general, abstract entity''.}
}

\TextPage{ L’idée substantive, l’idée adjective et l’idée verbale sont
  donc des idées tout à fait semblables aux autres idées; ce sont
  seulement celles de nos idées qui sont \emph{les plus générales} et,
  par conséquent, les plus abstraites. A ce titre, et à ce titre
  seulement, elles méritent une dénomination spéciale: je les
  appellerai les \emph{ idées grammaticales}. Ces idées sont
  évidemment abstraites; l’idée adjective, par exemple, est l’idée
  générale abstraite des adjectifs particuliers.

  Nous venons de voir que lorsqu’on examine les idées de plus en plus
  générales contenues dans un mot simple comme «cheval», on arrive
  finalement à une idée grammaticale. \emph{Cette idée caractérise le
    mot considéré}, c’est-à-dire que le même mot conduit toujours à la
  même idée grammaticale, quelle que soit la série des idées
  intermédiaires que l’on interpose. Ainsi, au lieu de considérer un
  «cheval» comme un animal vertébré, on peut le considérer comme un
  animal quadrupède, par exemple; le mot «cheval» n’en restera pas
  moins substantif, car l’idée «animal quadrupède» est aussi
  substantive; elle contient l’idée de «animal» et, par conséquent,
  aussi celle de «un être réel», et enfin celles de «un être» tout
  court et de «l’être» en général.

  On arrive donc au même résultat final, c’est-à-dire à la même idée
  grammaticale, que l’on considère l’une ou l’autre des deux séries
  d’idées sous-entendues: } %
{The nominal idea, the adjectival idea and the verbal idea are thus
  ideas completely like other ideas: they are simply those of our
  ideas that are \emph{the most general} and as a consequence, the
  most abstract. On that basis, and on that basis alone, they deserve
  a special terminology: I will call them the \emph{grammatical
    ideas.} These ideas are obviously abstract: the adjectival idea,
  for example, is the general idea abstracted from specific
  adjectives.

  We have just seen that when we examine the increasingly general
  ideas contained within a simple word like \emph{cheval} `horse', we
  arrive in the end at a grammatical idea. \emph{This idea
    characterizes the word under consideration,} that is the same word
  leads always to the same grammatical idea, regardless of the series
  of intermediate ideas that we put in between. Thus, instead of
  considering a ``horse'' as a vertebrate, we could consider it as a
  quadruped, for example: the word \emph{cheval} would remain
  nonetheless a noun, since the idea ``quadruped animal'' is also
  nominal: it contains the idea ``animal'' and consequently ``real
  entity'', and finally just ``an entity'' and ``entity'' in general.

  We thus arrive at the same final result, that is at the same
  grammatical idea, whether we consider the one or the other of the
  two series of understood ideas: }

\TextPage{
  \begin{center}
    \begin{tabular}[t]{cc}
      cheval & cheval\\
      \multicolumn{1}{l}{\emph{(animal}} & \multicolumn{1}{l}{\emph{(animal}}\\
      \multicolumn{1}{r}{\emph{mammifère)}} &
                                                \multicolumn{1}{r}{\emph{quadrupède)}}\\
      \emph{(animal vertébré)} & \emph{(animal)}\\
      \emph{(animal)} & \emph{(un être réel)}\\
      \emph{(un être réel)} & \emph{(un être)}\\
      \emph{(un être)} & \emph{(l'être)}\\
      \emph{(l'être)} & \\
    \end{tabular}             
  \end{center}

  Cette remarque montre que \emph{l’analyse des mots est indépendante
    de la manière dont on subdivise les idées}; elle est donc aussi
  indépendante des diverses théories scientifiques ou philosophiques,
  et dans chaque cas particulier, on emploiera la subdivision qui
  convient le mieux au point de vue auquel on s’est momentanément
  placé.

  La seule condition nécessaire (et qui d’ailleurs est forcément
  remplie) est que toutes les idées intercalées entre une idée
  particulière et l’idée grammaticale correspondante, contiennent
  elles-mêmes cette idée grammaticale. Autrement dit, si l’idée
  particulière donnée est, par exemple, substantive, toutes les idées
  plus générales intercalées entre cette idée et l’idée substantive
  sont forcément représentées aussi par des substantifs. C’est
  pourquoi j’ai écrit sous le mot «cheval»: «animal mammifère»,
  «animal vertébré», etc., et non pas simplement «mammifère»,
  «vertébré», etc., car ces mots sont des adjectifs.  } %
{
  \begin{center}
    \begin{tabular}[t]{cc}
      horse & horse\\
      \multicolumn{1}{l}{\emph{(mammalian}} &
                                              \multicolumn{1}{l}{\emph{(quadruped}}\\
      \multicolumn{1}{r}{\emph{animal)}} & \multicolumn{1}{r}{\emph{animal)}}\\
      \emph{(vertebrate} & \emph{(animal)}\\
      \multicolumn{1}{r}{\emph{animal)}} & \\
      \emph{(animal)} & \emph{(a real entity)}\\
      \emph{(a real entity)} & \emph{(an entity)}\\
      \emph{(an entity)} & \emph{(entity)}\\
      \emph{(entity)} & \\
    \end{tabular}  
  \end{center}

  This observation shows that \emph{the analysis of words is
    independent of the way we subdivide the ideas}; it is thus also
  independent of the various scientific and philosophical theories,
  and in each specific case we make use of the subdivision that is
  most suitable from the point of view taken at the moment.

  The only necessary condition (which, however, is necessarily
  fulfilled) is that all of the ideas interposed between a specific
  idea and the corresponding grammatical idea should themselves
  contain that grammatical idea. In other words, if the given specific
  idea is, for example, nominal, all of the more general ideas by
  which one passes from that idea and the nominal idea should also
  necessarily be represented by nouns. This is why I have written
  under the word \emph{horse} ``mammalian animal'', ``vertebrate
  animal, etc. and not just ``mammalian'', ``vertebrate'', etc., since
  these words are adjectives.  }

\TextPage{ Ce qu’il faut surtout ne pas oublier, c’est que pour
  l’analyse des mots, \emph{ce sont les idées générales qui sont
    contenues implicitement dans les idées particulières et non pas
    les idées particulières qui sont contenues dans les idées
    générales}, comme on pourrait quelquefois être tenté de le
  croire. Ainsi, par exemple, c’est l’idée «cheval» qui implique
  l’idée de «animal» et non pas l'idée «animal» qui implique l’idée
  «cheval», car tous les chevaux sont des animaux tandis que tous les
  animaux ne sont pas des chevaux.

  Dire que l’idée «cheval» contient celle de «animal», cela signifie
  qu’on n’ajoute rien à l’idée «cheval» en disant «cheval animal»; au
  contraire, en disant «animal cheval», on ajoute à l’idée «animal»
  une nouvelle idée qui spécialise la première, car elle signifie
  «animal, espèce particulière cheval»; on ne doit donc pas considérer
  l’idée «cheval» comme impliquée dans celle de «animal».

  En résumé, on peut comparer le dispositif des idées à une carte
  géographique: représentons les idées grammaticales par des pays
  indépendants, par exemple l’idée substantive par la France, l’idée
  adjective par la Grande-Bretagne et l’idée verbale par
  l’Allemagne. Alors toute idée substantive sera représentée par un
  endroit ou une région de la France; cet endroit étant d’autant plus
  petit que}%
{What must especially not be forgotten is that for the analysis of
  words, \emph{it is the general ideas that are contained implicitly
    in specific ideas, and not the specific ideas that are contained
    in general ideas,} as one is sometimes tempted to believe. Thus,
  for example, it is the idea ``horse'' that implies the idea
  ``animal'', and not the idea ``animal'' that implies the idea
  ``horse'', because all horses are animals but not all animals are
  horses.

  To say that the idea ``horse'' contains that of ``animal'' means
  that one adds nothing to the idea ``horse'' by saying ``animal
  horse'': on the contrary, in saying ``animal horse'' one adds to the
  idea ``animal'' a new idea that specializes it, because this means
  ``animal of the particular species horse''; one must thus not
  consider the idea ``horse'' as implied in that of ``animal''.

  In sum, we can compare the system of ideas to a geographic map: we
  represent the grammatical ideas by independent countries, for
  example the nominal idea by France, the adjectival idea by Great
  Britain and the verbal idea by Germany. Then every nominal idea will
  be represented by a place or region of France; and the smaller the place, }

\TextPage{\noindent l’idée en question est plus particulière; ainsi
  les villages, les bourgs, les villes de France pourront figurer les
  idées substantives les plus particulières, tandis que les communes,
  les départements, les provinces, etc., figureront les idées
  substantives plus générales.

  De même que l’idée «cheval» contient l’idée substantive de «un
  être», quelles que soient les idées intermédiaires intercalées, de
  même toute ville française, comme «Caen», contient l’idée «France»
  quelle que soit la manière dont on subdivise ce pays: si l’on divise
  la France en provinces, l’idée «Caen» contient l’idée «Normandie»;
  si on la divise en départements, l’idée «Caen» contient l’idée
  «Calvados»; mais, dans les deux cas, l’idée «France» reste contenue
  dans «Caen», parce que soit la Normandie, soit le Calvados sont des
  subdivisions de la France, et les deux schémas:

  \begin{center}
    \begin{tabular}[t]{cc}
      Caen & Caen \\
      \emph{(Normandie)} & \emph{(Calvados)} \\
      \emph{(France)} & \emph{(France)}
    \end{tabular}
  \end{center}

  \noindent sont analogues aux deux schémas que nous avons construits
  pour le mot «cheval». L’analyse des mots est indépendante de la
  manière dont on subdivise les idées.  } %
{\noindent the more specific the idea; thus the villages,
  towns, cities of France can represent the most specific nominal
  ideas, while the communes, the departments, the provinces will
  represent more general nominal ideas.

  Just as the idea ``horse'' contains the nominal idea of ``an
  entity'' whatever intermediate ideas come between them, so every
  French city, such as ``Caen'' contains the idea ``France'' in
  whatever way we subdivide this country.  If we divide France into
  provinces, the idea ``Caen'' contains the idea ``Normandy''; if we
  divide it into departments, the idea ``Caen'' contains the idea
  ``Calvados''; but in both cases the idea ``France'' remains
  contained in ``Caen'', because either Normandy or Calvados are
  subdivisions of France, and the two patterns:

  \begin{center}
    \begin{tabular}[t]{cc}
      Caen & Caen \\
      \emph{(Normandy)} & \emph{(Calvados)} \\
      \emph{(France)} & \emph{(France)}
    \end{tabular}
  \end{center}

  \noindent are analogous to the two patterns we constructed for the
  word \emph{cheval} `horse'.  The analysis of words is independent of
  the way in which we subdivide the ideas.  }

\TextPage{
  \begin{center}
    *\\[1ex]
    {*}\hspace{1em}{*}\\
  \end{center}

  \textsc{Classement des atomes}. — On peut classer tous les atomes
  (radicaux ou affixes) suivant la nature de l’idée la plus générale
  contenue dans chacun d’eux.

  Si cette idée la plus générale est celle de «l’être», de «ce qui
  est» (idée substantive), l’atome sera classé comme
  \emph{substantif}. Par exemple, «cheval» est un atome substantif
  d’après l’analyse faite ci-dessus. Le suffixe «iste» (dans
  «violoniste», «artiste», etc.) est aussi un atome substantif, car ce
  suffixe désigne une «personne» (dont la profession ou l’occupation
  habituelle est caractérisée par le radical auquel il est accolé); ce
  suffixe contient donc implicitement l’idée de «un être vivant», idée
  qui contient à son tour celle de «un être» tout court et de «l’être»
  en général (idée substantive).

  Lorsque l’idée la plus générale contenue dans un atome est une idée
  «qualificative», l’atome sera classé comme \emph{adjectif} car
  l’adjectif \emph{qualifie} le substantif. L’idée adjective abstraite
  est donc l’idée exprimée par le mot-radical «qual»\footnote{Du latin
    «qualis» (quel), d’où dérive le substantif «qual-ité». par
    opposition à «quantum» (combien), d’où dérive le substantif
    «quant-ité». On arrive ainsi à exprimer l’idée générale adjective
    par un atome irréductible.}  ou par le mot}%
{
  \begin{center}
    *\\[1ex]
    {*}\hspace{1em}{*}\\
  \end{center}
    
  \textsc{Classification of atoms.} --- We can classify all atoms
  (roots and affixes) according to the nature of the most general idea
  that each contains.

  If that most general idea is that of ``entity'', of ``that which
  is'' (the nominal idea), the atom will be classified as a
  \emph{noun}. For example, \emph{cheval} `horse' is a nominal atom
  according to the analysis given above. The suffix \emph{iste} `-ist'
  (in \emph{violoniste} `violinist', \emph{artiste} `artist', etc.) is
  also a nominal atom, because this suffix designates a ``person''
  (whose profession or habitual occupation is specified by the root to
  which it is attached); this suffix thus implicitly contains the idea
  of ``a living being'', an idea which contains in turn that of simply
  ``an entity'' and of ``entity'' in general (the nominal idea).

  When the most general idea contained in an atom is that of
  ``qualifying'', the atom will be classified as an \emph{adjective,}
  because adjectives \emph{qualify} nouns. The abstract adjectival
  idea is thus the idea expressed by the root word
  ``qual''\footnote{From Latin \emph{qualis} `what sort', from which
    the noun \emph{qual-ité} `quality' derives, as opposed to
    \emph{quantum} `how many', from which the noun \emph{quant-ité}
    `quantity' derives. We manage thus to express the general
    adjectival idea by an irreducible atom.} or by the word }

\TextPage{\noindent «propre», dans le sens\footnote{Il n’y a, en
    effet, guère de différence entre les «qualités» et les
    «propriétés» d’un être. Elles désignent «ce qui est qual» dans cet
    être, ou «ce qui est propre» à cet être. En allemand, «propre» se
    dit «eigen» et l’\emph{adjectif} est dénommé « Eigenschaftswort».}
  de «propre à», «propre à un être»\footnote{Dans l’expression «propre
    à un être», l’idée adjective est exprimée par le seul atome
    «propre»; le reste n’est qu’explicatif et indique simplement
    comment l'adjectif «propre» doit être uni au substantif, à l’être
    qu’il qualifie.}.

  Pareillement aux atomes substantifs, les atomes adjectifs ont tantôt
  la forme de radicaux, comme «grand», «riche», «sage», etc., tantôt
  la forme de suffixes, comme l’atome «able» (dans «louable»). En
  effet, «able» signifie «pouvant (être)», «digne (d’être)» [loué]; ce
  suffixe contient donc bien une idée qualificative, car «pouvant»,
  «digne», sont des adjectifs.

  Enfin, si l'idée la plus générale contenue dans un atome est l’idée
  \emph{dynamique} de «faire une action» ou l’idée \emph{statique} de
  «être dans un état», l’atome sera classé comme
  \emph{verbal}\footnote{Dans toute cette étude, je n’emploierai le
    mot verbal que comme adjectif du mot «verbe» opposé à «substantif»
    ou «adjectif», et non du mot «verbe» dans le sens de «parole»
    ({\selectlanguage{greek}λογος}).}. On peut représenter ces deux
  formes de l’idée verbale par les simples mots «agir» (ou «faire») et
  «être» (au sens statique, en }%
{\noindent ``characteristic, proper'' in the sense\footnote{In fact,
    there is hardly any difference between the ``qualities'' and the
    ``properties'' of an entity. They designate ``that which is qual''
    in this entity, or ``that which is characteristic'' of this
    entity. In German, ``characteristic'' is \emph{eigen} and the
    \emph{adjective} is called \emph{Eigenschaftswort}.} of
  ``characteristic of, proper to'', ``characteristic of an
  entity''\footnote{In the expression ``proper to an entity'', the
    adjectival idea is expressed by the single atom ``proper''; the
    rest is only explanatory and simply indicates how the adjective
    ``proper'' must be linked with the noun, with the entity which it
    qualifies.}.

  Parallel to nominal atoms, adjectival atoms have sometimes the form
  of roots, such as \emph{grand} `large, tall', \emph{riche} `rich',
  \emph{sage} `wise', etc., and sometimes the form of suffixes, such
  as the atom \emph{able} (in \emph{louable} `commendable'). Actually,
  \emph{able} means ``capable (of being)'', ``worthy (of being)''
  {[praised]}.  This suffix thus does contain a qualifying idea, since
  ``capable'', ``worthy'', are adjectives.

  Finally, if the most general idea contained in an atom is the
  \emph{dynamic} idea of ``perform an action'' or the \emph{static}
  idea of ``be in a state'', the atom will be classed as
  \emph{verbal}\footnote{Throughout this work, I will only use the
    word verbal as the adjective related to the word ``verb'' as
    opposed to ``noun'' or ``adjective'', and not to the homophonous
    French word with the sense ``language''
    ({\selectlanguage{greek}λογος}).}. We can represent these two
  forms of the verbal idea by the simple words ``(to) act'' (or ``(to)
  do'') and ``(to) be'' (in the static sense of }
  
\TextPage{\noindent latin \emph{stare}), mais ces mots contiennent
  encore des terminaisons verbales «ir» et «re» qui indiquent
  seulement l’infinitif, de sorte qu'elles sont inutiles au point de
  vue logique; pour exprimer l’idée contenue dans ces mots, les
  radicaux «ag» (ou «fai») et «sta» (ou «êt») suffisent. On arrive
  ainsi à représenter l'idée verbale abstraite par l'atome
  irréductible «ag» pour les verbes actifs et «sta» pour les verbes
  neutres. Nous montrerons, du reste, plus loin que l’idée «faire une
  action» se réduit à l’atome «ag» (agir) et que l’idée «être dans un
  état» ou, littéralement, dans une «station», se réduit à l’atome
  «sta» \mbox{(\emph{stare})\footnotemark}\footnotetext{Il y a le même rapport logique entre
    «état» (ou «estat») et «station» (ou «estation») qu’entre «acte»
    et «action». Du reste, en anglais, on dit régulièrement «State»,
    «station» et «act», «action».}.
  
  De même que les atomes substantifs ou les atomes adjectifs, les
  atomes verbaux sont tantôt des radicaux, comme «abonn» (abonner),
  «écri» (écrire), «dorm» (dormir), etc., tantôt des suffixes, comme
  «is» (dans «modern-is-er»), ou «ifi» (dans «béat-ifi-er), etc. Ces
  suffixes contiennent, en effet, une idée dynamique: «moderniser»
  signifie «rendre moderne», «béatifier», «rendre béat».

  Nous sommes donc naturellement amenés à classer les atomes (radicaux
  et affixes) suivant trois classes principales: la classe des atomes
  substantifs, }%
{\noindent Latin \emph{stare}), but these words still contain the
  verbal endings \emph{ir} and \emph{re} which indicate only the
  infinitive, such that they have no use from the perspective of
  logic: to express the idea contained in these words, the roots
  \emph{ag} `act' (or \emph{fai} `do') and \emph{sta} (or \emph{êt})
  `be' are sufficient. We thus come to represent the abstract verbal
  idea by the irreducible atom \emph{act} for active verbs and
  \emph{be} for neuter (stative) verbs. We will show below, besides,
  that the idea ``to perform an action'' reduces to the atom
  \emph{act} (to act) and that the idea ``to be in a state'', or
  literally, in a ``station'', reduces to the atom \emph{be}
  (\emph{stare})\footnote{There is the same logical relation between
    \emph{état} (or \emph{estat}) `state' and \emph{station} (or
    \emph{estation})`station' as between \emph{acte} `act' and
    \emph{action} `action'. Furthermore, in English one says regularly
    ``state'', ``station'' and ``act'', ``action''.}.

  Just as with nominal atoms and adjectival atoms, verbal atoms are
  sometimes roots, such as \emph{abonn} (\emph{abonner} `to
  subscribe'), \emph{écri} (\emph{écrire} `to write'), \emph{dorm}
  (\emph{dormir} `to sleep'), etc., and sometimes suffixes, such as
  \emph{is} `-ize' (in \emph{modern-is-er} `to modernize'), or
  \emph{ifi} `-ify' (in \emph{béat-ifi-er} `to beatify'), etc. These
  suffixes actually contain a dynamic idea: \emph{moderniser} means
  `to make modern', \emph{béatifier} `to make holy'.

  We are thus led naturally to classify atoms (roots and affixes) as
  belonging to three main classes: the class of nominal atoms, }

\TextPage{\noindent qui contiennent implicitement l'idée de «l’être»
  ou «ce qui est»; celle des atomes adjectifs, qui contiennent l'idée
  «qual» ou «propre (à)», et celle des atomes verbaux, qui contiennent
  l'idée «ag» ou«sta».
  
  Ceci revient à considérer tous les atomes substantifs comme des cas
  particuliers de l’atome substantif général «l’(être)», «un être», en
  latin \emph{ens}, ou «ce (qui est)»; tous les atomes adjectifs comme
  des cas particuliers de l’atome adjectif général «qual» ou «propre
  (à)», et tous les atomes verbaux comme des cas particuliers de
  l'atome verbal général «ag» ou «sta». On peut donc établir une
  classification des mots simples en trois colonnes correspondant
  respectivement aux trois rubriques: «ens», «qual» et «ag». Cette
  classification est indispensable pour l’analyse et la synthèse
  logique des mots. Elle a pour but d’associer à l’idée particulière
  exprimée explicitement par un atome, une idée grammaticale
  (implicite) et \emph{une seule}.

  Il y a, en effet, des cas où l’on serait tenté d’attribuer à un même
  atome deux idées générales différentes: on prend souvent, par
  exemple, un adjectif dans un sens substantif, en disant «un riche»
  pour «homme riche», «le beau» pour «l’être idéel \emph{beau}»,
  «abstraction \emph{beau}» (beauté), etc. Mais il est bien évident
  que dans ces expressions l'article «un» est l’atome
  substantificateur, car il remplace un substantif sous-entendu; on
  dit, par exemple, }%
{\noindent which contain implicitly the idea of ``entity'' or ``that
  which is''; that of adjectival atoms, which contain the idea
  ``qual'' or ``property of'', and that of verbal atoms, which contain
  the idea ``act'' or ``be''.

  This comes down to considering all nominal atoms as special cases of
  the general nominal atom ``entity'', ``an entity'', in Latin
  \emph{ens}, or ``that which is''; all adjectival atoms as special
  cases of the general adjectival atom ``qual'' or ``property (of)'',
  and all verbal atoms as special cases of the general verbal atom
  ``act'' or ``be''. We can thus set up a classification of simple
  words in three columns corresponding respectively to the three
  rubrics ``\emph{ens}'', ``qual'' and ``act''. This classification is
  indispensable for the logical analysis and synthesis of words. Its
  goal is to associate with the specific idea expressed explicitly by
  an atom an (implicit) grammatical idea and \emph{only one}.

  There are actually cases in which we would be tempted to attribute
  to the same atom two different general ideas: an adjective is often
  used, for example, in a nominal sense, as when we say \emph{un
    riche} `a rich (person)' for \emph{homme riche} `rich man',
  \emph{le beau} `the beautiful' for \emph{l'être idéel
    \underline{beau}}, \emph{abstraction \underline{beau}}
  (\emph{beauté} `beauty'), etc. But it is quite obvious that in
  these expressions the article \emph{un} `a' is the nominalizing
  atom, because it replaces an understood noun; we say, for example, }

\TextPage{\noindent «\emph{un} avare» (pour «homme avare»), «\emph{un}
  vertébré» (pour animal vertébré»), «\emph{un} désert» (pour «lieu
  désert»), «\emph{un} vide» (pour «espace vide»), etc. Ces formes
  échappent à l’analyse logique, parce que l'article «un» ne peut
  représenter par lui-même que l'idée substantive «un être» (réel ou
  idéel), et non pas un être particulier comme «homme», «lieu», etc.;
  l'analyse logique n’est possible que si l’on rétablit le substantif
  sous-entendu. Si, au contraire, on emploie l’article défini «le»
  devant un adjectif, l'analyse logique est facile, car «le» équivaut
  à «l’(être)» en général, «l’être idéel», abstrait de la réalité, par
  exemple «\emph{le} beau», «\emph{le} noir», etc.; de même dans les
  expressions substantives tirées de verbes, comme «\emph{le} manger»,
  «\emph{le} boire», «\emph{le} dormir»\footnote{Pour trouver la vraie
    signification d’un mot, simple ou composé, il faut considérer ce
    mot \emph{isolément}, sans y ajouter ni article, ni autre
    chose. On voit alors clairement que les atomes «beau», «riche»,
    etc., sont des atomes adjectifs. Il y a cependant quelques cas
    douteux (comme le mot «logique» par exemple), que nous examinerons
    plus loin.

    De même, quand nous disons que «le beau» signifie «l’être idéel,
    l’être abstrait beau» ou «beauté», nous entendons par là
    l’expression «le beau» prise isolément et sans contexte, car il
    est bien entendu que si dans le contexte on a parlé d’un «homme
    beau» ou du «beau temps», l’expression «le beau» pourrait se
    rapporter à cet homme ou au temps. En résumé, lorsqu’il n’y a pas
    de contexte, l’article «le» ne peut signifier que «l'être» en
    général et l’article «un», «un être quelconque» (réel ou
    idéel). }.  }%
{\noindent \emph{\und{un} avare} `\emph{a} miser' (for \emph{homme
    avare} `miserly man'), \emph{\und{un} vertébré} `\emph{a}
  vertebrate' (for \emph{animal vetébré} `vertebrate animal'),
  \emph{\und{un} désert} `\emph{a} desert' (for \emph{lieu désert}
  `deserted place'), \emph{\und{un} vide} `\emph{an} emptiness' (for
  \emph{éspace vide} `empty space'), etc. These forms elude logical
  analysis, because the article \emph{un} `a' by itself can only
  represent the nominal idea \og an entity\fg{} (real or ideal), and
  not a specific entity such as \og man,\fg{} \og place,\fg{} etc. The
  logical analysis is only possible by re-establishing the understood
  noun. If by contrast the definite article \emph{le} `the' is used
  before an adjective, the logical analysis is straightforward,
  because \emph{le} is equivalent to \emph{l'être} `the entity' in
  general, the ideal entity, abstracted from reality, such as
  \emph{\und{le} beau} `\emph{the} beautiful', \emph{\und{le} noir}
  `\emph{the} black', etc.; similarly for nominal expressions derived
  from verbs, such as \emph{\und{le} manger} `(the) food',
  \emph{\und{le} boire} `(the) drink', \emph{\und{le} dormir} `(the)
  sleep'.\footnote{To determine the true meaning of a word, simple or
    compound, it is necessary to consider the word \emph{in
      isolation,} without adding an article or anything else. It is
    then clearly to be seen that the atoms \emph{beau} `beautiful',
    \emph{riche} `rich', etc. are adjectival atoms. There are,
    however, some doubtful cases (like the word \emph{logique}
    `logic(al)' for example) which we will examine below.

    Similarly, when we say that \emph{le beau} means `the ideal
    entity, the abstract beautiful entity' or `beauty', we mean
    thereby the expression \emph{le beau} taken in isolation and
    without context, because it is clear that if in context there has
    been talk of an \emph{homme beau} `handsome man' or of \emph{beau
      temps} `nice weather', the expression \emph{le beau} can refer
    to that man or that weather. In brief, when there is no context,
    the article \emph{le} can only mean `the entity' in general, and
    the article \emph{un} `any entity' (real or ideal).}  }

\TextPage{Il n’y a pas ici d’idées particulières sous-entendues;
  l’article «le» ne contient que l’idée substantive générale, donc
  abstraite, et \emph{en vertu du principe de l’invariabilité des
    atomes}, l’expression «le beau» doit être considérée comme un mot
  composé, c’est-à-dire une molécule bi-atomique équivalente à
  «beau-té»; les affixes «le» ou «té» sont en effet tous deux des
  atomes ne représentant que l’idée substantive générale, «l’être en
  général», «l’être» abstrait de la réalité. On peut dire que le mot
  «beauté» est une molécule \emph{condensée} (forme synthétique),
  tandis que l’expression «le beau» est une molécule \emph{dissociée}
  (forme analytique). Nous verrons quels sont les rapports qui
  existent entre une molécule dissociée et la même molécule à l’état
  condensé; pour le moment, remarquons seulement que le principe de
  l’invariabilité des atomes s’applique aux molécules dissociées comme
  aux molécules condensées, c’est-à-dire que dans la molécule «le
  beau», comme dans «beau-té», l’atome «beau» est et reste toujours
  purement adjectif; cet atome ne contient en lui-même que l’idée
  adjective «qual»; ce n’est pas un être, mais seulement l’attribut
  propre à un être.

  On comprend donc quelle importance a le classement grammatical des
  atomes, puisque l’idée générale contenue dans un atome dépend de ce
  classement, donc aussi le sens interne de l’atome.

  Ce classement a un caractère plus ou moins arbi{[traire:]} }%
{There are no specific ideas assumed here: the article \emph{le}
  `the' contains only the general, and therefore abstract, nominal
  idea, and \emph{by virtue of the principle of the invariability of
    atoms,} the expression \emph{le beau} `the beautiful' must be
  considered as a compound word, that is, as a bi-atomic molecule
  equivalent to \emph{beau-té} `beau-ty'.  The affixes \emph{le} and
  \emph{té} are both effectively atoms that represent only the
  general nominal idea, that of `entity in general,' `entity'
  abstracted from reality. We can say that the word \emph{beauté} is
  a \emph{condensed} molecule (the synthetic form), while the
  expression \emph{le beau} is a \emph{dissociated} molecule (the
  analytic form). We will see what relations there are between a
  dissociated molecule and the same molecule in the condensed state;
  for the moment, let us only note that the principle of the
  invariability of atoms is applicable to dissociated molecules just
  as it is to condensed molecules, that is, in the molecule \emph{le
    beau} as in \emph{beau-té}, the atom \emph{beau} is and remains
  purely adjectival.  This atom contains in itself only the adjectival
  idea ``qual''; it is not an entity, but only an attribute proper to
  an entity.

  We thus see how important the classification of atoms is, since the
  general idea contained in an atom depends on that classification and
  thus also the internal sense of the atom.

  This classification has a character that is more or less
  arbi[trary:]

}

\TextPage{\noindent {[arbi]}traire: ainsi le mot «gaité» est dérivé de
  l’adjectif «gai»; le mot «joie» au contraire est un radical
  substantif; mais une fois ce classement fait, il doit être
  définitif, puisque le sens des mots dérivés où cet atome entre en
  jeu dépend en partie de ce classement.

  \textsc{Les Atomes fondamentaux}. — Avant de classer tous les atomes
  particuliers suivant l'idée générale (grammaticale) contenue dans
  chacun d’eux, il est bon de mettre à part tous les atomes qui ne
  contiennent pas d'idée particulière, mais qui contiennent seulement
  une des trois idées grammaticales (substantive, adjective ou
  verbale), car les trois atomes «le» (l’être), «qual» et «ag» ne sont
  pas les seuls qui permettent d’exprimer ces trois idées
  générales. Elles sont souvent aussi exprimées par des suffixes ou
  autrement; ces suffixes sont donc des \emph{synonymes} de l’un des
  trois atomes-radicaux «le», «qual» ou «ag».

  Les trois idées grammaticales sont des idées fondamentales à cause
  de leur généralité. J’appellerai donc \emph{atome fondamental} tout
  atome (radical ou affixe) qui ne contient rien d’autre qu’une idée
  grammaticale. Ces atomes servent de type aux autres, parce que leur
  constitution est tout à fait simple. En effet, plus une idée est
  générale, moins elle contient d’idées plus générales sous-entendues;
  les atomes fondamentaux ne contiennent donc qu’une seule idée (idée
  grammaticale) et ne peuvent }%
{\noindent {[arbi]}trary: thus, the word \emph{gaieté} `gaiety' is
  derived from the adjective \emph{gai} `gay'; the word \emph{joie}
  `joy', however, is a noun root; but once the classification is made,
  it must be definitive, since the sense of derived words in which
  this atom plays a part depends in part on that classification.

  \textsc{The basic Atoms} --- Before classifying all specific atoms
  according to the general (or grammatical) idea contained in each, it
  is best to classify separately all those atoms which contain no
  specific idea, but only one of the three grammatical ideas (nominal,
  adjectival or verbal), since the three atoms \emph{le} (entity),
  ``qual'' and ``ag'' are not the only ones that can express these
  three general ideas. These are often also expressed by suffixes or
  otherwise: those suffixes are thus \emph{synonyms} of one of the
  three root atoms \emph{le}, ``qual'' or ``ag''.

  The three grammatical ideas are the basic ideas, by virtue of their
  generality. I will thus designate as \emph{basic atom} any atom
  (root or affix) which contains nothing but a grammatical idea. These
  atoms serve as types for the others, because their composition is
  completely simple. In effect, the more general an idea is, the less
  it contains understood ideas that are more general. The basic atoms
  thus contain only a single idea (the grammatical idea), and can}

\TextPage{\noindent contenir rien d'autre, puisqu'il n’existe pas
  d'idée plus générale, plus abstraite, que les idées grammaticales.
  
  Examinons séparément chacun des trois groupes d’atomes fondamentaux:
  
  1. \emph{Atomes fondamentaux adjectifs}: les suffixes «ain» (dans
  «humain»), «el» (dans «industriel»), «ique» (dans «périodique»),
  «eux» (dans «fameux»), etc., ne contiennent que l’idée adjective
  générale. Ce sont donc des synonymes du radical adjectif général
  «qual», c’est-à-dire qu’ils sont interchangeables avec ce radical;
  ainsi, par exemple, «hum-ain» = «hom-qual» et «hum-an-ité» =
  «hom-qual-ité».
 
  Il y a encore d’autres atomes équivalents à l’idée
  adjective. L’adjectif est souvent un simple génitif; ainsi «une main
  humaine» signifie «une main d’homme», de sorte que la préposition
  génitive «de», «d’(un être){[»]}, doit être traitée comme un atome
  adjectif fondamental. Cet atome ne diffère des suffixes «ain»,
  «eux», etc., que par sa position dans la molécule, car il précède,
  au lieu de suivre, l’atome qu’il qualifie. D’autre part, la
  préposition «de» n’est pas un préfixe, car elle n’est pas soudée au
  mot auquel elle se rapporte; on peut dire que c’est un \emph{suffixe
    dissocié}, car cette préposition forme des molécules dissociées
  comme «d’homme», molécules dans lesquelles elle joue le même rôle
  que le }%
{\noindent contain nothing more, since there is no idea that is more
  general, more abstract than the grammatical ideas.

  Let us examine separately each of the three groups of basic atoms:

  1. \emph{Basic adjectival atoms}: The suffixes \emph{ain} (in
  \emph{humain} `human'), \emph{el} (in \emph{industriel}
  `industrial'), \emph{ique} (in \emph{périodique} `periodic'),
  \emph{eux} (in \emph{fameux} `famous'), etc. contain only the
  general adjectival idea. They are thus synonyms of the general root
  adjective ``qual''; that is, they are interchangeable with that
  root. Thus, \emph{hum-ain} = ``hom-qual'' and \emph{hum-an-ité} =
  ``hom-qual-ité''.

  There are still other atoms that are equivalent to the adjectival
  idea. The adjective is often a simple genitive: thus, \emph{une main
    humaine} `a human hand' means \emph{une main d'une homme} `a hand
  of a man', such that the genitive preposition \emph{de} `of',
  \emph{d'(un être)} `of (an entity)' must be treated as a basic
  adjectival atom. This atom differs from the suffixes \emph{ain,
    eux,} etc. only by its position within the molecule, since it
  precedes rather than following the atom that it qualifies. On the
  other hand, \emph{de} is not a prefix, because it is not bound to
  the word it relates to.  We can say that it is a \emph{dissociated
    suffix}, since this preposition forms dissociated molecules like
  \emph{d'homme} `of man', molecules in which it plays the same role
  as the }

\TextPage{\noindent suffixe «ain» dans la molécule condensée
  «humain». On a, en effet,\emph{ d'homme} = \emph{hum-ain}; de même
  «de qualité» est l’adjectif du substantif «qualité», lequel est tiré
  lui-même de l’adjectif «qual»; on a donc «de qualité» =
  «qual\footnote{En effet, la famille «qual», «qualité», «de
    qualité»), est analogue à la famille «beau», «beauté», «beautiful»
    (en anglais). On a «de qualité» = «qual», comme on a «beautiful» =
    «beau», parce que la deuxième dérivation (adjectivation d’un
    substantif) est exactement inverse de la première
    (substantification d’un adjectif).}». Ainsi, par exemple: \emph{un
    bâton de fer} = \emph{un bâton ferr-eux} ou encore \emph{de
    qualité fer} (\emph{fer-qual}).
  
  Enfin, il y a encore comme atome adjectif fondamental le mot «qui»,
  «qui (est)», «qui (a l’être)»; ainsi \emph{ac-tif} = \emph{qui
    ag(it)}, \emph{pallia-tif} = \emph{qui pallie}; le mot «qui»
  équivaut aussi à la finale du participe lorsque celle-ci est prise
  dans le sens adjectif: ainsi \emph{aim-ant} = \emph{qui aime};
  \emph{diffèr-ent} = \emph{qui diffère}; «aimant» est la forme
  condensée, «qui aime» est la forme dissociée, et l’on voit que
  l’atome «qui» est encore un suffixe dissocié ou déplacé, comme la
  préposition «de». Une autre preuve que l’atome «qui (est)» ne
  contient que l’idée générale adjective, c’est qu’on n’ajoute rien à
  un adjectif en lui accolant cet atome: on a toujours \emph{beau}
  =\emph{ qui (est) beau}, \emph{le grand arbre} = \emph{l’arbre qui
    (est) grand}\footnote{Ou encore \emph{l’arbre qui (a l’être)
      grand, qui a de la grandeur}, puisque \emph{le grand},
    \emph{l'être idéel }«\emph{grand}» = \emph{grand-eur}.}, etc.  }
{\noindent suffix \emph{ain} in the condensed molecule \emph{humain}
  `human'. Effectively, we have \emph{d'homme} = \emph{humain}, just
  as \emph{de qualité} `of quality' is the adjective from the noun
  \emph{qualité} `quality', which is itself taken from the adjective
  ``qual''; we thus have \emph{de qualité} =
  ``qual\footnote{Effectively, the family ``qual'', \emph{qualité},
    \emph{de qualité} is analogous to the family \emph{beau}
    `beautiful', \emph{beauté} `beauty', \emph{beautiful} (in
    English). We have \emph{de qualité} = ``qual'', just as
    \emph{beautiful} = \emph{beau}, because the second derivation
    (adjectivalization of a noun) is exactly the inverse of the first
    (nominalization of an adjective).}''. Thus, for example, \emph{un
    bâton de fer} `a stick of iron' = \emph{un bâton ferr-eux} `an
  iron stick' or also \emph{de qualité fer} (\emph{fer}-qual) `{[a
    stick]} with iron properties'.

  Finally, as a basic adjectival atom, there is the word \emph{qui}
  `who, which', \emph{qui (est)} `which (is)', \emph{qui (a l'être)}
  `which (the entity has)'. Thus, \emph{ac-tif} `active' = \emph{qui
    ag(-it)} `which acts', \emph{pallia-tif} `palliative' = \emph{qui
    pallie} `which allays, moderates'.  The word \emph{qui} is also
  equivalent to the ending of the participle when this is taken in an
  adjectival sense: thus, \emph{aim-ant} `lov-ing' = \emph{qui aime}
  `who loves', \emph{diffèr-ent} = \emph{qui diffère} `which
  differs'.  \emph{Aimant} is the condensed form, \emph{qui aime} is
  the dissociated form, and we see that the atom \emph{qui} is another
  dissociated or displaced suffix, like the preposition
  \emph{de}. Further proof that the atom \emph{qui (est)} contains
  only the general adjectival idea is that nothing is added to an
  adjective when this atom is attached: thus, we have \emph{beau} =
  \emph{qui (est) beau}, \emph{le grand arbre} `the tall tree' =
  \emph{l'arbre qui (est) grand} `the tree which (is) tall'\footnote{Or
    again \emph{l'arbre qui (a l'être) grand, qui a de la grandeur,}
    `the tree which (has) tallness, which has height', since \emph{le
      grand, l'être ideel ``grand''} = \emph{grandeur} `height'.},
  etc.

}

\TextPage{\noindent Même dans une phrase comme «Socrate est sage», il
  suffit de remplacer ce qui est sous-entendu [«Socrate est un (homme)
  sage»] pour pouvoir ajouter l’atome «qui (est)» sans rien changer au
  sens de la phrase: «Socrate est un (homme) qui est sage».
  
  L’expression «qui est» placée devant un adjectif est un simple
  pléonasme, une répétition de l’idée adjective générale, tout comme
  un suffixe verbal placé après un radical verbal n’est qu’une
  répétition de l’idée verbale déjà contenue dans ce radical,
  c’est-à-dire que \emph{beau} = \emph{qui est beau}, comme
  \emph{écri} = \emph{écri-re}. Au contraire, devant un verbe ou
  un substantif, l’atome adjectif «qui» ou «qui (est)», «qui (a
  l’être)», reprend toute sa valeur qualificative; ainsi «qui aime» ne
  signifie pas «aim» ou «aimer», mais «aim-ant», de même «qui est
  homme» ne signifie pas «homme», mais «humain»; par exemple, \emph{un
    être humain} = \emph{un être qui est homme}. Il faut seulement ne
  pas confondre l’atome adjectif «qui (est)» avec l’atome substantif
  «ce (qui est)»; ce dernier désigne «l’être», tandis que le premier
  ne désigne qu’un attribut de l’être; il y a la même différence entre
  «qui est» et «ce qui est» qu’entre «beau» et «le beau».

  On voit que l’idée adjective générale peut être représentée par
  beaucoup d’atomes différents, tous synonymes entre eux, c’est-à-dire
  interchangeables les uns avec les autres au point de vue logique. Si
} {\noindent Even in a sentence like ``Socrates is wise,'' it suffices
  to replace what is understood {[``Socrates is a wise (man)'']} to be
  able to add the atom \emph{qui est} `who (is)' without changing
  anything in the meaning of the sentence: ``Socrates is a (man) who
  is wise.''

  The expression \emph{qui est} `who/which is' before an adjective is
  simply a pleonasm, a repetition of the general adjectival idea, just
  as a verbal suffix placed after a verbal root is only a repetition
  of the verbal idea already contained in the root. That is,
  \emph{beau} `beautiful' = \emph{qui est beau} `who is beautiful',
  just as \emph{écri-} `write (root)' = \emph{écri-re} `(to) write
  (infinitive)'. In contrast, before a verb or a noun, the adjectival
  atom \emph{qui} or \emph{qui (est), qui (a 'l'être)} `who/which
  (is), which (the entity has)' recovers all of its qualifying
  value. Thus \emph{qui aime} `who loves' does not mean \emph{aim-}
  `love (root)' or \emph{aimer} `to love', but rather \emph{aim-ant}
  `loving', just as \emph{qui est homme} `who is (a) man' means not
  \emph{homme} `man', but \emph{humain} `human'; for example, \emph{un
    être humain} `a human being' = \emph{un être qui est homme} `a
  being that is (a) man'.  It is only required not to confuse the
  adjectival atom \emph{qui (est)} with the nominal atom \emph{ce (qui
    est)} `that (which is)'. The latter designates an ``entity'',
  while the former designates only an attribute of an entity. The
  difference between \emph{qui est} and \emph{ce qui est} is the same
  as between \emph{beau} `beautiful' and \emph{le beau} `the
  beautiful'.

  We see that the general adjectival idea can be represented by a
  number of different atoms, all synonyms of each other, that is,
  interchangeable with one another from the point of view of logic. If
}

\TextPage{\noindent donc on représente l’idée adjective par le symbole
  spécial \textbf{a}, on peut écrire:\\[1ex]
  
  \noindent{\setlength{\tabcolsep}{0pt}%
    \resizebox{\linewidth}{!}{\begin{tabular}[t]{lccl}
      Idée adjective \textbf{a} & = &atomes-&radicaux: \emph{qual}, \emph{propre} (à);\\
                                & = & » &suffixes: \emph{ain}, \emph{el}, \emph{ique}, \emph{eux}, etc.;\\
                                & = & » &suffixes dissociés: \emph{de}, \emph{qui} (est),\\
                                &   &   & \hspace{1em}\emph{qui} (a l’étre).\\
    \end{tabular}}}\\[1ex]

  Ces atomes sont interchangeables: ainsi l’idée adjective \textbf{a}
  ajoutée au substantif \emph{homme} donne l’une quelconque des
  molécules bi-atomiques suivantes: \textbf{hom-a} = \emph{hum-ain} =
  \emph{hom-qual} (molécules condensées) ou: \emph{propre} (à
  l’)\emph{homme},\emph{ d'homme}, \emph{qui} (est) \emph{homme},
  \emph{qui} (a l’ètre) \emph{homme} (molécules dissociées).

  2. \emph{Atomes fondamentaux substantifs}. — Jusqu’ici nous avons
  représenté l’idée substantive par les expressions «l’être» ou «ce
  qui est». Ces expressions contiennent ou semblent contenir plusieurs
  atomes, c’est-à-dire qu’elles ont l’apparence d’une molécule. Or,
  les idées grammaticales, servant de base aux idées particulières,
  doivent être représentées par des atomes irréductibles (radicaux ou
  suffixes) pouvant servir de modèles aux atomes particuliers. Ce
  n’est qu’après avoir défini et classé tous les atomes, que l’on peut
  entreprendre la construction des molécules composées de plusieurs
  atomes pour exprimer des idées plus complexes. La théorie }
{\noindent therefore we represent the adjectival idea by the special
  symbol \textbf{a}, we can write:\\[1ex]
  
  \noindent{\setlength{\tabcolsep}{0pt}%
    \resizebox{\linewidth}{!}{\begin{tabular}[t]{lccl}
      Adjective idea \textbf{a} & = &atom-&roots: \emph{qual}, \emph{specific} (to);\\                                        
                               & = & "  &suffixes: \emph{an}, \emph{ial}, \emph{ic}, \emph{ous}, etc.;\\
                               & = & "  & dissociated suffixes: \emph{of},\\
                               &   &    &  \hspace{1em}\emph{which} (is), \emph{which} (it has).
    \end{tabular}}}\\[1ex]

  These atoms are interchangeable: thus, the adjectival idea
  \textbf{a} added to the noun \emph{homme} `man' gives any one of
  following bi-atomic molecules: \textbf{hom-a} = \emph{hum-ain}
  `human' = \emph{hom-qual} (condensed molecules) or \emph{propre} (a
  l')\emph{homme} `specific (to the) man', \emph{d'homme} `of man,
  \emph{qui (est) homme} `which (is) man', \emph{qui (a l'être)
    homme} `which (the entity) man (has)' (dissociated molecules).

  2. \emph{Basic nominal atoms.} --- Up to this point we have
  represented the nominal idea by the expressions \emph{l'être} `the
  entity' or \emph{ce qui est} `that which is'. These expressions
  contain or appear to contain several atoms, that is, they have the
  appearance of a molecule.  Now the grammatical ideas, serving as the
  bases of specific ideas, must be represented by irreducible atoms
  (roots or suffixes) capable of serving as models for specific
  atoms. It is only after having defined and classified all of the
  atoms that we can undertake the construction of molecules composed
  of several atoms to express more complex ideas. The {[atomic]}
  theory }
    
\TextPage{\noindent atomique de la formation des mots n’a de valeur
  que si l’on part d’atomes irréductibles; en effet, si l’on
  définissait les idées fondamentales par des molécules polyatomiques,
  on tomberait dans un cercle vicieux, et la théorie même
  s’écroulerait, car on ne peut pas définir les mots-atomes par des
  mots-molécules; ce sont les molécules qui doivent être définies par
  les atomes qu’elles contiennent. Comme il y a des atomes
  particuliers et des atomes généraux, on peut encore faire rentrer
  les atomes particuliers dans les atomes généraux, mais c’est
  tout. Donc, en fin de compte, le dernier résidu, base irréductible
  de l’analyse des mots, ce sont les atomes généraux, ou fondamentaux,
  qui représentent les idées grammaticales. Ces atomes doivent avoir
  une forme tout à fait simple; c’est pour cela que nous avons montré
  que l’expression «de qualité», qui représente l’idée adjective
  générale, n’a que l’apparence d’une molécule polyatomique, mais
  qu’en réalité cette expression se réduit à l’atome «qual»; l’idée
  adjective a été ainsi définie par des éléments tout à fait
  irréductibles, tels que les radicaux «qual», «propre» ou les
  suffixes «ain», «ique», «eux», etc.
 
  De même, les expressions «l’être», «ce qui est», qui expriment
  l’idée substantive générale, n’ont que l’apparence de molécules
  polyatomiques dissociées. En réalité, le verbe «être» n’a, dans ces
  expressions, qu’un rôle explicatif; ce sont les atomes «le» }
{\noindent atomic {[theory]} of the formation of words only has force
  if we start from irreducible atoms. Indeed, if we were to define the
  fundamental ideas by polyatomic molecules, we would fall into a
  vicious circle and the theory itself would collapse, because we
  cannot define word-atoms by word-molecules. It is the molecules that
  must be defined by the atoms that they contain.  As there are
  specific atoms and general atoms, we can fit specific atoms into the
  general atoms, but that is all. Thus in the end, the last resort,
  the irreducible basis of the analysis of words is the {[set of]}
  general or basic atoms which represent the grammatical ideas. These
  atoms must have a completely simple form: that is why we have shown
  that the expression \emph{de qualité} `with the quality', which
  represents the general adjectival idea, only has the appearance of a
  polyatomic molecule: in reality this expression reduces to the atom
  ``qual''. The adjectival idea has thus been defined by completely
  irreducible elements, such as the roots ``qual'', \emph{propre}
  `proper (to)', or the suffixes \emph{ain, ique, eux} etc.

  Similarly, the expressions \emph{l'être} `the entity', \emph{ce
    qui est} `that which is', which express the nominal idea, only
  have the appearance of dissociated polyatomic molecules. In reality,
  the verb \emph{être} `to be' in these expressions has only an
  explanatory role; it is the atoms \emph{le} `the' }

\TextPage{\noindent (article) et «ce» (pronom) qui indiquent vraiment
  l’idée substantive; ces atomes sont des substantificateurs: dans
  l’expression «l’être», l’affixe «l’» ou «le» substantifïe le verbe
  «être» (exister), comme il substantifie l’adjectif «beau» dans
  l’expression «le beau» (beau-té).
  
  Mais il y a beaucoup d’autres atomes représentant l’idée
  substantive. Les atomes «le», «ce» représentent l’idée substantive
  sous sa forme la plus générale, la plus abstraite (l’être). Pour
  représenter l’idée substantive particulière sous sa forme concrète
  (un être), on emploie l’article indéfini «un». Cette forme de l’idée
  substantive est moins générale, car l’idée concrète «un» implique
  l’idée abstraite «le», tandis que la réciproque n’est pas
  vraie. Cependant, l’atome «un» représente encore l’idée substantive
  pure (un être), à condition de ne pas l’employer pour représenter un
  être spécial. Les expressions «un riche», «un beau», dans le sens de
  «homme riche», «homme beau» sont des expressions incomplètes, car on
  donne ici à l’article «un», non pas le sens de «un être quelconque»
  (réel ou idéel), mais le sens spécial de «un être réel» (par
  opposition à un être de langage), et même les sens encore plus
  spéciaux de «un être réel vivant», «un être réel vivant humain». Le
  sens logique total de l’expression «un beau», prise isolément et
  sans contexte, ne peut être que «un}%»
{\noindent (article) and \emph{ce} `this' (pronoun) that really
  indicate the nominal idea.  These atoms are nominalizers: in the
  expression \emph{l'être} `(the) entity', the affix \emph{l'} or
  \emph{le} nominalizes the verb \emph{être} (to exist), just as it
  nominalizes the adjective \emph{beau} `beautiful' in the expression
  \emph{le beau} (\emph{beau-té} `beau-ty').

  But there are many other atoms that represent the nominal idea. The
  atoms \emph{le, ce} represent the nominal idea in its most general
  and most abstract form (entity). To represent the specific nominal
  idea in its concrete form (an entity), the indefinite article
  \emph{un} `a' is used. This form of the nominal idea is less
  general, because the concrete idea \emph{un} `a' implies the
  abstract idea \emph{le} `the', while the inverse is not
  true. However, the atom \emph{un} still represents the pure nominal
  idea (an entity) so long as it is not used to represent a particular
  entity. The expressions \emph{un riche} `a rich (one)', \emph{un
    beau} `a handsome (one)' in the sense of \emph{homme riche} `rich
  man', \emph{homme beau} `handsome man' are incomplete expressions,
  because here the article \emph{un} is given not the sense of
  \emph{un être quelconque} `any entity' (real or ideal), but the
  specific sense of \emph{un être réel} `a real entity' (as
  opposed to a linguistic entity), and even the yet more specific
  sense of \emph{un être réel vivant} `a real living being', \emph{un être réel vivant humain}`a
  real living human being'. The entire logical sense of the expression
  \emph{un beau}, taken in isolation and without context, can only be
  \emph{un} }

\TextPage{\noindent être beau», cet être étant d’ailleurs quelconque,
  réel ou idéel; et, en effet, comme l’atome «un» contient en lui-même
  l’atome plus général «le», on a «un beau» = «un—le beau» = \mbox{«une
  beauté»\footnotemark}\footnotetext{On se rend mieux compte du sens logique de la
    molécule formée par un adjectif précédé de l’article indéfini en
    choisissant un adjectif qui n’évoque pas d’idée substantive
    spéciale; par exemple, «un vrai» = «une chose vraie» = «une
    vérité» = «un—le vrai». Ou encore: «un blanc» peut, signifier soit
    «un homme blanc», soit «un objet-unité blanc» (sens concret), soit
    «un blanc de l’esprit» (sens abstrait).} (puisque le beau =
  beau-té), expression qui s’applique bien à un être quelconque. En
  résumé, on peut considérer l’atome «un» comme atome fondamental
  substantif représentant l’idée particulière «un être»; il faut
  seulement faire attention qu’il y a souvent, dans cet atome, une
  idée substantive spéciale sous-entendue, idée qu’il faut rétablir
  sous une forme explicite avant de procéder à une analyse
  quelconque. Ainsi, au lieu de «un riche» on doit écrire «homme
  riche»; par contre, dans l’expression «un tout» il n’y a rien de
  sous-entendu, parce que l’atome «un» peut être appliqué ici à un
  être quelconque.
  
  L’idée substantive générale peut aussi être représentée par le mot
  «entité» (du latin ens), qui équivaut à l’expression «l’(être)», «un
  être abstrait de la}%»
{\noindent \emph{être beau} `a beautiful being', this entity being,
  however, any entity, real or ideal. Indeed, since the atom \emph{un}
  contains the more general atom \emph{le} within it, we have \emph{un
    beau} `a beautiful (one)' = \emph{un--le beau} = \emph{une
    beauté} `a beauty'\footnote{The logical sense of the molecule
    formed by an adjective preceded by the indefinite article is
    better shown by choosing an adjective that does not evoke a
    specific nominal idea: for example, \emph{un vrai} `a true (one)'
    = \emph{une chose vraie} `a true thing' = \emph{une vérité} `a
    truth' = \emph{un--le vrai}. Or again: \emph{un blanc} `a white
    (one)' can mean either \emph{un homme blanc} `a white man' or
    \emph{un objet-unité blanc} `a white object' (concrete sense) or
    \emph{un blanc d'esprit} `a mental blank' (abstract sense).}
  (since \emph{le beau} = \emph{beau-té} `beauty'), an expression
  completely applicable to any being. In sum, we can consider the atom
  \emph{un} as a basic nominal atom representing the specific nominal
  idea \emph{un être} `an entity'; it is only necessary to attend to
  the fact that there is often understood, within this atom, a
  specific nominal idea that must be restored in explicit form before
  proceeding to any analysis.  Thus, instead of \emph{un riche} `a
  rich (one)' we have to write \emph{homme riche} `rich man'; on the
  other hand, in the expression \emph{un tout} `a totality' there is
  nothing understood, because the atom \emph{un} can here be applied
  to any entity.

  The general nominal idea can also be represented by the word
  \emph{entité} `entity' (from Latin \emph{ens}), which is
  equivalent to the expression \emph{l'être} `an entity abstracted
  from }

\TextPage{\noindent réalité». (Voir plus loin l’analyse du mot
  «entité»).
  
  Enfin, il y a beaucoup de suffixes qui représentent l’idée
  substantive générale: ainsi les suffixes «ité» (dans «égalité»),
  «esse» (dans «richesse») «eur» (dans «grandeur»), «ment» (dans
  «abonnement»), «ture» (dans «écriture»), «tion» (dans «abdication»),
  etc., sont des atomes fondamentaux, car ce sont de simples
  substantificateurs d'adjectifs ou de verbes.

  Si donc on représente l’idée substantive générale par le symbole o,
  et l’idée substantive particulière par le symbole o1, on peut
  écrire:\\[1ex]

  \noindent{\setlength{\tabcolsep}{0pt}%
    \resizebox{\linewidth}{!}{\begin{tabularx}{1.25\linewidth}[t]{Xc@{=}cX} 
      \raggedright Idée substantive
      générale (ou abstraite)         &\textbf{o}&& {\raggedright atomes radicaux: \emph{ens}, (\emph{entité}), \emph{ce} (qui est).}\\
                              &&& atomes suffixes: \emph{ité, esse, eur, ment, tion, ture}, etc.\\
                              &&& suffixe dissocié: \emph{le, l',} (être).\\
      \raggedright Idée substantive particulière (ou concrète) &\textbf{o}\sxsub{1}&& suffixe diss.: \emph{un.}
    \end{tabularx}}}\\[1ex]


  La distinction entre l’idée générale \textbf{o} et l’idée
  particulière \textbf{o}\sxsub{1}, ne signifie pas qu’il y a deux
  sortes d’idée substantive, mais simplement que tout substantif peut
  être pris, soit dans le sens abstrait, soit dans le sens concret. À
  part cela, tous ces atomes sont interchangeables entre eux; les
  suffixes «ité», }%
{\noindent reality). (see below for the analysis of the word
  \emph{entité} `entity').

  Lastly, there are a number of suffixes that represent the general
  nominal idea: thus, the suffixes \emph{ité} (in \emph{égalité}
  `equality'), \emph{esse} (in \emph{richesse} `richness, wealth'),
  \emph{eur} (in \emph{grandeur} `greatness, size', \emph{ment} (in
  \emph{abonnement} `subscription'), \emph{ture} (in \emph{écriture}
  `writing'), \emph{tion} (in \emph{abdication} `abdication'),
  etc. are basic atoms, because they are simple nominalizers of
  adjectives or verbs.

  If we then represent the general nominal idea by the symbol
  \textbf{o} and the specific nominal idea by the symbol
  \textbf{o}\sxsub{1}, we can write:\\[1ex]

  \noindent{\setlength{\tabcolsep}{0pt}%
    \resizebox{\linewidth}{!}{\begin{tabularx}{1.25\linewidth}[t]{Xc@{=}cX}
      \raggedright General (or abstract) nominal idea &\textbf{o}&& root atoms: \emph{ens}, (\emph{entité}), \emph{ce} (qui  est).\\
                              &&& suffix atoms: \emph{ité, esse, eur, ment, tion, ture}, etc.\\
                              &&& dissociated suffix: \emph{le, l',} (être).\\
      \raggedright Specific (or concrete) nominal idea &\textbf{o}\sxsub{1}&& dissociated suffix: \emph{un.} 
    \end{tabularx}}}\\[1ex]

  The distinction between the general idea \textbf{o} and the specific
  idea \textbf{o}\sxsub{1} does not mean that there are two kinds of
  noun, but only that every noun can be taken either in the abstract
  or in the concrete sense. Apart from that, all of these atoms are
  interchangeable with one another; the suffixes \emph{ité}, }

\TextPage{\noindent «tion», par exemple, sont interchangeables avec
  l’atome «le», «l’être», «l’(être abstrait)»; en effet, les adjectifs
  et les verbes ne représentent pas des êtres, mais des attributs, des
  manières d’être ou l’agir d’un être; les suffixes «ité», «tion»,
  etc., qui servent à substantifier soit un adjectif, soit un verbe,
  font bien de ces derniers un être, mais comme cette
  substantification n’est qu’un simple procédé de langage qui
  n'apporte aucune réalité nouvelle, l’être ainsi obtenu ne peut être
  qu’un être idéel, abstrait de nature, puisqu’on l’a abstrait de la
  réalité adjective ou verbale pour en faire un être. On a donc
  \emph{beau-té} = \emph{le beau} (l’être abstrait «beau»), de même
  que \emph{bois-son} = \emph{le boire}. Mais ce qu’il faut remarquer,
  c’est que dans les molécules «le beau» et «beau-té», l’atome «beau»
  reste toujours adjectif (principe de l’invariabilité des atomes). Il
  ne faut pas mettre dans les suffixes «ité», «esse», etc., une idée
  de «qualité», car on ne peut pas avoir \emph{ité} =
  \emph{qual-ité}; les atomes «ité», «tion», sont l’expression la
  plus pure de l’idée substantive générale o, c’est-à-dire de l’idée
  «l’être (en général)», «l’être abstrait». Et, en effet, nous avons
  vu que tout adjectif, comme «beau», contient en lui-même l’idée
  adjective «qual»; donc si «beau-té» exprime une «qual-ité», ce n’est
  pas parce que le suffixe «ité» exprime la qualité, mais \emph{parce
    que l’atome adjectif «beau» contient implicitement en lui-même
    l’idée} }%
{\noindent \emph{tion}, for example, are interchangeable with the atom
  \emph{le, l'être, l'(être abstrait)} `the (abstract)
  (entity)'. Indeed, adjectives and verbs do not represent entities,
  but attributes, manners of being or acting of an entity.  The
  suffixes \emph{ité, tion}, etc. which serve to nominalize an
  adjective or a verb do make an entity from these items, but as this
  nominalization is only a simple linguistic process which brings in
  no new reality, the entity thereby obtained can only be something
  ideal, abstracted from nature, since one has abstracted from the
  adjectival or verbal reality to make an entity out of it.  We thus
  have \emph{beau-té} `beauty' = \emph{le beau} `the beautiful' (the
  abstract entity \emph{beau} `beautiful'), just as \emph{bois-son}
  `drink (n)' = \emph{le boire} `the drink (v)'.  But what has to be
  noted is that in the molecules \emph{le beau} and \emph{beau-té},
  the atom \emph{beau} still remains an adjective (principle of the
  invariability of atoms).  It is necessary not to find in the
  suffixes \emph{ité, esse}, etc. an idea of `quality', because we
  cannot have \emph{ité} = \emph{qual-ité}.  The atoms
  \emph{ité, tion}, are the purest expression of the general nominal
  idea, that is, the idea `entity (in general)', `abstract
  entity'. And indeed we have seen that every adjective, like
  \emph{beau}, contains in itself the adjectival idea `qual'.  Thus,
  if \emph{beau-té} expresses a `qual-ity', this is not because the
  suffix \emph{ité} expresses the quality, but \emph{because the
    adjectival atom \und{beau} implicitly contains in itself the}
  \emph{idea} }

\TextPage{\noindent\emph{adjective generale «qual»}. Nous reviendrons
  là-dessus à propos de l’analyse des molécules, et nous verrons alors
  quelle distinction on peut faire entre «le beau» et «beauté», «le
  boire» et «boisson»; il suffit, pour le moment, de constater que les
  atomes tels que «ité», «tion», etc., ne contiennent que l’idée
  substantive générale. L’atome «ité» ne peut pas plus être égal à la
  molécule «qualité» que l’atome «tion» ne peut être égal à la
  molécule «action», car un atome substantif général (comme «ité» ou
  «tion») ne peut pas contenir en même temps l’idée adjective ou
  l'idée verbale, car ces trois idées fondamentales sont
  essentiellement indépendantes les unes des autres.\footnote{Si l'on
    ne peut pas écrire «ité» = «qual-ité», on peut, par contre,
    écrire «ité» = «ent-ité», car le radical «ens» ne fait que
    répéter l'idée substantive «l'être» ou «un être».}
  
  3.  \emph{Atomes fondamentaux verbaux}. — Il y a deux idées verbales
  générales: l'idée dynamique «faire un acte» ou «une action»
  (correspondant aux verbes actifs) et l’idée statique «être dans un
  état» ou «station» (correspondant aux verbes neutres).

  L’idée statique n’est du reste qu’un cas particulier de l'idée
  dynamique: cette dernière implique en effet des forces en activité;
  si les forces ne se font pas équilibre on a l’idée dynamique
  proprement dite (action), c’est le cas général; si les forces }%
{\noindent \emph{general adjectival} {[\emph{idea}]}.  We will come
  back to this in connection with the analysis of molecules, and we
  will see then what distinction can be made between \emph{le beau}
  `the beautiful' and \emph{beau-té} `beauty', \emph{le boire} `the
  drink (v)' and \emph{boisson} `drink (n)'. For the moment it
  suffices to note that atoms such as \emph{ité, tion}, etc. contain
  only the general nominal idea.  The atom \emph{ité} can no more be
  equal to the molecule \emph{qualité} `quality' than the atom
  \emph{tion} can be equal to the molecule \emph{action} `action',
  because a general nominal atom (like \emph{ité} or \emph{tion})
  cannot contain at the same time the adjectival idea or the verbal
  idea, since these three basic ideas are fundamentally independent of
  one another.\footnote{Although we cannot write \emph{ité} =
    \emph{qual-ité} `quality', we can in contrast write \emph{ité}
    = \emph{ent-ité} `entity', because the root \emph{ens} does
    nothing but repeat the nominal idea \emph{l'être} or \emph{un
      être} `entity'.}

  3. \emph{Basic verbal atoms.} --- There are two general verbal
  ideas: the dynamic idea `perform an act' or `an action'
  (corresponding to active verbs) and the static idea `be in a state'
  or `position' (corresponding to neutral verbs).

  The static idea is however only a special case of the dynamic idea.
  The latter implies forces in activity: if the forces are not in
  equilibrium, we have the proper dynamic idea (action), which is the
  general case; if the forces }

\TextPage{\noindent viennent à se faire équilibre on a l’idée
  statique\footnote{Du reste en mécanique, la statique (état
    d’équilibre des forces) n’est qu’un chapitre de la dynamique
    (mouvement et forces en activité); de sorte qu’on peut dire que
    les sciences mécaniques sont les sciences du verbe.} (état). Et en
  effet dès que l’état change on retombe dans l’action (faite ou
  subie).

  Mais comme les idées fondamentales (ou grammaticales) doivent toutes
  être exprimées par des atomes simples, il est à présumer que les
  expressions «faire une action» et «être dans une station » sont
  réductibles à une forme plus simple, de même que l’idée adjective
  «de qualité» est réductible au simple atome «qual».

  Remarquons d’abord que l’idée verbale générale, que nous
  représenterons par le symbole \textbf{i}, peut être exprimée par une
  simple finale verbale, telle que la finale «er» (dans «couronn-er»,
  «clou-er», «sci-er», «entour-er», «rag-er», etc.), car les atomes
  «couronne», «clou», «entour», «scie», etc., étant des atomes non
  verbaux, et conservant toujours ce caractère, d’après le principe de
  l’invariabilité des atomes, le mot «couronn-er» signifie «faire
  l’action (caractérisée par) l’objet couronne», «entour-er» signifie
  «faire l’action autour», etc. Les finales «er», «ir», etc., ne sont
  donc pas seulement des désinences destinées à indiquer les
  différents temps et personnes de la conjugaison d’un }%
{\noindent are in equilibrium, we have the static
  \mbox{idea\footnotemark}\footnotetext{Besides, in mechanics the static (state of equilibrium
    of forces) is only a part of the dynamic (movement and forces in
    action), so that we can say that the mechanical sciences are the
    sciences of the verb.}  (state). And indeed as soon as the state
  changes, we fall back into action (performed or undergone).

  But since the basic (or grammatical) ideas must all be expressed by
  simple atoms, it is to be presumed that the expressions ``perform an
  action'' and ``be in a state'' are reducible to a simpler form, just
  as the adjectival idea ``of quality'' is reducible to the simple
  atom `qual'.

  Let us note initially that the general verbal idea, which we
  represent by the symbol \textbf{i}, can be expressed by a simple
  verbal ending, such as the ending \emph{er} (in \emph{couronn-er}
  `to crown', \emph{clou-er} `to nail (down)', \emph{sci-er} `to saw',
  \emph{entour-er} `to surround', \emph{rag-er} `to rage, be in a
  rage', etc.). Since the atoms \emph{couronne} `crown , \emph{clou}
  `nail', \emph{entour} `surrounding', \emph{scie} `saw', etc. are
  non-verbal atoms, and always preserve that character according to
  the principle of the invariability of atoms, the word
  \emph{couronn-er} means ``to perform the action (characterized by)
  the object `crown'\,'' \emph{entour-er} means ``to perform the
  action `around'\,'', etc. The endings \emph{er, ir}, etc. are thus
  not only desinences indicating different tenses and persons in the
  conjugation of a}

\TextPage{\noindent verbe, mais ce sont de véritables suffixes au
  point de vue logique, des atomes contenant en eux-mêmes l’idée
  verbale générale, puisque les molécules «couronn-er», «rag-er»,
  etc., sont des verbes, tandis que les atomes «couronne», «rage»,
  etc., sont de purs substantifs qui ne contiennent, en fait d’idées
  grammaticales, que l’idée d’un être (concret ou abstrait).

  Voyons maintenant quels sont les atomes-radicaux qui permettent
  d’exprimer par un simple atome l’idée verbale générale «faire un
  acte, une action», ou «être dans un état, une station». Tout
  d’abord, «faire une action» se réduit à «faire», car quand on
  «fait», on fait toujours «une action»; de même «être dans un état»
  se réduit à «être» (au sens statique, sens que nous avons représenté
  par le radical «sta»), car quand on «est» (au sens statique), on
  «est» toujours «dans un état». On peut donc considérer que dans les
  expressions «faire (une action)» ou «être (dans un état)», les
  parties entre parenthèses ne sont qu’explicatives; ainsi la
  parenthèse «dans un état» a pour but d’indiquer que le verbe «être»
  est pris ici au sens statique (\emph{stare}) et non dans le sens
  d’«exister». Par conséquent, on peut représenter les deux idées
  verbales par les deux atomes-radicaux «fai(re)» (verbe dynamique) et
  «sta(re)[»](verbe statique).

  Mais on peut procéder aussi autrement en remar-{[quant]} }%
{\noindent verb, but they are true suffixes from the point of view of
  logic, atoms containing in themselves the general verbal idea, since
  the molecules \emph{couronn-er} `to crown', \emph{rag-er} `to (be in
  a) rage', etc. are verbs, while \emph{couronne} `crown', \emph{rage}
  `rage', etc. are pure nouns that only contain, as far as grammatical
  ideas are concerned, the idea of an entity (concrete or abstract).

  Let us see now which root-atoms make it possible to express in a
  simple atom the general verbal idea ``to perform an act, an action''
  or ``to be in a state, a position''.  To begin with, ``to do,
  perform an action'' reduces to ``to do'', because when one ``does''
  one always performs ``an action''; similarly ``to be in a state''
  reduces to ``to be'' (in the static sense, the sense which we have
  represented by the root \emph{sta} `be'), because when one ``is''
  (in the static sense), one ``is'' always ``in a state''. We can thus
  consider that in the expressions ``to do, perform (an action)'' or
  ``to be (in a state)'', the parts in parentheses are only
  explicative; thus the parenthesized ``in a state'' serves to
  indicate that the verb ``to be'' is taken here in the static sense
  (\emph{stare}) and not in the sense ``to exist''.  Consequently we
  can represent the two verbal ideas by the two root-atoms ``do''
  (dynamic verb) and ``be'' (static verb).

  But we can also proceed differently in no{[ting]} }

\TextPage{\noindent {[remar]}quant que \emph{la dissociation d’une
    molécule condensée (ou la condensation d’une molécule dissociée)
    renverse l’ordre des atomes}: ainsi \emph{beau-té = le beau,
    hum-ain = d'homme, différ-ent = qui diffère}, etc. Donc une
  molécule bi-atomique dissociée \emph{(x)—(y)} est égale à la
  molécule condensée \emph{(y—x)}. La molécule dissociée
  \emph{(faire)—(action)} sera donc équivalente à la molécule
  condensée \emph{(actionn-er)}, puisque les atomes «faire» et «er»
  expriment tous deux l’idée verbale générale \textbf{i}. Et de même
  que la molécule «de qualité» se réduit à l’atome «qual», de même la
  molécule «actionner» se réduit à l’atome «ac», parce que la
  dérivation qui mène du verbe «ac» (ou «ag»)\footnote{L’atome
    \emph{ac} ou \emph{act} du mot \emph{action} est le même que
    l’atome \emph{ag} du mot \emph{agir}. Du reste, en anglais, la
    forme \emph{ag} n’existe pas et l’on a les deux formes régulières
    (\emph{to act} et \emph{action}) pour le verbe et son
    substantif. On peut se demander aussi s’il est plus logique de
    diviser le mot «action» en «ac-tion» ou en «act-ion»; mais comme
    le résultat de l’analyse logique est le même dans les deux cas, je
    choisirai la coupure «ac-tion», afin de rapprocher «ac» de «ag».}
  au substantif «action» est exactement inverse de celle qui mène du
  substantif «action» au verbe «actionner». On démontrerait de même
  que la molécule dissociée «être dans un état», ou mieux «être dans
  une station», est équivalente à la molécule condensée «stationner»,
  laquelle est réductible au simple atome «sta», de sorte que les deux
  idées verbales générales peuvent être représentées aussi}%
{\noindent {[no]}ting that \emph{the dissociation of a condensed
    molecule (or the condensation of a dissociated molecule) reverses
    the order of the atoms}: thus \emph{beau-té} `beauty' = \emph{le
    beau} `the beautiful', \emph{hum-ain} `human' = \emph{d'homme} `of
  man', \emph{différ-ent} `different' = \emph{qui diffère} `which
  differs', etc. Therefore a dissociated bi-atomic molecule
  \emph{(x)---(y)} is equal to the condensed molecule
  \emph{(y---x)}. The dissociated molecule \emph{(do)---(action)} will
  therefore be equivalent to the condensed molecule
  \emph{(action-\textsc{infinitive})} `to activate', since the atoms
  ``to do'' and ``[\textsc{infinitive}]'' both express the general
  verbal idea \textbf{i}. And just as the molecule ``of quality'' is
  reducible to the atom `qual', so the molecule ``activate'' is
  reducible to the atom `ac', because the derivation which leads from
  the verb ``ac'' (or ``ag'')\footnote{The atom ``ac'' or ``act'' in
    the word \emph{action} is the same as the atom ``ag'' in the word
    \emph{agir} `to act'. Besides, in English, the form ``ag'' does
    not exist and we have the two regular forms (\emph{to act} and
    \emph{action}) for the verb and its noun. We can also ask whether
    it is more logical to divide the word \emph{action} as \emph{ac-tion} or
    as \emph{act-ion}; but since the result of the logical analysis is
    the same in the two cases, I will choose the division \emph{ac-tion}
    in order to bring together ``ac'' and ``ag''.} to the noun
  \emph{action} is exactly the inverse of that which leads from the noun
  \emph{action} to the verb ``to activate''. We would similarly show that
  the dissociated molecule ``to be in a state'' or better ``to be in a
  position'' is equivalent to the condensed molecule \emph{stationner}
  ``to remain in a position'', which is reducible to to the simple
  atom ``sta'', such that the two general verbal ideas can also be
  represented }

\TextPage{\noindent par les deux atomes-radicaux: «ag» (pour les
  verbes actifs\footnote{Le mot «ac-tif» est une molécule qui signifie
    «de qualité \emph{ac}» et comme l’atome «ac» est le même que « ag»
    , l'expression \emph{verbe actif} signifie précisément: «verbe qui
    contient l’idée \emph{ag}».}) et «sta» (pour les verbes neutres ou
  statiques\footnote{Le mot «stat-ique» signifie «de qualité
    \emph{stat}», l’expression \emph{verbe statique} signifie
    précisément: «verbe qui contient l'idée «stat»» (ou «êt» au sens
    statique).}).

  En anglais, on peut encore représenter l'idée verbale \textbf{i} par
  le suffixe dissocié «to», par exemple: \emph{(to)-(crown) =
    (couronn-er)}, \emph{(to)-(nail) = (clou-er)}, etc. Ce suffixe se
  place avant le substantif (\emph{crown, nail}, etc.) auquel il se
  rapporte, conformément à la loi du renversement des atomes dans les
  molécules
  dissociées. En résumé, on peut écrire:\\[8pt]
  
  {\setlength{\tabcolsep}{0pt}
    \noindent
    \begin{tabular}[t]{rcl}
      Idée verbale \textbf{i} = &\ atomes-&radicaux:\\
                                &&\emph{ag, fai(re);
                                   sta, }\\
                                &&\emph{êt(re)}.\\
      = &	»	& suffixes: \\
                                &&\emph{er, ir, re}, etc.\\
      =&»& suffixe dissocié:\\
                                &&\emph{to} (en anglais).\\
    \end{tabular}}\\[8pt]

  Ces atomes sont interchangeables; considérons, par exemple, la série
  «couronne», «couronn-er», «couronn-e-ment»; on voit que l'atome
  verbal «er» de «couronner» se réduit à un simple «e» dans
  «couronn-e-ment», mais cet «e» est très important, car il représente
  à lui seul l’idée verbale «ag» ou «ac» dans le mot «couronnement»;
  on peut }%
{\noindent by the two root-atoms: ``ag'' (for active
  verbs\footnote{The word \emph{ac-tive} is a molecule which means ``of
    quality \emph{ac}'' and since the atom ``ac'' is the same as
    ``ag'', the expression \emph{active verb} means exactly ``verb
    that contains the idea \emph{ag}''.}) and ``sta'' (for neutral or
  static verbs\footnote{The word ``stat-ic'' means ``of quality
    \emph{stat}'', the expression \emph{static verb} means exactly
    ``verb that contains the idea ``stat'' (or ``be'' in the static
    sense).}).

  In English, we can again represent the verbal idea \textbf{i} by the
  dissociated suffix ``to'', for example: \emph{(to)-(crown) =
    (couronn-er)} `crown-\textsc{infinitive}', \emph{(to)-(nail) =
    (clou-er)} `nail-\textsc{infinitive}', etc. This suffix is placed
  before the noun (\emph{crown, nail,} etc.) to which it is related,
  according to the law of reversal of atoms in dissociated
  molecules. In summary, we can write:\\[8pt]

  {\setlength{\tabcolsep}{0pt}
    \noindent
    \begin{tabular}[t]{rcl}
      Verbal idea  \textbf{i} = &\ atoms:\ &\ roots\\
                                &&\emph{ag,
                                   (to) do; sta, }\\
                                &&\emph{(to) be}.\\
      = &"& (infin.) suffixes\\
                                &&\emph{er, ir, re}, etc.\\
      =&"& dissociated suffix\\
                                &&\emph{to} (in English).\\
    \end{tabular}}\\[8pt]
                        
  These atoms are interchangeable: consider, for example, the series
  \emph{couronne} `crown', \emph{couronn-er} `to crown',
  \emph{couronn-e-ment} `coronation'; we see that the verbal atom
  \emph{er} in \emph{couronner} is reduced to a simple \emph{e} in
  \emph{couronn-e-ment}, but this \emph{e} is very important, because
  it represents by itself the verbal idea ``ag'' or ``ac'' in the word
  \emph{couronnement} `coronation'; we can}

\TextPage{\noindent donc remplacer l'atome verbal «e» par l’atome
  équivalent «ag» ou «ac», et comme l’atome substantif «ment» de
  «couronnement» est interchangeable avec l’atome «tion» de «ac-tion»,
  on peut écrire:

    \begin{center}
      \emph{couronn-e-ment} = \emph{couronn-ac-tion}
    \end{center}

    \noindent ou symboliquement\footnote{Pour l’écriture symbolique il
      est préférable d’employer l’orthographe phonétique la plus
      internationale, ainsi \emph{couronn} s’écrira \textbf{kron} à
      cause de l’allemand \emph{Krone} et de l’anglais \emph{crown}.}:

    \begin{center}
      \textbf{kron-i-o} = \textbf{kron-ag-o}
    \end{center}

    \noindent comme on a écrit:

    \begin{center}
      \emph{hum-an-ité} = \emph{hom-qual-ité}
    \end{center}

    Ce ne sont donc pas les atomes «ment», «tion», etc., qui expriment
    l’action, car ces atomes sont de simples substantificateurs, comme
    les atomes «ité», «esse» (couronne-ment = le couronner). L’idée
    verbale «ac» est contenue dans l’atome verbal qui précède
    immédiatement les atomes substantifs «ment», «tion», etc. Cet
    atome verbal est quelquefois très réduit et à peine
    reconnaissable, mais sa présence est signalée par le fait que les
    atomes «ment», «tion» ne sont employés qu’après des atomes
    verbaux, comme les atomes «ité», «esse» ne le sont qu'après des
    atomes adjectifs\footnote{M. le prof. Ch. Bally, qui a bien voulu
      relire le ma-}.  }%
  {\noindent therefore replace the verbal atom \emph{e} with the
    equivalent atom ``ag'' or ``ac'', and since the nominal atom
    \emph{ment} in \emph{couronnement} `coronation' is interchangeable with
    the atom \emph{tion} of \emph{ac-tion}, we can write:

    \begin{center}
      \emph{couronn-e-ment} = \emph{couronn-ac-tion}
    \end{center}

    \noindent or symbolically\footnote{For the symbolic representation
      it is preferable to make use of the most international phonetic
      transcription; therefore \emph{couronne} `crown' will be written
      \textbf{kron} because of German \emph{Krone} and English
      \emph{crown}.}:

    \begin{center}
      \textbf{kron-i-o} = \textbf{kron-ag-o}
    \end{center}

    \noindent as we wrote:

    \begin{center}
      \emph{hum-an-ité} = \emph{hom-qual-ité}
    \end{center}

    Therefore it is not the atoms \emph{ment}, \emph{tion}, etc. that
    express the action, because these atoms are simple nominalizers,
    like the atoms \emph{ité}, \emph{esse} (\emph{couronne-ment}
    `coron-ation' = \emph{le couronner} `the to-crown'). The verbal
    idea ``ac'' is contained in the verbal atom which immediately
    precedes the nominal atoms \emph{ment}, \emph{tion} etc. This
    verbal atom is sometimes much reduced and barely recognizable, but
    its presence is signaled by the fact that the atoms \emph{ment},
    \emph{tion} are only used after verbal atoms, just as the atoms
    \emph{ité}, \emph{esse} are only used after adjective
    atoms.\footnote{Prof. Ch. Bally, who was so good as to reread the
      ma[nuscript]}}

  \TextPage{\emph{Remarque}.  \blfootnote{nuscrit de cet essai avant
      sa publication, fait ici une remarque très intéressante: «Si,
      dit-il, on se place au point de vue psychologique, on observe
      qu'un mot composé tend toujours à être conçu peu à peu comme un
      mot simple: l'idée adjective, qui n’est tout d'abord contenue
      que dans le radical «beau» du mot «beauté», finit par infecter
      et pénétrer le suffixe «té», de sorte que ce dernier
      (psychologiquement, sinon logiquement) participe de l’idée
      adjective et de l'idée substantive. Inversement, dans le mot
      «couronner», le radical «couronn» est contaminé par l'idée
      verbale contenue daus le suffixe «er».[»] — La remarque de
      M. Bally me semble très juste; néanmoins, pour le but que je me
      suis proposé, et ainsi que l'indique le titre même de cette
      brochure, le point de vue logique est le seul qui doive être
      pris ici en considération.}  — Les suffixes verbaux «er», «ir»,
    etc., s’accollent, non pas seulement à des atomes radicaux
    substantifs (comme «couronne», «clou», etc.) ou adjectifs (comme
    «grand», «gros», etc.), mais aussi à des atomes verbaux (comme
    «écri», «ouvr», «abdic»), puisque l’on dit: «écri-re», «ouvr-ir»,
    «abdiqu-er», aussi bien que «couronn\-er», «clou-er», «grand-ir»,
    etc. Au point de vue logique, il y a évidemment un pléonasme dans
    le mot «écrire», car le radical «écri» contient déjà implicitement
    en lui-même l’idée dynamique «ag», par le seul fait qu’il est
    verbal; comme l’atome suffixe «re» est lui-même équivalent à «ag»,
    on voit que l’idée verbale «ag» ou \textbf{i} est exprimée deux
    fois dans le mot «écri-re», ainsi que dans tous les verbes dont le
    radical est verbal, comme «ag-ir», }%
  {\emph{Remark}.\blfootnote{[ma]nuscript of this essay before its
      publication, makes a very interesting remark here: ``If, he
      says, we take the psychological point of view, we observe that a
      compound word always tends little by little to be conceived as a
      simple word: the adjective idea, which is initially contained
      only in the root \emph{beau} `beautiful' of the word \emph{beauté}
      `beauty' ends up by infecting and penetrating the suffix
      \emph{té} such that the latter draws (psychologically, if not
      logically) on the adjective idea and the nominal
      idea. Conversely, in the word \emph{couronner}, the root
      \emph{couronn} is contaminated by the verbal idea contained in the
      suffix \emph{er}.'' --- Prof. Bally's remark seems to me quite
      accurate; nevertheless, for my ends, and also as the very title
      of this booklet indicates, the logical point of view is the
      only only one that should be taken into consideration here.  }
    --- The {[infinitive]} verbal suffixes \emph{er}, \emph{ir}, etc. attach
    not only to root noun atoms (like \emph{couronne} `crown', \emph{clou}
    `nail', etc.) or adjectives (like \emph{grand} `big, tall', \emph{gros}
    `large, fat' etc.) but also to verbal atoms (such as \emph{écri}
    `write', \emph{ouvr} `open', \emph{abdic} `renounce'), since one
    says: \emph{écri-re}`(to) write', \emph{ouvr-ir} `(to) open',
    \emph{abdiqu-er} `(to) abdicate, renounce', etc.  From a logical
    point of view, there is a pleonasm in the word \emph{écrire} `(to)
    write', since the root \emph{écri} already contains in itself the
    dynamic idea ``ag'', from the very fact that it is verbal; as the
    suffix atom \emph{re} is itself equivalent to ``ag'', we see that the
    verbal idea ``ag'' or \textbf{i} is expressed twice in the word
    \emph{écri-re}, as well as in all verbs whose root is verbal, such
    as \emph{ag-ir} `to act', }

  \TextPage{\noindent «abdiqu-er», etc. On peut même dire que
    \emph{ag-ir} = \emph{ag-ag} = \textbf{i}-\textbf{i}, puisque
    l’atome «ir» est interchangeable avec l’atome «ag», ou l’atome
    symbolique \textbf{i}.

    Au point de vue purement logique, de tels pléonasmes sont
    contraires à l’un des principes qui gouvernent la formation des
    mots, comme nous le verrons en parlant de la synthèse des mots
    composés. Un suffixe doit toujours introduire dans un mot une idée
    qui n’y était pas encore contenue, c’est-à-dire que l’on ne doit
    pas répéter inutilement la même idée une ou plusieurs fois dans le
    même mot. Mais comme un même atome peut contenir en même temps une
    idée générale et une idée particulière, il y a des cas où l’on est
    obligé de répéter une idée générale déjà exprimée, afin
    d’introduire l’idée particulière non encore exprimée. Dans ces
    cas, le pléonasme est inévitable; lorsque l’on dit «frappez»,
    l’idée verbale exprimée par l’atome suffixe «ez» est déjà exprimée
    par l’atome radical «frap», mais cette répétition est permise
    parce que l’atome «ez» apporte dans le mot, outre l’idée générale
    «ag», les trois idées particulières de présent, d’impératif et de
    deuxième personne du pluriel. Du reste, ces pléonasmes n’ont aucun
    inconvénient; ils ne changent pas le sens du mot, car «écrire» est
    la même chose que «faire l’action écrire», de même que «beau» = «
    qui est beau » ou «de qualité beau», parce que l’atome «qui» et la
    molécule «de qua- }%»
  {\noindent \emph{abdiqu-er} `abdic-ate', etc. We could even say that
    \emph{ag-ir} `(to) act' = \emph{ag-ag} = \textbf{i}-\textbf{i},
    since the atom \emph{ir} is interchangeable with the atom ``ag'',
    or the symbolic atom \textbf{i}.

    From a purely logical point of view, such pleonasms are contrary
    to one of the principles that govern the formation of words, as we
    will see in discussing the synthesis of compound words. A suffix
    should always introduce into a word an idea that is not already
    contained within it, that is, one should not repeat uselessly the
    same idea one or several times within the same word.  But as the
    same atom can contain at the same time a general idea and a
    particular idea, there are cases where one is obliged to repeat a
    general idea that has already been expressed, in order to
    introduce the particular idea that has not yet been expressed. In
    those cases, the pleonasm is inevitable; when we say
    \emph{frapp-ez} `strike {[\textsc{2pl Imperative]}}', the verbal
    idea expressed by the suffix \emph{ez} `\textsc{2pl}' is already
    expressed by the root atom \emph{frap} `strike', but this repetition
    is permitted because the atom \emph{ez} brings to the word, besides
    the general idea ``ag'', the three particular ideas of present,
    imperative, and second person plural. Besides, these pleonasms
    are not disadvantageous: \emph{they do not change the sense of the
      word,} since \emph{écrire} `(to) write' is the same as
    ``perform the action \emph{écrire}'', just as \emph{beau}
    `beautiful' = ``which is beautiful'' or ``of beautiful quality'',
    since the atom \emph{qui} `which' and the molecule ``of
    qua[lity''] }

  \TextPage{\noindent lité» sont équivalents, à l’idée adjective
    \textbf{a}, laquelle est déjà contenue dans «beau».

    Ayant ainsi achevé le dénombrement des atomes
    fondamentaux\footnote{Quand je dis dénombrement, je n’entends pas
      dénombrement complet, car pour ne parler que de la langue
      française, il existe d’autres suffixes fondamentaux que ceux}
    (adjectifs, substantifs ou verbaux), c’est-à-dire des atomes qui
    ne contiennent rien d'autre que l’une des idées fondamentales
    \textbf{a}, \textbf{o} ou \textbf{i}, on peut maintenant procéder
    au classement de tous les atomes particuliers, c’est-à-dire des
    radicaux et des affixes qui contiennent une idée particulière en
    plus de l’idée fondamentale correspondante. On obtiendra ainsi un
    vocabulaire disposé comme suit:

    \begin{center}
      1. \textsc{Atomes fondamentaux}.
    \end{center}

    \noindent
    \resizebox{\linewidth}{!}{\mbox{\begin{tabular}[t]{l}
      \multicolumn{1}{c}{Adjectifs.}\\
      \multicolumn{1}{c}{\textbf{a}}\\
      \emph{qual}\\
      \emph{propre} (à)\\
      \emph{de}\\
      \emph{qui}\\
      -\emph{ain} (suffixe)\\
      -\emph{ique} (	»	)\\
      -\emph{eux} (	»	)\\
      -\emph{el} (	»	)\\
      \multicolumn{1}{c}{etc., etc.}\\
    \end{tabular}
    \begin{tabular}[t]{|l|}
      \multicolumn{1}{|c|}{Substantifs.}\\
      \multicolumn{1}{|c|}{\textbf{o}}\\
      \emph{ens} (entité)\\
      \emph{le} (ou \emph{un})\\
      \emph{ce}\\
      -\emph{ité} (suffixe)\\
      -\emph{esse} (	» )\\
      -\emph{eur} ( » )\\
      -\emph{tion} (	» )\\
      -\emph{ment}(	» )\\
      -\emph{ture} (	» )\\
      \multicolumn{1}{|c|}{etc., etc.}\\
    \end{tabular}
    \begin{tabular}[t]{l}
      \multicolumn{1}{c}{Verbes.}\\
      \multicolumn{1}{c}{\textbf{i}}\\
      \emph{ag} (ou \emph{sta})\\
      \emph{fai}\\
      \emph{to} (en\\
      \multicolumn{1}{r}{anglais)}\\
      \emph{er} (suffixe)\\
      \emph{re} (» )\\
      \emph{ir} (» )\\
      \multicolumn{1}{c}{etc., etc.}
    \end{tabular}}}
  } {\noindent {[qua]}lity'' are equivalent to the adjectival idea
    \textbf{a}, which is already contained in \emph{beau} `beautiful'.

    Having thus achieved the enumeration of the basic
    atoms\footnote{When I say enumeration, I do not mean complete
      enumeration, because even speaking only of the French language,
      there exist other basic suffixes than those} (adjectival,
    nominal or verbal), that is, the atoms that contain nothing other
    than the basic ideas \textbf{a}, \textbf{o}, or \textbf{i}, we can
    now proceed to the classification of all of the specific atoms,
    that is, of the roots and affixes that contain a particular idea
    in addition to the corresponding basic idea. We thus obtain a
    vocabulary organized as follows:

    \begin{center}
      1. \textsc{Basic atoms}.
    \end{center}

    \noindent
    \resizebox{\linewidth}{!}{\mbox{\begin{tabular}[t]{l}
      \multicolumn{1}{c}{Adjectives}\\
      \multicolumn{1}{c}{\textbf{a}}\\
      \emph{qual}\\
      \emph{propre} (à)\\
      \emph{de}\\
      \emph{qui}\\
      -\emph{ain} (suffix)\\
      -\emph{ique} (	")\\
      -\emph{eux} (	"	)\\
      -\emph{el} (	"	)\\
      \multicolumn{1}{c}{etc., etc.}\\
    \end{tabular}
    \begin{tabular}[t]{|l|}
      \multicolumn{1}{|c|}{Nouns}\\
      \multicolumn{1}{|c|}{\textbf{o}}\\
      \emph{ens} (entity)\\
      \emph{le} (or \emph{un})\\
      \emph{ce}\\
      -\emph{ité} (suffix)\\
      -\emph{esse} (	" )\\
      -\emph{eur} ( " )\\
      -\emph{tion} (	" )\\
      -\emph{ment}(	" )\\
      -\emph{ture} (	" )\\
      \multicolumn{1}{|c|}{etc., etc.}\\
    \end{tabular}
    \begin{tabular}[t]{l}
      \multicolumn{1}{c}{Verbs}\\
      \multicolumn{1}{c}{\textbf{i}}\\
      \emph{ag} (or \emph{sta})\\
      \emph{fai}\\
      \emph{to} (in\\
      \multicolumn{1}{r}{English)}\\
      \emph{er} (suffix)\\
      \emph{re} (" )\\
      \emph{ir} ( " )\\
      \multicolumn{1}{c}{etc., etc.}
    \end{tabular}
  }}}

  \TextPage{\blfootnote{que j’ai cités. Ainsi le suffixe \emph{ise}
      (dans \emph{gourmandise}), est le même atome que \emph{ité}
      (daus \emph{égalité}), c'est-à-dire que ce suffixe \emph{ise},
      et d’autres encore, est égal à l’idée substantive \textbf{o}. Le
      but de cette étude est de proposer une méthode d’analyse, non
      d’appliquer cette méthode d’une manière complète à une langue
      particulière.}
    \begin{center}
      2. \textsc{Atomes particuliers}.
    \end{center}

    \noindent
    \resizebox{\linewidth}{!}{\mbox{\begin{tabular}[t]{l}
      \multicolumn{1}{c}{Adjectifs}\\[.5ex]
      \emph{grand}\\
      \emph{beau} \\
      \emph{fort}\\
      \emph{riche}\\
      \emph{bon}\\
      \emph{lourd}\\
      \multicolumn{1}{c}{etc., etc.}\\
      \ \\
      \ \\
      \ \\
      -\emph{able}\\
      \multicolumn{1}{r}{(suffixe)}\\
      \ \\
      \ \\
      \ \\
      \multicolumn{1}{c}{etc., etc.}\\
    \end{tabular}
    % \hspace{-.3em}
    \begin{tabular}[t]{|l|}
      \multicolumn{1}{|c|}{Substantifs}\\[.5ex]
      \emph{homme}\\
      \emph{cheval}\\
      \emph{table}\\
      \emph{âme}\\
      \emph{éspace}\\
      \emph{science}\\
      \emph{théorie}\\
      \multicolumn{1}{|c|}{etc., etc.}\\
      \ \\
      \ \\
      -\emph{iste} (suffixe)\\
      -\emph{eur}\footnotemark\ ( » )\\
      -\emph{oir}\footnotemark\ ( » )\\
      -\emph{ie}\footnotemark\  ( » )\\
      -\emph{ard}\footnotemark\ ( » )\\
      \multicolumn{1}{|c|}{etc., etc.}\\
    \end{tabular}
%     \hspace{-1.5em}
    \begin{tabular}[t]{l}
      \multicolumn{1}{c}{Verbes.}\\[.5ex]
      \emph{écri}re)\\
      \emph{frapp}(er)\\
      \emph{abdiqu}(er)\\
      \emph{tend}(re)\\
      \emph{coud}(re)\\
      \emph{dorm}(ir)\\
      \emph{souffr}(ir)\\
      \multicolumn{1}{r}{etc., etc.}\\
      \ \\
      \emph{ifi} (suffixe)\\
      \emph{is} ( » )\\
      \multicolumn{1}{r}{etc., etc.}
      \ \\
      \ \\
      \ \\
      \ \\
    \end{tabular}}}
    \setcounter{footnote}{1}\footnotetext{Dans \emph{bross-eur}.}
    \stepcounter{footnote}\footnotetext{Dans \emph{abreuv-oir}.}
    \stepcounter{footnote}\footnotetext{Dans \emph{brosser-ie}.}
    \stepcounter{footnote}\footnotetext{Dans \emph{vieill-ard}.}
  }%
  {\blfootnote{that I have cited. Thus the suffix \emph{ise} (in
      \emph{gourmandise} `gluttony' is the same atom as \emph{ité}
      (in \emph{égalité} `equality'), that is, this suffix
      \emph{ise} among others is equal to the nominal idea
      \textbf{o}. The goal of this study is to propose a method of
      analysis, not to apply that method in a complete fashion to a
      specific language.}
    \begin{center}
      2. \textsc{Particular atoms}.
    \end{center}

    \noindent
    \resizebox{\linewidth}{!}{\mbox{\begin{tabular}[t]{l}
      \multicolumn{1}{c}{Adjectives}\\[.5ex]
      \emph{tall}\\
      \emph{beautiful} \\
      \emph{strong}\\
      \emph{rich}\\
      \emph{good}\\
      \emph{heavy}\\
      \multicolumn{1}{c}{etc., etc.}\\
      \ \\
      \ \\
      \ \\
      -\emph{able}\\
      \multicolumn{1}{r}{(suffix)}\\
      \ \\
      \ \\
      \ \\
      \multicolumn{1}{c}{etc., etc.}\\
    \end{tabular}
%     \hspace{-.3em}
    \begin{tabular}[t]{|l|}
      \multicolumn{1}{|c|}{Nouns}\\[.5ex]
      \emph{man}\\
      \emph{horse}\\
      \emph{table}\\
      \emph{soul}\\
      \emph{space}\\
      \emph{science}\\
      \emph{theory}\\
      \multicolumn{1}{|c|}{etc., etc.}\\
      \ \\
      \ \\
      -\emph{iste} (suffixe)\\
      -\emph{eur}\footnotemark ( " )\\
      -\emph{oir}\footnotemark ( " )\\
      -\emph{ie}\footnotemark ( " )\\
      -\emph{ard}\footnotemark ( " )\\
      \multicolumn{1}{|c|}{etc., etc.}\\
    \end{tabular}
%     \hspace{-1.7em}
    \begin{tabular}[t]{l}
      \multicolumn{1}{c}{Verbs.}\\[.5ex]
      (to) \emph{write}\\
      (to) \emph{strike}\\
      (to) \emph{abdicate}\\
      (to) \emph{stretch}\\
      (to) \emph{sew}\\
      (to) \emph{sleep}\\
      (to) \emph{suffer}\\
      \multicolumn{1}{r}{etc., etc.}\\
      \ \\
      \emph{ifi} (suffix)\footnotemark\\
      \emph{is} ( " )\footnotemark\\
      \multicolumn{1}{r}{etc., etc.}
      \ \\
      \ \\
      \ \\
      \ \\
    \end{tabular}
  }}
    \setcounter{footnote}{1}\footnotetext{In
      \emph{b[r]oss-eur} `hard worker';
      cf. \emph{bosser} `to work hard'.}
    \stepcounter{footnote}\footnotetext{In
      \emph{abreuv-oir} `drinking trough';
      cf. \emph{abreuver} `to water (an animal)'.}
    \stepcounter{footnote}\footnotetext{In
      \emph{brosser-ie} `brushery'; cf. \emph{brosser}
      `to brush'.}
    \stepcounter{footnote}\footnotetext{In
      \emph{vieill-ard} `old man'; cf. \emph{vieux,
        vieil} `old'.}
    \setcounter{footnote}{5}\footnotetext{In
      \emph{sign-ifi-er} `to signify'}
    \stepcounter{footnote}\footnotetext{In
      \emph{signal-is-er} `to signalize'}
    }
  
  \TextPage{
    \begin{center}
      § 2. — \textbf{Etude des molécules}.
    \end{center}
    \addcontentsline{toc}{subsection}{2 Etude des molécules}

    Tout mot composé de plusieurs atomes (radicaux ou affixes) est une
    \emph{molécule}. Ainsi le mot «hum-an-ité» est une molécule
    contenant trois atomes: l’atome substantif «hom» (atome
    particulier), l’atome adjectif «an» (atome général) et l’atome
    substantif «ité» (atome général). Nous avons vu, en outre, que
    certaines expressions composées de plusieurs mots, comme
    «d’homme», «le beau», etc., doivent être considérées comme des
    molécules à l’état dissocié.

    \textsc{Classement des molécules}. — On peut classer les mots
    composés en trois classes principales, comme les atomes. Du reste,
    le classement des molécules est immédiatement déterminé par celui
    des atomes, car on constate facilement que \emph{la classe d’une
      molécule est celle de son dernier atome}. Cette règle est très
    importante; elle montre qu’à ce point de vue il n’y a pas de
    différence entre les mots composés d’atomes-radicaux (comme
    «Schlafzimmer» en allemand) et les mots dits dérivés, c’est-à-dire
    composés de radicaux et d’affixes: ainsi, de même que le mot
    «Schlafzimmer» est un substantif, parce que son dernier atome
    «Zimmer» en est un; de même le mot «humanité» est substantif parce
    que son dernier atome «ité» est un atome substantif; le }%
  {\begin{center}
      § 2. — \textbf{The study of molecules}.
    \end{center}

    Every word composed of several atoms (roots or affixes) is a
    \emph{molecule}. Thus, the word \emph{hum-an-ité} `humanity' is
    a molecule containing three atoms: the noun atom \emph{hom}
    (particular atom), the adjective atom \emph{an} (general atom),
    and the noun atom \emph{ité} (general atom). We have seen in
    addition that certain expressions like \emph{d'homme} `of man',
    \emph{le beau} `the beautiful', etc. must be considered as
    molecules in the dissociated state.

    \textsc{Classification of molecules.}  --- We can categorize
    compound words into three main classes, like atoms. In addition,
    the classification of molecules is immediately determined by that
    of atoms, because we can observe easily that \emph{the class of a
      molecule is that of its final atom}. This rule is very
    important: it shows that from this point of view there is no
    difference between words compounded of root atoms (like
    \emph{Schlafzimmer} `bedroom' in German) and words said to be
    derived, that is, composed of roots and affixes: thus just as the
    word \emph{Schlafzimmer} is a noun, because its final atom
    \emph{Zimmer} `room' is one; similarly, the word \emph{humanité}
    is a noun because its final atom \emph{ité} is a nominal atom;
    the }

  \TextPage{\noindent mot «humain» est un adjectif parce que «ain» est
    un atome adjectif; le mot «clouer» est un verbe parce que «er» est
    un atome verbal, et ainsi de suite.

    Toutefois, dans les molécules dissociées, c'est-à-dire dans les
    expressions «le beau», «un vieux», «le boire», «to crown», etc.,
    ainsi que dans les mots composés comme «bateau à vapeur», «machine
    à coudre», «un porte-plume», etc., l’ordre des atomes est
    renversé, c’est-à-dire que c’est le premier atome, et non le
    dernier, qui détermine le classement de la molécule dissociée. En
    effet, «\emph{le} beau» «beau-\emph{té}», «\emph{un} (homme)
    vieux» = «vieill-\emph{ard}», «\emph{le} boire» =
    «bois-\emph{son}», « \emph{to} crown = cou-ronn-\emph{er} »,
    «\emph{chambre} à coucher» = «Schlaf-\emph{Zimmer}»,
    «\emph{machine} à coudre» = «sewing-\emph{machine}», «\emph{un}
    (objet) porte-plume» = «Federhalt-\emph{er}», «\emph{un} (objet)
    porte-chandelle» = «chandel-\emph{ier}», «(\emph{argent}) pour
    boire» = «Trink-\emph{geld}», etc., etc. Il faut donc compléter la
    règle de classement des molécules, en disant: \emph{La classe
      d'une molécule est celle de son dernier atome, à moins que la
      molécule ne soit dissociée; dans ce cas, la classe est celle du
      premier atome}, à cause du renversement atomique qui est produit
    par la dissociation.

    Au fond, l’ordre des atomes dans les molécules dissociées est
    l’ordre analytique, explicatif, et c’est la condensation, la
    synthèse de la molécule en un}%
  {\noindent the word \emph{humain}
    `human' is an adjective because \emph{ain} is an adjective atom; the
    word \emph{clouer} `to nail' is a verb because \emph{er} is a verbal
    atom, and so on.

    However, in dissociated molecules, that is in the expressions \emph{le
    beau} `the beautiful', \emph{un vieux} `an old (man)', \emph{le boire}
    `the drink', \emph{to crown}, etc., as well as in compound words like
    \emph{bateau à vapeur} `boat of steam: steamboat', \emph{machine à
    coudre} `machine to sew: sewing machine', \emph{un porte-plume} `a
    carry-pen: a penholder', etc., the order of atoms is reversed;
    that is, it is the first atom and not the last that determines the
    classification of the dissociated molecule.  Actually, ``\emph{le}
    beau'' [=] ``beau-\emph{té}'', ``\emph{un} (homme) vieux'' =
    ``vieill-\emph{ard}'', ``\emph{le} boire'' = ``bois-\emph{son}'',
    ``\emph{to} crown'' = ``couronn-\emph{er}'', ``\emph{chambre} à
    coucher'' `room to sleep' = ``Schlaf-\emph{zimmer}'',
    ``\emph{machine} à coudre'' = ``sewing \emph{machine}'',
    ``\emph{un} (objet) porte-plume'' = ``Federhalt-\emph{er}'',
    ``\emph{un} (object) porte-chandelle'' `an (object) carry-candle'
    = ``chandel-\emph{ier}'' `candlestick', ``(\emph{argent}) pour
    boire'' `(money) for drink' = ``Trink-\emph{geld}'' `drink-money:
    tip', etc. etc. We must thus supplement the classification rule
    for molecules by saying: \emph{The class of a molecule is that of
      its final atom, unless the molecule is dissociated: in that
      case, the class is that of its first atom,} because of the
    reversal of atoms that is produced by dissociation.

    Basically, the order of atoms in dissociated molecules is the
    \emph{analytic, explanatory} order, and it is condensation, the
    synthesis of a molecule into a }

  \TextPage{\noindent seul mot, qui renverse l’ordre des atomes et
    produit l’ordre synthétique. La soudure entre les atomes est donc
    l’effet provoqué par le renversement de leur ordre naturel. Mais,
    quel que soit l’ordre logique des atomes, la seule chose qui nous
    intéresse ici, c’est que dans la molécule dissociée l'ordre est
    inverse de ce qu’il est dans la molécule condensée.

    Pour analyser une molécule, il faut tenir compte de tout ce
    qu’elle contient, c’est-à-dire de tout ce que contient chacun des
    atomes qui la composent.

    Or, nous avons vu qu’en général un atome exprime non seulement une
    idée particulière, mais qu’il contient implicitement une ou
    plusieurs idées plus générales (et en particulier une idée
    grammaticale), qui accompagnent toujours cet atome et dont il faut
    tenir compte si l’on veut faire une analyse complète de la
    molécule.

    Cependant, nous savons qu’il existe un certain nombre d’atomes qui
    ne contiennent qu’une idée générale (\textbf{a}, \textbf{o} ou
    \textbf{i}) et aucune idée particulière: ce sont les atomes
    fondamentaux. Ces atomes ont une constitution plus simple que les
    atomes particuliers; ils montrent à nu tout leur contenu et ne
    cachent rien dans leur intérieur. Par conséquent, si un mot
    composé ne contient que des atomes fondamentaux, son analyse sera
    toute simple, parce que le sens total du mot résultera
    immédiatement de la } %
  {\noindent single word, which reverses the order of the atoms and
    produces the synthetic order. The fusion of the atoms is thus the
    effect provoked by the reversal of their natural order. But
    whatever the logical order of the atoms may be, the only thing
    that interests us here is that in the dissociated molecule the
    order is the opposite of what it is in the condensed molecule.

    To analyze a molecule, it is necessary to take account of
    everything that it contains, that is of all that is contained in
    each of the atoms that make it up.

    Now we have seen that in general, an atom expresses not only a
    specific idea, but that it implicitly contains one or several
    more general ideas (and in particular, a grammatical idea) which
    always go along with this atom and of which it is necessary to
    take account if we wish to make a complete analysis of the
    molecule.

    However, we know that there exist a certain number of atoms that
    contain only a general idea (\textbf{a}, \textbf{o}, or
    \textbf{i}), and no specific idea: these are the fundamental
    atoms. These atoms have a simpler composition than the specific
    atoms: they show their content openly and hide nothing in their
    interior. Consequently, if a compound word contains only
    fundamental atoms, its analysis will be quite simple, since the
    complete sense of the word results immediately from the }

  \TextPage{\noindent juxtaposition des idées contenues dans les
    différents atomes, à raison d’une seule idée par atome.

    \begin{center}
      A). \textsc{Molécules fondamentales}.
    \end{center}

    Il est donc naturel, avant d’aborder l’analyse d’un mot
    quelconque, d’étudier d’abord toutes les molécules que l’on peut
    obtenir en combinant entre eux les atomes fondamentaux. Ces
    molécules seront appelées \emph{fondamentales} ou \emph{mots
      fondamentaux}; en effet, puisque les atomes fondamentaux ne
    contiennent que les idées les plus générales (idées
    grammaticales), ces atomes expriment les idées fondamentales du
    langage, c'est-à-dire les idées abstraites qui servent de modèle,
    de chef de file aux idées particulières; il est donc à présumer
    que tous les mots composés uniquement d’atomes fondamentaux
    exprimeront aussi des idées fondamentales, des idées abstraites
    servant de modèle, de chef de file à des séries correspondantes
    d’idées particulières, celles-ci étant exprimées par des mots
    composés, de même type, mais contenant des atomes particuliers. Il
    en est en effet ainsi, et puisque nous connaissons déjà les atomes
    fondamentaux \textbf{a}, \textbf{o}, \textbf{i}, nous pouvons
    passer à l’étude des molécules fondamentales composées de deux
    atomes, c’est-à-dire des molécules:

    \begin{center}
      \textbf{(a-o)}, \textbf{(i-o)}; \textbf{(o-a)},
      \textbf{(o-i)};\textbf{ (a-i)}, \textbf{(i-a)}.
    \end{center}
  }%
  {\noindent juxtaposition of the ideas contained in the different
    atoms, at the rate of one single idea per atom.

    \begin{center}
      A). \textsc{Fundamental Molecules}.
    \end{center}

    It is thus natural, before undertaking the analysis of any word,
    to first study all of the molecules that we can obtain by
    combining the fundamental atoms with one another. These molecules
    will be called \emph{fundamental}, or \emph{fundamental words}:
    indeed, since the fundamental atoms contain only the most general
    ideas (grammatical ideas), these atoms express the fundamental
    ideas of the language, that is, the abstract ideas that serve as
    the model, as the leaders of the corresponding series of specific
    ideas; it is thus to be assumed that all words composed
    exclusively of fundamental atoms will also express fundamental
    ideas, abstract ideas that serve as the model, the leaders of the
    corresponding series of specific ideas, with these being expressed
    by compound words of the same type but containing specific
    atoms. This is the way things are, and since we already know the
    fundamental atoms \textbf{a}, \textbf{o}, \textbf{i}, we can move
    on to the study of fundamental molecules composed of two atoms,
    that is, the molecules:

    \begin{center}
      \textbf{(a-o)}, \textbf{(i-o)}; \textbf{(o-a)},
      \textbf{(o-i)};\textbf{ (a-i)}, \textbf{(i-a)}.
    \end{center}
  }

  \TextPage{
    \begin{center}
      \emph{Molécules fondamentales biatomiques}.
    \end{center}


    1. \textsc{Molécule} \textbf{(a-o)}. — Pour trouver l’équivalent
    en français de la molécule \textbf{(a-o)}, il suffît de remplacer
    les atomes fondamentaux \textbf{a} et \textbf{o} par un de leurs
    synonymes français choisis dans le tableau de la page 40. On doit
    seulement remarquer que dans la molécule \textbf{(a-o)}, l’atome
    \textbf{o} est la finale du mot et l’atome \textbf{a} en est le
    radical; il faut donc choisir le synonyme de \textbf{a} sous forme
    d’atome radical et celui de \textbf{o} sous forme
    d’atome-suffixe. Or, l’atome général adjectif \textbf{a} est
    exprimable par l’atome-radical «qual» ou par l’atome-radical
    «propre (à)»; on a donc symboliquement: molécule \textbf{(a-o)} =
    \emph{qual}-\textbf{o} ou \emph{propr}-\textbf{o}.

    L’atome \textbf{o} représente l’idée substantive générale «ce (qui
    est)», «l’être en général», «l’être abstrait», laquelle est
    exprimée en français par les suffixes synonymes «ité», «eur»,
    «esse», etc., lorsque cette idée suit un atome adjectif, comme
    «qual», par conséquent \textbf{(a-o)} = \emph{qual}-\textbf{o} =
    \emph{qual-ité}.

    Le mot «qualité» est donc un mot fondamental de la langue
    française, un mot exprimant une idée essentiellement abstraite,
    car il ne contient que des atomes généraux fondamentaux; c’est
    \emph{l'adjectivo-substantif} type, puisqu’on a:

    \begin{center}
      \emph{qual-ité} = \textbf{(a-o)}
    \end{center}
  }%
  {\begin{center} \emph{Biatomic fundamental Molecules}.
    \end{center}


    1. \textsc{Molecule} \textbf{(a-o)}. — To find the equivalent in
    French of the the molecule \textbf{(a-o)}, it suffices to replace
    the fundamental atoms \textbf{a} and \textbf{o} by one of their
    French synonyms found in the table on page 40.  We must only
    observe that in the molecule \textbf{(a-o)}, the atom \textbf{o}
    is final in the word and the atom \textbf{a} is the root; it is
    thus necessary to choose the synonym of \textbf{a} in the form of
    a root atom and that of \textbf{o} in the form of a suffix
    atom. Now the general adjective atom \textbf{a} can be expressed
    by the root atom ``qual'' or by the root atom \emph{propre (à)}; we
    thus have symbolically: molecule \textbf{(a-o)} =
    \emph{qual}-\textbf{o} or \emph{propr}-\textbf{o}.

    The atom \textbf{o} represents the general nominal idea ``that
    (which is)'', ``existence in general'', ``abstract existence'',
    which is expressed in French by the synonymous suffixes \emph{ité},
    \emph{eur}, \emph{esse}, etc. Since this idea follows an adjective idea
    such as ``qual'', it follows that \textbf{(a-o)} =
    \emph{qual}-\textbf{o} = \emph{qual-ité}.

    The word \emph{qualité} is thus a fundamental word of the French
    language, a word expressing an essentially abstract idea, because
    it only contains general fundamental atoms: it is the protoypical
    example of an \emph{adjectivo-nominal}, since we have:
    
    \begin{center}
      \emph{qual-ité} = \textbf{(a-o)}
    \end{center}
  }

  \TextPage{\noindent et ce mot servira de modèle à tous les
    adjectivo-substantifs particuliers (tels que «égal-ité», «bon-té»,
    «grand-eur», «rich-esse», etc.). Tous ces mots sont des molécules
    biatomiques irréductibles à un seul atome; ils expriment tous des
    «qualités», car ils se composent tous d’un radical adjectif
    particulier suivi de l’atome substantif général.

    On peut aussi écrire \emph{propr}-\textbf{o} =
    \emph{propri}-\emph{été}, donc aussi:

    \begin{center}
      \emph{propri-été} = \textbf{(a-o)}
    \end{center}


    \noindent ou en allemand:

    \begin{center}
      \emph{Eigen}-\emph{schaft} = \textbf{(a-o)}
    \end{center}

    \noindent car nous avons vu qu’il n’y a guère de différence entre
    l’idée «qual» et l’idée «propre (à)» ou, en allemand, «eigen»; ce
    qui justifie le terme «Eigenschaftswort» qui, en allemand, sert à
    désigner l’adjectif.

    Quant au sens de la molécule \textbf{(a-o)}, c’est-à-dire du mot
    «qualité», il résulte immédiatement de la juxtaposition des sens
    de ses atomes constituants, puisque ces atomes sont tous deux
    fondamentaux et ne contiennent rien de sous-entendu; l’atome
    \textbf{o} exprime l’idée générale substantive «ce», «ce (qui
    est)», et l’atome \textbf{a} l’idée générale adjective «qual»,
    «(qui est) qual», ou «propre (à)», «(qui est) propre (à)». Donc
    les mots «qualité», «propriété» signifient «\emph{ce} qui est
    qual», ou «ce qui est propre à». } %
  {\noindent and this word will
    serve as a model for all specific adjectivo-nominals (such as
    \emph{egal-ité} `equality', \emph{bon-té} `goodness', \emph{grand-eur}
    `size', \emph{rich-esse} `riches', etc.).

    All these words are biatomic molecules that cannot be reduced to a
    single atom; they all express ``qualities'', because they are all
    composed of a specific root adjective followed by the general
    nominal atom.

    We can thus also write \emph{propr}-\textbf{o} =
    \emph{propri}-\emph{été}, and therefore also:

    \begin{center}
      \emph{propri-été} = \textbf{(a-o)}
    \end{center}

    \noindent or in German:

    \begin{center}
      \emph{Eigen}-\emph{schaft} = \textbf{(a-o)}
    \end{center}

    \noindent because we have seen that there is no real difference
    between the idea ``qual'' and the idea ``proper (to)'' or, in
    German, \emph{eigen} `own, characteristic'; which is what
    justifies the term \emph{Eigenschaftswort} that, in German, serves
    to designate the adjective.

    As for the sense of the molecule \textbf{(a-o)}, that is, of the
    word \emph{qualité} `quality', this results immediately from the
    juxtaposition of the senses of its constituent atoms, since these
    atoms are both fundamental and do not contain anything understood:
    the atom \textbf{o} expresses the general nominal idea ``this,
    that'', ``that (which exists)'', and the atom \textbf{a} the
    general adjectival idea ``qual'', ``(which is) qual'', or ``proper
    (to)'', ``(which is) proper (to)''. Thus, the words \emph{qualité},
    \emph{propriété} `property' mean ``(\emph{that}) which is qual''
    or ``(\emph{that}) which is proper to''.  }

  \TextPage{\noindent Ainsi, par exemple, la phrase: \emph{La qualité
      de cette étoffe est mauvaise}, signifie: \emph{ce qui est
      «qual»}, (on pourrait même dire \emph{ce qui est de caractère
      adjectif}) \emph{dans cette étoffe est mauvais}.

    De même l’expression: \emph{Les propriétés d’un corps} signifie
    «\emph{ce qui est propre à ce corps}», «\emph{ce qui, dans ce
      corps, est de caractère adjectif}». Par exemple, l’étoffe ou le
    corps dont nous venons de parler est «lisse» ou «rugueux», «rouge»
    ou «noir», «lourd» ou «léger», etc.; autant d’adjectifs pour
    exprimer les qualités ou les propriétés d’un corps.

    Remarquons que la soudure entre deux atomes fondamentaux est bien
    une simple juxtaposition: puisque l’atome «ité» = «ce» et que
    «qual» = «qui est qual», la molécule \emph{qual-ité}= «\emph{ce}
    (qui est) \emph{qual}». Les deux membres de cette égalité sont
    identiques; l’ordre des atomes est seulement renversé, mais cela
    doit être, car on peut considérer l’expression «ce—qui est qual»
    comme une molécule biatomique à l’état dissocié, et nous savons
    que la condensation, la synthèse de la molécule, provoque le
    renversement de l’ordre de ses atomes.

    2. \textsc{Molécule} \textbf{(i-o)}. — Pour trouver l’équivalent,
    en français, de la molécule fondamentale \textbf{(i-o)}, il suffit
    de remplacer les atomes i et o par un de leurs synonymes français
    donnés dans le tableau de la page 40, en ayant soin de prendre le
    synonyme de \textbf{i} sous la

  } %
  {\noindent Thus, for example, the sentence \emph{La qualité de cette
    étoffe est mauvaise} `The quality of this fabric is bad' means
    \emph{that which is ``qual''} (we could even say \emph{that which
      is of adjective character}) \emph{in this fabric is bad.}

    Similarly the expression: \emph{Les propriétés d'un corps} `the
    properties of a body' means ``\emph{that which is proper to this
      body}'', ``\emph{that which, in this body, is of adjective
      character}''. For example, the fabric or the body that we were
    just speaking of is \emph{lisse} `smooth' or \emph{rugueux}
    `rough', \emph{rouge} `red' or \emph{noir} `black' \emph{lourd}
    `heavy' or \emph{léger} `light', etc.; as many adjectives as
    describe the qualities or properties of a body.

    We note that the juncture between two fundamental atoms is just a
    simple juxtaposition: since the atom \emph{ité} = ``that'' and
    ``qual'' = ``which is qual'', the molecule \emph{qual-ité} =
    ``\emph{that} (which is) \emph{qual}''. The two sides of this
    equation are identical: only the order of atoms is reversed, but
    that must be the case, because we consider the molecule ``that ---
    which is qual'' to be a biatomic molecule in the dissociated
    state, and we know that the condensation or synthesis of a
    molecule results in the reversal of the order of its atoms.

    2. \textsc{Molecule} \textbf{(i-o)}. To find the equivalent in
    French of the basic molecule \textbf{(i-o)}, it suffices to
    replace the atoms \textbf{i} and \textbf{o} with one of their French
    synonyms in the table on page 40, taking care to choose the
    synonym of \textbf{i} in the }

  \TextPage{\noindent forme d’un atome radical, et celui de \textbf{o}
    sous la forme d’une finale, c’est-à-dire d’un atome-suffixe.

    Or, le tableau montre que l’idée verbale générale \textbf{i} est
    synonyme, en français, des atomes-radicaux «ag» (agir, faire un
    acte, une «action») ou «êt», dans le sens «stat», (être, être dans
    un état, une «station»); on a donc d’abord \textbf{(i-o)} =
    \emph{ag}-\textbf{o} ou \emph{stat}-\textbf{o}. L’atome
    \textbf{o}, c’est-à-dire l’idée générale substantive «ce (qui
    est)», «ce (qui existe)», est représenté, en français, par les
    suffixes synonymes «tion», «ture», «ment», etc., lorsque cette
    idée suit un atome verbal comme «ag» ou «stat». On a donc
    finalement:

    \begin{center}
      \textbf{(i-o)} = \emph{ag}-\textbf{o} = \emph{ag}-\emph{tion} =
      \emph{ac}-\emph{tion}
    \end{center}

    Ou bien:

    \begin{center}
      \textbf{(i-o)} = \emph{stat}-\textbf{o} =
      \emph{sta}-\emph{tion}.
    \end{center}

    Ainsi, tout comme les mots «qualité», «propriété», les mots
    «action» et «station» (ou «état») sont des mots fondamentaux de la
    langue française, car ils ne contiennent que des atomes
    fondamentaux. De même que «qualité», «propriété», sont les types
    de l’adjectivo-substantif, de même les mots «action», «station»,
    sont les types du verbo-substantif. Ces mots désignent des êtres
    abstraits, qui serviront de modèle à tous les verbo-substantifs
    particuliers (comme «abdica-tion», «écri-ture», «abonne-ment»,
    etc.). Toutes ces idées particulières contiennent l'idée
    d’«action» (ou de «sta- }%»
  %
  {\noindent form of a root atom, and that of \textbf{o} in the form
    of a final, that is, of a suffix atom.

    Now the table shows that the general verbal idea \textbf{i} is
    synonymous, in French, with the root atoms ``ag'' (to act, to
    perform an act, an \emph{action}) or \emph{êt} in the sense ``stat''
    (to be, to be in a state, a ``position''); we therefore have,
    first, \textbf{(i-o)} = \emph{ag}-\textbf{o} or
    \emph{stat}-\textbf{o}. The atom \textbf{o}, that is the general
    nominal idea ``that (which is)'', ``that (which exists)'' is
    represented in French by the synonymous suffixes \emph{tion},
    \emph{ture}, \emph{ment} etc. when this idea follows a verbal atom like
    ``ag'' or ``stat''. Thus we have finally:

     \begin{center}
       \textbf{(i-o)} = \emph{ag}-\textbf{o} = \emph{ag}-\emph{tion} =
       \emph{ac}-\emph{tion}
     \end{center}

     or else:

    \begin{center}
      \textbf{(i-o)} = \emph{stat}-\textbf{o} =
      \emph{sta}-\emph{tion}.
    \end{center}

    Therefore, just like the words \emph{qualité} `quality',
    \emph{propriété} `property', the words \emph{action} and
    \emph{station} (or \emph{état} `state') are fundamental words of
    the French language, because they contain only fundamental
    atoms. Just as \emph{qualité}, \emph{propriété} are prototypes of
    the adjectivo-nominal, so the words \emph{action}, \emph{station}
    are prototypes of the verbo-nominal.  These words designate
    abstract entities which serve as models for all specific
    verbo-nominals (such as \emph{abdica-tion} `abdication',
    \emph{écri-ture} `writing', \emph{abonne-ment} `subscription,
    etc.). All of these specific ideas contain the diea of ``action''
    (or of ``sta[te''] }

  \TextPage{\noindent tion») et se composent d’un radical verbal
    particulier suivi de l’atome substantif général; ce sont des
    molécules biatomiques irréductibles à un simple atome.

    Quant au sens de la molécule \textbf{(i-o)}, c’est-à-dire des mots
    «action» et «station», il résulte immédiatement de la
    juxtaposition des sens de leurs atomes constituants, puisque ces
    atomes sont tous fondamentaux: l’atome \textbf{o} exprime l’idée
    substantive générale «ce (qui est)», et l’atome \textbf{i} l’idée
    verbale générale «ag» (agir) ou «sta» (stare). Donc la molécule
    \textbf{(i-o)} signifie «ce qui est \emph{ag}», «ce qui est
    \emph{agir}» ou bien «ce qui est \emph{sta}», «ce qui est
    \emph{stare}». Ainsi, par exemple, la phrase: \emph{L'action de
      cette machine est régulière}, signifie: \emph{ce qui est}
    «\emph{ag}», \emph{ce qui est} «\emph{agir}» (on pourrait même
    dire: \emph{ce qui est de caractère verbal}) \emph{dans cette
      machine est régulier}; de même la phrase: \emph{L'état de ces
      travaux est satisfaisant} signifie: \emph{ce qui est}
    «\emph{\emph{sta}}», \emph{ce qui est} «\emph{stare}» (ou \emph{ce
      qui est verbal}, dans le sens statique) \emph{dans ces travaux
      est satisfaisant}.

    Et de même que tout ce qui est qualité ou propriété d’un corps
    s’exprime par des adjectifs, de même tout ce qui est action ou
    état (station) s’exprime par des verbes. Ainsi la machine dont
    nous venons de parler «comprime», «concasse», «lamine», «coud»,
    [«]rabote», etc. De même l’état des travaux sera exprimé par des
    verbes neutres: «les travaux dorment, languissent, progressent»,
    etc. De même l'état des travaux sera exprimé par des verbes
    neutres: «les travaux dorment, languissent, progressent», etc.}
 %
  {\noindent {[sta]}te'') and are composed of a specific verbal root
    followed by the general nominal atom; they are biatomic molecules
    not reducible to a simple atom.

    As for the sense of the molecule \textbf{(i-o)}, that is of the
    words \emph{action} and \emph{station} `state', this follows
    immediately from the juxtaposition of the senses of their
    constituent atoms, since these atoms are all basic: the atom
    \textbf{o} expresses the general nominal idea ``that (which is)''
    and the atom \textbf{i} the general verbal idea ``ag''
    (\emph{agir} `to act') or ``sta'' (\emph{stare} `to be in a
    state').  Thus, the molecule \textbf{(i-o)} means ``that which is
    \emph{ag}'', ``that which is \emph{agir}'' or else ``that which is
    \emph{sta}'', ``that which is \emph{stare}''.  Thus, for example,
    the sentence \emph{L'action de cette machine est reguliére} `the
    action of this machine is regular' means \emph{that which is
      ``ag", that which is \emph{agir}} (we could even say \emph{that
      which is verbal in character}) \emph{in this machine is
      regular}; similarly, in the sentence \emph{L'état de ces travaux
      est satisfaisant} `the state of this job is satisfactory' means
    \emph{that which is ``sta'', that which is \emph{stare}} (or
    \emph{that which is verbal}, in the static sense) \emph{in this
      job is satisfactory}.

    And just as everything that is a quality or property of of a body
    is expressed by adjectives, similarly everything that is an action
    or a state is expressed by verbs. Thus, the machine we were just
    speaking of \emph{comprime} `compresses', \emph{concasse}
    `crushes', \emph{lamine} `rolls', \emph{coud} `sews',
    \emph{rabote} `planes, scrapes', etc. Similarly the state of the
    job will be expressed by neutral verbs: \emph{les travaux dorment,
      languissent, progressent'} `the job is dormant, languishes,
    progresses' etc.  }

  \TextPage{Il faut bien distinguer la «qualité» de
    l'«état». Cependant, il n’y a pas de limite fixe et précise entre
    ces deux notions; on peut, en effet, passer de la «qualité» à
    l’«état» d’une manière continue, comme on passe de la notion de
    froid à celle de chaud. Les qualités d’un corps sont «ce qui est
    propre» à ce corps, car on ne peut les séparer du corps sans
    altérer profondément la nature de celui-ci. La «qualité»
    \textbf{(a-o)} est adjective, donc indépendante du temps. Au
    contraire, l’«état» \textbf{(i-o)} est verbal, donc passager,
    temporel, et il n’affecte pas la nature même du corps, de l’être
    qui subit cet état, cette «station». Ainsi, la «dureté» est une
    qualité, le «sommeil», c’est-à-dire «le dormir», est un état, une
    «station»; en effet, c’est le substantif d’un verbe neutre, tandis
    que «dureté» est le substantif d’un adjectif.\footnote{Voir
      encore, à ce sujet, le dernier chapitre.}

    {\small
      \begin{center}
        \emph{Traduction en allemand de la molécule} \textbf{(i-o)}.
      \end{center}


      En allemand, le verbe est désigné par le vocable
      «Tätigkeitswort», c’est-à-dire «mot impliquant une action».

      Cependant, la traduction ordinaire du mot «action» en allemand
      est «Handlung» et non pas «Tätigkeit», car la série «Tat»,
      «tätig», «Tätigkeit» correspond à notre série «acte»,
      «actif», «activité». Or, il est facile de voir que le mot
      «Handlung» est bien l’équivalent du mot français «action»,
      c’est-à-dire l’équivalent allemand de la molécule fondamentale
      \textbf{(i-o)}.}

  }
 % 
  {It is quite necessary to distinguish the ``quality'' from the
    ``state''. However, there is no fixed and precise boundary between
    these two notions: we can, indeed, pass from the ``quality'' to
    the ``state'' in a continuous fashion, as we pass from the notion
    of cold to that of hot. The qualities of a body are ``that which
    is proper'' to that body, for one cannot separate them from the
    body without profoundly altering its nature. The ``quality''
    (\textbf{a-o}) is adjectival, thus independent of time. On the
    other hand, the ``state'' (\textbf{i-o}) is verbal, and thus
    passing, temporal, and it does not affect the very nature of the
    body, of the entity that goes through this state, this
    ``position.'' Thus, ``hardness'' is a quality, ``sleep'', that is
    ``sleeping'' is a state, a ``position''; indeed, it is the noun
    from a neutral verb, while ``hardness'' is the noun from an
    adjective.\footnote{See also, on this subject, the final chapter.}

    {\small
      \begin{center}
        \emph{Translation in German of the molecule} \textbf{(i-o)}.
      \end{center}

      In German, the verb is designated by the word
      \emph{Tätigkeitswort}, that is, ``word implying an action.''

      However, the usual translation of the word \emph{action} in
      German is \emph{Handlung} and not \emph{Tätigkeit}, since the
      series \emph{Tat}, \emph{tätig}, \emph{Tätigkeit} corresponds to
      our series ``act'', ``active'', ``activity''. Now it is easy to
      see that the word \emph{Handlung} is the equivalent of the
      French word \emph{action}, that is, the German equivalent of the
      basic molecule (\textbf{i-o}).  } }

  \TextPage{ {\small Pour comprendre le mot «Handlung», il faut
      considérer la série «Hand», «handeln», «Handlung», qui, traduite
      \emph{littéralement} en français, donne: \emph{Hand} =
      \emph{main}, \emph{hand-eln} = \emph{mani-er}, \emph{Hand-l-ung}
      = \emph{mani-e-ment}.

      On voit que le suffixe «eln», en allemand, ou «er», en français,
      est l'atome verbal général \textbf{i}; donc «handeln», ou
      «manier», signifie littéralement «faire (une action) avec la
      main». Mais en allemand l'idée «main» est ici prise au sens
      figuré; l’organe humain de l'action symbolise l’organe de
      l’action en général; c’est pourquoi l’on peut mettre l’idée
      spéciale «Hand» entre parenthèses. Les molécules «(hand)eln» et
      «(Hand)lung» deviennent alors des molécules fondamentales, car
      elles ne contiennent plus d’idées particulières. On a en effet:\\[1ex]

      atome allemand \emph{eln} = atome français \emph{er} = atome
      verbal \textbf{i};

      atome allemand \emph{ung} = atome français \emph{ment} = atome
      substantif \textbf{o}.\\[1ex]

      Or, dans le mot «Hand-l-ung», la lettre \emph{l} est ce qui
      reste de l’atome verbal \emph{eln}; de même, dans le mot
      «mani-e-ment[»], la lettre \emph{e} est tout ce qui reste de
      l’atome verbal \emph{er}. On a donc finalement:

     \begin{center}
       \emph{(hand)-eln} = \emph{(mani)-er} = \textbf{(man)-i}\\
       \emph{(Hand)-l-ung} = \emph{(mani)-e-ment} = \textbf{(man)-i-o}
     \end{center}


     Donc, à part l'idée particulière symbolique «Hand», le mot
     «Handlung» est bien égal au mot français «action», puisque:

     \begin{center}
       \emph{l-ung} =\textbf{ i-o} = \emph{ag}-\textbf{o} =
       \emph{ac-tion}.
     \end{center}

     Les mots «Eigenschaft» et «Handlung» sont donc les mots
     fondamentaux allemands qui correspondent aux mots français (ou
     anglais) «qualité» et «action», c’est-à-dire aux molécules
     fondamentales \textbf{(a-o)} et \textbf{(i-o)}.}

 }
 % 
 { {\small To understand the word \emph{Handlung} it is necessary to
     consider the series \emph{Hand}, \emph{handeln}, \emph{Handlung} which,
     translated \emph{literally} into French, gives \emph{Hand} =
     \emph{main} `hand', \emph{hand-eln} = \emph{man-ier} `(to)
     handle', \emph{Hand-l-ung} = \emph{mani-e-ment} `manipulation,
     handling'.

     We see that the suffix \emph{eln} in German, or \emph{er} in
     French, is the general verbal atom \textbf{i}; thus,
     \emph{handeln} or \emph{manier} means literally ``carry out (an
     action) with the hand.'' But in German the idea ``hand'' is here
     taken in a figurative sense: the human organ of action symbolizes
     the organ of action in general, which is why we can put the
     specific idea ``hand'' in parentheses. The molecules
     \emph{(hand)eln} and \emph{(Hand)lung} thus become basic
     molecules, since they no longer contain specific
     ideas. We thus actually have:\\[1ex]

     German atom \emph{eln} = French atom \emph{er} = verbal atom
     \textbf{i};

     German atom \emph{ung} = French atom \emph{ment} = nominal atom
     \textbf{o}.\\[1ex]

     Now in the word \emph{Hand-l-ung}, the letter \emph{l} is what
     remains of the verbal atom \emph{eln}; similarly, in the word
     \emph{mani-e-ment} the letter \emph{e} is all that remains of the
     verbal atom \emph{er}. We thus have finally:

     \begin{center}
       \emph{(hand)-eln} = \emph{(mani)-er} = \textbf{(man)-i}\\
       \emph{(Hand)-l-ung} = \emph{(mani)-e-ment} = \textbf{(man)-i-o}
     \end{center}

     Thus, other than the specific symbolic idea \emph{Hand}, the word
     \emph{Handlung} is quite equal to the French word \emph{action}, since:

     \begin{center}
       \emph{l-ung} =\textbf{ i-o} = \emph{ag}-\textbf{o} =
       \emph{ac-tion}.
     \end{center}

     The words \emph{Eigenschaft} and \emph{Handlung} are thus basic German
     words which correspond to the French (or English) words
     \emph{qualité} `quality' and \emph{action}, that is to the basic
     molecules \textbf{(a-o)} et \textbf{(i-o)}.}
     
   
 }
 
 \TextPage{En résumé, nous avons trouvé jusqu’ici comme mots
   fondamentaux:

   \emph{a}) Les atomes \textbf{o} = «l’(être)» ou «ce (qui est)»,
   \textbf{a} = \emph{qual} et \textbf{i} = \emph{ag}.

   \emph{b}) Les molécules biatomiques \textbf{(a-o)} =
   \emph{qual-ité} et \textbf{(i-o)} = \emph{ac-tion}. Etudions
   maintenant les molécules inverses \textbf{(o-a)} et \textbf{(o-i),}
   qui sont aussi fondamentales.

   3° \textsc{Molécule} \textbf{(o-a)}. — Cette molécule représente
   \emph{l'idée substantif adjectivée} ou l’adjectif de
   «l’(être)». En donnant à «l’être» le sens d’«essence», on peut
   traduire la molécule \textbf{(o-a)} par le mot français
   \emph{essentiel}, ou encore par les mots \emph{personn-el,
     ré-el}, en prenant l’idée substantive sous la forme concrète
   (personne ou chose). Cette molécule sert donc de type à tous les
   adjectifs dérivés de substantifs (comme «hum-ain», «industri-el»,
   «périod-ique», etc.).  Elle a un rôle important; d’ailleurs, outre
   le mot français «essentiel», traduction sous forme condensée de la
   molécule \textbf{(o-a)}, il existe des expressions permettant de
   traduire cette molécule sous la forme dissociée \textbf{(a)-(o)}.

   Dans les molécules dissociées, chaque partie entre parenthèses
   représente un mot à part; on doit donc traduire ici les atomes
   \textbf{a} et \textbf{o}, non par des suffixes, mais par des
   radicaux: l’idée adjective \textbf{a} par «propre (à)» et l’idée
   substantive \textbf{o} par «l’(être)», donc \textbf{(a)-(o)}
   =«propre à l’être». Et, en effet, les mots «personn-el», «ré-el»,
   «hum-ain», «industri-%»

 }
 %
 { In summary, to this point we have found as basic words:

   \emph{a)} the atoms \textbf{o} = ``the (entity)'' or ``that (which
   is)'', \textbf{a} = \emph{qual}, and \textbf{i} = \emph{ag}.

   \emph{b)} the biatomic molecules \textbf{(a-o)} = \emph{qual-ité}
   and \textbf{(i-o)} = \emph{ac-tion}. Let us now study the inverse
   molecules \textbf{(o-a)} and \textbf{(o-i)}, which are also basic.

   3. \textsc{Molecule} \textbf{(o-a)}. --- This molecule represents
   \emph{the adjectivized nominal idea} or the adjective of ``the
   (entity)''. In giving to ``the entity'' the sense of ``the
   essence'', we can translate the molecule \textbf{(o-a)} with the
   French word \emph{essentiel}, or also by the words
   \emph{personn-el} `personal', \emph{ré-el} `real', taking the
   nominal idea in the concrete form (person or thing). The molecule
   thus serves as the type for all adjectives derived from nouns (such
   as \emph{hum-ain}, \emph{industri-el} `industrial', \emph{périod-ique}
   `periodic', etc.).  It has an important role; moreover, besides the
   French word \emph{essentiel}, translation in condensed form of the
   molecule \textbf{(o-a)}, there are also expressions allowing this
   molecule to be translated by the dissociated form \textbf{(a)-(o)}.

   In disssociated molecules, each part in parentheses represents a
   separate word; we thus translate here the atoms \textbf{a} and
   \textbf{o} not by suffixes but by roots: the adjectival idea
   \textbf{a} by ``proper (to)'' and the nominal idea \textbf{o} by
   ``the (entity)'', thus \textbf{(a)-(o)} = ``proper to the
   entity''. And indeed, the words \emph{personn-el}, \emph{ré-el},
   \emph{industri-[el]}

 }

 \TextPage{\noindent el», etc., signifient «propre à la personne»,
   «propre à la chose», «propre à l’homme», «propre à l'industrie»,
   etc.

   Naturellement, on pourrait traduire aussi l'atome adjectif
   \textbf{a} par un des suffixes dissociés équivalents «de», «qui
   (est)», c'est-à-dire que, suivant les cas, on pourra traduire la
   molécule dissociée \textbf{(a)-(o)} par «propre à l'être», ou par
   «de l’être», ou encore par «qui (est) l'être». Ainsi «hum-ain»
   peut signifier, suivant les cas, «propre à l'homme», ou «d'un
   homme», ou encore «qui est un homme»; par exemple, «un acte humain»
   (propre à l’homme), «une main humaine» (d’homme), «un être humain»
   (qui est un homme), etc.

   4° \textsc{Molécule} \textbf{(o-i)}. — Cette molécule fondamentale
   n'existe pas à l'état condensé en français.  Elle représente\emph
   {l'idée substantive verbifiée}. Cette molécule servira donc de
   type à tous les verbes dérivés de substantifs (comme «couronn-er»,
   «clou-er», «sci-er», «pein-er», «rag-er». etc.). D’ailleurs, on
   peut traduire cette molécule en français sous la forme
   dissociée. La forme condensée \textbf{(o-i)} est égale à la forme
   dissociée \textbf{(i)-(o)}; les atomes \textbf{i} et \textbf{o}
   devant être traduits par des radicaux, puisqu'ils représentent des
   mots séparés dans la molécule dissociée, on a: \textbf{i} =
   \emph{ag} ou \emph{sta}, et \textbf{o} = \emph{un} (être) . Or,
   \emph{ag} équivaut à «agir» ou «faire une action», et \emph{sta}
   équivaut à «être dans un état» (station). La molécule dissociée

 }
 %
 {\noindent \emph{[industri-]el} etc. mean ``proper to the person'',
   ``proper to the thing'', ``proper to man'', ``proper to industry'',
   etc.

   Naturally, we can also translate the adjective atom \textbf{a} by
   one of the equivalent dissociated suffixes ``of'', ``which (is)'',
   that is, depending on the case, we can translate the dissociated
   molecule \textbf{(a)-(o)} by ``proper to the entity'' or ``of the
   entity'' or again by ``which (is) the entity''.  Thus, \emph{hum-ain}
   `human' can mean, depending on the case, ``proper to man'' or ``of
   a man'' or again ``which is a man''; for example, ``a human act''
   (proper to man), ``a human hand'' (of a man), ``a human being''
   (which is a man), etc.

   4. \textsc{Molecule} \textbf{(o-i)}. --- This basic molecule does
   not exist in the condensed state in French. It represents \emph{the
     nominal idea verbalized}. This molecule will serve as the type of
   all verbs derived from nouns (like \emph{couronn-er}, \emph{clou-er} `to
   nail', \emph{sci-er} `to saw', \emph{pein-er} `to pain (someone)',
   \emph{rag-er} `to rage', etc.). Moreover, we can translate this
   molecule in French by the dissociated form. The condensed form
   \textbf{(o-i)} is equivalent to the dissociated form
   \textbf{(i)-(o)}; the atoms \textbf{i} and \textbf{o} needing to be
   translated by roots, since they represent separate words in the
   dissociated molecule, we have \textbf{i} = \emph{ag} or \emph{sta},
   and \textbf{o} = \emph{a(n)} (entity).  Now \emph{ag} is equivalent
   to ``to act'' or ``to perform an action'', and \emph{sta} is
   equivalent to ``to be in a state'' (position). The dissociated
   molecule

 }

 \TextPage{\noindent \textbf{(i)-(o)} signifie donc «faire l’action
   (caractérisée par) un être» ou bien «être dans l’état (caractérisé
   par) un être» (réel ou idéel). Ainsi «couronn-er» signifie «faire
   l’action caractérisée par l’être-réel (l’objet) \emph{couronne}»,
   «rag-er» signifie «être dans l’état caractérisé par l’être-idéel
   (le sentiment) \emph{rage}», etc.

   5° \textsc{Molécule} \textbf{(i-a)}. — L’atome \textbf{i} sous la
   forme d’un radical est traduit par «agir» ou «stare» ou, en
   supprimant la terminaison infinitive, par «ag» ou «sta»
   (quelquefois «stat»); donc on a:

     \begin{center}
       \textbf{(i-a)} = \emph{ag}-\textbf{a} = \emph{ac-tif}
     \end{center}
     ou bien:
     \begin{center}
       \textbf{(i-a)} = \emph{stat}-\textbf{a} = \emph{stat-ique}.
     \end{center}

     Les mots «actif» et «statique» sont donc aussi des mots
     fondamentaux de la langue française; ils représentent
     \emph{l'idée verbale adjectivée}. «Actif» signifie «de qualité
     \emph{ag}», et «statique», «de qualité \emph{stat}». Ces mots
     servent donc de chef de file à tous les adjectifs dérivés de
     verbes, tels que: «pallia-tif», «préserva-tif», «purga-tif»,
     «différ-ent», etc.

     On peut aussi traduire «ac-tif» par «qui ag», «qui agit»,
     «pallia-tif» par «qui pallie», etc., puisque «qui» exprime l’idée
     adjective. Mais il faut soigneusement distinguer les adjectifs
     verbaux («actif», «préservatif», etc.), qui contiennent une idée
     qualitative, et les participes («agissant», «préservant», etc.),
     qui sont des formes purement verbales n’im-

   }
 %
   {\noindent \textbf{(i)-(o)} thus means ``to perform the action
     (characterized by) an entity'', or else ``to be in the state
     (characterized by) an entity'' (real or ideal). Thus,
     \emph{couronn-er} means ``to perform the action characterized by the
     real entity (the object) \emph{crown}'', \emph{rag-er} means ``to be
     in the state characterized by the ideal entity (the feeling)
     \emph{rage}'', etc.

     5. \textsc{Molecule} \textbf{(i-a)}. --- The atom \textbf{i} in
     the form of a root is translated by \emph{agir} or \emph{stare}, or in
     suppressing the infinitive ending, by ``ag'' or ``sta''
     (sometimes ``stat''); thus we have:

     \begin{center}
       \textbf{(i-a)} = \emph{ag}-\textbf{a} = \emph{ac-tive}
     \end{center}
     or else:
     \begin{center}
       \textbf{(i-a)} = \emph{stat}-\textbf{a} = \emph{stat-ic}.
     \end{center}

     The words \emph{actif} `active' and \emph{statique} `static' are
     thus also basic words of the French language; they represent
     \emph{the verbal idea adjectivized}. \emph{Actif} means ``of
     quality \emph{ag}'' and \emph{statique} ``of quality
     \emph{stat}''. These words serve as leading examples for all
     adjectives derived from verbs, such as \emph{pallia-tif}
     `palliate-ive', \emph{préserva-tif} `preserve-ative',
     \emph{purga-tif} `purge-ative', \emph{différ-ent} `differ-ent',
     etc.

     We can thus translate \emph{ac-tif} as ``who acts'', \emph{pallia-tif}
     by ``which palliates'', etc., since ``who, which'' translates the
     adjective idea. But it is necessary to carefully distinguish the
     verbal adjectives (\emph{actif}, \emph{préservatif}, etc.) which
     contain a qualitative idea from the participles \emph{agissant}
     `acting', \emph{préservant} `preserving', etc.) which are purely
     verbal forms im[plying]

   }

   \TextPage{\noindent pliquant qu’une idée d'action ou d'état. Nous
     avons dit, en effet, qu'il ne faut pas confondre la «qualité»
     (idée adjective) avec l’état ou l’action (idée verbale). Ainsi le
     «comité actif» d’une société peut n'ètre pas du tout
     «agissant». Il faut d’autant plus éviter cette confusion qu’en
     français le participe et l’adjectif ont souvent des formes très
     voisines (ex: «différ-ant» et «différ-ent», et même des formes
     identiques qui ne se distinguent plus que par l'accord de
     l’adjectif avec le substantif (ex.: «une fille aimant son père»
     et une «nature aimante»).

     Nous reviendrons, du reste, dans le dernier chapitre, sur les
     rapports qui existent entre le participe et l’adjectif.

     6° \textsc{Molécule} \textbf{(a-i)}. — L’atome \textbf{a} est
     traduit sous la forme d’un radical par «qual» ou «propre», mais
     on ne peut pas verbifier directement ces radicaux, car les verbes
     «qual-er», «propri-er» n’existent pas. Par contre, on peut dire
     «qual-ifi-er» et «ap-propri-er». Ces mots servent de modèle à
     toute une série de verbes obtenus en remplaçant l’adjectif
     général «qual» ou «propre» par des adjectifs particuliers: ainsi
     «qual-ifi-er» sert de modèle aux mots tels que «béat-ifi-er»,
     «pur-ifi-er», etc., ou encore «modern-is-er», etc., car on voit
     immédiatement que le suffixe «is» dans «moderniser» est le même
     que le suffixe «ifi» dans «béatifier». De même «ap-propri-er»
     sert de chef de file aux mots

   }
   % 
   {\noindent
     %
     {[im]}plying only an idea of action or state. We have said,
     indeed, that it is necessary not to confuse the ``quality''
     (adjectival idea) with the state or action (verbal idea). Thus,
     the ``active committee'' of a society can be not at all
     ``acting''. It is all the more necessary to avoid this confusion
     since in French the participle and the adjective often have very
     similar forms (e.g. \emph{différ-ant} `differ-ing' and
     \emph{différ-ent} `different'[)], and even identical forms which
     are only distinguished by the agreement of the adjective with the
     noun (e.g. ``une fille \emph{aimant} son père'' `a daughter
     \emph{loving} her father' and a ``nature \emph{aimant''}
     \emph{loving} nature').

     We will return in the final chapter, in addition, to the
     relations that exist between the participle and the adjective.

     6. \textsc{Molecule} \textbf{(a-i)}. --- The atom \textbf{a} is
     translated as a root by ``qual'' or ``proper (to)'', but we
     cannot make these roots directly into verbs, because the verbs
     \emph{qual-er} `to qual', \emph{propri-er} `to proper' do not exist. On
     the other hand, we can say \emph{qual-ifi-er} `to characterize' and
     \emph{ap-propri-er} `to adapt'. These words serve as models for a
     whole series of verbs obtained by replacing the general adjective
     ``qual'' or ``proper'' by specific adjectives; thus,
     \emph{qual-ifi-er} serves as the model for words such as
     \emph{béat-ifi-er} `to beatify', \emph{pur-ifi-er} `to purify', etc.,
     or also \emph{modern-is-er} `to modernize' etc., because we see
     immediately that the suffix \emph{is} in \emph{moderniser} is the same
     as the suffix \emph{ifi} in \emph{béatifier}. Similarly,
     \emph{ap-propri-er} serves as the leading example for words
     
   }
   
   \TextPage{\noindent tels que «a-grand-ir», etc. Le mot «béatifier»
     signifie «rendre béat», le mot «purifier» signifie «rendre pur»,
     «moderniser» signifie «rendre moderne», «agrandir» signifie
     «rendre grand», etc. Les affixes «ifi», «is», «a», etc., sont
     donc des atomes verbaux qui ne contiennent rien d’autre que
     \emph{l'idée verbale active}; ils ne contiennent pas d’idée
     particulière; cependant l’idée qu’ils contiennent n’est pas non
     plus tout à fait générale, ce n’est pas simplement l’idée
     verbale, c’est l’idée verbale \emph{active} («faire» ou
     «rendre»), par opposition à l’idée verbale \emph{neutre}
     («devenir»), qui peut être exprimée soit par la molécule
     dissociée «devenir qual» (par exemple «devenir grand»), soit par
     une fausse forme réfléchie de la forme active: «se qual-ifi-er»,
     «s'ap-propri-er»; (par exemple: «s’a-grand-ir» signifie aussi
     «devenir grand», «se modern-is-er» signifie aussi «devenir
     moderne»; ainsi la phrase: «les rois \emph{se moderni-is-ent}»
     équivaut à «les rois \emph{deviennent modernes}»).

     Donc, malgré les apparences d’irrégularité et la diversité des
     formes, le passage de l’adjectif au verbe ne présente que deux
     formes (forme active et forme neutre); ces deux formes sont
     symétriques quant à leur sens, mais elles ne sont pas exprimées
     par des formes symétriques, en français tout au moins, parce
     qu’il manque un suffixe neutre correspondant aux suffixes actifs
     «ifi», «is»,

   }
 %
   {\noindent
     %
     such as \emph{a-grand-ir} `to enlarge' etc. The word
     \emph{béatifier} means ``to render beatific'', the word
     \emph{purifier} means ``to make pure'', \emph{moderniser} means
     ``to make modern'', \emph{agrandir} means ``to make large'', etc.
     The affixes \emph{ifi}, \emph{is}, \emph{a}, etc. are thus verbal
     atoms which contain nothing but \emph{the active verbal idea};
     they contain no specific idea; however, the idea they contain is
     not completely general, it is not simply the verbal idea, it is
     the \emph{active} verbal idea (``to cause'' or ``to make''), as
     opposed to the \emph{neutral} verbal idea (``to become''), which
     can be expressed either by the dissociated molecule ``to become
     qual'' (for example ``to become large''), or by a false reflexive
     form of the active form: \emph{se qual-ifi-er'}`to qualify
     oneself', \emph{s'ap-propri-er'}`to adapt oneself' (for example:
     \emph{s'a-grand-ir} means also ``to become large'', \emph{se
       modern-is-er} means also ``to become modern''; thus the
     sentence ``les rois \emph{se modern-is-ent}'' `the kings
     modernize themselves' is equivalent to ``the kings become
     modern'').

     Thus, despite the appearance of irregularity and diversity of
     form, the passage from adjective to verb involves only two forms
     (active form and neutral form); these two forms are symmetric
     with respect to their sense, but they are not expressed by
     symmetric forms, at least in French, because a neutral suffix is
     lacking corresponding to the active suffixes \emph{ifi}, \emph{is},

   }

   \TextPage{\noindent
     %
     etc. Aussi, de même que nous avons représenté les idées
     grammaticales par les symboles conventionnels \textbf{a},
     \textbf{o}, \textbf{i}, de même nous pouvons adopter des nouveaux
     symboles pour représenter séparément et symétriquement les formes
     active et neutre de l’idée verbale générale \textbf{i}; par
     exemple, l’atome \textbf{ig}\footnote{Dans l’écriture symbolique
       et phonétique, la lettre \textbf{g} a toujours le son dur.}
     représentera l’idée active «rend(re)», et l’atome \textbf{ij}
     l’idée neutre «deven(ir)», de sorte que l’on peut écrire:
     \textbf{ig-i} = «rend-re», et \textbf{ij-i} = «deven-ir». Par
     suite:

     {\noindent
       \hspace{-.2em}{\setlength{\tabcolsep}{2pt}
         \resizebox{\linewidth}{!}{\begin{tabular}[t]{rr}
           idée verbale active: \textbf{ig} = &atomes-radicaux: \emph{fai, rend}.\\ 
                                              &	» -suffixes: \emph{ifi, is}, etc.\\
                                              &	» -préfixes: \emph{a-, em-, é-}, etc.\\
           \makebox[2em][r]{»} \makebox[2em][r]{»} \hspace{1em}neutre: \textbf{ij} = &atome-radical: \emph{deven},\\
                                              & » -suffixes doubles: \emph{se-ifi}, \\
                                              &  \emph{se-is}, etc., \emph{s'a,  s’em}, etc.
         \end{tabular}}}}\\[1ex]


     On ramène ainsi toutes ces formes différentes à deux types
     symétriques qui montrent que la molécule fondamentale
     \textbf{(a-i),} que nous étudions, n’a de sens précis que si on
     la scinde en deux autres; la molécule \textbf{(a-ig-i)}, qui
     signifie «rendre qual», «rendre propre» ou «qual-ifi-er»,
     «ap-propri-er», et la molécule \textbf{(a-ij-i)}, qui signifie
     «devenir qual», «devenir propre», ou «se qualifier»,
     «s’aproprier». On aura, par exemple:
     \begin{center}
       \emph{rend-re pur} = \emph{pur-ifi-er} = \textbf{pur-ig-i}.\\
       \emph{deven-ir grand} = \emph{s'a-grand-ir} =
       \textbf{grand-ij-i}.
     \end{center}
   }
   % 
   {\noindent
     % 
     etc. Also, just as we have represented the grammatical ideas by
     the conventional symbols \textbf{a}, \textbf{o}, \textbf{i}, we
     can similarly adopt new symbols to represent separately and
     symmetrically the active and neutral forms of the general
     verbal idea \textbf{i}: for example, the atom
     \textbf{ig}\footnote{In the symbolic and phonetic transcription,
       the letter \textbf{g} always has the hard sound.} will
     represent the active idea ``to make'', and the atom \textbf{ij}
     the neutral idea ``to become'', so that we can write:
     \textbf{ig-i} = \emph{rend-re} `make-\textsc{infinitive}' and
     \textbf{ij-i} = \emph{deven-ir} `become-\textsc{infinitive}'. Thus:

     {\noindent
       \hspace{-.2em}{\setlength{\tabcolsep}{2pt}
         \resizebox{\linewidth}{!}{\begin{tabular}[t]{rl}
           active verbal idea: \textbf{ig} = &root-atoms: \emph{fai, rend}.\\ 
                                             & suffix- '' : \emph{ifi, is}, etc.\\
                                             & préfix- '' : \emph{a-, em-, é-}, etc.\\
           neutral \makebox[2em][r]{''} \makebox[2em][r]{''}:
           \textbf{ij} = &root-atom: \emph{deven},\\
                                             & double suffix '': \emph{se-ifi}, \\
                                             &
                                               \emph{se-is},
                                               etc.,
                                               \emph{s'a,
                                               s’em},
                                               etc.
         \end{tabular}}}}\\[1ex]

     We thus reduce all of these different forms to two symmetric
     types which show that the basic molecule \textbf{(a-i)}, which we
     are studying, only has a precise sense when we break it down into
     two others: the molecule \textbf{(a-ig-i)}, which means ``to make
     qual'', ``to make proper to'' or \emph{qual-ifi-er},
     \emph{ap-propri-er}, and the molecule \textbf{(a-ij-i)} which means
     ``to become qual'', ``to become proper to'', or ``qualify oneself
     (as)'', ``adapt oneself (as)''. We would thus have, for example,
     \begin{center}
       \emph{rend-re pur} = \emph{pur-ifi-er} = \textbf{pur-ig-i}.\\
       \emph{deven-ir grand} = \emph{s'a-grand-ir} =
       \textbf{grand-ij-i}.
     \end{center}



   }
   
   \TextPage{Les formes ci-dessus sont les formes régulières et
     complètes pour passer de l’adjectif au verbe. Cependant, il
     arrive quelquefois que l’on verbifïe directement l’adjectif; par
     exemple, on dit \emph{grossir}, \emph{grandir}; dans ces cas,
     c'est le contexte qui montre si la verbification a lieu dans le
     sens actif ou dans le sens neutre; ainsi, «il a grossi» signifie
     «il est devenu gros»; au contraire, «il a grossi les faits»
     signifie «il a rendu les faits (plus) gros»; mais au point de vue
     logique, ces formes sont incomplètes.

     Je termine ici l’étude des molécules fondamentales bi-atomiques,
     car nous avons examiné tous les types possibles: \textbf{(a-o),
       (i-o); (o-a), (o-i); (a-i) et (i-a).}

     Il est vrai que l’on peut encore considérer les molécules de la
     forme \textbf{(a-a)}, \textbf{(o-o)} et \textbf{(i-i)}. Ces
     molécules n'offrent pas beaucoup d'intérêt; elles représentent de
     simples pléonasmes. Or, nous savons qu’un pléonasme, introduit
     dans un mot, ne modifie pas le sens de celui-ci. Si le pléonasme
     est volontaire, il sert simplement à renforcer une idée déjà
     exprimée; s’il n’est pas volontaire, il n'apporte aucune
     modification au sens du mot. Par conséquent, les molécules telles
     que \textbf{(a-a)} sont réductibles à l’atome \textbf{a}. Les
     pléonasmes de cette sorte, très rares chez les adjectifs et les
     substantifs, se rencontrent constamment chez les verbes.

   }
   % 
   {The forms above are the regular and complete forms for passing
     from adjective to verb. However, it sometimes happens that we
     make a verb directly from the adjective: for example, we say
     \emph{grossir} `to grow, get larger', \emph{grandir} `to grow,
     increase'; in these cases, it is the context that shows whether
     the verbalizing is in the active or the neutral sense: thus, ``il
     a grossi'' means ``he has become large''; in contrast, ``il a
     grossi les faits'' means ``he has made the facts large(r)'', but
     from the logical point of view these forms are incomplete.

     I conclude the study of basic bi-atomic molecules here, since we
     have examined all of the possible types: \textbf{(a-o), (i-o);
       (o-a), (o-i); (a-i) et (i-a).}

     It is true that we can also consider molecules of the form
     \textbf{(a-a)}, \textbf{(o-o)} and \textbf{(i-i)}. These
     molecules do not offer much of interest: they represent simple
     pleonasms. Now we know that a pleonasm, introduced into a word,
     does not modify its sense. If the pleonasm is deliberate, it
     serves simply to reinforce an idea already expressed; if it is
     not deliberate, it does not make any modification of the word's
     sense. Consequently, molecules such as \textbf{(a-a)} are
     reducible to the atom \textbf{a}. Pleonasms of this type, quite
     rare in the case of adjectives and nouns, are constantly found in
     verbs.
     
   }
   
   \TextPage{1. \textsc{Molécule} \textbf{(i-i)}. — Puisque \textbf{i}
     est l’idée verbale «\emph{ag}» ou «\emph{sta}», on a:

     \begin{center}
       \textbf{(i-i)} = \emph{ag-ir} ou \emph{sta-re}.
     \end{center}


     Nous avons déjà constaté précédemment que le mot «\emph{ag-ir}»
     contient un pléonasme, puisque soit le radical «ag», soit le
     suffixe «\textbf{ir}», expriment la même idée verbale \textbf{i},
     et comme la molécule \textbf{(i-i)} est une molécule
     fondamentale, le pléonasme qui se trouve dans le mot «agir» se
     retrouvera dans tous les verbes dont le radical est verbal, comme
     «écri-re», «abdiqu-er», etc. En effet, tous ces verbes sont des
     cas particuliers du verbe «agir», c’est-à-dire que les radicaux
     «écri», «abdiqu» contiennent implicitement l’idée «ag» (ou
     «\emph{sta}»), qui est exprimée une seconde fois par la finale
     verbale «re» ou «er». Ce pléonasme est inévitable, c’est-à-dire
     qu’on ne peut pas supprimer les finales «ir» ou «er», même si le
     radical lui-même est verbal, parce que ces finales ne servent pas
     seulement à exprimer l’idée verbale générale \textbf{i}, mais
     aussi les différents temps et personnes de la conjugaison.

     2. \textsc{Molécule} \textbf{(a-a)}. — Si l’on traduit l’atome
     \textbf{a} par le radical «qual» et par le suffixe «eux», on
     obtient \textbf{(a-a)} — «qual-eux», mot qui n’existe pas. On
     trouve cependant de rares exemples d’adjectifs particuliers
     construits sur ce type, par exemple le mot allemand «süss-lich»,
     dans lequel le suffixe «lich» égale le

   }
 %
   {1. \textsc{Molecule} \textbf{(i-i)}. --- Since \textbf{i} is the
     verbal idea ``\emph{ag}'' or ``\emph{sta}'', we have:

     \begin{center}
       \textbf{(i-i)} = \emph{ag-ir} ou \emph{sta-re}.
     \end{center}

     We have already noted previously that the word \emph{ag-ir} `to
     act' contains a pleonasm, since both the root ``ag'' and the
     suffix \emph{ir} express the same verbal idea \textbf{i}, and as
     the molecule \textbf{(i-i)} is a basic molecule, the pleonasm to
     be found in the word \emph{agir} is to be found in all verbs
     whose root is verbal such as \emph{écri-re} `to write',
     \emph{abdiqu-er} `to abdicate', etc.  Actually, all of these
     verbs are special cases of the verb \emph{agir}, that is, the
     roots \emph{écri}, \emph{abdiqu} implicitly contain the idea
     ``ag'' (or ``sta''), which is expressed a second time by the
     verbal ending \emph{re} or \emph{er}. This pleonasm is
     inevitable, that is, the endings \emph{re} or \emph{er} cannot be
     suppressed, even if the root itself is verbal, because these
     endings serve to express not only the general verbal idea
     \textbf{i}, but also the different tenses and persons of the
     conjugation.

     2. \textsc{Molecule} \textbf{(a-a)}. --- If we translate the atom
     \textbf{a} by the root ``qual'' and by the suffix \emph{eux}, we get
     \textbf{(a-a)} --- \emph{qual-eux}, a word that does not exist. We
     do however find rare examples of specific adjectives constructed
     according to that type, for example the German word
     \emph{süss-lich}, in which the suffix \emph{lich} equals the
     
   }

   \TextPage{\noindent
   %
     suffixe français «eux», c’est-à-dire l’atome symbolique
     \textbf{a}, quoique le radical «süss» soit déjà lui-même
     adjectif, c’est-à-dire contienne déjà implicitement l’idée
     \textbf{a} ou «qual». Le mot «süss-lich» rentre donc dans le type
     «qual-eux». Il est évident qu’au point de vue purement logique
     «süss-lich» est un simple pléonasme réductible à «süss»; nous
     savons, en effet, que le suffixe «lich» ou «eux» est équivalent
     au suffixe dissocié «qui», «qui est», c’est-à-dire que
     «süss-lich» signifie «qui est doux», ou simplement «doux», car
     l’atome «qui (est)» ne fait que répéter l’idée adjective déjà
     contenue dans «doux». En français, nous avons aussi les deux mots
     «doux» et «doucereux», qui sont aussi équivalents au point de vue
     logique, mais pour une autre raison; la série «doux», «douc-eur»,
     «douc-er-eux», est en effet analogue à la série «beau»,
     «beau-té», «beau-ti-ful» (en anglais); on a donc \emph{doucereux}
     = \emph{doux} pour la même raison que \emph{beau-tiful} =
     \emph{beau}. La somme (\emph{ti} + \emph{ful}), en anglais, ou
     \emph{eur} + \emph{eux}, en français, est nulle, parce que cette
     somme représente l’adjectivation d’un substantif tiré lui-même
     d’un adjectif, opération double, dont la seconde est l’inverse de
     la première.

     Bien entendu, si l’on emploie, en français, les deux formes
     «doux», «doucereux», et en allemand les deux formes «süss»,
     «süsslich», c’est pour les distinguer l’une de l’autre, pour
     exprimer deux

   }
 % 
   {\noindent
     %
     French suffix \emph{eux}, that is, the symbolic atom \textbf{a},
     while the root \emph{süss} is already itself an adjective, that
     is it already implicitly contains the idea \textbf{a} or
     ``qual''. The word \emph{süss-lich} is thus of the type
     \emph{qual-eux}. It is obvious that from the purely logical point
     of view \emph{süss-lich} is a simple pleonasm reducible to
     \emph{süss}; we know, indeed, that the suffix \emph{lich} or
     \emph{eux} is equivalent to the dissociated suffix ``which'',
     ``which is'' --- that is, that \emph{süss-lich} means ``which is
     sweet'', or simply ``sweet'', since the atom ``which (is)'' does
     nothing but repeat the adjectival idea already contained in
     ``sweet''. In French, we also have the two words \emph{doux}
     `sweet' and \emph{doucereux} `smooth, unctuous', which are also
     equivalent from the logical point of view, but for another
     reason: the series \emph{doux}, \emph{douc-eur},
     \emph{douc-er-eux} is actually equivalent to the series
     \emph{beau}, \emph{beau-té}, ``beau-ti-ful'' (in English); we
     thus have \emph{doucereux} = \emph{doux} for the same reason that
     \emph{beau-tiful}= \emph{beau}.  The sum (\emph{ti} + \emph{ful})
     in English, or \emph{eur} + \emph{eux} in French, is null, since
     that sum represents the adjectivization of a noun which is itself
     derived from an adjective, a double operation in which the second
     is the inverse of the first.

     Of course, if we use the two forms \emph{doux} and \emph{doucereux} in
     French, and in German the two forms \emph{süss} and
     \emph{süsslich}, it is to distinguish the one from the other, to
     express two
   
   }

   \TextPage{\noindent
   %
     nuances différentes de la même idée. Les langues naturelles
     recourent donc a un artifice pour donner à la même idée deux sens
     légèrement differents: elles construisent deux molécules,
     équivalentes l’une à l’autre au point de vue logique, mais de
     formes différentes\footnote{C’est, par un artifice analogue que
       l'on donne, en français, deux sens différents aux expressions
       logiquement équivalentes «un homme grand» et «un grand
       homme».}.  Aussi ces deux formes ne sont pas toutes deux
     irréductibles; l’une d’elles contient des atomes superflus, et en
     les supprimant on retrouve l’autre forme, celle qui est
     irréductible.

     3. \textsc{Molécule} \textbf{(o-o)}. — Cette molécule est
     intéressante, car sa traduction française existe. En effet,
     l’idée substantive \textbf{o} peut être traduite par le radical
     latin \emph{ens} («l’être» ou «un être») et par le suffixe
     français «ité» («l’être»); on a donc:

     \begin{center}
       \textbf{(o-o)} = \emph{ens-ité} = \emph{ent-ité:}
     \end{center}
   
     Si l'on prend le radical «ens» au sens abstrait («l’être»), ce
     radical a exactement le même sens que le suffixe «ité», et l’on a
     alors:

     \begin{center}
       \emph{ité} = \emph{ent-ité} =\textbf{ (o-o)}.
     \end{center}

     C’est un simple pléonasme, mais ce pléonasme sert à donner un
     corps au suffixe «ité», qui ne peut, à lui tout seul, former un
     mot complet.

   }
 %
   {\noindent
   %
     different nuances of the same idea. Natural languages thus have
     recourse to an artifice to give two slightly different senses to
     the same idea: they construct two molecules, equivalent to one
     another from the point of view of logic, but with different
     forms\footnote{It is by an analogous artifice that in French, we
       give two different senses to the logically equivalent
       expressions \emph{un homme grand} `a tall man' and \emph{un
         grand homme} `a great man'.}. Also these two forms are not
     both irreducible: the one contains superfluous atoms, and in
     suppressing these, we recover the other form, the one that is
     irreducible.

     3. \textsc{Molecule} \textbf{(o-o)}. — This molecule is
     interesting, because its French translation exists. Indeed, the
     nominal idea \textbf{o} can be translated by the Latin root
     \emph{ens} (``the entity'' or ``an entity'') and by the suffix
     \emph{ité} (``the entity''); we thus have:

     \begin{center}
       \textbf{(o-o)} = \emph{ens-ité} = \emph{ent-ité:}
     \end{center}
   
     If we take the root \emph{ens} in the abstract sense (``the
     entity''), this root has exactly the same sense as the suffix
     \emph{ité} and we then have:

     \begin{center}
       \emph{ité} = \emph{ent-ité} =\textbf{ (o-o)}.
     \end{center}

     This is a simple pleonasm, but that pleonasm provides a body for
     the suffix \emph{ité}, which cannot alone form a complete word.

   }

   \TextPage{Si l’on prend le radical «ens» au sens concret
     (\textbf{o\sxsub{1}} = «un être»), alors la molécule prend la
     forme \textbf{(o\sxsub{1}-o)} qui n’est plus un pléonasme,
     puisque le second \textbf{o} est l’idée substantive abstraite,
     tandis que le premier est l’idée substantive concrète; on a
     alors:\\[1ex]

   \begin{center}
     {\setlength{\tabcolsep}{4pt}
       \resizebox{\linewidth}{!}{\begin{tabular}[t]{rcccl}
         ent-ité&=&\textbf{(o\sxsub{1}-o)}&=&«un — être abstrait».\\
                &=& &=&«une-ité».\\
                & & &=&«une un-ité».
       \end{tabular}}}
   \end{center}
   puisque l’article «un» peut aussi représenter l’idée substantive
   concrète.

   Ayant ainsi terminé l’examen des molécules fondamentales
   bi-atomiques, nous pouvons passer aux molécules tri-atomiques.\\[1ex]

   \begin{center}
     \emph{Molécules fondamentales tri-atomiques}\footnote{On peut,
       sans inconvénient, sauter, en première lecture, tout ce qui est
       en petits caractères}.
   \end{center}


   {\small Il existe beaucoup de mots tri-atomiques, mais ce sont,
     pour la plupart, des mots particuliers, formés, en partie au
     moins, d’atomes particuliers. Ainsi \emph{hum-an-ité} est un mot
     tri-atomique, mais il conlient l’atome «\textbf{hom}», qui
     exprime une idée particulière.

     Comparativement au nombre de combinaisons possibles, il y a très
     peu de molécules tri-atomiques fondamentales; par exemple,
     \textbf{(i-o-i)} est traduisible par «ac-tionn-er» ou
     «sta-tionn-er», \textbf{(i-a-o)} par «ac-tiv-ité», mais presque
     tous les autres types n’ont pas de correspondant en français;
     c’est qu'en effet \emph{toute molécule poly-atomique} (condensée
     ou}

 }
 %
 {If we take the root \emph{ens} in the concrete sense
   (\textbf{o\sxsub{1}} = ``an entity''), then the molecule takes the
   form \textbf{(o\sxsub{1}-o)} which is no longer a pleonasm, since
   the second \textbf{o} is the abstract nominal idea, while the first
   is the concrete nominal idea; we then have:\\[1ex]

   \begin{center}
     {\setlength{\tabcolsep}{4pt}
       \resizebox{\linewidth}{!}{\begin{tabular}[t]{rcccl}
         ent-ité&=&\textbf{(o\sxsub{1}-o)}&=&``an — abstract entity''.\\
                &=& &=&``un-it''.\\
                & & &=&``a un-it''.
       \end{tabular}}}
   \end{center}
   since the article \emph{un} `a(n)' can also represent the concrete
   nominal idea.

   Having thus concluded the examination of the basic bi-atomic
   molecules, we can pass on to the tri-atomic molecules.

   \begin{center}
     \emph{Basic tri-atomic molecules}\footnote{On a first reading, one
         can without problem skip over everything that is in small
         type.}.
   \end{center}

   {\small
     %
     There are many tri-atomic words, but these are, for the most
     part, specific words, formed at least in part of specific
     atoms. Thus \emph{hum-an-ité} is a tri-atomic word, but it
     contains the atom ``\textbf{hom}'', which expresses a specific
     idea.

     In comparison to the number of possible combinations, there are
     very few basic tri-atomic molecules: for example,
     \textbf{(i-o-i)} can be translated by \emph{ac-tionn-er} `to
     activate' or \emph{sta-tionn-er} `to station', \textbf{(i-a-o)} by
     \emph{ac-tiv-ité} `activity', but nearly all of the other types
     have no correspondent in French; actually, \emph{every
       poly-atomic molecule} (condensed or }

 }

 \TextPage{\noindent
   %
   {\small dissociée) \emph{est réductible en fin de compte aux
       molécules bi-atomiques}, de sorte que tous les types
     fondamentaux sont ou mono- ou bi-atomiques.

     Prenons, par exemple, la molécule tri-atomique «hum-an-ité»;
     cette molécule représente le substantif dérivé de l’adjectif
     «humain», exactement comme «beau-té» représente le substantif
     dérivé de l'adjectif «beau»; la seule différence est que «beau»
     est un adjectif primitif, tandis que «humain» est un adjectif
     dérivé. On a donc: «humanité» = «human-ité», et non pas
     «hum-anité» (qui signifierait «la qualité homme», puisque le
     suffixe «an» = «qual»). Le mot «human-ité» ne contient donc que
     deux éléments dissociables: «humain» et «ité». A son tour, le mot
     «humain» en contient deux: «hum» et «ain»; la molécule «humanité»
     doit donc être représentée par le schéma:

     \begin{center}
       {$\big[$(hum-an)-ité$\big]$} ou {$\big[$\textbf{(hom-a)-o}$\big]$}
     \end{center}
     schéma qui montre que ce mot ne contient pas trois atomes
     indépendants, mais un atome et une molécule bi-atomique.

     On peut aller plus loin et constater que «humanitaire» =
     «humanit-aire» et «humanitarisme[»] = «humanitar-isme», de sorte
     que le schéma moléculaire de ce dernier mot est:

     \begin{center}
       $\big(\big\{\big[$(hum-an)-it$\big]$-ar$\big\}$-isme$\big)$ ou
       $\big(\big\{\big[$\textbf{(hom-a)-o}$\big]$\textbf{-a}$\big\}$\textbf{-o}$\big)$
     \end{center}
     schéma qui montre que tout mot composé est réductible de proche
     en proche à des types bi-atomiques. Nous pourrions donc nous
     dispenser complètement de l’étude des types fondamentaux
     tri-atomiques; cependant, j’en examinerai quelques-uns qui sont
     plus particulièrement intéressants.

     1. \textsc{Molécule} \textbf{(a-o-a).} — Cette molécule est
     réductible au simple atome \textbf{(a)} ou «qual», parce que la
     dérivation \textbf{o-a} est exactement inverse de la dérivation
     \textbf{a-o} (par exemple le}

 }
 % 
 {\noindent
   % 
   {\small
     %
     dissociated) \emph{is reducible in the end to bi-atomic
       molecules,} such that all of the basic types are either mono-
     or bi-atomic.

     Take, for example, the tri-atomic molecule \emph{hum-an-ité}; this
     molecule represents the noun derived from the adjective
     \emph{humain}, exactly as \emph{beau-té} represents the noun derived
     from the adjective \emph{beau}; the only difference is that \emph{beau}
     is a basic adjective, while \emph{humain} is a derived adjective. We
     thus have \emph{humanité} = \emph{human-ité} and not
     \emph{hum-anité} (which would mean ``the quality man'', since the
     suffix \emph{an} = ``qual''). The word \emph{human-ité} thus contains
     only two dissociable elements: \emph{humain} and \emph{ité}. In its
     turn, the word \emph{humain} contains two: \emph{hum} and \emph{ain}; the
     molecule \emph{humanité} must thus be represented by the schema:

     \begin{center}
       {$\big[$(hum-an)-ité$\big]$} or {$\big[$\textbf{(hom-a)-o}$\big]$}
     \end{center}
     a schema which shows that this word does not contain three
     independent atoms, but one atom and a bi-atomic molecule.

     We can go further and note that \emph{humanitaire} =
     \emph{humanit-aire} and \emph{humanitarisme} = \emph{humanitair-isme},
     such that the molecular schema of this last word is:

     \begin{center}
       $\big(\big\{\big[$(hum-an)-it$\big]$-ar$\big\}$-isme$\big)$ or
       $\big(\big\{\big[$\textbf{(hom-a)-o}$\big]$\textbf{-a}$\big\}$\textbf{-o}$\big)$
     \end{center}
     a schema which shows that every compound word is reducible
     gradually to bi-atomic types. We could thus dispense completely
     with the study of basic tri-atomic types; however, I will examine
     some which are particularly interesting.

     1. \textsc{Molecule} \textbf{(a-o-a)}. --- This molecule is
     reducible to the simple atom \textbf{(a)} or ``qual'', because
     the derivation \textbf{o-a} is exactly the inverse of the
     derivation \textbf{a-o} (for example, the

   } }

 \TextPage{\noindent
  %
   {\small couple «joie», «joyeux» est inverse du couple «gai»,
     «gaîté»). On peut d’ailleurs s’en rendre compte par des
     opérations symboliques, en remarquant que la molécule
     \textbf{(a-o-a)} ne peut désigner que l’adjectif de la molécule
     \textbf{(a-o)}. On a donc, en dissociant:

     \begin{center}
       \textbf{(a-o-a)} = {$\big[$\textbf{(a-o)-a}$\big]$} = \textbf{(a)} —
       \textbf{(a-o)} = «de-qualité»
     \end{center}
     et nous savons que la molécule «de qualité» se réduit à l’atome
     «qual» («hom-qual» = «de qualité homme»[)]. Comme exemple
     particulier de molécule \textbf{(a-o-a)} réductible à
     \textbf{(a),} nous avons cité le mot anglais «beau-ti-ful[»] ,
     qui est en effet équivalent au mot français mono-atomique «beau».

     Mais pour qu’une molécule tri-atomique du type\textbf{ (x-y-x)}
     soit réductible à l’atome \textbf{x}, il faut que le premier
     atome \textbf{x} soit exactement le même que le dernier. Or, pour
     différentes raisons, cela n’a pas toujours lieu. Il peut arriver,
     par exemple, que dans la molécule \textbf{(a-o-a)}, l’idée
     adjective exprimée par l’atome final \textbf{a} ne soit pas la
     même que l’idée adjective exprimée par l’atome \textbf{a}
     initial, car il y a dans l’idée adjective plusieurs nuances:
     l’idée «qual» (ou «de qualité») et l’idée «propre (à)» (qui
     signifie plutôt «relatif à», «appartenant à»).

     Ainsi, dans la série «sain», «san-té», «san-it-aire», le mot
     «sanitaire» rentre dans le type \textbf{(a-o-a)}, mais la
     molécule «sanitaire» n’est pas réductible à l’atome «sain», parce
     que l’idée adjective contenue dans «sain» est purement
     qualitative: «un homme sain» signifie «un homme de qualité
     \emph{santé}» ou «san-té-qual», car «san-té-qual» se réduit à
     «sain», puisque cette molécule est du type \textbf{(a-o-a),} où
     le dernier \textbf{a} a la même valeur que le premier. Au
     contraire, «san-it-aire» signifie «propre (à la) san-té»
     [\textbf{(a) — (a-o)}]; ainsi «un appareil sanitaire» est un
     appareil «propre à la santé», «relatif à la santé», et non pas
     «un appareil de qualité santé», car ce n’est pas l’appareil
     lui-même qui est «sain»;}

 }
%
 {\noindent
   %
   {\small
     %
     pair \emph{joie} `joy', \emph{joyeux} `joyous' is the inverse of the
     pair \emph{gai} `gay', \emph{gaîté} `gaiety'). We could, however, take
     account of this by symbolic operations, noting that the molecule
     \textbf{(a-o-a)} can only designate the adjective in the molecule
     \textbf{(a-o)}. We thus have, by dissociating:

     \begin{center}
       \textbf{(a-o-a)} = {$\big[$\textbf{(a-o)-a}$\big]$} = \textbf{(a)} —
       \textbf{(a-o)} = ``of-quality''
     \end{center}
     and we know that the molecule ``of-quality'' reduces to the atom
     ``qual'' (``hom-qual'' = ``of quality man'').  As a specific
     example of the molecule \textbf{(a-o-a)} reducible to
     \textbf{(a)}, we have cited the English word ``beau-ti-ful'',
     which is actually equivalent to the mono-atomic French word
     \emph{beau}.

     But in order for a tri-atomic molecule of the type
     \textbf{(x-y-x)} to be reducible to the atom \textbf{x}, it is
     necessary that the first atom \textbf{x} should be exactly the
     same as the last. Now for various reasons, that is not always the
     case. It can happen, for example, that in the molecule
     \textbf{(a-o-a)} the adjectival idea expressed by the final
     \textbf{a} atom is not the same as the adjectival idea expressed
     by the initial \textbf{a} atom, because there are a number of
     nuances in the adjectival idea: the idea ``qual'' (or ``of
     quality'') and the idea ``proper (to)'' (which means rather
     ``relative to'', ``belonging to'').

     Thus, in the series \emph{sain} `healthy', \emph{san-té} `health',
     \emph{san-it-aire} `sanitary', the word \emph{sanitaire} belongs to the
     type \textbf{(a-o-a)}, but the molecule \emph{sanitaire} is not
     reducible to the atom \emph{sain} because the adjectival idea
     contained in \emph{sain} is purely qualitative: \emph{un homme sain} `a
     healthy man' means ``a man of quality \emph{health}'' or
     \emph{san-té-qual}, because \emph{san-té-qual} is reducible to
     \emph{sain} since that molecule is of the type \textbf{(a-o-a)}
     where the last \textbf{a} has the same value as the first. On the
     other hand, \emph{san-it-aire} means ``proper (to) health''
     [\textbf{(a) — (a-o)}]; thus \emph{un appareil sanitaire} `a
     sanitary appliance' is an appliance ``proper to health'',
     ``relative to health'', and not ``an appliance of quality
     health'', because it is not the appliance itself that is
     ``healthy'';

   }

 }

 \TextPage{\noindent
  %
   {\small le mot «san-it-aire» est donc du type \textbf{(a-o-a)}, le
     second \textbf{a} signifiant «propre à», tandis que le premier
     (qui est contenu dans le radical «san») signifie «qual»; nous
     sommes dans le cas où la molécule est irréductible.

     2. \textsc{Molécule} \textbf{(i-o-i)}. — Nous avons aussi déjà
     rencontré cette molécule fondamentale, dont la traduction en
     français (sous forme synthétique) est «ac-tion-ner». Cette
     molécule se réduit à l’atome «ac» ou «ag», parce que le dernier
     atome \textbf{i} a la même valeur que le premier; en effet, les
     finales verbales «er», «ir», etc., sont équivalentes à l’atome
     «ag» et signifient «faire une action», ou simplement «faire», car
     «faire» est synonyme de «agir». On peut démontrer, du reste,
     toutes ces équivalences par la méthode symbolique, en se
     rappelant que la condensation ou la dissociation d’une molécule
     produit le renversement de l’ordre de ses atomes:

     \textbf{i} = «ag» = finales verbales «er», «ir», «re».

     \textbf{(i-i)} = «ag-ir» = «fai-re».

     \textbf{(i-o)} = «ac-tion».

     \textbf{(i-i)} — \textbf{(i-o)} = «(fai-re) — (ac-tion)».

     Comme \textbf{(i-i)} est un simple pléonasme qui se réduit à
     \textbf{i}, on a: «(fai-re) — (act-ion)» = \textbf{(i)} —
     \textbf{(i-o)} = \textbf{(i-o-i)} = «(ac-tionn-er)».

     Or, la molécule \textbf{(i-o-i)} se réduit à l’atome \textbf{i},
     on a donc bien: \textbf{i} = «faire une action». Cet exemple
     suffit pour montrer comment on peut opérer sur les symboles
     \textbf{a}, \textbf{o}, \textbf{i}, et retrouver toujours les
     mots ou les expressions logiquement équivalentes, malgré la
     diversité des formes apparentes. Ainsi l’atome verbal \textbf{i},
     ayant aussi le sens neutre ou statique «sta», est aussi
     équivalent à l’expression «être dans une station (un état)».

     3. \textsc{Molécule} \textbf{(o-a-o)}. — Cette molécule sera
     réductible à l’atome \textbf{o} ou, au contraire, irréductible,
     suivant que le dernier \textbf{o} aura ou non la même valeur que
     le premier. Donc, si chacun des deux \textbf{o} représentait
     l’idée substantive dans toute}

 }
%
 {\noindent
   %
   {\small
     %
     the word \emph{san-it-aire} is thus of type \textbf{(a-o-a)}, the
     second \textbf{a} signifying ``proper to'' while the first (which
     is contained in the root \emph{san}) means ``qual''; we thus have a
     case where the molecule is irreducible.

     2. \textsc{Molecule} \textbf{(i-o-i)}. --- We have also already
     encountered this basic molecule, whose translation in French (in
     synthetic form) is \emph{ac-tion-ner} `to activate'. This
     molecule reduces to the atom ``ac'' or ``ag'', because the final
     atom \textbf{i} has the same value as the first; indeed, the
     verbal endings \emph{er}, \emph{ir}, etc. are equivalent to the
     atom ``ag'' and mean ``to carry out an action'' or simply ``to
     do'', since ``to do'' is synonymous with \emph{agir} `to act'. We
     can, besides, demonstrate all of these equivalences by the
     symbolic method, recalling that the condensation or dissociation
     of a molecule results in the reversal of the order of its atoms:

     \textbf{i} = ``ag'' = verbal endings \emph{er}, \emph{ir}, \emph{re}

     \textbf{(i-i)} = \emph{ag-ir} = \emph{fai-re}

     \textbf{(i-o)} = \emph{ac-tion}

     \textbf{(i-i)} --- \textbf{(i-o)} = ``(fai-re) --- (ac-tion)''

     Since \textbf{(i-i)} is a simple pleonasm which reduces to the
     atom \textbf{i}, we have: ``(fai-re) --- (ac-tion)'' =
     \textbf{(i)} — \textbf{(i-o)} = \textbf{(i-o-i)} =
     «(ac-tionn-er)».

     Now the molecule \textbf{(i-o-i)} reduces to the atom \textbf{i},
     and we have: \textbf{i} = ``carry out an action''. This example
     suffices to show how we can operate on the symbols \textbf{a, o,
       i} and always recover the logically equivalent words or
     expressions, in spite of the diversity of their apparent
     forms. Thus, the verbal atom \textbf{i}, having also the neutral
     or static sense ``sta'', is also equivalent to the expression
     ``to be in a condition (a state)''.

     3. \textsc{Molecule} \textbf{(o-a-o)}. --- This molecule will be
     reducible to the atom \textbf{o}, or alternatively irreducible,
     depending on whether the final \textbf{o} has or has not the same
     value as the first. Thus if each of the two \textbf{o}s
     represents the nominal idea in all

   }

 }

 \TextPage{\noindent
  %
   {\small sa généralité, la molécule serait réductible. Mais, en
     réalité, dans la molécule \textbf{(o-a-o)}, le second \textbf{o}
     remplace un suffixe substantificateur (comme «ité»), c’est-à-dire
     l’idée substantive générale de «l’être», «l’être abstrait»
     \textbf{(o)}. Donc, si le premier \textbf{o} représente aussi un
     être abstrait, la molécule \textbf{(o-a-o)} sera réductible à
     l’atome \textbf{o}; au contraire, si le premier \textbf{o}
     représente un être concret \textbf{(o\sxsub{1})}, la molécule
     sera irréductible, parce qu’elle sera du type
     \textbf{(o\sxsub{1}-a-o)}, dans lequel le premier \textbf{o} n’a
     pas la même valeur que le dernier.

     II n’existe pas, en français, de mot unique pour traduire, sous
     forme synthétique, la molécule \textbf{(o\sxsub{1}-a-o)}, parce
     qu'il n’existe pas d’autre atome que l’article «un» pour désigner
     «un être concret», et que cet article ne peut pas entrer dans la
     composition des mots, au moins dans ce sens\footnote{En effet, le
       mot français «un-ic-ité» existe bien et pourrait servir à
       traduire la molécule en question, puisqu’on a:
       \textbf{o\sxsub{1}} = «un», \textbf{(o\sxsub{1}-a) }= «unique»,
       donc \textbf{(o\sxsub{1}-a-o)} = «un-ic-ité»; mais le mot «un»
       n’a pas ici le sens de «un être»; il représente l'idée «un» par
       opposition à «plusieurs».}. Mais si l’on remarque qu’un être
     concret est une «personne» ou une «chose», on peut dire que la
     molécule \textbf{(o\sxsub{1}-a-o)} est équivalente à l’un des
     deux mots «personn-al-ité» ou «ré-al-ité», (puisque l’atome
     \textbf{a} = «qual» = suffixes «el», «eux», «ique», etc., et que
     l’atome «chose» prend la forme latine «res» quand il entre dans
     la composition des mots).

     On a donc:
     \begin{center}
       {\setlength{\arraycolsep}{0pt} \textbf{o\sxsub{1}} = «un
         (être concret)» =$\left\{
           \begin{array}{l}
             \mbox{«personne»,}\\
             \mbox{«chose» (res),}
           \end{array}
         \right.$
    
         \textbf{(o\sxsub{1}-a)} = $\left\{
           \begin{array}{l}
             \mbox{«personn-el»,}\\
             \mbox{«ré-el».}
           \end{array}
         \right.$
    
         \textbf{o\sxsub{1}-a-o} = $\left\{
           \begin{array}{l}
             \mbox{«personn-al-ité,»,}\\
             \mbox{«ré-al-ité».}
           \end{array}
         \right.$ }
     \end{center}

     La molécule fondamentale \textbf{(o\sxsub{1}-a-o)} sert donc de
     chef de file à tous les mots, tels que «hum-an-ité»,
     [«]nébul-os-ité»,}

 }
% 
 {\noindent
   %
   {\small
     %
     its generality, the molecule will be reducible. But in reality,
     in the molecule \textbf{(o-a-o)}, the second \textbf{o} replaces
     a nominalizing suffix (like \emph{ité}), that is, the general
     nominal idea of ``the entity'', ``the abstract entity''
     \textbf{(o)}. Thus, if the first \textbf{o} also represents an
     abstract entity, the molecule \textbf{(o-a-o)} will be reducible
     to the atom \textbf{o}; on the other hand, if the first
     \textbf{o} represents a concrete entity \textbf{(o\sxsub{1})},
     the molecule will be irreducible, since it will be of the type
     \textbf{(o\sxsub{1}-a-o)}, in which the first \textbf{o} does not
     have the same value as the last.

     In French, there does not exist a unique word to translate in
     synthetic form the molecule \textbf{(o\sxsub{1}-a-o)}, because
     there exists no other atom than the article \emph{un} to designate
     ``a concrete entity'', and this article cannot enter into the
     composition of words, at least in this sense\footnote{Actually,
       the French word \emph{un-ic-ité} `uniqueness' does exist and
       could serve to translate the molecule in question, since we
       have: \textbf{o\sxsub{1}} = \emph{un}, \textbf{(o\sxsub{1}-a)} =
       \emph{unique}, thus \textbf{(o\sxsub{1}-a-o)} = \emph{un-ic-ité};
       but the word \emph{un} here does not have the sense of ``an
       entity''; it represents the idea ``one'' as opposed to
       ``several''.}. But if we note that a concrete entity is a
     ``person'' or a ``thing'', we can say that the molecule
     \textbf{(o\sxsub{1}-a-o)} is equivalent to one of the words
     \emph{personn-al-ité} or \emph{ré-al-ité} (since the atom
     \textbf{a} = ``qual'' = suffixes \emph{el}, \emph{eux}, \emph{ique}, etc.,
     and the atom \emph{chose} `thing' takes the Latin form \emph{res} when
     it enters into the composition of words).

     We thus have:
     \begin{center}
       {\setlength{\arraycolsep}{0pt} \textbf{o\sxsub{1}} = ``a
         (concrete entity)'' =$\left\{
           \begin{array}{l}
             \mbox{``person'',}\\
             \mbox{``thing'' (res),}
           \end{array}
         \right.$
    
         \textbf{(o\sxsub{1}-a)} = $\left\{
           \begin{array}{l}
             \mbox{``person-al'',}\\
             \mbox{``re-al''.}
           \end{array}
         \right.$
    
         \textbf{o\sxsub{1}-a-o} = $\left\{
           \begin{array}{l}
             \mbox{``person-al-ity,»,}\\
             \mbox{``re-al-ity''.}
           \end{array}
         \right.$ }
     \end{center}

     The basic molecule \textbf{(o\sxsub{1}-a-o)} thus serves as the
     leading form for all of the words like \emph{hum-an-ité}
     `humanity', \emph{nébul-os-ité} `cloudiness',

   }

 }
 
 \LargerTextPage{\noindent
  %
   {\small «électr-ic-ité», «caus-al-ité», etc., mots qui signifient
     tous «qual-ité d’une (personne ou chose)», «propri-été d’une
     (personne ou chose)», puisque tous les atomes adjectifs «an»
     («ain»). «os» («eux»), «ic» («ique»), «el» («al»), etc., sont,
     synonymes de l'atome «qual». Ainsi «hum-an-ité» = «hom-qual-ité»,
     «nébul-os-ité» = «nuage-qual-ité» (l'atome «nuage» prenant la
     forme «nébul» en composition), «électr-ic-ité[»] =
     «ambre-propri-été» (l’atome «ambre» prenant la forme grecque
     spéciale «électr» en composition, pour indiquer qu'il ne s’agit
     pas d’une propriété quelconque, mais de la propriété spéciale
     qu’a l’ambre d’attirer les corps légers, lorsqu’on le frotte),
     etc. Dans tous ces mots, le premier atome représente un être
     concret, comme «nuage», «ambre», «homme», «personne»,
     «chose». En effet, si dans la molécule \textbf{(o\sxsub{1}-a-o)}
     on prenait le substantif concret \textbf{(o\sxsub{1})} au sens
     abstrait \textbf{(o)}, la molécule deviendrait réductible au
     simple atome \textbf{o}, parce que l’opération \textbf{a-o}serait
     alors inverse de \textbf{o-a}. C'est pourquoi le mot «humanité»
     (ainsi que tous les autres mots analogues), peut prendre deux
     sens, suivant que I'on donne à l'atome «hom» le sens concret ou
     le sens abstrait. Si l'on dit «l’humanité de Jésus-Christ», cela
     signifie «la qualité humaine, la nature d’homme (concret) de
     Jésus-Christ»; de même «traiter quelqu’un avec humanité» signifie
     «traiter d’une manière humaine, propre à un homme (concret)», et
     dans ces deux cas, le mot «humanité» est tout différent de
     «homme»; l’un est une molécule du type \textbf{(o\sxsub{1}-a-o)},
     l’autre un atome du type \textbf{o\sxsub{1}}. Si, au contraire,
     on donne à l’atome «homme» le sens abstrait «homme en général»,
     alors le mot «humanité», qui en dérive, change de sens et ne
     signifie rien de plus que l’«homme eu général»; c’est qu’en
     effet, en donnant au mot «homme» le sens d’un être abstrait, le
     mot «humanité» n’est plus une molécule du type
     \textbf{(o\sxsub{1}-a-o)} mais du type \textbf{(o-a-o)}, lequel
     type se réduit à l atome \textbf{o}.}}
%
 {\noindent
  %
   {\small
     %
     \emph{électr-ic-ité}, \emph{caus-al-ité}, etc., words that all
     mean ``qual-ity of a (person or thing)'', ``proper-ty of a
     (person or thing)'', since all of the adjectival atoms \emph{an}
     (\emph{ain}), \emph{os} (\emph{eux}), \emph{ic} (\emph{ique}),
     \emph{el} (\emph{al}), etc. are synonyms of the atom
     ``qual''. Thus, \emph{hum-an-ité} = ``man-qual-ity'',
     \emph{nebul-os-ité} = ``cloud-qual-ity'' (the atom \emph{nuage}
     `cloud' taking the form \emph{nebul} in composition),
     \emph{électr-ic-ité} = ``amber-proper-ty'' (the atom \emph{ambre}
     `amber' taking the special Greek form \emph{électr} in
     composition, to indicate that it is not a question of just any
     property, but the special property that amber has of attracting
     light bodies when one rubs it), etc. In all of these words, the
     first atom represents a concrete entity, like ``cloud'',
     ``amber'', ``man'', ``person'', ``thing''. Indeed, if we take the
     concrete noun \textbf{(o\sxsub{1})} in the molecule
     \textbf{(o\sxsub{1}-a-o)} in the abstract sense \textbf{(o)}, the
     molecule would become reducible to the simple atom \textbf{o},
     because the operation \textbf{a-o} would then be the inverse of
     \textbf{o-a}. This is why the word \emph{humanité} (as well as
     all the other analogous words) can take on two senses, depending
     on whether we give the atom \emph{hom} the concrete sense or the
     abstract sense. If we say ``the humanity of Jesus Christ'', that
     means ``the human quality, the nature of man (concrete) of Jesus
     Christ''; similarly, ``to treat someone with humanity'' means
     ``to treat (them) in a human way, in a way proper to a man
     (concrete)'', and in both cases the word ``humanity'' is quite
     different from ``man''; the one is a molecule of the type
     \textbf{(o\sxsub{1}-a-o)}, the other an atom of the type
     \textbf{o\sxsub{1}}. If on the other hand we give the atom
     ``man'' the abstract sense ``man in general'', then the word
     ``humanity'' which derives from it changes its sense and means
     nothing more than ``man in general''; that is, in giving to the
     word ``man'' the sense of an abstract entity, the word
     ``humanity'' is no longer a molecule of the type
     \textbf{(o\sxsub{1}-a-o)} but rather of type \textbf{(o-a-o)},
     which type reduces to the atom \textbf{o}.
   }
 }

 \TextPage{ {\small On pourrait aussi dire que ce second sens du mot
     «humanité» provient de ce qu'on étend la notion de «qual-ité» (ou
     «an-ité») à l'«abstrait en général», par le fait que la qualité
     est une notion abstraite. Cela revient à prendre la partie pour
     le tout, car tous les abstraits ne sont pas des
     «qualités». L'être abstrait général est représenté par le suffixe
     «ité» et non par la molécule «qual-ité»; il suffirait donc de
     dire «hom-ité» (au lieu de «hum-an-ité[»]) pour désigner l'homme
     en général, l'homme au sens abstrait. C'est ce qu’on fait en
     allemand, où l’on a les deux mots: «Mensch-lich-keit» =
     «hum-an-ité» = «hom-qual-ité» \textbf{(o\sxsub{1}-a-o)} et
     «Mensch-heit» = «hom-ité», type \textbf{(o\sxsub{1}-o)} qui
     signifie «l'être abstrait homme», «l'homme en général\footnote{Le
       mot «Mensch-heit» ou «hom-ité» rentre dans le type «ent-ité»
       \textbf{(o\sxsub{1}-o)} analysé plus haut.}».

     Le second sens du mot «humanité» (Menschheit) n’a donc que
     l’apparence d'une molécule; en réalité, c'est un simple atome qui
     signifie «l’homme (en général)» et, par suite, «l'ensemble des
     hommes». Or, «l’ensemble des hommes» peut être considéré à son
     tour comme une nouvelle entité concrète \textbf{(o\sxsub{1})}
     donnant naissance à une nouvelle série de mots:

    \begin{center}
      {\footnotesize«humanité», «humanit-aire», «humanit-ar-isme»,}
    \end{center}
    série qui correspond exactement à la famille initiale:

    \begin{center}
      «homme», «hum-ain», «hum-an-ité»,
    \end{center}
    ou symboliquement:

    \begin{center}
      \textbf{hom, hom-a, hom-a-o,}
    \end{center}
    avec cette seule différence que dans cette dernière série le mot
    «hom» a le sens de «un homme», tandis que dans la première le mot
    «humanit» doit être traité comme un simple atome signifiant
    «l’ensemble des hommes», «les}%»

}
% 
{{\small
    %
    We could also say that this second sense of the word ``humanity''
    comes from extending the notion of ``qual-ity'' (or ``an-ity'') to
    the ``abstract in general'', from the fact that quality is an
    abstract notion. This comes down to taking the part for the whole,
    because not all abstracts are ``qualities''. The general abstract
    entity is represented by the suffix \emph{ité} `ity' and not by the
    molecule ``qual-ity''; it would then suffice to say \emph{hom-ité}
    (instead of \emph{hum-an-ité}) to designate man in general, man in
    the abstract sense. This is what we do in German, where we have
    the two words \emph{Mensch-lich-keit} = ``hum-an-ity'' =
    ``hom-qual-ity'' \textbf{(o\sxsub{1}-a-o)} and \emph{Mensch-heit} =
    ``hom-ity'', type \textbf{(o\sxsub{1}-o)}, which means ``the
    abstract entity man'', ``man in general\footnote{The word
      \emph{Mensch-heit} or ``hom-ity'' belongs to the type ``ent-ity''
      \textbf{(o\sxsub{1}-o)} analyzed above.}''.

    The second sense of the word ``humanity'' (Menschheit) only
    appears to be a molecule; in reality, it is a simple atom that
    means ``man (in general)'', and following that, ``the set of
    men''. Now ``the set of men'' can be considered in turn as a new
    concrete entity \textbf{(o\sxsub{1})} giving rise to a new series
    of words:

    \begin{center}
      {\footnotesize «humanité», «humanit-aire», «humanit-ar-isme»,}
    \end{center}
    a series that corresponds exactly to the initial series:

    \begin{center}
      «homme», «hum-ain», «hum-an-ité»,
    \end{center}
    or symbolically:

    \begin{center}
      \textbf{hom, hom-a, hom-a-o,}
    \end{center}
    with the sole difference that in this last series the word \emph{hom}
    `man' has the sense of ``a man'', while in the first the word
    \emph{humanit} must be treated as a simple atom signifying ``the set
    of men'',

  }
  
}

\TextPage{\noindent
  % 
  {\small hommes», cet ensemble étant considéré comme une nouvelle
    entité concrète, donnant par conséquent naissance à un nouveau
    substantif abstrait (humanit-ar-isme) du type
    \textbf{(o\sxsub{1}-a-o)}.

    Les mots français les plus généraux rentrant dans le type
    \textbf{(o\sxsub{1}-a-o)} sont les mots «personn-al-ité» et
    «ré-al-ité», puisque les suffixes «al» et «ité» sont des atomes
    généraux et que les radicaux «personne» et «chose» (\emph{res})
    sont, sinon des atomes généraux, du moins de simples subdivisions
    de l’idée substantive concrète \textbf{o\sxsub{1}}.  On a
    «personn-al-ité» = «personn-propri-été» = «ce qui est — propre à —
    une personne», lorsqu'on prend «personne» au sens concret; mais si
    l'on prend cet atome au sens abstrait, alors il n’y a plus de
    différence entre la molécule «personnalité» et l’atome «la
    personne en général». De même «réalité» signifie «la chose en
    général», parce que sous la forme latine \emph{res}, le mot
    «chose» a plutôt le sens abstrait. De même, «causalité» n’est que
    «la cause en général», si l’on donne à l’atome «cause» le sens
    abstrait.

    4. \textsc{Molécule} \textbf{(o-i-o)}. — Comme la molécule
    précédente, la molécule \textbf{(o-i-o)} sera réductible ou non,
    suivant que les deux atomes \textbf{o} auront ou non la même
    valeur. Comme nous savons déjà que \textbf{(i-o)} = «ac-tion» ou
    «sta-tion» (état), l’atome final \textbf{o} a toujours le sens
    abstrait \textbf{o} exprimé par les suffixes «tion», «ment»,
    «ture», etc., qui suivent toujours un atome verbal. La molécule
    sera donc irréductible si l'\textbf{o} initial est concret, car
    elle prend alors la forme \textbf{(o\sxsub{1}-i-o)}; elle sera
    réductible si l'\textbf{o} initial est abstrait, car alors
    \textbf{(o-i-o)} = \textbf{o}.

    On ne peut pas traduire en français la molécule
    \textbf{(o\sxsub{1}-i-o)} par une molécule condensée, mais comme
    on a, d’après la loi de renversement: \textbf{(o\sxsub{1}-i-o)} =
    \textbf{(i-o)} — \textbf{(o\sxsub{1})}, cette molécule signifie:
    «action (caractérisée par) un être concret» \textbf{(o\sxsub{1})},
    tel que, par exemple, «couronne», «main», etc. La molécule
    \textbf{(o\sxsub{1}-i-o)} sert donc de chef de file aux mots tels
    que:}

}
%
{\noindent
  % 
  {\small
    %
    ``men'', this set being considered as a new concrete entity,
    therefore giving rise to a new abstract noun (\emph{humanit-ar-isme}) of
    the type \textbf{(o\sxsub{1}-a-o)}.

    The most general French words belonging to the type
    \textbf{(o\sxsub{1}-a-o)} are the words \emph{personn-al-ité} and
    \emph{ré-al-ité}, since the suffixes \emph{al} and \emph{ité} are
    general atoms and the roots \emph{personne} and \emph{chose}
    (\emph{res}) are, if not general atoms, at least simple
    subdivisions of the concrete nominal idea \textbf{o\sxsub{1}}. We
    have \emph{personn-al-ité} = ``person-proper-ty'' = ``that which is
    --- proper to --- a person'', when we take ``person'' in the
    concrete sense; but if we take that atom in the abstract sense,
    then there is no longer a difference between the molecule
    ``personality'' and the atom ``the person in general''.
    Similarly, \emph{réalité} means ``things in general'', because in
    its Latin form \emph{res}, the word ``thing'' has rather the
    abstract sense. Similarly, \emph{causalité} is only ``cause in
    general'' if one gives the atom ``cause'' the abstract sense.

    4. \textsc{Molecule} \textbf{(o-i-o)}. — Like the preceding
    molecule, the molecule \textbf{(o-i-o)} is reducible or not
    depending on whether the two \textbf{o} atoms have the same value
    or not. As we already know that \textbf{(i-o)} = \emph{ac-tion} or
    \emph{sta-tion} (state), the final atom \textbf{o} always has the
    abstract sense \textbf{o} expressed by the suffixes \emph{tion},
    \emph{ment}, \emph{ture}, etc., which always follow a verbal atom. The
    molecule will thus be irreducible if the initial \textbf{o} is
    concrete, since it then takes the form \textbf{(o\sxsub{1}-i-o)};
    it will be reducible if the initial \textbf{o} is abstract, since
    then \textbf{(o-i-o)} = \textbf{o}.

    We cannot translate the molecule \textbf{(o\sxsub{1}-i-o)} in
    French by a condensed molecule, but since we have, by the law of
    reversal, \textbf{(o\sxsub{1}-i-o)} = \textbf{(i-o)} ---
    \textbf{(o\sxsub{1})}, this molecule means ``action (characterized
    by) a concrete entity'' \textbf{(o\sxsub{1})}, such as, for
    example, ``crown'', ``hand'' etc. The molecule
    \textbf{(o\sxsub{1}-i-o)} thus serves as leading example for words
    such as:
    
  }

}

\TextPage{\noindent
  % 
  {\small «couronn-e-ment», «mani-e-ment», etc., puisque le suffixe
    «ment» est égal à «tion» et que l’atome «e» est ce qui reste de la
    finale verbale «er» du verbe «courron-er», laquelle finale
    contient l’idée générale \textbf{i} ou «ac».

    Considérons encore la série «règle», «régl-er», «règl-e-ment»: si
    l’on donne au substantif «règle» le sens concret de «une règle»,
    le mot «règlement» signifie «l'action (faite d’après) une règle»,
    (l’action de régler), comme, par exemple, dans la phrase: «le
    règlement de cette question s’impose». Au contraire, si l’on donne
    au substantif «règle» le sens abstrait de «la règle en général»,
    la différence entre «règlement» et «règle» disparait, parce que la
    molécule rentre alors dans le type \textbf{(o-i-o)} réductible à
    \textbf{o}; ainsi, le «règlement d’une société» signifie «la règle
    en général», c’est-à-dire «l’ensemble des règles» de cette
    société, tout comme «humanité» signifie «l'homme en général»,
    «l’ensemble des hommes», lorsqu’on donne à l’atome «hum» le sens
    abstrait \textbf{o}. Lorsqu'on prend le mot «règlement» dans ce
    dernier sens, on peut le considérer comme n’étant plus, à
    proprement parler, une molécule tri-atomique, mais un atome
    signifiant «la règle» au sens abstrait, «l’ensemble des
    règles». Or, «l’ensemble des règles» (comme «l’ensemble des
    hommes») peut être considéré comme une nouvelle entité concrète
    pouvant donner à son tour naissance à une nouvelle famille de
    mots: :

    \begin{center}
      \makebox[4.4em][l]{\emph{règlement,}}
      \makebox[5.5em][l]{\emph{réglement-er},}
      \makebox[6em][l]{\emph{réglement-at-ion}}
    \end{center}
    tout à fait semblable à la famille initiale:

    \begin{center}
      \makebox[4.4em][l]{\emph{règle,}}
      \makebox[5.5em][l]{\emph{régl-er},}
      \makebox[6em][l]{\emph{règl-e-ment}}
    \end{center}
    ou symboliquement:

    \begin{center}
      \makebox[4.4em][l]{\textbf{regul},}
      \makebox[5.5em][l]{\textbf{regul-i},}
      \makebox[6em][l]{\textbf{regul-i-o}.}
    \end{center}

    La seule différence entre ces deux familles est que l’atome
    «règle», qui sert de point de départ à la famille}
  
}
% 
{\noindent
  % 
  {\small
    %
    \emph{couronn-e-ment} `coron-at-ion', \emph{mani-e-ment}
    `man(ipul)-at-ion', etc., since the suffix \emph{ment} is equal to
    \emph{tion} and the atom \emph{e} is what remains of the verbal ending
    \emph{er} of the verb \emph{couronn-er}, which ending contains the
    general idea \textbf{i} or ``ac''.

    Let us consider again the series \emph{règle} `rule', \emph{régl-er}
    `regulate', \emph{règl-e-ment} `regulation': if we give to the noun
    \emph{règle} the concrete sense of ``a rule'', the word
    \emph{règlement} means ``the action (done according to) a rule'',
    (the action of regulating), as for example in the sentence \emph{le
    règlement de cette question s'impose} `the regulation of this
    matter is obvious'. On the other hand, if we give to the noun
    \emph{règle} the abstract sense of ``rule in general'', the
    difference between \emph{règlement} and \emph{règle} disappears,
    because the molecule then belongs to the type \textbf{(o-i-o)}
    reducible to \textbf{o}; thus, the \emph{règlement d'une
    société} `regulation of a society' means ``rule in general'',
    that is ``the set of rules'' of that society, just as ``humanity''
    means ``man in general'', ``the set of men'' when one give the
    atom ``hum'' the abstract sense \textbf{o}. Once we take the word
    \emph{règlement} in this latter sense, we can consider it as not
    being, properly speaking, a tri-atomic molecule, but an atom
    meaning ``rule'' in the abstract sense, ``the set of rules''. Now
    ``the set of rules'', like ``the set of men'', can be considered a
    new concrete entity able to give rise in its turn to a new family
    of words: :

    \begin{center}
      \makebox[4.5em][l]{\emph{règlement,}}
      \makebox[6.0em][l]{\emph{réglement-er},}
      \makebox[6.5em][l]{\emph{réglement-at-ion}}\\
      `regulation', `to control', `rules, regulations'
    \end{center}
    entirely comparable to the initial family:

    \begin{center}
      \makebox[4.4em][l]{\emph{règle,}}
      \makebox[5.5em][l]{\emph{régl-er},}
      \makebox[6em][l]{\emph{règl-e-ment}}
    \end{center}
    or symbolically:

    \begin{center}
      \makebox[4.4em][l]{\textbf{regul},}
      \makebox[5.5em][l]{\textbf{regul-i},}
      \makebox[6em][l]{\textbf{regul-i-o}.}
    \end{center}

    The only difference between these two families is that the atom
    \emph{règle} serves as the point of departure for the [first]
    family }
    
  
}

\TextPage{\noindent
  % 
  {\small initiale, a le sens «une règle» (entité concrète), tandis
    que l’atome «règlement», qui sert de point de départ à l'autre
    famille, signifie «un ensemble de règles», par exemple «l’ensemble
    des règles» de telle ou telle société. Cet ensemble, considéré
    comme concret, peut alors engendrer un nouveau substantif abstrait
    (réglementation), qui aura la même relation avec le mot
    «règlement» (concret) que le mot «règlement» (abstrait) a avec le
    mot «règle» (concret).

    En effet, de même que le sens \mbox{tri-atomique\footnotemark}\footnotetext{On peut
      distinguer les deux sens des mots «réglement» et «humanité», en
      appelant l'un le sens tri-atomique (règl-e-ment) et l’autre le
      sens mono-atomique (règlement = régle, au sens général).} du
    mot «règl-e-ment» est «l'action (faite d après) une règle», le
    sens du mot «réglement-at-ion» est «l'action (faite d'après) un
    règlement».

    5. \textsc{Molécule} \textbf{(a-i-i).} — En étudiant les deux sens
    de la molécule bi-atomique \textbf{(a-i),} nous avons représenté
    le sens actif par la molécule \textbf{(a-ig-i)} et le sens neutre
    par \textbf{(a-ij-i)}. Les atomes \textbf{ig} et \textbf{ij} sont
    des atomes verbaux qui ne sont pas strictement fondamentaux: ils
    ne représentent pas l’idée verbale \textbf{i} dans toute sa
    généralité; ce sont des \emph{subdivisions} de l’idée verbale
    \textbf{i} en «actif» et «neutre», comme \textbf{o} et
    \textbf{o\sxsub{1}} sont des subdivisions de l’idée substantive en
    «abstrait» et «concret»; on pourrait donc représenter les idées
    «actif» et «neutre» par les symboles \textbf{i} et
    \textbf{i\sxsub{1}}, au lieu de \textbf{ig} et \textbf{ij}, et
    considérer les molécules \textbf{(a-i-i)},
    \textbf{(a-i\sxsub{1}-i)} comme des molécules fondamentales où
    l’on a seulement donné au premier \textbf{i} un sens spécialement
    «actif» ou spécialement «neutre»\footnote{J’ai employé les
      notations \textbf{o} et \textbf{o\sxsub{1}} pour désigner les
      formes abstraite et concrète de l'idée substantive, parce que
      l'idée substantive a naturellement la forme abstraite. De même,
      on peut employer la notation \textbf{i} et \textbf{i\sxsub{1}}
      pour désigner les formes active et neutre de l'idée verbale,
      parce que l'idée verbale a naturellement la forme active.}.

    On peut traduire la molécule \textbf{(a-i-i)} de plusieurs
    manières en français: 1° par le mot «qual-ifi-er», puisque
    \textbf{a} = «qual»}


}
% 
{\noindent
  % 
  {\small
    %
    first [family], in the sens of ``a rule'' (concrete entity) while
    the atom \emph{règlement}, which serves as the point of departure
    for the other family means ``a set of rules'', for example ``the
    set of rules'' of such and such a society. This set, considered as
    concrete, can then engender a new abstract noun (règlementation)
    which bears the same relation to the word \emph{règlement}
    (concrete) as the word \emph{règlement} (abstract) bears to the
    word \emph{règle} (concrete).

    Indeed, just as the tri-atomic sense\footnote{We can distinguish
      the two senses of the words \emph{règlement} and \emph{humanité}
      by calling one the tri-atomic sense (règl-e-ment) and the other
      the mono-atomic sense (règlement = rules, in the general
      sense).}  of the word \emph{règl-e-ment} is ``action (done
    according to) a rule'', the sense of the word
    \emph{règlement-at-ion} is ``action (done according to) the
    rules''.

    5. \textsc{Molecule} \textbf{(a-i-i).} — In studying the two
    senses of the biatomic molecule \textbf{(a-i)}, we have
    represented the active sense by the molecule \textbf{(a-ig-i)} and
    the neutral sense by \textbf{(a-ij-i)}. the atoms \textbf{ig} and
    \textbf{ij} are verbal atoms which are not strictly basic: they do
    not represent the verbal idea \textbf{i} in its full generality:
    they are \emph{subdivisions} of the verbal idea \textbf{i} into
    ``active'' and ``neutral'', as \textbf{o} and \textbf{o\sxsub{1}}
    are subdivisions of the nominal idea into ``abstract'' and
    ``concrete''. We can thus represent the ``active'' and ``neutral''
    ideas by the symbols \textbf{i} and \textbf{i\sxsub{1}} instead of
    \textbf{ig} and \textbf{ij}, and consider the molecules
    \textbf{(a-i-i)}, \textbf{(a-i\sxsub{1}-i)} as basic molecules
    where we have
    simply given the first \textbf{i} a specially
    ``active'' or ``neutral'' \mbox{sense\footnotemark.}\footnotetext{I have employed the
      notations \textbf{o} and \textbf{o\sxsub{1}} to designate the
      abstract and concrete forms of the nominal idea, because the
      nominal idea naturally has the abstract form. Similarly, we can
      employ the notion \textbf{i} and \textbf{i\sxsub{1}} to
      designate the verbal idea, because the
      verbal idea naturally has the active form.}

    We can translate the molecule \textbf{(a-i-i)} in several ways in
    French: 1. by the word \emph{qual-ifi-er}, since \textbf{a} =
    ``qual''

  }
  
}

\TextPage{\noindent
  % 
  {\small et \textbf{i} = \textbf{ig}= «ifi»; alors la molécule
    \textbf{(a-i-i)} sert de chef de file à tous les mots, tels que:
    «béat-ifi-er», «pur-ifi-er», «laïc-is-er», etc.; 2° par un trio de
    suffixes fondamentaux, comme, par exemple: «an-is-er», dans le mot
    «(hum)-an-is-er», «iqu-is-er», dans «(électr)-iqu\footnote{La
      forme régulière du mot «électriser» est «électr-iqu-is-er», car
      ce mot signifie «rend-re électr-ique» et l’on sait que «rend-re»
      = \textbf{ig-i} = «is-er». On devrait de même dire
      «class-é-ifi-er» (rend-re class-é) au lieu de
      «classifier»}-is-er», «é-ifi-er», dans «(class)-é-ifi-er». Tous
    ces assemblages de suffixes sont équivalents et interchangeables,
    car on a:

    \begin{center}
      \begin{tabular}[t]{l}
        \emph{qual} = \emph{ain} = \emph{ique} = \emph{eux} = etc. \\
        \emph{rend}(re) = \emph{ifi} = \emph{is} = etc.
      \end{tabular}
    \end{center}
    de sorte que, par exemple: (hum-an-is-er) = (hum-an-ifi-er) =
    (rend-re) — (hum-ain) — (rend-re) = (hom-qual), etc.

    6. \textsc{Molécule} \textbf{(a-i-o)}. — En remplaçant le dernier
    atome verbal \textbf{i} par l’atome substantif \textbf{o}, on
    obtient la molécule fondamentale:

    \textbf{(a-i-o) }= «qual-ifica-tion» = «an-isa-tion», molécule qui
    sert de chef de file à la série des mots tels que:
    «béat-ifica-tion», «pur-ifica-tion», «laïc-isa-tion»,
    «(hum)-an-isa-tion», etc.

    7. \textbf{Molécule} \textbf{(i-a-o)}. — Comme \textbf{i} =
    radicaux verbaux «ag» ou «ac», que \textbf{a} = suffixes adjectifs
    «ain», «if», etc., et que \textbf{o} = suffixes substantifs «ité»,
    etc., ôn a:

    \textbf{i} = «ac», \textbf{(i-a)} = «act-if», \textbf{(i-a-o)} =
    «act-iv-ité», la molécule \textbf{(i-a-o)} sert donc de chef de
    file aux mots tels que: «divis-ibil-ité», «collect-iv-ité», etc.,
    etc., puisque «divis», «collect», etc., sont des cas particuliers
    de l’idée verbale «ac» ou \textbf{i}, et que le suffixe «ibl» est
    un cas particulier de l’idée adjective «if» ou \textbf{a}.}

  Je termine ici l’étude des molécules fondamentales, car il n’existe
  guère d’autres types intéres- }
%
{\noindent
  % 
  {\small
    %
    and \textbf{i} = \textbf{ig} = \emph{ifi}; so the molecule
    \textbf{(a-i-i)} serves as the leading example for all words such
    as \emph{béat-ifi-er} `to beatify', \emph{pur-ifi-er} `to purify',
    laïc-is-er'' `to secularize', etc. 2. by a trio of basic
    suffixes, such as for example: \emph{an-is-er} in the word
    \emph{(hum)-an-is-er'}`to humanize', \emph{ique-is-er} in
    \emph{(électr)-iqu\footnote{The regular form of the word
      \emph{électriciser} is \emph{électr-iqu-is-er}, because this word
      means ``to make electric'' and we know that \emph{rend-re} `to
      make' = \textbf{ig-i} = \emph{is-er}. We ought similarly to say
      \emph{class-é-ifi-er} (\emph{rendr-re class-é}' `to make classed')
      instead of \emph{classifier} `to classify'.}-is-er} `to
    electrify', \emph{é-ifi-er} in \emph{(class)-é-ifi-er} `to
    classify'. All these assemblages of suffixes are equivalent and
    interchangeable, since we have:

    \begin{center}
      \begin{tabular}[t]{l}
        \emph{qual} = \emph{ain} = \emph{ique} = \emph{eux} = etc. \\
        \emph{rend}(re) = \emph{ifi} = \emph{is} = etc.
      \end{tabular}
    \end{center}
    such that, for example, (\emph{hum-an-is-er}) =
    (\emph{hum-an-ifi-er}) = (\emph{rend-re}) — (\emph{hum-ain}) =
    (\emph{rend-re}) — (\emph{hom}-qual), etc.

    6. \textsc{Molecule} \textbf{(a-i-o)}. — By replacing the final
    verbal atom \textbf{i} by the nominal atom \textbf{o}, we obtain
    the basic molecule:

    \textbf{(a-i-o) }= \emph{qual-ifica-tion} = \emph{an-isa-tion}, the
    molecule that serves as the leading form for the series of words
    such as: \emph{béat-ifica-tion}, \emph{pur-ifica-tion},
    \emph{laïc-isa-tion}, \emph{(hum)-an-isa-tion}, etc.

    7. \textbf{Molecule} \textbf{(i-a-o)}. — As \textbf{i} = verbal
    root ``ag'' or ``ac'', \textbf{a} = adjectival suffixes \emph{ain},
    \emph{if}, etc. and \textbf{o} = nominal suffixes \emph{ité}, etc., we
    have:

    \textbf{i} = ``ac'', \textbf{(i-a)} = \emph{act-if},
    \textbf{(i-a-o)} = \emph{act-iv-ité}.  The molecule
    \textbf{(i-a-o)} thus serves as the leading form for words such as
    \emph{divis-ibil-ité}, \emph{collect-iv-ité}, etc. etc., since
    \emph{divis}, \emph{collect}, etc. are specific cases of the
    verbal idea ``ac'' or \textbf{i}, and the suffix \emph{ibl} is a
    specific case of the adjectival idea \emph{if} or \textbf{a}.  }

  I conclude here the study of the basic molecules, because there
  exist hardly any other interes[ting] types
  
}

\TextPage{\noindent
  % 
  sants parmi les types tri-atomiques ou poly-atomiques. et les
  exemples précédents suffisent, d'autant plus qu’on peut toujours
  réduire un type poly-atomique quelconque à des types
  bi-atomiques. Nous pouvons donc passer à l’analyse des molécules
  \emph{non} fondamentales, c'est-à-dire des molécules ne contenant
  que des atomes particuliers, ou un mélange d'atomes fondamentaux et
  d'atomes particuliers. Mais auparavant je ferai une petite
  digression sur la symétrie des trois idées grammaticales.\\[1ex]

  \begin{center}
    \rule{2cm}{0.4pt}
  \end{center}
  \vspace*{1ex}

  \begin{center}
    \textsc{Digression sur la Symétrie du Verbe\\[.5ex]
      et de l’Adjectif\\[.5ex]
      par rapport au Substantif.}
  \end{center}


  \emph{Le substantif}, comme son nom l’indique, est la «substance»,
  le «corps» du langage: «homme», «table», etc., désignent «des êtres»
  (\emph{ens}); mais il ne faut pas oublier que le mot «substance» se
  rapporte au Cosmos, tandis que «substantif» se rapporte au langage;
  de sorte que le substantif comprend non seulement les mots qui
  désignent des êtres du Cosmos (comme «homme», «table», etc.), mais
  aussi des «êtres de raison» (\emph{ens rationis}), c’est-
  
}
% 
{\noindent
  % 
  {[interes]}ting [types] among the tri-atomic or poly-atomic types,
  and the preceding examples are sufficient --- all the more so since
  we can always reduce any poly-atomic type at all to bi-atomic
  types. We can now pass on to the analysis of \emph{non} basic
  molecules, that is, molecules containing only specific atoms or a
  mixture of basic atoms and specific atoms. But first I will make a
  small digression concerning the symmetry of the three grammatical
  ideas.\\[1ex]

  \begin{center}
    \rule{2cm}{0.4pt}
  \end{center}
  \vspace*{1ex}

  \begin{center}
    \textsc{Digression on the Symmetry of the Verb\\
      and the Adjective\\
      in relation to the Noun.}
  \end{center}

  The \emph{noun} [French \emph{substantif}], as its name indicates,
  is the ``substance'', the ``body'' of language: ``man'', ``table'',
  etc. designate ``entities'' (\emph{ens}); but it must not be
  forgotten that the word ``substance'' relates to the cosmos, while
  ``noun'' is related to language, so that the noun includes not only
  the words that designate beings in the cosmos (like ``man'',
  ``table'', etc.) but also ``beings of reason'' (\emph{ens
    rationis}), that is
  
}

\TextPage{\noindent
  % 
  à-dire des «entités» créées par l’homme en vue du langage, comme
  «science», «théorie», etc.

  D’autre part, le substantif tout seul ne peut pas plus fonctionner
  qu’un corps sans membres. Pour construire des phrases, et même pour
  construire des mots composés, des «molécules» représentant des idées
  complexes, le verbe et l’adjectif sont nécessaires. Je n’envisagerai
  la question qu'au point de vue de la formation des mots, celle des
  phrases étant en dehors de mon programme.

  De l’étude sommaire que nous avons faite des différents types de
  molécule, il ressort, non pas seulement que le verbe et l’adjectif
  sont les membres qui permettent au substantif de fonctionner, mais
  encore que ces deux membres ont des rôles \emph{symétriques} par
  rapport au substantif, de sorte que Verbe, Substantif et Adjectif
  forment une triade, un organisme dont le Substantif est le corps et
  dont le Verbe et l’Adjectif sont les deux ailes, les deux membres
  symétriques. Nous avons déjà vu, par exemple, que les trois atomes
  fondamentaux \textbf{a}, \textbf{o}, \textbf{i} donnent naissance
  aux deux molécules fondamentales \textbf{(a-o)} et \textbf{(i-o)},
  dans lesquelles l’adjectif \textbf{a} et le verbe \textbf{i} jouent
  des rôles symétriques par rapport au substantif \textbf{o} (tandis
  que \textbf{a} et \textbf{o} n’ont pas des rôles symétriques par
  rapport à \textbf{i}, ni \textbf{i} et \textbf{o} par rapport à
  \textbf{a}). Ces molécules \textbf{(a-o)} et \textbf{(i-o)}
  représentent d’ailleurs deux mots fondamentaux de la langue:
  «qual-ité»
  
}
% 
{\noindent
  % 
  {[that is]} ``entities'' created by man in language, like
  ``science'', ``theory'', etc.

  On the other hand, the noun cannot function alone any more than a
  body without limbs. To construct sentences, and even to construct
  compound words, the ``molecules'' representing complex ideas, the
  verb and the adjective, are necessary. I will consider the question
  only from the point of view of the formation of words, that of
  sentences being beyond the scope of my program.

  From the summary study which we have made of the different types of
  molecule, it emerges not only that the verb and the adjective are
  the limbs that allow the noun to function, but also that these two
  limbs have \emph{symmetric} roles in relation to the noun, such that
  Verb, Noun and Adjective form a triad, an organism of which the Noun
  is the body and the Verb and the Adjective are the two wings, the
  two symmetric limbs. We have already seen, for example, that the
  three basic atoms \textbf{a, o, i} give rise to two basic molecules
  \textbf{(a-o)} and \textbf{(i-o)}, in which the adjective \textbf{a}
  and the verb \textbf{i} play symmetrical roles in relation to the
  noun \textbf{o} (while \textbf{a} and \textbf{o} do not have
  symmetric roles in relation to \textbf{i}, nor \textbf{i} and
  \textbf{o} in relation to \textbf{a}). These molecules
  \textbf{(a-o)} and \textbf{(i-o)} represent moreover two fundamental
  words of the language: \emph{qual-ité}
  
}

\TextPage{\noindent
  %
  et «ac-tion», c’est-à-dire que la substantification d’un adjectif
  est le pendant de la substantification d’un verbe.

  On peut vérifier ce fait de plusieurs manières différentes:

  1° Les mots «grand-eur», «rich-esse», «bon-té», etc., font pendant
  aux mots: «abonne-ment», «abdica-tion», «écri-ture», etc.

  2° Les expressions «le beau», «le laid», «le propre», etc., font
  pendant aux expressions «le boire», «le manger», «le rire»,
  etc. (par exemple, suivant Rabelais «\emph{le rire} est \emph{le
    propre} de l’homme»).

  3° Nous avons vu que si un mot contient un atome adjectif, l’idée
  exprimée par ce mot contient une idée «qualificative», et si un mot
  contient un atome verbal, l’idée exprimée par ce mot contient une
  idée d’«agir» (ou de «stare»); aussi lorsqu’un substantif (comme
  «homme» ou symboliquement \textbf{hom}) ne contient pas d’idée
  qualificative et veut s’en assimiler une, il va la chercher chez
  l’adjectif («hum-ain» ou \textbf{hom-a}) et la ramène dans un
  nouveau substantif («hum-an-ité» ou \textbf{hom-a-o}); de même,
  lorsqu’un substantif (comme «règle» ou \textbf{regul}) ne contient
  pas d’idée d’«agir» et veut s’en assimiler une, il va la chercher
  chez le verbe («régl-er» ou \textbf{regul-i}) et la ramène dans un
  nouveau substantif («règl-e-ment» ou \textbf{regul-i-o}); il y a
  symétrie parfaite eutre les deux séries:

}
%
{\noindent
  %
  and \emph{ac-tion}. That is, the nominalization of an adjective is
  the counterpart of the nominalization of a verb.

  We can verify this fact in several different ways:

  1. The words \emph{grand-eur}, \emph{rich-esse}, \emph{bon-té}, etc. are
  the counterparts of the words \emph{abonne-ment} `subscription',
  \emph{abdica-tion}, \emph{écri-ture} `writing', etc.

  2. The expressions \emph{le beau} `the beautiful', \emph{le laid} `the ugly',
  \emph{le propre} `what is specific' , etc. are counterparts of the
  expressions \emph{le boire} `the drink', \emph{le manger} `the food', \emph{le
  rire} `laughter' etc. (for example, according to Rabelais
  ``\emph{le rire} est \emph{le propre} de l'homme'' `laughter is what
  is specific to man').

  3. We have seen that if a word contains an adjectival atom, the idea
  expressed by this word contains a ``qualifying'' idea, and if a word
  contains a verbal atom, the idea expressed by this word contains an
  idea of ``acting'' (or ``being in a state''); also, when a noun
  (like \emph{homme} `man' or symbolically \textbf{hom}) does not
  contain a qualifying idea and wishes to acquire one, it looks for
  this from an ajective (\emph{hum-ain} or \textbf{hom-a}) and brings
  this to a new noun (\emph{hum-an-ité} or \textbf{hom-a-o});
  similarly, when a noun (like \emph{règle} or \textbf{regul}) does
  not contain an idea of ``to act'' and wishes to incorporate one, it
  looks for this from a verb (\emph{régl-er} or \textbf{regul-i}) and
  brings this to a new noun (\emph{règl-e-ment} or
  \textbf{regul-i-o}). There is perfect symmetry between the two
  series:
 
}

\TextPage{\noindent
  %
  \begin{center}
    «homme», «humain», «humanité», «humanitaire», «humanitarisme»;
  \end{center}
  \begin{center}
    «règle», «régler», «règlement», «réglementer», «réglementation».
  \end{center}


  4° En fait de substantifs, tout adjectif primitif (comme «grand»)
  n’en engendre qu’un directement («grand-eur»), substantif abstrait,
  tandis qu’à tout adjectif dérivé (comme «hum-ain») correspondent
  deux substantifs («homme» et «human-ité»), obtenus l’un en
  supprimant, l'autre en conservant l’atome adjectif («ain»), et comme
  cet atome est qualificatif, abstractif, ces substantifs seront l’un
  concret, l’autre abstrait; de même, tout verbe primitif (comme
  «écri») n’engendre directement qu’un substantif («écri-ture»)
  d’espèce abstraite, tandis qu’à tout verbe dérivé (comme
  «couronn-er») correspondent deux substantifs («couronne» et
  «couronnement»), obtenus l’un en supprimant, l’autre en conservant
  l’atome verbal («er» ou «e»); ces substantifs seront aussi l’un
  concret, l’autre abstrait, car toutè «action[»] est une idée
  abstraite de la réalité «agir».

  On peut résumer les rapports entre le Verbe, le Substantif et
  l’Adjectif par le tableau suivant:

  \begin{center}
    \textsc{substantif} (\textbf{o} = \emph{ens})\\
    \textsc{adjectif} (\textbf{a} = \emph{qual})\hspace{2em}
    \textsc{verbe} (\textbf{i} = \emph{ag})
  \end{center}


}
%
{\noindent
  %
  \begin{center}
    \emph{homme}, \emph{humain}, \emph{humanité}, \emph{humanitaire}, \emph{humanitarisme};
  \end{center}
  \begin{center}
    \emph{règle}, \emph{régler}, \emph{règlement}, \emph{réglementer}, \emph{réglementation}.
  \end{center}
  
  4. In terms of nouns, every basic adjective (like \emph{grand}) only
  yields one directly (\emph{grand-eur}), an abstract noun, while to
  every derived adjective (like \emph{hum-ain}) there correspond two
  nouns (\emph{homme} and \emph{human-ité}) obtained in the one case
  by suppressing, and in the other by retaining the adjectival atom
  (\emph{ain}); and as this atom is qualifying, abstracting, one of
  these nouns is concrete and the other abstract. Similarly, every
  basic verb (such as \emph{écri}) only yields directly one noun
  (\emph{écri-ture}) of an abstract sort, while to every derived verb
  (like \emph{couronn-er}) there correspond two nouns (\emph{couronne}
  and \emph{couronnement}), obtained in the one case by suppressing
  and in the other by retaining the verbal atom (\emph{er} or
  \emph{e}). These nouns also are the one concrete, the other
  abstract, because every ``action'' is an idea abstracted from
  reality (``to act'').

  We can sum up the relations between the Verb, the Noun and the
  Adjective by the following table:

  \begin{center}
    \textsc{noun} (\textbf{o} = \emph{ens})\\
    \textsc{adjective} (\textbf{a} = \emph{qual})\hspace{2em}
    \textsc{verb} (\textbf{i} = \emph{ag})
  \end{center}

}

\TextPage{\noindent
  %
  qui montre que nous avons affaire à une triade symétrique dont le
  Substantif est le centre.

  \begin{center}
    B). \textsc{Molécules quelconques}.
  \end{center}


  Jusqu’ici nous n’avons étudié que les molécules fondamentales, car
  les molécules particulières mentionnées dans les paragraphes
  précédents ne l’ont été qu’à titre d’exemples.

  Il faut donc maintenant examiner de plus près les molécules
  particulières, c’est-à-dire celles qui contiennent des atomes
  particuliers, afin de pouvoir analyser un mot quelconque.

  Du reste, il n’y a pas de limite précise entre les idées générales
  et les idées particulières. C’est en subdivisant les premières qu’on
  obtient les secondes, mais il est difficile de dire à quel moment on
  passe des unes aux autres. Ainsi, en subdivisant l’idée substantive
  en abstrait \textbf{(o)} et en concret \textbf{(o\sxsub{1})}, on
  obtient des idées que nous avons considérées comme étant encore
  générales; de même, en subdivisant les êtres concrets en «choses» et
  «personnes», on obtient des idées plus particulières ou, si l’on
  veut, moins générales, et ainsi de suite.

  L’analyse des molécules particulières n’offre aucune difficulté,
  maintenant que nous connaissons les atomes et les molécules
  fondamentales, et que

}
%
{\noindent
  %
  which shows that we have to do with a symmetrical triad of which the
  noun is the center.

  \begin{center}
    B). \textsc{Molecules in General}.
  \end{center}

  Up to this point we have only studied basic molecules, since the
  specific molecules mentioned in preceding paragraphs have only been
  cited as examples.

  It is now necessary to examine more closely the specific molecules,
  that is, those which contain specific atoms, in order to be in a
  position to analyse any word whatsoever.

  In fact, there is no precise limit between general ideas and
  specific ideas. It is in subdividing the first that we obtain the
  second, but it is difficult to say at what point we pass from the
  one to the other. Thus, in subdividing the nominal idea into abstract
  \textbf{(o)} and concrete \textbf{(o\sxsub{1})}, we obtain ideas
  that we have still considered general; similarly, in subdividing
  concrete entities into ``things'' and ``persons'', we obtain ideas
  that are more specific, or if you wish, less general, and so on.

  The analysis of specific molecules presents no difficulty, now that
  we are familiar with atoms and the basic molecules, and that

}

\TextPage{\noindent
  %
  nous savons que chaque atome particulier contient, outre l’idée
  particulière qu’il exprime, une ou plusieurs idées générales qui
  sont implicitement contenues en lui. Ceci revient à considérer tout
  atome particulier comme un cas spécial d’un atome général qui lui
  sert de chef de file; par exemple l’atome verbal particulier
  «abdiqu» est un cas spécial de de l'atome verbal général «ag»;
  l’atome adjectif particulier «bon» est un cas spécial de l’atome
  adjectif général «qual».

  Donc, \emph{pour analyser un mot composé quelconque, il faut d'abord
    remplacer tous les atomes particuliers contenus dans ce mot par
    les atomes généraux correspondants; on obtient ainsi le mot
    composé fondamental qui sert de chef de file au mot particulier
    que l'on étudie.}

  Prenons, par exemple, les deux mots particuliers «abdication» et
  «bonté»; ces mots sont tous deux bi-atomiques («abdica-tion»,
  «bon-té») et ils contiennent tous deux un atome particulier (abdica,
  bon) et un atome général ou grammatical (tion, té). L’atome
  particulier «abdica» étant verbal a pour chef de file l’atome
  général «ag» ou «ac», et l’atome «bon» étant adjectif est un cas
  spécial de l’atome général «qual». Donc le mot «abdication» n’est
  qu’un cas spécial du mot fondamental «ac-tion» et le mot «bon-té»
  n’est qu’un cas spécial du mot fondamental «qual-ité».

}
%
{\noindent
  %
  we know that each specific atom contains, besides the specific idea
  that it expresses, one or more general ideas that are implicitly
  contained within it. This comes down to considering every specific
  atom as a special case of a general atom, which serves as its
  leading form.  For example, the specific verbal atom \emph{abdique} `abdic(ate)' is
  a special case of the general verbal atom ``ag''; the specific
  adjectival atom \emph{bon} `good' is a special case of the general
  adjectival atom ``qual''.

  Thus, \emph{to analyze an arbitrary compound word, it is necessary
    first to replace all of the specific atoms contained in the word
    by the corresponding general atoms; we obtain in that way the
    basic compound word which serves as the leading form for the
    specific word we are studying.}

  Let us take, for example, the two specific words \emph{abdication} and
  \emph{bonté} `goodness'. These words are both bi-atomic
  (\emph{abdica-tion} and \emph{bon-té}), and they each contain a specific
  atom (abdica, bon) and a general or grammatical atom (tion,
  té). The specific atom \emph{abdica} being verbal, it has as its
  leading form the general atom ``ag'' or ``ac'', and the atom \emph{bon}
  being an adjective, it is a special case of the general atom
  ``qual''.  Thus, the word \emph{abdica-tion} is only a special case of
  the basic word \emph{ac-tion}, and the word \emph{bon-té} is only a
  special case of the basic word \emph{qual-ité}.

}

\TextPage{Pour bien montrer que les idées particulières «abdica»,
  «bon», contiennent implicitement en elles-mêmes les idées générales
  «ac», «qual», on peut écrire ces dernières sous les premières et
  entre parenthèses:

  \begin{center}
    \begin{tabular}[t]{ll}
      \emph{abdica-tion}	&\emph{bon-té}\\
      \emph{(ac)}	&\emph{(qual)}
    \end{tabular}
  \end{center}



  Pour avoir maintenant le sens exact d’un mot particulier, il suffît
  de présenter ce mot comme un cas spécial du mot fondamental
  correspondant. On écrira donc:\\[1ex]

  \noindent
  {\setlength{\tabcolsep}{2pt}
    \resizebox{\linewidth}{!}{\begin{tabular}[t]{lr}
      \emph{abdica-tion}&= \emph{ac-tion} (espèce particulière\\
      \hspace{1em}(\emph{ac})                  &«\emph{abdica}»)\\
      
    \end{tabular}}}\\[1ex]


  \noindent
  et:\\[1ex]

 
  {\centering\emph{bon-té} = \emph{qual-ité} (espèce particulière
    \emph{bon})\par}
  \noindent(\emph{qual})\\[1ex]

  Cette méthode d’analyse est complète, car elle ne laisse plus aucun
  élément caché et elle permet de ramener l’analyse d’un mot
  quelconque à celle d’un petit nombre de mots fondamentaux. En effet,
  nous avons déjà étudié les mots généraux «action» et «qualité»; nous
  savons que dans ces mots la soudure est une simple juxtaposition
  (précisément parce que les atomes généraux ne contiennent rien de
  sous-entendu); donc, pour analyser le sens des mots «ac-tion» et
  «qual-ité», il suffît de dissocier

}
%
{To demonstrate that the specific ideas \emph{abdica} and \emph{bon}
  implicitly contain in themselves the general ideas ``ac'', ``qual'',
  we can write these last under the first and in parentheses:

  \begin{center}
    \begin{tabular}[t]{ll}
      \emph{abdica-tion}	&\emph{bon-té}\\
      \emph{(ac)}	&\emph{(qual)}
    \end{tabular}
  \end{center}

  Now to obtain the exact sense of a specific word, it suffices to
  present that word as a special case of the corresponding basic
  word. We thus write:\\[1ex]

  \noindent
  {\setlength{\tabcolsep}{2pt}
\begin{tabular}[t]{lr}
      \emph{abdica-tion}&= \emph{ac-tion} (specific type\\
      \hspace{1em}(\emph{ac}) &\emph{abdica})\\
    \end{tabular}}\\[1ex]

  \noindent
  and:\\[1ex]
 
  {\centering\setlength{\tabcolsep}{2pt}
    \begin{tabular}[t]{lr}
      \emph{bon-té}&= \emph{qual-ité} (specific type \emph{bon})\\
      (\emph{qual})
    \end{tabular}}
    \\[2ex]

  This method of analysis is complete, since it no longer leaves any
  element hidden and allows us to unite the analysis of an arbitrary
  word with that of a small number of basic words. Indeed, we have
  already studied the general words \emph{action} and \emph{qualité}: we
  know that in these words the juncture is a simple juxtaposition
  (precisely because the general atoms contain nothing understood).
  Thus, to analyze the sense of the words \emph{ac-tion} and
  \emph{qual-ité} it suffices to dissociate

}

\TextPage{\noindent
  %
  ces molécules en appliquant la loi du renversement des atomes, et
  nous avons trouvé ainsi:

  \begin{center}
    \begin{tabular}[t]{l}
      (ac-tion) = (ce qui est) — (ag)\\
      (qual-ité) = (ce qui est) — (qual)
    \end{tabular}
  \end{center}

  \noindent en nous rappelant que «ag» = «ag-ir» et «qual» = «de
  qualité». On ne peut aller plus loin dans l’analyse, car après avoir
  ramené les molécules particulières aux molécules générales, et après
  avoir dissocié celles-ci de manière à montrer leur sens, uniquement
  au moyen des atomes fondamentaux qu’elles contiennent et sans qu’il
  n’existe plus aucun lien, aucune soudure entre ces atomes (condition
  importante), on a fait le même travail que le chimiste qui, pour
  analyser une molécule particulière, la fait rentrer dans une
  famille, dans une molécule servant de type à toute une série, puis
  analyse le contenu de cette molécule-type en en séparant tous les
  atomes. L’analyse est alors terminée, car les atomes fondamentaux,
  c’est-à-dire les idées grammaticales \textbf{(a, i, o)} sont le
  résidu ultime de l’analyse, les derniers éléments irréductibles,
  nécessaires et suffisants pour définir le sens d’un mot fondamental
  et, par suite, d’un mot quelconque.

  Ainsi, si l’on ne veut pas tomber dans un cercle vicieux en faisant
  l'analyse des mots, puisqu’un mot ne peut être défini que par
  d'autres mots, \emph{il}

}
%
{\noindent
  %
  these molecules by applying the law of reversal of atoms, and we
  thus find:

  \begin{center}
    \begin{tabular}[t]{l}
      (ac-tion) = (that which is) — (ag)\\
      (qual-ité) = (that which is) — (qual)
    \end{tabular}
  \end{center}

  \noindent
  while recalling that ``ag'' = \emph{ag-ir} `to act' and ``qual'' = ``of
  quality''. We cannot go further in the analysis, because after
  having brought the specific molecules back to general molecules, and
  after having dissociated these so as to show their sense, solely by
  means of the fundamental atoms that they contain and without which
  there would not exist any link, any juncture between these atoms (an
  important condition), we have done the same work as the chemist who,
  to analyze a specific molecule, puts it into a family, with a
  molecule that serves as the type for an entire series, and analyzes
  the content of that type-molecule by separating its atoms. The
  analysis stops there, since the fundamental atoms, that is the
  grammatical ideas \textbf{(a, i, o)} are the ultimate residue of the
  analysis, the last irreducible elements, necessary and sufficient to
  define the sense of a basic word and consequently of any arbitrary
  word.

  Thus, if one does not wish to fall into a vicious circle in
  performing the analysis of words, since a word cannot be defined
  except by other words, \emph{it}

}

\TextPage{\noindent
  %
  \emph{faut définir le sens des mots particuliers par celui des mots
    généraux, puis définir ceux-ci en dissociant les atomes qui les
    constituent} et qui, du reste, ne sont qu’au nombre de trois
  (atomes fondamentaux \textbf{a, o, i}), car malgré leur diversité de
  formes, les atomes fondamentaux ne représentent que l’idée
  adjective, l’idée substantive ou l’idée verbale. Ces trois idées
  sont les données sur lesquelles tout le reste est construit.

  Ce serait donc une erreur, une pétition de principes que de définir
  les mots généraux comme «action», «qualité», etc., (qui sont des
  molécules fondamentales) par d’autres mots qui souvent sont moins
  généraux que ceux que l’on prétend définir. Cette erreur se
  rencontre chez beaucoup d’auteurs et dans la plupart des
  dictionnaires.

  Voyons par exemple, ce que dit Littré à propos des mots «action» et
  «qualité»:

  1° \emph{Action}. — A la page 72, on peut lire: «Action (gramm.) =
  \emph{ce qu’exprime, ce que marque le verbe}.». Cette définition est
  parfaite, car elle se compose (comme le mot «ac-tion») de deux
  parties: «ce» et «qu’exprime le verbe». Or, nous avons vu que «ce»
  représente précisément l’idée substantive, ainsi que le suffixe
  «tion», et à la question «qu’exprime le verbe?» nous avons répondu:
  «Il exprime l'idée représentée par l’atome «ac», atome qui ne
  contient qu’une idée ultime et%»

}
%
{\noindent
  %
  \emph{is necessary to define the meaning of specific words by that
    of general words, then define those by dissociating the atoms of
    which they are composed} and which, moreover, number only three
  (basic atoms \textbf{a, o, i}), for in spite of their diversity of
  forms, the basic atoms represent only the adjectival idea, the
  nominal idea or the verbal idea. These three ideas are the elements
  on which all the rest is built.

  It would thus be an error, begging the question, to define the
  general words such as \emph{action}, \emph{qualité}, etc. (which are basic
  molecules) by other words which are often less general than those
  which one claims to be defining. This error is found in many writers
  and in the majority of dictionaries.

  Let us see for example what the \textsl{Littré} says on the
  subject of the words \emph{action} and \emph{qualité}:

  1. \emph{Action} --- On p. 72, we can read: ``Action (gramm.) =
  \emph{ce qu'exprime, ce que marque le verbe.}'' `that which the verb
  expresses, marks'. This definition is perfect, because it is
  composed (like the word \emph{ac-tion}) of two parts: \emph{ce} `that' and
  \emph{qu'exprime le verbe} `which the verb expresses'.  Now we have
  seen that \emph{ce} represents precisely the nominal idea, like the
  suffix \emph{tion}, and to the question ``what does the verb express?''
  we have answered ``It expresses the idea represented by the atom
  ``ac'', an atom which contains only one ultimate [basic] idea

}

\TextPage{\noindent
  % «
  fondamentale que nous avons figurée par le symbole \textbf{i}[»];
  donc on a exactement:

  \noindent
  \begin{center}
    (ce) -- (qu’exprime le verbe) = (tion) -- (ac)\\
    = (ac-tion) = \textbf{(i-o)}.
  \end{center}


  Mais il ne faut pas oublier que c’est nous qui avons répondu à la
  question «qu’exprime le verbe?» en disant qu’il exprime l’idée «ac»
  (ou «sta»). En réalité, Littré répond tout autrement, car si l’on
  cherche la définition du mot «verbe» (à la page 1256), on trouve:
  «verbe» = «\emph{partie du discours qui exprime une action ou un
    état sous forme variable}» ! Cette définition ôte toute sa valeur
  à la première définition du mot «action»; car après avoir défini le
  mot «action» par le mot «verbe», Littré définit à son tour le mot
  «verbe» par le mot «action». On ne saurait concevoir de cercle
  vicieux plus parfait, mais comme il y a environ 1200 pages de texte
  entre les deux définitions, on a eu le temps d’oublier la première
  quand on lit la seconde. En réalité, la première définition de
  Littré est la seule juste; en disant que «ac-tion» est
  «ce—qu’exprime le verbe», il donne précisément la définition qui
  résulte de la dissociation des deux atomes «ac» et «tion», puisque
  «tion» est l’idée substantive «ce», et que, d’autre part,
  «qu’exprime le verbe?» sinon l’idée verbale \textbf{i} = «ac» (ou
  «sta»)?  La définition de Littré revient donc bien à dire que
  «action» =

}
%
{\noindent
  %
  basic [idea] which we have represented by the symbol \textbf{i}'':
  thus, we have exactly:

  \noindent
  \begin{center}
    (that) -- (which the verb expresses)\\
    = (tion) -- (ac)\\
    = (ac-tion) = \textbf{(i-o)}.
  \end{center}

  But it is necessary not to forget that it is we who have responded
  to the question ``what does the verb express?'' by saying that it
  expresses the idea ``ac'' (or ``sta''). In reality, \textsl{Littré}
  responds quite differently, for if we seek the definition of the
  word ``verb'' (on page 1256), we find: \emph{verbe} = ``\emph{part of
    speech which expresses an action or a state in variable form}''!
  This definition deprives the first definition of the word \emph{action}
  of all its value, since after defining the word \emph{action} by the
  word ``verb'', \textsl{Littré} defines in turn the word verb by
  the word ``action''. It would be hard to imagine a more perfect
  vicious circle, but since there are about 1200 pages of text between
  the two definitions, we have had time to forget the first when we
  read the second. In reality, the first definition of the
  \textsl{Littré} is the only legitimate one: in saying that
  \emph{ac-tion} is ``that --- which the verb express'' it gives
  precisely the definition which results from the dissociation of the
  two atoms ``ac'' and ``tion'', since ``tion'' is the nominal idea
  ``that''; and what is it, on the other hand, ``which the verb
  expresses?'' except the idea \textbf{i} = ``ac'' (or ``sta'')? The
  \textsl{Littré} definition thus comes down to saying that
  \emph{action} =

}

\TextPage{\noindent
  %
  \textbf{(i-o)}; l’idée «action» est donc simplement \emph{l'idée
    qui résulte de la juxtaposition de l'idée substantive à l'idée
    verbale}. Par contre, la définition du mot «verbe» donnée par
  Littré n’a aucune valeur, non seulement parce qu’elle constitue un
  cercle vicieux, mais encore parce qu’elle prétend définir un atome
  fondamental comme \textbf{i} au moyen du mot «action», qui est une
  molécule.

  En résumé, la seule méthode logique est la \emph{réduction aux
    atomes}; c’est l’atome «verbe» \textbf{i} qui doit définir le
  mot «action» et non l’inverse, et l’on peut écrire cette définition
  sous l’une quelconque des formes suivantes:

  \noindent
  {\setlength{\tabcolsep}{1pt}
    \resizebox{\linewidth}{!}{\begin{tabular}[t]{rl}
      (\emph{ac-tion})&= (\emph{ce}) -- (qu’exprime le \emph{verbe}),\\
                      &= (\emph{ce} qui est) -- (\emph{idée verbale}),\\
                      &= (\emph{ce} qui est) -- (\emph{ac} ou \emph{ag} ou \emph{agir}\footnotemark),\\
                      &= (\emph{tion}) --	(\emph{ac}),\\
                      &= (\emph{idèe substantive}) soudée à (\emph{idée verbale}),\\
                      &= \makebox[9em][c]{\textbf{(o)}}
                        \makebox[1em][c]{---}\makebox[4em][r]{\textbf{(i)},}\\
                      &= molécule \makebox[7.1em][r]{\textbf{(i-o)}.}\\
    \end{tabular}}
  }

\noindent
et c’est tout.

2° \emph{Qualité}. — Si nous cherchons maintenant dans Littré la
définition du mot «qualité», nous trouvons: «qualité» = «ce qui fait
qu’une chose est%»

\footnotetext{Car \emph{ag-ir} est un simple pléonasme équivalant à
  \emph{ag}.}

}
%
{\noindent
  %
  \textbf{(i-o)}; the idea \emph{action} is thus simply \emph{the idea
    that results from the juxtaposition of the nominal idea with the
    verbal idea.} On the other hand, the definition of the word
  ``verb'' given by the \textsl{Littré} has no value, not only because
  it constitutes a vicious circle, but also because it claims to
  define a basic atom like \textbf{i} by means of the word ``action''
  which is a molecule.

  In summary, the only logical method is the \emph{reduction to
    atoms}; it is the atom ``verb'' \textbf{i} that must define the
  word \emph{action} and not the reverse, and one can write this
  definition in any of the following ways:

  \noindent
  {\setlength{\tabcolsep}{1pt}
    \resizebox{\linewidth}{!}{\begin{tabular}[t]{@{}rl@{}}
      (\emph{ac-tion})&= (\emph{that}) -- (which the \emph{verb} expresses),\\
                      &= (\emph{that} which is) -- (\emph{verbal} idea),\\
                      &= (\emph{that} which is) -- (\emph{ac} or \emph{ag} or \emph{agir}\footnotemark),\\
                      &= (\emph{tion}) --	(\emph{ac}),\\
                      &= (\emph{nominal idea}) joined to (\emph{verbal idea}),\\
                      &= \makebox[8em][c]{\textbf{(o)}}
                        \makebox[1em][c]{---}\makebox[4em][r]{\textbf{(i)},}\\
                      &= molecule \makebox[7.1em][r]{\textbf{(i-o)}.}\\
    \end{tabular}}  
  }

  \noindent
  and that is all.  \footnotetext{Because \emph{ag-ir} `to act' is a
    simple pleonasm equivalent to \emph{ag}.}

  2. \emph{Quality}. --- If we now look in the \textsl{Littré} for
  the definition of the word \emph{qualité}, we find: \emph{qualité} =
  ``that which makes it that a thing is%''

}

\TextPage{\noindent
  %
  telle». Cette définition est bonne, mais peut être mise sous une
  forme plus symétrique. Tout d’abord, l’adjectif «tel» (\emph{talis})
  correspond à l’adjectif «quel» (\emph{qualis}); l’un est seulement
  la réponse de l’autre; quand on demande: «Quelle est cette chose?»
  on répond: «Elle est telle». On peut donc considérer l’atome «tel»
  comme atome adjectif fondamental, au même titre que «qual»; de sorte
  que l’on peut dire que «qual-ité» = «ce qui est — tel» ou «ce qui
  est — qual» dans une chose. Cette définition est la seule bonne,
  parce qu’elle explique le mot composé «qualité» au moyen des deux
  atomes qui le composent et sans utiliser autre chose.

  Littré dit aussi: «qualité» = «ce qui constitue la manière d’être
  d’une chose». Cette définition est moins lumineuse; néanmoins, on
  peut l’utiliser en la comparant à celle de l’adjectif
  qualificatif. Littré dit en effet: «adjectif» = «l'une des dix
  parties du discours, mot que l’on joint au substantif pour le
  qualifier,... mot qui sert à exprimer la manière d’être...». Comme
  nous avons vu que «qual-ifi-er» = «rendre qual», la définition de
  Littré signifie que l’adjectif sert à rendre «qual» le substantif
  (par exemple «hum-ain» = «hom-qual»); il ajoute donc au substantif
  l’idée «qual», ce qui revient à dire tout simplement que l’idée
  adjective («ain») est l’idée «qual».

  En outre, à la question: «qu’exprime l’adjectif?»

}
%
{\noindent
  %
  such''. This definition is good, but can be put into a more
  symmetric form. First of all, the adjective \emph{tel} `such'
  (\emph{talis}) corresponds to the adjective \emph{quel} `what (kind)'
  (\emph{qualis}): the one is simply the answer to the other.  When we
  ask \emph{Quelle est cette chose?} `What kind of thing is that?', we
  answer \emph{Elle est telle} `It is such (a thing)'.  We can thus
  consider the atom \emph{tel} as a basic adjective, in the same capacity
  as ``qual''; so that we can say that \emph{qual-ité} = ``that which
  is --- such'' or ``that which is --- qual'' in a thing. This
  definition is the only good one, because it explains the compound
  word ``quality'' by means of the two atoms that compose it and
  without using anything else.

  The \textsl{Littré} also says \emph{qualité} = ``that which
  constitutes the manner of being of a thing''.  This definition is
  less lucid; nevertheless, we can make use of it by comparing it to
  that of the qualifying adjective. \textsl{Littré} says indeed:
  \emph{adjectif} = ``one of the ten parts of speech, a word which one
  joins to a noun to qualify it, \ldots\ a word which serves to
  express the manner of being \ldots''.  As we have seen that
  \emph{qual-ifi-er} `to qualify' = ``to make qual'', the
  \textsl{Littré} definition means that the adjective serves to make
  the noun ``qual'' (for example \emph{hum-ain} = ``hom-qual'').  It thus
  adds to the noun the idea ``qual'', which comes down to saying
  simply that the adjectival idea (\emph{ain}) is the idea ``qual''.

  In addition, to the question ``what does the adjective express?''

}

\TextPage{\noindent
  %
  Littré répond: «Il exprime la manière d’être», et comme auparavant
  Littré a défini la «qualité» comme étant «ce qui constitue la
  manière d’être d’une chose», nous concluons immédiatement, en
  réunissant ces deux définitions, que «qualité» = «\emph{ce
    qu'exprime l'adjectif}», forme d’où nous retirons un double
  avantage, car en réunissant les deux définitions de Littré, nous
  avons complètement éliminé l’expression gênante de «manière d’être»
  (comme on élimine une variable superflue entre deux équations), et
  secondement la définition du mot «qualité» fait maintenant pendant à
  celle du mot «action», qui, d’après Littré même, signifie «ce
  qu’exprime le verbe». On ramène ainsi la définition de ces deux mots
  à l’analyse de leurs atomes et l’on considère les idées verbale et
  adjective comme des matériaux primitifs de la construction des
  mots. C’est l’atome adjectif (\textbf{a}) qui doit définir le mot
  «qualité» et non l’inverse; aussi, nous supprimerons la fin de la
  définition de Littré, dans laquelle il ajoute: «l’adjectif sert à
  exprimer la \emph{qualité} des choses», car cette partie de la
  définition est un simple cercle vicieux\footnote{Littré dit aussi
    «l’état des choses», mais il ne faut pas confondre la «qualité»
    avec «l’état». Voir le dernier chapitre.}. Il faut dire, au
  contraire, que la «qualité est \emph{l'idée qui résulte de la
    juxtaposition de l'idée substantive à l'idée adjective}».

}
%
{\noindent
  %
  \textsl{Littré} responds: ``it expresses the manner of being'', and
  since previously \textsl{Littré} defined the \emph{qualité} as being
  ``that which constitutes the manner of being of a thing'', we
  immediately conclude, by combining these two definitions, that
  ``quality'' is ``\emph{that which the adjective expresses}'', a form
  from which we derive a double advantage, because in combining the
  two definitions of the \textsl{Littré}, we have completely
  eliminated the troubling expression ``manner of being'' (as one
  eliminates a superfluous variable between two equations); and
  secondly, the definition of the word ``quality'' now corresponds to
  that of the word ``action'' which, according to the \textsl{Littré}
  itself, means ``that which the verb expresses''. We thus reduce the
  definition of these two words to the analysis of their atoms and
  consider the verbal and adjectival ideas as primary materials for
  the construction of words. It is the adjectival atom \textbf{(a)}
  which must define the word ``quality'', and not the reverse; also we
  will eliminate the last part of the \textsl{Littré} definition,
  which adds ``the adjective serves to express the \emph{quality} of
  things'', for that part of the definition is simply a vicious
  circle\footnote{\textsl{Littré} also says ``the state of things'',
    but it is necessary not to confuse the ``quality'' with the
    ``state''.  See the final chapter.} On the contrary, it is
  necessary to say that the ``quality is  \emph{the idea that
    results from the juxtaposition of the nominal idea with the
    adjectival idea}''.


}

\TextPage{En résumé, la définition du mot «qualité» peut prendre l’une
  quelconque des formes suivantes, qui sont d’ailleurs tout à fait
  symétriques des formes servant à définir le mot «action»:

  \noindent
  {\setlength{\tabcolsep}{1pt}
    \resizebox{\linewidth}{!}{\begin{tabular}[t]{rl}
      (\emph{qual-ité})&= (\emph{ce}) -- (qu’exprime \emph{l'adjectif}),\\
                         &= (\emph{ce} qui est) -- (\emph{idée adjective}),\\
                         &= (\emph{ce} qui est) -- (\emph{qual} ou \emph{tel})\\
                         &= \makebox[4em][r]{(\emph{ité})}
                           \makebox[1.5em][r]{--}\makebox[5em][r]{(\emph{qual}),}\\ 
                         &= (\emph{idèe substantive}) soudée à  (\emph{idée}\\
                         &\multicolumn{1}{r}{\emph{adjective}),}\\
                         &= \makebox[5em][c]{\textbf{(o)}}
                           \makebox[1em][c]{---}\makebox[4em][r]{\textbf{(a)},}\\
                         &= molécule \textbf{(a-o)}.\\
    \end{tabular}}
  }
  
  \textsc{Exemples d’analyses de mots}. — Nous sommes maintenant en
  possession de tous les éléments nécessaires pour procéder à
  l’analyse logique d’un mot particulier quelconque. Quelques exemples
  suffiront.

  1. \emph{Analyse du mot }«\emph{beauté}». — On commence par
  indiquer les atomes qui composent le mot en écrivant «beau-té»; on
  remplace l’atome particulier «beau» par l’atome général
  correspondant «qual», et comme l’atome «té» ou «ité» est déjà un
  atome fondamental, on conclut que «beau-té» est un cas particulier
  de «qual-ité», puisque «beau» est un cas particulier de «qual». On
  écrira donc:

}
%
{In summary, the definition of the word ``quality'' can take any one
  of the following forms, which are however completely symmetric with
  the forms that define the word ``action'':

  \noindent
  {\small {\setlength{\tabcolsep}{1pt}
      \resizebox{\linewidth}{!}{\begin{tabular}[t]{rl}
        (\emph{qual-ité})&= (\emph{that}) -- (which
                             the \emph{adjective} \\
                           &\multicolumn{1}{r}{expresses),}\\
                           &= (\emph{that} which is) --- (\emph{adjective
                             idea}),\\
                           &= (\emph{that} which is) --- (\emph{qual} or \emph{tel})\\
                           &= \makebox[5.4em][r]{(\emph{ité})}
                             \makebox[2em][r]{---}\makebox[4.5em][r]{(\emph{qual}),}\\ 
                           &= (\emph{nominal idea}) joined to
                             (\emph{adjectival}\\
                           &\multicolumn{1}{r}{\emph{idea}),}\\
                           &= \makebox[5.7em][r]{\textbf{(o)}}
                             \makebox[2.5em][c]{---}\makebox[3.5em][r]{\textbf{(a)},}\\
                           &= molecule \textbf{(a-o)}.\\
      \end{tabular}}
    } }

  \textsc{Examples of analyses of words}. --- We now have all of the
  elements necessary to to proceed to the logical analysis of an
  arbitrary specific word. Some examples will suffice.

  1. \emph{Analysis of the word \emph{beauté}}. ---We begin by
  indicating the atoms that compose the word in writing \emph{beau-té};
  we replace the atom \emph{beau} by the corresponding general atom
  ``qual'', and since the atom \emph{té} or \emph{-ité} is already a
  basic atom, we conclude that \emph{beau-té} is a specific case of
  \emph{qual-ité} since \emph{beau} is a specific case of ``qual. Thus we
  write:
  

}

\TextPage{\noindent
  %
  \begin{center}
    \begin{tabular}[t]{l}
      «beau-té» = «qual-ité \emph{beau}».\\
      (qual)
    \end{tabular}
  \end{center}
  
  L’analyse est terminée, puisque le mot fondamental «qualité» a déjà
  été analysé; mais l’équation ci-dessus nous montre différentes
  choses.

  Nous avons vu que dans les molécules fondamentales, la soudure entre
  les atomes est une simple juxtaposition (à part le renversement de
  l’ordre des atomes dû à la condensation de la molécule); ainsi, nous
  savons que: «ité» = «ce (qui est)», donc «qualité» = «ce (qui est) —
  qual». On a aussi «beauté» = «ce qui est beau», à condition de
  laisser à l’atome «ce» toute sa généralité; cet atome désigne
  «l’ètre en général», «l’être abstrait», donc «beau-té» =
  «l’abstraction beau» = «le beau». Mais l’analyse est plus précise et
  plus nette, lorsqu’on se sert des idées générales sous-entendues qui
  se trouvent dans les atomes particuliers. On a alors «beauté» =
  «qualité \emph{beau}», sens plus restreint que «abstraction
  \emph{beau}», car si toutes les qualités sont des abstraits, tous
  les abstraits ne sont pas des qualités. Il y a donc une nuance entre
  «beauté» («qualité \emph{beau}») et «le beau» («abstraction
  \emph{beau}»)\footnote{II y a une nuance analogue entre «boisson»
    («action boire») et «le boire» («abstraction boire»).}. Ce dernier
  sens est plus général que le

}
%
{\noindent
  %
  \begin{center}
    \begin{tabular}[t]{l}
      \emph{beau-té} = ``qual-ité \emph{beau}''.\\
      (qual)
    \end{tabular}
  \end{center}
  
  The analysis terminates, because the basic word \emph{qualité} has
  already been analyzed, but the equation above shows us different
  things.

  We have seen that in the basic molecules, the junction between atoms
  is a simple juxtaposition (apart from the reversal of the order of
  atoms resulting from the condensation of the molecule).  Thus, we
  know that \emph{ité} = ``that (which is)'', thus \emph{qualité} =
  ``that (which is) --- qual''. We thus have \emph{beauté} = ``that
  which is beautiful'', providing we give the atom \emph{ce} `that' its full
  generality; this atom designates ``being in general'', ``abstract
  being'', and thus \emph{beau-té} = ``the abstraction beautiful'' =
  ``the beautiful''. But the analysis is more precise and clearer,
  since one makes use of the understood general ideas that are present
  in specific atoms. We thus have \emph{beauté} = ``quality
  \emph{beautiful}'', a narrower sense than ``abstraction
  \emph{beautiful}'', because even if all qualities are abstractions,
  not all abstractions are qualities. There is thus a nuance between
  \emph{beauté} (``quality \emph{beautiful}'') and \emph{le beau}
  (``abstraction \emph{beautiful}'')\footnote{There is an analogous
    nuance between \emph{boisson} `drink(ing)' (``action \emph{to
      drink}'') and \emph{le boire} (``abstraction \emph{to drink}'').}
  This last sense is more general than the

}

\TextPage{\noindent
  %
  premier, puisque la «qualité» n’est qu’un cas de l’«abstraction».

  Les idées générales sous-entendues sont donc très utiles dans
  l’analyse des mots. Ce ne sont, du reste, pas seulement les idées
  les plus générales (idées grammaticales) qui peuvent servir à cette
  analyse. Prenons, par exemple, le mot «Lyonnais»; ce mot contient
  deux atomes «Lyon» et «ais»; l’atome «Lyon» est un atome particulier
  substantif, mais, outre l’idée substantive («un être concret»),
  l’idée «Lyon» contient des idées moins générales, comme, par
  exemple, l’idée «France», si l’on se place au point de vue
  géographique. On peut donc écrire:\\[1ex]

  «Lyonn-ais» = «Franç-ais de \emph{Lyon}.»

  (France)\\[1ex]

  \noindent
  %
  tout comme on a écrit\footnote{Ces équations sont justes au sens
    mathématique, c’est-à-dire que les deux membres sont exactement
    équivalents, quoiqu’il semble possible de faire l’objection
    suivante: puisque l’atome «beau», dans le premier membre de
    l’équation, contient en lui-même l’idée «qual», ce même atome
    «beau», dans le second membre, doit aussi contenir l’idée «qual»
    en lui-mème, et alors il n’y aurait qu’un atome «qual» dans le
    premier membre et deux dans le second; donc les deux membres ne
    seraient pas équivalents; en d'au-}:\\[1ex]

  «beau-té» = «qual-ité \emph{beau}».

  (qual)

}
%
{\noindent
  %
  first, since the ``quality'' is only one case of the
  ``abstraction''.

  The understood general ideas are thus very useful in the analysis of
  words. Moreover, it is not only the most general ideas (grammatical
  ideas) that can be made use of in this analysis. Take for example
  the word \emph{Lyonnais}: this word contains two atoms \emph{Lyon} and
  \emph{ais}. The atom \emph{Lyon} is a specific nominal atom, but besides
  the nominal idea ``\emph{a} concrete entity''), the idea ``Lyon''
  contains less general ideas, such as for example the idea
  ``France'', if one takes the point of view of geography.  We can
  thus write:\\[1ex]

  \emph{Lyonn-ais} = ``Franç-ais from \emph{Lyon}.''

  (France)\\[1ex]

  \noindent
  %
  just as we wrote\footnote{These equations are correct in the
    mathematical sense, that is, the two sides are exactly equivalent,
    even though it is possible to make the following objection: since
    the atom \emph{beau} in the first member of the equation contains in
    itself the idea ``qual'', this same atom \emph{beau} in the second
    member must also contain in itself the idea ``qual'', and then
    there would be only one atom ``qual'' in the first member and two
    in the
    second, and thus the two members would not be equivalent. In [other]}:\\[1ex]

  \emph{beau-té} = `` qual-ité \emph{beau}''.

  (qual)

}

\TextPage{Les idées générales sous-entendues servent donc à donner un
  sens précis aux mots composés, car c’est par elles que le sens
  général d'un mot est déterminé. Ainsi, dans le mot «beauté», les
  idées gé-

  \blfootnote{tres mots, si l’on relranche l'atome «beau» dans les
    deux membres, il reste «té» ou «ité» = «qual-ité», ce qui est
    faux.

    Cette objection n'est pas bien fondée, car, en réalité, l’atome
    «beau», dans le second membre, n’est plus du tout le même que
    l'atome «beau» dans le premier membre. Le premier atome «beau» est
    un vrai adjectif qui a un pouvoir qualificateur et qui, par
    conséquent, contient en lui-même l’idée générale adjective «qual»;
    c’est un atome complet, un atome vivant. Tout autre est l’atome
    «beau» dans le second membre: il n’a plus de pouvoir
    qualificateur, il ne qualifie plus le substantif qui est à côté de
    lui, car la «qualité \emph{beau}» n’est pas du tout la même chose
    qu’une «belle qualité»; l’atome «beau» du second membre n’exprime
    donc plus qu’une idée particulière et ne contient plus l’idée
    générale «qual»: cette idée «qual», qui constituait pour ainsi
    dire la vie, l'âme de l'atome «beau» dans le premier membre de
    l’équation, a été extirpée et mise en évidence dans le mot
    «qualité», où elle figure explicitement; l’atome «qual» du mot
    «qualité», dans le second membre, est donc bien le même que l’idée
    «qual» qui était contenue dans l’atome «beau» du premier membre;
    donc l’atome «beau» du second membre n’est plus qu’une sorte de
    cadavre, une coquille dont l’animal intérieur («qual») a été
    extirpé; ce n’est plus qu'un numéro; si l’on numérote tous les
    adjectifs du dictionnaire, la qualité «beau» sera, par exemple, la
    qualité N° 127; c est pourquoi, dans les équations ci-dessus, j’ai
    distingué les atomes vivants «beau», «Lyon» des mêmes atomes
    morts, en mettant ceux-ci en italiques.}

}
%
{The understood general ideas thus serve to give a precise sense to
  compound words, because it is through them that the general sense of
  a word is determined. Thus, in the word \emph{beauté}, the ge[neral]
  ideas

  \blfootnote{other words, if we delete the atom \emph{beau} on the two
    sides, what remains is \emph{té} or \emph{ité} = \emph{qual-ité},
    which is false.

    This objection is not well founded, because in reality, the atom
    \emph{beau} in the second member is not at all the same as the atom
    \emph{beau} in the first member. The first atom \emph{beau} is a true
    adjective which has qualifying power and which, as a result,
    contains in itself the general adjectival idea ``qual'': it is a
    complete atom, a living atom. The atom \emph{beau} in the second
    member is quite different: it no longer has qualifying power, it
    does not qualify the noun beside it any more, for the ``quality
    \emph{beau}'' is not at all the same thing as a ``beautiful
    quality''. The atom \emph{beau} in the second member thus expresses
    nothing more than a specific idea and no longer contains the
    general idea ``qual''. This idea ``qual'', which constitutes as it
    were the life, the soul of the atom \emph{beau} in the first member
    of the equation, has been rooted out and made obvious in the word
    \emph{qualité}, where it figures explicitly.  The atom ``qual'' of
    the word \emph{qualité} in the second member is thus just the same
    as the idea ``qual'' which was contained in the atom \emph{beau} of
    the first member.  Thus the atom \emph{beau} of the second member is
    no more than a sort of cadaver, a shell from which the animal
    inside (``qual'') has been extracted; it is no more than a number;
    if one enumerated all of the adjectives in the dictionary, the
    quality \emph{beau} would be, for example, \#127.  That is why, in
    the equations above, I have distinguished the living atoms
    \emph{beau}, \emph{Lyon}, from the same atoms in death, putting these in
    italics.}

}

\TextPage{\noindent
  %
  nérales sont \textbf{a} et \textbf{o}, et l’on peut écrire
  symboliquement:

  \begin{center}
    \begin{tabular}[t]{l}
      \textbf{bel-o} = \textbf{(a-o)} espèce \emph{bel}.\\
      \textbf{(a)}
    \end{tabular}
  \end{center}

  \noindent
  équation qui montre que le mot «beauté» rentre dans le type
  \textbf{(a-o)}, c’est-à-dire dans le type des adjectivo-substantifs
  (comme «richesse», «grandeur», etc.), dont le sens général est
  «qual-ité».

  On peut dire que le mot «beau» est égal à l’atome général \textbf{a}
  coiffé de l’idée particulière \emph{beau}, atome que l’on peut
  représenter symboliquement par \textbf{â}, l’accent circonflexe
  étant destiné à faire une distinction entre les atomes particuliers
  et les atomes généraux; ainsi j’écrirai:

  \begin{center}
    {\small «beau» = \textbf{â}, «té» = \textbf{o}, d’où «beau-té» =
      \textbf{â-o}.}
  \end{center}


  2. \emph{Analyse du mot «violoniste»}. — Cette molécule est
  bi-atomique («violon-iste») et les deux atomes qui la composent sont
  tous deux substantifs. Elle rentre donc dans le type général
  \textbf{(o-o),} qui contient par conséquent un pléonasme. Mais le
  pléonasme est ici à peine perceptible et ne joue qu’un rôle tout à
  fait secondaire, car ce qui importe dans le mot «violoniste», ce
  sont les idées particulières «violon» et «iste»; or, le pléonasme ne
  se rapporte pas à ces idées particulières (\textbf{ô}), mais
  seulement à l’idée substantive générale (\textbf{o}).

}
%
{
  \noindent
  %
  general [ideas] are \textbf{a} and \textbf{o}, and one can write
  symbolically:

  \begin{center}
    \begin{tabular}[t]{l}
      \textbf{bel-o} = \textbf{(a-o)} type \emph{bel}.\\
      \textbf{(a)}
    \end{tabular}
  \end{center}

  \noindent
  %
  an equation that shows that the word \emph{beauté} belongs to the
  type \textbf{(a-o)}, that is, to the type of nominalized adjectives
  (like \emph{richesse} `rich-ness: wealth', \emph{grandeur} `large-ness:
  size', etc.), whose general sense is \emph{qual-ité}.

  We can thus say that the word \emph{beau} is equal to the general atom
  \textbf{a} elaborated by the specific idea \emph{beau}, an atom that
  we can represent symbolically by \textbf{â}, the circumflex accent
  intended to make a distinction between specific atoms and the
  general atoms.  Thus, I will write:

  \begin{center}
    {\small \emph{beau} = \textbf{â}, \emph{té} = \textbf{o}, thus
      \emph{beau-té} = \textbf{â-o}.}
  \end{center}

  2. \emph{Analysis of the word \emph{violoniste}}. --- This molecule is
  bi-atomic (\emph{violon-iste}), and the two atoms that make it up are
  both nouns. It thus belongs to the general type \textbf{(o-o)},
  which contains as a result a pleonasm. But the pleonasm is hardly
  perceptible here, and only plays a completely secondary role,
  because what matters in the word \emph{violoniste} are the specific
  ideas \emph{violon} and \emph{iste} and pleonasm is not related to
  specific ideas (\textbf{ô}), but only to the general nominal idea
  (\textbf{o}).

}

\TextPage{Aussi, tandis que les molécules générales du type
  \textbf{(o-o)} ou \textbf{(a-a)}, etc., sont très rares, parce que
  ce sont de simples pléonasmes; les molécules particulières
  \textbf{(ô-ô)} ou \textbf{(â-â)}, etc., sont très fréquentes et ne
  sont pas réductibles, parce que les deux atomes expriment chacun une
  idée particulière différente et n’ont en commun que l’idée générale
  substantive (ou adjective). On s’en rend compte en écrivant sous
  chaque atome toutes les idées plus générales
  qui sont implicitement contenues dans cet atome:\\[1ex]

  \noindent
  {\setlength{\tabcolsep}{0pt}
    \resizebox{\linewidth}{!}{\begin{tabular}[t]{cccc}
      \emph{violon}&---&\emph{iste}&= \textbf{(ô-ô)}\\
      (objet)&&(personne)\\
      (un être concret)&&(un être concret)\\
      (un être)&& (un être)\\
      \textbf{o}&&\textbf{o}
    \end{tabular}}}
  \\[1ex]

  On a donc:

  \begin{center}
    {\small
      «iste» = un être-personne (espèce \emph{professionnelle}).\\
      «violon» = un être-objet (espèce \emph{violon}).}
  \end{center}

  \noindent
  et par suite, en dissociant et renversant les atomes, le sens du mot
  «violoniste» est: «un être-personne professionnel (caractérisé par)
  l’être-objet violon»\footnote{Ici, la soudure entre les deux atomes
    du mot «violoniste» n’est pas une simple juxtaposition. Nous
    reviendrons là-dessus dans le second chapitre.}.

  3. \emph{Analyse du mot «mammifère».} — L’atome «mammi» (mammelle)
  est un radical substantif

}
%
{Also, although the general molecules of the type \textbf{(o-o)} or
  \textbf{(a-a)} are quite rare, because they are simple pleonasms,
  the specifici molecules \textbf{(ô-ô)} or \textbf{(â-â)}, etc. are
  very frequent and are not reducible, because the two atoms each
  express a different specific idea and have in common only the
  general nominal (or adjectival) idea. We take this into account by
  writing under each atom all of the more general ideas that are
  implicitly
  contained in that atom:\\[1ex]

  \noindent
  {\setlength{\tabcolsep}{0pt}
    \resizebox{\linewidth}{!}{\begin{tabular}[t]{cccc}
      \emph{violon}&---&\emph{iste}&= \textbf{(ô-ô)}\\
      (object)&&(person)\\
      (concrete entity)&&(concrete entity)\\
      (an entity)&& (an entity)\\
      \textbf{o}&&\textbf{o}
    \end{tabular}}}
  \\[1ex]

  Thus we have:

  \begin{center}
    {\small
      \emph{iste} = a person-entity (type \emph{professional}).\\
      \emph{violon} = an object-entity (type \emph{violin}).}
  \end{center}

  \noindent
  and following that, by dissociating and reversing the atoms, the
  meaning of the word \emph{violoniste} is ``a person-entity
  \emph{professional} (characterized by) the object-entity
  \emph{violin}''\footnote{Here the juncture between the two atoms of
    the word \emph{violon-iste} is not a simple juxtaposition.  We will
    come back to this in the second chapter.}.

  3. \emph{Analysis of the word \emph{mammifère}}. --- The atom
  \emph{mammi} `teat' is a [specific] noun root

}

\TextPage{\noindent
  %
  particulier (\textbf{ô}); l’atome «fère» (qui porte) est un suffixe
  adjectif particulier (\textbf{â}).

  Cette molécule ne contient aucun pléonasme et s’analyse
  immédiatement, comme suit:\\[1ex]

  {\setlength{\tabcolsep}{0pt}
    \begin{tabular}[t]{cccc}
      \emph{mammi}&\makebox[2em][l]{---}&\emph{fère}&= \textbf{(ô-â)}\\
      (objet)&&\textbf{a}\\
      (un être concret)\\
      (un être)\\
      \textbf{o}
    \end{tabular}}
  \\[1ex]

  Le mot «mammifère» est donc un adjectif, et signifie, sous forme
  dissociée: «(qui porte) — (l’être-objet \emph{mammelle})».

  4. \emph{Analyse du mot «chandelier»}. — L’atome «chandel» est un
  radical substantif particulier (\textbf{ô}); l’atome «ier» est un
  suffixe substantif particulier (\textbf{ô}) qui signifie «objet qui
  porte». On a
  donc:\\[1ex]

  \noindent
  {\setlength{\tabcolsep}{0pt} {\small
      \begin{tabular}[t]{cccc}
        \emph{chandel}&---&\emph{ier}&= \textbf{(ô-ô)}\\
        (objet)&&(objet)\\
        (un être concret)&&(un être concret)\\
        (un être)&&(un être)\\
        (\textbf{o})&&\textbf{o}
      \end{tabular}}}
  \\[1ex]

  On voit que toutes les idées générales contenues dans «chandel» sont
  les mêmes que celles contenues dans «ier», et pourtant le mot
  «chandelier» ne peut pas être simplifié; il est irréductible, parce
  que les deux atomes (\textbf{ô}) différent par les idées }
%
{\noindent
  %
  specific [noun root] (\textbf{ô}); the atom \emph{fère} (which
  bears) is a specific adjective suffix (\textbf{â}).

  This molecule contains no pleonasm, and is immediately analyzed as follows:\\[1ex]

  {\setlength{\tabcolsep}{0pt}
    \begin{tabular}[t]{cccc}
      \emph{mammi}&\makebox[2em][l]{---}&\emph{fère}&= \textbf{(ô-â)}\\
      (object)&&\textbf{a}\\
      (concrete entity)\\
      (an entity)\\
      \textbf{o}
    \end{tabular}}
  \\[1ex]

  The word \emph{mammifère} is thus an adjective, and means, in
  dissociated form, ``(which bears) --- (the object-entity
  \emph{teat})''.

  4. \emph{Analysis of the word \emph{chandelier} `candelabra'}.---
  The atom \emph{chandel} `candle' is a specific noun root
  (\textbf{ô}); the atom \emph{ier} is a specific nominal suffix
  (\textbf{ô}) which means ``object
  that holds''. We thus have\\[1ex]

  \noindent
  {\setlength{\tabcolsep}{0pt} {\small
      \begin{tabular}[t]{cccc}
        \emph{chandel}&---&\emph{ier}&= \textbf{(ô-ô)}\\
        (object)&&(object)\\
        (concrete entity)&&(concrete entity)\\
        (an entity)&&(an entity)\\
        (\textbf{o})&&\textbf{o}
      \end{tabular}}}
  \\[1ex]

  We see that all of the general ideas contained in \emph{chandel} are
  the same as those contained in \emph{ier}, but nonetheless the word
  \emph{chandelier} cannot be simplified: it is irreducible, because the
  two (\textbf{ô}) atoms differ in terms of the [specific] ideas

}

\TextPage{\noindent
  %
  particulières («chandelle», «ier») qu’ils représentent. En
  dissociant les atomes, on a: «chandelier» = (être-objet qui
  \emph{porte}) — (l’être-objet \emph{chandelle}).

  5. \emph{Analyse du mot «héroïne»}. Ce mot est aussi du type
  (\textbf{ô-ô}), comme le précédent. Les atomes (\textbf{ô})
  représentent seulement des personnes au lieu d’objets. Le suffixe
  «ine» (dans héroïne) ou «esse» (dans «princesse»), désigne en effet
  une «personne du sexe
  féminin»:\\[1ex]

  \noindent
  {\setlength{\tabcolsep}{0pt} {\small
      \begin{tabular}[t]{cccc}
        \emph{héro}&{---}&\emph{ïne}&=\textbf{(ô-ô)}\\
        (personne)&&(personne)\\
        (un être concret)&&(un être concret)\\
        (un être)&&(un être)\\
        (\textbf{o})&&(\textbf{o})
      \end{tabular}}}
  \\[1ex]

  D’où, en dissociant: «héroïne» = (personne du sexe \emph{féminin}) —
  (du type \emph{héros}).

  Si, au lieu du mot «héroïne», nous prenons le mot «matronine», nous
  aurons l’analyse suivante: \\[1ex]

  \noindent
  {\setlength{\tabcolsep}{0pt} {\small
      \begin{tabular}[t]{cccc}
        \emph{matron}&{---}&\emph{ine}&\hspace{-1em}=\textbf{(ô-ô)}\\
        (personne féminine) &&(personne)\\
        (personne)&&(un être concret)\\
        (un être concret)&&(un être)\\
        (un être)&&(\textbf{o})\\
        (\textbf{o})
      \end{tabular}}}
  \\[1ex]

  Cette analyse montre que non seulement toutes les idées générales
  sous-entendues dans le suffixe «ine» sont les mêmes que celles qui
  sont sous-entendues dans l’atome «matrone», mais encore que


}
%
{\noindent
  %
  specific [ideas] that they represent (\emph{chandelle} `candle',
  \emph{ier}). By dissociating the atoms, we have \emph{chandelier} =
  (object-entity which \emph{holds}) --- (object-entity
  \emph{candle}).

  5. \emph{Analysis of the word \emph{héroïne} `heroine'}. --- This word is
  also of the type (\textbf{ô-ô}), like the preceding one. The
  (\textbf{ô}) atoms merely represent persons instead of
  objects. The suffix \emph{ine} (in héroïne) or \emph{esse} (in
  \emph{princesse}) actually designates a ``person of the feminine sex'':\\[1ex]

  \noindent
  {\setlength{\tabcolsep}{0pt} {\small
      \begin{tabular}[t]{cccc}
        \emph{héro}&{---}&\emph{ïne}&=\textbf{(ô-ô)}\\
        (person)&&(person)\\
        (concrete entity)&&(concrete entity)\\
        (an entity)&&(an entity)\\
        (\textbf{o})&&(\textbf{o})
      \end{tabular}}}
  \\[1ex]

  From which, by dissociating, \emph{héroïne} = (person of sex
  \emph{feminine}) --- (of type \emph{hero}).

  If instead of the word ``héroïne'' we take the word
  \emph{matronine}, we would have the following analysis: \\[1ex]

  \noindent
  {\setlength{\tabcolsep}{0pt} {\small
      \begin{tabular}[t]{cccc}
        \emph{matron}&{---}&\emph{ine}&\hspace{-1em}=\textbf{(ô-ô)}\\
        (person feminine) &&(person)\\
        (person)&&(concrete entity)\\
        (concrete entity)&&(an entity)\\
        (an entity)&&(\textbf{o})\\
        (\textbf{o})
      \end{tabular}}}
  \\[1ex]

  This analysis shows that not only all of the general ideas
  assumed by the suffix \emph{ine} are the same as those which are
  understood by the atom \emph{matron}, but also that

}

\TextPage{\noindent
  %
  l’idée \emph{particulière} «ine» (personne du sexe \emph{féminin})
  est elle-même sous-entendue dans l’idée particulière «matrone». Dès
  lors, le mot «matronine» contient un pléonasme inutile; l’atome
  «ine» est déjà implicitement et totalement contenu dans l’atome
  «matrone»; on peut donc le supprimer et réduire «matronine» à
  «matrone». Du reste, le mot «matronine» n’existe pas en français; en
  effet, l'usage même de la langue empêche l’introduction de suffixes
  inutiles, et si un mot nouvellement créé contenait un pareil
  suffixe, le principe du moindre effort aurait tôt fait de le faire
  disparaître.

  Evidemment, \emph{tout suffixe doit introduire dans le mot auquel on
    l'accole une idée} (générale ou particulière) \emph{qui n'y était
    pas encore contenue}; ainsi, le suffixe «eux», nécessaire dans
  «glorieux», est inutile dans «grandiose», car «os» =
  «eux\footnote{On peut constater que «ose» = «eux» dans les mots
    «nébul-eux», «nébul-os-ité».}» = idée adjective \textbf{a}; donc:

  \begin{center}
    \emph{glor-ieux} = \textbf{glor-a} (irréductible),
  \end{center}

  \noindent
  tandis que:\\[1ex]
  
  \noindent
  {\small \emph{grand-iose}\footnote{«Grandiose» n’est égal à «grand»
      qu'au point de vue logique. En réalité, dans «grandiose», le
      suffixe a un sens particulier (augmentatif), comme le suffixe
      «lich», dans «süsslich», a un sens particulier (diminutif).} =
    \emph{grand-eux} = \textbf{grand-a} = \textbf{grand}, }

}
%
{\noindent
  %
  the \emph{specific} idea \emph{ine} (person of sex \emph{feminine}) is
  itself understood in the specific idea \emph{matrone}
  `matron'. Therefore the word \emph{matronine} contains a useless
  pleonasm: the atom \emph{ine} is already implicitly and totally
  contained in the atom \emph{matrone}; we can thus eliminate it and
  reduce \emph{matonine} to \emph{matrone}. Moreover, the word \emph{matronine}
  does not exist in French: indeed, the very usage of the language
  prevents the introduction of useless suffixes, and if a newly
  created word contained such a suffix, the principle of least effort
  would soon have made it disappear.

  Apparently, \emph{every suffix must introduce into the word to which
    it is attached an idea} (general or specific) \emph{which was not
    already contained in it.} Thus, the suffix \emph{eux}, necessary
  in \emph{glorieux}, is useless in \emph{grandiose}, since \emph{ose}
  = \emph{eux}\footnote{We can establish that \emph{ose} = \emph{eux}
    in the words \emph{nébul-eux} `cloudy', \emph{nébul-os-ité}
    `cloudiness'.} = adjectival idea \textbf{a}; thus:

  \begin{center}
    \emph{glor-ieux} = \textbf{glor-a} (irreducible),
  \end{center}

  \noindent
  while:\\[1ex]
  
  \noindent
  {\small \emph{grand-iose}\footnote{\emph{Grandiose} is only equal to
      \emph{grand} from the logical point of view. Actually, in
      \emph{grandiose}, the suffix has a specific sense
      (augmentative), just as the suffix \emph{lich} in
      \emph{süsslich}, has a specific (diminutive) sense.} =
    \emph{grand-eux} = \textbf{grand-a} = \textbf{grand}, }

}

\TextPage{\noindent
  %
  car l’idée adjective \textbf{a} n’est pas contenue dans le
  substantif «gloire», mais elle l’est déjà dans l’adjectif «grand».

  Ces remarques nous amènent à étudier les lois de la \emph{synthèse}
  des mots composés, problème inverse de celui que nous venons de
  traiter jusqu’ici. Ces lois de synthèse sont particulièrement utiles
  pour les savants et les techniciens, qui forgent souvent des mots
  nouveaux plus ou moins bien construits.  \vspace{1.5in}
  \begin{center}
    \rule{0.25\textwidth}{0.4pt}
  \end{center}

}
%
{\noindent
  %
  since the adjectival idea \textbf{a} is not contained in the noun
  \emph{gloire} `glory', but it is so in the adjective \emph{grand}.

  These remarks lead us to study the laws of \emph{synthesis} in
  compound words, the inverse problem of that which we have just been
  treating up to this point. These laws of synthesis are quite useful
  for scientists and technicians, who often build new words that are
  more or less well constructed.\vspace{1.5in}
  \begin{center}
    \rule{0.25\textwidth}{0.4pt}
  \end{center}
}

\TextPage{\vspace{1in}
  \begin{center}
    {\large CHAPITRE II}
  \end{center}
  % \vspace{1ex}
  \begin{center}
    \rule{3em}{0.4pt}
  \end{center}
  \vspace{2ex}
  \begin{center}
    {\large\textbf{SYNTHÈSE DES MOTS}}
  \end{center}
  \addcontentsline{toc}{section}{2 Synthèse des mots}

  Le problème à résoudre pour pouvoir effectuer la synthèse des mots
  est l’inverse de celui que nous avons étudié dans le chapitre
  premier. On peut l’énoncer comme suit:

  \emph{Etant donnée une idée complexe, construire le mot composé qui
    représente cette idée}, c’est-à-dire trouver la combinaison de
  radicaux et d’affixes qui évoquera cette idée.

  Nous avons déjà dit que le tout étant l'ensemble de ses parties,
  l’idée totale évoquée par un mot composé est l’ensemble, ou, si l'on
  veut, la résultante des idées partielles évoquées par les
  différentes parties de ce mot. Donc, réciproquement, pour
  représenter par un mot composé une idée donnée, il faut introduire
  dans ce mot (au moyen de radicaux et d’affixes), toutes les idées
  partielles contenues dans l’idée totale à représenter.

}
%
{\vspace{1in}
  \begin{center}
    {\large CHAPTER II}
  \end{center}
  % \vspace{1ex}
  \begin{center}
    \rule{3em}{0.4pt}
  \end{center}
  \vspace{2ex}
  \begin{center}
    {\large\textbf{SYNTHESIS OF WORDS}}
  \end{center}

  The problem that must be resolved in order to carry out the
  synthesis of words is the inverse of that which we have studied in
  the first chapter.  We could formulate it as follows:

  \emph{Given a complex idea, construct the compound word that
    represents that idea}, that is, find the combination of roots and
  affixes that will evoke that idea.

  We have already said that the whole being the sum of its parts, the
  total idea evoked by a compound word is the set, or if you wish the
  result of the partial ideas evoked by the different parts of this
  word. Thus, reciprocally, to represent a given idea by a compound
  word, it is necessary to introduce into the word (by means of roots
  and affixes) all of the partial ideas contained in the total idea to
  be represented.

}

\TextPage{Mais, énoncée sous cette forme, la solution du problème
  n’aurait aucune valeur pratique, car: 1° une idée complexe contient
  une quantité d'idées partielles que l’on ne peut pas toutes
  énumérer; 2° quoique le nombre des mots simples soit considérable,
  le nombre des affixes est assez restreint, et l’on ne peut pas
  toujours exprimer exactement par un affixe l'idée partielle que l’on
  voudrait exprimer. Pour ces deux raisons, l’idée totale n’est pas
  toujours exprimable exactement au moyen de radicaux et
  d’affixes. Dans ce cas, il faut se contenter d’une solution
  approchée et enfermer l’idée à exprimer entre deux limites aussi
  rapprochées que possible de cette idée, à l’instar des
  mathématiciens qui, ne pouvant pas représenter exactement les
  quantités incommensurables par un nombre, enferment ces quantités
  entre deux limites commensurables aussi rapprochées que possible
  l’une de l’autre.

  On est donc conduit, afin de prévoir tous les cas possibles, à poser
  les deux principes suivants qui s’opposent et se complètent
  mutuellement et qui forment la base logique de la synthèse des mots:

  1. \textsc{Principe de nécessité}: \emph{Dans la formation d’un mot
    composé, il faut introduire (au moyen de radicaux et d'affixes)
    tous les éléments nécessaires pour évoquer clairement et
    complètement l'idée que ce mot doit représenter.}

}
%
{However, set out in that form, the solution to the problem would have
  no practical value, since 1. a complex idea contains so many partial
  ideas that one could not enumerate them all; 2. although the number
  of simple words is considerable, the number of affixes is rather
  limited, and one cannot always express through an affix exactly the
  partial idea that one would like to express. For these two reasons,
  the total idea is not always exactly expressible by means of roots
  and affixes. In that case, it is necessary to be content with an
  approximate solution, and to confine the idea to be expressed
  between two limits as close as possible to that idea, following the
  example of mathematicians who, not being able to represent
  irrational quantities exactly by a number, confine these quantities
  between two rational limits as close as possible to one another.

  We are thus led, in order to anticipate all possible cases, to set
  down the two following opposed and complementary principles which
  form the logical basis of the synthesis of words:

  1. \textsc{Principle of necessity}: \emph{In the formation of a
    compound word, it is necessary to introduce (by means of roots and
    affixes) all elements necessary to evoke clearly and completely
    the the idea that this word should represent.}

}

\TextPage{2. \textsc{Principe de suffisance}: \emph{Il ne faut pas
    répéter (sans nécessité) plusieurs fois la même idée dans le même
    mot, et il ne faut pas y introduire des idées étrangères non
    contenues dans l'idée totale à exprimer.}

  Lorsqu’un mot composé est construit conformément à ces deux
  principes, on est sûr que: 1° chaque idée partielle, nécessaire pour
  évoquer l’idée totale, est contenue dans quelque partie du mot
  composé qui exprime cette idée totale (ainsi l’idée totale exprimée
  par le mot «humanité» contient une idée partielle qualificative,
  laquelle se trouve contenue dans l’atome adjectif «an»); 2° toute
  idée contenue dans un élément du mot composé est une idée partielle
  nécessairement contenue dans l’idée totale représentée par ce mot
  (ainsi l’atome «hum» de «humanité» contenant, par exemple, l'idée de
  «personne», cette idée de personne se retrouve forcément aussi dans
  le mot «humanité»); 3° aucune idée n’est exprimée (sans nécessité),
  plus d’une fois.

  En résumé, \emph{le sens d'un mot ne dépend que de son propre
    contenu et de tout son contenu}, et non pas de la manière dont on
  peut supposer ce mot dérivé d’un autre; à condition, bien entendu,
  que l’on connaisse (par le classement des atomes de la page 40) la
  nature grammaticale de chacun des atomes dont ce mot est composé.

}
%
{2. \textsc{Principle of sufficiency}: \emph{It is necessary not to
    repeat (unnecessarily) the same idea multiple times in the same
    word, and it is necessary not to introduce alien ideas not
    contained in the total idea to be expressed.}

  When a compound word is constructed in conformity with these two
  principles, we can be sure that 1. each partial idea, necessary to
  evoke the total idea, is contained in some part of the compound word
  which expresses that total idea (thus the total idea expressed by
  the word \emph{humanité} contains a qualifying partial idea, which is
  found in the adjective atom \emph{an}); 2. every idea contained in an
  element of the compound word is a partial idea that is necessarily
  contained in the total idea represented by the word (thus since the
  atom \emph{hum} in \emph{humanité}, contains for example the idea of
  ``person'', that idea of person is necessarily found also in the
  word \emph{humanité}): 3. No idea is expressed (unnecessarily) more
  than once.

  In sum, \emph{the meaning of a word depends only on its own content,
    and on all of its content}, and not on the manner in which one
  supposes this word to be derived from another; on condition, of
  course, that we know (from the classification of atoms on page 40)
  the grammatical nature of each of the atoms of which the word is
  composed.

}

\TextPage{Pour construire le mot composé représentant une idée
  complexe donnée, le moyen le plus simple est d'exprimer d’abord
  cette idée complexe sous forme analytique, au moyen de plusieurs
  mots: si l’on considère alors cette définition analytique de l'idée
  comme une molécule dissociée, il suffit pour obtenir le mot cherché
  de condenser cette molécule en en expulsant (grâce au principe de
  suffisance) les pléonasmes qu’elle contient presque toujours, et en
  appliquant la loi du renversement des atomes.

  Deux cas peuvent se présenter suivant que l’idée à représenter est,
  ou non, exprimable au moyen des radicaux et suffixes dont dispose la
  langue. Dans le premier cas, le problème a une solution exacte, dans
  le second cas la solution n’est qu’approchée.

  \textsc{Exemples}: 1. Prenons d’abord comme exemple l’idée complexe
  représentée analytiquement par le groupe de mots: \emph{action}
  «\emph{d'écrire}». Si l’on remplace chaque atome par son équivalent
  symbolique («ac» = \textbf{i}, «tion» = \textbf{o}, «écri» =
  \textbf{î}, «re» = \textbf{i}) on voit que:

  \begin{center}
    «(ac-tion) d’(écrire)» = \textbf{(i-o)} — \textbf{(î-i)}
  \end{center}


  L’idée complexe en question est ainsi mise sous forme d’une molécule
  dissociée à deux éléments dont chacun est biatomique; mais cette
  molécule contient encore plusieurs pléonasmes inutiles: d’abord la
  molécule \textbf{(î-i)} se réduit à l’atome \textbf{(î)} ou

}
%
{To construct the compound word representing a given complex idea, the
  simplest means is first to express this complex idea in analytic
  form, by means of several words: if one then considers this analytic
  definition of the idea as a dissociated molecule, it suffices to
  obtain the word sought by condensing this molecule while expelling
  from it (thanks to the principle of sufficiency) the pleonasms that
  it almost always contains, and applying the law of the reversal of
  atoms.

  There can be two cases, depending on whether the idea to be
  represented is or is not expressible by means of the roots and
  suffixes that the language has. In the first case, the problem has
  an exact solution; in the second case, the solution can only be
  approximated.

  \textsc{Examples}: 1. Let us first take as an example the complex
  idea represented by the group of words \emph{action ``d'écrire''}
  `\emph{action ``to write''}'. If we replace each atom by its
  symbolic equivalent (``ac'' = \textbf{i}, ``tion'' = \textbf{o},
  \emph{écri} = \textbf{î}, \emph{re} = \textbf{i}), we see that:
  
  \begin{center}
    ``(ac-tion) d’(écri-re)'' = \textbf{(i-o)} — \textbf{(î-i)}
  \end{center}

  The complex idea in question is thus put into the form of a
  dissociated molecule with two elements, each of which is
  bi-atomic. However this molecule also contains several unnecessary
  pleonasms: first, the molecule \textbf{(î-i)} reduces to the atom
  \textbf{(î)} or

}

\TextPage{\noindent
  %
  «écri», car l’idée verbale générale \textbf{i} est déjà contenue
  dans l’idée verbale particulière «écri» \textbf{(î)}; en d’autres
  mots le mot «écri-re» \textbf{(î-i)} contient, au point de vue
  logique, un pléonasme analogue à celui du mot «grand-iose»
  \textbf{(â-a)}, et l’on a: «écrire» = «écri», comme «grandiose» =
  «grand». Si, maintenant, après avoir chassé ce premier pléonasme, on
  condense la molécule dissociée (en renversant l’ordre de ses
  éléments), on aura:

  \noindent
  {\small
    \resizebox{\linewidth}{!}{\begin{tabular}[t]{rcl}
      (ac-tion) d’(écri-re)&=& \textbf{(i-o)} — \textbf{(î)}\\
                           &=&\textbf{(î-i-o)} = (écri-ac-tion).
    \end{tabular}}
  }

  Sous cette forme condensée nous voyons qu’il reste encore un
  pléonasme, car l'idée verbale générale \textbf{i} (ou «ac») est
  contenue déjà dans l’atome précédent \textbf{î} (ou «écri»). On peut
  donc supprimer «ac» et il reste:

  \begin{center}
    (action) d’(écrire) = (écri-tion) = \textbf{(î-o)}
  \end{center}

  \noindent
  ou en remplaçant l’atome «tion» par l’atome synonyme «ture»:

  \begin{center}
    (action) d’(écrire) = (écri-ture) = \textbf{(î-o)}.
  \end{center}

  2. Prenons encore comme exemple l’idée complexe représentée
  analytiquement par le groupe de mots: \emph{la qualité}
  «\emph{grand}». Cette expression est aussi une molécule dissociée à
  deux éléments, dont

}
%
{\noindent
  %
  \emph{écri}, because the general verbal idea \textbf{i} is already
  contained in the specific verbal idea \emph{écri} \textbf{(î)}: in
  other words, the word \emph{écri-re} \textbf{(î-i)} contains, from
  the logical point of view, a pleonasm analogous to that of the word
  \emph{grand-iose} \textbf{(â-a)}, and we have \emph{écrire} = \emph{écri}
  like \emph{grandiose} = \emph{grand}. If now, having gotten rid of this
  first pleonasm, we condense the dissociated molecule (reversing the
  order of its elements), we will have:

   \noindent
   {\small
     \resizebox{\linewidth}{!}{\begin{tabular}[t]{rcl}
       (ac-tion) d’(écri-re)&=& \textbf{(i-o)} — \textbf{(î)}\\
                            &=&\textbf{(î-i-o)} = (écri-ac-tion).
     \end{tabular}}
   }

   In this condensed form, we see that another pleonasm remains, since
   the general verbal idea \textbf{i} (or ``ac'') is already contained
   in the preceding atom \textbf{î} (or \emph{écri}).  We can thus
   eliminate ``ac'', and what remains is:

 \begin{center}
   (action) d’(écrire) = (écri-tion) = \textbf{(î-o)}
 \end{center}

  \noindent
  or replacing the atom \emph{tion} by the synonymous atom \emph{ture}:

  \begin{center}
    (action) d’(écrire) = (écri-ture) = \textbf{(î-o)}.
  \end{center}

  2. Let us take now as an example the complex idea represented
  analytically by the group of words \emph{la qualité}
    ``\emph{grand}''. This expression is also a dissociated molecule with
  two elements, of which

}

\TextPage{\noindent
  %
  l’un est une molécule biatomique, et l’autre un simple atome. On a,
  en effet:

  \begin{center}
    (qual-ité) — (grand) = \textbf{(a-o)} — \textbf{(â)}
  \end{center}

  \noindent
  d’où en condensant et renversant les éléments:\\[1ex]

  \noindent
  \resizebox{\linewidth}{!}{\begin{tabular}[t]{rcl}
    (qual-ité) — (grand)&=&\textbf{(â-a-o)}\\
                        &=&(grand-qual-ité)
  \end{tabular}}\\[1ex]

  \noindent
  mais comme l’idée générale \textbf{a} ou «qual» est déjà contenue
  dans «grand» \textbf{(â)}, on peut la supprimer, et il reste:

  \begin{center}
    (qual-ité) — (grand) = (grand-ité) = \textbf{(â-o)}.
  \end{center}

  \noindent
  ou en remplaçant l’atome «ité» par son synonyme «eur»:

  \begin{center}
    (qual-ité) — (grand) = (grand-eur)
  \end{center}

  \noindent
  ou:

  \begin{center}
    \textbf{(a-o)} — \textbf{(â)} = \textbf{(â-o)}.
  \end{center}


  3. Si, au lieu de l’idée «qualité \emph{grand}» nous prenons
  «\emph{qualité humain}», nous aurons une molécule dissociée à deux
  éléments, dont chacun est une molécule bi-atomique, car l’atome
  «hom» désignant un être concret \textbf{(o\sxsub{1})}, et l’atome
  «ain» l'idée adjective \textbf{(a)}, on a:

  \noindent
  {\setlength{\tabcolsep}{2pt}
    \resizebox{\linewidth}{!}{\begin{tabular}[t]{rcl}
      (qual-ité) — (hum-ain)&=&\textbf{(a-o)} — \textbf{(ô\sxsub{1}-a)}\\
                            &=&\textbf{(ô\sxsub{1}-a-a-o)}\\
                            &=&(hum-an-qual-ité)
    \end{tabular}}}


}
%
{\noindent
  %
  one is a bi-atomic molecule and the other a simple atom. We have,
  indeed:

  \begin{center}
    (qual-ité) — (grand) = \textbf{(a-o)} — \textbf{(â)}
  \end{center}

  \noindent
  or by condensing and reversing the elements:\\[1ex]

  \noindent
  \resizebox{\linewidth}{!}{\begin{tabular}[t]{rcl}
    (qual-ité) — (grand)&=&\textbf{(â-a-o)}\\
                        &=&(grand-qual-ité)
  \end{tabular}}\\[1ex]

  \noindent
  but since the general idea \textbf{a} or «qual» is already contained
  in \emph{grand} \textbf{(â)}, we can eliminate it, and there remains:

  \begin{center}
    (qual-ité) — (grand) = (grand-ité) = \textbf{(â-o)}.
  \end{center}

  \noindent
  or, replacing the atom \emph{ité} by its synonym \emph{eur}:

  \begin{center}
    (qual-ité) — (grand) = (grand-eur)
  \end{center}

  \noindent
  or:

  \begin{center}
    \textbf{(a-o)} — \textbf{(â)} = \textbf{(â-o)}.
  \end{center}

  3. If instead of the idea ``qualité \emph{grand}'' we take
  ``\emph{qualité humain}'', we will have a dissociated molecule
  with two elements, each of which is a bi-atomic molecule. Because
  the atom ``hom'' designates a concrete entity \textbf{(o\sxsub{1})},
  and the atom \emph{ain} the adjectival idea \textbf{(a)}, we have:

  \noindent
  {\setlength{\tabcolsep}{2pt}
    \resizebox{\linewidth}{!}{\begin{tabular}[t]{rcl}
      (qual-ité) — (hum-ain)&=&\textbf{(a-o)} — \textbf{(ô\sxsub{1}-a)}\\
                            &=&\textbf{(ô\sxsub{1}-a-a-o)}\\
                            &=&(hum-an-qual-ité)
    \end{tabular}}}


}

\TextPage{\noindent
  %
  ou en supprimant le pléonasme causé par la présence de deux atomes
  \textbf{a} identiques:\\[1ex]

  \noindent
  {\small
    (qual-ité) — (hum-ain)=\textbf{(ô,-a-o)}=(hum-an-ité)}\\[1ex]

  4. Synthèse de l’idée complexe: «\emph{objet fait pour porter une
    chandelle}». Cette synthèse est très simple, car il se trouve
  qu’il existe en français un suffixe qui exprime précisément l’idée
  «objet qui porte»: ce suffixe est le suffixe substantif «ier». On a
  donc:\\[1ex]

  \noindent
  {\setlength{\tabcolsep}{0pt} \small
    \resizebox{\linewidth}{!}{\begin{tabular}[t]{rcl}
      (objet qui porte) - (chandelle)&=&(ier) - (chandelle)\\
                                     &=&\textbf{(ô)} — \textbf{(ô)}
    \end{tabular}}}\\[1ex]

  \noindent
  ou en condensant et renversant l’ordre des atomes:\\[1ex]

  \noindent
  {\small
    (objet qui porte) - (chandelle) = (chandel-ier).}\\[1ex]

  5. Synthèse de l’idée complexe: «\emph{mettre une couronne sur la
    tête de} (quelqu'un)». Il n’existe pas de suffixe pour exprimer
  l'idée particulière: «prendre un objet et le fixer sur un autre
  objet». Nous nous trouvons donc dans le cas où le problème de la
  synthèse n’est pas susceptible d’une solution exacte; dans ce cas il
  faut enfermer la solution entre deux autres aussi rapprochées que
  possible l’une de l’autre. L’une de ces solutions sera la solution
  par \emph{défaut} et l’autre, la solution par \emph{excès}, car la
  solution exacte étant comprise entre les deux solutions approchées,
  il faut nécessairement que l’une de celles-ci ne contienne pas
  toutes les idées exprimées par

}
%
{\noindent
  %
  or, eliminating the pleonasm produced by the presence of two
  identical atoms \textbf{a}:\\[1ex]

  \noindent
  {\small
    (qual-ité) — (hum-ain)=\textbf{(ô,-a-o)}=(hum-an-ité)}\\[1ex]

  4. Synthesis of the complex idea ``\emph{objet fait pour porter une
    chandelle}'' `\emph{object made to hold a candle}'. This synthesis
  is very simple, because it happens that there exists in French a
  suffix which expresses precisely the idea ``object that holds'':
  this is the nominal suffix \emph{ier}. Thus we have:\\[1ex]

  \noindent
  {\setlength{\tabcolsep}{0pt} \small
    \resizebox{\linewidth}{!}{\begin{tabular}[t]{rcl}
      (objet qui porte) - (chandelle)&=&(ier) - (chandelle)\\
                                     &=&\textbf{(ô)} — \textbf{(ô)}
    \end{tabular}}}\\[1ex]

  \noindent
  or in condensing and reversing the order of atoms:\\[1ex]

  \noindent
  {\small
    (objet qui porte) - (chandelle) = (chandel-ier).}\\[1ex]

  5. Synthesis of the complex idea ``\emph{mettre une couronne sur la
    tête de} (quelqu'un)'' `\emph{to put a crown on the head of}
  (someone)'. There is no suffix that expresses the specific idea ``to
  take an object and place it on another object''. Thus we find
  ourselves in the situation where the problem of synthesis is not
  amenable to an exact solution; in that case it is necessary to
  confine the solution between two others as close as possible to one
  another. One of these solutions will be the solution by
  \emph{rounding down}, and the other, the solution by \emph{excess}.  Since
  the exact solution is found between the two approximate solutions,
  one of these necessarily does not contain all of the ideas expressed
  by

}

\TextPage{\noindent
  %
  l’idée totale, tandis que l’autre, au contraire, contiendra des
  idées en excès, c’est-à-dire des idées étrangères non contenues dans
  l’idée totale à exprimer.

  Dans l’exemple ci-dessus, l’idée «prendre un objet et le fixer sur
  un autre objet» contient avant tout l’idée de «faire une action»,
  laquelle idée est représentée par le suffixe verbal général «er». On
  peut donc traduire d’une manière approximative l’expression «mettre
  une couronne sur la tête de» par le mot «couronn-er», et cette
  solution constitue une solution par défaut, puisque les atomes
  «couronn» et «er» ne contiennent pas en eux-mêmes toutes les idées
  qu’il s’agissait d’exprimer; le mot «couronn-er» ne signifie, en
  effet, que «faire l’action relative à une \emph{couronne} ou
  caractérisée par une \emph{couronne}».

  Au contraire, si l’on veut spécifier la nature de l’action faite sur
  l’objet «couronne», on doit, faute de suffixe ou de radical
  approprié, se contenter des mots simples existant, comme, par
  exemple, le verbe «garnir». On arrive alors à représenter l’idée
  «mettre une couronne sur la tête de (quelqu’un)» par le mot composé
  «couronn-garnir» en admettant que l’on puisse employer le
  mot-radical «garn» à la manière d'un suffixe\footnote{On ne peut pas
    employer ici le suffixe «ifi», car le mot «couronn-ifi-er»
    signifierait «rendre couronne», «transformer en couronne».}; mais
  cette solution est

}
%
{\noindent
  %
  the total idea, while the other, on the contrary, contains ideas in
  excess, that is, alien ideas not contained in the total idea to be
  expressed.

  In the example above, the idea ``to take an object and place it on
  another object'' contains above all the idea ``to perform an
  action'', which is represented by the general verbal suffix
  \emph{er}.  We can thus translate in an approximate way the
  expression ``to put a crown on the head of'' by the word
  \emph{couronn-er}, and this solution constitutes a solution by
  rounding down, since the atoms \emph{couronn} and \emph{er} do not
  contain in themselves all of the ideas which are supposed to be
  expressed: the word \emph{couronn-er} only means, actually, ``to
  perform an action relative to a \emph{crown} (\emph{couronne}), or
  characterized by a \emph{crown}''.

  On the other hand, if one wants to specify the nature of the action
  performed on the object ``crown'', we must, in the absence of an
  appropriate suffix or root, be content with existing simple words,
  such as for example the verb \emph{garnir} `to decorate'. Then we
  get to representing the idea ``to put a crown on the head of
  (someone)'' by the compound word \emph{couronn-garnir} if we allow
  the root word \emph{garn} to be used as a suffix\footnote{We cannot
    use the suffix \emph{ifi} here, because the word
    \emph{couronn-ifi-er} would mean ``to make (something) a crown'',
    ``to transform into a crown''.}; but this solution is

}

\TextPage{\noindent
  %
  évidemment une solution approchée par excès, car l'idée «garnir»
  contient des idées étrangères qui ne sont pas contenues dans l’idée
  totale à exprimer.

  Est-ce à dire que dans la pratique, tous les cas de cette nature (et
  ils sont nombreux) comportent deux solutions également acceptables?
  Certes non, car un rapide examen des deux solutions nous conduit à
  poser le principe suivant: \emph{Entre la solution par défaut et
    celle par excès il faut toujours choisir la solution par défaut.}

  Ce principe est justifié par celui du moindre effort; en effet, des
  deux solutions, par défaut et par excès, celle par défaut exige un
  moindre effort, puisqu'elle correspond à une description incomplète
  de l’idée à exprimer; on peut la comparer à un tableau qui ne serait
  pas complètement achevé. Du reste, le principe du moindre effort est
  conforme à la pratique de toutes les langues, et l’on dit
  «couronn-er» aussi bien en anglais (\emph{to crown}) qu’en allemand
  (\emph{krön-en}\footnote{En allemand le tréma n'indique que la
    verbification du radical \emph{Kron}; ce tréma fait double emploi
    avec la finale verbale \emph{en}.}).

  Au contraire, aucune langue n’admet les solutions par excès, et cela
  non pas seulement parce qu’elle coûte un effort plus grand, mais
  aussi parce qu’elle pourrait donner lieu à des erreurs
  d’interprétation,

}
%
{\noindent
  %
  obviously an approximate solution by excess, because the idea
  \emph{garnir} contains alien ideas that are not contained in the total
  idea to be expressed.

  Does this mean that, in practice, all cases of this nature (and they
  are many) have two equally acceptable solutions? Certainly not,
  because a quick examination of the two solutions leads us to posit
  the following principle: \emph{Between the solution by rounding down
    and that by excess, it is always necessary to choose the solution
    by rounding down.}

  This principle is justified by that of least effort: actually, of
  the two solutions, by rounding down and by excess, the one by rounding down
  requires less effort, since it corresponds to an incomplete
  description of the idea to be expressed.  We can compare it to a
  painting which is not completely finished. Furthermore, the
  principle of least effort is in agreement with the practice of all
  languages, and we say \emph{couronn-er} as well in English (\emph{to
    crown}) and in German (\emph{krön-en}\footnote{In German, the
    umlaut indicates only the verbalization of the root \emph{Kron};
    this umlaut does double duty with the verbal final \emph{en}.}).

  On the other hand, no language allows solutions by excess, and that
  not only because this would require a greater effort, but also
  because it could give rise to errors of interpretation,

}

\TextPage{\noindent
  %
  dues à ce fait que toute solution par excès contient forcément des
  idées étrangères à l’idée à exprimer; en effet, si elle ne contenait
  pas d’idées étrangères, ce ne serait pas une solution par excès, ce
  serait ou bien une solution exacte, ou bien une solution par
  défaut. Supposons un instant que l’on dise «couronn-garnir», au lieu
  de «couronn-er», le substantif dérivé de ce verbe serait:
  «couronn-garni-ture», au lieu de «couronn-e-ment», ce qui créerait
  une confusion entre les deux idées distinctes: «action de couronner»
  et «garniture de couronnes».

  6. — De même pour représenter l'idée: «\emph{fixer un objet sur un
    autre au moyen de colle}», comme nous venons de voir qu’il
  n’existe pas de suffixe pour exprimer exactement l’idée «fixer un
  objet sur un autre objet», on adoptera la solution par défaut
  «coll-er», solution qui signifie littéralement «faire l’action
  relative à la \emph{colle}, caractérisée par la \emph{colle}», et
  l’on rejettera la solution par excès (colle-garnir) parce qu’elle
  enfreint le principe du moindre effort, et parce que le mot
  «colle-garnir» contient des idées étrangères, attendu que «garnir de
  colle» (forme dissociée de «colle-garnir») n’est pas la même chose
  que «fixer au moyen de colle» (coller); en effet, un timbre-poste,
  par exemple, peut être mal «collé» sur une enveloppe, tout en étant
  bien «garni, enduit de colle».

}
%
{\noindent
  %
  due to the fact that every solution by excess necessarily contains
  ideas alien to the idea to be expressed; indeed, if it did not
  contain alien ideas, it would not be a solution by excess, it would
  be either an exact solution or else a solution by rounding
  down. Supposing for a moment that one said \emph{couronn-garnir}
  instead of \emph{couronn-er}, the noun derived from this verb would
  be \emph{couronn-garni-ture} instead of \emph{couronn-e-ment}, which
  would create confusion between the two distinct ideas ``action of
  crowning'' and ``decoration of crowns''.

  6. --- The same goes for representing the idea ``\emph{fixer un
    objet sur un autre au moyen de colle}'' `to attach an object to
  another by means of glue'. Since as we have just seen, there is no
  suffix that represents exactly the idea ``to attach an object to
  another object'', we will adopt here the solution by rounding down
  \emph{coll-er}, a solution which literally means ``to perform an action
  relative to \emph{glue} (\emph{colle}), characterized by \emph{glue}'', and
  we will reject the solution by excess (colle-garnir), because that
  would infringe the principle of least effort, and because the word
  \emph{colle-garnir} contains alien ideas, given that ``garnir de
  colle'' `to decorate with glue' (the dissociated form of
  \emph{colle-garnir}) is not the same thing as ``to attach by means of
  glue''; indeed, a postage stamp, for example, can be badly
  \emph{collé} `glued' to an envelope, while being quite well ``garni,
  enduit de colle'' `decorated, coated with glue'.

}

\TextPage{7. — Synthèse de l'idée «\emph{personne dont l’occupation ou
    la profession est de jouer du violon}». Il existe en français
  plusieurs suffixes («iste», «ien», «ier», etc.) qui désignent «une
  personne caractérisée par l’idée contenue dans le radical» auquel
  ces suffixes sont accollés. Donc le mot «violon-iste» (molécule
  biatomique du type \textbf{ô-o}) sera une solution approchée par
  défaut de la synthèse en question; en effet, cette solution évoque
  d’une manière claire l’idée qu'il s’agissait d’exprimer; cependant
  elle ne contient pas explicitement l’idée de «jouer», c’est pourquoi
  la solution n’est qu’approchée; elle enfreint légèrement le principe
  de nécessité, et, en effet, le mot «violoniste» pourrait aussi, à la
  rigueur, signifier «un fabricant» ou «un marchand de
  violons». Toutefois le but du violon étant, non pas de le fabriquer,
  mais de s’en servir, le mot «violoniste» est acceptable, et les
  personnes qui fabriquent ou vendent des violons devront être
  désignées d’une manière plus explicite, soit par des molécules
  dissociées comme «fabricant de violons»\footnote{En français on
    emploie le mot «luth-ier» pour désigner aussi bien les «fabricants
    de violon», que ceux de violoncelles, luth, etc. On a donc
    spécialisé le sens logique du mot «luthier» de manière à remplir
    le rôle de la molécule manquante «fabricant de violons».}, soit
  par

}
%
{7. --- Synthesis of the idea ``\emph{personne dont l’occupation ou la
    profession est de jouer du violon}'' `person whose occupation or
  profession is to play the violin'. There exist several suffixes in
  French (\emph{iste}, \emph{ien}, \emph{ier}, etc.) that designate
  ``a person characterized by the idea contained in the root'' to
  which these suffixes are attached. Thus, the word \emph{violon-iste}
  (a bi-atomic molecule of the type \textbf{(ô-o)} would be one
  approximate solution by rounding down for the synthesis in question.
  Actually, this solution evokes in a clear way the idea which is to
  be expressed; however, it does not explicitly contain the idea ``to
  play'', which is why this solution is only approximate. It slightly
  infringes the principle of necessity, and indeed, the word
  \emph{violoniste} can also, if necessary, mean ``a maker'' or ``a
  seller of violins''. However, the goal of the violin being not to
  manufacture it, but to make use of it, the word \emph{violoniste} is
  acceptable, and those persons who make or sell violins should be
  designated in a more explicit way, either by dissociated molecules
  like \emph{fabricant des violons} `maker of violins'\footnote{In
    French we use the word \emph{luth-ier} to designate ``makers of
    violins'' as well as those of cellos, lutes, etc. We have thus
    specialized the logical sense of the word \emph{luthier} so as to
    fill the role of the missing molecule `maker of violins'.}, or by

}

\TextPage{\noindent
  %
  des molécules condensées explicites, comme en allemand
  «Violinmacher», «Violinfabrikant\footnote{Du reste en allemand on
    dit aussi «Violinspieler»; ce mot constitue une solution exacte de
    la synthèse proposée, car il n’enfreint plus le principe de
    nécessité comme le mot français «violoniste».}».

  8. \emph{Synthèse des mots composés de plusieurs radicaux.} —
  Jusqu’ici nous nous sommes occupés surtout des mots composés d’un
  seul radical et d’un ou de plusieurs affixes; c’est, qu’en effet,
  les mots composés de plusieurs radicaux ne se distinguent pas des
  mots composés d’un radical et d’affixes; ils rentrent seulement dans
  la catégorie des solutions approchées par défaut: la molécule
  «bateau à vapeur», ou en allemand «Dampf-Schiff» est construite
  d'une façon analogue aux molécules «couronn-er» ou
  «violon-iste». Dans l’expression «bateau à vapeur», la relation
  entre les deux radicaux n'est pas exprimée explicitement, car on ne
  dit pas «bateau mû par vapeur», pas plus que l’idée de «jouer» n’est
  exprimée dans le mot «violoniste». Le mot «Dampf-Schiff» est un
  tableau incomplet comme le mot «violoniste», mais ce tableau est
  suffisant, car, en général, les idées non énoncées ne feraient
  qu’alourdir le mot sans augmenter beaucoup sa clarté.

  Il est donc inutile de faire une distinction entre

}
%
{\noindent
  %
  explicit condensed molecules, as in German \emph{Violinmacher},
  \emph{Violinfabrikant}\footnote{Furthermore, in German we also say
    \emph{Violinspieler}; this word constitutes an exact solution to
    the proposed synthesis, becasuse it no longer infringes the
    principle of necessity as the French word \emph{violoniste}
    does.}.

  8.  \emph{Synthesis of words composed of several roots.} --- Up to
  this point, we have been concerned especially with words composed of
  a single root and one or more affixes.  Actually, words composed of
  several roots are not distinct from words composed of a root and
  affixes: they simply belong to the category of approximate solutions
  by rounding down. The molecule \emph{bateau à vapeur} `steam ship',
  or in German \emph{Dampf-Schiff} is built in a way analogous to the
  molecules \emph{couronn-er} or \emph{violon-iste}. In the expression
  \emph{bateau à vapeur}, the relation between the two roots is not
  explicitly expressed, since we do not say \emph{bateau mû par
    vapeur} `boat moved by steam' any more than the idea ``to play''
  is expressed in the word \emph{violoniste}. The word
  \emph{Damp-Schiff} is an incomplete picture like the word
  \emph{violoniste}, but this picture suffices, because in general the
  ideas that are not set out would do nothing but weigh down the word
  without adding greatly to its clarity.

  It is thus useless to make a distinction between

}

\TextPage{\noindent
  %
  les mots composés à un seul radical et les mots composés à plusieurs
  radicaux. Beaucoup plus utile, au contraire, est la distinction
  entre les mots composés qui représentent une solution exacte (comme
  par exemple «qual-ité», «ac-tion», «princ-esse», «chandel-ier»,
  «lou-able», «grand-eur», etc.) et ceux qui ne représentent qu’une
  solution approchée par défaut (comme par exemple «couronn-er»,
  «vio-lon-iste», «Dampf-Schiff», «Schlaf-Zimmer», etc.). Dans le
  premier cas, la soudure entre les deux éléments de la molécule est
  une simple juxtaposition, ainsi «chandel-ier» = (objet qui porte) —
  (chandelle ), «qual-ité» = (ce qui est) — (qual), etc.: les soudures
  de cette sorte sont des soudures rigides. Dans le second cas la
  soudure est élastique, c'est-à-dire que le second élément de la
  molécule est simplement caractérisé par l’idée contenue dans le
  premier élément, c’est pourquoi la solution n’est qu’approchée au
  point de vue logique, et le sens précis du mot n’est fixé d’une
  manière tout à fait nette que par l'usage ou par le contexte. Il
  suffit pour s’en rendre compte de comparer les deux mots allemands
  «Dampf-Schiff» et «Luft-Schiff», qui, quoique semblables, puisque la
  vapeur et l'air sont tous deux des gaz, correspondent à des notions
  très différentes.

  Les exemples qui précèdent suffisent pour montrer comment il faut
  appliquer les principes de

}
%
{\noindent
  %
  compound words with a single root and words composed of several
  roots. Much more useful, on the other hand, is the distinction
  between compound words that represent an exact solution (like for
  example \emph{qual-ité}, \emph{ac-tion}, \emph{princ-esse},
  \emph{chandel-ier}, \emph{lou-able} `commendable', \emph{grand-eur},
  etc.)  and those that only represent an approximate solution by
  rounding down (like for example \emph{couronn-er},
  \emph{violon-iste}, \emph{Dampf-Schiff}, \emph{Schlaf-Zimmer},
  etc.).  In the first case, the juncture between the two elements of
  the molecule is a simple juxtaposition, thus \emph{chandel-ier} =
  (object that holds) --- (candle), \emph{qual-ité} = (that which is)
  --- (qual), etc.: junctures of this kind are rigid junctures. In the
  second case, the juncture is elastic, that is, the second element of
  the molecule is simply characterized by the idea contained in the
  first element.  That is why the solution is only approximate from
  the logical point of view, and the precise meaning of the word is
  fixed clearly only by usage or the context. To see this it suffices
  to compare the two German words \emph{Damp-Schiff} and
  \emph{Luft-Schiff} which, although similar since steam and air are
  both gases, correspond to very different notions.

  The preceding examples will suffice to show how it is necessary to
  apply the principles of

}

\TextPage{\noindent
  %
  nécessité et de suffisance pour opérer la synthèse d’une idée
  complexe et la représenter par un seul mot, c'est-à-dire, en somme,
  pour condenser une molécule dissociée. Ces principes peuvent être
  utiles aux techniciens qui ont à forger de nouveaux mots.

  \begin{center}
    \textsc{Notes Additionnelles.}
  \end{center}

  {\small 1. \emph{Des atomes synonymes}. — Les atomes synonymes sont
    les atomes qui expriment la même idée; ce sont des atomes
    identiques de sens, mais différents par la forme: ils sont donc
    interchangeables entre eux; ainsi, par exemple l’atome «ité» dans
    «probité» est synonyme de l'atome «esse» dans «richesse», ou «eur»
    dans «grandeur». Au point de vue logique, un seul atome suffirait
    pour exprimer l idée substantive générale \textbf{o}, et si les
    langues naturelles en emploient plusieurs, c’est surtout pour des
    raisons d’euphonie. Ainsi le suffixe «ien» dans «pharmacien», ou
    «ier» dans «bottier» est aussi synonyme de «iste» dans
    «violoniste», ou «eur» dans «vendeur».

    Mais la diversité de formes d’un même atome est aussi due à
    d’autres causes: \emph{cette diversité peut servir à indiquer le
      caractère grammatical de l'atome immédiatement précédent.} Ainsi
    nous avons vu par exemple que les suffixes «ment», «tion», «ture»,
    «age», etc., sont synonymes des suffixes «ité», «esse»,
    «eur». etc., mais il y a cette différence que les premiers suivent
    toujours un atome verbal («écri-ture», «abonne-ment», «abatt-age»,
    etc.), tandis que les seconds viennent toujours après un atome
    adjectif («prob-ité >», «rich-esse», etc.). Cette distinction est
    très utile, car il arrive souvent que le dit atome adjectif ou
    verbal a été à tel point réduit par l’usure du langage qu'il ne
    serait plus reconnaissable si sa présence n'était pas signa-} }
%
{\noindent
  %
  necessity and sufficiency to carry out the synthesis of a complex
  idea and to represent it by a single word; that is, in sum, to
  condense a dissociated molecule. These principles can be useful to
  technicians who have to construct new words.

  \begin{center}
    \textsc{Supplementary notes}
  \end{center}

  {\small 1. \emph{Synonymous atoms}. Synonymous atoms are atoms that
    express the same idea. These are atoms identical in sense, but
    different in form: thus, for example, the atom \emph{ité} in
    \emph{probité} `integrity'' is synonymous with the atom \emph{esse} in
    \emph{richesse} `richness, wealth' or \emph{eur} in \emph{grandeur}. From
    the point of view of logic, a single atom would suffice to express
    the general nominal idea \textbf{o}, and if natural languages make
    use of several, it is especially for reasons of euphony. Thus, the
    suffix \emph{ien} in \emph{pharmacien} `pharmacist' or \emph{ier} in
    \emph{bottier} `bootmaker' is also a synonym of \emph{iste} in
    \emph{violoniste} or \emph{eur} in \emph{vendeur} `seller, salesperson'.

    But the diversity of form of the same atom is also due to other
    causes: \emph{this diversity can serve to indicate the grammatical
      nature of the immediately preceding atom}. Thus we have seen for
    example that the suffixes \emph{ment}, \emph{tion}, \emph{ture}, \emph{age},
    etc. are synonymous with the suffixes \emph{ité}, \emph{esse},
    \emph{eur}, etc., but there is this difference, that the first ones
    always follow a verbal atom (\emph{écri-ture}, \emph{abonne-ment},
    \emph{abatt-age} `slaughter', etc.) while the second set always come
    after an adjectival atom (\emph{prob-ité}, \emph{rich-esse}, etc.).
    This distinction is quite useful, because it often happens that
    the adjectival or verbal atom in question has been reduced to such
    a point by the erosion of language that it would no longer be
    recognizable if its presence were not indica[ted]



  }

}

\TextPage{\noindent
  %
  {\small lée par la forme extérieure de l'atome suivant. C'est ainsi
    que dans le mot «couronn-e-ment», le caractère verbal de l'atome
    «e» n’est reconnu que grâce à l’emploi du suffixe «ment», qui suit
    toujours un atome verbal.

    2.\emph{ Importance des idées générales sous-entendues pour fixer
      le sens des mots composés.} — Lorsqu'on dérive d’un même radical
    une famille de mots, le sens des dérivés dépend du caractère
    grammatical de ce radical, c’est-à-dire de l’idée générale
    (substantive, adjective ou verbale) contenue implicitement dans
    ledit atome-radical.

    Prenons par exemple l’atome-radical «brosse» (ou symboliquement
    \textbf{bros}), et dérivons en le mot «bross-ier» en ajoutant
    l'atome «ier», synonyme de «iste». On pourra écrire indifféremment
    «bross-ier» ou \textbf{bros-ist} (symboliquement). Or, le sens de
    ce mot varie suivant l'idée générale contenue dans l’atome
    \textbf{bros}. En francais, le mot «brosse» est substantif,
    puisque la série «brosse», «bross-er», «bross-e-ment», est tout à
    fait analogue a «couronne». «couronn-er», «couronn-e-ment»; donc
    l'idée générale sous-entendue dans l'atome \textbf{bros} est celle
    d’un «être concret», d’un «objet»; donc: (\emph{bross-ier}) =
    (\emph{une personne}) caractérisée par (\emph{l'objet-brosse}),
    par exemple «un fabricant de brosses» ou «un vendeur de brosses».

    Si, au contraire, on voulait parler d’une personne dont
    l'occupation actuelle est de brosser, il faudrait d’abord
    verbifier l'atome substantif «brosse», ce qui donne la molécule
    «bross-er», ou symboliquement \textbf{bros-i} (type \textbf{ô-i});
    cette molécule ne désigne alors plus un objet, mais l’agir relatif
    à cet objet; et de même que de «bross-er» (\textbf{bros-i}) nous
    avons déjà dérivé le substantif «bross-e-ment» (\textbf{bros-i-o}
    ou \emph{bross-ac-tion}), de même on peut dériver aussi de
    «bross-er» (\textbf{bros-i}) le substantif «bross-e-eur»
    (\textbf{bros-i-ist}) qui signifiera «personne caractérisée par
    l'action de «brosser» [»], parce que l'atome «eur» est synonyme de
    «ier»: et si l’on choisit «eur» au lieu de «ier», c’est que ce}

}
%
{\noindent
  %
  {\small
    %
    [indica]ted by the external form of the following atom. It is thus
    that in the word \emph{couronn-e-ment}, the verbal character of the
    atom \emph{e} is only recognizable thanks to the use of the suffix
    \emph{ment} which always follows a verbal atom.

    2. \emph{Importance of understood general ideas to establish the
      sense of compound words}. --- When we derive a family of words
    from the same root, the sense of the derived words depends on the
    grammatical character of this root, that is, the general (nominal,
    adjectival or verbal) idea implicitly contained in the root atom
    in question.

    Let us take, for example, the root atom \emph{brosse} `brush' (or
    symbolically \textbf{bros}), and derive from it the word
    \emph{bross-ier} by adding the atom \emph{ier}, a synonym of
    \emph{iste}. We could write equally \emph{bross-ier} or
    \textbf{bros-ist} (symbolically). Now the meaning of this word
    varies according to the general idea contained in the atom
    \textbf{bros}. In French, the word \emph{brosse} is a noun, since
    the series \emph{brosse}, \emph{bross-er}, \emph{bross-e-ment} is
    completely analogous to \emph{couronne}, \emph{couronn-er},
    \emph{couronn-e-ment}; thus, the general idea understood in the
    atom \textbf{bros} is that of a ``concrete entity'', of an
    ``object'': thus: (\emph{bross-ier}) = (\emph{a person})
    characterized by (\emph{object-brush}), for example ``a maker of
    brushes'', or ``a seller of brushes''.

    If, on the other hand, we wanted to speak of a person whose
    current occupation is to brush, it would first be necessary to
    verbalize the nominal atom \emph{brosse}, which would yield the
    molecule \emph{bross-er} or symbolically \textbf{bros-i} (of type
    \textbf{ô-i}); this molecule no longer designates an object, but
    an action relative to that object; and just as from \emph{bross-er}
    (\textbf{bros-i}) we have already derived the noun
    \emph{bross-e-ment} (\textbf{bros-i-o} or \emph{bross-ac-tion}), we
    can also derive from \emph{bross-er} (\textbf{bros-i}) the noun
    \emph{bross-e-eur} (\textbf{bros-i-ist}) which would mean ``person
    characterized by the the action of brushing'', since \emph{eur} is a
    synonym of \emph{ier}, and if we choose \emph{eur} instead of \emph{ier},
    it is because this



  }

}

\TextPage{\noindent
  %
  {\small suffixe se met toujours après un atome verbal
    (ex. «acheteur», «vend-eur», etc.), de sorte que cet atome verbal
    «e» peut même disparaître complètement, le mot «bross-eur» n’en
    conservera pas moins sa signification, parce que le suffixe «eur»,
    quoique substantif, a une forme extérieure qui témoigne de la
    verbification du radical \textbf{bros}; ainsi en réalité le mot
    «bross-(e)-eur» est une molécule tri-atomique du type
    (\textbf{ô-i-ô}), tandis que «bross-ier» est une molécule
    biatomique du type (\textbf{ô-ô}); la première molécule contient
    l’idée verbale «ag», la seconde ne la contient pas.

    3. \emph{Des idées partielles contenues dans l'idée totale.} —
    D'après les principes de nécessité et de suffisance, toute idée
    partielle contenue dans l’idée totale à exprimer doit être
    contenue dans une partie du mot composé qui sert à exprimer cette
    idée totale. Ceci s’applique non pas seulement aux atomes qui
    composent la molécule, mais aux différentes sous-molécules qui
    peuvent être contenues dans la molécule totale.

    Ainsi par exemple l’idée «couronn-e-ment» (\textbf{kron-i-o})
    contient un atome verbal parce qu’elle contient l'idée d’agir
    (\emph{ag}). Mais en outre un «couronnement» est une «action»
    (\textbf{i-o}); la molécule «couronnement» doit donc contenir non
    seulement l’atome verbal \textbf{i}, mais la molécule
    (\textbf{i-o}), ce qui a lieu en effet. De meme le mot «écri-ture»
    représentant une action, doit contenir la molécule (\textbf{i-o})
    et cela a bien lieu, car l’atome «écri» est un atome verbal
    particulier \textbf{î}, c’est-à-dire qu’il contient
    implicitement en lui-même l'idée verbale générale \textbf{i}; donc
    la molécule particulière «écri-ture» ou (\textbf{î-o}) contient la
    molécule générale (\textbf{i-o}) Etc., etc.

    4. \emph{De la soudure entre les atomes.} — Nous avons dit qu’il y
    a deux sortes de soudure entre les atomes d’une molécule: 1° la
    \emph{soudure rigide}, qui est une simple juxtaposition des idées
    contenues dans les différents atomes (avec renversement de l’ordre
    des atomes); par exemple «qual-ité» =}

}
%
{\noindent
  %
  {\small suffix always comes after a verbal atom (e.g. \emph{achet-eur}
    `buy-er', \emph{vend-eur}, etc.), such that even if this verbal atom
    \emph{e} completely disappears, the word \emph{bross-eur} will
    nonetheless preserve its meaning, since the suffix \emph{eur} ,
    though itself nominal, has an external form that bears witness to
    the verbalization of the root \textbf{bros}.  Thus the word
    ``bross-(e)-eur'' is really a tri-atomic molecule of the type
    (\textbf{ô-i-o}), while \emph{bross-ier} is a biatomic molecule of
    the type (\textbf{ô-ô}). The first molecule contains the
    verbal idea ``ag'', which the second does not.

    3. \emph{Partial ideas contained in the total idea}. --- Following
    the principles of necessity and sufficiency, every partial idea
    contained in the total idea to be expressed must be contained in
    one part of the compound word that serves to express this total
    idea. This applies not only to the atoms that compose the
    molecule, but also to the different sub-molecules that may be
    contained within the total molecule.

    Thus for example the idea \emph{couronn-e-ment} (\textbf{kron-i-o})
    contains a verbal atom because it contains the idea of acting
    (\emph{ag}). But in addition a \emph{couronnement} is an ``action''
    (\textbf{i-o}); the molecule \emph{couronnement} must thus contain
    not only the verbal atom \textbf{i}, but the molecule
    (\textbf{i-o}), which is indeed the case. Similarly, since the
    word \emph{écri-ture} represents an action, it must contain the
    molecule (\textbf{i-o}), and it indeed does, because the atom
    \emph{écri} is a specific verbal atom \textbf{î}, that is, it
    implicitly contains in itself the general verbal idea \textbf{i},
    and so the specific molecule \emph{écri-ture} or
    (\textbf{î-o}) contains the general molecule
    (\textbf{i-o}). Etc. etc.

    4. \emph{On the juncture between atoms}. --- We have said that
    there are two sorts of juncture between the atoms of a molecule:
    1. \emph{rigid juncture}, which is a simple juxtaposition of the
    ideas contained in the different atoms (with reversal of the order
    of the atoms): for example \emph{qual-ité} =

  }

}

\TextPage{\noindent
  %
  {\small (ce qui est) — (qual); cette soudure se rencontre toutes les
    fois que le mot composé représente exactement et complètement
    l’idée qu’il s’agissait d’exprimer; 2° la \emph{soudure
      élastique}, qui est équivalente à l’expression «caractérisé
    par»; par exemple «Dampfschiff» — (Schiff) caractérisé par
    (Dampf); cette soudure se rencontre toutes les fois que le mot
    composé ne représente l’idée totale que d'une manière
    approximative (solution approchée par défaut).

    Mais, quelle que soit la nature de la soudure entre deux atomes,
    il est intéressant de remarquer que ces atomes peuvent être soudés
    \emph{soit directement, soit par l'intermédiaire des idées
      générales} implicitement contenues dans ces atomes. Dans
    l’exemple «qual-ité» = (ce qui est) — (qual), il n’y a qu’une
    manière d’opérer la soudure, parce que les atomes «qual» et «ité»
    représentent tous deux des idées grammaticales, donc ils ne
    contiennent pas d’idées plus générales sous entendues, puisqu'il
    n’y a pas d'idées plus générales que les idées grammaticales. Par
    contre, dans l’exemple «beau-té», l’atome «beau» contient l’idée
    plus générale «qual», de sorte que l’on peut souder l’atome «ité»
    soit à l’atome «beau», ce
    qui donne:\\[1ex]

    \noindent
    «beau-té»=(ce qui est) -- (beau)=(le) -- (beau),\\[1ex]
    
    \noindent
    soit à l’atome «qual» contenu dans «beau», ce qui donne:

    \begin{center}
      \resizebox{\linewidth}{!}{\begin{tabular}[t]{l@{ = }l}
        «beau-té» & (ce qui est) — (qual) [espèce \emph{beau}]\\
        (qual) & «qual-ité	\emph{beau}».
      \end{tabular}}
    \end{center}



    Cette deuxième manière est préférable, parce qu elle fournit une
    analyse plus complète.

    Si un atome contient plusieurs idées générales sous-entendues, on
    peut effectuer la soudure au moyen de l’une quelconque d’entre
    elles. Ainsi le mot «cheval-in», adjectif}

}
%
{\noindent
  %
  {\small (that which is) --- (qual); this juncture is found every
    time the compound word represents exactly and completely the idea
    which is to be represented; 2. \emph{elastic juncture}, which is
    equivalent to the expression ``characterized by'': for example
    \emph{Dampschiff} = (ship) characterized by (steam).  This juncture
    is found every time the compound word represents the total idea
    only approximately (solution by rounding down).

    But whatever the nature of the juncture between two atoms, it is
    interesting to note that these atoms can be joined \emph{either
      directly, or through the intermediary of the general ideas}
    implicitly contained in those atoms. In the example
    \emph{qual-ité} = (that which is) --- (qual), there is only one
    way of carrying out the juncture, since the atoms ``qual'' and
    \emph{ité} both represent grammatical ideas and therefore do not
    contain any more general ideas that are understood, because there
    are no ideas that are more general than the grammatical ideas. On
    the other hand, in the example \emph{beau-té}, the atom
    \emph{beau} contains the more general idea ``qual'', so that one
    can join the atom \emph{ité}
    either to the atom \emph{beau}, giving:\\[1ex]

    \noindent   
    «beau-té»=(that which is) --- (beautiful) = (the) --- (beautiful),\\[1ex]
    
    \noindent
    or to the atom ``qual'' contained in \emph{beau}, giving:

    \begin{center}
      \resizebox{\linewidth}{!}{\begin{tabular}[t]{lcl}
        «beau-té» &=&(that which is) — (qual) \\
                  &&\multicolumn{1}{r}{[type \emph{beautiful}]}\\
        (qual) &=& ``qual-ity \emph{beautiful}''.
      \end{tabular}}
    \end{center}

    This second way is to be preferred, since it yields a more
    complete analysis.

    If an atom contains several understood general ideas, we can carry
    out the juncture by means of any one of them. Thus, the word
    \emph{cheval-in} `equine', adjective

  }

}

\TextPage{\noindent
  %
  {\small du mot «cheval» (\textbf{cheval-a}) signifie «cheval-qual»
    ou «(propre) au (cheval)», et comme l'atome «cheval» contient
    implicitement les idées de plus en plus générales: «animal
    mammifères», «animal vertébré», «animal», un «être concret», etc.,
    on pourra souder l'atome adjectif «in», ou \textbf{a}, à l’une
    quelconque de ces idées sous-entendues, ce qui donne les analyses
    suivantes, de plus en plus complètes:\\[1ex]

    \noindent
    \resizebox{\linewidth}{!}{\begin{tabular}[t]{@{}l@{ = }l}
      «cheval-in» & (propre) au (cheval)\\
      «cheval-in» & (propre) à l'(animal \emph{mammifère,}\\
      \multicolumn{2}{r}{\emph{cheval})}\\
      \multicolumn{2}{l}{\footnotesize (a. mamm.)}
    \end{tabular}}\\[1ex]

    \noindent
    \resizebox{\linewidth}{!}{\begin{tabular}[t]{@{}l@{ = }l}
      «cheval-in» & (propre) à l'(animal \emph{vertébré,}\\
      \multicolumn{2}{r}{\emph{mammifère, cheval})}\\
      \multicolumn{2}{l}{\footnotesize (a. mamm.)}\\
      \multicolumn{2}{l}{\footnotesize (a. vert.)}\\
    \end{tabular}}\\[1ex]

    \noindent
    et ainsi de suite.

    Pour la même raison, l’idée «Lyon» contenant les idées plus
    générales «France», «Europe», etc., le mot dérivé «Lyonn-ais[»]
    contiendra les idées plus générales «Franç-ais», «Europé-en», etc.,
    puisque le suffixe «en» est synonyme de «ais». Donc:\\[1ex]

    «Lyonn-ais» = «Franç-ais de \emph{Lyon}»

    {\footnotesize (France)}\\[1ex]

    \noindent
    ou encore:

    \noindent
    \resizebox{\linewidth}{!}{\begin{tabular}[t]{@{}l@{ = }l}
      «Lyonn-ais» & «Europé-en de \emph{France}, de \emph{Lyon}»\\
      \multicolumn{2}{l}{\footnotesize (France)}\\
      \multicolumn{2}{l}{\footnotesize (Europe)}\\
    \end{tabular}}\\[1ex]
    
    5. \emph{Des mots superflus et des mots manquants}. — Les langues
    naturelles contiennent une quantité de mots superflus. Il ne
    s'agit pas ici des synonymes qui servent à exprimer différentes
    nuances d'une même idée, comme par exemple «sommeiller» et
    «dormir», mais des mots simples, comme} }
%
{\noindent
  %
  {\small from the word \emph{cheval} `horse' (\textbf{cheval-a}), means
    ``horse-qual'' or ``(proper) to a (horse)'', and as the atom
    \emph{cheval} implicitly contains the more and more general ideas
    ``mammalian animal'', ``vertebrate animal'', ``animal'',
    ``concrete entity'', etc., we can join the adjectival atom \emph{in}
    or \textbf{a} to any one of these understood ideas, which gives
    the following progressively more complete analyses:\\[1ex]

    \noindent
    \resizebox{\linewidth}{!}{\begin{tabular}[t]{@{}l@{ = }l}
      \emph{cheval-in} & (proper) to (horse)\\
      \emph{cheval-in} & (proper) to (animal \emph{mammal,}\\
      \multicolumn{2}{r}{\emph{horse})}\\
      \multicolumn{2}{l}{\footnotesize (a. mamm.)}
    \end{tabular}}\\[1ex]

    \noindent
    \resizebox{\linewidth}{!}{\begin{tabular}[t]{@{}l@{ = }l}
      \emph{cheval-in} & (proper) to (animal \emph{vertebrate,}\\
      \multicolumn{2}{r}{\emph{mammal, horse})}\\
      \multicolumn{2}{l}{\footnotesize (a. mamm.)}\\
      \multicolumn{2}{l}{\footnotesize (a. vert.)}\\
    \end{tabular}}\\[1ex]

    \noindent
    and so on.

    For the same reason, since the idea ``Lyon'' contains the more
    general ideas ``France'', ``Europe'', etc., the derived word
    \emph{Lyonn-ais} will contain the more general ideas ``French-man'',
    ``Europe-an'', etc., because the suffix \emph{en} is a synonym of
    \emph{ais}. Thus:\\[1ex]

    \emph{Lyonn-ais} = ``French-man from \emph{Lyon}''

    {\footnotesize (France)}\\[1ex]

    \noindent
    or also:

    \noindent
    \resizebox{\linewidth}{!}{\begin{tabular}[t]{@{}l@{ = }l}
      \emph{Lyonn-ais} & ``Europe-an from \emph{France}, from \emph{Lyon}''\\
      \multicolumn{2}{l}{\footnotesize (France)}\\
      \multicolumn{2}{l}{\footnotesize (Europe)}\\
    \end{tabular}}\\[1ex]

    5. \emph{Superfluous words and missing words.} --- Natural
    languages contain many superfluous words. The issue here is not
    that of synonyms that serve to express different nuances of the
    same idea, such as for example \emph{sommeiller} `to doze' and
    \emph{dormir} `to sleep', but that of simple words, like


  }

}

\TextPage{\noindent
  %
  {\small «jument», qui pourraient être traduits aussi exactement par
    une molécule biatomique, n’impliquant que des atomes déjà connus;
    en effet:\\[1ex]

    \noindent
    \resizebox{\linewidth}{!}{\begin{tabular}[t]{@{}l@{ = }l}
      «jument»&«cheval-femelle»\\
              &«cheval-esse» ou «cheval-ine\footnotemark»
    \end{tabular}}\\[1ex]
    
    \footnotetext{Il ne faut pas confondre ce suffixe \emph{ine}, qui
      désigne ici la femelle, avec le suffixe adjectif \emph{in} du
      mot français «chevalin», qui est le même suffixe que \emph{ain}
      dans «humain».}
    \noindent
    puisque les suffixes «esse» dans «princesse», ou «ine» dans
    «héroïne» désignent la femelle.

    De même le mot «chêne» est superflu, car ou pourrait aussi bien
    dire «gland-ier» (arbre qui porte des glands), comme on dit
    «poir-ier» (qui porte des poires) ou «chandel-ier» (qui porte des
    chandelles).

    Au contraire, lorsque le mot régulièrement dérivé n’existe pas, on
    le remplace par un synonyme qui a à peu près le même sens. Ainsi,
    par exemple, pour exprimer l’«action de dormir» ou
    «dorm-ir-ac-tion» (\textbf{î-i-i-o}) on pourrait dire, en
    supprimant les pléonasmes: «dormi-tion» (\textbf{î-o}); mais ce
    mot n’existant pas en français, on le remplace par le mol
    «sommeil», qui n’est pas une molécule du type (\textbf{î-o}), mais
    un simple atome substantif (\textbf{ô})- Le mot «sommeil» n'est
    donc qu’une traduction approchée de l’expression «action de
    dormir», d’abord parce que les radicaux «sommeil» et «dorm» ne
    sont pas tout à fait synonymes, l'idée «sommeil» impliquant une
    idée de lassitude qui ne se trouve pas dans le radical «dorm»,
    ensuite parce que la molécule «dormi-tion» (\textbf{î-o}) contient
    un atome verbal (\textbf{î}) qui ne se trouve pas dans l’atome
    «sommeil» (\textbf{ô}); cependant l’atome «sommeil» rentrant, en
    tant que substantif primitif, dans la catégorie des «êtres idéels»
    (abstraits de nature) est très voisin du mot «dormition», qui, en
    tant que substantif dérivé de verbe, rentre aussi dans la
    catégorie des «êtres}%»
}
%
{\noindent
  %
  {\small \emph{jument} `mare' which could also be translated exactly as
    a bi-atomic molecule, involving nothing but known atoms; actually:\\[1ex]

    \noindent
    \begin{tabular}[t]{@{}l@{ = }l}
      \emph{jument}&``horse-female''\\
                &\emph{cheval-esse} or \emph{cheval-ine\footnotemark}
    \end{tabular}\\[1ex]
    
    \footnotetext{It is necessary not to confuse this suffix
      \emph{ine}, which here means the female, with the adjectival
      suffix \emph{in} of the French word \emph{chevalin} `equine', which
      is the same suffix as \emph{ain} in \emph{humain}.}
    \noindent
    since the suffixes \emph{esse} in \emph{princesse} and \emph{ine}
    in \emph{héroïne} designate the female.

    Similarly, the word \emph{chêne} `oak (tree)' is superfluous,
    since we could just as well say \emph{gland-ier} (tree that bears
    \emph{glands} `acorns'), as we say \emph{poir-ier} (that bears
    \emph{poires} `pears') or \emph{chandel-ier} (that holds candles).

    On the other hand, when the regularly derived word does not exist,
    we replace it by a synonym that has almost the same meaning.
    Thus, for example, to express ``the action of sleeping'' or
    \emph{dorm-ir-ac-tion} (\textbf{î-i-i-o}) we could say, while
    suppressing the pleonasms, \emph{dormi-tion} (\textbf{î-o}), but this
    word does not exist in French, and we replace it with the word
    \emph{sommeil} `sleep (n)' which is not a molecule of the type
    (\textbf{î-o}) but rather a simple nominal atom
    (\textbf{ô}). The word \emph{sommeil} is thus only an approximate
    translation of the expression ``action of sleeping'', first
    because the roots \emph{sommeil} and \emph{dorm} are not completely
    synonymous, and then because the molecule \emph{dormi-tion}
    (\textbf{î-o}) contains a verbal atom (\textbf{î}) which is not
    found in the atom \emph{sommeil} (\textbf{ô}). However, the atom
    \emph{sommeil}, belonging as a basic noun to the category of ``ideal
    entities'' (natural abstractions), is quite close to the word
    \emph{dormition}, which as a noun derived from a verb, also belongs
    to the category of ``[ideal] entities['']


  } }

\TextPage{\noindent
  %
  {\small idéels»; nous avons vu en effet qu’il n’y a qu’une nuance
    entre «dormi-tion» et «le dormir», c’est-à-dire «l’être abstrait,
    l’idée abstraite \emph{dorm}»; «dormi-tion» exprime plus
    spécialement l'action ou l’état, et «le dormir» plus spécialement
    l’idée abstraite.

    Les caprices des langues naturelles, qui tantôt créent des mots
    superflus, tantôt se privent des mots régulièrement construits,
    proviennent surtout des exigences de l’euphonie, mais ces
    accidents n’infirment en rien les lois de la formation normale des
    mots.

    6. \emph{Des atomes à double sens.} — D’après le principe de
    l'invariabilité des atomes, chaque atome ne devrait avoir qu’un
    sens, et réciproquement à chaque idée ne devrait correspondre
    qu’un seul atome. En réalité, il existe souvent plusieurs atomes
    synonymes, comme «eur» (grandeur), «esse» (richesse), «ité»
    (probité), et réciproquement un même atome peut avoir deux sens
    totalement différents; ainsi «esse» dans «richesse» n’a aucun
    rapport avec «esse» dans «princesse»

    Ces coïncidences n’ont rien à voir avec l’analyse logique; elles
    sont encore dues aux exigences de l’euphonie et dans l’analyse
    logique, on peut éviter ces coïncidences en employant l'écriture
    symbolique; ainsi on peut par exemple représenter tout être
    femelle par le suffixe symbolique\footnote{Prononcez
      phonétiquement «inne».}  \textbf{in}; on voit alors qu’il n’y a
    aucune parenté entre «richesse» et
    «princesse», puisqu’on aura phonétiquement:\\[1ex]

  \noindent
  \emph{rich-esse} = \textbf{rich-o} et \emph{princ-esse} =
  \textbf{princ-in}.\\[1ex]

  7. \emph{Des adverbes}: Dans le tableau de la classification des
  atomes (p. 40), nous n’avons considéré que trois classes
  (substantifs, adjectifs et verbes). Ces trois classes ne comprennent
  pas tous les atomes, puisqu’il existe encore des } }
%
{\noindent
  %
  {\small [``]ideal [entities]''.  We have seen actually that there is
    only a nuance between \emph{dormi-tion} and \emph{le dormir} `sleep',
    that is ``the abstract entity or idea
    \emph{dorm}''. \emph{Dormi-tion} expresses more particularly the
    action or the state, and \emph{le dormir} more particularly the
    abstract idea.

    The vagaries of natural languages, which sometimes create
    superfluous words and sometimes deprive themselves of regularly
    constructed words, come especially from the requirements of
    euphony, but these accidents do not in any way weaken the laws for
    the normal formation of words.

    6. \emph{Atomes with two senses}. --- According to the principle
    of the invariability of atoms, every atom should not have more
    than one sense, and reciprocally, only a single atom should
    correspond to each idea. In reality, there are often several
    synonymous atoms, such as \emph{eur} (grandeur). \emph{esse} (richesse),
    \emph{ité} (probité); and reciprocally the same atom can have two
    totally different senses: thus \emph{esse} in \emph{richesse} has
    nothing to do with \emph{esse} in \emph{princesse}.

    These coincidences have nothing to do with the logical analysis;
    they are due to the requirements of euphony and in the logical
    analysis we can avoid these coincidences by making use of the
    symbolic transcription.  Thus we can for example represent any
    female entity with the symbolic suffix\footnote{Pronounced
      phonetically \emph{inne} (IPA [in] --- eds.)} \textbf{in}. We can
    then see that there is no relation between \emph{richesse} and
    \emph{princesse}, since we will have phonetically\\[1ex]

  \noindent
  \emph{rich-esse} = \textbf{rich-o} and \emph{princ-esse} = \textbf{princ-in}.\\[1ex]

  7. \emph{Adverbs}. --- In the table of classification of atoms
  (p. 40), we have only considered three classes (nouns, adjectives
  and verbs). These three classes do not include all atoms, since
  there are also

    
} }

\TextPage{\noindent
  %
  {\small atomes-adverbes, des atomes-prépositions, etc., qui eux
    aussi peuvent entrer dans la formation de mots composés; cependant
    il n’y a pas lieu de classer ces atomes, car ils ne contiennent
    pas en eux-mêmes d’idée générale permettant de faire un
    classement; on pourrait bien par exemple classer les adverbes, en
    adverbes contenant l’idée de \emph{temps}, adverbes contenant
    l’idée de \emph{lieu}, etc., mais ces idées de \emph{temps, lieu,}
    etc. sont encore des idées particulières, et ce classement
    n'aiderait pas à fixer le sens des mots composés. En effet, le
    classement des atomes n’est utile que si un même atome peut être
    transféré d'une classe dans une autre: ainsi l'atome «bros»
    pourrait aussi bien être un atome verbal qu’un atome substantif,
    et le sens des mots dérivés de «bros» dépend du classement adopté
    pour ce radical; au contraire les atomes-adverbes, prépositions,
    etc., ne sont pas susceptibles de contenir deux idées générales
    différentes; leur classement est donc inutile. Ainsi le mot
    «pré-dire» par exemple ne peut pas avoir plusieurs sens, car
    l’adverbe «pré» signifiant «d’avance», contient forcément l’idée
    de temps et n'en peut pas contenir d’autres.

    On peut toutefois mettre à part les adverbes de «manière», car
    cette catégorie comprend non seulement des adverbes primitifs,
    comme «ainsi», «comment», etc., mais tous les adverbes dérivés
    d’adjectifs, comme «agréable-ment», «facile-ment», etc., de sorte
    que cette classe d’adverbes sera aussi grande que la classe des
    adjectifs. Cette classe comprend aussi des molécules dissociées
    comme «en aimant», «en abondance», «à pied», «à cheval», «par
    écrit», etc., etc , en effet:\\[1ex]

    \noindent
    \footnotesize{
      \resizebox{\linewidth}{!}{\begin{tabular}[t]{@{}l@{ = }l}
        «en abondance»&«abondam-ment»\\
                      &«(d’une manière) -- (abondante)»,\\      
        «à pied»&«pédestre-ment»\\      
                      &«(à la manière) -- (pédestre)», etc.
      \end{tabular}}
    } } }
%
{\noindent
  %
  {\small adverb atoms, preposition atoms, etc. which can also enter
    into the formation of compound words. This is not, however, the
    place to classify these items, since they do not contain in
    themselves general ideas that would allow us to make a
    classification.  We could, for example, classify adverbs into
    adverbs containing the idea of \emph{time}, adverbs containing the
    idea of \emph{place}, etc., but these ideas of \emph{time, place}
    etc. are still specific ideas, and this classification would not
    help in fixing the meaning of compound words. Indeed, the
    classification of atoms is only useful if the same atom can be
    transferred from one class to another: thus, the atom \emph{bros}
    can just as well be a verbal atom as a nominal atom, and the sense
    of words derived from \emph{bros} depends on the classification
    adopted by that root.  On the contrary, adverb atoms,
    prepositions, etc. are not able to contain two different general
    ideas; their classification is thus pointless. Thus the word
    \emph{pré-dire} `to predict' cannot have multiple senses: since
    the adverb \emph{pré} means ``in advance'', it necessarily
    contains the idea of time and cannot contain any others.

    We can, however, treat separately adverbs of ``manner'', since
    this category includes not only basic adverbs like \emph{ainsi}
    `thus', \emph{comment} `how', etc., but also all adverbs derived
    from adjectives, like \emph{agréable-ment} `pleasant-ly',
    \emph{facile-ment} `easi-ly', etc., so that this class of adverbs
    is as large as the class of adjectives. This class also included
    dissociated molecules like \emph{en aimant} `in loving', \emph{en
      abondance} `in abundance', \emph{à pied} `on foot', \emph{à
      cheval}
    `on horseback', \emph{par écrit} `in writing', etc. etc. Actually:\\[1ex]

    \noindent
    \footnotesize{
      \resizebox{\linewidth}{!}{\begin{tabular}[t]{@{}l@{ = }l}
        \emph{en abondance}&\emph{abondam-ment}\\
                           &(in a fashion) -- (abundant),\\      
        \emph{à pied}&\emph{pédestre-ment}\\      
                           &(in a way) -- (pedestrian), etc.
      \end{tabular}}
    }



  }

}

\TextPage{\small
  %
  Si donc on représente par \textbf{e} l'idée adverbiale générale «à
  la manière», «d’une manière», on aura:\\[1ex]

  \noindent
  \resizebox{\linewidth}{!}{\begin{tabular}[t]{@{}l@{ = }l}
    idée adverbiale \textbf{e} & \emph{à la manière, d'une manière,}\\
                               &suffixe \emph{ment},\\
                               &suff. dissociés \emph{à, en, par,} etc.
  \end{tabular}}\\[1ex]
    
  Il faut seulement remarquer que les suffixes dissociés comme «à»,
  dans la molécule dissociée «(à) — (pied)», signifient bien «à la
  manière», mais \emph{à condition de prendre le second atome sous la
    forme adjective}; autrement dit, la molécule «à-pied» n’est pas
  une simple molécule biatomique du type (\textbf{e}) — (\textbf{ô});
  c’est une molécule triatomique du type (\textbf{e})
  — (\textbf{ô-a}). On a, en effet:\\[1ex]

    \noindent
    (\emph{à}) -- (\emph{pied}) = (\emph{à la manière}) --
    (\emph{péd-estre}) = (\textbf{e}) -- (\textbf{pied-a})\\[1ex]
      
    \noindent
    ou en condensant la molécule:\\[1ex]
      
    \noindent
    (\textbf{e}) — (\textbf{pied-a}) = (\textbf{pied-a-e}) =
    \emph{péd-estre-ment}.\\[1ex]

    Ainsi l’atome \textbf{e} doit toujours être précédé d’un atome
    adjectif; si cet atome manque, il faut le rétablir avant de faire
    l’analyse logique. Dans le mot «agréable-ment» ou symboliquement
    \textbf{agrabl-e}, l idée adjective \textbf{a} est contenue
    implicitement dans le radical adjectif \textbf{agrabl}; en effet,
    cette molécule est du type (\textbf{â-e}) ainsi que tous les
    adverbes dérivés d’adjectifs par le suffixe «ment».

    8. \emph{Des préfixes}. — Les préfixes sont en général des
    prépositions; ce sont donc des atomes qui ne contiennent pas en
    eux-mêmes d’idée générale; cependant les préfixes offrent ceci de
    particulier qu’ils jouent le rôle d’un adverbe dans la formation
    des mots composés. Ainsi «avant-poste» = «(poste) — (en avant)»,
    «prédire» = «(dire) —(d’avance)»,

  }
%
  {\small
  %
    If we thus represent by \textbf{e} the general adverbial idea ``in
    the way'', ``in a manner'', we will have:\\[1ex]

    \noindent
    \resizebox{\linewidth}{!}{\begin{tabular}[t]{@{}l@{ = }l}
      adverbial idea \textbf{e} & \emph{in a way, in the manner,}\\
                                &suffix \emph{ment},\\
                                &dissociated suff. \emph{à, en, par,} etc.
    \end{tabular}}\\[1ex]
  
    It is only necessary to note that the dissociated suffixes like
    \emph{à} in the dissociated molecule ``(à) --- (pied)'' do mean
    ``in the manner'' but \emph{only if we take the second atom in
      adjectival form}: in other words, the molecule \emph{à pied} is
    not a simple bi-atomic molecule of the type (\textbf{e}) —
    (\textbf{ô}); it is a tri-atomic molecule of the type (\textbf{e})
    ---
    (\textbf{ô-a}). We actually havet:\\[1ex]

    \noindent
    (\emph{à}) -- (\emph{pied}) = (\emph{in the manner}) --
    (\emph{ped-estrian}) = (\textbf{e}) -- (\textbf{pied-a})\\[1ex]
      
    \noindent
    or by condensing the molecule:\\[1ex]
      
    \noindent
    (\textbf{e}) — (\textbf{pied-a}) = (\textbf{pied-a-e}) =
    \emph{ped-estrian-ly}.\\[1ex]

    Thus the atom \textbf{e} must always be preceded by an adjective
    atom; if that atom is missing, it is necessary to restore it
    before carrying out the logical analysis. In the word
    \emph{agréable-ment} `pleasant-ly', or symbolically
    \textbf{agrabl-e}, the adjectival idea \textbf{a} is implicitly
    contained in the root adjective \textbf{agrabl}. Actually this
    molecule is of the type (\textbf{â-e}) just like all adverbs
    derived from adjectives with the suffix \emph{ment}.

    8. \emph{Prefixes}. --- Prefixes are in general prepositions; they
    are thus atoms that do not contain in themselves a general idea.
    However, these prefixes present in particular the fact that they
    play the role of an adverb in the formation of compound
    words. Thus \emph{avant-poste} `outpost' = ``(post) --- (forward)'',
    \emph{prédire} = ``(to say) --- (in advance)'',
    

  }

  \TextPage{\noindent
  %
    {\small «beau-frère» = «(frère) — (à la manière
      \emph{beau})». c'est-à-dire «(frère) — (par alliance)», «bon
      vouloir» = «(vouloir) — (à la manière \emph{bonne})»,
      «malheureux» = «(heureux)— (à la manière \emph{contraire})»,
      etc., etc. On peut donc dire que tous les préfixes contiennent
      implicitement en eux-mêmes l’idée adverbiale \textbf{e} (= à la
      manière); un mot composé tel que «avant-poste» est donc un
      molécule bi-atomique du type (\textbf{ê-ô}), etc.

      9. \emph{Des atomes à caractère grammatical douteux}. — La
      plupart des atomes ont un caractère grammatical très net; ainsi
      les atomes «homme», «table», «science», «âme». contiennent
      évidemment l’idée substantive. D’autres ont un caractère moins
      net; ainsi l’atome \textbf{bros}, considéré en lui-même,
      pourrait aussi bien être classé comme verbal que comme
      substantif; mais il suffit, pour lever le doute, de considérer
      la famille des molécules dérivées de l'atome «bros»: le mot
      «brosserie», par exemple, signifie «lieu où l’on tient, où l’on
      vend des brosses», et ceci montre que le radical \textbf{bros}
      est substantif (tout au moins en français), car si ce radical
      était verbal, «brosserie» aurait un autre sens et signifierait
      «lieu où l’on brosse», comme «laverie» signifie «lieu où l'on
      lave», parce que le radical \textbf{lav} est verbal. Et, en
      effet, le verbe «bross-er» est dérivé du substantif «brosse»,
      comme «couronn-er» est dérivé du substantif «couronne»; ces deux
      verbes sont des molécules du type (\textbf{ô-i}).

      Il y a cependant des cas où l’on a quelque peine à fixer le
      caractère grammatical d'un atome. Ce cas est dû simplement au
      fait que toute langue est un organisme vivant qui évolue et se
      transforme constamment; il arrive donc au bout d’un certain
      temps que des molécules perdent leur sens primitif, celui qui
      résulte de leur étymologie, c’est-à-dire de leur composition
      atomique, et acquièrent un sens nouveau, parce que l’ancienne
      molécule, en changeant de sens, est devenue un simple atome, qui
      peu à peu donne naissance à}

  }
%
  {\noindent
  %
    {\small \emph{beau-frère} `brother-in-law' = ``(brother) --- (in a
      \emph{beautiful} way)'', that is, ``(brother) --- (by
      marriage)'', \emph{bon vouloir} `good will' = ``(will) --- (in a
      \emph{good} way)'', \emph{malheureux} `unhappy' = ``(happy) --- (in
      the \emph{opposite} way)'', etc. etc. We can thus say that all
      prefixes implicitly contain in themselves the adverbial idea
      \textbf{e} (= in the manner); a compound word like
      \emph{avant-poste} `outpost' is thus a bi-atomic molecule of the
      type (\textbf{ê-ô}), etc.

      9. \emph{Atoms of doubtful grammatical character}. --- Most
      atoms have a very clear grammatical character; thus the atoms
      \emph{homme}. \emph{table}, \emph{science}, \emph{âme} `soul'
      obviously contain the nominal idea. Others have a less clear
      character: thus, the atom \textbf{bros} considered in itself
      could just as well be classified as verbal as nominal; but it
      suffices, to remove doubt, to consider the family of molecules
      derived from the atom \emph{bros}. The word \emph{brosserie},
      for example, means ``place where one keeps, or where one sells,
      brushes'', and that shows that the root \textbf{bros} is nominal
      (at least in French), for if this root were verbal,
      \emph{brosserie} would have a different sense and would mean
      ``place where one brushes'', just as \emph{laverie} `laundry'
      means ``place where one washes'', because the root \textbf{lav}
      is verbal. And indeed, the verb \emph{bross-er} is derived from
      the noun \emph{brosse}, as \emph{couronn-er} is derived from the
      noun \emph{couronne}; these two verbs are molecules of the type
      (\textbf{ô-i}).

      There are, however, cases in which one has some difficulty in
      determining the grammatical character of an atom. This is due
      simply to the fact that every language is a living organism
      which is constantly evolving and changing.  It happens thus that
      after a time some molecules lose their original sense, that
      which results from their etymology, and acquire a new sense,
      because the old molecule, in changing its sense, has become a
      simple atom, which little by little gives rise to



    }

  }

  \TextPage{\noindent
  %
    {\small une nouvelle famille de mots dérivés; mais tant que cette
      transformation ne s’est pas complètement effectuée, le caractère
      grammatical du nouvel atome reste plus ou moins caché.

      Tel est le cas pour tous les mots français tels que «logique»,
      «physique», «musique», etc., qui sont d’anciens adjectifs
      dérivés du type moléculaire (\textbf{ô-a}): «log-ique»,
      «phys-ique», «mus-ique», etc. Mais en prenant un sens précis
      différent du sens étymologique, ces mots sont devenus ou
      deviendront de simples atomes substantifs (\textbf{ô}), et ont
      donné ou donneront de nouveaux adjectifs dérivés; c’est ainsi
      que le nouvel atome «musique» a engendré le nouvel adjectif
      «music-al» du type (\textbf{ô-a}), parce que «music» est
      maintenant un simple atome du type (\textbf{ô}); les nouveaux
      atomes substantifs «logique», «physique», etc., n’ont pas encore
      donné naissance à des adjectifs régulièrement dérivés, puisqu’on
      dit encore «logique», «physique», etc., au sens adjectif: du
      moins en français, car en anglais on dit régulièrement
      «logic-al», «physic-al», etc. Il semble donc que ce n’est qu’une
      question de temps, et que lorsque le caractère substantif des
      atomes tels que «logique» se sera suffisamment affirmé en
      français, les adjectifs tels que «logic-al» apparaîtront aussi
      en français, tout comme l’adjectif «music-al».

      Le caractère grammatical douteux de certains atomes en état de
      transformation n’infirme donc en rien le principe de la
      spécificité grammaticale des atomes en général.

      10. \emph{Du critère par pléonasme}. — Le mot \emph{pléonasme}
      vient du grec et signifie «être surabondant». Littré définit le
      pléonasme: «l'emploi simultané de plusieurs mois ayant le même
      sens. Le pléonasme est une négligence ou un moyen de donner plus
      de force à la pensée. Il y a lieu de distinguer le pléonasme
      inconscient et le pléonasme employé comme procédé de style. Le
      premier est souvent une faute de langage résultant de
      l'ignorance ou de l’irréflexion (exem-%»

    } }
%
  {\noindent
  %
    {\small a new family of derived words; but insofar as this
      transformation has not been completely carried out, the
      grammatical character of the new atom may remain more or less
      hidden.

      This is the case for all of the French words like \emph{logique},
      \emph{physique}, \emph{musique}, etc., which are old derived
      adjectives of the molecular type (\textbf{ô-a}): \emph{log-ique},
      \emph{phys-ique}, \emph{mus-ique} etc. But in taking on a specific
      sense different from the etymological sense, these words have
      become or will become simple nominal atoms (\textbf{ô}) and
      have given or will give new derived adjectives: thus the new
      atom \emph{musique} has generated the new adjective \emph{music-al} of
      type (\textbf{ô-a}), because \emph{music} is now a simple atom of
      type (\textbf{ô}). The new nominal atoms \emph{logique},
      \emph{physique}, etc. have not yet given rise to regularly derived
      adjectives, since we still say \emph{logique}, \emph{physique},
      etc. in the adjectival sense --- at least in French, because in
      English we say regularly ``logic-al'', ``physic-al'', etc. It
      thus seems that it is just a question of time, and that once the
      nominal character of atoms such as \emph{logique} is sufficiently
      established in French, adjectives such as ``logic-al'' will also
      appear in French, just like the adjective \emph{music-al}.

      The doubtful grammatical character of certain atoms undergoing
      transformation does not at all infringe the principle of the
      grammatical specificity of atoms in general.

      10. \emph{On the criterion of pleonasm}. --- The word
      \emph{pleonasm} comes from Greek, and means ``superfluity''. The
      \textsl{Littré} defines pleonasm'': ``the simultaneous use of
      several words having the same sense. Pleonasm is careless, or a
      way of giving additional force to the thought. There is cause to
      distinguish unconscious pleonasm and pleonasm used as a
      stylistic technique. The first is often a linguistic mistake
      resulting from ignorance or thoughtlessness (exam-[ple]
    }

  }

  \TextPage{\noindent
  %
    {\small ple: une hémorragie de sang). Mais il arrive parfois que
      l’usage sanctionne les pléonasmes involontaires, lorsque la
      valeur étymologique des termes cesse d’être comprise (exemple:
      aujourd’hui).»

      Les pléonasmes jouent un rôle important dans la formation des
      mots et il faut autant que possible les éviter; cependant cela
      n’est pas absolument nécessaire, car si le pléonasme alourdit
      les mots ou les expressions, il n’en change pas le sens; le
      pléonasme est souvent inévitable, et il est quelquefois utile
      pour renforcer l’expression ou la rendre plus claire.

      On peut se servir de la notion de pléonasme comme critère dans
      l'analyse des mots, car si le pléonasme ne change pas le sens
      d’une molécule, on en conclut réciproquement que: \emph{si
        l'addition d'un atome à une molécule ne change rien au sens de
        celle-ci, cet atome forme un pléonasme}, c’est-à-dire que
      l’idée contenue dans cet atome se trouvait déjà contenue dans la
      molécule.

      Ainsi de ce que l’expression «hémorragie de sang» a le même sens
      que le simple mot «hémorragie», on en conclut que l’idée «sang»
      est déjà contenue dans l’idée «hémorragie». De ce que «grand» =
      «qui est grand», on conclut que l’idée «qui est» est déjà
      contenue dans l'adjectif «grand», et comme cela s’applique à un
      adjectif quelconque, on en conclut que l’expression «qui est»
      est une des formes de l’idée adjective générale
      \textbf{a}. L’équation «grand» = «(qui est) — (grand)» peut
      s’écrire symboliquement:

    \begin{center}
      \textbf{grand} = (\textbf{a}) — (\textbf{grand}) =
      (\textbf{grand-a})
    \end{center}

    De même l'idée «l’être», «un être» (concret ou abstrait) est une
    des formes de l’idée substantive générale \textbf{o}, parce que
    pour tout substantif, comme par exemple «homme», } }
%
{\noindent
  % ``
  {\small [exam]ple: a hemorrhage of blood). But it sometimes happens
    that usage sanctions involuntary pleonasms, once the etymological
    value of the terms ceases to be understood (example: aujourd'hui
    [`today'; orig. literally `on the day of today']).''

    Pleonasms play an important role in the formation of words, and it
    is necessary to avoid them insofar as possible. This is not,
    however, absolutely necessary, because even though pleonasm weighs
    down words and expressions, it does not change their sense;
    pleonasm is often inevitable, and it is sometimes useful to
    reinforce the expression or to make it clearer.

    We can make use of the notion of pleonasm as a criterion in the
    analysis of words, because if the pleonasm does not change the
    sense of a molecule, we can conclude in turn that: \emph{if the
      addition of an atom to a molecule does not at all change the
      molecule's sense, that atom forms a pleonasm}: that is, the idea
    contained in this atom is already contained in the molecule.

    Thus from the fact that the expression ``hemorrhage of blood'' has
    the seme meaning as the simple word \emph{hemorrhage} we can
    conclude that the idea ``blood'' is already contained in the idea
    \emph{hemorrhage}.  From the fact that ``big'' = ``which is big'',
    we conclude that the idea ``which is'' is already contained in the
    adjective ``big'', and since that applies to any adjective at all,
    we therefore conclude that the expression ``which is'' is one of
    the forms of the general adjectival idea \textbf{a}. The equation
    ``big'' = ``(which is) --- (big)'' can be written symbolically:

  \begin{center}
    \textbf{big} = (\textbf{a}) — (\textbf{big}) = (\textbf{big-a})
  \end{center}

  Similarly the idea ``the entity'', ``an entity'' (concrete or
  abstract) is one of the forms of the general nominal idea
  \textbf{o}, because for every noun, such as for example \emph{homme}, }

}

\TextPage{\noindent
  %
  \begin{minipage}[t]{\linewidth}
    \setcounter{mpfootnote}{\value{footnote}}
    \renewcommand{\thempfootnote}{\arabic{mpfootnote}}%
    {\noindent
  %
      {\small on a: «homme» = «le (ou un) être homme»\footnote{Comme
          l'idée substantive \textbf{o} peut être traduite par «ce qui
          est», on peut aussi dire que «homme» = «ce qui est homme»,
          mais il faut toujours se garder de confondre «qui est»
          (l'idêe adjective \textbf{a}) avec «ce qui est» (idee
          substantive \textbf{o}). Ainsi «grand» = «qui est grand»,
          mais non pas «ce qui est grand», car «ce qui est grand» est
          «l'être idéel grand», «la grandeur», et «grand» n'est pas
          «grand-eur»; en d'autres mots: \textbf{grand} =
          \textbf{grand-a}, mais non \textbf{grand-o}. De même «homme»
          = «ce qui est homme», mais non pas «qui est homme», car «qui
          est homme» est une forme de l'adjectif «hum-ain», et «homme»
          n’est pas égal à «hum-ain»; en d'autres termes symboliques:
          \textbf{hom} = \textbf{hom-o} mais non pas
          \textbf{hom-a}.}. ou symboliquement:

  \begin{center}
    \textbf{hom} = \textbf{(o)} — (\textbf{hom}) = (\textbf{hom-o}).
  \end{center}


  De même l’idée «faire l'action» est une des formes de l’idée verbale
  générale \textbf{i}, parce que pour tout verbe, comme par exemple
  «écri», on a: «écri» = «(faire l’action) — (écri)» = «(écri-re)»,
  puisque la finale «re» exprime aussi l’idée verbale. Donc
  symboliquement:

  \begin{center}
    \textbf{skrib} = (\textbf{i}) — (\textbf{skrib}) =
    (\textbf{skrib-i}).
  \end{center}


  Par contre \textbf{skrib} n’est pas égal à \textbf{skrib-o}, car de
  l’équation \textbf{skrib} = \textbf{skrib-i}, on tire:
  \textbf{skrib-o} (ou \emph{écri-ture}) = \textbf{skrib-i-o} =
  \textbf{(i-o)} — (\textbf{skrib}) = (\emph{ac-tion}) — (\emph{écri})
  ou \emph{action d'écrire}, puisque la molécule \textbf{(i-o)} =
  \emph{ac-tion}.

  De même de l'équation \textbf{grand} = \textbf{grand-a}, on tire
  \textbf{grand-o} (ou \emph{grand-eur}) = \textbf{grand-a-o} =
  \textbf{(a-o}) — \textbf{(grand)} = (\emph{qualité}) —
  (\emph{grand}).  } }
\end{minipage}

\begin{center}
  FIN DE LA PREMIÈRE PARTIE
\end{center}
}
%
{\noindent
  %
  \begin{minipage}[t]{\linewidth}
    \setcounter{mpfootnote}{\value{footnote}}
    \renewcommand{\thempfootnote}{\arabic{mpfootnote}}%
    {\noindent
  %
      {\small we have: ``man'' = ``the (or an) entity
        man''\footnote{Since the nominal idea \textbf{o} can be
          translated by ``that which is'', we can also say that
          ``man'' = ``that which is man'', but it is always necessary
          to avoid confusing ``which is'' (the adjective idea
          \textbf{a}) with ``that which is'' (nominal idea
          \textbf{o}). Thus ``big'' = ``which is big'', but not ``that
          which is big'', because ``that which is big'' is ``the ideal
          entity big'', ``bigness, size'', and ``big'' is not
          ``big-ness''. In other words, \textbf{big} = \textbf{big-a}
          but not \textbf{big-o}. Similarly ``man'' = ``that which is
          man'' but not ``which is man'', because ``which is man'' is
          a form of the adjective \emph{hum-ain} `hum-an' and \emph{homme}
          `man' is not equal to \emph{hum-ain}, or in symbolic terms
          \textbf{hom} = \textbf{hom-o} and not \textbf{hom-a}.}, or
        symbolically:
        
        \begin{center}
          \textbf{man} = \textbf{(o)} — (\textbf{man}) =
          (\textbf{man-o}).
        \end{center}
        
        Similarly the idea ``to perform the action'' is one of the
        forms of the general verbal idea \textbf{i}, since for every
        verb, like for example \emph{écri}, we have: \emph{écri} =
        ``(perform the action) --- (écri)'' = ``(écri-re)'' because
        the ending \emph{re} also expresses the verbal idea. Thus
        symbolically:

        \begin{center}
          \textbf{skrib} = (\textbf{i}) — (\textbf{skrib}) =
          (\textbf{skrib-i}).
        \end{center}

        In contrast, \textbf{skrib} is not equal to \textbf{skrib-o},
        for from the equation \textbf{skrib} = \textbf{skrib-i}, we
        derive: \textbf{skrib-o} (or \emph{écri-ture}) =
        \textbf{skrib-i-o} = \textbf{(i-o)} — (\textbf{skrib}) =
        (\emph{ac-tion}) — (\emph{écri}) or \emph{action d'écrire},
        since the molecule \textbf{(i-o)} = \emph{ac-tion}.

        Similarly from the equation \textbf{grand} = \textbf{grand-a},
        we derive \textbf{grand-o} (or \emph{grand-eur}) =
        \textbf{grand-a-o} = \textbf{(a-o}) — \textbf{(grand)} =
        (\emph{quality}) — (\emph{grand}).  }

    }
  \end{minipage}

      \begin{center}
        END OF THE FIRST PART
      \end{center}
    }




\end{sloppypar}

\restoregeometry
%%% Local Variables: 
%%% mode: latex
%%% TeX-master: "./RdS_Morphology.tex"
%%% End: 
