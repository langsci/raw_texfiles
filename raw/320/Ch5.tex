
\chapter{Consequences}\label{chapter_consequences} \label{ch5}
In this chapter, we explore  certain consequences of the Emergent Hypothesis,\is{Emergent Grammar!hypothesis} through five case studies. We show that  \is{productivity}productive, \is{derivability}derivable, \is{optimisation}optimising patterns are captured under the Emergent morph-based framework \il{Warembori}(Warembori, \Sec\ref{section_allophone}). Our second case involves a pattern that is not productive; we show that the Emergent Hypothesis seamlessly accounts for unproductive patterns that are nonetheless phonologically related (derivable)\is{derivability} and phonologically predictable (optimising)\is{predictability}\is{optimisation} \il{English}(English, \textsection\ref{section_English-nasal-assimilation}). We then turn to three examples which are traditionally analysed by appealing to abstract\is{abstractness!underlying representation}\is{underlying representation!abstractness} underlying representations --  ternary distribution  \il{Mayak}(Mayak, \textsection\ref{section_Mayak_low_vowels}), tone shift \il{Kinande}(Kinande, \textsection\ref{section_Kinande}), and \is{absolute neutralisation}absolute neutralisation \il{Polish}(Polish \textsection\ref{section_Polish}).

\section[Complementary distribution in Warembori]{Derivable, productive and optimising: Complementary distribution in Warembori }\label{section_allophone}\label{Warembori_section}\label{section_Warembori}
\largerpage[-1]
\is{complementary distribution|(}Underlying representations are key to the standard analysis of complementary distribution, namely to motivate assigning {\it allophones}\is{allophone} to a single abstract unit, the {\it phoneme}. Schematically, complementary distribution involves at least two sounds, [p\down{1}] and [p\down{2}], where [p\down{1}] occurs in contexts where [p\down{2}] does not occur and, conversely,  [p\down{2}] occurs in contexts where  [p\down{1}] is not found -- that is, the contexts for the sounds are complementary, the quintessential pattern driving the criteria for underlying representations (laid out in Chapter \ref{ch4}, (\ref{UR-criteria})). Rather than encode each sound in its appropriate contexts when such sounds are phonetically similar,\is{similarity!sounds} generative phonology\is{generative phonology} follows the structuralist analysis\is{structuralism!phoneme} of ``phonemes''\is{phoneme} and ``allophones'' (\citealt{Bloomfield:1926, Sapir:1933, Twaddell:1935, Bloch:1948}; see Chapter \ref{chapter_URs}): a single underlying phoneme is related to its surface manifestations by rules or constraints. The analysis might involve the relation seen in  (\ref{morph-schemas-abstract-base}a) of Chapter \ref{ch4}, selecting one sound as corresponding directly to  the underlying form and letting the grammar\is{grammar} convert the underlying form into surface forms in the appropriate contexts. Alternatively, the analysis might involve the relation seen in (\ref{morph-schemas-abstract-base}b), Chapter \ref{ch4}, where the underlying form is unlike any surface form in the language; one such strategy is to represent the underlying form as impoverished in some way (e.g.\  an underspecified segment,\is{underspecification} \citealt{Archangeli:1984, Archangeli:1988, Pulleyblank:1986}):  because the values for the alternating features are predictable, these are provided by the grammar, and not included as part of the underlying representation.  

Virtually any language could be used to illustrate the phenomenon: complementary distribution of sounds is ubiquitous. To illustrate the concrete Emergent analysis of sounds in complementary distribution, we choose an example  from Warembori (glottocode ware1253%, ISO 639-3 wsa
), a Lower Mamberamo\il{Mamberamo family} language of the Papua region of Indonesia,\is{Indonesia} relying on \citet{Donohue:1999}.\il{Warembori|(}

\subsection{The Warembori puzzle}

\largerpage[-1]
\citet[6]{Donohue:1999} provides a phonemic inventory\is{inventory!Warembori consonants} of  Warembori consonants; adding in the phonetically\is{phonetics} occurring but ``non-contrastive''\is{contrast} segments [\B] and [r], we make the classifications in \tabref{Warembori-Cs}. Consistent with the discussion in \textsection\ref{section_natural_classes} and \textsection\ref{intro-partitions-sec}, we partition\is{partition} the consonants  based on a combination of phonetic and phonological criteria. For reasons of simplicity, we place both [\B] and [r] in the class labelled \textit{obstruent}.\is{obstruent|(}   This classification is sufficient for our purposes and distinguishes the consonants from each other. We return to this partitioning in our discussion in \Sec\ref{section_Warembori_conclusion}.\footnote{The Warembori series of stops marked by \ps\ (oral and nasal) are referred to as \textit{heavy consonants}\is{heavy consonant!Warembori|(} in \citet[8]{Donohue:1999}. Generally, a syllable\is{syllable!Warembori} beginning with a heavy consonant is stressed\is{stress} (using \'V to show stress, to reserve \ps\ for heaviness): [\ps bóro] `thorn-{\sc ind}', *[\ps boró], cf.\ [boró] `fruit-{\sc ind}' where stress is predictably final in the absence of heavy consonants; a post-nasal heavy consonant is preceded by a lengthened nasal: [\ipa{nuamː\ps boro}] `coconut.thorn.{\sc ind}' vs.\ [nuamboro] `coconut.fruit.{\sc ind}', and heavy nasals involve a slight glottal onset when not followed by an oral stop: \ipa{[a\ps \textglotstop nːéro]} `jungle-{\sc ind}', cf.\ [anéro] `crocodile-{\sc ind}'. Relevant to our discussion here, heavy consonants do not alternate with continuants in intervocalic position: [ayo\ps boro], *[ayo\B oro] `tree thorn'. The phonetic\is{phonetics} correlates distinguishing the heavy consonants are not well understood, but they clearly have a distinct phonological behaviour, indicating that they are in a class distinct from the alternating consonants under discussion here. If closer analysis were to reveal that they do not have phonetic properties distinct from those of the stops that alternate with continuants, then Warembori would  present an example of the independence of morph sets from each other,  like the English\il{English} example discussed in \Sec\ref{section_MS-independence} and the Mayak example considered in \Sec\ref{section_Mayak_low_vowels}.\label{Warembori-heavies-note}}


\begin{table}
\caption{Warembori consonants\label{Warembori-Cs}} 
\begin{tabular}{l ccc l}
\lsptoprule
			&{labial}	&{coronal}	&dorsal\\\midrule
voiceless 	&p 		&t 			&k\\
voiced		&b		&d		\\
heavy		&\ps b	&\ps d	\\
voiceless continuant	&		&s	\\
voiced continuant	&\B		&r	&&obstruent\\
\midrule
nasal 		&m		&n	&&sonorant \\ 
nasal-heavy	&\ps m	&\ps n\\
approximant	&w 		&y \\
\lspbottomrule
\end{tabular}
\end{table}


\is{heavy consonant!Warembori}As shown by (\ref{Warembori-allophones-1}a, b, c) respectively, the voiced stop [b] is found word-initially and after nasals; the continuant [β] occurs after a vowel. The form in (\ref{Warembori-allophones-1}d) shows both  initial and post-vocalic labials in the same form. The same distribution is found for [d] and [r] in (\ref{Warembori-allophones-1}e-i). Numbers in the final column identify the appropriate page in \citet{Donohue:1999}.\footnote{We have faithfully followed representations in \citet{Donohue:1999}; Donohue uses the symbol ``v'' for [β] once the pattern has been explained, hence the parenthesised \textit{\ipa{baβa-ro}} in (\ref{Warembori-allophones-1}d). Another consequence is the apparent difference between [kɛ] and [ke] in (\ref{Warembori-allophones-1}c, i). This is only apparent: \citet[6]{Donohue:1999} notes that ``the vowels require little comment; they show remarkably little allophony, appearing with their expected phonetic value''. Where Donohue uses IPA, we find [ɛ]; when he stops using  IPA, we find ``e''. Vowel quality is not at issue in our discussion.}


\begin{example}\et{Warembori {\sc Indicative} (`It is...')}\label{Warembori-allophones-1}\smallskip\\
\begin{tabular}{@{}ll lr r@{}}
a.&\ipa{bo-ro} &`mouth-{\sc ind}' &59\\%&`...mouth' &59\\
%b.&\ipa{warɛm-bo-ro} &river-mouth-{\sc ind} &`(It is a) mouth of a river' \\
b.&\ipa{warɛm-bo-ro} &`river-mouth-{\sc ind}' &6\\%&`...mouth of a river'&6 \\
c.&\ipa{kɛ-βo-o-ro} &`{\sc 1pl.in.poss}-mouth-tooth-{\sc ind}'&9\\%&`...tooth (in mouth)'&9\\
d.&\ipa{bava-ro} (\ipa{baβa-ro})&`stone-{\sc ind}' &37\ee%&`...stone' &37\ee
e. &\ipa{doro-ro} &`rain-{\sc ind}' &62\\%&`...rain' &62\\
f.&\ipa{dan-do} &`water-{\sc ind}' &25\\%&`...water' &25\\
g.&\ipa{doro-ran-do} &`rain-water-{\sc ind}' &6\\%&`...rainwater'&6\\
h.&\ipa{daran-do} &`ear-{\sc ind}'	&63\\%&`...ear'&63\\
i.&\ipa{ke-raran-do}&`{\sc 1.pl.incl.poss}-ear-{\sc ind}'&54\\%&`...{\sc 1.pl.incl.poss}-ear'&54
\end{tabular}
\end{example}


\largerpage[-1]
Items such as those in (\ref{Warembori-allophones-1}) demonstrate that a morph may exhibit both the stop and the continuant forms, for example, \{bo, \ipa{βo}\}\down{\sc mouth}, \{do, ro\}\down{\sc indicative}, and \{dan, ran\}\down{\sc water}. Both these pairings and the selection between the alternants is perfectly regular. Of special interest, however, are forms with no alternation,\is{alternation!non-alternating} seen with the stem-medial consonants in  [\ipa{baβa-ro}] `stone-{\sc ind}' and [daran-ro] `ear-{\sc ind}'. In words such as these, the roots contain intervocalic voiced continuants, [\ipa{baβa}], *[baba] and [\ipa{daran}], *[dadan]. Nevertheless, the discussion in \citet{Donohue:1999} follows standard analysis of such patterns, positing stops underlyingly as   in (\ref{Warembori-ur}), under the assumption that the stop is the  basic form. For roots like `stone' and `ear', intervocalic stops are standardly posited underlyingly, /baba/ and /dadan/, even though such forms are never observed.\is{phoneme}


\begin{example} \et{Analysis positing underlying representations: alternating voiced consonants in Warembori with two ``phonemes'', /b/ \& /d/} \label{Warembori-ur}\smallskip\\
\begin{tabular}{@{}llp{1.5in}lll@{}}
a. &/baba/ &`stone' 	&c. 	&/dadan/ &`ear'\\
b. &/bo/ &`mouth' 		&d.	&/do/ &{\sc indicative}\\
%	&	&				&e. &/dan/&`water'
\end{tabular}
\end{example}


This kind of analysis is so widely adopted for cases of complementary distribution that it almost seems as though positing intervocalic stops is unquestionably plausible -- and yet the forms that a learner would actually encounter are [...ba\ipa{β}a...]/[...\ipa{βaβ}a...] and [...daran...]/[...raran...].  To posit an underlying stop in forms where only a continuant is ever observed requires that the learner establish an alternation\is{reverse engineering!alternation} pattern and then use it to work backwards and establish an underlying form (/baba/, /dadan/, etc.), despite the medial consonants never being realised as stops on the surface.\footnote{This  issue  has received considerable attention in the literature on the alternation condition,\is{alternation condition}  strict cycle condition,\is{strict cycle condition} and  elsewhere condition,\is{elsewhere condition}  e.g., \citealt{Kiparsky:1968-opacity, Kiparsky:1973pr, Kiparsky:1982lexical-phonology, Mascaro:1976}. See \citet{vandeWeijer:2012} on the irrelevance of the alternation condition if we assume strictly surface representations.}

Why would such a move be taken by a learner? The first reason could be that the theory requiring such a basic form is preferable for some reason, an argument we rejected  in \textsection\ref{morph-sets-as-building-blocks-section}. The second reason might be to capture the phonological generalisation that the distribution of  stops and continuants is predictable. Let us consider that latter point here. The relevant generalisations are given in (\ref{Warembori-generalisations}).


\begin{example}\et{Generalisations}\label{Warembori-generalisations}\\
\ea voiced stops only occur word-initially and after a consonant/nasal
\ex voiced continuants only occur after a vowel
\z
\end{example}

In the following section, we   show that an Emergent account explains these generalisations -- and does so without requiring abstract\is{abstractness!underlying representation}\is{underlying representation!abstractness} underlying representations.  

\subsection{Emergence and complementary distribution}

Under Emergence, the forms recorded in the lexicon simply mirror what is observed on the surface: \{\ipa{baβa, βaβa}\}\down{\sc stone}, \{dan, ran\}\down{\sc water}, \{daran, raran\}\down{\sc ear}, \{do, ro\}\down{\sc ind}, and so on.  The fact that morphs which begin with a voiced consonant have both stop-initial and  continuant-initial counterparts is captured in a Morph Set Relation,  (\ref{Warembori-C-MSR}).\footnote{Even though this pattern obtains only with labials and coronals, place need not be mentioned in the Warembori Morph Set Relation\is{Morph Set Relation!Warembori}  (\ref{Warembori-C-MSR}) because voiced velar obstruents do not occur in the language (neither [\ipa{ɡ}] nor [\ipa{ɣ}]), as shown in the consonant chart in \tabref{Warembori-Cs}. Note, however, that \citet[9]{Donohue:1999} notes that velar stops are ``written as {\it g}'' after a nasal suggesting that velar stops may be voiced in that one context.}  

\begin{whiteshadowbox}
\begin{example} \et{Warembori  stop/continuant Morph Set Relation, \WaremboriMSR} \label{Warembori-C-MSR}\is{Morph Set Relation!Warembori}

In a minimal morph set, there is a systematic relation between morphs with an initial voiced obstruent stop and morphs with an initial voiced obstruent continuant.\smallskip\\~\\

\begin{tabular}{@{}lp{5in}@{}}
{\it examples}	&\{\ipa{baβa, βaβa}\}\down{\sc stone}\\
				&\{\ipa{dan, ran}\}\down{\sc water}\\
				&\{\ipa{ro, do}\}\down{\sc ind}\ee
\end{tabular}

\begin{tabular}{@{}llll@{}}
\WaremboriMSR:&\{$\mathcal{M}$\down{\it i}, $\mathcal{M}$\down{\it j}\} &$\mathcal{M}$\down{\it i}: &\#$\begin{bmatrix}\textrm{stop}\\\textrm{voice}\\\textrm{obstruent}\end{bmatrix}$\ee
&&$\mathcal{M}$\down{\it j}: &\#$\begin{bmatrix}\textrm{continuant}\\\textrm{voice}\\\textrm{obstruent}\end{bmatrix}$ 
\end{tabular}
\end{example}
\end{whiteshadowbox}

The Warembori  stop/continuant Morph Set Relation is a symmetric relation that gives rise to the Morph Set Condition (MSC) in (\ref{Warembori_MSC}); it holds of morph sets with initial voiced obstruents, whether stops or continuants. 

While the presence of both continuant and stop variants in a morph set is achieved by the MSC\is{Morph Set Condition} (or by direct observation if both forms happen to be encountered), the choice between the variants within a morph set is achieved by the well-formedness conditions\is{well-formedness condition!Warembori} in (\ref{Warembori-EG-1}). 

\begin{whiteshadowbox}
\begin{example} \et{Warembori Morph Set Conditions, \WaremboriMSC}\is{Morph Set Condition}\label{Warembori_MSC}
\ea With respect to \WaremboriMSR, a minimal morph set is ill-formed if there is a morph with an initial voiced obstruent stop and no corresponding morph with an initial voiced obstruent continuant.
\ex With respect to \WaremboriMSR, a minimal morph set is ill-formed if there is a morph with an initial voiced obstruent continuant and no corresponding morph with an initial voiced obstruent stop.
\z

\WaremboriMSC: For $\mathcal{M}$\down{\it i}, $\mathcal{M}$\down{\it j} of \WaremboriMSR, *\{$\mathcal{M}$\down{\it i}, $\neg$$\mathcal{M}$\down{\it j}\}, *\{$\lnot$$\mathcal{M}$\down{\it i}, $\mathcal{M}$\down{\it j}\}\\\smallskip
\begin{tabular}{ll}
{Schematic}: &*\{d..., $\neg$r...\}, *\{$\lnot$d..., r...\}\\
             &*\{b..., $\neg$\B...\}, *\{$\lnot$b..., \B...\}
             \end{tabular}
\end{example}
\end{whiteshadowbox}

One condition penalises  voiced obstruent  continuants generally (\ref{Warembori-EG-1}a); we assume this is reflected in the frequency\is{frequency} of their occurrence. A second condition penalises a voiced obstruent stop specifically after a vowel. Since only morph-initial voiced obstruent stops exhibit alternation,\is{alternation!Warembori} as expressed in \WaremboriMSR,\is{Morph Set Relation!Warembori} it might be possible to express the phonotactic\is{phonotactics} condition more generally than in (\ref{Warembori-EG-1}b).

\begin{example} \et{Conditions for Warembori} \label{Warembori-EG-1} 
\ea *$\begin{bmatrix}\textrm{cont}\\\textrm{voice}\\\textrm{obstruent}\end{bmatrix}$, \tier: segments, \dom: word\smallskip\\\is{word!domain}
For all segments, assign a violation to a word for each voiced obstruent continuant.
\ex *V $\begin{bmatrix}\textrm{stop}\\\textrm{voice}\\\textrm{obstruent}\end{bmatrix}$, \tier: segments, \dom: morph, word\smallskip\\
For all segments, assign a violation to a morph or a word for each post-vocalic voiced obstruent stop.\is{morph!domain}
\z
\end{example}


With a domain setting for both morphs and words, (\ref{Warembori-EG-1}b) holds broadly in the language; in fact, \citet{Donohue:1999} implies that there are no counter-examples. As a morph condition,  (\ref{Warembori-EG-1}b) guides acquisition\is{acquisition!well-formedness condition} of new morphs; as a word condition, (\ref{Warembori-EG-1}b)  selects among different compilations. Illustrating with \{do, ro\}\down{\sc  indicative} in (\ref{Warembori-assessment-BaBaro}), the continuant-initial morph is chosen when following a vowel, due to (\ref{Warembori-EG-1}b) militating against stops in this context. The compilations in (\ref{Warembori-assessment-BaBaro}a, c) are eliminated by (\ref{Warembori-EG-1}b); selection devolves to (\ref{Warembori-EG-1}a), which eliminates the continuant-rich *[\B a\B aro] of (\ref{Warembori-assessment-BaBaro}d).


\begin{example} \et{Assessment for [\ipa{βaβaro}]\down{\sc stone-indicative}}
\label{Warembori-assessment-BaBaro}

{\it morph sets}: \{\ipa{baβa, βaβa}\}\down{\sc stone}; \{\ipa{do, ro}\}\down{\sc indicative}

\begin{center}
\renewcommand*{\arraystretch}{1.2}

\begin{tabular}{lll | c | c }
\hline
\hline 
\multicolumn{3}{c|}{{\sc stone}-{\sc indicative}} &*V $\begin{bmatrix}\textrm{stop}\\\textrm{voice}\\\textrm{obstruent}\end{bmatrix}$ &*$\begin{bmatrix}\textrm{cont}\\\textrm{voice}\\\textrm{obstruent}\end{bmatrix}$\\
\hline
&a. &\ipa{baβado}	&*!	&*	\\
\hline

\rightthumbsup
&b. &\ipa{baβaro}	&	&** 	\\
\hline
&c. &\ipa{βaβado}	&*!	&**	  	\\
\hline
&d. &\ipa{βaβaro}	&	&***! \\
\hline
\hline 
\end{tabular}
\end{center}
\end{example}

Compilations with no V-C sequence\is{sequence!Warembori}  are resolved solely by the cost assigned to voiced obstruent continuants (\ref{Warembori-EG-1}a), whether due to the alternating segment being word-initial or post-consonantal; both contexts are illustrated in (\ref{Warembori-assessment-dando}).   The form with no such continuants, (\ref{Warembori-assessment-dando}a), is selected; *V~[stop] plays no role.

\begin{example} \et{Assessment for [\ipa{dando}]\down{\sc water-indicative}}
\label{Warembori-assessment-dando}

{\it morph sets}: \{dan, ran\}\down{\sc water}; \{\ipa{do, ro}\}\down{\sc indicative}

\begin{center}
\renewcommand*{\arraystretch}{1.2}

\begin{tabular}{lll | c | c }
\hline
\hline 
\multicolumn{3}{c|}{{\sc water}-{\sc indicative}} &*V $\begin{bmatrix}\textrm{stop}\\\textrm{voice}\\\textrm{obstruent}\end{bmatrix}$ &*$\begin{bmatrix}\textrm{cont}\\\textrm{voice}\\\textrm{obstruent}\end{bmatrix}$\\
\hline
\rightthumbsup
&a. &dando		&	&\\
\hline

&b. &danro	&	&*! 	\\
\hline
&c. &rando	&	&*!	  	\\
\hline
&d. &ranro	&	&*!* \\
\hline
\hline 
\end{tabular}
\end{center}
\end{example}


Complementary distribution\is{complementary distribution!definition}  involves a phonological distribution of X and Y such that they occur in complementary contexts throughout a language -- in both underived and derived contexts. Such patterns are derivable,\is{derivability} productive,\is{productivity} and optimising\is{optimisation} (see Chapter \ref{ch4}, (\ref{UR-criteria})). The Emergent analysis expresses these properties with  (i) an MSR\is{Morph Set Relation} that relates morphs in a morph set (derivable); (ii) the related MSC (productive)\is{Morph Set Condition}, and (iii) a phonological well-formedness condition (optimising). Because this condition is relevant in both morph and word domains, it both selects between morphs  (word domain) and  governs acquisition\is{acquisition!morph} of new morphs (morph domain). Each of these elements is necessary independently of complementary distribution; when they cooccur, complementary distribution is the result. The phonological generalisations are expressed using morph sets, not unique underlying\is{underlying representation} representations.\is{obstruent|)}


\begin{dadpbox}{The category {[\B, r]}}{box-Warembori-Br}


\is{category}\is{partition}Partitions based on voicing, stop/continuant, and  what we have called ``obstruent/sonorant'' are critical to a phonological analysis of the Warembori facts. The first two are phonetically unambiguous. We address here our classification of  [r] as an obstruent, alongside other obstruent segments like [b], [d], and [\B], a classification supported by the Warembori phonological pattern.\\ \is{natural class!unnatural class}

In terms of an obstruent$\sim$sonorant distinction, we can quite unambiguously separate [p, t, k, b, d, \ps b, \ps d, s] from [m, n, \ps m, \ps n, w, y], for which we use the familiar terms ``obstruent'' and ``sonorant'' respectively. This is less clear for [\ipa{β}, r] where standard assumptions might suggest the assignment of [\ipa{β}] to the obstruent class and  [r] to the sonorant class. Based on the phonetics\is{phonetics} only, the assignments are not completely clear, however. A learner might classify [\ipa{β}] with the sonorants, particularly if it has only weak friction; a learner  might classify [r] as an obstruent based on its differences with the unambiguous members of the sonorant class.\is{acquisition!sounds} The phonetics are plausible either way: [\ipa{β}, r] are the only voiced nonvocalic continuants and so we might expect ambiguity in their classification (\citealt{Mielke:2008}). With respect to the phonology, the situation is simple. As long as [\ipa{β}] and [r] are members of the same class, their relation to [b, d] can be characterised in a uniform fashion: [b, d] are stops; [\ipa{β}, r] are continuants. As seen above, this allows for a simple analysis.\\

If features are assigned in a more phonetically rigid manner, with [\ipa{β}] analysed as an obstruent\is{obstruent} and [r] as a sonorant, the analysis becomes more complicated hence the pattern less expected. From a rule-based perspective, the stops /b, d/ must become continuants with their values for [sonorant] dependent on place: labials become continuant (obstruents) while coronals become continuant sonorants. The shift in [sonorant] is arbitrary and unexplained. In a constraint-based framework, the constraints can be analogous to those of (\ref{Warembori-EG-1}), with the post-vocalic prohibition on voiced obstruent stops and the context-free prohibition on voiced continuant consonantal segments. The problem is that general considerations of faithfulness would favour repairs involving only [continuant]. Since the constraints can be satisfied without a change of [sonorant], there would be no reason for such a shift -- short of positing an additional constraint with no independent motivation to penalise voiced coronal continuant obstruents. Both labial and coronal stops would be expected to alternate with labial and coronal continuants, with the same value for sonorant. In short, if feature specifications are universally specified,\is{innateness} there is rigidity in their utilisation. However, if featural classifications are established on the basis of experience, as proposed in Chapter \ref{ch2}, then this issue does not arise.
\end{dadpbox}



\subsection{Analyses with underlying representations}



Complementary distribution is a phenomenon that satisfies the three criteria for underlying representations\is{underlying representation}  given in Chapter \ref{ch4}, (\ref{UR-criteria}); every framework of phonology has some kind of analysis for these patterns. As noted in the introduction to this section, the standard analysis, exemplified by that in \citet{Donohue:1999}, is to assume that one type of underlying segment (voiced stops in Warembori) are converted by rule to another class (here, continuants), requiring reverse engineering\is{reverse engineering!acquisition} during acquisition to posit the \is{underlying representation!Warembori}underlying segment even in contexts where only a different sound appears (e.g.\ the medial continuants in Warembori morphs like [doro] `rain'). 



In contrast, Optimality Theory\is{Optimality Theory} does not impose restrictions on the underlying forms \is{Emergences vs.\ OT}(richness of the base\is{richness of the base}; \citealt{Prince+:1993}). Consequently, inputs may contain  both stops /b, d/ and continuants /\ipa{β, r}/ without positional restrictions. Gen\is{Gen} creates the corresponding possible outputs, and  the constraint hierarchy evaluates  input-output pairings to eliminate forms with stops or continuants ``in the wrong place'': a constraint penalising post-vocalic [voiced, obstruent, stop] would outrank a general constraint against [voiced, obstruent, continuant]; both must outrank faithfulness to [continuant]  and to [stop] in voiced obstruents. The upshot is that if either /\ipa{β}/ or /\ipa{r}/ were posited in an underlying representation\is{underlying representation!Warembori} then it would only surface unaltered if it happened to be post-vocalic on the surface; in any other context, it would be better to have [b] or [d]. Similarly, if /b/ or /d/ were posited underlyingly, then each would surface unaltered only when {\it not} post-vocalic. If the input for a form like [ba\ipa{β}a-ro] was /baba-do/, or /baba-ro/, or /ba\ipa{β}a-ro/, or indeed, /\ipa{βaba-do}/, the surface output would be the same. In fact, since Warembori has a very limited consonant inventory\is{inventory!Warembori consonants}, the possible underlying representations could be quite varied, e.g.\  /\ipa{βava-d̪o}/ involving somewhat gratuitous adjustments of place of articulation.\is{articulation!place} In such cases of complementary distribution, much of the specific input representation is immaterial. The crucial aspect of the OT analysis is the constraint set, with markedness outranking\is{ranking!OT} faithfulness -- a ranking that is comparable in a sense to the approach taken here in terms of the importance of markedness constraints. Indeed, if OT invokes an interpretation of lexicon optimisation\is{lexicon optimisation} where inputs should be maximally harmonic with surface forms (\citealt{Prince+:1993}), then even the inputs postulated by OT would be largely comparable to the forms obtained by direct observation. This is a necessary result of the Emergent framework but a possible stipulated property in OT.\is{Optimality Theory}\is{underlying representation!Warembori}



\subsection{Conclusion}\label{section_Warembori_conclusion}

Patterns that are both productive\is{productivity} and phonological (both in the optimising\is{optimisation} and the derivable\is{derivability} senses) provide the quintessential argument for the \is{underlying representation!unique|(}unique underlying representation. In a rule-based generative analysis,\is{generative phonology} complementary distribution requires the postulation of a unique underlying form since the ``conditioned variants'' must be prevented from appearing in contexts where the rule would not apply: a generative analysis of such cases always pairs a particular postulated form for the underlying representation with a particular form of the required rule. In some cases, the choice of underlying form is determined by the ease of characterising the contexts for each sound; in some cases, the contexts are equally expressible (e.g.\ vowel harmony\is{harmony!underspecification}\is{underspecification!harmony} contexts) and an arbitrary (or underspecified)\is{underspecification} representation is posited.

 In a constraint-based approach, this need for a unique underlying form disappears since conditions\is{well-formedness condition} govern all  variants. The motivation for an underlying representation is therefore lost.  Since Emergent Grammar as we model it is a condition-based approach, unique, abstract\is{underlying representation!abstractness}\is{abstractness!underlying representation} underlying forms are completely unmotivated.\is{Emergent Grammar} \is{underlying representation!unique|)}
 

The representations required under the Emergent account are neither unique nor abstract, consisting of morph sets containing all and only morphs that are directly related to surface forms.\is{surface-to-surface} In this case, complementary distribution\is{complementary distribution|)} results from  the confluence of three properties (each of which is independently necessary for cases not involving complementarity). (i) A word- and morph-domain\is{morph!domain} condition ensures that the pattern holds both in words and morphs (derived and underived contexts); word-domain conditions express optimisation.\is{word!domain} (ii)  A phonological Morph Set Relation\is{Morph Set Relation} characterises  derivable\is{derivability} patterns by relating morphs directly to each other. (iii) The paired Morph Set Conditions determine which morph sets are incomplete and so ill-formed; rectifying this ill-formedness results in productivity.\is{productivity}\il{Warembori}\il{Warembori|)}



We turn now to an example where the pattern is \is{derivability}derivable and optimisable\is{optimisation} but not productive,\is{productivity} the distribution of preconsonantal nasals in English.



\section{Phonological but unproductive: Limited place assimilation in English}\label{section_MS-independence}

\label{section_English-nasal-assimilation}\label{URs_English-nasals}

Every morph set involves  some degree of learning.\is{acquisition!morph set}  At a minimum, each morph set  involves the sort of learning involved in mapping some arbitrary\is{set!arbitrary} set of \is{sound-meaning correspondence}sounds onto a set of semantic and \is{syntax}syntactic markers, whether or not there is additional regularity to acquire.  While it is common cross-linguistically for morphs within a morph set to bear a systematic and productive relation to each other (characterised by Morph Set Relations and Morph Set Conditions,  \Sec\ref{Yangben-MSRs-section})  there is no necessity for such relations to be \is{productivity}productive  within a language, even within morph sets that bear significant similarities\is{similarity!morph sets} to each other. Each morph set is an entity unto itself. {While certain properties in a morph set involve productive regularity (expressed by MSRs and the related MSCs), there are also patterns within morph sets that are \is{non-productivity}{\it not} productive -- for example, when the multiple forms in a morph set simply reflect an earlier stage in the language's \is{diachrony}history when a pattern was productive (\citealt{Blevins:2004}).} The \il{English|(}English \is{assimilation!nasal place}nasal place assimilation example presented here illustrates the kind of case where morph choice is \is{optimisation!not productive}phonologically optimising, but the pattern observed is not productive (English: glottocode nort3314).%, ISO 639-3 eng

Our focus is the negative prefixes, {\it in-} as in {\it inattentive}, {\it un-} as in {\it unaccented}, and {\it non-}  as in {\it nonacademic}: each ends with [n] when prevocalic. The situation changes with consonant-initial stems:  the {\it in-} morph set has multiple members, while the morph sets for {\it un-} and {\it non-} have only one each. Consequently,  one morph set shows alternation\is{alternation!morph set} (because there are multiple related members) but the other morph sets show no \is{alternation!morph set}alternation (having only one member on the relevant dimension). There is no systematic pattern governing all nasal-final prefixes in English. Thus, although morph choice is phonologically \is{optimisation!non-productive}optimising, the pattern itself is not productive. We begin by reviewing the data.

\subsection{A non-productive pattern}

%American 
English exhibits a pattern seen in many languages where nasals  share place of articulation\is{articulation!place}\is{assimilation!nasal place|(} with\is{non-productivity|(} a following consonant, within an appropriate domain. In English this pattern is restricted \is{morphology}morphologically: some morph sets participate in the pattern while others do not (\citealt{Allen:1978}). Examples showing participation are given in (\ref{English-N-place-data}). 

\begin{example} \et{English nasal place assimilation}\label{English-N-place-data}\smallskip\\
\begin{tabular}{@{}llll@{}}
a.	&within morphs 			&b.	&between morphs\\
&\begin{tabular}{llll}
	{[}bʌmp]	&`bump'\\
	{[}tɛnt]	&`tent'\\
	{[}bænd]	&`band'\\
	{[}bæləns]	&`balance'\\
	{[}lɛnz]	&`lens'\\
	{[}b{æ}ŋk]	&`bank'\\
	{[}mɪŋks]	&`minx'\\
\end{tabular}  &            &\begin{tabular}{llll}
	\ipa{[ɪmbæləns]}	&`imbalance'\\
	\ipa{[ɪntɹænsɪʤənt]}	&`intransigent'\\
	\ipa{[ɪŋkənsɪdəɹət]}	&`inconsiderate'\\
	\ipa{[ɪnæptɪtud]}	&`inaptitude'\\~\\~\\~\\
\end{tabular}
\end{tabular}
\end{example}


The nasal place assimilation\is{assimilation!nasal place} pattern holds within morphs as shown in  (\ref{English-N-place-data}a), suggesting a morph-domain syntagmatic condition preferring NC sequences\is{sequence!English NC} which share place of articulation.\is{articulation!place} In some instances, this condition also holds between morphs, illustrated in  (\ref{English-N-place-data}b): as shown, the prefix in (\ref{English-N-place-data}b) has at least three morphs, \{ɪn, ɪm, ɪŋ\}\down{\sc negative}.\footnote{The discussion is simplified by omitting details such as that the \{ɪn\} morph also occurs before vowels, that there is an \{ɪ\} morph which occurs before sonorants, an \{\ipa{ɪ\textltailm}\} morph which occurs before labiodentals, and so on.} The appropriate polymorphic form is selected by a phonotactic preferring nasal-obstruent sequences with the same place of articulation.\is{articulation!place} Thus, as the forms in (\ref{English-N-place-data}) illustrate, the domain for this phonotactic is both morph and word,  (\ref{phonotactic_Eng-pl-assim}).\is{well-formedness condition!English}\is{assimilation!nasal place}


\begin{example} \et{English nasal place assimilation phonotactic}\is{phonotactics} \label{phonotactic_Eng-pl-assim}\ee
*$\begin{bmatrix}\textrm{nasal}\\\textrm{\up{$\wedge$}place\down{\it i}}\end{bmatrix} \begin{bmatrix}\textrm{obstruent}\\\textrm{place\down{\it i}}\end{bmatrix}$, \tier: segments, \dom: morph, word\smallskip\\\is{morph!domain}
For all segments, assign a violation to a morph or a word for each nasal-obstruent sequence\is{!English NC} where the nasal's place is in the complement class\is{complement class} of the place of the obstruent.\is{word!domain}
\end{example}

\largerpage
 \begin{dadpbox}{Complement class}{box-complement-class-notation}


The notion of complement class\is{complement class} (\citealt{Hayes+:2008}), introduced in \Sec\ref{section_natural_classes}, is relevant for two types of situations. In one, the complement class is characterised by some specifiable property (e.g., the complement class [\up{$\wedge$}nasal] can also be characterised by [oral]); in the other,  the complement class cannot be expressed by a single property or set of properties shared by all members.  In the first, the specifiable property cases,  we adopt the convention of representing the complement class by the relevant specification (e.g.\ [oral], not [\up{$\wedge$}nasal]). We reserve the notion of {\it complement class}\is{complement class!definition} for cases where  the complement class is not amenable to independent specification. Such reference follows from the notion of partitioning introduced in \textsection\ref{section_natural_classes} and box \ref{box:box-partitions}, p.\ \pageref{box:box-partitions}. As noted there, it does not follow that every phonetic\is{phonetics!set}\is{set!phonetics} partitioning will result in all subsets of the partition\is{partition} being characterisable in terms of phonetic\is{phonetics} properties.\\

English nasal place assimilation\is{assimilation!nasal place} makes critical use of the notion of complement class. The complement class\is{complement class} notation allows generalisation\is{generalisation!phonological} over several separate phonotactics, each prohibiting a nasal-obstruent sequence where place features do not match: 

\begin{center}
*$\begin{bmatrix}\textrm{nasal}\\\textrm{labial}\end{bmatrix}$  $\begin{bmatrix}\textrm{obstruent}\\\textrm{coronal}\end{bmatrix}$, *$\begin{bmatrix}\textrm{nasal}\\\textrm{labial}\end{bmatrix}$ $\begin{bmatrix}\textrm{obstruent}\\ \textrm{dorsal}\end{bmatrix}$, etc. 
\end{center}

\noindent The series of separate phonotactics\is{phonotactics} misses the general observation that nasal-obstruent sequences\is{sequence!English NC} are disallowed if their place features are distinct from each other.\\

An alternative to the complement class\is{complement class} concept would be some other type of convention, such as defining both identity\is{identity} (F\down{\it i}, F\down{\it j}, where {F\down{\it i} = F\down{\it j}}) and non-identity (F\down{\it i}, F\down{\it j}, where {F\down{\it i} $\neq$\ F\down{\it j}}) relations (represented through reference to subscripts), so that the nasal place phonotactic could be defined as shown below:\begin{center}*$\begin{bmatrix}\textrm{nasal}\\\textrm{place\down{\it i}}\end{bmatrix}$ $\begin{bmatrix}\textrm{obstruent}\\\textrm{place\down{\it j}}\end{bmatrix}$, where {[place\down{\it i}]} $\neq$ [place\down{\it j}] \end{center} Given that the simple notion of partitioning\is{partition} establishes both sets and their complements,\is{complement class} we adopt that device here rather than invoke some additional convention.
\end{dadpbox}

\largerpage
In English, nasal place assimilation is not required between members of a compound ({\it fan-boy}, *{\it fa[m]-boy}; {\it fa[n]-girl}, *{\it fa[ŋ]-girl}),\footnote{See \citet{Mohanan:1993} for discussion of some of the variables involved in causing nasal place assimilation\is{assimilation!nasal place} to apply differently in different domains.} and crucial for our purposes here, there are also nasal-final prefixes which do not exhibit place assimilation,  illustrated in (\ref{English-un-non}).

\begin{example} \et{English non-alternating nasal-final prefixes}\label{English-un-non}\smallskip\\
\begin{tabular}{@{}llll@{}}
a.	&\{ʌn\}\down{\sc negative} 			&b.	&\{\ipa{nɑn}\}\down{\sc negative}\\
&\begin{tabular}{@{}l@{~~}l@{}}
	{[}ʌnpɹ\ipa{ɑ}bləmætɪk]	&`unproblematic'\\
	{[}ʌnbælənst]	        &`unbalanced'\\
	{[}ʌntʌʧt]	            &`untouched'\\
	{[}ʌndɪsəplənd]	        &`undisciplined'\\
	{[}ʌnkaɪnd]	            &`unkind'\\
{[}ʌnɡlud]	             &`unglued'
\end{tabular} & &\begin{tabular}{@{}l@{~~}l@{}}
	\ipa{[nɑnpeɪpəl]}	&`nonpapal'\\
	\ipa{[nɑnbəlif]}	&`nonbelief'\\
	\ipa{[nɑntɑksɪk]}	&`nontoxic'\\
	\ipa{[nɑndɛɹi]}	    &`nondairy'\\
	\ipa{[nɑnkɹɛdɪt]}	&`noncredit'\\
	\ipa{[nɑnɡlɛɹ]}	    &`nonglare'\\
	\end{tabular}
\end{tabular}
\end{example}

The problem is clearly stated by \citet[2]{Allen:1978}: ``[s]ome property of the prefix \uline{in-}, other than its segmental composition, must be proposed in order for a rule of \is{assimilation!nasal place rule}Nasal Assimilation to operate in forms prefixed by \uline{in-}, but not in forms prefixed by \uline{un-} or \uline{non-}. At this point, there is no reason to rule out the possibility that the necessary `property' is simply a statement of the relevant facts''. \is{non-productivity|)}

\subsection{The Emergent analysis}
Our suggestion, under an Emergent analysis, is  that we do indeed want to simply provide ``a statement of the relevant facts''. There is only one morph per morph set for the two prefixes in (\ref{English-un-non}),  \{ʌn\}\down{\sc negative} in (\ref{English-un-non}a) and \{\ipa{nɑn}\}\down{\sc negative} in (\ref{English-un-non}b). The learner perceives one form of the prefix,  acquires the morph set,\is{acquisition!morph set} and has no \is{uncertainty}uncertainty when producing forms, nonce or otherwise. In contrast, for the \textit{in-} prefix in (\ref{English-N-place-data}b), there are multiple surface forms observed and the nasal place phonotactic\is{phonotactics} -- motivated morph-internally -- determines which form to use in any given case, again without uncertainty. Allen's arguments against such a transparent analysis are essentially of two types. First, the prefix \textit{in-} attaches to various non-words (e.g., {\it inert}, {\it implacable}, {\it intrepid}, {\it insipid}, {\it immaculate}) but fails to attach to certain productively derived words (e.g.\ *{\it inselfish}, *{\it inthoughtful}, *{\it infreckled}, *{\it inchildlike}, *{\it infriendly}). Second, the semantics of \textit{in-} derivatives is less compositional than with either \textit{un-} or \textit{non-} (whose differences she also discusses). Overall, we interpret Allen's observations as evidence for a lack of productivity with the prefix {\it in-}. Allen concludes that if {\it in-} prefixation is lexically restricted, and if this is the sole nasal-final prefix exhibiting place assimilation, then the need for a general rule of nasal assimilation is weakened. The Emergent analysis  given here captures the essential properties, (i)  it is unpredictable whether a nasal-final prefix will alternate and that is encoded appropriately in the morph sets; (ii) if it does alternate, the pattern is phonologically predictable due to the nasal place assimilation\is{assimilation!nasal place} phonotactic\is{phonotactics} condition (\ref{phonotactic_Eng-pl-assim}); (iii) the nasal place assimilation phonotactic\is{assimilation!nasal place} holds of underived forms (morphs) as well. The division of labour between morph sets and conditions results in exactly the patterns observed.\largerpage[2]


Assessments\is{assessment} comparing \{ɪn, ɪm, ɪŋ\}\down{\sc negative} and \{ʌn\}\down{\sc negative} are given in (\ref{English-assessment-imbalance}) and (\ref{English-assessment-imbalanced}) respectively. When there are multiple morphs in the morph set, as in (\ref{English-assessment-imbalance}), the nasal place assimilation\is{assimilation!nasal place} phonotactic selects among the available compilations.

\begin{example} \et{Assessment for [\ipa{ɪmbæləns}]\down{\sc negative-balance}}
\label{English-assessment-imbalance} \nopagebreak \ee
{\it morph sets}: \{ɪn, ɪm, ɪŋ\}\down{\sc negative}; \{\ipa{bæləns}\}\down{\sc balance} \nopagebreak \ee
\begin{center}
\renewcommand*{\arraystretch}{1.2}
\begin{tabular}{lll | c }
\hline
\hline
\multicolumn{3}{c|}{{\sc negative-balance}} &*$\begin{bmatrix}\textrm{nasal}\\\textrm{\up{$\wedge$}place\down{\it i}}\end{bmatrix}$ $\begin{bmatrix}\textrm{obstruent}\\\textrm{place\down{\it i}}\end{bmatrix}$\\
\hline
\rightthumbsup
&a. &\ipa{ɪm-bæləns}		&		\\
\hline

&b. &\ipa{ɪn-bæləns}	&*!	 	\\
\hline
&c. &\ipa{ɪŋ-bæləns}	&*!		  	\\
\hline
\hline 
\end{tabular}
\end{center}
\end{example}

When there is only one prefix morph, as in (\ref{English-assessment-imbalanced}),  there is no selection to be made.\footnote{We do not represent the choice of suffix morphs in the table here, including only the compilation that shows the correct morph, [t]. We also do not give an account of the selection of the prefixal [ɪn]  morph when prevocalic: presumably there is a type condition against non-coronal place features (comparable in its effect to the condition against voiced continuants in Warembori\il{Warembori} (\ref{Warembori-EG-1}a)).}

\begin{example} \et{Assessment for [\ipa{ʌn-bæləns-t}]\down{\sc negative-balance-adjective}}
\label{English-assessment-imbalanced}

{\it morph sets}: \{ʌn\}\down{\sc negative}; \{\ipa{bæləns}\}\down{\sc balance}; \{d, t, əd\}\down{\sc adjective}

\begin{center}
\renewcommand*{\arraystretch}{1.2}

\begin{tabular}{lll | c }
\hline
\hline
\multicolumn{3}{c|}{{\sc negative-balance-adjective}}&*$\begin{bmatrix}\textrm{nasal}\\\textrm{\up{$\wedge$}place\down{\it i}}\end{bmatrix}$ $\begin{bmatrix}\textrm{obstruent}\\\textrm{place\down{\it i}}\end{bmatrix}$\\
\hline
\rightthumbsup
&a. &\ipa{ʌn-bæləns-t}		&*		\\
\hline
\hline 
\end{tabular}
\end{center}
\end{example}

\subsection{Non-productive patterns and URs}
\is{non-productivity!underlying representation|(}
\is{underlying representation!non-productivity|(}The standard generative analysis\is{generative phonology|(} of these three  prefixes is not so straightforward. The invariant\is{morph!invariant} prefixes in (\ref{English-un-non}) are [n]-final in all instances, so it is logical to assume that these prefixes are /n/-final in their underlying representations.\is{underlying representation} However,  an underlying representation for the prefix in (\ref{English-N-place-data}b) would also be arguably /n/-final because it surfaces as [ɪn-] when pre-vocalic (where the nasal cannot acquire place features from a following consonant). This creates a conundrum for both generative and Optimality Theoretic frameworks:\is{acquisition!underlying representation}\is{underlying representation!acquisition} how are the assimilating and the non-assimilating instances of /n/ to be distinguished in English underlying representations, where the relation among morphs is derivable\is{derivability} and optimisable\is{optimisation} but not productive?\is{productivity} Recognising the phonological properties of the alternations,\is{alternation} linguists have attempted to bypass the productivity criterion\is{productivity} either by positing  enriched representations (for example, the alternating /n/ might be underspecified,\is{underspecification!English nasal} simply [+nasal], while each nonalternating /n/ has place specifications, \citealt{Archangeli:1984}) or enriched grammar structures (for example the different \is{Lexical Phonology}strata of Lexical Phonology, \citealt{Kiparsky:1982lexical-phonology}).

A different kind of problem arises in \is{Emergences vs.\ OT}Optimality Theory. Whatever the input, Gen\is{Gen} ought to produce candidate outputs where the nasal of the prefix agrees in place with a following consonant. The problem is to distinguish between cases of the \textit{in-} input -- where place agreement is optimal, more important than remaining faithful to some input form -- and the \textit{un-} and \textit{non-} inputs -- where faithfulness to input [n] is more important than \is{generative phonology|)}\is{assimilation!nasal place|)}place agreement.\il{English|)}\is{nderlying representation!non-productivity}
\is{non-productivity!underlying representation|)}
\subsection{Conclusion}
These problems stand in sharp contrast to the Emergent approach, where an asymmetric distribution of morphs across morph sets is not unexpected, because there is no requirement that featural patterns within morph sets will always be equivalent across morph sets. In particular, because there is no recurrent pattern of nasal-final morphs exhibiting multiple places of articulation,\is{articulation!place} there is no motivation in English for a general Morph Set Condition\is{Morph Set Condition} deriving multi-member nasal-final morph sets: the English pattern is not productive -- morph sets differ in  either having (i) invariant final [n],\is{morph!invariant} or (ii) having a final nasal differentiated across place features. However, inspection of members of the alternating morph set reveals that the individual members are phonologically related to each other (derivable\is{derivability} -- and expressible as a Morph Set Relation\is{Morph Set Relation} for this morph set), and that selection among morphs is achieved by a phonologically defined condition (optimising).\is{optimisation} 



\section{Ternary contrasts: Mayak low vowels} \label{section_Mayak_low_vowels}

\is{ternarity|(}
In this section,\is{contrast!ternarity} we consider a pattern involving similarly unproductive morph sets, a case where morph sets differ in terms of a single feature. The example we consider involves Mayak tongue root harmony.\is{harmony!Mayak}\il{Mayak|(} We show that morph sets can differ  in terms of  the tongue root specification(s) within a morph set -- retracted, advanced, or both. We demonstrate that ternary behaviour arises from the way tongue root features are distributed across morph sets.


Given the human capacity for learning both highly regular and highly \is{idiosyncrasy}idiosyncratic properties, we expect to find cases where morph sets share some, but not all, properties. Mayak (glottocode buru1301) is a Western Nilotic\il{Nilotic} language spoken predominantly in South Sudan,\is{South Sudan} a Northern Burun\il{Burun} language which, together with Southern Burun, forms ``one of the three branches of Western Nilotic in ...[the] internal subgrouping of the Nilotic language, the other two being the Nuer-Dinka\il{Nuer}\il{Dinka} languages and the Luo\il{Luo} languages'' (\citealt[1]{Andersen:1999-Vs}, following \citealt{Kohler:1955}). Mayak provides a particularly interesting case where the phonotactics\is{phonotactics} are transparent, while the behaviour of particular morph sets is determined by \is{idiosyncrasy}idiosyncratic properties of those sets. Hence phonological regularity is clear as far as the phonotactics are concerned, but not in the morph sets, illustrating a case that is derivable\is{derivability} and optimising\is{optimisation} but not productive.\is{productivity} Noteworthy in terms of the general conception of the Emergent framework, morph sets with minimal membership may result in words which violate conditions simply because there is no completely well-formed compilation\is{compilation} available. As we show, the case is problematic for models with underlying representations, whether using rules or optimality theoretic constraints and Gen.\is{Gen}
 


Mayak exhibits a wide variety of alternations:\is{alternation!Mayak} morphologically-conditioned vowel raising (\citealt{Andersen:1999-Vs, Trommer:2016}), rounding harmony\is{harmony!Mayak} (\citealt{Andersen:1999-Vs, McCollum:2017}), vowel length alternations (\citealt{Andersen:1999-Vs}), and interestingly complex tongue root harmony alternations (\citealt{Andersen:1999-Vs, Andersen:2000, Finley:2007, Ozburn:2019}), as well as a rich and regular system of stem-final consonant alternations (\citealt{Andersen:1999-Cs}). In this discussion we limit our attention to a select part of tongue root harmony,\is{tongue root!Mayak} %focussing on 
those patterns involving low-vowelled suffixes.
 

\subsection{Mayak harmony patterns}

\is{tongue root!Mayak harmony|(}Mayak has a symmetrical ten-vowel system,\is{inventory!Mayak} breaking along a cross-height contrast,\is{contrast!Mayak vowels} as shown in \tabref{Mayak_vowels}. Following \citet{Andersen:1999-Vs}, we characterise this contrast in terms of tongue root position.

\begin{table} 
\caption{Mayak vowels (\citealt[3]{Andersen:1999-Vs})\label{Mayak_vowels}}
\begin{tabular}{l ccccc ccccc cc}
\lsptoprule
&\multicolumn{5}{c}{[rtr]}&\multicolumn{5}{c}{[atr]}\\\cmidrule(lr){2-6}\cmidrule(lr){7-11}
high&\ipa{ɪ}&&&&\ipa{ʊ}&i&&&&u\\
mid&&ɛ& &ɔ&&&e &&o&\\
low&&&a&&&&&ʌ&&\\
\lspbottomrule
\end{tabular}
\end{table}

Of interest here are low vowel suffixes which either harmonise\is{harmony!Mayak} or fail to harmonise with the root vowel, depending on the suffix and the specific vowel that precedes. To situate this discussion, we sketch very briefly the general properties of progressive tongue root harmony in Mayak. 

As shown in \citet{Andersen:1999-Vs}, there are two contexts which exhibit progressive harmony. We illustrate both with high vowels, turning subsequently to the analysis of low vowels.\footnote{Mid vowels have a restricted distribution that we do not consider here. See \citet{Andersen:1999-Vs}.} First, when a root vowel is high advanced, we observe advanced suffixes.\footnote{We follow the transcription used in \citet{Andersen:1999-Vs} with one exception: \citet{Andersen:1999-Vs} notes that the language has  [t, ​t̪, d, d̪] but chooses to represent [t, d] with [ƭ, ɗ] to emphasise the visual difference --  [ƭ,  ɗ] vs.\ [​t̪, d̪] rather than [t, d] vs.\. [​t̪, d̪]. We use standard IPA symbols here.} (In (\ref{Mayak_prog_harmony_I/i_copy}a, b), the suffix [k] marks \textsc{plural}.)\is{harmony!Mayak}

\begin{example}\label{Mayak_prog_harmony_I/i_copy}\et{Mayak progressive  harmony I: \{ɪ, i\}\down{\sc 1.sg.poss}
(\citealt[10]{Andersen:1999-Vs})} \smallskip\\
\begin{tabular}{@{}llllll@{}}
%&\multicolumn{5}{l}{{\sc 1.singular.possessive}}  \\
&{noun}		&\{ɪ, i\}\down{\sc 1.sg.poss}		&{gloss}\\
a.&ŋɪn	&\ipa{ŋɪŋ-ɪ-k}	&`eyes'\\
b.&lɛk 		&lɛk-ɪ-k 		&`teeth'\\
c.&pal 			&pal-ɪ			&`navel'\\
d.&wɔŋ 	&wɔŋ-ɪ	&`eye'\\
e.&​t̪ʊk 	&​t̪\ipa{ʊɣ-ɪ}	&`outer mouth'\\
f. &ʔʌm		&ʔʌm-ɪ		&`thigh'\\
g. &ʔid 		&ʔid-i 			&`ear'\\
h. &ʔuŋ 	&ʔuŋ-i 	&`knee'\\
\end{tabular}
\end{example}

As seen in (\ref{Mayak_prog_harmony_I/i_copy}a-e), the first person singular possessive suffix is retracted ([ɪ]) when the root is retracted; similarly, when the root is low advanced (\ref{Mayak_prog_harmony_I/i_copy}f), the suffix is retracted. When the root is high advanced, however, the suffix vowel is advanced as well (\ref{Mayak_prog_harmony_I/i_copy}g, h). This pattern can be derived by positing both advanced and retracted variants in the morph set for the suffix (\{ɪ, i\}\down{\sc 1.sg.poss}) in combination with two conditions.\is{well-formedness condition!Mayak}\is{harmony!Mayak conditions}\is{tongue root!Mayak phonotactic}

\begin{example}\et{Mayak *[atr, high][rtr] phonotactic condition}\is{phonotactics}
\label{Mayak_phonotactics_1}\label{Mayak_high-atr-rtr_phonotactic}\smallskip\\
\Mahr,  \tier: vowels, \dom: morph, word\\\is{morph!domain}
With a focus on vowels, assign a violation to a word or a morph for each sequence of a high advanced vowel followed by a retracted vowel.
\ex\et{Mayak *[atr] type condition}\label{Mayak_atr_type_copied}\smallskip\\
\relax {*}[atr], \tier: vowels, \dom: word\\\is{word!domain}
With a focus on vowels, assign a violation to a word for each  advanced vowel.
\end{example}

The *[atr, high][rtr] condition ensures that the suffix is advanced after a high atr root vowel; the *[atr] condition prefers  retracted morphs.

Second, when a suffix is back, we see an additional instance\is{harmony!Mayak} of harmony.\footnote{\citet{Andersen:1999-Vs} analyses the difference between suffixes with [i] and with [u] as resulting from the former being specified for [$-$ATR] and the latter being underlyingly unspecified\is{underspecification!Mayak} (along with [a]). We reanalyse this through reference to the backness distinction between [ɪ,~i] on the one hand and [ʊ, u, a] on the other. See \citet{Ozburn:2019}. Finally, we ignore the vowel length alternations\is{alternation!Mayak vowel length} as well as the height alternations in the roots in (\ref{Mayak_a_prog_harmony-alt_copy}).}

\begin{example}\label{Mayak_a_prog_harmony-alt_copy} \et{Mayak progressive  harmony II: \{ʊk, uk\}\down{\sc plural}}  (\citealt[13]{Andersen:1999-Vs})\smallskip\\
\begin{tabular}{@{}llllll@{}}
&{singular}		&plural		&{gloss}\\
a.&\ipa{mɛɛk} &m\ipa{ɪɣ-ʊk}&`spider'\\
b.&\ipa{ɡɔɔc}&\ipa{ɡʊj-ʊk}&`bowl'\\
c. &cɪɪma&cim-uk&`knife'\footnotemark\\
d.&bul&bul-uk&`stomach'\\
e.&jaaŋ&jʌŋ-uk&`crocodile'\\
\end{tabular}
\end{example}
\footnotetext{\citet[13]{Andersen:1999-Vs} notes that certain roots in the plural exhibit ``grammatically conditioned root vowel alternation''. This accounts for the difference in tongue root values between the singular and plural root forms for `knife'. As Andersen notes, however, the behaviour of the suffix is phonologically regular.}



With a retracted root, the plural suffix is retracted (\ref{Mayak_a_prog_harmony-alt_copy}a,b); with a high advanced root, the plural suffix is advanced (\ref{Mayak_a_prog_harmony-alt_copy}c,d). These tongue root values are entirely analogous to those seen in (\ref{Mayak_prog_harmony_I/i_copy}) and would be accounted for by the conditions in (\ref{Mayak_high-atr-rtr_phonotactic}) and (\ref{Mayak_atr_type_copied}).\is{harmony!Mayak} 

Of particular interest is the comparison between the harmonic  %[jaaŋ] `crocodile-{\sc sg}' vs.
 [jʌŋ-uk] `croc\-o\-dile-{\sc pl}' (\ref{Mayak_a_prog_harmony-alt_copy}e) %and [ʔʌm] `thigh'
 vs.\ the disharmonic\is{disharmony!Mayak} [ʔʌm-ɪ] `thigh-{\sc 1.sg.poss}' (\ref{Mayak_prog_harmony_I/i_copy}f). We see that an advanced low vowel in a root does not cause a front vowel to be advanced in a suffix, but it does induce advancement on a back vowel, motivating the condition in (\ref{Mayak_back-atr-back-rtr_phonotactic}).\is{well-formedness condition!Mayak}

\begin{example} \et{Mayak *[atr, back][rtr, back] phonotactic condition}\is{phonotactics}
\label{Mayak_phonotactics_2}\label{Mayak_back-atr-back-rtr_phonotactic}\smallskip\\
\Mbabr, \tier: vowels, \dom: morph, word\\\is{morph!domain}
With a focus on vowels, assign a violation to a word\is{word!domain} or a morph for each sequence of a back advanced  vowel followed by a back retracted vowel.	
\end{example}

To conclude this brief introduction to progressive harmony, we have shown that Mayak patterns respond to two pressures. First, all else being equal, retracted vowels are preferred. Second, this general preference is overridden in two contexts: advanced vowels are preferred in suffixes after a high advanced vowel; back advanced vowels are preferred in suffixes after a back advanced vowel.\is{harmony!Mayak}

\begin{dadpbox}{Mayak regressive harmony}{box-Mayak-regressive}\is{harmony!regressive}\is{regressive harmony}
In addition to the progressive harmony that is of direct relevance to our discussion, Mayak also exhibits regressive harmony, cases where root vowels alternate as a result of a particular suffix (see \citealt{Andersen:1999-Vs} for discussion; examples in this box are from p.\ 7).  Verbs with subject suffixes illustrate the pattern:\medskip\\

\begin{tabular}{@{}lllll@{}}
retracted suffix&[\ipa{ɡɛb-ɛr}]	& `beat-{\sc 3s}' &[\ipa{ɡʊ\textsubbridge{d}-ɛr}] &`untie-{\sc 3s}' \\
high advanced suffix&[\ipa{ɡeb-ir}] &`beat-{\sc 2s}'&[\ipa{ɡu\textsubbridge{d}-ir}] &`untie-{\sc 2s}'
\end{tabular}\medskip\\

Accounting for such cases involves the interaction of three things. First, appropriately defined morph sets for roots must include both retracted and advanced forms, for this case, \{\ipa{ɡɛb, ɡeb}\}\down{\sc beat} and \{\ipa{ɡʊ\textsubbridge{d}}, \ipa{ɡu\textsubbridge{d}}\}\down{\sc untie}. The relevant Morph Set Relation\is{Morph Set Relation!Mayak} ensures appropriate two-member morph sets, relating  a retracted, nonlow  $\mathcal{M}$\down{{\it i}} and an advanced $\mathcal{M}$\down{{\it j}}. This MSR is asymmetrically productive:\is{productivity!asymmetric} retracted nonlow vowels have advanced counterparts, but [atr] vowels need not have retracted counterparts. Thus, two classes of roots show no alternations, those with low vowels, \{\ipa{ʔam}\}\down{\sc eat}: [\ipa{ʔam-b-ɛr}] `eat-{\sc 3s}', [\ipa{ʔam-b-ir}] `eat-{\sc 2s}', and those with advanced vowels, \{\ipa{ʔib}\}\down{\sc shoot} ([\ipa{ʔib-ɛr}] `shoot-{\sc 3s}', [\ipa{ʔib-ir}] `shoot-{\sc 2s}'; \{\ipa{pʌ\textsubbridge{d}}\}\down{\sc untie}, [\ipa{pʌ\textsubbridge{d}-ɛr}] `untie-{\sc 3s}', [\ipa{pʌ\textsubbridge{d}-ir}] `untie-{\sc 2s}'. \\

The next relevant factor is the type condition in (\ref{Mayak_atr_type_copied}), *[atr], causes morphs with retracted vowels to be preferred, all else being equal. \\

The final element is a syntagmatic condition\is{well-formedness condition!Mayak} that causes the advanced form to be preferred before a high, advanced suffix: {*}[rtr][atr, hi]; \tier: vowels; \dom: morph, word.\is{morph!domain} (With a focus on vowels, assign a violation to a morph or a word for any sequence\is{sequence!Mayak} of a retracted vowel followed by a high advanced vowel.)\\\is{word!domain}

The phonotactic\is{phonotactics} {*}[rtr][atr, hi], responsible for regressive harmony,\is{regressive harmony} requires that the second vowel be high. When alternating roots like \{\ipa{ɡɛb, ɡeb}\}\down{\sc beat} and \{\ipa{ɡʊ\textsubbridge{d}}, \ipa{ɡu\textsubbridge{d}}\}\down{\sc untie} are followed by a low advanced vowel, the harmony phonotactic does not override the general preference for retracted vowels: [\ipa{ɡɛb-ʌr}] `beat-{\sc 1s}', [\ipa{ɡʊ\textsubbridge{d}-ʌr}] `untie-{\sc 1s}'. (Low vowel retracted roots are similarly unaffected by a low advanced suffix, [\ipa{ʔam-b-ʌr}] `eat-{\sc 1s}'. This is doubly expected since such roots have no advanced forms and the low advanced suffix is exempt from the regressive harmony condition.)\is{harmony!regressive}\is{regressive harmony}
\end{dadpbox}

\subsection{Three patterns for suffixes with low vowels}

We now turn to the focus of this section, the behaviour of low vowel suffixes in the progressive harmony\is{harmony!Mayak} contexts outlined above. As seen in \tabref{Mayak-variable-low-suffix-summary-blood} and \tabref{Mayak-variable-low-suffix-summary-iron}, low vowel suffixes exhibit three distinct harmonic patterns.\footnote{For additional data, including examples with mid vowels, see \citet{Andersen:1999-Vs, Andersen:2000}.}


\begin{table}
\caption{Low vowel suffixes: alternating behaviour (\citealt{Andersen:1999-Vs})\label{Mayak-variable-low-suffix-summary-blood}} 
\begin{tabular}{llllllllll}
\lsptoprule
&\multicolumn{2}{l}{\{a​t̪, ʌ​t̪\}, p.\ 13}\\ 
&{\sc sg}&{\sc pl}&{\it gloss}\\\midrule
{[}ɪ] &rɪm-a​t̪&rɪm&`blood'\\
% &bɪl&bɪl-ak&`iron' &\ipa{\textsubbridge{d}ɪɪm-b-ɛr} &\ipa{\textsubbridge{d}ɪɪm-b-ʌr} &`weed'  \\
{[}a] &daal-a​t̪&daal&`flower''\\%  &kac&kaj-ak&`leopard' &caab-ɛr&caab-ʌr  	&`cook' \\
{[}ʊ] &kʊm-a​t̪ &kʊm&`egg''\\% &\ipa{kʊr}&kʊr-ak&`boat' &\ipa{ɟʊʊɟ-ɛr}&\ipa{ɟʊʊɟ-ʌr} 	&`find' \ee

%{[}i] &ʔin-ʌ​t̪&ʔin&`intestine'	&kic&kij-ak&`bee' &wiin-\textrtaild-ɛr&wiin-\textrtaild-ʌr   	&`cook'\\
{[}i] &ʔin-ʌ​t̪&ʔin&`intestine''\\%	&kic&kij-ak&`bee' &wiin-d-ɛr&wiin-d-ʌr   	&`cook'\\
%{[}u] &ruuj-ʌ​t̪ &ruuc&`worm' &kuʈ&kuɗ-ak&`nest' &puur-d̪-ɛr&puur-d̪-ʌr  	&`hoe'\ee
{[}u] &ruuj-ʌ​t̪ &ruuc&`worm''\\% &kut&kud-ak&`nest' &puur-d̪-ɛr&puur-d̪-ʌr  	&`hoe'\ee

{[}ʌ] &ʔʌʌw-ʌ​t̪&ʔʌʌp &`bone''\\% &kʌm&kʌm-ak&`elbow' &ʔʌʌb-ɛr&ʔʌʌb-ʌr 	&`catch~in the air'\\
\lspbottomrule
\end{tabular}
\end{table}

The singular suffix \{\ipa{a​t̪, ʌ​t̪}\}\down{\sc singular} has both retracted and advanced forms, alternating as a function of the harmonic context. This alternation contrasts with both the plural suffix \{ak\}\down{\sc plural}, which is invariably retracted without regard for its harmonic context, and with the first singular subject suffix, which is similarly invariant,\is{morph!invariant} but advanced rather than retracted, \{ʌr\}\down{\sc 1.singular}. These differences are summarised in (\ref{Mayak-low-suffixes-copy}) -- given what a learner encounters,\is{acquisition!morph set} these are the morph sets\is{acquisition!morph set} that will be acquired. We also include a fourth suffix \{ʌn\}\down{\sc singular}, given in \citet{Andersen:2000}, which has an invariant advanced low vowel. 

\begin{table}
\caption{Low vowel suffixes: invariant behaviour (\citealt{Andersen:1999-Vs})\label{Mayak-variable-low-suffix-summary-iron}} 
\begin{tabular}{llllllllll}
\lsptoprule
&\multicolumn{2}{l}{\{ak\}, p.\ 12}&&\multicolumn{2}{l}{\{ʌr\}, p.\ 8} \\
&{\sc sg}&{\sc pl}&{gloss}&{\sc 3sg}&{\sc 1sg}&{\it gloss} \\\cmidrule(lr){2-4}\cmidrule(lr){5-7}
{[}ɪ] %&rɪm-a​t̪&rɪm&`blood'
&bɪl&bɪl-ak&`iron' &\ipa{\textsubbridge{d}ɪɪm-b-ɛr} &\ipa{\textsubbridge{d}ɪɪm-b-ʌr} &`weed'  \\
{[}a] %&daal-a​t̪&daal&`flower' 
&kac&kaj-ak&`leopard' &caab-ɛr&caab-ʌr  	&`cook' \\
{[}ʊ] %&kʊm-a​t̪ &kʊm&`egg'
&\ipa{kʊr}&kʊr-ak&`boat' &\ipa{ɟʊʊɟ-ɛr}&\ipa{ɟʊʊɟ-ʌr} 	&`find' \ee

%{[}i] &ʔin-ʌ​t̪&ʔin&`intestine'	&kic&kij-ak&`bee' &wiin-\textrtaild-ɛr&wiin-\textrtaild-ʌr   	&`cook'\\
{[}i] %&ʔin-ʌ​t̪&ʔin&`intestine'
&kic&kij-ak&`bee' &wiin-d-ɛr&wiin-d-ʌr   	&`cook'\\
%{[}u] &ruuj-ʌ​t̪ &ruuc&`worm' &kuʈ&kuɗ-ak&`nest' &puur-d̪-ɛr&puur-d̪-ʌr  	&`hoe'\ee
{[}u] %&ruuj-ʌ​t̪ &ruuc&`worm'
&kut&kud-ak&`nest' &puur-d̪-ɛr&puur-d̪-ʌr  	&`hoe'\ee

{[}ʌ] %&ʔʌʌw-ʌ​t̪&ʔʌʌp &`bone'
&kʌm&kʌm-ak&`elbow' &ʔʌʌb-ɛr&ʔʌʌb-ʌr 	&`catch~in the air'\\\lspbottomrule
\end{tabular}
\end{table}

\begin{example} \et{Three types of suffixes with initial low vowels in Mayak}\label{Mayak-low-suffixes-copy}\smallskip\\
\begin{tabular}{@{}lll@{}}
a. 	&alternating	[rtr]$\sim$[atr]	&\{\ipa{a​t̪, ʌ​t̪}\}\down{\sc singular}  \\
b.	&invariant [rtr]	&\{ak\}\down{\sc plural}\\
c.	&invariant [atr] 	&\{ʌr\}\down{\sc 1.singular}\\
	&				&\{ʌn\}\down{\sc singular}\footnotemark\\
\end{tabular}
\end{example}
\footnotetext{\citet{Andersen:2000} lists an invariant suffix [-ʌni​t̪] `singular' but notes that it is ``probably morphologically complex, consisting of the suffixes /-ʌn/ and /-i​t̪/'' \citealt[34]{Andersen:2000};  both are glossed as {\sc singular}). Phonologically, both [-ʌn] and [-i​t̪] are invariant, as would be the putative [-ʌni​t̪]. We adopt Andersen's proposal that [-ʌni​t̪] is bimorphemic ([-ʌn-i​t̪]), and interpret his examples accordingly. Since these morphs  are harmonically invariant, regardless of whether [-ʌni​t̪] is compositional, the proposals made here are not affected other than in the postulation or non-postulation of a harmonically invariant \{ʌni​t̪\}\down{\sc singular} in addition to the harmonically invariant \{ʌn\}\down{\sc singular} and \{i​t̪\}\down{\sc singular}. Note that interpreting [-ʌni​t̪] as bimorphemic creates a form that is doubly-marked for {\sc singular}, assuming the adequacy of the gloss for the morph's syntactic and semantic properties.\label{un-it_note}} 

In terms of the criteria discussed in Chapter \ref{ch4},  (\ref{UR-criteria}), for relating morphs, we see first that the only case with multiple morphs is \{\ipa{a​t̪, ʌ​t̪}\}\down{\sc singular}. Regarding {\it derivability},\is{derivability} these two morphs are certainly related to each other phonologically: one is retracted, the other is advanced; all other features are the same. Similarly, as we are about to show, the choice between the two is phonologically {\it optimising},\is{optimisation} determined by the dictates of tongue root harmony.\is{harmony!Mayak} Like English\il{English} \is{assimilation!nasal place}nasal place assimilation (\textsection\ref{section_MS-independence}), however, there is no {\it productivity}\is{productivity} in the observed variation. Because of the small number of such suffixes and their relatively even distribution across the three categories (alternating, invariably retracted, invariably advanced), each morph set must be learned through exposure to the morphs themselves.\is{acquisition!morph set} {It seems unlikely that a single pair is sufficient to establish a Morph Set Relation\is{Morph Set Relation} governing advanced and retracted morphs in the low vowel morph sets in Mayak; however, even if such a relation were established, there is no recurring pattern in the observed morph sets to motivate a productive Morph Set Condition.}\is{Morph Set Condition} Once the learner has identified suffix morph set membership (through observation), assessment of the morph compilation\is{compilation} follows directly from the proposed conditions on sequences of vowels, (\ref{Mayak_high-atr-rtr_phonotactic}) and (\ref{Mayak_back-atr-back-rtr_phonotactic}), and  on vowel types, (\ref{Mayak_atr_type_copied}), where the syntagmatic conditions take precedence over the type condition.\is{well-formedness condition!Mayak}


We provide sample assessments with each of the three types of low-vowelled suffixes in turn. The crucial point is that given the structure of each morph set, selection follows straightforwardly. There is no need to treat one suffix type as the ``normal'' pattern and the other two as ``exceptional''.

\subsubsection{The alternating low suffix}
We turn first to the case where the learner has acquired a  suffix with an alternating [low] vowel, \{a​t̪, ʌ​t̪\}\down{\sc singular}: the retracted morph appears after retracted roots and the advanced morph appears after advanced roots.


With a high advanced noun, the choice between the advanced and retracted suffix options is illustrated in (\ref{Mayak_intestine1}). The prohibition against [atr, high][rtr] sequences\is{sequence!Mayak} eliminates the retracted form of the suffix, *[ʔin-a​t̪], thereby selecting the fully harmonic form\is{harmony!Mayak}, [ʔin-ʌ​t̪].


\begin{example}Assessment for  [\ipa{ʔin-ʌ​t̪}]\down{\sc intestine-1.singular}\label{Mayak_intestine1} \ee

{\it morph sets}: \{\ipa{ʔin}\}\down{\sc intestine}; \{a​t̪, ʌ​t̪\}\down{\sc singular}

\begin{tabular}{lllp{.6in}|c:c|c}
\hline\hline
\multicolumn{4}{c|}{\sc intestine-sg}	&\Mhiatr &\Mbkatr	&\Matr \\
\hline
&&a.		&ʔin-a​t̪		&*!		&&*		\\
\hline
&\rightthumbsup
&b.	&ʔin-ʌ​t̪			&		&&**	\\  \hline\hline
\end{tabular}
\end{example}

With a low advanced root, harmony\is{harmony!Mayak} is again predicted since low vowels are [back].

\begin{example}Assessment for  [\ipa{ʔʌʌw-ʌ​t̪}]\down{\sc bone-1.singular}\ee\label{Mayak_bone1}

{\it morph sets}: \{\ipa{ʔʌʌw}\}\down{\sc bone}; \{a​t̪, ʌ​t̪\}\down{\sc singular}

\begin{tabular}{lllp{.6in}|c:c|c}
\hline\hline
&\multicolumn{3}{c|}{\sc bone-sg}	&\Mhiatr &\Mbkatr	&\Matr \\
\hline
&&a.		&ʔʌʌw-a​t̪	&		&*!	&*		\\
\hline
&\rightthumbsup
&b.	&ʔʌʌw-ʌ​t̪			&		&	&**	\\  \hline\hline
\end{tabular}
\end{example}

Hence the alternating low vowel suffix follows precisely the pattern established above for high vowels: the low vowel is advanced if the root is [atr, high] (\ref{Mayak_intestine1}), advanced if the root is [atr, back] (\ref{Mayak_bone1}), and retracted otherwise -- where we illustrate the retracted case in (\ref{Mayak_blood-sg}). (In our assessment of [rɪm-a​t̪] `blood-{\sc singular}', we include both [rtr] and [atr] morphs for the root for `blood'. This is because regressive\is{harmony!regressive}\is{regressive harmony} harmony (see box \ref{box:box-Mayak-regressive}, p.~\pageref{box:box-Mayak-regressive}) \label{regressive-reference} motivates the possibility of an advanced morph for such a root. This is immaterial in general, since the retracted form will be preferred due to *[atr] -- only in contexts forcing regressive harmony will such a root be chosen. We include the advanced morph in the morph set here for two reasons: (i) completeness, (ii) to show that positing such a form, as required for regressive harmony,\is{harmony!regressive}\is{regressive harmony} results in no difficulties for the analysis).


\begin{example}Assessment of [rɪm-a​t̪]\down{\sc blood-singular}\label{Mayak_blood-sg}\\

{\it morph sets}:  \{rɪm, rim\}\down{\sc blood}; \{a​t̪, ʌ​t̪\}\down{\sc singular} \nopagebreak \\

\begin{tabular}{lllp{.6in}|c:c|c}
\hline\hline
&\multicolumn{3}{c|}{\sc blood-sg}	&\Mhiatr &\Mbkatr	&\Matr \\  \hline
&\rightthumbsup
&a.		&rɪm-a​t̪		&		&		&	\\
\hline
&&b.	&rɪm-ʌ​t̪		&		&		&*!	\\\hline
&&c.	&rim-a​t̪			&*!		&		&*	\\
\hline
&&d.	&rim-ʌ​t̪		&		&		&*!*	\\\hline\hline
\end{tabular}
\end{example}


A case like (\ref{Mayak_blood-sg}) might make it appear that the harmony\is{harmony!Mayak} conditions  play a crucial role in selecting the surface form with \{a​t̪, ʌ​t̪\}\down{\sc singular}. Careful consideration of such forms, however, shows that there is invariably some other condition that would be sufficient to determine the attested form. In (\ref{Mayak_blood-sg}), for example, *[atr] is crucial (necessary to avoid indeterminacy) and would independently eliminate the form  (\ref{Mayak_blood-sg}c) that also violates a harmonic condition:  [rɪm-a​t̪] (\ref{Mayak_blood-sg}a) is incidentally harmonic and  the harmony phonotactics\is{phonotactics} play no crucial role.\is{harmony!Mayak}

We turn now to suffixes with low vowels that do not  alternate. As seen in (\ref{Mayak-low-suffixes-copy}), there are three suffixes with invariant low vowels:\is{morph!invariant} one with retracted [a] and two with advanced [ʌ]. We begin with the retracted case: \{ak\}\down{\sc plural} invariably surfaces as retracted, regardless of root vowel.

\largerpage[-1]
\subsubsection{Nonalternating low retracted suffix} 
When an affix is invariant,\is{morph!invariant} the learner acquires a morph set with exactly one morph,\is{acquisition!morph set} in this case, \{ak\}\down{\sc pl}. If there is also only one morph in the stem morph set, compilation\is{compilation} gives only one option. That one option is the surface form, regardless of violations. The assessment in (\ref{Mayak_bee}) illustrates the point.\footnote{This same strategy is adopted in Construction Grammar:\is{Construction Grammar} \citet[27]{Valimaa-Blum:2011} observes ``[s]hould a lexical \is{morphology!Construction Grammar}\is{Construction Grammar!morphology}morpheme only have one sound shape in the lexicon, the schemas would consequently use this one form in all derivation.''} 


\begin{example}Assessment for  [kij-ak]\down{\sc bee-pl}
\label{Mayak_bee} \ee

{\it morph sets}: \{\ipa{kij}\}\down{\sc bee}; \{ak\}\down{\sc pl}\nopagebreak \\

\begin{tabular}{lllp{.6in}|c:c|c}
\hline\hline
&\multicolumn{3}{c|}{\sc bee-pl}	&\Mhiatr &\Mbkatr	&\Matr \\   \hline
&\rightthumbsup&a.		&kij-ak		&*		&		&*\\
\hline\hline
\end{tabular}
\end{example}

Such a word violates harmony since the advanced stem vowel is followed by a retracted suffix, but there are no options in the morph sets that would avoid this violation. In addition, there is an [atr] vowel and again no way to avoid a violation of \Matr.

\subsubsection{Nonalternating low advanced suffixes} 
We turn now to the two suffixes with invariant [ʌ],\is{morph!invariant}  where, regardless of the quality of the root vowel, the suffixes surface with the vowel [ʌ].

Like \{ak\}\down{\sc plural}, the morph sets of these suffixes have exactly one morph, i.e.\ \{ʌn\}\down{\sc singular}, and \{ʌr\}\down{\sc 1.singular}. Consequently, when these suffixes are attached to a non-alternating stem, i.e.\ a stem with only one advanced morph, the result is harmonic, simply because both stem and suffix are advanced. This is because, as with \{ak\}\down{\sc plural}, when there is also only one morph in the stem morph set, morph compilation gives only one option -- and with only one option, that is the surface form. The assessment in (\ref{Mayak_intestine2}) illustrates the point.


\begin{example}Assessment for  [ʔin-ʌn-i​t̪]\down{\sc intestine-singular}
\label{Mayak_intestine2} \ee
{\it morph sets}: \{\ipa{ʔin}\}\down{\sc intestine}; \{ʌn\}\down{\sc singular}; \{i​t̪\}\down{\sc singular}
\begin{tabular}{lllp{.7in}|c:c|c}
\hline\hline
&\multicolumn{3}{c|}{\sc intestine-sg}	&\Mhiatr &\Mbkatr	&\Matr \\   \hline
&\rightthumbsup&a.		&ʔin-ʌn-i​t̪		&	&	&***	\\
\hline\hline
\end{tabular}
\end{example}



The same result obtains with an alternating root, as illustrated in (\ref{Mayak_find-1sg}). While the harmonic prohibitions discussed here prohibit particular sequences\is{sequence!Mayak} where the second vowel is retracted, in this case the suffix vowel  is advanced. Hence neither of the harmony conditions is relevant. The motivation for such alternating roots, as noted in box  \ref{box:box-Mayak-regressive} on p.~\pageref{box:box-Mayak-regressive}, \label{regressive-reference2} is their behaviour before high advanced suffixes ({*}[rtr][atr, high]; \tier: vowels; \dom: morph, word).\is{morph!domain} Low vowels do not trigger regressive\is{harmony!regressive}\is{regressive harmony} harmony so the presence of a low advanced suffix does not override the general preference for retracted vowels in the root (*[atr]).\is{word!domain}


\begin{example}Assessment for [\ipa{ɟʊʊɟ-ʌr}]\down{\sc find-1sg}
\label{Mayak_find-1sg} \ee

{\it morph sets}: \{\ipa{ɟʊʊɟ, ɟuuɟ}\}\down{\sc find}; \{ʌr\}\down{\sc 1.singular} \\

\begin{tabular}{lllp{.6in}|c:c|c}
\hline\hline
&\multicolumn{3}{c|}{\sc hair-sg}	&\Mhiatr &\Mbkatr	&\Matr \\  \hline 
&\rightthumbsup&a.		&\ipa{ɟʊʊɟ-ʌr}	&	&	&*	\\
\hline
&&b.	&\ipa{ɟuuɟ-ʌr}						&	& 	&**!\\  \hline\hline
\end{tabular}
\end{example}



\subsection{A three-way contrast without enriched representations}

The three types of low vowel suffixes are summarised in \tabref{Mayak-low-atr-summary}.\is{contrast!ternarity}\is{contrast!Mayak vowels}


\begin{table} 
\caption{Summary of low vowel morph set types\label{Mayak-low-atr-summary}}
\begin{tabular}{l@{~~}lll}
\lsptoprule
\multicolumn{2}{l}{Morph set} & {Harmony role} & {Preference for retracted vowels}\\\midrule
alternating &\{a​t̪, ʌ​t̪\}\down{\sc sg} &crucial &plays a role, all else being equal \\
nonalternating &\{ak\}\down{\sc pl} &not crucial &plays a role, all else being equal \\
nonalternating &\{ʌn\}\down{\sc sg} &not crucial &plays a role, all else being equal \\
&\{ʌr\}\down{\sc 1.sg}\\
\lspbottomrule
\end{tabular}
\end{table}



As demonstrated, morph sets including low vowels do not exhibit a consistent pattern. When the learner encounters an advanced low vowel, for example, it is impossible to predict whether that vowel would have a retracted counterpart in another context or not -- the same unpredictability is found with a retracted low vowel and a possible advanced counterpart. Each morph set must simply be learned through exposure to appropriately affixed words.\is{acquisition!Morph Set Condition} There is no productive Morph Set Condition.\is{Morph Set Condition}\is{tongue root!Mayak harmony|)}


\subsection{Discussion and conclusion: Ternary distinctions}\label{section_three-way_distinctions}

\is{ternarity!opacity|(}The cases from English (\textsection\ref{URs_English-nasals})\il{English} and Mayak (\textsection\ref{section_Mayak_low_vowels}) highlight a problem that derives directly from the postulation of underlying representations. Consider again the Mayak case.\il{Mayak} As just seen in \tabref{Mayak-low-atr-summary}, low vowel suffixes are of three types: (i) alternating, [atr] or [rtr], (ii) nonalternating [rtr], (iii) nonalternating [atr]. Assuming underlying representations,\is{underlying representation} it might seem straightforward to assume that the nonalternating [rtr] forms are underlyingly retracted and that the nonalternating [atr] forms are underlyingly advanced.


\begin{example} \et{Putative underlying representations for Mayak}\label{putative-ur-Mayak}\smallskip\\
\begin{tabular}{@{}lll@{}}
   &{\it surface form} &{\it underlying representation} \\
a. &{[}rtr] &/rtr/ \\
b. &{[}atr] &/atr/
\end{tabular}
\end{example}

The problem is what then to do with the alternating forms.\is{alternation!Mayak} The expected options would be that underlying /atr/ becomes surface [rtr] in some context (a harmonic context in this case) or that underlying /rtr/ becomes surface [atr] in some context (again, a harmonic context). A core aspect of the underlying representation hypothesis is that once an underlying form is postulated, there is no ``look-ahead'' access to the eventual surface form. Hence if /rtr/ is posited, then one cannot look ahead to see that the underlying representation in question will become [atr] in appropriate contexts. The result therefore is that a postulated /rtr/ for an alternating form becomes indistinguishable from the /rtr/ of (\ref{putative-ur-Mayak}a), just as a postulated /atr/ for an alternating form would be indistinguishable from the /atr/ of (\ref{putative-ur-Mayak}b). A third type of representation, not directly motivated by the values of the feature in question, becomes necessary.



An entirely comparable problem arises in the English case.\il{English} Since the negative prefix [ɪ, ɪn, ɪm, ɪŋ] appears prevocalically as [ɪn] (\textit{inelegant})  while the preconsonantal environment conditions the appearance of different morphs, the plausible underlying representation for the prefix would be /ɪn-/ (\citealt{Allen:1978}). This means that however the rule or condition causing \is{assimilation!nasal place}assimilation is formulated, it needs to target coronal nasals. A problem analogous to that of Mayak now arises. If the underlying nasal at the end of /ɪn-/ is blind to the fact that it ultimately surfaces in a variety of ways ([ɪ, ɪn, ɪm, ɪŋ]), then it is indistinguishable from the underlying -- but nonalternating -- nasal of \textit{non-} and \textit{un-}.


These cases do not appear to be isolated or unusual. In Margi\il{Margi} (\citealt{Hoffmann:1963, Pulleyblank:1986, Archangeli+:2018routledge}), the morphs in a morph set may bear a consistently low tone, a consistently high tone, or may alternate between low and high. In Polish\il{Polish} (\citealt{Bethin:1978, Sanders:2003}), morph set members may exhibit a back vowel that is consistently mid, a back vowel that is consistently high, or may exhibit back vowels that alternate between mid and high. In Nuu-chah-nulth\il{Nuu-chah-nulth} (\citealt{Davidson:2002, Stonham:1990, Kim:2003, Archangeli+:2018Henry}), barring the effect of certain ``shortening'' or ``lengthening'' suffixes, morph sets may have vowels which are consistently short, consistently long, or that alternate between short and long. In Barrow Inupiaq\il{Barrow Inupiaq} (\citealt{Kaplan:1981, Archangeli+:1994}) and Kashaya\il{Kashaya} (\citealt{Buckley:1994}), certain apparently phonetically identical segments diverge behaviourally, where one set acts as a target and not a trigger and the other as a trigger and not as a target. And this list goes on: in \Sec\ref{URs_Polish_section}, we discuss another three-way alternation, the Polish {\it yer} pattern. 

Since postulating underlying representations is the source of the difficulty, any theory assuming such representations must seek a solution that involves either enrichment of the theory's representations or enrichment of the theory's architecture. Both have been proposed. For example, underspecification\is{underspecification!ternarity}\is{ternarity!underspecification} has been proposed as a solution to cases such as those seen in Mayak and Margi. For Margi,\il{Margi} \citet{Pulleyblank:1986} suggests that the consistently low \is{morpheme!Margi}morphemes in Margi have a lexical L, the consistently high morphemes have a lexical H, and the alternating morphemes have no lexical tone; the surface tone of an alternating morpheme is determined contextually or by\is{tone!default}\is{default!tone} default.\footnote{We use the term ``morpheme''\is{morpheme} advisedly here, referring to earlier analyses. An Emergent account would represent the distribution\is{distribution!morph set} in terms of morph sets with only low-toned morphs, only high-toned morphs, or both low- and high-toned morphs respectively.} For Mayak, \citet{Andersen:1999-Vs} proposes that the alternating low vowel suffixes  are unspecified for the tongue root\is{tongue root!underspecification}\is{underspecification!tongue root} feature, whereas the nonalternating suffixes are underlyingly specified, and for Barrow Inupiaq,\il{Barrow Inupiaq} \citet{Archangeli+:1994} argues for an unspecified vowel contrasting with a (partially) specified /i/,\is{contrast!Barrow Inupiaq vowels} both surfacing as [i] in most environments. In a \is{prosody}prosodic case such as Nuu-chah-nulth\is{contrast!Nuu-chah-nulth length} length,\il{Nuu-chah-nulth} representational enrichments have been proposed for syllable structure:\is{syllable} \citet{Stonham:1990} proposes two types of long vowels, distinguished by reference to a syllable nucleus and a syllable rhyme; \citet{Kim:2003} proposes a distinction between two types of long vowels underlyingly, one prelinked to two moras and one that is only partially linked. Architecturally, cases such as the English\il{English} one have been proposed to involve multiple \is{Lexical Phonology}lexical strata (\citealt{Kiparsky:1982lexical-phonology, Mohanan:1986}). If the rule of \is{assimilation!nasal place}place assimilation is assigned to a stratum where /ɪn-/ is attached but not to a \is{Lexical Phonology}stratum where /n\ipa{ɑ}n-/ and /ʌn-/ are attached, then the differences in behaviour can be accounted for \is{morphology}morphologically.

These enrichments are necessary  in rule-based as well as constraint-based frameworks. In a rule-based framework, the enrichment is to ensure that the relevant rule applies to only a subset of the cases that appear to match the rule's structural description. In a constraint-based framework such as Optimality Theory,\is{Optimality Theory} something is needed to ensure that the combination of Gen\is{Gen} plus markedness constraints results in one form being optimal in one type of case but some other form being optimal in a different class of cases. For example, faithfulness to one type of structural input (e.g., unspecified) works differently from faithfulness to a different type of structural input (e.g., specified). Rule-based and constraint-based frameworks are comparable in this regard.

Strikingly, in the Emergent Grammar approach,\is{Emergent Grammar} there is no comparable problem. At all levels, the creation of higher level structure does not entail destruction of lower level structure. This has been amply motivated in the exemplar\is{Exemplar Theory} literature (\citealt{Pierrehumbert:2001ed, Pierrehumbert:2003PhoneticDiversity}, etc.). In the current context, when the learner establishes the relation between [a​t̪] and [ʌ​t̪] in Mayak, for example, this does not erase the actual words in which [a​t̪] and [ʌ​t̪] occur, nor does the postulation of a morph set \{a​t̪, ʌ​t̪\} erase in some way the information that this form involves two possible surface forms.

 
At issue in the Emergent approach are two questions:\is{Emergent Grammar}\is{morph set}

\begin{enumerate}[label=(\roman*),noitemsep,topsep=0pt]
\item Is there a  relation between the members of a morph set?  
\item If so, does the relation define productive augmentations of morph sets? 
\end{enumerate}

The answer to (i) is that if there are multiple members in a morph set, the {\Identity}\is{Identity Principle} predicts a high degree of similarity\is{similarity!morphs} among those members. The human's predilection for generalisation\is{generalisation} leads us to expect a high degree of systematic variation: relations of a systematic nature are expected between the members of a morph set, our Morph Set Relations; nonsystematic differences are expected to be less common and more difficult to acquire.\is{acquisition} In cases like the Yangben vowels discussed in \Sec\ref{Yangben-MSRs-section} and the Warembori voiced obstruents examined in \Sec\ref{Warembori_section}, large numbers of morph pairs are related in this kind of systematic manner, warranting the grammaticisation of Morph Set Relations.\largerpage


Stability\is{stability} in a phonological system is predicted when the pressures of phonotactics engage smoothly with systematic options made available by morph sets. When morph sets productively provide precisely the options in morphs needed to satisfy word-domain conditions,\is{domain \dom} the grammar\is{grammar} is able to maximally satisfy the phonotactic conditions that it imposes. Compare, for example, (\ref{Mayak_blood-sg}) with (\ref{Mayak_bee}); by having both advanced and retracted morphs in its morph set, the polymorph {\sc sg} suffix in (\ref{Mayak_blood-sg}) allows the harmony\is{harmony!Mayak} phonotactics to be satisfied in a way that the monomorph {\sc pl} suffix in (\ref{Mayak_bee})  cannot. In response to (ii), therefore, we would expect Morph Set Relations to tend to define productive augmentation of morph sets: the absence of a paired morph may render it impossible to satisfy some word-level\is{word} phonotactic.\is{phonotactics} Hence where an unpaired morph is dispreferred, Morph Set Conditions\is{Morph Set Condition} are imposed to penalise non-compliant (``singleton'') morph sets. In Warembori,\il{Warembori} stop-initial morphs are systematically paired with fricative-initial morphs (\ref{Warembori-C-MSR}); neither stop-initial nor fricative-initial morphs are allowed as singletons, captured in the symmetric Warembori MSC (\ref{Warembori_MSC}); in Yangben,\il{Yangben} advanced morphs are paired with retracted morphs (MSR\down{\sc [tr]}, (\ref{Yangben-rtr-root-MSR}), Chapter \ref{ch3}); pairing is asymmetrically imposed, with singleton [rtr] morphs penalised by the Yangben MSC\down{\sc [tr]} but with singleton [atr] morphs allowed. It is also possible for both pairs and singletons to be well-formed. In Polish, discussed in \Sec\ref{section_Polish},  there is a systematic relation observed between many pairs of morphs (enough to establish MSR\down{\it yer}), yet both the sets of paired morphs and the two types of singleton morph sets occur. That is, in Polish,\il{Polish} there is an MSR\is{Morph Set Relation!Polish} but no corresponding MSC.



There is no expectation that such distinctions in the structure of morph sets should be limited to any particular type of feature. That  distinctions occur for tongue root (Mayak),\il{Margi} place features (English),\il{English} tone (Margi),\il{Margi} length (Nuu-chah-nulth),\il{Nuu-chah-nulth} and so on is unproblematic. The prediction for the Emergent framework is that any property that can give rise to lexical distinctions could give rise to the sort of three-way distinctions that we have observed in this section. \il{Mayak|)}


Moreover,  the theory predicts a maximum differentiation of three: for any contrast\is{contrast!ternarity} involving {\it P} and {\it \up{$\wedge$}P}, the complement class\is{complement class} of {\it P}, the set of options includes maximally the morph sets including \{P\}, \{\up{$\wedge$}P\}, \{P, \up{$\wedge$}P\}. Whether the restriction to a maximally three-way possibility holds or doesn't hold of an approach invoking abstract\is{abstractness!underlying representation}\is{underlying representation!abstractness} underlying representations depends on the specific nature of the representation or architectural enrichment invoked.
\is{ternarity!opacity|)}\is{ternarity|)}

\section{Contrast (re)location: Kinande tone shift} \label{section_Kinande}\largerpage

\label{URs_Kinande_H-shift}\il{Kinande}\il{Kinande|(}\is{tone!Kinande|(}
\is{contrast!Kinande tone location}
In the next two sections, we address examples which illustrate another  key difference between the concepts of  morph set and of underlying representation: while members of morph sets are concrete and directly related to observable forms, underlying representations may be abstract, only indirectly related to observable forms -- Hockett's \is{theoretical base form}``theoretical base form''.  The difference is illustrated by schemas in (\ref{morph-schemas-1-revised}), which show possible lexical representations for an observed form [X] under the different frameworks.  Given [X]\down{α}, the  one-member  morph set in (\ref{morph-schemas-1-revised}a) is the only possible corresponding morph set  under Emergence. With underlying representations there are two types of representations: (i) a concrete representation, depicted in (\ref{morph-schemas-1-revised}b), where  underlying /X/  corresponds directly to surface [X], and (ii) an abstract\is{abstractness!underlying representation}\is{underlying representation!abstractness} representation, depicted in (\ref{morph-schemas-1-revised}c),  where underlying /Z/ corresponds indirectly to  surface [X]. (Again, ``$\upalpha$'' indicates the syntactic/semantic unit.)

\begin{example}\et{Lexical representations given [X]\down{$\upalpha$} \label{morph-schemas-1-revised}}\\\multicolsep=.25\baselineskip
\begin{multicols}{3}\raggedcolumns
\ea Emergence:\\
    \{X\}\down{$\upalpha$}\columnbreak
\ex Concrete UR\\
    \begin{forest}
    [{[X]} [/X/,name=X]]
    \node [right=7pt of X.base,inner sep=0pt] {\down{\sc $\upalpha$}};
    \end{forest}\columnbreak
\ex Abstract UR\\
    \begin{forest}
    [{[X]} [/Z/,name=Z]]
    \node [right=7pt of Z.base,inner sep=0pt] {\down{\sc $\upalpha$}};
    \end{forest}
\z
\end{multicols}
\end{example}

In the abstract case, (\ref{morph-schemas-1-revised}c), the observed form is not postulated at the abstract level (as /X/) and the abstract form is not realised (as [Z]) on the surface. 

We illustrate the abstract schema of (\ref{morph-schemas-1-revised}c) with discussion of tonal distribution in Kinande.\footnote{A similar example in a different featural domain is found in Esimbi\il{Esimbi}, where prefix vowel height is determined largely by the root to which a prefix is attached. See \citet{Hyman:1988} for a standard account of Esimbi, and \citet{Archangeli+:2015_Frontiers} for an EG account.\is{Emergent Grammar}} Kinande, or Nande, (glottocode nand1264), is a Narrow Bantu\il{Bantu} language, D42 in Guthrie's classification (\citealt{Guthrie:1967}), spoken in the Democratic Republic of the Congo\is{Democratic Republic of the Congo} by around 900,000 in 1991 (\citealt{ethnologue:kinande}). Data are primarily taken from \citet{Mutaka:1994} and \citet{Akinlabi+:2001}.\footnote{Big thanks to Philip Ngessimo Mutaka for thought-provoking discussion of this section. Data from \citet{Mutaka:1994} are indicated with M; from \citet{Akinlabi+:2001} by AM. A few forms were provided by Philip Ngessimo Mutaka (personal communication), indicated by Mpc.}


\subsection{The Kinande tone puzzle}\largerpage[1.5]

\is{tone|(}The two core observations about Kinande tone are that (i)  in most cases the prefix tone, high (H) or low (L), is determined by the  root that the prefix is attached to, not by the prefix itself, and (ii) verb roots are typically L on the surface, while noun roots are either L or H-L (\citealt{Hyman+:1985, Mutaka:1994, Akinlabi+:2001}).  Both points are illustrated  in (\ref{URs_Kinande_nouns-n-verbs}).\footnote{In our transcriptions, vowel quality is represented in IPA, rather than with the orthographic conventions used in \citet{Mutaka:1994}; tongue root values are indicated and all tones, both L and H, are marked. We do not address tongue root\is{harmony!Kinande}\is{tongue root!Kinande harmony} harmony here, but present all data in a way that is consistent with the correct surface forms; a more complete treatment of Kinande would include both advanced and retracted morphs in the appropriate morph sets. Roots are indicated by square brackets. Noun classes are indicated by ``C'' followed by the number of the class.} In (\ref{URs_Kinande_nouns-n-verbs}a, b), the prefixes immediately to the left of the root surface as L, while in (\ref{URs_Kinande_nouns-n-verbs}c, d)  the immediately pre-root prefixes surface as H -- the difference results from the root to which the prefixes are attached.\footnote{We restrict our discussion here to the tonal forms that occur in \is{phrase!Kinande}non-phrase-final position; in phrase-final position, an additional penultimate H tone is present in many forms (\citealt{Hyman+:1985, Mutaka:1994}). Our analysis here follows that of \citet{Archangeli+:2015_K-tone}; see that work for  a  more complete discussion of the Kinande basic tone pattern and see \citet{Archangeli+:2014abidjan} for extension to a class of more complex verbal tone patterns.}


\begin{example} \et{Kinande phrase-medial verbs \& nouns}\label{URs_Kinande_nouns-n-verbs}\smallskip\\
\begin{tabular}{@{}llllll@{}}
a.  &\multicolumn{2}{l}{\it Verbs: L prefixes} \\
    &\ipa{\`{ɛ}-{r\`{ɪ}}-[h\`{ʊ}m]-à màɡ\'{ʊ}ːl\`{ʊ}}&`to hit Magulu'  &AM336 \\
&\ipa{\`{ɛ}-{r\`{ɪ}}-nà-[h\`{ʊ}m]-\`{ɪ}r-à màɡ\'{ʊ}ːl\`{ʊ}}&`to just hit for Magulu' &AM336  \ee

b.  &\multicolumn{2}{l}{\it Nouns: L prefixes} \\
    &\ipa{\`{ɔ}-{k\`{ʊ}}-[ɡ\`{ʊ}l\`{ʊ}] kù-l{\í}ːt\`{ɔ}}&`heavy leg' (C15) &M155\\
%&\ipa{\`{ɔ}-{m\`{ʊ}}-[ɡ\'{ɔ}nɡ\`{ɔ}] mù-l{\í}ːt\`{ɔ}} &`heavy back' (C3) &M155\ee
&\ipa{\`{a}-{k\`{a}}-[ɡ\'{ɔ}nɡ\`{ɔ}] kà-lw{\'ɛ}ːr\`{ɛ}} &`the back is sick' (C12) &M158\ee

c.  &\multicolumn{2}{l}{\it Verbs: H prefixes} \\
    &\ipa{\`{ɛ}-{r\'{ɪ}}-[t\`{ʊ}m]-à màɡ\'{ʊ}ːl\`{ʊ}} &`to send Magulu' &AM338\\
&\ipa{\`{ɛ}-{r\`{ɪ}}-ná-[t\`{ʊ}m]-\`{ɪ}r-à màɡ\'{ʊ}ːl\`{ʊ}} &`to just send for Magulu' &AM338\ee

d.  &\multicolumn{2}{l}{\it Nouns: H prefixes} \\
    &\ipa{\`{ɔ}-{k\'{ʊ}}-[b\`{ɔ}k\`{ɔ}] kù-l{\í}ːt\`{ɔ}} &`heavy arm'  (C15) &M155\\
%&\ipa{à-{ká}-[h\'{ʊ}kà] k\`ə-l{\í}ːt\`{ɔ}} &`heavy insect' (C12) &M156
&\ipa{à-{ká}-[h\'{ʊ}kà] kà-lw{\'ɛ}ːr\`{ɛ}} &`the insect is sick' (C12) &M158
\end{tabular}
\end{example}


A standard autosegmental\is{autosegmental} analysis of this kind of distribution  postulates an \is{underlying representation!Kinande tone}\is{tone!Kinande underlying representation}underlying feature on the root that spreads onto the affix, and then delinks from its original position. \citet{Mutaka:1994} does exactly that for Kinande, positing an underlying high tone on the stem that surfaces one mora to the left of its underlying location via a derivation  illustrated in \figref{okuboko_derivation}. \is{noniterativity!Kinande Noniterative Association}Noniterative Association causes the H tone to associate to a preceding mora, while Delink Rightmost removes the association between the H tone and its underlying host. L tones are assigned to all unspecified vowels by default\is{default!tone}\is{tone!default} (\citealt{Pulleyblank:1986}). We illustrate this with the examples \ipa{\`{ɔ}-{k\`{ʊ}}-[ɡ\`{ʊ}l\`{ʊ}]...} (\ref{URs_Kinande_nouns-n-verbs}b) and \ipa{\`{ɔ}-{k\'{ʊ}}-[b\`{ɔ}k\`{ɔ}]...} (\ref{URs_Kinande_nouns-n-verbs}d).\footnote{In \figref{Kinande_UR-SR_tone_schemas}, we give the hypothesised \is{underlying representation!Kinande tone}\is{tone!Kinande underlying representation}underlying representations for the noun and verb roots found in (\ref{URs_Kinande_nouns-n-verbs}), including the cases illustrated in \figref{okuboko_derivation}. The roots with H tones undergo Noniterative \is{noniterativity!Kinande Noniterative Association}Association and Delink Rightmost to derive the surface tones.}


\begin{figure} \caption{Kinande root-tone based derivations\label{okuboko_derivation}}\small
\begin{minipage}[b]{.3\linewidth}
~
\end{minipage}\begin{minipage}[b]{.33\linewidth}
a. `leg'
\end{minipage}\hfill\begin{minipage}[b]{.33\linewidth}
b. `arm'
\end{minipage}\\
\begin{minipage}[b]{.3\linewidth}
\textit{input representation}
\end{minipage}\begin{minipage}[b]{.33\linewidth}
\begin{forest} for tree = {s sep=1pt} 
    [,phantom [{ɔ}][-][k][{ʊ}][-][ɡ][{ʊ}][l][{ʊ}]] 
\end{forest} 
\end{minipage}\hfill\begin{minipage}[b]{.33\linewidth}
\begin{forest} for tree = {s sep=1pt,grow'=90} 
    [,phantom [ɔ][-][k][ʊ][-][b][ɔ [H]][k][ɔ] ] 
\end{forest} 
\end{minipage}\\
\begin{minipage}[b]{.3\linewidth}
\textit{Noniterative Association}\is{noniterativity!Kinande Noniterative Association}
\end{minipage}\begin{minipage}[b]{.33\linewidth}
    n/a 
\end{minipage}\hfill\begin{minipage}[b]{.33\linewidth}
\begin{forest} for tree = {s sep=1pt,grow'=90} 
    [,phantom [ɔ][-][k][ʊ,name=u][-][b][ɔ [H,name=H]][k][ɔ] ] 
    \draw [dotted] (u.north) -- (H.south);
\end{forest}
\end{minipage}\\
\begin{minipage}[b]{.3\linewidth}
{\it Delink Rightmost}
\end{minipage}\begin{minipage}[b]{.33\linewidth}
    n/a 
\end{minipage}\hfill\begin{minipage}[b]{.33\linewidth}
\begin{forest} for tree = {s sep=1pt,grow'=90} 
    [,phantom [ɔ][-][k][ʊ,name=u][-[H,name=H,no edge,xshift=3pt]][b][ɔ,name=o][k][ɔ] ] 
    \draw (u.north) -- (H.south);
    \draw (o.north) -- (H.south) node [midway,rotate=-45] {||};
\end{forest}
\end{minipage}\\
\begin{minipage}[b]{.3\linewidth}
\textit{Default L tone}
\end{minipage}\begin{minipage}[b]{.33\linewidth}
\begin{forest} for tree = {s sep=1pt,grow'=90} 
    [,phantom [ɔ,name=o][-][k][ʊ,name=u1][- [L,no edge,name=L]][ɡ][ʊ,name=u2][l][ʊ,name=u3]] 
    \path (L.south) edge (o.north)
                    edge (u1.north)
                    edge (u2.north)
                    edge (u3.north);
\end{forest} 
\end{minipage}\hfill\begin{minipage}[b]{.33\linewidth}
\begin{forest} for tree = {s sep=1pt,grow'=90} 
    [,phantom [ɔ[L]][-][k][ʊ[H]][-][b][ɔ,name=o1][k[L,no edge,name=L]][ɔ,name=o2] ] 
    \path (L.south) edge (o1.north)
                    edge (o2.north);
\end{forest} 
\end{minipage}\\
\begin{minipage}[b]{.3\linewidth}
\textit{output representation} 
\end{minipage}\begin{minipage}[b]{.33\linewidth}
[\ipa{\`{ɔ}k\`{ʊ}ɡ\`{ʊ}l\`{ʊ}}]...
\end{minipage}\hfill\begin{minipage}[b]{.33\linewidth}
[\ipa{\`{ɔ}k\'{ʊ}b\`{ɔ}k\`{ɔ}}]...
\end{minipage}\\
\end{figure} 

\begin{figure}\small
\caption{Types of input/output relations in spreading account of Kinande\label{Kinande_UR-SR_tone_schemas}} 
\centering
\begin{tabular}{@{}l@{~}ll@{~}ll@{~}ll@{~}l@{}}
\multicolumn{4}{@{}l}{\it Underlying representations}\\
a.&
/\ipa{kʊ}/  &b.&/\ipa{hʊm}/, /\ipa{ɡʊlʊ}/ &c.& \begin{forest} [H,asr,calign child=3 [/] [t][ʊ,edge=draw,baseline][m][/]] \end{forest}, 
                                               \begin{forest} [H,asr,calign child=3 [/][b][ɔ,edge=draw,baseline][k][ɔ][/] ]\end{forest}
&d. & \begin{forest} [H,asr,calign child=6 [/][ɡ][ɔ][n][ɡ][ɔ,edge=draw,baseline][/]] \end{forest},
      \begin{forest} [H, asr, calign child=4 [/][h][ʊ,baseline,edge=draw][k][a,edge=draw][/]]
      \end{forest}\\
\multicolumn{4}{@{}l}{\it Surface forms}\\
 
& \begin{forest} [L,asr,calign child=3 [{[}] [k][ʊ,edge=draw,baseline] [{]}]] \end{forest}, \begin{forest} [H,asr,calign child=3 [{[}] [k][ʊ,edge=draw,baseline] [{]}] ] \end{forest}
& & \begin{forest} [L,asr,calign child=3 [{[}] [h][ʊ,edge=draw,baseline][m] [{]}] ] \end{forest}, \begin{forest}  [L,asr,calign child=4 [{[}][ɡ][ʊ,edge=draw,baseline][l][ʊ,edge=draw][{]}]] \end{forest}
& & \begin{forest} [L,asr,calign child=3 [{[}] [t][ʊ,edge=draw,baseline][m] [{]}] ] \end{forest}, \begin{forest} [L,asr,calign child=4 [{[}] [b][ɔ,edge=draw,baseline][k][ɔ,edge=draw][{]}]] \end{forest}
& & \begin{forest}
        [,phantom,asr
          [{[},tier=word] [ɡ,tier=word][H[ɔ,edge=draw,tier=word,baseline]][n,tier=word][ɡ,tier=word][L[ɔ,edge=draw,tier=word]] [{]},tier=word]
        ] 
    \end{forest}, 
    \begin{forest}
        [,phantom,asr
          [{[},tier=word] [h,tier=word][H[ʊ,tier=word,edge=draw]][k,tier=word][L[a,edge=draw,baseline,tier=word]] [{]},tier=word]
        ]
    \end{forest}                                                                                
\end{tabular}
\end{figure}

A comparable analysis is proposed in an optimality theoretic account\is{Optimality Theory} in \citet{Akinlabi+:2001}: a H tone is introduced as in Mutaka's autosegmental account, but prevented from being realised in that position by an anti-faithfulness ``{\sc AvoidSponsor}'' constraint. What is remarkable about such analyses of Kinande is that the \is{underlying representation!Kinande tone}\is{tone!Kinande underlying representation}underlyingly postulated H tones never surface attached to the vowel they start with. Rather, they attach to a neighbour and detach from the underlying vowel host. This gives rise to  the kinds of input/output relations shown in \figref{Kinande_UR-SR_tone_schemas} for sample morphs from (\ref{URs_Kinande_nouns-n-verbs}).

As  seen by comparing the \is{underlying representation!Kinande tone}\is{tone!Kinande underlying representation}underlying representations and surface forms in \figref{Kinande_UR-SR_tone_schemas}, there is no direct relation between any of the tones of the underlying \is{morpheme!Kinande}morphemes and the tones observed in  the surface forms. A vowel that is underlyingly toneless does not surface as toneless; however, it can surface in multiple ways tonally -- as variably L or H (\figref{Kinande_UR-SR_tone_schemas}a), as invariably L (\figref{Kinande_UR-SR_tone_schemas}b), or as invariably H (\figref{Kinande_UR-SR_tone_schemas}d). In contrast, a vowel that is underlyingly H surfaces  invariably  as L, \figref{Kinande_UR-SR_tone_schemas}c, d, never as H. Representations of the type in \figref{Kinande_UR-SR_tone_schemas}d show  a toneless-H sequence\is{sequence!Kinande} paired with a H-L sequence, due to the shifting H tone. In this tone-shift approach, underlying representations have moved away from the structuralist\is{structuralism} notion of building blocks. Rather, \is{reverse engineering!underlying representation}underlying representations are  worked out by the reverse engineering\is{reverse engineering} of rules: here, prefixal H tone is assumed to result from the tone of the following \is{morpheme!Kinande}morpheme and this is accomplished by spreading, ergo the following (L-toned) vowel must be underlyingly H. As seen in Kinande, this means that underlying representations must be highly abstract entities -- the underlying H tones are in \is{underlying representation!abstractness}locations where they are not observed on the surface, abstractness\is{abstractness!underlying representation}\is{underlying representation!abstractness} of the type schematised in (\ref{morph-schemas-1-revised}c).


\subsection{The Emergent analysis: Lexical classes in Kinande}

In a framework like Emergence, where representations reflect observed forms directly, such an analysis is not possible. How, then, is the Kinande tone ``shift'' to be understood? Abstracting away from vowel quality changes due to tongue root harmony,\is{harmony!Kinande}\is{tongue root!Kinande harmony} observation establishes that the relevant prefixes have two forms, \{\ipa{{m\`{ʊ}}, m\'{ʊ}}\}\down{\sc c3}, \{\ipa{{kà}, ká}\}\down{\sc c12}, \{\ipa{k\`{ʊ}, k\'{ʊ}}\}\down{\sc c15}, \{\ipa{r\`{ɪ}, r\'{ɪ}}\}\down{\sc infinitive} and \{{\ipa{{nà}, ná}}\}\down{\sc just}; in contrast, roots typically have one form, e.g.\  \{\ipa{ɡ\`{ʊ}l\`{ʊ}}\}\down{\sc leg}, \{\ipa{b\`{ɔ}k\`{ɔ}}\}\down{\sc arm} and \{\ipa{h\'{ʊ}kà}\}\down{\sc insect}, etc. The emergent morph sets correspond directly to the surface\is{surface-to-surface} forms of (\ref{morph-schemas-1-revised}). Yet there is still a  difference to capture since there are two classes of roots -- a class whose prefix bears  H tone --  \{\ipa{b\`{ɔ}k\`{ɔ}}\}\down{\sc arm}, \{\ipa{h\'{ʊ}kà}\}\down{\sc insect}, etc. -- and a class whose prefix bears L tone -- \{\ipa{ɡ\`{ʊ}l\`{ʊ}}\}\down{\sc leg}, \{\ipa{ɡ\'{ɔ}nɡ\`{ɔ}}\}\down{\sc back},  etc. The  distribution of H tone cannot be predicted from any observable phonological property of the roots concerned -- it is phonologically arbitrary.   We use the label α\ for the class whose prefixes  typically have H tone. Lexical items like \{\ipa{b\`{ɔ}k\`{ɔ}}\}\down{\sc arm} would be members of the α\ class, \{\ipa{b\`{ɔ}k\`{ɔ}}\}\down{{\sc arm}, α}.\footnote{See box \ref{box:box-alpha-classes}, p.\ \pageref{box:box-alpha-classes} for discussion of arbitrary\is{class!arbitrary} lexical classes (like the α\ class) and their interaction with conditions in the grammar.}\largerpage

Reviewing the distribution\is{distribution} of H tone in Kinande, we see that in general  Kinande vowels have L tone (\citealt{Pulleyblank:1986, Mutaka:1994}), leading to positing a general prohibition on high tones, *[H] (\ref{K-tone_cond}a). Nonetheless, in the variable Kinande prefix morph sets, the H-toned morph does surface with roots of class α, achieved by a condition stating that α\ roots must not occur with a preceding L tone: {*[L]\{...\}\down{α}}, given in (\ref{K-tone_cond}b).\footnote{Since surface moras in Kinande have either a H or L tone, this condition could equivalently be expressed positively, by a condition requiring H tones before the {α} class, H\{...\}\down{α}. This was how the condition was expressed in \citet{Archangeli+:2015_K-tone}. However, as laid out in (\ref{types-schema-revised}), (\ref{phonotactics-schema-revised}), and (\ref{paradigmatic-schema-revised}) in Chapter \ref{ch6}, we are assuming all conditions are expressed as prohibitions, hence this, too, is expressed negatively.} (The general formalism for type conditions  like (\ref{K-tone_cond}a) and syntagmatic conditions like  (\ref{K-tone_cond}b) are introduced in \textsection\ref{section_choice}; the schemas are summarised in Chapter \ref{ch6}, in (\ref{types-schema-revised}) and (\ref{phonotactics-schema-revised}) respectively.) Both conditions are given a word domain: (\ref{K-tone_cond}a) is not true generally of individual morphs and (\ref{K-tone_cond}b) requires two morphs so must hold of a domain larger than the morph.\is{well-formedness condition!Kinande}\is{well-formedness condition!tone}\is{tone!well-formedness condition}


\begin{example} {\et{Kinande tone conditions}}\label{K-tone_cond}
\ea *[H], \tier: vowels, \dom: word\\\is{word!domain}
With a focus on vowels, assign a violation to a word for each high-toned vowel.
\ex *[L]\{...\}\down{α}, \tier: vowels, \dom: word\\
With a focus on vowels, assign a violation to a word  for each sequence of a low-toned vowel preceding a member of the α\ class.
\z
\end{example}

Assessment is illustrated in (\ref{URs_Kinande_noun_selection}). With \REF{URs_Kinande_noun_selectiona}, there is no α\ class noun stem, so *[L]\{...\}\down{α} is irrelevant. In this case, *[H] is the decider, selecting the L-toned prefix morph. In \REF{URs_Kinande_noun_selectionb}, the noun stem is a member of the α\ class,  so *[L]\{...\}\down{α} is relevant: the H-toned prefix morph wins out over the  L-toned prefix morph.



\begin{example} \et{Kinande noun assessments} (vowel alternations not represented in morph sets) \label{URs_Kinande_noun_selection}
\ea \ipa{\`{ɔ}-{k\`{ʊ}}-[ɡ\`{ʊ}l\`{ʊ}]}...\\{\it morph sets:} \{\ipa{\`{ɔ}}\}\down{\sc det}, \{\ipa{k\'{ʊ}, k\`{ʊ}}\}\down{\sc c15}, \{\ipa{g\`{ʊ}l\`{ʊ}}\}\down{\sc leg}\smallskip\\\label{URs_Kinande_noun_selectiona}
\begin{tabular}{lll ||c|cc}
\hline\hline
	&&{\sc det-c15-leg}		&*[L]\{...\}\down{α}&*H\\
\hline
\rightthumbsup
&a.	&\ipa{\`{ɔ}-{k\`{ʊ}}-[ɡ\`{ʊ}l\`{ʊ}]}...	&	&	\\
\hline
&b.	&\ipa{\`{ɔ}-{k\'{ʊ}}-[ɡ\`{ʊ}l\`{ʊ}]}...&&*! \\
\hline \hline
\end{tabular}

~\ee

\ex \ipa{\`{ɔ}}-\ipa{k\'{ʊ}}-[\ipa{b\`{ɔ}k\`{ɔ}}\down{α}]...\\{\it morph sets:} \{\ipa{\`{ɔ}}\}\down{\sc det}, \{\ipa{k\'{ʊ}, k\`{ʊ}}\}\down{\sc c15}, \{\ipa{b\`{ɔ}k\`{ɔ}}\}\down{\sc arm.}\down{α}\\\label{URs_Kinande_noun_selectionb}
\begin{tabular}{lll ||c|cc}
\hline\hline
&&{\sc det-c15-arm}\down{α}&*[L]\{...\}\down{α}&*H\\
\hline
&c.	&\ipa{\`{ɔ}}-\ipa{k\`{ʊ}}-[\ipa{b\`{ɔ}k\`{ɔ}}\down{α}]...&*!	&	\\
\hline
\rightthumbsup
&d.	&\ipa{\`{ɔ}}-\ipa{k\'{ʊ}}-[\ipa{b\`{ɔ}k\`{ɔ}}\down{α}]...&&* \\
\hline \hline
\end{tabular}
\z
\end{example}



\subsection{Consequences}

There are three desirable consequences of this analysis. First, we have already seen morph sets with a single L-toned morph and morph sets with both H- and L-toned morphs. There is the logical possibility of a morph set with a single H-toned morph, a morph which surfaces as H-toned regardless of the context.\is{morph!invariant} In fact, such  prefixes exist,  for example \{ká\}\down{\sc continuous}, illustrated in (\ref{URs_Kinande_stable-H}) where the continuous is compared with infinitives for the same verb (repeated from (\ref{URs_Kinande_nouns-n-verbs})).\footnote{As noted in \citet{Mutaka:2001}, the marker {\it ka} exhibits different tonal behaviour in tenses other than the present. We do not address these tenses here.}

\begin{example} \et{Invariant H tones in Kinande (M219)} \label{URs_Kinande_stable-H}\smallskip\\
\begin{tabular}{@{}ll@{}}
%	\{ká\}\down{\sc cont}	&&\{\ipa{rì, rí}\}\down{\sc infinitive}\\
%\ipa{[tù-{ká}-[[hùm-à]]]...} &`we are hitting'&\ipa{[è-{rì}-[[hùm-à]]]... }&`to hit'\\%95, 116
%\ipa{[tù-{ká}-[[tùm-à]]]...} &`we are sending'&\ipa{[è-{rí}-[[tùm-à]]]... }&`to send'\\%95, 116
\{ká\}\down{\sc continuous}	&   \\                                                  
%\ipa{t\`{ʊ}-{ká}-[h\`{ʊ}m-à] vàl\`{ɪ}náːnd\`{ɛ}} &`we are hitting Valinande' 
\ipa{t\`{ʊ}-{ká}-[h\`{ʊ}m-à] vàl\`{ɪ}náːnd\`{ɛ}} &`we are hitting Valinande'\\
%\ipa{t\`{ʊ}-{ká}-[t\`{ʊ}m-à] vàl\`{ɪ}náːnd\`{ɛ}} &`we are sending  Valinande'
\ipa{t\`{ʊ}-{ká}-[t\`{ʊ}m-à] vàl\`{ɪ}náːnd\`{ɛ}} &`we are sending  Valinande'\\   
\end{tabular}\smallskip\\
\noindent \begin{tabular}{@{}ll@{}}
\multicolumn{2}{@{}l}{\{\ipa{r\`{ɪ}, r\'{ɪ}}\}\down{\sc infinitive}}\\
\ipa{\`{ɛ}-{r\`{ɪ}}-[h\`{ʊ}m-à]... }&`to hit...'\\%95, 116
\ipa{\`{ɛ}-{r\`{ɪ}}-[h\`{ʊ}m-à]... }&`to hit...'\\%95, 116
\ipa{\`{ɛ}-{r\'{ɪ}}-[t\`{ʊ}m-à]... }&`to send...'\\%95, 116
\ipa{\`{ɛ}-{r\'{ɪ}}-[t\`{ʊ}m-à]... }&`to send...'\\%95, 116    
\end{tabular}
\end{example}

Under the Emergent analysis, there is simply a single morph in this morph set, \{ká\}\down{\sc continuous}. Since there are no other options, a H tone is realised regardless of the type of verb it attaches to. In contrast, an analysis with abstract underlying representations\is{abstractness!underlying representation}\is{underlying representation!abstractness} is challenged because these H tones cannot arise from the \is{morpheme!Kinande}morpheme to which they are attached: \{\ipa{h\`{ʊ}m}\}\down{\sc hit} does not typically appear with a preceding H; see (\ref{URs_Kinande_nouns-n-verbs}a). Nor do these H tones shift to the preceding morpheme, unlike other H tones. This state of affairs is expected under Emergence. An affix with only a H-toned morph is a logical possibility; given such a morph set, compilations are assessed normally. The apparent ``lack of tone shift'' follows under Emergence because the affix is not of class-α, so does not require a preceding H; selection of a preceding tone is made by *H.

Second, extending this line of thought further, the EG\is{Emergent Grammar} analysis predicts four types of morphs based on tone type and whether or not the morph is a member of the α-class: L, not α-class; H, not α-class; L, α-class; H, α-class. All four types are found in Kinande.

We have already seen examples of the L and H morphs that are not members of the α\ class, that is, that do not condition a H tone to their left. Both roots (\{\ipa{ɡ\`{ʊ}l\`{ʊ}}\}, \{\ipa{ɡ\'{ɔ}nɡ\`{ɔ}}\}, (\ref{URs_Kinande_nouns-n-verbs}b))\footnote{As observed above, only noun roots may belong to the ``H'' class, with a H on the initial vowel.} and affixes (\{\ipa{r\`{ɪ}, r\'{ɪ}}\}, \{\ipa{nà, ná}\}, (\ref{URs_Kinande_nouns-n-verbs}a, c)) with this behaviour can be observed. Moreover, as just noted, the prefix \{ká\}\down{\sc continuous} is invariably H and does not condition a preceding H (\ref{URs_Kinande_stable-H}).

As for morphs of the α-class, we have already seen roots in (\ref{URs_Kinande_nouns-n-verbs}c, d). Affixes too may be members of the α-class, as seen in (\ref{Kinande_prefix_alpha}).

%\newpage
\begin{example}\et{Kinande prefixes of the α-class} \label{Kinande_prefix_alpha}
\begin{tabular}{@{} llll @{}}
%&{L prefix, preceding H} \\
a. &\{tà\down{α}\} &\ipa{\`{ɛ}-r\'{ɪ}-tà-[h\`{ʊ}m]-à màɡ\'{ʊ}ːl\`{ʊ}} &M36/Mpc \\
   &               & `to merely hit Magulu' \\
   &\{tá\down{α}\} &\ipa{\`{ɛ}-r\'{ɪ}-tá-[t\`{ʊ}m]-à màɡ\'{ʊ}ːl\`{ʊ}} &Mpc\\
   &               & `to merely send Magulu' \\
b. &\{\ipa{m\`{ʊ}}\}         &\ipa{\`{ɛ}-r\`{ɪ}-m\`{ʊ}-[h\`{ʊ}m]-\`{ɪ}r-à màɡ\'{ʊ}ːl\`{ʊ}}     &AM336\\
   &                         &`to hit him for Magulu' \\
   &\{\ipa{m\`{ʊ}}\}         &\ipa{\`{ɛ}-r\`{ɪ}-nà-m\`{ʊ}-[h\`{ʊ}m]-\`{ɪ}r-à màɡ\'{ʊ}ːl\`{ʊ}} &AM336\\
   &                         &`to just hit him for Magulu'\\
   &\{\ipa{m\'{ʊ}}\down{α}\} &\ipa{\`{ɛ}-r\'{ɪ}-m\'{ʊ}-[t\`{ʊ}m]-\`{ɪ}r-à màɡ\'{ʊ}ːl\`{ʊ}}   &AM338\\
   &                         &`to send him for Magulu'\\
   &\{\ipa{m\'{ʊ}}\down{α}\} &\ipa{\`{ɛ}-r\`{ɪ}-ná-m\'{ʊ}-[t\`{ʊ}m]-\`{ɪ}r-à màɡ\'{ʊ}ːl\`{ʊ}} &AM338\\
   &                         &`to just send him for Magulu'
%a. &\{tà\down{α}\} &\ipa{\`{ɛ}-r\'{ɪ}-tà-[h\`{ʊ}m]-à màg\'{ʊ}:l\`{ʊ}} &`to merely hit M.' &M36/Mpc \\
%&\{tá\down{α}\} &\ipa{\`{ɛ}-r\'{ɪ}-tá-[t\`{ʊ}m]-à màg\'{ʊ}:l\`{ʊ}} &`to merely send M.' &Mpc \ee
%b. &\{\ipa{m\`{ʊ}}\} &\ipa{\`{ɛ}-r\`{ɪ}-m\`{ʊ}-[h\`{ʊ}m]-\`{ɪ}r-à màg\'{ʊ}:l\`{ʊ}} &`to hit him for M.' &AM336\\
%&\{\ipa{m\`{ʊ}}\} &\ipa{\`{ɛ}-r\`{ɪ}-nà-m\`{ʊ}-[h\`{ʊ}m]-\`{ɪ}r-à màg\'{ʊ}:l\`{ʊ}} &`to just hit him for M.' &AM336\\
%&\{\ipa{m\'{ʊ}}\down{α}\} &\ipa{\`{ɛ}-r\'{ɪ}-m\'{ʊ}-[t\`{ʊ}m]-\`{ɪ}r-à màg\'{ʊ}:l\`{ʊ}} &`to send him for M.'&AM338\\
%&\{\ipa{m\'{ʊ}}\down{α}\} &\ipa{\`{ɛ}-r\`{ɪ}-ná-m\'{ʊ}-[t\`{ʊ}m]-\`{ɪ}r-à màg\'{ʊ}:l\`{ʊ}} &`to just send him for M.' &AM338\\
\end{tabular} 
\end{example}

As seen in (\ref{Kinande_prefix_alpha}), some morphs must be assessed individually, not by morph set, for whether they are members of class α, requiring the presence of a preceding H. For `merely', both L and H morphs belong to class α\ (\ref{Kinande_prefix_alpha}a); for the object marker (\ref{Kinande_prefix_alpha}b), the L-toned morph does not condition a preceding H while the H-toned morph does.

These patterns are striking -- and they are problematic for a rule-based approach. `Merely' has two morphs: \{\ipa{tà\down{α}, tá\down{α}}\}\down{\sc merely}. Both morphs require a preceding H; the choice between the two morphs depends on the verb root that follows. Similarly, the object marker has two morphs: \{m\`{ʊ}, m\'{ʊ}\down{α}\}\down{\sc 3.sg.obj}. Again, the choice between the two morphs is determined by the verb root that follows. But there is a crucial difference: in \{m\`{ʊ}, m\'{ʊ}\down{α}\}\down{\sc 3.sg.obj}, only  the H morph  requires a preceding H tone. This double-H effect seen with the object marker required special treatment in a spreading account.  \citet{Mutaka:1994} proposed a special architecture for the object marker so that the object marker triggered a special cycle of rule application, one where spreading would apply twice (once to the object marker from a ``H-toned'' root and a second time from the object marker to the preceding prefix), but delinking would apply only once. In their optimality theoretic\is{Optimality Theory} account, \citet{Akinlabi+:2001} proposed an interaction between two tonal alignment constraints and the constraint mentioned above, {\sc AvoidSponsor}, which prevents a H from being realised on the vowel it is lexically a part of. Note the problem for such an analysis of the continuous forms where \{ká\}\down{\sc continuous} is consistently H, regardless of what follows. According to the analysis presented here, we propose that the H-toned morph for a third person object is a member of the α-class while the corresponding L-toned morph is not: \{m\`{ʊ}, m\'{ʊ}\down{α}\}\down{\sc 3.sg.obj}. As such the L-toned morph imposes no requirements on a preceding morph while the H-toned morph is subject to *[L]\{...\}\down{α}.

 
Finally, the Emergent analysis of Kinande tone is unlike other analyses of Kinande in that it makes no appeal to a tone spreading rule, \is{iterativity}iterative or noniterative. This is significant because the traditional Kinande spreading analysis crucially requires that tone not only spreads, but that it does so only to the next vowel over -- a rare instance of \is{noniterativity}noniterative spreading. Distinguishing between iterative and noniterative spread poses a variety of problems, problems that disappear under the EG\is{Emergent Grammar} analysis since it has no ``spread'', iterative or not. In general, we would expect a phonotactic\is{phonotactics} to constitute pressure for \is{iterativity}iterativity since *XY would be violated in domino fashion by changes in a sequence\is{sequence!Kinande} involving multiple potential targets: XXXXY $\rightarrow$ XXXYY $\rightarrow$ XXYYY $\rightarrow$ XYYYY $\rightarrow$ YYYYY. In contrast, a morpho-phonotactic\is{morpho-phonotactics} condition constitutes pressure only on the targeted element adjacent to the relevant morph: *X\{~\}\down{α} is satisfied by XXXY\{~\}\down{α} hence there is no pressure to become YYYY\{~\}\down{α}.\footnote{See \citet{Kaplan:2008} on issues involving noniterativity.\is{noniterativity} For additional discussion of Kinande tone in an Emergent framework, see \citet{Archangeli+:2015_K-tone, Archangeli+:2014abidjan}; for an alternative take on Kinande within a non-emergentist framework, see \citet{Jones:2014}.}




\subsection{Discussion: Diacritics vs.\ representations}

\is{diacritics|(}In Kinande, a morph's phonological properties cannot be predicted from the phonological make-up of the morph in question: the tone pattern is non-derivable.\is{derivability} The effect of a root  on the prefix tone cannot in any way be predicted from  phonological properties of the root: the pattern is not optimising.\is{optimisation} Only through observation do we know that a root must cooccur with a H-toned prefix, by observing the H tone on the prefix and tracking which root the H-toned prefix happens to have occurred with: the pattern is unproductive. Despite meeting none of the criteria for an \is{underlying representation!criteria}underlying representation, generativists\is{generative phonology!diacritics} have attempted a phonological analysis requiring abstract underlying representations\is{abstractness!underlying representation}\is{underlying representation!abstractness} and special non-iterative rules. Essentially, the traditional analysis  observes the H on the prefix and then ``explains'' its presence and its location by positing an otherwise unmotivated underlying H on a different \is{morpheme!Kinande}morpheme. This is circularity, not an explanation: the H is a diacritic masquerading as phonological motivation for the observed effect.\footnote{See \citet{Hyman:2018} for an analysis of tone ``displacement'' in Mijikenda,\il{Mijikenda} for another example of this type of analysis.}  

We propose instead a lexical class α\ and a morpho-phonotactic condition (i.e.\ a condition referring to both phonological and morphological content) referring to that class. The morpho-phonotactic\is{morpho-phonotactics} condition outranks\is{ranking!Kinande} the general *H type condition. This, along with representations involving H tones and L tones, accounts for the basic patterns of tone distribution in Kinande. There is of course much remaining to be examined for a full account of Kinande tone; see, for example, \citet{Hyman+:1985, Mutaka:1994, Mutaka:2001, Jones:2014, Archangeli+:2014abidjan}.\is{diacritics|)}\is{tone|)}\is{tone!Kinande|)}\il{Kinande|)}


\section{Abstract segments: Polish yers}\label{section_Polish}
\label{URs_Polish_section}\il{Polish|(}\is{yers|(}

The final issue we address in this chapter involves another three-way distinction:\is{abstractness} Slavic\il{Slavic} languages are known for having a three-way contrast\is{contrast!ternarity} among lexical items involving  \textit{yer} vowels. The  yer shows a V/$\emptyset$  alternation;\is{alternation!Polish yers} analyses typically posit a highly abstract \is{underlying representation!abstractness}underlying representation for the yer since the V/$\emptyset$ alternation contrasts with both non-alternating V and non-alternating CC. This leads to surface opacity\is{opacity} of two types, (i) alternations that occur despite there being no surface trigger for the alternation (\textit{overapplication}),\is{opacity!overapplication}\is{overapplication|see{opacity!overapplication}} and (ii) alternations that do not occur despite the presence of a trigger (\textit{underapplication}).\is{opacity!underapplication}\is{underapplication|see{opacity!underapplication}} As we show here, under Emergence, opacity is an expected consequence of the way that conditions assess compilations\is{compilation} of  morph sets with a three-way difference along a particular dimension. Opacity,\is{opacity} rather than motivating novel representations (\citealt{Gussmann:1980, Rubach:1984, Spencer:1986, Rubach:1986, Kenstowicz+:1987, Piotrowski+:1992, Szpyra:1992}),   unnatural rule-ordering\is{rule-ordering} (\citealt{Kiparsky:1971, Kiparsky:1973abstractness, Bakovic:2011-Wiley}) or challenging  the architecture of the theory itself (\citealt{Anttila:1997, Anttila:2002, Jarosz:2005_CLS, Gouskova:2012, Iwan:2015}), is simply business as usual in the Emergent framework.

We illustrate our analysis of opacity\is{opacity} with the distribution of yers  in Polish (Indo-European\il{Indo-European}, Slavic; glottocode poli1260). 

\subsection{The yer phenomenon}

A comparison of stem-final sequences in Polish contrasts alternating roots with roots that do not alternate along the same dimension. As seen in \tabref{Polish_yer_data}a, there can be an alternation between a root-final {\it CɛC} sequence and a root-final consonant cluster. Yet this alternation\is{alternation!Polish yers} is not necessary: we also find roots that invariably end in {\it CɛC}, \tabref{Polish_yer_data}b and roots that invariably end in a consonant cluster, \tabref{Polish_yer_data}c.  Stems are indicated with square brackets in \tabref{Polish_yer_data}.\footnote{As argued in \citet{Szpyra:1992}  the alternation\is{alternation!Polish yers} cannot be triggered by syllable\is{syllable} structure since all combinations of sonorants and obstruents are found with each of   the three types of roots (alternating, CVC\#, and CC\#).}


\begin{table}
\caption{Stem-final variation in Polish (\citealt[181]{Jarosz:2005_BLS})\label{Polish_yer_data}}
\begin{tabular}{l lll l@{~}l l@{~}l}
\lsptoprule
			  &a.	&\multicolumn{2}{l}{yer (alternating vowel)}	&b.	&final CVC	&c.	&final cluster\\\midrule
{\sc nom.sg}  &	    &[sfɛtɛr]	&*[sfɛtr]		&	&[rovɛr]				&	&[pʲotr]\\
{\sc gen.sg}  &	    &[sfɛtr]a	&*[sfɛtɛr]a		&	&[rovɛr]a				&	&[pʲotr]u\\
{\sc instr.pl}&	    &[sfɛtr]ami	&*[sfɛtɛr]ami		&	&[rovɛr]ami				&	&[pʲotr]ami\\
			  &	    &`sweater'	&		&	&`bicycle'				&	&`Peter'\\
\lspbottomrule
\end{tabular}
\end{table}


The alternating  yer vowel is illustrated in \tabref{Polish_yer_data}a, where stems in the {\sc nom.sg} form end with a CɛC sequence\is{sequence!Polish},  e.g. [sfɛt\uline{ɛ}r] (where the  yer is underlined), while in both the {\sc gen.sg} and the {\sc instr.pl}, the stem ends with a CC sequence, e.g.\ [sfɛ\uline{tr}]a (where the stem-final CC is underlined). Forms with yers  contrast with forms which invariably end with a CɛC sequence, as in [rovɛr], [rovɛr]a, *[rovr]a, \tabref{Polish_yer_data}b, as well as with those which invariably end with a consonant cluster, as in  [pʲotr], [pʲotr]a, *[pʲotɛr], \tabref{Polish_yer_data}c.

Analyses of this pattern typically invoke both a representational component and a relational component. For the former, the yer must have an abstract underlying representation\is{abstractness!underlying representation}\is{underlying representation!abstractness} since, as seen, the yer contrasts both with invariant\is{morph!invariant} {\it CɛC} and with invariant {\it CC}. Relationally, there must be a means of expressing when the yer surfaces and when it does not; the typical analysis involves some version of Havlík's Law,\is{Havlík's Law}  which states that  yers   surface when there is a following yer, but delete otherwise.  Havlík's Law\is{Havlík's Law} is shown in \tabref{Polish_schematic_rules}b; an obligatory \is{neutralisation}neutralisation rule, \tabref{Polish_schematic_rules}c, removes all yers  that have not been converted to [ɛ]. 


 
\begin{table} 
\caption{Schematic derivational analysis\label{Polish_schematic_rules}}
\begin{tabular}{l l cc}
\lsptoprule
	&		&i. {\sc sweater-nom.sg}&ii. {\sc sweater-gen.sg}\\\midrule
a.	&URs	&/sfɛt(V)r + (V)/ &/sfɛt(V)r + a/\\
b.	&(V) $\rightarrow$ [ɛ] / \uline{~~~}C\down{$\emptyset$}(V)	&sfɛtɛr + (V)&---\\
c.	&(V) $\rightarrow$ $\emptyset$	&sfɛtɛr&sfɛtr + a \\
d.	&SR				&[sfɛtɛr]	&[sfɛtra]\\
\lspbottomrule
\end{tabular}
\end{table}

In \tabref{Polish_schematic_rules}i, ii, the UR /sfɛt(V)r/  has a yer, depicted here by ``(V)''. When a yer-containing suffix such as /(V)/ `{\sc nom.sg}' is attached as in \tabref{Polish_schematic_rules}i, the first of the two  yers  surfaces according to Havlík's Law. The {\sc nom.sg}, being a final suffix, can never be followed by a yer; the consequence is that it never surfaces, since it is composed solely of a yer. Similarly, when /a/ `{\sc gen.sg}' is attached as in \tabref{Polish_schematic_rules}ii, there is only one yer vowel in the word, the yer in /sfɛt(V)r/, which cannot surface since there is no following yer.

\subsection{The Emergent analysis}

Under Emergence, the analysis is concrete: morphs are constructed from material observed phonetically; there can be no ``(V)'' representation. Rather, the three-way contrast\is{contrast!ternarity} of \tabref{Polish_yer_data} is represented directly, by three distinct types of morph sets, as in (\ref{Polish_3-morph-sets}): one has two morphs differing by the presence/absence of the vowel [ɛ] (\ref{Polish_3-morph-sets}a), while the other two have single morphs, showing no variation along this dimension (\ref{Polish_3-morph-sets}b, c).


\begin{example} 
\et{Examples of three types of morph sets in Polish}\label{Polish_3-morph-sets}
\ea \{sfɛtɛr, sfɛtr\}\down{\sc sweater}
\ex \{rovɛr\}\down{\sc bicycle}
\ex \{pʲotr\}\down{\sc Peter}
\z
\end{example}


These morphs sets are established based on observed forms. While there is a systematic relation between the morphs in a yer set, captured with a Morph Set Relation\is{Morph Set Relation!Polish} (\ref{Polish_yer_MSR}), this is not a productive\is{productivity} relation. Whether a morph set has two members can only be learned through observation.\is{acquisition!morph set}  (This step is true also for frameworks which propose an abstract representation for  yers, the typical solution when assuming \is{underlying representation!Polish yers}underlying representations. However, positing an abstract representation\is{abstractness!underlying representation}\is{underlying representation!abstractness} requires an additional stage of learning -- after identifying that there are two forms corresponding to the same meaning\is{sound-meaning correspondence} and the phonological relation between the two forms, the learner would then also devise a third, abstract, representation corresponding to the two surface variants.)

Under the Emergent framework, the learner posits morph sets with one or two morphs in them, depending on what has been observed.\is{acquisition!morph set} In the case of the polymorph sets, there is a consistent relationship between the morphs -- the presence or absence of [ɛ] near the end of the morph. Systematic relationships are expressed in MSRs.\is{acquisition!Morph Set Relation} MSR\down{\it yer}, (\ref{Polish_yer_MSR}), captures the two  consistent and generalisable phonological properties\is{generalisation!yers} of yers, namely the  quality and location of the vowel in the alternating class: the vowel [ɛ] is found between the last two stem consonants (\citealt[30]{Piotrowski+:1992}, \citealt[184--185]{Jarosz:2005_BLS}).\footnote{\citet[1140]{Rubach:2013} notes that there is a smattering of alternating [ɔ] words, again with the vowel between the final two consonants:  [kɔtɕɔł]\down{\sc nom.sg}, [kɔtła]\down{\sc gen-pl} `cauldron',  [ɔɕɔł]\down{\sc nom-sg},  [ɔsła]\down{\sc gen-pl} `donkey',  [kɔʑɔł]\down{\sc nom-sg},  [kɔzła]\down{\sc gen-pl} `goat'. This is not unexpected under Emergence; class membership is arbitrary\is{class!arbitrary} and must be learned.}


\begin{whiteshadowbox}
\begin{example} \et{Polish {yer} MSR}\is{Morph Set Relation!Polish} \label{Polish_yer_MSR} 

{In a minimal morph set, there is a systematic relation between morphs with a final CɛC sequence and morphs with a final CC sequence.}


\begin{tabular}{lp{6in}}
~\\
{\it example}		&\{sfɛtr, sfɛtɛr\}\down{\sc sweater} \\
				&\{marxɛv, marxv\}\down{\sc carrot}\footnotemark \\~\\
\end{tabular}

~\\

\begin{tabular}{lll}
MSR\down{\it yer}:&\{$\mathcal{M}$\down{\it i}, $\mathcal{M}$\down{\it j}\} &$\mathcal{M}$\down{\it i}: CC\#\\
&&$\mathcal{M}$\down{\it j}: CɛC\# \ee
\end{tabular}
\end{example}
 \end{whiteshadowbox}
\footnotetext{The example for {\sc carrot} is from \citet[185]{Jarosz:2005_BLS}.}


The pattern is not productive\is{non-productivity} because of the numerous invariant stems\is{morph!invariant} ending with either CC\# or with CɛC\#, (\ref{Polish_3-morph-sets}b, c) respectively. Consequently, there is no related MSC.\is{Morph Set Condition}

\largerpage
\begin{dadpbox}{What counts towards {\it identity}?}{box-identity-calculating}


\is{identity|(}In an MSR such as (\ref{Polish_yer_MSR}), we do not specify in the relation of ...CC\# with ...CɛC\# that the first C of ...CC\# must correspond with the first C of ...CɛC\#, and the same with the second C of both -- although we do assume that such correspondence holds. That is, we do not stipulate something like:\\

\begin{tabular}{lll}
MSR\down{\it yer}:&\{$\mathcal{M}$\down{\it i}, $\mathcal{M}$\down{\it j}\} &$\mathcal{M}$\down{\it i}: C\down{\it p}C\down{\it q}\#\\
&&$\mathcal{M}$\down{\it j}: C\down{\it p}ɛC\down{\it q}\# \\~\\
\end{tabular}

Such specification is unnecessary, though assumed, due to a combination of {\it The Identity Principle}\is{Identity Principle} (\ref{identity-assumption}) (Chapter \ref{ch3}) with our assumptions about the substance of phonological representations. {\it Identity} holds both of the featural make-up of segments and of properties of precedence\is{precedence} and adjacency. A pair of morphs such as \{sfɛtr, sfɛtɛr\}\down{\sc sweater} satisfies featural identity (except for the presence/absence of [ɛ]) and satisfies precedence since [t] precedes [r] in both morphs. Adjacency\is{adjacency} is not identical in both due to the presence/absence of [ɛ] and is stipulated in the MSR.\is{Morph Set Relation!Polish}\is{identity|)}
\end{dadpbox}


The above provides the Emergent analysis of the representations necessary for the yers  in Polish. We turn now to determining when to use which morph: Havlík's Law\is{Havlík's Law} is not an option because there is no abstract yer\is{abstractness} in any representation. Three contexts must be considered:  the word-final context seen with the {\sc nom.sg}, where the yer is preferred, and two types of pre-suffix contexts, one which prefers the yer and the other which does not, illustrated by representative forms in \tabref{Polish_V-no-V}.  Again, the stem is bracketed for clarity.


\begin{table} 
\caption{Vowel \& no-vowel contexts in Polish\label{Polish_V-no-V}}
\begin{tabular}{llllll}
\lsptoprule
&   & word-final & suffix 1 & suffix 2\\\midrule
	&				&{\sc nom.sg}	&{\sc nom.sg-dim} &{\sc gen.sg}\\
a.	&`staple'		&[skɔbɛl]		&[skɔbɛl]ɛk	&[skɔbl]a&\citet[1141]{Rubach:2013}\\
b.	&`sweater'	&[sfɛtɛr]			&\ipa{[sfɛtɛr]ɛk}	&[sfɛtr]a	&\citet[186]{Jarosz:2005_BLS}\\
\lspbottomrule
\end{tabular}
\end{table}


Inspection of the forms in \tabref{Polish_V-no-V} reveals that the [ɛ]-form surfaces when the morph is word-final ({\it word-final} column). Before suffixes, the choice of root forms is phonologically arbitrary, simply depending on the suffix which is added. \tabref{Polish_V-no-V} illustrates this with two vowel-initial suffixes, where type 1 requires the {\it ...CɛC} form of the root ({\it suffix 1} column) and type 2 requires the {\it ...CC} form ({\it suffix 2} column). Were the phonological shape of the suffixes to drive the selection between morphs, we would expect the same result with both types of suffixes, yet they differ.   We take the general pattern to be the {\it ...CɛC} pattern, derived by the condition *CC, a prohibition on consonant clusters.\footnote{Whether the {\it ...CC} pattern or the {\it ...CɛC} pattern is generally preferred shapes the nature of the analysis. Taking the {\it ...CC} pattern as the more general pattern would require a prohibition on open syllables\is{syllable} to penalise, e.g., *[skɔbɛla] and *[sfɛtɛra]. We present only the *CC analysis here.} Since CC clusters abound within morphs, this can only be a prohibition at the word domain.\is{well-formedness condition!Polish}


\begin{example}\et{Polish CC condition} \label{Polish_*CC}\\
*CC, \tier: segments, \dom: word\\\is{word!domain}
With a focus on segments, assign a violation to a word for each sequence of two consonants.
\end{example}

Despite the preference to avoid CC clusters, certain suffixes like \{a\}\down{\sc gen.sg} or \{ami\}\down{\sc Instr.pl} require the CC-final form -- [sfɛtra]\down{\sc sweater-gen.sg}, *[sfɛtɛra]. We propose that this class of suffixes avoids a particular phonological shape in the sequence immediately preceding the suffix, namely a preceding VC sequence, with the result that the otherwise nonoptimal CC sequence results. There is no way to predict whether or not a suffix imposes the \PolVC\  condition: this condition is imposed by a lexically \is{class!arbitrary}arbitrary class, hence reference to this arbitrary ``α-class'' must be built into the condition.\is{morpho-phonotactic} As with the CC condition, it holds at the word domain since it requires a combination of morphs to meet its conditions.\is{well-formedness condition!Polish}

\begin{example} \et{Polish VC condition} \label{Polish_*VC}\\
\PolVC, \tier: segments, \dom: word\\
With a focus on segments, assign a violation to a word for each VC sequence preceding a member of the α\ class.
\end{example}

Selection of relevant forms is illustrated by  assessments. In (\ref{sfeter}), the {\sc nom.sg} is a null affix; there are only two compilations and *CC is the deciding factor. For comparison, (\ref{assess_pjotr}) shows the assessment for \{pʲotr\}\down{\sc peter.nom.sg}: since this is a root with a single form, there is only one compilation to consider hence no competition for selection.


\begin{example} \et{Assessment for [{sfɛtɛr}]\down{\sc sweater-nom.sg}}\\ \label{sfeter}
{\it morph sets}: \{sfɛtɛr, sfɛtr\}\down{\sc sweater}; \{$\emptyset$\}\down{\sc nom.sg}\\
\begin{center}
\begin{tabular}{lll | c   | c }
\hline\hline
\multicolumn{3}{c|}{{\sc sweater-nom.sg}}
					&\PolVC		&*CC \\
\hline
\rightthumbsup
&a. &\ipa{sfɛtɛr}	&	 	 		&* 	\\
\hline

&b. &\ipa{sfɛtr}	&	 		 	&**!  	\\
\hline
\hline 
\end{tabular}
\end{center}
\end{example}



\begin{example} \et{Assessment for [pʲotr]\down{\sc peter-nom.sg}}\label{assess_pjotr}

{\it morph sets}: \{pʲotr\}\down{\sc peter}; \{$\emptyset$\}\down{\sc nom.sg}\\

\begin{center}
\begin{tabular}{lll | c   | c }
\hline\hline
\multicolumn{3}{c|}{{\sc Peter-nom.sg}}
						&\PolVC			&*CC \\
\hline
\rightthumbsup
&a. &\ipa{pʲotr}	&	 	 		&* 	\\
\hline\hline
\end{tabular}
\end{center}

\end{example}


The assessments in (\ref{sfetra}) and (\ref{rovera}) illustrate the effect of adding an  α-class suffix. In the case of \{{sfɛtɛr}, sfɛtr\}\down{\sc sweater}, there is a CC-final option (\{sfɛtr\}); [\ipa{sfɛtr-a}] is selected over [\ipa{sfɛtɛr-a}].


\begin{example} \et{Assessment for [sfɛtɛra]\down{\sc sweater-gen.sg}} \label{sfetra}

{\it morph sets}: \{sfɛtɛr, sfɛtr\}\down{\sc sweater}; \{a\}\down{{\sc gen.sg}, α}

\begin{center}
\begin{tabular}{lll | c   | c }
\hline\hline
\multicolumn{3}{c|}{{\sc sweater-gen.sg}\down{$\upalpha$}}
					&\PolVC		&*CC \\
\hline
&a. &\ipa{sfɛtɛr-a\down{$\upalpha$}}	&*!	 	 		&{*} 	\\
\hline
\rightthumbsup
&b. &\ipa{sfɛtr-a\down{$\upalpha$}}	&	 		 	&** 	\\
\hline
\hline 
\end{tabular}
\end{center}
\end{example}


With a single-morph set, like \{rovɛr\}\down{\sc bicycle} in (\ref{rovera}), there is only one compilation so there is no competition;  [rovɛra] is selected despite the \PolVC\ violation.


\begin{example} \et{Assessment for [rovɛra]\down{\sc bicycle-gen.sg}} \label{rovera}

{\it morph sets}: \{rovɛr\}\down{\sc bicycle}; \{a\}\down{{\sc gen.sg}, α}
\begin{center}

\begin{tabular}{lll | c   | c }
\hline\hline
\multicolumn{3}{c|}{{\sc bicycle-gen.sg}\down{$\upalpha$}}
					&\PolVC		&*CC \\
\hline
\rightthumbsup
&a. &\ipa{rovɛr-a\down{$\upalpha$}}	&*	 	 		& 	\\
\hline
\hline
\end{tabular}
\end{center}
\end{example}


\subsection{Stacking up alternating suffixes}

A striking property of the Polish pattern is that it is possible to concatenate sequences of morph sets containing multiple alternating vowels, illustrated in (\ref{rubach}). As established in \tabref{Polish_V-no-V} and repeated in (\ref{rubach}a),   [\ipa{skɔbɛl}]\down{\sc staple} patterns like [\ipa{sfɛtɛr}]\down{\sc sweater}, having a final CɛC with {\sc nom.sg} but a final CC-cluster with {\sc gen.sg}. When the diminutive\is{diminutive!Polish|(} suffix is added in (\ref{rubach}b), we see a V/$\emptyset$ alternation\is{alternation!Polish yers} in the diminutive suffix itself, \{ɛk,~k,~ɛ\tS,~\tS\}\down{\sc dim}, depending on the following suffix, as well as in the noun stem. Polish also has a  double diminutive construction, shown in (\ref{rubach}c).


\begin{example} \et{Multiple alternating vowels (data from \citealt[1141]{Rubach:2013})} \label{rubach}

\begin{tabular}{llllll}
	&						&{\sc masc.nom.sg}	&{\sc gen.sg}\\
a.	&`staple'				&skɔbɛl			&skɔbla\\
b.	&`staple-{\sc dim}'	&skɔbɛlɛk		&skɔbɛlka\\
c.	&double {\sc dim}				&skɔbɛlɛ{\tS}ɛk&skɔbɛlɛ\tS ka\\
\end{tabular}
\end{example}


Since the diminutive attaches to [skɔbɛl] and not [skɔbl], we conclude that it does not belong to class α. Adding the  morph set \{ɛk, k, ɛ\tS, \tS\}\down{\sc dim} is all that is needed to account for the patterns seen in (\ref{rubach}), shown by the assessments in (\ref{assess-skobelek}) and (\ref{assess-skobelka}). %In (\ref{assess-skobelek}),  the *CC phonotactic\is{phonotactics!Polish} makes the crucial determination, while (\ref{assess-skobelka}) relies on \PolVC\ to eliminate combinations where the suffix \{a\}\down{{\sc gen.sg}, α} follows a VC sequence\is{sequence!Polish}.\footnote{The distribution of palatal consonants in Polish reveals intriguing alternations,\is{alternation!Polish palatals} including those that give rise to the four morphs in the {\sc dim} morph set. We do not explore the puzzle of Polish palatalisation\is{palatalisation} here; those facts suggest, among other things, that there may be two distinct diminutive suffixes, differing by their interaction with palatalisation.  See \citet{Lubowicz:2016} and \citet{Czaplicki:2014} for discussion of palatalisation and Polish diminutives; see \citet{Manova+:2011} for discussion of multiple diminutives. See also \citet{Mihajlovic:2020} for an Emergent analysis of palatalisation in Bosnian/Croatian/Serbian.\il{Bosnian}\il{Croatian}\il{Serbian}\label{Polish-palatal-footnote} In assessments, we simplify the surface possibilities by including only those forms that have the appropriate morphs in terms of palatalisation, leaving for future work the conditions which select among the different consonantal possibilities.}


\begin{example} \et{Assessment for [{\ipa{skɔbɛlɛk}]\down{\sc staple-dim-nom.sg}}}
\label{assess-skobelek}

{\it morph sets}: \{\ipa{skɔbɛl, skɔbl}\}\down{\sc staple}; \{ɛk, k,~ɛ\tS,~\tS\}\down{\sc dim}; \{$\emptyset$\}\down{\sc nom.sg}

\begin{center}
\begin{tabular}{lll | c   | c }
\hline\hline
\multicolumn{3}{c|}{{\sc staple-dim-nom.sg}}
					&\PolVC		&*CC \\
\hline
\rightthumbsup
&a. &\ipa{skɔbɛl-ɛk}	&	 	 		&* 	\\
\hline
&b. &\ipa{skɔbɛl-k}	&	 		 	&**!  	\\
\hline
&c. &\ipa{skɔbl-ɛk}	&	 		 	&**! 	\\
\hline
&d. &\ipa{skɔbl-k}	&	 		 	&**!*  	\\
\hline
\hline 
\end{tabular}
\end{center}
\end{example}

In (\ref{assess-skobelek}),  the *CC phonotactic\is{phonotactics!Polish} makes the crucial determination, while (\ref{assess-skobelka}) relies on \PolVC\ to eliminate combinations where the suffix \{a\}\down{{\sc gen.sg}, α} follows a VC sequence\is{sequence!Polish}.\footnote{The distribution of palatal consonants in Polish reveals intriguing alternations,\is{alternation!Polish palatals} including those that give rise to the four morphs in the {\sc dim} morph set. We do not explore the puzzle of Polish palatalisation\is{palatalisation} here; those facts suggest, among other things, that there may be two distinct diminutive suffixes, differing by their interaction with palatalisation.  See \citet{Lubowicz:2016} and \citet{Czaplicki:2014} for discussion of palatalisation and Polish diminutives; see \citet{Manova+:2011} for discussion of multiple diminutives. See also \citet{Mihajlovic:2020} for an Emergent analysis of palatalisation in Bosnian/Croatian/Serbian.\il{Bosnian}\il{Croatian}\il{Serbian}\label{Polish-palatal-footnote} In assessments, we simplify the surface possibilities by including only those forms that have the appropriate morphs in terms of palatalisation, leaving for future work the conditions which select among the different consonantal possibilities.}


\begin{example} \et{Assessment for [{\ipa{skɔbɛlka}]\down{\sc staple-dim-gen.sg}}}
\label{assess-skobelka}

{\it morph sets}: \{\ipa{skɔbɛl, skɔbl}\}\down{\sc staple}; \{ɛk, k,~ɛ\tS,~\tS\}\down{\sc dim}; \{a\}\down{{\sc gen.sg}, α}

\begin{center}
\begin{tabular}{lll | c   | c }
\hline\hline
\multicolumn{3}{c|}{{\sc staple-dim-gen.sg}\down{α}}
						&\PolVC		&*CC \\
\hline
&a. &\ipa{skɔbɛl-ɛk-a\down{$\upalpha$}}	&*!	 	 		&{*} 	\\
\hline
\rightthumbsup
&b. &\ipa{skɔbɛl-k-a\down{$\upalpha$}}	&	 		 	&**  	\\
\hline
&c. &\ipa{skɔbl-ɛk-a\down{$\upalpha$}}	&*!	 		 	&{**} 	\\
\hline
&d. &\ipa{skɔbl-k-a\down{$\upalpha$}}	&	 		 	&***!  	\\
\hline
\hline 
\end{tabular}
\end{center}
\end{example}



When two diminutive suffixes are added, again the phonotactic\is{phonotactics!Polish} *CC is the deciding factor when there are no α-class suffixes  to make the selection, shown in (\ref{assess-skobelekek}). (See footnote \ref{Polish-palatal-footnote} on the simplified morph compilations in (\ref{assess-skobelekek}), (\ref{assess-skobelekka}).)

\begin{example} \et{Assessment for [\ipa{skɔbɛlɛ\tS ɛk}]\down{\sc staple-dim-dim-nom.sg}} \label{assess-skobelekek}

{\it morph sets}: \{\ipa{skɔbɛl, skɔbl}\}\down{\sc staple}; \{ɛk, k,~ɛ\tS,~\tS\}\down{\sc dim}; \{$\emptyset$\}\down{{\sc nom.sg}}

\begin{center}

\begin{tabular}{lll | c   | c }
\hline\hline
\multicolumn{3}{c|}{{\sc staple-dim-dim-nom.sg}}
					&\PolVC		&*CC \\
\hline
\rightthumbsup
&a. &\ipa{skɔbɛl-ɛ\tS-ɛk}	&	 	 		&* 	\\
\hline
&b.	&\ipa{skɔbɛl-ɛ\tS-k}	&	 	 		&**! 	\\
\hline
&c. &\ipa{skɔbɛl-\tS-ɛk}	&	 		 	&**!  	\\
\hline
&d. &\ipa{skɔbɛl-\tS-k}	&	 		 	&**!*  	\\
\hline
&e. &\ipa{skɔbl-ɛ\tS-ɛk}	&	 		 	&**! 	\\
\hline
&f. &\ipa{skɔbl-ɛ\tS-k}	&	 		 	&**!*  	\\
\hline
&g. &\ipa{skɔbl-\tS-ɛk}	&	 		 	&**!*	\\
\hline
&h. &\ipa{skɔbl-\tS-k}	&	 		 	&**!**  	\\
\hline
\hline 
\end{tabular}
\end{center}
\end{example}


With a morph compilation involving a class α\ suffix, such as \{a\}\down{{\sc gen.sg}, α}, the condition \PolVC\ prevents the immediately preceding suffix from surfacing with a vowel, here the rightmost of the two diminutive suffixes. That is, ...CC-\{a\}\down{α} is preferred to ...VC-\{a\}\down{α}. The form in (\ref{assess-skobelekka}b) is therefore selected, with a CC cluster only before the suffix \{a\}\down{{\sc gen.sg}, α}.
 

\begin{example} \et{Assessment for [\ipa{skɔbɛlɛ\tS ka}]\down{\sc staple-dim-dim-gen.sg}} \label{assess-skobelekka}

{\it morph sets}: \{\ipa{skɔbɛl, skɔbl}\}\down{\sc staple}; \{ɛk, k, ~ɛ\tS,~\tS\}\down{\sc dim}; \{a\}\down{{\sc gen.sg}, α}

\begin{center}

\begin{tabular}{lll | c   | c }
\hline\hline
\multicolumn{3}{c|}{{\sc staple-dim-dim-gen.sg}\down{α}}
											&\PolVC	&*CC \\
\hline
&a. &\ipa{skɔbɛl-ɛ\tS-ɛk-a}\down{α}	&*!	 	 		& * 	\\
\hline
\rightthumbsup
&b.	&\ipa{skɔbɛl-ɛ\tS-k-a}\down{α}	&	 	 		&** 	\\
\hline
&c. &\ipa{skɔbɛl-\tS-ɛk-a}\down{α}		&*!	 		 	& **  	\\
\hline
&d. &\ipa{skɔbɛl-\tS-k-a}\down{α}		&	 		 	&***!  	\\
\hline
&e. &\ipa{skɔbl-ɛ\tS-ɛk-a}\down{α}		&*!	 		 	& ** 	\\
\hline
&f. &\ipa{skɔbl-ɛ\tS-k-a}\down{α}		&	 		 	&***!  	\\
\hline
&g. &\ipa{skɔbl-\tS-ɛk-a}\down{α}		&*!	 		 	& ***	\\
\hline
&h. &\ipa{skɔbl-\tS-k-a}\down{α}		&	 		 	&***!*  	\\
\hline
\hline 
\end{tabular}
\end{center}
\end{example}
\is{diminutive!Polish|)}

\subsection{The position and quality of  yers }

Under the Emergent account, the  morph set representations directly correspond to the surface forms:  the alternating\is{alternation!Polish yers}\is{alternation!morph set} effect is represented by morph sets with multiple members, e.g.\ \{{sfɛtɛr}, sfɛtr\}\down{\sc sweater} while the lack of alternation is represented by morph sets with single members, \{rovɛr\}\down{\sc bicycle}, \{pʲotr\}\down{\sc peter}. There are no abstract representations\is{abstractness} of the Slavic\il{Slavic} yer.  Selection of the appropriate morph is achieved by a general prohibition on CC sequences\is{sequence!Polish} mediated by a prohibition on a preceding VC sequence before certain suffixes.

One consequence of this analysis is that it explains both the location and the quality of the yer vowels, issues raised in \citet{Jarosz:2005_BLS}:

\begin{quote}``All word-final yer vocalization follows the template XCC-Y $\sim$ XCɛC-$\emptyset$, where Y is any overt inflection, and X, any vowel or consonant....In contrast, alternations of the type CCC-Y $\sim$ CɛCC-$\emptyset$, where a yer vocalizes between the first and second consonant of the triple, are conspicuously absent.'' \citet[184--5]{Jarosz:2005_BLS} \end{quote}



Given the Emergent analysis,  these patterns are explained. Formally, the vowel quality and the location are encoded in the  Morph Set Relation\is{Morph Set Relation!Polish} (\ref{Polish_yer_MSR}). Analytically, if a morph set were posited with the alternating vowel in a different location, e.g.\ \{...CɛCC, ...CCC\}, the final CCC would never surface: in the special environment where class-α\ morphs avoid a preceding VC, both morphs would be equivalent with respect to the \PolVC\ well-formedness condition\is{well-formedness condition!Polish} since neither ends in a VC; the choice between the two would default to *CC which would select ...CɛCC (one violation) over ...CCC (two violations). {In contexts where the class-α\ condition is not relevant, the ...CɛCC morph would continue to incur fewer violations than the ...CCC morph. In other words, the ...CCC morph would never be chosen. The result is that only morph pairs differing by  ...CC  vs.\ ...CɛC  at the right edge could result in different surface forms.} Functionally, this analysis ensures that the relevant suffixes avoid sequences\is{sequence!Polish} of light open syllables, satisfying \PolVC.

\begin{dadpbox}{Syllables}{box-syllables}
\is{syllable|(}\is{prosody}

Throughout this work, we have been deliberately agnostic as to the necessity for syllables as a \is{constituency!prosodic|(}constituent, how they might be structured and how they might be acquired. We note in this regard though that both phonetic\is{phonetics} and distributional cues may lead to such a constituent (\citealt{Maddieson:1985, Turk:1994}); the acquisition\is{acquisition!syllable} literature shows that  patterns of learning support acquisition of such a constituent (\citealt{Carter:1999, Carter+:2004}); further, as pointed out in \citet{vandeWeijer:2012}, the tip-of-the-tongue phenomenon suggests some concept of ``syllable'' in mental representations. \\

If syllables are indeed motivated, it remains to be seen whether the \textit{syllable} is best characterised in terms of the distribution of phonetic\is{phonetics} cues or by a more abstract construct; we are noncommittal at this point in our research.  As a simplifying move, we have therefore omitted morph set members that might be distinguished by syllabically correlated phonetic cues. For example, we use \{CVC\}, not making a distinction between \{CVC, CV.C\}, even when there is evidence that the final consonant is syllabified as a coda in some words (CVC) and as an onset in others (CV.C).\is{constituency!prosodic|)}\is{syllable|)}

\end{dadpbox}

\subsection{Discussion: Opacity}\label{section_opacity-in-Polish}
\is{opacity|(}

\largerpage[-1]
The crux of the Polish case is that while there are two phonological shapes in play (CɛC\#, CC\#), there are three lexical classes (alternating final CɛC/CC, non-alternating final CɛC, non-alternating final CC). This three-way contrast\is{ternarity!opacity} leads  to the issue of opacity,  a phenomenon that raises analytic challenges unique to the \is{underlying representation!Polish yers}underlying representation hypothesis.  Two types of opacity, overapplication opacity\is{opacity!overapplication} and underapplication opacity,\is{opacity!underapplication} can be identified (\citealt{Kager:1999,McCarthy:1999, Idsardi:2000}).\footnote{For more on opacity, see \citet{Kiparsky:1971, Kiparsky:1973abstractness, Kiparsky:1979, Kiparsky:2000, McCarthy:1996, Bakovic:2011-Wiley}, among others.}


\begin{example} \et{Opacity}\ee
\label{opacity}

``An opaque generalization is a generalization that does crucial work in the analysis, but which does not hold of the output form.''  (\citealt[338]{Idsardi:2000})\is{opacity!definition}

\begin{tabular}{llll}
	&{\it type}		&{\it environment}	&{\it generalisation}\\
a.	&overapplication&not surface-apparent	&holds anyway\\
b.	&underapplication&surface-apparent	&does not always hold\\ 
\end{tabular}
\end{example}


Overapplication opacity\is{opacity!overapplication} is relevant for Polish because there is no surface-ap\-par\-ent context triggering the appearance of the  yers, yet in some contexts the yer surfaces.

We also see cases of underapplication opacity:\is{opacity!underapplication} even if a context were clear for the appearance of the yer, there are both vowels that always surface, even where the yer does not (compare  [rovɛra] `bicycle-{\sc gen.sg}' and [sfɛtra] `sweater-{\sc gen.sg}'), and there are final CC sequences\is{sequence!Polish} that are never interrupted by a vowel, in contrast to where the yer surfaces (compare [pʲotr] `Peter-{\sc nom.sg}' and  [\ipa{sfɛtɛr}] `sweater-{\sc nom.sg}').

Opacity\is{opacity} in general presents a challenge to both rule-based and constraint-based frameworks: rules that insert vowels should apply in forms like /pʲotr/, yet do not; at the same time, rules that delete vowels should apply in /rovɛr-a/, yet do not. Optimality Theory\is{Optimality Theory} is challenged by opacity as well due to the role of Gen:\is{Gen} constraints that eliminate *[sfɛtɛra] in favour of [sfɛtra] would erroneously eliminate [rovɛra] in favour of  *[rovra]; constraints that eliminate *[sfɛtr] in favour of [sfɛtɛr] would erroneously eliminate [pʲotr] in favour of *[pʲotɛr].

The concrete Emergent analysis contrasts sharply with the abstraction forced by unique \is{underlying representation!unique}underlying representations: under the assumption of unique underlying representations, Polish  yers   rely on a vowel with an abstract\is{abstractness} representation  (to distinguish it from other vowels, including /ɛ/, which do not alternate with $\emptyset$). Due to the reliance on Havlík's Law,\is{Havlík's Law}  the {\sc nom.sg} is represented \is{underlying representation!Polish yers}underlyingly as a yer despite the fact that there is no surface reflex of this suffix. Exactly how a specific analysis plays out varies. Derivational proposals include analyses relying on abstract vowels with \is{absolute neutralisation}absolute neutralisation  (\citealt{Gussmann:1980, Rubach:1984}), featureless skeletal slots (\citealt{Spencer:1986}), floating features (\citealt{Rubach:1986, Kenstowicz+:1987}),  epenthesis (\citealt{Czaykowska-Higgins:1988}, after unpublished work by Alicja Gorecka),\footnote{An advantage of epenthesis would be that there is no need for the abstract vowels\is{abstractness}, and syllabification\is{syllable!Polish} places the epenthetic vowel in exactly the right position; \citet{Szpyra:1992} argues strongly against epenthesis.} lexical syllabic consonants (\citealt{Piotrowski+:1992}), and  underspecification\is{underspecification!Polish yers} (\citealt{Szpyra:1992}). Optimality theoretic\is{Optimality Theory} analyses also propose abstract vowels (\citealt{Yearley:1995}); additionally, such analyses involve partial ordering (\citealt{Jarosz:2005_CLS}; see also \citealt{Anttila:1997, Anttila:2002} on partial ordering), \is{morpheme!exceptionality}morpheme exceptionality (\citealt{Gouskova:2012}), and intermediate levels of representation (\citealt{Iwan:2015}). These accounts involve segments that do not surface in the language and so require \is{absolute neutralisation}absolute neutralisation; such representational approaches are not available -- or necessary -- under EG.\is{Emergent Grammar}\footnote{There are a small number of alternative assumptions about  yers, for example, \citet{Jarosz:2005_CLS, Gouskova:2012}. There are also prefixes/proclitics which have been argued to contain  yers;  \citet{Pajak+:2010} argue that there is no yer in these forms. Our analysis, as shown in this section, is that some morph sets have one member; others have two members, and well-formedness conditions assess morph compilations appropriately.}


To conclude, at least in Polish, opacity\is{opacity} -- whether of underapplication or of overapplication -- is a non-issue under Emergence. There is no call for abstract segments, special orderings, more complex formal systems, etc. The  apparent opacity effect arises due to the presence of MSR\down{\it yer} and the absence of a corresponding MSC:\is{Morph Set Condition} this results in a three-way distinction in Polish morph sets, (i) those which always end with a CC, (ii) those which always end with a CɛC, and (iii) those with two morphs, one ending with CC and the other ending with CɛC; choices among morph compilation members are adjudicated by two conditions, \PolVC\ and *CC.\is{opacity|)}\is{yers|)}\il{Polish|)} 


\section{Conclusion: Dispensing with abstract representations}

The examples in this chapter all share the property of seemingly ``requiring'' abstract\is{abstractness|(} underlying representations; we have provided analyses using concrete representations in accordance with Emergence. Derivable,\is{derivability} productive,\is{productivity} and optimisable,\is{optimisation} the three criteria (defined in Chapter \ref{ch4}, (\ref{UR-criteria})) that are often taken as indicative of \is{underlying representation!criteria}underlying representations, are expressed under Emergence without appeal to that mechanism. Additionally, in our examples in this chapter, we see cases where a phonological pattern is not derivable, not productive,\is{non-productivity} and/or not optimisable, summarised in \tabref{chapter_summary}.\is{assimilation!nasal place}

\begin{table} 
\caption{Summary of  examples\label{chapter_summary}} \il{Warembori}\il{English}\il{Mayak}\il{Kinande}\il{Polish}
\fittable{\begin{tabular}{l@{ }llccc}
\lsptoprule
	&Language	&Phenomenon&Derivable	&Productive	&Optimisable\\ \midrule
a.	&Warembori	&stop/fricative	&yes	&yes	&yes\\ \midrule
b.	&English		&nasal place assimilation&possible	&no	&yes\\
c.	&Mayak		&3 low V suffix types&possible	&no	&yes\\	\midrule
d.	&Kinande		&``tone shift''	&possible	&no	&no	\\
e.	&Polish		&yers 	&yes	&no	&no\\
\lspbottomrule
\end{tabular}}
\end{table}


Whether a relation is derivable is expressed in Emergence by whether or not there is evidence for a Morph Set Relation.\is{Morph Set Relation} In Warembori, the pattern is pervasive, and in Polish, the alternation is seen in about a third of the possible cases; there is sufficient reason for a learner to posit MSRs in both these cases.\is{acquisition!Morph Set Relation} For the other three languages, the alternations are observed in a limited number of affixes; once the alternants have been observed, there may be no further generalisation so there is no need for an MSR. However, a learner might posit a relation with very limited use, hence ``possible'' in the \textit{derivable} column in \tabref{chapter_summary}; no significant generalisation\is{generalisation} is missed if there is no relevant MSR in these cases nor is communication impacted. 

Productivity,\is{productivity} characterised by an MSC,\is{Morph Set Condition} is found only in Warembori;\il{Warembori} in the other four languages, whether or not a morph set exhibits an alternation is learned on a case-by-case basis.\is{non-productivity} Finally, in three of the languages, the distribution of morphs is optimisable, characterised by phonological conditions.\is{well-formedness condition!phonological} In the other two languages, \il{Kinande}Kinande and \il{Polish}Polish, the distribution of morphs is not optimisable. In Kinande, the appearance of a H tone on a prefix is made based on conditions\is{well-formedness condition!lexical} that refer to an \is{class!arbitrary}arbitrary lexical class rather than to phonological properties. In Polish, while a C\ipa{ɛ}C sequence is preferred phonologically to a CC sequence\is{sequence!Polish}, the occurrence of a CC sequence\is{sequence!Polish} in cases of \is{alternation!non-productive}alternation occurs only when required by a class that is lexically arbitrary.\footnote{A very interesting analysis of Irish\il{Irish} initial consonant \is{consonant mutation}mutation, \citet{McCullough:2020, McCullough:2021}, provides an example of a derivable\is{derivability} and productive\is{productivity} pattern that is not optimisable.\is{optimisation}}

To analyse each of these cases in terms of single \is{underlying representation!unique}underlying representations requires the introduction of otherwise-unmotivated representations, changes to the theory of relations among representations, or both. In contrast, an analysis in Emergent theoretic terms requires three independently motivated mechanisms, the Morph Set Relation (MSR), the Morph Set Condition (MSC), and word-domain\is{domain \dom} well-formedness conditions.\is{well-formedness condition!domain}  Each of these devices serves to characterise observations about surface forms in the language; together, they represent the alternations found under \is{morphology!concatenation}morphological concatenation.\is{word!domain}

These cases collectively demonstrate that \is{underlying representation!criteria}underlying representations have been posited in cases where the three criteria are not simultaneously satisfied, despite these being the criteria used to motivate underlying representations. But the examples go further: in each case, we provide explanatory analyses under Emergence, with \is{surface-to-surface}surface-based representations, not unique \is{underlying representation!unique}underlying representations, solving long-standing challenges like ternarity, abstractness,\is{abstractness|)} and opacity\is{opacity} without additional mechanisms.\is{ternarity} Representations are concrete,  phonotactics\is{phonotactics} express general properties, and morpho-phonotactics\is{morpho-phonotactics} pinpoint exactly where sub-patterns occur. The formal mechanisms used are Morph Set Relations\is{Morph Set Relation}, to codify differences among members of a morph set, Morph Set Conditions\is{Morph Set Condition}, to codify the productivity of an MSR, and word-domain well-formedness conditions,\is{domain \dom} to select among competing compilations.\is{compilation}  
