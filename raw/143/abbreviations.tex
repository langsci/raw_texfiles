% -*- coding: utf-8 -*-
\addchap{Symboles et abréviations}
% \addchap{Abbreviations and symbols}


\begin{multicols}{2} 

\noindent
\begin{tabular}{lp{4.5cm}} 
\multicolumn{2}{l}{\bfseries Théorie des ensembles :}\\
$\set{\,}$        &    Contenu d'un \kwab{ensemble} ; \set{a;b;c} est l'ensemble composé des éléments $a$, $b$ et $c$. \\
$\tq$                & \emph{Tel que} dans la notation \set{x\tq p}, l'ensemble de tous les $x$ tels que la condition $p$ est vérifiée.\\
$\in$        &    \kwab{Appartenance} ; $a\in A$ : $a$ est un élément de l'ensemble $A$ ou $a$ appartient à $A$. \\
$\inclus$        &    \kwab{Inclusion} ; $A\inclus B$ : $A$ est un sous-ensemble de $B$ ou $A$ est inclus dans $B$, \ie\ tous les éléments de $A$ sont aussi des éléments de $B$. \\
$\eVide$                & \kwab{Ensemble vide} : l'ensemble qui ne possède aucun élément ; $\eVide\inclus A$ quel que soit $A$.  \\
$\cap$        &    \kwab{Intersection} ; $A\cap B$ est l'ensemble de tous les éléments qui sont dans $A$ et dans $B$. \\
$\cup$        &    \kwab{Union} ; $A\cup B$ est l'ensemble des tous les éléments qui sont dans $A$ et/ou dans $B$. \\
$-$        &    \kwab{Différence ensembliste} ; $A-B$ est l'ensemble de tous les éléments qui sont dans $A$ mais pas dans $B$. \\
$\powerset(\,)$        &    \kwab{Ensemble des parties} d'un ensemble ; $\powerset(A)$ est l'ensemble composé de tous les sous-ensembles de $A$ ;  exemple : $\powerset(\set{a;b;c})=\set{\eVide;\set a;\set b;\set c;\set{a;b};\set{a;c};\set{b;c};\set{a;b;c}}$.\\
\end{tabular}
%
\begin{tabular}{lp{4.5cm}} 
$\Card{\cdot}$ & \kwab{Cardinal} d'un ensemble. $\Card A$ est le nombre d'éléments que contient $A$.\\
\tuple{\,}        &    Contenu d'une \kwab{liste} ou d'un \kwab{$n$-uplet}.  \\
$\times$        &    \kwab{Produit cartésien} d'ensembles.  $A\times B$ est l'ensemble de tous les 2-uplets de la forme \tuple{x,y} avec $x\in A$ et $y\in B$.\\
$\mapsto$        &   Descripteur de  \kwab{fonction}: $x\longmapsto f(x)$ est la fonction qui à tout $x$ associe $f(x)$. \\
$B^A$        &    Ensemble de toutes les \kwab{fonctions} allant de l'ensemble $A$ vers l'ensemble $B$, parfois noté aussi $A\Vers B$. \\
%...        &    ... \\
\end{tabular}
%
\noindent%
\begin{tabular}{lp{4.5cm}} 
\multicolumn{2}{l}{\bfseries Logique et sémantique :}\\
$\Xlo\neg$ & \kwab{Négation}.\\
$\Xlo\wedge$ & \kwab{Conjonction}.\\
$\Xlo\vee$ & \kwab{Disjonction}.\\
$\Xlo\implq$ & \kwab{Implication matérielle}.\\
$\Xlo\ssi$ & \kwab{Équivalence matérielle}.\\
$\Xlo=$ & \kwab{Identité} (de dénotation).\\
$\Xlo\forall$ & Symbole de \kwab{quantification universelle}.\\
$\Xlo\exists$ & Symbole de \kwab{quantification existentielle}.\\
$\Xlo\atoi$ & Opérateur de \kwab{description définie}.\\
$\Xlo\doit$ & Opérateur de \kwab{nécessité}.\\
$\Xlo\peut$ & Opérateur de \kwab{possibilité}.\\
$\Xlo\doitn{n}$ & Variante multimodale de l'opérateur de \kwab{nécessité}.\\
$\Xlo\peutn{n}$ & Variante multimodale de l'opérateur de \kwab{possibilité}.\\
\end{tabular}

\noindent%
\begin{tabular}{lp{4.5cm}} 
$\mP$ & Opérateur de \kwab{passé}.\\
$\mF$ & Opérateur de \kwab{futur}.\\
$\Xlo\Intn$ & Opérateur d'\kwab{intension\-na\-li\-sa\-tion} (ou intenseur).\\
$\Xlo\Extn$ & Opérateur d'\kwab{extension\-na\-li\-sa\-tion} (ou extenseur).\\
$\Xlo\lambda$ & Opérateur d'\kwab{abstraction}.\\
$\Xlo\gand$ & \kwab{Conjonction généralisée}.\\
$\Xlo\gor$ & \kwab{Disjonction généralisée}.\\
$\Xlo\gneg$ & \kwab{Négation généralisée}.\\
$\satisf$ & \kwab{Conséquence logique} et \kwab{satisfaction}.\\
$\denote{\cdot}$ & \kwab{Valeur sémantique}.\\
$1$ & \kwab{Vrai}.\\
$0$ & \kwab{Faux}.\\
$\Modele$ & \kwab{Modèle}.\\
\Unv A & \kwab{Domaine d'interprétation} (ensemble de tous les individus du modèle).\\
\FI & \kwab{Fonction d'interprétation} des constantes non logiques.\\
$\leadsto$ & Relation de \kwab{traduction} de la langue naturelle vers le langage sémantique.\\ 
$\Ftrad$ & Fonction de \kwab{traduction} de la langue naturelle vers le langage sémantique.\\
$\Tps$ & Ensemble des \kwab{instants}.\\
$\Unv W$ & Ensemble des \kwab{mondes possibles}.\\
$\ME$ & Ensemble des \kwab{expressions interprétables} du langage sémantique.\\ 
$\DoM$ & \kwab{Domaines de dénotation}.\\
\typ e & \kwab{Type} des expressions dénotant des \kwab{entités}.\\
\typ t & \kwab{Type} des expressions dénotant des \kwab{valeurs de vérité}.\\
\typ s & \kwab{Type} des \kwab{indices intensionnels} (mondes possibles).\\
\Types & \kwab{Ensemble des types} du langage sémantique.
\end{tabular}

\noindent%
\begin{tabular}{lp{4.5cm}} 
\multicolumn{2}{l}{\bfseries Jugements linguistiques :}\\
$^*$ & \kwab{Agrammaticalité}.\\
{\zarb} & \kwab{Anomalie} sémantique ou \kwab{inadéquation} contextuelle.\\
{\urgh} {\uurgh}& \kwab{Acceptabilité douteuse} ou très douteuse.\\
\end{tabular}

\noindent%
\begin{tabular}{lp{4.5cm}} 
\multicolumn{2}{l}{\bfseries Abréviations :}\\
ssi & \emph{Si et seulement si}.\\
{\ie} & \emph{Id est} (c'est-à-dire).\\
t.q. & \emph{Tel que}.\\
ang. & \emph{En anglais}.\\
{\vs} & \emph{Versus} (contre, par opposition à).\\
\LO & \emph{Langage objet}, langage de représentation sémantique.\\
\LOz & \emph{Langage objet} «à deux sortes».
\end{tabular}
 
\end{multicols} 
