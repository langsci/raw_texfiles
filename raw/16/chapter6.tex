\chapter{Conclusion}\label{sec6}
\epigraph{1.25in}{In substance lies\\A form that's pure\\That is all lies\\I'm not so sure}{Nino Logoratti}{}

The working hypothesis which animated this book was that insights from research on phonetic detail at the prosodic level can be usefully incorporated into phonological models of intonation. This is consistent with our historist understanding of phonetic detail as systematically produced and perceived phonetic information which is not yet included in abstract phonological representations.\is{phonetic detail} Under such perspective, if phonological categories are flexible enough to be enriched with phonetic information which proves to be systematically produced and perceptually relevant, phonetic detail is not only consistent with exemplar-based approaches, but can also lead to a refinement of accounts based on abstractionist assumptions. 

In this book I explored whether and how one particular abstractionist model of intonation, the Autosegmental-Metrical (\textit{AM}) framework, should account for detailed phonetic information in \textit{f0} contours and durational patterns. The evidence gathered in the experimental chapters will be reviewed in the next section (Section~\ref{sec61}), by grouping results according to their relevance to production or perception and to intonation or tempo (see Table~\ref{tab11} in Section~\ref{sec13}). I will then provide a brief overview of the tools for the exploration of prosodic detail developed or fine-tuned across the various experimental chapters, thus grouping the methodological outcomes of this work in Section~\ref{sec62}. I will conclude by discussing the wider theoretical implications of our findings and commenting on the polyvalence of prosodic detail, which can be accommodated in both exemplar-based (Section~\ref{sec631}) and abstractionist (Section~\ref{sec632}) accounts of prosody.

\section{Summary of findings}\label{sec61}
Evidence from the experimental chapters points to the need of a partial enrichment of phonological categories in the AM framework. By examining functional contrasts between narrow focus questions and both partial topic statements (Section~\ref{sec2} and Section~\ref{sec32}) and narrow focus statements (Section~\ref{sec33} and Sections~\ref{sec4}--\ref{sec5}), we have found that some phonetic detail in the shape of \textit{f0} contours should be included in abstract representations of tunes (Sections~\ref{sec2}--\ref{sec3}), whereas phonetic information about durational patterns can indeed be regarded as negligible detail (Sections~\ref{sec4}--\ref{sec5}), at least for this contrast and in this variety. In the following subsections, instead of presenting results for individual studies as in the experimental chapters, we group them according to the phonetic dimension involved (melodic detail, Sections~\ref{sec2}--\ref{sec3}, see Section~\ref{sec611}; temporal detail, Sections~\ref{sec4}--\ref{sec5}, see Section~\ref{sec612}) and to the mechanisms explored (production, Sections~\ref{sec2}--\ref{sec4}, and perception, Sections~\ref{sec3}--\ref{sec5}, see Section~\ref{sec613}).

\subsection{Intonation}\label{sec611}
In the AM framework, continuous phonetic information relative to \textit{f0} contours is discretized into phonological tunes. Tunes are composed by a series of tonal events, namely pitch accents and boundary tones, which are phonetically represented by points in the \textit{f0}-time plane. As a result, \textit{f0} contours between such tonal events are considered as context-determined, inferable by rule, and, ultimately, phonologically irrelevant.\is{f0 dynamics} However, we provided in Section~\ref{sec2} some evidence for systematically produced differences in the \textit{f0} contour between the two tones composing the nuclear pitch accents in narrow focus questions (\textit{QNF}) and partial topic statements (\textit{SPT}) in Neapolitan Italian (NI). Both pitch accents are realized phonetically as a rise which begins at stressed syllable onset and reaches its peak at the end of the stressed vowel. Alignment and scaling of rise start and end are not significantly different in the two contexts, but the \textit{f0} contour between the two is: the rise is more convex in QNF and more concave in SPT.\is{partial topic} If \textit{f0} contours had to be reduced to a sequence of points on the \textit{f0}-time plane connected by irrelevant interpolations, there would be no way to account for these observed regularities in production. 

As Section~\ref{sec32} shows, however, differences in \textit{f0} rise shape do not seem to be used as a perceptual cue to the contrast between QNF and SPT. We resynthesized stimuli at different points along a continuum of rise shape, ranging from very concave to very convex. Listeners' responses to a two-alternatives forced-choice identification task showed no correlation with stimulus manipulation. That is, it seems that differences in rise shape, while consistently produced, are not always used as a perceptual cue to pragmatic contrasts. Under these circumstances, pitch accent internal rise shape can not be considered as phonetic detail, and does not need to be included in the phonological representations of nuclear pitch accents in order to contrast QNFs and SPTs. Nuclear pitch accents in both contexts might use the prosodic transcription already suggested in the literature for QNF, namely L*+H. The contrast between the two contexts is rather expressed at the tune level, by different paradigmatic options in terms of boundary tones and postnuclear pitch accents.

The fact that rise shape does not play a perceptual role in contrasting QNF and SPT does not mean that rise shape is phonologically irrelevant altogether. In Section~\ref{sec33} we examined the perceptual role of rise shape differences in another pragmatic contrast, the one between QNF and narrow focus statements (SNF), whose nuclear rises are also more concave compared to questions.\is{sentence modality} It has been long acknowledged that the contrast between QNF and SNF is primarily signalled by differences in tonal alignment - that is, in the synchronization of \textit{f0} movements with the segmental string. As we said above, in QNF the \textit{f0} peak is reached at the end of the stressed vowel; in SNF, on the other hand, the peak is reached around the stressed vowel midpoint, and the pitch accent is accordingly transcribed as L+H*. We hypothesized that, if alignment information were made ambiguous, rise shape could have been the only cue for listeners to rely on. A two-alternative forced choice identification task of stimuli with ambiguous alignment showed that listeners do use phonetic information in rise shape when categorizing (narrow focus) questions and statements.

These findings do not necessarily have to impact the conventions in use for prosodic transcription in the AM framework. We can still continue to label NI nuclear pitch accents as L*+H in questions and as L+H* in statements, as long as we acknowledge that these are used as shortcuts to richer phonetic descriptions.\is{pitch accent} This might not always be the case, as shown by \citeauthor{petrone2011tones}'s \citeyearpar{petrone2011tones} work on phonetic information in the prenuclear region, according to which a new structural position (a phrase accent) is required to account for sentence modality contrasts. It is important to stress that the exploration of melodic detail is consistent with different outcomes, ranging from the validation of information reduction (as we have seen in the case of QNF vs SPT rises) to an enrichment of phonetic representations which does not affect transcription conventions (as in the case of QNF vs SNF rises) and to the suggestion of different structural interpretations (as in the case of prenuclear falls across sentence modality). The interest of studying phonetic detail lies indeed in this rich range\enlargethispage{1em} of solutions which can be suggested for the research questions it raises.

\subsection{Tempo}\label{sec612}
Phonetic information fed into phonological categories in the AM framework is not only reduced with respect to the discretization of \textit{f0} contours into a sequence of contrastive tonal events and irrelevant transition. Information is also reduced by concentrating on \textit{f0} contours alone, thus discarding information on other dimensions, such as duration, intensity, voice quality and spectral proprieties. We thus tested whether sentence modality contrasts (again QNF vs SNF) are characterized acoustically by differences along other dimensions, and whether eventual differences are used as perceptual cues. We decided to focus on the temporal dimension, since in the last ten years the literature on sentence modality contrasts has shown that questions and statements often differ with respect to either global measures of speech rate or local measures in the duration of linguistic units of various sizes, ranging from segments to phonological words.\is{tempo}

We thus collected two corpora of sentences uttered as both questions and statements, by controlling focus placement as well. Results of a first experiment (Section~\ref{sec43}) show that global utterance duration and thus speech rate do not vary across sentence modality. Whereas \citet{vanheuven2005speech} suggested that questions might display a universal trend to faster speech rate just as they show a trend to higher pitch, our findings are rather consistent with language-specific encoding of sentence modality contrasts.\is{frequency code} Differences between questions and statements along the temporal dimension, however, were found when analyzing our corpora in more detail. If overall utterance duration is the same, segmental durations have been found to vary in the two conditions. In particular, statements have longer initial segments, whereas the final segment (a vowel in our corpora) is systematically longer in questions. The magnitude of these effects is not negligible, especially for final vowel in questions, which are about 20 ms longer than in statements. 

The existent differences in segmental duration within utterance of the same global duration suggested the use of an integrated metric for the evaluation of durational patterns. In a second experiment (Section~\ref{sec44}) we thus adapted the algorithm proposed by \citet{pfitzinger2001phonetische} in order to capture local variations of speech rate.\is{Local Phone Rate} This allowed us to show that speech rate indeed follows different patterns across sentence modality, being globally increasing in statements and decreasing in questions.

A subsequent experiment (Section~\ref{sec5}) was devised in order to establish whether durational differences at the segmental level were also consistently used as perceptual cues, in which case they should be considered as relevant prosodic detail and be somehow incorporated into phonological representations of sentence modality contrasts. We had to manipulate durational patterns independently of \textit{f0} contours, which represent by themselves a very strong cue to sentence modality contrasts. As in the case of the perceptual study on melodic detail in QNF vs SNF pitch accents (Section~\ref{sec33}), in which peak alignment information was made unavailable in order to evaluate the role of rise shape, in the study of durational patterns we manipulated the test stimuli so as to have an ambiguous \textit{f0} contour. In addition, unlike the previous experiment on melodic detail, we also manipulated durational patterns in utterances with clear question or statement intonation.\is{resynthesis} This enabled us to assess the perceptual importance of temporal information, by testing whether it is used constantly and independently (that is, in addition to intonational cues) or only when other primary cues are not available. 

A two-alternative forced-choice identification task showed however that listeners' responses are not affected by manipulations of durational patterns, not even when intonation was made ambiguous. These results are not consistent with the hypothesis that temporal detail is evaluated as a perceptual cue in its own right. This finding has been replicated in a shorter experiment, in which subjects only listened to intonationally ambiguous stimuli, in order to maximize their attention on temporal cues. However, the fact that listeners responses are not affected by temporal manipulations does not entail that durational differences are not processed at all. Listeners might perceive durational information but ultimately discard it when intonational cues have been evaluated. For this reason, we measured reaction times to stimuli with either congruous or incongruous cues on the melodic and temporal level. Stimuli with congruous information (e.g. with statement-like \textit{f0} contour and durational pattern) were predicted to elicit faster responses than stimuli with incongruous information (e.g.with statement-like \textit{f0} contour and question-like durational pattern). However, this prediction was not borne out either. Reaction times are only slightly longer when intonation is ambiguous - a fact which contributes to show that, in NI sentence modality contrasts, durational information is negligible detail.

\subsection{Production and perception}\label{sec613}
We are thus faced with an extremely interesting pattern of results, where production experiments show consistent acoustic differences in both melodic detail (between QNF and SPT nuclear rises) and temporal detail (in durational patterns across sentence modality), but perceptual experiments fail in attesting their use as perceptual cues. Of course, we cannot exclude that our negative results in perception are due to poor methodological choices in the set-up of the experiments. However, especially in the case of the perception of temporal detail, the conditions for appropriate testing were probably met (see Sections~\ref{sec541}--\ref{sec542} for discussion). According to \citeauthor{frick1995accepting}'s \citeyearpar{frick1995accepting} ``good effort criterion'', we should even accept the null hypothesis of no durational information in phonological categories for sentence modality contrasts in NI, rather than simply stating that the alternative hypotheses are not supported.\is{good effort} In any case, this does not allow us to conclude that durational patterns play no role at all in the perception of any contrast in any language, and thus we cannot exclude that phonetic information at the temporal level is stored and used in perception of post-lexical contrasts, as exemplar-based approaches would predict.

Our goal, however, is not to rule out the possibility of an ``exemplar prosody'' altogether. We rather aim to show that evidence from both production and perception is needed when working on prosodic detail, from either an abstractionist or an exemplarist viewpoint. Quite recently, \citet{nguyen2009dynamical} observed that ``much of the available evidence for long-term storage of FPD in the mental lexicon comes from studies of speech production''. The observation is even more true for research on exemplar prosody, which deals exclusively with production data, as we will see shortly (Section~\ref{sec631}). This is understandable, since research in this field is still very young. But when suggesting a new understanding of phonological structures, evaluating the impact of phonetic detail on perception is no less important. This is clearly shown, for example, by research on incomplete neutralization, dealing with allegedly neutralized phonological contrasts which are still reflected by surface phonetic differences. 

For example, a devoicing process has been said to neutralize voicing contrasts in domain-final obstruents in German, thus making \textit{Rat} (advice) and \textit{Rad} (wheel) homophones.\footnote{Among the vast bibliography on the topic, see \citet{port1981neutralization,odell1983discrimination,charlesluce1985word,port1985neutralization,port1989incomplete,port1996discreteness,kleber2010implications,rottger2011robustness,winterFORTHnature}. Similar phenomena have been explored in other languages, such as Dutch \citep{warner2004incomplete,warner2006orthographic,ernestus2006functionality}, Russian \citep{pye1986word,dmitrieva2010phonological}, Polish \citep{slowiaczek1985neutralizing,slowiaczek1989perception} and Catalan \citep{dinnsen1984phonological,charlesluce1987reanalysis}. We exclude from our review the seminal paper by \citet{dinnsen1971three}, which was unfortunately not available to us. Experimental results or theoretical arguments against incomplete neutralization are provided by \citet{fourakis1984incomplete,mascaro1987underlying,jassem1989neutralization,kopkalli1993phonetic,manaster1996letter}.} However, subtle sub-phonemic durational differences can be found in speakers' production of underlying voiceless and devoiced obstruents.\is{incomplete neutralization} Along the phonetic continua of vowel duration, burst duration and closure voicing, devoiced obstruents are somewhere in between the extremes occupied by voiced and voiceless sounds. Crucially to our discussion, the perceptual role of this consistently produced phonetic detail has been investigated since the very first studies on incomplete neutralization - that is, at least since \citet{port1981neutralization}. Constant methodological improvements enabled the exclusion of possible experimental confounds, as in the case of orthography-induced biases \citep{rottger2011robustness}. Likewise, determining whether such contrasts are perceptible is instrumental in deciding of their functional role: whereas \citet{port1981neutralization} first thought that ``this `semicontrast' must be nearly useless in conversation'', \citet{ernestus2006functionality} recently suggested that incomplete neutralization might be ``a subphonemic cue to past-tense formation'' in Dutch. Ultimately, it is this long-term exploration of both production and perception mechanisms which allowed researchers to recast the incomplete neutralization issue in abstractionist/exemplarist terms, as in \citet{kleber2010implications}. We hope that our investigation of prosodic detail, however far from conclusive, might at least demonstrate that the recent work on exemplar prosody based on production evidence (see Section~\ref{sec631}) must be necessarily complemented by a thorough examination of perceptual mechanisms.

\section{Tools for prosodic detail research}\label{sec62}\is{ASSI}
Besides suggesting a potentially interesting research topic, we also aimed at providing some experimental tools which might be useful in its exploration. This was particularly needed in the case of the study of temporal detail, which has not been analyzed in the literature as thoroughly as melodic detail. However, the tools briefly presented in the experimental chapters on temporal detail might also prove relevant in the study of prosodic detail in general. 

\subsection{Automatic Speech Segmentation for Italian}\label{sec621}
The study of temporal detail in production required the collection of a matrix with a great number of segment durations (Section~\ref{sec422}). Pooling data from the \textit{Orlando} and the \textit{Danser} corpora (see Section~\ref{sec421}), we had to segment 2376 utterances, each composed of 8 CV syllables. With more than 35.000 segmental boundaries to be placed, manual segmentation was simply not an option. However, tools for automatic segmentation of Italian were not available either.\footnote{\textit{EasyAlign} \citep{goldman2011easyalign} only works with French, English, Brazilian Portuguese, Spanish and Taiwan Min, while \textit{SPPAS} \citep{bigi2012speech}, which works with French, English, Italian and Chinese, was only released after our experiment was planned, executed and published.} Our solution has been to develop our own tool for Italian forced alignment, \textit{ASSI} \citep{cangemi2011automatic}. In forced alignment, audio files are segmented according to an orthographic transcription and a phonetized lexicon provided by the user. The first is a plain text file containing for each row an audio file name and its orthographically transcribed content, as in (1) for the \textit{Danser} transcription file:

\begin{description}
   \item[(1)] {\tt Q1BD1.wav danilo\_vola\_da\_roma}
\end{description}\label{ex61}

\noindent The second contains for each row an orthographic word-form and a phonetic transcription of the expected pronunciation (variants are allowed), as in (2) for the \textit{Orlando} lexicon file: 

\begin{description}
   \item[(2)] {\tt ralego r:a:l:[e:/E:]g:o:}
\end{description}\label{ex62}

Forced alignment is especially suited for the segmentation of read speech, since for this kind of data the experimenter can provide an orthographic transcription with no effort. Moreover, when working on sentence modality and/or focus placement in NI, which only use prosodic cues to express these contrasts, the use of forced alignment is even more indicated: the same orthographic transcription based on the same phonetized lexicon can be used for a variety of experimental items. For example, the \textit{Orlando} corpus contained three sentences composed by 16 segments. These were uttered in the six combinations between the two levels of the sentence modality factor (question, statement) and the three levels of the narrow focus placement factor (on subject, verb or object). Each of 30 speakers read three repetitions of three 16-segment sentences uttered in six contexts, thus requiring the extraction of 25.920 phone durations in total. This was accomplished by providing a single lexicon file with the phonetic transcription of \textit{six} words, and a single transcription file containing orthographic transcriptions for \textit{three} sentence types.

\subsection{Multi-parametric continuous resynthesis}\label{sec622}\is{resynthesis}
Whereas the exploration of temporal detail in production required a tool which merely speeded up an already existing procedure (viz. manual segmentation), to test our hypotheses on the perception of temporal detail we had to resynthesize stimuli using a new procedure altogether (see Section~\ref{sec52}). The procedure is partly based on \citet{gubian2010automatic,gubian2011joint}, and through a set of \textit{Praat} \citep{boersma2008praat} and \textit{R} \citep{r2008r} scripts yields input files for the \textit{PSOLA} \citep{moulines1990pitchsyncronous} resynthesis engine in \textit{Praat}. Given two utterances, the question and statement version of a same sentence, we needed to resynthesize each one using (1) durational patterns and/or \textit{f0} contours of the other one (\textit{cross-modality} resynthesis) and (2) ambiguous durational patterns and/or \textit{f0} contours between the two (\textit{ambiguous} resynthesis).

However, as we have seen discussing NI intonation (see ~\ref{sec123}), the synchronization of \textit{f0} movements with the segmental string is crucial in signalling sentence modality contrasts. This means that, as far as cross-modality resynthesis is concerned, intonational and temporal cues must be jointly manipulated: one cannot simply extract the \textit{f0} of the first utterance and use it to resynthesize the second. Similarly, segmental durations cannot be modified without transforming \textit{f0} contours as well. Thus we extracted \textit{f0} contours and segmental durations for each utterance, then the two \textit{f0} contours were time-warped by aligning their corresponding segmental boundaries. This landmark registration procedure creates two intermediate contours having identical underlying phone durations. These intermediate contours can be combined with actual durational patterns and thus be ready to be resynthesized onto an actual utterance. 

The results of cross-modality resynthesis are particularly satisfying. As we have seen in Section~\ref{sec541}, listeners' responses to stimuli resynthesized by applying question \textit{f0} contours onto statement bases are not significantly different from listeners' responses to natural questions. Informal testing shows encouraging results in the resynthesis of other contrasts as well, as for example in the case of focus placement, and even when the lexical material is different between the two sentences. For example, \textit{f0} contour and durational pattern of a (prepositional) \textit{object} narrow focus statement utterance of the sentence \textit{Danilo vola da Roma} were used to resynthesize a \textit{subject} narrow focus statement utterance of the sentence \textit{Serena vive da Lara} (see Section~\ref{sec4212}), yielding a stimulus which was identified as having narrow focus on the object. In this case, performances could be even improved by adding manipulation of landmark-registered intensity contours, which could be easily included as an additional module in the resynthesis procedure. Of course, these excellent results are at least in part motivated by the use of sentences with identical metrical and syllable structures at both ends of the resynthesis procedure. However, we believe that very different sentences could also be used, if phonologically motivated assumptions guided the choice of the landmarks to be registered.

Our second goal was the creation of ambiguous stimuli, with respect to \textit{f0} contours and/or durational patterns. This was achieved by averaging phone durations, in the case of durational patterns, and by averaging intermediate \textit{f0} contours (i.e. landmark-registered contours expressed in normalized time with identical underlying phone durations), prior to resynthesis. By using a simple average, we obtained \textit{acoustically} ambiguous stimuli. These stimuli would have also been perceptually ambiguous only if the perceptual space between questions and statements was perfectly linear. Unsurprisingly, this proved not to be the case (for a discussion of how this affected the interpretation of our results, see Section~\ref{sec541}): responses to stimuli with acoustically ambiguous intonation had a significant question bias, probably due to the postnuclear region.\footnote{The subject narrow focus utterances used in the experiment have an audible fall in the question condition (see Figure~\ref{fig201}, bottom panel) and flat \textit{f0} contour in the statement condition, which are respectively transcribed as !H* and !H+L*. However, in postnuclear position even slight \textit{f0} falls (as those in the acoustically ambiguous condition) can be salient, and thus bias the listener towards the question response. The impact of the postnuclear region on acoustical and perceptual ambiguity can thus be easily tested by using gated or object-focussed stimuli.} What is relevant to the present discussion is that our resynthesis procedure allows us to address very explicitly the issue of the difference between acoustically and perceptually ambiguous stimuli in a multidimensional prosodic space. In this sense, this procedure could prove a useful tool in the investigation of questions which are not directly addressed in this book.

\section{Theoretical implications}\label{sec63}
In this final section we interpret our findings on prosodic detail in Neapolitan Italian by discussing their relevance for recent exemplar-based approaches to prosody focussing on frequency effects in production (Section~\ref{sec631}). We conclude by highlighting that a close examination of phonetic detail is necessary for the construction of phonologically adequate categories (Section~\ref{sec632}): neither excessive unanalyzed phonetic information nor bony minimalist abstract categories are viable options when prosody is analyzed in production and perception.

\subsection{Exemplar prosody}\label{sec631}\is{exemplar prosody}
As we said in the introductory pages (Section~\ref{sec113}), an exemplar-based approach to prosody would provide a natural setting for the accommodation of prosodic detail. Let us flesh out this statement in this section. According to Johnson, \begin{quote}an exemplar is an association between a set of auditory properties and a set of category labels. The auditory properties are output from the peripheral auditory system, and the set of category labels includes any classification that may be important to the perceiver, and which was available at the time that the exemplar was stored - for example, the linguistic value of the exemplar, the gender of the speaker, the name of the speaker, and so on. \cite[147]{johnson1997speech}\end{quote}
As we explained above (see Section~\ref{sec111}), in this approach the normalization phase is no longer necessary: new prompts activate both ``linguistic value'', thus feeding word recognition, and indexical information (e.g. the gender and name of the speaker), thus feeding talker recognition. But what happens if \textit{f0} contours are stored as part of the auditory properties set, and pragmatic or information structure contrasts are stored as part of the category labels set (specifically, its ``linguistic value'')? By enriching exemplars with information on both the substantial and the functional sides, the model could perform talker recognition, word recognition and extraction of post-lexical meaning at the same time.

Recent research has addressed the issue of whether \textit{f0} contours are stored into exemplars and connected to post-lexical meaning.\footnote{For storage of fundamental frequency information connected to lexical contrasts, see \citet{sekiguchi2006effects}.} Most work has focussed on frequency effects in pitch accent production. Exemplar models have been extended to production since \citet{pierrehumbert2001exemplar}. In her most basic model, \begin{quote}the decision to produce a given category is realized through activation of that label. The selection of a phonetic target, given the label, may be modelled as a random selection of an exemplar from the cloud of exemplars associated with the label.\cite[§3.1]{pierrehumbert2001exemplar}\end{quote}
By positing activation of a region in the exemplar cloud rather than that of a single exemplar, the model can account for entrenchment effects, according to which productions become less variable with practice. In this case, phonetic variability is expected to decrease when the cloud is denser because the exemplars are produced and perceived with higher frequency.

Recent work by Katrin Schweitzer combines the hypothesis of \textit{f0} contour storage with predictions on entrenchment. In her model, \begin{quote}during speech production a speaker selects a stored exemplar as a production target. Assuming that pitch accents can be stored with the word, the speaker would select an exemplar that matches not only the intended word but also the intended pitch accent. \cite[138]{schweitzer2010frequency}\end{quote}
If, instead of selecting a single exemplar, a whole region of exemplars is activated, as in Pierrehumbert's refined model, entrenchment would predict that more frequent pitch accents are less variable. In the last few years, a number of corpus studies has used parametrized descriptions of pitch accents (based on \textit{PaIntE}, see \citealt{mohler1998parametric,mohler2001improvements}) to explore whether phonetic variability is affected by frequency of occurrence. The results seem to provide mixed evidence, ranging from the absence of frequency effects in German \citep{walsh2008examining}, to more prominent \textit{f0} movements in frequent word/pitch accent combinations \citep{schweitzer2010frequency} and to entrenchment in English collocations \citep{schweitzer2011prosodic}. In sum, even if the authors conclude that ``there is still a great deal to be understood about how lexicalised storage interacts with `top-down' information in the production of prosody'' \cite[4]{schweitzer2011prosodic}, these results are taken as supporting an exemplar-based view of prosody.

\subsection{Substance, form and function}\label{sec632}
This approach is surely intriguing, and we are persuaded that it will receive a great deal of attention in the coming years. Its elaboration, however, might benefit from a close inspection of its theoretical underpinnings, in order to rule out possible aporetic developments. At this point, it must be clear that in a model where exemplars are conceived as associations between \textit{f0} information and post-lexical function labels, there is no longer room for phonological representations, which are at best redundant. Substance is no longer linked to function by abstract forms, but rather through activation of exemplars using similarity functions. This is indeed the perspective of the so-called \textit{functional} approaches to prosody \citep{shriberg1998can,noth2000verbmobil,batliner2001prosodic}, which suggest that formal entities such as ``the unfortunate notion of pitch accent'' should be pruned by Occam's razor \citep[25]{batliner2005prosodic}.\is{pitch accent} 

However, two possible objections arise at this point. The first is that, as \citet[20]{ladd2008intonational} puts it, ``whether we should adopt a `phonological' approach to intonation is not primarily a matter of taste, but an \textit{empirical question}''. Or, in other words, in addition to dealing with previously unaccounted-for phenomena (such as frequency effects), an exemplar-based approach to prosody must also account for phenomena which have already been framed in phonological terms.\footnote{A few examples might be the role of accentedness in discourse structure \citep{hawkins1991factors}, the disentangling of linguistic and paralinguistic meaning \citep{scherer1984vocal}, and evidence from imitation studies \citep{cole2011phonology}.} This objection, of course, is nothing more than an empirical argument: in \citeauthor{smith1981categories}'s \citeyearpar[33]{smith1981categories} terms, ``it is a statement about what has happened so far, not about what can happen''. And since functional models have been seriously explored for only a decade, we surely cannot consider empirical arguments as conclusive.

The second objection is perhaps more cogent. It regards the covert use of phonological forms, even when the general system architecture is claimed to posit a direct link between phonetic substance and post-lexical meaning. In the discussion of frequency of usage in exemplar-based models of prosody, for example, we have seen that its effects have been explored in terms of phonetic variability of pitch accents. That is, even the arguments adduced in favour of exemplar dynamics eventually posit somehow abstract categories. At this point, it is unclear whether exemplars actually link phonetic substance to post-lexical categorical labels or rather to pitch accents - that is, forms which themselves bridge substance and functions.\footnote{This state of affair is already exemplified by the titles of relevant studies in this perspective, such as ``Relative frequency affects pitch accent realisation'' \citep{schweitzer2010relative} and ``Frequency of occurrence effects on pitch accent realisation'' \citep{schweitzer2010frequency}.} An operationalized version of abstractions corresponding to pitch accents also seems to be required by current text-to-speech systems. In \citeauthor{vansanten2000quantitative}' \citeyear[278]{vansanten2000quantitative} quantitative model, for example, phonetic differences between \textit{f0} contours interpreted as having the same function are accounted for by time warping of a common \textit{template}.

It is unclear, at this point, whether formal representations of prosody can really be dismissed, even in exemplar-based approaches. Functional approaches criticize AM-like phonological approaches to intonation because \begin{quote}The classical phonological concept of the Prague school has been abandoned in contemporary intonation models, namely that phonemes - be they segmental or suprasegmental - should only be assumed if these units make a difference in meaning. This functional point of view has given way to more formal criteria such as economy of description. Thus, the decision on the descriptive units is not based on differences in meaning but on formal criteria, and only afterwards are functional differences sought that can be described with these formal units. \cite[§1.1]{batliner2005prosodic}\end{quote}
However, it is crucial to stress out that, in principle, ``formal criteria'' \textit{include} consideration of contrasts in meaning. And that, in that very same Prague school, ``phonemes - be they segmental or suprasegmental'' are \textit{formal} entities. In this sense, despite their claims, functional approaches do not actually advocate for the suppression of phonological representations and of inventories of formal units. 

It is true that, in AM accounts of intonation, the mapping between forms and functions is often confusing. In discussing prenuclear fall shape across sentence modalities in NI (see Section~\ref{sec242}) and German (see Section~\ref{sec31}), we have seen a clear example of how meaningful phonetic detail can be accommodated by enriching either the tonal inventory or the sequential grammar. We agree with proponents of functional approaches that if a single pragmatic contrast is expressed by a given phonetic difference, having two competing phonological analyses is symptomatic of the unstable state of the formal descriptions available. But again, we must acknowledge that this objection is nothing more than an empirical argument: it does not prove that phonological representations are useless, but rather that they are not yet capable of providing a unified account. 

Ultimately, the central point is that if decisions on the descriptive units must be based on differences in meaning, then we need to start from a catalogue of different functions. But as we have seen in the discussion of the gap between segmental and supra-segmental phonology (Section~\ref{sec243}), there is no such thing as a theory-independent set of post-lexical functions. Moreover, individual theories of pragmatic and discourse meaning use as \textit{explicans} the very same set of phenomena which is \textit{explicandum} in a functional account of intonation.\footnote{One example is the case of B-accents in \citet{jackendoff1972semantic}.} The risk of circularity here is very high. Theories of intonation must acknowledge the need of a constant exchange between evidence provided by research on substance and by research on function: the formal level is indeed the central processor which permits the incorporation of insights coming from both directions.

Frequency effects on pitch accent variability shows that research on prosodic detail in production can provide arguments supporting an exemplar-based view of prosody, but also that no framework for the study of intonation has actually dismissed a somehow abstract level of representation altogether. Our results on the perception of prosodic detail support this view by showing that some consistently produced phonetic information does not function as a cue to some post-lexical contrasts. The strongest interpretation of these results is that an abstract representation in terms of phonological categories is useful and necessary in the study of intonation. However, current phonological models need to be refined with respect to both the richness of phonetic specification (as in the case of the nuclear rise shape across sentence modalities) and in the mechanisms used to link substance with function (as in the case of the competing analyses of prenuclear falls across sentence modalities). According to the minimalist interpretation, on the other hand, no inferences are drawn about the role of abstract forms in intonation, but whenever phonological categories are indeed assumed, they must be thoroughly explored in production and perception to rule out the exclusion of useful prosodic detail.

\enlargethispage{1em}In any case, the exploration of prosodic detail appears to be crucial for both exemplar-based and abstractionist approaches to intonation, and will probably provide the common ground for their integration into a truly hybrid model.