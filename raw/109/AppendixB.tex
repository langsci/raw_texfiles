\chapter{Historical and ethnological perspectives} 
\label{chap:appendixb}
\label{chap:historyanthropologysociology}

This Appendix aims to shed light on the “chains of societies” (\textit{chaînes de sociétés}: \citealt{amselle1990}; see also \citealt[329--331]{tryon1998}) that shaped Na ethnicity.

%Readers who wish to go straight to the heart of tonal matters can safely skip the 50"=page historical and ethnographic section (\sectref{sec:presentationofthenalanguageandnasocietyandreviewofearlierstudies}) and start from \sectref{sec:chronologyofthestudyelicitationproceduresandonlinematerials} or even from Chapter~\ref{chap:thelexicaltonesofnouns}.

%This long section (almost fifty pages) goes into more detail than is strictly necessary to the discussion of tones, and focused tonologists can safely skip it. / It provides background information about the Na and their language. 


\section{The history of Yongning in outline}
\label{sec:historicaloutline}

\subsection{Prehistory}
\label{sec:prehistory}

The \ili{Naxi} scholars Guo
Dalie \zh{郭大烈} and 
He Zhiwu \zh{和志武} believe that the name 
\textit{Móshā} \zh{摩沙} appearing in a~fourth-century chronicle refers, beyond any doubt, to “the \ili{Naxi}”, a~concept which they define as including the speakers of the Yongning Na language \citep[102-103]{guoetal1994}.\footnote{“This is the first certain and unequivocal mention of the {Naxi} in recorded history.” \textit{Original text:} \zh{这是纳西族在历史上首次明确无误的记录。}} They thus project today's ethnic identity into a~period distant by a~millenium and a~half. They proceed to track this people through a~sequence of changes in the {Chinese} terms used to designate it: the term \textit{Móshā} \zh{摩沙} used in the Jin dynasty is followed by \textit{Móxiē} \zh{磨些} in the Tang dynasty, then \textit{Móxiē} \zh{麽些}, \textit{Móxiē} \zh{摩些}, \textit{Mósuō} \zh{摩娑} and \textit{Mòxiē} \zh{末些} in the Yuan dynasty and later. As to the earlier origin of this people, they propose that it originates in an~admixture of Qiang \zh{羌} people to an~earlier aboriginal population, left unnamed.


\begin{quote} 
	In the formation of the \ili{Naxi} people, the main component consisted of aboriginals, who blended with Qiang \zh{羌} people, and later assimilated some other peoples at their periphery; conversely, in peripheral areas, some \ili{Naxi} were assimilated into other peoples. 
	\citep[24]{guoetal1994}\footnote{\textit{Original text:} 
		\zh{纳西族的形成以土著为主,融合了北来羌人,以后又同化了周围其他一些民族,边缘地区则是纳西族被其他民族同化。}} 
\end{quote}

%Command \noindent added to avoid having an indent. Proofreader suggestion: since this sentence continues the argument, it is better not to indent. 
{\noindent}A~further association is suggested between the \ili{Naxi} and the Shiguanzang \zh{石棺葬} culture, attested during the first millenium \,BC over areas that match present-day \ili{Naxi} and Na settlement \citep[66-67]{guoetal1994}. The association of incoming \il{Sino-Tibetan}Sino"=Tibetan peoples with a~certain type of graves raised hopes similar to the association of Kurgan pit-graves with the “Indo-Europeans” \citep{gimbutas1977,anthony2010}. The Shiguanzang culture was characterized by telltale stone graves, typically located on tablelands near sites of confluence between rivers, and short bronze swords of a~type also attested in China's northern steppes. The absence of clear attestation of associated settlements is suggestive of a~nomadic, pastoral people using metal, contrasting with the~people indigenous to the area, whose abundant settlements are clearly indicative of a~Neolithic agriculturalist culture transitioning into the Bronze Age. The nomadic people are identified with the \ili{Yi} \zh{夷} tribes of {Chinese} chronicles, considered as ancestor to the present-day \ili{Yi} \zh{彝} ethnic group as well as to the Na and \ili{Naxi} \citep[64-66]{guoetal1994}. 

This fits the overall scenario of migration of speakers of \ili{Sino-Tibetan} languages from the valley of the Yellow River, hypothesizing the Yangshao \zh{仰韶} culture as the point of origin, c.~5000\,BC to 3000\,BC: one of the main {reconstructed} lines of migration is “south"=west down the river valleys along the eastern edge of the \ili{Tibetan} plateau through what has been called the \textit{ethnic corridor}” (\citealt[236]{lapolla2001}; emphasis in original).

However, Guo Dalie \& He Zhiwu's proposals raise issues such as to what extent archaeological evidence lends itself to pigeon-holing into such broad cultural types, and on what evidence the naming of peoples in Chinese chronicles was based. “Most excavation reports describing and interpreting burial material from Southwest China tend to associate grave type with archaeological culture; hence their urgent desire to arrive
at a~clear classification of burial types; however, ({\dots}) one cultural or ethnic group can be characterized by
a~number of different burial rituals, while other practices might be common across such
boundaries” \citep[31]{hein2013}. 

A~1,400-page study of cultural geography and interregional contacts in prehistoric times offers a~more fine"=grained exploration than was possible at the time of writing of Guo \& He's \textit{History}. Systematic study of the available evidence leads to distinguish no fewer than “four subregions
showing fairly distinct archaeological assemblages, burial patterns, and subsistence
systems, indicating that they were probably inhabited by different cultural groups” \citep[588]{hein2013}. 

The first is that of the Anning river \zh{安宁河} valley. Settlements from the third millenium\,BC yield relatively similar finds, including “coarse sand-tempered low fired red-brown ceramics (mainly large urns with finger-tip impressed appliqué strip below the rim, \textit{bo} and \textit{wan} bowls, vases, and a~few lids and rarely spouts), accompanied by a~few polished stone woodworking tools, arrowheads, and among the surface finds also perforated
stone-knives” \citep[559]{hein2013}. The interpretation provided is that the communities in these settlements “probably shared
similar cultural tradition and thus identified themselves as part of the same larger group” \citep[589]{hein2013}. One of the sites, Dayangdui \zh{大洋堆}, shows evidence of outside influence c.~2000\,BC followed by {assimilation}: 

\begin{quotation}
	Both in ceramic quality and
	execution, the early Dayangdui ceramics ({\dots}) strongly resemble ceramics from sites in 
	Gansu \zh{甘肅} and Qinghai \zh{青海} attributed to the Qijia \zh{齊傢} culture. It is therefore not
	unlikely that the earth-pit graves at Dayangdui were built by a~group of Qijia origin. This would
	suggest a~date between 2200 and 1800\,BC ({\dots}). \citep[562]{hein2013}
	
	The middle and late Dayangdui assemblages do not contain any metal objects, however,
	and they show a~mixture of both early Dayangdui and local Neolithic trades that indicate some
	form of acculturation of the group of immigrants. As no similar sites of clear foreign origin have
	been identified in the Anning River Valley, it is likely that migration of whole groups from the
	North occurred rarely. \citep[594]{hein2013}
\end{quotation}	

%Command \noindent added to avoid having an indent. Proofreader suggestion: since this sentence continues the argument, it is better not to indent. 
{\noindent}In this area, there appeared megalithic graves, which were then imitated in neighbouring areas. 	

\begin{quotation}
	Graves with stone-construction parts are common throughout Southwest China, but
	megalithic graves seem to be unique to the Anning River Valley. The ceramics associated with
	these graves indicate a~local origin of this burial tradition in the Xichang area. This impression is
	supported by the fact that all early megalithic graves ({\dots}) are located in Xichang, while the megalithic graves in other regions
	such as Dechang, Mianning, Puge, and Xide all date to Phase IIa at the earliest. Why this kind of
	burial mode arose is uncertain, but its overall development and spread is relatively clear: it
	started with small constructions used for a~single instance of interment of several people,
	possibly in a~secondary mode of burial. During or after the burial, communal drinking rituals
	took place which seem to have become more extensive over time, as the large number of
	drinking vessels both in later graves and related ceramic pits shows. ({\dots})
	
	As far as daily life and mode of subsistence are concerned, the tool assemblages from
	megalithic graves and related settlement sites in the Anning River Valley show an agricultural
	and probably settled mode of living involving the planting of rice and probably other cereals,
	often supplemented by hunting, and in some places fishing. Only the sites in Puge show a
	continued primary reliance on hunting. Metal seems to have mainly been used for personal
	ornaments and only secondarily weapons or tools. \citep[595-599]{hein2013}
	
\end{quotation}	

Archaeological remains from the Neolithic to Bronze Age collected in Yongning in 1958 are considered as connected to the megalithic graves \citep[933]{hein2013}; it is relevant to the history of Yongning that the spread of this burial mode is hypothesized to have taken place through cultural diffusion (presumably by persons who had participated in communal rituals and later reproduced these patterns in their home settlements) rather than through military conquest. It is for future excavations to verify the existence of this connection. 

The second subregion is one of remote mountains, a~harsh environment where “groups of different origins conducted different kinds of
burial rituals next to each other, apparently respecting each other's monuments and even
adopting part of each others' burial customs and objects. In this meeting place of different groups,
cultural and other forms of identity (or at least their expression in the choice of grave form, burial
mode, and object assemblage) thus seem to have been extremely fluid” \citep[602]{hein2013}.

The third subregion is one of fertile valleys, to the Southeast. The inhabitants of the earliest settlements “practiced a~hunter-gatherer lifestyle, using caves and open-air sites either as seasonal or hunting camps rather than living in permanent settlements”, later “practising incipient agriculture in a~particularly
congenial environment, living either in permanent or semi-permanent settlements” \citep[605-606]{hein2013}.

The fourth and final subregion is that of the high-altitude mountains, plateaus and valleys of the Southwest, a~geographical area that includes Yongning. 

\begin{quotation}
	The people living in Yanyuan \zh{盐源} and Ninglang \zh{宁蒗} ({\dots}) belonged to a~clearly
	separate cultural group for whom armed combat~-- sometimes combined with horseback-riding~-- was a~central part of their life and identity. The emphasis on horse"=riding, the interment of horse heads and sheep shoulder blades in
	graves and the overall metal assemblage (in particular the staff heads) seen at Yanyuan are
	essentially foreign to the research area. Pictorial evidence for horse"=riding is known from the
	Dian \zh{滇} culture context, but horse skulls or long bones have never been. The interment of horse
	bones is instead common in the Northern Steppe and the Ordos region, and elements of horse
	gear similar to those seen in Yanyuan have been reported from there as well. ({\dots})
	It is therefore likely that the burying group of the “warrior graves” in Yanyuan is of a
	northern origin, be it the upper Minjiang \zh{岷江} or even the steppe. \citep[616-618]{hein2013}
\end{quotation}


%\il{Sinitic}Chinese historical sources on Na and \ili{Naxi} history as summarized by \citet{chavannes1912}
%(reprinted in \citealt{bacot1913}); the historical outline presented by \citet{gros1996}; and Chapter 9 of \citet{mathieu2003}. 


%
%"That such a~labor-intensive custom as erecting megalithic graves was taken up
%by groups that were originally culturally distinct, is likely the outcome of the personal encounter
%with the powerful rituals surrounding them and the bonds they created between those who
%participated in them, forming a~supra-local sense of community throughout the Anning River
%Valley and neighboring regions, leading to the emergence of a~new kind of identity that
%transcended the previous cultural and local group boundaries without necessarily destroying
%them, as the continuation of local particularities in each of the various location shows." 624
%
%"However, it is not enough to point out single objects or features and then jump to
%inferences such as "migration," "influence," or "contact," which are so readily used as blanket
%explanations for similarities between the material remains in different places. In a~first step, we
%have to define what we mean by "contact," what different types of contact there may be, and how
%we can identify them in the archaeological record. The challenging terrain of the research area
%furthermore makes it imperative to consider possible routes and reasons for contact."

Yanyuan \zh{盐源}, a~strategic area due to its abundance in rock salt, was conquered in 225\,AD by the Chinese, who refer to the population that they defeated there as the “Mosha” \zh{摩沙}. Returning to the claim of \ili{Naxi} historians Guo Dalie and He Zhiwu that these “Mosha” are the ancestors of today's \ili{Naxi}, the hypothesis could be rephrased as follows: military defeat led clans of this population of warriors to withdraw West, conquering new territories that included Yongning, the banks of the Yangtze, and later the plain of Lijiang~-- which to this day remain their area of settlement. 

But there does not seem to be decisive evidence linking the Yanyuan “warrior culture” to the \ili{Naxi} rather than with other \ili{Sino-Tibetan} groups, who later migrated into more distant areas in present"=day Yunnan and Burma. A~small piece of evidence on this topic comes from the path that ritual practitioners dictate to the soul of the deceased for its journey back to the ancestral homeland. These ritual journeys, which have been studied in many parts of East and Southeast Asia, can be shown to relate to historical migrations \citep{blackburn2004, gaenszle2012, mckhann2012}. Different clans, in different \ili{Naxi} areas, have different paths, but they all pass back through Yongning, where they join with the paths of the Na, and continue northward \citep[50-55]{guoetal1994}. The identification of place names soon becomes difficult or impossible as the distance from Yongning increases. The identification of Minya Konka \zh{贡嘎山} as the endpoint of the journey is not unlikely to be a~later addition based on the prestige of this 7,500-meter high mountain, which is a~mooring point for traditions and beliefs of various peoples of this part of the Himalayas. On the other hand, it may be relevant that the path for returning souls does not go through Yanyuan: there is thus no evidence of the group's forebears ever dwelling further East than the Yongning area. Of course, this piece of ethnological evidence does not carry considerable weight, as paths could have been modified at any point in the chain of {oral transmission}. But it suggests that the identification of the “Mosha” \zh{摩沙} as direct ancestors of the Na and \ili{Naxi} should not be taken as~proven.

\begin{quotation}
	Local oral narratives and scholarly writings which discuss the origins 
	and migrations of hill peoples of the far eastern Himalaya often share the 
	same propositions. Firstly, both types of sources tend to plot routes of  
	migration between an assumed original homeland area or origin place and 
	a present"=day dwelling location; direction of movement and itineraries 
	are of shared importance here. Second, they both claim identification of
	contemporary populations with their purported ancestors from past times 
	and distant places, with implicit and explicit claims of ethnic continuity. \citep[83]{huber2012}
\end{quotation}

Whatever the exact relationship of the “Mosha” to the present-day Na and \ili{Naxi}, conquest of the Yanyuan area by the Chinese in 225\,AD was a~major landmark. Ties with the heart of the Chinese empire were established, and never entirely cut off thereafter, even during periods when the Chinese central power was weakest, such as the following four centuries. 

\subsection{Empires and indigenous chieftains}
\label{sec:feudal}

%: archaeology reveals major discontinuities during the Han dynasty, as “the thousand year old tomb-building culture of the Zuo \zh{笮} was suddenly replaced by the earth graves culture of the Kunming \zh{昆明} and Sui \zh{嶲} tribes, and, in the north east, by the tomb building culture of the Han” \citep[366]{mathieu2003}. 
%\il{Sinitic}Chinese chronicles provide glimpses into a~troubled history, allowing for the building of hypotheses such as that the Yongning area was conquered in the first century \,AD “by the Kunming \zh{昆明} and Sui \zh{嶲} from what is today Dali \zh{大理}” \citep[367]{mathieu2003}. 

In 794\,AD, the Nanzhao \zh{南诏} kingdom, with its centre on the fertile land around lake 
Erhai (\zh{洱海}, currently
a~Bai"=speaking area), conquered a~broad area including Yongning as well as 
Lijiang \zh{丽江}.
After the fall of Nanzhao in 902, the kingdom of 
Dali (\zh{大理国}, 937--1253), likewise centered
around lake Erhai, exercised control over Yongning and Lijiang, which remained ruled by indigenous chieftains.

At the outset of the Yuan dynasty, a~new feudal chieftain
(\textit{tǔsī} 
\zh{土司}) was installed in Yongning by the Mongolians, who passed
through Yongning on their way to attack the kingdom of Dali. A~chapter of the imperial geography \textit{Yuan \ili{Yi} Tongzhi} \zh{《元一 统志》}{\kern-4pt}, dated 1286, contains a~transcription of the name given to Yongning as \zh{楼头} (present"=day \ili{Mandarin} reading: \textit{lóutóu}). Using the system proposed by \citet{coblin2007}, the name \zh{楼头} reconstructs as *\ipa{ləw dəw}, which is clearly cognate with the present-day name of Yongning: \ili{Naxi} /\ipa{ly˧dy˩}/ and Na /\ipa{ɬi˧di˩}/, discussed in Chapter 1, \sectref{sec:presentationofthenalanguageandnasocietyandreviewofearlierstudies}. It is likely that the authors, who provide a~transcription for the names ‘Lijiang’ (\zh{样渠头}) and ‘Yongning’ (\zh{楼头}), based themselves on the pronunciation used in Lijiang, a~more important centre than Yongning (being more densely populated and more accessible); in this sense, their transcription does not constitute a~direct testimony about the language spoken in Yongning. Still, this constitutes a~reasonable basis on which to hypothesize that there has been linguistic continuity in Yongning since the thirteenth century. 

The Yongning chieftain who surrendered to the Mongolians in 1253 reported a~genealogy of thirty-one generations since his ancestors conquered Yongning. Assuming linguistic continuity, Yongning Na would have been introduced into the area at a~date in the range 500"=650\,AD, counting twenty to twenty-five years between generations. Of course, it may also be that an~earlier form of the language was already spoken in and around Yongning earlier on, and the change in the ruling class c.~500"=650\,AD had no great linguistic impact.

The introduction of \ili{Tibetan} Buddhism dates back to about the same period as the Mongolian conquest, with the missionary
efforts of monks from Muli from 1276 onward \citep[389]{guoetal1994}. In 1356, a~Kagyupa (\textit{bka' brgyud pa})
monastery was established; in 1556, a~large Gelugpa (\textit{dge"=lugs pa}) monastery was established
in Yongning (\ili{Tibetan} name: \textit{dgra} \textit{med} \textit{dgon} \textit{pa}). Earlier cults
remained, with a~division of labour between the Buddhist monks and Na /\ipa{dɑ˧pɤ˧}/ ritual
practitioners; but from that time, Buddhism became a~dominant religion in Yongning. (At the time of
Communist takeover, there were over 700 monks at the Yongning monastery.) This led to
an~increasing cultural distance between Yongning and the Lijiang plain. In Lijiang, no school of
\ili{Tibetan} Buddhism was able to establish and maintain a~central role, as sudden turns followed one
another in the course of an~eventful religious history; the \ili{Naxi} /\ipa{to˧mbɑ˩}/ tradition (with \ili{Tibetan} Bön religion likely as a~major influence) almost acquired the status of an~official cult \citep{jackson1979}.

\begin{photofigure}[t!!]
	\caption{The Yongning monastery. The dialect under study is spoken in Alawa, a~hamlet adjacent to the monastery. Autumn 2006.}
	\includegraphics[width=\textwidth]{figures/MonasteryAndPlain.jpg}
\end{photofigure}

\begin{photofigure}[t]
	\caption{Worshippers at the Yongning monastery. Autumn 2006.}
	\includegraphics[width=\textwidth]{figures/worshippers.jpg}
\end{photofigure}

During the Yuan and Ming dynasties, incessant wars took place between the feudal chieftains of Yongning,
Lijiang and 
Yanyuan \citep[430--431]{guoetal1994}. In 1545, Yongning
united with the neighbouring areas of the five \textit{suǒ} \zh{所} 
(Zuosuo \zh{左所}, Yousuo \zh{右所}, 
Qiansuo \zh{前所}, Housuo \zh{后所}, and 
Zhongsuo \zh{中所}). The Yangtze river constituted
the border between the territories of Yongning and Lijiang.

During the Qing dynasty (1644--1912), the \ili{Naxi} chiefdom of Lijiang came under direct Chinese administration, whereas the
feudal chieftain system was continued in Yongning due to the failure of attempts at military conquest of
the 
Liangshan \zh{凉山} \ili{Yi} area, which constitutes the gateway to Yongning \citep[460]{guoetal1994}. This led to
further cultural differentiation between increasingly Sinicized Lijiang, on the one hand, and on the other hand peripheral areas such as Yongning. The Yongning feudal chieftains actively
resisted Chinese migration into the area, prohibiting rice cultivation and establishing alliances with \ili{Yi} chieftains for military support. This accelerated the pace at which \ili{Yi} families settled in the Yanyuan and
Yanbian areas, eventually replacing the earlier inhabitants~-- speakers of closely related language
varieties, referred to as “\ili{Naxi} dialects” by Guo Dalie \& He Zhiwu (p. 461), which in this volume will be referred to as “\ili{Naish} languages”, as explained further
below. The \ili{Laze}, a~small group of some four hundred people who migrated from Yanbian to their
current location in Muli towards the end of the 19\textsuperscript{th} century, are apparently among the speakers of \ili{Naish} languages who left Yanbian as
it became a~dominantly \ili{Yi} area (about the \ili{Laze} language, see \citealt{huang2009}).

\newpage 
During the 1920s, Yongning became a~link in commercial chains between Tibet, the \ili{Yi} territories centered around Ninglang, Lijiang, and inland China. “This was the first time in
history that the Moso had frequent visitors from the outside on a~considerable scale. ({\dots})~this period was also the onset of the syphilis epidemic” \citep{shihetal2002}, an epidemic curbed in the 1950s.\footnote{This epidemic is alluded to by Goullart, who stayed in Lijiang in the 1940s: “The Nakhi men ({\dots}) knew well enough that most of the Liukhi [/\ipa{ly˧-çi˧}/, the {Naxi} term for the Na of Yongning] tribe was infected with venereal disease”
	%	, and it was only this dread of almost certain infection that made the
	%	Nakhi and other sensible men give a~wide berth to the Liukhi enchantresses” 
	(Chapter 3 of \citealt{goullart1955}).} 

\subsection{The People's Republic of China}
\label{sec:prc}

After the founding of the People's Republic of China in 1949, the central government made plans for the graduated integration of frontier areas along the outer margins of the dynastic power realm: those known during the Qing dynasty as ‘vassal states’ (\textit{fānshǔ} \zh{藩屬}).

\begin{quotation}
	Instead of ‘mobilizing the masses’ like in the Yunnan interior and elsewhere in China, land reform on the volatile frontier took the form of “uniting the feudal to fight feudalism” (\textit{lianhe fengjian fan fengjian} \zh{联合封建反封建}), a scheme designed to win over the ethnic elite. The CCP united front work saw a~revival of traditions in forms of pledging allegiance, conferring official titles, and the ritual of court audiences~-- those who cooperated were invited to serve in the new government and local dignitaries were taken to Beijing to have an audience with the CCP leaders. \citep[43-44]{guoState2008}
\end{quotation}

\begin{sloppypar}
The Liangshan \zh{凉山} \ili{Yi} area, which constitutes the gateway to Yongning, proved a~hard nut to crack. Armed rebellion broke out in 1956. By 1958, the large"=scale ‘pacification’ operations were complete \citep[228-231]{guoState2008}, and administration of the area by the Communist state began. Since then, reforms have been applied essentially top"=down. The policy is that the majority group points the way forward, and leads minority groups towards modernity and eventual {assimilation}. In a~study about the Drung, an~ethnic
group located in an~even less accessible area of Yunnan, \citet{gros2014} argues that these policies bring about a~fundamental change from the earlier relationships of
vassalage between local powers and the state(s). Feudal chieftains paid tributes, and received titles in
return; the balance of this exchange~-- how much tribute was paid, and how much recognition and
autonomy was granted in return~-- was weighed by both parties. By contrast, top"=down state policies
do not partake in a~logic of exchange~-- \textit{don et contre"=don}, as emphasized in the classic
study by Marcel \citet{mauss1990}. Post-1956 events remain a~highly sensitive topic. Mazard's observation about the Nusu (\ili{Yi}) also applies to the Na:
\end{sloppypar}

\begin{quotation}
	They generally
	treat the years from 1958 to around 1979 as a~single historical period circumscribed
	by the collectivisation and de-collectivisation of their land. ({\dots}) Why do many of them ({\dots}) identify this period as a~whole with the
	Cultural Revolution, even though they know and employ the term ‘Great Leap
	Forward’ as well?
	
	One reason may be that the CCP allows overt (though
	limited) criticism of the Cultural Revolution; not so the Great Leap Forward (or the Anti"=Rightist
	Campaign). The Party has never denounced the Great Leap Forward as a~{mistake} ({\dots}). The
	Cultural Revolution has its villains (the Gang of Four); so do the Civil War (the Guomindang) and
	the Second World War (the {Japanese}). Discussion of the Cultural Revolution is not easy, but it is
	possible. When Nusu elders allude to their suffering as occuring under the ‘Cultural Revolution’
	(even if it took place in 1959), they lay claim to a~permitted register of complaints. \citep[172]{mazard2011}
\end{quotation}

Recent history (since the 1980s) will only be addressed indirectly here, through a~discussion of ethnic classification (\sectref{sec:ethnicclass}) and a~review of studies about the impact of tourism on Na society (\sectref{sec:presentdaysociologicalstudiestheimpactoftourismsincethe1990s}).

\section{Ethnic classification: Naxi, Mongolian, Mosuo, or Na}
\label{sec:ethnicclass}

Ethnic categorization as defined by the state “crafted the prism through which the modern Chinese state, and increasingly the people of China and the world at large, have come to view and understand non"=Han Chinese identity” \citep[5]{mullaney2010}. This categorization, which stands on each individual's identity card (\textit{shēnfen zhèng} \zh{身份证}, literally a~‘certificate of identity’), has such a~strong bearing on present"=day identities that it warrants separate discussion.

\begin{quotation}
	{\dots}~in the China of Chiang Kai-shek, the Nationalist regime vociferously argued that the country was home to only one people, “the Chinese people” (\textit{Zhonghua minzu} \zh{中華民族}), and that the supposedly distinct groups of the republic were merely subvarieties of a~common stock. At the same time, a~counterdiscourse emerged among Chinese scholars in the newly formed disciplines of ethnology and linguistics, a~discourse in which China was reimagined as home to many dozens of unique ethnic groups~-- a~newly imported concept also translated using the term \textit{minzu} \zh{民族}. (\citealt[2]{mullaney2010}; see also \citealt{bulag2012})
\end{quotation}

The first census of the People's Republic of China, in 1953-1954, recorded over four hundred different ethnic identities, more than half of which concerned the province of Yunnan, which borders on multi"=ethnic areas on all sides (Vietnam, Laos, Burma, Tibet, Sichuan, Guizhou and Guangxi). Until recently, little information was available on the process of ethnic identification (\textit{mínzú shíbié} \zh{民族识别}) whereby these were subsequently grouped into some fifty officially recognized nationalities (\textit{mínzú} \zh{民族}). Sources that have become available in the past decade reveal how small teams of researchers from diverse social science backgrounds evaluated possibilities for groupings, and gave names to these groupings, against a~tight agenda (less than six months). “In the years following the end of the project, cultural and scientific works rewrote the history of China and its
diversity in an effort to promote a~so-called “historic” and “ancestral” model of the 56 \textit{minzu}
components” \citep{frangville2011}. At first, this did not exclude some fine"=tuning of the ethnic categories: adjustments were made in the 1960s and 1970s, including the recognition of two ethnic minorities that were absent from the 1954 classification. In 1987, however, it was clarified that no additional nationalities would be recognized, and the figure of fifty-six was final.

\begin{quotation}
	{\dots}~the idea of China as a~“unified, multinational country” (\textit{tongyi de duominzu guojia} \zh{统一的多民族国家}) is a~central, load"=bearing concept within a~wide and heterogeneous array of discourses and practices in the contemporary People's Republic. China is a~plural singularity, this orthodoxy maintains, composed of exactly fifty-six ethnonational groups (\textit{minzu} \zh{民族}): the Han ethnic majority, which constitutes over ninety percent of the population, and a~long list of fifty-five minority nationalities who account for the rest. Wherever the {question} of diversity is raised, this same taxonomic orthodoxy is reproduced, forming a~carefully monitored orchestra of remarkable reach and consistency: anthropology museums with the requisite fifty-six displays, “nationalities doll sets” with the requisite fifty-six figurines, book series with the requisite fifty-six “brief histories” of each group, Olympic ceremonies with fifty-six delightfully costumed children, and the list goes on. Fifty-six stars, fifty-six flowers, fifty-six \textit{minzu}, one China. \citep[1]{mullaney2010}
\end{quotation}

The teams who conducted surveys for the national project of ethnic identification in 1953-1954 operated separately in each province, and decisions were also made province by province. The Na living in Yunnan were considered as part of the \ili{Naxi} minority. Those living in Sichuan were classified as Mongolian; this surprising choice was no doubt influenced by the powerful ring which the name ‘Mongolian’ retains in the area since the Mongolian army’s impressive crossing of the Himalayas (passing through Yongning) and victory over the Song dynasty.

\begin{quotation}
	Due to historical tensions between the Na and the \ili{Naxi}, when the Sichuan Na learned that they
	would be classified as \ili{Naxi} in the early 1950s, they protested by taking over the county
	government offices. As the federal government limits recognition to the fifty-six
	ethnicities, local officials were perplexed as to what to do, and a~face-saving compromise
	was established such that the Sichuan Na could be classified as Mongolian, on the basis
	that the Mongols had invaded the area seven hundred years previously, and perhaps the
	Na were descendants of these Mongols. Although this designation is within historical
	memory, the Na in Sichuan have clearly adopted their designation as Mongolian, and
	colorful plastic plaques of Genghis Khan hang prominently on the walls in homes.
	%Sichuan Na also disavow designation as Mosuo, likely because of the associations with
	%the term ‘Mosuo’ developed in the tourist industry. 
	\citep[9]{lidz2010}
\end{quotation}

%Command \noindent added to avoid having an indent. Proofreader suggestion: since this sentence continues the argument, it is better not to indent. 
{\noindent}The claim of Mongolian descent apparently predates the systematic process of “ethnic identification” carried out by the People's Republic of China. 

\begin{quotation}
	Moso chiefs in both Yunnan and Sichuan Provinces claimed that they were descendants of the Mongols. According to Joseph Rock, who personally befriended key members of the Yongning chief's family ({\dots}), the general superintendent (\textit{zongguan} \zh{總管}) of Yongning at that time “was proud of his Mongol origin, for he was a~descendant of one of the Mongol officers left by Kublai Khan in Yung-ning to govern that territory” \citep[359]{rock1947}. Abundant historical records indicate that it was commonplace for the Mongolian conquerors to leave troops of Mongol or non-Mongol ethnic background to govern the newly subjugated territories. ({\dots}) The problem is that in all cases other than the Moso chief's, there is evidence~-- such as records on stelae, tombs and tombstones (some inscribed in Mongolian), records of genealogy, language or vocabulary, and legends in one form or another~-- to substantiate the claim. The Moso aristocrats, however, had nothing to support their claim of Mongol ancestry. \citep[40-41]{shih2010}
\end{quotation}

%Command \noindent added to avoid having an indent. Proofreader suggestion: since this sentence continues the argument, it is better not to indent. 
{\noindent}Shih suggests that the new chieftain was “a Xifan [\ili{Pumi}] officer in the Mongol troops left by Kublai Khan to
rule Yongning” \citep[51]{shih2010}. One piece of evidence that he adduces is the identification of the Yongning chieftain's ethnicity as “Xifan” \zh{西番} in Ming"=dynasty chronicles. The interpretation of the term “Xifan” as referring specifically to the \ili{Pumi} is not self"=evident, however: the label may have been used in a~broad sense that included speakers of the language ancestor to Yongning Na. To this day, the \ili{Namuyi} of Muli, who are speakers of a~{Naic} language (about which see Chapter 1, \sectref{sec:thepositionofnaandnaxiwithinsinotibetan}), are included among the “Xifan” \zh{西番}, a~cover term for various non"=\ili{Tibetan} groups. Another piece of evidence is ethnographic: at the succession of the Yongning chieftain, the \ili{Pumi} would perform a~ritual akin to /\ipa{sɯ˧kʰɯ˩}/, the ritual associated to the giving of a~household member (typically, the giving of a~bride); this is consistent with Shih's hypothesis that their ethnic group was the donor of the Yongning chieftain himself. Shih states his interpretation as follows:

\begin{quotation}
	In the case of \textit{sike} [phonetic transcription: /\ipa{sɯ˧kʰɯ˩}/], a~household member was given to become a~particular person's wife in another household. Because the status of wife necessarily ended with the life of the woman in {question}, \textit{sike} was a~one-time ritual between the two families concerned. In the succession ritual, however, as the historical records suggest, when a~member of the \ili{Pumi} was given to become the chieftain of Yongning, a~territory dominated by the Moso, the status of chieftaincy was perpetual, as was the ritual of interrogation. In both cases, the rituals were performed to dramatize a~reassertion of the unbreakable blood bond between the deceased and her or his natal family. \citep[48]{shih2010}
\end{quotation}

Inclusion in the Mongolian minority proved a~comfortable fiction: it paradoxically granted the Na of Sichuan a~place of their own within the landscape of the recognized ethnic groups of Sichuan. Initially the ‘Mongolian’ label was taken very seriously and attempts were made to teach the Mongolian script to \ili{Naish} speakers of Sichuan, with predictably poor results. The initial ethnic identification was not modified since, because of a~national policy to keep the Pandora’s box of ethnic labels closed. Their fictitious cousins of Mongolia do not appear to have found a~subject for quarrel in the term ‘Mongolian’ being applied to this small group in Sichuan; and the label prevented the Na from being pooled together with closer neighbours, such as the \ili{Naxi}. By contrast, the inclusion of the Na of Yunnan among the \ili{Naxi} made them a~minority within a~minority, restricting the power and number of their representatives at various institutional levels. 

In view of the historical outline summarized in \sectref{sec:historicaloutline}, it is not difficult to understand why the Na would tend to think of themselves as distinct from the \ili{Naxi} despite the conspicuous similarities between their respective languages. Resentment about inclusion in the \ili{Naxi} minority led to a~search for recognition as a~distinct group. From this vantage point, the \isi{endonym} /\ipa{nɑ˩˧}/ is less than ideal: /\ipa{nɑ˩˧}/, presumed to mean ‘black, dark’, is also found in the \isi{endonym} of the \ili{Naxi}: /\ipa{nɑ˩çi˧}/, where /\ipa{çi˧}/ means ‘person, human being’. The quasi"=identity of endonyms might cast doubt on the legitimacy of a~sharp separation. Instead, the Na of Yunnan came to favour the \isi{exonym} ‘Mosuo’ (\textit{mósuō} \zh{摩梭}),\footnote{Alternative transcriptions of this name are shown in \tabref{tab:thenamesofthenaendonymsandexonyms}. As mentioned above, this {exonym}'s origin is unclear. \citet[132]{chavannes1912} cites Chinese chronicles as indicating that the Mo-so tribe was formed during the Nanzhao period out of two distinct elements, the Mo and the So.} a~name formerly used in the Chinese records, which was officially replaced after 1949 by ‘\ili{Naxi}' (\textit{nàxī} \zh{纳西}).\footnote{Thus, the dictionary of pictograms originally published by Li Lin-ts’an \zh{李霖灿}, Chang K’un \zh{张琨} and Ho Ts’ai \zh{和才} as 
	\textit{Dictionary of Mo-So hieroglyphics} (\citeyear{lietal1953}) was reprinted in \citeyear{lietal2001} on the mainland under the title \textit{Dictionary of {Naxi} pictograms}; all occurrences of \textit{Móxiē} \zh{麽些} in the book were replaced by \textit{Nàxī} \zh{纳西}.}

Reviving the demised term ‘Mosuo’ to refer to the Na of the Yongning area is a~felicitous choice to substantiate claims to recognition as a~group separate from the \ili{Naxi}, since the words ‘Mosuo’ and ‘\ili{Naxi}’ are clearly distinct from each other phonetically. Moreover, the term ‘Mosuo’ presents the twofold advantage of being a~term of great antiquity, having been used in the Chinese chronicles since the first millenium \,AD, and of having fallen into disuse in the middle of the 20\textsuperscript{th} century, which lends it a~quaint charm and a~touch of mystery. In 1990, following heated protest against the label ‘\ili{Naxi}’, the ‘Mosuo’ of Yunnan were granted recognition at the provincial level as a~separate subgroup within the \ili{Naxi} minority. This label currently seems set to become the standard in Sichuan as well \citep[10-11]{lidazhu2015}. La Mingying \zh{喇明英}, a~member of the Sichuan Academy of Social Sciences who identifies herself through the official label ‘Mongolian’ but expresses a~preference for the label ‘Na’, reports that self-identification as ‘Mosuo’ is gaining ground among the younger generations on the Sichuan side of the designated ‘Mosuo’ tourist area. 

\begin{quotation}
	Disagreement among the Na about names [ethnonyms] is a~cause for disputes; people sometimes even come to blows. Acknowledging one's ethnic identity, and having a~sense of belonging to the community, constitute the most basic and essential factors in “ethnic identification”. The multiplication of ethnic denominations for the Na of the Lake Lugu area generates great perplexity about their ethnic identity, to the point of causing prejudice to their sense of belonging to a~community and to their ethnic cohesion. \citep[53]{lamingying2015}\footnote{\textit{Original text}: \zh{因称呼的分歧,纳人内部时有争论甚至打架的情况发生。对民族身份的承认和群体归属感是“民族认同”最基本的要素。泸沽湖地区纳人族称的多元化在很大程度上对其民族身份的认同造成很大的困惑,甚至影响了纳人的群体归属感和民族凝聚力。}}
\end{quotation}

Before Communist takeover, there were three hereditary castes among the Na of Yongning: the family of the chieftain, /\ipa{sɯ˧pʰi˧}/, constituted the nobility, as distinct from commoners, /\ipa{dze˧kʰɤ˧˥}/, who were the majority group (about 640 families in the late 1950s); finally, a~smaller group (280 families in the late 1950s) were serfs, /\ipa{wɤ˧}/. Historically, when outsiders joined the community~-- as~war captives, or as~immigrants from areas near and far~--, they would be integrated to the serf caste, which also accommodated commoners stripped of their rank as a~punishment for rebellion \citep{liu1981}. Ethnic identity as assigned by the administration (whether as ‘\ili{Naxi}’ or ‘Mongolian’) lay flat earlier differences between castes; it also went along with the end of the integration of newcomers into Na society. The steady influx of settlers into Yongning, recorded in the successive editions of
the county Annals (\textit{xiànzhì} \zh{县志}; the entire collection
bears the 
name {\kern-3pt}\zh{《中华人民共和国地方志丛书》}{\kern-4pt}, “Collection of local chronicles of the People’s
Republic of China”), results in cohabitation of persons whose official identities remain distinct, some of them “Han”, others “Mosuo”, “\ili{Naxi}”, “\ili{Pumi}”, “\ili{Yi}” and so on, in keeping with the ideology of a~multi"=ethnic and unified China. 



\section[Anthropological research: The fascination of the Na family structure]{Anthropological research: The fascination of the Na's kinship system and family structure}
\label{sec:anthropologicalresearchthefascinationofnafamilystructure}

The rich morphotonology of the Na language, which forms the central topic of this book, apparently went unnoticed until the early 21\textsuperscript{st} century. On the other hand, peculiarities of the Na kinship system and family structure have long been famous well beyond the circles of specialized ethnologists. Here is an excerpt from the highly exoticized account provided by Peter Goullart, a~Russian"=born traveller, explorer and author who lived among the \ili{Naxi} in Lijiang from 1942 to 1949, the last years of the period of intense caravan traffic (1920s to 1940s).

\newenvironment{amquote}{\list{}{\rightmargin0pt\leftmargin7mm} 
\item\relax}
{\endlist}

\begin{amquote}
	The arrival of the members of a~certain matriarchic tribe, living about seven days by caravan
	north of Likiang, always created a~furore in Likiang. Whenever these men and women passed through
	the market or Main Street on their shopping expeditions, there was indignant whispering, giggling
	and squeals of outraged modesty on the part of Likiang women and girls, and salacious remarks from
	men. They were the inhabitants of the Yungning duchies across the Yangtze at the apex of the great
	bend. The Nakhi [\ipa{nɑ˩-çi˧}] called them Liukhi [\ipa{ly˧-çi˧}] and they called themselves Hlihin [\ipa{ɬi˧-hĩ˧}]. The structure of their
	society was entirely matriarchal. The property passed from mother to daughter. Each woman had
	several husbands and the children always cried, ‘We have mama but no papa.’ The mother’s husbands
	were addressed as uncles and a~husband was allowed to stay on only as long as he pleased the
	woman, and if he didn’t, could be thrown out without much ceremony. The Yungning country was a~land of free love, and all efforts of the Liukhi women were concentrated on enticing more lovers
	in addition to their husbands. Whenever a~\ili{Tibetan} caravan or other strangers were passing
	Yungning, these ladies went into a~huddle and secretly decided where each man should stay. The
	lady then commanded her husbands to disappear and not to reappear until called. She and her
	daughters prepared a~feast and danced for the guest. Afterwards the older lady bade him to make a~choice between ripe experience and foolish youth. ({\dots})
	
	With their lips heavily rouged and eyes painted, they walked slowly, or rather undulated, through
	the streets, swaying their hips, smiling and casting an~amorous eye on this man or that. That
	alone was enough to incense the less sophisticated Nakhi women. But when they walked slowly along
	hanging on the neck of a~husband or a~lover, and being held by the waist, this was too much for
	even the brazen Nakhi women, who spat or giggled nervously.~\citep[Chapter~3]{goullart1955}
	%  ({\dots})
	%
	%Only twice was my path crossed by Liukhi women and in both cases it resulted in a~mild scandal.
\end{amquote}


%Command \noindent added to avoid having an indent. Proofreader suggestion: since this sentence continues the argument, it is better not to indent. 
{\noindent}This sample of travellers’ reports about “a land of free love” suffices to explain why the Na exert an~enduring fascination on the general public. The present review does not attempt extensive coverage of the considerable anthropological, ethno"=historical and sociological literature about the Na. Its aim is to convey a~sense of the development of the field, of the historical evolution of approaches and viewpoints, and of the consequences in terms of local people’s perception of social scientists who come to Yongning for fieldwork. 


\subsection[Surveys conducted in the 1960s]{A major source of information: Surveys conducted in the 1960s}
\label{sec:themainsourceofinformationonnafamilystructuresurveysconductedinthe1960s}

The in"=depth research report based on sociological surveys conducted in the 1960s
\citep{bianjizuguojiaminweiminzushehuilishidiaochayunnanshengbianjizu1986} constitutes a~major
resource for the study of Na society. The results of the survey are organized by village, and bring
out subtle differences among villages and among individual households. Most later scholarship
builds on the data reported in the three volumes of this report~-- close to one thousand pages in
total.

The survey clarifies that, until the 1950s, the typical family structure in the Yongning Plain was
matrilinear, with lifelong matrilocal residence. In non"=technical terms, this means that brothers and sisters lived all their lives in their mother’s house, together with their
relatives on the mother’s side: cousins, aunts and uncles, and grandmother and her brothers and
sisters. 

This situation bears some resemblance to that found among the Minangkabau (Indonesia) as described by \citet{hadler2008} and the Nayar (India) as described by \citet{fuller1976}. 

\begin{quotation}
	Men marry into an extended family, but remain attached to their mothers' houses. They return to that house daily to work the fields, convalesce there in times of sickness, and are eventually buried in the maternal family graveyard. A~husband and father is an evanescent figure. In the words of a~Minangkabau aphorism, “The \textit{urang sumando} is like a~horsefly on the tail of a~buffalo, or like ashes on a~burned tree trunk. [When a~little wind blows, it is gone.]” ({\dots}) Minangkabau culture has been termed matrifocal because, although men can be part of the lives of their wives and children, it is mother"=centeredness that grounds the family. \citep[6]{hadler2008}
\end{quotation}

%Command \noindent added to avoid having an indent. Proofreader suggestion: since this sentence continues the argument, it is better not to indent. 
{\noindent}Among the Nayar of India, the husband resides with his sister and visits his wife at night \citep{fuller1976}; in pre"=1956 Na society, lovers met discreetly at the woman’s home. Among the Na, as among the Minangkabau and the Nayar, the answer to the “matrilineal puzzle”~\citep{richards1950}~-- the potential conflict in authority between father and maternal uncle~-- was that authority rests with the uncle. Fathers did not have a~prominent social role; men have commitments to their sisters' children, not to their own, who grow up in another household. “According to tradition, it is the \textit{mamak} (maternal uncle) who provides male authority in the lives of children” \citep[6]{hadler2008}; likewise, among the Na, the male figure of authority was the maternal uncle, as evidenced by proverbs such as (\ref{ex:eagle}).

\begin{exe}
	\ex
	\label{ex:eagle}
	\ipaex{mv̩˧ʁo˥ {\kern2pt}|{\kern2pt} dze˩-hĩ˩-dʑo˥, {\kern2pt}|{\kern2pt} kɤ˩-nɑ˧mi˧; {\kern2pt}|{\kern2pt} di˧qo˧ se˧-dʑo˩, {\kern2pt}|{\kern2pt} ə˧v̩˧˥.}\\
	\gll mv̩˧ʁo˥\$	dze˩\textsubscript{a}	-hĩ˥	-dʑo˥	kɤ˩-nɑ˧mi˧	di˧qo˧	se˥	-dʑo˥	ə˧v̩˧˥\\
	heavens		to\_fly		\textsc{nmlz}	\textsc{top}	eagle	plain	to\_walk		\textsc{top}		maternal\_uncle\\
	\glt ‘As the Eagle is greatest of all that fly in the sky, so the Uncle is greatest of all that walk the earth.'
\end{exe}

%Command \noindent added to avoid having an indent. Proofreader suggestion: since this sentence continues the argument, it is better not to indent. 
{\noindent}A difference is that while the Minangkabau have (flimsy) marriage ties, and the Nayar practise marriage, this institution was marginal among the Na of the Yongning plain before the post-1956 social upheaval. The chieftains, being in contact with their patrilinear Chinese, {Tibetan}, {Naxi} or {Pumi} peers, had wives, at least as a~diplomatic façade. Among commoners, on the other hand, there was no marital exchange between clans or families, and no dowry or brideprice. 

\largerpage[-2] 
The method of data collection used in the 1960s survey bears the stamp of the historical context: a~time when the young
People’s Republic of China took stock of its new Western possessions. Clearly, unconditional
obedience to instructions was expected from the surveyed human subjects. It seems that the objective
was reached, and that, for the sake of the survey, the subjects provided candid,
detailed statements about their family history and sentimental life story. The fact that all the
data was eventually published (and reprinted in 2009), including the real names of the people who
entrusted information on their private lives to the visiting ethnographers, is at variance with
present"=day concerns about the privacy of personal information (as set out in anthropology
handbooks, e.g.~\citealt{fluehrlobban2014}, which contains a~reproduction of the Code of Ethics of the
American Anthropological Association). On the other hand, the social structures described in the
survey have undergone such changes since then that the survey report is simply
irreplaceable.\footnote{A consultant told me in 2008 that during the Cultural Revolution, cereal
	rations in Yongning were made conditional to the possession of a~marriage certificate. Beyond this
	report, indicative of a~perception of a~historical divide, an~anthropologist would want to obtain
	fuller details, verifying this information with other consultants, and investigating how the policies
	were implemented in the various villages and how the local society responded. “The Cultural
	Revolution” is sometimes used as a~cover term because (as mentioned at the end of \sectref{sec:historicaloutline}) the ruling party allows some criticism of the Cultural Revolution, and not of other episodes such as the Great Leap Forward \citep[172]{mazard2011}.} To venture a~comparison with the history of exploratory techniques used in experimental phonetics, the results of the 1960s survey can be likened to the X-ray data collected from the 1930s to the 1970s. This was the window of time between the technical advances that made it possible to carry out cineradiography and the realization that exposure to the high doses of radiation involved carried serious health risks for the person being filmed. This heritage data still constitutes a~precious resource to study the sounds of the world’s languages~\citep{fant1960,leroyetal1974,bothoreletal1986,bouarourouetal2008}.


\subsection[Marxist interpretation of the Na family structure]{Marxist interpretation: Na family structure as a~confirmation of Morgan’s theory}
\label{sec:marxistinterpretationnafamilystructureasaconfirmationofmorganstheory}
\largerpage[-2] 

In the early 1980s, several researchers involved in the survey published books based on these
materials \citep{zhanetal1980,yanetal1984}, before the publication of the original report. These
authors adhered to an~evolutionary perspective, which led them to the conclusion that Na society was
a “living fossil” 
(\textit{huóhuà shí} \zh{活化石}), a~remnant of a~matriarcal society that existed
prior to patriarchy, constituting decisive proof of the reality of Lewis Henry Morgan’s theory
(1877), as embraced by Marx: that family structure evolved from the consanguine family via the
matrilineal clan to the patrilineal nuclear family. 

From the point of view of anthropological theory, Na family structure was taken as confirmation of an~established theory (itself a~“living fossil”: Morgan's theory had been thoroughly discredited in the West for many decades, and only survived in China by the power of dogma), so that, in effect, the fresh data did not make a~significant contribution to
progress in the field. This may be likened to interpretations given of sunspots (black patches on
the surface of the Sun): in the 9\textsuperscript{th} century AD, they were interpreted as planetary transits
obscuring part of the Sun \citep[93]{Wilson1917}; in the 17\textsuperscript{th} century, they were taken as evidence
of the sun’s decay, confirming the pessimistic vision of the world’s gradual decadence, as expressed in the works of Walter Raleigh and Thomas Browne. “Scientific evidence can only answer the questions
that scientists think fit to ask” \citep[21]{Hampson1968}.

To preview a~topic which will be taken up below, the view of Na society as a~“living fossil”,
popularized by advertisements for the Na area as a~tourist destination, created no small amount of
resentment on the part of the Na (this is reported e.g.~by \citealt[132]{shih2010}).


\subsection[Cai Hua’s \textit{A society without fathers or husbands}]{Bringing Na family structure to the attention of Western anthropologists: Cai Hua’s \textit{A society without fathers or husbands}}
\label{sec:bringingnafamilystructuretotheattentionofwesternanthropologistscaihuas19972001asocietywithoutfathersorhusbands}
\nocite{cai1997,cai2001}

While Na society got straightforwardly pigeonholed into one of the evolutionary stages postulated by
Marxist"=Leninist anthropology (as the earliest stage: matriarchy), it did not conform to postwar
Western models of the anthropology of kinship. Browsing through the first pages of Murdock’s classic
study of family structure (\citeyear{murdock1949}: 1--3), it is clear that the Na family does not
fit within the typology. Murdock’s typology considers the “first and most basic” type of family
organization to be the nuclear family (“a married man and woman with their offspring”), of which the
other two types of families acknowledged in his typology constitute “combinations”: the polygamous
family “consists of two or more nuclear families affiliated by plural marriages”, and the extended
family “consists of two or more nuclear families affiliated through an~extension of the parent"=child
relationship rather than of the husband"=wife relationship, i.e., by joining the nuclear family of
a~married adult to that of his parents” (p. 2). One can imagine the excitement with which
a~researcher working in Western anthropological circles would pursue the theoretical implications of
the observations made in Yongning, which contradict two of Murdock’s assumptions: the universality of marriage, and the universality of the nuclear family.

Such was the perspective adopted by 
Cai Hua \zh{蔡华}, a~Yunnan"=born anthropologist who wrote
a~Ph.D.\ at the \textit{École des hautes études en sciences sociales} in Paris with Kristofer Schipper,
Françoise Héritier and Olivier Herrenschmidt as advisors. Cai Hua’s \textit{Une société sans père ni
	mari: les Na de Chine} (A society without fathers or husbands: the Na of China) \citep{cai1997}
was the first book to present a~study of Na family structure to a~non"=Chinese"=reading audience. The
title announces the author’s vantage point: presenting the Na as a~{counterexample} to generalizations
that previously seemed firmly established. Cai Hua’s book aims to draw the attention of the international
community of anthropologists to a~social structure that calls into {question} tenets of the
anthropology of kinship, such as the presumed universality of marriage. Its diffusion was
facilitated by an~{English} translation \citep{cai2001}. As a~sample of the enthusiastic response to the dizzying blend of theoretical
challenges and juicy stories contained in the book, an~article in the \textit{New York Review of
	Books} \citep{geertz2001} points out challenges to Lévi"=Strauss’s views on kinship
\citep{levistrauss1949} and also provides a~racy summary of the titillating part of the story:

\largerpage[-2]
\begin{quotation}
	Sexual intercourse takes place between casual, opportunistic lovers, who develop no broader, more
	enduring relations to one another. The man “visits,” usually furtively, the woman at her home in
	the middle of the night as impulse and opportunity appear, which they do with great
	regularity. Almost everyone of either sex has multiple partners, serially or simultaneously;
	simultaneously usually two or three, serially as many as a \hfill hundred
	\newpage\noindent
	or two. There are no nuclear
	families, no in"=laws, no stepchildren.~\citep{geertz2001}\footnote{The same two aspects~--
		scientific significance and sexual fascination~-- recur in reviews of the book. Here is another
		example: “le propos a~de quoi mettre sens dessus dessous la théorie anthropologique qui fait
		reposer le principe même des sociétés humaines sur l'alliance de mariage. Mais la lecture de ce
		livre à la fois savant et ingénu est aussi recommandée à ceux que ce problème laisse froids~: le
		tableau des mœurs libertines des Na est digne des plus joyeux fantasmes qui circulaient en
		Europe dans les années 70”~\citep{journet1998}.}
\end{quotation}

\largerpage[-2]
From the vantage point of a~Western audience, an~additional bonus is that the study’s author is
Chinese. The workings of a~society “without fathers or husbands” are all the more fascinating as
they are narrated by an~anthropologist whose background is a~patriarchal and marriage"=centered
culture that takes a~disparaging view of Na culture as “backward”. In examining Na society, Cai
Hua is careful to distance himself from former colleagues such as Yan Ruxian, denouncing the
evolutionary bias in their writings. This earns him the somewhat patronizing praise of
colleagues who emphasize the author’s scientific achievement: freeing himself from two
preconceptions that could have biased his research, namely Marxist ideology and Chinese
prejudice against forms of kinship and sexuality that are remote from Chinese
culture (e.g.\ \citealt[57--58]{cartieretal1998}).\footnote{“Narrating the transformations imposed on Na society represents a~significant achievement in itself on the part of a~Han ethnologist, especially if he proves capable of standing back from Marxist ideology and from specifically Chinese prejudice about patterns of kinship and sexuality that are very distant from his own culture. Cai Hua, a young Yunnanese researcher who came over to Paris to complement his training, therefore deserves to be congratulated for achieving this difficult task, literally applying a~\textit{tabula rasa} approach to cast aside most of the prejudices that could have hampered his research.” \textit{Original text:} Relater ({\dots}) les transformations
	imposées à la société Na représentait en soi une contribution fort honorable de la part d’un
	ethnologue Han, surtout s’il se montrait capable de prendre ses distances par rapport à
	l’idéologie marxiste et aux préjugés proprement chinois relatifs à des formes de parenté et de
	sexualité très éloignées de sa culture. On saura donc tout particulièrement gré à Cai Hua, un
	jeune chercheur yunnanais venu à Paris compléter sa formation ({\dots}), d’avoir réussi ce difficile
	exercice en faisant littéralement table rase de la plupart des préjugés qui auraient pu
	handicaper sa recherche.} The book gained international fame, and received a~response from Claude Lévi"=Strauss~\citep{levistrauss2004}.

The less positive side of the success story is that, in his desire to emphasize the scoop~-- that Na
society presents radical challenges to the anthropology of kinship~--, Cai Hua stretches the
evidence. A~reader who had access to the Chinese literature examined Cai’s
argument in detail and concluded that, “in setting out to make certain points, Cai picks his
unreferred cases rather selectively and ignores the cases which do not fit his argument”
\citep{wellens2003}. The book is selective in its presentation of the data, in order to
bring out forcefully the uniqueness of this society, represented as “the ‘other’ of the Han Chinese:
a~society free of the constrictions of Confucian morality” \citep[147]{wellens2003}. 

Cai Hua became
a~professor in anthropology at Peking University, where he continued to focus on the anthropology of
kinship. But he did not publish a~Chinese version of his book. One possible reason for this choice is that he was
aware that the book, tailored for a~Western audience, would encounter a~more critical reception on
the part of scholars who have access to the ethnographic reports, and now to linguistic data as well. To call Yongning Na society “a society without fathers or husbands” is
to stretch the point. The notion of ‘father’ is by no means absent from the language: the
word /\ipa{ə˧dɑ˥\$}/\footnote{The combination \ipa{˥\$} in this word's transcription refers to one of the lexical tone categories of Yongning Na: see Chapter 2, \sectref{sec:wordfinalandmorphologicalnucleusfinalHtones}.} unambiguously means ‘father’. If one wishes to formulate the key observations
by contrast with Chinese marriage customs, a~more adequate description 
% proposed by He
% Xueguang \zh{和学光} (p.c.\ 2008)  %% in fact: is used in earlier publications, such as the 1966 film about Yongning society.
is \textit{bù} \textit{qǔ} \textit{bú} \textit{jià} 
\zh{不娶不嫁}: men do not
\textit{take} a~wife into their family (\textit{qǔ} \zh{娶}), and women do not \textit{leave} their
family to join their partner’s (\textit{jià} \zh{嫁}).

In his later foreign"=languages publications, Cai Hua continued to lend Na society a~prominent
position in the typology of kinship structures. In a~book published in {French} \citep{cai2008}, the author discusses four family structures: Chinese; Na; {French}; and Samo
(Burkina Faso), which was studied by Cai’s Ph.D.\ advisor Françoise Héritier. These four societies are neatly
arranged into a~system of binary oppositions. The first two are described as monolateral, and the latter two as
bilateral; the Chinese family as masculine, and Na as feminine. {French} is bilateral"=symmetrical, and
Samo bilateral"=asymmetrical. The publisher’s blurb emphasizes that the author advances “new
epistemological proposals which call into {question} a~certain Western rationalism and would also be
useful, it seems, to other human and social sciences”.\footnote{\textit{Original text}: {\dots}~de nouvelles propositions épistémologiques
	qui remettent en {question} un certain rationalisme occidental et seraient utiles, selon toute
	apparence, aux autres sciences humaines et sociales.} The different orientations of the author’s
anthropological publications in Chinese and in {French} provide an~illuminating example of the
enduring divide between ‘Western’ and ‘Chinese’ scholarship, the former apparently encouraging
epistemological boldness~-- sometimes at the expense of breadth and depth of typological surveys,
and of precision in detail.


\subsection[Studies of Na society in comparative perspective]{Beyond the initial scoop: Studies of Na society in comparative perspective}
\label{sec:shih19932010andweng1993}

The initial scoop~-- encountering a~society with uncommon family structure~-- opens into a~wealth of issues for anthropologists to explore. Two important Ph.D.\ dissertations about the Na were completed in 1993: those of Shih Chuan"=kang (\citeyear{shih1993}) and Weng Naiqun (\citeyear{weng1993}). These were followed in \citeyear{chao1995} by Emily Chao’s, which has a~stronger focus on the \ili{Naxi}. A~Chinese translation of Shih’s dissertation was published in \citeyear{shih2008}, and
an~enlarged {English} edition, with additional fieldwork results, in \citeyear{shih2010}. These
studies provide an~in"=depth analysis of Na society, on the basis of new fieldwork data.

To venture a~critical note about Shih Chuan"=kang's publications, his conclusions on linguistic
issues are sometimes hasty, as when accepting the {folk etymology} of the place name ‘Yongning’ (discussed in Chapter 1, \sectref{sec:presentationofthenalanguageandnasocietyandreviewofearlierstudies}). One may likewise entertain
reasonable doubt about Shih Chuan"=kang’s interpretation of the \isi{exonym} ‘Mosuo’ found in
Chinese chronicles: 

%\Hack{\newpage}
\begin{quotation}
	In the summer of 2001, I made another field trip to Yongning under the auspices of the National Science Foundation. While being jolted around in a~Mitsubishi SUV on the way from Lijiang to Yongning, I was ruminating yet again over the candidate words for which the term \textit{Mosuo} and its variants might have been transliterated. When I was mulling over the phrase \textit{mosi}, the legend about the {English} word \textit{kangaroo} suddenly occurred to me.
	
	In the 1770s, the story goes, when Captain Cook and his explorers in Australia saw a~large quadruped hopping animal they had never seen in Europe, they asked: “What is the name of this animal?” “Kangaroo,” the aborigines replied. The British assumed this must be the name of the animal and introduced the word into the {English} vocabulary as such. It turned out, according to the legend, that the word was not the name of the animal. Rather it meant “I don't understand.”
	
	Inspired by this legend, I wondered how I could have missed the point for so long. In both the {Naxi} and Naru [i.e.\ Na] languages, \textit{mosi} means “not know,” which can be used as an independent phrase to answer a~{question}. The pronunciation of this phrase is identical in both languages. I had asked this phrase in the field countless times but never thought it was the answer to my long"=standing {question}.
	
	Neither the historical accuracy of the kangaroo legend nor the exact meaning of the word \textit{kangaroo} in the aboriginal language bear any direct relevance to the origin of the word in {question}. Rather, the significance of this legend is that it vividly depicts a~conceivable scenario in which cultural and linguistic misunderstandings could arise during the initial contact of different cultures. It is not difficult to envision another such scenario: One of the first Chinese speakers to get in touch with the group under consideration asked: “Who are you?” Responding to a~language that he did not understand, the person said: “\textit{Mosi},” meaning “I don't know (what you are talking about).” The Chinese speaker just took it as an answer to his {question} and recorded or repeated this “name of the people” in the closest sounds in his own language.~\citep[25--26]{shih2010}
\end{quotation}

%Command \noindent added to avoid having an indent. Proofreader suggestion: since this sentence continues the argument, it is better not to indent. 
{\noindent}“Kangaroo legend” is an apt label. ‘Kangaroo' does not mean ‘I don't know': it is the name of a~species of kangaroo in Guugu Yimidhirr, a~language of the Pama-Nyungan family. To relate the earliest Chinese names for the Na to /\ipa{mɤ˧-sɯ˥}/ ‘[I] don’t know’ is to build another legend in blissful ignorance of linguistic methods. Shih compares present"=day {Southwestern Mandarin} pronunciations
with present"=day Na \citep[26--27]{shih2010}, but etymological research at such historical depth
should be based on \is{comparative method (historical linguistics)}reconstructed forms. The earliest Chinese term, \zh{摩沙}, goes back to the Jin dynasty
(265--420\,AD). Reconstructions of Old Chinese suggest that the realization of \zh{沙} may have been close to
*\ipa{sræ} or *\ipa{sræj} \citep{baxter2000}. This does not match up well with reconstructions
proposed to date for the proto"={Naish} stage: ‘to know’ is \is{comparative method (historical linguistics)}reconstructed as *\ipa{si} \citep{jacquesetal2011}.

To the linguist, these slight shortcomings in linguistic aspects of an~anthropologist’s publications
serve as a~word to the wise: great care should be exercised to avoid oversimplifications in areas
other than one’s own. In the same way as a~lack of precision in linguistic analyses on the part of
anthropologists casts the shadow of a~doubt on their anthropological conclusions, linguists run
a~risk of missing linguistic insights by taking a~simplistic view of social phenomena, and paying
insufficient attention to the social nature of language.

\largerpage[-2]%longdistance
On historical topics, a~salient aspect of Shih’s study is the author’s relentless insistence that
the \ili{Naxi} and Na are distinct peoples. He proposes a~distinct ancestry for the two groups, tracing
the one back to the tribes referred to as \textit{Máoniúzhǒng} 
\zh{牦牛种} in Han"=dynasty \il{Sinitic}Chinese chronicles, and the other to the 
\textit{Rǎnmáng}
\zh{冉駹} of \il{Sinitic}Chinese chronicles. Identifications between present"=day ethnic minorities and names
given to “barbarian” tribes in early Chinese writings are the subject of sustained debate in Chinese
scholarship. These identifications are highly speculative, however \citep{gros2014b}. Shih’s statement that
“patrilineal descent has been the norm for thousands of years” among the {Naxi}’s forebears is not
supported by convincing evidence. One may have an~impression that the author, who expresses great
sympathy for Na society throughout his study (witness the 2010 title \textit{Quest} \textit{for}
\textit{Harmony}), adopts his consultants’ viewpoint that they are clearly distinct from the {Naxi},
reifies this perceived difference as a~binary opposition between Na and {Naxi} as ethnic categories,
and projects this opposition into the indefinite past.

\begin{quotation}
	Shih is so preoccupied with establishing the exceptionality of the Moso that he
	has rather too hastily dismissed the comparative potential of similar practices found in some
	regions bordering Tibet. As a~matter of fact, one of the merits of the book is that it presents a
	clear ethnography on the basis of which regional comparisons could be drawn, while fuelling
	the debate within anthropology of kinship in general. \citep{gros2011}\footnote{\textit{Original text:} préoccupé d’établir l’exceptionnalité du cas Moso, l’auteur écarte un peu trop rapidement l’intérêt comparatif de cas assez similaires relevés dans certaines régions voisines de la bordure sino-tibétaine. C’est pourtant un des mérites de son ouvrage que de nous livrer une ethnographie claire à même de servir à une entreprise comparative régionale, comme d’alimenter le débat au sein de l’anthropologie de la parenté plus généralement.}
\end{quotation}

\largerpage[-2]
An unfortunate consequence of Shih Chuan"=kang’s entrenched belief in the great historical depth of
the Na"=vs."={Naxi} divide is that it leads him to dismiss the studies of researchers who
hypothesize that the Na and {Naxi} share a~common ancestry, and who put forward a~historical synopsis
of their gradual divergence. An extreme version of this hypothesis is explored by
\citet[33–46]{jackson1979}, who points out “strong resemblances with regard to their kinship
patterns in particular”, and suggests that the main differences between {Naxi} and Na societies only
have a~time depth of about three centuries: in his view, they mostly result from the in"=depth Sinicization
of {Naxi} culture since the 18\textsuperscript{th} century. Shih
vigorously rejects Jackson’s theses, and the \citeyear{shih2010} edition of his book does not mention Christine Mathieu’s (\citeyear{mathieu2003}) study, \textit{A history and anthropological study of the ancient kingdoms of the Sino"=Tibetan borderland~-- {Naxi} and Mosuo},
which explicitly sets out to explore the historical relationship between the Na and {Naxi}.

Admittedly, Jackson’s study calls for a~thorough revision in light of more recent
documentation. It must be remembered that much larger amounts of material are now available than at
the time of Jackson’s study; this goes a~long way towards explaining occasional mistakes, such as
the interpretation of the \ili{Naxi} name of the Na, /\ipa{ly˧-çi˧}/ (romanized as \textit{Lü"=khi}), as
“the people of Lü, the Chinese name for the area” (p. 36), when it actually means ‘the people of the
Centre’, and is an~exact parallel (cognate) to the Na \isi{endonym} /\ipa{ɬi˧-hĩ˧}/.\footnote{For the sake of simplicity, this noun is provided here in surface phonological transcription. Its underlying form is //\ipa{ɬi˧-hĩ\#˥}//, with a~{floating} High tone. This tonal category is analyzed in Chapter 2, \sectref{sec:afloatinghtonewithcomparativeevidencepointingtoitsorigin}.} Also, some
formulations are deliberately provocative: Jackson likes to sketch {Naxi} history in broad strokes,
emphasizing decisive junctures such as the year “1723 A.D. when the Mu family was ignominiously
dismissed and the area was ‘nationalized’ by the Chinese” (p. 35). During that year, Lijiang was
placed under direct Chinese rule and the Mu \zh{木} family of feudal chieftains who had ruled the area since
the Yuan dynasty ceased to exercise real control. This is undoubtedly a~major landmark in {Naxi}
history. However, one may want to emphasize that the deliberate introduction of Chinese culture and
Confucian ideology had begun much earlier: the Mu feudal chieftains’ unswerving allegiance to China dates
back to the beginning of their rule, in the 14\textsuperscript{th} century. Viewed in this light, the integration of
Lijiang into Chinese territory in the 18\textsuperscript{th} century is not without links to decisions that were
made by the ruling family several centuries earlier.

\begin{quotation}
	The Mu paid tribute to the imperial court and guarded the frontier on behalf of the Chinese emperors. To develop their realm, they pacified, conscripted, and taxed the local tribes (against fierce resistance), and they also called on large numbers of Chinese migrants from the interior~– peasants, artists, craftsmen, literati, Taoist and Chinese Buddhist adepts~– who worked on their estates, joined their armies, populated garrisoned villages and towns in tribal territories, and assimilated into the {Naxi} population. The Mu kings prided themselves on their civilization, in other words: their Sinicization. They were soldiers, and they became scholars, poets and calligraphists.  They built palaces in Chinese style; they also built Confucian, Taoist and Buddhist temples, and dedicated arches to the chastity of their wives in Confucian fashion. \citep[359-360]{mathieu2015}
\end{quotation}

\largerpage[-2]
A well-documented typological parallel for the appearance of cultural differences due to Sinicization is the case of Vietnam: in"=depth sinicization in the course of the first millenium \,AD resulted in differences of mentality between the Vietnamese (speakers of an Austroasiatic language deeply influenced by Chinese) and their Austroasiatic neighbours.\footnote{Haudricourt draws a~parallel with the Germanization of the Czechs, a~Western Slav group; readers unfamiliar with Haudricourt's style should be warned that allowance must be made for Haudricourt's taste for thought-provoking shortcuts. “The Vietnamese are what they are because at bottom they are culturally Chinese. This is exactly like Czechs: they speak a~Slavic language, but their civilization is German. Literary German is the Prague variety of German as it was used by the imperial administration of the House of Luxembourg. This explains why there have always been insoluble national problems between Czechs and Slovaks. The Vietnamese have roughly the same history: they have assimilated enough Chinese culture to become unclassifiable in the eyes of their neighbours” \citep[97-98]{haudricourtetal1987}. \textit{Original text:} Les Vietnamiens sont ce qu'ils sont parce qu'en fait ils sont chinois. C'est exactement comme les Tchèques, Marcel Mauss m'avait fait remarquer avant la guerre que les Tchèques parlent une langue slave mais qu'ils ont une civilisation allemande. L'allemand littéraire c'était l'allemand de Prague utilisé par l'administration impériale de la dynastie des Luxembourg. Ce qui explique qu'il y ait toujours eu des problèmes nationaux insolubles entre les Tchèques et les Slovaques. Les Vietnamiens ont à peu près la même histoire, ils ont assimilé assez de civilisation chinoise pour se rendre inclassables aux yeux de leurs voisins.} 
Again adopting the linguist’s (admittedly narrow) perspective, there appears to be evidence supporting the view of a~gradual divergence between the Na and Naxi. Careful examination of kinship terms in \ili{Naxi} suggests that words for relatives on the
father’s side are mostly borrowings or recent coinages, as they are not cognate across dialects. The same holds true of terms relating to
marriage, such as ‘husband’, ‘wife’, and ‘daughter"=in"=law’. By contrast, the terms for relatives on
the mother’s side are of greater antiquity, with cognates in Na and {Laze}. Viewed in this light, the hypothesis of a~divergence in terms of family structure as the {Naxi} underwent growing Chinese (Confucian) influence
should not be lightly dismissed. Jackson’s {phrasing} (p. 37) is: “This is the missing key to the
confusion on Nakhi kinship: legal patrilinearity yet traditional matrilinearity”.


\largerpage[-2]
Last but not least in this review of studies of the Na's kinship system and family structure, a~two"=volume set of collected works in Chinese needs to be mentioned: \citet{latami2006}. This collection is organized by themes (volume I: ethnology and anthropology; volume II: language, customs, religion, culture, music, and book reviews). It covers the period from 1960 to 2005. It has some minor limitations, such as incomplete information about the original publication references of work reprinted in the collection, and typographical issues for Latin characters and transcriptions in the International Phonetic Alphabet.


\subsection[Present"=day sociological studies]{Present"=day sociological studies: The impact of tourism since the 1990s}
\label{sec:presentdaysociologicalstudiestheimpactoftourismsincethe1990s}

Since the 1990s, tourism has developed at a~staggering pace in the Yongning Na area. A~number of
books, both in Chinese and in Western languages, cater for the tourist industry by presenting
idealized pictures of Na society against its beautiful background: Lake Lugu and the Yongning
plain~\citep[e.g.][]{refflet2006,lamu1998}. There also exists abundant scholarly literature on the effects of tourism on Na society. An especially striking contrast could be observed in the
2010s between rituals and songs as practised by villagers among themselves in the village of
Lijiazui (\zh{利家嘴}), far from the tourist area, and the performances staged for tourists on the shores of Lake Lugu \citep{milan2013}. Anthropologists report in"=depth effects of tourism such as decreasing reliance on matrilineal kin as wealth increases.

\begin{quotation}
	{\dots}~increased individual access to resources is associated with	diminished importance of the kinship group in organising behaviour, a~shift away	from matrilineal inheritance and {erosion} of the non-conjugal visiting system. ({\dots}) Respondents in tourist-impacted areas showed more deviation from matrilineal ideology in terms of household composition and preference for marriage, trends that seem to be associated more with wealth than with cultural {assimilation}. \citep[171]{mattison2010}
\end{quotation}

Foreign sociologists and anthropologists often take a~critical stance,
pointing out that “official representations of China’s ethnic minorities have created an~image of
minority people as dangerous, feminine, and erotic”, and that, in the case of the
Na/{\allowbreak}Mosuo, “early state categorizations of Mosuo gender practices have led to
representations of Mosuo ethnicity built around notions of women freely available for sex, to whom
present lovers have no future commitments, or of a~land where women rule. Matriarchy and sexual
availability are central in tourists’ desire to visit the Mosuo” (\citealt[449–450]{walsh2005}; see also \citealt{schein1997, blumenfield2010}). These authors bring to light the ironic reversal whereby “the cultural characteristics the Maoist
government tried to change became celebrated as markers of Mosuo cultural uniqueness and value”
\citep[457]{walsh2005}.

The detailed analyses proposed by Stéphane \citet{gros2001} in his study of the
Drung (Dulong) ethnic group also apply to other groups in Yunnan and in China at large: images of
“minority” identities are constructed to suit the country’s projects. Assimilationist policies,
which culminated during the Great Proletarian Cultural Revolution (1966--1976), translate into dual
visions of pre"=Liberation and post"=Liberation societies. Relative toleration during the 1980s,
China’s “Reforms and opening up” decade \citep{zhu2014}, led to the recognition of acceptable cultural
features that need not be eliminated along with “bad” inheritance from the past, e.g.\ granting
cultural value to “religion” (\textit{zōngjiào} \zh{宗教}) as (precariously) distinguished from “superstition” (\textit{míxìn} \zh{迷信}).
After 1989, the {conservative} backlash was accompanied by folklorization of ethnic minorities:
providing timeless and monolithic representations of the officially defined ethnic minorities,
marketed to cater for the tourist industry and contribute to the country’s GDP. A~constant is that
images of the “minorities” serve as a~means to assert by contrast the homogeneity and modernity of
the Han “majority” \citep[31]{gros2001}. Another constant is the pressure towards {assimilation}.

\begin{quotation}
	Today, on the mere surface, “leisure culture” represents market reasoning rather than a~statist logic. This is, after all, what hegemony is all about: naturalization of ruling technologies. \citep[242]{sigley2013} 
\end{quotation}

For obvious reasons, publications by scholars with institutional or family ties to mainland China tend to refrain from such criticisms. An extensive literature focuses on proposals for striking a~reasonable balance between the competing demands of
economic development and “cultural preservation” 
(\textit{wénhuà bǎohù} \zh{文化保护}). This perspective is in line with the national mottos of economic development, on the one
hand, and preservation of social harmony, on the other. Overviews are provided by \citet{knodel1995} and \citet{he2008}. Here is an~{English}"=language sample.

\begin{quotation}
	With its unique natural landscape of a~plateau
	lake and matriarchal culture, Lugu Lake region has
	recently become an attractive destination for tourists
	and researchers. Although the present environmental
	conditions in Lugu Lake region are good,
	rapid economic and tourism development in
	recent years has impacted on the regional environment
	and the traditional Mosuo culture. ({\dots}) Unlike the attitude
	of other races to nature, the Mosuo’s attitudes
	towards nature greatly benefit environmental
	protection. \citep[49-51]{yanetal2008}
\end{quotation}

%Command \noindent added to avoid having an indent. Proofreader suggestion: since this sentence continues the argument, it is better not to indent. 
{\noindent}The article leads up to a~list of recommendations, the first being that “Government should play a~more positive role in conservation of traditional Mosuo culture and the local environment through increased investment” \citep[54]{yanetal2008}.

On a~much less predictable note, the \citeyear{lamu2007} collection of articles by Mosuo 
scholar and poet Lamu Gatusa \zh{拉木·嘎吐萨} preserves in (novelized) writing some interesting pieces of local history, and publications by Mosuo anthropologist Latami Dashi \zh{拉他咪·达石} span a~range of topics and locations, including the Mosuo villages of the Ninglang plain \citep{latami2009, latami2016}. 
