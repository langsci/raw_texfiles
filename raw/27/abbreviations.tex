\addchap{Liste des abréviations utilisées dans les gloses des exemples}
 
\begin{multicols}{2}
\begin{tabbing}
distr \hspace{1em} \= marquage différentiel de l’objet direct\kill
1 \> première personne\\
2 \> deuxième personne\\
3 \> troisième personne\\
\textsc{acc} \> accusatif\\
\textsc{adj} \> adjectif\\
\textsc{adv} \> adverbe\\
\textsc{art} \> article\\
\textsc{aux} \> auxiliaire\\
\textsc{comp} \> complémenteur\\
\textsc{cond} \> conditionnel\\
\textsc{conj} \> conjonction\\
\textsc{dat} \> datif\\
\textsc{decl} \> déclarative\\
\textsc{def} \> défini\\
\textsc{dem} \> démonstratif\\
\textsc{det} \> déterminant\\
\textsc{distr} \> distributif\\
\textsc{dom} \> marquage différentiel de \\
	\> l’objet direct\\
\textsc{dur} \> aspect duratif\\
\textsc{excl} \> exclamative\\
\textsc{f} \> féminin\\
\textsc{foc} \> focus\\
\textsc{fut} \> futur\\
\textsc{gen} \> génitif\\
\textsc{imp} \> impératif/impérative\\
\textsc{ind} \> indicatif\\
\textsc{indf} \> indéfini\\
\textsc{inf} \> infinitif\\
\textsc{inter} \> interrogative\\
\textsc{intr} \> intransitif\\
\textsc{ipfv} \> imparfait\\
\textsc{m} \> masculin\\
\textsc{n} \> neutre\\
\textsc{neg} \> négation, négatif\\
\textsc{nom} \> nominatif\\
\textsc{obj} \> objet\\
\textsc{obl} \> oblique\\
\textsc{pass} \> passif\\
\textsc{perf} \> aspect perfectif\\
\textsc{pl} \> pluriel\\
\textsc{pred} \> prédicatif\\
\textsc{prf} \> parfait\\
\textsc{prs} \> présent\\
\textsc{pst} \> passé\\
\textsc{ptcp} \> participe\\
\textsc{q} \> marqueur interrogatif\\
\textsc{refl} \> réfléchi\\
\textsc{rel} \> relatif\\
\textsc{sbj} \> sujet\\
\textsc{sbjv} \> subjonctif\\
\textsc{sg} \> singulier\\
\textsc{sup} \> supin\\
\textsc{top} \> topique\\
\textsc{tr} \> transitif\\
\textsc{voc} \> vocatif\\ 
\end{tabbing}
\end{multicols}