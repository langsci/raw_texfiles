\addchap{Remerciements}

Plusieurs personnes ont contribué, de près ou de loin, à l’aboutissement de cet ouvrage et je tiens à les en remercier chaleureusement. 

D’abord, je voudrais remercier \textit{Language Science Press} d’avoir accepté la publication de ce livre. Je tiens à remercier particulièrement Stefan Müller et Berthold Chrysmann pour l’intérêt qu’ils ont porté à mon travail et la patience avec laquelle ils ont attendu la version finale du manuscrit. Je remercie aussi les deux évaluateurs anonymes du manuscrit ; leurs commentaires très fins et leurs suggestions ont nettement amélioré la qualité de cet ouvrage. Un grand merci à Sebastian Nordhoff, qui m’a accompagnée tout au long du processus de mise en forme, afin de remplir les critères techniques de publication de \textit{Language Science Press}.

Cet ouvrage reprend une bonne partie de ma thèse de doctorat, soutenue à l’Université Paris Diderot – Paris 7 en 2011. Je voudrais exprimer toute ma reconnaissance à ma directrice de thèse, Anne Abeillé, qui a dirigé mes recherches depuis le master et m’a permis de réaliser mes travaux de recherche dans les meilleures conditions possibles. Ses intuitions très fines, sa souplesse et son ouverture d’esprit ont contribué de manière significative à mon parcours scientifique. 

J’ai eu la chance de me former au milieu d’une équipe exceptionnelle de chercheurs et d’enseignants-chercheurs du Laboratoire de Linguistique Formelle et de l’Université Paris Diderot – Paris 7. Comme doctorante et jeune chercheur, j’ai énormément profité de leurs savoirs et expérience. Plusieurs ont eu un apport essentiel et je tiens à les remercier en particulier. Je suis très reconnaissante à François Mouret, qui m’a aidée à comprendre plusieurs aspects liés à la coordination et à l’ellipse. Son suivi, le sérieux et la précision de ses commentaires m’ont toujours aidée à avancer dans mon travail. Je remercie aussi Olivier Bonami pour son aide précieuse sur la partie formelle de cet ouvrage ; ses commentaires m’ont aidée à résoudre certains problèmes relevant de la méthodologie, la description ou encore la formalisation de faits linguistiques. J’ai eu la chance de côtoyer Jean-Marie Marandin et Danièle Godard ; les discussions avec eux ont répondu à mes questions délicates et m’ont ouvert de nouveaux horizons scientifiques. Je remercie aussi Philip Miller d’avoir lu certaines sections de cet ouvrage ; ses suggestions très pointues m’ont aidée dans la rédaction du chapitre général dédié aux phrases elliptiques.  

Je remercie les personnes qui ont accepté d’être membres de mon jury de thèse. Je suis très reconnaissante à Carmen Dobrovie-Sorin pour toute son aide apportée à divers titres depuis mon arrivée à l’Université Paris 7 comme étudiante Erasmus. Je remercie Jonathan Ginzburg d’avoir soulevé des questions profondes liées à l’analyse formelle de l’ellipse. Je tiens à exprimer ma reconnaissance à Emil Ionescu, un des premiers professeurs m’ayant initiée à la linguistique à l’Université de Bucarest. Ma gratitude s’adresse aussi à Jason Merchant, qui, par l’intermédiaire de ses présentations et articles, m’a aidée à approfondir le champ d’étude de l’ellipse ; j’apprécie beaucoup son honnêteté intellectuelle et son ouverture d’esprit. Enfin, je remercie Marleen Van Peteghem d’avoir accepté de lire ma thèse malgré des conditions difficiles. 

Un grand merci à mes chers collègues et amis Frédéric Laurens et Grégoire Winterstein, qui ont eu un apport non négligeable aux résultats de cette thèse, par leurs collaborations et leurs discussions stimulantes. Merci à Margot Colinet pour sa disponibilité même à des heures indues, ma référence en matière de français. Merci à mes collègues et amis à Paris 7, et notamment à Israel de la Fuente, Anna Gazdik, Fabiola Henri, Jana Strnadova et Delphine Tribout. Merci à Clément Plancq, qui m’a soutenue dans mes problèmes informatiques.

Je clos enfin ces remerciements en dédiant cet ouvrage à ma famille et aux quelques amis très proches (sans oublier Daniel Lavalette), qui m’ont soutenue tout au long de ces années de travail. Et, plus que quiconque, je remercie Răzvan, mon soutien sans faille, d’avoir sublimé les 2500 km entre Bucarest et Paris pour venir à mes côtés.