\documentclass[output=paper]{langsci/langscibook}
\ChapterDOI{10.5281/zenodo.1406989}

\title{Lexemes, categories and paradigms: What about cardinals?}


\author{Gilles Boyé \affiliation{Université Bordeaux-Montaigne \& UMR5263 (CNRS)}} 

\abstract{In Word and Paradigm frameworks such as Network Morphology 
%(Corbett and Fraser, 1993)
\citep{CorbettFraser93}
 and Paradigm Function Morphology 
 %(Stump, 2001)
 \citep{Stump01}, %
 %
categories and lexemes are taken as granted and usually associated with an inflectional \is{inflection} paradigm \is{paradigm} relevant for all the lexemes in a given category. In Section \ref{Sec1}, we explore the status of \ili{French} cardinals \is{cardinals} as lexemes based on the characteristic properties defined by %
%Fradin (2003)
\citet{Fradin2003}%
%Fradin
%
: i) abstraction over form-variation, ii) autonomous forms, iii) stable meaning, iv) belonging to a major category, v) open-ended set of units that can serve as input and/or output of morphology. We start with the simple cardinals \is{cardinals} and argue, following %
%Saulnier (2008)
\citet{Saulnier08}%
%Saulnier
%
's discussion, that \ili{French} cardinals \is{cardinals} fit all the lexemic criteria but (iv), belonging to a major category, and should be considered full lexemes even though they constitute a sub-category of determiner, a minor category in Fradin's terms. In Section \ref{Sec2}, moving from simple cardinals \is{cardinals} to complex ones, we show that the idiosyncratic morphophonological properties of \ili{French} cardinals plead for a morphological analysis rather than a syntactic one, giving an analysis of their construction as multi-layered compounds. In Section \ref{Sec3}, we describe the inflectional \is{inflection} paradigms \is{paradigm} of \ili{French} cardinals \is{cardinals} as dependent on their rightmost element using the Right Edge mechanism introduced by %
%Miller (1992) 
\citet{Miller92} %
%Miller
%
and %
%Tseng (2003) 
\citet{Tseng03} %
%Tseng
%
for other phenomena in \ili{French}.
In the conclusion, we show that some complex cardinals \is{cardinals} have to be analyzed as multi-layered morphological compounds \is{compounding} due to their morphophonological idiosyncrasies but this does not entail that all complex cardinal should be. The fact that syntactic combinations of \ili{French} cardinals \is{cardinals} do not respect lexical integrity indicates that to some extent, complex cardinals \is{cardinals} are in the shared custody of morphology and syntax.
}

\maketitle
\begin{document}
\selectlanguage{english}

\section{Introduction}\label{Sec0}
In this paper, following the lead of \cite{Saulnier08,Saulnier10}, we explore the status of \ili{French} cardinals \is{cardinals} and their place in Word and Paradigm frameworks, within theories of morphology focusing on lexemes as their fundamental unit. In general, this topic poses interesting problems for linguistic theories:
\begin{itemize}
\item Are they lexemes? To what category do they belong: determiners, nouns, adjectives?
\item Are they built by syntax or in the lexicon?
\item Is there an inflectional \is{inflection} paradigm \is{paradigm} for cardinals? If so, where does it come from?
\end{itemize}


In Section \ref{cardLexemes}, we explore the categorial status of simple cardinals.
%The discussion is intentionally limited to simple cardinals to separate the issue of their lexical representation as lexemes from the question of complex cardinals being composed in syntax or in morphology.
In Section \ref{cardCat}, we argue that complex cardinals \is{cardinals} are lexemes, like simple cardinals, even though they constitute a subcategory of determiners.\footnote{This does not mean that all determiners are lexemes but rather that cardinals \is{cardinals} have to be treated as an exception.} We outline a syntagmatic analysis  to create complex cardinals \is{cardinals} in morphology as compounds. In the last section, we
%examine the inflectional paradigms of cardinals and propose a model to produce the appropriate forms
propose an analysis of the inflectional \is{inflection} paradigm \is{paradigm} of cardinals
based on the Right Edge mechanism introduced by \citet{Miller92} and \citet{Tseng03} for other phenomena in \ili{French}.
% définir le pb intéressant.
%In this paper, we explore the case of simple and complex cardinals in French in a Word and Paradigm framework, looking at their status in the lexicon, in syntax and at their paradigms.

%Simple cardinals belong in the lexicon but complex cardinals do not all belong there.
% opposer aussi les cardinaux lexicalisés aux cardinaux à former en ligne
%To some extent some complex cardinals are probably lexicalized, just as high frequency inflected forms are supposed to be, but most of them are derived online when the need arises.

%At one end of the spectrum, the derivation could be syntax as usual, combining simple cardinals syntagmatically to produce complex ones, at the other end, it could be contrived as inflection of a single numeral lexeme,
% est-ce que la flexion pourrait se concevoir pour une seule unité
% est-ce que la dérivation pourrait se concevoir pour une seule unité
%while, in between, it could be a word-formation process deriving the complex numerals from the simple ones.

%Numbers vs others
%	A syntactic combination? A numeral lexeme? Numeral lexemes?
%	A single category?
%	A uniform paradigm?
%
%From a semantic point of view, French complex cardinals are completely compositional \citep{Saulnier08}.

\section{French cardinals: Lexemes?}\label{cardLexemes}\label{Sec1}
In this section, we
%discuss
examine
the lexical status of \ili{French} cardinals.\footnote{For complex cardinals, see Section \ref{cardCat}.}

%: are they lexemes in the sense of \citeauthor{Fradin03}?

%\cite[102]{Fradin03} distinguishes two types of atomic units in the lexicon: morphological units, \emph{lexemes}, typically nouns, verbs, adjectives, adverbs and grammatical units, \emph{grammemes}, prepositions, determiners, conjunctions. But rather than an arbitrary contrast between word classes, he gives the following criteria\footnote{Translated from French} for lexemes:
Following \cite[102]{Fradin03}, we distinguish two types of atomic units in the lexicon: lexemes and grammemes. Lexemes are typically nouns, verbs, adjectives, adverbs, while grammemes are grammatical units such as prepositions, determiners, conjunctions. \citeauthor{Fradin03} identifies the following characteristic properties of lexemes:
\ea\label{lexCriteria}
	\ea It is an abstract unit to which word-forms are related; this unit captures the variations across word-forms.
	\ex It possesses a phonological representation which gives it prosodic autonomy.
	\ex Its meaning is stable and unique.
	\ex It belongs to a category and can have an argument structure.
	\ex It belongs to an open-ended set and can serve as output and input of derivational morphology.
	\z
\z
Whatever the analysis of \ili{French} complex cardinals \is{cardinals} such as \emph{vingt-et-un} \tradnum{21},
%simple cardinals like \emph{vingt} \tradnum{20} and \emph{un} \tradnum{1} have their place in the lexicon and have to be lexemes or grammemes. In the rest of this section, we look at the arguments for the lexematic status of lexical cardinals based only on the simple ones.
simple cardinals \is{cardinals} like \emph{vingt} or \emph{un} are underived and therefore have to be listed in the lexicon. In what follows, we argue that simple cardinals \is{cardinals} in \ili{French} pattern with lexemes rather than grammemes.


%The elements in \refp{simpleCardinals} are used both as cardinals and as construction units for complex cardinals.
In \ili{French}, the simple cardinals \is{cardinals} are the elements listed in \refp{simpleCardinals} that serve as cardinals \is{cardinals} and as building blocks for complex cardinals.\footnote{The elements \emph{million} and \emph{milliard} are not simple cardinals \is{cardinals} in \ili{French}; their respective values are realized as \emph{un million} (`one million') and \emph{un milliard} (`one billion'). They semantically belong to the quantity noun series in \emph{-aine} (see Table \ref{derivCARD}, p. \pageref{derivCARD})}
%Simple cardinals have straightforward semantics, they denote counting values, but their morphosyntax is complex.

\ea\label{simpleCardinals}
\emph{un} \tradnum{1}, \emph{deux} \tradnum{2}, \emph{trois} \tradnum{3}, \emph{quatre} \tradnum{4}, \emph{cinq} \tradnum{5}, \emph{six} \tradnum{6}, \emph{sept} \tradnum{7}, \emph{huit} \tradnum{8}, \emph{neuf} \tradnum{9}, \\\emph{dix} \tradnum{10}, \emph{onze} \tradnum{11}, \emph{douze} \tradnum{12}, \emph{treize} \tradnum{13}, \emph{quatorze} \tradnum{14}, \emph{quinze} \tradnum{15}, \emph{seize} \tradnum{16}, \\\emph{vingt} \tradnum{20}, \emph{trente} \tradnum{30}, \emph{quarante} \tradnum{40}, \emph{cinquante} \tradnum{50}, \emph{soixante} \tradnum{60}, \\\emph{cent} \tradnum{100}, \emph{mille} \tradnum{1000}
\z



%%%%%%% Liaison and Simple cardinals
Simple cardinals \is{cardinals} have the properties (\ref{lexCriteria}b--c). They can be used as single word answers, meaning they have an autonomous phonological representation. They have straightforward semantics, denoting counting values.

\subsection{\textbf{Form variation abstraction}}\label{formVariation}
\largerpage
As for property (\ref{lexCriteria}a), while \emph{un} \tradnum{1} is the only simple cardinal varying in gender (\textsc{m}: [œ̃] % was:\textipa{œ̃}
 \emph{un}, \textsc{f}: [yn] % was:\textipa{yn}
 \emph{une}), many simple cardinals \is{cardinals} are subject to \emph{liaison} (linking), a morphosyntactic phenomenon whereby \ili{French} words can change in form depending on the phonological properties of the following word. For example, in \refp{simpleLiaison}, the adjective \textsc{bon} agrees in gender and number with the following noun, in both cases masculine and singular. But in a liaison \is{liaison (French)} context such as prenominally, the form bɔ̃ % was:\textipa{bÕ}
 appears in \refp{simpleLiaisonA} in front of a word starting with a consonant (not a liaison \is{liaison (French)} trigger: \lmoins) and the form bɔn % was:\textipa{bOn}
 appears in \refp{simpleLiaisonB} in front of a vowel-initial word (a liaison \is{liaison (French)} trigger: \lplus). Outside liaison \is{liaison (French)} context (\lpause), adjectives assume the same form as in liaison \is{liaison (French)} context without trigger (\lpause=\lmoins{}).\footnote{For more details about the morpho-syntactic aspects of liaison \is{liaison (French)} see \cite{BoBoTse04}.}
\ea\label{simpleLiaison}
	\ea\label{simpleLiaisonA}\gll un bon collègue \\ 	
	œ̃ % was:\textipa{œ̃} 
	bɔ̃\lmoins{} % was:\textipa{bÕ\lmoins}
 kolɛɡ % was:\textipa{kolEg}
 \\ \\\trad{a good colleague}
	\ex\label{simpleLiaisonB}\gll un bon ami \\ œ̃ % was:\textipa{œ̃}
 bɔn% was:\textipa{bOn}
\lplus{} ami % was:\textipa{ami}
 \\ \\ \trad{a good friend}
	\ex\label{noLiaison}\gll bon à manger \\ bɔ̃\lpause{} a % was:\textipa{a}
 mɑ̃ʒe % was:\textipa{mÃZe}
 \\ \\ \trad{ready to eat}
	\z
\z
Unlike adjectives, cardinals \is{cardinals} can have three different forms for the three contexts above.\footnote{See \cite{Plenat08,Plenat11} and the citations therein for a detailed description.} For example, \emph{six} \tradnum{6} has different realizations (si, siz, sis% was:\textipa{si, siz, sis}
) for the three contexts:
\ea
	\ea in liaison \is{liaison (French)} context without a liaison \is{liaison (French)} trigger  $\Rightarrow$ si % was:\textipa{si}
\lmoins\\
		\gll six souris \\ si\lmoins{ % was:\textipa{si\lmoins{}
} suʁi % was:\textipa{suKi}
\\ \\ \trad{six mice}
	\ex in liaison \is{liaison (French)} context with a liaison \is{liaison (French)} trigger $\Rightarrow$ siz % was:\textipa{siz}
\lplus\\
		\gll six écureuils\\ siz\lplus{ % was:\textipa{siz\lplus{}
} ekyʁœj % was:\textipa{ekyKœj}
\\ \\ \trad{six squirrels}
	\ex not in liaison \is{liaison (French)} context $\Rightarrow$ sis % was:\textipa{sis}
\lpause\\
		\gll six à attraper\\ sis\lpause{} a % was:\textipa{a}
 atʁape % was:\textipa{atKape}
\\ \\ \trad{six to catch}
	\z
\z
%Considering the different forms and their distributions, the liaison variations give rise to five types of simple cardinals as shown in Table \ref{typesCardinals}.
Not all cardinals \is{cardinals} have different forms in all three contexts. Table \ref{typesCardinals} gives the five different patterns of syncretism found with the simple cardinals.
Type A cardinals \is{cardinals} are not sensitive to liaison and thus display only one form; in type B the \lmoins{} and the \lpause{} are identical and the \lplus{} has an additional consonant at the end, while in type C all three forms are distinct. In type D, \lmoins{}  is overabundant with a long form and a short form, and the long form is also used in the two other contexts. Type E is a variant of type B where instead of having an additional consonant for \lplus{}, the final fricative alternates between voiceless f % was:\textipa{f}
 and  its voiced counterpart v.\footnote{In the case of type E, there is also  hesitation for the \lplus{} form between nœv % was:\textipa{nœv}
 and nœf % was:\textipa{nœf}
 as they can both provide an onset for the following trigger unlike in type B.}
\begin{table}
	\resizebox{1\textwidth}{!}{
\begin{tabular}{crllll}
\lsptoprule
 Type  & Example & \lmoins & \lplus & \lpause & Cardinals \\
 \midrule
 A  & 4 & katʁ % was:\textipa{katK}
 & katʁ % was:\textipa{katK}
 & katʁ % was:\textipa{katK}
 & 4, 7, 11, 12, 13, 14, 15, 16, 30, 40, 50, 60, 1000 \\
 B & 2 & dø % was:\textipa{dø}
 & døz % was:\textipa{døz}
 & dø % was:\textipa{dø}
 & 1, 2, 3, 20, 100 \\
 C  & 6 & si % was:\textipa{si}
 & siz % was:\textipa{siz}
 & sis % was:\textipa{sis}
 & 6, 10  \\
 D  & 5 & sɛ̃/sɛ̃k % was:\textipa{sẼ/sẼk}
 & sɛ̃k % was:\textipa{sẼk}
 & sɛ̃k % was:\textipa{sẼk}
 & 5, 8 \\
 E  & 9 & nœf % was:\textipa{nœf}
 & nœv % was:\textipa{nœv}
 & nœf % was:\textipa{nœf}
 & 9 \\
\lspbottomrule
\end{tabular}}
\caption{Type of simple cardinal variation according to liaison}
\label{typesCardinals}
\end{table}

%So the elements in \refp{simpleCardinals} are abstractions capturing the form variation described in Table \ref{typesCardinals}, therefore fitting criterion (\ref{simpleCardinals}a).
The simple cardinals \is{cardinals} in \refp{simpleCardinals} have an associated form paradigm \is{paradigm} for liaison, \rephrase{fitting}{which fit} \citeauthor{Fradin03}'s property (\ref{lexCriteria}a). This property is part of the conceptual definition of lexeme; it is neither required nor sufficient by itself. Definite determiners which have form paradigms \is{paradigm} in \ili{French} and German are not considered lexemes, while English adjectives are lexemes even though their forms do not vary.

We turn now to the two remaining properties (\ref{lexCriteria}d--e): belonging to an open-ended category and participating as the output and potentially the input of derivational morphology.

% Card
%% A morphological category
%%% Derivations on Card (van Marle)
\subsection{\textbf{Morphological input}}

%\cite{Saulnier08,FradinSaulnier09,Saulnier10} have shown that simple cardinals serve as input for several morphological derivations
%as shown in the following table adapted from \cite{FradinSaulnier09}. \\
In \ili{French}, simple cardinals \is{cardinals} clearly serve as input for several morphological derivations as summarised in Table~\ref{derivCARD} below %
%\citep[see][for detailed discussion]{Saulnier08,FradinSaulnier09,Saulnier10}
(see \citealt{Saulnier08,FradinSaulnier09,Saulnier10} for a detailed discussion).\footnote{While belonging to the same series of nouns designating groups of approximate cardinality, \emph{millier} (`thousand'), \emph{million} (`million'), \emph{milliard} (`billion') are derived from \emph{mille} with different suffixes (\emph{-ier}, \emph{-ion}, \emph{-iard}).}
\begin{table}
\begin{tabular}{lll}
\lsptoprule
Suffix & Derivation & Category \\
\midrule
\emph{-ième} & \emph{deux (2) \fldr{} deuxième} (`second' ordinal) & Adj \\
\emph{-ième} & \emph{cinq (5) \fldr{} cinquième} (`fifth' part) & Adj/N \\
\emph{-ain} & \emph{quatre (4) \fldr{} quatrain} (`quatrain') & N \\
\emph{-aine} & \emph{douze (12) \fldr{} douzaine} (`dozen') & N \\
\emph{-aire} & \emph{trente (30) \fldr{} trentenaire} (`thirty-year-old') & Adj/N \\
\lspbottomrule
\end{tabular}
\caption{Some derivations on French cardinals \citep[adapted from][201]{FradinSaulnier09}}\label{derivCARD}
\end{table}


As bases for the ordinals, simple cardinals \is{cardinals} are part of a morphological category in terms of \cite{vanMarle85} namely the derivational domain of ordinals, but to satisfy (\ref{lexCriteria}d), simple cardinals \is{cardinals} have to belong to a unique morphosyntactic category.

%%%%% Zéro and the variables
\subsection{\textbf{Morphosyntactic category}}\label{catCARD}

Following \cite{Saulnier10}, we consider simple cardinals \is{cardinals} to be a sub-category of indefinite determiners, CARD.

\cite[31--40]{Saulnier10} applies the discriminating contexts defined in \cite{Leeman04}'s work on \ili{French} indefinite determiners. She shows that cardinals \is{cardinals} have the following distribution across the six diagnostic contexts.
\ea\label{cardCriteria}
\begin{description}
    \item \emph{en} dislocation: $+$ \fldr{} il a \emph{deux} solutions = il en a \emph{deux} 		\\\hfill\trad{he has 2 solutions = he has 2}
    \item only alone before N: $-$  \fldr{} mes \emph{deux} livres (*mes \emph{plusieurs} livres) 	\\\hfill\trad{my 2 books (*my several books)}
    \item following the indefinite: $-$ \fldr{} *un \emph{deux} livres (un \emph{certain} livre)		\\\hfill\trad{*a 2 books (a certain book)}
    \item following the definite: $+$ \fldr{} les \emph{deux} livres (*les \emph{certains} livres)		\\\hfill\trad{the 2 books (*the certain books)}
    \item followed by the definite: $-$ \fldr{} *\emph{deux} les livres (\emph{tous} les livres)		\\\hfill\trad{*2 the books (all the books)}
    \item followed by \emph{de} NP: $+$ \fldr{} \emph{deux} de mes collègues				\\\hfill\trad{2 of my colleagues}
\end{description}
\z

With these criteria in mind for the category CARD, it becomes clear that there are simple cardinals \is{cardinals} that were not listed in \refp{simpleCardinals} because they do not participate in the formation of complex cardinals.
%The elements in \refp{simpleCardinals} are not the only French simple cardinals in CARD.

\emph{Zéro} \tradnum{0}, for example, is not a construction unit for complex cardinals \is{cardinals} but it behaves like a CARD in all the contexts in \refp{cardCriteria}. \cite[38]{Saulnier10} considered \emph{zéro} to depart from the cardinals \is{cardinals} distribution because she could not find examples for the contexts in \refp{exampleZero}, expecting \emph{zéro} to be singular.\footnote{In the same contexts, \citeauthor{Saulnier10} does not examine \emph{un} and the surprising plural number that arises when it follows a definite or a possessive. For example, in \emph{pour ses/son \textbf{un} mois} \trad{for his one month anniversary}, the masculine singular form of the possessive \emph{son} is far less common than the plural \emph{ses}; the possessive can take its plural form \emph{ses} despite the presence of the cardinal \emph{un} \tradnum{1}.}
%The examples in \refp{exampleZero} show \emph{zéro} appearing in these contexts.

  
\newpage 
\ea\label{exampleZero} %\cite{Saulnier10} missing examples\footnote{Examples from the web (26/12/2016)} (p. 38)
Examples from the web (26/12/2016)
\begin{description}
\item \emph{en} dislocation: $+$ \fldr{}  Il a des tas d'contacts, des tonnes de numéros pour remplir son phone mais des vrais potes il en a \emph{zéro}.%
\footnote{\trad{He's got many contacts, tons of numbers to fill his phone, but real mates, he's got zero.}\\\protect\url{https://genius.com/Enz-narcisse-and-cassandre-lyrics}}
\item only alone before N: $-$ \fldr{} Et il ne nous restera alors que \textbf{nos\indi{pl}} \emph{zéro} euros d'augmentation pour pouvoir demander un crédit.%
\footnote{\trad{Then we will only have our 0 euros of raise to ask for a credit.}\\\url{http://psasochaux.reference-syndicale.fr/files/2015/04/Tract-avril-15.pdf}}
\item following the indefinite: $-$ \fldr{} *un \emph{zéro} livre/livres.%
\footnote{\trad{*a zero book/books}}
\item following the definite: $+$ \fldr{}  Je vote pour \textbf{les\indi{pl}} \emph{zéro} heures payées trente-cinq.%
\footnote{\trad{I'm voting for the 0 hours being paid as 35.}\\\url{https://fr.toluna.com/opinions/762230/Etes-vous-pour-ou-contre-les-35-heures}}
\item followed by \emph{de} NP: $+$ \fldr{}  Mais même les potes des autres viennent ici et \emph{zéro} \textbf{de mes potes} sont venus me voir.%
\footnote{\trad{But while even the other guys' pals come here, 0 of mine have come to see me.}\\\url{https://twitter.com/MisHyding/status/762360289329307649}}
\end{description}

\noindent And contra \citet{Saulnier10}, \emph{zéro} also appears in \emph{zéro}+N subject NPs:
\begin{description}
\item \emph{zéro}+N subject: $+$ \fldr{}  Pendant ce temps, \textbf{\emph{zéro} personnes\indi{pl}} sont mortes de  surdoses de marijuana.%
\footnote{\trad{All this while, 0 persons have died of marijuana overdose.}\\\url{https://anarchocommunismelibertaire.wordpress.com/}}
\end{description}
%	\ea On est fatigués de vos discours alarmistes avec \emph{zéro} solutions en face.\footnote{\url{https://www.mouvement-republicain-caledonien.com/2013/09/18/les-transferts-de-l-article-27-optionnels-inutiles-et-dangereux/}, }
\z
In derivational morphology, \emph{zéro} also gives a corresponding ordinal \emph{zéroième} following the pattern of other simple cardinals.

\subsection{\textbf{Morphological output}}

Apart from fixed value cardinals, \ili{French} uses variable cardinals \is{cardinals} such as \emph{n} \trad{n} (pronounced [ɛn] % was:\textipa{[En]}
) or  \emph{x} \trad{x} (pronounced [iks] % was:\textipa{[iks]}
). Like \emph{zéro}, these variable cardinals \is{cardinals} do not participate in complex cardinal formation but they appear in the contexts in \refp{cardCriteria} and allow a subset of the derivations for fixed value cardinals \is{cardinals} (e.g. \emph{énième} \trad{nth} pronounced [ɛnjɛm] % was:\textipa{[EnjEm]}
 and \emph{xième} \trad{xth} pronounced  [iksjɛm] % was:\textipa{[iksjEm]}
).
\ea\label{lettreCardinals}
	\ea Une solution consiste à rechercher les \emph{N} meilleures solutions pour chaque ville épelée.\footnote{\trad{A solution would be to search for the N best possibilities for every city name.}\\\url{http://www.afcp-parole.org/spip.php?article152}}
	\ex Donc l'installateur fait des bidouilles avec les \emph{X} paramètres qui en [soi] ne sont pas très clairs ou pas forcément adaptés aux diverses situations des clients...\footnote{\trad{So the installer switches around the X parameters which are a bit obscure or not necessarily adapted to the various customer situations.}\\\url{https://www.bricozone.fr/t/reglage-chaudiere-viessman.11296/page-7}} 	
	\ex Aujourd'hui, je constate que pour la \emph{énième} fois, une voiture est garée devant mon entrée de garage, m'empêchant de sortir.\footnote{\trad{Today, for the nth time, I see a car parked in front of my garage door, blocking my way.}\\\url{https://goo.gl/lOrTuo}}
	\z
\z
These cardinals \is{cardinals} are obtained by converting letter names, usually \ili{French} or Greek, to cardinals, making them the output of a morphological process and therefore fitting part of criterion (\ref{lexCriteria}e).

\subsection{\textbf{Open-ended set}}

In the general domain or in mathematical contexts this practice is limited to the conversion of a few letter names, but in computer programming names for integer-valued variables are created all the time and behave as simple cardinals \is{cardinals}, making CARD an open-ended category.\footnote{Note that the \ili{French} complex cardinals \is{cardinals} are not an open-ended set but rather a large set containing one trillion elements, as \ili{French} speakers can count from 0 to \numprint{999999999999}.} Even the derived ordinals appear in computer program descriptions.
\ea\label{progCardinals}
	\ea Lance le son à partir de la \emph{nbième} [ɛnbejɛm] seconde.\footnote{\trad{Run the soundbite from the nbth second.}\\\url{http://www.forum-dessine.fr/index.php?id=06038}}
	\ex appFunc(NUM):	Renvoie l'adresse de la \emph{NUMième} fonction de la page courante\footnote{\trad{Returns the address of the NUMth function in the current page.}\\\url{https://goo.gl/LHh46c}}
	\z
\z

The preceding discussion shows that \ili{French} simple cardinals \is{cardinals} are part of an open-ended set with the productive coinage of integer variables. As we have seen above, ordinal derivation takes simple cardinals \is{cardinals} as input and letter name conversion gives simple cardinals \is{cardinals} as output. These three observations indicate that \ili{French} simple cardinals \is{cardinals} fit the property (\ref{lexCriteria}e).

\subsection{\textbf{Interim conclusion: the lexical status of simple cardinals}}

%To answer the question raised in this section, simple cardinals fit all the criteria from \cite{Fradin03} to be lexemes. Their behaviour coincides with the intuition that new lexemes are instantly created by derivation, borrowing and arbitrary coining while grammemes emerge through diachronic phenomena.
In this section, we have shown that simple cardinals \is{cardinals} in \ili{French} have all the properties deemed characteristic of lexemes by \citet{Fradin03}. Like typical lexemes, elements of CARD are created by borrowing and arbitrary coining while grammemes emerge through diachronic phenomena.
% Grammemes tend to appear one by one, and lexemes in series.
% What about Fradin's criteria? And the intuition also bears on complex cardinals
Considering simple cardinals \is{cardinals} to be lexemes might seem at odds with the fact that we have taken them to be a sub-category of determiners, usually not regarded as a lexeme-based category. In the following section, we argue that CARD, in general, are a part of the syntactic category of determiners but constitute a morphological % morphosyntactic
category of their own.
\bigskip


%%% Card creations: zéro, n, x

% => Saunier philosophical considerations
% => Lexemes

% \citep{Hurford98}
% Simple cardinals
% % Almost perfect Semantics => IM06
%%% Case assignment

% Complex cardinals
%% Almost perfect X-Bar => Hurford
%%% Plurality and number mismatch
%% Phonological idiosyncrasies

% Cardinals => Lexical units
%% Extra-linguistical meaning (proper names)
%% CARD => Saunier

\section{French cardinals: Category?}\label{cardCat}\label{Sec2}
In this section, we examine the status of \ili{French} cardinals, simple and complex. We start with an overview of `The Composition of Complex Cardinals' \citep{IM06}, as an example of a completely syntactic view of cardinal derivation. Then we argue that the phonological idiosyncrasies of complex cardinals \is{cardinals} are best modelled with a morpholexical system.

\subsection{Complex cardinals in syntax}

\cite[316]{IM06} argue that `complex cardinals \is{cardinals} are composed entirely in syntax and interpreted by the regular rules of semantic composition'.

\subsubsection{\textbf{Semantics}}
Their analysis describes the semantics of complex cardinals \is{cardinals} and their syntax in several languages, focusing particularly on \ili{Russian}.
To allow for the semantic combination of Cards in CardP, they propose that simplex cardinals \is{cardinals} have the type \textless{}\textless{}e,t\textgreater{}, \textless{}e,t\textgreater{}\textgreater{} so that a series of simplex cardinal followed by a noun predicate of type \textless{}e,t\textgreater{} will be able to combine step by step with a parent simplex cardinal as in \refp{combET} and result in a type \textless{}e,t\textgreater{}.
\ea\label{combET}
\Tree [ .\textless{}e,t\textgreater{}
		[ .\textless{}\textless{}e,t\textgreater{},\textless{}e,t\textgreater{}\textgreater{}\\\emph{two} ]
		[ .\textless{}e,t\textgreater{}
			[ .\textless{}\textless{}e,t\textgreater{},\textless{}e,t\textgreater{}\textgreater{}\\\emph{hundred} ]
			[ .\textless{}e,t\textgreater{}\\\emph{books}
			]
		]
	]
\z
The actual semantic combination is not described in detail but the authors seem to rely on the packing strategy of \cite{Hurford07} where complex cardinals \is{cardinals} are analyzed based on the simple set of syntagmatic rules associated with calculations in \refp{numberSG}. Figure \ref{fig:Boye:numberPhrase} gives the corresponding structure for \numprint{5002600}.
\ea\label{numberSG}
\begin{itemize}
\item NUMBER \fldr{} \begin{avm}\{\,DIGIT\\ \,PHRASE (NUMBER)\,\}\end{avm}\vspace{3pt}\\ $\text{value}(\text{NUMBER}) = \text{value}(\text{PHRASE}) + \text{value}(\text{NUMBER})$\\[6pt]
\item PHRASE \fldr{} (NUMBER) M\\ $\text{value}(\text{PHRASE}) = \text{value}(\text{NUMBER}) \times \text{value}(\text{M})$
\end{itemize}
\z
\begin{figure}
\small
\Tree [ .NUMBER
		[ .PHRASE
			[ .NUMBER [ .DIGIT five ] ]
			[ .M million ]
		]
		[ .NUMBER
			[ .PHRASE
				[ .NUMBER [ .DIGIT two ] ]
				[ .M thousand ]
			]
			[ .NUMBER
				[ .PHRASE
					[ .NUMBER [ .DIGIT six ] ]
					[ .M hundred ]
				]
			]
		]
	]
\caption{Syntagmatic analysis of \numprint{5002600} from \citet{Hurford07}}
\label{fig:Boye:numberPhrase}
\end{figure}

\citeauthor{Hurford07} describes the packing strategy as a constraint on the syntagmatic grammar in \refp{numberSG}:
\begin{itemize}
\item The sister constituent of a NUMBER must have the highest possible value.\footnote{This constraint is intended to have the same effect as converting time in seconds into complex units such as days/hours/minutes/seconds, maximising the number of days first, then hours, minutes and finally seconds.}
\end{itemize}

The semantic analysis proposed by \citet{IM06} does not warrant a syntactic view of complex cardinals. From an external perspective, it manages to treat complex cardinals \is{cardinals} and simple cardinals \is{cardinals} in the same manner, giving them the same semantic type and the same combinatorial constraints on the counted noun (atomicity and countability).

\subsubsection{\textbf{Syntax}}
 \largerpage
Concerning syntax, \cite{IM06} describe two phenomena relevant to \ili{French} cardinals: case assignment and number morphology. In \ili{Russian}, cardinal-contain\-ing NPs do not realize the direct cases (nominative \& accusative) the same way as other NPs. For example, the NPs in \refp{nomNP} could all be used as subjects or direct objects. In (\ref{nomNP}a), \emph{\v{s}ag} \trad{step} has the nominative/accusative plural form expected for a direct argument but in (\ref{nomNP}b) it has the genitive singular form (paucal in the terms of \citeauthor{IM06}) and, in (\ref{nomNP}c), the genitive plural form.
\ea\label{nomNP}
	\ea\gll \v{s}ag-i \\ step-\textsc{nom.pl}\\ \\\trad{steps}
	\ex\gll \v{c}etyre \v{s}ag-á \\ four step-\textsc{gen.sg}\\ \\\trad{four steps}       
	\ex\gll \v{s}est' \v{s}ag-ov \\ six step-\textsc{gen.pl}\\ \\\trad{six steps}
	\z
\z
The case and number appearing on the head noun depend on the last simple cardinal in CardP. Cardinal 1 does not interfere with direct cases, cardinals \is{cardinals} 2--4 assign genitive singular and the other cardinals \is{cardinals} assign genitive plural.

This phenomenon also happens inside CardP in multiplicative contexts such as \refp{caseMille}. \emph{Tysja\v{c}a} \tradnum{1000} appears in the nominative singular alone, but in the genitive singular with 4 and in the genitive plural with 5.
\ea\label{caseMille}
	\ea\gll tysja\v{c}-a \v{s}ag-ov \\ thousand-\textsc{nom.sg} step-\textsc{gen.pl}\\ \\\trad{one thousand steps}
	\ex\gll \v{c}etyre tysja\v{c}-i \v{s}ag-ov \\ four thousand-\textsc{gen.sg} step-\textsc{gen.pl}\\ \\\trad{four thousand steps}
	\ex\gll pjat' tysja\v{c} \v{s}ag-ov \\ five thousand.\textsc{gen.pl} step-\textsc{gen.pl}\\ \\\trad{five thousand steps}
	\z
\z
The form variations above do not interfere with the external case and number. The case and number realized internally on the head noun and the multiplied cardinals \is{cardinals} in the CardP do not affect the case and number of the NP in its relation to the rest of the sentence.

\ili{French} does not have an inflectional \is{inflection} case system similar to \ili{Russian} but cardinals \is{cardinals} still display similar properties.
In syntax, the CARD category identified for morphology in section \ref{catCARD} opposes the cardinals \is{cardinals} ending with elements \emph{million} and \emph{milliard}, infelicitous in \refp{numN}, with all other cardinals infelicitous in \refp{numDeN}.\footnote{This could be contrived as \emph{million} and \emph{milliard} being classifiers but their behavior in complex numerals shows that they are indeed cardinal construction elements.
\ea un milliard trois cents millions \textbf{d'}euros \trad{\numprint{1300000000} euros}
\ex un\indi{\tiny\textsc{m}} million une\indi{\tiny\textsc{f}} pages\indi{\tiny\textsc{f}} \trad{\numprint{1000001} pages}
\z}
\ea\label{numOpposition}
	\ea\label{numN} Paul a deux/cent/*un million euros à la banque. \\\trad{Paul has X euros in his account.}
	\ex\label{numDeN} Paul a *deux/*cent/un million \textbf{d'}euros à la banque. \\\trad{Paul has X \textbf{of} euros in his account.}
	\z
\z
The data in \refp{numOpposition} could be interpreted as a difference in category, \emph{un million} being considered as a noun rather than a CARD.
But while the use of \emph{un million} changes the shape of the NP, it does not affect its external relations to the sentence, just as in \ili{Russian}. It appears that \emph{millions} and \emph{milliard} assign genitive plural to the head noun resulting in a \emph{de} NP without changing the overall distribution of the cardinal-containing NP.
Both structures participate in the contexts \refp{cardCriteria} used by \cite{Saulnier10} repeated below.
\ea
\begin{description}
    \item \emph{en} dislocation: + \fldr{} il en a \emph{deux}/\emph{un million}\\\trad{he has 2/1,000,000}
    \item only alone before N: - \fldr{} mes \emph{deux} livres/mes \emph{un million} de livres\\\trad{my 2/1,000,000 books}
    \item following the definite: + \fldr{} les \emph{deux} livres/les \emph{un million} de livres\\\trad{the 2/1,000,000 books}
    \item followed by \emph{de} NP: + \fldr{} \emph{deux}/\emph{un million} de mes collègues\\\trad{2 /1,000,000 of my colleagues}
\end{description}
\z
Including \emph{million}, \emph{milliard} and their combinations in the CARD category with different controlling features captures the external similarity while retaining the appropriate contrast between the different NP structures CARD N vs CARD \emph{de} N in the examples above.

\ili{French} also displays number morphology inside complex cardinals \is{cardinals}, like \ili{Russian}. The marks are visible in liaison \is{liaison (French)} contexts before triggers as shown in \refp{numCards}.
\ea\label{numCards}
	\ea\gll \emph{cent} ans \\ \emph{sɑ̃t % was:\textipa{sÃt}
} ɑ̃ % was:\textipa{Ã}
 \\ \\\trad{one hundred years}
	\ex\gll deux \emph{cents} ans \\ dø % was:\textipa{dø}
 \emph{sɑ̃z % was:\textipa{sÃz}
} ɑ̃ % was:\textipa{Ã}
 \\ \\\trad{two hundred years}
	\ex\gll \emph{vingt} ans \\ \emph{vɛ̃t % was:\textipa{vẼt}
} ɑ̃ % was:\textipa{Ã}
 \\ \\\trad{twenty years}
	\ex\gll quatre -\emph{vingts} ans \\ katʁə % was:\textipa{katK@}
-\emph{vɛ̃z % was:\textipa{vẼz}
} ɑ̃ % was:\textipa{Ã}
 \\ \\\trad{eighty years}
	\z
\z
The  \lplus{} forms of simple cardinals \is{cardinals} \emph{cent} and \emph{vingt} end in t % was:\textipa{t}
 but their final consonant is replaced by z % was:\textipa{z}
 in multiplicative contexts.%
\footnote{In liaison \is{liaison (French)} contexts, the t-final % was:\textipa{t}
 \lplus{} forms alternate with the \lmoins{} forms depending on collocations. Frequent ones such as \emph{vingt ans} \trad{20 years} and \emph{cent ans} \trad{100 years} are generally pronounced with \lplus{} forms (vɛ̃t\lplus{}ɑ̃, sɑ̃t\lplus{}ɑ̃), but rarer collocations like \emph{vingt écureuils}  \trad{20 squirrels} and \emph{cent écureuils} \trad{100 squirrels} are often found with the \lpause{} forms (vɛ̃\lpause{}ekyʁœj, sɑ̃\lpause{}ekyʁœj). But in any case, the emergence of a z-final \lplus{} form outside multiplicative contexts is considered faulty: *vɛ̃z\lplus{}ekyʁœj, *sɑ̃z\lplus{}ekyʁœj.}
This change does not seem to be mandated by plural marking as \emph{cent} and \emph{vingt} are already plural controllers.\footnote{\cite[Section 3]{Hurford98} describes a case in Finnish were number marking on cardinals \is{cardinals} makes a difference. Plural cardinals \is{cardinals} count groups of N while singular cardinals \is{cardinals} count N individuals.}
%% A syntactic category
%%% 3 Liaison context
%%% External distribution
%%% Internal distribution

All in all, \cite{IM06} and \cite{Hurford07} provide an interesting framework in which to analyze \ili{French} cardinals \is{cardinals} as a unique syntactic category. The differentiated control properties and the idiosyncrasic number morphology they propose allows for a uniform syntactic analysis where all complex cardinals \is{cardinals} are constructed in the same way. However, the phonological aspects of \ili{French} cardinals \is{cardinals} do not go along with the perfectly predictable semantics and syntax of the complex cardinals \is{cardinals} on which \cite{IM06} build their syntactic view of the process.
% Restructure the last sentence to make it understandable.

\subsection{Complex cardinals and phonology}
From a phonological standpoint, idiosyncrasies are everywhere in the construction of \ili{French} complex cardinals. In the following we review the various combinatorial exceptions in the formation of complex cardinals \is{cardinals} and argue that it would be difficult to account for these with a purely syntactic analysis.

As we have seen in section \ref{formVariation}, \ili{French} simple cardinals \is{cardinals} are subject to form variation according to liaison \is{liaison (French)} contexts.
%Whereas other syntactic categories contrast only two linking contexts depending on the presence of a triggering word on their right, cardinals distinguish three types of linking context.
In the derivation of complex cardinals, however, simple cardinals \is{cardinals} use the same forms but in quite different distributions. For example, \emph{vingt} \tradnum{20} and \emph{cent} \tradnum{100} belong to the same type B in Table \ref{typesCardinals}, p. \pageref{typesCardinals}: both combine with simple cardinals \is{cardinals} 2--9, but \emph{vingt} uses the \lplus{} form vɛ̃t\footnote{Note that this holds true independently of the fact that the \lpause{} form of 20 is subject to diatopic variation between vɛ̃ % was:\textipa{vẼ}
 and vɛ̃t % was:\textipa{vẼt}
.} even though these cardinals \is{cardinals} are not liaison \is{liaison (French)} triggers, while \emph{cent} uses the \lmoins{} form sɑ̃ % was:\textipa{sÃ}
 in the same context, as shown in Table \ref{vingtCent}.
\begin{table}
\fittable{
\begin{tabular}{rcccccccc}
\lsptoprule
&2&3&4&5&6&7&8&9\\
\midrule
20\lplus{} & vɛ̃t-dø % was:\textipa{vẼt-dø}
 & vɛ̃t-tʁwa % was:\textipa{vẼt-tKwa}
 &vɛ̃t-katʁ % was:\textipa{vẼt-katK}
 &vɛ̃t-sɛ̃k % was:\textipa{vẼt-sẼk}
 &vɛ̃t-sis % was:\textipa{vẼt-sis}
 &vɛ̃t-sɛt % was:\textipa{vẼt-sEt}
 &vɛ̃t-ɥit % was:\textipa{vẼt-4it}
 &vɛ̃t-nœf % was:\textipa{vẼt-nœf}
 \\
100\lmoins{} & sɑ̃-dø % was:\textipa{sÃ-dø}
 & sɑ̃-tʁwa % was:\textipa{sÃ-tKwa}
 &sɑ̃-katʁ % was:\textipa{sÃ-katK}
 &sɑ̃-sɛ̃k % was:\textipa{sÃ-sẼk}
 &sɑ̃-sis % was:\textipa{sÃ-sis}
 &sɑ̃-sɛt % was:\textipa{sÃ-sEt}
 &sɑ̃-ɥit % was:\textipa{sÃ-4it}
 &sɑ̃-nœf % was:\textipa{sÃ-nœf}
 \\
\lspbottomrule
\end{tabular}
}
\caption{\emph{vingt} and \emph{cent} combinations with simple cardinals from 2 to 9}
\label{vingtCent}
\end{table}
\medskip

Combinations involving \emph{cinq} \tradnum{5} and \emph{huit} \tradnum{8} in the construction of multiples of 100 and 1000 are not parallel even though they belong to the same type D of simple cardinals \is{cardinals} in Table \ref{typesCardinals}, with two alternating realisations for the \lmoins{} form: sɛ̃/sɛ̃k % was:\textipa{sẼ/sẼk}
, ɥi/ɥit. % was:\textipa{4i/4it}
With \emph{cinq} both of the \lmoins{} forms can be used in the combinations but with \emph{huit} only the short \lmoins{} form ɥi % was:\textipa{4i}
 is felicitous:
\ea
	\ea 500 sɛ̃-sɑ̃/sɛ̃k-sɑ̃ % was:\textipa{sẼ-sÃ/sẼk-sÃ}
, 5000 sɛ̃-mil/sɛ̃k-mil % was:\textipa{sẼ-mil/sẼk-mil}

	\ex 800 ɥi-sɑ̃/*ɥit-sɑ̃ % was:\textipa{4i-sÃ/*4it-sÃ}
, 5000 ɥi-mil/*ɥit-mil % was:\textipa{4i-mil/*4it-mil}

	\z
\z
\medskip

Moreover, the same simple cardinal \emph{dix} \tradnum{10} combines with 7--9 and with 1000, none of which are liaison \is{liaison (French)} triggers, but it uses the \lplus{} form in the first case  and the  \lmoins{} in the second:
 \ea
	\ea 17 diz-sɛt % was:\textipa{diz-sEt}
, 18 diz-ɥit % was:\textipa{diz-4it}
, 19 diz-nœf % was:\textipa{diz-nœf}

	\ex 10000 di-mil % was:\textipa{di-mil}

	\z
\z
\medskip

Finally, instances of \emph{quatre-vingt} have to be pronounced with an r % was:\textipa{r}
 at the end of \emph{quatre}, even for speakers who usually drop it in word-final complex codas.
\ea
	\ea un arbre frappé par la foudre\\œ̃n aʁbʁə fʁape paʁ la fudʁə % was:\textipa{œ̃n aKbK@ fKape paK la fudK@}
 = œ̃n aʁb fʁape paʁ la fud % was:\textipa{œ̃n aKb fKape paK la fud}

	\ex vingt-quatre francs \\ vɛ̃tkatʁə fʁɑ̃ % was:\textipa{vẼtkatK@ fKÃ}
 = vɛ̃tkat fʁɑ̃ % was:\textipa{vẼtkat fKÃ}

	\ex quatre-vingts francs \\ katʁəvɛ̃ fʁɑ̃ % was:\textipa{katK@vẼ fKÃ}
 $\neq$ *katvɛ̃ fʁɑ̃\footnote{katvɛ̃ % was:\textipa{katvẼ}
 is correct, however, for the decimal number \tradnum{4.20}.}
	\z
\z
% refaire un tableau pour les éléments simples en composition
We conclude that even though both the semantic and syntactic dimensions of complex cardinal formation are simple and regular, the combinatory principles at work at the phonological level are far from simple and must be specific to cardinal formation, leading us away from syntax and towards a lexical account of the derivation of complex cardinals.\is{cardinals}


% Cardinals => Single category (French)

% Quid du nombre sur N dans DET-PL 1 N, les un an(s)

\subsection{\textbf{Complex cardinals in CARD}}

As complex cardinals \is{cardinals} have the same distribution in the \citeauthor{Saulnier10}-\citeauthor{Leeman04} contexts in \refp{cardCriteria} and serve as input for the ordinal derivation,
%To account for the fact that cardinals themselves constitute a lexical category,
we analyze numerical cardinals \is{cardinals} as compounds \is{compounding} created by means of a phrase structure grammar similar to those proposed by \cite{Hurford75,Hurford94,Hurford98,Hurford07}. The analysis will be presented in two parts. We first introduce a model limited to the structure of 2-digit cardinals \is{cardinals} where most of the phonological and syntagmatic idiosyncrasies occur and then generalize it to the rest of the cardinals.

\subsubsection{2-digit cardinals}

Cardinal components are categorized according to their combinatorial properties (Table \ref{2D-Cat}). To demonstrate the mechanics of the analysis, we use arbitrary categories rather than motivated features to differentiate elements. The category names reflect their purpose in the system. Unit categories start with u for digits (u, u1, u4, u7) and uv (uv, uv1) for units under 20, while categories for multiples of ten begin with d (d, d1, d2, d6).\footnote{To account for the Swiss and Belgian cardinal systems, the category d would have to include \emph{septante} \tradnum{70}, \emph{octante/huitante} \tradnum{80} and \emph{nonante} \tradnum{90}.}

\begin{savenotes}
\begin{table}
\begin{tabular}{llll}
 \lsptoprule
 Cat  & Components & Example \\
 \midrule 
 u & deux (2), trois (3), cinq (5), six (6) & vingt-\textbf{deux} &20+\textbf{2}=22\\
 u1 & un (1) & vingt-et-\textbf{un} &20\&\textbf{1}=21\\
 u4 & quatre (4)& \textbf{quatre}-vingts &\textbf{4}x20=80\\
 u7 & sept (7), huit (8), neuf (9)&dix-\textbf{sept} &10+\textbf{7}=17\\
 \midrule
 uv & douze (12), treize (13), quatorze (14) & soixante-\textbf{douze} &60+\textbf{12}=72\\
 & quinze (15), seize (16) & \\
uv1 & onze (11)& soixante-et-\textbf{onze} &60\&\textbf{11}=71 \\
\midrule
 d & trente (30), quarante (40), cinquante (50) & \textbf{trente}-deux &\textbf{30}+2=32 \\
 d1 & dix (10)& *\textbf{dix}-deux$\neq$douze &\textbf{10}+2$\neq$12 \\
      &             & soixante-\textbf{dix} &60+\textbf{10}=70 \\
 d2 & vingt (20)&quatre-\textbf{vingts} &4x\textbf{20}=80\\
 d6 & soixante (60) & \textbf{soixante}-treize &\textbf{60}+13=73\\
\midrule
 et & et (\&) & trente-\textbf{et}-un &30\textbf{\&}1=31\\
\lspbottomrule
\end{tabular}
\caption{Categories of cardinal components for 2-digit cardinals}\label{2D-Cat}
\end{table}
\end{savenotes}
% 10 does not combine with u but rather with u7

The rules in Table \ref{2D-Rules} generate all 2-digit cardinals \is{cardinals} (category Digit2). Rule 1 states that simple cardinals \is{cardinals} are de facto Digit2. Rule 2 generates \emph{dix-sept}, \emph{dix-huit}, \emph{dix-neuf}. Rules 3 and 5 assemble \emph{et un} and \emph{et onze}. Rule 4 produces DixP for number between \emph{vingt} \tradnum{20} and \emph{cinquante-neuf} \tradnum{59}.\footnote{In rule 4, the \lplus{} form is selected for the first term: d2.\lplus{}=vɛ̃+t % was:\textipa{vẼ+t}
} Rule 6 makes the \emph{soixante} compounds \is{compounding} from \emph{soixante} \tradnum{60} to \emph{soixante-dix-neuf} \tradnum{79} and rules 7 to 9 create the compounds \is{compounding} based on \emph{quatre-vingt} for number between \emph{quatre-vingts} \tradnum{80} and \emph{quatre-vingt-dix-neuf} \tradnum{99}.\footnote{In rule 7, the \lmoins{} form is selected for the first term,  Dix8X.\lmoins{}=katʁəvɛ̃. In rule 9, The liaison \is{liaison (French)} consonant for d2 changes to z, \lplus{} becomes vɛ̃+z.} Finally, rule 10 elevates all intermediary compounds \is{compounding} to Digit2.


\begin{savenotes}
\begin{table}
\fittable{
\begin{tabular}{rll}
\lsptoprule
& Rule & Comment \\
\midrule 
 \nex[1]&Digit2 \fldr{} u/u1/u4/u7/uv/uv1/d/d1/d2/d6 & simplex cardinals \\
 \nex&Dix1P \fldr{} d1.\lplus{} u7  & diz % was:\textipa{diz}
 (10) for 10+7..9 \\
  \nex&Et1 \fldr{} et u1  & eœ̃ % was:\textipa{eœ̃}
(\&1) for 20/30/40/50\&1 \\
 \nex&DixP \fldr{} d/d2.\lplus{} u/u4/u7/Et1 &  20/30/40/50+2..9/\&1\\
 \nex&Et11 \fldr{} et u1/uv1 & eœ̃/eɔ̃z % was:\textipa{eœ̃/eÕz}
(\&1/\&11) for 60\&1/11 \\
\nex&DixP \fldr{} d6 u/u4/u7/d1/Et11/uv/Dix1P & 60+2..10/12..16/(10+7..9)/\&(1/11) \\
%\nex&DixP \fldr{} Dix8X & 80 \\
\nex&Dix8X \fldr{} u4 d2.z % was:\textipa{z}
 & 4x20 for 80\\
\nex&DixP \fldr{} Dix8X & complex 80\\
\nex&DixP \fldr{} Dix8X.\lmoins{} u/u1/u4/u7/d1/uv/uv1/Dix1P & 80+1..16/(10+7..9)\\
\nex&Digit2 \fldr{} DixP & complex cardinals \\
\lspbottomrule
\end{tabular}
}
\caption{Syntagmatic rules for 2-digit cardinals}
\label{2D-Rules}
\end{table}
\end{savenotes}
%\footnotetext{The selected combining form for d2.\lplus{} is vɛ̃t % was:\textipa{vẼt}}
%\footnotetext{The selection applies only when there is a combination}
%\footnotetext{The liaison consonant for d2 changes to z % was:\textipa{z}}


\FloatBarrier

The syntagmatic rules in Table \ref{2D-Rules} integrate constraints stipulating the combining forms:
\ea
	\ea Rules 2 and 4 use the linking form \lplus{} of the first component;
	\ex Rule 7 changes the liaison \is{liaison (French)} consonant of the second component from t % was:\textipa{t}
 to z; % was:\textipa{z}

	\ex Rule 9 uses the \lmoins{} form  of the first component.
	\z
\z
Figure \ref{rules1-4} illustrates the application of rules 1--4, and more particularly the way diz-sɛt % was:\textipa{diz-sEt}
 and vɛ̃t-sɛt % was:\textipa{vẼt-sEt}
 are obtained with d1.\lplus{} diz % was:\textipa{diz}
 and d2.\lplus{} vɛ̃t % was:\textipa{vẼt}
.

\begin{figure}
\begin{tabular}{ccccccc}
\Tree 	[ .Digit2
			[ .u1 un ]
		]
&
\Tree 	[ .Digit2
			[ .u7 sept ]
		]
&
\Tree 	[ .Digit2
			[ .d1 dix ]
		]
&
\Tree 	[ .Digit2
			[ .uv1 onze ]
		]
&
\Tree 	[ .Digit2
			[ .DixP
				[ .d1.\lplus{} dix ]
				[ .u7 sept ]
			]
		]
&
\Tree 	[ .Digit2
			[ .d2 vingt ]
		]
&
\Tree 	[ .Digit2
			[ .DixP
				[ .d2.\lplus{} vingt ]
				[ .u7 sept ]
			]
		]
\end{tabular}
\caption{Phrase structures for 1, 7, 10, 11, 17, 20, 27}\label{rules1-4}
\end{figure}

Figure \ref{rules60} shows how \emph{et onze} \trad{\& 11} and intermediary compounds \is{compounding} such as \emph{dix-sept} \tradnum{17} are combined with \emph{soixante} \tradnum{60}.

\begin{figure}
\begin{tabular}{ccc}
\Tree 	[ .Digit2
			[ .DixP
				[ .d6 soixante ]
				[ .d1 dix ]
			]
		]
&
\Tree 	[ .Digit2
			[ .DixP
				[ .d6 soixante ]
				[ .Et11
					[ .et et ]
					[ .uv1 onze ]
				]
			]
		]
&
\Tree 	[ .Digit2
			[ .DixP
				[ .d6 soixante ]
				[ .Dix1P
					[ .d1.\lplus{} dix ]
					[ .u7 sept ]
				]
			]
		]
\end{tabular}
\caption{Phrase structures for 70, 71, 77}\label{rules60}
\end{figure}

Finally, Figure \ref{rules80} displays the combinations involving the \emph{quatre-vingt} intermediary compound. When Dix8X is formed, the linking consonant of \emph{vingt} is changed from t % was:\textipa{t}
 to z, % was:\textipa{z}
but when the Dix8X is itself combined with another element by means of rule 9, its \lmoins{} form is selected rendering the previous change invisible. Thus we obtain the \lplus{} form katʁə-vɛ̃z % was:\textipa{katK@-vẼz}
 for \emph{quatre-vingts} \tradnum{80} and the forms katʁə-vɛ̃-sɛt % was:\textipa{katK@-vẼ-sEt}
 and katʁə-vɛ̃-diz-sɛt % was:\textipa{katK@-vẼ-diz-sEt}
 for \emph{quatre-vingt-sept} \tradnum{87} and \emph{quatre-vingt-dix-sept} \tradnum{97}.

\begin{figure}
\begin{tabular}{ccc}
\Tree 	[ .Digit2
			[ .DixP
				[ .Dix8X [ .u4 quatre ] [ .d2.z % was:\textipa{z}
 vingt ] ]
			]
		]
&
\Tree 	[ .Digit2
			[ .DixP
				[ .Dix8X.\lmoins{} [ .u4 quatre ] [ .d2.z % was:\textipa{z}
 vingt ] ]
				[ .u7 sept ]
			]
		]
&
\Tree 	[ .Digit2
			[ .DixP
				[ .Dix8X.\lmoins{} [ .u4 quatre ] [ .d2.z % was:\textipa{z}
 vingt ] ]
				[ .Dix1P [ .d1 dix ] [ .u7 sept ] ]
			]
		]
\end{tabular}
\caption{Phrase structures for 80, 87, 97}\label{rules80}
\end{figure}
Even though we provide rules for all Digit2 cardinals \is{cardinals} in Table \ref{2D-Rules}, most of these compounds \is{compounding} are probably lexicalized. The rules are like redundancy generalizations \emph{à la} \cite{Lieber82} or \cite{Koenig99}, stating observable regularities in existing lexemes.
% these generalization might not be part of the grammar
% redundancy rules à la Lieber

\subsubsection{Numerical cardinals}

With most of the idiosyncrasies residing below 100, the fragment in Table \ref{numRules}\movedfootnote{We found no critical data for or against adding a linking z % was:\textipa{z}
    to the \lplus{} form of multiplied \emph{million} and \emph{milliard}, rules 15 and 17.} for the composition \is{compounding} of the higher combinations is simpler. It breaks the compounding \is{compounding} into four levels corresponding to the counting units \emph{cent} \tradnum{100}, \emph{mille} \tradnum{1000}, \emph{million} \trad{million}, and \emph{milliard} \trad{billion}. Each level is composed of two rules, one to multiply the unit level and one to add the units from the level below.

\begin{savenotes}
\begin{table}
\begin{tabular}{rll}
\lsptoprule
& Rule & Comment \\
\midrule 
\nex[11] &CentX \fldr{} u/u4/u7.\lmoins{} Cent.z % was:\textipa{z}
 & hundreds\\
\nex& CentP \fldr{} CentX/Cent.\lmoins{} (Digit2) & adding the Digit2\\
\nex& MilleX \fldr{} CentP/Digit2P.\lmoins{} Mille & thousands\\
\nex& MilleP \fldr{} MilleX/Mille (CentP/Digit2) & adding the hundreds\\
\nex & MionX \fldr{} CentP/Digit2.\lmoins{} Mion.z % was:\textipa{z}
 & millions\\
\nex& MionP \fldr{} MionX.\lmoins{} (MilleP) & adding the thousands\\
\nex& MiardX \fldr{} CentP/Digit2.\lmoins{} Miard.z % was:\textipa{z}
 & billions\\
\nex&MiardP \fldr{} MiardX.\lmoins{} (MionP)&  adding the millions \\
\lspbottomrule
\end{tabular}
\caption{Syntagmatic rules for 3-digit+ cardinals}
\label{numRules}
\end{table}
\end{savenotes}

For example, rule 11 assembles the multiples of \emph{cent} \tradnum{100} and rule 12 adds the units from the level Digit2.\footnote{These two rules could be modified to generate the 11 to 19 multiples of \emph{cent} (e.g. \emph{dix-huit cents} \tradnum{1800}). The rest would also have to be adapted to avoid the generation of aberrations such as \emph{*un million dix-huit cents mille} \tradnum{2800000}.} In rules 12, 16 and 18, the selection of the \lmoins{} form\footnote{To be more precise, rules 16 and 18 select the \textsc{m}.\lmoins{} form (i.e œ̃ % was:\textipa{œ̃}
 for \emph{un}).} happens only in the presence of the optional second term.

Figure \ref{numStruct} shows how the two sets of rules combine in the analysis of numerical cardinals \is{cardinals} in general.

\begin{figure}
\caption{Phrase structure for 600 and 697}\label{numStruct}
\begin{tabular}{cc}
\Tree [ .CentP
		[ .CentX
			[ .u.\lmoins{} six ]
			[ .Cent.z % was:\textipa{z}
 cent ]
		]
	]
&
\Tree [ .CentP
		[ .CentX.\lmoins{}
			[ .u.\lmoins{} six ]
			[ .Cent.z % was:\textipa{z}
 cent ]
		]
		[ .Digit2
			[ .DixP
				[ .Dix8X.\lmoins{} [ .u4 quatre ] [ .d2.z % was:\textipa{z}
 vingt ] ]
				[ .Dix1P [ .d1.\lplus{} dix ] [ .u7 sept ] ]
			]
		]
	]
\end{tabular}
\end{figure}

The analysis presented here relies on 26 combination elements, the 23 in \refp{simpleCardinals} plus \emph{et}, \emph{million} and \emph{milliard}. All numerical cardinals\is{cardinals}, including the simple ones, are derived from these elements. So, on the one hand, cardinal elements belong to special categories in the lexicon while, on the other hand, all numerical cardinals, including the simple ones, are CARDs derived from cardinal elements.
%\FloatBarrier



\section{French cardinals: Paradigm?}\label{Sec3}

%In this section, we combine our previous observations about gender, liaison and compounding to propose a uniform paradigm and an analysis allowing to fill the cells with the appropriate forms and associate each numerical cardinal with its proper syntactic frame.
In this section, we propose an analysis for a uniform paradigm \is{paradigm} of simple and complex cardinals. The analysis combines the observations about gender, liaison \is{liaison (French)} and compounding \is{compounding} to (i) give a set of rules that fills the cells of the paradigm \is{paradigm} with the appropriate forms and (ii) associate each numerical cardinal with its proper syntactic frame.

As lexemes belonging to the CARD category, \ili{French} cardinals, simple and complex, undergo inflection \is{inflection} with a paradigm \is{paradigm} based on two features:
\begin{itemize}
\item \textsc{liaison}: \lmoins, \lpause, \lplus
\item \textsc{gender}: \textsc{m, f}
\end{itemize}
This results in the six-cell paradigm \is{paradigm} exemplified in Table \ref{uniformParadigm} with simple cardinals.

\begin{table}[hbp]
\begin{tabular}{cccc}
\lsptoprule
\begin{tabular}{rcc}
\tradnum{1} & \textsc{m} & \textsc{f} \\
\midrule
\lmoins & œ̃ % was:\textipa{œ̃}
&yn % was:\textipa{yn}
 \\
\lpause & œ̃ % was:\textipa{œ̃}
&yn % was:\textipa{yn}
 \\
\lplus & œ̃n % was:\textipa{œ̃n}
&yn % was:\textipa{yn}
 \\
\end{tabular}
&
\begin{tabular}{rcc}
\tradnum{2}& \textsc{m} & \textsc{f} \\
\midrule
\lmoins & dø % was:\textipa{dø}
&dø % was:\textipa{dø}
 \\
\lpause &dø % was:\textipa{dø}
 &dø % was:\textipa{dø}
 \\
\lplus & døz % was:\textipa{døz}
  & døz % was:\textipa{døz}
\\
\end{tabular}
&
\begin{tabular}{rcc}
\tradnum{4}& \textsc{m} & \textsc{f} \\
\midrule
\lmoins & katʁ % was:\textipa{katK}
&katʁ % was:\textipa{katK}
 \\
\lpause &katʁ % was:\textipa{katK}
 &katʁ % was:\textipa{katK}
 \\
\lplus & katʁ % was:\textipa{katK}
  & katʁ % was:\textipa{katK}
\\
\end{tabular}
&
\begin{tabular}{rcc}
\tradnum{6}& \textsc{m} & \textsc{f} \\
\midrule
\lmoins & si % was:\textipa{si}
&si % was:\textipa{si}
 \\
\lpause &sis % was:\textipa{sis}
 &sis % was:\textipa{sis}
 \\
\lplus & siz % was:\textipa{siz}
  & siz % was:\textipa{siz}
\\
\end{tabular}\\
\lspbottomrule
\end{tabular}
\caption{Uniform paradigm of cardinals}
\label{uniformParadigm}
\end{table}%

The paradigm \is{paradigm} of complex cardinals \is{cardinals} follows the pattern of the rightmost element in the compound. For example, in Table \ref{rightEdge}, \textsc{trente-et-un}, \textsc{quatre-vingt-un} and \textsc{cent-un} share the pattern of \textsc{un}, and \textsc{trente-six}, \textsc{quatre-vingt-six} and \textsc{cent-six} inflect like \textsc{six}.

\begin{table}[bp]

		\resizebox{0.95\linewidth}{!}{
\begin{tabular}{ccc}
\lsptoprule
\begin{tabular}{rcc}
\tradnum{31}& \textsc{m} & \textsc{f} \\
\midrule
\lmoins & tʁɑ̃teœ̃ % was:\textipa{tKÃteœ̃}
&tʁɑ̃teyn % was:\textipa{tKÃteyn}
 \\
\lpause &tʁɑ̃teœ̃ % was:\textipa{tKÃteœ̃}
 &tʁɑ̃teyn % was:\textipa{tKÃteyn}
 \\
\lplus & tʁɑ̃teœ̃n % was:\textipa{tKÃteœ̃n}
  & tʁɑ̃teyn % was:\textipa{tKÃteyn}
 \\
\end{tabular}
&
\begin{tabular}{rcc}
\tradnum{81}& \textsc{m} & \textsc{f} \\
\midrule
\lmoins & katʁəvɛ̃œ̃ % was:\textipa{katK@vẼœ̃}
&katʁəvɛ̃yn % was:\textipa{katK@vẼyn}
 \\
\lpause &katʁəvɛ̃œ̃ % was:\textipa{katK@vẼœ̃}
 &katʁəvɛ̃yn % was:\textipa{katK@vẼyn}
 \\
\lplus & katʁəvɛ̃œ̃n % was:\textipa{katK@vẼœ̃n}
  & katʁəvɛ̃yn % was:\textipa{katK@vẼyn}
 \\
\end{tabular}
&
\begin{tabular}{rcc}
\tradnum{101}& \textsc{m} & \textsc{f} \\
\midrule
\lmoins & sɑ̃œ̃ % was:\textipa{sÃœ̃}
&sɑ̃yn % was:\textipa{sÃyn}
 \\
\lpause & sɑ̃œ̃ % was:\textipa{sÃœ̃}
&sɑ̃yn % was:\textipa{sÃyn}
 \\
\lplus & sɑ̃œ̃n % was:\textipa{sÃœ̃n}
&sɑ̃yn % was:\textipa{sÃyn}
 \\
\end{tabular}
\\
\tablevspace
\begin{tabular}{rcc}
\tradnum{36}& \textsc{m} & \textsc{f} \\
\midrule
\lmoins & tʁɑ̃tsi % was:\textipa{tKÃtsi}
&tʁɑ̃tsi % was:\textipa{tKÃtsi}
 \\
\lpause &tʁɑ̃tsis % was:\textipa{tKÃtsis}
 &tʁɑ̃tsis % was:\textipa{tKÃtsis}
 \\
\lplus & tʁɑ̃tsiz % was:\textipa{tKÃtsiz}
  & tʁɑ̃tsiz % was:\textipa{tKÃtsiz}
\\
\end{tabular}
&
\begin{tabular}{rcc}
\tradnum{86}& \textsc{m} & \textsc{f} \\
\midrule
\lmoins & katʁəvɛ̃si % was:\textipa{katK@vẼsi}
&katʁəvɛ̃si % was:\textipa{katK@vẼsi}
 \\
\lpause &katʁəvɛ̃sis % was:\textipa{katK@vẼsis}
 &katʁəvɛ̃sis % was:\textipa{katK@vẼsis}
 \\
\lplus & katʁəvɛ̃siz % was:\textipa{katK@vẼsiz}
  & katʁəvɛ̃siz % was:\textipa{katK@vẼsiz}
\\
\end{tabular}
&
\begin{tabular}{rcc}
\trad{106}& \textsc{m} & \textsc{f} \\
\midrule
\lmoins & sɑ̃si % was:\textipa{sÃsi}
&sɑ̃si % was:\textipa{sÃsi}
 \\
\lpause &sɑ̃sis % was:\textipa{sÃsis}
 &sɑ̃sis % was:\textipa{sÃsis}
 \\
\lplus & sɑ̃siz % was:\textipa{sÃsiz}
  & sɑ̃siz % was:\textipa{sÃsiz}
\\
\end{tabular}
\\
\lspbottomrule
\end{tabular}}
\caption{Inflection on the Right Edge}
\label{rightEdge}
\end{table}%

The only exception are \emph{vingt} \tradnum{20} and \emph{cent} \tradnum{100}, which change their linking consonant from t % was:\textipa{t}
 to z % was:\textipa{z}
 in rules 7 and 11 (p. \pageref{2D-Rules} \& p. \pageref{numRules}).
\begin{table}

\begin{tabular}{cc}
\lsptoprule
\begin{tabular}{rcc}
20& \textsc{m} & \textsc{f} \\
\midrule
\lmoins & vɛ̃ % was:\textipa{vẼ}
&vɛ̃ % was:\textipa{vẼ}
 \\
\lpause &vɛ̃ % was:\textipa{vẼ}
 &vɛ̃ % was:\textipa{vẼ}
 \\
\lplus & vɛ̃t % was:\textipa{vẼt}
  & vɛ̃t % was:\textipa{vẼt}
 \\
\end{tabular}
&
\begin{tabular}{rcc}
80& \textsc{m} & \textsc{f} \\
\midrule
\lmoins & katʁəvɛ̃ % was:\textipa{katK@vẼ}
&katʁəvɛ̃ % was:\textipa{katK@vẼ}
 \\
\lpause &katʁəvɛ̃ % was:\textipa{katK@vẼ}
 &katʁəvɛ̃ % was:\textipa{katK@vẼ}
 \\
\lplus & katʁəvɛ̃z % was:\textipa{katK@vẼz}
  & katʁəvɛ̃z % was:\textipa{katK@vẼz}
\\
\end{tabular}\\
\tablevspace
\begin{tabular}{rcc}
100& \textsc{m} & \textsc{f} \\
\midrule
\lmoins & sɑ̃ % was:\textipa{sÃ}
&sɑ̃ % was:\textipa{sÃ}
 \\
\lpause &sɑ̃ % was:\textipa{sÃ}
 &sɑ̃ % was:\textipa{sÃ}
 \\
\lplus & sɑ̃t % was:\textipa{sÃt}
  & sɑ̃t % was:\textipa{sÃt}
 \\
\end{tabular}
&
\begin{tabular}{rcc}
200& \textsc{m} & \textsc{f} \\
\midrule
\lmoins & døsɑ̃ % was:\textipa{døsÃ}
&døsɑ̃ % was:\textipa{døsÃ}
 \\
\lpause &døsɑ̃ % was:\textipa{døsÃ}
 &døsɑ̃ % was:\textipa{døsÃ}
 \\
\lplus & døsɑ̃z % was:\textipa{døsÃz}
  & døsɑ̃z % was:\textipa{døsÃz}
\\
\end{tabular}\\
\lspbottomrule
\end{tabular}
\caption{Number morphology on the Right Edge}
\label{numMor}
\end{table}%
%The paradigm view shown in Table \ref{uniformParadigm} contain 6 cells but all cardinals belong to the same 5 types defined in Table \ref{typesCardinals} for simple cardinals.

Not only do the forms of complex cardinals \is{cardinals} depend on the element on the right edge, but their controlling properties are also derived from the right edge element. This distinguishes cardinals \is{cardinals} ending in \emph{million/milliard} from the others as seen below in \refp{cardOpposition} and in \refp{numOpposition} (p. \pageref{numOpposition}).
\ea\label{cardOpposition}
	\ea\label{cardN} un milliard trois cents millions cinq cent \emph{mille} chinois/*de chinois \\ \trad{\numprint{1300500000} Chinese}
	\ex\label{cardDeN} un milliard trois cent \emph{millions} *chinois/de chinois \\ \trad{\numprint{1300000000} Chinese}
	\z
\z
Cardinals ending with \emph{million} \trad{million} or \emph{milliard} \trad{billion} impose a \emph{de}-NP structure. We use a \textsc{de} feature to encode this difference: \textsc{de} $= +$ for \refp{cardDeN}, \textsc{de} $= -$ for \refp{cardN}.

Both the \textsc{de} feature and the inflectional \is{inflection} paradigm \is{paradigm} of compound \is{compounding} cardinals \is{cardinals} can be constructed using the Right Edge mechanism introduced by \citet{Tseng03} and \citet{BoBoTse04} to model \ili{French} phrasal affixes (\emph{à} \trad{at}, \emph{de} \trad{of}) and liaison. The proposed mechanism ensures that the properties of the rightmost element are propagated to the top of the construction by copying the relevant features of the last component to its parent node at every level of compounding \is{compounding} represented by the arrows in Figure \ref{cardP}. Rules combining two elements get a specific form from the left paradigm \is{paradigm} and prefix it to the paradigm \is{paradigm} on the right.

\begin{figure} %\protectsɛ̃kmiljɔ̃dømilsisɑ̃katʁəvɛ̃tʁwa % was:\textipa{sẼkmiljÕdømilsisÃkatK@vẼtKwa}
% cinq million deux mille six cent quatre-vingt trois
\begin{center}
%\Tree
%	[ .MionP
%		[ .MionX
%			[ .CentP
%				[ .CentX [ .u cinq ] [ .Cent cent ] ]
%				[ .Digit2 [ .u six ] ]
%			]
%			[ .Mion million ]
%		]
%		[ .MilleP
%			[ .MilleX
%	        			[ .CentP
%        					[ .CentX [ .u deux ] [ .Cent cent ] ]
%        					[ .Digit2
%						[ .DixP
%							[ .d trente ]
%							[ .u trois ]
%						]
%					]
%        				]
%				[ .Mille mille ]
%			]
%			[ .CentP
%   				[ .CentX [ .u six ] [ .Cent cent ] ]
%				[ .Digit2
%					[ .DixP
%						[ .d6 soixante ]
%						[ .Dix1P [ .d1 dix ] [ .u7 sept ] ]
%					]
%				]
%			]
%		]
%	]
\resizebox{0.8\linewidth}{!}{
\begin{forest}
for tree={s sep=0em}
	[\textbf{MionP}
		[MionX.\lmoins
			[CentP.\lmoins
				[CentX.\lmoins [u.\lmoins [{cinq}, edge=<-] ] [{Cent.z % was:\textipa{z}
},edge=<- [{cent},edge=<-] ] ]
				[{Digit2},edge=<- [{u},edge=<- [{six}, edge=<-] ] ]
			]
			[{Mion.z % was:\textipa{z}
},edge=<- [{million},edge=<-] ]
		]
		[{\textbf{MilleP}},edge={<-,thick}
			[MilleX
        				[Digit2.\lmoins
					[{DixP},edge=<-
						[d.\lplus [{trente},edge=<-] ]
						[{u},edge=<- [{trois},edge=<-] ]
					]
				]
				[{Mille},edge=<- [{mille},edge=<-] ]
			]
			[{\textbf{CentP}},edge={<-,thick}
   				[CentX.\lmoins [u.\lmoins [{six},edge=<-] ] [{Cent.z % was:\textipa{z}
},edge=<- [{cent},edge=<-] ] ]
				[{\textbf{Digit2}},edge={<-,thick}
					[{\textbf{DixP}},edge={<-,thick}
						[d6 [{soixante},edge=<-] ]
						[{\textbf{Dix1P}},edge={<-,thick} [d1.\lplus [{dix},edge=<-] ] [{\textbf{u7}},edge={<-,thick} [{\textbf{sept}},edge={<-,thick}] ] ]
					]
				]
			]
		]
	]
\end{forest}}
\end{center}
\caption{Phrase structure for \numprint{506033677}}
\label{cardP}
\end{figure}

For example, on the right side, in \refp{dixSept}, the Dix1P prefixes the \textsc{m}.\lplus{} form of \textsc{dix} diz % was:\textipa{diz}
 to all forms of \textsc{sept} and carries the controlling property \textsc{de} $= -$ from \textsc{sept}. In \refp{sixCents}, the combination selects the \textsc{m}.\lmoins{} form of \textsc{six} si % was:\textipa{si}
 and combines it with the modified paradigm \is{paradigm} of \textsc{cent} where the linking consonant of the \lplus{} forms t % was:\textipa{t}
 has been changed to z % was:\textipa{z}
.
\ea
\begin{exe} 
 \begin{xlist}
\exbox{\label{dixSept}
\begin{tabular}{rcc}
\multicolumn{3}{c}{\tradnum{10}}\\
& \textsc{m} & \textsc{f} \\
%\lsptoprule
\lmoins & di % was:\textipa{di}
&di % was:\textipa{di}
 \\
\lpause &dis % was:\textipa{dis}
 &dis % was:\textipa{dis}
 \\
\lplus & \textbf{diz % was:\textipa{diz}
}  & diz % was:\textipa{diz}
\\
\midrule
\multicolumn{3}{l}{\textsc{de} $= -$}\\
\end{tabular}
\emph{
\begin{tabular}{rcc}
\multicolumn{3}{c}{\textnormal{\tradnum{7}}}\\
& \textnormal{\textsc{m}} & \textnormal{\textsc{f}}\\
%\lsptoprule
\lmoins & sɛt % was:\textipa{sEt}
&sɛt % was:\textipa{sEt}
 \\
\lpause &sɛt % was:\textipa{sEt}
 &sɛt % was:\textipa{sEt}
 \\
\lplus & sɛt % was:\textipa{sEt}
  & sɛt % was:\textipa{sEt}
\\
\midrule
\multicolumn{3}{l}{\textsc{de} $= -$}\\
\end{tabular}
}
\fldr{}
\begin{tabular}{rcc}
\multicolumn{3}{c}{\tradnum{17}} \\
& \textsc{m} & \textsc{f} \\
%\lsptoprule
\lmoins & dizsɛt % was:\textipa{dizsEt}
&dizsɛt % was:\textipa{dizsEt}
 \\
\lpause &dizsɛt % was:\textipa{dizsEt}
 &dizsɛt % was:\textipa{dizsEt}
 \\
\lplus & dizsɛt % was:\textipa{dizsEt}
  & dizsɛt % was:\textipa{dizsEt}
\\
\midrule
\multicolumn{3}{l}{\textsc{de} $= -$}\\
\end{tabular}
}
\medskip
\exbox{\label{sixCents}
\begin{tabular}{rcc}
\multicolumn{3}{c}{\tradnum{6}}\\
& \textsc{m} & \textsc{f} \\
%\lsptoprule
\lmoins & \textbf{si % was:\textipa{si}
}&si % was:\textipa{si}
 \\
\lpause &sis % was:\textipa{sis}
 &sis % was:\textipa{sis}
 \\
\lplus & siz % was:\textipa{siz}
  & siz % was:\textipa{siz}
\\
\midrule
\multicolumn{3}{l}{\textsc{de} $= -$}\\
\end{tabular}
\emph{
\begin{tabular}{rcc}
\multicolumn{3}{c}{\textnormal{\tradnum{100}}}\\
& \textnormal{\textsc{m}} & \textnormal{\textsc{f}}\\
%\lsptoprule
\lmoins & sɑ̃ % was:\textipa{sÃ}
&sɑ̃ % was:\textipa{sÃ}
 \\
\lpause &sɑ̃ % was:\textipa{sÃ}
 &sɑ̃ % was:\textipa{sÃ}
 \\
\lplus & sɑ̃z % was:\textipa{sÃz}
  & sɑ̃z % was:\textipa{sÃz}
\\
\midrule
\multicolumn{3}{l}{\textsc{de} $= -$}\\
\end{tabular}
}
\fldr{}
\begin{tabular}{rcc}
\multicolumn{3}{c}{\tradnum{600}} \\
& \textsc{m} & \textsc{f} \\
%\lsptoprule
\lmoins & sisɑ̃ % was:\textipa{sisÃ}
&sisɑ̃ % was:\textipa{sisÃ}
 \\
\lpause &sisɑ̃ % was:\textipa{sisÃ}
 &sisɑ̃ % was:\textipa{sisÃ}
 \\
\lplus & sisɑ̃z % was:\textipa{sisÃz}
  & sisɑ̃z % was:\textipa{sisÃz}
\\
\midrule
\multicolumn{3}{l}{\textsc{de} $= -$}\\
\end{tabular}\\
}
\end{xlist}
\end{exe}
\z

%\newpage 
The percolations proceed level by level, and yield a structure at the top with a full paradigm \is{paradigm} and the appropriate value of the \textsc{de} feature.%
\footnote{\begin{tabular}[t]{@{}r@{\hspace{3pt}}c@{\hspace{3pt}}c@{}}
\multicolumn{3}{c}{\tradnum{506033677}} \\
& \textsc{m} & \textsc{f} \\
%\lsptoprule
\lmoins & sɛ̃ksɑ̃similjɔ̃trɑ̃ttʁwamilsisɑ̃swasɑ̃tdizsɛt % was:\textipa{sẼksÃsimiljÕtrÃttKwamilsisÃswasÃtdizsEt}
&sɛ̃ksɑ̃similjɔ̃trɑ̃ttʁwamilsisɑ̃swasɑ̃tdizsɛt % was:\textipa{sẼksÃsimiljÕtrÃttKwamilsisÃswasÃtdizsEt}
 \\
\lpause &sɛ̃ksɑ̃similjɔ̃trɑ̃ttʁwamilsisɑ̃swasɑ̃tdizsɛt % was:\textipa{sẼksÃsimiljÕtrÃttKwamilsisÃswasÃtdizsEt}
 &sɛ̃ksɑ̃similjɔ̃trɑ̃ttʁwamilsisɑ̃swasɑ̃tdizsɛt % was:\textipa{sẼksÃsimiljÕtrÃttKwamilsisÃswasÃtdizsEt}
 \\
\lplus & sɛ̃ksɑ̃similjɔ̃trɑ̃ttʁwamilsisɑ̃swasɑ̃tdizsɛt % was:\textipa{sẼksÃsimiljÕtrÃttKwamilsisÃswasÃtdizsEt}
  & sɛ̃ksɑ̃similjɔ̃trɑ̃ttʁwamilsisɑ̃swasɑ̃tdizsɛt % was:\textipa{sẼksÃsimiljÕtrÃttKwamilsisÃswasÃtdizsEt}
\\
\midrule
\multicolumn{3}{@{}l}{\textsc{de} $= -$}\\
\end{tabular}\smallskip}


The model outlined here relies on the propagation of ready-made elementary para\-digms \is{paradigm} via a phrase structure grammar rather than rules of exponence or referral based on the inflectional \is{inflection} features of the different cardinals \is{cardinals} as is common with Word and Paradigm syntagmatic frameworks\footnote{See \cite{BoSchal16} for a typology of views on inflectional \is{inflection} paradigms \is{paradigm} in different theories.} such as A-Morphous Morphology \citep{Anderson92}, Paradigm Function Morphology \citep{Stump01} or the Information-Based Model of \cite{Bonami13d}. It is more in line with paradigm-oriented models like Network Morphology \citep{CorbettFraser93}.
%On the right edge, \emph{sept} sends its control and inflectional properties to the top following the bold arrows. At the same time,

%\Tree [ .CentP
%		[ .CentX
%			[ .Uni100 six ]
%			[ .Cent cent ]
%		]
%		\qroof{quatre vingt trois}.Digit2
%	]

%In a way, the solution is reminiscent of the construction of irregular adjectives advanced by \cite{BoBo05} where the derivation itself builds the inflectional paradigm to distinguish deverbal adjectives in -œʁ/ʁis % was:\textipa{-œK/Kis}
% from their counterpart in -œʁ/øz % was:\textipa{-œK/øz}
%, only on a larger scale.

% A derived paradigm
%% Card compounding: a syntagmatic analysis
%% Right Edge Inflectional Head: a uniform paradigm


% Remaining problems
%% Lexical integrity and Coordination
%%% Lexical integrity and Compounds => same problem with Cardinals
\section{Conclusion}\label{Sec4}
In this chapter, we set out to discuss the place of cardinals \is{cardinals} in \ili{French} morphology with a focus on their status as lexemes, their categories and their inflectional \is{inflection} paradigms. Taking into account the number of phonological idiosyncrasies in the formation of \ili{French} cardinals, we argued that they should be considered as lexemes. Following \cite{Saulnier08, Saulnier10, FradinSaulnier09}, we examined both their morphotactic properties and their syntactic distribution and concluded that they belonged to a morphosyntactic category CARD inside the determiners. We showed that there are two types of cardinals \is{cardinals} regarding the way they associate with nouns, the direct type like \textsc{cinquante-deux} (\emph{cinquante-deux années}) and the indirect type like \textsc{un-milliard-trois-cents-millions} (\emph{un milliard trois cents millions d'années}).  This distinction making no difference on the outside of the NP, we analyzed them as compounds \is{compounding} based on 26 simple elements\footnote{Nothing would prevent \ili{French} from using more elements. In fact, it has been proposed since the 15th century to expand the counting system by including \emph{billion}, \emph{trillion}, \emph{quadrillion}, etc. \citep[see][147--151 for an overview of the proposals]{Saulnier10}.} using a phrase structure grammar, even though the cardinals \is{cardinals} below 100 are probably lexicalized. Our compounding \is{compounding} mechanism propagates the inflectional \is{inflection} and syntactic properties of the rightmost component to the entire compound \is{compounding} to create its paradigm \is{paradigm} and percolate its type (\textsc{de} = $\pm$).

The type of compounds \is{compounding} we advocate for is different from the usual two-compo\-nent ones. It expands the ternary compounds \is{compounding} described in the biomedical domain by \cite{Namer05} to higher levels of composition. The extended compounding \is{compounding} mechanism allows to generate all numerical cardinals \is{cardinals} as CARD without having to cast them into the different subcategories that would be needed to break the compounding \is{compounding} process into binary operations. It does not presuppose that complex cardinals \is{cardinals} are lexicalised but only that they can be created online by morphology, as [+morphological, -lexical] compounds \is{compounding} in the sense of \cite{GaetaRicca09}.

The model outlined here should be integrated with the formal analysis of \cite{BoBoTse04} of liaison \is{liaison (French)} in HPSG \citep{Pollard94}. It would be interesting to examine data from the cardinals \is{cardinals} in other languages to parallel the work of \cite{Stump10} on the ordinals\footnote{\ili{French} ordinals are derived from their cardinal counterparts by \emph{-ième} suffixation as proposed by \cite[p. 228]{Stump10} with the notable exceptions of \emph{millionième} and \emph{milliardième} which drop the \emph{un} from \emph{un million} and \emph{un milliard}.} and from the composition \is{compounding} of the decimals and its interference with the integers.\footnote{Many ill-formed cardinals \is{cardinals} are in fact well-formed decimals. For example, \emph{cinq vingt} is automatically understood as \tradnum{5.20}. Furthermore, \emph{un million un}  \tradnum{1000001}, when not followed by a counted noun, is usually perceived as \tradnum{1100000} with \emph{million} interpreted as a measure unit.}

\subsection{Remaining questions}

Cardinal coordinations do not respect lexical integrity. Examples like \refp{multCoord} are common, and
% but seem to occur mostly with the leftmost element of multiplying combinations.
even stranger coordinations appear with ordinals where the first ordinal is realised as a gender-agreeing cardinal as in \refp{ordCoord}.
\ea\label{cardCoord}
	\ea\label{multCoord} Quelques \emph{soixante-dix} ou \emph{quatre-vingt} mille personnages sont passés à la trappe, 35 000 sont à l'ombre.\footnote{\trad{Some seventy or eighty thousand persons have disappeared, 35,000 are in jail.}\\\url{http://plumenclume.org/blog/173-erdogan-consolide-son-emprise-par-israel-adam-shamir}}
	\ex\label{ordCoord} Ses débuts, il les fit, dans sa ville natale, au début du siècle dernier dans sa vingt-et-\emph{une} ou vingt-deux\emph{ième} année.\footnote{\trad{His debut, he made at the beginning of last century, when he was in his twenty-first or twenty-second year.}\\\url{http://www.www.dutempsdescerisesauxfeuillesmortes.net/fiches_bio/darbon/darbon.htm}}
	\z
\z

\cite{Saulnier08} observes that \emph{quelques} follows the syntactic distribution of CARDs and derives the \emph{quelquième} ordinal found in \emph{trente et quelquième} `thirty-somethingth'.\footnote{\cite{FradinSaulnier09} also mention \emph{combien/combientième} `how many', \emph{quel/quellième} `which' as potential cardinal/ordinal pairs (\emph{quantième/tantième} look more like fractions than ordinals).}

The arguments developed in this chapter for a morphological analysis of the composition \is{compounding} of cardinals \is{cardinals} rely on the idiosyncrasies of complex cardinals \is{cardinals} below 100. To capture the phenomenon in \refp{cardCoord}, it would be possible to propose a morphological analysis of lower complex cardinals \is{cardinals} as compounds \is{compounding} and lexemes, while still allowing syntactic composition\is{compounding} for higher complex cardinals\is{cardinals}.

\section*{Acknowledgements}

I wish to thank Patricia Cabredo Hofherr, Georgette Dal, Olivier Bonami and Gauvain Schalchli for their helpful comments as well as the countless millionaires present at various morphology conferences for answering so many questions about their first millions, and last but not least the Coach for bringing our community together and making it count.

{\sloppy
    \printbibliography[heading=subbibliography,notkeyword=this]
}

\end{document}
\endinput
