\documentclass[output=paper]{langsci/langscibook}
\ChapterDOI{10.5281/zenodo.1407023}
\title{Word formation in LFG-based layered morphology and two-level semantics} 
\shorttitlerunninghead{Layered Morphology and Two-Level Semantics}
\author{Christoph Schwarze}

\abstract{This article treats the problem of how the semantics\is{word-formation!semantics of} of word formation can be accounted for in terms of rules and representations. A comprehensive model of multilayered, \textsc{lfg}-based morphology is proposed.
It comprises four layers of representation: phonology, constituent structure, functional feature structure and lexical semantics. The meaning of derived words is treated in the framework of two-level semantics. It is assumed that rules of word formation derive underspecified semantic forms\is{semantic form}, parting from which the actual meanings are construed by recourse to conceptual structure\is{conceptual structure}. The model is illustrated on the basis of three morphological processes: \ili{French}
é-prefixation\is{prefixation!in \emph{é}-}, \ili{Italian} denominal verbs\is{verb!denominal} of removal, and noun-to-verb conversion\is{conversion!noun-to-verb} in \ili{French}. The analyses of é-prefixation\is{prefixation!in \emph{é}-} and of verbs\is{verb!denominal} of removal are taken from the literature; the study on noun-to-verb conversion\is{conversion!noun-to-verb} is original work.}

\maketitle

\begin{document}
\selectlanguage{english}

\is{morphology!LFG-based layered morphology|(}
\is{two-level semantics|(}

\section{Introduction}

The hypothesis that the semantics\is{word-formation!semantics of} of word formation is an aspect of grammar assumes that the processes of word formation concern both form and meaning. However, actual work on this basis encounters considerable challenges. The data available for the study of a given process of word formation never seem to show a perfect parallelism between form and meaning: forms that stem from a given generative process often have meanings on which it seems to be impossible to form a descriptive generalization. It is the aim of this paper to show how challenges to the semantics\is{word-formation!semantics of} of word formation can be dealt with.

I will first address the question of how morphological processes and structures can comprehensively be represented. I will then present three hypotheses concerning the semantics\is{word-formation!semantics of} of word formation, namely

\begin{enumerate}
\def\labelenumi{\roman{enumi}.}
\item \begin{quote}

 The semantic output of morphological rules is underspecified. \end{quote}
\item \begin{quote}

 The meanings of the words that constitute the data arise from a sequence of steps. \end{quote}
\item \begin{quote}

 These steps are a. the morphological rule defines an underspecified semantic form\is{semantic form}, b. semantic form\is{semantic form} is turned into a specified meaning on the basis of conceptual knowledge, c. the derived word enters the lexicon, and d. the lexicalized word may have its own development, independently from morphology, and its original meaning may thus be changed and its morphological origin be obscured. \end{quote}
\end{enumerate}

\section{\textsc{lfg}-based Layered Morphology}\label{lfg-based-layered-morphology}

\textsc{lfg}-based Layered Morphology (\textsc{llm}) integrates essential properties of Construction Grammar Morphology,\footnote{See %
%Booij (2010)
\citet{Booij10}%
%Booij
%
, %
%Booij and Audring (2015)
\citet{Booij2017}% première publication en 2015, journal daté de 2017
%?Booij-Audring 
%
.} which does justice to the multi-layered nature of the lexicon, and HPSG-based morphology, which elaborates on the features that syntax receives from morphology.\footnote{For work on \ili{French}, see %
%Fradin (2005) 
\citet{Fradin2005} %
%Fradin
%
and %
%Tribout (2010a)
\citet[]{Tribout2010a}%
%Tribout
%
.}

Notice that \textsc{llm} is a model, not a theory or a hypothesis. Unlike theories and hypotheses, which can be empirically evaluated with reference to observable data, models can only be evaluated with respect to their usefulness for the progression of knowledge. This kind of usefulness cannot be measured, it can only be shown by actual work on specific phenomena. That is what I will try to do in this study.

%\subsection{An outline}\label{an-outline}

Lexicalist models of grammar commonly assume that words are linguistic objects with layered representations, phonological, syntactic and semantic. Accordingly, morphological processes operate simultaneously at various layers or levels of representation.\footnote{I fully agree with %
%Arnaugue et %
%Plénat (2014: 1)
\citet[1]{Aurnague2008}%
%Plénat
%
 when they say: \enquote{Une lexie est un n-tuplet de représentations reliées entre elles, mais relevant chacune d'un niveau linguistique (phonologique, syntaxique, sémantique, etc.) distinct. La description d'un mode de formation lexical productif suppose par conséquent que soient relevées et expliquées les régularités apparaissant à chacun de ces niveaux.}} In accordance with Lexical Functional Grammar (\textsc{lfg}) the \textsc{llm} model makes a distinction between the level of constituents, called the c-structure level, and a level of functional features, called the f-structure level.%
\footnote{\textsc{llm} was first presented in a seminar held by the author at the University of Padova in 2008 and subsequently applied to the formation of \ili{Italian} past and passive participles in %
%Schwarze (2011)
\citet{Schwarze2011}%
%Schwarze
%
.} The latter contains features concerning agreement, tense, mood, inflectional class etc. It also contains grammatical functions and, importantly,
predicate features, which are labels of lexical meanings and encode grammatical functions, the syntactic reflex of argument structure.

In addition to these two \enquote{syntactic} levels, morphological representations need to comprise a phonological level to account for non-concatenative morphological processes, like \ili{German} \emph{Umlaut}; cf.
Germ. \emph{krank} /krank/ `sick' + --\emph{lich /}lɪχ\emph{/} `ly' \textrightarrow{}
\emph{kränklich} /krɛnklɪχ/ `sickly'.

And, of course, there is a semantic level, where the lexical meanings encoded in the lexicon are represented and processed. Resuming,
morphological representations and processes are located at

\begin{itemize}
\item The level of constituent structure (the c-structure level)

\item The functional level (the f-structure level)

\item The phonological level (the p-structure level)

\item The semantic level (the s-structure level)

\end{itemize}

Unlike syntax, morphology may manipulate predicates, thus deriving new predicates.

\section{\texorpdfstring{A sample analysis: French
\emph{é-}prefixation}{A sample analysis: French é-prefixation}}\label{a-sample-analysis-french-uxe9-prefixation}

\is{prefixation!in \emph{é}-}
\il{French}

I will illustrate \textsc{llm} on the basis of %
%Namer \& Jacquey's (2003)
\citepos{NamerJacquey2003}%
%
\footnote{In a subsequent article, %
%Namer and Jacquey (2012) 
\citet{NamerJackey2012} %
%Namer-Jacquey
%
proposed a modelization of the N\textgreater{}V vs. V\textgreater{}N derivations within the framework of the Generative Lexicon.} article on the \ili{French} \emph{é-}prefixation\is{prefixation!in \emph{é}-}, which endevours to give a formalized version of the findings of %
%Aurnague et Plénat (1997)
\citet{Aurnague1997a}%
%Aurnague-Plénat
%
. In one of its modalities, \emph{é-}prefixation\is{prefixation!in \emph{é}-} turns nouns into transitive verbs\is{verb!denominal} that denote events where the referent of the base noun is distanced, removed,
or separated from the referent of the direct object, as in (\ref{ex:Schwarze:1}):\movedfootnote{Changes like adding /j/ to the stem as in \emph{épouiller} are idiosyncratic and must be accounted for in the lexicon.}
\il{French|(}
\ea\label{ex:Schwarze:1} FR.\nolistbreak
      \ea \gll \emph{é} + \emph{branche} \textrightarrow{} \emph{ébrancher (un arbre)}\\
      {} {} `branch' {} {`to prune a tree'}\\
      \ex \gll \emph{é} + \emph{feuille} \textrightarrow{} \emph{effeuiller (x)}\\
      {} {} `leaf' {} {`to strip the leaves or petals from x'}\\
      \ex \gll \emph{é} + \emph{gorge} \textrightarrow{} \emph{égorger (x)}\\
      {} {} `throat' {} {`to cut x's throat'}\\
      \ex \gll \emph{é} + \emph{pou} \textrightarrow{} \emph{épouiller (x)}\\
      {} {} `louse' {} {`to delouse x'}\\
\z\z

Moreover, as has been shown by %
%Aurnague et Plénat (1997, 2007, 2014)
\citet{Aurnague1997a,AurnaguePlenat2007,Aurnague2008}%
%Aurnague-Plénat;Aurnague-Plénat;?Aurnague-Plénat
%
,
the relation that holds between the two dissociated entities must be
\enquote{usual} and \enquote{natural},\footnote{\enquote{[\ldots{}] les dérivés en \emph{é-} expriment la dissociation [\ldots] par un agent intentionnel [\ldots] d'une relation d'attachement habituel [\ldots] créée naturellement [\ldots] et à laquelle il s'oppose [\ldots]} %
%(Aurnague et Plénat {[}2014:28{]})}
\citep[28]{Aurnague2008}.} %
 or, as \citeauthor{NamerJacquey2003} put it: 
 
 \begin{quote} [D]escribing the process consisting in clearing a tree of e.g.
the magpies (\emph{pie}) or the cats (\emph{chat}) that colonize it cannot be performed by processes referred to by the
?\emph{épier}\footnote{Not to be confounded with existing \emph{épier qu.} `to spy on someone'.} or ?\emph{échatter} impossible derived verbs\is{verb!denominal}.
\citep[2]{NamerJacquey2003}
\end{quote}

Table~\ref{tab:Schwarze:1} gives the rule that generates verbs\is{verb!denominal} like \emph{ébrancher},
\emph{effeuiller}, \emph{égorger} or \emph{épouiller} in the \textsc{llm}
notation.

\begin{table}
\begin{tabularx}{\textwidth}{lQ}
\lsptoprule

{{c-structure}} &

{[}\emph{é}{]}\textsubscript{V\_prefix} + N\textsubscript{stem} \textrightarrow{}
{[}\emph{é}N\textsubscript{stem}{]}\textsubscript{V\_stem}\\

{{f-structure}} &

\textsc{(\textuparrow{} pred1)$=$`p' \textrightarrow{} (\textuparrow{} pred2)$=$`dissociate (\textuparrow{} subj),(\textuparrow{} obj)'}\\

{{p-structure}} &

\textless{}no morphologically relevant change\textgreater{}\\
% TODO: non conforme aux guidelines, footnote dans table

{{s-structure}}&
$\ly \f{p}{y} \to \lx \ly \E{z} $\newline $\f{dissociate}{x,y,z} \et \f{natural\_relationship}{y,z} \et \f{agent}{x} \et \f{theme}{y}$ \\
\lspbottomrule
\end{tabularx}
\caption{The \textsc{llm} rule for \emph{é-}prefixation\is{prefixation!in \emph{é}-}}
\label{tab:Schwarze:1}
\end{table}

\il{French|)}


The c-structure change as formulated in \tabref{tab:Schwarze:1} should be self-explanatory, whereas a few comments on the f- and s-structure part of the rule will be useful.

\textsc{pred} is a feature attribute, whose value identifies a word's lexical meaning and argument structure. The input, \textsc{(\textuparrow{} \textsc{pred}1)=`p'},
contains a predicate variable, \emph{p}, which ranges over the nominal predicates associated with constituent N\textsubscript{stem}. The up-arrow is an abbreviation for a function that projects the feature to the dominant c-structure node. The output of the semantic change is a new predicate \textsc{pred}2, which is defined by the rule. It has two arguments,
an agent and a theme, realized as the subject and the object respectively. Notice that the prefix, in accordance with %
%Namer \&
%
%Jacquey (2003)
\citet{NamerJacquey2003}%
%Jacquey
%
, has no direct functional representation, because it has no referential meaning.\footnote{\enquote{Our purpose is \ldots{} to represent the verb class obtained by the \emph{é-}prefix derivation. To achieve this, two basic ways are provided: (i) representing the prefix itself or (ii) representing an abstract, parametrized lexical unit describing the output (verbal) class. The motivation for the first choice would be the fact that the affix can be seen as some kind of predicate, operating on and controlling two arguments, the base and the derived word, from a structural, categorial and above all semantic points of view. However, the nature itself of the affix is a counterargument: according to the morphological theory defended here, an affix does not belong to any of the major categories. In addition, we have seen that it bears no referential meaning: consequently, it is not foreseeable to modelize its semantic content, as it has no proper semantic content} %
%(Namer \& Jacquey {[}2003:3{]})
\citep{NamerJacquey2003}%
. I follow this argumentation, with the exception that not belonging to a major category does not generally imply the lack of functional or semantic information.}
As to the s-structure level, the derived predicate, `dissociate', has three semantic arguments: $x$, which is the subject and refers to the agent, $y$, which is the object and refers to the theme, and $z$, which is incorporated in the verb's meaning and refers to the entity which is dissociated from $y$ . The additional predication, repeated as (\ref{ex:Schwarze:2}), is needed to constrain the range of $y$ and $z$:

\ea\label{ex:Schwarze:2} $\f{natural\_relationship}{y,z}$
\z This part of the representation expresses the fact that the relation between $y$ and $z$ must not be a merely accidental one, as reported above.

Notice that the change in s-structure as expressed in \tabref{tab:Schwarze:1} does not predict the full actual meanings of the verbs\is{verb!derived} derived by
\emph{é}-prefixation\is{prefixation!in \emph{é}-}: the derived representation is underspecified.%
\footnote{This assumption is quite common in the literature, see the survey in %
%Tribout (2010a: 282-284)
\citet[282--284]{Tribout2010a}%
%Tribout
%
.} In the following section I will give some background for such an assumption.

\section{Two-level semantics}\label{two-level-semantics}

If we assume that lexical morphology is a generative subsystem that feeds the lexicon, then its semantics is part of lexical semantics. Now,
the fundamental question is to which extent lexical semantics is an affair of grammar. According to the conception known as two-level-semantics, lexical meaning is represented at two distinct levels, semantic form\is{semantic form} (\textsc{sf}) and conceptual structure\is{conceptual structure} (\textsc{cs}).\footnote{For a critical state-of-the-art overview, see %
%Lang \& Maienborn (2011)
\citet{LangMaienborn2011}%
%Lang-Maienborn
%
.}

Semantic form is linguistic knowledge. \textsc{sf}s \enquote{are systematically connected to, and hence covered by, lexical items and their combinatorial potential to form more complex expressions} %
%(Lang \& Maienborn {[}2011:711{]})
\citep[711]{LangMaienborn2011}%
%
. They \enquote{form an integral part of the information cluster represented by the lexical entries of a given language} %
%(Lang \& Maienborn {[}2011:711{]})
\citep[711]{LangMaienborn2011}%
%
. They are \enquote{accessibly stored in long-term memory} (ib.). They are underspecified with respect to \textsc{cs} representations %
%(Lang \& Maienborn {[}2011: 713{]})
\citep[713]{LangMaienborn2011}%
%
. And, importantly,
\textsc{sf} is the level at which two-level semantics endeavors to represent the compositionality of lexical meaning and the grammatical role of lexical decomposition %
%(Lang \& Maienborn {[}2011: 723{]})
\citep[723]{LangMaienborn2011}%
.

Returning to the semantics\is{word-formation!semantics of} of word formation, it is an aspect of grammar, as far as semantic form\is{semantic form} is concerned. Most of the characteristics of \textsc{sf} that hold for ordinary lexical semantics also hold for the semantics\is{word-formation!semantics of} of word formation, with one exception:
compositionality is not a general feature of lexical morphology. In fact, non-concatenative processes may be absolutely regular, but cannot be compositional, since compositionality presupposes concatenation.

In order to see whether the semantics of lexical morphology can reach out to phenomena that are situated beyond \textsc{sf}, let us see what two-level semantics means by conceptual structure\is{conceptual structure}.

Conceptual structure can be said to be world knowledge %
%(Lang \& Maienborn 2011:711)
\citep[711]{LangMaienborn2011}%
%
. That does not mean, however, that it has nothing to do with language, actually, it is closely related to \textsc{sf}: \textsc{cs} representations are built upon and enrich \textsc{sf} representations. Thus, semantic representations typically contain both, \textsc{cs} and \textsc{sf} features. This happens in such a way that, for the representation of a given lexical meaning, the \textsc{cs} features specify and enrich \textsc{sf} representations, thus enabling words to denote their referents.\footnote{In Lang and Maienborn's words: \enquote{\ldots{}for every linguistic expression \emph{e} in language \emph{L} there is a \textsc{cs} representation \emph{c} assignable to it via \textsc{sf}(e), but not vice versa} %
%(Lang \& Maienborn {[}2011:711{]})
\citep[711]{LangMaienborn2011}%
%
; \enquote{\ldots{} \textsc{cs} representations are taken to belong to, or at least to be rooted in, the non-linguistic mental systems based on which linguistic expressions are interpreted and related to their denotations.}}\textsuperscript{,}%
\footnote{This conception has an important consequence: if the features retrieved from \textsc{cs} are combined with or replace \textsc{sf} features, doing lexical semantics does not mean to represent the entire bulk of knowledge and beliefs that we have about the referents of the lexemes under investigation.}\textsuperscript{,}%
\footnote{As to the mental status and processing of \textsc{cs} representations, they are assumed to be \enquote{activated and compiled in working memory}, contrarily to \textsc{sf} representations, which, as has been said above, are stored in long-term memory %
%(Lang \& Maienborn {[}2011:712{]})
\citep[712]{LangMaienborn2011}%
%
. I am not sure about the mental status of \textsc{cs}: it may safely be assumed that concepts, once they are lexicalized as meaning components, are as stable as \textsc{sf}s.}

\section{A second sample analysis: Italian denominal verbs of removal}\label{a-second-sample-analysis-italian-denominal-verbs-of-removal}

\il{Italian}
\is{verb!denominal} 

It will be useful to illustrate underspecification and its resolution with an example from derivational morphology. I will briefly present the analysis of the denominal verbs\is{verb!denominal} derived by \emph{s-} prefixation\is{prefixation!in \emph{s}-} in \ili{Italian} as proposed by %
%Heusinger \& Schwarze (2006)
\citet{Heusinger2006}%
%Heusinger-Schwarze
%
.\footnote{Giuseppina %
\citet{Giuseppina2017} %
applies the %
%Heusinger \& Schwarze (2006)
\citet{Heusinger2006} 
 approach to prefixed deadjectival verbs\is{verb!deadjectival} in \ili{Italian}.}
 
The morphological process generates verb stems from noun stems by prefixing the constituent
\emph{s-} to the noun stem;\footnote{Notice that \ili{Italian} also has a V\textrightarrow{}V \emph{s-}prefixation\is{prefixation!in \emph{s}-}, which derives verbs\is{verb!derived} of reversal, see %
%Mayo et al. (1995:932)
\citet[932]{Mayo1995}%
%Mayo-al.
%
, among others. This is a different morphological process, which I do not discuss here.} cf. (\ref{ex:Schwarze:3}) and (\ref{ex:Schwarze:4}):

\ea\label{ex:Schwarze:3}
    \ea\label{ex:Schwarze:3a} \gll {crem(a)}\textsubscript{N} \textrightarrow{}
    {screma(re)}\textsubscript{V} \\
    `cream' {} {`to skim'}\\
    \ex\label{ex:Schwarze:3b} \gll {carcer(e)}\textsubscript{N} \textrightarrow{}
    {scarcera(re)}\textsubscript{V} \\
    `prison' {} {`to release from prison'}\\
    \z
\ex\label{ex:Schwarze:4}
    \ea\label{ex:Schwarze:4a} \gll La mattina, la nonna scremava il latte.\\
    the morning the grandmother skimmed\textsubscript{IMPERFECT} the milk\\
    \glt `In the morning, Grandmother used to skim the milk'\\

    \ex\label{ex:Schwarze:4b} \gll Il giudice ha scarcerato Giovanni Rossi. \\
    the judge has released-from-prison Giovanni Rossi \\
    \glt `The judge released Giovanni Rossi from prison.'\\
    \z
\z

Both verbs\is{verb!derived}, \emph{scremare} and \emph{scarcerare}, mean `x removes y from z'\emph{.} However, they differ with respect to the role of the nominal base in the verbs' meaning. In terms of Leonard 
\citepos{Talmy1985}
lexical typology of motion events, the entity denoted by the base noun may be the Figure\is{figure} or the Ground\is{ground}. In (\ref{ex:Schwarze:4a}) the cream (\emph{crema}) is the Figure\is{figure}; it is removed from the milk (\emph{latte}), which is the Ground\is{ground}.
Inversely, in (\ref{ex:Schwarze:3b}) the prison (\emph{carcere}) is the Ground\is{ground}, from which Giovanni Rossi, the Figure\is{figure}, is released. Thus the speaker needs to decide on the assignment of Figure\is{figure} and Ground\is{ground} for every single verb generated by N\textrightarrow{}V \emph{s-}prefixation\is{prefixation!in \emph{s}-}. In a two-level semantics, \textsc{sf} will only state that the verbs\is{verb!derived} under discussion denote caused motion, the role of the incorporated noun being left open. The general semantic form\is{semantic form} of these verbs\is{verb!derived} may thus be written as (\ref{ex:Schwarze:6}):\footnote{In %
%Heusinger \& Schwarze (2006) 
\citet{Heusinger2006} %
%Heusinger-Schwarze
%
the representation given here as (\ref{ex:Schwarze:6}) is not the final version, which uses indices in order to account for the correlation between ambiguity of role assignment and the alternative of quantification. In fact, if the predicate of the base noun is incorporated in the verb, it is only existentially bound by \E. If it becomes the direct object, it is bound by the \lda operator. In (\ref{ex:Schwarze:6}), quantification is omitted for the sake of easier reading.}

\ea\label{ex:Schwarze:6} \f{cause}{x,\f{become}{\neg\f{located}{y,z}}} \;\&\; {[}\f{N}{y}\ou\f{N}{z}{]}
\z % Internal labels follow submitted paper's numbering for easier replacement of references. The paper skipped n°5, so I am doing so too.

The first part of representation (\ref{ex:Schwarze:6}), \f{cause}{x,\f{become}{\neg\f{located}{y,z}}},
is the lexical decomposition of the main feature of all verbs\is{verb!derived} of removal, \f{remove}{x,y,z}. The second part, \f{N}{y}\ou\f{N}{z}, expresses the underspecification of \emph{s-}prefixed verbs\is{verb!derived} of removal by a disjunction, where N is the predicate of the base noun.

The ambiguity expressed by this disjunction is resolved at the \textsc{cs} level.
According to %
%Heusinger \& Schwarze's (2006) 
\citepos{Heusinger2006} %
%Heusinger-Schwarze
%
analysis, the resolution of the underspecification passes through the following phases: the concepts associated with the base noun predicates are looked up in \textsc{cs} and checked regarding their aptitude to be a Figure\is{figure} or a Ground\is{ground} in a motion event. Objects that may contain something, are apt to take the role of Ground\is{ground}, objects that may easily perform or undergo motion are apt to be the Figure\is{figure}. Some objects, such as a sheet of paper, may meet both criteria and may consequently motivate derived verbs\is{verb!derived} with two alternative fully specified meanings. \ili{Italian} \emph{scartare}, derived from \emph{carta} `paper', is such a case: it may be used as both a Ground\is{ground} verb or a Figure\is{figure} verb; cf. (\ref{ex:Schwarze:7}):

\ea\label{ex:Schwarze:7} \gll Mario scarta il regalo.\\
Mario \emph{s}-paper\textsubscript{3sg.ind.pres} the gift\\
\glt `Mario takes the gift out of the paper' \hfill \emph{Ground\is{ground} verb}
\glt `Mario takes the paper off the gift' \hfill \emph{Figure\is{figure} verb}
\z

\tabref{tab:Schwarze:2} gives the rule that derives denominal \emph{s-}prefixed verbs\is{verb!derived} of removal in the \textsc{llm} format, with the semantic layer formulated in such a way as to generate underspecified \textsc{sf} representations.\movedfootnote{For easier reading, I do not express here the case-marking of the Oblique, which must be \emph{ne} if its predicate is `\textsc{pro}' and must be marked by preposition \emph{da} elsewhere.}\textsuperscript{,}\movedfootnote{The {[}s{]} vs. {[}z{]} realization of the prefix is a matter of post-lexical phonology, hence it is not expressed in the morphological rule.}

 
\begin{table}
\begin{tabular}{lp{0.7\textwidth}}
\lsptoprule

{{c-structure}} &

{[}s{]}\textsubscript{V\_prefix} + N\textsubscript{stem} \textrightarrow{}
{[}sN\textsubscript{stem}{]}\textsubscript{V\_stem}\\

{{f-structure}} &


(\textuparrow{} \textsc{pred1)=`P' \textrightarrow{} } (\textuparrow{} \textsc{pred2)=`remove (\textuparrow{} subj),(\textuparrow{} obj),
(\textuparrow{} obl)'}\\
% TODO: non conforme aux guidelines, footnote dans table
{{p-structure}} &


\textless{}no morphologically relevant change\textgreater{}\\
% TODO: non conforme aux guidelines, footnote dans table

{{s-structure}} &
% π(y)\textrightarrow{} cause(x,become(¬located(y,z))) \& {[}N(y)∨N(z){]}
$\pi(y) \to \f{cause}{x,\f{become}{\neg\f{located}{y,z}}} \;\&\; {[}\f{N}{y}\ou\f{N}{z} {]}$ \\
\lspbottomrule
\end{tabular}
\caption{The rule for deriving \ili{Italian} denominal \emph{s-}prefixed verbs\is{verb!derived}}
\label{tab:Schwarze:2}
\end{table}


\section{A third sample analysis: French N\textrightarrow{}V conversion}\label{a-third-sample-analysis-french-nv-conversion}

\il{French}
\is{conversion!noun-to-verb}

I will now present a case study of \ili{French} N\textrightarrow{}V conversion\is{conversion!noun-to-verb}, as exemplified by the pairs in (\ref{ex:Schwarze:8}):
\ea\label{ex:Schwarze:8}
    \ea \gll {amidon} -- {amidonner}\\
    `starch' {} {`to starch'}\\
    \ex \gll {archives} -- {archiver}\\
    `archives' {} {`to archive'}\\
    \ex \gll {bêche} -- {bêcher}\\
    `spade' {} {`to dig'}\\
    \ex \gll {sucre} -- {sucrer}\\
    `sugar' {} {`to sugar'}\\
  \z
\z
The relation between the nouns and the respective verbs\is{verb!denominal} in (\ref{ex:Schwarze:8}) is clearly directed, which does not hold for other noun-verb pairs, as those given in (\ref{ex:Schwarze:9}):

\ea\label{ex:Schwarze:9}
    \ea \gll {chant} -- {chanter} \\
          `song' {} {`to sing'}\\

    \ex \gll {gel} -- {geler} \\
          `frost' {} {`to freeze'}\\

    \ex \gll {prêt} -- {prêter} \\
          `loan' {} {`to lend'}\\

    \ex \gll {vent} -- {venter} \\
          `wind' {} {`to be windy'}\\
\z\z

The difference between (\ref{ex:Schwarze:8}) and (\ref{ex:Schwarze:9}) is due to the ontological class of the nouns' predicates: whereas the nouns in (\ref{ex:Schwarze:8}) denote objects or substances and thus are clearly distinct from the respective verbs\is{verb!denominal},
those given in (\ref{ex:Schwarze:9}) denote events or results of events and thus are not clearly distinct from the verbs they relate to. The derivational direction in (\ref{ex:Schwarze:8}) clearly is N\textrightarrow{}V, because event predicates may be built upon object or substance predicates, but not inversely.\footnote{Cf. the more explicit formulation by %
%Tribout (2010a:140)
\citet[140]{Tribout2010a}%
%Tribout
%
: \enquote{\ldots{} le recours aux propriétés sémantiques des deux lexèmes pour déterminer l'orientation de la conversion\is{conversion} repose, par exemple, sur l'idée que le lexème dérivé est nécessairement défini par le biais de son lexème base, tandis que le lexème base est sémantiquement indépendant de son lexème dérivé. Ainsi pour la paire \textsc{clou}$\thicksim$\textsc{clouer}, \textsc{clouer} est nécessairement défini relativement à \textsc{clou} comme \enquote{faire quelque chose avec des clous} tandis que \textsc{clou} est défini comme un petit objet pointu, indépendamment de clouer. Cette asymétrie dans la relation sémantique entre les deux lexèmes permet de prédire une orientation de la conversion\is{conversion!noun-to-verb} de nom à verbe.}} On the contrary, the conversion\is{conversion!noun-to-verb} in (\ref{ex:Schwarze:9}) may be the opposite, V\textrightarrow{}N,\footnote{For a state-of-the-art discussion on the direction of the \ili{French} N\textrightarrow{}V vs. V\textrightarrow{}N conversion see %
%Tribout (2010b: 348-356) 
\citet[348--356]{Tribout2010}.} or non-directional, N$\leftrightarrow$V, because the nouns' and the verbs' predicates are identical or very closely related.

As for the semantics of N\textrightarrow{}V conversion\is{conversion!noun-to-verb}, I assume that the rule defines an underspecified semantic form\is{semantic form}, from which full meanings are derived by a retrieval of conceptual structure\is{conceptual structure}.\footnote{%
%Tribout (2010a: 284-290) 
\citet[284--290]{Tribout2010a} %
%Tribout
%
criticizes the underspecification approach; instead she proposes and spells out a fully specified semantics, based upon a classification of the output verbs. I am trying to show that an underspecification-based analysis of the \ili{French} N\textrightarrow{}V conversion\is{conversion!noun-to-verb} is an achievable goal.} To account for actual meanings that are not predicted on this basis, post-morphological processes are taken into account. It is also assumed that there are certain verbs that look like N\textrightarrow{}V converts,
but are idiosyncratic items not derived by the rule.

\subsection{A database}\label{a-database}

As a descriptive basis for the study, I established a database of 170
verbs\is{verb!derived} that clearly are N\textrightarrow{}V converts. 19 of these verbs\is{verb!derived} are prefixed and have no lexicalized unprefixed counterpart, such as \emph{emprisonner}
`to imprison'.

I consider including prefixed verbs\is{verb!derived} of this kind as legitimate, because the prefixes involved, \emph{en-}, \emph{dé-} and \emph{re-}, require a verbal base. \emph{Emprisonner}, e.g., thus has the derivational history shown by (\ref{ex:Schwarze:10}):

\ea\label{ex:Schwarze:10} {{prison}\textsubscript{N}} \textrightarrow{}
{prisonner}\textsubscript{V} \textrightarrow{}
{{emprisonner}\textsubscript{V}}
\z

In addition to the verbs and their base nouns, the database contains the following information:

\begin{itemize}
\item The verb's underspecified semantic form\is{semantic form} (\textsc{sf}), if there is one

\item The specified semantic representation (\textsc{sr})

\item The conceptual class of the base noun

\item The presence of a prefix, if there is one

\item Remarks on formal and semantic properties of the derived verbs\is{verb!derived}

\end{itemize}

\subsection{The underspecified semantic forms}\label{the-underspecified-semantic-forms}
\is{semantic form}
Underspecified semantic forms\is{semantic form} could be construed for 142 of the 170
verbs\is{verb!derived}. The predominant one, which holds for 136 of the 170 verbs\is{verb!derived} contained in the database, states the following:

\begin{itemize}
\item The verb denotes an event, which is an action

\item It involves an agent and a theme

\item The denotation of the noun from which the verb is derived is a salient

 component of the action
\end{itemize}

For an illustration, see example (\ref{ex:Schwarze:11}):

\ea\label{ex:Schwarze:11} \gll Le secrétaire a archivé la correspondence.\\
the secretary has archived the correspondence\\
\glt `The secretary archived the correspondence.'
\z

The \textsc{sf} underlying (\ref{ex:Schwarze:11}) states that the sentence describes an action. The denotation of the noun \emph{archives} is a salient component of that action. The verb, \emph{archivé}, has two arguments, \emph{le secrétaire} and \emph{la correspondence}, whose roles are agent and theme respectively.

In addition to the predominant \textsc{sf}, two more \textsc{sf}s have been identified;
they are closely related to the predominant one, see examples (\ref{ex:Schwarze:12}) and
(\ref{ex:Schwarze:13}). (\ref{ex:Schwarze:12}) describes an action, but unlike (\ref{ex:Schwarze:11}), the verb has no argument in the role of theme. (\ref{ex:Schwarze:13}), where the reflexive pronoun is the operator of the middle voice, describes a process, the verb's only argument is in the role of theme.

\ea\label{ex:Schwarze:12} \gll Nous avons passé l'après-midi à magasiner. {(Canadian Fr.)}\\
we have spent {the afternoon} to window-shopping {}\\
\glt `We spent the afternoon window-shopping.'\\

\ex\label{ex:Schwarze:13} \gll Leurs genoux se sont ankylosés.\\
their knees \textsc{Reflexive\_pronoun} are ankylosed\\
\glt `Their knees have become stiff.'\\
\z

All \textsc{sf}s assumed for the verbs\is{verb!derived} contained in the database are shown in \tabref{tab:Schwarze:3}, which also shows the forms of the semantic predicates involved,
the mapping of the arguments onto grammatical functions and the number of verbs\is{verb!derived} for each \textsc{sf}.\footnote{There are two questions that I cannot address here in detail. First, how productive is the process analyzed here? \ili{French} is a language that overwhelmingly prefers affixation to conversion\is{conversion!noun-to-verb}. I assume that N\textrightarrow{}V is fully productive, but that much of its output is blocked by the output of competing rules of affixation. Second, can the non-dominant \textsc{sf}s be derived from the predominent one? Further research is needed here.}


\begin{table}
\resizebox{\textwidth}{!}{
\begin{tabular}[c]{@{}lp{0.15\textwidth}p{0.2\textwidth}p{0.4\textwidth}l@{}}
\lsptoprule
& Semantic\newline predicate & Grammatical\newline functions & \textsc{sf} & verbs\\
\midrule

\textsc{sf}1 & \f{P}{e,x,y}

\f{agent}{x}

\f{theme}{y} &  \Tree[.{P (\textsc{subj}),} x ]\Tree[.(\textsc{obj}) y ]

 & x accomplishes an action on y, N is salient in that action. &
136\\
\textsc{sf}2 & \f{P}{e,x}

\f{agent}{x} & \Tree[.{P (\textsc{subj})} x ]\hspace{2cm}
 & x accomplishes an action. N is salient in that action. &
4\\
\textsc{sf}3 & \f{P}{e,x,y}

\f{theme}{x} & \Tree[.{P (\textsc{subj})} x ]\hspace{2cm}
& x undergoes a process. N is salient in that process. &
4\\
\lspbottomrule
\end{tabular}
}
\caption{Underspecified semantic forms\is{semantic form} of converted denominal verbs}
\label{tab:Schwarze:3}
\end{table}

We can now formulate the rule for \ili{French} N\textrightarrow{}V conversion\is{conversion!noun-to-verb}, see \tabref{tab:Schwarze:4}.\movedfootnote{Except the selection of alternative lexicalized stem variants, see fn. 8. In the table I omit quantification again in order to make reading easier.} At the semantic layer, only the predominant \textsc{sf} is given.

\begin{table}
\begin{tabular}{lp{0.7\textwidth}}
\lsptoprule

{{c-structure}} &

N\textsubscript{stem} \textrightarrow{}V\textsubscript{stem ,} 1\textsuperscript{st}
inflectional class\\

{{f-structure}} &

(\textuparrow{} \textsc{pred)=`P1' \textrightarrow{} } (\textuparrow{} \textsc{pred)=`P2 (\textuparrow{} subj),(\textuparrow{} obj)'}\\
{{p-structure}} &

\textless{}no morphologically relevant change\textgreater{}\\
% TODO: non conforme aux guidelines, footnote dans table

{{s-structure}} &

 $\f{p1}{z}\to \f{p2}{e,x,y} \et \f{agent}{x}\et \f{theme}{y} \et \f{salient\_component\_of}{e} = \f{p1}{y}$

% \f{p1}{z}\to \f{p2}{e,x,y}
% \et\f{agent}{x}\et\f{theme}{y}\et\f{salient\_component\_of}{e}=\f{p1}{y}\footnote{I omit quantification again in order to make reading more easy.}
\\
\lspbottomrule
\end{tabular}
\caption{The layered rule for N\textrightarrow{}V conversion\is{conversion!noun-to-verb}}
\label{tab:Schwarze:4}
\end{table}

\section{Resolving the underspecified semantic forms}\label{resolving-the-underspecified-semantic-forms}
\is{semantic form}
As has already been pointed out, the underspecified semantic forms\is{semantic form} cannot be used in discourse, because they are unable to refer to the specific actions denoted by the verbs. Hence the underspecification needs to be resolved. This happens by accessing the conceptual knowledge associated to the base nouns. Regarding N\textrightarrow{}V conversion\is{conversion!noun-to-verb}, I assume that the speaker or hearer looks up the conceptual knowledge associated with the noun, inspects the event types in which the noun's denotation is typically involved, and finally creates a new semantic predicate in which one of these event types is, so to speak, incorporated. The noun's meaning is then turned into a feature of the new predicate, a feature that becomes visible by lexical decomposition. I will try to illustrate this idea by means of two examples, the first is (\ref{ex:Schwarze:14}):

\ea\label{ex:Schwarze:14} \gll L' orfèvre a ciselé leurs noms sur les alliances.\\
the goldsmith has chiseled their names on the {wedding\_rings}\\
\glt `The goldsmith engraved their names on the wedding rings.'\\
\z

The verb contained in (\ref{ex:Schwarze:14}) has the general, underspecified semantic form\is{semantic form} listed as \textsc{sf}1 in \tabref{tab:Schwarze:3}, and repeated here as (\ref{ex:Schwarze:15}):

\ea\label{ex:Schwarze:15} X accomplishes an action on y; N is salient in that action.
\z

For \emph{ciseler} `to chisel, to engrave' we replace N with \enquote{a chisel},
getting (\ref{ex:Schwarze:16}):

\ea\label{ex:Schwarze:16}
 X accomplishes an action on y, a chisel (Fr. \emph{ciseau}) is salient in that action.
\z

The conceptual knowledge associated with \emph{ciseau} contains, among others, the information given under (\ref{ex:Schwarze:17}):

\ea\label{ex:Schwarze:17}
A chisel is a tool, used for cutting wood, stone or metals.
\z

The predicate \f{cut}{x,y} is the semantic counterpart of the concept of cutting. Going back from conceptual structure\is{conceptual structure} to semantic form\is{semantic form}, the speaker inserts it into the decomposed semantic representation of the new predicate created by the conversion\is{conversion!noun-to-verb} rule. The meaning of the new predicate also contains \f{chisel}{x}, taken from the base noun. Since,
according to (\ref{ex:Schwarze:17}), a chisel is a tool, i.e. an instrument, the feature will be \f{instrument\_used}{e,x,y}=\f{chisel}{z}. Notice that $z$ is not an argument of the new predicate and will not be realized in the sentence.
(\ref{ex:Schwarze:18}) is the assumed semantic representation of \emph{ciseler}, after the resolution of underspecification.

\ea\label{ex:Schwarze:18}
\E{e} \lx \ly \f{chisel}{e,x,y}\footnote{For readers not familiar with the \ili{French} language, I use English to name semantic features, even though this may make the analysis somewhat inaccurate.}\\
$\f{event\_type}{e}=\f{action}{e}$\\
$\f{action\_type}{e}=\f{cut}{x,y}$\\
$\f{agent}{e}=x$\\
$\f{theme}{e}=y$\\
$\f{instrument\_used}{e}=\f{chisel}{z}$\\
\z

The first line of (\ref{ex:Schwarze:18}) gives the semantic representation of
\emph{ciseler} in the standard notation. The remaining lines give its decomposed meaning in terms of features, written as equations, in the tradition of unification grammars. (This notation mainly shows its usefulness when larger sections of the lexicon are analyzed: it makes it easy to express feature inheritance, and it helps to control the consistency of the features declared.)

The second example I give for the resolution of underspecification is
(\ref{ex:Schwarze:19}):

\ea\label{ex:Schwarze:19} \gll Les chasseurs ont huilé leurs fusils.\\
          the hunters have oiled their shotguns\\
        \glt `The hunters oiled their shotguns.'\\
\z

The verb \emph{huiler} `to oil' has the same \textsc{sf} as \emph{ciseler}.
Applied to the base noun \emph{huile} `oil' it reads:

\ea\label{ex:Schwarze:20} X accomplishes an action on y, oil (Fr. \emph{huile}) is salient in that action.\\
\z
Accessing the conceptual knowledge associated with \emph{huile,} the speaker gets, among others, the information given under (\ref{ex:Schwarze:21}):

\ea\label{ex:Schwarze:21} Oil is a substance used to lubricate a mechanism.\\
\z
The predicate \f{lubricate}{x,y} is the semantic counterpart of the concept of lubricating. The speaker inserts it into the decomposed semantic representation; the meaning of the new predicate also contains \f{oil}{x},
taken from the base noun. Since, according to (\ref{ex:Schwarze:21}), oil is a substance,
the feature will  be \f{substance\_used}{e,x,y} = \f{oil}{z}. (\ref{ex:Schwarze:22}) is the assumed semantic representation of \emph{huiler}:

\ea\label{ex:Schwarze:22}
$\E{e} \lx \ly \f{oil}{e,x,y}$\\
$\f{event\_type}{e}=\f{action}{e}$\\
$\f{action\_type}{e}=\f{lubricate}{x,y}$\\
$\f{agent}{e}=x$\\
$\f{theme}{e}=y$\\
$\f{substance\_used}{e}=\f{oil}{z}$\\
\z


\subsection{Polysemy in lexical morphology}\label{polysemy-in-lexical-morphology}

The conceptual categorization of `oil' I assumed for the above sample analysis, i.e. that `oil' is a substance used to lubricate a mechanism,
is far from being the only one.\footnote{I inserted this section as a response to a comment I received from an anonymous reviewer. For the analysis of polysemy in lexical morphology, also see %
%Schwarze (2012)
\citet{Schwarze2012}%
%Schwarze
%
.} As we know, oil also is used to preserve wood or iron, to cook and season food, it also is a fuel, and an ingredient of oil paint. As linguists, we do not have scientific methods to find out to what extent knowledge of this kind is contained in the conceptual structure\is{conceptual structure} and we have no precise knowledge of how conceptual structure\is{conceptual structure} is processed during the resolution of semantic underspecification. However, we can look at the lexicon and see those elements of conceptual structure\is{conceptual structure} that show up in the lexical meanings of a given language. Thus we can observe that, in the meaning variation of the \ili{French} verb \emph{huiler} `to oil' the following bits of information clearly play a role:

\begin{enumerate}
\def\labelenumi{\roman{enumi}.}
\item Oil is a lubricant (\ref{ex:Schwarze:23}), repeated from (\ref{ex:Schwarze:19}).

\item Oil is a preservative (\ref{ex:Schwarze:24}).

\end{enumerate}

\ea\label{ex:Schwarze:23} \gll Les chasseurs ont huilé leurs fusils.\\
          the hunters have oiled their shotguns\\
        \glt `The hunters oiled their shotguns.'\\
\ex\label{ex:Schwarze:24} \gll Cette table a besoin d' être huilée.\\
          this table has need to be oiled\\
        \glt `This table needs to be oiled.'\\
\z

As to using oil for preparing or seasoning food, the situation is less clear. According to the reviewer of this article, whom I believe to be a native speaker of \ili{French}, \emph{huiler} cannot mean `to season with oil'. I briefly searched the Internet and found out that there were zero hits for \emph{huiler la viande} (\emph{viande} means `meat') and
\emph{huiler les steaks}. There were several hits for \emph{huiler la salade}, but only two of them were from real text (\ref{ex:Schwarze:25}) and (\ref{ex:Schwarze:26}), the others being citations from dictionaries.

\ea\label{ex:Schwarze:25} J'aime faire des vinaigrettes qui ne font pas qu'assaisonner ou huiler la salade mais qui apportent plutôt une valeur ajoutée.\footnote{\url{http://brutalimentation.ca/2017/01/14/salade-festive-vinaigrette-digestive} [2017-08-29].}
   \glt `I like to make vinaigrettes that do not only season or oil the salad but rather bring an additional value.'\\

\ex\label{ex:Schwarze:26} Ne pas huiler la salade, car ainsi suivant son goût chacun fera sa propre vinaigrette, et puis s'il reste de la salade, elle se conservera plus facilement sans vinaigrette.\footnote{\url{http://ilovecuisine.blogspot.ch/2013/09/ma-salade-de-lete-la-salade-nicoise.html}
[2017-08-29].}\\
   \glt `Don't oil the salad, because that's how everyone will make their own vinaigrette to their taste, and then, if some salad is left over, it will be preserved more easily without vinaigrette.'\\
\z




The remaining known uses of oil do not seem to play a role in the meaning variation of \ili{French} \emph{huiler}. Instead of speculating about why this should be so, let us pass on to a question that immediately arises from what we could observe.

Assuming that the accessible conceptual structure\is{conceptual structure} offers competing information for the resolution of the underspecified meaning generated by the morphological process, the full meaning of \emph{huiler} shows the following variants:

\ea \label{ex:Schwarze:27}
  \ea `To lubricate with oil'
  \ex `To preserve with oil'
  \ex `To prepare or season with oil'
\z\z

The question now is: How do speakers pick out the convenient reading in producing or parsing utterances? This is a very general question, not specific to the semantics\is{word-formation!semantics of} of word formation. In the case of transitive verbs such as \emph{huiler}, a sort of semantic agreement is at work,
which checks the compatibility of the verb's reading with the conceptual class of the direct object.

Regarding the avoidance of \emph{huiler} with a direct object denoting meat, there may be practical reasons or no reason at all; there are phenomena in verbal behavior that are beyond the reach of linguistic analysis.

\section{Restrictions on the input}\label{restrictions-on-the-input}

It can easily be seen that many nouns are not fit to be a base in the \ili{French} N\textrightarrow{}V conversion\is{conversion!noun-to-verb}. In a list of the first 100 non-eventual nouns contained in the \emph{Petit Larousse}, only two are a base of N\textrightarrow{}V converts, and only one of them, \emph{acier} `steel', is the stem of a verb with a transparent meaning, \emph{aciérer} `to cover with steel'.%
\footnote{The other, \emph{abîme} `abyss', has \emph{abîmer} `to damage' as a convert, but that verb has a meaning that does not seem to be derived  in a straightforward way from the noun's meaning.}
Notice, however, that this finding rests on a very weak empirical basis.
The nouns considered are very few, and the data are limited to strongly lexicalized items. More research is needed to get reliable quantitative results. So I will just characterize the database with respect to the
143 nouns that are the base of verbs with a transparent meaning. Turning these observations into well-founded constraints and disentangling grammatical constraints on the input and conditions for use and lexicalization of the output must be left to further research.

The following semantic characteristics of the base nouns can be gathered from the database:

\begin{itemize}
\item Most base nouns denote an instrument (42 items),\footnote{Cf. \enquote{Les verbes converts instrumentaux sont parmi les plus nombreux. Ils sont mentionnés dans toutes les études portant sur la conversion\is{conversion!noun-to-verb} et sont généralement définis comme signifiant `utiliser N', selon le schéma \ldots{} X utiliser Nb} %
%(Tribout 2010a: 263)
\citep[263]{Tribout2010a}%
%Tribout
%
.} a substance (36 items), a container (seven items), or a body part (nine items); see Tables 8 to 11 in the Appendix. \item Only one noun, \emph{enfant} `child', denotes a human being. The
 derived verb, \emph{enfanter} `to give birth to', is infrequent and strongly marked as belonging to the literary register.
\item Only two nouns denote an animal, \emph{raton} `young rat', and
 \emph{zèbre} `zebra'. \emph{Ratonner} `to commit a racial attack (\emph{ratonnade}) on North-African immigrants' has no transparent semantic relationship to its base. As to \emph{zèbre}, the derived verb, \emph{zébrer} `to stripe', is only weakly transparent: rather than to the animal, it refers to a visual pattern, black stripes upon a white ground.
\end{itemize}

Regarding the formal properties of the base nouns, short words are preferred: most of them are mono- or disyllabic, only three
(\emph{ankylose} `ankylosis', \emph{courbature} `ache, stiffness', and
\emph{magasin} 'store') have three and only one (\emph{photographie}
`photography') has four syllables.

Nouns consisting of one morpheme only are clearly preferred; only
\emph{tambourin} `tambourine' and \emph{photographie} `photography' may be segmented into morphemes. There are no agent nouns in
\emph{--(at)eur} and no quality nouns in \emph{--(i)té} in the stems of derived verbs\is{verb!derived}.

\section{Reduced or lacking transparency -- construed lexemes in time}\label{reduced-or-lacking-transparency-construed-lexemes-in-time}

The database contains several verbs whose relationship with the base noun is not fully transparent or not transparent at all. For none less than 25 of the 170 verbs\is{verb!derived}, no underspecified semantic form\is{semantic form} could be identified, which means that the meaning of the base noun is not a feature of the derived verb, see the examples in (\ref{ex:Schwarze:28}):

\ea \label{ex:Schwarze:28}
\begin{tabular}[t]{@{}lll}
a. & fourrager & fourrage\\
   &`to rummage through' & `forage'\\
b. & fronder &fronde \\
   & `to satirize' & `slingshot; revolt' \\
c. & gueuler &   gueule \\
   & `to yell, to bawl' & `mouth' \\
\end{tabular}
\z

Ten verbs\is{verb!derived} can be analyzed as having undergone some post-morphological change along one of the familiar paths of semantic change or variation,
such as narrowing or widening an original meaning. Examples are shown in \tabref{tab:Schwarze:5}:

  \begin{table}\resizebox{\linewidth}{!}{
\begin{tabular}[c]{@{}lllll@{}}
\lsptoprule
{N} & {English} & {V} & {English} &
{kind of change}\\
\midrule

\emph{fer} & `iron' & \emph{ferrer} & `to shoe (a horse)', `to strike (a fish)' & narrowing\\
\emph{jardin} & `garden' & \emph{jardiner} & `to do some gardening' &
narrowing\\
\emph{mur} & `wall' & \emph{emmurer} & `to wall (a prisoner)' &
narrowing\\
\emph{ombre} & `shadow' & \emph{ombrer} & `to shade, to hatch' &
narrowing\\
\emph{peau} & `skin' & \emph{peler} & `to peel' &
narrowing\\
\emph{piste} & `trace' & \emph{dépister} & `to track down (a game)' &
narrowing\\
\emph{plume} & `feather' & \emph{plumer} & `to pluck (a bird)' &
narrowing\\
\emph{tapis} & `carpet' & \emph{tapisser} & `to decorate (a wall and similar)' & widening\\
\lspbottomrule
\end{tabular}}
\caption{Post-morphological change along some familiar paths}
\label{tab:Schwarze:5}
\end{table}

A particular kind of incomplete semantic transparency of the converted verb is due the fact that, rather than the verb, the base noun underwent a change after the derived verb entered the mental lexicon. Examples are
\emph{échafauder} `to put up scaffolding' and \emph{mitrailler} `to machine-gun'. The base noun of \emph{échafauder}, \emph{échafaud}, does not mean `scaffolding' any longer, it means `executioner's platform' in modern-day \ili{French}. The verb's meaning came about when \emph{échafaud} still meant
`scaffolding'. Likewise, \emph{mitrailler} `to machine-gun' was created when the noun, \emph{mitraille}, still meant `machine gun'. Its meaning changed to `hail of bullets', which lessened the semantic transparency of the derived verb.

The formal transparency may also be obscured, i.e. the noun's stem may differ to some extent from the derived verb's stem.\footnote{For a complete list of the kinds of allomorphy involved in N\textrightarrow{}V conversion\is{conversion!noun-to-verb} see %
%Tribout (210a: 114f)
\citet[114f]{Tribout2010a}%
%
%
. She argues that even totally opaque pairs such as \emph{pierre} `stone' and \emph{lapider} `to stone' may be analyzed as cases of conversion\is{conversion!noun-to-verb}, because they are related by suppletion %
%(Tribout {[}2010a: 110, 118{]})
\citep[110, 118]{Tribout2010a}%
%
.} The variation in such cases mostly is due to morphologization of a phonological variation existing at an earlier stage of the language and may be made less opaque by the existence, in modern \ili{French}, of other examples that exhibit the same lexical variation. The variation between /o/ and /ɛl/ or /ǝl/ as in
\emph{peau} /po/ `skin' -- /pɛl/ `peels' and \emph{peler} /pǝle/ `to peel' is such a case. Its transparency is improved by the presence of numerous items like those given in (\ref{ex:Schwarze:29}):

\ea \label{ex:Schwarze:29}
\ea \glll {nouveau}\\
          /nuvo/\\
          {`new.\textsc{mas}'}\\
\ex \glll {nouvelle}\\
          /nuvɛl/\\
          {`new.\textsc{fem}'}\\
\ex \gll {renouveler}\\
          /rǝnuvǝle/\\
    \glt    {`to renew'}\\
\ex \glll {niveau} --- {niveler}\\
          /nivo/ {} /nivǝle/\\
          `level' {} {`to level'}\\
\z\z

\newpage 
But this is not always the case. See the right-most column in \tabref{tab:Schwarze:6}.

\begin{table}\resizebox{\linewidth}{!}
{
\begin{tabular}[c]{@{}lllll@{}}
\lsptoprule
{N} & {English} & {V} & {English} &
{remarks}\\
\midrule

\emph{ciseau} /sizo/ & `chisle' & \emph{ciseler} /sizǝle/ & `to chisle'
& see (\ref{ex:Schwarze:26})\\
\emph{faux} /fo/ & `scythe' & \emph{faucher} /foʃe/ & `to scythe' &
isolated variation\\
\emph{grain} /gʁɛ̃/ & `grain' & \emph{engrener} /ɑ̃gʁəne/ & `to engage with' & no transparency\\
\emph{hiver} /ivɛʁ/ & `winter' & \emph{hiverner} /ivɛʁne/ & `to winter'
& cf. \emph{jour -journée}\\
\emph{marteau} /maʁto/ & `hammer' & \emph{marteler} /maʁtǝle/ & `to hammer & see (\ref{ex:Schwarze:27})\\
\emph{nœud} /nø/ & `knot' & \emph{nouer} /nue/ & `to knot' & cf.
\emph{jeu-jouer}\\
\emph{poil} /pwal/ & `hair' & \emph{peler} /pǝle/ & `to peel' & cf.
\emph{moi-me}\\
\emph{poil} /pwal/ & `hair' & \emph{épiler} /epile/ & `to depilate' &
native vs. borrowed\\
\emph{sang} /sɑ̃/ & blood' & \emph{saigner} /sɛɲe/ & `to bleed' &
isolated variation\\
\lspbottomrule
\end{tabular}}
\caption{Stem variation in N-V pairs}
\label{tab:Schwarze:6}
\end{table}

Most of these cases of reduced or lacking transparency have originated from the development of the grammar combined with the effects of lexicalization. N\textrightarrow{}V conversion\is{conversion!noun-to-verb} has been a persistent rule in a changing grammar. It was present at the Latin stage of the language (see Table~\ref{tab:Schwarze:7}), and endured throughout the centuries up to the present day, while there happened important changes elsewhere in the grammar.

\begin{table}
\resizebox{\linewidth}{!}{
\begin{tabular}[c]{@{}llll@{}}
\lsptoprule N & English & V & English\\
\midrule

\emph{cor cordis} & heart & \emph{recordor} & to call to mind, to remember\\
\emph{glacies} & ice & \emph{glaciō} & to freeze\\
\emph{navigium} & vessel, ship & \emph{navigō} & to navigate, to sail\\
\emph{onus oneris} & cargo, burden, load & \emph{onerō} & to load, to burden\\
\emph{pignus pignoris} & bet, stake, pledge & \emph{pignorō} & to pledge\\
\emph{pilum} & hair & \emph{pilō} & to depilate\\
\emph{pugnus} & fist & \emph{pugnō} & to fight\\
\emph{sal} & salt & \emph{salō} & to salt\\
\emph{vēlum} & curtain, sail, covering & \emph{vēlō} & to enfold,
envelop, veil\\
\lspbottomrule
\end{tabular}}
\caption{N\textrightarrow{}V conversion\is{conversion!noun-to-verb} in Latin}
\label{tab:Schwarze:7}
\end{table}

When speakers found it useful for communication, the output of the rule entered into usage and was lexicalized. This happened at various periods, when the meaning of the base noun could be different from today's, and when there was a regular phonological variation given up later. But the original forms and meanings could remain in the lexicon.

Moreover, once a construed word has entered the mental lexicon, its meaning may develop freely, which leads to reduced or lacking transparency with respect to the original meaning, founded on some \textsc{sf}
and its conceptual resolution.

What does that mean for the morphological process as a part of mental grammar? Remember that word formation rules are thought to have a double purpose: they create possible words, and they analyze existent words.
Hence the N\textrightarrow{}V conversion\is{conversion!noun-to-verb} rule will not create opaque or semi-transparent forms. However, as a means of learning and understanding construed lexemes, it will also cope with semi-transparent forms, to the extent that suitable variation patterns are present in the lexicon. Thus speakers will presumably be able to relate \emph{ciseler} /sizǝle/ to
\emph{ciseau} /sizo/ or \emph{marteler} /maʁtǝle/ to \emph{marteau}
/maʁto/, because these pairs show a variation pattern that is also present elsewhere in the lexicon. In addition, a clear semantic relationship between the noun and the verb certainly is a strong support to transparency. It would be interesting to see experimental research on this point.


\section*{Acknowledgments}

I am most grateful to Fabio Montermini, who thouroughly read the present text and gave valuable comments that helped me improve its form and content. I also am indebted to an anonymous reviewer, who discovered several remaining errors and made most constructive comments.

\section*{Appendix}

The Appendix contains some tables that would have disturbed the reading process of the main text.

\begin{table}
\resizebox{\linewidth}{!}{
\begin{tabular}[c]{@{}llll@{}}
\lsptoprule
{V} & {English} & {N} &
{English}\\
\midrule

\emph{ciller} & to blink & \emph{cil} & eyelash\\
\emph{enculer} & to sodomize & \emph{cul} & arse\\
\emph{doigter} & to use one's fingers correctly on a piano and similar &
\emph{doigt} & finger\\
\emph{griffer} & to scratch & \emph{griffe} & claw\\
\emph{gueuler} & to yell & \emph{gueule} & mouth\\
\emph{manier} & to handle & \emph{main} & hand\\
\emph{peler} & to peel & \emph{peau} & skin\\
\emph{plumer} & to pluck (a bird) & \emph{plume} &
feather\\
\emph{dépiler} & to depilate & \emph{poil} & hair\\
\emph{sourciller} & to raise one's eyebrows & \emph{sourcil} &
eyebrow\\
\emph{talonner} & to follow someone's heels & \emph{talon} &
heel\\
\emph{zyeuter} & to take a look at & \emph{yeux} & eyes\\
\lspbottomrule
\end{tabular}}
\caption{Verbs derived from nouns that denote a body part}
\label{tab:Schwarze:11}
\end{table}




\begin{table}
\resizebox{\linewidth}{!}{
\renewcommand{\arraystretch}{.9}
\begin{tabular}[c]{@{}llll@{}}
\lsptoprule
{V} & {English} & {N} &
{English}\\
\midrule

\emph{ancrer} & to anchor & \emph{ancre} & anchor\\
\emph{arquer} & to curve & \emph{arc} & bow\\
\emph{basculer} & to topple over & \emph{bascule} &
seesaw\\
\emph{bêcher} & to dig with a spade & \emph{bêche} &
spade\\
\emph{boulonner} & to bolt & \emph{boulon} & bolt\\
\emph{brosser} & to brush & \emph{brosse} & brush\\
\emph{ceinturer} & to surround & \emph{ceinture} & belt\\
\emph{ciseler} & to chisle & \emph{ciseau} & chisle\\
\emph{claironner} & to shout from the rouftops & \emph{clairon} &
bugle\\
\emph{clouer} & to nail & \emph{clou} & nail\\
\emph{cravacher} & to whip & \emph{cravache} & whip\\
\emph{crocheter} & to pick (a door, a lock) & \emph{crochet} &
picklock\\
\emph{chaîner} & to put on snow chains & \emph{chaîne} &
chain\\
\emph{faucher} & to scythe & \emph{faux} & scythe\\
\emph{filtrer} & to filter & \emph{filtre} & filter\\
\emph{flinguer} & to blow away, to shoot & \emph{flingue} &
gun\\
\emph{flûter} & to produce a flute-like sound & \emph{flûte} &
flute\\
\emph{fouetter} & to flog, to whip & \emph{fouet} & whip\\
\emph{fourcher} & to split & \emph{fourche} & fork\\
\emph{freiner} & to brake & \emph{frein} & brake\\
\emph{fronder} & to satirize & \emph{fronde} & sling,
revolt\\
\emph{fusiller} & to shoot (a condemned person) & \emph{fusil} &
rifle\\
\emph{hacher} & to chop & \emph{hache} & ax\\
\emph{griller} & to grill & \emph{gril} & grill\\
\emph{limer} & to file & \emph{lime} & file\\
\emph{marteler} & to beat, to pound & \emph{marteau} &
hammer\\
\emph{menotter} & to handcuff & \emph{menottes} &
handcuffs\\
\emph{miner} & to mine, to sap & \emph{mine} & mine\\
\emph{mitrailler} & to machine-gun & \emph{mitraille} & hail of bullets\\
\emph{peigner} & to comb & \emph{peigne} & comb\\
\emph{photographier} & to photograph & \emph{photographie} &
photography\\
\emph{pilonner} & to bombard, to grind & \emph{pilon} &
pestle\\
\emph{poignarder} & to stab & \emph{poignard} & dagger\\
\emph{raboter} & to plane & \emph{rabot} & plane\\
\emph{sabrer} & to cut down & \emph{sabre} & sword\\
\emph{scier} & to saw & \emph{scie} & saw\\
\emph{tambouriner} & to hammer, to drum & \emph{tambourin} &
tambourin\\
\emph{tamiser} & to sieve & \emph{tamis} & sieve\\
\emph{téléphoner} & to phone & \emph{téléphone} & phone\\
\emph{se tirebouchonner} & to be twisted, to be wrinkled &
\emph{tirebouchon} & corkscrew\\
\emph{visser} & to screw on & \emph{vis} & screw\\
\emph{vriller} & to bore, to pierce & \emph{vrille} &
spiral\\
\lspbottomrule
\end{tabular}}
\caption{Verbs derived from nouns that denote an instrument}
\label{tab:Schwarze:8}
\end{table}


\begin{table}
%\resizebox{\linewidth}{!}{
% \renewcommand{\arraystretch}{.9}
\begin{tabularx}{\textwidth}{lXll}
\lsptoprule
{V} & {English} & {N} &
{English}\\
\midrule
\emph{aérer} & to air & \emph{air} & air\\
\emph{amidonner} & to starch & \emph{amidon} & starch\\
\emph{argenter} & to silver & \emph{argent} & silver\\
\emph{bétonner} & to concrete & \emph{béton} & concrete\\
\emph{beurrer} & to butter & \emph{beurre} & butter\\
\emph{bitumer} & to asphalt, to tarmac & \emph{bitume} &
bitumen\\
\emph{charbonner} & to blacken & \emph{charbon} & coal\\
\emph{chiffonner} & to crumple & \emph{chiffon} & mousseline,
rag\\
\emph{cimenter} & to cement & \emph{ciment} & cement\\
\emph{cirer} & to polish (shoes, the floor) & \emph{cire} &
wax\\
\emph{crotter} & to muddy & \emph{crotte} & dropping\\
\emph{cuivrer} & to bronze, to copper & \emph{cuivre} &
copper\\
\emph{émailler} & to enamel & \emph{émail} & enamel\\
\emph{fariner} & to flour & \emph{farine} & flour\\
\emph{ferrer} & to shoe (a horse) & \emph{fer} & iron\\
\emph{feutrer} & to felt & \emph{feutre} & felt\\
\emph{enfieller} & to fill with bile & \emph{fiel} & bile,
venom\\
\emph{enfumer} & to fill with smoke & \emph{fumée} &
smoke\\
\emph{gazer} & to gas & \emph{gaz} & gas\\
\emph{givrer} & to frost over & \emph{givre} & frost\\
\emph{goudronner} & to tarmac & \emph{goudron} & tar\\
\emph{graisser} & to grease & \emph{graisse} & grease\\
\emph{huiler} & to oil & \emph{huile} & oil\\
\emph{larder} & to lard & \emph{lard} & fat streaky bacon\\
\emph{pimenter} & to put chillies in & \emph{piment} & hot pepper\\
\emph{plastiquer} & to carry out a bomb attack on & \emph{plastic} &
plastic explosive\\
\emph{plâtrer} & to plaster & \emph{plâtre} & plaster\\
\emph{plomber} & to fill (a tooth), to seal & \emph{plomb} &
lead\\
\emph{poivrer} & to pepper & \emph{poivre} & pepper\\
\emph{poudrer} & to powder & \emph{poudre} & powder\\
\emph{rouiller} & to rust & \emph{rouille} & rust\\
\emph{sabler} & to sandblast & \emph{sable} & sand\\
\emph{saigner} & to bleed & \emph{sang} & blood\\
\emph{savonner} & to rub soap on & \emph{savon } & soap\\
\emph{saler} & to salt & \emph{sel} & salt\\
\emph{sucrer} & to put sugar in & \emph{sucre} & sugar\\
\lspbottomrule
\end{tabularx}
%}
\caption{Verbs derived from nouns that denote a substance}
\label{tab:Schwarze:9}
\end{table}

\begin{table}
%\resizebox{\linewidth}{!}{
\begin{tabularx}{\textwidth}{lXll}
\lsptoprule
{V} & {English} & {N} &
{English}\\
\midrule

\emph{archiver} & to archive & \emph{archives} & archive\\
\emph{cuver} & to ferment & \emph{cuve} & tank\\
\emph{engainer} & to put into its sheath & \emph{gaine} &
sheath\\
\emph{engranger} & to gather in, to store & \emph{grange} &
barn\\
\emph{emmagasiner} & to store & \emph{magasin} & store\\
\emph{emprisonner} & to imprison & \emph{prison} & prison\\
\emph{enregistrer} & to register & \emph{registre} &
register\\
\lspbottomrule
\end{tabularx}%}
\caption{Verbs derived from nouns that denote a container}
\label{tab:Schwarze:10}
\end{table}





\is{morphology!LFG-based layered morphology|)}
\is{two-level semantics|)}
{\sloppy
\printbibliography[heading=subbibliography,notkeyword=this]
}


\end{document}
