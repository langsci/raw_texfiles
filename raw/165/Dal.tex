\documentclass[output=paper]{langsci/langscibook}
\ChapterDOI{10.5281/zenodo.1406995}
\title{\texorpdfstring{Les adverbes en \textit{-ment} du français~: Lexèmes ou formes d'adjectifs~?}{Les adverbes en {-ment} du français~: Lexèmes ou formes d'adjectifs~?}}
\renewcommand{\lsCollectionPaperFooterTitle}{Les adverbes en \noexpand\textit{-ment} du français~: Lexèmes ou formes d'adjectifs~?}
\author{Georgette Dal \affiliation{Univ. Lille, CNRS, UMR 8163 - STL - Savoirs Textes Langage, F-59000 Lille, France}}
 \abstract{Cet article cherche à déterminer le statut des adverbes en \emph{-ment}\is{adverb!in --\emph{ment}} du français~: s'agit-il de lexèmes résultant de l'application d'une règle de construction de lexèmes\is{lexeme formation rule}, ou de mots-formes\is{wordform} relevant du paradigme\is{paradigm} de l'adjectif~? Contrairement à d'autres langues comme l'anglais, ou, pour ce qui est des langues romanes, l'espagnol ou l'italien, la question a été peu débattue en français dans des travaux récents, à l'exception de %
\citet{Dal07}%
%Dal
%
. Or, un examen attentif des propriétés de ces adverbes et, dans le même temps, de la règle dont ils sont le produit fait clairement opter pour une analyse flexionnelle. La conclusion est par conséquent que les adverbes en \emph{--ment}\is{adverb!in --\emph{ment}} constituent des variantes contextuelles d'adjectifs, dont ils sont des mots-formes\is{wordform}.

%Mots-clés
%
%Flexion \emph{vs} dérivation~; adverbes ; adjectifs~; paradigme flexionnel adjectival 
}

\maketitle

\begin{document}
\selectlanguage{french}
\il{French|(}

\section{Introduction}

La séquence \emph{--ment} présente dans des adverbes du français pouvant être mis en relation formelle et sémantique avec un adjectif comme \emph{joyeusement~}/~\emph{joyeux, prestement~}/~\emph{preste} ou \emph{timidement~}/~\emph{timide} est en général tenue pour dérivationnelle, au point qu'elle figure comme telle en bonne place dans les ouvrages universitaires à visée pédagogique %
%(par exemple, Huot 2006~; Niklas-Salminen 2015~; Gardes-Tamine \& al. 2015)
\citep[par exemple, ][]{Huot06,Niklas15,Gardes15}%
%Huot;Niklas-Salminen;Gardes-Tamine-al.
%
, sans parler des manuels ou ressources en ligne à destination de jeunes publics où, bien souvent, la formation d'adverbes en \emph{--ment}\is{adverb!in --\emph{ment}} constitue l'exemple archétypal de dérivation\is{derivation}.

Le statut dérivationnel de la règle dont \emph{--ment} est l'exposant\is{exponent} --par conséquent, le caractère lexématique des adverbes qu'elle permet de former--, n'est pas davantage remis en cause dans les travaux de recherche, y compris chez les morphologues %
%(voir par exemple Corbin 1982 et 1987~; van Willigen 1983~; Bonami \& Boyé 2005~; Roché 2010~; Boyé \& Plénat 2015~; Detges 2015~; Rainer 2016)
\citep[voir par exemple ][]{Corbin82,Corbin87,Willigen83,Bonami05,Roche10,Boye15,Detges15,Rainer16b}%
%Corbin;Corbin;?van Willigen;Bonami-Boyé;Roché;Boyé-Plénat;Detges;Rainer
%
, même dans un cadre comme celui de la morphologie naturelle dans lequel l'opposition flexion\is{inflection} / dérivation\is{derivation} n'est pas discrète mais scalaire %
%(pour des points récents sur ce courant, cf. Dressler 2005 et Luschützky 2015)
\citep[pour des points récents sur ce courant, cf. ][]{Dressler05,Luschutzky15}%
%Dressler;Luschützky
%
. Or, si l'on considère attentivement les caractéristiques des adverbes en \emph{--ment}\is{adverb!in --\emph{ment}} du français, il apparaît que le caractère dérivationnel de la règle dont cette séquence est l'exposant\is{exponent} n'a aucun caractère d'évidence. C'est ce que cherche à (re)mettre en lumière cette recherche, dans le prolongement de %
%Dal (2007)
\citet{Dal07}%
%Dal
%
.

Le présent chapitre débutera par un état de l'art sur le traitement de quelques homologues des adverbes en \emph{--ment}\is{adverb!in --\emph{ment}} du français dans plusieurs langues romanes et en anglais. Cet état de l'art sera l'occasion de poser quelques jalons pour la suite. Dans un deuxième temps, j'examinerai si les adverbes en \emph{--ment}\is{adverb!in --\emph{ment}} du français répondent aux attendus des produits d'une règle de construction de lexèmes\is{lexeme formation rule}. À l'issue de cet examen, il apparaîtra que la réponse est négative sur tous les plans et qu'à l'instar de leurs homologues dans d'autres langues romanes et en anglais, ces adverbes peuvent être tenus pour des variantes contextuelles d'adjectifs instanciant une case du paradigme\is{paradigm} des adjectifs auxquels ils sont morpho-sémantiquement appariables.

\section{État de l'art}

Si peu, pour ne pas dire pas, de travaux récents, excepté %
%Dal (2007)
\citet{Dal07}%
%Dal
%
, s'interrogent sur la nature de la règle associant à un adjectif donné un adverbe en \emph{--ment}\is{adverb!in --\emph{ment}} en français (son statut dérivationnel est en général asserté sans discussion), celle des règles produisant des adverbes à partir d'adjectifs a fait l'objet de davantage de questionnement dans plusieurs langues du monde. On se concentrera ici sur la suffixation en ‑\textsc{mente}\footnote{La notation en capitales ‑\textsc{mente} neutralise ici les réalisations sous les formes \emph{-mente} ou \emph{--ment} selon les langues concernées.} dans plusieurs langues romanes en dehors du français et en --\emph{ly} en anglais, et l'on verra que la question est loin d'être résolue, même dans les travaux les plus récents\footnote{On trouvera dans %
%Ricca (2015) 
\citet{Ricca15} %
%Ricca
%
une synthèse très documentée de la question pour d'autres langues du monde.}.
\il{French|)}

\subsection{Les adverbes en -\textsc{mente} dans les langues romanes (hors français)}
\is{adverb!in -\textsc{mente}|(}

La question du statut de la séquence -\textsc{mente} des adverbes des langues romanes, en dehors du français, a été abordée dans de nombreux travaux. Quatre hypothèses ont été formulées~: l'hypothèse compositionnelle (§ \ref{section:dal:2.1.1}), l'hypothèse dérivationnelle (§ \ref{section:dal:2.1.2}), l'hypothèse de l'affixe syntagmatique (§ \ref{section:dal:2.1.3}) et l'hypothèse flexionnelle (§ \ref{section:dal:2.1.4}).

\subsubsection{L'hypothèse compositionnelle}\label{section:dal:2.1.1}
\il{Spanish|(}
\il{Portuguese|(}
\il{Catalan|(}


Une hypothèse récurrente est que les adverbes en -\textsc{mente} seraient des composés\is{compounding}, partant, que -\textsc{mente} serait un nom conformément à son étymon latin \emph{mens, mentis} («~esprit~»). L'hypothèse a été développée pour l'espagnol %
%(cf., parmi d'autres, Bello 1847~; Hockett 1958~; Seco 1972~; Zagona 1990~; Kovacci 1999)
\citep[cf., parmi d'autres][]{Bello1847,Hockett58,Seco72,Zagona90,Kovacci99}%
%Hockett;Seco;Zagona;Kovacci
%
. On la trouve aussi formulée en filigrane pour le catalan et le portugais dans %
%Chircu (2007)
\citet{Chircu07}%
%Chircu
%
.

Outre l'argument étymologique, l'argument majeur sur lequel se fondent les partisans de cette hypothèse est la possibilité que présente -\textsc{mente} dans certaines langues romanes d'être élidé et mis en facteur commun en cas de coordination d'adverbes, -\textsc{mente} étant porté par le premier ou le dernier adverbe de la série selon les langues. La possibilité est attestée au moins en espagnol, catalan et portugais, comme l'indiquent les exemples (\ref{ex:Dal:1}-\ref{ex:Dal:3}) empruntés à la Toile~:

\ea\label{ex:Dal:1} (esp.) Inspira \textbf{lenta y profundamente} {[}lentement et profondément{]}.

\ex\label{ex:Dal:2} (cat.) \textbf{Ràpidament i silenciosa} {[}rapidement et silencieusement{]}, l'Elena se'ls va acostar.

\ex\label{ex:Dal:3} (port.) Quantas pessoas foram, \textbf{severa e cruelmente} {[}sévèrement et cruellement{]}, torturadas por se oporem ao regime?

\z

Certains linguistes, comme %
%Saporta (1990)
\citet{Saporta90}%
%Saporta
%
, ont tiré argument de cette possibilité pour voir dans les adverbes en \textsc{-mente} des composés\is{compounding} endocentriques dont la tête serait le nom \textsc{-mente}.

La double accentuation des adverbes en \textsc{-mente}, une première fois sur l'adjectif repérable dans leur structure, une seconde sur la séquence \textsc{-mente}, est un autre des arguments parfois avancés en faveur de la composition %
%(Saporta 1990~; Detges 2015)
\citep{Saporta90,Detges15}%
%Saporta;Detges
%
. C'est particulièrement vrai de l'espagnol (cf. \ref{ex:Dal:4}) où, normalement, un lexème issu d'un processus de dérivation\is{derivation} ne comporte qu'un seul accent, tandis que les composés\is{compounding} permettent une double accentuation~:

\ea\label{ex:Dal:4} (esp.) literàlmènte~; ràpidamènte~; cuidadósamènte ~
\z

Le dernier argument parfois invoqué, à vrai dire davantage contre l'hypothèse dérivationnelle qu'en faveur de l'hypothèse compositionnelle, est celui de la forme féminine de l'adjectif à laquelle s'adjoindrait -\textsc{mente}. Si ce dernier était un suffixe dérivationnel, il ne pourrait pas s'appliquer postérieurement à une règle flexionnelle\is{inflection!rule} (on reviendra ultérieurement sur ce point)~: or, si la séquence \textsc{-mente} n'est pas un suffixe dérivationnel, les adverbes en \textsc{-mente} ne peuvent être que des composés\is{compounding}, et \textsc{-mente} un nom, comme son étymon.

\il{Portuguese|)}
\il{Catalan|)}
\subsubsection{L'hypothèse dérivationnelle}\label{section:dal:2.1.2}
\il{Italian|(}

L'hypothèse dérivationnelle, que formulent entre autres %
%Karlsson (1981)
\citet{Karlsson81}%
%Karlsson
%
, %
%Bosque (1989)
\citet{Bosque89}%
%Bosque
%
, %
%Varela Ortega (1990) 
\citet{Varela90} %
%Varela Ortega
%
ou %
%Rainer (1996 et 2016) 
\citet{Rainer96,Rainer16b} %
%Rainer;Rainer
%
à propos de l'adjonction de -\textsc{mente} à un adjectif pour former un adverbe, est en général une réponse aux faiblesses de l'hypothèse compositionnelle. Les arguments, dont on trouve une synthèse récente dans %
%Torner (2016)
\citealt{Torner2016}%
%
, sont en substance les suivants~:
\begin{enumerate}[label=(\roman*)]
\item la séquence -\textsc{mente} présente dans les adverbes des langues romanes n'a plus la valeur pleine du nom latin \emph{mens, mentis} «~esprit~», et les adverbes qui en sont pourvus peuvent avoir des types sémantiques variés~: au moins pour l'italien et l'espagnol, adverbes de manière (\emph{lentamente}), ou de point de vue (\emph{economicamente}), adverbes orientés sujet (\emph{francamente}), etc.~;

\item si les lexèmes en -\textsc{mente} étaient des composés\is{compounding} endocentriques à tête nominale, ils devraient être des noms et non pas des adverbes %
%(cf. aussi Fábregas 2007)
\citep[cf. aussi ][]{Fabregas07}%
%Fábregas
%
~;

\item l'adverbe en -\textsc{mente} hérite la structure argumentale de l'adjectif que l'on repère dans sa structure %
%(Bosque 1989~; Cifuentes 2002)
\citep{Bosque89,Cifuentes02}%
%Bosque
%
, comme l'indiquent les exemples espagnols sous (\ref{ex:Dal:5}) empruntés à %
%Fábregas (2007)
\citet{Fabregas07}%
%Fábregas
%
, ce que ne permet pas de prédire l'hypothèse compositionnelle~:
\ea\label{ex:Dal:5}
       \ea paralelo \textbf{a esto} / paralelamente \textbf{a esto}

        \ex independiente \textbf{de ello} / independientemente \textbf{de ello}

        \ex proporcional \textbf{al resultado} / proporcionalmente \textbf{al resultado}
\z\z

\item le type sémantique de l'adjectif détermine le type sémantique de l'adverbe en \mbox{-\textsc{mente}}~: les adjectifs relationnels produisent des adverbes de domaine ou de point de vue~; les adjectifs exprimant la manière d'agir d'un agent produisent des adverbes orientés agent~; etc.
De la même manière, les restrictions de sélection lexicale de l'adverbe sont corrélées à celles de l'adjectif.

\end{enumerate}

L'hypothèse dérivationnelle n'entre pas en conflit avec la catégorisation adverbiale des séquences en -\textsc{mente} à base adjectivale (cf. l'argument (ii) ci-dessus), et est davantage en conformité avec l'héritage, du lexème-base par le lexème-dérivé, de propriétés syntaxiques (cf. iii) et sémantiques (cf. iv). Si elle ne résout pas la variété des types sémantiques d'adverbes en \emph{--ment}\is{adverb!in --\emph{ment}} (cf. i), du moins n'est-elle pas incompatible avec elle.

\il{Italian|)}
\subsubsection{L'hypothèse de l'affixe syntagmatique}\label{section:dal:2.1.3}

Reprenant une notion mise au jour par %
%Zwicky (1987)
\citet{Zwicky1987c}%
%Zwicky
%
, %
%Nevis (1985) 
\citet{Nevis85} %
%Nevis
%
et %
%Miller (1992) 
\citet{Miller92} %
%Miller
%
et principalement appliquée aux clitiques, %
%Torner (2005 et 2016) 
\citet{Torner2005,Torner2016} %
%
 voit dans le statut d'affixe syntagmatique une alternative aux hypothèses compositionnelle et dérivationnelle.

L'hypothèse de l'affixe syntagmatique se fonde sur le caractère hybride de la séquence -\textsc{mente} des adverbes de l'espagnol. L'argument majeur réside dans l'application de cette séquence à (ce qui se donne à voir comme) la forme féminine de l'adjectif, autrement dit à une forme flexionnelle construite en syntaxe. Or, selon l'universel 28 de %
%Greenberg (1963)
\citet{Greenberg1963}%
%Greenberg
%
\footnote{``If both derivation and inflection follow the root, or they both precede the root, the derivation is always between the root and the inflection'' %
%(Greenberg 1963 : 93)
\citep[93]{Greenberg1963}%
%Greenberg
%
.}, que réinvestit à sa manière l'hypothèse de la morphologie scindée (\emph{split morphology}) développée par %
%Anderson (1977, 1982 et 1992) 
\citet{Anderson77,Anderson82,Anderson92} %
%Anderson;Anderson;Anderson
%
et %
%Perlmutter (1988)
\citet{Perlmutter88}%
%Perlmutter
%
, la flexion\is{inflection} est réputée s'appliquer après la dérivation\is{derivation}.

Même si, à la suite de %
%Rainer (1996~: 87)
\citet[87]{Rainer96}%
%Rainer
%
, %
%Torner (2005~: 131)
\citet[131]{Torner2005} %
%
 convient que ce choix d'une forme féminine est davantage vestigial, étant donné l'étymon de -\textsc{mente}, que requis par la syntaxe, il s'agit pour lui d'un argument décisif, qui explique en outre la possibilité, soulignée plus haut, d'une élision de la séquence en cas de coordination d'adverbes. Dans l'hypothèse de l'affixe syntagmatique, il n'y a en fait pas d'élision, mais plutôt un attachement de -\textsc{mente} à un syntagme adjectival %
%(Torner 2005~: 132)
\citep[132]{Torner2005}%
%
, autrement dit à une séquence syntaxique (d'où la notion d'affixe syntagmatique), formée par conséquent postérieurement à l'application d'une marque flexionnelle à l'adjectif.

\il{Spanish|)}

\subsubsection{L'hypothèse flexionnelle}\label{section:dal:2.1.4}
\il{Italian|(}

L'hypothèse flexionnelle semble avoir été moins explorée que les hypothèses compositionnelle et dérivationnelle pour expliquer le statut de -\textsc{mente} dans les langues romanes.

Pour l'espagnol, on la trouve néanmoins formulée dans %
%Hjelmslev (1928)
\citet{Hjelmslev28}%
%Hjelmslev
%
, et, à sa suite, dans %
%Alarcos Llorach (1951~: 85)
\citet[85]{Alarcos51}%
%Alarcos Llorach
%
, pour qui la «~forma adverbial del adjetivo en \emph{-mente} debe considerarse como un `casus adverbialis', pues su morfema es exigido por el `verbo' regente~». %
%Pottier (1966) 
\citet{Pottier66} %
%Pottier
%
considère pareillement qu'en espagnol\il{Spanish}, les adverbes en \emph{-mente}\is{adverb!in --\emph{mente}} ne sont rien d'autre que la forme que revêt l'adjectif sous rection verbale et, donc, que \emph{-mente} y est une marque casuelle.

En ce qui concerne l'italien, on peut citer %
%Scalise (1990) 
\citet{Scalise90} %
%Scalise
%
et %
%Ricca (1998 et 2004)
\citet{Ricca98,Ricca04}%
%Ricca;Ricca
%
, même si, au terme de leur examen, ni l'un ni l'autre ne retiennent l'hypothèse flexionnelle.

Selon %
%Scalise (1990)
\citet{Scalise90}%
%Scalise
%
, le principal écueil auquel elle se heurte en italien réside dans la productivité\is{productivity} limitée de la suffixation en \emph{-mente}, où \emph{productif} est à entendre comme «~apte à s'appliquer dès que sont réunies les conditions catégorielles favorables à l'application~»\footnote{On distingue ici cet emploi de la notion de productivité\is{productivity} de celui qu'en fait %
%Schultink (1961) 
\citet{Schultink1961} %
%Schultink
%
(en substance~: possibilité, pour les locuteurs d'une langue, de former, de façon non intentionnelle, un nombre en principe infini de nouveaux mots morphologiquement complexes à l'aide d'un procédé donné). Pour un point récent sur la notion de productivité\is{productivity}, cf. %
%Gaeta \& Ricca (2015) 
\citet{Gaeta15} %
%Gaeta-Ricca
%
et %
%Dal \& Namer (2016)
\citet{Dal16}%
%Dal-Namer
%
.}. En effet, là où la flexion\is{inflection} passe pour être entièrement productive --~par exemple, en français, tout adjectif peut être fléchi en nombre~--, la dérivation\is{derivation} le serait moins. Ce contraste figure en bonne place parmi les très nombreux travaux s'interrogeant sur les critères cherchant à opposer flexion\is{inflection} et dérivation\is{derivation} (cf., entre autres, %
%Dressler 1989~; Scalise 1988~; Haspelmath 1996~; Blevins 2001~; Kilani-Schoch \& Dressler 2005~;~Stump 2005
\citealt{Dressler89,Scalise1988,Haspelmath1996,Blevins2001,Kilani-Schoch05,Stump05}%
%
\footnote{%
%Stump (2005~: 54) 
\citet[54]{Stump05} %
%Stump
%
préfère utiliser le terme de \emph{completeness} à celui de \emph{productivity}.}, %
%ten Hacken 2014~; Štekauer 2015
\citealt{Hacken14,Stekauer15}%
%
). Or, s'agissant de \emph{-mente}\is{adverb!in --\emph{mente}} en italien, %
%Scalise (1990)
\citet{Scalise90} %
%Scalise
%
 recense plusieurs catégories d'adjectifs qui seraient rétifs à son adjonction. Si, comme lui, l'on exclut le cas des possessifs, démonstratifs, indéfinis, numéraux au motif que leur statut adjectival est discutable, il s'agit, pour l'essentiel %
%(le marquage par un astérisque est le fait de Scalise 1990)
\citep[le marquage par un astérisque est le fait de ][]{Scalise90}%
%Scalise
%
~:
\begin{enumerate}[label=(\alph*)]

\item des adjectifs exprimant des propriétés physiques (\emph{calvo} `chauve' / *\emph{calvamente}), dont les adjectifs chromatiques (\emph{giallo} `jaune' / *\emph{giallamente}),

\item de l'acception littérale des adjectifs possédant une acception littérale et une acception métaphorique (\emph{aridamente} ne serait possible qu'avec l'interprétation figurée de \emph{arido} `dépourvu de sentiment'),

\item de l'acception spatiale des adjectifs présentant une lecture spatiale et une lecture temporelle~: ainsi, à l'adjectif \emph{lungo} `long (dans le temps ou dans l'espace)' correspond bien un adverbe, \emph{lungamente}, mais ce dernier exprime une propriété exclusivement temporelle,

\item d'un certain nombre d'adjectifs construits~: évaluatifs (\emph{leggerino} `assez léger' / *\emph{leggerinamente}), adjectifs de relation en --\emph{acco} (\emph{polacco} `polonais' / *\emph{polaccamente}), en --\emph{ale} (\emph{postale} `postal' / *\emph{postalmente})\emph{,} en --\emph{ano} (\emph{isolano} `insulaire' / *\emph{isolanamente}), etc., adjectifs en --\emph{bile} à base verbale sous leur forme positive (\emph{utilizzabile} `utilisable' / *\emph{utilizzabilamente}). Pour S. Scalise, 45 des 65 suffixes formant des adjectifs en italien bloqueraient ainsi l'application postérieure de la suffixation en \emph{-mente}, sans qu'il ne s'agisse toutefois d'une impossibilité structurelle catégorique, comme en attestent \emph{naturalmente}, \emph{temporaneamente}, \emph{barbaricamente}, \emph{amabilmente}, etc., que cite %
%Scalise (1990)
\citet{Scalise90}%
%Scalise
%
\footnote{On relève également sur la Toile des occurrences de ces séquences marquées comme impossibles par %
%Scalise (1990)
\citet{Scalise90}%
%Scalise
%
. Par exemple \emph{polaccamente} (litt. «~polonaisement~»)~: «~(\ldots{}) e il segretario particolare di Giovanni Paolo, un prete polacco dal nome \textbf{polaccamente} impossibile~».}.

\end{enumerate}

Les adjectifs résultant d'un processus de composition seraient pareillement impropres à donner lieu à un adverbe en \emph{-mente}\is{adverb!in --\emph{mente}} en italien~: *\emph{dolceamaramente}, *\emph{storicocriticamente}, etc.

Se fondant sur ce qu'il considère comme une applicabilité limitée de la suffixation en \emph{-mente}\footnote{Les mêmes impossibilités ont été peu ou prou signalées pour l'espagnol\il{Spanish}~: cf. %
%Egea (1993)
\citet{Egea93}%
%Egea
%
, %
%Garcia Page (1991)
\citet{Garcia91}%
%Garcia Page
%
, %
%Kovacci (1999)
\citet{Kovacci99}%
%Kovacci
%
, %
%Fábregas (2007)
\citet{Fabregas07}%
%Fábregas
%
.}, %
%Scalise (1990) 
\citet{Scalise90} %
%Scalise
%
rejette par conséquent l'hypothèse flexionnelle et lui préfère l'hypothèse dérivationnelle.

Pour ce qui est de D. Ricca, son rejet de l'hypothèse flexionnelle pour expliquer la suffixation en \emph{-mente} en italien est moins irrémédiable. En effet, à l'issue de l'examen des différentes caractéristiques de cette suffixation, %
%Ricca (1998) 
\citet{Ricca98} %
%Ricca
%
conclut qu'elle constitue un bon exemple de cas intermédiaire entre flexion\is{inflection} et dérivation\is{derivation}, et ce, autant d'un point de vue synchronique que d'un point de vue diachronique. Dans %
%Ricca (2004)
\citet{Ricca04}%
%Ricca
%
, il nuance cette position et considère qu'au sein du système morphologique de l'italien, du fait des restrictions de natures morphologique et sémantique auxquelles elle est sujette et malgré sa productivité\is{productivity} très élevée %
%(cf. aussi Gaeta 2008)
\citep[cf. aussi ][]{Gaeta08}%
%Gaeta
%
, la suffixation en \emph{-mente} relève de la dérivation\is{derivation}, même s'il ne s'agit pas là d'une dérivation\is{derivation} prototypique %
%(Ricca 2004~: 473)
\citep[473]{Ricca04}%
%Ricca
%
.
\il{Italian|)}
\is{adverb!in -\textsc{mente}|)}

\subsection{Les adverbes en --\emph{ly} de l'anglais}\label{section:dal:2.2}
\is{adverb!in --\emph{ly}|(}
\il{English|(}

En anglais, la question du statut de la séquence --\emph{ly} figurant dans des adverbes comme \emph{beautifully} ou \emph{rapidly} sous (\ref{ex:Dal:6}) a été abordée de façon récurrente~:

\ea\label{ex:Dal:6}
    \ea She sings beautifully.
    \ex The birds moved rapidly.
\z\z

Les discussions portent sur le statut dérivationnel ou flexionnel de la règle à laquelle est associée la séquence --\emph{ly}, à l'exclusion de toute autre hypothèse. Contrairement à ce qu'on a vu pour -\textsc{mente}, l'hypothèse compositionnelle n'est en effet pas explorée, malgré l'étymon nominal de --\emph{ly}, \emph{lic}, signifiant «~forme, apparence, corps~» en vieil anglais %
%(cf. notamment Jespersen 1954~; Ricca 2015)
\citep[cf. notamment ][]{Jespersen54,Ricca15}%
%Jespersen;Ricca
%
.

\subsubsection{L'hypothèse flexionnelle}\label{section:dal:2.2.1}

Pour les tenants de la piste flexionnelle, que défendent entre autres %
%Hockett (1958~: 110)
\citet[110]{Hockett58}%
%Hockett
%
, %
%Lyons (1968)
\citet{Lyons68}%
%Lyons
%
, 
%Sugioka \& Lehr (1983)
\citet{SugiokaLehr1983}%
%
,
%Miller (1991~: 95)
\citet[95]{Miller91}%
%Miller
%
, %
%Haspelmath (1996~: 49-50)
\citet[49--50]{Haspelmath1996}%
%Haspelmath
%
, %
%Baker (2003~: 230-235)
\citet[230--235]{Baker03}%
%Baker
%
, ou, plus récemment, %
%Giegerich (2012) 
\citet{Giegerich12} %
%Giegerich
%
et %
%Pittner (2015)
\citet{Pittner15}%
%Pittner
%
, les arguments sont en substance les suivants\footnote{On peut encore citer %
%Emonds (1976)
\citet{Emonds76}%
%Emonds
%
, %
%Radford (1988)
\citet{Radford88}%
%Radford
%
, %
%Plag (2003) 
\citet{Plag.2003g} %
%Plag
%
ou %
%Bassac (2004)
\citet{Bassac04}%
%Bassac
%
, qui sont des manuels ayant contribué, en tant que tels, à disséminer la thèse flexionnelle.}~:

\begin{itemize}
\item[---] la productivité\is{productivity} réputée très élevée de la suffixation en --\emph{ly}, que s'accordent à reconnaître tous les travaux qui lui sont consacrés indépendamment du statut qui lui est dévolu. En anglais, tout adjectif est en effet susceptible de donner lieu à un adverbe en --\emph{ly} %
(cf. notamment %
%Bybee 1985
\citealt{Bybee85}%
~: 84, 
%Scalise \& Guevara 2005
\citealt{Scalise05}
~: 159)
%\citep[cf. notamment ][84~; Scalise \& Guevara 2005~: 159]{Bybee85}%
%Bybee
%
, sauf quelques cas régulièrement cités~: adjectifs auxquels correspond un adverbe irrégulier (ex.~: \emph{good} / \emph{well} / *\emph{goodly}) et, surtout, adjectifs terminés par --\emph{ly}\footnote{Cf. %
%Bybee (1985~: 84-5)
\citet[84--85]{Bybee85}%
%Bybee
%
, %
%Anderson (1992~: 195)
\citet[195]{Anderson92}%
%Anderson
%
. Sur l'évitement des adverbes se terminant par la séquence --\emph{lily}, cf. %
%Bauer (1983~; 1992~; 2001)
\citet{Bauer83,Bauer92,Bauer01}%
%Bauer;Bauer;Bauer
%
. Pour un examen détaillé des adjectifs se terminant par la séquence /ly/, cf. %
%Bauer, Lieber \& Plag (2013, chap. 15)
\citet[chap. 15]{Bauer.2013}%
%Bauer-Lieber-Plag
%
.}, encore que l'exemple emblématique \emph{sillily} (sur \emph{silly} «~bête, stupide~»), régulièrement cité comme impossible, soit attesté dans l'\emph{Oxford English Dictionary} ainsi qu'entre autres, dans le \emph{Wester's Online Dictionary}, à côté de, notamment, \emph{burlily, chillily}, \emph{cleanlily}, \emph{comelily}, \emph{deadlily}, \emph{friendlily}
 ou \emph{ghastlily}. Cette contrainte morphophonologique que %
%Štekauer (2005~: 216) 
\citet[216]{Stekauer2005a} %
%Štekauer
%
impute au phénomène de \isi{dissimilation} prohibant la consécution de deux séquences /li/ de part et d'autre d'une frontière constructionnelle, pour autant qu'elle soit avérée, est donc pour le moins faible~;

\item[---] le fait que le recours à un adverbe en --\emph{ly} plutôt qu'au lexème adjectival auquel il est apparenté est motivé par la syntaxe, l'adverbe apparaissant dans des contextes non nominaux, autrement dit dans des contextes impropres à accueillir un modifieur dont le statut adjectival est évident. Par exemple, si les verbes \emph{sing} et \emph{move} plus haut en (\ref{ex:Dal:6}) requièrent un adverbe en --\emph{ly}, les noms \emph{song} et \emph{movement} en (\ref{ex:Dal:7}) ne peuvent cooccurrer qu'avec un authentique adjectif~:

\ea\label{ex:Dal:7}
    \ea She sings beautifully. / Her song is beautiful. A beautiful song.
    \ex The birds moved rapidly. / Their movements are rapid. Rapid movements.
\z\z

\item[---] le traitement différencié que nécessiteraient les adverbes en --\emph{ly} relativement aux adverbes dépourvus d'un affixe en anglais, si l'on traitait les premiers comme dérivationnels. Selon %
%Giegerich (2012)
\citet{Giegerich12}%
%Giegerich
%
, un tel traitement aurait pour effet de masquer des généralisations pouvant être faites sur la classe des adverbes non affixés, dans lesquels il voit des adjectifs (on reviendra sur ce point dans le § \ref{section:dal:3.3.1}.)~;

\item[---] le caractère mutuellement exclusif de la suffixation en --\emph{ly} et de celles en --\emph{er} et --\emph{est} marquant la comparaison et le haut degré. Cette observation fait conclure à %
%Hockett (1958) 
\citet{Hockett58} %
%Hockett
%
que les formes adverbiales en --\emph{ly} relèvent du même paradigme\is{paradigm} que les formes adjectivales en --\emph{er} et en --\emph{est}, donc que, comme --\emph{er} et --\emph{est}, --\emph{ly} est flexionnel %
%(cf. aussi Giegerich 2012)
\citep[cf. aussi ][]{Giegerich12}%
%Giegerich
%
.
S'agissant du point d'achoppement que peut constituer la catégorisation comme adverbes des mots en --\emph{ly} relativement aux adjectifs auxquels ils sont liés --~la flexion\is{inflection} est en effet réputée conserver intègre la catégorie lexicale du lexème sur lequel elle opère~--, deux explications sont en concurrence parmi les partisans de la thèse flexionnelle~:

\begin{itemize}
\item[---] %
%Haspelmath (1996) 
\citet{Haspelmath1996} %
%Haspelmath
%
fait l'hypothèse de l'existence de procédés flexionnels pouvant agir sur la catégorie lexicale des inputs (il parle de \emph{transpositional inflection}), tels les participes présents en allemand ou les adjectifs en position épithète en turc, et étend précisément cette hypothèse aux adverbes en --\emph{ly} de l'anglais %
%(cf. aussi Bybee 1985)
\citep[cf. aussi ][]{Bybee85}%
%Bybee
%
~;

\item[---] %
%Sugioka \& Lehr (1983)
\citet{SugiokaLehr1983} %
%
 mettent en question la pertinence même de la catégorie de l'adverbe %
%(cf. aussi entre autres Fábregas 2014)
\citep[cf. aussi entre autres ][]{Fabregas14}%
%Fábregas
%
, et considèrent que ce qu'on a coutume de nommer «~adverbe en --\emph{ly}~» n'est rien d'autre que la forme du paradigme\is{paradigm} de l'adjectif qui est sélectionnée dans un contexte non nominal.

\end{itemize}
\end{itemize}
On reviendra plus longuement sur cette question de la catégorisation comme adverbe dans le § \ref{section:dal:3.3.1}, lorsqu'il s'agira de déterminer le statut dérivationnel ou flexionnel de la règle dont \emph{--ment} est l'exposant\is{exponent} en français.

\subsubsection{L'hypothèse dérivationnelle}\label{section:dal:2.2.2}

Pour les tenants de la piste dérivationnelle dont font partie %
%Zwicky (1995) 
\citet{Zwicky1995} %
%Zwicky
%
--~en réponse à %
%Sugioka \& Lehr (1983)
\citet{SugiokaLehr1983}%
%
~--, et %
%Payne \& al. (2010)
\citet{Payne10}%
%Payne-al.
%
, que reprend %
%Ricca (2015)
\citet{Ricca15}%
%Ricca
%
, les arguments sont les suivants~:

\begin{itemize}
\item[---] il est faux qu'adjectifs et adverbes en --\emph{ly} correspondants apparaissent dans des environnements syntaxiques mutuellement exclusifs~: noms pour les adjectifs, autres environnements pour les adverbes. C'est là l'argument central de %
%Payne \& al. (2010)
\citet{Payne10}%
%Payne-al.
%
, qui considèrent que, à condition d'être postposés aux noms qu'ils modifient, les adverbes en --\emph{ly} sont aptes à figurer dans la fonction de modifieurs de noms, comme, par exemple, \emph{globally} et \emph{internationally} dans les exemples sous (\ref{ex:Dal:8}), repris de cette étude~:


\ea\label{ex:Dal:8}
    \ea {[}The unique role \textbf{globally} of the Australian Health Promoting Schools Association{]}, as a non-government organization specifically established to promote the concept of the health promoting school, is described.

    \ex The NHS and {[}other health organisations \textbf{internationally}{]} clearly need methodologies to support benefit analysis of merging healthcare organisations.
\z\z

\item[---] les adverbes en --\emph{ly} n'héritent pas tous de la structure argumentale des adjectifs auxquels ils sont apparentés, comme l'indique le contraste \emph{proud of his daugther} / *\emph{proudly of his daughter};

\item[---] certains types sémantiques d'adjectifs ne donnent pas lieu à la production d'un adverbe en --\emph{ly}. On retrouve, mentionnées ici, les catégories d'adjectifs citées par %
%Scalise (1990) 
\citet{Scalise90} %
%Scalise
%
pour l'italien~: adjectifs chromatiques~; adjectifs exprimant une propriété sensorielle, sauf à ce que l'adverbe ait une valeur métaphorique (ex. \emph{warmly})~; etc.

\end{itemize}

\is{adverb!in --\emph{ly}|)}
\il{English|)}
\subsection{Discussion}\label{section:dal:2.3}

L'état de l'art qui précède a mis en évidence au moins un point~: le statut des règles morphologiques produisant des adverbes à partir d'adjectifs dans plusieurs langues romanes et germaniques a donné lieu à des discussions nourries, parfois virulentes, et la question n'est toujours pas résolue. À cet égard, on ne peut qu'être surpris qu'en français, peu de travaux se soient penchés sur le statut de la règle à laquelle ressortit la séquence adverbiale \emph{-ment.}

Il est par ailleurs remarquable que, dans les travaux dont il a été question dans cet état de l'art, l'hypothèse dérivationnelle n'ait jamais été abordée positivement~: soit elle constitue une réponse aux faiblesses de l'hypothèse compositionnelle (cf. § \ref{section:dal:2.1.2}), soit elle tempère les généralisations de l'hypothèse flexionnelle (cf. § \ref{section:dal:2.1.4} et § \ref{section:dal:2.2}), mais elle met rarement, pour ne pas dire jamais, en avant d'arguments irréfutables montrant que les adverbes en -\textsc{mente} ou en --\emph{ly} résultent de l'application d'une règle de construction de lexèmes\is{lexeme formation rule}, partant, que ces adverbes sont des lexèmes à part entière.

\section{Quel statut pour les adverbes en \emph{--ment}\is{adverb!in --\emph{ment}} du français~?}\label{section:dal:3}
\il{French|(}
Étant admis que l'hypothèse compositionnelle est exclue en français --~l'argument de l'élision ou de la mise en facteur commun de \emph{--ment} entre plusieurs adverbes, jugé décisif par les partisans de cette hypothèse en espagnol, ne tient pas pour le français moderne\footnote{%
%Meyer-Lübke (1894~: 638) 
\citet[638]{Meyer-Lubke} %
%Meyer-Lübke
%
signale cette possibilité en ancien français au travers de l'exemple «~Ainzi fu la guere maintenue Si cruel e si longuement~», également cité dans %
%Karlsson (1981~: 60)
\citet[60]{Karlsson81}%
%Karlsson
%
.}~\mbox{--,} l'alternative est la même que pour --\emph{ly} en anglais~: flexion\is{inflection} ou dérivation\is{derivation}~?, à moins que les adverbes en \emph{--ment}\is{adverb!in --\emph{ment}} du français ne relèvent de l'une de ces «~zones grises~» %
%(Bybee 1985)
\citep{Bybee85}%
%Bybee
%
, indécidables entre flexion\is{inflection} et dérivation\is{derivation}.

Pour tenter d'apporter des éléments de réponse à cette question, je me propose de reprendre dans ce qui suit les attendus d'une règle de construction de lexèmes\is{lexeme formation rule}. Auparavant, je discuterai de la forme du radical\is{stem} de l'adjectif à laquelle s'attache \emph{--ment} afin d'évacuer cette question de la discussion\emph{.}

\subsection{Forme du radical\is{stem} de l'adjectif}\label{section:dal:3.1}

La séquence \emph{--ment} du français est réputée s'appliquer à la forme féminine de l'adjectif auquel est apparenté l'adverbe %
%(cf. entre autres Guimier 1996~; Molinier \& Levrier 2000~: 28-29)
\citep[cf. entre autres ][28--29]{Guimier96a,Molinier00}%
%Guimier;Molinier-Levrier
%
, autrement dit à une forme fléchie. Comme on l'a vu précédemment, ce même constat effectué pour, entre autres, l'espagnol et l'italien a été porté au crédit de l'hypothèse flexionnelle et de celle de l'affixe syntagmatique, dans la mesure où une règle dérivationnelle est supposée ne pas pouvoir s'appliquer postérieurement à une opération de flexion\is{inflection}.

Or, la notion aronovienne de morphome %
%(Aronoff 1994)
\citep{Aronoff94}%
%Aronoff
%
, selon laquelle certaines unités morphologiques n'expriment aucune propriété morphosyntaxique ou sémantique --~ce sont de pures formes, ou, selon les termes de %
%Bonami \& Boyé (2005~: 82)
\citet[82]{Bonami05}%
%Bonami-Boyé
%
, de «~purs objets morphologiques~»~--, offre une explication élégante et neutre vis-à-vis de l'attribution d'un quelconque statut à la règle à laquelle est associée la séquence \emph{--ment}.

Recourant à la notion de morphome, %
%%
%Bonami \& Boyé (2005)
\citet{Bonami05} %
%Bonami-Boyé
%
font l'hypothèse que les adjectifs du français possèdent un espace thématique\is{stem!stem space} constitué de deux thèmes\is{stem}, identiques ou distincts, \rephrase{n'exprimant}{qui n'expriment} aucune propriété morphosyntaxique, et \rephrase{servant}{qui servent} % sinon, "constuire" dans la marge
à construire les cinq formes de leur paradigme\is{paradigm}~: les quatre formes traditionnelles faisant intervenir les catégories de genre et de nombre, plus une forme de liaison du masculin singulier en position prénominale. Le tableau \ref{tab:Dal:1}, emprunté à %
%Bonami \& Boyé (2005)
\citet{Bonami05}%
%Bonami-Boyé
%
, donne les thèmes\is{stem} de quelques adjectifs du français~:

\begin{table}
\begin{tabular}{lll}
\lsptoprule
Lexème & Thème\is{stem} 1 & Thème\is{stem} 2 \\
\midrule
\textsc{livide}  & /livid/ & /livid/ \\
\textsc{sec}     & /sɛk/ & /sɛʃ/ \\
\textsc{vif}     & /vif/ & /viv/ \\
\textsc{vieux}   & /vjø/ & /vjɛj/ \\
\textsc{nouveau} & /nuvo/ & /nuvɛl/ \\
\lspbottomrule
\end{tabular}
\caption{Espace thématique\is{stem!stem space}  de quelques adjectifs en français}
\label{tab:Dal:1}
\end{table}

En flexion\is{inflection}, le thème\is{stem} 1 est utilisé pour le masculin, hors liaison (\emph{arbre sec}~; \emph{regard vif}~; \emph{vieux fauteuil}~; \emph{nouveau manteau})~; le thème\is{stem} 2 l'est pour le féminin (\emph{branche sèche}~; \emph{riposte vive}~; \emph{vieille ferme}~; \emph{nouvelle tenue}). Pour ce qui est de la forme de liaison au masculin singulier en position prénominale, selon les adjectifs, sont mobilisés le thème\is{stem} 1 (\emph{sec entretien}~: {[}sɛkɑ̃trətjɛ{]}) ou le thème\is{stem} 2 (\emph{vieil avion~}: {[}vjɛjavjɔ̃{]}).

Pour rendre compte de la forme du radical\is{stem} des adverbes en \emph{--ment}\is{adverb!in --\emph{ment}}, la solution, amorcée dans %
%Dal (2007) 
\citet{Dal07} %
%Dal
%
et largement développée dans %
%Boyé \& Plénat (2015)
\citet{Boye15}%
%Boyé-Plénat
%
, consiste à ajouter un troisième thème\is{stem} à l'espace thématique\is{stem!stem space}  de l'adjectif en français. Selon les cas, ce troisième thème\is{stem} peut être (i) homophone du thème\is{stem} 2, (ii) homophone du thème\is{stem} 1, (iii) différent des thèmes\is{stem} 1 et 2. Majoritairement, les adverbes en \emph{--ment}\is{adverb!in --\emph{ment}} mettent en jeu un thème\is{stem} homophone du thème\is{stem} 2 (par exemple, /sɛʃmɑ̃/ relativement à /sɛʃ/), autrement dit le thème\is{stem} qui sert aussi très majoritairement à former le féminin des adjectifs. Cette observation explique l'assertion récurrente selon laquelle, dans les adverbes, la séquence \emph{--ment} s'appliquerait à une forme fléchie au féminin singulier ainsi, d'ailleurs, que les hésitations des scripteurs lorsque les formes de masculin et féminin sont homophones sans être homographes~: à titre d'exemple, /ʒolimɑ̃/, que les scripteurs orthographient \emph{joliment} (9 millions d'occurrences sur la Toile au moyen du moteur de recherche Google fin juin 2017) ou \emph{joliement} (300~000 occurrences). D'autres choix de radicaux\is{stem} sont toutefois possibles, comme le montrent %
%Boyé \& Plénat (2015)
\citet{Boye15}%
%Boyé-Plénat
%
~:

\begin{itemize}
\item[---] lorsque le thème\is{stem} 1 de l'adjectif se termine par /ɑ̃/, c'est à un homophone de ce thème\is{stem} que s'applique le plus souvent \emph{--ment}, modulo la dénasalisation de la voyelle finale %
%(cf. Pagliano 2003)
\citep[cf.][]{Pagliano03}%
%Pagliano
%
~: cf. \emph{méchamment} préféré à \emph{méchantement} (qui compte quand même une petite centaine d'occurrences sur la Toile fin juin 2017), \emph{violemment} préféré, en français moderne, à \emph{violentement}, sauf quand cette finale est précédée d'une nasale ou d'une labiale, auquel cas le thème\is{stem} sélectionné est préférentiellement homophone du thème\is{stem} 2 (cf. \emph{charmantement} préféré à \emph{charmamment}\footnote{Si l'on exclut les emplois en mention et les pages redondantes, \emph{charmantement} (parfois sous la forme \emph{charmentement}) compte environ 160 occurrences sur la Toile au 1\textsuperscript{er} octobre 2016 (ex.~: «~La pluie continuait de tomber. J'étais \textbf{charmantement} abritée~»), contre une dizaine pour \emph{charmamment / charmament} (ex.~: «~Évidemment un peu vieux jeu, \textbf{charmamment} démodé~»)\emph{.}})~;

\item[---] dans d'autres cas, le thème\is{stem} 3 est différent des thèmes\is{stem} 1 et 2, et se caractérise par l'émergence d'un /e/ concaténé au thème\is{stem} 2 servant à former le féminin (\emph{obtusé\-ment})\footnote{L'émergence de ce /e/ n'est pas aléatoire~: il apparaît, de façon récurrente, après une consonne nasale (\emph{cochonnément, communément,} \emph{conformément, opportunément,} \emph{uniformément\ldots{}}) ou après une fricative, le plus souvent sifflante, sonore (\emph{concisément, confusément, précisément\ldots{}}) ou sourde (\emph{densément}, \emph{expressément}\ldots{}), plus rarement liquide (\emph{aveuglément}) ou vibrante (\emph{obscurément}). S'agissant du premier cas, l'émergence de ce /e/ pourrait avoir pour objectif de satisfaire la contrainte dissimilative déjà citée. L'option prise ici, comme dans %
%Boyé \& Plénat (2015)
\citet{Boye15}%
%Boyé-Plénat
%
, est que /e/ fait partie du radical\is{stem}. Je renvoie à ce travail pour une argumentation.}, ou est simplement imprédictible (\emph{brièvement}) du point de vue synchronique.
\end{itemize}
Le tableau \ref{tab:Dal:2}, adapté de %
%Boyé \& Plénat (2015)
\citet{Boye15}%
%Boyé-Plénat
%
, récapitule ces résultats.
\medskip

\begin{table}
\begin{tabular}{llll}
\lsptoprule
Lexème & Thème\is{stem} 1 & Thème\is{stem} 2 & Thème\is{stem} 3 \\
\midrule
\textsc{joli} & \cellcolor{gray!30}   /ʒoli/  &\cellcolor{gray!30} /ʒoli/     &\cellcolor{gray!30} /ʒoli/ \\
\textsc{sec} &    /sɛk/    &\cellcolor{gray!30} /sɛʃ/      &\cellcolor{gray!30} /sɛʃ/ \\
\textsc{charmant} &    /ʃaʁmɑ̃/ &\cellcolor{gray!30} /ʃaʁmɑ̃t/   &\cellcolor{gray!30} /ʃaʁmɑ̃t/ \\
\textsc{méchant} &    /meʃɑ̃/  & /meʃɑ̃t/    &\cellcolor{gray!30} /meʃɑ̃m/ \\
\textsc{obtus} &    /ɔpty/  & /ɔptyz/    &\cellcolor{gray!30} /ɔptyse/ \\
\textsc{bref} &    /bʁɛf/  & /bʁɛv/     &\cellcolor{gray!30} /bʁijɛv/ \\
\lspbottomrule
\end{tabular} % \cellcolor{gray!30}
\caption{Proposition d'ajout d'un thème\is{stem} 3 dans l'espace thématique\is{stem!stem space}  de quelques adjectifs en français.}
\label{tab:Dal:2}
\end{table}


La solution de l'ajout d'un troisième thème\is{stem} au paradigme\is{paradigm} de l'adjectif pour former des adverbes en \emph{--ment}\is{adverb!in --\emph{ment}}, résumée ici à partir de %
%Boyé \& Plénat (2015)
\citet{Boye15}%
%Boyé-Plénat
%
, est orthogonale à la question du statut de la règle à laquelle est associé l'exposant\is{exponent} \emph{--ment}, dans la mesure où tant les règles flexionnelles\is{inflection!rule}  que les règles dérivationnelles peuvent sélectionner tel ou tel thème\is{stem} de l'espace thématique\is{stem!stem space} , de façon exclusive ou privilégiée %
%(cf. Bonami et al. 2009)
\citep[cf.][]{Bonami2009a}%
%Bonami-al.
%
. Elle permet par conséquent d'évacuer de la discussion la forme du radical\is{stem} à laquelle s'adjoint la forme \emph{--ment} et évite de tirer argument de cette forme identifiée, à tort, comme étant un féminin~: plus exactement, si le radical\is{stem} affecte le plus souvent la forme d'un féminin, c'est parce que la formation d'adverbes en \emph{--ment}\is{adverb!in --\emph{ment}}, quel qu'en soit le statut, et la formation du féminin de l'adjectif en français opèrent toutes deux de façon privilégiée sur le thème\is{stem} 2, ou sur un homophone de ce thème\is{stem}.

On note du reste que, sans toutefois mobiliser explicitement la notion de morphome,  % Problème avec la gestion du "ten" ???
%Hacken (2014~: 19) 
\citet[19]{Hacken14} %
%Hacken
%
considère pareillement que, pour concilier les données du français et l'universel 28 de \citeauthor{Greenberg1963}, une solution est de considérer que, dans \emph{lentement}, \emph{lente} est une variante du radical\is{stem} de l'adjectif. Pour sa part, %
%Ricca (2015~: 1392) 
\citet[1392]{Ricca15} %
%Ricca
%
recourt à la notion de morphome pour expliquer la voyelle /a/ qui clôt le radical\is{stem} de certains adverbes en -\textsc{mente}\is{adverb!in -\textsc{mente}} en italien, portugais et espagnol.

\subsection{Attendus d'une Règle de Construction de Lexèmes\is{lexeme formation rule}}\label{section:dal:3.2}

Une Règle de Construction de Lexèmes\is{lexeme formation rule} (désormais, RCL\is{lexeme formation rule!LFR}) peut être schématiquement définie comme un ensemble de régularités observables entre deux séries de lexèmes dont les uns, les outputs, ont un degré de complexité supérieur aux autres, les inputs.

Selon %
%Fradin (2003)
\citet{Fradin03}%
%Fradin
%
, le schéma de représentation d'une RCL\is{lexeme formation rule!LFR} relevant du procédé de dérivation\is{derivation} est le suivant (Tableau \ref{tab:Dal:3})~:

\begin{table}
\begin{tabular}{lcl}
\lsptoprule
Inputs & \hspace*{3cm} & Outputs \\
\midrule 
Phonologie 1 & & Phonologie 2 \\
Syntactique 1 & $\Leftrightarrow$ & Syntactique 2 \\
Sémantique 1 & & Sémantique 2 \\

\lspbottomrule
\end{tabular}
\caption{Schéma de représentation d'une RCL\is{lexeme formation rule!LFR} relevant du procédé de dérivation\is{derivation} selon %
%Fradin (2003)
\citet{Fradin03}%
%Fradin
%
.}
\label{tab:Dal:3}
\end{table}

Ce schéma revient à dire qu'une RCL\is{lexeme formation rule!LFR} opère sur trois plans~: le plan phonologique, le plan syntaxique et le plan sémantique.

De façon générale, des contraintes de différents types peuvent opérer sur les inputs et sur les outputs. Si l'on exclut les contraintes phonologiques qui opèrent davantage au niveau de tel ou tel lexème (ou ensemble de lexèmes) particulier qu'au niveau de la règle en tant que telle, pour l'essentiel, il s'agit~:

\largerpage
\begin{itemize}
\item[---] de contraintes sémantiques~: chaque procédé constructionnel s'applique à un type sémantique de bases (par exemple, bases exprimant des propriétés, référant à des événements, des parties naturelles, etc.), ou demande des bases qu'il sélectionne qu'elles-mêmes relèvent (ou ne relèvent pas) d'un certain type sémantique. Pareillement, le sens des outputs est une fonction du sens des inputs, cette fonction se caractérisant par une constante --~celle, précisément, qui enregistre la contribution sémantique de la RCL\is{lexeme formation rule!LFR}~-- et par une variable, représentée par le sens de l'input~;

\item[---] de contraintes syntaxiques --~une RCL\is{lexeme formation rule!LFR} s'applique sur un certain type catégoriel de bases et forme un certain type catégoriel de dérivés~--, qui peuvent être vues comme une conséquence des contraintes sémantiques %
%(cf., notamment, Dal 2004)
\citep[cf., notamment,][]{Dal04}%
%Dal
%
.
\end{itemize}
D'autres contraintes peuvent jouer (contraintes historiques, pragmatiques, notamment), nous les laissons de côté ici.

S'agissant de la règle qui forme les adverbes en \emph{--ment}\is{adverb!in --\emph{ment}} à partir d'adjectifs en français, une fois la question de la forme le plus souvent féminine du radical\is{stem} résolue grâce au recours à la notion de morphome et l'ajout d'un troisième thème\is{stem} dans l'espace thématique\is{stem!stem space}  de l'adjectif, il s'agit désormais de déterminer si les contraintes en entrée et en sortie dont elle s'assortit satisfont ce que demande une RCL\is{lexeme formation rule!LFR}.

\subsection{Examen}\label{section:dal:3.3}
\largerpage
\subsubsection{Contraintes syntaxiques}\label{section:dal:3.3.1}

\subsubsubsection{Contraintes syntaxiques d'entrée}

La règle dont \emph{--ment} est l'exposant\is{exponent} prend très majoritairement en entrée des d'adjectifs (notons cette propriété P\textsubscript{1}).

Pour donner un ordre d'idée, le corpus réuni par %
%Pagliano (2003) 
\citet{Pagliano03} %
%Pagliano
%
compte 2746 adverbes dont 2725 formés à partir d'adjectifs ou de participes, soit plus de 99\%.

Le 1\% restant est constitué d'adverbes figurant~:

\begin{enumerate}[label=(\roman*)]
\item dans des séquences formulaires du type \emph{X-ment vôtre} ou \emph{X-ment parlant}, comme dans les exemples sous (\ref{ex:Dal:9}) relevés sur la Toile\footnote{Sur la morphologie des séquences en \emph{X-ment parlant} et \emph{X-ment vôtre}, cf. %
%Boyé \& Plénat (2015)
\citet{Boye15}%
%Boyé-Plénat
%
, ainsi que, pour ces dernières, %
%Mora (2007)
\citet{Mora07}%
%Mora
%
.}~\emph{~}:

\ea\label{ex:Dal:9}
    \ea\label{ex:Dal:9a} Internet'ment vôtre~; rock'n'roll'ment vôtre~; jazz'ment vôtre~; meuh..ment vôtre
    \ex\label{ex:Dal:9b} Le script est crade \textbf{HTML ment} parlant.
    \ex\label{ex:Dal:9c} Il n'est pas bizarre,  \textbf{marketing-ment} parlant, de faire ça.~
\z\z

\item dans des créations ludiques, comme \emph{ordinateurement} ou \emph{mousquetairement} sous (\ref{ex:Dal:10}), également empruntées à la Toile~:

\ea\label{ex:Dal:10}
    \ea\label{ex:Dal:10a} Protection contre les maladies \textbf{ordinateurement} transmissibles.
    \ex\label{ex:Dal:10b} Blafard de teint, ses cheveux aplatis, sa barbe pointue et sa moustache «~\textbf{mousquetairement}~» retroussée rutilent comme l'or.
\z\z

\newpage 
\item dans des adverbes désanthroponymiques, comme les exemples littéraires \emph{baudelairement} ou \emph{lamartinement} sous (\ref{ex:Dal:11}) %
%(cf. Amiot \& Flaux 2005)
\citep[cf.][]{Amiot05}%
%Amiot-Flaux
%
~:

\ea\label{ex:Dal:11}
    \ea Je suis dans un jour où je vois tout idéalement et douloureusement, et enfin, s'il m'est possible de m'exprimer ainsi, \textbf{lamartinement} (Sainte-Beuve, Portr. Littér.)

    \ex Une manière de fatalité (...) qu'à présent il nomme moins \textbf{baudelairement} le train-train de l'existence (Verlaine, \emph{Œuvres posthumes})
\z\z
\end{enumerate}
Une hypothèse est que ces séquences soient formées par analogie %
%(cf. Dal 2003) 
\citep[cf.][]{Dal2003a} %
%Dal
%
avec des séquences mettant en jeu un adverbe à support adjectival\footnote{Dans le cadre de la grammaire de construction, une autre explication, non incompatible avec celle qui est proposée ici, serait que \emph{--ment} sous (\ref{ex:Dal:9})/(\ref{ex:Dal:11}) force une lecture adjectivale de l'item auquel il est concaténé %
%(cf. Audring \& Booij 2016)
\citep[cf.][]{audringbooij16}%
%Audring-Booij
%
.}~:

\begin{itemize}
\item[---] il est assez probable que les formules de politesse ludiques telles celles sous (\ref{ex:Dal:9a}) fassent écho à des formules à support indéniablement adjectival comme \emph{cordialement vôtre} ou \emph{amicalement vôtre}~;

\item[---] dans les séquences en \emph{Xment parlant}, les adverbes sont principalement formés à partir d'adjectifs relationnels (cf., relevés sur la Toile, \emph{grammaticalement parlant}, \emph{philosophiquement parlant}, \emph{financièrement parlant}, \emph{culturellement parlant}). S'il est plus difficile de trouver un chef de file pour les séquences sous (\ref{ex:Dal:9b}) que pour celles sous (\ref{ex:Dal:9a}), on peut néanmoins considérer que ces séquences à support adjectival leur aient servi de modèle~;

\item[---] dans une séquence comme celle sous (\ref{ex:Dal:10a}), on ne peut pas ne pas remarquer le jeu échoïque avec la séquence quasi-figée \emph{maladies sexuellement transmissibles} (ce même jeu échoïque avec une séquence quasi-figée s'observe pareillement dans, par exemple, «~Paysage \emph{ordinateurement} modifié~»\footnote{L'analogue est bien sûr ici \emph{génétiquement modifié}.}, également relevé sur la Toile)~;

\item[---] enfin, tant avec l'exemple sous (\ref{ex:Dal:10b}) qu'avec les désanthroponymiques sous (\ref{ex:Dal:11}), la suffixiformité adjectivale de la finale du nom support (\emph{mousque\-taire}, \emph{Baudelaire}, \emph{Lamartine}) est un facteur favorisant l'émergence de l'adverbe %
(%
%Amiot \& Flaux 2005
\citeauthor{Amiot05}%
%
, font une remarque analogue pour les désanthroponymiques)
%\citep[, font une remarque analogue pour les désanthroponymiques]{Amiot05}%
%Amiot-Flaux
%
\footnote{Dans certaines langues, la séquence finale de séquences paraphrasables par «~à la manière de X~», où X est un nom, est traitée comme un marqueur du cas essif, donc comme flexionnelle %
%(par exemple, en hongrois --\emph{kent} dans \emph{turistakent} «~à la manière d'un touriste~»~; cf. Ricca 2015~: 1399)
\citep[par exemple, en hongrois --\emph{kent} dans \emph{turistakent} «~à la manière d'un touriste~»~; cf. ][1399]{Ricca15}%
%Ricca
%
.}. Lorsque l'anthroponyme n'a pas de finale suffixiforme, la tendance est de transiter par un adjectif relationnel (c'est le cas dans cet exemple relevé sur la Toile~: «~(\ldots{}) en mettant \textbf{molièresquement} tous les rieurs de son côté~»).

\end{itemize}

En somme, ce 1\% résulterait d'une pression lexicale, et serait formé par analogie avec des adverbes (ou des séquences comportant un adverbe) à support authentiquement adjectival.

Relativement \rephrase{à la question du statut}{au statut} de la règle à laquelle --\emph{ment} est associé, la contrainte d'entrée P\textsubscript{1} --~\emph{-ment} s'applique massivement à des adjectifs~-- n'est pas décisive~: si les adverbes en \emph{--ment}\is{adverb!in --\emph{ment}} sont produits par une règle dérivationnelle, cette dernière prendrait des adjectifs en entrée~; s'ils le sont par une règle de réalisation de lexème\is{inflection!rule} (par la flexion\is{inflection}, donc), on s'attend à ce qu'ils soient des mots-formes\is{wordform} d'une catégorie unique, qui serait en l'occurrence celle des adjectifs.

\subsubsubsection{Contraintes syntaxiques de sortie}

Admettons donc que les supports des mots en \emph{--ment} soient des adjectifs. Il n'en reste pas moins que ces mots sont catégorisés comme adverbes. Appelons cette propriété P\textsubscript{2}. Or, l'une des propriétés régulièrement invoquées pour différencier la flexion\is{inflection} de la dérivation\is{derivation} est que seules les règles dérivationnelles peuvent former des lexèmes relevant d'une catégorie différente de celle des lexèmes qu'elles prennent en entrée~: on tiendrait là l'argument décisif en faveur du caractère dérivationnel de la règle ayant \emph{--ment} pour exposant\is{exponent}.

Toutefois, on a vu plus haut que, pour %
%Haspelmath (1996) 
\citet{Haspelmath1996} %
%Haspelmath
%
qui suit en cela la proposition amorcée dans %
%Bybee (1985)
\citet{Bybee85}%
%Bybee
%
, la flexion\is{inflection} peut avoir un effet sur la catégorie des sorties et que, selon lui, en anglais, la suffixation en --\emph{ly} serait précisément l'une de ces règles flexionnelles\is{inflection!rule}  transpositionnelles %
(%
%Scalise 1988
\citealt{Scalise1988} %
%
 envisage également le cas de règles flexionnelles\is{inflection!rule}  dont les outputs ne relèveraient pas de la catégorie des inputs)%
%\citep[ envisage également le cas de règles flexionnelles dont les outputs ne relèveraient pas de la catégorie des inputs]{Scalise1988}%
%Scalise
%
. La formation d'adverbes en \emph{--ment}\is{adverb!in --\emph{ment}} du français pourrait être passible de la même explication.

Par ailleurs, même si l'on récuse cette possibilité, on a déjà souligné plus haut la difficulté à cerner de façon satisfaisante la catégorie de l'adverbe, qui se caractérise, pour le moins, par une très grande hétérogénéité %
%(Ricca 2015)
\citep{Ricca15}%
%Ricca
%
, au point que certains linguistes remettent en question son existence même, parfois de façon péremptoire. C'est le cas d'%
%Aronoff (1994~: 10)
\citet[10]{Aronoff94}%
%Aronoff
%
, qui affirme~: «~I assume without argument that adverbs are adjectives~».

\largerpage
Reprenons les principaux arguments avancés, ou pouvant l'être, en faveur de la remise en cause, totale ou partielle, de la catégorie de l'adverbe.

Pour %
%Giegerich (2012)
\citet{Giegerich12}%
%Giegerich
%
, les arguments sont d'abord morphologiques. Pour lui, en anglais, ce qu'il est convenu d'appeler «~adverbes~» ne présente aucune propriété morphologique qui distinguerait cette catégorie de celle des adjectifs~: il en conclut que les adverbes sont des formes d'adjectifs. Cette «~single-category claim~», qui vaut tant pour les adverbes en --\emph{ly} que pour les adverbes dépourvus de marque affixale (il fait de ces derniers des adjectifs non fléchis), expliquerait le fait que, contrairement aux catégories du nom, de l'adjectif et du verbe, la catégorie de l'adverbe ne puisse pas servir d'input à une quelconque règle dérivationnelle, compte tenu de l'ordre d'application dérivation\is{derivation}, puis flexion\is{inflection}\footnote{~Les contre-exemples apparents qu'il reprend à %
%Payne \& al. (2010~: 63) 
\citet[63]{Payne10} %
%Payne-al.
%
tels \emph{soonish, soonness}, \emph{seldomness}, \emph{unseldom} mettent en jeu des affixations qui, précisément, s'appliquent typiquement à des adjectifs.}. Parallèlement, l'hypothèse d'une catégorie unique réunissant adjectifs et adverbes expliquerait que si, pour le français, l'on excepte les cas à la marge comme \emph{baudelairement} vus plus haut, aucun adverbe ne dérive de nom ou de verbe, là où, pour les catégories lexicales majeures authentiques que sont les noms, les adjectifs et les verbes, toutes les combinaisons sont deux à deux possibles.

De surcroît, alors que les noms, adjectifs et verbes peuvent servir d'inputs à plus d'une règle dérivationnelle, dans l'hypothèse de l'attribution d'un statut dérivationnel à la suffixation en --\emph{ly}, l'adverbe serait atypique en ceci qu'outre la conversion d'adjectif à adverbe (on reviendra plus loin sur ce point), il ne mettrait en jeu que cette seule suffixation.

La situation est stricto sensu transposable au français~: il apparaît que la catégorie de l'adverbe ne sert pas d'input au système constructionnel du français et qu'en sortie, une seule marque, \emph{--ment}, appliquée à la seule catégorie de l'adjectif, serait possible, en plus de la conversion.

Comme pour l'anglais, en faisant de l'adverbe un cas d'espèce de l'adjectif et de \emph{--ment} une marque flexionnelle, la position atypique des adverbes dans le système dérivationnel du français trouve une explication~: l'adverbe ne peut pas servir d'input à une règle dérivationnelle, parce que c'est un mot-forme\is{wordform} et non pas un lexème, et il ne constitue la sortie que de la catégorie adjectivale, parce qu'il occupe une case du paradigme\is{paradigm} de cette catégorie.

Pour %
%Giegerich (2012)
\citet{Giegerich12}%
%Giegerich
%
, du point de vue de la flexion\is{inflection}, l'adverbe en anglais ne présente pas davantage de propriétés qui le distingueraient de l'adjectif. La variation morphologique en degré est possible pour l'adverbe, mais elle n'affecte que les adverbes dépourvus de --\emph{ly}, et les marques flexionnelles utilisées sont précisément celles que connaît également l'adjectif (\emph{big~}: \emph{bigger}, \emph{biggest}~; \emph{soon~}: \emph{sooner}, \emph{soonest}). Comme on l'a déjà vu, pour sa part, le fait que les adverbes en --\emph{ly} n'acceptent pas de marquage en degré au moyen de marques flexionnelles s'explique dans l'hypothèse flexionnelle défendue par Giegerich, puisque, en tant que mots-formes\is{wordform}, ils occupent une case du paradigme\is{paradigm} de l'adjectif~: les exposants\is{exponent} --\emph{er}, --\emph{est} et --\emph{ly} permettant d'instancier des mots-formes\is{wordform} du même paradigme\is{paradigm}, ils sont mutuellement exclusifs.

Pour ce qui est du français, la situation est comparable, au moins en partie, dans la mesure où l'adverbe y est réputé invariable. %
%Hummel (2013 et 2014) 
\citet{Hummel13,Hummel14} %
%Hummel;Hummel
%
remet en effet en cause l'invariabilité des «~short adverbs~», en même temps que celle de l'appartenance de ces derniers à la catégorie de l'adverbe. Pour lui comme pour %
%Abeillé \& Godard (2004)
\citet{Abeille04}%
%Abeillé-Godard
%
, \emph{gras} dans \emph{manger gras} ou \emph{direct} dans \emph{Pierre et Marie vont direct au café} ne sont pas des adverbes, mais des «~adjectifs non marqués~» ou «~adjectifs en fonction adverbiale~». Son argumentation tout à la fois convoque des arguments diachroniques et exploite des données de corpus actuelles, dans une perspective variationniste. En effet, dans les langues qui connaissent la flexion\is{inflection} de l'adjectif comme le français, une tendance observée dans la langue contemporaine dans des emplois non standard renoue avec celle qui a eu cours jusqu'au XVII\textsuperscript{e} siècle d'accorder les adverbes courts. Cet accord s'observe avec le sujet ou avec l'objet interne, comme on le voit sous (\ref{ex:Dal:12a}), relevé sur la Toile, et (\ref{ex:Dal:12b}), emprunté à %
%Hummel \& Gadzik (2014)
\citet{Hummel14b}%
%
~:\newpage 

\ea\label{ex:Dal:12}
    \ea\label{ex:Dal:12a} Ils jouent forts, et souvent faux, ponctuent les chansons d'exclamations en espagnol, sont d'une bonne humeur resplendissante et communicative.

    \ex\label{ex:Dal:12b} Je suis sur le point d'arrêter nette ma conso de cannabis.
\z\z

L'hypothèse de M. Hummel est qu'il s'agit là d'une stratégie destinée à maintenir la cohésion thématique au sein de la prédication avec l'un des arguments, interne ou externe, du verbe. On observe toutefois que cet accord est favorisé par une homophonie de l'adverbe court et de la forme de masculin de l'adjectif. Ainsi, si l'on relève sur la Toile des exemples comme ceux sous (\ref{ex:Dal:13}) :

\ea\label{ex:Dal:13}
    \ea Ce que la nouvelle recherche suggère, c'est que les bénéfices de la course à pieds pourraient s'arrêter nets plus tard dans la vie.

    \ex En juin 2011, un généalogiste amateur originaire de l'Aude et résidant depuis quelques années dans l'Hérault a vu ses recherches piétiner pour s'arrêter nettes.
\z\z

des requêtes telles «~joue(nt) forte(s)~», «~joue(nt) fausse(s)~» ramènent beaucoup moins de résultats utiles\footnote{À titre d'exemple, en juillet 2017, «~jouent fausses~» ramène une trentaine de résultats utiles contre environ 450 pour «~arrêtent nettes~».}.

Quoi qu'il en soit, l'adverbe court ne se distingue en français par aucune marque flexionnelle qui lui serait exclusive~: soit, dans une perspective normée de la langue, il est invariable~; soit, dans une perspective plus en prise avec l'usage, il recourt aux marques flexionnelles de genre et nombre de l'adjectif.

Du point de vue de la syntaxe, lorsque le degré est exprimé syntaxiquement, de nouveau, adjectifs et adverbes partagent les mêmes marqueurs. Ce qui vaut de l'anglais -- les deux peuvent remplacer X dans, par exemple, le comparatif «~more X than~», et admettent les mêmes modifieurs adverbiaux~: par exemple, \emph{very expensive} / \emph{very quickly}~; \emph{too big} / \emph{too slowly}~-- vaut aussi du français. Dans les exemples attestés ci-dessous, les marqueurs \emph{très, plutôt}, \emph{un peu, extrêmement} portent aussi bien sur des adjectifs (\ref{ex:Dal:14}) que sur des adverbes, avec ou sans \emph{--ment} (\ref{ex:Dal:15})~:

\ea\label{ex:Dal:14}
    \ea Il faut généralement agir de façon \textbf{très} stupide pour se retrouver exilé ici.
    \ex Même s'il était \textbf{plutôt} \ul{maigre}, \textbf{plutôt} \ul{petit} et ma foi \textbf{un peu} ridicule, je pouvais imaginer que (\ldots{})
    \ex Pourquoi mes muscles sont \textbf{extrêmement} \ul{douloureux} après l'exercice~?
    \z
\ex\label{ex:Dal:15}
    \ea Nous nous levâmes \textbf{très} \ul{tôt}, nous fûmes \textbf{très} \ul{rapidement} habillées.
    \ex L'ensemble contrastait \textbf{plutôt} \ul{désagréablement} avec le reste de la demeure.
    \ex On s'est engagé \textbf{un peu} \ul{vite}, sans évaluation suffisante des impacts sur la santé.
    \ex J'ai été affecté \textbf{extrêmement} \ul{douloureusement} par tout cela.
\z\z

En conclusion, il apparaît que, pas plus que P\textsubscript{1}, P\textsubscript{2} n'est irréfutablement décisive quant au statut dérivationnel de la règle à laquelle ressortit l'exposant\is{exponent} \emph{-ment~}: certes, les séquences en \emph{--ment} sont des adverbes, mais on vient de voir que la pertinence même de la catégorie de l'adverbe comme catégorie distincte de celle de l'adjectif peut être mise en cause sous de nombreux aspects, et que, si l'on considère qu'en récuser l'existence est excessif, l'hypothèse transpositionnelle, qui pose que la flexion\is{inflection} peut produire des séquences ne relevant pas de la catégorie de ce sur quoi elle s'applique, affaiblit l'hypothèse P\textsubscript{2}.

Examinons dans ce qui suit si les contraintes sémantiques sont davantage décisives.

\subsubsection{Contraintes sémantiques}\label{section:dal:3.3.2}

\subsubsubsection{Contraintes sémantiques d'entrée}

La règle qui forme des adverbes en \emph{--ment} en français peut s'appliquer à des types sémantiques d'adverbes variés~:

\begin{itemize}
\item[---] adjectifs qualificatifs exprimant une propriété~: \emph{étrange / étrangement}~; \emph{sale~/ salement~};

\item[---] adjectifs dits relationnels, mettant en relation le référent du nom sur lequel ils sont construits et le référent de leur nom recteur, comme en témoignent sous (\ref{ex:Dal:16}) les adverbes relevés sur la Toile pouvant être mis en relation avec un adjectif en --\emph{al,} --\emph{aire}, --\emph{el}, --\emph{esque,} --\emph{ien,} --\emph{ique} et --\emph{if}\footnote{Sur la productivité\is{productivity} des adverbes en \emph{--ment}\is{adverb!in --\emph{ment}}, cf. %
%Molinier (1992)
\citet{Molinier92}%
%Molinier
%
.}~:

\end{itemize}

\ea\label{ex:Dal:16}
    \ea\label{ex:Dal:16a} La France n'est-elle pas déjà \textbf{présidentiellement} rayonnante~?
    \ex\label{ex:Dal:16b} Il n'y a pas de frontières, du moins pas de frontières définies \textbf{géographiquement}.
    \ex Si j'avais su que commander à La Redoute impliquait de se faire spammer à ce point, \textbf{électroniquement} et \textbf{postalement}, je dormirais encore sur mon matelas.
    \ex Les 10 Chefs qui ont marqué \textbf{mondialement} l'Année gastronomique 2014.
    \ex En effet, c'est un mandarin qui a vécu \textbf{insulairement} (un peu comme le français de Québec par rapport à la France).
    \ex (\ldots{}) en mettant \textbf{molièresquement} tous les rieurs de son côté.
    \ex (\ldots{}) Ou si, \textbf{rabelaisiennement} nourri d'un savoir immense, (\ldots{})
    \ex Un nouveau fléau guetterait les jeunes~: les maladies transmises \textbf{auditivement}.
\z\z

S'agissant des adjectifs qualificatifs, il a toutefois été souligné, notamment pour l'italien (cf. § \ref{section:dal:2.1.4}) et pour l'anglais (cf. § \ref{section:dal:2.2.2}), que certains types sémantiques d'adjectifs sont rétifs à l'adjonction d'un exposant\is{exponent} adverbialisateur. L'observation a été faite en particulier pour les adjectifs chromatiques et, plus généralement, pour les adjectifs exprimant des propriétés physiques ou sensorielles.

En premier lieu, pour se limiter ici aux seuls chromatiques, on remarquera qu'il ne s'agit pas là d'une impossibilité structurelle, comme le montrent les exemples relevés sur la Toile sous (\ref{ex:Dal:17})\footnote{Sur les adverbes à valeur chromatique, cf. %
%Mora-Millan (2005)
\citet{Mora-Millan05}%
%
.}, dans lesquels, contrairement à des adjectifs lexicalisés comme \emph{vertement}, \emph{blanchement} ou \emph{noirement} qu'atteste le \emph{Trésor de la Langue française}, les séquences en \emph{--ment} présentent bien la valeur chromatique de leur adjectif support :


\ea\label{ex:Dal:17}
    \ea Bientôt la machine {[}la guillotine{]} aura sans doute déclenché son couperet : la vie d'une vieillarde et de deux gamins se répandra \textbf{rougement}.

    \ex Les puces de Cugnat avaient dû aller chercher ailleurs un abri et le charbonnier ne montrerait plus jamais le bout \textbf{violettement} épaté de son nez.

    \ex Tout jeune, il avait trouvé sa voie : vagabonder sur les fortifications dont les talus, \textbf{jaunement} verdis de gazon brûlé par le soleil, viennent mourir près du viaduc.
\z\z

En second lieu, plutôt que de considérer, comme %
%Scalise (1990) 
\citet{Scalise90} %
%Scalise
%
ou %
%Ricca (2015) 
\citet{Ricca15} %
%Ricca
%
pour l'italien, que l'obstacle vient d'une incompatibilité entre le sens de l'adjectif et les con\-train\-tes sémantiques que fait peser sur ses inputs la règle dont \emph{--ment} est l'exposant\is{exponent}, je réitère l'hypothèse faite dans %
%Dal (2007) 
\citet{Dal07} %
%Dal
%
que la rareté d'adverbes en \emph{--ment} à valeur chromatique et, plus généralement, en lien avec un adjectif exprimant une propriété physique ou sensorielle, tient au fait que, si l'on admet que la caractéristique des adverbes en --\emph{ment}\is{adverb!in --\emph{ment}} est d'émerger dans des contextes non nominaux, dans la mesure où ce à quoi renvoient une phrase, un verbe, un adjectif ou un adverbe n'a pas d'extension spatiale, on peut difficilement lui associer des propriétés physiques ou sensorielles. En somme, je rejoins %
%Fábregas (2007)
\citet{Fabregas07}%
%Fábregas
%
, qui considère que, les adjectifs de couleur ou de forme étant fortement associés à des entités physiques %
%(Quine 1960)
\citep{Quine60}%
%Quine
%
, il est attendu que les adverbes en --\emph{ment}\is{adverb!in --\emph{ment}} correspondants, voués de ce fait eux aussi à exprimer des propriétés chromatiques ou physiques, trouvent peu de contextes non nominaux dans lesquels émerger. La contrainte ne tient donc pas à la morphologie en tant que telle, mais est purement sémantique. Elle ne diffère guère de l'impossibilité d'utiliser un adjectif chromatique avec un nom ne référant pas à une entité physique, en préservant la valeur chromatique initiale de l'adjectif~: le fait qu'une délibération puisse difficilement être dite violette ou un exploit marron ne signifie pas pour autant que \emph{violet} ou \emph{marron} ne sont pas des adjectifs.

La règle à laquelle ressortit \emph{--ment} ne semble \rephrase{par conséquent}{donc} pas faire peser de contraintes sémantiques sur les lexèmes qu'elle prend en entrée, les impossibilités, toutes relatives, pointées pour certains types sémantiques d'adjectifs pouvant s'expliquer sans en faire supporter la responsabilité à la morphologie.

\subsubsubsection{Contraintes sémantiques de sortie}

Du point de vue des sorties, il ne semble pas davantage que l'on puisse définir de fonction sémantique qui soit commune à l'ensemble des adverbes en \emph{--ment}\is{adverb!in --\emph{ment}}. En effet, comme le remarquent %
%Plag (2003~: 196) 
\citet[196]{Plag.2003g} %
%Plag
%
pour l'anglais et %
%Fábregas (2007~: 6) 
\citet[6]{Fabregas07} %
%Fábregas
%
pour l'espagnol, la règle à laquelle la séquence \emph{--ment} est associée n'encode pas de signification lexicale particulière, et l'adverbe garde intègre le sens de l'adjectif~auquel il correspond. Plus précisément, aux adjectifs exprimant des qualités correspondent des adverbes classiquement rangés parmi les adverbes de manière (\ref{ex:Dal:18})~; aux adjectifs à sens relationnel correspondent des adverbes de point de vue ou de domaine (cf. (\ref{ex:Dal:19}) qui reprend (\ref{ex:Dal:16b}))~:


\ea\label{ex:Dal:18} Il déploie joyeusement sur la toile ses émotions et ses visions avec une belle énergie.
\ex\label{ex:Dal:19} Il n'y a pas de frontières, du moins pas de frontières définies géographiquement.
\z

Le cas des adverbes dits de phrase peut sembler démentir cette constante.

%
%Molinier (1990) 
\citet{Molinier90} %
%Molinier
%
définit les adverbes de phrase, desquels il propose une typologie\footnote{Il opère une première dichotomie entre adverbes conjonctifs, qui requièrent un contexte gauche (\emph{subséquemment, semblablement}\ldots{}) et adverbes disjonctifs, qui n'imposent pas cette condition. Ces derniers sont à leur tour répartis entre disjonctifs de style (\emph{honnêtement, franchement}), disjonctifs d'attitude -- eux-mêmes classés en disjonctifs d'habitude, évaluatifs et modaux --, disjonctifs d'attitude orientés sujet.}, comme croisant les deux propriétés suivantes~: (i)~pouvoir figurer en tête de phrase négative~; (ii) ne pas pouvoir être extraits dans \emph{c'est \ldots{} que}. Ainsi, dans (\ref{ex:Dal:20}), \emph{sincèrement} et \emph{étrangement} sont des adverbes de phrase~:

\ea\label{ex:Dal:20}
    \ea Sincèrement, je ne pensais pas qu'un groupe pareil s'intéresserait un jour à moi.

    \ex Étrangement, le chasseur ne semblait pas du tout gêné par l'odeur.
\z\z

Certains adverbes de phrase peuvent être homomorphes d'un adverbe de manière. C'est le cas des adverbes de (\ref{ex:Dal:20}), comme le montrent les exemples relevés sur la Toile sous (\ref{ex:Dal:21})~:

\ea\label{ex:Dal:21}
    \ea Si tu t'estimes sincèrement dans ton bon droit, (\ldots{})

    \ex À l'accueil de l'hôtel, la réceptionniste le regarde étrangement.
\z\z

D'autres, tel \emph{certainement,} ne semblent pouvoir être utilisés que comme adverbes de phrase, même si, pour %
%Molinier (1990)
\citet{Molinier90}%
%Molinier
%
, ils ont pu connaitre un emploi comme adverbes de manière jusqu'au XIX\textsuperscript{e} siècle.

La difficulté que posent ces adverbes relativement à l'assertion selon laquelle l'adverbe garde intègre le sens de l'adjectif auquel il correspond et, en particulier, qu'à un adjectif qualificatif fait écho un adverbe de manière est qu'elle ne prédit pas l'existence des adverbes de phrase, ni la possibilité d'adverbes présentant un double emploi comme ceux sous (\ref{ex:Dal:20}) et (\ref{ex:Dal:21}). Une façon de résoudre cette difficulté est de considérer que, de quelque type qu'elle soit, l'opération d'ajout de la séquence \emph{--ment} à un adjectif est transparente sémantiquement, mais qu'une autre opération, indépendante de la première, permet d'employer les adverbes en \emph{--ment}\is{adverb!in --\emph{ment}} comme des adverbes de phrase. Pour %
%Lamiroy \& Charolles (2004)
\citet{Lamiroy04}%
%Lamiroy-Charolles
%
, cette seconde opération relève du phénomène de pragmaticalisation, qu'ils définissent comme le passage de la composante grammaticale à la composante pragmatique ou discursive du langage.

Quoi qu'il en soit, si l'hypothèse flexionnelle n'offre pas de meilleure explication à ce phénomène, l'hypothèse dérivationnelle y achoppe tout autant.

En bref, l'adjonction de la séquence \emph{--ment} à un adjectif ne s'assortit pas d'une fonction sémantique repérable, qui serait dévolue à une RCL\is{lexeme formation rule!LFR}.

\subsection{La formation d'adverbes en \emph{--ment}\is{adverb!in --\emph{ment}} en français : une règle flexionnelle\is{inflection!rule}}\label{section:dal:3.4}

\subsubsection{Les adverbes en \emph{--ment}\is{adverb!in --\emph{ment}} ~: des formes d'adjectifs dans des contextes non nominaux}\label{section:dal:3.4.1}

Au terme de l'examen qui précède, il apparaît que la règle morphologique permettant de former des adverbes en \emph{--ment}\is{adverb!in --\emph{ment}} en français ne possède, de façon irréfutable, aucune des propriétés attendues d'une règle de construction de lexèmes\is{lexeme formation rule}, aussi bien du point de vue syntaxique que du point de vue sémantique~: l'existence même de la catégorie de l'adverbe peut être mise en question, et, sans aller jusqu'à nier la pertinence de cette catégorie, pour le moins, on pourrait être ici face à un cas de transposition flexionnelle~; du point de vue du système, tous les types d'adjectifs semblent pouvoir se voir associer un adverbe en \emph{--ment}\is{adverb!in --\emph{ment}}~; sémantiquement, l'adjonction de \emph{--ment} préserve le sens de l'adjectif, les adverbes de phrase en \emph{--ment} pouvant être considérés comme constituant des emplois spécifiques d'adverbes de manière.

A contrario, une fois levées les objections auxquelles elle semble achopper, la formation d'adverbes en \emph{--ment}\is{adverb!in --\emph{ment}} passe avec succès l'ensemble des critères permettant de distinguer la flexion\is{inflection} de la dérivation\is{derivation} qu'on peut trouver dans, entre autres, %
%Bauer (1997)
\citet{Bauer1997}%
%Bauer
%
, %
%Dressler (2005)
\citet{Dressler05}%
%Dressler
%
, %
%Stump (2005) 
\citet{Stump05} %
%Stump
%
ou %
%Štekauer (2005)
\citet{Stekauer2005a}%
%Štekauer
%
~: parmi ces critères, on retiendra ici le fait qu'à tout adjectif peut correspondre un adverbe en \emph{--ment}\is{adverb!in --\emph{ment}} sans que l'application de cette séquence ne s'assortisse d'une opération sémantique constante repérable.

La conclusion qui s'impose est par conséquent que la formation d'adverbes en \emph{--ment}\is{adverb!in --\emph{ment}} relève de la flexion\is{inflection}, et, partant, que ces adverbes sont la forme que peuvent revêtir les adjectifs dans des contextes non nominaux. Autrement dit, il s'agit là d'un cas d'espèce de flexion contextuelle\is{inflection!contextual inflection}, pour reprendre la terminologie de Booij (cf. entre autres 
%1994
\citeyear{Booij94}%
%
, 
%1996
\citeyear{Booij96} %
%
 et 
% 2000
\citeyear{Booij00}%
). Dans un cadre théorique différent, ce résultat rejoint ceux, anciens, de %
%Kuryłowicz (1936 : 83)
\citet[83]{Kurylowicz36}%
%Kuryłowicz
%
, qui voit en \emph{--ment} un «~morphème syntaxique~», donc une marque flexionnelle, et de %
%Moignet (1963)
\citet{Moignet63}%
%Moignet
%
, dans la perspective de la psychomécanique.

À l'appui de ce résultat, on peut convoquer les exemples sous (\ref{ex:Dal:22}), relevés sur la Toile et/ou partiellement repris de %
%Dal (2007)
\citet{Dal07}%
%Dal
%
, que l'adverbe soit interne au domaine verbal ou qu'il fonctionne comme modifieur d'un adjectif ou d'un adverbe. Ainsi, le choix de \emph{soigneux} vs \emph{soigneusement} en (a/a') est lié à la catégorie du lexème sur lequel portent ces formes, selon qu'il s'agit d'un nom (a) ou d'un verbe (a'). La remarque vaut en (b/b') avec \emph{réponse rapide} vs \emph{répondre rapidement}, en (c/c') avec \emph{applaudissements bruyants} vs \emph{applaudir bruyamment} et en (d/d') avec \emph{marcheur lent} vs \emph{marcher lentement}. En (e/e'), c'est le contexte adjectival qui déclenche l'émergence de l'adverbe \emph{rapidement} en (e'), tandis qu'en (f/f'), le déclencheur est l'adverbe \emph{vite} %
(plus probablement adjectif si on suit %
%Giegerich 2012
\citealt{Giegerich12}%
\footnote{\emph{Vite} a d'ailleurs été longtemps catégorisé comme adjectif en français, cette catégorisation étant confirmée par le nom de propriété \emph{vitesse}, ainsi que la citation suivante de Vialar, que mentionne le \emph{\cite{TLF}}~: «~En tête, c'est Pandore~: un chien vite et solide, et qui prend bien les erres sur la feuille~».}~)
%\citep[plus probablement adjectif si on suit ][\footnote{\emph{Vite} a d'ailleurs été longtemps catégorisé comme adjectif en français, cette catégorisation étant confirmée par le nom de propriété \emph{vitesse}, ainsi que la citation suivante de Vialar, que mentionne le \emph{TLF}~: «~En tête, c'est Pandore~: un chien vite et solide, et qui prend bien les erres sur la feuille~».}~]{Giegerich12}%
%Giegerich
%
. Dans ces divers exemples, les adverbes en \emph{--ment}\is{adverb!in --\emph{ment}} satisfont la définition, communément admise, qu'%
%Anderson (1992~: 83) 
\citet[83]{Anderson92} %
%Anderson
%
donne de la flexion\is{inflection} selon laquelle «~Inflection thus seems to be just the morphology that is accessible to and/or manipulated by rules of syntax~»~:

\ea\label{ex:Dal:22}
  \begin{xlist}
    \exi{a.~} «~Sac à dos~» à roulettes d'une grande capacité est semi-rigide afin de permettre un \textbf{rangement soigneux} et une protection optimum.

    \exi{a'.} L'album photo 26x30 est l'outil parfait pour \textbf{ranger soigneusement} vos précieux clichés.

    \exi{b.~} Vous recevez une \textbf{réponse anonyme} et gratuite à vos questions.

    \exi{b'.} Plus de 16 000 collégiens et lycéens de 12 à 18 ans \textbf{ont répondu anonymement} à un questionnaire détaillé.

    \exi{c.~} Alors tout le bois résonne des \textbf{applaudissements bruyants} des spectateurs et des cris ardents des supporters.

    \exi{c'.} Il savait qu'ils ne pouvaient plus remonter, lui répondit Harry, en criant lui aussi pour couvrir le vacarme, mais sans cesser d'\textbf{applaudir bruyamment}.

    \exi{d.~} Je suis un \textbf{marcheur lent} qui ne cherche pas la performance mais le plaisir de la marche dans un cadre sublime.

    \exi{d'.} Commencez à \textbf{marcher lentement}, puis accélérez le pas et marchez rapidement pour les 5 prochaines minutes.

    \exi{e.~} Il m'avait laissée tomber pour une fille qui se prenait pour un gars et qui était d'une \textbf{laideur abominable}.

    \exi{e'.} Autant le dire tout de suite, c'est \textbf{abominablement laid}.

    \exi{f.~} Ce qui m'ennuie plutôt c'est la \textbf{vitesse atroce} et la stabilité ... emm... très «~délicate~» ... mais je réserve mon jugement pour plus tard ...

    \exi{f'.} Je suis désolée d'avoir mis si longtemps à donner de mes nouvelles mais le temps passe \textbf{atrocement vite} non~??
  \end{xlist}
\z

On relève bien sur la Toile quelques exemples marginaux similaires à ceux dont se servent %
%Payne \& al. (2010) 
\citet{Payne10} %
%Payne-al.
%
pour récuser le fait que les adjectifs et les adverbes apparaissent en distribution complémentaire, donc l'hypothèse flexionnelle en anglais (cf. supra, § \ref{section:dal:2.2.2}). Ainsi en (\ref{ex:Dal:23}), l'adverbe émerge dans un contexte nominal et il semble commutable avec un adjectif~:

\ea \label{ex:Dal:23} Dans une pure tradition franco-britannique et dans la signature de cet hommage résolu à l'absurde du comique anglais, nous nous attaquons sans commune mesure à un pan entier de la culture d'une \emph{île} \textbf{insulairement} sans frontière terrestre ni avec la Hollande\ldots{}
\z

Toutefois, si tant est que, dans (\ref{ex:Dal:23}), \emph{insulairement} fonctionne bien comme modifieur post-nominal du nom \emph{île}\footnote{On peut aussi considérer qu'il fonctionne comme adverbe de point de vue glosable par «~du point de vue insulaire~» et portant sur le syntagme prépositionnel qui suit.}, il n'en demeure pas moins que, dans la grande majorité des cas, adjectifs et ce qu'il est convenu d'appeler adverbes en \emph{--ment}\is{adverb!in --\emph{ment}} figurent en distribution complémentaire, comme le note pareillement %
%Giegerich (2012~: 356)
\citet[356]{Giegerich12}%
%Giegerich
%
~: les quelques exemples de ce type ne suffisent pas à invalider l'hypothèse flexionnelle.

\subsubsection{Quelques autres propriétés}

Au moins trois propriétés remarquables des adverbes en \emph{--ment}\is{adverb!in --\emph{ment}} du français trouvent en outre une explication sous l'éclairage de l'hypothèse flexionnelle~:

\begin{itemize}

\item[---] la position en clôture de mot de la séquence \emph{--ment}. La remarque a été faite pour l'italien par %
%Ricca (1998)
\citet{Ricca98}%
%Ricca
%
, et, pour l'anglais, notamment par %
%Geuder (2000) 
\citet{Geuder00} %
%Geuder
%
ainsi qu'indirectement, par l'ensemble des travaux qui listent --\emph{ly} parmi les affixes de niveau 2, selon la généralisation de %
%Siegel (1979)
\citet{Siegel79}%
%Siegel
%
\footnote{Selon l'\emph{Affix Ordering Generalization} de %
%Siegel (1979)
\citet{Siegel79}%
%Siegel
%
, les affixes se répartissent en affixes de niveau 1 et affixes de niveau 2~: selon ce principe, très discuté %
%(par ex. Fabb 1988)
\citep[par ex. ][]{Fabb88}%
%Fabb
%
, un lexème résultant d'une affixation de niveau 2 ne peut pas servir de base à une affixation de niveau 1.}. Or, les règles de réalisation de lexèmes sont réputées s'appliquer postérieurement aux règles de construction de lexèmes\is{lexeme formation rule} --~cf. de nouveau l'universel 28 de \citeauthor{Greenberg1963} et son incarnation dans l'hypothèse de la morphologie scindée --, du moins quand il s'agit de flexion contextuelle\is{inflection!contextual inflection}.

Si les adverbes en \emph{--ment}\is{adverb!in --\emph{ment}} du français constituent la réalisation d'adjectifs dans un contexte non nominal, on comprend que --\emph{ment}se situe en clôture de mot et, puisqu'il s'agit de flexion contextuelle\is{inflection!contextual inflection}, qu'une forme en \emph{--ment} ne puisse pas servir d'input à une RCL\is{lexeme formation rule!LFR}. Faire de \emph{--ment} l'exposant\is{exponent} d'une RCL\is{lexeme formation rule!LFR} revient en revanche à entériner cette propriété sans l'expliquer~;

\item[---] le fait que, pour ce qui est des adverbes de manière, ils ne diffèrent des adjectifs correspondants ni par leur fonction sémantique --~les uns et les autres expriment des propriétés, d'individus et événements pour l'adjectif \emph{vs} événements seulement pour l'adverbe\footnote{Sans entrer dans le détail, s'agissant des adverbes orientés agents (par ex. \emph{soigneusement}), l'hypothèse a été faite qu'ils possèdent aussi un argument de type individu. La remarque vaut pour les adverbes résultatifs (par ex. \emph{confortablement}), dont l'argument individu serait constitué de l'objet implicite, résultant de l'événement. Pour une argumentation, cf. %
%Geuder (2000) 
\citet{Geuder00} %
%Geuder
%
repris en partie dans %
%Bonami \& al. (2004)
\citet{Bonami04}%
%Bonami-al.
%
.}~--, ni par leur fonction pragmatique, pour reprendre les distinctions opérées par %
%Croft (2003~: 185)
\citet[185]{croft03}%
%Croft
%
. En effet, les adjectifs et les adverbes de manière en \emph{--ment} correspondants assument la même fonction pragmatique de modification, même s'il est probable qu'il faille faire une distinction selon que cette modification s'exerce sur un référent de type objet ou de type événement (Croft, c.p.)~;

\item[---] le fait que la classe des adverbes en \emph{--ment}\is{adverb!in --\emph{ment}} soit une classe ouverte, comme l'est, du reste, celle des «~short adverbes~», au contraire des autres sous-catégories d'adverbes, réputées fermées. À l'échelle des langues du monde, on oppose en effet les catégories des noms, verbes et adjectifs, qui constituent des classes ouvertes, à toutes les autres (adpositions, conjonctions, articles, etc.), qui constituent des classes fermées, cette partition ouvert / fermé allant de pair avec l'opposition lexème / grammème %
(catégorie lexicale majeure / catégorie lexicale mineure~; \emph{content word} / \emph{function word~}; etc. Pour une remise en cause partielle, cf. %
%Croft 2000
\citealt{Croft00}%
%
)
%\citep[catégorie lexicale majeure / catégorie lexicale mineure~;\emph{ content word} / \emph{function word~}; etc. Pour une remise en cause partielle, cf. ][]{Croft00}%
%Croft
%
. Or, dans les langues connaissant la catégorie de l'adverbe, toutes les sous-classes de la catégorie de l'adverbe sont fermées, sauf précisément celle des adverbes de manière %
(cf. pour l'anglais %
%Haspelmath 2001
\citealt{Haspelmath01}%
%
~: 16544~; pour le français, %
%Fradin 2003
\citealt{Fradin03}%
%
~: 18)
%\citep[cf. pour l'anglais ][16544~; pour le français, Fradin 2003~: 18]{Haspelmath01}%
%Haspelmath
%
. L'hypothèse qui consiste à faire des adverbes de manière et de domaine, avec ou sans \emph{--ment}, des formes d'adjectifs a ceci d'intéressant qu'elle vide la catégorie de l'adverbe de sa seule sous-classe présumément ouverte, et que, dès lors, la catégorie de l'adverbe, si on la maintient, s'homogénéise et devient clairement une catégorie lexicale mineure. On tient, en même temps, une explication plausible au fait que le nombre des adverbes de manière puisse s'accroître~: ils tiennent cette possibilité du fait que ces adverbes (avec ou sans marque affixale) instancient une case du paradigme\is{paradigm} des adjectifs, donc du paradigme\is{paradigm} d'une catégorie elle-même ouverte.
\end{itemize}

\subsubsection{Conséquence pour l'organisation de la catégorie de l'adjectif}

On a vu plus haut que la notion de morphome résolvait la question de la forme le plus souvent apparemment féminine du radical\is{stem} sur lequel \emph{--ment} s'applique, à condition d'ajouter un troisième radical\is{stem} à l'espace thématique\is{stem!stem space}  de l'adjectif, le plus souvent homophone du thème\is{stem} 2, auquel s'applique l'exposant\is{exponent} \emph{--ment}.

Dans l'hypothèse flexionnelle défendue ici, la conséquence est que l'adjectif connaît deux modes de variation~: l'un premier en contexte nominal, l'autre second en contexte non nominal, et que le paradigme\is{paradigm} de l'adjectif en français passe de cinq à six cases~:

\begin{itemize}
\item[---] en contexte nominal, l'adjectif varie en français selon les catégories traditionnelles du genre et du nombre, avec, en outre, une forme spécifique dédiée à la forme de liaison au masculin singulier (FLMS) selon l'hypothèse %
%Bonami \& Boyé (2005) 
\citet{Bonami05} %
%Bonami-Boyé
%
rappelée plus haut (§ \ref{section:dal:3.1})~;

\item[---] en contexte non nominal, si l'on intègre au dispositif les hypothèses d'%
%Abeillé \& Godard (2004) 
\citet{Abeille04} %
%Abeillé-Godard
%
et de %
%Hummel (2013 et 2014) 
\citet{Hummel13,Hummel14} %
%Hummel;Hummel
%
qui font des adverbes courts des formes d'adjectifs (cf. supra, § \ref{section:dal:3.3.1}), deux formes seraient en compétition dans une même case du paradigme\is{paradigm}~: une forme longue avec \emph{--ment}, une forme courte, sans \emph{--ment}. Sur ce dernier point, en flexion\is{inflection}, il existe en effet des cas avérés \emph{d'overabondance} %
%(cf. Thornton 2012)
\citep[cf. ][]{Thornton12}%
%Thornton
%
, autrement dit de compétition entre plusieurs formes pour une même case de paradigme\is{paradigm}. Pour le français, c'est par exemple le cas du verbe \emph{asseoir} que citent %
%Apothéloz \& Boyé (2004) 
\citet{Apotheloz04} %
%Apothéloz-Boyé
%
et qui possède les trois formes {[}asɛj{]}, {[}asje{]}, {[}aswa{]} pour la même structure de traits \{ind, prés, 3pl\}. La différence, ici, serait que cette compétition ne serait pas occasionnelle, mais systématique pour la catégorie de l'adjectif dans son ensemble. Il resterait à explorer plus en avant la compétition en contexte non nominal, ce qui déborde le propos du présent article\footnote{Une piste à explorer, que me souffle Dany Amiot, serait une distribution complémentaire tendancielle entre les formes courtes, préférentiellement affectées aux adjectifs exprimant une propriété perceptible par les sens (\emph{parler haut / fort / bas~; jouer gros / petit,} etc.) et les formes longues, qui ont peu d'affinité avec ce type sémantique d'adjectifs.}.
\end{itemize}

Le tableau \ref{tab:Dal:4} propose une représentation du paradigme\is{paradigm} qui intègre la proposition qui précède. Dans la langue standard, l'adverbe court est homomorphe de l'adjectif fléchi au masculin, singulier, hors liaison :


\begin{table}
  \centering
\begin{tabularx}{.8\textwidth}{Xll}
\lsptoprule
\multicolumn{3}{c}{Contexte nominal} \\
 & Singulier & Pluriel \\
\midrule
  & A \{masc., sg, -liaison \}& \multirow{2}{*}{A \{masc., plur.\}} \\
\multirow{-2}{*}{Masculin}& A \{masc., sg, +liaison prénominale \}& \\
\tablevspace
Féminin & A \{fem., sg \}& A \{fem., pl \} \\
\midrule
\end{tabularx}
\begin{tabularx}{.8\textwidth}{C}
\midrule
  Contexte non nominal \\
  \midrule
  Adverbe en \emph{--ment}\is{adverb!in --\emph{ment}} \\
  Adverbe court \\
  \lspbottomrule
\end{tabularx}
\caption{Paradigme\is{paradigm} de l'adjectif en français}
\label{tab:Dal:4}
\end{table}

\section{Conclusion}

En première intention, dans une théorie qui prend le lexème pour unité de base, la réponse à la question de déterminer le statut des séquences adverbiales en \emph{--ment} est a priori aisée à établir~: si ce sont des lexèmes, ce sont des produits d'une règle de construction de lexèmes\is{lexeme formation rule} formant, en tant que telle, des lexèmes différents de ceux qu'elle prend en entrée~; si ce sont des mots-formes\is{wordform}, ils résultent d'une règle flexionnelle\is{inflection!rule}, servant par conséquent à réaliser des mots-formes\is{wordform} des lexèmes sur lesquelles elle opèrent.

S'agissant des adverbes en \emph{--ment}\is{adverb!in --\emph{ment}} du français, il est apparu que ce qui est cité comme le cas de dérivation\is{derivation} par excellence chez de nombreux linguistes et dans de nombreux manuels à vocation pédagogique mérite largement discussion. À la lumière des travaux menés pour d'autres langues, un faisceau d'arguments donne à penser que leurs propriétés sont davantage celles de mots-formes\is{wordform} que de lexèmes, et que «~adverbe en \emph{--ment}\is{adverb!in --\emph{ment}}~» est une étiquette commode pour nommer la forme que peut revêtir un adjectif dans un contexte non nominal~: l'adjonction de \emph{--ment} du français, loin de constituer une zone grise entre flexion\is{inflection} et dérivation\is{derivation}, serait ainsi pleinement une règle flexionnelle\is{inflection!rule}.

Il resterait toutefois quelques points à étayer, énoncés ici sous forme de questions, pour que l'hypothèse flexionnelle emporte définitivement l'adhésion~:
\begin{itemize}
\item[---] selon quel(s) critère(s) le choix entre la forme courte et la forme en \emph{--ment} de l'adverbe s'effectue-t-il~?

\item[---] existe-t-il d'autres cas avérés, en flexion\is{inflection}, de mots-formes\is{wordform} s'émancipant du lexème au paradigme\is{paradigm} duquel ils relèvent~?
\end{itemize}
\section*{Remerciements}

J'adresse tous mes remerciements à Dany Amiot, Olivier Bonami, Stéphanie Lignon et Fiammetta Namer, pour leur relecture attentive d'une version précédente du présent chapitre et pour leurs diverses suggestions d'amélioration, dont j'ai tenté de tirer profit au mieux.

%\nocite{Abeille04}
%\nocite{Alarcos51}
%\nocite{Amiot05}
%\nocite{Anderson77}
%\nocite{Anderson82}
%\nocite{Anderson92}
%\nocite{audringbooij16}
%\nocite{Apotheloz04}
%\nocite{Aronoff94}
%\nocite{Baker03}
%\nocite{Bassac04}
%\nocite{Bauer83}
%\nocite{Bauer92}
%\nocite{Bauer01}
%\nocite{Bauer.2013}
%\nocite{Bello1847}
%\nocite{Blevins2001}
%\nocite{Bonami05}
%\nocite{Bonami2009a}
%\nocite{Bonami04}
%\nocite{Booij94}
%\nocite{Booij1996}
%\nocite{Booij00}
%\nocite{Bosque89}
%\nocite{Boye15}
%\nocite{Bybee85}
%\nocite{Chircu07}
%\nocite{Cifuentes02}
%\nocite{Corbin82}
%\nocite{Corbin87}
%\nocite{Croft00}
%\nocite{croft03}
%\nocite{Dal2003a}
%\nocite{Dal04}
%\nocite{Dal07}
%\nocite{Dal16}
%\nocite{Detges15}
%\nocite{Dressler89}
%\nocite{Dressler05}
%\nocite{Egea93}
%\nocite{Emonds76}
%\nocite{Fabb88}
%\nocite{Fabregas07}
%\nocite{Fabregas14}
%\nocite{Fradin03}
%\nocite{Gaeta08}
%\nocite{Gaeta15}
%\nocite{Garcia91}
%\nocite{Gardes15}
%\nocite{Geuder00}
%\nocite{Giegerich12}
%\nocite{Greenberg1963}
%\nocite{Guimier96a}
%\nocite{Hacken14}
%\nocite{Haspelmath1996}
%\nocite{Haspelmath01}
%\nocite{Hay02}
%\nocite{Hjelmslev28}
%\nocite{Hockett58}
%\nocite{Hummel13}
%\nocite{Hummel14}
%\nocite{Hummel14b}
%\nocite{Huot06}
%\nocite{Jespersen54}
%\nocite{Karlsson81}
%\nocite{Kilani-Schoch05}
%\nocite{Kovacci99}
%\nocite{Kurylowicz36}
%\nocite{Lamiroy04}
%\nocite{Luschutzky15}
%\nocite{Lyons68}
%\nocite{Meyer-Lubke}
%\nocite{Miller91}
%\nocite{Miller92}
%\nocite{Moignet63}
%\nocite{Molinier90}
%\nocite{Molinier92}
%\nocite{Molinier00}
%\nocite{Mora-Millan05}
%\nocite{Mora07}
%\nocite{Nevis85}
%\nocite{Niklas15}
%\nocite{Pagliano03}
%\nocite{Payne10}
%\nocite{Perlmutter88}
%\nocite{Pittner15}
%\nocite{Plag.2003g}
%\nocite{Pottier66}
%\nocite{Quine60}
%\nocite{Radford88}
%\nocite{Rainer96}
%\nocite{Rainer16b}
%\nocite{Ricca98}
%\nocite{Ricca04}
%\nocite{Ricca15}
%\nocite{Roche10}
%\nocite{Saporta90}
%\nocite{Scalise1988}
%\nocite{Scalise90}
%\nocite{Scalise05}
%\nocite{Schultink1961}
%\nocite{Seco72}
%\nocite{Siegel79}
%\nocite{Stekauer2005a}
%\nocite{Stekauer15}
%\nocite{Stump05}
%\nocite{Thornton12}
%\nocite{Varela90}
%\nocite{Willigen83}
%\nocite{Zagona90}
%\nocite{Zwicky1987c}
%\nocite{Zwicky1995}

\il{French|)}

{\sloppy \printbibliography[heading=subbibliography,notkeyword=this] }

\end{document}
