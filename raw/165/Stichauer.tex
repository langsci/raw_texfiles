\documentclass[output=paper]{langsci/langscibook}
\ChapterDOI{10.5281/zenodo.1406999}
\title{Some remarks on clipping of deverbal nouns in French and Italian}
\author{Pavel Štichauer\affiliation{Charles University, Prague}}
 \abstract{This chapter deals with the restricted class of clipped deverbal nominals in French (e.g. \emph{introduction} → \emph{intro}) and especially in Italian (e.g. \emph{giustificazione} → \emph{giustifica}) and aims to show that subtle semantic restrictions seem to constrain such clipping, although there are some differences between the two languages. First, I introduce the well-known distinction between \isi{event} (E) and \isi{result}/referential\is{reference} (R) nouns\is{noun!event}\is{noun!referential} that has been further elaborated by %
%Melloni (2006; 2007; 2011)
\citet{Melloni2006,Melloni2007,Melloni2011}%
%Melloni
%
. I then proceed to discuss a class of formations where clipping seems to be sensitive to a special \isi{result}/object meaning which is very close to what %
%Pustejovsky (1995: 174; cf. Melloni 2011: 109, 111, 142) 
\citeauthor{pustejovsky91} (\citeyear{pustejovsky91}: 174; see \citealt{Melloni2011}: 109, 111, 142) %
%\citet[174; cf. Melloni 2011: 109, 111, 142]{Pustejovsky.1991} % The Generative Lexicon ???
%?Pustejovsky
%
calls \emph{information object}\is{information-object}. On the basis of a limited class of examples (both attested and hypothetical, e.g. \emph{quantificazione} → \emph{quantifica}), I argue that where there is such an information object reading available to the relevant nominal, the clipping rule may apply. I take these phenomena to be relevant for %
%Fradin \& Kerleroux's (2009: 84-86) 
\citeposp{Fradin09}{84--86} %
%Fradin-Kerleroux
%
\emph{Maximal Specification Hypothesis}, according to which word-formation rules\is{word-formation rule} can apply, especially in the case of polysemous lexemes, to specific semantic features inherent in the overall meaning of the base. I demonstrate that clipping can have access to precisely these semantic features.
 }

\maketitle

\begin{document}
\selectlanguage{english}
\il{French|(}
\il{Italian|(}
\is{clipping|(}

\section{Introduction}\label{Stekauer-introduction}

It is widely held that morphological phenomena such as clipping (or
\isi{truncation} and \isi{blending}) can be well explained within a sociolinguistic
or pragmatic framework where specific stylistic, diaphasic and/or
diastratic factors are at work. Under this view, the only
morphologically relevant issue would be that of phonological conditions
and constraints on the bases. Nevertheless, there have recently been 
some attempts to show that there might also be specific semantic
constraints that, in some cases, rule out the possibility of such
morphological reduction, regardless of any pragmatically constrained
context. Such studies demonstrate that \isi{truncation} may operate in a
highly systematic way that involves access to specific semantic
information of a given base.

In this chapter, I intend to show that, within the
restricted class of clipped deverbal nominals\is{noun!deverbal} in French (e.g.
\emph{introduction → intro}) and especially in Italian (e.g.
\emph{giustificazione} \emph{→ giustifica}), which will be the focus of
the present text, special and subtle semantic restrictions seem to
constrain the availability of these formations, though the two languages do not cover
exactly the same group of formations.

In what follows, I will assume the traditional, though much debated,
distinction between inflection and derivation %
%(see, e.g., Spencer 2013:
%38-43)
\citep[see, e.g.,][38-43]{Spencer13}%
%Spencer
%
. Such a distinction is fundamental in that it posits two
different roles of morphology: inflectional morphology is supposed to
realize the inflected forms of a given lexeme, while derivational
morphology serves to create new lexemes.\footnote{Inflectional
  morphology provides the word forms inhabiting the cells in the
  lexeme's paradigm. {[}...{]}, a derivational process defines a new
  lexeme, which may well have a completely new set of inflectional
  properties. Therefore, derivational morphology cannot be defined using
  the same machinery as inflectional morphology, because a derived
  lexeme is not paradigmatically related to its base and cannot be
  considered a word form of anything. Rather, it defines an entirely new
  set of (possibly inflected) word forms. %
%(Spencer 2013: 2)
\citep[2]{Spencer13}%
%Spencer
%
.} However,
the difficulty of the topic to be tackled in the following pages lies
precisely in the fact that \emph{clipping} (or \emph{truncation})\is{truncation} does
not always seem to deliver an entirely new lexeme.

I shall argue, following %
%Fradin \& Kerleroux's (2009: 84-86)
\citeposp{Fradin09}{84--86} %
%Fradin-Kerleroux
%
\emph{Maximal Specification Hypothesis}, that word-formation rules\is{word-formation rule}
apply, especially in the case of polysemous lexemes, to specific
semantic features inherent in the overall meaning of the base, and that
clipping can have access to precisely these semantic features.

The text is organized as follows. In Section 2, I first lay out the
well-known distinction between \isi{event} (E) and \isi{result}/referential\is{reference} (R)
nouns\is{noun!event}\is{noun!referential}  that has been further elaborated by %
%Melloni (2006; 2007; 2011) 
\citet{Melloni2006,Melloni2007,Melloni2011} %
%Melloni
%
and
that, at first sight, seems to capture some of the known cases. In
Section 3, I briefly comment on the French data taken from %
%Kerleroux (1997)
\citet{Kerleroux1997}%
, %
%Fradin \& Kerleroux (2003)
\citet{Fradin03b}%
%Fradin-Kerleroux
%
, and %
%Fradin (2003)
\citet{Fradin2003}%
%Fradin
%
. In Section 4, I
take up the Italian data, based on %
%Thornton (1990; 2004)
\citet{Thornton1990a,Thornton2004}%
%Thornton
%
, %
%Štichauer
%(2006)
\citet{Stichauer2006}%
, and %
%Montermini \& Thornton (2014) 
\citet{MonterminiThornton2014} %
%Montermini-Thornton
%
which are, in some fundamental
aspects, different with respect to French. In Section 5, I conclude
by putting forward a (falsifiable) hypothesis according to which such
deverbal nouns\is{noun!deverbal} are liable to undergo clipping only when special semantic
and pragmatic conditions are met. I point out that, contrary to what is
usually assumed (especially for Italian), the shortened forms may not
always be completely synonymous with their ``full'' parental nominals.

\section{Event/Referential nouns\is{noun!event}\is{noun!referential}  and
clipping}\label{Stekauer-eventreferential-nouns-and-clipping}

Since %
%Grimshaw (1990)
\citet{Grimshaw1990}%
%
, the distinction between \emph{complex event
nouns}\is{noun!complex event}, \emph{simple event nouns}\is{noun!simple event}  and \emph{result nouns}\is{noun!result}  has become
widely accepted, though there has been much critical discussion about
the various criteria that Grimshaw herself proposed to individuate the
three groups %
%(see Melloni 2011: 21-34)
\citep[see][21--34]{Melloni2011}%
%Melloni
%
.

It has also been thought that only \emph{complex event nouns}\is{noun!complex event} can give
rise to various \isi{result} interpretations where the \isi{result} reading is
normally associated with the outcome of the corresponding complex event
noun\is{noun!complex event}. Traditional examples of such \isi{event}/\isi{result} (E/R) ambiguity are
given in~(\ref{ex:Stekauer:1}), where the English examples (\ref{ex:Stekauer:1a}., \ref{ex:Stekauer:1b}.) are given an
equivalent version in Italian (\ref{ex:Stekauer:1c}., \ref{ex:Stekauer:1d}.) and French (\ref{ex:Stekauer:1e}., \ref{ex:Stekauer:1f}.).

\begin{exe}
\ex\label{ex:Stekauer:1}
\begin{xlist}
\ex\label{ex:Stekauer:1a} {The construction of that house (by the company) took place
forty years ago} → E

\ex\label{ex:Stekauer:1b} {The construction is breathtaking} → R

\ex\label{ex:Stekauer:1c} {La costruzione di quella casa (da parte dell'impresa) ebbe
luogo quarant'anni fa} → E

\ex\label{ex:Stekauer:1d} {La costruzione è molto bella} → R

\ex\label{ex:Stekauer:1e} {La construction de la maison (de la part de la compagnie) a eu
lieu il y a quarante ans} → E

\ex\label{ex:Stekauer:1f} {La construction est très belle} → R
\end{xlist}
\end{exe}

\emph{Simple event nouns}\is{noun!simple event} (e.g. \emph{party}), instead, do not have
an associated \isi{event} structure, so that the \isi{event}/\isi{result} polysemy is not
available. Moreover,  simple event nouns\is{noun!simple event} are said to pattern with
result nouns\is{noun!result} in that they share the same set of properties  %
%(cf. Melloni 2011: 24-25)
\citep[see][24--25]{Melloni2011}%
%
. In what follows, I will assume the general divide between
an event-based reading\is{event!event-based reading} and result-based reading\is{result!result-based reading} of the derived nominals,
discussing various problems in due course.

When it comes to clipping, the general divide between E/R nominals turns
out to be relevant as there are specific constraints on the semantic
status of the deverbal noun\is{noun!deverbal}. In fact, as 
%Kerleroux 
\citeauthor{Kerleroux1997} %
%
claims 
(%
%1997
\citeyear{Kerleroux1997}%
%
: 155),
``nouns denoting complex events\is{noun!complex event} may not be apocopated''.


However, as we shall see, the situation is more complicated since there
are more subtle semantic conditions that allow for clipping. More
precisely, the clipping rule seems to eliminate the possibility of event
noun\is{noun!event} interpretation (E) regardless of the fact whether the affected noun
is a complex event\is{noun!complex event} or simple event nominal\is{noun!simple event}. Rather, what is required is
a specific \isi{result}/object -- or \emph{referential}\is{reference} (R) denotation of the
corresponding deverbal noun\is{noun!deverbal}, as illustrated in~(\ref{ex:Stekauer:2}).

\begin{exe}

\ex\label{ex:Stekauer:2} French
\begin{xlist}
\ex[]{{La récupération des naufragés fut longue} → E}
\ex[*]{{La récupe des naufragés fut longue} → *E\footnote{Georgette
  Dal (p.c.) observes that, on the Internet, we can easily find some
  examples of the eventive reading\is{event!eventive reading} as well, such as \emph{``La recup(e)
  a été longue car j'avais une centaine de courriers à récupérer}.''}
\trans`The rescue operation of the shipwrecked was long'}
\ex[]{{J'ai des récupérations à prendre avant Noël} → R}
\ex[]{{J'ai des récupes à prendre avant Noël} → R
\trans `I have some extra days of holiday to take before Christmas'}
\ex[]{{Il s'oppose à l'introduction du loup à Paris} → E}
\ex[*]{{Il s'oppose à l'intro du loup à Paris} → *E
\trans`He is against the introduction of the wolf into Paris'}
\ex[]{{Il a apprécié l'introduction de ton livre} → R}
\ex[]{{Il a apprécié l'intro de ton livre} → R
\trans`He enjoyed the introduction of your book'}
\end{xlist}
 \end{exe}

As far as Italian is concerned, the situation is more intricate.
Following %
%Thornton (2004) 
\citet{Thornton2004} %
%Thornton
%
and %
%Montermini \& Thornton (2014)
\citet{MonterminiThornton2014}%
%Montermini-Thornton
%
, a
distinction must be made between those deverbal nouns\is{noun!deverbal} in \emph{-a} which
are the result of the unproductive process of \isi{conversion} (e.g., \emph{la
conquista, la sosta, la firma} etc.), and the apparently identical
deverbal nouns\is{noun!deverbal} in -\emph{a} such as \emph{bonifica, condanna, conferma}
whose (diachronic) origin is to be sought in the \isi{truncation} of the
actional suffix \emph{-zione}\is{suffixation!in -\emph{zione}}\is{suffixation!actional}   %
%(see Montermini \& Thornton 2014: 187
%ff.)
\citep[see][187 ff.]{MonterminiThornton2014}%
%Montermini-Thornton
%
.

Although the diachronic account is surely on the right track,
synchronically the behaviour of pairs of full vs. clipped formations is
far from being identical. As I will argue below, it is worth drawing a
distinction between three groups.

The first group comprises the pairs of formations which seem to be
totally interchangeable displaying (purportedly) absolute \isi{synonymy}, such
as \emph{modificazione / modifica} (\ref{ex:Stekauer:3}), where both forms display
regular E/R ambiguity:

\begin{exe}
\ex\label{ex:Stekauer:3} Italian
\begin{xlist}
\ex {La modificazione del testo (da parte dell'autore) è stata molto
lunga} → E

\ex {La modifica del testo (da parte dell'autore) è stata molto
lunga} → E\footnote{In French, the clipped form \emph{la modif} would
  also seem to be possible as some examples from the Internet show, such
  as ``\emph{ceux qui sont grisés apparaissent comme dégrisés après la
  modif du texte}.'' (Georgette Dal, p.c.).}

\trans `The modification of the text (by the author) took a long time'

\ex {La modificazione del testo sarebbe subito saltata fuori} → R

\ex {La modifica del testo sarebbe subito saltata fuori} → R

\trans `The modification of the text would have surfaced immediately'

\ex {La modificazione (del testo) è sul tavolo} → R

\ex {La modifica (del testo) è sul tavolo} → R

\trans `The modification (of the text) is on the table'
\end{xlist}
\end{exe}


The second group involves partly synonymous formations in which the
difference is claimed to lie exclusively at the stylistic level, such as
\emph{giustificazione / giustifica} (\ref{ex:Stekauer:4}), but which may display deeper semantic differences, as I will show, especially when it comes to the
difference between an event\is{event!eventive reading} vs. referential reading\is{reference!referential reading}. In fact, as the
examples in (\ref{ex:Stekauer:4}) show, the event reading\is{event!eventive reading} of the clipped form tends to be
rather unacceptable.

\begin{exe}
\ex\label{ex:Stekauer:4} Italian
\begin{xlist}
\ex[]{{Le ripetute giustificazioni dell'assenza (da parte degli
studenti) sono intollerabili} → E}
\ex[*]{{Le ripetute giustifiche dell'assenza (da parte degli studenti)
sono intollerabili} → *E
\trans `The frequent justifications for absence (on the part of the students) are
intolerable'}
\ex[]{{La giustificazione dell'assenza è falsa} → R}
\ex[]{{La giustifica dell'assenza è falsa} → R
\trans `The  justification for absence is false'}
\ex[]{{La giustificazione è sul tavolo} → R}
\ex[]{{La giustifica è sul tavolo} → R
\trans `The justification is on the table'}
\end{xlist}
\end{exe}

Finally, a third group, explicitly not addressed in the literature,
would involve impossible, unacceptable formations where the
clipping of the suffix is disallowed even when the full noun in
\emph{-zione} displays some referential reading\is{reference!referential reading}. The examples in (\ref{ex:Stekauer:5})
illustrate.


\begin{exe}
\ex\label{ex:Stekauer:5}
\begin{xlist}
\ex[]{{La riunificazione delle due Germanie è stata un processo
complesso} → E}
\ex[*]{{La riunifica delle due Germanie è stata un processo complesso}
→ *E
\trans `The reunification of the two Germanies was a complex process'}
\ex[]{{Questo sedimento è la stratificazione di rocce diverse} → R}
\ex[*]{{Questo sedimento è la stratifica di rocce diverse} →
*R\footnote{I owe this example to Fabio Montermini.}
\trans `This sediment is a (result of the) stratification of various rocks'}
\end{xlist}
\end{exe}

In what follows, I shall concentrate precisely on these two groups where
we find, on the one hand, some attested pairs of full vs. clipped
formations with presumably slightly different semantics, and, on the
other hand, unattested, yet possible or impossible clipped forms. To
begin with, I posit that what the two clipping rules, in French and in
Italian, respectively, seem to have in common is a sort of (partial)
elimination of event reading\is{event!eventive reading} of the deverbal noun\is{noun!deverbal} in favour of a salient
referential interpretation\is{reference!referential reading}. At the same time, a specific semantic
condition on the kind of object (i.e. the type of referential reading\is{reference!referential reading})
is required for the rule in question. In the next sections, after first
considering some French and -- in more detail -- Italian examples, I
will argue that a special typology of result nominals\is{noun!result} %
%(elaborated by
%Melloni 2011) 
\citep[elaborated by][]{Melloni2011} %
%Melloni
%
is needed in order to account for the phenomena in
question. I intend to show that a lexical semantic typology of the base
verbs will be able to predict, to a large extent, the possibility of
clipping.

\section{Clipped deverbal nominals\is{noun!deverbal} in
French}\label{Stekauer-clipped-deverbal-nominals-in-french}

In this section, I  briefly review the French data, taken from the
literature, focusing on the general condition for the clipping rule,
which will turn out to be useful  in the discussion of the Italian
examples as well.

In French, the clipping rule, as far as deverbal nouns\is{noun!deverbal} with the suffix
-\emph{tion}\is{suffixation!in -\emph{tion}} are concerned, may apply to a number of formations.\footnote{I deliberately leave aside the general context for \isi{truncation}
  which, in French, is not limited to complex words (having as its
  target only the suffix) but may be applied to a wide range of bases,
  such as \emph{documentation -- doc, information -- info, actualité --
  actu,} etc. As %
%Montermini \& Thornton (2014: 183) 
\citet[183]{MonterminiThornton2014} %
%Montermini-Thornton
%
point out, in cases
  where the truncated material coincides with the suffix (e.g.,
  \emph{invitation -- invite}), the coincidence is to be taken as purely
  fortuitous.} When clipped, the noun receives a special \isi{result}/object
reading although some aspects of \isi{event} interpretation are maintained.
The clipped nouns thus become similar to \emph{simple event nouns}\is{noun!simple event}. The
internal arguments of the base verb are, in such a formation, excluded
%
%(cf. Kerleroux 1997: 171)
\citep[see][171]{Kerleroux1997}%
%Kerleroux
%
:

\begin{exe}
\ex\label{ex:Stekauer:6}
\begin{xlist}
\ex[]{{La manifestation de la vérité aura pris cinquante ans} → E}
\ex[*]{{La manif de la vérité aura pris cinquante ans} → E
\trans `The demonstration of the truth will have taken fifty years'}
\ex[]{{La manifestation (des étudiants) a duré cinq heures} → E}
\ex[]{{La manif (des étudiants) a duré cinq heures} → E
\trans `The demonstration (of the students) took five hours'}
\end{xlist}
\end{exe}

According to %
%Kerleroux (1997: 155)
\citet[155]{Kerleroux1997}%
%Kerleroux
%
, already cited  above, the difference
lies precisely in the complex / simple \isi{event} dichotomy. Complex
event nominals\is{noun!complex event}, which maintain their internal argument structure, cannot
undergo clipping, whilst in the case of simple event nouns\is{noun!simple event}, such as
\emph{manifestation} in the sense of `demonstration', clipping is
allowed.

In the following example, the possibility of clipping is limited to a
more concrete (and not \emph{eventive})\is{event} interpretation of
`introduction', that of \emph{information-object}\is{information-object}.\footnote{As Fabio
  Montermini notes (p.c.), such an information-object\is{information-object} feature does not
  prevent, in principle, an event-based reading\is{event!event-based reading}, as witnessed by the
  acceptability of \emph{l'intro de son discours a duré une heure},
  where \emph{discours} `speech', being a simple event noun\is{noun!simple event}, enables
  clipping.} This notion will be of great importance in the discussion of
the Italian data.

\begin{exe}
\ex\label{ex:Stekauer:7}
\begin{xlist}
\ex[]{{L'introduction du lynx dans le massif du Vercors par les responsables de l'ONF} → E}
\ex[*]{{L'intro du lynx dans le massif du Vercors par les responsables
de l'ONF} → *E
\trans `The introduction of the lynx into Vercors Massif by the authorities of
the ONF (National Forest Office)'}
\ex[]{{L'introduction (de ton livre) compte quatre pages} → R}
\ex[]{{L'intro (de ton livre) compte quatre pages} → R
\trans `The introduction (of your book) has four pages'}
\end{xlist}
\end{exe}

The important point is that clipping in French does not seem to
eliminate eventive readings\is{event!eventive reading} altogether. In the case of event nouns\is{noun!event}, the
difference between pure transpositions\is{transposition} (complex event nominals\is{noun!complex event}) and what
we might call ``names of specific events''\is{event} is relevant. Indeed, as
Fradin states, the  condition on clipping seems  to be that

\begin{quote}

(...) d'une manière générale, ne peuvent être accourcies que des
expressions nominales fonctionnant comme des dénominations
(\emph{names}) d'entités diverses (individu, objet, comportement...).
{[}Generally, what can be shortened are the expressions functioning as
denominations, names of various entities such as individuals, objects,
behaviour{]}. \citep[250]{Fradin2003}%
\end{quote}

I now turn to the Italian data in order to see further semantic
constraints on what kind of entities these generally need to be for
clipping to take
place.

\section{Clipped deverbal nominals\is{noun!deverbal} in
Italian}\label{Stekauer-clipped-deverbal-nominals-in-italian}

According to %
%Thornton (1990; 2004: 519)
\citet[519]{Thornton1990a,Thornton2004}%
%Thornton
%
, the Italian shortened forms are
to be taken simply as stylistic variants of their corresponding full
nominals. Furthermore, as %
%Montermini \& Thornton (2014: 193-194) 
\citet[193--194]{MonterminiThornton2014} %
%Montermini-Thornton
%
show on
the basis of  corpus frequency, many shortened forms (especially
those in \emph{-ifica}) have by now become far more frequent than their
full counterparts.

%
%Štichauer (2006) 
\citet{Stichauer2006} %
%?Štichauer
%
proposes, as already mentioned above, to distinguish
three groups of such clipped nominals that behave differently with
respect to the original deverbal nouns\is{noun!deverbal} with the suffix -\emph{zione}\is{suffixation!in -\emph{zione}}.

The first group comprises the pairs such as
\emph{modificazione-modifica} (\ref{ex:Stekauer:3}) or
\emph{verificazione-verifica} (\ref{ex:Stekauer:8}) in which the clipped form has
already assumed the same syntactic distribution; moreover, in this case
of \emph{verificazione/verifica}, the clipped form is far more
acceptable because of its increasing frequency of use.

\begin{exe}
\ex\label{ex:Stekauer:8}
\begin{xlist}
\ex {La verificazione della
teoria (da parte degli scienziati) è stata affrettata} → E

\ex {La verifica della teoria (da parte degli scienziati) è stata
affrettata} → E

\trans `The verification of the theory (by the scholars) was hasty'

\ex {La verificazione (della teoria) va pubblicata su una rivista
importante} → R

\ex {La verifica (della teoria) va pubblicata su una rivista
importante} → R

`The verification (of the theory) is to be published in an important
journal'

\ex {La verificazione (della teoria) è sul tavolo} → R

\ex {La verifica (della teoria) è sul tavolo} → R

\trans `The verification is on the table'
\end{xlist}
\end{exe}

In the second group of formations we should take into consideration
cases in which, on the contrary, we find a shortened form that has a
specialized meaning with respect to the noun in -\emph{zione}\is{suffixation!in -\emph{zione}}, e.g.
\emph{permutazione - permuta}. While the former noun is a normal event
nominal\is{noun!event}, the latter refers to a specialized type of property
exchange.\footnote{%
%Montermini \& Thornton (2014: 196-198) 
\citet[196-198]{MonterminiThornton2014} %
%Montermini-Thornton
%
rectify
  %
%Štichauer's (2006: 33) 
\citeposp{Stichauer2006}{33} %
%?Štichauer
%
incorrect claim about the loss of a
  transpositional\is{transposition} relation between the verb \emph{permutare} and
  \emph{permuta}. In fact, \emph{permuta} clearly functions as an event
  noun\is{noun!event} being thus similar to the relation between, say, the French verb
  \emph{manifester} with respect to \emph{manifestation} and
  \emph{manif}. Moreover, %
%Montermini \& Thornton (2014: 198) 
\citet[198]{MonterminiThornton2014} %
%Montermini-Thornton
%
suggest
  that \emph{permuta} is to be taken as a converted form rather than a
  clipped formation.} (\ref{ex:Stekauer:9}):\footnote{The examples are taken from the corpus
\emph{La Repubblica} and slightly modified.}


\begin{exe}
\ex\label{ex:Stekauer:9}
\begin{xlist}
\ex {Questo poemetto (...) si fonda sulla permutazione dei ruoli tra
l'uomo e l'ani\-ma\-le}

\ex {Questo poemetto (...) si fonda sulla *permuta}\footnote{In fact,
  web search on google.it
  (http://www.ilcovile.it/news/archivio/00000420.html) provides one
  example of the shortened form \emph{permuta} in precisely this
  context. The sequence \emph{permuta dei ruoli} can be found in the
  Italian translation of Jankélévitch's book \emph{Le Paradoxe de la morale}.} \emph{dei ruoli tra l'uomo e l'animale}

\trans `This short poem is based on the permutation of  roles between 
man and  animal'

\ex {Che dire poi di coloro che cedono la propria auto in permuta?}

\ex {Che dire poi di coloro che cedono la propria auto in
*permutazione?}

\trans `What can we say then about those who trade in their cars?'
\end{xlist}
\end{exe}
Finally, the third group of nouns would be the one in which clipping is impossible. Although this question is not directly
addressed in the literature, I maintain that it is interesting to uncover the constraints
that seem to regulate the possibility or impossibility of a hypothetical
\isi{nonce-formation}. In fact, if only stylistic constraints
were at work, we should find many more examples in various
administrative texts than we actually encounter.\footnote{For instance,
  in the corpus of \emph{La Repubblica} (330 million tokens), we find about
  150 different types in -\emph{ificazione}\is{suffixation!in -\emph{ificazione}}, and about 90 forms ending
  in \emph{-ifica}, where after  careful post-processing,
  about a dozen  formations remain and virtually no \emph{hapax}
  qualifying as a real neologism can be found (\emph{la chiarifica}
  being probably the only exception).} Moreover, if only such diaphasic
differences were responsible for the clipping rule, many a
\isi{nonce-formation}, e.g. \emph{la continua desertificazione del pianeta --
la continua *desertifica del pianeta} (`the continuous desertification
of the planet'), might become acceptable under specific stylistic
circumstances. However, this does not seem to be the case.

I will limit my analysis to a narrow sample of nouns in
\emph{-ificazione} that seem to be the most frequent deverbal nominals\is{noun!deverbal}
that might, under specific conditions to be stated below, undergo 
clipping of the suffix \emph{-zione}. For the present, I will assume
that where clipping is allowed, a special \isi{result}/object denotation is
required or imposed by the mechanism in question; at the same time, the
complex or simple event reading\is{event!eventive reading} is, in some cases, partially eliminated.

I shall consider the following six examples: \emph{riunificazione,
mercificazione, reificazione, quantificazione, giustificazione} and
\emph{falsificazione}. I will employ roughly the same ``diagnostic''
contexts also used by %
%Melloni 2011
\citealt{Melloni2011}%
%
. This step is obviously problematic
for the simple reason that the diagnostic contexts do not always yield
an entirely natural example, attested or ``attestable'' in the corpora.
I attempt to remedy this shortcoming by modifying or integrating the
examples according to real data present in the corpus
CORIS/CODIS\footnote{Accessible at:
  http://corpora.dslo.unibo.it/TCORIS/. Accessed  September-October,
  2016.}, \emph{La Repubblica},\footnote{Accessible at:
  http://dev.sslmit.unibo.it/corpora/corpus.php?path=\&name=Repubblica.
  Accessed  Sep\-tember-October, 2016.} or on the Internet (by a general
search on google.it). When necessary, I also add a clarifying footnote
(especially when native speakers' judgements tend to give variable
results).

\newpage 
I begin with \emph{riunificazione}. In (\ref{ex:Stekauer:10}), we see that the only
available reading is that of an \isi{event}, all possible \isi{result}/referential\is{reference}
readings\is{reference!referential reading} are excluded simply because \emph{riunificare} does not belong
to any product-oriented verbs\is{verb!product-oriented} %
%(in the sense of Melloni 2011: 184 ff.)
\citep[in the sense of][184 ff.]{Melloni2011}%
%Melloni
%
:

\begin{exe}
\ex\label{ex:Stekauer:10}
\begin{xlist}
\ex[]{{La riunificazione delle due Germanie ha richiesto molto tempo}
→ E}
\ex[*]{{La riunifica delle due Germanie ha richiesto molto tempo} → *E
\trans `The  reunification of the two Germanies took a long time'}
\ex[*]{{La riunificazione è falsa} → *R}
\ex[*]{{La riunifica è falsa} → *R
\trans (intended) `The reunification is false'}
\ex[*]{{La riunificazione è sul tavolo} → *R}
\ex[*]{{La riunifica è sul tavolo} → *R
\trans (intended) `The reunification is on the table'}
\end{xlist}
\end{exe}

In the case of \emph{mercificazione} (`commodification') we find
essentially the same situation.

\begin{exe}
\ex\label{ex:Stekauer:11}
\begin{xlist}

\ex[]{{Questo processo di (continua) mercificazione del corpo femminile} → E}
\ex[*]{{Questo processo di (continua) mercifica del corpo femminile} →
E
\trans `This process of (continuous) commodification of the female body'}
\ex[*]{{Le presenti mercificazioni del corpo femminile non sono
affatto belle} → *R}
\ex[*]{{Le presenti mercifiche del corpo femminile non sono affatto
belle} → *R
\trans (intended) `The present commodifications of the female body are not nice at all'}
\ex[*]{{La mercificazione è sul tavolo} → *R}
\ex[*]{{La mercifica è sul tavolo} → *R
\trans `The commodification is on the table'}
\end{xlist}
\end{exe}

It could be argued, however, that the verb \emph{mercificare} is
semantically close to verbs of creation (by modification)\is{verb!creation}\is{verb!creation by modification}. The
impossibility of having an R-reading might be due to the same reasons for
which \emph{edificazione} from \emph{edificare}, as a typical creation
verb\is{verb!creation}, does not display any \isi{result}/object interpretation. %
%Melloni (2011:
%189) 
\citet[189]{Melloni2011} %
%Melloni
%
suggests that a possible R-interpretation is blocked by the
existing lexeme \emph{edificio}.

Analogous behaviour is also exhibited by \emph{reificazione} (\ref{ex:Stekauer:12})
(`reification'), which is acceptable only in the eventive reading\is{event!eventive reading}.

\largerpage
\begin{exe}
\ex\label{ex:Stekauer:12}
\begin{xlist}
\ex[]{{Le osservazioni di L. C. sulla (costante) reificazione dei
bambini meritano}... → E}
\ex[]{{Le osservazioni di L. C. sulla (costante) *reifica dei bambini
meritano}... → *E
`L. C.'s remarks on the (constant) reification of  children
deserve...'}
\ex[*]{{La reificazione è interessante} → *R}
\ex[*]{{La reifica è interessante} → *R
\trans (intended) `The reification is interesting'}
\ex[*]{{La reificazione è sul tavolo} → *R}
\ex[*]{{La reifica è sul tavolo} → *R
\trans (intended) `The reification is on the table'}
\end{xlist}
\end{exe}

In the nouns in (\ref{ex:Stekauer:10}-\ref{ex:Stekauer:12}) we thus find that the only possible
interpretation is the one associated with event nominals\is{noun!event},  the \isi{result} reading of the \emph{construction}-type nouns\is{noun!\emph{construction}-type} being ruled
out. Arguably,
the absence of such a \isi{result}/object aspect is the factor that does not
allow for further clipping of the formation. Indeed, the \isi{result}/object
reading seems to be a necessary, albeit not sufficient, condition. As we
will see in the examples below (\ref{ex:Stekauer:13}-\ref{ex:Stekauer:17}), clipping seems to be sensitive
to a special \isi{result}/object meaning which is very close to what
%
%Pustejovsky %
%(1995: 174; cf. Melloni 2011: 109, 111, 142) 
\citeauthor{pustejovsky91} (\citeyear{pustejovsky91}: 164; see \citealt{Melloni2011}: 109, 111, 142) %
%?Pustejovsky
%
calls
\emph{information object}\is{information-object}. It thus appears that where there is such an
information object reading\is{information-object!reading} available to the relevant nominal, the
clipping rule may apply.

I now pass to the discussion of such nouns. I start with
\emph{quantificazione}. In example (13b), we can see that the
shortened form is less acceptable in the eventive reading\is{event!eventive reading}.\footnote{Some
  speakers tend to accept the shortened form even in this eventive\is{event}
  context (Fabio Montermini finds it totally acceptable without
  perceiving any difference whatsoever). Thus, it would be necessary to
  see whether all possible \emph{eventive}\is{event} contexts, offered below for
  \emph{giustifica}, would equally yield a more or less acceptable
  formation. The corpora  offer no example. However, an internet search conducted in July 2017 found 7 hits, including an example where the
  author puts the formation within quotation marks in order to signal
  its peculiar (neological?) status: \emph{Secondo me è una discreta
  opportunità di lavoro con contratto biennale, ma ho bisogno di una
  ``}quantifica'' dei costi \emph{che io non so proprio fare.}} The
referential reading\is{reference!referential reading} -- conveying an information-object\is{information-object!reading} interpretation --
allows for clipping giving rise to a possible \isi{nonce-formation} \emph{°la
quantifica}.\footnote{I follow here \citepos{Corbin87} use of the ° sign to mark
  possible, yet unattested formations. However, as we have seen, the
  formation \emph{quantifica} is modestly attested (albeit to a very
  limited extent).}

\begin{exe}

\ex\label{ex:Stekauer:13}
\begin{xlist}


\ex[]{{La quantificazione dei costi deve essere effettuata al più
presto} → E}
\ex[?*]{{La quantifica dei costi deve essere effettuata al più presto}
→ ?*E
\trans `The quantification of the costs must be carried out immediately'}
\ex[]{{La quantificazione (dei costi) contiene un errore} → R}
\ex[°]{{La quantifica (dei costi) contiene un errore} → R
\trans `The quantification (of the costs) contains an error'}
\ex[]{{La quantificazione è sul tavolo} → R}
\ex[°]{{La quantifica è sul tavolo} → R
\trans `The quantification is on the table'}
\end{xlist}
\end{exe}
I argue that the pair \emph{giustificazione / giustifica}, seen above in
example (\ref{ex:Stekauer:4}), repeated here as (\ref{ex:Stekauer:14}), shows essentially the same behaviour
despite %
%Montermini \& Thornton's (2014: 192) 
\citeposp{MonterminiThornton2014}{192} %
%Montermini-Thornton
%
claim about its total
\isi{synonymy}.

\begin{exe}
\ex\label{ex:Stekauer:14}
\begin{xlist}
\ex[]{{Le frequenti giustificazioni dell'assenza (da parte degli
studenti) sono intollerabili} → E}
\label{ex:Stekauer:14a}
\ex[*]{{Le frequenti giustifiche dell'assenza (da parte degli
studenti) sono intollerabili} → *E
\trans `The frequent justifications for absence (on the part of the students) are
intolerable'}
\label{ex:Stekauer:14b}
\ex[]{{La giustificazione dell'assenza è falsa} → R}
\ex[]{{La giustifica dell'assenza è falsa} → R
\trans `The  justification for absence is false'}
\ex[]{{La giustificazione è sul tavolo} → R}
\ex[]{{La giustifica è sul tavolo} → R
\trans `The justification is on the table'}
\end{xlist}
\end{exe}

The example thus deserves more discussion. %
%Montermini \& Thornton (2014:
%192) 
\citet[192]{MonterminiThornton2014} %
%Montermini-Thornton
%
claim that \emph{giustificazione} and \emph{giustifica} are
absolutely synonymous (differing only in the register, the latter being
typical of a school jargon). To support this apparently indubitable
fact, they adduce not only their native speaker judgements but also some
corpus evidence, such as the (fixed) sequence \emph{libretto delle
giustificazioni / libretto delle giustifiche} which appears in a large
number of official school rules and regulations. However, I argue that
the \isi{synonymy} of this pair is limited to just the \emph{referential}
reading\is{reference!referential reading} where, indeed, the two formations are wholly interchangeable.
Yet, in the eventive readings\is{event!eventive reading}, the \isi{synonymy} is far less obvious.

First, as shown above in examples (\ref{ex:Stekauer:14a}, \ref{ex:Stekauer:14b}), if subjected to different tests
of \isi{actionality}, the form \emph{giustifica} turns out to be ruled out.
Drawing (loosely) on %
%Anscombre's (1986) 
\citepos{Anscombre1986} %
%Anscombre
%
tests of \isi{actionality}, I point
out that the following constructions highlight the problems at hand.

\begin{exe}
\ex\label{ex:Stekauer:15}
\begin{xlist}
\ex {Gli studenti hanno sempre trovato un metodo di giustificazione
/*giustifica delle loro assenze}

\trans `The students have always found a method of justification of their
absences'

\ex {In caso di mancata giustificazione / ?? giustifica dell'assenza
da parte degli alunni, verrà attivata un'azione disciplinare} \footnote{For
  some speakers, in fact, \emph{giustifica} is acceptable even in this
  dynamic reading, while for others it tends to be ruled out.}

\trans `Failure on the part of the student to provide justification of the absence may
result in disciplinary action'

\ex {Ora procediamo alla giustificazione/*giustifica delle assenze }

\trans `Now let's move on to justifying the absences'

\ex {Non si può accettare una giustificazione/*giustifica così
frettolosa}

\trans `It's not possible to accept such a hasty justification'
\end{xlist}
\end{exe}

What I stress is that the clipped form, displaying a clear
information-object\is{information-object!meaning} meaning (\emph{la giustifica} is primarily a written
document), is far less acceptable in all eventive readings\is{event!eventive reading} enhanced by
the constructions of the type seen in (\ref{ex:Stekauer:15}). I argue that such a semantic
condition, though being probably just a slight tendency, can be best
seen in the example of \emph{falsificazione}. The underlying verb,
\emph{falsificare}, can have two meanings, a material one of
\emph{falsificare la moneta, la carta di credito} etc. (to falsify the
money, the credit card) and a Popperian sense of \emph{falsificare
un'ipotesi} (to falsify a hypothesis). When  \emph{falsification} is
understood in the ``material'' sense, clipping seems to be ruled out (\ref{ex:Stekauer:16}), but when it comes to the other meaning, an information-object\is{information-object!reading}
reading appears to be more acceptable (\ref{ex:Stekauer:17}) given that the
falsification of a hypothesis may in fact be a written document.

\begin{exe}
\ex\label{ex:Stekauer:16}
\begin{xlist}
\ex[]{{La falsificazione delle carte di credito (da parte di alcune
persone) è sempre stata facile} → E}
\label{ex:Stekauer:16a}
\ex[*]{{La falsifica delle carte di credito (da parte di alcune
persone) è sempre stata facile} → *E
\trans `The falsification of the credit cards (by some people) was always
easy'
\label{ex:Stekauer:16b}}
\ex[]{{Questa carta di credito è una falsificazione} → *R}
\label{ex:Stekauer:16c}
\ex[*]{{Questa carta di credito è una falsifica} → *R
\trans `This credit card is a falsification'
\label{ex:Stekauer:16d} }
\ex[]{{La falsificazione (della carta di credito) è sul tavolo} → *R
}\label{ex:Stekauer:16e}
\ex[*]{{La falsifica (della carta di credito) è sul tavolo} → *R
\trans `The falsification (of the credit card) is on the table'}
\label{ex:Stekauer:16f} 
\end{xlist}
\ex\label{ex:Stekauer:17}
\begin{xlist}
\ex[]{{La falsificazione di
quella ipotesi (da parte dello studioso) non ha richiesto molto tempo} →
E}
\label{ex:Stekauer:17a}
\ex[*]{{La falsifica di quella ipotesi (da parte dello studioso) non
ha richiesto molto tempo} → *E
\trans `The falsification of that hypothesis (by the scholar) didn't take much
time'}
\label{ex:Stekauer:17b} 
\ex[]{{La falsificazione (di quella ipotesi) è geniale} → R}
\label{ex:Stekauer:17c} 
\ex[°]{{La falsifica (di quella ipotesi) è geniale} → R
\trans `The falsification (of that hypothesis) is brilliant'}
\label{ex:Stekauer:17d} 
\ex[]{{La falsificazione è sul tavolo} → R}
\label{ex:Stekauer:17e} 
\ex[°]{{La falsifica è sul tavolo} → R
\trans `The falsification is on the table'}
\label{ex:Stekauer:17f} 
\end{xlist}
\end{exe}

What the two contexts have in common is a possibility of having a
\isi{result}-object interpretation. But while in (\ref{ex:Stekauer:16c}--\ref{ex:Stekauer:16f}) the
referential reading\is{reference!referential reading} is more ``material'', in (\ref{ex:Stekauer:17c}--\ref{ex:Stekauer:17f}), the
information-object\is{information-object!reading} reading of \emph{falsificazione} strongly favours the
acceptability of the clipped variant \emph{falsifica} %
%(see also
%Montermini \& Thornton 2014: 196, n. 16 on \emph{falsifica})
\citep[see also][196, note 16 on \emph{falsifica}]{MonterminiThornton2014}%
%Montermini-Thornton
%
.\footnote{The
  form \emph{falsifica} is in fact attested on the Internet only a
  couple of times, in a context that seems to be due to strong analogy
  with \emph{verifica}: ``...sostituire alle procedure rigorose di
  verifica e \textbf{falsifica di} proposizioni scientifico-sperimentali
  un metodo simile a quello storico-comprensivo...''; ``...i dati
  sperimentali sono il fondamento della verifica/\textbf{falsifica di}
  ogni ipotesi scientifica...''; ``...isolare e selezionare quei fatti,
  e quei modi di viverli, che consentono la verifica (o
  \textbf{falsifica) di} date ipotesi...''; ``...Gli epistemologi hanno
  così iniziato a riflettere e a cercare situazioni di verifica o di
  \textbf{falsifica di} queste ipotesi...''}
I take this case, along with the others discussed above, as an example
of Frazdin's hypothesis that  %
%Fradin \& Kerleroux's (2009: 86) 
 %
%Fradin-Kerleroux
%
hypothesis according to which

\begin{quote}

[\ldots] un procédé dérivationnel donné opère de manière discriminante sur
l'une ou l'autre de ces significations. {[}a given derivational process
operates differentially on one or the other of these
meanings.{]} \citep[86]{Fradin09}
\end{quote}

\section{Concluding remarks}\label{Stekauer-concluding-remarks}

On the basis of the data so far analyzed -- which represent only a very
limited sample -- I now conclude by summarizing my main proposal.

I maintain that the clipping rule is sensitive to the information-object\is{information-object!meaning}
meaning of the construction in -\emph{zione}\is{suffixation!in -\emph{zione}}. Such an information-object\is{information-object!meaning}
meaning can be predicted from the general semantics of the base verb.

What
%
%Melloni (2011: 108) 
\citet[108]{Melloni2011} %
%Melloni
%
considers to be the core meaning of what she calls the R
nominals  may be captured in the following four or five classes
based on the semantics of the underlying verb: the \emph{product}\is{verb!product},
\emph{means}\is{verb!means}, \emph{path and measure}\is{verb!path and measure}, \emph{entity in state}\is{verb!entity in state} verbs and
the \emph{sense extensions}\is{sense extension}. She shows that inside of the
\emph{product}-\emph{oriented} nominals a further division is to be made
between \emph{creation/result-object} verbs\is{verb!creation} \is{verb!result-object}(such as \emph{costruire}),
\emph{creation-by-representation verbs}\is{verb!creation by representation} (such as \emph{tradurre}) and
\emph{creation-by-modification verbs}\is{verb!creation by modification} (such as \emph{correggere}). The
\emph{representation} (and also \emph{modification}) class of
\emph{creation verbs}\is{verb!creation} can, as %
Melloni puts it

\begin{quote}

[\ldots]  undergo a metonymic displacement and convey the concrete
interpretation of its container object, (a piece of paper, for instance)
[\ldots]. \citep[201]{Melloni2011}
\end{quote}

Furthermore, still inside this class of \emph{creation verbs}\is{verb!creation}, there is
another non-proto\-typical group of \emph{speech act verbs}\is{verb!speech act} %
%(cf. Melloni 2011: 213-214)
\citep[see][213--214]{Melloni2011} %
%
 which convey a proposition that can be, once again,
understood as \emph{information object}\is{information-object} à la %
%Pustejovsky (1995)
\citet{pustejovsky91}%
%?Pustejovsky
%
, as, for
example, \emph{confessione, communicazione} etc. In such a perspective,
we could also reconsider the already lexicalized nouns as, for instance,
\emph{condanna, confisca, deroga, proroga, ratifica, nomina} etc. %
%(cf.
%Thornton 2004: 519)
\citep[see][519]{Thornton2004}%
%Thornton
%
. But this is, of course, a matter of future
research. For the present, I only wished to show that a general
\emph{information-object}\is{information-object!meaning} meaning can indeed be a relevant factor in a
(marginal) process of clipping of the Italian nouns in
\emph{-ificazione}.

\section*{Acknowledgments}

This paper was first presented back in
  2008 at the \emph{13th International Morphology Meeting} in Vienna,
  and then shelved for various reasons. I wish to thank all those who
  were willing, back then and now, to provide me with their critical
  comments: Fabio Montermini, Anna M. Thornton, Antonietta Bisetto,
  Chiara Melloni, Georgette Dal, and, \emph{last but not least}, Fabio
  Ripamonti. Of course, none of them is to be held responsible for the
  (controversial) ideas expressed here. This study was supported by the 
  Charles University project Progres 4 (Language in the shiftings of time, 
  space, and culture) and by the European Regional Development Fund, Project 
  ``Creativity and Adaptability as Conditions of the Success of Europe in 
  an Interrelated World'' (No. CZ.02.1.01/0.0/0.0/16\_019/0000734).


%\nocite{Anscombre1986}
%\nocite{Fradin09}
%\nocite{Fradin03b}
%\nocite{Fradin2003}
%\nocite{Kerleroux1997}
%\nocite{Melloni2006}
%\nocite{Melloni2007}
%\nocite{Melloni2011}
%\nocite{MonterminiThornton2014}
%\nocite{Spencer13}
%\nocite{Stichauer2006}
%\nocite{Thornton1990a}
%\nocite{Thornton2004}

\is{clipping|)}
\il{French|)}
\il{Italian|)}

{\sloppy
    \printbibliography[heading=subbibliography,notkeyword=this]
}


\end{document}
