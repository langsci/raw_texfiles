\documentclass[output=paper]{langsci/langscibook}
\ChapterDOI{10.5281/zenodo.1407003}
\author{Hilaria Cruz\affiliation{Dartmouth College}  \lastand Gregory Stump\affiliation{Professor emeritus, University of Kentucky}}
\title{The morphology of essence predicates in Chatino}
\setlength{\tabcolsep}{3pt} 

\abstract{ In the Chatino language [\ili{Oto-Manguean}; Mexico], essence predicates are a class of predicative lexemes exhibiting a special complex of properties that distinguishes them from other kinds of
predicates. We characterize this complex of properties with evidence from the San Juan Quiahije (SJQ) variety of Chatino. After examining the principal morphosyntactic characteristics of essence predicates, we focus particular attention on their patterns of person/number marking, on which basis we distinguish two possible hypotheses about the grammatical status of essence predicates:  the possessed-subject hypothesis and the compound predicate hypothesis. We then assess these hypotheses in light of four kinds of evidence:  the structural variety of essence predicates, their external syntax, their general lack of semantic compositionality, and their relation to the distributional flexibility of subject-agreement marking in Chatino.  On the basis of this evidence, we conclude that neither the possessed-subject hypothesis nor the compound predicate hypothesis is fully adequate; we therefore propose an alternative way of situating essence predicates in the wider context of Chatino morphosyntax. }

\maketitle
\begin{document}
\selectlanguage{english}

\il{Chatino|(}
\il{Chatino!San Juan Quiahije variety|(}
\is{predicate!essence predicate|(}

\section{\rephrase{}{Introduction}} 
\largerpage[-3]
Our intention here is to characterize a distinctive class of predicates in Chatino; we call this the class of \textsc{essence predicates}.  As we show, the members of this class share certain distinctive morphosyntactic characteristics; at the same time, they are also heterogeneous with respect to various criteria. Their interest here resides in the superficial ambiguity of their structure:  in some ways, this resembles the syntactic combination of a verb and its subject, while in other ways, it resembles the morphological structure of a compound predicate.  In section 1, we examine the fundamental features of essence predicates.  Their patterns of person/number marking (section 2) suggest two alternative analyses of their structure, one syntactic, the other morphological.  In section 3, we examine four characteristics of essence predicates as a way of gauging the relative adequacy of the two competing analyses.  In view of the equivocal outcome of this examination, we conclude (section 4) that essence predicates  are, in fact, neither verb-subject combinations nor ordinary compound predicates, but lexemes whose realization is invariably periphrastic\is{periphrasis} and whose content stems from the special function of a handful of grammaticalized nouns.

\section{Basic characteristics of essence predicates}


One of the defining features of essence predicates  is their structure, which comprises a \isi{predicative base} followed by a \is{noun!nominal component}nominal component.  For example, the essence predicate \emph{ndi}\expo{4}
\emph{riq}\expo{2} `s/he was thirsty'\footnote{Here and throughout, we generally use a verb's third-person singular completive form as its
citation form; deviations from this practice are duly noted. We employ the following abbreviations:
\textsc{cpl} `completive aspect', \textsc{prog} `progressive aspect', \textsc{hab} `habitual aspect', \textsc{pot} `potential mood'; \textsc{dem}
`demonstrative'; \textsc{abs} signifies that a referring expression's referent is absent; \textsc{ess} = \emph{riq}\expo{2}, \emph{tye}\expo{32} or \emph{qin}\expo{4}; \textsc{ev.mod} = \isi{event modifier}; and \textsc{cbm} = \isi{cranberry morpheme}. A superscript 0 represents a
floating super high \isi{tone}, 1 a high \isi{tone}, 2 a mid high \isi{tone}, 3 a low mid \isi{tone}, and 4 a low \isi{tone}.
Contour tones are represented as combinations of these numerals. For details concerning the SJQ Chatino tone system\is{tone}, see \cite{ECruz2011}, \cite{WoodburyToAppear}.} comprises the \isi{predicative base} \emph{ndi}\expo{4} `be thirsty' and the noun \emph{riq}\expo{2} `essence'; its inflectional paradigm is given in \tabref{tab:CruzStump:sjq-1}.  Essence predicates exhibit a wide range of predicative bases\is{base!predicative base}, but there is only a handful of choices for the \is{noun!nominal component}nominal component, the most common being \emph{riq}\expo{2}. 


\begin{table}


%\resizebox{\textwidth}{!}{%
\begin{tabular}{lllll}
\lsptoprule
&\textsc{completive}&\textsc{progressive}&\textsc{habitual}&\textsc{potential}\\
\midrule
\textsc{1sg}&\emph{ndi}\expo{4} \emph{renq}\expo{20}&\emph{ndi}\expo{32} \emph{renq}\expo{20}&\emph{ndyi}\expo{4} \emph{renq}\expo{20}&\emph{tyi}\expo{32} \emph{renq}\expo{20}\\
\textsc{2sg}&\emph{ndi}\expo{4} \emph{riq}\expo{1}&\emph{ndi}\expo{32} \emph{riq}\expo{1}  &\emph{ndyi}\expo{4} \emph{riq}\expo{1}&\emph{tyi}\expo{32} \emph{riq}\expo{1}\\
\textsc{3sg}&\emph{ndi}\expo{4} \emph{riq}\expo{2}&\emph{ndi}\expo{32} \emph{riq}\expo{2}&\emph{ndyi}\expo{4} \emph{riq}\expo{2}&\emph{tyi}\expo{32} \emph{riq}\expo{2}\\
\textsc{1incl}&\emph{ndi}\expo{4} \emph{renq}\expo{2} \emph{en}\expo{1}&\emph{ndi}\expo{32} \emph{renq}\expo{2} \emph{en}\expo{1}&\emph{ndyi}\expo{4} \emph{renq}\expo{2} \emph{en}\expo{1}&\emph{tyi}\expo{32} \emph{renq}\expo{2} \emph{en}\expo{1}\\
\textsc{1excl}&\emph{ndi}\expo{4} \emph{riq}\expo{2} \emph{wa}\expo{42}&\emph{ndi}\expo{32} \emph{riq}\expo{2} \emph{wa}\expo{42}&\emph{ndyi}\expo{4} \emph{riq}\expo{2} \emph{wa}\expo{42}&\emph{tyi}\expo{32} \emph{riq}\expo{2} \emph{wa}\expo{42}\\
\textsc{2pl}&\emph{ndi}\expo{4} \emph{riq}\expo{2} \emph{wan}\expo{1}&\emph{ndi}\expo{32} \emph{riq}\expo{2} \emph{wan}\expo{1}&\emph{ndyi}\expo{4} \emph{riq}\expo{2} \emph{wan}\expo{1}&\emph{tyi}\expo{32} \emph{riq}\expo{2} \emph{wan}\expo{1}\\
\textsc{3pl}&\emph{ndi}\expo{4} \emph{riq}\expo{2} \emph{renq}\expo{1}&\emph{ndi}\expo{32} \emph{riq}\expo{2} \emph{renq}\expo{1}&\emph{ndyi}\expo{4} \emph{riq}\expo{2} \emph{renq}\expo{1}&\emph{tyi}\expo{32} \emph{riq}\expo{2} \emph{renq}\expo{1}\\
\lspbottomrule
\end{tabular}%}
\caption{Paradigm of the essence predicate \emph{ndi}\expo{4} \emph{riq}\expo{2} `s/he was thirsty' [thirsty essence] in SJQ Chatino}

\label{tab:CruzStump:sjq-1}
\end{table}

\largerpage[-2]

In view of its structure, the inflectional morphology of essence predicates differs from that
of simple verbal lexemes. These differences can be seen by comparing \pagebreak  the inflectional paradigm of the
essence predicate \emph{ndi}\expo{4} \emph{riq}\expo{2} `s/he was thirsty' in \tabref{tab:CruzStump:sjq-1} with that of the simple verbal lexeme \emph{yqan}\expo{42}
`s/he washed' in \tabref{tab:CruzStump:sjq-2}.\footnote{The 1\textsc{incl} clitic appearing as \emph{en}\expo{1} in \tabref{tab:CruzStump:sjq-1} and as \emph{an}\expo{42} $\sim$ \emph{an}\expo{1} in \tabref{tab:CruzStump:sjq-2} gets its vowel quality from its
host and is manifested as a lengthening of the preceding vowel mora.  (Note, however, that verbs with \isi{tone} 14 do not undergo mora lengthening in the first person inclusive, so that superficially, they appear to lack the 1\textsc{incl} enclitic, as in \tabref{tab:CruzStump:sjq-2}.)  Its \isi{tone} is generally determined by a process of progressive \is{tone!tone sandhi}tone sandhi~\citep{Chen2004}; but verbs whose basic \isi{tone} is 4 instead exhibit a regressive process by which their \isi{tone}  becomes 24.  It is evidently the historical reflex
of a clitic that was once constant in form. This constant form survives as the clitic \emph{na}\expo{4} in
Zenzontepec Chatino\il{Chatino!Zenzontepec variety} \citep{Campbell2011}.  For details of the idiosyncratic \isi{sandhi} exhibited by the 1\textsc{incl} enclitic, see \citet{ECruz2011}.} 

  

\begin{table}


%\resizebox{\textwidth}{!}{%
\begin{tabular}{lllll}
\lsptoprule
&\textsc{completive}&\textsc{progressive}&\textsc{habitual}&\textsc{potential}\\
\midrule
\textsc{1sg}&\emph{yqan}\expo{24}&\emph{ntyqan}\expo{1}&\emph{ntyqan}\expo{24}&\emph{yqan}\expo{24}\\
\textsc{2sg}&\emph{yqan}\expo{32}&\emph{ntyqan}\expo{32}&\emph{ntyqan}\expo{42}&\emph{yqan}\expo{42}\\
\textsc{3sg}&\emph{yqan}\expo{42}&\emph{ntyqan}\expo{32}&\emph{ntyqan}\expo{24}&\emph{yqan}\expo{24}\\
\textsc{1incl}&\emph{yqan}\expo{42} \emph{an}\expo{42}&\emph{ntyqan}\expo{1} \emph{an}\expo{1}&\emph{ntyqan}\expo{14}&\emph{yqan}\expo{14}\\
\textsc{1excl}&\emph{yqan}\expo{42} \emph{wa}\expo{42}&\emph{ntyqan}\expo{32} \emph{wa}\expo{42}&\emph{ntyqan}\expo{14} \emph{wa}\expo{42}&\emph{yqan}\expo{14} \emph{wa}\expo{42}\\
\textsc{2pl}&\emph{yqan}\expo{42} \emph{wan}\expo{4}&\emph{ntyqan}\expo{32} \emph{wan}\expo{4}&\emph{ntyqan}\expo{24} \emph{wan}\expo{32}&\emph{yqan}\expo{24} \emph{wan}\expo{32}\\
\textsc{3pl}&\emph{yqan}\expo{42} \emph{renq}\expo{4}&\emph{ntyqan}\expo{32} \emph{renq}\expo{4}&\emph{ntyqan}\expo{24} \emph{renq}\expo{32}&\emph{yqan}\expo{24} \emph{renq}\expo{32}\\
\lspbottomrule
\end{tabular} %}
\caption{\normalfont Paradigm of the simple verbal lexeme \emph{yqan}\expo{42} `s/he washed' in SJQ Chatino}
\label{tab:CruzStump:sjq-2}
\end{table}


As \tabref{tab:CruzStump:sjq-2} shows, the singular forms of a simple verbal lexeme are single, synthetic word
forms inflected both for aspect/mood and for subject person and number. The corresponding plural
forms consist of a verb form inflected for aspect/mood and an enclitic pronominal element marking
person and number; in general, this pronominal element appears only in the absence of an overt
subject constituent, in the presence of which the verb simply appears in its default third-person singular form.  
As \tabref{tab:CruzStump:sjq-1} shows, essence predicates differ from simple verbal lexemes in satisfying what \cite{Rasch02} calls the \isi{Compound Inflection Criterion}, according to which an essence predicate exhibits aspect/mood marking on its \isi{predicative base} but person and number marking on its \is{noun!nominal component}nominal component, where, again, the marking of plural persons takes the form of an enclitic that only  appears in the absence of an overt subject constituent.  The one complication is that in the first-person plural inclusive, subject \isi{agreement} is marked twice, not only by the clitic \textit{en}\expo{1}, but also by ablauting of the \isi{nominal  component}, which appears as \emph{renq}\expo{2} rather than as \emph{riq}\expo{2}  in \tabref{tab:CruzStump:sjq-1}. 

Tables \ref{tab:CruzStump:sjq-1} and \ref{tab:CruzStump:sjq-2} show that the essence predicate \emph{ndi}\expo{4} \emph{riq}\expo{2} is like a verb in inflecting for
aspectual and modal properties; but not all essence predicates are similarly verb-like. We take this
as evidence that essence predicates are heterogeneous with respect to their syntactic category
membership. 
In SJQ Chatino, the criteria in (\ref{ex:CruzStump:1}) are diagnostics of the distinction between verbs and adjectives. By
criterion (\ref{ex:CruzStump:1a}), the predicate \emph{yqan}\expo{42} `s/he washed' in \tabref{tab:CruzStump:sjq-2} is a verb because it exhibits distinct
completive, progressive, habitual and potential subparadigms. By contrast, the predicate \emph{tqi}\expo{4} `sick'
in \tabref{tab:CruzStump:sjq-3} does not, and is therefore an adjective according to criterion (\ref{ex:CruzStump:1a}). Similarly, \emph{yqan}\expo{42} and \emph{tqi}\expo{4} may both be used
predicatively (as in (\ref{ex:CruzStump:2})), but only \emph{tqi}\expo{4} is used attributively (e.g. (\ref{ex:CruzStump:3}a)); in order to modify a noun as
part of a noun phrase, \emph{yqan}\expo{42} must appear as part of a relative clause introduced by the pronominal \emph{no}\expo{4} `one', as
in (\ref{ex:CruzStump:3}b). Thus, criterion (\ref{ex:CruzStump:1b}) also leads to the conclusion that \emph{yqan}\expo{42} is a verb and \emph{tqi}\expo{4}, an adjective.

\ea \label{ex:CruzStump:1}
		\ea\label{ex:CruzStump:1a}	 Verbs inflect for aspect and mood, but adjectives do not.
		\ex\label{ex:CruzStump:1b} Adjectives may be used attributively, but verbs cannot (except as part of a relative clause).
		\z
\ex \label{ex:CruzStump:2} \begin{xlist}
\ex \gll{tqi}\expo{4} {no}\expo{4} {kiqyu}\expo{1}.\\
sick one(s) male\\
\trans `The men are sick.'
\ex\gll {ntyqan}\expo{32} {no}\expo{4} {kiqyu}\expo{1}.\\
		wash.\textsc{prog} one(s) male \\
\trans		`The men are washing.'
		\end{xlist}
\ex \label{ex:CruzStump:3} 
\begin{xlist}
\ex\gll  {no}\expo{4} {kiqyu}\expo{1} {tqi}\expo{4}\\
          one(s) male sick\\
\trans `the sick men'
\ex\gll {no}\expo{4} {kiqyu}\expo{1} *({no}\expo{4})  {ntyqan}\expo{32}\\
     one(s) male  \textcolor{white}{*(}one(s) wash.\textsc{prog}\\
\trans `the men that are washing'
\end{xlist}
\z

\begin{table}


\begin{tabular}{l l}
\lsptoprule
\textsc{1sg}&\emph{tqen}\expo{20}\\
\textsc{2sg}&\emph{tqi}\expo{32}\\
\textsc{3sg}&\emph{tqi}\expo{4}\\
\textsc{1incl}&\emph{tqen}\expo{24} \emph{en}\expo{32}\\
\textsc{1excl}&\emph{tqi}\expo{4} \emph{wa}\expo{42}\\
\textsc{2pl}&\emph{tqi}\expo{4} \emph{wan}\expo{4}\\
\textsc{3pl}&\emph{tqi}\expo{4} \emph{renq}\expo{4}\\
\lspbottomrule
\end{tabular}
\caption{Paradigm of the adjective \emph{tqi}\expo{4} `sick' in SJQ Chatino}
\label{tab:CruzStump:sjq-3}
\end{table}


By these diagnostics, it appears that some essence predicates are verbs and others, adjectives.
Unlike the essence predicate \emph{ndi}\expo{4} \emph{riq}\expo{2} `s/he was thirsty' but like the adjective \emph{tqi}\expo{4} `sick', the essence
predicate \emph{tqi}\expo{4} \emph{riq}\expo{2} [sick essence] `s/he was scornful' in \tabref{tab:CruzStump:sjq-4} does not inflect for aspect and mood.
Moreover, a comparison of (\ref{ex:CruzStump:4}) and (\ref{ex:CruzStump:5}) reveals that while \emph{tqi}\expo{4} \emph{riq}\expo{2} readily appears in attributive
position, the attributive use of \emph{ndi}\expo{4} \emph{riq}\expo{2} requires a relative clause construction. Thus, although \emph{ndi}\expo{4} \emph{riq}\expo{2} and \emph{tqi}\expo{4} \emph{riq}\expo{2} are both essence predicates, the diagnostics in (\ref{ex:CruzStump:1}) suggest that the former is a verb\footnote{This conclusion further implies that \emph{ndi}\expo{4} is itself a verb, but its status as a verb is not independently demonstrable, given that it is a kind of \isi{cranberry morpheme}, appearing as part of the essence predicate \emph{ndi}\expo{4} \emph{riq}\expo{2} and nowhere else.} and the latter, an
adjective.\footnote{The question naturally arises whether an essence predicate's \isi{predicative base} is ever a noun.  There are occasional instances in which this superficially appears to be the case, but closer scrutiny leaves room for doubt.  For example, the essence predicate \emph{tnya}\expo{3} \emph{riq}\expo{2}  `s/he is hardworking' seems to have the noun \emph{tnya}\expo{3} `work' as its \isi{predicative base}, but \emph{tnya}\expo{3} also seems to have adjectival uses, as in
\center{\emph{No}\expo{4} \emph{nga}\expo{24} \emph{tnya}\expo{4} [one be.\textsc{prog} working] `the ones who are authorities'.}}



\begin{table}


\begin{tabular}{l l  }
\lsptoprule
\textsc{1sg}&\emph{tqi}\expo{4} \emph{renq}\expo{20}\\
\textsc{2sg}&\emph{tqi}\expo{4} \emph{riq}\expo{1}\\
\textsc{3sg}&\emph{tqi}\expo{4} \emph{riq}\expo{2}\\
\textsc{1incl}&\emph{tqi}\expo{4} \emph{renq}\expo{2} \emph{en}\expo{1}\\
\textsc{1excl}&\emph{tqi}\expo{4} \emph{riq}\expo{2} \emph{wa}\expo{42}\\
\textsc{2pl}&\emph{tqi}\expo{4} \emph{riq}\expo{2} \emph{wan}\expo{1}\\
\textsc{3pl}&\emph{tqi}\expo{4} \emph{riq}\expo{2} \emph{renq}\expo{1}\\
\lspbottomrule
\end{tabular}
\caption{Paradigm of the essence predicate \emph{tqi}\expo{4} \emph{riq}\expo{2} [sick essence] `s/he is scornful' in SJQ Chatino}

\label{tab:CruzStump:sjq-4}
\end{table}
\ea\label{ex:CruzStump:4}\gll
 {{Ntqan}\expo{24} } {{qin}\expo{32} } {{no}\expo{4} } {{kiqyu}\expo{1}} {{tqi}\expo{4} } {{riq}\expo{2} } {{qnya}\expo{1} } {{kanq}\expo{42}.} \\
 {see.\textsc{cpl}.\textsc{1sg} } {\textsc{acc}} {one} {male } {sick} {essence.3\textsc{sg}} {me} {\textsc{dem}}\\
\glt {`I saw the guy scornful of me.'}\\

\ex\label{ex:CruzStump:5}\gll {{Ntqan}\expo{24} } {{qin}\expo{32} } {{no}\expo{4} } {{kiqyu}\expo{1}} {*({no}\expo{4}) } {{ndi}\expo{32}} {{riq}\expo{2}}{kanq}\expo{42}. \\
 {see.\textsc{cpl}.\textsc{1sg}} {\textsc{acc}} {one} {male } {*(one)} {thirsty.\textsc{prog}} {essence.3\textsc{sg}}\textsc{dem} \\
\glt {`I saw the guy who is thirsty.'}\\

\z


	Most essence predicates denote a particular psychological state or disposition, as the representative examples in \tabref{tab:CruzStump:sjq-5} reveal. Some essence predicates, however, denote a physical state, as in \tabref{tab:CruzStump:sjq-6};
and there are also occasional examples that have an active rather than a stative or dispositional
meaning, as in \tabref{tab:CruzStump:sjq-7}.

 \begin{table}


\begin{tabularx}{\textwidth}{XXX}
\lsptoprule
Essence predicate  & Gloss  &Component parts\\
\midrule
\emph{nkqan}\expo{4} \emph{riq}\expo{2} &  `s/he remembered' & [sit essence] \\
\emph{sa}\expo{4} \emph{riq}\expo{2} &  `s/he was smart, fast, agile' &[light essence] \\
\emph{ndon}\expo{42}  \emph{riq}\expo{2} &  `s/he was happy' &[standing essence] \\
\emph{sqwi}\expo{4} \emph{riq}\expo{2} &`s/he remembered'  & [exist essence] \\
\emph{senq}\expo{14} \emph{riq}\expo{0} & `s/he was upset' & [\textsc{cbm} essence] \\
\emph{qna}\expo{3} \emph{riq}\expo{2} & `s/he pitied' &[pity essence] \\
\emph{xkuq}\expo{42} \emph{riq}\expo{2} & `s/he was sad' &[turn.around essence] \\
\emph{ndwe}\expo{4} \emph{riq}\expo{2} &`s/he worried' &[minced essence] \\
\emph{skwa}\expo{3} \emph{riq}\expo{2} & `s/he was fed up' &[lying essence] \\
\emph{tqi}\expo{4} \emph{riq}\expo{2} & `s/he hated' &[sick essence] \\
\emph{sqwe}\expo{3} \emph{riq}\expo{2}/\emph{tye}\expo{32} & `s/he was generous/happy' & [good essence/chest] \\
\emph{liqa}\expo{14} \emph{riq}\expo{0} &  `s/he was taciturn' &[slow essence] \\
\emph{chin}\expo{4} \emph{nga}\expo{24} 	\emph{tye}\expo{32} & `s/he was scared/queasy' &[ugly feel chest] \\
\emph{ndya}\expo{32} \emph{riq}\expo{2} \emph{tye}\expo{32}&`s/he liked' &[like essence chest] \\
\emph{xqan}\expo{1} \emph{nga}\expo{04} \emph{tye}\expo{32} & `s/he felt angry' &[mean feel chest] \\
\lspbottomrule
\end{tabularx}

\caption{Some representative essence predicates in SJQ Chatino}
\label{tab:CruzStump:sjq-5}
\end{table}


\begin{table}


%\resizebox{\textwidth}{!}{%
\begin{tabularx}{\textwidth}{XXX}
\lsptoprule
&Essence predicate & Gloss of component parts\\
\midrule
`s/he is fair-skinned' &\emph{lwi}\expo{3} \emph{riq}\expo{2} $\sim$ \emph{lwi}\expo{3} \emph{tye}\expo{32} &[clean essence $\sim$ chest] \\
`s/he was thirsty, wheezing'&\emph{ndyi}\expo{4} \emph{riq}\expo{2} &[\textsc{cbm} essence] \\
`s/he is dark-skinned' &\emph{nta}\expo{14} \emph{riq}\expo{0} &[dark essence] \\
`s/he is hungry' &\emph{nteq}\expo{32} \emph{riq}\expo{2} &[hungry essence] \\
`s/he is skinny' &\emph{ti}\expo{4} \emph{riq}\expo{2} &[skinny essence] \\
`s/he is sturdy'&\emph{tjoq}\expo{4} \emph{riq}\expo{2} &[strong essence] \\
`s/he is cold'&\emph{tlyaq}\expo{4} \emph{riq}\expo{2} &[cold essence] \\
`s/he is skinny'&\emph{tyjyan}\expo{20} \emph{riq}\expo{2} &[skinny essence] \\
`s/he is hot'&\emph{tykeq}\expo{14} \emph{riq}\expo{0} &[hot essence] \\
 \lspbottomrule
 \end{tabularx}%}
\caption{Some essence predicates denoting physical states in SJQ Chatino}

\label{tab:CruzStump:sjq-6}
\end{table}



\begin{table}


%\resizebox{\textwidth}{!}
 \caption{Some essence predicates with an active denotation in SJQ Chatino}

\label{tab:CruzStump:sjq-7}
\end{table}

 In nearly all cases, \emph{riq}\expo{2} `essence' seems to be interpretable as `X's self', making the essence predicate
similar to a lexically reflexive verb\is{verb!reflexive} in a language like \ili{French}; \emph{skeq}\expo{1} \emph{riq}\expo{0} `il se m\'eprend', \emph{sqwi}\expo{4} \emph{riq}\expo{2}
`elle se souvient', \emph{ndwe}\expo{4} \emph{riq}\expo{2} `il s'inqui\`ete', \emph{tno}\expo{4} \emph{nga}\expo{24} \emph{tye}\expo{32} `elle se sent courageuse'. Note, however,
that argument reflexives are expressed by means of a reflexive pronoun in Chatino, as in (\ref{ex:CruzStump:6}) and (\ref{ex:CruzStump:7}).  We return to the semantic issues raised by essence predicates in \sectref{sec:stump:3.3}.

\ea \label{ex:CruzStump:6}\gll
 {{Ti}\expo{2} } {{kwenq}\expo{42} {en}\expo{42} } {{qnyi}\expo{4} }{qnya}\expo{4}.\\
 {\textsc{ev.mod}:only} {myself} {hit.\textsc{cpl.obj.pron.1sg}}\\
\glt {`I flagellated myself.'}\\

\ex  \label{ex:CruzStump:7}\gll {{ti}\expo{2} } {{kwiq}\expo{42} } {{ti}\expo{4}} {{Tyu}\expo{14}} {{kwa}\expo{0}} {{qnyi}\expo{1} }{qin}\expo{24}.\\
 {\textsc{ev.mod}:still} {himself} {\textsc{ev.mod}:only} {Peter} {\textsc{det}} {hit.\textsc{cpl.obj.pron.3sg}} \\
\glt {`Peter flagellated himself.'}\\

\z



\section{Person/number marking in essence predicates}

An essence predicate exhibits person/number marking on its \is{noun!nominal component}nominal component.  Person/number marking has a complex distributional pattern in Chatino; in this section, we propose to situate essence predicates within this complex pattern by comparing them with simplex verbs, inalienably possessed\is{inalienable possession!possessed} nouns, and compound verbs\is{compounding!compound verb}.  These comparisons lead us to entertain two competing hypotheses about the morphosyntax of essence predicates:  the \isi{possessed-subject hypothesis} (according to which essence predicates embody a verb-subject construction, defined by the syntax of Chatino) and the \isi{compound predicate hypothesis} (according to which essence predicates belong to a larger class of predicative\textemdash mainly verbal\textemdash compounds, defined by the morphology of the language).

\subsection{Comparison to person/number marking in simplex verbs}

A prominent feature of Chatino grammar is the heavy use of \isi{tone} contrasts in its inflectional system \citep{ECruz2011,CruzWoodbury2013}.
Consider, for example, the paradigm of the simple verb \emph{sqi}\expo{2} `s/he bought' in \tabref{tab:CruzStump:sjq-8}. In this
paradigm, contrasts in aspect/\linebreak mood are marked in three ways:  (i) a nasal prefix distinguishes the progressive and the habitual from the completive and the potential, (ii) a stem-initial consonant alternation distinguishes the completive and the progressive (both with stem-initial \emph{s}) from the habitual (stem-initial \emph{ch}) and the potential (stem-initial \emph{x}), and (iii) the completive and the progressive share one pattern of \isi{tone} alternation, while the habitual and the potential share another.  Within a
particular aspect/mood subparadigm, contrasts in person and number are marked both tonally and\textemdash
in the plural forms\textemdash by the use of personal clitics (in the absence of an overt subject constituent); in first-person singular and first-person plural inclusive forms, the verb stem also exhibits nasalization, sometimes in combination with ablaut.  
Verbs fall into a number of different conjugation classes that are distinguished mainly by their
paradigms' patterns of \isi{tone} alternation. Thus, despite some similarities, the pattern of \isi{tone}
alternation in the paradigm of \emph{sqi}\expo{2} `s/he bought' contrasts with the pattern of \emph{yqan}\expo{42} `s/he washed'
observed earlier in \tabref{tab:CruzStump:sjq-2}; these contrasting \isi{tone} patterns are given in \tabref{tab:CruzStump:sjq-9}. For extensive
details on conjugation-class distinctions in Chatino, see \cite{CruzWoodbury2013}, \cite{WoodburyToAppear}.

\begin{table}


%\resizebox{\textwidth}{!}{%
\begin{tabular}{lllll}
\lsptoprule
&\textsc{completive}&\textsc{progressive}&\textsc{habitual}&\textsc{potential}\\
\midrule
\textsc{1sg}&\emph{sqen}\expo{40}&\emph{nsqen}\expo{40}&\emph{nchqin}\expo{40}&\emph{xqin}\expo{40}\\
\textsc{2sg}&\emph{sqi}\expo{1}&\emph{nsqi}\expo{1}&\emph{nchqi}\expo{20}&\emph{xqi}\expo{20}\\
\textsc{3sg}&\emph{sqi}\expo{2}&\emph{nsqi}\expo{2}&\emph{nchqi}\expo{14}&\emph{xqi}\expo{14}\\
\textsc{1incl}&\emph{sqen}\expo{2} \emph{en}\expo{1}&\emph{nsqen}\expo{2} \emph{en}\expo{1}&\emph{nchqin}\expo{14}&\emph{xqin}\expo{14}\\
\textsc{1excl}&\emph{sqi}\expo{2} \emph{wa}\expo{42}&\emph{nsqi}\expo{2} \emph{wa}\expo{42}&\emph{nchqi1}\expo{40} \emph{wa}\expo{42}&\emph{xqi1}\expo{40} \emph{wa}\expo{42}\\
\textsc{2pl}&\emph{sqi}\expo{2} \emph{wan}\expo{1}&\emph{nsqi}\expo{2} \emph{wan}\expo{1}&\emph{nchqi}\expo{14} \emph{wan}\expo{0}&\emph{xqi}\expo{14} \emph{wan}\expo{0}\\
\textsc{3pl}&\emph{sqi}\expo{2} \emph{renq}\expo{1}&\emph{nsqi}\expo{2} \emph{renq}\expo{1}&\emph{nchq}\expo{14} \emph{renq}\expo{0}&\emph{xqi}\expo{14} \emph{renq}\expo{0}\\
\lspbottomrule
\end{tabular} %}
\caption{Paradigm of the verbal lexeme \emph{sqi}\expo{2} `s/he bought' in SJQ Chatino}

\label{tab:CruzStump:sjq-8}
\end{table}

\begin{table}

%\resizebox{\textwidth}{!}{%
\begin{tabular}{llllll }
\lsptoprule
&&\textsc{completive}&\textsc{progressive}&\textsc{habitual}&\textsc{potential}\\
\midrule
\emph{sqi}\expo{2}&\textsc{1sg}&40&40&40&40\\
`s/he bought'&\textsc{2sg}&1&1&20&20\\
&\textsc{3sg}&2&32&14&14\\
&\textsc{1incl}&2-1&2-1&14&14\\
&\textsc{1excl}&2-42&2-42&140-42&140-42\\
&\textsc{2pl}&2-1&2-1&14-0&14-0\\
&\textsc{3pl}&2-1&2-1&14-0&14-0\\
\midrule
\emph{yqan}\expo{42}&\textsc{1sg}&24&24&24\\
`s/he washed'&\textsc{2sg}&32&32&42&42\\
&\textsc{3sg}&42&32&24&24\\
&\textsc{1incl}&42-42&1-1&14&14\\
&\textsc{1excl}&42-42&32-42&14-42&14-42\\
&\textsc{2pl}&42-4&32-4&24-32&24-32\\
&\textsc{3pl}&42-4&32-4&24-32&24-32\\
\lspbottomrule
\end{tabular}%}
\caption{Tone\is{tone} patterns of two verbal lexemes in SJQ Chatino}

\label{tab:CruzStump:sjq-9}
\end{table}

Essence predicates participate in this system of \isi{tone} contrasts, but in a different manner from simplex verbs.  In the inflection of a simplex verb, a verb form's \isi{tone} exhibits a kind of cumulative exponence, serving to distinguish (or to help distinguish) both the form's aspect/mood and its person/number. In the inflection of an essence predicate, by contrast, forms do not exhibit this sort of cumulation, but conform to the \isi{Compound Inflection Criterion}, with the \isi{predicative base} carrying the \isi{tone} relevant to identifying its aspect or mood and its \is{noun!nominal component}nominal component carrying the \isi{tone} that helps distinguish its person and number.  (See again the inflection of \emph{ndi}\expo{4} \emph{riq}\expo{2} `s/he was thirsty' in \tabref{tab:CruzStump:sjq-1}.)


\subsection{Comparison to person/number marking in inalienably possessed\is{inalienable possession!possessed}  nouns}

The exponents of person and number employed in verb inflection also appear in the
inflection of nouns, where they serve to express the properties of an inalienable possessor\is{inalienable possession!possessor}. Thus, in
the paradigm of the noun \emph{skon}\expo{2} `arm' in \tabref{tab:CruzStump:sjq-10}, the person and number of an inalienable possessor\is{inalienable possession!possessor} 
are expressed by \isi{tone} and\textemdash in the plural (in the absence of an overt possessor constituent)\textemdash by a
clitic. Different nouns exhibit different patterns of \isi{tone} alternation in their inflection for an
inalienable possessor\is{inalienable possession!possessor}; thus, the \isi{tone} pattern in the paradigm of \emph{yqan}\expo{1} `mother' (\tabref{tab:CruzStump:sjq-11}) is different
from that of \emph{skon}\expo{2} `arm'. \cite{ECruz2016} distinguishes seven classes of nouns according to their
patterns of \isi{tone} alternation.


\begin{table}
%\resizebox{\textwidth}{!}
\caption{Inflection of the noun \emph{skon}\expo{2} `arm' for an inalienable possessor's\is{inalienable possession!possessor} person and number in SJQ Chatino (E. Cruz) }

\label{tab:CruzStump:sjq-10}
\end{table}

\begin{table}

\begin{tabular}{l l  l  }
\lsptoprule
Possessor&Possessum\\
\midrule
\textsc{1sg}&\emph{yqan}\expo{20}&`my mother'\\
\textsc{2sg}&\emph{yqan}\expo{42}&`your (sg) mother'\\
\textsc{3sg}&\emph{yqan}\expo{1}&`his/her mother'\\
\textsc{1incl}&\emph{yqan}\expo{1} \emph{an}\expo{1}&`our mother'\\
\textsc{1excl}&\emph{yqan}\expo{1} \emph{wa}\expo{42}&`our mother'\\
\textsc{2pl}&\emph{yqan}\expo{1} \emph{wan}\expo{24}&`your (pl) mother'\\
\textsc{3pl}&\emph{yqan}\expo{1} \emph{renq}\expo{24}&`their mother'\\
\lspbottomrule
\end{tabular}
\caption{Inflection of the noun \emph{yqan}\expo{1} `mother' for an inalienable possessor's\is{inalienable possession!possessor} person and number in SJQ Chatino (E. Cruz)}

\label{tab:CruzStump:sjq-11}
\end{table}


In view of this correspondence of form between a verb's subject-agreement\is{agreement} marking and a noun's inalienable possessor\is{inalienable possession!possessor} marking, one might hypothesize that an essence predicate's \is{noun!nominal component}nominal component is in fact a subject denoting an individual's inalienably possessed\is{inalienable possession!possessed} essence, and that its person-number marking therefore marks the person and number of the
possessor of this essence. Indeed, \emph{riq}\expo{2} belongs to an inflection class differing minimally from
that of \emph{skon}\expo{2} `arm', exhibiting the same pattern of \isi{tone} alternation as in \tabref{tab:CruzStump:sjq-10} except in the
first-person singular (where \emph{riq}\expo{2} exhibits \isi{tone} 20 instead of \isi{tone} 40). Accordingly, given the
additional fact that Chatino is verb-initial, one might be drawn to conclude that the literal sense of the form \emph{ndi}\expo{4} \emph{renq}\expo{20} (analyzed in
\tabref{tab:CruzStump:sjq-1} as `I was thirsty') is `my essence is thirsty'\textemdash that of a verb-subject combination whose subject is the noun \emph{riq}\expo{2} `essence'
inflected for a first-person singular inalienable possessor\is{inalienable possession!possessor}  and whose predicate is, appropriately, the third-person
singular progressive form of \emph{ndi}\expo{32} `be thirsty'. On this \textsc{possessed-subject hypothesis}\is{Possessed-Subject Hypothesis}, an overt noun phrase apparently serving as the
subject of an essence predicate is instead seen as a possessor, so that (i) \emph{no}\expo{4} \emph{kyqyu}\expo{1} \emph{kwa}\expo{3} `that
guy' is a possessor in (\ref{ex:CruzStump:8}) exactly as in (\ref{ex:CruzStump:9}), and (ii) the head of the subject constituent in (\ref{ex:CruzStump:8}) is \emph{riq}\expo{2}
`(his) essence', paralleling \emph{tqwa}\expo{4} `(his) mouth' in (\ref{ex:CruzStump:9}). 

\begin{exe}
	\ex\label{ex:CruzStump:8}\gll {{La}\expo{1}} {{riq}\expo{2}} {{no}\expo{1}} {{kyqyu}\expo{1}} {{kwa}\expo{3}.}\\
	 {open} {essence} {one} {male} {that} \\
	\glt {`That guy is friendly, talkative.'}
	{[}= `That guy's essence is open', according to the \isi{possessed-subject hypothesis}.{]} \\

	\ex\label{ex:CruzStump:9}\gll {{La}\expo{1} } {{tqwa}\expo{4} } {{no}\expo{4} } {{kyqyu}\expo{1}} {{kwa}\expo{3}.}\\
	 {open} {mouth} {one} {male} {that}\\
	\glt {`The guy's mouth is open.'}\\
\end{exe}

This is a tempting analysis, but there is also an alternative possibility\textemdash the \textsc{compound predicate hypothesis}\is{Compound-Predicate Hypothesis}, according to which essence predicates are a class of compound predicates taking mostly experiencer subjects.  In order to evaluate this hypothesis, we now consider person/number marking in compound predicates in SJQ Chatino.  



\subsection{Comparison to person/number marking in compound verbs}
\is{compounding!compound verb|(}
\label{sec:stump:3.3}

Consider the compound verbs \emph{yku}\expo{4} \emph{jyaq}\expo{3} `s/he tasted' [eat amount] and \emph{ykwiq}\expo{4}  \emph{sla}\expo{3} [speak tiredness] `s/he dreamed', whose paradigms are given in Tables \ref{tab:CruzStump:sjq-12} and \ref{tab:CruzStump:sjq-13}.  Each compound consists of a verbal element (\emph{yku}\expo{4} `s/he ate', \emph{ykwiq}\expo{4} `s/he spoke') and a nominal element (\emph{jyaq}\expo{3}  `amount', \emph{sla}\expo{3} `tiredness').  The verbal element is like an essence predicate's \isi{predicative base}, inflecting for aspect/mood but not ordinarily for person and number (though the verbal element sometimes exhibits \isi{agreement} in the first person singular, as in \tabref{tab:CruzStump:sjq-12}); likewise, the nominal element is like an essence predicate's \is{noun!nominal component}nominal component, since it carries the person/number inflection. In other words, the inflectional pattern again tends to conform to Rasch's \isi{Compound Inflection Criterion}.\footnote{Compound predicates are nevertheless somewhat varied in their properties in SJQ Chatino.  Compound verbs whose inflection deviates from the \isi{Compound Inflection Criterion} may do so in more than one way.
 In the inflection of some compound verbs, person and number, like aspect and mood, are marked on the first, verbal element rather than on the following nominal element (e.g. \emph{snyi}\expo{4} \emph{chaq}\expo{3} `s/he dealt, negotiated' [grab word]); in the inflection of other compound verbs, aspect and mood, like person and number, are marked on the second, nominal element rather than on the preceding verbal element (e.g. \emph{xi}\expo{42} \emph{skwa}\expo{3} `s/he turned (s.o.) over' [cause be.in.elevated.position]); still others sporadically exhibit person/number marking on both the verbal and the nominal elements (as with \emph{ykon}\expo{1} \emph{jyanq}\expo{3} `I tasted' in \tabref{tab:CruzStump:sjq-12}); and yet others exhibit marking of aspect and mood on both the verbal and the nominal elements (e.g. \emph{sti}\expo{1} \emph{qo}\expo{20} `s/he made fun of' [laugh with]). See \cite{CruzWoodbury2013} for details concerning these deviations from the \isi{Compound Inflection Criterion} in SJQ Chatino.}

\begin{table}

	\resizebox{\textwidth}{!}{%
\begin{tabular}{lllll }
\lsptoprule
&\textsc{completive}&\textsc{progressive}&\textsc{habitual}&\textsc{potential}\\
\midrule
\textsc{1sg}&\emph{ykon}\expo{1} \emph{jyanq}\expo{3}&\emph{ntykon}\expo{1} \emph{jyanq}\expo{3}&\emph{ntykon}\expo{20} \emph{jyanq}\expo{3}&\emph{kon}\expo{20} \emph{jyanq}\expo{3}\\
\textsc{2sg}&\emph{yku}\expo{4} \emph{jyaq}\expo{1}&\emph{ntyku}\expo{32} \emph{jyaq}\expo{1}&\emph{ntyku}\expo{4} \emph{jyaq}\expo{1}&\emph{ku}\expo{4} \emph{jyaq}\expo{1}\\
\textsc{3sg}&\emph{yku}\expo{4} \emph{jyaq}\expo{3}&\emph{ntyku}\expo{32} \emph{jyaq}\expo{3}&\emph{ntyku}\expo{4} \emph{jyaq}\expo{3}&\emph{ku}\expo{4} \emph{jyaq}\expo{3}\\
\textsc{1incl}&\emph{yku}\expo{4} \emph{jyan}\expo{2} \emph{anq}\expo{1}&\emph{ntyku}\expo{32} \emph{jyanq}\expo{2} \emph{an}\expo{1}&\emph{ntyku}\expo{4} \emph{jyanq}\expo{2} \emph{an}\expo{1}&\emph{ku}\expo{4} \emph{jyanq}\expo{2} \emph{an}\expo{1}\\
\textsc{1excl}&\emph{yku}\expo{4} \emph{jyaq}\expo{3} \emph{wa}\expo{42}&\emph{ntyku}\expo{32} \emph{jyaq}\expo{3} \emph{wa}\expo{42}&\emph{ntyku}\expo{4} \emph{jyaq}\expo{3} \emph{wa}\expo{42}&\emph{ku}\expo{4} \emph{jyaq}\expo{3} \emph{wa}\expo{42}\\
\textsc{2pl}&\emph{yku}\expo{4} \emph{jyaq}\expo{3} \emph{wan}\expo{24}&\emph{ntyku}\expo{32} \emph{jyaq}\expo{3} \emph{wan}\expo{24}&\emph{ntyku}\expo{4} \emph{jyaq}\expo{3} \emph{wan}\expo{24}&\emph{ku}\expo{4} \emph{jyaq}\expo{3} \emph{wan}\expo{24}\\
\textsc{3pl}&\emph{yku}\expo{4} \emph{jyaq}\expo{3} \emph{renq}\expo{24}&\emph{ntyku}\expo{32} \emph{jyaq}\expo{3} \emph{renq}\expo{24}&\emph{ntyku}\expo{4} \emph{jyaq}\expo{3} \emph{renq}\expo{24}&\emph{ku}\expo{4} \emph{jyaq}\expo{3} \emph{renq}\expo{24}\\
\lspbottomrule
\end{tabular}}

\caption{Paradigm of the compound predicate \emph{yku}\expo{4} \emph{jyaq}\expo{3} `s/he tasted' [eat amount] in SJQ Chatino}
\label{tab:CruzStump:sjq-12}
\end{table}



\begin{table}

	\resizebox{\textwidth}{!}{%
\begin{tabular}{lllll}
\lsptoprule
&\textsc{completive}&\textsc{progressive}&\textsc{habitual}&\textsc{potential}\\
\midrule
\textsc{1sg}&\emph{ykwiq}\expo{4} \emph{slan}\expo{40}&\emph{ntykwiq}\expo{32} \emph{slan}\expo{40}&\emph{ntykwiq}\expo{4} \emph{slan}\expo{40}&\emph{tykwiq}\expo{4} \emph{slan}\expo{40}\\
\textsc{2sg}&\emph{ykwiq}\expo{4} \emph{sla}\expo{1}&\emph{ntykwiq}\expo{32} \emph{sla}\expo{1}&\emph{ntykwiq}\expo{4} \emph{sla}\expo{1}&\emph{tykwiq}\expo{4} \emph{sla}\expo{1}\\
\textsc{3sg}&\emph{ykwiq}\expo{4} \emph{sla}\expo{3}&\emph{ntykwiq}\expo{32} \emph{sla}\expo{3}&\emph{ntykwiq}\expo{4} \emph{sla}\expo{3}&\emph{tykwiq}\expo{4} \emph{sla}\expo{3}\\
\textsc{1incl}&\emph{ykwiq}\expo{4} \emph{slan}\expo{2} \emph{an}\expo{1}&\emph{ntykwiq}\expo{32} \emph{slan}\expo{2} \emph{an}\expo{1}&\emph{ntykwiq}\expo{4} \emph{slan}\expo{2} \emph{an}\expo{1}&\emph{tykwiq}\expo{4} \emph{slan}\expo{2} \emph{an}\expo{1}\\
\textsc{1excl}&\emph{ykwiq}\expo{4} \emph{sla}\expo{3} \emph{wa}\expo{42}&\emph{ntykwiq}\expo{32} \emph{sla}\expo{3} \emph{wa}\expo{42}&\emph{ntykwiq}\expo{4} \emph{sla}\expo{3} \emph{wa}\expo{42}&\emph{tykwiq}\expo{4} \emph{sla}\expo{3} \emph{wa}\expo{42}\\
\textsc{2pl}&\emph{ykwiq}\expo{4} \emph{sla}\expo{3} \emph{wan}\expo{14}&\emph{ntykwiq}\expo{32} \emph{sla}\expo{3} \emph{wan}\expo{14}&\emph{ntykwiq}\expo{4} \emph{sla}\expo{3} \emph{wan}\expo{14}&\emph{tykwiq}\expo{4} \emph{sla}\expo{3} \emph{wan}\expo{14}\\
\textsc{3pl}&\emph{ykwiq}\expo{4} \emph{sla}\expo{3} \emph{renq}\expo{24}&\emph{ntykwiq}\expo{32} \emph{sla}\expo{3} \emph{renq}\expo{24}&\emph{ntykwiq}\expo{4} \emph{sla}\expo{3} \emph{renq}\expo{24}&\emph{tykwiq}\expo{4} \emph{sla}\expo{3} \emph{renq}\expo{24}\\
\lspbottomrule
\end{tabular}}
\caption{Paradigm of the compound verb \emph{ykwiq}\expo{4}  \emph{sla}\expo{3} [speak tiredness] `s/he dreamed' in SJQ Chatino}

\label{tab:CruzStump:sjq-13}
\end{table}

As \cite{Rasch02} and \cite{CruzWoodbury2013} observe, compound verbs in Chatino are quite varied in their structure, consisting of a verb paired with a stem of any of a range of categories to form either a head-complement structure (as in (\ref{ex:CruzStump:10}a)) or a head-modifier structure (as in (\ref{ex:CruzStump:10}b)), but not, in general, to form a verb-subject structure.\footnote{Despite initial resemblances, a compound verb such as \emph{ykwiq}\expo{4}  \emph{sla}\expo{3} `s/he dreamed' cannot be seen as the phrasal combination of a verb with an independent postverbal constituent.   As a VSO language, Chatino ordinarily positions a verb's subject between the verb and a following complement or modifier, as in (i); but a compound verb is followed by its subject, as in (ii).  Moreover, the \is{noun!nominal component}nominal component of a compound verb carries the verb's person/number inflection, as in (iii), but a verb's object does not, as (iv) shows.

\ea
\label{ex:CruzStump:i}\gll {{Ykwiq}\expo{4}} {{no}\expo{4} } { {qan}\expo{1} } {{kwa}\expo{3}} {{ska}\expo{4} } {{poema}\expo{24}.}\\
	 {speak.\textsc{cpl}} {one } { female} {that} {one} {poem}\\
	\glt {`That woman spoke a poem.'}\\


\ex \label{ex:CruzStump:ii}\gll {{Ykwiq}\expo{4} } {{sla}\expo{3} } {{no}\expo{4} } { {qan}\expo{1} } {{kwa}\expo{3}.}\\
	 {speak.\textsc{cpl}} {tiredness} {one } { female} {that}\\
	\glt {`That woman dreamt.'}\\


\ex  \label{ex:CruzStump:iii}\gll {{Ykwiq}\expo{4} } {{slan}\expo{40}} {{nka}\expo{3}.}\\
	 {speak.\textsc{cpl}} {tiredness.\textsc{1sg}} {yesterday}\\
	\glt {`I dreamed yesterday.'}\\


\ex \label{ex:CruzStump:iv}\gll  {{Ykwenq}\expo{1} } {{chaq}\expo{3}-{xlya}\expo{10} } {{nka}\expo{3}}\\
	 {speak.\textsc{cpl.1sg}} {word-Castilian} {yesterday.}\\
	\glt {`I spoke Spanish yesterday.'}
\z}

\begin{exe}
\ex\label{ex:CruzStump:10}
\begin{xlist}
	\ex {{nchu}\expo{1} {yaq}\expo{2}
	\glt  `s/he clapped' [hit hand]}
  \ex {{yku}\expo{4} {na}\expo{2}
	\glt `s/he ate in secret' [eat hidden]}
	\end{xlist}
\end{exe}



Whether as a verb-complement structure or a verb-modifier structure, the compound verb tends to conform to the \isi{Compound Inflection Criterion}. This similarity between an essence predicate such as \emph{ndi}\expo{4} \emph{riq}\expo{2} `s/he was thirsty' and a compound verb such as \emph{yku}\expo{4} \emph{jyaq}\expo{3} `s/he tasted' raises the possibility that essence predicates are in fact simply a subclass of compound predicates.  If this is so, then an essence predicate's \is{noun!nominal component}nominal component does not obviously function as an argument of its \isi{predicative base}.  Instead, it seems to serve as a quasi-adverbial modifier:  \emph{ndi}\expo{4} \emph{renq}\expo{20} `I was thirsty inside'.  On this analysis, the person/number marking on an essence predicate's \is{noun!nominal component}nominal component is not an expression of possession, but (as in the compound verb \emph{yku}\expo{4} \emph{jyaq}\expo{3} `s/he tasted') an ordinary expression of subject agreement.
\is{compounding!compound verb|)}

In the following section, we assess the relative adequacy of the possessed-subject and compound predicate hypotheses in light of four kinds of evidence. 


\section{Assessing the possessed-subject and compound predicate hypotheses}

We now consider four important characteristics of essence predicates in SJQ Chatino:  their structural variety, their external syntax, their general lack of semantic \isi{compositionality}, and their relation to the distributional flexibility of subject-\isi{agreement} marking.  As we show, this evidence reveals that neither the \isi{possessed-subject hypothesis} nor the \isi{compound predicate hypothesis} accounts for the full range of characteristics exhibited by essence predicates.

\subsection{Structural variety}

Essence predicates vary in their structure in at least three ways. First, there is
variation with respect to the identity of the \is{noun!nominal component}nominal component, which we have so far exemplified mainly
with \emph{riq}\expo{2} `essence'. Second, there is variation with respect to the possibility of employing more than one \is{noun!nominal component}nominal component within the same essence predicate.    And third, essence predicates vary with respect to their \isi{predicative base}\textemdash specifically, with respect to whether the \isi{predicative base} has independent uses apart from its use in an essence predicate.  Consider each of these areas of variation. 

\subsubsection{Choice of \is{noun!nominal component}nominal component}

The examples of essence predicates cited so far have nearly all had the noun \emph{riq}\expo{2} `essence' as their \isi{nominal
component}. This is, indeed, the most usual \is{noun!nominal component}nominal component for essence predicates. There is,
however, a sizeable class of essence predicates whose \is{noun!nominal component}nominal component is instead \emph{tye}\expo{32} `chest';
one such predicate is \emph{nqne}\expo{42} \emph{tye}\expo{32} `s/he dared', whose paradigm is given in \tabref{tab:CruzStump:sjq-14}. Still
another class of essence predicates has the \is{noun!nominal component}nominal component \emph{qin}\expo{4} (whose low \isi{tone} makes it
frequently susceptible to \is{tone!tone sandhi}tone sandhi; an example is the predicate \emph{skeq}\expo{1} \emph{qin}\expo{24} `he (wrongly) thought or believed' [imagine essence] in
\tabref{tab:CruzStump:sjq-15}.



\begin{table}


\resizebox{\textwidth}{!}{%
\begin{tabular}{lllll}
\lsptoprule
&\textsc{completive}&\textsc{progressive}&\textsc{habitual}&\textsc{potential}\\
\midrule
\textsc{1sg}&\emph{qne}\expo{42} \emph{tyin}\expo{20}&\emph{nqne}\expo{32} \emph{tyin}\expo{20}&\emph{nqne}\expo{24} \emph{tyin}\expo{20}&\emph{qne}\expo{24} \emph{tyin}\expo{20}\\

\textsc{2sg}&\emph{qne}\expo{42} \emph{tye}\expo{32}&\emph{nqne}\expo{32} \emph{tye}\expo{32}&\emph{nqne}\expo{24} \emph{tye}\expo{32}&\emph{qne}\expo{24} \emph{tye}\expo{32}\\

\textsc{3sg}&\emph{qne}\expo{42} \emph{tye}\expo{32}&\emph{nqne}\expo{32} \emph{tye}\expo{32}&\emph{nqne}\expo{24} \emph{tye}\expo{32}&\emph{qne}\expo{24} \emph{tye}\expo{32}\\

\textsc{1incl}&\emph{qne}\expo{42} \emph{tyin}\expo{1}\emph{in}\expo{1}&\emph{nqne}\expo{32} \emph{tyin}\expo{1} \emph{in}\expo{1}&\emph{nqne}\expo{24} \emph{tyin}\expo{1}\emph{in}\expo{1}&\emph{qne}\expo{24} \emph{tyin}\expo{1}\emph{in}\expo{1}\\

\textsc{1excl}&\emph{qne}\expo{42} \emph{tye}\expo{32} \emph{wa}\expo{42}&\emph{nqne}\expo{32} \emph{tye}\expo{32} \emph{wa}\expo{42}&\emph{nqne}\expo{24} \emph{tye}\expo{32} \emph{wa}\expo{42}&\emph{qne}\expo{24} \emph{tye}\expo{32} \emph{wa}\expo{42}\\

\textsc{2pl}&\emph{qne}\expo{42} \emph{tye}\expo{32} \emph{wan}\expo{4}&\emph{nqne}\expo{32} \emph{tye}\expo{32} \emph{wan}\expo{4}&\emph{nqne}\expo{24} \emph{tye}\expo{32} \emph{wan}\expo{4}&\emph{qne}\expo{24} \emph{tye}\expo{32} \emph{wan}\expo{4}\\

\textsc{3pl}&\emph{qne}\expo{42} \emph{tye}\expo{32} \emph{renq}\expo{4}&\emph{nqne}\expo{32} \emph{tye}\expo{32} \emph{renq}\expo{4}&\emph{nqne}\expo{24} \emph{tye}\expo{32} \emph{renq}\expo{4}&\emph{qne}\expo{24} \emph{tye}\expo{32} \emph{renq}\expo{4}\\
\lspbottomrule
\end{tabular}}

\caption{Paradigm of the essence predicate \emph{nqne}\expo{42} \emph{tye}\expo{32} `s/he dared' [do chest] in SJQ Chatino }
\label{tab:CruzStump:sjq-14}
\end{table}

\begin{table}


\resizebox{\textwidth}{!}{%
\begin{tabular}{l l ll l  }
\lsptoprule
&\textsc{completive} & \textsc{progressive} & \textsc{habitual} & \textsc{potential} \\
\midrule
\textsc{1sg} & \emph{skeq}\expo{1} \emph{qnya}\expo{24} & \emph{nskeq}\expo{1} \emph{qnya}\expo{24} & \emph{nxkeq}\expo{1} \emph{qnya}\expo{24} & \emph{xkeq}\expo{1 } \emph{qnya}\expo{24} \\
\textsc{2sg} & \emph{skeq}\expo{1} \emph{qin}\expo{42} & \emph{nskeq}\expo{1} \emph{qnya}\expo{24} & \emph{nxkeq}\expo{1} \emph{qnya}\expo{24} & \emph{xkeq}\expo{1 } \emph{qnya}\expo{24} \\
\textsc{3sg} & \emph{skeq}\expo{1} \emph{qin}\expo{24} & \emph{nskeq}\expo{1} \emph{qin}\expo{24} & \emph{nxkeq}\expo{1} \emph{qin}\expo{24} & \emph{xkeq}\expo{1} \emph{qin}\expo{24} \\
\textsc{1incl} & \emph{skeq}\expo{1} \emph{qin}\expo{24} & \emph{nskeq}\expo{1} \emph{qin}\expo{24} & \emph{nxkeq}\expo{1} \emph{qin}\expo{24} & \emph{xkeq}\expo{1} \emph{qin}\expo{24} \\
\textsc{1excl} & \emph{skeq}\expo{1} \emph{qwa}\expo{42} & \emph{nskeq}\expo{1} \emph{qwa}\expo{42} & \emph{nxkeq}\expo{1} \emph{qwa}\expo{42} & \emph{xkeq}\expo{1 } \emph{qwa}\expo{42} \\
\textsc{2pl} & \emph{skeq}\expo{1} \emph{qwan}\expo{14} & \emph{nskeq}\expo{1} \emph{qwan}\expo{14} & \emph{nxkeq}\expo{1} \emph{qwan}\expo{14} & \emph{xkeq}\expo{1} \emph{qwan}\expo{14} \\
\textsc{3pl} & \emph{skeq}\expo{1} \emph{qin}\expo{24} \emph{renq}\expo{24} & \emph{nskeq}\expo{1} \emph{qin}\expo{24} \emph{renq}\expo{24} & \emph{nxkeq}\expo{1} \emph{qin}\expo{24} \emph{renq}\expo{24} & \emph{xkeq}\expo{1} \emph{qin}\expo{24} \emph{renq}\expo{24} \\
\lspbottomrule
\end{tabular}}

\caption{Paradigm of \emph{skeq}\expo{1} \emph{qin}\expo{24} `s/he (wrongly) thought or believed' [imagine essence] in SJQ Chatino}
\label{tab:CruzStump:sjq-15}
\end{table}

The identity of \emph{qin}\expo{4} in \emph{skeq}\expo{1} \emph{qin}\expo{24} `s/he wrongly thought or believed' is debatable, since \emph{qin}\expo{4} has a variety
of functions in Chatino; for example, \emph{qin}\expo{4} functions (with \is{tone!tone  sandhi}tone  sandhi) as a third-person singular pronoun in (\ref{ex:CruzStump:11a}), but arguably as an animal classifier in (\ref{ex:CruzStump:11b}).

\begin{exe}

	\ex \label{ex:CruzStump:11}
			\ea \label{ex:CruzStump:11a}\gll {{Ye}\expo{42} } {{qa}\expo{24} } {{yku}\expo{24} } {{tykwen}\expo{1}} {{qin}\expo{24} } {{sen}\expo{32}.}\\
	 				{ {very}} {\textsc{emph}} {eat.\textsc{cpl}} {bedbug} {\textsc{obj.pron:3sg}} {last.night}\\
					\glt {`Bedbugs bit her last night.'}\\
	 		\ex \label{ex:CruzStump:11b}\gll {{Yla}\expo{42}} {{qin}\expo{4}} {{qo}\expo{1}} {{snyiq}\expo{24}} {{qin}\expo{24}.}\\
	 				{ { arrive.\textsc{cpl}}} {\textsc{animal}.\textsc{clf}} {with} {offspring} {\textsc{obj.pron:3sg}}\\
					\glt {`The (animal) returned home with his offspring.'}\\
	\z
\end{exe}

Although \emph{riq}\expo{2}, \emph{tye}\expo{32} and \emph{qin}\expo{4} are not freely interchangeable as the \is{noun!nominal component}nominal component of an essence predicate, they do exhibit a partial overlap in their
distribution; in cases of overlap, the choice of \is{noun!nominal component}nominal component may or may not serve to express a difference in meaning. The forms in (\ref{ex:CruzStump:12}) constitute a minimal triplet in which the
\isi{predicative base} \emph{sqwe}\expo{3} `good' combines with \emph{riq}\expo{2} (`essence'), \emph{tye}\expo{32} (`chest'), or \emph{qin}\expo{4} (`his or her essence'), with each
combination expressing a different meaning.

\begin{exe}
	\ex \label{ex:CruzStump:12}
		\ea\label{ex:CruzStump:12a} {sqwe}\expo{3} {riq}\expo{2}\\
		\glt `s/he was in a good mood'\\
		\ex\label{ex:CruzStump:12b} {sqwe}\expo{3} {tye}\expo{32}\\
		\glt `s/he was generous'\\
		\ex\label{ex:CruzStump:12c} {sqwe}\expo{3} {qin}\expo{24}\\
		\glt `s/he was affable'\\
		\z
\end{exe}

Several cases in which \emph{riq}\expo{2}, \emph{tye}\expo{32} and \emph{qin}\expo{4} may be used more or less interchangeably are listed in
\tabref{tab:CruzStump:sjq-16}a. The essence predicates in \tabref{tab:CruzStump:sjq-16}b involve \emph{riq}\expo{2} and \emph{tye}\expo{32} but have no alternative with
\emph{qin}\expo{4}; conversely, those in \tabref{tab:CruzStump:sjq-16}c involve \emph{riq}\expo{2} and \emph{qin}\expo{4} and have no alternative with \emph{tye}\expo{32}. Those
in \tabref{tab:CruzStump:sjq-16}d involve \emph{riq}\expo{2} but not \emph{tye}\expo{32} or \emph{qin}\expo{4}; those in \tabref{tab:CruzStump:sjq-16}e involve \emph{tye}\expo{32} but not \emph{riq}\expo{2} or
\emph{qin}\expo{4}; and those in \tabref{tab:CruzStump:sjq-16}f involve \emph{qin}\expo{4} but not \emph{riq}\expo{2} or \emph{tye}\expo{32}.\footnote{It might appear that in \tabref{tab:CruzStump:sjq-16}d, \emph{tqi}\expo{4} \emph{riq}\expo{2} `s/he hates' has a counterpart with \emph{tye}\expo{32}, but \emph{tqi}\expo{4} \emph{tye}\expo{32} only has the literal meaning `her/his chest hurts', not that of an essence predicate.}

\begin{table}[t]
\footnotesize
\begin{tabularx}{\textwidth}{l@{~} Q l@{}l@{}l l@{}}
\lsptoprule
&&Based on \emph{riq}\expo{2}&Based on \emph{tye}\expo{32}&Based on \emph{qin}\expo{4}& \\
\midrule
a.&`s/he understood'&\emph{nkwa}\expo{2} \emph{jyaq}\expo{3} \emph{riq}\expo{2} &\emph{nkwa}\expo{2} \emph{jyaq}\expo{3} \emph{tye }\expo{24}&\emph{[nkwa}\expo{2} \emph{jyaq}\expo{3}\emph{qin}\expo{24} &[realized \textsc{ess}] \\
&&&&`s/he was tried']& \\
&`s/he is na{\"i}ve'&\emph{ntu}\expo{1} \emph{riq}\expo{0}&\emph{ntu}\expo{1} \emph{tye}\expo{0}&\emph{ntu}\expo{1} \emph{qin}\expo{0}&[stupid \textsc{ess}] \\
&`s/he is getting angry'&\emph{ntykwen}\expo{3} \emph{riq}\expo{24}&\emph{ntykwen}\expo{3} \emph{tye}\expo{24}&[\emph{ntykwen}\expo{3} \emph{qin}\expo{24}&[choke \textsc{ess}] \\
&&&&`s/he choked on sth']& \\
&`s/he misperceives' &\emph{skeq}\expo{1} \emph{riq}\expo{0} &\emph{skeq}\expo{1} \emph{tye}\expo{32} &\emph{skeq}\expo{1} \emph{qin}\expo{24}&[imagine \textsc{ess}] \\
%&`s/he thought erroneously'&\emph{skeq}\expo{1} \emph{riq}\expo{4}&\emph{skeq}\expo{1} \emph{tye}\expo{4}&\emph{skeq}\expo{1} \emph{qin}\expo{4}&[imagine \textsc{ess}] \\
&`s/he is happy'&\emph{stu}\expo{1} \emph{riq}\expo{1} &\emph{stu}\expo{1} \emph{tye}\expo{32} &\emph{stu}\expo{1} \emph{qin}\expo{0}&[gusto \textsc{ess}] \\
&`s/he is fuzzy'&\emph{swaq}\expo{24} \emph{riq }\expo{32}&\emph{swaq}\expo{24} \emph{tye}\expo{32}&\emph{swaq}\expo{24} \emph{qin}\expo{32}&[slovenly \textsc{ess}] \\
&`s/he is cool (not hot)'&\emph{tlaq}\expo{14} \emph{riq}\expo{0}&\emph{tlaq}\expo{14} \emph{tye}\expo{32}&\emph{[tlaq}\expo{14} \emph{qin}\expo{0}&[cool \textsc{ess}] \\
&&&& `s/he is in peace']& \\
&`s/he is hot'&\emph{tykeq}\expo{14} \emph{riq}\expo{0}&[\emph{tykeq}\expo{14} \emph{tye}\expo{0}&\emph{tykeq}\expo{14} \emph{qin}\expo{0}&[hot \textsc{ess}] \\
&&&`s/he is angry']&& \\
b.&`s/he pities' &\emph{qna}\expo{3} \emph{riq}\expo{2} &\emph{qna}\expo{3} \emph{tye}\expo{32} &&[pity \textsc{ess}] \\
&`s/he remembers'&\emph{sqwi}\expo{4} \emph{riq}\expo{2} &[\emph{sqwi}\expo{4} \emph{tye}\expo{32} &&[exist \textsc{ess}] \\
&&&`s/he holds a grudge']&& \\
&`s/he is greedy'&\emph{tkonq}\expo{1} \emph{riq}\expo{2} &\emph{tkonq}\expo{1} \emph{tye}\expo{32} && [greedy \textsc{ess}] \\
&`s/he is sad' &\emph{xkuq}\expo{42} \emph{riq}\expo{2} &\emph{xkuq}\expo{42} \emph{tye}\expo{32} &&[\textsc{cbm} \textsc{ess}] \\
c.&`s/he is fast or in a hurry'&\emph{sa}\expo{4} \emph{riq}\expo{2}, \emph{ndla}\expo{2} \emph{riq}\expo{2}&&\emph{ndla}\expo{2}\emph{qin}\expo{0}&[hurry \textsc{ess}] \\
&`s/he is satisfied/ satiated'&\emph{ylaq}\expo{42} \emph{riq}\expo{2}&&\emph{sa}\expo{4} \emph{qin}\expo{4}, \emph{ylaq}\expo{42} \emph{qin}\expo{4}&[satiated \textsc{ess}] \\
d.&`s/he hates'&\emph{tqi}\expo{4} \emph{riq}\expo{2} &&&[sick \textsc{ess}]  \\
&`s/he knows/is aware' &\emph{jlyo}\expo{20} \emph{riq}\expo{2} &&&[\textsc{cbm} \textsc{ess}]  \\
&`s/he is worry' &\emph{ndwe}\expo{32} \emph{riq}\expo{2} &&&[minced \textsc{ess}]  \\
&`s/he remembers'&\emph{nkya}\expo{42} \emph{yqwi}\expo{32} \emph{riq}\expo{1} &&&[remember \textsc{ess}]  \\
&`s/he is disgusted' &\emph{stya}\expo{4} \emph{riq}\expo{2} &&&[place \textsc{ess}]  \\
&`s/he is ecstatic'&\emph{styi}\expo{1} \emph{riq}\expo{2} &&&[laugh \textsc{ess}]  \\
e.&`s/he is scared/ queasy' &&\emph{chin}\expo{4} \emph{nga}\expo{24} \emph{tye}\expo{32} &&[ugly \textsc{ess}]  \\
&`s/he likes' &&\emph{ndya}\expo{24} \emph{riq}\expo{2} \emph{tye}\expo{32}&&[like \textsc{ess}]  \\
&`s/he feels brave' &&\emph{tno}\expo{4} \emph{nga}\expo{24} \emph{tye}\expo{32} &&[big \textsc{ess}]  \\
&`s/he feels angry' &&\emph{xqan}\expo{10} \emph{nga}\expo{24} \emph{tye}\expo{32} &&[mean \textsc{ess}]  \\
&`s/he feels sad' &&\emph{tqwa}\expo{14} \emph{nka}\expo{24} \emph{tye}\expo{32} &&[cool \textsc{ess}]  \\
f.&`s/he is affable' &&&\emph{sqwe}\expo{3} \emph{qin}\expo{24}&[good \textsc{ess}]  \\
&`s/he is a thief or fast &&'&\emph{sa}\expo{4} \emph{qin}\expo{4}&[light \textsc{ess}]  \\
\lspbottomrule
\end{tabularx}
 \caption{Some essence predicates in SJQ Chatino}
\parbox{\textwidth}{\footnotesize In the three central columns, bracketed essence predicates have a meaning different from that of the corresponding essence predicate with  \emph{riq}\expo{2}. Note that \is{tone!tone sandhi}tone sandhi alters the expected tonality of third-person singular \emph{riq}\expo{2} in some of these forms.}
\label{tab:CruzStump:sjq-16}
\end{table}

Even where the choice of \is{noun!nominal component}nominal component corresponds to a difference of meaning, it is not clear that the nature of this difference is predictable.
For example, the general sense of pity may be expressed by an essence predicate consisting of \emph{qna}\expo{3} and
either \emph{riq}\expo{2} or \emph{tye}\expo{32}, and the nuanced difference expressed by this choice in (\ref{ex:CruzStump:13}) is not obviously predictable from the semantic difference between \emph{riq}\expo{2} `essence' and \emph{tye}\expo{32} `chest'.  Note, by way of contrast, that the meaning of disgust expressed by the essence predicate \emph{stya}\expo{4} \emph{riq}\expo{2} has no counterpart with \emph{tye}\expo{32}: *\emph{stya}\expo{4} \emph{tye}\expo{32}. Moreover, the
meaning `s/he is sad' may be expressed by an essence predicate with either \emph{riq}\expo{2} or \emph{tye}\expo{32} (as
either \emph{xkuq}\expo{42} \emph{riq}\expo{2} or \emph{xkuq}\expo{42} \emph{tye}\expo{32}), but the meaning `s/he feels sad' is expressed by an essence predicate requiring
\emph{tye}\expo{32} rather than \emph{riq}\expo{2} (as \emph{tqwa}\expo{14} \emph{nka}\expo{24} \emph{tye}\expo{32} but not *\emph{tqwa}\expo{14} \emph{nka}\expo{24} \emph{riq}\expo{2}). 

\begin{exe}
	\ex\label{ex:CruzStump:13}
		\ea \gll {{Qna}\expo{3}} {{qa}\expo{24} } {{riq}\expo{2} } {{La}\expo{20}{ya}\expo{24} } {{kwa}\expo{3} } {{xneq}\expo{2} } {{qin}\expo{2}. } {}\\
	 { {pity}} {very} {essence} {Hilaria} {that} {dog} {\textsc{poss.3sg} } {}\\
	\glt {`Hilaria feels bad for her dog.'}\\

	 \ex \gll{{Qna}\expo{3}} {{qa}\expo{24} } {{tye}\expo{32} } {{La}\expo{20}{ya}\expo{24} } {{kwa}\expo{3}, } {{nkjwi}\expo{42} } {{xneq}\expo{2} } {{qin}\expo{1}.}\\
	  { {pity}} {very} {essence} {Hilaria} {that} {die.\textsc{cpl}} {dog} {\textsc{poss.3sg}}\\
	\glt {`Hilaria is pitiable, her dog died.'}\\
	\z
\end{exe}


These facts suggest that choices among the nominal components\is{noun!nominal component} \emph{riq}\expo{2}, \emph{tye}\expo{32} and \emph{qin}\expo{4} in essence predicates are often (perhaps always) determined by lexical stipulation.

  
\subsubsection{Combinability of nominal components\is{noun!nominal component}} 

It is often possible to use \emph{riq}\expo{2} and \emph{tye}\expo{32} in tandem, as in \tabref{tab:CruzStump:sjq-17}.\footnote{In \tabref{tab:CruzStump:sjq-17} and some later tables, `\#' marks forms that we have not encountered and that aren't
clearly acceptable, but whose acceptability to at least some speakers we do not wish to rule out.} In such cases, it is \emph{tye}\expo{32} rather
than \emph{riq}\expo{2} that exhibits the person-number \isi{agreement}; for instance, the first-person singular completive
form of \emph{njlya}\expo{32} \emph{riq}\expo{2} \emph{tye}\expo{32} `s/he forgot' is \emph{njlya}\expo{32} \emph{riq}\expo{2} \emph{tyin}\expo{20} `I forgot'. It is not clear that \emph{qin}\expo{4}
appears in tandem with either \emph{riq}\expo{2} and \emph{tye}\expo{32} in its function as the \is{noun!nominal component}nominal component of an
essence predicate; in those cases in which it might appear to do so, it instead serves one of its other functions, e.g. that of
an animal classifier (as in \emph{tkonq}\expo{1} \emph{riq}\expo{2} \emph{tye}\expo{32} \emph{qin}\expo{24} `that animal is gluttonous').


\begin{table}
\footnotesize
\begin{tabularx}{\textwidth}{QlQl}
\lsptoprule
Gloss&\emph{riq}\expo{2} + \emph{tye}\expo{32}&Gloss&\emph{riq}\expo{2} + \emph{tye}\expo{32} \\
\midrule
`s/he forgives'&\emph{chaq}\expo{3} \emph{tlyu}\expo{2} \emph{riq}\expo{2} \emph{tye}\expo{32}&`s/he likes sth'&\emph{snyi}\expo{4} \emph{riq}\expo{2} \emph{tye}\expo{32} \\
`s/he forgets'&\emph{jlya}\expo{32} \emph{riq}\expo{2} \emph{tye}\expo{32}&`s/he is generous/happy'&\emph{sqwe}\expo{3} \emph{riq}\expo{2} \emph{tye}\expo{32}  \\
`s/he knows/is aware'&\emph{jlyo}\expo{20} \emph{riq}\expo{2} \emph{tye}\expo{32}&`s/he remembers'&\emph{sqwi}\expo{4} \emph{riq}\expo{2} \emph{tye}\expo{32} \\
`s/he mischievous, playful'&\emph{jnya}\expo{20} \emph{riq}\expo{2} \emph{tye}\expo{32}&`s/he is strong/sturdy'&\emph{sqye}\expo{14} \emph{riq}\expo{0} \emph{tye}\expo{32} \\
`s/he worries'&\emph{ndwe}\expo{4} \emph{riq}\expo{2} \emph{tye}\expo{32}&`s/he is happy'&\emph{stu}\expo{1} \emph{riq}\expo{0} \emph{tye}\expo{32}  \\
`s/he is open, extroverted'&\emph{la}\expo{1} \emph{riq}\expo{2} \emph{tye}\expo{32}&`s/he is disgusted'&\emph{stya}\expo{4} \emph{riq}\expo{2} \emph{tye}\expo{32} \\
`s/he is taciturn'&\emph{liqa}\expo{24} \emph{riq}\expo{1} \emph{tye}\expo{32}&`s/he is ecstatic'&\emph{styi}\expo{1} \emph{riq}\expo{2} \emph{tye}\expo{32} \\
`s/he is fair-skinned'&\emph{lwi}\expo{3} \emph{riq}\expo{2} \emph{tye}\expo{32}&`s/he is hard-working'&\emph{t(j)nya}\expo{3} \emph{riq}\expo{2} \emph{tye}\expo{32} \\
`s/he mocks'&\emph{lyeq}\expo{3} \emph{riq}\expo{2} \emph{tye}\expo{32} &`s/he is frugal/takes care of sth' & \emph{tjenq}\expo{3} \emph{riq}\expo{2} \emph{tye}\expo{32} \\
&&&\\
`s/he realizes'&\emph{ndi}\expo{20} \emph{riq}\expo{2} \emph{tye}\expo{32}&`s/he is sturdy'&\emph{tjoq}\expo{4} \emph{riq}\expo{2} \emph{tye}\expo{32} \\
`s/he is happy'&\emph{ndon}\expo{42} \emph{riq}\expo{2} \emph{tye}\expo{32}&`s/he is greedy'&\emph{tkonq}\expo{1} \emph{riq}\expo{2} \emph{tye}\expo{32}  \\
`s/he is worry'&\emph{ndwe}\expo{32} \emph{riq}\expo{2} \emph{tye}\expo{32}&`s/he placates, calms'&\emph{tlaq}\expo{14} \emph{riq}\expo{0} \emph{tye}\expo{32} \\
`s/he likes, loves'&\emph{ndya}\expo{24} \emph{riq}\expo{2} \emph{tye}\expo{32}&`s/he is cold'&\emph{tlyaq}\expo{4} \emph{riq}\expo{2} \emph{tye}\expo{32} \\
`s/he is thirsty, wheezing'&\emph{ndyi}\expo{32} \emph{riq}\expo{2} \emph{tye}\expo{32}&`s/he is tired'&\emph{tnyaq}\expo{4} \emph{riq}\expo{2} \emph{tye}\expo{32} \\
`s/he remembers'&\emph{nkqan}\expo{4} \emph{riq}\expo{2} \emph{tye}\expo{32}&`s/he is fully conscious'&\emph{tqa}\expo{24} \emph{riq}\expo{2} \emph{tye}\expo{32} \\
`s/he realizes'&\emph{nkwa}\expo{2} jyaq\expo{3} \emph{riq}\expo{2} \emph{tye}\expo{32} &`s/he scorns'&\emph{tqi}\expo{4} \emph{riq}\expo{2} \emph{tye}\expo{32} \\
`s/he remembers'&\emph{nkya}\expo{42} \emph{yqwi}\expo{32} \emph{riq}\expo{1} \emph{tye}\expo{32}&`s/he is flirtatious'&\emph{tsa}\expo{3} \emph{riq}\expo{2} \emph{tye}\expo{32} \\
`s/he is dark-skinned'&\emph{nta}\expo{14} \emph{riq}\expo{0} \emph{tye}\expo{32}&`s/he is smart'&\emph{tya}\expo{20} \emph{riq}\expo{2} \emph{tye}\expo{32} \\
`s/he is hungry'&\emph{nteq}\expo{32} \emph{riq}\expo{2} \emph{tye}\expo{32}&`s/he is slow'&\emph{tyaq}\expo{4} \emph{riq}\expo{2} \emph{tye}\expo{32} \\
`s/he is feeling lazy'&\emph{ntja}\expo{1} \emph{riq}\expo{2} \emph{tye}\expo{32}&`s/he is hot'&\emph{tykeq}\expo{14} \emph{riq}\expo{0} \emph{tye}\expo{32} \\
`s/he is weak'&\emph{ntqan}\expo{1} \emph{riq}\expo{1} \emph{tye}\expo{32}&`s/he made up her/his mind'&\emph{wa}\expo{2} \emph{xtya}\expo{20} \emph{riq}\expo{2} \emph{tye}\expo{32} \\
`s/he is stupid'&\emph{ntu}\expo{10} \emph{riq}\expo{1} \emph{tye}\expo{32} &`s/he is sad'& \emph{xkuq}\expo{42} \emph{riq}\expo{2} \emph{tye}\expo{32} \\
`s/he gets mad'&\emph{ntykwen}\expo{3} \emph{riq}\expo{2} \emph{tye}\expo{32}&`s/he is afraid'&\emph{xqnyi}\expo{4} \emph{riq}\expo{2} \emph{tye}\expo{32} \\
`s/he gets used to'&\emph{ntyqan}\expo{1} \emph{riq}\expo{2} \emph{tye}\expo{32}&`s/he is fed up with/tired of'&\emph{xyaq}\expo{2} \emph{riq}\expo{2} \emph{tye}\expo{32} \\
`s/he pities'&\emph{qna}\expo{3} \emph{riq}\expo{2} \emph{tye}\expo{32}&`s/he bullies'&\emph{xyuq}\expo{1} \emph{riq}\expo{2} \emph{tye}\expo{32} \\
`s/he is smart, fast, agile'&\emph{sa}\expo{4} \emph{riq}\expo{2} \emph{tye}\expo{32}&`s/he believes/is gullible'&\emph{ya}\expo{42} \emph{ntyqan}\expo{4} \emph{riq}\expo{1} \emph{qin}\expo{24} \emph{tye}\expo{32} \\
`s/he is upset'&\emph{senq}\expo{24} \emph{riq}\expo{1} \emph{tye}\expo{32}&`s/he is satisfied/satiated'&\emph{ylaq}\expo{42}/\emph{ndlaq}\expo{42} \emph{riq}\expo{2} \emph{tye}\expo{32} \\
`s/he is standoffish'&\emph{siyeq}\expo{3} \emph{riq}\expo{2} \emph{tye}\expo{32}&`s/he is shy'&\emph{yqu}\expo{20} \emph{riq}\expo{2} \emph{tye}\expo{32} \\
`s/he misperceives'&\emph{skeq}\expo{1} \emph{riq}\expo{2} \emph{tye}\expo{32}&`s/he breaks a bad habit/learns a lesson'&\emph{\#ksa}\expo{4} \emph{riq}\expo{2} \emph{tye}\expo{32} \\
`s/he is fed up'&\emph{skwa}\expo{3} \emph{riq}\expo{2} \emph{tye}\expo{32}&`s/he is skinny'&\emph{\#ti}\expo{4} \emph{riq}\expo{2} \emph{tye}\expo{32} \\
`s/he takes a liking to'&\emph{skwi}\expo{1} \emph{riq}\expo{2} \emph{tye}\expo{32}&`s/he is skinny'&\emph{\#tyjyan}\expo{20} \emph{riq}\expo{2} \emph{tye}\expo{32} \\
`s/he is desirous'&\emph{snya}\expo{1} \emph{riq}\expo{2} \emph{tye}\expo{32} \\
\lspbottomrule
\end{tabularx}
\caption{Instances of \emph{riq}\expo{2} used in tandem with \emph{tye}\expo{32} in SJQ Chatino}

\label{tab:CruzStump:sjq-17}
\end{table}




\subsubsection{Cranberry predicative bases}
\is{base!predicative base|(}
Essence predicates also vary with respect to the independence of their predicative base. On one
hand, there are essence predicates whose predicative base also appears independently (though
usually not with the same meaning as the essence predicate), as in (\ref{ex:CruzStump:14}). On the other hand, there are
instances whose predicative base does not have an independent use as a predicate, as in (\ref{ex:CruzStump:15})\textendash(\ref{ex:CruzStump:18});
such predicative bases are in effect cranberry morphemes\is{morpheme!cranberry morpheme}.

\begin{exe}
	\ex\label{ex:CruzStump:14}
	\begin{xlist}
		\ex[]{\gll {{Sqwe}\expo{3}} {{riq}\expo{2} / {tye}\expo{32} } {{no}\expo{4} } {{kyqyu}\expo{1}.}\\
	 { {good}} {essence / chest.\textsc{3sg}} {the} {male (ones)}\\
	\glt {`The men are in a good mood / generous.'}}
	 	\ex[]{\gll {{Sqwe}\expo{3} } {{no}\expo{4} } {{kyqyu}\expo{1}.} {}\\
	 { {good}} {the} {male (ones)} {}\\
	\glt {`The men are good.'}}
	\z
	\ex\label{ex:CruzStump:15}
	\begin{xlist}
		\ex[]{\gll {{Ndi}\expo{32} } {{riq}\expo{2} } {{Xwa}\expo{3}} {{ kwa}\expo{3}.}\\
	 { {thirsty.\textsc{prog}}} {essence.\textsc{3sg}} {Juan} {that}\\
	\glt `Juan is thirsty.'}
	 \ex[*]{\gll {{Ndi}\expo{32} } {{Xwa}\expo{3}} {{ kwa}\expo{3}.} {}\\
	 { {thirsty.\textsc{prog}}} {Juan} {that} {}\\
	\glt Sought interpretation: `Juan is thirsty.'}
	\z

	\ex\label{ex:CruzStump:16}
	\begin{xlist}
		\ex[]{\gll {{Ndi}\expo{32} } {{riq}\expo{2} } {{sti}\expo{1}} {{Xwa}\expo{3}} {{ kwa}\expo{3}.}\\
		 { {thirsty.\textsc{prog}}} {essence.\textsc{3sg}} {father.\textsc{3sg}} {Juan} {that}\\
		\glt `Juan's father is thirsty.'}
		 \ex[*]{\gll {{Ndi}\expo{32}} {{sti}\expo{4}} {{Xwa}\expo{3}} {{ kwa}\expo{3}.} \\
		 { {thirsty.\textsc{prog}}} {father.\textsc{3sg}} {Juan} {that} \\
		\glt Sought interpretation: `Juan's father is thirsty.'}
		\z


	\ex\label{ex:CruzStump:17}
	\begin{xlist}
		\ex[]{\gll {{Ndi}\expo{32} } {{renq}\expo{20}.}\\
	 { {thirsty.\textsc{prog}}} {essence.1sg}\\
	 \glt {`I am thirsty.'}}
	 \ex[*]{\gll {{Ndi}\expo{32}. } \\
	 { {thirsty.\textsc{prog}}} \\
	\glt Sought interpretation: `I am thirsty.'}
	\z


	\ex\label{ex:CruzStump:18}
	\begin{xlist}
		\ex[]{\gll {{Ndi}\expo{32}} {{riq}\expo{2}} {{sten}\expo{1}.}\\
	 { {thirsty.\textsc{prog}}} {essence.\textsc{3sg}} {father.\textsc{1sg}}\\
	\glt `My father is thirsty.'}
	 \ex[*]{\gll {{ndi}\expo{32}} {{sten}\expo{4}.} \\
	 { {thirsty.\textsc{prog}}} {father.\textsc{1sg}} \\
	\glt Sought interpretation: `My father is thirsty.'}
	\z
\end{exe}

\tabref{tab:CruzStump:sjq-18} lists some essence predicates whose predicative bases have independent uses, and \tabref{tab:CruzStump:sjq-19}, some whose predicative bases are cranberry morphemes\is{morpheme!cranberry morpheme}. As inspection of both tables reveals,
the meaning expressed by an essence predicate L usually cannot be equivalently expressed by using
L's predicative base by itself; either the predicative base of L differs in meaning from L (as in (\ref{ex:CruzStump:14}))
or it is simply unavailable for use as an independent predicate (as in (\ref{ex:CruzStump:15})\textendash(\ref{ex:CruzStump:18})).

\begin{table}
\small
\begin{tabularx}{\textwidth}{lQQlQ}
\lsptoprule
\multicolumn{2}{l}{\normalsize Essence predicate} &\multicolumn{1}{l}{\normalsize Gloss} &\multicolumn{2}{l}{\normalsize Independent use of predicative base} \\
\midrule
\emph{chaq}\expo{3} \emph{tlyu}\expo{2} \emph{riq}\expo{2} &`s/he forgives'&[strong essence]&\emph{chaq}\expo{3} \emph{tlyu}\expo{2} \emph{no}\expo{4} \emph{kyqyu}\expo{1} &`because the men are strong' \\
\emph{ksa}\expo{4} \emph{riq}\expo{2} &`s/he breaks a bad habit/learns a lesson'&[break essence]&\emph{ksa}\expo{4} \emph{yka}\expo{4}&`to break a piece of wood' \\
\emph{ndwe}\expo{4} \emph{riq}\expo{2} &`s/he worries'&[minced essence]&\emph{ndwe}\expo{4} \emph{no}\expo{4} \emph{kyqyu}\expo{1} &`The men will be minced.' \\
\emph{la}\expo{1} \emph{riq}\expo{2} &`s/he is open, extroverted'&[open essence]&\emph{la}\expo{1} \emph{no}\expo{4} \emph{kyqyu}\expo{1} &`The men got open.' \\
\emph{lwi}\expo{3} \emph{riq}\expo{2} &`s/he is fair-skinned'&[clean essence]&\emph{lwi}\expo{3} \emph{no}\expo{4} \emph{kyqyu}\expo{1} &`The men are clean.' \\
\emph{ndo}\expo{42} \emph{riq}\expo{2} &`s/he is happy'&[stand essence]&\emph{ndo}\expo{42} \emph{no}\expo{4} \emph{kyqyu}\expo{1} &`The men are standing.' \\
\emph{ndya}\expo{24} \emph{riq}\expo{2} &`s/he likes, loves'&[arrive essence]&\emph{ndya}\expo{24} \emph{no}\expo{4} \emph{kyqyu}\expo{1} &`The men arrived.' \\
\emph{nkqan}\expo{4} \emph{riq}\expo{2} &`s/he remembers'&[sit essence]&\emph{nkqan}\expo{4} \emph{no}\expo{4} \emph{kyqyu}\expo{1} &`The men are sitting.' \\
\emph{nta}\expo{14} \emph{riq}\expo{0} &`s/he is dark-skinned'&[black essence]&\emph{nta}\expo{14} \emph{no}\expo{0} \emph{kyqyu}\expo{1} &`Men are black.' \\
\emph{ntja}\expo{1} \emph{riq}\expo{2} &`s/he is feeling lazy'&[lazy essence]&\emph{ntja}\expo{1} \emph{no}\expo{4} \emph{kyqyu}\expo{1} &`Men are lazy.' \\
\emph{ntu}\expo{10} \emph{riq}\expo{0} &`s/he is stupid'&[stupid essence]&\emph{ntu}\expo{1} \emph{no}\expo{0} \emph{kyqyu}\expo{1}  &`The men are stupid or slow.' \\
\emph{ntykwen}\expo{3} \emph{riq}\expo{2} &`s/he gets mad'&[climb essence]&\emph{ntykwen}\expo{3} \emph{no}\expo{4} \emph{kyqyu}\expo{1} &`The men climbed up.' \\
\emph{qna}\expo{3} \emph{riq}\expo{2} &`s/he pities'&[poor essence]&\emph{qna}\expo{3} \emph{no}\expo{4} \emph{kyqyu}\expo{1} &`the poor men' \\
\emph{sa}\expo{4} \emph{riq}\expo{2} &`s/he is smart, fast, agile'&[light/fast essence]&\emph{sa}\expo{4} \emph{no}\expo{4} \emph{kyqyu}\expo{1} &`The men are light or fast.' \\
\emph{siyeq}\expo{3} \emph{riq}\expo{2}&`s/he is standoffish'&[vain essence]&\emph{siyeq}\expo{3} \emph{no}\expo{4} kyqyu\expo{1}&`Men are vain.' \\
\emph{skwa}\expo{3} \emph{riq}\expo{2} &`s/he is fed up'&[lie.elevated essence]&\emph{skwa}\expo{3} \emph{no}\expo{4} \emph{kyqyu}\expo{1} &`The men are lying elevated.' \\
\emph{snyi}\expo{4} \emph{riq}\expo{2} &`s/he likes sth'&[grab essence]&\emph{snyi}\expo{4} \emph{no}\expo{4} \emph{kyqyu}\expo{1} &`The men grabbed .' \\
\emph{sqwe}\expo{3} \emph{riq}\expo{2}/\emph{tye}\expo{32} &`s/he is generous/ happy'&[good essence]&\emph{sqwe}\expo{3} \emph{no}\expo{4} \emph{kyqyu}\expo{1} &`Men are good.' \\
\emph{sqwi}\expo{4} \emph{riq}\expo{2} &`s/he remembers'&[exist essence]&\emph{sqwi}\expo{4} \emph{no}\expo{4} \emph{kyqyu}\expo{1} &`The men exist.' \\
\emph{sqye}\expo{14} \emph{riq}\expo{0} &`s/he is strong/ sturdy'&[strong essence]&\emph{sqye}\expo{14} \emph{no}\expo{0} \emph{kyqyu}\expo{1} &`The men are strong.' \\
\emph{styi}\expo{1} \emph{riq}\expo{2} &`s/he is ecstatic'&[laugh essence]&\emph{styi}\expo{1} \emph{no}\expo{4} \emph{kyqyu}\expo{1} &`The men are laughing.' \\
\midrule
\end{tabularx}
\caption{Essence predicates whose predicative bases are also used independently in SJQ Chatino}
\label{tab:CruzStump:sjq-18}
\end{table}

\begin{table}[t]
\small
\begin{tabularx}{\textwidth}{lQQlQ}
\midrule
\emph{t(j)nya}\expo{3} \emph{riq}\expo{2} &`s/he is hard-working'&[work essence]&\emph{no}\expo{4} \emph{kyqyu}\expo{1} \emph{tnya}\expo{3}  &`the men who are authorities' \\
\emph{ti}\expo{4} \emph{riq}\expo{2} &`s/he is skinny'&[skinny essence]&\emph{ti}\expo{4} \emph{no}\expo{4} \emph{kyqyu}\expo{1} &`The men are skinny.' \\
\emph{tjenq}\expo{3} \emph{riq}\expo{2} &`s/he is frugal or to take care of sth'&[sticky essence]&\emph{tjenq}\expo{3} \emph{no}\expo{4} \emph{kyqyu}\expo{1} &`The men are sticky.' \\
\emph{tjoq}\expo{4} \emph{riq}\expo{2} &`s/he is sturdy'&[strong essence]&\emph{tjoq}\expo{4} \emph{no}\expo{4} \emph{kyqyu}\expo{1} &`The men are strong.' \\
\emph{tkonq}\expo{1} \emph{riq}\expo{2} &`s/he is greedy'&[ambitious essence]&\emph{tkonq}\expo{1} \emph{no}\expo{4} \emph{kyqyu}\expo{1} &`The men are ambitious.' \\
\emph{tlaq}\expo{14} \emph{riq}\expo{0} &`s/he placates, calms'&[cool essence]&\emph{tlaq}\expo{14} \emph{no}\expo{4} \emph{kyqyu}\expo{1} &`The men are cooled, calm.' \\
\emph{tnyaq}\expo{4} \emph{riq}\expo{2} &`s/he is tired'&[tired essence]&\emph{tnyaq}\expo{4} \emph{no}\expo{0} \emph{kyqyu}\expo{1} &`The men are tired.' \\
\emph{tqa}\expo{24} \emph{riq}\expo{2} &`s/he is fully conscious'&[complete essence]&\emph{tqa}\expo{24} \emph{no}\expo{4} \emph{kyqyu}\expo{1} &`all men' \\
\emph{tqi}\expo{4} \emph{riq}\expo{2} &`s/he hates'&[sick essence]&\emph{tqi}\expo{4} \emph{no}\expo{4} \emph{kyqyu}\expo{1} &`The men are sick.' \\
\emph{tya}\expo{20} \emph{riq}\expo{2} &`s/he is smart'&[smart essence]&\emph{tya}\expo{20} \emph{no}\expo{4} \emph{kyqyu}\expo{1} &`The men are smart.' \\
\emph{tyaq}\expo{4} \emph{riq}\expo{2} &`s/he is slow'&[slow essence]&\emph{tyaq}\expo{4} \emph{no}\expo{4} \emph{kyqyu}\expo{1} &`The men are slow.' \\
\emph{tyjyan}\expo{20} \emph{riq}\expo{2} &`s/he is skinny'&[skinny essence]&\emph{tyjyan}\expo{20} \emph{no}\expo{4} \emph{kyqyu}\expo{1} &`The men are skinny.' \\
\emph{tykeq}\expo{14} \emph{riq}\expo{0} &`s/he is hot'&[hot essence]&\emph{tykeq}\expo{14} \emph{no}\expo{0} \emph{kyqyu}\expo{1} &`The men are hot (temp).' \\
\emph{xyaq}\expo{2} \emph{riq}\expo{2}&`s/he is fed up with/tired of'&[mix essence]&\emph{xyaq}\expo{2} \emph{no}\expo{4} \emph{kyqyu}\expo{1} &`The men will be mixed.' \\
\lspbottomrule
\end{tabularx}
\end{table}

\begin{table}

\begin{tabular}{l l  }
\lsptoprule
Essence predicate&Gloss \\
 \midrule
\emph{jlya}\expo{32} \emph{riq}\expo{2} &`s/he forgot' \\
\emph{jlyo}\expo{20} \emph{riq}\expo{2} &`s/he knews/was aware' \\
\emph{jnya}\expo{20} \emph{riq}\expo{2} &`s/he was mischievous, playful' \\
\emph{ngwi}\expo{3} \emph{riq}\expo{2} &`s/he realized' \\
\emph{ndwe}\expo{4} \emph{riq}\expo{2} &`s/he worried' \\
\emph{ndya}\expo{32} \emph{riq}\expo{2} &`s/he liked, loved' \\
\emph{ndi}\expo{4} \emph{riq}\expo{2} &`s/he was thirsty, wheezing' \\
\emph{nteq}\expo{4} \emph{riq}\expo{2} &`s/he was hungry' \\
\emph{ntqan}\expo{1} \emph{riq}\expo{1} &`s/he was weak' \\
\emph{ntyqan}\expo{1} \emph{riq}\expo{2} &`s/he got used to' \\
\emph{qna}\expo{3} \emph{riq}\expo{2} &`s/he pitied' \\
\emph{senq}\expo{24} \emph{riq}\expo{1} &`s/he was upset' \\
\emph{skeq}\expo{1} \emph{riq}\expo{2} / \emph{qin}\expo{4} &`s/he mistook, misperceived' \\
\emph{skwi}\expo{1} \emph{riq}\expo{2} &`s/he took a liking to'  \\
\emph{snya}\expo{1} \emph{riq}\expo{2} &`s/he was desirous' \\
\emph{stu}\expo{1} \emph{riq}\expo{0} &`s/he was happy' \\
\emph{stya}\expo{4} \emph{riq}\expo{2}&`s/he was disgusted' \\
\emph{tlyaq}\expo{4} \emph{riq}\expo{2} &`s/he was cold' \\
\emph{tsa}\expo{3} \emph{riq}\expo{2} &`s/he was flirtatious' \\
\emph{tya}\expo{20} \emph{riq}\expo{2} &`s/he was careful'  \\
\emph{xkuq}\expo{42} \emph{riq}\expo{2} &`s/he was sad' \\
\emph{xqnyi}\expo{4} \emph{riq}\expo{2} &`s/he was afraid' \\
\emph{xyaq}\expo{2} \emph{riq}\expo{2} &`s/he was fed up with/ tired of'  \\
\emph{xyuq}\expo{1} \emph{riq}\expo{2}&`s/he bullied' \\
\emph{ylaq}\expo{42}/ \emph{ndlaq}\expo{42} \emph{riq}\expo{2} &`s/he was satisfied/satiated' \\
 \lspbottomrule
 \end{tabular}
 
 \caption{Essence predicates whose predicative bases are not used independently in SJQ Chatino}
\label{tab:CruzStump:sjq-19}
\end{table}
\is{predicative base|)}

Summarizing, we have seen that essence predicates exhibit three sorts of structural variety:  in their choice of \is{noun!nominal component}nominal component; in whether they exhibit one \is{noun!nominal component}nominal component or two; and in whether their \isi{predicative base} has uses apart from the essence predicate.  None of these sorts of structural variety is unexpected under the \isi{compound predicate hypothesis}.  Because a compound constitutes a lexeme, two compounds may differ in lexically idiosyncratic ways. Despite their closely related meanings, the \ili{English} compound nouns \textit{German
 shepherd} and \textit{Shetland sheepdog} differ in their internal logic; while one can imagine alternative combinations such as \textit{Germany sheepdog} and \textit{Shetlander shepherd}, each breed has its own conventional name agreed upon on the occasion of its coinage.  In the same way, the use of \emph{riq}\expo{2}, \emph{tye}\expo{32}, \emph{qin}\expo{4} or the combination \emph{riq}\expo{2} \emph{tye}\expo{32} as an essence predicate's \is{noun!nominal component}nominal component is a matter of convention enforced by the lexicon of Chatino.  The incidence of essence predicates whose \isi{predicative base} is a \isi{cranberry morpheme} is further testimony to their lexical status; in such cases, the predicative base, like the \textit{were-} in \ili{English}  \textit{werewolf}, persists long after losing its status as an independent lexeme.  
If one instead views essence predicates as predicates having inalienably possessed\is{inalienable possession!possessed}
subjects, the structural variety examined here is somewhat unexpected.  On that conception, the choice among \emph{riq}\expo{2}, \emph{tye}\expo{32} and \emph{qin}\expo{4} as subjects should seemingly be independent of the choice of predicate,  and they should not appear in tandem (any more than \textit{you} and \textit{they} should appear in tandem to produce sentences such as *\textit{You they left}).

\subsection{External syntax}



With only occasional exceptions, the components of an essence predicate can be
interrupted by members of a small class of elements; their syntax relative to these elements is a revealing criterion for evaluating the possessed-subject and compound predicate hypotheses.  The class of interruptors includes the elements in (\ref{ex:CruzStump:19}), some of which \cite[10]{Rasch02} labels \textsc{event
modifiers}; we extend his terminology to the full class. These may intervene between a verb and its
subject, as in examples (\ref{ex:CruzStump:20})\textendash (\ref{ex:CruzStump:25}) (where verb and subject are in boldface).  Correspondingly, they may sometimes intervene between an essence predicate's \isi{predicative base} and its nominal element, as in (\ref{ex:CruzStump:26})\textendash(\ref{ex:CruzStump:33}), in which the interrupted essence predicates are in boldface.

\begin{exe}

	\ex\label{ex:CruzStump:19}

	 \ea {{sqwe}\expo{3}}\\  {`good, well'} \\
	 \ex {{ka}\expo{24}}\\  {`able to; expression of emphasis'}\\
\largerpage
	 \ex {{ye}\expo{42}}\\  {`very'} \\
	 \ex {{la}\expo{24}}\\  {`comparative'}\\
	 \ex {{qa}\expo{24}}\\  {`very'} \\
	 \ex {{kcha}\expo{4}}\\  {`crazy'}\\
\newpage 	 
	 \ex {{ti}\expo{2}, {ti}\expo{4}}\\  {`very, still, just' }
	 \ex {{kcha}\expo{4} {qa}\expo{1}}\\  {`crazy'}\\
	 \z


	\ex\label{ex:CruzStump:20}
	\gll { \textbf{{Ntqan}\expo{42}}} {{ti}\expo{4}} { \textbf{{La}\expo{20}{ya}\expo{24}}} {{kwa}\expo{3} } {{kna}\expo{1},} {{kwen}\expo{42}  } {{qa}\expo{24} } {{ntqo}\expo{1} } { {xqya}\expo{24}.}\\
	 {see.\textsc{cpl}} {\textsc{ev.mod}:just} {Hilaria} {that} {snake} {loud} {very} {leave.\textsc{cpl}} {scream}\\
	\glt `As soon as Hilaria saw the snake, she screamed very loudly.'\\


	\ex\label{ex:CruzStump:21}
	\gll {\textbf{{Nkwa}\expo{2}}  } { \textbf{{tqi}\expo{4}} } {  {ka}\expo{3}  } { \textbf{{nten}\expo{14}} } {  \textbf{{no}\expo{0}}  } { \textbf{{yku}\expo{24}}  } { \textbf{{kla}\expo{32}}  } { \textbf{{xi}\expo{1}}  } { \textbf{{kanq}\expo{42}} } {{qa}\expo{24}.}\\
	 {be.\textsc{cpl} } { sick } { \textsc{ev.mod:emph} } { people } { that } { eat.\textsc{cpl} } { fish } { bad } { that.\textsc{abs} } {very}\\
	\glt `People who ate the bad fish got really sick.'\\



	\ex\label{ex:CruzStump:22}
	\gll {{Ti}\expo{2}  } { \textbf{{ykwiq}\expo{1}}  } { {ye}\expo{42} } {  \textbf{{silya}\expo{14}} } { {qo}\expo{0} } {{chaq}\expo{3}} {{tyqo}\expo{1}} {qo}\expo{1}  {{ ja}\expo{4}  } { {slya}\expo{1}}  { {qa}\expo{1}. } \\
	 {\textsc{ev.mod}:very } { speak.\textsc{cpl} } { \textsc{ev.mod}:very } { police } { with.\textsc{3sg} } { to } {leave} {and}  { \textsc{neg} } { agree.\textsc{cpl} } { \textsc{neg} }\\
	 \glt `The police pleaded with him to leave and he refused (to leave).'\\


	\ex\label{ex:CruzStump:23}
	\gll { \textbf{{Ykwiq}\expo{4}} } { {la}\expo{1}  } { \textbf{{sti}\expo{4}}-\textbf{{qo}\expo{2}}  } { \textbf{{kwa}\expo{4}}  } { {ke}\expo{4}  } { {neq}\expo{4}-{sya}\expo{10}  } { {kwa}\expo{1}.}\\
	 { speak.\textsc{cpl} } { \textsc{ev.mod}:more } { father-saint } { that } { then } { type.people-justice } { that}\\
	\glt `The priest spoke more than the authorities.'\\



	\ex\label{ex:CruzStump:24}
	\gll {\textbf{{Ya}\expo{42}}  } { {kcha}\expo{4}  } { \textbf{{no}\expo{4}}  } { \textbf{{qan}\expo{1}}    } { \textbf{{lyuq}\expo{04}} } { \textbf{{kwa}\expo{3}}.}\\
	 {go.away.\textsc{cpl} } { \textsc{ev.mod}:crazy } { one } { female } { little } { that}\\
	\glt `That little girl went (somewhere) aimlessly.'\\



	\ex\label{ex:CruzStump:25}
	\gll {\textbf{{Qya}\expo{42}} } {  {kcha}\expo{4} } {  {qa}\expo{1}  } { \textbf{{kyo}\expo{24}}  } { {nka}\expo{3}.}\\
	 { fall.\textsc{cpl} } { \textsc{ev.mod}:crazy } { \textsc{ev.mod}:very } { rain } { yesterday}\\
	\glt `It rained crazy, unpredictably yesterday.'\\



	\ex\label{ex:CruzStump:26}
	\gll
	\small
	 {\textbf{{Nkqan}\expo{4}} } {{sqwe}\expo{3}} {\textbf{{riq}\expo{2}} } {{no}\expo{4}} {{qan}\expo{1}} {{chaq}\expo{3} } {{tsa}\expo{24} } {{kwaq}\expo{3}.}\\
	 {sitting.ground.\textsc{prog}} {\textsc{ev.mod}:well} {essence} {the} {female (one) } {\textsc{compl}} {go.away.\textsc{pot}} {that}\\
	\glt `That woman remembers well that she has to go to the party.'\\



	\ex\label{ex:CruzStump:27}
	\gll {\textbf{{sqwa}\expo{140}}} {{ye}\expo{42}} {\textbf{{chaq}\expo{3}}} { \textbf{{tye}\expo{32}}.}\\
	 {put \textsc{cpl}.\textsc{3sg}} {\textsc{ev.mod}:very} { word/thing  } {chest}\\
	\glt `He really encouraged him.'\\



	\ex\label{ex:CruzStump:28}
	\gll {\textbf{{Ndon}\expo{42}} } {{qa}\expo{24}} {\textbf{{riq}\expo{2}} } {{Xwa}\expo{3}} {{ kwa}\expo{3}} {{ndon}\expo{42}} {{tqwa}\expo{4}{-tqan}\expo{4}} {{qin}\expo{4}.}\\
	 {stand.\textsc{prog}} {\textsc{ev.mod}:very} {essence} {Juan} {that} {stand.\textsc{prog}} {mouth-house } {his}\\
	\glt `Juan was very happy standing in is front porch.'\\



	\ex\label{ex:CruzStump:29}
	\gll {\textbf{{Ndon}\expo{42}} } {{ti}\expo{4}} {\textbf{{riq}\expo{2}}} {{Xwa}\expo{3}} {{ kwa}\expo{3}} {{ndon}\expo{42}} {{tqwa}\expo{4}{-tqan}\expo{4}} {{qin}\expo{4}.}\\
	 {stand.\textsc{prog}} {\textsc{ev.mod}:only} {essence} {Juan} {that} {stand.\textsc{prog}} {mouth-house } {his}\\
	\glt `Juan was just happy standing in his front porch.'\\



	\ex\label{ex:CruzStump:30}
	\gll {\textbf{{Ndon}\expo{42}} } {{ka}\expo{24}} {\textbf{{riq}\expo{2}}} {{Xwa}\expo{3}} {{ kwa}\expo{3}} {{ndon}\expo{42}} {{ tqwa}\expo{4}{-tqan}\expo{4}} {{qin}\expo{4}.}\\
	 {stand.\textsc{prog}} {\textsc{ev.mod}:very} {essence} {Juan} {that} {stand.\textsc{prog}} {mouth-house } {his}\\
	\glt `Juan was sure very happy standing in his front porch.'\\



	\ex\label{ex:CruzStump:31}
	\gll
	 {\textbf{{Ndon}\expo{42}} } {{la}\expo{24}} {\textbf{{riq}\expo{2}}} {{Xwa}\expo{3}} {{kwa}\expo{3}} {{ndon}\expo{42}} {{tqwa}\expo{4}{-tqan}\expo{4} } {{qin}\expo{4}.}\\
	 {stand.\textsc{prog}} {\textsc{ev.mod}:more} {essence} {Juan} {that} {stand.\textsc{prog}} {mouth-house } {his}\\
	\glt `Juan was happier standing in his front porch.'\\



	\ex\label{ex:CruzStump:32}
	\gll {\textbf{{Ndo}\expo{42}}} {{kcha}\expo{4} } {{qa}\expo{1}} {\textbf{{riq}\expo{2}}.}\\
	 {happy} {\textsc{ev.mod}:crazy} {\textsc{ev.mod}:very} {essence}\\
	\glt `He is crazy happy.'\\


	\ex\label{ex:CruzStump:33}
	\gll
	\small
	 {\textbf{{Stu}\expo{10}} } {{kcha}\expo{0} } {*{(qa}\expo{24})} {\textbf{{riq}\expo{1}} } {{Xwa}\expo{3} } {{kwa}\expo{3} } {{nkya}\expo{24}.}\\
	 {gusto} {\textsc{ev.mod}:crazy} {\textcolor{white}{*(}\textsc{ev.mod}:very} {essence} {Juan} {that} {go.away.\textsc{cpl}}\\
	\glt `Juan left awfully happy.'\\

\end{exe}

Strikingly, compound predicates generally resist the intrusion of an event modifier, a fact reflected by the unacceptability of (\ref{ex:CruzStump:34}).  When an event modifier combines with a compound predicate, it generally follows it, as in (\ref{ex:CruzStump:35}).  Yet, event modifiers in general do not follow essence predicates, as the evidence in (\ref{ex:CruzStump:36}) and (\ref{ex:CruzStump:37}) attests.  Similarly, event modifiers do not typically follow the subject of a clause.  Thus, in (\ref{ex:CruzStump:38}), the event modifier may intrude between the verb \textit{ylu}\expo{2} `it grew' and its subject \textit{yka}\expo{24}-\textit{knyi}\expo{24}  \textit{kwa}\expo{3} `that tree graft' (as in (\ref{ex:CruzStump:38a})) but cannot follow the subject (*(\ref{ex:CruzStump:38b})).  The overarching generalization is that an event modifier typically follows the head of a predicate phrase, whether this head be simplex or compound.  This generalization suggests that because an event modifier typically follows an essence predicate's \isi{predicative base}, the essence predicate itself is phrasal. 

\begin{exe}

	\ex[*]{
	\gll
	 {\textbf{{Ykon}\expo{1}}} {{ten}\expo{24}} {\textbf{{jyanq}\expo{24}}} {{skwa}\expo{1}} {{qin}\expo{24}.}\\
	  {{eat}.\textsc{cpl.1sg}} {{\textsc{ev.mod}:only}.\textsc{1sg}} {{measure}.\textsc{1sg}} {{soup}} {{his}}\\
	\glt Sought interpretation: `I only tasted his soup.'}
	\label{ex:CruzStump:34}



	\ex[]{
	\gll
	  {\textbf{{Ykon}\expo{1}}} {\textbf{{jyanq}\expo{24}}} {{ten}\expo{24}} {{skwa}\expo{1}} {{qin}\expo{24}.}\\
	  {{eat}.\textsc{cpl.1sg}} {{measure}.\textsc{1sg}} {{\textsc{ev.mod}:only}.\textsc{1sg}} {{soup}} {{his}}\\
	\glt `I only tasted his soup.'}
	\label{ex:CruzStump:35}



	\ex\label{ex:CruzStump:36}
	\begin{xlist}
	\ex[]{ \gll {\textbf{{Ndon}\expo{42}}} {{qa}\expo{24}} {\textbf{{riq}\expo{2}}.}\\
	 {{stand.\textsc{prog}}} {\textsc{ev.mod}:very} {essence}\\
	\glt `S/he is very happy.'}
	\label{ex:CruzStump:36a}
	 \ex[*]{ \gll { \textbf{{Ndon}\expo{42}}} {\textbf{{riq}\expo{2}}} { qa\expo{1}.}\\
	 {{stand.\textsc{prog}}} {essence} {\textsc{ev.mod}:very}\\
	\glt Sought interpretation: `S/he is very happy.'}
	\label{ex:CruzStump:36b}
	\z



	\ex\label{ex:CruzStump:37}
	\begin{xlist}
	\ex[]{\gll {\textbf{{Qne}\expo{42}}} {{sqwe}\expo{3}} {\textbf{{tye}\expo{3}}.}\\
	 {{do.\textsc{cpl}}} {\textsc{ev.mod}:good} {chest.\textsc{3sg}}\\
	\glt `S/he dared do something.'}
	 \ex[*]{\gll { \textbf{{Qne}\expo{42}}} {\textbf{{tye}\expo{32}}} {{sqwe}\expo{3}.}\\
	 {{do.\textsc{cpl}}} {chest.\textsc{3sg}} {\textsc{ev.mod}:good}\\
	\glt Sought interpretation: `S/he dared do something.'}
	\z



	\ex\label{ex:CruzStump:38}
	\begin{xlist}
	\ex[]{\gll {\textbf{{Ylu}\expo{2}}} {{sqwe}\expo{3}} {\textbf{{yka}\expo{24}-{knyi}\expo{24}}} {\textbf{{kwa}\expo{3}}.}\\
	 {{grow.\textsc{cpl}}} {\textsc{ev.mod}:well} {tree-graft} {that}\\
	\glt `That grafted tree grew really well.'}\label{ex:CruzStump:38a}
	 \ex[*]{\gll {*\textbf{{Ylu}\expo{2}}} {\textbf{{yka}\expo{24}-{knyi}\expo{24}}} {{sqwe}\expo{3}.} {}\\
	 {{grow.\textsc{cpl}}} {tree-graft} {\textsc{ev.mod}:good} {}\\
	\glt Sought interpretation: `That grafted tree grew really well.'\\}
	\label{ex:CruzStump:38b}
	\z




\end{exe}
This distributional generalization about event modifiers is, however, deceptively broad, because event modifiers exhibit a number of idiosyncrasies in their interaction with essence predicates.   On one hand, the event modifiers \emph{ti}\expo{2} / \emph{ti}\expo{4} `very, still, just', \emph{ka}\expo{24} `able to', \emph{la}\expo{24} `comparative', \emph{kcha}\expo{4} `crazy', and \emph{kcha}\expo{4} \emph{qa}\expo{1} `crazy' intervene quite freely between the parts of an essence predicate with two components; thus, all of these event modifiers may appear in the contexts in (\ref{ex:CruzStump:39}). On the other hand, if an essence predicate has three or more components, these event modifiers exhibit a much more variable pattern of distribution, as the examples in (\ref{ex:CruzStump:40}) suggest.  

\begin{exe}

	    \ex\label{ex:CruzStump:39}%
	\begin{tabular}[t]{l  lll  }
	`s/he worries'&{ndwe}\expo{4} &{}\textemdash{}   &{riq}\expo{2}\\
	`s/he remembers'&{nkqan}\expo{4} &{}\textemdash{}  &{riq}\expo{2} \\
	`s/he is standoffish'&{siyeq}\expo{3} &{}\textemdash{}  &{riq}\expo{2}\\
	`s/he is daring'&{tno}\expo{4} &{}\textemdash{}  &{tye}\expo{32}\\
	`s/he is afraid'&{xqnyi}\expo{4} &{}\textemdash{}  &{riq}\expo{2} \\
	\end{tabular}
	\normalfont


	    \ex\label{ex:CruzStump:40}% 
	\begin{tabular}[t]{l  l lllll@{}l@{}}
	`s/he forgives'&{chaq}\expo{3}	&\underline{ * } 	&{tlyu}\expo{2} 	&\underline{ * } &{riq}\expo{2} &\hspace*{-5mm}(very idiomatic)\\
	`s/he realizes'&{nkwa}\expo{2}	&\underline{ * } &{jyaq}\expo{3}	 &\underline{ ✔} &{riq}\expo{2} &\\
	`s/he made up her/his mind'&{wa}\expo{2}	&\underline{ * } 	&{xtya}\expo{20} &\underline{ ✔} &{riq}\expo{2}/{tye}\expo{32} &\\
	`s/he feels sad'&{tqwa}\expo{14}	&\underline{  ✔}	&{nka}\expo{24} &\underline{ * } &{tye}\expo{32}&\\
	`s/he believes/is gullible'&{ya}\expo{42}	&\underline{ * } 	&{ntyqan}\expo{4} &\underline{ ✔} &{riq}\expo{1} \underline{  * }  {qin}\expo{24} &\\
\end{tabular}
\end{exe}



Moreover, the event modifiers \emph{sqwe}\expo{3} `good', \emph{ye}\expo{42} `very' and \emph{qa}\expo{24} `very' exhibit a much higher degree
of idiosyncrasy in their capacity to intervene between the parts of an essence predicate, as the examples in \tabref{tab:CruzStump:sjq-20} show. This irregularity very likely has more than one cause. Some interventions are semantically improbable, e.g. *\emph{senq}\expo{24} \emph{sqwe}\expo{3} \emph{riq}\expo{1} `s/he is well upset'.  But it also appears that essence predicates are simply more fully grammaticalized as tightly bound units, more strongly resisting intrusive formatives. 


 \begin{table}

\begin{tabular}{l l  c  c   }
\lsptoprule
\multirow{2}{*}{Gloss} & \multirow{2}{*}{Essence predicate} & \emph{sqwe}\expo{3} `good' & \multirow{2}{*}{\emph{qa}\expo{24} `very' }\\ & &  \emph{ye}\expo{42} `very' & \\
\midrule
`s/he remembered' & \emph{nkqan}\expo{4} \normalfont{\textemdash} \emph{riq}\expo{2} & ✔ & ✔ \\
`s/he was smart, fast, agile'  & \emph{sa}\expo{4} \normalfont{\textemdash} \emph{riq}\expo{2} & ✔ & ✔ \\
`s/he was happy' & \emph{ndon}\expo{42} \normalfont{\textemdash}  \emph{riq}\expo{2} & ✔&* \\
`s/he remembered' &\emph{sqwi}\expo{4} \normalfont{\textemdash} \emph{riq}\expo{2} & ✔&* \\
`s/he was upset' &\emph{senq}\expo{14} \normalfont{\textemdash}  \emph{riq}\expo{0} &*& ✔ \\
`s/he pitied' &\emph{qna}\expo{3} \normalfont{\textemdash} \emph{riq}\expo{2} &*& ✔ \\
`s/he was sad' &\emph{xkuq}\expo{42} \normalfont{\textemdash} \emph{riq}\expo{2} &*&* \\
`s/he worried' &\emph{ndwe}\expo{4} \normalfont{\textemdash} \emph{riq}\expo{2} &*&* \\
`s/he was fed up' &\emph{skwa}\expo{3} \normalfont{\textemdash} \emph{riq}\expo{2} & \#& \# \\
`s/he hated' &\emph{tqi}\expo{4} \normalfont{\textemdash} \emph{riq}\expo{2} & \#& \# \\
`s/he was generous/happy' &\emph{sqwe}\expo{3} \normalfont{\textemdash} \emph{riq}\expo{2}/\emph{tye}\expo{32} & \#&* \\
`s/he was taciturn' &\emph{liqa}\expo{24} \normalfont{\textemdash} \emph{riq}\expo{1} & \# &* \\
`s/he was scared/queasy' &\emph{chin}\expo{4} \normalfont{\textemdash} \emph{nga}\expo{24} 	\emph{tye}\expo{32} &*&* \\
 &\emph{chin}\expo{4} \emph{nga}\expo{24} \normalfont{\textemdash} \emph{tye}\expo{32} &*&* \\
`s/he liked' &\emph{ndya}\expo{32} \normalfont{\textemdash} \emph{riq}\expo{2} \emph{tye}\expo{32}&*& ✔ \\
&\emph{ndya}\expo{24} \emph{riq}\expo{2} \normalfont{\textemdash} \emph{tye}\expo{32}& \# & \# \\
`s/he felt angry' &\emph{xqan}\expo{10} \normalfont{\textemdash} \emph{nga}\expo{24} \emph{tye}\expo{32} &*& \# \\
&\emph{xqan}\expo{10} \emph{nga}\expo{24} \normalfont{\textemdash} \emph{tye}\expo{32} &*&* \\
\lspbottomrule
\end{tabular}

\caption{Intervention of the event modifiers\is{event!event modifier} \emph{sqwe}\expo{3} `good', \emph{ye}\expo{42} `very' and \emph{qa}\expo{24} `very' into essence predicates in SJQ Chatino}
\label{tab:CruzStump:sjq-20}
\end{table}

We conclude that although the distribution of event modifiers exhibits a number of idiosyncrasies, essence predicates resemble verb + subject combinations more closely than they resemble compound predicates as regards their interaction with event modifiers.  Thus, this evidence militates in favor of the \isi{possessed-subject hypothesis} and against the \isi{compound predicate hypothesis}.

\subsection{Lack of \isi{compositionality}}

As we have seen, essence predicates tend to refer psychological states, with some exceptions.  In a large proportion of cases, essence predicates are not transparently compositional\is{compositionality}. There are, to be sure, those whose semantics is directly deducible from their parts; examples are the essence predicates in \tabref{tab:CruzStump:sjq-21}. But a substantial number of essence predicates exhibit various degrees of departure from \isi{compositionality}; the examples in \tabref{tab:CruzStump:sjq-22} illustrate. The analogy of essence predicates to lexically reflexive verbs\is{verb!reflexive} (noted 
in section 1) is again apt, since reflexive predicates are often idiosyncratic in their semantics; compare \emph{attendre} `wait for' to \textit{s'attendre} (\textit{\`a}) `expect', \emph{douter} `doubt' to \emph{se} \emph{douter} `suspect', \emph{rendre} `return' to \emph{se} \emph{rendre} (\textit{\`a}) `go to'.  In the case of essence predicates whose \isi{predicative base} is a \isi{cranberry morpheme} appearing in no context other than the essence predicate itself (see again \tabref{tab:CruzStump:sjq-19}), there is no real question of \isi{compositionality}. Here, too, the analogy to lexically reflexive verbs\is{verb!reflexive} holds, since they also may be based on cranberry morphemes\is{morpheme!cranberry morpheme}, as in the case of \ili{French} \textit{s'\'evanouir} `faint' (whose verbal base \textit{\'evanouir} has no independent use).

\begin{table}

\begin{tabular}{l l  l  }
\lsptoprule
&Essence predicate&Glosses of component parts \\
 \midrule
`s/he was hard-working'&\emph{t(j)nya}\expo{3} \emph{riq}\expo{2}&[work essence] \\
`s/he was open, extrovert'&\emph{la}\expo{1} \emph{riq}\expo{2}&[open essence] \\
`s/he realized'&\emph{ngwi}\expo{3} \emph{riq}\expo{2}&[awake essence] \\
`s/he got used to'&\emph{nt(y)qan}\expo{1}  \emph{riq}\expo{2}&[used.to essence] \\
`s/he was hungry' &\emph{ntenq}\expo{32} \emph{riq}\expo{2}&[hungry essence] \\
`s/he was feeling lazy'&\emph{ntja}\expo{1}  \emph{riq}\expo{2}&[lazy essence] \\
`s/he was stupid'&\emph{ntu}\expo{1} \emph{riq}\expo{0}&[stupid essence] \\
`s/he misperceived'&\emph{skeq}\expo{1} \emph{riq}\expo{0}&[imagine essence] \\
`s/he was strong/ sturdy'&\emph{sqye}\expo{14} \emph{riq}\expo{0}&[strong essence] \\
`s/he was skinny'&\emph{ti}\expo{4}  \emph{riq}\expo{2}&[skinny essence] \\
`s/he was sturdy'&\emph{tjoq}\expo{4}  \emph{riq}\expo{2}&[sturdy essence] \\
`s/he was greedy'&\emph{tkonq}\expo{1} \emph{riq}\expo{2}&[greedy essence] \\
`s/he was cold'&\emph{tlyaq}\expo{4}  \emph{riq}\expo{2}&[cold essence] \\
`s/he was tired'&\emph{tnyaq}\expo{4} \emph{riq}\expo{2}&[tired essence] \\
`s/he was slow'&\emph{tyaq}\expo{4} \emph{riq}\expo{2}&[slow essence] \\
`s/he was skinny'&\emph{tyjyan}\expo{20} \emph{riq}\expo{2}&[skinny essence] \\
`s/he was hot'&\emph{tykeq}\expo{14} \emph{riq}\expo{0}&[hot essence] \\
`s/he was shy'&\emph{yqu}\expo{20} \emph{riq}\expo{2}&[embarrassed essence] \\
 \lspbottomrule
 \end{tabular}
 \caption{Semantically transparent essence predicates in SJQ Chatino}
\label{tab:CruzStump:sjq-21}

\end{table}


\begin{table}
\fittable{
\begin{tabular}{l l  l  }
  \lsptoprule
&Essence predicate&Glosses of component parts \\
\midrule
`s/he was mischievous, playful'&\emph{jnya}\expo{20} \emph{riq}\expo{2}&[borrow essence] \\
`s/he broke a bad habit/learned a lesson'&\emph{ksa}\expo{4} \emph{riq}\expo{2}&[break essence] \\
`s/he worried'&\emph{ndwe}\expo{4} \emph{riq}\expo{2}&[minced essence] \\
`s/he was fair-skinned'&\emph{lwi}\expo{3}  \emph{riq}\expo{2}&[clean essence] \\
`s/he was standoffish'&\emph{lyaq}\expo{14} \emph{riq}\expo{0}&[quiet essence] \\
`s/he mocked'&\emph{lyeq}\expo{3} \emph{riq}\expo{2}&[fun essence] \\
`s/he was satisfied/satiated'&\emph{ndla}\expo{2} \emph{riq}\expo{2}&[fast essence] \\
`s/he was happy'&\emph{ndon}\expo{42} \emph{riq}\expo{2}&[stand essence] \\
`s/he remembered'&\emph{nkqan}\expo{4} \emph{riq}\expo{2}&[sit essence] \\
`s/he realized'&\emph{nkwa}\expo{2} \emph{jyaq}\expo{3} \emph{riq}\expo{2}&[be.able measure essence] \\
`s/he was dark-skinned'&\emph{nta}\expo{14} \emph{riq}\expo{0}&[dark essence] \\
`s/he pitied'&\emph{qna}\expo{3}  \emph{riq}\expo{2}&[poor essence] \\
`s/he was smart/fast agile'&\emph{sa}\expo{4} \emph{riq}\expo{2}&[airy essence] \\
`s/he was fed up'&\emph{skwa}\expo{3} \emph{riq}\expo{2}&[lie.elevated essence] \\
`s/he took a liking to'&\emph{skwi}\expo{1} \emph{riq}\expo{2}&[round essence] \\
`s/he liked sth'&\emph{snyi}\expo{4}  \emph{riq}\expo{2}&[grab essence] \\
`s/he was excited'&\emph{sti}\expo{1}  \emph{riq}\expo{2}&[laugh essence] \\
`s/he was standoffish'&\emph{syeq}\expo{3}  \emph{riq}\expo{2}&[happy essence] \\
`s/he was frugal or took care of sth'&\emph{tjenq}\expo{3} \emph{riq}\expo{2} &[sticky essence] \\
`s/he placated'&\emph{tlaq}\expo{14} \emph{riq}\expo{0}&[cool essence] \\
`s/he was fully conscious'&\emph{tqa}\expo{24} \emph{riq}\expo{2}&[complete essence] \\
`s/he was envious'&\emph{tqi}\expo{4} \emph{riq}\expo{2}&[sick essence] \\
`s/he was afraid'&\emph{xqnyi}\expo{4} \emph{riq}\expo{2}&[sad essence] \\
`s/he was fed up with/tired of'&\emph{xyaq}\expo{2}  \emph{riq}\expo{2}&[mix essence] \\
`s/he bullied'&\emph{xyuq}\expo{1} \emph{riq}\expo{2}&[naughty essence] \\
\lspbottomrule
\end{tabular}
  }
 \caption{Semantically opaque essence predicates in SJQ Chatino}
\label{tab:CruzStump:sjq-22}

\end{table}

\newpage 
These facts about the semantics of essence predicates might be seen as favoring the \isi{compound predicate hypothesis}; the observed variability in semantic transparency is, of course, typical of compounds.  But the semantic noncompositionality\is{compositionality} of many essence predicates might be reconciled with the possessed-subject hypothesis by regarding them as idioms; even the incidence of essence predicates with cranberry morphemes\is{morpheme!cranberry morpheme} might be likened to the fact that idioms sometimes involve words that have no use outside the idiom (e.g. \textit{jiffy} in the idiom \textit{in a jiffy}, \textit{dint} in \textit{by dint of}, \textit{fro} in \textit{to and fro}). Nevertheless, recurring commonalities of form and content among essence predicates might be argued to make them different from idioms, which tend not to possess this high degree of systematicity.

\subsection{Distributional flexibility of subject-\isi{agreement} marking}
 
An important feature of Chatino subject-\isi{agreement} marking is its flexibility:  in the inflection of a simplex verb, subject-\isi{agreement} marking is expressed cumulatively with aspect/mood marking (as in the case of \emph{sqi}\expo{2} `s/he bought'\textemdash \tabref{tab:CruzStump:sjq-8}); but in the inflection of a compound predicate, aspect/mood is marked on the first member, and subject \isi{agreement} is marked separately, on the second member (as in the case of \emph{yku}\expo{4} \emph{jyaq}\expo{3} 's/he tasted'\textemdash \tabref{tab:CruzStump:sjq-12}).  This flexibility extends even farther:  If a simplex verb is followed by an \isi{event modifier}, the event modifier may carry the verb's subject-\isi{agreement} morphology; thus, compare the inflection of \emph{ykwiq}\expo{4} `s/he spoke' in \tabref{tab:CruzStump:sjq-23} with that of  \emph{ykwiq}\expo{4} \emph{ti}\expo{4} `s/he just spoke' [speak \textsc{event.modifier}] in \tabref{tab:CruzStump:sjq-24}.\footnote{Note that as in the inflection of the \is{compounding!compound verb}compound verb  \emph{yku}\expo{4} \emph{jyaq}\expo{3} `s/he tasted' [eat amount] in \tabref{tab:CruzStump:sjq-12}, the inflection of the verb + \isi{event modifier} combination \emph{ykwiq}\expo{4} \emph{ti}\expo{4} `s/he just spoke' [speak \textsc{event.modifier}]\is{event!event modifier} exhibits ablaut of its verbal element in the first person singular.}

\begin{table}

\begin{tabularx}{\textwidth}{lXXXX}
\lsptoprule
&\textsc{completive}&\textsc{progressive} &\textsc{habitual}&\textsc{potential}\\
\midrule
\textsc{1sg}&\emph{ykwenq}\expo{1}&\emph{ntykwenq}\expo{1}&\emph{ntykwenq}\expo{20}&\emph{tykwenq}\expo{20}\\
\textsc{2sg}&\emph{ykwiq}\expo{42}&\emph{ntykwiq}\expo{42}&\emph{ntykwiq}\expo{42}&\emph{tykwiq}\expo{42}\\
\textsc{3sg}&\emph{ykwiq}\expo{4}&\emph{ntykwiq}\expo{32}&\emph{ntykwiq}\expo{4}&\emph{tykwiq}\expo{4}\\
\textsc{1incl}&\emph{ykwenq}\expo{24} \emph{en}\expo{32}&\emph{ntykwenq}\expo{1} \emph{en}\expo{32}&\emph{ntykwenq}\expo{24} \emph{en}\expo{32}&\emph{tykwenq}\expo{24} \emph{en}\expo{32}\\
\textsc{1excl}&\emph{ykwiq}\expo{4} \emph{wa}\expo{42}&\emph{ntykwiq}\expo{32} \emph{wa}\expo{42}&\emph{ntykwiq}\expo{4} \emph{wa}\expo{42}&\emph{tykwiq}\expo{4} \emph{wa}\expo{42}\\
\textsc{2pl}&\emph{ykwiq}\expo{4} \emph{wan}\expo{4}&\emph{ntykwiq}\expo{32} \emph{wan}\expo{4}&\emph{ntykwiq}\expo{4} \emph{wan}\expo{4}&\emph{tykwiq}\expo{4} \emph{wan}\expo{4}\\
\textsc{3pl}&\emph{ykwiq}\expo{4} \emph{renq}\expo{4}&\emph{ntykwiq}\expo{32} \emph{renq}\expo{4}&\emph{ntykwiq}\expo{4} \emph{renq}\expo{4}&\emph{tykwiq}\expo{4} \emph{renq}\expo{4}\\
\lspbottomrule
\end{tabularx}
\caption{Paradigm of the verb \emph{ykwiq}\expo{4} `s/he spoke' in SJQ Chatino}
\label{tab:CruzStump:sjq-23}

\end{table}

\begin{table}
\small
\begin{tabularx}{\textwidth}{Xllll}
\lsptoprule
&		\textsc{completive}	&	\textsc{progressive}	&	\textsc{habitual}	&	\textsc{potential}		\\
\midrule
\textsc{1sg}	&\emph{ykwenq}\expo{1} \emph{ten}\expo{24}	&\emph{ntykwenq}\expo{1} \emph{ten}\expo{24}	&\emph{ntykwenq}\expo{20} \emph{ten}\expo{24}	&\emph{tykwenq}\expo{20} \emph{ten}\expo{24} \\
\textsc{2sg}	&\emph{ykwiq}\expo{4} \emph{ti}\expo{42}	&\emph{ntykwiq}\expo{32} \emph{ti}\expo{42}	&\emph{ntykwiq}\expo{4} \emph{ti}\expo{42}	&\emph{tykwiq}\expo{4} \emph{ti}\expo{42} \\
\textsc{3sg}	&\emph{ykwiq}\expo{4} \emph{ti}\expo{4}	&\emph{ntykwiq}\expo{32} \emph{ti}\expo{4}	&\emph{ntykwiq}\expo{4} \emph{ti}\expo{4}	&\emph{tykwiq}\expo{4} \emph{ti}\expo{4} \\
\textsc{1incl}	&\emph{ykwiq}\expo{4} \emph{ten}\expo{24} \emph{en}\expo{32}	&\emph{ntykwiq}\expo{32} \emph{ten}\expo{24} \emph{en}\expo{32}	&\emph{ntykwenq}\expo{24} \emph{ten}\expo{24} \emph{en}\expo{32}	&\emph{tykwenq}\expo{24} \emph{ten}\expo{24} \emph{en}\expo{32} \\
\textsc{1excl}	&\emph{ykwiq}\expo{4} \emph{ti}\expo{4} \emph{wa}\expo{42}	&\emph{ntykwiq}\expo{32} \emph{ti}\expo{4} \emph{wa}\expo{42}&	\emph{ntykwiq}\expo{4} \emph{ti}\expo{4} \emph{wa}\expo{42}	&\emph{tykwiq}\expo{4} \emph{ti}\expo{4} \emph{wa}\expo{42} \\
\textsc{2pl}	&\emph{ykwiq}\expo{4} \emph{ti}\expo{4} \emph{wan}\expo{4} 	&\emph{ntykwiq}\expo{32} \emph{ti}\expo{4} \emph{wan}\expo{4}	&\emph{ntykwiq}\expo{4} \emph{wan}\expo{4}	&\emph{tykwiq}\expo{4} \emph{wan}\expo{4} \\
\textsc{3pl}	&\emph{ykwiq}\expo{4} \emph{ti}\expo{4} \emph{renq}\expo{4}	&\emph{ntykwiq}\expo{32} \emph{ti}\expo{4} \emph{renq}\expo{4}	&\emph{ntykwiq}\expo{4} \emph{renq}\expo{4}	&\emph{tykwiq}\expo{4} \emph{renq}\expo{4} \\
\lspbottomrule
\end{tabularx}
\caption{Paradigm of \emph{ykwiq}\expo{4} \emph{ti}\expo{4} `s/he just spoke' [speak \textsc{event.modifier}]\is{event!event modifier} in SJQ Chatino}
\label{tab:CruzStump:sjq-24}

\end{table}


The \isi{compound predicate hypothesis} entails that in the inflection of an essence predicate, the \is{noun!nominal component}nominal component (\emph{riq}\expo{2}, \emph{tye}\expo{32} or \emph{qin}\expo{4}, alone or in combination) functions very much like the \isi{event modifier} \emph{ti}\expo{4} in the inflection of \emph{ykwiq}\expo{4} \emph{ti}\expo{4} `s/he just spoke':  not as a subject, but as an adverbial or quasi-adverbial modifier of the predicate's head; in either instance, the modifier's adjacency to the preceding head makes it available to carry the head's \isi{agreement} morphology.  On this view, the literal meaning of an essence predicate's \is{noun!nominal component}nominal component does not combine in a compositional\is{compositionality} way with the literal meaning of the \isi{predicative base}; instead, the \is{noun!nominal component}nominal component has been grammaticalized with a meaning something like that of \ili{English} \emph{inside} in experiencer-based expressions such as \emph{ntykwen}\expo{3} \emph{riq}\expo{24} `s/he got angry inside'; note again that reflexive pronouns have been grammaticalized with much the same function in expressions such as \textit{elle s'est f\^ach\'ee} 'she got angry inside'.  Thus, the \isi{compound predicate hypothesis} situates the expression of subject \isi{agreement} in essence predicates within a larger, independently motivated system in which other compound predicates and verb + \isi{event modifier} combinations also participate in parallel fashion.  
The distributional flexibility of subject \isi{agreement} therefore yields equivocal results.  Both the \isi{possessed-subject hypothesis} and the \isi{compound predicate hypothesis} relate the person/number marker on an essence predicate's \is{noun!nominal component}nominal component to an independent phenomenon in Chatino:  according to the \isi{possessed-subject hypothesis}, the person/number marking on an essence predicate's \is{noun!nominal component}nominal component can be identified with a noun's inflection for the person and number of an inalienable possessor\is{inalienable possession!possessor}; by contrast, the \isi{compound predicate hypothesis} entails that an essence predicate's \is{noun!nominal component}nominal component reflects a more general pattern in which the person and number of a predicate's subject are marked on a nonsubject constituent\textemdash on the second member of a compound predicate, on an \isi{event modifier}, or on a quasi-adverbial essence word.  Given that both of these patterns of person/number marking must in any event be countenanced in an adequate grammar of Chatino, it is not clear that the present criterion provides compelling evidence for choosing either of the two hypotheses over the other.

\section{Essence predicates:  A formal interpretation}


Superficially, the properties of essence predicates seem ambiguous in their implications for a formal analysis.  The essence predicate in (\ref{ex:CruzStump:41}) on the one hand resembles the verb-subject construction in (\ref{ex:CruzStump:42}):  in both cases, the predicative word (in boldface) is inflected for aspect/mood and the nominal element (in italics) is inflected for person and number.   At the same time, the essence predicate in (\ref{ex:CruzStump:41}) resembles the \is{compounding!compound verb}compound verb in (\ref{ex:CruzStump:43}):  here, too, the boldface predicative word is inflected for aspect/mood and the nominal element is inflected for person and number.  Finally, the essence predicate in (\ref{ex:CruzStump:41}) resembles the verb + \isi{event modifier} combination in (\ref{ex:CruzStump:44}), where the predicative word is again inflected for aspect/mood and the \isi{event modifier}, for person and number.

\begin{exe}

	\ex\label{ex:CruzStump:41}
	\gll
	 {\textbf{{Ndi}\expo{4}}} {\textit{{riq}\expo{2}}} {{no}\expo{4}} {{kyqyu}\expo{1}} {{kwa}\expo{3}.}\\
	 {thirsty.\textsc{cpl}} {essence.\textsc{3sg}} {one} {male} {that}\\
	\glt `That guy was thirsty.'\\



	\ex\label{ex:CruzStump:42}
	\gll {\textbf{{Nkya}\expo{24}}} {\textit{{sti}\expo{1}}} {{Xwa}\expo{3}} {{kwa}\expo{3}.}\\
	 {go.baseward.\textsc{cpl}} {father.\textsc{3sg}} {Juan} {that}\\
	\glt `Juan's father left.'\\



	\ex\label{ex:CruzStump:43}
	\gll {\textbf{{Ykwiq}\expo{4}}} {\textit{{sla}\expo{3}}} {{no}\expo{4}} {{qan}\expo{1}} {{kwa}\expo{3}.}\\
	 {speak.\textsc{cpl}} {tiredness.\textsc{3sg}} {one} {female} {that}\\
	\glt `That woman dreamt.'\\



	\ex\label{ex:CruzStump:44}
	\gll {\textbf{{Ykwiq}\expo{4}}} {\textit{{ti}\expo{4}}} {{no}\expo{4}} {{kyqyu}\expo{1}} {{kwa}\expo{3}.}\\
	 {speak.\textsc{cpl}} {\textsc{ev.mod.3sg}} {one} {male} {that}\\
	\glt `That guy just spoke.'\\

\end{exe}


According to the \isi{possessed-subject hypothesis}, an essence predicate is a predicate-subject construction comparable to that of \marginpar{Sacha: r??f??rences fausses, j'ai modifi?? pour correspondre aux exs ci-dessus} (\ref{ex:CruzStump:42}):  its nominal element (\textit{riq}\expo{2} `essence' in (\ref{ex:CruzStump:41})) is a subject, and as in (\ref{ex:CruzStump:42}), the inflectional marking on the subject expresses the person and number of an inalienable possessor\is{inalienable possession!possessor}; this entails that \emph{no}\expo{4} \emph{kyqyu}\expo{1}  \emph{kwa}\expo{3} `that guy' is not the subject of (\ref{ex:CruzStump:41}), but instead denotes an inalienable possessor\is{inalienable possession!possessor}, like \emph{Xwa}\expo{3} `Juan' in (\ref{ex:CruzStump:42}).

According to the \isi{compound predicate hypothesis}, an essence predicate is a compound predicate comparable to those of (\ref{ex:CruzStump:43}) and (\ref{ex:CruzStump:44}).  In a compound predicate, the second element is not a subject, but is either a complement or a modifier of the predicate (as in (\ref{ex:CruzStump:43}) and (\ref{ex:CruzStump:44}) respectively), so that its inflection encodes the person and number of the predicate's subject rather than that of an inalienable possessor\is{inalienable possession!possessor}.  
This suggests that through grammaticalization, an essence predicate's \is{noun!nominal component}nominal component has come to serve a quasi-adverbial function, ordinarily causing the predicate to refer to the psychological or physical state of its subject's referent.

In section 3, we examined four characteristics of essence predicates:  their structural variety, their external syntax relative to event modifiers\is{event!event modifier}, their general lack of semantic \isi{compositionality}, and their possible relation to the distributional flexibility of Chatino subject-\isi{agreement} marking.  As we have seen, these four criteria do not decisively favor either of the two hypotheses under consideration.  The criterion of external syntax seems to favor the \isi{possessed-subject hypothesis}; the criteria of structural variety and lack of \isi{compositionality} seem to favor the \isi{compound predicate hypothesis}; and the criterion of the distributional flexibility of subject \isi{agreement} marking does not clearly favor either hypothesis.

It is clear from this impasse that a third hypothesis is necessary to account for the properties of essence predicates.  We therefore suggest the following account.
\begin{itemize}
\item We regard an essence predicate as a lexeme whose \isi{predicative base} and \is{noun!nominal component}nominal component act as separate constituents in syntax.\footnote{There is abundant evidence that 
lexemes may inflect periphrastically\is{periphrasis}; for discussion, see \cite{BoerjarsEtAl}, \cite{Sadler:Spencer01}, \cite{AckermanStump}, \cite{AckermanStumpWebelhuth}, \cite{SurreyPeriphrasis}, \cite{BonamiSamvelian}, and \cite{BonamiPeriphrasis}.  In many languages, a lexeme's paradigm may include both synthetic and periphrastic\is{periphrasis} realizations; that is, \isi{periphrasis} is used for the realization of particular morphosyntactic property sets (as in \ili{Latin}, where periphrastic\is{periphrasis} realizations occupy the perfective passive cells in paradigms whose other cells are realized synthetically).  An essence predicate, however, is uniformly periphrastic\is{periphrasis} in its realization; that is, the incidence of \isi{periphrasis} is not restricted to the realization of particular morphosyntactic property sets, but is characteristic of all of an essence predicate's realizations.  This view of essence predicates as lexemes whose realization is invariably periphrastic\is{periphrasis} recalls the similar conception of Persian complex predicates proposed by \cite{BonamiSamvelianComplexPreds}.}  They are different from compound predicates in Chatino:  their parts may be interrupted by event modifiers\is{event!event modifier}, while those of a compound predicate in general cannot. 
 \item We propose that every Chatino predicate has an ``inflectional domain'' to which its \isi{agreement} morphology is confined.  A predicate is ordinarily its own domain; this is true whether the predicate is simplex or compound.  In the former case, aspect/mood and \isi{agreement} are expressed cumulatively.  In the latter case, inflection is regulated by a principle of distributed exponence which we here equate with the \isi{Compound Inflection Criterion}; according to this principle, a compound predicate's inflection is ordinarily bipartite, with aspect/mood marked on its head and \isi{agreement} marked on nonhead component.  (The details of this principle are complicated by deviations from this ordinarily bipartite pattern, as e.g. in the first-person singular forms in Tables \ref{tab:CruzStump:sjq-12} and \ref{tab:CruzStump:sjq-24}; see \cite{CruzWoodbury2013} for discussion of the range of such deviations.)
\item Certain kinds of syntactic combinations also constitute inflectional domains.  If a simplex verb is modified by an adjacent \isi{event modifier}, these two words compose an inflectional domain, whose inflection again involves the distributed exponence prescribed by the \isi{Compound Inflection Criterion}.\footnote{There are also cases in which the combination of a compound predicate with an adjacent \isi{event modifier} constitutes an inflectional domain in which subject \isi{agreement} is marked both on the compound predicate's non-head component and on the \isi{event modifier}; sentence (\ref{ex:CruzStump:35}) is an example of this sort.}  In addition, an essence predicate is a lexeme whose periphrastic\is{periphrasis} realization functions as an inflectional domain, exhibiting the same pattern of distributed exponence.  In particular, its person/number marking is situated on its \is{noun!nominal component}nominal component and is an expression of subject \isi{agreement} rather than \isi{inalienable possession}.
\item An essence predicate's \is{noun!nominal component}nominal component is not a subject, but has been grammaticalized as a quasi-adverbial formative ordinarily serving to express the psychological or physical state of the referent of the essence predicate's subject.
\item The structural variety of essence predicates and their semantic idiosyncrasy reflect their status as lexemes listed in the lexicon.
\item In most instances, the parts of an essence predicate are recognizably associated with independent lexemes, but this is not invariably the case.  In \ili{English}, the derivational suffix \textit{-ize} may transparently relate a verb with a causative or inchoative meaning to a nominal or adjectival stem (\textit{magnet}$\rightarrow$ \textit{magnetize}, \textit{popular} $\rightarrow$ \textit{popularize}) but may also simply mark a causative or inchoative verb that is not synchronically related to any nominal or adjectival base (\emph{baptize}, \emph{ostracize},
\textit{recognize}). Analogously, a Chatino essence predicate denoting a psychological or physical state may be transparently related (in form if not in content) to an independent predicate (as in (\ref{ex:CruzStump:45})) but there are also "intrinsic" essence predicates that are synchronically unrelated to any independent predicate (as in (\ref{ex:CruzStump:46})).  The observed parallelism of reflexive verbs\is{verb!reflexive} is again telling: \emph{demander} `ask' $\rightarrow$ \emph{se} \emph{demander} `wonder', but \emph{se} \emph{moquer} `mock' (*\emph{moquer}).
\end{itemize}

\begin{exe}

	\ex \label{ex:CruzStump:45} {{skwa}\expo{3} {riq}\expo{2} \hspace*{3em}{($\leftarrow$ \emph{skwa}\expo{3} `s/he lay elevated')}\\ `s/he was fed up'} \\


	\ex \label{ex:CruzStump:46} {{ndi}\expo{32} {riq}\expo{2} \hspace*{3em}{(*\emph{ndi}\expo{32} without \emph{riq}\expo{2})\\
	`s/he was thirsty'} }\\

\end{exe}

Other \ili{Oto-Manguean} languages possess essence predicates exhibiting both similarities to and differences from those of SJQ Chatino; future work on these similarities and differences will likely shed additional light on the properties of this distinctive class of predicates.

\is{predicate!essence predicate|)}

\il{Chatino|)}
\il{Chatino!San Juan Quiahije variety|)}

\section*{Acknowledgements}
\largerpage
We wish to thank Tony Woodbury and Ryan Sullivant  for discussions that contributed substantially to the realization of this paper.  Thanks also to Olivier Bonami for several helpful suggestions.

{\sloppy
    \printbibliography[heading=subbibliography,notkeyword=this]
}

\end{document}
