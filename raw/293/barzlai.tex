\documentclass[output=paper]{langscibook}
\ChapterDOI{10.5281/zenodo.5578820}

%This is where you put the authors and their affiliations
\author{Maya L. Barzilai\affiliation{Georgetown University} and Nubantood Khalil\affiliation{Nubian Language Society}}

\title{Morphologically conditioned phonological variation in Nobiin}
\abstract{This paper provides an account of a morphophonological phenomenon of Nobiin (Nile-Nubian; Egypt, Sudan), in which assimilation occurs at two morpheme boundaries, but the assimilation patterns are different for each of the two affixes. Additionally, there are two possible surface forms with one affix but not with the other. We capture these facts in an analysis based on cophonologies, in which each affix is associated with its own ranking of the same phonological constraints.}

\begin{document}
\multicolsep=-.25\baselineskip
\maketitle

\section{Introduction}

This paper presents original data from Nobiin, an underdocumented Nile-Nubian language (North Eastern Sudanic, Nilo-Saharan) indigenous to southern Egypt and northern Sudan. We show that progressive forms (\ref{progex}) and accusative forms (\ref{accex}) in Nobiin both have consecutive consonants in the input, but the phonological interactions that produce the attested output forms of these sequences are different for each of the two affixes.



\ea

\begin{multicols}{2}

\begin{xlist}
\ex Progressive marking: \label{progex}\\
 {}[ammirin] $\sim$ {}[aamirin]\\
/ag-mirin/\\
\textsc{prog}-run.\textsc{3sg.prs}\\
`is running'\\
%\bg. {}[a:mirin]\\
%/ag-mirin/\\
%\textsc{prog}-run.\textsc{3sg.prs}\\
%`is running'\\
\ex Accusative marking: \label{accex}\\
{}[egetta] (*{}[egeeta])\\
/eged-ka/\\
sheep-\textsc{acc}\\
`sheep (acc.)'\\
%\bg. *{}[ege:ta]\\
%/eged-ka/\\
%sheep-\textsc{acc}\\
%`sheep (acc.)'\\

\end{xlist}

\end{multicols}

\z

We propose that a Cophonology Theory analysis accounts both for the presence of variation with progressive prefixation (\ref{progex}) and the absence of variation with accusative suffixation (\ref{accex}), and also for the different assimilation patterns attested at the different morpheme boundaries.

In the following section, we provide background on Nobiin, and specifically on the aspects of Nobiin phonology relevant to the phenomena described and analyzed here. In \S \ref{data} we present the empirical facts and in \S \ref{analysis} we provide an analysis of the data. In \S \ref{discussion} we discuss why our proposed analysis is superior to other potential analyses, and also point out areas for future research. Finally, in \S \ref{conclusion}, we conclude.
 
\section{Background} \label{background}
\subsection{Nobiin}
Nobiin (ISO 639-3 fia; also called Mahas) is a Nile-Nubian language of the North Eastern Sudanic branch of the Nilo-Saharan language family. The language is native to southern Egypt and northern Sudan in the area along the Nile River, known as Nubia. Geopolitical circumstances over the past several centuries have led to the displacement of Nobiin speakers; of the approximately 545,000 current speakers, about 45,000 live outside of Nubia in diaspora communities. The language is under-documented and is considered endangered.

\subsection{Nobiin phonology}\largerpage
Table \ref{inventory} shows Nobiin's consonant inventory. There are plosives at the bilabial, alveolar, and velar places of articulation, with the voiceless bilabial plosive missing from the inventory. This lack of /p/ is a property common to many languages spoken in northeastern Africa and the Arabian peninsula. There is one underlying affricate and voiceless fricatives at the bilabial, alveolar, and palatal places of articulation. The inventory also contains a series of four nasal stops, at the bilabial, alveolar, palatal, and velar places of articulation, respectively. The Nobiin rhotic can surface as a tap or a trill, and is listed as a liquid along with the lateral as these two sounds pattern together in the language; neither appears word-initially in Nobiin, and they are both variably epenthesized at the beginning of some prosodic phrases \citep{barzilaispips}. 

\begin{table}[h]
\centering
\begin{tabular}{l c c c c}
\lsptoprule
& Bilabial & Alveolar & Palatal & Velar\\
\midrule
Stop & b & t d & & k g  \\
Affricate & & d͡ʒ & & \\
Fricative & f & s & ʃ & \\
Nasal & m & n & ɲ & ŋ \\
Liquid & & l r & & \\
\lspbottomrule
\end{tabular}
\caption{Nobiin consonant inventory\label{inventory}}
\end{table}

Consonants in Nobiin can be underlyingly singleton or geminate. As described throughout the paper, geminates can also be derived by phonological processes across morpheme boundaries.{\interfootnotelinepenalty=10000\footnote{Phonetic work has shown that underlying and derived geminates have the same duration on average, both lasting about twice as long as singleton consonants \citep{nobiingeminates}. Based on this finding, long consonants derived at morpheme boundaries are analyzed here as geminates.}} Word-medial consonant clusters containing either an obstruent and a liquid or a plosive and a fricative are also permitted in Nobiin, though all of these can be analyzed as heterosyllabic. Nobiin contrasts between H and L tones,\footnote{High tones are marked with an acute accent mark in transcriptions throughout; low tones are unmarked.} and also shows contrastive vowel length.
 
\section{The data} \label{data}
The following sections show the empirical data on which our analysis is based. The data comes from a Nobiin speaker currently living in the Washington, D.C. area. The data is supplemented by the native-speaker judgements of the second author and checked against \citet{werner} where possible.

\subsection{Assimilation in progressive marking}\label{progmarking}
Progressive forms of verbs in Nobiin are formed with the /ag-/ prefix (\ref{prognoass}).

%
%\subsubsection{Assimilation in Progressive Forms}
%\begin{itemize}
%\item The prefix /ag-/ marks the progressive (\ref{prognoass}).
\ea  \label{prognoass}

\begin{multicols}{2}

\begin{xlist}
\ex {}[agárɲin]\\
/ag-árɲin/\\
\textsc{prog}-sneeze.\textsc{3sg.npst}\\
`is sneezing'\\

\ex {}[agg\'oɲill]\\
/ag-g\'oɲill]\\
\textsc{prog}-build.\textsc{3sg.npst}\\
`is building'\\

\end{xlist}

\end{multicols}

\z

This prefixation triggers assimilation at the morpheme boundary when the verb begins with a consonant. This assimilation results in a derived geminate as in the forms in (\ref{progass}).\largerpage[2]

\ea \label{progass}
\begin{multicols}{2}\raggedcolumns
\ea
{}[ammirin]\\
/ag-mirin/\\
\textsc{prog}-run.\textsc{3sg.prs}\\
`is running'\\\columnbreak
\ex {}[akkabin]\\
/ag-kabin/\\
\textsc{prog}-eat.\textsc{3sg.prs}\\
`is eating'\\\columnbreak
%
\ex {}[assokk\'ilokkom]\\
/ag-sokk\'ilokkom/\\
\textsc{prog}-lift.\textsc{2pl.prs}\\
`are lifting'\\\columnbreak
\ex {}[abbe\'esir]\\
/ag-be\'esir/\\
\textsc{prog}-comb.\textsc{1sg.prs}\\
`am combing'\\
\z
\end{multicols}
\z

These geminates are derived via complete regressive assimilation, surfacing as a geminate version of the stem-initial consonant.

\subsection{Variable forms of the progressive}

The surface realization of progressive forms described in \S \ref{progmarking} is variable. Geminated forms derived from progressive affixation are in free variation with forms comprised of a long vowel and a singleton consonant at the same morpheme boundary. Example (\ref{progcomplength}) repeats the forms in (\ref{progass}), including for each example the other possible surface form.
%

\ea \label{progcomplength}
\begin{multicols}{2}
\begin{xlist}
\ex {}[ammirin] $\sim$ [aamirin]\\
/ag-mirin/\\
\textsc{prog}-run.\textsc{3sg.prs}\\
`am/are running'\\
\ex {}[akkabin] $\sim$ [aakabin]\\
/ag-kabin/\\
\textsc{prog}-eat.\textsc{3sg.prs}\\
`am/are eating'\\
\ex {}[assokk\'ilokkom] $\sim$ [aasokkílokkom]\\
/ag-sokk\'ilokkom/\\
\textsc{prog}-lift.\textsc{2pl.prs}\\
`am/are lifting'\\
\ex {}[abbe\'esir] $\sim$ [aabe\'esir]\\
/ag-be\'esir/\\
\textsc{prog}-comb.\textsc{1sg.prs}\\
`am/are combing'\\
\end{xlist}
\end{multicols}
\z

\subsection{Assimilation in accusative marking}
The suffix /-ka/ marks accusative case in Nobiin (\ref{accnoass}) \citep{khalil}. When the accusative suffix is added to nouns ending with a consonant,\footnote{The accusative marker may surface as either [-ka] or [-ga] when affixed to a word-final vowel, in an alternation possibly based on the tonal patterns of the noun. We leave a full description of accusative marking on vowel-final nouns to future work.} 
the final consonant of the root sometimes assimilates to the initial consonant of the accusative suffix, depending on the identity of that consonant (\ref{accass}--\ref{accnas}). This assimilation results in a derived geminate that surfaces at the morpheme boundary. The system of assimilation seen at this morpheme boundary is more complex than the patterns described above in \S \ref{progmarking}; whether assimilation occurs, and the type of assimilation that occurs, is dependent upon the identity of the root-final consonant.\pagebreak


\ea \label{accnoass}
\begin{multicols}{3}
\ea {}[fuulka]\\
/fuul-ka/\\
fava\_beans-\textsc{acc}\\
`fava beans (acc.)'\\
\ex {}[kad\'isska]\\
/kad\'iss-ka/\\
cat-\textsc{acc}\\
`cat (acc.)'\\
\ex {}[gisirka]\\
/gisir-ka/\\
bone-\textsc{acc}\\
`bone (acc.)'\\
\z
\end{multicols}

\ex \label{accass}
\begin{multicols}{3}
\ea {}[egetta]\\ \label{egetta}
/eged-ka/\\
sheep-\textsc{acc}\\
`sheep (acc.)'\\
\ex {}[dit͡ʃt͡ʃa]\\ \label{ditstsa}
/di\t{dʒ}-ka/\\
five-\textsc{acc}\\
`five (acc.)'\\
\ex {}[fákka]\\ \label{fakka}
/fág-ka/\\
goat-\textsc{acc}\\
`goat (acc.)'\\
\z
\end{multicols}
\ex \label{accnas}

\begin{multicols}{3}
\ea {}[simsimka]\\ \label{simsimka}
/simsim-ka/\\
sesame-\textsc{acc}\\
`sesame (acc.)'\\
\ex {}[ámáŋŋa]\\ \label{amanga}
/ámán-ka/\\
water-\textsc{acc}\\
`water (acc.)'\\
\ex {}[soriŋŋa]\\ \label{soringa}
/soriŋ-ka/\\
nose-\textsc{acc}\\
`nose (acc.)'\\
\z
\end{multicols}
\z

If the root-final consonant is a stop or affricate, the place of articulation of the derived geminate is that of the root-final consonant; this occurs via progressive assimilation (\ref{egetta}--\ref{ditstsa}) and applies vacuously in cases with a root-final velar consonant (\ref{fakka}). The geminates surfacing in these cases are always voiceless, which is unsurprising given the cross-linguistic tendency to avoid voiced geminates \citep{ohala}.\footnote{While voiced geminates are avoided in this derived environment, the language does not ban voiced geminates entirely, as evidenced by the assimilation at the progressive boundary. Additionally, there are some instances of underlying geminate /dd/ in monomorphemes, which surface as voiced [dd], as in /hiddo/ `where' and /oddi/ `illness'. There appear to be relatively few of these words across the lexicon and at least one, /eddi/ `hand', is a borrowing from Arabic.} 

Assimilation at the accusative morpheme boundary can also be bidirectional, when the final consonant of the noun is a nasal. Whereas final /m/ does not trigger assimilation (\ref{simsimka}), alveolar and velar nasals trigger both progressive manner assimilation and regressive place assimilation, resulting in a surface geminate [ŋŋ] at the morpheme boundary. (\ref{amanga}--\ref{soringa}).\footnote{We believe that the surface accusative forms of words that end with the palatal nasal /ɲ/ before a velar also trigger place assimilation for a surface geminate [ŋŋ], but further phonetic analysis is necessary to confirm this.}

%Root-final nasals other than /m/ (see (\ref{accnoass}c)) trigger progressive manner assimilation and regressive place assimilation. This results in a geminate [ŋŋ] at the morpheme boundary (\ref{accass}d). If the root-final consonant is not a nasal, the place of articulation of the derived geminate is that of the root-final consonant; this occurs via progressive assimilation (\ref{accass}b-\ref{accass}c) unless the root-final consonant is velar and there assimilation of the suffix consonant is not required (\ref{accass}a). Root-final oral consonants always result in a voiceless geminate, via regressive voicing assimilation. This generalization that oral geminates surface as voiceless is in keeping with the cross-linguistic tendency to avoid voiced geminates \citep{ohala}.\footnote{There are, however, instances of underlying geminate /dd/ in Nobiin, as in /hiddo/ `where', /eddi/ `hand', and /oddi/ `illness'.} Interestingly, the result of these patterns is that accusative assimilation can be bidirectional, such that there is progressive place of articulation assimilation and regressive voicing assimilation.


Crucially, accusative forms exhibiting a long vowel and singleton consonant, similar to those in (\ref{progcomplength}), are illicit with accusative nouns, as revealed in (\ref{badacc}).

\ea \label{badacc}
\begin{multicols}{2}
\begin{xlist}
\ex {}[fákka] (*{}[fáaka])\\
/fág-ka/\\
goat-\textsc{acc}\\
`goat (acc.)'\\
\ex {}[egetta] (*{}[egeeta])\\
/eged-ka/\\
sheep-\textsc{acc}\\
`sheep (acc.)'\\
\end{xlist}
\end{multicols}
\z

The fact that two forms are possible in (\ref{progcomplength}), but only one in (\ref{badacc}), shows that the variation between geminate consonants and a sequence of a long vowel followed by a single consonant is morphophonologically constrained.

\subsection{Assimilation at morpheme boundaries}
Table \ref{asssumm} summarizes the different assimilation patterns that surface at the progressive and accusative morpheme boundaries, as described above.

\begin{table}
\centering
\begin{tabular}{c c c}
\lsptoprule
Consonant & Progressive & Accusative\\
C & /ag-C/ & /C-ka/\\
\midrule
/b/ & [abb] %$\sim$ [a:b] 
& [ppa] \\
\midrule
/d/ & [add] %$\sim$ [a:d] 
& \multirow{2}{2em}{[tta]}  \\
/t/ &[att] %$\sim$ [a:t] 
& \\
\midrule
/g/ & [agg] %$\sim$ [a:g] 
& \multirow{2}{2em}{[kka]} \\
/k/ & [akk] %$\sim$ [a:k] 
& \\
\midrule
/d͡ʒ/ & [ad͡ʒd͡ʒ] %$\sim$ [a:d͡ʒ]
& [t͡ʃt͡ʃa]\\
\midrule
/f/ & [aff]\footnote{[agf] is marginally acceptable here.} %$\sim$ [a:f]  
& \cellcolor{lightgray} [fka] \\
/s/ & [ass] %$\sim$ [a:s] 
& \cellcolor{lightgray} [ska] \\ 
/ʃ/ & [aʃʃ] %$\sim$ [a:ʃ]
& \cellcolor{lightgray} [ʃka] \\
\midrule
/m/ & [amm] %$\sim$ [a:m]
& \cellcolor{lightgray} [mka] \\
\midrule
/n/ & [ann] %$\sim$ [a:n] 
& \multirow{3}{2em}{[ŋŋa]} \\
/ɲ/ & (?) & \\
/ŋ/ & -- & \\
\midrule
/l/ & -- & \cellcolor{lightgray} [lka]\\
/r/ & -- & \cellcolor{lightgray} [rka]\\
%/h/ & (?) & \cellcolor{lightgray} [hka] \\
\lspbottomrule
\end{tabular}
\caption{Summary of assimilation patterns. Gray cells: no assimilation in surface form. --: consonant not possible word-initially in Nobiin. (?): consonant not possible word-initially for verbs in Nobiin (confirmed by native-speaker intuitions of the second author).\label{asssumm}}
\end{table}

Table \ref{asssumm} shows that geminates at the boundary between a progressive suffix and a verb have all of the same features as the initial consonant of the verb. The final /g/ of the progressive suffix assimilates entirely to the following consonant. On the other hand, assimilation at the boundary between a noun and an accusative marker, shown in the Accusative column, is slightly more complex. When the final consonant of the noun is a liquid, a fricative, or /m/, it surfaces faithfully with a following /k/, yielding a heterosyllabic surface cluster. When the final consonant of the noun is a nasal other than /m/, accusative suffixation results in a geminate velar nasal, [ŋŋ]. Finally, when the final consonant of the noun is a stop or an affricate, accusative suffixation results in a surface voiceless geminate that has the same place of articulation as the underlying root-final consonant.


\section{Analysis} \label{analysis}
In this section we present our analysis, which accounts for the variation in progressive forms by arguing that surface forms must be faithful to their input forms with respect to their moraic structure. We model the differences in assimilation patterns in the two affixation processes in a cophonologies framework, showing that assigning different constraint rankings to the different affixes yields the attested output forms. 

\subsection{Mora faithfulness}
The variation shown in (\ref{progcomplength}) is in line with historically and typologically common patterns of compensatory lengthening \citep{hayes89, quantity},  in which the deletion or shortening of a surface segment is counteracted by a lengthening of an adjacent or local segment. It is interesting to note that equivalent sequences (i.e. combinations of short vowel with geminate or long vowel with singleton) are contrastive elsewhere in the language, as evidenced by the possessive pronouns in (\ref{pronounalt}). This further strengthens the claim, suggested above, that the variable compensatory lengthening processes seen in Nobiin progressive formation are specific to that morphosyntactic context and do not represent a general phonological process in the language. 

\ea \label{pronounalt}
\begin{multicols}{2}
\begin{xlist}
\ex {}[uun\'\i]\\
/uu-n\'\i/\\
\textsc{1pl-poss}\\
`our'\\
\ex {}[unn\'i]\\
/ur-n\'i/\\
\textsc{2pl-poss}\\
`your (pl)'\\
\end{xlist}
\end{multicols}
\z


The variable compensatory lengthening in the forms in (\ref{progcomplength}) is supported by a moraic analysis of geminates \citep{mccarthy, scheinsteriade}. The patterning seen here is evidence that codas are moraic in Nobiin, and that long vowels are bimoraic. Intervocalic geminates are syllabified as the moraic coda of the first syllable and as the (non-moraic) onset of the following syllable. Therefore, the progressive marker /ag-/ contains two moras underlyingly, one from its nucleic vowel and the second from its coda. These moras may surface either as a vowel followed by a geminated consonant (\ref{agsgem}), or as a long vowel (\ref{ags}c); in either case, the output is faithful to the input with respect to its moraic structure. The moraic structure of the input, along with the two possible output forms, are shown in the example in (\ref{ags}).

\ea \label{ags}
\begin{xlist}
\begin{multicols}{2}\raggedcolumns
\ex  /ag- kabin/ \\
\begin{tikzpicture}[sibling
distance=3pt, level distance=20pt]
\tikzset{level distance=40pt}
\Tree
[.σ
[.μ a ] [.μ g ]]
\begin{scope}[xshift=0.5in]
\Tree
[.σ
[k ]  [.μ a ]]
\end{scope}
\begin{scope}[xshift=1.05in]
\Tree
[.σ
[b ]  [.μ i ] [.μ n ]]
\end{scope}
\end{tikzpicture}\columnbreak

\ex  {}[akkabin]\\ \label{agsgem}
\begin{tikzpicture}[sibling
distance=3pt, level distance=20pt]
\tikzset{level distance=40pt}
\begin{scope}
\Tree
[.σ [.μ a ] [.μ \node(l){kk};] ]
\end{scope}
\begin{scope}[xshift=0.5in]
\Tree
[.\node(O){σ}; [.μ  [. a ]]]
%[.R [.μ a ] [ l ] ] ]
\draw[] (O.south) -- (l.north);
\end{scope}
\begin{scope}[xshift=0.9in]
\Tree
[.σ
[b ]  [.μ i ] [.μ n ]]
\end{scope}
\end{tikzpicture}
\end{multicols}
\end{xlist}
\begin{xlist}

\exi{c.} {}[aakabin]\\
\begin{tikzpicture}[sibling
distance=3pt, level distance=20pt]
\tikzset{level distance=40pt}
\begin{scope}
\Tree
[.σ
[.\node(mora1){μ}; ]  [.μ \node(a){aa}; ]]
\draw[] (mora1.south) -- (a.north);
\end{scope}
\begin{scope}[xshift=0.5in]
\Tree
[.σ
[k ]  [.μ a ]]
\end{scope}
\begin{scope}[xshift=1in]
\Tree
[.σ
[b ]  [.μ i ] [.μ n ]]
\end{scope}
\end{tikzpicture}

\end{xlist}

\z

The faithfulness constraint in (\ref{faithmu}), adapted from \citet{davissinhala}, militates for the mora faithfulness that is maintained in both possible output forms in (\ref{ags}).\footnote{There are other documented phonological processes in Nobiin that have been analyzed as preserving input prosodic structure in the output. In a phonological process independent from the assimilation analyzed here, liquids are epenthesized before vowel-initial words, even when the preceding word ends in a consonant and therefore epenthesis results in a consonant cluster. This, similarly to the phenomenon described here, has been analyzed as prosodic faithfulness: epenthetic liquids ensure that syllable boundaries coincide with the prosodic boundaries between morphosyntactic constituents \citep{barzilaispips}.} This constraint interacts with the faithfulness constraint in (\ref{ident}).\largerpage

\ea

\begin{xlist}

\ex \label{faithmu} \textsc{Max}\textsubscript{μ}: Every mora in the input must have a corresponding mora in the output.
\ex \label{ident} \textsc{Ident}: Segments in the output must have the same features as the corresponding segments in the input.

\end{xlist}

\z

\begin{sloppypar}
We model the variation in surface progressive forms with partially-ordered constraints (POC, \citealt{coetzee2006, anttila}), with Max\textsubscript{μ} (\ref{faithmu}) outranking  \textsc{Max\textsubscript{Seg}} and \textsc{Ident} (\ref{ident}), which are in turn unranked with respect to each other.\footnote{This synchronic variation could also be modeled with weighted constraints, as in, e.g., Harmonic Grammar \citep{HG}. An analysis of this sort could also account for the relative frequencies of the two possible output forms. Given that we do not have the data on these relative frequencies, we leave this analysis to future work.}
\end{sloppypar}

\begin{table}
\ShadingOn
\begin{tableau}{c|c:c}
\inp{\ips{ag + kabin}}   \const*{\textsc{Max\textsubscript{μ}}}  \const*{\textsc{Max\textsubscript{Seg}}}         \const{Ident}
\cand{a.ka.bin}         \vio{*!}                                    \vio{*}                                         \vio{}
\cand{a.ga.bin}         \vio{*!}                                    \vio{*}                                         \vio{}
\cand[\Optimal]{ak.ka.bin} \vio{}                                   \vio{}                                          \vio{*}
\cand[\Optimal]{aa.ka.bin}  \vio{}                                  \vio{*}                                         \vio{}
\end{tableau}
\caption{Mora faithfulness with POC} \label{POC}
\end{table}


In Table \ref{POC}, candidates a. and b. are both eliminated as a result of a \textsc{Max\textsubscript{μ}} violation; in each one, there are fewer moras in the output than in the input. The two winning candidates in this derivation are both licit, as neither one violates the undominated \textsc{Max\textsubscript{μ}} constraint. In this analysis, candidates c. and d. are evaluated as equally optimal, though we do not at this point make any claims about their relative optimality or about which form surfaces more frequently in Nobiin speech.

\subsection{Cophonologies analysis}
The generalization that variable compensatory lengthening is seen in progressive formation but not in accusative formation can be modeled in Cophonology Theory \citep{anttilaCT, inkelaszollCT}. In this type of analysis, each affix is associated with its own cophonology, which is comprised of a unique ranking of phonological constraints. 

Though compensatory lengthening in accusative contexts would be faithful to the underlying moraic structure, these forms are nonetheless illicit, as shown in (\ref{badacc}). Therefore, in the cophonology associated with the accusative suffix, \textsc{Max\textsubscript{Seg}} is reranked so that it strictly outranks \textsc{Ident}, as opposed to these two constraints remaining unranked with respect to each other as they do in the cophonology of the progressive affix. This reranking rules out candidates similar to candidate d. in Table \ref{POC}.


%\begin{table}[h]
%\centering
%\begin{tabular}{| r || c: c | c |}
%\hline
%/fág + ka/ & \textsc{Max\textsubscript{μ}} & \textsc{Max\textsubscript{Seg}} & \textsc{Ident} \\
%\hline
%\hline
%a. fá.ka & *! & * & \cellcolor{lightgray} \\
%\hline
%b. fá.ga & *! & * & \cellcolor{lightgray} \\
%\hline
%\leftpointright c. fák.ka & & & * \\
%\hline
%d. fáa.ka & & *! & \\
%\hline
%\end{tabular}
%\caption{Accusative Marking} \label{acc}
%\end{table}

\begin{table}
\ShadingOn
\begin{tableau}{c | c | c}
\inp{\ips{f\'ag + ka}}   \const*{\textsc{Max\textsubscript{μ}}}  \const*{\textsc{Max\textsubscript{Seg}}}         \const{Ident}
\cand{f\'a.ka}         \vio{*!}                                    \vio{*}                                         \vio{}
\cand{f\'a.ga}         \vio{*!}                                    \vio{*}                                         \vio{}
\cand[\Optimal]{f\'ak.ka} \vio{}                                   \vio{*}                                          \vio{}
\cand{f\'aa.ka}  \vio{}                                  \vio{*!}                                         \vio{}
\end{tableau}
\caption{Accusative marking} \label{acc}
\end{table}

In Table \ref{acc}, candidates a. and b. are eliminated because they are not faithful to the input moraic structure, like in Table \ref{POC}. Though candidate d. does not incur a similar \textsc{Max\textsubscript{μ}} violation, it loses to candidate c. because it violates \textsc{Max\textsubscript{Seg}}, deleting an input consonant. Therefore, in this derivation, c. is the only harmonic candidate.

\begin{sloppypar}
The constraint rankings associated with the respective cophonologies of the progressive and accusative affixation processes are summarized in (\ref{cophonologies}). By reranking \textsc{Max\textsubscript{Seg}} above \textsc{Ident} in the cophonology associated with the accusative marker, we correctly rule out an output form with compensatory lengthening, such as the one in d. in Table \ref{acc}.
\end{sloppypar}

\ea \label{cophonologies}

Progressive ranking: \textsc{Max\textsubscript{μ}} \guillemotright  \textsc{Max\textsubscript{Seg}} , \textsc{Ident}\\
Accusative ranking: \textsc{Max\textsubscript{μ}} \guillemotright  \textsc{Max\textsubscript{Seg}} \guillemotright \textsc{Ident}
 
\z


\subsection{Assimilation patterns with cophonologies}
Having shown how a cophonologies analysis can account for the differences in variability between the two affixation processes discussed, we now complete the analysis by incorporating constraints that account for the specific assimilation patterns observed. In order to capture the assimilation patterns associated with each affix discussed here, we propose the following additional constraints:

\ea \label{assimconstraints}

\begin{xlist}

\ex \label{assim} \textsc{Assim}: Assign one violation for each surface consonant cluster in which the two segments differ in at least one feature.
\ex \label{identons} \textsc{Ident\textsubscript{L}}: Output segments at the left edge of a morphological constituent must have the same features as the corresponding segments at left edges in the input.
% 
% Onsets in the output must have the same features as the corresponding onsets in the input.\footnote{Following others (e.g., \citealp{beckmandiss}), we assume that syllable structure is available at the point in the derivation at which this and the other constraints apply.} %MLB ask HS for a citation?
\ex \label{stardd} \textsc{*DD}: Assign one violation for every geminate voiced obstruent in the output. 
\ex \label{identnas} \textsc{Ident\textsubscript{Nas}}: Nasal features in the input must have a corresponding nasal feature in the output.
\ex \label{identweak} \textsc{Ident\textsubscript{2}}: Assign one violation for every segment in the output that differs from its corresponding input segment in two or more features.

\end{xlist}

\z

Before showing that these constraints capture the attested data, we first elaborate upon two of the constraint formulations in (\ref{assimconstraints}). First, the \textsc{Assim} constraint in (\ref{assim}) is intended to stand in for two individual constraints that each militate against specific surface clusters. One of these constraints is a markedness constraint that prohibits sequences of consecutive plosives. Another markedness constraint prohibits /TN/ sequences, where T stands for any obstruent and N stands for any nasal. Given that all word-initial consonants in Nobiin are either obstruents or nasals (the liquids /l/ and /r/ are restricted to non-word-initial positions, as shown in Table \ref{asssumm}), these two constraints together militate against any surface [g\#C] clusters at the progressive boundary. These constraints are also compatible with the observation that all morpheme-internal consonant clusters Nobiin are either sequences of a plosive and a fricative (e.g., /makʃa/ `table', /t\'usk\'o/ `three') or of an obstruent and a liquid (e.g., /sarbee/ `finger', /aʃri/ `beautiful').\footnote{There are very rare exceptions to the prohibition of plosive-plosive clusters found in loan words from English, e.g., /aktevist/ `activist'. These claims are based on a corpus of just under 3,000 unique words and about 2,350 sentences in Nobiin.} The data present no evidence for the ranking of these two markedness constraints with respect to each other. Therefore, the constraint \textsc{Assim} here represents these constraints for simplicity and clarity.


Second, the \textsc{Ident\textsubscript{2}} constraints in (\ref{identweak}) militates against output forms with segments that differ from their corresponding input segments in two or more features. Candidates that do not violate this constraint are those in which input segments have undergone assimilation in order to form a surface geminate, but this assimilation has been derived by changing no more than one of the input features. This result is related to the typological generalization that segments participating an agreement relationship, such as one that results in assimilation, are highly similar to each other \citep{rosewalker}. In the \citet{rosewalker} long-distance consonant agreement (LDCA) model, agreement relationships are not established if the segments in questions are too featurally dissimilar to one another. Similarly, in the current analysis, \textsc{Ident\textsubscript{2}} prevents assimilation between consonants in a way that alters too many of the input features of these consonants.

The \textsc{Ident\textsubscript{2}} constraint, and the phonological generalization that it captures, could also be modeled with local conjunction (\citealp{smolenskyLC} et seq.); in this type of analysis, two separate \textsc{Ident} constraints would be locally conjoined such that a simultaneous violation of both would result in a less harmonic candidate than a violation of just one or the other. In a local conjunction analysis, the active conjoined constraint would differ depending on the identity of the consonants in the input cluster. If the input consonant is a stop or affricate, the active conjoined constraint would be [\textsc{Ident\textsubscript{Place}} \& \textsc{Ident\textsubscript{Voi}}], ensuring that the place of articulation of the root consonant, as well as the voicelessness of the suffix consonant, are maintained. On the other hand, if the input consonant stop is a nasal (other than /m/), the active constraint would be [\textsc{Ident\textsubscript{Place}} \& \textsc{Ident\textsubscript{Manner}}], which would result in a surface nasal with the same place of articulation as that of the suffix onset /k/. Crucially, the domain of locality in this analysis would have to be the segment \citep{lubowicz}, such that a given candidate could only incur a violation if both conjoined \textsc{Ident} constraints are violated by the same output segment.

Similarly, a weighted constraints analysis such as Harmonic Grammar \citep{HG}, in which constraints have relative weights as opposed to strict relative weightings, could also capture the cumulative effects of multiple \textsc{Ident} constraints. Given that one of the benefits of Harmonic Grammar accounts is capturing relative frequencies of variable output forms, and since our data set does not include such frequency information, we leave this type of account to future work.

An alternative analysis in which locally conjoined \textsc{Ident} constraints operating on an individual segment correctly derive the attested geminates in accusative forms, or one in which constraints are weighted and not strictly ranked, could be equally successful as the one presented here. Though local conjunction has been argued to make incorrect typological predictions, including over-generation (e.g., see discussion in \citealp{farristrimble}), \citet{shih} has nonetheless shown that in a weighted constraints analysis, conjoined constraints are necessary to capture superadditive gang effects. Therefore, we acknowledge that both a constraint conjunction analysis and a weighted constraints analysis could be as theoretically satisfactory as our current analysis.

Having motivated the constraints in (\ref{assimconstraints}), we turn now to our proposed analysis. Tables \ref{agOTniil} and \ref{agOTdaaril} (pages \pageref{agOTniil} and \pageref{agOTdaaril}) show the derivation of two progressive forms, incorporating the constraints in (\ref{assimconstraints}) to account for the identity of the geminate consonants in the winning forms in e. In both derivations, candidates a. and b. are eliminated for violating \textsc{Max\textsubscript{μ}}, candidate c. is eliminated because of its illicit cluster, and candidate d. is eliminated because of an \textsc{Ident\textsubscript{L}} violation. For each of the two input forms, candidates e. and f. are equally optimal, as they avoid violating the high-ranked \textsc{Assim} and \textsc{Ident\textsubscript{L}} constraints, and maintain the moraic structure of the input form.


\begin{sidewaystable}
\ShadingOn
\begin{tableau}{c : c : c | c : c | c : c : c}
\inp{\ips{ag + n\'\i il}}  \const*{\textsc{Max\textsubscript{μ}}} \const{Assim} \const*{\textsc{Ident\textsubscript{L}}} \const*{\textsc{Max\textsubscript{Seg}}} \const{Ident} \const{*DD} \const*{\textsc{Ident\textsubscript{Nas}}} \const{\textsc{Ident\textsubscript{2}}}
\cand{a.n\'\i il} \vio{*!} \vio{} \vio{} \vio{*} \vio{} \vio{} \vio{} \vio{} 
\cand{a.g\'\i il} \vio{*!} \vio{} \vio{*} \vio{*} \vio{} \vio{} \vio{} \vio{}
\cand{ag.n\'\i il} \vio{} \vio{*!} \vio{} \vio{} \vio{} \vio{} \vio{} \vio{}
\cand{ag.g\'\i il} \vio{} \vio{} \vio{*!} \vio{} \vio{*} \vio{*} \vio{*} \vio{*}
\cand[\Optimal]{an.n\'\i il} \vio{} \vio{} \vio{} \vio{} \vio{*} \vio{} \vio{} \vio{*} 
\cand[\Optimal]{aa.n\'\i il} \vio{} \vio{} \vio{} \vio{*} \vio{} \vio{} \vio{} \vio{} 
\end{tableau}
\caption{Progressive /ag-/ marking in Nobiin (n\'\i il `drink.\textsc{1sg.npst}')} \label{agOTniil}
\end{sidewaystable}

\begin{sidewaystable}
\ShadingOn
\begin{tableau}{c : c : c | c : c | c : c : c}
\inp{\ips{ag + da\'aril}}  \const*{\textsc{Max\textsubscript{μ}}} \const{Assim} \const*{\textsc{Ident\textsubscript{L}}} \const*{\textsc{Max\textsubscript{Seg}}} \const{Ident} \const{*DD} \const*{\textsc{Ident\textsubscript{Nas}}} \const{\textsc{Ident\textsubscript{2}}}
\cand{a.da\'a.ril} \vio{*!} \vio{} \vio{} \vio{} \vio{} \vio{} \vio{} \vio{}
\cand{a.ga\'a.ril} \vio{*!} \vio{} \vio{} \vio{} \vio{} \vio{} \vio{} \vio{}
\cand{ag.da\'a.ril} \vio{} \vio{*!} \vio{} \vio{} \vio{} \vio{} \vio{} \vio{}
\cand{ag.ga\'a.ril} \vio{} \vio{} \vio{*!} \vio{} \vio{} \vio{*} \vio{} \vio{}
\cand[\Optimal]{ad.da\'a.ril} \vio{} \vio{} \vio{} \vio{} \vio{*} \vio{*} \vio{} \vio{}
\cand[\Optimal]{aa.da\'a.ril} \vio{} \vio{} \vio{} \vio{*} \vio{} \vio{} \vio{} \vio{}
\end{tableau}
\caption{Progressive /ag-/ marking in Nobiin (daáril `sit.\textsc{1sg.npst}')} \label{agOTdaaril}
\end{sidewaystable}

By reranking the same set of constraints, we derive the attested output of accusative forms, shown in Tables \ref{kaOTaman} and \ref{kaOTeged} (pages \pageref{kaOTaman} and \pageref{kaOTeged}). Here, *DD and \textsc{Ident\textsubscript{Nas}} ensure the correct assimilation pattern, depending on the identity of the first consonant in the input cluster. Additionally, \textsc{Ident\textsubscript{2}} allows forms that change one feature for each of two input segments, but not those that change two features of a single input segment. This constraint allows for forms in which assimilation is bidirectional, and the surface geminate takes, for instance, its place features from the underlying suffix onset /k/ and its manner features from the underlying root-final nasal. 

\begin{sidewaystable}
\ShadingOn
\begin{tableau}{c : c : c : c : c : c | c : c }
\inp{\ips{aman + ka}}    \const*{\textsc{Max\textsubscript{μ}}}  \const{Assim}  \const{*DD} \const*{\textsc{Ident\textsubscript{Nas}}}  \const*{\textsc{Ident\textsubscript{2}}}   \const*{\textsc{Max\textsubscript{Seg}}}  \const{Ident}  \const*{\textsc{Ident\textsubscript{L}}}
\cand{a.ma.ka} \vio{*!} \vio{} \vio{} \vio{} \vio{} \vio{*} \vio{} \vio{}
\cand{a.ma.na} \vio{*!} \vio{} \vio{} \vio{} \vio{} \vio{*} \vio{} \vio{}
\cand{a.man.ka} \vio{} \vio{*!} \vio{} \vio{} \vio{} \vio{} \vio{} \vio{}
\cand{a.man.na} \vio{} \vio{} \vio{} \vio{} \vio{*!} \vio{} \vio{**} \vio{*} 
\cand{a.mak.ka} \vio{} \vio{} \vio{} \vio{*!} \vio{*} \vio{} \vio{**} \vio{}
\cand{a.mat.ta} \vio{} \vio{} \vio{} \vio{*!} \vio{} \vio{} \vio{**} \vio{*}
\cand[\Optimal]{a.maN.Na} \vio{} \vio{} \vio{} \vio{} \vio{} \vio{} \vio{**} \vio{*}
\cand{a.maa.Na} \vio{} \vio{} \vio{} \vio{} \vio{} \vio{*!} \vio{*} \vio{*}
\end{tableau}
\caption{Accusative /-ka/ marking in Nobiin (aman `water')} \label{kaOTaman}
\end{sidewaystable}

\begin{sidewaystable}
\ShadingOn
\begin{tableau}{c : c : c : c : c : c | c : c }
\inp{\ips{eged + ka}}    \const*{\textsc{Max\textsubscript{μ}}}  \const{Assim}  \const{*DD} \const*{\textsc{Ident\textsubscript{Nas}}}  \const*{\textsc{Ident\textsubscript{2}}}   \const*{\textsc{Max\textsubscript{Seg}}}  \const{Ident}  \const*{\textsc{Ident\textsubscript{L}}}
\cand{e.ge.ka} \vio{*!} \vio{} \vio{} \vio{} \vio{} \vio{*} \vio{} \vio{}
\cand{e.ge.da} \vio{*!} \vio{} \vio{} \vio{} \vio{} \vio{*} \vio{} \vio{*}
\cand{e.ged.ka} \vio{} \vio{*!} \vio{} \vio{} \vio{} \vio{*} \vio{} \vio{}
\cand{e.ged.da} \vio{} \vio{} \vio{*!} \vio{} \vio{*} \vio{} \vio{} \vio{*}
\cand{e.gek.ka} \vio{} \vio{} \vio{} \vio{} \vio{*!} \vio{} \vio{} \vio{}
\cand{e.geg.ga} \vio{} \vio{} \vio{*!} \vio{} \vio{} \vio{} \vio{} \vio{*}
\cand{e.get.ta} \vio{} \vio{} \vio{} \vio{} \vio{} \vio{} \vio{**} \vio{*}
\cand{e.gee.ta} \vio{} \vio{} \vio{} \vio{} \vio{} \vio{*!} \vio{} \vio{}
\end{tableau}
\caption{Accusative /-ka/ marking in Nobiin (eged `sheep')} \label{kaOTeged}
\end{sidewaystable}


In addition to deriving the correct assimilation patterns, the difference in the rankings between Tables \ref{agOTniil}--\ref{kaOTeged} also captures the difference in variability: there are two possible output forms in progressive /ag-/ affixation, but only one possible form for each root when the accusative /-ka/ suffix is added. 

The complete constraint rankings associated with the cophonologies of each of the affixes discussed here are summarized in (\ref{cophonologiesassim}). 

\ea \label{cophonologiesassim}
Progressive ranking: \\ \textsc{Max\textsubscript{μ}}, \textsc{Assim}, \textsc{Ident\textsubscript{L}} \guillemotright  \textsc{Max\textsubscript{Seg}}, \textsc{Ident} \guillemotright *DD, \textsc{Ident\textsubscript{Nas}}, \textsc{Ident\textsubscript{2}}\\
Accusative ranking: \\  \textsc{Max\textsubscript{μ}}, \textsc{Assim}, *DD, \textsc{Ident\textsubscript{Nas}}, \textsc{Ident\textsubscript{2}}, \textsc{Max\textsubscript{Seg}} \guillemotright \textsc{Ident}, \textsc{Ident\textsubscript{L}}
\z

%
\section{Discussion} \label{discussion}
In this section we show how our proposed cophonologies analysis of the empirical data is preferable to an indexed constraint analysis. Second, we discuss the prefix/suffix asymmetry in our data and its implications, and finally we suggest directions for future work on Nobiin morphophonology.

\subsection{Indexed constraints analysis}
An alternative analysis to the one we present here is one that uses indexed constraints \citep{beckman1995, beckman1997}. Under this type of analysis, constraints refer to the morphological properties of the input and not just its surface phonological properties. In some views of phonological and morphophonological theory, phonological derivations should not directly refer to morphological structure, even if morphology and phonology can be shown empirically to interact. Regardless of this theoretical debate, we show here that even if an indexed constraints analysis were preferred, it could not account for the Nobiin assimilation data. 

In an indexed constraints analysis of the data presented here, an indexed constraint such as the high-ranking one in (\ref{ICex}) could be used in an analysis of the Nobiin data presented here. Candidates would incur a violation of this constraint only if it is a root segment that is not faithful to its input. By ranking this indexed constraint above a non-indexed faithfulness constraint, at least some of the features of the root consonant in both assimilation contexts would be preserved. The variation in surface forms of the progressive would be as straightforwardly accounted for in an indexed constraints analysis as it is in the present cophonologies analysis. 

\ea \label{ICex}
\textsc{Ident\textsubscript{root}} \guillemotright \textsc{Ident}
\z


However, the patterns of assimilation in Nobiin accusative formation would lead to a ranking paradox in this type of analysis. When the noun ends in a nasal (\ref{paradox1}), the place of articulation of this root consonant changes, but not its voicing or manner. On the other hand, when the nouns ends in a voiced stop or an affricate (\ref{paradox2}), the voicing of the root changes but not its place of articulation. Each assimilation pattern would require a different relative ranking of \textsc{Ident-Voi\textsubscript{root}} and \textsc{Ident-Place\textsubscript{root}}, as demonstrated in (\ref{rankingparadox}).

\ea \label{rankingparadox}

\begin{xlist}

\ex /aman-ka/ $\rightarrow$ [amaŋŋa] \\ \label{paradox1} \textsc{Ident-Voi\textsubscript{root}} \guillemotright \textsc{Ident-Place\textsubscript{root}}\\
\ex/eged-ka/ $\rightarrow$ [egetta]\\ \label{paradox2} \textsc{Ident-Place\textsubscript{root}} \guillemotright \textsc{Ident-Voi\textsubscript{root}}

\end{xlist}

\z


Therefore, though the data reveals an apparent generalization that the root consonant always determines the identity of a geminate that results from assimilation across a morpheme boundary, the details of this pattern cannot actually be modeled as a unified phonological process, even when leveraging the morphologically-specific power of indexed constraints. 

\subsection{Prefix/suffix asymmetry}\label{edgeasymmetries}
The cophonologies analysis of the Nobiin data presented above assigns a different ranking of the proposed constraint set to each of the two morphemes in question, thereby deriving the attested assimilation and variation patterns. This analysis captures the generalizations in the data without making reference to morphological information; rather, the difference between the two morphemes falls out of the phonological constraints and their rankings. While we believe that deriving morphophonological patterns using only phonological constraints can be considered theoretically desirable, as mentioned above, we acknowledge that our approach does not directly address the fact that one morpheme in question here is a prefix while the other is a suffix. 

It is a well-established typological generalization that prefixes and suffixes, as well as left and right word edges, behave differently from each other (e.g., \citealp{byedelacy, hyman2008, pycha}). Specifically, suffixation is more cross-linguistically common than prefixation. When both affix types are present in a language, left edges are more resistant to neutralization than right edges \citep{byedelacy}. In other words, suffixes are more ``cohering'' than prefixes; phonological processes at a suffix boundary are often bidirectional, whereas at the prefix boundary phonological processes tend to be unidirectional, dictated solely by the segments in the root \citep{hyman2008}. These typological facts align with the Nobiin data presented here: the phonological alternations seen at the prefix boundary are different from those at the suffix boundary, such that assimilation at the prefix boundary is unidirectionally anticipatory whereas the assimilation at the suffix boundary is bidirectional. One proposal for a theoretical account of this typological asymmetry is the edge-asymmetry hypothesis (EAH), which asserts that no constraint may refer to the right edge of a constituent \citep{byedelacy}. By employing the \textsc{Ident\textsubscript{L}} constraint (\ref{identons}), and by avoiding constraints that conversely refer to the right edges of constituents, we abide by the EAH and contribute to the body of morphophonological analyses that capture edge asymmetries.

\subsection{Future work}
The data and analysis shown here present several directions for future work. First, the speaker who provided the bulk of the data presented here occasionally mentioned instances in which Nobiin speakers from other dialectal regions may have different output forms. For instance, there are varieties of Nobiin in which /ag\,+\,kabin/ might surface as [a.ga.ka.bin], with a vowel being epenthesized to avoid a surface consonant cluster as opposed to the assimilation patterns discussed throughout this paper. It is likely, then, that the constraints on consonant assimilation have a different ranking with respect to other active faithfulness constraints in such varieties. Future work is required to determine whether the phonological patterning of these dialects can shed light on the processes that are described here.

Our analysis also raises several theoretical questions about Nobiin phonology in general. For instance, we do not at this point posit a default phonological grammar based on which the constraints in the morpheme-specific cophonologies are reranked. Further research on the language more broadly is needed to determine which of the rankings active in the derivations presented here are pervasive in the Nobiin phonology. We believe that a more thorough understanding of independent phonological processes in the language may further inform the assimilation patterns we show. 

As mentioned above in \S \ref{edgeasymmetries}, the data analyzed here reveals a clear difference in phonological patterning between one prefix and one suffix in Nobiin. Future work will investigate phonological processes taking place at other morpheme boundaries, with the specific aim of determining whether the edge asymmetries seen here hold across the morphophonology of the language. If this is the case, future theoretical models of Nobiin morphophonology should address these edge asymmetries more explicitly than the analysis presented here.

Finally, though tone is peripheral to the consonant assimilation processes described here, there is very little work on the tonal phonology and morphophonology of Nobiin at this point. Future work on Nobiin will examine phonological processes involving tone, including whether they are morphophonologically constrained like the alternations described throughout this paper.


\section{Conclusion} \label{conclusion}
We have argued for a Cophonology Theory analysis of progressive and accusative affixation in Nobiin. We show that the variation in output progressive forms can be modeled as faithfulness to input moraic structure. Our proposed cophonologies analysis can account for the presence of variation in the surface forms resulting from one type of affixation but not the other. This account also models the differences in assimilation patterns between two affixes: one triggers bidirectional assimilation while the other triggers complete regressive assimilation. Finally, we show that though an indexed constraints analysis seems well-suited to capture the surface facts, cophonologies are needed in order to provide a unified phonological account of the various processes presented.


%\section*{Abbreviations}

%Abbreviations should be included in this section. 

\section*{Acknowledgements}
We are grateful to Tanutamon Gerais for providing the grammaticality judgements on which this work is based. Thanks also to Hannah Sande, Maddie Oakley, Ruth Kramer, and the audience at ACAL50 for feedback on previous versions of this work.

{\sloppy\printbibliography[heading=subbibliography,notkeyword=this]}
\end{document}
