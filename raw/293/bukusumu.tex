\documentclass[output=paper]{langscibook} 
\ChapterDOI{10.5281/zenodo.5578822}

\author{Justine Sikuku\affiliation{Moi University} and Michael Diercks\affiliation{Pomona College}}
\title{Object marking in Lubukusu: Information structure in the verb phrase}
\abstract{Object marker (OM) doubling (i.e. clitic doubling) in Lubukusu has previously been argued to necessarily generate a verum (focus) reading of the clause. We argue for a new empirical generalization: OM-doubling is licit when there is focus in/on the verb phrase, and verum results when that is not otherwise possible (as an elsewhere case). We demonstrate these patterns with a large range of novel empirical data, providing a fuller picture of clitic doubling in Lubukusu.}

\begin{document}
\maketitle

\section{Background and summary of core contributions}

The properties of object markers/clitics (OMs) have long been areas of deep syntactic interest. This paper addresses Lubukusu (Bantu, Luyia subgroup, Kenya), building on \citet*{SikukuEtAl:2018:LubukusuOM} and falsifying some key details of their proposals.\footnote{Lubukusu is a (Luyia) Bantu language; it has been estimated that there are at least 23 different Luyia varieties spoken in Western Kenya and Eastern Uganda \citep{Marlo:2009:LuyiaTonalDialects}. \citet{Lewis:2016:Ethnologue} list the number of Lubukusu speakers at 1,433,000 based on the 2009 census. Originally classified as E31c, an earlier edition of the Ethnologue reclassified it to J30, and \citet{Maho:2008:GuthrieUpdate} to JE31c.} Example (\ref{FirstExampleNoDoubling}) illustrates the OM in Lubukusu, showing that it generally cannot co-occur with a transitive object in neutral pragmatic contexts.\footnote{Examples cited from \citet{SikukuEtAl:2018:LubukusuOM} have tone marking as provided by Michael Marlo, a co-author on that paper; new data in this paper are not marked for tone.}

\begin{exe}  \ex \label{FirstExample}
\begin{xlist}
\ex \label{FirstExampleNoOM}
\gll N-á-βon-a	Weekesa.	\\	
1\Sg{.}\Sm-\Pst-see-\Fv{} 1Wekesa \\ \jambox*{Lubukusu}
\glt `I saw Wekesa.’ \citep[360]{SikukuEtAl:2018:LubukusuOM}

\ex \label{FirstExampleNoDoubling}
\gll N-á-\circled{mu-}βon-a	(\#Weekesa).	\\
1\Sg{.}\Sm-\Pst-1\Om-see-\Fv{} (\#Wekesa) \\ \jambox*{No OM-doubling}
\glt `I saw him.’  (\textit{licit in a context where Wekesa is salient in the discourse})\\
*`I saw Wekesa.' \citep[360]{SikukuEtAl:2018:LubukusuOM}
\end{xlist}
\end{exe}

\noindent Investigations of OMs in Bantu languages have usually centered on whether they can co-occur with (i.e. \textit{double}) overt objects (and, if so, under what conditions),  how they come to occur in the positions that they occur in, and therefore whether OMs are pronominal forms, or agreement markers, or fall under some more nuanced designation. These alternatives center on a core diagnostic of whether or not the OM is in complementary distribution with an overt, \textit{in situ} lexical object.\footnote{A host of relevant references lay behind these core syntactic proposals in the Bantu syntax literature. See \citet{MartenKula:2012:BantuOmParameters} and \citet{MartenKulaThwala:2007:BantuParameters} for broad typological overviews; \citet{BresnanMchombo:1987:ChichewaSmOm}, \citet{Jelinek:1984:IncorporatePronouns}, \citet{Baker:2003:KinandeAgrDislocate}, \citet{VanDerSpuy:1993:NguniOm}, \citet{Zeller:2009:ZuluClld}, \citet{Zerbian:2006:Thesis}, \citet{ByarushengoHymanTenenbaum:1976:Haya}, \citet{Marlo:2014:ExceptionalOmsBantu}, \citet{Marlo:2015:ExceptionalReflexive,Marlo:2015:NumberOfOms}, \citet{DurantiByarushengo:1977:HayaDirectObject}, \citet{Tenenbaum:1977:HayaDislocation}, \citet{Riedel:2009:Thesis}, \citet{Henderson:2006:Thesis}, \citet{Zeller:2012:ZuluOM,Zeller:2015:ZuluDrD,Zeller:2014:ZuluOm3Types}, \citet{Letsholo:2013:IkalangaOM}, \citet{MartenRamadhani:2001:KiluguruOM}, \citet{Keach:1995:SwahiliSmOm}, \citet{Woolford:2001:RuwundOm}, \citet{BaxDiercks:2012:ManyikaOm}, \citet{DiercksRaneroPaster:2014:KuriaClitic}, among others.}

OMs in Lubukusu monotransitives can co-occur with a postverbal object, but that object obligatorily occurs after a clearly discernable prosodic break (marked by a comma below) and receives an afterthought topic reading, both of which are typical characteristics of right-dislocated phrases, suggesting that the lexical object in (\ref{ProsodicBreakExample}) is right-dislocated (\citealt{Riedel:2009:Thesis}, among many others). 

\ea \label{ProsodicBreakExample}
\gll N-á-\circled{ki-}βon-a \#(,) \circled{ée-m-bwa}. \\
1\Sg{.}\Sm-\Rem-9\Om-see-\Fv{} {} 9-9-dog \\ %\jambox{\textbf{Prosodic break before OMed object}} 
\glt `I saw it, the dog.' \citep[366]{SikukuEtAl:2018:LubukusuOM}
\z 

This suggests a pronoun analysis of the OM, as the OM and the \textit{in situ} lexical object are in complementary distribution, and \citet{SikukuEtAl:2018:LubukusuOM} confirm this pattern with various diagnostics.

There are, however, some systematic exceptions to Lubukusu's restrictions on OM-doubling, as illustrated in (\ref{FirstExampleVerumDoubling}):

\begin{exe}  \ex \label{FirstExampleVerumDoubling}
\gll n-aa-\circled{βu-}l-íílé	\circled{βúu-suma}. \\
1\Sg{.}\Sm-\Pst-14\Om-eat-\Pfv{}	14.14-ugali \\
\glt `I \textit{did} eat the ugali!'  \citep[360]{SikukuEtAl:2018:LubukusuOM} %\\
%\textit{(licit if somebody is doubting this is true, e.g. in an argument)}
\end{exe} 

\citet{SikukuEtAl:2018:LubukusuOM} show that co-occurrence of an OM and an object (OM-doubling) is in fact available, but only in pragmatic contexts that license \textsc{verum focus}, similar to English emphatic \textit{do}. \citet{SikukuEtAl:2018:LubukusuOM} propose that the doubling OM and non-doubling OM in Lubukusu have distinct syntactic derivations: non-doubling OMs are incorporated pronouns, and doubling OMs are agreement morphemes arising on an Emphasis head, which introduces a verum focus reading. Centrally for our concerns here, this analysis predicts that OM-doubling should always require a verum reading. We have recently discovered, however, that the empirical generalizations reported in \citet{SikukuEtAl:2018:LubukusuOM} are incomplete. Notably, there are additional contexts where OM-doubling is licensed without a verum reading: 

\ea \label{ex:sikuku:FirstNewExMannerAdverbial}
\begin{xlist}
\exi{Q:} 
\gll W-a-teekh-a ka-ma-kanda \textbf{o-rieena} ? \\
2\Sg.\Sm-\Pst-cook-\Fv{}	6-6-beans	2\Sg-how \\ %\jambox{[Lubukusu]}
\glt `How did you cook the beans?'

\exi{A:} \label{DoublingFocusedMannerAdverb}
\gll N-a-\circled{ka-}teekh-a \circled{ka-ma-kanda} \textbf{bwaangu} \\
1\Sg.\Sm-\Pst-6\Om-cook-\Fv{} 6-6-beans quickly \\
    \glt ‘I cooked the beans \textit{quickly}.’ (not: `I \textit{did} cook the beans quickly.’) 

\end{xlist}
\z 

This shows that the analysis from \citet{SikukuEtAl:2018:LubukusuOM} cannot be correct in a strict sense. The purpose of this paper is to clarify the conditions under which OM-doubling is possible in Lubukusu (and its various syntactic/pragmatic correlates). In this brief paper we do not give an explanatory analysis -- more research is necessary before that is within reach. But we are able to demonstrate a broader set of generalizations licensing OM-doubling in Lubukusu, concluding that the verum doubling analyzed in our previous work reflects only a subset of the possible OM-doubling contexts. The new set of generalizations suggests that the reason that (\ref{DoublingFocusedMannerAdverb}) is acceptable without verum is that OM-doubling triggers conjoint/disjoint-like effects within the verb phrase: doubling creates a focal effect in \textit{v}P that requires focused material in the verb phrase.\footnote{As we will show in \S \ref{SectAnalysis} and \S \ref{SectNeFocusMarker}, this is a mild simplification.} In the absence of such material, verum focus results (which is the set of patterns described by \citealt{SikukuEtAl:2018:LubukusuOM}). \S \ref{SectFocusLicensesDoubling} shows that focus licenses OM-doubling, and \S \ref{SectVpInternal} shows that the focused material must be \textit{v}P-internal to do so. \S \ref{SectInterpretationDoubling} gives some initial data on the interpretation of OM-doubling. \S \ref{SectLitReview} points out an empirical parallel in conjoint/disjoint constructions that heavily factors into the informal analysis that we offer in \S \ref{SectAnalysis}. \S \ref{SectDoublingWithoutFocus} and \S \ref{SectNeFocusMarker} show OM-licensing conditions that are predicted by the informal analysis that we present.

\section{Focus licenses OM-doubling} \label{SectFocusLicensesDoubling}

In this section we illustrate the generalization that focused material licenses OM-doubling on a distinct object in the verb phrase. 

\subsection{New information focus licenses OM-doubling}

We saw in \REF{ex:sikuku:FirstNewExMannerAdverbial} that when a manner adverbial is focused, OM-doubling the direct object is licensed (without a verum reading). Likewise, when a temporal adjunct is questioned or bears new information focus, OM-doubling an object is licit without a verum reading:

\ea
\begin{xlist}

\exi{Q:} 
\gll Ba-ba-ana ba-a-\circled{(ka)-}kes-a \circled{ka-ma-indi} \textbf{liina}? \\
2-2-children 2\Sm-\Pst-6\Om-harvest-\Fv{} 6-6-maize when \\
\glt `When did the children harvest the maize?’ \textit{OK without verum}

\exi{A:}
\gll Ba-ba-ana ba-\circled{(ka)-}kes-ile \circled{ka-ma-indi} \textbf{likolooba}. \\
2-2-children 2\Sm-6\Om-harvest-\Pfv{} 6-6-maize yesterday \\
\glt `The children harvested the maize yesterday.’ \textit{OK without verum}

\end{xlist}
\z

For the sake of space we don't include the data here, but similar patterns arise with lexical ditransitives, with instrumental, benefactive, and causative double object constructions, and with reason adjuncts. In all of these instances, OM-doubling an object is licit in the event that some other constituent in the verb phrase (argument or adjunct) is interpreted as focused.\footnote{All of these data are being compiled in our ongoing work \citep{SikukuDiercks2020BukusuOmBook}.}

% Likewise, when a \textsc{theme} bears focus, a \textsc{benefactive} object can be OM-doubled without verum.

% \ea
% \begin{xlist}

% \exi{Q:} 
% \gll Ba-ba-ana ba-a-\circled{(ba)-}kes-el-a \circled{ba-b-ebusi} \textbf{siina}? \\
% 2-2-children 2\Sm-\Pst-(2\Om)-harvest-\Appl-\Fv{} 2-2-parents what \\
% \glt `What did the children harvest for their parents?’ \textit{OK without verum}

% \exi{A1:}
% \gll Ba-ba-ana ba-a-\circled{(ba)-}kes-el-a \circled{ba-b-ebusi} \textbf{ka-ma-indi} \\
% 2-2-children 2\Sm-\Pst-(2\Om)-harvest-\Appl-\Fv{} 2-2-parents 6-6-maize \\
% \glt `The children harvested maize for their parents.’ \textit{OK without verum}

% \end{xlist}
% \z


%\subsubsection{New Information Focus in Instrumental Ditransitives}

%We see the same pattern in an instrumental applicative double object construction. In (\ref{FocusPlusDoublingInstrumental}) questioning the instrument pr having new information focus on the instrument in the answer to the question readily license OM-doubling: 

%\ea \label{FocusPlusDoublingInstrumental}
%\begin{xlist}

%\exi{Q:} 
%\gll Maisha a-\circled{(ka)-}kulul-il-a \circled{ka-ma-kanda} \textbf{siina}?  \\
%1Maisha 1\Sm.\Pst-(6\Om)-stir-\Appl-\Fv{} 6-6-beans 7what \\ 
%\glt `What did Maisha stir beans with?' \textit{OK without verum} 

%\exi{A1:} 
%\gll Maisha a-\circled{(ka)-}kulul-il-a \circled{ka-ma-kanda} \textbf{ku-mu-kango}   \\
%1Maisha 1\Sm.\Pst-(6\Om)-stir-\Appl-\Fv{} 6-6-beans 3-3-spoon \\ 
%\glt `Maisha stirred beans with a spoon.' \textit{OK without verum} 

%\end{xlist}
%\z


%\subsubsection{New Information Focus in Causative Ditransitive}

%As above, in a causative double object construction, focus on one of the objects (the theme, here) licenses OM-doubling with the other (here, the causee). 

%\ea 
%\begin{xlist}

%\exi{Q:} 
%\gll O-mw-aana a-li-isy-a chi-khafu siina?\\
%1-1-child 1\Sm.\Pst-eat-\Caus-\Fv{} 10-cows 7what \\
%\glt `What did the child feed the cows?' 

%\exi{A1:} 
%\gll O-mw-aana a-\circled{chi-}li-isy-a \circled{chi-}khafu \textbf{ka-ma-indi}\\
%1-1-child 1\Sm.\Pst-10\Om-eat-\Caus-\Fv{} 10-cows 6-6-maize  \\
%\glt `The child fed the cows maize.' \textit{OK without verum}

%\end{xlist}
%\z

%\subsubsection{Lexical DOC}

%\hl{To be added: ``give'' or ``show''  with new information focus} 

%JS: This still appears accurate, but only in DOCs, and is better with recipient argument other than the patient. So, Babaana babokesya bakhaana kamareeba (the children showed the girls the questions) is better than babaana bakokesya kamareeba bakhaana. Perhaps it may also be interesting to know the judgements when the agreed with argument is not in IAV position. Babaana babokesya kamareeba bakhaana/ babaana bakokesya bakhaana kamareeba – The separation sounds better for both. 

\subsection{Focus with \textit{-ong'ene} `only' licenses OM-doubling}

OM-doubling is licensed if you put focus on a constituent using \textit{-ong'ene} `only':

\ea %\underline{Benefactive ditransitive, focus on THEME} \\
\gll Ba-ba-ana ba-a-\circled{(ba)-}rer-er-a \circled{ba-b-ebusi} \textbf{ka-m-echi} \textbf{k-ong’ene}. \\
2-2-children 2\Sm-\Pst-(2\Om)-bring-\Appl-\Fv{} 2-2-parents 6-6-water 6-only \\
\glt `The children brought their parents only water.' \textit{OK without verum}
\z 

\noindent Additional instances of this `only' pattern appear throughout the rest of the paper.


%\ea 
%\gll Babaana babarerera babebusi kamechi, se bakeenda busa ta. \\
%\hl{gloss needed from Sikuku it's confusing to MJKD} \\
%\glt `The children only brought their parents only water.'  \textit{OK without verum}
%\z 

%\ea \underline{Lexical ditransitive, focus on THEME} \\
%\gll Ba-ba-ana ba-a-\circled{b-}okesy-a \circled{baa-khaana} \textbf{ka-ma-reeba} \textbf{k-ong’ene} \\
%2-2-children 2\Sm-\Pst-2\Om-show-\Fv{} 2-girls 6-6-questions 6-only \\
%\glt `The children showed the girls ONLY THE QUESTIONS' (i.e. they didn't show them the answers)\\
%\textit{OK with or without verum}
%\z

%\ea \underline{Benefactive ditransitive, focus on THEME} \\
%\gll Babaana baabakesela babebusi kamaindi kong’ene \\
%gloss here \\
%\glt `The children harvested ONLY MAIZE for their parents (they left the other things in the field).' \textit{OK without verum}
%\z 

%\ea \underline{Causative DOC, focus on THEME} \\
%\gll O-mw-aana a-a-\circled{(chi)-}li-isy-a \circled{chi-khafu} \textbf{ka-ma-indi} \textbf{k-ong’ene} \\
%1-1-child 1\Sm-\Pst-10\Om-eat-\Caus-\Fv{} 10-cows 6-6-maize 6-only  \\
%\glt `The child fed the cows ONLY MAIZE (didn’t feed them anything else).' \textit{OK without verum} (sounds a bit (?))
%\z

\subsection{Contrastive focus licenses OM-doubling}

Contrastive focus shows the same effects as the patterns shown above. When a {\lilv}P-internal constituent is contrastively focused (here diagnosed by a continuation that clarifies which constituent is contrastively focused), OM-doubling is natural without a verum reading. 

\ea \label{ContrastiveFocusManner}
\gll Ba-ba-ana ba-a-\circled{bu-}ly-a \circled{bu-suma} \textbf{bwangu}, se-li kalaa ta. \\
2-2-children 2\Sm-\Pst-14\Om-eat-\Fv{} 14-ugali quickly \Neg-be slowly \Neg \\
\glt `The children ate the ugali QUICKLY, not slowly.’ \textit{OK without verum}
\z 

\section{Focused phrases must be overtly {\lilv}P-internal for doubling} \label{SectVpInternal}

We have seen that OM-doubling is facilitated by focused material without the need for a verum reading; that said, the structural position of the focused material is relevant. The preceding examples are all instances of focused phrases that are likely internal to the verb phrase. Here, we show that material that is external to the verb phrase cannot license OM-doubling.

\subsection{\textit{Ex situ} focus does not license OM-doubling}

%These are updated data from the 8.8.2018 questionnaire

%We have seen extensive evidence in what preceded that focusing a VP-internal constituent is a straightforward way to license OM-doubling on an object without a verum interpretation of the clause. This section shows, as illustrated in (\ref{InitialExSituVerum}), that this focused element must remain \textit{in situ} within the verb phrase to have this effect. A clefted phrase (raised out of the verb phrase) does not license OM-doubling in the same way. As (\ref{InitialExSituVerum}) shows, OM-doubling an object with a clefted temporal adverb requires verum to be acceptable: 

%\ea \label{InitialExSituVerum}
%\gll \#Ya-/lwa-b-ele \textbf{lu-kolooba}\sub{k} ni-lwo ba-ba-ana ba-\circled{ka-}kes-ile \circled{ka-ma-indi} t\sub{k}  \\
%9\Sm-/11\Sm-be-\Pfv{} 11-yesterday COMP-11 2-2-children 2\Sm-6\Om-harvest-\Pfv{} 6-6-maize \\
%\glt ‘It was yesterday that the children OM-harvested the maize.’ \textit{requires verum to be acceptable}
%\z 

%\ea 
%\gll Ya-/mwa-b-eele mu-mu-kunda	ni-mwo	ba-ba-ana	ba-ka-kes-ile	ka-ma-indi \\
%9\Sm-/18\Sm-be-\Pfv{} 18-3-shamba \Comp-18 2-2-children 2\Sm-6\Om-harvest-\Pfv{} 6-6-maize \\
%\glt `It was in the shamba that the children OM-harvested the maize.’ \textit{good with verum, not without}
%\z 

It is the the surface positions of focused phrases that is relevant for licensing doubling. To illustrate, the \textit{in situ} questions in \REF{ex:sikuku:InSituExSituQuestions} license OM-doubling (\ref{DoublingOkInSituQuestion}), but doubling an object that occurs inside a wh-cleft (with nothing else in the verb phrase apart from the doubled object) results in a verum reading of the clause \REF{ex:sikuku:VerumExSituQuestion}. 

\ea \label{ex:sikuku:InSituExSituQuestions}
\begin{xlist}

\ex \label{DoublingOkInSituQuestion}
\gll Ba-ba-ana ba-a-\circled{ka-}kes-a \circled{ka-ma-indi} \textbf{liina}? \\
2-2-children 2\Sm-\Pst-6\Om-harvest-\Fv{} 6-6-maize when \\
\glt `When did the children harvest the maize?' \textit{Does not require verum}

\ex \label{ex:sikuku:VerumExSituQuestion}
\gll \textbf{Liina} ni-lwo ba-ba-ana ba-a-\circled{(\#ka)-}kes-a \circled{ka-ma-indi}? \\
when \Comp-11 2-2-children 2\Sm-\Pst-6\Om-harvest-\Fv{} 6-6-maize  \\
\glt `When did the children harvest the maize?' \textit{Requires verum}

\end{xlist}
\z

A parallel set of facts emerges in the answers to the questions in \REF{ex:sikuku:InSituExSituQuestions}. Either sentence in \REF{ex:sikuku:InSituExSituAnswers} can answer either question in \REF{ex:sikuku:InSituExSituQuestions}, but only the \textit{in situ} focused temporal adjunct licenses OM-doubling \REF{ex:sikuku:InSituAnswer}. As with the questions, focus on an object via a cleft construction when nothing else remains postverbal with the doubled object necessarily results in a verum reading \REF{ex:sikuku:ExSituAnswer}.

\ea \label{ex:sikuku:InSituExSituAnswers}
\begin{xlist}

\ex \label{ex:sikuku:InSituAnswer}
\gll Ba-ba-ana ba-\circled{ka-}kes-ile \circled{ka-ma-indi} \textbf{li-kolooba}. \\
2-2-children 2\Sm-6\Om-harvest-\Pfv{} 6-6-maize 11-yesterday \\
\glt `The children harvested maize yesterday.' \textit{Does not require verum}

\ex \label{ex:sikuku:ExSituAnswer}
\gll \textbf{Li-kolooba} nilwo ba-ba-ana ba-\circled{(\#ka)-}kes-ile \circled{ka-ma-indi}.  \\
11-yesterday \Comp11 2-2-children 2\Sm-6\Om-harvest-\Prf{} 6-6-maize \\
\glt `It was yesterday that the children harvested maize.' \textit{Requires verum}

\end{xlist}
\z

%\noindent We have seen these same patterns with various kinds of objects as well: \textit{ex situ} focus does not license OM-doubling of a separate object inside the verb phrase.

%\subsubsection{Direct object question}

%\underline{In situ direct object question:} 
%\ea
%\begin{xlist}

%\exi{Q:} 
%\gll Ba-ba-ana ba-a-\circled{ba-}kes-el-a \circled{ba-b-ebusi} \textbf{siina}? \\
%2-2-children 2\Sm-\Pst-2\Om-harvest-\Appl-\Fv{} 2-2-parents what \\
%\glt `What did the children harvest for their parents?’ \textit{Does not require verum}

%\exi{A1:}
%\gll Ba-ba-ana ba-a-\circled{ba-}kes-el-a \circled{ba-b-ebusi} \textbf{ka-ma-indi}.  \\
%2-2-children 2\Sm-\Pst-2\Om-harvest-\Appl-\Fv{} 2-2-parents 6-6-maize \\
%\glt `The children harvested maize for (their) parents.' \textit{Does not require verum}

%\exi{A2:}
%\gll \#ka-a-ba \textbf{ka-ma-indi} ni-ko ba-a-\circled{ba-}kes-el-a \circled{ba-b-ebusi}   \\
%6\Sm-\Pst-be 6-6-maize \Comp-6 2\Sm-\Pst-2\Om-harvest-\Appl-\Fv{} 2-2-parents  \\
%\glt `It was the maize that the children harvested for the parents.' \textit{requires verum }

%\end{xlist}
%\z

%\noindent \underline{Ex situ direct object question:}

%\ea
%\begin{xlist}

%\exi{Q:} 
%\gll \#\textbf{Siina} ni-syo ba-ba-ana ba-a-\circled{ba-}kes-el-a \circled{ba-b-ebusi}?    \\
%7what \Comp-7 2-2-children 2\Sm-\Pst-2\Om-harvest-\Appl-\Fv{} 2-2-parents  \\
%\glt `What did the children harvest for their parents?’ \textit{requires verum}

%\exi{A1:}
%\gll Ba-ba-ana ba-a-\circled{ba-}kes-el-a \circled{ba-b-ebusi} \textbf{ka-ma-indi}.   \\
%2-2-children 2\Sm-\Pst-2\Om-harvest-\Appl-\Fv{} 2-2-parents 6-6-maize \\
%\glt `The children harvested maize for (their) parents.' \textit{Does not require verum}

%\exi{A2:}
%\gll \#\textbf{ka-ma-indi} ni-ko ba-a-\circled{ba-}kes-el-a \circled{ba-b-ebusi} \\
%6-6-maize \Comp-6 2\Sm-\Pst-2\Om-harvest-\Appl-\Fv{} 2-2-parents \\
%\glt `It was the maize that the children harvested for the parents.' \textit{requires verum}

%\end{xlist}
%\z

\subsection{Subject focus does not license OM-doubling without verum}

%Crucially, putting focus on the subject does not create the appropriate conditions for OM-doubling to occur without a verum reading. 

Focus on preverbal subjects is incapable of licensing OM-doubling an object. (\ref{SubjectQandA}) shows that subject questions and answers cannot contain OM-doubling without verum:

\ea \label{SubjectQandA}
\begin{xlist}

\exi{Q:} 
\gll \textbf{Naanu} w-a-\circled{(\#ka)-}kes-ile \circled{ka-ma-indi}? \\
1who 1\Sm-\Pst-6\Om-harvest-\Pfv{} 5-5-maize \\
\glt `Who harvested the maize?’ \textit{Doubling requires verum}

\exi{A:} \label{NoSubjFocusForDoubling}
\gll \textbf{Ba-ba-ana} ba-a-\circled{(\#ka)-}kes-ile \circled{ka-ma-indi} \\
2-2-children 2\Sm-\Pst-6\Om-harvest-\Pfv{} 6-6-maize. \\
\glt `The children harvested the maize.' \textit{Doubling requires verum}

\end{xlist}
\z

\noindent Likewise, \textit{-ong’ene} `only’ on the subject does not license doubling without verum:

\ea 
\gll \textbf{Ba-ba-ana} \textbf{b-ong’ene} ba-a-\circled{(\#ba)-}rer-er-a \circled{ba-b-ebusi} ka-m-echi. \\
2-2-children 2-only 2\Sm-\Pst-2\Om-bring-\Appl-\Fv{} 2-2-parents	6-6-water \\
\glt `Only the children brought their parents water.’ \textit{Doubling requires verum}
\z 

\noindent And in the same way, contrastive focus on the subject does not license OM-doubling without verum:\footnote{Note that subject focus does not \textit{exclude} OM-doubling an object; rather, subject focus itself cannot license doubling. OM-doubling may occur with subject focus if the conditions for doubling are met independently of the subject focus.} 

\ea 
\gll \textbf{Ba-ba-ana} ba-a-\circled{(\#bu)-}ly-a \circled{bu-suma}, se-li ba-b-ebusi ta. \\
2-2-children	2\Sm-\Pst-14\Om-eat-\Fv{} 14-ugali \Neg-be 2-2-parents \Neg{} \\
\glt `The children ate ugali, not the parents.' (i.e. the parents didn’t eat ugali) \textit{Doubling requires verum}
\z 


\subsection{Locative adjuncts do not license doubling}

In addition to subjects, focus on locative adjuncts is insufficient to license OM-doubling an object without verum focus. 

\ea 
\begin{xlist}

\exi{Q:} 
\gll Ba-ba-ana ba-a-\circled{(\#ka)-}kes-a \circled{ka-ma-indi} \textbf{wae}? \\
2-2-children 2\Sm-\Pst-6\Om-harvest-\Fv{} 6-6-maize where \\
\glt `Where did the children harvest maize?' \textit{Doubling requires verum}

\exi{A:}
\gll Ba-ba-ana ba-a-\circled{(\#ka)-}kes-a \circled{ka-ma-indi} \textbf{mu-mu-kunda}. \\
2-2-children 2\Sm-\Pst-6\Om-harvest-\Fv{} 6-6-maize 18-3-shamba \\
\glt `The children harvested maize in the shamba.' \textit{Doubling requires verum}

%\exi{A2:} 
%\gll \#Babaana ba-a-\circled{ka-}kes-el-a \circled{ka-ma-indi} \textbf{mu-mu-kunda}. \\
%2-2-children 2\Sm-\Pst-6\Om-harvest-\Appl-\Fv{} 6-6-maize 18-3-shamba \\
%\glt `The children harvested maize in the shamba.' \textit{requires verum to be natural, ? otherwise}

\end{xlist}
\z



\noindent Locative adjuncts are clearly not within the domain where focus licenses OM-doubling. A broad range of postverbal focused material \textit{does} qualify, including manner adjuncts, temporal adjuncts, themes, recipients, benefactives, causees, and instruments: all of these are plausibly \textit{v}P-internal. Locative adjuncts consistently occur to the right of all of these, suggesting that locative adjuncts are outside the \textit{v}P (see \S \ref{SectWordOrder} for additional evidence in this regard). Given this and the subject facts, we therefore assume the relevant domain for focus to license OM-doubling is \textit{v}P.

\section{On the interpretation of OM-doubled objects} \label{SectInterpretationDoubling}

We have not yet arrived at a formal analysis of the interpretation of OM-doubling; that said, we can report a broad range of relevant empirical facts, some of which are very familiar from clitic-doubling constructions cross-linguistically.

\subsection{OM-doubling yields specific readings}

As is common for clitic doubling cross-linguistically, OM-doubled objects in Lubukusu are interpreted as \textit{specific}: 

\ea 
\begin{xlist}

\ex 
\gll N-a-w-a o-mw-aana ka-ma-beele. \\
1\Sg.\Sm-\Pst-give-\Fv{} 1-1-child 6-6-milk \\
\glt `I gave a child milk.' (could be any child) 

\ex 
\gll N-a-\circled{mu-}w-a \circled{o-mw-aana} ka-ma-beele.  \\
1\Sg.\Sm-\Pst-1\Om-give-\Fv{} 1-1-child 6-6-milk \\
\glt `I gave a specific child milk.' (i.e. it is known who the child is) \\
\textit{Assuming focus conditions are met to license doubling}

\end{xlist}
\z 

\noindent As would be expected based on the observation above, an object DP that contains a demonstrative allows OM-doubling much more naturally than a bare nominal object: 

\begin{exe}
\ex \label{DemonstrativeImprovesDoubling}
%\begin{xlist}

%\ex \label{DemonstrativeImprovesDoublingNoDem}
%\gll ?n-a-ba-bon-a ba-ba-ana. \\
%1\Sg{}.\Sm-\Pst-2\Om-see-\Fv{} 2-2-children \\
%\glt I DID see (the) children. (\textit{requires verum})

%\ex \label{DemonstrativeImprovesDoublingWithDem}
\gll n-a-\circled{ba-}bon-a \circled{ba-ba-ana \hspace{2mm} ?(abo)} \\
1\Sg{}.\Sm-\Pst-2\Om-see-\Fv{} {2-2-children 2\Dem{}} \\
\glt `I \textit{did} see those children.'  (\textit{requires verum})
%Needs verum to be correct but without verum its better than if it was without demonstrative
%\end{xlist}
\end{exe}

(\ref{DemonstrativeImprovesDoubling}) requires a verum reading to be acceptable, but the presence of the demonstrative marks a significant improvement in naturalness over its absence.  Likewise, (\ref{DemonstrativeImprovesDoublingFocusRight}) shows that when the focal requirements of OM-doubling are met, OM-doubling is more natural with a demonstrative than without one: 


\ea \label{DemonstrativeImprovesDoublingFocusRight}
\begin{xlist}

\exi{Q:} 
\gll \textbf{Naanu} ni-ye w-a-bon-a? \\
1who \Comp-1 2\Sg.\Sm-\Pst-see-\Fv{} \\
\glt `Who did you see?'

\exi{A:} \label{DoubleMonotransitiveFocusedObject}
\gll n-a-\circled{ba-}bon-a \circled{\textbf{ba-ba-ana ?(abo)}}. \\
1\Sg.\Sm-\Pst-2\Om-see-\Fv{} {2-2-children 2\Dem{}} \\
\glt `I saw those children.' \textit{OK without verum}

\end{xlist}
\z 


%\ea
%\begin{xlist}

%\exi{Q:}
%\gll Wabona babaana abo liina?  \\
%gloss here \\
%\glt `When did you see those children?'

%\exi{A:}
%\gll n-a-ba-bon-e babaana abo likolooba  \\
%gloss here \\
%\glt `I OM-saw those children yesterday.' \textit{OK without verum}
%\end{xlist}

%\z 


%\ea
%\begin{xlist}

%\exi{Q:}
%\gll Wabone naanu likolooba?\\
%gloss here \\
%\glt `Who did you see yesterday?'

%\exi{A:} 
%\gll n-a-ba-bon-e babaana abo likolooba\\
%gloss here \\
%\glt `I OM-saw those children yesterday.' \textit{OK without verum}

%\end{xlist}
%\z 

\subsection{OM-doubling acceptable with D-linked wh-phrases}

It is unacceptable to OM-double a bare wh-phrase:

\ea 
\gll Ba-ba-ana ba-a-\circled{(*ba)-}kes-el-a \circled{\textbf{naanu}} ka-ma-indi? \\
2-2-children 2\Sm-\Pst-(*2\Om)-harvest-\Appl-\Fv{} 2who 6-6-maize \\
\glt `Who did the children harvest maize for?’ 
\z 

\noindent However, D-linked wh-phrases can be readily OM-doubled. 

\ea 
%\begin{xlist}

\gll Ba-ba-ana ba-a-\circled{ba-}kes-el-a \circled{\textbf{ba-andu siina}} ka-ma-indi? \\
2-2-children 2\Sm-\Pst-2\Om-harvest-\textsc{ap}-\Fv{} {2-people \hspace{1mm} 7what} 6-6-maize \\
\glt `Which people did the children harvest maize for?’ 

%\ex 
%\gll Ba-ba-ana ba-a-\circled{ba-}kes-el-a  ka-ma-indi \circled{\textbf{baandu siina}} ? \\
%2-2-children 2\Sm-\Pst-\Om-harvest-\Appl-\Fv{} 6-6-maize {people 7what} \\
%\glt `Which people did the children harvest maize for?’ 

%\end{xlist}

\z 

%\noindent Possible to OM-double theme, in a specific word order: 

%\ea
%\begin{xlist}
%\ex 
%\gll Ba-ba-ana ba-a-ka-kes-el-a ba-b-ebusi ma-indi siina? \\
%2-2-children 2\Sm-\Pst-6\Om-harvest-\Appl-\Fv{} 2-2-parents 6-maize what \\
%\glt `Which maize did the children harvest for their parents?' \textit{OK without verum}

%\ex 
%\gll ?Babaana baakakesela maindi siina babebusi? \\
%gloss here \\
%\glt `Which maize did the children harvest for their parents?' %\textit{questionable without verum -ok with verum}

%\end{xlist}
%\z 

\subsection{OM-doubling possible with focused objects}

Throughout \S \ref{SectFocusLicensesDoubling} we showed that focused phrases license doubling an object. The same focus requirement continues to hold, but there is no restriction against an OM-doubled object itself being focused. (\ref{DoubleFocusedBenefactive}) shows that it is possible to OM-double a \textsc{recipient} that bears new information focus in a benefactive double object construction:

\ea \label{DoubleFocusedBenefactive}
\begin{xlist}

\exi{Q:} 
\gll Ba-ba-ana ba-a-kes-el-a \textbf{naanu} ka-ma-indi? \\
2-2-children 2\Sm-\Pst-harvest-\Appl-\Fv{} 1who 6-6-maize \\
\glt `Who did the children harvest maize for?’ 

\exi{A1:}
\gll Ba-ba-ana ba-a-\circled{ba-}kes-el-a \circled{\textbf{ba-b-ebusi}} ka-ma-indi.\\
2-2-children 2\Sm-\Pst-2\Om-harvest-\Appl-\Fv{} 2-2-parents 6-6-maize\\
\glt `The children harvested maize for (their) parents.' \textit{OK without verum}
\end{xlist}
\z

\noindent The same pattern emerges in a lexical ditransitive with \textit{-ong’ene} `only' focus on the recipient, where that same recipient can be doubled:  

\ea 
\gll Ba-ba-ana ba-a-\circled{b-}okesy-a \circled{\textbf{ba-a-khaana b-ong’ene}} ka-ma-reeba. \\
2-2-children 2\Sm-\Pst-2\Om-show-\Fv{} {2-2-girls \hspace{8mm} 2-only} 6-6-questions \\
\glt `The children showed \textit{only the girls} the questions.' (i.e. they didn't show the boys) \textit{OK without verum}
\z

And in fact, we saw above in (\ref{DoubleMonotransitiveFocusedObject}) that it is possible to OM-double a monotransitive object with nothing else in the \textit{v}P, as long as that object bears focus.

%\noindent \underline{OM-doubling a focused \textsc{instrument} in an instrumental DOC:} 

%\ea 
%\begin{xlist}

%\exi{Q:} 
%\gll Maisha a-kulul-il-a ka-ma-kanda \textbf{siina}?  \\
%1Maisha 1\Sm.\Pst-stir-\Appl-\Fv{} 6-6-beans 7what\\ 
%\glt `What did Maisha stir beans with?'  

%\exi{A2:} 
%\gll Maisha a-\circled{ku-}kulul-il-a \circled{\textbf{ku-mu-kango}} ka-ma-kanda? \\
%1Maisha 1\Sm.\Pst-(3\Om)-stir-\Appl-\Fv{} 3-3-spoon 6-6-beans \\ 
%\glt `Maisha stirred beans with a (THE??) spoon.' \textit{OK without verum}\footnote{The attentive reader will notice a word order shift in this example, whereas none of the other examples in this section show that. There is a lot of variable word order that we discuss in \S \hl{WHAT}, for most doubling constructions. In this section our goal is simply to show the extent of OM-doubling with focused objects, so for the most part we've included evidence with no word order shifts. OM-doubling a focused right-edge instrument is unacceptable without verum, however, which is why we include this example, to show that doubling a focused instrumental object is acceptable, but requires a leftward movement of that object.}

%\end{xlist}
%\z

%\noindent \underline{OM-doubling a focused \textsc{theme} in an instrumental DOC:} 

%\ea 
%\begin{xlist}

%\exi{Q:} 
%\gll Maisha a-kulul-il-a \textbf{siina} ku-mu-kango? \\
%1Maisha 1\Sm.\Pst-stir-\Appl-\Fv{} 7what 3-3-spoon \\ 
%\glt `What did Maisha stir with a spoon?'  


%\exi{A1:} 
%\gll Maisha a-\circled{ka-}kulul-il-a \circled{\textbf{ka-ma-kanda}} ku-mu-kango?  \\
%1Maisha 1\Sm.\Pst-(6\Om)-stir-\Appl-\Fv{} 6-6-beans 3-3-spoon  \\ 
%\glt `Maisha stirred a/the??? beans with a spoon.' \textit{OK without verum (degraded)}


%\end{xlist}
%\z


%\noindent \underline{OM-doubling a focused \textsc{causee} in an causative DOC:} 

%\ea 
%\begin{xlist}

%\exi{Q:} 
%\gll O-mw-aana a-li-isy-a \textbf{siina} ka-ma-indi?  \\
%1-1-child 1\Sm.\Pst-eat-\Caus-\Fv{} 7what 6-6-maize \\
%\glt `What did the child feed maize to?'

%\exi{A1:} 
%\gll O-mw-aana a-\circled{chi-}li-isy-a \circled{\textbf{chi-khafu}} ka-ma-indi\\
%1-1-child 1\Sm.\Pst-10\Om-eat-\Caus-\Fv{} 10-cows 6-6-maize  \\
%\glt `The child fed the cows maize.' \textit{OK without verum}

%\end{xlist}
%\z

%\noindent \underline{OM-doubling a focused \textsc{theme} in an causative DOC:} 

%\ea 
%\begin{xlist}

%\exi{Q:} 
%\gll O-mw-aana a-li-isy-a chi-khafu \textbf{siina}?\\
%1-1-child 1\Sm.\Pst-eat-\Caus-\Fv{} 10-cows 7what \\
%\glt `What did the child feed the cows?' 

%\exi{A1:} 
%\gll O-mw-aana a-\circled{ka-}li-isy-a chi-khafu \circled{\textbf{ka-ma-indi}} \\
%1-1-child 1\Sm.\Pst-6\Om-eat-\Caus-\Fv{} 10-cows 6-6-maize \\
%\glt `The child fed the cows maize.' \textit{OK without verum}

%\end{xlist}
%\z

%\subsubsection{OM-doubling objects with `only' focus}



%\ea \underline{Benefactive DOC, focus on \textsc{recipient}} \\
%\gll Ba-ba-ana ba-a-\circled{ba-}kes-el-a \circled{\textbf{baa-bebusi b-ong’ene}} ka-ma-indi \\
%2-2-children 2\Sm-\Pst-2\Om-harvest-\Appl-\Fv{} {2-parents \hspace{4mm} 2-only} 6-6-maize \\
%\glt `The children harvested maize for ONLY THEIR PARENTS (not for anyone else).' \textit{OK without verum}
%\z 

%\ea 
%\begin{xlist}
%\ex 
%\gll Babaana baabarerera babebusi bong’ene  kamechi \\
%gloss here \\
%\glt `The children brought only their parents water.'\textit{OK without verum}
%\ex 
%\gll Babaana baabarerera kamechi babebusi bong’ene.\\
%gloss here \\
%\glt `The children brought only their parents only water.' %\textit{OK without verum}
%\end{xlist}
%\z 


%\subsection{Contrastive focus}

%\hl{WHEN WE GET EXAMPLES OF CONTRASTIVE FOCUS being OM-DOUBLED}

\subsection{``Aboutness'' topics require OM-doubling}
OM-doubled phrases receive an ``aboutness'' interpretation that can be discerned by explicitly requiring an aboutness interpretation of the relevant object: 

\ea Prompt: ``Tell me something about Wekesa.''\\
%\begin{xlist}

%\exi{Prompt:} `Tell me something about Wekesa.'

%\gll \hl{Bukusu translation of ``tell me something about Wekesa.''} \\
%\hl{GLOSS} \\
%\glt `Tell me something about Wekesa.'

%\exi{Response:} 
\label{DoublingAboutness}
\gll N-a-\circled{\#?(mu)-}w-el-a \circled{Wekesa} ba-ba-ana bi-anwa.  \\
1\Sg.\Sm-\Pst-1\Om-give-\Appl-\Fv{} 1Wekesa 2-2-children 8-gifts \\
\glt `I gave the children gifts for Wekesa.' 

%enyuma wekhukhumba biachana ne ndi Nairobi (original data continued with this expansion of the comment
%after he gave them to me when I was in Nairobi
%\end{xlist}
\z 

We see in (\ref{DoublingAboutness}) that an object that is an aboutness topic is preferably OM-doubled. In this sense there is some ``topicality'' to an OM-doubled phrase, but it's important to note that this does not exclude focused phrases and discourse-new information being OM-doubled. \citet{SikukuEtAl:2018:LubukusuOM} report that focus does not license OM-doubling on an object (in apparent contrast to what we have reported above):\footnote{Minor aspects of the transcriptions in (\ref{ApparentFocusNotLicensingDoubling}) were altered to match our transcription conventions in this paper: following orthographic conventions in the Lubukusu-speaking community, in this paper we represent the velar fricative as 〈kh〉 and the bilabial fricative/stop as 〈b〉. And note that while we have shown in \REF{ContrastiveFocusManner} that contrastive focus can license doubling, what is at issue in \REF{ApparentFocusNotLicensingDoubling} is that the object itself is focused, as opposed to a manner adverb in \REF{ContrastiveFocusManner}.} 

\ea \label{ApparentFocusNotLicensingDoubling}
\gll Lionéeli k-á-\circled{(\#ku)-}ly-a \circled{\textbf{kú-mú-chéele}}, se-k-á-ly-á búu-sumá tá. \\
1Leonnel 1\Sm-\Pst-3\Om-eat-\Fv{} 3-3-rice \Neg-1\Sm-\Pst-eat-\Fv{} 14.14-ugali \Neg{} \\
\glt `Leonell ate the rice, he didn’t eat the ugali.’  \citep[376]{SikukuEtAl:2018:LubukusuOM} \\ (\textit{OM-doubling requires verum}) \\
(\textit{See comments below for alternative licensing conditions})

\z 

\noindent Adding a demonstrative to the doubled object in (\ref{ApparentFocusNotLicensingDoubling}) is not sufficient to license doubling without verum. But if (\ref{ApparentFocusNotLicensingDoubling}) is a response to the prompt ``Tell me about what Lionell ate,'' (\ref{ApparentFocusNotLicensingDoubling}) becomes acceptable without verum. 

Clearly specificity, aboutness, and focus are all important aspects of OM-dou\-bling.
An aboutness interpretation appears to be central to licensing OM-dou\-bling. Specificity is also linked with OM-doubling, but it appears to be insufficient to license OM-doubling on its own.\footnote{Our current thought is that it's an effect of OM-doubling but not a cause or licensing condition of OM-doubling.} We are proposing that OM-doubled phrases are aboutness topics in a topic-comment information structure, but they are \textit{not} topical in the sense of being necessarily discourse-old. 

\section{Conjoint/disjoint\,+\,OMing in Zulu \citep{Zeller:2015:ZuluDrD}} \label{SectLitReview}

%It is well known that information structure across a range of African languages has central grammatical effects, to the extent of being a fundamental organizing principle of a language's grammar (see, for example: \citealt{HymanWatters:1984:AuxFocus},  \citealt{Schwarz:2007:KikuyuFocus}, \citealt{AbelsMuriungi:2008:TharakaFocus}, \citet{Hyman:2010:AghemFocus}, \citet{HymanPolinsky:2010:AghemFocus}, \citealt{LandmanRanero:2018:KuriaFocus}).

%The long-documented patterns of grammaticalized focus in Aghem are another key point of comparison: for length reasons we don't summarize those here, but see \S \ref{SectAghem} in the appendix for an overview.

%\subsection{Broad/Predicate Focus}

%\noindent Some northeastern Bantu languages require obligatory focus marking of the verb in affirmative declarative environments, including Kikuyu and Tharaka (); in left-peripheral focus contexts this focus marker can appear on the focused constituent. The examples below are from Kuria (). \hl{SCHWARZ 2007, Abels and Muriungi}; \hl{LANDMAN RANERO}).

%\ea
%\gll Ichi-ng’iti *(n-)cha-a-it-ir-e ege-toocho. \\
%10-hyena (\Foc-)10\Sm-\Pst-kill-\Prf-\Fv{} 7-rabbit \\ 
%\glt `The hyenas killed the rabbit.’
%\z

%\ea
%\gll *(N-)ke ichi-ng’iti (*n-)cha-a-it-ir-e? \\
%(\Foc-)what 10-hyena (\Foc-)10\Sm-\Pst-kill-\Prf-\Fv{} \\
%\glt `What did the hyenas kill?’
%\z 



%\subsection{Conjoint/Disjoint in Bantu Languages}


%\ea \underline{Tswana (S31, Creissels 1996: 109, glosses added)} 

%\begin{xlist}

%\ex 
%\gll Dikgomó dí-fúla kwa nokeng. (conjoint) \\
%10.cows 10\Sm-graze at river \\
%\glt `The cows graze/are grazing at the river.’

%\ex 
%\gll Dikgomó dí-á-fúla. (disjoint) \\
%10.cows 10\Sm-\Prs{}.\Dj-graze \\
%\glt`The cows are grazing.’

%\end{xlist}
%\z 

It is well known that information structure has central grammatical effects across a range of African languages, to the extent of being a fundamental organizing principle of some grammatical systems.\footnote{See, for example: \citet{HymanWatters:1984:AuxFocus},  \citet{Schwarz:2007:KikuyuFocus}, \citet{AbelsMuriungi:2008:TharakaFocus}, \citet{Hyman:2010:AghemFocus}, \citet{HymanPolinsky:2010:AghemFocus}, \citet{LandmanRanero:2018:KuriaFocus}.} A relatively well-studied example of this is the conjoint/disjoint distinction that appears on verbal forms in many Bantu languages and which reflects focal properties of the clause (see \citealt{VanDerWalHyman:2017:ConjointDisjoint} for an overview). Conjoint forms on a verb show a closer connection between the verb and what follows, and disjoint forms are used when there is a looser connection with what follows or when nothing follows the verb \citep{VanDerWalHyman:2017:ConjointDisjoint}.\largerpage[2] 

In Zulu, the predominant analysis is that the conjoint/disjoint distinction\linebreak tracks the presence of overt morphosyntactic content inside \textit{v}P and that focal effects are secondary (see \citealt{Halpert:2016:Book}, \citealt{Zeller:2015:ZuluDrD}, and references cited therein). Conjoint is used when a constituent is inside \textit{v}P; disjoint is used when \textit{v}P is empty.\footnote{We follow Zeller in not marking tone or phrasal penult lengthening in the Zulu data, though these prosodic properties have also been shown to mark the edge of the verb phrase \citep{VanDerSpuy:1993:NguniOm,ChengDowning:2009:ZuluTopic}.}

\ea 
\begin{xlist}

\ex 
\gll U-mama u-phek-a i-n-yama  {]}\sub{\textit{v}P}\\
\Aug-1a.mother 1\Sm-cook-\Fv{} \Aug-9-meat\\\jambox*{(conjoint)}
\glt `Mother is cooking the meat.’

\ex 
\gll  *U-mama u-phek-a  {]}\sub{\textit{v}P}\\
\Aug-1a.mother 1\Sm-cook-\Fv{}\\\jambox*{(conjoint)}
\glt Intended: `Mother is cooking.’

\ex 
\gll U-mama u-\textbf{ya}-phek-a  {]}\sub{\textit{v}P}\\
\Aug-1a.mother 1.\Sm-\Dj-cook-\Fv{}\\\jambox*{(disjoint)}
\glt `Mother is cooking.’ \citep{Zeller:2015:ZuluDrD}

\end{xlist}
\z 

%\indent These patterns interact with object marking in many languages. In Tswana, if an OM is on the verb a postverbal coreferent object cannot occur with the conjoint form: 

%\ea \underline{Tswana (S31, Creissels 1996: 112, 113)}
%\begin{xlist}

%\ex
%\gll Re-thúsá Kítso. (conjoint) \\
%1\Pl.\Sm-help 1.Kitso \\
%\glt `We help Kitso.’

%\ex
%\gll Re-a-mo-thúsá Kitso (disjoint) \\ 1pl.sm-\Dj-1om-help 1.Kitso \\
%\glt `We help him, Kitso.’ 

%\ex * Re-mo-thúsá Kítso. (conjoint)

%\ex * Re-a-thúsá Kítso. (disjoint)

%\end{xlist}
%\z

%\subsection{Zulu Object Marking \citep{Zeller:2015:ZuluDrD}}

\noindent There is a long history of research on Zulu object marking.\footnote{Selected references include \citet{Adams:2010:Thesis,Buell:2005:Thesis,Buell:2006:ZuluConjointDisjoint,ChengDowning:2009:ZuluTopic,Halpert:2016:Book,VanDerSpuy:1993:NguniOm,Zeller:2012:ZuluOM,Zeller:2014:ZuluOm3Types,Zeller:2015:ZuluDrD}.} The data and discussion below are from \citet{Zeller:2015:ZuluDrD}. In Zulu, OM-doubling in a transitive requires the disjoint verb form:  

\ea 
%\begin{xlist}

%\ex 
\gll U-mama u-*(\textbf{ya})-\circled{yi-}phek-a {]}\sub{\textit{v}P} \circled{i-n-yama}.\\
\Aug-1a.mother 1\Sm-\Dj-9\Om-cook-\Fv{} {} \Aug-9-meat \\ %\jambox{\textbf{[Zulu]}}
\glt `Mother is cooking it, the meat.’ \citep[20]{Zeller:2015:ZuluDrD}

%\ex 
%\gll Ngi-yi-fund-ile/*-e  {]}\sub{\textit{v}P} i-n-cwadi.  \\ 1\Sg-9\Om-read-\Pst.\textsc{dj}/\Pst{}.\textsc{cj} {} \Aug-9-book \\
%\glt `I was reading/studying it, the book.’ \citep[20]{Zeller:2015:ZuluDrD}

%\end{xlist}
\z 


%\ea \underline{OM-doubled arguments occur to the right of manner adverbs:}
%\begin{xlist}

%\ex 
%\gll Si-bon-a i-n-kosi kahle. \\
%1\Sg.\Sm-see-\Fv{} \Aug-9-chief well \\ \hfill \textbf{[Zulu]}
%\glt `We are seeing the chief well.’

%\ex 
%\gll *Si-yi-bon-a i-n-kosi kahle. \\
%1\Sg.\Sm-9\Om-see-\Fv{} \Aug-9-chief well \\

%\ex 
%\gll *Si-bon-a kahle i-n-kosi. \\
%1\Sg.\Sm-see-\Fv{} well \Aug-9-chief  \\ 

%\ex \label{ZuluOmDoublingOutsideMannerAdv}
%\gll Si-\circled{yi-}bon-a kahle \circled{i-n-kosi}. \\
%1\Sg.\Sm-9\Om-see-\Fv{} well \Aug-9-chief \\
%\glt `We are seeing him well, the chief.’ \citep[20]{Zeller:2015:ZuluDrD}

%\end{xlist}
%\z 

%\noindent Assuming that manner adverbs adjoin at the edge of \textit{v}P, this suggests that OM-doubled arguments move outside \textit{v}P: 

%\ea . . .siyibona kahle]\sub{\textit{v}P} . . . inkosi = (\ref{ZuluOmDoublingOutsideMannerAdv}) \citep[20]{Zeller:2015:ZuluDrD}
%\z

%\noindent It is well-established that focused phrases in Zulu must appear \textit{v}P-internally: 

%\ea 
%\begin{xlist}

%\ex 
%\gll Ku-fik-e u-Sipho kuphela. \\
%17.EXPL-arrive-\Pst{} \Aug-1a.Sipho only \\
%\glt `Only Sipho arrived.’

%\ex 
%\gll *U-Sipho kuphela u-fik-ile. \\
%AUG-1a.Sipho only 1.SM-arrive-PAST.DIS \\

%\ex 
%\gll Ku-sebenz-e bani?  \\
%17.EXPL-work-PAST 1a.who \\
%\glt `Who worked?’

%\ex 
%\gll *U-bani u-sebenz-ile?  \\
%AUG-1a.who 1.SM-work-PAST.DIS \\

%\end{xlist}
%\z 

%\noindent It is therefore expected that OM-doubling focused phrases would be impossible, which is confirmed below: 

%\ea 
%\begin{xlist}

%\ex 
%\gll Ngi-bon-e \textbf{u-Sipho} \textbf{kuphela}]\sub{\textit{v}P}. \\
%1S-see-PAST AUG-1a.Sipho only \\
%\glt `I saw only Sipho.’

%\ex 
%\gll *Ngi-\circled{m-}bon-ile]\sub{\textit{v}P} \circled{\textbf{u-Sipho} \hspace{7mm} \textbf{kuphela}}. \\
%1S-1.OM-see-PAST.DIS {AUG-1a.Sipho only} \citep[(6)]{Buell:2008:ZuluDpsVpDislocation} \\

%\end{xlist}
%\z 

%\ea 
%\begin{xlist}

%\ex 
%\gll U-cul-e \textbf{i-phi} \textbf{i-n-goma}]\sub{\textit{v}P}?  \\
%2S-sing-PAST 9.ADJ-which AUG-9-song \\
%\glt `Which song did you sing?’

%\ex 
%\gll *U-\circled{yi-}cul-ile]\sub{\textit{v}P} \circled{\textbf{i-phi} \hspace{10mm} \textbf{i-n-goma}}? \\
%2S-9.OM-sing-PAST.DIS {9.ADJ-which AUG-9-song} \citep[(5)]{Buell:2008:ZuluDpsVpDislocation} \\

%\end{xlist}
%\z 
%\ea  \hl{TREE (65)}
%\z

\noindent Zulu also has ``symmetrical'' OMing in ditransitives, wherein either object can be OMed. The resulting word order shows that the doubled object is dislocated, as it must appear on the right edge: 

\ea 
\begin{xlist}

\ex 
\gll Ngi-\circled{m-}theng-el-a u-bisi {]}\sub{\textit{v}P} \circled{u-Sipho}.\\
1\Sg-1\Om-buy-\Appl-\Fv{} \Aug-11.milk {} \Aug-1a.Sipho \\\jambox*{(conjoint)}
\glt `I’m buying him milk, Sipho.’

\ex 
\gll *?Ngi-\circled{m-}theng-el-a \circled{u-Sipho} u-bisi. {]}\sub{\textit{v}P}\\
1\Sg-1\Om-buy-\Appl-\Fv{} \Aug-1a.Sipho \Aug-11.milk \\\jambox*{(conjoint)}

\ex 
\gll Ngi-\circled{lu-}theng-el-a u-Sipho {]}\sub{\textit{v}P} \circled{u-bisi}. \\
1\Sg-11\Om-buy-\Appl-\Fv{} \Aug-1a.Sipho {} \Aug-11.milk \\\jambox*{(conjoint)}
\glt `I’m buying it for Sipho, the milk.’ \citep[22]{Zeller:2015:ZuluDrD}

\end{xlist}
\z

\noindent Note that all of the examples above use the conjoint form, because the non-doubled object in each case is inside \textit{v}P, creating a conjoint environment. It is possible to double an object in a ditransitive with a disjoint verb form, however, as \citet[23]{Zeller:2015:ZuluDrD} shows: 

\begin{exe}
%\ex
%\begin{xlist}

\ex Double right dislocation: both objects dislocated: \label{DRDRecTheme} \\
\gll Ngi-\textbf{ya}-\circled{m-}theng-el-a {]}\sub{\textit{v}P} \circled{u-Sipho} u-bisi. (disjoint) \\
1\Sg-\Dj-1\Om-buy-\Appl-\Fv{} {} \Aug-1a.Sipho \Aug-11.milk \\
\glt `I AM buying milk for Sipho.’ 

%\ex \label{DRDThemeRec}
%\gll Ngi-ya-m-theng-el-a u-bisi u-Sipho. \\
%1S-DIS-1.OM-buy-APPL-FV AUG-11.milk AUG-1a.Sipho \\
%\glt `I AM buying milk for Sipho.’ \citep[23]{Zeller:2015:ZuluDrD}

%\end{xlist}
\end{exe}
(\ref{DRDRecTheme}) uses the disjoint form: both objects have vacated the \textit{v}P. 

\begin{quote}
As indicated by the translations, constructions such as [(\ref{DRDRecTheme})] are typically interpreted as expressing verum (polarity) focus, an interpretation that is not available for [other] right dislocation constructions. Other interpretations occasionally reported by speakers are narrow verb focus, or habituality of the activity expressed by the verb. All these interpretations fall under the category `auxiliary focus’ discussed in Hyman and Watters (1984), which is defined as focus ‘placed on any of the semantic parameters which serve as operators on propositions: tense, aspect, mood, polarity.' \citep[236]{Zeller:2015:ZuluDrD}
\end{quote}

Zeller's analysis is that anti-focus features on a functional head F in the middlefield of the clause probe and find an anti-focus object. The agreed-with object raises to a right-facing Spec,FP, arising at the right-edge. If the \textit{v}P doesn't have additional content, a disjoint form appears on the verb: the typical case of OMing. In double-right-dislocation constructions like (\ref{DRDRecTheme}) the \textsc{recipient} undergoes this OMing process. However, the \textsc{theme} is unfocused and can't remain inside \textit{v}P (a focus domain in Zulu) and therefore is right-dislocated (without interacting with the probe on F).

%\item ``the obligatory dislocation of all non-focused material in DRD-constructions specifically may be a consequence of the auxiliary focus interpretation typically associated with these constructions (see footnote 8). Interestingly, Hyman and Watters (1984) show that in the Bantu language Kirundi, auxiliary focus is expressed by the disjoint form of the verb (cf. Meeussen, 1959; Ndayiragije, 1999). The same may be the case in Zulu, but recall that the choice of the disjoint form in Zulu also depends on constituency: the disjoint form of the verb cannot be followed by nP-internal material (see Section 2). Therefore, dislocation of all \textit{v}P-internal material may be required in Zulu to create the syntactic environment in which the morphological exponent of auxiliary focus is licensed.'' \citep[32]{Zeller:2015:ZuluDrD}

The pattern that we see in Zulu, then, is that OM-doubled objects move to the right edge of \textit{v}P: if \textit{v}P still has content, the verb appears in a conjoint form, but if \textit{v}P is empty, it appears in a disjoint form. It is possible to use a disjoint form when doubling a single object in a double object construction, but Zeller analyzes both of the objects as dislocated, and a verum-like reading of the clause results. And this pattern of facts reflects common cross-linguistic patterns from related constructions, as shown in \tabref{CjDjTable}.


\begin{table}
\caption{Cross-linguistic properties of conjoint vs. disjoint (and similar constructions)  (modified from \citealt[328]{Guldemann:2003:ProgressiveFocusBantu})}
\label{CjDjTable}
\begin{tabularx}{\textwidth}{lQQ}
\lsptoprule
 & Disjoint form & Conjoint form \\
\midrule 
Postverbal XP  &  optional  &  obligatory  \\ 
Verb position  & can be clause-final  & not clause-final   \\ 
Postverbal material  & discourse-old  & discourse-new, asserted  \\
Complement is  & anaphoric, definite, generic  & indefinite  \\
Object marking is  & possible  & impossible  \\
Emphasis on  & positive truth value (verum)  & postverbal constituent  \\
%In polar questions and answers  & In constituent questions and answers  \\
Focus pattern  &  predicate within the scope of focus, complement/adjunct extrafocal & complement/adjunct within the scope of focus, predicate extrafocal \\
\lspbottomrule
\end{tabularx}
\end{table} 


\section{Initial observations regarding word order} \label{SectWordOrder}

It is tempting to analyze Lubukusu like Zulu, correlating OM-doubling with movement out of \textit{v}P. Potential evidence for this is that OM-doubling makes it sound more natural for an object to be moved to the right edge. Parenthetical judgments in the following examples are alternative positions for the doubled object.
    
\ea 
\begin{xlist}

\ex 
\gll Ba-ba-ana ba-a-\circled{(ka)-}kes-a \circled{ka-ma-indi} \textbf{liina}? \\
2-2-children 2\Sm-\Pst-6\Om-harvest-\Fv{} 6-6-maize when \\
\glt `When did the children harvest the maize?’ \textit{Doubling OK without verum}

\ex
\gll Ba-ba-ana ba-\circled{(ka)-}kes-ile \circled{ka-ma-indi} \textbf{likolooba} (\Checkmark). \\
2-2-children 2\Sm-6\Om-harvest-\Pfv{} 6-6-maize yesterday \\
\glt `The children harvested the maize yesterday.’ \textit{Doubling OK without verum}

%\exi{A2:}
%\gll Ba-ba-ana ba-\circled{ka-}kes-ile \textbf{likolooba} \circled{ka-ma-indi} \\
%2-2-children 2\Sm-(6\Om)-harvest-\Pfv{} yesterday  6-6-maize \\
%\glt `The children harvested the maize yesterday.’ \textit{Doubling OK without verum}

\end{xlist}
\z


%\begin{itemize}
    
%    \item As (\ref{NoDoublingMovementOutsideLocative}) shows, while an OM-doubled recipient in a benefactive applicative can appear somewhat readily to the right of the theme, it is unacceptable to the right of a locative adjunct.

%\end{itemize}

%\begin{exe}
%\ex \label{NoDoublingMovementOutsideLocative}
%\begin{xlist}

%\exi{Q:} 
%\gll \textbf{Siina} ni-syo Wafula a-a-teekh-el-a mayi wewe mu-chikoni? \\
%7what \Comp-7 1Wafula 1\Sm-\Pst-cook-\Appl-\Fv{} 1mother 1his 18-kitchen  \\
%\glt `What did Wafula make for his mother in the kitchen?' 

%\exi{A:} 
%\gll Wafula a-a-\circled{mu-}fuk-il-a \textbf{bu-suma} \circled{mayi \hspace{3mm} wewe} mu-chikoni (*?) \\
%1Wafula 1\Sm-\Pst-1\Om-cook-\Appl-\Fv{} 14-ugali {1mother 1his} 18-kitchen  \\
%\glt `Wekesa made ugali for his mother in the kitchen.' \textit{Doubling OK without verum}

%\exi{A2:} 
%\gll *?Wafula a-a-\circled{mu-}fuk-il-a \textbf{bu-suma} mu-chikoni \circled{mayi \hspace{3mm} wewe}  \\
%1Wafula 1\Sm-\Pst-1\Om-cook-\Appl-\Fv{} 14-ugali 18-kitchen {1mother 1his}   \\


%\end{xlist}
%\end{exe}

The Lubukusu facts are non-identical to Zulu, however:  while a doubled object can occur at the right edge of the verb phrase, the preferred position of a doubled object is the leftmost position in (\ref{ChanaWordOrderAtvP}) (which in this example is the position immediately after the verb). 
    
%1. omwalimu amuwa omwaana chana sitabu bulayi
%2. omwalimu amuwa sitabu omwaana chana  bulayi 
%3. omwalimu amuwa sitabu bulayi omwaana chana
%
%- 1 is the neutral..2 and 3 seem both OK with me but if I have to rank then 2 would be better. In 3 the benefactive feels rather more dislocated.

%If we made it “only the book” I wonder if 3 feels less dislocated? Or the same? Maybe we could add a locative adjunct too and see if omwana Chana has to be inside or outside it

%1. omwalimu amuwa omwaana chana sitabu syong’ene bulayi
%2. omwalimu amuwa sitabu syong’ene omwaana chana  bulayi 
%3. omwalimu amuwa sitabu syong’ene bulayi omwaana chana

%3 feels less dislocated. If you add a locative adjunct like khusoko...on the market, omwaana chana must be inside

%it sounds like you can also move omwana Chana to the right of sitabu syong’ene but inside bulayi as well. Is that true? Especially with sitabu focused? (YES)

\ea \label{ChanaWordOrderAtvP}
%\begin{xlist}

%\ex 
\gll o-mw-alimu a-a-\circled{mu-}w-a \circled{o-mw-aana chana} \textbf{si-i-tabu} \textbf{sy-ong’ene} (\Checkmark) bulayi (?) khu-soko (*). \\
1-1-teacher 1\Sm-\Pst-1\Om-give-\Fv{} {1-1-child \hspace{3.7mm} \textsc{about}} 7-7-book 6-only {} well {} 17-9market \\
\glt `The teacher gave the child (that I'm talking about) only a book well in the market.' (i.e. did a good job giving) \textit{Doubling OK without verum}

%\ex 
%\gll o-mw-alimu a-a-\circled{mu-}w-a  \textbf{si-i-tabu} \textbf{sy-ong’ene} \circled{o-mw-aana chana} \underline{bulayi} khu-soko (preferred order) \\
%1-1-teacher 1\Sm-\Pst-1\Om-give-\Fv{} 7-7-book 6-only {1-1-child \hspace{3.7mm} \textsc{about}} well 17-9market \\
%\glt `The teacher gave the child (that I'm talking about) only a book well in the market.' (i.e. did a good job giving)

%\ex 
%\gll o-mw-alimu a-a-\circled{mu-}w-a  \textbf{si-i-tabu} \textbf{sy-ong’ene} \underline{bulayi} \circled{o-mw-aana chana} khu-soko (preferred order) \\
%1-1-teacher 1\Sm-\Pst-1\Om-give-\Fv{} 7-7-book 6-only well {1-1-child \hspace{3.7mm} \textsc{about}} 17-9market \\
%\glt `The teacher gave the child (that I'm talking about) only a book well in the market.' (i.e. did a good job giving)

%\ex 
%\gll *o-mw-alimu a-a-\circled{mu-}w-a  \textbf{si-i-tabu} \textbf{sy-ong’ene} \underline{bulayi} khu-soko \circled{o-mw-aana chana}  (preferred order) \\
%1-1-teacher 1\Sm-\Pst-1\Om-give-\Fv{} 7-7-book 6-only well 17-9market {1-1-child \hspace{3.7mm} \textsc{about}}  \\
%\glt `The teacher gave the child (that I'm talking about) only a book well in the market.' (i.e. did a good job giving)

%\end{xlist}

\z 

We include a locative adjunct here because we have shown above that they behave as if they are outside \textit{v}P: if a doubled object were to appear to the right of a locative adjunct, it would be strong evidence for dislocation of that object. The unacceptability of a doubled object outside the locative is consistent with our conclusions that locatives are structurally higher than manner adjuncts and are outside the relevant domain of OM-doubling.\footnote{The impossibility of the object to the right of the locative adjunct does not imply that the object is not moved to a position at the right edge of \textit{v}P that is below the locative adjunct, of course.} The available positions on either side of the manner adverb are amenable to an account of a position being available at the edge of \textit{v}P.\footnote{It remains to be seen if there are interpretive distinctions between those two object positions.}


%\noindent We see the same pattern replicated below: 

%\begin{exe}
%\ex 
%\begin{xlist}

%\exi{Q:} 
%\gll Ba-ba-ana ba-a-(ba)-kes-el-a ba-b-ebusi siina mumukunda? \\
%2-2-children 2\Sm-\Pst-(2\Om)-harvest-\Appl-\Fv{} 2-2-parents what 18-3-shamba? \\
%\glt `What did the children harvest for their parents?’ \textit{OK without verum}

%\exi{A1:}
%\gll Ba-ba-ana ba-a-(ba)-kes-el-a ba-b-ebusi ka-ma-indi mumukunda \\
%2-2-children 2\Sm-\Pst-(2\Om)-harvest-\Appl-\Fv{} 2-2-parents 6-6-maize 18-3-shamba \\
%\glt `The children OM-harvested maize for their parents.’ \textit{OK without verum}

%\exi{A2:} 
%\gll Ba-ba-ana ba-a-(ba)-kes-el-a ka-ma-indi ba-b-ebusi mumukunda \\
%2-2-children 2\Sm-\Pst-(2\Om)-harvest-\Appl-\Fv{} 6-6-maize 2-2-parents 18-3-shamba \\
%\glt `The children OM-harvested maize for their parents.’ \textit{OK without verum}

%\exi{A3:} \hl{missing - to the right of LOC}

%\end{xlist}
%\end{exe}

We could attempt to maintain a Zulu-like account of OM-doubling linked with movement out of \textit{v}P by claiming that apparent \textit{in situ} doubling is actually movement to the left edge of \textit{v}P. In fact, it looks like quite the opposite is happening: when a temporal adverb is included that is plausibly analyzed as being adjoined at the left edge of \textit{v}P, the undoubled object preferably occurs to the left of it (\ref{LunoBareObjectMoves}), and the OM-doubled object is preferably to the right (\ref{OmDoubleInSitu}).

\ea 
\begin{xlist}

\ex \label{LunoBareObjectMoves}
\gll Wekesa a-a-w-ele \circled{ba-ba-ana} luno (??) bi-anwa bi-ong'ene. \\
1Wekesa 1\Sm-\Pst-give-\Pfv{} 2-2-children this.time {}  8-gifts 8-only \\
\glt `Wekesa gave the children only gifts this time.' 

\ex \label{OmDoubleInSitu}
\gll Wekesa a-a-\circled{ba-}w-ele (??) \underline{luno} \circled{ba-ba-ana} bi-anwa bi-ong'ene. \\
1Wekesa 1\Sm-\Pst-2\Om-give-\Pfv{} {} this.time 2-2-children 8-gifts 8-only \\
\glt `Wekesa gave the children only gifts this time.' 

\end{xlist}
\z 

Therefore a Zulu-like account of doubled objects vacating the \textit{v}P appears to be unlikely. (\ref{OmDoubleInSitu}) suggests that OM-doubled phrases are quite natural \textit{in situ}.

% It's worth noting that the correlations between word order, doubling, and specificity do not seem to be direct, however. An object bearing a demonstrative may be either to the right or the left of \textit{luno} `this time' (\ref{LunoDemonstrativeNoDoubling}), but an OM-doubled object bearing a demonstrative is preferred to the left of \textit{luno} (\ref{LunoDemonstrativeDoubling}).

% \ea 
% \begin{xlist}

% \ex \label{LunoDemonstrativeNoDoubling}
% \gll Wekesa a-w-ele {[}ba-ba-ana abo{]} \underline{luno} (\Checkmark) \textbf{bi-anwa} \textbf{bi-ong'ene} \\
% 1Wekesa 1\Sm-give-\Pfv{} 2-2-children 2\Dem{} this.time {} 8-gifts 8-only \\
% \glt `Wekesa gave these children only gifts this time.' 

% \ex \label{LunoDemonstrativeDoubling}
% \gll Wekesa a-\circled{ba-}w-ele \circled{babaana \hspace{4mm} abo} \underline{luno} (??) \textbf{bianwa} \textbf{biong'ene} \\
% 1Wekesa 1\Sm-give-\Pfv{} {2-2-children 2\Dem{}} this.time {} 8-gifts 8-only \\
% \glt `Wekesa gave these children only gifts this time.' 


% \end{xlist}
% \z 

% \noindent Therefore there are times when doubled phrases appear to have moved outside \textit{v}P, but this appears to be conditioned by additional factors beyond the OM-doubling itself: in (\ref{LunoDemonstrativeNoDoubling}) the presence of a demonstrative is the main influencer of this movement, as a \textit{v}P-\textit{in-situ} position is preferred for the OM-doubled bare noun phrase in (\ref{OmDoubleInSitu}). 

% \hl{We see the same thing with D-linked wh-phrases: 

% Evidence from whatsapp here}


%A. Wekesa a(?ba)wele luno babaana abo bianwa biong'ene 
%B. Wekesa a(ba)wele babaana abo luno bianwa biong'ene
%C. Wekesa awele luno babaana abo bianwa biong'ene 
%D. Wekesa awele babaana abo luno bianwa biong'ene
%B is now quite good, A worsens. C and D are almost the same...BTW, the correct tense with luno is the immediate past...Wekesa awele....

\section{The beginnings of an analysis} \label{SectAnalysis}

\subsection{Generalizations}\largerpage

\ea 
  \begin{xlist}
  \ex OM-doubled lexical DP objects are interpreted as \textit{specific}.  
  \ex There is a link between OM-doubling and interpretation of those objects as \textsc{aboutness topics}.
  \ex OM-doubling is a generally available operation in Lubukusu, but the pragmatic interpretation of the sentence is highly dependent on the content of \textit{v}P. 
  \end{xlist}
\z 

As noted above, the particular content of \textit{v}P is central to the resulting interpretations from OM-doubling. OM-doubling is possible without verum when the \textit{v}P contains a focused constituent distinct from the doubled object. If there is a constituent in the \textit{v}P distinct from the doubled object, but this remaining constituent does \textit{not} bear focus, the clause receives a verum reading. A range of patterns from \citealt{SikukuEtAl:2018:LubukusuOM} demonstrate this to be the case, as well as all of the sentences marked as \# in this paper. If the doubled object itself bears focus, the sentence is acceptable in (otherwise) neutral pragmatic contexts. But if the doubled object does not bear focus and there is no other focused element in the verb phrase, the sentence requires verum to be acceptable.

\begin{table} 
\caption{Pragmatics of Lubukusu doubling configurations}
\begin{tabular}{cll}
\lsptoprule
focus on/in \textit{v}P? & \textit{v}P configuration & verum-like reading?  \\ \midrule
%yes & {[} \circled{Doubled Object} XP YP {]}\sub{\textit{v}P} & no verum \\ 
yes  & {[} \circled{Doubled Object} \textbf{XP\sub{FOC}} {]}\sub{\textit{v}P} & no verum \\
yes  & {[} \circled{\textbf{Doubled Object\sub{FOC}}}  {]}\sub{\textit{v}P} & no verum \\
\rowcolor{Gray}
no & {[} \circled{Doubled Object} XP {]}\sub{\textit{v}P} & verum \\
\rowcolor{Gray}
no & {[} \circled{Doubled Object}  {]}\sub{\textit{v}P} & verum \\
%yes  & \textbf{\Ne-}{[} \circled{Doubled Object} {]}\sub{\textit{v}P} & no verum 
\lspbottomrule
\end{tabular}
%[   ]\sub{\textit{v}P} & no verum \\
%[   ]\sub{\textit{v}P} & verum \\
\end{table}

\subsection{Toward an informal analysis}

Informally speaking, it appears that OM-doubling activates a conjoint/disjoint-like system, in that it appears to be dependent on overt \textit{v}P content and directly correlates with focus properties of the verb phrase. In this conjoint/disjoint-like system, OM-doubling appears to remove the doubled object from consideration in this system. Apart from the doubled object, then, there are similarities to conjoint/disjoint systems, such as focus on/in the \textit{v}P, patterns of OM-doubling dependent on \textit{overt} \textit{v}P content, and verum/predicate focus in the absence of relevant \textit{v}P content (a common property of disjoint forms) \citep{Guldemann:2003:ProgressiveFocusBantu,VanDerWalHyman:2017:ConjointDisjoint}.
    
That said, there are important distinctions from other such systems. Elsewhere, conjoint/disjoint patterns exist independently of OMing but interact with it; in Lubukusu, the presence of the conjoint/disjoint-like system only emerges when  OM-doubling occurs. Since surface \textit{v}P constituency clearly matters in Lubukusu with respect to OM-doubling, it is tempting to claim that (like Zulu) OM-doubling removes an object from \textit{v}P, and an empty \textit{v}P results in verum focus. This appears to not be the case, however: verum occurs in OM-doubling with an additional \textit{v}P constituent \textit{if} that constituent is unfocused, doubled objects appear to be able to remain \textit{in situ}, and the doubled object ``counts'' with respect to the focus requirement -- a focused object can license OM-doubling on itself. A final point is that intransitive verbs don't show this system of interpretations in Lubukusu. Conjoint/disjoint systems generalize across verbs of different valencies in other languages (i.e. intransitive verbs bear disjoint forms). But in Lubukusu, intransitive verbs don't necessarily require verum readings -- it would be difficult to draw a strict correlation between the verum properties that sometimes result from object marking and an empty \textit{v}P. Rather, it does seem that OM-doubling somehow activates this system.

Our initial analytical thoughts are that OM-doubling is linked with a topic-comment structure inside the \textit{v}P (using the term topic-comment relatively pre-theoretically here to refer to some version of the well-attested distinctions between presupposition and assertion, givenness and focus, theme and rheme). OM-doubling requires an aboutness topic reading of the doubled object because it is generated via Agree with Topic features at the edge of \textit{v}P (precise position to be determined) \citep[see][]{Mursell:2018:SwahiliOmTopic}. However, identification of a \textsc{topic} requires a \textsc{comment} about that \textsc{topic}: the content of \textit{v}P therefore bears focus. We suggest that the focus requirement on \textit{v}P is realized in various ways. If there is one distinct (non-topical) constituent within the \textit{v}P, either its semantics or the discourse context must be naturally compatible with it bearing a focused interpretation; we deem this a pragmatic effect of a single constituent being the entire comment about the topic. If there is no other (non-topical) constituent within the verb phrase, however, verum focus results (interpreted here as focus on the entire predicate itself).

This approach makes several predictions. By analyzing the locus of the focus requirement as \textit{v}P instead of individual constituents, it naturally captures how \textit{any} \textit{v}P-internal constituent can bear focus and license the OM-doubling of a separate object argument. Because there is not in fact a requirement for  term focus inside the \textit{v}P, but instead a focus requirement on the \textit{v}P itself, we would expect \textit{v}P-level properties to be capable of licensing OM-doubling without verum. This is in fact what we find: \S \ref{SectDoublingWithoutFocus} shows that if there are \textit{multiple} non-topical constituents inside the \textit{v}P, these constituents collectively can bear broad focus and are sufficient \textit{v}P-content to license OM-doubling both without verum and without term focus on an individual constituent. \S \ref{SectNeFocusMarker} shows that a (structurally low) predicate focus marker licenses OM-doubling as well.

\subsection{Term focus is unnecessary when \textit{v}P contains sufficient material} \label{SectDoublingWithoutFocus}

As mentioned above, a major prediction of this preliminary analysis is that the strong focal effects on a single constituent should be mitigated if additional constituents are inside \textit{v}P when an object is OM-doubled. This is because on the approach sketched here, those focal effects are only the result of a single constituent serving as the comment about the topic. This is in fact what happens. In the intuitions of the first author, the more things there are in \textit{v}P, the more natural OM-doubling sounds, and term focus becomes unnecessary.

\ea 
%\begin{xlist}

%\ex 
%\gll N-a-w-el-a Wekesa ba-ba-ana bi-anwa. \\
%1\Sg.\Sm-\Pst-give-\Appl-\Fv{} 1Wekesa 2-2-children 8-gifts \\
%\glt `I gave the children gifts for Wekesa.' \textit{OK without verum, without additional context}

%\ex 
\gll N-a-\circled{mu-}w-el-a \circled{Wekesa} ba-ba-ana bi-anwa. \\
1\Sg.\Sm-\Pst-1\Om-give-\Appl-\Fv{} 1Wekesa 2-2-children 8-gifts \\
\glt `I gave the children gifts for Wekesa.' \\ \textit{OK without verum, without additional context}

%\end{xlist}
\z 

\noindent In general, adding more \textit{v}P-level material makes an OM-doubled sentence sound increasingly natural. The following sentences are very natural with OM-doubling and without any exclusive focus on a single constituent:

\ea 
\begin{xlist}

\ex 
\gll N-a-\circled{mu-}w-el-a \circled{Wekesa} ba-ba-ana bi-anwa bulayi. \\
1\Sg.\Sm-\Pst-1\Om-give-\Appl-\Fv{} 1Wekesa 2-2-children 8-gifts well \\
\glt `I gave the children gifts well for Wekesa.' \\ \textit{OK without verum, without additional context}

\ex 
\gll N-a-\circled{mu-}w-el-a \circled{Wekesa} ba-ba-ana bi-anwa likolooba. \\
1\Sg.\Sm-\Pst-1\Om-give-\Appl-\Fv{} 1Wekesa 2-2-children 8-gifts yesterday  \\
\glt `I gave the children gifts yesterday for Wekesa.' \\ \textit{OK without verum, without additional context}

%\ex 
%\gll N-a-mu-w-el-a Wekesa ba-ba-ana bi-anwa enyuma wekhukhumba biachana ne ndi Nairobi \\
%1\Sg.\Sm-\Pst-1\Om-give-\Appl-\Fv{} 1Wekesa 2-2-children 8-gifts \hl{GLOSS} \\
%\glt `I gave the children gifts for Wekesa after he gave them to me when I was in Nairobi.' \\ \textit{OK without verum, without additional context}

\end{xlist}
\z 

%\subsection{Towards a Formalized Analysis}


%\begin{exe}
%\ex

%\Tree [.TopP \qroof{(topic)}.XP [.Top\1 Top^\circ{} \qroof{(comment)}.YP ] ] \\ \citep[286]{Rizzi:1997}

%\end{exe}

%\begin{exe}
%\ex

%\Tree [.TopP \sout{\textsc{obj}}\sub{k} [ {Top^\circ{}\\{[}\ph:obj{]}} \qroof{\ldots \hspace{0.8mm} \sout{\textsc{subj}} \ldots \hspace{0.8mm} \textsc{obj}\textsubscript{[\textsc{top}]}\sub{k} \ldots \hspace{0.8mm} \textbf{XP\textsubscript{\textsc{focus}}} \ldots }.\textit{v}P ]] 

%\end{exe}

\section{\Ne{}-focus marking licenses OM-doubling} \label{SectNeFocusMarker}

An additional piece of evidence supporting a \textit{v}P-level topic-comment approach is \Ne-focus. \citet[335]{Wasike:2007:Thesis} documents a morpheme in Lubukusu that appears on the main verb in compound tenses which he analyzes as wh-agreement, a reflex of A'-movement: 

\ea 
\gll Siina ni-syo mw-a-ba \textbf{ne}-mu-khol-a? \\
7what \Comp-7 2\Pl.\Sm-\Pst-be \textbf{\Ne}-2\Pl.\Sm-do-\Fv{} \\
\glt `What was it that you were doing?'
\z 

\noindent It is clear that \textsc{n(e)-} cannot itself be wh-agreement, as it readily appears in non-extraction contexts:\footnote{There certainly are interactions with extraction: even for the first author on this paper, certain extraction environments make \textsc{n(e)-} obligatory. So there is still work to be done to explain these patterns.} 

\begin{exe}
\ex 
%\begin{xlist}

%\ex \label{CtNeAbsent}
%\gll Wekesa a-ba a-a-nyw-a ka-ma-lwa buli nyanga.\\
%1Wekesa 1\Sm-be 1\Sm-\Pst-drink-\Fv{} 6-6-alcohol every day \\
%\glt `Wekesa used to drink alcohol everyday.'

 \label{CtNePresent}
\gll Wekesa a-ba (\textbf{n})-a-a-nyw-a ka-ma-lwa buli nyanga \\
1Wekesa 1-be \textbf{\Ne}-1\Sm-\Pst-drink-\Fv{} 6-6-alcohol every 9day \\
\glt `Wekesa (certainly) used to drink alcohol everyday.' \\

%\end{xlist}
\end{exe}


The interpretive contribution of \textsc{n(e)-} is hard to pin down, but it has some kind of connection to \textsc{focus} or \textsc{emphasis}. With \textsc{n(e)-}, the speaker is more committed to the truth of (\ref{CtNePresent}). Without \textsc{n(e)-}, (\ref{CtNePresent}) is more or less neutral.\footnote{There are a variety of complex facts related to agreement and extraction around the properties of \textsc{n(e)-}, but for now we focus on a few core properties relevant to OMing.}  

\subsection{OM-doubling in compound tenses with \Ne{}-}

If OM-doubling results in focus on the \textit{v}P, OM-doubling should be acceptable if the \textit{v}P is focused independently of its internal content. OM-doubling sounds natural with the \textsc{n(e)-} focus morpheme in a compound tense (without verum). 

\begin{exe}
\ex 
\begin{xlist}

\ex 
\gll Wekesa a-ba a-a-\circled{(\#ka)-}nyw-a \circled{ka-ma-lwa} buli nyanga.\\
1Wekesa 1\Sm-be 1\Sm-\Pst-6\Om-drink-\Fv{} 6-6-alcohol every 9day \\
\glt `Wekesa used to drink alcohol everyday.' \\ \textit{Requires verum for OM-doubling to be acceptable}

\ex \label{BasicExDoublingWithNe}
\gll Wekesa a-ba \textbf{n}-a-a-\circled{ka-}nyw-a \circled{ka-ma-lwa} buli nyanga \\
1Wekesa 1-be \textbf{\Ne}-1\Sm-\Pst-6\Om-drink-\Fv{} 6-6-alcohol every 9day \\
\glt `Wekesa (certainly) used to drink alcohol everyday.' \\ \textit{Doubling OK without verum}

\end{xlist}
\end{exe}

%\begin{itemize}

%\item Doubling sounds better with the focus marker; there is added confidence about the truth of the assertion in (\ref{BasicExDoublingWithNe}) above. 

%\item This strengthens the possibility that this is a focus morpheme, perhaps historically connected to the predicate focus morphemes that appear in Kuria, Kikuyu, and Tharaka.

%\end{itemize}

\subsection{\Ne{}- and imperatives (and doubling)}

The focus morpheme \textsc{ne}- can also occur on imperatives; this tends to have the interpretive effect of increasing the force/urgency of the speaker's command. 

\begin{exe}
\ex 
\begin{xlist}

%\ex 
%\gll Nyw-a echai y-oo! \\
%drink-\Fv{} 9tea 9-your \\
%\glt `Drink your tea!'

\ex 
\gll kh-o-nyw-e echai yoo   \\
\textsc{kh}-2\Sg.\Sm-drink-\Subj{} 9tea 9your \\
\glt `Drink your tea.' 

\ex 
\gll \textbf{n}-o-nyw-e echai yoo! \\
\Ne-2\Sg.\Sm-drink-\Subj{} 9tea 9-your \\
\glt `Drink your tea!'

\end{xlist}
\end{exe}


OM-doubling the object is acceptable but requires a verum interpretation (\ref{ImperativeDoublingVerum}). If \textsc{ne}- is used, doubling does not require a verum interpretation (\ref{ImperativeNeDoublingNoVerum}).

\begin{exe}
\ex 
\begin{xlist}

\ex \label{ImperativeDoublingVerum}
\gll \#\circled{Ki-}nyw-e \circled{echai yoo}!  \\
9\Om-drink-\Subj{} {9tea \hspace{2mm} 9-your} \\
\glt `Drink your tea!' \textit{Requires verum, i.e. `DO drink your tea.'}

\ex \label{ImperativeNeDoublingNoVerum}
\gll \textbf{n}-o-\circled{ki-}nyw-e \circled{echai yoo}! \\
\Ne-2\Sg.\Sm-9\Om-drink-\Subj{} {9tea \hspace{2mm} 9-your} \\
\glt `Drink your tea!' \textit{OK without verum}

\end{xlist}
\end{exe}

%\noindent The same pattern emerges in an imperative construction with an additional argument. The same patterns of imperatives in subjunctive and not, with NE and not, occurs here:  

%\begin{exe}
%\ex 
%\begin{xlist}

%\ex 
%\gll Ra murere khumesa! \\
%gloss here \\
%\glt `put the murhere on the table!'
 
%\ex 
%\gll khore murere khumesa \\
%gloss here \\
%\glt `put the murhere on the table (subjunctive)'

%\ex 
%\gll No-re murere khumesa!\\
%gloss here \\
%\glt `Ne-put the murhere on the table!'

%\end{xlist}
%\end{exe}


%\begin{itemize}

%\item As we might expect at this point, doubling sounds more natural with an additional verbal argument than with a lone object as in the examples above (\ref{ImperativeDoublingThemeVerumISH}). 

%\item When that argument is focused, OM-doubling the theme is much more natural (\ref{ImperativeDoublingThemeFocusedLocation}).  

%\item But even without this additional focus on the location, if \textsc{ne-} is present OM-doubling can occur naturally without a verum reading: 

%\end{itemize}

%\begin{exe}
%\ex 
%\begin{xlist}

%\ex \label{ImperativeDoublingThemeVerumISH}
%\gll \#Ki-r-e murere khu-mesa! \\
%9\Om-put-\Subj{} 9murere 17-9table \\
%\glt `OM-put the murere on the table! \textit{OK with verum} 
%(without verum better than the ‘tea’ example above)

%\ex \label{ImperativeDoublingThemeFocusedLocation}
%\gll ?Ki-r-e murere khu-mesa khw-ong’ene! \\
%9\Om-put-\Subj{} 9murere 17-9table 17-only \\
%\glt `Put murere only on the table (nowhere else).' \textit{(OK without verum, but slightly marginal compared to (\ref{ImperativeDoublingNe}))}

%\ex \label{ImperativeDoublingNe}
%\gll N-o-ki-r-e murere khumesa! \\
%\Foc-2\Sg.\Sm-9\Om-put-\Subj{} 9murere 17-9table \\
%\glt `Put the murere on the table!' \textit{OK without verum}

%\end{xlist}
%\end{exe}

\subsection{Intermediate summary, \Ne-focus}

What we see from this section, then, is that the \Ne-focus marker is capable of licensing OM-doubling an object independently of any other focused phrase inside the verb phrase. Like the pattern discussed in \S \ref{SectDoublingWithoutFocus}, this is another process centered on the broader verb phrase itself, rather than any particular constituent inside the verb phrase. This therefore further supports an approach where the focus requirement for OM-doubling applies to the verb phrase as a whole, despite the fact that it is often realized by term focus on an individual constituent inside the verb phrase. 

\section{Conclusions} \label{SectConclusions}

% In this section we very briefly review our findings in this paper (and how they build on our previous work), and discuss the direction that the research continues to take.

\subsection{Empirical generalizations}

The main contribution of this paper is in expanding the empirical generalizations on the properties of OMing and OM-doubling in Lubukusu. \REF{SummaryEmpiricalGen2018} summarizes the relevant pre-existing empirical generalizations that were arrived at in \citet{SikukuEtAl:2018:LubukusuOM}, and \REF{NewEmpiricalGen2020} summarizes the new empirical generalizations reached in this paper.

\ea Selected empirical generalizations from \citet{SikukuEtAl:2018:LubukusuOM} \label{SummaryEmpiricalGen2018}
   \begin{xlist}
   \ex Doubling in simple monotransitives is unacceptable in neutral discourse contexts.
   \ex Doubling in simple monotransitives requires a verum-licensing context to be acceptable.    
   \end{xlist}
   \z 
   
\ea New generalizations: Lubukusu OM-doubling \label{NewEmpiricalGen2020}
  \begin{xlist}
  \ex OM-doubled lexical DP objects are interpreted as \textsc{specific}.  
  \ex Objects that are interpreted as \textsc{aboutness topics} require OM-doubling.
  \ex OM-doubling is a generally available operation in the language, but the pragmatic interpretation of the sentence is highly dependent on the content of \textit{v}P.
  \end{xlist}
\z 

As for the pragmatic effects in particular, if there are 2+ distinct constituents in the \textit{v}P other than the doubled object, there are no discernible pragmatic effects (i.e. no focus effects). OM-doubling is possible without verum when the \textit{v}P contains a focused constituent distinct from the doubled object. If there is a constituent in the \textit{v}P distinct from the doubled object, but this remaining constituent does \textit{not} bear focus, the clause receives a verum reading. If the doubled object bears focus, the sentence is acceptable in neutral pragmatic contexts. If the doubled object does not bear focus, the sentence requires verum to be acceptable. We have also identified one additional pathway to non-verum OM-doubling: if the verb bears the \textsc{n(e)-} emphatic marker, OM-doubling is natural without any additional term-level focus inside the \textit{v}P (this \textsc{n(e)-} morpheme only appears in a compound tense or imperative). \tabref{DoublingConfigurationsChart} sketches the core configurations of OM-doubling that we considered in this paper, including their interactions with focus effects. 

\begin{table}
\caption{Pragmatics of Lubukusu doubling configurations\label{DoublingConfigurationsChart}}
\begin{tabular}{cll}
\lsptoprule
focus on/in \textit{v}P? & \textit{v}P configuration & verum-like focus?  \\ \midrule
yes & {[} \circled{Doubled Object} XP YP {]}\sub{\textit{v}P} & no verum \\ 
yes  & {[} \circled{Doubled Object} \textbf{XP\sub{FOC}} {]}\sub{\textit{v}P} & no verum \\
yes  & {[} \circled{\textbf{Doubled Object\sub{FOC}}}  {]}\sub{\textit{v}P} & no verum \\
\rowcolor{Gray}
no & {[} \circled{Doubled Object} XP {]}\sub{\textit{v}P} & verum \\
\rowcolor{Gray}
no & {[} \circled{Doubled Object}  {]}\sub{\textit{v}P} & verum \\
yes  & \textbf{\Ne-}{[} \circled{Doubled Object} {]}\sub{\textit{v}P} & no verum \\
\lspbottomrule
\end{tabular}
%[   ]\sub{\textit{v}P} & no verum \\
%[   ]\sub{\textit{v}P} & verum \\
\end{table}

\subsection{Future research}\largerpage[-1]

There are a number of persistent analytical questions, and the work is ongoing. First, the project is still a work in progress; as we mentioned above, our current direction of analysis is to analyze OM-doubling as a result of topic agreement or givenness agreement, but the nature of topicality/givenness puts explicit semantic requirements on the complement of the head generating this agreement. As pointed out to us by Rose Marie Déchaine, these Lubukusu patterns show a large degree of similarity to  destressing patterns in English focalization (an area of particularly intense analytical and theoretical work) \citep{Wagner:2012:FocusGivenness,Williams:1997:BlockingAnaphora,Schwarzschild:1999:Givenness}.

\citet{Wagner:2012:FocusGivenness} proposes a semantics of accent shift in English that analyzes givenness and focus as two mutually necessary sides of the same coin: marking something as given necessitates marking something else as focused. Our ongoing work looks to integrate these observations from the English destressing/focus literature with the properties of Lubukusu OM-doubling.

Beyond the analysis itself, there are a number of empirical domains to be looked into, including investigating properties of variable word order postverbally both with and without OM-doubling. The patterns are quite complex and finding reliable diagnostics of syntactic position has been a challenge, but we continue to work in this area. Likewise we continue to investigate the properties of \Ne-emphasis and to look for additional diagnostic contexts to specifically clarify the interpretation of the OM-doubling itself. 

As raised by several reviewers, there are two important areas of research on object marking that need additional work. First, an active area of research is whether both objects in a ditransitive may be OM-doubled (and if so, under what conditions: i.e. whether object marking is (a)symmetrical). Our ongoing research suggests that structurally lower objects in Lubukusu may be OM-doubled, but that this significantly changes the focus properties of the sentence (restricting the focus to only the OM-doubled object). This intersects with another question raised by a reviewer: we have shown a few instances where an OM-doubled object itself may bear focus, rather than some other element in the verb phrase (e.g. (\ref{DemonstrativeImprovesDoublingFocusRight})). This does not straightforwardly translate to a topic/focus bifurcation in the verb phrase where the OM-doubled element is the topic and something else is the focus. While addressing this goes far beyond what we can accomplish in this short paper, we expect that both of these empirical areas will be central to resolving a precise analysis of Lubukusu OMing and are a part of the investigation in \citet{SikukuDiercks2020BukusuOmBook}.  

%\footnote{\citet{Zeller:2014:ZuluOm3Types} places a high importance on the ability/requirement (or not) of object relative clauses to OM the head of their relative clause in analyzing the nature of the OM. We are still investigating Lubukusu OMing with respect to these patterns, and the data are puzzling (at present, it appears that given conditions expected to license OM-doubling, OMing the extracted object of an object relative clause is impossible without verum, but possible with verum, despite the presence of a focused phrase in \textit{v}P).} 

Work is still underway, but it appears that very similar patterns appear in Wanga, Tiriki, and Logoori (which are all Luyia languages). That said, we have encountered speakers of each of these varieties who appear to lack these patterns, instead appearing to allow what looks like an incorporated pronoun analysis of OMs (doubling is always impossible). Given the deep contextual dependence of these patterns, it's impossible to rule out pragmatic licensing of some sort for those speakers, but the best we can tell, some speakers of these languages completely lack these patterns. So while we can say (based on our preliminary work in these other languages) that patterns like this are relatively broadly attested in Luyia languages, it's unclear whether it is appropriate to say they are pervasively present among all speakers of any particular language (including Lubukusu).  

\section*{Abbreviations}

Below are listed only those abbreviations that do not adhere to or are beyond the scope of the Leipzig Glossing Rules.

%\printglossaries
\begin{multicols}{3}
\begin{tabbing}
AAAA    \=BBBBBBBBBBBBBB  \kill
\Aug{}  \>Augment        \\
\Cj{}	\>Conjoint       \\
\Dj{}	\>Disjoint       \\
\Fv{}   \>Final vowel    \\
\Om{}	\>Object marker   \\
\Rem{}	\>Remote	      \\
\Sm{} 	\>Subject marker \\
\end{tabbing}
\end{multicols}

\section*{Acknowledgments}

This research builds on \citet{SikukuEtAl:2018:LubukusuOM}, and we gratefully acknowledge the contributions of Michael Marlo in this ongoing work. We have benefited from comments and criticisms from Rose-Marie Déchaine, Travis Major, John Gluckman, Mark Baker, Vicki Carstens, Jesse Harris, Ruth Kramer, Rodrigo Ranero, Kristina Riedel, Ken Safir, Jenneke van der Wal, and Jochen Zeller. Audiences in many venues have contributed much to our understanding of these phenomena, most recently the audience at ACAL 50 at UBC. We also extend our gratitude to the two anonymous reviewers of this paper and to the editorial team as well. Portions of this research were funded from a Doctoral Dissertation Research Improvement Grant (BCS-0843868), a Hirsch Research Initiation Grant from Pomona College, a NSF Collaborative Research Grant (Structure and Tone in Luyia: BCS-1355749), and Pomona College (including ongoing support from the Paul and Susan Efron fund and the Robert Efron Lectureship in Cognitive Science at Pomona College). The first author provided the Lubukusu data in the paper. The first and second author jointly identified the empirical generalizations, developed the (fledgling) analysis, and developed the argumentation in the paper. The second author took primary responsibility for putting these conclusions into written form.

{\sloppy\printbibliography[heading=subbibliography,notkeyword=this]}
\end{document}
