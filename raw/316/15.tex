\documentclass[output=paper]{langscibook} 
\ChapterDOI{10.5281/zenodo.5082478}

\author{Flóra Lili Donáti\affiliation{SFL, Université Paris 8} and
Yasutada Sudo\affiliation{University College London}}

\title[\textup{Even} superlative modifiers]{\textit{Even} superlative modifiers}  

\abstract{We observe that numerals with superlative modifiers -- \textit{at least} and \textit{at most} -- are systematically unacceptable with certain focus particles, most notably \textit{even}. We analyze the infelicity of such sentences as arising from a clash between the presupposition of the focus particle and the obligatory implicature of the superlative modifier. We claim that to obtain these results it is crucial to make the following two assumptions: (i) the set of alternatives that focus particles operate on is generated by the same mechanism as the set of alternatives for implicatures, and (ii)~additive presuppositions are \textit{de re}.

\keywords{superlative modifiers, even, additive presupposition, ignorance implicature, alternatives}}

\begin{document}
\maketitle



\section{Introduction}

The main puzzle that we would like to grapple with in this paper consists in the observation that numerals with superlative modifiers -- \textit{at least} and \textit{at most} -- are unacceptable with focus particles like \textit{even}, as demonstrated by \REF{don-sud:langs}. We employ nominal ellipsis in this example to force the intended, narrow focus structure. We mark the focused element by $F$ throughout this paper.

    \ea
    I speak two languages.
    \#James even speaks [at least five]$_F$.\label{don-sud:langs}
    \z

\noindent To show that the infelicity of sentences like this is indeed a puzzle, let us go through some similar cases. Firstly, observe that when associating with a bare numeral, \textit{even} means that the number is big in a given situation. 

    \ea[]{I speak two languages. James even speaks five$_F$.\\
    $\leadsto$ James speaks many languages \label{don-sud:even-more}}
    \z
  
\noindent A similar inference is observed with a comparative modifier.

    \ea[?]{I speak two languages. James even speaks [more than four]$_F$.\\
    %\phantom{(?)}
    $\leadsto$ James speaks many languages
    \label{don-sud:comp}}
    \z
   
\noindent Although some of the speakers of English we consulted do not like \REF{don-sud:comp} as much as \REF{don-sud:even-more}, all of them judge \REF{don-sud:langs} to be worse. Crucially, the contrast between \REF{don-sud:langs} and \REF{don-sud:comp} suggests that the intended meaning of \REF{don-sud:langs} itself is not the source of its infelicity.

    It is also important to point out that focussing a numeral with a superlative modifier does not necessarily result in infelicity. Concretely, when the focus is interpreted broadly, \textit{even}\,+\,\textit{at least $n$} becomes felicitous, as the following example demonstrates.

    \ea James did everything to impress the interviewers.\\He sang songs in three different languages, and even [answered questions in at least five]$_F$ during the interview.\label{don-sud:broad}
    \z

\noindent Furthermore, we observe that \textit{only}, another focus particle, can felicitously associate with \textit{at most $n$}, as well as \textit{fewer than $n$}, as shown in \REF{don-sud:onlyde}.\footnote{It turns out that \textit{even} cannot felicitously associate with \textit{at most $n$}, and \textit{only} cannot felicitously associate with \textit{at least $n$}. But we think that these cases need a separate explanation, as their comparative counterparts are also infelicitous. We will discuss relevant examples and sketch an analysis in the appendix.\label{don-sud:fn:appendix}}

    \ea I speak five languages. James only speaks $\left\{\begin{array}{@{\,}l@{}}\text{a. {[at most three]$_F$}.}\\\text{b.\ {[fewer than four]$_F$}.}\end{array}\right.$
    \label{don-sud:onlyde}
    \z

\noindent These observations suggest that the infelicity of examples like \REF{don-sud:langs} is not due to failure of focus association. Then, why can \REF{don-sud:langs} not mean something similar to \REF{don-sud:comp}?\largerpage

We claim that the culprit is a conflict between the \textit{obligatory ignorance implicature} of \textit{at least five} and the \textit{additive presupposition} of \textit{even}. Simply put, the ignorance implicature of \REF{don-sud:langs} implies that the speaker is not sure whether James speaks exactly five languages or more than five languages, but its additive presupposition requires that the speaker be sure that James speaks $n$ languages, for some particular number $n\geq 5$. Evidently these two inferences cannot hold at the same time.

We furthermore claim that in order to obtain this result, it is necessary to assume that the set of alternatives that focus particles operate on is generated by the same mechanism as the set of alternatives used for computing implicatures, as previously proposed by \citet{rooth:92} and \citet{foxkatzir} on independent grounds, and that the additive presupposition of \textit{even} is \textit{de re}, in the sense to be made clear later (\citealt{kripke}).

The paper is organized as follows. We will first discuss the semantics and pragmatics of superlative modifiers in detail in \sectref{don-sud:sec:ignorance}, and the presuppositions that \textit{even} triggers in \sectref{don-sud:sec:even}. In \sectref{don-sud:sec:analysis}, we will then put these two ingredients together to show how the infelicity of examples like \REF{don-sud:langs} can be accounted for. We will also discuss some predictions of our analysis there. \sectref{don-sud:sec:conclusion} contains conclusions and remarks on some additional open questions.


\section{The implicatures of numerals with superlative modifiers}
\label{don-sud:sec:ignorance}

\subsection{The ignorance inference as an obligatory implicature}

One of the notable characteristics of numerals with superlative modifiers is that they often give rise to \textit{ignorance inferences} very robustly (\citealt{krifka, buring, geurts-nouwen}, among others). Concretely, consider the following examples.

    \ea
        \begin{xlist}
        \ex[??]{I have at least three children.}
        \ex[??]{I have at most four children.}
        \end{xlist}
    \z

\noindent These examples very strongly suggest that the speaker does not know the exact number of his or her children, which, in normal circumstances, is perceived to be odd. A similar remark applies to the following examples.

    \ea
        \begin{xlist}
        \ex[??]{A triangle has at least two sides.}
        \ex[??]{A triangle has at most four sides.}
        \end{xlist}
    \z

\noindent What exactly is the content of the ignorance inference of a numeral with a superlative modifier? It is clear that it is not ignorance about every number in the range of the modified numeral. That is, \REF{don-sud:children2} below does not imply that for each number $n$ greater than two, the speaker does not know whether or not Jacopo has exactly $n$ many children, schematically: $\forall n> 2 [\neg B(n) \land \neg B(\neg n)]$, where each $n$ represents the proposition that Jacopo has exactly $n$ children, and $>$ orders these propositions according to the natural order of natural numbers. This is evidently too strong, because the sentence is perfectly felicitous even when the speaker is sure that Jacopo does not have 10 or more children, for example.

    \ea Jacopo has at least three children. \label{don-sud:children2}\z

\noindent Similarly, the ignorance inference is not that for each number $n$ greater than two, the speaker either believes the negation of the proposition that Jacopo has exactly $n$ many children or is not certain about the truth of this proposition, schematically: $\forall n> 2 [B(\neg n) \lor \neg B(n)]$. This is weaker than the previous hypothesis, but it is now too weak, because this is compatible with the speaker believing that Jacopo does not have exactly three children, and has at least four, as long as he or she is not certain about any number above three. This is a bad prediction, as the sentence is perceived as infelicitous if that is the case.

The ignorance inference of \REF{don-sud:children2} can more aptly characterized as follows (see \citealt{buring, mayr, schwarz}): the speaker is uncertain about whether or not Jacopo has exactly three children, and about whether or not he has more than three children, schematically: $\neg B(3) \land \neg B(\neg 3) \land \neg B(\mathord{>}3) \land \neg B(\neg \mathord{>}3)$. Similarly, the ignorance inference of \REF{don-sud:children3} is that the speaker is uncertain about whether or not Jacopo has exactly three children, and whether or not he has fewer than three children, schematically: $\neg B(3) \land \neg B(\neg 3) \land \neg B(\mathord{<}3) \land \neg B(\neg \mathord{<}3)$.

    \ea Jacopo has at most three children. \label{don-sud:children3}
    \z
    
\noindent We will assume these characterizations of the ignorance inferences of superlative modifiers in the rest of the paper.

Previous studies on this topic, furthermore, regard the ignorance inference of a superlative modifier to be a kind of implicature, and we adopt this idea (\citealt{buring, mayr, schwarz, buccolahaida:18, mendia}; see also \citealt{geurts-nouwen, coppock-brochhagen, cohen-krifka} for other related ideas). Empirical support for this analysis comes from the observation that it exhibits characteristic properties of implicatures with respect to certain linguistic operators. For instance, under a necessity operator, the ignorance inference can disappear.

    \ea Andy doesn't need to write papers, but Patrick needs to write at least three.\label{don-sud:need}\z

\noindent This example has a reading without ignorance inferences (in addition to one with ignorance inferences). Instead, it has a scalar implicature implying that it is ok if Patrick writes exactly three papers, and it is also ok if Patrick writes more than three papers.

This behavior is reminiscent of more familiar cases of (generalized) implicatures that arise from items like \textit{or}. Specifically, \textit{or} gives rise to ignorance implicatures and a scalar implicature in sentences like the following.
\ea Katie speaks French or German.\z
The ignorance implicatures of this example are that the speaker does not know whether or not Katie speaks French or whether or not she speaks German, and the scalar implicature is that Katie does not speak both French and German. When embedded under a universal quantifier, these ignorance implicatures turn into scalar implicatures, as demonstrated by \REF{don-sud:frgr}.

\ea Katie is required to speak French or German.\label{don-sud:frgr}\z

\noindent That is, \REF{don-sud:frgr} has a reading with scalar implicatures that Katie is not required to speak French and that she is not required to speak German. As we will discuss below, this observation is standardly accounted for by theories of scalar implicatures. Given the parallel behavior exhibited by numerals with superlative modifiers, it would be desirable to extend the scalar implicature approach to them as well.

Before moving on, it should be remarked that implicatures of this kind are very robust, especially in comparison to particularized conversational implicatures, and sometimes even considered to be obligatory. To capture this, it could be hypothesized that \textit{or} and superlative modifiers obligatorily activate alternatives and demand some inference to be derived from them, for example. This is a well-discussed issue in the current theoretical literature, and why that is so is far from settled and different views have been proposed in different theoretical frameworks (see, for example, \citealt{levinson, magri, schwarz, buccolahaida}).  For the purposes of this paper, fortunately, we need not make theoretical commitments regarding this issue, although as we will discuss now, we will have to make specific assumptions about the alternatives that superlative modifiers activate.

\subsection{Alternatives of superlative modifiers}

We assume the assertive meanings of \textit{at least $n$} and \textit{at most $n$} to be simply lower-bounded at $n$ and upper-bounded at $n$, respectively. The compositional details of how that is derived do not matter much here (but see \sectref{don-sud:sec:conclusion}). To derive the ignorance inference of a superlative modifier as an implicature, previous studies postulate particular sets of implicature alternatives for them (\citealt{krifka,buring,mayr,schwarz,mendia}; see also \citealt{coppock-brochhagen}). We adopt the following idea from \citet{buring} and \citet{schwarz}.

    \ea  \label{don-sud:alts}
        \ea \cnst{alt}(\qq{\text{at least $n$}}) = \{\qq{\text{at least $n$}},\qq{\text{at least $n+1$}}, \qq{\text{exactly $n$}}\}
        \ex \cnst{alt}(\qq{\text{at most $n$}}) = \{\qq{\text{at most $n$}}, \qq{\text{at most $n-1$}}, \qq{\text{exactly $n$}}\}
        \z
    \z
    
\noindent Note that the assertive meanings of the alternatives \textit{at least} $n+1$ and \textit{exactly} $n$ are independent from each other, but both of them are stronger than that of \textit{at least} $n$ (in terms of generalized entailment). Similarly, the assertive meanings of \textit{at most} $n-1$ and \textit{exactly} $n$ are independent from each other, but are both stronger than that of \textit{at most} $n$.

Notice importantly that if these stronger alternatives are both negated, the overall meaning will be contradictory. In order to see this, consider \REF{don-sud:ii-ex}.

    \ea James speaks at least five languages. \label{don-sud:ii-ex}
        \ea Alternatives: James speaks at least six languages.
        \ex Alternatives: James speaks exactly five languages.
        \z
    \z

\noindent The assertive meaning of \REF{don-sud:ii-ex} is that James speaks five or more languages. If the first alternative is negated, it will imply that James speaks exactly five languages. If the second alternative is also negated, then, the entire meaning will be contradictory.

Generally, when there are non-weaker alternatives that cannot be negated simultaneously while maintaining consistency with the assertion, each of them gives rise to an ignorance implicature \citep{sauerland, fox, mayr, meyer, schwarz}. This is exactly how the ignorance implicatures of \textit{or} are accounted for. For instance, consider the following example with the three alternatives given here.

    \ea Katie speaks French or German.\label{don-sud:or}
        \ea Alternative: Katie speaks French.\label{don-sud:fr}
        \ex Alternative: Katie speaks German.\label{don-sud:gr}
        \ex Alternative: Katie speaks French and German.\label{don-sud:frandgr}
        \z
    \z

\noindent Among these alternatives, \REF{don-sud:fr} and \REF{don-sud:gr} cannot be negated simultaneously to maintain consistency with what is asserted, and they indeed give rise to ignorance implicatures that the speaker does not know whether or not Katie speaks French, and does not know whether or not she speaks German.

The classical way to derive ignorance implicatures is by resorting to the maxim of quantity.\footnote{Alternatively, we could use a ``grammatical theory'' of ignorance implicatures \citep{meyer, buccolahaida} without any crucial changes in our analysis.} Notice that in the above example, all three alternatives are stronger than what is asserted. It is reasonable to assume that utterances of these alternatives would have respected the maxim of manner and the maxim of relevance, so given the speaker must be obeying the maxim of quantity, it must be the case that the speaker would have flouted the maxim of quality. What this implies is that the speaker's beliefs do not support the truths of these alternatives.  Together with the assumption that the speaker respects the maxim of quality and so believes the truth of what she asserted, this amounts to the ignorance implicatures of the sentence.

Now, using the same mechanism, we can derive the ignorance implicatures of numerals with superlative modifiers. They simply amount to the fact that the speaker's beliefs do not entail the truths of the stronger alternatives to the prejecent. Together with the assertive meaning of the prejacent, the overall meaning entails the ignorance inferences we wanted to derive.

In the case of \textit{or} there is also a scalar implicature to be accounted for. For \REF{don-sud:or} above, for example, the scalar implicature is that \REF{don-sud:frandgr} is false. This needs an additional explanation.  \citet{sauerland}, for example, assumes that scalar implicatures are also derived from ignorance implicatures by additional reasoning called the epistemic step, which strengthens the above quantity implicatures to the speaker's beliefs about the falsity of the alternatives, as long as consistency with the rest of the meaning can be maintained. Alternatively, one could assume that scalar implicatures are derived by a separate mechanism, as proposed by \citet{fox}, for example (see also \citet{buccolahaida} for more discussion). According to \citet{fox}, the scalar implicatures are first generated by negating all the alternatives that can be negated while maintaining consistency, and then those that were not negated in this process give rise to scalar implicatures.

For the purposes of this paper, we do not have to choose between these theoretical options, but one nice consequence of the implicature approach we are considering here is that it also explains with the same set of alternatives cases where scalar implicatures are observed instead of ignorance implicatures, such as \REF{don-sud:need} and \REF{don-sud:frgr}. Let us consider the former example (the analysis of the latter is parallel). The relevant alternatives are:

\ea
  \ea Patrick needs to write exactly three (papers).
  \ex Patrick needs to write at least four (papers).\z
\z

\noindent Since the negations of these alternatives are consistent with what is asserted, they give rise to scalar implicatures, rather than ignorance implicatures.\footnote{As we remarked in passing, \REF{don-sud:need} also allows for a reading with an ignorance implicature. One way to derive this is by assigning wider scope to \textit{at least three}, above the necessity modal, but the compositional details are a little complicated, as the (implicit) existential quantifier should stay under the scope of the modal. See, for example, \citet{krifka, hackl, beck} for relevant discussion.}


\section{The presuppositions of \textit{even}}
\label{don-sud:sec:even}

Let us now discuss the second ingredient, the presuppositions of \textit{even}. It is standardly considered that the focus particle \textit{even} triggers two presuppositions, an \textit{additive presupposition} and a \textit{scalar presupposition}, based on a contextually relevant set of focus alternatives to the sentence it modifies, the \textit{prejacent} (see \citealt{karttunenpeters, rooth, kay, wilkinson, crnic}, among others).\footnote{For our purposes we can assume that \textit{even} always takes propositional scope. Depending on one's syntactic assumptions, some examples might require covert movement of \textit{even}, but we could also dispense with such a scope-taking mechanism by type-generalizing the meaning given here, as done by \citet{rooth} (see also \citealt{panizzasudo}).\label{don-sud:scope}}

    \ea\label{don-sud:evensem} \qq{\text{Even $\phi$}} presupposes:
        \ea $\phi$ is relatively unlikely among $\cnst{alt}(\phi)$ \hfill \text{Scalar}
        \ex $\psi$ is true, for some $\psi\in \cnst{alt}(\phi)$ that is not entailed by $\phi$ \hfill \text{Additive}
        \z
    \z

\noindent A couple of remarks are in order. Firstly, we state the scalar presupposition in terms of likelihood, but whether or not this is an accurate characterization in the general case is highly controversial and alternative ideas have been put forward that make use of other kinds of ordering among alternatives \citep{rooth, kay, herburger, greenberg}. Furthermore, these previous studies do not agree on how exactly the alternatives are quantified over. Specifically some argue that \textit{all} the alternatives distinct from the prejacent must be ranked higher with respect to the relevant ordering, while others assume something weaker like we do above, or even weaker with existential quantification. There is no consensus on these issues in the literature, and we certainly cannot settle them in this paper, so we remain somewhat loose on these points. Therefore, our account to be developed below should ideally not rely on a particular way of stating the scalar presupposition.


Secondly, there is a separate debate as to whether the additive presupposition is actually part of the core semantics of \textit{even} or it  comes from something else \citep{rullmann, crnic, francis}. One of the main reasons to think that it is not inherently part of the semantics of \textit{even} is that the additive presupposition does not seem to be present in certain examples, although the judgments might not be stable across speakers, as noted by \citet{francis}. For now, we treat the additive presupposition as part of the semantics of \textit{even}, as in \REF{don-sud:evensem}, and discuss relevant cases and issues they pose for our account at the end of the paper.

It should also be noted that we state the additive presupposition in a particular way, namely as a \textit{de re} presupposition, rather than as an existential presupposition about the existence of a true non-entailed alternative. We will come back to this point, after presenting our analysis in the next section.

Now, let us illustrate how the above semantics of \textit{even} works with a simple example in \REF{don-sud:dance}.

    \ea Even James$_F$ danced.\label{don-sud:dance}
    \z

\noindent We follow \citet{foxkatzir} in assuming that focus alternatives are contextually relevant expressions that are obtained by replacing the $F$-marked constituent with alternative expressions. Without loss of generality, let us assume the following set of alternatives here.
     \[\left\{\begin{array}{ll}
    \text{James danced}, & \text{Katie danced},\\
    \text{Lucas danced}, & \text{Ruoying danced}\end{array}\right\}\]

\noindent The scalar presupposition is that James was relatively unlikely to dance, compared to the other people mentioned here, and the additive presupposition requires there to be someone else than James that danced. This seems to capture the intuitive meaning of the sentence in \REF{don-sud:dance}.


\section{Analysis}
\label{don-sud:sec:analysis}

\subsection{Putting the ingredients together}

With what we discussed in the previous two sections, we are now ready to come back to our main puzzle. We will use the following sentence as a representative example.

    \ea[\#]{James even speaks [at least three]$_F$ languages.\label{don-sud:rep}}\z

\noindent What are the focus alternatives that \textit{even} operates on here? Following \citet{foxkatzir}, we crucially assume that the alternatives that focus particles operate on and the alternatives used for computing implicatures are generated by the same mechanism. Concretely, \textit{even} in \REF{don-sud:rep} will operate on the following set of alternatives.
  \[\left\{\begin{array}{l}
    \text{James speaks at least three languages,}\\
    \text{James speaks {at least four} languages,}\\
    \text{James speaks {exactly three} languages}
  \end{array}\right\}\]

\noindent With this set of alternatives, let us compute the scalar and additive presuppositions predicted by the semantics of \textit{even} reviewed in the previous section. If either of them is not satisfiable, we have an account of the infelicity of the sentence.\largerpage

The scalar presupposition will be that the prejacent of \textit{even}, i.e.\ the top sentence in the above set, is relatively unlikely to be true. Note that this is unsatisfiable, because it is the weakest element in this set in the sense that the other two alternatives asymmetrically entail it. Since probability is monotonic with respect to entailment, the prejacent can at most be as likely as the other two, and cannot be less likely.

However, we are reluctant to see this as a satisfactory account of the infelicity of \REF{don-sud:rep}. As we mentioned in the previous section, there is a debate about how the scalar presupposition of \textit{even} should be stated, in particular, with respect to which ordering to use. The above explanation depends crucially on the monotonicity of probability with respect to entailment, but if the scalar presupposition turns out to be able to use ordering that is not monotonic with respect to entailment, the scalar implicature may actually come out as satisfiable. In fact, such notions as remarkableness or noteworthiness are non-monotonic with respect to entailment and seem to be good candidates for the semantics of \textit{even}.

Moreover, a more empirical reason to eschew this explanation comes from the fact that \textit{only} can felicitously modify \textit{at most $n$}, as we saw in \REF{don-sud:onlyde}. When associating with a numeral, \textit{only} generally triggers a scalar inference that the amount in question is small. The acceptability of the inference in \REF{don-sud:onlyde} suggests that a scalar inference and the ignorance inference of a superlative modifier are compatible with each other.

For these reasons, we think that the scalar presupposition is actually not problematic after all. Rather, we propose that the real culprit is the additive presupposition. We will present additional evidence that this is the case later that comes from an additive particle like \textit{too}, but let's first see how it can render the example in \REF{don-sud:rep} infelicitous.

The additive presupposition says of at least one alternative that is not weaker than the prejacent that it is true. In the above set, therefore, either it is presupposed that James speaks at least four languages or that James speaks exactly three languages.

Turning now to the ignorance implicatures of \REF{don-sud:rep} there are two candidates for the set of alternatives: (i) the set of alternatives is identical to the set we considered above for computing the presupposition of \textit{even}, or (ii) it is the following set, where each member contains \textit{even}.\footnote{An anonymous reviewer asks if (i) is possible at all. If one assumes the Roothian Alternative Semantics \citep{rooth:92}, as we do here, there is a natural way of making sense of it. Under this framework, the set of alternatives for \textit{even} is structurally represented as the complement of the $\mathord{\sim}$-operator, and there is no reason why the mechanism used for generating ignorance implicatures cannot make use of the same set. It should also be noted that for (ii), it needs to be assumed that \textit{even} does not always make the set of alternatives trivial, contrary to what \citet{rooth:92} stipulates. As \citet{krifka:91} and \citet{panizzasudo} discuss, there is independent evidence for abandoning Rooth's stipulation.}

  \[\left\{\begin{array}{l}
    \text{James even speaks at least three languages,}\\
    \text{James even speaks {at least four} languages,}\\
    \text{James even speaks {exactly three} languages}
  \end{array}\right\}\]

\noindent The only difference between the two sets is the presence/absence of the particle \textit{even}, whose assertive meaning is vacuous. It is currently a hotly debated issue how presuppositions behave in the computation of implicatures (e.g.\ \citealt{gajewskisharvit, spectorsudo, marty, anvari}), and the current literature contains no explicit discussion of the behavior of \textit{even} in implicatures, or how presuppositions of alternatives behave in the computation of ignorance implicatures, as opposed to scalar implicatures. For this reason, this issue will remain as another open question, but for our purposes in this paper, it is enough to approach this issue bottom-up. That is, the examples in \REF{don-sud:broad} and \REF{don-sud:onlyde} contain numerals with superlative modifiers and focus particles, and crucially, they have the same ignorance implicatures as the versions of these sentences without the focus particles. Extending this to example \REF{don-sud:rep}, we expect it to have the same ignorance implicatures as the version of the sentence without \textit{even}. Theoretically, we could obtain this result by forcing the option (i) above, or by adopting (ii) but somehow making sure that the computation of ignorance implicatures ignores \textit{even}, which we will leave open here.

Now notice crucially that the ignorance implicatures of \REF{don-sud:rep} contradict the additive presupposition of \textit{even}. Specifically, the additive presupposition is either that James speaks at least four languages or that James speaks exactly three languages, but then it must be the case that the speaker believes it to be true, and so cannot be ignorant about its truth (cf.\ \citealt{stalnaker}). We claim that this conflict is what is behind the infelicity of \REF{don-sud:rep}.\largerpage

To reiterate the crucial assumption of our analysis, the focus alternatives that \textit{even} operates on are generated in the same way as the alternatives that give rise to ignorance implicatures, based on the alternatives of numerals with superlative modifiers in \REF{don-sud:alts} (cf.\ \citealt{rooth:92, foxkatzir}). If not, the additive presupposition could well be compatible with the ignorance implicatures. That is, if the additive presupposition could be about an alternative that was not in the set of alternatives for the ignorance implicatures, the truth of that alternative would not conflict with the ignorance implicatures.

\subsection{\textit{De re} additive presuppositions}

Notice at this point that it is crucial for us that the additive presupposition is about a particular alternative, and in this sense \textit{de re}. More specifically, the additive presupposition of \textit{even} $\phi$ is satisfied in a given context if the truth of some alternative not weaker than $\phi$ is common ground. This contrasts with an existential presupposition, which says that it is common ground that some alternative not weaker than $\phi$ is true. Such an existential presupposition is too weak for our purposes, as it is compatible with the ignorance implicature that the speaker does not know which non-weaker alternative is true.
  
There is independent empirical reason to adopt the \textit{de re} additive presupposition. \citet{kripke} argues that the additive presuppositions that additive particles like \textit{too} trigger are similarly stronger than existential, based on examples like the following (see also \citealt{geurtsvandersandt}; but see \citealt{ruys}).

    \ea Sam$_F$ is having dinner in New York tonight, too.\label{don-sud:ny}
    \z

\noindent If it is merely existential, the presupposition that there is someone else having dinner in New York tonight will be very easy to satisfy. Rather, the intuition tells us that this sentence requires a prior mention of some particular individual, who is at least known to be in New York tonight, and perhaps also known to be going to have dinner there.

We observe that \textit{even} behaves similarly in this regard. To see this, consider the following example.

     \ea Even Daniele$_F$ has a bike.\z

\noindent Intuitively, this example similarly requires it to be clear in the context which alternative or alternatives are relevant, at least.

\citet{kripke} analyzes the additive presupposition of \textit{too} to be anaphoric. That is, it is not merely propositional but contains an anaphoric component that needs to be resolved to an antecedent accessible in the discourse that satisfies the relevant property. For example, the additive presupposition of \REF{don-sud:ny} above has an anaphoric component that needs to be resolved, and then it furthermore presupposes that that individual is going to have dinner in New York (see also \citealt{geurtsvandersandt}). We essentially adopt this analysis for \textit{even}, but the way we state it is slightly weaker, as it does not have an anaphoric component, but an existential quantifier over alternatives that is \textit{de re} with respect to the presuppositional attitude. At this point, we cannot tease apart these two analytical possibilities on empirical grounds, and we could as well adopt \citeauthor{kripke}'s idea, but crucially, both types of analyses, when applied to \REF{don-sud:rep}, will result in a conflict with the ignorance inference.


\subsection{Predictions}

One prediction that our analysis makes is that the additive particle should also give rise to infelicity, when used in a sentence like \REF{don-sud:rep} in place of \textit{even}, because the conflict should arise as long as an additive presupposition is triggered. This prediction is borne out. Note, however, the infelicity of an example like the following is not telling, because the truth-conditional meanings of the two sentences are simply incompatible with each other anyway.

    \ea Daniele speaks exactly two languages. \#He speaks [at least three]$_F$, too.\z

\noindent Rather, we need to look at examples like \REF{don-sud:daniele}.

    \ea Daniele is allowed to smoke exactly two cigarettes today. \#He is allowed to smoke [at least three]$_F$, too.\label{don-sud:daniele}\z

\noindent Here, the truth-conditional meanings of the two sentences should be compatible with each other. In fact, the comparative version of this example is perfectly felicitous.

    \ea Daniele is allowed to smoke exactly two cigarettes today. He is allowed to smoke [more than two]$_F$, too.\z

\noindent According to our account, \REF{don-sud:daniele} is rendered infelicitous because of the clash between the additive presupposition and the ignorance implicature.\footnote{It is actually an open issue why the second sentence of \REF{don-sud:daniele} has to have an ignorance implicature, rather than a scalar implicature. See \citet{buccolahaida:18} for discussion.}

Another prediction we make is that a similar conflict should arise with a scalar implicature of a superlative modifier as well. In order to see this, consider the following example, which is infelicitous.

    \ea
    Andy is giving two lectures at the summer school.
    \#Patrick is even required to give [at least four]$_F$.
    \z

\noindent Recall that a superlative modifier gives rise to a scalar implicature under a universal quantifier like a necessity modal. The second sentence of this example, therefore, has a scalar implicature that Patrick is not required to give exactly four lectures and he is not required to give more than four lectures. On the other hand, the additive presupposition requires that it be presupposed that Patrick is required to give exactly four lectures, or that he is required to give more than four. This clash explains the infelicity.

  
 \section{Conclusion and open issues}
 \label{don-sud:sec:conclusion}

To summarize, we have developed an account of the observation that numerals with superlative modifiers are not compatible with focus particles like \textit{even}, which as far as we know has not been previously discussed. We proposed that what causes the infelicity is the additive presupposition triggered by \textit{even}, which conflicts with the ignorance implicature of the numeral with the superlative modifier. We remarked that in order to obtain these results, two theoretical assumptions are necessary: (i) the set of alternatives for implicatures and the set of alternatives for focus operators are generated in the same way based on the alternatives for numerals with superlative modifiers \citep{foxkatzir, rooth:92, mendia}, and (ii)~the additive presupposition is stronger than a merely existential presupposition, and is \textit{de re} \citep{kripke, geurtsvandersandt}. Both of these points have been proposed in the literature on independent grounds, and we hope to have provided further support for them in this paper. Before closing, we will discuss some open issues that arise from our analysis.

\subsection{Alternatives of superlative modifiers}

In Section \ref{don-sud:sec:ignorance}, we simply followed previous analyses and postulated particular sets of alternatives for numerals with superlative modifiers, but we did not provide a principled account as to why these alternatives must be used. In fact, this is one of the open issues discussed in \citet{schwarz} and \citet{mayr}, and unfortunately we do not have anything additional to offer. Having said that, however, we would like to discuss how to extend our analysis to other uses of superlative modifiers, which might shed some light on this question.

So far, we have only looked at cases where superlative modifiers combine directly with a numeral, but superlative modifiers can modify other types of expressions as well. In fact, it is reasonable to assume that superlative modifiers are focus sensitive operators themselves. Concretely, in examples like the following, one observes the usual focus association effects.

\ea
  \ea Andy at least introduced Patrick$_F$ to Tom.
  \ex Andy at least introduced Patrick to Tom$_F$.
  \z
\z

\noindent In order to capture this, we can analyze \textit{at least} and \textit{at most} as focus sensitive operators. As in the case of \textit{even}, let us assume that the superlative modifiers take sentential scope, although this assumption is strictly speaking not necessary (cf.\ fn.~\ref{don-sud:scope}).

\eanoraggedright
  \eanoraggedright
  \qq{\text{at least}\;\phi} requires contextually determined partial ordering $\leq$ among $\cnst{alt}(\phi)$, and asserts the grand disjunction of $\{\phi' | \phi \leq \phi'\}$.\smallskip\\Furthermore, $\cnst{alt}(\text{at least}\;\phi)=\{\qq{\text{at least} \;\phi}, \qq{\text{Exh}_{\{\phi'\in \cnst{alt}(\phi) | \phi <\phi' \}}\,\phi},\break \qq{\text{at least}\;\psi}\}$ where $\psi$ is the grand disjunction of $\{\phi' | \phi < \phi'\}$.
  \ex
 \qq{\text{at most}\;\phi} requires contextually determined partial ordering $\leq$ among $\cnst{alt}(\phi)$, and asserts the grand disjunction of $\{\phi' | \phi' \leq \phi\}$ and the negation of the grand disjunction of $\{\phi'' | \phi < \phi''\}$.\smallskip\\Furthermore, $\cnst{alt}(\text{at most}\;\phi)=\{\qq{\text{at most}\;\phi}, \qq{\text{Exh}_{\{\phi'\in \cnst{alt}(\phi) | \phi' <\phi \}}\,\phi},\break \qq{\text{at most $\psi$}}\}$ where $\psi$ is the grand disjunction of $\{\phi' | \phi' < \phi\}$.
  \z
\z

\noindent Exh here is the exhaustivity operator à la \citet{fox}. The notion of innocent exclusion used in its definition is crucial to state the above semantics in a general way.
\ea \qq{\text{Exh$_A$ $\phi$}} is true iff $\phi$ is true and all innocently excludable alternatives to $\phi$ with respect to $A$ are false.\z

\eanoraggedright
  \eanoraggedright $\psi$ is an innocently excludable alternative to $\phi$ with respect to $A$ iff $\psi$ is a member of every maximal set of excludable alternatives with respect to $\phi$ and $A$.
  \ex A set $S$ is a set of excludable alternatives with respect to $\phi$ and $A$ iff\linebreak $S\subseteq A$ and $\phi$ and the negation of the grand disjunction of $S$ are consistent.
  \z
\z

\noindent To see how this works, let us apply it to the following example.

\ea Pietro invited at least Daniele$_F$.\z

\noindent The alternatives to \textit{Daniele} need to be ordered here in some way. One of the most natural options here is the following kind of set of alternatives, partially ordered by generalized entailment.
\[
\left\{\begin{array}{l}
  \vdots\\
  \text{Pietro invited Danile and Taka and Ruoying}, \dots\\
  \text{Pietro invited Daniele and Taka}, \dots\\
\text{Pietro invited Daniele},\dots\\
\end{array}\right\}
\]

\noindent The assertive meaning only concerns those alternatives that are commensurable with the prejacent, and so will be that Pietro invited Daniele (and possibly someone else), and the ignorance implicature will amount to  the speaker's lack of certainty whether Pietro only invited Daniele or if he invited someone else.

The analysis also works for cases like the following where the scale is dense.

\ea
  \ea Daniele is at least [180 cm]$_F$ tall.
  \ex It at most takes [15 min]$_F$.
  \z
\z

\noindent It also works when the scale is not ordered by (generalized) entailment, as in \REF{don-sud:rank}.

\ea\label{don-sud:rank}
  \ea Daniele is at least a [postdoc]$_F$.
  \ex Andy won at most the silver$_F$ medal.
  \z
\z

\noindent Crucially, this analysis predicts the following examples to be infelicitous for the same reason as numerals with superlative modifiers are incompatible with focus operators with additive presuppositions, which is a good prediction.

\ea
  \ea[\#]{Pietro even invited [at least Daniele$_F$]$_F$.}
  \ex[\#]{Pietro also invited [at least Daniele$_F$]$_F$.}
  \z
\z

\subsection{Comparative modifiers}

As we saw in several places in this paper, numerals with comparative modifiers behave differently from numerals with superlative modifiers. Part of this comes from the fact that numerals with comparative modifiers do not give rise to robust ignorance implicatures. For example, the following sentences sound more felicitous than their superlative counterparts.

\ea
  \ea I have more than two kids.
  \ex I have fewer than three kids.
  \z
\ex
  \ea A triangle has more than two sides.
  \ex A triangle has fewer than four sides.
  \z
\z

\noindent If numerals with comparative modifiers do not necessarily give rise to ignorance implicatures, then it is predicted that they should be compatible with focus particles with additive presuppositions.

However, this matter is not as clearcut as one might hope. That is, the sentences like those above actually do often have have inferences that amount to something similar to an ignorance implicature or an indifference/irrelevance implicature \citep{meyer, lauer}. How this arises and what alternatives are used are interesting questions, but we cannot offer a concrete account here, and as far as we know, they are currently debated in the literature (see \citealt{foxhackl, mayr, schwarz}).

In addition to this question about alternatives, the morphosyntactic difference between comparative and superlative modifiers is also puzzling. As we discussed in the previous subsection, superlative modifiers are focus sensitive operators and can appear in all sorts of adverbial positions. By contrast, there is no indication that comparative modifiers are focus sensitive, and in fact their distribution seems to be more constrained. We are presently not aware of a satisfactory account of why this is so.


\subsection{When \textit{even} is not additive}

The last open issue we would like to mention has to do with the additive presupposition of \textit{even}. As mentioned in passing, the additive presupposition of \textit{even} sometimes seems to be absent \citep{rullmann, crnic}. The following is a well-discussed example of this.

\ea Andy even won the silver$_F$ medal.\label{don-sud:silver}\z

\begin{sloppypar}
\noindent This sentence has a reading that does not imply that Andy also won another medal, although according to \citet{francis}, these judgments are not stable across speakers of English.
\end{sloppypar}

As \citet{crnic} discusses in great detail, there is currently no satisfactory account of exactly when the additive presupposition of \textit{even} arises, and we have nothing insightful to add here. Yet, it is our prediction that in the absence of an additive presupposition, \textit{even} should be compatible with superlative modifiers. One might then think that the fact that an example like \REF{don-sud:evensilver} is infelicitous might be problematic for our account.
\ea Patrick won the bronze medal. ??Andy even won [at least the silver$_F$]$_F$ medal.\label{don-sud:evensilver}\z
However, in the absence of a good understanding of the distribution of the additive presupposition, we cannot be sure if this example actually lacks an additive presupposition. In particular, entailment among the focus alternatives might be one relevant factor that correlates with the presence of additivity, as \citet{crnic} claims, and if so, the presence of \textit{at least} in \REF{don-sud:evensilver} should matter crucially, as with it, the focus alternatives presumably stand in an entailment relation (cf.\ the semantics of \textit{at least} above).\footnote{An anonymous reviewer asks about examples like \textit{I even doubt that one$_F$ person came} (cf.\ \citealt{crnic}). If the numeral and its alternatives receive lower-bounded readings, then indeed the additive presupposition would be problematic because the alternatives would be entailed. However, it is well known that numerals generally can easily receive bilateral readings even in negative contexts (\citealt{geurts, breheny}, among others). With this as an option, such examples do not pose an issue. For the above example, the additive presupposition would be satisfied if the speaker doubts that exactly $n$ people came for at least one $n>1$, which seems to be a reasonable analysis.}

Due to these complications, we cannot offer a conclusive example involving \textit{even}, but it should be remarked that nothing rules out the existence of a scalar particle like \textit{even} that is never associated with an additive presupposition in a natural language. For instance, Italian \textit{addirittura} is a good candidate (Daniele Panizza, p.c.). If this is the case, we predict it to be compatible with superlative modifiers. We, however, have left investigation of this for future research.


\section*{Acknowledgements}

We would like to thank Katherine Fraser and James Gray for judgments and helpful discussion. We also benefitted from feedback from the audiences of the 20th Szklarska Poręba Workshop in March 2019 and SinFonIJA 12 held at Masaryk University in Brno in September 2019. All remaining errors are ours.


\section*{Appendix}\largerpage

We saw in the main part of this paper that a numeral with \textit{at least} is incompatible with \textit{even}, as in \REF{don-sud:langs} while a numeral with \textit{at most} is compatible with \textit{only}, as in \REF{don-sud:onlyde}. Our explanation for the former is that the additive presupposition clashes with the obligatory implicature of \textit{at least}. For the latter, we could resort to the fact that \textit{only} does not trigger an additive presupposition and hence does not cause a conflict.

We also mentioned in fn.~\ref{don-sud:fn:appendix}, a numeral modified by \textit{at least} is incompatible with \textit{only} and a numeral with \textit{at most} is incompatible with \textit{even}, as shown below.

\ea
  \ea I speak five languages. \#James only speaks [at least two]$_F$.\label{don-sud:onlyatleast}
  \ex I speak two languages. \#James even speaks [at most five]$_F$.\label{don-sud:evenatmost}
  \z
\z

\noindent For \REF{don-sud:evenatmost}, we could extend our analysis and maintain that the additive presupposition clashes with the ignorance implicature, but \REF{don-sud:onlyatleast} is not amenable to this explanation. Generally, when associating with a quantity expression, \textit{only} gives rise to an inference that the relevant quantity is small, so one would expect the second sentence of \REF{don-sud:onlyatleast} to mean James speaks more than one language, and he speaks many languages.

We think that examples like these require an entirely different explanation anyway, because their comparative counterparts are equally unacceptable.

\ea
  \ea I speak five languages. \#James only speaks [more than one]$_F$.\label{don-sud:onlymore}
  \ex I speak two languages. \#James even speaks [fewer then six]$_F$.\label{don-sud:evenfewer}
  \z
\z

\noindent Here is a sketch of a possible analysis. As mentioned above, \textit{only} associating with a quantity expression triggers an inference that the named amount is small. \textit{Even}, on the other hand, triggers an inference that the named amount is large.

\ea
  \ea James only speaks five$_F$ languages.
  \hfill $\leadsto$ five is a small amount
  \ex James even speaks five$_F$ languages.
  \hfill $\leadsto$ five is a large amount
  \z
\z

\noindent There are several accounts of when and how \textit{only} can trigger such a scalar inference \citep{grosz, coppockbeaver, alxatib}, which we will not get into here. For \textit{even}, the semantics we discussed in \sectref{don-sud:sec:even} can derive a scalar inference with reasonable assumptions about the flavor of the scalar presupposition and about the alternatives of numerals.

What seems to us to be going on in the above cases with modified numerals is that these scalar inferences arise from all numerals in their range. That is, the scalar inference of \REF{don-sud:onlyatleast}/\REF{don-sud:onlymore} is that each $n>1$ is a small amount, and that of \REF{don-sud:evenatmost}/\REF{don-sud:evenfewer} is that each $n<6$ is a large amount. We leave open the compositional details of how these inferences arise from the semantics of the focus operators and modified numerals here, but they account for the infelicity of these examples. Further support for this analysis comes from the following contrast.

\ea Katie speaks four languages, which is a lot.
  \ea[]{ James only speaks [at most two]$_F$ languages.}
  \ex[\#]{James only speaks [at most ten]$_F$ languages.}
  \z
\z

{\sloppy\printbibliography[heading=subbibliography,notkeyword=this]}

\end{document}
