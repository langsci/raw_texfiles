\addchap{Preface}


\textit{African Linguistics on the Prairie} features select peer-reviewed papers from the 45\textsuperscript{th} Annual Conference on African Linguistics (ACAL 45).  The conference was held on April 17-19, 2014 at the University of Kansas in Lawrence, Kansas and was hosted by the Department of Linguistics.  The plenary speakers for ACAL 45 were: Kofi Agyekum, University of Ghana; Chris Collins, New York University; Ruth Kramer, Georgetown University; Michael R. Marlo, University of Missouri; Carlos M Nash, University of California, Santa Barbara; Bonny Sands, Northern Arizona University; and Malte Zimmermann, Universität Potsdam.  The theme of the conference was “Africa’s Endangered Languages: Documentary and Theoretical Approaches”.  In conjunction with the conference, a special three-day workshop, supported by a grant from the National Science Foundation (NSF DEL 1360823), was organized around the conference theme.  The workshop brought together scholars with varied perspectives and research agendas to address the unique challenges facing endangered languages, language documentation, and revitalization efforts in Africa.  Select papers from that workshop appear in the volume \textit{Africa’s Endangered Languages}, published in 2017 by Oxford University Press.  The articles appearing in this volume were for the most part presented in the main session of the conference. 

The articles that comprise this volume reflect the enormous diversity of African languages, as they focus on varieties from all of the major African language phyla.  The articles here also reflect the many different research perspectives that frame the work of linguists in the Association for Contemporary African Linguistics. The diversity of views presented here are thus indicative of the vitality of current African linguistics research. As a perusal of the titles hints, the work published in this volume covers fields ranging from phonetics, phonology, morphology, typology, syntax, and semantics to sociolinguistics, discourse, analysis, language acquisition, computational linguistics and beyond.  This broad scope and the quality of the articles contained within holds out the promise of continued advancement in linguistic research on African languages.

ACAL 45 would not have been possible without financial support from multiple institutions.  We gratefully acknowledge support from the University of Kansas Department of Linguistics, the Kansas African Studies Center, the office of the Dean of the College of Liberal Arts and Sciences, and the National Science Foundation.  The ACAL 45 Organizing Committee, Travis Major, Ibrahima Ba, Mfon Udoinyang, Carlos Nash and Peter Ojiambo, played a central role in the success of the conference and we thank them immensely for their tireless work and support.  The University of Kansas Linguistics Department provided more than just material support.  Department faculty, graduate students, and staff volunteered hours of their time at all stages of the planning and execution of the conference. We thank: Saad Aldossari, Chia-Ying Chu, Katrina Connell, Kate Coughlin, Lauren Covey, Philip Duncan, Robert Fiorentino, Alison Gabriele, Longcan Huang, Corinna Johnson, Allard Jongman, David Kummer, Mingxing Li, Beatriz Lopez Prego, María Martínez García, Andrew McKenzie, Utako Minai, Zhen Qin, Maria Rangel, Sara Rosen, Joan Sereno, Khady Tamba, Wenting Tang, Annie Tremblay, Xiao Yang, and Jie Zhang.  The following individuals served as session chairs at the conference and we would like to express our thanks to them as well: Akin Akinlabi, Andrew McKenzie, Mike Cahill, Jeanine Ntihirageza, Vicki Carstens, Allard Jongman, Lindley Winchester, Martha Michieka, Ibrahima Ba, Timothy M. Stirtz, Christopher Green, Annie Tremblay, Jie Zhang, Kasangati Kinyalolo, Lisa Zsiga, Rebecca Hale, Peter Ojiambo, Claire Halpert, Lee Bickmore, Mohamed Mwamzandi, Heike Tappe, Laura McPherson, Tucker Childs, Philip Rudd, Mary Paster, Peter Jenks, Rose-Marie Déchaine, Michael Diercks, Olanike Orie, Mamadou Bassene, and James Essegbey.  In putting together this volume, we have relied upon the many specialists who generously agreed to serve as reviewers, thus ensuring the high quality of articles that appear within.  We are exceedingly grateful to: Oluseye Adesola, Colleen Ahland, Assibi Amidu, Anton Antonov, Rebekah Baglini, Nicholas Baier, Anna Belew, Kelly Harper Berkson, Leston Buell, Tucker Childs, Caitlin Coughlin, Rose-Marie Déchaine, Katherine Demuth, Michael Diercks, Philip Duncan, John Gluckman, Scott Grimm, Claire Halpert, Claire Harter, Stefanie Harves, Peter Jenks, Gregory Kobele, Ruth Kramer, Jorge Emilio Rosés Labrada, Florian Lionnet, Victor Manfredi, Michael Marlo, Andrew McKenzie, Laura McPherson, Leonard Muaka, Samuel Gyasi Obeng, Doris Payne, Clifton Pye, Philip Rudd, Russell Schuh, Anne Schwarz, Peter Trudgill, Matthew Tucker, Jenneke van der Wal, and Malte Zimmermann.  Thanks also to Joan Maling at the National Science Foundation and Kate Lorenz at the University of Kansas Institute for Policy and Social Research for valuable assistance in helping us navigate through the complexities of external funding.  We also acknowledge the valuable assistance provided by Akin Akinlabi and Lee Bickmore, who not only answered countless questions and offered excellent advice, but helped run the ACAL organization and created the infrastructure for the publication of this and future ACAL volumes with Language Science Press.  Lastly, we heartily thank Sebastian Nordhoff at Language Science Press for his invaluable help with the many technical aspects of publishing this volume.

 