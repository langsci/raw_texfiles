\documentclass[output=paper,
modfonts
]{langscibook} 
% \bibliography{localbibliography}
\ChapterDOI{10.5281/zenodo.1251724}

 
\title{Consonant substitution in child language (Ikwere)} 

\author{Roseline I. C. Alerechi \affiliation{University of Port Harcourt}}

% \chapterDOI{} %will be filled in at production 

\abstract{The Ikwere language is spoken in four out of the twenty-three Local Government Areas (LGAs) of Rivers State of Nigeria, namely, Port Harcourt, Obio/Akpor, Emohua and Ikwerre LGAs. Like Kana, Kalabari and Ekpeye, it is one of the major languages of Rivers State of Nigeria used in broadcasting in the electronic media. The Ikwere language is classified as an Igboid language of the West Benue-Congo family of the Niger-Congo phylum of languages (\citealt[67, 71]{Williamson1988}, \citealt[31]{Williamson2000}). This paper treats consonant substitution in the speech of the Ikwere child. It demonstrates that children use of a language can contribute to the divergent nature of that language as they always strive for simplification of the target language. Using simple descriptive method of data analysis, the paper identifies the various substitutions of consonant sounds, which characterize the Ikwere children’s utterances. It stresses that the substitutions are regular and rule governed and hence implies the operation of some phonological processes. Some of the processes are strengthening and weakening of consonants, loss of suction of labial implosives causing them to become labial plosives, devoicing of voiced consonants, etc. While some of these processes are identical with the adult language, others are peculiar to children, demonstrating the relationships between the phonological processes in both forms of speech. It is worthy of note that highlighting the relationships and differences will make for effective communication between children and adults. 
}
%\keywords{Divergent nature, substitution, simplification, rule governed, phonological processes}

\begin{document} 

\maketitle  


\section{Introduction}\label{sec:alerechi:1}

The \ili{Ikwere} language is spoken in four out of the twenty-three Local Government Areas (LGAs) of Rivers State of \isi{Nigeria}, namely, Port Harcourt, Obio/Akpor, Emohua and Ikwerre LGAs. Like \ili{Kana}, Kalabari and Ekpeye, it is one of the \isi{major} languages of Rivers State used in broadcasting in the electronic media \citep[1]{Alerechi2007a}. \citet[67, 71]{Williamson1988} classifies \ili{Ikwere} as one of the \ili{Igboid} group of languages as well as \ili{Igbo}, Ekpeye, ogba, Echie, to mention but a few.  \citet[31]{Williamson2000} locate \ili{Igboid} under the node of West Benue-\isi{Congo} family of Niger-\isi{Congo} \isi{phylum} of languages. The \ili{Ikwere} language comprises twenty-four divergent dialects, which are mutually intelligible. It is yet to develop a standard dialect. However, there are published works such as \citegen{Donwa-IfodeEkwulo1987} \textit{\ili{Ikwere} Orthography}, \textit{Tẹsitament Iikne} (a translated New Testament Bible), and some recent scholarly works in the language. 

Some of the works like \citet{Williamson1980},  \citet{Donwa-Ifode2001}, among others, observe different realization of some phonological segments in \ili{Ikwere}. In fact, \citet{Alerechi2007a} specifically identified some phonological processes responsible for the different realizations of segments, which may have contributed to the divergent nature of the \ili{Ikwere} language. Some of such processes are loss of suction of labial \isi{implosives} causing them to become labial plosives, the \isi{spirantization} (weakening) of labial and \isi{alveolar} plosives to labial and \isi{alveolar} \isi{fricatives}, respectively, the voicing of \isi{alveolar} \isi{fricatives}, etc. 

It is interesting to note that these studies are focused on the adult use of the \ili{Ikwere} language to the neglect of the area concerning \isi{child language}. This serves as a motivation for this paper. Given that the general trend for children is to change the sounds of the language in an attempt to use them, this paper is aimed at identifying such changes and consequently the phonological processes characteristic of \ili{Ikwere} children aged 3 to 4 years. Following the assertion of \citet[358]{FromkinEtAl2003} that early phonological rules generally reflect natural phonological processes that occur in the adult (target) language, this paper further investigates if the phonological processes in \isi{child language} are identical with those of the adult (target language).

This paper focuses on O̩deegnu (Odgn), E̩mowha (Emwh), \ili{Akpo}, \ili{Aluu} and \ili{Omuanwa} (Omnw) dialects of \ili{Ikwere}, an \ili{Igboid} language spoken in Rivers State of \isi{Nigeria}. Even though the sound \isi{substitution} in \isi{child language} involves both consonants and vowels, this paper specifically focuses on the \isi{substitution} of consonants of the target language for those characteristic of the child’s language.

\subsection{Literature review}\label{sec:alerechi:1.1}

This section gives a brief review of literature in \isi{child language acquisition}. It specifically treats the \isi{phonological development}, phonological processes and outlining the target sounds.

\subsection{Phonological development}\label{sec:alerechi:1.2}

Communication is a natural phenomenon to every human being. Thus, to enable children to communicate with others in their environment, they need to acquire the language. \citet[361]{O’Grady2011} state that the ability of children to produce speech sounds begins to emerge at around six months with the onset of babbling. Babbling enables the children to experiment with and begin to gain control over their vocal apparatus. It increases in frequency until the age of about twelve months, when the children begin to produce their first understandable words. 

Scholars have investigated the order of acquisition of sounds by the children and some observed that among the earlier sounds are the back \isi{velar} sounds [k]and [g], and front vowels like [a], [i] and [e]. Others recognize the bilabials [m], [p] and the \isi{alveolar} sounds [n] and [d] demonstrated in such sequences as ma, pa, di (\citealt[283]{Bolinger1975}; \citealt[344]{Labarba1981}; \citealt[69]{Ojukwu2011}). There is a contrary view that children acquire \isi{velar} consonants before the bilabials and alveolars \citep[45]{Anthony1971}. In line with this view, \citet[257]{Alerechi2012} observe that Ekpeye children below age three replace the \isi{velar} plosives [k] or [g] with the \isi{alveolar} plosives [t] or [d], respectively. This implies that despite the similarity observed in the order in which children acquire speech sounds, individual differences still abound. In fact, each child develops his own systematic way of producing adult forms within his limited scope of sound sequences \citep[813]{Menn1992}.

In spite of a good deal of variation observed from one child to the other in terms of the order of mastering sounds in production and perception, the general tendencies as outlined by \citet[362]{O’Grady2011} seem to exist. Based on the manner of articulation (stricture), stops tend to be acquired before other consonants. In terms of place of articulation, labials are often acquired first followed (with some variation) by alveolars, velars, alveopalatals.

Realizing that sounds do not exist in isolation but in sequences to form morphemes or words, such sequences comprise vowel and consonant. Thus, vowel and consonant occur in a sequence to make up syllable structure and children tend to simplify the syllable or word structure of the target language. Children structures are mainly CV, CVCV \citep[25]{Akpan2004}. This implies that in the acquisition of the \isi{adult speech} by the children certain phonological processes are in operation.

\subsection{Phonological processes}\label{sec:alerechi:1.3}

Phonological processes are those changes which segments undergo that result in the various phonetic realizations of the underlying phonological segments \citep[144]{Yul-Ifode1999}  Children adopt certain phonological processes to attain to the adult sounds. According to \citet[26]{Akpan2004}, phonological processes in children are short-cut processes that operate on the child’s speech in his attempt to attain the adult target. As the child’s chronological age increases, the phonological processes decrease to conform to the phonological system of the language. \citet[27]{Akpan2004} further notes three \isi{major} classifications of the phonological processes: \isi{substitution}, assimilatory and syllable structure processes. In addition to these \isi{major} classifications, \citet[2]{Yul-Ifode2003} recognizes dissimilation (intervocalic consonant \isi{devoicing}) as a fourth \isi{major} process. Of the four \isi{major} phonological processes identified in the literature, the present paper focuses on \isi{consonant substitution} in children speech. 

Substitution is the systematic replacement of one linguistic feature for another or one phoneme for another that the child finds easier to articulate (\citealt[357]{Fromkin2003}; \citealt[491]{Akmajian2008}; \citealt[365]{O’Grady2011}. \citet[365]{O’Grady2011} identified common \isi{substitution} processes to include stopping, the replacement of a \isi{fricative} by a corresponding stop; fronting, the moving forward of a place of articulation; gliding, the replacement of a liquid by a glide; and \isi{denasalization}, the replacement of a nasal stop by a non-nasal counterpart. Scholars like \citet{Akmajian2008}, \citet{Akpan2008}, \citet{David2009}, \citet{Akpan2010},  \citet{Alerechi2010} and  \citet{Ojukwu2011} have equally identified various forms of sound \isi{substitution} in \isi{child language} in different languages. The present paper treats the \isi{consonant substitution} in \ili{Ikwere}.

\subsection{The target sounds}\label{sec:alerechi:1.4}

% \todo{please check prose description against provided images (8 vs 9 vowels)}
There are nine phonetic oral vowels [i ɪ e ɛ a o ɔ u ʊ], and eight phonetic nasalized vowels [{\~{\.\i}} \~ɪ ẽ ã õ \~ɔ \~u \~ʊ] in \ili{Ikwere} (\citealt[42--43]{Donwa-IfodeEkwulo1987}; \citealt{Alerechi2007a}). They are summarized in the vowel charts on \figref{fig:alerechi:1} and \figref{fig:alerechi:2} respectively.

\begin{figure}
% \parbox{.45\textwidth}{                                                           
% \includegraphics[height=.15\textheight]{figures/alerechi1.png}
% }
\parbox{.45\textwidth}{
\begin{vowel}
\putcvowel{i}{1} 
\putcvowel{e}{2}
\putcvowel{[ɛ]}{3}
\putcvowel{a}{4}
\putcvowel{ɔ}{6}
\putcvowel{o}{7}
\putcvowel{u}{8} 
\putcvowel{ɪ}{13}
\putcvowel{ʊ}{14} 
\end{vowel}
}

\caption{The Ikwere phonetic oral vowels.}
\label{fig:alerechi:1}
\end{figure}
 

\begin{figure}
% \parbox{.45\textwidth}{
% \includegraphics[height=.15\textheight]{figures/alerechi2.png}
% }
\parbox{.45\textwidth}{
\begin{vowel}
\putcvowel{{\~{\.\i}}}{1} 
\putcvowel{\~e}{2}

\putcvowel{\~a}{4}
\putcvowel{\~ɔ}{6}
\putcvowel{\~o}{7}
\putcvowel{\~u}{8} 
\putcvowel{\~ɪ}{13} 
\putcvowel{\~ʊ}{14} 
\end{vowel}
}

\caption{The Ikwere phonetic nasalized vowel.}
\label{fig:alerechi:2}
\end{figure}

Contrary to \citet[42--43]{Donwa-IfodeEkwulo1987}, which see [e] and [ɛ] as the allophonic variants of the phoneme /e/ in \ili{Ikwere}, \citet[65]{Alerechi2007a} observed that the vowel [ɛ] contrasts with other vowels in some dialects and is an allophonic variant of /e/ in some other dialects. It is also noted that beside the distinctive nasal vowels recorded in the language, vowels in the environment of nasal consonant may or may not be nasalized. 

On the other hand, \ili{Ikwere} records thirty-one phonetic consonants as shown in \tabref{tab:alerechi:1}. There is no long consonant in the language. 

  

\begin{table}
%  \includegraphics[height=.3\textheight]{figures/alerechi3.png}
\resizebox{\textwidth}{!}{\begin{tabular}{>{\raggedright}p{1.7cm} *{14}{c}}
 \lsptoprule
  Manner of articulation & \multicolumn{14}{c}{Place of articulation}\\\cmidrule(lr){2-15}
			& \multicolumn{2}{c}{Labial} & \multicolumn{2}{c}{Alveolar} & \multicolumn{2}{c}{Palatal} & \multicolumn{2}{c}{Velar} & \multicolumn{2}{c}{Labialized velar} & \multicolumn{2}{c}{Labial-velar} & \multicolumn{2}{c}{Glottal}\\\midrule
 Nasal 			& & m & & n & & ɲ & & ŋ & &  ŋʷ \\\tablevspace
 Plosives		& p & b & t & d & & & k & ɡ &\textbf{kʷ} & \textbf{ɡʷ} & & & ʔ & \\\tablevspace
 Implosives		& ƥ & ɓ \\\tablevspace
 Af\isi{fricatives}		& & & & & tʃ & dʒ \\\tablevspace
 Fricatives		& f & v & s & z & ʃ & ʒ & & ɰ & hʷ & & & & h &\\\tablevspace
 Tap			&   &   &  & ɾ \\\tablevspace
 Central\newline approximants   &  &  &  &  &  & j & & & & & &  w \\\tablevspace
 Lateral\newline approximant    &   &   &  & l \\ 
 \lspbottomrule
\end{tabular}}
\caption{The phonetic consonants of Ikwere (adapted from \citealt{Donwa-IfodeEkwulo1987}).\label{tab:alerechi:1}}
\end{table}

Of the thirty-one consonants recorded in the language twenty-eight of them are phonemic. Following \citet[98]{Alerechi2007a}, the number of the phonemic consonants in each of the dialects, however, varies. It ranges from twenty-six to twenty-eight.

   
 
\begin{table}
% \includegraphics[height=.3\textheight]{figures/alerechi4.png}
\resizebox{\textwidth}{!}{\begin{tabular}{>{\raggedright}p{1.7cm} *{14}{c}}
 \lsptoprule
  Manner of articulation & \multicolumn{14}{c}{Place of articulation}\\\cmidrule(lr){2-15}
			& \multicolumn{2}{c}{Labial} & \multicolumn{2}{c}{Alveolar} & \multicolumn{2}{c}{Palatal} & \multicolumn{2}{c}{Velar} & \multicolumn{2}{c}{Labialized velar} & \multicolumn{2}{c}{Labial-velar} & \multicolumn{2}{c}{Glottal}\\\midrule
 Nasal 			& & m & & n & & ɲ & & ŋ & &  ŋʷ \\\tablevspace
 Plosives		& (p) & (b) & t & d & & & k & ɡ &\textbf{kʷ} & \textbf{ɡʷ} & & &  & \\\tablevspace
 Implosives		& ƥ & ɓ \\\tablevspace
 Af\isi{fricatives}		& & & & & tʃ & dʒ \\\tablevspace
 Fricatives		& (f) & (v) & s & z &  &  & & ɰ & hʷ & & & & h &\\\tablevspace
 Tap			&   &   &  & ɾ \\\tablevspace
 Central\newline approximants   &  &  &  &  &  & j & & & & & &  w \\\tablevspace
 Lateral\newline approximant    &   &   &  & l \\ 
 \lspbottomrule
\end{tabular}}
\caption{The phonemic consonants of Ikwere.}
\label{tab:alerechi:2}
\end{table}

All the sounds in parentheses in \tabref{tab:alerechi:2} occur in Ọgkr. Whereas one or more of these sounds are allophone(s) in some dialects, and do not exist in others. \citet[99]{Alerechi2007a} gives clear picture of the occurrence of [p b f v] in the dialects of \ili{Ikwere} as shown in  \tabref{tab:alerechi:3} - \tabref{tab:alerechi:7}.

  
\tabref{tab:alerechi:3} shows that /p b f v/ are phonemic in Ọgkr.
\begin{table}
\begin{tabular}{cc}
\lsptoprule
 p& b\\
 f& v\\
\lspbottomrule
\end{tabular}
\caption{Ọgkr.}
\label{tab:alerechi:3}
\end{table}

\tabref{tab:alerechi:4} records /p b v/ as phonemes in the above dialects.

\begin{table}
\begin{tabular}{cc}
\lsptoprule
 p& b\\
 -- & v\\
\lspbottomrule
\end{tabular}
\caption{Ozha, Ọmnw, Ubma, Akpb, Egbd, Elle, Omdg, Ubmn Omrl and Apni.}
\label{tab:alerechi:4}
\end{table}

\tabref{tab:alerechi:5} illustrates that /b f v/ are phonemic in \emph{\textup{Ẹ}}mwh.

\begin{table}
\begin{tabular}{cc}
\lsptoprule
 -- & b\\
 f& v\\
\lspbottomrule
\end{tabular}
\caption{\emph{Ẹ}mwh.}
\label{tab:alerechi:5}
\end{table}

The dialects in \tabref{tab:alerechi:6} record only /p b/ as phonemes.

\begin{table}
\begin{tabular}{cc}
\lsptoprule
 p& b\\
 -- & -- \\
\lspbottomrule
\end{tabular}
\caption{Akpọ , Obio, Alụu, Igwr Omgw, Iskp, Ipo and Omdm.}
\label{tab:alerechi:6}
\end{table}
 
\tabref{tab:alerechi:7} is the reverse of \tabref{tab:alerechi:6} as only /f v/ are phonemic in the four dialects.

\begin{table}[h!]
\begin{tabular}{cc}
\lsptoprule
 -- & -- \\
 f& v\\
\lspbottomrule
\end{tabular}
\caption{Rmkp, Rndl, Ọdgn and Ib/Ob.}
\label{tab:alerechi:7} 
\end{table}

\clearpage 
The foregoing \tabref{tab:alerechi:3} - \tabref{tab:alerechi:7} demonstrate the occurrence the consonants /p b f v/ in the various dialects of \ili{Ikwere}. The present study, therefore, intends to find out among other things, if children from the dialects without any of these sounds would manifest such in course of acquiring the language.

\subsection{Methodology}\label{sec:alerechi:5}

The wordlist used in collecting data from the subjects was drawn from the wordlist collected by \citet{Alerechi2007b} for treating labial variation in \ili{Ikwere}. It contains seventy-two words of everyday life obtainable in the environment of the subjects. The words contain different sounds of the language and are made up of monosyllabic, disyllabic and polysyllabic structures, giving such structures as V, CV, VCV, VCVCV, etc. The data were collected from each of the subjects by imitation and object pointing methods at their residence. The visit to each subject’s residence during the period of data collection was about two to three times in order to elicit the accurate forms of the subject’s speech. The subjects’ speech forms were recorded manually and finally transcribed for analysis. The study adopts a descriptive approach in analyzing the data. It focuses on identifying and analyzing the \isi{substitution} patterns and processes of consonants in the speech of the \ili{Ikwere} children. The data was also analyzed using SPSS. Descriptive statistics was carried out to describe the performance of the subjects. The different occurrence of the consonants produced by the adult and subject in each dialect, were converted to quantitative data which was presented as percentages using Bar Charts.

\subsection{The subjects}\label{sec:alerechi:1.6}

The subjects consulted during the period of data collection are seven; however, five of them were selected for the analysis because a comparison of the speech forms of two subjects from the same dialect area showed a replication of the other. The five subjects comprise three female and two male and fall within the age range of 3 to 4 years (3, 3, 3 ½, 3 ½ , 4). They were selected from O̩deegnu, Emowha, \ili{Akpo}, \ili{Aluu} and \ili{Omuanwa} so as to investigate if in the course of sound change, a child from a particular area would manifest forms typical of those of other area(s) or not. \tabref{tab:alerechi:8} summarizes details of the subjects. 

\begin{table}
\begin{tabularx}{\textwidth}{lXXXXX}
\lsptoprule
Subject & 1. VN & 2. MA & 3. EE & 4. IN & 5. GW\\
\midrule
Gender & Female & Male & Female & Female & Male \\
Age & 3 years & 3 years & 3 ½ years & 3 ½ years & 4 years\\
Village & Rumuodogo & Elibarada & Rumuolumini & Omuokiri & Ubordu \\
Dialect & Odeegnu & Emowha & {Akpo} & {Aluu} & {Omuanwa}\\
\lspbottomrule
\end{tabularx}
%%please move \begin{table} just above \begin{tabular
\caption{Data on the subjects.}
\label{tab:alerechi:8}
\end{table}

\subsection{Consonant substitution}\label{sec:alerechi:1.7}

There are twenty-eight phonemic consonants /m n ɲ ŋ ŋʷ p b t d k g kʷ gʷ ƥ ƃ ʧ ʤ f v s z ɰ hʷ h r j w l/in \ili{Ikwere}\textbf{. }Some of these consonants are replaced with some others in \isi{child language}. The pattern of \isi{substitution} reflects those involving different states of the \isi{glottis}, places of articulation and manners of articulation. The pattern reflecting different states of the \isi{glottis} sometimes overlap with those of places of articulation. I, therefore, present the various substitutions based on manner of articulation.

\subsubsection{Substitution according to manners of articulation}\label{sec:alerechi:1.7.1}

The different manners of articulation observed in the data involve the plosives, \isi{fricatives}, affricates, \isi{implosives}, approximants, etc. 

\subsubsection{Substitution of plosive with plosive}\label{sec:alerechi:1.7.2}

The substitutions here reflect those involving states of the \isi{glottis} or places of articulation. \citet[16]{Hyman1975} observed that the general tendencies in \isi{child language} include the learning of voiceless stops before voiced stops. This phenomenon is identified in this paper in the utterances of children above the age of 3 to 4. Thus, where the target language records the voiced stops [b] or [d], the tendency is for the children above age 3 to 4 to replace them with their voiceless counterparts [p] or [t], respectively. \tabref{tab:alerechi:9} is, therefore, strong evidence that \ili{Ikwere} children are not left out in first acquiring voiceless consonants and subsequently revert to the target forms. The table shows the Odgn subject replacing the voiceless \isi{velar} \isi{plosive} [k] with the voiceless \isi{alveolar} \isi{plosive} [t] of the target.

\begin{table}
\fittable{
\begin{tabular}{llllllllllp{1cm}}
\lsptoprule
\multicolumn{2}{c}{Odeegnu} & \multicolumn{2}{c}{Emowha} & \multicolumn{2}{c}{Akpo}  & \multicolumn{2}{c}{Aluu}  & \multicolumn{2}{c}{\ili{Omuanwa} } & Gloss\\\cmidrule(lr){1-2}\cmidrule(lr){3-4}\cmidrule(lr){5-6}\cmidrule(lr){7-8}\cmidrule(lr){9-10}
Target & VN 3yrs & Target & MA 3½yrs & Target & EE 3½yrs & Target & IN 3½yrs & Target & GW 4yrs\\
\midrule
\`{m}ƥòró ákâ & \`{m}pòtó átà & \`{m}ƥòró ákâ & \`{m}pèrè-áká & \`{m}ƥèré-ák\={a} &  & ìsíní ákâ & ìpèní á{\downstep}kâ & ḿƥèré ákâ & ípèlé á{\downstep}kâ & `elbow'\\
àƃà & àbà & àƃà & àbà & àb\`{ã} & àbà & àbà & àbà & àbà & àpà & `jaw'\\ 
zɪ & ʤɪ & zì & zì & zì &  & zɪ & zɪ & dɪ & tè & `is'\\
àkítì & àtítì & àkɪdɪ & àkɪdɪ & àkídì &  & àkísì & àkítì & ákɪdɪ & àkʷà & Beans\newline (brown)\\
\lspbottomrule
\end{tabular}
}

\caption{d → t: b → p, k → t.}
\label{tab:alerechi:9}
\end{table}

There is also the tendency of the \ili{Ikwere} subjects replacing complex articulated (\isi{labialized}) sounds with those of simple articulation (single segments). This phenomenon is predominant in the speech form of the Odgn child, even though children from other dialect areas manifest traces of this phenomenon. Consider the data in \tabref{tab:alerechi:10}. Note that the data include the simplifying of \isi{labialized} \isi{fricative} and nasal.

\begin{table} 
\fittable{
\begin{tabular}{lllllllllll}
\lsptoprule
\multicolumn{2}{c}{Odeegnu} & \multicolumn{2}{c}{Emowha } & \multicolumn{2}{c}{\ili{Akpo} } & \multicolumn{2}{c}{\ili{Aluu} } & \multicolumn{2}{c}{Omuanwa} & Gloss\\\cmidrule(lr){1-2}\cmidrule(lr){3-4}\cmidrule(lr){5-6}\cmidrule(lr){7-8}\cmidrule(lr){9-10}
Target & VN 3yrs & Target & MA 3½yrs & Target & EE 3½yrs & Target & IN 3½yrs & Target & GW 4yrs\\
\midrule
vékʷ\={u} & vétù & békʷǔ & békʷǔ & békʷǔ & békʷǔ & békʷǔ & békʷǔ & békʷù & békù & Greet\\
εkʷ\'{ã} & εtá & εkʷ\'{ã} & áŋkʷá & àkʷ\'{ã} & àk\'{ã} & àkʷ\'{ã} & ŋʹká & ákʷá & ákʷá & Cry\\
εkʷ\^{ã} & àtâ & εkʷâ & εkʷâ & àkʷá & àká & àkʷâ & á{\downstep}kʷâ & àkʷâ & àkʷâ & Egg\\
ɛŋʷʊ & εmʊ & ɛŋʷʊ & εmʊ & àŋʷʊ & àŋʷʊ & àŋʷʊ & áŋʷʊ & áŋʷʊ & áwʊ & Death\\
ɛkʷà & àtà & ɛkʷà & àkʷà & àkʷà & àkʷà & àkʷà & àkʷà & àkʷà & àkʷà & Bush fowl\\
ɛ{\downstep}ŋʷâ & ámà & ɔ{\downstep}ŋʷâ & εŋʷâ & ɔ{\downstep}ŋʷá & ɔ{\downstep}má & ɔ{\downstep}ŋʷâ & ɔ{\downstep}ŋʷâ & ɔ{\downstep}ŋʷâ & ɔ{\downstep}ŋʷâ & Moon\\
εh\^{ĩ} & εjɪ & εh\^{ĩ} & àjî & εh\'{ĩ} & àhɪ & àhʷʊ & àwʊ & àhʷʊ & àwʊ & Body\\
ʤɪ & ʤɪ & ʤɪ- & ʤɪ & ʤɪ & ʤɪ & gʷʊ & gɔ & gʷʊ & gʷʊ & Given (name) \\
ɔʧɪ & ɔʧɪ & ɔʧɪ & ɔʧɪ & ɔʧɪ & ɔʧɪ & ɔkʷʊ & ɔkʊ & ɔkʷʊ & ɔkʷʊ & Leg\\
\lspbottomrule
\end{tabular}
}
\caption{kw → k; kw → t; gw → g; hw → w, ŋw → m.}
\label{tab:alerechi:10}
\end{table}

\subsubsection{Substitution of plosive with fricative}\label{sec:alerechi:1.7.3}
\largerpage
The pattern of \isi{substitution} treated in this section reflects those involving [v] with [b], and vice versa, and that of [s] with [t]. While children from Emwh and \ili{Akpo} show preference for [b] instead of [v] of the target language, they, conversely, replace [v] with [b] as demonstrated in \tabref{tab:alerechi:11}. The data further show the children in the choice of [t] for [s] of \ili{Aluu}. The \isi{substitution} in this section agrees with the observation of \citet[242]{Crystal1997} that the replacement of \isi{fricatives} with stops is one of the possible trends for children in \isi{language acquisition}. It is worthy of note that this \isi{substitution} seem peculiar to children that are above age 3 to 4.    

\begin{table}
\fittable{
\begin{tabular}{llllllllllp{1.5cm}}
\lsptoprule
\multicolumn{2}{c}{Odeegnu} & \multicolumn{2}{c}{Emowha } & \multicolumn{2}{c}{\ili{Akpo} } & \multicolumn{2}{c}{\ili{Aluu} } & \multicolumn{2}{c}{Omuanwa} & Gloss\\\cmidrule(lr){1-2}\cmidrule(lr){3-4}\cmidrule(lr){5-6}\cmidrule(lr){7-8}\cmidrule(lr){9-10}
Target & VN\,3yrs & Target & MA\,3½yrs & Target & EE\,3½yrs & Target & IN\,3½yrs & Target & GW\,4yrs\\
\midrule 
t\'{õ}- & tɔ- & tò\~{r}\'{ũ} & tò\~{r}\'{ũ} & s\'{õ} & ʧ\'{õ} & só & só & tò & tò\~{r}\'{ũ} & `follow'\\
óvírízí & óvílìʤì & óbírízí & óbíjíz & óbírízí & óbílízí & óbírízí & óvírízí & óbúrúzù & óbúlúsù & `sympathy'\\
èvùlù & àvùlù & èvùlù & èvùnù & èvùlù & èbùlù & èbùlù & èbùrù & \`{m}fùlù & èvùlù & `ram'\\
dívjà & - & díbjà & dívjà & díbjà & - & díbjà & díbjà & díbjà & díbjà & `doctor'\\
tó & ʤɔ & tó & tó & só & - & só & só & tó & tó & `grow'\\
\`{m}vɔ & \`{m}vɔ & \`{m}vɔ & \`{m}bú & \`{m}bɔ & \`{m}bɔ & ḿ{\downstep}bɔ & ḿ{\downstep}bɔ & ḿ{\downstep}vɔ & ḿ{\downstep}fɔ & `comb~(in)'\\
ɔvɔʧɪ & ɔvɔʧɪ & ɔvɔʧɪ & ɔbɔʧɪ & ɔbɔʧɪ & ɔbɔʧɪ & ɔbɔʧɪ & ɔbɔʧɪ & ɔbɔʧɪ & ɔbɔʧɪ & `day'\\
òvèʤè & òvèʤì & òbèʤè & èvèʤè & òbòʤò & òbòjò & òbèʤè & òbòʤè & òbèʤè & òbòʤò & `mudskipper'\\
àkítì & àtítì & àkɪdɪ & àkɪdɪ & àkídì & - & àkísì & àkítì & ákɪdɪ & àkwà & Beans\newline (brown)\\
\lspbottomrule
\end{tabular}
}
\caption{s → t; v →b; b → v.}
\label{tab:alerechi:11}
\end{table}

\subsubsection{Substitution of fricative with fricative}\label{sec:alerechi:1.7.4}
 
In addition to substituting voiced stops with their voiceless counterparts, \tabref{tab:alerechi:12} further proves that the replacement of voiced consonants with their voiceless counterparts extends to the \isi{fricatives}. The \isi{substitution} is, however, predominant in the speech of an Omnw child of 4years old than those of Odgn of 3years and \ili{Akpo} of 3\textsuperscript{1}/\textsubscript{2 }yrs as the data demonstrate. Thus, the children replace [v] and [z] of the target language with [f] and [v], respectively. Our data show that the changes occur both word-initially and word-medially indicating that there is no conditioning factor for the change. The data further demonstrate the tendency of the child from Omnw replacing some of the vowels in initial position with a syllabic nasal. This additional peculiarity observed in the speech of the Omnw child, further strengthens the claim of the presence of a slight speech problem in this child’s language. This, however, requires further investigation in other to confirm our claim. 

\begin{table}
\fittable{
\begin{tabular}{lllllllllll}
\lsptoprule
\multicolumn{2}{c}{Odeegnu} & \multicolumn{2}{c}{Emowha } & \multicolumn{2}{c}{\ili{Akpo} } & \multicolumn{2}{c}{\ili{Aluu} } & \multicolumn{2}{c}{Omuanwa} & Gloss\\\cmidrule(lr){1-2}\cmidrule(lr){3-4}\cmidrule(lr){5-6}\cmidrule(lr){7-8}\cmidrule(lr){9-10}
Target & VN\,3yrs & Target & MA\,3½yrs & Target & EE\,3½yrs & Target & IN 3½yrs & Target & GW\,4yrs\\
\midrule
s\'{ã} & s\'{ã} & s\'{ã} & sá & z\'{ã} & - & z\'{ã} & zá & z\'{ã} & sá & `imitate'\\
ú{\downstep}sû & úʤù & ú{\downstep}sû & \'{n}{\downstep}s\^{ũ} & i{\downstep}z\'{ũ} & ɪʧu & í{\downstep}z\'{ũ} & í{\downstep}ʤû & ú{\downstep}z\'{ũ} & \'{n}{\downstep}s\^{ũ} & `corpse'\\
s\'{ũ} & ʧ\'{ũ} & s\'{ũ} & s\'{ũ} & Z\'{ũ} & zú & z\'{ũ} & ʤú & z\'{ũ} & sú & `steal'\\
ɔs\^{ʊ} & ɔʧʊ & ɔs\^{ʊ} & às\^{ʊ} & àz\'{ʊ} & àz\'{ʊ} & àz\^{ʊ} & àʤû & àz\^{ʊ} & às\^{ʊ} & `back'\\
ɔ{\downstep}s\^{ʊ} & ɔʧʊ & ɔs\`{ũ} & ɔs\`{ũ} & áz\`{ʊ} & áʧʊ & áz\`{ʊ} & áʤʊ & áz\`{ʊ} & ás\`{ʊ} & `fish'\\
zʊ & ʤʊ & zʊ & zʊ & zʊ & sʊ & zʊ & ʤʊ & zʊ & sʊ & `buy'\\
zɔ & ʤɔ & zɔ & zɔ & zɔ & ʤɔ & zɔ & zɔ & zɔ & sɔ & `step on'\\
ézè & - & ézè & ézè & ézè &  & ézè & éʤè & ézè & ésè & `king'\\
- & - & ɪvù & ɪvú & ɪvù & ɪvú & ɪbû & ɪ{\downstep}bû & ívû & ḿfû & `load'\\
ɔɲì & ɪvǔ & ɪvù & ɪvù & ɪvù & ɪvù & ɪbù & ɪbù & ɪvù & ḿfù & `fat'\\
èvùlù & àvùlù & èvùlù & èvùnù & èvùlù & èbùlù & èbùlù & èbùrù & \`{m}fùlù & èvùlù & `ram'\\
ɔzà & áʤà & ɔzʊzà & àzʊzà & ɔzʊzà & ɔsìsà & ɔzìzà & ɔʤìzà & ɔzìzà & ɔsìsà & `broom'\\
v\`{ã} & f\v{a} & v\`{ã} & v\`{ã} & b\`{ã}- & b\v{a} & b\'{ã} & b\v{a} & b\`{ã} & bà & `enter'\\
\lspbottomrule
\end{tabular}
}
\caption{v → f, z → s.}
\label{tab:alerechi:12}
\end{table}

\subsubsection{Substitution of fricative with affricate}\label{sec:alerechi:1.7.5}
\largerpage

\begin{table}[b]
\fittable{
\begin{tabular}{llllllllllp{1.5cm}}
\lsptoprule
\multicolumn{2}{c}{Odeegnu} & \multicolumn{2}{c}{Emowha } & \multicolumn{2}{c}{\ili{Akpo} } & \multicolumn{2}{c}{\ili{Aluu} } & \multicolumn{2}{c}{Omuanwa} & Gloss\\\cmidrule(lr){1-2}\cmidrule(lr){3-4}\cmidrule(lr){5-6}\cmidrule(lr){7-8}\cmidrule(lr){9-10}
Target & VN 3yrs & Target & MA 3½yrs & Target & EE 3½yrs & Target & IN 3½yrs & Target & GW 4yrs\\
\midrule
ú{\downstep}sû & úʤù & ú{\downstep}sû & \'{n}{\downstep}s\^{ũ} & i{\downstep}z\'{ũ} & ɪʧu & í{\downstep}z\'{ũ} & í{\downstep}ʤû & ú{\downstep}z\'{ũ} & \'{n}{\downstep}s\^{ũ} &  `corpse'\\
s\'{ũ} & ʧ\'{ũ} & s\'{ũ} & s\'{ũ} & Z\'{ũ} & zú & z\'{ũ} & ʤú & z\'{ũ} & sú & `steal'\\
ɔs\^{ʊ} & ɔʧʊ & ɔs\^{ʊ} & às\^{ʊ} & àz\'{ʊ} & àz\'{ʊ} & àz\^{ʊ} & àʤû & àz\^{ʊ} & às\^{ʊ} &  `back' \\
ɔ{\downstep}s\^{ʊ} & ɔʧʊ & ɔs\`{ũ} & ɔs\`{ũ} & áz\`{ʊ} & áʧʊ & áz\`{ʊ} & áʤʊ & áz\`{ʊ} & ás\`{ʊ} & `fish' \\
zʊ & ʤʊ & zʊ & zʊ & zʊ & sʊ & zʊ & ʤʊ & zʊ & sʊ & `buy' \\
zɔ & ʤɔ & zɔ & zɔ & zɔ & ʤɔ & zɔ & zɔ & zɔ & sɔ &  `step~on' \\
ó{\downstep}s\^{ũ} & óʧù & ó{\downstep}s\^{ũ} & ós\'{ũ} & ó{\downstep}s\'{ũ} & ó{\downstep}sú & é{\downstep}sû & é{\downstep}ʧû & é{\downstep}s\^{ũ} & \'{n}{\downstep}s\^{ũ} & `millipede'\\
sʊ & ʧʊ & sʊ & sʊ & sʊ & ʧʊ & sʊ & ʧʊ & sʊ & s\'{ʊ} & `pound (yam)' \\
s\'{ʊ} & ʧ\'{ʊ} & s\'{ʊ} & s\'{ʊ} & s\'{ʊ} & ʧ\'{ʊ} & s\'{ũ} & sú & s\'{ʊ} & s\'{ʊ} & `wash' \\
ɔsʊ & ɔʧʊ & ɔsʊ & ɔs\'{ʊ} & ɔsʊ & ɔʧʊ & ɔsʊ & ɔsʊ & ɔsʊ & ɔsʊ & `bat' \\
sɔ & ʧɔ & sɔ & sɔ & sɔ & ʧɔ & sɔ & sɔ & sɔ & sɔ & Forbid~or\newline respect \\
t\'{õ}- & tɔ & t\'{õ} & tò- & s\'{õ} & ʧ\'{õ} & s\'{õ} & só & s\`{õ} & tò- & Follow\\
zɪ & ʤɪ & zì & zì & zì & - & zɪ & zɪ & dɪ & tè & `is'\\
ǹzí & ǹʤì & ǹzí &  & ǹzí & ǹzí & ǹzí & ǹʤí & ǹdí & ìdí & `husband'\\
óvírízí & óvílìʤì & óbírízí & óbíjíz & óbírízí & óbílízí & óbírízí & óvírízí & óbúrúzù & óbúlúsù & `sympathy'\\
\lspbottomrule
\end{tabular}
}
\caption{Fricative versus affricate: s → ʧ; z → ʧ, z → ʤ.}
\label{tab:alerechi:13}
\end{table}

There is also the \isi{substitution} of the \isi{alveolar} \isi{fricatives} [s] and [z] of the adult utterance with the palato-\isi{alveolar} affricates [ʧ]and [ʤ], respectively, in children pronunciation. Thus, where Odgn, \ili{Akpo} and \ili{Aluu} adult articulate [s] or [z], the choice for the children is [ʧ] or [ʤ], respectively, indicating the \isi{affrication} of these \isi{fricatives}. Occasionally, the children substitute [s] for [ʤ] or [z] for [ʧ] as demonstrated in \tabref{tab:alerechi:13}. The \isi{substitution} of \isi{alveolar} \isi{fricatives} with palato-\isi{alveolar} affricates is restricted to the child of 3years, whereas the replacement of the voiced \isi{alveolar} \isi{fricative} with the voiceless counterpart seems peculiar to the child of 4. This phenomenon, though, geographically determined is observed in the \isi{adult speech}, the impression of this paper is that it may be a case of speech impediment in the utterance of this 4 year old child.



  
\subsubsection{Substitution of fricative with approximant/null}\label{sec:alerechi:1.7.6}

The majority of the children have not acquired the \isi{glottal} \isi{fricative} [h]. The data in \tabref{tab:alerechi:14} demonstrate that they either delete it wherever it occurs in the target form or replace it with [j]. See \tabref{tab:alerechi:14}.

\begin{table}
\fittable{
\begin{tabular}{llllllllllp{1cm}}
\lsptoprule
\multicolumn{2}{c}{Odeegnu} & \multicolumn{2}{c}{Emowha } & \multicolumn{2}{c}{\ili{Akpo} } & \multicolumn{2}{c}{\ili{Aluu} } & \multicolumn{2}{c}{Omuanwa} & Gloss\\\cmidrule(lr){1-2}\cmidrule(lr){3-4}\cmidrule(lr){5-6}\cmidrule(lr){7-8}\cmidrule(lr){9-10}
Target & VN 3yrs & Target & MA 3½yrs & Target & EE 3½yrs & Target & IN 3½yrs & Target & GW 4yrs\\
\midrule
h\'{ã} & \'{ã} & - & - & ƥě & pě &  & - & ƥé & pè & Peel\newline (orange) \\
úhjé & újeá & úhjé & újé & ɪhjé & ɪjé & ɪhjé & ɪjé & úhjí & újé & Red \\
εh\^{ĩ} & εjɪ & εh\^{ĩ} & àjî & εh\'{ĩ} & àhɪ & àhʷʊ & àwʊ & àhʷʊ & àwʊ & Body\\
\lspbottomrule
\end{tabular}
}
\caption{h → j, h →${\emptyset}$.}
\label{tab:alerechi:14}
\end{table}

\subsubsection{Substitution of tap with lateral or nasal}\label{sec:alerechi:1.7.7}

The tendency is also recorded of \ili{Ikwere} children to use the \isi{alveolar} lateral approximant instead of tap or nasal. Thus, where the adult use the \isi{alveolar}[r], the children show preference for the \isi{alveolar} lateral [l] or the \isi{alveolar} nasal [n]. This \isi{substitution} is characteristic of children that cut across ages 3 to 4 as shown in \tabref{tab:alerechi:15}. 

\begin{table}
\fittable{
\begin{tabular}{llllllllllp{1.5cm}}
\lsptoprule
\multicolumn{2}{c}{Odeegnu} & \multicolumn{2}{c}{Emowha } & \multicolumn{2}{c}{\ili{Akpo} } & \multicolumn{2}{c}{\ili{Aluu} } & \multicolumn{2}{c}{Omuanwa} & Gloss\\\cmidrule(lr){1-2}\cmidrule(lr){3-4}\cmidrule(lr){5-6}\cmidrule(lr){7-8}\cmidrule(lr){9-10}
Target & VN 3yrs & Target & MA 3½yrs & Target & EE 3½yrs & Target & IN 3½yrs & Target & GW 4yrs\\
\midrule
ɛƥ\'{ã}r\'{ã} & ɔpálá & ɔƥ\'{ã}r\'{ã} & ɔpálá & ɔƥ\'{ã}r\'{ã} & ɔpárá & ɔƥ\'{ã}r\'{ã} & ɔpáná & ɔƥ\'{ã}r\'{ã} & ɔpálá & `first son'\\
óvírízí & óvílìʤì & óbírízí & óbíjíz & óbírízí & óbílízí & óbírízí & óvírízí & óbúrúzù & óbúlúsù & `sympathy' \\
\`{m}ƥòró ákâ & \`{m}pòtó átà & \`{m}ƥòró ákâ & \`{m}pèrè-áká & \`{m}ƥèré-ák\={a} & - & ìsíní ákâ & ìpèní á{\downstep}kâ & ḿƥèré ákâ & ípèlé á{\downstep}kâ & `elbow' \\
ùrì & ùlì & ùrì & ùlì & \`{I}rì & ìlì & ìrì & ùlì & ùrì & ùlì & `indigo' \\
rí & - & rí & rí & rí & - & rí & lí & rí & lí & `eat'\\
ŋʷɔ & - & ŋʷɔ & ŋʷ\'{õ} & ŋʷɔ & - & rí & wɔ & rí & lí/lílí & `drink' \\
ɔ{\downstep}l\^{o} & ɔrɔ & ó{\downstep}l\^{o} & ó{\downstep}r\^{o} & ó{\downstep}ló & ɔ{\downstep}rɔ & é{\downstep}l\^{o} & á{\downstep}rɔ & é{\downstep}l\^{o} & é{\downstep}l\^{o} & `antelope' \\
\lspbottomrule
\end{tabular}
}
\caption{r →l, l → r.}
\label{tab:alerechi:15}
\end{table}

\subsubsection{Substitution of implosive with plosive}\label{sec:alerechi:1.7.8}

The replacement of the \isi{labial implosive} [ƥ] of the target language with the \isi{labial plosive} [p] serves as another trend in the speech of \ili{Ikwere} children. Thus, where the choice in the target language is [ƥ], the children use [p] as demonstrated in \tabref{tab:alerechi:16}. The data show that the \isi{substitution} of [ƥ] with [p] cuts across ages 3 to 4. A similar trend is observed with the voiced counterparts [ƃ] and [b]. This implies that the acquisition of [ƥ] and [ƃ] is a later development in the language of children.

\begin{table}
\fittable{
\begin{tabular}{llllllllllp{1.5cm}}
\lsptoprule
\multicolumn{2}{c}{Odeegnu} & \multicolumn{2}{c}{Emowha } & \multicolumn{2}{c}{\ili{Akpo} } & \multicolumn{2}{c}{\ili{Aluu} } & \multicolumn{2}{c}{Omuanwa} & Gloss\\\cmidrule(lr){1-2}\cmidrule(lr){3-4}\cmidrule(lr){5-6}\cmidrule(lr){7-8}\cmidrule(lr){9-10}
Target & VN 3yrs & Target & MA 3½yrs & Target & EE 3½yrs & Target & IN 3½yrs & Target & GW 4yrs\\
\midrule
ɛƥ\'{ã}r\'{ã} & ɔpálá & ɔƥ\'{ã}r\'{ã} & ɔpálá & ɔƥ\'{ã}r\'{ã} & ɔpárá & ɔƥ\'{ã}r\'{ã} & ɔpáná & ɔƥ\'{ã}r\'{ã} & ɔpálá & First son\\
\`{m}ƥòró ákâ & \`{m}pòtó átà & \`{m}ƥòró ákâ & \`{m}pèrè-áká & \`{m}ƥèré-ák\={a} &  & ìsíní ákâ & ìpèní á{\downstep}kâ & ḿƥèré ákâ & ípèlé á{\downstep}kâ & Elbow\\
ƥ\'{ʊ} & - & ƥ\'{ʊ} & pé & ƥ\'{ʊ} &  & ƥ\'{ʊ} & pé & ƥ\'{ʊ} & pè & Scrape\\
ƥó & - & ƥ\'{õ} & pó & ƥ\'{õ} & Pó & ƥ\v{õ} & p\v{o} & ƥ\'{õ} & pó & Pack waste\\
h\'{ã} & \'{ã} & - & - & ƥě & pě & - & - & ƥé & pè & Peel (orange)\\
- & - & v\'{õ} & vú & ƥ\'{õ} & Pó & b\'{õ} & bó & b\'{õ} & bó & Accuse\\
àƃà & àbà & àƃà & àbà & àb\`{ã} & àbà & àbà & àbà & àbà & àpà & Jaw\\
\lspbottomrule
\end{tabular}
}
\caption{ƥ → p, ƃ → b.}
\label{tab:alerechi:16}
\end{table}

The data in this paper demonstrate that sound \isi{substitution} in \isi{child language} also involves the vowels. From the data, however, the replacement of sounds involving vowel are not as recurrent as those of the consonants. This agrees with previous studies that vowels are acquired earlier by children than consonants. By the age of 3 years most vowels sounds would have been established hence, no need for much \isi{substitution}. Tonal \isi{substitution} in the language is not significant in children speech as sown in most of the data. 

\subsection{Substitution processes}\label{sec:alerechi:1.8}

The various \isi{substitution} patterns observed in this paper give additional evidence of the simplification of adult (target) language by children. It is observed that as children develop, the substituted sounds are dropped to conform to the adult forms when they have gained greater articulatory control. Recalling that the substitutions are not haphazard but rule governed, an interesting question would be what rules do children impose to simplify the \isi{adult language}? In other words, what are the phonological processes operating to relate the child utterances with the target forms? Considering the divergent nature of the \ili{Ikwere} language, \citet{Alerechi2007a} identified a number of phonological processes relating one form of speech of a particular geographical location with the form of the others, one of which is the reflex of the proto-form, while others are likely innovations. Thus this section does not only identify the phonological processes in operation, but also draws attention to the processes that are identical with those of the adults as in the speech of different geographical areas and those that are typical of \isi{child language}. The following subsections discuss the phonological processes observed in this paper.

\subsection{Strengthening and spirantization (weakening)}\label{sec:alerechi:1.9}

Bearing in mind the replacement of plosives with \isi{fricatives} and vice versa, I observed the processes of strengthening and \isi{spirantization} (weakening), respectively, in \tabref{tab:alerechi:11}. \citet[262]{Alerechi2007a} earlier observed the \isi{spirantization} (weakening) of [b] of some dialects to [v] in some other dialects but not the strengthening of [v] to [b]. In \isi{child language}, however, there is an addition of the phonological process of strengthening of [v] of the adult form to [b] showing that the process reflecting the change of [v] to [b] is typical of children in \ili{Ikwere}, whereas that involving the change of [b] to [v] is identical with the \isi{adult speech}. 

\subsection{Loss of suction and a shift to plosive}\label{sec:alerechi:1.10}

Concerning the \isi{substitution} of [ƥ] with [p], we note a loss of suction of \isi{labial implosive} of the \isi{adult language} and a shift to \isi{labial plosive} in \isi{child language}. \citet[260]{Alerechi2007a} recognized this phonological process in \ili{Ikwere}. Here children from the dialect areas with the preponderant use of [ƥ] show preference for [p] of some other dialect areas. This change does not only reveal that the phonological process in \isi{child language} is identical with the \isi{adult language}, it also proves that phonological \isi{substitution} characteristic of \isi{child language} is rule governed.

\subsection{(Successive) affrication (and (de)voicing)}\label{sec:alerechi:1.11}

For the change of \isi{fricatives} with affricates, two phonological processes are in operation. One is the \isi{affrication} of [s] to [ʧ] and [z] to [ʤ] depending on voicing. Another is the \isi{successive affrication} and voicing or \isi{devoicing} of \isi{fricatives} as in [s] changing to [ʤ] and [z] changing to [ʧ]. Even though \citet[261]{Alerechi2007a} observed \isi{affrication} in \isi{adult language}, this \isi{affrication} is different as it reflects [s] and [ʧ] or [z] and [ʤ] and not [t], [ʧ], [ʃ] and [s] as in the \isi{adult language}.

\subsection{De-labialization}\label{sec:alerechi:1.12}

De-labialization is a process whereby the feature of lip rounding on the primary stricture is lost. In \tabref{tab:alerechi:10}, it is observed that all the subjects replaced the main stricture in either one or more of these changes [kʷ→ k; kʷ→ t; gʷ→ g; hʷ→ w, ŋʷ→ m]. Even though a child may have produced any of the \isi{labialized} sounds accurately, \isi{labialized} sounds are among the difficult consonants for the children to acquire since all the subjects manifested de-labialization process. 

\subsection{Gliding}\label{sec:alerechi:1.13}

Gliding in child phonology is a process whereby any consonant is realized as a glide \citep[255]{Yul-Ifode2008}. In addition to producing accurately [j] as in the target form, the Odeegnu and Emowha subjects produce [h] as [j] intervocalically, indicating that [j] is among the early sounds in \isi{child language acquisition}.

\subsection{Denasalization}\label{sec:alerechi:1.14}

Even though this paper focuses on consonants, we observed a preponderance loss of \isi{nasalization} of nasalized vowels in \isi{child language} in our data especially in \tabref{tab:alerechi:16}. Thus the phonological process of \isi{denasalization} of certain vowel segments of the target language is evident of \isi{child language}. Though \citet[249]{Alerechi2007a} observed \isi{denasalization} in the \isi{adult language}, it is not as predominant as that in \isi{child language}, showing that loss of \isi{nasalization} is one of the ways children actually simplify their pronunciation. In fact nasalizing a vowel segment requires extra energy or force and more natural for children to neglect it than the adults. This explains why we record a preponderance loss of \isi{nasalization} in children speech forms more than those of the adults.

\section{Interpretations}\label{sec:alerechi:2}

The foregoing substitutions of segments in the language of \ili{Ikwere} children are evidence of simplification of items of the target pronunciation of the \ili{Ikwere} consonants. In an attempt to articulate the sounds of the language, \ili{Ikwere} children, like other children, change the sounds when trying to attain the target form thus, resulting in imperfect rendering of some of the target sounds. The imperfect representations of the adult sounds generate ambiguous forms and this contributes significantly to the communication gap existing between children and adults. While the children understand the adults, but find it difficult to communicate effectively, the adults, on the other hand, have achieved greater articulatory control of the target language, but cannot grasp fully the intentions of the children. This could create serious problems such as frustration on the part of the adults, particularly, the impatient ones and dissatisfaction, resentment and most likely, seclusion on the part of the children if the gap is not bridged as the adult may not reach their needs. However, it is the parents that understand their children better than any other person. 

The various phonological processes observed in \sectref{sec:alerechi:5}, which demonstrate that the processes in both adults and children’s forms of speech are to a great extent identical with slight differences also has implication. The similarities indicate that the gap in communication existing between adults and children could be bridgedparticularly if the adults are conversant with the forms of the varieties spoken in other geographical locations. For the forms peculiar to children, an exposure to what should be expected could facilitate communication and reduce communication gap to the barest minimum.

\subsection{Performance scores of consonants}\label{sec:alerechi:2.1}

\tabref{tab:alerechi:17} represents the phonetic consonants of \ili{Ikwere} observed in the data. It shows the articulatory activity of each subject. The total occurrence of each consonant is obtained by counting the number of occurrences of the sound in the words elicited from both the target and the subjects. These were converted to quantitative data and presented as percentages using bar charts. The bar charts displaying the performance of the children from Odeegnu, Emowha, \ili{Akpo}, \ili{Aluu} and \ili{Omuanwa} are represented in \figref{fig:alerechi:3} to \figref{fig:alerechi:7}. It should be noted that twenty-three (23) consonants were observed in the data collected. Due to dialectal difference all the consonants did not manifested in the target of all the dialects. Thus Odeegnu records 17; Emowha and \ili{Akpo} have 19 consonants each, while \ili{Aluu} and \ili{Omuanwa} used 21 and 23, respectively.

% \todo[inline]{Last Column appears to be missing?}
\begin{sidewaystable}
\begin{tabular}{l *{15}{d{1}} }
\lsptoprule
Sound & \multicolumn{3}{c}{Odeegnu} & \multicolumn{3}{c}{Emowha} & \multicolumn{3}{c}{Akpo} & \multicolumn{3}{c}{Aluu} & \multicolumn{3}{c}{Omuanwa}\\\cmidrule(lr){2-4}\cmidrule(lr){5-7}\cmidrule(lr){8-10}\cmidrule(lr){11-13}\cmidrule(lr){14-16}
& T & 1 & \% & T & 2 & \% & T & 3 & \% & T & 4 & \% & T & 5 & \%\\
\midrule
ŋʷ & 3 & 0 & 0 & 3 & 2 & 66.7 & 2 & 1 & 50 & 2 & 2 & 100 & 2 & 1        & 50\\
p & - &  &  & - &  &  & 1 & 1 & 100 & 3 & 3 & 100 & 2 & 2               & 100\\
b & - &  &  & 4 & 2 & 50 & 9 & 9 & 100 & 16 & 15 & 93.8 & 12 & 11       & 91.7\\
t & 4 & 3 & 75 & 4 & 4 & 100 & 2 & 2 & 100 & 2 & 2 & 100 & 2 & 2        & 100\\
d & 3 & 2 & 66.7 & 4 & 4 & 100 & 2 & 2 & 100 & 3 & 3 & 100 & 6 & 4      & 66.7\\
k & 3 & 1 & 33.3 & 4 & 4 & 100 & 1 & 1 & 100 & 3 & 3 & 100 & 4 & 3      & 75\\
g & 1 & 0 & 0 & 2 & 2 & 100 & 1 & 1 & 100 & 3 & 3 & 100 & 2 & 2         & 100\\
kʷ & 4 & 0 & 0 & 4 & 3 & 75 & 5 & 3 & 60 & 5 & 3 & 60 & 7 & 6           & 85.7\\
gʷ & - &  &  & - &  &  & - &  &  & 1 & 0 & 0 & 1 & 1                    & 100\\
ƥ & 2 & 0 & 0 & 5 & 0 & 0 & 3 & 0 & 0 & 4 & 0 & 0 & 5 & 1               & 20\\
ƃ & - &  &  & 2 & 0 & 0 & - &  &  & 1 & 1 & 100 & 1 & 0                 & 0\\
ʧ & 2 & 2 & 100 & 4 & 4 & 100 & 4 & 4 & 100 & 1 & 1 & 100 & 1 & 1       & 100\\
ʤ & 4 & 4 & 100 & 4 & 4 & 100 & 4 & 3 & 75 & 2 & 2 & 100 & 2 & 2        & 100  \\
f & 5 & 4 & 80 & 4 & 4 & 100 & 1 & 1 & 100 & - &  &  & 2 & 2            & 100\\
v & 10 & 9 & 90 & 10 & 7 & 70 & 4 & 3 & 75 & - &  &  & 4 & 0            & 0\\
s & 11 & 2 & 18.2 & 11 & 11 & 100 & 7 & 2 & 28.6 & 9 & 6 & 66.7 & 8 & 6 & 75\\
z & 6 & 0 & 0 & 8 & 8 & 100 & 10 & 4 & 40 & 13 & 5 & 38.5 & 11 & 0      & 0\\
h & 3 & 0 & 0 & 2 & 0 & 0 & 2 & 1 & 50 & 1 & 0 & 0 & 1 & 0              & 0\\
hʷ & - &  &  & - &  &  & - &  &  & 1 & 0 & 0 & 1 & 0                    & 0\\
r & 6 & 0 & 0 & 7 & 3 & 42.9 & 4 & 1 & 25 & 7 & 1 & 14.3 & 8 & 0        & 0\\
j & 3 & 2 & 66.7 & 4 & 4 & 100 & 4 & 4 & 100 & 5 & 5 & 100 & 5 & 5      & 100\\
w & - &  &  & - &  &  & - &  &  & 1 & 0 & 0 & 1 & 0                     & 0\\
l & 4 & 3 & 75 & 4 & 2 & 50 & 4 & 2 & 50 & 4 & 1 & 25 & 4 & 3           & 75\\
\midrule 
Total & 74 & 32 & 43.2 & 90 & 68 & 75.6 & 70 & 45 & 64.3 & 87 & 56 & 64.4 & 92 & 52 & 56.5\\
\lspbottomrule
\end{tabular}
\caption{Frequency of occurrence for consonants.}
\label{tab:alerechi:17}
\end{sidewaystable}

\subsection{Consonants in Odeegnu}\label{sec:alerechi:2.2}

\tabref{tab:alerechi:17} and \figref{fig:alerechi:3} show that the Odeegnu subject aged 3 years scored 100\% in the production of [ʧ] and [ʤ]. It could be that the number of these sounds is limited or that the child has mastered them completely. The subject has acquired [v f t l d j] to a reasonable extent but not completely. The table and figure also show that the subject could not produce [ŋʷg kʷ ƥ z r] and had difficulty in pronouncing [k]. The total performance of the subject in the production of the target consonants is 43.2\%, which is below the average.

\begin{figure}[p]
\begin{tikzpicture}
\begin{axis}[
  ybar,
  ymin=0,
  nodes near coords,
  symbolic x coords={C, C3,  C4,  C5,  C6,  C7, C9, CS1, CS2, CS3, CS4, CS5, CS6, CS7,CS9, S1, S3},
  xtick=data,
  width=\textwidth,
  height=6cm,
  xticklabel style={font=\footnotesize},
  ylabel=Percentage,
  xlabel={Consonants in Odeegnu}
]

\addplot [fill=\lsSeriesColor!40] coordinates {
(C,0)
(C3,75)
(C4,66.66667)
(C5,33.33333)
(C6,0)
(C7,0)
(C9,0)
(CS1,100)
(CS2,100)
(CS3,80)
(CS4,90)
(CS5,18.18182)
(CS6,0)
(CS7,0)
(CS9,0)
(S1,66.66667)
(S3,75)
};
\end{axis}
\end{tikzpicture}
\begin{tabularx}{\textwidth}{XXXXXXXXXXXXXXXXXXXXXXX}
\lsptoprule
\textsc{c} & \textsc{c}\oldstylenums{1} & \textsc{c}\oldstylenums{2} & \textsc{c}\oldstylenums{3} & \textsc{c}\oldstylenums{4} & \textsc{c}\oldstylenums{5} & \textsc{c}\oldstylenums{6} & \textsc{c}\oldstylenums{7} & \textsc{c}\oldstylenums{8} & \textsc{c}\oldstylenums{9} & \textsc{cs} & \textsc{cs}\oldstylenums{1} & \textsc{cs}\oldstylenums{2} & \textsc{cs}\oldstylenums{3} & \textsc{cs}\oldstylenums{4} & \textsc{cs}\oldstylenums{5} & \textsc{cs}\oldstylenums{6} & \textsc{cs}\oldstylenums{7} & \textsc{cs}\oldstylenums{8} & \textsc{cs}\oldstylenums{9} & \textsc{s}\oldstylenums{1} & \textsc{s}\oldstylenums{2} & \textsc{s}\oldstylenums{3}\\
\midrule
ŋʷ & p & b & t & d & k & g & kʷ & gʷ & ƥ & ƃ & ʧ & ʤ & f & v & s & z & h & hʷ & r & j & w & l\\
\lspbottomrule
\end{tabularx}
\caption{The Odeegnu subject performance in consonants.}
\label{fig:alerechi:3}
\end{figure}

\subsection{Consonants in Emowha}\label{sec:alerechi:2.3}


\begin{figure}[p]
\begin{tikzpicture}
\begin{axis}[
  ybar,
  ymin=0,
  nodes near coords,
  symbolic x coords={C, C2, C3,  C4,  C5,  C6,  C7, C9,  CS, CS1, CS2, CS3, CS4, CS5, CS6, CS7, CS9, S1, S3},
  xtick=data,
  width=\textwidth,
    height=6cm,
  xticklabel style={font=\footnotesize},
  ylabel=Percentage,
  xlabel={Consonants in Emowha}
]
\addplot [fill=\lsSeriesColor!40] coordinates {
(C, 66.66667)
(C2, 50)
(C3, 100)
(C4, 100)
(C5, 100)
(C6, 100)
(C7, 75)
(C9, 0)
(CS, 0)
(CS1,100)
(CS2,100)
(CS3,100)
(CS4,70)
(CS5,100)
(CS6,100)
(CS7,0)
(CS9,42.85714)
(S1, 100)
(S3, 50)
};
\end{axis}
\end{tikzpicture}\caption{The Emowha subject performance in consonants.}
\label{fig:alerechi:4}
\begin{tabularx}{\textwidth}{XXXXXXXXXXXXXXXXXXXXXXX}
\lsptoprule
\textsc{c} & \textsc{c}\oldstylenums{1} & \textsc{c}\oldstylenums{2} & \textsc{c}\oldstylenums{3} & \textsc{c}\oldstylenums{4} & \textsc{c}\oldstylenums{5} & \textsc{c}\oldstylenums{6} & \textsc{c}\oldstylenums{7} & \textsc{c}\oldstylenums{8} & \textsc{c}\oldstylenums{9} & \textsc{cs} & \textsc{cs}\oldstylenums{1} & \textsc{cs}\oldstylenums{2} & \textsc{cs}\oldstylenums{3} & \textsc{cs}\oldstylenums{4} & \textsc{cs}\oldstylenums{5} & \textsc{cs}\oldstylenums{6} & \textsc{cs}\oldstylenums{7} & \textsc{cs}\oldstylenums{8} & \textsc{cs}\oldstylenums{9} & \textsc{s}\oldstylenums{1} & \textsc{s}\oldstylenums{2} & \textsc{s}\oldstylenums{3}\\
\midrule 
ŋʷ & p & b & t & d & k & g & kʷ & gʷ & ƥ & ƃ & ʧ & ʤ & f & v & s & z & h & hʷ & r & j & w & l\\
\lspbottomrule
\end{tabularx}
\end{figure}


From \tabref{tab:alerechi:17} and \figref{fig:alerechi:4}, it is observed that the Emowha male subject of 3 ½ years produced [t d k g ʧ ʤ f s z j] accurately and above the average in [kʷ v ŋʷ]. He, however, scored zero percent in the production of [ƥ ƃ h]. This subject seems to have acquired most of target sounds as against those that are yet to be included in his inventory. \tabref{tab:alerechi:17} puts the total performance of the Emowha subject in the production of target consonants at 75.6\%.

\subsection{Consonants in Akpo}\label{sec:alerechi:2.4}

From \tabref{tab:alerechi:17} and \figref{fig:alerechi:5}, the \ili{Akpo} female subject of 3 ½ years scored 100\% in the production of [p b t d k g ʧ f j], above average in [ʤ v kʷ] and has attained average score in [ŋʷ h l]. On the other hand, she scored zero percent in the production of [ƥ] and less than average in [r s z]. \tabref{tab:alerechi:17} also shows that the subject has acquired 64.3\% in the production all the target sounds.

\begin{figure}
\begin{tikzpicture}
\begin{axis}[
  ybar,
  ymin=0,
  nodes near coords,
  symbolic x coords={C, C1, C2,  C3,  C4,  C5,  C6,  C7, C9, CS1, CS2, CS3, CS4, CS5, CS6, CS7, CS9, S1, S3},
  xtick=data,
  width=\textwidth,
    height=6cm,
  xticklabel style={font=\footnotesize},
  ylabel=Percentage,
  xlabel={Consonants in Akpo}
]
\addplot [fill=\lsSeriesColor!40] coordinates {
(C,50)
(C1,100)
(C2,100)
(C3,100)
(C4,100)
(C5,100)
(C6,100)
(C7,60)
(C9,0)
(CS1,100)
(CS2,75)
(CS3,100)
(CS4,75)
(CS5,28.57143)
(CS6,40)
(CS7,50)
(CS9,25)
(S1,100)
(S3,50)};
\end{axis}
\end{tikzpicture}
\caption{The Akpo subject performance in consonants.}
\label{fig:alerechi:5}
\begin{tabularx}{\textwidth}{XXXXXXXXXXXXXXXXXXXXXXX}
\lsptoprule
\textsc{c} & \textsc{c}\oldstylenums{1} & \textsc{c}\oldstylenums{2} & \textsc{c}\oldstylenums{3} & \textsc{c}\oldstylenums{4} & \textsc{c}\oldstylenums{5} & \textsc{c}\oldstylenums{6} & \textsc{c}\oldstylenums{7} & \textsc{c}\oldstylenums{8} & \textsc{c}\oldstylenums{9} & \textsc{cs} & \textsc{cs}\oldstylenums{1} & \textsc{cs}\oldstylenums{2} & \textsc{cs}\oldstylenums{3} & \textsc{cs}\oldstylenums{4} & \textsc{cs}\oldstylenums{5} & \textsc{cs}\oldstylenums{6} & \textsc{cs}\oldstylenums{7} & \textsc{cs}\oldstylenums{8} & \textsc{cs}\oldstylenums{9} & \textsc{s}\oldstylenums{1} & \textsc{s}\oldstylenums{2} & \textsc{s}\oldstylenums{3}\\
\midrule 
ŋʷ & p & b & t & d & k & g & kʷ & gʷ & ƥ & ƃ & ʧ & ʤ & f & v & s & z & h & hʷ & r & j & w & l\\
\lspbottomrule
\end{tabularx}
\end{figure}

\subsection{Consonants in Aluu}\label{sec:alerechi:2.5}

\tabref{tab:alerechi:17} and \figref{fig:alerechi:6} show the \ili{Aluu} male subject of 3 ½ years with 100\% score in the production of the consonants [ŋʷ p t d k g ƃ ʧ ʤ j], 93.8\% in [b] and above the average in [s kʷ]. However, the subject scored zero percent in the production of [gʷ ƥ h hʷ w], and 14.3\% and 25\% performance scores in [r] and [l], respectively. The total percentage for the production of all the target consonants is 64.4\%, indicating above average mastery of the target sounds.

\begin{figure}
\begin{tikzpicture}
\begin{axis}[
  ybar,
  ymin=0,
  nodes near coords,
  symbolic x coords={C, C1, C2,  C3,  C4,  C5,  C6,  C7,  C8,  C9,  CS, CS1, CS2, CS5, CS6, CS7, CS8, CS9, S1, S2, S3},
  xtick=data,
  width=\textwidth,
    height=6cm,
  xticklabel style={font=\footnotesize},
  ylabel=Percentage,
  xlabel={Consonants in Aluu}
]
\addplot [fill=\lsSeriesColor!40] coordinates {
(C,100)
(C1,100)
(C2,93.75)
(C3,100)
(C4,100)
(C5,100)
(C6,100)
(C7,60)
(C8,0)
(C9,0)
(CS,100)
(CS1,100)
(CS2,100)
(CS5,55.55556)
(CS6,0)
(CS7,0)
(CS8,100)
(CS9,71.42857)
(S1,0)
(S2,0)
(S3,25)
};
\end{axis}
\end{tikzpicture}
\caption{The Aluu subject performance in consonants.}
\label{fig:alerechi:6}
\begin{tabularx}{\textwidth}{XXXXXXXXXXXXXXXXXXXXXXX}
\lsptoprule
\textsc{c} & \textsc{c}\oldstylenums{1} & \textsc{c}\oldstylenums{2} & \textsc{c}\oldstylenums{3} & \textsc{c}\oldstylenums{4} & \textsc{c}\oldstylenums{5} & \textsc{c}\oldstylenums{6} & \textsc{c}\oldstylenums{7} & \textsc{c}\oldstylenums{8} & \textsc{c}\oldstylenums{9} & \textsc{cs} & \textsc{cs}\oldstylenums{1} & \textsc{cs}\oldstylenums{2} & \textsc{cs}\oldstylenums{3} & \textsc{cs}\oldstylenums{4} & \textsc{cs}\oldstylenums{5} & \textsc{cs}\oldstylenums{6} & \textsc{cs}\oldstylenums{7} & \textsc{cs}\oldstylenums{8} & \textsc{cs}\oldstylenums{9} & \textsc{s}\oldstylenums{1} & \textsc{s}\oldstylenums{2} & \textsc{s}\oldstylenums{3}\\
\midrule 
ŋʷ & p & b & t & d & k & g & kʷ & gʷ & ƥ & ƃ & ʧ & ʤ & f & v & s & z & h & hʷ & r & j & w & l\\
\lspbottomrule
\end{tabularx}
\end{figure}

\subsection{Consonants in Omuanwa}\label{sec:alerechi:2.6}



The \ili{Omuanwa} male subject of 4 years scored 100\% in the production of [p t g gʷ ʧ ʤ f j]; 91.7\% and 85.7\% in [b] and [kʷ], respectively, and 75\% in [k s l]. For the difficult sounds, the subject obtained zero percent score for [ƃ v z h hʷ r w] and 20\% for [ƥ]. The total performance of this subject in the production of all the consonants is 56.5\% of the target. These are shown in \tabref{tab:alerechi:17} and \figref{fig:alerechi:7}.

\begin{figure}
\begin{tikzpicture}
\begin{axis}[
  ybar,
  ymin=0,
  nodes near coords,
  symbolic x coords={C, C1, C2,  C3,  C4,  C5,  C6,  C7,  C8,  C9,  CS, CS1, CS2, CS3, CS4, CS5, CS6, CS7, CS8, CS9, S1, S2, S3},
  xtick=data,
  width=\textwidth,
    height=6cm,
  xticklabel style={font=\footnotesize},
  ylabel=Percentage,
  xlabel={Consonants in Omuanwa}
]
\addplot [fill=\lsSeriesColor!40] coordinates {
(C,50)
(C1,100)
(C2,91.66667)
(C3,100)
(C4,66.66667)
(C5,75)
(C6,100)
(C7,85.71429)
(C8,100)
(C9,20)
(CS,0)
(CS1,100)
(CS2,100)
(CS3,100)
(CS4,0)
(CS5,75)
(CS6,0)
(CS7,0)
(CS8,0)
(CS9,0)
(S1,100)
(S2,0)
(S3,75)};
\end{axis}
\end{tikzpicture}
\caption{The Omuanwa subject performance in consonants.}
\label{fig:alerechi:7}
\begin{tabularx}{\textwidth}{XXXXXXXXXXXXXXXXXXXXXXX}
\lsptoprule
\textsc{c} & \textsc{c}\oldstylenums{1} & \textsc{c}\oldstylenums{2} & \textsc{c}\oldstylenums{3} & \textsc{c}\oldstylenums{4} & \textsc{c}\oldstylenums{5} & \textsc{c}\oldstylenums{6} & \textsc{c}\oldstylenums{7} & \textsc{c}\oldstylenums{8} & \textsc{c}\oldstylenums{9} & \textsc{cs} & \textsc{cs}\oldstylenums{1} & \textsc{cs}\oldstylenums{2} & \textsc{cs}\oldstylenums{3} & \textsc{cs}\oldstylenums{4} & \textsc{cs}\oldstylenums{5} & \textsc{cs}\oldstylenums{6} & \textsc{cs}\oldstylenums{7} & \textsc{cs}\oldstylenums{8} & \textsc{cs}\oldstylenums{9} & \textsc{s}\oldstylenums{1} & \textsc{s}\oldstylenums{2} & \textsc{s}\oldstylenums{3}\\
\midrule 
ŋʷ & p & b & t & d & k & g & kʷ & gʷ & ƥ & ƃ & ʧ & ʤ & f & v & s & z & h & hʷ & r & j & w & l\\
\lspbottomrule
\end{tabularx}

\end{figure}

\section{Discussion}\label{sec:alerechi:3}

A close look at the performance of the subjects demonstrates that the majority of the subjects recorded 100\% accuracy in the production of 9 to 10 consonants, except the Odeegnu subject that recorded only 2 consonants as shown in \tabref{tab:alerechi:17} and \figref{fig:alerechi:3} to \figref{fig:alerechi:7}.

Comparing the total performance of target sounds by the subjects, therefore, it is observed that the Emowha subject has acquired a greater percentage of 75.6 of the adult sounds, followed by the percentage scores of 64.4 and 64.3 by the \ili{Aluu} and \ili{Akpo} subjects, respectively. The \ili{Omuanwa} subject scored 56.5\% and Odeegnu scored below average of 43.2\%. While the reason for the low performance of the Odeegnu subject could be attributed to age factor (3 years) that of the \ili{Omuanwa} subject may be due to delayed acquisition of the target or slight speech disorders. This, of course, requires further investigation before conclusion could be drawn.

\newpage 
Having noted the performance of individual subject in the target consonant, one may say that the easy sounds for \ili{Ikwere} subjects generally are [p t k g d b ʧ ʤ f j] particularly if they occur in the target speech. On the other hand, [ƥ r h hʷ] appear more problematic than other consonants. Some others not listed as either easy or difficult may be easy or difficult based on the unique articulatory performance of the subject. Thus, the plosives, affricates, \isi{fricatives} and approximants seem not to constitute areas of difficulty, whereas the \isi{implosives}, tap and \isi{glottal} \isi{fricatives} do. 

\section{Conclusions}\label{sec:alerechi:4}

Inthis paper, we have been able to identify the various changes children impose on the \ili{Ikwere} (target) language as they articulate certain consonants. Some of these changes conform to the forms used by speakers of different geographical areas, while others are characteristic of \isi{child language}. Adult speakers of \ili{Ikwere} should be aware of the existing varieties of the language and the forms peculiar to children as this could facilitate communication between children and adults, thereby, preventing problems engendered by communication gap.
 
 
{\sloppy
\printbibliography[heading=subbibliography,notkeyword=this]
}
\end{document}