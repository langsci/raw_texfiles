\documentclass[output=paper,
modfonts
]{langscibook} 
% \bibliography{localbibliography}
\ChapterDOI{10.5281/zenodo.1251736}

%%% \usepackage{langsci-optional}
\usepackage{langsci-gb4e}
\usepackage{langsci-lgr}

\usepackage{listings}
\lstset{basicstyle=\ttfamily,tabsize=2,breaklines=true}

%added by author
% \usepackage{tipa}
\usepackage{multirow}
\graphicspath{{figures/}}
\usepackage{langsci-branding}

 

\title{Factive relative clauses in Pulaar} 

\author{Ibrahima Ba  \affiliation{The University of Kansas} }

\abstract{Drawing from \citet{Kayne:1994}, this chapter shows that Headed Relative Clauses and Factive Clauses in Pulaar are built from similar structures. Both display word order similarities, and in each case the complementizer, which is homophonous with the determiner, agrees with the (null or overt) head NP in Spec,CP. The verb form is also the same in Headed Relatives and Factive Relatives, and it undergoes the same agreement pattern. Furthermore, Headed Relatives and Factives in Pulaar both exhibit island constraints such that extraction out of either construction is impossible; this indicates that they all involve movement of some sort. The difference between these constructions is that the Headed Relative has an overt head noun whereas Factives have null head nouns.}

\begin{document}
\maketitle
% [Warning: Draw object ignored][Warning: Draw object ignored][Warning: Draw object ignored][Warning: Draw object ignored][Warning: Draw object ignored][Warning: Draw object ignored][Warning: Draw object ignored][Warning: Draw object ignored][Warning: Draw object ignored][Warning: Draw object ignored][Warning: Draw object ignored][Warning: Draw object ignored][Warning: Draw object ignored][Warning: Draw object ignored][Warning: Draw object ignored][Warning: Draw object ignored][Warning: Draw object ignored][Warning: Draw object ignored][Warning: Draw object ignored][Warning: Draw object ignored][Warning: Draw object ignored][Warning: Draw object ignored][Warning: Draw object ignored][Warning: Draw object ignored][Warning: Draw object ignored][Warning: Draw object ignored][Warning: Draw object ignored][Warning: Draw object ignored][Warning: Draw object ignored][Warning: Draw object ignored][Warning: Draw object ignored][Warning: Draw object ignored][Warning: Draw object ignored][Warning: Draw object ignored]\textbf{}
            

 
\section{Introduction}\label{sec:ba:1}
This paper investigates \isi{factive} relative clauses in \ili{Pulaar}, a West Atlantic language spoken in \isi{Senegal} and other West African countries. The \ili{Pulaar} variety described here is spoken in the southern part of \isi{Senegal}. Specifically, the paper provides an analysis of two \isi{factive} constructions in \ili{Pulaar}, namely the verbal \isi{factive} and the \textit{ko}{}-\isi{factive}, as (\ref{ex:ba:1}a) and (\ref{ex:ba:1}b) respectively: in (\ref{ex:ba:1}a), the \isi{infinitive} form of the \isi{verb} is fronted and followed by the \isi{complementizer}; in (\ref{ex:ba:1}b), the particle \textit{ko}\footnote{\textit{Ko} has a variety of meanings in \ili{Pulaar}, most of which are not related semantically. I treat these various instances of \textit{ko} as homophones, which have meanings/functions such as \isi{focus}/topic (see \citealt{Cover2006}), copula, \isi{noun class}, \isi{complementizer}, \isi{pronoun}.} (glossed as a relative \isi{complementizer}) always appears at the leftmost edge of the \isi{clause}.\footnote{The two meanings of (\ref{ex:ba:1}a) are discussed in \sectref{sec:ba:3.2}.}

\settowidth\jamwidth{Verbal Factive}
\ea\label{ex:ba:1}
\ea
\gll [\textbf{def-go}      ngo      \textbf{ndef}-mi      ñebbe     ngo]        bettu     Hawaa.\\
    cook-\textsc{inf}   \textsc{c}.\textsc{\textsubscript{rel}}     cook-\textsc{1sg}  beans     \textsc{cl.}the     surprise Hawaa\\\jambox{Verbal Factive\footnotemark{}}
\glt   ‘The fact that I cooked beans surprised Hawaa.’\\
  ‘The cooking that I cooked the beans surprised Hawaa.’

\settowidth\jamwidth{\textit{ko}{}-Factive}
\ex  
\gll [\textbf{ko}         \textbf{ndef}{}-mi     ñebbe     ko]       bettu     Hawaa.\\
   \textsc{c.}\textsc{\textsubscript{rel}} cook-\textsc{1sg} beans    \textsc{cl.}the   surprise Hawaa                \\\jambox{\textit{ko}{}-Factive}
\glt   ‘The fact that I cooked beans surprised Hawaa.’  
\z
\z




%Notice that (\ref{ex:ba:1}a) has two meanings. I will discuss this further in \sectref{sec:ba:3.2}.
% \todo{Unfortunately, Section 3.3 was not included}

The main claim in this paper is that the constructions in \REF{ex:ba:1} are \isi{relative clause} constructions with a derivation similar to headed relative clauses in \ili{Pulaar}, as in \REF{ex:ba:2}:

\ea\label{ex:ba:2}
 \gll  Musa  ñaam-ma   [ñebbe   ɗe       ndef-mi      ɗe].\\
 Musa   eat-\textsc{perf}    beans   \textsc{c.}\textsc{\textsubscript{rel}}     cook-\textsc{1sg}    \textsc{cl}.the\\
\glt ‘Musa ate the beans that I cooked.’
\z

I argue that headed relatives as well as \isi{factive} relatives can be derived from the same underlying structure in \REF{ex:ba:3} following \citet{Kayne1994}. The structure in \REF{ex:ba:3} is composed of a D and a CP complement.

%\ea\label{ex:ba:3}                
%\begin{forest} nice empty nodes
%[DP [] [D' [D] [CP [] [C' [C] [TP [~~~~~~~~~~~,roof] ] ]]]] 
%\end{forest}
%\z


\ea \label{ex:ba:3}
\begin{forest}
[DP [~~] [D' [D] [CP [~~] [C' [C] [TP [~~~~~~~~,roof]]]]]]          
\end{forest}\z

This is explicitly shown in the structures in \REF{ex:ba:4} where we can see the different \isi{movement} operations that occur in the derivation of the different clauses. Specifically, the entire CP moves to Spec, DP.

\largerpage
\ea\label{ex:ba:4}
\begin{xlista}
\ex Verbal Factive 
\begin{forest} 
[DP [~~,name=empty1] [D' [D\\ngo,align=center,base=top] [CP,name=cp [defgo,name=defgo] [C' [C\\ngo,base=top,align=center] [TP [ndef-mi \~{n}ebbe,roof,name=ndef]] ] ] ] ] ] 
\path [-{Stealth[]}] (ndef) edge [bend left=90] (defgo);
\node [draw, ellipse, fit=(cp) (ndef) (defgo),inner sep=-4pt] (group) {};
\path [-{Stealth[]}] (group.west) edge [bend left, out=45] (empty1.center);
\end{forest}\vspace*{.5\baselineskip}

\clearpage 
\begin{multicols}{2}
\ex \textit{Ko}-Factive\\
\begin{forest}
[DP [~~,name=empty1] [D' [D\\ko,align=center,base=top] [CP,name=cp [∅,name=defgo] [C' [C\\ko,base=top,align=center] [TP [ndef-mi \~{n}ebbe,roof,name=ndef]] ] ] ] ] ] 
\path [-{Stealth[]}] (ndef) edge [bend left=90] (defgo);
\node [draw, ellipse, rotate=-20, fit=(cp) (ndef) (defgo),inner sep=0pt] (group) {};
\path [-{Stealth[]}] (group.west) edge [bend left, out=45] (empty1.center);
\end{forest}

\ex  Headed RC\\
\begin{forest}
[DP [~~,name=empty1] [D' [D\\ɗe,align=center,base=top] [CP,name=cp [\~{n}ebbe,name=nebbe] [C' [C\\ɗe,base=top,align=center] [TP [ndef-mi t,roof,name=ndef] ] ] ] ] ]
\path [-{Stealth[]}] (ndef) edge [bend left=90] (nebbe);
\node [draw, ellipse, fit=(ndef) (cp) (nebbe),inner sep=-4pt] (group) {};
\path [-{Stealth[]}] (group.west) edge [bend left, out=45] (empty1.center);
\end{forest}
\end{multicols}
\end{xlista}
\z

The remainder of this paper is structured as follows: \sectref{sec:ba:2} provides a short background on \ili{Pulaar} which will include the basic \isi{word order}, some properties of the \isi{noun} and the agreement morphology. The distribution of \isi{factive} clauses is laid out in \sectref{sec:ba:3}. \sectref{sec:ba:4} deals with the structural similarities that exist between Headed Relatives and Factives in \ili{Pulaar}. \sectref{sec:ba:5} demonstrates that both headed relatives and factives are islands and \sectref{sec:ba:6} shows the derivation of Headed Relatives and Factive clauses. \sectref{sec:ba:7} presents concluding remarks. 


 
\section{Background on Pulaar}\label{sec:ba:2}
\citet{Lewis2009} states that \ili{Pulaar} belongs to Atlantic branch of the Niger-\isi{Congo} language family. There is a large number of \ili{Pulaar} dialects with varying levels of mutual intelligibility, spoken from \isi{Senegal} to \isi{Cameroon} and \isi{Sudan} and all the countries in-between. There are at least four dialects of \ili{Pulaar} in \isi{Senegal}: Futa \ili{Tooro} region (north-east), Fula(kunda) spoken in the Kolda region (south), Pular (spelled with one ‘a’ ) spoken by people originally from \isi{Guinea} Republic; and the dialect spoken in Kabaadaa (south and east of Kolda), also known as Toore, which this paper is based on.


\subsection{Word order}\label{sec:ba:2.1}


\ili{Pulaar} is used here as a general term to refer to the language. It is a Subject-Verb-Object (SVO), prepositional language, as shown in the sentence below.

 
 \ea\label{ex:ba:5}
\gll  Taalibe      mo       jangu-m                deft-are    nde        les       lekki.\\
     student      \textsc{cl.}the    read-\textsc{perf.neut}     book-\textsc{cl} \textsc{cl.}the    under   tree\\
\glt    ‘The student has read the book under a tree.’              
\z
 
\isi{Focus} in \ili{Pulaar} is generally encoded by the particle \textit{ko} which precedes the focused phrase, as shown in the example below:
 
 \ea\label{ex:ba:6}
 \ea\settowidth\jamwidth{DP focus}
 \gll  (Ko)  raandu   ndu     Musaa  yii-noo.\\
       \textsc{foc}    dog.\textsc{cl}  \textsc{cl}.the  Musaa  see-\textsc{past} \\\jambox{DP focus}
\glt       ‘It’s the dog that Musaa saw.’
\ex\settowidth\jamwidth{Verb focus}
\gll Musaa (ko)  yii-no      raandu ndu.\\
       Musaa  \textsc{foc}  see-\textsc{past} dog.\textsc{cl} the.\textsc{cl}\\\jambox{Verb focus}
\glt       ‘Musaa saw the dog (not heard it bark).’
\z
\z
\
The parentheses indicate that \textit{ko} is optional. In the absence of \textit{ko}, \isi{focus} can still be interpreted from the \isi{verb} ending. Long vowels indicate DP \isi{focus} whereas short vowel indicate Verb \isi{focus}, regardless of the presence or absence of of the \isi{focus} particle \textit{ko}. \textit{Ko} is also used in Wh-questions, as in the following example:

 \ea\label{ex:ba:7}\settowidth\jamwidth{Wh-question}
 \gll Ko Musaa  yii-\textbf{noo?}\\ 
            what  Musaa  see-past\\\jambox{Wh-question}
\glt           ‘What did Musaa see?’
\z
 
\subsection{Nouns in Pulaar}\label{sec:ba:2.2}
 
\ili{Pulaar} is a \isi{noun class} language. It has twenty-two \isi{noun} classes and the \isi{noun class} marker follows the \isi{noun} \citep[34]{Sylla1982}.

\ea\label{ex:ba:8}
\ea
\gll raa-ndu       ndu\\
   dog-\textsc{cl}        \textsc{cl}.the\\
\glt               ‘the dog’ 
 
 \ex
 \gll daa-ɗi           ɗi \\
      dog-\textsc{cl}         \textsc{cl}.the\\
\glt      ‘the dogs’
\z
\z



The \isi{noun} in (\ref{ex:ba:8}a) can be analyzed as the \isi{root} \isi{noun} \textit{raa} “dog” and a suffix \textit{ndu.} Thus, the \isi{noun} always occurs as a combination of the \isi{noun} and the suffix, like \textit{raandu} “a dog”.



The \isi{infinitive} in \ili{Pulaar} is composed of the \isi{verb} \isi{root} and the \isi{infinitive} suffix \textit{go}, as seen in the examples in (\ref{ex:ba:9}a-b). This \isi{infinitive} form occurs in a variety of positions within a sentence. The examples below show the different positions that the \isi{infinitive} can occupy.


\renewcommand{\currentjam}{Complement of V}\settowidth\jamwidth{\currentjam}
 \ea\label{ex:ba:9}
  \ea
  \gll Mbiɗo   yiɗi/foti          \textbf{def-go}      maaro.\\
        1\textsc{sg}        want/should   cook-\textsc{inf}   rice       \\\jambox{\currentjam}
\glt       ‘I want to cook rice.’
\ex \renewcommand{\currentjam}{As a \isi{noun} + adjective}\settowidth\jamwidth{\currentjam}
\gll O     ñoot-ma  tuuba  am  ba       \textbf{ñoot-go}  wesoo.              \\
  3\textsc{sg}  sew-\textsc{perf}  pants my \textsc{cl}.the sew-\textsc{inf}  beautiful\\\jambox{\currentjam}
\glt  ‘He has sewn my pants a beautiful sewing.’
\z
\z


(\ref{ex:ba:9}b) shows that the \isi{infinitive} in \ili{Pulaar} can be modified by an adjective, which suggests that it behaves as a \isi{noun} belonging to the \textit{ngo} class. \tabref{tab:ba:1} shows the \isi{noun} classes in \ili{Pulaar}. 
 
\begin{table}
\begin{tabular}{rlll} 
\lsptoprule
& { {Noun class}}  & { {example}}  & { {gloss}} \\
\midrule
{  1} & { mo}  & { suko mo}  & { the child} \\
{  2} & { nde}  & { hoore nde}  & { the head} \\
{ 3} & { ndi}  & { ngaari ndi}  & { the ox} \\
{ 4} & { ndu}  & { raandu ndu}  & { the dog} \\
{ 5} & { nge}  & { nagge nge}  & { the cow} \\
{ \textbf{6}} & { \textbf{ngo}}  & { jungo ngo}  & { the hand} \\
{ 7} & { ngu}  & { pucuu ngu}  & { the horse} \\
{ 8} & { nga}  & { damnga nga}  & { the door} \\
{ 9} & { ba}  & { mbabba ba}  & { the donkey} \\
{ 10} & { ka}  & { laanaa ka}  & { the plane, boat} \\
{ 11} & { ki}  & { leɓii ki}  & { the knife} \\
{ \textbf{12}} & { \textbf{ko}}  & { huuko ko}  & { the grass} \\
{ 13} & { ɗum}  & { ɓaleejum ɗum}  & { the black thing} \\
{ 14} & { ɗam}  & { ndiyam ɗam}  & { the water} \\
{ 15} & { nge}  & { laacee nge}  & { the little tail} \\
{ 16} & { ka}  & { leyka ka}  & { the small land} \\
{ 17} & { ngi}  & { damngii ngi}  & { the huge door} \\
{ 18} & { nga}  & { neɗɗaa nga}  & { the huge person} \\
{ 19} & { ɓe} & { yimɓe ɓe} & { the people}\\
{ 20} & { ɗe} & { gite  ɗe} & { the eyes}\\
{ 21} & { ɗi} & { babaaji  ɗi} & { the donkeys}\\
{ 22} & { koñ} & { laanoñ  koñ} & { the small boats}\\
\lspbottomrule
\end{tabular}
\caption{Noun Classes in Pulaar.}
\label{tab:ba:1}
\end{table}

Noun classes 1 to 18 are singular and \isi{noun} classes 19 to 22 are plural. The \isi{noun class} 1 is used for humans and borrowed words. It has two plural forms: 19 for humans and 21 for borrowed words. However, while 19 relates specifically to humans, 21 is not only related to borrowed words; it is also the plural of other \isi{noun} classes such as 3, 4, 5, 7 etc. The \isi{noun class} 20 is also the plural of several \isi{noun} classes such 8, 10, 2, etc. The \isi{noun class} 22 is the plural for diminutives 15 and 16. The augmentative classes 17 and 18, however, have the regular plural class 20 even when the “augmented” \isi{noun} denotes a human referent. 


%%please move \begin{table} just above \begin{tabular
\begin{table}
\begin{tabular}{ll}
\lsptoprule
{ {Singular}} & { {Plural}}\\\midrule
{ mo} & { ɓe (humans), ɗi (loanwords)}\\
{ ndi, ndu, nge, ngu, ba, ko, ɗum, ɗam} & { ɗi}\\
{ nde, ngo, ka, ki \& the augmentatives nga, ngi} & { ɗe}\\
{ nge, ka (diminutives)} & { koñ}\\
\lspbottomrule
\end{tabular}
\caption{Singular/Plural Mapping of Noun Classes.}
\label{tab:ba:2}
\end{table}


For the remainder of this paper, I will be spelling nouns as one single unit, for instance \textit{raandu} instead of a split word \textit{raa-ndu.}

 
\subsection{Consonant mutation}\label{sec:ba:2.3}
 
Consonant mutation refers to the change of one consonant into another under certain conditions. According to \citet{Sylla1982} and \citet{McLaughlin2005}, \ili{Pulaar} exhibits \isi{consonant mutation}, for instance the alternation between \textit{y}, \textit{g} and \textit{s}, \textit{c} below: 

\ea\label{ex:ba:10}
\ea
 \textbf{y}itare        ‘eye’
\ex  \textbf{g}ite           ‘eyes’
\z
\z

\ea
\ea
\textbf{s}engo        ‘side’
\ex  \textbf{c}engle      ‘sides’ 
\z
\z


\tabref{tab:ba:3} shows the alternation patterns that can be found in \ili{Pulaar}.

\begin{table}
\begin{tabularx}{.4\textwidth}{XX}
\lsptoprule
\multicolumn{2}{l}{Initial consonant of the verb}\\
Simple & Mutated\\\midrule
Ø\footnotemark{}, g & ŋg\\
f & p\\
h & k\\
b, w & mb\\
s & c\\
j, y & ñj\\
d, r & nd\\
\lspbottomrule
\end{tabularx}
\caption{Mutating Initial Consonants.}
\label{tab:ba:3}
\end{table}

\footnotetext{The symbol ‘Ø’ represents cases when the \isi{verb} starts with a vowel. In such cases, [ŋg] becomes the mutated sound in the right context.} 

Alternations like these occur in a variety of contexts such as \isi{subject agreement} on the \isi{verb}, singular/plural alternation on nouns, but also affixation. In what follows, I show an example of each of these alternations. In matrix clauses for instance, \isi{subject agreement} is shown on the \isi{verb} through the mutation of the initial consonant when the subject is plural.


\ea\label{ex:ba:11}
 \ea \renewcommand{\currentjam}{Singular}\settowidth\jamwidth{\currentjam}
 \gll mi/a/o     \textbf{s}ood-ma    oto.\\
 I/you/he/she   buy-\textsc{perf.neut}    car \\\jambox{\currentjam}
  \glt ‘I/you have bought a car.’
\ex  \renewcommand{\currentjam}{Plural}\settowidth\jamwidth{\currentjam}
\gll En/on/ɓe     \textbf{c}ood-ma      oto.\\
We/you/they    buy-\textsc{perf.neut}   car \\\jambox{\currentjam}
\glt    ‘We have bought a car.’
\z
\z

In (\ref{ex:ba:12}a) the sentence has a singular subject and the \isi{verb} ‘buy’ starts with [s]. In (\ref{ex:ba:12}b), however, where the subject is plural the \isi{verb} ‘buy’ begins with <c> and is pronounced [ʧ].

Consonant mutation may also occur in \isi{nominalization}; that is when a \isi{verb} is turned into a \isi{noun}, as shown in the following examples:


\ea\label{ex:ba:12}
 { Verb to Noun Alternations}\\
\ea \label{ex:ba:13a} \gll \textbf{{s}}{urku-go}    $\to$     \textbf{{c}}{urki} ‘smoke’ \\
smoke-\textsc{inf} \\
\glt  ‘to smoke’

\ex \label{ex:ba:13b}
\gll \textbf{y}im-go            $\to$ \textbf{{j}}{imoo}  ‘a song’\\
sing-\textsc{inf}   \\
\glt ‘to sing’       
\z\z

Note the alternations in examples (\ref{ex:ba:13a}) and (\ref{ex:ba:13b}) in which the initial consonant of the \isi{verb} changes in the corresponding \isi{noun}.

 
\section{Distribution and semantic interpretations of factives}\label{sec:ba:3}


 
\subsection{Distribution of factives}\label{sec:ba:3.1}
 

Both \isi{factive} \isi{clause} types occur as subjects and complements to \isi{factive} predicates, i.e. predicates that presuppose the truth of their subjects or complements. For instance, the sentence in \REF{ex:ba:14}, from \citet{Kiparsky1970}, involves the non-\isi{factive} \isi{verb} ‘claim’. In other words, a claim may be proven either right or wrong, as shown in (\ref{ex:ba:14}b-c):

\ea\label{ex:ba:14}  \renewcommand{\currentjam}{Non-\isi{factive} Predicate}\settowidth\jamwidth{\currentjam}
\ea  John claims that he offended Mary.\jambox{\currentjam}                        
\ex    … and in fact, he did.
\ex    … but in fact, he did not.
\z
\z

The example in \REF{ex:ba:15}, however, involves a \isi{factive} \isi{verb}. That means it refers to an event that has necessarily occurred, as shown in (15b-c):

\largerpage 
\ea\label{ex:ba:15} \renewcommand{\currentjam}{Factive Predicate}\settowidth\jamwidth{\currentjam}
\ea[]{John regrets that he offended Mary.\jambox{\currentjam}}
 \ex[]{… and in fact, he did.}
 \ex[\#]{… but in fact, he did not.}
\z
\z

The examples in (\ref{ex:ba:16}b) and (\ref{ex:ba:16}c) respectively show verbal and \textit{ko} factives as subjects:
  \ea \label{ex:ba:16}
 \ea \renewcommand{\currentjam}{(input to (16b-c)}\settowidth\jamwidth{\currentjam}
 \gll ɓe       nguju-m   deftare.\\
      3.\textsc{pl}    steal-\textsc{perf} book    \\\jambox{\currentjam}
\glt      ‘They stole a book.’


\ex \renewcommand{\currentjam}{Verbal-Factive}\settowidth\jamwidth{\currentjam}
\gll [wuju-go   ngo        ɓe     nguj-i        deftare    ngo]    bettu-mii-m.\\
       steal-\textsc{inf}   \textsc{c}.\textsc{\textsubscript{rel}}      3.\textsc{pl}   steal-\textsc{perf}   book     \textsc{cl.}the   surprise-\textsc{1sg-perf}\\\jambox{\currentjam}
\glt      ‘The fact that they stole the book surprised me.’

\ex \renewcommand{\currentjam}{\textit{ko}{}-Factive}\settowidth\jamwidth{\currentjam}
\gll [ko      ɓe      nguj-i        deftare     ko]       bettu-mii-m.\\
      \textsc{c.}\textsc{\textsubscript{rel}}   3\textsc{.pl}    steal-\textsc{perf}   book      \textsc{cl}.the    surprise-1\textsc{sg}{}-perf\\\jambox{\currentjam}
\glt     ‘(The fact) that they stole the book surprised me.’
\z
\z

In \ili{Pulaar}, \isi{factive} clauses occur as arguments of \isi{factive} verbs like \textit{bettugo} ‘surprise’, \textit{loɓgo} ‘to be angry’, \textit{ricitaago} ‘to regret’. Factive clauses can, thus, be complements to \isi{factive} verbs, as in the following examples where the verbal and the \textit{ko} \isi{factive} are objects of the \isi{verb} \textit{ricitaago} ‘to regret’:
 
\ea
 \ea \renewcommand{\currentjam}{Verbal-Factive}\settowidth\jamwidth{\currentjam}
 \gll ɓe  ndicit-iim   [wuju-go   ngo   ɓe     nguj-i         deftare    ngo].\\
      \textsc{1pl} regret\textsc{{}-perf} steal-\textsc{inf}   \textsc{c}.\textsc{\textsubscript{rel}}  3.\textsc{pl}   steal-\textsc{perf}   book     \textsc{cl.}the \\\jambox{\currentjam}
\glt  ‘They regret the fact that they stole the book.’

  \ex \renewcommand{\currentjam}{\textit{ko}{}-Factive}\settowidth\jamwidth{\currentjam}
  \gll ɓe  ndicit-iim   [ko      ɓe      nguj-i        deftare     ko].\\
       \textsc{1pl} regret\textsc{{}-perf} \textsc{c.}\textsc{\textsubscript{rel}}   3\textsc{.pl}    steal-\textsc{perf}   book      \textsc{cl}.the    \\\jambox{\currentjam}
\glt       ‘They regret (the fact) that they stole the book.’
\z
\z

Also, \isi{factive} clauses do not occur as arguments of non-\isi{factive} verbs like \textit{siɓ-go} ‘to doubt’, as shown in the following examples:

 \ea
 \ea[*]{Verbal-Factive\\
 \gll mbiɗo   siɓ-i       [wuju-go   ngo    ɓe   nguj-i         deftare   ngo].\\
        1\textsc{sg}       doubt-\textsc{perf} steal-\textsc{inf}   \textsc{c}.\textsc{\textsubscript{rel}}  \textsc{3}.\textsc{pl}  steal-\textsc{perf}   book      \textsc{cl.}the\\
\glt       Intended: ‘I doubt the fact that they stole a book.’}


\ex[*]{\textit{ko}{}-Factive\\
\gll mbiɗo   siɓ-i             [ko     ɓe     nguj-i         deftare  ko].\\
        1\textsc{sg}       doubt-\textsc{perf}   \textsc{c.}\textsc{\textsubscript{rel}}  \textsc{3}.\textsc{pl}  steal-\textsc{perf}    book    \textsc{cl}.the      \\
\glt         Intended: ‘I doubt that they stole a book.’}
\z
\z

\subsection{Semantic interpretations of Pulaar factive clauses}\label{sec:ba:3.2}

There are interpretive differences between the verbal \isi{factive} and the \textit{ko}-\isi{factive} in \ili{Pulaar}. In fact, whereas the verbal \isi{factive} is ambiguous between an eventive and a manner readings, the \textit{ko}-\isi{factive} can be interpreted as an event. 

\ea
\ea \label{ex:ba:19a} \renewcommand{\currentjam}{Verbal-Factive}\settowidth\jamwidth{\currentjam}
\gll [\textbf{def-go}      ngo      \textbf{ndef}{}-mi      ñebbe     ngo]        bettu     Hawaa.  \\
      cook-\textsc{inf}   \textsc{c}.\textsc{\textsubscript{rel}}     cook-\textsc{1sg}     beans     \textsc{cl.}the     surprise Hawaa\\\jambox{\currentjam}
\glt  ‘The fact that I cooked beans surprised Hawaa.’\\
     ‘The cooking that I cooked the beans surprised Hawaa.’
\ex \label{ex:ba:19b} \renewcommand{\currentjam}{\textit{ko}{}-Factive}\settowidth\jamwidth{\currentjam}
\gll [\textbf{ko}         \textbf{ndef}{}-mi     ñebbe     ko]       bettu     Hawaa \\
       \textsc{c.}\textsc{\textsubscript{rel}}         cook-\textsc{1sg}    beans    \textsc{cl.}the   surprise Hawaa\\\jambox{\currentjam}    
\glt ‘The fact that I cooked beans surprised Hawaa.’
\z
\z

The example in \REF{ex:ba:19a} can mean that Hawaa did not expect the speaker to cook the beans in the first place; maybe they agreed that the beans were for sale. In addition to this eventive reading, the verbal \isi{factive} has a manner reading under which (\ref{ex:ba:19a}) would mean that Hawaa expected the speaker to cook the beans but the cooking turned out to be either so good or so bad that Hawaa is somehow surprised.

As for the \textit{ko}-\isi{factive}, it only has an eventive reading. In (\ref{ex:ba:19b}) for instance, Hawaa is surprised that the speaker cooked the beans. There may be a few reasons to this; Hawaa may not have expected or wanted the beans to be cooked or she may not have expected or wanted the speaker to cook the beans he/she does not like cooking or is a terrible cook, etc.


\section{Pulaar relative clauses}\label{sec:ba:4}

In this section I show the morphological similarities between \isi{factive} clauses and headed relative clauses. Specifically, I show that \isi{factive} clauses are types of relative clauses. In addition to being head initial, these three constructions have \isi{agreeing complementizer}, final determiner, similar placement for subject DP or \isi{pronoun}. They also have the same agreement properties.
 
\subsection{Clause structure of headed relative clauses}\label{sec:ba:4.1}
 

\ili{Pulaar} has head-initial relative clauses. The \isi{relativizer} (or \isi{complementizer}) agrees with and follows the head \isi{noun}. It is homophonous with the \isi{clausal determiner} at the end of the \isi{clause} which encodes definiteness. When it is omitted, the head \isi{noun} is indefinite. The relative \isi{complementizer} is obligatory. 

\ea \label{ex:ba:20}
\ea \renewcommand{\currentjam}{Headed  Relative Clause}\settowidth\jamwidth{\currentjam}
\gll simis     \textbf{mo}       Hawaa   loot-i            \textbf{mo}\\
             shirt      \textsc{c.}\textsc{\textsubscript{rel}}      Hawaa  wash-\textsc{perf}   \textsc{cl}.the\\\jambox{\currentjam}
\glt                   ‘the shirt that Hawaa washed’

\ex
  \gll simis   *(\textbf{mo)}     Hawaa  loot-i               \\
             shirt       \textsc{c.}\textsc{\textsubscript{rel}}      Hawaa  wash-\textsc{perf}  \\
\glt                   ‘(some) shirt that Hawaa washed’
\z\z

\ea \label{ex:ba:21}
\ea
\gll faɗoo    \textbf{ngo}    Hawaa  watt-ii         \textbf{ngo}    \\
              shoe      \textsc{c.}\textsc{\textsubscript{rel}}   Hawaa  wear-\textsc{perf}  \textsc{cl}.the\\
\glt                    ‘The shoe that Hawaa is wearing’
  \ex
  \gll faɗoo   *(\textbf{ngo)}    Hawaa  watt-ii            \\
             shoe         \textsc{c.\textsubscript{rel}}    Hawaa  wash-\textsc{perf}  \\
\glt                   ‘(some) shoe that Hawaa is wearing’
\z
\z

The examples in \ref{ex:ba:20} have all the same material, the only difference is that (\ref{ex:ba:20}a) ends with a determiner which is missing in (\ref{ex:ba:20}b). However, the \isi{complementizer} in (\ref{ex:ba:20}b) cannot be deleted. The same can be said \REF{ex:ba:21} where the only difference is that (\ref{ex:ba:21}b) is lacking the final determiner; and again the \isi{complementizer} is mandatory.

Subject agreement is shown on the \isi{verb} through \isi{consonant mutation} for plural subjects, as in matrix clauses. This is shown in the examples below:

\ea \label{ex:ba:22}
\ea \renewcommand{\currentjam}{\oldstylenums{3}\textsc{sg} subject}\settowidth\jamwidth{\currentjam} 
\gll ñebbe    ɗe      Hawaa    \textbf{d}ef-i             ɗe   \\
              beans    \textsc{c.}\textsc{\textsubscript{rel}}   Hawaa   cook-\textsc{perf}    \textsc{cl}.the\\\jambox{\currentjam}
\glt        ‘the beans that Hawaa cooked’
 
 \ex \renewcommand{\currentjam}{\oldstylenums{1}\textsc{sg} subject}\settowidth\jamwidth{\currentjam}
 \gll ñebbe    ɗe       \textbf{nd}ef-mi           ɗe  \\
              beans    \textsc{c.}\textsc{\textsubscript{rel}}    cook-\textsc{1sg}         \textsc{cl}.the\\\jambox{\currentjam}
\glt       ‘the beans that I cooked’
\ex \renewcommand{\currentjam}{\oldstylenums{3}\textsc{pl} subject}\settowidth\jamwidth{\currentjam}
\gll ñebbe    ɗe       rewɓe     ɓe       \textbf{nd}ef-i           ɗe            \\
             beans    \textsc{c.}\textsc{\textsubscript{rel}}    women   \textsc{cl}.the cook-\textsc{perf}    \textsc{cl}.the\\\jambox{\currentjam}
\glt       ‘the beans that the women cooked’
\z
\z

The initial consonant of the \isi{verb} changes from [d] in (\ref{ex:ba:22}a) to [nd] in (\ref{ex:ba:22}b,c). DP subjects in relative clauses always precede the \isi{verb}. 
 
\begin{table}
\caption{Pulaar subject pronouns.}
\label{tab:ba:4}
\begin{tabular}{ll}
\lsptoprule
{ {Singular}}  & { {Plural}} \\\midrule
{ mi} & { min, en}\\
{ a} & { on}\\
{ o} & { ɓe}\\
\lspbottomrule
\end{tabular}
\end{table}

The \isi{word order} of the headed object relative clauses in \ili{Pulaar} is as follows: 

\ea
  NP   \hspace{1em}     \textsc{c.}\textsc{\textsubscript{rel}}     \hspace{1em}    S     \hspace{1em}     V     \hspace{1em}    O\textsubscript{trace}   \hspace{1em}    \textsc{det.cl}  
\z

\newpage
\subsection{Clause Structure of factive clauses}\label{sec:ba:4.2}

Verbal factives are so called because a form of the \isi{verb} (the \isi{infinitive} or gerundive) is treated as a \isi{noun} heading the \isi{factive} \isi{clause}. In this \isi{clause}, the nominalized form of the \isi{verb} is followed by an agreeing \isi{relativizer} which is homophonous with the determiner at the end of the \isi{clause}. This can be seen in the examples below:

\ea \renewcommand{\currentjam}{Verbal Factive}\settowidth\jamwidth{\currentjam} 
\gll   loot-go      \textbf{ngo}      Hawaa    loot-i            wutte    \textbf{ngo} \\                           
wash-\textsc{inf}    \textsc{c.}\textsc{\textsubscript{rel}}       Hawaa    wash-\textsc{perf}     shirt    \textsc{cl.}the\\\jambox{\currentjam}
\glt ‘the fact that Hawaa washed a shirt’
\z

\ea
\gll   \textbf{ko}       Hawaa     loot-i             wutte      \textbf{ko}\\
   \textsc{c.}\textsc{\textsubscript{rel}}   Hawaa    wash-\textsc{perf}   shirt      \textsc{cl.}the  \\
   \glt ‘(the fact) that Hawaa washed a shirt’
   \z


When the determiner is omitted, the verbal \isi{noun} is indefinite\footnote{This is still interpreted as a \isi{factive}. Structures like \REF{ex:ba:26} and \REF{ex:ba:27} can be answers to a question like: ‘What is so and so mad about’ where the person answering the question is not making it sound like their interlocutor knew about that specific event.}. The relative \isi{complementizer} is obligatory. This is shown in the following examples:

\ea\label{ex:ba:26}
 \gll  Loot-go     *(\textbf{ngo)}     Hawaa     loot-i           wutte  \\
wash-\textsc{inf}     \textsc{c.}\textsc{\textsubscript{rel}}         Hawaa    wash-\textsc{perf}    shirt  \\ 
\glt ‘A/some washing that Hawaa washed a shirt’
\z

\ea\label{ex:ba:27}
\gll \textbf{*ko}      Hawaa        loot-i            wutte{\rmfnm}\\
  \textsc{c.}\textsc{\textsubscript{rel}}     Hawaa     wash-\textsc{perf}     shirt                 \\
\glt ‘the fact that Hawaa washed a shirt’
\z

\footnotetext{This is just interpreted as a subject \isi{focus} construction and means something along the lines: ‘It’s Hawaa who cooked/washed…’.}  
In verbal \isi{factive} constructions, the \isi{verb} appears to show some form of agreement. Subject agreement is shown on \isi{verb} through \isi{consonant mutation} for plural subjects, as in matrix clauses. However, singular subjects also trigger \isi{consonant mutation} when they follow the \isi{verb}. This is shown in the examples below:

\ea \label{ex:ba:28}
\ea \renewcommand{\currentjam}{\oldstylenums{3}\textsc{sg} subject}\settowidth\jamwidth{\currentjam} 
\gll def-go     ngo     Hawaa   \textbf{d}ef-i          ñebbe     ngo     \\
              cook-\textsc{inf}   \textsc{c.}\textsc{\textsubscript{rel}}    Hawaa  cook-\textsc{perf}   beans     \textsc{cl}.the\\\jambox{\currentjam}
\glt   ‘the fact that Hawaa cooked beans’
\ex \renewcommand{\currentjam}{\oldstylenums{1}\textsc{sg} subject}\settowidth\jamwidth{\currentjam}
\gll def-go      ngo     \textbf{nd}ef-mi          ñebbe   ngo     \\
             cook-\textsc{inf}    \textsc{c.}\textsc{\textsubscript{rel}}   cook-\textsc{1sg}         beans   \textsc{cl}.the\\\jambox{\currentjam}
\glt      ‘the fact that I cooked beans’

\ex \renewcommand{\currentjam}{\oldstylenums{3}\textsc{pl} subject}\settowidth\jamwidth{\currentjam}
\gll  def-go     ngo       ɓe           \textbf{nd}ef-i            ñebbe  ngo      \\
             cook-\textsc{inf}  \textsc{c.}\textsc{\textsubscript{rel}}    \textsc{subj.}pro  cook-\textsc{perf}     beans   \textsc{cl}.the\\\jambox{\currentjam}
\glt       ‘the fact that they cooked beans’
\z
\z

The initial consonant of the main \isi{clause} \isi{verb} changes from [d] in (\ref{ex:ba:28}a) to [nd] in (\ref{ex:ba:28}b,c). DP subjects in relative clauses always precede the \isi{verb}, as in (\ref{ex:ba:28}a). However, all subject pronouns, except 3\textsc{sg/pl}, have to follow the \isi{verb}. In this case, the initial consonant of the \isi{verb} mutates even when the subject \isi{pronoun} is singular, as in (\ref{ex:ba:28}b).

The \isi{word order} in a verbal \isi{factive} appears to be the following:

\ea
   V\textsubscript{N}\textsc{\textsubscript{om}}   \hspace{1em}      \textsc{c.}\textsc{\textsubscript{rel}}    \hspace{1em}    S   \hspace{1em}       V    \hspace{1em}    O   \hspace{1em}      \textsc{det.cl}
\z

I assume that the \isi{infinitive} form of the relative \isi{verb} (V\textsubscript{NOM}) is moved to Spec,CP to fill in for a null \isi{noun} ‘fact’ (which does not exist in \ili{Pulaar}) along the lines of  \citet{Collins1994} and \citet{TambaTorrence2013}. Assuming that only the \isi{verb} \isi{root} has been moved, the presence of the \isi{infinitive} suffix can be justified by the need for agreement; V\textsubscript{NOM}, the \isi{complementizer} and the determiner must all \isi{agree}.
 
\subsection{Clause structure of the ko-factive} \label{sec:ba:4.3} 

With \textit{ko} as a \isi{relativizer}, the \textit{ko}-\isi{factive} is headless, or it is rather headed by a null \isi{noun}. This is due to the fact that \ili{Pulaar} does not have the word ‘fact’. But one piece of evidence is also that this null \isi{noun} is associated with an existing \isi{noun class} \textit{ko}. When the determiner is omitted, the structure cannot be interpreted as a \isi{factive}. The relative \isi{complementizer} is obligatory. This is shown in the following examples:

\ea
 \textbf{*ko}      Hawaa        loot-i            wutte{\rmfnm}\\
  \textsc{c.}\textsc{\textsubscript{rel}}     Hawaa     wash-\textsc{perf}     shirt                               \\
  \glt ‘the fact that Hawaa washed a shirt’
\z
\footnotetext{This is just interpreted as a subject \isi{focus} construction and means something along the lines: ‘It’s Hawaa who cooked/washed…’.}  
  
\ea
 \gll *\textbf{ko}        Hawaa       def-i             ñebbe   \\
\textsc{c.}\textsc{\textsubscript{rel}}    Hawaa     cook-\textsc{perf}   beans            \\
\glt ‘the fact that Jeyla cooked beans’
\z

Similar to verbal \isi{factive} and headed relative constructions, the \isi{verb} show of agreement morphology in \textit{ko}-factives. Subject agreement is shown on \isi{verb} through \isi{consonant mutation} for plural subjects, as in matrix clauses. This is shown in the examples below:

\ea \label{ex:ba:32}
\ea \renewcommand{\currentjam}{\oldstylenums{3}\textsc{sg} subject}\settowidth\jamwidth{\currentjam} 
\gll ko     Hawaa      \textbf{d}ef-i           ñebbe     ko   \\
              \textsc{c.}\textsc{\textsubscript{rel}}   Hawaa     cook-\textsc{perf}   beans    \textsc{cl}.the\\\jambox{\currentjam}
\glt        ‘the fact that Hawaa cooked beans’
\ex \renewcommand{\currentjam}{\oldstylenums{1}\textsc{sg} subject}\settowidth\jamwidth{\currentjam}  
       \gll ko            \textbf{nd}ef-mi    ñebbe   ko      \\
               \textsc{c.}\textsc{\textsubscript{rel}}        cook-\textsc{sg}      beans   \textsc{cl}.the\\\jambox{\currentjam}
\glt         ‘the fact that I cooked beans’
\ex \renewcommand{\currentjam}{\oldstylenums{3}\textsc{pl} subject}\settowidth\jamwidth{\currentjam}
\gll ko        ɓe       \textbf{nd}ef-i           ñebbe    ko      \\
               \textsc{c.}\textsc{\textsubscript{rel}}   \textsc{3}\textsc{\textsuperscript{rd}}\textsc{.pl}   cook-\textsc{perf}     beans   \textsc{cl}.the\\\jambox{\currentjam}
\glt         ‘the fact that they cooked beans’  
        \z
        \z

The initial consonant of the \isi{verb} changes from [d] in (\ref{ex:ba:32}a) to [nd] in (\ref{ex:ba:32}b,c). DP subjects always precede the \isi{verb}. However, all subject pronouns, except 3\textsc{sg/pl}, have to follow the \isi{verb}. In this case, the initial consonant of the \isi{verb} mutates even when the subject \isi{pronoun} is singular, as seen (\ref{ex:ba:32}b).

The \isi{word order} in a \textit{ko}-\isi{factive} appears to be the following:

\ea
   Ø\textsubscript{NP}   \hspace{1em}    \textsc{c.}\textsc{\textsubscript{rel}} \hspace{1em}        S      \hspace{1em}     V   \hspace{1em}      O  \hspace{1em}       \textsc{det.cl} 
\z

Based on the data presented here, the headed \isi{relative clause} and \isi{factive} relative clauses share a similar structural pattern, as shown below:

\ea
\begin{tabular}[t]{*8{l}}
a. & NP                    &    C.\textsubscript{REL}    &   S     &      V   &       O\textsubscript{trace} &    DET.CL    &      Headed relative       \\
b. & V\textsubscript{NP}   &    C.\textsubscript{REL}    &   S     &      V   &       O                      &    DET.CL    &      Verbal \isi{factive}        \\           
c. & Ø\textsubscript{NP}   &    C.\textsubscript{REL}    &   S     &      V   &       O                      &    DET.CL    &      \textit{ko}{}-\isi{factive} \\
\end{tabular}
\z

Factive clauses involve a null \isi{noun} for the \textit{ko}-\isi{factive} and a \isi{verb} with nominal features for the verbal \isi{factive} and both of these nominals \isi{agree} with a specific \isi{complementizer} and the corresponding homophonous determiner or \isi{noun class}. I assume the presence of a null \isi{noun} in the \textit{ko-}\isi{factive} due to the fact that it agrees with a \isi{noun class}, but also there is no \isi{noun} ‘fact’ in \ili{Pulaar}.

The clear parallel that exist between the headed \isi{relative clause} and \isi{factive} relative clauses suggest that these constructions look like NP [CP] Det. I will follow \citet{Kayne1994} and analyze relative clauses as involving a D + CP, as in the structure in \REF{ex:ba:37}: 

\ea \label{ex:ba:37}
\begin{forest}
[DP [~~] [D' [D] [CP [DP/NP\textsubscript{i}] [C' [C\textsubscript{-\textsc{rel}}] [TP [~~~~~~~~,roof]]]]]]          
\end{forest}\z

\newpage 
However, whether these constructions are all derivable from the same structure is dependent upon whether or not they all involve some type of \isi{movement}.

The data below suggest that \isi{relativization} and factivization involve \isi{movement}. In fact, \isi{relativization} or ‘factivization’ out of a \isi{relative clause} is impossible in headed relatives as well as the verbal and \textit{ko}-\isi{factive} clauses. The examples below illustrate this fact:

\settowidth\jamwidth{\textit{Ko}-F}
\ea\label{ex:ba:36}
\ea[]{
\gll ɗa     yiɗ-i         [suko    mo      Isa     tott-i            ñebbe    mo.]\\
2\textsc{sg}   like-\textsc{perf}     child   \textsc{cl}\textsc{\textsubscript{rel}}    Isa     give-\textsc{perf}   beans   \textsc{cl}.the \\
\glt ‘I like the boy that Isa gave beans’}

\ex[*]{
\gll  ɗa    yiɗ-i          [ñebbe    ɗe      [suko    mo    Isa     {tott-i    \uline{~~~~~~~}}      mo].\\
 2\textsc{sg}   like-\textsc{perf}   [beans   \textsc{cl}\textsubscript{rel}    [child   \textsc{cl}\textsubscript{rel}   Isa    give-\textsc{perf]}           \textsc{cl}.the \\\jambox{RC}
 \glt ‘You like the beans that boy that Isa gave’}
  
\ex[*]{
\gll ɗa   yiɗ-I        [tottu-go   ngo  [suko    mo   Isa  tott-i]        ñebbe  mo     ngo.\\
 2\textsc{sg}  like-\textsc{perf}  [give-\textsc{inf}  \textsc{cl}\textsubscript{rel} [child  \textsc{cl}\textsubscript{rel}   Isa  give-\textsc{perf]} beans  \textsc{cl}.the \textsc{cl}.the \\\jambox{VF}}

 \ex[*]{
\gll ɗa   yiɗ-i        [ko    [suko    mo     Isa  tott-i]       mo       ko       ñebbe     ɗe.\\
 2\textsc{sg} like-\textsc{perf}  [\textsc{cl}\textsubscript{rel} [child   \textsc{cl}\textsubscript{rel}  Isa  give-\textsc{perf}]  \textsc{cl}.the  \textsc{cl}.the  beans  \textsc{cl}.the \\\jambox{\textit{Ko}-F}}
 \z
\z

The examples in (\ref{ex:ba:36}b-d) show that it is impossible to relativize (or ‘factivize’) out of a \isi{relative clause}. The examples (\ref{ex:ba:36}b), (\ref{ex:ba:36}c) and (\ref{ex:ba:36}d) show, respectively, a \isi{relative clause}, a verbal \isi{factive} and a \textit{ko}-\isi{factive}. The impossibility of extracting out of a \isi{relative clause} or relativizing out of a \isi{relative clause} indicates that these constructions involve some type of \isi{movement} and are islands.


 
\section{Derivation of relative and factive clauses} \label{sec:ba:5}
 

In this section, I provide a unified analysis of RCs and \isi{factive} clauses. Following \citet{TambaTorrence2013}, \citet{Torrence2005} and \citet{Kayne1994}, I assume that in \ili{Pulaar}, headed relatives and factives can be derived from the same underlying structure which consists of a D and a CP complement. I argue that in this structure CP raises to Spec,DP.

I first analyze relative clauses like \REF{ex:ba:37}:


\ea \renewcommand{\currentjam}{Headed  Relative Clause}\settowidth\jamwidth{\currentjam} 
\gll   wutte\textsubscript{i}  \textbf{mo}       Hawaa   loot-i     t\textsubscript{i}      \textbf{mo} \\                      
     shirt    \textsc{cl.rel}   Hawaa  wash-\textsc{perf} {} \textsc{cl}.the\\\jambox{\currentjam}
\glt    ‘The shirt that Hawaa washed’
\z 

In this construction, the head (object) NP moves to Spec,CP as shown in \REF{ex:ba:38}: 

\ea \label{ex:ba:38}
\footnotesize
\begin{forest}
[DP [~~,name=empty1] [D', s sep=10mm [D\\mo\\\textsc{cl}.the,base=top,align=center] [CP,name=cp [DP/NP\textsubscript{i}\\wutte\\shirt,name=shirt,base=top,align=center] [C' [C\\mo\\C\textsubscript{-\textsc{rel}},base=top,align=center] [TP [Hawaa loot-i t\\Hawaa washed,base=top,align=left,roof,name=hawaa]]]]]] 
\node [draw, circle, fit=(hawaa) (shirt) (cp),inner sep=-4pt] (group) {};
\path [-{Stealth[]}] (hawaa) edge [bend left=45] (shirt);
\path [-{Stealth[]}] (group.west) edge [bend left=90,looseness=2] (empty1);
\end{forest} 
\z

In the second step of the derivation, CP moves to Spec,DP to yield the surface structure, as it appears in \REF{ex:ba:39}.

Turning to verbal factives, along the lines of \citet{TambaTorrence2013} and following \citet{Collins1994} and \citet{Aboh2005}, I argue that in the Verbal Factive in (\ref{ex:ba:39}a), a copy of the \isi{verb}, which is relativized and carries the \isi{infinitival} –\textit{go}, is moved to Spec,CP. The \isi{complementizer} agrees in \isi{noun class} with the \isi{infinitival} \isi{verb} in Spec,CP. As have I have pointed out, the \isi{infinitive} form the \ili{Pulaar} \isi{verb} exhibits nominal properties\footnote{See example (\ref{ex:ba:9}b).}.
 

\ea \label{ex:ba:39}
\ea
\gll loot-go      \textbf{ngo}      Hawaa     loot-i           wutte    \textbf{ngo}\\
Wash-\textsc{inf} \textsc{c.}\textsc{\textsubscript{rel}}      Hawaa    wash-\textsc{perf}     shirt    \textsc{cl.}the                    \\
\glt ‘the fact that Hawaa washed a shirt’
\ex 
\footnotesize
\begin{forest}
[DP [~~,name=empty1] [D', s sep=10mm [D\\ngo\\\textsc{cl}.the,base=top,align=center,name=ngo] [CP,name=cp [DP/NP\textsubscript{i}\\\textbf{loot-go}\footnotemark\\wash-\textsc{inf},name=wash,base=top,align=center] [C' [C\\ngo\\C\textsubscript{-\textsc{rel}},base=top,align=center] [TP [Hawaa loot-i wutte\\Hawaa washed shirt,base=top,align=left,roof,name=hawaa]]]]]] 
\node [draw, circle, fit=(hawaa) (wash) (cp),inner sep=-4pt] (group) {};
\path [-{Stealth[]}] (hawaa) edge [bend left=45] (wash);
\path [-{Stealth[]}] (ngo.west) edge [bend left,out=90] (empty1.west);
\end{forest} 
\z
\z
\footnotetext{A reviewer notes that the fact the \isi{verb} copy is \isi{infinitival} indicates that there is more structure involved. I leave for future research the precise nature of the nominal constituent in Spec,CP and how a \isi{verb} becomes nominalized.}

Once the \isi{infinitival} \isi{verb} has moved to Spec,CP, the whole CP node is then moved to Spec,DP generating the expected surface structure.

This analysis correctly derives the \isi{word order} of the Verbal Factive construction in (\ref{ex:ba:39}a) in a way similar to the derivation of the headed relative.

I now move to the \textit{ko}-\isi{factive} structure. The \textit{ko}-Factive Relative is slightly different from the other relative types because it involves a null NP meaning ‘fact’. But the presence of this null NP is signaled by its agreement with some \isi{noun class}, in this case \textit{ko}.

In order to derive a \textit{ko}-Factive like the one in (\ref{ex:ba:40}a), we can posit the \isi{movement} of the null NP from inside the TP to Spec,CP. As a second step, the \isi{movement} of CP to Spec,DP yields the surface \isi{word order} along the lines of Headed Relatives and Verbal Factives, as we can see in (\ref{ex:ba:40}b):

\ea \label{ex:ba:40}
\ea
 \gll ko      Jeyla    loot-i           wutte   ko\\
 \textsc{c.}\textsc{\textsubscript{rel}}   Jeyla    wash-\textsc{perf}   shirt    \textsc{cl.}the  \\
 \glt ‘(the fact) that Jeyla washed a shirt’
\ex
\footnotesize
\begin{forest}
[DP [~~,name=empty1] [D', s sep=10mm [D\\ko\\\textsc{cl}.the,base=top,align=center,name=ngo] [CP,name=cp [DP/NP\textsubscript{i}\\∅\textsubscript{\textsc{np}},name=wash,base=top,align=center] [C' [C\\ko\\C\textsubscript{-\textsc{rel}},base=top,align=center] [TP [Hawaa loot-i wutte\\Hawaa washed shirt,base=top,align=left,roof,name=hawaa]]]]]] 
\node [draw, circle, fit=(hawaa) (wash) (cp),inner sep=-4pt] (group) {};
\path [-{Stealth[]}] (hawaa) edge [bend left=45] (wash);
\path [-{Stealth[]}] (ngo.west) edge [bend left,out=90] (empty1.west);
\end{forest} 
\z
\z
          

As the analysis has shown, Headed Relatives and Factive Relatives in \ili{Pulaar} can all be derived from the same hierarchical structure in a relatively similar manner.

\section{Concluding remarks}\label{sec:ba:6}\label{sec:ba:7}

In this paper, I have argued that Headed Relatives and Factive Relatives have similar structure in sense that they have a similar \isi{word order} and in all of them the \isi{complementizer} agrees with the (null or overt) head NP in Spec,CP and is homophonous with the determiner. 

In my analysis, the differences between the three types has to do with the material in Spec,CP. In headed RCs, it is a lexical \isi{noun}. In the verbal factives, it is a nominalized copy of the \isi{verb}, while in the \textit{ko-}factives it is a null \isi{noun} of the \textit{ko} class.
 


% \section*{Abbreviations}
% \section*{Acknowledgements}

{\sloppy
\printbibliography[heading=subbibliography,notkeyword=this]
}
\end{document}
