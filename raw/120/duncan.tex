\documentclass[output=paper,
modfonts
]{langscibook} 
% \bibliography{localbibliography}
\ChapterDOI{10.5281/zenodo.1251740}
 

\title{Searching high and low for focus in Ibibio} 

\author{Philip T. Duncan\affiliation{University of Kansas}\and Travis Major\affiliation{University of California, Los Angeles}\lastand  Mfon Udoinyang \affiliation{University of Kansas}}

% \chapterDOI{} %will be filled in at production
% \epigram{}

\abstract{
This paper discusses two strategies in Ibibio for focusing verbs: contrastive verb focus and exhaustive verb focus. We demonstrate how these constructions differ crucially in the syntactic configurations and derivations that underlie each. Exhaustive verb focus is marked by the presence of the focus operator \textit{kp\'{ɔ}t} 'only', which is base-generated high in the left periphery and triggers phrasal movement of the TP containing the focused verb via pied-piping. Contrastive verb focus is marked by verb doubling produced by head movement, and it invokes a low focus phrase situated in the middle field, somewhere at the boundary of the inflectional and verbal domains. Both types of verb focus in Ibibio are thus syntactically-driven, but the locus of each is split across the clausal spine, and each Foc head can probe independent of the other. Ibibio thus furnishes further evidence that multiple foci can occur in a single clause, and it also provides independent support for the existence of a low focus phrase.
}

\begin{document}
\maketitle


 
\section{Introduction}
In this paper, we discuss morphosyntactic properties of two types of \isi{focus} involving verbs in \ili{Ibibio}: contrastive \isi{verb} \isi{focus} (\ref{ex:duncan:1b}) and exhaustive \isi{verb} \isi{focus} (\ref{ex:duncan:1c}). 

\settowidth\jamwidth{Contrastive Verb Focus}
\ea\label{ex:duncan:1}
\ea 
\gll ékpê á-mà á-kót \`{ŋ}-wèt \\
 ekpe 3\textsc{sg-pst} 3\textsc{sg}-read \textsc{nmlz}-write\\\jambox{(input to 1b,c)}
\glt ‘Ekpe read a book.’
\ex \label{ex:duncan:1b}
\gll ékpê á-ké á-kòó-kót \`{ŋ}-wèt \\
 ekpe 3\textsc{sg}-\textsc{pst.foc} 3\textsc{sg}-\textsc{con.foc}-read \textsc{nmlz}-write\\\jambox{Contrastive Verb Focus}
\glt ‘Ekpe READ the book (not, say, take it away).’

\ex \label{ex:duncan:1c}
\gll ékpê á-ké á-kót \`{ŋ}-wèt kp\'{ɔ}t \\
 ekpe 3\textsc{sg-pst.foc} \textsc{3sg}-read \textsc{nmlz}-write only\\\jambox{Exhaustive Verb Focus}
\glt ‘Ekpe only read the book.’
\z
\z
We motivate and explore two distinct \isi{focus} positions corresponding to each construction, a high \isi{focus} phrase (HFocP) in the C domain, and a low \isi{focus} phrase (LFocP) in the inflectional domain. We also show that \ili{Ibibio} has both syntactically and semantically distinct loci of \isi{verb} \isi{focus}. Exhaustive \isi{verb} \isi{focus} recruits structure high in the left periphery, and is derived by phrasal \isi{movement} where the TP is pied-piped. On the other hand, contrastively focused verbs are situated much closer to VP and are generated by \isi{head movement}, where V\textsuperscript{0} is attracted to the lower \isi{focus} head. Moreover, we discuss how these distinct structural configurations allow for double \isi{verb} \isi{focus} constructions. The structural superiority of the phrasal projection that houses the exhaustively focused \isi{verb} triggers scope effects such that exhaustive \isi{focus} takes wide scope over contrastive \isi{focus} obligatorily.

\ili{Ibibio} thus provides independent evidence for multiple foci occurring in a single \isi{clause} (\citealt{Krifka1992,Rizzi1997,Kiss1998}) and further support for the existence of a low \isi{focus} position \citep{Belletti2004}. Our proposed analysis is given in \REF{ex:duncan:2}, which shows the derivation for both exhaustive \isi{verb} \isi{focus} and contrastive \isi{verb} \isi{focus}:

\ea\label{ex:duncan:2}
 \begin{forest}% nice empty nodes
 [HFocP [~~,name=empty] [HFoc’ [HFoc,name=HFoc [\textit{kp\'{ɔ}}t]] [CP, s sep=10mm [C] [TP,name=TP [DP\textsubscript{SUBJ}] [T' [T] [LFocP [\sout{DP\textsubscript{SUBJ}}]  [LFoc' [LFoc,name=LFoc] [VP [\sout{DP\textsubscript{SUBJ}}] [V' [V,name=V] [DP\textsubscript{OBJ},name=DPOBJ  ] ] ] ] ] ] ] ] ] ]
 \node [dashed,draw,circle,fit={(DPOBJ) (TP)}] (group) {};
 \path[-{Triangle[]}, dashed] (V) edge [bend left=45] (LFoc) ; 
 \path[-{Triangle[]}, dashed] (group) edge [bend left=45] (empty) ; 
 \draw (V.165) -- ++(-2cm,0) |- (LFoc) node[pos=.25,align=left,left] {\small\scshape probe-\\\small\scshape goal\textsubscript{CONFOC}};
 \draw (V.195) -- ++(-6cm,0) |- (HFoc) node[pos=.25,align=left,left] {\small\scshape probe-\\\small\scshape goal\textsubscript{EXFOC}};
 \end{forest}
\z
\largerpage
This derivation illustrates our analyses for contrastive and exhaustive \isi{verb} \isi{focus}, and it also demonstrates how both of these structurally distinct foci can be activated to generate double \isi{focus}. In contrastive \isi{verb} \isi{focus}, LFoc\textsuperscript{0} probes V\textsuperscript{0} and triggers head raising. In exhaustive \isi{focus}, HFoc\textsuperscript{0} forms a probe-goal relation with V\textsuperscript{0}; instead of generating \isi{head movement}, though, the TP is pied-piped to Spec,HFocP. When both foci are activated, ordering is critical: contrastive \isi{verb} \isi{focus} must be embedded under exhaustive \isi{focus} for the derivation to be sustained.

This paper is organized as follows. In \sectref{sec:duncan:2}, we provide a basic background of \ili{Ibibio}, focusing on \isi{word order} and agreement, and motivating the existence of \isi{verb} raising in the language. Following this, in \sectref{sec:duncan:3} we turn to argument \isi{focus} and \textit{wh-}questions to provide a backdrop for understanding \isi{verb} \isi{focus} constructions. \sectref{sec:duncan:4}–\ref{sec:duncan:6} provide our analyses of contrastive \isi{verb} \isi{focus}, exhaustive \isi{verb} \isi{focus}, and double \isi{verb} \isi{focus}, respectively. \sectref{sec:duncan:7} concludes.

\section{Background}\label{sec:duncan:2}
\subsection{Word order \& agreement}

\ili{Ibibio} is a Lower Cross Niger-\isi{Congo} language spoken in \ili{Akwa} Ibom state in southeastern \isi{Nigeria}. It is an SVO language with both \isi{subject agreement} and object agreement (Essien \citeyear*{Essien1990grammar}; \citealt{BakerWillie2010}):

\ea \label{ex:duncan:3}
\ea 
\gll èkà \textbf{á-}mà \textbf{á-}fát áy\'ɨn\\
 mother 3\textsc{sg-pst} 3\textsc{sg}-hug child\\
\glt ‘The mother hugged the child.’
\ex
\gll èkà \textbf{á-}mà \textbf{á-\'ɱ-}fát (míèn)\\
 mother 3\textsc{sg-pst} \textsc{3sg-1sg}-hug 1\textsc{sg}\\
\glt ‘The mother hugged me.’
\z
\z


\noindent As seen in \REF{ex:duncan:3}, \isi{subject agreement} surfaces on both T\textsuperscript{0} and V\textsuperscript{0}, leading to multiple \isi{subject agreement}. Object agreement occurs on V\textsuperscript{0} only, and is not always visible in the surface form.\footnote{Oftentimes because of phonological reasons (e.g. vowel hiatus resolution) object agreement is difficult to discern. All person markers in \ili{Ibibio} are vowels except \oldstylenums{\textsc{1sg}}, which is a nasal that assimilates to the onset of the \isi{verb} \isi{root}. Thus, object marking always survives in cases involving 1sg objects because the nasal does not delete.}

\ili{Ibibio} matrix clauses project not only TP, but also a series of functional layers such as AspP and MoodP. Subject agreement has “no firm upper limit” and is present on “every verbal functional head” (\citealt[110]{BakerWillie2010}):

\ea \label{ex:duncan:4}
\gll ú-kpá ú-ké ú-sé ú-màná-ké ú-nám\\
2\textsc{sgS-cond} \textsc{2sgS-perf} \textsc{2sgS-impf} \textsc{2sgS}-do.again-\textsc{neg} 2\textsc{sgS}-do\\
\glt ‘You should not have been doing it again.’ (\citealt[118]{BakerWillie2010})
\z


\noindent These facts yield the following \isi{word order} in a standard declarative \isi{clause}:

\ea \label{ex:duncan:5}
{DP\textsc{\textsubscript{subj}} Agr\textsc{\textsubscript{subj}}-T […] Agr\textsc{\textsubscript{subj}}-/Agr\textsc{\textsubscript{obj}}-V DP\textsc{\textsubscript{obj}}}
\z

\subsection{Verb raising}

Verb \isi{movement} occurs in several contexts in \ili{Ibibio}. One of these is \isi{negation}, illustrated below:

\settowidth\jamwidth{Affirmative}
\ea\label{ex:duncan:6}
\ea 
\gll òkón á-mà á-\textbf{tóŋŋó}\\
 okon 3\textsc{sg-pst} \textsc{3sg}-start\\\jambox{Affirmative}
\glt ‘Okon had started.’

\ex\label{ex:duncan:6b}
\gll òkón í-ké í-\textbf{tóŋŋó-ké} \\
 okon 3\textsc{sg-pst.foc} \textsc{3sg}-start-\textsc{neg}\\\jambox{Negative}
\glt ‘Okon had not started.’

\ex
\gll ímá á-mà á-\textbf{dép}\\
 ima 3\textsc{sg-pst} \textsc{3sg}-buy\\\jambox{Affirmative}
\glt ‘Ima bought it.’

\ex\label{ex:duncan:6d}
\gll ímá í-ké í-\textbf{dép-pé}\\
 ima 3\textsc{sg-pst.foc} \textsc{3sg}-buy-\textsc{neg}\\\jambox{Negative}
\glt ‘Ima didn’t buy it.’
\z
\z


Note in these examples that \isi{negation} surfaces as a CV suffix, which in these cases is either \textit{-ké} (\ref{ex:duncan:6b}) or an assimilated suffix (\ref{ex:duncan:6d}).\footnote{There is also a third allomorph, -ɣV, which surfaces on monosyllabic \isi{verb} roots. See \citet{AkinlabiUrua2002}, who also treat the various allomorphs of the negative suffix to be “underlyingly /ké/” (\citealt[127]{AkinlabiUrua2002}).} These forms provide evidence for the order of Tense and Negation, as well as morphosyntactic consequences of V raising (\citealt{BakerWillie2010}).\footnote{In agreement with \citet{BakerWillie2010}, we believe that V raising is supported by the fact that \isi{negation} surfaces preverbally as a separate word, ké, in small \isi{clause} constructions (e.g. \isi{causatives}) and subjunctives, which may lack the TP layer. We remain agnostic at present with respect to the possibility of V raising through Neg to T, though we feel this is a viable option (see \citealt{Baker2008}).}\textsuperscript{,}\footnote{We discuss below our account for the change in tense markers.}\textsuperscript{ }The abbreviated tree in \REF{ex:duncan:7} shows the formation of the \isi{complex head} in (\ref{ex:duncan:6d}):

\ea\label{ex:duncan:7}
 \begin{forest}
[T' 
	[T\textsuperscript{0}
		 [\textit{í-ké}]
	]
	[NegP 
		[Neg\textsuperscript{0} 
			[\textit{í-dép\textsubscript{j}},name=idep] 
			[\textit{-pé}]
		] 
		[VP 
			[V\textsuperscript{0} 
				[\textit{t}\textsubscript{j},name=tj] 
			] 
			[~~] 
		] 
	] 
] 
\draw[-{Triangle[]}] (tj) -| (idep);
\end{forest}
\z

As in \REF{ex:duncan:7}, NegP dominates VP, and V-to-Neg raising results in \isi{negation} surfacing postverbally.\footnote{Alternatively, one reviewer points out, the negative suffix could result from V raising around Neg followed by Neg encliticizing onto V (see \citealt{Pollock1989}, for \ili{French}; \citealt{HolmbergPlatzack1995}, for Scandinavian). For our purposes, though, either analysis predicts the same output, as we merely wish here to motivate the existence of \isi{verb} raising in \ili{Ibibio} independent of contrastive \isi{verb} \isi{focus}.}

Verb raising also occurs in reciprocal constructions, which are bipartite in \ili{Ibibio}, producing a suffix that resembles \isi{negation}. Reciprocal morphology is circumfixal, as in (\ref{ex:duncan:8c}), consisting of a \textit{du-} prefix and a CV suffix\footnote{As with the negative suffix, the reciprocal suffix form is assimilative and varies according to the syllable structure and phonetic form of the \isi{verb} \isi{root}.}:

\settowidth\jamwidth{Bipartite reciprocal}
\ea\label{ex:duncan:8}
\ea
\gll é-mà é-k\'ɨt \\
 3\textsc{pl-pst} 3\textsc{pl}-see\\\jambox{Affirmative}
 \glt ‘They saw.’

\ex
\gll í-ké í-\textbf{k\'ɨt-té}\\
 \textsc{3pl-pst.foc} \textsc{3sg}-see-\textsc{neg}\\\jambox{Negation}
\glt ‘They didn’t see.’

\ex\label{ex:duncan:8c}
\gll é-mà é-dû-\textbf{k\`ɨt-tè}\\
 \textsc{3pl-pst} \textsc{3pl-rec}-see-\textsc{rec}\\\jambox{Bipartite reciprocal}
\glt ‘They saw each other.’
\z
\z

Negated reciprocals have stacked suffixes, as seen in \REF{ex:duncan:9}, and \isi{negation} appears farther away from the \isi{verb} than the reciprocal suffix:\footnote{%
  The semantics of negated reciprocals support our ordering where Neg {\textgreater}{\textgreater} Rec. Negation always takes wide scope over the reciprocal, which suggests that the reciprocal \isi{verb} constitutes the input to \isi{negation}. Additionally, \isi{negation} can appear before a reciprocal \isi{verb} in the effect \isi{clause} of a \isi{causative}:
  \begin{exe}
  \ex\label{ex:duncan:fnex1}
  \gll eno á-mà á-nám \'{ɔ}mm\^{ɔ} ké í-dú-k\`ɨt-tè\\
  eno \oldstylenums{3}\textsc{sg}-\textsc{pst} \oldstylenums{3}\textsc{sg}-make they \textsc{neg} \oldstylenums{3}\textsc{pl}-\textsc{rec}-\textsc{see}-\textsc{rec}\\
  \glt ‘Eno made them not see each other.’
  \end{exe}
  Following \citet{BakerWillie2010} we take it that the preverbal negative particle in \REF{ex:duncan:fnex1} is the morphological exponent of Neg\textsuperscript{0} when the \isi{verb} does not raise. Note, though, that the \isi{verb} bears reciprocal morphology, which again suggests that the reciprocal suffix attaches to the \isi{verb} before the negative suffix.
}

\settowidth\jamwidth{Reciprocal + Negation}
\ea\label{ex:duncan:9}
\gll í-ké í-dú-\textbf{k\`ɨt-tè-kè} \\
 \textsc{3pl-pst.foc} \textsc{3pl-rec}-see-\textsc{rec-neg}\\\jambox{Reciprocal + Negation}
\glt ‘They didn’t see each other’
\z

\ili{Ibibio} verbs thus raise for structurally superior heads to surface postverbally (à la Baker's \citeyear{Baker1985} Mirror Principle). In \REF{ex:duncan:9}, the bipartite reciprocal is formed prior to \isi{negation}, and the ordering of the stacked suffixes gives insight into syntactic structure. The schematic in \REF{ex:duncan:10} shows the derivation based on the \isi{hierarchy} we posit to derive the aforementioned properties of negatives and reciprocals:
 
 \ea\label{ex:duncan:10}
 \begin{forest}
 [TP [~~] [T’ [T\textsuperscript{0}] [NegP [Neg\textsuperscript{0},name=Neg] [RecP [Rec\textsuperscript{0},name=Rec] [VP [V\textsuperscript{0},name=V] [~~] ] ] ] ] ]
 \draw[-{Triangle[]}] (V) -| (Rec);
 \draw[-{Triangle[]}] (Rec) -| (Neg);
 \end{forest}
\z

Thus, if RecP intervenes between NegP and VP, V raising ensures that the reciprocal suffix surfaces closest to the \isi{verb}, since the \isi{verb} head raises first to Rec\textsuperscript{0}. This forms a \isi{complex head} that provides the input to Neg\textsuperscript{0}. As we argue below, \isi{verb} raising and the architecture in \REF{ex:duncan:10} are significant for understanding contrastive \isi{verb} \isi{focus}, which also involves \isi{head movement}.

\section{Argument focus \& \textit{wh}-questions}     \label{sec:duncan:3}

When arguments are focused, the past tense marker -\textit{mà} is replaced with -\textit{ké}:\footnote{We here only present data in the past tense, though present and future tenses pattern similarly in this regard.}

\settowidth\jamwidth{Object \textit{wh-}question}
\ea \label{ex:duncan:11}
\ea 
\gll ànìyè (ówó) ké èkà á-\textbf{ké/*mà} á-fát\\
 who person \textsc{comp} mother 3\textsc{sg-pst.foc/*pst} \textsc{3sg}-hug\\\jambox{Object \textit{wh-}question}
\glt ‘Who did the mother hug?’\\

\ex
\gll (á-dò) áy\'ɨn ké èkà á-\textbf{ké/*mà} á-fát\\
 3\textsc{sg}-be child \textsc{comp} mother 3\textsc{sg-pst.foc/*pst} \textsc{3sg}-hug\\\jambox{Object focus}
\glt ‘It was the child that the mother hugged.’
\z
\z

Thus, in past tense, -\textit{mà} is incompatible with argument \isi{focus} (\citealt{Essien1990grammar,Essien1990aspect}; \citealt[244]{WillieUdoinyang2012}). “Focus” -\textit{ké} surfaces obligatorily in argument \isi{focus} contexts (for past tense), as well as \textit{wh-}questions (Note that the 1\textsuperscript{st} \textit{ké} in (\ref{ex:duncan:11}a--b) is the \isi{complementizer}; the inflected \textit{á-ké}---relevant for our discussion---is obligatory). Following \citet{Rizzi1997}, we take it that the landing site of focused constituents and \textit{wh-}expressions is a \isi{focus} phrase located in the C domain. In this paper, we call this projection HFocP to distinguish it from a second \isi{focus} phrase that we argue projects rather low in the clausal spine. HFoc\textsuperscript{0} bears a \isi{focus} feature that draws a phrasal element to its \isi{specifier}, presumably because such \isi{movement} is induced by the need to satisfy a focus-criterion \citep{Rizzi1997}.

In contrast to object \textit{wh-}questions and object \isi{focus}, an overt C is illicit in subject \textit{wh-}questions and subject \isi{focus}. Moreover, past tense \textit{-mà} cannot occur in these constructions, and the fact that \textit{-mà} and “\isi{focus}” \textit{-ké} are in \isi{complementary distribution} suggests that the \textit{-ké} in (\ref{ex:duncan:12}a, c) is “\isi{focus}” \textit{-ké}, not the \isi{complementizer}.

\settowidth\jamwidth{Subject \textit{wh-}question}
\ea\label{ex:duncan:12}
  \ea[]{ 
  \gll ànìyé í-\textbf{ké/*mà} í-fát áy\'ɨn\\
  who 3\textsc{sg-pst.foc/*pst} \textsc{3sg}-hug child\\\jambox{Subject \textit{wh-}question}
  \glt ‘Who hugged the child?’}
  \ex[*]{
  \gll ànìyé \textbf{ké} í-\textbf{ké} í-fát áy\'ɨn \\
  who \textsc{comp} 3\textsc{sg-pst.foc} \textsc{3sg}-hug child\\
  \glt (Intended: ‘Who hugged the child?’ or ‘Who is it that hugged the child?’)}
  \ex[]{
  \gll (á-dò) èkà á-\textbf{ké}/*\textbf{mà} á-fát áy\'ɨn\\
  3\textsc{sg}-be mother 3\textsc{sg-pst.foc/*pst} \textsc{3sg}-hug child\\\jambox{Subject focus}
  \glt ‘It was the mother that hugged the child (not the father).’\\}
  \ex[*]{
  \gll èkà \textbf{ké} á-\textbf{ké} á-fát áy\'ɨn \\
  mother \textsc{comp} 3\textsc{sg-pst.foc} \textsc{3sg}-hug child\\
  \glt (Intended: ‘It was the mother that hugged the child [not the father].’)}
  \z
\z

This subject-object asymmetry in argument \isi{focus} suggests a “that-trace effect” (\citealt{Perlmutter1971,ChomskyLasnik1977}) disallowing subject \isi{extraction} over overt complementizers.

In summary, argument \isi{focus} in \ili{Ibibio} requires a special tense marker (“\isi{focus}” \textit{-ké} in past tense), and the neutral tense marker is illicit in such constructions. Focused arguments and \textit{wh-}items undergo \isi{movement} to HFocP in the \isi{complementizer} domain, and land higher than the C head. As we discuss below, these properties of \isi{focus} constructions are significant for differentiating between the two types of \isi{verb} \isi{focus} under consideration here: exhaustively focused verbs pattern much like argument \isi{focus} constructions and involve phrasal \isi{movement} to the left periphery, whereas contrastively focused verbs do not activate structure in the C system, and instead are derived in the inflectional domain via \isi{head movement}.

\section{Contrastive verb focus}     \label{sec:duncan:4}

\subsection{Morphophonological properties}

When verbs are contrastively focused, \isi{verb} morphology expresses \isi{focus} (\citealt[103--106]{Essien1990grammar}; \citealt{AkinlabiUrua2000,AkinlabiUrua2002}; see \citealt{Cook2002} for \isi{verb} \isi{focus} in the closely related Efik).

\settowidth\jamwidth{(input to \ref{ex:duncan:13b})}
\ea \label{ex:duncan:13}
\ea 
\gll ákùn á-mà á-dép \`{ŋ}-wèt \\
 akun 3\textsc{sg-pst} \textsc{3sg}-buy \textsc{nmlz}-write\\\jambox{(input to \ref{ex:duncan:13b})}
\glt ‘Akun bought the book.’

\ex \label{ex:duncan:13b}
\gll ákùn á-ké á-\textbf{dèé-dép} \`{ŋ}-wèt í-ké í-y\`ɨp-pé-y\`ɨp\\
 akun 3\textsc{sg-pst.foc} \textsc{3sg-con.foc}-buy \textsc{nmlz}-write 3\textsc{sg-pst.foc} 3\textsc{sg.con.foc}-\textsc{neg}-steal\\
\glt ‘Akun BOUGHT the book, she didn’t STEAL it.’
\z
\z

 
Forms of focused verbs demonstrate interactions between phonology, morphology, and syntax. In affirmative forms, the \isi{focus} component “takes the shape of a heavy (bimoraic) syllable” (\citealt[156]{AkinlabiUrua2002}), which appears on the surface to be some type of prefixal “reduplicant.” Vowel lengthening occurs on the “reduplicant,” and the initial CV sequence of the \isi{verb} \isi{root} becomes a “reduplicative prefix” of the form CVV-${\surd}$. This prefix bears a tone pattern (LH or HH) that is sensitive to the tone melody on the \isi{root}. The -ATR vowels /ɨ, ʉ, ʌ/ cannot be lengthened in \ili{Ibibio}, and these change to [e, u, ɔ] in order to be lengthened. Finally, \isi{verb} roots with underlyingly low tones become HL falling tones in contrastive reduplication. These properties can be seen in the examples of affirmative contrastively focused verbs in \tabref{tab:duncan:1}, which are given for each of the vowels and simple tones in \ili{Ibibio}.

\begin{table}
\caption{Contrastive verb focus forms}
\label{tab:duncan:1}
\begin{tabularx}{\textwidth}{p{1.4cm}Qp{1cm}p{3cm}Q}
\lsptoprule
{Vowel (w/~tone)} & {Permissible syllable~type} & {~\newline Verb} & {\ili{English}\newline  gloss} & {Focused stem (affirmative)}\\
\midrule {}
 [í]& CV(C) & dí & ‘come’ & dìídí\\{}
 [ì]& CV(C) & kpì & ‘cut’ & kpìíkpî\\{}
 [\'ɨ]& CVC & t\'ɨm & ‘pound’ & tèét\'ɨm\\{}
 [\`ɨ]& CVC & n\`ɨm & ‘keep’ & nèén\^ɨm\\{}
 [é]& CV(C) & sé & ‘look’ & sèésé\\{}
 [è]& CV(C) & wèt & ‘write’ & wèéwêt\\{}
 [ú]& CV(C) & túúk & ‘touch’ & tùútúúk\\{}
 [ù]& CV(C) & fù & ‘be lazy’ & fùúfû\\{}
 [\'{ʉ}]& CVC & bʉn & ‘keep many things’ & bùúb\'{ʉ}n\\{}
 [\`{ʉ}]& CVC & bʉm & ‘break’ & bùúb\^{ʉ}m\\{}
 [ó]& CV(C) & bót & ‘mold’ & bòóbót\\{}
 [ò]& CV(C) & bòn & ‘begat’ & bòóbôn\\{}
 [\'{ɔ}]& CVC & t\'{ɔ}k & ‘urinate’ & t\`{ɔ}\'{ɔ}t\'{ɔ}k\\{}
 [\`{ɔ}]& CVC & t\`{ɔ}k & ‘verbally abuse’ & t\`{ɔ}\'{ɔ}t\^{ɔ}k\\{}
 [\'{ʌ}]& CVC & fʌk & ‘cover’ & f\`{ɔ}\'{ɔ}f\'{ʌ}k\\{}
 [\`{ʌ}]& CVC & tʌk & ‘grate’ & t\`{ɔ}\'{ɔ}t\^{ʌ}k\\{}
 [á]& CV(C) & má & ‘love’ & màámá\\{}
 [à]& CV(C) & mà & ‘complete’ & màámâ\\ 
\lspbottomrule
\end{tabularx} 
\end{table}

\subsection{Morphosyntactic structure}

Unlike argument \isi{focus}, which recruits structure in the C domain, we claim that the derivation for \isi{verb} \isi{focus} is more local, that is, TP-internal:

\ea \label{ex:duncan:14}
\gll ákùn [\textsubscript{TP} á-ké á-yèé-y\^ɨp \`{ŋ}-wèt ] í-ké í-dép-pé-dép\\
    akun ~ 3\textsc{sg-pst.foc} \textsc{3sg-con.foc}-steal \textsc{nmlz}-write ~ 3\textsc{sg-pst.foc} 3\textsc{sg}-buy-\textsc{neg}-buy \\
\glt ‘Akun STOLE the book, she didn’t BUY it.’
\z

Evidence for our claim comes from the position of contrastively focused verbs with respect to T\textsuperscript{0}. We take it that the presence or absence of “\isi{focus}” -\textit{ké} is a diagnostic of activation (or not) of the left periphery. Unlike argument \isi{focus}, where “\isi{focus}” \textit{-ké} tense marker appears obligatorily, contrastively focused verbs can occur with the standard past tense -\textit{mà} and without “\isi{focus}” -\textit{ké}:

\ea \label{ex:duncan:15}
\gll ímà á-mà á-ɲèé-ɲ\'ɨmmé\\
 ima 3\textsc{sg-pst} \textsc{3sg-con.foc}-\isi{agree}\\
\glt ‘Ima AGREED (she didn’t disagree).’
\z


Thus, contrastive \isi{verb} \isi{focus} does not activate the left edge. Instead, the focused \isi{verb} \textit{ɲééɲ\'ɨmmé} ‘AGREED’ in \REF{ex:duncan:15} surfaces below the T\textsuperscript{0} \textit{-mà}.

To account for this, we posit a low \isi{focus} projection that dominates VP, and propose that verbs undergo \isi{movement} to LFoc\textsuperscript{0} in contrastive \isi{verb} \isi{focus}. This is shown in the abbreviated tree in \REF{ex:duncan:16}, which shows the derivation of \REF{ex:duncan:15}:

\ea\label{ex:duncan:16}
\begin{forest}
[TP[\textit{ímà}\\Ima,align=center,base=top] [T’ [\textit{á-mà}\\\textsc{\oldstylenums{3}sg-pst}] [ [… \hspace{1em} LFocP,roof  [LFoc\textsuperscript{0} [\textit{á-ɲèé-ɲ\'ɨmmé}\textsubscript{i}\\ \textsc{\oldstylenums{3}sg-con.foc}-agree,name=agree]] [VP [\textit{t}\textsubscript{i},name=ti]  [~~ ] ] ] ] ] ]
\draw[-{Triangle[]}] (ti) -- ++(0,-1cm) -| (agree);
\end{forest}
% \todo[inline]{LFoc\textsuperscript{0} had only one child, so I changed the tree accordingly.}
\z

\newpage
We argue that the contrastive \isi{verb} \isi{focus} “morpheme” is the product of the \isi{verb} headmoving to LFoc\textsuperscript{0} (see \REF{ex:duncan:2} above), and that the syntax provides input to phonology, which results in this special \isi{verb} morphology. In the derivation in \REF{ex:duncan:16}, LFoc\textsuperscript{0} probes for V\textsuperscript{0} (\citealt{Chomsky2000,Chomsky2001}) and attracts it to itself. We take it that this probing and attraction is driven by an interpretable \isi{focus} feature on LFoc\textsuperscript{0}, which V\textsuperscript{0} values following head adjunction. Focus “reduplication” is a post-syntactic consequence that results from head raising. Interestingly, in \ili{Ibibio} this low \isi{focus} position is uniquely associated with contrastive semantics for verbs, which is sort of an unexpected restriction.\footnote{\citet{Belletti2004} shows that, in \ili{Italian}, low \isi{focus} involving postverbal subjects is associated with new information.} We stipulate—but leave for future investigation—that \ili{Ibibio} LFoc\textsuperscript{0} has a property such that it probes for features exclusive to verbs, and this disallows attracting phrasal units.

Negated verbs may offer insight into the syntactic structure of \isi{verb} \isi{focus}. As noted above, V raising produces a CV negative suffix, as seen in (\ref{ex:duncan:17b}).

\settowidth\jamwidth{Affirmative}
\ea\label{ex:duncan:17}
  \ea 
  \gll à-mà á-f\'{ɔ}p\\
  2\textsc{sg-pst} 3\textsc{sg}-burn\\\jambox{Affirmative}
  \glt ‘You burned it.’\\
  \ex\label{ex:duncan:17b}
  \gll ú-ké ú-f\'{ɔ}p\textbf{-p\'{ɔ}}\\
  2\textsc{sg-pst.foc} 2\textsc{sg}-burn-\textsc{neg}\\\jambox{Negative}
  \glt ‘You didn’t burn it.’
  \z
\z

Instead of a phonologically reduced copy of the \isi{verb} that appears in affirmative contrastive \isi{verb} \isi{focus} forms, negative focused verbs exhibit two full copies of the \isi{verb} (irrespective of syllable type) with Neg intervening:

\ea \label{ex:duncan:18} \settowidth\jamwidth{Neg + Contrastive Verb Focus}
  \ea 
  \gll ú-ké ú-\textbf{f\'{ɔ}p-p\'{ɔ}-f\'{ɔ}p}\\
  2\textsc{sg}-\textsc{pst.foc} 2\textsc{sg}-burn\textsc{-neg}-burn\\\jambox{Neg + Contrastive Verb Focus}
  \glt ‘You didn’t BURN it.’\\
  \ex
  \gll í-ké í-\textbf{dép-pé-dép} \`{ŋ}-wèt á-ké  á-yèé-y\^ɨp \\
  3\textsc{sg-pst.foc} \textsc{3sg}-buy-\textsc{neg}-buy \textsc{nmlz}-write 3\textsc{sg-pst.foc} \textsc{3sg-con.foc}-steal\\
  \glt ‘She didn’t BUY the book, she STOLE it.’
  \z
\z


Similar to our analysis of negated reciprocals above, we take it that the suffix closest to the \isi{verb} attaches first as a result of \isi{verb} raising. In \REF{ex:duncan:18}, this is the negative suffix, either \textit{-p\'{ɔ}} (\ref{ex:duncan:18}a) or \textit{-pé} (\ref{ex:duncan:18}b). Negation thus precedes contrastive \isi{focus}, and the negated \isi{verb} forms the input to the low \isi{focus} position.

We propose \REF{ex:duncan:19} as the derivation of (\ref{ex:duncan:18}a):

\ea\label{ex:duncan:19}
\begin{forest}
[TP [~~] [T’ [T\textsuperscript{0} 
               [\textit{ú-ké}\\\oldstylenums{2}\textsc{sg-pst.foc},align=center,base=top] 
             ] 
             [LFocP 
                   [\textit{ú-} {[} \textit{f\'{ɔ}p\textsubscript{j}}\textit{-p\'{ɔ}}\textsubscript{k} {]} \textit{-f\'{ɔ}p}\\\oldstylenums{2}\textsc{sg-}burn-\textsc{neg}-burn,base=top,align=center,name=burn] 
	              [NegP
	                  [\textit{t}\textsubscript{j} + \textit{t}\textsubscript{k},name=tjtk] [VP
		                  [\textit{t}\textsubscript{j},name=tj] [~~]
                  ]
	             ]
	        ]
	      ]
]        
\draw[-{Triangle[]}] (tj) -| (tjtk);
\draw[-{Triangle[]}] (tjtk) -| (burn);
\end{forest}
\z

In the derivation of negative contrastive \isi{verb} \isi{focus}, LFoc\textsuperscript{0} probes for V\textsuperscript{0} (as in \REF{ex:duncan:16} above), but it attracts the morphologically complex \isi{verb} that has first raised to Neg\textsuperscript{0}. The negative suffix is a consequence of V-to-Neg (similar to patterning of reciprocals), and the negative suffix + a full \isi{verb} copy are a consequence of V-to-Neg-to-Foc.

Why is affirmative contrastive \isi{focus} a heavy CVV “prefix” while negative contrastive \isi{focus} retains a full copy? We tentatively propose (but leave for future analysis) the possibility that the grammar disprefers adjacent copies in contrastive \isi{focus} constructions and instead prefers to dissimilate and maintain distinction. Support for this comes from other instances of contrastive \isi{focus} in the language. Full reduplication exists elsewhere in \ili{Ibibio}, as in \REF{ex:duncan:20} below, but when items are contrastively focused some strategy for differentiation is employed, as in \REF{ex:duncan:21}:

\ea\label{ex:duncan:20}
  \ea 
  ìt\'{ɔ}k\\
  \glt ‘(a) race’
  \ex
    ìt\'{ɔ}k ìt\'{ɔ}k\\
  \glt ‘hurriedly’
  \z
\z

\begin{multicols}{2}
\ea\label{ex:duncan:21}
  \begin{xlista}
  \ex\label{ex:duncan:21a}
  {{éwá }\textbf{{ámì}}    
  \glt ‘this dog’          
  \ex\label{ex:duncan:21b}
  {éwá }\textbf{{ókò}}       
  \glt ‘that (visible) dog’  
  \ex\label{ex:duncan:21c}
  {éwá }\textbf{\textit{ódò}}   
  \glt ‘that (not visible) dog’
  \vspace*{\baselineskip}
  
  \exi{a'.}
  {éwá }\textbf{{ámì-ŋí\`{m}mí}}}\\
  \glt ‘THIS dog (not that one)’
  \exi{b'.}
  {éwá }{\textit{ókó-ŋó\`{ŋ}kó}}\\
  ‘THAT (visible) dog (not this one)’
  \exi{c'.}
   {éwá }\textbf{{ódò-ŋóǹdó}}\\
  \glt ‘THAT (not visible) dog (not this one)’
  \end{xlista}
\z
\end{multicols}

\newpage 
Thus, (phonologically) maintaining a distinction seems to be specific to contrastively focused items – either verbs or demonstratives – in \ili{Ibibio}; identical adjacent items in non-contrastive constructions are permitted \REF{ex:duncan:20}. The patterning of contrastively focused demonstratives in \REF{ex:duncan:21} could be explained in a way that is analogous to the narrative of contrastive \isi{verb} \isi{focus} we develop here; that is, if the syntax generates adjacent items that are phonologically identical then the phonological system resorts to a post-syntactic strategy to differentiate them.

In our analysis, then, \isi{verb} \isi{focus} morphology is a syntactic consequence of focused verbs undergoing V-to-LFoc \isi{movement}. This enables us to provide a more unified account of both affirmative and negative contrastive \isi{verb} forms, since the same derivation underlies both, despite their superficial dissimilarity. However, more work is needed in this area to determine what additional morphophonological processes generate the affirmative forms (such as those proposed by \citealt{AkinlabiUrua2000,AkinlabiUrua2002}). What we see as critically important is that the presence of intervening material (e.g. the negative suffix) blocks phonological reduction, though full copies of the verbs are present in the syntactic derivation of both affirmative and negative forms.

\section{Exhaustive focus}\label{sec:duncan:5}

A second type of \isi{focus} construction in \ili{Ibibio} corresponds to exhaustive \isi{focus}, which is illustrated below in \REF{ex:duncan:22}. As with argument \isi{focus} – and unlike contrastive \isi{verb} \isi{focus} – “\isi{focus}” \textit{ké} surfaces obligatorily in exhaustive \isi{focus} constructions.

\ea\label{ex:duncan:22}\settowidth\jamwidth{Object exhaustive focus}
  \ea 
  \gll ìmá á-mà á-fèɰé ít\`{ɔ}k \\
  ima 3\textsc{sg-pst} \textsc{3sg}-run race\\\jambox{(input to \ref{ex:duncan:22}b,c)}
  \glt ‘Ima ran the race.’
  \ex
  \gll ìmá \textbf{kp\'{ɔ}t} á-ké á-fèɰé ít\`{ɔ}k \\
  ima only 3\textsc{sg-pst.foc} 3\textsc{sg}-run race\\\jambox{Subject exhaustive focus}
  \glt ‘Only Ima ran the race (not Ekpe or Akun).’/*‘Ima only ran the race (she didn’t go to the party).’
  \ex
  \gll ít\`{ɔ}k \textbf{kp\'{ɔ}t} ké ìmá á-ké á-fèɰé \\
  race only \textsc{comp} ima \textsc{3sg-pst.foc} 3\textsc{sg-}run\\\jambox{Object exhaustive focus}
  \glt ‘It was only the race that Ima ran.’
  \ex 
  \gll èté â-ké-dép-pé àkàrà á-mà á-kót \`{ŋ}-wèt\\
  man 3\textsc{sg-pst.foc}-buy-\textsc{rel} bean.cake 3\textsc{sg-pst} \textsc{3sg}-read \textsc{nmlz}-write\\\jambox{(input to \ref{ex:duncan:22e})}
  \glt ‘The man who bought the bean cake read the book.’
  \ex \label{ex:duncan:22e} 
  \gll  èté â-ké-dép-pé àkàrà \textbf{kp\'{ɔ}t} á-ké á-kót \`{ŋ}wèt\\
  man 3\textsc{sg-pst.foc}-buy-\textsc{rel} bean.cake only 3\textsc{sg-pst.foc} 3\textsc{sg}-read \textsc{nmlz}-write\\
  \glt ‘Only the man who bought the bean cake read the book (not Ima or Akun).’/*‘The man who bought the bean cake only read the book (he didn’t read the magazine/he didn’t sell the book).’\footnote{An anonymous reviewer points out that this structure is ambiguous, and that it has the additional meaning ‘The man who bought only the bean cake read the book.’ We assume here that relative clauses of the form in (\ref{ex:duncan:22}d-e) involve raising-to-C, which accounts for the appearance of the relative suffix. We take it that this additional meaning is still compatible with \isi{movement} to a high \isi{focus} position, since the subject and relativized \isi{verb} also undergo \isi{movement} to the C domain. However, we leave a more precise account of \isi{relative clause} structures for future investigation.}
  \z
\z

The \isi{focus} particle \textit{kp\'{ɔ}t} ‘only’ acts as an exhaustive \isi{focus} operator, and it appears to the right of the focused element. We posit that \textit{kp\'{ɔ}t} heads its own phrasal projection, which is a high \isi{focus} phrase in the \isi{complementizer} domain.\footnote{Note, too, that the \isi{complementizer} ké is required when an object is exhaustively focused, as in (\ref{ex:duncan:22}c), and that this \isi{complementizer} appears after \textit{kp\'{ɔ}t}. An overt C\textsuperscript{0} is illicit when subjects are exhaustively focused, which is reminiscent of the subject-object asymmetry observed in argument \isi{focus} constructions due to the “that-trace effect” (see \sectref{sec:duncan:3}). It may be the case that \textit{kp\'{ɔ}t} constructions do not require the type of Spec-Head configuration that we propose. However, what is most important for our analysis is that exhaustive \isi{focus} in \ili{Ibibio} recruits structure high in the left periphery.}\textsuperscript{,}\footnote{An alternative analysis could treat \textit{kp\'{ɔ}t} as a focus-sensitive adjunct much like ‘only’ in \ili{English} (this point was raised by an audience member of LSA 2015 and an anonymous reviewer). In such an account, \textit{kp\'{ɔ}t} would not head a projection in the C domain; instead, it would adjoin to an XP that bears a \isi{focus} feature. We take it that this is indeed a viable option, but at present it is difficult to distinguish with much certainty. Importantly, data suggest that \textit{kp\'{ɔ}t}-focused constituents (including exhaustive \isi{verb} \isi{focus}) activate a left-peripheral \isi{focus} projection in a way that parallels argument \isi{focus} and wh-questions in the language. That is, exhaustive \isi{focus} requires “\isi{focus}” tense/aspect morphology (just like in cases of \=A-\isi{extraction}), which is not a requirement of \isi{verb} \isi{focus} constructions that recruit the low \isi{focus} projection.} Exhaustively focused XPs that are attracted by HFoc\textsuperscript{0} thus land in Spec, HFocP (\citealt{Rizzi1997,Kayne1998}; \'{E}. \citealt{Kiss1998}), which guarantees that \textit{kp\'{ɔ}t} always follows it’s focused constituent, as the examples in \REF{ex:duncan:22} show. The structure in \REF{ex:duncan:23} shows the derivation of (\ref{ex:duncan:22}b) along these lines.

\largerpage
\ea \label{ex:duncan:23}
\small
 \begin{forest}
[HFocP [DP\textsc{\textsubscript{subj}} [\textit{ìmá}\\Ima,align=center,base=top,roof] ] [HFoc’ [HFoc\textsuperscript{0} [\textit{kp\'{ɔ}t}\\only, base=top,align=center] ] [TP [\sout{ DP\textsc{\textsubscript{subj}}}] [T' [T [\textit{á-ké}\\\oldstylenums{3}\textsc{sg-pst.foc}, base=top,align=center]] [VP [\sout{ DP\textsc{\textsubscript{subj}}}] [V' [V [\textit{á-fèɰé}\\\oldstylenums{3}\textsc{sg-}run, base=top,align=center]] [DP\textsubscript{OBJ} [\textit{ít\`{ɔ}k}\\race,base=top,align=center,roof] ] ] ] ] ] ] ]
\end{forest}
\z

When verbs are exhaustively focused in \ili{Ibibio}, the entire TP is targeted for \isi{movement}. Consequently, the exhaustive \isi{focus} operator \textit{kp\'{ɔ}t} always appears to the right edge of the TP, as seen in \REF{ex:duncan:24}.\footnote{One anonymous reviewer notes that, with a change of tone, the second readings in \REF{ex:duncan:24} are possible. Thus, if the final tone of the \isi{verb} complexes in (\ref{ex:duncan:24a}) and (\ref{ex:duncan:24b}) are high, the grammaticality judgments are reversed. We assume that the different readings have similar underlying structures. The fact that \textit{kp\'{ɔ}t} can scope over the whole TP or the subject, but not the object, further supports our pied-piping analysis. It could be that the object is too deeply embedded inside TP to be focused in this construction.}

\ea\label{ex:duncan:24}
\ea \label{ex:duncan:24a}
\gll  [\textsc{\textsubscript{HFocP}} [\textsubscript{TP} ìmá á-ké á-fèɰè ít\`{ɔ}k ] \textbf{kp\'{ɔ}t} ]\\
 ~ ~ ima 3\textsc{sg-pst.foc} \textsc{3sg}-run race ~ only\\
\glt ‘Ima only ran the race.’/*‘Only Ima ran the race.’

\ex \label{ex:duncan:24b}
\gll [\textsc{\textsubscript{HFocP}} [\textsubscript{TP} ékpê á-ké á-kòt \`{ŋ}-wèt  ] \textbf{kp\'{ɔ}t} ]\\
 ~ ~ ekpe 3\textsc{sg-pst.foc} \textsc{3sg}-read \textsc{nmlz}-write  ~ only\\
\glt ‘Ekpe only read the book.’/*‘Only Ekpe read the book.’ 
\z
\z

Moreover, exhaustive \isi{verb} \isi{focus} constructions bear an affinity to subject \isi{focus} in that an overt \isi{complementizer} is not permitted:

\ea \label{ex:duncan:25}
\gll * ìmá á-ké á-fèɰé ít\`{ɔ}k \textbf{kp\'{ɔ}t} \textbf{ké}\\
 {} ima 3\textsc{sg-pst.foc} \textsc{3sg}-run race only \textsc{comp}\\
 \glt (Intended: ‘Ima only ran the race.’)
\z


Exhaustively focused verbs thus demonstrate a “that-trace effect,” which is the same configuration for subject \isi{focus}.

We propose the analysis for exhaustive TP \isi{focus} shown in \REF{ex:duncan:26}, where HFoc\textsuperscript{0} probes for V\textsuperscript{0} and pied-pipes \citep{Ross1967} TP.

\ea\label{ex:duncan:26}
 \begin{forest}
[HFocP [TP\textsubscript{i} [~~~~~,roof,name=empty] ] [HFoc’ [HFoc\textsuperscript{0} [\textit{kp\'ɔt}] ] [\textit{t}\textsubscript{i},name=ti] ] ]
\draw[-{Triangle[]}] (ti) -- ++(0,-1.5cm) -| (empty.center);
\end{forest}
 \z
 
Our analysis of (\ref{ex:duncan:24a}) is given in \REF{ex:duncan:27}:

\ea\label{ex:duncan:27}
\begin{forest}
[HFocP [TP\textsubscript{i} [\textit{ìmá}\\Ima,base=top,align=center] [T' [ \textit{á-ké}\\\oldstylenums{3}\textsc{sg-pst.foc}] [VP [\textit{á-fèɰé}\\\oldstylenums{3}\textsc{sg}-run,base=top,align=center] [\textit{ít\`{ɔ}k}\\race,base=top,align=center] ] ] ] [HFoc’ [\textit{kp\'{ɔ}t}\\only,base=top,align=center] [\textit{t}\textsubscript{i}]
]
]
\end{forest}
\z

Exhaustive \isi{verb} \isi{focus} constructions thus require a different structural configuration than that of contrastively focused verbs. Exhaustive VP \isi{focus} targets XPs for \isi{movement} to the C layer, rather than being derived by \isi{head movement} more local to VP. Exhaustively focused constituents move to Spec,HFocP and appear to the left of the exhaustive \isi{focus} operator.

\section{Double focus}\label{sec:duncan:6}

Given the tree in \REF{ex:duncan:1}, \ili{Ibibio} should allow “real multiple \isi{focus}” \citep{Krifka1992}. That is, if there are truly two distinct \isi{focus} projections, it should be able to activate both \isi{focus} heads in a single \isi{clause}; \ili{Ibibio} shows that this can indeed happen.

\citet[298]{Rizzi1997} noted for \ili{Italian} that \textit{wh-}questions are incompatible with \isi{focus} constructions, since they compete for the same position:


\ea \label{ex:duncan:28}
{Italian}
\ea[*]{ 
 A chi IL PREMIO NOBEL dovrebbero dare?\\
\glt ‘To whom THE NOBEL PRIZE should they give?’ \citep[298]{Rizzi1997}}
\ex[*]{
 IL PREMIO NOBEL a chi dovrebbeo dare?\\
\glt ‘THE NOBEL PRIZE to whom should they give?’ \citep[298]{Rizzi1997}}

\z
\z

In \ili{Ibibio}, \textit{wh-}questions allow \isi{movement} to the left periphery, while contrastive \isi{verb} \isi{focus} is derived in the inflectional domain. As a result, \ili{Ibibio} contrastive \isi{verb} \isi{focus} is compatible with \textit{wh-}questions, as seen in \REF{ex:duncan:29} below.

\ea\label{ex:duncan:29}
\ea 
\gll  \textbf{ǹsǒ} ké (àfò) à-\textbf{dìá-díá}\\
 what \textsc{comp} 2\textsc{sg} \textsc{2sg-con.foc}-eat\\
\glt ‘What the hell are you EATING?’

\ex
\gll \textbf{ǹsǒ} ké (àfò) mmé-ú-ké-ú-\textbf{dìá-ɰá-díá} \\
 what \textsc{comp} 2\textsc{sg} \textit{mmé}\textsc{-2sg-pst.foc-2sg-con.foc-neg}-eat\\
\glt ‘What the hell didn’t you EAT?’
\z
\z

In \REF{ex:duncan:29}, the left edge HFoc and lower LFoc are both activated, allowing the \textit{wh}-element \textit{ǹsǒ} ‘what?’ to move to the C domain and the \isi{verb} \textit{dìá} ‘eat’ to raise to the lower \isi{focus} position. This type of double \isi{focus} interestingly produces a \textit{wh-}the-hell reading (\citealt{Pesetsky1987,DenDikkenGiannakidou2002}).

\ili{Ibibio} also permits double \isi{focus} (contrastive + exhaustive) with verbs probed for by both Foc heads:

\ea \label{ex:duncan:30}
\gll é-ké é-\textbf{b\`{ɔ}\'{ɔ}-bw\'{ɔ}t} ák\'{ʌ}k \textbf{kp\'{ɔ}t}\\
 3\textsc{pl-pst.foc} \textsc{3pl-con.foc}-borrow money only\\
\glt ‘They only BORROWED money.’ (Response to: ‘Did they steal money?’)
\z


As with \REF{ex:duncan:29}, our analysis can account for the simultaneity of these \isi{focus} types since they correspond to distinct structural configurations.

We propose the (truncated) structure in \REF{ex:duncan:31} for double \isi{verb} \isi{focus} in \ili{Ibibio}:

\ea\label{ex:duncan:31}\settowidth\jamwidth{Double \isi{verb} focus}
\begin{forest}
	[HFocP, s sep=10mm [~~,name=empty] [HFoc’, s sep=10mm [\textit{kp\'{ɔ}t}] [TP,name=TP [T\textsuperscript{0}] [LFocP [LFoc\textsuperscript{0},name=LFoc] [VP [V\textsuperscript{0},name=V] [DP,name=DP] ] ] ] ] ]
	\draw[-{Triangle[]}] (V) -| (LFoc);
	\node[draw,circle,fit={(TP) (DP)},inner sep=-3pt] (group) {};
	\draw[-{Triangle[]}] (group) -| (empty.center);
\end{forest}\jambox{Double \isi{verb} focus}
\z
 
\largerpage
Here, the order in the derivation is critical, and the derivation of contrastive \isi{verb} \isi{focus} precedes exhaustive \isi{verb} \isi{focus}. Accordingly, LFoc\textsuperscript{0} probes for V\textsuperscript{0} and causes it to raise, which ensures the \isi{verb} \isi{focus} morphology unique to contrastively focused verbs. Following this, HFoc\textsuperscript{0} also probes for V\textsuperscript{0} and pied-pipes the whole TP to Spec,HFocP since exhaustive \isi{focus} constructions require phrasal \isi{movement}. This ensures that the contrastively focused \isi{verb} along with the object DP surface to the left of the exhaustive \isi{focus} operator. Our analysis of \REF{ex:duncan:30} is seen below in \REF{ex:duncan:32}:

\ea\label{ex:duncan:32}
 \begin{forest}
[HFocP [
		{[}\textsubscript{TP} \textit{é-ké} {[}\textsubscript{LFoc} {[}\textit{é-b\`{ɔ}\'{ɔ}-bwót àk\'{ʌ}k}{]]}\textsubscript{i}\\\oldstylenums{3}\textsc{pl-pst.foc \oldstylenums{3}pl-con.foc}-borrow money,base=top,align=center,name=borrow
	   ]
	   [HFoc’
		 [\textit{kp\'{ɔ}t}\\only,base=top,align=center] [\textit{t}\textsubscript{i},name=ti]
	   ]
]
\draw[-{Triangle[]}] (ti) -- ++(0,-1cm) -| (borrow);
\end{forest}
\z

Again, in our proposal contrastively focused verbs must be derived prior to pied-piping of the TP for \isi{verb} \isi{focus} morphology to occur on the \isi{verb} in a double \isi{verb} \isi{focus} construction.\footnote{%
Our analysis predicts that exhaustive \isi{focus} always takes wide scope over contrastive \isi{focus}, which is indeed borne out:
    \begin{exe}
     \ex 
     \begin{xlista}
      \ex
      \gll  ìyó, ékpê [\textsc{\textsubscript{HFocP }} [\textsc{\textsubscript{TP}} á-ke á-kót \`{ŋ}-wèt ] kp\'{ɔ}t ]   \\
      no ekpe ~ ~ \oldstylenums{3}\textsc{sg-pst.foc} \oldstylenums{3}\textsc{sg-}read   \textsc{nmlz}-write ~ only\\
      \glt ‘No, Ekpe only [read the book] (not the magazine/he didn’t even do his laundry/*he 
      did not take it away).’
      \ex\label{ex:duncan:iib}
      \gll ìyó, ékpê [\textsc{\textsubscript{HFocP}} [\textsc{\textsubscript{LFocP}} á-ke á-\textbf{kòó-kót} \`{ŋ}-wèt ] \textbf{kp\'{ɔ}t} ]\\
      no ekpe  ~ ~ \oldstylenums{3}\textsc{sg-pst.foc} \textsc{\oldstylenums{3}sg-con.foc}-read \textsc{nmlz}-write ~ only\\
      \glt ‘No, Ekpe only [READ the book] (he did not take it away/*he didn't even do his 
      laundry/*not the magazine).’
      \end{xlista}
    \end{exe}
    Thus, \REF{ex:duncan:iib} only corresponds to the interpretation where reading (not doing something else to) the book is the only thing that Ekpe did; it cannot mean that some object other than the book was read.
}

To summarize, in this section we presented data to show that \ili{Ibibio} does allow double \isi{focus} constructions, and that such constructions involve real multiple \isi{focus}. Given the distinct structural configurations required for exhaustive and contrastive \isi{verb} \isi{focus}, our analysis comports rather nicely with these facts. Further, given that the different \isi{focus} positions correspond to particular semantic interpretations when verbs are focused, our proposal also accounts for the scope effects present in double \isi{verb} \isi{focus}.

\section{Conclusions}\label{sec:duncan:7}

We have argued in this paper that \ili{Ibibio} \isi{verb} \isi{focus} constructions are not unified. We motivated the existence of two types of syntactically-driven \isi{focus} constructions involving verbs in \ili{Ibibio}: \isi{verb} raising to a low \isi{focus} position in the inflectional domain corresponds to contrastive \isi{focus}, and TP pied-piping to the C layer corresponds to exhaustive \isi{focus}. Since these \isi{focus} types are structurally distinct, both Foc probes can target V. Thus, double \isi{verb} \isi{focus} is permitted in \ili{Ibibio} \citep{Krifka1992}, and exhaustively focused verbs always take wide scope \citep{Krifka1992,Kiss1998} over contrastively focused ones. From a typological perspective, \ili{Ibibio} \isi{verb} \isi{focus} constructions are significant in that they provide independent evidence for a low \isi{focus} position associated with a “specialized” semantic interpretation \citep{Belletti2004}. The low \isi{focus} position in \ili{Ibibio} is rather unique, however, in that it seems to exclusively target verbs and not, say, NPs. Further exploration into \ili{Ibibio} contrastive \isi{verb} \isi{focus} could thus yield interesting theoretical insights into the nature of low \isi{focus} and what is possible cross-linguistically.
 
\newpage
\section*{Abbreviations}  
\noindent\begin{tabularx}{.45\textwidth}{>{\scshape}ll}
 1 & 1st person\\
 2 & 2nd person\\
 3 & 3rd person\\
 comp & \isi{complementizer}\\
 con & contrastive\\
 cond & conditional\\
 foc & \isi{focus}\\
 impf & imperfective\\
 indf & indefinite\\
\end{tabularx}
\begin{tabularx}{.45\textwidth}{>{\scshape}ll}
 neg & \isi{negation}\\
 nmlz & nominalizer\\
 obj & object\\
 perf & \isi{perfective}\\
 pl & plural\\
 pst & past\\
 sg & singular\\
 S/subj & subject\\
 \\
 \end{tabularx}
 
 \bigskip
\noindent
 \ili{Ibibio} is tonal, and tones are marked in the following manner: 
 
\bigskip 
\noindent
\begin{tabularx}{\textwidth}{lX}
 \'{V} & high tone\\
 \`{V} & low tone\\
 \^{V} & falling tone (note that tones are marked on either vowels or syllabic nasals in the data).
 \end{tabularx}

\section*{Acknowledgements}

We would like to thank our fellow participants in the 2014 Field Methods in Linguistic Description class and the 2014 Research in Field Linguistics seminar at KU, and audience members at both ACAL 45 and LSA 2015 for helpful comments and discussions. We are especially grateful to Harold Torrence and Jason Kandybowicz for their many insights and critiques. Edopeseabasi Udoinyang also graciously provided examples and judgments that informed the final version of this paper. Finally, we thank two anonymous reviewers for their help in suggesting improvements. Unless otherwise noted, the data used in this paper are from Mfon and Edopeseabasi and reflect their judgments.

{\sloppy
\printbibliography[heading=subbibliography,notkeyword=this]
}
\end{document}
