\documentclass[output=paper,
modfonts
]{langscibook}  

\ChapterDOI{10.5281/zenodo.1251738}
\abstract{In Seereer (Atlantic, Senegal), singular pronominal objects are obligatorily marked by an object suffix on the verb. This paper provides the first comprehensive description of this object suffixation pattern, a topic that has been only cursorily described in the extant literature on Seereer (cf. \citealt{Renaudier2012}). In addition, I provide a preliminary theoretical account of the Seereer object suffix system. I argue that Seereer object suffixes are best analyzed as incorporated pronouns. Evidence for such an analysis comes from the following: (i) an object suffix may never occur with an in situ object DP; (ii) an object suffix may not double an extracted object in relative clauses, wh-questions, or focus constructions; (iii) there is only one object suffix allowed per clause; and (iv) an object suffix may reference either object in a double object construction. I argue that object suffixes raise to Spec-\lilv{}P and are subsequently incorporated in the verb via m-merger (\citealt{Matushansky2006}, \citealt{Kramer2014}, \citealt{Harizanov2014}). This analysis elegantly derives the behaviors listed above. Such an approach also allows us to integrate the Seereer object suffixation data into the broader understanding of cliticization patterns crosslinguistically, thereby enriching our understanding of object marking systems in verbs.}

\title{Object suffixes as incorporated pronouns in Seereer}
\author{Nico Baier} 


\begin{document}
\maketitle 
 
% \todo{the command Ext was not defined. It is marked in red in the text}


 

\section{Introduction}\label{sec:baier:1}

In \ili{Seereer} (Atlantic; \isi{Senegal}), singular object pronouns are marked by a suffix on the \isi{verb}, as shown in (\ref{suffix-ex}). Plural object pronouns are realized as a full pronominal DP (\ref{pl-obj-ex}).\footnote{Plural object pronouns are preceded by the differential object marker \textit{a}. This marker is required with objects that are pronouns or proper names. I will not discuss the differential object marker here.}


%\begin{multicols}{2}
\begin{exe}
\ex \label{suffix-ex} \textbf{Singular Object Suffixes}
\begin{multicols}{2}
\begin{xlista}
\ex \gll Jegaan a naf-a-\ovalbox{\textbf{xam}}.\\
Jegaan 3 hit-\Dv{}-\Fsg{}.\Obj{} \\
\glt `Jegaan hit me.'

\ex \gll Jegaan a naf-a-\ovalbox{\textbf{ang}}.\\
Jegaan 3 hit-\Dv{}-\Ssg{}.\Obj{} \\
\glt `Jegaan hit you.'

\ex \gll Jegaan a naf-a-\ovalbox{\textbf{an}}.\\
Jegaan 3 hit-\Dv{}-\Tsg{}.\Obj{} \\
\glt `Jegaan hit him/her/it.'
\end{xlista}
\end{multicols}
\end{exe}

\begin{exe}
\ex \label{pl-obj-ex} \textbf{Plural Object Pronouns}
\begin{multicols}{2}
\begin{xlista}
\ex \gll Jegaan a naf-a a \textbf{in}.\\
Jegaan 3 hit-\Dv{} \Obj{} \Fpl{} \\
\glt `Jegaan hit us.'

\ex \gll Jegaan a naf-a a \textbf{nuun}.\\
Jegaan 3 hit-\Dv{} \Obj{} \Spl{}  \\
\glt `Jegaan hit you guys.'

\ex \gll Jegaan a naf-a a \textbf{den}.\\
Jegaan 3 hit-\Dv{} \Obj{} \Tpl{}  \\
\glt `Jegaan hit them.'
\end{xlista}
\end{multicols}
\end{exe}
%\end{multicols}

\noindent There are only singular object suffixes; no equivalent plural object suffixes exist in the language. Alongside the suffixes, \ili{Seereer} has a full set of free pronouns for all person/ number combinations. The object suffixes and the free pronouns are shown below in \tabref{table:base-suf}:

\begin{table}
\begin{center}
% {\renewcommand{\arraystretch}{1.25} 
% \renewcommand{\tabcolsep}{0.25cm} 
\begin{tabular}{r  c c c c c c }
\lsptoprule
& \oldstylenums{1}\textsc{sg} & \oldstylenums{2}\textsc{sg} & \oldstylenums{3}\textsc{sg} & \oldstylenums{1}\textsc{pl} & \oldstylenums{2}\textsc{pl} & \oldstylenums{3}\textsc{pl} \\
\midrule
Object Suffix & \textit{-aam} & \textit{-ong} & \textit{-in} & & &  \\
Free Pronoun & \textit{mi} & \textit{wo'} & \textit{ten(o)} & \textit{in} & \textit{nun} & \textit{den(o)}  \\
\lspbottomrule
\end{tabular}
\caption{Object Suffixes vs. Free Pronouns}\label{table:base-suf}
%\label{table:baier:pro}
\end{center}
\end{table}

\noindent In \tabref{table:base-suf}, the object suffixes are given in their underlying forms. In most cases, these underlying forms are obscured by morphonological processes. For reasons of space I will not discuss these processes here.\footnote{Though see \citet{Renaudier2012} for discussion of the morphonology of object suffixes in a different \ili{Seereer} dialect, \ili{Seereer}-Marlodj.}

Although there is a small amount of published work on \ili{Seereer} \citep{Faye:1982,McLaughlin:1994,McLaughlin:2000,Renaudier2012}, there is no comprehensive description of the object suffix system. This paper aims to fill this gap. I show that object suffixes are best analyzed as pronouns that are morphologically incorporated into the \isi{verb}, rather than object agreement. I also sketch a preliminary analysis of the \isi{pronoun incorporation} process. Building on analyses of pronominal clitics by \citet{Harizanov2014} and \citet{Kramer2014}, I propose that object suffixes originate in an argument position as pronouns and undergo \isi{head movement} to \hdzero{\lilv{}}. 

The structure of this paper is as follows. In \sectref{sec:baier:2}, I show that the distribution of object suffixes is identical to the distributio of free object pronouns, and argue that this shows object suffixes to be incorporated pronouns. I then discuss constraints on object \isi{suffixation} in \sectref{sec:baier:3}. Based on these facts, I present my analysis in \sectref{sec:baier:4}. Section \sectref{sec:baier:5} provides conclusions.


\section{Object suffixes are pronouns}\label{sec:baier:2}

In this section, I show that object suffixes have the same distribution as other object pronouns and are therefore best analyzed as incorporated pronouns. Evidence for this comes from the fact that object suffixes cannot double an \textit{in situ} object DP; that they cannot co-occur with an \abar-moved object; and that they must resume a left-dislocated object.

\subsection{Doubling of full NPs}\label{sec:baier:2.1}

\textit{In situ} full DP objects can never co-occur with a coreferential object suffix on the \isi{verb}, as shown by the pair of examples in (\ref{dist:no.double1}a-b):

\begin{exe}
\ex \label{dist:no.double1}
\begin{xlista}
\ex {\gll Mataar a jaw-a [\xp{DP} \textbf{maalo} \textbf{fe}] \\
Mataar 3 cook-\Dv{} {} rice \Det{} \\
\glt `Mataar cooked the rice.'}

\ex[*] {\gll Mataar a jaw-a-\ovalbox{\textbf{an\textsubscript{i}}} [\xp{DP} \textbf{maalo} \textbf{fe}]\textsubscript{i} \\
Mataar 3 cook-\Dv{}-\Tsg{}.\Obj{} {} rice \Det{} \\
\glt Intended: `Mataar cooked the rice.'}
\end{xlista}
\end{exe}

\noindent In (\ref{dist:no.double1}a), there is a single full, post-verbal full DP object, \textit{maalo fe} `the rice'. When a object suffix coreferential with \textit{maalo fe} is added to the \isi{verb} in (\ref{dist:no.double1}b), the sentence becomes ungrammatical. \ili{Seereer} is completely invariant with respect to this constraint. As shown in (\ref{dist:no.double2}), an object suffix can never double any kind of full DP object: 

\begin{exe}
\ex \label{dist:no.double2}
\begin{xlista}

\ex[*] {\gll Jegaan a bug-a-\textbf{an\textsubscript{i}} [\xp{DP} \textbf{ya'} \textbf{um} \textbf{oxe}]\textsubscript{i}.\\
Jegaan 3 love-\Dv{}-\Tsg{}.\Obj{} {} mother 3\Poss{} \Det{} \\
\glt Intended: `Mataar loves his mother.' \hfill {Kinship term}}

\ex[*] {\gll Jegaan a ga'-a-\textbf{xam\textsubscript{i}} [\xp{DP} \textbf{a} \textbf{mi}]\textsubscript{i}.\\
Jegaan 3 see-\Dv{}-\Fsg{}.\Obj{} {}  \Obj{} {\sc 1sg} \\
\glt Intended: `Jegaan saw me.' \hfill {Free pronoun}}

\ex[*] {\gll Jegaan a ga'-a-\textbf{an\textsubscript{i}} [\xp{DP} \textbf{okoor} \textbf{oxe}]\textsubscript{i}.\\
Jegaan 3 see-\Dv{}-\Tsg{}.\Obj{} {} man \Det{} \\
\glt Intended: `Jegaan saw the man.' \hfill {Human animate}}

\ex[*] {\gll Jegaan a ga'-a-\textbf{an\textsubscript{i}} [\xp{DP} \textbf{muus} \textbf{ne}]\textsubscript{i}.\\
Jegaan 3 see-\Dv{}-\Tsg{}.\Obj{} {} cat \Det{} \\
\glt Intended: `Jegaan saw the cat.' \hfill {Non-human animate}}

\ex[*] {\gll Jegaan a jik-a-\textbf{an\textsubscript{i}} [\xp{DP} \textbf{mbin} \textbf{ne}]\textsubscript{i} \\
Jegaan 3 buy-\Dv{}-\Tsg{}.\Obj{} {} house \Det{} \\
\glt Intended: `Jegaan bought the house.' \hfill {Inanimate}}

\end{xlista}
\end{exe}

\noindent So, the basic observation is that full, post-verbal DP objects are in \isi{complementary distribution} with object suffixes. This observation is immediately explained if we assume that object suffixes and full DP objects occupy the same structural position at some point in the derivation. Thus, object suffixes and full DP objects compete for an argument position, as there can only be one argument per structural position. This, in turn, straightforwardly follows if we assume that object suffixes are pronouns that have been incorporated morphologically into the \isi{verb}.


\subsection{Object extraction contexts}\label{sec:baier:2.2}

Object suffixes are also in \isi{complementary distribution} with an \abar-extracted object. This is true for all constructions that involve \abar-\isi{extraction} in \ili{Seereer}: \wh-questions, \isi{focus} clauses, and relative clauses.\footnote{Evidence that these clauses involve \abar-\isi{extraction} of the object comes from the fact that the \isi{verb} takes the final suffix \textit{-u}, which only occurs when \abar-\isi{movement} has occurred in a \isi{clause} within which the \isi{verb} is contained. See \citet{Baier:2014} for extensive discussion.} First, an object suffix cannot co-occur with an extracted object \wh-phrase, as shown in (\ref{dist:wh.ban}a-b):

\begin{exe}
\ex \label{dist:wh.ban}
\begin{xlista}
\ex[*] {\gll {\bf xar\textsubscript{i}} Ami a jik-u-\ovalbox{{\bf n\textsubscript{i}}}? \\
what Ami 3 buy-\Ext{}-3\Sg.\Obj{} \\
\glt Intended: `What did Ami buy?' \hfill {Inanimate \wh-word}}

\ex[*] {\gll {\bf an\textsubscript{i}} Ami a bug-u-\ovalbox{{\bf n\textsubscript{i}}}? \\
who Ami 3 love-\Ext{}-3\Sg.\Obj{} \\
\glt Intended: `Who does Ami love?' \hfill {Animate \wh-word}}
\end{xlista}
\end{exe}

\noindent This constraint is also active in object \isi{focus} clauses, as shown in (\ref{dist:focus.ban}a-b):

\begin{exe}
\ex \label{dist:focus.ban}
\begin{xlista}
\ex[*] {\gll {\bf Jegaan\textsubscript{i}}{\subscript{{\footnotesize {\sc foc}}}} Ami a bug-u-\ovalbox{{\bf n\textsubscript{i}}}.\\
Jegaan Ami 3 love-\Ext{}-3\Sg.\Obj{} \\
\glt Intended: `It's Jegaan that Ami loves.' \hfill {DP focus}}

\ex[*] {\gll (a) {\bf wo'\textsubscript{i}}{\subscript{{\footnotesize {\sc foc}}}} Ami a bug-\ovalbox{{\bf ong\textsubscript{i}}}.\\
\Obj{} 2\Sg{} Ami 3 love-2\Sg.\Obj{}.\Ext{} \\
\glt Intended: `It's you that Ami loves.' \hfill {Pronoun focus}}
\end{xlista}
\end{exe}

\noindent Finally, in object relative clauses, an object suffix may not double the extracted DP, as seen in (\ref{dist:rel.ban}):

\begin{exe}
\ex[*] {\label{dist:rel.ban} \gll [\xp{N} {\bf maalo}]\textsubscript{i} [\xp{CP} {\bf ne} Ami a \~{n}am-uu-\ovalbox{{\bf n\textsubscript{i}}}-a] \\
{} rice {} \Rel{}.\Det{} Ami 3 eat-\Ext{}-3\Sg.\Obj{}-\Rel{} \\
\glt Intended: `the rice that Ami ate' }
\end{exe}

\noindent The data in (\ref{dist:wh.ban})-(\ref{dist:rel.ban}) also follow from the idea that object suffixes are underlyingly pronouns that saturate argument positions. An \abar-extracted argument must be generated in an argument position before it undergoes \abar-\isi{movement}, and this blocks an object suffix from being generated in the same argument position. Note that plural object pronouns, which do not have a suffixal form, are also blocked from co-occuring with an extracted plural DP object:

\begin{exe}
\ex \label{dist:pl.ban}
\begin{xlista}
\ex[*] {\gll {\bf aniin\textsubscript{i}} Ami a bug-u a  \textbf{den}\textsubscript{i}? \\
who.\Pl{} Ami 3 love-\Ext{} \Obj{} 3\Pl{} \\
\glt Intended: `Who all does Ami love?' \hfill {Plural \wh-word}}

\ex[*] {\gll (a) {\bf nuun\textsubscript{i}}{\subscript{{\footnotesize {\sc foc}}}} Ami a ga'-u a \textbf{nuun}\textsubscript{i}.\\
\Obj{} 2\Pl{} Ami 3 see-\Ext{} \Obj{} 2\Pl{} \\
\glt Intended: `It's you all that Ami saw.' \hfill {Plural DP focus}}
\end{xlista}
\end{exe}

\noindent So object suffixes have the exact same distribution as free, plural object pronouns in cases of object \abar-\isi{extraction}. This is further evidence that object suffixes are pronouns that are incorporated into the \isi{verb}.

\subsection{Left dislocation contexts}\label{sec:baier:2.3}

Object suffixes must double a left-dislocated full DP object. As shown in (\ref{dist:sg.disl}), when the dislocated DP is singular, an object suffix is required on the \isi{verb}:

\begin{exe}
\ex \label{dist:sg.disl}
\begin{xlista}

\ex {\gll \textbf{maalo fe}, Mataar a jaw-a-\ovalbox{\textbf{an}}.\\
 {rice \Det{}} Mataar 3 cook-\Dv{}-\Tsg{}.\Obj{} \\
\glt `The rice, Mataar cooked it.' \hfill {Suffix}}

\ex[*] {\gll \textbf{maalo fe}, Mataar a jaw-a-\ovalbox{\textbf{\O{}}}.\\
 {rice \Det{}} Mataar 3 cook-\Dv{} \\
\glt Intended: `The rice, Mataar cooked it.' \hfill {No Suffix}}
\end{xlista}
\end{exe}

\noindent Free singular object pronouns may also be dislocated. Resumption by an object suffix is also required in this case: 

\begin{exe}
\ex \label{dist:pro.disl}
\begin{xlista}

\ex {\gll \bf (a) mi, Mataar a bug-a-*(\textbf{xam}).\\
\Obj{} \Fsg{} Mataar 3 cook-\Dv{}-\Fsg{}.\Obj{} \\
\glt Intended: `Me, Mataar likes.'}

\ex {\gll \bf (a) wo', Mataar a bug-a-*(\textbf{ang}).\\
\Obj{} \Ssg{} Mataar 3 cook-\Dv{}-\Fsg{}.\Obj{} \\
\glt Intended: `You, Mataar likes.'}

\end{xlista}
\end{exe}

\noindent Again, the behavior of object suffixes is the same as that of free plural object pronouns. When a plural object DP is left dislocated, a plural \isi{pronoun} is required  as a \isi{resumptive}, (\ref{dist:pl.disl}a); lack of one results in ungrammaticality (\ref{dist:pl.disl}b):

\begin{exe}
\ex \label{dist:pl.disl}
\begin{xlista}

\ex {\gll \textbf{goor we}, Mataar a ga'-a a \ovalbox{\textbf{den}}.\\
 {men \Det{}} Mataar 3 see-\Dv{}  \Obj{} \Tpl{} \\
\glt `The men, Mataar saw them.' \hfill {Pronoun}}

\ex[*] {\gll \textbf{goor we}, Mataar a ga'-a \ovalbox{\textbf{\O{}}}.\\
 {men \Det{}} Mataar 3 see-\Dv{} \\
\glt Intended: `The men, Mataar saw them.' \hfill {No Pronoun}}

\end{xlista}
\end{exe}

\noindent Left dislocation in \ili{Seereer} does not involve \abar-\isi{movement}. Evidence for this comes from the fact that \isi{left dislocation} does not trigger the presence of the \abar-sensitive final suffix \textit{-u}.\footnote{For further discussion, see \citet{Baier:2014}.} Instead, \isi{left dislocation} involves base generation of a DP in the left periphery and resumption in an argument position in the main part of the \isi{clause}. Since \isi{resumptive} elements are usually pronouns \citep{McCloskey:2006a}, this supports the idea that object suffixes are themselves pronouns. Again, this idea is reinforced by the fact that they pattern identically to free plural pronouns in this construction.


\section{Syntactic constraints on object suffixation}\label{sec:baier:3}

In the previous section, I presented distributional evidence that object suffixes are in fact pronouns that end up as a morphological subunit of the \isi{verb} word. Following this line of thought, I assume that, as pronouns, object suffixes are generated as D heads in object position as the complement to V. This is shown in (\ref{con:struct}), where `OS' stands for object suffix:

\begin{exe}
\ex \label{con:struct} {[\xp{VP} V [\xp{D} OS ]]}
\end{exe}

\noindent Thus, object suffixes are simply generated in argument position like any other object and later become associated morphologically with the \isi{verb}. But why do object suffixes incorporate into the \isi{verb}? In this section, I present evidence that object \isi{suffixation} is constrained by the syntactic structure of the \isi{clause} and therefore object \isi{suffixation} is a fundamentally syntactic process. The specific data are derived from the following contexts:

\begin{exe}
\ex 
\begin{xlista}
\ex The obligatoriness of object suffixes
\ex Multiple object constructions: Ditransitives, applicatives, \isi{causatives}
\ex Object suffixes in \isi{passive} clauses
\end{xlista}
\end{exe}

\subsection{Obligatoriness}\label{sec:baier:3.1}

If there is only one singular object \isi{pronoun}, it must \textit{always} surface as a suffix, never as a free \isi{pronoun}, as shown by (\ref{con:no.free}). 

\begin{exe} 
\ex \label{con:no.free}
\begin{xlista}
\ex {\gll Jegaan a fal-a-\ovalbox{\textbf{ang}}.\\
Jegaan 3 kick-\Dv{}-\Ssg{}.\Obj{} \\
\glt `Jegaan kicked you.' \hfill {Object suffix}}

\ex[*] {\gll Jegaan a fal-a (a) \ovalbox{\textbf{wo'}}.\\
Jegaan 3 kick-\Dv{} \Obj{} {\sc 2sg} \\
\glt Intended: `Jegaan kicked you.' \hfill {Free pronoun}}
\end{xlista}
\end{exe}

\noindent Regardless of the presence of other post-verbal constituents, a singular 
object \isi{pronoun} must be realized as a suffix. Consider (\ref{con:sg.order}), which shows that a free singular object \isi{pronoun} is impossible in such contexts:

\begin{exe} 
\ex \label{con:sg.order}
\begin{xlista}
\ex {\gll Jegaan a fal-a-\ovalbox{\textbf{ang}} faak.\\
Jegaan  3 kick-\Dv{}-\Ssg.\Obj{} yesterday \\
\glt `Jegaan kicked you yesterday.'}

\ex[*] {\gll Jegaan a fal-a faak (\textbf{a}) \textbf{wo'}.\\
Jegaan  3 kick-\Dv{} yesterday \Obj{} {\sc 2sg} \\
\glt Intended: `Jegaan kicked you yesterday.'}

\ex[*] {\gll Jegaan a fal-a (\textbf{a}) \textbf{wo'} faak.\\
Jegaan  3 kick-\Dv{} \Obj{} {\sc 2sg} yesterday \\
\glt Intended: `Jegaan kicked you yesterday.'}
\end{xlista}
\end{exe}

\noindent Note that, otherwise, objects are generally freely ordered with regards to other post-verbal constituents. As shown in (\ref{con:free.order}), plural object pronouns and full DP objects may precede or follow an adverb such as \textit{faak} `yesterday':

\begin{exe} 
\ex \label{con:free.order}
\begin{xlista}
\ex {\gll Jegaan a ga'-a {(a \textbf{nuun})} faak {(a \textbf{nuun})}.\\
Jegaan  3 kick-\Dv{} {\Obj{} {\sc 2pl}} yesterday {\Obj{} {\sc 2pl}} \\
\glt  `Jegaan saw you guys yesterday.' \hfill {Plural pronoun}}

\ex {\gll Jegaan a ga'-a {(\textbf{otew oxe})} faak {(\textbf{otew oxe})}.\\
Jegaan  3 kick-\Dv{} {woman \Det{}} yesterday {woman \Det{}} \\
\glt  `Jegaan saw the woman yesterday.' \hfill {Full DP}}
\end{xlista}
\end{exe}


\noindent These data are important in that they show that object \isi{suffixation} is insensitive to \isi{linear order}. If object were sensitive to \isi{linear order}, we would expect a \isi{clause} like (\ref{con:no.free}b), in which an adverbial intervenes between a singular object \isi{pronoun} and the \isi{verb}, to be grammatical (as the plural counterpart in (\ref{con:free.order}a) is). However, this order is not possible. Since syntactic operations not sensitive to \isi{linear order}, this points to a syntactic account of object \isi{suffixation}.

\subsection{Multiple object constructions}\label{sec:baier:3.2}

\ili{Seereer} has several types of \isi{double object} constructions (DOC). Such constructions occur with lexical \isi{ditransitive} verbs, such as \textit{ci'} `give'; verbs bearing one of the \isi{applicative} suffixes \textit{-an} `benefactive' and \textit{-(i)t} `instrumental/locative'; and \isi{causative} verbs derived with the \isi{causative} suffix \textit{-noor}. Lexical \isi{ditransitive} verbs and \isi{applicative} verbs pattern together with regard to \isi{word order} and object \isi{suffixation}, while \isi{causative} verbs pattern differently than the first two classes with regard to these diagnostics. 
 
Ditransitive verbs and \isi{applicative} verbs in \ili{Seereer} are \textsc{symmetrical} \isi{double object} constructions (following the terminology of \citeplain{Bresnan:1990}). When \isi{ditransitive} and \isi{applicative} verbs have two full DP arguments and both are post-verbal, these arguments are freely ordered. This is shown for ditransitives in (\ref{con:ditr.order}) and for the benefacative \isi{applicative} \textit{-an} in (\ref{con:appl.order}). In the following examples, `$\leftrightarrow$' indicates that the bracketed constituents can be reversed in order:

\begin{exe}
\ex \label{con:ditr.order} 
\begin{xlista}

\ex \gll Jegaan a ci'-a [\xp{DP} okoor oxe]\goal{} $\leftrightarrow$ [\xp{DP} atere le]\theme{}.\\
Jegaan 3 give-\Dv{} {} man \Det{} {} {} book \Det{} {} \\
\glt `Jegaan gave the man the book.'\hfill  \cmark{} {{\sc goal $<$ theme}} /  \cmark{} {{\sc theme $<$ goal}}

\end{xlista}
\end{exe}

\begin{exe}
\ex \label{con:appl.order} 
\begin{xlista}

\ex \gll a jaw-an-a [\xp{DP} okoor oxe]\baierben{} $\leftrightarrow$ [\xp{DP} maalo fe]\theme{}.\\
3 cook-\Ben{}-\Dv{} {} man \Det{} {} {} rice \Det{} {} \\
\glt `He cooked the rice for the man.'\hfill  \cmark{} {{\sc ben $<$ theme}} / \cmark{} {{\sc theme $<$ ben}}

\end{xlista}
\end{exe}

\noindent When one of the objects of a \isi{ditransitive} or \isi{applicative} \isi{verb} is a singular \isi{pronoun}, it \textit{must} be realized as a suffix, as shown for a \isi{ditransitive} \isi{verb} in (\ref{con:ditr.suf}).\footnote{For reasons of space, I will use data only from lexical ditransitives for the remainder of this section. The judgements also apply to all applicatives.} This constraint holds regardless of order, as shown by (\ref{con:ditr.suf}b-c):

\begin{exe}
\ex \label{con:ditr.suf} 
\begin{xlista}

\ex \gll Jegaan a ci'-a-\ovalbox{\textbf{ang}\goal{}} [\xp{DP} atere le]\theme{}.\\
Jegaan 3 give-\Dv{}-\Ssg{}.\Obj{} {} book \Det{} {} \\
\glt `Jegaan gave you the book.'\hfill {Object suffix}

\ex[*] {\gll Jegaan a ci'-a  [\xp{DP} a wo']\goal{} [\xp{DP} atere le]\theme{}.\\
Jegaan 3 give-\Dv{}  {} \Obj{} \Ssg{} {} book \Det{} {} \\
\glt Intended: `Jegaan gave the book to you.' \hfill {Free pronoun}}

\ex[*] {\gll Jegaan a ci'-a  [\xp{DP} atere le]\theme{} [\xp{DP} a wo']\goal{}.\\
Jegaan 3 give-\Dv{}  {} book \Det{} {} \Obj{} \Ssg{} {} \\
\glt Intended: `Jegaan gave the book to you.' \hfill {Free pronoun}}

\end{xlista}
\end{exe}

\noindent When a \isi{ditransitive} or \isi{applicative} \isi{verb} takes two singular object pronouns, either argument may surface as a suffix, as shown in (\ref{ditrans3}{a-b}).\footnote{In cases where one object is a speech act participant and the other is not, my consultant showed a preference for \isi{suffixation} of the SAP object. However, this is not a hard and fast constraint. Examples like (\ref{ditrans3}{a}) are perfectly grammatical.} However, there is a \textit{maximum of one} object suffix per \isi{verb} form; the \isi{verb} cannot take multiple object suffixes, as shown by (\ref{ditrans3}c):

\begin{exe}
\ex \label{ditrans3}
\begin{xlista}

\ex \gll Jegaan a ci'-a-\ovalbox{\textbf{ang}\subscript{{\sc goal}}} [\xp{DP} a ten]\subscript{{\sc theme}}.\\
Jegaan 3 give-\Dv{}-\Ssg{}.\Obj{} {} \Obj{} \Tsg{}  \\
\glt `Jegaan gave you it.'\hfill {Goal suffix}

\ex \gll Jegaan a ci'-a-\ovalbox{\textbf{an}\subscript{{\sc theme}}} [\xp{DP} a wo']\subscript{{\sc goal}}.\\
Jegaan 3 give-\Dv{}-\Tsg{}.\Obj{} {} \Obj{} \Ssg{}  \\
\glt `Jegaan gave it to you.'\hfill {Theme suffix}

\ex[*] {\gll Jegaan a ci'-a-\ovalbox{\textbf{ang}\subscript{{\sc goal}}\textbf{-in}\subscript{{\sc theme}}}.\\
Jegaan 3 give-\Dv{}-\Ssg{}.\Obj{}-\Tsg{}.\Obj{} \\
\glt Intended: `Jegaan gave you it.'\hfill {Two suffixes}}

\end{xlista}
\end{exe}

\noindent So these particular multiple object constructions are symmetrical with regard to object \isi{suffixation}, in that either object may be realized as an object suffix when they are both singular pronouns.

On the other hand, \isi{causatives} of transitive verbs derived with the suffix \textit{-noor} are \textbf{asymmetrical} \isi{double object} constructions. Such verbs take two objects: the subject of the caused event (the \textsc{causee}) and the underlying object of the caused event. With regard to \isi{word order}, a full DP \isi{causee} must \textit{always} precede a full DP object, as shown in (\ref{caus1}):

\begin{exe}
\ex \label{caus1}
\begin{xlista}
\ex {\gll Jegaan a fal-\textbf{noor}-a [\xp{DP} okoor oxe]\causee{} [\xp{DP} naak le]\object{}.\\
Jegaan 3 kick-\Caus{}-\Dv{} {} man \Det{} {} cow \Det{} \\
\glt `Jegaan made the man kick the cow.' \hfill \cmark{}{{\sc \isi{causee} $<$ object}}}

\ex[*] {\gll Jegaan a fal-\textbf{noor}-a [\xp{DP} naak le]\object{} [\xp{DP} okoor oxe]\causee{}.\\
Jegaan 3 kick-\Caus{}-\Dv{} {} cow \Det{} {} man \Det{} \\
\glt Intended: `Jegaan made the man kick the cow.' \hfill {{\sc *object $<$ causee}}}
\end{xlista}

\end{exe}

\noindent This is the opposite of what we saw for \isi{ditransitive} and \isi{applicative} verbs, where either ordering was licit. Also unlike \isi{ditransitive} and \isi{applicative} verbs, there is an asymmetry for \isi{causative} verbs with regards to which argument is able to appear as an object suffix. The \isi{causee} \textit{must} be an object suffix if it is a singular \isi{pronoun}, as shown by (\ref{caus2}): 

\begin{exe}
\ex \label{caus2}
\begin{xlista}
\ex {\gll Jegaan a fal-\textbf{noor}-a-\ovalbox{\textbf{ang}\subscript{{\sc causee}}} [\xp{DP} naak le]\subscript{{\sc object}}.\\
Jegaan 3 kick-\Caus{}-\Dv{}-\Ssg{}.\Obj{} {} cow \Det{} \\
\glt `Jegaan made you kick the cow.' \hfill {Object suffix}}

\ex[*] {\gll Jegaan a fal-\textbf{noor}-a [\xp{DP} a wo']\subscript{{\sc causee}} [\xp{DP} naak le]\subscript{{\sc object}}.\\
Jegaan 3 kick-\Caus{}-\Dv{} {} \Obj{} \Ssg{} {} cow \Det{} \\
\glt `Jegaan made you kick the the cow.' \hfill {Free pronoun}}

\end{xlista}
\end{exe}

\noindent However, the object of the \isi{causative} \isi{verb} \textit{cannot} be realized as an object suffix, even if it is the only singular object \isi{pronoun} in the \isi{clause}, as shown by (\ref{caus3}a):

\begin{exe}
\ex \label{caus3}
\begin{xlista}
\ex {\gll Jegaan a fal-\textbf{noor}-a [\xp{DP} okoor oxe]\subscript{{\sc causee}} [\xp{DP} a wo' ]\subscript{{\sc object}}.\\
Jegaan 3 kick-\Caus{}-\Dv{} {} man \Det{} {} \Obj{} \Ssg{} \\
\glt `Jegaan made the man kick the cow.' \hfill {Free pronoun} }

\ex[*] {\gll Jegaan a fal-\textbf{noor}-a-\ovalbox{\textbf{ang}\subscript{{\sc object}}} [\xp{DP} okoor oxe]\subscript{{\sc causee}}.\\
Jegaan 3 kick-\Caus-\Dv{}-\Ssg.\Obj{} {} man \Det{} \\
\glt `Jegaan made the man kick you.' \hfill {Object suffix}}
\end{xlista}
\end{exe}

\noindent Again, this is exactly the opposite of what we saw with ditransitives and applicatives. Like those verbs, however, it is also impossible for a \isi{causative} \isi{verb} to take two object suffixes, as shown by (\ref{caus4}):

\begin{exe}
\ex[*] {\label{caus4} \gll Jegaan a fal-\textbf{noor}-a-\ovalbox{\textbf{ang}\subscript{{\sc causee}}\textbf{-in}\subscript{{\sc object}}}.  \\
Jegaan 3 kick-\Caus{}-\Dv{}-\Ssg{}.\Obj{}-\Tsg{}.\Obj{} \\
\glt Intended: `Jegaan made you kick it.'\hfill {Two suffixes}}
\end{exe}

\noindent All of the facts just discussed are summarized in \tabref{tab:baier:doc}:

\begin{table}
\begin{center}
% {\renewcommand{\arraystretch}{1.24} 
% \renewcommand{\tabcolsep}{0.2cm} 
\begin{tabular}{c  c c c}
\lsptoprule
Type & Word Order & Object Suffix & Multiple Suffixes \\
\midrule
Ditransitive & {\sc sym} & {\sc sym} & \xmark{}  \\
Applicative & {\sc sym}   & {\sc sym} & \xmark{}  \\
Causative & {\sc asym}  & {\sc asym} & \xmark{}  \\
\lspbottomrule
\end{tabular}
\caption{Sereer double object constructions}
\label{tab:baier:doc}
\end{center}
\end{table}

\noindent The differences between symmetrical (\isi{ditransitive}/\isi{applicative}) and asymmetrical (\isi{causative}) \isi{double object} constructions are a convincing argument in favor of a syntactic account of object \isi{suffixation}. As we will see below, these differences can be relativized to independent principles of locality in which \isi{causatives} include a barrier to object \isi{suffixation} of the internal argument of the causativized \isi{predicate}, whereas ditransitives and applicatives do not.\footnote{See \cite{Baker:2012} for such an approach to similar data in \ili{Lubukusu}.} A non-syntactic account would have to stipulate these differences.

In addition, the general ban on multiple suffixes is an argument against approaches to object \isi{suffixation} that do not take place in the syntax, as such accounts would have to posit a different set of weak pronouns that occur as suffixes, and a stipulation would be required to block these suffixes from co-occuring. A syntactic approach, on the other hand, can take advantage of the idea that the operation triggering \isi{incorporation} of a \isi{pronoun} into the \isi{verb} only applies once per structure. 


\subsection{Passives}\label{sec:baier:3.3}

The final constraint on object \isi{suffixation} concerns passives. When a \isi{ditransitive} \isi{verb} is passivized, one of the underlying objects is promoted to subject, while the other object is left behind in the post-verbal position and treated as an object. Either object may be promoted to subject, as shown in (\ref{pass-di}):

\begin{exe}
\ex \label{pass-di}
\begin{xlista}
\ex \gll [\xp{DP} okoor oxe]\subscript{{\sc goal}} a ci'-\textbf{e}  [\xp{DP} atere le]\subscript{{\sc theme}} \\
{} man \Det{} 3 give-\Pass{} {} book \Det{} \\
\glt `The man was given the book.' \hfill {Goal subject}

\ex \gll [\xp{DP} atere le]\subscript{{\sc theme}} a ci'-\textbf{e} [\xp{DP} okoor oxe]\subscript{{\sc goal}} \\
{} book \Det{} 3 give-\Pass{} {} man \Det{} \\
\glt `The book was given to the man.' \hfill {Theme subject}
\end{xlista}
\end{exe}

\noindent In (\ref{pass-di}a), the goal argument is promoted to subject and the theme remains post-verbal as an object. In (\ref{pass-di}b), the theme is promoted to subject and the goal argument remains behind. When the object that remains post-verbal is a singular \isi{pronoun}, it \textit{cannot} be realized as a suffix. This is true regardless of which argument it refers to, as shown by (\ref{pass-di-suff}):

\begin{exe}
\ex \label{pass-di-suff}
\begin{xlista}
\ex[*] {\gll [\xp{DP} okoor oxe]\subscript{{\sc goal}} a ci'-\textbf{e}-\ovalbox{\textbf{n}} \\
{} man \Det{} 3 give-\Pass{}-\Tsg{}.\Obj{}  \\
\glt Intended: 'The man was given it.' \hfill {Goal suffix}}

\ex[*] {\gll [\xp{DP} atere le]\subscript{{\sc theme}} a ci'-\textbf{e}-\ovalbox{\textbf{n}}\\
{} book \Det{} 3 give-\Pass{}-\Tsg{}.\Obj{} \\
\glt Intended: `The book was given to him/her.' \hfill {Theme suffix}}
\end{xlista}
\end{exe}

\noindent In (\ref{pass-di-suff}a), the object suffix on the \isi{verb} corresponds to the theme argument. In (\ref{pass-di-suff}b), the object suffix on the \isi{verb} refers to the goal argument. Both examples are ungrammatical. This ungrammaticality is avoided by realizing the pronominal object as a full, free \isi{pronoun}. 

\begin{exe}
\ex \label{pass-di-pro}
\begin{xlista}
\ex \gll [\xp{DP} okoor oxe]\subscript{{\sc goal}} a ci'-\textbf{e}  [\xp{DP} a ten]\subscript{{\sc theme}} \\
{} man \Det{} 3 give-\Pass{} {} \Obj{} \Tsg{} \\
\glt `The man was given it.'

\ex \gll [\xp{DP} atere le]\subscript{{\sc theme}} a ci'-\textbf{e} [\xp{DP} a ten]\subscript{{\sc goal}} \\
{} book \Det{} 3 give-\Pass{} {} \Obj{} \Tsg{} \\
\glt `The book was given to him/her.' 
\end{xlista}
\end{exe}

\noindent As seen in (\ref{pass-di-pro}), a post-verbal object in a \isi{ditransitive} \isi{passive} is grammatical, while a object suffix is not. This observation is another argument for a syntactic approach to object \isi{suffixation}, as we expect different \isi{voice} types to enforce different syntactic constraints. An account that locates object \isi{suffixation} in a post-syntactic module of the grammar would have to appeal to a stipulation by stating that singular pronouns cannot be realized as suffixes in a structure with a \isi{passive}. Alternatively, one could say that there is a templatic restriction banning \isi{incorporation} into a \isi{passive} \isi{verb}. A syntactic analysis, on the other hand, can appeal to differences in the structure of active and passive sentences to account for the availability of object \isi{suffixation}. For instance, perhaps object \isi{suffixation} is triggered by a head present in the active that is not present in the \isi{passive}. I now move on to sketching such an approach in section 4.

\section{Towards an analysis}\label{sec:baier:4}

Before moving on to my analysis, I present a summary of the generalizations made above concerning object \isi{suffixation} in (\ref{summary}):

\begin{exe}
\ex  \label{summary} \textbf{Characteristics of Object Suffix}
\begin{xlista}
\ex There are only singular object suffixes
\ex An object suffix may not co-occur with an \textit{in situ} DP.
\ex An object suffix may not co-occur with an \abar-extracted object (\isi{focus}/\wh{}-relative)
\ex An object must co-occur with a topicalized object. 
\ex There is a limit of one object suffix per \isi{verb}.
\ex An object suffix is obligatory where possible.
\ex An object suffix may refer to either argument in a symmetrical DOC.
\ex An object suffix cannot refer to the theme of a causativized transitive \isi{verb}.  
\ex An object suffix cannot occur in a \isi{passive} \isi{verb}.
\end{xlista}
\end{exe}

\noindent In this section, I sketch an analysis that aims to capture the generalizations given above.

The core idea of my analysis is that object \isi{suffixation} involves \isi{head movement} of a \isi{pronoun} (\hdzero{D}) to the head \hdzero{\lilv{}}, which causes it to be morphologically incorporated into the \isi{verb}. This idea is schematized in (\ref{syntax3}):

\begin{exe}
\ex \label{syntax3} {[\xp{\lilv{P}} V+\lilv{}+OS [\xp{VP} \sout{V} [\xp{D} \sout{OS} ]]]} 
\end{exe}

\noindent There are two questions that must be answered with regards to the structure in (\ref{syntax3}). First, what triggers \isi{movement} of a pronominal \hdzero{D} to \hdzero{\lilv{}} and why does it only target singular pronouns? Second, why is the \isi{head movement} impossible in some circumstances, such as when there are multiple objects or when the \isi{verb} is \isi{passive}?

Building on analyses of \ili{Bulgarian} pronominal clitics by \citet{Harizanov2014} and \ili{Amharic} object suffixes by \citet{Kramer2014}, I suggest that \isi{incorporation} of a \isi{pronoun} into \hdzero{\lilv{}} is motivated by the operation Agree which is triggered by a \isi{probe} on \hdzero{\lilv{}}. Both Harizanov and Kramer and adopt the conception of \isi{head movement} developed by \citet{Matushansky2006} in which \isi{head movement} is taken to be regular phrasal \isi{movement} to a \isi{specifier} followed by a special operation \textsc{m-merger} which fuses a \isi{specifier} with its head. They argue that \isi{clitic doubling} in \ili{Amharic} and \ili{Bulgarian} derives from \isi{movement} of a DP to \isi{specifier} of \lilv{}, after which the DP m-merges with \hdzero{\lilv{}}. For Harizanov, m-merger of a XP reduces that projection to its label, yielding a \isi{complex head}. This is shown in (\ref{syntax4}):

\begin{exe}
\ex \label{syntax4} 
\begin{xlista}
\ex {[\xp{\lilv{P}} \tikz[baseline,remember picture] \node[anchor=base] (BaierDP1) {DP}; [ \lilv{} [\xp{VP} V \rnode{1a}{\tikz[baseline,remember picture] \node[anchor=base] (BaierDP2) {\sout{DP}}; ]]]} \hfill DP moves to Spec-\lilv{}P
\begin{tikzpicture}[overlay,remember picture]
 \draw[-{Stealth[]}] (BaierDP2.south east) -- ++(0,-.75\baselineskip) -| node[near start,fill=white] {\textsc{move}} (BaierDP1);
\end{tikzpicture}

% % % \ncbar[angleA=-90,angleB=-90,armA=1em,armB=1em,nodesep=3pt,linearc=2pt]{<-}{1b}{1a}
% % % \mput*{{\sc move}}
\vspace{1.5em}
\ex {[\xp{\lilv{P}} D+\lilv{} [\xp{VP} V \sout{DP}
P} ]]]} \hfill M-Merger of DP 
\end{xlista}
\end{exe}

\noindent Under this analysis, object \isi{suffixation} in \ili{Seereer} occurs because \hdzero{\lilv{}} is equipped with a \isi{probe} that causes a \isi{pronoun} to move to its \isi{specifier}. Later, that \isi{pronoun} undergoes m-merger with \hdzero{\lilv{}}, resulting in morphological \isi{incorporation} of the \isi{pronoun} into the \isi{verb}. 

I propose that active \hdzero{\lilv{}} in \ili{Seereer} is equipped with a \textsc{number probe} ([u\#]) that triggers \isi{movement} of an argument in VP to Spec-\lilv{}P. I follow much work on the operation Agree in assuming that probes can be relativized to search for specific values of a feature \citep{Bejar:2008, Bejar:2009,Preminger:2011b}. In this case, I assume that the number \isi{probe} on \hdzero{\lilv{}} is relativized to search for singular features. I represent this as [u\#\subscript{{\sc sg}}]. 

Assuming that the \#-\isi{probe} on \hdzero{\lilv{}} is relativized to search only for singular features immediately derives the fact that only singular pronouns will incorporate into the \isi{verb} in \ili{Seereer}, yielding only singular object suffixes. But how do we derive the fact that no doubling of an in situ DP object is possible in \ili{Seereer}? Recall that the \isi{head movement} approach I am employing assumes that DPs can undergo m-merger to form a \isi{complex head} with \hdzero{\lilv{}}. Thus, \isi{clitic doubling} should, in principle, be possible. 

I propose that the ability for XPs to undergo m-merger is subject to parametric variation. In languages like \ili{Bulgarian} it is possible, and therefore \isi{clitic doubling} occurs. In languages like \ili{Seereer}, however, it is not possible, and therefore DPs can never be doubled by object suffixes, as these suffixes are impossible to generate. Thus, we have two situations in \ili{Seereer}, given in (\ref{whenpro}) and (\ref{whenDP}).

% % % \todo{it is unclear in how far pstricks can be made to work with xelatex. Please try forest.}
\begin{exe}
\ex \label{whenpro} \textbf{Singular Pronoun = m-merger}\\
\begin{forest}
[\textit{v}P [\textit{v} [D] [\textit{v}+{[}u\#\textsubscript{\textsc{SG}}{]}] ] [VP [V] [\sout{D\textsubscript{SG}}]]] 
\end{forest}
% % % % {\small \jtree[xunit=2.6em,yunit=1.0em]
% % % % \def\\{[labelgapb=-1.2ex]}
% % % % % \everymath={\rm}
% % % % \! = {\lilv{P}}
% % % % : {{\lilv{}}}!a [scaleby=2.1 1] {VP}
% % % % : {V} {\sout{{D}\subscript{{\sc sg}}}}@a2 . 
% % % % \!a = : {{D}}@a1  {{\lilv{}+[u\#\subscript{{\sc sg}}]}}. 
% % % % \endjtree}
\end{exe}

\begin{exe}
\ex \label{whenDP} \textbf{Singular DP = no m-merger}\\
\begin{forest} nice empty nodes
[\textit{v}P [DP] [ [\textit{v}+{[}u\#\textsubscript{\textsc{SG}}{]}] [VP [V] [\sout{D\textsubscript{SG}}] ] ] ]
\end{forest}
% % % % {\small \jtree[xunit=2.6em,yunit=1.0em]
% % % % \def\\{[labelgapb=-1.2ex]}
% % % % % \everymath={\rm}
% % % % \! = {\lilv{P}}
% % % % : {DP} 
% % % % : {{\lilv{}+[u\#\subscript{{\sc sg}}]}}  {VP}
% % % % : {V} {\sout{{DP}\subscript{{\sc sg}}}}@a2 . 
% % % % \endjtree}
\end{exe}


\noindent In (\ref{whenpro}), the complement of V is a singular \isi{pronoun}, a minimal \hdzero{D}, and therefore, object \isi{suffixation} occurs. In (\ref{whenDP}), on the other hand, the complement of V is a singular DP. Therefore, m-merger of DP is not possible after it moves to Spec-\lilv{}P and no object suffix surfaces. This derives the fact that there is no doubling of full DPs by object suffixes in \ili{Seereer}.

\largerpage
A key characteristic of object \isi{suffixation} in \ili{Seereer} is that it is obligatory when it is possible, but when it is impossible, no ungrammaticality results. This is problematic for the idea that \isi{suffixation} is triggered by Agree, as we would expect sentences without singular objects and active \lilv{} to be ungrammatical. To alleviate this problem, I follow \citet{Preminger:2011b} in assuming that the failure of a \isi{probe} to find matching features does not result in crash. Therefore, a \#-\isi{probe} can be present on every active \lilv{}, but derivations without a singular DP object will not crash. This derives the generalization that object suffixes are obligatory when there is a singular object \isi{pronoun}, but when there is not one, the sentence is fine.\footnote{An alternative would be to posit that the \isi{probe} is only sometimes present on \hdzero{\lilv{}}. However, pursuing this line of thinking would require one to devise a way to enfore the \isi{probe}'s presence when there is at least one singular object \isi{pronoun} in the structure. I will avoid this discussion here, and leave the comparison of the two analyses to future work.}

Furthermore, because there is only one \#-\isi{probe} on \hdzero{\lilv{}}, only one object suffix is possible on any given \isi{verb}. Thus, I assume that once the \#-\isi{probe} on \hdzero{\lilv{}} has found a matching singular DP, it does not have to \isi{probe} further, and is satisfied. Thus, when there are two singular object pronouns in the structure, as in a DOC, the higher object \isi{pronoun} in the structure is found by the \#-\isi{probe} on \hdzero{\lilv{}}, and that \isi{pronoun} incorporates. The other is left free:

\begin{exe}
\ex \label{syntax5} {[\xp{\lilv{P}} [ \tikz[baseline,remember picture] \node[anchor=base] (Baierv) {\textit{v}}; [\xp{VP} \tikz[baseline,remember picture] \node[anchor=base] (BaierDi) {D\textsubscript{i}}; [ V D\textsubscript{k} ]]]]} \hfill DP moves to Spec-\lilv{}P
\begin{tikzpicture}[overlay,remember picture]
 \draw[{Circle[]}-{Circle[]}] (BaierDi.south) -- ++(0,-.75\baselineskip) -| (Baierv.south);
\end{tikzpicture}
\vspace*{1.5em}
% % % % \ncbar[angleA=-90,angleB=-90,armA=.8em,armB=.8em,nodesep=3pt,linearc=2pt]{*-*}{1b}{1a}
\end{exe}



\noindent In (\ref{syntax5}), the \#-\isi{probe} on \hdzero{\lilv{}} finds the higher of two object pronouns, and thus that one is the only one that is incorporated. 

Finally, this analysis is able to derive two further constraints on object \isi{suffixation}. First, because \lilv{} is responsible for encoding the \isi{voice} of the \isi{clause}, it is reasonable to assume that the \#-\isi{probe} is limited to certain \lilv{} heads. Namely, \isi{passive} \lilv{} lacks the \#-\isi{probe}, and therefore, no object suffix is possible in \isi{passive} structures. Second, the differences between symmetrical DOCs (ditransitives/applicatives) and asymmetrical DOCs (\isi{causatives}) can be derived by appealing to Phase-based locality \citep{Chomsky:2001a, Chomsky:2008}. 
In \isi{causative} DOCs, there is a phase boundary between the \isi{causee} object and the theme object which blocks Agree with the theme. In symmetrical DOCs, on the other hand, there is no such boundary, and therefore both objects can occur as suffixes. 


\section{Conclusion}\label{sec:baier:5}
\largerpage
In this paper, I have presented a description of \ili{Seereer} object suffixes, focusing on their distribution and syntactic behavior. On the basis of their distributional characteristics, I have argued that they are best analyzed as pronouns that are morphologically incorporated into the \isi{verb}. I have further argued that this process of \isi{incorporation} occurs in the syntax, in that it is constrained by syntactic structure. These constraints include the fact that object \isi{suffixation} is obligatory; that it cannot occur more than once per \isi{verb}; that it is sensitive to the \isi{voice} of the \isi{clause}; and that it is sensitive to the structure of \isi{double object} constructions. I have also also sketched an implementation of the syntactic approach based on the idea that active \hdzero{\lilv{}} in \ili{Seereer} bears a \#-\isi{probe} relativized to search for singular DPs, and that this \isi{probe} triggers \isi{head movement} of pronouns to adjoin to~\hdzero{\lilv{}}.

\section*{Acknowledgments}
I thank Peter Jenks and Line Mikkelsen for insightful comments, guidance, and discussion on this project, as well as the audience at ACAL 45 at the University of Kansas. I am also indebted to my consultant Malick Loum for taking the time to share his knowledge of \ili{Seereer} with me. All data in this paper were gathered during the 2012--2013 UC Berkeley Field Methods class and subsequent follow-up work with John Merrill in 2013--2014. 


\section*{Abbreviations}
\begin{tabularx}{.45\textwidth}{lQ}
 \Det{} & determiner\\
 \Dv{} & default vowel\\
 \Ext{} &  {extraction} suffix\\
 \Inf{} &  {infinitive}\\
 \Obj{} & object\\
 \Pl{} & plural\\
 \end{tabularx}
\begin{tabularx}{.45\textwidth}{lQ}
 \Rel{} & relative\\
 \Sg{} & singular\\
 1 & first person\\
 2 & second person\\
 3 & third person\\
 \\
 \end{tabularx}




{\sloppy
\printbibliography[heading=subbibliography,notkeyword=this]
}
\end{document}