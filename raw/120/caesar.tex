\documentclass[output=paper,
modfonts
]{langscibook} 
% \bibliography{localbibliography}
\ChapterDOI{10.5281/zenodo.1251726}

 
\title{A morphosyntactic analysis of adjectives in two Kwa languages: Ga and Dangme} 

\author{Regina Oforiwah Caesar\affiliation{University of Education, Winneba, Ghana}\lastand 
Yvonne A. A. Ollennu\affiliation{University of Education, Winneba, Ghana}}

\abstract{The adjective category normally serves as attribute for the nouns in languages that do have them. The paper investigates the morphosyntactic properties of adjectives in two Kwa languages, Ga and Dangme. Both languages have derived and non-derived adjectives. The paper which is mainly descriptive, examines the similarities and differences that exist between these two Kwa languages in terms of their morphological and syntactic features. The paper reveals that though similarities exist in the occurrence of adjectives syntactically, there exist differences in their morphological properties. On the other hand, Ga and Dangme show agreement in terms of number with the head noun for all adjectives used attributively. The paper concludes that in both languages, adjective occur after the head noun in attributive position. Predication of adjectives can occur in nominal forms and the verbal equivalence is also employed in both languages. Plural marking in adjectives is through reduplication and affixation in Ga while in Dangme, it is only through affixation. Data for this paper were collected from both primary and secondary sources. }
% \keywords{Adjectives, predicative, attributive, pluralization, reduplication}

\begin{document}
\maketitle

\section{Introduction} 
Adjectives as one of the subclasses of modifiers have been studied in Ghanaian languages by several linguists (e.g., \citealt{Osam2003,Ameka2003,Otoo2005,Adjei2007,Amfo2007,Danti2007,Dzameshie2007,Naden2007,PokuaaEtAl2007,Caesar2013}, a.o.). This study investigates the morpho-syntactic properties of adjectives in \ili{Ga} and \ili{Dangme}. \ili{Ga} and \ili{Dangme} belong to the Kwa group of languages from the \isi{Niger} \isi{Congo} family. \ili{Ga} is a two tone language whiles \ili{Dangme} is a three tone language. \ili{Ga} has twenty six letters in its alphabet whiles \ili{Dangme} has thirty letters of the alphabet. Both languages have seven oral vowels and five nasal vowels. 



\ili{Ga} is spoken along the coastal area in Accra. These areas include Ga  {Mashi}, Osu, La, Teshi, Nungua, Tema, Oyibi, Bawaleshi, and its surrounding villages. The area stretches to the foot of the Akwapim hills, the Nyanam hill up to Ninobi, and then to the southwards of Langma hill in the southern part of Kasoa, whiles \ili{Dangme} is spoken in two regions of \isi{Ghana}: Eastern and Greater Accra mainly in South-Eastern \isi{Ghana} along the coastal belt and the forest areas. \ili{Dangme} speakers thus, inhabit the coastal area of the Greater Accra Region, east of Accra and part of Eastern Region of \isi{Ghana}. \ili{Dangme} has seven dialects while \ili{Ga} has no dialects but there may be vocabulary differences geographically in terms of pronunciation. The seven dialects of \ili{Dangme} include: Ada, Nugo, Kpone, Gbugblaa/Prampram, Osudoku, Sɛ and Krobo (Yilo and Manya). There are several small communities east of the Volta Region that trace their origins to \ili{Dangme} land; most of these have shifted to \ili{Ewe} as a language of daily life but others have not (Dakubu 1966, Sprigge 1969 cited in \citealt{Ameka2008}). Patches of speakers are also found in \isi{Togo} land for instance, Nyetoe and Gatsi. Data used for this study were collected from some native speakers of \ili{Ga} and \ili{Dangme} and were cross-checked with other native speakers. Data were also drawn from books (\citealt{Ablorh-Odjidjah1961,Dakubu1987,Dakubu2000,Adams1999,Adams2000,Amfo2007,Adi2003,Odonkor2004,Otoo2005}; and \citealt{Caesar2013}). The aim of this paper is to find out how similar or different the usage of adjectives in these two languages looks like. It is said that a concept in a language may be expressed in another, using a word from a particular word class but the same concept may be expressed in another language using a word from a different word class \citep{Dixon1977,Dixon1982,Dixon2004}. The theoretical framework is based on \citegen{Dixon2004} classification of adjectives. 


\subsection{Theoretical framework}\label{sec:caesar:1.1} 

\citet{Dixon2004} identifies a set of semantic types of property concepts that are encoded by the adjective class in languages that have them. There are thirteen classes in his recent work, which are:


\begin{itemize}
\setlength\itemsep{0em}
 \item \textsc{dimension} e.g. big, small, long, deep, etc
 \item \textsc{physical property}, e.g. hard, strong, sweet, cheap, etc
 \item \textsc{speed}, e.g. fast, quick, rapid, etc
 \item \textsc{age}, e.g. new, old young, modern, etc
 \item \textsc{colour}, e.g. black, white, golden, etc
 \item \textsc{value}, e.g. good, bad, lovely, pretty, etc
 \item \textsc{difficulty}, e.g. easy, tough, hard, simple, etc
 \item \textsc{volition}, e.g accidental, purposeful, deliberate, etc
 \item \textsc{qualification}, (this has subtypes) e.g. true, obvious, normal, right, etc
 \item \textsc{human propensity}, (this also has subtypes), e.g. angry, jealous, clever, sad, etc
 \item \textsc{similarity}, e.g. different, equal (to) analogous (to), etc
 \item \textsc{quantification}, e. g. many, few, plenty, little, etc
 \item \textsc{position}, e.g. high, low, etc. 
\end{itemize}
 
 
For the purpose of this study, we use examples from the following classes: dimension, colour, value, age and physical property. This is because the adjectives in these two Kwa languages are mostly found in these groups. Other semantic groupings will be investigated in future.


\section{Sources of adjectives}\label{sec:caesar:1.3} 

Linguistic scholars \citep{Dakubu1987,Adi2003,Adams1999,Adams2000,Odonkor2004,Amfo2007,Caesar2013} have identified that \ili{Ga} and \ili{Dangme} have both deep level and derived adjectives. Deep level adjectives are monomorphemic, that is, they cannot be segmented into morphemes to be meaningful. Examples of deep level are found below in \REF{ex:caesar:1}.


 
\ea\label{ex:caesar:1}  
\caesarbox{\textbf{Ga:} }{\textbf{\ili{Dangme}:} }\\
\caesarbox{\textit{agbo}  ‘big’ }{\textit{yumu} ‘black’  }\\ 
\caesarbox{\textit{kpitioo} ‘short’ }{\textit{kpiti}  ‘short’ }\\ 
\caesarbox{\textit{kpakpa} ‘good’}{\textit{kpakpa} ‘good’ }\\ 
\caesarbox{\textit{kakadaŋŋ} ‘long’}{\textit{wayoo} ‘small’ }\\ 
\caesarbox{\textit{gojoo} ‘huge’ }{\textit{gojoo} ‘huge’ }\\  
\z
% \todo{kpakpa \& gojoo double. Intended?} 
\subsection{Adjectives derived from verbs}\label{sec:caesar:2.1} 

Adjectives in \ili{Dangme} could also be derived from verbs through either total or partial reduplication. When adjectives are derived through total reduplication in \ili{Dangme}, the reduplicant takes an additional segment base on the vowel of the verbs stem, that is, \isi{verb} stems that end in \{u, o, or ɔ\} take \{i, e or ɛ\} to arrive at the adjective. Two processes occur in partial reduplication: the deletion of a consonant and a rise in tongue high level of the vowel of the reduplicant morpheme. Consider the following examples in \ili{Dangme}:


\ea\label{ex:caesar:2}
\textbf{\ili{Dangme}:} \\
\caesarbox{\textbf{Verb} }{\textbf{Reduplicated form} }\\ 
\caesarbox{\textit{bla} ‘to join’ }{\textit{ba{$\sim$}blɛ} ‘joint’}\\ 
\caesarbox{\textit{ngla} ‘to burn’  }{\textit{nga{$\sim$}nglɛ} ‘burnt}\\ 
\caesarbox{\textit{sa}  ‘to spoil’ }{\textit{sa{$\sim$}sɛ}  ‘rotten’}\\ 
\caesarbox{\textit{tsu}  ‘to redden’ }{\textit{tsu{$\sim$}tsu} ‘red’}\\ 
\caesarbox{\textit{gbo} ‘to die’  }{\textit{gbo{$\sim$}gboe} ‘dead’}\\ 
\caesarbox{\textit{pɔ}  ‘to wet’ }{\textit{pɔ{$\sim$}pɔe} ‘wet’}\\ 
\caesarbox{\textit{fi}  ‘to tie’ }{\textit{fi{$\sim$}fii}  ‘tried’}\\ 
\caesarbox{\textit{si} ‘to fry’ }{\textit{si{$\sim$}sii} ‘fried’}\\ 
\caesarbox{\textit{ngma} ‘to write’ }{\textit{ngma{$\sim$}ngmɛɛ} ‘written’}\\  
\z


The lateral /l/ is elided in the base of the reduplicated forms \textit{bablɛ} ‘joint’ and \textit{nganglɛ} ‘burnt’ and there is a rise in tongue height. That is, the low front vowel /a/, has changed to the low mid vowel /ɛ/ as exemplified in the first two examples of \REF{ex:caesar:2} above. From the data, it is observed also that reduplicated adjectives in \ili{Dangme} can be formed by copying the whole of the base and by adding a vowel to the reduplicated part of the \isi{verb} as shown in the last five examples of \REF{ex:caesar:2} above.



Adjectives could also be derived from verbs in \ili{Ga} through affixation and reduplication. Affixation is characterised with the \isi{suffixation} of \{-ŋ\}, \{-ru\} and \{-ra\} to the base form of some verbs to form adjectives. \{-ŋ\} is attached to \isi{verb} stems that end in /i, ɛ/. \{-ru\} is suffixed to \isi{verb} stems that end in /u/, and \{-ra\} is attached to \isi{root} forms that end in /a/. Consider the \ili{Ga} examples below:


\ea\label{ex:caesar:3}
\textbf{Ga:} \\
\caesarbox{\textbf{Verb} }{\textbf{Adjective}}\\ 
\caesarbox{\textit{gbi} ‘to dry’  }{\textit{gbi-ŋ} ‘dried’ }\\ 
\caesarbox{\textit{di} ‘to blacken’  }{\textit{di-ŋ} ‘black’}\\ 
\caesarbox{\textit{yɛ} ‘to whiten’  }{\textit{yɛ-ŋ} ‘white’}\\ 
\caesarbox{\textit{tsu}  ‘to ripe’ }{\textit{tsu-ru} ‘red’}\\ 
\caesarbox{\textit{sha} ‘to spoil’ }{\textit{sha-ra}  ‘spoilt’}\\  
\z




The \ili{Ga} examples in \REF{ex:caesar:3} above have adjectives formed from verbs through affixation. The illustrations below also show the reduplication process. The \isi{verb} is either suffixed with a segment or morpheme before reduplicated, or it is reduplicated and then suffixed with the /i/ segment as in \REF{ex:caesar:4} below: 


\ea\label{ex:caesar:4}
\textbf{Ga:} \\
\caesarbox{\textit{boda} ‘to be bent’ }{\textit{boda{$\sim$}bodai}  ‘crooked’ }\\
\caesarbox{\textit{kwɔ} ‘to be deep’ }{\textit{kwɔŋ{$\sim$}kwɔŋ} ‘deep’ }\\
\caesarbox{\textit{nyaŋe} ‘to shun/despise’ }{\textit{nyaŋemɔ{$\sim$}nyaŋemɔ} ‘disgusting’ }\\ 
\z


In complete reduplication in Ga, the segment /i/ is attached to the reduplicated part of some verbs as in \textit{boda-boda-i} ‘crooked’. However, a \isi{verb} such as \textit{kwɔ} ‘to be deep’ is suffixed with the \isi{velar} nasal /ŋ/ to become \textit{kwɔŋ} before it is reduplicated as \textit{kwɔŋkwɔŋ} ‘deep’. It is realised that the \isi{velar} nasal, /ŋ/ is attached to the base as well as the reduplicated part of the adjective. Similarly, the \isi{verb} \isi{root}, \textit{nyaŋe} ‘to shun/despise’ has been suffixed with \{-mɔ\} to become \textit{nyaŋemɔ} before it is reduplicated as \textit{nyaŋemɔnyaŋemɔ} ‘disgusting’. This class of verbs is limited in Ga.


\subsection{Adjectives derived from nouns}\label{sec:caesar:2.2} 

The stock of \ili{Dangme} and \ili{Ga} adjectives could also be added to through reduplication of nouns. In Ga, the nouns are mostly pluralised first and then reduplicated. In \ili{Dangme}, however, some of the nouns just go through a total reduplication while others are pluralised as in \ili{Ga} before they are reduplicated to derive adjectives. Consider the \ili{Dangme} examples below: 


\ea\label{ex:caesar:5}
\textbf{\ili{Dangme}:}\\ 
\caesarbox{\textbf{Noun} }{\textbf{Reduplicated form (Adjective)}}\\
\caesarbox{\textit{zu}  ‘sand’ }{\textit{zu{$\sim$}zu} ‘sandy’ }\\
\caesarbox{\textit{mamu} ‘powder’ }{\textit{mamu{$\sim$}mamu} ‘powdery’ }\\
\caesarbox{\textit{nyu} ‘water’ }{\textit{nyu{$\sim$}nyu} ‘watery’ }\\
\caesarbox{\textit{zↄ} ‘oil’ }{\textit{zↄ{$\sim$}zↄ} ‘oily’ }\\
\caesarbox{\textit{tso} ‘tree’ }{\textit{tsohi{$\sim$}tsohi}  ‘spongy’ }\\
\caesarbox{\textit{tɛ} ‘stone’ }{\textit{tɛhi{$\sim$}tɛhi} ‘rocky’ }\\
\caesarbox{\textit{wu} ‘bone’ }{\textit{wuhi{$\sim$}wuhi} ‘bony’ }\\
\caesarbox{\textit{kpↄ} ‘lump’ }{\textit{kpↄhi{$\sim$}kpↄhi}  ‘lumpy’ }\\ 
\z


It is observed in the \ili{Dangme} examples in \REF{ex:caesar:5} that the first four examples of the reduplicated nouns have not taken on any affix after the reduplication. However, the last three have attached the plural marker for common nouns, \{-hi\}, to indicate that they are countable.


\ea\label{ex:caesar:6}
\textbf{Ga:}\\
\caesarbox{\textbf{Noun} }{\textbf{Reduplicated form (Adjective)}}\\
\caesarbox{\textit{tɛ} ‘stone’ }{\textit{tɛi{$\sim$}tɛi}  ‘stony/rocky’ }\\
\caesarbox{\textit{tso} ‘tree }{\textit{tsei{$\sim$}tsei} ‘spongy’ }\\
\caesarbox{\textit{nu} ‘water’ }{\textit{nui{$\sim$}nui} ‘watery’ }\\
\caesarbox{\textit{shia} ‘sand’ }{\textit{shia{$\sim$}shiai} ‘sandy’ }\\
\caesarbox{\textit{kpaa} ‘rope’ }{\textit{kpaa{$\sim$}kpai} ‘knotty’  }\\
\caesarbox{\textit{kpɔ} ‘lump’ }{\textit{kpɔi{$\sim$}kpɔi} ‘lumpy’ }\\
\caesarbox{\textit{wu} ‘bone’ }{\textit{wui{$\sim$}wui} ‘bony’ }\\ 
\z



\ili{Ga} generally attaches the plural suffix \{–i\} to count nouns while \ili{Dangme} adds \{-hi\} to common count nouns to form its plural. It is observed from the data on \ili{Ga} that unlike \ili{Dangme} which pluralises only the count nouns, \ili{Ga} pluralises the count and some non-count nouns by attaching the morpheme \{-i\} to the base and the reduplicated forms of the word in many cases like \textit{nu} ‘water’ becoming \textit{nu-i-nu-i} ‘watery’. It is however observed that the base form of the \isi{noun}, \textit{shia} ‘sand’ has not been pluralised in the reduplicated form \textit{shia-shia-i} ‘sandy’. A few of the nouns however, do not add the \{-i\} segment to the nouns to form adjectives in Ga. See some examples in \ili{Ga} below:


\ea\label{ex:caesar:7} 
\caesarbox{ \textit{ŋmↄtↄ}  ‘mud’ }{\textit{ŋmↄtↄ{$\sim$}ŋmↄtↄ } ‘muddy’  }\\
\caesarbox{ \textit{kotsa} ‘sponge’ }{\textit{kotsa{$\sim$}kotsa} ‘spongy’ }\\
\caesarbox{ \textit{ŋoo} ‘salt’ }{\textit{ŋoo{$\sim$}ŋoo}  ‘salty’  }\\ 
\z

 
\section{Morphosyntactic properties}

This section discusses the morphological process of number agreement and reduplication of adjectives in these two Kwa Languages. Adjectives in \ili{Ga} and \ili{Dangme} can inflect for number (\citealt{Dakubu1987,Dakubu2000,Adams1999,Adams2000}). The \ili{Ga} adjective is marked overtly to indicate plurality to show agreement with the head \isi{noun} it modifies. However, in \ili{Dangme}, adjectives are not marked morphologically to show number agreement in the constructions when the definite article is present in the NP. The number agreement is marked on the definite article to indicate plurality and therefore has scope over the entire \isi{noun} phrase. 


\subsection{Plural formation in adjectives in Ga and Dangme}

In \ili{Ga} and \ili{Dangme}, the plural affix of a \isi{noun}, is either attached to the adjective that qualifies the preceding \isi{noun} or follows the NP. The adjective suffixes in \ili{Ga} include \{-i. -ji. -bii\} and the zero morpheme in few instances. \ili{Dangme} also has \isi{noun} plural suffixes \{-hi, -mɛ, -bi, -wi, -li\}. The animate or human nouns in \ili{Dangme} are what are marked morphologically for plural in \ili{Dangme}. In \isi{noun} phrases where adjectives are present, the adjectives select the \{-hi\} plural marker. \{-mɛ\} however, is attached to definite and indefinite articles, such as \{ɔ, a or ko\}, to form the plural of the articles. The plural marking on any of the articles scopes over the entire \isi{noun} phrase. Consider the examples in \ili{Dangme} below:


\protectedex{
\ea\label{ex:caesar:8}
\textbf{Dangme}:\\
\gll {Bo}  \textbf{{tsutsu-hi}} {ngɛ} {daka} {a} {mi.}\\
cloth red-\textsc{pl} are box \textsc{def} in \\
\glt ‘There are red cloths in the box.’
\z
}


\ea\label{ex:caesar:9}
\gll {Mangoo}  \textbf{{mumu}} \textbf{{ɔ{}-mɛ}} {sa.}\\
mango fresh \textsc{def}-\textsc{pl} rotten.\\
\glt ‘The fresh mangoes are rotten.’
\z


\ea\label{ex:caesar:10}
\gll {Duku} \textbf{{futa}} \textbf{{a-mɛ}} {sɛ mu.}\\
scarf white \textsc{def}-\textsc{pl} dirty.\\
\glt ‘The white scarves are dirty.’
\z




In example \REF{ex:caesar:8}, the adjective \textit{tsutsu-hi} ‘red.\textsc{pl}’ has taken the plural marker of the \isi{noun} \textit{bo} ‘cloth’ which is \{-hi\} retaining the \isi{noun} in its singular form in the \isi{clause}. In example \REF{ex:caesar:9}, the adjective \textit{mumu} ‘fresh’ comes in between the subject \isi{noun} \textit{mangoo} ‘mango’ and the definite plural marker \textit{‘ɔmɛ’} in the morphology. Similarly in example \REF{ex:caesar:10}, the adjective \textit{futaa} ‘white’ comes in between the subject \isi{noun} \textit{duku} ‘scarf’ and the definite plural marker, \textit{amɛ}.



In \ili{Dangme} we observe that to show number agreement in a \isi{noun} phrase, it is the plural form of the definite or indefinite articles that is used. The definite or indefinite article informs us that the \isi{noun} and adjective are in their plural forms even though the nouns are not marked morphologically as in (\ref{ex:caesar:8}-\ref{ex:caesar:10}) above. Consider the \ili{Ga} examples in (\ref{ex:caesar:11}-\ref{ex:caesar:13}).
\newpage 






\ea\label{ex:caesar:11}
\textbf{Ga:}\\
\gll {Wo-ji} \textbf{{he-i}} {lɛ} {dara.}\\
 book-\textsc{pl} new-\textsc{pl} \textsc{def} big.\textsc{iter}\\
\glt  ‘The new books are big.’
\z



\ea\label{ex:caesar:12}
\gll {Atade-i} \textbf{{fɛɛfɛ-ji}} {lɛ} {elaaje.}\\
 dress-\textsc{pl} beautiful-\textsc{pl} \textsc{def} lost\\
\glt  ‘The beautiful dresses are lost.’
\z

\ea\label{ex:caesar:13}
\gll {Tse-i} \textbf{{kakada-ji}} {lɛ} {kumɔ.}\\
tree-\textsc{pl} long-\textsc{pl} \textsc{def} broke.\textsc{iter} \\
\glt ‘The tall trees broke.’
\z




On the other hand, \ili{Ga} has a suffix on both the nouns \textit{wo-ji} ‘books’ \textit{atade-i} ‘dresses’ and adjectives \textit{he-i} ‘new’, \textit{fɛɛfɛ-ji} ‘beautiful.\textsc{pl}’, \textit{tse-i} ‘tress’, and \textit{kakada-ji} ‘tall ones’ to show number agreement. The definite article however, is not marked as it has no plural form in Ga. Another observation from the above \ili{Ga} example is that the definite article retains its shape for both the singular and the plural forms of the nouns. Some adjectives are also reduplicated in \ili{Ga} and \ili{Dangme} to express plural number in the entity named as in (\ref{ex:caesar:14}-\ref{ex:caesar:17}) below:

\ea\label{ex:caesar:14}
\textbf{Ga:}\\
\gll {E-he} \textbf{{shikpɔŋ}} \textbf{{lɛkɛtɛɛ}}.\\
 3\textsc{sg}-buy land wide\\
\glt ‘He/she bought a wide land.’
\z



\ea\label{ex:caesar:15}
\gll {E-he} \textbf{{shikpɔ-ji}} \textbf{{lɛkɛtɛ{$\sim$}lɛkɛtɛɛ.}}\\
 3\textsc{sg}-buy land-\textsc{pl} wide{$\sim$}\textsc{red}\\
\glt ‘He/she bought wide lands.’
\z


\ea\label{ex:caesar:16}
\textbf{\ili{Dangme}:}
\gll {E} {jua} \textbf{{blodo}} \textbf{{daka.}}\\
3\textsc{sg} sell.\textsc{aor} bread box\\
\glt ‘He/she sold a box full of bread.’
\z



\ea\label{ex:caesar:17}
\gll {E} {juaa} \textbf{{blodo}} \textbf{{daka{$\sim$}daka.}}\\
3\textsc{sg} sell.\textsc{aor} bread box{$\sim$}\textsc{red}\\
\glt ‘He/she sold boxes of bread.’
\z

The \ili{Ga} example \textit{lɛkɛtɛɛ} ‘wide’ in \REF{ex:caesar:14} denotes a singular number. On the other hand, \textit{lɛkɛtɛ-lɛkɛtɛɛ} ‘wide-wide’, denotes plurality in the plural \isi{noun}, \textit{shikpɔ-ji} it qualifies in \REF{ex:caesar:15}. But in \ili{Dangme}, the mere reduplication of the adjective does denote plurality. Plural marking affixes are however, not expressed on the \isi{noun} being modified as exemplified in (\ref{ex:caesar:16}-\ref{ex:caesar:17}). 

\subsection{Reduplication of adjectives}\label{sec:caesar:3.3} 

\ili{Ga} and \ili{Dangme} adjectives can also be reduplicated. Normally when an adjective is reduplicated, it shows intensity. As in the NPs, the reduplicated adjectives are pluralised in both their base and the reduplicant parts. Below are examples to illustrate: 



\ea\label{ex:caesar:18}

\textbf{Ga:}\\
\caesarbox{\textit{wulu}  ‘big’ }{\textit{wuji{$\sim$}wuji} ‘big’ }\\
\caesarbox{\textit{kpitioo}  ‘short’ }{\textit{kpitibii{$\sim$}kpitibii} ‘short’ }\\
\caesarbox{\textit{bibioo}  ‘small’ }{\textit{bibii{$\sim$}bibii}  ‘small’ }\\
\caesarbox{\textit{wamaa}  ‘large’ }{\textit{wamaa{$\sim$}wamaa} ‘large’ }\\ 
\z

\protectedex{
\ea\label{ex:caesar:19}
\textbf{\ili{Dangme}:}\\
\caesarbox{\textit{agbo}  ‘big’ }{\textit{agbo{$\sim$}agbo}  ‘big’ }\\
\caesarbox{\textit{nyafi} ‘small’ }{\textit{nyafi{$\sim$}nyafi}  ‘small’ }\\
\caesarbox{\textit{yumu}  ‘black’ }{\textit{yumu{$\sim$}yumu} ‘blackened’ }\\
\caesarbox{\textit{tsutsu} ‘red’ }{\textit{tsutsu{$\sim$}utsu} ‘reddish’ }\\ 
\z
}


The reduplication process can be total or partial. This is demonstrated in example \REF{ex:caesar:19} where \textit{tsutsu} 'red' becomes \textit{tsutsuutsu} 'reddish'. The reduplication process in \ili{Dangme} is total whereas in Ga, there is generally the \isi{suffixation} of \{-i, -bii, -jii\} as plural affixes to adjectives and nouns. With the exception of few adjectives such as \textit{wamaa –wamawamaa,} all other nouns and adjectives take any of the three suffixes above. Their conditioning is not discussed in this paper. See below how some of these reduplicated forms of adjectives can occur in sentences.

\ea\label{ex:caesar:20}
\textbf{Ga:}

\gll {Gbekɛ} \textbf{{bibioo}} {lɛ} {e-wↄ.}\\
child small \textsc{def} \textsc{perf}-sleep\\
\glt  ‘The little child is asleep.’
\z



\ea\label{ex:caesar:21}
\gll {Gbekɛbii} \textbf{{bibii{$\sim$}bibii}} {lɛ} {e-wↄ.}\\
 child-\textsc{pl} small{$\sim$}very \textsc{def} .\textsc{perf}-sleep\\
\glt  ‘The little children are asleep.’
\z

\ea\label{ex:caesar:22}

\textbf{\ili{Dangme}:}

\gll {Tade} \textbf{{yumu}} {ɔ} {gba.} \\
 dress black \textsc{def} tear.\textsc{aor} \\
\glt ‘The black dress is torn.’
\z



\ea\label{ex:caesar:23}
\gll {Tade} \textbf{{yumu{$\sim$}yumu}}  {ɔ} {ngɛ} {tsu} {ↄ} {mi}. \\
 dress black{$\sim$}very \textsc{def} is room \textsc{def} inside\\
\glt ‘The very black dress is in the room.’ 
\z

In the \ili{Ga} examples in (\ref{ex:caesar:20}-\ref{ex:caesar:21}), it is observed that the reduplicated form of \textit{bibioo} ‘a small…’ is \textit{bibiibii} ‘very small ones’. \textit{Bibioo} expresses singularity in the entity being discussed while \textit{bibiibii} expresses intensity and plurality. In the \ili{Dangme} examples in (\ref{ex:caesar:22}-\ref{ex:caesar:23}), it is observed that \textit{yumu} ‘black’ has been reduplicated as \textit{yumuyumu} ‘very black’. The reduplicated form, \textit{yumuyumu,} shows the intensity of the colour \textit{yumu,} ‘black’.



Both \ili{Ga} and \ili{Dangme} nominalise the adjective to be the head of an NP and also to be a subject of a sentence. To nominalise an adjective in \ili{Ga} and \ili{Dangme}, the prefix \{e-\} is attached to certain class of adjectives. However some of these adjectives are not attached with the prefix but remain in the same form as nominals. The prefix \{e-\} nominalises a class of adjectives that denotes colour and age of objects. For instance, in Ga, \textit{yɛŋ} `white' becomes \textit{e-yɛŋ} ‘white one’, \textit{diŋ} `black' becomes \textit{e-diŋ} ‘black one’, \textit{tsuru} `red' becomes \textit{e-tsuru} ‘red one’, \textit{hee} `new' becomes \textit{e-hee} ‘new/new one’, \textit{momo} `old' becomes \textit{e-momo} ‘old one’, \textit{ŋmↄŋ} ‘fresh’ becomes \textit{e-ŋmↄŋ} ‘fresh one’. Likewise, in \ili{Dangme}, \textit{ku} ‘male’ becomes \textit{ku-e-ku} ‘male one’, \textit{yo} ‘female’ becomes \textit{e-yo} ‘female’, \textit{he} ‘new' becomes \textit{e-he} ‘new one', \textit{agbo} ‘big’ becomes \textit{e-agbo} ‘big one’, \textit{wayoo} ‘small’ becomes \textit{e-wayoo} ‘small one’. The conditioning for the zero allomorph is yet to be investigated. 

\section{Functions of adjectives}\label{sec:caesar:3} 

\citet{Dixon2004} asserts that adjectives typically fill two roles in the grammar of a language. These two roles are the attributive and predicative use of adjectives. In addition to these roles, the adjective can occur in comparative constructions. When an adjective plays the attributive role, it serves as a modifier to the head \isi{noun}. When the adjective is used predicatively, it occurs as a copula complement in most languages. \citet{Dixon2004} notes however that these two roles, attributive and predicative may not occur for all adjectives in all languages. In certain instances, only one of these roles may be found. Such adjectives can occur within sentences as exemplified below. We begin with adjectives in attributive position.

\subsection{Adjectives in attributive position}\label{sec:caesar:3.1} 




\ili{Dangme} adjectives as well as \ili{Ga} adjectives that qualify nouns have the reversal structure as compared to \ili{English}. Such adjectives come after the nouns they qualify in a phrase, \isi{clause} or of any kind. Consider the following examples in \ili{Dangme} and Ga:

\ea\label{ex:caesar:24}
\textbf{\ili{Dangme}:}


\gll {Womi} \textbf{{he}} {ↄ} {ka.}\\
book new \textsc{def} be.long\\
\glt  ‘The new book is long.’
\z



\ea\label{ex:caesar:25}
\gll {Bɔɔlu} \textbf{{momo}} {ɔ} {pɛ.}\\
 ball old \textsc{def} burst\\
\glt  ‘The old ball is burst.’
\z



\ea\label{ex:caesar:26}
\gll {Kɔɔpoo} \textbf{{agbo}} {ɔ} {hyi.}\\
 cup big \textsc{def} be.full\\
\glt ‘The big cup is full.’
\z



\ea\label{ex:caesar:27}
\gll {E} {juaa} {blodo} \textbf{{bɔdɔɔ.}}\\
 he/she sell.\textsc{hab} bread soft\\
\glt ‘He/she sells soft bread.’
\z

\ea\label{ex:caesar:28}
\textbf{Ga:} 

\gll {Nuu} \textbf{{agbo}}  {lɛ} {wɔ} {vii.}\\
man big \textsc{def} sleep. \textsc{pst} deeply\\
\glt ‘The big man slept soundly/deeply.’
\z

\ea\label{ex:caesar:29}
\gll {Wolo} \textbf{{hee}} {lɛ} {da.}\\
 book new \textsc{def} be big\\
\glt ‘The new book is big.’
\z



\ea\label{ex:caesar:30}
\gll {Atade}  \textbf{{momo}}  {lɛ} {e-tsere.} \\
 dress old \textsc{def} 3\textsc{sg}-tear\\
\glt  ‘The old dresses are torn.’
\z



\ea\label{ex:caesar:31}
\gll {Blodo} \textbf{{bɔdɔɔ}} {e-hɔ-ɔ.}\\
 bread soft 3\textsc{sg}-sell-\textsc{hab} \\
\glt  ‘He/she sells soft bread.’
\z

In examples (\ref{ex:caesar:24}-\ref{ex:caesar:25}) in \ili{Dangme}, \textit{he} ‘new’ and \textit{momo} ‘old’ are expressing age. \textit{Agbo} ‘big’ indicates dimension in \REF{ex:caesar:26} and \textit{bɔdɔɔ} ‘soft’ shows physical property. All these adjectives come after the nouns, \textit{womi} ‘book’, \textit{bɔɔlu} ‘ball’, \textit{kɔɔpoo} ‘cup’ and \textit{blodo} ‘bread’ in \REF{ex:caesar:27} they respectively qualify. In example (\ref{ex:caesar:28}-\ref{ex:caesar:31}) on Ga, it is realised also that \textit{agbo} ‘big’, \textit{bɔdɔɔ} ‘soft’, \textit{hee} ‘new’ and \textit{momo} ‘old’ express dimension, physical property, and age as in the \ili{Dangme} examples in (\ref{ex:caesar:24}-\ref{ex:caesar:27}). These adjectives in Ga, are preceded by the head nouns \textit{nuu}, ‘man’, \textit{wolo} ‘book’, \textit{atade} ‘dress’ and \textit{blodo} ‘bread’. From the above examples in (\ref{ex:caesar:24}-\ref{ex:caesar:27}) in \ili{Dangme} and (\ref{ex:caesar:28}-\ref{ex:caesar:31}) in Ga, we see the adjectives occurring after the nouns they modify, followed by the definite article, if present in the \isi{clause}. Thus, the adjective comes in between the \isi{noun} and the definite or indefinite article as in other Ghanaian languages. 

The adjective for ‘beautiful’ in \ili{Ga} is \textit{fɛɛfɛo}. In example \REF{ex:caesar:32}, \textit{fɛɛfɛo} is used attributively in a \isi{clause}. \ili{Dangme} on the other hand, employs a whole phrase such as \textit{‘he ngɛ fɛu’} or \textit{‘kɛ e he fɛu’} in other to express the attributive use of the adjective, ‘beautiful’ as in example \REF{ex:caesar:33} below:


\ea\label{ex:caesar:32}
\textbf{Ga:}\\
\gll {Asupaatere} \textbf{{f}}\textbf{{ɛɛ}}\textbf{{f}}\textbf{{ɛ}}\textbf{{o}} {l}{ɛ} {etse.} \\
shoe/sandals beautiful \textsc{def} tear.\textsc{pst}\\
\glt ‘The beautiful pair of shoes is torn.’
\z

\ea\label{ex:caesar:33}
\textbf{\ili{Dangme}:}\\
\gll  {Tokota} \textbf{{kɛ}} \textbf{{e}} \textbf{{he}} \textbf{{fɛu}}  {ɔ} {hia.}\\
shoe/sandal with 3\textsc{sg}.\textsc{poss} part beauty \textsc{def} tear.\textsc{aor}\\
\glt ‘The beautiful pair of shoes is torn.’
\z



\subsection{More than one adjective in attributive position}\label{sec:caesar:4.2} 

The paper also investigates when more than one adjective is used attributively for a \isi{noun}. In Ga, it is observed that sometimes the last adjective used in a sequence is prefixed with \{e-\}. Consider the \ili{Ga} examples (\ref{ex:caesar:34}-\ref{ex:caesar:37}) below:
\newline

\ea\label{ex:caesar:34}
\textbf{Ga:}\\
\gll {Tɛ} \textbf{{bibioo}}{}  \textbf{{e-hee}} {ko} {ka} {jɛmɛ.}\\
 stone little \textsc{nmlz-}new \textsc{def} lie there.\\
\glt ‘There is a new small stone lying there.’
\z



\ea\label{ex:caesar:35}
\gll {Gbee} \textbf{agbo} \textbf{kpitioo} {lɛ} {gbo.}\\
 dog big short \textsc{def} die.\textsc{pst}\\
\glt ‘The big short dog is dead.’
\z


\ea\label{ex:caesar:36}
\gll {Mi-he} {shia} \textbf{fɛɛfɛo} \textbf{agbo} {ko.}\\
 1\textsc{sg-}buy.\textsc{pst} house beautiful big certain.\\
\glt ‘I bought a certain big beautiful house.’
\z



\ea\label{ex:caesar:37}
\gll {Gbe-i} \textbf{{agbo-i}} \textbf{e-di-ji} {lɛ} {egbo.}\\
 dog-\textsc{pl} big-\textsc{pl} \textsc{nmlz-}black-\textsc{pl} \textsc{def} \textsc{perf}.die\\
\glt  ‘The big black dogs are dead.’
\z

In the \ili{Ga} examples in (\ref{ex:caesar:34}-\ref{ex:caesar:37}), the adjectives are \textit{bibioo} ‘small’ and \textit{ehee} ‘new’, \textit{agbo} ‘big’ and \textit{kpitioo} ‘short’, \textit{fɛɛfɛo} ‘beautiful’ and \textit{agbo} ‘big’, \textit{agbo-i} ‘big ones’ and \textit{edi-ji} ‘black ones’ have respectively occurred in a sequence. In \ili{Dangme}, the category of age, value or colour may precede those with physical or dimension properties. Those from the physical property and human property tend to be used in copula complement function. Sentences (\ref{ex:caesar:38}-\ref{ex:caesar:41}) present some examples of adjective sequencing in \ili{Dangme}.


\ea\label{ex:caesar:38}
\textbf{\ili{Dangme}:}\\
\gll {Mangoo} \textbf{{ngmlikiti}} \textbf{{gaga}}  {nↄ} {si.}\\
 mangoo ripe.\textsc{neg} long fall down.\\
\glt  ‘The unripe oval shaped mango has fallen down.’
\z


\ea\label{ex:caesar:39}
\gll {To} \textbf{{futa}} \textbf{{agbo}} {kɛ} {e} {nane} {gagaaga} {a} {laa.}\\
 sheep/goat white big with 3\textsc{sg}.\textsc{poss} leg long.\textsc{red} \textsc{def} lost\\
\glt  ‘The white fat sheep/goat with the long legs is lost.’
\z



\ea\label{ex:caesar:40}
\gll {Sɛ} \textbf{{yumu}} \textbf{{nyafii}} {ngɛ} {sukuu} {tsu} {ↄ} {mi.}\\
 Stool/chair black \textsc{dim} is school room \textsc{def} inside\\
\glt  ‘The black small stool/chair is in the classroom.’
\z



\ea\label{ex:caesar:41}
\gll {Lo} \textbf{{momo}} \textbf{{sasɛ}}  {ↄmɛ} {ngɛ} {tso} {goga} {a} {mi.}\\
 fish/meat old rotten \textsc{def}.\textsc{pl} at wood bucket \textsc{def} inside\\
\glt ‘The little rotten old meat/fish is in the wooden bucket.’
\z



\subsection{Predicative use of the adjective}\label{sec:caesar:4.3} 

The paper now examines whether the adjective can function predicatively in the two languages. \citet[106]{Dixon2004} asserts that adjectives function as copula complement usually referred to as predicative adjectives. A predicatively used adjective is one kind of subject complement. It is an adjective that modifies the subject of the sentence. In addition, they function as adjectives usually to qualify the NPs they occur with within sentences as exemplified below:


\ea\label{ex:caesar:42}
\textbf{Ga:} \\
\gll  {Asapaatere} {lɛ} {yɛ} \textbf{{fɛo.}}\\
 Shoe/sandals \textsc{def} possess beauty\\
\glt  ‘The pair of shoe/sandals is beautiful.’
\z


\ea\label{ex:caesar:43}
\textbf{\ili{Dangme}:} \\
\gll {Tokota} {a} {ngɛ} \textbf{{fɛu.}}\\
 Shoe/sandals \textsc{def} possess beauty.\\
\glt  ‘The pair of shoe/sandals is beautiful.’
\z




In the examples \REF{ex:caesar:42} and \REF{ex:caesar:43} the use of the adjectives in complement position are normally nominalised. The copula \isi{verb} \textit{yɛ} and \textit{ngɛ} are used respectively in \ili{Ga} and \ili{Dangme}. The adjective, \textit{fɛo} in \REF{ex:caesar:42} and \textit{fɛu} in \REF{ex:caesar:43} are the nominal forms of the adjectives in \ili{Ga} and \ili{Dangme}. The \ili{Ga} adjective normally does not occur in \isi{predicate} position except when prefixed with \{e-\} or remains in the same form (zero morpheme). Sometimes adjectives which have verbal equivalents are used by native speakers of both \ili{Ga} anad \ili{Dangme} languages. The constructions below indicate this phenomenon:

\ea\label{ex:caesar:44}
\textbf{\ili{Dangme}:}\\
\gll {Siadeyo} \textbf{{ka.}}\\
 Siadeyo be.tall\\
\glt ‘Siadeyo is tall.’
\z



\ea\label{ex:caesar:45}
\gll Bo ɔ \textbf{{pɔ.}}\\
cloth \textsc{def} be.wet\\
\glt ‘The cloth is wet.’
\z



\ea\label{ex:caesar:46}
\gll \ili{Tso} ɔ \textbf{{gbli.}}\\
 tree \textsc{def} be.dry\\
\glt ‘The tree is dried.’
\z



\ea\label{ex:caesar:47}
 \textbf{Ga:} \\
\gll {Aku}  \textbf{{kwɔ.}}\\
 Aku be.tall \\
\glt  ‘Aku is tall.’
\z



\ea\label{ex:caesar:48}
\gll {Atade} {lɛ} \textbf{{e-fɔ.}}\\
 dress \textsc{def} \textsc{perf}-wet\\
\glt  ‘The dress is wet.’
\z



\ea\label{ex:caesar:49}
\gll {Tso} {lɛ}  \textbf{{gbi.}}\\
 tree \textsc{def} \textsc{perf}.dry\\
\glt ‘The tree is dry.’
\z

Examples (\ref{ex:caesar:44}-\ref{ex:caesar:46}) are intransitive clauses with one core argument each in the subject position. The morphemes in bold print are intransitive predicates with their heads being verbs. All verbs which denote adjectival meanings in sentence (\ref{ex:caesar:44}-\ref{ex:caesar:46}) occupy the predicative position but modify the \isi{noun} \textit{Siadeyo} ‘a personal name’, \textit{bo} ‘cloth’ and \textit{tso} ‘tree’ in the \ili{Dangme} clauses. In a similar way, \textit{kwɔ} ‘be tall’, \textit{fɔ} ‘wet’ and \textit{gbi} ‘dry’ modify the nouns \textit{Aku,} ‘a personal name’, \textit{atade} ‘dress’ and \textit{tso} ‘tree’ in the \ili{Ga} examples in (\ref{ex:caesar:47}-\ref{ex:caesar:49}) above.

From the examples in (\ref{ex:caesar:44}-\ref{ex:caesar:49}), it was realised that some of these adjectives when used predicatively make use of verbs when there are equivalence in \ili{Ga} and \ili{Dangme}. It was also noted that in Ga, when the \isi{verb} equivalence is absent, there is the copula construction and the nominalised form of the adjective is used. When the adjectives are from the human propensity class they normally tend to be nouns. Relative clauses are also sometimes used in both languages. Below are some other verbs that could be used to express adjectival meanings.

\parbox{.4\textwidth}{
\ea 
\textbf{\ili{Dangme}:}\\
\textit{tí} ‘ to be thick’ \\
\textit{fú} ‘to be ripe’  \\
\textit{gbó} ‘to die’  \\
\textit{kle} ‘to be big’  \\
\textit{hì} ‘to be good’  \\ 
\z
}
\parbox{.4\textwidth}{
\ea 
\textbf{Ga:}\\
\textit{ti}  ‘to be thick’ \\
\textit{tsu}  ‘to be ripe’ \\
\textit{gbo} ‘to die’ \\
\textit{da} ‘to be big’ \\
\textit{hi} ‘to be good’ \\ 
\z
}




These are verbs as they take all aspect and tense markers like other verbs. \citet{Dixon2004} refers to such adjectives as verbal adjectives. These can be negated as shown in \REF{ex:caesar:52} and \REF{ex:caesar:53}. In \ili{Ga} and \ili{Dangme}, each of these takes a negative suffix. For instance:

\parbox{.4\textwidth}{
\ea\label{ex:caesar:52}
\textbf{\ili{Dangme}:}\\
\textit{ti we} ‘not thicken’   \\
\textit{fu-i} ‘not ripe’  \\
\textit{gbo we} ‘not dead’   \\
\textit{klee we} ‘not big’  
\z
}
\parbox{.4\textwidth}{
\ea\label{ex:caesar:53}
\textbf{Ga:}\\
\textit{tii-i} ‘not thicken’ \\
\textit{tsuu-u} ‘not ripe’ \\
\textit{gboo-o}  ‘not dead’ \\
\textit{daa-a} ‘not big’ 
\z
}

In Ga, there are two \isi{verb} classes, one class takes circumfixes most often to mark negativity and aspect and the other class employs suffixes. These have shown up in \REF{ex:caesar:53} where the affirmative forms of the verbs are either suffixed or circumfixed with the negative marker(s). That is, verbs of this category can also take on the full range of tense/aspect markers just like any other \isi{verb} in \ili{Ga} and \ili{Dangme}. According to \citet[19]{Dixon2004}, when an adjective occurs as an intransitive \isi{predicate}, it may take some or all the morphological processes available to verbs in the slot, thus, tense, aspects, mood, polarity, etc. An adjective in \ili{Dangme} functions directly as a modifier of a \isi{noun} in a \isi{noun} phrase, acting as a copula complement and shows morphological categories similar to those of nouns especially those related to numbers.


\section{Similarities and differences}\label{sec:caesar:4.4} 

\ili{Ga} and \ili{Dangme} both have deep level adjectives which are monomorphemic. Both languages have derived adjectives from verbs and nouns which are obtained through total or partial reduplication and an addition of a segment. In the process of reduplication, \ili{Dangme} attaches \{-i, -e, or -ɛ\} to \isi{verb} stems that end in \{-u, -o or -ↄ
\} while \ili{Ga} attaches \{-ŋ, -ru, -ra\} to \isi{verb} stems that end in \{-i, -ɛ, -u, -a\} to derive adjectives. We found that in the process of deriving adjectives from nouns in \ili{Ga} and \ili{Dangme} involves two processes. That is, either the \isi{root} \isi{noun} goes through a complete reduplication only, or reduplicates completely and then attaches a plural marker to both the \isi{root} \isi{noun} and the reduplicated parts of the \isi{noun}. \ili{Dangme} attaches the \{-hi\} plural marker while in Ga, \{-i\} is attached as discussed in examples (\ref{ex:caesar:5}-\ref{ex:caesar:7}). It is to be noted that whilst in \ili{Dangme}, non-countable nouns are not pluralised either in a part or in all the morphemes of the reduplicated form of the \isi{noun}, \ili{Ga} does attach a plural affix, \{-i\} to each of the free forms of countable and non-countable nouns reduplicated to form an adjective, with the exception of a few that do not pluralise the \isi{root} \isi{noun} of the reduplicated word as in \textit{shia-shia-i} ‘sandy’. Adjectives can also be reduplicated to show intensity and express plural number in both languages.



Adjectives in \ili{Ga} and \ili{Dangme} function attributively. It is to be noted that just as it occurs in other Ghanaian languages such as \ili{Akan}, \ili{Ewe}, Gurene, Dagbani and the like, \ili{Ga} and \ili{Dangme} adjectives and nouns have a reversed structure in the NP as compared to \ili{English}, i.e. the adjective occurs after the \isi{noun} it qualifies. It is also to be noted that while \ili{Ga} has lexical nominal forms of adjectives such as \textit{fɛɛfɛo} for the \ili{English} word ‘beautiful’, \ili{Dangme} on the other hand employs phrases such as ‘\textit{…he ngɛ fɛu’} or ‘\textit{kɛ e he fɛu’.}


\newpage
It was realised that plural adjectives are marked morphologically on the adjective and not on the \isi{noun} stem in \ili{Dangme} when the definite article is absent in a phrase, a \isi{clause} or a sentence. It is also noted that, where determiners come into play, the adjectives come in between the \isi{noun} and the determiner. However, the definite article in \ili{Ga} does not take any plural affix whiles the definite article in \ili{Dangme} have plural form. The definite article denotes number concord in the \isi{noun} and adjective in \ili{Ga} and \ili{Dangme}. When there is more than one adjective in a construction in Ga, they are all marked to express plurality. In \ili{Dangme} however, it is only the final adjective that takes the plural suffix.



The study has identified that predicative adjectives are preceded by copula verbs in both languages. While \ili{Ga} employs the copula \isi{verb}, \textit{yɛ}, \ili{Dangme} uses \textit{ngɛ}. In Ga, when the adjective is used predicatively, it is prefixed with \{e-\} or may occur in the same form. Sometimes, verbs which denote adjectival meanings are used. Where there is no verbal equivalence, the construction becomes a copula one. 


\section{Conclusion}\label{sec:caesar:5} 

This paper has examined certain morphological and syntactic features of adjectives in \ili{Ga} and \ili{Dangme}, two Kwa languages spoken in \isi{Ghana}. It considered the similarities and differences between adjectives in \ili{Ga} and \ili{Dangme}, It was observed that the membership for the adjective class in these two languages is increased by deriving adjectives from other sources such as reduplication and the derivation of nouns and verbs. Although, \ili{Ga} and \ili{Dangme} employ these three ways of forming adjectives, some of the processes vary. When adjectives are derived through total reduplication, the reduplication template takes an additional segment base on the vowel of the verbs stem to express adjectival properties in \ili{Ga} and \ili{Dangme}. It was identified that the two languages have the potential of using reduplicated verbs and nouns as adjectives to modify other nouns in the language. Adjectives can also be reduplicated to show intensity as in other languages of the world. In dealing with categories of adjectives, one can see that adjectives in \ili{Ga} and \ili{Dangme} help differentiate one nominal from the other as in other languages. 



The study established that even though \ili{Ga} and \ili{Dangme} are related to a large extent in the area of adjectives, there are identifiable differences in their morphological and syntactic properties of adjectives as discussed under the similarities and differences in \sectref{sec:caesar:4.4}. It is hoped that this study will add to the typological study on adjectives universally.

 
 
\section*{Abbreviations}
\begin{tabularx}{.45\textwidth}{>{\scshape}lQ}
1	 &first person \\
3	 &third person \\
aor  &aorist    \\
def  &definite article  \\
hab  &habitual          \\
nmlz &nominalizer \\
     \end{tabularx}
\begin{tabularx}{.45\textwidth}{>{\scshape}lQ}     
np   & {noun} phrase             \\
perf & {perfective}        \\
pl   &plural            \\
poss &possessive        \\
pst  &past tense        \\
%postp&postposition      \\
sg  &singular    \\ 
\end{tabularx}

\section*{Acknowledgements}

We are grateful to Management of the University of Education, Winneba for permitting us to attend and present this paper at ACAL45 in Kansas, USA. 

{\sloppy
\printbibliography[heading=subbibliography,notkeyword=this]
}
\end{document}
