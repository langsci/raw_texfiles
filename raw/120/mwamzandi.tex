\documentclass[output=paper,
modfonts
]{langscibook} 
\ChapterDOI{10.5281/zenodo.1251756}
\author{Mohamed Mwamzandi}
\title{A corpus study of the Swahili demonstrative position}

\abstract{Synchronic studies on Swahili adnominal demonstratives have not addressed the interplay between syntactic position and pragmatic function of these structures. This study shows how referential givenness of discourse entities may explain Swahili word order variation in Swahili adnominal demonstratives. Class 1 (animate nouns) demonstratives are examined in the two attested word orders: NP+DEM and DEM+NP. A close analysis of dataset extracted from the Helsinki Corpus of Swahili reveals that the two structures have distinct pragmatic values. The NP+DEM order is used for active topics while the DEM+NP order reactivates semiactive/inactive topics. This study reveals how the syntax-pragmatics interplay may explain distinct structures viewed as semantic equivalents by native speakers.
}

\begin{document}
\maketitle

% \todo{please check sectioning against the original}
\section{Introduction}\label{sec:mwamzandi:1}

This paper explores\isi{word order} variation in \ili{Swahili} adnominal demonstratives  via corpus analysis. The term ``adnominal demonstrative'' is used in the literature to distinguish demonstratives that co-occur with nouns from stand-alone pronominal demonstratives. While an adnominal demonstrative forms a constituent with an adjacent \isi{noun}, a pronominal demonstrative is a \isi{noun} phrase in its own right. More specifically, I analyze the pragmatic use of \ili{Swahili} demonstratives as outlined by \citet{Fillmore19751971,Fillmore1982,Fillmore1997}. Thereafter, I present a qualitative and quantitative analysis of the pre and \isi{postnominal} position of the \ili{Swahili} demonstrative. I focus on the relationship that exists between cognitive level of the hearer on discourse entities and the choice of referring expressions \citep{Chafe1987,Ariel1988,Ariel1991,Ariel2001,GundelEtAl1993}.

\ili{Swahili} has various proximal and \isi{distal} demonstrative forms that obligatorily \isi{agree} with the nominal class of the \isi{noun} they modify as exemplified in \tabref{tab:mwamzandi:1}.

\begin{table}
\begin{tabularx}{\textwidth}{XXX}
\lsptoprule
 {\bfseries Noun class} & {\bfseries Proximal Dem} & {\bfseries Distal Dem}\\
\midrule
 1 & \textit{hu-yu} & \textit{yu-le}\\
 2 & \textit{ha-wa} & \textit{wa-le}\\
 3 & \textit{hu-u} & \textit{u-le}\\
 4 & \textit{hi-i} & \textit{i-le}\\
\lspbottomrule
\end{tabularx}

\caption{Proximal and distal demonstrative forms of the first four noun classes.}
\label{tab:mwamzandi:1}
\end{table}

Notice that the hV- stem is used for the proximal demonstrative while the \textit{{}-le} stem is used for the \isi{distal} demonstratives. Further, the agreement affix varies with \isi{noun class} hence \textit{yu-} and \textit{wa-} for class 1 and 2 and \textit{u-} and \textit{i-} for class 3 and 4.

Besides the semantic distinction of \isi{distal} and proximal demonstratives, there are two demonstrative constructions that vary in their \isi{word order}: NP + DEM as seen in \REF{ex:mwamzandi:1} and DEM + NP as seen in \REF{ex:mwamzandi:2}.

\ea\label{ex:mwamzandi:1}
\gll {\ob}Msichana yule] a-li-ingia.\\
     {\db}1.girl \textsc{1.dist}.\textsc{dem} \textsc{1.sm}{}-\textsc{pst}{}-enter\\
\glt ‘That girl entered.’
\z

\ea\label{ex:mwamzandi:2}
\gll {\ob}Yule msichana] a-li-ingia.\\
     {\db}\textsc{1.dist}.\textsc{dem} 1.girl \textsc{1.sm}{}-\textsc{pst}{}-enter\\
\glt ‘That girl entered.’
\z

The \isi{distal} demonstrative \textit{yule} ‘that’ is \isi{postnominal} in \REF{ex:mwamzandi:1} but prenominal in \REF{ex:mwamzandi:2}. The general tendency in studies on the demonstrative position in Bantu is to claim that the \isi{postnominal} demonstrative \REF{ex:mwamzandi:1} is the unmarked form reserved for the basic gestural function, while the prenominal demonstrative \REF{ex:mwamzandi:2} is an innovation aimed at marking definite reference (\citealt{Ashton1944,Carstens1991,Carstens2008,Tamanji2006}). \citet{Amidu2006} points out that both the pre and \isi{postnominal} demonstrative positions as seen in \REF{ex:mwamzandi:1} and \REF{ex:mwamzandi:2} can be referential (anaphoric due to previous mention) but does not discuss the pragmatic implications of these demonstratives.

In my analysis, I first discern the adnominal demonstrative function as gestural, anaphoric, or recognitional \citep{Fillmore19751971,Fillmore1982,Fillmore1997,Himmelmann1996,Diessel1999}. Thereafter, I qualitatively and quantitatively analyze the pragmatic function of the \ili{Swahili} demonstrative position. I posit that the \isi{postnominal} demonstrative as seen in \REF{ex:mwamzandi:1} indicates that the intended referent is “active” \citep{Chafe1987}. On the other hand the prenominal demonstrative as seen in \REF{ex:mwamzandi:2} indicates that the intended referent is “semi-active” or “inactive”. “Semiactive” referents are those discourse entities reintroduced in the discourse after topic shift (change of topic) as well as discourse entities within the conversational context. Topic in this study is what an utterance is about. “Inactive” topics are (re)introduced in the discourse after a long gap of absence or are familiar to the interlocutors.

It is important to note here that \citegen{Chafe1987} activation states as outlined above do not make specific claims on the relationship that exists between activation level and forms of referring expressions. To tackle this absence of matching activation level with forms of referring expression, I invoke the Accessibility Hierarchy (\citealt{Ariel1988,Ariel1991,Ariel2001}) and the Givenness Hierarchy \citep{GundelEtAl1993} cognitive theories which associate referential choice with “referential givenness”: The awareness level of interlocutors to discourse entities (\citealt{GundelFretheim2006}). These two cognitive hierarchies rank demonstrative expressions as mid-accessibility markers. Pronouns are ranked higher than demonstratives while explicit NPs are ranked lower.

A few things on the scope and limitations of this study are worth mentioning. In this study, I examine class 1 (animate nouns) proximal (\textit{hu-yu}) and \isi{distal} (\textit{yu-le}) demonstratives. Class 1 is chosen because of the relative prominence and sustainability of animate nouns as opposed to inanimate nouns in discourse \citep{Givón1976,Givón1983}. The applicability of the results is therefore limited to class 1 demonstratives though an extension of the findings to other \isi{noun} classes is plausible. Further, this study does not look at the distribution of referential demonstratives. Referential demonstratives such as \textit{huyo} are formed by suffixing the “O” of reference to the proximal demonstrative and then deleting the final vowel of the demonstrative \citep{Ashton1944}. While the referential demonstrative is mainly used in discourse to mark definiteness, the use of a proximal/\isi{distal} demonstrative is not limited to this function (See \sectref{sec:mwamzandi:2.2.1}). Due to its difference in form and functional limitation, the distribution of the referential demonstrative is left out for future research.

The organization of the rest of the paper is as follows. In \sectref{sec:mwamzandi:2} I explain the methodology. In \sectref{sec:mwamzandi:3} I present and discuss the results of the study. \sectref{sec:mwamzandi:4} presents the conclusion and theoretical implications.

\section{Methodology}\label{sec:mwamzandi:2}

In this section, I explain \isi{extraction} of the dataset from the Helsinki Corpus of \ili{Swahili} in \sectref{sec:mwamzandi:2.1}. I then discuss how the dataset was coded in \sectref{sec:mwamzandi:2.2}.

\subsection{Extraction of the dataset from the corpus}\label{sec:mwamzandi:2.1}

The source of data in this study is the Helsinki Corpus of \ili{Swahili} (HCS) which has 14 annotated corpora. The corpora contain current newspaper articles as well as excerpts of literary texts, education and science material written in the mid to late 20th century. Due to the absence of annotations on anaphora resolution in the corpora, I limit the analysis to the HCS books (cf. \citealt{Mitkov1994}). The HCS books sub-corpus has 1,055,425 words in 71 documents. The documents are mainly \ili{Swahili} literary texts and education manuscripts.

To obtain the dataset, four queries were made in the HCS. Due to limitations associated with functionality of corpus software, the queries asked for all nouns adjacent to demonstratives whether the demonstrative and the adjacent \isi{noun} formed a syntactic unit or not. Thus, a manual postediting process aimed at eliminating all the DEM+NP collocations that did not form a syntactic unit was done. Most of these cases were \isi{ditransitive} verbs with a demonstrative adjacent to both the direct object and the oblique argument as seen in \REF{ex:mwamzandi:3}.

\ea\label{ex:mwamzandi:3}
\gll Njoo u-m-pat-i-e [kijana huyu] [maji ya kunywa].\\
     come \textsc{2sg}{}-\textsc{om}{}-get-\textsc{appl}{}-\textsc{imp} teenager \textsc{prox}.\textsc{dem} water of drinking\\
\glt ‘Come and give this teenager some water to drink.’
\z

In \REF{ex:mwamzandi:3}, the proximal demonstrative \textit{huyu} is modifying the direct object \textit{kijana} ‘teenager’ but was also displayed by the HCS concordancer as a prenominal demonstrative modifying the indirect object \textit{maji ya} kunywa ‘drinking water’. Other cases that were eliminated include pronominal identification demonstratives in which the copula introducing the demonstratum was deleted; and adnominal demonstratives from poems whose pre or \isi{postnominal} position may be driven by metrical requirements. \tabref{tab:mwamzandi:2} shows the number of adnominal demonstratives before and after disambiguation.

\begin{table}
\begin{tabularx}{\textwidth}{lSSr}
\lsptoprule
{\bfseries And-Dem} & {\bfseries And-Dems before disambiguation} & {\bfseries And-Dems after disambiguation} & {\bfseries Difference}\\
\midrule
{ Prenominal proximal} & { 133} & { 109} & { 24}\\
{ Postnominal proximal} & { 135} & { 124} & { 11}\\
{ Prenominal distal} & { 140} & { 126} & { 14}\\
{ Postnominal distal} & { 114} & { 75} & { 39}\\
{ Total} & { 522} & { 434} & { 88}\\
\lspbottomrule
\end{tabularx}

\caption{Adnominal demonstratives before and after postediting.}
\label{tab:mwamzandi:2}
\end{table}

Each of the disambiguated demonstrative expression was then displayed in its narrow context (in the HCS of \ili{Swahili} concordancer) as well as its wider context (in the original text) for contextual and statistical analysis.

\subsection{Coding the data}\label{sec:mwamzandi:2.2}

Each demonstrative expression was coded for the following variables: dem-type (proximal, \isi{distal}), dem-function (gestural, anaphoric, recognitional), dem-position (prenominal, \isi{postnominal}) and the activation state (active, semiactive, inactive). Anaphoric demonstratives were further coded for referential distance. While coding for dem-type and dem-position was straightforward after displaying the queries in their wider context, coding for the dem-function, referential distance and activation state needs further elaboration. Each of these variables is explained in turn in \sectref{sec:mwamzandi:2.2.1}, \sectref{sec:mwamzandi:2.2.2} and \sectref{sec:mwamzandi:2.2.3}.

\subsubsection{Demonstrative function}\label{sec:mwamzandi:2.2.1}

Adnominal demonstratives as referring expressions have mostly been analyzed by looking at the demonstrative function: gestural, anaphoric, and recognitional. Coding for these demonstrative functions is explained below.

‘Gestural’ here does not necessarily mean actual pointing but rather situations which need ‘pointing’ of some sort to establish reference. In the dataset there are instances such as \REF{ex:mwamzandi:4} where a cue word may indicate that the demonstrative in question is gestural.

\ea\label{ex:mwamzandi:4}
\gll {\ob}Yule bwana] u-na-mu-on-a: Mzalamo yule?\\
     {\db}\textsc{dist}.\textsc{dem} person \textsc{2sg}{}-\textsc{prs}{}-\textsc{om}{}-see-\textsc{fv} Zaramo \textsc{dist}.\textsc{dem}\\
\glt ‘Do you see that person: is he a Zamoro (ethnic community)?’
\z

In \REF{ex:mwamzandi:4}, the demonstrative expression \textit{yule bwana} ‘that person’ was coded gestural because the \isi{verb} \textit{on-a} ‘see’ draws the attention of the hearer to a potential discourse entity within the conversational context. Only first mentions of referents within conversational contexts were coded as gestural. Subsequent mentions were coded as anaphoric.

Anaphoric demonstratives track discourse entities across clauses (intra-sentential) \REF{ex:mwamzandi:5} as well as across sentences (inter-sentential) \REF{ex:mwamzandi:6} (\citealt{BotleyMcEnery2000}).

\ea\label{ex:mwamzandi:5}
\gll A-li-po-fik-a kwa [mzee Malongo], [mzee yule] a-ka-shangaa.\\
     \textsc{sm}{}-\textsc{pst}{}-when-arrive-\textsc{fv} at old.man Malongo old.man \textsc{dist}.\textsc{dem} \textsc{sm}{}-\textsc{seq}{}-surprise\\
\glt ‘When he (Kiliilo) arrived at mzee Malongo’s home, that old man (Malongo) was surprised (to see him).’
\z

\ea\label{ex:mwamzandi:6}
\ea 
\gll  U-ki-vuka bahari saba, ku-na [chewa]\textsubscript{i} mkubwa.\\
     \textsc{2sg}{}-\textsc{cond}{}-cross seas seven, \textsc{17sm}{}-be grouper big\\
\glt ‘If you cross the seven seas, there is a grouper (type of fish).’
\ex
\gll {\ob}Chewa huyu]\textsubscript{i} a-ki-vuta pumzi\\
     {\db}1grouper this \textsc{1sm}{}-\textsc{cond}{}-breath air\\
\glt ‘When this grouper is breathing…’
\z
\z

In \REF{ex:mwamzandi:5}, the NP \textit{mzee Malongo} in the matrix \isi{clause} is the antecedent of the demonstrative expression \textit{mzee yule} ‘that mzee’ in the embedded \isi{clause}. In (\ref{ex:mwamzandi:6}a), the \isi{noun} \textit{chewa} ‘grouper’ is the antecedent of the demonstrative NP \textit{chewa huyu} ‘this grouper’ in (\ref{ex:mwamzandi:6}b). Demonstratives used to track referents in intra and intersentential contexts were coded as anaphoric.

Demonstratives used recognitionally indicate common knowledge and therefore do not have a co-specification element in the surrounding situation or preceding discourse \citep{Diessel1999}. This is illustrated in \REF{ex:mwamzandi:7}.

\ea\label{ex:mwamzandi:7}
\gll {\ob}Huyu \ili{Juma}] ka-shindw-a ku-ku-tunz-a.\\
     {\db}\textsc{prox}.\textsc{dem} \ili{Juma} \textsc{sm}.\textsc{prf}{}-defeat-\textsc{fv} \textsc{inf}{}-\textsc{om}{}-take.care-\textsc{fv}\\
\glt ‘This \ili{Juma} has failed to provide for you.’
\z

In \REF{ex:mwamzandi:7} the proximal demonstrative \textit{huyu} indicates that \textit{Juma} is the man the speaker and the hearer all know. The demonstrative expression here is not anaphoric since the referring expression \textit{Juma} has no apparent antecedent in the preceding discourse. It is also not gestural because the referent \textit{Juma} is not physically present in the conversational context.

Although recognitional demonstrative expressions are overwhelmingly used in first mentions to indicate common knowledge, there are instances where subsequent mentions via a demonstrative expression may mark the referent as familiar at that point of discourse. This is illustrated in \REF{ex:mwamzandi:8}.

\ea\label{ex:mwamzandi:8}
\gll  Kumbe [yule mtu mweupe] amba-ye a-li-kuwa a-me-nusur-ik-a ku-ua-w-a na wenyeji\\
     \textsc{intj} \textsc{dist}.\textsc{dem} man white \textsc{comp}{}-\textsc{rel} \textsc{sm}{}-\textsc{aux} \textsc{sm}{}-\textsc{perf}{}-save-\textsc{stv}{}-\textsc{fv} \textsc{inf}{}-kill-pass-\textsc{fv} by natives\\
\glt ‘Alas, that white man who had escaped being killed by the natives {\dots} ’
\z

In \REF{ex:mwamzandi:8}, the referential distance between the adnominal demonstrative and its antecedent was 118 clauses. The writer is aware of the “referential problem” \citep{Auer1984} caused by topic shift and therefore adds more information to the adnominal demonstrative in form of a restrictive \isi{clause} to ensure successful identification of the referent. Following \citet[230]{Himmelmann2006}, in addition to first mention of discourse entities to indicate common knowledge, I also coded demonstrative expressions as recognitional if the gap of absence after previous mention was too long to warrant “additional anchoring or descriptive information to make the intended referent more accessible”.

\subsubsection{Referential discourse}\label{sec:mwamzandi:2.2.2}

Referential distance has been described as the most important diagnostic tool for measuring referential givenness. \citet[36]{Givón1983}, for example, explains that the effect of referential givenness on accessibility correlates with other factors such as interference from other possible discourse entities since “a high referential distance would show - all other things being equal - more interfering topics in the preceding register.” Interfering topics are other topics mentioned other than the immediate topic before its previous mention in the discourse.

Since the finite \isi{clause} is the locus for topic update, referential distance in this study is the number of finite clauses from the relevant adnominal demonstrative expression to a co-specifying explicit NP to its left (cf. \citealt{Kameyama1998,PoesioEtAl2004,TaboadaZabala2008}). The \isi{clause} as the ‘locus for topic update’ implies that it is at the \isi{clause} level that more information about the topic is added. Example \REF{ex:mwamzandi:9} illustrates how coding for referential distance with the finite \isi{clause} as the unit of analysis was done. Notice that each of the clauses in (\ref{ex:mwamzandi:9}a-c) contains new information about the topic (\textit{mjumbe} ‘messenger’). The letter \textit{u} stands for ‘utterance’ – the minimal unit of analysis in discourse, in this case, the finite \isi{clause} (cf. \citealt{GroszEtAl1995}).

\ea\label{ex:mwamzandi:9}
\ea 
(u1) mjumbe wa tano alipotakikana, (u2) alitokea bila ya ajizi. (u3) Mjumbe huyu alikuwa Ridhaa\\
\gll {\ob}Mjumbe] wa tano a-li-po-tak-ik-an-a,\\
     {\db}messenger of fifth \textsc{sm}{}-\textsc{pst}{}-when-want-\textsc{stv}{}-\textsc{recp}{}-\textsc{fv}\\
\glt When the call for the fifth messenger was made,
\ex
\gll  a-li-toke-a bila ya ajizi\\
     \textsc{sm}{}-\textsc{pst}{}-appear-\textsc{fv} without of fail\\
\glt he came forth without fail.
\ex
\gll {\ob}Mjumbe huyu] a-li-kuwa Ridhaa.\\
     {\db}messenger \textsc{prox}.\textsc{dem} \textsc{sm}{}-\textsc{pst}{}-\textsc{aux} Ridhaa\\
\glt This messenger was Ridhaa.
\z
\z

The ref-distance in \REF{ex:mwamzandi:9} was coded as 2, that is, there are two finite clauses before the subsequent mention of the topic \textit{mjumbe} in (\ref{ex:mwamzandi:9}c).

\subsubsection{Activation states}\label{sec:mwamzandi:2.2.3}

Depending on the referential distance between the adnominal demonstrative under consideration and its antecedent, the adnominal demonstrative in question was coded as active, semi-active or inactive. A question that arises under this description adapted from \citet{Chafe1987} is: What is the number of intervening utterances that qualify a discourse entity to be active/semiactive/inactive?

The intended referent of an active referent is within the immediate consciousness of the discourse participants. Thus, an adnominal demonstrative expression was coded as ‘active’ if there was an apparent antecedent in the preceding utterance as is the case in \REF{ex:mwamzandi:10}.

\ea\label{ex:mwamzandi:10}
  \ea
  \gll  Mtu wa pili ku-kut-an-a na-ye a-li-kuwa [mzee].\\
      person of second \textsc{inf}{}-meet-\textsc{recp}{}-\textsc{fv} with-\textsc{3sg} \textsc{sm}{}-\textsc{pst}{}-\textsc{aux} old.man\\
  \glt ‘The second person to meet me was an old man.’
  \ex
  \gll  {\ob}Mzee huyu] a-li-kuwa a-ki-peleka ng’ombe wake mtoni.\\
      {\db}old.man \textsc{prox}.\textsc{dem} \textsc{sm}{}-\textsc{pst}{}-\textsc{aux} \textsc{sm}{}-\textsc{ipfv}{}-take cows his river-\textsc{loc}\\
  \glt ‘This old man was taking his cows to the river.’
  \z
\z

In (\ref{ex:mwamzandi:10}a) the NP \textit{mzee} ‘old man’ is an apparent antecedent of the adnominal expression \textit{mzee huyu} ‘this old man’ in (\ref{ex:mwamzandi:10}b). The adnominal demonstrative expression \textit{mzee huyu} ‘this old man’ in (\ref{ex:mwamzandi:10}b) was therefore coded as active.

Semiactive referents in this study were of two types: situational (in conversational context) and textual (in discourse texts). Consequently, all gestural adnominal demonstratives were coded as semiactive. In discourse texts, a referent was coded as semiactive if there was an intervening topic(s) between the previous explicit mention of the antecedent NP to the adnominal demonstrative expression under consideration. This is illustrated in \REF{ex:mwamzandi:11} and \REF{ex:mwamzandi:12}.

\ea\label{ex:mwamzandi:11}
\gll Mbele ya-ngu ku-li-kuwa bado watu wawili [yule mzee] na [msichana mmoja].\\
     in.front \textsc{poss}{}-\textsc{1sg} \textsc{loc17}{}-\textsc{pst}{}-\textsc{aux} still people two \textsc{dist}.\textsc{dem} old.man and girl one\\
\glt ‘In front of me, there were still two people remaining, that old man and one girl.’
\z

\ea\label{ex:mwamzandi:12}
\gll  {\ob}Msichana huyu], a-li-ye-kuwa bado a-me-weka kitambaa\\
     {\db}girl \textsc{prox}.\textsc{dem} \textsc{sm}{}-\textsc{pst}{}-\textsc{rel}{}-\textsc{aux} still \textsc{sm}{}-\textsc{prf}{}-put handkercheif\\
\glt ‘This girl, who still had a handkercheif placed {\dots} ’
\z

In \REF{ex:mwamzandi:11}, yule mzee ‘that old man’ and \textit{msichana mmoja} ‘a girl’ are the potential topics for the following utterance. A potential topic is a referent within an utterance that can be chosen by the speaker to be the center (topic) of the next utterance (cf. \citealt{GroszEtAl1995}). In the following 4 utterances (not presented above), the \textit{mzee} ‘old man’ is established and continued as the topic. In \REF{ex:mwamzandi:12}, the demonstrative expression \textit{msichana huyu} ‘this girl’ reintroduces the girl mentioned in \REF{ex:mwamzandi:11}. The adnominal demonstrative \textit{msichana huyu} ‘this girl’ in \REF{ex:mwamzandi:12} was therefore coded as semiactive because of the interfering topic, \textit{mzee} ‘old man’.

All recognitional demonstratives were coded as “inactive” because their identification depends on retrieval of the discourse participants from the memory (see \sectref{sec:mwamzandi:2.2.1}).

\section{Results and discussion}\label{sec:mwamzandi:3}

In this section, I discuss the relevance of the demonstrative function in explaining the demonstrative position in \sectref{sec:mwamzandi:3.1}. I then discuss the relationship that exists between the demonstrative position and activation states (active, semiactive, inactive) in \sectref{sec:mwamzandi:3.2}.
% \todo{Section 3.2 was not included}

\subsection{Demonstrative function and position}\label{sec:mwamzandi:3.1}

Of the 434 adnominal demonstratives in the dataset, gestural demonstratives were 52, anaphoric 308 and recognitional 74. The frequencies of dem-type (proximal and \isi{distal}) in both the pre and \isi{postnominal} position are presented in \tabref{tab:mwamzandi:3}.

\begin{table}
\begin{tabularx}{\textwidth}{X rrr  rrr  rrr} 
\lsptoprule
& \multicolumn{3}{c}{{\bfseries Gestural}} & \multicolumn{3}{c}{{\bfseries Anaphoric}} & \multicolumn{3}{c}{{\bfseries Recognitional}}\\\cmidrule(lr){2-4}\cmidrule(lr){5-7}\cmidrule(lr){8-10}
& Pre & Post & Total & Pre & Post & Total & Pre & Post & Total\\
\midrule
{ Proximal} & 38 & 9 & 47 & 49 & 110 & 159 & 22 & 5 & 27\\
{ Distal} & 2 & 3 & 5 & 83 & 66 & 149 & 41 & 6 & 47\\
\lspbottomrule
\end{tabularx}

\caption{Dem-function and dem-position in proximal and distal demonstratives.}
\label{tab:mwamzandi:3}
\end{table}

The pragmatic value of the demonstrative position for each of the demonstrative functions will be discussed in turn.

\subsubsection{Gestural demonstratives}\label{sec:mwamzandi:3.1.1}

\tabref{tab:mwamzandi:3} above shows that the proximal gestural demonstratives in prenominal position were 38 and 9 in \isi{postnominal}. There were 2 \isi{distal} gestural demonstratives in prenominal position and 3 in \isi{postnominal}. These frequencies show that, first, the proximal demonstrative is mostly used as the deictic expression for the gestural function. The total frequency of proximal gestural demonstratives is 47 while the total frequency of the \isi{distal} gestural demonstratives is 5. This frequency difference is significant (X\textsuperscript{2} (1,N=52)=33.92, p < 0.001). Second, the gestural demonstratives have a higher frequency count in prenominal position than \isi{postnominal}. The total number of prenominal demonstratives is 40 while in the \isi{postnominal} position the total number is 12. This frequency difference is also significant (X\textsuperscript{2} (1,N=52)=15.08, p < 0.001).

The difference in the demonstrative position for the gestural demonstratives can be explained by recalling the grammaticalization of the \ili{Swahili} prenominal demonstrative to express definite reference \citep{Ashton1944,Givón1976,Carstens1991,Carstens2008}. In their paper on definite reference in \ili{English}, \citet[38]{ClarkMarshall1981}, mention \textsc{physical copresence} (presence in conversational contexts) as one of the reasons which license definite reference in \ili{English}. Based on the contextual analysis of the corpus data, I posit that the prenominal demonstratives are mostly used to point to definite referents due to \textsc{physical copresence} as seen in \REF{ex:mwamzandi:4} repeated here as \REF{ex:mwamzandi:13}.

\ea\label{ex:mwamzandi:13}
\gll  {\ob}Yule bwana] u-na-mu-on-a: Mzalamo yule?\\
     {\db}\textsc{dist}.\textsc{dem} person \textsc{2sg}{}-\textsc{prs}{}-\textsc{om}{}-see-\textsc{fv} Zaramo \textsc{dist}.\textsc{dem}\\
\glt ‘Do you see that person: is he a Zamoro (ethnic community)?’
\z

Based on the high frequency of gestural demonstratives in prenominal position, it can be deduced that the prenominal position is mainly used to mark the referents as accessible (semi-active) in conversational contexts. The examples in \REF{ex:mwamzandi:14} and \REF{ex:mwamzandi:15} further illustrate this.

\ea\label{ex:mwamzandi:14}
\gll  Huyu kondoo tu-m-pelek-e kwa Mfalme Ndevu.\\
     \textsc{prox}.\textsc{dem} sheep \textsc{1pl}{}-\textsc{om}{}-take-\textsc{imp} to King Ndevu\\
\glt ‘This sheep, let us take her to King Ndevu.’
\z

\ea\label{ex:mwamzandi:15}
\gll  Mfalme a-ki-m-pat-a kondoo huyu a-ta-furahi sana.\\
     King \textsc{sm}{}-\textsc{sbjv}{}-\textsc{om}{}-get-\textsc{fv} sheep \textsc{prox}.\textsc{dem} \textsc{sm}{}-\textsc{fut}{}-happy very\\
\glt ‘If the King gets this sheep, he will be very happy.’
\z

In \REF{ex:mwamzandi:14}, because the discourse participants are all aware of the sheep’s presence within the conversational context, the prenominal demonstrative in \textit{huyu kondoo} ‘this sheep’ marks the referent as definite due to \textsc{physical copresence}. In \REF{ex:mwamzandi:15}, however, the \isi{postnominal} position of the demonstrative marks the previously mentioned \textit{kondoo} ‘sheep’ as anaphoric. As it will be seen in \sectref{sec:mwamzandi:3.1.2}, the unmarked position for anaphoric demonstratives is \isi{postnominal}.

\subsubsection{Anaphoric demonstratives}\label{sec:mwamzandi:3.1.2}

The distribution of the 308 anaphoric demonstratives was as follows. There were 49 proximal demonstratives in prenominal position but 110 in \isi{postnominal} position (X\textsuperscript{2} (1,N=159)=23.40, p<0.001). The \isi{distal} demonstratives were 83 in prenominal position but 66 in \isi{postnominal} position, p>0.05. These results show that for the anaphoric demonstratives the proximal demonstrative has a higher frequency in \isi{postnominal} position than in prenominal. When contrasted with the \isi{distal} \isi{postnominal} demonstrative, the proximal \isi{postnominal} demonstrative frequency is also significantly higher (X\textsuperscript{2} (1,N=176)=11.00, p<0.001). In the prenominal position, the \isi{distal} demonstrative has a significantly higher frequency than the proximal demonstrative (X\textsuperscript{2} (1,N=132)=8.76, p<0.005).

In order to further explore the frequency tendencies of the anaphoric demonstratives, the referential distance of the anaphoric demonstrative expressions in the dataset was analyzed. The results are presented in \sectref{sec:mwamzandi:3.1.3}.

\subsubsection{The effect of referential distance on the anaphoric demonstrative position}\label{sec:mwamzandi:3.1.3}

In measuring the referential distance, the number of finite clauses between an adnominal demonstrative and a co-referential NP to its left was counted and recorded in a database. The raw data was then log transformed to reduce the skewness of the data distribution. After log-transformation, the Shapiro-Wilk test revealed that the data distribution for the \isi{distal} and proximal prenominal demonstratives was normal, p>0.05. The skewness of the \isi{distal} and proximal \isi{postnominal} demonstrative data was greatly reduced but not completely eliminated, p<0.05.\footnote{The statistical operations conducted in this study assume normal distribution. Log transformation of the variables enhances normal distribution, hence reducing the influence of outliers on the results \citep{Baayen2008}.}

\tabref{tab:mwamzandi:4} and \tabref{tab:mwamzandi:5} report the descriptive statistics of the demonstrative position for the raw and log-transformed data respectively. The number in parentheses is the standard deviation while the number outside the parentheses is the mean referential distance.

\begin{table}
\begin{tabularx}{.75\textwidth}{XSS}
\lsptoprule
{\bfseries \ili{Dem}\_Type} & {\bfseries Prenominal} & {\bfseries Postnominal}\\
\midrule
{ Proximal} & { 5.55 (5.39)} & { 5.25 (5.06)}\\
{ Distal} & { 7.40 (6.55)} & { 5.29 (4.35)}\\
\lspbottomrule
\end{tabularx}
\caption{Mean referential distance and standard deviation of raw data.}
\label{tab:mwamzandi:4}
\end{table}

\begin{table}
\begin{tabularx}{.75\textwidth}{XSS}
\lsptoprule
{\bfseries \ili{Dem}\_Type} & {\bfseries Prenominal} & {\bfseries Postnominal}\\
\midrule
{ Proximal} & { 1.34 (0.87)} & { 1.25 (0.92)}\\
{ Distal} & { 1.70 (0.77)} & { 1.30 (0.89)}\\
\lspbottomrule
\end{tabularx}
\caption{Mean referential distance and standard deviation of log-transformed data.}
\label{tab:mwamzandi:5}
\end{table}

The mean referential distance of the prenominal demonstratives is higher than that of \isi{postnominal} demonstratives. A non-repeated measures ANOVA with ref-distance as the dependent variable and dem-type and dem-position as the independent variables reveal that there is a significant main effect of ref-distance on dem-type (F(1,308)=6.09, p<0.05) and dem-position (F(1,308)=5.90, p<0.05). There was no significant interaction between dem-type and dem-position, p>0.05.

Further, a planned comparison using the t.test reveals that the mean ref-distance of the \isi{distal} prenominal demonstrative is higher that of the proximal prenominal demonstrative, p<0.05. The nonparametric Wilcoxon test applied to compare the median of the \isi{distal} prenominal demonstrative and the \isi{distal} \isi{postnominal} demonstrative indicates that the medians of these two vectors and their distributions are different. Hence, the mean referential distance of the \isi{distal} prenominal demonstrative is also higher than that of the \isi{postnominal} \isi{distal} demonstratives, p<0.05. However, there is no significant difference between the mean ref-distance of the proximal pre and \isi{postnominal} demonstratives. These statistics show that:

\begin{enumerate}
 \item The difference in referential distance between the proximal and \isi{distal} \isi{postnominal} demonstrative is not significant.

 \item The \isi{distal} prenominal demonstrative tends to be separated from its antecedent by longer referential distance than the \isi{distal} \isi{postnominal} demonstrative as well as the proximal pre and \isi{postnominal} demonstrative.

 \item The difference in referential distance between the proximal pre and \isi{postnominal} demonstratives is not significant.

\end{enumerate}
 

I illustrate these observations with examples from the corpus.

These statistics show that the proximal demonstrative is frequently used in \isi{postnominal} position when the referential distance is short (See example \REF{ex:mwamzandi:15}). The insignificant difference in referential distance between the proximal and \isi{distal} \isi{postnominal} demonstratives further suggests that there are cases when a \isi{distal} \isi{postnominal} demonstrative may be used after a short referential distance as seen in \REF{ex:mwamzandi:16}.

\ea\label{ex:mwamzandi:16}
\ea 
\gll  Adili a-li-po-taka ku-ingia ndani,\\
     Adili \textsc{sm}{}-\textsc{pst}{}-when-want \textsc{inf}{}-enter inside\\
\glt ‘When Adili was about to go inside (the house), ’
\ex
\gll  a-li-ona [mtu] a-me-simama mlango-ni\\
     \textsc{sm}{}-\textsc{pst}{}-see person \textsc{sm}{}-\textsc{prf}{}-stand door-\textsc{loc}\\
\glt ‘he saw a person standing at the door {\dots} ’
\ex
\gll  Adili a-li-dhani [mtu yule] a-li-kuwa bawabu.\\
     Adili \textsc{sm}{}-\textsc{pst}{}-assume person \textsc{dist}.\textsc{dem} \textsc{sm}{}-\textsc{pst}{}-\textsc{aux} security.officer\\
\glt ‘Adili thought that the person was a security officer.’
\z
\z

The referent \textit{mtu} ‘person’ introduced in (\ref{ex:mwamzandi:16}b) is continued in (\ref{ex:mwamzandi:16}c). The \isi{postnominal} position of the demonstrative in (\ref{ex:mwamzandi:16}c) marks the referent as ‘active’. The use of the \isi{postnominal} \isi{distal} demonstrative \textit{yule} ‘that’ instead of the proximal demonstrative \textit{huyu} ‘this’ has a special effect of marking the “narrative distance” (\citealt{Leonardo1987,Wilt1987}), that is, the author is narrating events from a third person’s perspective. In the third person’s perspective style of narration, the narrator is not involved in the events of the story.

Further, the results show that the \isi{distal} prenominal demonstrative is separated from its antecedent by long referential distance as illustrated in \REF{ex:mwamzandi:17}.

\ea\label{ex:mwamzandi:17}
\gll {\ob}Yule msichana] a-li-ingi-a.\\
     {\db}\textsc{dist}.\textsc{dem} girl \textsc{sm}{}-\textsc{pst}{}-enter-\textsc{fv}\\
\glt ‘That girl entered.’
\z

In \REF{ex:mwamzandi:17}, the demonstrative expression \textit{yule msichana} reintroduces the girl as the topic after 45 clauses.

It is important to mention here that most corpus generalizations are based on statistical tendencies (See \citealt{Mwamzandi2014} for more examples). In general, anaphoric proximal and \isi{distal} demonstrative are used postnominally after a short referential distance to mark the intended referent as active. Anaphoric \isi{distal} demonstrative are used prenominally after topic shift to mark the referent as semiactive.

\subsubsection{Recognitional demonstratives}\label{sec:mwamzandi:3.1.4}

The frequency of recognitional proximal demonstratives in prenominal position was 22, and 5 in \isi{postnominal} position (X\textsuperscript{2} (1,N=27)=10.70, p < 0.01). In prenominal position, the frequency of \isi{distal} demonstratives was 41, and 6 in \isi{postnominal} position (X\textsuperscript{2} (1,N=47)=26.06, p < 0.001). The difference between the recognitional demonstratives in pre and \isi{postnominal} positions is statistically significant (X\textsuperscript{2} (1,N=74)=36.54, p < 0.001). It can be inferred from the results that a demonstrative is preferred in prenominal position if used recognitionally.

Contrary to \citegen{Himmelmann1996} claim that only one of the demonstratives, mostly the \isi{distal} demonstrative, is preserved for the recognitional function across languages, both the \isi{distal} and proximal demonstratives can be used for this function in \ili{Swahili} as seen in \REF{ex:mwamzandi:18} and \REF{ex:mwamzandi:19}.

\ea\label{ex:mwamzandi:18}
\gll  {\ob}Yule mtoto wako] a-na-ye-fundisha {Chuo Kikuu},\\
     {\db}\textsc{dist}.\textsc{dem} child your \textsc{sm}{}-\textsc{prs}{}-\textsc{rel}{}-teach university\\
\glt ‘That child of yours who teaches at the university, {\dots}’
\z

\ea\label{ex:mwamzandi:19}
\gll  Hii ni kazi ya majirani zetu, hasa [huyu mjukuu wa Ndenda].\\
     9\textsc{prox}.\textsc{dem} is 9work of neighbours our especially \textsc{prox}.\textsc{dem} grandchild of \ili{Ndanda}\\
\glt ‘This is the work of our neighbours, especially this grandchild of Ndenda.’
\z

In \REF{ex:mwamzandi:18}, the speaker uses the \isi{distal} demonstrative \textit{yule} ‘that’ to signal familiarity. However, in \REF{ex:mwamzandi:19} the proximal demonstrative \textit{huyu} ‘this’ signals not only “larger situation” familiarity \citep{Hawkins1978} but also “community membership”, that is, the referent (mjukuu wa Ndenda) lives within the speaker’s neighborhood (\citealt{ClarkMarshall1981}). The use of the \isi{distal} demonstrative expression \textit{yule mjukuu wa Ndenda} in \REF{ex:mwamzandi:19} instead of the proximal demonstrative \textit{expression huyu mjukuu wa Ndeda} eliminates the “community membership” implication.


\subsection{Activation states}\label{sec:mwamzandi:3.2}

In this section, I discuss the effect of the active, semiactive and inactive activation states on the form of the adnominal expression in the following paragraphs in turn.

As mentioned earlier, subsequent mentions of referents via anaphoric demonstrative expressions if the referent was a continued topic were coded as active. \tabref{tab:mwamzandi:6} presents the frequencies of the demonstrative expressions coded as active.

\begin{table}
\begin{tabularx}{.66\textwidth}{Xrrr} 
\lsptoprule
& {\bfseries Prenominal} & {\bfseries Postnominal} & {\bfseries Total}\\
\midrule
{ Proximal} & 42 & 88 & 130\\
{ Distal} & 32 & 47 & 79\\
\lspbottomrule
\end{tabularx}

\caption{Demonstrative expressions coded as active.}
\label{tab:mwamzandi:6}
\end{table}

A few things can be said about these frequencies. First, the frequencies show that the proximal demonstrative is used more frequently than the \isi{distal} demonstrative if the activation state of the intended referent is active (X\textsuperscript{2} (1,N=209)=12.45, p < 0.001). Second, there is a higher frequency of proximal demonstrative in \isi{postnominal} position than in prenominal position if the activation state of the intended referent is active (X\textsuperscript{2} (1,N=130)=16.28, p < 0.001). Third, though insignificant, the frequency of the \isi{distal} demonstrative in \isi{postnominal} position is higher than in prenominal position if the activation state of the intended referent is active, p > 0.05. These results corroborate the statistics I presented on the effect of referential distance on demonstrative position of anaphoric demonstratives in \sectref{sec:mwamzandi:3.1.3}.

All gestural demonstratives as well as demonstratives used anaphorically after topic shift were coded as semi-active. \tabref{tab:mwamzandi:7} and \tabref{tab:mwamzandi:8} present the frequencies of the gestural and anaphoric semi-active demonstratives.

\begin{table}
\begin{tabularx}{.66\textwidth}{Xrrr}
\lsptoprule
 & {\bfseries Prenominal} & {\bfseries Postnominal} & {\bfseries Total}\\
\midrule
{ Proximal} & 38 & 9 & 47\\
{ Distal} & 2 & 3 & 5\\
\lspbottomrule
\end{tabularx}

\caption{Semiactive gestural demonstratives.}
\label{tab:mwamzandi:7}
\end{table}

\begin{table}
\begin{tabularx}{.66\textwidth}{Xrrr}
\lsptoprule
 & {\bfseries Prenominal} & {\bfseries Postnominal} & {\bfseries Total}\\
\midrule
{ Proximal} & 7 & 22 & 29\\
{ Distal} & 51 & 19 & 70\\
\lspbottomrule
\end{tabularx}

\caption{Semiactive anaphoric demonstratives.}
\label{tab:mwamzandi:8}
\end{table}

I have discussed the significance of the gestural demonstrative frequencies in pre and \isi{postnominal} position in \sectref{sec:mwamzandi:3.1.1}. Here I discuss the frequencies of the anaphoric demonstratives coded as semiactive. First, the frequencies of the anaphoric semiactive demonstratives show that the \isi{distal} demonstrative is used more frequently than the proximal demonstrative (X\textsuperscript{2} (1,N=99)=16.98, p < 0.001). Second, the frequency of proximal \isi{postnominal} demonstrative is higher than proximal prenominal demonstratives (X\textsuperscript{2} (1,N=29)=7.76, p < 0.01). Third, the frequency of \isi{distal} demonstrative in prenominal position is higher than \isi{postnominal} (X\textsuperscript{2} (1,N=70)=14.63, p < 0.001).

As for inactive activation state, all first mentions of familiar referents and subsequent mentions of discourse entities after a long referential distance via adnominal demonstrative expressions were coded as inactive. \tabref{tab:mwamzandi:9} presents the frequencies of the adnominal demonstratives coded as inactive.

\begin{table}
\begin{tabularx}{.66\textwidth}{Xrrr}
\lsptoprule
& {\bfseries Prenominal} & {\bfseries Postnominal} & {\bfseries Total}\\
\midrule
{ Proximal} & 22 & 5 & 27\\
{ Distal} & 41 & 6 & 47\\
\lspbottomrule
\end{tabularx}

\caption{Inactive adnominal demonstratives.}
\label{tab:mwamzandi:9}
\end{table}

I have also discussed the significance of these frequencies in \sectref{sec:mwamzandi:3.1.4}. In summary, the prenominal position is used more frequently than \isi{postnominal} if the referent is inactive.

\section{Conclusion and theoretical implications}\label{sec:mwamzandi:4}

The observation that pre and \isi{postnominal} demonstratives can be used as referring expressions for discourse entities has ramifications on the analysis of \ili{Swahili} demonstrative expressions in pragmatics as well as syntax. In pragmatics, it has been observed cross-linguistically that activation level of topics may be represented via different forms of referring expressions (\citealt{GundelEtAl1993,Ariel1988,Ariel1991,Ariel2001}). In \ili{Swahili}, the demonstratives co-occur with the \isi{noun} to mark different activation levels of referents. The results of this study show that \isi{postnominal} demonstratives are high accessibility markers, prenominal demonstratives are mid-accessibility markers, and prenominal demonstratives followed by a restrictive \isi{clause} are low accessibility markers.

This functional role of the demonstrative position independently motivates a syntactic analysis of the \ili{Swahili} demonstrative in pre and \isi{postnominal} position (\citealt{Carstens1991,Carstens2008}). In the \isi{postnominal} position, the unmarked order of \ili{Swahili} \isi{noun} modifiers is: N>POSS>DEM>Quantifier \REF{ex:mwamzandi:20} (cf. \citealt{Rugemalira2007}).

\ea\label{ex:mwamzandi:20}
\gll  eneo langu hili lote\\
     area \textsc{5agr}.\textsc{poss}.\textsc{1sg} \textsc{prox}.\textsc{dem} all\\
\glt ‘all this area of mine’
\z

Of these three types of modifiers, only the demonstrative may occur prenominally. The functional distinction of the demonstrative in pre and \isi{postnominal} position as observed in this study rules out the possibility of these demonstratives orders being manifestation of a single abstract syntactic structure. The different N + DEM/DEM + N constructions correspond to different discourse needs.

\section*{Acknowledgements}
I thank Dr. Laurel Stvan, Dr. Jason Kandybowicz and Dr. Jeffrey Witzel for their input on earlier versions of this paper. I also thank the ACAL reviewers for their comments and recommendations.

\section*{Abbreviations}

Unless indicating person, numbers in glosses indicate \isi{noun class}. Abbreviations follow Leipzig Glossing Rules, with the following exceptions:
\bigskip

\begin{tabular}{ll}
 \textsc{fv} & final vowel\\
 \textsc{intj} & interjection\\
\textsc{om} & object marker\\
\end{tabular}
\begin{tabular}{ll}
 \textsc{seq} & sequential\\
 \textsc{sm} & subject marker\\
 \textsc{stv} & stative
\end{tabular}
 
{\sloppy
\printbibliography[heading=subbibliography,notkeyword=this]
}
\end{document}