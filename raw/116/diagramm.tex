\begin{table}[h!]
%\centering
%\begin{longtable}{llll}
		\begin{tabularx}{\columnwidth}{XXX}
\midrule
\textbf{Phänomen} & \textbf{Quellen \hai{chrLiJi1}} (max. 53 Quellen) & \textbf{Quellen \hai{jüdLiJi1} }(max. 10 Quellen) \\
		\midrule

\hai{V24} & 41 & 10 \\

\hai{V22} & 36 & 8 \\

\isi{a-Verdumpfung} & 34  & 9 \\

\hai{V44} & 33 & 8 \\

\hai{V42} & 33 & 6 \\

\hai{V34} (mhd. \textit{iu}) & 23 & 8 \\

ü > i & 14 & 6 \\

ü > e & 10 & 5 \\

ö > e & 10 & 4 \\

ö > i & 1& 2 \\

 /u/ > /y/ & 13 & 5\\ 
 o > u & 17 & 7 \\ 
u > o & 14 & 7 \\

<ai> statt <ei>$^*$ & 33 & 9\\

<ey> statt <ei>$^*$ & 13 & – \\
<ay> statt <ei>$^*$ & 1 & – \\%\hdashline %rs! ich habe hier Schwierigkeiten, die gepunktete Linie einzusetzen, weil Overleaf Probleme mit der Verwendung der Pakete tabularx, lontable und arydshln hat; mit dem Paket arydshln habe ich Schwierigkeiten mit lontable, vor allem in part 2 am Ende; kannst du auf die Linie verzichten?
<scht> statt <st> & 13 & 5 \\
ç > ʃ & 11 & 7 \\
<ß> für <z> & 6  & 3 \\
<s> für <z> & 5 & 4 \\
<ß> für <s> & 2 & 3 \\
<z> für <s> & 1& 1 \\

<z> für <ts> & – & – \\
<s> für <ß> & – & – \\

<t> für <d> (im \isi{Anlaut}) & 6 & –\\
<t> für <d> (im Inlaut) & 5 & – \\
<p> für <b> (im \isi{Anlaut}) & 7 & 2 \\
<k> für <g> (im \isi{Anlaut}) & 7 & 1 \\
<k> für <g> (im Inlaut) & 4 & – \\
<k> für <g> (im \isi{Auslaut}) & 2 & – \\
Erhalt von germ. *-\textit{pp}- & 3 & 7 \\
<b> für <w> & 2  & 1 \\ \midrule
  \end{tabularx}
		 \caption{Häufigkeiten phonologischer Manipulationen im \hai{LiJi}}
		 \label{tblphonallhaeuf}
		 
		 \begin{footnotesize}{*Betrifft nur Abweichungen vom texteigenen System. Wenn ein Text im Deutschen z.\,B. <ey> setzt, wo im Gegenwartsdeutschen <ei> steht und dieses System auch im \hai{LiJi} aufrecht erhält, wurde das in der Analyse nicht berücksichtigt, sondern nur, wenn etwa neben <ey> im \hai{LiJi} <ay> gesetzt wurde.}\end{footnotesize}
	 \end{table}