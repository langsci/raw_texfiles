\chapter{Tonale Gesten}
\label{chap:07}

\section{Kopplung von tonalen und oralen Gesten}
\label{sec:0701}

Im Folgenden geht es um die zeitliche Synchronisation von Intonationsmustern (tonale Gesten) und \isi{Artikulation} (orale Gesten). Wie bereits in Kapitel~\ref{chap:03} aufgezeigt, lassen sich artikulatorische Gesten mit nichtlinearen Planungsoszillatoren assoziieren. Diese Oszillatoren fungieren als \isi{Taktgeber}. Sie sind paarweise gekoppelt und bilden ein multiples Netzwerk der Selbstorganisation. Ihre Schwingungsmuster erzeugen stabile \isi{Koordinationsmuster} zwischen unterschiedlichen Gesten. Dieser Ansatz bietet die Möglichkeit, tonale Alignierungsmuster von melodischer und textueller Schicht unter \isi{Kopplung} tonaler und oraler Gesten dynamisch zu modellieren. 

\subsection{Was sind tonale Gesten?}
\label{subsec:070101}

Im Autosegmental-Metrischen-Modell (AM-Modell) der Intonation sind Töne als diskrete tonale Ereignisse definiert. Intonationsmuster setzen sich dabei nicht aus Konturen der Grundtonbewegung, sondern vielmehr aus Sequenzen unterschiedlicher, diskreter Tonstufen wie Hoch- und Tiefpunkten (High und Low) zusammen (vgl. \citealt{Grice2002}, \citealt{Ladd2008}). Die Tonstufen markieren Abfolgen tonaler Zielpunkte (tonale Targets), zwischen denen die Intonationskontur linear interpoliert wird. Dabei können auch komplexe Töne abgebildet werden; ein steigender bitonaler LH-Akzent setzt sich aus einem lokalen Tiefpunkt L, der den Beginn des tonalen Anstiegs markiert, und einem Gipfel H, der dessen Ende markiert, zusammen. Den tonalen Zielspezifikationen L und H werden auf der Realisierungsebene lokale Minima (Täler) und Maxima (Gipfel) in der F0-Kontur zugeordnet.

Die \isi{Segmentale Ankerhypothese} beschreibt im Rahmen des AM-Modells die Prinzipien der Synchronisation von Intonationsmustern und Text (tonale \isi{Alignierung}). Sie basiert auf der Beobachtung, dass Wendepunkte in der F0-Kontur (tonale Zielunkte) systematisch in der Nähe von lexikalisch betonten Silben auftreten, mit denen der Ton assoziiert ist (vgl. \citealt{Arvaniti1998} für Griechisch; \citealt{Ladd1999} für Englisch; \citealt{Ladd2000} für Holländisch; \citealt{Prieto2007b} für Spanisch; \citealt{Dimperio2007} für \ili{Italienisch}; \citealt{Atterer2004} und \citealt{Mücke2008b} für unterschiedliche Varietäten des Deutschen, \citealt{Ladd2008} für einen Überblick über die \isi{Segmentale Ankerhypothese}). Werden die F0-Wendepunkte in Bezug zu segmentalen Landmarken gesetzt, beispielsweise zum akustischen Beginn des initialen Konsonanten der akzentuierten \isi{Silbe}, so ergeben sich stabile Alignierungsmuster. Diese Muster enthalten so viel Information, dass sie Ähnlichkeiten und Unterschiede in der tonalen \isi{Alignierung} zwischen Sprachen und Varietäten phonetisch abbilden können.

Abbildung~\ref{figure:0701} zeigt Alignierungseigenschaften von pränuklearen LH-Ton\-ak\-zen\-ten in verschiedenen Sprachen. Der Beginn des Anstiegs (der Zielpunkt L) ist im Englischen und Griechischen konstant mit dem linken Rand der akzentuierten \isi{Silbe} -- dem Beginn des initialen Konsonanten ${C}_{1}$ -- aligniert. Diese Alignierungsmuster für L in steigenden Tonakzenten ist häufig in der Literatur gefunden worden, darunter auch im Holländischen \citep{Ladd2000}, im Italienischen \citep{Dimperio2002}, im Spanischen \citep{Prieto2007b} und im Katalanischen \citep{Prieto2007a}. Im Gegensatz dazu zeigt sich im Deutschen, dass pränuklear steigende LH-Akzente später ansteigen. Im Norddeutschen (hier eine Niederfränkische Sprachregion, Düsseldorf) tritt der Tiefpunkt L zeitlich erst in der Mitte des initialen Konsonanten, ${C}_{1}$, und im Süddeutschen (Wienerisch) sogar erst im darauffolgenden \isi{Vokal}, ${V}_{1}$, auf.

\begin{figure}
	\includegraphics[width=\textwidth]{figures/7-1_Alignierung_nach_Ladd.png}
	\caption{Schematische Übersicht über die Alignierungseigenschaften von pränuklearen LH-Akzenten für das Griechische und Englische (adaptiert nach \citealt{Atterer2004}) und durch deutsche Varietäten aus Düsseldorf (Norddeutsch) und Wien (Süddeutsch) ergänzt (vgl. \citealt{Mücke2008b}). C und V stilisieren akustisch definierte Segmente.}
	\label{figure:0701}
\end{figure}

\newpage
Bei der Beschreibung der Synchronisation von tonalen Konturen und der segmentalen Ebene ist die \isi{phonologische Assoziation} nicht mit der phonetischen \isi{Alignierung} gleichzusetzen.

So beziehen sich \isi{phonologische Assoziation} und phonetische Realisierung auf zwei verschiedene Ebenen. Es besteht mehr systematische Variation im phonetischen Signal, als sich in der phonologischen Spezifikation im Rahmen des AM-Modells abbilden lässt.

Die Koordination von Ton und Text lässt sich auch im Rahmen der Artikulatorischen Phonologie beschreiben. Hierbei sind die Grundeinheiten sowohl für die segmentale als auch für die tonale Ebene artikulatorische Gesten, die als invariante funktionale Bewegungsintervalle die momentane Ausformung des Vokaltraktes definieren. Gesprochene Sprache lässt sich beschreiben als wechselnde Konstellation von diskreten Gesten, deren besondere Eigenschaft es ist, dass sie miteinander überlappen (Kapitel~\ref{chap:01} und \ref{chap:02}). Regularitäten und Variabilitäten des intergesturalen Timings \citep{Byrd1994,Byrd1996a,Byrd1996b,Cho2001,Bombien2010} können als selbstorganisierendes Netzwerk aus paarweise gekoppelten Oszillatoren als Kopplungsgraphen modelliert werden \citep[u.a.][]{Browman2000, Saltzman2000, Nam2003, Nam2007a, Goldstein2007a, Goldstein2009}.

In diesem Netzwerk ist, wie in Kapitel~\ref{chap:03} ausgeführt, jede \isi{Geste} mit einem \isi{Oszillator} (einem \isi{Taktgeber}) assoziiert, der paarweise mit anderen Oszillatoren eingekoppelt ist. Daraus ergibt sich ein Netzwerk konkurrierender Zielspezifikationen mit intrinsischen (in-phase / synchron und anti-phase / sequentiell; vgl. \citealt{Turvey1990}) und nicht-intrinsischen Phasenmodi (exzentrische Phasen).

\largerpage
Abbildung~\ref{figure:0702} fasst noch einmal die wichtigsten Kopplungsgraphen für Silbenstruktur zusammen. Im Verlauf dieses Kapitels werden diese Kopplungsgraphen auf die Koordination von Tönen und oralen Konstriktionsgesten übertragen. In der Silbenkopplungshypothese werden vor allem zwei intrinsische Modi verwendet, um die Organisation von Gesten in Silben abzubilden \citep{Goldstein2009, Nam2009b}. Dabei wird die stabilere In-Phase für die Beschreibung der Onset-Nukleus-Relation und die Anti-Phase für die Nukleus-Koda-Relation verwendet. Bei der Onset-Nukleus-Relation in Abbildung~\ref{figure:0702}~(a) besteht keine \isi{Phasenverschiebung} zwischen der konsonantischen und der vokalischen \isi{Geste} (In-Phase; \isi{Phasenverschiebung} 0°), so dass beide Gesten gleichzeitig starten. Da vokalische Gesten eine geringere \isi{Ausführungsgeschwindigkeit} als Konsonanten haben, sind auf der akustischen Oberfläche beide wahrnehmbar. Bei der Nukleus-Koda-Relation in Abbildung~\ref{figure:0702}~(b) besteht eine \isi{Phasenverschiebung} der gestischen Aktivierung von 180°, so dass die vokalische und die konsonantische \isi{Geste} nacheinander aktiviert werden.

\begin{figure}
% 	\includegraphics[width=.8\textwidth]{figures/7-2_Cluster_inAP.png}
	
\begin{tikzpicture} 
\node[minimum width=3cm, minimum height=.7cm, inner sep=0pt, draw, fill=black!10] (rect1) {V};
   
\node[minimum width=3cm, minimum height=.7cm, inner sep=0pt, draw, fill=black!10] (rect2) [below=of rect1] {V};

\node[minimum width=3cm, minimum height=.7cm, inner sep=0pt, draw, fill=black!10] (rect3) [below=of rect2] {~~~~~~~~~~V};


\node[minimum width=1.4cm, minimum height=.5cm, inner sep=0pt, draw, fill=white] (inset1) [below=of rect1,yshift=1.25cm,xshift=-.8cm] {C};
\node[minimum width=1.4cm, minimum height=.5cm, inner sep=0pt, draw, fill=white] (inset2) [below=of rect2,yshift=1.25cm,xshift=1.5cm] {C};
\node[minimum width=1.4cm, minimum height=.5cm, inner sep=0pt, draw, fill=white] (inset3) [below=of rect3,yshift=1.5cm,xshift=-1.5cm] {C};
\node[minimum width=1.4cm, minimum height=.5cm, inner sep=0pt, draw, fill=white] (inset4) [below=of rect3,yshift=1.2cm,xshift=-.5cm] {C};

 

\node[circle,draw, minimum size=3mm] (C1) [right=of rect1,xshift=5mm] {C};
\node[circle,draw, minimum size=3mm] (V1) [right=of C1   ] {V};

\node[circle,draw, minimum size=3mm] (V2) [right=of rect2,xshift=5mm]   {V};
\node[circle,draw, minimum size=3mm] (C2) [right=of V2]  {C};

\node[circle,draw, minimum size=3mm] (C3)  [right=of rect3,xshift=5mm]  {C};
\node[circle,draw, minimum size=3mm] (C4)  [right=of C3   ]  {C};
\node (V3) [circle,draw, minimum size=3mm,below = of $(C3)!0.5!(C4)$,yshift=4mm]  {V};

\draw 	      (C1) -- (V1);
\draw[dashed] (V2) -- (C2);
\draw[dashed] (C3) -- (C4);
\draw         (V3) -- (C3);
\draw         (V3) -- (C4);

\node (Gestenpartitur) [above = of rect1,yshift=-.8cm] {Gestenpartitur};
\node (Kopplungsgraph) [right = of Gestenpartitur,xshift=5mm] {Kopplungsgraph};

\node () [left = of rect1] {(a)};
\node () [left = of rect2] {(b)};
\node () [left = of rect3] {(c)};

\end{tikzpicture}
	
	
	\caption{Gestenpartitur und Kopplungsgraph für CV (In-Phase, Phasenverschiebung 0°), VC (Anti-Phase, Phasenverschiebung 180°) und CCV (konkurrierende Kräfte der Zielspezifikationen). Graue Linie = In-Phase; schwarz gestrichelter Pfeil = Anti-Phase; schwarz durchgezogener Pfeil = exzentrische Phase (vgl. Kapitel~\ref{sec:0302}, Tabelle~\ref{table:0301}).}
	\label{figure:0702}
\end{figure}

Bei verzweigenden Onsets, CCV, spielt die Anti-Phase eine Rolle \citep{Nam2009b}. Hier gibt es einen Wettbewerb zwischen den beiden konsonantischen Gesten ${C}_{1}$ und ${C}_{2}$, denn beide konkurrieren um eine In-Phase-Relation mit dem \isi{Vokal}. Um zu verhindern, dass ${C}_{1}$ und ${C}_{2}$ gleichzeitig aktiviert werden, sind sie miteinander zusätzlich mit Anti-Phase-Relation (auch \isi{exzentrische Phase} mit einer \isi{Phasenverschiebung} von 90° möglich als Variante einer out-of-phase Beziehung, vgl. Kapitel~\ref{sec:0302}) gekoppelt, die bewirkt, dass als Kompromiss aus den konkurrierenden Kräften der Zielphasen ${C}_{1}$ früher und ${C}_{2}$ später gestartet werden. Dabei entsteht eine schuppenartige \isi{Überlappung} zwischen ${C}_{1}$ und ${C}_{2}$ \citep[\isi{C-Center Effekt};][]{Browman1988, Browman2000, Bombien2010, Gao2009, Goldstein2007a, Goldstein2009, Hermes2008b, Hermes2008b, Nam2007a, Nam2009b, Shaw2009}.

Wie verhalten sich \isi{Koordinationsmuster} zwischen oralen und tonalen Gesten? In einem dynamischen System sind tonale Gesten -- die orale Gesten -- als koordinierte Bewegungseinheiten des Vokaltraktes definiert. Anders als bei oralen Gesten beziehen sich die tonalen Zielspezifikationen auf den F0-Raum bzw. F0-Verlauf \citep[vgl.][]{Gao2009, Mücke2009b, Niemann2011, Mücke2012}. Ein steigender H-\isi{Tonakzent}, beispielsweise eine Hochtongeste, beinhaltet eine F0-Bewegung in Richtung eines tonalen Zielpunktes. Wird das tonale Gestenintervall aktiviert, beginnt auch die Grundfrequenz sich in Richtung der Zielspezifikation zu bewegen. Sie steigt an, vgl. Abbildung~\ref{figure:0703}~(obere Ebene). Bei einem hypothetischen steigenden LH-Akzent, der sich aus einer \isi{Tieftongeste} (L-\isi{Geste}) und einer Hochtongeste (H-\isi{Geste}) zusammensetzt, fällt die Aktivierung der H-\isi{Geste} mit der Deaktivierung der vorangehenden L-\isi{Geste} zusammen, wobei der Beginn der L-\isi{Geste} im F0-Verlauf unklar ist.

Bei tonalen Gesten handelt es sich um gestische Aktionseinheiten. Jedes \isi{Aktivierungsintervall} hat einen Beginn und ein Ende. Das ist ein wichtiger Unterschied zum AM-Modell, bei dem tonale Zielpunkte als punktuelle Ereignisse ohne zeitliche Ausdehnung definiert sind, schematisiert als L und H in Abbildung~\ref{figure:0703}~(untere Ebene).

\begin{figure}
	\includegraphics[width=\textwidth]{figures/7-3_Tonale_Geste_in_APundAM.png}
	\caption{Analyse von steigenden LH-Tonakzenten: Töne als gestische Aktionseinheiten / Intervalle in der Artikulatorischen Phonologie (oben) und Töne als tonale Zielpunkte / Ereignisse im Autosegmental-Metrischen-Modell (unten).}
	\label{figure:0703}
\end{figure}

\largerpage
Tonale Gesten sind wie orale Gesten mit gekoppelten Oszillatoren als \isi{Taktgeber} assoziiert, so dass sich \isi{Koordinationsmuster} zwischen oralen und tonalen Gesten aus deren Kopplungsgraphen im multiplen Netzwerk ableiten. Paarweise Kopplungen in den verschiedenen Phasenmodi (In-Phase, Anti-Phase, \isi{exzentrische Phase}) sind somit nicht nur zwischen tonalen Gesten oder zwischen oralen Gesten, sondern auch zwischen tonalen und oralen Gesten möglich. So nimmt \citet{Gao2006} und \citet{Gao2009} beispielsweise für den Ton 1 (Ton 1–H, high level) im \ili{Mandarin} eine H-\isi{Geste} an. Die H-\isi{Geste} koppelt sie dann mit oralen Konstriktionsgesten in CV-Silben (Beispiel: [ma]). Da in der kinematischen Dimension die konsonantische, vokalische und tonale \isi{Geste} zeitversetzt in der Reihenfolge C-V-T (consonant, vowel, tone) starten, schlussfolgert sie, dass sich die tonale H-\isi{Geste} wie ein Konsonant verhält. Wie bei Konsonantenclustern in verzweigenden Silbenonsets (vgl. Abbildung~\ref{figure:0702}~(c)) ergeben sich zwischen Konsonant, \isi{Vokal} und H-Ton konkurrierende Zielspezifikationen: C und H sind In-Phase mit dem \isi{Vokal} und zusätzlich Anti-Phase (oder einer näher zu spezifizierenden exzentrischen Phase) miteinander gekoppelt, vgl. \isi{Kopplungsgraph} in Abbildung~\ref{figure:0704}.

\begin{figure}[hbtp]
% 	\includegraphics[width=3cm,height=3cm]{figures/7-4_CVH_Ton.png}

\begin{tikzpicture}	
\node[circle,draw, minimum size=3mm] (C)    {C};
\node[circle,draw, minimum size=3mm] (H)  [right=of C]  {H};
\node (V) [circle,draw, minimum size=3mm,below = of $(C)!0.5!(H)$,yshift=4mm]  {V};
\node()[above = of V, yshift=.1cm]{\LARGE CVH};

\draw[dashed] (C) -- (H);
\draw         (V) -- (C);
\draw         (V) -- (H);
\end{tikzpicture} 
	\caption{Hypothetischer Kopplungsgraph für die Koordination von konsonantischer und vokalischer Geste, CV, und tonaler Hochtongeste, H, mit konkurrierende Kräfte der Zielspezifikationen. Durchgezogene Linie = In-Phase; gestrichelte Linie = Anti-Phase \citep[adaptiert von][]{Gao2009}.}
	\label{figure:0704}
\end{figure}

\subsection{Lexikalische Töne im Mandarin}
\label{subsec:070102}

\citet{Gao2006} und \citet{Gao2009} untersuchte die zeitliche Koordination von tonalen und oralen Gesten mit Hilfe taktgebender Oszillatoren (In-Phase-, Anti-Phase-Beziehungen) bei lexikalischen Tönen im \ili{Mandarin} Chinesisch. Bei der \isi{Modellierung} von lexikalischen Tönen im \ili{Mandarin} geht \citet{Gao2006} und \citet{Gao2009} davon aus, dass sich Töne in ihrer Koordination wie Konsonanten verhalten. Sie ordnet die jeweiligen Töne der zeitlichen Organisation von Onset-Nukleus-Relationen einer \isi{Silbe} zu. Ihr Modell ist wegweisend für die Darstellung lexikalischer Töne als artikulatorische Gesten. Dennoch kann mit diesem Ansatz nicht der Abbildung von starken kontextbedingten Tonvariationen, wie Ton 3 Sandhi, Rechnung getragen werden. Ihr Ansatz wurde von \citet{Hsieh2011} erweitert, der Töne auch der zeitlichen Organisation von Nukleus-Koda-Relationen zuordnet.

Abbildung~\ref{figure:0705} zeigt die Repräsentation des F0-Verlaufs von Ton 1 (high-level), Ton 2 (rising), Ton 3 (low-falling) und Ton 4 (high-falling), schematisiert nach den Daten von \citet{Gao2009}. Für ihre Studie hatte sie Zielwörter mit der Struktur CV und CVC -- wie [ma] oder [man] -- verwendet. Die Trägersätze waren so konzipiert, dass die Zielwörter in einer Umgebung mit nicht konfligierenden tonalen Targets auftraten, beispielsweise Ton 1 (H) wurde in der tonalen Umgebung HL---H---LH elizitiert.

\begin{figure}[p]
	\includegraphics[width=\textwidth]{figures/7-5_Mandarin_Konturen.png}
	\caption{Schematischer Grundtonverlauf im Mandarin nach den Daten von \citet{Gao2009}.}
	\label{figure:0705}
\end{figure}

\begin{figure}[p]
% 	\includegraphics[width=\textwidth]{figures/7-6_Tonale_Gesten_Mandarin.png} 
	\begin{tikzpicture}	
\node[circle,draw, minimum size=3mm] (C1)    {C};
\node[circle,draw, minimum size=3mm] (T1)  [right=of C]  {T};
\node (V1) [circle,draw, minimum size=3mm,below = of $(C1)!0.5!(T1)$,yshift=4mm]  {V};
\node () [above = of $(C1)!0.5!(T1)$,yshift=-4mm]  {\LARGE CVT}; 

\draw[dashed] (C1) -- (T1);
\draw         (V1) -- (C1);
\draw         (V1) -- (T1);


\node[circle,draw, minimum size=3mm] (C2)  [right=of T1]  {C};
\node[circle,draw, minimum size=3mm] (T2)  [right=of C2]  {T};
\node[circle,draw, minimum size=3mm] (T3)  [right=of T2]  {T};
\node[circle,draw, minimum size=3mm] (V2)  [below=of T2,yshift=7mm]  {V}; 
\node () [above = of $(C2)!0.5!(T3)$,yshift=-4mm]  {\LARGE CVT}; 

\draw[dashed] (C2) -- (T2);
\draw[dashed] (T2) -- (T3);
\draw         (V2) -- (C2);
\draw         (V2) -- (T2);
\draw         (V2) -- (T3);

\node  [minimum width=2.5cm, minimum height=.7cm, inner sep=0pt, draw ] (rH1) [below=of C1,  anchor=west, xshift=-.25cm, yshift=-1cm] {H};
\node  [minimum width=2.5cm, minimum height=.7cm, inner sep=0pt, draw ] (rH2) [below=of rH1] {H};
\node  [minimum width=1cm, minimum height=.7cm, inner sep=0pt, draw ] (rL2) [above=of rH2,xshift=-.75cm,yshift=-1cm] {L};
\node  [minimum width=2.5cm, minimum height=.7cm, inner sep=0pt, draw ] (rH3) [below=of rH2,yshift=.5cm] {H};


\node (Ton1) [left=of rH1,xshift=8mm] {Ton 1};
\node (Ton2) [left=of rH2,xshift=8mm,yshift=.4cm] {Ton 2};
\node (Ton3) [left=of rH3,xshift=8mm] {Ton 3};
\node (Ton4) [right=of rH1,xshift=-5mm] {Ton 4};
\node  [minimum width=1.25cm, minimum height=.7cm, inner sep=0pt, draw ] (rH4) [right=of Ton4,xshift=-1cm] {H};
\node  [minimum width=1.25cm, minimum height=.7cm, inner sep=0pt, draw ] (rL2) [right=of rH4,xshift=-1cm] {L};
\end{tikzpicture} 
	\caption{Unten: Gesturale Repräsentation für vier Töne im Mandarin (H = high tone gesture, L = low tone gesture). Oben: Zugehörige Kopplungsgraphen (durchgezogene Linien für In-Phase, gestrichelte Pfeile für Anti-Phase) \citep[adaptiert von][]{Gao2009}.}
	\label{figure:0706}
\end{figure}
Abbildung~\ref{figure:0706} zeigt nun die Transformation des F0-Verlaufs auf die entsprechenden Repräsentationen als tonale Gesten (\isi{Gestenpartitur}, unten) und deren zugehörige Kopplungsgraphen (oben) nach \citet{Gao2009}. Analog zu oralen Konstriktionsgesten zeigt die Partitur für die tonalen Gesten \isi{Aktivierungsintervalle} vom Start bis zum Erreichen des Targets der damit verbundenen tonalen \isi{Bewegungsaufgabe}. Entgegen den oralen Konstriktionsgesten ist das \isi{Target} jedoch nicht im Vokaltrakt sondern im F0-Raum spezifiziert. Es wird von zwei distinktiven tonalen Gesten ausgegangen, die miteinander kombiniert werden können: die Hochtongeste (high tone gesture) und die \isi{Tieftongeste} (low tone gesture). Beide Gesten steuern dieselbe \isi{Traktvariable} im Task-Dynamic-Modell an, aber sie unterscheiden sich in ihren Targets.


Für Ton 1---H  (high-level) verwendet \citet{Gao2009} bei den Gestenpartituren (Abbildung~\ref{figure:0706}~(unten)) das \isi{Aktivierungsintervall} für eine Hochtongeste. Für Ton 2---LH (rising) kombiniert sie eine tiefe und eine hohe Tongeste, wobei sie für die Hochtongeste eine geringere \isi{Eigenfrequenz} und somit eine geringere \isi{Steifheit} annimmt. Somit erreicht die Hochtongeste im Vergleich zur \isi{Tieftongeste} ihr \isi{Target} später -- ähnlich wie wir es von Vokalen und Konsonanten kennen. Das entspricht dem perzeptiven Eindruck von Ton 2---LH, dass der kurze frühe Abfall des Grundtons nur eine aktive Vorbereitung für die kommende Steigung ist. Für Ton 3---L (low-falling) nimmt sie eine \isi{Tieftongeste} an und für Ton 4---HL (high-falling) wieder eine Kombination aus Hoch- und \isi{Tieftongeste}, diesmal aber nicht überlappend sondern sequenziell mit gleicher \isi{Eigenfrequenz}. Bei Ton 4 entspricht die Sequenz HL wieder dem perzeptiven Eindruck, dass der kurze frühe Anstieg nur eine Vorbereitung für den kommenden Fall des Grundtons ist.

Für die kanonische Form von Silben mit nur einem Ton (Ton 1---H und Ton 3---L) zeigt sie im kinematischen Signal, dass die oralen Konstriktionsgesten CV und die Tongesten T in der Reihenfolge C-V-T starten. Dabei findet sie vergleichsweise große zeitliche Abstände zwischen den Onsets der jeweiligen Bewegungen: Der Konsonant wird im Durchschnitt ca.~$50$~ms vor dem \isi{Vokal} aktiviert und Ton startet entsprechend ca.~$50$~ms nach dem \isi{Vokal}. Gao schlussfolgert aus diesen Ergebnissen, dass sich die Tongeste wie eine konsonantische \isi{Geste} verhalte. Konsonantische und tonale \isi{Geste} bildeten gemeinsam mit dem \isi{Vokal} konkurrierende Zielspezifikationen. Konsonant und Ton seien dann beide im In-Phase-Modus mit dem \isi{Vokal} und untereinander in Anti-Phase gekoppelt. Diese Koordination ist in Form von Kopplungsgraphen in Abbildung~\ref{figure:0706}~(oben links) skizziert. Es ergibt sich aus den konkurrierenden Zielspezifikationen eine Stabilität des Zentrums von Konsonant und Ton relativ zum \isi{Vokal}: Als Konsequenz bewegt sich der Konsonant -- relativ zum \isi{Vokal} -- nach links und der Ton nach rechts.

\largerpage
Der \isi{Kopplungsgraph} in Abbildung~\ref{figure:0706}~(oben links) ist ebenfalls für Ton 2---LH gültig. So symbolisiert in diesem Graph der Ton T entweder eine Hoch- oder \isi{Tieftongeste} oder eine Kombination aus beiden. Beide Gesten, Tief- und Hochtongeste, starten synchron und bilden gemeinsam mit dem Konsonanten eine Zentrumskoordination (c-center coordination), aber die Hochtongeste hat eine geringere \isi{Ausführungsgeschwindigkeit} als der Tiefton.

Die einzige Ausnahme bildet Ton 4---HL. Hier nimmt \citet{Gao2006} und \citet{Gao2009} eine sequentielle \isi{Kopplung} zwischen den Tönen an (Abbildung~\ref{figure:0706}~(oben rechts)). Somit gibt es in diesem \isi{Kopplungsgraph} drei konkurrierende Zielspezifikationen, und zwar zwischen C und V, zwischen  ${T}_{1}$ und V sowie zwischen  ${T}_{2}$ und V. Als Konsequenz startet im Output des Modells der \isi{Vokal} und ${T}_{1}$ mehr oder weniger zeitglich, da sich  ${T}_{1}$ im Zentrum der ${C}-{T}_{1}-{T}_{2}$-Verbindung befindet.

\subsection{Kontextbedingten Variation bei Tönen}
\label{subsec:070103}

In der Analyse von Gao ist Ton 3 (low-falling) in der folgenden tonalen Umgebung eingebettet: LH---L---HL. Somit folgt auf Ton 3 ein Hochton. \citet{Hsieh2011} diskutiert die \isi{Modellierung} von \citet{Gao2009} und merkt an, dass sich Ton 3 in verschiedenen Kontexten unterschiedlich verhalte \citep[vgl. auch][]{Xu1997,Cho2011}. Demnach reiche eine Analyse als einzelne \isi{Tieftongeste} für die quantitative Abbildung der Variationen von Ton 3 nicht aus. Vielmehr bestehe Ton 3 aus zwei tonalen Gesten, einer Tief- und Hochtongeste, die in verschiedenen Kontexten zu unterschiedlichem Output führten.

\citet{Hsieh2011} beschreibt die folgenden drei Variationen für Ton: Full~Tone~3, Low~Tone~3 und Sandhi~Tone~3 (Abbildung \ref{figure:0707}).

\begin{figure}
	\includegraphics[width=.8\textwidth]{figures/7-7_Ton3_Sandhi.png}
	\caption{Schematische Repräsentation des F0-Verlaufs für Varianten von Ton~3 im Mandarin nach den Daten von \citet{Hsieh2011}.}
	\label{figure:0707}
\end{figure}

Bei Ton~3 (full) ist die \isi{Tieftongeste} in-phase und die Hochtongeste in Anti-Phase mit dem \isi{Vokal} gekoppelt. Somit verhält sich die \isi{Tieftongeste} wie ein Onset-Konsonant und die Hochtongeste wie ein Koda-Konsonant (vgl. Abbildung~\ref{figure:0708}). Als Konsequenz ergibt sich ein relativ später Anstieg von \isi{F0} relativ zu \isi{Silbe}.

\begin{figure}[t]
% 	\includegraphics[width=4.64cm,height=2.81cm]{figures/7-8_Kopplungsgraph_Ton3.png}
	
	\begin{tikzpicture}	
\node[circle,draw, minimum size=3mm] (C)                {C};
\node[circle,draw, minimum size=3mm] (L)  [right=of C]  {L};
\node[circle,draw, minimum size=3mm] (H)  [right=of L]  {H};
\node[circle,draw, minimum size=3mm] (V)  [below=of L]  {V}; 

\draw[dashed] (C) -- (L);
\draw         (V) -- (C);
\draw         (V) -- (L);
\draw[dashed] (V) -- (H);
	\end{tikzpicture} 

	\caption{Kopplungsgraph für Ton~3 (full) und Ton~3 (low) nach \citealt[891]{Hsieh2011}; durchgezogene Linien = In-Phase, gestrichelte Linien = Anti-Phase. Die tonale H Geste verhält sich wie ein Konsonant in der Koda und als Resultat ergibt sich ein später Anstieg des F0.}
	\label{figure:0708}
\end{figure}
\begin{figure}[t]
%   \includegraphics[width=\textwidth]{figures/7-9_Ueberlappung_HochtingesteHsie.png}
  
  \begin{tikzpicture}	
	  
  \node[circle,draw, minimum size=3mm] (V1)                 {V};
  \node[circle,draw, minimum size=3mm] (V2) [right=of V1]   {V};
  \node[circle,draw, minimum size=3mm] (L)  [above=of V1,yshift=5mm]  {L};
  \node[circle,draw, minimum size=3mm] (H1)  [right=of L,xshift=-2mm, yshift=-1mm]  {H};
  \node[circle,draw, minimum size=3mm] (H2)  [above=of V2,yshift=-.5cm]  {H};
  \node  () [right=of H1,yshift=-5mm] {\parbox[t]{5cm}{\raggedright H aufgrund der Überlappung nur auf σ2 hörbar. σ1 klingt tief.}};

  \draw         (V1) -- (L);
  \draw[dashed] (V1) -- (H1);
  \draw[dashed] (V1) -- (V2);
  \draw         (V2) -- (H2);
  \node  () [below=of V1,yshift=8mm] {\parbox[t]{1cm}{\centering σ1\\Ton 3}};
  \node  () [below=of V2,yshift=8mm] {\parbox[t]{1cm}{\centering σ2\\Ton 2}};


  \end{tikzpicture}
  \caption{Überlappung der finalen Hochtongeste H von Ton 3 (low) mit nachfolgender Hochtongeste in Ton 1. Durchgezogene Linien = In-Phase, gestrichelte Linien = Anti-Phase. Darstellung für V und T \citep[nach][891]{Hsieh2011}.}
  \label{figure:0709}
\end{figure}

Die Realisation von Ton~3 (low) ist das Ergebnis einer nachfolgenden Hochtongeste, die sich mit der finalen Hochtongeste von Ton~3 überlappt. Die Variation ergibt sich also -- ähnlich wie bei der \isi{Assimilation} von oralen Gesten -- aus einem quantitativen Grad der gesturalen \isi{Überlappung} (hier von Koda H und nachfolgendem \isi{Onset} H, vgl. Abbildung~\ref{figure:0709}). Das ist deshalb möglich, weil die Hochtongeste H von Ton~3 in Anti-Phase mit dem \isi{Vokal} gekoppelt ist und somit eine Nukleus-Koda-Relation bildet. Sie fällt in der beschriebenen Umgebung außerhalb der mit Ton~3 assoziierten \isi{Silbe} und es wird auf dieser \isi{Silbe} nur ein tiefer Ton perzipiert.

\largerpage
Bei Ton~3 Sandhi wird Ton~3, der von einem weiteren Ton~3 gefolgt wird, wie ein Ton~2 realisiert. Nach \citet{Hsieh2011} kommt es bei Ton~3~Sandhi zu einer Änderung im \isi{Kopplungsgraph} zwischen den Phasenbeziehungen des „Koda“-Tons H. Folgt auf Ton~3 ein weiterer Ton~3, so wird eine zusätzliche In-Phase-\isi{Kopplung} zwischen \isi{Tieftongeste} L und Hochtongeste H angenommen. Als Konsequenz ergibt sich ein früherer Anstieg der F0-Bewegung in der Zielsilbe, der den perzeptiven Eindruck von Ton~2 erweckt. \citet{Hsieh2011} schlussfolgert, dass eine solche \isi{Phasenverschiebung} vom weniger stabilen Anti-Phase-Modus zwischen L und H zum stabilen In-Phase-Modus diachron das Ergebnis einer Destabilisierung des Anti-Phase-Modus unter Erhöhung der \isi{Artikulationsrate} sein könnte: Eine ansteigende \isi{Artikulationsrate} begünstigt Synchronie zwischen den Oszillatoren, die mit den Tongesten assoziiert sind.

\section[Postlexikalische Töne: Tonale Anstiege (Katalanisch -- Deutsch)]{Postlexikalische Töne: Tonale Anstiege im Katalanischen und Deutschen}
\label{sec:0702}

Wie können Tonakzente in Nicht-Tonsprachen wie \ili{Deutsch}, Englisch, \ili{Italienisch}, Spanisch oder \ili{Katalanisch} modelliert werden? Eine Möglichkeit besteht darin, dass Tonakzente sich ähnlich wie Töne im \ili{Mandarin} verhalten und zusammen mit konsonantischen und vokalischen Gesten konkurrierende Zielspezifikationen bilden. Das würde bedeuten, dass tonale Gesten die internen Silbenkopplungsgraphen beeinflussen (Konsonanten würden dann \isi{kinematisch} vor dem \isi{Vokal} starten und nicht mehr gleichzeitig mit ihm aktiviert werden). Es ist aber auch möglich, dass sich das Kopplungsnetzwerk bei Tonakzenten nach anderen Gesetzmäßigkeiten organisiert. So sind tonale Gesten im \ili{Mandarin} Teil der lexikalischen Repräsentation von Wörtern und Silben und könnten dadurch in das interne Netzwerk der Silbenorganisation integriert sein. Artikulatorische Studien haben gezeigt, dass postlexikalische Tonakzente hingegen keinen Einfluß auf den intrasilbischen Kopplungsgraphen nehmen (\citealt{Mücke2009b, Mücke2012} für \ili{Katalanisch} und Wiener \ili{Deutsch}; \citealt{Niemann2011} für \ili{Italienisch} und \ili{Deutsch}; \citealt{Niemann2015,Niemann2017} für \ili{Deutsch}, Norddeutsche Varietät).

Im Folgenden wird aufgezeigt, wie steigende Tonakzente für \ili{Katalanisch} und \ili{Deutsch} im \isi{Gestenmodell} abgebildet werden können. Wir beschäftigen uns mit der Frage, wie der Beginn des tonalen Anstiegs, L, mit konsonantischen und vokalischen Gesten koordiniert ist. Eine detaillierte Beschreibung dieser Modellierungsformen befindet sich in \citet{Mücke2012}. Kapitel~\ref{subsec:070202} gibt dann einen Ausblick, wie sich der Zielpunkt der Hochtongeste, H, im Hinblick auf die Silbenkopplung im \il{Deutsch!Wien}Wiener Deutschen verhält. Hier findet sich eine detailliertere Beschreibung in \citet{Mücke2008b} zum \il{Deutsch!Wien}Wiener Deutschen, sowie eine Studie zum Standarddeutschen in \citet{Niemann2015} und \citet{Niemann2017}.

\subsection{Methode: Tonaler Anstieg im Katalanischen und Wiener Deutschen}
\label{subsec:070201}

Im Katalanischen werden steigende LH-Tonakzente für die Markierung von weitem und kontrastivem \isi{Fokus} verwendet. Beide Akzente zeigen ein vergleichbares Alignierungsmuster \citep{Prieto2007a}: Der Beginn des Tonhöhenanstiegs, L, findet in der Nähe des Beginns der \isi{Akzentsilbe} statt, während der Gipfel, H, gegen Ende der \isi{Akzentsilbe} erreicht wird. Während beide Tonakzente sich temporal ähnlich verhalten, unterscheiden sie sich jedoch räumlich durch die erreichte Gipfelhöhe im F0-Verlauf. Im kontrastiven \isi{Fokus} (Abbildung~\ref{figure:0710}~(unten)) -- dem prominenteren Akzent -- wird ein höherer Gipfel für die Realisierung von H in der F0-Kontur erreicht (peak height) als im weiten \isi{Fokus} (Abbildung~\ref{figure:0710}~(oben)).

\begin{figure}
	\includegraphics[width=\textwidth]{figures/7-10_Schreenshot_Mimami_F0.png}
	\caption{Oszillogramm, F0-Verlauf und Sonagramm für die katalanische Äußerung <La MiMAmi> im kontrastiven Fokus. Die Grenzen des Zielwortes <MiMAmi> sind mit vertikalen Linien markiert, die Akzentsilben <MA> grau schattiert.}
	\label{figure:0710}
\end{figure}

Bei den katalanischen Zielwörtern handelt es sich um fiktive Namen wie <Mimami>. In allen Antwort-Äußerungen folgte dem \isi{Nuklearakzent} ein tiefer Grenzton.

%%(7.11)
\begin{exe}
	\ex Weiter \isi{Fokus}:\label{ex:0711}
	\sn F: Qui va venir? --- (lit.: Wer kam?)
	\sn A: [La MiMAmi]\textsubscript{Fokus} --- (lit.: Die Mimami.)
\end{exe}

%%(7.12)
\begin{exe}
	\ex Kontrastiver \isi{Fokus}:\label{ex:0712}
	\sn F: Va venir la MiMAmila? --- (lit.: Kam die Mimamila?)
	\sn A: No, [la MiMAmi]\textsubscript{Fokus} --- (lit.: Nein, die MiMAmi)
\end{exe}

Es wurden acht Zielwörter mit der lexikalisch betonten \isi{Silbe} als Zielsilbe verwendet. Dabei wurden systematisch die Silbenstruktur (offen und geschlossen wie in 'CV.CV und 'CVC.CV), der Artikulationsort des initialen Konsonanten der Zielsilbe (C~=~labial oder alveolar) und die Größe des Fußes (zwei versus drei Silben) variiert. Diese Variationen kommen zustande, weil die \isi{Alignierung} des F0-Gipfels, H, von prosodischen Effekten betroffen ist. So ist für das Katalanische ähnlich wie für das Spanische oder Deutsche bekannt, dass tonale Gipfel -- das Ende des tonalen Anstiegs -- in \isi{nuklear} steigenden Akzenten systematisch später in geschlossenen als in offenen Silben aligniert sind \citep[u.a.][]{Prieto2007b,Prieto2007a,Mücke2009b,Mücke2012}. Hinzu kommt, dass die Fußgröße der Testwörter als Faktor zu berücksichtigen ist, weil die kompensatorische Kürzung polysilbischer Formen (polysyllabic shortening) einen Einfluss auf die Gipfelposition haben kann. Die Zielwörter waren Mami, Mamila, Mamzi, Mamzila, Nani, Nanila, Nanmi, Nanmila.

Für das Deutsche wurde die Wiener Varietät untersucht. Es wurden Frage-Antwort-Paare mit Zielwörtern im kontrastiven \isi{Fokus} verwendet, vgl. Beispiel~\ref{ex:0713}. In \il{Deutsch}deutschen Deklarativsätzen werden vor allem in \isi{nuklear} kontrastiven Akzente steigende LH-Konturen produziert \citep{Baumann2006}. Den Nuklearakzenten folgte ein tiefer Grenzton.

%%(7.13)
\begin{exe}
	\ex Kontrastiver \isi{Fokus}:\label{ex:0713}
	\sn F: Hat sie die Mammi oder die Nanni bestohlen?
	\sn A: Sie hat [die MAMMi]\textsubscript{ Fokus} bestohlen.
\end{exe}

Analog zum Katalanischen wurden die Silbenstruktur (offen/geschlossen) und der Artikulationsort des initialen Konsonanten der Zielsilbe (labial/alveolar) variiert. Dabei wurde die Silbenstruktur mit Hilfe der phonologischen Vokallänge variiert (Langvokale 'CV:.CV und Kurzvokale 'CVCV). Im Deutschen treten Kurzvokale nicht in offenen Silben auf, die lexikalisch betont sind, so dass Ambisyllabizität für den intervokalischen Konsonanten in der Sequenz 'CVCV angenommen wird. Das verdeutlichen auch die psycholinguistischen Experimente von \citet{Schiller1997}. Diese zeigen, dass Sprecher des Holländischen dazu tendieren, Silben mit Kurzvokal als geschlossene Silben zu produzieren. Darüber hinaus lässt sich im Deutschen annehmen, dass bei den verwendeten Zielwörtern der definite Artikel zusammen mit dem benachbarten Inhaltswort ein prosodisches Wort ergibt (<die Mahmi>). Die Zielwörter waren Mahmi, Mammi, Nahni, Nanni.

Abbildung~\ref{figure:0711} gibt ein Beispiel für die Äußerung <Sie hat die MAHmi bestohlen.> Der F0-Verlauf für den \isi{Nuklearakzent} (kontrastiver \isi{Fokus}) zeigt eine vergleichsweise späte \isi{Alignierung}; er beginnt erst spät im \isi{Vokal} der CV-\isi{Akzentsilbe} zu steigen und erreicht seinen Gipfel erst in der folgenden lexikalisch unbetonten \isi{Silbe}.

\begin{figure}
	\includegraphics[width=\textwidth]{figures/7-11_Screenshot_MahmiDeutsch_F0.png}
	\caption{Oszillogramm, F0-Verlauf und Sonagramm für die die Äußerung <Sie hat die MAHmi bestohlen> im kontrastiven Fokus. Die Grenzen des Zielwortes <MAHmi> sind mit durchgezogenen Linien markiert, die Akzentsilben <MAH> grau schattiert.}
	\label{figure:0711}
\end{figure}

Die Aufnahmen wurden mit einem Carstens AG100 (5 Kanäle) mittels 2-D Elektromagnetischer \isi{Artikulographie} im Labor des I\textit{f}L Phonetik der Universität zu Köln durchgeführt. An dem Experiment nahm für das Katalanische eine Sprecherin aus Zentralkatalonien und für das \il{Deutsch!Wien}Wiener Deutsche eine Sprecherin aus Wien teil. Sensoren wurden auf den unteren Lippenrand, Zungenblatt und Zungenrücken befestigt, um die Bewegungen der primären Artikulatoren aufzuzeichnen. Die Probandinnen lasen das Sprachmaterial in pseudorandomisierter Form von einem Computerbildschirm ab. Es wurden 120 Targetwörter (\ili{Katalanisch}: 8 Targetwörter x 5 Wiederholungen x 2 Fokuskonditionen; \ili{Deutsch}: 4 Targetwörter x 10 Wiederholungen) aufgezeichnet. Beide Sprecherinnen realisierten durchgängig \isi{nuklear} steigende Tonakzente auf den Zielwörtern \citep[vgl. auch][]{Mücke2012}.

Es wurden tonale Landmarken in der F0-Kontur, segmentale Landmarken in der akustischen Wellenform und gesturale Landmarken in der kinematischen Wellenform identifiziert. Für den steigenden LH-Akzent wurden lokale Wendepunkte in der F0-Kontur manuell identifiziert (Abbildung~\ref{figure:0712}). Dabei wurde das lokale Minimum zu Beginn des F0-Anstiegs (L) und das Maximum am Ende des Anstiegs (H) annotiert. Für die akustische Analyse wurden L und H als tonale Zielpunkte gemäß des AM-Modells behandelt. In der kinematischen Analyse (AP-Modell) wurde der Beginn der tonalen \isi{Geste} zu dem Zeitpunkt festgelegt, an dem die F0-Bewegung in Richtung der gestischen Zielspezifikation beginnt. Das bedeutet, dass das tonale Label für L im AM-Modell und der \isi{Onset} der H-\isi{Geste} in der AP zeitgleich auftreten. In dem verwendeten Material fällt der \isi{Onset} der H-\isi{Geste} mit dem Offset der vorangehenden L-\isi{Geste} zusammen.

\begin{figure}
	\includegraphics[width=\textwidth]{figures/7-12_LH_in_Mahmi.png}
	\caption{Annotationsbeispiel für das Setzen von Landmarken im F0-Verlauf für die Äußerung <Sie hat die MAHmi bestohlen> im \il{Deutsch!Wien}Wiener Deutsch. Es werden Wendepunkte für L und H im F0-Verlauf bestimmt (Annotation nach AM-Model), die dann in Relation zu tonalen Gesten gesetzt werden (nach AP Modell).}
	\label{figure:0712}
\end{figure}
 
\newpage  
Für die akustischen Analysen wurden Segmentgrenzen von Konsonanten und Vokalen in den Zielwörtern identifiziert. Dazu wurden Oszillogramme und zugehörige Breitbandsongramme gleichzeitig dargestellt. Die Segmentgrenzen zwischen Konsonanten und Vokalen wurden zum Zeitpunkt des abrupten Energieabfalls bei der Bildung des konsonantischen Verschlusses im Spektrum identifiziert. Das galt für Nasale (starke spektrale Dämpfung), Laterale (hier insbesondere in den höheren Formantstrukturen) und Frikative (zusätzlich Auftreten aperiodischer Wellenformen bei der geräuschverursachenden Engebildung). Auf Basis der Segmentgrenzen wurden zeitliche Abstände zwischen dem Beginn des F0-Anstiegs (der Tiefpunkt für L) und dem Beginn des initialen ${C}_{1}$ Segments der akzentuierten \isi{Silbe} ermittelt (Abstand: Ton- ${C}_{1}${}-Segment).

\begin{figure}[t]
	\includegraphics[width=\textwidth]{figures/7-13_Mimami_Artiklation_Screenshot.png}
	\caption{Von oben nach unten Oszillogramm, vertikale Positions- und Geschwindigkeitskurve für den Zungenrücken zur Erfassung der Vokalproduktion /a/ und für die untere Lippe zur Erfassung der Konsonantenproduktion /m/ in der Zielsilbe. <ma> (Katalanisch). Die Aktivierungsintervalle von Onset bis Offset der Bewegung in /a/ und /m/ sind grau schattiert und die zugehörigen Nulldurchgänge in den Geschwindigkeitskurven mit Punkten markiert. Auf Basis der kinematischen Landmarken wurden Zeitabstände zwischen Startpunkten (Onsets) der tonalen Gesten (H-Gesten) und der oralen Gesten (V und C) berechnet. Die kinematischen Berechnungen basieren auf Onset-zu-Onset Abständen, d.\,h., es wurden Abstände zwischen Zeitpunkten kalkuliert, an denen gestische Aktivierungsintervalle beginnen.}
	\label{figure:0713}
\end{figure}

Die kinematischen Landmarken wurden anhand der vertikalen Bewegungen der Artikulatoren ermittelt. Dazu wurden die vertikalen Positionstrajektorien der Sensoren auf der Unterlippe für /m/, der Zungenspitze für /n/ und dem Zungenrücken für die Vokalbewegungen verwendet. Die Startpunkte der konsonantischen und vokalischen Bewegungen wurden anhand von Nulldurchgängen in den jeweiligen Geschwindigkeitskurven bestimmt. Die Abbildung~\ref{figure:0713} zeigt ein kinematisches Landmarkenschema für die Zielsilbe <ma> im weiten \isi{Fokus} (Zielwort <MiMAmi>). Es ist gut zu erkennen, dass der \isi{Vokal} und der Konsonant gleichzeitig aktiviert werden, der \isi{Vokal} jedoch eine langsamere \isi{Ausführungsgeschwindigkeit} hat.

  

Sowohl für die akustische als auch für die kinematische Analyse werden zeitliche Abstände in der Form  $A-B$ ermittelt. Dabei indizieren negative Werte, dass A früher als B auftritt, und positive Werte, dass $B$ früher als $A$ auftritt. Tabelle~\ref{table:0701} fasst die relevanten Messvariablen zusammen.

  
\begin{table}
\small
%	\resizebox{\textwidth}{!}{
		\begin{tabularx}{\textwidth}{llX} \lsptoprule
			\textbf{Variable} & \textbf{Ebene} & \textbf{Beschreibung}\\ \midrule
			\textbf{L-C1onset} & Akustik & Zeitlicher Abstand (ms) zwischen dem Beginn des nuklearen F0-Anstiegs (L) in der F0-Kontur und der linken Segmentgrenze des initialen Konsonanten in der akzentuierten \isi{Silbe}. Negative Werte indizieren, dass der linke Rand der \isi{Akzentsilbe} vor dem {L}{}-Ton auftritt.\\ \cmidrule{1-3}
			\textbf{Ton-zu-V} & \isi{Artikulation} & Zeitlicher Abstand (ms) zwischen dem Beginn des nuklearen F0-Anstiegs (\isi{Onset} der H-\isi{Geste}) in der F0-Kontur und dem Beginn der vokalischen \isi{Geste} in der akzentuierten \isi{Silbe}. Negative Werte indizieren, dass die vokalische Bewegung vor der tonalen Bewegung startet.\\ \cmidrule{1-3}
			\textbf{Ton-zu-C} & \isi{Artikulation} & Zeitlicher Abstand (ms) zwischen dem Beginn des nuklearen F0-Anstiegs (\isi{Onset} der H-\isi{Geste}) in der F0-Kontur und dem Beginn der konsonantischen \isi{Geste} in der akzentuierten \isi{Silbe}. Negative Werte indizieren, dass die konsonantische Bewegung vor der tonalen Bewegung startet.\\ \cmidrule{1-3}
			\textbf{C-zu-V} & \isi{Artikulation} & Zeitlicher Abstand (ms) zwischen dem Beginn der konsonantischen \isi{Geste} und der vokalischen Bewegung in der akzentuierten \isi{Silbe}; \isi{kinematische Messung}. Negative Werte indizieren, dass die konsonantische Bewegung vor der vokalischen Bewegung startet.\\ \lspbottomrule
		\end{tabularx}
%	}
	\caption{Messvariablen auf akustischer und artikulatorischer Ebene.}
	\label{table:0701}
\end{table}

\subsection{Ergebnisse: Tonale Anstiege im Katalanischen und Deutschen}
\label{subsec:070202}

Zunächst sollen die akustischen Alignierungsmuster im Katalanischen und \il{Deutsch!Wien}Wiener Deutschen im Rahmen des AM-Modells verglichen werden. Hierfür wurden die Abstände zwischen dem L-Ton und dem Beginn des initialen Konsonanten, ${C}_{1}$, in der akzentuierten \isi{Silbe} berechnet(${L}-{C1}_{onset}$). Eine Tabelle, die sämtliche Werte für die beiden Datensätze \ili{Katalanisch} und Wiener \ili{Deutsch} abbildet, findet sich im Anhang (siehe Kapitel~\ref{chap:app01}). Wir konzentrieren uns im Folgenden auf den labialen Datensatz.

Abbildung~\ref{figure:0714} zeigt die Häufigkeitsverteilungen im Katalanischen (hellgrau) und \il{Deutsch!Wien}Wiener Deutschen (dunkelgrau) für Zielwörter im kontrastiven \isi{Fokus} mit offener Zielsilbe. Negative Werte auf der x-Achse indizieren, dass der L-Tiefton systematisch vor ${C}_{1}$ auftritt. Positive Werte indizieren, dass er nach ${C}_{1}$ auftritt. Es sind zwei unterschiedliche Verteilungen erkennbar. Im Katalanischen tritt der Tiefton L vor ${C}_{1}$ auf (durchschnittlich $28\,ms$ vor ${C}_{1}$), während er im \il{Deutsch!Wien}Wiener Deutschen nach ${C}_{1}$ auftritt (durchschnittlich $71\,ms$). Der Tiefton L ist demnach in den labialen Datensätzen knapp $100\,ms$ früher im Katalanischen als im Deutschen aligniert und die Messwerte überlagern sich in den Stichproben nicht.

\begin{figure}
	\includegraphics[width=.8\textwidth]{figures/7-14_density_plot_akustik.png}
	\caption{Häufigkeitsverteilung für die akustischen Alignierungsabstände L-${C}_{1}$ Onset (Katalanisch in hellgrau und \il{Deutsch!Wien}Wiener Deutsch in dunkelgrau). Negative Werte indizieren, dass L vor der Akzentsilbe auftritt. Positive Werte indizieren, dass L nach der Akzentsilbe auftritt.}
	\label{figure:0714}
\end{figure}

In Abbildung~\ref{figure:0715} sind die akustischen Alignierungsmuster für den Zielfuß [ma.mi] im Katalanischen (links) und [maː.mi] im Wienerischen (rechts) schematisiert. Die Abbildung ist zeitlich skaliert und basiert auf statistischen Mittelwerten, die in der genannten Abbildung aufgeführt sind. Die lexikalische betonte \isi{Silbe} ist grau schattiert.

\begin{figure}
	\includegraphics[width=.8\textwidth]{figures/7-15_Alignierung_schematisiert.png}
	\caption{Schematisiertes akustisches Alignierungsmuster für nuklear steigende LH-Akzente, kontrastiver Fokus im Katalanischen (links) und kontrastiver Fokus im Wiener Deutschen (rechts).}
	\label{figure:0715}
\end{figure}

In beiden Sprachen tritt der Tiefton L in der Nähe des C1-Segments auf (initialer Konsonant der lexikalisch betonten \isi{Silbe}). Im Katalanischen tritt L vor dem akustischen \isi{Onset} von C1 und im \il{Deutsch!Wien}Wiener Deutschen nach ihm auf. Es zeigt sich, dass mit Hilfe der Segmentalen Ankerhypothese im Rahmen des AM-Modells Alignierungsmuster zwischen Sprachen gut darstellbar sind (vgl. u.a. \citealt{Arvaniti1998} für \ili{Griechisch}; 
\citealt{Prieto2007b} für \ili{Spanisch}; 
\citealt{Ladd1999} für \ili{Englisch}; 
\citealt{Ladd2000} für \ili{Holländisch};
\citealt{Dimperio2007} für Neapolitanisches \ili{Italienisch}) und regiolektalen Varietäten innerhalb einer Sprache \citep[u.a.][]{Atterer2004, Braun2007, Mücke2008a, Mücke2008b, Kleber2008, Mücke2009a, Mücke2009b} für unterschiedliche nord-, süd- und ostdeutsche Varietäten).\il{Deutsch} Während die deskriptive Darstellung der Algnierungsunterschiede gut gelingt, ist ihre \isi{Modellierung} oder Phonologisierung jedoch problematisch. So basiert die AM-Beschreibung auf akustischen Ankern wie „linker Rand des initialen Konsonanten in der lexikalisch betonten \isi{Silbe}“ für das \ili{Katalanisch} und „Mitte des initialen Konsonanten in der lexikalisch betonten \isi{Silbe}“ im \il{Deutsch!Wien}Wiener Deutschen. Deshalb spiegeln die Alignierungsunterschiede im Rahmen des AM-Modells nicht unterschiedliche phonologische Assoziationen wider, sondern beschreiben vielmehr phonetisches Detail \citep{Atterer2004,Ladd2008}. Diese Auffassung wird durch die Ergebnisse der Studien zum Katalanischen und \il{Deutsch!Wien}Wiener Deutschen gestützt: Der initiale Konsonant der lexikalisch betonten \isi{Silbe} ist in beiden Sprachen der segmentale Anker. Die ermittelten Alignierungsunterschiede zwischen \ili{Katalanisch} und Wiener \ili{Deutsch} sind dabei phonetischer, gradueller Natur.

Die artikulatorischen Daten lassen sich am besten veranschaulichen, wenn man zunächst sein Augenmerk auf die Koordination der silbeninternen Koordination von konsonantischer und vokalischer \isi{Geste}, CV, richtet. Hierzu wurde der zeitliche Abstand (ms) zwischen dem Beginn der konsonantischen \isi{Geste} und der vokalischen Bewegung in der akzentuierten \isi{Silbe} gemessen. Negative Werte indizieren, dass die konsonantische Bewegung vor der vokalischen Bewegung startet. Die Werte sind in der Häufigkeitsverteilung in Abbildung~\ref{figure:0716} dargestellt, jeweils für \ili{Katalanisch} (hellgraue Verteilung) und Wiener \ili{Deutsch} (dunkelgraue Verteilung). Das Modell der gekoppelten Oszillatoren propagiert für die Selbstorganisation von CV-Silben (Kapitel~\ref{subsec:030301}) eine In-Phase-\isi{Kopplung} von C und V. Dem Modell zufolge starten beide Gesten gleichzeitig. Da die vokalische \isi{Geste}, V, eine geringere \isi{Ausführungsgeschwindigkeit} als die konsonantische \isi{Geste}, C, aufweist, entsteht auf akustischer Oberfläche der Eindruck einer Abfolge von C und V. Eben dieses Timing finden wir in unseren Stichproben für das Katalanische und Deutsche (Abbildung~\ref{figure:0716}). Im Katalanischen tritt die konsonantische \isi{Geste} sehr kurz vor der vokalischen \isi{Geste} auf (durchschnittlich $-2$~ms in den labialen Daten). Im \il{Deutsch!Wien}Wiener Deutschen verhält es sich umgekehrt, denn hier tritt die konsonantische \isi{Geste} sehr kurz nach der vokalischen auf (durchschnittlich $5$~ms). Die Unterschiede sind aber sehr gering, und sie basieren auf jeweils nur einem Sprecher bzw. einer Sprecherin pro Sprache. Die Häufigkeitsverteilungen in beiden Stichproben überlappen beinah vollständig. In beiden Sprachen starten C und V synchron. Diese In-Phase-Koordination ist in beiden Sprachen von der Anwesenheit eines Tonakzents unbeeinflusst.

\begin{figure}[p]
	\includegraphics[width=\textwidth]{figures/7-16_density_Artikulation_CV.png}
	\caption{Häufigkeitsverteilung für die artikulatorischen Alignierungsabstände C-zu-V (Katalanisch in hellgrau und Wiener Deutsch in dunkelgrau). Negative Werte indizieren, dass C vor V auftritt und umgekehrt.}
	\label{figure:0716}
\end{figure}

\largerpage
Betrachten wir nun die Koordination der tonalen Gesten mit der \isi{Silbe}. Da die konsonantische und die vokalische \isi{Geste} gleichzeitig starten, genügt ein Vergleich mit einer der beiden oralen Gesten. Hier wird als Beispiel die vokalische \isi{Geste} bevorzugt, weil sie den Nukleus der \isi{Silbe} bildet und weil im AM Modell der \isi{Tonakzent} mit dem Nukleus der \isi{Akzentsilbe} assoziiert ist. Des Weiteren beschränken wir uns analog zur akustischen Analyse auf Zielwörter mit offener Silbenstruktur, da eine Varianzanalyse in der Stichprobe Wiener \ili{Deutsch} einen Effekt der Silbenstruktur auf die Ton-zu-V Koordination gezeigt hat.

\clearpage 

Abbildung~\ref{figure:0717} zeigt die Häufigkeitsverteilungen für die Messvariable Ton-zu-V für \ili{Katalanisch} und Wiener \ili{Deutsch}. Es handelt sich dabei um den zeitlichen Abstand (ms) zwischen dem Beginn des nuklearen F0-Anstiegs (\isi{Onset} der H-\isi{Geste}) in der F0-Kontur und dem Beginn der vokalischen \isi{Geste} in der akzentuierten \isi{Silbe}. Negative Werte indizieren, dass die vokalische Bewegung vor der tonalen Bewegung startet. Im Katalanischen (hellgraue Verteilung) starten tonale und vokalische \isi{Geste} synchron. Die tonale \isi{Geste} startet durchschnittlich nur 4\,ms vor der vokalischen \isi{Geste}. Das bedeutet für \ili{Katalanisch} einen synchronen Trigger der tonalen und oralen Gesten. Innerhalb der \isi{Silbe} starten C und V in-phase und zwischen tonaler \isi{Geste} und V besteht ebenfalls eine in-phase-Relation. Im \il{Deutsch!Wien}Wiener Deutschen (dunkelgraue Verteilung) verhält es sich jedoch anders. Hier startet die tonale \isi{Geste} deutlich später als die vokalische \isi{Geste}, durchschnittlich 133\,ms nach der Aktivierung der vokalischen \isi{Geste}. Die tonalen Gesten werden im \il{Deutsch!Wien}Wiener Deutschen im Hinblick auf orale Gesten stark verzögert aktiviert. Demgegenüber bleibt die Synchronisation der oralen Gesten untereinander unbeeinflusst: C und V starten -- wie im Katalanischen -- synchron.

\begin{figure}[p]
	\includegraphics[width=\textwidth]{figures/7-17_density_TonC.png}
	\caption{Häufigkeitsverteilung für die artikulatorischen Alignierungsabstände Ton-zu-C (Katalanisch in hellgrau und Wiener Deutsch in dunkelgrau). Negative Werte indizieren, dass der Ton vor dem initialen Konsonanten, C, auftritt. Positive Werte indizieren, dass der Ton nach C auftritt.}
	\label{figure:0717}
\end{figure}


Abbildungen~\ref{figure:0718} und~\ref{figure:0719} veranschaulichen die gesturale Koordination für \ili{Katalanisch} und Wiener \ili{Deutsch} mit Hilfe von Gestenpartituren (zu Grundlagen der Gestenpartituren vgl. Kapitel~\ref{chap:02}). Die Partituren zeigen \isi{Aktivierungsintervalle} für orale und glottale Gesten, die auf kinematischen Daten der Bewegungsaktivierung und Deaktivierung der ausführenden Artikulatoren (Start bis Ziel der Bewegung) basieren. Diese Abbildungen können in Bezug zu den Alignierungsmustern um Rahmen der segmentalen Ankerhypothese gesetzt werden (Abbildung~\ref{figure:0715}). Die segmentale Ankerhypothese zeigt bereits, dass der Anstieg des Nuklearakzents im Katalanischen früher (kurz vor Beginn des Zielwortes) und im Wiener \ili{Deutsch} später (Mitte des initialen Konsonanten) beginnt, wenn man die akustische Oberfläche betrachtet. Diese Unterschiede spiegeln sich in den Gestenpartituren wider, die das Auftreten von \isi{F0} in Bezug zu artikulatorischen Konstriktionsgesten setzen. Allerdings können die Gestenpartituren noch zusätzlich die F0-Bewegung relativ zum Beginn der Vokalgeste veranschaulichen, was auf akustischer Ebene nicht möglich ist, da in der Akustik der Eindruck einer Sequenz von C und V entsteht.

Die Gestenpartituren zeigen von oben nach unten die Aktivierung der Tongesten L und H, den labialen Verschluss für {LA labial closure} [m] und die Zungenrückenkonstriktion für die Vokale {TB pharyngeal wide} [a] und {TB palatal narrow} [i]. Die gepunktete Linie indiziert, dass der Start der L-Tongeste im bitonalen LH-Akzent nicht aus den Daten berechnet werden kann.

 

\begin{figure}[t]
	\includegraphics[width=\textwidth]{figures/7-18_Gestenparttur_katalan.png}
	\caption{Gestenpartitur für Katalanisch ['ma.mi]. Die Abbildung ist skaliert und basiert auf statistischen Mittelwerten für zehn Wiederholungen des Zielwortes.}
	\label{figure:0718}
\end{figure}

\begin{figure}[t]
	\includegraphics[width=\textwidth]{figures/7-19_Gestenpartitur_deutsch.png}
	\caption{Gestenpartitur für Wiener Deutsch ['ma:.mi], kontrastiver Fokus. Die Abbildung ist skaliert und basiert auf statistischen Mittelwerten.}
	\label{figure:0719}
\end{figure}

\newpage   
Die \isi{Gestenpartitur} für das Katalanische zeigt für die Produktion der lexikalisch betonten \isi{Silbe}, dass die Onsets für die konsonantische {LA labial closure}, vokalische {TB pharyngeal wide} und die H-\isi{Geste} alle zeitgleich auftreten. Für weiten und kontrastiven \isi{Fokus} startet der \isi{Onset} der tonalen H-\isi{Geste} synchron mit den Onsets der oralen Konstriktionsgesten für den \isi{Vokal} und den initialen Konsonanten in der lexikalisch betonten \isi{Silbe}. Dieses Muster spiegelt eine In-Phase-Koordination von tonalen und oralen Gesten wider: Konsonant, \isi{Vokal} und Ton sind synchron initiiert. Bei genauer Betrachtung starten die Gesten in der Reihenfolge C-V-T, aber die zeitlichen Unterschiede sind extrem klein.


  
Bei Betrachtung der Ergebnisse für die Alignierungsmuster im \il{Deutsch!Wien}Wiener Deutschen fällt auf, dass der \isi{Onset} der tonalen H-\isi{Geste} im Vergleich zur vokalischen \isi{Geste} mehr als $100$\,ms zeitverzögert auftritt. Dennoch starten die oralen Konstriktionsgesten für den initialen Konsonanten und den Folgevokal synchron. Nur die tonale \isi{Geste} startet also später.

 
\begin{figure}[b]
% 	\includegraphics[width=\textwidth]{figures/7-20_Kopplung_katalan.png}
	\caption{Gestenpartituren und Kopplungsgraphen für Katalanisch; die Kopplungsgraphen zeigen In-Phase- (durchgezogene Linien) und Anti-Phase-Zielspezifikationen (gestrichelte Linien).}
	\label{figure:0720}
\fittable{
\begin{tikzpicture}
\node[circle,draw, minimum size=3mm] (L)  []  {L};
\node[circle,draw, minimum size=3mm] (H)  [right=of L]  {H};
\node (V) [circle,draw, minimum size=3mm,below = of $(L)!0.5!(H)$,yshift=4mm]  {V};

\draw[dashed] (L) -- (H);
\draw 	      (V) -- (H);


\node  [minimum width=1.5cm, minimum height=.7cm, inner sep=0pt, draw ] (rH) [left=of L] {H};
\node  [minimum width=1.5cm, minimum height=.7cm, inner sep=0pt, draw ] (rV) [below=of rH,yshift=.9cm] {V};
\node  [minimum width=1.5cm, minimum height=.7cm, inner sep=0pt, draw ] (rL) [left=of rH,xshift=.9cm] {L};
\node  [minimum width=3mm, minimum height=1cm, inner sep=0pt, fill=white ] (rLc) [left=of rL,xshift=1.1cm] {};
\end{tikzpicture} 
}
\end{figure}
\begin{figure}[b]
% 	\includegraphics[width=\textwidth]{figures/7-21_Kopplung_deutsch.png}
\fittable{
  \begin{tikzpicture}
  \node[circle,draw, minimum size=3mm] (L)  []  {L};
  \node[circle,draw, minimum size=3mm] (H)  [right=of L]  {H};
  \node (V) [circle,draw, minimum size=3mm,below = of $(L)!0.5!(H)$,yshift=4mm]  {V};

  \draw[dashed] (L) -- (H);
  \draw         (V) -- (H);
  \draw         (V) -- (L);


  \node  [minimum width=1.5cm, minimum height=.7cm, inner sep=0pt, draw ] (rH) [left=of L] {H};
  \node  [minimum width=2cm, minimum height=.7cm, inner sep=0pt, draw ] (rV) [below=of rH,yshift=.95cm,xshift=-1.3cm] {V};
  \node  [minimum width=1.5cm, minimum height=.7cm, inner sep=0pt, draw ] (rL) [left=of rH,xshift=.9cm] {L};
  \node  [minimum width=3mm, minimum height=.8cm, inner sep=0pt, fill=white ] (rLc) [left=of rL,xshift=1.1cm] {};
  \end{tikzpicture} 
}
	\caption{Gestenpartituren und Kopplungsgraphen für \il{Deutsch!Wien}Wiener Deutsch; die Kopplungsgraphen zeigen In-Phase- (durchgezogene Linien) und Anti-Phase- (gestrichelte Linien) Zielspezifikationen.}
	\label{figure:0721}
\end{figure}





\largerpage
Um diese Unterschiede zwischen \ili{Katalanisch} und Wiener \ili{Deutsch} im Rahmen der AP modellieren zu können, werden die distinktiven Kopplungsgraphen in Abbildung~\ref{figure:0720} und~\ref{figure:0721} angenommen und im artikulatorischen Task-Dynamic-Sprachsynthesystem (TaDA) der Haskins Laboratorien \citep{Nam2004} getestet (vgl. Kapitel~\ref{sec:0102}). Dabei dienten die Kopplungsgraphen als Systeminput, um unterschiedliche Gestenpartituren des Katalanischen und \il{Deutsch!Wien}Wiener Deutschen zu generieren. Die TaDA-Gestenpartituren zeigten dabei deutlich die in den Daten gefundenen, späteren Onsets für die tonale H-\isi{Geste} im \il{Deutsch!Wien}Wiener Deutschen.


Die tonalen Gesten L und H sind in den beiden Sprachen sequentiell angeordnet und deshalb miteinander in Anti-Phase gekoppelt (gepunktete Linien). Der Unterschied zwischen den Sprachen liegt in den Kopplungseigenschaften der tonalen und oralen Gesten.


Im Katalanischen (Abbildung~\ref{figure:0720}) ist die H-\isi{Geste} in-phase mit der vokalischen \isi{Geste} gekoppelt (durchgezogene Linie). Die L-\isi{Geste} dagegen ist nicht direkt mit der vokalischen \isi{Geste} gekoppelt und startet zu einem früheren Zeitpunkt vor der akzentuierten \isi{Silbe}. Die H-\isi{Geste} und die vokalische \isi{Geste} starten gleichzeitig.


Im \il{Deutsch!Wien}Wiener Deutschen sind beide tonale Gesten L und H mit der vokalischen \isi{Geste} In-Phase gekoppelt. Zusätzlich sind L und H miteinander sequentiell gekoppelt. Die daraus resultierenden konkurrierenden Zielspezifikationen bewirken eine \isi{Rechtsbewegung} der H-\isi{Geste}, um Platz für die vorangehende L-\isi{Geste} zu schaffen; das \isi{Target} der H-\isi{Geste} (der F0-Gipfel) kann aufgrund der \isi{Rechtsbewegung} auch außerhalb der lexikalisch betonten \isi{Silbe} auf die postlexikalische \isi{Silbe} fallen. Dies \isi{Koordinationsmuster} ist vergleichbar mit den konkurrierenden Zielspezifikationen (C-Center) in Konsonantenclustern ${C}_{1}{C}_{2}$, bei denen beide Konsonanten, in-phase mit dem \isi{Vokal} und in Anti-Phase zueinander gekoppelt sind (vgl. \citealt{Browman1989,Browman2000,Nam2003,Nam2007a,Goldstein2009,Marin2008,Hermes2013}), was zu einer Linksbewegung von ${C}_{1}$ und einer \isi{Rechtsbewegung} von ${C}_{2}$ relativ zum folgenden \isi{Vokal} in der Kinematik führt.

Es zeigt sich, dass die Alignierungsdifferenzen in Form von Kopplungsgraphen festgehalten und in Form von Zeitrelationen zwischen Taktgebern phonologisiert werden können. Somit ist die H-\isi{Geste} nicht mit einem arbiträren Zeitpunkt synchronisiert, sondern resultiert automatisch aus den konkurrierenden Zielspezifikationen des Kopplungsgraphen.

Es gibt jedoch auch im AM-Modell die Möglichkeit, einen phonologischen Unterschied zwischen den Alignierungsmustern der \isi{nuklear} steigenden LH-Akzente im Katalanischen und \il{Deutsch!Wien}Wiener Deutschen abzubilden. Dafür kann angenommen werden, dass im Katalanischen der steigende \isi{Tonakzent} keinen führenden L-Ton besitzt (leading Tone) und die Analyse nur einen H*-Akzent zeigt. Diese Form der Analyse würde zu einer nicht-konkurrierenden Zielspezifikation für die tonale H-\isi{Geste} mit der vokalischen \isi{Geste} führen. Für Wiener \ili{Deutsch} könnte ein L*H statt eines LH* angenommen werden, um dem späteren Anstieg des F0-Verlaufs relativ zum Start der \isi{Akzentsilbe} gerecht zu werden. Es existiert jedoch im Deutschen keine klare kategoriale Unterscheidung zwischen L*H und LH* (vgl. die Diskussion in \citealt{Braun2003} und \citealt{Braun2007}).



Es bestehen außerdem Gemeinsamkeiten zwischen den angenommenen Unterschieden in den Kopplungsgraphen in der Artikulatorischen Phonologie und dem Assoziationsdiagramm im AM-Modell von \citet{Grice1995}, bei dem bitonale Tonakzente entweder als Sequenzen bzw. Cluster (Abbildung~\ref{figure:0722}~(a)) oder als Einheiten (Abbildung~\ref{figure:0722}~(b)) analog zur Darstellung von Affrikaten in der segmentalen Domäne \citep{Yip1989} dargestellt werden.


\begin{figure}
% \includegraphics[width=\textwidth]{figures/7-22_Ton_inAP.png} 
\subfigure[Katalanisch]{
\begin{tikzpicture}  
\node[circle,draw, minimum size=3mm] (kL)  []  {L};
\node[circle,draw, minimum size=3mm] (kH)  [right=of L]  {H};
\node (kV) [circle,draw, minimum size=3mm,below = of $(kL)!0.5!(kH)$,yshift=4mm]  {V};
\node()[above of=kL,anchor=west,yshift=-3mm,xshift=-2mm]{tonal cluster};
\draw[dotted] (kL) -- (kH);
\draw 	      (kV) -- (kH);



\node (kEnd) [left=of kL,xshift=5mm] {};
\node (kStart) [left=of kEnd, xshift=-10mm] {};
\draw[dashed] (kStart) -- (kEnd);
\node[circle,draw, inner sep=0.75mm] (kLbubble) [below=of kStart,yshift=9mm,xshift=4mm]  {};
\node[circle,draw, inner sep=0.75mm] (kHbubble) [below=of kEnd,  yshift=9mm,xshift=-4mm]  {};
\node(kLText)[below=of kLbubble] {L};
\node(kHText)[below=of kHbubble] {H};
\node[right=of kLText,xshift=-7.5mm]{+};
\node[right=of kHbubble,xshift=-1.05cm]{*};
\draw (kLbubble) -- (kLText);
\draw (kHbubble) -- (kHText);
\node(kSigmaanchor) [above=of kHbubble,yshift=-10mm]  {};
\node(kSigma)[above=of kSigmaanchor] {σ\parbox{0mm}{\,*}};
\node(kLText)[below=of kLbubble] {L};
\node[right=of kLText,xshift=-7.5mm]{+};
\draw (kSigmaanchor) -- (kSigma); 
\end{tikzpicture} 
\qquad
}
\subfigure[Deutsch (Wien)]{
\begin{tikzpicture}   
\node[circle,draw, minimum size=3mm] (wL)  []  {L};
\node[circle,draw, minimum size=3mm] (wH)  [right=of wL]  {H};
\node (wV) [circle,draw, minimum size=3mm,below = of $(wL)!0.5!(wH)$,yshift=4mm]  {V};

\draw[dotted] (wL) -- (wH);
\draw         (wV) -- (wH);
\draw         (wV) -- (wL); 
\node()[above of=wL,yshift=-3mm,anchor=west]{tonal unit};


\node (wEnd) [left=of wL,xshift=5mm] {};
\node (wStart) [left=of wEnd, xshift=-10mm] {};
\draw[dashed] (wStart) -- (wEnd);
\node[circle,  inner sep=0.75mm] (wLbubble) [below=of wStart,yshift=9mm,xshift=4mm]  {};
\node[circle,  inner sep=0.75mm] (wHbubble) [below=of wEnd,  yshift=9mm,xshift=-4mm]  {};
\node[circle, draw, inner sep=0.75mm] (wMiddlebubble) [below=of wEnd,  yshift=9mm,xshift=-12mm]  {};
\node(wLText)[below=of wLbubble] {L};
\node(wHText)[below=of wHbubble] {H};
\node[right=of wLText,xshift=-7.5mm]{+};
\node[right=of wMiddlebubble,xshift=-1.05cm]{*};
\draw (wMiddlebubble) -- (wLText);
\draw (wMiddlebubble) -- (wHText);
\node(wSigmaanchor) [above=of wMiddlebubble,yshift=-10mm]  {};
\node(wSigma)[above=of wSigmaanchor] {σ\parbox{0mm}{\,*}};
\node(wLText)[below=of wLbubble] {L};
\node[right=of wLText,xshift=-7.5mm]{+};
\draw (wSigmaanchor) -- (wSigma);
\end{tikzpicture}
}

\caption{(a) Tonales Cluster zusammengesetzt aus zwei (bitonalen) Tonakzenten mit zwei tonalen Knotenpunkten. (b) Tonale Einheit mit einem gemeinsamen sich verzweigenden tonalen Knotenpunkt \citep{Grice1995}.}
\label{figure:0722}
\end{figure}


Die Koordination von Tonakzenten im Katalanischen und \il{Deutsch!Wien}Wiener Deutschen und die daraus resultierenden Kopplungsgraphen unterscheiden sich in einer wichtigen Eigenschaft von denen der lexikalischen Töne im \ili{Mandarin}, wie sie von \citet{Gao2009} analysiert wurden. Im \ili{Mandarin} werden Silben mit H oder L Tönen mit einer beträchtlichen Zeitverzögerung von (circa 50~ms) zwischen konsonantischen und vokalischen Gesten und zwischen vokalischen und tonalen Gesten produziert. Im Gegensatz dazu starten im Katalanischen und \il{Deutsch!Wien}Wiener Deutschen die konsonantischen und vokalischen Gesten gleichzeitig, unabhängig davon, ob ein \isi{Tonakzent} anwesend ist. Eine Interpretation dieser Ergebnisse (auf der Basis der oben angeführten Kopplungsgraphen) wäre die Folgende: Wenn eine tonale \isi{Geste} im Sprachen mit postlexikalischen Tönen wie \ili{Katalanisch} oder \ili{Deutsch} zu einer \isi{Silbe} hinzugefügt wird, hat die tonale \isi{Geste} keinen Einfluss auf die silbeninterne CV-Koordination. Das bestätigen auch Studien von \citet{Niemann2011} zu einer norddeutschen Varietät. Sie fanden keinen Unterschied in der CV-Koordination bei Zielwörtern in akzentuierter (kontrastiver \isi{Fokus}) und unakzentuierter Position. Wortinitial wurden konsonantische und vokalische Gesten in allen Konditionen synchron initiiert.

\newpage 
Demnach sind lexikalische Töne im \ili{Mandarin} vollständig im silbeninternen Kopplungsnetzwerk integriert und fungieren hier wie zusätzliche Konsonanten. Demgegenüber sind prosodische (post-lexikalische) Tonakzente im Katalanischen mit der \isi{Silbe} gekoppelt (oder mit dem \isi{Vokal}), ohne jedoch die silbeninternen Kopplungsverhältnisse zu beeinflussen. Die Ergebnisse der Daten des \il{Deutsch!Wien}Wiener Deutschen betonen noch stärker die Unterschiede zwischen Tonakzenten und lexikalischen Tönen. Aufgrund der konkurrierenden Zielspezifikationen zwischen der tonalen Sequenz (L-H) und dem \isi{Vokal} entsteht eine starke Verzögerung zwischen dem Auftreten der H-\isi{Geste} verglichen zur vokalischen \isi{Geste}. Dennoch wird davon nicht das CV-Timing beeinflusst, denn konsonantische und vokalische \isi{Geste} starten auch im \il{Deutsch!Wien}Wiener Deutschen synchron. Die Kopplungsrelationen zwischen \isi{Tonakzent} und \isi{Silbe} (vokalische \isi{Geste} als Trigger) beeinflussen anders als im \ili{Mandarin} nicht die silbeninternen Kopplungsrelationen.

\section{Ausblick: Split-Gesten als Anker für tonale Gesten?}
\label{sec:0703}

Mehr Möglichkeiten, um Variationen in der Koordination von tonalen und oralen Gesten zu modellieren, besteht in der Annahme von Split-Gesten. Split-Gesten unterteilen beispielsweise konsonantische Gesten in Verschluss- und Lösungsgeste (\citealt{Nam2007b,Nam2007a}, \citealt{Pouplier2011a}).%%\citep{Nam2007a,Nam2007b,Pouplier2011a}%%

Neben dem oralen Verschluss wird dann zusätzlich die Lösungsgeste modelliert. Dieser Ansatz ist entwickelt worden, um Asymmetrien bei der Konsonantenproduktion abzubilden, beispielsweise, wenn eine Verschlussgeste durch die nachfolgende Öffnung trunkiert worden ist \citep{Harrington1995}. Der Split-Gesten Ansatz lässt sich aber auch für die \isi{Modellierung} von Tönen und oralen Gesten anwenden, da er ermöglicht, sowohl den Verschluss als auch die Lösung einer oralen Konstriktionsgeste als Anker bzw. Koordinationspunkt zu verwenden. Für die \isi{Alignierung} steigender nuklearer Akzente im \il{Deutsch!Wien}Wiener Deutschen könnte für den späten Gipfel dann auch eine Koordination mit der Lösungsgeste des initialen Konsonanten angenommen werden. Aus dieser Perspektive könnte man sagen, dass im Katalanischen die tonale Hochtongeste in steigenden Nuklearakzenten mit der Verschlussgeste des initialen Konsonanten In-Phase gekoppelt ist, während sie im \il{Deutsch!Wien}Wiener Deutschen mit der Lösungsgeste des initialen Konsonanten In-Phase gekoppelt ist. Diese Sichtweise geht aber von Beobachtungen in CV-Silben aus, die nicht mehrere Konsonanten im Silbenonset haben. Bei komplexen Onsets könnte sich das Realisierungsmuster ändern.

Was spricht für die Annahme einer Lösungsgeste in der Artikulatorischen Phonologie? \citet{Browman1992b} haben bereits aufgezeigt, dass die Lösung eines Verschlusses durch die Annahme einer \isi{Geste} kontrolliert werden müsse. Diese Idee weicht von dem Grundmodell der Artikulatorischen Phonologie ab, bei dem sich orale Gesten nach Verschlusslösung auf eine Neutralposition als nicht näher spezifizierten Parameter zubewegen. \citet{Nam2007b} zeigt jedoch, dass die konsonantische REL-\isi{Geste} Besonderheiten in den Parametern \isi{Steifheit} und \isi{Target} aufweist. Wenn beispielsweise Konsonant und \isi{Vokal} die gleiche \isi{Traktvariable} teilen (/ka/, /ki/, \isi{Traktvariable} TB), so entspricht im \isi{Onset} die \isi{Steifheit} von REL der \isi{Steifheit} eines Konsonanten. Das \isi{Target} von REL jedoch entspricht dem \isi{Target} des begleitenden Vokals.

Werden orale Konsonanten in Verschluss- und Lösungsgeste aufgeteilt, so besteht dann jeder Konsonant aus zwei Gesten (CLO und REL), die eigene Kopplungen beispielsweise mit dem \isi{Vokal} eingehen \citep{Nam2007b, Nam2007a}. Abbildung~\ref{figure:0723} zeigt die Kopplungsgraphen für CV und VC mit den konsonantischen Split-Gesten CLO und REL. Im \isi{Onset}, CV, sind beide Split-Gesten in einer In-Phase-\isi{Kopplung} mit dem \isi{Vokal} und in Anti-Phase zueinander. Sie verhalten sich streng genommen wie ein Cluster, obwohl es sich nur um einen einzelnen Konsonanten handelt. In der Nukleus-Koda-Relation, VC, hingegen ist nur der Verschluss CLO direkt mit dem \isi{Vokal} in Anti-Phase gekoppelt und es folgt sequentiell die Lösung REL.

\begin{figure}
% 	\includegraphics[width=\textwidth]{figures/7-23_SplitGeste.png}
\begin{tikzpicture}
\node[circle,draw, minimum size=9mm] (clo1)  []  {\textsc{clo}};
\node[circle,draw, minimum size=9mm] (rel1)  [right=of L]  {\textsc{rel}};
\node (V1) [circle,draw, minimum size=9mm,below = of $(kL)!0.5!(kH)$,yshift=4mm]  { V};
\node()[above of=clo1,xshift=9mm,yshift=.5mm]{\Large CV};
\draw[dotted] (clo1) -- (rel1);
\draw 	      (V1) -- (clo1);
\draw 	      (V1) -- (rel1);

\node[circle,draw, minimum size=9mm, right=of rel1] (V2)  []  {V};
\node[circle,draw, minimum size=9mm] (clo2)  [right=of V2]  {\textsc{clo}};
\node[circle,draw, minimum size=9mm] (rel2)  [right=of clo2]  {\textsc{rel}};
\node()[above of=clo2,yshift=.5mm]{\Large VC};
\draw[dotted] (V2) -- (clo2);
\draw[dotted] (clo2) -- (rel2);
\end{tikzpicture}

	\caption{Kopplungsgraphen CV und VC nach \citet{Nam2007a} mit Split-Gesten für den Konsonanten C (CLO = closure, REL = release). Durchgezogene Linien = In-Phase, gestrichelte Linien = Anti-Phase.}
	\label{figure:0723}
\end{figure}

Die Organisation in CV kann unterschiedliche Regularitäten abbilden. So startet im \isi{Onset} die vokalische Bewegung häufig erst nach dem Beginn des konsonantischen Verschlusses. Diese Abfolge CLO-V-REL ergäbe sich dann aus der zentrumsartigen Organisation (C-Center) von CLO und REL mit dem \isi{Vokal}.

Durch die konkurrierenden Zielspezifikationen wird CLO relativ zum \isi{Vokal} nach links verschoben und startet somit früher. Auch führt die feste Bindung in der „molekularen Struktur“ in CV zu einer erhöhten Stabilität in der zeitlichen Koordination. So hat \citet{Byrd1996b} beobachtet, dass es weniger \isi{Variabilität} in der Verschlussdauer von Konsonanten im \isi{Onset} als in der Koda gibt. Der Ansatz der Split-Gesten führt noch einen weiteren Schritt weg von dem traditionellen Segmentbegriff \citep{Pouplier2011a} und eröffnet, wie eingangs erörtert, neue Möglichkeiten in der \isi{Kopplung} oraler und glottaler Gesten, da sie mehr Anker- oder Koordinationspunkte zur Verfügung stellt.