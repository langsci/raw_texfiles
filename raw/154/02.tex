\chapter{Gestenpartituren}
\label{chap:02}

Die Gestenpartituren bilden den Output des Linguistischen Gestenmodells. Sie bilden die höhere Struktur von Gesten und somit deren intergesturale Koordination ab. Dabei wird bei der zeitlichen Koordination nicht von einem externen „Trigger“ (external clock) ausgegangen, sondern vielmehr von einer Koordination der dynamischen Zustände der Gesten untereinander (\isi{Phasing}; vgl. \citealt{Kelso1987, Browman1991a, Browman1992a}). Auf der vertikalen Achse befinden sich die \isi{Traktvariablen}, auf der horizontalen Achse die diskreten \isi{Aktivierungsintervalle} einer \isi{Geste} mit den entsprechenden dynamischen Parametern. Die Gestenpartituren dienen als Input für das Task-Dynamic-Modell (vgl. Abbildung~\ref{figure:0102}), bei dem dann die kontextabhängigen Kurvenverläufe der Artikulatoren generiert werden.

Gestenpartituren sind sprachspezifisch. Mit ihrer Hilfe werden diskrete phonologische Kontraste sowie graduelle kontextbedingte Variationen generiert. Letztere sind auf syntagmatischer Ebene nicht das Ergebnis phonologischer Reorganisationen sondern die Konsequenz unterschiedlicher Überlappungsgrade zwischen invarianten Gesten.

\begin{quote}
	Much of the richness of phonological structure, in the gestural framework, lies in the patterns of how gestures are coordinated in time with respect to one another (…). Utterances comprised of the same gestures may contrast with one another in how the gestures are organized, i.e. the same gestures can form different constellations. \citep[][162]{Browman1992a}
\end{quote}



\begin{figure}
	\includegraphics[height=.15\textheight]{figures/2-1_tea.png}
	\caption{Gestenpartituren mit Gestenspezifikationen und Traktvariablen für <tea> im Englischen; VEL = Velum, TT = Zungenspitze, TB = Zungenrücken, GLO = Glottis.}
	\label{figure:0201}
\end{figure}

Abbildung~\ref{figure:0201} zeigt eine \isi{Gestenpartitur} für die \il{Englisch}englische Äußerung <tea>. Die Partituren sind vom Prinzip her wie Musikpartituren aufgebaut: Auf der vertikalen Achse befinden sich die \isi{Traktvariablen} (im vorliegenden Beispiel VEL = Velum, TT = Zungenspitze, TB=Zungenrücken und GLO = Glottis) ähnlich wie die musikalischen Einzelstimmen und auf der horizontalen Achse die Zeit. Die gestischen \isi{Aktivierungsintervalle} stellen in Form von Kästchen dar, welche Aufgabe die \isi{Traktvariable} jeweils ausführen soll. Das ist ähnlich wie Noten, mit denen musikalische Aufgaben für Einzelstimmen notiert werden. Die gestischen Bewegungsaufgaben sind in Form von Deskriptoren kodiert, wie beispielsweise ein alveolarer Verschluss der Zungenspitze für die Bildung von /t/, {TT alveolar closure} in <tea> in Abbildung~\ref{figure:0201} (links). Zu Anfang eines Aktivierungsintervalls beginnt die \isi{Traktvariable} mit der Ausführung der \isi{Bewegungsaufgabe}, die am Ende des Intervalls erreicht sein sollte. Nachdem die \isi{Geste} nicht mehr aktiv ist, wendet sich die zugehörige \isi{Traktvariable} entweder einer neuen \isi{Bewegungsaufgabe} zu oder sie wird deaktiviert und bewegt sich zurück in die Neutralstellung. Der Default entspricht in diesem Modell der Neutralposition des Vokaltraktes bei der Produktion des Zentralvokals Schwa, /ə/. Hier finden keine speziellen Engebildungen im Vokaltrakt statt; die Hohlraumkonfigurationen entsprechen am ehesten denen eines einseitig geöffneten Rohres. Spontane Stimmhaftigkeit sowie das Anheben des Velums zur Abkopplung des Nasentraktes gehören ebenfalls zum Default, und werden hier nicht gekennzeichnet. Bei der \isi{Gestenpartitur} wird auch das zeitliche Zusammenspiel von glottalen und oralen Gesten während der Produktion für /t/ deutlich. So ist die glottale \isi{Geste} für Stimmlosigkeit {GLO wide} noch aktiv, obwohl Zungenspitzengeste {TT alveolar closure} bereits beendet ist. Auf der akustischen Oberfläche entsteht somit \isi{Aspiration}.

Für eine bessere Übersicht bietet es sich an, jeweils nur die \isi{Traktvariablen} in der Partitur abzubilden, die aufgrund der gestischen Spezifikationen von Bewegungsaufgaben angesteuert werden -- in einer musikalischen Partitur gibt es schließlich auch keine \enquote{leeren} Einzelstimmen für Instrumente, die nicht mitspielen. 


\section{Lexikalische Kontraste}
\label{sec:0201}
Es gibt drei grundlegende Prinzipien der gesturalen Organisation, um lexikalische Kontraste wie <packen> und <backen> innerhalb einer Sprache zu bilden. 


\subsection{Prinzip 1: An- oder Abwesenheit von Gesten}
\label{subsec:020101}

Beim ersten Prinzip entscheidet allein die An- oder Abwesenheit der \isi{Geste} über den linguistischen Kontrast. 

\begin{description}
	\item[Prinzip (1):] Linguistische Kontraste entstehen durch die An- oder Abwesenheit von Gesten.
\end{description}

Abbildung~\ref{figure:0202} illustriert dieses Prinzip für die \il{Englisch}englischen Äußerungen <tea>, <dee> und <knee> in Form einer \isi{Gestenpartitur}. Von oben nach unten zeigt die Abbildung die \isi{Traktvariablen} VEL = Velum, TT = Zungenspitze, TB = Zungenrücken und GLO = Glottis eingezeichnet. Die Kästchen schematisieren die gestischen \isi{Aktivierungsintervalle} für die jeweiligen \isi{Traktvariablen}; zu Beginn eines Intervalls startet die \isi{Traktvariable} mit der \isi{Bewegungsaufgabe} und am Ende des Intervalls hat sie das Ziel erreicht bzw. beendet die Aufgabenausführung. Die Deskriptoren in den Kästchen kodieren die Bewegungsaufgaben mit \isi{Konstriktionsgrad} und -ziel, beispielsweise /n/ = {TT alveolar closure} und {VEL wide} in <knee>.


\begin{figure}
	\includegraphics[width=\textwidth]{figures/2-2_tea_dee_knee.png}
	\caption{Gestenpartituren mit Gestenspezifikationen und Traktvariablen für <tea>, <dee> und <knee> im Englischen; VEL = Velum, TT = Zungenspitze, TB = Zungenrücken, GLO = Glottis.}
	\label{figure:0202}
\end{figure}


Vergleichen wir nun die drei Gestenpartituren in Abbildung~\ref{figure:0202} im Hinblick auf die Bildung lexikalischer Kontraste miteinander. Alle drei Äußerungen, <tea>, <dee> und <knee>, werden mit einem geschlossenen Vorderzungenvokal /i/ gebildet, bei dem jeweils der Zungenrücken TB involviert ist, {TB narrow palatal}. Obwohl sie sich \isi{segmental} in den initialen Konsonanten /t/, /d/, /n/ unterscheiden, haben diese Wörter gleiche orale Gesten im Mundraum auszuführen, beginnend mit dem Vollverschluss der Zungenspitze an den Alveolen {TT closure alveolar}. Die Äußerungen unterscheiden sich nur durch die An- oder Abwesenheit einer glottalen bzw. velischen \isi{Geste}. So unterscheidet sich <tea> von <dee> aufgrund der glottalen Abduktionsgeste {GLO wide} zur Produktion von Stimmlosigkeit für /t/, und <dee> und <knee> durch die Aktivierung des Velums {VEL wide} zur Öffnung des nasalen Traktes für /n/. Die vokalische \isi{Geste} und die Gesten des initialen Konsonanten starten gleichzeitig, aber die vokalische \isi{Geste} ist aufgrund geringerer Ausführungsgeschwindigkeiten länger aktiviert.



\begin{figure}[b]
	\includegraphics[width=\textwidth]{figures/2-3_Fahnefadeschade.png}
	\caption{Gestenpartituren mit Gestenspezifikationen und Traktvariablen für <Fahne>, <fade> und <schade> im Deutschen; LA = Lippenöffnung, TT = Zungenspitze, TB = Zungenrücken, GLO = Glottis.}
	\label{figure:0203}
\end{figure}



Abbildung~\ref{figure:0203} gibt ein Beispiel aus dem \ili{Deutsch}en, <Fahne> /fa:nə/, <fade> /fa:də/ und <schade>  /ʃa:də/. Zwischen <Fahne> und <fade> bestimmt die An- bzw. Abwesenheit der Velumsgeste {VEL wide} zu Beginn der zweiten \isi{Silbe} über den lexikalischen Kontrast. Zwischen <fade> und <schade> liegen unterschiedliche Gesten für den initialen Konsonanten im oralen System vor: für /f/ wird ein labiodentaler Beinahverschluss der der Lippen {LA critical dental} und für /ʃ/ ein postalveolarer Beinahverschluss der der Zungenspitze {TT critical alveolar} spezifiziert. Der \isi{Vokal} in der zweiten \isi{Silbe}, der Schwalaut, ist hier als {TB mid uvu-pharyngeal} spezifiziert (der \isi{Konstriktionsgrad} ist „mid“, um ihn vom tiefen Schwa {TB wide uvu-pharyngeal} im \ili{Deutsch}en unterscheiden zu können). Schwa entspricht im Grunde dem Default des Modells und müsste nicht unbedingt spezifiziert werden. Jedoch wird für die zweite \isi{Silbe} eine \isi{Bewegungseinheit} benötigt, die als zeitlicher Trigger für die Aktivierung des Systems fungiert, so dass eine Spezifizierung sinnvoll scheint.

\newpage 
Da die Äußerungen in Abbildung~\ref{figure:0203} aus jeweils zwei Silben bestehen, wird an diesen Beispielen bereits deutlich, dass die \isi{Aktivierungsintervalle} für die Vokale nahtlos aneinander anschließen, so dass die Produktion der Vokale die der Konsonanten vollständig überlagert. Diese Überlagerung manifestiert sich an akustischer Oberfläche u.a. durch Formanttransitionen und fließende Segmentgrenzen, die den zugrundeliegenden Vokalzyklus widerspiegeln. Die Beobachtung, dass es einen zugrundeliegenden Vokalzyklus gibt, der von konsonantischen Verschlüssen überlagert wird, ist bereits in \citet{Öhman1966} beschrieben und wurde u.a. von \citet{Fowler1977, Fowler1980} vertieft. So hat \citet{Öhman1966} anhand von akustischen Studien zu V1CV2-Sequenzen gezeigt, dass sich direkte koartikulatorische Effekte von V2 bereits in V1 finden, und das umgekehrt V1 auch V2 beeinflusst, obwohl es einen intervokalischen Konsonanten gibt.


\subsection{Prinzip 2: Unterschiede in gestischen Deskriptoren}
\label{subsec:020102}

Beim zweiten Prinzip entscheiden die parametrischen Spezifikationen -- die gestischen Deskriptoren -- über den linguistischen Kontrast, beispielsweise ein Vollverschluss für einen Plosiv gegenüber eines Beinah-Verschlusses für einen Frikativ.

\begin{description}
	\item[Prinzip (2):] Linguistische Kontraste entstehen aufgrund von unterschiedlichen gestischen Deskriptoren bzw. Parametern (z.\,B. CD, CL).
\end{description}

Um das zweite Prinzip zu veranschaulichen wird die \il{Englisch}englische Äußerung <tea> in Abbildung~\ref{figure:0204} den \il{Englisch}englischen Äußerungen <sea> und <she> gegenübergestellt. Diesmal unterscheiden sich die Äußerungen nicht durch die An- oder Abwesenheit von Gesten sondern vielmehr durch die parametrische Spezifikation des initialen Konsonanten. Zwischen <tea> und <sea> besteht der Kontrast in dem Grad der Zungenspitzenkonstriktion (CD, constriction degree), d.\,h. im Vollverschluss {TT alveolar closure} für /t/ im Gegensatz zu einem Beinah-Verschluss {TT alveolar critical} für /s/. Zwischen <sea> und <she> liegt der Unterschied in dem Ort der Konstriktion (CL, constriction location), d.\,h. zwischen alveolar {TT alveolar critical} für /s/ im Gegensatz zu postalveolar {TT postalveolar critical} für /ʃ/.



\begin{figure}[t]
	\includegraphics[height=.15\textheight]{figures/2-4_tea_see_she.png}
	\caption{Gestenpartituren mit Gestenspezifikationen und Traktvariablen für <tea>, <sea> und <she> im Englischen; TT = Zungenspitze, TB = Zungenrücken, GLO = Glottis. 
	}
	\label{figure:0204}
\end{figure}


Dieses Prinzip greift auch bei dem folgenden Beispiel, <Diebe> /di:bə/, <Siebe> /zi:bə/ und <schiebe> /ʃi:bə/ im \ili{Deutsch}en (Abbildung~\ref{figure:0205}). Das Gestentableau für die initialen Konsonanten in /di:bə/ und /zi:bə/ unterscheidet sich nur in der Spezifizierung des Konstriktionsgrades der Zungenspitze, CL. In /di:bə/ ist ein Vollverschluß spezifiziert {TT alveolar closure} und in /zi:bə/ ein Beinahverschluss {TT alveolar critical} spezifiziert. Beim Vergleich von /zi:bə/ und /ʃi:bə/ unterscheiden sich die Gesten für die initialen Konsonanten nur im Deskriptor für den Konstriktionsort, CL. In /zi:bə/ wird die Konstriktion an den Alveolen {TT alveolar critical} und in /ʃi:bə/ postalveolar {TT postalveolar critical} spezifiziert.


\begin{figure}[t]
	\includegraphics[height=.2\textheight]{figures/2-5_SiebeDiebe.png}
	\caption{Gestenpartituren mit Gestenspezifikationen und Traktvariablen für <Diebe>, <Siebe> und <schiebe> im Deutschen; LA = Lippenöffnung, TT = Zungenspitze, TB = Zungenrücken.}
	\label{figure:0205}
\end{figure}


\subsection{Prinzip 3: Phasing}
\label{subsec:020103}
Das dritte Prinzip betrifft die Koordination (\isi{Phasing}) der Gesten zueinander, siehe Abbildung~\ref{figure:0206}. Hier entscheidet allein die Koordination der Gesten zueinander über den linguistischen Kontrast. 

\begin{description}
	\item[Prinzip (3):] Linguistische Kontraste entstehen aufgrund unterschiedlicher Koordinationen zwischen Gesten (\isi{Phasing}).
\end{description}

\begin{figure}
	\includegraphics[height=.15\textheight]{figures/2-6_buddub.png}
	\caption{Gestenpartituren mit Gestenspezifikationen und Traktvariablen für <bud>, <dub> im Englischen; TT = Zungenspitze, TB = Zungenrücken, LA = Lippenöffnung.}
	\label{figure:0206}
\end{figure}

\begin{figure} 
	\includegraphics[height=.2\textheight]{figures/2-7_BohneMode.png}
	\caption{Gestenpartituren mit Gestenspezifikationen und Traktvariablen für <Bohne> und <Mode> im Deutschen; VEL = Velum, LA = Lippenöffnung, TT = Zungenspitze, TB = Zungenrücken.}
	\label{figure:0207}
\end{figure}

So bestehen die \il{Englisch}englischen Äußerungen <bud> und <dub> aus der gleichen Gestenauswahl, jedoch sind die konsonantischen Gesten für den labialen Verschluss in {LA labial closure} und den alveolaren Verschluss {TT alveolar closure} zeitlich unterschiedlich mit der Vokalgeste {TB wide uvular} koordiniert. Das dritte Prinzip greift auch im \ili{Deutsch}en in den Äußerungen <Bohne> /bo:nə/ und <Mode> /mo:də/. In Abbildung~\ref{figure:0207} sind die beiden Gestenpartituren gegenübergestellt. Es zeigt sich, dass beide Partituren aus dem gleichen Set an Gesten bestehen, jedoch die Velumsgeste {VEL wide} in /bo:nə/ mit dem initialen Konsonanten der ersten \isi{Silbe} {LA closure}und in /mo:də/ mit dem initialen Konsonanten der zweiten \isi{Silbe} {TT alveolar closure} synchronisiert ist.



\subsection{Beispielpartituren}
\label{subsec:020104}
\begin{figure}[t]
	\includegraphics[width=\textwidth]{figures/2-8_GestenpartiurenASS.png}
	\caption{Gestenpartituren mit Gestenspezifikationen und Traktvariablen für verschiedene Äußerungen des Deutschen; VEL = Velum, LA = Lippenöffnung, TT = Zungenspitze, TB = Zungenrücken, GLO = Glottis}.
	\label{figure:0208}
\end{figure}

Die folgenden Gestenpartituren in Abbildung~\ref{figure:0208} verdeutlichen die Anwendung der ersten drei Prinzipien in unterschiedlichen Äußerungen des \ili{Deutsch}en. So unterscheiden sich <das> und <nass> durch die Anwesenheit bzw. Abwesenheit der velischen \isi{Geste} {VEL wide}, die in <nass> mit der alveolaren Vollverschlussgeste {TT alveolar closure} zeitlich synchronisiert ist, jedoch in <das> nicht auftritt. Die Äußerungen <bass> und <das> unterscheiden sich ebenfalls durch das Auftreten unterschiedlicher Gesten im Silbenanlaut. So ist der initiale Plosiv in <das> durch die alveolare Vollverschlussgeste {TT alveolar closure} spezifiziert, während <Bass> im Anlaut eine labiale Vollverschlussgetse {LA labial closure} aufweist. Die Äußerungen <Bass> und <Fass> weisen beide im Anlaut eine Konstriktionsgeste auf, die der \isi{Traktvariablen} Lippenöffnung (LA) zugeordnet ist. Die Gesten unterscheiden sich jedoch in beiden Deskriptoren (CL, constriction location und CD, constriction degree). Bei <Bass> handelt es sich um eine labiale Vollverschlussgeste {LA labial closure} und bei <Fass> um eine labiodentale Beinahverschlussgeste {LA dental critical}. <Bass> und <Pass> weisen das gleiche Tableau an glottalen Gesten auf, in <Pass> jedoch kommt eine \isi{glottale Abduktionsgeste} {GLO wide} für Stimmlosigkeit hinzu. Bei <Pass> endet die glottale \isi{Geste} später als die orale Konstriktionsgeste, so dass auf akustischer Oberfläche \isi{Aspiration} entsteht. Vergleicht man <Hass> und <Pass>, so unterscheiden sich die beiden Äußerungen darin, dass in <Pass> ein Lippenvollverschluss {LA labial closure} des oralen Systems spezifiziert ist, in <Hass> aber nicht.  

\section{Kontextbedingte Variation}
\label{sec:0202}

Neben der Bildung lexikalischer Kontraste spielt die \isi{kontextbedingte Variation} eine wichtige Rolle in der \isi{Modellierung} der Sprachproduktion. Anhand von traditionellen phonologischen Merkmalen lassen sich allophonische Repräsentationen wie der Grad der \isi{Aspiration} nicht abbilden. Phonologische Merkmale basieren auf kategorialen Darstellungen, die weder graduelle Unterschiede zwischen Sprachen noch kontextbedingte Variationen innerhalb einer Sprache erfassen können. 

  
Kontextbedingte Variationen können beispielsweise den jeweiligen Grad einer \isi{Aspiration} (Behauchung) von Plosiven innerhalb einer Sprache betreffen, also sich in der glottal-oralen Kontrolle manifestieren. So ist phonologisch innerhalb einer Sprache mittels des Merkmals [${\;\pm\;}$spread glottis] spezifiziert, ob ein Plosiv aspiriert ist, aber der Grad der \isi{Aspiration} kann beispielsweise in Folge von kontextuellen und prosodischen Einflüsse systematisch variieren. So spezifiziert das Merkmal [${\;\pm\;}$spread glottis] lediglich, ob Plosive in einem bestimmten Sprachsystem aspiriert vorkommen oder nicht, d.\,h., ob die Glottis nach der Lösung des Plosivs noch offen ist und Aspirationsrauschen auf der akustischen Oberfläche erzeugt. 

Auch bei der Beschreibung von Prozessen wie \isi{Assimilation} und Tilgung wird traditionell von der vollständigen Änderung oder dem Wegfall eines Segments ausgegangen. Mit Hilfe von phonologischen Merkmalen lassen sich keine Zwischenstufen abbilden. So ist ein Konsonant entweder assimiliert <i[m] Berlin> oder nicht <i[n] Berlin>, wenngleich in der erstgenannten Variante, <i[m] Berlin>, häufig eine durch den Lippenverschluss verdeckte Zungenspitzengeste auftritt. Ähnliches gilt für die Tilgung. Entweder wird in einer Äußerung wie <Er hat Paris erreicht> auf segmentaler Ebene der alveolare Konsonant /t/ realisiert [hat paʁis] oder nicht [hapaʁis], wenngleich artikulatorisch häufig verschiedenste Zwischenformen beobachtbar sind (u.a. \citealt{Barry1991}; \citealt{Kohler1995}; \citealt{Ellis2002}; \citealt{Jaeger2007}; \citealt{Mücke2008c}; \citealt{Bergmann2008}). 

Im Gegensatz zu traditionellen Modellen sind im \isi{Gestenmodell} diese Formen kontinuierlicher Variationen beschreibbar. Bei der \isi{Modellierung} kontextbedingter Variationen werden Gesten nicht hinzugefügt oder weggenommen, sondern es ändert sich der Grad der \isi{Überlappung} zwischen zwei Gesten und/oder der Grad der Ausdehnung eines Gestenintervalls. Dabei ist es wichtig, neben dem rein segmentalen Kontext auch die \isi{prosodische Struktur} als relevanten Faktor für den Grad der kontextbedingten Variation einzubeziehen.

\subsection{Prinzip 4 und 5: Glottale und orale Koordination}
\label{subsec:020201}

Für die Koordination von glottalen und oralen Gesten in ausgesuchten westgermanischen Sprachen wie dem Englischen schlagen \citet{Browman1992a} zwei Ordnungsprinzipien vor, aus denen sich sprachspezifische Beschränkungen bezüglich ihrer temporalen Organisation ableiten lassen.

\begin{description}
	\item[Prinzip (4):] Bei stimmlosen Frikativen in wortinitialer Position tritt das Maximum der glottalen \isi{Öffnungsgeste} (peak glottal opening, glottaler Gipfel) zeitgleich mit der Mitte der Frikativgeste auf (midpoint of the fricative gesture). Bei stimmlosen Plosiven ist der glottale Gipfel mit der Lösung der Plosivgeste (release) synchronisiert.
\end{description}

Dieses Prinzip lässt sich artikulatorisch an den Beispielen <tea> und <sea> in Abbildung~\ref{figure:0209} illustrieren. In <sea> ist das Maximum der glottalen \isi{Öffnungsgeste} (der glottale Gipfel) mit der Mitte der Frikativgeste synchronisiert. In <tea> hingegen ist der glottale Gipfel erst mit dem Lösen des Vollverschlusses synchronisiert. Letzteres führt bei Plosiven in wortinitialer Position auf akustischer Ebene zur \isi{Aspiration} (\citealt{Browman1992a}, \citealt{Pouplier2011b}; \citealt{Yoshioka1981} für \ili{Englisch}; \citealt{Yeoul2008} für \il{Arabisch!marokkanisch}Marokkanisches Arabisch; vgl. auch \citealt{Hoole2006} für eine kritische Diskussion). 

\begin{figure}[t]
	\includegraphics[width=\textwidth]{figures/2-9_TEASEA.png}
	\caption{Gestenpartituren mit Gestenspezifikationen und Traktvariablen für <tea> und <sea> im \ili{Englisch}en; TT = Zungenspitze, TB = Zungenrücken, GLO = Glottis.}
	\label{figure:0209}
\end{figure}

Eine solche Koordination ließe sich auch für das Standarddeutsche annehmen, beispielweise für <Tal> und <Schal>, wobei /t/ in <Tal> erwartungsgemäß auf akustischer Oberfläche aspiriert wäre (Abbildung~\ref{figure:0210}). Artikulatorisch erwarten wir im Vergleich zu <Schal> einen späteren glottalen Gipfel relativ zur oralen \isi{Geste} (Abbildung~\ref{figure:0210}). 

\begin{figure}
	\includegraphics[width=\textwidth]{figures/2-10_TalSchal.png}
	\caption{Gestenpartituren mit Gestenspezifikationen und Traktvariablen für <Tal> und <Schal> im Deutschen; TT = Zungenspitze, TB = Zungenrücken, GLO = Glottis. Dass Sternchen verdeutlicht, dass es sich bei /l/ um einen lateralen Verschluss handelt.}
	\label{figure:0210}
\end{figure}

Dieses \isi{Koordinationsmuster}, das zur \isi{Aspiration} bei Plosiven führt, ist innerhalb einer Sprache oder Varietät festgelegt. Beispielsweise zeigen \citet{Sawashima1980}, dass bei nicht-aspirierten Plosiven im \ili{Französisch}en oder \ili{Hindi} die glottale \isi{Geste} mit der Plosivgeste zeitlich so synchronisiert, dass sie mit dem Lösen des Plosivs endet. Beide gestischen \isi{Aktivierungsintervalle} starten und enden gleichzeitig. Als Konsequenz ergibt sich keine \isi{Aspiration} auf akustischer Oberfläche, da mit oraler Verschlusslösung des Plosivs die Stimmhaftigkeit des glottalen Systems wiedereinsetzt. Es sei hier jedoch kurz angemerkt, dass die Landmarke für den glottalen Gipfel nicht unumstritten ist, da sie nicht immer im Signal bestimmbar ist (\citealt{Hoole2006}; \citealt{Pouplier2011a}).

Das fünfte Prinzip beschreibt die \isi{Modellierung} von wortinitialen /s/+Plosiv-Sequenzen in Sprachen wie dem \ili{Englisch}en \citep{Browman1986} oder \ili{Deutsch}en \citep{Hoole2006}. In diesen Sprachen werden Plosive, die auf Frikative folgen, häufig nicht aspiriert. 

\begin{description}
	\item[Prinzip (5):] Bei wortinitialen /s/+Plosiv-Clustern findet sich nur eine einzelne glottale \isi{Öffnungsgeste}.
\end{description}

Abbildung~\ref{figure:0211} zeigt die Äußerung <steal> /sti:l/ im \ili{Englisch}en und <Stahl> /ʃta:l/ im \ili{Deutsch}en. In beiden Fällen würde der Plosiv /t/ ohne \isi{Aspiration} realisiert, weil ihm jeweils ein stimmloser Frikativ vorangeht. Die glottale Aktivität ist hier im Hinblick auf das Konsonantencluster organisiert, d.\,h. die Glottis ist bereits während des Frikativs maximal geöffnet und nicht erst nach der Lösung des Folgeplosivs. Als Ergebnis setzen Stimme und orale Verschlusslösung für den Folgevokal gleichzeitig ein und die Stimmeinsatzzeit beträgt NULL. 



\begin{figure}
	\includegraphics[width=\textwidth]{figures/2-11_STEALSTAHL.png}
	\caption{Gestenpartituren mit Gestenspezifikationen und Traktvariablen für <steal> im Englischen und <Stahl> im Deutschen; TT = Zungenspitze, TB = Zungenrücken, GLO = Glottis. Dass Sternchen verdeutlicht, dass es sich bei /l/ um einen lateralen Verschluss handelt.}
	\label{figure:0211}
\end{figure}

\largerpage
In den genannten Beispielen findet sich also in Frikativ-Plosiv Sequenzen nur eine glottale \isi{Öffnungsgeste} (nur ein glottaler Gipfel) für das gesamte Cluster. Diese \isi{Geste} ist im Hinblick auf den maximalen Verschluss der Frikativgeste synchronisiert, so dass diese Cluster keine \isi{Aspiration} des nachfolgenden Plosivs zeigen. Es gibt jedoch sprachspezifische Variationen. Beispielsweise zeigen \citet{Munhall1988} für die \ili{Englisch}e Sequenz <Kiss Ted>, dass die Anzahl der glottalen Gipfel abhängig von der Sprechgeschwindigkeit ist, hier über eine Morphemgrenze hinweg. Bei langsamer Sprechrate treten zwei einzelne glottale Gipfel auf, die den Konsonanten /s/ und /t/ zugeordnet werden können. Bei zunehmender Sprechgeschwindigkeit werden die glottalen Gesten jedoch nach und nach ineinander geblendet und es tritt schlussendlich bei schneller \isi{Artikulationsrate} nur noch ein einzelner glottaler Gipfel auf. 

Im \ili{Tashlhiyt} Berber \citep{Ridouane2006} lassen sich in wortinitialen Sequenzen wie /sk/ mehrere glottale Gipfel nachweisen. Die Anzahl dieser Gipfel und somit die Aktivierung der glottalen \isi{Öffnungsgeste}(n) sind u.a. von der segmentalen Struktur des jeweiligen Clusters abhängig. Vermutlich resultieren sie auch aus Restriktionen der Silbenstruktur, denn \ili{Tashlhiyt} Berber erlaubt keine verzweigenden Onsets. Eine Äußerung wie <kfik> („gib dir selbst“) besteht aus zwei Silben, /k.fik/, bei dem der wortinitiale Konsonant /k/ nicht zur Folgesilbe zählt \citep{Hermes2011b,Hermes2011a}. 


\subsection{Reduktion und Assimilation}
\label{subsec:020202}

Reduktionen können zeitlicher und räumlicher Natur sein. Nach dem H\&H Modell \citep{Lindblom1990} nutzen Sprecher ein Kontinuum zwischen Hypo- und \isi{Hyperartikulation} aus und bringen in Abhängigkeit des kommunikativen Nutzens ein unterschiedliches Maß an artikulatorischem Aufwand auf. Bei sehr sorgfältigem Sprechen ist der Aufwand hoch und die Sprache \textit{hyper}artikuliert. Dabei wird mit einem größeren artikulatorischen Aufwand gesprochen, wodurch das Maß an \isi{Koartikulation} abnimmt. Bei verschliffenem Sprechen hingegen ist der Aufwand niedrig und die Sprache wird \textit{hypo}artikuliert. Der Grad an artikulatorischen Aufwand nimmt ab, und es treten mehr Reduktionserscheinungen auf der akustischen Oberfläche auf (vgl. auch \citealt{DeJong1993} und \citealt{Kröger1998}). Im Folgenden werden die Grundprinzipien der Reduktion und \isi{Assimilation} im Rahmen der Artikulatorischen Phonologie dargestellt. Auf den Zusammenhang von \isi{Assimilation} und prosodischer Struktur wird dann in Kapitel~\ref{chap:06} eingegangen.

Reduktionen treten im \ili{Deutsch}en vermehrt in lexikalisch unbetonten Silben auf, die meist mit kürzeren Dauern und weniger distinkten Lautqualitäten artikuliert sind. Hier können Vollvokale häufig stufenlos bis zum Neutralvokal Schwa reduziert werden. Die folgenden Prinzipien zeigen, wie sich Reduktionen im \isi{Gestenmodell} kontinuierlich modellieren lassen. Dabei kann zunächst wie im Prinzip (6) die dem Segment zugehörige \isi{Geste} zeitlich und räumlich modifiziert werden.

\begin{description}
	\item[Prinzip (6):] Reduktionsformen können durch räumliche und zeitliche Reduktion der Gestengröße modelliert werden.
\end{description}

Eine Form der Reduktion ist die Tilgung eines Segmentes auf akustischer Oberfläche. Ein getilgtes Segment ist in der Äußerung nicht mehr hörbar und fällt somit aus segmentaler Sicht weg. Artikulatorisch gesehen fallen die dem Segment zugehörigen Gesten jedoch nicht weg, sondern sind vielmehr vollständig durch andere Gesten überlappt, vgl. Prinzip (7). 

\begin{description}
	\item[Prinzip (7):] Gesten unterschiedlicher \isi{Traktvariablen} überlappen sich bis hin zur Verdeckung. Auf der akustischen Oberfläche kommt es zur Tilgung oder \isi{Assimilation} (gestural hiding).
\end{description}



Abbildung~\ref{figure:0212} zeigt Gestenpartituren für <nicht mal> in kanonischer [nɪçt mal] und verschliffener [nɪçmal] Aussprache der \il{Deutsch}deutschen Äußerung <Er geht nicht mal einkaufen>. Die Gestenpartituren zeigen für eine bessere Übersicht nur das orale System. In kanonischer Aussprache findet keine \isi{Überlappung} über Wortgrenzen hinweg statt, in dem Beispiel der verschliffenen Sprache dagegen eine vollständige \isi{Überlappung}. Die Gesten befinden sich auf unterschiedlichen Ebenen, d.\,h. sie steuern die unterschiedlichen \isi{Traktvariablen} Zungenspitze (TT) und Lippenöffnung (LA) an. Der alveolare Verschluss {TT alveolar closure} für /t/ wird von dem bilabialen Verschluss {LA labial closure} für /m/ verdeckt (gestural hiding); auf der akustischen Oberfläche kommt es zur Tilgung des Segments /t/. 

\begin{figure}[t]
	\includegraphics[width=\textwidth]{figures/2-12_PartiturNICHTMAL.png}
	\caption{Gestenpartituren der Äußerung <nicht mal> im Deutschen in kanonischer und verschliffener Form; orales System: TT = Zungenspitze, TB = Zungenrücken. Das Sternchen verdeutlicht, dass es sich bei /l/ um einen lateralen Verschluss handelt.}
	\label{figure:0212}
\end{figure}

Eine weitere Form der Reduktion ist die \isi{Assimilation}, bei der auf akustischer Oberfläche ein Segment in mindestens einem phonologischen Merkmal einem Nachbarsegment angeglichen wird. Abbildung~\ref{figure:0213} zeigt Gestenpartituren für <bat mich> in kanonischer [ba:t mɪç] und verschliffener [ba:pmɪç] Aussprache der \il{Deutsch}deutschen Äußerung <Er bat mich rein>. Auf akustischer Oberfläche ist eine \isi{Assimilation} des alveolaren Plosivs an den folgenden Labiallaut über die Wortgrenze hinweg zu erwarten. Aus gestischer Sicht wird der alveolare Verschluss {TT alveolar closure} für /t/) durch den labialen Verschluss {LA labial closure} für /m/ vollständig verdeckt (gestural hiding). Auf der akustischen Ebene führt das Verdecken dieser \isi{Geste} zur Ortsassimilation, und /t/ wird als [p] perzipiert, vgl. Prinzp (7). Evidenzen für verdeckte Gesten finden sich in kinematischen Studien zur Ortsassimilation im Deutschen \citep{Kohler1995, Jaeger2007, Bergmann2008, Mücke2008c} und \ili{Englisch}en \citep{Barry1991, Ellis2002}. In diesen Studien wird unter Einsatz von Elektropalatographie (EPG) und Elektromagnetischer \isi{Artikulographie} gezeigt, dass koronale Gesten artikulatorisch häufig noch vorhanden sind, auch wenn sie akustisch-perzeptiv überdeckt sind. Dabei kann unterschieden werden zwischen vollständiger Ausführung der Gesten (keine \isi{Assimilation}), abgeschwächte Ausführung der Gesten (partielle \isi{Assimilation}) und Ausfall der Gestenausführung (segmentale Substitution).

\begin{figure}[t]
	\includegraphics[height=.2\textheight]{figures/2-13_PartiturBATMICH.png}
	\caption{Gestenpartituren der Äußerung <bat mich> im Deutschen in kanonischer und verschliffener Form; orales und velisches System: LA = Lippenöffnung, TT = Zungenspitze, TB = Zungenrücken, VEL = Velum.}
	\label{figure:0213}
\end{figure}

 \largerpage
Gesten derselben Ebene können jedoch nicht stärker überlappen, weil sie die gleiche \isi{Traktvariable} ansteuern. Sie liegen miteinander im Wettbewerb um diese \isi{Traktvariable} \citep{Browman1989}. In diesem Fall werden Gesten überblendet \citep{Fowler1993}. Durch Blending können ebenfalls Tilgungen und Assimilationen auf der akustischen Oberfläche entstehen. Das haben beispielsweise \citet{Munhall1988} für die glottale und orale Koordination im \ili{Englisch}en gezeigt. Hier werden bei zunehmender Sprechgeschwindigkeit die beiden glottalen Gesten für Stimmlosigkeit in /s/ und /t/ in der Sequenz <Kiss Ted> über die Morphemgrenze hinweg ineinander geblendet werden bis nur noch ein einzelner glottaler Gipfel auftritt (vgl. Kapitel~\ref{subsec:020201}).
 
\begin{description}
	\item[Prinzip (8):] Gesten, die gleiche \isi{Traktvariablen} ansteuern, werden ineinander geblendet. Auf der akustischen Oberfläche kommt es zur Tilgung oder \isi{Assimilation} (gestural blending).
\end{description}

Abbildung~\ref{figure:0214} gibt ein Beispiel für Blending, bei der Gestenpartituren für <das Spiel> der \il{Deutsch}deutschen Äußerung <Das Spiel endete mit einem Unentschieden> einmal in kanonischer [das ʃpi:l] und einmal in verschliffener Aussprache [daʃ ʃpi:l] für das orale System dargestellt sind. Die Gesten {TT alveolar critical} für /s/ und {TT postalveolar closure} für /ʃ/ werden über die Wortgrenze hinweg ineinander geblendet. Es findet eine Ortsassimilation auf akustischer Oberfläche statt, die hier mit einem diskreten Symbolwechsel <da[ʃ ʃ]piel> statt <da[s ʃ]piel> transkribiert ist. Dennoch finden insbesondere in Fällen von Sibilanten in der Sprechrealität selten vollständige Assimilationen statt. Meist handelt es sich weniger um eine Gemination, als um eine Doppelartikulation. Letztere zeigt sich in der Akustik durch Transitionen im Frikativspektrum, vgl. auch das Beispiel <This shop is a fish shop> von \citet{Holst1995}, bei dem sich die Sibilantensequenz in <This shop> qualitativ von <fish shop> unterscheidet.

\begin{figure}
	\includegraphics[width=\textwidth]{figures/2-14_PartiturDASSPIEL.png}
	\caption{Gestenpartituren der Äußerung <das Spiel>im Deutschen in kanonischer und verschliffener Form; orales System: LA = Lippenöffnung, TT = Zungenspitze, TB = Zungenrücken. Das Sternchen verdeutlicht, dass es sich bei /l/ um einen lateralen Verschluss handelt.}
	\label{figure:0214}
\end{figure}

\newpage 
In den aufgeführten Fällen führt eine Zunahme des Überlappungsgrades zwischen Gesten zu den unterschiedlichen Reduktions- und Assimilationsformen, und die \isi{Modellierung} ist rein quantitativer Natur. Es sei jedoch angemerkt, dass die Annahme, dass alle Reduktions- und Assimilationsformen aus dem Grad der gestischen \isi{Überlappung} resultieren, auch in der Kritik steht. So liegen in besonders starken Reduktionsformen wie [mɪm] der kanonischen Äußerung des \ili{Deutsch}en <mit dem> vermutlich keine apikalen Gesten mehr vor \citep{Kohler1992}. \citet{Pouplier2007} zeigt darüber hinaus, dass auch Intrusionen von Gesten beobachtbar sind. Mit Hilfe von artikulographischen Aufnahmen, die sie im Rahmen der Versprecher-Forschung (speech error) für das Englische aufgezeichnet hat, kann sie eine Vielzahl von Insertionen von Gesten (gestural intrusion error) nachweisen. Im Gegensatz zu \citet{Kohler1992} stellt sie fest, dass Intrusionen eher zu beobachten seien als Tilgungen.