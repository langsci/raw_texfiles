\chapter{Einführung in die gesturale Analyse}
\label{chap:01}

\largerpage[2]
\begin{figure}[b]
	\includegraphics[width=.95\textwidth]{figures/1-1_Lina_Screenshot.png}
	\caption{Oszillogramm (oben) und vertikale Positionskurven für Zungenspitze (Mitte) und Zungenrücken (unten) in der Zielsilbe /li/ in dem Zielwort <Lina>.}
	\label{figure:0101}
\end{figure}
Gesprochene Sprache besteht aus überlappenden Bewegungseinheiten der artikulierenden Organe wie Zunge, Mundlippen, Kiefer und Glottis. Es ist anhand des Sprachsignals nicht möglich zu sagen, wo ein Laut endet und ein neuer anfängt. Vielmehr sind Segmente kontextabhängig und kodieren multiple Gesten, die miteinander zeitlich und räumlich koordiniert sind. Dieses Phänomen wird als \isi{Koartikulation} bezeichnet \citep{Menzerath1933, Mattingly1981, Farnetani1999}. Während sich \isi{Koartikulation} artikulatorisch durch die \isi{Überlappung} von verschiedenen konsonantischen und/oder vokalischen Bewegungseinheiten -- den artikulatorischen Gesten -- ausdrückt, zeigt sie sich akustisch durch die Beeinflussung der konsonantischen Transitionen durch die Umgebungsvokale \citep{Öhman1966}.

Wie stark die Laute bei der \isi{Artikulation} ineinander verzahnt sind, wird bei der direkten Beobachtung der \isi{Artikulation} im kinematischen Signal deutlich. Die Abbildung~\ref{figure:0101} veranschaulicht dieses Phänomen anhand der Zielsilbe /li/. Es handelt sich um die betonte \isi{Silbe} in <Lina> in der Äußerung <Er geht mit der \textbf{LI}na viel lieber>. Die Abbildung zeigt von oben nach unten das akustische Signal in Form eines Oszillogramms sowie die Positionskurven für die Bewegungen der Zungenspitze und des Zungenrückens. Es handelt sich jeweils um vertikale Positionskurven, die mit dem Öffnungsgrad des Vokaltraktes assoziiert sind, d.\,h. niedrige Werte stellen hier eine offene, und hohe Werte eine geschlossene Stellung der Artikulatoren dar. Die Bewegungsintervalle für Start und Ende der konsonantischen Bewegung sind grau schattiert: die Zungenspitze wird für den alveolaren Verschluss in /l/ angehoben, und der Zungenrücken wird für die Öffnung des Vokals /i/ angehoben. Beide Bewegungsintervalle starten im kinematischen Signal gleichzeitig; den Bewegungsstartpunkt bildet der vorangehende \isi{Vokal} (das tiefe Schwa in <der>). Allerdings wird die \isi{Bewegungsaufgabe} des Zungenrückens für /i/ langsamer als die der Zungenspitze für /l/ ausgeführt. Somit wird das Ziel für den \isi{Vokal} deutlich später erreicht. Obwohl sich die beiden Bewegungseinheiten vollständig überlappen, entsteht aufgrund der unterschiedlichen Ausführungsgeschwindigkeiten von Konsonanten und Vokalen auf der akustischen Oberfläche der Eindruck von einer Abfolge von Segmenten.


Die Gleichzeitigkeit von Konsonanten und Vokalen in CV-Silben wird in den traditionellen Analysen nicht berücksichtigt \citep{Mücke2016}. Diese verwenden meist sprachliche Grundeinheiten wie Merkmale oder Segmente, und betrachten die \isi{kontextbedingte Variation} häufig als einen rein phonetischen Effekt, der sich phonologisch über ein Set von Regeln und Algorithmen vorhersagen lässt. Neuere, dynamische Theorien hingegen betrachten Variation als Teil des linguistischen Systems, das konkret Aufschluss über zugrundeliegende Strukturen gibt. Hier wird keine künstliche Schnittstelle zwischen Phonetik und Phonologie angenommen, sondern die Repräsentationsebenen sind vollständig integriert. Dabei werden als Grundeinheiten artikulatorische Gesten angenommen, die miteinander überlappen können. Die Diskrepanz in der Definition sprachlicher \isi{Primitiva} lässt sich am besten verstehen, wenn man das Problem wissenschaftstheoretisch betrachtet.

\largerpage
In der traditionellen Phonologie wurde davon ausgegangen, dass mentale Repräsentationen beim Menschen diskreter Natur sein müssten. Sie verwenden als sprachliche \isi{Primitiva} deshalb Einheiten wie Segmente oder Merkmale, die an symbolischen Repräsentationen orientiert sind. Diese Einheiten stehen jeweils für die kategoriale Zuordnung eines bestimmten Wertes. So ist ein \isi{Vokal} entweder nasaliert [+ nasal] oder nicht [- nasal]. Einen Zwischenwert gibt es nicht. So gelten beispielsweise [balkɔ] und [balkɔŋ] als alternative Aussprachen für <Balkon>. Dass in der letzteren Variante etwas Nasalierung feststellbar ist, kann mit diesem Set diskreter Einheiten nicht ausgedrückt werden.

Später erkannte man, dass mentale Repräsentationen beim Menschen auch kontinuierlicher Natur sein können. Dies ging mit der Entwicklung dynamischer Systeme einher. Dynamische Systeme verwenden keine Schnittstelle zwischen symbolorientierten, diskreten Repräsentationen und deren Abbildung in der physikalischen, kontinuierlichen Welt. Vielmehr formulieren sie die physikalischen Vorgänge als Gesetzmäßigkeiten und beschreiben die Entwicklung von Objekten innerhalb eines Systems. Solche Systeme können in der Biologie Räuber-Beute-Verhältnisse und in der Linguistik das Zusammenspiel sprachlicher \isi{Primitiva} wie artikulatorischen Gesten sein. In diesen Ansätzen wird die Variation als Teil der Systementwicklung gesehen, die grundlegende Eigenschaften der in ihnen verankerten Objekte reflektiert.

Auch wenn die Definition von sprachlichen \isi{Primitiva} in dynamischen Systemen (Gesten) sich grundsätzlich von denen in traditionellen phonologischen Theorien unterscheiden (Segmente, Merkmale), so lassen sich doch auch große Übereinstimmungen finden. Das bedeutet, dass die Theorien durchaus miteinander verbunden werden können oder einander ergänzen. Dies liegt nicht zuletzt daran, dass Gesten --  auch wenn sie gleichzeitig auftreten -- auditiv und akustisch durchaus den Eindruck von einer Abfolge von Segmenten mit bestimmten Eigenschaften vermitteln. 

Im Folgenden werden die artikulatorischen Gesten und ihre Organisation als kognitive Grundeinheiten gesprochener Sprache als \isi{dynamisches System} dargestellt. Es wird aufgezeigt, nach welchen Prinzipien artikulatorische Gesten linguistische Information enkodieren. Mit Hilfe von gestischen Organisations- bzw. Koordinationsmustern werden phonologische Prozesse wie Reduktion, \isi{Assimilation} und Tilgung dynamisch abgebildet und in unterschiedlich starken Graden modelliert. Das Kapitel beginnt mit einer kurzen Einführung in das Prinzip der dynamischen Systeme am Beispiel des Task-Dynamic-Modells, das als Grundlage für die \isi{Modellierung} von artikulatorischen Gesten dient. 

\section{Grundlagen eines dynamischen Systems}
\label{sec:0101}

Ein Werkzeug der mathematischen \isi{Modellierung}, welches ohne die Verwendung einer Schnittstelle sowohl diskrete als auch kontinuierliche Aspekte komplexer Systeme ausdrücken kann, ist die Theorie der nichtlinearen Dynamik (u.a. \citealt{Kelso1995}; \citealt{Kugler1987}; \citealt{Gafos2006}). Mit Hilfe von dynamischen Systemen können physikalische Vorgänge als Gesetzmäßigkeiten formuliert werden, die die Entwicklung von Objekten innerhalb eines Systems über die Zeit beschreiben. Derartige Vorgänge können aus unserer erfahrbaren Welt stammen, wie beispielsweise ein Feder-Masse-System. Auch gesprochene Sprache kann als Vorgang mit seinen Gesetzmäßigkeiten als \isi{dynamisches System} modelliert werden, wie beispielsweise im Task Dynamic Modell (u.a. \citealt{Fowler1980}; \citealt{Saltzman1989}; \citealt{Browman1986}). Bei einer solchen \isi{Modellierung} werden Gleichungen für eine gesuchte Funktion verwendet, die selbst Ableitungen der Funktion enthalten (Differentialgleichungen). Diese Differentialgleichungen können als die mathematische Gestalt von Entwicklungsgesetzen verstanden werden, und als solche sind sie von invarianter Natur. 

\begin{quotation}
	(\dots{}) cognition is best understood using a single formal language that can express both discrete and continuous aspects of complex systems, the mathematics of nonlinear dynamics. In this view, the key constructs are not symbol strings (representations) and algorithms for their manipulation (discrete computation), but rather laws stated in the form of differential equations. These laws prescribe how some behavior’s essential parameters (e.g., perceptual response or relative phase in interlimb coordination) change as contextual parameters are modified (e.g., stimulus properties, oscillation frequency). \citep[][906]{Gafos2006}
\end{quotation}

\citet{Browman1986} veranschaulichen das Prinzip eines dynamischen Systems an einem einfachen Feder-Masse-Modell, das zum Schwingen gebracht wird. Eine Masse (ein Objekt) wird an einer Feder befestigt. Zunächst verändert sich das System nicht, denn das Objekt befindet sich in seiner Ruheposition (Gleichgewichtslage). Wenn ich an dem Objekt ziehe, spannt sich die Feder über ihre Gleichgewichtslage hinaus. Lasse ich die Masse los, so beginnt das System sinusförmig um seine Ruhelage zu schwingen, angenommen das System ist ohne Reibung. Die Bewegung des Objektes lässt sich als \isi{Bewegungstrajektorie} der Masse abbilden. Sie ist mathematisch gesehen das Ergebnis der Differenzialgleichung einer nichtgedämpften Schwingung (vgl. Formel~\ref{eq:diff01}). Weil bei einer Differenzialgleichung das Ergebnis eine Funktion ist, kann diese die \isi{Bewegungstrajektorie} abbilden, in diesem Fall als Funktion von \enquote{Kraft = Federkonstante * Weg}:

\begin{equation}
\label{eq:diff01}
m\ddot{x}+k\left(x-x_{0}\right)=0
\end{equation}
wobei gilt:
\begin{align*}
m = & \; \text{Masse des Objekts} \\%
k = & \; \text{\isi{Steifheit} der Feder} \\%
x_{0} = & \; \text{\isi{Gleichgewichtslage der Feder} (neues \isi{Target})}\\%
x = & \; \text{Momentanwert des Objekts (aktuelle Position der Masse)} \\%
\ddot{x} = & \; \text{Momentanbeschleunigung des Objekts}\\%
\end{align*}

Es zeigt sich, dass unterschiedliche dynamische Parameter wie Masse, \isi{Steifheit} und Ruheposition der Feder $(m$,  $k$,  ${x}_{0}$ an das System übergeben werden können \citep{Browman1986}. Außerdem wird die Ausgangsposition des Objekts mit einberechnet. Die Gleichung selbst ändert sich dabei nicht; sie ist \isi{invariant}. Es variieren lediglich je nach Parameterübergabe die unterschiedlichen Trajektorien des beschriebenen Objekts.

Verändere ich in diesem System die \isi{Steifheit} der Feder $k$, so verändert sich die Frequenz der Oszillation und ich erziele eine zeitliche Variation (\isi{Steifheit} ist auch als Eigenperiode bzw. \isi{Eigenfrequenz} bezeichnet). Verändere ich die aktuelle Position/Lage der Masse und die \isi{Gleichgewichtslage der Feder} (die \isi{Zielposition}, bei der die Feder zur Ruhe kommt), so nehme ich Einfluss auf die \isi{Bewegungsauslenkung} und erziele eine räumliche Variation.

Das Modell der Task Dynamics verwendet dynamische Systeme für die \isi{Modellierung} der biologischen und physikalischen Prinzipien von Bewegungs-Tasks (Bewegungsaufgaben). Zunächst wurde das Modell auf nicht sprachliche Aufgaben angewendet, beispielsweise um die Dynamik von Fingerbewegungen zu untersuchen. In einer Studie von \citet{Kelso1980} hatten die Probanden die Aufgabe, ihre Finger in hoher Geschwindigkeit auf eine gelernte \isi{Zielposition} hin zu bewegen. Die Probanden konnten diese Aufgabe trotz Perturbationen ausführen, d.\,h. die Finger erreichten stets die finale Position. Hier zeigt sich das Prinzip der Äquifinalität \citep{Bertalanffy1968}: Systemobjekte in Feder-Masse-Modellen erreichen trotz verschiedener Anfangsbedingungen denselben Endzustand (Zielgleichheit, vgl. \citealt{Browman1986}; \citealt{Saltzman1989}; \citealt{Hawkins1992}; \citealt{Pouplier2011a}; \citealt{BrowmanGoldstein}). Bewegungsaufgaben können mit Hilfe unterschiedlicher Bewegungsabläufe und sogar mittels unterschiedlicher Organgruppen ausgeführt werden (Motor-Äquivalenz; \citealt{Hebb1949}). Motor-Äquivalenz zeigt sich beispielsweise in der persönlichen Handschrift: So kann beim Schreiben ein Stift unterschiedlich gehalten werden, je nachdem ob man auf Papier, an eine Wandtafel oder sogar mit dem Fuß in den Sand schreibt \citep{Wing2000}. Obwohl für Bewegungsaufgaben während des Zeitraums ihrer Ausführung invariante und kontextunabhängige Targets zugrunde liegen, ist die ausgeführte \isi{Bewegungstrajektorie} variabel und kontextabhängig. 

Task-Dynamic-Modelle können auch auf sprachliche Aufgaben angewendet werden (u.a. \citealt{Fowler1977}; \citealt{Fowler1980}; \citealt{Saltzman1986}; \citealt{Browman1986}; \citealt{Browman1988}; \citealt{Saltzman1987}; \citealt{Saltzman1989}; eine zusammenfassende Einführung findet sich in \citealt{Hawkins1992}). In diesem Fall beschreibt es die dynamische Koordination und Kontrolle von linguistisch relevanten Bewegungsaufgaben des Sprechtraktes (Tasks). Sprechen ist ein kontinuierlicher Vorgang und die komplexen Bewegungen der Artikulatoren wie Zunge, Kiefer, Lippen oder Velum führen zu sich beständig verändernden Hohlraumkonfigurationen im \isi{Sprechtrakt}, die für die Klangeigenschaften des akustischen Signals relevant sind. Die Komplexität dieser Bewegungsabläufe wird in sprachliche \isi{Primitiva} zerlegt: die artikulatorischen Gesten \citep{Saltzman1989}. Solche Gesten definieren im Feder-Masse-Modell ein Set von diskreten Bewegungsaufgaben. Sie kontrollieren und koordinieren dabei die Objekte, die die Aufgaben ausführen. Die Objekte beschreiben den Aufgabentyp und sind in dem Modell als eine Gruppe von Task-Variablen bzw. Trakt-Variablen definiert (\citealt[vgl.][]{Hawkins1992}). 

Konkret bedeutet das für die Gleichung im Feder-Masse-Modell in Formel~\ref{eq:diff01}: Hat eine \isi{Bewegungsaufgabe} einen bilabialen Verschluss der Lippen zum Ziel, so liefert das Feder-Masse-Modell eine Beschreibung für die artikulatorischen Bewegungen, die mit diesem Lippenverschluss assoziiert sind (\citealt{Browman1986}). Zunächst soll aus Gründen der Einfachheit nur die \isi{Bewegungstrajektorie} der unteren Lippe betrachtet werden; später wird sich zeigen, dass die Lippen bei einem labialen Verschluss gemeinsam mit dem Kiefer als eine Organgruppe agieren.

In der Gleichung \ref{eq:diff01} beschreibt die Variable $x$ die vertikale Bewegung der unteren Lippe. Wenn sich die Lippen schnell bewegen (beispielsweise bei schneller globaler \isi{Artikulationsrate} oder lokal bei nicht prominenten Reduktionssilben), so wird die Federsteifheit $k$ erhöht.

\begin{quotation}
	The stiffer the gesture, the higher its frequency of oscillation and therefore the less time it takes for one cycle. Note this also means that, for a given equilibrium position, the stiffer the gesture, the faster the movement of the associated articulators will be. \citep[][348]{Browman1991a})
\end{quotation}

\newpage 
Soll nun der räumliche Weg, den der \isi{Artikulator} zurücklegt, verkürzt werden (geringere \isi{Auslenkung} der \isi{Bewegungstrajektorie}), so kann die Differenz zwischen dem neuen \isi{Target} $x_{0}$ und der momentanen Position für die untere Lippe verringert werden (geringere \isi{Auslenkung} bzw. geringeres \enquote{Displacement}). Umgekehrt verhält es sich dann bei der \isi{Modellierung} von \isi{Prominenz}, bei der von geringeren Artikulationsgeschwindigkeiten (geringere \isi{Steifheit} $k$) und größeren Bewegungsauslenkungen $(x-x_{0})$ ausgegangen werden kann. Es lassen sich demnach durch Manipulationen der \isi{Steifheit} $k$ und der \isi{Bewegungsauslenkung} $(x-x_{0})$ Strategien der Hyper- und Hypoartikulation modellieren (\citealt{Lindblom1990}; H\&H Model, vgl. auch Kapitel~\ref{chap:02} in diesem Buch). 

Bei der vertikalen Bewegung der Lippen handelt es sich um nicht-oszillierende Bewegungen. Deshalb geht in die Beschreibung der Faktor Dämpfung ein. Es wird dabei von einer kritischen Dämpfung ausgegangen, d.\,h. das in Schwingung versetzte Objekt (hier die Mundlippen) nähert sich der Nullauslenkung asymptotisch an, ohne das \isi{Target} zu erreichen.

Die folgende Gleichung \ref{eq:diff02} ist gegenüber Gleichung~\ref{eq:diff01} um die Dämpfung erweitert (\citealt{Saltzman1989}; \citealt{Hawkins1992}; \citealt{BrowmanGoldstein}). Das Objekt entspricht hier der Taskvariablen und später bei sprachlichen Bewegungsaufgaben auch den \isi{Traktvariablen}.

\begin{equation}
\label{eq:diff02}
m\ddot{x}+b\dot{x}+k\left(x-x_{0}\right)=0
\end{equation}
wobei gilt:
\begin{align*}
m = & \; \text{Masse des Objekts}\\%
b = & \; \text{Dämpfung des Systems}\\%
k = & \; \text{\isi{Steifheit} der Feder}\\%
x_{0} = & \; \text{\isi{Gleichgewichtslage der Feder} (neues \isi{Target})}\\%
x = & \; \text{Momentanwert des Objekts (aktuelle Position der Masse)}\\%
\dot{x} = & \; \text{Momentangeschwindigkeit des Objekts}\\%
\ddot{x} = & \; \text{Momentanbeschleunigung des Objekts}\\%
\end{align*}

Für die dynamische \isi{Modellierung} von Sprechbewegungsaufgaben sind die Parameter $m$ (Masse) und $b$ (Dämpfung) für die meisten Objekte (Taskvariablen) festgesetzt und somit dem System bekannt, während $k$ (\isi{Steifheit}) und ${x}_{0}$ (\isi{Target}) unter Einbeziehung von $x$ (aktuelle Lage des Objektes) eine wichtige Rolle für die jeweilige \isi{Modellierung} des Schwingungsverhalten -- beispielsweise für \isi{Prominenz} -- des Systems spielen. Es sei hier kurz angemerkt, daß im Task-Dynamic-Modell die Objektmasse  $m$ und das Dämpfungsverhältnis \(b:{\left(2\;{\cdot}\;{\left[{mk}\right]}^{1/{2}}\right)}\)  zumeist den konstanten Wert $1,0$ \citep{Hawkins1992} bekommen. Insbesondere die Definition eines konstanten Dämpfungsverhältnisses kann jedoch problematisch sein, insbesondere, wenn bei den Gesten sogenannte Haltephasen in Form von Plateaus entstehen \citep[vgl.][]{Fuchs2011}.

Bei sprachlichen Aufgaben ist die Motor-Äquivalenz ein weiteres Prinzip und führt zu einem dynamischen System mit multiplen Freiheitsgraden: Wenn der Kiefer des Sprechers fixiert wird, kann trotzdem ein Lippenverschluss gebildet werden. Die Lippen kompensieren dabei unmittelbar die fehlende Kieferbewegung durch größere Bewegungsauslenkungen und erhöhte \isi{Steifheit} der Bewegungsausführung (\citealt{Kelso1984b}; \citealt{Ito2000}).

\section{Artikulatorische Phonologie}
\label{sec:0102}

Die \isi{Artikulatorische Phonologie} basiert auf dem Task-Dynamic-Modell (\citealt{Browman1986}; \citealt{Browman1988}; \citealt{Browman1991a}). Sie macht sich zu Nutze, dass Bewegungsaufgaben während des Zeitraums ihrer gestischen Aktivierung diskret, \isi{invariant} und kontextunabhängig sind, ihre Ausführungen aber kontinuierlich, variabel und kontextgebunden verlaufen. Die \isi{Artikulatorische Phonologie} verwendet ebenfalls die artikulatorische \isi{Geste} als sprachliche Grundeinheit, beschreibt aber darüber hinaus deren Funktion als kombinatorische Einheiten. Dabei werden Gestenpartituren und Gestenstrukturen verwendet, um die Koordination von Gesten als \enquote{Atome} in Form von \enquote{Molekülen} gesprochener Sprache abzubilden (vgl. \citealt{Pouplier2011a}).

Die folgende Abbildung~\ref{figure:0102} skizziert das \emph{TA}sk \emph{D}ynamic \emph{A}pplication (TADA) Computermodell, mit dessen Hilfe Sprache artikulatorisch synthetisiert werden kann. Dieses Modell hat verschiedene Submodelle mit unterschiedlichen Abstraktionsgraden. Die drei Hauptkomponenten sind das Linguistische \isi{Gestenmodell}, das Task-Dynamic-Modell und das Vokaltrakt-Modell; die Modelle nehmen in dieser Reihenfolge im Abstraktionsgrad -- von der intendierten Äußerung hin zum akustischen Output -- ab.

\begin{figure}[ht]
	\includegraphics[width=.8\textwidth]{figures/1-2_Computermodel_AP.png}
	\caption{Computergestützte Modellierung von Gesten mittels der dynamisch artikulatorischen Systeme, TADA, \citealt[nach][342]{Browman1991a}.}
	\label{figure:0102}
\end{figure}

Das Linguistische \isi{Gestenmodell} beschreibt die artikulatorische Struktur von Gesten in Form von gestenparametrischer Koordination und Kombination. Dabei generiert es entsprechende Partituren (\emph{gestural scores}), die nicht nur einzelne invariante Bewegungsaufgaben (die \enquote{Atome}) sondern auch deren linguistische Koordination (die \enquote{molekulare} Struktur) enthalten. Die Partituren dienen als Input für das Task-Dynamic-Modell. Die Aufgabe des Task-Dynamic-Modells besteht in der Kontrolle der \enquote{Artikulatoren}. Dabei verwendet es die \isi{Traktvariablen} als Objekte und generiert als deren Output Bewegungstrajektorien. Die Trajektorien selbst sind immer noch abstrakt, dienen aber als Input für das Vokaltrakt-Modell, welches mittels Areafunktionen das akustische Signal generiert (\citealt{Browman1991a}).

\subsection{Traktvariablen}
\label{subsec:010201}

Die Taskvariablen des Task-Dynamic-Modells beschreiben Bewegungsaufgaben unter Verwendung von gedämpften Differenzialgleichungen zweiter Ordnung (\citealt{Browman1992a}). Bei sprachlichen Bewegungsaufgaben entsprechen sie den Variablen des Vokaltraktes (\isi{Traktvariablen}; \citealt[vgl.][]{Saltzman1986}; \citealt{Saltzman1987}; \citealt{Saltzman1989}; \citealt{Browman1991a}; \citealt{Browman1992a}; \citealt{BrowmanGoldstein}).

Während der Aktivierung einer \isi{Traktvariablen} versucht diese eine neue, dem gestischen Ziel entsprechende Gleichgewichtslage oder Ruheposition zu erreichen. In der Analogie zum Feder-Masse-Modell entspräche das \isi{Bewegungsmuster} einer einzelnen \isi{Traktvariablen} nicht dem einer einzelnen Feder (eines einzelnen Artikulators) sondern vielmehr dem eines Federsystems (einer artikulatorischen Organgruppe). Solche Organgruppen bilden funktionale Synergien, bei denen verschiedene Kräfte zusammenwirken. Für die Bildung eines Lippenverschlusses sind beispielsweise Kiefer, untere und obere Lippe als koordinative Struktur involviert. Gemeinsam bilden sie ein virtuelles Federsystem (\citealt{Saltzman1986}; \citealt{BrowmanGoldstein}). Die Distanz wischen oberer und unterer Lippe wird von der \isi{Traktvariablen} \emph{Lip Aperture} (\isi{Zwischenlippendistanz}) reguliert, welche Kiefer und Lippen einbezieht. Der Wert der \isi{Traktvariablen} \emph{Lip Aperture} beträgt bei einem Vollverschluss $0{cm}$ (Null). Positive Werte beschreiben eine Öffnung zwischen den Lippen, negative Werte deren Kompression. 

Die \isi{Traktvariablen} lassen sich drei Subsystemen zuordnen: dem oralen, dem velischen und dem glottalen System, vgl. Abbildung~\ref{figure:0103}.

\begin{figure}[ht]
	\includegraphics[width=\textwidth]{figures/1-3_Saggi_Traktvariablen.png}
	\caption{Orales, velisches und glottales Subsystem, schematisiert nach \citealt{Hewlett2006}.}
	\label{figure:0103}
\end{figure}

\largerpage
Die \isi{Traktvariablen} des oralen Systems (Lippen, Zungenspitze und -rücken) greifen teilweise auf gleiche Artikulatoren zurück und zeigen somit Abhängigkeiten und Synergien, wenn sie gleichzeitig aktiv sind. Sie sind jeweils in Paare (LP-LA, TTCL-TTCD, TBCL-TBCD; vgl. Tabelle~\ref{table:0101}) auf zwei Beschreibungsdimensionen des virtuellen vertikal-horizontalen Vokaltraktes aufgeteilt: Eine \isi{Traktvariable} beschreibt dabei jeweils den Grad einer Konstriktion (constriction degree, CD, vertikale Ebene), die andere den Ort der Konstriktion (location of constriction, CL, horizontale Ebene). In Tabelle~\ref{table:0101} sind die acht \isi{Traktvariablen} und die zugehörigen Artikulatoren gelistet. 

\begin{table}[htpb]
	\resizebox{\textwidth}{!}{
		\begin{tabular}{lll} \lsptoprule
		 	& {\bfseries Traktvariable} & \bfseries Organgruppe\\ \midrule
			LP & Lippenrundung &  {Lippen, Kiefer}\\
			LA & Lippenöffnung & {Lippen, Kiefer}\\ 
			\tablevspace
			 {TTCD} & Zungenspitze  \isi{Konstriktionsgrad}& Zungenspitze und -rücken, Kiefer\\  
			{TTCL} & Zungenspitze  Konstriktionsort &Zungenspitze und -rücken, Kiefer \\ 
			\tablevspace
			 {TBCD} & Zungenrücken  \isi{Konstriktionsgrad}&  {Zungenrücken, Kiefer} \\  
			 {TBCL} & {Zungenrücken} Konstriktionsort&  {Zungenrücken, Kiefer} \\
			\tablevspace
			VEL & Velum & Velum\\
			\tablevspace
			GLO & Glottis & Glottis\\ \lspbottomrule
		\end{tabular}
	}
	\caption{Traktvariablen und zugehörige Artikulatoren nach \citet{Browman1992a}.}
	\label{table:0101}
\end{table}

Für das velische und das glottale System (Kontrolle von Velum und Glottis) sind bislang eindimensionale Spezifizierungen ausreichend; die \isi{Traktvariablen} treten hier im Gegensatz zu den anderen Organgruppen nicht in Paaren auf. Derzeit sind acht \isi{Traktvariablen} in den gängigen Systemen der Artikulatorischen Phonologie implementiert; die Anzahl der verwendeten \isi{Traktvariablen} ließe sich jedoch noch erweitern. So könnten für die Zunge TT und TB noch Deskriptoren für die Zungenform (constriction shape CS, \citealt{Browman1989}) oder weitere glottale Deskriptoren hinzugefügt werden: \enquote{Additional laryngeal variables are required to allow for pitch control and for vertical movement of the larynx, required, for example, for ejectives and implosives.} (\citealt[][73]{Browman1989})

\subsection{Artikulatorische Gesten}
\label{subsec:010202}

Die Grundeinheiten der Artikulatorischen Phonologie sind die artikulatorischen Gesten. Gesten sind im Rahmen des Task-Dynamic-Modells spezifiziert. Das Bewegungsziel (\emph{Task}) einer \isi{Geste} ist die Bildung eines linguistisch relevanten Verschlusses. Das gestische \isi{Aktivierungsintervall} beschreibt das Intervall vom Start bis zum Ziel einer \isi{Geste}. Gesten kontrollieren die Bewegungen des Sprechtraktes mit Hilfe der \isi{Traktvariablen} (\citealt[vgl.][]{Browman1991a}; \citealt{Browman1992a}). Bei oralen Gesten koordiniert eine \isi{Geste} jeweils ein Paar von \isi{Traktvariablen} (horizontal-vertikale Dimension). Diese \isi{Traktvariablen} greifen auf gleiche Organgruppen zurück (LP-LA, TTCL-TTCD, TBCL-TBCD; vgl. Abbildung~\ref{figure:0103}). Beim velischen und glottalen System kontrolliert die \isi{Geste} jeweils eine \isi{Traktvariable} (VEL, GLO).

\begin{quotation}
	That is, for \isi{oral} gestures, two dynamical equations are used, one for constriction location and one for constriction degree. Since the glottal and velic aperture tract variables do not occur in pairs, they map directly onto glottal and velic gestures, respectively. (\citealt[][3]{Browman1991a})
\end{quotation}

\begin{table}[b] 
		\begin{tabularx}{\textwidth}{lQQ} \lsptoprule
			\bfseries \isi{Traktvariable} & \bfseries CD (\isi{Konstriktionsgrad}) & \bfseries CL (Konstriktionsort)\\ \midrule
			{Lippenöffnung LA} &  {geschlossen (close)}
			\newline {kritisch (critical)}
			\newline {eng (narrow)} & labial, labiodental\\
			\tablevspace
			{Lippenrundung LP} & {gerundet (protruded)} & \\
			\tablevspace
			{Zungenspitze TT} & {geschlossen (close)} {kritisch (critical)}
			\newline {eng (narrow)} & {dental, alveolar, postalveolar}\\
			\tablevspace
			{Zungenrücken TB} & {geschlossen (close)}
			\newline {kritisch (critical)}
			\newline {eng (narrow)
			\newline mittel (mid)} & {palatal, velar, uvular, pharyngal, uvu-pharyngal}\\
			\tablevspace
			{Velum VEL} & {offen (wide)} & \\
			\tablevspace
			Glottis GLO & {offen (wide)} & \\ \lspbottomrule
		\end{tabularx} 
	\caption{Gängige Deskriptoren für Traktvariablen.}
	\label{table:0102}
\end{table}
Deskriptoren für Gesten beschreiben die vertikale und horizontale Dimension der Konstriktion (constriction degree CD, constriction location CL) sowie die \isi{Steifheit} $k$ (\citealt{Browman1989}). Solche Deskriptoren oder Parameter sind sprachabhängig und müssen im jeweiligen Sprachsystem festgelegt werden (\citealt{Browman1992a}). Tabelle~\ref{table:0102} illustriert die gängigen Deskriptoren für die jeweiligen \isi{Traktvariablen}; obwohl die Lippenöffnung, LA, in der Regel mit nur einem Deskriptor für den \isi{Konstriktionsgrad} auskommt, finden sich in der Literatur auch Spezifikationen für den Konstriktionsort. Die genauen Spezifikationen ergeben sich jeweils aus dem phonologischen Modell für die zu untersuchende Sprache.


Der Grad einer Konstriktion (CD) kann wie folgt spezifiziert werden: geschlossen (closed; vollständige Blockade des Luftstroms bei der Plosivproduktion), kritisch (critical; geräuschverursachende Engebildung bei der Frikativproduktion), eng (narrow; nicht-geräuschverursachende Engebildung bei der Approximantproduktion), mittel (mid) und offen (wide). \citet{Nam2007b} gibt konkrete Beispiele für unterschiedliche Targetspezifikationen den Grad der Konstriktion, CD, betreffend. In seinen Äußerungen des Englischen haben  Plosive die Werte $-2{mm}$, Frikative $1{mm}$ als Abstandsziel; einen offenen \isi{Vokal} /a/ definiert er mit $11{mm}$. Die negativen Werte kommen zustande, weil ein \isi{Target} nur approximiert, aber nicht erreicht wird.

Deskriptoren für den Ort der Konstriktion (CL) sind gerundet (protruded), labial, dental, labiodental, alveolar, post-alveolar, palatal, velar, uvular und pharyngal (pharyngeal). Der Steifheitsparameter  kann vokalische und konsonantische Gesten unterscheiden: Bei einem nicht-silbischen Halbvokal [j] und einem silbischen Entsprechungsvokal [i] unterscheiden sich die Parameter CD und CL nicht; beide gestischen Ziele liegen in der Bildung eines palatalen Beinahverschlusses des Zungenrückens \{TB narrow palatal\}. Die \isi{Steifheit} ist jedoch beim Halbvokal [j] höher als beim Vollvokal [i] (\citealt{Browman1989}). Mit der Erhöhung der \isi{Steifheit} (Eigenperiode, \isi{Eigenfrequenz}) lässt sich mittels eines artikulatorischen Synthesizers (\citealt{Kröger1993}) ein \isi{Vokal} in einen Halbvokal überführen, beispielsweise im Deutschen die Aussprachevarianten von <Dahlie>, [da:l.jə] und [da:l.i.ə] (\citealt[vgl.][]{Mücke1999}). 

  
Der Steifheitsparameter $k$ verweist bereits auf die funktionale Unterscheidung von konsonantischer und vokalischer Funktion von Gesten. Diese geht auf Beobachtungen zurück, die in akustisch-spektrographischen Analysen von VCV-Sequenzen gemacht worden sind (\citealt{Öhman1966}): Die Vokalartikulation überlagert die Konsonantenproduktion fast vollständig, während sich Vokale untereinander kaum überlagern. Für eine kinematische Analyse bedeutet das einen Unterschied in der intrinsischen Dauer der gesturalen Aktivierung von Vokalen und Konsonanten. Im dynamischen Modell haben vokalische Gesten deshalb eine geringere \isi{Eigenfrequenz} (eine geringere \isi{Steifheit} $k$) als konsonantische Gesten. Auch wenn die konsonantische und die vokalische \isi{Geste} gleichzeitig starten, erreicht die vokalische \isi{Geste} ihr \isi{Target} später, da sie mit langsamerer Geschwindigkeit ausgeführt wird und länger aktiviert ist als die konsonantische \isi{Geste} (\citealt{Goldstein2006}). 

  
\begin{figure} 
	\includegraphics[width=\textwidth]{figures/1-4_LaMimami_Screenshot.png}
	\caption{Oszillogramm (oben) und vertikale Positionskurven für Zungenrücken (Mitte) und Unterlippe (unten) in der Zielsilbe [ma] in der katalanischen Äußerung <La MiMAmi>.}
	\label{figure:0104}
\end{figure}

\newpage  
\largerpage
Abbildung~\ref{figure:0104} gibt analog zu Abbildung~\ref{figure:0101} ein Beispiel für die Gleichzeitigkeit von Konsonant- und Vokalproduktion am Beispiel des Zielwortes <MiMAmi> in der katalanischen Äußerung <La MiMAmi>, aufgenommen mit elektromagnetischer \isi{Artikulographie}. Das Zielwort ist ein fiktiver Name im Katalanischen. Das Zielwort trägt einen nuklearen LH-Akzent (weiter \isi{Fokus}); die lexikalisch betonte \isi{Silbe} ist zur Veranschaulichung mittels Großbuchstaben und Fettdruck hervorgehoben. Die Zielsilbe [ma] zeigt, dass \isi{Vokal} und Konsonant gleichzeitig starten, Vokale jedoch geringere Ausführungsgeschwindigkeiten und längere \isi{Aktivierungsintervalle} (Intervall vom Start bis zum Ziel einer Bewegung) haben. Hierbei werden im oralen System zwei linguistisch relevante Bewegungsaufgaben ausgeführt: ein konsonantischer Verschluss der Lippen {LA labial closure} für /m/ und ein vokalischer Verschluss des Zungenrückens {TB pharyngeal wide} für /a/. Von oben nach unten zeigt die Abbildung~\ref{figure:0104} die akustische Wellenform sowie die kinematischen Bewegungstrajektorien des Zungenrückens und der unteren Lippe. Es handelt sich jeweils um vertikale Positionskurven (niedrige Werte indizieren eine offene und hohe Werte eine geschlossene Stellung der Artikulatoren). Die gestischen \isi{Aktivierungsintervalle} für Start und Ende der Bewegung /m/ und /a/ sind grau schattiert. Die Bewegungsintervalle starten gleichzeitig. Die Bewegungen des Zungenrückens verlaufen langsamer und erreichen deutlich später als die des Konsonanten ihr Ziel. Für die Koordination der Gesten untereinander (\isi{Phasing}) ist deshalb die funktionale Unterscheidung in konsonantische und vokalische Gesten mit unterschiedlichen dynamischen Parameterspezifikationen relevant; in Kapitel~\ref{chap:03} wird ausgeführt, auf welche Weise beide Gestentypen miteinander assoziiert sind, um ein \enquote{Silbenmolekül} zu formen. 


Die gestischen Deskriptoren zeigen Ähnlichkeiten zu Merkmalen (Merkmalsgeometrie, \citealt{Clements1985}), unterscheiden sich aber in wesentlichen Aspekten von ihnen (\citealt{Browman1989}; \citealt{Browman1992a}; \citealt{Pouplier2011a}). Während Merkmale eine Kombination aus akustischen und artikulatorischen Eigenschaften darstellen (vgl. \citealt{Pike1943}; \citealt{Ladefoged1996}), sind Gesten als artikulatorische Einheiten lexikalisiert. Obwohl die gestischen Deskriptoren der Gesten selbst nicht hierarchisch organisiert sind, ergibt sich eine indirekte Hierarchie aus den beteiligten Organgruppen des Vokaltraktes (Abbildung~\ref{figure:0105}). 

\begin{figure}[h]  
\caption{Artikulatorische Baumstruktur, adaptiert von \citealt[][12]{Brent1996} mit CD = Konstriktionsgrad, CL = Ort der Konstriktion.}
\label{figure:0105} 
\begin{forest}
 [Sprechtrakt 
  [Larynx\\{[}CD{]}]
  [Mundraum 
    [Lippen\\{[}CL{,} CD{]}]
    [Zunge
      [Spitze\\{[}CL{,} CD{]}]
      [Dorsum\\{[}CL{,} CD{]}]
      [Wurzel\\{[}CL{,} CD{]}]
    ]
  ]
  [Velum\\{[}CD{]}]
] 
\end{forest}
	
\end{figure}


Die \isi{Traktvariablen} TT und TB (Zungenspitze und -rücken) teilen sich die Zunge als \isi{Artikulator}, und gemeinsam mit den LIPPEN (Mundlippen) referieren sie auf den Kiefer. Hieraus ergeben sich natürliche Klassen der artikulatorischen Geometrie. Der Grad und Ort der Konstriktion (CL, CD) werden von den Artikulatoren(-knoten) dominiert (\citealt{Browman1992a}), während in der Merkmalsgeometrie die Merkmale der Artikulationsart direkt mit dem Wurzelknoten verbunden sind \citep{Clements1985}.


Gesten haben als \isi{Primitiva} der Artikulatorischen Phonologie und somit auch als Einheiten dynamischer Systeme eine duale Funktion. Die Festlegung der gestischen \isi{Bewegungsaufgabe} ist phonologischer Natur, während deren Ausführung eine phonetische Aktion darstellt und als zielgerichteter Bewegungsablauf der Sprechorgane modelliert ist (Tabelle~\ref{table:0103}). Somit sind Gesten gleichzeitig Einheiten der Information (diskret) und der Aktion (kontinuierlich). Sie können gleichzeitig invariante, kontextunabhängige Information abbilden und variable, kontextabhängige Trajektorien generieren, ohne eine gesonderte Schnittstelle zwischen den Repräsentationsebenen annehmen zu müssen. Gesten haben somit kognitiven und gleichzeitig physikalischen Status. Die Prinzipien zur Bildung phonologischer Kontraste sowie sprachspezifische Aspekte und kontextbedingte Variationen werden im Folgekapitel mit Hilfe von Gestenpartituren illustriert.


\begin{table}[h]
	\resizebox{\textwidth}{!}{
	\begin{tabularx}{\textwidth}{XX} \lsptoprule
	\bfseries Kombinatorische Einheit & \bfseries Physikalisch messbar\\ \midrule
	{Diskret}
	
	{Kontextunabhängig}
	
	{Zeitlich invariant}
	
	{Wenig-dimensional}
	
	Kognitiv & {Kontinuierlich}
	
	{Kontextabhängig}
	
	{Zeitlich variabel}
	
	{Mehrdimensional}
	
	Physikalisch\\ \lspbottomrule
\end{tabularx}
}
\caption{Die duale Funktion der Gesten (aus \citealt{BrowmanGoldstein}).}
\label{table:0103}
\end{table}