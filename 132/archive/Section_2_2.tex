\subsection{Comparison with other approaches}
\label{sec:ComparisonWithOtherApproaches}

Up to now in translation studies, two procedures to cut a translation process into units have been tried and both of them ignore the fact that the length of different writing pauses (AA, BA, AB) correlate differently with typing pause length. Both approaches considered gaps between two writing actions longer than a fixed threshold pauses in translation, that is, these gaps were considered the boundaries of translation process units. They were not understood as pauses in writing nor as pauses in typing nor as pauses in finger motion.

The first fixed threshold used in translation studies was user-unspecific and was picked by the researcher him or herself. Early thresholds ranged between 5 and 6 seconds \citep{Hansen:1999wn,Hansen:2002wu,Alves:2003va,PACTE:2005vu}. The second approach to cut the writing process into units during translation was proposed by \citet{Dragsted:2004tj,Dragsted:2005vl}. Her approach consisted of finding a writing pause\footnote{Typing speed in her terms since she did not distinguish typing actions from writing actions} for each participant that `seemed to reveal a certain pattern of syntactic units\footnote{Here called \emph{grammatical structures}}'. The attempt is valid and it does reveal a certain patterning that looks similar to a word/group/phrase based cutting of the target text production.

However, even though Dragsted's approach is much better at capturing the writing rhythm of fast and slow writers, it tells little about how much `cognitive' effort was put in each pause. It is also a bad indicator of cognitive effort since it tends to find boundaries of translation process units before all capital letters. This might lead researchers to believe that sentence beginnings and German nouns are specially charged with cognitive effort, when that is not really what is going on. It just takes longer for a person to type a capital letter than a small letter. Dragsted's approach also has the tendency to underestimate the cognitive effort of writing pauses between two small letters, which are typically shorter because less typing occurs during them. Examples \ref{ex:12} and \ref{ex:13} show respectively an undervaluation and and overvaluation of pauses.

\begin{exe}
  \ex\label{ex:12}
\tt{$\cdot$\={ }A\u{ }s\u{ }p\u{ }e\u{ }k\u{ }t\u{ }$\cdot$} (TPW cut)\\
\tt{$\cdot$•A\u{ }s\u{ }p\u{ }e\u{ }k\u{ }t\u{ }$\cdot$} (Dragsted's cut)
\end{exe}

\begin{exe}
  \ex\label{ex:13}
\tt{$\cdot$\u{ }z\u{ }u\u{ }$\cdot$57\={ }53$\cdot$2P\u{ }r\u{ }o\u{ }z\u{ }e\u{ }n\u{ }t\u{ }$\cdot$} (TPW cut)\\
\tt{$\cdot$\u{ }z\u{ }u\u{ }$\cdot$•7\u{ }5\u{ }$\cdot$•P\u{ }r\u{ }o\u{ }z\u{ }e\u{ }n\u{ }t\u{ }$\cdot$} (Dragsted's cut)
\end{exe}

In Example \ref{ex:12}, the tpw\textsubscript{\={ }} pause before the \textbf{`A' char insert} action was taken by Dragsted method as being significant whereas I estimate this pause to be in the borderline of being a typing pause or not. It is barely longer than the regular gap between two typing actions of the translator. In contrast, Example \ref{ex:13} shows an undervaluation of pauses. Tpw\textsubscript{\={ }} and tpw\textsubscript{3} pauses between \textbf{standard char insert} actions are not recognised whereas a tpw\textsubscript{2} pause before a capital letter is. In other words, even though Dragsted's approach adapts better to the writing speed of each participant, it might give us a skewed view of pauses in translation.