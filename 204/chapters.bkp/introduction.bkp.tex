\documentclass[output=paper,
modfonts,
%bulgarian,greek,polish,portuguese,romanian,russian,english,hebrew
]{langscibook} 
%\bibliography{localbibliography}


\title{Preface} %Invent a better title 

\author{%
 Stella Markantonatou\affiliation{Institute for Language and Speech Processing, Athena RIC, Greece}\and 
 Carlos Ramisch\affiliation{Aix Marseille Université, CNRS, LIS, Marseille, France}\and
 Agata Savary\affiliation{Université de Tours, France}\lastand
 Veronika Vincze\affiliation{University of Szeged, Hungary}
}

\begin{document}

\maketitle

%The index of abbreviations in p.~\pageref{sec:abbreviations} gives the meaning of all language codes and of the grammatical codes used in glosses. 

%%%%%%%%%%%%%%%%%%%%%%%%%%%
\section{Notational conventions}

Some chapters cite MWE examples in several languages. These multilingual \emph{numbered examples}, as (\ref{preface:en:take-on}) to (\ref{preface:fa:have-sleep-for-sb}), contain: 

\begin{enumerate}[(i)]%[label=(\roman*)]
\item\label{ex-line:use} a sample use of the VMWE, followed by the 2-letter ISO 639-1 language code (cf. the list of abbreviations p.~\pageref{sec:lang-codes}),
\item\label{ex-line:transcription} a transcription, if the language is written with a non-Latin alphabet,
\item\label{ex-line:gloss} a gloss following the Leipzig Glossing Rules,\footnote{\url{https://www.eva.mpg.de/lingua/pdf/Glossing-Rules.pdf}} 
\item\label{ex-line:trans} a literal translation, followed by an idiomatic translation in single quotes. 
\end{enumerate}

For English examples, items (\ref{ex-line:transcription})--(\ref{ex-line:trans}) are irrelevant or optional but idiomatic translation might sometimes be useful to easy the comprehension by non-native readers. For right-to-left languages (e.g. Farsi or Hebrew), item (\ref{ex-line:use}) is spelled right-to-left, item (\ref{ex-line:trans}) left-to-right and items (\ref{ex-line:transcription})--(\ref{ex-line:gloss}) left-to-right within components, and right-to-left from one component to another. 
Lexicalized components of the VMWE, i.e. those which are always realized by the same lexeme (cf. SAVARY-CHAPTER \secref{sec:def-scope}, p. \pageref{sec:def-scope}) are highlighted in bold face. 

%\begin{exe}

\ea\label{preface:en:take-on}
She reluctantly \lex{took} \lex{on} this task \jambox{(EN)}
\z

\ea \label{preface:sl:skrivati-glavo-v-pesek}
\gll Ida \lex{skriva} \lex{glavo} \lex{v} \lex{pesek} \\
Ida hide.\textsc{3.sg} head in sand \\ \jambox{(SL)}
% Eva thrusts head in sand \\
\glt Ida hides her head in the sand `Ida pretends not to see a problem’
\z

\ea \label{preface:el:take-decision}
\glll Η Ζωή \lex{παίρνει} μία \lex{απόφαση} \\
i zoi perni mia apofasi \\
Zoe take.\textsc{3.sg} a decision \\ \jambox{(EL)}
\glt Zoe takes a decision `Zoe makes a decision'
\z

\ea \label{preface:fa:have-sleep-for-sb}
\glll\PRL{است} \lex{\PRL{دیده}} \lex{\PRL{خواب}} \PRL{من} \lex{\PRL{برای}} \PRL{کافی} \PRL{قدر} \PRL{به} \\
%æst di:de xɒ:b mæn bærɒ:je kɒ:fi: qædre be  \\
ast \lex{dide} \lex{khab} man \lex{baraye} kafi qadre be  \\
is seen sleep me for enough quantity to\\ \jambox{(FA)}
\glt he had enough sleep for me `he has many plans for me’
\z

%\end{exe}

These examples are to be typeset by the dedicated LSP commands, e.g. example (\ref{preface:el:take-decision}) is typeset like this:

\begin{verbatim}
\ea \label{preface:el:take-decision}
\glll Η Ζωή \lex{παίρνει} μία \lex{απόφαση} \\
i zoi perni mia apofasi \\
Zoe take.\textsc{3.sg} a decision \\ \jambox{(EL)}
\glt Zoe takes a decision `Zoe makes a decision'
\z
\end{verbatim}


\emph{In-line examples}, used for brevity, are preceded by the 2-letter language code, contain items (\ref{ex-line:use}) and (\ref{ex-line:trans}) only, and the idiomatic translation (if any) is introduced by a double arrow `$\Rightarrow$'. For instance, an in-line example corresponding to itemized example (\ref{preface:sl:skrivati-glavo-v-pesek}) would be the following: (SL) \exlitidio{Ida \lex{skriva} \lex{glavo} \lex{v} \lex{pesek}}{Ida hides her head in the sand}{Ida pretends not to see a problem}. If the language under study is written with a non-Latin alphabet, the inline example should not be in italics, e.g. (EL) \nlexidio{Η Ζωή \lex{παίρνει} μία \lex{απόφαση}}{Zoe takes a decision}{Zoe makes a decision}. Sometimes, the literal or the idiomatic translation is superfluous or too verbose, and can be skipped, as in:  (EL) \nlexidio{Η Ζωή \lex{παίρνει} μία \lex{απόφαση}}{Zoe makes a decision}. In-line examples are to be typset by commands specific to this volume, defined in the \texttt{localcommands.tex} file. For instance, the two last examples are typeset as follows:

\begin{small}
\begin{verbatim}
(SL) \exlitidio{Ida \lex{skriva} \lex{glavo} \lex{v} \lex{pesek}}
{Ida hides her head in the sand}{Ida pretends not to see a problem} 
(EL) \nlexidio{Η Ζωή \lex{παίρνει} μία \lex{απόφαση}}{Zoe makes a decision}
\end{verbatim}
\end{small}

%In each usage example the head verb is either in a finite form or in the infinitive (only when the subject is not lexicalized). 



%We put forward these conventions as a future notation standard for multilingual examples of MWEs.

%%%%%%%%%%%%%%%%%%%%%%%%%%%
\section{Acknowledgement}
Cite the ICT COST Action
IC1207 Parsing and multi-word expressions. Towards linguistic precision and computational efficiency in natural language processing (PARSEME), 2013–2017. 

Here goes the list of reviewers: \ldots

Acknowledgement for editorial help from LangSci.

%%%%%%%%%%%%%%%%%%%%%%%%%%%
\section{Language codes}
\label{sec:lang-codes}

\begin{table}
\begin{tabularx}{.45\textwidth}{ll}
%\sout{AR} & Arab \\
\textsc{BG} & Bulgarian \\
\textsc{CS} & Czech \\
\textsc{DE} & German \\
\textsc{EL} & Greek \\
\textsc{EN} & English \\
\textsc{ES} & Spanish \\
\textsc{ET} & Estonian \\
%\sout{EU} & Basque \\
\textsc{FA} & Farsi \\
\textsc{FR} & French \\
\textsc{HE} & Hebrew \\
%\sout{HI} & Hindi \\
\textsc{HR} & Croatian 
\end{tabularx}
\begin{tabularx}{.45\textwidth}{ll}
\textsc{HU} & Hungarian \\
%\sout{ID} & Indonesian\\
%\sout{JA} & Japanese \\
\textsc{IT} & Italian \\
\textsc{LT} & Lithuanian \\
\textsc{MT} & Maltese \\
\textsc{PL} & Polish \\
\textsc{PT} & Brazilian Portuguese \\
\textsc{RO} & Romanian \\
\textsc{SL} & Slovene \\
\textsc{SV} & Swedish \\
\textsc{TR} & Turkish \\
\textsc{YI} & Yiddish \\
%\sout{ZH} & Chinese 
\end{tabularx}
\end{table}


\printbibliography[heading=subbibliography,notkeyword=this]

\end{document}
