\documentclass[output=paper
,modfonts
,nonflat]{langsci/langscibook} 
%\bibliography{localbibliography}
%\bibliography{mwe2017bib3}  
%% add all extra packages you need to load to this file

\usepackage{tabularx,multicol}
\usepackage{url}
\urlstyle{same}

\usepackage{listings}
\lstset{basicstyle=\ttfamily,tabsize=2,breaklines=true}

\usepackage{langsci-basic}
\usepackage{langsci-optional}
\usepackage{langsci-lgr}
\usepackage{langsci-osl}
% \usepackage{./langsci/styles/langsci-lgr}
% \usepackage{./langsci/styles/langsci-osl}
% \usepackage{langsci-gb4e}

\usepackage{tikz}
\usetikzlibrary{patterns,calc}
\pgfdeclarepatternformonly{south east lines}{\pgfqpoint{-0pt}{-0pt}}{\pgfqpoint{3pt}{3pt}}{\pgfqpoint{3pt}{3pt}}{
    \pgfsetlinewidth{0.6pt}
    \pgfpathmoveto{\pgfqpoint{0pt}{3pt}}
    \pgfpathlineto{\pgfqpoint{3pt}{0pt}}
    \pgfpathmoveto{\pgfqpoint{.2pt}{-.2pt}}
    \pgfpathlineto{\pgfqpoint{-.2pt}{.2pt}}
    \pgfpathmoveto{\pgfqpoint{3.2pt}{2.8pt}}
    \pgfpathlineto{\pgfqpoint{2.8pt}{3.2pt}}
    \pgfusepath{stroke}}
    
\usepackage{stmaryrd}
\usepackage{wasysym}
\usepackage{multirow}
\usepackage{caption}
\usepackage{subcaption}
\usepackage{mathrsfs}
\usepackage{qtree}

\usepackage{linguex}


%%% hyphenation points for line breaks
%% Normally, automatic hyphenation in LaTeX is very good
%% If a word is mis-hyphenated, add it to this file
%%
%% add information to TeX file before \begin{document} with:
%% %% hyphenation points for line breaks
%% Normally, automatic hyphenation in LaTeX is very good
%% If a word is mis-hyphenated, add it to this file
%%
%% add information to TeX file before \begin{document} with:
%% %% hyphenation points for line breaks
%% Normally, automatic hyphenation in LaTeX is very good
%% If a word is mis-hyphenated, add it to this file
%%
%% add information to TeX file before \begin{document} with:
%% \include{localhyphenation}
\hyphenation{
    Beck-man
    Ngu-yen
    back-chan-nel
    back-chan-nels
    mo-not-o-nous
    ste-reo-typ-i-cal
}

\hyphenation{
    Beck-man
    Ngu-yen
    back-chan-nel
    back-chan-nels
    mo-not-o-nous
    ste-reo-typ-i-cal
}

\hyphenation{
    Beck-man
    Ngu-yen
    back-chan-nel
    back-chan-nels
    mo-not-o-nous
    ste-reo-typ-i-cal
}

%%pminos do not split footnotes
% \interfootnotelinepenalty=10000 %Footnote in Laporte chapters has to be split SN


%\DeclareIndexNameFormat{default}{%
%\nameparts{#1}%
%\usebibmacro{index:name}%
%{\index[names]}%
%{\namepartfamily}%
%{\namepartgiveni}%
% {}% L1
% {}% L2
%{\namepartprefix}% generates spurious space L3
%{\namepartsuffix}% generates spurious space L4
%}

%  {\DeclareIndexNameFormat{default}{%
%     \usebibmacro{index:name}{\index[names]}{#1}{#3}{#5}{#7}}}

%\DeclareIndexNameFormat{default}{%
%  \usebibmacro{index:name}{\sindex[nom]}{#1}{#3}{#5}{#7}}

%\DeclareIndexNameFormat{default}{%
%  \usebibmacro{index:name}{\sindex[person]}{#1}{#3}{#5}{#7}}
%\DeclareIndexNameFormat{default}{%
%\nameparts{#1} \usebibmacro{index:name}{\sindex[person]]}{\namepartfamily}{‌​\namepartgiven}{\nam‌​epartprefix}{\namepa‌​rtsuffix}}

%\newcommand{\smiley}{:)}

%\renewbibmacro*{index:name}[5]{%
%\usebibmacro{index:entry}{#1}%
%{\iffieldundef{usera}{}{\thefield{usera}\actualoperator}\mkbibindexname{#2}{#3}{#4}{#5}}}

% \newcommand{\noop}[1]{}

%remove for final
%\overfullrule=1mm

\newcommand{\tobi}[2]}}
\renewcommand{\S}[1]{\tobi{#1}{\textsc{*}}}

% this volume references
% puts: [this volume]
% already defined: \citetv
%\newcommand{\citepv}[1]{(\citeauthor{#1} \citeyear*{#1} [this volume])}
\newcommand{\citealtv}[1]{\citeauthor{#1} \citeyear*{#1} [this volume]}

%parentheses around example number
\newcommand{\pref}[1]{(\ref{#1})}

% in-text examples

\newcommand{\lnex}[1]{\textit{#1}} %target lang word
\newcommand{\lnlit}[1]{(lit.: `#1')} %literal reading
\newcommand{\lnlat}[1]{(#1)} % latinization
\newcommand{\lntrans}[1]{`#1'} %translation
\newcommand{\lnexl}[2]%
{\lnex{#1}{} \lnlat{#2}} % ex with latinization
\newcommand{\lnexlat}[3]{\lnex{#1}{} \lnlat{#2}{} \lntrans{#3}} % ex with latinization and tranl.

%ch01
\newcommand{\co}[1]{\mbox{\textbf{#1}}}

%ch09

\newcommand{\cyrbulg}[1]{\begin{otherlanguage*}{bulgarian}#1\end{otherlanguage*}}


%ch10
\newcommand{\nlp}{{\small NLP}}
\newcommand{\mwe}{{\small MWE}}
\newcommand{\rae}{{\small RAE}}
\newcommand{\lvc}{{\small LVC}}
\newcommand{\pos}{{\small P}o{\small S}}
%\newcommand{\todo}[1]{ \textcolor{red}{#1} }

%\renewcommand{\labelenumi}{\theenumi}
%\ainamefmt{{vv}{ll}{, ff}{, jj}} % fullname

\newcommand{\biberror}[1]{{\color{red}#1}}

\newcommand{\osenovaitem}{--~} 



\title{Comparing bilingual word embeddings to translation dictionaries for extracting multilingual collocation equivalents}
\author{%
 Marcos Garcia\affiliation{Universidade da Coruña}
}
% \chapterDOI{} %will be filled in at production

% \epigram{}

% Language Index: English, Portuguese, Spanish
% Subject Index: collocations, phraseology, distributional semantics, parallel corpora, dependency syntax,
%                universal dependencies, machine translation, language learning,
%                corpus linguistics, word embeddings, multiwords, natural language processing

\abstract{
  This chapter introduces a strategy for the automatic extraction of multilingual
  collocation equivalents which takes advantage of parallel corpora to train
  bilingual word embeddings. First, monolingual collocation candidates are retrieved
  using syntactic dependencies and standard association measures. Then, the distributional
  models are applied to search for equivalents of the elements of each collocation in the target languages.
  The proposed method extracts not only collocation equivalents with direct translation
  between languages, but also other cases where the collocations in the two languages
  are not literal translations of each other. Several experiments -- evaluating collocations
  with five syntactic patterns -- in English, Spanish, and Portuguese show that this
  approach can effectively extract large sets of bilingual equivalents with an average
  precision of about $85\%$. Moreover, preliminary results on comparable corpora suggest
  that the distributional models can be applied for identifying new bilingual collocations
  in different domains. This strategy is compared to both hand-crafted bilingual dictionaries
  and to probabilistic translation dictionaries learned from the same resources as the
  bilingual word embeddings, showing that it achieves much larger recall values
  while keeping high precision results.
}

\begin{document}

\maketitle
\label{GARCIA-CHAPTER}

\section{Introduction}
MWEs have been repeatedly classified as an important problem for developing
Natural Language Processing (NLP) tools, as well as to automatically analyze linguistic
utterances \citep{Sag2002a}. Among the different types of MWEs, processing
collocations in an automatic way may pose various problems due to their intrinsic properties
such as compositionality or unpredictability \citep{melcuk98}.

From a theoretical perspective, there are at least two main views on collocations.
On the one hand, there is a tendency to consider any frequent pair of words to be
a collocation \citep{smadja1993,evert2003,kilgarriff2006}.
On the other hand, the phraseological tradition needs both a lexical restriction and a syntactic
relation to consider two Lexical Units (LUs) as a collocation.\footnote{An overview of different visions on collocations
  -- both from theoretical and practical perspectives -- can be found in \citet{seretan2011syntax}.}
From this phraseological point of view, a collocation is a restricted binary co-occurrence of LUs
between which a syntactic relation holds, and that one of the LUs (the \textsc{base})
is chosen according to its meaning as an isolated LU, while the other (the \textsc{collocate})
is selected depending on the base and the intended meaning of the co-occurrence as a whole,
rather than on its meaning as an isolated LU \citep{melcuk98}. Thus, a noun in English
such as \textit{picture} (as a direct object) requires the verb \textit{to take} (and not \textit{to do}, or \textit{to make})
in the phrase \textit{take a picture}, while \textit{statement} selects \textit{to make} (\textit{make a statement}).

In a bilingual (or multilingual) scenario, equivalent collocations are needed
to produce more natural utterances in the target language(s). In this regard,
the referred noun \textit{picture} would select the verb \textit{tirar} in Portuguese
-- `to remove' -- (\textit{tirar uma fotografia}). Similarly the Spanish \textit{vino}
(`wine') would require the adjective \textit{tinto} (\textit{vino tinto}), which is not
the main translation of \textit{red} (\textit{red wine}).

The unpredictability of these structures poses problems for tasks such as
machine translation, whose performance can benefit from lists of multilingual
collocations (or transfer rules for these units) \citep{orliac2003}.
In areas like second language learning, it has been shown that even advanced learners
need to know which word combinations are allowed in a specific linguistic
variety \citep{altenberg2001,ramos2010}. Thus, obtaining resources of multilingual
equivalent collocations could be useful for a variety of applications such as those
mentioned above. However, this kind of resource is scarce, and constructing
them manually requires a large effort from expert lexicographers.

Since the 1990s, a number of approaches were implemented aimed at extracting
bilingual collocations, both from parallel corpora \citep{kupiec1993,smadja1996,wu2003},
and from comparable or even from non-related monolingual resources \citep{lu2004,Rivera2013},
often combining statistical approaches with the use of bilingual dictionaries to
find equivalents of each \textsc{base}.

This chapter explores the use of distributional semantics (by means of bilingual
word embeddings) for identifying bilingual equivalents of monolingual collocations:
On the one hand, monolingual collocation candidates are extracted using a harmonized
syntactic annotation -- provided by \textsc{Universal Dependencies} (UD)\footnote{\scriptsize{\url{http://universaldependencies.org/}}} --,
as well as standard measures for lexical association. On the other hand, bilingual
word embeddings are trained using lemmatized versions of noisy parallel corpora.
Finally, these bilingual models are employed to search for semantic equivalents of both
the \textsc{base} and the \textsc{collocate} of each collocation.

Several experiments using the OpenSubtitles2016 parallel corpora \citep{opus} in English,
Portuguese, and Spanish  show that the proposed method successfully
identifies bilingual collocation equivalents with different patterns: \emph{adjective-noun},
\emph{noun-noun}, \emph{verb-object}, \emph{verb-subject}, and \emph{verb-adverb}.
Furthermore, preliminary results in comparable corpora suggest that the same strategy
can be applied in this kind of resources to extract new pairs of bilingual
collocations. In this regard, this chapter is an extended version of a previous work
in bilingual collocation extraction \citep{mwe17}, including new collocation
patterns and a larger evaluation which compares the proposed approach to probabilistic
translation dictionaries \citep{hiemstra1998,natools}.

Apart from this introduction, \sectref{garcia:sec:rw} includes a review of previous work on collocation
extraction, specially on papers dealing with bilingual resources. Then, \sectref{garcia:sec:method}
and \sectref{garcia:sec:experiments} present and evaluate the method, respectively. Finally, some conclusions
and further work are discussed in \sectref{garcia:sec:conclusions}.


%
\section{Previous studies on collocation extraction}
\label{garcia:sec:rw}

The extraction of monolingual collocation candidates (as well as other MWEs)
from corpora is a well-known topic in corpus and computational linguistics which deserved a
lot of work in different languages.

In this respect, most strategies use statistical association measures on windows of
n-grams with different sizes \citep{church1990, smadja1993}. Other methods, such as the
one presented by \citet{lin1999}, started to apply dependency parsing aimed at better identifying
combinations of words which occur in actual syntactic relations.

More recently, the availability of better parsers allowed researchers to combine automatically
obtained syntactic information with statistical methods to extract collocations more accurately \citep{evert2008,seretan2011syntax}.

A different perspective on collocation extraction focuses not only on their retrieval, but on
semantically classifying the obtained collocations, in order to make them more useful for NLP
applications \citep{wanner2006,wanner2016}.

Concerning the extraction of bilingual collocations, most works rely on parallel corpora to find the
equivalent of a collocation in a target language. In this regard, \citet{smadja1992} and \citet{smadja1996}
first identify monolingual collocations in English (the source language), and then use \textsc{Mutual
  Information} (\textsc{mi}) and the \emph{Dice coefficient} to find the French equivalents of the source
collocations.

\citet{kupiec1993} also uses parallel corpora to find noun phrase equivalents between English
and French. The method consists of applying an expectation maximization (EM) algorithm to previously
extracted monolingual collocations.

Similarly, \citet{haruno1996} obtain Japanese-English chunk equivalents by computing
their \textsc{mi} scores and taking into account their frequency and position in the aligned corpora.

Another work which uses parallel corpora is presented by \citet{wu2003}. The authors
extract Chinese and English n-grams from aligned sentences by computing their \textsc{log-likelihood} ratio.
Then, competitive linking algorithm is used to decide whether each bilingual pair actually
corresponds to a translation equivalent.

\citet{seretan2007} took advantage of syntactic parsing to extract bilingual collocations from parallel corpora.
The strategy consists of first extracting monolingual collocations using \textsc{log-likelihood},
and then searching for equivalents of each \textsc{base} using bilingual dictionaries.
The method also uses the position of the collocation in the corpus, and relies on the syntactic analysis
by assuming that equivalent collocations will occur in the same syntactic relation in both languages.

\citet{Rivera2013} present a framework for bilingual collocation retrieval
which can be applied (using different modules) in both parallel and comparable corpora.
As in other works, monolingual collocations (based on n-grams) are extracted in a first
step, and then bilingual dictionaries (or WordNet, in the comparable corpora scenario) are used
to find the equivalents of the \textsc{base} in the aligned sentence (or in a small window of
adjacent sentences) of the source collocation.

A different approach, which uses non-related monolingual corpora for finding bilingual
collocations, was presented in \citet{lu2004}. Here, the authors apply dependency parsing
and the \textsc{log-likelihood} ratio for obtaining English and Chinese collocations. Then,
they search for translations using word translation equivalents with the same dependency
relation in the target language (using the EM algorithm and a bilingual dictionary).

Although not focused on collocations, Pascale Fung applied methods based on distributional
semantics to build bilingual lexica from comparable corpora \citep{fung1998}.
This approach takes into account that in this type of resources the position and the frequency
of the source and target words are not comparable, and also that the translations of the source
words might not exist in the target document.

Similarly, the strategy presented in this chapter leverages noisy parallel corpora for building
bilingual word embeddings. However, with a view to applying it in other scenarios (such as comparable
corpora), it identifies equivalents without using information about the position of the collocations
-- neither their comparative frequency -- in the corpora. Furthermore, it does not take advantage
of external resources such as bilingual dictionaries, making it easy to extend to other languages.
\citet{pacorbook} had introduced a naive version of this approach, including experiments in Portuguese and Spanish with just one collocation pattern.


%
\section{A new method for bilingual collocation extraction}
\label{garcia:sec:method}
This section presents the proposed method for automatically extracting bilingual collocations from corpora.
First, the approach for identifying candidates of monolingual collocations using syntactic dependencies
is briefly described. Then, the process of creating the bilingual word embeddings is shown,
followed by the strategy for discovering the collocation equivalents between languages.

%
\subsection{Monolingual dependency-based collocation extraction}
\label{garcia:sec:monolingual}
Early works on n-gram based collocation extraction already pointed out the
need for using syntactic analysis for better identifying collocations from corpora
\citep{smadja1993,lin1999}. Syntactic analysis can, on the one hand, avoid the
extraction of syntactically unrelated words which occur in a small context
windows. On the other hand, it can effectively identify the syntactic relation
between lexical items occurring in long-distance dependencies \citep{evert2008}.

Besides, and even though it is not always the case \citep{lu2004}, the method presented
in this chapter assumes that most bilingual equivalent of collocations bear the same syntactic
relation in both the source and the target languages.

In order to better capture the syntactic relations between the \textsc{base} and
the \textsc{collocate} of each collocation, the strategy uses state-of-the-art dependency
parsing. Apart from that, and aimed at obtaining harmonized syntactic information
between languages, the method relies on \textsc{universal dependencies} annotation,
which makes it possible to use the same strategy for extracting and analyzing the collocations in multiple
languages.\footnote{\scriptsize{\url{http://universaldependencies.org/u/dep/all.html}}}

\subsubsection{Preprocessing:} Before extracting the collocation candidates from each
corpus, a pipeline of NLP tools is applied in order to annotate the text with the
desired information. Thus, the output of this process consists of a parsed corpus
in CoNLL-U format, where each word is assigned to its surface form, its lemma, its
POS-tag and morphosyntactic features, its syntactic head as well as the UD
relation of the word in context.\footnote{\scriptsize{\url{http://universaldependencies.org/format.html}}}

From this analyzed corpus, the word pairs belonging to the desired relations
(collocation candidates) are extracted. We keep their surface forms, POS-tags,
and other syntactic dependents which may be useful for the identification
of potential collocations. Besides, a list of triples is retained in order to apply association measures,
containing (i) the syntactic relation, (ii) the head, and (iii) the dependent (using their lemmas
together with the POS-tags). Thus, from a sentence such as \emph{John took a great responsibility},
the following triples (among others) are obtained:

\begin{quotation}
  \noindent\textsc{nsubj}(take\textsubscript{\textsc{Verb}},John\textsubscript{\textsc{PropN}})\\
  \textsc{amod}(responsibility\textsubscript{\textsc{Noun}},great\textsubscript{\textsc{Adj}})\\
  \textsc{dobj}(take\textsubscript{\textsc{Verb}},responsibility\textsubscript{\textsc{Noun}})
\end{quotation}

This information (and also the corpus size and the frequency of the different elements
of the potential collocations) is saved in order to rank the candidates.

\subsubsection{Collocation patterns:} In this chapter, candidates of five different syntactic
patterns of collocations are extracted in three languages (Spanish -- ES --, Portuguese
-- PT --, and English -- EN):\footnote{In this chapter we address the European variety of Portuguese.
  However, even if we use a European Portuguese corpus (see \sectref{garcia:sec:experiments}),
  it contains some texts in the Brazilian dialect.}

\begin{itemize}
\item Adjective---Noun (\textsc{amod}): these candidates are pairs of adjectives
(as \textsc{collocates}) and nouns (as \textsc{bases}) where the former syntactically depends
of the latter in a \textsc{amod} relation.
Example: \textbf{killer}\textsubscript{base};\textbf{serial}\textsubscript{collocate}.

\item Noun---Noun (\textsc{nmod}): this pattern consists of two common
nouns related by the \textsc{nmod} relation, where the head is the \textsc{base}
and the dependent is the \textsc{collocate} (optionally with a \textsc{case} marking
dependent preposition: \emph{of} in English, \emph{de} in Portuguese and Spanish).
Example: \textbf{rage}\textsubscript{base};\textbf{fit}\textsubscript{collocate}.\footnote{Some
  collocations belonging to this pattern are analyzed in UD -- mainly in English --
  using the \textsc{compound} relation, so they are not extracted in the experiments performed in this chapter.}

\item Verb---Object (\textsc{vobj}): \emph{verb-object} collocations consist
of a verb (the \textsc{collocate}) and a common noun (the \textsc{base})
occurring in a \textsc{dobj} relation.
Example: \textbf{care}\textsubscript{base};\textbf{take}\textsubscript{collocate}.

\item Subj---Verb (\textsc{vsubj}): the \textsc{vsubj} collocation pattern contains a
  common noun (the \textsc{base}, acting as a subject) and the verb it depends on
  (the \textsc{collocate}).
  Example: \textbf{ship}\textsubscript{base};\textbf{sink}\textsubscript{collocate}.
  
\item Verb---Adverb (\textsc{advmod}): in this case, a \textsc{collocate} adverb modifies
  a verb (the \textsc{base}) in an \textsc{advmod} relation.
  Example: \textbf{want}\textsubscript{base};\textbf{really}\textsubscript{collocate}.
\end{itemize}

\subsubsection{Identification of candidates:} For each of the five patterns of
collocations, a list of potential candidates for the three languages is extracted.
After that, the candidates are ranked using standard association measures that have
been widely used in collocation extraction (\textsc{mi, t-score, z-score, Dice, log-likelihood},
etc.) \citep{evert2008}.

In the current experiments, two statistical measures were selected, whose results
complement each other: \textsc{t-score}, which prefers frequent dependency pairs,
and has been proved useful for collocation extraction \citep{krenn2001}, and
\textsc{Mutual Information}, which is useful for a large corpus, even if it
tends to assign high scores to candidates with very low-frequency \citep{pecina2010lexical}.

The output of both association measures is merged in a final list for each
language and collocation pattern, defining thresholds of \emph{\textsc{t-score$>$=2}} and
\emph{\textsc{mi}$>$=3} \citep{stubbs1995}, and extracting only collocations with a frequency
of \emph{f$>$=10}. This large value was defined to reduce the extraction of incorrect
entries from a noisy corpus and from potential errors of the automatic analysis.

It must be noted that, since these lists of monolingual collocations have been built
based on statistical measures of collocability, their members need not be
\emph{bona fide} collocations in the phraseological meaning. Thus, the lists can include
idioms (e.g., kick the bucket), quasi-idioms (e.g., big deal) \citep{melcuk98},
or free combinations (e.g., buy a drink).

%
\subsection{Bilingual word embeddings}
\label{garcia:sec:models}
Word embeddings are low-dimensional vector representations of words which
capture their distributional context in corpora. Even though distributional
semantics methods have been largely used in previous years, approaches based
on word embeddings have gained popularity with the publication of \emph{word2vec}
\citep{mikolov2013}.

Based on the \emph{Skip-gram} model of \emph{word2vec}, \citet{bivec} proposed
\emph{BiSkip}, a model of word embeddings which learns bilingual representations
using aligned corpora, thus being able to predict words crosslinguistically.

The method presented in this chapter uses lemmas (instead of surface forms) to identify the
collocation candidates, so the bilingual models of word embeddings are also trained on
lemmatized corpora. Therefore, the raw parallel corpus is lemmatized keeping the original sentence alignment.

The bilingual models are built (on the lemmatized version of the corpora) using MultiVec,
an implementation of \emph{word2vec} and \emph{BiSkip} \citep{MultiVec}. As the approach
is evaluated in three languages, three different bilingual models are needed:
Spanish-English, Spanish-Portuguese, and Portuguese-English.

As it will be shown, the obtained models can predict the similarity between words
in bilingual scenarios by computing the cosine similarity between their vectors. As
the models learn the distribution of single words (lemmas), they deal with different
semantic phenomena such as polysemy or homonymy. Concerning collocations, this means
that, ideally, the bilingual models could predict not only the equivalents of a
\textsc{base}, but also to capture the (less close) semantic relation between the
bilingual \textsc{collocates}, if they occur an enough number of times in the data.

%
\subsection{Bilingual collocation alignment}
\label{garcia:sec:align}
In order to identify the bilingual equivalent of a collocation (in a target language),
the method needs (i) lists of monolingual collocations (ideally obtained from similar
resources), and (ii) a bilingual \emph{source-target} model of word embeddings.

With these resources, the following strategy is applied: For each collocation
in the source language (e.g., \textbf{lío}\textsubscript{base};\textbf{tremendo}\textsubscript{collocate},
`huge mess' in Spanish) the system selects its \textsc{base} and obtains -- using the bilingual model --
the \emph{n} most similar lemmas in the target language (where \emph{n=5} in the experiments
performed in this chapter): \textit{trouble}, \textit{mess}, etc. Then, starting from the most similar lemma,
we search in the target list for collocations containing the equivalents of
the \textsc{base} (\textit{trouble}\textsubscript{base};\textit{little}\textsubscript{collocate},
\textit{trouble}\textsubscript{base};\textit{deep}\textsubscript{collocate}, \textit{mess}\textsubscript{base};\textit{huge}\textsubscript{collocate}, \textit{mess}\textsubscript{base};\textit{fine}\textsubscript{collocate}, etc.). If a collocation with
a \textsc{base} equivalent is found, the cosine similarity between both \textsc{collocates}
(\textit{tremendo} versus \textit{little}, \textit{deep}, \textit{huge}, and \textit{fine}) is computed, and
they are selected as potential candidates if their similarity is higher than a given threshold
(empirically defined in this chapter as $0.65$), and if the target candidate is among the \emph{n}
most similar words of the source \textsc{collocate} (again, \emph{n=5}). Finally, if these conditions
are met, the source and target collocations are aligned, assigning the average distance between the
\textsc{bases} and the \textsc{collocates} as a confidence value, as in the following example:
ES-EN:\textbf{lío}\textsubscript{base}; \textbf{tremendo}\textsubscript{collocate}=\textbf{mess}\textsubscript{base};\textbf{huge}\textsubscript{collocate};0.721.

%
\section{Evaluation}
\label{garcia:sec:experiments}
This section presents the experiments carried out in order to evaluate the proposed distributional
method (henceforth \textsc{DiS}) in the three analyzed languages, using the five collocation
patterns defined in \sectref{garcia:sec:monolingual}. The approach presented in this chapter is
compared to a baseline system (\textsc{Bas}) which uses hand-crafted bilingual dictionaries,
and to probabilistic translation dictionaries (\textsc{Nat}).\footnote{The extractions
  of these three methods are available at\\\scriptsize{\url{http://www.grupolys.org/~marcos/pub/pmwe-dis.tar.bz2}}}

\paragraph*{Corpora:} Monolingual collocations were extracted from a subset of the
OpenSubtitles2016 corpus \citep{opus}, which contains parallel corpora from TV and Movie
subtitles. This resource was selected because it is a large and multilingual parallel corpus
likely to contain different types of collocations (also from an informal register),
thus being useful for comparative studies.\footnote{Note, however, that
  OpenSubtitles2016 includes non-professional translations with some noisy elements such as typos
  or case inconsistencies, among others.}

From the English, Spanish, and Portuguese corpora, those senteces which appear in the three
languages were selected (a total of 13,017,016). These sentences were tokenized, lemmatized
and POS-tagged with a multilingual NLP pipeline \citep{Citius}, obtaining three corpora of
$\approx91$M (ES and PT), and $\approx98$M (EN) tokens. The resulting data
were enriched with syntactic annotation using statistical models trained with MaltParser \citep{MaltParser}
and the $1.4$ version of the UD treebanks \citep{univdep}.

\begin{table}
  \begin{center}
    %\setlength\tabcolsep{4pt}
    {\small{
        \caption{\label{tab:mono} Number of unique input dependencies for each syntactic pattern (\emph{deps}),
          and final monolingual collocation candidates (\emph{colls}).}
    \begin{tabular}{lrrrrrrrrrr}
      \lsptoprule
      \textbf{\multirow{2}{*}{Lg}} & \multicolumn{2}{c}{\textbf{\textsc{amod}}} & \multicolumn{2}{c}{\textbf{\textsc{nmod}}} & \multicolumn{2}{c}{\textbf{\textsc{vobj}}} & \multicolumn{2}{c}{\textbf{\textsc{vsubj}}} & \multicolumn{2}{c}{\textbf{\textsc{advmod}}} \\
      & \emph{deps} & \emph{colls} & \emph{deps} & \emph{colls} & \emph{deps} & \emph{colls} & \emph{deps} & \emph{colls} & \emph{deps} & \emph{colls}\\
      \midrule
      \emph{ES}   & 373k & 13,870   & 644k & 5,673   & 423k & 17,723 & 287k & 4,914 & 124k & 5,526 \\
      \emph{PT}   & 361k & 12,967   & 709k & 5,643   & 544k & 20,984 & 283k & 3,927 & 142k & 6,660 \\
      \emph{EN}   & 381k & 14,175   & 517k & 3,133   & 483k & 15,492 & 264k & 2,663 & 162k & 6,711 \\
      \lspbottomrule
    \end{tabular}
    }}
  \end{center}
\end{table}

\paragraph*{Collocations:} From each corpus, five patterns of collocation candidates
were extracted: \textsc{amod}, \textsc{nmod}, \textsc{vobj}, \textsc{vsubj}, and \textsc{advmod}.
For each language and pattern, a single list of collocations was obtained by merging the \textsc{mi}
and \textsc{t-score} outputs as explained in \sectref{garcia:sec:monolingual}. \tabref{tab:mono}
shows the number of filtered collocations in each case (\emph{colls}).

Another version of each corpus was created only with the lemma of each token,
keeping the original sentence alignments. These corpora were used for training
three bilingual word embeddings models with MultiVec (with $100$ dimensions and
a window-size of $8$ words): ES-EN, ES-PT, and PT-EN.\footnote{These models
  are available at \scriptsize{\url{http://www.grupolys.org/~marcos/pub/mwe17_models.tar.bz2}}}

\paragraph*{Baseline (\textsc{Bas}):} The performance of the method described in \sectref{garcia:sec:align}
was compared to a baseline which follows the same strategy, but using bilingual
dictionaries instead of the word embeddings models. Thus, the \textsc{Bas} method obtains the
equivalents of both the \textsc{base} and the \textsc{collocate} of a source collocation,
and verifies whether there is a target collocation with the translations. The bilingual
dictionaries provided by the \emph{apertium} project (SVN revision 75,477) were used for
these experiments \citep{apertium}.\footnote{\scriptsize{\url{https://svn.code.sf.net/p/apertium/svn/}}}

The Spanish-Portuguese dictionary has $14,364$ entries, and the Spanish-English one
contains $34,994$. The Portuguese-English dictionary (not provided by \emph{apertium})
was automatically obtained by transitivity from the two other lexica, with
a size of $9,160$ pairs.

\paragraph*{Probabilistic translation dictionaries (\textsc{Nat}) :} The distributional method
was also compared to probabilistic translation dictionaries. Probabilistic dictionaries are
bilingual resources which contain, for each word in a source language, possible translations
in the target language together with the probability of the translation being correct.
To obtain these dictionaries NATools was used, which is a set of tools to work with parallel corpora
that can be utilized for different tasks such as sentence and word alignment, or to extract bilingual
translation dictionaries by means of statistical methods \citep{natools}. The probabilistic dictionaries
are obtained by applying the EM algorithm on sparse matrices of bilingual word co-occurrences, previously
built from parallel corpora \citep{hiemstra1998}.

Aimed at a better comparison to the \textsc{DiS} model, \textsc{Nat} dictionaries were
extracted from the same lemmatized resources used for training the bilingual word embeddings.
Thus, this method only differs from the \textsc{DiS} one in the bilingual resources used to
search for equivalents of the \textsc{bases} and the \textsc{collocates}.\footnote{After preliminary
  evaluations, the translation probability thresholds of both LUs were empirically defined as $0.1$.}

%
\subsection{Results}
With a view to knowing the performance of \textsc{Bas}, \textsc{Nat}, and \textsc{DiS}
in the different scenarios, $100$ bilingual collocation pairs were randomly selected
from each language and pattern, creating a total of 45 lists (15 from each of the three methods).\footnote{Except for those baseline extractions with less than $100$ elements, where all of them were selected.}

Three reviewers worked during the evaluation process. Each bilingual collocation
pair was labeled as (i) correct, (ii) incorrect, or (iii) dubious (which includes pairs where the
translation might be correct in some contexts even if they were not considered faithful
translations).\footnote{Some of these dubious equivalents are actual translations in the
  original corpus, such as the Spanish-English \textit{copa de champaña} (`champagne cup') --- \textit{cup of wine},
  even if they are semantically different.}
Correct collocation equivalents are those pairs where the monolingual extractions were
considered correct (both in terms of co-occurrence frequency and of collocation pattern
classification), and that their translations were judged by the reviewers as potential
translations in a real scenario.
In the \textsc{Bas} and \textsc{DiS} outputs, two reviewers labeled each collocation pair,
achieving 92\% and 83\% inter-annotator agreement, respectively (with an average $\kappa=0.39$, which
indicates the difficulty of this kind of annotation). Those pairs with correct/incorrect disagreement
were discarded for the evaluation. Those with at least one dubious label were checked by a
third annotator, deciding in each case whether they were correct, incorrect, or dubious.
This third annotator evaluated the outputs of \textsc{Nat} using exactly the same guidelines.

\begin{table}
  \begin{center}
    %{\small{
    %\setlength\tabcolsep{4pt}
    \caption{\label{tab:extract} Number of bilingual extractions of the baseline,
      \textsc{Nat}, and \textsc{DiS} systems.}
    \begin{tabular}{llrrr}
      \lsptoprule
      \textbf{Pattern} & \textbf{model} & \textbf{ES-PT} & \textbf{ES-EN} & \textbf{PT-EN}\\
      \midrule
	\multirow{3}{*}{\textsc{amod}} & \textsc{Bas} & 657 & 248 & 213\\ 
	& \textsc{Nat} & 1,329 & 1,113 & 1,005\\ 
	& \textsc{DiS} & 9,464 & 7,778 & 7,083\\ \hline
	\multirow{3}{*}{\textsc{nmod}} & \textsc{Bas} & 320 & 32 & 43\\ 
	& \textsc{Nat} & 704 & 138 & 136\\ 
	& \textsc{DiS} & 3,867 & 890 & 917\\ \hline
	\multirow{3}{*}{\textsc{vobj}} & \textsc{Bas} & 529 & 183 & 241\\ 
	& \textsc{Nat} & 1,443 & 1,461 & 1,544\\ 
	& \textsc{DiS} & 12,887 & 8,865 & 9,206\\ \hline
	\multirow{3}{*}{\textsc{vsubj}} & \textsc{Bas} & 188 & 27 & 55\\ 
	& \textsc{Nat} & 382 & 346 & 323\\ 
	& \textsc{DiS} & 2,522 & 1,344 & 1,298\\ \hline
	\multirow{3}{*}{\textsc{advmod}} & \textsc{Bas} & 58 & 19 & 22\\ 
	& \textsc{Nat} & 113 & 104 & 106\\ 
	& \textsc{DiS} & 3,721 & 2,301 & 2,412\\
      \lspbottomrule
    \end{tabular}
    %}}
  \end{center}
\end{table}


\begin{table}
  \begin{center}
    \caption{\label{tab:bas} Precision, recall and f-score of the baseline (\textsc{Bas}) system (\emph{average} is macro-average).}
    \begin{tabular}{llrrrr}
      \lsptoprule
      \multicolumn{2}{l}{\textbf{Pattern}} & \textbf{ES-PT} & \textbf{ES-EN} & \textbf{PT-EN} & \textbf{average}\\
      \midrule
	\multirow{3}{*}{\textsc{amod}} & P & 99.0 & 95.8 & 97.9 & 97.6\\ 
	& R & 5.0 & 1.7 & 1.6 & 2.8\\ 
	& F1 & 9.6 & 3.4 & 3.2 & 5.4\\ \hline
	\multirow{3}{*}{\textsc{nmod}} & P & 97.8 & 100\phantom{.0} & 91.7 & 96.5\\ 
	& R & 5.5 & 1.0 & 1.3 & 2.6\\ 
	& F1 & 10.5 & 2.0 & 2.5 & 5.1\\  \hline
	\multirow{3}{*}{\textsc{vobj}} & P & 98.7 & 100\phantom{.0} & 92.1 & 96.9\\ 
	& R & 3.0 & 1.2 & 1.4 & 1.9\\ 
	& F1 & 5.7 & 2.3 & 2.8 & 3.6\\  \hline
	\multirow{3}{*}{\textsc{vsubj}} & P & 93.8 & 96.3 & 92.7 & 94.3\\ 
	& R & 4.5 & 1.0 & 1.9 & 2.5\\ 
	& F1 & 8.6 & 1.9 & 3.8 & 4.8\\  \hline
	\multirow{3}{*}{\textsc{advmod}} & P & 96.7 & 100\phantom{.0} & 95.7 & 97.4\\ 
	& R & 1.0 & 0.3 & 0.3 & 0.6\\ 
	& F1 & 2.0 & 0.7 & 0.6 & 1.1\\  \hline
	\multirow{3}{*}{\emph{average}} & P & 97.2 & 98.4 & 94.0 & 96.5\\ 
	& R & 3.8 & 1.0 & 1.3 & 2.1\\ 
	& F1 & 7.3 & 2.1 & 2.6 & 4.0\\
      \lspbottomrule
    \end{tabular}
  \end{center}
\end{table}


\begin{table}
  \begin{center}
    \caption{\label{tab:nat} Precision, recall and f-score of the probabilistic (\textsc{Nat}) system (\emph{average} is macro-average).}
    \begin{tabular}{llrrrr}
      \lsptoprule
      \multicolumn{2}{l}{\textbf{Pattern}} & \textbf{ES-PT} & \textbf{ES-EN} & \textbf{PT-EN} & \textbf{average}\\
      \midrule
	\multirow{3}{*}{\textsc{amod}} & P & 92.5 & 92.5 & 83.3 & 89.5\\
	& R & 9.5 & 7.4 & 6.5 & 7.8\\
	& F1 & 17.2 & 13.8 & 12.0 & 14.3\\ \hline
	\multirow{3}{*}{\textsc{nmod}} & P & 91.1 & 98.7 & 91.4 & 93.7\\
	& R & 11.4 & 4.4 & 4.0 & 6.6\\
	& F1 & 20.2 & 8.3 & 7.6 & 12.1\\ \hline
	\multirow{3}{*}{\textsc{vobj}} & P & 95.2 & 80.0 & 92.7 & 89.3\\
	& R & 7.8 & 7.5 & 9.2 & 8.2\\
	& F1 & 14.3 & 13.8 & 16.8 & 15.0\\ \hline
	\multirow{3}{*}{\textsc{vsubj}} & P & 82.4 & 78.6 & 79.2 & 80.0\\
	& R & 8.0 & 10.2 & 9.6 & 9.3\\
	& F1 & 14.6 & 18.1 & 17.1 & 16.6\\ \hline
	\multirow{3}{*}{\textsc{advmod}} & P & 59.2 & 78.8 & 83.3 & 73.8\\
	& R & 1.2 & 1.5 & 1.3 & 1.3\\
	& F1 & 2.4 & 2.9 & 2.6 & 2.6\\ \hline
	\multirow{3}{*}{\emph{average}} & P & 84.1 & 85.7 & 86.0 & 85.3\\
	& R & 7.6 & 6.2 & 6.1 & 6.6\\
	& F1 & 13.8 & 11.4 & 11.2 & 12.1\\
      \lspbottomrule
    \end{tabular}
  \end{center}
\end{table}


\begin{table}
  \begin{center}
    \caption{\label{tab:sys} Precision, recall and f-score of \textsc{DiS} system (\emph{average} is macro-average).}
    \begin{tabular}{llrrrr}
      \lsptoprule
      \multicolumn{2}{l}{\textbf{Pattern}} & \textbf{ES-PT} & \textbf{ES-EN} & \textbf{PT-EN} & \textbf{average}\\
      \midrule
	\multirow{3}{*}{\textsc{amod}} & P & 92.9 & 92.0 & 90.5 & 91.8\\ 
	& R & 67.8 & 51.6 & 49.5 & 56.3\\ 
	& F1 & 78.4 & 64.3 & 64.0 & 68.9\\ \hline
	\multirow{3}{*}{\textsc{nmod}} & P & 93.8 & 88.0 & 90.0 & 90.6\\ 
	& R & 64.3 & 25.0 & 26.3 & 38.5\\ 
	& F1 & 76.3 & 38.9 & 40.1 & 51.9\\ \hline
	\multirow{3}{*}{\textsc{vobj}} & P & 90.1 & 84.0 & 83.9 & 86.2\\ 
	& R & 66.0 & 48.1 & 49.9 & 54.7\\ 
	& F1 & 76.5 & 61.2 & 62.6 & 66.7\\ \hline
	\multirow{3}{*}{\textsc{vsubj}} & P & 80.3 & 81.2 & 74.1 & 78.5\\ 
	& R & 51.6 & 41.0 & 36.1 & 42.9\\ 
	& F1 & 62.8 & 54.5 & 48.6 & 55.3\\ \hline
	\multirow{3}{*}{\textsc{advmod}} & P & 77.6 & 83.3 & 67.4 & 76.1\\ 
	& R & 52.2 & 34.7 & 24.4 & 37.1\\ 
	& F1 & 62.4 & 49.0 & 35.8 & 49.1\\ \hline
	\multirow{3}{*}{\emph{average}} & P & 86.9 & 85.7 & 81.2 & 84.6\\ 
	& R & 60.4 & 40.1 & 37.3 & 45.9\\ 
	& F1 & 71.3 & 53.6 & 50.2 & 58.4\\
      \lspbottomrule
    \end{tabular}
  \end{center}  
\end{table}

From these data, the precision values for each case were obtained by dividing the number of
correct collocation equivalents by the number of correct, incorrect, and dubious cases (so dubious
cases were considered incorrect). Recall (\textit{r}) was obtained by multiplying the precision values (\textit{p})
for the number of extracted equivalents (\textit{e}), and dividing the result by the smallest number of input collocations
for each pair (\textit{i}, see \tabref{tab:mono}). For instance, the Spanish-Portuguese baseline recall for the \textsc{amod}
pattern was estimated as follows (see \tabref{tab:mono}, \tabref{tab:extract}, and \tabref{tab:bas}):
$r = \frac{p * e}{i} = \frac{99 * 657}{12,967} = 5.01$.\footnote{Note that these recall results assume that
  every collocation in the shortest input list of each pair has an equivalent on the other language,
  which is not always the case. Thus, more realistic recall values (which would need an
  evaluation of every extracted pair) will be higher than the ones obtained in these experiments.}
Finally, f-score values (the harmonic mean between precision and recall) were obtained
for each case, and the macro-average results were calculated for each language, pattern, and approach.

\tabref{tab:extract} contains the number of bilingual collocation equivalents extracted by
each method in the 15 settings, from the input lists of monolingual data (\tabref{tab:mono}).
These results clearly show that the baseline approach extracts a lower number of bilingual equivalents.
\textsc{Nat} obtains much more bilingual collocations than \textsc{Bas}, but both methods
extract less equivalents than the distributional approach. This might have happened due to the
size of the dictionaries (in \textsc{Bas}) and because of the internal properties of the collocations
(in both \textsc{Bas} and \textsc{Nat}), where the \textsc{collocates} may not be direct translations
of each other. Moreover, it is worth noting that with the three strategies, the bilingual extractions
including English are smaller than the Spanish-Portuguese ones.

Concerning the performance of the three approaches, \tabref{tab:bas} (baseline),
\tabref{tab:nat} (\textsc{Nat}), and \tabref{tab:sys} (\textsc{DiS}) contain the precision,
recall and f-score for each language pair and collocation pattern.

\textsc{Bas} obtains high-precision results in every language and collocation
pattern (91.7\% in the worst scenario), with a macro-average value of 96.5\%. These
results are somehow expected due to the quality of the hand-crafted dictionaries. However,
because of the poor recall numbers, the general performance of \textsc{Bas} is low,
achieving f-score results around $4\%$. Interestingly, the size of the dictionary
does not seem crucial to the results of the baseline. In this respect, the
Spanish-Portuguese results are much better (specially in recall) than Spanish-English, whose
dictionary size is more than the double. Also, the Portuguese-English results are slightly
better than the Spanish-Portuguese ones, the latter being obtained using a dictionary
built by transitivity.

The use of probabilistic translation dictionaries (\textsc{Nat}) increases the recall
(by a factor higher than 3) when compared to the baseline, but with a cost in precision
(which drops, in average, from $96.5\%$ to $85.3\%$). However, these differences
allow the \textsc{Nat} approach to obtain much better f-score results than the
baseline. When looking at the different collocation patterns, it is worth noting
that while \textsc{amod, nmod}, and \textsc{vobj} have precision values of about $90\%$,
\textsc{vsubj}, and specially \textsc{advmod} (also with very low recall values)
do not overpass $80\%$ (with one case, ES-PT, with $<60\%$). As it will be shown
in \sectref{garcia:sec:error}, some preprocessing issues might be the source of the some
errors of the \textsc{advmod} extractions.

As for the \textsc{DiS} model, its precision is again lower than the baseline and very similar
to the \textsc{Nat} approach, with average results of $84.6\%$. However, the distributional
strategy finds much more bilingual equivalents than the dictionaries, so
recall values increase to an average of more than $45\%$. Again, \textsc{vsubj} and
\textsc{advmod} showed worse precision values than the other three patterns.
Besides, the \textsc{nmod} extractions of the pairs including English have very low recall
when compared to the other results, maybe derived from not having extracted nouns analyzed
as \textsc{compound} (\sectref{garcia:sec:monolingual}).
As in the other two methods, the \textsc{DiS} Spanish-Portuguese results are better than the two
other language pairs, so the linguistic distance seems to play an important role on bilingual
collocation extraction.

The method proposed in this chapter assigns a confidence value (obtained from the cosine
similarity between the vectors of the \textsc{base} and the \textsc{collocate} equivalents)
to each bilingual pair of collocations. In this respect, \figref{fig:curva} plots
the average performance and confidence curves versus the total number of extracted pairs.
This figure shows that using a high confidence value ($>90\%$), it is possible
to extract $\approx40,000$ bilingual pairs with a high degree of precision. Besides, it is
worth mentioning that filtering the extraction with confidence values higher than
$90\%$ does not increase the precision of the system, so it can be inferred that the errors
produced in the most confident pairs arise due factors other than the semantic similarity
(e.g. different degrees of compositionality). However, as the confidence value decreases
the precision of the extraction also gets worse, despite the rise in the number of extractions
which involves higher recall and consequently better f-score.

Finally, all the bilingual collocations extracted by \textsc{DiS} were merged
into a single list with the three languages, thus obtaining new bilingual
equivalents (not extracted directly by the system) by transitivity.\footnote{The merging process
  obtained $6,969$ new bilingual collocation equivalents not present in the original extractions, and it also
  includes more than one translation for some collocations.}
This final multilingual resource has $74,942$ entries, $38,629$ of them with translations in the three languages.

\begin{figure}
  \centering
  \includegraphics[width=\textwidth]{figures/curve.eps}
  \caption{\label{fig:curva}Average precision, recall, f-score, and confidence
    curves (from 0 to 1) versus total number of extractions of the \textsc{DiS} model.}
\end{figure}

%
\subsection{Error analysis}
\label{garcia:sec:error}
The manually annotated lists of bilingual collocations were used to perform an error analysis
of the presented strategy. These errors were classified in five types depending on
their origin. \tabref{tab:error} contains, for each error type, the macro-average rates
of each collocation pattern as well as the final distribution of the error typology.

\begin{enumerate}
\item \textbf{Bilingual model (\textit{BiModel}):} The bilingual word embeddings approach,
  though useful, produces some errors such as the identification of antonyms
  (with a similar distribution), which can align opposite collocation
  equivalents (such as PT-EN:\textbf{tecido}\textsubscript{base};\textbf{vivo}\textsubscript{collocate}=\textbf{tissue}\textsubscript{base};\textbf{dead}\textsubscript{collocate}, instead of \textit{living tissue})
  where the extracted equivalent of the \textsc{collocate} \textit{vivo} (`living' -- in this context -- or `alive', in Portuguese)
  was \textit{dead}. In most cases, however, the system obtained similar
  (but not synonym) collocations: PT-EN:\textbf{chá}\textsubscript{base};\textbf{preto}\textsubscript{collocate}=\textbf{coffee}\textsubscript{base};\textbf{black}\textsubscript{collocate}
  (\textit{black tea}, \textit{black coffee}).

\item \textbf{Monolingual extraction (\textit{MonoExtract}):} The extraction of \textsc{base} and
  \textsc{collocate} pairs produced incorrect collocations such as
  \textbf{plan}\textsubscript{base};\textbf{figure}\textsubscript{collocate},
  instead of obtaining the phrasal verb \textit{figure out} as \textsc{collocate}.  
  
\item \textbf{Preprocessing (\textit{NLP}):} Several errors derived from issues produced
  by the NLP pipeline, such as POS-tagging or dependency parsing:
  e.g., \textit{pain}\textsubscript{Noun},\\\textit{end}\textsubscript{Verb} was labeled
  as \textsc{dobj} (instead of \textsc{nsubj}).
  An special case of preprocessing errors was the analysis of some Portuguese
  and Spanish adverbs ending in \emph{--mente} (\emph{-ly} adverbs in English),
  whose suffix was wrongly removed during the extraction process: e.g. \textit{brutalmente}
  (`brutally') $\rightarrow$ \textit{brutal}. These issues -- which can be easily corrected --
  caused the alignment of Spanish and Portuguese incorrect collocations with
  English candidates, such as the PT-EN:\textbf{matar}\textsubscript{base};\textbf{brutal}\textsubscript{collocate}=\textbf{kill}\textsubscript{base};\textbf{brutally}\textsubscript{collocate} (instead of \textbf{matar}\textsubscript{base};\textbf{brutalmente}\textsubscript{collocate}=\textbf{kill}\textsubscript{base};\textbf{brutally}\textsubscript{collocate}), and it was the main source of errors of the \textsc{advmod} relation.

\item \textbf{Lemmatization and gender (\textit{Gender}):} The lemmatization of some words
  differs from language to language, so working with lemmas instead of
  tokens also might involve some errors. For instance, the word (ES)
  \textit{hija} (`daughter') is lemmatized as (ES) \textit{hijo} (`son')
  (also in Portuguese: \textit{filha}, \textit{filho}), while in English \textit{son}
  and \textit{daughter} appear as different entries. Thus, some bilingual
  collocations differ in the gender of their \textsc{bases}, such as the
  Spanish-English \textbf{hijo}\textsubscript{base};\textbf{encantador}\textsubscript{collocate}=\textbf{daughter}\textsubscript{base};\textbf{lovely}\textsubscript{collocate} (instead of \textbf{hijo}\textsubscript{base};\textbf{encantador}\textsubscript{collocate}=\textbf{son}\textsubscript{base};\textbf{lovely}\textsubscript{collocate}).

\item \textbf{Other errors (\textit{Other}):} Some other errors were caused by mixed languages
  in the original corpus (e.g., the verb form \emph{are}, in English, was analyzed
  as a form of the verb \textit{arar} -- `to plow' --, in Spanish) and from noise
  and misspellings in the corpora (proper nouns with lowercase letters, etc.).
\end{enumerate}

It is worth mentioning that, in general, the error type distribution was similar across
the different collocation patterns, showing much higher variation between different patterns
of the same language pair (the distribution, for instance, of Spanish-English \textsc{amod} errors
is similar to the Portuguese-English \textsc{amod} one, while the typology of the Spanish-Portuguese \textsc{nmod}
errors is different to those of Spanish-Portuguese \textsc{amod} equivalents).

\begin{table}
  \begin{center}
    \caption{\label{tab:error} Error rate of each of the defined types of \textsc{DiS} system (\emph{average} is macro-average).}
    \begin{tabular}{lrrrrrr}
      \lsptoprule
      \textbf{\emph{Type}}&\textbf{\textsc{amod}}&\textbf{\textsc{nmod}}&\textbf{\textsc{vobj}}&\textbf{\textsc{vsubj}}&\textbf{\textsc{advmod}}&\textbf{\emph{average}}\\
      \midrule
      \textbf{BiModel} & 70.57 & 93.52 & 59.23 & 45.74 & 32.61                 & 60.33\\
      \textbf{MonoExtract} & 0\phantom{00} & 0\phantom{00} & 21.43 & 21.85 & 44.94 & 17.64\\
      \textbf{NLP} & 8.34  & 0\phantom{00} & 16.96 & 11.48 & 20.49         & 11.45\\
      \textbf{Gender} & 21.10 & 2.78  & 2.38  & 19.07 & 0\phantom{00}         & 9.07\\
      \textbf{Other} & 0\phantom{00} & 3.70  & 0\phantom{00} & 1.85  & 1.96  & 1.50\\
      \lspbottomrule
    \end{tabular}
  \end{center}
\end{table}

Among the different errors produced by the presented method, an interesting case
are \emph{incongruent} collocations \citep{nesselhauf2003}. These expressions are
those equivalents where the translation of both elements is not coherent, e.g.,
EN-PT:\textbf{requirement}\textsubscript{base};\textbf{meet}\textsubscript{collocate}=\textbf{condição}\textsubscript{base};\textbf{cumprir}\textsubscript{collocate} (since the verb \textit{to meet} is usually translated into Portuguese as \textit{conhecer}, not as \textit{cumprir}).
For these collocation equivalents to be correctly extracted by our method, they should
appear with some frequency in the training corpus, which is not always the case. This fact
may lead us to explore new compositional models, aimed at learning the distribution of the
whole collocation (not of its constituents), in further work.

\subsection{Comparable corpora}
A final experiment was carried out in order to know (i) whether the bilingual
word embeddings -- trained in the same parallel corpora as those used for
extracting the collocations -- could be successfully applied for aligning collocations
obtained from different resources, and (ii) the performance of the proposed
method in comparable corpora.

Therefore the same strategy for monolingual collocation extraction was applied in the Spanish and Portuguese
\emph{Wikipedia Comparable Corpus 2014}.\footnote{\scriptsize{\url{http://linguatools.org/tools/corpora/wikipedia-comparable-corpora/}}}
Then, we calculated the semantic similarity between the collocations using the same word embeddings
models as in the previous experiments.

From these corpora, filtered lists of $89,285$ and $140,900$ candidate collocations in Portuguese
and Spanish were obtained, respectively (from 140M, and 80M of tokens). From the $59,507$ bilingual
collocations obtained by the \textsc{DiS} approach, $150$ Spanish-Portuguese pairs were randomly selected
and evaluated.

The precision of the extraction was $87.25\%$, with a recall of $58.15\%$ (again computed
using the whole set of monolingual collocations), and $69.79\%$ f-score. These results are
in line with those obtained in the OpenSubtitles Spanish-Portuguese pair ($\approx2\%$ lower),
so the method works well in different corpora and domains. It is worth noting that
$49,259$ of the extracted collocation equivalents ($83\%$) had not been retrieved from
the OpenSubtitles corpus.

This last experiment shows that (i) the bilingual word embeddings can be used for identifying
collocation equivalents in different corpora than those used for training, and that
(ii) they can also be applied in corpora of different domains to obtain previously unseen
multilingual collocations.

%
\section{Conclusions}
\label{garcia:sec:conclusions}
This chapter presents a new strategy to automatically discover multilingual collocation
equivalents from both parallel and comparable corpora. First, monolingual collocation candidates from five different patterns are
extracted using syntactic analysis provided by harmonized UD annotation,
together with a combination of standard association measures. Besides, bilingual word embeddings are trained in parallel corpora
that have previously been lemmatized. These bilingual models are then
used to find distributional equivalents of both the \textsc{base} and
the \textsc{collocate} of each source collocation in the target language.

The performed experiments, using noisy parallel corpora in three languages,
showed that the proposed method achieves an average precision %in the bilingual alignment of collocations
of about $85\%$, with reasonable recall values.
A systematic comparison to translation dictionaries pointed out that
the distributional approach achieves similar precision results with much
higher recall values than the probabilistic dictionaries.
Furthermore, the evaluation showed that setting up a confidence value as
a threshold is useful for retaining only high-quality bilingual
equivalents, which could benefit the work on multilingual lexicography.

Finally, preliminary tests using comparable corpora suggested that
the bilingual word embeddings can be efficiently applied in different
corpora than those used for training, discovering new bilingual collocations
not present in the original resources.

The multilingual resources generated by the proposed method can be used in several
scenarios in which MWEs play an important role, such as machine
translation or second language learning. In this respect, corpora from various registers
and linguistic varieties could be used in order to obtain a wider diversity of collocation
equivalents that can be useful for different purposes.

The work presented in this chapter enables us to propose a number of directions for
further work. First, the results of the error analysis should be taken into account
in order to reduce both the issues produced by the NLP pipeline, and those
which arise from the word embedding models. On the one hand, understanding collocations
as directional combinations may lead us to evaluate other association measures
which are not symmetrical (e.g., \emph{Delta-P}). On the other hand, it could be
interesting to evaluate other approaches for the alignment of bilingual collocations
which make use of better compositionality models, and which effectively learn the semantic
distribution of collocations as single units, in order to deal with cases of incongruent
collocation equivalents.


\begin{section}*{Abbreviations}
    \begin{tabularx}{.49\textwidth}{ll}
      \textsc{em} &   expectation maximization    \\
      \textsc{en} &   English    \\
      \textsc{lu}  &  lexical unit \\
      \textsc{mi} & mutual information    \\      
       \end{tabularx}
      \begin{tabularx}{.49\textwidth}{ll}      
      \textsc{mwe} &  multiword expression  \\            
      \textsc{nlp}  & natural language processing\\
      \textsc{pt} &  Portuguese    \\
      \textsc{es} &   Spanish    \\      
      \textsc{ud} &     universal dependencies  \\
    \end{tabularx}
\end{section}


\section*{Acknowledgements}
This work has been supported by the Spanish Ministry of Economy, Industry and
Competitiveness through the project with reference FFI2016-78299-P,
by two \emph{Juan de la Cierva} grants (FJCI-2014-22853 and IJCI-2016-29598),
and by a 2017 Leonardo Grant for Researchers and Cultural Creators, BBVA Foundation.

\printbibliography[heading=subbibliography,notkeyword=this]

\end{document}
