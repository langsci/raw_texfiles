\chapter{The nominal system}

Nouns are modified grammatically and pragmatically by means of pre- and postnominal elements. Common nouns are not inflected for number, case or gender in Pichi. In the personal pronoun paradigm, number and case are, however, morphologically marked. Generally, a noun phrase (henceforth NP) headed by a common noun has the structure given in \figref{fig:key:5.1}, which provides a (constructed) complex NP for exemplification.

\begin{figure}
\caption{Structure of the noun phrase} 
\label{fig:key:5.1}
\fittable{
\small 
\begin{tabular}{llllllllllllll}
\textsc{qnt} & \textsc{def/dem} & \textsc{pron} & \textsc{card} & \textsc{ord} & \textsc{mod} & \textsc{n} & \textsc{pl} & \textsc{adv} & \textsc{poss} & \textsc{qnt} & \textsc{foc} & \textsc{top} & \textsc{relc}\\
\textit{ɔ́l} & \textit{dí} & \textit{mi} & \textit{tú} & \textit{lás} & \textit{fáyn} & \textit{torí} & \textit{dɛn} & \textit{yá} & \textit{fɔ} \textit{tidé} & \textit{(ɔ́l)} & sɛ́f & \textit{náw} & \textit{wé}\\
all & this & my & two & last & \multicolumn{1}{c}{ nice} & \multicolumn{1}{c}{story} & \textsc{pl} & here & of today & (all) & self & now & that\\
\hhline{------~-------}
\multicolumn{6}{c}{ Prenominal} & Head & \multicolumn{7}{c}{ Postnominal}\\
\multicolumn{14}{c}{ ‘As for all these my two last nice stories here of today that (...)’}\\
\end{tabular}
}
\end{figure}
The possibilities for modifying nouns with determiners (\textsc{def} and \textsc{dem}) and quantifiers (\textsc{qnt}) depend on their lexical class. Pichi nouns fall into three lexical classes: count nouns (e.g.\textit{ hós} ‘house’) including collective nouns (e.g. \textit{pípul} ‘people’), mass nouns (e.g. \textit{watá} ‘water’), and proper nouns (e.g. place names, such as \textit{Panyá} ‘Spain’, as well as personal names like \textit{Tokobé).} 


The slot \textsc{def/dem} indicates that the definite article \textit{di} (\textsc{def}) and the proximal and distal demonstratives \textit{dí} and \textit{dán} (\textsc{dem}) do not cooccur. Possessive pronouns (\textsc{pron}) precede the head and may co-occur with demonstratives but not with the definite article. NP constituents in other slots featuring a single function label in \figref{fig:key:5.1} may coocur. \is{noun phrase structure}



There are two quantifier slots. The quantifiers \textit{ɔ́l} ‘all’ and \textit{dásɔl} ‘only’ (\textsc{qnt}) can be floated and may occur either in a pre- or post-head position (hence the postnominal \textit{ɔ́l} in brackets). The possessor in compounds, associative constructions, and dislocated possessive constructions is best seen to fill the modifier (\textsc{mod}) slot. Several modifiers can therefore co-occur (e.g. \textit{bíg blák kichin-písis} ‘big black kitchen rag’). The possessor in a \textit{fɔ}-prepositional construction follows the head, but its exact position in the postnominal slot may depend on pragmatic factors, e.g. either before or after \textit{sɛ́f} or \textit{náw} depending on the scope of \textsc{foc} or \textsc{top}. Relative clauses (\textsc{relc}) invariably follow the head noun. 


\section{Determination}\label{sec:5.1}

This section covers the distribution and functions of the definite article, indefinite determiners, demonstratives, and number marking. Quantifiers are treated separately in section \sectref{sec:5.3}.

\subsection{Definiteness and specificity}

Definiteness and specificity of nouns are marked by the prenominal definite article \textit{di} ‘\textsc{def}’ and the indefinite determiners \textit{wán} ‘one, a’ and \textit{sɔn} ‘some, a’. In addition, bare nouns without a preceding determiner are marked for definiteness and specificity by default. Some relevant characteristics of definiteness marking are presented in \tabref{tab:key:5.1}. The use of bare nouns is covered in more detail in \sectref{sec:5.1.4}.


%%please move \begin{table} just above \begin{tabular
\begin{table}
\caption{Characteristics of definiteness marking}
\label{tab:key:5.1}
\resizebox{\linewidth}{!}{
\begin{tabular}{lllll}
\lsptoprule
 & \textstyleTablePichiZchn{di} ‘\textsc{def}’ & \textstyleTablePichiZchn{wán} ‘one, a’ & \textstyleTablePichiZchn{sɔn} ‘some, a’ & Bare noun\\
\midrule
Definiteness & \textsc{def} & \textsc{indf} & \textsc{indf} & \textsc{indf}\\
Specificity & \textsc{spec} & \textsc{spec} & \textsc{spec/non-spec} & \textsc{non-spec}\\
Number & \textsc{sg/pl} & \textsc{sg} & \textsc{sg/pl} & \textsc{sg/pl}\\
Pronominal use & No & Yes & Yes & n.a\\
Used within negative scope? & Yes & Yes & No & Yes\\
\lspbottomrule
\end{tabular}
}
\end{table}
The definite article \textit{di} signals definiteness of a noun phrase. It is neutral as to number and can be used with count, mass, and proper nouns alike. \textit{Di} may precede NPs headed by full nouns (cf. \ref{ex:key:169} below), the numeral \textit{wán} ‘one’ in its function as a pronominal \REF{ex:key:165}, or any element functioning as a noun, such as the deverbal noun \textit{dú} in \REF{ex:key:166}\textsc{:} 


\ea%165
    \label{ex:key:165}
    \gll Di  láyf  fɔ́s  tɛ́n    e    bin  swít    pás    \textbf{di}  \textbf{wán}    tidé.\\
\textsc{def}  life  first  time    \textsc{3sg.sbj}  \textsc{pst}  be.sweet  pass    \textsc{def}  one    today\\

\glt ‘Life in the past was more enjoyable than that of today.’ [\textstylePichiexamplenumberZchnZchn{ab03ay 104]}
\z


\ea%166
    \label{ex:key:166}
    \gll Mék    e    bít    yú,    mék    e    dú  yú    \textbf{di}  \textbf{dú} e    wánt,  mék    e    hála,  \op...\cp\\
\textsc{sbjv}    \textsc{3sg.sbj}  beat    \textsc{2sg.indp}  \textsc{sbjv}    \textsc{3sg.sbj}  do  \textsc{2sg.indp}  \textsc{def}  do
\textsc{3sg.sbj}  want  \textsc{sbjv}    \textsc{3sg.sbj}  shout\\

\glt ‘Let him beat you, let him do to you what he wants to, let him shout (...)’ [\textstylePichiglossZchn{bo03cb 135]}
\z

Proper nouns, such as the place name \textit{Camerún} ‘Cameroon’ and personal names, do not usually co-occur with the article \REF{ex:key:167}, but may appear with it if required \REF{ex:key:168}: 


\ea%167
    \label{ex:key:167}
    \gll Porque  a    bin  pás  na  \textbf{Camerún}    fɔ́s.\\
because  \textsc{1sg.sbj}  \textsc{pst}  pass  \textsc{loc}  \textsc{place}    first\\

\glt ‘Because I passed through Cameroon first.’ [fr03ft 98]
\z


\ea%168
    \label{ex:key:168}
    \gll Na  \textbf{di}  \textbf{sén}    \textbf{Jorge}  wé    a    sabí    nɔ́?\\
\textsc{foc}\textstylePichiexamplenumberZchnZchn{} \textsc{def} \textstylePichiexamplenumberZchnZchn{  same} \textsc{name}\textstylePichiexamplenumberZchnZchn{} \textsc{sub}    \textsc{1sg.sbj} \textstylePichiexamplenumberZchnZchn{  know}  \textsc{intj}\\

\glt ‘It is the same Jorge that I know, right?’ [nn07fn 227]
\z

The definite article \textit{di} is employed in contexts in which a noun is specific, identifiable, and familiar to discourse participants either through its presence in the immediate physical surrounding (e.g. \textit{maíz} ‘maize’) \REF{ex:key:169}, or through situational inference (e.g. \textit{mɔ́nin mɔ́nin tɛ́n} ‘early in the morning’) \REF{ex:key:170}:


\ea%169
    \label{ex:key:169}
    \gll Yu  ték    \textbf{di}  \textbf{maíz}  yu  hól=an.\\
\textsc{2sg}  take    \textsc{def}  maize  \textsc{2sg}  hold=\textsc{3sg.obj}\\

\glt ‘You take the maize and hold it.’ [fr03do 003]
\z


\ea%170
    \label{ex:key:170}
    \gll \'{A}fta    a    de  mít=an    nía    di  klós    dɛn \textbf{di}  \textbf{mɔ́nin}  \textbf{mɔ́nin}  tɛ́n.\\
then  \textsc{1sg.sbj}  \textsc{ipfv}  meet=\textsc{3sg.obj}  near    \textsc{def}  clothing  \textsc{pl} 
 \textsc{def}  morning  \textsc{rep}    time\\

\glt ‘Then I met her near the clothes early in the morning.’ [ru03wt 037]
\z

The associative use of the article is exemplified in \REF{ex:key:171}. The referent \textit{leche} ‘milk’ has been established earlier on in discourse. The Spanish noun \textit{animal} ‘animal’ is therefore definite by association with the antecedent \textit{leche}:


\ea%171
    \label{ex:key:171}
    \gll Es  que,    e    fáyn    wé  yu  nó  sabí    sé 
\textbf{e}    kɔmɔ́t    fɔ  \textbf{di} \textbf{animal}.\\
it.is  that    \textsc{3sg.sbj}  be.fine  \textsc{sub}  \textsc{2sg}  \textsc{neg}  know  \textsc{quot} 
\textsc{3sg.sbj}  come.out  \textsc{prep}  \textsc{def}  animal\\

\glt ‘It’s that it is fine when you don’t know that it [the milk] has just come out of the animal.\textstylePichiglossZchn{’} [ed03sp 105]
\z

The anaphoric use of the article can be seen in the following examples. The referent \textit{mán} ‘man’ is introduced in \REF{ex:key:172a} by the speaker abbreviated as (hi) (cf. \tabref{tab:1:1.1} in \sectref{sec:1.7}) and taken up as a definite NP by speaker (bo) in (b). Note the presence of the Nigerian Pidgin form \textit{haws} ‘house’ instead of Pichi \textit{hós} in (b):\is{anaphora}


\ea%172
    \label{ex:key:172}
    \ea{
        \label{ex:key:172a}
    \gll Dɛn  kin  fíɛ  \textbf{dɛn}  \textbf{mán} dán  káyn  stáyl.\\
  \textsc{3pl}  \textsc{hab}  fear  \textsc{3pl}  man    that  kind    style\\

\glt   ‘They (usually) fear their husbands and the like.’ [hi03cb 131]
}

\ex{
    \label{ex:key:172b}
\gll    Yu  de  fíɛ  \textbf{di}  \textbf{mán} mék    e    nó  bít  yú    ɔ  mék   e    nó  drɛ́b    yú    fɔ  haws  ó.\\
  \textsc{2sg}  \textsc{ipfv}  fear  \textsc{def}  man    \textsc{sbjv}    \textsc{3sg.sbj}  \textsc{neg}  beat  \textsc{2sg.indp}  or  \textsc{sbjv} \textsc{3sg.sbj}  \textsc{neg}  drive  \textsc{2sg.indp}  \textsc{prep}  house  \textsc{sp}\\
  
\glt   ‘You fear your man lest he should beat you or drive you out of the house.’ [hi03cb 132]
}
\z
\z




Cataphoric use of the article – where the identity of the definite noun is established in following discourse – can be seen in the relative construction in \REF{ex:key:173}: 


\ea%173
    \label{ex:key:173}
    \gll Yu  nó  fít,  porque  yu  mamá  nó  go  hébul    pé  \textbf{ɔ́l}
\textbf{di}  \textbf{wók}    \textbf{wé}  dán  mán    dɔ́n  dú  fɔ  yú\\
\textsc{2sg}  \textsc{neg}  can  because  \textsc{2sg}  mother  \textsc{neg}  \textsc{pot}  be.capable  pay  all
\textsc{def}  work  \textsc{sub}  that  man    \textsc{prf}  do  \textsc{prep}  \textsc{2sg.indp}\\

\glt ‘You can’t because your mother wouldn’t be able to pay all that work 
that the man has done for you.’ [ab03ay 021]\is{article}
\z

Singular count nouns are marked for indefiniteness with the cardinal numeral\is{cardinal numerals} \textit{wán} ‘one’ \REF{ex:key:174}, or with the quantifier \textit{sɔn} ‘some, a’ (cf. \ref{ex:key:175} below). The numeral \textit{wán} is not a fully grammaticalised indefinite article. In many contexts, \textit{wán} retains its lexical meaning of ‘one’. \textit{Wán} also has pronominal functions and can itself be preceded by the demonstratives \textit{dí} and \textit{dán} and the definite article \textit{di} (e.g. \ref{ex:key:165}).


\ea%174
    \label{ex:key:174}
    \gll A    gɛ́t  \textbf{wán}  \textbf{bíg}  \textbf{sísta}  wé  na  mulata.\\
\textsc{1sg.sbj}  get  one  big  sister  \textsc{sub}  \textsc{foc}  African-European.\textsc{f}\\

\glt ‘I have a/one big sister who is African-European.’ [fr03ft 022]
\z

When used with count nouns, \textit{wán} usually signals a higher degree of specificity than \textit{sɔn}. However, there is no categorical distinction between specific and non-specific deixis in Pichi. This can be seen in the following two sentences. Here the noun \textit{fébɔ} ‘favour’ appears with \textit{sɔn} ‘some, a’ in \REF{ex:key:175} and \textit{wán} ‘one, a’ in a specific and emphatic setting in \REF{ex:key:176}: 


\ea%175
    \label{ex:key:175}
    \gll A    wánt  mék    yu  dú  mí    \textbf{sɔn} \textbf{    fébɔ},
mék    yu  wás    mi    sɔn    klós    dɛn.\\
\textsc{1sg.sbj}  want  \textsc{sbjv}    \textsc{2sg}  do  \textsc{1sg.indp}  some  favour
\textsc{sbjv}    \textsc{2sg}  wash  \textsc{1sg.indp}  some  clothing  \textsc{pl}\\

\glt ‘I want you to do me a favour (and) wash some clothes for me.’ [\textstylePichiexamplenumberZchnZchn{ru03wt 030]}
\z


\ea%176
    \label{ex:key:176}
    \gll Na  sé,    na  layk  sé    di  mán  de  mék    yú    \textbf{wán} \textbf{   fébɔ}.\\
\textsc{foc}  \textsc{quot}    \textsc{foc}  like  \textsc{quot}    \textsc{def}  man  \textsc{ipfv}  make  \textsc{2sg.indp}  one    favour\\

\glt ‘It is that, it is as if the man is doing you a favour.’ [hi03cb 180]
\z

Given that Pichi does not mark number on nouns morphologically, \textit{wán}, rather than \textit{sɔn}, is used to express that singular number is a significant feature of the referent as in \textit{wán motó} ‘one car’ \REF{ex:key:177}\is{cardinal numerals}. Here an interpretation of \textit{wán} as a numeral would appear awkward, since the speaker does not have more than one car in mind:


\ea%177
    \label{ex:key:177}
    \gll Yu  sabí    sé    \textbf{wán}    \textbf{motó}  fɔ  wán  mún    na  cincuenta  dólar,
ɛf  yu  hól    wán    motó  fɔ  wán  mún.\\
\textsc{2sg}  know  \textsc{quot}    one    car    \textsc{prep}  one  month  \textsc{foc}  fifty      dollar
if  \textsc{2sg}  hold    one    car    \textsc{prep}  one  month  \\


\glt ‘You know that a car for one month is fifty dollars, if you keep a car for only one month.’
[ed03sp 076]
\z

\textit{Wán} rather than \textit{sɔn} is also common in emphatic contexts. The data does not contain a single sentence in which a noun is preceded by \textit{sɔn} in an equative clause of the type in \REF{ex:key:178}, in which the identified entity is highly specific. The numeral \textit{wán} may also signal additional emphasis\is{emphasis} when it precedes a noun under cleft focus\is{cleft constructions} in a presentative construction, as in \REF{ex:key:179} (cf. also \sectref{sec:7.4.4}): 


\ea%178
    \label{ex:key:178}
    \gll \textbf{Na}  \textbf{wán}    \textbf{ɔnkúl}  \textbf{directo},  fɔ  mi    mamá  in    papá
in    fámbul  pát.\\
\textsc{foc}  one    uncle  direct  \textsc{prep}  \textsc{1sg.poss}  mother  \textsc{3sg.poss}  father
\textsc{3sg.poss}  family  part\\

\glt ‘(He) is a direct uncle on my mother’s father’s family’s side.’ [fr03ft 051]
\z


\ea%179
    \label{ex:key:179}
    \gll E    dé    complicado,  \textbf{na}  \textbf{wán}    \textbf{tín}    \textbf{dat}.\\
\textsc{3sg.sbj}  \textsc{be.loc}  complicated  \textsc{foc}  one    thing  that\\

\glt ‘It’s complicated, it’s one (kind of a) thing.’ [ye07de 017]
\z

Contrary to what one would expect of a cardinal numeral that signals singular number, \textit{wán} can also modify a noun containing a numeral above one \REF{ex:key:180}. Such usage of \textit{wán} is often found in conjunction with Spanish numerals and head nouns and is likely to be a case of structural borrowing\is{borrowing} from Spanish. In Spanish, the plural indefinite article (\textit{unos}/\textit{unas}) fulfills an identical function (cf. also \sectref{sec:13.3.1}): \is{article}


\ea%180
    \label{ex:key:180}
    \gll \'{A}fta    wi  kán  mít    layk    \textbf{wán}    \textbf{seis}    \textbf{años}  después.\\
then  \textsc{1pl}  real  meet  like    one    six    year.\textsc{pl}  afterwards\\

\glt ‘Then we met again some six years later.’ [fr03ft 191]
\z

With plural count nouns, indefiniteness is signalled through the presence of \textit{sɔn} alone \REF{ex:key:181} or the absence of a definiteness expression altogether (cf. \sectref{sec:5.1.4}). Mass nouns may only be modified by \textit{sɔn} for indefiniteness, or they occur devoid of any determiner \REF{ex:key:182}: 


\ea%181
    \label{ex:key:181}
    \gll Wi  gɛ́t  \textbf{sɔn}    \textbf{fámbul}  dé,    na  dán  yu,  na  yu  prima.\\
\textsc{1pl}  get  some  family  there  \textsc{foc}  that  \textsc{2sg}  \textsc{foc}  \textsc{2sg}  cousin.\textsc{f}\\

\glt ‘We have a family member there, it’s your, it’s your female cousin.’ [ge07ga 048]
\z


\ea%182
    \label{ex:key:182}
    \gll \textbf{Blɔ́d}  de  kɔmɔ́t    na  in    nós,    e    de  kɔmɔ́t
na  in    mɔ́t.\\
blood  \textsc{ipfv}  come.out  \textsc{loc}  \textsc{3sg.poss}  nose  \textsc{3sg.sbj}  \textsc{ipfv}  come.out
\textsc{loc}  \textsc{3sg.poss}  mouth\\

\glt ‘Blood was coming out of her nose, it was coming out of her mouth.’\textstylePichiexamplenumberZchnZchn{ [ab03ay 125]}
\z

Furthermore, \textit{wán}, but not \textit{sɔn}, may occur with NPs that are within the scope of negation, even if only with an emphatic meaning \REF{ex:key:183}. In the absence of emphasis, NPs do not usually appear with a marker of indefiniteness in negative clauses \REF{ex:key:184} (cf. \sectref{sec:7.2.2} for details): 


\ea%183
    \label{ex:key:183}
    \gll Sóté    a    \textbf{nó}  tɔ́k  \textbf{nó}  \textbf{wán}    \textbf{wɔ́d}.\\
until  \textsc{1sg.sbj}  \textsc{neg}  talk  \textsc{neg}  one    word\\

\glt ‘Until I didn’t say a single word (anymore).’ [ab03ay 088]
\z


\ea%184
    \label{ex:key:184}
    \gll Yu  sabí    sé    yu  \textbf{nó}  gɛ́t  \textbf{pikín}?\\
\textsc{2sg}  know  \textsc{quot}    \textsc{2sg}  \textsc{neg}  get  child\\

\glt ‘Do you (really) know that you don’t have a child?’ [fr03wt 181]
\z

Both \textit{wán} and \textit{sɔn} can function as pronominals and refer anaphorically to a preceding indefinite NP. While \textit{wán} is limited to anaphoric reference of a singular count noun, \textit{sɔn} may be used to refer to preceding singular or plural count and mass nouns. 


In both \REF{ex:key:185} and \REF{ex:key:186}, \textit{wán} and \textit{sɔn} refer to a preceding NP \textit{televisión} ‘TV set’. When referring to a plural noun, \textit{sɔn} may optionally be followed by the pluraliser \textit{dɛn} ‘\textsc{pl}’ \REF{ex:key:187}: \is{anaphora}



\ea%185
    \label{ex:key:185}
    \gll Yɛ́s,    a    gɛ́t  \textbf{wán}.\\
yes    \textsc{1sg.sbj}  get  one\\

\glt ‘Yes, I have one [a TV set].’ [dj05ae 078]
\z


\ea%186
    \label{ex:key:186}
    \gll Na  só    mi    yón     sɛ́f,  a    jɔ́s  báy  \textbf{sɔn}.\\
\textsc{foc}  \textstylePichiexamplenumberZchnZchn{like.that}  \textsc{1sg.poss}  \textstylePichiexamplenumberZchnZchn{own} \textsc{emp} \textsc{1sg.sbj} \textstylePichiexamplenumberZchnZchn{just} \textstylePichiexamplenumberZchnZchn{buy} \textstylePichiexamplenumberZchnZchn{some}\\

\glt ‘That’s how it is with me as well, I just bought one [a TV set].’ [ma0305hm 072]
\z


\ea%187
    \label{ex:key:187}
    \gll A    gɛ́t  \textbf{sɔn}    \textbf{dɛn}.\\
\textsc{1sg.sbj}  get  some  \textsc{pl}\\

\glt ‘I have some (\textsc{pl}).’ [ro05fe 002]
\z

\textit{Sɔn} and \textit{wán} may also be used with a partitive\is{partitive} reading when followed by a definite possessed noun. Once more the nominal referent preceded by \textit{sɔn} tends to receive a less specific reading than the one featuring \textit{wán}. The same meaning may alternatively be expressed if \textit{sɔn} or \textit{wán} are followed by a definite \textit{fɔ}-prepositional phrase (cf. e.g. \ref{ex:key:247}):


\ea%188
    \label{ex:key:188}
    \gll \textbf{Sɔn}   in    sísta    \op...\cp\\
some  \textsc{3sg.poss}  sister\\

\glt ‘A sister of hers (...)’ [ab03ay 058]
\z


\ea%189
    \label{ex:key:189}
    \gll A    sé,    \textbf{wán}    mi    kɔ́mpin  nɔ́,  \op...\cp\\
\textsc{1sg.sbj}  \textsc{quot}    one    \textsc{1sg.poss}  friend  \textsc{intj}  \\

\glt ‘I say one of my friends, right, (...)’ [ye07ga 001]
\z

Finally, only the quantifier and indefinite, non-specific determiner \textsc{sɔn} appears in NPs which function as nominal and adverbial indefinite pronouns and involve generic noun{\fff}s like \textit{tín} ‘thing’, \textit{pɔ́sin} ‘person’, \textit{tɛ́n} ‘time’, \textit{sáy} ‘side’, \textit{plés} ‘place’, \textit{áwa} ‘hour, time’, and \textit{stáyl} ‘style’. Compare the following two examples (cf. \sectref{sec:5.4.3} for a complete listing):{\fff}


\ea%190
    \label{ex:key:190}
    \gll \textbf{Sɔn} \textbf{   áwa}    a    nó  kin  hébul    mɔ́,    mi
sísta    dɛn  kin  sɛ́n    mi    mɔní.\\
some  hour  \textsc{1sg.sbj}  \textsc{neg}  \textsc{hab}  be.capable  more  \textsc{1sg.poss}
sister  \textsc{pl}  \textsc{hab}  send  \textsc{1sg.indp}  money\\

\glt ‘Sometimes I wouldn’t cope any more, (so) my sisters would send me money.’ [ed03sp 087]
\z


\ea%191
    \label{ex:key:191}
    \gll Wán    dé  \textbf{sɔn} \textbf{   pɔ́sin}  bin  kán    sé,    e    de
tɔ́k  sé    yu  dɔ́n  gí  wán    golpe  \textbf{sɔn} \textbf{    sáy}.\\
one    day  some  person  \textsc{pst}  come  \textsc{quot}    \textsc{3sg.sbj}  \textsc{ipfv}
talk  \textsc{quot}    \textsc{2sg}  \textsc{prf}  give  one    blow  some  side.\\

\glt ‘One day somebody came that, he was saying that you had given a blow
somewhere [you had fathered a child somewhere].’ [fr03wt 185]\is{definiteness}
\z

\subsection{Demonstratives}

Pichi has a two-term demonstrative system that serves to express the notions of proximity and distance with the speaker as the deictic centre. The demonstratives \textit{dí/dís} ‘this’ and \textit{dá/dán/dat} ‘that’ and sometimes \textit{dɛn} ‘those’ express the spatial, temporal, and discourse functions of proximal and distal reference respectively. \tabref{tab:key:5.2} gives an overview of the forms and functions of Pichi demonstratives.

%%please move \begin{table} just above \begin{tabular
\begin{table}
\caption{Demonstratives}
\label{tab:key:5.2}

%\fittable{
\begin{tabular}{lllll}
\lsptoprule
Deixis type & Attributive & Pronominal & Presentative & Deictic adverbial\\
\midrule
Proximal & \itshape dí/dís & \itshape dí/dís wán; dís & \itshape dís & \itshape yá\\
Distal & \itshape dá/dán & \itshape dá/dán wán; dat & \itshape dat & \itshape dé\\
& \itshape dɛn & \itshape — & \itshape {}--- & not attested\\
\lspbottomrule
\end{tabular}
%}
\end{table}

It is unclear whether \textit{dí} and \textit{dá} are distinct realisations or phonological variants with a deleted final consonant of the forms \textit{dís} and \textit{dán/dát}. The differentiation between \textit{dán} and \textit{dat} suggests that the “short” and the “long” forms may be distinct developments from their respective English etymons (< ‘this/that’). Likewise, the use of either form as attributive demonstratives could not be correlated to any (socio-)linguistic conditioning factor. 


In contrast, it is very likely that \textit{di} ‘\textsc{def}’ is a reflex of English \textit{the}, while \textit{dí} ‘this’ is a reflex of the English proximal demonstrative \textit{this}. The evidence is prosodic. Pichi \textit{di} ‘\textsc{def}’ was lexicalised as L-toned because English \textit{the} is usually unstressed, while \textit{dí} ‘this’ received a lexical H because \textit{this} is usually stressed in English. 



Demonstratives may be used attributively as prenominal modifiers. The forms \textit{dí} and \textit{dís} are equivalent in function, although \textit{dí} is more common as a proximal demonstrative \REF{ex:key:192}:



\ea%192
    \label{ex:key:192}
    \gll Djunais  tɔ́k  sé,    nɔ́  Rubi    \textbf{dí}  \textbf{gɛ́l}  lɛ́k  yú.\\
\textsc{name}  talk  \textsc{quot}    \textsc{intj}  \textsc{name}  this  girl  like  \textsc{2sg.indp}\\

\glt ‘Djunais said, really Rubi, this girl likes you.’ [ru03wt 021]
\z

The two forms \textit{dá} and \textit{dán} serve as distal attributive demonstratives \REF{ex:key:193}. The form \textit{dán} is used in the majority of cases, irrespective of the word-initial onset of the following noun. NPs featuring an attributively used demonstrative are pluralised in the usual way by means of the postposed pluraliser\is{pluraliser} \textit{dɛn} \REF{ex:key:193}:


\ea%193
    \label{ex:key:193}
    \gll Ɔ́l  \textbf{dán}  pikín  \textbf{dɛn}  na  dán  mán    in    yón.\\
all  that  child  \textsc{pl}  \textsc{foc}  that  man    \textsc{3sg.poss}  own\\

\glt ‘All those children are that man’s.’ [hi03cb 190]
\z

\textit{Dí} and \textit{dís} \REF{ex:key:194}, as well as \textit{dá} and \textit{dán} (cf. e.g. \ref{ex:key:202}) may combine with the numeral and pronominal \textit{wán} ‘one’, in order to form singular \REF{ex:key:194} and plural \REF{ex:key:195} demonstrative pronominals:


\ea%194
    \label{ex:key:194}
    \gll A    tínk    sé    \textbf{dí}  \textbf{wán}  na  wán  problema  fɔ  Afrika,  ɛ́n.\\
\textsc{1sg.sbj}  think  \textsc{quot}    this  one  \textsc{foc}  one  problem    \textsc{prep}  \textsc{place}  \textsc{sp}\\

\glt ‘I think that this is a problem in Africa.’ [fr03ft 105]
\z


\ea%195
    \label{ex:key:195}
    \gll Na  dé    \textbf{dís}  \textbf{wán}    \textbf{dɛn}  mamá  dɛn  de  mék    ɛ́ni    tín.\\
\textsc{foc}  there  this  one    \textsc{3pl}  mother  \textsc{3pl}  \textsc{ipfv}  make  every  thing\\

\glt ‘It is then that these ones’ mothers do every thing.’\textstylePichiglossZchn{ [ab03ay 047}]
\z

The forms \textit{dís} and \textit{dat} may be employed as independent pronominals on their own, although this use is marginal compared to that involving the pronominal \textit{wán}:


\ea%196
    \label{ex:key:196}
    \gll \textbf{Dís}  nóto  Manolete.\\
this  \textsc{neg}.\textsc{foc}  \textsc{name}\\
\glt ‘This is not Manolete (oil).’ [ab03ab 029]\\
\z

\ea%197
    \label{ex:key:197}
    \gll \textbf{Dát} nó  go  dú  ó!\\
that  \textsc{neg}  \textsc{pot}  do  \textsc{sp}\\

\glt ‘That really won’t do.’ [nn07fn 216]
\z

\textit{Dís} and \textit{dát}, but never \textit{dí} and \textit{dá/dán}, also occur in sentence-final position in a presentative construction of the type presented in \REF{ex:key:198} and \REF{ex:key:199}, where the demonstratives are anaphoric to an antecedent focused NP (cf. \sectref{sec:7.4.4})\textsc{:} \is{anaphora}


\ea%198
    \label{ex:key:198}
    \gll Sé    \textbf{na} ín    \textbf{dís},  \textbf{na} yu  húman  \textbf{dís},  yu  wánt
ɔ  yu  nó  wánt,  \textbf{na} in    \textbf{dís}. \\
\textsc{quot}    \textsc{foc}  \textsc{3sg.indp}  this  \textsc{foc}  \textsc{2sg}  woman  this  \textsc{2sg}  want
or  \textsc{2sg}  \textsc{neg}  want  \textsc{foc}  \textsc{3sg.poss}  this \\

\glt ‘(She said) this is her, this is your wife, you like it or not, 
this is her.’ [ed03sp 009]
\z


\ea%199
    \label{ex:key:199}
    \gll \textbf{Na} in    vida    \textbf{dát}.\\
\textsc{foc}\textstylePichiexamplenumberZchnZchn{} \textsc{3sg.poss} \textstylePichiexamplenumberZchnZchn{  life}    \textstylePichiexamplenumberZchnZchn{that}\\

\glt ‘That’s his (kind of) life.’ [he07fn 228\textstylePichiexamplenumberZchnZchn{]}
\z

Demonstrative adjectives do not co-occur with the definite article. They may, however, precede proper nouns \REF{ex:key:200} and possessive pronouns \REF{ex:key:201}:\is{article}


\ea%200
    \label{ex:key:200}
    \gll Lúk=an,    di  dé  wé  \textbf{dís} \textbf{ Paquita}  in    papá
bin  kán    ték=an,    e    pé  avioneta.\\
look=\textsc{3sg.obj}  \textsc{def}  day  \textsc{sub}  this  \textsc{name}  \textsc{3sg.poss}  father 
\textsc{pst}  come  take=\textsc{3sg.obj}  \textsc{3sg.sbj}  pay  small.aircraft\\

\glt ‘Look at this, the day that Paquita’s father came to take her, 
he hired a small aircraft.’ [ab03ay 140]
\z


\ea%201
    \label{ex:key:201}
    \gll Cuñado,      mí    gɛ́fɔ    fɛ́n    \textbf{dán}  \textbf{mi}
prima  ó,  Cristina.\\
brother-in-law  \textsc{1sg.indp}  have.to  look.for  that  \textsc{1sg.poss} 
cousin.\textsc{f}  \textsc{sp}  \textsc{name}\\

\glt ‘Brother(-in-law), I [\textsc{emp}] really have to look for that my 
(female) cousin, Cristina.’ [ge07ga 046]
\z

Demonstratives are often reinforced through the deictic locative adverbs\is{locative adverbials} \textit{yá} ‘here’, \textit{dé} ‘there’, and sometimes \textit{yandá} ‘yonder, over there’ \REF{ex:key:202}:


\ea%202
    \label{ex:key:202}
    \gll Ɛhɛ́,    wán  glás    watá  aparte,  yu  pút=an    ínsay,  \textbf{dán}  \textbf{wán}
\textbf{dé,}    yu  fít  ték    medio  fɔ  dán    sén    glas  \op...\cp\\
\textsc{intj}    one  glass  water  separate  \textsc{2sg}  put=\textsc{3sg.obj}  inside  that  one
there  \textsc{2sg}  can  take    half    \textsc{prep}  that    same  glass  \\

\glt ‘Exactly, one glass of water separately, you put it inside, as for that one, you can 
take half in that very glass (...)’ [\textstylePichiexamplenumberZchnZchn{dj03do 054]}
\z

The idiom \textit{dís-tín} ‘this-thing’ may substitute for an inanimate noun. Example \REF{ex:key:203} shows that this expression has been lexicalised to an extent which allows the occurrence of the demonstrative \textit{dán} ‘that’ with its full referential meaning: 


\ea%203
    \label{ex:key:203}
    \gll A    ték    tú  peso  a    báy  \textbf{dán}    \textbf{dís-tín}  \op...\cp\\
\textsc{1sg.sbj}  take    two  peso  \textsc{1sg.sbj}  buy  that    this-thing\\

\glt ‘I took two pesos (and) I bought this whatsit (...)’ [ed03sp 083]
\z

The \textsc{3pl} dependent personal pronoun and pluraliser \textit{dɛn} occasionally occurs in the determiner position at the very left of the NP. In this position, \textit{dɛn} simultaneously functions as a plural definite article and a demonstrative with a largely discourse deictic function. Prenominal \textit{dɛn} usually also has emphatic force. This use of \textit{dɛn} however is marginal in the corpus. Note the additional presence of \textit{dɛn} as a pluraliser\is{pluraliser} after the noun \textit{fronteras} ‘borders’ \REF{ex:key:204}: \is{article}  

\ea%204
\label{ex:key:204}
\gll Wet    ɔ́l  \textbf{dɛn}    \textbf{fronteras}  \textbf{dɛn}  wé  dɛn  de  chénch.\\
with    all  those  borders    \textsc{pl}  \textsc{sub}  \textsc{3pl}  \textsc{ipfv}  change\\

\glt ‘With all those borders that are changing.’ [fr03ft 102]
\z

In their function as markers of spatial deixis, the proximal and distal demonstratives serve to locate referents in physical space with the speaker as the deictic centre \REF{ex:key:205}:


\ea%205
    \label{ex:key:205}
    \gll Wi  de  gó  dɔ́n,    wi  de  gó  lɛ́f=an    di  sáy
\textbf{dán}    \textbf{motó}\textstylePichiexamplespaceZchn{} \textbf{dé}.\\
\textsc{1pl}  \textsc{ipfv}  go  down  \textsc{1pl}  \textsc{ipfv}  go  leave=\textsc{3sg.obj}  \textsc{def}  side
that    car    \textsc{be.loc}\\

\glt ‘We’re going down, we are going to leave it where that car is.’ [ma03ni 043]
\z

The demonstrative pronouns also serve to express discourse-pragmatic deixis. I reiterate example \REF{ex:key:202} above in \REF{ex:key:206} below in context. In the excerpt, speaker (dj) explains how to cook corn porridge. The interjection \textit{ɛhɛ́} ‘exactly’ confirms the interruptive question posed in \REF{ex:key:206a}. The topical \textit{dán wán dé} in (b) is therefore anaphoric to the process explained just beforehand in the same sentence. 


The anaphoric function of the distal demonstrative pronoun is frequently made use of in order to refer to preceding NPs, phrases, and entire sentences. \textit{Dán sén glás} ‘that very glass’ represents in \REF{ex:key:202} an additional means of referent tracking via the use of the focus and emphasis\is{emphasis} marker \textit{sén} ‘same, very’: \is{anaphora}



\ea%206
    \label{ex:key:206}
    \ea{
    \label{ex:key:206a}
    \gll Wán    glás    watá?\\
  one    glass  water\\

\glt   ‘One glass of water?’ [fr03do 053]
}
\ex{
\gll Ɛhɛ́,    wán    glas    watá  aparte,  yu  pút=an    ínsay,  \textbf{dán}  \textbf{wán}
  \textbf{dé}    yu  fít  ték    medio  fɔ  dán  sén    glas  \op...\cp\\
  exactly  one    glass  water  separate  \textsc{2sg}  put=\textsc{3sg.obj}  inside  that  one
  there  \textsc{2sg}  can  take    half    \textsc{prep}  that  same  glass  \\

\glt 
  ‘Exactly, one glass of water separately, you put it inside, that one [that method],
   you can take half in that very glass (...)’ [dj03do 054]\is{demonstratives}
}
\z
\z

\subsection{Number}

Pichi marks plural number via the postposed pluraliser\is{pluraliser} \textit{dɛn} which is identical to the \textsc{3pl} dependent pronoun. The pluraliser is clitic-like in one respect: It may not be separated from the noun it refers to by any constituent. Typically, the pluraliser occurs with count nouns \REF{ex:key:207}, but it may also follow collective nouns like \textit{pípul} ‘people’ \REF{ex:key:208}: \is{cliticisation}


\ea%207
    \label{ex:key:207}
    \gll Yu  nó  fít  jɔ́s  trowé    di  \textbf{tín}  \textbf{dɛn}  na  strít    só.\\
\textsc{2sg}  \textsc{neg}  can  just  throw.away  \textsc{def}  thing  \textsc{pl}  \textsc{loc}  street  like.that\\

\glt ‘You can’t just throw the things into the street like that.’ [hi03cb 031]
\z


\ea%208
    \label{ex:key:208}
    \gll Fɔ  \textbf{pípul}  \textbf{dɛn},    \textbf{pípul}  \textbf{dɛn}  kin  dé    na  ród,    plɛ́nte.\\
\textsc{prep}  people  \textsc{pl}    people  \textsc{pl}  \textsc{hab}  \textsc{be.loc}  \textsc{loc}  road    plenty\\

\glt ‘Because of people, people are usually on the road, a lot.’ [ma03ni 011]
\z

The pluraliser is also encountered with mass nouns denoting liquids such as \textit{watá} ‘water’ \REF{ex:key:209} or \textit{leche} ‘milk’ in \REF{ex:key:210}:


\ea%209
    \label{ex:key:209}
    \gll Fít  sifta    ín    sóté    tú  tɛ́n    mék    mék  
dán  smɔ́l  smɔ́l  \textbf{watá}  \textbf{dɛn}  nó  lɛ́f.\\
can  sieve  \textsc{3sg.indp}  until  two  time    make  \textsc{sbjv}
that  small  \textsc{rep}    water  \textsc{pl}  \textsc{neg}  leave\\

\glt ‘(You) can sieve it up to two times in order not to make that 
little bit of water remain.’ [dj03do 008]
\z


\ea%210
    \label{ex:key:210}
    \gll A    bin  de  vɔ́mit  dán  \textbf{leche}  \textbf{dɛn}  fɔ́s  fɔ́s  tɛ́n    dɛn.\\
\textsc{1sg.sbj}  \textsc{pst}  \textsc{ipfv}  vomit  that  milk    \textsc{pl}  first \textsc{rep}  time    \textsc{pl}\\
\glt ‘I was throwing up that milk during the first few times.’\textstylePichiglossZchn{ [ed03sp 104]}
\z

NPs featuring a cardinal numeral \is{cardinal numerals}can also optionally be marked for plural number \REF{ex:key:211}, although in the majority of instances, speakers prefer not to use the pluraliser together with a numeral \REF{ex:key:212}: 


\ea%211
    \label{ex:key:211}
    \gll E    gɛ́t  \textbf{tú}  \textbf{pikín}  \textbf{dɛn}  na  Panyá  sɛ́f.\\
\textsc{3sg.sbj}  get  two  child  \textsc{pl}  \textsc{loc}  Spain  \textsc{emp}\\

\glt ‘She even has two children in Spain.’ [fr03ft 140]
\z


\ea%212
    \label{ex:key:212}
    \gll E    bríng  \textbf{trí}    \textbf{kasára},  e    lé  dɛ́n    pantáp  di  tébul.\\
\textsc{3sg.sbj}  bring  three  cassava  \textsc{3sg.sbj}  lie  \textsc{3pl.indp}  on    \textsc{def}  table\\

\glt ‘He brought three cassavas and put them on the table.’ [li07pe 067]
\z

Furthermore, the pluraliser may co-occur with quantifiers that indicate plurality of the referent such as \textit{ɔ́l} ‘all’ \REF{ex:key:213}, and \textit{bɔkú} ‘many, much’ \REF{ex:key:214}, although the absence of plural marking is equally common \REF{ex:key:215}: 


\ea%213
    \label{ex:key:213}
    \gll Yu  wánt  báy  cuaderno,    bolí  \textbf{ɔ́l}  \textbf{dán}  \textbf{tín}    \textbf{dɛn}
na  wet    dólar.\\
\textsc{2sg}  want  buy  exercise.book    pen  all  that  thing  \textsc{pl}
\textsc{foc}  with    dollar\\

\glt ‘You want to buy an exercise book, pen and all those things, it’s with dollars.’ [ed03sp 096] 
\z


\ea%214
    \label{ex:key:214}
    \gll \textbf{Bɔkú}  \textbf{ motó} \textbf{  dɛn} dé    yá    só,    \op...\cp\\
much  car    \textsc{pl}  \textsc{be.loc}  here    like.that\\

\glt ‘(Since) there were many cars around, (...)’ [ye03cd 178]
\z


\ea%215
    \label{ex:key:215}
    \gll Mí,    lɛk  háw  yu  de  sí  mí,    a    dɔ́n  
sí  \textbf{plɛ́nte}  \textbf{tín}.\\
\textsc{1sg.indp}  like  how  \textsc{2sg}  \textsc{ipfv}  see  \textsc{1sg.indp}  \textsc{1sg.sbj}  \textsc{prf}  
see  plenty  thing \\

\glt ‘As for me, as you see me (now), I’ve seen many things (in life).’ [ab03ab 023]
\z

The pluraliser is also consistently made use of with inserted Spanish nouns marked with the Spanish plural morpheme \{-s\} \REF{ex:key:216}. The same is true of the few instances in the corpus, in which the nouns \textit{bɔ́y} ‘boy’ and \textit{gál} ‘girl’ are marked for plural with the marginal Pichi plural morpheme \{-s\} as in \REF{ex:key:217}:


\ea%216
    \label{ex:key:216}
    \gll \'{A}fta    dɛ́n    na  mi    \textbf{sobrinos}    \textbf{dɛn}.\\
then  \textsc{3pl.indp}  \textsc{foc}  \textsc{1sg.poss}  nephew.\textsc{pl}  \textsc{pl}\\

\glt ‘So, they are my nephews.’ [fr03ft 060]
\z


\ea%217
    \label{ex:key:217}
    \gll Ɔ́l  Ghána  \textbf{bɔ́y-s}  \textbf{dɛn},    wé  dɛn  dé  \op...\cp\\
all  Ghana  boy-\textsc{pl}  \textsc{pl}    \textsc{sub}  \textsc{3pl}  \textsc{be.loc}\\

\glt ‘All the Ghanaian guys that were around (...)’ [ed03sp 076]
\z

Personal names may be pluralised in order to form an associative plural \REF{ex:key:218}. The resulting meaning is ‘X and those associated with her/him habitually or at the time of reference’:


\ea%218
    \label{ex:key:218}
    \gll \MakeUppercase{A}   dɔ́n    explica  \textbf{Boyé}  \textbf{dɛn},    sé    na  só
mi    de  mɛ́mba,    ɔ́l  tín.\\
\textsc{1sg.sbj}  \textsc{prf}    explain  \textsc{name}  \textsc{pl}    \textsc{quot}  \textsc{foc}  like.that  
\textsc{1sg.indp}  \textsc{ipfv}  remember  all  thing\\

\glt ‘I have explained to Boyé and the others that this is how I remember everything.’ [ru03wt 045]
\z

Plural number need not be marked on the head noun of a relative clause\is{relative clauses} and may instead be expressed via the coreferential subject pronoun in the relative clause:


\ea%219
    \label{ex:key:219}
    \gll \textbf{Di}  \textbf{húman}  wé  \textbf{dɛn}  fáyn    mɔ́    na  {América Latina}
húman  dɛn.\\
\textsc{def}  woman  \textsc{sub}  \textsc{3pl}  fine    more  \textsc{loc}  \textsc{place}
woman  \textsc{pl}\\

\glt ‘The women who are the most beautiful are Latin American women.’ [ed03sp 025]
\z

Syntactic factors may also constrain plural marking. One of the instances in which plurality is not overtly expressed and left to inferral is in dislocated possessive constructions.


I repeat sentence \REF{ex:key:195} in \REF{ex:key:220} below. As is generally the case in dislocated possessive constructions, a personal pronoun coreferential with the possessor (\textit{dɛn} ‘\textsc{3pl}’) links the plural possessor (\textit{dís wán dɛn} ‘these ones’) and the possessed noun (\textit{mamá} ‘mother’). I interpret the linker \textit{dɛn} in these cases as the \textsc{3pl} pronoun rather than the pluraliser, since singular possessors require the use of the corresponding singular possessive pronoun \textit{in} ‘\textsc{3sg.poss}’ in the same position. Hence the pluraliser remains unexpressed in the construction in order to avoid doubling of the two homophonous forms:



\ea%220
    \label{ex:key:220}
    \gll Na  dé    dís  wán    dɛn    mamá  \textbf{dɛn}  de  mék    ɛ́ni    tín.\\
\textsc{foc}  there  this  one    \textsc{3pl}    mother  \textsc{3pl}  \textsc{ipfv}  make  every  thing\\

\glt ‘It is then that these ones’ mothers do every thing.’\textstylePichiglossZchn{ [ab03ay 047}]
\z

In \REF{ex:key:221}, we encounter a similar overlap of \textsc{pl} and \textsc{3pl}. Here, \textit{dɛn} may be interpreted as the pluraliser postposed to the NP or instead, as a resumptive pronoun\is{resumptive pronouns} and the subject of the following verb. In contexts such as these, where a predicate immediately follows a plural-referring \textsc{NP}, the distinction between the pluraliser and a \textsc{3pl} resumptive pronoun is not possible, since doubling of the form is normally avoided. The distributional characteristics of \textit{dɛn} in these contexts indicate the significant functional overlap of \textsc{NP} and verbal number marking in Pichi:


\ea%221
    \label{ex:key:221}
    \gll Estudiante  fɔ  Guinea  \textbf{dɛn}    de  sɔ́fa    plɛ́nte.\\
student    \textsc{prep}  \textsc{place}  \textsc{3pl/pl}  \textsc{ipfv}  suffer  plenty\\

\glt ‘Guinean students were suffering a lot.’ [ed03sp 086]
\z

Finally, I point out that Pichi has at least two nouns with suppletive plural forms which are occasionally employed instead of the regular plural involving \textit{dɛn} ‘\textsc{pl’}. The relevant singular-plural pairs are \textit{gál-gáls} ‘girl-girls’ and \textit{bɔ́y-bɔ́ys} ‘boy-boys’. However, these forms are not suppletive in the true sense, since they feature the segmentable but only marginally productive plural morpheme \{-s\}, which is only attested with these two nouns. As example \REF{ex:key:217} above shows, these forms may also be followed by the pluraliser \textit{dɛn}.\is{number!nouns} 

\subsection{Genericity}\label{sec:5.1.4}

Generic reference of an NP can be established through the use of bare nouns with or without plural marking, as well as the use of the definite article \textit{di} ‘\textsc{def’}. A noun phrase may consist of only a bare noun. The demarcation between count and mass nouns is blurred when they are used as “non-individuated” \citep{Mufwene1986} nouns in this way, since the number distinction is now irrelevant for both entity types. 


Generalisations may be made about a whole class of referents by using the bare form of the corresponding count noun in generic statements like the following ones: 



\ea%222
    \label{ex:key:222}
    \gll Na  \textbf{mán}  in    suerte.\\
\textsc{foc}  man    \textsc{3sg.poss}  luck\\

\glt ‘That’s the fortune of men.’ [\textstylePichiexamplenumberZchnZchn{fr03ft 194]}
\z


\ea%223
    \label{ex:key:223}
    \gll \textbf{Dɔ́g}    kin  bɛ́t.\\
dog    \textsc{hab}  bite\\

\glt ‘Dogs bite.’ [dj07ae 371]
\z

In contrast, the use of the bare form is the normal way of referring to indefinite and non-specific mass nouns like \textit{chɔ́p} ‘food’ and \textit{pamáyn} ‘oil’, while definite (and specific by default) mass nouns are preceded by the definite article \textit{di} ‘\textsc{def’} like count nouns:\is{article}

\ea%224
 \label{ex:key:224}
 \gll  \textbf{Chɔ́p}  dé    na  hós,    \textbf{pamáyn}   dé     \op...\cp\\
food    \textsc{be.loc}  \textsc{loc}  house  oil    \textsc{be.loc}\\

\glt ‘There’s food in the house, there’s oil (...)’ [ro05rt 050]
\z


\ea%225
    \label{ex:key:225}
    \gll Yu  fɔ  trowé  di  \textbf{watá}  yá    só.\\
\textsc{2sg}  \textsc{prep}  pour  \textsc{def}  water  here    like.that\\

\glt ‘You have to pour (out) the water here.’ [dj03do 039]
\z

In Pichi, weather mass nouns like \textit{brís} ‘wind’, \textit{tináda} ‘thunderstorm’, and \textit{rén} ‘rain’ also have non-specific NP marking and reference when they occur in weather condition clauses like the following one:


\ea%226
    \label{ex:key:226}
    \gll \textbf{Brís}  de  bló.\\
air  \textsc{ipfv}  blow\\

\glt ‘The wind is blowing.’ [dj07ae 242]
\z

However, with count nouns, generic reference can also be established by employing a plural noun without a determiner \REF{ex:key:227}:\is{number!nouns}


\ea%227
    \label{ex:key:227}
    \gll \textbf{Mán}  \textbf{dɛn}  nó  de  bísin  fɔ  mék    fám    mɔ́.\\
man    \textsc{pl}  \textsc{neg}  \textsc{ipfv}  be.busy  \textsc{prep}  make  farm  more\\

\glt ‘People are no more into farming.’ [ed03sp 053]
\z

Further, the reference of the definite article \textit{di} ‘\textsc{def’} may also be construed as generic if it co-occurs with generic TMA marking. In this example, imperfective marking expresses a habitual\is{habitual aspect}, generic sense, and the nouns \textit{gabonés} and \textit{guineano} designate the whole class of referents rather than specific ones: \is{article}


\ea%228
    \label{ex:key:228}
    \gll Pero    \textbf{di}  \textbf{gabonés}    wé  de  tɔ́k  Bata    wet    \textbf{di}  \textbf{guineano} 
wé  de  tɔ́k   Bata,  di  sonido  nó  dé    di  sén.\\
but    \textsc{def}  Gabonese  \textsc{sub}  \textsc{ipfv}  talk  Fang  with    \textsc{def}  Guinean    
\textsc{sub}  \textsc{ipfv}  talk Fang    \textsc{def}  sound  \textsc{neg}  \textsc{be.loc}  \textsc{def}  same\\

\glt ‘But the Gabonese who talks Fang and the Guinean who talks Fang, the sound is not the same.’ [ma03hm 048]
\z

Example \REF{ex:key:229} illustrates how generic meaning arises through the interplay of NP marking (the bare NP \textit{tidé pikín} ‘children of today’), impersonal use of \textsc{2sg}, and the habitual\is{habitual aspect} reading of the potential modality marker \textit{go}: 


\ea%229
    \label{ex:key:229}
    \gll \textbf{Tidé}  \textbf{pikín}  \textbf{yu}  \textbf{go}  gɛ́t  bɛlɛ́,    \textbf{yu}  púl=an
\textbf{yu}  \textbf{go}  dáy  wet    bɛlɛ́.\\
today  child  \textsc{2sg}  \textsc{pot}  get  belly  \textsc{2sg}  remove=\textsc{3sg.obj}
\textsc{2sg}  \textsc{pot}  die  with    belly\\

\glt ‘As for children of today, they get pregnant, they abort it and die because of the pregnancy.’ [ab03ay 105]
\z

Bare nouns are also encountered in many idiomatic verb-object collocations involving count nouns such as \textit{mék fám} ‘to farm’, \textit{gɛ́t bɛlɛ́} ‘to be pregnant’, or \textit{fála húman} ‘to womanise’. Such noun phrases are also characterised by genericity by virtue of their non-specific reference. They equally reflect a general tendency to omit indefiniteness and number marking with non-specific objects{\fff} \REF{ex:key:230}: {\fff}


\ea%230
    \label{ex:key:230}
    \gll \MakeUppercase{A}   ralla    ín    \textbf{wet}    \textbf{rallador}.\\
\textsc{1sg.sbj}  grate  \textsc{3sg.indp}  with    grater\\

\glt ‘I grated it with a grater.’ [dj03do 004]
\z

\section{Noun phrase modification}\label{sec:5.2}

Nouns are modified by pre- and post-nominal modifiers and possessive constructions. Postnominal modification via focus and topic markers is treated separately in sections \sectref{sec:7.4.2} and \sectref{sec:7.5}, respectively. Nouns may also be modified through relative clause\is{relative clauses}s (cf. \sectref{sec:10.6}) and noun complement clauses (cf. \sectref{sec:10.5.8}).

\subsection{Prenominal modification}\label{sec:5.2.1}

Head nouns of noun phrases may be modified prenominally by other nouns and by verbs in compounds, by nouns in associative constructions, as well as by quantifiers and property items that have been converted to attributive adjectives. In \REF{ex:key:231}, the nouns \textit{mán} ‘man’ and \textit{húman} ‘woman’ are modified by the preposed property item \textit{bíg} ‘(be) big’. 


\ea%231
    \label{ex:key:231}
    \gll Bɔt  wé  di  mán  na  \textbf{bíg}  mán,  di  húman  sɛ́f  na  \textbf{bíg}  húman,
porque  ɔ́l  tɛ́n  na  húman  dé    bɔtɔ́n  mán.\\
but  \textsc{sub}  \textsc{def}  man  \textsc{foc}  big  man    \textsc{def}  woman  \textsc{emp}  \textsc{foc}  big  woman
because  all  time  \textsc{foc}  woman  \textsc{be.loc}  under  man\\
\glt ‘But when the man is a big man, the woman, too is a big woman, because 
it is always the woman who is below the man. [\textstylePichiexamplenumberZchnZchn{hi03cb 152]}
\z

An ordinal numeral\is{ordinal numerals} or similar quantifier such as \textit{ɔ́da} ‘other’ immediately follows the article and precedes other modifiers \REF{ex:key:232}:


\ea%232
    \label{ex:key:232}
    \gll Yu  pút  \textbf{ɔ́da}    \textbf{nyú}    wán    ínsay,  dán    wán    sé
mék    e    nó  smɛ́l.\\
\textsc{2sg}  put  other  new    one    inside  that    one    \textsc{quot}
\textsc{sbjv}    \textsc{3sg.sbj}  \textsc{neg}  smell\\

\glt 
\textstylePichiexamplenumberZchnZchn{‘(Then) you put another one inside, that in order for it} 
\textstylePichiexamplenumberZchnZchn{not to smell.’ [dj03do 048]}
\z

Speakers show clear preferences in their use of verbs for prenominal modification in NPs. Firstly, only numerals and other quantifying expressions (e.g. \textit{nɛ́ks} ‘next’, \textit{plɛ́nte} ‘(be) plenty’) as well as other property items usually function as attributive modifiers. 


Secondly, the following more “basic” semantic types of property items have the highest likelihood of occurring as prenominal modifiers to head nouns: dimension (e.g. \textit{bíg} ‘(be) big’ in \REF{ex:key:231} and \textit{smɔ́l} ‘(be) small), age (e.g. \textit{ól} ‘(be) old’, cf. \ref{ex:key:232}), value (e.g. \textit{bád} ‘(be) bad’, \textit{bɛ́ta} ‘(be) very good’, \textit{fáyn} ‘(be) fine, beautiful’, \textit{trú} ‘(be) true’, and \textit{(s)trɔ́n} ‘be strong, profound (cf. \ref{ex:key:233}), colour (e.g. \textit{blák} ‘(be) black’, \textit{wáyt} ‘(be) white’, and \textit{rɛ́d} ‘(be) red’):



\ea%233
    \label{ex:key:233}
    \gll E    gɛ́t  wán  \textbf{trɔ́n}    stáyl  fɔ  tɔ́k=an.\\
\textsc{3sg.sbj}  get  one  strong  style  \textsc{prep}  talk=\textsc{3sg.obj}\\

\glt ‘There’s a profound way of saying it.’ [ye07je 020]
\z


\ea%234
    \label{ex:key:234}
    \gll Dán  \textbf{wáyt}  tín    wé  e    dé    na  in    yáy.\\
that  white  thing  \textsc{sub}  \textsc{3sg.sbj}  \textsc{be.loc}  \textsc{loc}  \textsc{3sg.poss}  eye\\

\glt ‘That white thing that’s in his eye.’ [dj03cd 103]
\z

Physical properties (e.g. \textit{swít} ‘(be) tasty’, \textit{évi} ‘(be) heavy’, \textit{hád} ‘(be) hard’, \textit{sáf} ‘(be) soft’) are far less likely to appear in the prenominal position. So are human propensities, be they lexicalised as dynamic (e.g. \textit{krés} ‘(be) crazy’, \textit{jɛ́lɔs} ‘(be) envious’) or inchoative-stative verbs (e.g. \textit{wíkɛd} ‘(be) wicked’). Further, the corpus contains no instance of a prenominal, modifying use of labile{\fff} change-of-state verbs like \textit{brók} ‘(be) broken, break,’ \textit{lɔ́s} ‘(be) lost, lose,’ \textit{lɔ́k} ‘close, (be) closed’, and locative verbs{\fff} like \textit{sidɔ́n} ‘sit, seat’. 


Instead, members of the semantic classes listed above preferably occur in other kinds of modifying structures, such as relative constructions \REF{ex:key:235} and compounds \REF{ex:key:236}\is{compounding}: 



\ea%235
    \label{ex:key:235}
    \gll Na  wán    \textbf{mán} wé  e    \textbf{lɔ́s}.\\
\textsc{foc}  one    man    \textsc{sub}  \textsc{3sg.sbj}  lose.\\

\glt ‘He’s a lost man [a hopeless case].’ [be07fn 217]
\z


\ea%236
    \label{ex:key:236}
    \gll Wán    dé  {wán  dé}  dís  húman  go  tɔ́n  \textbf{kres-húman}.\\
one    day  \textsc{rep}      this  woman  \textsc{pot}  turn  crazy.\textsc{cpd}{}-woman\\

\glt ‘Someday this woman will turn into a crazy woman.’ [ro05ee 039]
\z

The few members of the Pichi adjective class (e.g. \textit{fáyn} ‘be fine) may appear in the prenominal modifier position like other property items. However, only adjectives may function as complements\is{complements} to the locative-existential copula \textit{dé} in predicate adjective constructions (cf. \sectref{sec:7.6.5}).

\subsection{Postnominal modification}

Nouns may be modified by postposed elements of two types: focus particles (cf. \sectref{sec:7.4.2}), the topic marker\is{topic marker} \textit{náw} ‘now’, and optionally, by quantifiers like \textit{wán} ‘alone’ (cf. \ref{ex:key:257}–\ref{ex:key:258}), \textit{ɔ́l} ‘all’ (cf. \ref{ex:key:260}), and \textit{dásɔl} ‘only’ (cf. \ref{ex:key:270}). 

\subsection{Possessive constructions}\label{sec:5.2.3}

Pichi employs four types of possessive constructions\is{possessive constructions} through which possessive relations and relations of modification are established between nouns: compounding, the associative construction, the “dislocated possessive construction” \citep[160]{Kouwenberg1994} and a prepositional phrase construction involving the associative preposition \textit{fɔ}. Compounding shares much of its functional space with the associative construction and both constructions are covered extensively in section \sectref{sec:4.4}.

%%please move \begin{table} just above \begin{tabular
\tabref{tab:key:5.3} shows that the order of the participating NPs and forms of linkage are relevant for the way in which possessive relations and relations of modification are established. For ease of exposition, these relations are summarily referred to as “possessive” constructions and the participating NPs as “possessor” and “possessed”, respectively:

%%please move \begin{table} just above \begin{tabular
\begin{table}
\caption{Possessive constructions}
\label{tab:key:5.3}

\begin{tabularx}{\textwidth}{lXlX}
\lsptoprule
Construction & NP 1 & Type of linkage & NP 2\\
\midrule
Compound\is{compounding} & Possessor & Tonal derivation & Possessed\\
Associative & Possessor & Juxtaposition & Possessed\\
Dislocated possessive & Possessor & \textstyleTablePichiZchn{in} ‘\textsc{3sg}.\textsc{poss’}, \textstyleTablePichiZchn{dɛn} ‘\textsc{3pl’} & Possessed\\
\textstyleTablePichiZchn{fɔ-}prepositional & Possessed & \textstyleTablePichiZchn{fɔ} ‘\textsc{prep’} & Possessor\\
\lspbottomrule
\end{tabularx}
\end{table}
In the associative construction, two nouns are juxtaposed, whereby the “possessor” (the modifier noun) modifies the “possessed” noun (the modified noun). Firstly, this construction is always employed when the possessor is instantiated in a possessive pronoun. Secondly, associative constructions express various relations of modification, either exclusively or in complementarity with compounds (cf. \sectref{sec:4.4}). One relation of modification that is always expressed as an associative construction if the possessor is not a multi-constituent NP is a “measure/entity” relation \REF{ex:key:237}. In such constructions, the modifier noun is the measure (\textit{glás} ‘glass’) and the modified noun the entity measured (\textit{watá} ‘water’): \is{associative constructions}


\ea%237
    \label{ex:key:237}
    \gll Wán    \textbf{glás}    \textbf{watá}.\\
one    glass  water\\

\glt ‘One glass of water’ [dj03do 053]
\z

Unlike the associative construction, which typically instantiates a relation of modification between two noun phrases, the dislocated possessive construction typically serves to express a possessive relation. The possessor is therefore usually animate and human – the data contains no instance of a dislocated possessive construction involving an inanimate possessor. 


In the dislocated possessive construction, a possessive pronoun\is{resumptive pronouns} that is co-referential with the possessor intervenes as a linker between the possessor and the possessed noun. With a singular possessor, the \textsc{3sg} possessive pronoun \textit{in} is therefore chosen, and with a plural possessor the \textsc{3pl} possessive pronoun \textit{dɛn}: \is{number!nouns}



\ea%238
    \label{ex:key:238}
    \gll Pero    chico  na  yu  pikín  \textbf{in}    láyf.\\
but    boy    \textsc{foc}  \textsc{2sg}  child  \textsc{3sg.poss}  life\\

\glt ‘But boy, it is your child’s life.’ [hi03cb 133]
\z


\ea%239
    \label{ex:key:239}
    \gll \op...\cp{}  wáyt  pípul  \textbf{dɛn}  wáyf.\\
{} white  people  \textsc{pl}  wife\\

\glt ‘(...) white people’s wives.’ [ed03sp 042]
\z

The dislocated possessive construction requires coreferentiality of the possessive pronoun and the possessor. Hence \REF{ex:key:240}, which involves a \textsc{2sg} person possessor, is ungrammatical:


\ea[*]{%240
    \label{ex:key:240}
    \gll Na  \textbf{yú}    \textbf{in}    hós.\\
  \textsc{foc}  \textsc{2sg.indp}  \textsc{3sg.poss}  house\\
\glt Intended: ‘It’s your house.’ [ne07fn 231]
}
\z

Recursive possessive relations can be expressed by the juxtaposition of possessive constructions, as in \REF{ex:key:241}: 


\ea%241
    \label{ex:key:241}
    \gll Na  dé    a    kán  sabí    \textbf{mi}    mamá  \textbf{in}    papá
\textbf{in}    fámbul. \\
\textsc{foc}  there  \textsc{1sg.sbj}  \textsc{pfv}  know  \textsc{1sg.poss}  mother  \textsc{3sg.poss}  father
\textsc{3sg.poss}  family\\

\glt ‘It is there that I got to know my mother’s father’s family.’ [fr03ft 044]
\z

In the \textit{fɔ}-prepositional construction, the possessed noun is followed by a prepositional phrase that contains a full noun functioning as a possessor \REF{ex:key:242} or modifier \REF{ex:key:243}: 


\ea%242
    \label{ex:key:242}
    \gll \'{A}fta    Miguel  Ángel  wé  na  di  lás  pikín  \textbf{fɔ} mi    antí.\\
then  \textsc{name}  \textsc{name}  \textsc{sub}  \textsc{foc}  \textsc{def}  last  child  \textsc{prep}  \textsc{1sg.poss}  aunt\\

\glt ‘Then (there is) Miguel Ángel who is the last child of my aunt.’ [\textstylePichiexamplenumberZchnZchn{fr03ft 143]}
\z


\ea%243
    \label{ex:key:243}
    \gll \'{A}fta    dɛn  de  gɛ́t  fisionomía  \textbf{fɔ}  Afrika  dɛn.\\
then  \textsc{3pl}  \textsc{ipfv}  get  physiognomy  \textsc{prep}  \textsc{place}  \textsc{pl}\\

\glt ‘Then, they have African physiognomies.’ [\textstylePichiexamplenumberZchnZchn{ed03sp 031]}
\z

Unlike the dislocated possessive construction, the “possessor” in the \textit{fɔ}-construction may be inanimate. This construction therefore typically expresses a relation of modification between a modified (“possessed”) and a modifier (“possessor”) entity. The construction may express various semantic roles including source\index{} \REF{ex:key:244} and material\index{} \REF{ex:key:245} (cf. \sectref{sec:9.1.3} for a complete description of the semantic roles covered by \textit{fɔ} ‘\textsc{prep’}):


\ea%244
    \label{ex:key:244}
    \gll Yu  nó  go  gɛ́t  hambɔ́g  \textbf{fɔ}  \textbf{pípul}  dɛn.\\
\textsc{2sg}  \textsc{neg}  \textsc{pot}  get  irritation  \textsc{prep}  people  \textsc{pl}\\

\glt ‘You won’t get any irritation from people.’ [ma03ni 009]
\z


\ea%245
    \label{ex:key:245}
    \gll Dán  casa    verde,  dán  casa    \textbf{fɔ}  \textbf{madera}  \op...\cp\\
that  house  green  that  house  \textsc{prep}  wood\\

\glt ‘That green house, that wooden house (...)’ [hi03cb 037]
\z

The \textit{fɔ}-construction is also used to express part-whole relations in the idiomatic expression \textit{pát fɔ} ‘part of’ \REF{ex:key:246} or in a partitive\is{partitive} construction involving the determiner \textit{sɔn} ‘some’ \REF{ex:key:247}: 


\ea%246
    \label{ex:key:246}
    \gll Gɔ́bna    de  gí  yu  \textbf{pát}  \textbf{fɔ}  \textbf{di}  \textbf{mɔní}.\\
government  \textsc{ipfv}  give  \textsc{2sg}  part  \textsc{prep}  \textsc{def}  money\\

\glt ‘Government gives you part of the money.’ [hi03cb 064]
\z


\ea%247
    \label{ex:key:247}
    \gll \textbf{Sɔn}    \textbf{fɔ}  \textbf{di}  \textbf{watá}  dé    yét.\\
some  \textsc{prep}  \textsc{def}  water  \textsc{be.loc}  yet\\

\glt ‘Some of the water still remains.’ [ab07fn 224]
\z

The \textit{fɔ}-construction is also employed to express a possessive relation in the same way as the dislocated possessive construction. There appears to be a stronger likelihood for the use of \textit{fɔ}-prepositional constructions instead of dislocated possessive constructions when the possessed NP is complex and features more than one constituent. This is the case in the following example, in which the possessed noun \textit{pikín} ‘child’ is additionally modified by the quantifier \textit{lás} ‘last’: 


\ea%248
    \label{ex:key:248}
    \gll \'{A}fta    Miguel  Ángel  wé  na  di  \textbf{lás}  \textbf{pikín}  \textbf{fɔ}  \textbf{mi}    \textbf{antí}.\\
then  \textsc{name}  \textsc{name}  \textsc{sub}  \textsc{foc}  \textsc{def}  last  child  \textsc{prep}  \textsc{1sg.poss}  aunt\\

\glt ‘Then, there is Miguel Ángel who is the last child of my aunt.’ \textstylePichiexamplenumberZchnZchn{[fr03ft 143]}
\z

Another factor that contributes to the use of the \textit{fɔ}-construction is animacy. The resumptive pronoun\is{resumptive pronouns} in the dislocated possessive construction is typically coreferential with an animate, usually human possessor. Therefore, an inanimate possessor is best expressed through the \textit{fɔ}-construction: \is{animacy}


\ea%249
    \label{ex:key:249}
    \gll Na  wán  prensa  internacional    \textbf{wán}    \textbf{ministro}  \textbf{fɔ}  \textbf{Gabón}
kán  tɔ́k  sé    dán    isla    na  Gabón.\\
\textsc{loc}  one  press  international    one    minister    \textsc{prep}  \textsc{place}
\textsc{pfv}  talk  \textsc{quot}    that    island  \textsc{foc}  \textsc{place}\\

\glt ‘In an international press [newspaper] a secretary of state of Gabon said that that island is [belongs to] Gabon.’ [fr03ft 111]\is{possessive constructions}
\z

\section{Quantification}\label{sec:5.3}

Quantification is expressed through numerals, as well as a variety of relative, absolute, and negative quantifying expressions.

\subsection{Numerals}\label{sec:5.3.1}

Pichi has a decimal numeral system. Cardinal numerals up to ten are listed in \tabref{tab:key:5.4}.

%%please move \begin{table} just above \begin{tabular
\begin{table}
\caption{Cardinal numerals}
\label{tab:key:5.4}
\begin{tabularx}{.8\textwidth}{lXX}
\lsptoprule
 Numeral & Cardinal & Ordinal\\
 \midrule 
 1 & \textit{wán} & \textit{fɔ́s}\\
 2 & \textit{tú} & \textit{sɛkɔ́n, sɛ́kɔn; nɔmba-tú}\\
 3 & \textit{trí} & \textit{nɔmba-trí}\\
 4 & \textit{fó} & \textit{nɔmba-fó}\\
 5 & \textit{fáyf} & \textit{nɔmba-fáyf}\\
 6 & \textit{síks} & \textit{nɔmba-síks}\\
 7 & \textit{sɛ́ven} & \textit{nɔmba-sɛ́ven}\\
 8 & \textit{ét} & \textit{nɔmba-ét}\\
 9 & \textit{náyn} & \textit{nɔmba-náyn}\\
 10 & \textit{tɛ́n} & \textit{nɔmba-tɛ́n}\\
\lspbottomrule
\end{tabularx}
\end{table}

In the corpus, no numeral higher than seven was used in natural speech and no speaker except one could list numerals higher than ‘ten’ without fault. The \ili{Spanish} numeral system is employed by all speakers and has largely replaced Pichi cardinal numerals above three (cf. \sectref{sec:13.3.1} for additional details). Cardinal numerals occur in the prenominal modifier position \REF{ex:key:250} and may be used independently as pronominals \REF{ex:key:251}. The repetition of cardinal numerals renders a distributive sense \REF{ex:key:252}: 


\ea%250
    \label{ex:key:250}
    \gll So  a    dɔ́n  gɛ́t  \textbf{trí}    \textbf{nacionalidad}    na  dís  wɔ́l.\\
so  \textsc{1sg.sbj}  \textsc{prf}  get  three  nationality    \textsc{loc}  this  world\\

\glt ‘So I have three nationalities in this world.’ [fr03ft 102]
\z


\ea%251
    \label{ex:key:251}
    \gll Ɛf  yu  de  ték  \textbf{trí},    treinta  mil.\\
if  \textsc{2sg}  \textsc{ipfv}  take  three  thirty  thousand\\

\glt ‘If you take three, (it is) thirty thousand.’ [f103fp 016]
\z


\ea%252
    \label{ex:key:252}
    \gll Yu  fít  kɛ́r    dɛ́n    \textbf{tú}  \textbf{tú}.\\
\textsc{2sg} \textstylePichiexamplenumberZchnZchn{  can} \textstylePichiexamplenumberZchnZchn{ carry} \textsc{3pl.indp} \textstylePichiexamplenumberZchnZchn{  two} \textstylePichiexamplenumberZchnZchn{\textsc{rep}}\\
\glt ‘You can carry them two by two.’ [\textstylePichiexamplenumberZchnZchn{bo07fn 231]}\is{cardinal numerals}
\z

Pichi has the three lexical ordinal numerals \textit{fɔ́s} ‘first’ \REF{ex:key:253}, \textit{sɛkɔ́n/sɛ́kɔn} ‘second’ \REF{ex:key:254}, and \textit{lás} ‘last’ \REF{ex:key:255}. The first two occur as attributive prenominal modifiers like other property items, while \textit{lás} ‘last’ preferably occurs in quantifier compounds {\fff}: 


\ea%253
    \label{ex:key:253}
    \gll Na  di  \textbf{fɔ́s}  tín  \op...\cp\\
\textsc{foc}  \textsc{def}  first  thing\\

\glt ‘It’s the first thing (...)’ [ab0310ay 010]
\z


\ea%254
    \label{ex:key:254}
    \gll E    gó  blánt  wet    di  \textbf{sɛkɔ́n}  papá.\\
\textsc{3sg}.\textsc{sbj}  go  reside  with    \textsc{def}  second  father\\

\glt ‘She went to stay with the second father [stepfather].’ [hi07fn 225]
\z


\ea%255
    \label{ex:key:255}
    \gll Mí    na  di  \textbf{las-mán}\textbf{\textmd{.}}\\
\textstylePichiexamplenumberZchnZchn{\textsc{1sg.indp}}  \textsc{foc}  \textsc{def}  \textstylePichiexamplenumberZchnZchn{last.}\textstylePichiexamplenumberZchnZchn{\textsc{cpd}}\textstylePichiexamplenumberZchnZchn{{}-man}\\

\glt ‘I’m the last person (here).’ [\textstylePichiexamplenumberZchnZchn{nn07fn 234]}
\z

Ordinal numerals except ‘first’ may also be formed productively through the use of quantifier compounds involving the modifier noun \textit{nɔ́mba} ‘number’ and a cardinal numeral as the head. Most people also use this construction to express ‘second’ \REF{ex:key:256}:


\ea%256
    \label{ex:key:256}
    \gll Di  \textbf{nɔmba}-\textbf{tú}    pikín,  e    kán  tɛ́l  mí    di  sén    tín.\\
\textsc{def}  number.\textsc{cpd}{}-two  child  \textsc{3sg.sbj}  \textsc{pfv}  tell  \textsc{1sg.indp}  \textsc{def}  same  thing\\

\glt ‘(As for) the second child, she told me the same thing.’ [ed03sb 027]\is{ordinal numerals}
\z

The numeral \textit{wán} has a number of functions that are derived from its cardinality sense. We have seen that it functions as an indefinite deteminer and a pronominal or nominal substitute (cf. \ref{ex:key:194}–\ref{ex:key:195}). The adverbialising suffix \textit{-wán} ‘adv’ is also etymologically related to the cardinal numeral \textit{wán} (cf. also \sectref{sec:5.2.1} and \sectref{sec:5.4.4}). The numeral \textit{wán} also expresses adverbial meanings such as ‘alone, single-handedly’ with an emphatic{\fff} nuance, as in \REF{ex:key:257}. When used in this way, \textit{wán} may modify a head noun post-nominally like a postnominal modifier, such as the focus particle \textit{sɛ́f} ‘self, emp’. However, \textit{wán} does not modify full nouns by itself. It rather appears after an independent (emphatic) personal pronoun that is coreferential with the full noun in question \REF{ex:key:258} (cf. also \ref{ex:key:289}–\ref{ex:key:290}):


\ea%257
    \label{ex:key:257}
    \gll Dɛn    tɛ́l=an    sé    “nóto  \textbf{ín}    \textbf{wán}”.\\
\textsc{3pl}    tell=\textsc{3sg.obj}  \textsc{quot}    \textsc{neg}.\textsc{foc}  \textsc{3sg.indp}  one\\

\glt ‘They told her “it’s not only her”.’ [ed03sb 067]
\z


\ea%258
    \label{ex:key:258}
    \gll Mi    \textbf{brɔ́da}  \textbf{ín}    \textbf{wán}    mɛ́n    di  pikín. \\
\textsc{1sg.poss}  brother  \textsc{3sg.indp}  one    raise  \textsc{def}  child\\

\glt ‘My brother raised the [his] child single-handedly.’ [he07fn 444]
\z

\subsection{Other quantifying expressions}

Non-numeral words express relational, absolute and negative quantification (cf. \tabref{tab:key:5.5}). Some of these words modify nouns in a way similar to determiners. One of them is the indefinite determiner \textit{sɔn} ‘some, a’. Some are only employed attributively with nouns (e.g. \textit{hól} ‘whole’). Yet others are only used as pronominals (e.g. \textit{nátin} ‘nothing’).

%%please move \begin{table} just above \begin{tabular
\begin{table}
\caption{Non-numeral quantifiers}
\label{tab:key:5.5}

\begin{tabularx}{\textwidth}{XXll}
\lsptoprule

Type & \multicolumn{2}{c}{Quantifier} & Pronominal use\\
\midrule
Relational & \itshape ɔ́l & ‘all’ & Yes\\
& \itshape ɛ́ni & ‘every’ & No\\
& \itshape ɔ́da & ‘other, next’ & No\\
& \itshape nɛ́ks & ‘next’ & No\\
& \itshape hól & ‘whole’ & No\\
& \itshape háf & ‘half’ & Yes\\
& \itshape ónli & ‘only’ & No\\
& \itshape dásɔl & ‘only’ & No\\
& \itshape sósó & ‘only, abundant(ly)’ & No\\
& \itshape grén & ‘only, exactly’ & No\\

\tablevspace
Absolute & \itshape sɔn & ‘some, a’ & Yes\\
& \itshape bɔkú & ‘much, many’ & Yes\\
& \itshape plɛ́nte & ‘plenty’ & Yes\\
& \itshape smɔ́l & ‘a bit, few’ & Yes\\
& \itshape mɔ́ch & ‘much’ & No\\

\tablevspace
Negative & \itshape nó & ‘no’ & No\\
& \itshape nátin & ‘nothing’ & Yes\\
\lspbottomrule
\end{tabularx}
\end{table}
The quantifier \textit{ɔ́l} ‘all’ occurs with count and mass nouns alike. \textit{ɔ́l} is encountered in a pre- \REF{ex:key:259}, and postnominal position \REF{ex:key:260}, yet without any effect on its quantificational properties:


\ea%259
    \label{ex:key:259}
    \gll \op...\cp{}  yu  de  bák      \textbf{ɔ́l}  \textbf{di}  \textbf{mɔní}  \op...\cp\\
 {} \textsc{2sg}  \textsc{ipfv}  give.back  all  \textsc{def}  money\\

\glt ‘(...) you return all the money (...)’ [hi03cb 184]
\z


\ea%260
    \label{ex:key:260}
    \gll \textbf{Di} \textbf{  pikín} \textbf{  ɔ́l}  sé    na  mi    yón   bikɔs
a    dɔ́n  pé  mɔní.\\
\textsc{def}  child  all  \textsc{quot}  \textsc{foc}  \textsc{1sg.poss}  own  because
\textsc{1sg.sbj}  \textsc{pfv}  pay  money\\

\glt ‘(...) all the children are mine, because I have paid money [the dowry].’ [hi03cb 196]
\z

When \textit{ɔ́l} appears immediately before the noun, it is most often found to modify generic noun{\fff}s like \textit{tín} ‘thing’, \textit{tɛ́n} ‘time’, \textit{pɔ́sin} ‘person’, \textit{mán} ‘human being’, \textit{plés} ‘place’, \textit{sáy} ‘side, place’, and \textit{stáyl} ‘manner’, as in the two following sentences (cf. \sectref{sec:5.4.3}. for a complete listing):


\ea%261
    \label{ex:key:261}
    \gll \textbf{Ɔ́l} \textbf{  mán}  kin  lúk=an,    yu  go  sí  wi  nó  go
mít    nó  bɔ́di    na  hós.\\
all  man    \textsc{hab}  look=\textsc{3sg.obj}  \textsc{2sg}  \textsc{pot}  see  \textsc{1pl}  \textsc{neg}  \textsc{pot}
meet  \textsc{neg}  body  \textsc{loc}  house\\

\glt ‘Everybody watches it, you’ll see, we won’t run into anybody 
in the house.’ [ma03ni 038]
\z


\ea%262
    \label{ex:key:262}
    \gll Porque  na  mí    mí    de  prepara  \textbf{ɔ́l  } \textbf{tín}.\\
because  \textsc{foc}  \textsc{1sg.indp}  \textsc{1sg.indp}  \textsc{ipfv}  cook  all  thing\\

\glt ‘Because it was me, I was cooking everything.’ [dj03do 025]
\z

Rather than seeing syntagmas like \textit{ɔ́l mán} ‘everybody’ and \textit{ɔ́l tín} ‘everything’ above as belonging to a word class\is{word classes} termed “indefinite pronouns”, they are best seen as ordinary NPs involving a quantifer and a generic noun, which may function as equivalents of nominal and adverbial indefinite pronouns in other languages. This analysis is supported by the fact that the generic nouns involved retain their full distributional potential as ordinary nouns; there are no signs of specialisation or grammaticalisation (cf. \citealt[182–183]{Haspelmath1994}).


The occurrence of plural marking in the quantifier phrase in \REF{ex:key:263} also illustrates that a distinction between the meanings of ‘everybody’ and ‘all persons/people’ is irrelevant in Pichi, since genericity can be expressed through bare “singular” nouns and plural-marked nouns alike (cf. \sectref{sec:5.1.4}):



\ea%263
    \label{ex:key:263}
    \gll Mí    sɛ́f,  \textbf{ɔ́l}  \textbf{pɔ́sin}  dɛn  kin  áks  mí    sé    yu  dɔ́n  bɔ́n?\\
\textsc{1sg.indp}  \textsc{emp}  all  person  \textsc{pl}  \textsc{hab}  ask  \textsc{1sg.indp}  \textsc{quot}  \textsc{2sg}  \textsc{pfv}  give.birth\\

\glt ‘As for me, all people ask me, “do you have a child?”’ [fr03ft 152]
\z

\textit{Ɔ́l} ‘all’ may quantify over temporal \REF{ex:key:264} and locative \REF{ex:key:265} expressions. This function may also be fulfilled by the attributive quantifier \textit{hól} ‘whole’ \REF{ex:key:266}. In general, the use of \textit{hól} is, however, rare: 


\ea%264
    \label{ex:key:264}
    \gll “\textbf{Ɔ́l}  \textbf{tidé}    e    bin  de  kɔ́l  mí”,    e    kɔ́l  mí
wán    tɛ́n    dásɔl.\\
\phantom{“}all  today  \textsc{3sg.sbj}  \textsc{pst}  \textsc{ipfv}  call  \textsc{1sg.indp}  \textsc{3sg.sbj}  call  \textsc{1sg.indp}
one    time    only\\

\glt ‘“All of today he was calling me [so he says]”, he [actually] called me only 
once.’ [fr03cd 022]
\z


\ea%265
    \label{ex:key:265}
    \gll \textbf{Ɔ́l} \textbf{  hía}    pák  polvo.\\
all  here    pack  dust\\

\glt ‘All this place is full of dust.’ [ge07fn 127]
\z


\ea%266
    \label{ex:key:266}
    \gll \op...\cp{}  adɔnkɛ́  e    nó  sí  yú    wán    \textbf{hól}    \textbf{dé},  \op...\cp{}\\
 {} even.if  \textsc{3sg.sbj}  \textsc{neg}  see  \textsc{2sg.indp}  one    whole  day  \\

\glt ‘(...) even if she didn’t see you for a whole day, (...)’
\z

The quantifiers \textit{ónli} ‘only’ and \textit{sósó} ‘only, abundant’ have a distribution similar to \textit{hól} above and may appear as prenominal, attributive modifiers to the noun. However, contrary to \textit{hól}, both \textit{ónli} and \textit{sósó} may additionally function as quantifying adverbs. Compare the attributive (a) and adverbial (b) uses of \textit{ónli} \REF{ex:key:267} and \textit{sósó} \REF{ex:key:268} in the following two sentence pairs\textstyleannotationreference{:} 


\ea%267
    \label{ex:key:267}
    \ea{\label{ex:key:267a}
    \gll Di  \textbf{ónli} \textbf{   lángwech}  wé  dɛn  de  tɔ́k  fáyn    fáyn,  \op...\cp{}\\
  \textsc{def}  only    language    \textsc{sub}  \textsc{3pl}  \textsc{ipfv}  talk  fine    \textsc{rep}\\

\glt   ‘The only language that they speak really well (...)’ [au07se 265]
}
    \ex{\label{ex:key:267b}
    \gll \textbf{\'{O}nli}    dɛn  wánt  hía    Panyá.\\
  only    \textsc{3pl}  want  hear    Spanish\\
\glt   ‘They only want to hear Spanish.’ [au07se 211]
}
\z
\z


\ea%268
    \label{ex:key:268}
    \ea{\label{ex:key:268a}
    \gll A    bin  bríng  wán  bláy  só,    \textbf{sósó    } \textbf{jakató}.\\
  \textsc{1sg.sbj}  \textsc{pst}  bring  one  bag  like.this  only    bitter.tomato\\

\glt   ‘I brought a bag like this, full of bitter tomatoes.’ [ro05rt 068]
}
\ex{\label{ex:key:268b}
\gll Aa  \textbf{ sósó }    yandá.\\
  \textsc{intj}  only    yonder\\

\glt   ‘Ah, all the way over there.’ [ge07ga 050] 
}
\z
\z

In contrast, the relational quantifier \textit{dásɔl} ‘only’ behaves like the universal relational quantifier \textit{ɔ́l} ‘all’. Hence, \textit{dásɔl} may appear to the very left of the reference noun \REF{ex:key:269} or occur after the reference noun \REF{ex:key:270}. Aside from that, \textit{dásɔl} is used as a sentence adverb and clause linker (cf. \sectref{sec:10.7.9}):

\ea%269
    \label{ex:key:269}
    \gll \textbf{Dásɔl} \textbf{ wán}     \textbf{smɔ́l},  wán    glas,    yu  fúlɔp=an.\\
only    one    small  one    glass    \textsc{2sg}  fill=\textsc{3sg.obj}\\

\glt ‘Only one small, one glass, you fill it up.’ [dj03do 052]
\z


\ea%270
    \label{ex:key:270}
    \gll Pero    di  fíba    bin  kɛ́r    wán  dé  \textbf{dásɔl}.\\
but    \textsc{def}  fever  \textsc{pst}  carry  one  day  only\\

\glt ‘But the fever lasted only one day.’ [ru03wt 062]
\z

The quantifier \textit{ɛ́ni} ‘every’ quantifies over sets. It therefore has a distributive meaning and can only occur with singular count nouns \REF{ex:key:271}:


\ea%271
    \label{ex:key:271}
    \gll \textbf{Ɛ́ni}    \textbf{dé}  dɛn  de  chɔ́p  rɛ́s,  \textbf{ɛ́ni}    \textbf{dé}.\\
every  day  \textsc{3pl}  \textsc{ipfv}  eat    rice  every  day\\

\glt ‘Every day they eat rice, every day.’ [ed03sp 117]
\z

The quantifier \textit{grén} ‘only, exactly’ (< \textit{grén} ‘grain’) only occurs in fixed collocations as a measure word with a preceding cardinal numeral, and followed by a count noun. Like \textit{ɛ́ni} ‘every’, \textit{grén} therefore quantifies over sets. The resulting quantifier compound{\fff} functions as an attributive quantifier to the following noun \textit{pikín}:


\ea%272
    \label{ex:key:272}
    \gll Na  yu  \textbf{wan-grén} pikín.\\
\textsc{foc}  \textsc{2sg}  one.\textsc{cpd}{}-grain  child\\

\glt ‘That’s your one and only [single] child.’ [ge07fn 015]
\z

The relative or partitive{\fff} quantifiers \textit{sɔn} ‘some’, \textit{bɔkú} ‘much’, \textit{plɛ́nte} ‘plenty’, and \textit{smɔ́l} ‘few, a bit’ may quantify over count and mass nouns alike. NPs featuring one of these forms may be compared to an implicit standard of comparison, like \textit{smɔ́l} ‘few, a bit’ in \REF{ex:key:273} and \textit{sɔn} ‘some’ in \REF{ex:key:274}: 


\ea%273
    \label{ex:key:273}
    \gll \MakeUppercase{A}   kin  wánt  kɔ́f    dɛn  de  trowé  \textbf{smɔ́l} \textbf{ mélk},
leche  tibia      na  mi    trót.\\
\textsc{1sg.sbj}  \textsc{hab}  want  cough  \textsc{3pl}  \textsc{ipfv}  pour  small  milk
milk    lukewarm  \textsc{loc}  \textsc{1sg.poss}  throat\\
\glt ‘I would have to cough (and) they would throw away a little bit of milk,
lukewarm milk inside my throat.’ [ab03ay 087]
\z


\ea%274
    \label{ex:key:274}
    \gll \textbf{Sɔn} \textbf{   fés}  dɛn  dé    wé  a    sabí    nɔ́.\\
some  face  \textsc{pl}  \textsc{be.loc}  \textsc{sub}  \textsc{1sg.sbj}  know  \textsc{intj}\\

\glt ‘There are some faces that I know, right.’ [fr03ft 033]
\z

When the standard of comparison is explicit, the quantifier participates in a partitive\is{partitive} construction. Compare \textit{bɔkú} ‘much, many’ in \REF{ex:key:275} which precedes the standard \textit{mi kɔntri-mán dɛn} ‘my countrymen’:


\ea%275
    \label{ex:key:275}
    \gll Bikɔs  a    gɛ́t  \textbf{bɔkú}  \textbf{mi}    \textbf{kɔntri}-\textbf{mán}    dɛn  
wé  dɛn  húman  kin  dé    fɔ  Annobón.\\
because  \textsc{1sg.sbj}  get  much  \textsc{1sg.poss}  country.\textsc{cpd}{}-man  \textsc{pl}
\textsc{sub}  \textsc{3pl}  woman  \textsc{hab}  \textsc{be.loc}  \textsc{prep}  \textsc{place}\\
\glt ‘Because I have many of my countrymen whose wives are (usually) 
in Annobón.’ [ed03sb 157]
\z

The negative quantifier \textit{nó} ‘\textsc{neg}, no’ is preposed to its referent. This includes the inherently negative indefinite pronoun \textit{nátin} ‘nothing’. Additionally, negative quantifier phrases generally appear with support from verb negation. The resulting clause always yields a single negation reading (cf. \sectref{sec:7.2.3} for more details). Compare the following sentence: 


\ea%276
    \label{ex:key:276}
    \gll \textbf{Nó}  nátín  \textbf{nó}  dé    pantáp=an.\\
\textsc{neg}  nothing  \textsc{neg}  \textsc{be.loc}  on=\textsc{3sg.obj}\\

\glt ‘Nothing is on it [the table].’ [li07pe 011]
\z

Some of the quantifiers covered can function as pronominals, as exemplified with \textit{ɔ́l} ‘all’ in \REF{ex:key:277} (cf. \tabref{tab:key:5.5} for a complete overview). However, a quantifier phrase featuring a generic noun (e.g. \textit{ɔ́l tín} ‘all thing’ = ‘everything’) is usually preferred: 


\ea%277
    \label{ex:key:277}
    \gll \textbf{Ɔ́l}  \textbf{di}  \textbf{tín}    wé  yú    an  dán    mán    bin  gɛ́t,
ɔ́l  de  lɛ́f    fɔ  dán    mán.\\
all  \textsc{def}  thing  \textsc{sub}  \textsc{2sg.indp}  and  that    man    \textsc{pst}  get
all  \textsc{ipfv}  remain  \textsc{prep}   that  man \\

\glt ‘All the things that you and that man had, all remains for that man.’ [hi03cb 191]
\z


\section{Pronouns}\label{sec:5.4}

Pronouns may occur in the syntactic positions of common nouns. At the same time, they fulfil specific grammatical functions and are characterised by distributional preferences and restrictions. 

\subsection{Personal pronouns}\label{sec:5.4.1}

Four features are distinguished in the use of personal pronouns: person, number, syntactic (in)dependence, and case (cf. \tabref{tab:key:5.6} below). The majority of “dependent pronouns” (with the exception of \textit{mi} ‘\textsc{1sg.poss}’ and \textit{in} ‘\textsc{3sg.poss}’) employed for subject case are also used for the expression of possessive case. Where the “possessive” column has no entry, the corresponding “subject” form is used. None of the forms in the “subject” and “possessive”\is{possessive pronouns} columns are simultaneously employed as object pronouns. 


In addition, there is an overlap in forms for the expression of object case. The “object \& emphatic” columns are employed as object pronouns and emphatic pronouns at the same time.  However, the \textsc{3sg} pronouns \textit{=an} and \textit{ín} are suppletive allomorphs. The choice of either of the two forms is phonologically conditioned (cf. \sectref{sec:3.2.5}). One of these forms, i.e. the clitic \textit{=an} ‘\textsc{3sg.obj’,} is the only dependent object pronoun of Pichi\is{dependent object pronoun}. \is{cliticisation}



The \textsc{2pl} pronoun \textit{una/unu} is normally invariable throughout the entire paradigm. Both forms are employed with any difference in meaning, but \textit{una} is used in the vast majority of cases. Independent personal pronouns may undergo tonal derivation in order to participate in compound pronouns which express universal and dual number (cf. \tabref{tab:key:5.7}).\is{number!pronouns}


%%please move \begin{table} just above \begin{tabular
\begin{table}
\caption{Personal pronouns}
\label{tab:key:5.6}

\begin{tabularx}{\textwidth}{Xllll}
\lsptoprule

Person \& Number & \multicolumn{3}{c}{ Dependent pronouns} & Independent pronouns\\
& Subject & \multicolumn{1}{c}{ Possessive} & \multicolumn{1}{c}{ Object} & Object \& emphatic\\
\midrule
\textsc{1sg} & \itshape a & \itshape mi &  & \itshape mí\\
\textsc{2sg} & \itshape yu &  &  & \itshape yú\\
\textsc{3sg} & \itshape e & \itshape in & \itshape =an & \itshape ín\\
\textsc{1pl} & \itshape wi &  &  & \itshape wí\\
\textsc{2pl} & \itshape una, unu &  &  & \itshape una, unu\\
\textsc{3pl} & \itshape dɛn &  &  & \itshape dɛ́n\\
\lspbottomrule
\end{tabularx}
\end{table}

Dependent subject pronouns always occur in finite clauses together with verbs. They may only be separated from the verb by TMA markers, the negator, and preverbal adverbs. Only independent\is{independent pronouns} personal pronouns \is{emphatic pronouns}may be focused \REF{ex:key:278}, topicalised, modified by postposed elements, and conjoined by the coordinators \textit{an} ‘and’ or \textit{ɔ} ‘or’ \REF{ex:key:279}:

\ea%278
    \label{ex:key:278}
    \gll \textbf{Mí}    gɛ́t  tú  brɔ́da.\\
\textsc{1sg.indp}  get  two  brother\\

\glt ‘I [\textsc{emp}] have two brothers.’ [ro07fn 501]
\z

\ea%279
    \label{ex:key:279}
    \gll Bɔt  di  gɛ́l  nó  kán  grí    mék    e    gí  in    bɔ́y  frɛ́n
ɔ  di  pikín  \textbf{ɔ} \textbf{ ín}    sénwe,  e    kán  rɔ́n.\\
but  \textsc{def}  girl  \textsc{neg}  \textsc{pfv}  agree  \textsc{sbjv}    \textsc{3sg.sbj}  give  \textsc{3sg.poss}  boy  friend
or  \textsc{def}  child  or  \textsc{3sg.indp}  self    \textsc{3sg.sbj}  \textsc{pfv}  run\\

\glt ‘But the girl didn’t agree to surrender her boyfriend or the child or herself (and) she ran (away).’ [ed03sb 032]
\z

A focused or topicalised independent pronoun may be followed by a resumptive dependent pronoun \REF{ex:key:280}. This alternative is not very common in the data:

\ea%280
    \label{ex:key:280}
    \gll \textbf{Mí}    \textbf{a}    nó  gɛ́t.\\
\textsc{1sg.indp}  \textsc{1sg.sbj}  \textsc{neg}  get\\

\glt ‘\textsc{As} for me, I don’t have (one).’ [ma03ni 041]
\z

Likewise, only independent personal pronouns occur under focus in cleft constructions {\fff}involving the focus markers \textit{na} ‘\textsc{foc}’ \REF{ex:key:281}, and \textit{nóto} ‘\textsc{neg.foc}’. The example also shows the use of independent pronouns (i.e. \textit{dɛ́n} ‘\textsc{3pl.indp}’) as regular object pronouns (save the clitic \textit{=an} for \textsc{3sg.obj}):


\ea%281
\label{ex:key:281}
\gll E    wás    dí  klós    dɛn,    e    dráy    dɛ́n, nɔ́  \textbf{na}  \textbf{mí}    dráy    dɛ́n.\\
\textsc{3sg.sbj}  wash  this  clothing  \textsc{pl}    \textsc{3sg.sbj}  dry    \textsc{3pl.indp} \textsc{intj}  \textsc{foc}  \textsc{1sg.indp}  dry    \textsc{3pl.indp}\\

\glt ‘She washed the clothes, she dried them, no, it is me (who) dried them.’
[ru03wt 034]
\z

The independent form is also selected when a personal pronoun heads a relative clause\is{relative clauses} \REF{ex:key:282} or is employed as a vocative\is{vocatives} \REF{ex:key:283}:


\ea%282
    \label{ex:key:282}
    \gll Lɛk  náw    só,    \textbf{mí} \textbf{\textmd{[}}\textbf{wé} a    nó  máred,
ɛf  a    bɔ́n      pikín]?\\
like  now    like.that  \textsc{1sg.indp}  \textsc{sub}  \textsc{1sg.sbj}  \textsc{neg}  marry
if  \textsc{1sg.sbj}  give.birth  child \\

\glt ‘Like right now, me who is not married, if I had a child?’ [ab03ab 193]
\z


\ea%283
    \label{ex:key:283}
    \gll \textbf{Yú},    kán    yá!\\
\textsc{2sg.indp}  come  here\\

\glt ‘(Hey) you, come here!’ [ch07fn 232]
\z

%%please move \begin{table} just above \begin{tabular
\tabref{tab:key:5.6} above shows that suppletion and grammatical tone are employed for case and number marking. The following two sentences exemplify the use of tone for pronominal inflection. Sentence \REF{ex:key:284} is a double-object construction. The object and emphatic pronoun \textit{mí} is high-toned. Hence \textit{mí} must be interpreted as the maleficiary\is{maleficiary} object of the verb \textit{tíf} ‘steal’, while \textit{ordenador} ‘computer’ functions as the patient\is{patient} object:


\ea%284
    \label{ex:key:284}
    \gll Dɛn  tíf    \textbf{mí}    ordenador.\\
\textsc{3pl}  steal  \textsc{1sg.indp}  computer\\

\glt ‘They stole a computer from me.’ [ge07fn 169]
\z

Conversely, \REF{ex:key:285} is a single object construction. The low-toned ponoun \textit{mi} is a possessive pronoun to the noun \textit{ordenador} ‘computer’ which functions as a patient NP to the verb \textit{tíf} ‘steal’:


\ea%285
    \label{ex:key:285}
    \gll Dɛn  tíf    \textbf{mi}    ordenador.\\
\textsc{3pl}  steal  \textsc{1sg.poss}  computer\\

\glt ‘They stole my computer.’ [ge07fn 170]
\z

The form \textit{=an} ‘\textsc{3sg.obj}’ is exclusively employed to express object case. It functions as a pronominal object to verbs, prepositions, and locative nouns\is{locative nouns}. It is a clitic that forms a single phonological word with the immediately preceding verb, preposition or locative noun\is{cliticisation}. The pronoun \textit{=an} is sometimes employed indiscriminately for singular or plural reference. In such cases, it may be considered to function as a kind of transitivity or verbal agreement marker. In \REF{ex:key:286}, \textit{=an} is coreferential with the plural-referring pronominal \textit{ɔ́l}: 


\ea%286
    \label{ex:key:286}
    \gll Mí    sénwe  a    mɛ́n=\textbf{an}      \textbf{ɔ́l}.\\
\textsc{1sg.indp}  \textsc{emp}    \textsc{1sg.sbj}  care.for=\textsc{3sg.obj}  all\\

\glt ‘I \textsc{[emp]} myself brought them [the children] all up.’ [ma03ni 030]
\z

Dependent possessive pronouns\is{possessive pronouns} appear before the noun and may in turn be preceded by a demonstrative\is{demonstratives} \REF{ex:key:287}:


\ea%287
    \label{ex:key:287}
    \gll Pero    \textbf{dís}  \textbf{ una} baf-rúm.\\
but    this  \textsc{2pl}    bath.\textsc{cpd}{}-room\\

\glt ‘But this your [\textsc{pl}] bathroom [look how dirty it is].’ [ge07fn 184]
\z

Independent possessive pronouns\is{independent possessive pronouns} are formed by placing a possessive pronoun to the left of the pronominal\is{possessive pronominal} \textit{yón} ‘own’ \REF{ex:key:288}:


\ea%288
    \label{ex:key:288}
    \gll E    sé    a    gó  mɛ́n    pikín  dásɔl  ɛf  a
dɔ́n  sí  \textbf{yu} \textbf{ yón}.\\
\textsc{3sg.sbj}  \textsc{quot}  \textsc{1sg.sbj}  go  care.for  child  only    if  \textsc{1sg.sbj}
\textsc{prf}  see  \textsc{2sg}  own\\

\glt \textstylePichiexamplenumberZchnZchn{‘She said I will only care for a child when I have seen yours.’ [fr03ft 159]}\is{case}
\z

\subsection{Modification of personal pronouns}\label{sec:5.4.2}

Subject and object pronouns can be modified by postposed quantifiers\is{quantifiers} including numerals, focus markers and the topic marker,\is{topic marker} as well as nouns. Aside from that, the pronominal system may be extended through the formation of compound pronouns.


In \REF{ex:key:289}, the pronoun \textit{yú} ‘\textsc{2sg.indp}’ is modified by \textit{wán} ‘one, alone’. \textit{Wán} is semantically compatible with plural referents \REF{ex:key:290}. In \REF{ex:key:291}, the pronoun \textit{ín} ‘\textsc{3sg.indp}’ is modified by \textit{dásɔl} ‘only’. Note the obligatory use of independent (emphatic) pronouns \is{emphatic pronouns}with these quantifiers: 



\ea%289
    \label{ex:key:289}
    \gll Ɛf  yu  bin  dé    \textbf{yú} \textbf{   wán}   yu  nó  bin  fɔ    tɔ́k  só.\\
if  \textsc{2sg}  \textsc{pst}  \textsc{be.loc}  \textsc{2sg.indp}  one    \textsc{2sg}  \textsc{neg}  \textsc{pst}  \textsc{cond}    talk  like.that\\

\glt ‘If you had been alone, you wouldn’t have talked like that.’ [nn07fn 390]
\z


\ea%290
    \label{ex:key:290}
    \gll Na  \textbf{dɛ́n}    \textbf{wán}    de  disfruta  ó.\\
\textsc{foc}  \textsc{3pl}    one    \textsc{ipfv}  enjoy  \textsc{sp}\\

\glt ‘It is them alone who are enjoying [it].’ [ed07fn 280]
\z


\ea%291
    \label{ex:key:291}
    \gll Na  \textbf{ín} \textbf{   dásɔl}  dán  húman  dɔ́n  de  wók    fɔ.\\
\textsc{foc}  \textsc{3sg.indp}  only    that  woman  \textsc{prf}  \textsc{ipfv}  work  \textsc{prep}\\

\glt ‘It is only that that that woman is working for.’ [hi03cb 219]
\z

Sentence \REF{ex:key:292} provides an example of modification by a noun. The country name \textit{Camerún} ‘Cameroon’ modifies the personal pronoun \textit{una} ‘\textsc{2pl}’ by apposition. The modifier noun does not take the pluraliser\is{pluraliser} \textit{dɛn} ‘\textsc{pl}’:


\ea%292
    \label{ex:key:292}
    \gll A    sé    bikɔs  \textbf{una}  \textbf{Camerún}  una  gɛ́t  \op...\cp{}\\
\textsc{1sg.sbj}  \textsc{quot}    because  \textsc{2pl}  \textsc{place}    \textsc{2pl}  get\\

\glt ‘I said because you Cameroonians, you have (...)’ [ab03ay 151]\is{independent pronouns}
\z

Compound pronouns feature a personal pronoun and the quantifiers\is{quantifiers} \textit{tú} ‘two’ and/or \textit{ɔ́l} ‘all’. They are formed by the same means as other compounds: The lexical H tone of the initial component(s) is erased and replaced by an L tone while the final component retains its lexically assigned H tone. Evidence that compounding is indeed at work in the formation of compound pronouns comes from (\ref{ex:key:289}–\ref{ex:key:291}) above. The presence of the postposed quantifiers \textit{wán} ‘alone’ and \textit{dásɔl} ‘only’ in these examples requires the use of H-toned emphatic personal pronouns. In contrast, the \textsc{3pl} form of the personal pronoun in \REF{ex:key:293} below is L-toned, although the quantifier \textit{ɔ́l} ‘all’ is in the same syntactic position as \textit{wán} and \textit{dásɔl} in (\ref{ex:key:289}–\ref{ex:key:291}) above.\is{number!pronouns} 


The collocation \textit{dɛn-ɔ́l} ‘\textsc{pl}-all’ may be employed in order to signal inclusivity of all referents. The use of a resumptive simplex dependent pronoun as in \REF{ex:key:293} is optional but very common: 



\ea%293
    \label{ex:key:293}
    \gll \textbf{Dɛn-ɔ́l}      \textbf{dɛn}  de  salút  dɛn  sɛ́f.\\
\textsc{3pl.indp.cpd-}all  \textsc{3pl}  \textsc{ipfv}  greet  \textsc{3pl}  self\\

\glt ‘They are all greeting each other.’ [dj07re 009]
\z

A compound pronoun may also feature the numeral \textit{tú} ‘two’ as the second component and thereby express dual number \REF{ex:key:294}. Such dual compound pronouns are most frequently formed by additionally incorporating the quantifie\is{quantifiers}r \textit{ɔ́l} ‘all’ into the compound \REF{ex:key:295}. The data contains no trial compound pronouns formed with the numeral \textit{trí} ‘three’:


\ea%294
    \label{ex:key:294}
    \gll Dɛn  go  reúne,  \textbf{dɛn-tú}      \textbf{dɛn}  go  kɔ́l  di  bɔ́y  \op...\cp{}\\
\textsc{3pl}  \textsc{pot}  meet  \textsc{3pl.indp.cpd-}two  \textsc{3pl}  \textsc{pot}  call  \textsc{def}  boy\\

\glt ‘They would meet, the two of them would call the boy (...)’  [ab03ay 042]
\z


\ea%295
    \label{ex:key:295}
    \gll Yu  sí,  \textbf{dɛn}{}-\textbf{ɔl}{}-\textbf{tú}     júmp  fɔ  bɔ́t    di  bɔ́l.\\
\textsc{2sg}  see  \textsc{3pl.indp.cpd-}all.\textsc{cpd}{}-two  jump  \textsc{prep}  head  \textsc{def}  ball\\

\glt ‘You see, they both jumped to head the ball.’ [au07se 058]
\z

Compound personal pronouns are employed in a regular and conventionalised way in order to express dual number with any of the three plural personal pronouns. Note the deletion of the H tones and replacement by L tones over all components of the dual object \textit{wi-ɔl-tú} ‘the two of us’ save the last one (i.e. \textit{tú} ‘two’, which bears its original lexical H tone) in \REF{ex:key:296}:


\ea%296
    \label{ex:key:296}
    \gll Lɛk  sé    dɛn  de    hía    \textbf{wi-ɔl-tú}    \textbf{wi}  de  tɔ́k  yét.\\
like  \textsc{quot}  \textsc{3pl}  \textsc{ipfv}  hear    \textsc{1pl.indp.cpd-}all.\textsc{cpd}{}-two  \textsc{1pl}  \textsc{ipfv}  talk  yet\\

\glt ‘Like if they heard both of us still talking.’ [au07se 217]
\z

Examples \REF{ex:key:294} and \REF{ex:key:296} also show that dual pronouns are anaphorically referred to (i.e. through the resumptive pronoun\is{resumptive pronouns}s \textit{dɛn} ‘\textsc{3pl}’ and \textit{wi} ‘\textsc{1pl}’, respectively) by making use of the corresponding plural pronoun. \is{anaphora}


The extension of the Pichi pronominal system by compounding is summarised in \tabref{tab:key:5.7}. Compound object, subject, and emphatic pronouns are identical. For possessive and resumptive pronouns, the regular plural pronouns are employed. Optional elements are in parentheses:


%%please move \begin{table} just above \begin{tabular
\begin{table}
\caption{Compound personal pronouns}
\label{tab:key:5.7}


\begin{tabularx}{\textwidth}{XXX}
\lsptoprule

Person \& number & Subject/object/

emphatic & Possessive/

resumptive\\
\midrule
1 dual & \itshape wi-(ɔl)-tú & \itshape wi\\
2 dual & \itshape una-(ɔl)-tú & \itshape una\\
3 dual & \itshape dɛn-(ɔl)-tú & \itshape dɛn\\
1 universal & \itshape wi-ɔ́l & \itshape wi\\
2 universal & \itshape una-ɔ́l & \itshape una\\
3 universal & \itshape dɛn-ɔ́l & \itshape dɛn\index{}\\
\lspbottomrule
\end{tabularx}
\end{table}
\subsection{Indefinite pronouns} \label{sec:5.4.3}

In Pichi, the functional equivalents of indefinite pronouns are phrases involving generic nouns preceded by the quantifier{\fff} and indefinite determiner \textit{sɔn} ‘some, a’ as well as the quantifier{\fff}s \textit{ɔ́l} ‘all’, \textit{ɛ́ni} ‘every’, and \textit{nó} ‘\textsc{neg}’. \tabref{tab:key:5.8} provides an overview of ‘some’ and ‘every’ indefinites involving the generic nouns \textit{pɔ́sin} ‘person’, \textit{mán} ‘man, person’, \textit{tín} ‘thing’, \textit{sáy} ‘side, place’, \textit{(káyn) stáyl} ‘(kind of) style’, \textit{tɛ́n} ‘time’, and \textit{áwa} ‘hour, time’. Some examples for their use are provided in (\ref{ex:key:190}–\ref{ex:key:191}) above as well as (\ref{ex:key:261}–\ref{ex:key:263}) above. 


An extensive treatment of ‘no’ and ‘any’ forms, hence negative phrases with the functions of negative indefinites, is provided in \sectref{sec:7.2.3}.\is{generic noun}


%%please move \begin{table} just above \begin{tabular
\begin{table}
\caption{Indefinite pronouns}
\label{tab:key:5.8}

\begin{tabularx}{\textwidth}{lXlXl}
\lsptoprule
 & ‘Some’ &  & ‘Every’ & \\
 \midrule 
person & \itshape sɔn pɔ́sin & ‘somebody’ & \itshape ɔ́l pɔ́sin, & ‘everybody’\\
& \itshape sɔn mán &  & \itshape ɔ́l mán & \\

\tablevspace
thing & \itshape sɔn tín & ‘something’ & \itshape ɔ́l tín, & ‘everything’\\
&  &  & \itshape ɛ́ni tín & \\

\tablevspace
place & \itshape sɔn sáy & ‘somewhere’ & \itshape ɔ́l sáy, & ‘everywhere’\\
&  &  & \itshape ɛ́ni sáy & \\
\tablevspace
manner & \itshape sɔn (káyn)\newline stáyl & ‘somehow’ & \itshape ɔ́l (káyn) stáyl,\newline ɛ́ni (káyn)\newline stáyl & ‘(in) every way’\\
\tablevspace
time & \itshape sɔn tɛ́n dɛn & ‘sometimes’ & \itshape ɔ́l tɛ́n & ‘always’\\
& \itshape sɔn áwa (dɛn) &  & \itshape ɔ́l áwa & \\
&  &  & \itshape ɛ́ni tɛ́n & ‘every time’\\
\tablevspace
kind & \itshape sɔn káyn & ‘some kind of’ & \itshape ɔ́l káyn & ‘every kind of’\\
&  &  & \itshape ɛ́ni káyn & \\
\lspbottomrule
\end{tabularx}
\end{table}

A few characteristics of the NPs in \tabref{tab:key:5.8} are worthy of note. Firstly, Pichi makes no difference between “some” indefinites used in affirmative and realis modality declarative sentences and “free-choice” indefinites (\citealt[48–52]{Haspelmath1997}) of the “any” type.


Secondly there are a few idiosyncracies in the formation of indefinites: while \textit{sɔn pɔ́sin} ‘somebody’ is more common than \textit{sɔn mán}, \textit{ɔ́l mán} ‘everybody’ is favoured over \textit{ɔ́l pɔ́sin}; “manner” is equally often expressed as \textit{sɔn stáyl} as it is involving the modifier substitute \textit{káyn} ‘kind’. Finally, note that ‘sometimes’ is expressed as \textit{sɔn tɛ́n dɛn}, hence a plural NP while \textit{sɔn.tɛ́n} is a lexicalised collocation functioning as an adverb with the meaning ‘perhaps’. Also note that \textit{tɛ́n} ‘time’ is a count noun, hence quantification with \textit{ɛ́ni} ‘every’ renders the distributive meaning ‘every time’. 


\subsection{Pronominals}\label{sec:5.4.4}

The pronominals \textit{sɛ́f} ‘self’, \textit{yón} ‘own’, {\fff}and \textit{nátin} ‘nothing’ occur in the syntactic positions of nouns. At the same time, they are characterised by a preference for specific environments or show distributional restrictions. The anaphoric pronominals \textit{sɛ́f} ‘self’ and \textit{yón} ‘own’ are employed to form independent reflexive{\fff} and possessive pronouns and do not co-occur with determiners either. Instead, they are usually preceded by possessive pronouns. The negative indefinite pronoun \textit{nátin} ‘nothing’ only occurs in negative clauses. 


There is a transition from these more specialised pronominals characterised by restrictions to pronominals like \textit{káyn} ‘kind’ and \textit{wán} ‘one’, which favour specific environments, to generic nouns like \textit{mán} ‘man, person’, \textit{sáy} ‘place’, \textit{stáyl} ‘manner’, and \textit{tɛ́n} ‘time’, which behave like other common nouns but fulfil important functions in the grammatical system of Pichi. For example, \textit{káyn} ‘kind’ and \textit{wán} ‘one’ may co-occur with a determiner or a prenominal modifier. \textit{Káyn} appears as a head noun in question words and as a generic noun{\fff} in the modifier or modified position of certain conventionalised collocations (e.g. \textit{na wán káyn tín} ‘\textsc{foc} one kind thing’ = ‘that’s really something’). 



\textit{Wán} also functions as a generic\is{generic nouns} substitute for any other common noun, and in this function, it may be preceded by prenominal modifiers or determiners (e.g. \textit{di ɔ́da wán} ‘\textsc{def} other one’ = ‘the other one’).\is{pronouns} 


\section{Coordination}\label{sec:5.5}
The form which is most commonly employed for signalling coordination between two noun phrases is the comitative preposition \textit{wet} ‘with’ \REF{ex:key:297}. The form \textit{an} ‘and’ is also used to coordinate noun phrases \REF{ex:key:298} next to being employed as a sentential coordinator (cf. \sectref{sec:10.3}). However, most speakers have a clear preference for \textit{wet} rather than \textit{an}: 


\ea%297
    \label{ex:key:297}
    \gll Lydia  \textbf{wet}    Junior,  na  dɛ́n    a    sabí.\\
\textsc{name}  with    \textsc{name}  \textsc{foc}  \textsc{3pl.indp}  \textsc{1sg.sbj}  know\\

\glt ‘Lydia and Junior, it’s them I know.’ [fr03ft 134]
\z


\ea%298
    \label{ex:key:298}
    \gll Ɔ́l  di  tín    wé  yú    \textbf{an}  dán  mán    bin  gɛ́t \op...\cp{}\\
all  \textsc{def}  thing  \textsc{sub}  \textsc{2sg.indp}  and  that  man    \textsc{pst}  get\\

\glt ‘All the things that you and that man had (...)’ [hi03cb 191]
\z

The disjunctive coordinator is \textit{ɔ} ‘or’, which alternates in pronunciation between [ɔ̀] and [ò]. This variation in form is likely to be reinforced by the existence of the equivalent Spanish coordinator \textit{o} ‘or’:


\ea%299
    \label{ex:key:299}
    \gll \op...\cp{}  wé  a    tínk    sé    na  judías  blancas  \textbf{ɔ}  rɛ́s.\\
 {}  \textsc{sub}  \textsc{1sg.sbj}  think  \textsc{quot}    \textsc{foc}  bean.\textsc{pl}  white.\textsc{pl}  or  rice.\\

\glt ‘(...) of which I think that it is white beans or rice.’ [ed03sp 122]\is{coordination!noun phrases}
\z

