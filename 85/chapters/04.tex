\chapter{Morphology}

Pichi nouns and verbs constitute two major word classes. Adjectives, prepositions, and adverbs constitute minor word classes with a few members each. Pichi word formation strategies are predominantly analytic. Besides that, the use of one (adverb-deriving) affix and morphological tone play a role in Pichi derivation and inflection.

\section{Word classes}\label{sec:4.1}

Pichi word classes are differentiated by their syntactic functions (e.g. a noun may head an \textsc{NP}), distribution within the sentence (e.g. a preposition may not be preceded by an article), the morphosyntactic categories that may be specified for them (e.g. verbs may be specified for tense\is{tense}, \isi{aspect}, and \isi{modality}), their derivational potential (e.g. personal pronouns and prepositions are not normally reduplicated, and adverbs do not function as nouns), as well as semantic criteria (dynamic states-of-affairs and property concepts are generally expressed as verbs). 


The major underived word classes, with the most members and the potential to occur in the largest range of environments, are nouns and verbs. The noun-verb distinction in Pichi is quite strong: although verbs may function as nouns in specific (e.g. in emphatic) contexts, the reverse is not usually the case. The verb-adjective distinction is weak. There are only a handful of adjectives, which are indistinguishable from verbs in most environments. The minor word classes consist of adjectives, prepositions, adverbs, as well as various sentence elements that contribute to the meaning of the sentence. 


\subsection{Nominals}

Nouns appear as one of up to three core participant\is{core participants}s of a verb, i.e. as subjects or up to two objects\is{objects}. Nouns also occur as objects of prepositions, and they may function as adverbials. They may be modified by other elements of the noun phrase (e.g. \textit{di} ‘\textsc{def}’, \textit{dá(n)} ‘that’, \textit{sɔn} ‘some, a’ or \textit{dɛn} ‘\textsc{pl’)}, including other nouns in associative constructions and compounds. \is{associative constructions}The vast majority of nouns bears a single H tone and belongs to one of the major tone classes (cf. \tabref{tab:key:3.2}). Underived nouns typically denote time-stable object concepts. Nouns also belong to an open class which may be extended by compounding, conversion, and borrowing\is{borrowing} from \ili{Spanish}.


Personal pronouns, pronominals, and compound question word{\fff}s are subsets of nominals that exhibit a more restricted distribution. Personal pronouns are found in the same syntactic positions as noun phrases but do not cooccur with preposed modifiers. The latter usually also holds true for the pronominals \textit{nátin} ‘nothing’, \textit{sɛ́f} ‘self’, and \textit{yón} ‘own’. The pronominals \textit{káyn} ‘kind’ and \textit{wán} ‘one’ have a wider distribution but are also characterised by specific syntactic preferences. Locative nouns{\fff} form a further subclass of nominals characterised by distributional specificities. Locative nouns are not often preceded by modifiers or determiners, and their distribution overlaps with that of prepositions. 


\subsection{Verbs and adjectives}\label{sec:4.1.2}

Verbs occupy the centre of the predicate. The predicate is best seen to include a number of functional elements that form a tightly-knit unit with the verb in order to constitute clauses: TMA markers, preverbal adverbs\is{preverbal adverbs}, the negator, dependent personal pronouns, as well as the clitic \textsc{3sg} object pronoun. Verbs are usually preceded by a subject noun, pronoun, or both. Verbs may optionally be followed by objects\is{objects}. They are typically mono- or bisyllabic and usually belong to one of the three major tone classes. 


There are numerous subclasses of verbs which can be defined along formal and semantic lines: Aspectual and modal verbs, transfer and communication verbs, stative, inchoative-stative, and dynamic verbs, labile verbs{\fff}, and copula verbs. Other than reduplication, Pichi only has marginally productive means of verb derivation through compounding{\fff}. There are numerous other strategies for the creation of new verbal meanings, e.g. light verb constructions, involving \textit{gɛ́t} ‘get, have’, \textit{mék} ‘make’, or \textit{gí} ‘give’, as well as systematic borrowing{\fff} from Spanish.



There is just a handful of adjectives in Pichi. A small set of property items alternates between uses as inchoative-stative verbs and as adjectives (cf. \tabref{tab:key:7.11} in \sectref{sec:7.6.5}). {\fff}The overwhelming majority of property concepts are lexicalised as inchoative-stative verbs in Pichi. The following “semantic types” \citep[3]{dixon_adjective_2004} are expressed through inchoative-stative verbs: Dimension (e.g. \textit{bíg} ‘be big’, \textit{smɔ́l} ‘be small’, and \textit{lɔ́n} ‘be long’), age (e.g. \textit{ól} ‘be old’ and \textit{yún} ‘be young’), value (e.g. \textit{bád} ‘be bad’, \textit{fáyn} ‘be good’, and \textit{trú} ‘be true’), and colour (e.g. \textit{blák} ‘be black’, \textit{wáyt} ‘be white’, \textit{rɛ́d} ‘be red’, and \textit{yɛ́lo} ‘be yellow’). Most physical properties are also lexicalised as inchoative-stative verbs (e.g. \textit{hád} ‘be hard’, \textit{sáf} ‘be soft’, \textit{sók} ‘be wet’, \textit{évi} ‘be heavy’, \textit{hɔ́t} ‘be hot’, \textit{swít} ‘be tasty’). 



Human propensities are divided between inchoative-stative (e.g. \textit{gudhát} ‘be good-hearted’, \textit{wíkɛd} ‘be wicked’, \textit{badhát} ‘be mean’, \textit{klɛ́va} ‘be clever’) and dynamic verbs (e.g. \textit{gládin} ‘be glad’, \textit{jɛ́lɔs} ‘be envious’) according to whether they denote intrinsic or transient properties. Resultatives{\fff} are exclusively expressed through the stative readings of labile change-of-state verbs (e.g. \textit{brók} ‘break, be broken’, \textit{chɛ́r} ‘tear, be torn’, \textit{lɔ́s} ‘lose, be lost’ and \textit{wɛ́r} ‘be dressed’). Semantic types like position or location are expressed through other means, such as copula clauses featuring the locative-existential copula \textit{dé} (cf. e.g. \ref{ex:key:793}–\ref{ex:key:794}) in combination with adverbials, or through locative verbs like {\fff}\textit{lé} ‘lie’ and \textit{tínap} ‘stand (up)’ (cf. \sectref{sec:8.1.3}).


\subsection{Other word classes}\label{sec:4.1.3}

Most prepositions must be followed by an object, although some may be stranded\is{stranding}, that is, they may occur in the clause-final position. Prepositional phrases are found in the clause-initial or -final position. A majority of prepositions is monosyllabic, a few are bisyllabic. Pichi exhibits a division of labour between prepositions, locative nouns, locative adverbs, and locative verbs in order to express spatial relations. The language has a small number of underived adverbs, amongst them a group of four preverbal adverbs\is{preverbal adverbs}.


Each of the following groups of modifiers may also be said to constitute minor word classes unto themselves, because they occupy distinct syntactic positions in the noun phrase or predicate: the article, demonstratives, quantifiers, prenominal attributive modifiers, numerals, the pluraliser\is{pluraliser}, emphasis markers, topicalisers, TMA markers, aspectual and modal verbs, the general negator, interjections, and ideophones\is{ideophones}. Certain elements modify sentences in their entirety with respect to pragmatic status (e.g. question words, tags, focus particles, interjections) or link sentences with each other (e.g. clause linkers and conjunctions). These sentential elements may also each be considered a separate word class due to their functions and syntactic behaviour.\is{word classes} 


\section{Inflection}\label{sec:4.2}

Most grammatical functions are realised analytically by independent words without the morphological modification of heads or dependents. Participant-marking is taken care of by prepositions and locative nouns, serial verb constructions, and word order, and nominal modification by juxtaposition of adjectives and other modifiers\is{word orderes}. Number-marking is achieved by post-nominal modification. 

The verbal category of number is signalled by personal pronouns and reduplication. Complementisers, preverbal TMA markers, serial verb constructions, and adverbs participate in expressing the grammatical categories of tense{\fff}, modality, and aspect. Comparison is expressed by adverbs of degree, ideophones, verbs, phrasal expressions, suprasegmental modification, serial verb constructions, and prepositions. There are, however, exceedingly rare cases of number marking on \textit{gɛ́l/gál} ‘girl’ and \textit{bɔ́y} ‘boy’ by an apparently marginal plural affix \{-s\}, hence \textit{gɛ́l-s}, \textit{bɔ́y-s}. 


A description of the only inflectional morphological processes follows. The expression of the grammatical relations of subject, object, and possessive case may be seen to involve the use of (tonal) suprafixation, summed up in \tabref{tab:key:4.1} (cf. \sectref{sec:5.4.1} for the full pronominal paradigm and examples).


%%please move \begin{table} just above \begin{tabular
\begin{table}
\caption{Suprafixation with personal pronouns}
\label{tab:key:4.1}

\begin{tabularx}{.75\textwidth}{Xl}
\lsptoprule

Category expressed & Suprafix\\
\midrule
Object case \& independent pronouns & H tone\\
Subject \& possessive case & L tone\\
\lspbottomrule
\end{tabularx}
\end{table}

Tone-conditioned suppletive allomorphy also fulfils inflectional functions in Pichi, even if it involves outright substitution rather than morphological modification (cf. \sectref{sec:3.3}). It has been suggested that the cognate form of the Pichi imperfective marker \textit{de} ‘\textsc{ipfv}’ be analysed as an inflectional verbal prefix in Jamaican Creole \citep[30]{Farquharson2007}. In Pichi too, the use of resumptive imperfective marking with the preverbal aspectual adverbs \textit{jís/jɔ́s} ‘just’ and \textit{stíl} ‘still’ suggests a tighter-than-usual syntagmatic relation between the imperfective aspect marker and the verb it modifies:


\ea%98
    \label{ex:key:98}
    \gll   Náw    dɛn  \textbf{de}  \textbf{jís}  \textbf{de}  kán.\\
now    \textsc{3pl}  \textsc{ipfv}  just  \textsc{ipfv}  come\\

\glt ‘Now, they’re just coming.’ [ye07je 179] \is{preverbal adverbs}
\z

\section{Derivation}\label{sec:4.3}

Pichi makes use of morphological processes for the purpose of derivation. One is a tonal process which derives compounds, including reduplications. The other is adverb-deriving suffixation. Compounding and reduplication are two highly productive derivational processes in Pichi. 

\subsection{Affixation}
\tabref{tab:key:4.2} summarises the derivational processes found in Pichi. This section covers formal aspects of compounding and reduplication, which both receive a more detailed functional treatment in \sectref{sec:4.4} and \sectref{sec:4.5.1}, respectively. Adverb-deriving suffixation is covered in this section in both its formal and functional aspects.

%%please move \begin{table} just above \begin{tabular
\begin{table}
\caption	{Derivational processes}
\label{tab:key:4.2}
{\small\begin{tabularx}{\textwidth}{QQ lQ l}
\lsptoprule
Category expressed & Word class applied to & (Supr-)affix & Process & Productivity\\
\midrule
Verbal plurality & Dynamic verbs & L tone + \textsc{red} & Tone deletion + iteration & High\\

\tablevspace
Nominal and verbal compound & Nouns, pronouns verbs, adverbs, phrases & L tone & Tone deletion & Fair\\

\tablevspace
Manner adverb & Verbs, adjectives & {}-\textstyleTablePichiZchn{wán} \textsc{‘adv’} & Suffixation & Low\\
\lspbottomrule
\end{tabularx}}
\end{table}

Compounding and reduplication both make use of the same tonal derivation. Reduplication is therefore best seen as a form of (self-)compounding in Pichi. In the process, the H tone over the initial component(s) is deleted and replaced by an L tone. The final component retains its original tone configuration. The resulting compound word then features a single H tone like most Pichi words. Pichi compounds are therefore right-headed; the L-toned initial component(s) function as modifier(s) to the final component, which is the head. 


Nouns, verbs, adjectives, and adverbs participate in compounds. The resulting structures may function as nouns or verbs. Personal pronouns may also participate as modifiers in compound personal pronouns (cf. \sectref{sec:5.4.2}). Compounding is fairly productive (cf. \sectref{sec:4.4} for details). Compare the compound in \REF{ex:key:99} featuring the modifier noun \textit{kɔ́ntri} ‘country, home town’ and the modified noun \textit{chɔ́p} ‘food’. While \textit{kɔ́ntri} loses the H tone over its first syllable, the head noun \textit{chɔ́p} retains its original H tone:



\ea%99
    \label{ex:key:99}
    \gll   Na  in    \textbf{kɔntri}{}-\textbf{chɔ́p}.\\
\textsc{foc}  \textsc{3sg.poss}  country.\textsc{cpd}{}-food\\

\glt ‘That’s his local food.’ [au07ec 007]
\z

Compounding through tone deletion also characterises the reduplication of dynamic verbs in order to derive verbal number \REF{ex:key:100}. This kind of derivation is fully productive for all dynamic verbs. Equally, it can be observed with a small number of lexicalised reduplications involving other word classes (cf. \sectref{sec:4.5.3}):


\ea%100
    \label{ex:key:100}
    \gll Kán  tót    bɛlɛ́,    bigín  de  \textbf{hala}-\textbf{hála}  náw,  \textbf{hala}-\textbf{hála}.\\
\textsc{pfv}  carry  belly  begin  \textsc{ipfv}  \textsc{red.cpd-}shout  now    \textsc{red.cpd-}shout\\

\glt ‘(Then she) became pregnant, (and) began lamenting and lamenting.’ [ab03ay 118]\is{derivation!tonal}
\z

Adverbs are derived from verbs and adjectives by means of the suffix -\textit{wán} \textsc{‘adv’}, etymologically related to the numeral \textit{wán} ‘one’. Amongst its numerous other uses (cf. \sectref{sec:5.3.1}), the cardinal numeral \textit{wán} ‘one’ serves as a pronominal substitute for nouns in \textsc{NPs} featuring attributively used property items (i.e. \textit{di blák wán} ‘the black one’; \textit{di bíg wán} ‘the big one’). When such \textsc{NPs} appear in an adverbial slot in the clause, the resulting structure functions as a manner adverb. 


The semantic link between the function of -\textit{wán} \textsc{‘adv’} as an adverbialising suffix and the meaning of \textit{wán} in other contexts is opaque. This warrants the analysis of -\textit{wán} ‘\textsc{adv’} as a suffix rather than seeing it as the second component of a compound word. \is{derivation!affixation}The derivation of adverbs is a derivational process distinct from compounding and does not involve the tone deletion that accompanies the latter kind of word formation. In the following examples, the property items \textit{fáyn} ‘(be) fine’ \REF{ex:key:101} and \textit{smɔ́l} ‘(be) small’ \REF{ex:key:102} and the affix \textit{-wán} retain their lexically assigned H tone. The resulting adverbs are bisyllabic, bimorphemic words with an H\textsc{{}-}H (downstepped H) tone configuration:\is{cardinal numerals}



\ea%101
    \label{ex:key:101}
    \gll E    mék=an    \textbf{fáyn-wán}.\\
\textsc{3sg.sbj}  make=\textsc{3sg.obj}  fine\textsc{{}-adv}\\

\glt ‘She made it nicely.’ [ra07ve 017]
\z


\ea%102
    \label{ex:key:102}
    \gll E    fáyn    fɔ  dríng  \textbf{smɔ́l-wán}.\\
\textsc{3sg.sbj}  fine    \textsc{prep}  drink  small\textsc{{}-adv}\\

\glt ‘It’s good to drink moderately.’ [ma03hm 071]
\z

The derivation of manner adverbials through the suffixation of -\textit{wán} is not particularly productive. In the corpus, it is unanimously accepted with a limited number of monosyllabic property items denoting physical properties, such as \textit{smɔ́l} ‘be small’, \textit{kól} ‘be cold’, \textit{hɔ́t} ‘be hot’, \textit{fáyn} ‘be fine’. In contrast, the formation of adverbials with many other property items was rejected by informants, amongst them \textit{dɔtí} ‘be dirty’, \textit{bád} ‘be bad’, \textit{bɛlfúl} ‘be satiated’, \textit{nékɛd} ‘be naked’, \textit{táya} ‘be tired’, \textit{lét} ‘be late’, \textit{frɛ́s} ‘be fresh’, \textit{rɛ́p} ‘be ripe’, and \textit{sáful} ‘be slow, diligent’. 


The generic noun \textit{tɛ́n} occurs in a small number of more or less lexicalised expressions functioning as sentence and time adverbs. All of the expressions contained in the corpus are listed in \REF{ex:key:103}. Like derived adverbs featuring the suffix -\textit{wán} ‘\textsc{adv}’, these bisyllabic expressions are not compounds, since there is no tonal derivation. 



\is{cardinal numerals}The meanings of these expressions are semantically distinct from the meanings of their components in varying degrees. The degree of semantic opaqueness of each collocation is reflected in the orthographic choice of writing them as single or separate words. A good indicator of the  degree of semantic unity of the collocations in \REF{ex:key:103} is their behaviour during repetition\is{repetition} for emphasis (cf. \ref{ex:key:152} further below). Even in the lexicalised expressions (e.g. \textit{bádtɛn} ‘unfortunately) each morpheme nevertheless retains its original pitch, as shown by tone marking. This renders complex words with a sequence of two H tones (the second H undergoes downstep).

\eabox{\label{ex:key:103}
\begin{tabularx}{.9\textwidth}{lll}
Construction & Components & Gloss\\
\itshape lɔ́n tɛ́n & long time & ‘long time ago’\\
\itshape (di) fɔ́s tɛ́n & (the) first time & ‘(the) first time, formerly’\\
\itshape wán tɛ́n & one time & ‘once’\\
\itshape wán.tɛn & one.time & ‘at once, suddenly’\\
\itshape bád.tɛn & bad.time & ‘unfortunately’\\
\itshape smɔ́l.tɛn & small.time & ‘shortly, nearly’ \\
\itshape sɔn tɛ́n dɛn & some time \textsc{pl} & ‘sometimes’\\
\itshape sɔn.tɛ́n & some.time & ‘perhaps’\\
\end{tabularx}
}
The largely unpredictable meanings of the adverbs in \REF{ex:key:103} are reason enough to consider them as lexicalised phrasal expressions, rather than analysing \textit{tɛ́n} as a productive adverbialising suffix. 

\subsection{Conversion}

Some word classes\is{word classes} are characterised by multifunctionality. They may undergo conversion and appear in a syntactic position reserved for another class without morphological derivation. \tabref{tab:key:4.3} provides an overview of productive conversion. Some processes are unidirectional, others bidirectional. Arrows indicate the direction of conversion\textstyleannotationreference{.} The productivity of conversion varies with word class and is often subject to lexical idiosyncracies. 

%%please move \begin{table} just above \begin{tabular
\begin{table}
\caption{Conversion}
\label{tab:key:4.3}

{\small\begin{tabularx}{\textwidth}{llcl}
\lsptoprule

Type of conversion & Word class & Direction & Word class\\
\midrule
Change in & Verb & \phantom{←  }→ & Noun\\
word class & Predicate adjective & \phantom{←  }→ & \textit{\textup{Verb}}\\
& \textit{\textup{Verb (property concept)}} & ←  → & Attributive adjective\\
& Noun & \phantom{←  }→ & \textit{\textup{Adverbial}}\\

\tablevspace
No change in & Inchoative-stative verb & ←  → & Dynamic verb\\
word class & Noun & ←  → & Modifier noun\\
\lspbottomrule
\end{tabularx}}
\end{table}

Verbs may be employed in the syntactic position of nouns. This process of conversion is very productive. The meanings of such nominalised verbs vary in accordance with their lexical aspect. A dynamic verb used as a noun denotes the nominalised activity, while an (inchoative-)stative verb used in such a way denotes the corresponding nominalised state. 


In \REF{ex:key:104}, the dynamic verb \textit{hála} ‘shout’ is used as a dynamic noun or “action nominal”, (\citealt{ComrieThompson1985}). In \REF{ex:key:105}, the inchoative-stative verb \textit{gúd} ‘be good’ is employed as a stative noun or “state nominal” (\citealt{ComrieThompson1985}). The use of nominalised verbs as cognate objects \is{cognate objects}is common for emphasis (cf. \sectref{sec:9.3.3}). Cognate objects\is{objects} behave no differently from other nominalised verbs:



\ea%104
    \label{ex:key:104}
    \gll E    sé    frɔn    \textbf{dán}    \textbf{hála}    dí  pikín  nó  slíp    mɔ́.\\
\textsc{3sg.sbj}  \textsc{quot}    from  that    shout  this  child  \textsc{neg}  sleep  again\\

\glt ‘She said from that shout(ing) onwards this child didn’t sleep anymore.’ [ab03ab 075]
\z


\ea%105
    \label{ex:key:105}
    \gll \'{A}fta    ínsay  \textbf{dán}    \textbf{gúd}    wé  a    trata  yú    na  dé
mi    mán    go  chɛ́k  sé    mi    rabia  dɔ́n  fínis.\\
then  inside  that    good  \textsc{sub}  \textsc{1sg.sbj}  treat  \textsc{2sg.indp}  \textsc{foc}  there
\textsc{1sg.poss}  man    \textsc{pot}  think  \textsc{quot}    \textsc{1sg.poss}  anger  \textsc{prf}  finish\\
\glt ‘Then through that goodness that I treated you with, that’s where my 
husband would think that my anger has finished.’ [ro05rr 003]
\z

A verb can also appear in the nominal position together with its object, although this is rarely heard in natural speech: 


\ea%106
    \label{ex:key:106}
    \gll Na  \textbf{di}  \textbf{wás}    \textbf{klós},  na  di  tín    mék    yu  táya.\\
\textsc{foc}  \textsc{def}  wash  clothing  \textsc{foc}  \textsc{def}  thing  make  \textsc{2sg}  be.tired\\

\glt ‘It’s the washing of clothing, that’s why you’re tired.’ [dj05be 039]
\z

In contrast, very few nouns are attested in the syntactic position of verbs. The noun \textit{bɛlɛ́} ‘belly’ \REF{ex:key:107} may be used as a verb with the meaning ‘impregnate’ \REF{ex:key:108}. Other noun-verb pairs in the corpus that may be employed in a similar way are \textit{kaká} ‘defecate, faeces’, \textit{pipí} ‘urinate, urine’, \textit{rút} ‘root, uproot’, \textit{latrín} ‘toilet, go to toilet’. These rare cases are not listed in \tabref{tab:key:4.3} because they are lexicalised, and there is hence no productive noun-verb conversion. 


\ea%107
    \label{ex:key:107}
    \gll Tidé    pikín,  yu  go  gɛ́t  \textbf{bɛlɛ́}    yu  púl=an
yu  go  dáy  wet    \textbf{bɛlɛ́}.\\
today  child  \textsc{2sg}  \textsc{pot}  get  belly  \textsc{2sg}  remove=\textsc{3sg.obj}
\textsc{2sg}  \textsc{pot}  die  with    belly\\

\glt ‘(As for) children of today, you could get pregnant and remove it 
and you could die due to pregnancy.’ [ab03ay 105]
\z


\ea%108
    \label{ex:key:108}
    \gll A    fía  sé    dɛn  go  \textbf{bɛlɛ́}      mi    pikín  fɔ  mí.\\
  \textsc{1sg.sbj}  fear  \textsc{quot}    \textsc{3pl}  \textsc{pot}  impregnate  \textsc{1sg.poss}  child  \textsc{prep}  \textsc{1sg.indp}\\
\glt ‘I feared that they could impregnate me my child.’ [dj05be 055]\is{derivation!conversion}
\z

Other word class\is{word classes}es are also characterised by multifunctionality. Members of the small adjective class of Pichi may be used as inchoative-stative verbs without a change in form (cf. \sectref{sec:7.6.5}). Property items, whether adjectives or verbs, may be employed as attributive adjectives (i.e. prenominal modifiers, cf. \sectref{sec:5.2.1}), and nouns may modify other nouns in associative constructions without an overt process of derivation (cf. \sectref{sec:4.4.2}). Further, labile verbs\is{labile verbs} may be used in their respective lexical aspect classes without any formal change (cf. \sectref{sec:9.2.3}). Such multifunctionality with respect to lexical aspect class is very productive. It is lexically restricted to the class of labile verbs, which constitutes a large verb class in Pichi. Aside from that, members of the small class of adverbs are not usually employed as nouns or verbs.\is{conversion} 

\section{Compounding}\label{sec:4.4}

Pichi makes extensive use of compounding in order to derive nouns, verbs, and personal pronouns. Compound words are formed by combining two, sometimes more lexical items. Most types of compounding are covered in \sectref{sec:4.4}. Reduplication, which also involves compounding, is covered separately in section \sectref{sec:4.5.1}. Aspects of the morphophonology of compounding are covered in \sectref{sec:3.2.4}.\is{derivation}

\subsection{General characteristics}

Compounding forms part of a continuum of possessive constructions or relations of modification between constituents (cf. also \sectref{sec:5.2.3}). I only refer to those possessive constructions as “compounds” which form single phonological words via the tonal derivation described in \sectref{sec:3.2.4}. I nevertheless use the term “compounding” as a generic term to designate the formative processes that derive compounds associative constructions and \textit{fɔ}{}-constructions. Compounds relate in interesting ways to associative constructions and \textit{fɔ}{}-prepositional phrase constructions. The two latter types of possessive constructions are formed by syntactic concatenation alone. In the following, I refer to the individual lexical items occurring in these three types of possessive constructions as “components”. \tabref{tab:key:4.4} provides an overview of relevant characteristics of the three types of compounding: 

%%please move \begin{table} just above \begin{tabular
\begin{table}
\caption{Characteristics of compounding}
\label{tab:key:4.4}
\fittable{
\begin{tabular}{llll}
\lsptoprule
Features & Compounds & Associative constructions & \textit{Fɔ}{}-construction\\
\midrule
Morphosyntax & Tonal derivation & Syntactic concatenation & Syntactic concatenation\\
Productivity & Medium & Medium & High \\
Lexicalisation & High & Medium & Low\\
\lspbottomrule
\end{tabular}
}
\end{table}

Phonological and semantic factors determine the choice between compounding and the use of associative constructions for word formation. Speakers may opt to use a compound when the relevant concepts are commonly associated with each other, and the entire structure is conventionalised or lexicalised. In contemporary Pichi, there is no formal difference between compounds that may have been carried over from English (e.g. \textit{pan-kék} ‘pancake’, \textit{ren-sísin} ‘rain(y) season’) and language-internal formations (e.g. \textit{kɔntri-chɔ́p} ‘local food’). The meanings of both groups may be more compositional or more idiosyncratic, and both undergo the same tonal derivation characteristic of compounding:

\eabox{\label{ex:key:109}
\begin{tabularx}{.9\textwidth}{XXX} 
         Compound & Components & Gloss\\
\itshape kɔntri-chɔ́p & country-food & ‘local food’\\
\itshape kichin-písis & kitchen-cloth & ‘kitchen rag’\\
\itshape waka-stík & walk-stick & ‘walking stick’\\
\itshape ren-sísin & rain-season & ‘rainy season’\\
\itshape pan-kék & \textit{\textup{pan-cake}} & ‘pancake’\\ 
\end{tabularx}
}

Some semantically opaque compounds also exist, in which one component has no independent meaning (\ref{ex:key:110}a) or where one component is obsolete (b). It is noteworthy that the initial components of the first two compounds below exhibit a regular sound-meaning relation with the verbs \textit{spót} ‘be stylish’ and \textit{lúk} ‘look’, respectively, although there is no nominalising suffix \textit{*-in} in Pichi. However, there is one verb-noun pair in the corpus, in which the noun (\textit{bɛ́rin} ‘burial’) is the action nominal to a verb (\textit{bɛ́r} ‘bury’). The compound in (c) is therefore transparent and fully segmentable. Opaque and exocentric compounds are written without a hyphen in this work and their components are separated by a dot where relevant: 

\eabox{\label{ex:key:110}
\begin{tabularx}{.9\textwidth}{lXXX}
   & Compound & Components & Gloss\\
a. & \itshape spotin.bɔ́y & \textstyleTablePichiZchn{*spotin}.boy & ‘stylish guy’\\
   & \itshape lukin.glás & \textstyleTablePichiZchn{*lukin}.glass & ‘mirror’\\
   & \itshape kobo.fút & \textstyleTablePichiZchn{*kobo}.foot & ‘bowlegs’\\
\tablevspace   
b. & \itshape faya-wúd & \itshape \textup{fire}{}-?wúd & ‘firewood’\\
\tablevspace   
c. & \itshape bɛrin-grɔ́n & burial-ground & ‘burial ground’\\
\end{tabularx}
}

Other collocations are also partially opaque but exhibit the prosodic characteristics of either associative constructions or compounds. These are  structures that have inherited varying degrees of semantic opacity and lexicalisation from English, cf. (\ref{ex:key:111}–\ref{ex:key:113}). In the compounds in \REF{ex:key:111}, both components before and after the dot retain their original pitch configurations. In collocations involving the generic noun \textit{dé} ‘day’ as a modified noun, the “modifier” has no meaning of its own:

\eabox{\label{ex:key:111}
\begin{tabularx}{.9\textwidth}{XXX}
Compound & Components & Gloss\\
\itshape hɔ́li.dé & *\textstyleTablePichiZchn{hɔ́li}.\textstyleTableEnglishZchn{day} & ‘holiday’\\
\itshape yɛ́sta.dé & *\textit{yɛ́sta}.day & ‘yesterday’\\
\itshape sáti.dé & *\textstyleTablePichiZchn{sáti}.\textstyleTableEnglishZchn{day} & ‘Saturday’\\
\end{tabularx}
}
The structure of two sets of kinship terms is also of interest. The root \textit{gran-} ‘grand-’ is segmentable and has a discernible meaning. However, the root is never found independently of the word it modifies. It only appears in compounds (\ref{ex:key:112}a), which can, in turn, be preceded by the prenominal modifier \textit{grét} ‘great’ (b):

\eabox{\label{ex:key:112}
\begin{tabularx}{.9\textwidth}{lXXX}
         & Compound & Components & Gloss\\
a. & \itshape gran-mɔ́da & grand-mother & ‘grandmother’\\
& \itshape gran-má,~gran-mamá & grand-ma/-mother & ‘grandma/grandmother’\\
& \itshape gran-pá, gran-papá & grand-pa/-father & ‘grandpa/grandfather’\\
& \itshape gran-pikín & grand-child & ‘grandchild’\\
b. & \itshape grét gran-pikín & great grand-child & ‘great grandchild’\\
\end{tabularx}
}

The second set of kinship-denoting compounds contains the segmentable root \textit{lɔ́} ‘law’ as the final component. The composite meanings of these compounds are idiosyncratic. Additionally, some of the structures are fully segmentable, with the first component constituting an independent word (\ref{ex:key:113}a). Further, we find variants of group (a) compounds with slightly altered initial components (b). With these, the etymology is clear, but the altered initial component never occurs on its own. A final group contains an opaque initial element, which is a fossilised English morpheme that does not exist (any longer) in contemporary Pichi (c): 

\eabox{\label{ex:key:113}
\begin{tabularx}{.9\textwidth}{lXXX}
         & Compound & Components & Gloss\\
a. & \itshape mɔda-lɔ́ & mother-law & ‘mother-in-law’\\
& \itshape fada-lɔ́ & father-law & ‘father-in-law’\\
& \itshape brɔda-lɔ́ & brother-law & ‘brother-in-law’\\
& \itshape sista-lɔ́ & sister-law & ‘sister-in-law’\\
b. & \itshape mɔdɛ-lɔ́ & \textstyleTablePichiZchn{*mɔdɛ}.law & ‘mother-in-law’\\
& \itshape sistɛ-lɔ́ & \textstyleTablePichiZchn{*sistɛ}.law & ‘sister-in-law’\is{kinship terminology}\\
c. & \itshape dɔta.lɔ́ & \textstyleTablePichiZchn{*dɔta}.law & ‘daughter-in-law’\\
& \itshape sɔni.lɔ́ & \textstyleTablePichiZchn{*sɔni}.law & ‘son-in-law’\\
\end{tabularx}
}
In Spanish compounds and neologisms involving Spanish components (e.g. \textit{busca-blanco} ‘female sex worker specialised to white men’), the initial component(s) is/are always low-toned, while the final component bears H tone on the penultimate or only syllable \REF{ex:key:114}. This also holds for reduplicative compounds involving Spanish-derived dynamic verbs. The H tone is therefore found on the syllable that is stressed in standard Spanish. However, when these Spanish-derived compounds are employed in Pichi clauses, the H tone over the final component may not be shifted to other components of the compound for focus or emphasis (as the placement of stress may be in Spanish). This speaks for an analyisis of these collocations as Pichi-style compounds featuring the tonal derivation that other compounds have:

\eabox{\label{ex:key:114}
\begin{tabularx}{.9\textwidth}{lXlX}
Compound & Transcription & Components & Translation\\
\itshape vídeo-club & [vìdjò klúb] & video-club & {‘video\newline rental shop’}\\
\itshape busca-blanco & [bùskà-blánkò] & search-white.male & {‘female sex\newline worker spe-\newline cialised to \newline white men’}\\
\itshape tres mil & [trɛ̀s míl] & three thousand & {‘three thou-\newline sand’}\\
\itshape cuarenta y siete & {[kwàrɛ̀nta ì\newline sjétè]} & forty and seven & ‘forty-seven’\\
\itshape cruza-cruza & [krùsà-krúsá] & cross-cross & ‘cross repeat-\newline edly’\\
\end{tabularx}
}
Although in many cases conventionalisation is a good indicator for the use of compounding, phonology may override semantics. Compounds are shunned in favour of associative constructions where the first component belongs to the L.H tone class featuring a word-final H tone. We have seen that this tone class remains unaffected by other tonal and intonational processes as well (cf. e.g. \sectref{sec:3.4.1}). Hence the concepts in \REF{ex:key:115}, although conventionalised, are expressed as associative constructions, syntactic phrases consisting of prosodically independent components:

\eabox{\label{ex:key:115}
\begin{tabularx}{.9\textwidth}{lllX}
Ass. construction & Components & Gloss\\
\itshape bangá súp & palm-nut soup & ‘palm-nut soup’\\
\itshape dɔtí pán & dirt pan & ‘dustbin’\\
\itshape plantí fufú & plantain fufu & ‘fufu made from plantain’\\
\end{tabularx}
}
The tonal derivation characteristic of compounding also distinguishes lexicalised compound verbs (\ref{ex:key:116}a) from verb-object phrases (b) (cf. also \sectref{sec:4.4.3}):

\eabox{\label{ex:key:116}
\begin{tabularx}{.9\textwidth}{lllX}
         & Construction & Components & Gloss\\
a. & \itshape e opin.yáy & \textsc{3sg.obj} open.eye & ‘s/he is enlightened, cultivated’\\
b. & \itshape e ópin yáy & \textsc{3sg.sbj} open eye & ‘s/he opened (her) eye(s)’\\
\end{tabularx}
}

\subsection{Compound nouns}\label{sec:4.4.2}

Compound nouns function as nouns in a clause. Their final component is always a noun, while their initial component(s) may be a noun, verb, or an adverb. Compound nouns are the most common type of compound in the corpus. They instantiate a relation of modification, with the first component serving as the modifier and the second as the modified element. 


In a large number of collocations in the corpus, the modified noun is one of the generic nouns\is{generic nouns} listed in \REF{ex:key:117}, which serve other important functions in the language as well (cf. \citealt[252]{Faraclas1996}):


\eabox{\label{ex:key:117}
\begin{tabularx}{.9\textwidth}{llX}
Type & Generic noun & Gloss\\
 Human & \itshape mán & ‘man, person’\\
  & \itshape húman & ‘woman’\\
  & \itshape bɔ́y & ‘boy’\\
  & \itshape gɛ́l & ‘girl’\\
  & \itshape pikín & ‘child, member of group’\\
  & \itshape pɔ́sin & ‘person’\\
  & \itshape pípul & ‘people’\\
 Place & \itshape sáy & ‘side, place’\\
  & \itshape pát & ‘part, place’\\
  & \itshape plés & ‘place’\\
 Manner & \itshape stáyl & ‘style’\\
  & \itshape fásin & ‘manner’\\
 Time & \itshape tɛ́n & ‘time’\\
  & \itshape áwa & ‘hour, time’\\
 Entity & \itshape tín & ‘thing’\\
  & \itshape wán & ‘one’\\
  & \itshape káyn & ‘kind’\\
\end{tabularx}
}
The tendencies of nominal compounding are summarised in \tabref{tab:key:4.5}. The column “modifier/modified” in \tabref{tab:key:4.5} lists the types of modification relations attested in the data. I have added the third relevant possessive construction, the “\textit{fɔ}{}-construction” for comparison. The columns headed by “compound”, “associative construction”, and “\textit{fɔ}{}-construc\-tion” contain a cross ($\times$) if the structure is employed to express the corresponding relation in the leftmost column. A blank space indicates that the structure is not employed for this purpose. 

%%please move \begin{table} just above \begin{tabular
\begin{table}
\caption{Tendencies of nominal compounding}
\label{tab:key:4.5}
\fittable{
\begin{tabular}{lccc}
\lsptoprule
Modifier/modified & Compound & Associative construction & \textit{Fɔ}{}-construction\\
\midrule 
Group/member of &  & $\times$ & $\times$\\
\tablevspace
Gender of/creature &  & $\times$ & \\
\tablevspace
Measure/entity &  & $\times$ & \\
\tablevspace
Kind of/entity & $\times$ & $\times$ & $\times$\\
\tablevspace 
Activity/agent & $\times$ &  & \\
\lspbottomrule
\end{tabular}
}
\end{table}

Compounds, associative constructions, and \textit{fɔ}{}-prepositional constructions form part of a continuum of “possessive” constructions. In this continuum, associative constructions may express the widest range of modification relations, including most relations that may also be expressed as compounds and \textit{fɔ}{}-prepositional constructions (cf. also \sectref{sec:5.2.3}). \tabref{tab:key:4.5} shows that compound nouns are only used to express “kind of/entity” relations – the “activity/agent” relation being a subtype of the “kind of/entity” relation in which the first component is a dynamic verb and the second a human-denoting noun.


In turn, associative constructions represent the conventional means of expressing a measurement relation (referred to as “measure\slash entity” in \tabref{tab:key:4.5}), a “group\slash member of” relation featuring the modified noun \textit{pikín} ‘child’, and a “gender of\slash creature” relation featuring the gender nouns \textit{mamá} ‘mother’ and \textit{papá} ‘father’, \textit{mán} ‘man’ and \textit{húman} ‘woman’, or \textit{bɔ́y} ‘boy’ and \textit{gál} ‘girl’ in the modifier position. 



Secondly, associative constructions are the default option for expressing “kind of\slash entity” relations when these are not expressed as compounds. One criterion that determines the use of an associative construction as a default option is the nature of the modifier noun. Modifier nouns with an L.H pitch configuration and/or with more than two syllables are less likely to undergo the tone deletion that derives compound nouns. A second, subsidiary criterion is the lack of conventionalisation or lexicalisation of the collocation. In all other cases, “kind of\slash entity” relations, including “activity\slash agent” relations are usually expressed through compounds. Nevertheless, allowance must be made for numerous lexicalised exceptions to these tendencies.



In “kind of/entity” compounds, the first component modifies the second as to certain qualities. These compounds encompass bicomponental food items and dishes (\ref{ex:key:118}a) and body parts (b), as well as other concepts commonly associated with each other (c). Note that \textit{kaka-rás} ‘arse’ in (b) is a lexicalised compound and an exception to the tendency for collocations featuring an L.H modifier noun to be realised as associative constructions (the other most common exception being \textit{bɛlɛ́} ‘belly’ when used in the modifier position of a compound, cf. \ref{ex:key:124}). Compounds are also employed to form highly conventionalised quantifier compounds which express ordinal numerals{\fff} (d) as well as dual and \textit{ɔ́l} ‘all’ extensions of the pronominal system (e).



In sum, the use of “kind of/entity” compounds therefore reflects the degree of conventionalisation of the linkage between the participating nouns and in that a certain degree of inalienability\is{inalienability}:


\eabox{\label{ex:key:118}
\begin{tabularx}{.9\textwidth}{lXXX}
         & Compound & Components & Gloss\\
a. & \itshape pɛpɛ-súp & pepper-soup & ‘pepper soup’\\
& \itshape bwɛl-plantí & boil-plantain & ‘boiled plantain’\\
& \itshape bit-fufú & beat-fufu & ‘pounded fufu’\\
b. & \itshape finga-nél & finger-nail & ‘finger nail’\\
& \itshape kaka-rás & faeces-arse & ‘arse’\\
c. & \itshape hɔt-watá & hot-water & ‘hot water’\\
& \itshape kol-watá & cold-water & ‘cool water’\\
\end{tabularx}
}

\begin{exe}
\sn
\begin{tabularx}{.9\textwidth}{lXXX}
d. & \itshape nɔmba-tú & number-two & ‘second’\\
& \itshape nɔmba-trí & number-three & ‘third’\\
& \itshape las-nɛ́t & last-night & ‘last night’\\
& \itshape las-mán & last-man & ‘last person’\\
e. & \itshape wi-ɔl-tú & \textsc{1pl}{}-all-two & ‘the two of us’\\
& \itshape dɛn-ɔ́l & \textsc{3pl}{}-all & ‘they all’\\
\end{tabularx}
\end{exe}

Certain “kind of/entity” relations follow in \REF{ex:key:119} that are expressed through associative constructions rather than compounds. Group (a) features collocations, in which the modifier noun belongs to the L.H tone class. Here we also find some highly conventionalised collocations (b). The words in (c) contain associative constructions that involve trisyllabic modifier nouns from different tone classes. Other concepts are not sufficiently conventionalised or lexicalised to appear in compounds even if they present no formal obstacles (d). Also note the “kind of/entity” relations listed in \REF{ex:key:119}: 

\eabox{\label{ex:key:119}
\begin{tabularx}{.9\textwidth}{l llX}
         & Compound & Components & Gloss\\
a. & \itshape granát pamáyn & groundnut oil & ‘groundnut oil’\\
& \itshape Lubá topé & \textsc{place} palmwine & ‘Palmwine from Luba’\\
b. & \itshape dɔtí pán & dirt pan & ‘dustbin’\\
& \itshape plantí fufú & plantain fufu & ‘fufu made from plantain’\\
c. & \itshape kápinta wók & carpenter work & ‘work of a carpenter’\\
& \itshape wahála húman & trouble woman & ‘female trouble maker’\\
& \itshape aráta hól & rat hole & ‘rat hole’\\
& \itshape dominó stón & domino stone & ‘domino stone’\\
d. & \itshape Ghána mɔní & \textsc{place} money & ‘Ghanaian money’\\
& \itshape Píchi wɔ́d & Pichi word & ‘Pichi word’\\
& \itshape skúl plába & school problem & ‘problems related to school’\\
\end{tabularx}
}
Other “kind of/entity” relations are also expressed through associative constructions, although they do not present any phonotactic or semantic obstacles either. For example, the generic noun \textit{tɛ́n} ‘time’ is only recorded as a modified noun in the associative constructions listed in \REF{ex:key:120}, even though these structures are lexicalised and occur very frequently. Note, however, that other, lexicalised collocations involving \textit{tɛ́n} are not expressed as compounds either (cf. \ref{ex:key:103} above):

\eabox{\label{ex:key:120}
\begin{tabularx}{.9\textwidth}{XXX}
          Compound & Components & Gloss\\
\itshape mɔ́nin tɛ́n & morning time & ‘morning’\\
\itshape sán tɛ́n & sun time & ‘(after)noon’\\
\itshape ívin tɛ́n & evening time & ‘evening’\\
\end{tabularx}
}
Compounds involving \textit{sáy} ‘side, place’ are equally scarce. This noun is only attested as a modified noun in three compounds in the corpus, all of which have partially idiosyncratic meanings (\ref{ex:key:121}a). Other equally conventionalised collocations involving \textit{sáy} are expressed through associative constructions (b) or via \textit{fɔ}{}-prepositional constructions (c):

\eabox{\label{ex:key:121}
\begin{tabularx}{.9\textwidth}{lllX}
         & Compound & Components & Gloss\\
a. & \itshape wok-sáy & work-side & ‘work-place’\\
& \itshape rɔn-sáy & wrong-side & ‘inside out, upside-down, reverse’\\
& \itshape gud-sáy & good-side & ‘the right way round’\\
b. & \itshape ɔ́p sáy & up side & ‘(at the) upper part, up (there)’\\
& \itshape bihɛ́n sáy & behind side & ‘(at the) rear’\\
& \itshape dɔ́n sáy & down side & ‘(at the) lower part, down (there)’\\
c. & \itshape sáy fɔ chɔ́p & place \textsc{prep} eat & ‘eating place, restaurant’\\
& \itshape sáy fɔ wás & place \textsc{prep} wash & ‘place for washing, washhouse’\\
\end{tabularx}
}
“Group/member of” structures feature the human-denoting noun \textit{pikín} ‘child’ in the modified position. The conventional way of expressing this relation is through the associative construction. The modified noun \textit{pikín} may acquire quite an idiosyncratic meaning in the collocations listed under (\ref{ex:key:122}b). In these associative constructions, \textit{pikín} ‘child’ denotes a typical member of the group specified by the modifier noun rather than a kind of child (cf. \citealt[91–97]{ClaudiHünnemeyer1991}). For example, the construction \textit{Guinea pikín} is best translated as ‘person of Equatoguinean stock, typically Equatoguinean person’: 

\eabox{\label{ex:key:122}
\begin{tabularx}{.9\textwidth}{lllX}
         & Compound & Components & Gloss\\
a. & \itshape tidé pikín & today child & ‘child(ren) of today’\\
& \itshape gɔ́d pikín & God child & ‘child of God’\\
b. & \itshape Guinea pikín & \textsc{place} child & ‘person of Equatoguinean stock’\\
& \itshape gál pikín & girl child & ‘girl’ (but cf. also \ref{ex:key:123} below)\\
\end{tabularx}
}
“Gender of/creature” structures in which the modifier noun specifies the gender of a modified noun are also expressed as associative constructions. Compare the following collocations involving nouns with diverse pitch configurations: 

\eabox{\label{ex:key:123}
\begin{tabularx}{.9\textwidth}{llX}
          Compound & Components & Gloss\\
 \itshape bɔ́y pikín & boy child & ‘male child, son’\\
 \itshape gál pikín & girl child & ‘female child, daughter’\\
 \itshape húman fɔ́l & woman fowl & ‘hen’\\
 \itshape mán dɔ́g & man dog & ‘male dog’\\
 \itshape mamá Krió & mother Krio & ‘(elderly) Fernandino woman’\\
\end{tabularx}
}
The human-denoting nouns \textit{mán} ‘man, person’, \textit{húman} ‘woman’, \textit{pípul} ‘people’, and \textit{pɔ́sin} ‘person’ usually appear as modified nouns in compounds only \REF{ex:key:124}. The list also contains two compounds featuring \textit{bɛlɛ́} ‘belly’ as a modifier noun. \textit{Bɛlɛ́} and \textit{kaká} ‘faeces’ are the only attested nouns with an L.H pattern that are subjected to the tonal derivation characteristic of compounding. In the two compounds, the H tone over \textit{bɛlɛ́} has been deleted: 

\eabox{\label{ex:key:124}
\begin{tabularx}{.9\textwidth}{lllX}
         & Compound & Components & Gloss\\
a. & \itshape kɔntri-mán & country-man & ‘person from the same place of origin’\\
& \itshape layf-mán & life-man & ‘bon vivant’\\
& \itshape bɛlɛ-mán & belly-man & ‘pot-bellied man’\\
b. & \itshape bɛlɛ-húman & belly-woman & ‘pregnant woman’\\
& \itshape makit-húman & market-woman & ‘market-woman’\\
c. & \itshape yun-gɛ́l & young-girl & ‘(female) youngster’\\
& \itshape yun-bɔ́y & young-boy & ‘(male) youngster’\\
d. & \itshape jɛntri-pípul & riches-people & ‘rich people’\\
& \itshape ya-pípul & here-people & ‘people of this place’\\
& \itshape Ghana-pípul & \textsc{place}{}-people & ‘Ghanaians’\\
\end{tabularx}
}
The noun \textit{mán} ‘man’ is encountered in “activity/agent” compounds in which the first component is a dynamic verb with \textit{mán} instantiating the agent or “doer”. Such compounds are a subtype of the “kind of/entity” type of compound and serve to form agentive nouns as in the examples provided in \REF{ex:key:125}:

\eabox{\label{ex:key:125}
\begin{tabularx}{.9\textwidth}{lll}
          Compound & Components & Gloss\\
 \itshape fisin-mán & fish-man & ‘fisher’\\
 \itshape hɔnti-mán & hunt-man & ‘hunter’\\
 \itshape tif-mán & steal-man & ‘thief’\\
 \itshape chak-mán & get.drunk-man & ‘drunkard’\\
\end{tabularx}
}
Certain compounds involving \textit{mán} ‘man’ are neutral in their gender reference (\ref{ex:key:126}a) and equivalent to the far less common \textit{pɔ́sin} ‘person’ (b) in “activity/agent” compounds. However, \textit{mán} is also employed with the meaning ‘person’ in other contexts (e.g. \textit{na mán} ‘\textsc{foc} man’ = ‘that’s a human being’). Hence the gender-neutral use of \textit{mán} is not necessarily an indication of the generalisation of its function. In fact, \textit{húman} ‘woman’ always occurs as the “doer” when a female reference is desired (c) (cf. also \textit{mákit-húman} ‘market woman’ in \ref{ex:key:124} above). The generic noun \textit{mán} ‘man’ therefore falls short of functioning as an agentive suffix, in spite of its general, gender-neutral meaning in some contexts: 

\eabox{\label{ex:key:126}
\begin{tabularx}{.9\textwidth}{lllX}
         & Compound & Components & Gloss\\
a. & \itshape day-mán & die-man & ‘dead person, corpse’\\
b. & \itshape day-pɔ́sin & die-person & ‘dead person, corpse’\\
c. & \itshape day-húman & die-woman & ‘dead woman’\is{associative constructions}\is{compounds!nouns}\\
\end{tabularx}
}
\subsection{Compound verbs}\label{sec:4.4.3}

Three types of compounds may function as verbs in a clause: verb-verb reduplications, adverb-verb degree compounds, and verb-noun property compounds. The latter two are treated in this section; reduplication is extensively covered in section \sectref{sec:4.5.1}.


A verb may appear as the head of a compound featuring the multifunctional word \textit{óva} ‘over, be excessive, too much’ as the first component. The resulting compound verb expresses an excessive degree of the situation denoted by the verb. It is therefore normally formed with verbs denoting properties, such as \textit{dráy} ‘be dry, lean’ \REF{ex:key:127}, or verbs whose meaning contains an implicit gradation, such as \textit{dríng} ‘drink (alcohol)’ \REF{ex:key:128}. 



Such compounding is therefore an integral part of the Pichi system of comparison and emphasis\is{emphasis} (cf. \sectref{sec:6.9.1}). Other degree compounds found in the data are \textit{ova-stáwt} ‘be too corpulent’, \textit{ova-hɔ́t} ‘overheat, be too hot’, \textit{ova-klín} ‘clean excessively, be excessively clean’, and \textit{ova-fáyn} ‘be excessively beautiful’:



\ea%127
    \label{ex:key:127}
    \gll Dí  gɛ́l  pikín  \textbf{ova}-\textbf{dráy}    ó.\\
this  girl  child  over.\textsc{cpd}{}-be.dry  \textsc{sp}\\

\glt ‘This girl is really too lean.’ [dj07ae 207]
\z


\ea%128
    \label{ex:key:128}
    \gll \MakeUppercase{A}   \textbf{ova-dríng}.\\
\textsc{1sg.sbj}  over.\textsc{cpd}{}-drink\\

\glt ‘I drank too much.’ [au07ec 051]
\z

Many speakers do not accept degree compounds formed with verbs that are not property items. The alternative to the ungrammatical example \REF{ex:key:129} is provided in \REF{ex:key:130}: 

\ea[*]{%129
    \label{ex:key:129}
    \gll \MakeUppercase{A}   dɔ́n  \textbf{ova-blánt}   na  Panyá.\\
  \textsc{1sg.sbj}  \textsc{prf}  over\textsc{.cpd}{}-reside  \textsc{loc}  Spain\\
\glt Intended: ‘I have lived in Spain for too long.’ [au07ec 052]
}\z


\ea%130
    \label{ex:key:130}
    \gll \MakeUppercase{A}   dɔ́n  \textbf{tú}  \textbf{mɔ́ch}  \textbf{sté}    na  Panyá.\\
\textsc{1sg.sbj}  \textsc{prf}  too  much  stay    \textsc{loc}  Spain\\

\glt ‘I have stayed in Spain for too long.’ [au07ec 053]
\z

Equally, degree compounding is not accepted with a degree verb like \textit{bɔkú} ‘be much’ \REF{ex:key:131}. Instead, \textit{óva} may be employed as a degree verb on its own \REF{ex:key:132}:


\ea[*]{%131
    \label{ex:key:131}
    \gll Di  chɔ́p  \textbf{ova-bɔkú}.\\
  \textsc{def}  food    over\textsc{.cpd}{}-much\\
\glt Intended: ‘The food is too much.’ [au07ec 041]
}\z


\ea%132
    \label{ex:key:132}
    \gll Di  chɔ́p  \textbf{óva}.\\
\textsc{def}  food    be.over\\

\glt ‘The food is too much.’ [au07ec 042]\is{comparative degree}
\z

Property compounds are lexicalised compounds consisting of a property item and noun. Many of these compounds denote human propensities and emotions and involve a body part\index{} as the second component. The resulting structures are idiosyncratic and unpredictable in their meanings. Most property compounds are therefore exocentric. Consider \textit{bad-hát} ‘bad.\textsc{cpd}{}-heart’ = ‘be mean’ in \REF{ex:key:133}:


\ea%133
    \label{ex:key:133}
    \gll Dɛn  nó  lɛ́k  pɔ́sin,  dɛn  tú  \textbf{bad}-\textbf{hát}.\\
\textsc{3pl}  \textsc{neg}  like  person  \textsc{3pl}  too  bad.\textsc{cpd}{}-heart\\

\glt ‘They don’t like people, they’re too mean.’ [ma03hm 012]\is{derivation!verbs}
\z

Other compounds of this type are \textit{trɔn-yés} ‘strong.\textsc{cpd}{}-ear’ = ‘be disobedient’, \textit{trɔn-héd} ‘strong.\textsc{cpd}{}-head’ = ‘be stubborn’, \textit{gud-hát} ‘good.\textsc{cpd}{}-heart’ = ‘be good hearted’, \textit{brok-hát} ‘break.\textsc{cpd}{}-heart’ = ‘be broken-hearted’, and \textit{opin-yáy} ‘open.\textsc{cpd}{}-eye’ = ‘be enlightened, cultivated’ (cf. \ref{ex:key:116} above).


 There are also some semantically transparent endocentric compounds in the corpus involving dynamic verbs  that nevertheless denote properties. Compare the nominalised compound verb \textit{chɔp-mɔní} ‘eat.\textsc{cpd}{}-money’ = ‘expensive’ in \REF{ex:key:134}:\is{degree modification}



\ea%134
    \label{ex:key:134}
    \gll Dán    sáy,    na  \textbf{chɔp-mɔní}.\\
that    side    \textsc{foc}  eat.\textsc{cpd}{}-money\\

\glt ‘That place, it’s expensive.’ [ro07fn 203]
\z

\section{Iteration}\label{sec:4.5}

This section describes structures that involve the full iteration of a word. There are two distinct types of iteration in Pichi. Reduplication\is{reduplication} involves a morphological operation in addition to iteration, namely the tonal derivation also used in compounding (cf. \sectref{sec:3.2.4}). Repetition\is{repetition} involves iteration alone, and is therefore limited to syntactic concatenation. Reduplication is only employed with dynamic verbs and expresses various meanings associated with verbal number. Repetition is attested with a wider range of word class\is{word classes}es than reduplication and produces distributive, emphatic, and intensifying meanings \citep{Yakpo2012}. 


A limited number of Pichi words consist of identical components that cannot be separated and used on their own. Such unsegmentable, lexicalised iterations are found in various word classes, including ideophones. In spite of the formal differences between them, reduplication and repetition are characterised by a functional overlap. Both types of iteration are associated with quantification. \tabref{tab:key:4.6} summarises relevant features of the two types of iteration in Pichi.


%%please move \begin{table} just above \begin{tabular
\begin{table}
\caption{Types of iteration}
\label{tab:key:4.6}
\small
\begin{tabularx}{\textwidth}{lQp{3.5cm}}
\lsptoprule

Features & Reduplication\is{reduplication} & Repetition\is{repetition}\\
\midrule
Morphosyntactic process & Iteration + tonal derivation & Iteration\\
\tablevspace
Word classes & Dynamic verbs & Any lexical word class\\
\tablevspace
Phonological domain & Lexical word & {(Phonological) word,\newline phrase}\\
\tablevspace
Meanings & Verbal number: Iterative aspect \& dispersive readings & Intensity and emphasis; lexicalisation\\
\tablevspace
Number of iterations & Duplication & Duplication, triplication and more\\
\lspbottomrule
\end{tabularx}
\end{table}
\subsection{Reduplication}\label{sec:4.5.1}

As a productive derivational process, reduplication is only attested with dynamic verbs. However, the pattern is also found in a few lexicalised iterations involving nouns (cf. \sectref{sec:4.5.3}). Reduplication involves a complex morphological process consisting of the two distinct and simultaneous processes of iteration and tonal derivation. In the process, the verb is reduplicated, and the high tone over the first, reduplicated component is deleted and replaced by an L tone.


Therefore, this kind of reduplication is formally no different from compounding, except that the first component is a copy of the root; hence it involves “self-compounding” \citep[6]{Downing2001} (cf. \sectref{sec:3.2.4} for a detailed treatment of the pitch-related aspects of reduplication). The application of the morphological process of tone deletion to the first component of the reduplicated verb suggests that Pichi reduplications, like compounds, are right-headed (cf. \citealt[117]{Odden1996}).



Reduplication modifies the meaning of the verb root. The reduplicated verb may therefore appear in any syntactic position that a non-reduplicated verb may be found in. In \REF{ex:key:135}, a reduplicated \textit{wáka} ‘walk’ appears as a V2 in an SVC. Sentence \REF{ex:key:136} features a reduplicated \textit{rɔ́n} ‘run’ as a nominalised verb preceded by the demonstrative \textit{dí} ‘this’: 

\ea%135
    \label{ex:key:135}
    \gll Yɛ́stadé    wi  kán  gó  \textbf{waka-wáka}  mɔ́.\\
yesterday  \textsc{1pl}  \textsc{pfv}  go  \textsc{red}.\textsc{cpd}{}-walk  more\\
\glt ‘Yesterday we went walking around again.’ [ye 07fn 044]\\

\z

\ea%136
    \label{ex:key:136}
    \gll Pero    dí  \textbf{rɔn-rɔ́n}    nó  de  gí  nó  nátín  dé.\\
but    this  \textsc{red.cpd}{}-run  \textsc{neg}  \textsc{ipfv}  give  \textsc{neg}  nothing  there\\

\glt ‘But this running about aimlessly does not lead anywhere there.’ [dj07re 016]
\z

In the same vein, reduplication may be applied to a complement verb irrespective of its reduced finiteness:


\ea%137
    \label{ex:key:137}
    \gll Kán  tót    bɛlɛ́,    bigín  de  \textbf{hala-hála},    náw    \textbf{hala-hála}.\\
\textsc{pfv}  carry  belly  begin  \textsc{ipfv}  \textsc{red.cpd-}shout    now    \textsc{red.cpd-}shout\\

\glt ‘(Then she) became pregnant, (and) began lamenting and lamenting.’ [ab03ay 118]
\z

Reduplication expresses verbal number. The range of meanings associated with verbal reduplication spans the semantically close notions of iterative aspect, dispersive, distributive, low intensity, and casualness. A befitting cover term for these functions therefore is “temporal and/or spatial disaggregation”. Reduplication also often co-occurs with several nominal participants. Pichi reduplication is “event-internal” \citep[238]{Cusic1981}; it denotes the reiteration of a single event on a single occasion, consisting of repeated internal phases. Therefore reduplication does not express habitual aspect and is only found with dynamic verbs (cf. \sectref{sec:6.3.6} for details on the expression of iterative aspect). 


The iterative notion expressed by reduplication harmonises with the meanings expressed by imperfective aspect. There is a much stronger tendency for reduplicated predicates to co-occur with the imperfective aspect marker \textit{de} ‘\textsc{ipfv}’ than with any other TMA marker. The presence of the imperfective marker and the reduplicated verb \textit{rɔ́b} ‘rub’ in \REF{ex:key:138}. Since the unmarked reduplicated verb acquires a factative reading (hence past and perfective) by default, the presence of \textit{de} ‘\textsc{ipfv’} provides an imperfective sense to the clause:



\ea%138
    \label{ex:key:138}
    \gll Na  ús=káyn  tín    mék    yu  \textbf{de}   \textbf{rɔb-rɔ́b}    yu  sɛ́f  nía  mí
bifó    mi    fámbul?\\
\textsc{foc}  \textsc{q}=kind  thing  make  \textsc{2sg}  \textsc{ipfv}  \textsc{red}.\textsc{cpd-}rub  \textsc{2sg}  self  near  \textsc{1sg.indp}
before  \textsc{1sg.poss}  family\\

\glt ‘Why are you constantly rubbing yourself up to me [getting all cosy with me] in front of my family?’ [ge07fn 129]
\z

Further, iterative reduplication is also attested with the potential mood marker \textit{go} ‘\textsc{pot’}, as in the following example, and the habitual marker \textit{kin} (cf. \ref{ex:key:142}):


\ea%139
    \label{ex:key:139}
    \gll \MakeUppercase{A}   nó  wánt  nó  nátín  wé  \textbf{go}  \textbf{tayt-táyt}      mi    skín.\\
\textsc{1sg.sbj}  \textsc{neg}  want  \textsc{neg}  nothing  \textsc{sub}  \textsc{pot}  \textsc{red.cpd-}tighten  \textsc{1sg.poss}  body\\

\glt ‘I don’t want anything [clothes] that would be too tight for me (in various places).’ [ra07fn 045]
\z

Further, the interaction of verbal and nominal plurality often characterises the use of iterative aspect. The presence of plural referents generally induces a sense of iterative-distributive action of the situation denoted by the verb. For example, the light verb construction in \REF{ex:key:140} features the reduplicated nominalised verb \textit{jwɛ́n} ‘join’. The presence of the plural subject \textit{mí wet Rubi} ‘me and Rubi’, which is picked up by the resumptive pronoun{\fff} \textit{wi} ‘\textsc{1pl}’, induces a cumulative meaning of the reduplicated and deverbal noun \textit{jwɛ́n} ‘join’:


\ea%140
    \label{ex:key:140}
    \gll Mí    wet    Rubi    wi  \textbf{mék}    \textbf{jwɛn-jwɛ́n},  wi  báy  pía,
wi  báy  sadín,  wi  báy  tomates,    wi  desayuna.\\
\textsc{1sg.indp}  with    \textsc{name}  \textsc{1pl}  make  \textsc{red}.\textsc{cpd}{}-join  \textsc{1pl}  buy  avocado
\textsc{1pl}  buy  sardine  \textsc{1pl}  buy  tomatoes  \textsc{1pl}  breakfast\\
\glt ‘Me and Rubi, we joined up, we bought avocados, we bought sardines, we 
bought tomatoes, we had breakfast.’ [ye03cd 152]
\z

In turn, the presence of the plural object \textit{nɔ́mba dɛn} ‘numbers’ in the following sentence renders an iterative and distributive reading of the reduplicated verb \textit{chénch} ‘change’. 


\ea%141
    \label{ex:key:141}
    \gll Wétin  yu  de \textbf{  chench-chénch}  \textbf{nɔ́mba}  \textbf{dɛn}  só?\\
what  \textsc{2sg}  \textsc{ipfv}  {\textsc{red.cpd-}change}  number  \textsc{pl}  like.that\\

\glt ‘Why do you constantly change (telephone) numbers like that?’ [ye03cd 131]
\z

The iterative-distributive sense of the reduplicated verb is particularly evident in a reciprocal{\fff} construction like \REF{ex:key:142}. We have seen that a single form, the pronominal \textit{sɛ́f} ‘self, \textsc{emp}’ is employed as both the reflexive{\fff} and reciprocal anaphor. Hence there is room for ambiguity between the reflexive and reciprocal senses when a clause features a plural subject. One disambiguating feature amongst others is the presence of a reduplicated verb. There is no formal feature contained in \REF{ex:key:142} that would categorically force a reciprocal interpretation on the clause. But the use of reduplication, the presence of plural referents, and the meaning of the verb \textit{cháp} ‘chop’ and its instrument{\fff} object \textit{kɔ́tlas} ‘cutlass’ collude to induce a reciprocal rather than a reflexive meaning of the clause: 


\ea%142
    \label{ex:key:142}
    \gll Dɛn  kin  de  \textbf{chap-cháp}  dɛn  sɛ́f  \textbf{kɔ́tlas}  ó.\\
\textsc{3pl}  \textsc{hab}  \textsc{ipfv}  \textsc{red.cpd-}chop  \textsc{3pl}  self  cutlass  \textsc{sp}\\

\glt ‘(Mind you) they have the habit of chopping each other up with cutlasses
[referring to political violence in northern Nigeria].’ [ye07fn 239]
\z

Conversely, where there are no plural subjects or objects, the iterative meaning of the reduplicated verb shades off into the nuances of low intensity or casualness of the action denoted by the verb. Once again, it is the cumulative meaning of the various elements of the clause that tilts the balance towards this particular reading. 


In \REF{ex:key:143}, the intransitive use of the reduplicated verb \textit{tɔ́n} ‘turn’, in concert with the singular subject \textit{e} ‘\textsc{3sg.sbj}’, favours the related readings of low intensity or casualness. Further examples for these nuances are the reduplication of \textit{rɔ́b} ‘rub’ in \REF{ex:key:138} above, and of \textit{táyt} ‘tighten’ in \REF{ex:key:145} below. All these examples may also be seen to involve a nuance of lack of control by the subject:



\ea%143
    \label{ex:key:143}
    \gll E    sé    e    wánt  kán    \textbf{tɔn-tɔ́n} fɔ  Guinea.\\
\textsc{3sg.sbj}  \textsc{quot}    \textsc{3sg.sbj}  want  come  \textsc{red.cpd-}turn  \textsc{prep}  Equatorial.Guinea\\

\glt ‘He said he wanted to come move around a little in Equatorial Guinea.’ [ed03sb 190]
\z

The distribution of verbal reduplication in my corpus also suggests that it principally occurs in contexts of low transitivity, even if reduplication does not categorically function as a detransitivising device. Hence, preceding examples featuring reduplication for one part involve verbs characterised by a low transitivity, such as locomotion verbs (\textit{wáka} ‘wáka’, \textit{rɔ́n} ‘run’) and other verbs denoting body movement (\textit{tɔ́n} ‘turn, move around’, \textit{rɔ́b} ‘rub (oneself)’, as well as verbs of sound emission (\textit{hála} ‘shout’, \textit{kráy} ‘cry’) in intransitive clauses. 


Further, where reduplicated verbs (irrespective of their semantic class) do appear in transitive clauses, these clauses involve less prototypical transitivity, such as reflexive\is{reflexivity} and reciprocal\is{reciprocity} constructions, lexicalised verb-noun collocations (\textit{chénch} \textit{nɔ́mba} ‘change one’s telephone number’) or verbs followed by quantifier phrases like \textit{ɔ́l sáy} ‘all place’ = ‘everywhere’. The latter type of phrase is functionally equivalent to an adverbial indefinite and is therefore not a prototypical undergoer object either: 



\ea%144
    \label{ex:key:144}
    \gll Dɛn  de  \textbf{lɔk-lɔ́k}    ɔ́l  sáy.\\
\textsc{3pl}  \textsc{ipfv}  \textsc{red.cpd-}lock  all  side\\

\glt ‘They’re constantly closing every place.’ [pa07fn 467]
\z

Additionally, where reduplicated verbs with a higher transitivity occur, they are far more frequent in intransitive clauses. In the following sentence, the reduplicated Spanish-origin verb \textit{pica} ‘snip, cut up’ appears without a patient\is{patient} object: 


\ea%145
    \label{ex:key:145}
    \gll \MakeUppercase{A}   bigín  de \textbf{  pica-píca},    wi  fráy  patata,  wi  fráy  plantí.\\
\textsc{1sg.sbj}  begin  \textsc{ipfv}  \textsc{red.cpd-}cut.up  \textsc{1pl}  fry  potato  \textsc{1pl}  fry  plantain\\

\glt ‘I began to  (casually) snip (the trimmings), we fried potatoes, we fried plantain.’ [ye03cd.172]
\z

\subsection{Repetition}

Repetition in Pichi is a syntactic operation during which an item is duplicated or triplicated (more repetitions are not attested in the data). Although a pause or boundary tone is not normally inserted between the repeated elements, repetition does not involve the tonal process that characterises compounding and reduplication. Hence every repeated constituent retains its lexically determined tone pattern. Repetition involves syntactic concatenation. Normally, there is no pause or boundary tone between the repeated elements. Hence, the morphological operation characteristic of compounding and reduplication is not employed with this kind of iteration. Repetition is attested with a wider range of word class\is{word classes}es than reduplication. My data features repetition of nouns, verbs, attributively used property items, adverbs, and ideophones\is{ideophones}. 


Repetition produces a range of emphatic, intensifying nuances. The core meaning of repetition is augmentative, hence an iconic “more of the same”. However, the expression of plural number does not lie within the functional range of repetition. In the following three examples, we witness the use of intensifying repetition for emphasis with the temporal adverb \textit{náw} ‘now’ \REF{ex:key:146}, the locative noun \textit{dɔ́n} ‘down’ \REF{ex:key:147}, the common noun \textit{fámbul} ‘family’, and the attributively used property item \textit{bɔkú} ‘(be) much’ \REF{ex:key:148}: 



\ea%146
    \label{ex:key:146}
    \gll \MakeUppercase{A}   de  kɔmɔ́t    na  tɔ́n    \textbf{náw}    \textbf{náw}.\\
\textsc{1sg.sbj}  \textsc{ipfv}  come.out  \textsc{loc}  town  now    \textsc{rep}\\

\glt ‘I coming from town right now.’ [ro05ee 076]
\z


\ea%147
    \label{ex:key:147}
    \gll Bɔt  ín    sidɔ́n  \textbf{dɔ́n}    \textbf{dɔ́n}    \textbf{dɔ́n}    yandá.\\
but  \textsc{3sg.indp}  stay    down  \textsc{rep}    \textsc{rep}    yonder\\

\glt ‘But he stays far down over there.’ [ma03ni 026]
\z

\ea%148
    \label{ex:key:148}
    \gll Fɔ  mi    \textbf{fámbul}  \textbf{fámbul}  \textbf{fámbul}  a    nó  sabí  
\textbf{bɔkú}  \textbf{bɔkú}  pɔ́sin  dɛn.\\
\textsc{prep}  \textsc{1sg.poss}  family  \textsc{rep}    \textsc{rep}    \textsc{1sg.sbj}  \textsc{neg}  know
much  \textsc{rep}    person  \textsc{pl}\\

\glt ‘Within my immediate family I don’t know that many people.’ [fr03wt 031]
\z

The repetition of numerals renders a distributive sense. Clauses in which numerals are used with a distributive sense very often also feature plural nominal participants. In this example, the repetition \textit{tú tú} ‘two \textsc{rep}’ functions as a depictive adjunct and is oriented towards the plural object pronoun \textit{dɛ́n} ‘\textsc{3pl.indp}’:


\ea%149
    \label{ex:key:149}
    \gll Yu  fít  kɛ́r    dɛ́n    \textbf{tú}  \textbf{tú}.\\
\textsc{2sg} \textstylePichiexamplenumberZchnZchn{  can} \textstylePichiexamplenumberZchnZchn{carry} \textsc{3pl.indp} \textstylePichiexamplenumberZchnZchn{two} \textstylePichiexamplenumberZchnZchn{\textsc{rep}}\\
\glt ‘You can carry them in pairs.’ [\textstylePichiexamplenumberZchnZchn{bo07fn 231]}
\z

Numerals of \ili{Spanish} origin may be repeated for distributive meaning in the same way as Pichi numerals. Sentence \REF{ex:key:150} features the threefold repetition of the Spanish numeral \textit{quinientos} ‘five hundred’. It is worthy of note that repeating the numeral more than twice merely extends the distributive sense to additional participants rather than providing an additional emphatic nuance as with the repetition of members of other word class\is{word classes}es:


\ea%150
    \label{ex:key:150}
    \gll \textbf{Quinientos}  \textbf{quinientos}  \textbf{quinientos}.\\
five.hundred  \textsc{rep}      \textsc{rep}\\

\glt ‘Five hundred each.’ [hi03cb 058]
\z

The preceding examples have shown that various syntactic categories may be subjected to repetition. Nevertheless, the by far most commonly repeated categories are property items functioning as prenominal attributive modifiers like \textit{bɔkú} in \REF{ex:key:148} above, distributive numerals used as depictive modifiers like \textit{tú} ‘two’ in \REF{ex:key:149} above, and time expressions like \textit{náw} ‘now’ in \REF{ex:key:146} above. This distribution points towards the fact that repetition is strongly associated with gradable, quantity- and quality-denoting lexical items, as well as with distribution. 


The quantificational essence of repetition also transpires when it is applied to time expressions. The corpus contains numerous instances of repeated time expressions with an emphatic, quantificational meaning. The repetition of a temporal adverb like \textit{náw} ‘now’ \REF{ex:key:146} above or a temporal noun like \textit{mɔ́nin} ‘morning’ in the following sentence renders an intensive meaning ‘early in the morning, at dawn’: 



\ea%151
    \label{ex:key:151}
    \gll \'{A}fta    a    de  mít=an    nía    di  klós    dɛn
di  \textbf{mɔ́nin}  \textbf{mɔ́nin}  tɛ́n.\\
then  \textsc{1sg.sbj}  \textsc{ipfv}  meet=\textsc{3sg.obj}  near    \textsc{def}  clothing  \textsc{pl}
\textsc{def}  morning  \textsc{rep}    time\\

\glt ‘Then I ran into her by the clothes at dawn.’ [ru03wt 037]
\z

Other time expressions that allow some form of gradation are also frequently repeated in this way. For example the property item \textit{lɔ́n} ‘(be) long’ in the collocation \textit{lɔ́n tɛ́n} ‘long time ago’ is very often repeated in order to indicate a larger degree of time-depth: 


\ea%152
    \label{ex:key:152}
    \gll E    bin  dɔ́n  pás  lɔ́n    tɛ́n,    nóto  \textbf{lɔ́n}    \textbf{lɔ́n}  tɛ́n.\\
\textsc{3sg.sbj}  \textsc{ipfv}  \textsc{prf}  pass  long    time    \textsc{neg}.\textsc{foc}  long    \textsc{rep}  time\\

\glt ‘It happened long ago, not very long ago.’ [ma03sh 001]
\z

The repetition of time expressions involving the generic noun{\fff} \textit{tɛ́n} ‘time’ depends in form on the degree of semantic independence of the components of the collocation. When the collocation is endocentric, only the modifier element is reduplicated. In the following sentence, only \textit{wán} ‘one’ is therefore repeated rather than the entire expression \textit{wán tɛ́n} ‘once’. The same holds for \textit{lɔ́n tɛ́n} ‘long ago’ in the preceding example:


\ea%153
    \label{ex:key:153}
    \gll Na  \textbf{wán}    \textbf{wán}    tɛ́n    dásɔl.\\
\textsc{foc}  one    \textsc{rep}    time    only\\

\glt ‘It’s just once in a while.’ [fr03ft 053]
\z

In contrast, once the two words \textit{wán} and \textit{tɛ́n} are employed as part of the lexicalised expression \textit{wántɛn} ‘at once’, the entire collocation is repeated:


\ea%154
    \label{ex:key:154}
    \gll Na  wán    mán    wé  de  abraza  tú  húman  \textbf{wántɛn}  \textbf{wántɛn}  só.\\
\textsc{foc}  one    man    \textsc{sub}  \textsc{ipfv}  embrace  two  woman  at.once  \textsc{rep}    like.that\\

\glt ‘That’s a man embracing two women at once.’ [dj07re 038]
\z

Further, the repetition of periods of the day other than \textit{mɔ́nin (tɛ́n)} ‘morning (time)’ is not encountered in the data. Expressions like \textit{ívin tɛ́n} ‘evening’ or \textit{sán tɛ́n} ‘noon’ do not appear to lend themselves to some concept of quantification or gradation. This is possibly so because the corresponding period is of no cultural relevance, while ‘at dawn’ in \REF{ex:key:151} above is, since this is when people usually get up. Hence, for example, there is no instance of ?\textit{sán sán tɛ́n} with the intended reading ‘exactly at noon’. 


We are therefore once more dealing with a degree of lexical specialisation here. Such lexicalisation is also attested with other common repetitions. For example, the two dimension concepts \textit{bíg} ‘(be) big’ and \textit{smɔ́l} ‘(be) small’ are two of the most commonly encountered repeated property items in the corpus. Compare the following two examples: 



\ea%155
    \label{ex:key:155}
    \gll A    de  sí  \textbf{bíg}  \textbf{bíg}  fáya.\\
\textsc{1sg.sbj}  \textsc{ipfv}  see  big  \textsc{rep}  fire\\

\glt ‘I was seeing a huge fire.’ [ab03ay 067]
\z


\ea%156
    \label{ex:key:156}
    \gll E    de  sɛ́l  e    de  pút  \textbf{smɔ́l}  \textbf{smɔ́l}  wán  fɔ  kɔ́na.\\
\textsc{3sg.sbj}  \textsc{ipfv}  sell  \textsc{3sg.sbj}  \textsc{ipfv}  put  small  \textsc{rep}    one  \textsc{prep}  corner\\

\glt ‘She’s selling (and) she’s putting tiny ones [amounts] to the side.’ [hi03cb 220]
\z

In the rarer cases where verbs that function as predicates rather than prenominal modifiers are repeated, these are usually not property items. Property items are most commonly repeated when they precede a head noun as attributive modifiers; there is not a single instance of a repeated property item functioning as a predicate, e.g. ?\textit{e bíg bíg} ‘it is very big’. 


The meanings of repeated verbs are closely tied to their semantic structure. Hence, a verb like \textit{kɔ́t} ‘cut’ may imply a series of cyclic repetitions, particularly in the context of cooking as in \REF{ex:key:157}. The resulting meaning of the repetition is very close to that of iterative reduplication in an example like \REF{ex:key:145} above. Note that this verb is repeated together with its clitic object pronoun \textit{=an} ‘\textsc{3sg.obj’}: 



\ea%157
    \label{ex:key:157}
    \gll Di  dé  yu  bwɛ́l  jakató      yu  \textbf{kɔ́t}=an    \textbf{kɔ́t}=an
\textbf{kɔ́t}\textbf{\textmd{=an}}  yu  báy  wán    sardina\\
\textsc{def}  day  \textsc{2sg}  boil    bitter.tomato    \textsc{2sg}  cut=\textsc{3sg.obj}  \textsc{rep}
\textsc{rep}    \textsc{2sg}  buy  one    sardine.\\

\glt ‘The day you boil bitter tomato, you cut it up into small bits (and) you buy a sardine.’ [ro05rt 063]

\z

A similar case can be made for the repetition of the locomotion verb \textit{júmp} ‘jump’. This verb also naturally lends itself to a cyclical movement. In \REF{ex:key:158}, reduplication and the simultaneous use of repetition of the reduplicated sequence build up to an emphatic iterative sense with a cyclical meaning: 


\ea%158
    \label{ex:key:158}
    \gll Sɔntɛ́n  e    bin  de  \textbf{jump-júmp}    \textbf{jump-júmp},
pero  e    strét    náw.\\
perhaps  \textsc{3sg.sbj}  \textsc{pst}  \textsc{ipfv}  \textsc{red.cpd}{}-jump    \textsc{rep}
but    \textsc{3sg.sbj}  be.straight  now\\

\glt ‘Let’s assume she was constantly jumping around but she’s upright now.’ [ye07je 111]
\z

Two words in the corpus allow partial iteration. With the two inchoative-stative verbs and property items \textit{wɔwɔ́} ‘(be) ugly, messed up’ and \textit{lílí} ‘(be) little, tiny’, one syllable rather than the entire word may be iterated. Both words share the characteristic that they already constitute lexicalised iterations or at least appear so by their their segmental structure. Sentence \REF{ex:key:159} exemplifies the partial iteration of \textit{lílí} ‘(be) little’. A simplex word *\textit{lí} does not exist in Pichi. Since there is no sign of tone deletion over the first component of the iteration, I analyse \textit{lílí-lí} as an instance of partial repetition rather than reduplication: 


\ea%159
    \label{ex:key:159}
    \gll Pero    como  di  harina  tú  \textbf{lílí-lí},  kɔ́n    tú  smɔ́l  náw,
a    mezcla  ín    ɔ́l.\\
but    since  \textsc{def}  flour  too  little-\textsc{rep}  corn    too  be.small  now  
\textsc{1sg.sbj}  mix    \textsc{3sg.indp}  all\\

\glt ‘But since the flour is too little, the corn is too little now, I mixed all of it [in making the porridge].’ [dj03do 044]
\z


Now compare the fully \REF{ex:key:160} and partially iterated \REF{ex:key:161} alternatives for \textit{wɔwɔ́} ‘(be) ugly, messed up’. In both examples, the property item \textit{wɔwɔ́} is employed as a prenominal modifier. Note that a monosyllabic root \textit{*wɔ} does not exist in Pichi:


\ea%160
    \label{ex:key:160}
    \gll Na  Afrika  e    gɛ́t  \textbf{wɔwɔ́}  \textbf{wɔwɔ́}  tín    dɛn 
wé  a    nó  sabí.\\
\textsc{loc}  \textsc{place}  \textsc{3sg.sbj}  get  ugly    \textsc{rep}    thing  \textsc{pl}  
\textsc{sub}  \textsc{1sg.sbj}  \textsc{neg}  know\\

\glt ‘In Africa there are really messy things [happening] that I don’t know [how to explain].’ [ed03sb 187]

\z


\ea%161
    \label{ex:key:161}
    \gll Aa,  guineano  tú  dé    sɔn    ?\textbf{wɔ-wɔwɔ́}  stáyl.\\
\textsc{intj}  Guinean    too  \textsc{be.loc}  some  ?\textsc{red.cpd}{}-ugly  style\\

\glt ‘Guineans behave in a too messed up way.’ [ed03sp 055]
\z

The tonal characteristics of the partial iteration of \textit{wɔwɔ́} in \REF{ex:key:161} above are of interest. In the example, the original lexical H tone over the first syllable of the \textit{wɔ-wɔwɔ́} before the ligature has been replaced by an L tone. The presence of tone deletion points to the operation of partial reduplication rather than repetition. This contrasts with the iteration of other, attributively used property items in a similar way. In \REF{ex:key:155} and \REF{ex:key:156} above, \textit{bíg} and \textit{smɔ́l} undergo repetition, not reduplication. Although this example stands alone, it may be indicative of an area of transition between reduplication and repetition not only in meaning but also in form. 


There is often no sharp distinction in meaning between the repetition of single words and the iteration of larger chunks of a sentence. This is particularly so if the repeated elements are not separated from each other by a pause or declarative intonation (hence an utterance-final fall) as in the sentence below\is{declarative intonation}. The iteration of the NP \textit{in estómago} ‘her stomach’ in \REF{ex:key:162} conveys a repetitive and emphatic meaning in very much the same way as the verb-object phrase \textit{kɔ́t=an} ‘cut=\textsc{3sg.obj}’ in \REF{ex:key:157}: 



\ea%162
    \label{ex:key:162}
    \gll Nɔ́,  \textbf{in}    \textbf{estómago}  {\textbf{in}  \textbf{estómago}}  {\textbf{in}  \textbf{estómago}}.\\
\textsc{intj}  \textsc{3sg.poss}  stomach    \textsc{rep}        \textsc{rep}\\

\glt ‘[She would repeatedly say] No, (it’s) her stomach, her stomach, her stomach [rather than a pregnancy].’ [ab03ay 122]\is{repetition}
\z

\subsection{Lexicalised iteration}\label{sec:4.5.3}

A limited number of Pichi words consist of identical components that cannot be separated and used on their own. Such unsegmentable, lexicalised iterations are found in various word classes. An example follows featuring the ideophonic noun \textit{wuruwúrú} ‘confusion’. The (lexicalised) iteration of ideophones is covered in section \sectref{sec:12.1}. 


\ea%163
    \label{ex:key:163}
    \gll Dɛn  de  mék    \textbf{wuruwúrú}.\\
\textsc{3pl}  \textsc{ipfv}  make  confusion\\

\glt ‘They’re causing confusion.’ [be07fn 147]
\z

The pitch structure of lexicalised iteration is characterised by diversity. Some words feature a pitch configuration suggestive of reduplication, others feature a configuration that points towards repetition. The former group comprises cases of lexicalised iterations (\ref{ex:key:164}a) with no attested simplex form but whose etymology can be established. It also encompasses words with identical components, of which the origin of the simplex form is difficult or impossible to establish – these words are probably reflexes of English or Portuguese lexicalised iterations (b). The group also contains words which have a deducible, but idiosyncratic semantic relation with a simplex form (c). With all these words, we find an L tone over the first component of the word, while the second component bears an H tone. Hence this is the pitch configuration that we have already seen with iterative, verbal reduplication in section \sectref{sec:4.5.1}. The only difference is that \REF{ex:key:164} also includes nouns:

\eabox{\label{ex:key:164}
\begin{tabularx}{.9\textwidth}{llX}
          a. & \itshape bya.byá & ‘beard’\\
  & \itshape san.sán & ‘sand, soil’\\
  & \itshape was.wás & ‘wasp’\\
  & \itshape wɔ.wɔ́ & ‘be ugly, messed up’\\
 b. & \itshape ka.ká & ‘defecate, faeces’\\
  & \itshape ma.má & ‘mother’\\
  & \itshape pi.pí & ‘urinate, urine’\\
  & \itshape pa.pá & ‘father’\\
 c. & \itshape chuk.chúk & ‘thorn’ (< \textstyleTablePichiZchn{chúk} ‘pierce, sting’)\\
  & \itshape hayd.háyd & ‘secretely’ (< \textstyleTablePichiZchn{háyd} ‘hide’)\\
\end{tabularx}
}
