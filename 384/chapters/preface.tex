\addchap{Preface}

This book is directed at anyone interested in second language (L2) speech, be they student, teacher, linguist or lay reader. It summarizes the results of a study focused on the melodic patterns produced by Czech and German native speakers when they learn a foreign language, in this case Spanish or Italian.



The basis for this research stems from my interest in the sounds of languages and especially their melodies. From 2008 to 2011 I participated in a research project on the intonation of Buenos Aires Spanish, well-known for sounding “Italian”. In the project we examined the influence of the most important migration-induced contact language, Italian, on the prosodic system of this Argentinean variety of Spanish. This made me speculate about how Buenos Aires Spanish would sound today if -- instead of Italians -- a huge wave of, for instance, Czech or German immigrants had moved to Buenos Aires during the 19th and early 20th centuries. The question about how external and internal factors can be involved in language variation and change, particularly at the phonological level, constitutes the primary motivation for the present study.



A secondary motivation derives from my curiosity about foreign accents. I am fond of the anecdote that Roman Jakobson spoke six languages, but all of them in Russian. Everybody who learns a foreign language has probably faced the situation of being asked where s/he comes from because of his/her speech “sounding somehow different”. This perceived “otherness” and questions related to language learning and non-native speech have attracted the interest of not only researchers from very different fields, including linguistics, neurology, psychology, didactics, computer science and communication, but also laypersons. Although we never lose the ability to learn a new language, it is well known that speech in a foreign language deviates to one degree or another from native speech and that native-like pronunciation is very difficult~-- perhaps impossible~-- to achieve. The role of our native or first language’s sound system usually plays a crucial role here.



In comparison to research on the acquisition of segments (vowels and consonants), the research on L2 intonation -- the melodic pattern of an utterance -- is still limited. In fact, intonation is essential for communication because it conveys meaning in many ways. Differences in intonation can lead to a lot of misunderstanding. For example, a question ending in a low tone can be interpreted as a statement, a statement ending in a high target can be mistakenly understood as a question and the displacement of stress can radically change in meaning of the related word or sentence. Many pragmatic nuances such as obviousness, sarcasm, irony, politeness or surprise may completely be missed or misinterpreted if a listener is not properly attuned to intonational cues. Languages may differ greatly in their intonation patterns, and intonation may therefore be perceived differently across them. For example, the wider pitch excursion of tonal elements in English may sound excessive to Czech speakers, while by contrast the lower pitch range of Czech can signal boredom or disinterest to English speakers.



Finally, the choice of languages in this study deserves some clarification. First, as a native speaker of Czech it was natural that I would find it interesting to explore how the intonation of my own language might leave traces in a foreign language. At the same time, having been a resident of Germany for some time and thus having become familiar with German, I have observed that German intonation differs from Czech in various ways. Recalling my prior research experience with Spanish intonation and how it interacted with Italian intonation in Buenos Aires Spanish, an obvious direction to take in my research was to explore the intonational interface among these four languages, namely Czech, German, Italian and Spanish. Access to nearby language study contexts where Germans and Czechs were learning (one of) the two Romance languages provided me with an ideal laboratory. To the best of my knowledge there is no other study that focuses on this particular combination of languages.



In more general terms, based on a large production experiment, the present research aims to describe non-native intonational patterns in foreign languages and to identify the principles which govern the acquisition mechanisms of intonation. It is thus my hope to make a contribution to research on second language acquisition and second language speech in general. I also hope that, after reading this work, you will hear languages differently.


\bigskip
\hfill Andrea Pešková\\
\hbox{}\hfill Hamburg, August 2023\hbox{}
