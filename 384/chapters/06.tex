

\chapter{Conclusions}\label{ch:6}
\section{Summary and contribution}\label{sec:6.1}

The main objective of the present cross-sectional study was to shed new light on a still relatively unexplored area in non-native speech research: the production of second language intonation.


The approach embodied by this study was ground-breaking in several ways. First and foremost, the study investigated F0 patterns in Spanish and Italian as foreign languages acquired by Czech and German L1 speakers, a combination of languages that is entirely novel in the field of intonation acquisition. Another novelty is the application of multidirectional \textit{Contrastive Interlanguage Analysis} \citep{Granger1996} to a comparison of, on the one hand, one L2 (Spanish) as produced by learners with two different L1s (Czech and German), and, on the other, two different L2s (Spanish and Italian) as produced by learners with the same L1 (Czech). The value of this dual approach is that it allowed us to test hypotheses about the role of cross-linguistic influence in the acquisition of L2 intonation. The results showed that Czech learners of Spanish differed more from Czech learners of Italian in their ability to approximate the respective L2 intonation patterns than Czech learners differed from German learners in their ability to approximate Spanish intonation. A contrastive analysis showed that intonation is learnable and that learners of the same background \textit{notice} (in the sense used by \citealt{Schmidt1990}) target languages differently. This was made clear, as we saw in \sectref{sec:5.1}, by the fact that contrasts between the interlanguage varieties were found in different types of sentence, with different tonal events (i.e., pitch accents and boundary tones) and in different positions in sentence.



As for methodology, the corpus analysed here was obtained by means of an intonation questionnaire developed for the present study as a part of a large production experiment (\sectref{sec:3.1} and \sectref{sec:3.3}), whose methodology was based on the (\textit{Inter-}) \textit{Fonología del Español Contemporáneo} ((I)FEC) corpus project (\citealt{PustkaEtAl2016,PustkaEtAl2018}). The audio dataset comprising the corpus was elicited from 20 German and 20 Czech learners of Spanish and 20 Czech learners of Italian (\sectref{sec:3.2}). Half of these participants were intermediate and half advanced learners according to language abilities as categorized in the \textit{Common European Framework of Reference for Languages}. The justification for having the two competence levels was primarily to examine in which ways the two groups -- intermediate (B1--B2) vs. advanced learners (C1--C2) -- would differ from each other and to explain intonation development in an L2. I will come back to the results pertaining to this issue below. Moreover, six Spanish, six Italian, six Czech and six German native speakers served as control groups. With regard to the material selected for the final evaluation, 18 sentences per speaker were acoustically analysed with Praat, including neutral and biased statements, neutral and biased yes/no questions, neutral and biased wh-questions and neutral vocatives. The fact that the present study examined a variety of sentences is another important contribution to the field. For the tonal analysis, I chose phonetically ToBI-based labels (\sectref{sec:2.2} and \sectref{sec:3.4}), which provide simplified representations of F0 contours and which proved to be useful for systematically comparing the L2 patterns in the data. The results for the tonal realizations of all sentences in L2 Spanish and L2 Italian were presented and extensively discussed in \chapref{ch:4}.



This study provided an important empirical support for the second language acquisition theory (see, e.g., \citealt{TowellHawkins1994} and \sectref{sec:1.2}), several tenets of which were illustrated by the results. First, all L2 learners exhibited \textit{incomplete} acquisition, which is characterized by patterns transferred from their L1 and by cross-linguistic mixed patterns or patterns for which the specific type of CLI is difficult to pinpoint. Next, L2 varieties showed high, especially interlearner, variation. In order to explain this variation, I tested two factors, \textit{L1} \textit{Background} and \textit{Language Proficiency}. Whereas the learner’s L1 showed a statistically significant effect, proficiency did not reveal any statistical difference in spite of various contrasts between intermediate and advanced learners. This result (see \sectref{sec:5.4}) gives reason to speculate that intonation contours are already fossilized at the B1 or B2 level. Fossilization is understood here as a stagnation in L2 learning that can be either temporary or permanent. Such processes and factors involved in them have to be proved in future.



A second pillar of my theoretical orientation was \textit{L2 Intonation Learning theory} (LILt) \citep{Mennen2015}, a recent proposal that addresses the issue of L2 intonation in particular. This theory is based on the Autosegmental-Metrical model of intonational phonology (\citealt{Pierrehumbert1980}, see also \sectref{sec:2.3}) and its core assumption is that languages differ across four dimensions (systemic/phonological, realizational/phonetic, semantic/functional and frequency/recurrence) (see \sectref{sec:2.4}, \sectref{sec:5.3}, \sectref{sec:5.4}). These dimensions permitted us to formulate predictions and offered explanations for intonation contours in the L2s under study. During the thorough treatment of L2 patterns, we observed that intonation ``errors'' did not occur in all the four dimensions in the same way, with the phonetic and semantic dimensions representing the most challenging area of acquisition. With the goal of determining which group of learners was the most successful in L2 intonation acquisition, I established an \textit{accuracy} model (\sectref{sec:5.2.1}), according to which German learners were on average more accurate (62.8\%) than Czech learners of Spanish (58.8\%) and this latter group was more accurate than Czech learners of Italian (49.6\%). This is not very surprising, since German and Spanish are typologically closer (both are head-prominence languages) and share more tonal similarities (e.g., in yes/no questions, vocatives, focus marking and lexical prosody). Czech learners of Italian probably scored worse than Czech learners of Spanish because they had to deal with more unfamiliar patterns (e.g., L+H*+L) or new tonal combinations (e.g., H+L* LH\% in nuclear position). These findings regarding average accuracy should, however, be taken with caution: the model is based on an L1 “ideal” speaker and intended for guidance only.



The findings of this study have important implications for the teaching of intonation in the foreign language classroom, since they point to a need to increase learners’ awareness of intonational patterns, in both L2 and L1, as discussed in \sectref{sec:5.2.2}. These ideas are also exemplified by my proposal for how intonation skills can be implemented into the CEFR’s characterization of \textit{Phonology control} of the CEFR.



Finally, the most important contribution of this research to the field is the formulation of a \textit{Developmental L2 Intonation Hypothesis}, which provides a tentative answer to the question of whether there are any “universals” in the acquisition of L2 intonation. The hypothesis offers the following nine generalizations in this regard:


\ea%1
    \label{ex:1}
    Phonological features of intonation are acquired earlier than phonetic features of intonation.
    \z

For example, here we saw that some Czech learners had no particular difficulty with a high boundary tone in Spanish or Italian yes/no questions, but implemented the terminal patterns with different L1-based F0 contours (H\%, H!H\%, !H\%, LH\%).

\ea%2
    \label{ex:2}
    Pragmatically unmarked structures are acquired earlier than marked structures.
    \z

For example, both Czech and German learners showed more difficulty with the statements of the obvious and counterexpectational yes/no and wh-questions than with neutral sentences. The reason for this tendency may be lower pragmatic/semantic knowledge in the L2.

\ea%3
    \label{ex:3}
    Patterns that exist in both L1 and L2 are acquired earlier than new patterns provided that they convey the same meaning.
    \z

It was shown that non-native speakers more quickly acquire those patterns that are similar in the L1 and L2 and do not present any changes in the semantic dimension. For example, learners had no difficulties with a low boundary tone in statements; German learners were also very successful in focus marking or in vocatives in Spanish due to positive transfer.

\ea%4
    \label{ex:4}
  Patterns with a strong semantic weight are acquired earlier than patterns with no changes in meaning.
  \z

When an L1-based tonal contour leads easily to misinterpretation, learners will be forced to acquire the correct target pattern faster. For example,  \citet{MéndezSeijas2018} showed that some L1 English learners of Spanish stopped using \textit{uptalk} in statements after having been abroad for a certain period, presumably because \textit{uptalk} had led to misinterpretation (i.e., statements had been heard as questions). Méndez Seijas added that the same speakers did not change the alignment of prenuclear accents at all, presumably because it had no impact on meaning.

\ea%5
    \label{ex:5}
    Patterns that do not involve substantial changes in the semantic dimension fossilize faster.
    \z

This point is indirectly related to point \REF{ex:4} and posits that learners either need more time to acquire certain patterns or their learning stagnates, and fossilization occurs. This can be the case for certain pitch accents in the prenuclear position of statements (L*+H, L+H* and L+<H*), which differ in terms of alignment but do not change meaning. Another example is H+L*, which L2 Spanish and L2 Italian learners used in medial position of statements instead of the target patterns L+<H* (Spanish) or L+H* (Italian).

\ea%6
    \label{ex:6}
    Phonetically similar patterns that exist in both L1 and L2 fossilize faster than phonetically different patterns.
    \z

For example, Czech learners of Italian tended to fossilize a L*+H pattern in prenuclear position in statements because the L1 Czech accentual phrase (L* Ha) is phonetically similar to the L1 Italian L+H* pattern. In contrast, they did not fossilize the L1 Czech-based L*+H in the nuclear position of pragmatically marked statements: in this type of sentence, they were able to assimilate the Italian L+H*+L pattern, which is phonetically very different from the L1 Czech L*+H focus pattern.

\ea%7
    \label{ex:7}
    Patterns in functionally weaker positions fossilize faster than patterns in functionally stronger positions.
    \z

For example, pitch accents in medial position in statements showed the lowest accuracy in all interlanguage varieties. The related intonation “errors” were interpreted as a case of negative transfer and a result of the fact that medial pitch accents have a weaker impact on meaning than initial pitch accents or nuclear configurations.

\ea%8
    \label{ex:8}
    New but frequent and perceptually prominent patterns tend to be subject to overgeneralization.
    \z

We saw that Czech learners of Italian overgeneralized the (L+)H*+L nuclear pattern, which is characteristic of emphasis and focus in L1 Italian and occurs only in nuclear position. In the L2 data, it was detected in prenuclear positions in different types of sentences as well as in the nuclear position of neutral vocatives, where other patterns would have been expected. I assume that learners interpret this perceptually prominent pattern -- which does not exist in their L1 -- as a kind of “Italianish” feature.

\ea%9
    \label{ex:9}
    Rising boundary tones (as unmarked or “universal” forms) tend to be overgeneralized in all types of questions.
    \z

Not only the present study but also previous research revealed a preference for a H\% boundary tone in different types of question, even where the L1 required a falling pattern.

Needless to say, this set of hypotheses requires further consideration and corroboration in future research involving different interlanguage combinations and data.

\section{Limitations and directions for future research}\label{sec:6.2}

Although this research has brought us a step closer to an understanding of the phenomenon being studied and offers interesting findings, it shows several limitations and leaves open questions for future research.


One primary limitation is related to the amount of data (one token per sentence type) and the type of data analysed here. I selected a popular method used to investigate L1 intonation, an intonation questionnaire, which I adapted for the purposes of the present study (\sectref{sec:3.1}). The main advantage of the method was that it covered different types of sentences set in natural contexts and permitted a one-to-one comparison across the varieties under study. The analysis presented here covered so far a selection of the 25 sentence types the questionnaire elicits. In future I would like to follow up by examining the remaining sentence types and also include read and spontaneous material collected within the large production experiment. Different data of this sort would enrich the study inasmuch as they would make it possible to investigate other parameters such as prosodic phrasing, fluency, and the use of pauses. Moreover, future research should also control for number of words, number of syllables and stress placement. The control of the latter two parameters would be of particular interest in connection with L1 Czech learners. While German, Spanish and Italian are all prototypical intonation or head-prominence languages with variable stress, Czech is a language that differs from the other three typologically: it has a fixed stress on the first syllable and belongs to phrase or head/edge-prominence languages (\sectref{sec:2.3.2}), meaning that it assigns prominence with both pitch accents and boundaries at the (accentual) phrase level. Hence, the phonetic implementation of tonal cues may be more difficult for Czech learners.



The second limitation is linked with the L1 variety of the participants and the L2 variety they had been exposed to. It was not possible to ensure that all the participants in this project were speaking the same L1 variety and were learning or had learnt the same regional variety of L2. Most of the learners had had experience with more than one Spanish or Italian variety, for reasons that included different L2 teachers, different stays abroad or different Spanish- or Italian-speaking friends and contacts. Findings would be more robust if these factors could be controlled for. It also goes without saying that future work along these lines could include L2 Italian produced by L1 German learners, as well as other combinations of languages.



Another limitation of the present study is the fact that it analyses production data without taking perception data into account. We still do not know how learners of different backgrounds perceive the tonal patterns of target languages. Do German and Czech learners perceive Spanish and Italian intonation in the same way? Are intonation deviations in L2 caused by problems of perception, production or both? And we do not know how natives would judge and interpret foreign intonation in their L1. This also leaves the question open as to which parameters are involved in the perception of a foreign accent. What are the relative weights in the perception of “foreign accentedness” of errors in tonal alignment, slope, pitch range or duration play? Based on a pilot experiment, I speculate that intonation deviations are partly responsible for creating what is perceived as a foreign accent but hypothesise that they are not as negatively perceived as non-native segments (this holds at least for the present language scenarios). However, this area remains unexplored. Foreign accent rating experiments executed by native listeners would also be very helpful to verify the accuracy model I proposed in \sectref{sec:5.2.1} These should be followed up by foreign accent detection tasks or some type of discrimination experiments in order to understand which differences and parameters matter to native speakers. The results of the present study could be very helpful for framing such experiments.



The last issues that require special attention are associated with inter-learner and intra-learner variation. What sequence does the development of second language intonation follow in individuals? Do all L2 acquisitions proceed along the same path? Or does the developmental path differ depending on the type of sentence or type of tonal event? Previous research (see, e.g.,  \citealt{MéndezSeijas2018}) provides evidence that whereas some learners show a clearly linear path of development in their acquisition of L2 intonation patterns, others show stagnation in their learning. The present study can provisionally confirm this tendency since it has shown that some speakers show fossilization earlier than others. Moreover, it is conjectured that the acquisition of L2 intonation does not necessarily go hand in hand with the acquisition of segments and other areas of grammatical knowledge, another issue that deserves careful examination in future. Hence, further research should aim at collecting second language longitudinal data from individuals and compare these data with their first language data. Only in this way can we properly examine whether growth in L2 knowledge occurs systematically across learners.



Finally, one last limitation of this study consists of its focus on linguistic background (\sectref{sec:2.1.1}) and proficiency level (\sectref{sec:2.1.2}) in order to explain variation in data, with other possible factors only touched on marginally. Among these other factors, age of learning (\sectref{sec:2.1.3}), quality of input, length of time spent in an L2-speaking environment and formal instruction (\sectref{sec:2.1.4}), phonological awareness (\sectref{sec:2.1.5}) and a series of personal factors (\sectref{sec:2.1.6}, \sectref{sec:2.1.7}) related to general talent for pronunciation, music skills, mimicry or memory, are widely thought to play important roles in second language speech and therefore deserve further study. It is also still unclear how much control “learners themselves can exert over their non-native accents” \citep[146]{Cutler2014}.



Despite these limitations, it is my hope that the present study has offered insights into the acquisition of L2 intonation and will inspire not only future researchers in this field but also those who are devoted to language and pronunciation teaching.
