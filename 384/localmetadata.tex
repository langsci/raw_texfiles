\title{L2 Spanish and Italian intonation}
\subtitle{Accounting for the different patterns displayed by L1 Czech and German learners}
\BackBody{The main aim of this book is to contribute to our understanding of the acquisition of second language intonation, by comparing Czech learners of Spanish with German learners of Spanish and Czech learners of Italian. By means of a large production database, the study seeks to uncover how L1-to-L2 intonational transfer works and what role prosodic (dis)similarities between languages play. Contrary to most previous research, the work presents an original multidirectional cross-linguistic comparison and examines different types of sentence, such as neutral and non-neutral statements, yes/no questions, wh-questions, exclamatives and vocatives. The findings reveal positive and negative transfer from L1 to L2, and the formation of mixed patterns as well as native-like patterns, which are mainly constrained by linguistic factors such as the type of sentence and the position of the tonal event in the utterance. The results are discussed within Mennen’s (2015) \emph{L2 intonation learning theory} and lead to the formulation of a \emph{developmental L2 intonation hypothesis} that makes several generalizations to characterize interlanguage intonation. This volume not only represents a step forward in the study of the acquisition of L2 intonation in general but also offers valuable findings that can be directly or indirectly applied in the classroom and will hopefully inspire further research.}

\author{Andrea Pešková} 
 
\renewcommand{\lsID}{384}
\renewcommand{\lsISBNdigital}{978-3-96110-418-5}
\renewcommand{\lsISBNhardcover}{978-3-98554-076-1}
\BookDOI{10.5281/zenodo.8153997}
% \typesetter{}
\proofreader{Alexandr Rosen, Carla Bombi, Elliott Pearl, Jean Nitzke}
% \lsCoverTitleSizes{51.5pt}{17pt}% Font setting for the title page


\renewcommand{\lsSeries}{orl} 
\renewcommand{\lsSeriesNumber}{5} 
