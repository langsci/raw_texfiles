\newcommand{\keywords}[1]{\noindent\textbf{Keywords:} {#1}}

% % \sloppy
% % \pagenumbering{roman}

%non-italics in examples in footnotes

\renewcommand{\fnexfont}{\footnotesize\upshape}
\renewcommand{\fnglossfont}{\footnotesize\upshape}
\renewcommand{\fntransfont}{\footnotesize\upshape}
\renewcommand{\fnexnrfont}{\fnexfont\upshape}

% 01
\forestset{qtree/.style={for tree={parent anchor=south, child anchor=north,align=center,inner sep=0pt}}}

% 02

\forestset{
  nice nodes/.style={
  for tree={
  inner sep=1pt, s sep=12pt,
  fit=band,
},
},
% begin fairly nice empty nodes
fairly nice empty nodes/.style={
            delay={where content={}{shape=coordinate,for parent={
                  for children={anchor=north}}}{}}
},
% end fairly nice empty nodes
% begin pretty nice empty nodes
pretty nice empty nodes/.style={
    for tree={
      calign=fixed edge angles,
      parent anchor=children,
      delay={if content={}{
          inner sep=0pt,
          edge path={\noexpand\path [\forestoption{edge}] (!u.parent anchor) -- (.children)\forestoption{edge label};}
        }{}}
    },
  },
% end pretty nice empty nodes
default preamble={ 
nice nodes,
%nice empty nodes, % uncomment the one you want (and delete the ones you don't)
%fairly nice empty nodes,
%pretty nice empty nodes
}
}

% 03
\newcommand{\possessivecite}[1]{\citeauthor{#1}'s \citeyearpar{#1}}
\newcommand{\sx}[1]{$\llbracket${#1}$\rrbracket$}
\newcommand{\sembra}[1]{\ensuremath{\, [ \! [ }\mbox{#1}\ensuremath{] \! ] \,}}
\newcommand{\un}[1]{$_{\mbox{\scriptsize{#1}}}$} %for non-math subscripts
\newcommand{\cnst}[1]{\textsf{\small{\MakeUppercase{#1}}}}
\newcommand{\uncnstfn}[1]{$_{\mbox{\textsf{\miniscule{\MakeUppercase{#1}}}}}$}

% 04 

\newcommand{\uncnst}[1]{$_{\mbox{\textsf{\tiny{\MakeUppercase{#1}}}}}$}

\DeclareMathSymbol{\Alpha}{\mathalpha}{operators}{"41}
\DeclareMathSymbol{\Mu}{\mathalpha}{operators}{"4D}
\DeclareMathSymbol{\Chi}{\mathalpha}{operators}{"58}

% defining \rightlsquigarrow, \nrightlsquigarrow and \ngg from MnSymbol

\DeclareFontFamily{U} {MnSymbolA}{}
\DeclareFontShape{U}{MnSymbolA}{m}{n}{
  <-6> MnSymbolA5
  <6-7> MnSymbolA6
  <7-8> MnSymbolA7
  <8-9> MnSymbolA8
  <9-10> MnSymbolA9
  <10-12> MnSymbolA10
  <12-> MnSymbolA12}{}
\DeclareFontShape{U}{MnSymbolA}{b}{n}{
  <-6> MnSymbolA-Bold5
  <6-7> MnSymbolA-Bold6
  <7-8> MnSymbolA-Bold7
  <8-9> MnSymbolA-Bold8
  <9-10> MnSymbolA-Bold9
  <10-12> MnSymbolA-Bold10
  <12-> MnSymbolA-Bold12}{}
\DeclareSymbolFont{MnSyA} {U} {MnSymbolA}{m}{n}
\DeclareMathSymbol{\rightlsquigarrow}{\mathrel}{MnSyA}{160}

\DeclareFontFamily{U} {MnSymbolB}{}
\DeclareFontShape{U}{MnSymbolB}{m}{n}{
  <-6> MnSymbolB5
  <6-7> MnSymbolB6
  <7-8> MnSymbolB7
  <8-9> MnSymbolB8
  <9-10> MnSymbolB9
  <10-12> MnSymbolB10
  <12-> MnSymbolB12}{}
\DeclareFontShape{U}{MnSymbolB}{b}{n}{
  <-6> MnSymbolB-Bold5
  <6-7> MnSymbolB-Bold6
  <7-8> MnSymbolB-Bold7
  <8-9> MnSymbolB-Bold8
  <9-10> MnSymbolB-Bold9
  <10-12> MnSymbolB-Bold10
  <12-> MnSymbolB-Bold12}{}
\DeclareSymbolFont{MnSyB} {U} {MnSymbolB}{m}{n}
\DeclareMathSymbol{\nrightlsquigarrow}{\mathrel}{MnSyB}{160}


\DeclareFontFamily{U} {MnSymbolD}{}
\DeclareFontShape{U}{MnSymbolD}{m}{n}{
  <-6> MnSymbolD5
  <6-7> MnSymbolD6
  <7-8> MnSymbolD7
  <8-9> MnSymbolD8
  <9-10> MnSymbolD9
  <10-12> MnSymbolD10
  <12-> MnSymbolD12}{}
\DeclareFontShape{U}{MnSymbolD}{b}{n}{
  <-6> MnSymbolD-Dold5
  <6-7> MnSymbolD-Dold6
  <7-8> MnSymbolD-Dold7
  <8-9> MnSymbolD-Dold8
  <9-10> MnSymbolD-Dold9
  <10-12> MnSymbolD-Dold10
  <12-> MnSymbolD-Dold12}{}
\DeclareSymbolFont{MnSyD} {U} {MnSymbolD}{m}{n}
\DeclareMathSymbol{\ngg}{\mathrel}{MnSyD}{201}


% macros for citations:

\newcommand{\citeposst}[1]{\citeauthor{#1}'s (\citeyear{#1})} % produces Chomsky's (1995)
\newcommand{\citeposstpg}[2]{\citeauthor{#1}'s (\citeyear[#2]{#1})} % produces Chomsky's (1995: page)
\newcommand{\citepossalt}[1]{\citeauthor{#1}'s \citeyear{#1}} % produces Chomsky's 1995
\newcommand{\citepossaltpg}[2]{\citeauthor{#1}'s \citeyear[#2]{#1}} % produces Chomsky's 1995: page

% 05
\newcolumntype{P}[1]{>{\centering\arraybackslash}p{#1}}

% 13

\makeatletter
\ifcase \@ptsize \relax% 10pt
  \newcommand{\miniscule}{\@setfontsize\miniscule{4}{5}}% \tiny: 5/6
\or% 11pt
  \newcommand{\miniscule}{\@setfontsize\miniscule{5}{6}}% \tiny: 6/7
\or% 12pt
  \newcommand{\miniscule}{\@setfontsize\miniscule{5}{6}}% \tiny: 6/7
\fi
\makeatother

% 14
\newcommand{\evalfun}[2][]{\ensuremath{\left\llbracket \mbox{#2} \right\rrbracket^{#1}}}
\newcommand{\semdot}{\hspace{1pt}.\hspace{2pt}}
\newcommand{\smallcheck}{{\scriptsize \Checkmark}}
\renewcommand{\checkmark}{\ding{51}}
\newcommand{\sub}[1]{\textsubscript{#1}}

% 16
\newcommand{\nomm}{\textsc{nom}}
\newcommand{\accc}{\textsc{acc}}
\newcommand{\datt}{\textsc{dat}}
\newcommand{\genn}{\textsc{gen}}
\newcommand{\inst}{\textsc{inst}}
\newcommand{\locc}{\textsc{loc}}
\newcommand{\ergg}{\textsc{erg}}
\newcommand{\abss}{\textsc{abs}}
\newcommand{\msg}{\textsc{sg.m}}
\newcommand{\mpl}{\textsc{pl.m}}
\newcommand{\fsg}{\textsc{sg.f}}
\newcommand{\fpl}{\textsc{pl.f}}
\newcommand{\nsg}{\textsc{sg.n}}
\newcommand{\npl}{\textsc{pl.n}}
\newcommand{\lr}{[+lr]}
\newcommand{\hr}{[+hr]}
\newcommand{\nocase}{[{}{}{}]}
\newcommand{\littlev}{\textit{v}}
\newcommand{\up}{$\uparrow$\textsc{Agr}$\uparrow$}
\newcommand{\down}{$\downarrow$\textsc{Agr}$\downarrow$}
\newcommand{\before}{$\succ$}

% 17
\forestset{
  roof first-line-width/.style={
    split option={content}{\\}{roof first-line-width-a,gobble}
  },
  roof first-line-width-a/.style={
    TeX={\setbox0=\hbox{#1}},
    edge path'/.expanded={%
      ($(.parent)+(-\the\dimexpr 0.5\wd0\relax,0pt)$)
      --(!u.children)--
      ($(.parent)+(\the\dimexpr 0.5\wd0\relax,0pt)$)
      --cycle
    }
  },
  gobble/.style={},
}

% 19
\newcommand{\ins}{\textsc{ins}}
\newcommand{\fem}{\textsc{fem}}
\newcommand{\masc}{\textsc{masc}}
\newcommand{\sg}{\textsc{sg}}
\newcommand{\pl}{\textsc{pl}}

% 20
\newcommand{\semantictype}[1]{\langle{#1}\rangle} %puts stuff into semantic type brackets; necessary to use in the form $\st{}$
