\documentclass[output=paper]{langscibook}

\ChapterDOI{10.5281/zenodo.7777532}
\author{Caroline Wellmann\orcid{0000-0003-2797-0851}\affiliation{University of Potsdam}
and Julia Holzgrefe-Lang\orcid{0000-0001-7112-2943}\affiliation{University of Potsdam} and Hubert Truckenbrodt\affiliation{Leibniz-Zentrum Allgemeine Sprachwissenschaft (ZAS), Berlin}
and Isabell Wartenburger\orcid{0000-0001-5116-4441}\affiliation{University of Potsdam}
and Barbara Höhle\orcid{0000-0002-9240-6117}\affiliation{University of Potsdam}}

\lsConditionalSetupForPaper{}

\title[Developmental changes]
{Developmental changes in prosodic boundary cue perception in German-learning infants}

%                    ABSTRACT

\abstract{Previous investigations suggest that the main prosodic cues characterizing intonation phrase boundaries (IPBs), namely pitch change, final lengthening, and pause, have different weightings in perception. Such a weighting of IPB cues may be subject to crosslinguistic variation and seems to develop during the first year of life. For German, eight-month-old infants were found to detect an IPB signaled by pitch change combined with final lengthening in a behavioral discrimination task, even in the absence of a pause (\citealt{Wellmann2012}). Assessing the developmental course of prosodic boundary detection, the present study tested six-month-old German-learning infants with the same discrimination task in the headturn preference paradigm. Stimuli were presented in two different prosodic groupings, as a sequence either without or with an internal boundary after the second name, \textit{[Moni und Lilli und Manu]\textsubscript{IP}} vs. \textit{[Moni und Lilli]\textsubscript{IP} [und Manu]\textsubscript{IP}}. The internal IPB was systematically varied with respect to the amount and combination of cues. Infants were familiarized to sequences without an IPB and then tested on their discrimination of both prosodic groupings. We found that infants detected the boundary when it was cued by all cues (Exp. 1) and by pause and lengthening (Exp. 3). However, when the IPB was only marked by a combination of pitch and lengthening, they failed (Exp. 2), even when familiarization duration was doubled (Exp. 2a). This points to a crucial role of the pause cue at six months. Taken together with previous results, our data suggest a development towards an adult-like boundary perception that no longer requires the pause cue between six and eight months. We argue that this behavioral change reflects a shift of attention to boundary markings that are functionally relevant in the ambient language.}

\lehead{Caroline Wellmann et al.}
\begin{document}
\renewcommand{\lsChapterFooterSize}{\footnotesize}
\lehead{Caroline Wellmann et al.}
\maketitle
%                    INTRO

\section{Introduction}
During their first year of life infants pass through a phase of perceptual reorganization, in which their speech perception is sharpened for acoustic properties that are functional in the language they are exposed to but attenuates for acoustic properties that are not (for a review, see \citealt{Maurer2014}). Perceptual reorganization was initially shown for vowels and consonants, with numerous findings suggesting an increasing ability to discriminate native sound contrasts but decreasing performance with non-native sounds (for vowels, e.g., \citealt{Kuhl1992, Polka1994}; for consonants, e.g., \citealt{Best2003, Kuhl2006, Werker1984}). More recently, perceptual reorganization has also been shown for suprasegmental prosodic aspects of language like lexical tone (\citealt{Gotz2018, Mattock2006, Mattock2008, Yeung2013}) and lexical stress (\citealt{Bijeljac-Babic2012, Hohle2009, Jusczyk1993, Skoruppa2009}).

The present paper deals with a potential developmental change in a further area of prosody, namely in the perception of acoustic cues that mark major prosodic boundaries, specifically boundaries at the edges of Intonation Phrases (IP, \citealt{Nespor1986, Selkirk1986}). Three main cues are considered to mark these boundaries across languages: a change in fundamental frequency (F0) in terms of boundary tones or a pitch reset, a lengthening of preboundary segments, and the insertion of a pause (e.g., \citealt{Hirst1998, Nespor1986, Price1991, Vaissiere1983}).

The edges of IPBs usually coincide with syntactic clause boundaries (\citealt{Selkirk2005}; for German: \citealt{Truckenbrodt2005}), and infants as well as adults benefit from this close syntax-prosody mapping. Over the past thirty years, multiple studies have provided consistent evidence that infants are highly sensitive to prosodic boundary information and use it to segment the continuous speech signal into linguistically relevant units (\citealt{Gout2004, Hirsh-Pasek1987, KemlerNelson1989, Nazzi2000, Schmitz2008, Shukla2011}). Assuming that each prosodic boundary cue contributes individually to boundary perception, previous research has focused on the specific roles of pitch change, final lengthening, and pause in adult as well as infant listeners (e.g., \citealt{Aasland2003, Johnson2008, Lehiste1976, Petrone2017, Sanderman1997, Scott1982, Seidl2007, Streeter1978, Zhang2012}).

Studies with American English-learning infants found evidence for a developmental change in perception of these boundary cues (\citealt{Seidl2007, Seidl2008}). In a series of experiments with varying cue manipulations, \citet{Seidl2007} tested six-month-old American English-learning infants, applying an experimental design based on \citet{Nazzi2000}. In a familiarization phase, infants were presented with two sequences of the same words extracted from two different naturally recorded text passages of infant-directed speech. One familiarization sequence was a complete clause (within clause stimuli), e.g., \textit{\# Leafy vegetables taste so good \#}. The other one contained an internal clause boundary and thus a major prosodic boundary (non-clause stimuli), e.g., \textit{leafy vegetables \# taste so good}. During the test phase, the original text passages from which the familiarization sequences had been extracted were presented. In the first experiment, the internal boundary of the non-clausal familiarization sequence was fully marked by all three boundary cues, that is, a pitch reset at the juncture, final lengthening, and a pause. Results indicated that infants were better able to recognize the clausal than the non-clausal sequence in the continuous speech presented during the test phase. In subsequent experiments, the familiarization sequences were varied by acoustic manipulation of one or two of the three essential prosodic cues to find out if clause segmentation would still be possible. When either pause or final lengthening was neutralized, infants still preferred the passage with the familiar clausal sequence. Consequently, neither of these two cues was necessary to evoke the detection of a prosodic boundary. However, when the pitch cue was neutralized, infants no longer showed a preference for the clausal sequence. Yet pitch alone was not a sufficient marker: when it was present as a single cue with neutralized pause and final lengthening, clause segmentation was not successful. Hence, \citet{Seidl2007} concluded that American English-learning six-month-old infants need the pitch cue in combination either with pause or with final lengthening for clause segmentation.

\citet{Seidl2008} extended these investigations to four-month-old American English-learning infants. Tested with the same experimental material and design, the younger group was only successful in clause segmentation when all three cues in combination signaled the boundary. The authors concluded that four-month-old infants’ clause segmentation is based on a holistic processing that relies on a coalition of all boundary cues while six-month-old American English-learning infants are already able to process the cues independently of each other.
Crosslinguistic evidence provides first hints that boundary perception by six-month-olds is modulated by specific prosodic properties of the language that the infants are learning. \citet{Johnson2008} used the same experimental procedure as \citet{Seidl2007} to test Dutch-learning six-month-olds with Dutch materials. In contrast to their American English age-mates, the Dutch infants only showed evidence for clause segmentation when the prosodic boundaries were marked by the combination of pitch, lengthening, and a pause, but not when the pause was removed from the materials. The authors argue that pauses may be a strong marker of prosodic boundaries in Dutch and thus important for infants’ boundary detection. In fact, a comparison of the speech materials that were naturally recorded for the American English and the Dutch experiments showed that in the Dutch passages the pauses at the clause boundaries were twice as long as in the American English ones while the American English passages showed a much larger pitch reset after the juncture than the Dutch materials.

Broadening the crosslinguistic perspective, \citet{Wellmann2012} investigated German-learning infants on their perception of boundary cues in bracketed lists of names as \textit{[A and B and C]} in contrast to \textit{[A and B] [and C]} indicating either one group of three people or a group of two and a single person. According to \citet{Petrone2017}, German speakers typically employ the IP to signal such a grouping (see also \citetv{chapters/huttenlauch}). Indeed, analyzing the respective cues at the natural internal boundary in Wellmann et al.’s materials revealed typical characteristics of an IPB: pitch changes, specifically an upstep on the second peak and a partial pitch reset, a lengthening of the preboundary vowel, and the employment of a pause (for similar findings concerning prosodic boundary cues in German, see also \citealt{Fery2010}, \citetv{chapters/schuboe}, \citealt{Truckenbrodt2007a, Truckenbrodt2016}). Hence, the sequences either formed a single IP, \textit{[Moni und Lilli und Manu]\textsubscript{IP}} or were made up of two IPs with an internal IPB after the second name, \textit{[Moni und Lilli]\textsubscript{IP} [und Manu]\textsubscript{IP}}. The internal IPB was the focus of our investigations on prosodic cue perception.

Unlike previous studies (\citealt{Johnson2008, Nazzi2000, Seidl2007}), \citet{Wellmann2012} and the present study tested the detection of the prosodic boundary not by a clause segmentation task, but by a discrimination task. In this discrimination task, two groups of eight-month-old infants were familiarized with a sequence of one prosodic type, either with or without an internal IPB. In the subsequent test phase, all infants were presented with sequences of both prosodic types to test whether they discriminated between them. Given that previous research on infants’ attunement to features of segmental phonology also used discrimination tasks (\citealt{Mattock2006, Mattock2008, Polka1994, Werker1984}), we assumed that such a discrimination task should be suited to reveal differences in prosodic boundary information as well. We considered the methodology as an important contribution since the same materials were suitable for use in a behavioral study with adults, as well as for use in ERP studies with adults and infants (\citealt{Holzgrefe-Lang2016, Holzgrefe-Lang2018}). Moreover, the material can in principle be used in other languages as well (for French, \citealt{VanOmmen2020}).

\begin{sloppypar}
\citet{Wellmann2012}'s results revealed that eight-month-old infants preferred to listen to sequences of the new prosodic grouping (i.e., sequences with an IPB) after familiarization with sequences without an IPB. This indicated successful discrimination of the prosodic patterns. In subsequent experiments the prosodic boundary information was systematically varied by adding a cue or a subset of cues to the original sequence without an internal IPB after the second name. When the IPB was signaled by a pitch rise and final lengthening in combination but without a pause, eight-month-olds still successfully discriminated the sequences with boundary cues from the sequences without an internal IPB. However, when the IPB was signaled solely by a pitch rise or by final lengthening, infants did not discriminate the two prosodic conditions. These findings suggest that pitch change and lengthening in combination, but not as single cues, are sufficient for IPB detection in eight-month-old German-learning infants while the presence of a pause is not necessary.
\end{sloppypar}

Interestingly, the discrimination pattern of the German eight-month-olds mirrored a pattern that \citet{Holzgrefe-Lang2016} observed in a prosodic judgment task using the same stimuli with German-speaking adults. In this task, participants judged via button-press whether the stimuli contained an internal boundary or not. The results revealed that stimuli containing a pitch change and final lengthening in combination but no pause were judged as sequences with an IPB. In contrast, stimuli that contained only a pitch change were predominantly judged as sequences without an internal boundary, and sequences with only lengthening were judged at chance level, indicating no categorization.

Although the two tasks -- the discrimination task with infants and the prosodic judgment task with adults -- may place some different requirements on the participants and each group’s data was analyzed on its own, the similarity in the results across the two studies indicates that German-learning eight-month-olds’ sensitivity to prosodic boundary cues already resembles that of German adults. The question arises whether the discrimination pattern found in the German eight-month-olds is in fact the result of a perceptual attunement from a solely acoustically driven perception based on the presence of the salient pause cue to a more sophisticated linguistically affected perception relying on pitch change and lengthening.

Therefore, six-month-old German-learning infants were studied with the same experimental paradigm and the same stimuli as used by \citet{Wellmann2012}; that is, sequences without an internal IPB and sequences with an internal boundary cued by the full set or a subset of pitch, lengthening, and pause cues had to be discriminated.
In Experiment 1, detection of a prosodic boundary that is fully marked by the combination of all naturally occurring cues was investigated. We assumed that infants are able to detect this boundary, as previous research presenting infants with artificial pauses at boundary and non-boundary locations has revealed that German infants are highly sensitive to the correlation of prosodic boundary information already in their first half year of life (\citealt{Schmitz2008}).
Experiment 2 examined whether pitch change and final lengthening in combination are sufficient boundary markers or whether pause is a necessary cue for this age group. Here we aimed to clarify whether prosodic boundary detection at six months already reflects an attunement towards linguistically relevant markings or whether it is rather influenced by the perceptual salience of cues.
In Experiment 2a we used the same stimuli as in Experiment 2, but with a prolonged familiarization. We hypothesized that infants develop a sensitivity towards boundaries that are not cued by pause between six and eight months. In Experiment 2a, we asked whether this sensitivity would show up already at six months under optimized experimental conditions, that is, after a prolonged exposure. We hypothesized that the double amount of presentations of the familiarization sequence might lead to a more stable mental representation and would thus release (working) memory capacity to thoroughly explore the new stimulus with its differences.
In Experiment 3 we investigated whether the combination of pause and final lengthening provides sufficient information for boundary detection or whether -- as has been shown for younger American English-learning infants -- only a combination of all cues would evoke boundary detection.
In the following, we will successively introduce each experiment with its participants, stimuli, procedure, and its descriptive results. Subsequently, a statistical analysis across all four experiments will be reported, followed by a general discussion.

% Experiment 1


\section{Experiment 1: The influence of pitch, final lengthening, and pause}
Experiment 1 tested whether German-learning six-month-old infants are able to perceive an IPB that is signaled by the three main prosodic cues pitch change, final lengthening, and pause.

\subsection{Participants}
A group of twenty-four six-month-old infants (12 girls, 12 boys) was tested. Their mean age was 6 months, 11 days (range: 6 months, 2 days to 6 months, 27 days). Nine additional infants were tested but not included in the data for the following reasons: failure to complete the experiment (1), crying or fussiness (4), mean listening times of less than \WellmannSeconds{3} per condition (1), technical problems (2), and noise in the surroundings due to construction work (1).

All infants who participated in this and the following experiments were from monolingual German-speaking families, born full-term, and with normal hearing. They were recruited from birth lists obtained through the Potsdam city hall archives. All parents signed informed consent. None of the infants tested in the present study participated in more than one experiment.

\subsection{Stimuli}
All stimuli used in Experiments 1 and 2 were identical to those that were presented to eight-month-olds in the study by \citet{Wellmann2012}: the stimuli consisted of a sequence of three German names containing only sonorant sounds \textit{(Moni, Lilli, Manu)}, which allowed a reliable measure of F0 and were suitable for acoustic manipulation. The names were coordinated by \textit{und} (‘and’). A young female adult, a German native speaker from the Brandenburg area, was instructed to read the sequence in two different prosodic groupings indicated by different bracketing:

\ea (Moni und Lilli und Manu) -- without internal IPB
\ex (Moni und Lilli) und Manu -- with internal grouping
\z

Both sequences contained the same string of names and differed only in grouping either all three names together as shown in (1) or grouping the first two names together and the final one apart as shown in (2). Sequences of type (1) were produced as a single IP, without an internal boundary. In contrast, sequences of type (2) consisted of two IPs, with an internal IPB after the second name. The speaker repeated each sequence six times, resulting in six recordings per prosodic type. The intended grouping was confirmed by two independent listeners who were naïve with respect to the given bracketing. Recordings were made in an anechoic chamber equipped with an Audio-Technica AT4033A studio microphone, using a C-Media Wave soundcard at a sampling rate of \qty[group-separator={,},group-digits=all]{22050}{\hertz} with 16-bit resolution. Examples of both kinds of prosodic phrasing are depicted in \figref{fig1}A and B.
 


\begin{figure}% [h]
\includegraphics[width=\textwidth]{figures/fig1.png}
\caption{Oscillograms and pitch contours aligned to the text. Vertical lines mark the segmental boundaries. The hash mark indicates the silent pause after the IPB. (A) Sequence without an IPB used in Exp. 1, (B) Sequence with a fully marked internal IPB used in Exp. 1, (C) Sequence with pitch change and final lengthening used in Exp. 2 and 2a, (D) Sequence with pause and final lengthening used in Exp. 3.}
 \label{fig1}
\end{figure}

\begin{table}
  \caption{Mean values (range) of the acoustic correlates of prosodic boundary in the six experimental sequences \textit {Moni und Lilli und Manu} without and with an internal IPB, respectively. BT: boundary tone (Hz), maxF0 on Name2's final vowel; PR: pitch rise (Hz), maxF0 on Name2 minus minF0 on Name2; PPR: {partial pitch reset} (Hz),  maxF0 on Name2 minus minF0 on Name2; FL: {final lengthening} (ms), duration of Name2's final vowel; P: {pause} (ms), duration of pause after Name2. $\dagger$: in semitones.\label{tab.1}}
% % % \begin{tabular}{lcccc}
% % %   \lsptoprule
% % %   &  \multicolumn{2}{c}{Without internal IPB} & \multicolumn{2}{c}{With internal IPB} \\
% % %   Boundary cue & mean & \textit{SD} & mean & \textit{SD}\\\midrule
% % %   boundary tone (Hz) & 277 & 264-293 & 397 & 371 - 422\\ 
% % %   pitch rise (Hz)$^\dagger$ & 88 (6.7) & 77 - 110 (5.8 - 8.2) & 220 (14.0) & 197 - 240 (12.8 - 14.6)\\ 
% % %   partial pitch reset (Hz)$^\dagger$  & 5x rise: -12 (-0.7)  1x fall: 18 (1.1) & -23 - -5 (-1.3 - -0.3) & 6x fall: 55 (2.5)  & 36 - 96 (1.7 - 4.4)\\
% % %     final lengthening (ms) & 99 & 91 - 110 & 175 & 162 - 186\\ 
% % %   pause (ms) & 0 & 0 & 506 & 452 - 556\\ 
\fittable{\begin{tabular}{lccccc}
\lsptoprule
& \multicolumn{5}{c}{Boundary cue}\\\cmidrule(lr){2-6}
& BT  &  PR$^\dagger$ & PPR$^\dagger$    & FL  & P\\ \midrule
\multicolumn{6}{l}{\itshape Without internal IPB}\\
Mean     & 277      &  88 (6.7)            & 5$\times$rise: $-12$ ($-0.7$)    & 99      & 0  \\
         &          &                      & 1$\times$fall: 18 (1.1)          &         &    \\
SD       & 264--293 &  77--110 (5.8--8.2)  & $-23$ -- $-5$ ($-1.3$ -- $-0.3$) & 91--110 & 0  \\\addlinespace
\multicolumn{6}{l}{\itshape With internal IPB}\\
Mean     & 397      &  220 (14.0)            & 6$\times$fall: 55 (2.5) & 175       & 506       \\
SD       & 371--422 &  197--240 (12.8--14.6) & 36--96 (1.7--4.4)       & 162--186  & 452--556\\ 
\lspbottomrule
\end{tabular}}
% \footnotesize\begin{labeling}[~--]{boundary tone}\vspace{0.2pt}\footnotesize
% \item[boundary tone] maxF0 on Name2's final vowel
% \item[pitch rise] maxF0 on Name2 minus minF0 on Name2
% \item[partial pitch reset] maxF0 on Name2 minus minF0 on Name2
% \item[final lengthening] duration of Name2's final vowel
% \item[pause] duration of pause after Name2 
% \end {labeling}
 \end{table}
% \caption{boundary tone - maxF0 on Name2's final vowel; pitch rise - maxF0 on Name2 minus minF0 on Name2; partial pitch reset - maxF0 on Name2 minus minF0 on Name2; final lengthening - duration of Name2's final vowel; pause - duration of pause after Name2}
%\end{table}
The acoustic analysis of the recordings revealed clear acoustic differences between the two prosodic phrasings on and after the second name (see \tabref{tab.1}).
\clearpage
\subsubsection{Preboundary pitch movement} 
Sequences without an internal IPB that form one single IP were characterized by F0 lowering, with an accentual pitch rise on the first name, followed by a smaller pitch rise on the second name, that is, a downstep pattern (\citealt{Truckenbrodt2007b}). Sequences with an internal IPB exhibited a flat tonal contour (plateau) on the first name, followed by a large pitch rise on the second name, starting at the second syllable, and leading to an upstepped peak, a high boundary tone, at the final vowel. This pitch rise on the second name (measured as the difference between the maximum and minimum pitch on the second name in semitones) was 2.5 times greater and led to a higher maximum pitch than the small rise occurring at the same location in sequences without an IPB. Hence, the F0 contour of sequences with an internal IPB clearly indicated the following prosodic boundary. This tonal contour resembled the most common realization of internal IPBs in similar German sequences of names investigated in \citet{Petrone2017}.

\subsubsection{Postboundary pitch movement} 
Postboundary pitch reset was measured as the difference between the maximum pitch on the final vowel of the second name and the maximum pitch on the vowel of the conjunction. In sequences without an internal IPB there was no relevant pitch difference: in five of the six recordings the height of the downstepped second peak was slightly higher at the conjunction (on average by 0.7 semitones), whereas in one recording it was slightly lower (by 1.1 semitones). In sequences with an internal IPB a partial pitch reset, one step below the preboundary upstep, occurred (see \citealt{Truckenbrodt2007b} for a similar partial reset). This was expressed by a pitch fall of 2.5 semitones on average.

\subsubsection{Final lengthening} 
To explore final lengthening, the duration of the second name’s final vowel [i] was compared in sequences with and without an internal IPB. Its duration was 1.8 times longer in the grouping with an IPB, indicating a strong lengthening cue (cf. \citealt{Kohler1983}).

\subsubsection{Pause duration} 
Finally, the duration of the silent interval after the second name was measured. A pause with an average duration of \qty{506}{\ms} occurred in sequences with an internal IPB, whereas no pause was present in sequences without an internal IPB.


Taken together, in sequences with an internal IPB all acoustic correlates of the three main prosodic boundary cues were observed: a change in F0 (mainly, a preboundary pitch rise), a lengthening of the preboundary vowel, and the occurrence of a pause.
The recorded sequences were used to create sound files for presentation during the experiment. All recordings were scaled to a mean intensity of 70 dB. For each prosodic type, the six recordings were randomly concatenated with a silent interval of \WellmannSeconds{1} inserted between them. In this way, six sound files per prosodic grouping were created such that each file consisted of a different order.

Due to the missing durational cues (final lengthening and pause), sequences without an internal IPB were shorter than those with an internal IPB. The average duration of sequences without an internal IPB was \WellmannSeconds{1.76} (range: 1.67–\WellmannSeconds{1.87}), while it was \WellmannSeconds{2.16} (range: 2.13–\WellmannSeconds{2.2}) for sequences with an internal IPB. To match the sound files of the two prosodic types with respect to overall length, the number of sequences within each file was varied. As a result, sound files of the condition with an internal IPB contained six sequences and had an average duration of \WellmannSeconds{18.97}, and sound files of the condition without an internal IPB contained seven sequences (i.e., one random recording was repeated), leading to an average duration of \WellmannSeconds{19.32} (range: 19.16–\WellmannSeconds{19.43}). The difference in the number of sequences was crucial in order to present sound files of similar lengths during the experimental trials.

\subsection{Procedure}
In all experiments presented here, infants were tested using the headturn preference procedure (HPP) including a familiarization phase. During the experiment, the infant was seated on the lap of a caregiver in the center of a test booth. Inside this booth three lamps were fixed: a green one on the center wall, and a red one on each of the side walls. Directly above the green lamp was an opening for the lens of a video camera. Behind each of the red lights a JBL Control One loudspeaker was mounted. Each experimental trial started with the blinking of the green center lamp. When the infant oriented to the green lamp, it was turned off and one of the red lamps on a side wall started to blink. When the infant turned her head towards the red lamp, the speech stimulus was started, delivered via a Sony TA-F261R audio amplifier to the loudspeaker on the same side. The trial ended when the infant turned her head away for more than \WellmannSeconds{2}, or when the end of the speech file was reached. If the infant turned away for less than \WellmannSeconds{2}, the presentation of the speech file continued but the time spent looking away was not included in the total listening time. The whole session was digitally videotaped. The experimenter’s coding was recorded and served for the calculation of the duration of the infant’s head turns during the experimental trials. The caregiver listened to music over headphones to prevent influences on the infant’s behavior. Furthermore, she was instructed not to interfere with the infant’s behavior during the experiment. The experimenter sat in an adjacent room, where she observed the infant’s behavior on a mute video monitor and controlled the presentation of the visual and the acoustic signals by a button box. The experimenter was blind with respect to the type of acoustic stimuli presented during familiarization and testing.

An experimental session consisted of a familiarization phase immediately followed by a test phase. For Experiment 1, we familiarized half of the infants with sequences without an internal IPB, while the other half listened to sequences with an internal IPB. For both groups, familiarization lasted until at least 20 sequences were presented. Given that sequences without an IPB were shorter than sequences with an internal IPB, familiarization timing differed slightly. That is, when infants were familiarized with sequences without an internal IPB, familiarization lasted until \WellmannSeconds{55} of listening time had been accumulated. When familiarized with sequences with an internal IPB, infants had to accumulate \WellmannSeconds{63} of listening time.

After familiarization, infants immediately passed through the test phase that comprised twelve trials. Half of the test trials contained the identical sound files previously presented during familiarization (familiar test trials). The other six trials contained the sound files of the other prosodic grouping (novel test trials). The test trials were grouped in three blocks of four trials each. Two out of these four trials contained sequences with an internal IPB, the others contained sequences without an internal IPB. Within each block, test trials were randomly ordered with the side of presentation being counterbalanced for each prosodic type. Based on the infant’s head turns the listening time to each test trial was measured. The duration of each experimental session varied between four and six minutes, depending on the infant’s behavior.

\subsection{Descriptive results}\largerpage
We analyzed the data for each familiarization group on a descriptive level. Within this sample we observed clear numerical differences in the listening times. The group of infants that was familiarized with sequences with an internal IPB listened on average for  \WellmannSeconds{7.69} (SD = \WellmannSeconds{3.41}) to novel test trials and for \WellmannSeconds{7.52} (SD = \WellmannSeconds{2.90}) to familiar test trials; that is, the mean listening time between novel and familiar test trials only differed by \textit{M\textsubscript{Diff}} = \WellmannSeconds{0.17}. Six out of twelve infants listened longer to the novel test trials.

The group of infants that was familiarized with sequences without an internal IPB listened on average for \WellmannSeconds{9.65} (SD = \WellmannSeconds{2.96}) to novel test trials and for \WellmannSeconds{8.06} (SD = \WellmannSeconds{3.02}) to the familiar ones (see \figref{fig2}). Hence, the familiarization with sequences without an internal IPB yielded a novelty preference with a mean listening time difference of \textit{M\textsubscript{Diff}} = \WellmannSeconds{1.58}. Nine out of twelve infants listened longer to the novel test trials.

\begin{sloppypar}
Overall, a clearer numerical difference between the two prosodic patterns showed up after familiarization with sequences without internal IPB, whereas the familiarization with sequences with an internal IPB seemed to be much less or even not effective, as also evidenced in other studies (see \citealt{VanOmmen2020} and \citealt{Wellmann2012} for a discussion of this asymmetric behavior). Given the constraints we usually encounter in infant research (small sample sizes, high drop-out rates related to infant behavior) we therefore decided to run only the familiarization with sequences without an internal IPB in the subsequent experiments of the present study.\footnote{With the same Experiment 1, we also tested 24 four-month-old infants ($n=12$ in each familiarization group) in a slightly modified HPP setup (to adapt for the limited head movements at that age the position of the side lamps was moved to the edges of the front wall). Four-month-old infants that were familiarized to sequences without an IPB had mean listening times of \WellmannSeconds{10.5} (SD = \WellmannSeconds{3.59}) to novel test trials and \WellmannSeconds{10.25} (SD = \WellmannSeconds{3.26}) to the familiar ones, $t(11) = 0.483,\allowbreak p = 0.639$, two-tailed. The group familiarized to sequences with an IPB listened on average \WellmannSeconds{10.01} (SD = \WellmannSeconds{4.12}) to the novel test condition and \WellmannSeconds{10.67} (SD = \WellmannSeconds{4.90}) to the familiar one, $t(11) = -1.1, p = 0.295$, two-tailed. Given the null result we did not continue in testing four-month-olds.}
\end{sloppypar}

\section{Experiment 2: The influence of pitch and final lengthening}\largerpage[2]
Experiment 2 examined whether a subset of prosodic cues would suffice to trigger the perception of a boundary in six-month-old infants. Specifically, the impact of the combination of a rising pitch contour and final lengthening was under focus, questioning the necessity of the pause cue.

% Experiment 2


\subsection{Participants}
Sixteen infants (8 girls, 8 boys) were tested. The mean age was 6 months, 14 days (range: 5 months, 28 days to 6 months, 29 days). Four additional infants were tested but not included in the data analysis for the following reasons: failure to complete the experiment (1), crying or fussiness (1), and mean listening times of less than \WellmannSeconds{3} per condition (2).

\subsection{Stimuli}
Experiment 2 involved stimuli of two prosodic types: One condition comprised the same six sequences without an internal IPB as in Experiment 1. The other one consisted of six sequences with only pitch rise and lengthening cues indicating the boundary. For this prosodic type, sequences without an internal IPB were locally acoustically manipulated with respect to F0 on the second name and the duration of its final vowel. A specific pitch reset cue, that is, a manipulation of F0 at the position of the postboundary conjunction, was not implemented, since in the stimuli of Experiment 1 the postboundary peak was utterance-final and generally low. Stimuli manipulations were carried out the same way as in \citet{Wellmann2012}.

The stimuli without a boundary were selected as the basis for the acoustic manipulations to avoid a potential influence of additional cues that may contribute to IPB marking and perception. Thus, the crucial boundary information, here a rising pitch contour and final lengthening, was added to the sequences without an internal IPB. Hence, experimental effects can clearly be attributed to the acoustic properties under investigation. By using these local cue manipulations the stimuli with IPB cues differed from sequences that were used in the condition without an internal IPB only within a predefined critical region and by controlled acoustic properties. However, this local manipulation led to the concession that sequences with inserted pitch rise and lengthening differed from natural sequences with an internal IPB with respect to the pitch contour of the first name. The original recordings without an internal IPB had an accentual peak on the first name, that is, a pitch cue to a phonological phrase boundary (\figref{fig1}C). The accentual peak on the first name was always lower than the peak of the H\% in the second name (\textit{M\textsubscript{F0MAX}} = \qty{317}{\hertz} vs. \qty{388}{\hertz}). However, this kind of cue was not present in the naturally produced sequences with an internal IPB, but was preserved in sequences with inserted cues since sequences without an internal IPB were the base for sequences with inserted cues and cue manipulations were restricted to the second name. Hence, the inserted pitch rise was preceded by another smaller pitch cue. If the pitch rise on the second name is parsed globally, that is, in relation to the previous pitch contour, the pitch rise cue in the manipulated sequences is less pronounced and potentially less salient than the pitch rise cue in natural sequences with an internal IPB.

The acoustic manipulation was carried out with Praat (\citealt{Boersma2010}). As the phonetic magnitude of prosodic cues differs across languages, and also within a language depending on the syntactic structure (e.g., for pausing in German, see \citealt{Butcher1981}), there is no unique value for each prosodic cue. Hence, we decided to implement the same phonetic magnitude for each cue that was present in the corresponding naturally produced stimulus with an internal IPB, proceeding in the same way as other studies that have employed cue manipulations (e.g., \citealt{Seidl2007}).

Manipulation steps were the following: To implement the pitch rise, first, the pitch contour of the sequences without an internal IPB was stylized (two semitones). This transformation decreases pitch perturbations by reducing the number of pitch points. Second, the pitch points on the second name were set to the reference values. The reference values of F0 were measured on the second name in the six original sequences with an internal IPB (used in Experiment 1), namely at the midpoints of the four segments [l], [ı], [l], [i] and at the position of the maximum pitch present on the final vowel. For the manipulation of the pitch contour, pitch points with the mean values at these time points (\qty{176}{\hertz}, \qty{183}{\hertz}, \qty{224}{\hertz}, \qty{305}{\hertz}, and \qty{397}{\hertz}) were inserted into the stylized sequences without an internal IPB at the same positions. After PSOLA resynthesis in Praat, the six new stimuli contained a natural sounding pitch rise of \qty{212}{\hertz} (13.65 semitones) leading to an H\% with a mean value of \qty{388}{\hertz}.
To implement final lengthening, the final vowel [i] of the second name was lengthened to 180\%. This factor was chosen because in the natural stimuli, the crucial vowel was on average 1.8 times longer in sequences with an internal IPB than in sequences without an internal IPB (\tabref{tab.1}). A sequence with manipulated pitch and lengthening is depicted in \figref{fig1}C.

To avoid comparing natural with acoustically manipulated material, we carried out a slight acoustic manipulation in sequences without an internal IPB as well, that is, the stylization of the pitch contour (two semitones). After pitch stylization, sequences were resynthesized using the PSOLA function.

Sequences without an internal IPB lasted on average \WellmannSeconds{1.76} (range: 1.67–\WellmannSeconds{1.87}), while sequences with inserted pitch and lengthening had a mean duration of \WellmannSeconds{1.84} (range: 1.74–\WellmannSeconds{1.96}). From these sequences six differently ordered sound files per prosodic type were created to be used as experimental trials. The interstimulus interval between the sequences and within a sound file was \WellmannSeconds{1}. All sound files contained seven sequences (one random recording was repeated). The files containing sequences without an internal IPB had an average duration of \WellmannSeconds{18.33} (range: 18.23–\WellmannSeconds{18.43}) and the files containing sequences with inserted pitch and lengthening cues lasted on average \WellmannSeconds{18.81} (range: 18.79–\WellmannSeconds{19.01}).

\subsection{Procedure}
All infants were familiarized with sequences without an internal IPB. The familiarization lasted until at least 20 sequences had been presented, resulting in a minimum of \WellmannSeconds{52} of accumulated listening time.
The familiarization was immediately followed by a test phase with twelve trials. As in Experiment 1, half of the test trials contained the same sound files that the infants had heard during familiarization. The other half contained the files of the sequences with pitch and lengthening cues, which had not been presented during familiarization. All twelve test trials were grouped in three blocks of four trials each (two of each prosodic type in a random order). The infant’s listening time to each test trial was measured.

\subsection{Descriptive results}
Infants tested in Experiment 2 showed a mean listening time of \WellmannSeconds{7.09} (SD = \WellmannSeconds{2.1}) to the novel test trials and a mean listening time of \WellmannSeconds{7.20} (SD = \WellmannSeconds{2.43}) to the familiar test trials (see \figref{fig2}). Eight out of 16 infants had longer listening times to the familiar test trials.

% Experiment 2a

\section{Experiment 2a: The influence of pitch and final lengthening after prolonged familiarization}

To verify that the non-discrimination in Experiment 2 was due to the composition of the stimuli, we modified the experimental design by doubling the familiarization time. Considering a longer familiarization to enable successful discrimination stems from findings of studies that tested French-learning infants’ discrimination of rhythmic patterns (\citealt{Bijeljac-Babic2012, Hohle2009, Skoruppa2009}). For French, a language without contrastive stress at the word level, the perception of prosodic cues indicating lexical stress has been shown to be hard for infant learners and adult listeners (\citealt{Bhatara2013, Hohle2009}). In a study by \citet{Bijeljac-Babic2012} monolingual ten-month-old French-learning infants exhibited a null result in discriminating an iambic and a trochaic version of a pseudo-word after a one-minute familiarization. However, when familiarization duration was increased to two minutes, they were successful at discriminating the stress patterns as indicated by a novelty effect. Bijeljac-Babic et al. concluded that the null result after the short familiarization could not be interpreted as a general inability to distinguish the two stress patterns, but was due to the short familiarization. Regarding the novelty effect after long familiarization, they drew on the model by \citet{Hunter1988} that would predict novelty preferences in relatively easy discrimination conditions.

In our case of discriminating lists of names with and without IPB cues, it is important to consider that boundary perception without the pause cue is successful in older infants. This sensitivity seems to arise between six and eight months. If six-month-old German-learning infants are already at the beginning of this development, in the light of the Hunter and Ames model, discrimination might show up with reduced task difficulty. Following \citet{Bijeljac-Babic2012} we hypothesized that a more robust mental representation of the stimuli presented during familiarization may improve the ability to detect differences between the familiar and the novel stimuli. In the following experiment, we therefore doubled the amount of presentations of the familiarization stimulus in order to help infants building up a more robust mental representation of the sequences without an internal IPB. Through this modification, infants might be able to accomplish a still difficult task for their age such as the detection of a boundary signaled only by pitch and lengthening cues.\largerpage[2]

\subsection{Participants}
Twenty-three six-month-old infants (12 girls, 11 boys) were tested. The mean age was 6 months, 15 days (range: 6 months, 0 days to 6 months, 26 days). All infants were from monolingual German-speaking families, born full-term and normal-hearing. Thirty-one additional infants were tested but their data were not included in the analysis for the following reasons: failure to complete the experiment (6), crying or fussiness (15), mean listening times of less than 3 seconds per condition (4), technical problems (2), experimenter error (2), parental interference (1) and outlying listening times due to steady fixation (1).
Drop-out rate was especially high, primarily due to infants’ fussiness and failure to finish the experiment (accounting for 68\% of all drop-outs). The longer lasting familiarization which increased the total duration of the experiment to about 6 to 10 minutes (in contrast to six minutes with the original familiarization duration) may have reduced infants’ attention.

\subsection{Stimuli}
Stimuli were exactly the same as in Experiment 2.

\subsection{Procedure}
Infants were familiarized with sequences without an internal IPB. The familiarization duration was set to \WellmannSeconds{104}. After familiarization, infants listened to exactly the same twelve test trials as used in Experiment 2, half of them being sequences with inserted pitch and lengthening cues, the other half sequences without an internal IPB.

\subsection{Descriptive results}
Infants tested in Experiment 2a showed a mean listening time of \WellmannSeconds{6.86} (SD = \WellmannSeconds{2.01}) to the novel test trials, and a listening time of \WellmannSeconds{7.52} (SD = \WellmannSeconds{2.25}) to the familiar test trials (see \figref{fig2}). Fifteen out of 23 infants listened longer to familiar test trials.


                % Experiment 3

\section{Experiment 3: The influence of pause and final lengthening}\largerpage

In the following Experiment 3 the boundary was cued by a pause in combination with final lengthening, but without any pitch cue. The aim of this experiment was to investigate whether six-month-olds would respond to a boundary that is cued by a subset of the naturally occurring cues including a pause. The combination of pause and lengthening was chosen because a pause rarely occurs as the only cue in German (only at 1.3\% of all boundaries in the analysis by \citealt{Peters2005PhonetMerk}), and would thus sound unnatural as the only inserted cue\footnote{We are grateful to one reviewer who raised the question whether discrimination would be possible with pause as the only boundary cue. We hypothesize that a similar experiment with a boundary cued by pause only would lead to successful discrimination as well. This hypothesis is based on infants’ successful discrimination between clauses with artificially inserted pauses at non-boundary locations and clauses with pauses at natural boundary positions with co-occurring boundary cues (\citealt{Schmitz2008}).}, and because the combination of pause with final lengthening occurs more frequently (8.4\%) in spoken German than the combination of pause and pitch (4.9\%, values by \citealt{Peters2005PhonetMerk}). Moreover, this combination is interesting to look at crosslinguistically, as six-month-old American English-learning infants failed to perceive a boundary signaled only by the combination of pause and lengthening cues (\citealt{Seidl2007}).


\subsection{Participants}
Sixteen infants (8 girls, 8 boys) were tested. The mean age was 6 months, 10 days (range: 5 months, 14 days to 6 months, 28 days). Eleven additional infants were tested but not included in the data for the following reasons: failure to complete the experiment (1), crying or fussiness (6), mean listening times of less than \WellmannSeconds{3} per condition (2), and technical problems (2).

\subsection{Stimuli}\largerpage
In Experiment 3, we contrasted sequences without an internal IPB and sequences that contained a pause and final lengthening. To create the latter, five recordings of sequences without an internal IPB were acoustically manipulated on and after the second name. We did not use exactly the same set of recordings without an internal IPB as in Experiments 1 and 2 because some of the sequences contained co-articulation between the final vowel of the second name and the initial vowel of the conjunction such that the insertion of a pause would have created an unnaturally sounding stimulus. Hence, for Experiment 3, we chose five sequences with no or only minimal co-articulation: three sequences that had been used in the previous experiments and two more sequences recorded with the same speaker. First, any co-articulation between the second name and the subsequent conjunction, that is, the section of formant transition from the final vowel [i] to the vowel [u], was cut out at zero crossings. Second, a silent interval of \qty{500}{\ms} -- corresponding to the mean duration of pauses measured in natural sequences with an internal IPB from Experiment 1 -- was inserted at the offset of the final vowel. Then, the final vowel was lengthened to 180\%, according to the average lengthening factor found in the acoustic analysis of sequences with an internal IPB in Experiment~1. A sequence with inserted pause and lengthening cues is depicted in \figref{fig1}D. For both stimulus conditions, the pitch contours were stylized (two semitones) and sequences were resynthesized using the PSOLA function in Praat.

Sequences without an internal IPB lasted on average \WellmannSeconds{1.82} (range: 1.71–\WellmannSeconds{1.89}), while sequences with inserted pause and lengthening had a mean duration of \WellmannSeconds{2.36} (range: 2.27–\WellmannSeconds{2.42}). The sound files for the condition without an internal IPB contained seven sequences (two of the five recordings were randomly chosen and repeated at the end of a sound file) and had an average duration of \WellmannSeconds{18.75} (range: 18.59–\WellmannSeconds{18.86}). To achieve a similar mean duration, the sound files for the condition with inserted pause and lengthening cues contained only six sequences, resulting in an average duration of \WellmannSeconds{19.2} (range: 19.09–\WellmannSeconds{19.24}). The interstimulus interval between the sequences in each type of sound file was \WellmannSeconds{1}.

\subsection{Procedure}
The procedure was identical to that of Experiments 1 and 2. All infants were familiarized with sequences without an internal IPB until at least 20 sequences had been presented. This led to a minimum of \WellmannSeconds{54} of accumulated listening time. The familiarization was immediately followed by a test phase of twelve test trials, half of them containing familiar sequences without an internal IPB, the other half, containing new sequences with an internal IPB cued by pause and final lengthening.

\subsection{Descriptive results}
Infants tested in Experiment 3 showed a mean listening time of \WellmannSeconds{8.65} (SD = \WellmannSeconds{3.97}) to the novel test trials and a mean listening time of \WellmannSeconds{7.1} (SD = \WellmannSeconds{3.52}) to the familiar test trials (see \figref{fig2}). Eleven out of 16 infants had longer listening times to the novel test trials.


            % Joint ananlysis
\section{Joint statistical analysis of the experiments}\largerpage

We statistically analyzed the data of all four experiments in a repeated-measures ANOVA with Familiarity as within-subject factor (mean listening times to novel vs. familiar test trials) and Experiment as between-subject factor (Exp. 1, 2, 2a, 3).\footnote{Note that from Experiment 1 only the data from the group familiarized without IPB was considered ($n=12$).} This revealed a significant main effect of Familiarity, $F(1,63) = 5.480,\allowbreak p = 0.022$, but not of Experiment, $F(3,63) = 1.334,\allowbreak p = 0.271$. However, there was a significant interaction of Familiarity and Experiment, $F(3,63) = 5.628,\allowbreak p = 0.002$.\largerpage

\begin{figure}
\includegraphics[width=\textwidth]{figures/fig2.png}
\caption{Mean listening times in seconds to familiar and novel test trials after familiarization with sequences without an internal IPB in Experiment 1 (all cues), 2 (pitch and lengthening, short familiarization), 2a (pitch and lengthening, long familiarization) and 3 (pause and lengthening). Error bars indicate ± 1 SE.}
 \label{fig2}
\end{figure}

To dissolve the significant interaction and to determine which experiments differed from each other we carried out pairwise comparisons. Therefore, we compared the results of each experiment with those from Experiment 2 as a control experiment – the one that yielded the smallest listening time differences between novel and familiar test trials. We ran a post-hoc $t$-test on the difference scores (mean listening time to novel test trials minus mean listening time to familiar test trials) in Experiment 2 versus 1, Experiment 2 versus 2a, and Experiment 2 versus 3. Difference scores are depicted in \figref{fig3}.

\begin{figure} %[htbp]
\includegraphics[width=\textwidth]{figures/fig3.png}
\caption{Mean listening time differences to novel minus familiar test trials in seconds. Error bars indicate ± 1 SE.}
 \label{fig3}
\end{figure}

To adjust for multiple comparisons, the alpha-level of post-hoc $t$-tests was corrected according to \citet{Holm1979}. With three levels of comparisons this resulted in $\alpha = 0.05$ for the largest $p$-value, $\alpha = 0.025$ for the mid $p$-value, and $\alpha = 0.017$ for the smallest $p$-value.

\subsection{Pairwise comparison of Experiment 2 versus 1}\largerpage
The post-hoc test for Experiment 2 versus 1 failed to reach significance, \textit{M\textsubscript{Diff}} = \WellmannSeconds{1.694}, $t(26) = -2.073,\allowbreak p=0.048,\allowbreak \alpha = 0.025$. However, on the descriptive level, we see a much larger listening time difference in Experiment 1 compared to Experiment 2. Infants in Experiment 1 had a mean listening time difference of \textit{M\textsubscript{Diff}} = \WellmannSeconds{1.584} with a preference for novel test trials, whereas infants in Experiment 2 had a mean listening time difference of \textit{M\textsubscript{Diff}} = \WellmannSeconds{-0.110}. Considering the small sample size that presumably prevents statistical significances, this may indicate that six-month-old infants might tend to discriminate the two types of prosodic patterns when the IPB is indicated by pitch, lengthening, and pause, but not when it is cued by pitch and lengthening only (also see the comparison to Experiment 3 below).

\subsection{Pairwise comparison of Experiment 2 versus 2a}
The post-hoc test for Experiment 2 versus 2a was not significant, \textit{M\textsubscript{Diff}} = \WellmannSeconds{0.558}, $t(37) = 0.843,\allowbreak p = 0.404,\allowbreak \alpha = 0.05$. Infants in Experiment 2a had a mean listening time difference of \textit{M\textsubscript{Diff}} = \WellmannSeconds{-0.668} with a slight preference for familiar trials. Infants in Experiment 2 had a mean listening time difference of \textit{M\textsubscript{Diff}} = \WellmannSeconds{-0.110}. This comparison indicates that infants’ behavior does not differ between Experiments 2 and 2a. Hence, the data do not support the hypothesis that a longer familiarization phase leads to better discrimination in six-month-olds suggesting that – unlike eight-month-olds – they still need the pause cue to detect the boundary (see Exp. 1). However, it is possible that doubling the familiarization time may have reduced infants’ general attention during the test phase and may have obscured their discrimination of the test stimuli. This is also indicated by the high drop-out rate, which suggests the modified version of the experiment was especially hard.

\subsection{Pairwise comparison of Experiment 2 versus 3}
The post-hoc test for Experiment 2 versus 3 almost reached significance, \textit{M\textsubscript{Diff}} = \WellmannSeconds{1.688}, $t(30) = -2.512,\allowbreak p = 0.018,\allowbreak \alpha = 0.017$. Infants in Experiment 3 had a mean listening time difference of \textit{M\textsubscript{Diff}} = \WellmannSeconds{1.578} with a preference for novel test trials. Infants in Experiment 2 had a mean listening time difference of \textit{M\textsubscript{Diff}} = \WellmannSeconds{-0.110}. We interpret this as a tendency towards a better discrimination of stimuli, in which pause and lengthening indicate the boundary, instead of pitch and lengthening.
Moreover, the results obtained from Experiment 3 support the interpretation that infants in Experiment 1 detected the boundary that was marked by pause, lengthening, and pitch. Note that the number of participants was higher in Experiment 3 ($n=16$) compared to Experiment 1 ($n=12$). This underlines the issue of low statistical power in Experiment 1. Across Experiments 1 and 3, the mean listening time scores were very similar with both revealing a numerically strong novelty effect.

Overall, we interpret the six-month-olds’ data as an indicator for successful perception of boundaries that are marked by the full set of cues or by the subset of pause and lengthening. In contrast, the combination of pitch and lengthening seems to be a non-sufficient marking. This points to a crucial role of the pause cue in early prosodic boundary processing in German.

            % General discussion

\section{General discussion}
Perceptual reorganization in early speech perception has been reported extensively and in numerous languages for aspects of segmental phonology. However, research on the development of prosody – specifically phrasal prosody – is still sparse. The present study, in combination with the findings from \citet{Wellmann2012} and \citet{Holzgrefe-Lang2016, Holzgrefe-Lang2018}, contributes to uncovering a developmental change in the processing of prosodic boundary information in German.

\begin{sloppypar}
The experiments presented in this paper have addressed the role of pitch change, final lengthening, and pause in boundary detection by German-learning infants. Although the statistical power of our experiments is small due to the low number of infants and the limited amount of trials, our data yield three major results that need to be discussed with caution. First, German six-month-olds (but not four-month-olds) are able to detect a major prosodic boundary signaled by all the cues. Second, pitch change combined with final lengthening did not appear as a sufficient marking for six-month-olds, neither after a prolonged familiarization. Third, six-month-olds do not generally need a combination of all the three cues, but a combination of pause and final lengthening is sufficient to detect the boundary.
\end{sloppypar}

We will focus our discussion on the questions that were raised in the introduction: First, what do these results tell us about developmental changes in infant prosodic cue perception? Second, we will embed the results into the previous research on American English- and Dutch-learning infants, focusing on crosslinguistic similarities and differences in prosodic cue weighting. Beyond, we will compare the present behavioral outcomes to electrophysiological findings (\citealt{Holzgrefe-Lang2018}) and discuss the cognitive demands underlying a potential asymmetry in prosodic cue perception.

\subsection{Developmental changes in German boundary perception}\largerpage
The results suggest that German-learning infants have developed a sensitivity to fully marked IPBs by six months\footnote{Four-month-old infants displayed a null result in the same Experiment 1 which allows for two interpretations: either, four-month-olds are not yet able to detect fully marked IPBs, or they are able to, but can’t show their ability with this kind of method. Even though there are few studies showing that the HPP methods in principle works with four-month-olds (e.g. \citealt{Bosch2001, Herold2008, Seidl2008}), the familiarization design might not be optimal for this young age group as it depends on a rather high working memory load (process the auditory information and store them to detect the change in the test phase). A study by \citet{Hohle2009} revealed a null result for four-month-olds in a familiarization technique, whereas \citet{Herold2008} evidenced discrimination in the same age group, with the same materials, but a change in the experimental setup to a discrimination technique without a familiarization phase. Hence, a more simple preference paradigm might have worked better with our materials in the four-month-olds; however, the data would not be directly comparable to the data of the older infants.}  and even to a subset of boundary cues containing pause and final lengthening. While six-month-olds heavily rely on the pause cue to detect a boundary, two months later infants have enhanced their sensitivity to boundaries by perceiving a subtle difference indicated by pitch and lengthening (\citealt{Wellmann2012}). Perceptual attunement in the acquisition of German phrasal prosody mainly concerns the necessity of the pause cue. Two observations from German, 1) a rather inconsistent occurrence of the pause cue in German adult-directed speech (ADS) as well as 2) no or only short pauses at minor syntactic boundaries, underline the usefulness of a German learner’s ability to detect a boundary without pause, and hence support an enhancement in the sensitivity towards boundary cues.

At first glance, the occurrence of a pause seems to be highly reliable with respect to its function as a linguistic boundary cue in infant-directed speech (IDS): whenever pauses occur they are likely to indicate a sentence boundary (\citealt{Fernald1984, Fisher1996}). However, a pause does not seem to be the predominant boundary cue in German ADS and occurs only rarely as a single cue: \citet{Peters2005PhonetMerk} analysis of phrase boundary markings in the German Kiel Corpus of spontaneous ADS\footnote{In this analysis, all auditory breaks that occurred turn-internally within the continuous speech stream were classified as phrase boundaries.} showed that pauses occurred only at 38\% of all boundaries, while pitch changes did so at 74\% and lengthening at 66\%. An essential finding was that cue combination at boundaries was a frequent pattern, occurring at 61.6\% of all boundaries. Among these, the co-occurrence of pitch and lengthening (24.6\%) and the coalition of all three cues (23.7\%) were the most frequent. Cue combinations including only pause and one additional cue were comparatively infrequent: only 8.4\% of all boundaries were marked by a combination of pause and lengthening, and 4.9\% by a combination of pause and pitch. Each prosodic cue also occurred as a single cue: pitch alone marked 20.8\% of all boundaries, while lengthening cued 9.4\%, and pause only 1.3\%. In brief, \citet{Peters2005PhonetMerk} revealed pause to be the least frequent and the combination of pitch and lengthening to be the most frequent marker. This implies that, at least in German ADS, a large proportion of phrase boundaries are not signaled by a pause, which would cause a segmentation problem for learners who overly rely on the occurrence of a pause. Unfortunately, corresponding data on cue frequency in German IDS are missing, but \citet{Fernald1984} report longer pause duration and a higher correspondence between pause and sentence boundaries in German IDS compared to ADS. Also, a systematic review by \citet{Ludusan2016} across several languages suggests that pause duration is increased in IDS. So, it may be the case that a high reliance on pause as a boundary cue is appropriate when exposed to IDS, but not when exposed to ADS, making a change in the reliance on this cue necessary to become a proficient processor of ADS.

The second argument that relying solely on the pause cue is not an optimal strategy comes from the fact that pauses signal major prosodic and syntactic boundaries, whereas minor prosodic and syntactic boundaries like phrase boundaries are less often marked by a pause (\citealt{Strangert1991, Terken1992}) or they are marked by pauses of shorter durations (e.g., \citealt{Butcher1981, Goldman-Eisler1972}). Thus, if prosodic cues are essential for infants for detecting not only major clause boundaries in the signal but also boundaries of smaller units within these larger domains, children must become more sensitive to boundary markings that do not involve a pause. To sum up, the developmental change evidenced here seems to be in line with the requirements of the ambient language German.

\subsection{Crosslinguistic comparison} 
Turning to the next point of the discussion -- the crosslinguistic dimension of boundary cue perception -- our results reveal similarities as well as dissimilarities in Dutch, German, and American English infants. First, the finding that six-month-old German infants are sensitive to naturally occurring, fully marked IPBs is in line with results from previous studies with Dutch- and American English-learning infants (\citealt{Johnson2008, Nazzi2000, Seidl2007,Soderstrom2005}) and thus expands the crosslinguistic evidence that young infants are sensitive to natural prosodic phrasing. For American English, even four-month-olds have been shown to use fully marked boundaries for the segmentation of complex clauses.

\begin{sloppypar}
We consider our materials – rather short and phonologically highly-controlled sequences with successively inserted cues – an important extension to the crosslinguistic field of infant prosodic boundary perception. A recent study by \citet{VanOmmen2020} created similar stimuli in French to be presented to French- and German-learning infants. The sequences were three coordinated French names either with a major prosodic boundary \textit{[Loulou et Manu][et Nina]} or without a boundary \textit{[Loulou et Manou et Nina]}. Hence, materials were identical in their structure to the concatenation of three German names used in the present study. Also the procedure was the same, a discrimination task in the HPP paradigm with a familiarization phase followed by a test phase. The results showed that French six- and eight-month-olds perceived the boundary when it was signaled by all the three cues, but none of the two age groups was successful when the boundary was cued by pitch and lengthening only. Interestingly, in contrast, German infants presented with the same French materials perceived the pitch-lengthening cued boundary at eight months, but not at six months. This result reinforces the data from the present study and from \citet{Wellmann2012} and supports the interpretation of a developmental change in German boundary perception related to the pause cue. Moreover, no such developmental change can be observed in French infants’ performance pointing towards language-specificity of the respective development.
\end{sloppypar}

\subsubsection{Crosslinguistic similarities}
A similarity found across the American English, Dutch, German, and French studies is that pause seems to be a necessary boundary cue for the youngest groups of tested infants (Dutch- and German-learning six- and American English-learning four-month-olds, for French even in both six- and eight-month-olds) -- independently of whether the specific task requires a segmentation or a discrimination of stimuli. This points to a language-general way of processing prosodic boundaries in the first months of life that is strongly related to the acoustically salient pause cue.

A strong reliance on the pause is useful since among the three main boundary cues, pause is the most “universal” one. In the languages in which infant boundary perception has been studied, pauses have many pragmatic and paralinguistic functions; however, when it comes to linguistic structure, it serves only one function, that is, the marking of syntactic boundaries. This may render pause a crosslinguistically highly reliable cue. Note that preboundary lengthening and pitch may bear more than one linguistic function. Duration as the acoustic correlate of lengthening is also used to express lexical and/or phrasal stress as well as phonemic contrasts (vowel duration). Regarding pitch, the majority of the world languages are tonal; this means that pitch is used to express different lexical items. In pitch-accent languages like Japanese or stress languages such as English pitch can also be used to distinguish word meanings. Moreover, pitch bears several functions at the sentence level, for example the distinction between declaratives and questions. Moreover, pause is a perceptually rather salient feature of an acoustic signal and it provides categorical information that can be processed locally because the presence or absence of silence can be detected immediately in the signal. This may be different for the other two boundary cues, pitch changes and lengthening, which constitute relational information and require the parsing of longer strings to recognize any changes in pitch and duration at the location of the boundary in relation to the whole speech string.

Therefore, the available results by infants learning American English, Dutch, German, and French revealing that the occurrence of a pause is initially required for boundary detection may reflect a rather universal processing that initially relies on the pause as an acoustically salient categorical cue that can be easily processed independent of the contextual information in every language environment.\footnote{We are grateful to one reviewer who suggested to link our findings to those of individuals with acquired language impairments in which the special relevance of the pause cue is also evident (\citealt{Aasland2003}). When tested on resolving syntactic ambiguities in coordinate structures, a group of individuals with aphasia after left-hemispheric brain damage – in contrast to a control group – was not able to consistently identify the phrase boundary cued by lengthening and pause (with neutral pitch). However, when the pause duration was increased beyond normal ranges, accuracies improved. Thus, pause also seems to play a crucial role in impaired comprehension and may enable boundary detection even in the absence of the pitch cue.}

\subsubsection{Crosslinguistic differences}
The major difference in the development between Dutch, American English, German, and French concerns if, and if so, when, infants respond to boundaries that are not marked by a pause. For Dutch, we only know that six-month-olds need the pause for boundary detection since older infants were not tested (\citealt{Johnson2008}). French infants still need the pause cue at eight months (\citealt{VanOmmen2020}). In American English-learning infants, the necessity of a pause as a boundary cue disappears already between the ages of four and six months (\citealt{Seidl2007, Seidl2008}). This is when in German-learning infants the perception of fully cued boundaries first emerges. Only between the ages of six and eight months a developmental change occurs that makes the pause cue no longer necessary.

\citet{Seidl2008} interpret the behavior of the four-month-old American English-learning infants as a so-called holistic processing in which all cues are equally attended to. They argue that this reflects a general processing mechanism rather than a linguistically based strategy. By six months, American English-learning infants do assign more weight to pitch. Seidl and Cristià explain this development through the increased language exposure allowing to observe the distribution of boundary cues in their native language. Infants may have learned by this age, that pauses are unreliable boundary cues, whereas pitch is a more reliable cue to syntactic boundaries in American English.

Comparing the developmental trajectory between American English- and Ger\-man-learn\-ing infants, the data reveal that the development is different at six months. American English-learning six-month-olds need the pitch cue (in any combination with another cue), while this seems not to be the case for the German six-month-olds. This may suggest that pitch information is more salient and thus more important for American English learners than for German learners. In fact, crosslinguistic comparisons of pitch in IDS have found that the mean, minimum, and maximum F0 as well as the F0 variability is significantly higher in American English IDS than in German IDS (\citealt{Fernald1984, Fernald1989}). Hence, American English-learning infants may be more prone to attend to pitch variation in their input than German-learning infants.

\begin{sloppypar}
A comparison of German- and Dutch-learning six-month-olds (\citealt{Johnson2008}) reveals similarities. Like their German age-mates, Dutch-learning infants did not respond to the prosodic boundary marked by pitch change and final lengthening, indicating a crucial role of the pause in Dutch as well. Interestingly, Dutch – like German – IDS was found to show a lower mean and a lower range in F0 difference compared to American English IDS (\citealt{Fernald1989, Weijer1997}), suggesting again that the properties of the specific speech input relate to crosslinguistic differences in how infants process prosodic information and that with less pitch variation pauses may become a more crucial cue for the marking and the perception of prosodic boundaries.
\end{sloppypar}

At eight months, German infants’ sensitivity has developed to perceiving a pitch-lengthening cued boundary to such an extent that it can even be applied in a non-native language (\citealt{VanOmmen2020}). The result that French eight-month-olds’ boundary detection still depends on all three cues points to a delay in comparison to the German-learning infants. This is supported by van Ommen et al.’s experiment with French and German adults who did not differ in their discrimination of sequences with pitch and lengthening only. Apparently, French listeners catch up at one point. \citet{VanOmmen2020} argue that the language-specific differences at eight months might stem from a higher prosodic variability in German providing a larger basis to attend to prosodic details. French does not use prosodic characteristics to mark lexical stress. It uses prosody for phrasal stress; however, phrasal stress coincides with phrasal boundaries by default. Hereby, French is highly regular in the employment of prosodic cues. German, on the contrary, employs a larger variety of tonal and duration patterns at the phrasal as well as at the lexical level, and these are not strictly aligned to boundaries. This might explain why the German-learning infants show an earlier sensitivity to the specific cue combinations than their French peers.

The comparisons between the German, American English, Dutch, and French studies (that, notably, varied in the experimental paradigms: discrimination vs. segmentation) only give first, still vague indications of crosslinguistic differences in prosodic boundary cue weighting that support the assumption that perceptual attunement occurs in this domain. Future research using more comparable materials and methods across languages is necessary to provide a reliable picture of potential crosslinguistic effects of boundary perception and their development. In addition, it is not clear whether the few studies so far have used acoustic instantiations of the different cues that are typical for the specific language and typical for the infants’ input. Corresponding prosodic analyses of ADS and IDS in the respective languages are therefore needed to broaden our understanding of the early prosodic development. A further limitation in the interpretation of the results concerns the strength of single cues, which might differ between different cue constellations. There is evidence that marking of prosodic boundaries is subject to cue trading relations, that is, an interaction between the strength of the cues that mark the boundary with one cue being stronger when another cue is weaker (e.g., \citealt{Beach1991}). The cue insertion applied to the German stimuli was based on the acoustic parameters of pitch, lengthening, and pause that had been measured in natural sequences with a fully marked IPB. Although these values were already quite high (a pitch rise of \qty{212}{\hertz}/13.65 semitones, a final lengthening factor of 1.8, and a pause of \qty{500}{\ms} duration; cf. \citet{Peters2005a}\footnote{In a perception study with adult listeners, \citet{Peters2005a} implemented lengthening factors of 1.2, 1.4, and 1.6 and pauses with a duration continuum between \qty{50}{\ms} and \qty{890}{\ms} -- based on the values found in German ADS (\citealt{Peters2005PhonetMerk})}), they might be even higher in natural sequences with a boundary that is only marked by the subset of pitch and lengthening; in other words, when pause is missing, the other cues may be enhanced. Thus, we cannot exclude that six-month-olds might also be able to detect a boundary without pause if we had implemented stronger pitch and/or lengthening cues. Given that continuous stimulus manipulations can hardly be investigated in behavioral tasks in infants, the present experimental design did not consider cue trading relations. Still, we can conclude that developmental changes in behavior occur, since the eight-month-olds were able to detect the boundary using identical materials.

\subsection{Behavioral versus neurophysiological methods}
In the final section, we compare the present behavioral finding to those of a previous electrophysiological study. Using the very same stimuli, \citet{Holzgrefe-Lang2018} investigated boundary perception in eight- and also six-month-old German-learning infants by means of event-related potentials (ERP). In adults, the processing of IPBs with and without a pause evokes a specific ERP component, the so-called closure positive shift (CPS; e.g., \citealt{Holzgrefe-Lang2016, Steinhauer1999}), which is assumed to reflect the perception of a prosodic boundary. \citet{Holzgrefe-Lang2018} investigated whether an infant CPS can be elicited in response to different cue constellations. Specifically, they compared six- and eight-month-old infants’ brain response to stimuli containing either no boundary cue, a combination of pitch change and lengthening, or only a pitch cue. The ERPs in response to the latter condition did not differ from the condition without any boundary cues, but the combined occurrence of pitch change and final lengthening elicited a positivity that resembled the adult CPS in both age groups (\citealt{Holzgrefe-Lang2016}). Hence, the electrophysiological data suggests that six- and eight-month-old German infants do not differ in IPB perception, whereas the current HPP data provides no evidence that six-month-olds detect pitch and lengthening cued boundaries, but suggests a developmental change between the ages of six and eight months. Thus, prosodic boundary perception without the pause cue is evidenced earlier at the electrophysiological level (but see \citealt{Mannel2009, Mannel2013} for data that indicate that stimuli with neutralized pause cues would only elicit a CPS in children older than three years). In line with this asymmetry, there is ample evidence from other studies (\citealt{Friederici2007, Hohle2009, Schipke2012}) that a specific brain response may precede the corresponding behavioral response in the course of development. For instance, the recognition of the ambient language’s dominant stress pattern has been shown for four-month-old German learners using ERPs (\citealt{Friederici2007}), whereas a behavioral preference is evident only at six months (\citealt{Hohle2009}).

We assume that the diverging results across the different methods are due to different cognitive demands during testing. ERPs are measured on-line during infants’ passive listening, and hence do not depend on task demands or an overt response performance involving additional processing requirements (see \citealt{Mannel2008}). In the case of the ERP study by \citet{Holzgrefe-Lang2018}, the brain response indicating the perception of a prosodic boundary marked by pitch and lengthening occurs right after the presentation of the phrase-final syllable. Hence, the ERPs represent immediate processing responses evoked by the presence of specific boundary cues in the stimuli. In the HPP, the infants’ behavioral response is measured by the amount of listening time indicated by the infant’s head turn towards the side of presentation. The expectation to observe differences in listening times between the conditions in the HPP experiment is based on the assumption that a representation of the familiarization stimulus has been formed during the familiarization phase and that the stimuli presented during the test phase are mapped onto this representation. Establishing this representation requires that at least some memory traces survive the switch to the test phase of the experiment. Such a long-lasting memory component is not involved in the ERP paradigm. Nevertheless, the fact that the six-month-olds show effects in the HPP when the boundary is marked by the full set of boundary cues or by the subset of pause and lengthening suggests that memory requirements alone are not sufficient to explain the null effect in the condition with only lengthening and pitch. Rather, differences in the level of attention might account for the different outcomes across the ERP and HPP measurements as well as for the different outcomes across ages in the behavioral studies. Considering that listening times are an indicator of attention, the change that we observed between the six- and the eight-month-olds in the HPP data may suggest that infants have sharpened their attention towards pitch and lengthening, which are functionally relevant cues to German phrase boundaries, at the age of eight months.

            % Conclusion

\section{Conclusion}
To conclude, the present study provides evidence that German six-month-old infants are able to detect a major prosodic boundary characterized by the three main cues. This ability is crucial for the first steps in language acquisition as it equips the naïve learner with a tool to chunk the continuous speech stream into clauses. In a headturn preference procedure discrimination task, we found that for six-month-olds pitch and lengthening cues are not sufficient, but they need the pause cue. Boundary detection on the basis of combined relational prosodic cues like pitch changes and final lengthening shows up only by eight months. We argued that this behavioral change displays an enhancement in sensitivity that is reflected in a shift of attention to boundary markings that are functionally relevant in the ambient language. The ability to detect a boundary with the full as well as with a subset of cues enables syntactic parsing of not only major, but also minor, syntactic units. This ability is also necessary for the adult listener, especially in the case of structural ambiguities. Therefore, being able to detect these boundaries with their language-specific markers is essential to becoming an efficient processor of a given language.


\section*{Acknowledgements}
We thank T. Fritzsche for technical assistance in the Potsdam BabyLab. Thanks to A. Beyer, S. Fischer, B. Graf, T. Leitner, M. Orschinsky, and M. Zielina for their help in recruiting and testing the infants. We are grateful to R. Räling for stimuli production. Thanks to all parents and their children who participated in this study. Many thanks to S. Jähnigen and B. Smolibocki for help with the acoustic manipulation of the stimuli. 

\section*{Funding information}
This work was funded by the Deutsche Forschungsgemeinschaft (DFG, German Research Foundation), priority program SPP\,1234, project no. 18181517 (GZ HO 1960\slash 13-1 and FR2865\slash 2-1).

\printbibliography[heading=subbibliography,notkeyword=this]

\end{document}
