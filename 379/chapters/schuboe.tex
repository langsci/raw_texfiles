\documentclass[output=paper]{langscibook}
\ChapterDOI{10.5281/zenodo.7777526}

\author{Fabian Schubö\orcid{0000-0002-9956-2438}\affiliation{University of Stuttgart} 
and Sabine Zerbian\orcid{0000-0002-4631-369X}\affiliation{University of Stuttgart}}



\lsConditionalSetupForPaper{}

\title[Pre-boundary lengthening in German]{The patterns of pre-boundary lengthening in German}
\abstract{It has been observed for English that pre-boundary lengthening (PBL) is initiated on the syllable with main stress of the phrase-final word. Furthermore, the results of some studies suggest that the scope of PBL varies depending on the segmental composition of the phrase-final word. The present study investigates the scope of PBL in German. We report on a production experiment that tested for the position of word stress (penultimate vs. antepenultimate) and the presence\slash absence of an additional segment at the end of the word (CV.CV.CV vs. CV.CV.CVC) as predictors for the initiation of PBL. The results revealed that the initiation of PBL occurred on the stressed syllable across conditions. Furthermore, PBL was initiated later when a coda consonant was added to the words with penultimate stress, shifting the initiation point from the onset consonant to the following vowel of the stressed syllable. These observations suggest that the nuclear vowel of the main stress syllable serves as an anchor for PBL, but the initiation occurs earlier if the amount of material between the nuclear vowel and the prosodic boundary is limited. Thus, in line with findings from other languages, the scope of PBL in German is determined by the prosodic structure as well as the segmental composition of the phrase-final word.}


\begin{document}
\SetupAffiliations{mark style=none}
\maketitle

\section{Introduction}
Pre-boundary lengthening (PBL) has been identified as one of the major correlates of prosodic phrasing. That is, segments in phrase-final position are produced with longer duration than the same segments in phrase-medial position. This effect has been attested a stable correlate of boundary production and a reliable cue for boundary perception (e.g., \citealt{PetroneEtal2017}). The observation that PBL occurs in various languages, involving different prosodic systems, suggests that it might be a universal phenomenon (see, e.g., \citealt{Vaissiere83}). At the same time, it has been found that PBL is implemented language-specifically with reference to the given phonological system, which suggests that it must be learnt by speakers (e.g., \citealt{NakaiEtal2009} with regard to Northern Finnish).

It has been found that PBL operates primarily on the rime of the phrase-final syllable. Additionally, in some languages, PBL can affect the material of the penultimate syllable and even reach until the antepenultimate syllable of the phrase-final word. The scope of PBL may thus span several syllables. This scope is henceforth referred to as “PBL domain”. The extent of the PBL domain is connected to aspects of prosodic prominence. In languages with word stress, it has been observed that PBL is initiated on the last syllable with main stress preceding the boundary (e.g., \citealt{White2002} for British English). That is, the domain reaches from the main stress syllable to the end of the phrase-final word. Depending on the location of this syllable, the point of PBL initiation occurs earlier or later in a word so that the PBL domain varies in size. Furthermore, it has been found that the PBL domain is of fixed duration and overlaps with a part of the phrase-final word (e.g., \citealt{ByrdSaltzman2003} with reference to American English). Thus, the initiation of PBL differs as to the number (and type) of given segments immediately preceding the phrase boundary. If material is added to the end of the word, the initiation point is expected to shift further to the right. As for the temporal dynamics of the affected segments, a strong tendency towards a pattern of progressive lengthening has been observed across languages (e.g., \citealt{Kohler1983}, \citealt{Silverman1990} for German; \citealt{ByrdEtal2006} for American English). That is, the closer the location of the affected segment is to the prosodic boundary, the larger is the relative amount of lengthening. %In some languages, this pattern can be interrupted locally (e.g., Turk \& Shattuck-Hufnagel 2007 for American English).

The present study investigates the patterns of PBL in German. The aim is to delineate the PBL domain and the temporal dynamics of the affected material. We report on a production experiment that tested if (a) prosodic prominence and (b) the segmental composition of the phrase-final word affect the initiation of PBL.\footnote{A subset of the data gained in this study and a preliminary account were presented in \citet{SchuboeZerbian2020}. This subset includes the productions from 12 out of 24 subjects comprising the target words with penultimate word stress. The chapter at hand presents the full dataset and theoretical account.} The design controlled for the position of main word stress (penultimate vs. antepenultimate), the composition of the final syllable rime in words with penultimate stress (CV.ˈCV.CV vs. CV.ˈCV.CVC), and the presence\slash absence of a following prosodic boundary. The results revealed that (a) PBL is initiated on the stressed syllable across conditions, and that (b) the presence of an additional coda consonant in the words with penultimate stress shifts the initiation to the next segment. Furthermore, our findings suggest a weak form of progressive lengthening: The amount of PBL gradually increased by tendency, but also showed an interruption of this pattern in some instances on the material preceding the final rime. The final rime consistently showed an abrupt increase of lengthening on the nuclear vowel (independent of the stress pattern). This supports the view that the final rime has a central role in the implementation of a prosodic boundary by means of PBL (e.g., \citealt{TurkShattuck-Hufnagel2007}).

The paper is structured as follows: The following subsections summarize the relevant background from prior studies on PBL. This is followed by an introduction of the major aspects of German prosody and the statement of the research question and hypotheses. Section \ref{methods} details the methods employed in the production experiment and presents the steps in the data analysis. Section \ref{results} presents the results. Section \ref{discussion} discusses the findings, also addressing crosslinguistic aspects and ends with some concluding remarks.

\subsection{The initiation and scope of pre-boundary lengthening}\label{initiation}
The results from several studies suggest that the initiation of PBL is connected to a specific phonological constituent, such as a syllable or rime. In particular studies on English have provided evidence for the assumption that the initiation of PBL occurs on the last main stress syllable preceding the prosodic boundary (\citealt{White2002}, \citealt{TurkShattuck-Hufnagel2007}). This effect has been termed the Word Rime hypothesis, reflecting the observation that the first location of lengthening often occurs on the vowel or coda consonant of the main stress syllable. For example, \citet{White2002} tested words like \textit{SPECtre} and \textit{SPECtacle} in British English and found that PBL occurs on the coda consonant of the main stress syllable independent of its position. Similarly, \citet{TurkShattuck-Hufnagel2007} tested words like \textit{MIchigan} and \textit{JaMAIca} in American English and found that PBL occurs on the vowel of the main stress syllable. Their study also controlled for the presence\slash absence of an accent on the stressed syllable and found that PBL applies to the main stress syllable independent of the presence of an accent. This suggests that word stress, and not phrasal stress, is the relevant predictor. Several studies on different languages observed a pattern that is compatible  with the Word Rime hypothesis (e.g., \citealt{Kohler1983} for German; \citealt{Cambier-Langeveld1997} for Dutch; \citealt{Krull1997} for Estonian; \citealt{Cambier-Langeveld2000} for British English; \citealt{NakaiEtal2009} for Northern Finnish). However, the results from these studies do not provide independent evidence for this pattern, as alternative explanations could also account for the data (see below for Kohler's findings on German). Furthermore, it has been found that PBL can be initiated earlier in words with pre-final stress than in words with final stress, but the initiation point is not necessarily located on the stressed syllable (\citealt{Katsika2016} for Greek).

As for German, most studies that investigated PBL only tested for an effect on the final syllable (e.g., \citealt{PetersEtal2005, SchuboeEtal2015, PetroneEtal2017}). To our knowledge, only two prior studies addressed the extent of the domain of PBL in German: \citet{Kohler1983} analysed acoustic speech data from two speakers who produced the indefinite pronouns \textit{eine} [ˈaɪ.nə] (`one') and \textit{einige} [ˈaɪ.ni.gə] (`some') in utterance-medial and utterance-final position, respectively. He found that the initiation of PBL occurred on the stressed syllable in both words. This is compatible with the Word Rime hypothesis; yet, as the main stress syllable is initial in both of these words, it could also be the case that PBL operates on the entire prosodic word in German (see also \citealt{Silverman1990} and \citealt{TurkShattuck-Hufnagel2007} for this point). Moreover, it is unclear whether PBL also affects material preceding the utterance-final word. In order to link the initiation of PBL to the main stress syllable, we must exclude the possibility that the syllable preceding the disyllabic test word \textit{eine} also underwent PBL. \citet{Silverman1990} addressed these problems by comparing the durational patterns in a pair of trisyllabic words that differed only as to the location of main word stress. One of the words comprised penultimate stress (\textit{umLAgern} `to besiege') whereas the other one comprised antepenultimate stress (\textit{UMlagern} `to relocate'). In Silverman's study, two German native speakers were recorded, who produced these words six times in phrase-medial and phrase-final position. Silverman's analysis of the acoustic speech data revealed that PBL operates on the entire prosodic word in German. The presence of PBL on material preceding the main stress syllable must however be assumed to result from a different factor. Similar to this finding for German, experimental data on American English also revealed that PBL can occur on the antepenultimate syllable of a word with penultimate stress (\citealt{ChoEtal2013}). The variability found in these languages calls for further investigation of the patterns of PBL.

Some models do not posit a connection between PBL initiation and phonological constituency, but assume that the scope of PBL has fixed duration and overlaps with the phrase-final material (e.g., \citealt{ByrdSaltzman2003, ByrdEtal2005, ByrdEtal2006}). From this point of view, the PBL domain is aligned with the phrase boundary at its right edge whereas its left edge is determined by the phrase-final material in terms of the number and intrinsic length of given segments. In the framework of Articulatory Phonology (e.g., \citealt{BrowmanGoldstein1992, GoldsteinEtal2006}), this pattern has been accounted for as resulting from a clock-slowing gesture, the so-called π-gesture (e.g., \citealt{ByrdSaltzman2003, ByrdEtal2005, ByrdEtal2006}), which slows down the articulatory movements at the end of a prosodic phrase and thus leads to a lengthening effect. Such models entail that the initiation of PBL depends on the number of segments overlapped by the PBL domain or π-gesture (depending on the theoretical account). This predicts that PBL is initiated at a later point in a word that has an additional consonant in the final syllable coda than in the same word without the additional consonant. For example, we would expect a later point of PBL initiation in the word \textit{bananas} (comprising a plural suffix) than in the word \textit{banana} (without the plural suffix). We will refer to this assumption as the Overlap hypothesis.

In some languages, both phonological constituency and the type of phrase-final segments have an impact on the scope of PBL. For example, it has been found for Dutch that PBL mainly occurs on the rime of the final syllable; however, in case the rime comprises a vowel that is not expandable, such as a schwa, the initiation point occurs on preceding material (\citealt{Cambier-Langeveld1997}). A combination of phonological structure and segmental composition has also been observed in Japanese: \citet{SeoEtal2019} found that in disyllabic words PBL is initiated on the vowel of the penultimate syllable as long as it does not contain a coda consonant. In disyllabic words consisting of two CVN syllables, the initiation point occurs on the coda consonant of the penultimate syllable, which can be understood as a shift induced by additional phonetic content. Yet, the authors also attested an impact of the word prosodic structure: In case the words bore a lexical pitch accent anchored to the initial syllable, there was no effect of PBL on the final syllable.

These findings on the PBL domain are inconsistent in several ways. This particularly applies to findings on American English: For example, while \citet{TurkShattuck-Hufnagel2007} observed that PBL has a large scope reaching until pre-final main stress syllables, \citet{ByrdEtal2006} found that the scope is limited and that there is no interaction with stress. Furthermore, \citet{ByrdRiggs2008} observed that the initiation of PBL was shifted to pre-final stressed syllables only by one out of three subjects. Furthermore, \citet{ChoEtal2013} observed the presence of PBL on the antepenultimate syllable in a word with penultimate stress. These inconsistencies call for further research on the initiation of PBL with reference to the position of main word stress. The reason for these inconsistent findings might result from differences in the methods used in prior studies, as the studies differed with regard to aspects such as stimuli, type of data collection, and number of participants.

\subsection{The amount and distribution of pre-boundary lengthening}\label{amount}
Various studies observed a pattern of progressive lengthening towards the phrase boundary (e.g., \citealt{Kohler1983}, \citealt{Silverman1990} for German; \cite{Berkovits1994} for Hebrew; \citealt{ByrdEtal2006} for American English; \citealt{NakaiEtal2009} for Northern Finnish; \citealt{SeoEtal2019} for Japanese). That is, the amount of PBL progressively increases from one segment to the next in the PBL domain, so that the effect is strongest on the final segment. This pattern might be affected by the expandability potential of specific segments; for example, it has been found that oral stops involve a lower amount of PBL than other types of consonants in American English (\citealt{Klatt1976}), Hebrew (\citealt{Berkovits1993a,Berkovits1993b}), and Dutch (\citealt{HofhuisEtal1995}). It has also been found that, once initiated, PBL can be interrupted on intermediate elements (\citealt{Cambier-Langeveld1997} for Dutch; \citealt{TurkShattuck-Hufnagel2007} for American English). For example, \citet{TurkShattuck-Hufnagel2007} found that American English words with antepenultimate stress involved PBL on the rime of the stressed syllable and on the rime of the final syllable, but not on the intermediate material. They also observed ``a weaker version of progressive lengthening" (\citeyear[459]{TurkShattuck-Hufnagel2007}), which entails that the amount of PBL globally increases from left to right, but this increase can be interrupted locally, resulting in a lower amount of PBL on a segment in comparison to the amount of the prior segment (thus, it progresses with a “medial dip”).

The progressive lengthening pattern often involves a comparatively large increase of lengthening on the phrase-final syllable, leading to a larger slope of progressive lengthening in this position (e.g., \citealt{Klatt1975}, \citealt{Kohler1983}, \citealt{Berkovits1994}, \citealt{TurkShattuck-Hufnagel2007}, \citealt{SeoEtal2019}).  For example, \citet{Kohler1983} found that German words with pre-final stress involve a considerably larger amount of PBL on the final syllable (87--176\%) than on the penultimate syllable (15--31\%). The data from some production studies suggest that the large amount of increase occurs on the rime of the final syllable: For example, in their study on American English, \citet{TurkShattuck-Hufnagel2007} observed 15 percent of lengthening on the onset of the final syllable followed by 71 percent of lengthening on the following syllable rime (mean percentages based on the data from four subjects). Furthermore, \citet{SeoEtal2019} argue that the final rime constitutes the major unit for the distribution of PBL in Japanese, showing, among other things, that the amount of lengthening on the rime of an open syllable (CV) is comparable to the amount on the rime of a closed syllable (CVN). %These observations suggest that the final rime constitutes a predictor for the distribution of PBL across languages.

\subsection{German prosody}
Before turning to the present study, we will briefly outline the prosodic properties of German. German prosody closely resembles the prosodic system of English. Syllables may be open or closed and can contain single consonants or consonant clusters both in onset and in coda position. With a few exceptions, consonant clusters comply with the Sonority Sequencing Principle (SSP), that is, the degree of sonority decreases towards the edges of a syllable (see, e.g., \citealt{Selkirk1984} for this principle). Voiceless alveolar and post-alveolar obstruents can occur at the peripheries of a consonant cluster in violation of the SSP. Voiced coda obstruents undergo devoicing. The rhotic is usually vocalized as [ɐ] in coda position. Several phonotactic constraints apply, including the prohibition of the glottal fricative and the palatal glide in coda position (see, e.g., \citealt{Hall1992} for an overview).

Word stress is assigned to either of the last three syllables in a morphologically simple word; yet, polysyllabic words with final stress are rare in German. In words with three or more syllables, there is a tendency for penultimate stress if the penultimate syllable is closed, and for antepenultimate stress if the penultimate syllable is open (\citealt{Wiese1996}). According to \citet{Delattre1965}, German exhibits a tendency towards word-initial stress, but trisyllabic words do not statistically differ as to the frequency of penultimate and antepenultimate stress. The most prevalent phonetic correlate of word stress in German is duration: Vowels and consonants exhibit longer duration in stressed than in unstressed syllables (e.g., \citealt{DogilWilliams1999}). %Unlike in English, vowels are not reduced in unstressed positions in German. 

Phrasal stress is assigned with respect to syntactic structure, rhythmic patterns, information structural conditions, and other meaning-related aspects (see, e.g., \citealt{Truckenbrodt2006}). Phrasal stress is realized by a pitch accent aligned with a main stress syllable as well as by longer duration. Different systems employing Tone and Break Indices (ToBI) are offered in the literature (e.g., \citealt{GriceEtal2005, Peters2018}). For the annotation of tonal events, the present study adopts the system proposed in \citet{GriceEtal2005}, referred to as German Tone and Break Indices (GToBI). This system assumes a set of six pitch accents (L*, H*, L+H*, L*+H, H+L*, H+!H*), two phrase tones (L-, H-), and two boundary tones (L\%, H\%). Nuclear stress is usually assigned to the rightmost phrasal stress position and implemented by means of a pitch accent with relatively larger prominence than the preceding ones in the utterance. The nuclear pattern at the end of a prosodic phrase (i.e., the last pitch accent in combination with the following phrase and\slash or boundary tone) may express specific pragmatic meanings (see, e.g., \citealt{GriceEtal2005}). For example, a pattern involving (L+)H* H-(\%) is often employed for expressing incompleteness whereas a pattern involving L+H* L-\% is often employed for expressing a contrastive assertion (\citealt{GriceEtal2005}: 71).

Two levels of prosodic phrasing are distinguished in GToBI, referred to as Intonational Phrase (IP) and intermediate phrase (ip), respectively. The former involves a relatively stronger and the latter a relatively weaker prosodic boundary at its right edge. The prosodic boundaries on both levels can be expressed by means of boundary tones, PBL and pauses (see \citealt{PetroneEtal2017} for a study on the production and perception of these cues). The pitch movements induced by boundary tones can involve rising, falling, or falling-rising patterns on the material between the last pitch accent and the end of the phrase. In utterance-medial position, they usually involve a rising or falling-rising pattern, whereas in utterance-final position they usually involve a falling pattern (see, e.g., \citealt{Truckenbrodt2002, Truckenbrodt2007}). As stated above, PBL initiation was found on the last main stress syllable preceding the boundary (\citealt{Kohler1983}), but there is also some evidence for lengthening of the prior syllable (\citealt{Silverman1990}). The amount of lengthening has been found to increase progressively towards the end of the prosodic phrase (\citealt{Kohler1983}, \citealt{Silverman1990}).

\subsection{Research questions and hypotheses}
The present study investigates the patterns of PBL in German speech production, addressing the question of what determines the initiation of PBL. Specifically, it is tested if the initiation of PBL is affected by (a) the position of main word stress and\slash or (b) the number of segments in the phrase-final word. The respective hypotheses are stated in (\ref{1}). The statement in (\ref{1}a) captures the Word Rime hypothesis, which predicts that PBL begins on the last main stress syllable before the prosodic boundary. Thus, if this hypothesis holds, it is expected that words with different stress positions differ with regard to the point of PBL initiation and the scope of the PBL domain. Attesting this pattern for German would strengthen the assumption that prosodic prominence serves as a predictor for PBL initiation in languages with a stress-based prosodic system. The statement in (\ref{1}b) is in compliance with the Overlap hypothesis, which entails that the scope of PBL is of fixed duration and overlaps with a portion of the phrase-final word. Thus, if this hypothesis holds, it is expected that additional material at the end of the word leads to a shift of PBL initiation to a later point, such as the following segment or syllable. It is also possible that both hypotheses hold, in which case the initiation of PBL would change in accordance with the position of main word stress and at the same time would shift to a later position if additional material is present at the end of the word.

\ea \label{1}
    \ea PBL is initiated on the nuclear vowel of the main stress syllable and persists until the end of the phrase-final word (Word Rime hypothesis).
    \ex The initiation of PBL is delayed if a coda consonant is added to the final syllable (Overlap hypothesis).
    \z
\z

Furthermore, this study addresses the relative amount of lengthening among the segments affected by PBL. According to the Progressive Lengthening hypothesis (\ref{2}), it is expected that the amount is relatively larger on segments that are relatively closer to the end of the prosodic phrase; however, given prior findings from other languages (see \sectref{amount}), this pattern might not be applied consistently so that the relative amount of lengthening locally decreases or lengthening is completely absent in intermediate positions.

\ea
The amount of PBL progressively increases towards the end of the prosodic phrase (Progressive Lengthening hypothesis).
\label{2}
\z

The predictions were tested by conducting a production experiment, which is reported on in the next section. The experiment involved the elicitation and audio-recording of read speech in a laboratory setting. Given the inconsistencies found in prior studies (see \sectref{initiation}), we chose to employ a carefully controlled design.

\section{Methods}\label{methods}
\subsection{Stimuli}\label{stimuli}
The stimuli employed in the production experiment were controlled for the position of main word stress, the presence\slash absence of a final coda consonant, and the presence\slash absence of a prosodic boundary. We employed two types of target words, which were both trisyllabic proper names. The first type comprised CV.ˈCV.CV structure, involving penultimate word stress (e.g., RaMOna). These words were elicited under two conditions affecting the final rime: In one condition, they were in accusative case and retained their structure. In the other condition, they were in genitive case and comprised a suffix \textit{-s}, which is implemented as a voiceless alveolar fricative in the word-final coda, yielding a CV.ˈCV.CVC structure (e.g., RaMOnas). The second type of target words comprised antepenultimate main word stress (e.g., KArolin). These words varied with regard to the presence of a coda consonant in the penultimate and\slash or final syllable. Given that proper names with antepenultimate stress and the same internal syllable structure as the words with penultimate stress are rare in German, we decided to also include words that deviated with regard to the presence of coda consonants.

Prosodic boundaries after the target words were elicited by means of lists of the type [N1 or N2 and N3], which can be interpreted as comprising a left-branching structure [[N1 or N2] and [N3]] or a right-branching structure [[N1] or [N2 and N3]]. The target words were in position N2. Prior studies showed that speakers disambiguate such lists by means of prosodic phrasing, inserting a boundary after N2 in the left-branching case and after N1 in the right-branching case (e.g., \citealt{KentnerFery2013}, \citealt{PetroneEtal2017}, \citealt{huttenlauchetal2021} for German; \citealt{Wagner2005}, \citealt{TurkShattuck-Hufnagel2007} for English; see also \citetv{chapters/huttenlauch} and \citetv{chapters/wellmann}). The lists were medially embedded in carrier sentences. The sentences were preceded by a short context story. The branching structure was indicated by setting the list in italics and underlining its sub-constituents. An example item is given in (\ref{3}). The target word of the pair in (\ref{3}a) involves penultimate stress and lacks a final coda consonant (\textit{Ramona}). The first sentence comprises a right-branching structure, which renders the target word in phrase-medial position, and the second sentence comprises a left-branching structure, which renders the target word in phrase-final position. The target word in (\ref{3}b) involves penultimate stress and a final coda consonant (\textit{Ramonas}). In this case, the sentence comprises an elliptic right-node-raising construction. Finally, the target word in (\ref{3}c) involves antepenultimate stress (\textit{Karolin}). Here, the structure of the sentences is the same as in (\ref{3}a), but the first two names are exchanged, so that the name with antepenultimate stress occurs in N2 position.\largerpage

\ea\label{3}
    \ea Ich werde \textit{\uline{Karolin} oder \uline{Ramona und Peter}} einladen.\\    
    Ich werde \textit{\uline{Karolin oder Ramona} und \uline{Peter}} einladen.
    \glt ‘I will invite Karolin or Ramona and Peter.’
    
    \ex Ich werde \textit{\uline{Karolins} oder \uline{Ramonas und Peters Freunde}} einladen.\\    
    Ich werde \textit{\uline{Karolins oder Ramonas} und \uline{Peters Freunde}} einladen.
    \glt ‘I will invite Karolin's or Ramona's and Peter's friends.’
    
    \ex Ich werde \textit{\uline{Ramona} oder \uline{Karolin und Peter} einladen.}\\    
    Ich werde \textit{\uline{Ramona oder Karolin} und \uline{Peter}} einladen.
    \glt ‘I will invite Ramona or Karolin and Peter.’
    \z
\z

The context stories consisted of three to four sentences, as illustrated in (\ref{4}). The story in (\ref{4}a) preceded the sentences in (\ref{3}a) and (\ref{3}c). Since the object in (\ref{3}b) had a different structure, the context story was slightly modified for reasons of coherence, as illustrated in (\ref{4}b).

\ea\label{4}
    \ea Max feiert bald seinen Geburtstag. Er hat bereits seine besten Freunde eingeladen. Nun überlegt er, wen er noch einladen soll, Max denkt:
    \glt ‘Max will soon celebrate his birthday. He already invited his best friends. Now he is wondering who else he could invite. Max is thinking:’
    \ex Max feiert bald seinen Geburtstag. Er hat bereits seine besten Freunde eingeladen. Nun überlegt er, auch noch deren Freunde einzuladen. Max denkt:
    \glt ‘Max will soon celebrate his birthday. He already invited his best friends. Now he is considering to invite their friends as well. Max is thinking:’
    \z
\z

In order to facilitate the interpretation of the lists, pictures with drawings of persons grouped according to the constituent structure were presented below the target sentences. Figure \ref{fig:images} illustrates the pictures for the sentences in (\ref{3}a). The persons were marked with the initial letter of the respective name in the list (here, \textit{Karolin}, \textit{Ramona}, and \textit{Peter}). The picture in (a) represents the right-branching structure, where the target word is in phrase-medial position, and the one in (b) represents the left-branching structure, where the target word is in phrase-final position. 

\begin{figure}
    \begin{subfigure}[b]{.45\linewidth}\centering%
    \includegraphics[width=3.2cm]{figures/Image_RB.png}
    \caption{right-branching structure}
    \end{subfigure}\hfill
    \begin{subfigure}[b]{.45\linewidth}\centering%
    \includegraphics[width=5cm]{figures/Image_LB.png}
    \caption{left-branching structure}
    \end{subfigure}
    \caption{Examples of pictures showing persons grouped according to the constituent structure of the lists. Ramona (R) is the target.}
    \label{fig:images}
\end{figure}

\subsection{Design, subjects and procedure}
As described above, the design involved six conditions (2 phrase positions $\times$ 2 stress positions + 2 coda conditions). We employed six names with penultimate and six names with antepenultimate stress as target words and created twelve items of the sort presented in \sectref{stimuli} (each name occurred in two different items). This yielded a total of 72 target expressions, which are given in Appendix~\ref{appendix:schuboe:a}. Furthermore, we employed 90 filler expressions. The stimuli were pseudo-randomized and presented to the subjects in a within-subject design. We audio-recorded 24 native speakers of German from the Stuttgart area aged between 18 and 25 years. This yielded 288 productions per condition (12 items $\times$ 24 subjects) and 1,728 productions in total ($288 \times 6$ conditions). The recording sessions took place in a sound-attenuated booth at the University of Stuttgart and lasted 44 minutes on average across subjects. The recordings were made digitally with a Sennheiser ME 62 microphone and stored on hard disk in Waveform Audio File Format (mono sound, 44.1 kHz sampling frequency, 16-bit resolution).

The recording session was self-controlled by the subject using a computer mouse. During the session, the subject and the experimenter were sitting at a table, separated by a shoulder-high screen. The stimuli were presented to the subject one by one on a display screen. The subject was instructed to first read the context and the target sentence silently and then to decide which of the two interpretations was indicated by the underlining of the constituents and the picture given below. After that, the subject started a recording phase of six seconds by clicking on a button on the display screen and then read the target sentence out loud. If the subject was not satisfied with the production, he\slash she could repeat it after clicking on the recording button again. In this case, the recording of the prior production was deleted. The subject was allowed to repeat the production of a sentence as often as he\slash she wanted to. After the recording, the subject mouse-clicked a different button on the display screen in order to move on to the next stimulus. The software used for this procedure was written in Python by the first author.\largerpage

Furthermore, the elicitation procedure involved a communication task (similar to \citealt{PetroneEtal2017}). The subject was instructed to produce the sentences in such a way that the experimenter could understand which of the two branching structures was expressed. The experimenter saw a printed list with both alternatives for each sentence and had to assign the production to one out of the two alternatives by checking a box on the list. The subject did not see the experimenter's decision and no feedback was given. The experimenter identified the correct structure in 97 percent of cases across subjects and conditions. This procedure was supposed to make the subjects produce the disambiguating prosodic cues more reliably, as it has been found that speakers use prosodic cues for disambiguation in a consistent way only when they are aware of a need for disambiguation (e.g., \citealt{SnedekerTrueswell2003}, \citealt{SchuboeEtal2015}). We acknowledge that this procedure might have elicited a focus intonation pattern, as the participants were required to communicate one out of two possible structures. Given that lengthening is also a correlate of focus in German (e.g., \citealt{FeryKuegler2008}), we cannot exclude the possibility that the participants produced focus-related lengthening in addition to PBL.

Preceding the recording session, the subjects were familiarized with the type of sentences and the ambiguity involved. They saw an example for each branching structure (including the respective underlining of the constituents and the picture) and read a short text describing the meaning difference. At the beginning of the recording session, the subjects produced five sentences that were not part of the test material, but involved the same type of ambiguity and indication of structure. These productions did not enter the analysis and were deleted after the recording session.

In a prior study on German (\citealt{PetroneEtal2017}), it was found that an experiment design such as the one employed in the present study leads to a consistent production of IP boundaries (rather than ip boundaries). We chose to employ a design of this type in order to consistently elicit IP boundaries. Designs with less control might cause variation in the type of prosodic boundaries. Given that the amount of PBL is expected to be larger at relatively stronger boundaries (e.g., \citealt{PetersEtal2005}), such a variation would be problematic.

\subsection{Pre-analysis: Pitch accents and boundary tones}
A pre-analysis of the intonation patterns on the target words was performed using the GToBI system (\citealt{GriceEtal2005}). This was applied in order to verify that the productions were consistent with regard to phrasal prominence on the target words and that they involved the expected phrasing patterns. The first author (who is highly familiar with the intonation patterns of German and the GToBI system) manually annotated each production as to the presence and type of pitch accent on the target word. Furthermore, the presence\slash absence and type of prosodic boundary immediately following and immediately preceding the target word was annotated by the first author. The types of pitch accent and boundary tones were identified based on the local shape of the F0 contours, the global F0 pattern, and the auditory impression of the tonal event. In 30 productions, the target word was unaccented (19 from the right-branching and 11 from the left-branching condition). These tokens were excluded from the subsequent analyses, as the absence of an accent might have affected segment duration. Furthermore, 40 productions (39 from the right-branching and 1 from the left-branching condition) involved a prosodic boundary immediately preceding the target word (i.e., after the conjunction \textit{oder} `or'). These tokens were also excluded from the subsequent analyses, as phrase-initial strengthening might have affected the duration of the word-initial segments. Altogether, 1,658 productions were included in the subsequent analyses, which corresponds to 96 percent of the collected tokens.

\tabref{tab:Tab1} shows the frequency and type of prosodic boundaries realized under the right- and left-branching condition, respectively, indicated by the boundary tone types from the GToBI system. As expected, the majority of productions from the right-branching condition did not involve a prosodic boundary after the target word whereas the majority of productions from the left-branching condition did involve a prosodic boundary in this position. In the latter case, the vast majority of instances ended with an H\% boundary tone, indicating the presence of an Intonation Phrase (IP) boundary with a continuation rise at the right edge. Also, there were 20 productions from the right-branching conditions that did involve a prosodic boundary after the target word (16 with H\%, 3 with H-, and 1 with L\%; see \tabref{tab:Tab1}) and 15 productions from the left-branching condition that did not involve a prosodic boundary after the target word. These productions were included in the subsequent analyses and treated as phrase-medial or -final in accordance with the presence\slash absence of a boundary tone after the target word. Thus, the categorization of the productions was based solely on their surface prosodic pattern and not on the condition under which they were elicited.

\begin{table}
    \caption{Frequency of boundary tone types in the right-branching and left-branching condition\label{tab:Tab1}}
    \begin{tabular}{lrrrrr}
    \lsptoprule
    Boundary tone & H\% & H- & L\% & L- & none \\
    \midrule
    Right-branching & 16 & 3 & 1 & 0 & \textbf{786} \\
    Left-branching & \textbf{821} & 2 & 10 & 4 & 15 \\
    \lspbottomrule
    \end{tabular}
\end{table}

The GToBI annotations revealed that the vast majority of boundary tones signal an IP boundary (i.e., H\% and L\%). This is in line with the findings by \citet{PetroneEtal2017}, who used a similar design and observed a consistent use of IP boundaries. Our data is thus largely consistent with regard to the phrasing level, which avoids variation in PBL resulting from different types of boundaries.\largerpage

\tabref{tab:Tab2} presents the frequency of pitch accent types on the target word in phrase-medial position (productions not involving a boundary tone after the target word) and phrase-final position (productions involving a boundary tone after the target word). The most frequent type of pitch accent in phrase-medial position was H* ($n=500$; 62 percent), followed by L+H* ($n=191$; 24 percent), and L*+H ($n=92$; 11 percent). In phrase-final position, the most frequent type was L+H* ($n=567$; 66 percent), followed by L*+H ($n=241$; 28 percent). Only few monotonal pitch accents occurred in this position. Thus, the target words in the different phrase positions involve different tendencies as to the distribution of pitch accent types. We acknowledge that these tendencies might constitute a confound that could affect the duration of specific segments in the target words (in particular in the stressed syllable and the following one). Testing for a correlation of pitch accent type and segment or syllable duration is beyond the scope of this study and should be addressed in future research. Moreover, inter-speaker differences with regard to these aspects should be explored.

\begin{table}
    \caption{Frequency of pitch accent types in phrase-medial and phrase-final position\label{tab:Tab2}}
    \begin{tabular}{lrrrrr}
    \lsptoprule
    Pitch accent & H* & L* & L+H* & L*+H & unclear\\
    \midrule
    Phrase-medial & 500 & 12 & 191 & 92 & 6 \\
    Phrase-final & 23 & 25 & 567 & 241 & 1 \\
    \lspbottomrule
    \end{tabular}
\end{table}
    
The penultimate stress words with and without a final coda consonant (e.g., \textit{Ramona} vs. \textit{Ramonas}) were elicited by means of different syntactic structures: The words with a final coda consonant were part of an elliptic right-node raising construction, which was not the case for the words without a final coda consonant. In order to check if the different syntactic constructions might have induced different pitch accent patterns, we compared the frequency of the most common pitch accent types (H*, L+H*, and L*+H) realized on the words with penultimate stress with and without a final coda consonant in phrase-medial and -final position, respectively. \tabref{tab:Tab3} indicates that both conditions show the same tendencies with regard to the distribution of pitch accent types for each phrase position. In phrase-medial position, the most common type in both conditions is H*, followed by L+H*. Only relatively few instances of L*+H occur in this position. The words with a final coda consonant were less often produced with an H* and slightly more often produced with an L+H*, but the relative distribution among the pitch accent types is similar for both conditions. In phrase-final position, the most common type in both conditions is L+H* and fewer instances of L*+H were produced on the words with a final coda consonant, but the relative distribution is similar.\largerpage[2]

\begin{table}[H]
    \caption{Frequency of H*, L+H*, and L*+H in phrase-medial and phrase-final position for the penultimate stress words in elliptic constructions\slash coda present and non-elliptic constructions\slash coda absent}
    \begin{tabular}{lcccccc}
    \lsptoprule
     & \multicolumn{3}{c}{Phrase-medial} &  \multicolumn{3}{c}{Phrase-final} \\\cmidrule(lr){2-4}\cmidrule(lr){5-7}
    Pitch accent & H* & L+H* & L*+H & H* & L+H* & L*+H \\
    \midrule
    Coda absent  & 184 & 76 & 12 & 4 & 190 & 72 \\
    Coda present & 133 & 89 & 27 & 10 & 220 & 59\\
    \lspbottomrule
    \end{tabular}
    \label{tab:Tab3}
\end{table}

\subsection{Analysis}
\subsubsection{Acoustic speech segmentation}
We manually annotated the segment boundaries of the target words based on spectrographic and waveform information, following the guidelines for acoustic speech segmentation provided by \citet{TurkEtal2006}. These guidelines suggest that the locations of the segment boundaries should be identified based on abrupt spectral changes caused by the onsets and releases of consonantal constrictions (rather than by voicing criteria). Thus, the segmentation procedure primarily relied on acoustic landmarks that were caused by the consonantal constriction gestures. For example, sibilants were segmented based on the onset and offset of frication energy whereas nasal stops were segmented based on abrupt spectral changes at the points of closure and release, marking an abrupt decrease of energy in comparison to the surrounding vowels. As a secondary cue, the onset and offset of F2 energy was taken into account, which the guidelines suggest particularly with regard to weak fricatives and pre-pausal or utterance-final vowels. For visual inspection and annotation, we employed the acoustics analysis software Praat (\citealt{BoersmaWeenink2019}). After the annotation process was completed, the duration values of the intervals defined by the identified consonantal constrictions were extracted by means of an automated procedure.

\subsubsection{Statistics}
For statistical analyses, we employed the software environment R (\citealt{RCoreTeam2019}) and the \texttt{lme4} package (\citealt{BatesEtal2015}). Separate linear mixed effects (LME) models were fitted to the data for each type of target word (penultimate stress with coda, penultimate stress without coda, antepenultimate stress). The models account for \textsc{duration} as a function of \textsc{phrase position} (levels: medial, final) and \textsc{segment position} (with interaction term). The levels of \textsc{segment position} included all relevant combinations of the syllable position in the word (antepenultimate, penultimate, final) and the internal syllable structure (onset, nucleus, coda). The interaction term is motivated based on the assumption that the amount of PBL is relatively larger on segments that are closer to the prosodic boundary. As random factors, we included intercepts and slopes for \textsc{subject} and intercepts for \textsc{item}. Due to non-convergence, the slopes for \textsc{subject} were removed from the model fitted to the data for the words with antepenultimate stress. Significance between the levels of \textsc{phrase position} was tested at each \textsc{segment position} by using the \texttt{multcomp} package (\citealt{HothornEtal2008}). The results are presented below. Model outputs for all coefficients are given in Appendix~\ref{appendix:schuboe:b}.

\section{Results}\label{results}
\subsection{The scope of lengthening}
Figure \ref{fig:pen_short} presents the duration data for the target words with penultimate stress and CV.ˈCV.CV structure (e.g., RaMOna). The light boxes show the data from the productions in phrase-medial position and the dark boxes show the data from the productions in phrase-final position. The codes above the plots indicate the significance level of the \textit{p}-values obtained by the post-hoc comparisons. The boxplots for the segments in the antepenultimate syllable (C1 and V1) do not suggest a significant difference between the two phrase positions, and the respective comparison did not yield a significant effect (C1: $\beta=3.6,\allowbreak \text{SE}=1.9,\allowbreak t=1.9,\allowbreak p=0.0598$; V1: $\beta=-0.1,\allowbreak \text{SE}=1.9,\allowbreak t=-0.5,\allowbreak p=0.616$). The comparison for C1 was however near significant. The model estimated a longer duration of 4 ms in phrase-final position. The boxplots for the consonants and vowels of the penultimate and final syllable clearly suggest a longer duration in phrase-final position, and the comparisons yielded highly significant effects, respectively (C2: $\beta=10.7,\allowbreak \text{SE}=1.9,\allowbreak t=5.6,\allowbreak p<0.001$; V2: $\beta=13.9,\allowbreak \text{SE}=1.9,\allowbreak t=7.3,\allowbreak p<0.001$; C3: $\beta=8.1,\allowbreak \text{SE}=1.9,\allowbreak t=4.3,\allowbreak p<0.001$; V3: $\beta=80.1,\allowbreak \text{SE}=1.9,\allowbreak t=42.2,\allowbreak p<0.001$). Thus, \textsc{phrase position} affected \textsc{duration} in the last four segments, causing an increase in phrase-final position.

\begin{figure}%[H]%[htp]
    \centering
    \includegraphics[width=10cm]{figures/boxplts_pen_short_revised.png}
    \caption{Duration plots (ms) with confidence intervals for the segments of the penultimate stress words with CV.ˈCV.CV structure (e.g., RaMOna) across subjects (*** $p<0.001$, ** $p<0.01$, * $p<0.05$, n.s. not significant; light boxes: phrase-medial, dark boxes: phrase-final)}
    \label{fig:pen_short}
\end{figure}

Figure \ref{fig:pen_long} presents the results for the penultimate stress words with a final coda consonant (e.g., RaMOnas). The post-hoc comparisons did not yield a significant effect for the segments of the antepenultimate syllable (C1: $\beta=2.2,\allowbreak \text{SE}=2.3,\allowbreak t=1,\allowbreak p=0.323$; V1: $\beta=-0.5,\allowbreak \text{SE}=2.3,\allowbreak t=-0.2,\allowbreak p=0.831$). There also was no significant effect for the onset consonant of the penultimate syllable (C2: $\beta=1.9,\allowbreak \text{SE}=2.3,\allowbreak t=0.8,\allowbreak p=0.408$). The plots for the vowel of the penultimate syllable and all following segments clearly suggest longer duration in phrase-final than in phrase-medial position, and the respective comparisons yielded a significant effect (V2: $\beta=12.9,\allowbreak \text{SE}=2.3,\allowbreak t=5.7,\allowbreak p<0.001$; C3: $\beta=5.3,\allowbreak \text{SE}=2.3,\allowbreak t=2.4,\allowbreak p<0.0183$; V3: $\beta=65,\allowbreak \text{SE}=2.3,\allowbreak t=28.9,\allowbreak p<0.001$; C4: $\beta=56.7,\allowbreak \text{SE}=2.3,\allowbreak t=25.2,\allowbreak p<0.001$). Thus, \textsc{phrase position} affected \textsc{duration} in the last four segments, causing an increase in phrase-final position.


\begin{figure}%[H]%[htp]
    \centering
    \includegraphics[width=11cm]{figures/boxplts_pen_long.png}
    \caption{Duration plots (ms) with confidence intervals for the segments of the penultimate stress words with CV.ˈCV.CVC structure (e.g., RaMOnas) across subjects (*** $p<0.001$, ** $p<0.01$, * $p<0.05$, n.s. not significant; light boxes: phrase-medial, dark boxes: phrase-final)}
    \label{fig:pen_long}
\end{figure}

The results for the target words with antepenultimate stress (e.g., KArolin) are illustrated in Figure \ref{fig:ant}. The coda consonant of the penultimate syllable (C3) was present in only one of the target words (VAlentin) and the coda consonant of the final syllable (C5) was absent in one of the target words (GIsela). The remaining consonants were present in all target words. The plots for the initial onset consonant (C1) suggest a slight tendency towards longer duration in phrase-final position than in phrase-medial position, but the comparison did not yield a significant effect (C1: $\beta=2.4,\allowbreak \text{SE}=1.8,\allowbreak t=1.3,\allowbreak p=0.187$). The plots for all following segments (V1--C5) clearly suggest longer duration in phrase-final position than in phrase-medial position, and the comparisons yielded a significant effect, respectively (V1: $\beta=7.6,\allowbreak \text{SE}=1.8,\allowbreak t=4.1,\allowbreak p<0.001$; C2: $\beta=3.6,\allowbreak \text{SE}=1.8,\allowbreak t=2,\allowbreak p=0.0456$; V2: $\beta=6.7,\allowbreak \text{SE}=1.8,\allowbreak t=3.7,\allowbreak p<0.001$; C3: $\beta=9.2,\allowbreak \text{SE}=4.4,\allowbreak t=2.1,\allowbreak p<0.0359$; C4: $\beta=6.8,\allowbreak \text{SE}=1.8,\allowbreak t=3.7,\allowbreak p<0.001$; V3: $\beta=51.8,\allowbreak \text{SE}=1.8,\allowbreak t=28.4,\allowbreak p<0.001$; C5: $\beta=49.9,\allowbreak \text{SE}=2,\allowbreak t=25,\allowbreak p<0.001$). Thus, \textsc{phrase position} affected \textsc{duration} in all three syllables, causing longer duration in phrase-final position than in phrase-medial position.

\begin{figure}%[H]%[htp]
    \centering
    \includegraphics[width=12cm]{figures/boxplts_ant.png}
    \caption{Duration plots (ms) with confidence intervals for the segments of the antepenultimate stress words (e.g., KArolin) across subjects (*** p<.001, ** p< .01, * p<.05, n.s. not significant; light boxes: phrase-medial, dark boxes: phrase-final; C3 was present in only one target word)}
    \label{fig:ant}
\end{figure}

\subsection{Progressive lengthening}
This sub-section presents the results for the distribution of PBL in the target words based on the estimates provided by the LME models. Figure \ref{fig:comp} presents the amounts of durational increase in phrase-final compared to phrase-medial position in percentages for the three types of target words. The percentages were calculated as follows: pct = (coefficient value × 100) \div \ intercept value. The dashed line presents the amount of increase for the words with penultimate stress and CV.ˈCV.CV structure (e.g., RaMOna). The initial onset consonant (C1) involves an increase of 6 percent. The following vowel involves a decrease of 1 percent. After that, on the penultimate syllable, the amount of increase rises to 20 percent on C2 and then slightly falls to 16 percent on V2. On the final syllable, the amount of increase slightly rises to 17 percent on C3 and then undergoes a large increase on the final vowel (V3), reaching 81 percent. The dash-dotted line presents the results for the words with penultimate stress and CV.ˈCV.CVC structure (e.g., RaMOnas). The pattern on the antepenultimate syllable is similar to the prior case. On the penultimate syllable, there is a relatively small increase on C2 (4 percent), followed by a larger increase on V2 (14 percent). On the final syllable, the increase slightly decreases to 12 percent on C3 and then shows a large rise on V3 (63 percent) and further rises on C4 (72 percent). The solid line presents the results for the words with antepenultimate stress (e.g., KArolin). In this case, the increase gradually rises from C1 (5 percent) to C3 (18 percent). After that, there is a large increase on V3 (67 percent) and C4 (104 percent).\largerpage[-2]

\begin{figure}%[H]%[htp]
    \centering
    \includegraphics[width=12cm]{figures/lineplots_pct_comp_new.png}
    \caption{Increase of duration in phrase-final position in percent for the three types of target words across subjects, based on the estimates provided by the LME models (the data for the coda consonant of the penultimate syllable in one of the words with antepenultimate stress is omitted here because the other word forms did not include a coda consonant in this position)}
    \label{fig:comp}
\end{figure}

Figure \ref{fig:comp} illustrates that the relatively small amount on C1 is comparable across target words. A significant effect was absent in this position across target words. On V1, the words with penultimate stress show an amount of nearly 0 whereas the words with antepenultimate stress show an amount of 9 percent. The comparisons yielded a significant effect only in the latter case. On C2, the types of words with penultimate stress show different patterns: The words with CV.ˈCV.CVC structure involve a considerably smaller increase (4 percent) than the words without such a consonant (20 percent). On the following vowel (V2) and the consonant after that (C3) both types show a similar amount of increase. On the vowel of the final syllable (V3), the types of target words with a following coda consonant show a similar amount of increase (63 and 67 percent, respectively) whereas the target words without such a consonant show a larger amount (82 percent).

It has been found in prior studies that vowels in closed syllables are shorter in duration than vowels in open syllables (e.g., \citealt{Jones1950}), which is referred to as closed-syllable vowel shortening. Thus, the final vowel in the penultimate stress words might be shorter in duration when a final coda consonant is present than when it is absent. In order to test how much difference in duration between the vowels in these conditions must be attributed to this phenomenon, we compared the duration of the final vowel with and without a final coda consonant in phrase-final position. As shown in Figure \ref{fig:closed_syllable}, the vowels in open syllables were significantly longer than the vowels in closed syllables. A linear mixed effects model accounting for \textsc{duration} as a function of \textsc{coda condition} (levels: absent, present) was fitted to the data of these vowels. Random intercepts and slopes were included for \textsc{subject} and random intercepts for \textsc{item}. The model estimated that, in closed syllables, the vowel duration was 10 ms shorter than in open syllables ($\beta=10.2$, $\text{SE}=4.8$, $t=-2.1$). The model was tested against a reduced model without \textsc{coda condition} as a fixed factor by means of a likelihood ratio test, which yielded a significant effect ($\chi^2(1)=4.11$, $p=0.043$). The smaller amount of increase in V3 with a following coda consonant might be related to the fact that the vowel is inherently shorter as well as with the fact that another segment is following (C4, which shows an even larger amount of lengthening). A detailed exploration of a connection between closed syllable shortening and the patterns of PBL is beyond the scope of this study and should be addressed in future research.

\begin{figure}
    \includegraphics[width=9.8cm]{figures/closed-syllable.png}
    \caption{Duration plots (ms) with confidence intervals for the vowel of the final syllable in the penultimate
stress words with and without a final coda consonant across subjects (* $p<0.05$)\label{fig:closed_syllable}}
\end{figure}

\section{Discussion and conclusion}\label{discussion}
\subsection{Initiation and scope of PBL}
The results suggest that word stress affects the initiation and scope of PBL in German. The initiation of PBL occurred on the main stress syllable across conditions, resulting in a later initiation point in words with penultimate stress than in words with antepenultimate stress. This pattern supports the assumption that the position of main stress serves as an anchor for PBL. The findings, however, deviate from the prediction of the Word Rime hypothesis in one respect: As stated in (\ref{1}a), this hypothesis predicts that PBL applies to the nuclear vowel of the main stress syllable and all succeeding segments in the phrase-final word (but not to the segments preceding the rime of the main stress syllable). In the penultimate stress words without a final coda consonant (e.g., RaMOna), PBL initiation, however, occurred on the onset consonant of the main stress syllable, that is, the segment immediately preceding the expected initiation point. This suggests a less strict reading of the Word Rime hypothesis. In the other types of target words, the initiation point was on the nuclear vowel of the main stress syllable. The observed connection between PBL initiation and the main stress syllable is compatible with the results by \citet{Kohler1983}, who found that PBL occurred across all syllables in di- and trisyllabic words with initial stress. However, the results differ from the observation by \citet{Silverman1990} that the antepenultimate syllable also undergoes PBL in words with penultimate stress in German. Generally, the observed connection between PBL initiation and the main stress syllable is in line with prior findings from other languages (e.g., \citealt{Berkovits1994} for Hebrew; \citealt{White2002} for British English; \citealt{TurkShattuck-Hufnagel2007} for American English); yet, there are differences with regard to the details. For example, \citet{TurkShattuck-Hufnagel2007} found that PBL is interrupted on the penultimate syllable in words with antepenultimate stress, which was not attested in the present study. It has also been observed that the main stress syllable is relevant for phrase-initial segment lengthening (\citetv{chapters/napoleao}), which suggests a more general connection between stress positions and boundary-related lengthening.

As for the addition of a coda consonant to the end of the word, the results suggests that the initiation of PBL shifted to the following segment if an additional coda consonant was present. In the penultimate stress words (e.g., RaMOna\slash RaMOnas), PBL was initiated on the onset consonant of the penultimate syllable when a final coda consonant was absent, but on the following nuclear vowel when such a consonant was present. Thus, in both conditions, PBL occurred on the portion of the phrase-final word that included the last four segments. Thus, the prediction stated in (\ref{1}b), capturing the Overlap hypothesis is supported by the present findings. This is in line with prior studies that found an overlap effect (e.g., \citealt{ByrdSaltzman2003} for American English, \citealt{SeoEtal2019} for Japanese).

Altogether, our results suggest that both the position of main word stress and the presence\slash absence of word-final material affect the point of PBL initiation. These findings are compatible with an account that assumes the nuclear vowel of the main stress syllable as the default point of PBL initiation, but allows for PBL on earlier segments if the amount of material between the nuclear vowel and the phrase-boundary is limited. That is, the PBL domain is by default aligned with the nuclear vowel of the main stress syllable at its left and the phrase boundary at its right, but can include earlier segments if this span is too short. This gives rise to the working hypothesis that the PBL domain must have a minimum size in German and thus extends to a segment preceding the nuclear vowel of the main stress syllable. In our data, this occurred when the words had penultimate word stress and lacked a final coda consonant (e.g., RaMOna). The other word forms (e.g., RaMOnas, KArolin) contained enough material between the anchor and the end of the word, so that the PBL domain did not include the preceding onset consonant. This explanation is also compatible with the observation that PBL can occur on material preceding the syllable bearing main word stress, as has been found for words with penultimate stress in German (\citealt{Silverman1990}) and American English (\citealt{ChoEtal2013}). The expansion of the PBL domain to earlier material might also be a strategy of the speaker to signal the presence of a prosodic boundary to the listener.\footnote{Many thanks to an anonymous reviewer for pointing this out to us.} This should be addressed in future research.

The conclusions drawn from the comparison between the penultimate stress words with and without a final coda consonant (e.g., RaMOna\slash RaMOnas) are limited for several reasons in the present study. First, the final coda consonant was always the voiceless alveolar fricative [s]; second, this fricative constitutes a suffix, which yields different morphological structures in the word forms; and, third, the word forms were elicited in different syntactic constructions, which might have affected relative boundary strength. Future research should test mono-mor\-phem\-ic words with various types of coda consonants that are elicited in the same syntactic construction as the words without a final coda consonant.

\subsection{Progressive lengthening}
We observed a general tendency of progressive lengthening, that is, the amount of PBL gradually increased towards the phrase boundary. This pattern was not consistently applied on the material preceding the final rime, where the amount of PBL showed a slight decrease from one segment to a following one in some cases. This finding suggests that German employs a weak form of progressive lengthening. When the final rime was complex, the amount increased from the nuclear vowel to the following coda consonant, so that the largest amount of PBL always occurred on the final consonant. Furthermore, we found a large increase of PBL on the vowel of the final rime across conditions. That is, the increase of PBL in comparison to the prior segment was strongest on the vowel of the final rime, independent of the rime-internal structure. These patterns are similar to those observed in American English (\citealt{TurkShattuck-Hufnagel2007}) and Japanese (\citealt{SeoEtal2019}). Unlike to American English (\citealt{TurkShattuck-Hufnagel2007}), PBL was not interrupted on the penultimate syllable in words with antepenultimate stress (e.g., KArolin) in our data.

\subsection{PBL in a crosslinguistic perspective}
The findings of the present study support the view that the extent of the PBL domain is determined by the position of word stress as well as by the segmental composition of the phrase-final word across languages. Like English (e.g., \citealt{White2002, TurkShattuck-Hufnagel2007}), German exhibits a connection between PBL initiation and the main stress syllable. Similarly, Greek shows a tendency to pull the initiation of PBL towards the main stress syllable (\citealt{Katsika2016}). Dutch, on the other hand, does not show an influence of the main stress syllable, but initiates PBL on the final syllable, unless the final syllable contains only a schwa (\citealt{Cambier-Langeveld1997}). Altogether, these findings suggest that word stress tends to affect PBL across languages, but languages differ with regard to implementation.

Future research should test the production patterns of PBL with the same or similar materials across languages, as this would provide crosslinguistic data that is directly comparable. The type of materials used in the present study can easily be adapted to other languages. More research is particularly needed on languages with diverse prosodic systems, including languages with an edge-based prosodic system and languages with lexical tone. \citet{SchuboeEtal2021} used the same type of materials as in the present study to investigate boundary-related lengthening in Tswana (Southern Bantu), a tone language that expresses specific prosodic boundaries by means of lengthening of the penultimate syllable. They found that PBL occurs on the final syllable in addition to the penultimate lengthening effect, and that the amount of lengthening is comparable on both syllables. This pattern is different from the pattern of progressive lengthening found in German and other languages, which suggests that Tswana has two independent lengthening mechanisms for expressing a prosodic boundary.

The present study found that PBL was strongest on the rime of the final syllable. This suggests that the duration of the final rime might constitute the most salient cue for listeners in the perception of a prosodic boundary based on PBL. Further research is needed on the relevance of the location and amount of lengthening for the perception of a prosodic boundary. Testing the role of these factors for speech perception requires a detailed understanding of their impact on the production of PBL in a given language (see also \citealt{TurkShattuck-Hufnagel2007} for this point). For example, in order to test if PBL must occur within the designated portion of the phrase-final word or may as well be located on other material near the potential boundary location, we need to understand which factors affect the scope and distribution of PBL. The present study provided insights for German that are essential for such investigations.


\section*{Acknowledgements}
Many thanks to Nadja Spina for comments and technical support.

\section*{Funding information}
This research was funded by the Deutsche Forschungsgemeinschaft (DFG, German Research Foundation), project ``Preboundary lengthening in a cross-lin\-guis\-tic perspective" (project no. 416902968, GZ ZE 940/3-1), held by Sabine Zerbian. 


\begin{paperappendix}\largerpage
\section{Stimuli}\label{appendix:schuboe:a}
\ea
\ea Ich werde \uline{Ramona oder Karolin} und \uline{Peter} einladen.
\ex Ich werde \uline{Ramona} oder \uline{Karolin und Peter} einladen.
\ex Ich werde \uline{Karolin oder Ramona} und \uline{Peter} einladen.
\ex Ich werde \uline{Karolin} oder \uline{Ramona und Peter} einladen.
\ex Ich werde \uline{Karolins oder Ramonas} und \uline{Peters Freunde} einladen.
\ex Ich werde \uline{Karolins} oder \uline{Ramonas und Peters Freunde} einladen.
\z
\ex
\ea Ich werde \uline{Marina oder Salomon} und \uline{Paula} besuchen.		
\ex Ich werde \uline{Marina} oder \uline{Salomon und Paula} besuchen.			
\ex Ich werde \uline{Salomon oder Marina} und \uline{Paula} besuchen.		
\ex Ich werde \uline{Salomon} oder \uline{Marina und Paula} besuchen.			
\ex Ich werde \uline{Salomons oder Marinas} und \uline{Paulas Oma} besuchen.
\ex Ich werde \uline{Salomons} oder \uline{Marinas und Paulas Oma} besuchen.
\z
\ex
\ea Ich werde \uline{Verena oder Jonathan} und \uline{Stefan} helfen.
\ex Ich werde \uline{Verena} oder \uline{Jonathan und Stefan} helfen.
\ex Ich werde \uline{Jonathan oder Verena} und \uline{Stefan} helfen.
\ex Ich werde \uline{Jonathan} oder \uline{Verena und Stefan} helfen.
\ex Ich werde \uline{Jonathans oder Verenas} und \uline{Stefans Schwester} helfen.
\ex Ich werde \uline{Jonathans} oder \uline{Verenas und Stefans Schwester} helfen.
\z
\ex
\ea Ich werde \uline{Rosina oder Valentin} und \uline{Anna} verwarnen.
\ex Ich werde \uline{Rosina} oder \uline{Valentin und Anna} verwarnen.
\ex Ich werde \uline{Valentin oder Rosina} und \uline{Anna} verwarnen.
\ex Ich werde \uline{Valentin} oder \uline{Rosina und Anna} verwarnen.
\ex Ich werde \uline{Valentins oder Rosinas} und \uline{Annas Bruder} verwarnen.
\ex Ich werde \uline{Valentins} oder \uline{Rosinas und Annas Bruder} verwarnen.
\z
\ex
\ea Ich werde \uline{Simona oder Fridolin} und \uline{Lisa} suchen.
\ex Ich werde \uline{Simona} oder \uline{Fridolin und Lisa} suchen.
\ex Ich werde \uline{Fridolin oder Simona} und \uline{Lisa} suchen.
\ex Ich werde \uline{Fridolin} oder \uline{Simona und Lisa} suchen.
\ex Ich werde \uline{Fridolins oder Simonas} und \uline{Lisas Geschwister} suchen.
\ex Ich werde \uline{Fridolins} oder \uline{Simonas und Lisas Geschwister} suchen.
\z
\ex
\ea Ich werde \uline{Selina oder Gisela} und \uline{Martin} holen.
\ex Ich werde \uline{Selina} oder \uline{Gisela und Martin} holen.
\ex Ich werde \uline{Gisela oder Selina} und \uline{Martin} holen.
\ex Ich werde \uline{Gisela} oder \uline{Selina und Martin} holen.
\ex Ich werde \uline{Giselas oder Selinas} und \uline{Martins Eltern} holen.
\ex Ich werde \uline{Giselas} oder \uline{Selinas und Martins Eltern} holen.
\z
\ex
\ea Ich werde \uline{Ramona oder Salomon} und \uline{Anna} befragen.
\ex Ich werde \uline{Ramona} oder \uline{Salomon und Anna} befragen.
\ex Ich werde \uline{Salomon oder Ramona} und \uline{Anna} befragen.
\ex Ich werde \uline{Salomon} oder \uline{Ramona und Anna} befragen.
\ex Ich werde \uline{Salomons oder Ramonas} und \uline{Annas Bruder} befragen.
\ex Ich werde \uline{Salomons} oder \uline{Ramonas und Annas Bruder} befragen.
\z
\ex
\ea Ich werde \uline{Marina oder Jonathan} und \uline{Stefan} abholen.
\ex Ich werde \uline{Marina} oder \uline{Jonathan und Stefan} abholen.
\ex Ich werde \uline{Jonathan oder Marina} und \uline{Stefan} abholen.
\ex Ich werde \uline{Jonathan} oder \uline{Marina und Stefan} abholen.
\ex Ich werde \uline{Jonathans oder Marinas} und \uline{Stefans Geschwister} abholen.
\ex Ich werde \uline{Jonathans} oder \uline{Marinas und Stefans Geschwister} abholen.
\z
\ex
\ea Ich werde \uline{Rosina oder Valentin} und \uline{Lisa} beschuldigen.
\ex Ich werde \uline{Rosina} oder \uline{Valentin und Lisa} beschuldigen.
\ex Ich werde \uline{Valentin oder Rosina} und \uline{Lisa} beschuldigen.
\ex Ich werde \uline{Valentin} oder \uline{Rosina und Lisa} beschuldigen.
\ex Ich werde \uline{Valentins oder Rosinas} und \uline{Lisas Freunde} beschuldigen.
\ex Ich werde \uline{Valentins} oder \uline{Rosinas und Lisas Freunde} beschuldigen.
\z
\ex
\ea Ich werde \uline{Verena oder Fridolin} und \uline{Martin} anrufen.
\ex Ich werde \uline{Verena} oder \uline{Fridolin und Martin} anrufen.
\ex Ich werde \uline{Fridolin oder Verena} und \uline{Martin} anrufen.
\ex Ich werde \uline{Fridolin} oder \uline{Verena und Martin} anrufen.
\ex Ich werde \uline{Fridolins oder Verenas} und \uline{Martins Eltern} anrufen.
\ex Ich werde \uline{Fridolins} oder \uline{Verenas und Martins Eltern} anrufen.
\z
\ex
\ea Ich werde \uline{Simona oder Gisela} und \uline{Peter} begleiten.
\ex Ich werde \uline{Simona} oder \uline{Gisela und Peter} begleiten.
\ex Ich werde \uline{Gisela oder Simona} und \uline{Peter} begleiten.
\ex Ich werde \uline{Gisela} oder \uline{Simona und Peter} begleiten.
\ex Ich werde \uline{Giselas oder Simonas} und \uline{Peters Oma} begleiten.
\ex Ich werde \uline{Giselas} oder \uline{Simonas und Peters Oma} begleiten.
\z
\ex
\ea Ich werde \uline{Selina oder Karolin} und \uline{Paula} ermahnen.
\ex Ich werde \uline{Selina} oder \uline{Karolin und Paula} ermahnen.
\ex Ich werde \uline{Karolin oder Selina} und \uline{Paula} ermahnen.
\ex Ich werde \uline{Karolin} oder \uline{Selina und Paula} ermahnen.
\ex Ich werde \uline{Karolins oder Selinas} und \uline{Paulas Schwester} ermahnen.
\ex Ich werde \uline{Karolins} oder \uline{Selinas und Paulas Schwester} ermahnen.
\z
\z
\section{LME model outputs}\label{appendix:schuboe:b}

\begin{table}[H]
    \caption{LME model output for the words with penultimate stress and CV.ˈCV.CV structure (significant $t$ values are boldfaced)}
    \begin{tabular}{l S[table-format=-2.1] c r}
    %\begin{tabular}{lccc}
    \lsptoprule
     \ & \mc{Estimate} & \mc{SE} & \mc{$t$}  \\
    \midrule
    (Intercept) & 98.4 & 1.8   & \textbf{55.4} \\
    \textsc{phrase position} \textit{final} & 80.1 & 1.9 & \textbf{42.2} \\
    \textsc{segment position} \textit{2} & 51.4 & 1.8 & \textbf{−28.2} \\
    \textsc{segment position} \textit{3} & -7.7 & 1.8 & \textbf{−4.2} \\
    \textsc{segment position} \textit{4} & -32.1 & 1.8 & \textbf{−17.6} \\
    \textsc{segment position} \textit{5} & -24.7 & 1.8 & \textbf{−13.6} \\
    \textsc{segment position} \textit{6} & -32.7 & 1.8 & \textbf{−18.0} \\
    \textsc{phrase position} \textit{final} : \textsc{segment position} \textit{2} & -72 & 2.6 & \textbf{−27.8} \\
    \textsc{phrase position} \textit{final} : \textsc{segment position} \textit{3} & -66.2 & 2.6 & \textbf{−25.6} \\
    \textsc{phrase position} \textit{final} : \textsc{segment position} \textit{4} & -69.4 & 2.6 & \textbf{−26.9} \\
    \textsc{phrase position} \textit{final} : \textsc{segment position} \textit{5} & -81 & 2.6 & \textbf{−31.3} \\
     \textsc{phrase position} \textit{final} : \textsc{segment position} \textit{6} & -76.5 & 2.6 & \textbf{−29.6} \\
    \lspbottomrule
    \end{tabular}
\end{table}


\begin{table}%[H]
    \caption{LME model output for the words with penultimate stress and CV.ˈCV.CVC structure (significant $t$ values are boldfaced)}
    %\begin{tabular}{lccc}
    \begin{tabular}{l S[table-format=-2.1] c r}
    \lsptoprule
     \ & \mc{Estimate} & \mc{SE} & \mc{$t$}  \\
    \midrule
    (Intercept) & 79 & 1.8   & \textbf{43.3} \\
    \textsc{phrase position} \textit{final} & 56.7 & 2.3 & \textbf{25.2} \\
    \textsc{segment position} \textit{2} & 24.3 & 1.9 & \textbf{12.9} \\
    \textsc{segment position} \textit{3} & -29.9 & 1.9 & \textbf{−15.9} \\
    \textsc{segment position} \textit{4} & 12.2 & 1.9 & \textbf{6.5} \\
    \textsc{segment position} \textit{5} & -10.4 & 1.9 & \textbf{−5.6} \\
    \textsc{segment position} \textit{6} & -5.3 & 1.9 & \textbf{−2.8} \\
    \textsc{segment position} \textit{7} & -13.2 & 1.9 & \textbf{−7.0} \\
    \textsc{phrase position} \textit{final} : \textsc{segment position} \textit{2} & 8.3 & 2.6 & \textbf{3.2} \\
    \textsc{phrase position} \textit{final} : \textsc{segment position} \textit{3} & -51.4 & 2.6 & \textbf{−20.1} \\
    \textsc{phrase position} \textit{final} : \textsc{segment position} \textit{4} & -43.8 & 2.6 & \textbf{−17.1} \\
    \textsc{phrase position} \textit{final} : \textsc{segment position} \textit{5} & -54.8 & 2.6 & \textbf{−21.4} \\
    \textsc{phrase position} \textit{final} : \textsc{segment position} \textit{6} & -57.2 & 2.6 & \textbf{−22.4} \\
    \textsc{phrase position} \textit{final} : \textsc{segment position} \textit{7} & -54.5 & 2.6 & \textbf{−21.3} \\
    \lspbottomrule
    \end{tabular}
\end{table}


\begin{table}[H]
    \caption{LME model output for the words with penultimate stress and CV.ˈCV.CVC structure (significant $t$ values are boldfaced)}
   % \begin{tabular}{lccc}
    \begin{tabular}{l S[table-format=-2.1] c r}
    \lsptoprule
      & \mc{Estimate} & \mc{SE} & \mc{$t$}   \\
    \midrule
    (Intercept) & 50.8 & 3.6   & \textbf{14.2} \\
    \textsc{phrase position} \textit{final} & 50 & 2.0 & \textbf{25.0} \\
    \textsc{segment position} \textit{2} & 24 & 2.0 & \textbf{12.3} \\
    \textsc{segment position} \textit{3} & -0.3 & 2.0 & −0.4 \\
    \textsc{segment position} \textit{4} & 10.1 & 3.4 & \textbf{3.0} \\
    \textsc{segment position} \textit{5} & 12.8 & 2.0 & \textbf{6.6} \\
    \textsc{segment position} \textit{6} & -3.3 & 2.0 & \textbf{−1.7} \\
    \textsc{segment position} \textit{7} & 42.1 & 2.0 & \textbf{21.6} \\
    \textsc{segment position} \textit{8} & 14.3 & 2.0 & \textbf{7.3} \\
    \textsc{phrase position} \textit{final} : \textsc{segment position} \textit{2} & 1.9 & 2.7 & 0.7 \\
    \textsc{phrase position} \textit{final} : \textsc{segment position} \textit{3} & -43.1 & 2.7 & \textbf{−16.0} \\
    \textsc{phrase position} \textit{final} : \textsc{segment position} \textit{4} & -40.7 & 4.8 & \textbf{−8.5} \\
    \textsc{phrase position} \textit{final} : \textsc{segment position} \textit{5} & -43.2 & 2.7 & \textbf{−16.0} \\
    \textsc{phrase position} \textit{final} : \textsc{segment position} \textit{6} & -46.3 & 2.7 & \textbf{−17.1} \\
    \textsc{phrase position} \textit{final} : \textsc{segment position} \textit{7} & -42.3 & 2.7 & \textbf{−15.7} \\
    \textsc{phrase position} \textit{final} : \textsc{segment position} \textit{8} & -47.5 & 2.7 & \textbf{−17.6} \\
    \lspbottomrule
    \end{tabular}
\end{table}

\end{paperappendix}

\printbibliography[heading=subbibliography,notkeyword=this]

\end{document}
