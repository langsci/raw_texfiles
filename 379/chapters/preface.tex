\addchap{Preface}
\begin{refsection}
In spoken language comprehension, the hearer is faced with a more or less continuous stream of auditory information. Prosodic cues, such as pitch movement, pre-boundary lengthening, and pauses, incrementally help to organize the incoming stream of information into prosodic phrases, which often coincide with syntactic units. Prosody is hence central to spoken language comprehension and is ``the skeletal structure on which the rest of the utterance depends" \citep[248]{frazier_prosodic_2006}. Accordingly, some models assume that the speaker produces prosody in a consistent and hierarchical fashion (e.g., \cite{nespor_prosodic_1986}). While there is manifold empirical evidence that prosodic boundary cues are reliably and robustly produced and effectively guide spoken sentence comprehension across different populations and languages, the underlying mechanisms and the nature of the prosody-syntax interface still have not been identified sufficiently. This is also reflected in the fact that most models on sentence processing completely lack prosodic information. 

This edited book volume is grounded in a workshop that was held in 2021 at the annual conference of the \textit{Deutsche Gesellschaft für Sprachwissenschaft} (DGfS). The five chapters cover selected topics on the production and comprehension of prosodic cues in various populations and languages, all focusing in particular on processing of prosody at structurally relevant prosodic boundaries.

With respect to the prosodic cues investigated, the different contributions refer to the most common cues crosslinguistically, such as increased segment duration in phrase-initial position (referred to as domain-initial strengthening in \citetv{chapters/napoleao}) and in phrase-final position (referred to as final lengthening in \citetv{chapters/ots}, \citetv{chapters/huttenlauch}, and \citetv{chapters/wellmann}, referred to as pre-boundary lengthening in \citetv{chapters/schuboe}). Pauses and F0-related measures, such as F0-range (also referred to as pitch range), and rises, are also addressed in several studies (\citetv{chapters/ots}, \citetv{chapters/huttenlauch}, \citetv{chapters/wellmann}). We have refrained from unifying the terminology used across individual chapters, especially in the case of the boundaries themselves. Here, reference is made to prosodic units and domains, prosodic phrases, Intonational Phrases, clause boundaries and breaks, and each contributor defines these terms in the respective chapter as relevant.

Regarding theory and modelling, the contributions by \citetv{chapters/schuboe} and \citetv{chapters/napoleao} deal with the interrelation of prosodic boundaries and lexical prominence. Boundary processes are explored across four languages, and it emerges that the syllable carrying the main stress serves as an anchor for boundary phenomena. Both studies share the reference to the work of \citet{Katsika2016} who showed for Greek that stress on phrase-final words has an effect on boundary-related lengthening but not on phrase-initial lengthening of the following word. In her proposed (gestural) account, lexical and phrasal prosody interact in a systematic and coordinated way at prosodic boundaries.


From a methodological perspective, \textcitetv{chapters/schuboe}, \textcitetv{chapters/wellmann}, and \textcitetv{chapters/huttenlauch} worked with lists of three names which differ in their syntactic branching. Due to these parallels in the stimuli, the results of these studies can inform each other with reference to the Proximity/Similarity Model (\cite{KentnerFery2013}), specifically as the Principle of Anti-Proximity is concerned in the rendition of the middle name with respect to the strengthening of prosodic cues at its boundary. 

With respect to the specific aims, research questions and theoretical contributions of the different studies in this volume,  \textcitetv{chapters/schuboe} investigate the initiation and scope of pre-boundary lengthening on the phrase-final word in German. Native adult speakers read out sentences in which the target word varied with respect to the position of word stress (penultimate vs. antepenultimate syllable) and the presence/absence of an additional segment at the end of the word. In result, pre-boundary lengthening was reliably found on the stressed syllable – and its start was shifted from the onset consonant to the following vowel of the stressed syllable when a coda consonant was added to the words with penultimate stress. This indicates that the scope of pre-boundary lengthening in German is determined by the prosodic structure as well as the segmental composition of the phrase-final words as it has been claimed for other languages as well.

Moving away from prosodic boundary phenomena in a single language, a crosslinguistic comparison of boundary phenomena at the beginning of prosodic units was undertaken by \textcitetv{chapters/napoleao}. This chapter compares domain-initial strengthening in three lexical stress languages: English, Spanish, and Portuguese. The study addresses the question of how domain-initial strengthening is expressed acoustically across the three different languages, and how it affects the acoustic properties of segments in fully unstressed syllables in prenuclear post-boundary positions. From the crosslinguistic comparison, the study concludes that acoustic correlates of boundary marking extend beyond the initial segment in unstressed CV syllables (in Spanish and Portuguese the vowel is affected as is the stressed syllable in all three languages). 

How prosodic cues can be used by non-native listeners to chunk a speech stream of a language they do not know compared to native speakers of that language has been studied by \textcitetv{chapters/ots}. To this end, German and Estonian listeners were asked to listen to spontaneous utterances spoken in Estonian and to mark the point in time when they perceived a break between words. While Estonian listeners were guided by clause boundaries marked by longer pauses and intonational rises, German listeners were also sensitive to phrase-final lengthening and intensity drop. This indicates that non-native listeners rely on bottom-up processing for prosodic boundary identification, while native adult speakers also apply their top-down knowledge to chunk the incoming speech stream.

A similar bottom-up processing strategy needs to be applied by newborns and infants in language acquisition. \textcitetv{chapters/wellmann} investigate developmental changes in the processing of intonation phrase boundaries which (in German) are mainly characterized by pitch change, final lengthening, and a silent pause. These cues have been shown to have different weightings in perception in different languages. Between the age of six and eight months, infants’ prosodic processing undergoes an important development: moving away from the necessity of all of these three cues or a combination of pause and final lengthening for detection of a boundary to a detection based on a combination of pitch and final lengthening without a pause. This shift towards more ``independence" from the pause cue reflects a language-specific shift of attention to boundary markings that are functionally relevant in the to-be-learned ambient language. 

Finally, \textcitetv{chapters/huttenlauch} investigate how the other end of the age spectrum affects the realization of intonation phrase boundaries. Here, the productions of younger speakers of German (\cite{huttenlauchetal2021}) are compared to older speakers of German while they produced coordinated three-name sequences without and with internal grouping of the first two names. Both age groups marked the grouping globally using all three prosodic cues (i.e., F0 range, final lengthening, and pause) in line with the Proximity/Similarity model by \citet{kentner_new_2013}. Prosodic grouping was unaffected by age despite age-related longer absolute durations and larger variability in the productions of older participants. Furthermore, as both groups do not or only minimally adapt to different virtual communication partners, the data support models of situational independence of disambiguating prosody (e.g., \cite{speer_situationally_2011}).

\begin{sloppypar}
With this volume, we hope to have compiled an interesting compendium on different prosodic boundary phenomena which comprises crosslinguistic evidence as well as evidence from non-native listeners, infants, adults, and elderly speakers, highlighting the important role of prosody in both language production and comprehension. 
\end{sloppypar}

{\sloppy\printbibliography[heading=subbibliography]}
\end{refsection}
