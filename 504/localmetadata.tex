\title{Locative and existential predication}
\subtitle{On forms, functions and neighboring domains}
\BackBody{Locative and existential predications are fundamental linguistic
constructions that exhibit significant formal overlap while serving
distinct communicative functions. Locative clauses typically anchor a
definite referent to a spatial context, whereas existential clauses
introduce new, often indefinite, referents into discourse. Despite their
central role in syntactic and typological research, the cross-linguistic
diversity of these predications remains largely underexplored.
This collective volume originates from workshops held in 2023 at the
Annual SLE Meeting in Athens and the International Conference on
Historical Linguistics in Heidelberg. It brings together in-depth
analyses of locative and existential predications across a wide range of
languages, drawing on diverse methodological and theoretical approaches.
Rather than imposing a single framework, the volume deliberately allows
for variation in how these constructions are defined and analyzed,
reflecting the complexity and diversity of linguistic structures.
A key theme of the book is the relationship between locative,
existential, and possessive predication. Many of the included studies
highlight the formal and functional connections between these domains,
illustrating how different languages encode possession through
structures that overlap with locative and existential constructions. The
volume also challenges conventional assumptions about structural
distinctions between these predications, showing that in many languages,
such boundaries are blurred or even nonexistent.
The introductory chapter reviews key findings from prior research and
offers a refined typology of locative and existential predications. It
also highlights the major insights from the remaining chapters, each of
which provides a detailed empirical analysis of these constructions in
one or several underdescribed languages. The contributions address (i)
the structural and functional properties of locative and existential
clauses, (ii) criteria for distinguishing these constructions in
languages where formal differentiation is minimal, (iii) their frequency
and usage in natural discourse, and (iv) grammaticalization pathways
that link locative, existential, and possessive predication.
By integrating data from a broad range of languages and perspectives,
this volume advances our understanding of locative and existential
predication and offers a foundation for future research in typology,
syntax, and historical linguistics.}
\author{Chris Lasse Däbritz and Josefina Budzisch and Rodolfo Basile}

\renewcommand{\lsISBNdigital}{978-3-96110-537-3}
\renewcommand{\lsISBNhardcover}{978-3-98554-157-7}
\BookDOI{10.5281/zenodo.16759903}
\typesetter{Sebastian Nordhoff}
\proofreader{Alexis Michaud,
Bev Erasmus,
David Carrasco Coquillat,
Elliott Pearl,
Kate Lynn Lindsey,
Katja Politt,
Ksenia Shagal,
Lachlan Mackenzie,
Mahamane Laoualy Abdoulaye,
Mary Ann Walter,
Nicoletta Romeo,
Patricia Cabredo,
River Tae Smith,
Sercan Karakas,
Silvie Strauß,
Yvonne Treis
}
% \lsCoverTitleSizes{51.5pt}{17pt}% Font setting for the title page

\renewcommand{\lsSeries}{rcg}
\renewcommand{\lsSeriesNumber}{6}
\renewcommand{\lsID}{504}

