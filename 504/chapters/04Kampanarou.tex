\documentclass[output=paper,colorlinks,citecolor=brown]{langscibook}
\ChapterDOI{10.5281/zenodo.16838060}
\author{Anna Kampanarou\affiliation{University of Crete \& Aristotle University of Thessaloniki}}
\title[Locative phrases in existential and locative sentences in Greek]{Locative phrases as arguments or adjuncts: Existential and locative sentences in Greek}
\abstract{This chapter investigates the status of the locative constituent in Greek locative and existential sentences. Given that several types of sentences qualify as locatives and existentials in the language, this study offers a window into the syntax exploited for these two functions. Specifically, it shows that the locative function requires a structure in which the locative constituent is part of the main predication. In contrast, the existential one is achieved via the same structure with an additional change in the word order or via a second structure in which the locative merges as an adjunct. In this sense, this chapter suggests that the syntax behind locatives and existentials may be unified, but not necessarily.}


\IfFileExists{../localcommands.tex}{
   \addbibresource{../localbibliography.bib}
   % add all extra packages you need to load to this file

\usepackage{tabularx,multicol}
\usepackage{url}
\urlstyle{same}

\usepackage{listings}
\lstset{basicstyle=\ttfamily,tabsize=2,breaklines=true}

\usepackage{langsci-basic}
\usepackage{langsci-optional}
\usepackage{langsci-lgr}
\usepackage{langsci-osl}
% \usepackage{./langsci/styles/langsci-lgr}
% \usepackage{./langsci/styles/langsci-osl}
% \usepackage{langsci-gb4e}

\usepackage{tikz}
\usetikzlibrary{patterns,calc}
\pgfdeclarepatternformonly{south east lines}{\pgfqpoint{-0pt}{-0pt}}{\pgfqpoint{3pt}{3pt}}{\pgfqpoint{3pt}{3pt}}{
    \pgfsetlinewidth{0.6pt}
    \pgfpathmoveto{\pgfqpoint{0pt}{3pt}}
    \pgfpathlineto{\pgfqpoint{3pt}{0pt}}
    \pgfpathmoveto{\pgfqpoint{.2pt}{-.2pt}}
    \pgfpathlineto{\pgfqpoint{-.2pt}{.2pt}}
    \pgfpathmoveto{\pgfqpoint{3.2pt}{2.8pt}}
    \pgfpathlineto{\pgfqpoint{2.8pt}{3.2pt}}
    \pgfusepath{stroke}}
    
\usepackage{stmaryrd}
\usepackage{wasysym}
\usepackage{multirow}
\usepackage{caption}
\usepackage{subcaption}
\usepackage{mathrsfs}
\usepackage{qtree}

\usepackage{linguex}


   %pminos do not split footnotes
% \interfootnotelinepenalty=10000 %Footnote in Laporte chapters has to be split SN


%\DeclareIndexNameFormat{default}{%
%\nameparts{#1}%
%\usebibmacro{index:name}%
%{\index[names]}%
%{\namepartfamily}%
%{\namepartgiveni}%
% {}% L1
% {}% L2
%{\namepartprefix}% generates spurious space L3
%{\namepartsuffix}% generates spurious space L4
%}

%  {\DeclareIndexNameFormat{default}{%
%     \usebibmacro{index:name}{\index[names]}{#1}{#3}{#5}{#7}}}

%\DeclareIndexNameFormat{default}{%
%  \usebibmacro{index:name}{\sindex[nom]}{#1}{#3}{#5}{#7}}

%\DeclareIndexNameFormat{default}{%
%  \usebibmacro{index:name}{\sindex[person]}{#1}{#3}{#5}{#7}}
%\DeclareIndexNameFormat{default}{%
%\nameparts{#1} \usebibmacro{index:name}{\sindex[person]]}{\namepartfamily}{‌​\namepartgiven}{\nam‌​epartprefix}{\namepa‌​rtsuffix}}

%\newcommand{\smiley}{:)}

%\renewbibmacro*{index:name}[5]{%
%\usebibmacro{index:entry}{#1}%
%{\iffieldundef{usera}{}{\thefield{usera}\actualoperator}\mkbibindexname{#2}{#3}{#4}{#5}}}

% \newcommand{\noop}[1]{}

%remove for final
%\overfullrule=1mm

\newcommand{\tobi}[2]}}
\renewcommand{\S}[1]{\tobi{#1}{\textsc{*}}}

% this volume references
% puts: [this volume]
% already defined: \citetv
%\newcommand{\citepv}[1]{(\citeauthor{#1} \citeyear*{#1} [this volume])}
\newcommand{\citealtv}[1]{\citeauthor{#1} \citeyear*{#1} [this volume]}

%parentheses around example number
\newcommand{\pref}[1]{(\ref{#1})}

% in-text examples

\newcommand{\lnex}[1]{\textit{#1}} %target lang word
\newcommand{\lnlit}[1]{(lit.: `#1')} %literal reading
\newcommand{\lnlat}[1]{(#1)} % latinization
\newcommand{\lntrans}[1]{`#1'} %translation
\newcommand{\lnexl}[2]%
{\lnex{#1}{} \lnlat{#2}} % ex with latinization
\newcommand{\lnexlat}[3]{\lnex{#1}{} \lnlat{#2}{} \lntrans{#3}} % ex with latinization and tranl.

%ch01
\newcommand{\co}[1]{\mbox{\textbf{#1}}}

%ch09

\newcommand{\cyrbulg}[1]{\begin{otherlanguage*}{bulgarian}#1\end{otherlanguage*}}


%ch10
\newcommand{\nlp}{{\small NLP}}
\newcommand{\mwe}{{\small MWE}}
\newcommand{\rae}{{\small RAE}}
\newcommand{\lvc}{{\small LVC}}
\newcommand{\pos}{{\small P}o{\small S}}
%\newcommand{\todo}[1]{ \textcolor{red}{#1} }

%\renewcommand{\labelenumi}{\theenumi}
%\ainamefmt{{vv}{ll}{, ff}{, jj}} % fullname

\newcommand{\biberror}[1]{{\color{red}#1}}

\newcommand{\osenovaitem}{--~}
   %% hyphenation points for line breaks
%% Normally, automatic hyphenation in LaTeX is very good
%% If a word is mis-hyphenated, add it to this file
%%
%% add information to TeX file before \begin{document} with:
%% %% hyphenation points for line breaks
%% Normally, automatic hyphenation in LaTeX is very good
%% If a word is mis-hyphenated, add it to this file
%%
%% add information to TeX file before \begin{document} with:
%% %% hyphenation points for line breaks
%% Normally, automatic hyphenation in LaTeX is very good
%% If a word is mis-hyphenated, add it to this file
%%
%% add information to TeX file before \begin{document} with:
%% \include{localhyphenation}
\hyphenation{
    Beck-man
    Ngu-yen
    back-chan-nel
    back-chan-nels
    mo-not-o-nous
    ste-reo-typ-i-cal
}

\hyphenation{
    Beck-man
    Ngu-yen
    back-chan-nel
    back-chan-nels
    mo-not-o-nous
    ste-reo-typ-i-cal
}

\hyphenation{
    Beck-man
    Ngu-yen
    back-chan-nel
    back-chan-nels
    mo-not-o-nous
    ste-reo-typ-i-cal
}

   \boolfalse{bookcompile}
   \togglepaper[4]%%chapternumber
}{}

%\pretocmd{\gll}{\def\eachwordone{\itshape}\def\eachwordtwo{\normalfont}}{}{}

\begin{document}
\maketitle

\section{Introduction}
\subsection{Setting the scene}\label{KampanarouSec1.1}

This chapter focuses on several types of locational sentences, i.e., sentences that relate entities to locations, and examines the syntactic properties of the locative phrase, which is a common constituent contained in them. Drawing on evidence from Greek (Indo-European, [mode1248]), this line of research leads us to assume that there is variation in the syntactic structures behind locational sentences. In particular, this chapter shows that their structure, i.e., their syntax, and their pragmatic function, i.e., their use in the discourse, are strongly but not exclusively dependent on each other. To argue for it, I focus on two types of locational sentences, namely locative and existential ones. After defining what types of sentences fall into each category, their syntax is brought into scrutiny. 

To begin with, it must be noted that the terminology surrounding locative and existential sentences is highly controversial, primarily due to the vast literature on these topics. Identifying the commonalities between structures across languages and labeling them accordingly has proven particularly challenging. Furthermore, different perspectives endorsed by the researchers have led to various subcategorizations and definitions (see Haspelmath (\citeyear{Haspelmath2021}, \citeyear{Haspelmath2025}, \citeyear{HaspelmathNonverbal})).

Even though several proposals have been made in the literature, adopting function-based definitions for these labels is optimal because, among other reasons, they create a more useful and insightful dataset.\footnote{See \citet{Croft2022} for a similar view. A popular alternative is to define existentials, in particular, in terms of their specialized properties in their surface forms (see 
\cite{Kuno1971}, \cite{Milsark1974}, \cite{McNally1997, McNally2011}, \cite{Moro1997, Moro2006}, \cite{BentleyEtAl2015}, among others). For instance, \citet[1829]{McNally2011} defines existentials as “specialized or non-canonical constructions which express a proposition about the existence or the presence of someone or something”. However, I will not adopt this type of definition due to an observation made later by McNally herself, in \citet{McNally2016}, where she explains that since ‘canonical’ is different across languages, ‘non-canonical’ varies significantly, too. Thus, capturing what is deviant at a cross-linguistic level is not easy. Further, adopting this definition would exclude sentences including \textit{be} or \textit{exist} because, morphologically, they are canonical. Only sentences with expletives, peculiar morphological marking, or atypical word order would qualify as existentials. This would significantly restrict the dataset from which generalizations would be drawn and thus question the validity of these generalizations.} By defining locatives and existentials in terms of their pragmatic function, researchers can investigate a range of patterns used for the same purpose. Hence, they can investigate structural similarities and differences among the sentences falling under a single category or between the groups of sentences subsumed under each label.

For present purposes, I adopt Haspelmath’s \citeyearpar{Haspelmath2025} distinction between locatives and existentials for whom locative sentences (e.g., in (\ref{KampanarouIntroLoc})) qualify as predicational sentences in the sense that they include a locative predication with a definite subject. Specifically, the author uses the label ‘predlocative constructions’ for sentences like the ones presented in (\ref{KampanarouIntroLoc}) and defines them as ``a clause construction in which a \textbf{definite} (emphasis mine) subject argument is said to be located in a place expressed by a locative phrase''. 

He also proposes calling the subject/located element the \textit{locatum} (or, more specifically, the \textit{predlocatum}). 

\ea \label{KampanarouIntroLoc}
\ea \label{KampanarouEnglishLoc}
\textit{The cat is on the piano.} \\
\ex \label{KampanarouRussianLoc}
\langinfo{Russian}{Indo-European, [russ1263]}{\cite[212]{ParteeBorschev2004}} \\
\gll Doktor by-l v gorod-e. \\
doctor be-\textsc{pst.m.sg} in town-\textsc{loc.sg} \\
\glt `The doctor was in town.' 
\z
\z

The specification that predlocative constructions involve a definite locatum is crucial because this characteristic distinguishes them from existential sentences, as the ones presented in (\ref{KampanarouIntroEx}).

\ea \label{KampanarouIntroEx}
\ea \label{KampanarouEnglishEx}
\textit{There is a cat on the piano.} \\
\ex \label{KampanarouRussianEx}
\langinfo{Russian}{}{\cite[212]{ParteeBorschev2004}} \\
\gll V gorod-e by-l doktor. \\
in town-\textsc{loc.sg} be-\textsc{pst.m.sg} doctor \\
\glt `There was a doctor in town.' 
\z
\z

Likewise, an \textit{existential clause construction} is “a clause in which an indefinite nominal phrase (the \textit{existent}) is said to be in some location” according to \citet{Haspelmath2025}. For the rest of this paper, I will use the term \textit{locative} for predlocative constructions and \textit{existential} for existential clause constructions. 

Several researchers investigating these types of sentences have observed an affinity between them. This affinity derives from the fact that locative and existential sentences describe a locative relationship, i.e., a relationship between a locatum/existent, e.g., \textit{a cat} or \textit{a doctor}, and a location, e.g., \textit{a piano} or \textit{a town}, in the sentences above. To use Borschev \& Partee’s \citeyearpar{BorschevPartee1998} formalization, they both express a \textsc{be (thing, loc)} proposition where \textsc{be} represents the locative relationship, \textsc{thing} stands for the movable part of this relationship (locatum/existent), and \textsc{loc} represents the stable point, the location in which this part appears. \citet{Creissels2014, Creissels2019} and \citet{Kampanarou2023} treat this two-part locative relationship as a Figure-Ground relationship by adopting Talmy’s \citeyearpar{Talmy1975} distinction between Figures and Grounds.\footnote{\citet[419]{Talmy1975} defines the Figure as “a moving or conceptually movable point whose path or site is conceived as a variable, the particular value of which is the salient issue”. In contrast, the Ground is “a reference point, having a stationary setting within a reference-frame, with respect to which the Figure’s path or site receives characterization”.} Regardless of the terminology, these sentences are also truth-conditionally the same since they are true on the same conditions, i.e., when a cat appears on a piano for (\ref{KampanarouEnglishLoc}, \ref{KampanarouEnglishEx}) or when a doctor appears in town for (\ref{KampanarouRussianLoc}, \ref{KampanarouRussianEx}). 

A question that has been posed several times in the literature is whether this affinity derives from a unified syntax. In other words, do the locatum/existent and the location have the same syntactic status in both types of sentences?

Both answers have been given in the literature. On the one hand, many researchers supporting transformational approaches have posited that the syntax behind locatives and existentials is unified (\cite{Lyons1967}; \cite{Kuno1971}; \cite{Milsark1974}; \cite{Stowell1978}; \cite{Safir1987}; \cite{HoekstraMulder1990}; \cite{Freeze1992}; \cite{Kayne1993}; \cite{SagPollard1994}; \cite{Kempchinsky1996}; \cite{Moro1997, Moro2006}; \cite{Abdoulaye2006}; \cite{denDikken2006}; \cite{PeetersEtAl2006}, among others). On the other hand, a second strand of the literature has proposed that each sentence type has its own underlying structure. Each structure may have similarities with another structure, but this is not necessary (\cite{BarwiseCooper1981}; \cite{Williams1984}; \cite{Chung1987}; \cite{McNally1997}; \cite{Hazout2004}; \cite{Francez2007, Francez2009}; \cite{McCloskey2014}; \cite{Myler2018}, among others).

In this chapter, I present novel evidence from Greek suggesting that both strands of analysis have a grain of truth. Specifically, I show that the locative function stems from one single syntactic structure. In existentials, the existential function is achieved via distinct structures. In particular, the structure assumed for existentials is either identical or similar to the structure posited for locatives. To argue for this, I focus on the properties and the behavior of the locative constituent, which appears in both types of sentences. 

\subsection{Greek locatives and existentials}

To give some background, according to the above definitions, Greek has two types of locative sentences. The first one typically has a definite noun phrase preceding the verb \textit{ime} ‘be’ and a locative constituent (a prepositional phrase (PP) or adverb) following it. The verb fully agrees (in person and number) with the DP subject (\ref{Kampanarou3}).\footnote{All data are drawn on \citet{Kampanarou2023} and constitute naturally-occurring sentences that have been judged by approximately 30 speakers of Greek. Note, also, that the typical agreement pattern in \textit{ime}-sentences is not readily clear because of morphological syncretism: the form \textit{ine} is used for the third person singular and the third person plural.} 

\ea \label{Kampanarou3}
\ea \label{Kampanarou3a}
\gll I ɣata ine sto pjano. \\
\textsc{def.nom} cat\textsc{.nom} be.\textsc{3sg} on.\textsc{def.acc} piano.\textsc{acc}\\
\glt `The cat is on the piano.'
\ex \label{Kampanarou3b}
\gll Ta fruta ine sto trapezi. \\
\textsc{def.pl.nom} fruit.\textsc{pl.nom} be.\textsc{3pl} on.\textsc{def.acc} table.\textsc{acc} \\
\glt `The fruits are on the table.' 
\z
\z

The second type of locative sentence has the same surface order and agreement pattern but uses the non-active version of the verb \textit{vrisko} ‘find’, namely \textit{vriskome} ‘be found’ instead of \textit{ime} ‘be’ (\ref{Kampanarou4}).

\ea \label{Kampanarou4}
\ea \label{Kampanarou4a}
\gll O Petros vriskete sto spiti. \\
\textsc{def.nom} Peter\textsc{.nom} be.found.\textsc{3sg} at.\textsc{def.acc} house.\textsc{acc}\\
\glt `Peter is at home.'
\ex \label{Kampanarou4b}
\gll Ta koritsja vriskode stin Kriti. \\
\textsc{def.pl.nom} girl.\textsc{pl.nom} be.found.\textsc{3pl} in.\textsc{def.acc} Kriti.\textsc{acc} \\
\glt `The girls are in Crete.' 
\z
\z

Although different existent-location pairs are chosen to illustrate each sentence type, the two verbal items are interchangeable in all contexts above. 

On the other hand, there are three types of existentials in Greek: a \textsc{have}-based one, a \textsc{be}-based one, and an \textsc{exist}-based one. In all these sentences, the verbal item typically surfaces in sentence-initial position, the existent follows it, and the locative constituent appears sentence-finally. In \textsc{have}-sentences based on \textit{exi} `have' (\ref{Kampanarou5}), the existent is marked with accusative case, and the verb appears invariant marked for third-person singular (see \citealt{Kampanarou2021, Kampanarou2023} for more details). This pattern is also familiar in Romance languages (\cite[141]{BentleyEtAl2015}; \cite[38]{Cruschina2015}).

\ea \label{Kampanarou5}
\ea \label{Kampanarou5a}
\gll Exi ena skilo stin {peðiki xara}. \\
have.\textsc{3sg} \textsc{indf.acc} dog.\textsc{acc} at.\textsc{def.acc} playground.\textsc{acc}\\
\glt `There is a dog in the playground.'
\ex \label{Kampanarou5b}
\gll Exi kati skilus stin {peðiki xara}. \\
have.\textsc{3sg} some dog.\textsc{pl.acc} at.\textsc{def.acc} playground.\textsc{acc} \\
\glt `There are some dogs in the playground.' 
\z
\z

In contrast, \textsc{be}- and \textsc{exist}-existentials based on \textit{ime} `be' and \textit{iparxo} `exist' (\ref{Kampanarou6}, \ref{Kampanarou7}), respectively,  exhibit the typical agreement pattern of intransitive verbs in Greek.\footnote{Note that in this paper, I describe the distribution of \textit{ime}-existentials based on the judgments of most Modern Greek speakers. In \citet{Kampanarou2023}, I separate a smaller group of Modern Greek speakers who accept a second type of \textit{ime}-existentials plausibly due to a specific dialectal background since this group consists of speakers from southern parts of Greece.}

\ea \label{Kampanarou6}
\ea \label{Kampanarou6a}
\gll Ine ena peði sti mesi tu jipeðu. \\
be.\textsc{3sg} \textsc{indf.nom} kid\textsc{.nom} in.\textsc{def.acc} middle.\textsc{def.acc} \textsc{def.gen} field.\textsc{gen}\\
\glt `There is a child in the middle of the (football) field.'
\ex \label{Kampanarou6b}
\gll Ine kati peðja sti mesi tu jipeðu. \\
be.\textsc{3pl} some kid.\textsc{pl.nom} in.\textsc{def.acc} middle.\textsc{def.acc} \textsc{def.gen} field.\textsc{gen} \\
\glt `There are some children in the middle of the (football) field.' 
\z
\z

\ea \label{Kampanarou7}
\ea \label{Kampanarou7a}
\gll Iparxi kabosos kafes sto dulapi. \\
exist.\textsc{3sg} plenty.of\textsc{.nom} coffee\textsc{.nom} in.\textsc{def.acc} cupboard.\textsc{acc}\\
\glt `There is plenty of coffee in the cupboard.'
\ex \label{Kampanarou7b}
\gll Iparxun kabosi kafeðes sto dulapi. \\
exist.\textsc{3pl} plenty.of.\textsc{pl.nom} coffee.\textsc{pl.nom} in.\textsc{def.acc} cupboard.\textsc{acc} \\
\glt `There are plenty of (types of) coffees in the cupboard.' 
\z
\z

Again, even though each set of examples includes a different existent-location pair, the three verbal items are interchangeable in all the above cases. 

The verb \textit{iparxo} `exist' can also be used for the so-called ``hyparctic'' clauses or clauses of ontological existence that establish the existence of a kind (\ref{Kampanarou8}). However, it has a more extensive use than its English or Romance counterparts since it also appears in the locational \textit{iparxo}-existentials displayed above. 

\ea \label{Kampanarou8}
\ea \label{Kampanarou8a}
\gll Ðen ksero an iparxi θeos. \\
\textsc{neg} know.\textsc{1sg} if exist.\textsc{3sg} God\textsc{.nom}\\
\glt `I do not know if God exists.'
\ex \label{Kampanarou8b}
\gll I paniða ðen iparxi xoris ti xloriða. \\
\textsc{def.nom} fauna\textsc{.nom} \textsc{neg} exist.\textsc{3sg} without \textsc{def.acc} flora.\textsc{acc} \\
\glt `Fauna does not exist without flora.' 
\z
\z

In this article, I will focus on the locative constituent alone, i.e., the locative PP or adverb contained in locatives and existentials, to investigate whether its syntactic status is uniform across the board. Specifically, I will show that this item is an argument in \textsc{be.found}-locatives (based on \textit{vriskome}), \textsc{be}-locatives and \textsc{be}-existentials (both based on \textit{ime}), whereas it is an adjunct in \textsc{have}-existentials (based on \textit{exi}) and \textsc{exist}-existentials (based on \textit{iparxo}). This will constitute sufficient evidence for the assumption that existentials do not have the same underlying structure as locatives. To argue for this, I will investigate the behavior of this item concerning (a) its optionality or obligatoriness in being overt, (b) its ability to drop its preposition, (c) its ability to act as a stage topic, (d) its scopal behavior with respect to the adverbial \textit{again}, (e) the scopal behavior of quantifiers contained in it, and (f) its binding properties. Each criterion is discussed in \sectref{KampanarouSec2}. 

\section{Investigating the behavior of the locative constituent} \label{KampanarouSec2}
\subsection{Optionality in its overt realization}

The difference in the obligatoriness of being overt is a first property that separates two groups of sentences. First, witness that the locative constituent cannot be omitted under any circumstances in \textit{vriskome}-locatives (\ref{Kampanarou9a}), \textit{ime}-locatives (\ref{Kampanarou9b}), and \textit{ime}-existentials (\ref{Kampanarou9c}).

\ea \label{Kampanarou9}
\ea \label{Kampanarou9a}
\gll Ta vivlia vriskode \textup{*(}sto trapezi \textup{/} eðo\textup{)}. \\
\textsc{def.pl.nom} book.\textsc{pl.nom} be.found.\textsc{3pl} on.\textsc{def.acc} table.\textsc{acc} \textsc{} here\\
\glt `The books are *(on the table/here).'
\ex \label{Kampanarou9b}
\gll Ta vivlia ine \textup{*(}sto trapezi \textup{/} eðo\textup{)}. \\
\textsc{def.pl.nom} book.\textsc{pl.nom} be.\textsc{3pl} on.\textsc{def.acc} table.\textsc{acc} \textsc{} here\\
\glt `The books are *(on the table/here).'
\ex \label{Kampanarou9c}
\gll Ine kati vivlia \textup{*(}sto trapezi \textup{/} eðo\textup{)}. \\
be.\textsc{3pl} some book.\textsc{pl.nom} on.\textsc{def.acc} table.\textsc{acc} \textsc{} here\\
\glt `There are some books (on the table/here).'
\z
\z

In contrast, the locative constituent can be omitted in \textit{exi}- and \textit{iparxo}-existentials like in (\ref{Kampanarou10}).

\ea \label{Kampanarou10}
\gll Exi \textup{/} Iparxun kati vivlia  \textup{([}sto trapezi\textup{]} \textup{/} \textup{[}eðo\textup{])}. \\
have.\textsc{3sg} \textsc{} exist.\textsc{3pl} some book.\textsc{pl.nom} on.\textsc{def.acc} table.\textsc{acc} \textsc{} here\\
\glt `There are some books (on the table/here).'
\z

Notably, the locative constituent can be omitted only if easily retrieved through context or shared or common knowledge. This means that it is semantically present even if it is not phonetically overt. That is, a location is always implied.

Moreover, omitting the overt locative PP or adverb does not make the sentences hyparctic. The crucial difference between hyparctic sentences, e.g. (\ref{Kampanarou8}), and existentials with an omitted locative constituent is that the omitted locative in hyparctic sentences is not contextually determined. It is a vague spatiotemporal constituent that has more or less the meaning “in the universe, on the planet”. This contrasts with existentials, where the omitted locative may have several denotations depending on the discourse. 

Nonetheless, the locative constituent’s ability to be omitted in some existentials is a property indicative of adjuncts. 

\subsection{P(reposition)-drop}

According to \citet{GehrkeLekakou2013}, who investigate PPs in independent contexts, P(reposition)-drop is only available when PPs including the preposition \textit{se} (which is conflated with the definite determiner) are arguments. This occurs with verbs of directed motion and verbs of location. Adjunct PPs cannot drop their prepositional head, at least in most cases (see also \cite{IoannidouDenDikken2009}, \cite{Terzi2010}, \cite{Kouneli2014}, \cite{Sifaki2020}).

The researchers show this contrast based on the following examples. In (\ref{Kampanarou11a}), the complex P \textit{sto}, which comprises \textit{se} and the definite determiner \textit{to}, can be dropped as the PP is an argument of the verbs \textit{ftano} ‘arrive’ and \textit{epistrefo} ‘return’. In contrast, PPs that are adjunct modifiers cannot drop their preposition (\ref{Kampanarou11b}). 

\ea \label{Kampanarou11}
\ea \label{Kampanarou11a}
\gll Ftano \textup{/} Epistrefo \textup{/} Ime (sto) panepistimio. \\
arrive.\textsc{1sg} \textsc{} return.\textsc{1sg} \textsc{} be.\textsc{1sg} at.\textsc{def.acc} university.\textsc{acc}\\
\glt `I am arriving at/I am coming back to/I am at the university.'
\ex \label{Kampanarou11b}
\gll Perpatisa \textup{/} Xorepsa *(sto) jimnastirio \textup{/} sxolio \textup{/} ɣrafio \textup{/} eklisia. \\
walk.\textsc{pst.pfv.1sg} \textsc{} dance.\textsc{pst.pfv.1sg} at.\textsc{def.acc} gym.\textsc{acc} \textsc{} school.\textsc{acc} \textsc{} office.\textsc{acc} \textsc{} church.\textsc{acc}\\
\glt intended: `I walked/danced at the gym/school/office/church.' 
\z
\z

To employ this as a diagnostic, it is predicted that P-drop will be permitted with non-adjunct PPs. In the cases under discussion, P-drop becomes available only in \textit{vriskome}- and \textit{ime}-sentences (\ref{Kampanarou12}).\footnote{I have left aside examples including \textit{spiti} ‘home’ since \citet{Terzi2010} shows that this nominal behaves exceptionally, as is the case cross-linguistically, e.g., with English \textit{home}, German \textit{nachhause} / \textit{zuhause}, Russian \textit{doma} / \textit{domoj}.}

\ea \label{Kampanarou12}
\ea \label{Kampanarou12a}
\gll Ta peðja vriskode \textup{/} ine \textup{(}sto\textup{)} sxolio \textup{/} jimnastirio \textup{/} ɣrafio. \\
\textsc{def.pl.nom} kid.\textsc{pl.nom} be.found.\textsc{3pl} \textsc{} be.\textsc{3pl} at.\textsc{def.acc} school.\textsc{acc} \textsc{} gym.\textsc{acc} \textsc{} office.\textsc{acc}\\
\glt `The kids are at the school/gym/office.'
\ex \label{Kampanarou12b}
\gll Ine pola peðja \textup{(}sto\textup{)} sxolio \textup{/} jimnastirio \textup{/} ɣrafio simera. \\
be.\textsc{3pl} many kid.\textsc{pl.nom} at.\textsc{def.acc} school.\textsc{acc} \textsc{} gym.\textsc{acc} \textsc{} office.\textsc{acc} today \\
\glt `There are many kids at the school/gym/office today.' 
\z
\z

\textit{Exi}- and \textit{iparxo}-sentences do not allow the omission of the preposition heading the locative PP (\ref{Kampanarou13}).

\ea \label{Kampanarou13}
\gll Exi / Iparxun peðja *\textup{(}sto\textup{)} sxolio \textup{/} jimnastirio \textup{/} ɣrafio simera. \\
have.\textsc{3sg} \textsc{} exist.\textsc{3pl} kid.\textsc{pl.nom/acc} at.\textsc{def.acc} school.\textsc{acc} \textsc{} gym.\textsc{acc} \textsc{} office.\textsc{acc} today\\
\glt intended: `There are kids at the school/gym/office today.'
\z

Therefore, if we consider P-drop a diagnostic for non-adjuncts, we can argue that the PP is not an adjunct in \textit{vriskome}- and \textit{ime}-locatives and \textit{ime}-existentials. In contrast, it is an adjunct in \textit{exi}- and \textit{iparxo}-sentences. 

\subsection{Ability to act as stage topics}

Building on \citet{CohenErteschik2002} and \citet{Alexiadou2009}, it is expected that non-adjunct locative constituents will act as stage topics (i.e., as the items that indicate the spatiotemporal parameters of the sentence, the here-and-now of the discourse) contra locative adjuncts. Although the contrast is not very sharp, the speakers’ judgments point in the expected direction. The locative PP can act as a stage-topic in locative sentences independent of the verbal item. 

\ea \label{Kampanarou14}
\ea \label{Kampanarou14a}
\gll Oso ja to parti, ta peðja vriskode \textup{/} ine akoma eki. \\
as for \textsc{def.acc} party.\textsc{acc} \textsc{def.pl.nom} kid.\textsc{pl.nom} be.found.\textsc{3pl} \textsc{} be.\textsc{3pl} still there\\
\glt `As for the party, the kids are still there.'
\ex \label{Kampanarou14b}
\gll Oso ja to paljo spiti, ta epipla vriskode \textup{/} ine akoma eki. \\
as for \textsc{def.acc} old.\textsc{acc} house.\textsc{acc} \textsc{def.pl.nom} furniture.\textsc{pl.nom} be.found.\textsc{3pl} \textsc{} be.\textsc{3pl} still there \\
\glt `As for the old house, the furniture is still there.' 
\z
\z

It does so in \textit{ime}-existentials, too, as presented in (\ref{Kampanarou15}).

\ea \label{Kampanarou15}
\ea \label{Kampanarou15a}
\gll Oso ja to parti, ine pola peðja akoma eki. \\
as for \textsc{def.acc} party.\textsc{acc} be.\textsc{3pl} many kid.\textsc{pl.nom} still there\\
\glt `As for the party, there are many kids still there.'
\ex \label{Kampanarou15b}
\gll \textup{(?)} Oso ja to paljo spiti, ine pola epipla akoma eki. \\
\textsc{} as for \textsc{def.acc} old.\textsc{acc} house.\textsc{acc} be.\textsc{3pl} many furniture.\textsc{pl.nom} still there \\
\glt `As for the old house, there is much furniture still there.' 
\z
\z

Conversely, the PP in existential sentences with \textit{exi} and \textit{iparxo} cannot act as a stage topic, at least not as easily as in the cases above, cf. (\ref{Kampanarou16}).

\ea \label{Kampanarou16}
\ea \label{Kampanarou16a}
\gll \textup{??} Oso ja to parti, exi \textup{/} iparxun pola peðja akoma eki. \\
\textsc{} as for \textsc{def.acc} party.\textsc{acc} have.\textsc{3pl} \textsc{} exist.\textsc{3pl} many kid.\textsc{pl.acc/nom} still there\\
\glt intended: `As for the party, there are many kids still there.'
\ex \label{Kampanarou16b}
\gll \textup{??} Oso ja to paljo spiti, exi \textup{/} iparxun pola epipla akoma eki. \\
\textsc{} as for \textsc{def.acc} old.\textsc{acc} house.\textsc{acc} have.\textsc{3sg} \textsc{} exist.\textsc{3pl} many furniture\textsc{.pl.acc/nom} still there \\
\glt intended: `As for the old house, there is much furniture still there.' 
\z
\z

Therefore, this diagnostic indicates that the locative constituent is an adjunct only in \textit{exi}- and \textit{iparxo}-existentials. 

\subsection{Scopal behavior of \textit{again}}

Another piece of evidence is drawn on the \textit{again}-distribution, which, when used as a diagnostic, shows that the PP in \textit{exi}- and \textit{iparxo}-sentences behaves as an adjunct because it can be left outside the adverb’s scope and, hence, the main predication. Although this test has been used for identifying complex event structure by \citet{McCawley1968, McCawley1971}, \citet{Dowty1979}, and \citet{vonStechow1996}, it is not the use it will receive in this case since the constructions we are investigating are arguably mono-eventive. 

The adverb distribution is used as a diagnostic for severing the PP from the predicative layer, i.e., from the constituent where a Figure-Ground relationship is construed. Specifically, it predicts that if the PP can remain outside the scope of the adverb alone, then the PP is not part of the main predication. The examples below show that the adverb leaves the PP outside its scope and scopes over the NP (and the verb, as is always the case) only in \textit{exi}- and \textit{iparxo}-sentences (\ref{Kampanarou20}) and not in \textit{vriskome}- (\ref{Kampanarou17}) or \textit{ime}-sentences (\ref{Kampanarou18}, \ref{Kampanarou19}). 

In detail, witness that the locative sentence in (\ref{Kampanarou17b}) can only mean that the toys are again in Peter’s room. This is evidenced by the fact that the location cannot be negated (\textit{ala oxi} ‘but not’ is unacceptable). This means that \textit{ksana} `again' takes scope over the nominal and the locative PP. 

\ea \label{Kampanarou17}
\ea \label{Kampanarou17a}
\gll Xtes ta pexniðja vriskodan sto ðomatio tu Petru. \\
yesterday \textsc{def.pl.nom} toy.\textsc{pl.nom} be.found.\textsc{pst.3pl} in.\textsc{def.acc} room.\textsc{acc} \textsc{def.gen} Peter.\textsc{gen}\\
\glt `Yesterday, the toys were in Peter’s room.'
\ex \label{Kampanarou17b}
\gll Simera ta pexniðja vriskode ksana \textup{(}*ala oxi\textup{)} sto ðomatio tu Petru. \\
today \textsc{def.pl.nom} toy.\textsc{pl.nom} be.found.\textsc{3pl} again but not in.\textsc{def.acc} room.\textsc{acc} \textsc{def.gen} Peter.\textsc{gen} \\
\glt `Today, the toys are again (*but not) in Peter’s room.' 
\z
\z

The same is true for \textit{ime}-locatives. 

\ea \label{Kampanarou18}
\ea \label{Kampanarou18a}
\gll Xtes ta pexniðja itan sto ðomatio tu Petru. \\
yesterday \textsc{def.pl.nom} toy.\textsc{pl.nom} be.\textsc{pst.3pl} in.\textsc{def.acc} room.\textsc{acc} \textsc{def.gen} Peter.\textsc{gen}\\
\glt `Yesterday, the toys were in Peter’s room.'
\ex \label{Kampanarou18b}
\gll Simera ta pexniðja ine ksana \textup{(}*ala oxi\textup{)} sto ðomatio tu Petru. \\
today \textsc{def.pl.nom} toy.\textsc{pl.nom} be.\textsc{3pl} again but not in.\textsc{def.acc} room.\textsc{acc} \textsc{def.gen} Peter.\textsc{gen} \\
\glt `Today, the toys are again (*but not) in Peter’s room.' 
\z
\z

In \textit{ime}-existentials, too, the locative PP cannot remain outside the scope of again alone. The adverb takes scope over the NP-PP complex. 

\ea \label{Kampanarou19}
\ea \label{Kampanarou19a}
\gll Xtes itan kati pexniðja sto ðomatio tu Petru. \\
yesterday be.\textsc{pst.3pl} some toy.\textsc{pl.nom} in.\textsc{def.acc} room.\textsc{acc} \textsc{def.gen} Peter.\textsc{gen}\\
\glt `Yesterday, there were some toys in Peter’s room.'
\ex \label{Kampanarou19b}
\gll Simera ine ksana kati pexniðja \textup{(}*ala oxi\textup{)} sto ðomatio tu Petru. \\
today be.\textsc{3pl} again some toy.\textsc{pl.nom} but not in.\textsc{def.acc} room.\textsc{acc} \textsc{def.gen} Peter.\textsc{gen} \\
\glt `Today, there are again some toys (*but not) in Peter’s room.' 
\z
\z

Unlike the above, \textit{exi}- and \textit{iparxo}-existentials (\ref{Kampanarou20b}) have a reading in which the locative PP is left outside the scope of \textit{ksana} `again'. The fact that the PP can be negated suggests that only the NP is within the adverb’s scope (along with the verb). 

\ea \label{Kampanarou20}
\ea \label{Kampanarou20a}
\gll Xtes ixe \textup{/} ipirxan kati pexniðja sto ðomatio tu Petru. \\
yesterday have.\textsc{pst.3sg} \textsc{} exist.\textsc{pst.3pl} some toy.\textsc{pl.acc/nom} in.\textsc{def.acc} room.\textsc{acc} \textsc{def.gen} Peter.\textsc{gen}\\
\glt `Yesterday, there were some toys in Peter’s room.'
\ex \label{Kampanarou20b}
\gll Simera exi \textup{/} iparxun ksana kati pexniðja \textup{(}ala oxi\textup{)} sto ðomatio tu Petru. \\
today have.\textsc{3sg} \textsc{} exist.\textsc{3pl} again some toy.\textsc{pl.acc/nom} but not in.\textsc{def.acc} room.\textsc{acc} \textsc{def.gen} Peter.\textsc{gen} \\
\glt `Today, there are again some toys but not in Peter’s room.' 
\z
\z

\largerpage
The fact that the adverb can take scope only over the NP in the latter two cases indicates that the NP and the PP do not form a single constituent before the insertion of \textit{ksana} ‘again’. In other words, only the NP is part of the predication brought at least by the verb (if not by a head projected lower). If the PP stays outside the main predication in \textit{exi}- and \textit{iparxo}-existential sentences, the assumption that it merges as an adjunct is the only possible explanation. 

To sum up, the diagnostics presented so far suggest that the NP and the PP are part of the same constituent before the insertion of the verb in locatives and \textit{ime}-existentials. In contrast, they do not form a single constituent, at least by the time the verb is inserted in the structure of \textit{exi}- and \textit{iparxo}-existentials. For present purposes, I assume that this constituent is the predication layer (\textit{Pred}), i.e., the layer in which a Figure-Ground relationship is typically construed. In the former case (\ref{Kampanarou21a}), the predication layer contains both the Figure (NP) and the Ground (PP). In the latter case, (\ref{Kampanarou21b}), the same layer contains only the Figure (NP). The Ground is introduced later as an adjunct.  

\ea \label{Kampanarou21}
\ea \label{Kampanarou21a}
[\textsubscript{vP} v \textit{ime}\textup{/}\textit{vriskome} [\textsubscript{Pred} NP PP]] \\
\ex \label{Kampanarou21b}
[\textsubscript{vP} [\textsubscript{vP} v \textit{exi}\textup{/}\textit{iparxo} [\textsubscript{Pred} NP]] PP] \\
\z
\z

The rest of this chapter corroborates this conclusion and goes one step further as it attempts to pin down the hierarchy between the two constituents, i.e., whether the PP is inserted before the NP or vice versa. If this is conceived of in terms of a syntactic-tree representation, inserting the PP before the NP means that the former constituent is lower than the latter (NP>>PP). Conversely, the insertion of the PP after the NP suggests that the former is higher than the latter (PP>>NP) (see more in \sectref{KampanarouSec2.7}).

\subsection{Quantifier-scope readings}

The available quantifier-scope readings, as presented below, provide support for the assumption that the overt locative constituent in \textit{exi}- and \textit{iparxo}-existentials is severed from the main predication, i.e., it is an adjunct, and adjoins to a position hierarchically higher than the existent. To illustrate this, I consider the distribution in (\ref{Kampanarou22}) and (\ref{Kampanarou23}). 

\ea \label{Kampanarou22}
\ea \label{Kampanarou22a}
\gll Kapjo loɣotexniko vivlio vriskete se kaθe taksi. \\
some literary book\textsc{.nom} be.found.\textsc{3sg} in each classroom.\textsc{acc}\\
\glt `A literary book is in every classroom.' (∃>∀, ∀>∃)
\ex \label{Kampanarou22b}
\gll Kapjo loɣotexniko vivlio ine se kaθe taksi. \\
some literary book\textsc{.nom} be.\textsc{3sg} in each classroom.\textsc{acc}\\
\glt `A literary book is in every classroom.' (∃>∀, ∀>∃)
\ex \label{Kampanarou22c}
\gll Ine kapjo loɣotexniko vivlio se kaθe taksi. \\
be.\textsc{3sg} some literary book\textsc{.nom} in each classroom.\textsc{acc}\\
\glt `There is a literary book in every classroom.' (∃>∀, ∀>∃)
\z
\z

Keep in mind that when the interpretation reached is that of the surface order (∃>∀), the sentences mean that the same book (i.e., a copy of the same literary book) appears in every classroom. In contrast, under the inverse scope reading (∀>∃), there is a copy of a different literary book in every classroom. Note that getting the reading according to which the phrase \textit{kapjo loɣotexniko vivlio} ‘some literary book’ refers to the copy of one book is not easily accessible for a couple of my informants, yet for those that it is, the judgments point towards a distinction. 

As presented above, \textit{vriskome}- and \textit{ime}-sentences allow both scope readings. Building on assumptions made for ditransitives (see \cite{denDikken1995}, \cite{Harley1995, Harley2002}, \cite{Pesetsky1995}, \cite{Anagnostopoulou2003}), this suggests that the NP and the PP are contained in the same predication layer and the hierarchy between the constituents is depicted in the surface form: the NP is higher than the PP. In contrast, the existential sentences in (\ref{Kampanarou23}) do not receive the reading according to which the same copy appears in every classroom. They are mostly, if not solely, interpreted as conveying that each classroom has a different literary book.  

\ea \label{Kampanarou23}
\gll Exi \textup{/} Iparxi kapjo loɣotexniko vivlio se kaθe taksi. \\
have.\textsc{3sg} \textsc{} exist.\textsc{3sg} some literary book.\textsc{acc/nom} in each classroom.\textsc{acc}\\
\glt `There is a literary book in every classroom.' (*/?? ∃>∀, ∀>∃)
\z

Interestingly, \textit{exi}- and \textit{iparxo}-existential sentences lack the reading corresponding to the surface word order. According to the authors mentioned above, this means that these constituents are not in the same predication layer. On the other hand, it also shows that the surface word order does not correspond to the structural order of the constituents. In other words, it hints that the locative constituent is introduced in a position higher than the nominal (PP>>NP). The following section capitalizes on that. 

\subsection{Binding properties}

As further evidence for the hierarchy between the nominal and the locative constituent, I take into account the binding properties of the latter. Consider that the following sentences are used to describe books written by an author that integrates many traits of his personality and autobiographical information in his books. 

First, note that the complex NP \textit{o eaftos mu} constitutes a true anaphor in Greek, exempt from logophoric uses (\cite{AnagnostopoulouEveraert1999}; \cite{AngelopoulosEtAl2020}). Then, notice that a proper name contained in the locative PP in \textit{vriskome}- and \textit{ime}-locatives (\ref{Kampanarou24a}) and \textit{ime}-existentials (\ref{Kampanarou24b}) cannot bind an anaphor contained within the argumental noun phrase. An attempt to let the locative PP bind the anaphor leads to strong ungrammaticality. 

\ea \label{Kampanarou24}
\ea[*]{ \label{Kampanarou24a}
\gll  Ta komatja {\textup{[}tu eaftu tu\textup{]}\textsubscript{i}} vriskode \textup{/} ine sta vivlia \textup{[}tu Petru\textup{]}\textsubscript{i}. \\
 \textsc{def.pl.nom} part.\textsc{pl.nom} {of himself} be.found.\textsc{3pl} {} be.\textsc{3pl} in.\textsc{def.pl.acc} book.\textsc{pl.acc} \textsc{def.gen} Peter.\textsc{gen}\\
\glt intended: `The parts of himself are in Peter’s books.'
}
\ex[*]{ \label{Kampanarou24b}
\gll Ine pola komatja {\textup{[}tu eaftu tu\textup{]}\textsubscript{i}} sta vivlia \textup{[}tu Petru\textup{]}\textsubscript{i}. \\
 be.\textsc{3pl} many part.\textsc{pl.nom} {of himself} in.\textsc{def.pl.acc} book.\textsc{pl.acc} \textsc{def.gen} Peter.\textsc{gen} \\
\glt `There are many parts of himself in Peter’s books.' 
}
\z
\z

In contrast, the PP can bind an anaphor contained in the nominal argument in \textit{exi}- and \textit{iparxo}-existentials.

\ea \label{Kampanarou25}
\gll Exi \textup{/} Iparxun pola komatja {\textup{[}tu eaftu tu\textup{]}\textsubscript{i}} sta vivlia \textup{[}tu Petru\textup{]}\textsubscript{i}. \\
have.\textsc{3sg} {} exist.\textsc{3pl} many part.\textsc{pl.nom} {of himself} in.\textsc{def.pl.acc} book.\textsc{pl.acc} \textsc{def.gen} Peter.\textsc{gen} \\
\glt `There are many parts of himself in Peter’s books.' 
\z

Given that the antecedent binding the anaphor must be in a higher c-comman-\\ding position, the above distribution suggests that the surface order of the constituents in \textit{exi}- and \textit{iparxo}-existentials does not correspond to their hierarchical order in the syntax. In other words, the locative PP is in a structurally higher position than the nominal in these two cases (PP>>NP). 

\subsection{Interim summary}\label{KampanarouSec2.7}

Overall, the behavior of these sentences  suggests that \textit{vriskome}- and \textit{ime}-locatives and \textit{ime}-existentials are unified, at least regarding the behavior of the locative constituent which appears to act as a non-adjunct. In contrast, \textit{exi}- and \textit{iparxo}-sentences include a locative constituent exhibiting the opposite behavior. \tabref{tab:KampanarouLocativeConstituent} summarizes the properties of the locative constituent with respect to several diagnostics discussed in the previous sections. 

Note that the first four criteria indicate properties typical of adjuncts (\ding{52}) or non-adjuncts (\langscicross), whereas the latter two provide evidence for the hierarchy between the locative (PP) and the nominal (NP) constituent. 

\begin{table}
\caption{The behavior of the locative constituent}
\label{tab:KampanarouLocativeConstituent}
\fittable{
\begin{tabular}{llllll}
\lsptoprule
\textsc{Property} & \textit{vriskome}-\textsc{loc} & \textit{ime}-\textsc{loc} & \textit{ime}-\textsc{ex} & \textit{exi}-\textsc{ex} & \textit{iparxo}-\textsc{ex}\\
  \midrule
  Omission possible  & \langscicross & \langscicross & \langscicross & \ding{52} & \langscicheckmark\\
  \midrule
  P-drop possible  & \langscicross & \langscicross & \langscicross & \langscicheckmark & \langscicheckmark\\
  \midrule
  Act as stage-topic  & \langscicross & \langscicross & \langscicross & \langscicheckmark & \ding{52}\\
  \midrule
  Outside the scope of \textit{again} & \langscicross & \langscicross & \langscicross & \ding{52} & \ding{52}\\
  \midrule
  Surface quantifier-scope & NP>>PP & NP>>PP & NP>>PP & PP>>NP & PP>>NP\\
  \midrule
  PPs bind anaphors in NPs & NP>>PP & NP>>PP & NP>>PP & PP>>NP & PP>>NP\\
  \lspbottomrule
 \end{tabular}
 }
 \end{table}

Roughly represented in a syntactic tree, the findings of \sectref{KampanarouSec2} are as follows. The structure of \textit{vriskome}-locatives and \textit{ime}-sentences (\ref{Kampanarou26}) is presented in \figref{KampanarouFigure1}, where \textit{Pred} stands for the predication layer and \textit{vP} for the verbal phrase. If an adverb were to instantiate the locative constituent, it would still occupy the same position as the PP. \footnote{To answer a question posed by a reviewer, if the preposition heading the prepositional phrase was dropped, it would be assumed that the P-head of the locative PP is silent, i.e., phonetically not realized.}

\ea \label{Kampanarou26}
\ea \label{Kampanarou26a}
\gll Ta vivlia vriskode \textup{/} ine sto trapezi. \\
\textsc{def.pl.nom} book.\textsc{pl.nom} be.found.\textsc{3pl} {} be.\textsc{3pl} on.\textsc{def.acc} table.\textsc{acc}\\
\glt `The books are on the table.'
\ex \label{Kampanarou26b}
\gll Ine kati vivlia sto trapezi. \\
be.\textsc{3pl} some book.\textsc{pl.nom} on.\textsc{def.acc} table.\textsc{acc} \\
\glt `There are books on the table.' 
\z
\z

\begin{figure}
\begin{forest}
[vP
    [{v}
        [{ine/vriskode}]
    ]
    [Pred
        [NP
            [{ta/kati vivlia},roof]
        ]
        [PP
            [{sto trapezi},roof]
        ]
    ]
]
\end{forest}
%     \includegraphics[width=0.4\linewidth]{figures/Kampanarou-Figure1.png}
    \caption{The underlying structure of \textit{vriskome}-locatives and \textit{ime}-sentences}
    \label{KampanarouFigure1}
\end{figure}

In contrast, the structure of \textit{exi}- and \textit{iparxo}-existentials (\ref{Kampanarou27}) is shown in \figref{KampanarouFigure2}. Note that following traditional assumptions, I hold that the adjunct rewrites the verbal phrase (vP). Again, if an adverb appeared instead of the PP the adjunction process would stay the same.

\ea \label{Kampanarou27}
\gll Exi \textup{/} Iparxun kati vivlia sto trapezi \textup{/} eðo. \\
have.\textsc{3sg} {} exist.\textsc{3pl} many  book.\textsc{pl.nom/acc} on.\textsc{def.acc} table.\textsc{acc} {} here \\
\glt `There are some books on the table/here.' 
\z

\begin{figure}
%     \includegraphics[width=0.4\linewidth]{figures/Kampanarou-Figure2.png}
    
    \begin{forest}
    [vP
        [vP
                [v
                    [exi/iparxun]
                ]
                [Pred
                    [NP
                        [kati vivlia,roof]
                    ]
                ]
            ]
        [PP
            [sto trapezi,roof]
        ]
    ]
    \end{forest}
    \caption{The underlying structure of \textit{exi}- and \textit{iparxo}-existentials}
    \label{KampanarouFigure2}
\end{figure}

The following section addresses what these conclusions contribute to our understanding of the syntax underneath locative and existential sentences. It also offers a preliminary account for the fact that the sentences under discussion exhibit a variation in the surface word order: in locatives, the NP appears in a pre-verbal position, whereas in existentials, in a post-verbal one. 

\section{Towards a syntactic analysis}
\subsection{Tying our conclusions to existing analyses}

Although I will refrain from providing a formal syntactic analysis (see \cite{Kampanarou2023} for one), I will consider how compatible the findings of this research are with some widespread assumptions about the syntax behind locatives and existentials. 

First, we have concluded that the locative constituent in \textit{exi}- and \textit{iparxo}-exist-\\entials behaves as an adjunct, i.e., as a modifier. This means that it does not appear as an argument in the structure, and, thus, it must be severed from the main predication. This is in line with \citet{McNally1997} and \citet{Francez2007, Francez2009}, who put forth an adjunct-analysis for the locative constituent in existentials (see also \cite{Myler2018}).

Second, the locative constituent is argued to be part of the main predication in the rest of the cases. This conclusion has a two-fold interpretation. On the one hand, this could mean that the locative PP or adverb is the main predicate of the sentence, i.e., the constituent that introduces and construes the Figure-Ground relationship. In that case, the nominal constituent, i.e., the locatum in \textit{ime}- and \textit{vriskome}-locatives or the existent in \textit{ime}-existentials, would constitute the subject of the predicate, i.e., its argument. Further, the verb would be considered semantically vacuous, merging in the structure to “sentencify” the non-verbal predication brought by the locative. In broad terms, in this scenario, the locative constituent builds a Figure-Ground relationship by establishing a Ground that awaits an argument to take over the role of the Figure. This line of thinking is closely related to early assumptions made by \citet{Stowell1978}, \citet{Chomsky1981}, and \citet{Safir1982}, among others. 

On the other hand, since the diagnostic of P-drop suggests that the locative constituent in locatives and \textit{ime}-existentials is not only part of the main predication, but an argument, in particular, it could be posited that the verb serves as the predicate. As the main predicate, the verb selects the locative as its argument instantiating the Ground. In this case, the nominal constituent would be the second argument selected by the verb and would instantiate the Figure. This would concur with \citet{Milsark1974}, \citet{Keenan1987}, \citet{SagPollard1994}, and \citet{Dalmi2021}, among others.\footnote{Alternatively, following assumptions, prevalent in the generative tradition, holding that the predication is provided by a covert predicational head, e.g., a phonetically null \textit{Pred}-head that is in turn selected by the verb (\cite{Moro1997}, \cite{McCloskey2014}, \cite{Irwin2018}, and \cite{Myler2018}, among others), the distribution described in this chapter could be also considered compatible with the hypothesis that the overt locative constituent in locatives and \textit{ime}-existentials is an argument of this head. Kampanarou (\citeyear{Kampanarou2023}) offers an analysis in this spirit.} 

Even though the specifics of a syntactic analysis are not currently presented, this chapter has shown that the structure of existential sentences does not necessarily reduce to the structure of locatives. It has been demonstrated that the syntax behind the two types of sentences is identical, at least with respect to the status of the overt locative constituent, in the case of \textit{vriskome}- and \textit{ime}-locatives and \textit{ime}-existentials. In contrast, it is not the same if \textit{ime}-sentences and \textit{vriskome}-locatives are compared to \textit{exi}- and \textit{iparxo}-existentials. This confirms that both strands of analyses mentioned in \sectref{KampanarouSec1.1} have some merit of truth.

Nonetheless, it is essential, that despite their structural affinity, the surface word order of \textit{ime}-locatives and \textit{ime}-existentials is not the same. The following section attempts to provide an explanation for this.

\subsection{A note on word order patterns}

Even though we posit an identical structure for \textit{ime}-locatives and \textit{ime}-existentials, their surface form differs since the nominal argument that appears pre-verbally in the former case appears post-verbally in the latter case. This is schematically demonstrated in (\ref{Kampanarou28}).

\ea \label{Kampanarou28}
\ea \label{Kampanarou28a}
[Nominative DP]\textsubscript{\textsc{locatum}} [\textit{ime}] [Locative PP/Adverb]\textsubscript{\textsc{location}} \\ (Locative) \\
\ex \label{Kampanarou28b}
[\textit{ime}] [Nominative DP]\textsubscript{\textsc{existent}} [Locative PP/Adverb]\textsubscript{\textsc{location}} \\ (Existential)\\
\z
\z

Interestingly, ‘‘manipulating’’ the word order in order to change the function of the sentence is a common strategy cross-linguistically. However, as pointed out for Gawarbati \citep{PanovaLiljegren2025} and several other languages (\cite{CohenErteschik2002}; \cite{BeaverEtAl2005}; \cite{ParteeBorschev2006}; \cite{Francez2007, Francez2009}; \cite{GastHaas2011}; \cite{Cruschina2012}; \cite{Bentley2013}; \cite{Halevy2022}), changing the word order is not a simple surface scrambling. 

According to \citet[217]{ParteeBorschev2004}, locatives and existentials differ in their perspectival center, i.e., “the participant chosen as the point of departure for structuring the situation”. The authors claim that in (\ref{Kampanarou28a}), the proposition, i.e., the Figure-Ground relationship in our terms, is expressed through the Figure's perspective. In contrast, the same proposition is expressed through the perspective of the Ground in (\ref{Kampanarou28b}). For the same authors, this variation is depicted on the surface order: the Figure is fronted only in the case of locatives. 

A question that has considerably concerned research through the years is whether this shift in perspective is related to the information structure, i.e., the part of syntax where the speaker/addressee’s mental representations of how the information is “packaged” are encoded. Although several assumptions have been made in the literature, particularly with respect to locatives and existentials (see the references above), \citet{Kampanarou2023} shows that the two are co-related yet not directly as the word order is what creates the link between them: information Structure manipulates the word order to determine the topic of the sentence, and this leads to the variation in the perspectives. 

In more detail, in the sense of \citet{Lambrecht1994}, the topic is primarily a concept independent of linguistic expressions: the topic is the referent whom the situation is about (see also \cite{Strawson1964}, \cite{Reinhart1981}). More specifically, topics are known referents that function as the anchor for the utterance. For this reason, it is often assumed that topics provide the perspective of the utterance, i.e., they act as their perspectival centers. Focusing on Greek, \citet{RoussouTsimpli2006}, \citet{SpyropoulosRevithiadou2009}, and \citet{Sifaki2013} argue that topics in this language are items that are structurally introduced in the vP-layer, broadly speaking, and move to the upper part of the structure, i.e., to information sructure. 

In the cases under discussion, this is what happens in locative sentences: the nominal constituent is the known referent. Thus, it becomes the topic of the sentence, is fronted, and constitutes its perspectival center. In contrast, the fact that the verbal item surfaces sentence-initially in existentials suggests that neither argument constitutes the topic of the sentence and its perspectival center. The utterance does not have a participant as its anchor point. Nonetheless, it could be assumed that the word order it exhibits suggests that the whole situation is presented from the perspective of the situation itself. Therefore, as for \textit{ime}-sentences, it could be said that despite their structural similarities, their variation in the surface word order is determined later in the derivation, i.e., in their information structure. 

In any case, it is essential that word order is tied primarily to the choice of the perspectival center. Whether the two are also dependent on information structure needs further support, although the influential work by \citet{CohenErteschik2002} and the remarks in \citet[70--73, 198--201]{Kampanarou2023} point towards the postulation that it is. 

\section{Conclusions}

The discussion in this paper has highlighted that to achieve the existential function, the grammar (a) opts for using a structure that is already available for locative sentences or (b) makes use of a structure that is similar but not identical to the one exploited for locatives. Specifically, the Greek data has shown that the locative function is achieved via a single structure, at least in regard to the status of the locative constituent. In this structure, which is assumed for \textit{vriskome}- and \textit{ime}-locatives, the locative constituent is part of the main predication. On the other hand, the existential function is reached via a syntactic structure similar to locatives (this is what happens with \textit{ime}-existentials) or through a different structure where the locative constituent is an adjunct (\textit{exi}- and \textit{iparxo}-existentials). 

Overall, the Greek distribution has shown there are several strategies available for achieving the locative and existential functions. It is up to the language to choose which one(s) will be used for each function and whether additional strategies (e.g., manipulating the word order through information structure) may interfere. This chapter has made an initial contribution regarding this matter. A more detailed investigation of how the rest of the items contained in these sentences behave and, thus, what their status is, will shed more light on the structure lying underneath locatives and existentials. 


\section*{Acknowledgments}

I want to thank Artemis Alexiadou, Eirini Apostolopoulou, Elena Anagnostopou-\\lou, Evripidis Tsiakmakis, as well as the participants of the workshop “Locative and existential predication -- Core and periphery” that was held at SLE56 in 2023, specifically Anne Carlier and Evangelia Vlachou, for fruitful discussions regarding several parts of this chapter. Special thanks are also owed to Martin Haspelmath and Anastasia Panova for their reviews of my manuscript and to the editors of this volume, Rodolfo Basile, Josefina Budzisch, and Chris Lasse Däbritz for their insightful comments and their assistance throughout the publishing process. 

\sloppy
\printbibliography[heading=subbibliography,notkeyword=this]
\end{document}
