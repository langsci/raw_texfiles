\documentclass[output=paper,colorlinks,citecolor=brown]{langscibook}
\ChapterDOI{10.5281/zenodo.16838068}
\author{Jorge Agulló\orcid{}\affiliation{University of Cambridge}}
%\ORCIDs{}

\title[Language contact and the decline of the definiteness effect]{Language contact and the obsolescence of the Definiteness Effect: New data from Spanish in contact with Catalan}

\abstract{Existential constructions have been held to be sensitive to the Definiteness Effect, i.e., a ban against definiteness and/or specificity in the pivot position. The Definiteness Effect is known to be subject to wide cross-linguistic variation: whereas some languages show a strong version (Spanish, French), others have weaker versions of it (Italian, Catalan). This chapter surveys varieties of Spanish in contact with Catalan. A huge corpus of existential constructions has been gathered from the Audible Corpus of Spoken Rural Spanish. The hypothesis is put forth here that the Definiteness Effect weakens when Spanish meets Catalan: definite, specific pivots are found in these varieties.}

\IfFileExists{../localcommands.tex}{
   \addbibresource{../localbibliography.bib}
   % add all extra packages you need to load to this file

\usepackage{tabularx,multicol}
\usepackage{url}
\urlstyle{same}

\usepackage{listings}
\lstset{basicstyle=\ttfamily,tabsize=2,breaklines=true}

\usepackage{langsci-basic}
\usepackage{langsci-optional}
\usepackage{langsci-lgr}
\usepackage{langsci-osl}
% \usepackage{./langsci/styles/langsci-lgr}
% \usepackage{./langsci/styles/langsci-osl}
% \usepackage{langsci-gb4e}

\usepackage{tikz}
\usetikzlibrary{patterns,calc}
\pgfdeclarepatternformonly{south east lines}{\pgfqpoint{-0pt}{-0pt}}{\pgfqpoint{3pt}{3pt}}{\pgfqpoint{3pt}{3pt}}{
    \pgfsetlinewidth{0.6pt}
    \pgfpathmoveto{\pgfqpoint{0pt}{3pt}}
    \pgfpathlineto{\pgfqpoint{3pt}{0pt}}
    \pgfpathmoveto{\pgfqpoint{.2pt}{-.2pt}}
    \pgfpathlineto{\pgfqpoint{-.2pt}{.2pt}}
    \pgfpathmoveto{\pgfqpoint{3.2pt}{2.8pt}}
    \pgfpathlineto{\pgfqpoint{2.8pt}{3.2pt}}
    \pgfusepath{stroke}}
    
\usepackage{stmaryrd}
\usepackage{wasysym}
\usepackage{multirow}
\usepackage{caption}
\usepackage{subcaption}
\usepackage{mathrsfs}
\usepackage{qtree}

\usepackage{linguex}


   %pminos do not split footnotes
% \interfootnotelinepenalty=10000 %Footnote in Laporte chapters has to be split SN


%\DeclareIndexNameFormat{default}{%
%\nameparts{#1}%
%\usebibmacro{index:name}%
%{\index[names]}%
%{\namepartfamily}%
%{\namepartgiveni}%
% {}% L1
% {}% L2
%{\namepartprefix}% generates spurious space L3
%{\namepartsuffix}% generates spurious space L4
%}

%  {\DeclareIndexNameFormat{default}{%
%     \usebibmacro{index:name}{\index[names]}{#1}{#3}{#5}{#7}}}

%\DeclareIndexNameFormat{default}{%
%  \usebibmacro{index:name}{\sindex[nom]}{#1}{#3}{#5}{#7}}

%\DeclareIndexNameFormat{default}{%
%  \usebibmacro{index:name}{\sindex[person]}{#1}{#3}{#5}{#7}}
%\DeclareIndexNameFormat{default}{%
%\nameparts{#1} \usebibmacro{index:name}{\sindex[person]]}{\namepartfamily}{‌​\namepartgiven}{\nam‌​epartprefix}{\namepa‌​rtsuffix}}

%\newcommand{\smiley}{:)}

%\renewbibmacro*{index:name}[5]{%
%\usebibmacro{index:entry}{#1}%
%{\iffieldundef{usera}{}{\thefield{usera}\actualoperator}\mkbibindexname{#2}{#3}{#4}{#5}}}

% \newcommand{\noop}[1]{}

%remove for final
%\overfullrule=1mm

\newcommand{\tobi}[2]}}
\renewcommand{\S}[1]{\tobi{#1}{\textsc{*}}}

% this volume references
% puts: [this volume]
% already defined: \citetv
%\newcommand{\citepv}[1]{(\citeauthor{#1} \citeyear*{#1} [this volume])}
\newcommand{\citealtv}[1]{\citeauthor{#1} \citeyear*{#1} [this volume]}

%parentheses around example number
\newcommand{\pref}[1]{(\ref{#1})}

% in-text examples

\newcommand{\lnex}[1]{\textit{#1}} %target lang word
\newcommand{\lnlit}[1]{(lit.: `#1')} %literal reading
\newcommand{\lnlat}[1]{(#1)} % latinization
\newcommand{\lntrans}[1]{`#1'} %translation
\newcommand{\lnexl}[2]%
{\lnex{#1}{} \lnlat{#2}} % ex with latinization
\newcommand{\lnexlat}[3]{\lnex{#1}{} \lnlat{#2}{} \lntrans{#3}} % ex with latinization and tranl.

%ch01
\newcommand{\co}[1]{\mbox{\textbf{#1}}}

%ch09

\newcommand{\cyrbulg}[1]{\begin{otherlanguage*}{bulgarian}#1\end{otherlanguage*}}


%ch10
\newcommand{\nlp}{{\small NLP}}
\newcommand{\mwe}{{\small MWE}}
\newcommand{\rae}{{\small RAE}}
\newcommand{\lvc}{{\small LVC}}
\newcommand{\pos}{{\small P}o{\small S}}
%\newcommand{\todo}[1]{ \textcolor{red}{#1} }

%\renewcommand{\labelenumi}{\theenumi}
%\ainamefmt{{vv}{ll}{, ff}{, jj}} % fullname

\newcommand{\biberror}[1]{{\color{red}#1}}

\newcommand{\osenovaitem}{--~}
   %% hyphenation points for line breaks
%% Normally, automatic hyphenation in LaTeX is very good
%% If a word is mis-hyphenated, add it to this file
%%
%% add information to TeX file before \begin{document} with:
%% %% hyphenation points for line breaks
%% Normally, automatic hyphenation in LaTeX is very good
%% If a word is mis-hyphenated, add it to this file
%%
%% add information to TeX file before \begin{document} with:
%% %% hyphenation points for line breaks
%% Normally, automatic hyphenation in LaTeX is very good
%% If a word is mis-hyphenated, add it to this file
%%
%% add information to TeX file before \begin{document} with:
%% \include{localhyphenation}
\hyphenation{
    Beck-man
    Ngu-yen
    back-chan-nel
    back-chan-nels
    mo-not-o-nous
    ste-reo-typ-i-cal
}

\hyphenation{
    Beck-man
    Ngu-yen
    back-chan-nel
    back-chan-nels
    mo-not-o-nous
    ste-reo-typ-i-cal
}

\hyphenation{
    Beck-man
    Ngu-yen
    back-chan-nel
    back-chan-nels
    mo-not-o-nous
    ste-reo-typ-i-cal
}

   \boolfalse{bookcompile}
   \togglepaper[7]%%chapternumber
}{}

%\pretocmd{\gll}{\def\eachwordone{\itshape}\def\eachwordtwo{\normalfont}}{}{}

\begin{document}
\maketitle

\section{Introduction} \label{Introduction-Agullo}
\subsection{Aim and scope} \label{AimAndScope}

This work casts some light onto the linguistic consequences of contact between Spanish (Romance, Indo-European, [stan1288]) and Catalan (Romance, Indo-European, [stan1289]). It will mainly concentrate on a grammatical restriction labelled \textsc{definiteness effect} or \textsc{definiteness restriction}, which bans definite nouns in the pivot position of the existential sentence. The restriction will be argued to display a sharp linguistic variation: Spanish observes a strong version of the restriction, whereas Catalan allows for definite or specific nouns in the pivot position and hence abides by a weaker version of the effect. Linguistic contact between these two languages will be shown to trigger immunity to the grammatical restriction, as so-called \textsc{interdialectal} variants emerge. New data will be brought to the fore in support of the hypothesis that Spanish in contact with Catalan amnesties the definiteness effect.

The study of grammatical restrictions is by no means a cursory matter, insofar as its formulation is theoretically dependent, i.e., it bears upon a theory and a set of axioms. Grammatical restrictions, which are fairly abstract in nature, are thought to be hypotheses about the limits of possible human grammars. Some of these grammatical restrictions were brought to the fore by \citet{Ross1967} and were since then at the core of linguistic inquiry within Government and Binding Theory (\cite{Chomsky1981,Chomsky1982}). Grammatical restrictions were soon called into question both by theoreticians and typologists, but some of them have fared better than others. Milsark’s (\citeyear{Milsark1974}, \citeyear{Milsark1977})  definiteness effect is illustrative in this respect: the restriction has prevailed, even though grammatical restrictions in general have been progressively substituted for more general principles of grammar, particularly within Chomsky’s (\citeyear{Chomsky1995})  minimalist framework. Scant attention has been paid, nevertheless, to the intra-linguistic variation of the effect. The path I follow here, accordingly, fills this void, as it tackles the grammatical variation of the effect in Ibero-Romance languages and furthers our knowledge of new syntactic variants in contact situations between Spanish and Catalan.


A few words of caution are, nevertheless, in order. Whether grammatical restrictions are relevant for the analysis or not relies on the validity bestowed upon the theoretical assumptions underlying them. If the wisdom of such an approach is questioned, there remains no intuitive force to the notion of grammatical restriction. To sharpen the issues here, let us define existential constructions as construction-functions, in Croft’s (\citeyear{Croft2014}, \citeyear{Croft2016}) sense: existentials assert the existence or location of a brand-new or non-familiar entity, as already argued, for instance, in \citet{Abbott1992,Abbott1993,Abbott1999}. Note that, in passim, I think of existential constructions as defined by a primitive, cross-linguistically tenable conceptual semantic property (cf. \cite{Haspelmath2010} and \cite{Newmeyer2010}), which is far from mere descriptive categories (\cite{Haspelmath2018}). Given that indefiniteness has been linked either to the novelty of the noun phrase or to its non-familiarity (cf. \cite{Heim1982,Heim2019}) for different approaches), definiteness of the pivot is hence restricted in the existential.


If these arguments prove to be on the right track, the question shall be raised as to whether Milsark’s (\citeyear{Milsark1974}, \citeyear{Milsark1977}) definiteness effect is nothing more than an \textit{ad hoc} stipulation about the indefiniteness of the pivot. I will assume here, nonetheless, that grammatical restrictions can be used as merely descriptive artifacts, that is, as descriptions of what is not possible in the grammar of a given language.

\subsection{Basic descriptive data} \label{BasicDescriptiveData}
Prior to tackling basic descriptive data, let us assume, namely as a matter of execution, Aissen’s (\citeyear{Aissen2003}) Definiteness Scale as formulated in (\ref{Agullo1}):

\ea \label{Agullo1}
  {\textsc{definiteness hierarchy}}{}{\citealt[437]{Aissen2003}}\\
Personal pronoun > Proper name > Definite NP > Indefinite specific NP > Non-specific NP\\
\z

The definiteness scale in (\ref{Agullo1}) has been refined by \citet{Farkas2000,Farkas2002} and \citet{Sabbagh2016}, among others. Milsark’s (\citeyear{Milsark1974}, \citeyear{Milsark1977}, \citeyear{Milsark1979}) definiteness effect restricts the presence in the pivot position of the existential of definite noun phrases, i.e., items to the left of the definiteness hierarchy. This is illustrated by the English (Germanic, Indo-European, [stan1293]) examples in (\ref{Agullo2}). Definite noun phrases (\ref{Agullo2a}), proper nouns and personal pronouns (\ref{Agullo2b}), exhaustive quantifiers (\ref{Agullo2c}), or quantificational determiners (\ref{Agullo2d}) are barred from the pivot position of the existential sentence.

\ea \label{Agullo2}
(\cite[195]{Milsark1974})\\
\ea \label{Agullo2a}
*There is \{the $\sim$ John’s $\sim$ that\} dog in the room.\\
\ex \label{Agullo2b}
*There is \{John $\sim$ him\} in the room.\\
\ex \label{Agullo2c}
*There are \{all $\sim$ both\} dogs in the room.\\
\ex \label{Agullo2d}
*There is \{every $\sim$ each\} dog in the room.\\
\z
\z

Existential sentences in Spanish are known to behave alike. Relevant data is brought to the fore in (\ref{Agullo3})--(\ref{Agullo7}) below:

\ea \label{Agullo3}
\langinfo {Mexican Spanish}{}{CORPES XXI, Mexico}\\
\gll Había gente por todas partes. \\
there.were people for every parts \\
\trans ‘There were people everywhere.’\\
\z

\ea \label{Agullo4}
\langinfo {Argentinian Spanish}{}{CORPES XXI, Argentina}\\
\gll Había un olor raro. \\
there.was a	smell weird	 \\
\trans ‘There was a weird smell.’\\
\z

\ea \label{Agullo5}
\langinfo {Spanish}{}{personal knowledge}\\
\gll *Había él en la habitación. \\
there.was he in \textsc{art} room \\
\trans ‘There was him in the room.’\\
\z

\ea \label{Agullo6}
\langinfo {Spanish}{}{personal knowledge}\\
\gll *Hay Jorge en la habitación. \\
there.was Jorge in \textsc{art} room \\
\trans ‘There was Jorge in the room.’\\
\z

\ea \label{Agullo7}
\langinfo {Spanish}{}{personal knowledge}\\
\gll Hay	\{*el $\sim$ *tu\}	libro	en	la	habitación. \\
there.is	\{the $\sim$ your\}	book	in	ART	room \\
\trans ‘There is \{the $\sim$ your\} book in the room.’\\
\z

\ea \label{Agullo8}
\langinfo {Spanish}{}{personal knowledge}\\
\gll Hay \{*cada $\sim$ *todo\}	libro	en	la habitación. \\
there.is	\{each $\sim$ every\}	book	in	ART	room \\
\trans `There is \{each $\sim$ every\} book in the room.'\\
\z

The pivot position of the existential sentence can be occupied by bare noun phrases (\ref{Agullo3}) and indefinite noun phrases (\ref{Agullo4}). Definiteness seems to be restricted. The definiteness effect prevents personal pronouns (\ref{Agullo5}), proper nouns (\ref{Agullo6}), and definite constituents (\ref{Agullo7}) from occupying the pivot position. Quantified constituents, as those headed by \textit{cada} ‘each’ or \textit{todo} ‘every’, are also barred, unless they quantify over types (cf. \cite{McNally1997}).
The definiteness effect in existentials is thought to be subject to wide cross-linguistic variation, even within the same linguistic family. It is robust in languages like Spanish (cf. \ref{Agullo3}--\ref{Agullo8}), Galician (Romance, Indo-European, [gali1258]), and French (Romance, Indo-European, [stan1290]) (cf. \cite{Bouchard1997} and \cite{PaykinVanDeVelde2021}), whereas Catalan and Italian (Romance, Indo-European, [ital1282]) have weaker versions of it (see, among others, \cite{Leonetti2008}). Catalan allows proper nouns and definite constituents, as shown by 
\citet{Rigau1988} and \citet{BrucartEtAl2006}, in the pivot position of \textit{haver-hi} ‘be’, as shown in (\ref{Agullo9})--(\ref{Agullo10}) below.

\ea \label{Agullo9}
\langinfo {Catalan}{}{\cite[45]{Ramos1998}}\\
\gll Hi havia la Marta. \\
\textsc{loc} there.was \textsc{art}	Marta	 \\
\trans ‘There was Marta.’\\
\z


\ea \label{Agullo10}
\langinfo {Catalan}{}{\cite[396]{Rigau1997}}\\
\gll Hi havia el president. \\
\textsc{loc} there.was \textsc{art}	president	 \\
\trans ‘There was the president.’\\
\z

As suggested by a reviewer, the point should be made clear that the sentences in (\ref{Agullo9}) and (\ref{Agullo10}) cannot be interpreted as possessive clauses: the presence of locative \textit{hi} triggers an existential interpretation. 
Personal pronouns, at the beginning of Aissen’s (\citeyear{Aissen2003}) Definiteness Scale, have been labeled as ungrammatical by some grammarians (e.g., \cite{Rigau1988}, \cite{Ramos1998}, and \cite{Brucart2006}). In fact, \citet{Rigau1988} derives this ungrammaticality from Belletti’s (\citeyear{Belletti1987}, \citeyear{Belletti1988}) proposal that the pivot of the existential is marked with Partitive, not Accusative case (see also \cite[§2.6.4]{Chomsky1995}); hence nominative personal pronouns are excluded. Several authors (e.g., \cite{Villalba2016} and \cite{GraciaRoca2017}) have cast some doubt on this generalisation: there are thought to be some rather strict conditions under which personal pronouns in Catalan are acceptable, such as coordination, or the use of adverbials such as \textit{nomès} ‘only’ or \textit{també} ‘also’ (cf. \cite{Cruschina2016}, who reports similar facts in Italian). 

\ea \label{Agullo11}
\langinfo {Catalan}{}{CTILC, d’Ors, Eugeni: Gualba, la de mil veus, 1935}\\
\gll Hi ha ella mateixa i sa figura. \\
\textsc{loc} there.was	her herself and her figure \\
\trans ‘There was she herself and her figure.’\\
\z

\ea \label{Agullo12}
\langinfo {Catalan}{}{CTILC, Juan Arbó, Sebastià: Tino Costa, 1947}\\
\gll Aquella porta darrera la qual hi havia la seva filleta, hi havia ell.\\
that door behind \textsc{art.f} which \textsc{loc} there.was \textsc{art.f} \textsc{poss.3.f} daughter \textsc{loc} there.was him \\
\trans ‘That door behind which there was her little daughter, there was him[ — the safe company.]’’\\
\z

The fact that Catalan allows for both proper names (\ref{Agullo9}) and personal pronouns (\ref{Agullo11} and \ref{Agullo12}) is interesting in several ways. Personal pronouns and proper names are generally thought to be located in the highest position of Aissen’s (\citeyear{Aissen2003}) Definiteness Hierarchy (see, among others, \cite{Farkas2000}). 

\subsection{Previous accounts} \label{PreviousAccounts}
An exhaustive survey of the literature yields a clear diagnosis: close attention has been paid to the theoretical formulation and prediction of the definiteness effect, but insights into its intra-linguistic variation remain scant, apart from data on Italian dialects (see \cite{Bentley2015} and \cite{Cruschina2016}). Theoretical explanations of the definiteness effect may be classified as to whether more or less abstract syntactic principles are invoked (e.g., \cite{Safir1982}, \cite{LongaEtAl1996,LongaEtAl1998}, \cite{Belletti1987,Belletti1988}, \cite{LaFauciLoporcaro1997}), semantic ones (e.g., \cite{Milsark1974}, \cite{Lumsden1988}, \cite{Enç1991}), or pragmatic-discursive ones (e.g., \cite{Abbott1993,Abbott2006}, \cite{WardBirner1995}). It is widely accepted, particularly in relation to Romance languages, that the theoretical complexity of the definiteness effect boils down to a conspiracy between syntactic, semantic, and discoursive properties: \citet{Zucchi1995,Zucchi2003}, \citet{McNally1997}, and \citet{Leonetti2008}, for instance, acknowledge that the definiteness effect seems to be at the interplay between syntax, semantics, and some discourse-pragmatic factors.

The crosslinguistic distribution of the definiteness effect has been tackled and partially unveiled (see, for instance, \cite{McNally2016}). In fact, there are thought to be deep cross-linguistic differences with regard to the effect. Italian and Catalan have been argued to be immune to the effect (e.g.,  \cite{LongaEtAl1996,LongaEtAl1998}, \cite{LaFauciLoporcaro1997}, \cite{Bartra1987}, \cite{Villalba2013}) (cf., nonetheless, \cite{Leonetti2008} and \cite{Bentley2013}), whereas Spanish or Galician adhere more consistently to the effect (see \cite{Rodríguez-Mondoñedo2007} for Spanish). The hypothesis that the definiteness effect can be parametrised has been contended by several authors: crosslinguistic differences can be attributable to a difference between locative-flavoured (i. e. preposed locative coda) constructions and proper existentials (e.g., \cite{Cruschina2016}). It has come to be known that languages generally allow for definite pivots under certain circumstances (cf. \cite{Bentley2013} and the data in \sectref{VarietiesAtTheCrossroads}), but what is subject to crosslinguistic variation are (i) the conditions under which definite pivots are acceptable and (ii) the positions of Aissen’s (\citeyear{Aissen2003})  Definiteness Hierarchy (see \ref{Agullo2}) that might be grammatical as pivots of existential sentences under those conditions.

The attention rivetted by the definiteness effect from a theoretical point of view stands in stark contrast to our knowledge of its variation within a particular language, which remains notably sketchy. Spanish in contact with Catalan has been devoted close attention with regard to the phonetic inventory, lexical alternations, semantic loans, and several morphological or morphosyntactic phenomena (see, among many others, \cite{GargalloGil2001}, \cite{MartínZorraquinoFortCanellas1996}, \cite{Sinner2004} and references cited therein, \cite{BlasArroyo2011a}, \cite{BlasArroyo2011b}, \cite{RamalloAbalo2023}). Purely syntactic phenomena, nevertheless, have been scarcely accounted for. A conspiracy of two factors, as I now tend to believe, seems to be at play: (a) from a theoretical point of view, syntax has been conceived as less variable or more stable than other areas, such as phonetics or the lexicon (e.g., \cite{Chomsky1981}, \cite{Muysken2005,Muysken2012}); and, additionally, (b) the quantitative perspective has assumed, namely as a matter of execution, that syntactic variables are less easily formulated or demarked (e.g., \cite{Labov1975,Labov1994}, \cite{Tagliamonte2014}, \cite{Silva-Corvalán1984,Silva-Corvalán1989}, \cite[n. 10]{MartínButragueño1994}, \cite[§2.3]{CheshireEtAl2005}, \cite{Schwenter2011}).

Some observations have been made in relation to particular syntactic phenomena: 1) Spanish in contact with Catalan is said to use right dislocations more readily (\cite{Villalba2011}); 2) differential object marking (see \sectref{BasicDescriptiveData} above) has been deemed unstable (or even more less frequent) in these varieties (\cite{GomezAlvarez2022}); 3) preverbal double negation is found in these areas (e.g., \cite{CasanovasCatala1996}, \cite{PaaschSinner2021}, \cite{Sanromán2022}).
An overview of these approaches yields a striking diagnosis: while each of these phenomena has been tackled in the source language, its transmission in a contact situation remains unaccounted for. The same holds true for the definiteness effect in existential sentences: our knowledge of its distribution in Catalan is thorough (see, among many others, \cite{Rigau1988,Rigau1993,Rigau1997,Rigau2008}, \cite{Ramos1998}, \cite{GraciaRoca2017}), as it is in Spanish (e.g., \cite{Leonetti2008}), but no attention has yet been paid to varieties of Spanish in Catalan-speaking regions, apart from Agulló’s (\citeyear{Agulló2022}) sketchy observations. As a consequence, our comprehension of varieties of Spanish in contact with Catalan can safely be regarded as far from complete.

\section{Research methodology} \label{ResearchMethodology}
\subsection{Research hypothesis} \label{ResearchHypothesis}

Catalan existential sentences differ greatly from their Spanish counterparts. A general research hypothesis, hence, reveals itself, as formulated in (\ref{Agullo13}):

\ea \label{Agullo13}
Varieties of Spanish in contact with Catalan are immune to the definiteness effect.\\
\z

Some new data will be brought to the fore that supports the validity of the claim in (\ref{Agullo13}). The hypothesis will be sharpened and argued to be derived from more general principles of contact situations.

\subsection{Materials and methods} \label{MaterialsAndMethods}
This study is based upon reflection and observation of a rather large database (\textit{N} = 4995) of \textit{haber} ‘to be’ existential constructions in European Spanish gathered from the \textsc{Audible Corpus of Spoken Rural Spanish} (henceforth COSER, after its translation in Spanish) (\cite{Fernández-Ordóñez2005}), a lemmatised corpus of rural varieties of European Spanish. The aim of this study is to bring to the fore hitherto unnoticed data in support of the hypothesis in (\ref{Agullo13}) and to provide some insight into how these observed differences can be accounted for.

Some observations on the data collection methodology are in order, as carefully pointed out by a reviewer. The COSER data is obtained through sociolinguistic interviews conducted in small, rural villages. Informants meet the following criteria: (a) advanced age, (b) lifetime or prolonged residence in the town, (c) rural work as their main occupation, and (d) low (or null) educational level (see \cite{Fernández2004,Fernández2009}; and \cite{FernándezPato2020} for methodological remarks). Note, albeit incidentally, that the target speaker of the corpus overrides what has come to be known in traditional dialectology as NORM ‘non-mobile older rural male’, in that both men and women are equally selected as informants. As a consequence, data from rural vernacular varieties, i.e., representative of the language spoken in the village, is collected in a consistent and uniform way. More specifically, the methodological guidelines of the corpus have a direct effect on the conclusions drawn here: generalisations are based upon data from rural vernacular varieties.

\section{Varieties at the crossroads between Spanish and Catalan: a shift in the Definiteness Effect} \label{VarietiesAtTheCrossroads}

\subsection{Data from contact varieties: the obsolescence of the Definiteness Effect} \label{DataFromContactVarieties}

In this section, I will tackle data as in (\ref{Agullo14}) -- (\ref{Agullo16}), which show definite, specific nouns in the pivot position (\ref{Agullo14}), proper names (\ref{Agullo15}), and even so-called eventive existential constructions, as the ones in (\ref{Agullo16}).

\ea \label{Agullo14}
\ea \label{Agullo14a}
\langinfo {Spanish}{Son Macià (Manacor) [Mallorca]}{COSER-4915\_01}\\
\gll Después había la República. \\
after there.was \textsc{art} republic	 \\
\trans ‘Then, there was the Republic.’ \\

\ex \label{Agullo14b}
\langinfo {Spanish}{Mura [Barcelona]}{COSER-0804\_01}\\
\gll Ahora hay la calefacción. \\
now	there.is \textsc{art} heating \\
\trans ‘Now, there is the heating.’ \\

\ex \label{Agullo14c}
\langinfo {Spanish}{La Serra (Torre de Claramunt) [Barcelona]}{COSER-0803\_01}\\
\gll Cuando	vine de	la	mili había la carretera. \\
when came.1\textsc{sg} of \textsc{art} military.service there.was \textsc{art} road \\
\trans ‘When I returned from the military service, there was the road.’ \\
\z
\z

\ea \label{Agullo15}
\langinfo {Spanish}{La Serra (La Torre de Claramunt) [Barcelona]}{COSER-0803\_01}\\
\gll Te	obligaban a	hablar en castellano, cuando había Franco.	 \\
you.\textsc{acc} forced to speak in	Castilian when there.was Franco	 \\
\trans ‘You were forced to speak in Castilian when Franco was there.’ (Lit. ‘when there was Franco.’)\\
\z

\ea \label{Agullo16}
\ea \label{Agullo16a}
\langinfo {Spanish}{La Serra (Torre de Claramunt) [Barcelona]}{COSER-0803\_01}\\
\gll Había los tapones de gasoil en el  suelo. \\
there.were \textsc{art.pl} caps	of	diesel on \textsc{art} floor \\
\trans ‘There were the caps of diesel on the floor.’ \\

\ex \label{Agullo16b}
\langinfo {Spanish}{La Serra (Torre de Claramunt) [Barcelona]}{COSER-0803\_01}\\
\gll Hay los balcones abiertos,	 yo. \\
there.are \textsc{art} balconies opened me. \\
\trans ‘There are my balconies opened.’ (Lit. ‘There are the balconies opened, me’.)	 \\

\ex \label{Agullo16c}
\langinfo {Spanish}{San Climent (Mao) [Menorca]}{COSER-5003\_01}\\
\gll Había la iglesia, la	iglesia	 abierta. \\
there.was \textsc{art} church, \textsc{art}	church opened \\
\trans ‘There was the church opened.’  \\

\ex \label{Agullo16d}
\langinfo {Spanish}{San Climent (Mao) [Menorca]}{COSER-5003\_01}\\
\gll Había la cocina encendida. \\
there.was \textsc{art} kitchen	on \\
\trans ‘There was the kitchen turned on.’	  \\

\z
\z

The data in (\ref{Agullo14})--(\ref{Agullo16}) directly support the hypothesis formulated in (\ref{Agullo13}): the definiteness effect seems to weaken whenever Spanish and Catalan come into contact. Let us sharpen the issues here. As the sequences in (\ref{Agullo14})--(\ref{Agullo16}) show, the restriction on the definiteness of the pivot position of the existential is lifted in varieties of Spanish in contact with Catalan. The existential construction, hence, seems to no longer observe the restriction.

This immunity to the definiteness restriction is, as I will demonstrate, far from hazardous or trivial. On the contrary, it can easily be made to follow from a principled explanation based on language contact. Evidence in support of this hypothesis comes from the cartographical representation of existentials as those in (\ref{Agullo14})--(\ref{Agullo16}), as gathered from the main corpus, which are depicted in Map \ref{AgulloFigure1}.

\begin{figure}[ht]
\includegraphics[width=\textwidth]{figures/Agullo_Map.png}
\caption{Geographical distribution of \textit{haber} with definite and specific pivots (created with \cite{QGIS2024})}
\label{AgulloFigure1}
\end{figure}

Existential constructions with definite, specific pivots that show immunity to the definiteness effect are distributed with a clear geographical coherence: they are located chiefly on the coastal side of the Iberian Peninsula and the Balearic Islands. What is even more striking: villages where examples like (\ref{Agullo14})--(\ref{Agullo16}) are found bear close contact with and are heavily influenced by Catalan.

Special care, nevertheless, should be exercised when handling dialectal data in contact scenarios — even more when attributing results to language contact, as it could well be the case that they owe to different reasons. 

Carlota de Benito (p. c.) was the first to draw my attention to one of the major -- and plausibly not few -- shortcomings of the hypothesis in (\ref{Agullo13}): the hypothesis that the definiteness effects is superseded because of language contact falls short to predict that the effect can also be amnestied outside contact varieties. Let us just stress, however, that the hypothesis in (\ref{Agullo13}) is a speculation about the origin of this case of linguistic variation; by no means is it thought to be an exhaustive principle. Differently put, the claim in (\ref{Agullo13}) is a generalisation about the principles that account for the immunity to the definiteness effect, but it does not exhaust all plausible explanations.

The results in Map \ref{AgulloFigure1} are grouped consistently, some exceptions aside (i.e., the values in Galicia and Navarra), but they cannot be studied on independent grounds. On the contrary, demolinguistic data should be brought to the fore. Rural population shows a steady tendency to decrease as a consequence of urbanisation: the urban drift of the young population usually strands the elder, who remain in town (e.g., \cite{Aldoma2009}). The elderly population, usually with low or null instructional level, are held to use Catalan more frequently — a fact that seemingly holds only in rural areas. As a consequence, elder sections of the population in rural communities are a fairly illustrative example of the effects of language contact between Spanish and Catalan.

The distribution of grammatical categories across Aissen’s (\citeyear{Aissen2003}) Definiteness Hierarchy, as formulated in (\ref{Agullo1}), can offer a fairly intuitive account of the facts in (\ref{Agullo14})--(\ref{Agullo16}). The definiteness effect in Spanish bars personal pronouns, proper names, and definite nouns from the pivot position, but, whenever in contact with Catalan, the pivot can be a definite noun or even a proper noun. Spanish in contact with Catalan can thus be safely argued to enrich the categories allowed in the pivot position of the existential construction. A question arises, nevertheless, as to which is the \textsc{locus} of this phenomenon of syntactic variation. Is it the verbal predicate, which does not observe the restriction any longer, or is it the pivot of the existential, which, in spite of being definite, is recategorised as indefinite? The question shall also be addressed as to what the general principles of language are that account for the obsolescence of the definiteness effect or which specific mechanisms of language contact are at stake in these cases.

If the explanation of the facts in (\ref{Agullo14})--(\ref{Agullo16}) is thought to be dependent upon more general principles of grammar, a line, I believe, worth pursuing, these questions (and several others) have to be tackled. This chapter is, in fact, an attempt to provide a sound answer to some of them.

\subsection{Syntactic \textsc{transfer} in language contact: some insights} \label{SyntacticTransfer}

In what follows, I will shift gears to demonstrate that it is in close relation to linguistic contact that the obsolescence of the definiteness effect is to be accounted for. Specifically, I will put forth the hypothesis that the statement in (\ref{Agullo13}) can be straightforwardly predicted if we assume that language contact can outmanoeuvre or supersede a grammatical restriction. If a new syntactic variant emerges that does not observe the effect, the corresponding variety will, consequently, allow for definite and specific pivots in the existential.

The explanation I will develop here is underlaid by the assumption that syntax can, indeed, be borrowed or transferred. The hypothesis is by no means trivial, as it hinges on the more general concept of language contact: what can be borrowed and what cannot be borrowed in contact scenarios? There seems to be wide disagreement over the borrowability or transferability of syntax. Authors such as \citet{ThomasonKaufman1988}, \citet{Campbell1993}, \citet{Bowern2008} and \citet{Thomason2014}, among many others, have argued that syntactic borrowing is possible, whereas \citet{Prince1988} and \citet{Givon1979} have argued that syntactic borrowing is almost impossible or have even rejected the hypothesis that syntax can be transferred at all. It seems clear from variationist approaches to language that syntax and morphology are “the domains of linguistic structure least susceptible to the influence of contact” (\cite[513]{Sankoff2013}), contrary to the lexicon or the phonemic inventory, which can be readily borrowed.

The theoretical controversy, nevertheless, is misplaced. If lexical items can be borrowed and so can phonological rules, there seems to remain no intuitive appeal in the idea that syntactic categories or morphological constituents cannot be borrowed. Purely syntactic categories have, in fact, proven accessible to variation and change (see \cite{ThomasonKaufman1988}, \cite{Fischer1992}, \cite{Corrigan2009}, \cite{Thomason2014}). The same holds true for morphological constituents, even though it seems that independent morphemes are more easily borrowed than bound morphemes (cf. \cite{Thomason2014}). The question should be raised as to whether all or some cases of syntactic variation boil down to lexical features, lexical alternations, or other areas of grammar, but these issues do not compromise the idea that syntax can be borrowed or transferred. It remains to be ascertained, accordingly, if those apparent instances of syntactic variation are truly syntactic or morphological in nature or are, on the contrary, due to some features within the lexicon or some lexical alternations. 

The observation that the definiteness effect is overridden in contact with Catalan, in fact, is to be regarded as fairly robust evidence in support of the hypothesis that syntax can be borrowed, whether it be as a result of lexical features or not. Grammatical restrictions bear upon abstract principles of the architecture of language and undoubtedly involve syntax, although it is frequently the case that their clustering effects pervade other areas of grammar, such as semantics or the lexicon. Whenever syntax has been thought of as accessible to variation and change, it is usually the transfer or borrowing of certain grammatical items or more abstract syntactic features that is invoked. 

Similar mechanisms seem to be at play in varieties of Spanish in contact with Catalan, even though I believe that it is not the grammatical restriction that is being transferred or borrowed, as though it were a lexical or grammatical item (i.e., so-called superficial syntactic borrowing in \cite{Bowern2008}). Purely syntactic transfer may involve more or less abstract or complex categories (e.g., relative pronouns, as shown by \cite{Corrigan2009}), but grammatical restrictions do not seem likely to be borrowed or transferred. Moreover, if their consequences are made to follow from more general principles, as the cross-linguistically tenable primitive that defines existentials (see \cite{Haspelmath2010} or \cite{Newmeyer2010}), I believe there remains no doubt that grammatical restrictions cannot be transferred. 

The theoretical solution to this problem lies, ultimately, in redefining the locus of grammatical variation. If we assume that the definiteness effect cannot be transferred from one variety to another, what is it, then, that accounts for the data in (\ref{Agullo14})--(\ref{Agullo16})? A plausible hypothesis is that the immunity to the definiteness effect, i.e., the fact that Catalan does not observe the restriction, can pervade new dialectal varieties of Spanish in contact with Catalan. Under this purview, then, what is being transferred is not the grammatical restriction per se, but the inobservance of that restriction. By virtue of this immunity to the restriction, Spanish in contact with Catalan goes a step further with regard to Aissen’s (\citeyear{Aissen2003}) definiteness scale, as it now allows for definite nouns, or even proper names in the pivot position.

\subsection{New \textit{interdialect} or \textit{hybrid} variants} \label{NewInterdialect}
The question could be raised as to which are the precise mechanisms of language contact that predict the hypothesis formulated in (\ref{Agullo13}).

Theories on linguistic variation and change, particularly since pioneering research by \citet{WeinreichEtAl1968} and \citet{Labov1964,Labov1972,Labov1978}, have furthered our knowledge of the structural and social consequences of linguistic contact. It is generally agreed upon that, whenever grammar is subject to change, the process “entails not merely formal differences but functional differences as well” (\cite[314]{Harris1984}). Contact between different languages and their dialectal varieties is generally acknowledged to lay the foundations for the emergence of (a) linguistic variants hitherto non-existent or unnoticed (i.e. newly created variants), (b) new dialectal varieties, and (c) the obsolescence of other variants (e.g., the loss of gender agreement in adjective predicates in Romansch under influence of German; see \cite{WeinreichEtAl1968}) (see, among many others, \cite{Siegel1985}, \cite{Kerswill2013}, \cite{CerrutiTsiplakou2020}). 

Alternatively, already in-use variants may be reallocated, i.e., recycled or used differently. New linguistic variants -- so-called \textsc{interdialect variants} in \citet{Trudgill1989a,Trudgill1989b,Trudgill1992}, \citet{Britain2005} and \citet{Almeida2020}, among many others -- can, thus, arise when two languages come into contact with each other. Interdialect variants can be thought of as “novel features not found in any of the established contributing dialects” (\cite[325]{Tuten2001}). Previously existing features may be rearranged and, as a result, new linguistic variants may emerge. Interdialectalisms, which are, essentially, a special kind of dialect mixing, result when “speaker-learners reanalyse or rearrange forms and features of the contributing dialects” (\cite[187]{Tuten2006}). Interdialect forms stem, thus, from a mixture, rearrangement, or readaptation of different features or constructions belonging to two or more dialects. 

Data in (\ref{Agullo14})--(\ref{Agullo16}) above, which have shown the obsolescence of the definiteness effect in contact situations between Spanish and Catalan, prove easily amenable to the hypothesis that language contact (and, by extension, dialect contact) may rearrange or alter features in a given linguistic variety. The hypothesis put forth here is that the obsolescence of the definiteness effect is a consequence of interdialect forms. In fact, newly emerged linguistic variants, as those in (\ref{Agullo14})--(\ref{Agullo16}), render the definiteness effect obsolete. I believe the syntactic variants in (\ref{Agullo14})--(\ref{Agullo16}) can be held to be interdialect variants to the extent that definite, specific noun phrases are found in the pivot of the existential in Spanish due to contact with Catalan. The novel variant is, in fact, the use of this sort of pivot, which parallels the Catalan usage: existentials in Catalan allow for definite, specific pivots, as those in (\ref{Agullo9}, \ref{Agullo10}), or even personal pronouns, as in (\ref{Agullo11}, \ref{Agullo12}).

\section{Conclusions and prospects} \label{ConclusionsAndProspects}
Throughout this chapter, the claims in (\ref{Agullo17}) have borne close inspection:

\ea \label{Agullo17}
\ea \label{Agullo17a}
The definiteness effect is amnestied in varieties of Spanish in contact with Catalan. \\

\ex \label{Agullo17b}
As a consequence, existential sentences allow for definite, specific pivots. \\
\z
\z


What the claims in (\ref{Agullo17}) state is, differently put, that a grammatical restriction can be amnestied in contact situations. The pivot of the existential in Catalan has been shown to go one — plausibly two — steps further along Aissen’s (\citeyear{Aissen2003}) Definiteness Scale in (\ref{Agullo1}): definite, specific pivots are grammatical as pivots. These facts have been argued to derive from a more general principle: Spanish shows a strong version of the definiteness effect (e.g.,  \cite{LongaEtAl1996,LongaEtAl1998}, \cite{Leonetti2008}), whereas Catalan observes a weak version of it (see, particularly, \cite{Brucart2006}, \cite{BrucartEtAl2006}, \cite{Villalba2013}). As a consequence, varieties of Spanish in contact with Catalan allow for definite, specific nouns in the pivot position of the existential.

The claims in (\ref{Agullo17}) do not exhaust all theoretical possibilities, nor do they account for all the relevant dimensions of the syntactic variation at stake. Particularly, the question shall be raised as to whether this instance of variation is neatly syntactic in nature or owed to some alternation within the lexicon. Spanish existential constructions are widely held to alternate between \textit{haber} ‘to be’ and \textit{estar} ‘to be’, but the alternation is far from free or casual: in general Spanish, definite and specific pivots are generally introduced with \textit{estar} ‘to be’. If a decrease in frequency of presentational \textit{estar} ‘to be’ was found in dialects of Spanish in contact with Catalan, claims in (\ref{Agullo17}) could be easily argued to boil down to a lexical alternation. This and many other questions await further research.

\section{Acknowledgments} \label{Acknowledgments}
Financial support was received from the research project “El Corpus Oral y Sonoro del Español Rural (COSER): edición digital y análisis lingüístico” (PID2022-138497NB-I00) (PI: Inés Fernández-Ordóñez), funded by the Spanish Ministerio de Ciencia e Innovación (MCIN/AEI/10.13039/501100011033) and the FSE+. Preliminary versions of this chapter were presented as a talk, “Definiteness Effects and Linguistic Variation: Spanish in contact with Catalan”, delivered at the Annual Meeting of the Societas Linguistica Europaea (Athens, 29 August – 1 September 2023). Thanks are due to the members of the audience and particularly Florencio del Barrio and Susanne Michaelis for pointing out to me some possible lines of inquiry. I wish to thank Josefina Budzisch, Chris Lasse Däbritz, and Rodolfo Basile for organizing the workshop on “Locative and existential predication - Core and periphery” and for their patience with my constant delays.

\sloppy
\printbibliography[heading=subbibliography,notkeyword=this]
\end{document}
