\documentclass[output=paper,colorlinks,citecolor=brown]{langscibook}
\ChapterDOI{10.5281/zenodo.16838074}

\author{Erin C. SanGregory\affiliation{University of Oregon \& SIL International}}

\title[Locative, existential and possessive clauses in Wakhi]{An information structure analysis of locative, existential, and possessive clauses in Wakhi}
\abstract{Wakhi, a lesser-described language of northeastern Afghanistan, uses nonverbal constructions for a number of different types of clauses, including locative, existential, and possessive clauses. These three types of clauses share syntactic similarities, but they also exhibit puzzling syntactic variations. In this paper, I demonstrate that a formal structural approach is insufficient to account for these variations. Instead, I argue that the data can only be explained in terms of pragmatics. I propose two information structure constructions that model the syntactic variations exhibited by locative, existential, and possessive clauses in Wakhi. With this analysis, I affirm the significance of the relationship between information structure and syntax and provide a foundation for future research into how these two domains interact in other areas of Wakhi syntax.}

\IfFileExists{../localcommands.tex}{
   \addbibresource{../localbibliography.bib}
   % add all extra packages you need to load to this file

\usepackage{tabularx,multicol}
\usepackage{url}
\urlstyle{same}

\usepackage{listings}
\lstset{basicstyle=\ttfamily,tabsize=2,breaklines=true}

\usepackage{langsci-basic}
\usepackage{langsci-optional}
\usepackage{langsci-lgr}
\usepackage{langsci-osl}
% \usepackage{./langsci/styles/langsci-lgr}
% \usepackage{./langsci/styles/langsci-osl}
% \usepackage{langsci-gb4e}

\usepackage{tikz}
\usetikzlibrary{patterns,calc}
\pgfdeclarepatternformonly{south east lines}{\pgfqpoint{-0pt}{-0pt}}{\pgfqpoint{3pt}{3pt}}{\pgfqpoint{3pt}{3pt}}{
    \pgfsetlinewidth{0.6pt}
    \pgfpathmoveto{\pgfqpoint{0pt}{3pt}}
    \pgfpathlineto{\pgfqpoint{3pt}{0pt}}
    \pgfpathmoveto{\pgfqpoint{.2pt}{-.2pt}}
    \pgfpathlineto{\pgfqpoint{-.2pt}{.2pt}}
    \pgfpathmoveto{\pgfqpoint{3.2pt}{2.8pt}}
    \pgfpathlineto{\pgfqpoint{2.8pt}{3.2pt}}
    \pgfusepath{stroke}}
    
\usepackage{stmaryrd}
\usepackage{wasysym}
\usepackage{multirow}
\usepackage{caption}
\usepackage{subcaption}
\usepackage{mathrsfs}
\usepackage{qtree}

\usepackage{linguex}


   %pminos do not split footnotes
% \interfootnotelinepenalty=10000 %Footnote in Laporte chapters has to be split SN


%\DeclareIndexNameFormat{default}{%
%\nameparts{#1}%
%\usebibmacro{index:name}%
%{\index[names]}%
%{\namepartfamily}%
%{\namepartgiveni}%
% {}% L1
% {}% L2
%{\namepartprefix}% generates spurious space L3
%{\namepartsuffix}% generates spurious space L4
%}

%  {\DeclareIndexNameFormat{default}{%
%     \usebibmacro{index:name}{\index[names]}{#1}{#3}{#5}{#7}}}

%\DeclareIndexNameFormat{default}{%
%  \usebibmacro{index:name}{\sindex[nom]}{#1}{#3}{#5}{#7}}

%\DeclareIndexNameFormat{default}{%
%  \usebibmacro{index:name}{\sindex[person]}{#1}{#3}{#5}{#7}}
%\DeclareIndexNameFormat{default}{%
%\nameparts{#1} \usebibmacro{index:name}{\sindex[person]]}{\namepartfamily}{‌​\namepartgiven}{\nam‌​epartprefix}{\namepa‌​rtsuffix}}

%\newcommand{\smiley}{:)}

%\renewbibmacro*{index:name}[5]{%
%\usebibmacro{index:entry}{#1}%
%{\iffieldundef{usera}{}{\thefield{usera}\actualoperator}\mkbibindexname{#2}{#3}{#4}{#5}}}

% \newcommand{\noop}[1]{}

%remove for final
%\overfullrule=1mm

\newcommand{\tobi}[2]}}
\renewcommand{\S}[1]{\tobi{#1}{\textsc{*}}}

% this volume references
% puts: [this volume]
% already defined: \citetv
%\newcommand{\citepv}[1]{(\citeauthor{#1} \citeyear*{#1} [this volume])}
\newcommand{\citealtv}[1]{\citeauthor{#1} \citeyear*{#1} [this volume]}

%parentheses around example number
\newcommand{\pref}[1]{(\ref{#1})}

% in-text examples

\newcommand{\lnex}[1]{\textit{#1}} %target lang word
\newcommand{\lnlit}[1]{(lit.: `#1')} %literal reading
\newcommand{\lnlat}[1]{(#1)} % latinization
\newcommand{\lntrans}[1]{`#1'} %translation
\newcommand{\lnexl}[2]%
{\lnex{#1}{} \lnlat{#2}} % ex with latinization
\newcommand{\lnexlat}[3]{\lnex{#1}{} \lnlat{#2}{} \lntrans{#3}} % ex with latinization and tranl.

%ch01
\newcommand{\co}[1]{\mbox{\textbf{#1}}}

%ch09

\newcommand{\cyrbulg}[1]{\begin{otherlanguage*}{bulgarian}#1\end{otherlanguage*}}


%ch10
\newcommand{\nlp}{{\small NLP}}
\newcommand{\mwe}{{\small MWE}}
\newcommand{\rae}{{\small RAE}}
\newcommand{\lvc}{{\small LVC}}
\newcommand{\pos}{{\small P}o{\small S}}
%\newcommand{\todo}[1]{ \textcolor{red}{#1} }

%\renewcommand{\labelenumi}{\theenumi}
%\ainamefmt{{vv}{ll}{, ff}{, jj}} % fullname

\newcommand{\biberror}[1]{{\color{red}#1}}

\newcommand{\osenovaitem}{--~}
   %% hyphenation points for line breaks
%% Normally, automatic hyphenation in LaTeX is very good
%% If a word is mis-hyphenated, add it to this file
%%
%% add information to TeX file before \begin{document} with:
%% %% hyphenation points for line breaks
%% Normally, automatic hyphenation in LaTeX is very good
%% If a word is mis-hyphenated, add it to this file
%%
%% add information to TeX file before \begin{document} with:
%% %% hyphenation points for line breaks
%% Normally, automatic hyphenation in LaTeX is very good
%% If a word is mis-hyphenated, add it to this file
%%
%% add information to TeX file before \begin{document} with:
%% \include{localhyphenation}
\hyphenation{
    Beck-man
    Ngu-yen
    back-chan-nel
    back-chan-nels
    mo-not-o-nous
    ste-reo-typ-i-cal
}

\hyphenation{
    Beck-man
    Ngu-yen
    back-chan-nel
    back-chan-nels
    mo-not-o-nous
    ste-reo-typ-i-cal
}

\hyphenation{
    Beck-man
    Ngu-yen
    back-chan-nel
    back-chan-nels
    mo-not-o-nous
    ste-reo-typ-i-cal
}

   \boolfalse{bookcompile}
   \togglepaper[10]%%chapternumber
}{}

% \pretocmd{\gll}{\def\eachwordone{\itshape}\def\eachwordtwo{\normalfont}}{}{}

\begin{document}
\maketitle 
%\shorttitlerunninghead{}%%use this for an abridged title in the page headers
\section{Introduction} \label{Section1}

Word order is a foundational tool in the belt of many linguists. Thanks to the work of \citet{Greenberg1966} and \citet{Dryer1997,Dryer2013}, our most broad typological categories are based on word order. Word order is also commonly used to distinguish different types of clauses (cf. \citealt{Clark1978}). But sometimes word order fails us – sometimes a language thought to have rigid word order exhibits surprising variations, or two supposedly distinct types of clauses share the same word order, or what seems to be a single clause type uses multiple word orders. One language in which this is the case is Wakhi (Southeastern Iranian, Indo-European, [wakh1245]), a lesser-described language of Central Asia.

In Wakhi, locative, existential, and possessive clauses are nonverbal clause constructions. Word order as a diagnostic tool is unreliable at distinguishing these clauses from each other, due to both variations in word order within constructions and overlaps in syntactic structure across different constructions. For example, sentences (\ref{SanGregory1}) and (\ref{SanGregory2}) are nearly identical except for the appositive at the end of (\ref{SanGregory1}). Both sentences contain an indefinite noun phrase and a definite locative phrase; the only difference is that the noun phrase occurs clause-initially in (\ref{SanGregory1}), but in (\ref{SanGregory2}) the locative phrase is initial. Are these locative or existential clauses? Or is each a different type of clause? Based simply on word order, it is impossible to say.

\ea 
\label{SanGregory1}
[25.200] \\
\gll Ji  t͡ʃiz  ɑ=drəm  tu  ruxn.\\
     one  thing  \textsc{emph}=\textsc{loc.prox}  \textsc{cop.pst}  white\\
\glt `A thing was here, a white thing.' or: \\ 
`There was something here, something white.' 
\z 


\ea 
\label{SanGregory2}
[09.09] \\
\gll Dɑm  kɑf  ji  dʊstkɑwl  tu.\\
     \textsc{loc.prox}  palm  one  purse  \textsc{cop.pst}\\
\glt `In her hand there was a purse.' or:\\
`A purse was in her hand.' 
\z 

Locatives and existentials are not the only clauses in Wakhi that are difficult to classify. Examples (\ref{SanGregory3}) and (\ref{SanGregory4}) are both possessive clauses, but in (\ref{SanGregory3}) the possessor precedes the possessee, while in (\ref{SanGregory4}) the possessor follows the possessee.

 
\ea
\label{SanGregory3}
[23.02] \\
\gll ʐʊ-nən  ji  t͡ʃiz  tʊi.\\
\textsc{1sg}{}-\textsc{poss}  one  thing  \textsc{cop}\\
\glt `I have a [certain] thing.' 
\z 

\ea
\label{SanGregory4}
[05.55] \\
\gll Zur  ʐʊ-n  tʊi.\\
strength  \textsc{1sg}{}-\textsc{poss}  \textsc{cop}\\
\glt `I have strength.' 
\z 

When confronted with perplexing cases such as these, we can write phrase structure rules and draw syntax trees, but such devices merely allow us to describe the data in different ways. If we want to truly understand the reasons for variations in word order, we must determine how such variations correlate to different meanings or communicative functions \citep{Payne1999,Croft2013}.

In this paper, I adopt a usage-based perspective in analyzing how speakers of Wakhi use different syntactic structures to communicate effectively. In particular, I investigate the relationship between information structure and word order. A number of people have suggested that information structure can affect word order (\citealt{DikEtAl1981}; \citealt{Mithun1987}; \citealt{Payne1993}; \citealt{DooleyLevinsohn2023}). I argue that information structure and word order are closely connected in Wakhi: specifically, I show that topic, focus, and pragmatic function (i.e., whether a clause is predicational or presentational) are all crucial factors that work together to determine word order in locative, existential, and possessive clauses.

This paper is structured as follows. In \sectref{Section2}, I review background information related to nonverbal clause constructions and word order, provide a brief overview of Wakhi and the parameters of the current study, and present the problem of structural variation in Wakhi. In \sectref{Section3}, I introduce the various syntactic structures of locative, existential, and possessive clauses. In \sectref{Section4}, I discuss several effects of information structure on syntax as documented in the literature. I then show how different syntactic patterns correlate with information structure and pragmatic function, and I propose two information structure constructions that allow for a unified analysis of the various syntactic structures found in locative, existential, and possessive clauses. Finally, I offer a summary and some concluding remarks in \sectref{Section5}.

\section{\label{Section2}Background}
\subsection{\label{Section2.1}Three types of nonverbal clauses}

In this section, I discuss terminology and background information to provide a theoretical foundation for the rest of the paper, beginning with \textsc{nonverbal} \textsc{clauses}. Nonverbal clauses are clauses that contain semantic predicates that are not verbs. These clauses predicate ``concepts other than action concepts'' \citep[289]{Croft2022}, such as identity, classification, temporary and permanent properties, location, existence, and possession. The semantic predicate of a nonverbal clause is typically either nominal, adjectival, or adverbial (\citealt{Hengeveld1992}; \citealt{Stassen1997}; \citealt{OverallEtAl2018}; \citealt{HaspelmathNonverbal}).

Depending on the language, a nonverbal clause may or may not require some kind of semantically vacuous/structurally defective verb or copula to accompany the semantic predicate and contribute grammatical information (\citealt{Hengeveld1992}; \citealt{OverallEtAl2018}; \citealt{Croft2022}; \citealt{HaspelmathNonverbal}).\footnote{By ``structurally defective,'' I mean a verb or copula that is an invariant particle or that takes minimal, if any, morphological inflection compared to the possibilities available for regular verbs.} \citet[289--290]{Croft2022} calls such constructions ``nonprototypical predicates'' since a language may encode some or all non-action predicates via a verbal strategy. \citet[2]{HaspelmathNonverbal} contends (1) that not all constructions of this type involve predication, and (2) that the verbs used in such constructions are not prototypical verbs; therefore, he proposes ``nonverbal clause construction'' as a better term. I follow Haspelmath and use the term \textsc{nonverbal} \textsc{clause} \textsc{construction} throughout this paper to refer to these types of constructions. I use the term \textsc{construction} to refer specifically to a unique combination of form and meaning (cf. \citealt{Goldberg1995}; \citealt{Croft2001}).

My focus in this paper is on three types of nonverbal clause constructions: locative, existential, and possessive clauses. Similarities between these three types of nonverbal clause constructions have been recognized for some time (\citealt{Lyons1967}; \citealt{Clark1978}), but it was not until more recently that work in historical linguistics found evidence of diachronic relationships between these constructions (\citealt{Heine1997Cognitive}; \citealt{Heine2001}; \citealt{Stassen2009}; \citealt{OverallEtAl2018}; \citealt{Croft2022}). In the remainder of this section, I give a brief description of each of these three types of nonverbal clause constructions, beginning with locative clauses.

\textsc{Locative} \textsc{clauses} are one of two types of ``localizing predicates'' according to \citet[94]{Hengeveld1992}, the other type being existential clauses. A locative clause has an adverbial locative phrase as its semantic predicate, the primary function of which is to situate an object at a concrete physical location (\citealt{Hengeveld1992}; \citealt{Stassen1997}; \citealt{OverallEtAl2018}; \citealt{HaspelmathNonverbal}). For example, \textit{the book is on the table} is a locative clause because it functions to assert the location of a presupposed book. In addition to the locative adverbial predicate, some languages require a dedicated locative copula or verb(s) (often posture verbs) as part of a locative clause construction; others use a generic copula or a zero-copula strategy (\citealt{Stassen1997}; \citealt{OverallEtAl2018}; \citealt{Croft2022}). Historical and comparative studies of nonverbal predicates show that locative clauses are a common historical source for both existential and possessive clauses (\citealt{OverallEtAl2018}; \citealt{Croft2022}).\footnote{\citet{ChappellLü2022} propose that existentials are a requisite intermediate stage between locatives and possessives; \citet{OverallEtAl2018} recognize this as one possible pathway of change but show that a direct link from locatives to possessives is also attested.}

\textsc{Existential} \textsc{clauses} have classically been defined as clauses that assert the existence of some object or entity \citep{Croft2022}. However, a growing number of linguists assert that the primary function of existential clauses is actually to introduce a new object or participant to the discourse, and that the predication of existence is a secondary function (\citealt{Hengeveld1992}; \citealt{Givon2001}; \citealt{Gaeta2013}; \citealt{OverallEtAl2018}).

It is not uncommon for an existential clause to include a locative adverbial. An example of such a clause can be seen in the English sentence, \textit{There is a book on the table}. The presence of a locative phrase in an existential clause is seen as a reflection of the natural state of affairs in the physical world: an assertion that an entity exists entails that the entity exists in a particular location, even if that location is not stated explicitly (\citealt{Lyons1967}; \citealt{Clark1978}; \citealt{Koch2012}; \citealt{Gaeta2013}; \citealt{HaspelmathNonverbal}). The definiteness of the subject noun phrase is frequently used to distinguish existential clauses from locative clauses – locatives commonly have a definite subject, but existentials an indefinite one (\citealt{Clark1978}; \citealt{Gaeta2013}; \citealt{OverallEtAl2018}; \citealt{HaspelmathNonverbal}; \citealt{PanovaLiljegren2025}). This correlation between the type of clause and the definiteness of the subject is common but not universal; exceptions can be found even within an individual language \citep{Creissels2019}.

Two terms commonly used in discussions of locative and existential clauses are \textsc{figure} and \textsc{ground}. The figure is the referent that is ``conceptually moveable'' or salient (\cite[419]{Talmy1975}; \cite[312]{Talmy2000}; \cite{Creissels2019}). The figure of a locative clause is the entity whose location is asserted, while the figure of an existential clause is the referent being introduced \citep{Croft2022}. The ground, on the other hand, is the entity used as a reference point for locating the figure in space (\citealt{Talmy1975}; \citealt{Talmy2000}; \citealt{Danziger2010}; \citealt{LevinsonMeira2003}; \citealt{Croft2022}). It logically follows that the ground of a locative or existential clause is the locative phrase. In this paper, I follow the convention of using the terms \textsc{figure} and \textsc{ground} to label the primary constituents of locative and existential clauses.

The concept of possession is commonly understood to be a universal of human cognition (\citealt{Heine2001}; \citealt{Stassen2009}). \textsc{Possessive} \textsc{clauses} are clauses that predicate possession and, in so doing, express some degree of control exercised by a possessor over a possessee (\citealt{Heine1997}; \citealt{HerslundBaron2001}; \citealt{Dixon2010a}; \citealt{DanesiBarðdal2018}; \citealt{Croft2022}). An instance of predicative possession must therefore include a possessor, a possessee, and some way of defining the relationship between them. However, neither the semantic types of relationships that languages encode between possessor and possessee nor the the ways that they encode those relationships are universal. For example, typological studies have identified a number of different types of possession as well as numerous strategies for encoding possession. These strategies include verbal and nonverbal clause constructions, as well as differences in the morphological marking of and syntactic properties associated with possessor and possessee (\citealt{Clark1978}; \citealt{Heine1997Cognitive}; \citealt{Heine1997}; \citealt{Stassen1997}; \citealt{Stassen2009}; \citealt{Dixon2010a}; \citealt{OverallEtAl2018}; \citealt{Croft2022}). In this paper, I am concerned with nonverbal possessive clause constructions.

In this section, I have defined locatives as clauses that assert the physical location of an entity, and I have defined possessives as clauses that express a possessor's control over a possessee. Both of these functions are classified as \textsc{predicational} because they are characteristic of clauses in which the predicate asserts something about the subject (\citealt{Stassen1997}; \citealt{Croft2022}). The predicational functions of locative and possessive clauses are their primary or most prototypical functions, but not their only functions. Like existential clauses, both locative and possessive clauses may also be used to introduce a new referent into the discourse. Clauses with this type of introductory function are classified as \textsc{presentational} because they ``present'' a new referent to the listener (\citealt{Lambrecht1994}; \citealt{Stassen1997}; \citealt{Croft2022}). All existential clauses are presentational; locative and possessive clauses may be either predicational or presentational. As \citet{Hengeveld1992} notes, the overlap in function between existential clauses and presentational locative clauses often correlates with similarities in form. I return to the topic of predicational vs. presentational clausal functions in \sectref{Section4}.

\subsection{Word order} \label{Section2.2}

The fascination with determining the word order of a language is not new. Starting at least as far back as \citet{Behaghel1909}, various linguists have investigated the potential significance of word order. For example, \citet{Greenberg1966} posits the six-way typology of word order still in wide use today. \citet{Dryer1997,Dryer2013} advocates an alternate typology of word order based on the relative order of S vs. V, V vs. O, and the interactions between those two pairs of constituents. \citet{Comrie1989}, \citet{Tomlin2016}, and \citet{Dixon2010b}, among others, point out that what Greenberg and Dryer call ``word order'' is in fact ``constituent order'' since it refers to the order of major sentential constituents, not the order of individual words. In this paper, I follow the terminological tradition of using ``word order'' to refer to constituent order.

Although word order can be a useful area of study, it is neither as straightforward nor as productive as some make it out to be. For example, \citet{Dryer1997} observes that Greenberg's \citeyearpar{Greenberg1966} use of both S and O in his typology has led many linguists to neglect the word order of intransitive clauses in their research. The exclusion of intransitive clauses is problematic since a language may use one word order for certain constructions or types of clauses, but a different word order for others \citep{Comrie1989}. Some languages do not make a clear distinction between the categories of subject and object (see, for example, La Polla's \citeyearpar{LaPolla1993} analysis of Chinese); typologies based on those categories are irrelevant for such languages. Another difficulty is presented by languages that exhibit such a high degree of fluidity that it would be both impossible and inaccurate to select one order as more basic than any other (\citealt{Mithun1987}; \citealt{Comrie1989}; \citealt{Dixon2010b}; \citealt{Tomlin2016}). In regards to the topic of the present study, the different types of nonverbal clause constructions are sometimes assumed to be distinguishable on the basis of word order (cf. \citealt{Clark1978}). While it may sometimes be the case that each type of nonverbal clause is encoded via a dedicated syntactic construction, it seems to be much more common for at least some types to be ``structurally intertwined'' (\citealt[8]{OverallEtAl2018}) and therefore more difficult to identify in a particular language.

A number of linguists have investigated the correlation between word order and information structure notions such as topicality, focus, identifiability, and accessibility (\citealt{Chafe1987}; \citealt{Payne1987}; \citealt{Lambrecht1994}; \citealt{Dryer1997}; \citealt{DooleyLevinsohn2023}). For example, \citet{Mithun1987} demonstrates that pragmatics conditions word order in several languages that display an otherwise bewildering array of options for word order, and she suggests that many more languages would benefit from a pragmatic approach to analyzing word order. I address the intersection of information structure and syntax in \sectref{Section4}.

\subsection{\label{Section2.3}Overview: Wakhi and the current study}

Wakhi is a Southeastern Iranian language of the so-called Pamir group that is spoken by an estimated 52,000 speakers in Afghanistan, Pakistan, Tajikistan, and China.\footnote{Historical linguists debate whether the so-called Pamir languages comprise a family, a \textit{Sprachbund}, or a dialect chain (\citealt{Skjærvø1989}; \citealt{Wendtland2009}; \citealt{Novák2014}).} My focus is on the Afghan variety, which is spoken as a first language by approximately 17,000 people living in the Wakhan Corridor of Badakhshan Province, Afghanistan (\citealt{EberhardEtAl2023}).

In recent years, both the Pakistani and Tajik varieties of Wakhi have been studied and described to some degree (\citealt{Bashir1986}; \citealt{Bashir2009}; \citealt{Erschler2010}; \citealt{Hughes2011}; \citealt{Hughes2014}; \citealt{Fuchs2015}; \citealt{Obrtelová2017}). The Afghan variety has received considerably less attention, with only three publications devoted to it in the last decade (\citealt{Beck2013}; \citealt{SanGregory2015}; \citealt{SanGregory2018}). Much remains to be studied and described if we are to more fully understand Afghan Wakhi.

The data for this paper come from my own fieldwork on Afghan Wakhi. My current corpus comprises approximately 1.5 hours/10,000 words of naturalistic oral texts recorded in 2015--2016 and interlinearized in FieldWorks Language Explorer (FLEx) \citeyearpar{Flex}. In terms of composition, the corpus contains 6 dramatic roleplays (dialogues) and 18 narratives elicited using picture strip stories (\citealt{TakahashiFrauman-Prickel2015}) and flashcards \citep{Sardinha2011}, as well as one spontaneously narrated folktale.\footnote{These texts are legacy data; as such, the transcriptions and audio are not time-aligned. All speakers granted permission for their language data to be used in research on the condition that their identities and audio recordings be kept confidential.} The speakers were all young men. Examples from this corpus are cited by text and sentence number; for example, [23.02] refers to the second sentence of text 23.

From this corpus, I selected 106 locative, existential, and possessive clauses for analysis. I included clauses of both affirmative and negative polarity, as well as clauses of declarative, interrogative, and subjunctive moods. I also included clauses that express figurative location (e.g., accompaniment), but I excluded subordinate clauses and idiomatic expressions because these types of clauses can exhibit special syntactic patterns. I treated narrative clauses and speech clauses (i.e., dialogue, quoted and reported speech, and reported thoughts) as belonging to two separate genres; however, I do unify my analyses and descriptions when the features exhibited by both genres overlap.

The default word order of transitive clauses in Wakhi is SOV, and the default word order of intransitive clauses is SV, as seen in (\ref{SanGregory5}) and (\ref{SanGregory6}), respectively. Deviations from this word order are attested \citep{SanGregory2018}, but a discussion of atypical word order in verbal clauses is outside the scope of the current paper.

 
\ea
\label{SanGregory5}
[25.32] \\
\gll Sɑk  ti  gɑp-i  wʊdər-ən.\\
\textsc{1pl.nom}  \textsc{2sg.gen}  talk-\textsc{acc2}  hold-\textsc{2pl}\\
\glt `We heed your word.' 
\z 

\ea
\label{SanGregory6}
[25.114] \\
\gll Diu  wəz-di.\\
ogre  come-\textsc{pst}\\
\glt `The ogre came.' 
\z 

Wakhi uses nonverbal clause constructions to express predicates of identification, categorization, permanent and temporary properties, location, existence, and possession. Most of these clause types may be expressed either with or without a copula in the nonpast tense, but a copula is required to express negative polarity, subjunctive mood, or any tense other than the nonpast.

The remainder of this section provides a brief orientation to the Wakhi copula in order to facilitate better understanding of the many examples in this paper that include a copula. Wakhi has a single generic copula that is used in stative nonverbal clause constructions.\footnote{The verb \textit{wɔtsɑk} `to become' is used instead of the regular copula in inchoative and change-of-state nonverbal clause constructions \citep{SanGregory2022}.} The copula has distinct, invariant forms in the nonpast tense for both affirmative and negative polarity. The past tense form of the copula is also an invariant particle, but it takes the regular negative prefix \textit{nʊ-} that is used with verbs in Wakhi. The basic tense and polarity forms of the Wakhi copula are given in \tabref{Copula}.


\begin{table}
\caption{Basic tense and polarity forms of the Wakhi copula} \label{Copula}
\begin{tabular}{lll}
\lsptoprule
 & {affirmative} & {negative}\\
\midrule
{nonpast} & {\itshape tʊi} & {\itshape nɑst}\\
{past} & {\itshape tu} & {\itshape nʊ-tu}\\
\lspbottomrule
\end{tabular}
\end{table}

The copula also has a nonpast subjunctive stem.\footnote{The copula is unique in this regard. For all true verbs, and for the copula in the past tense, subjunctive mood is indicated by the preverbal subjunctive particle \textit{t͡sʊ}.} Although the copular forms given above never index the subject using regular verbal morphology,\footnote{Although the copula does not take the regular verbal subject-indexing suffixes, copular clauses can be marked with subject-indexing clitics \citep{SanGregory2015}.} the subjunctive stem is obligatorily inflected via the standard subject indexing suffixes used on nonpast verbs (cf. \citealt{SanGregory2015}). The subjunctive paradigm of the copula can be seen in \tabref{Subjunctive}.

\begin{table}
\caption{Subjunctive copular paradigm} \label{Subjunctive}
\begin{tabular}{lll}
\lsptoprule
& {singular} & {plural}\\
\midrule
{1} & {\itshape ʊmi-əm} & {\itshape ʊmi-ən}\\
{2} & {\itshape ʊmi} & {\itshape ʊmi-əv}\\
{3} & {\itshape ʊm-ʊt} & {\itshape ʊmi-ən}\\
\midrule
\end{tabular}
\end{table}

\subsection{The problem: structural variation in Wakhi}

In Afghan Wakhi, each of the three types of nonverbal clause constructions introduced in \sectref{Section2.1} – locative, existential, and possessive clauses – exhibits structural variations. Some syntactic structures can be found in more than one type of clause. These formal overlaps, especially between locative and existential clauses, can make them difficult to distinguish from one another, as I show in \sectref{Section1}. Therefore, I ask whether each type of clause has one structure that may be considered more ``basic'' or ``default'' than the other possible options at least in terms of frequency. I also seek to determine if Wakhi locative and existential clauses can be syntactically distinguished, or if they are merely different functions of a single construction. Finally, I investigate the synchronic and diachronic relationship between locative, existential, and possessive clauses in Wakhi.

I argue that we cannot understand the variety of syntactic patterns exhibited by locative, existential, and possessive clause constructions in Wakhi without considering the effect that information structure has on word order. Therefore, I conclude that there is no such thing as default word order in locative, existential, or possessive clauses in Wakhi. I propose that the constituent structure of each clause is determined by one of two separate information structure constructions; which construction is selected for a particular clause depends on the discourse function of that clause. Furthermore, I suggest that there is no formal distinction between locative and existential clauses in Wakhi (cf. the analysis of Gawarbati by \citealt{PanovaLiljegren2025}). Instead, the syntactic differences commonly observed between them are merely a result of the interaction between information structure and the type of information being asserted in a given clause.

In \sectref{Section2}, I have provided definitions and a theoretical context for my study of locative, existential, and possessive clauses. I have also introduced Afghan Wakhi, the corpus, and the scope of the study. Having provided this background, I now move on to the topic of structural variation in locative, existential, and possessive clauses in Wakhi.

\section{\label{Section3}Locative, existential, and possessive structures}
\subsection{\label{Section3.1}Locative clauses}

This section is devoted to describing locative, existential, and possessive clauses in Wakhi. In it, I present examples of these three types of nonverbal clause constructions and describe the different syntactic structures they exhibit. I also provide a schematic representation of each structural variation. These data and descriptions provide a foundation for the analysis that follows in \sectref{Section4}.

In Wakhi, locative adverbs (\ref{SanGregory7}) and locative prepositional phrases (\ref{SanGregory8}) may both serve as the semantic predicate/ground of a locative clause. As seen in both of these examples, a copula may or may not appear in locative clauses in the nonpast tense. This is true not just for declarative clauses, but also for interrogative ones, as shown in (\ref{SanGregory9}); a question word or particle suffices to indicate that the clause is a question. In contrast, a copula is required to indicate negative polarity (\ref{SanGregory10}) or the past tense (\ref{SanGregory11}).

 
\ea
\label{SanGregory7}
[25.178] \\
\gll Jəm  ɣɑl  drəm.\\
\textsc{3sg.nom}  still  \textsc{loc.prox}\\
\glt `He is still here.' 
\z 
 
\ea
\label{SanGregory8}
[25.500.02] \\
\gll Jəm-əʃ  tʊt  pɑlɯ.\\
\textsc{3.nom-pl}  up.to.\textsc{med}  side\\
\glt `They are up on that side.' 
\z 

\ea
\label{SanGregory9}
[03.13] \\
\gll Xɑi  ɑ=ʒɑrʒ  jɑn  niv  kumr?\\
\textsc{interj}  \textsc{emph}=milk  then  now  where\\
\glt `Well then, where is the milk now?' 
\z

\ea
\label{SanGregory10}
[03.06] \\
\gll Jɑ  ʒɑrʒ  drʊt  tɑ  xun  nɑst.\\
\textsc{dem.dist}  milk  \textsc{loc.med}  up.to  house  \textsc{cop}.\textsc{neg}\\
\glt `The milk isn't up there at the house.' 
\z 

\ea
\label{SanGregory11}
[11.11] \\
\gll Jɑn  ti  dʊstkɑwl  dʊ  ʐʊ  kɑf  tu.\\
and \textsc{2sg.gen}  purse  \textsc{loc}  \textsc{1sg.gen}  palm  \textsc{cop}.\textsc{pst}\\
\glt `And your purse was in my hand.' 
\z 

As seen in the preceding examples, the figure precedes the ground in the majority of locative clauses in Wakhi. These data also show that the figure of a locative clause is usually definite (cf. \sectref{Section2.1}), as is the ground. This structure can be represented as [\textsc{fig}\textsc{\textsubscript{def}} \textsc{grnd}\textsc{\textsubscript{def}} \textsc{(cop)}].

Example (\ref{SanGregory12}) is both similar to and yet somewhat different from the prototypical locative clauses seen in (\ref{SanGregory7}) – (\ref{SanGregory11}). This sentence exhibits the same figure-before-ground word order seen in the preceding examples, but its figure \textit{ji mʊʃqɔb} `a plate' is indefinite. Based on the syntactic similarity, I analyze (\ref{SanGregory12}) as a type of locative clause and represent its constituent structure as [\textsc{fig}\textsc{\textsubscript{indf}} \textsc{grnd}\textsc{\textsubscript{def}} \textsc{(cop)}]; I discuss the pragmatic differences between the different types of locative clauses in \sectref{Section4.2.2}.

 
\ea
\label{SanGregory12}
[08.02] \\
\gll Jɑn  ji  mʊʃqɔb  dʊ  ʐʊ  dʊ  ʐʊ  kɑf  tu.\\
then  one  plate  \textsc{loc}  \textsc{1sg.gen}  \textsc{loc}  \textsc{1sg.gen}  palm  \textsc{cop}.\textsc{pst}\\
\glt `Then a plate was in my, in my hand.' 
\z 

\subsection{\label{Section3.2}Existential clauses}

Recall from \sectref{Section2.1} that the primary function of an existential clause is to introduce a new entity to the discourse, whereas the primary function of a locative clause is to assert that a given entity is located in a particular place. In Wakhi, another difference between the two types of clauses is the position of the ground relative to the figure. In locative clauses, the ground follows the figure, but in existential clauses, the ground precedes the figure. I rely on both the difference in discourse function and the difference in constituent order to identify existential clauses in Wakhi.

In Wakhi, existential clauses can occur either with or without a locative ground. When a ground is included in a nonpast existential clause, it appears in clause-initial position. The copula may or may not be used in conjunction with a ground, as seen in (\ref{SanGregory13}) and (\ref{SanGregory14}), respectively. As is the case for locative clauses, existentials require a copula in negative polarity clauses and in tenses other than the nonpast. An example of a negative polarity existential clause is given in (\ref{SanGregory15}).

 
\ea
\label{SanGregory13}
[05.03] \\
\gll Trəm  ɣɑ  ʝuz.\\
over.to.\textsc{prox}  \textsc{ints}  firewood\\
\glt `Over here there's a lot of firewood.' 
\z 

\ea
\label{SanGregory14}
[25.500] \\
\gll Ji  pɑlɯ  bɔi  dəʝɑr  bʊ  tʊi.\\
one  side  cave  another  also  \textsc{cop}\\
\glt `On one side is another cave.' 
\z 

\ea
\label{SanGregory15}
[06.48] \\
\gll Drəm  ʊt͡ʃ  t͡ʃiz  nɑst.\\
\textsc{loc.prox}  none  thing  \textsc{neg}.\textsc{cop}\\
\glt `Here there is nothing.' 
\z 

Although the examples above show that a locative ground may be included in an existential clause, many existential clauses in Wakhi do not include a ground. When no ground is included, an existential clause must have a copula, regardless of tense, aspect, mood, or polarity, as shown in examples (\ref{SanGregory16}) -- (\ref{SanGregory19}). Sentence (\ref{SanGregory16}) gives an example of an existential clause in the nonpast tense, while sentence (\ref{SanGregory17}) displays an existential clause in the past tense. An example of an existential clause in the subjunctive mood is given in (\ref{SanGregory18}), and an example of a negative existential clause is provided in (\ref{SanGregory19}).

\ea
\label{SanGregory16}
[06.58.02] \\
\gll Miwa  bʊ  tʊi.\\
fruit  also  \textsc{cop}\\
\glt `There's also fruit.' 
\z 

\ea
\label{SanGregory17}
[25.01] \\
\gll Ji  pɔtʃɔ  tu.\\
one  king  \textsc{cop}.\textsc{pst}\\
\glt `There was a king.' 
\z 

\ea
\label{SanGregory18}
[12.07] \\
\gll Ji  dərɑxt  bɔjɑd  ʊm-ʊt.\\
one  tree  should  \textsc{cop.sbjv{}-3sg}\\
\glt `There should be a tree.' 
\z

\ea
\label{SanGregory19}
[09.01] \\
\gll Jɑrk  nʊ-tu.\\
work  \textsc{neg}{}-\textsc{cop}\\
\glt `There was no work.' 
\z 

The data given in examples (\ref{SanGregory13}) – (\ref{SanGregory19}) show that, in contrast to locative clauses, which usually have a definite figure, existential clauses always have an indefinite figure. These data also show that the ground of an existential clause may be either definite or indefinite, whereas the ground of a locative clause is usually definite. The syntactic structure for existential clauses can be represented as [\textsc{(grnd)} \textsc{fig}\textsc{\textsubscript{indf}} \textsc{(cop)}], where either the ground or the copula may be absent, but not both – at least one of them must be present for an existential clause to be well-formed.

\subsection{\label{Section3.3}Possessive clauses}

Possessive clauses in Wakhi contain two arguments: a possessor noun phrase (\textsc{posr}) and a possessed noun phrase (\textsc{posd}\MakeUppercase{)}. The \textsc{posr} is marked with the predicative genitive case \textit{{}-ən} `\textsc{poss}',\footnote{In previous work, I abbreviated the predicative genitive as \textsc{gen2} to distinguish it from the adnominal genitive, which I called \textsc{gen1} \citep{SanGregory2018}. In this paper, I label the case of the predicate possessor as \textsc{poss} and the case of the adnominal possessor as \textsc{gen} to make the functions of these cases more transparent.} and the \textsc{posd} is morphologically unmarked. When describing possessive clauses, I primarily use \textsc{posr} and \textsc{posd}, rather than figure and ground, to avoid confusion.

Both \textsc{posr} and \textsc{posd} arguments can occur clause-initially. An example of a possessive clause with an initial \textsc{posr} argument can be seen in (\ref{SanGregory20}). As both (\ref{SanGregory20}) and (\ref{SanGregory21}) show, the copula may or may not be used in affirmative nonpast clauses.

 
\ea
\label{SanGregory20}
[11.02] \\
\gll Jɑn  ʐʊ-nən  ji  tufɑŋgt͡ʃɑ  tu.\\
and  \textsc{1sg}{}-\textsc{poss}  one  gun  \textsc{cop}.\textsc{pst}\\
\glt `And I had a gun.'
\z 
 
\ea
\label{SanGregory21}
[25.157] \\
\gll A=jəm  pɑri-ən  ji  çɯi  dʊ  diu{}-i  sɑfed  kɑf.\\
\textsc{emph}=\textsc{dem.prox}  fairy-\textsc{poss}  one  sister  \textsc{loc}  ogre-\textsc{ez}  white  palm\\
\glt `This fairy has a sister in the hand of the white ogre.' 
\z 

Examples (\ref{SanGregory20}) – (\ref{SanGregory21}) also show that the \textsc{posr} is typically definite, while the \textsc{posd} is typically indefinite. All but two \textsc{posr}{}-initial clauses in my corpus clearly follow this definite-before-indefinite tendency. These two clauses, which are given in (\ref{SanGregory22}) – (\ref{SanGregory23}), comprise a question-answer pair from a dialogue. The definiteness of the \textsc{posr} is ambiguous in both clauses because Wakhi does not have a definite article. A determiner can make the definiteness of a noun unambiguous. For example, the number \textit{ji} `one' often functions as an indefinite article; a demonstrative or an adnominal possessor marks a noun as definite. However, an unmodified noun may be interpreted as indefinite, definite, or generic depending on the context. Thus, the \textsc{posr} \textit{ʂtʊr-ən} `camel-\textsc{poss}' could felicitously be translated as `a camel', `the camel', or `camels'. Since generics refer to an identifiable class of referents, many languages treat them as definite (\citealt{Lambrecht1994}; \citealt{DooleyLevinsohn2023}). I analyze the \textsc{posr} arguments in (\ref{SanGregory22}) – (\ref{SanGregory23}) as generics that behave as definites.

 
\ea
\label{SanGregory22}
[01.27] \\
\gll ʂtʊr-ən  t͡sum  kʊp?\\
camel-\textsc{poss}  how.many  hump\\
\glt `How many humps do camels have?' $\sim$ `How many humps [does] the camel [have]?' 
\z 

 
\ea
\label{SanGregory23}
[01.28] \\
\gll ʂtʊr-ən  bu  kʊp.\\
camel-\textsc{poss}  two  hump\\
\glt `Camels [have] two humps.' $\sim$ `The camel [has] two humps.' 
\z 

The majority of possessive clauses in my corpus have an initial \textsc{posr} argument like those shown in (\ref{SanGregory20}) – (\ref{SanGregory23}). For this reason, I propose that the unmarked (i.e., most frequent, as per \citealt{Haspelmath2006}) constituent structure of possessive clauses is [\textsc{posr}\textsc{\textsubscript{def}} \textsc{posd}\textsc{\textsubscript{indf}} \textsc{(cop)}].

In some of the possessive clauses in my corpus, the \textsc{posd} precedes the \textsc{posr}. All of the \textsc{posd}{}-initial examples found in dialogue or reported speech are idiomatic. For example, consider the syntactically possessive clause given in (\ref{SanGregory24}). In this clause, the speaker chooses to use an idiomatic expression with the literal meaning `It doesn't have a soul' to communicate that the item under discussion (a plate) is not alive rather than using a more literal attributive clause.
 
\ea
\label{SanGregory24}
[08.24] \\
\gll d͡ʒɔn  jʊt-ən  nɑst.\\
soul \textsc{3sg-poss}  \textsc{cop}.\textsc{neg}\\
\glt `It's not alive.' (Lit. `It doesn't have a soul.') 
\z 

Only two examples of \textsc{posd}{}-initial possessive clauses (one of which is a restatement of the other) occur in narrative; neither is idiomatic. The first clause of the pair is given in (\ref{SanGregory25}). Two things are worth noticing in this example. First, the second half of the clause (\ref{SanGregory25b}) is an appositive noun phrase that serves to clarify the identity of the boy referred to in the first line. Second, the constituent structure of the main part of the clause (\ref{SanGregory25a}) is remarkably similar to that of the secondary type of locative clause given in (\ref{SanGregory12}). I analyze \textsc{posd}{}-initial possessive clauses as having the structure [\textsc{posd}\textsc{\textsubscript{indf}} \textsc{posr}\textsc{\textsubscript{def}} (\textsc{cop})].\footnote{Although I treat case markers as suffixes, the use of the possessive case marker at the end of the appositive noun phrase in example (\ref{SanGregory25b}) shows that case markers in Wakhi are actually simple clitics of the phrasal affix type \citep{SanGregory2018}.}

\newpage
\ea \label{SanGregory25}
[25.55] \\
\ea \label{SanGregory25a}
\gll Ji  xɑnduni  əm  kɑʂ-ən,\\
one puppy \textsc{dem.prox} boy-\textsc{poss}\\
\glt `This boy has a puppy,\\
\ex \label{SanGregory25b}
\gll əm	pɔtʃɔ-ən əm	xɔr-jɔr	pʊtr-ən.\\
\textsc{dem.prox}  king-\textsc{poss}  \textsc{3sg.gen}  disfavored-person  son-\textsc{poss}\\
\glt this king's unbeloved son [has].' 
\z  
\z 

Cross-linguistically, the possessor and possessee sometimes form a single syntactic constituent in possessive clauses that use genitive marking. This type of possessive clause can be represented with the general formula \textsc{pr's pe is/exists} (\citealt[107, 113]{Stassen2009}). At first glance, possessor-initial clauses in Wakhi appear to follow this structure, especially since possessors precede possessees in adnominal possession as well.\footnote{Cf. the possessive noun phrase \textit{ʐʊ kɑf} `my hand' in example (\ref{SanGregory11}).} For example, (\ref{SanGregory20}) could potentially be given the literal translation seen in (\ref{SanGregory26}). However, the constituent structure seen in (\ref{SanGregory26}) is not the only option. As I have already shown in (\ref{SanGregory24}), the \textsc{posd} may precede the \textsc{posr}. In addition, one of the constituents may be postposed after the copula, as the \textsc{posd} is in (\ref{SanGregory27}).\footnote{I discuss postposing along with other minor structural variations in \sectref{Section4.4}.} Therefore, I conclude that the \textsc{posr} and \textsc{posd} do not form a single constituent in Wakhi, but rather are distinct constituents.

 
\ea
\label{SanGregory26}
[11.02] \\
\gll Jɑn  ʐʊ-nən  ji  tufɑŋgt͡ʃɑ  tu.\\
and  \textsc{1sg}{}-\textsc{poss}  one  gun  \textsc{cop}.\textsc{pst}\\
\glt `And I had a gun.' (Lit. `And mine one gun was.') 
\z 

\ea
\label{SanGregory27}
[25.03] \\
\gll əm  pɔtʃɔ-ən  tu  bu  kʊnd  ʊt  tru  pʊtr.\\
\textsc{dem.prox}  king-\textsc{poss}  \textsc{cop}.\textsc{pst}  two  wife  and  three  son\\
\glt `This king had two wives and three sons.' (Lit. `This king's were two wives and three sons.') 
\z 

In \sectref{Section3}, I have described the syntactic structure of locative, existential, and possessive clauses. Each type of clause exhibits similarities in grammar and, in particular, some variations in the order of constituents that seem unexpectedly parallel. In \sectref{Section4}, I explore a unified explanation for the variations across all three types of nonverbal clause constructions.

\section{\label{Section4}Syntactic variation and information structure}
\subsection{\label{Section4.1}Overview: information structure}

In this section, I introduce the relevance of information structure for the analysis of the different types of nonverbal clause constructions discussed in \sectref{Section3}. Although word order is often considered to be static, in actuality, it is both more variable and more closely related to information structure than we usually acknowledge. Pragmatic factors are often the driving force behind grammatical variety, as \citet[281]{Payne1993} recognizes when she urges that ``discourse-pragmatic factors must be built into the models [of grammar] from the ground up.''

Moving forward, I provide a brief overview of known connections between information structure, pragmatics, and syntax, paying particular attention to the notions of topic and focus. These terms have been defined in various ways by various people (cf. \citealt{DikEtAl1981}; \citealt{Chafe1987}; \citealt{DooleyLevinsohn2023}); for the purposes of this paper, I assume Lambrecht's \citeyearpar{Lambrecht1994} definitions.

In his seminal book on information structure, Lambrecht addresses sentence \textsc{topic} (as opposed to discourse topic), defining it as what a sentence is about. The topic is not necessarily the subject, nor even a noun phrase – for example, a locative phrase may be the topic of a thetic sentence \citep[145]{Lambrecht1994} – but it is always part of the presupposition. \textsc{Focus}, on the other hand, is ``the element of information whereby the presupposition and assertion differ from each other [...] the unpredictable or pragmatically non-recoverable element in an utterance" \citep[207]{Lambrecht1994}. In other words, focus is essential new information. Lambrecht proposes three subtypes of focus: predicate focus, argument focus, and sentence focus; each correlates with distinct pragmatic functions and domains (i.e., they have different scopes).

A number of linguists have observed the high frequency of topic-comment sentences in languages of the world and have concluded that topic-comment is universally the most basic/pragmatically unmarked type of sentence (\citealt{Hockett1966}; \citealt{Tomlin2016}). For some, a corollary to this conclusion is a universal, clause-initial topic position (\citealt{Clark1978}; \citealt{Tomlin2016}). \citet{Lambrecht1994} cites the existence of verb-initial languages as clear evidence that a clause-initial topic position is not universal since, in his framework, verbs cannot be topics; additional evidence is provided by data from languages such as O'odham (Uto-Aztecan; toho1245) \citep{Payne1993}. Inaccurate though it may be to say that an initial topic position is universal, there is nonetheless a cross-linguistic tendency for topics to occur at the beginning of the clause \citep{Lambrecht1994}.

Focus is not strongly correlated with any particular syntactic position, but marked word order is one of several methods commonly used to encode focus (\citealt{DikEtAl1981}; \citealt{Payne1993}; \citealt{DooleyLevinsohn2023}). In various languages, focus may appear in different positions; for example, preposed focus, sentence-final focus, and preverbal focus positions are a few commonly attested options (\citealt{DikEtAl1981}; \citealt{Lambrecht1994}; \citealt{DooleyLevinsohn2023}). Furthermore, a language may also have multiple focus positions that are used to encode different types of focus (\citealt{DooleyLevinsohn2023}).

Thus far, I have mentioned only morphosyntactic exponents of information structure. I recognize that prosodic features, such as intonation and pauses, are also important; however, I do not examine prosody in this paper. An in-depth study of prosodic features and their pragmatic functions in Wakhi is work that remains for the future.

\subsection{\label{Section4.2}Information structure in nonverbal clauses}

\subsubsection{\label{Section4.2.1}Predicational vs. presentational clauses}

In \sectref{Section3.1} – \sectref{Section3.3}, I describe the constituent structures found in locative, existential, and possessive clauses in Wakhi in terms of figure-ground relationships. However, each of these three types of nonverbal clause constructions serves a different information structure purpose. In the following sections, I review the pragmatic functions of locative, existential, and possessive clauses as defined in the literature. I give examples of Wakhi clauses with these functions and reframe my structural representations of these clauses in terms of topic and focus.

Before I analyze the information structure of locative, existential, and possessive clauses in Wakhi, I must briefly revisit the difference between predicational and presentational clauses. Recall from \sectref{Section2.1} that a predicational clause is one in which the predicate asserts something about the subject (or the figure, in Wakhi nonverbal clause constructions). In essence, ``predicational'' is another term for a topic-comment clause (\citealt{Stassen1997}; \citealt{Croft2022}). A presentational clause, on the other hand, is a type of thetic clause;\footnote{The other type of thetic clause is what \citet[143--144]{Lambrecht1994} calls ``event-reporting;'' i.e., a verbal clause (usually intransitive) that introduces an activity or event into the discourse.} its primary function is to introduce a referent into a discourse (\citealt{Lambrecht1994}; \citealt{Stassen1997}; \citealt{Croft2022}). All existential clauses are presentational; locative and possessive clauses may be either predicational or presentational, as I demonstrate in the coming sections.

\subsubsection{\label{Section4.2.2}Locative clauses}

The basic function of a predicative locative clause is to assert the location of the figure \citep{Croft2022}. The figure of a locative clause is usually given, definite, and topical; although the locative ground may be either definite or indefinite, it is usually unpredictable/unrecoverable and is therefore the focus (\citealt{Clark1978}; \citealt{DooleyLevinsohn2023}).

An example of a locative clause with this information structure can be seen in (\ref{SanGregory11}), repeated below as (\ref{SanGregory28}). This sentence comes from a short first-person narrative in which the narrator tells the (fictional) story of having stolen a purse from the second-person addressee. The discourse context given in (\ref{SanGregory28}) shows that the definite figure \textit{ti dʊstkɑwl} `your purse' is given and highly accessible in the discourse; therefore, I analyze it as the topic of clause (\ref{SanGregory28}). The locative ground \textit{dʊ ʐʊ kɑf} `in my hand' may or may not be unrecoverable – although the speaker never said what he did with the purse, the possibility of its being in his hand is certainly evoked by the frame. Regardless of the accessibility of the ground, the purpose of the entire clause is to assert the location of the purse; thus, the ground is the focus. Since in Wakhi the locative ground is the semantic predicate (cf. \sectref{Section3.1}), I analyze this sentence as having a topic-comment information structure with predicate focus. I propose the schema [\textsc{top} \textsc{foc} \textsc{(cop)}] to represent this structure, which is found in predicational locative clauses.

\ea
\label{SanGregory28} 
[11.11] \\
Context: I pulled out my gun and stopped you. I said: Give me your purse! And I took your purse from your hand. You left, and I fled from before you. Then when I was going, the police saw me. The police saw me and followed me. I fled from the police.\\
\gll Jɑn  ti  dʊstkɑwl  dʊ  ʐʊ  kɑf  tu.\\
and  \textsc{2sg.gen}  purse  \textsc{loc}  \textsc{1sg.gen}  palm  \textsc{cop}.\textsc{pst}\\
\glt `And your purse was in my hand.' 
\z 

As mentioned in \sectref{Section3.1}, presentational locative clauses have an indefinite figure and a different information structure from predicational clauses. For example, consider (\ref{SanGregory29}) in its context, as shown by the preceding sentence from the discourse. The indefinite figure \textit{ji ʃɑmʃir} `a sword' cannot be topical because it is brand-new; the definite locative ground \textit{dʊtən} `with him' refers to the boy from the previous sentence and thus is both given and topical. Sentence (\ref{SanGregory29}) functions pragmatically to introduce the sword and make it accessible for reference in the continuing discourse. Given this introductory function, I analyze this clause as representative of the presentational locative clause type. The marked information structure schema [\textsc{foc} \textsc{top} (\textsc{cop})] is used to indicate this presentational function.

\ea
\label{SanGregory29}
[25.187] \\
Context: The boy goes to the corner; he comes out from the house and puts himself in the corner and hides.\\
\gll Ji  ʃɑmʃir  bʊ  dʊt-ən  tʊi.\\
one  sword  also  with.\textsc{med}{}-\textsc{abl}  \textsc{cop}\\
\glt `And a sword is with him.' 
\glt Suddenly he sees that it has turned white. And suddenly the white ogre came. He saw that the white ogre came and like this was going up to enter the house. And he struck it with the sword, and it died.
\z 

\subsubsection{\label{Section4.2.3}Existential clauses}

As mentioned in \sectref{Section2.1}, the primary function of existential clauses is to introduce a new entity into the discourse; despite their name, the assertion of existence is a secondary function (\citealt{Hengeveld1992}; \citealt{Givon2001}; \citealt{Gaeta2013}; \citealt{OverallEtAl2018}). Both functions present entirely new/unpredictable information (i.e., they are thetic (\citealt{Lambrecht1994}; \citealt{Croft2022}; \citealt{DooleyLevinsohn2023})). Therefore, existential clauses exhibit sentence focus, and their figures are typically indefinite (\citealt{Clark1978}; \citealt{Lambrecht1994}; \citealt{PanovaLiljegren2025}).

An example of a Wakhi existential clause used to introduce a referent can be seen in (\ref{SanGregory30}). This example contains an indefinite figure (\textit{ji} \textit{xɑnduni} `a puppy') but no ground. The speaker introduces the puppy earlier in the folktale, but then does not mention it again for more than ten clauses. In the intervening time, he describes the road taken by the main character on his quest. After concluding this description, the speaker uses the existential clause given in (\ref{SanGregory30}) to reintroduce the puppy to the discourse.

\ea
\label{SanGregory30}
[25.69] \\
Context: The disfavored one goes on the road [called] ``Go and don't return.'' That is, you go and don't come back. So he went on this road.\\
\gll ʊt  degɑ  ji  xɑnduni,  xɑnduni  bʊ  tʊi\\
and  also  one  puppy  puppy  also  \textsc{cop}\\
\glt `And also, a puppy – there is also a puppy.' 
\z 

Cross-linguistically, many existential clauses contain a locative phrase that functions as the ground. The locative ground is often definite and tends to precede the figure (\citealt{Clark1978}; \citealt{PanovaLiljegren2025}). As seen in \sectref{Section3.2}, the ground of an existential clause in Wakhi may be either indefinite or definite. Regardless of its definiteness, a locative ground is the pragmatic topic of the existential clause in which it occurs; the fact that the initial locative phrase may or may not be included without affecting the primary function of the clause is evidence for its topicality (cf. \citealt{Lambrecht1994}: 224).\footnote{\citet[216]{Lambrecht1994} states that ``focus domains must be allowed to contain non-focal elements.'' He also observes that thetic sentences can have a locative topic \citep[145]{Lambrecht1994}.} For example, consider (\ref{SanGregory14}), repeated below as (\ref{SanGregory31}). The storyteller mentions a cave in the previous sentence in the discourse, but then realizes that the listener is not aware of the existence of a second cave yet. He takes a rhetorical step back and uses the existential clause in (\ref{SanGregory31}) to officially introduce the second cave. Although this existential clause includes a clause-initial ground \textit{ji pɑlɯ} `on one side', it would be just as effective in achieving its pragmatic purpose (introducing the cave into the discourse) without the ground, i.e., \textit{bɔi dəʝɑr bʊ tʊi} `There is another cave.'

\ea
\label{SanGregory31}
[25.500] \\
Context: Suddenly he [the boy] sees that a fox is in the neighboring cave, like this.\\
\gll Ji  pɑlɯ  bɔi  dəʝɑr  bʊ  tʊi.\\
one  side  cave  another  also  \textsc{cop}\\
\glt `On one side is another cave.' 
\z 

Based on my pragmatic analysis, I propose that existential clauses in Wakhi have an optional topic, an obligatory focus, and an optional copula. This pragmatic structure can be represented by the schema [(\textsc{top}) \textsc{foc} (\textsc{cop})]. Although both the topical ground and the copula are represented as optional in this schema, recall from \sectref{Section3.2} that at least one of the two must be present for an existential clause to be grammatical.

\subsubsection{\label{Section4.2.4}Possessive clauses}

Like locative clauses, possessive clauses can be either predicational or presentational. The pragmatic function of a possessive clause closely correlates with the syntactic structure of the clause. In the literature, the pragmatic function of a possessive clause is usually distinguished on the basis of definiteness and word order. Predicational possessives commonly have a definite possessee in the topic position, while presentational possessives typically have an indefinite possessee that occurs in the focus position. In other words, presentational possessives have the same constituent structure as other presentational clauses (\citealt{Hengeveld1992}; \citealt{Koch2012}; \citealt{Croft2022}).

In \sectref{Section3.3}, I demonstrate that the most frequent syntactic structure of possessive clauses in my corpus is [\textsc{posr}\textsc{\textsubscript{def}} \textsc{posd}\textsc{\textsubscript{indf}} \textsc{(cop)}]. Sentence (\ref{SanGregory32}), repeated from (\ref{SanGregory20}), exhibits this structure. The clause-initial \textsc{posr} is definite and given based on the context provided by the preceding sentence, so I analyze it as the topic; in contrast, the \textsc{posd} is indefinite and new, and therefore focal.\footnote{I have no examples of indefinite possessors in main clauses in my corpus, nor do I have any examples of definite possessees. Without these data, I cannot say with certainty whether the apparent correlation between definiteness, argument type, and pragmatic function is significant or just a coincidental gap in the data.} The schema [\textsc{top} \textsc{foc} (\textsc{cop})] represents this pragmatic structure.

\ea
\label{SanGregory32}
[11.02] \\
Context: One day I came to the bazaar when you had a purse in your hand.\\
\gll Jɑn  ʐʊ-nən  ji  tufɑŋgt͡ʃɑ  tu.\\
and  \textsc{1sg}{}-\textsc{poss}  one  gun  \textsc{cop}.\textsc{pst}\\
\glt `And I had a gun.' 
\z 

A different syntactic structure can be seen in sentence (\ref{SanGregory25}), repeated below as (\ref{SanGregory33}): the indefinite \textsc{posd} is the first constituent of the clause, followed by the definite \textsc{posr}. The \textsc{posr} \textit{əm kɑʂən} `this boy' is given and topical based on the preceding context, despite the fact that its intended referent must be disambiguated from other possible antecedents in the following phrase. The \textsc{posd} \textit{ji xɑnduni} `a puppy' is brand-new and completely unpredictable from the preceding discourse, so I analyze it as the focus. This pragmatic structure can be represented as [\textsc{foc} \textsc{top} (\textsc{cop})].

\ea
\label{SanGregory33}
[25.55] \\
Context: When they [i.e., the unbeloved boy and his brothers] depart, they go a very long way to two roads.\\
\gll Ji  xɑnduni  əm  kɑʂ-ən.\\
one  puppy  \textsc{dem.prox}  boy-\textsc{poss}\\
\glt `This boy has a puppy, [the king's unbeloved son does. He had a puppy.]' 
\z

Examples (\ref{SanGregory32}) and (\ref{SanGregory33}) are opposites in terms of syntax and information structure, but the definiteness of their \textsc{posr} and \textsc{posd} arguments is the same. The fact that both clauses introduce a new, indefinite \textsc{posd} could be taken as evidence that both are presentational; however, I argue that only (\ref{SanGregory33}) is presentational and that (\ref{SanGregory32}) is actually predicational for two reasons. First, as I noted in \sectref{Section3.3}, the \textsc{posr}{}-initial structure of (\ref{SanGregory32}) is far more frequent in my corpus than the \textsc{posd}{}-initial structure is. All else being equal, a more frequent structure is usually an unmarked structure. Second, (\ref{SanGregory33}) clearly functions to introduce the puppy as a new character in the discourse. The puppy remains a significant character throughout the folktale as he repeatedly helps the boy on his quest, but it is not so clear that (\ref{SanGregory32}) introduces the gun as similarly significant. It seems rather to provide more information about the \textsc{posr} – that is, it paints a picture of him as an armed and dangerous person – in which case it is part of the comment in a topic-comment construction, i.e., a predicational clause.\footnote{See \citet[102]{Stassen1997} and Croft's \citeyearpar[291]{Croft2022} summary of Stassen for further discussion of the ``file-expanding'' function of predicative clauses.}

Based on the reasoning above, I analyze (\ref{SanGregory32}) as an example of a predicational possessive clause, and I propose the information structure of such clauses to be [\textsc{top} \textsc{foc} (\textsc{cop})]. I further analyze (\ref{SanGregory33}) as an example of a presentational possessive clause, which has the marked information structure [\textsc{foc} \textsc{top} (\textsc{cop})].

\subsection{\label{Section4.3}A unified analysis of nonverbal clause structure}

In \sectref{Section4.2.2} – \sectref{Section4.2.4}, I examine the syntactic structure of locative, existential, and possessive clauses through the lens of information structure and revise my structural schemata in terms of topic and focus. In this section, I propose two information structure constructions that not only unite these clauses but also account for the majority of syntactic variations they exhibit.

The first construction I propose is a basic information structure construction of the form [(\textsc{top)} \textsc{foc} (\textsc{cop})]. This construction models the default word order found in existential, predicational locative, and predicational possessive clauses, as I explain below.

The position of content question words in Wakhi provides evidence for the focus position I have proposed. The default location of a question word/phrase is immediately before the verb or copula regardless of the constituent being questioned. For example, in (\ref{SanGregory34}) the questioned subject occurs immediately before the verb, not before the object as it typically does in declarative clauses. If a nonverbal clause without a copula is in the interrogative mood, the question word/phrase is clause-final, as seen in (\ref{SanGregory35}).

\ea
\label{SanGregory34}
[25.577] \\
\gll ʐʊ  t͡ʃigɑsʊk-i  tilɔ-i  ki  wɔzɔm-di?\\
\textsc{1sg.gen}  shelf-\textsc{ez}  gold-\textsc{adj}  who  bring-\textsc{pst}\\
\glt `Who stole my golden shelf?' (Lit. `My golden shelf who brought?') 
\z 
 
\ea
\label{SanGregory35}
[25.232] \\
\gll ɑw  kum  d͡ʒɑi?\\
\textsc{3sg.nom}  which  place\\
\glt `Where is it?' 
\z 

In predicational clauses, the focus domain is the predicate (\citealt{Lambrecht1994}; \citealt{Croft2022}). Since the Wakhi copula (like copulas in many languages) is not a true verb, the semantic predicate of a nonverbal clause is the focused element. Therefore, the focus of a predicative locative clause is the locative phrase (ground), and the focus of a predicational possessive clause is the possessee, as seen in (\ref{SanGregory36}) and (\ref{SanGregory37}), repeated from (\ref{SanGregory8}) and (\ref{SanGregory32}), respectively. Both of these predicational clauses have the information structure [\textsc{top} \textsc{foc} (\textsc{cop})], as I show in \sectref{Section4.2.2} and \sectref{Section4.2.4}.

\ea
\label{SanGregory36}
[25.500.02] \\
\gll Jəm-əʃ  tʊt  pɑlɯ.\\
\textsc{3.nom-pl}  up.to.\textsc{med}  side\\
\glt `They are up on that side.' 
\z  

\ea
\label{SanGregory37}
[11.02] \\
\gll Jɑn  ʐʊ-nən  ji  tufɑŋgt͡ʃɑ  tu.\\
and  \textsc{1sg}{}-\textsc{poss}  one  gun  \textsc{cop}.\textsc{pst}\\
\glt `I had a gun.' 
\z 

In \sectref{Section4.2.3}, I show that existential clauses may include a topical locative phrase (ground) that precedes the focal figure of the clause, as seen in example (\ref{SanGregory38}). However, a locative topic may or may not appear as part of an existential clause, as seen in (\ref{SanGregory39}), repeated from (\ref{SanGregory17}). I conclude that the initial topic is an optional element in the clause, and I schematize the construction as [(\textsc{top}) \textsc{foc} (\textsc{cop})]. My analysis is consistent with Lambrecht's (\citeyear[206]{Lambrecht1994}) claim that all sentences have a focus, but not all sentences have a topic.

\ea
\label{SanGregory38}
[09.09] \\
\gll Dɑm  kɑf  ji  dʊstkɑwl  tu.\\
\textsc{loc.prox}  palm  one  purse  \textsc{cop}.\textsc{pst}\\
\glt `In her hand was a purse.' 
\z 
 
\ea
\label{SanGregory39}
[25.01] \\
\gll Ji  pɔtʃɔ  tu.\\
one  king  \textsc{cop}.\textsc{pst}\\
\glt `There was a king.' 
\z 

\tabref{LocExPoss-IS} displays a summary of the constituent structure vs. information structure of predicational locative, existential, and predicational possessive clauses. Although each type of clause has a different constituent structure, all three have the same information structure, i.e., [(\textsc{top}) \textsc{foc} (\textsc{cop})]. Either figure or ground (possessor or possessee in possessive clauses) can be mapped to topic or focus depending on the pragmatic function of the clause; thus, pragmatic function is responsible for producing the different word orders found in the different types of clauses.

\begin{table}
\caption{\label{LocExPoss-IS} Constituent vs. information structure in existential clauses and predicational clauses} 
\begin{tabular}{lllllll} 
\lsptoprule
& \multicolumn{3}{l}{{Constituent structure}} & \multicolumn{3}{l}{{Information structure}}\\
\midrule
{Locative} & {[\textsc{fig}\textsc{\textsubscript{def}}} & {\textsc{grnd}\textsc{\textsubscript{def}}} & {(\textsc{cop})]} & {[\textsc{top}} & {\textsc{foc}} & {(\textsc{cop})]}\\
{Existential} & {[(\textsc{grnd})} & {\textsc{fig}\textsc{\textsubscript{indf}}} & {(\textsc{cop})]} & {[(\textsc{top})} & {\textsc{foc}} & {(\textsc{cop})]}\\
{Possessive} & [\textsc{posr}\textsc{\textsubscript{def}} & \textsc{posd}\textsc{\textsubscript{indf}} & (\textsc{cop})] & [\textsc{top}  & \textsc{foc} & {(\textsc{cop})]}\\
\lspbottomrule
\end{tabular}
\end{table}

Although the basic information structure construction proposed above models word order in existential, predicational locative, and predicational possessive clauses, it fails to do so for the alternate syntax found in some locative and possessive clauses. Example (\ref{SanGregory40}), given earlier as (\ref{SanGregory26}), has the same general constituent structure as a predicative locative clause but with an indefinite figure. In (\ref{SanGregory41}), repeated from (\ref{SanGregory33}), the order of the \textsc{posr} and \textsc{posd} is reversed from that found in predicational possessive clauses. The constituent structure of these clauses can be represented as [\textsc{fig}\textsc{\textsubscript{indf}} \textsc{grnd}\textsc{\textsubscript{def}} (\textsc{cop})] and [\textsc{posd}\textsc{\textsubscript{indf}} \textsc{posr}\textsc{\textsubscript{def}} (\textsc{cop})], respectively. 

\ea
\label{SanGregory40}
[25.187] \\
\gll Ji  ʃɑmʃir  bʊ  dʊt-ən  tʊi.\\
one  sword  also  with.\textsc{med}{}-\textsc{abl}  \textsc{cop}\\
\glt `And a sword is with him.' 
\z 

\ea
\label{SanGregory41}
[25.55] \\
\gll Ji  xɑnduni  əm  kɑʂ-ən.\\
one  puppy  \textsc{dem.prox}  boy-\textsc{poss}\\
\glt `This boy has a puppy.' 
\z 

In \sectref{Section4.2.2} and \sectref{Section4.2.4}, I analyze these sentences as presentational clauses because they function pragmatically to introduce a significant new entity into the discourse. I show that in both examples, the clause-initial indefinite figure/possessee is the new entity being introduced and thus the focus. Similarly, the definite ground/possessor is given and topical. Therefore, I propose that the construction for presentational locatives and possessives is [\textsc{foc} \textsc{top} (\textsc{cop})].

The presentational construction differs from the basic information construction in both constituent order and optionality of the topic. These differences are what signal the different functions of the two constructions. The basic information structure construction is called such because it used to encode the most ``basic'' or primary function of a clause. For locative and possessive clauses, the basic function is predicative: a locative clause encoded with this construction asserts the location of the given topic, and a possessive clause encoded with this construction expands our knowledge of the \textsc{posr} by asserting that a \textsc{posd} belongs to him or her. For existential clauses, the basic function is the only function: they introduce a new referent.

In contrast to the basic information construction, which signals that a clause should be interpreted according to its primary function, the marked information structure of the presentational construction signals that a clause should be interpreted according to its secondary function. For both locative and possessive clauses, the secondary function is to introduce a new entity – and not just any new entity, but one that is significant in the discourse context. It is this added meaning of discourse significance that is unique and that motivates my distinction between the two constructions.

\subsection{\label{Section4.4}Relationships among nonverbal clauses}

One of the questions I ask in \sectref{Section2.3} regards the nature of the relationship between locative and existential clauses. Specifically, can these clauses be syntactically distinguished, or are they simply different functions of the same construction? I have demonstrated that these two clauses have different pragmatic functions: the purpose of a locative clause is to assert the location of a given entity, but the purpose of an existential clause is to introduce a new entity. As information structure interacts with what is being predicated, topic and focus are mapped to different constituents, thereby producing different syntactic patterns. Since syntax is determined by information structure, I conclude that there is no formal distinction between existential clauses and predicative locative clauses in Wakhi and that the difference between them is merely pragmatic. In other words, Wakhi lacks a dedicated existential construction (in terms of the typology given in \citealt{Creissels2019}), and what appear to be locative and existential clauses are merely different pragmatic uses of a single locative-existential construction. 

The question still remains: what is the position of possessive clauses relative to the locative-existential type? \citet[50]{Stassen2009} proposes that ``possessive constructions in which the PR [possessor] has genitive case marking are [...] cases of the Locational Possessive''. In \sectref{Section4.3}, I show that Wakhi possessive clauses have the same information structure as locative clauses. These data are consistent with the areal typology of possession: Indo-European languages of Asia, including Iranian languages, are commonly analyzed as using a locative strategy to encode predicative possession (\citealt{Stassen2009}; \citealt{DanesiBarðdal2018}).

The use of a locative strategy does not necessarily indicate a diachronic relationship between locative and possessive clauses in Wakhi. Another piece of evidence to consider is the form of the possessive (i.e., predicative genitive) case marker: it is identical to the ablative case marker \textit{{}-ən}. As a result of this syncretism, the possessive and ablative cases must be distinguished by their different functions and by the different stems to which they attach. The possessive marker attaches to a genitive stem, while the ablative marker attaches to an accusative stem \citep[51]{SanGregory2018}. The ablative case is used to mark a variety of semantic roles, including separation/partition, accompaniment, instrument, comparison, source, and origin \citep{SanGregory2018}. The semantic roles of source and origin (and sometimes accompaniment) are locative in nature. An ablative of source can be seen in (\ref{SanGregory42}).

\ea
\label{SanGregory42}
[03.03] \\
\gll Wuz=əm  t͡sʊ  xun-ən  niʊʂ-ti.\\
\textsc{1sg.nom}=\textsc{1sg}  from  house-\textsc{abl}  come.out-\textsc{pst}\\
\glt `I came out of the house.' 
\z 

Based on data such as that seen in (\ref{SanGregory42}), some linguists analyze possessors as animate locations cross-linguistically (\citealt{Lyons1967}; \citealt{Clark1978}; \citealt{DeLancey2003}). Others disagree, saying that the concepts of possession and location should be kept separate (\citealt{Heine1997Cognitive}; \citealt{Stassen2009}). I am inclined to agree with Heine and Stassen on the conceptual distinction between possession and location. However, the question of the diachronic relationship between locative and possessive clauses in Wakhi is a different matter entirely. The syncretism between the possessive and ablative cases certainly suggests a historical connection between the two types of clauses. Therefore, I tentatively propose that a historical locative construction was the source of synchronic possessive clauses in Wakhi. Further historical/comparative work on case in Wakhi and related languages is needed in order to test the plausibility of this hypothesis.

\subsection{\label{Section4.5}Residue}

The two constructions that I propose in \sectref{Section4.3} account for nearly all of the data in my corpus. However, there are a few residual sentences that I cannot explain at this point. Example (\ref{SanGregory43}) exhibits a constituent structure found in several locative clauses in which the locative ground follows the copula rather than preceding it. Sentence (\ref{SanGregory44}) represents several existential clauses that have a clause-final ground, rather than the expected clause-initial ground. Example (\ref{SanGregory45}) also presents an existential clause with an unexpected constituent structure, but in this case the figure occurs in clause-final position. In (\ref{SanGregory46}), the compound \textsc{posd} follows the copula, and in (\ref{SanGregory47}), the copula occurs at the beginning, not the end, of the possessive clause.

\ea
\label{SanGregory43}
[25.580] \\
\gll əm  tu  dɑ  rɑʒ  sɑr.\\
\textsc{3sg.nom}  \textsc{cop}.\textsc{pst}  \textsc{loc}  sitting.area  head\\
\glt `It [the golden shelf] was at the top of the sitting area.' 
\z
 
\ea
\label{SanGregory44}
[01.30] \\
\gll Tɑswir  bʊ  tu  dʊ  kitɔb.\\
illustration  also  \textsc{cop}.\textsc{pst}  \textsc{loc}  book\\
\glt `There were also illustrations, in the book.' 
\z
 
\ea
\label{SanGregory45}
[17.01.02] \\
\gll Dʊ  ji  t͡ʃɔinʊk  tu  ʒɑrʒ.\\
\textsc{loc}  one  teapot  \textsc{cop}.\textsc{pst}  milk\\
\glt `In a teapot was [some] milk.' 
\z
 
\ea
\label{SanGregory46}
[25.03] \\
\gll əm  pɔtʃɔ-ən  tu  bu  kʊnd  ʊt  tru  pʊtr.\\
\textsc{dem.prox}  king-\textsc{poss}  \textsc{cop}.\textsc{pst}  two  wife  and  three  son\\
\glt `This king had two wives and three sons.' (Lit. `This king's were two wives and three sons.') 
\z
 
\ea
\label{SanGregory47}
[05.56] \\
\gll Nɑst  ti-nən  zur.\\
\textsc{cop}.\textsc{neg}  \textsc{2sg}{}-\textsc{poss}  strength\\
\glt `You have no strength!' (Lit. `[It] is not, yours strength!') 
\z 

Three of these syntactically anomalous clauses occur in dialogue: sentences (\ref{SanGregory43}) and (\ref{SanGregory44}) are responses to questions, and in sentence (\ref{SanGregory47}) the speaker contradicts his conversation partner's claim of strength. Examples (\ref{SanGregory45}) and (\ref{SanGregory46}) both occur in narrative and appear to be some kind of extra-marked presentational.

Prosodically, (\ref{SanGregory43}), (\ref{SanGregory45}), and (\ref{SanGregory46}) all exhibit a single intonation contour. In comparison, the atypically located constituents in (\ref{SanGregory44}) and (\ref{SanGregory47}) are both set off from the rest of the sentence by an intonation break, which is indicated by a comma in the free translations. It seems clear to me that each of these syntactic structures serves some pragmatic purpose – for example, the postposed constituents in (\ref{SanGregory44}) and (\ref{SanGregory47}) appear to be afterthoughts – but at this time, I cannot say what exactly the purpose of each variation may be. Further study of prosody, preposing, and postposing in both nonverbal and verbal clauses is needed in order to determine how these features are used in Wakhi.

\section{\label{Section5}Conclusion}

In this paper, I have shown that locative, existential, and possessive clauses in Wakhi each exhibit several syntactic structures. Most locative clauses use the same [\textsc{fig} \textsc{grnd}\textsc{\textsubscript{def}} \textsc{(cop)}] structure, but the definiteness of the figure may vary. The structure [\textsc{(grnd)} \textsc{fig}\textsc{\textsubscript{indf}} \textsc{(cop)}] is found in existential clauses; besides being optional, the ground of an existential may be either definite or indefinite. In most possessive clauses, the possessor occurs clause-initially in a [\textsc{posr}\textsc{\textsubscript{def}} \textsc{posd}\textsc{\textsubscript{indf}} (\textsc{cop})] structure; however, the structure [\textsc{posd}\textsc{\textsubscript{indf}} \textsc{posr}\textsc{\textsubscript{def}} (\textsc{cop})] is also attested for clauses in which the possessee is initial. Word order provides a starting point for identifying these clauses, but formal criteria alone are not sufficient to distinguish between any of them, nor to explain why multiple syntactic variations are possible for each type of clause.

Having found the word order approach to be insufficient, I have instead taken a usage-based approach to analyzing locative, existential, and possessive clauses in Wakhi. I have analyzed the data from an information structure perspective and have discovered patterns correlated with pragmatic function, topic, and focus. Based upon these patterns, I have proposed that two information-structure-based constructions can account for the majority of the syntactic variations found in locative, existential, and possessive clauses in Wakhi. The first of these constructions is the basic information structure construction [(\textsc{top}) \textsc{foc} (\textsc{cop})]. This construction represents the constituent structure of clauses serving their primary function: existential clauses, predicational locatives, and predicational possessives. The second construction is the presentational construction. The presentational construction uses the marked pragmatic structure [\textsc{foc} \textsc{top} (\textsc{cop})] and contributes the additional meaning of discourse significance; it is found in presentational locatives and possessives, which serve to introduce a significant new referent into the discourse.

The copula contributes very little (if any) lexical meaning to either of the constructions I have proposed. Its primary function is to convey grammatical meaning, including tense, polarity, and subjunctive mood (cf. \sectref{Section2.3}), which cannot otherwise be indicated on a nonverbal semantic predicate. The copula frequently does not appear in nonverbal clauses; a clause without a copula is interpreted as being nonpast and affirmative. Since the copula is not the semantic predicate of a nonverbal clause, and since its presence/absence does not affect the grammaticality of affirmative nonpast clauses, I mark it as optional in both constructions.

I analyze the basic information structure construction as the unmarked construction of the pair for two main reasons. First, it occurs more frequently than does the presentational construction. Second, it is used to convey the ``prototypical'' meaning of each clause. That is, a locative clause with this structure asserts location; a possessive clause with this structure expands the listener's knowledge about the possessor by asserting a possessee to be in his/her control; and an existential clause (which can use only this structure) introduces a referent into the discourse. The presentational construction is much more limited, since it is used only with locative and possessive clauses and only to signal that such clauses serve their secondary function of introducing a significant new entity.

Taking a usage-based approach to locative, existential, and possessive clauses in Wakhi has provided clarity that a formal structural approach could not. With this approach, I have both unified the various syntactic structures exhibited by these clauses into two constructions and also proposed an explanation for why Wakhi speakers choose one construction over another for an individual speech event. Furthermore, I have demonstrated that information structure determines the syntax of locative, existential, and possessive clauses in Wakhi. In so doing, I have not only confirmed the necessity of including pragmatic factors in our syntactic models (\citealt{Mithun1987}; \citealt{Payne1993}), but I have also provided a foundation for examining other aspects of Wakhi syntax from a functional perspective.

\section*{Acknowledgements}

I am grateful to Don Daniels, Spike Gildea, and the participants in the Spring 2023 Workshop on Language Documentation and Revitalization at the University of Oregon for their input and encouragement. I am also grateful to the many Wakhi people who have shared their language and culture with me. The research reported in this paper was funded in part by a grant from the Pike Center for Integrative Scholarship.\footnote{\url{http://www.pikecenter.org}}

\section*{Abbreviations}
\begin{tabularx}{.6\textwidth}{@{}lQ@{}}
\textsc{emph} & emphasis\\
\textsc{ez} & ezafe (Persian linking particle)\\
\textsc{fig} & figure\\
\textsc{grnd} & ground\\
\textsc{interj} & interjection\\
\textsc{ints} & intensifier\\
\end{tabularx}%
\begin{tabularx}{.4\textwidth}{@{}lQ@{}}
\textsc{med} & medial\\
O & object\\
\textsc{posd} & possessed\\
\textsc{posr} & possessor\\
S & subject\\
V & verb\\
\end{tabularx}

\sloppy
\printbibliography[heading=subbibliography,notkeyword=this]
\end{document}
