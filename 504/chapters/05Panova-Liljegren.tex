\documentclass[output=paper,colorlinks,citecolor=brown]{langscibook}
\ChapterDOI{10.5281/zenodo.16838062}

\author{Anastasia Panova\affiliation{Stockholm University} and Henrik Liljegren\affiliation{Stockholm University}}

\title[Locative and existential predication in Gawarbati]{Locative and existential predication contrasts in Gawarbati (Indo-Aryan) and the surrounding region}

\abstract{This paper analyses the morphosyntactic variation in locative (\textsc{loc}) and locational-existential (\textsc{loc-ex}) clauses in Gawarbati, an underdescribed Indo-Aryan language which has no dedicated formal marking of the \textsc{loc} vs. \textsc{loc-ex} contrast. The analysis also compares figure-ground predications in geographically adjacent languages. The results show that there are three morphosyntactic parameters which reflect, to varying degrees, the \textsc{loc} vs. \textsc{loc-ex} status of the predication: word order, indefiniteness marking and the lexical identity of the predicate. The parameter reflecting the \textsc{loc} vs. \textsc{loc-ex} alternation most consistently is word order. However, as shown in the corpus study of Gawarbati, what is primarily encoded by word order variation is a range of available information-structural patterns, and they do not always easily match with the \textsc{loc} vs. \textsc{loc-ex} distinction.}

\IfFileExists{../localcommands.tex}{
   \addbibresource{../localbibliography.bib}
   % add all extra packages you need to load to this file

\usepackage{tabularx,multicol}
\usepackage{url}
\urlstyle{same}

\usepackage{listings}
\lstset{basicstyle=\ttfamily,tabsize=2,breaklines=true}

\usepackage{langsci-basic}
\usepackage{langsci-optional}
\usepackage{langsci-lgr}
\usepackage{langsci-osl}
% \usepackage{./langsci/styles/langsci-lgr}
% \usepackage{./langsci/styles/langsci-osl}
% \usepackage{langsci-gb4e}

\usepackage{tikz}
\usetikzlibrary{patterns,calc}
\pgfdeclarepatternformonly{south east lines}{\pgfqpoint{-0pt}{-0pt}}{\pgfqpoint{3pt}{3pt}}{\pgfqpoint{3pt}{3pt}}{
    \pgfsetlinewidth{0.6pt}
    \pgfpathmoveto{\pgfqpoint{0pt}{3pt}}
    \pgfpathlineto{\pgfqpoint{3pt}{0pt}}
    \pgfpathmoveto{\pgfqpoint{.2pt}{-.2pt}}
    \pgfpathlineto{\pgfqpoint{-.2pt}{.2pt}}
    \pgfpathmoveto{\pgfqpoint{3.2pt}{2.8pt}}
    \pgfpathlineto{\pgfqpoint{2.8pt}{3.2pt}}
    \pgfusepath{stroke}}
    
\usepackage{stmaryrd}
\usepackage{wasysym}
\usepackage{multirow}
\usepackage{caption}
\usepackage{subcaption}
\usepackage{mathrsfs}
\usepackage{qtree}

\usepackage{linguex}


   %pminos do not split footnotes
% \interfootnotelinepenalty=10000 %Footnote in Laporte chapters has to be split SN


%\DeclareIndexNameFormat{default}{%
%\nameparts{#1}%
%\usebibmacro{index:name}%
%{\index[names]}%
%{\namepartfamily}%
%{\namepartgiveni}%
% {}% L1
% {}% L2
%{\namepartprefix}% generates spurious space L3
%{\namepartsuffix}% generates spurious space L4
%}

%  {\DeclareIndexNameFormat{default}{%
%     \usebibmacro{index:name}{\index[names]}{#1}{#3}{#5}{#7}}}

%\DeclareIndexNameFormat{default}{%
%  \usebibmacro{index:name}{\sindex[nom]}{#1}{#3}{#5}{#7}}

%\DeclareIndexNameFormat{default}{%
%  \usebibmacro{index:name}{\sindex[person]}{#1}{#3}{#5}{#7}}
%\DeclareIndexNameFormat{default}{%
%\nameparts{#1} \usebibmacro{index:name}{\sindex[person]]}{\namepartfamily}{‌​\namepartgiven}{\nam‌​epartprefix}{\namepa‌​rtsuffix}}

%\newcommand{\smiley}{:)}

%\renewbibmacro*{index:name}[5]{%
%\usebibmacro{index:entry}{#1}%
%{\iffieldundef{usera}{}{\thefield{usera}\actualoperator}\mkbibindexname{#2}{#3}{#4}{#5}}}

% \newcommand{\noop}[1]{}

%remove for final
%\overfullrule=1mm

\newcommand{\tobi}[2]}}
\renewcommand{\S}[1]{\tobi{#1}{\textsc{*}}}

% this volume references
% puts: [this volume]
% already defined: \citetv
%\newcommand{\citepv}[1]{(\citeauthor{#1} \citeyear*{#1} [this volume])}
\newcommand{\citealtv}[1]{\citeauthor{#1} \citeyear*{#1} [this volume]}

%parentheses around example number
\newcommand{\pref}[1]{(\ref{#1})}

% in-text examples

\newcommand{\lnex}[1]{\textit{#1}} %target lang word
\newcommand{\lnlit}[1]{(lit.: `#1')} %literal reading
\newcommand{\lnlat}[1]{(#1)} % latinization
\newcommand{\lntrans}[1]{`#1'} %translation
\newcommand{\lnexl}[2]%
{\lnex{#1}{} \lnlat{#2}} % ex with latinization
\newcommand{\lnexlat}[3]{\lnex{#1}{} \lnlat{#2}{} \lntrans{#3}} % ex with latinization and tranl.

%ch01
\newcommand{\co}[1]{\mbox{\textbf{#1}}}

%ch09

\newcommand{\cyrbulg}[1]{\begin{otherlanguage*}{bulgarian}#1\end{otherlanguage*}}


%ch10
\newcommand{\nlp}{{\small NLP}}
\newcommand{\mwe}{{\small MWE}}
\newcommand{\rae}{{\small RAE}}
\newcommand{\lvc}{{\small LVC}}
\newcommand{\pos}{{\small P}o{\small S}}
%\newcommand{\todo}[1]{ \textcolor{red}{#1} }

%\renewcommand{\labelenumi}{\theenumi}
%\ainamefmt{{vv}{ll}{, ff}{, jj}} % fullname

\newcommand{\biberror}[1]{{\color{red}#1}}

\newcommand{\osenovaitem}{--~}
   %% hyphenation points for line breaks
%% Normally, automatic hyphenation in LaTeX is very good
%% If a word is mis-hyphenated, add it to this file
%%
%% add information to TeX file before \begin{document} with:
%% %% hyphenation points for line breaks
%% Normally, automatic hyphenation in LaTeX is very good
%% If a word is mis-hyphenated, add it to this file
%%
%% add information to TeX file before \begin{document} with:
%% %% hyphenation points for line breaks
%% Normally, automatic hyphenation in LaTeX is very good
%% If a word is mis-hyphenated, add it to this file
%%
%% add information to TeX file before \begin{document} with:
%% \include{localhyphenation}
\hyphenation{
    Beck-man
    Ngu-yen
    back-chan-nel
    back-chan-nels
    mo-not-o-nous
    ste-reo-typ-i-cal
}

\hyphenation{
    Beck-man
    Ngu-yen
    back-chan-nel
    back-chan-nels
    mo-not-o-nous
    ste-reo-typ-i-cal
}

\hyphenation{
    Beck-man
    Ngu-yen
    back-chan-nel
    back-chan-nels
    mo-not-o-nous
    ste-reo-typ-i-cal
}

   \boolfalse{bookcompile}
   \togglepaper[5]%%chapternumber
}{}

%\pretocmd{\gll}{\def\eachwordone{\itshape}\def\eachwordtwo{\normalfont}}{}{}

\begin{document}
\maketitle

\section{Introduction} \label{Introduction-PanLilje}
To describe the physical location of one object (\textit{figure}) with respect to another object (\textit{ground}), languages commonly use two types of clauses (see \cite{Clark1978}, \cite{Freeze1992} and \cite{Hengeveld1992}, among others):

\begin{itemize}
  \item locative predications (\textsc{loc}), e.g. \textit{the book is on the table},
  \item locational-existential predications (\textsc{loc-ex}), e.g. \textit{there is a book on the table}.
\end{itemize}

When referring to both \textsc{loc} and \textsc{loc-ex} types of clauses, we use the term \textit{figure-ground predications}.\footnote{\citet{HaspelmathNonverbal} groups these two types of predications under the general term \textit{locational clause}.}

The first aim of this paper is to contribute to the research on figure-ground predications by focusing on the problem of distinguishing between \textsc{loc} and \textsc{loc-ex} clauses in languages where the \textsc{loc} vs. \textsc{loc-ex} contrast is not clearly reflected in their morphosyntax. There is a large number of such languages; for instance, \citet{Creissels2019} shows that more than half of the world’s languages lack a dedicated \textsc{loc-ex} construction such as, for example, \textit{there is} in English. Although the \textsc{loc} vs. \textsc{loc-ex} distinction is to some extent reflected in the morphosyntactic variation found in figure-ground predications (in particular, definiteness marking and word order), we will mostly address cases where there is no dedicated marking differentiating \textsc{loc} and \textsc{loc-ex} clauses.

The second aim of this paper is to bring more data from non-European low-resource languages to the discussion on figure-ground predications. We mainly focus on Gawarbati, an understudied Indo-Aryan language, but also discuss the languages of the surrounding area, namely the Hindu Kush-Karakoram region (henceforth, Hindu Kush). The comparison of Gawarbati with its geographical neighbours also allows us to contribute to the discussion on the diversity and development of \textsc{loc} and \textsc{loc-ex} clauses in an areal-typological perspective, cf. \citet{ZeitounEtAl1999} on Formosan languages, \citet{BentleyEtAl2015} on Italo-Romance varieties and \citet{Creissels2019Sudanic} on the languages of the Sudanic belt. The data for the study of figure-ground predications in Gawarbati come from a corpus of spoken language and are complemented by some directly elicited material. The data for the study of figure-ground predications in other Hindu Kush languages are mainly collected by using existing descriptive literature.

\largerpage
The research questions of the paper can be formulated more precisely as follows:
\begin{itemize}
  \item What types of figure-ground predications can be distinguished in Gawarbati, based on formal grounds?
  \item How do these types correspond to the semantic/pragmatic criteria for differentiating \textsc{loc} and \textsc{loc-ex} clauses cross-linguistically, as discussed in the literature?  (These criteria are described in \sectref{Theoretical-remarks}.)
  \item What areal patterns in figure-ground predications are found in the Hindu Kush region, and to what extent can the results of the Gawarbati study be generalised to other languages of the area? Can any of the observed patterns be explained in terms of language relatedness or language contact?
\end{itemize}

The structure of the paper is the following. \sectref{Theoretical-remarks} provides a theoretical background on the difference between \textsc{loc} and \textsc{loc-ex} clauses. \sectref{Background} introduces the Gawarbati language and other languages of the Hindu Kush. \sectref{Gawarbati} describes the study of figure-ground predications in Gawarbati. \sectref{Hindu-Kush} is dedicated to an areal survey. In \sectref{Conclusions}, the main findings of the paper are summarised and discussed from a theoretical and typological perspective.

\section{Theoretical remarks} \label{Theoretical-remarks}
In the literature, the precise difference between \textsc{loc} and \textsc{loc-ex} predications remains a matter of debate. \tabref{tab:approaches} summarises three common approaches to defining the \textsc{loc} vs. \textsc{loc-ex} contrast.

\begin{table}
\begin{tabularx}{\textwidth}{lQQ}
\lsptoprule
Parameter & \textsc{loc} & \textsc{loc-ex} \\
\midrule
Information structure & figure = topic & figure = focus \\
Definiteness & figure is definite & figure is indefinite \\
Perspectivisation & the Perspectival Center is the figure & the Perspectival Center is the ground \\
\lspbottomrule
\end{tabularx}
\caption{Approaches to differentiating \textsc{loc} and \textsc{loc-ex} predications}
\label{tab:approaches}
\end{table}

According to \citet{Koch2012}, the main criterion for differentiating between \textsc{loc} and \textsc{loc-ex} clauses is information structure. In \textsc{loc} clauses, the figure is the topic of the sentence, while in \textsc{loc-ex} clauses, the figure is the focus of the sentence.

An alternative approach, recently proposed by \citet{HaspelmathNonverbal}, suggests that the criterion for distinguishing \textsc{loc} and \textsc{loc-ex} predications is definiteness of the figure: when the figure is definite, it is a \textsc{loc} predication, and when the figure is indefinite, it is a \textsc{loc-ex} predication. 


The third influential view found in the literature suggests that the difference between \textsc{loc} and \textsc{loc-ex} predications is determined by perspectivisation, a parameter distinct from information structure or definiteness and also underlying, for example, the choice between active and passive voice (\cites[]{BorschevPartee2002}[]{ParteeBorschev2007}[49--50]{Creissels2019}). The key element of the perspective structure is the Perspectival Center, i.e. “the participant chosen as the point of departure for structuring the situation” \citep[124]{BorschevPartee2002}: in \textsc{loc} the Perspectival Center is the figure, in \textsc{loc-ex} the Perspectival Center is the ground.

Perspectivisation is a rather abstract notion that has to do with our perception and categorisation of the world, but, to some extent, it can be operationalised through the notion of presupposition. As noted by \citet{BorschevPartee2002}, an important property of the Perspectival Center is that it must be presupposed to exist, while the other participant does not have to be presupposed. The obligatory presupposition of existence for the Perspectival Center has implications for the formation of negative \textsc{loc} and \textsc{loc-ex} clauses. Since the presupposition cannot be cancelled under negation, in \textsc{loc} clauses it is not possible to negate the existence of the figure (the Perspectival Center): the scope of negation includes only the ground, cf. \textit{the book is not on the table}. By contrast, in \textsc{loc-ex} clauses the existence of the figure can be cancelled under negation, cf. \textit{there are no books on the table}.

Obviously, the three parameters presented in \tabref{tab:approaches} correlate with each other. For example, indefinite non-specific referents typically lack the presupposition of existence; topical nominals are usually definite. However, we also find examples of discrepancies between the parameters (\cites[]{ParteeBorschev2007}[47--50]{Creissels2019}[]{HaspelmathNonverbal}). For instance, the figure in \REF{ex:English} is definite but focal, the figure in \REF{ex:Russian} is indefinite but topical.

\ea\label{ex:English}
THE DOG is on the bed. \citep{HaspelmathNonverbal}
\z 

\ea\label{ex:Russian}
\langinfo{Russian}{Indo-European, [russ1263]}{based on \cite[150]{Paducheva2008}}\footnote{Slashes in this example indicate intonation contour.} \\
\gll Kreslo \textup{~/} na verande est’ \textup{~\textbackslash} . \\
chair \textsc{} on porch.\textsc{loc.sg}	be.\textsc{prs}  \\
\glt (If you are looking for chairs) ‘There is a chair on the porch.’ 
\z

In both examples, the ground seems to be associated with a greater presuppositionality of existence than the figure, although the Perspectival Center in affirmative clauses is not always perfectly clear and its annotation usually requires more context (see also discussion in \sectref{Annotation}).

This paper mainly follows the core vs. periphery approach, i.e. clauses which can be defined as \textsc{loc} and \textsc{loc-ex} according to all three parameters are considered core \textsc{loc} and \textsc{loc-ex} clauses, whereas all other clauses are considered peripheral. In the corpus study (\sectref{Corpus-study}), we do not immediately merge the three parameters into one single category. Instead, we consider them separately, so that the results of the study can be interpreted in the light of each one of the described approaches.

In the rest of the paper, we focus on the morphosyntactic variation in figure-ground predications and its relation to the \textsc{loc} and \textsc{loc-ex} distinction, using the data from Gawarbati and the languages of the surrounding region.

\section{Gawarbati and other languages of the Hindu Kush} \label{Background}
\subsection{Gawarbati and Gawarbati grammar}
Gawarbati [gawa1247] is an Indo-Aryan language spoken by approximately 20,000 people in a number of villages scattered along the Kunar River in the mountainous border area of Pakistan and Afghanis\-tan (Fazal Akbar, p.c.), in Pakistan’s Lower Chitral District, Khyber Pakhtunkhwa Province, and in Afghanistan’s Nari District, Kunar Province. The approximate location of the Gawarbati-speaking area can be seen in \figref{map} at the beginning of \sectref{Hindu-Kush:Areal}. According to the genealogical classification suggested in \citet{Bashir2003}, Gawarbati together with its two near-extinct varieties Shumashti [shum1235] and Grangali (in Afghanistan, [gran1245]) and the neighbouring language Dameli (in Pakistan, [dame1241]) constitute a Kunar group of Indo-Aryan, although the exact classification of Dameli remains uncertain (\cites[258]{Strand2001}[9--10]{Halfmann2022}). 

The available literature on Gawarbati grammar is limited to three papers \citep{Grierson1919, Morgenstierne1950, Kohistani1993}, based on first-hand data, and two short descriptions by \citet{Edelman1983} and \citet{Bashir2003}. The mentioned works together with the currently ongoing but not yet published corpus study of Gawarbati provide insight into the main morphosyntactic characteristics of the language, some of which are briefly discussed below.

Plural marking in Gawarbati is optional and occurs rather rarely in spontaneous speech. The case system includes (unmarked) Nominative, Oblique, Genitive/Ablative, Dative and Ergative/Instrumental cases. The Oblique case encodes direct objects, cf. \textit{heʦi\nobreakdash-a} [cow-\textsc{obl}] in \REF{ex:oblique}, as well as locative and temporal arguments. The Oblique is also the most frequent case used for nouns in postpositional phrases, cf. \textit{muʈʰa\nobreakdash-a} [tree-\textsc{obl}] in \REF{ex:oblique}. The numeral \textit{jak}/\textit{ja} ‘one’ (the distribution between the two variants is not clear), also illustrated in \REF{ex:oblique}, can be used as an indefinite article (see a more detailed discussion in \sectref{jak}).

\ea\label{ex:oblique}
\gll heʦi-a ja muʈʰa-a manzi genʈ-us\\
cow-\textsc{obl} \textsc{indf} tree-\textsc{obl} with/at tie-\textsc{pst.3sg} \\
\glt ‘[He] tied the cow to a tree.’\footnote{If not stated otherwise, all Gawarbati examples come from the corpus.}
\z

In terms of alignment typology, Gawarbati case marking displays a tripartite system, as seen in the distinct case marking of S, A and P in \REF{ex:intransitive}-\REF{ex:transitive}.

\ea\label{ex:intransitive}
\gll mon-a se mumaː ʥo pe-san bua \\
my-\textsc{m} \textsc{dist.sg[\textbf{nom}]} uncle[\textsc{\textbf{nom}}] \textsc{ptcl} sleep-\textsc{ptcp} be.\textsc{pst.3m.sg} \\
\glt ‘My uncle was sleeping.’
\z

\ea\label{ex:transitive}
\gll tuĩ asa rupai-a kenna aːn-u \\
you.\textsc{sg.\textbf{erg}} \textsc{prox.\textbf{obl}.sg} rupee-\textsc{\textbf{obl}} where.from bring-\textsc{pst.2sg} \\
\glt ‘Where did you bring this money from?’
\z

However, the TAM-based ergativity split and differential object marking often lead to neutralisation of these distinctions. The Ergative marking of A is limited to perfective TAM categories, whereas in imperfective TAM categories, A appears in the Nominative form. P is marked by the Oblique case only when it is definite; otherwise it remains in the Nominative form. In verbal agreement, Gawarbati follows a Nominative-Accusative pattern, thus showing different alignment types in head marking and dependent marking.

The TAM system of Gawarbati is composed of both synthetic and periphrastic forms and includes some modal constructions. Non-verbal predications are formed with a copula which cannot be left out in any of the tenses (see also \cite[262]{Bashir2023}). The arguments of predicates, by contrast, are freely omitted if they can be inferred from the previous context.

The unmarked basic word order in Gawarbati is SOV, but it is flexible and heavily dependent on information structure. The topical constituents appear in the clause-initial position, and focus is strongly associated with the immediately preverbal position. For example, in \REF{ex:IS} the topic is \textit{saudaː} ‘goods’. It is moved to the beginning of the sentence, while the preverbal position is occupied by the focal argument \textit{lawans-ana} ‘from Drosh’.

\ea\label{ex:IS}
\gll \textup{[}saudaː\textup{]}\textsubscript{\textup{topic}} ama \textup{[}lawans-ana\textup{]}\textsubscript{\textup{focus}} aːn-imek \\
goods.\textsc{nom/obl} we Drosh-\textsc{gen/abl} bring-\textsc{prs.1pl}\\
\glt (From where do you bring the stuff for your shop?) ‘We bring the stuff from Drosh.’
\z

Gawarbati employs a few particles such as \textit{ʥo} in \REF{ex:intransitive} above, but none of them is uncontroversially associated with any information structure category.

\subsection{Gawarbati documentation and corpus building}

The data used in this study come from a spoken corpus which is currently being compiled. It is part of a larger, ongoing project, documenting Gawarbati in close collaboration with the speaker community. The project was launched in 2021 and consists at the time of writing of a Swedish-Pakistani 6-member team:\footnote{Note that there are 7 people involved, whereby the team coordinator position was filled by one person initially then taken over by another.} Henrik Liljegren (principal investigator and project leader), Anastasia Panova (research assistant/doctoral student), Fazal Hadi (local team coordinator), Allauddin Torwali (deputy local team coordinator), Fazal Akbar (language consultant, mainly responsible for lexicographic data collection and processing), Nasirullah Nasir (language consultant, mainly responsible for recording and processing meta data) and Abdullah Soan (language consultant, mainly responsible for annotation of recorded data). The activities in Pakistan take place under the auspices of the Forum for Language Initiatives (FLI), a regional language resource centre founded in 2002, based in Islamabad \citep[72--77]{LiljegrenAkhunzada2017}. The aim of this project is to collect, annotate and analyse language data; to produce a typologically informed grammatical description; to compile a lasting record of the language in the form of an annotated audio and video corpus (eventually based on at least 20 hours of recordings); and to build a lexical database.

The version of the corpus used in the study of figure-ground predications comprised  40 texts (=recorded events) altogether, recorded in 2021--2022, in the village Arandu in Pakistan, then segmented, transcribed and given an utterance-by-utterance free translation, first into Urdu (Indo-European, [urdu1245]), and then into English, by the team members affiliated with FLI. Subsequently, it was morphologically analysed and glossed by Anastasia Panova, based on the former annotation.\footnote{The translations presented in the paper rely more on glosses than on translations extracted from the corpus, as the latter were often not precise enough.} It consists of about 38,000 tokens, and corresponds to four hours of recordings.

\subsection{The genealogical and areal context of Gawarbati}

Gawarbati and the other languages of the Kunar group (mentioned above) are included into a cluster of Indo-Aryan, often referred to as Dardic. However, it should be noted that “Dardic” or “Dardic languages”, since Morgenstierne’s groundbreaking work (\citeyear[138--139]{Morgenstierne1961}), should be understood as geographic or areal labels rather than genealogical terms. It is applied to the many Indo-Aryan languages spoken in the Hindu Kush region, a mountainous territory comprising northeastern Afghanistan, northernmost Pakistan and the northern parts of Kashmir and Ladakh. To stress the areal affinities between these languages rather than enforcing a misguided idea of a genealogical designation or intermediate node on the Indo-Aryan tree, Liljegren (\citeyear[13--16]{Liljegren2016}) has suggested the term “Hindu Kush Indo-Aryan (HKIA)” to replace Dardic. At least 25 distinctly different Indo-Aryan languages have been at home in this particular region for an extended period of time. Those can in their turn, following Bashir (\citeyear[824--825]{Bashir2003}) or Strand (\citeyear[258]{Strand2001}), be grouped into smaller relatedness clusters or groups, such as Pashai, Kunar/Pech, Chitral, Kohistani, Shina and Kashmiri. In addition to languages with a longstanding presence in the region, we also find a number of Indo-Aryan varieties that have closer genealogical and historical ties with languages of the Indo-Pakistani plains \citep[110]{Liljegren2017}.  

Apart from Indo-Aryan, which is the numerically dominant group of the Hindu Kush, five more phylogenetic groups are represented in this region: Iranian, Nuristani, Sino-Tibetan, Turkic and the language isolate Burushaski [buru1296]. As shown by Liljegren (\citeyear[217--221]{Liljegren2020}), the languages of the Hindu Kush have phonological and lexico-grammatical features that are shared by languages forming a core of the Hindu Kush, regardless of phylogenetic identity, such as large sets of retroflex consonants, utterance-final question markers and vigesimal numeral bases. In addition to that, there is a significant level of subareality apparent throughout the Hindu Kush, with more local feature clustering in geographical subregions \citep[221--225]{Liljegren2020}. The latter appears to be particularly prominent in the western part of the Hindu Kush, related to the fact that the ethnolinguistic communities in Nuristan, situated in the remote mountainous parts of northeastern Afghanistan and the immediately surrounding areas, did not become part of the larger Islamic world until very recently. Until then, these communities formed a relatively tight-knit cultural unit vis-à-vis the already Islamised outside world, linguistically reflected in e.g. the presence of retroflex/rhotacised vowels, a certain type of kin-term polysemy pattern, pronominal kinship suffixes, geomorphicity, and bisyndetic contrast marking \citep{DiCarlo2011, HeegårdMørch2004, HeegårdLiljegren2018, Liljegren2021, LiljegrenSvärd2017}. Gawarbati finds itself at the outskirts of this subarea, thus sharing some, but not all, of those characteristics. 

All of this indicates that language contact has played an important role, both in the larger region, and in the immediate surroundings of Gawarbati. Hence, studying and paying attention to the areal-typological context is essential to understand the development of particular linguistic phenomena in the Hindu Kush languages, a matter we will have reason to return to in further detail in \sectref{Hindu-Kush}.


\section{Figure-ground predications in Gawarbati} \label{Gawarbati}

\subsection{The basic morphosyntax of \textsc{loc} or \textsc{loc-ex} predications} \label{basic-morphosynt}

Direct elicitation of \textsc{loc} or \textsc{loc-ex} predications from English (or Urdu) to Gawarbati gives quite straightforward results. Both types of clauses involve a nominal phrase, a locative phrase and the copula. The difference between predications lies primarily in word order.\footnote{Word order should here, and henceforth, be understood as the order of syntactic constituents.} We can see that in the \textsc{loc} clause \REF{ex:elicited-boy-loc}, the figure \textit{ʈekura} ‘boy’ precedes the ground \textit{baːɣ} ‘garden’, whereas in the \textsc{loc-ex} clause \REF{ex:elicited-boy-locex}, the figure follows the ground. In addition, the figure in the \textsc{loc-ex} clause \REF{ex:elicited-boy-locex} is modified by the marker \textit{jak} which seems functionally similar to the indefinite article in the English stimulus.

\ea \label{ex:elicited-boy}
\ea \label{ex:elicited-boy-loc}
\gll ʈekura baːɣ-a tʰana\\
 boy garden-\textsc{obl} be.\textsc{prs.3m.sg}\\
\glt ‘The boy is in the garden.’ [elicited]\\
\ex \label{ex:elicited-boy-locex}
\gll baːɣ-a jak ʈekura tʰana\\
garden-\textsc{obl} \textsc{indf} boy be.\textsc{prs.3m.sg}\\
\glt ‘There is a boy in the garden.’ [elicited]\\
\z
\z

Despite the different positions in the clause, the figure seems to be the subject in both types of predications, as it appears in the Nominative case and controls the copula agreement, cf. \REF{ex:elicited-boy} and \REF{ex:elicited-mosque} where the copula agrees with the figure in the masculine gender and in the feminine gender, respectively.


\ea \label{ex:elicited-mosque}
\ea \label{ex:elicited-mosque-loc}
\gll uzal-i ʥamaːit laːm-an-a manzi tʰini\\
white-\textsc{f} mosque village-\textsc{gen-m} inside be.\textsc{prs.3f.sg}\\
\glt ‘The white mosque is in the middle of the village.’ [elicited]\\
\ex \label{ex:elicited-mosque-locex}
\gll laːm-an-a manzi jak ɖal ʥamaːit tʰini\\
village-\textsc{gen-m} inside \textsc{indf} big mosque be.\textsc{prs.3f.sg}\\
\glt ‘There is a big mosque in the middle of the village.’ [elicited]\\
\z
\z

Negative \textsc{loc-ex} clauses are formed with the default negative particle \textit{na} \REF{ex:negative-elicited}. In addition, we find one more possible strategy of deriving negative \textsc{loc-ex} clauses in the corpus, namely with the negative predicative particle \textit{nahn-} which shows agreement with the \figref{ex:negative-nahn}.

\ea\label{ex:negative-elicited}
\gll baːɣ-a kara xalaq na tʰanet \\
garden-\textsc{obl} any people \textsc{neg} be.\textsc{prs.3pl} \\
\glt ‘There are no people in the garden.’ [elicited]
\z

\ea\label{ex:negative-nahn}
\gll ite waːri xalaq nahn-et \\
here other people \textsc{neg}.be-\textsc{3pl} \\
\glt ‘There are no other people here.’
\z

All of the examples provided so far included the Present tense form of the copula. \tabref{tab:copula} shows the full paradigm of the copula in the Present as well as in the Past tense (copular forms in other tenses are also attested in the corpus but they rarely occur in \textsc{loc} and \textsc{loc-ex} clauses). In the Past tense, the stem is \textit{b-}, see also \REF{ex:copula-past}. The original meaning of the verb stem \textit{b-} is ‘become’, but synchronically the forms of \textit{tʰan-}/\textit{tʰin-} in the Present and the forms of \textit{b-} in the Past constitute subparts of one and the same paradigm.\footnote{The stem \textit{tʰan-}/\textit{tʰin-} also has a shortened variant \textit{tʰ-} to which regular Present tense affixes attach, cf. \REF{ex:perspective-loc} below. Those compete paradigmatically with the forms of \textit{tʰan-}/\textit{tʰin-}, although they are much less frequent. }

\begin{table}
\begin{tabularx}{\textwidth}{XXXXXXl}
\lsptoprule
 & \multicolumn{3}{c}{Present} & \multicolumn{3}{c}{Past} \\
 \cmidrule(rl){2-4}\cmidrule(rl){5-7}
 & \textsc{sg.m} & \textsc{sg.f} & \textsc{pl}  & \textsc{sg.m} & \textsc{sg.f} & \textsc{pl}\\
\midrule
1 & \textit{tʰanem} & \textit{tʰinim} & \textit{tʰanek} & \textit{bom} & \textit{bum} & \textit{bok} \\
2 & \textit{tʰanes} & \textit{tʰinis} & \textit{tʰaneu} & \textit{bos} & \textit{bus} & \textit{bo} \\
3 & \textit{tʰana} & \textit{tʰini} & \textit{tʰanet} & \textit{bua} & \textit{bui} & \textit{bot} \\
\lspbottomrule
\end{tabularx}
\caption{The Gawarbati copula in the Present and Past tenses}
\label{tab:copula}
\end{table}


\ea\label{ex:copula-past}
\gll madina-an-i rijaːsat-a xo insaːf bui \\
Madinah-\textsc{gen-f} state-\textsc{obl} \textsc{ptcl} justice be.\textsc{pst.3f.sg} \\
\glt ‘There was justice in the state of Madinah.’
\z

The copula is also used in possessive predications, which are structurally similar to \textsc{loc-ex} with the exception of case encoding — the possessor is usually encoded by the Genitive case \REF{ex:possessive}.

\ea\label{ex:possessive}
\gll tasan-i sat zu bot \\
\textsc{dist.gen.sg-f} seven daughter be.\textsc{pst.3pl} \\
\glt ‘He had seven daughters.’
\z


Although copula-based clauses represent the primary strategy for describing figure-ground relations, there is arguably an alternative strategy available for \textsc{loc} predications that does not involve the copula. In the corpus, we find recurrent cases of the verb \textit{niɕ-} ‘sit (down)’ in \textsc{loc} predications: sometimes it is intended as a posture verb but often used in a more abstract sense ‘be (located in)’. For example, in \REF{ex:sitting}, the interviewer says that the interviewee and the interviewer himself are “sitting” at the upper mill. However, it is clear from the video recording that at the moment of saying that, both participants are in fact standing in front of the entrance to the watermill. This contradiction shows that the verb \textit{niɕ-} can indicate physical presence without specifying posture.

\ea\label{ex:sitting}
\gll antar ʥanɬ-a niɕ-isan tʰanek \\
upper water.mill-\textsc{obl} sit-\textsc{ptcp} be.\textsc{prs.1pl} \\
\glt 'We are at the upper mill.' (lit. [We] are sitting at the upper mill.)
\z

It is noteworthy, however, that when speakers were asked to translate examples like \REF{ex:sitting} into English without context, they consistently favoured the literal translation with ‘sit (down)’ over the more abstract translation with ‘be (located in)’.

Another semantic extension of \textit{niɕ-} ‘sit (down)’, less relevant for the present paper but confirming the gradual semantic bleaching of the verb, is illustrated in \REF{ex:living}, where it is used in the sense of ‘live’.

\ea\label{ex:living}
\gll time tine niɕ-isan bot \\
\textsc{dist.pl} there sit-\textsc{ptcp} be.\textsc{pst.3pl} \\
\glt (Allah gave them a good place) ‘They started living there.’
\z

Having outlined how figure-ground predications are structured, we will now provide a usage-based account of these types of clauses.

\subsection{A corpus study: towards a bottom-up typology of figure-ground predications} \label{Corpus-study}

\subsubsection{Annotation} \label{Annotation}

The aim of the corpus study was to identify the types of figure-ground predications in Gawarbati using natural discourse data and then examine how these types relate to the \textsc{loc} vs. \textsc{loc-ex} contrast.

After a manual check of all the sentences in the corpus, the clauses describing temporary or permanent relations between figure and ground were selected for further analysis. We did not include clauses involving the negative predicative particle \textit{nahn-} as dedicated negative existential constructions often behave very differently across languages \citep{Veselinova2013}. We also excluded clauses with with predicates other than the copula, i.e. with the verb \textit{niɕ-} ‘sit (down)’ (see \sectref{basic-morphosynt}), clauses with omitted subject, as in \REF{ex:no-subject}, and clauses describing abstract spatial relations and time as space metaphors, as in \REF{ex:temporal}.

\ea\label{ex:no-subject}
\gll nori te muʈʰa-a manzi tʰanem \\
now \textsc{dem.obl.sg} tree-\textsc{obl} inside be.\textsc{prs.1m.sg} \\
\glt ‘[I] am now inside this tree.’
\z

\ea\label{ex:temporal}
\gll amn-a te laːm-a ja ɕil-i wax bui ze \\
our-\textsc{m} \textsc{dem.obl.sg} village-\textsc{obl} \textsc{indf} such-\textsc{f} time be.\textsc{pst.3f.sg} \textsc{comp} \\
\glt ‘There was a time in our village when…’
\z

The final sample included 87 examples, of which 74 were affirmative clauses and 13 were negative clauses.

All the examples included in the sample were annotated for a number of parameters.\footnote{The annotated dataset is available at \url{https://osf.io/bq98g/?view_only=a61f393ae43649f48a8501a8fe2fd735}.} First, we analysed formal variation across the clauses, specifically variation in word order and the presence of the marker \textit{ja}/\textit{jak} — these were the main features which varied across the examples on the structural level. Second, we also annotated the sample according to the parameters which have been discussed as relevant with respect to the \textsc{loc} vs. \textsc{loc-ex} contrast in the previous literature, namely definiteness of the figure, information structure, and perspectivisation.

In the annotation of the definiteness of the figure, three categories were distinguished — definite, indefinite specific and indefinite non-specific, the latter being associated primarily with negative contexts like \textit{there are no books on the table}.

For encoding information structure, we used topic vs. comment and focus vs. background distinctions, following \citet{Krifka2008} and \citet{Aissen2023} as the main references.\footnote{In the annotation, we also differentiated between plain and contrastive versions of topic and focus, but for the purposes of this paper these distinctions were merged.}  Both figure and ground were encoded as being (a part of) one or several information structure categories. For example, in \REF{ex:topic-comment}, the figure is the topic, and the ground is part of the comment, while in \REF{ex:background-focus}, the figure is both the focus and part of the comment, and the ground is both the topic and part of the background.

\ea\label{ex:topic-comment}
\gll \textup{[}ama\textup{]}\textsubscript{\textup{topic}} \textup{[}dam haranu-a tʰanek\textup{]}\textsubscript{\textup{comment}} \\
we Dam Arandu-\textsc{obl} be.\textsc{prs.1pl} \\
\glt ‘We are in Arandu Dam.’
\z

\ea\label{ex:background-focus}
\glll \textup{[}\textit{ate} \textit{hadika-a}\textup{]}\textsubscript{\textup{topic}} \textup{[}\textit{sik-aːn} \textit{bot}\textup{]}\textsubscript{\textup{comment}} \\
\textup{[}\textit{ate} \textit{hadika-a}\textup{]}\textsubscript{\textup{background}} \textup{[}\textit{sik-aːn}\textup{]}\textsubscript{\textup{focus}} \textup{[}\textit{bot}\textup{]}\textsubscript{\textup{background}} \\
here place-\textsc{obl} Sikh-\textsc{pl} be.\textsc{pst.3pl} \\
\glt (Actually, it was not Godibar.) ‘Sikhs used to live here.’
\z

Some sentences, especially at the beginning of the recorded events, were annotated as thetic (comment-only), because none of their parts were introduced, either explicitly or implicitly, in the previous context.

Finally, to identify the Perspectival Center, we analysed the presupposition vs. assertion structure of the sentences. As mentioned in \sectref{Theoretical-remarks}, the Perspectival Center is normally presupposed to exist, so if the figure lacks a presupposition of existence, the Perspectival Center is usually the ground. The clearest cases are negative clauses like \REF{ex:perspective-neg}, where the existence of the figure is negated. Hence, the Perspectival Center is the ground.

\ea\label{ex:perspective-neg}
\gll \textup{[}ate dameɽ-a\textup{]}\textsubscript{\textup{PerspectivalCenter}} iskul na tʰana \\
here Damil-\textsc{obl} school \textsc{neg} be.\textsc{prs.3m.sg} \\
\glt ‘There is no school in Damil.’
\z

In affirmative clauses, the figure was encoded as not having a presupposition of existence when the previous context was insufficient to assume that it existed. For example, in \REF{ex:perspective-aff}, although the general subject of the utterance fragment — the earthquake — is clear from the previous context, the existence of the cave cannot be assumed. Therefore, the cave cannot be a Perspectival Center.

\ea\label{ex:perspective-aff}
\gll \textup{[}ite aman-a moŋgoldam-an-a rataj ite hadika-a\textup{]}\textsubscript{\textup{PerspectivalCenter}} bi ɖal ɣaːr tʰana\\
here our-\textsc{m} Mongoldam-\textsc{gen-m} on here place-\textsc{obl} \textsc{add} big cave be.\textsc{prs.3m.sg}\\
\glt (The places of worship were destroyed by the volcano. There are still large black rocks lying here. There is also an earthquake mountain here.) ‘There is also a large cave here in Mongoldam.’ (There is a big rock, and hot air comes from inside.)
\z

The opposite situation, i.e. where the figure is annotated as a Perspectival Center, is illustrated in \REF{ex:perspective-loc}. In such examples both the figure and the ground can be presupposed to exist, or the ground can be associated with a lesser presuppositionality of existence. This can be compared with the following example from Partee and Borschev (\citeyear[157]{ParteeBorschev2007}), in which the existence of the ground is cancelled: \textit{The students were not at the concert. There was no concert.}

\ea\label{ex:perspective-loc}
\gll \textup{[}time\textup{]}\textsubscript{\textup{PerspectivalCenter}} te laːm-a tʰ-imet ʥo \\
\textsc{dist.pl} \textsc{dem.obl.sg} village-\textsc{obl} be-\textsc{prs.3pl} \textsc{ptcl} \\
\glt (These people come to Arandu. When they come to Arandu,) ‘they are in this village because\dots’
\z

It should be noted, however, that the perspective structure in affirmative sentences, generally, was the most difficult to annotate. The results of the annotation of the Perspectival Center should therefore be taken with caution.

Summarising the results of annotating definiteness, information structure and perspectivisation, we conclude that the groups of the examples identified by different parameters largely overlap. The diagram presented in \figref{combinations}\footnote{The diagram was created with R \citep{R2022} package \textit{ggforce} \citep{Pedersen2024}.} shows the proportion of the examples which can be identified as \textsc{loc} and \textsc{loc-ex} based on three parameters together (the red and green parts respectively), and what proportion of the examples has to be left out as instances of the periphery of the domain (the blue part).

\begin{figure}
\includegraphics[width=0.8\textwidth]{figures/PanovaLiljegren-diagram.png}
\caption{Combinations of annotated features in the sample and their respective proportions}
\label{combinations}
\end{figure}

Although the number of examples that ended up in the periphery might seem too big to represent the periphery (they constitute approximately one third of the sample), \figref{combinations} nevertheless shows that it is possible to identify core \textsc{loc} and \textsc{loc-ex} types of figure-ground predications on purely semantic/pragmatic grounds. Examples of clauses which can be considered \textsc{loc} and \textsc{loc-ex} according to all three parameters are provided in \REF{ex:core-loc} and \REF{ex:core-locex}, respectively. 

\ea\label{ex:core-loc}
\gll se baloʨistaːn-a bua \\
\textsc{dist.sg} Balochistan-\textsc{obl} be.\textsc{pst.3m.sg} \\
\glt ‘He was in Balochistan.’
\z

\ea\label{ex:core-locex}
\gll tasa manzi hast-a gila bua \\
\textsc{dist.obl.sg} inside hand-\textsc{obl} thing be.\textsc{pst.3m.sg} \\
\glt ‘There was something in his hand.’
\z


\REF{ex:definite-focal}--\REF{ex:topical-nonexist} are examples of discrepancies between the three parameters, and thus they are annotated as peripheral. In \REF{ex:definite-focal}, the figures are definite but focal; in \REF{ex:topical-nonexist}, the figure in the second clause is topical but lacks a presupposition of existence (see also \REF{ex:topic-not-presupposed} for the same type of discrepancy).

\ea\label{ex:definite-focal}
\gll haranu-a asan-a aːma tʰana dokalaːm-a asan-i taːnu sasi tʰini \\
Arandu-\textsc{obl} \textsc{prox.gen.sg-m} house be.\textsc{prs.3m.sg} Dokalam-\textsc{obl} \textsc{prox.gen.sg-f} self’s sister be.\textsc{prs.3f.sg} \\
\glt ‘In Arandu was his house, (whereas) in Dokalam was his sister's (house).’
\z

\ea\label{ex:topical-nonexist}
\gll te ki ki ɕai tʰanet time tine na tʰanet \\
here what what thing be.\textsc{prs.3pl} \textsc{dist.pl} there \textsc{neg} be.\textsc{prs.3pl} \\
\glt ‘What things exist here (in Pakistan) but do not exist there (in Afghanistan)?’ (lit. ‘What things are here that are not there?’)
\z
	 
The core vs. periphery approach confirms that \textsc{loc} and \textsc{loc-ex} categories as defined through definiteness, information structure and perspective structure alike are in principle relevant to the Gawarbati grammar and discourse. As a next step, we checked whether there are any correlations between the parameters relevant to the \textsc{loc} vs. \textsc{loc-ex} contrast and the formal variation present in the examples, i.e. variation in definiteness marking and word order.


\subsubsection{Results: the marker \textit{ja}/\textit{jak}} \label{jak}

The sample contains only eight examples featuring the marker \textit{ja}/\textit{jak} which modifies the figure. \tabref{tab:jak-indef} shows how the presence of \textit{ja}/\textit{jak} corresponds to the actual (in)definiteness status of the figure.

\begin{table}
\begin{tabularx}{0.8\textwidth}{l YY}
\lsptoprule
 & \textit{ja}/\textit{jak} & ∅ \\
\midrule
the figure is definite & 0 & 34 \\
the figure is indefinite specific & 8 & 22  \\
the figure is indefinite non-specific & 0 &  23 \\
\lspbottomrule
\end{tabularx}
\caption{The presence of \textit{ja}/\textit{jak} and definiteness of the figure in the Gawarbati sample}
\label{tab:jak-indef}
\end{table}


In all examples containing the marker \textit{ja}/\textit{jak}, the figure is indefinite specific, cf. \REF{ex:jak}.


\ea\label{ex:jak}
\gll mo manzi ite ja baːja tʰana \\
I.\textsc{obl} with here \textsc{indf} brother be.\textsc{prs.3m.sg} \\
\glt ‘There is a brother with me here.’ (His name is Hisham.)
\z

It seems that the marker \textit{ja}/\textit{jak}, apparently grammaticalised from the numeral \textit{jak} ‘one’, is indeed used only in indefinite specific contexts, thus not covering the whole indefinite domain (cf. the similar analysis of the article \textit{áa(k)} in related Palula by Becker (\citeyear[132, 266--267]{Becker2021})). Moreover, \textit{ja}/\textit{jak} often does not mark the figure even when it is indefinite specific, due to its incompatibility with plural referents and mass nouns. Finally, we find examples where \textit{ja}/\textit{jak} is absent, although there are no apparent reasons for it to be omitted, as in \REF{ex:jak-absent}.
	
\ea\label{ex:jak-absent}
\gll maːlai sudi bui \\
below ladder be.\textsc{pst.3f.sg} \\
\glt ‘There was a ladder below.’
\z

These observations indicate that the marker \textit{ja}/\textit{jak} is not a fully grammaticalised indefinite article. Although its presence implies the indefinite status of the figure and, subsequently, the \textsc{loc-ex} status of the clause (if we use definiteness to distinguish between \textsc{loc} and \textsc{loc-ex}), its absence does not imply anything with respect to the \textsc{loc} vs. \textsc{loc-ex} contrast.

Interestingly, the eight examples where \textit{ja}/\textit{jak} is found are also quite homogeneous in terms of perspective and information structures. In all of these cases, the figure is part of the comment, while the ground is usually the topic. Furthermore, the Perspectival Center in these examples is always the ground. Thus, technically the presence of \textit{ja}/\textit{jak} in our sample predicts the \textsc{loc-ex} status of the predication, regardless of the criteria we choose to distinguish between \textsc{loc} and \textsc{loc-ex} types of figure-ground constructions.

\subsubsection{Results: word order}

Overall, in the clauses included in the sample, the order of the figure (F), the ground (G) and the verb (V) follows one of three patterns: in 25 of the examples, the word order is F–G–V; in 59, it is G–F–V; and three of the examples display the order F–V–G. Word order is often motivated by the mechanism of anchoring, i.e. the tendency to start the utterance with a reference point and then relate the rest of the information to it (in the literature, anchoring has been primarily discussed for possessive constructions, see, e.g., \cite{Koptjevskaja-Tamm2002}, but it can also be applied to the clause level). Within this view, we expect participants which are definite, topical and presupposed to serve as better anchors, and therefore would appear at the beginning of the clause.

\tabref{tab:wo-def} shows how the order of the figure, the ground and the verb corresponds to definiteness of the figure.

\begin{table}
\begin{tabularx}{\textwidth}{l YYY}
\lsptoprule
 & F–G–V & G–F–V & F–V–G \\
\midrule
the figure is definite & 17 & 16 & 1 \\
the figure is indefinite specific & 4 & 26 & 0 \\
the figure is indefinite non-specific & 4 & 17 & 2 \\
\lspbottomrule
\end{tabularx}
\caption{Word order and definiteness of the figure in the Gawarbati sample}
\label{tab:wo-def}
\end{table}

An indefinite figure tends to occupy the second position, but a definite figure occurs almost as often at the beginning of the clause as in second position.

The correspondence between word order and information structure is shown in \tabref{tab:wo-io}.

\begin{table}
\begin{tabularx}{\textwidth}{X rrr}
\lsptoprule
 & F–G–V & G–F–V & F–V–G \\
\midrule
figure ∈ topic (+background); ground ∈ comment (+focus) & 13 & 0 & 0 \\
ground ∈ topic (+background); figure ∈ comment (+focus) & 6 & 51 & 2 \\
other & 6 & 8 & 1 \\
\lspbottomrule
\end{tabularx}
\caption{Word order and information structure in the Gawarbati sample}
\label{tab:wo-io}
\end{table}

As already mentioned in \sectref{basic-morphosynt}, the relationship between word order and information structure in Gawarbati manifests itself in the topic-first principle and the existence of a preverbal focus position. Both of these patterns are observed in figure-ground predications. The figure and the ground tend to occur at the beginning of the clause when they are topical, and in preverbal position when they are focal. This is illustrated in \REF{ex:contrastive-topic}--\REF{ex:focus}, provided with a partial annotation of information structure.

\ea\label{ex:contrastive-topic}
\gll \textup{[}gronɖ\textup{]}\textsubscript{\textup{topic}} \textup{[}lau duraːje\textup{]}\textsubscript{\textup{focus}} tʰini \\
ground very far be.\textsc{prs.3f.sg} \\
\glt (Why don't you play on the ground? Why do you play on the road?) ‘The ground is far away.’ (The road is near.)
\z


\ea\label{ex:focus}
\gll ãː ki xabar ze \textup{[}te\textup{]}\textsubscript{\textup{topic}} \textup{[}me\textup{]}\textsubscript{\textup{focus}} tʰaneu \\
I what news \textsc{comp} here you.\textsc{pl} be.\textsc{prs.2pl} \\
\glt (I thought it was a closed area, then) ‘I found out that you were here.’
\z

In line with the principles described above, there are no examples with a topical figure in the preverbal position. At the same time, surprisingly, the sample includes six examples with a topical ground immediately preceding the verb. For instance, in \REF{ex:topic-before-the-verb}, the ground \textit{tine} ‘there’ refers to the place discussed in the previous context and functions as a plain aboutness topic, but it nevertheless follows the figure which, in turn, presents a new piece of information about the ground.


\ea\label{ex:topic-before-the-verb}
\gll \textup{[}lutr-a sum-naːm\textup{]}\textsubscript{\textup{comment}} \textup{[}tine\textup{]}\textsubscript{\textup{topic}} \textup{[}tʰanet\textup{]}\textsubscript{\textup{comment}} \\
red-\textsc{m} soil-\textsc{pl} there be.\textsc{prs.3pl} \\
\glt (…whenever they dig, in that place they [long pipes] come out.) ‘There is a lot of red soil there.’ (They brought these pipes, and they used to make pots.)
\z


There is a possibility that the intended information structure, in examples like \REF{ex:topic-before-the-verb}, is expressed by intonation rather than by word order, but a systematic investigation of intonation in Gawarbati is outside the scope of this study.


Regarding the examples showing “other” types of information structure, several of them were grouped separately because of the presence of a temporal argument serving as the topic of the clause, cf. \REF{ex:temporal-topic}. Consequently, both the figure and the ground in such examples are included in the comment.


\ea\label{ex:temporal-topic}
\gll au laka \textup{[}aːinda-a ʂaʦi\textup{]}\textsubscript{\textup{topic}} \textup{[}tu ate tʰanes\textup{]}\textsubscript{\textup{comment}} \\
and like future-\textsc{obl} for you.\textsc{sg} here be.\textsc{prs.2sg} \\
\glt (A: We haven’t seen such good doctors who would come and work here anymore. After him, we have had only young people. B:) ‘And, like, in the future, you will be here.’
\z

In several examples, the verb represents a separate information structural unit, cf. \REF{ex:focused-verb} where the verb is focused and contrasted with the negated verb in the next clause. In such examples, both the figure and the ground belong to the background part. 

\ea\label{ex:focused-verb}
\gll \textup{[}aman-i laːm-a primeri zanaːna sikul\textup{]}\textsubscript{\textup{background}} \textup{[}tʰanet\textup{]}\textsubscript{\textup{focus}} \\
our-\textsc{f} village-\textsc{m} primary woman school be.\textsc{prs.3pl} \\
\glt ‘In our village, the primary girl schools exist,’ (while the middle high schools do not.)
\z


In addition, a few examples from the “other” category are thetic sentences.


It is also shown in \tabref{tab:wo-io} that the ground occurs after the verb in three examples. On the one hand, the clause-final occurrences of the ground can be after-thoughts, i.e. fragments that do not necessarily share the syntactic and information structures with the rest of the clause. On the other hand, it is possible that Gawarbati has a certain information structure (sub-)category associated with the post-verbal position. The latter analysis has previously been proposed for Urdu. According to \citet{ButtKing1996}, the post-verbal position in Urdu is reserved for not-new and non-prominent information. Likewise, in Gawarbati the constituents that come after the verb normally convey information that can be reconstructed from the context. For instance, in \REF{ex:post-verb}, the fact that the speaker is talking about the Arandu valley is evident from the part of the discourse preceding the example (not provided here due to space restrictions).


\ea\label{ex:post-verb}
\gll biʥli-an-a sahulat tʰana gal faːr \\
electricity-\textsc{gen-m} facilities be.\textsc{prs.3sg.m} valley yonder \\
\glt ‘There is electricity in the valley.’
\z

To sum up, the word order in figure-ground predications generally follows the information structure, although there are exceptions. This observation, in turn, suggests that if we choose information structure as the criterion for distinguishing \textsc{loc} and \textsc{loc-ex} clauses, word order can be considered a formal expression of the typology of figure-ground constructions. At the same time, since there are more than two possible information structures and more than two possible word order patterns available in figure-ground predications, the resulting typology must be significantly more complex than simply a \textsc{loc} vs. \textsc{loc-ex} dichotomy.


Finally, moving on to the correspondences between word order and perspective structure, we observe a tendency to start the clause with the Perspectival Center, cf. \tabref{tab:wo-perspective}.

\begin{table}
\begin{tabularx}{\textwidth}{l YYY}
\lsptoprule
 & F–G–V & G–F–V & F–V–G \\
\midrule
Perspectival Center: figure & 16 & 5 & 1 \\
Perspectival Center: ground & 9 & 54 & 2 \\
\lspbottomrule
\end{tabularx}
\caption{Word order and perspective structure in the Gawarbati sample}
\label{tab:wo-perspective}
\end{table}

For example, in \REF{ex:ground-first}, the ground \textit{tine} ‘there’ has the presupposition of existence, while the figure does not, and therefore the clause starts with the ground.


\ea\label{ex:ground-first}
\gll \textup{[}tine\textup{]}\textsubscript{\textup{PerspectivalCenter}} zaːtak bot \\
there young.people be.\textsc{pst.3pl} \\
\glt (I saw the defense checkpoint near the bridge.) ‘There were boys there.’
\z

A closer examination of the exceptions to this tendency reveals two patterns. First, the exceptions include the six sentences already discussed above as examples of discrepancies between word order and information structure — apparently, the word order in these sentences is influenced by some other factors, not included in the annotation. Second, in some of the examples where the Perspectival Center occupies a position other than the first place, the word order can be explained in terms of information structure. For instance, the figure in \REF{ex:topic-not-presupposed} lacks a presupposition of existence, and therefore it cannot be the Perspectival Center. Instead, it is the topic of the sentence, and that is the reason for its appearance at the beginning of the clause.

\ea\label{ex:topic-not-presupposed}
\gll ime ɕil-a ʈelar \textup{[}te\textup{]}\textsubscript{\textup{PerspectivalCenter}} na tʰanet \\
\textsc{prox.pl} such-\textsc{m} tailor here \textsc{neg} be.\textsc{prs.3pl} \\
\glt (How much does a good tailor charge? <…>) ‘There are no such tailors here.’
\z

From that, we conclude that the word order in the examples of our sample can be accounted for by information structure much more readily than by perspective structure.

\subsubsection{Summary}

The corpus study has shown what the bottom-up typology of figure-ground predications in Gawarbati looks like. Based on formal grounds, i.e. the presence of the marker \textit{ja}/\textit{jak} and word order, six groups of examples can be distinguished (although one of the groups lacks examples altogether), see \tabref{tab:wo-indefinite-marking}.

\begin{table}
\begin{tabularx}{0.5\textwidth}{l YY}
\lsptoprule
 &  \textit{ja}/\textit{jak} & ∅ \\
\midrule
F–G–V & 1 & 24 \\
G–F–V & 7 & 52 \\
F–V–G & 0 & 3 \\
\lspbottomrule
\end{tabularx}
\caption{The formal typology of figure-ground predications in the Gawarbati sample}
\label{tab:wo-indefinite-marking}
\end{table}


The variation shown in Table 7 is related to the \textsc{loc} vs. \textsc{loc-ex} contrast as defined through different parameters in a rather complex way. The presence of \textit{ja}/\textit{jak} predicts the indefinite status of the figure (although far from all indefinite figures are modified by \textit{ja}/\textit{jak}). The word order variation seems to primarily mark information structure, so it can be considered a syntactic expression of the \textsc{loc} vs. \textsc{loc-ex} contrast, but it also means that the resulting typology (presented in \tabref{tab:wo-io}) is significantly more complex than a simple dichotomy. As for perspective structure, it appears not to be directly reflected in the morphosyntax, or if it is reflected in word order, information structure appears to be the stronger factor (cf. \REF{ex:topic-not-presupposed}, \REF{ex:topical-nonexist}).


\tabref{tab:core-periph} demonstrates how the groups of examples identified in \tabref{tab:wo-indefinite-marking} align with the \textsc{loc} vs. \textsc{loc-ex} distinction as defined within the core vs. periphery approach, incorporating all three parameters (definiteness, information structure, perspectivisation).

\begin{table}
\begin{tabularx}{0.9\textwidth}{l YYYYY}
\lsptoprule
 & F–G–V & F–G–V, & G–F–V & G–F–V, & F–V–G \\
  & & \textit{ja}/\textit{jak} &  & \textit{ja}/\textit{jak} &  \\
\midrule
core \textsc{loc} & 12 & 0 & 0 & 0 & 0 \\
core \textsc{loc-ex} & 5 & 1 & 31 & 6 & 2\\
periphery & 7 & 0 & 21 & 1 & 1\\
\lspbottomrule
\end{tabularx}
\caption{Correspondences between formal and functional types of figure-ground predications in the Gawarbati sample}
\label{tab:core-periph}
\end{table}

The results show that all of the examples annotated as core \textsc{loc} clauses have the word order F–G–V and do not contain the \textit{ja}/\textit{jak} marker modifying the figure, while the morphosyntax of core \textsc{loc-ex} clauses is much more variable. Thus, in terms of the ability to account for the formal variation in figure-ground predications, the core vs. periphery approach does not seem to be a more powerful predictor than some of the parameters used alone, in particular, information structure.

As was noted in \sectref{basic-morphosynt}, in addition to indefiniteness marking and word order, Gawarbati occasionally marks the \textsc{loc} and \textsc{loc-ex} distinction by the lexical identity of the predicate: the verb \textit{niɕ-} ‘sit’ is used exclusively in \textsc{loc} clauses, and the predicative negative particle \textit{nahn-} is used exclusively in \textsc{loc-ex} clauses. These strategies were ignored in the corpus study due to their rare occurrence, but in other languages of the Hindu Kush region, the lexical identity of the predicate often plays a much more significant role. In the next section, we present an overview of the strategies for marking the \textsc{loc} vs. \textsc{loc-ex} contrast in the Hindu Kush languages and pay special attention to the interaction of the predicate's lexical identity with (in)definiteness marking and word order.



\section{Figure-ground predications in the Hindu Kush region} \label{Hindu-Kush}

\subsection{Areal survey}\label{Hindu-Kush:Areal}

The following is only a preliminary survey of marking strategies in the surrounding region. Examples are mostly drawn from grammatical descriptions, text collections of varying scope, and — to a more limited extent — own field data. Emphasis has been on identifying the more prominent strategies for upholding contrasts between \textsc{loc} and \textsc{loc-ex}, rather than offering any in-depth systematic analysis of each language in this respect, as few if any descriptions of languages of the region are explicit about such marking strategies, possibly with the exception of Appelgren’s (\citeyear[]{Appelgren2023}) study on locatives and existentials in Indo-Aryan Khowar [khow1242]. Starting with the more immediate Indo-Aryan neighbours of Gawarbati, the geographical scope will gradually widen as to include languages of all the six more prominent phylogenetic groupings that are represented in the Greater Hindu Kush, as shown in \figref{map}. No conscious attempts have been made at regularising the written representations of the languages, and we are mostly following the transcriptions applied in each of the cited works, while the glossing in a few cases has been adjusted (and somewhat simplified) to display greater uniformity.


\begin{figure}
\includegraphics[width=\textwidth]{figures/PanovaLiljegren-Map_Sample_HK_compressed.png}
\caption{Languages included in the areal sample}
\label{map}
\end{figure}

\subsection{Figure-ground predications in neighbouring speaker communities}

As of today, Gawarbati speakers primarily use two other languages of wider communication in local interaction with non-Gawarbati speakers, namely Pashto (Iranian, [pash1269]) and Khowar, the former on both sides of the border, and the latter on the Pakistani side. However, historically — especially if taking the last few centuries into account — contact patterns have most likely been rather more diverse and might not even have included those two languages, at least not to any greater extent. Among the more likely interactional candidates are — on the one hand — the upriver Indo-Aryan languages Dameli, Chitral Kalasha [kala1372] and Palula [phal1254], and downriver Sauji (closely related to Palula, [savi1242]), and possibly one or more varieties of Pashai [pash1270]. Dameli, for example, contrasts between \textsc{loc} \REF{ex:Dameli-loc} and \textsc{loc-ex} \REF{ex:Dameli-locex} by means of word order (F—G—V vs. G—F—V), entirely parallel to the structures in the elicited Gawarbati material.

\ea\label{ex:Dameli-loc}
\langinfo{Dameli}{}{\cite[131]{Perder2013}} \\
\gll šey-nam bum ṣaa daro \\
thing-\textsc{pl} ground on be.\textsc{inan.prs.3}  \\
\glt ‘The things are on the ground.’
\z

\ea\label{ex:Dameli-locex}
\langinfo{Dameli}{}{\cite[124]{Perder2013}} \\
\gll yee tʰaan-a gram beru \\
3\textsc{sg.inan} place-\textsc{loc} village be.\textsc{inan.pst.3}  \\
\glt ‘In this place was a village.’
\z


The same word-order alternation between \textsc{loc} and \textsc{loc-ex} is found in comparable sentences in Chitrali Kalasha \citep[52, 110]{HeegårdPetersen2006}, Palula (HL field data), Sauji \citep[66]{Buddruss1967} as well as in the better described Southeastern variety of Pashai \citep[381, 383]{Lehr2014}. In all of those languages, an utterance-final copula occurs, displaying agreement with the head of the figure NP along a variety of language-specific agreement patterns, whereas the \textsc{loc} or \textsc{loc-ex} identity itself does not appear to play a role in the choice of copula.   

Like in Gawarbati, an indefinite marker (derived from the numeral ‘one’) can occur in Palula (as well as in Sauji), but not obligatorily so, when the figure constitutes a part of the comment, as in \REF{ex:Palula-one}.

\ea\label{ex:Palula-one}
\langinfo{Palula}{}{B:FOY047} \\
\gll haṛáa táapeṛ-a ak tang ɣaár hín-i \\
there.\textsc{dist} mountain-\textsc{obl} \textsc{indf} narrow cave be.\textsc{prs-f} \\
\glt ‘There in the mountain is a narrow cave.’
\z


Such an indefinite marking, encoding something specific but not identifiable by the hearer, often translatable as “a certain X”, also occurs in Chitral Kalasha, as in \REF{ex:Kalasha-one}, i.e. `in a (certain) very beautiful forest area'.


\ea\label{ex:Kalasha-one}
\langinfo{Chitral Kalasha}{}{\cite[114]{HeegårdPetersen2015}} \\
\gll tasa dur ek bo ɕiɕojak adrak-una ɕ-iu \\
3\textsc{sg.rem.obl} house \textsc{indf} very beautiful forest-\textsc{loc} be-\textsc{prs/fut.3sg.ns} \\
\glt ‘His house lies in a very beautiful forest area.’
\z


Besides the ties with these neighbouring Indo-Aryan communities, Gawarbati has been in frequent contact for a very long time with some of the Nuristani groups in the mountainous areas in the Northwest, particularly with speakers of Nuristani Kalasha [waig1243] and Katë [kati1270] (\cites[6--7]{Morgenstierne1950}[241--248]{Cacopardo2001}). Here, too, word order is the most obvious strategy for differentiating between \textsc{loc} \REF{ex:NKalasha-loc} and \textsc{loc-ex} \REF{ex:NKalasha-locex}. 


\ea\label{ex:NKalasha-loc}
\langinfo{Nuristani Kalasha}{}{\cite[116, 285]{Degener1998}} \\
\gll toba üċ uzag di \textup{[}…\textup{]} murāygol-iw oṛ-ay \\
3\textsc{sg.gen} well today still \textup{[}…\textup{]} Muraygol-\textsc{loc} be-\textsc{prs.3sg} \\
\glt ‘His well is even today… in Muraygol.’
\z


\ea\label{ex:NKalasha-locex}
\langinfo{Nuristani Kalasha}{}{\cite[116, 313]{Degener1998}} \\
\gll nišeygrām-iw \textup{[}…\textup{]} atröm-čemi bröm-čemi oṛ-at \\
Nisheygram-\textsc{loc} […] upper-villager lower-villager be-\textsc{prs.3pl} \\
\glt ‘There are both up-valley and down-valley inhabitants in Nisheygram.’
\z


Distinctly different copulas, or copular realisations, are used in Nuristani Kalasha \citep[115--118]{Degener1998} as well as in Katë \citep[407--419]{Halfmann2022}, but in neither case does this appear to reflect a differentiation between \textsc{loc} and \textsc{loc-ex} per se. More interesting is the occurrence of posture verbs that apart from their posture usage have come to be used in copular-like ways. In Palula, the verb \textit{bheš-} ‘sit (down)’ spans across a rather wide range of meanings. While in some cases, the posture itself is intended, in many other contexts, it should be understood as meaning nothing more specific than ‘stay’, ‘wait’, ‘settle’, or ‘be or remain present in a certain place for an extended time’, as in \REF{ex:Palula-sit}. 

\ea\label{ex:Palula-sit}
\langinfo{Palula}{}{A:BEZ019} \\
\gll eetáa ugheeṇíi-a bheš-í de \\
there Pashtun-\textsc{pl} sit.down-\textsc{cvb} \textsc{pst} \\
\glt (The women had gone to pick wild herbs and brought them down to Red Plateau.) ‘Some Pashtuns were there (lit. had sat down).’
\z


A similar meaning is expressed by Katë \textit{ǰé-} ‘sit down’ in \REF{ex:Katë-sit}. 

\ea\label{ex:Katë-sit}
\langinfo{Katë}{}{\cite[140]{Halfmann2022}} \\
\gll ṍċ kómbr̆om ǰé-nëzëm \\
there Kómbr̆om.\textsc{loc} sit.down-\textsc{prog.pst.1sg.m}\\
\glt ‘I was staying in Kómbr̆om (Kāmdēš).’
\z



To what extent this is related to a \textsc{loc} vs. \textsc{loc-ex} differentiation is still very much an open question. In Palula, posture verbs are primarily (if not exclusively) used in \textsc{loc-ex}, and Baart (\citeyear[126--127]{Baart1999}) explicitly points out the use of a copula preceded by perfective adjectival participles of the verbs ‘sit’ (see \REF{ex:Gawri-sit}), ‘go’, ‘build’, and ‘adhere’ to introduce new participants in the discourse in Gawri/Kalam Kohistani [kala1373], an Indo-Aryan language spoken in the Panjkora and Swat valleys (to the East of Chitral and the Gawarbati-speaking area). He remarks that this predicate “is rather empty, and the main purpose of the clause is to assert the existence of the participant referred to” \citep[126]{Baart1999}.    

\ea\label{ex:Gawri-sit}
\langinfo{Gawri}{}{\cite[127]{Baart1999}} \\
\gll käṇō̃ rǟkä čor ḍākūǟn bäy-ǟl thu \\
ridge on four robbers sit-\textsc{adj.ptcp} be.\textsc{prs.m}\\
\glt ‘There were four robbers sitting on the ridge.’
\z



Although ‘sit’ appears to be the more common posture verb to display this particular usage, as also seen in the Nuristani Ashkun [ashk1246] example in \REF{ex:Ashkun-sit}, again with an indefinite specific figure NP, an analogous use of ‘lie’ can be seen in the Nuristani Prasun  [pras1239] example in \REF{ex:Prasun-lie}.

\ea\label{ex:Ashkun-sit}
\langinfo{Ashkun}{}{\cite[233]{Morgenstierne1929}} \\
\gll zaŋgal-wa pə kūč a pakīra niši-nestəgə \\
forest-\textsc{obl} in middle \textsc{indf} ascetic sit-\textsc{pfv.pst.3sg} \\
\glt ‘There was an ascetic in the middle of the forest.’
\z

\ea\label{ex:Prasun-lie}
\langinfo{Prasun}{}{\cite[222]{BuddrussDegener2017}} \\
\gll üštyöb-pan zəma atinyog əžə \\
tree-on snow lie.\textsc{pass.ptcp} be.\textsc{prs.3sg} \\
\glt ‘There is snow on the tree.’
\z


\subsection{Figure-ground predications in the wider surrounding region}

When widening the scope even further, geographically and phylogenetically, the same pattern prevails as far as the correspondence between word order and the \textsc{loc} vs. \textsc{loc-ex} contrast is concerned, as can be seen in the Burushaski examples in \REF{ex:Burushaski-loc} and \REF{ex:Burushaski-locex}, \textsc{loc} and \textsc{loc-ex} respectively. Apart from in Burushaski,\footnote{Burushaski \citep{Berger1998, Yoshioka2012}.} this general pattern is evidenced in Indo-Aryan,\footnote{Brokskat ([brok1247], \cite{Ramaswami1982, Sharma1998}), Chitral Kalasha \citep{HeegårdPetersen2006, HeegårdPetersen2015}, Dameli \citep{Perder2013}, Gawri \citep{Baart1999}, Gilgiti Shina \citep{RadloffShakil1998}, Indus Kohistani \citep{Lubberger2014, Lubberger2022}, Kalkoti ([kalk1245], own data), Kashmiri \citep{KoulWali1997}, Khowar \citep{Appelgren2023,Bashir2023}, Kohistani Shina ([kohi1248], \cite{Schmidt2006, SchmidtKohistani2008}), Palula (own data), Southeastern Pashai \citep{Lehr2014}, Torwali ([torw1241], \cite{Lunsford2001}).} Nuristani,\footnote{Ashkun \citep{Morgenstierne1929}, Katë \citep{Halfmann2022}, Nuristani Kalasha \citep{Degener1998}, Prasun \citep{BuddrussDegener2017}.} Iranian,\footnote{Munji ([munj1244], \cite{Morgenstierne1938}), Northern Pashto \citep{David2014}, Wakhi ([wakh1245], \cite{Bashir2009, SanGregory2018}), Yidgha  ([yidg1240], \cite{Morgenstierne1938}).}  Sino-Tibetan,\footnote{Balti \citep{Bielmeier1985, Jones2009, Read1934}.}  and Turkic\footnote{Uzbek ([uzbe1247], \cite{Sjoberg1963}).}  alike, as far as the wider Hindu Kush region is concerned. 

\ea\label{ex:Burushaski-loc}
\langinfo{Burushaski}{}{\cite[179]{Berger1998}} \\
\gll in basí-ul-o b-ái \\
he garden-\textsc{loc-ess} \textsc{cop-3sg.hm} \\
\glt ‘He is in the garden.’
\z

\ea\label{ex:Burushaski-locex}
\langinfo{Burushaski}{}{\cite[184]{Yoshioka2012}} \\
\gll gán-ul-o han bar-ċhil-an b-ilú-m \\
way-\textsc{loc-ess} one valley-water-\textsc{indf.sg} \textsc{cop-3sg.y-nprs} \\
\glt ‘There was a stream on the way.’
\z


The presence of an enclitic or suffixal indefinite or singulative marker \textit{-an} (probably derived from the numeral \textit{han} ‘one’) in \REF{ex:Burushaski-locex}, marking an indefinite specific figure NP in the \textsc{loc-ex}, is not unique to Burushaski. It occurs in quite a few of the neighbouring Indo-Aryan varieties — at least in Gilgiti Shina ([gilg1242] \textit{ek} ‘one’; =\textit{ek} \textsc{indf}, \cite[21--22]{RadloffShakil1998}), Kohistani Shina (\textit{ek} ‘one’; =\textit{(e)k} \textsc{indf}, \cite[75--79]{SchmidtKohistani2008}), Indus Kohistani ([indu1241] \textit{ék} ‘one’; =\textit{uk} \textsc{indf}, \cite{Lubberger2022}), Kashmiri ([kash1277] \textit{akh} ‘one’; =\textit{aː(h)} \textsc{indf}, \cite[94]{KoulBhat2014}) and Domaaki ([doma1260]; \textit{ek} ‘one’; =\textit{(e)k} \textsc{indf}, \cite[181--182]{Weinreich2011}) — as well as in Tibetan Balti (Sino-Tibetan, [balt1258]; \textit{čik} ‘one’; =\textit{či} \textsc{indf}, \cite[86, 162]{Bielmeier1985}), as duly noted by Yoshioka (\citeyear[47]{Yoshioka2012}). Neither is the co-occurrence of the suffix and a preceding “full” numeral ‘one’ a phenomenon restricted to Burushaski but has been noted in particular by \citet{Lubberger2022} for Indus Kohistani, and who describes at length the various uses of the indefinite specific enclitic \textit{=uk} in this language: a) to introduce participants/entities (\citeyear[287]{Lubberger2022}), as exemplified in \REF{ex:IKohistani-one}, b) to introduce a subset of an already known entity (\citeyear[291]{Lubberger2022}), and c) to encode approximation (\citeyear[293]{Lubberger2022}). While the enclitic \textit{=uk} implies specificity, the preceding \textit{ék}, also encoding indefiniteness, remains neutral in respect to specificity (\citeyear[297]{Lubberger2022}), and, when occurring alone, it often carries a generic meaning.

\ea\label{ex:IKohistani-one}
\langinfo{Indus Kohistani}{}{\cite[287]{Lubberger2022}} \\
\gll qasá gài gài ék baačàa=uk ã̀ãs \\
story go.\textsc{pfv.f} go.\textsc{pfv.f} \textsc{indf} king=\textsc{indf} be.\textsc{pst.m.sg} \\
\glt ‘Once upon a time there was a king.’
\z


It is entirely plausible that this parallel use actually originates in Burushaski and has subsequently spread through language contact; alternatively it is a substratal feature in speaker communities whose ancestors shifted from Burushaski to an Indo-Aryan language. An argument in favour of the latter, rather than the former, is the existence of a plural indefinite suffix \textit{-ik} in Burushaski \citep[43--44]{Berger1998} lacking paradigmatic parallels in all of these other languages except for Domaaki (Indo-Aryan, [doma1260]; \cite[182]{Weinreich2011}), a language that definitely stands at the receiving end of contact-induced change vis-à-vis Burushaski and Gilgiti Shina (\cites[80--83]{Backstrom1992}[314--315]{Weinreich2008}).

We concluded that there is relatively little evidence in the immediate neighbourhood of Gawarbati of a correlation between a \textsc{loc} vs. \textsc{loc-ex} differentiation and the preference for one or the other copula or copula-like verb used in such predications. However, such preferences can be found with more certainty in the wider region, particularly in the Western part of the Hindu Kush. Iranian Pashto (now a close neighbour, but more peripheral in a historical sense) is a case in point. While an inflected copula, used in other non-verbal predications, typically occurs in \textsc{loc} predications and displays agreement with the figure NP, as in \REF{ex:Pashto-cop}, a separate non-inflected existential particle/copula \textit{šta} (or \textit{šte}) is instead used in \textsc{loc-ex} predications, as in \REF{ex:Pashto-exist}. 


\ea\label{ex:Pashto-cop}
\langinfo{Northern Pashto}{}{\cite[308]{David2014}} \\
\gll zmā noṭ-una pə kitābče	ke	di \\
my note-\textsc{pl} in notebook.\textsc{obl} in be.\textsc{cont.prs.3pl.m} \\
\glt ‘My notes are in the notebook.’
\z

\ea\label{ex:Pashto-exist}
\langinfo{Northern Pashto}{}{\cite[93]{TegeyRobson1996}} \\
\gll pə kor ke woṛə šta \\
in house in flour \textsc{exist} \\
\glt ‘There is flour in the house.’
\z


In yet another set of languages, an explicit (existential) copula occurs, like in Pashto, in \textsc{loc-ex} predications, but can be left out altogether in \textsc{loc} predications. This is the case in Iranian Wakhi, in \REF{ex:Wakhi-omitted}, and Munji \citep[145]{Morgenstierne1938}, as well as in Turkic Uzbek, as in \REF{ex:Uzbek-omitted}. In the first-mentioned language, pronominal clitics with copular functions can co-occur with an overt copula to emphasise existence in \textsc{loc-ex}, while in \textsc{loc} both may be (but are not always) absent \citep[841--842]{Bashir2009}.\footnote{It should be noted that not everyone analyses the Wakhi clitics as being pronominal and/or copular, see \citet{SanGregory2015}.}


\ea\label{ex:Wakhi-omitted}
\langinfo{Wakhi}{}{\cite[126]{SanGregory2018}} \\
\gll ja mɔl-əʃ kɯ ra ʂad \\
\textsc{dem} flock-\textsc{pl} all down.to sheep.pen.\textsc{acc} \\
\glt ‘The flocks are all in the sheep pen.’
\z

\ea\label{ex:Uzbek-omitted}
\langinfo{Uzbek}{}{\cite[122]{Sjoberg1963}} \\
\gll ɔna-m taškent-ta \\
mother-\textsc{1sg.poss} Tashkent-\textsc{loc} \\
\glt ‘My mother is in Tashkent.’
\z


This clearly contrasts with the obligatory use of copulas in Tibetan Balti (in the East). Although Balti has more than one copula verb, \textit{yot} is the one that is normally used in \textsc{loc} as well as in \textsc{loc-ex}, apart from also occurring in some adjectival predications and in possessive constructions (\cites[35--37]{Read1934}[28--29]{Jones2009}). In \REF{ex:Balti-yot}, the use of \textit{yot} in a \textsc{loc-ex} is immediately followed by its use in a \textsc{loc}.

\ea\label{ex:Balti-yot}
\langinfo{Balti}{}{\cite[71]{Jones2009}} \\
\gll hundar-iŋ kʰoŋ-i kʰar yot, ŋai apičo-e astano	saŋ ina hundar-iŋ yot\\
Hundar-\textsc{loc} \textsc{3pl-gen} palace \textsc{exist} \textsc{1pl.excl.gen} grandmother.\textsc{hon-gen} tomb also there Hundar-\textsc{loc} \textsc{exist} \\
\glt ‘In Hundar there is their palace. Our grandmother’s tomb is also there in Hundar.’
\z


All of the languages exemplified so far, regardless of their phylogenetic identity, are canonically V-final. Thus, any major word order alternation is the one observed between figure and ground only. The only obvious exception to V-finality in the wider region, as defined here, is Kashmiri, normally considered a verb second word order (V2) language, exceptional both in terms of its areal affinities and in relation to other Indo-Aryan languages \citep[175]{Verbeke2013}. However, as far as contrasting \textsc{loc} and \textsc{loc-ex}, that is usually obtained by an F—V—G vs. G—V—F alternation \citep[338]{KoulWali1997}.

\subsection{Summary}


This rough areal survey, covering 27 language varieties (including Gawarbati), belonging to six phylogenetic groups, has shown that there is an overwhelmingly frequent correlation between \textsc{loc} predications and the constituent order F—G, and between \textsc{loc-ex} and the order G—F. This is not to say that other word orders are absent, only that the canonical order observed in Gawarbati has very clear areal parallels, both in the immediate geographical surroundings and in the wider Hindu Kush region, just like SOV is the only basic word order pattern represented in the region (with the marginal exception of the V2 pattern in Indo-Aryan Kashmiri). This correlation holds regardless of the phylogenetic affiliation. 

More extensive research and the availability of larger language-specific corpora would be essential in order to detect and explain more exact correlations between word order and the parameters discussed in \sectref{Gawarbati}, but a few key observations have also been made in regard to indefiniteness marking and the predicate's lexical identity. Indefinite markers, often derived from the numeral ‘one’, frequently occur with the figure NP of \textsc{loc-ex} predications. The semantics of such NPs is often one of indefinite specificity. A subset of languages, particularly those spoken in the eastern part of the Hindu Kush, have developed a suffixal or enclitic marker, in addition to a preposed numeral/indefinite article, possibly modelled on, or parallel to, a grammaticalisation process originating in Burushaski.

Posture verbs, most often based on ‘sit’, alternate with the use of more generally used copulas, particularly in the Indo-Aryan and Nuristani languages in the vicinity of Gawarbati. Although that, too, would need more careful investigation and larger corpora for a more authoritative interpretation to be made, the initial impression is that such posture verbs occur more often in \textsc{loc-ex} predications than in \textsc{loc} predications, in the languages that have them, i.e. the opposite pattern to the one noted in Gawarbati. Another observation made is a tendency in Iranian and Turkic, in the Northwest of the region, to encode \textsc{loc} and \textsc{loc-ex} in lexically distinct ways, such as using different copulas or completely leaving out the copula, primarily in \textsc{loc} predications.      


\section{Conclusions} \label{Conclusions}
 
In this paper, we analysed the morphosyntactic variation in figure-ground constructions in Gawarbati and other languages of the Hindu Kush region — languages which, in terms of Creissels’ (\citeyear{Creissels2019}) typology, lack a dedicated \textsc{loc-ex} construction that is distinct from a \textsc{loc} construction. In identifying the \textsc{loc} vs. \textsc{loc-ex} status of the clause, we mostly relied on the translations into English where \textsc{loc} and \textsc{loc-ex} clauses are formally distinct. In the case of Gawarbati, we also conducted a more fine-grained analysis of the corpus data, attempting to relate the observed morphosyntactic variation in the sample to the semantic/pragmatic criteria discussed in the literature as differentiating \textsc{loc} and \textsc{loc-ex} clauses cross-linguistically (definiteness of the figure, information structure and perspectivisation).


The investigation of Gawarbati presented in this paper is, to our knowledge, the first systematic corpus-based study of figure-ground predications in a language which lacks formal marking dedicated to differentiating \textsc{loc} and \textsc{loc-ex} clauses. Moreover, it is one of only a few corpus-based studies of figure-ground predications in general (another example is \cite{Budzisch2025}). In the process of annotation of natural spoken data, two remarks, relevant to the general discussion on the \textsc{loc} vs. \textsc{loc-ex} distinction, have been made. First, it has been shown that the majority of figure-ground predications occurring in natural discourse can be clearly assigned to the \textsc{loc} type or to the \textsc{loc-ex} type regardless of the criteria we choose for defining the \textsc{loc} vs. \textsc{loc-ex} contrast (definiteness, information structure or perspectivisation). In one third of the examples, however, the choice of the definition becomes crucial as different criteria give different results. Second, the annotation of the perspective structure in affirmative clauses turned out to be quite challenging. Thus, although defining the \textsc{loc} vs. \textsc{loc-ex} contrast through perspectivisation might seem promising, in practice the perspective structure in affirmative clauses, where the presence of the existence presupposition is less clear, is difficult to determine.


The corpus-based analysis, complemented by the comparison with areally close languages, allowed us to verify some of the claims made in the earlier literature and make several new observations. Overall, the results suggest that there are three morphosyntactic parameters which reflect, to varying degrees, the \textsc{loc} vs. \textsc{loc-ex} status of the predication in the Hindu Kush: word order, indefiniteness marking and the lexical identity of the predicate.
The word order variation reflects the \textsc{loc} vs. \textsc{loc-ex} alternation in a very consistent way across all languages studied in this paper. The consistency of this association is likely due to the similarity of word order patterns in the Hindu Kush region and is thus a phenomenon of areal nature. In all discussed Hindu Kush languages (except V2 Kashmiri), the unmarked word order is SOV, and in all of them, word order is flexible, allowing for the use of different word order patterns in figure-ground predications. As argued in the previous literature \citep[556--557]{Freeze1992}, SOV languages usually have the word order F—G—V in \textsc{loc} clauses and the word order G—F—V in \textsc{loc-ex} clauses; the results of our study confirm these generalisations. What has been less discussed in the literature, however, is the precise mechanism behind the link between the \textsc{loc} vs. \textsc{loc-ex} status and word order. The corpus study of Gawarbati has shown that word order primarily encodes the information structure of the clause: for example, when the figure is topical, it appears at the beginning of the clause, and when the figure is focused, it appears in the preverbal position. Furthermore, it means that word order reflects a whole set of information-structural patterns available in figure-ground constructions which are not always easily matched with a simple \textsc{loc} vs. \textsc{loc-ex} opposition. Given the uniformity of word-order-related processes in the Hindu Kush, it is reasonable to suggest that, upon closer examination, a similar conclusion could be drawn for other languages discussed in this paper.


The (in)definiteness marking of the figure shows slightly more variation. In all Hindu Kush languages, we observe some kind of indefinite marking of nouns, based on the numeral ‘one’. When the figure has an indefinite marker, the clause is likely to be a \textsc{loc-ex} predication, but, as shown in the corpus study of Gawarbati, this only works in one direction: far from all \textsc{loc-ex} predications have the indefinite marker. In Gawarbati, the indefinite marker has many restrictions. In particular, it appears only in indefinite specific contexts. Therefore, it is a very poor predictor of the \textsc{loc} vs. \textsc{loc-ex} status, and, most probably, the same applies to the indefinite markers in some of the neighboring languages, e.g. Dameli and Palula. It is important to note, however, that in some languages, primarily in the Eastern Hindu Kush with a possible source in Burushaski, the numeral ‘one’-based marking may be doubled: the indefinite article preceding the noun may co-occur with the cognate suffix or enclitic attached to the noun on the right side. In Indus Kohistani \citep{Lubberger2022}, the clitic marker has the function of introducing the participant to the discourse, a function that is often associated with \textsc{loc-ex} clauses (cf. Croft’s (\citeyear{Croft2022}) term “presentational location” for \textsc{loc-ex}). This might suggest that some languages have developed a slightly more specialised means of marking the \textsc{loc-ex} clauses than general indefinite marking.

The lexical identity of the predicate is the least stable feature differentiating \textsc{loc} and \textsc{loc-ex} clauses in the Hindu Kush languages. First, some variation can be seen in the use of posture verbs in figure-ground predications, a phenomenon widely attested cross-linguistically \citep{AmekaLevinson2007} and present in the Indo-Aryan and Nuristani languages. While in some languages, e.g. in Palula, Gawri and Ashkun, posture verbs in figure-ground predications can serve as indicators of the \textsc{loc-ex} status of the clause, in other languages, e.g. in Gawarbati, the rare copula-like uses of the same verbs are found exclusively in \textsc{loc} clauses. Second, individual languages allow omission of the copula in \textsc{loc} clauses and/or have specialised existential predicative markers in \textsc{loc-ex} clauses. The more complex structure of \textsc{loc-ex} predication, while not typologically surprising, is not a prevalent feature throughout the Hindu Kush region, but rather is associated with non-Indo-Aryan languages of the western part of the area.

To sum up, word order and general indefinite marking are more stable features in the area, but at the same time they are less specialised in terms of marking the \textsc{loc} vs. \textsc{loc-ex} distinction. In contrast, introductory marking of the figure and special existential verbs in \textsc{loc-ex} are less stable and more idiosyncratic features in the area, but they represent a more advanced stage of grammaticalisation of the \textsc{loc} vs. \textsc{loc-ex} contrast. Within this view, Gawarbati appears to represent the least developed stage of grammaticalisation of the \textsc{loc} vs. \textsc{loc-ex} distinction, in an areal comparison.





\section*{Abbreviations}
\subsection*{In running text}

\begin{tabularx}{.5\textwidth}{@{}lQ@{}}
\textsc{f} & figure\\
\textsc{g} & ground\\
\textsc{loc} & locative predication\\
\textsc{loc-ex} & locational-existential predication\\
\textsc{np} & noun phrase\\
\end{tabularx}%
\begin{tabularx}{.5\textwidth}{@{}lQ@{}}
\textsc{o} & object\\
\textsc{s} & subject\\
\textsc{tam} & tense, aspect, mood\\
\textsc{v} & verb\\
\textsc{v2} & verb second word order\\
\end{tabularx}

\subsection*{In glossed examples}
The following abbreviations are not found in the Leipzig Glossing Rules:\medskip

\noindent
\begin{tabularx}{.5\textwidth}{@{}lQ@{}}
\textsc{add} & additive\\
\textsc{cont} & continuous\\
\textsc{ess} & essive\\
\textsc{exist} & existential particle or copula\\
\textsc{hm} & human masculine gender\\
\textsc{hon} & honorific\\
 & \\

\end{tabularx}%
\begin{tabularx}{.5\textwidth}{@{}lQ@{}}
\textsc{inan} & inanimate\\
\textsc{nprs} & non-present\\
\textsc{ns} & non-specific\\
\textsc{ptcl} & particle\\
\textsc{rem} & remote\\
\textsc{y} & y-gender (Burushaski-specific class)\\
\end{tabularx}

\section*{Acknowledgements}

This work is part of the project \textit{Gawarbati: Documenting a vulnerable linguistic community in the Hindu Kush}, supported by Vetenskapsrådet, the Swedish Research Council (VR 2020-01500). We are grateful to Rodolfo Basile and Erin SanGregory for their many insightful comments and suggestions which helped us improve the text.

\sloppy
\printbibliography[heading=subbibliography,notkeyword=this]
\end{document}
