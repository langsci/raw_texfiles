\documentclass[output=paper,colorlinks,citecolor=brown]{langscibook}
\ChapterDOI{10.5281/zenodo.16838082}
\author{Eleanor Yacopetti\orcid{}\affiliation{Monash University} and Laurits Stapput Knudsen\orcid{}\affiliation{University of Newcastle} and Tom Ennever\orcid{}\affiliation{University of Surrey}}
%\ORCIDs{}

\title[Posture verbs across three Australian languages]{Posture verbs in locative and existential predication across three Australian languages}

\abstract{In this chapter, we examine the role of posture verbs in locative and existential predication in three Australian Aboriginal languages: Kukatja (Western Desert, Pama-Nyungan), Kune (a variety of Bininj Kunwok; Gunwinyguan), and Wik-Mungkan (Paman, Pama-Nyungan). We detail the semantic range of each posture verb and consider the extent to which individual posture verbs are recruited in more grammaticalised roles. Posture verbs exist alongside verbless predicational strategies in all three languages and the frequency of their use differs across the three languages in notable ways depending on the animacy of the figure referent.  We utilise standardised elicitation tasks to quantify observed variation in posture verb use and find that in each language, different posture verbs show differing degrees of grammaticalisation.}

\IfFileExists{../localcommands.tex}{
   \addbibresource{../localbibliography.bib}
   % add all extra packages you need to load to this file

\usepackage{tabularx,multicol}
\usepackage{url}
\urlstyle{same}

\usepackage{listings}
\lstset{basicstyle=\ttfamily,tabsize=2,breaklines=true}

\usepackage{langsci-basic}
\usepackage{langsci-optional}
\usepackage{langsci-lgr}
\usepackage{langsci-osl}
% \usepackage{./langsci/styles/langsci-lgr}
% \usepackage{./langsci/styles/langsci-osl}
% \usepackage{langsci-gb4e}

\usepackage{tikz}
\usetikzlibrary{patterns,calc}
\pgfdeclarepatternformonly{south east lines}{\pgfqpoint{-0pt}{-0pt}}{\pgfqpoint{3pt}{3pt}}{\pgfqpoint{3pt}{3pt}}{
    \pgfsetlinewidth{0.6pt}
    \pgfpathmoveto{\pgfqpoint{0pt}{3pt}}
    \pgfpathlineto{\pgfqpoint{3pt}{0pt}}
    \pgfpathmoveto{\pgfqpoint{.2pt}{-.2pt}}
    \pgfpathlineto{\pgfqpoint{-.2pt}{.2pt}}
    \pgfpathmoveto{\pgfqpoint{3.2pt}{2.8pt}}
    \pgfpathlineto{\pgfqpoint{2.8pt}{3.2pt}}
    \pgfusepath{stroke}}
    
\usepackage{stmaryrd}
\usepackage{wasysym}
\usepackage{multirow}
\usepackage{caption}
\usepackage{subcaption}
\usepackage{mathrsfs}
\usepackage{qtree}

\usepackage{linguex}


   %pminos do not split footnotes
% \interfootnotelinepenalty=10000 %Footnote in Laporte chapters has to be split SN


%\DeclareIndexNameFormat{default}{%
%\nameparts{#1}%
%\usebibmacro{index:name}%
%{\index[names]}%
%{\namepartfamily}%
%{\namepartgiveni}%
% {}% L1
% {}% L2
%{\namepartprefix}% generates spurious space L3
%{\namepartsuffix}% generates spurious space L4
%}

%  {\DeclareIndexNameFormat{default}{%
%     \usebibmacro{index:name}{\index[names]}{#1}{#3}{#5}{#7}}}

%\DeclareIndexNameFormat{default}{%
%  \usebibmacro{index:name}{\sindex[nom]}{#1}{#3}{#5}{#7}}

%\DeclareIndexNameFormat{default}{%
%  \usebibmacro{index:name}{\sindex[person]}{#1}{#3}{#5}{#7}}
%\DeclareIndexNameFormat{default}{%
%\nameparts{#1} \usebibmacro{index:name}{\sindex[person]]}{\namepartfamily}{‌​\namepartgiven}{\nam‌​epartprefix}{\namepa‌​rtsuffix}}

%\newcommand{\smiley}{:)}

%\renewbibmacro*{index:name}[5]{%
%\usebibmacro{index:entry}{#1}%
%{\iffieldundef{usera}{}{\thefield{usera}\actualoperator}\mkbibindexname{#2}{#3}{#4}{#5}}}

% \newcommand{\noop}[1]{}

%remove for final
%\overfullrule=1mm

\newcommand{\tobi}[2]}}
\renewcommand{\S}[1]{\tobi{#1}{\textsc{*}}}

% this volume references
% puts: [this volume]
% already defined: \citetv
%\newcommand{\citepv}[1]{(\citeauthor{#1} \citeyear*{#1} [this volume])}
\newcommand{\citealtv}[1]{\citeauthor{#1} \citeyear*{#1} [this volume]}

%parentheses around example number
\newcommand{\pref}[1]{(\ref{#1})}

% in-text examples

\newcommand{\lnex}[1]{\textit{#1}} %target lang word
\newcommand{\lnlit}[1]{(lit.: `#1')} %literal reading
\newcommand{\lnlat}[1]{(#1)} % latinization
\newcommand{\lntrans}[1]{`#1'} %translation
\newcommand{\lnexl}[2]%
{\lnex{#1}{} \lnlat{#2}} % ex with latinization
\newcommand{\lnexlat}[3]{\lnex{#1}{} \lnlat{#2}{} \lntrans{#3}} % ex with latinization and tranl.

%ch01
\newcommand{\co}[1]{\mbox{\textbf{#1}}}

%ch09

\newcommand{\cyrbulg}[1]{\begin{otherlanguage*}{bulgarian}#1\end{otherlanguage*}}


%ch10
\newcommand{\nlp}{{\small NLP}}
\newcommand{\mwe}{{\small MWE}}
\newcommand{\rae}{{\small RAE}}
\newcommand{\lvc}{{\small LVC}}
\newcommand{\pos}{{\small P}o{\small S}}
%\newcommand{\todo}[1]{ \textcolor{red}{#1} }

%\renewcommand{\labelenumi}{\theenumi}
%\ainamefmt{{vv}{ll}{, ff}{, jj}} % fullname

\newcommand{\biberror}[1]{{\color{red}#1}}

\newcommand{\osenovaitem}{--~}
   %% hyphenation points for line breaks
%% Normally, automatic hyphenation in LaTeX is very good
%% If a word is mis-hyphenated, add it to this file
%%
%% add information to TeX file before \begin{document} with:
%% %% hyphenation points for line breaks
%% Normally, automatic hyphenation in LaTeX is very good
%% If a word is mis-hyphenated, add it to this file
%%
%% add information to TeX file before \begin{document} with:
%% %% hyphenation points for line breaks
%% Normally, automatic hyphenation in LaTeX is very good
%% If a word is mis-hyphenated, add it to this file
%%
%% add information to TeX file before \begin{document} with:
%% \include{localhyphenation}
\hyphenation{
    Beck-man
    Ngu-yen
    back-chan-nel
    back-chan-nels
    mo-not-o-nous
    ste-reo-typ-i-cal
}

\hyphenation{
    Beck-man
    Ngu-yen
    back-chan-nel
    back-chan-nels
    mo-not-o-nous
    ste-reo-typ-i-cal
}

\hyphenation{
    Beck-man
    Ngu-yen
    back-chan-nel
    back-chan-nels
    mo-not-o-nous
    ste-reo-typ-i-cal
}

   \boolfalse{bookcompile}
   \togglepaper[14]%%chapternumber
}{}


\begin{document}
\maketitle

\section{Introduction} \label{YacopettiSection1}
\subsection{Background and aims} \label{YacopettiSection1.1}
The basic human stances of sit, lie, and stand are likely encoded in all languages and are sometimes referred to as ‘primary postures’ (\cite{Newman2022}). Morphemes encoding these primary postures have been investigated cross-linguistically revealing recurrent patterns that further motivate their typological grouping beyond their shared function depicting universally shared human postures (see \cite[850]{AmekaLevinson2007}; \cite{Newman2002CrossLinguistic}; \cite{Newman2022}). Typical shared features of this semantic class include: high frequency (with respect to other posturals), semantic extensions reflecting experiential realities (e.g. ‘lie’ extending to ‘sleep’) and certain differential paradigmatic features. One particular area of typological interest is the extent to which posture verbs are allowed -- or are, in fact, required -- when encoding the location or existence of a figure: functions often referred to as locative (\textit{x} \textsc{is at} \textit{y}) and existential (\textit{x} \textsc{exists} [\textsc{at} \textit{y}]).\footnote{The terms \textit{figure} and \textit{ground} were popularised by \citet{Talmy1983} and in simple terms can be understood as the relation being located and the relation against which something is located respectively.} 
Such uses can be exemplified in English by the posture verb \textit{lie}, which can be used for locative functions (\textit{the city lies to the east}) and existential functions (\textit{a river lies to the east}). In English, such strategies are not the only construction type for predicating location and existence but rather constitute minor strategies compared to the more prevalent use of the copula \textit{be}. The usage of posture verbs is thus rather specialised in English, tending to occur for only certain types of spatial scenes involving certain types of figures and grounds.

For a subset of the world’s languages, however, posture verbs are relied upon as an essential predicational element within a “Basic Locative Construction” (\cite{AmekaLevinson2007}). A language’s Basic Locative Construction is defined in accordance with a functional frame, namely, it is the unmarked construction type used to answer the question: “where is \textit{x}?”. In languages like Arrernte [aran1263], Dutch [dutc1256] and Goemai [goem1240], to express an unmarked predicational notion \textsc{be at}, a posture verb must be used. Crucially, there are still very few detailed studies of individual Australian languages exploring how obligatory posture verbs are in languages that make use of them in this way, what motivates their use (e.g. animacy/geometric properties of figure) and precisely what their semantic contribution to various clause types is. Posture verbs in many languages have been shown to take on a range of additional functions in locative or existential contexts, such as: signifying temporal meanings other than present tense (\cite[104]{Austin1981}; \cite{Hercus1994}); marking aspect (\cite[25]{Blansitt1975}; \cite{BybeeEtAl1994}; \cite{Kuteva1999}; \cite{Newman2002}); marking emphasis or contrast (\cite[179]{Nordlinger1998}); categorising a located subject entity (\cite[7]{Newman2022}); disambiguating polysemous nouns \citep{Wilkins1989,Wilkins2006}; conceptualising a located entity as a mass or individuated entity (\cite{Kaufman2013}; \cite{Newman2022}); contrasting temporary and permanent location (\cite{HaspelmathNonverbal}); or signalling containment (\cite{Lemmens2002}). What all of these studies show is that, even if a posture verb may appear to be a ``semantically neutral" element with respect to its postural specification in locative or existential constructions, such an element may still have separate, or additional, meaningful functions. Such findings suggest that posture verbs can vary considerably cross-linguistically, and motivate further detailed studies, especially for lesser described languages.

In this paper, we explore the role of posture verbs in locative and existential predication in three Australian Aboriginal languages. The existing literature for Australian languages provides two fairly well-known starting observations: i) that posture verbs are both diachronically and synchronically involved in copula constructions (\cite{Dixon2002Copula}); and ii) that locative and existential meanings are often encoded by the same construction type (\cite[241--242]{Dixon2002Australian}; \cite[224--229]{GoddardHarkins2002}). Our aim is to extend the description of postural semantics among Australian languages, enriching earlier observations about the typological profile of the region. We present findings on three languages with which we are familiar: two Pama-Nyungan languages (Kukatja [kuka1246] and Wik-Mungkan [wikm1247]), and one Gunwinyguan language (Kune [gune1238]). Using novel data and existing descriptions for each of the surveyed Australian languages, we ask:

\begin{itemize}
    \item What is the lexical semantic range of posture verbs? (§\ref{YacopettiSection3.1})
    \item How are posture verbs recruited in the predication of location and existence? (§\ref{YacopettiSection3.2})
    \item Are locative and existential predicates distinguished in the languages? (§\ref{YacopettiSection3.3})
\end{itemize}

These descriptive questions give rise to the observation that posture verbs exhibit substantial semantic flexibility in the study languages, more than is suggested in many of the existing descriptions. We subsequently analyse the frequency of posture verb use in locative clauses (§\ref{YacopettiSection3.4}) in order to provide some insights into their versatility. 
To better orient the reader with respect to how posture verbs feature in locative and existential functions, we first review the previous literature on posture verbs in Australian languages and report frequently attested meanings, functions, and semantic extensions (§\ref{YacopettiSection1.2}). In (§\ref{YacopettiSection2}) we outline the methodology for the current study, introduce the three languages (Kukatja, Kune and Wik-Mungkan) as well as the data collection process.
A terminological note of order before proceeding is that we initially refer to constructions that semantically predicate the closely related notions of location or existence simply as \textit{locative-existential} although the status of these terms is considered more explicitly in (§\ref{YacopettiSection3.3}).

\subsection{Posture verbs in Australian languages} \label{YacopettiSection1.2}
\largerpage
As part of a survey of copula predication strategies in Australian languages, \citet{Dixon2002Copula} notes a recurrent pattern of copula verbs having grammaticalised from erstwhile posture verbs in several independent contexts, while also drawing attention to the synchronic copula-like functions exhibited by posture verbs in many languages. Research specifically dedicated to posture verbs has also been conducted previously for the languages Arrernte [aran1263] (\cite{GoddardHarkins2002}; \cite{Wilkins1989,Wilkins2006}), the closely related Western Desert languages Pitjantjatjara [pitj1243] and Yankunytjatjara [yank1247] (\cite{GoddardHarkins2002}), Jaminjung [jami1236] (\cite{Schultze-Berndt2006}), Ngan’gityemerri [nang1259] (\cite{Reid2002}), and Warrwa [warr1258] (\cite{McGregor2006}).\footnote{Arrernte (Arrandic), Pitjantjatjara (Western Desert), and Yankunytjatjara (Western Desert) are all Pama-Nyungan languages, the same language family to which Wik-Mungkan and Kukatja belong. The third language in this study, Kune, is unrelated to any of the aforementioned languages.}  Here we summarise the key parameters of observed variation in posture verbs from these and other works, in terms of: their degree of grammaticalisation, their role in locative and existential predication while also drawing attention to the additional roles they may serve in such contexts.

Aligning with global trends, the most common posture verbs are those encoding the so-called primary postures of ‘sit’, ‘stand’, and ‘lie’, although one does find languages which have additional lexical verbs that are treated as a primary posture on the basis of their aligned semantics and/or morphosyntax: for example \textit{pupa-} ‘be crouched over, bent over’ in Pitjantjatjara and Yankunytjatjara (\cite[214]{GoddardHarkins2002}); \textit{yoongke} ‘hang’ in Kuuk Thaayorre (Pama-Nyungan, [thay1249]; \cite[383]{Gaby2017}) and ‘perch’ in Ngan’gityemerri (\cite[243]{Reid2002}); among others. Dixon’s survey of copula constructions in Australian languages identified that posture verbs constitute the main stock of copula elements across the continent, with a smaller number of languages making use of a dedicated ‘remain/reside’ or ‘go’ verb in copula function (\cite{Dixon2002Copula}). While the use of posture verbs in copula predication varies across a range of parameters (e.g. functions, frequency, preferred posture verb, etc.), they are frequently core components of many locative and existential clauses (\cite[28]{Dixon2002Copula}).

In some Australian languages, the postural semantics of individual posture verbs show evidence of complete or near-complete bleaching when used in certain constructions. Consider examples (\ref{Yacopetti1}) and (\ref{Yacopetti2}) from central Australian languages Arabana and Warlmanpa respectively, where the posture verbs \textit{thangk} and \textit{ka} are used in identificational and presentational clauses respectively without any residual postural semantics evident (see also discussion in \cite[22]{Dixon2002Copula}). 

\ea \label{Yacopetti1}
\langinfo{Arabana}{Pama-Nyungan, [arab1267]}{\cite[295]{Hercus1994}} \\
\gll Antha		minpaRu	\textbf{thangk-ka}. \\
\textsc{1sg} doctor		\textbf{be(lit.sit)-\textsc{pst}} \\
\trans ‘I am a doctor.’ \\
\z

\ea \label{Yacopetti2}
\langinfo{Warlmanpa}{Pama-Nyungan, [warl1255]}{\cite[385]{Browne2021}} \\
\gll Yumpa=ju	\textbf{ka-nya}		kirtana. \\
this=\textsc{1sg.ns} \textbf{be(lit.sit)-\textsc{prs}} father \\
\trans ‘This is my father.’ \\
\z

The examples in (\ref{Yacopetti1}) and (\ref{Yacopetti2}) are notable for the unique argument structures that they instantiate in their respective languages; they are bivalent with two formally unmarked nominal arguments (i.e. they are in the non-overt absolutive case). For these languages, the availability of the [\textsc{abs-abs}] case frame is diagnostic of a posture verb having a copula use or a copula sense since such a case frame is unique to these verbal predicates.
An interesting feature of posture verb systems in Australian languages is that typically just one of the posture verbs is recruited in bona fide copula roles (beyond locative-existential predication) of the kind illustrated in (\ref{Yacopetti1}) and (\ref{Yacopetti2}). \citet[210]{Dench1987} makes this point explicitly for Martuthunira (Pama-Nyungan, [mart1255]), a language which permits all three of its posture verbs (\textit{nyina-} ‘sit’, \textit{karri-} ‘stand’, \textit{wanti-} ‘lie’) to be used in copula function, but in which:

\begin{quote}
only \textit{nyina-} can be said to function like a true dummy copula; both \textit{karri-} and \textit{wanti-} \textbf{retain something of their core meaning in any copula construction} [emph. our own].
\end{quote}

        
Moreover, when one compares languages, there is substantial variation with respect to which member of this class has developed specialized functions (\cite[241--242]{Dixon2002Australian}): for some languages it is the ‘sit’ verb (e.g. Martuthunira: \cite[210]{Dench1987} and Kunbarlang (Gunwiniguan, [kunb1251]): \cite[307]{Kapitonov2021}), in others it is ‘lie’ (as in Gooniyandi (Bunaban, [goon1238]): \cite[310--311]{McGregor1990}) or ‘stand’ (as in Pitjantjatjara and Yankunytjatjara: \cite[224]{GoddardHarkins2002}).  \citet[28]{Dixon2002Copula} was the first to note that even amongst geographically and genetically close neighbours, individual languages appear to have developed these grammaticalised functions in a somewhat independent manner. In Pitjantjatjara and Yankunytjatjara (both members of the Western Desert subgroup of the Pama-Nyungan family), it is the ‘stand’ verb (both sharing a form \textit{ngara-}) that has the broadest range of copula functions (\cite[224]{GoddardHarkins2002}) while in Kukatja, a northern member of the same tight-knit subgroup, it is the ‘sit’ verb \textit{nyina-} (and not \textit{ngara-} ‘stand’) that has the broadest range of functions. One finds very similar diachronic evidence from Maric languages in Queensland: Biri [biri1256] and Gidabal [cond1242] have a copula \textit{wara-} which is cognate with ‘stand’ in neighbouring languages while Bidjara [bidy1243] and Gunja [guny1241] have a general copula \textit{wiyi}, cognate with a ‘sit’ verb in neighbouring languages (e.g. Ngiyambaa [wang1291], \cite[28]{Dixon2002Copula}).


Australian Aboriginal languages also differ with respect to whether they require a posture verb in locative or existential clauses across the board or may freely omit the verbal component in favour of a truly verbless (viz. copula-free) construction. In Warrwa, for example, both the posture verb and figure NP may be omitted in locative constructions, as long as the ground nominal is retained (\cite[123--124]{McGregor2006}). On the other hand, a posture verb may be required in copula constructions but only in specific contexts. Posture verbs frequently operate as a semantically neutral “host” or “dummy” in order to encode verbal categories such as tense, aspect, and/or mood (\cite[17]{Dixon2002Copula}). For example, Wilkins describes the variable presence of posture verbs in copula constructions in Arrernte as being tense-dependent, posture verbs being required when non-present meanings are expressed (\cite[32]{Wilkins2006}). Similar descriptions of posture verb use in copula function have been made for Martuthunira (\cite[209--210]{Dench1987}) and Gooniyandi (\cite[308--312]{McGregor1990}). Differences in predicational strategy have also been observed to be dependent on the nature of the locative component. In Ngiyambaa, \citet[232--233]{Donaldson1980} reports that if the locative component involves a noun as a head, then a posture verb is required. If the locative component is a demonstrative or an interrogative, then a (lexically distinct) copula is used. It is unknown to what extent this pattern is unique to Ngiyambaa or has simply gone unnoticed due to incomplete documentation.

\largerpage
A further point of variation across the continent relates to how posture verbs in different languages show differing degrees of sensitivity to the geometric properties of inanimate figures. In Ngan’gityemerri, for instance, an inanimate figure that is ``longer" in its horizontal dimension than its vertical dimension is said to ‘lie’ (e.g. rivers, roads or rocks) whereas other inanimate figures that are ``taller" in their canonical vertical orientation than their horizontal dimension are said to ‘stand’ (e.g. fridges, antbeds).\footnote{Again, similar observations regarding copula choice (with small differences) have been made for a number of other Australian languages such as Yankuntjatjara and Pitjantjatjara (\cite[214--223]{GoddardHarkins2002}), Gooniyandi (\cite[310]{McGregor1990}), Warlpiri (\cite[215]{Simpson1991}) and Kunbarlang (\cite[306--308]{Kapitonov2021}).} ‘Sit’ is the only posture verb in Ngan’gityemerri available for human figures in existential predication, distinguishing unmarked stance existential meanings from those entailing a postural implication (\cite[247--248]{Reid2002}). Leveraging the relation between the axial geometry of the figure object and the chosen posture verbs, the choice of posture verb can also serve to narrow the possibilities of intended senses of a polysemous nominal denoting the figure. In the Arrernte example in (\ref{Yacopetti3}), the sense ‘stick’ is selected from possible denotations of \textit{arne} ‘stick, tree’ because the former referent is pragmatically more associated with ‘lie’, whereas in (\ref{Yacopetti4}) the latter sense ‘tree’ is most applicable because trees prototypically ‘stand’. The use of posture verbs to narrow denotational meanings has also been described as a posture verb function in a role that is highly reminiscent of noun classifier systems (see \cite{Aikhenvald2000}).

\ea \label{Yacopetti3}
\langinfo{Arrernte}{}{\cite[32]{Wilkins2006}} \\
\gll \textbf{Arne} 		yanhe-Ø 		arlpentye	\textbf{inte-me}. \\
\textbf{tree/stick} 	that(mid)-\textsc{nom} 	tall/long 	\textbf{lie-\textsc{npp}} \\
\trans ‘That stick is long.’ (lit. ‘That stick lies long.’)  \\
\z

\ea \label{Yacopetti4}
\langinfo{Arrernte}{}{\cite[32]{Wilkins2006}} \\
\gll \textbf{Arne} 		yanhe-Ø 		arlpentye 	\textbf{tne-me}. \\
\textbf{tree/stick} 	that(mid)-\textsc{nom} 	tall/long 	\textbf{stand-\textsc{npp}} \\
\trans ‘That tree is tall.’ (lit. ‘That tree stands long.’)	 \\
\z

Finally, posture verbs have also been identified as part of some specific diachronic pathways, namely in the emergence of verbal aspect—a process that has also been identified in languages outside of Australia (\cite{Kuteva1999}; \cite{Newman2002,Newman2002CrossLinguistic}). Most of the evidence for this within the Australian context primarily comes from languages spoken in the north of the continent. Specifically, \citet[239]{Reid2002} has identified that in serialised verb constructions, “stance verbs are, at least to some extent, bleached of posture/stance meanings, and typically convey ‘continuous’, or ‘progressive’ aspect”. Relevant languages that show evidence of this include: Ngan’gityemerri, Emmi (Western Daly, [amii1238]), Djapu (Yolngu, [djap1238]), Garrwa (Garrwan, [gara1269]; \cite[299]{Mushin2012}) and various Maningrida languages (see also a Gunwinyguan language spoken in the Maningrida region, Kunbarlang: \cite[284--288]{Kapitonov2021}). In the Diyari language (Pama-Nyungan, [dier1241]) spoken in the south of the continent, an auxiliary verb \textit{parra-} homophonous with a verb ‘to lie’ is used for recent past tense (\cite[95]{Austin1981}).

In sum, the existing literature provides ample indication that posture verbs have a wide range of functional capabilities within and across Australian languages when occurring in locative and existential contexts. The present study seeks to provide a detailed comparison of such variation as can be found in three languages for which posture verbs have been minimally described: Kune (Gunwinyguan), Kukatja (Pama-Nyungan) and Wik-Mungkan (Pama-Nyungan). 

\section{Methodology} \label{YacopettiSection2}
\subsection{Languages in the study} \label{YacopettiSection2.1}

The genetic groupings of the languages in the study are shown in \tabref{tab:YacopettiTable1}. Kukatja and Wik-Mungkan are both members of the widespread Pama-Nyungan language family but are spoken in the geographically disparate regions of Cape York (the far northeast of the continent) and the Tanami Desert region of Western Australia. They are only distantly related within the Pama-Nyungan family and are members of the Western Desert and Middle Paman subgroups respectively. Kune is a member of the Gunwinyguan language family, a wholly unrelated family spoken in Arnhem Land (\figref{YacopettiFigure1}).


\begin{table}
\fittable{
    \begin{tabular}{llll}
    \lsptoprule
        {Language} &  {Glottocode} &  {Language family} &  {Sub-grouping}\\
    \midrule
    Kukatja  & kuka1246 & Pama-Nyungan & Western Desert \\
       Wik-Mungkan & wikm1247 & Pama-Nyungan & Middle Paman \\
       Kune  & gune1238 & Gunwinyguan & — (Variety of Bininj Kunwok) \\
    \lspbottomrule
    \end{tabular}
    }
    \caption{Genetic affiliations of the focus languages}
    \label{tab:YacopettiTable1}
\end{table}

\begin{figure}
\includegraphics[width=.9\textwidth]{figures/Yacopetti_Fig1_new.jpg}
\caption{Map of languages in the study}
\label{YacopettiFigure1}
\end{figure}

The language sample was chosen for both novelty and convenience. All three languages, while having been described to varying levels of detail (see §\ref{YacopettiSection2.2}), have not been the focus of any systematic study with respect to copula predication or the role of posture verbs in locative or  existential predication. Typologically, all three languages are still being acquired as first languages by children. Importantly, the three authors have conducted extensive linguistic documentation within the relevant speech communities and have had the opportunity to carefully elucidate posture verb semantics via dedicated elicitation techniques (§\ref{YacopettiSection2.2}), going beyond what is typically used for the basis of description in published grammars. The three survey languages all share the property of making frequent use of posture verbs but vary along several dimensions with respect to how these posture verbs feature in locative-existential clause types and other types of copula constructions. Typologically, all three languages exhibit a number of properties characteristic of non-configurational languages, especially with respect to grammatically free word order, and elision of nominal arguments. While Wik-Mungkan and Kukatja are overwhelmingly suffixing in both nominal and verbal morphology, Kune, as a non-Pama-Nyungan language, has both prefixing and suffixing morphology and a significantly more complex verbal word. Argument relations in Wik-Mungkan are predominately encoded by case-marking on dependents while Kune predominately encodes argument relations within the verbal complex. Kukatja is distinct from both in exhibiting a mixed system of both dependent marking but also relying on cross-referencing of arguments in a second position clitic complex.

Of relevance to the present study are the commonalities and differences with respect to the organisation of the class of ‘verbs’ in each language. Kukatja has two classes of lexical items which are appropriately considered ‘verbal’ in nature. One lexical type (termed \textit{inflecting verbs} or just \textit{verbs}) is a small, closed class of lexemes that take TAM inflection according to four conjugation classes and whose members encode relatively generic or schematic semantic content. Another lexical type (termed \textit{preverbs}) is a larger open class of lexemes which do not inflect for TAM values but whose members may have quite specific verbal semantic content and do not share distributional or inflectional properties of nominals or adverbs. These two classes of verbal elements frequently combine to form a type of complex predicate, sometimes termed a \textit{preverb} or \textit{coverb} construction (see \cite{OsgarbyBowern2023} for a recent overview). Lexical items with postural semantics occur in both the ‘preverb’ and ‘verb’ word classes within Kukatja.\footnote{Inflecting verbs and/or complex verbs also occur in serialised constructions.}  By contrast, Wik-Mungkan has just one major verb class of which the posture verbs constitute members and does not make use of complex predication. For Kune, unlike the two Pama-Nyungan languages, verbal words are more polysynthetic in their structure, requiring -- at  minimum -- affixation of pronominal arguments and TAM inflections. The verb stem occupies but one slot in a complex verbal template which further permits nominal arguments to be incorporated into the verbal structure. Additionally, Kune has several serial verb constructions which may be recruited in aspectual or spatial functions.

\subsection{Data and analysis} \label{YacopettiSection2.2}

In all three languages, we analysed data from both existing and novel material. For these languages, we consulted the existing descriptions and analyses put forward by: \citet{Peile1971} and \citet{Valiquette1993} for Kukatja; \citet{Evans2003} for Kune; and \citet{KilhamEtAl1986} and \citet{Ray2021} for Wik-Mungkan. Novel data were also collected with a specific focus on postural semantics and locative clauses, all in collaboration with language experts over the course of 2020--2023. These included both direct elicitation as well as broader corpus data (mainly monologic narratives). As part of our elicitation methods, we employed the tools of DELOCA (\cite{Guerrero2022}) and the Topological Relations Picture Series (TRPS) (\cite{BowermanPeterson1992}). In DELOCA elicitation, consultants are shown videos and/or photo stimuli and are asked to describe either: where a selected object is located, or what action a person is performing. In many cases, consultants were asked to judge whether the use of more than one posture verb was admissible for a given scene. In TRPS, consultants are shown line drawings on cards and asked to explain where a figure is located. The use of these standardised, picture-prompted elicitation techniques allows for crosslinguistic comparison—see, for example, the descriptions in \citet{LevinsonWilkins2006GrammarsOfSpace}, or the typology presented by \citet{AmekaLevinson2007}. In each language, we conducted these tasks with at least four different speakers. Their responses were transcribed and coded according to the stimulus type (animacy of figure/ground, number of figure), and response (type of verb used, reduplication). 
\tabref{tab:YacopettiTable2} is an overview of the participants from whom data were collected using the standardized tools.\footnote{The three communities that contributed language data all share internal variation amongst their populations with respect to multilingualism. Generally, all communities have exposure to Standard Australian English but vary with respect to additional linguistic influences such as the degree of multilingualism in additional Australian Aboriginal languages or their degree of utilisation of contact varieties such as Kriol [krio1252] (see \cite{Dickson2023}). Data collection in each location focussed on just the single target language. Fully accounting for community-internal variation was not the aim of data collection or of this chapter.}

\begin{table}
    \fittable{
    \begin{tabular}{lll}
    \lsptoprule
        {Language} &  {DELOCA} &  {TRPS} \\
    \midrule
       Kukatja  &  4 speakers (2 male, 2 female) & 5 speakers (3 female, 2 male) \\
       Kune & 5 speakers (3 female, 2 male) & 10 speakers (8 female; 2 male) \\
       Wik-Mungkan & 4 speakers (3 male, 1 female) & 7 speakers (4 male, 3 female) \\
    \lspbottomrule
    \end{tabular}
    }
    \caption{Standardised elicitation tasks probing locative-existential predication}
    \label{tab:YacopettiTable2}
\end{table}

We supplemented these core data with additional spatial elicitation tasks in order to examine postural use in a slightly richer array of contexts. These additional data included: responses to the Frog Story (\cite{Mayer1969}); the Man and Tree task (cf. \cite{KnudsenEtAl2023}; \cite{LevinsonEtAl1992}; \cite{EnneverEtAl2024}; \cite{TerrillBurenhult2008}; \cite{Ennever2024}); and the Rotating Scene Machine (\cite{BlytheEtAl2022}). Where a posture verb was present, these utterances were included and coded in line with the DELOCA/TRPS data.

In the following section, we introduce the basic properties of posture verbs in each language and compare their semantic range (§\ref{YacopettiSection3.1}). In §\ref{YacopettiSection3.2}, we explore how posture verbs are used in the predication of location-existence, demonstrating how posture verb choice is sensitive to ontological properties of the figure; and how individual posture verbs are used in non-locational copula constructions. Discussion of the ambiguity between locative and existential meanings is withheld until §\ref{YacopettiSection3.3}. The results from the frequency analysis of posture verb usage across the standardised elicitation tasks are presented in §\ref{YacopettiSection3.4}.

\section{Posture verbs in Kukatja, Kune and Wik-Mungkan} \label{YacopettiSection3}
\subsection{Lexical semantic range} \label{YacopettiSection3.1}
\subsubsection{Lexical polysemy} \label{YacopettiSection3.1.1}

The forms that constitute the basic posture verbs in each language are shown in \tabref{tab:YacopettiTable3}. The Kukatja and Wik-Mungkan forms are given in their uninflected citation forms while the Kune forms are given in their verbal root forms.\footnote{Across the three languages, only the \textit{nyiin-} and \textit{nyina-} forms in Wik-Mungkan and Kukatja respectively have been demonstrated to be cognate, descending from a common proto-Pama-Nyungan form \textit{*nyiːna} (\cite[509--510]{Alpher2004}).} 

\begin{table}
    \begin{tabular}{llll}
    \lsptoprule
        {Language} &  {‘sit’} &  {‘stand’} &  {‘lie’} \\
    \midrule
       Kukatja  &  \textit{nyina-} & \textit{ngara-} & \textit{ngarri-} \\
       Kune & \textit{-ni} & \textit{-di} & \textit{-yo}\\
       Wik-Mungkan & \textit{nyiin-} & \textit{than-} & \textit{wun-} \\
    \lspbottomrule
    \end{tabular}
    \caption{Forms of the basic posture verbs in Kukatja, Kune and Wik-Mungkan}
    \label{tab:YacopettiTable3}
\end{table}

Posture verbs in all three languages belong to a lexical class of (intransitive) verbs. The core semantics of the ‘sit’, ‘stand’ and ‘lie’ verbs are generally consistent across the three languages, particularly when it comes to the postural meanings they ascribe to human figures. The semantic consistency of the compared verbs mirrors a general tendency to encode a threeway ‘sit’–‘stand’–‘lie’ distinction, which \citet[1]{Newman2022} considers to constitute “primary postures” or “basic, recurring features of human behaviour.”\footnote{The three languages do differ with respect to the form and/or derivation of dynamic posture predicates, used to describe the act of assuming a particular posture, i.e. predications of the kind realised in English by the expressions ‘stand up’, ‘sit down’, and ‘lie down’ but this is beyond the scope of this study. For language particular descriptions of this aspect of posture verbs, see \citet[181--184]{Ennever2024} for Kukatja, \citet[310-312]{Ray2021} for Wik-Mungkan, and \citet[379-380]{Evans2003} for Kune.} 

All the languages allow their posture verbs to be used without any accompanying locative expression simply to predicate the postural disposition of an animate subject NP. Examples from Wik-Mungkan of these types of expressions involving animate figures are presented in (\ref{Yacopetti5})–(\ref{Yacopetti7}).

\ea \label{Yacopetti5}
\langinfo{Wik-Mungkan\footnote{Word final -\textit{a} is generally analysed as ‘rhythmic juncture’ in Wik-Mungkan (see \cite[56--57]{Ray2021} and references therein) although its precise function is not well understood, nor agreed upon by analysts. Here we gloss it rather agnostically as ‘-a’.}}{}{Michella Pahimbaung; deloca\_mp\_230610} \\
\gll \textbf{Than$\sim$than-pul}-a.  \\
\textbf{\textsc{rdp}$\sim$stand-\textsc{3du.prs}}-a \\
\trans ‘The two are standing.’  \\
\z

\ea \label{Yacopetti6}
\langinfo{Wik-Mungkan}{}{Maree Kalkeeyorta; navigation-task\_mk\_230716} \\
\gll Waath-a \textbf{nyiin-Ø}. \\
crow-a	\textbf{sit-3\textsc{sg.pst}} \\
\trans ‘A crow sat.’  \\
\z

\ea \label{Yacopetti7}
\langinfo{Wik-Mungkan}{}{Rufus Namponan; deloca\_rn-rw\_230610} \\
\gll Nil 	\textbf{wun$\sim$wun-Ø}, 		a' 	ek'-Ø 		ngul. \\
3\textsc{sg}	\textbf{\textsc{rdp}$\sim$lie-3\textsc{sg.pst}}	and	get.up-3.\textsc{sg.pst}	then \\
\trans‘He was laying down, and then he got up.’  \\
\z

Even though the verbs for ‘sit’, ‘stand’, and ‘lie’ can be used in a corresponding fashion when predicating posture in all three languages, they pattern differently in terms of their semantic extensions. The distribution of polysemies across the class of verbs in the three languages is illustrated in the crosslinguistic semantic map shown in \figref{YacopettiFigure2}. Note that the semantic map includes an aspectual function (\textsc{durative}) specific to the ‘sit' verb in Kune. This role is restricted to its occurrence in a type of aspectual serial verb construction of the kind mentioned in §\ref{YacopettiSection1.2}—a function not discussed further in the present paper (see \cite[659]{Evans2003} for Kune-specific details).

\begin{figure}[ht!]
\includegraphics[width=\textwidth]{figures/Yacopetti_Fig2_new.png}
\captionsetup{width=0.9\hsize}
\caption{Comparative semantic map of ‘sit’, ‘stand’ and ‘lie’ verbs in Kukatja, Kune, and Wik-Mungkan}
\label{YacopettiFigure2}
\end{figure}

The semantic extensions identified here are not unusual for Australian languages and broadly align with many of the experientially-motivated semantic extensions identified by \citet[10--11]{Newman2022}. The ‘sit’ verbs tend to exhibit extensions invoking a durative component of the kind ‘dwell’, ‘reside’ which Newman hypothesises as a reflection of ‘sitting’ as a kind of “default” posture \citep[10]{Newman2022}. Kukatja and Kune, for example, follow this cross-linguistic tendency, utilising their respective ‘sit’ verbs to mean ‘live/reside’ (\ref{Yacopetti8} and \ref{Yacopetti9})  or ‘stay/remain’ (\ref{Yacopetti10}). By contrast ‘stand’ verbs are typically excluded from such extensions and are instead used for expressing temporary location. 

\ea \label{Yacopetti8}
\langinfo{Kukatja: reside}{}{TKU: PEILE\_A06-001932A: 1393.1} \\
\gll Tjitja	turru	pirti-ngka		\textbf{nyi-nin}.  \\
\textsc{ident}	bird	hollow-\textsc{loc}		\textbf{sit-\textsc{prs}} \\
\trans ‘This bird [the \textit{luurn}] lives in hollows.’  \\
\z

\ea \label{Yacopetti9}
\langinfo{Kune: reside}{}{Kamanj Michelle Martin; KLLC\_20211004a} \\
\gll Kondanj	ngarri-\textbf{ni}	bad	kondanj nga-djordm-inj,	Kune		darrme.  \\
here	1\textsc{aug.excl}-\textbf{sit.\textsc{npst}} \textsc{conj}	here	3\textsc{min}-grow\_up-\textsc{pst}	Kune	\textsc{loc} \\
\trans ‘We live here but I (also) grew up here, on Kune country.’  \\
\z

\ea \label{Yacopetti10}
\langinfo{Kukatja: stay/remain}{}{\cite[20]{Peile1971}} \\
\gll Ya-nku-marranpa=rna,	langa-kutjupa-rri-ngu=rna \textbf{nyi-nama}=lpi=rna.  \\
go-\textsc{irr-avert=1sg.sbj}		ear-other-\textsc{inch-pst=1sg.sbj}	\textbf{sit-\textsc{ipfv}}=then=\textsc{1sg.sbj} \\
\trans ‘I was going to come, but I changed my mind (lit. ‘I became the other ear’) and so I stayed.’  \\
\z

In Wik-Mungkan, however, the sense ‘live/reside’ is a feature of the ‘lie’ verb, \textit{wun} (\ref{Yacopetti11}). Like ‘sitting’, ‘lying’ postures are similarly able to be maintained for a long period of time and are experientially associated with the body at rest.

\ea \label{Yacopetti11}
\langinfo{Wik-Mungkan: reside}{}{\cite[240]{KilhamEtAl1986}} \\
\gll Wachan		\textbf{wun-an}	thantt=am. \\
long\_way	\textbf{lie-\textsc{3sg.prs}}	\textsc{3pl.dat=abl}   \\
\trans ‘(He) lives a long way from them.’   \\
\z

In both Wik-Mungkan (\ref{Yacopetti12}) and Kune (\ref{Yacopetti13}), the ‘lie’ verb is polysemous with a sense of ‘sleep’, again reflecting intuitive experiential associations.

\ea \label{Yacopetti12}
\langinfo{Wik-Mungkan: sleep}{}{Perry Yunkaporta; space-elicitation\_py\_230411} \\
\gll My 	cousin 	nan	\textbf{wun-Ø}. \\
my	cousin		\textsc{med}	\textbf{lie-}\textsc{\textbf{3sg.pst}} \\
\trans ‘My cousin slept there.’   \\
\z

\ea \label{Yacopetti13}
\langinfo{Kune: sleep}{}{\cite[156]{Evans2003}} \\
\gll Karri-\textbf{yo}			ku-kawadj. \\
1\textsc{aug.incl}-\textbf{lie.\textsc{npst}}	\textsc{loc}-sand \\
\trans ‘We sleep on the sand.’   \\
\z

Interestingly in Kukatja, there is no equivalent extension of \textit{ngarri-} ‘lie’. Instead, a separate verbalised predicate\textit{ nyarli-rri-} ‘sleep-\textsc{inch}’ (from a nominal predicate \textit{nyarli} ‘(a)sleep’) is used to refer to the activity of sleeping (\ref{Yacopetti14}). Nevertheless, \textit{ngarri-} ‘lie’ does extend to the narrower sense of ‘(a night’s) camp’, or ‘camp overnight’, a usage that features heavily in travelogues, such an example is shown in (\ref{Yacopetti15}). In Kune and Wik-Mungkan the semantic space of ‘lie’, ‘sleep’ and ‘stay/camp overnight’ is covered by just the one posture verb, ‘lie’. 

\ea \label{Yacopetti14}
\langinfo{Kukatja: ‘sleep’}{}{ALL: TEN2-2022\_025-03: 1069.800} \\
\gll Ngatja	kunyarr-pa	\textbf{nyarli-rri-n} waru-wana.  \\
\textsc{prox}	dog-\textsc{ep}	\textbf{sleep-\textsc{inch-prs}}	fire-\textsc{perl}\\
\trans ‘This dog is sleeping by the fire.’  \\
\z

\ea \label{Yacopetti15}
\langinfo{Kukatja: ‘camp (overnight)’}{}{MSK: Manungka\_01-035158: 1839.553} \\
\gll Tjii-tjanu:::	Laka-kutu.	Laka-ngka=latju \textbf{ngarri-rnu}.  \\
\textsc{ident-abl}	Laka-\textsc{all}	Laka-\textsc{loc=1pl.excl.sbj}	\textbf{lie-\textsc{pst}}	\\
\trans‘After that (we went) to Laka. We stayed/camped overnight at Laka.’  \\
\z

Lastly, both ‘sit’ and ‘stand’ verbs in Kukatja can be used with the sense ‘wait (for)’ in combination with a dative argument as in (\ref{Yacopetti16}) and (\ref{Yacopetti17}), and the ‘stand’ verb has an additional lexical sense: ‘stop’. In Kune, the ‘stand’ verb can similarly mean ‘stop’ or ‘wait’ (see, e.g., \cite[685]{Evans2003}). Neither of these senses are covered by any posture verb in Wik-Mungkan and Wik-Mungkan is further notable for not demonstrating any clear instances of polysemy for the ‘sit’ member.

\ea \label{Yacopetti16}
\langinfo{Kukatja: ‘wait’}{}{TKU: PEILE\_A06-001932A: 1393.1} \\
\gll \textbf{Nyi-ninpa}=rna=ra	tjarlu-ku.  \\
\textbf{sit}-\textsc{\textbf{prs}=1sg.sbj=3sg.d}	big-\textsc{dat}	\\
\trans ‘I am waiting for the boss.’  \\
\z

\ea \label{Yacopetti17}
\langinfo{Kukatja: ‘wait’}{}{NKI: PEILE\_A06\_001933B: 581.774} \\
\gll Wiya=tju=n		\textbf{ngara-ku}!  \\
\textsc{neg=1sg.d=2sg.sbj}	\textbf{stand-\textsc{pot}} \\
\trans ‘Don’t wait for me!’   \\
\z

\subsubsection{Posture verbs and wider systems of postural specification} \label{YacopettiSection3.1.2}

Before proceeding with the analysis of posture verbs in locative-existential and other copula constructions, we first provide a brief illustration of how the three survey languages express postures other than ‘sit’, ‘stand’ or ‘lie’. Here we find quite a diversity of strategies across the verbal and nominal domain. However, barring a single exception (\textit{werndi} ‘be elevated’ in Kune), none of these strategies described are standardly recruited in locative-existential functions (without a specifically postural meaning), nor in copula functions.
Within the verbal domain, Kune is notable for having several morphologically complex verb stems that involve the posture verb thematics, especially \textit{di} ‘stand’, as in \textit{djab-di} [upright-stand] ‘be upright’ (\ref{Yacopetti18}) or \textit{barddurrungkulh-di} [ground\_with\_knees-stand] ‘kneel’ (\ref{Yacopetti19})--see also \citet[339]{Evans2003}.

\ea \label{Yacopetti18}
\langinfo{Kune}{}{Belinj Jay Jurrupula Rostron: KLLC\_20220807h} \\
\gll Kun-dulk	ka-\textbf{djab-di}.  \\
\textsc{iv}-tree		3\textsc{min}-\textbf{upright-stand.\textsc{npst}} \\
\trans ‘The tree is upright.’  \\
\z

\ea \label{Yacopetti19}
\langinfo{Kune}{}{Bulanj Russell Brian: KLLC\_20230621a\_RuB} \\
\gll Kabini-\textbf{barddurrungkulh-di}.  \\
3\textsc{ua}-\textbf{ground\_with\_knees-stand.\textsc{npst}} \\
\trans ‘They are kneeling.’  \\
\z

In (\ref{Yacopetti18}), the postural meaning of ‘be upright’ is encoded by the combination of verbal prepound \textit{djab-} and verbal thematic \textit{-di} while in (\ref{Yacopetti19}), the postural meaning of ‘kneel’ is compositionally formed via the stand verb \textit{-di} in combination with the lexicalised verbal prepound \textit{barddurrungkulh-}, involving the morpheme \textit{bard-} ‘knee’. The Kune complex verb \textit{-werndi} ‘be elevated’ (\ref{Yacopetti18}) will be returned to in §\ref{YacopettiSection3.4} since it alone is recruited in copula function and is even treated as a ‘stance’ verb by \citet[339]{Evans2003}, despite its semantics being primarily positional rather than postural.


Kukatja has both complex verbs, involving postural preverbs (e.g. \textit{wartu-} ‘prostrate’), as well as secondary predicate and adjunct based strategies. Examples are shown in (\ref{Yacopetti20})–(\ref{Yacopetti22}).

\ea \label{Yacopetti20}
\langinfo{Kukatja}{}{PSM: TEN2-2023\_003-01: 573.796} \\
\gll (Tutju	lamparn	manga)	\textbf{wartu-ngarri-npa}.  \\
woman	small		girl		\textbf{prostrate-lie-\textsc{prs}}	\\
\trans ‘The little girl is lying face-down.’  \\
\z

\ea \label{Yacopetti21}
\langinfo{Kukatja}{}{MMN: TEN2-2023\_004-01: 676.227} \\
\gll Ngatja	tutju		\textbf{nyina-ti-n}		\textbf{karlitjalpi}.  \\
\textsc{prox}	woman	\textbf{sit-\textsc{act-prs}}		\textbf{cross\_legged} \\
\trans ‘This woman is sitting down cross-legged.’ (\textit{karli.tjalpi} lit. ‘boomerang awkward’)   \\
\z

\ea \label{Yacopetti22}
\langinfo{Kukatja}{}{GLE: TEN2-2023-001-02: 98.1} \\
\gll Tutju	lamparn-pa	\textbf{murti-kurlu}	\textbf{ngari-n}.  \\
woman	small-\textsc{ep}	\textbf{knee-\textsc{prop}}	\textbf{stand-\textsc{prs}} \\
\trans ‘This young woman is kneeling (lit. ‘standing with knees.’).’   \\
\z

In (\ref{Yacopetti20}), the complex postural meaning of ‘lying face-down’ is encoded in the preverb-verb combination \textit{wartu-ngarri-}. In (\ref{Yacopetti21}), a nominal \textit{karlitjalpi} ‘cross-legged’ is used as a resultative secondary predicate assigning a resultant postural state to the subject figure brought about by the dynamic main predicate \textit{nyina-ti-} ‘sit down/become seated’. Finally in (\ref{Yacopetti22}), a proprietive case marked nominal expression, \textit{murti-kurlu} ‘knee-\textsc{prop}’, does not invoke a posture as a resultant state of a dynamic predicate but rather implicates a body part (specifically ‘knees’) as being involved in the postural event of the main verb \textit{ngari-} ‘stand’, an idiomatic way of expressing ‘kneeling’.\footnote{Proprietive (\ref{Yacopetti20}) and instrumental (\ref{Yacopetti21}) case-marked NPs in many Australian languages have (secondary) predicative functions as depictives wherein the case-marked nominal serves to characterise a property of the subject NP throughout the duration of the verbal event.} 

Wik-Mungkan is a little more restricted than either Kune or Kukatja in its lexical and constructional resources for specifying particular postures but exhibits a similar strategy to the body part strategy of Kukatja shown in (\ref{Yacopetti22}), through case-marked body parts as shown in (\ref{Yacopetti23}). Specifically, the body-part \textit{pungk} ‘knee’ is case-marked with the polyfunctional instrumental/locative case suffix and has the syntactic role of an adjunct modifier, external to the verb phrase.

\ea \label{Yacopetti23}
\langinfo{Wik-Mungkan}{}{Rosalie Wolmby; deloca\_rn-rw\_230610} \\
\gll Nil 		\textbf{pungk=ang} 		nyiin-an. \\
\textsc{3sg.nom}	\textbf{knee=\textsc{ins/loc}} sit-\textsc{3sg.prs} \\
\trans ‘He is kneeling/sits with knees.’   \\
\z

\subsection{Posture verbs in the predication of location and existence} \label{YacopettiSection3.2}
\subsubsection{Overview of posture verbs and construction} \label{YacopettiSection3.2.1}

In all three languages, the posture verbs ‘sit’, ‘stand’ and ‘lie’ are used as the predicative element in locative-existential constructions, coding a spatial relation between a figure and a ground, i.e. ‘be (at)’. While all three languages pattern similarly with respect to the use of posture verbs in this function with animate figures (§\ref{YacopettiSection3.2.2}), the languages show differences when one considers inanimate figures (§\ref{YacopettiSection3.2.3}) and their use in copula constructions beyond locative-existentials (§\ref{YacopettiSection3.2.4}). This information is shown in \tabref{tab:YacopettiTable4}. Note that in the final column ‘copula functions’ refers to contexts in which no locative-existential meaning is present and the copula links a subject with a (non-spatial) complement.


\begin{sidewaystable}
\small
\begin{tabularx}{\textwidth}{llQ@{}ll@{}}
\lsptoprule
{{Language}} & \multirow{2}{40mm}{Posture verb (gloss and \mbox{function as stance predicate)}} & \multicolumn{2}{c}{{{Locative-existential constructions}}} & \multirow{2}{13mm}{{Copula}  {functions}} \\
 \cmidrule{3-4}
 &  & {{Animate figure}} & {Inanimate figure} &  \\
 \midrule
 {Kukatja} & \textit{nyina-} ‘sit’ & {‘be in seated position (at)’} & \textbf{‘be (at)’ (unmarked for stance)} & \textbf{COP} \\
 \tablevspace
 & \textit{ngara-} ‘stand’ & {‘be in vertical position (at)’} & ‘be (at)’ (of vertical figure) & -- \\
 \tablevspace
 & \textit{ngarri-} ‘lie’ & {‘be in horizontal position (at)’} & ‘be (at)’ (of horizontal figure) & -- \\
 \midrule
 {Kune} & \textit{-ni} ‘sit’ & {‘be in seated position (at)’} & \textsc{\textit{rare}} & -- \\
 \tablevspace
 & \textit{-di} ‘stand’ & {‘be in upright or vertical position (at)’ (supported by legs)} & \textbf{‘be (at)’ (unmarked for stance)} & \textbf{COP} \\
 &  & {‘be (at)’ (unmarked for stance)} &  &  \\
 \tablevspace
 & \textit{-yo} ‘lie’\footnote{A fourth Kune verb \textit{werndi} ‘be up high’ (\textit{barndi} in related dialects) is included by \citet[339]{Evans2003} as a posture verb. It is not included in the table here since our focus is primarily on the semantically defined set of three (sit, stand and lie). However, see §\ref{YacopettiSection3.4} for its quantified use in the experimental tasks.} & {‘be in horizontal position (at)’} & ‘be (at)’ (of horizontal figure) & -- \\
 \midrule
 {Wik-Mungkan} & \textit{nyiin} ‘sit’ & {‘be in seated position (at)’  (also crouching  with support from feet)} & \textsc{\textit{rare}} & -- \\
 \tablevspace
 & \textit{than} ‘stand’ & {‘be in upright position (at)’} & ‘be (at)’ (of upright figure) & -- \\
 \tablevspace
 & \textit{wun} ‘lie’ & {‘be in horizontal position (at)’} & \textbf{‘be (at)’ (unmarked for stance)} & \textbf{COP} \\
 &  & {‘be (at)’ (unmarked for stance)} &  &  \\
 \lspbottomrule
\end{tabularx}
    \caption{Functions of basic posture verbs across both construction and figure type}
    \label{tab:YacopettiTable4} 
\end{sidewaystable}

The key differences in \tabref{tab:YacopettiTable4} relate to the choice of posture verb that is recruited both as a default verb (in locative-existentials) and in fully grammaticalised copula function in other copula constructions. For Kukatja the member is \textit{nyina-} ‘sit’, for Kune it is \textit{-di} ‘stand’ and for Wik-Mungkan, it is \textit{wun-} ‘lie’.

\subsubsection{Locative-existential predications with animate figures} \label{YacopettiSection3.2.2}

For all three languages, when the figure referent is animate, each of the core posture verbs can be used to describe a postural configuration that in some ways aligns with the human prototype. Contrastive examples of posture verbs with animates were given for Wik-Mungkan in (\ref{Yacopetti5})–(\ref{Yacopetti7}) and we provide additional examples of such stance-focussed predicates for the three languages in (\ref{Yacopetti24})–(\ref{Yacopetti26}). In each example, the focus of the proposition is that the figure subject is in a specific posture.

\ea \label{Yacopetti24}
\langinfo{Kukatja}{}{\cite[164]{Ennever2024}} \\
\gll Ngayunpa=la=nta	\textbf{nyi-nin}	kinti	nyuntu-ngka.  \\
\textsc{1pl=1pl.excl.sbj=2sg.obj/l}	\textbf{sit-\textsc{prs}}	close	\textsc{2sg-loc} \\
\trans ‘We are sitting close to you.’   \\
\z

\ea \label{Yacopetti25}
\langinfo{Kune}{}{Bulanj Simon Brian; KLLC\_20220709b} \\
\gll Kun-rak-kah	ka-\textbf{ni$\sim$ni}.  \\
\textsc{iv}-fire-\textsc{loc}	3\textsc{min-\textbf{rdp}}$\sim$\textbf{sit.}\textsc{\textbf{npst}} \\
\trans ‘He (the man) is sitting by the fire.’  \\
\z

\ea \label{Yacopetti26}
\langinfo{Wik-Mungkan}{}{Kay Tybingoompa; trps\_kt\_230416} \\
\gll Ku'waak pek$\sim$pek=ang=an 	  table=ang=an	\textbf{nyiin-Ø}. \\
cat	\textsc{rdp}$\sim$below=\textsc{loc=def}  table=\textsc{loc=def}	\textbf{sit-\textsc{3sg.pst}} \\
\trans ‘A cat sat under the table.’   \\
\z

In examples (\ref{Yacopetti24})–(\ref{Yacopetti26}), the respective ‘sit’ verb in each language is used in an expression of a locational relation. In all the languages, the chosen verb is reflecting an actual postural configuration of the figure. We can contrast, for example, the sitting posture of the animate figure ‘the cat’ in (\ref{Yacopetti26}) with the description of a dog lying down inside a dog house in (\ref{Yacopetti27}). 

\ea \label{Yacopetti27}
\langinfo{Wik-Mungkan}{}{Perry Yunkaporta; trps\_py\_221116} \\
\gll Ku'=an 	awuuch 	many=ang=an 	\textbf{wun-Ø-a}. \\
dog=\textsc{def}	house		small=\textsc{loc=def}	\textbf{lie-3\textsc{sg.pst}-a} \\
\trans ‘The dog lay inside the small house.’   \\
\z

The distinct postural specification of the dog in (\ref{Yacopetti27}) as compared to that of the cat (\ref{Yacopetti26}) is accomplished through the use of a contrastive postural verb \textit{wun-} ‘lie’ rather than \textit{nyiin-} ‘sit’. In fact, in all three languages, this is the preferred method for contrasting basic ‘sitting’, ‘standing’ and ‘lying’ postures for all animate figures (not just humans).

\subsubsection{Locative-existential predications with inanimate figures} \label{YacopettiSection3.2.3}

Moving to inanimate figures, the situation is a little more complex. All three posture verbs are attested in use with inanimate figures in each language. However, each language differs with respect to which posture verb is most commonly used: the preferred member is \textit{nyina-} ‘sit’ in Kukatja, \textit{-di} ‘stand’ in Kune, and \textit{wun-} ‘lie’ in Wik-Mungkan—see examples (\ref{Yacopetti28})–(\ref{Yacopetti30}). In what we might call a ‘default’ context, the posture verb primarily serves as a predicational element in the locative construction. In Kukatja and Wik-Mungkan, the verb simply provides a host for TAM marking as shown in (\ref{Yacopetti28}) and (\ref{Yacopetti30}). In Kune, with its more polysynthetic word structure, the posture verb can oftentimes combine with an incorporated nominal expressing the ground relation—such as \textit{kolh(-no)} ‘water’ in (\ref{Yacopetti29})—and further hosts a prefixed subject marker expressing the figure.

\ea \label{Yacopetti28}
\langinfo{Kukatja}{}{ALL: TEN2-2022\_022-01: 913.08} \\
\gll Sand-pa=lu 	\textbf{nyi-nin}	motika-ngka.  \\
sand-\textsc{ep=sg.l} 	\textbf{sit-\textsc{prs}}	car-\textsc{loc} \\
\trans ‘Sand is in the vehicle.’   \\
\z

\ea \label{Yacopetti29}
\langinfo{Kune}{}{Bulanj Simon Brian; KLLC\_20220711b} \\
\gll Ka-kolh-\textbf{di}.  \\
3\textsc{min}-water-\textbf{stand.\textsc{npst}}	 \\
\trans  ‘The water is in there.’  \\
\z

\ea \label{Yacopetti30}
\langinfo{Wik-Mungkan}{}{Beulah Pootchemunka; trps\_bp\_230412} \\
\gll Cup=an 	an=an 	table=ang=an 	\textbf{wun-Ø}. \\
cup=\textsc{def}	\textsc{dist=def}	table=\textsc{loc=def}	\textbf{lie-\textsc{3sg.pst}} \\
\trans ‘The cup was on that table.’   \\
\z

The question that arises from data such as (\ref{Yacopetti28})–(\ref{Yacopetti30}) is to what extent the ‘posture’ verbs being used here represent semantically-contentful predicates or whether they can be said to be entirely bleached of their semantic makeup. We find evidence in all three languages that there is not a simple categorical answer to this but rather each of the verbs can be used with varying levels of semantic-specificity along a cline from ‘stance-specified intransitive verbs’ through to ‘semantically null’ or ‘dummy’ copula. We also find evidence that these items are likely interpreted at the pragmatic level based on properties of the figure, properties of the ground, use of gesture, and the wider discourse context. 

We first justify this position by demonstrating that the default posture verbs can be used at ends of this functional continuum. On one end, we have already seen that with animate figures, the posture verbs are used contrastively to assert human or human-like postures (see Wik-Mungkan examples in (\ref{Yacopetti5})–(\ref{Yacopetti7}) and §\ref{YacopettiSection3.2.2}). At the other end of the spectrum, all three languages at least show some evidence of posture verbs being used in contexts where their postural semantic content is at odds with the postural properties of the figure being located (posture verbs used in non-locative copula constructions are also described in §\ref{YacopettiSection3.2.4}). This constitutes reasonable evidence that such terms can be used, not in their fully semantically specified sense in locative-existential predication, but in a purely relational role. In (\ref{Yacopetti31}), a Wik-Mungkan speaker responded to the pictorial prompt of an upright bottle on a table with a locative expression involving the ``default" posture verb \textit{wun-} ‘lie’. We can say here that the role of \textit{wun-} is simply to serve as the predicational element ‘be (at)’ in a locative construction.

\ea \label{Yacopetti31}
\langinfo{Wik-Mungkan}{}{Perry Yunkaporta; space-elicitation\_py\_230411} \\
\gll Chang=an-a 	wonk-pal	kuuw	an-a	\textbf{wun-Ø}. \\
water=\textsc{def}-a	side-\textsc{cis.loc}	west	\textsc{dist}-a	\textbf{lie-3\textsc{sg.pst}} \\
\trans ‘The water bottle is towards that west side.’   \\
\z

Default uses of copula verbs such as \textit{wun-} in (\ref{Yacopetti31}) are not uncommon with inanimate figures in Wik-Mungkan. While such default uses rarely occur with animate figures, they do crop up in elicitation and wider corpora. An example of the stance meaning being bleached in the context of predicating location of an animate figure in Wik-Mungkan is shown in (\ref{Yacopetti32}), describing the scene in \figref{YacopettiFigure3}.

\begin{figure}[ht!]
\includegraphics[width=0.3\textwidth]{figures/Yacopetti-Fig3.png}
\captionsetup{width=0.9\hsize}
\caption{Picture 34 from the Topological Relations Picture Series (\cite{BowermanPeterson1992})}
\label{YacopettiFigure3}
\end{figure}

\ea \label{Yacopetti32}
\langinfo{Wik-Mungkan}{}{Beulah Pootchemunka; trps\_bp\_230412} \\
\gll Keny=ang	\textbf{wun}-\textbf{an}-a	ngeen=ang=an, \textbf{than$\sim$than}-\textbf{an}. \\
above=\textsc{loc}	\textbf{lie}-\textbf{\textsc{3sg.prs}}-a	something=\textsc{loc=def}, \textbf{\textsc{rdp}$\sim$stand}-\textbf{\textsc{3sg.prs}} \\
\trans ‘(The man) is (lit. lies) on top of the something, (he) is standing.’   \\
\z

In the initial main clause in (\ref{Yacopetti32}) \textit{wun} ‘lie’ is used in copula function, while in the prosodically separated clause, (reduplicated) \textit{than} is used to assert the actual posture (‘stand’) of the man being located on top of the building. This kind of flexibility in the deployment of posture verbs across the postural-locational semantic space was also revealed in meta-discussion with speakers in the course of elicitation. For example, following the elicitation of the copula use of \textit{wun-} in (\ref{Yacopetti31}), later conferral with the speaker revealed that \textit{than-} ‘stand’ was still considered equally appropriate for such a locative scene, but instead involved a shift in focus to the bodily (or more accurately ‘geometric’) stance of the figure itself (in line with characterisations represented in \tabref{tab:YacopettiTable4}). This shows how the bleaching of the postural semantics of \textit{wun-} ‘lie’ verb does not preclude a speaker from using a non-prototypical posture verb in order to emphasise, contrast or pick out a certain property of the figure. In other words: while movable inanimate objects are generally described with \textit{wun-} ‘lie’ in locative expressions in Wik-Mungkan, if \textit{than} ‘stand’ is used, the pragmatic consequence is a focus on an upright configuration.
Very similar situations were found to hold for both Kukatja and Kune. In Kune, while \textit{-di} is the unmarked posture verb in locative-existential predications involving an inanimate figure, we nevertheless find data of the kind shown in (\ref{Yacopetti33}) where \textit{-yo} ‘lie’ is used instead. 

\ea \label{Yacopetti33}
\langinfo{Kune}{}{Bulanjdjan Risselle Brian; KLLC\_20210917f} \\
\gll Spun	ka-\textbf{yo}$\sim$\textbf{yo}	kanjdji	kun-madj.  \\
spoon	3\textsc{min-\textbf{rdp}}$\sim$\textbf{lie.}\textsc{\textbf{npst}}	under		\textsc{iv}-cloth	 \\
\trans  ‘The spoon is (lying) underneath the napkin.’  \\
\z

Remarking on data of the kind shown in (\ref{Yacopetti33}), Evans has made the following parallel observations regarding locative-existential clauses in Kune, emphasising the flexibility to use other (non-default) copulae in the language, such as \textit{barndi} ‘be high’:

\begin{quote}
There is some room to stress salient facts of the situation: normally one says of fruit in a tree that are not too high up, \textit{kamimdi} [it-fruit-stands] ‘there are fruit’, but if one wanted to stress how high up they were one could say \textit{kamimbarndi} [it-fruit-be\_high] ‘there are fruit up high’. (\cite[561]{Evans2003})
\end{quote}

Such flexibility in the deployment of posture verbs in locative-existential constructions has been noted cross-linguistically. As \citet[7]{Newman2022} describes, “the use of the posture morphemes in locative/existential constructions may not completely displace the original posture meaning”. This is also the case in the languages investigated here. A working hypothesis is that the chosen posture verb unmarked for stance in locative predication types is indicative of a general level of bleaching of the given verb. This would result in the demands of the figure object semantics being less strict. Indeed, as we shall see, the bleached verb is the one used for copula predication in which a postural meaning is entirely absent for two of the languages. 

\subsubsection{Posture verb choice and polysemous expressions for the figure} \label{YacopettiSection3.2.3.1}

While we saw in the previous sections that posture verb selection with inanimates show some flexibility, there is one specific context in which we find some evidence that posture verb choice has a contrastive, semantic role. This is primarily in the disambiguation of polysemous nominal expressions (as noted in the Australian context by \citealt{Wilkins2006} for Arrernte, for instance). This function has only been observed for scenes involving inanimate figure subjects. For example, the polysemous noun \textit{yuk} in Wik-Mungkan can both mean ‘tree’, ‘stick’, and is also a generic classifier noun for ‘stuff’. Examples  (\ref{Yacopetti34}) and  (\ref{Yacopetti35}) demonstrate how the posture verbs can be used to align and thus disambiguate the sense-type of \textit{yuk} that is being described.

\ea \label{Yacopetti34}
\langinfo{Wik-Mungkan}{}{Dennis Wolmby; deloca\_dw\_230620} \\
\gll \textbf{Yuk}	many	an=an		bowl	al=ant=ang=an \textbf{wun$\sim$wun-Ø}. \\
\textbf{tree}	small	\textsc{dist=def}	bowl	\textsc{dist=dat=loc=def} \textbf{\textsc{rdp}$\sim$lie-3\textsc{sg.pst}} \\
\trans ‘That little stick was lying in the bowl there.’   \\
\z

\ea \label{Yacopetti35}
\langinfo{Wik-Mungkan}{}{Kay Tybingoompa: trps\_kt\_230416} \\
\gll \textbf{Yuk}=an 		an=an-a 	\textbf{than-Ø} yoik=ang=an-a. \\
\textbf{tree}=\textsc{def}	\textsc{dist=def}-a	\textbf{stand-3\textsc{sg.pst}} mountain=\textsc{loc=def}-a  \\
\trans ‘That tree stood on the mountain.’   \\
\z

Here, \textit{than-} ‘stand’ is used for a vertical inanimate figure, while \textit{wun-} ‘lie’ is used for a horizontal figure. \textit{Wun-} ‘lie’ is the posture verb appropriate for any inanimate figure and especially one arranged or distributed horizontally on the ground. These examples demonstrate one function of the posture verb selection. The ‘classifying function’ of posture verbs evident in contrastive examples like (\ref{Yacopetti34}) and (\ref{Yacopetti35}) relies on the fact that the posture verbs themselves are not completely bleached of their semantic content in locative-existential function but can be sensitive to  geometric properties of inanimate figures.

\subsubsection{Beyond locative-existential predication} \label{YacopettiSection3.2.4}

All three languages have just one posture verb which is used as a copula in construction types that do not encode either existence or location. These represent the most ‘bleached’ functions of these lexical items since the postural semantics are wholly absent and the copula simply serves to link a subject relation to a subject complement. The argument structure of the copula construction in these cases involves two unmarked (absolutive) arguments, or it may involve an unmarked subject and another argument in a semantic case. Examples are (\ref{Yacopetti36})–(\ref{Yacopetti37}).

\ea \label{Yacopetti36}
\langinfo{Kune}{}{Bulanj Russell Brian; KLLC\_20230621a\_RuB} \\
\gll Laid	on	ka-\textbf{di}.  \\
light	on	3\textsc{min}-\textbf{be.\textsc{npst}}	 \\
\trans  ‘The light is on.’  \\
\z

\ea \label{Yacopetti37}
\langinfo{Kukatja}{}{\cite[27]{GreeneEtAl1992}} \\
\gll Ngaanpa=ya	\textbf{lukuti-yuru}	\textbf{nyi-nama}.  \\
3\textsc{pl=3pl.sbj}	\textbf{grub-\textsc{sembl}}	\textbf{be-\textsc{ipfv.pst}} \\
\trans ‘These (men) were like witchetty grubs.’   \\
\z

In (\ref{Yacopetti36}), the underlined elements represent codeswitches from Kriol within a Kune matrix clause. The nominal stative predicate \textit{on} ‘on’ combines with the main clause predicate, the Kune verb \textit{-di} ‘be (stand)', expressing the stative predicate ‘be on’. The subject of the predication is the Kriol-sourced \textit{laid} ‘light’ which shows agreement with the Kune verb via the third person minimal subject prefix \textit{ka-}. The clause lacks any invocation of postural semantics and is best analysed as involving a ‘copula’ use of \textit{-di}. In the Kukatja example in (\ref{Yacopetti37}), the predicational head is the semblative case-marked nominal expression \textit{lukuti-yuru} ‘like witchetty grubs’ which is linked via the copula \textit{nyi-nama} ‘be(sit)-\textsc{ipfv.pst}’ to the subject relation instantiated by the demonstrative pronoun \textit{ngaanpa} ‘these ones’. The subject is further cross-referenced by the second position subject enclitic \textit{=ya}.

Example (\ref{Yacopetti38}) shows a copula use of \textit{wun-} ‘lie’ in Wik-Mungkan. The copula hosts the portmanteau of optative verbal inflection and subject person/number, specifying the subject and temporal/modal conditions of the ascriptive nominal predicate \textit{meech} ‘(be) hungry’.

\ea \label{Yacopetti38}
\langinfo{Wik-Mungkan}{}{Kay Tybingoompa, fieldnotes\_240129} \\
\gll Meech 	\textbf{wun-iyin}! \\
hungry	\textbf{be-1\textsc{pl.excl.opt}}  \\
\trans ‘We might go hungry!’   \\
\z

\subsection{On the ambiguity of locative and existential predication} \label{YacopettiSection3.3}

So far we have focussed on the role posture verbs play in predicating a particular stance to a figure or predicating a location to a figure. An as yet unaddressed and complicating factor is that the same types of constructions used for location are also used in the study languages for predicating ‘existence’. Here we are referring to the meaning differences realised in English between the clauses (a) and (b) in (\ref{Yacopetti39}).

\ea \label{Yacopetti39}
\ea \label{Yacopetti39a}
Location predication \\
\textit{The cat is on the table.}
\ex \label{Yacopetti39b}
Existential predication \\
\textit{There is a cat on the table.}
\z
\z

The terminology in the literature for distinguishing predications of the sort shown in (\ref{Yacopetti39}) is rather fraught. Since languages vary widely with respect to their encoding of the semantic differences between (\ref{Yacopetti39a}) and (\ref{Yacopetti39b}), some of the typological literature has instead attempted to ground the differences not in a single structural property but instead through a distinction in cognitive perspectivisation: the former involving perspectivisation starting from the figure; and the latter involving perspectivisation starting from the ground (see \cite{Creissels2019}). Ultimately, the differences between these two relate to information packaging and the respective salience of figure and ground relations. \citet{HaspelmathNonverbal} argues that constructions of the kind (a) and (b) in (\ref{Yacopetti39}) can be typological defined on the basis of  formal definite marking of the figure (in the former and not the latter) presumably since this provides a sufficient cross-linguistic metric for information structure. Haspelmath terms the constructions so defined ‘predlocative’ and ‘existential’ respectively.\footnote{Other analysts prefer the terms \textsc{plain-locational} and \textsc{inverse-locational} (after \citealt{Creissels2013,Creissels2019}), reflecting the salience of the figure and ground respectively and reserve the term ‘existential’ clauses for those which do not predicate a location to a figure, i.e. wholly non-locative existential clauses of the kind instantiated by the English ‘Aliens exist’ (\cite{HaspelmathNonverbal} refers to these as ‘hyparctics’).}  Word order (as in English) may also play a role. However not all languages encode information structure in the same way and so morphosyntactic based distinctions may (understandably) not result in meaningful distinctions in every language.

For the languages in the present survey, two of them (Kukatja and Kune) lack any dedicated marking of definiteness, although demonstratives in these languages can have determining roles amongst their various other functions (more generally specifying identifiability based on distance relations or anaphoricity).\footnote{See \citet{Louagie2016} for a typological discussion on the status of determining elements in Australian Aboriginal languages more generally.}  None of the three surveyed languages show a consistent correspondence between word order and locative \textit{vs.} existential interpretations. Since the posture verbs involved in these constructions can also have stance-predication functions, we are in fact left with a threeway ambiguity in the predication of stance, location and existence in two of the surveyed languages, Kune and Kukatja. The argument structures of the three closely aligned clause types are presented in \tabref{tab:YacopettiTable5}. Note that for existential clauses either the verb or the locational adjunct can go unexpressed but not both. Wik-Mungkan only differs from Kukatja and Kune insofar as existential clauses can be distinguished based on the presence of definite marking of the figure in locative and stance-focussed clause-types, and lack of definite marking in existential clauses.


\begin{table}[t]
    \begin{tabularx}{\textwidth}{XXXl}
    \lsptoprule
     {Clause type} & \multicolumn{3}{c}{ {Argument structure}} \\
     \cmidrule{2-4}
         & <Figure> & <Ground> &  \\
    \midrule
    stance-focussed & subject & (adjunct) & verb \\
    locative  & subject & complement\footnote{Unmarked, or marked with spatial case (e.g. locative, perlative, etc.)} & (verb) \\
    existential & subject & (adjunct) & (verb) \\
    \lspbottomrule
    \end{tabularx}
    \caption{Argument structure of postural, locative, and existential clauses}
    \label{tab:YacopettiTable5}
\end{table}

Since each of the three clause types only differs with respect to the obligatory realisation of specific elements, we describe the types of ambiguity in cases i) with all three arguments; ii) without a verb; and iii) without a ground relation. 
For Kukatja and Kune, when a posture verb, subject and spatial complement/adjunct are all expressed, there is threeway ambiguity between locative, existential and postural meanings for animates and generally two-way ambiguity for inanimates. A Kukatja example is shown in (\ref{Yacopetti40}).

\ea \label{Yacopetti40}
\langinfo{Kukatja}{}{ALL: TEN2-2022\_022-01: 913.08} \\
\ea \label{Yacopetti40a}
\gll Kunyarr-pa	kalyu-ngka	nyi-nin.  \\
dog-\textsc{ep}	water-\textsc{loc}	sit-\textsc{prs} \\
\trans ‘A/the dog is in the water.’ / ‘There is a/the dog in the water.’ / ‘A/the dog is sitting in the water.’   \\
\ex \label{Yacopetti40b}
\gll Sand-pa=lu		nyi-nin	motika-ngka.  \\
sand-\textsc{ep}=3\textsc{sg.l}	sit-\textsc{prs}	vehicle-\textsc{loc} \\
\trans ‘Sand is in the vehicle.’ / ‘There is sand in the vehicle.’ 
\z
\z

When a posture verb is absent, there is ambiguity between locative and existential readings. A Kune example is shown in (\ref{Yacopetti41}).

\ea \label{Yacopetti41}
\langinfo{Kune}{}{Bulanjdjan Risselle Brian; KLLC\_20220708f} \\
\gll Kun-dulk	konda-beh 	koyek, 	kureh-beh 	bininj 	walem.  \\
\textsc{iv}-tree	here-side	east	there-side	man	south\\
\trans  ‘The tree is on this side, east; the man is on that side, south.’ / 
‘There is a tree on this side, east; there is a man on that side, south.’  \\
\z

When a verb is present but the locative argument is absent, a ‘locative’ interpretation is of course not possible. However, in these contexts, there is still structural ambiguity between postural and existential readings as illustrated in the Kune example in (\ref{Yacopetti42}).\footnote{\citet[478]{Evans2003} observes that one of the contexts in which nominals are incorporated with intransitive subjects in Bininj Kunwok languages is when they introduce new referents or assert existence. This is not the only function of such clause types, although it is a tendency observed in the data collected for Kune by the first author.} 

\ea \label{Yacopetti42}
\langinfo{Kune}{}{Bulanj Russell Brian; KLLC\_20220711b} \\
\gll Kunj 		na-wern 	ka-ni.  \\
kangaroo	\textsc{i}-many 	3\textsc{min}-sit.\textsc{npst} \\
\trans  ‘There are lots of kangaroos.’ / ‘Lots of kangaroos are sitting’.  \\
\z

Unlike Kune and Kukatja, Wik-Mungkan allows for definite marking of nominal expressions, providing formal grounds to distinguish between locative and existential clauses, in line with widespread typological observations. An example of an existential clause in Wik-Mungkan is shown in (\ref{Yacopetti43}).

\ea \label{Yacopetti43}
\langinfo{Wik-Mungkan}{}{mt\_mk-cak\_1\_230613} \\
\gll A' 	\textbf{keny=angk=an-a} 	pam-a 	than-an. \\
and	\textbf{top=\textsc{loc=def}-a}	man-a		stand-3\textsc{sg.prs}  \\
\trans ‘And on the top (there) is a man.’   \\
\z

In this example, the figure relation \textit{pam} ‘man’ is not definite marked, and is located on the top side. The ground object is the locational noun \textit{keny} ‘top’. The ground object is definite marked and marked as a spatial adjunct by the locative case. This example is an extract from a Man and Tree game, and was preceded by a description of the tree's location. In this context then, the new information is the existence (and location) of the man.
However, beyond definite marking, the same flexibility with respect to the realisation of arguments is observed in Wik-Mungkan as found for the other two languages (\ref{Yacopetti44})–(\ref{Yacopetti47}). These range, respectively, from the most minimal realisation to one in which all possible constituents are overtly realised. In (\ref{Yacopetti44}), the minimal locative marked complement NP is expressed, describing the location of an unexpressed but discourse topical subject figure as being at the house. In (\ref{Yacopetti45}), both the locative marked complement nominal expression (this mountain) and the figure subject NP (the tree) are expressed. In (\ref{Yacopetti46}), the obligatory complement NP (the plate) and a posture verb functioning as a copula-like verb is expressed. Lastly, in example (\ref{Yacopetti47}) the maximal argument structure is illustrated. 

\ea \label{Yacopetti44}
\langinfo{Wik-Mungkan}{}{kt: trps\_kt\_230416} \\
\textbf{Complement (locative predication)} \\
\gll Awuuch=ang=an-a. \\
house=\textsc{loc=def}-a  \\
\trans ‘[It is] at the house.’   \\
\z

\ea \label{Yacopetti45}
\langinfo{Wik-Mungkan}{}{kt: trps\_kt\_230416} \\
\textbf{Subject + complement (locative predication)} \\
\gll Yuk=an-a 	yoik=ang 		in=an=iy-a. \\
tree=\textsc{def}-a	mountain=\textsc{loc} 	\textsc{prox}=\textsc{def}=\textsc{top}-a  \\
\trans ‘The tree [is] on this mountain.’	   \\
\z

\ea \label{Yacopetti46}
\langinfo{Wik-Mungkan}{}{mp: deloca\_mp\_230610} \\
\textbf{Complement + copula verb (locative/stance predication)} \\
\gll Plate=ang=an 	wun-Ø-a. \\
plate=\textsc{loc=def}	lie-3\textsc{sg.pst}-a  \\
\trans ‘[It] was/lay on the plate.’	   \\
\z

\ea \label{Yacopetti47}
\langinfo{Wik-Mungkan}{}{kt: trps\_kt\_230416} \\
\textbf{Subject + complement + copula (locative/stance predication)} \\
\gll Lat	in=an=iy-a 		table=ang 	than-an. \\
book	\textsc{prox=def=top}-a	table=\textsc{loc}	stand-3\textsc{sg.prs}  \\
\trans‘This book is/stands on the table.’	   \\
\z

\subsection{Frequency comparison of posture verbs} \label{YacopettiSection3.4}

The existing descriptions of the use of posture verbs in locative and existential functions in at least Kune and Wik-Mungkan provide a characterisation of predictability and categoricity (see \cite[318]{Ray2021} for Wik-Mungkan; and \cite[560--562]{Evans2003} for Kune). Contrary to these descriptions, we find that in present day Wik-Mungkan and Kune, the use (and omission) of posture verbs are used with greater flexibility and with sensitivity to the various factors laid out in §\ref{YacopettiSection3.2}-\ref{YacopettiSection3.3}. In this section we compare flexibility of posture verbs using results from standardised elicitation tasks introduced in §\ref{YacopettiSection2.2}. 

Despite the structural possibility of posture verbs in locative-existential predications, they are not syntactically obligatory, nor are they even necessarily the quantitatively most frequent construction type for some of these predicate types in these languages. An under-researched question, then, is to what extent posture verbs are a preferred strategy amongst possible alternative construction types for encoding locative-existential meanings. In Wik-Mungkan and Kukatja (in contrast to Kune), the use of posture verbs in locative-existential contexts is a marked way of expressing these types of predicates. Instead, truly verbless locative and existential expressions are more common, as in examples (\ref{Yacopetti48}) and (\ref{Yacopetti49})—a construction type for which Australian languages are reasonably well-known (\cite[198]{Lynch1998}; \cite[227--228]{Simpson2023}, \cite{Dixon2002Copula}).

\ea \label{Yacopetti48}
\langinfo{Kukatja}{}{GLE: TEN2-2023\_001-02: 374} \\
\gll Lingka	lamparnpa	kalyu-\textbf{ngka}.  \\
snake	small		water-\textbf{\textsc{loc}} \\
\trans ‘The small snake is in the water.’   \\
\z

\ea \label{Yacopetti49}
\langinfo{Wik-Mungkan}{}{Michella Pahimbaung; deloca\_mp\_230610} \\
\gll Yoik 		keny=\textbf{ang}-a 		wuuch=an-a. \\
mountain	above=\textbf{\textsc{loc}}-a	house=\textsc{def}-a \\
\trans ‘The house is on top of a mountain.’	   \\
\z

In the Wik-Mungkan example in (\ref{Yacopetti49}), the figure \textit{wuuch} ‘house’ is located on top of the ground object \textit{yoik} ‘mountain’ without use of a verb in a postural or copula function. Instead, the locative case-marked expression \textit{yoik keny=ang-a} ‘on top=\textsc{loc}’ itself serves as the sole predicational element within the clause. Similarly in (\ref{Yacopetti48}), the locative case suffix \textit{-ngka} attaches to the nominal \textit{kalyu} ‘water’ and forms a predicational head \textit{kalyu-ngka} that does not need a linking copula to predicate a location of the figure (the subject \textit{lingka lamparn} ‘small snake’). The ability for a locative case suffix to function predicatively is a feature observed in Australian Aboriginal languages more broadly (\cite[227--228]{Simpson2023}).

We utilised results from the two main elicitation task (DELOCA and TPRS) to provide some quantified gauge of preferences among speakers for types of locative descriptions. Results were coded as to whether they invoked a posture verb (and which member), a fully verbless strategy (as in (\ref{Yacopetti48}) and (\ref{Yacopetti49})), or a miscellaneous ‘other’ category (to be exemplified shortly). A key hypothesis is that the animacy of the figure will influence the preference for posture verb usage: specifically, referents higher on the animacy hierarchy are expected to show more frequent use of posture verbs in locative-existential predications; while referents lower on the animacy hierarchy will show more frequent usage of non-posture verb strategies. All three posture verbs are anticipated to be used frequently with animate figures while inanimates will show either: i) a general dispreference to be predicated of a location via a posture verb or, alternatively ii) a preference to use but a single (bleached) posture verb in copula function.

We first present the results for Kukatja as shown in \tabref{tab:YacopettiTable6}. The Kukatja data show a reasonably robust relationship between figure type (based on a basic animacy hierarchy) and predicational strategy. Posturals were relatively rarely used for inanimate figures in Kukatja, accounting for just 23\% of locative descriptions while they were used for some 66\% of scenes that involved an animate figure.

\begin{table}
    \begin{tabularx}{\textwidth}{l YYY YY}
    \lsptoprule
    & \multicolumn{3}{c}{ {Posture verb}} & \multicolumn{2}{c}{ {Non posture verb}}\\
    \cmidrule(lr){2-4}\cmidrule(lr){5-6}
    Figure type  & \textit{nyi-}\newline ‘sit’ & \textit{ngara- }‘stand’ & \textit{ngarri-} ‘lie’ & Verbless & Other\\
    \midrule
    Inanimate (\textit{n}=91) & 11\% & 9\% & 4\% & 40\% & 36\%\\
         &  &  & \textbf{24\%} &  & \textbf{76\%}\\
    Animate (\textit{n}=20) & 40\% & 10\% & 0\% & 45\% & 5\%\\
         &  &  & \textbf{50\%} &  & \textbf{50\%}\\
    Human (\textit{n}=30) & 30\% & 30\% & 7\% & 7\% & 26\%\\
         &  &  & \textbf{67\%} &  & \textbf{33\%}\\
    \lspbottomrule
    \end{tabularx}
    \caption{Kukatja strategies for predicating location in DELOCA and TPRS}
    \label{tab:YacopettiTable6}
\end{table}

In Kune, the situation is similar although a hierarchy is less pronounced as shown in \tabref{tab:YacopettiTable7}. Inanimates in Kune are slightly more likely to be predicated of a location via a stance verb than in Kukatja (being used 33\% of the time rather than 23\%). However, posture verb strategies do not become quite as pervasive as one moves up the animacy hierarchy: posture verbs only being observed 52\% of the time compared to 66\% in Kukatja. However, the Kune picture changes depending on how one counts the positional verb \textit{-werndi} ‘be high’, which involves the posture verb element \textit{-di} ‘stand’. If tokens of \textit{-werndi} are added to the core set of posture verbs in Kune, the overall use of elements with postural semantics goes up significantly, counting for 44\%, 56\% and 63\% for inanimate, animate and human figures respectively. If we take this slightly more permissive set of what constitutes a ‘posture verb’, then we observe two findings: the differences in predicational strategy for humans and animates becomes rather minor in Kune and the use of a ‘posture verb’ strategy for inanimates is much higher in Kune than in Kukatja.

\begin{table}
\small
    \begin{tabularx}{\textwidth}{l@{}rrr Y rr}
    \lsptoprule
    & \multicolumn{3}{c}{ {Posture verb}} &  {Positional verb} & \multicolumn{2}{l}{ {Non-posture verb}} \\

    \cmidrule(lr){2-4}\cmidrule(lr){5-5}\cmidrule(lr){6-7}
    Figure type  & \textit{ni-} ‘sit’ & \textit{di- }‘stand’ & \textit{yo-}  ‘lie’ & \textit{wern.di} `be.high' & Verbless & Other\\
     \midrule
    Inanimate (\textit{n}=539) & 2\% & 23\% & 8\% & 11\% & 11\% & 45\% \\
         &  &  & \textbf{33\%} & \textbf{11\%} & & \textbf{56\%} \\
    Animate (\textit{n}=121) & 22\% & 21\% & 8\% & 15\% & 8\% & 36\% \\
         &  &  & \textbf{41\%} & \textbf{15\%} & & \textbf{44\%} \\
    Human (\textit{n}=115) & 21\% & 13\% & 18\% & 11\% & 1\% & 36\% \\
         &  &  & \textbf{52\%} & \textbf{11\%} & & \textbf{37\%} \\
    \lspbottomrule
    \end{tabularx}
    \caption{Kune strategies for predicating location in DELOCA and TPRS}
    \label{tab:YacopettiTable7}
\end{table}

The findings for Wik-Mungkan (shown in \tabref{tab:YacopettiTable8}) pattern in some ways like Kukatja on the one hand and like Kune on the other. Like Kukatja, posture verbs are a relatively disfavoured strategy for predicating location of an inanimate figure (just 32\% of scenes were described in this way). Like Kune and unlike Kukatja, however, there was no major difference in predicational strategies for scenes involving human and animate figures—both utilising posture verbs for over 60\% of the scenes described.

\begin{table}
    \begin{tabularx}{\textwidth}{lrrrrr}
    \lsptoprule
    & \multicolumn{3}{c}{ {Posture verb}} & \multicolumn{2}{c}{ {Non posture verb}} \\
    \cmidrule(lr){2-4}\cmidrule(lr){5-5}\cmidrule(lr){6-6}
    Figure type  & \textit{nyiin} ‘sit’ & \textit{than} ‘stand’ & \textit{wun} ‘lie’ & Verbless & Other \\
     \midrule
    Inanimate (\textit{n}=258) & 2\% & 9\% & 22\% & 45\% & 22\%\\
         &  &  & \textbf{33\%} &  & \textbf{67\%} \\
    Animate (\textit{n}=60) & 33\% & 2\% & 27\% & 28\% & 10\% \\
         &  &  & \textbf{62\%} &  & \textbf{38\%} \\
    Human (\textit{n}=55) & 31\% & 15\% & 20\% & 16\% & 18\% \\
         &  &  & \textbf{66\%} &  & \textbf{44\%} \\
    \lspbottomrule
    \end{tabularx}
    \caption{Wik-Mungkan strategies for predicating location in DELOCA and TPRS }
    \label{tab:YacopettiTable8}
\end{table}

The data thus support a view that there are differing degrees of grammaticalisation of posture verbs across the three languages when one compares the relative frequency of individual posture verbs with non-human figures. For inanimates, there is evidence of a single posture verb being preferred in all three languages. In Kukatja, the ‘sit’ member’ was the most frequent choice and the ‘lie’ member was barely used. In contrast, for Kune, it is the ‘stand’ member \textit{-di} (and the related compound \textit{-werndi} ‘be elevated’) that is the preferred posture verb (43\%) while \textit{-ni} ‘sit’ is only minimally attested (being used for 2\% of the relevant scenes). Finally in Wik-Mungkan it is \textit{wun-} ‘lie’ that is preferred (22\%) followed distantly by \textit{than-} ‘stand’ and \textit{nyiin-} ‘sit’ (2\%). Thus for all languages, the counts of posture verbs used with inanimate figures reflected the qualitative impression that there is a ‘default’ verb for inanimate figures.

For non-human animates, the three languages showed some un-anticipated results, however. For Wik-Mungkan speakers, the location of non-human animates was rarely described with locative clauses involving \textit{than-} ‘stand’, which tends to be reserved for either humans or profiling the upright configuration of certain inanimate entities. For scenes where ‘stand’ might have been anticipated (e.g. a dog ‘standing’ on all fours), verbless strategies were instead observed. For Kukatja, \textit{ngari-} ‘stand’ is used with animate non-humans but \textit{ngarri-} ‘lie’ was entirely unattested for such figure types. Kune speakers utilised all three posture verbs with animate non-humans but, like Kukatja speakers, dispreferred the use of the ‘lie’ verb.\footnote{Since the stimuli utilised in this quantitative study were not specifically controlled for equal numbers of scenes depicting each anticipated posture type, we did not anticipate equal proportions of each individual posture verb to be used. However, the differences observed here go well beyond variation in scenes depicted. It is an area of future study to carefully investigate posture verb use while controlling carefully for scene type (beyond referential properties of the figure).} 

\subsection{Non-posture verb strategies} \label{YacopettiSection3.4.1}

In the preceding sections, it was shown that a significantly large number of descriptions emerged utilising a fully verbless predicational strategy for locative descriptions. However, discussion of alternatives with various consultants and running the task with multiple speakers made it clear that supplying a posture verb was nearly always possible in all languages, reflecting a flexibility in strategy choice.

Other construction types used in the surveyed languages to express locative-existential predication are sampled below. In all three languages, a resultative or impersonal construction is attested for a number of locational scenes, particularly those depicting small manipulable inanimates objects as figures (cf. \cite[50]{LevinsonWilkins2006Patterns}) (\ref{Yacopetti50}). For some scenes, Kune speakers make use of a ‘have’ predicate (\ref{Yacopetti51}), while Kukatja speakers can make use of proprietive case marking on the figure for certain figure-ground relationships (\ref{Yacopetti52}), especially those where the figure is perceived as a characteristic property of the ground relation. 

\ea \label{Yacopetti50}
\langinfo{Kune: resultative}{}{Bangardidjan Tara Rostron; KLLC\_ 20220805b} \\
\gll Nane 	bid-no 	\textbf{birri-kurrme-ng}  durn-no		darrme. \\
\textsc{dem} finger-\textsc{prt} 	\textbf{\textsc{3pl>3sg}-put-\textsc{pst}} string-\textsc{prt} \textsc{loc}	 \\
\trans  ‘This peg  (lit. ‘finger’) is on the line’ (lit. ‘“they” put this peg on the line’)  \\
\z

\ea \label{Yacopetti51}
\langinfo{Kune: ‘have’ construction}{}{Bulanj Simon Brian; KLLC\_20210917a} \\
\gll Hill 	ka-\textbf{karrme} 		man-me. \\
hill 	3\textsc{min}-\textbf{have.}\textsc{\textbf{npst}} 	\textsc{iii}-food	 \\
\trans  ‘There is food on the hill.’ (lit. ‘the hill “has” food’)   \\
\z

\ea \label{Yacopetti52}
\langinfo{Kukatja: proprietive case}{}{George Lee: LM2019-04-15-150744} \\
\gll Tjurnu 		nyi-nama	\textbf{kalyu-kurlu}. \\
rockhole 	sit-\textsc{ipfv}	\textbf{water-\textsc{prop}}	 \\
\trans  ‘There is water in the rockhole.’ (lit. ‘the rockhole was “with” water’)    \\
\z

\section{Conclusion} \label{YacopettiSection4}

We have examined novel and existing material on three Australian Aboriginal languages (Kukatja, Kune, and Wik-Mungkan). The findings for Kukatja, Kune, and Wik-Mungkan generally align with the previous documentation of Australian language posture verbs as set out in §\ref{YacopettiSection1.2}, but careful semantic analysis and quantification reveal underappreciated variation and diversity. 

In §\ref{YacopettiSection3.1} we investigated the lexical semantic space covered by the three posture verbs in each language. Beyond their core stance functions (‘sit’, ‘stand’, ‘lie’), they exhibit overlapping but non-uniform extensions into various other senses such as ‘sleep’, ‘reside’, or ‘wait (for)’. For example, the sense ‘reside’ in Kukatja and Kune aligns with the ‘sit’ verb, whereas in Wik-Mungkan the meaning is associated with \textit{wun-} ‘lie’. Some of these sense types are also characterised by different argument structures for the verb (e.g. \textit{nyina-} ‘sit’ and \textit{ngari-} ‘stand’ in Kukatja take a dative non-subject argument in their sense ‘wait’). We also illustrated how a posture verb can be used for sense selection for polysemous figure nouns. Our detailed semantic analysis of the terms enriches existing descriptions of posture verbs typologically (\cite{Newman2002CrossLinguistic,Newman2022}). 

In §\ref{YacopettiSection3.2} we examined how posture verbs feature in strategies for the predication of location and existence, highlighting that for all three languages it is not always possible to analytically separate the postural meaning from a copula-type use. Overall, it is only when the semantics of a posture verb are incongruent with the arrangement of the figure that it is possible to make an unambiguous determination of a copula use of the posture verb. We also provided evidence that posture verb use in locative and existential predication in all three languages is generally flexible rather than grammatically obligatory. 

In §\ref{YacopettiSection3.3} we showed that for Kukatja and Kune, there is no clear distinction between locative and existential clause types, whereas Wik-Mungkan distinguishes the two based on obligatory encoding of definiteness of the figure in the former.

Finally in §\ref{YacopettiSection3.4}, we quantified some of the variation in posture verb use identified in §\ref{YacopettiSection3.2} and demonstrated that each of the languages differs with respect to their overall use of posture verbs in locative-existential predication. An animacy hierarchy was shown to be clearly operative in Kukatja and influential in Kune and Wik-Mungkan—higher order animates being more likely to be predicated of a location with an overt posture verb than inanimates. Frequency usage was shown to support our impressionistic observations that just one posture verb in each language was favoured for predicating the location of inanimate figures: \textit{nyina-} ‘sit’ for Kukatja; \textit{-di} ‘stand’ for Kune and \textit{wun-} ‘lie’ for Wik-Mungkan. 

The present study suggests that there is some benefit to reviewing existing claims of both posture verbs and copula predication in Australian Aboriginal languages in conjunction with quantified standardised tasks or corpus data. This would pave the way for a renewed and more informed typology of copula predication on the continent, especially given the number of detailed grammatical descriptions published since Dixon’s (\citeyear{Dixon2002Copula}) study.

 
\section*{Abbreviations}
\begin{tabularx}{.45\textwidth}{lQ}
\textsc{act} & action verbalizer \\
\textsc{aug} & augmented number \\
\textsc{avert} & avertive \\
\textsc{cis.loc} & cislocative (“hither”) \\
\textsc{conj} & conjunction \\
\textsc{d} & dative (bound pronoun) \\
\textsc{ep} & epenthetic \\
\end{tabularx}
\begin{tabularx}{.45\textwidth}{lQ}
\textsc{i} & noun class \textsc{i} \\
\textsc{ident} & identificational \\
\textsc{ii} & noun class \textsc{ii} \\
\textsc{iii} & noun class \textsc{iii} \\
\textsc{inch} & inchoative \\
\textsc{iv} & noun class \textsc{iv} \\
\textsc{l} & locational (bound pronoun) \\
\end{tabularx}

\begin{tabularx}{.45\textwidth}{lQ}
\textsc{med} & medial \\
\textsc{min} & minimal number \\
\textsc{npp} & non-past progressive \\
\textsc{npst} & non-past \\
\textsc{ns} & non-subject \\
\textsc{opt} & optative \\
\textsc{perl} & perlative \\
\end{tabularx}
\begin{tabularx}{.45\textwidth}{lQ}
\textsc{pot} & potential \\
\textsc{prop} & proprietive \\
\textsc{prt} & part noun marker \\
\textsc{rdp} & reduplication \\
\textsc{sembl} & semblative \\
\textsc{ua} & unit augmented \\
\\
\end{tabularx}

\section*{Acknowledgements}
We are indebted to the speakers of Kukatja in Balgo, speakers of Kune in Arnhem Land, and speakers of Wik-Mungkan in Aurukun. Special thanks go to Kune consultants ✝Kodjok R.Redford, Bulanj Russell Brian, and Kamanj Michelle Martin for their insightful contributions. We would also like to thank Alice Gaby and Jill Vaughan, for comments on an earlier version of this chapter. Finally, we acknowledge our two anonymous reviewers whose feedback significantly improved this chapter.
Institutional approval to conduct this research was provided by Monash University Human Research Ethics \& Compliance (Kukatja: ref 27309; Kune: ref 35986); and University of Newcastle Human Research Ethics Committee (Wik-Mungkan: Protocol Number H-2022-0147). The research on which this chapter was based was variously funded by: an Australian Research Council Discovery Project grant (DP200101079), an Australian Linguistics Society Research Grant, multiple Australian Government Research Training Program Scholarships, a Monash Graduate Excellence Scholarship and The University of Western Australia GRS Travel Award. 

\sloppy
\printbibliography[heading=subbibliography,notkeyword=this]
\end{document}
