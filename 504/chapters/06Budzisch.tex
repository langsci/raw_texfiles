\documentclass[output=paper,colorlinks,citecolor=brown]{langscibook}
\ChapterDOI{10.5281/zenodo.16838066}
\title{Predlocatives, existentials, and predpossessives in Nenets} 
\author{Josefina Budzisch\orcid{}\affiliation{University of Hamburg}}

\abstract{This chapter investigates predlocative, existential, and predpossessive clauses in Tundra and Forest Nenets, North-Samoyedic languages within the Uralic family. While prior research predominantly focuses on Tundra Nenets, this study addresses both varieties. Emphasis is placed on choice of the verbal element in predlocative, existential, and predpossessive clauses. The investigation is conducted at corpus-based level to assess the accuracy of descriptions related to Tundra Nenets and to determine their applicability to Forest Nenets.

It is observed that in locative sentences in Tundra Nenets both copulas \textit{ŋa-} and \textit{me-} are employed, with a distinction based on the animacy of the figure. In existential and possessive sentences, the existential verb \textit{tańa-} is predominantly used, except in the case of a quantified figure, where copula \textit{ŋa-} or zero copula is employed. The latter distinction is also evident in negation. Forest Nenets exhibits similar patterns in locative and possessive sentences, but demonstrates a much broader range of possible verbal elements in existential sentences.}



\IfFileExists{../localcommands.tex}{
   \addbibresource{../localbibliography.bib}
   % add all extra packages you need to load to this file

\usepackage{tabularx,multicol}
\usepackage{url}
\urlstyle{same}

\usepackage{listings}
\lstset{basicstyle=\ttfamily,tabsize=2,breaklines=true}

\usepackage{langsci-basic}
\usepackage{langsci-optional}
\usepackage{langsci-lgr}
\usepackage{langsci-osl}
% \usepackage{./langsci/styles/langsci-lgr}
% \usepackage{./langsci/styles/langsci-osl}
% \usepackage{langsci-gb4e}

\usepackage{tikz}
\usetikzlibrary{patterns,calc}
\pgfdeclarepatternformonly{south east lines}{\pgfqpoint{-0pt}{-0pt}}{\pgfqpoint{3pt}{3pt}}{\pgfqpoint{3pt}{3pt}}{
    \pgfsetlinewidth{0.6pt}
    \pgfpathmoveto{\pgfqpoint{0pt}{3pt}}
    \pgfpathlineto{\pgfqpoint{3pt}{0pt}}
    \pgfpathmoveto{\pgfqpoint{.2pt}{-.2pt}}
    \pgfpathlineto{\pgfqpoint{-.2pt}{.2pt}}
    \pgfpathmoveto{\pgfqpoint{3.2pt}{2.8pt}}
    \pgfpathlineto{\pgfqpoint{2.8pt}{3.2pt}}
    \pgfusepath{stroke}}
    
\usepackage{stmaryrd}
\usepackage{wasysym}
\usepackage{multirow}
\usepackage{caption}
\usepackage{subcaption}
\usepackage{mathrsfs}
\usepackage{qtree}

\usepackage{linguex}


   %pminos do not split footnotes
% \interfootnotelinepenalty=10000 %Footnote in Laporte chapters has to be split SN


%\DeclareIndexNameFormat{default}{%
%\nameparts{#1}%
%\usebibmacro{index:name}%
%{\index[names]}%
%{\namepartfamily}%
%{\namepartgiveni}%
% {}% L1
% {}% L2
%{\namepartprefix}% generates spurious space L3
%{\namepartsuffix}% generates spurious space L4
%}

%  {\DeclareIndexNameFormat{default}{%
%     \usebibmacro{index:name}{\index[names]}{#1}{#3}{#5}{#7}}}

%\DeclareIndexNameFormat{default}{%
%  \usebibmacro{index:name}{\sindex[nom]}{#1}{#3}{#5}{#7}}

%\DeclareIndexNameFormat{default}{%
%  \usebibmacro{index:name}{\sindex[person]}{#1}{#3}{#5}{#7}}
%\DeclareIndexNameFormat{default}{%
%\nameparts{#1} \usebibmacro{index:name}{\sindex[person]]}{\namepartfamily}{‌​\namepartgiven}{\nam‌​epartprefix}{\namepa‌​rtsuffix}}

%\newcommand{\smiley}{:)}

%\renewbibmacro*{index:name}[5]{%
%\usebibmacro{index:entry}{#1}%
%{\iffieldundef{usera}{}{\thefield{usera}\actualoperator}\mkbibindexname{#2}{#3}{#4}{#5}}}

% \newcommand{\noop}[1]{}

%remove for final
%\overfullrule=1mm

\newcommand{\tobi}[2]}}
\renewcommand{\S}[1]{\tobi{#1}{\textsc{*}}}

% this volume references
% puts: [this volume]
% already defined: \citetv
%\newcommand{\citepv}[1]{(\citeauthor{#1} \citeyear*{#1} [this volume])}
\newcommand{\citealtv}[1]{\citeauthor{#1} \citeyear*{#1} [this volume]}

%parentheses around example number
\newcommand{\pref}[1]{(\ref{#1})}

% in-text examples

\newcommand{\lnex}[1]{\textit{#1}} %target lang word
\newcommand{\lnlit}[1]{(lit.: `#1')} %literal reading
\newcommand{\lnlat}[1]{(#1)} % latinization
\newcommand{\lntrans}[1]{`#1'} %translation
\newcommand{\lnexl}[2]%
{\lnex{#1}{} \lnlat{#2}} % ex with latinization
\newcommand{\lnexlat}[3]{\lnex{#1}{} \lnlat{#2}{} \lntrans{#3}} % ex with latinization and tranl.

%ch01
\newcommand{\co}[1]{\mbox{\textbf{#1}}}

%ch09

\newcommand{\cyrbulg}[1]{\begin{otherlanguage*}{bulgarian}#1\end{otherlanguage*}}


%ch10
\newcommand{\nlp}{{\small NLP}}
\newcommand{\mwe}{{\small MWE}}
\newcommand{\rae}{{\small RAE}}
\newcommand{\lvc}{{\small LVC}}
\newcommand{\pos}{{\small P}o{\small S}}
%\newcommand{\todo}[1]{ \textcolor{red}{#1} }

%\renewcommand{\labelenumi}{\theenumi}
%\ainamefmt{{vv}{ll}{, ff}{, jj}} % fullname

\newcommand{\biberror}[1]{{\color{red}#1}}

\newcommand{\osenovaitem}{--~}
   %% hyphenation points for line breaks
%% Normally, automatic hyphenation in LaTeX is very good
%% If a word is mis-hyphenated, add it to this file
%%
%% add information to TeX file before \begin{document} with:
%% %% hyphenation points for line breaks
%% Normally, automatic hyphenation in LaTeX is very good
%% If a word is mis-hyphenated, add it to this file
%%
%% add information to TeX file before \begin{document} with:
%% %% hyphenation points for line breaks
%% Normally, automatic hyphenation in LaTeX is very good
%% If a word is mis-hyphenated, add it to this file
%%
%% add information to TeX file before \begin{document} with:
%% \include{localhyphenation}
\hyphenation{
    Beck-man
    Ngu-yen
    back-chan-nel
    back-chan-nels
    mo-not-o-nous
    ste-reo-typ-i-cal
}

\hyphenation{
    Beck-man
    Ngu-yen
    back-chan-nel
    back-chan-nels
    mo-not-o-nous
    ste-reo-typ-i-cal
}

\hyphenation{
    Beck-man
    Ngu-yen
    back-chan-nel
    back-chan-nels
    mo-not-o-nous
    ste-reo-typ-i-cal
}

   \boolfalse{bookcompile}
   \togglepaper[6]%%chapternumber
}{}

%\pretocmd{\gll}{\def\eachwordone{\itshape}\def\eachwordtwo{\normalfont}}{}{}

\begin{document}
\maketitle

\section{Introduction}\label{sec:introduction}
In the Nenets languages (Uralic, [nene1249] and [fore1274]), the structures of locative and existential clauses are characterized by distinctions in word oder and the verbal element employed. Possessive clause structures exhibit a close relationship with existential clauses.

The aim of this chapter is to provide an overview of locative, existential, and possessive structures in both Tundra and Forest Nenets, with a specific emphasis on the verbal component. This chapter aims to elucidate which elements are employed in various constructions and whether these correspond to existing descriptions for Tundra Nenets. Furthermore, it explores the alignment of Forest Nenets within this framework.

The chapter is structured as follows: the introduction is divided into three subsections: \sectref{sec:theory} theoretical background, offering the theoretical context for the study; \sectref{sec:lang} language and data, providing insights into the Nenets languages and data sources; and \sectref{sec:research} the description of previous research on locative, existential, and possessive structures in Nenets, delineating the focus and scope of the research. Subsequently, the chapter examines the core components in detail, dedicating separate sections to each of the three clause structures -- \sectref{sec:loc} predlocatives; \sectref{sec:ex} existentials; and \sectref{sec:poss} predpossessives; each split into a section about Tundra and Forest Nenets, and a concluding summary. Finally, the chapter concludes by summarizing the findings and giving an outlook in \sectref{sec:summary}.

\subsection{Theoretical background}\label{sec:theory}
Locative and existential clauses are often closely related, and it is commonly assumed that they can be traced back to the same structure: Both express the presence or absence of a figure (also: theme, pivot) in a ground (also: location, coda) (e.g. \cite{Lyons1967}; \cite{Clark1978}; \cite{Freeze1992}; \cite{Koch2012}). However, the precise delineation has often proven challenging and remains a point of contention in theoretical literature. Moreover, different perspectives by researchers have resulted in diverse subcategorizations and definitions. For instance, some define existential clauses based on deviations in surface form (e.g. \textcite[1829]{McNally2011}: ``The term ‘existential sentence’ is used to refer to a specialized or non-canonical construction which expresses a proposition about the existence or the presence of someone or something"). However, this approach excludes structures and, thus, possibly entire languages that do not display such deviations but only ``canonical structures", i.e. structures that are merely using equivalents of English `exist'. From a typological standpoint, it is more advantageous to prioritize function over form to avoid overlooking phenomena or entire languages.

For this chapter, I will hence reference Haspelmath's \citeyearpar{HaspelmathNonverbal, Haspelmath2025} concise definitions:
\begin{itemize}
\item predlocative clause: a clause in which the subject is definite and the predicate is a locative phrase;
\item existential clause: a clause in which an indefinite nominal phrase is said to be in some location.
% \item appertentive clause: a clause in which the subject is definite and the predicate is a possessor nominal
\end{itemize}

I will therefore henceforward talk about existential and predlocative clauses. This approach allows to investigate structural similarities and differences among clauses, enhancing insights into their pragmatic function, thus giving an idea of how the construction expresses the function and not limiting it to a specific strategy \citep[17]{Croft2022}.

In this chapter, I will address not only existential and predlocative sentences but also possessive sentences, given their structural resemblance to existential constructions in Nenets. The majority of Uralic languages lack a \textit{habeo} verb. Consequently, an alternative approach involves employing an existential construction with an animate ``ground". I also resort to the definition provided by \citet{HaspelmathNonverbal, Haspelmath2025} for this type of sentence, which is as follows:

\begin{itemize}
\item predpossessive clause: a clause in which an indefinite nominal phrase is said to be possessed by a definite possessor.
\end{itemize}


Building on these foundational concepts, we now proceed to a more detailed examination of Nenets sentences of these types.

\subsection{Language and data}\label{sec:lang}
Tundra Nenets and Forest Nenets, both belonging to the North-Samoyedic group within the Samoyedic branch of the Uralic language family, have traditionally been viewed as dialects of the Nenets language. However, more recent research recognizes them as distinct languages. Tundra Nenets is spoken across the tundra zone of the Russian north, spanning from the eastern coast of the White Sea in the west to the right-bank tributaries in the lower reaches of the Yenisei river in the east. It extends from the Arctic Ocean in the north to the northern border of the forest zone in the south. Forest Nenets is found in the forest-tundra and taiga zones along the Pur and Agan rivers, as well as in the area around Lake Num-To. The settlements are also shown in \figref{fig:map}. 

\begin{figure}
    \includegraphics[width=1\linewidth]{figures/9.1b Nenets_current.png}
    \caption{Current settlements of Tundra and Forest Nenets \citep{RantanenTolvanen2021}}
    \label{fig:map}
\end{figure}

Tundra Nenets is divided into western, central, and eastern dialect groups, whose boundaries often coincide with geographical features such as the Ural Mountains. These dialect groups are further subdivided into sub-dialects. Despite the vast geographical area over which these dialects are spoken, the dialects of Tundra Nenets are mutually intelligible. Forest Nenets has undergone significant changes throughout the 20\textsuperscript{th} century due to industrial activities. Today, it is categorized into the Pur, Agan, and Numto dialects. The respective groups live relatively isolated from one another, resulting in the individual dialects being considerably more heterogeneous compared to Tundra Nenets. However, the differences between the dialects are not significant enough to hinder mutual understanding \citep[182--184]{Burkova2010}. An overview of the dialect groups and respective sub-dialects is presented in \tabref{tab_dialects}.

\begin{table}
\caption{Dialects and sub-dialects in Tundra and Forest Nenets (\cite[182--184]{Burkova2010}) \label{tab_dialects}}
\begin{tabular}{cccc}
\lsptoprule
\multicolumn{3}{c}{Tundra Nenets} & Forest Nenets \\
\midrule
Western & Central & Eastern & \\
\cmidrule(rl){1-3}
\multicolumn{1}{c}{Kanin} &   \multicolumn{1}{c}{Bolshezemelskiy}      & Priural & Pur \\ 
{Tyiman} & &  Yamal & Agan \\{Kolguev} & & Nadym & Numto \\
{Malozemelskiy} & & Taz & \\
& & Gyda & \\
& & Taimyr & \\
\lspbottomrule
\end{tabular}
\end{table}

According to the 2021 Russian Census, the total Nenets population was reported as 49,787 people with 38,405 speakers of Nenets; only around 1,500 of those belonging to the Forest Nenets subpopulation.  The preservation of the Nenets languages exhibits regional variations. Tundra Nenets is better preserved in remote areas characterized by denser habitation and traditional reindeer herding practices. In contrast, Forest Nenets has seen a substantial decline in regions impacted by urbanization and the influence of the oil industry. Both Nenets languages enjoy legal protection and inclusion in school curricula, with efforts aimed at promoting their usage in new media platforms. Nonetheless, their application in public domains remains limited in practice (for more information see e.g. \cite{Burkova2022}).

Nenets is an agglutinative language. Nouns in Nenets undergo inflection for a range of cases, which includes the nominative, accusative, genitive, and locative cases, and these are applied in singular, dual and plural forms in the core grammatical cases, while in locative cases, only singular and plural occur and the dual forms are expressed by periphrastic constructions. Nouns are also inflected for possession, encoding the possessor's person and number. Moreover, the complexity of inflection is evident not only in nominal but also in verbal morphology; verbs in Nenets are marked for tense, aspect, and mood. To convey nuanced meanings and depict relationships between actions, Nenets employs a complex system of participles and converbs (for more about inflection in Nenets, see e.g. \cite{Salminen1997}). Nenets has a subject-object-verb (SOV) word order, a characteristic shared with many Uralic languages. The language utilizes postpositions to indicate spatial and temporal relationships (for more on Tundra Nenets syntax, see \cite{Tereshchenko1973} and \cite{Nikolaeva2014} and for Forest Nenets, \cite{Koshkareva2005}). 

The present study is based on a corpus being compiled in Hamburg as part of the INEL project \citep{BudzischWagner-Nagy2024}. The corpus comprises archived data, including both unpublished and published texts, and it mostly features narratives and folklores. In this context, a total of 457 existential, predpossessive, and predlocative clauses in Forest and Tundra Nenets were analyzed. 

As previously mentioned, both Tundra and Forest Nenets are divided into further subgroups. The dataset examined here is primarily based on the eastern Taimyr dialect of Tundra Nenets. Forest Nenets is largely represented by the Pur and Agan dialects, but some Numto data is also included to reflect the fact that the Forest Nenets dialects exhibit greater differences compared to the Tundra Nenets dialects. For all examples taken from the corpus, the language (Tundra Nenets: TN, Forest Nenets: FN) as well as the dialect group and the sub-dialects are indicated. The transcriptions in the examples shown below are either directly taken from the corpus or sourced from the literature and have been adopted without significant modifications. This implies that there are some inconsistencies, particularly concerning the marking of vowel length or shortness, which, however, are not pertinent to the thematic focus of the present study and therefore will not be discussed further.

\subsection{On the descriptions of predlocative, existential and predpossessive structures in Nenets}\label{sec:research}
The already existing research addressing the construction of predlocative, existential, and predpossessive structures predominantly focuses on Tundra Nenets (e.g. \cite[42--47, 74, 115]{Hajdú1968}, \cite{Tereshchenko1973}, \cite[10, 13, 23, 27, 39]{Honti2007}, \cite[250--251, 263--264]{Nikolaeva2014}, \cite[221--222, 226--229]{Wagner-Nagy2016}, \cite[82--85]{Mus2015} and \cite[244--245, 248--251]{Mus2022}) with some remarks on Forest Nenets in some sources (\cite[195--199, 228--237]{Wagner-Nagy2011}). 

It is generally stated that predlocative structures have figure-ground word order and use the verbs \textit{ŋa-} and \textit{me-}, the former for inanimate and the latter for animate figures,\footnote{Following \textcite[248]{Mus2022}, I assume that within the Nenets languages, an entity is considered animate when it is alive; however, it may be treated as inanimate once it is no longer alive. Consequently, the copula \textit{ŋa-} is applicable to prototypical animate figures within a specific context.} and existential and predpossessive structures exhibit a ground-figure (respective possessor-possessee) word order and make use of the existential verb \textit{tańa-}; in predpossessive structures, \textit{ŋa-} is also used in a few instances or the verb is dropped completely. All these structures are negated with the negative existential verb \textit{jaŋko-}. All four verbs take TAME markers. In \tabref{tab_verb} these findings are summarized regarding the verb used in the different structure types. 

\begin{table}
\caption{Verbal elements in Tundra Nenets for predlocative, existential, and predpossessive structures \label{tab_verb}}
\begin{tabular}{lll}
\lsptoprule
structure & affirmative & negative \\
\midrule
\multirow{2}{*}{predlocative} & \textit{ŋa-} {[}with inanimate figure{]}    & \textit{jaŋko-}\\
 & \textit{me-} {[}with animate figure{]} & \textit{jaŋko-} \\
 \tablevspace
existential & \textit{tańa-}                          & \textit{jaŋko-} \\
\tablevspace
predpossessive & \textit{tańa-}; (\textit{ŋa-} / zero)  & \textit{jaŋko-}\\
\lspbottomrule
\end{tabular}
\end{table}

In accordance with these descriptions, typical affirmative representatives for these three clause types appear as follows:\footnote{I will gloss the verb \textit{me-} as ‘be.there' and \textit{ŋa-} as simply ‘be' to make the distinction more clearly visible; this is diverging from the sources quoted here.}

\ea
\langinfo{Tundra Nenets}{}{\cite[244]{Mus2022}}
\ea{ \label{lit:loc_inan}
\gll Toľ labe-kăna ŋa.\\
       table room-\textsc{loc} be.\textsc{3sg}\\}\jambox*{[Predlocative with inanimate figure]}
        \glt ‘The table is in the room.'\\
        \ex{ \label{lit:loc_an}
         \gll Igoŕ xarăd-ʔ m'uńa me.\\
         Igor house-\textsc{gen} inside be.there.\textsc{3sg}\hspace*{-1cm}\\}\jambox*{[Predlocative with animate figure]}
         \glt ‘Igor is in the house.’\\
\z\z
  
  \ea
  \label{lit:ex}
    \langinfo{Tundra Nenets}{}{\cite[244]{Mus2022}}\\
    {\gll Labe-kana xasawa tăńa.\\
    room-\textsc{loc} man exist.\textsc{3sg}\\}\jambox*{[Existential]}
    \glt ‘There is a man in the room.’\\
\z


\ea 
   \label{lit:poss}
   \langinfo{Tundra Nenets}{}{\cite[13]{Vanuyto2012}, cited after \cite[229]{Wagner-Nagy2016}}\\
    {\gll Mań ńeka-mi, sʲiďa ńe ńa-mi tańa.\\
    \textsc{1sg.gen} brother-\textsc{1sg} two woman sister-\textsc{1sg} exist.\textsc{3sg}\\}\jambox*{[Predpossessive]}
        \glt ‘I have one brother and two sisters.’\\
\z
 
In \REF{lit:loc_inan}, the ground is expressed with a noun in a locative case, but it is also possible to use postpositional phrases (as seen in \REF{lit:loc_an}) or adverbs \citep[203]{Nikolaeva2014}. In both sentences, the figure precedes the ground. In \REF{lit:ex}, the existential sentence, the order is reversed, with the ground preceding the figure. In the predpossessive sentence in \REF{lit:poss}, the possessor is expressed via a possessive pronoun and suffix here, but it would also suffice to only have the suffix and drop \textit{mań} from \REF{lit:ex}. It is, however, also possible that the possessor is lexically expressed and marked with the nominative (as shown in \REF{lit:poss_overt}), genitive or locative \citep[237]{Wagner-Nagy2011}; \citet[251]{Nikolaeva2014} also presents an example with the possessor in the dative.

\ea
\label{lit:poss_overt}
\langinfo{Tundra Nenets}{}{\cite[16]{Lehtisalo1956}, cited after \cite[231]{Wagner-Nagy2011}}\\
{\gll Pida sʲiďa ńe-da ŋä-wi.\\
\textsc{3sg} two woman-\textsc{nom.sg.3sg} be-\textsc{nar.3sg.s}\\}\jambox*{[Predpossessive]}
\glt ‘He had two wives.’\\
\z

Regarding the verb, we can observe the distribution shown in \tabref{tab_verb}: in locative clauses, there is a distinction between animate and inanimate figures, while in existential and predpossessive clauses, the existential verb is employed. Hence, ``[i]n Nenets the use of the copula already indicates whether the sentence is locative or existential.” \citep[228]{Wagner-Nagy2016}

However, it has become evident that the distribution of verbs within these clause types is not as straightforward as previously described. As \textcite[230]{Wagner-Nagy2011} herself notes (``[\REF{lit:ex_na}] clearly illustrates that this verb form \mbox{[\textit{ŋa-}]} can appear in existential sentences [...]"), and as \textcite{Hegedűs2017,Hegedűs2018} also observe, the distribution is not entirely definitive, and they provide several examples that deviate from the described pattern. In the following, two existential structures with ``unexpected" verbs are presented: 

\ea
\label{lit:ex_na}
\langinfo{Tundra Nenets}{}{\cite[70]{Labanauskas2001}, cited after \cite[230]{Wagner-Nagy2011}}\\
{\gll Laxana-ko-r manma: pädare-ta jaxa ŋä-wi.\\
   story-\textsc{dim}-\textsc{2sg} say.\textsc{3sg} forest-\textsc{nom.sg.3sg} river be-\textsc{nar.3sg}\\}\jambox*{[Existential]}
    \glt ‘They say that there was a river that had a forest on its bank.’
\z

\ea
\label{lit:ex_me}
\langinfo{Tundra Nenets}{}{\cite[5]{Hegedűs2017}}\\
{\gll Labe-kana xasawa me.\\
room-\textsc{loc} man be.\textsc{3sg}\\}\jambox*{[Existential]}
\glt ‘There is a man in the room.’\\
\z

In \REF{lit:ex_na}, we encounter an existential sentence using \textit{ŋa-}, while \REF{lit:ex_me} provides an example of the use of \textit{me-} in an existential construction; it is structurally the same as \REF{lit:ex}, the only difference being the verb. Interestingly, \textcite[227]{Wagner-Nagy2016} states that this is deemed impossible, saying that \textit{me-} cannot be used in an existential sentence with an animate figure. It should be noted that the example cited here is elicited, meaning it appears out of context. 

Regarding predlocative clauses as well, there are examples found in the literature that deviate from the previously described pattern.


\ea \label{lit:loc_tana}
        \langinfo{Tundra Nenets}{}{\cite[5]{Hegedűs2017}; glossing slightly modified}\\
    {\gll T'uku xasawa labe-kana tańa.\\
  this man room-\textsc{loc} exist.\textsc{3sg} \\}\jambox*{[Predlocative]}
    \glt ‘This man is in the room.’\\
\z

In \REF{lit:loc_tana}, the existential verb is employed with a definite figure, yielding an unambiguous predlocative reading. 

The word order in both the existential and the predlocative clause still corresponds to what is typically assumed for their respective clause types, despite the unexpected verb: in the existential clause, the ground precedes the figure, while in predlocative clause, it is the other way around.

It is worth noting that, based on the same data as the examples quoted here from \textcite{Hegedűs2017, Hegedűs2018}, \textcite[250]{Mus2022} has observed in a more recent publication that the definiteness status of the figure (and consequently the categorization of clause structures, as discussed in \sectref{sec:theory}) is closely associated with the use of specific verbs. This suggests that the ``original" classification (as presented in \tabref{tab_verb}), which had been challenged by the examples \REF{lit:ex_na}--\REF{lit:loc_tana}, might indeed be correct:

\begin{quote}
    Outside of these marginal exceptions, it is only the verb \textit{tăńa-} that is acceptable in pure existential contexts, i.e., in contexts in which a new referent is introduced into the discourse. [...] Similarly, the verbs \textit{ŋa-}/\textit{me-} can only be used in contexts in which their subjects get a definite interpretation.\\
    \citep[250]{Mus2022}
\end{quote}


Regarding Forest Nenets, the limited mentions in the presented literature do not exhibit structural deviations from the descriptions of Tundra Nenets, with the exception of some variations in the shape of the verb stems. These variations are presented in \tabref{tab_verbFN}. 

\begin{table}
\caption{Verbal elements in Forest Nenets for predlocative, existential and predpossessive structures \label{tab_verbFN}}
\begin{tabular}{lll}
\lsptoprule
structure & affirmative & negative \\
\midrule
\multirow{2}{*}{predlocative} & \textit{ŋa-} {[}with inanimate figure{]}    & \textit{ďako-}    \\
& \textit{me-} {[}with animate figure{]} & \textit{ďako-} \\
\tablevspace
existential & \textit{tad’a-}                          & \textit{ďako-} \\
\tablevspace
predpossessive & \textit{tad’a-}; (\textit{ŋa-}) & \textit{ďako-}\\
\lspbottomrule
\end{tabular}
\end{table}

So, in summary, it can be stated that the classification of verbs within predlocative, existential, and predpossessive structures is not firmly established in either Tundra or Forest Nenets, with ongoing debates within the research community regarding the distribution. It may be more complex than initially outlined, and a comprehensive description encompassing both Tundra and Forest Nenets is still pending. The following sections aim to address this research gap and, at a minimum, offer an overview of the available possibilities based on a corpus.

\section{Predlocatives}\label{sec:loc}
It must be noted initially that the dataset for predlocative clauses in the corpus is less robust than those for existential and predpossessive clauses. This sentence type is simply less represented in the data, with only 54 predlocative sentences out of a total of 453 sentences (24 of them are Tundra Nenets, 30 Forest Nenets). Nevertheless, several tendencies can be discerned.

\subsection{Tundra Nenets}

As previously mentioned, the expectation derived from existing research literature is that predlocative clauses in (Tundra) Nenets employ two distinct verbs based on the animacy of the figure: \textit{ŋa-}\footnote{Note that I will refer to the forms presented in above tables in the text while the actual representation might vary a bit in the examples.} with inanimate figures and \textit{me-} with animate figures. This is also found in the Tundra Nenets dataset here. In \REF{cor:TN:loc_inan} a predlocative clause with an inanimate figure -- \textit{mʼado} ‘tent' -- and the verbal element \textit{ŋa-} is presented, while in \REF{cor:TN:loc_an} the grandfather \textit{jirʼi} is a human being, and thus animate, who is referred to with the verb \textit{me-}.

\ea
\ea \label{cor:TN:loc_inan}
        \langinfo{Tundra Nenets}{Eastern, Taimyr}{YaVD\_1976\_OldWomansSon\_flk.161}\\
        \gll Mʼado-naʔ ŋarka laŋg nʼi-nʼa ŋa-č.\\
      tent-\textsc{nom.sg.1pl} big steep.bank.\textsc{gen.sg} on-\textsc{loc.adv} be.\textsc{3sg.s}-\textsc{pst}\\
        \glt ‘Our tent was on a big steep bank.'\\
        \ex \label{cor:TN:loc_an}
        \langinfo{Tundra Nenets}{Eastern, Taimyr}{YaVD\_1976\_YoungShaman\_flk.137}\\
         \gll Jirʼi-r pʼi-xʼinʼa me.\\
        grandfather-\textsc{nom.sg.2sg} outside-\textsc{loc.pl} be.there.\textsc{3sg.s}\\
         \glt ‘Your grandfather is outside.’\\
\z\z

In addition to sentences conforming to the structure presented in the literature as the ``standard", there are instances where \textit{ŋa-} is employed contrary to the conventional description. In \REF{cor:TN:loc_an_na}, the verb is used with an animate figure. 

\ea\label{cor:TN:loc_an_na}
\langinfo{Tundra Nenets}{Eastern, Taimyr}{YaVD\_1974\_OldWar\_flk.024}\\
         \gll Čiki manto-r sʼidʼa to-xoʔ pont ŋe-b-wi\\
        that Tundra.Enets-\textsc{nom.sg.2sg} two lake-\textsc{obl.du} between be-\textsc{dur-nar.3sg.s}\\
         \glt ‘The Enets\footnote{The Enets are another Samoyedic ethnic group; their language is related to Nenets.} was between two lakes.’\\
\z

However, these exceptions are rare, and overall, the Tundra Nenets data supports the description of predlocative sentences. \tabref{tab:loc:TN} provides a percentage breakdown of the distributions, keeping in mind that the dataset is not very extensive.

\begin{table}
\caption{Percentage of animate and inanimate figures in predlocative clauses in Tundra Nenets \label{tab:loc:TN}}
\begin{tabular}{lrr}
\lsptoprule
 & animate & inanimate  \\
 \midrule
\textit{ŋa-}     & 8.33\%     & 91.67\%       \\
\textit{me-} & 100\%      & 0\%\\
\lspbottomrule
\end{tabular}
\end{table}

Up to this point, only affirmative structures have been described. However, it is noteworthy that no negated Tundra Nenets predlocative clauses are found in the corpus; further investigation of this sentence type must be omitted at this juncture. Nevertheless, it is worth mentioning that both \textcite{Wagner-Nagy2011} and \textcite{Mus2015} provide comprehensive descriptions of negated predlocative clauses in Tundra Nenets, and these studies are based on substantial datasets.

\subsection{Forest Nenets}

If we now turn our attention to Forest Nenets, the less extensively documented variety of Nenets, we observe in the corpus data that, similarly, the two copula verbs \textit{ŋa-} and \textit{me-} are frequently utilized: in \REF{cor:FN:loc_inan}, we encounter \textit{ŋa-} with the inanimate figure \textit{kaxeŋ todʼaaj} ‘sacred lake', and in \REF{cor:FN:loc_an}, \textit{me-} is used with a first person singular pronoun.

\ea
\label{cor:FN:loc_inan}
       \ea \langinfo{Forest Nenets}{Pur}{TPG\_2002\_HolyPlaces\_flk.003}\\
        \gll \dots Kaxe-ŋ to-dʼaaj    šat-nʼa-ŋi  käw-xana-n-t    ŋa.\\
     spirit-\textsc{gen.sg} lake-\textsc{aug.nom.sg} face-direction-\textsc{adj} side-\textsc{loc.sg-co-obl.2sg} be.\textsc{3sg.s} \\
        \glt ‘[If you stand up and look towards the sun,] the Sacred Lake will be on the left side.'\\
        
        \ex \label{cor:FN:loc_an}
        \langinfo{Forest Nenets}{Agan}{AAK\_200311\_MyLife\_nar.285}\\
         \gll Manʼ kaɬti-xĭna šan tʼaɬʼa-ŋ me-n-ŋa-m\dots\\
  \textsc{1sg.nom.sg} town-\textsc{loc.sg} more day-\textsc{gen.sg} be.there-\textsc{fut-co-1sg.sg.o}  \\
         \glt ‘I'll be in town for a few days [...]. ’\\
\z\z

Interestingly, there are no exceptions to this distribution in the data, i.e., these two verbs are not found in reverse distribution in the corpus (so far).

There are, however, additional structures that have not yet been mentioned. As demonstrated in the following examples \REF{cor:FN:loc_zero1} and \REF{cor:FN:loc_zero2}, the verb can be omitted. Instances of this phenomenon are found in the corpus exclusively in interrogative sentences. The figure can either be inanimate \REF{cor:FN:loc_zero1} or animate \REF{cor:FN:loc_zero2}.

\ea
\ea \label{cor:FN:loc_zero1}
        \langinfo{Forest Nenets}{Pur}{TPG\_2002\_Magpie\_flk.013}\\
        \gll Puxe-j xät-mi Salʼej-ŋ xät-mi kumši kuna?\\
      old.woman-\textsc{nom.sg.1sg} sew-\textsc{ptcp.pst} salyander-\textsc{gen.sg} sew-\textsc{ptcp.pst} coat.\textsc{nom.sg} where\\
        \glt ‘Where is my kumshi, sewn by Salyander's wife?'\\
                \ex \label{cor:FN:loc_zero2}
        \langinfo{Forest Nenets}{Pur}{TPG\_2002\_OldShaman\_flk.006}\\
         \gll Nʼu-ɬi kuna?\\
    child-\textsc{nom.sg.2sg} where\\
         \glt ‘Where is the daughter?’\\
\z\z

In addition to the ``classic" predlocative structures with copula verbs (or zero copula), a few constructions in Forest Nenets involve the use of a posture verb. The semantic weight carried by the verb itself in these constructions raises questions about the threshold at which these sentences can still be considered predlocative. The following \REF{cor:FN:loc_post} features the verb \textit{dʼošita-} ‘lie'. The scarce instances of this structure always involve an inanimate figure which is due to the fact that instances with animate figures have been excluded because it is assumed that animate referents have usually the ability to act or be positioned according to the semantic value of the verb and making it thus not strictly locational.  

\ea
\label{cor:FN:loc_post}
\langinfo{Forest Nenets}{Pur}{TPG\_2002\_Jay\_flk.005}\\
    \gll Pon čikʼe-xeʔna dʼošita-štu šuʔmʼa-n-ta ŋamsa.\\
 long this-\textsc{loc.pl} lie-\textsc{hab.3sg.s} jay-\textsc{gen-obl.3sg} meat.\textsc{nom.sg}\\
    \glt ‘For a long time, the meat of the jay lay there.’\\
 \z

In \tabref{tab:loc:FN}, the distribution of animacy status for the verbal elements in Forest Nenets is presented, highlighting that, aside from zero copula, the animacy of the figure appears to be the determining factor.

\begin{table}
\caption{Percentage of animate and inanimate figures in predlocative clauses in Forest Nenets \label{tab:loc:FN}}
\begin{tabular}{lrr}
\lsptoprule
 & animate & inanimate  \\
 \midrule
\textit{ŋa-}     & 0\%     & 100\%       \\
\textit{me-} & 100\%      & 0\%     \\
zero     & 50\%     & 50\%       \\
posture     & 0\%     & 100\%       \\
\lspbottomrule
\end{tabular}
\end{table}

While there are not many examples of negated sentences in Forest Nenets, these findings indicate that, on one hand, the expected strategy of negation with the negative existential verb is present in the data \REF{cor:FN:loc_neg1}. On the other hand, standard negation (negation verb + verb with connegative) is also employed \REF{cor:FN:loc_neg2}. Due to the limited number of examples, it is not possible to make definitive statements about the distribution of both possibilities.

\ea
\label{cor:FN:loc_neg1}
        \langinfo{Forest Nenets}{Pur}{TPG\_2002\_OldShaman\_flk.005}\\
    \gll \dots nʼe nʼu-dʼo-m čikʼi mʼaʔ-kna-n-ta dʼiku.\\
 woman.\textsc{nom.sg} child-\textsc{dim-acc.sg} this tent-\textsc{loc.sg-co-obl.3sg} \textsc{neg.ex.3sg.s}\\
    \glt ‘[The wife is crying,] but the daughter is not in the tent.’\\
\z

\ea
\label{cor:FN:loc_neg2}
\langinfo{Forest Nenets}{Agan}{AAK\_200311\_MyLife\_nar.290}\\
\gll Kuptana nʼi-t me-n kaɬti-xit.\\
long \textsc{neg-1sg.s} be.there-\textsc{cng} town-\textsc{abl.sg}\\
\glt ‘I will not be far from the town.’\\
\z

\subsection{Summary}
Overall, it becomes apparent that Tundra Nenets largely behaves as documented in the literature. While the division is not entirely accurate, the trend unmistakably indicates a close association between the distribution of the two copula verbs and the animacy of the respective figure. As of now, there are no examples of negation in the corpus, preventing any definitive statements or conclusions on this matter. In affirmative sentences in Forest Nenets, various strategies are evident. Alongside the utilization of the two copula verbs, the corpus shows two alternative methods for forming predlocative clauses: through the zero copula or posture verbs. The first strategy appears in the data exclusively in interrogative sentences; the latter playing a subordinate role in this context only. The corpus data suggest that in Forest Nenets as well, the verb \textit{ŋa-} is used with inanimate figures, while \textit{me-} is found with animate figures. Zero copula can occur with both, while posture verbs exclusively accompany inanimate figures. Negation examples are scarce in the dataset for Forest Nenets as well. However, alongside the anticipated strategy involving the negative existential verb, instances of employing standard negation are also found in the data.

However, it has been demonstrated that, essentially, the descriptions of predlocative sentences in Tundra Nenets can be extrapolated to Forest Nenets. The distribution discussed in \tabref{tab_verb} and \tabref{tab_verbFN}  in \sectref{sec:research} for this sentence type holds true.

\section{Existentials}\label{sec:ex}
In contrast to predlocative sentences, there are significantly more instances of existential sentences, allowing for more robust statements. In total, there are 239 sentences of this type with 140 Tundra Nenets and 99 Forest Nenets examples.

\subsection{Tundra Nenets}
Considering Tundra Nenets, based on the current state of research literature as seen in \sectref{sec:research}, it can be assumed that existential sentences are formed with the specific existential verb \textit{tańa-}, as illustrated in examples \REF{ex:TN:ex_tana1} and \REF{ex:TN:ex_tana2} from the corpus. Note that the verb can be marked for mood (as shown here with the narrative) and tense.

\ea
\label{ex:TN:ex_tana}
\ea \label{ex:TN:ex_tana1}
          \langinfo{Tundra Nenets}{Eastern, Taimyr}{YaVD\_1976\_OldWomansSon\_flk.172}\\
 \gll Ŋawna-wa-xana sʼiw xabʼi tanʼa-wi-sʼ.\\
    earlier-\textsc{emph-loc.sg} seven Khanty.\textsc{nom.sg} exist-\textsc{nar.3sg.s-pst}\\
         \glt ‘Long ago, there were seven Khanty people.’\\
              \ex \label{ex:TN:ex_tana2}
 \langinfo{Tundra Nenets}{Eastern, Taimyr}{YaVD\_1977\_SacredLandYadne\_flk.003}\\
         \gll Ŋawna-n-ta xebʼidʼa ja tanʼa-wi.\\
    earlier-\textsc{co-obl.3sg} sacred earth.\textsc{nom.sg} exist-\textsc{nar.3sg.s}\\
         \glt ‘Long ago, there was a sacred land.’\\
\z\z

However, when considering all sentences, it becomes evident that in the dataset, only about half of the sentences are formed in this manner. In the other half, the verb is either omitted entirely, or it is the copula   \textit{ŋa-}, a combination of  \textit{tańa-} and   \textit{ŋa-}, or a posture verb. 
\tabref{tab:ex:TN} illustrates the distribution of the various possibilities, arranged in descending order of frequency.

\begin{table}
\caption{Frequency of the verbal elements in existential sentences in Tundra Nenets \label{tab:ex:TN}}
\begin{tabular}{lr}
\lsptoprule
 \textit{tańa-}   & 49,47\%            \\
 zero     & 26,32\%         \\
  \textit{ŋa-}     & 21,05\%        \\
  \textit{tańa-}   + \textit{ŋa-}& 2,11\%      \\
posture     & 1,05\%      \\
\lspbottomrule
\end{tabular}
\end{table}

The second most common option is, therefore, a zero copula. However, based on the data, this is only possible in specific cases: most instances in the corpus involve the quantifier \textit{ŋoka} ‘many' \REF{ex:TN:many}  but the quantifier \textit{čanʼo} ‘few' \REF{ex:TN:few} is also possible. These structures represent ``predicative quantifier predication type" (\cite[168]{Hengeveld1992}); these predications are employing a predicative quantifier which serves as locative and possessive predication. Additionally, with figures specified by a numeral, zero copula can appear \REF{ex:TN:number}. Omission of the verb seems to be impermissible without either of these modifiers. Furthermore, the verb can be omitted with tense markers (as seen in \REF{ex:TN:few} and which is also the case for other nominal predication) but not with mood markers (see below). Note that is something that has already been described for (structurally alike) predpossessive clauses by \textcite[235]{Wagner-Nagy2011}.

\ea
\ea \label{ex:TN:many}
          \langinfo{Tundra Nenets}{Eastern, Taimyr}{YaVD\_1973\_ForestGiant\_flk.013}\\
 \gll Tanʼana ŋilʼeka-ʔ ŋoka-nʼuʔ.\\
    there evil.spirit-\textsc{nom.pl} many.\textsc{3sg.s-emph}\\
         \glt ‘There are a lot of evil spirits out there.’\\
                 \ex \label{ex:TN:few}
          \langinfo{Tundra Nenets}{Eastern, Taimyr}{YaVD\_1973\_ForestGiant\_flk.003}\\
         \gll Nʼenej nʼeneč-a-ʔ ŋulʼiʔ čanʼo-č.\\
    true person-\textsc{ep-nom.pl} totally few.\textsc{3sg.s-pst}\\
         \glt ‘And there were few real people.’\\
               \ex \label{ex:TN:number}
              \langinfo{Tundra Nenets}{Eastern, Taimyr}{YaVD\_1977\_Inggrav\_flk.045}\\
              \gll Čeč-u-mčej mʼade-na-n-ta ŋob mʼaʔ\\
    four-\textsc{ep-ord} place.of.tent-\textsc{loc.adv-co-obl.3sg} one tent.\textsc{nom.sg}\\
         \glt ‘There was one tent on the fourth nomad camp place.’\\
             
\z\z

As the previously described option is only possible under certain circumstances, the third most common possibility -- \textit{ŋa-} in existential sentences -- is also frequently used with quantifiers: in about two-thirds of these examples, there is also a quantifier such as ‘many' or a specific numeral that modifies the nominal phrase. In \REF{ex:TN:na_many}, \textit{ŋoka} ‘many' is combined with the copula \textit{ŋa-} which cannot be omitted in this context because \textit{ŋoka} cannot take mood markers or other derivational suffixes itself: ``nominal predicates can take on tense, but no mood markers in Nenets" \citep[235]{Wagner-Nagy2011}. Hence, the copula is needed to express the habitual. 

In \REF{ex:TN:na_inani}, the figure is marked with \textit{nʼaxarʔ} ‘three'. Both these examples represent again the predicative quantifier predication type (\cite[168]{Hengeveld1992}) but as can be seen in \REF{ex:TN:na_ani}, it is also possible when the figure is in prenominal position specified by a numeral, as here with \textit{ŋop} ‘one'. In both cases, the copula \textit{ŋa-} must be present to take on the narrative marker. 

It is notable that only \textit{ŋa-} is used, while \textit{me-} has not been documented in existential sentences in the Tundra Nenets corpus so far. Both possibilities have been mentioned in the existing research literature (see \REF{lit:ex_na} and \REF{lit:ex_me} in \sectref{sec:research}), but the example using \textit{me-} in this type of sentence has been elicited without further context. It might be the case that this actually does not happen in a discourse situation. The lacking of one of the verbs also means that the distribution established for predlocative sentences regarding the animacy of the figure is not maintained here. In the dataset, there are instances of using \textit{ŋa-} with both animate (\REF{ex:TN:na_many} and \REF{ex:TN:na_ani}) and inanimate \REF{ex:TN:na_inani} figures in existential sentences.

\ea
\ea \label{ex:TN:na_many}
          \langinfo{Tundra Nenets}{Eastern, Taimyr}{YaML\_1973\_ThreeEnetsSisters\_flk.007}\\
         \gll \dots ŋarka to-xona xalʼa-da ŋoka ŋe-sʼti.\\
    big lake-\textsc{loc.sg} fish-\textsc{nom.sg.3sg} many be-\textsc{hab.3sg.s}\\
         \glt ‘[...] there were many fish in the big lake.’\\
                   \ex \label{ex:TN:na_inani}
                 \langinfo{Tundra Nenets}{Eastern, Taimyr}{YaYA\_1973\_ThreeDaughters\_flk.036}\\
         \gll Mʼa-tu nʼaxarʔ ŋe-wi.\\
    tent-\textsc{nom.sg.3pl} three be-\textsc{nar.3sg.s}\\
         \glt ‘There were three tents.’\\
            \ex \label{ex:TN:na_ani}
          \langinfo{Tundra Nenets}{Eastern, Taimyr}{YaVD\_1973\_Tesyada\_nar.070}\\
 \gll Mʼa-ki mʼa-kăna ŋopoj-rʼi xasawa ŋe-wi.\\
    tent-\textsc{adjz} tent-\textsc{loc.sg} one-\textsc{lim} man.\textsc{nom.sg} be-\textsc{nar.3sg.s}\\
         \glt ‘There was only one man in the tent.’
\z\z

In one-third of cases with \textit{ŋa-}, however, the verb also occurs without a quantifier, as illustrated in the following example \REF{ex:TN:na_unquant}. In all these cases, the verb is further marked for mood or with other derivational suffixes (such as habitual). One could stipulate that this is an assimilation to the Russian pattern, where the existential verb (or existential participle) is only used in the present tense, even though \textit{tańa-} can take tense and mood markers (as shown in \REF{ex:TN:ex_tana}) which would make it possible to use the ``standard" Tundra Nenets existential verb. 

\ea \label{ex:TN:na_unquant}
\langinfo{Tundra Nenets}{Eastern, Taimyr}{YaVD\_1979\_HolyLegend1\_flk.077}\\
\gll Sew xun-kana lapta ŋe-wi.\\
eye.\textsc{gen.sg} length-\textsc{loc.sg} field.\textsc{nom.sg} be-\textsc{nar.3sg.s}\\
\glt ‘How many views, there was a tundra plain everywhere.’\\
\z

Another less frequent option is the combination of two verbs, namely the existential verb and the copula \textit{ŋa-}. Once again, the animacy of the figure is not decisive (in \REF{ex:TN:com1} the figure is animate, while in \REF{ex:TN:com2} it is inanimate), but upon examining the data, a different pattern becomes apparent: 
In the examples, if-conditions are expressed, indicating dubitative or conditional statements.

\ea
\ea \label{ex:TN:com1}
\langinfo{Tundra Nenets}{Eastern, Taimyr}{YaVD\_1976\_ThreeShamans\_flk.034}\\
 \gll Nʼenej nʼeneč-a-ʔ tanʼa-wi ŋe-ptu \dots\\
true person-\textsc{ep-nom.pl} exist-\textsc{ptcp.pst} be-\textsc{dub.3sg.s}\\
\glt ‘If there were real humans [...].’\\
\ex \label{ex:TN:com2}
\langinfo{Tundra Nenets}{Eastern, Taimyr}{YaVD\_1977\_MoonAndSun\_flk.031}\\
\gll \dots xajerʔ tanʼa-wi ŋe-bʔ \dots \\
sun.\textsc{nom.sg} exist-\textsc{ptcp.pst} be-\textsc{cond}\\
\glt ‘[...] if the sun existed, [it would be very good].’\\
\z\z

The last and smallest group of verbs in existential clauses consists of posture verbs. Once again, it is challenging to draw an exact boundary, so these examples must be treated with caution. Nonetheless, posture verbs do not play a significant role in Tundra Nenets existential clauses and do not occur regularly or grammaticalized. In \REF{ex:TN:post}, the use of the posture verb is justified by the fact that it involves humans who can also assume a bodily position, even though the humans mentioned here are dead and, thus, seen as inanimate.

\ea
\label{ex:TN:post}
          \langinfo{Tundra Nenets}{Eastern, Taimyr}{YaVD\_1977\_Inggrav\_flk.047}\\
    \gll Mʼad mʼu-nʼa čet xalʼmʼar jusʼida.\\
  tent.\textsc{gen.sg} inside-\textsc{loc.adv} four dead.man.\textsc{nom.sg} lie.\textsc{3sg.s}\\
    \glt ‘There were four dead men lying in the tent..’\\
\z

Overall, it becomes evident that Tundra Nenets (based on corpus data) essentially has two possible verbal elements in affirmative existential sentences: the existential verb \textit{tańa-} and the copula \textit{ŋa-}. Based on the available data, it is not possible to determine the reasons for choosing one option over the other. \tabref{tab:ex:TN_per} also illustrates the distribution concerning the animacy of the figures, which, in this case, also does not appear to be the determining factor. The use of zero copula or the combination of \textit{tańa-} and \textit{ŋa-} is also unaffected by animacy, but there are other limiting factors; posture verbs do not play a significant role in Tundra Nenets.

\begin{table}
\caption{Percentage of animate and inanimate figures in existential clauses in Tundra Nenets \label{tab:ex:TN_per}}
\begin{tabular}{lrr}
\lsptoprule
& animate & inanimate  \\
 \midrule
\textit{tańa-}   & 25.53\%     & 74.47\%       \\
 zero     & 56\%     & 44\%  \\
   \textit{ŋa-}     & 50\%     & 50\%       \\
 \textit{tańa-}   + \textit{ŋa-}& 50\%     & 50\%       \\
posture     & 0\%     & 100\%       \\
\lspbottomrule
\end{tabular}
\end{table}


Negative existential sentences in Tundra Nenets are, as expected, mostly expressed through the negative existential verb \REF{ex:TN:neg1}, but there are a few instances where standard negation is used \REF{ex:TN:neg2}. Again, it is not evident from the data what triggers this variation but the latter option is of relatively minor importance and appears sporadically. It is, however, noticeable that the emphatic clitic \textit{wi} always occurs with the standard negation, which might be the sole reason as it cannot be combined with the negative existential verb.  

\ea
\label{ex:TN:neg1}
\langinfo{Tundra Nenets}{Eastern, Taimyr}{LEP\_1977\_YonggadaSyarmuj\_flk.052}\\
    \gll Nʼi-i-nʼiʔ siŋgrʼo-du tamna juŋgu-nʼuʔ.\\
  sibling-\textsc{pl-obl.1du} news-\textsc{nom.sg.3pl} still \textsc{neg.ex.3sg.s-emph}\\
    \glt ‘There are still no news from our family.’\\
  \z

\ea
\label{ex:TN:neg2}
  \langinfo{Tundra Nenets}{Eastern, Taimyr}{LEP\_1977\_YonggadaSyarmuj\_flk.117}\\
    \gll \dots ja sʼi-dʼa wi-nʼi tanʼa-ʔ.\\
  earth.\textsc{gen.sg} hole-\textsc{nom.sg.3sg} \textsc{emph-neg.3sg.s} exist-\textsc{cng}\\
    \glt ‘[...] there's no such thing as a hole in the ground.’\\
\z

Overall, in affirmative existential sentences in Tundra Nenets, approximately half are expressed with the existential verb \textit{tańa-}, while a fifth use the copula \textit{ŋa-} without a clear trigger. Zero copula, however, is only possible with quantifiers, and the combination of \textit{tańa-} and \textit{ŋa-} is also only possible in certain contexts. Posture verbs do not play a systematic role in Tundra Nenets. The negation of these structures is as expected, but there are some instances where standard negation is also used.

\subsection{Forest Nenets}

For Forest Nenets, it also was assumed that the use of the existential verb is the default option in existential sentences. The use of this verb is illustrated in the following example:

\ea
\langinfo{Forest Nenets}{Pur}{PIK\_TPG\_2002\_KontscheeyKopanyta\_flk.PIK.023}\\
\gll \dots tajnʼa šexʼeɬi-udʼi tadʼa \dots\\
  there road-\textsc{dim.nom.sg} exist.\textsc{3sg.s}\\
\glt ‘[...] there is a path [...]’\\
\z

However, the picture is even more complex than in Tundra Nenets, and there are additional options in the Forest Nenets data. \tabref{tab:ex:FN} provides an overview of the distribution and usage in descending order. Here, it becomes apparent that there are a total of seven options, and the existential verb (standing alone) accounts for only about one-fifth of them.

\begin{table}
\caption{Frequency of the verbal elements in existential sentences in Forest Nenets\label{tab:ex:FN}}
\begin{tabular}{lr}
\lsptoprule
  \textit{tad’a-} & 21,28\%           \\
  zero     & 21,28\%       \\
    \textit{ŋa-}     & 17,02\%       \\
  \textit{tad’a-} + \textit{ŋa-}& 14,89\%       \\
    \textit{me-} & 10,64\% \\
    posture     & 8,51\%       \\
  \textit{tad’a-} + \textit{me-} & 6,38\% \\
  \lspbottomrule
\end{tabular}
\end{table}


While in Tundra Nenets, it was observed that zero copula, although common, is subject to conditions, in Forest Nenets, no such structured limitation is evident. While such a structure can also occur with a quantified figure \REF{ex:FN:zero1}, it is much more common for this not to be the case,  but the figure being an unmodifed noun phrase \REF{ex:FN:zero2}. An example with \textit{ŋoka} ‘many' with zero copula is not attested in the data at all, contrary to the Tundra Nenets data presented above. It can be assumed, therefore, that this option -- which appears in the data as frequently as the structure with the existential verb -- is also a normal means of forming existential sentences in Forest Nenets.

\ea
        \ea \label{ex:FN:zero1}
         \langinfo{Forest Nenets}{Pur}{TPG\_2002\_DogHalf\_flk.041}\\
 \gll \dots ŋisi-n-ta nʼaxaɬ tʼu ŋaškʼi.\\
   settlement-\textsc{lat.sg-obl.3sg} three ten child.\textsc{nom.sg}\\
         \glt ‘[...] there are thirty children in his camp.’\\
                   \ex \label{ex:FN:zero2}
             \langinfo{Forest Nenets}{Pur}{TPG\_2002\_Pine1\_flk.005}\\
 \gll\dots mʼaʔ-ta käw-xana poŋku. \\
   tent-\textsc{obl.3sg} side-\textsc{loc.sg} whirl.\textsc{nom.sg}\\
         \glt ‘There is a whirlwind near the tent.’\\
\z\z


This is also true for the use of \textit{ŋa-} -- it is the third most common way to form existential sentences. There seem to be no decisive factors that distinguish the choice of this linking element from the other two mentioned options. However, there is one limiting factor regarding  \textit{ŋa-} in existential clauses in Forest  Nenets:  this verb -- like in the predlocative clauses -- almost exclusively appears with inanimate figures. This is contrary to the data from Tundra Nenets, where \textit{ŋa-} was used with both animate and inanimate figures. One of the reasons for that might be that -- again in contrast to Tundra Nenets where the verb \textit{me-} was not used in existential sentences -- there are instances in Forest Nenets where \textit{me-} is used in existential sentences (in about a tenth of the clauses), and this copula is almost exclusively paired with animate figures. Thus, the animacy distinction, established for predlocative clauses in \sectref{sec:loc}, is also (partly) present in Forest Nenets existential clauses. In \REF{ex:FN:na}, \textit{ŋa-} is shown with an inanimate figure, and in \REF{ex:FN:me} the people are animate and are thus more likely to be paired with \textit{me-}.

\ea
\label{ex:FN:na}
         \langinfo{Forest Nenets}{Agan}{ILA\_200411\_Lake\_nar.033}\\
    \gll \dots ŋaɬka to-j ŋä-š.\\
big lake-\textsc{nom.sg.1sg} be.\textsc{3sg.s-pst}  \\
    \glt ‘[...] there was a big lake.’\\
\z

\ea
\label{ex:FN:me}
\langinfo{Forest Nenets}{Pur}{PIK\_TPG\_2002\_KontscheeyKopanyta\_flk.PIK.095}\\
\gll Ŋajŋ čikʼe-xʼenna ŋin-i-ʔko-ta-xaŋ nʼeša-ʔ ŋʼimi-ʔ me-xeŋ.\\
again this-\textsc{loc.sg} bow-\textsc{ep-dim-obl.3sg-nom.du} person-\textsc{nom.pl} other-\textsc{nom.pl} be.there-\textsc{3du.s}\\
\glt ‘Again, there are two people with bows.’\\
\z

Another option is the combination of the existential verb and one of the two copulas. While, in Tundra Nenets, this option was only found with \textit{ŋa-} and subject to conditions when it occurred (if-clauses), this cannot be transferred to Forest Nenets, nor can any other specific conditions be identified. However, it remains the case that \textit{ŋa-} strongly prefers to be combined with an inanimate figure \REF{ex:FN:tadana}, while \textit{me-} is usually paired with an inanimate figure \REF{ex:FN:tadame}, thus keeping the animacy distinction of the two copulas intact. It is notable that in almost all cases, the preceding existential verb is in the converb form, and all tense and mood markers are attached to the following copula verb.

\ea\label{ex:FN:tadana}
         \langinfo{Forest Nenets}{Agan}{AAK\_200311\_MyLife\_nar.190}\\
    \gll Wajma tʼa-xaʔna waŋk-tʼaaj-ʔ tatʼa ŋe-štuʔ.\\
  bad earth-\textsc{loc.pl} pit-\textsc{aug-nom.pl} exist.\textsc{cvb} be-\textsc{hab-3pl.s}\\
    \glt ‘There were large pits in those bad places.’\\
   \z

\ea
  \label{ex:FN:tadame}
         \langinfo{Forest Nenets}{Pur}{TPG\_2002\_DewakuRiver\_flk.023}\\
    \gll Čikʼe-xäna ŋup wäʔku, ŋup puxša, nʼi nʼu-č tadʼa me.\\
  this-\textsc{loc.sg} one old.man.\textsc{nom.sg} one old.woman.\textsc{nom.sg} woman.\textsc{nom.sg} child-\textsc{nom.sg.3du} exist.\textsc{cvb} be.there.\textsc{3sg.s}\\
    \glt ‘There  [in the tent] were an old man, an old woman and their daughter.’\\
    \z

Similar to Tundra Nenets, also in Forest Nenets there are some non-systema\-tized posture verbs that can be interpreted existentially, as shown in \REF{ex:FN:post}. Here, the verb \textit{nu-} ‘stand' is used.

\ea
  \label{ex:FN:post}
         \langinfo{Forest Nenets}{Agan}{ILA\_200411\_Lake\_nar.019}\\
    \gll I ŋaɬka tʼetʼaj mʼa-t käw-xana nʼuča tʼetʼaj mʼat-uutʼi nu-štu \dots\\
  and big leather.covered tent-\textsc{gen.sg} side-\textsc{loc.sg} small leather.covered tent-\textsc{dim.nom.sg} stand-\textsc{hab.3sg.s}\\
    \glt ‘And near the big tent, there was a small tent [...]’\\
  \z

As mentioned earlier, it could not be deduced from the data, apart from posture verbs, which play a very minor role, whether there is a specific context in which a particular verbal element can be used. The only distinction that can be made is the separation between \textit{ŋa-} and \textit{me-} with regard to the animacy of the figure -- even though especially the use of \textit{me-} can divert from that distinction as shown in \REF{ex:FN:meinani}. 

\ea
  \label{ex:FN:meinani}
        \langinfo{Forest Nenets}{Pur}{TPG\_2002\_Wader\_flk.008}\\
    \gll Čikʼe-n kanutaɬma-ŋ mʼu-ŋ mʼaʔ-dʼi mʼi-xe-j.\\
this-\textsc{loc.adv} deer.path-\textsc{gen.sg} inside-\textsc{lat.adv} tent-\textsc{dim.nom.sg} be.there-\textsc{hort-3sg.m}\\
    \glt ‘[...] Come on, [our tent] will be right on the road.’\\
\z

However, the overall tendencies are still apparent in the numbers presented in \tabref{tab:ex:FN_per}, where each option is categorized by whether the figures are animate or inanimate. Posture verbs, as in all cases so far, can only occur with inanimate entities.

\begin{table}
\caption{Percentage of animate and inanimate figures in existential clauses in Forest Nenets \label{tab:ex:FN_per}}
\begin{tabular}{lrr}
 \lsptoprule
 & animate & inanimate  \\
 \midrule
 \textit{tad’a-} & 10\%     & 90\%       \\
 zero     & 30\%     & 70\%       \\
  \textit{tad’a-} + \textit{ŋa-}& 14.29\%     & 85.71\%       \\
  \textit{ŋa-}     & 0\%     & 100\%       \\
   \textit{tad’a-} + \textit{me-} & 100\%     & 0\%       \\
\textit{me-} & 80\%      & 20\%     \\
posture     & 0\%     & 100\%       \\
\lspbottomrule
\end{tabular}
\end{table}

Negative existential sentences in Forest Nenets function similarly to Tundra Nenets: for the most part, negation is expressed through the negative existential verb \REF{ex:FN:neg1}, but there are also some examples of standard negation \REF{ex:FN:neg2}. In these cases, the emphatic clitic is always used.

\ea
\label{ex:FN:neg1}
        \langinfo{Forest Nenets}{Pur}{PIK\_TPG\_2002\_KontscheeyKopanyta\_flk.PIK.169}\\
    \gll \dots kem-ta dʼiku \dots\\
  blood-\textsc{nom.sg.3sg} \textsc{neg.ex.3sg.s}\\
    \glt ‘[He cut off that woman's tongue,] but there was no blood [...]’\\
\z

\ea
\label{ex:FN:neg2}
\langinfo{Forest Nenets}{Pur}{PIK\_TPG\_2002\_KontscheeyKopanyta\_flk.PIK.224}\\
\gll Nʼa-xăt-ta tačipʼa ŋuʔ wi-nʼi tadʼa-ʔ.\\
  friend-\textsc{abl.sg-obl.3sg} shaman.\textsc{nom.sg} also \textsc{emph-neg.3sg.s} exist-\textsc{cng}\\
    \glt ‘There is no shaman besides her.’\\
\z

In summary, for Forest Nenets, it can be stated that it is more fragmented and less clear-cut than Tundra Nenets: there are no clear triggers for a particular verbal element apart from animacy. It is, therefore, evident that a separate examination of Forest Nenets is imperative, and drawing conclusions from Tundra Nenets may not be warranted.

\subsection{Summary}

For Tundra Nenets, it is established that existential sentences primarily use the verb \textit{tańa-}, marked for mood and tense. However, only half of the sentences in the dataset follow this pattern; the rest use \textit{ŋa-}, a combination of \textit{tańa-} and \textit{ŋa-}, a posture verb, or omit the verb entirely. The zero copula occurs with quantifiers or numerals, while \textit{ŋa-} is often used with quantifiers when mood is marked. The verb \textit{me-} is not documented in existential sentences, indicating that \textit{ŋa-} can pair with both animate and inanimate figures. Additionally, combinations of \textit{tańa-} and \textit{ŋa-} express dubitative or conditional statements. Posture verbs are rare and not grammaticalized. Negation primarily uses the negative existential verb, with some instances of standard negation. 

Existential clauses in Forest Nenets are more complex than in Tundra Nenets: The existential verb accounts for only about one-fifth of the options, with seven different structures observed. Unlike in Tundra Nenets, zero copula is common without specific conditions, often occurring with an unmodified noun phrase. \textit{ŋa-} is frequently used but almost exclusively with inanimate figures, unlike in Tundra Nenets. \textit{me-} appears in about a tenth of the clauses and is paired with animate figures, maintaining an animacy distinction. Combining the existential verb with copulas is less conditional than in Tundra Nenets, with \textit{ŋa-} preferring inanimate and \textit{me-} preferring animate figures. Non-systematized posture verbs can also be used existentially. Negative existential sentences use the negative existential verb, with some instances of standard negation. 

Thus, the data reveal distinct patterns in the use of existential verbs in Tundra Nenets and Forest Nenets.

\section{Predpossessives}\label{sec:poss}
The dataset for possessive sentences consists of 160 sentences, divided between Tundra Nenets (109) and Forest Nenets (51). 

\subsection{Tundra Nenets}
For Tundra Nenets, it was shown in \tabref{tab_verb} that primarily the existential verb should be used, but \textit{ŋa-} is also a possible option that occurs and the verb can be omitted. Since the focus here is on the linking element, the exact formal structure of the possessor is secondary but will be touched upon later.

In \REF{pos:TN:ex}, there is an illustration of predpossessive sentences with the existential verb.  This is the default option in Tundra Nenets predpossessive clauses. It is also evident that the animacy of the possessee does not interact with the verb, as was the case in existential sentences. In \REF{pos:TN:ex1}, the possessee is inanimate, while in \REF{pos:TN:ex2}, it is animate. Note also, that the possessor is only marked with the respective possessive suffix which is the most frequent option.

\ea
\label{pos:TN:ex}
\ea \label{pos:TN:ex1}
         \langinfo{Tundra Nenets}{Eastern, Taimyr}{YaVD\_1973\_ForestGiant\_flk.020}\\
 \gll Xar-mʼi tanʼa, tipka-mʼi tanʼa \dots\\
    knife-\textsc{nom.sg.1sg} exist.\textsc{3sg.s} axe-\textsc{nom.sg.1sg} exist.\textsc{3sg.s}\\
         \glt ‘I have a knife, I have an axe [...]’\\
                \ex \label{pos:TN:ex2}
         \langinfo{Tundra Nenets}{Eastern, Taimyr}{YaML\_1973\_ThreeEnetsSister\_flk.004}\\
 \gll Nʼe nʼa-du tanʼa-wi.\\
    woman.\textsc{nom.sg} older.sister-\textsc{nom.sg.3pl} exist-\textsc{nar.3sg.s}\\
         \glt ‘They had a sister.’\\
           
\z\z

As described by \textcite[235]{Wagner-Nagy2011} -- and analogously to existential sentences described above -- the use of both \textit{ŋa-} and zero copula in predpossessive clauses is limited: it is only possible when the figure is marked with a quantifier or numeral. The copula is used accompanied by mood markers or additional derivational suffixes, while the verb is omitted when nothing or only tense (as it is possible to attach it to the nominal predicate) is marked. In the following example \REF{pos:TN:na}, two clauses are shown, in both the copula is in the narrative mood and, thus, present, while in both instances in \REF{pos:TN:zero} the verb is omitted. Note, however, that again as in existential structures in Tundra Nenets but in contrary to predlocative clauses, \textit{ŋa-} is not limited to be paired with inanimate figures; the same holds true for zero copula. Also as already shown for existential structures, there are both ``predicative quantifier predication type" (\cite[168]{Hengeveld1992}) but also instances where the numeral or quantifier is preceding the figure as in \REF{pos:TN:zero1}.

\ea
\label{pos:TN:na}
\ea \label{pos:TN:na1}
         \langinfo{Tundra Nenets}{Eastern, Taimyr}{YaVD\_1977\_Inggrav\_flk.007}\\
 \gll Xanjasej-da ŋulʼiʔ čanʼo ŋe-sʼta-wi.\\
    prey-\textsc{nom.sg.3sg} totally few be-\textsc{hab-nar.3sg.s}\\
         \glt ‘They had only a little prey.’\\
                \ex \label{pos:TN:na2}
         \langinfo{Tundra Nenets}{Eastern, Taimyr}{YaVD\_1977\_TwoVaj\_flk.006}\\
 \gll \dots ti-dʼiʔ nʼaxarʔ jurʔ ŋe-wi.\\
    reindeer-\textsc{nom.sg.3du} three hundred be-\textsc{nar.3sg.s}\\
         \glt ‘[...] they had merely three hundred reindeer.’\\
          
\z\z

\ea
\label{pos:TN:zero}
\ea \label{pos:TN:zero1}
         \langinfo{Tundra Nenets}{Eastern, Taimyr}{MMN\_1975\_TwoLostMen\_flk.003}\\
 \gll Sʼidʼa mʼa-čiʔ.\\
    two tent-\textsc{nom.sg.3du}\\
         \glt ‘They had two tents.’\\
                \ex 
         \langinfo{Tundra Nenets}{Eastern, Taimyr}{YaNN\_1976\_SonOfGod\_flk.086}\\
 \gll Te-waʔ ŋulʼiʔ čanʼo.\\
    reindeer-\textsc{nom.sg.1pl} totally few\\
         \glt ‘We have so few reindeer.’\\
          
\z\z

The distribution of options described above is also reflected in the negation of this sentence type: while the negative existential verb is the default option to negate predpossessive sentences \REF{pos:TN:exneg}, sentences with a quantifier are negated with standard negation \REF{pos:TN:copneg}, and this example is again of \citet[168]{Hengeveld1992} ``predicative quantifier predication type". 

So, we have an interplay of the existential verb in an affirmative clause with the negative existential verb in the negated counterpart, and the copula in an affirmative clause with negated copula in the negated clause. 

\ea
    \label{pos:TN:exneg}
         \langinfo{Tundra Nenets}{Eastern, Taimyr}{YaNN\_1976\_SonOfGod\_flk.003}\\
    \gll Nʼu-dʼiʔ juŋgu-wi.\\
  child-\textsc{nom.sg.3du} \textsc{neg.ex-nar.3sg.s}\\
    \glt ‘They had no children.’\\
  \z

\ea
  \label{pos:TN:copneg}
         \langinfo{Tundra Nenets}{Eastern, Taimyr}{LEP\_1977\_YonggadaSyarmuj\_flk.004}\\
    \gll Ti-du ŋoka nʼi-wi ŋa-ʔ \dots\\
  reindeer-\textsc{nom.sg.3pl} many \textsc{neg-nar.3sg.s} be-\textsc{cng}\\
    \glt ‘They didn't have many reindeer [...]’\\
     \z

\subsection{Forest Nenets}

In Forest Nenets, structures comparable to Tundra Nenets are found in this sentence type. This means that in the majority of predpossessive sentences, the existential verb is used, and the possessor is marked through the possessive suffix. \REF{poss:FN:ex} illustrates this.

\ea\label{poss:FN:ex}
\ea 
         \langinfo{Forest Nenets}{Pur}{PIK\_TPG\_2002\_KontscheeyKopanyta\_flk.PIK.040}\\
 \gll Dʼet-a-ɬ tadʼa \dots \\
    kettle-\textsc{ep-nom.sg.2sg} exist.\textsc{3sg.s}\\
         \glt ‘You have a cauldron [...]’\\
         \ex 
         \langinfo{Forest Nenets}{Pur}{ALY\_200206\_Life\_nar.011}\\
 \gll \dots ti-uwdʼe-ta tadʼa-ʔ \dots\\
    deer-\textsc{dim-nom.sg.3sg} exist-\textsc{3pl.s}\\
         \glt ‘[...] he has deer [...]’\\
\z\z

However, different strategies are employed when quantifiers are used. In statements about the future or when the verb is in a mood other than the indicative, the copula \textit{ŋa-} is used in this context \REF{poss:FN:na}, while the verb is omitted when it is in the indicative present or past (the according suffixes are than attached directly to the nominal predicate). This is the case in \REF{poss:FN:zero1} where the past marker \textit{-š} is found on the nominal phrase; in \REF{poss:FN:zero2} there is no tense or mood marker. This is true for structures in which the quantifier is the predicate of the sentences as well as in clauses where the figure is preceded by a number. 

\ea
\ea \label{poss:FN:na}
\label{poss:FN:na1}
\langinfo{Forest Nenets}{Pur}{PIK\_TPG\_2002\_KontscheeyKopanyta\_flk.PIK.006}\\
 \gll Nʼaxaɬ me-ta ŋe-što-maj, nʼaxaɬ pejču-n-ta dʼulaʔk ŋe-što-maj. \\
    three sort.of-\textsc{nom.sg.3sg} be-\textsc{hab-nar.3sg.s} three chock-\textsc{gen-obl.3sg} log.\textsc{nom.sg} be-\textsc{hab-nar.3sg.s}\\
         \glt ‘He had three of those, he used to have three logs.’\\
          \ex 
         \langinfo{Forest Nenets}{Pur}{TPG\_2002\_DewakuRiver\_flk.072}\\
 \gll Šatčet nʼu-č ŋaj-maj \dots\\
    eight child-\textsc{nom.sg.3du} be-\textsc{nar.3sg.s}\\
         \glt ‘They had eight children [...]’\\
         
\z\z

\ea
\label{poss:FN:zero}
\ea \label{poss:FN:zero1}
\langinfo{Forest Nenets}{Agan}{AAK\_200311\_MyLife\_nar.248}\\
\gll Manʼ šiča pʼemʼi-tʼenʼ-š. \\
\textsc{1sg.nom.sg} two shoe-pair-\textsc{pst}\\
\glt ‘I had two pairs of shoes.’\\
\ex \label{poss:FN:zero2}
\langinfo{Forest Nenets}{Pur}{TPG\_2002\_BaldHead\_flk.003}\\
 \gll Nʼimʼi-ŋ ŋup nʼuča, nʼimʼi-ŋ šiča nʼu-ta.\\
    other-\textsc{gen.sg} one small other-\textsc{gen.sg} two child-\textsc{nom.sg.3sg}\\
\glt ‘One had one child, and the other had two.’\\
\z\z

So far, in Forest Nenets, the same structures as in Tundra Nenets are found. However, there is an additional structure in FN: the combination of the existential verb \textit{tadʼa-} and \textit{me-} -- which was also found in existential sentences, thus, showing that these structures are alike. This seems to be usable only when expressing kinship relations. In \REF{poss:FN:me}, it is expressed that a certain man has a wife and a daughter. 

\ea
\label{poss:FN:me}
\langinfo{Forest Nenets}{Pur}{TPG\_2002\_OldShaman\_flk.003}\\
\gll Puxša-tta tadʼa me-maj, nʼe nʼu-ta tadʼa me-maj.\\
wife-\textsc{nom.sg.3sg} exist.\textsc{cvb} be.there-\textsc{nar.3sg.s} woman.\textsc{nom.sg} child-\textsc{nom.sg.3sg} exist.\textsc{cvb} be.there-\textsc{nar.3sg.s}\\
\glt ‘He had a wife and a daughter.’\\
\z

Turning to the negation of predpossessive sentences in Forest Nenets, it becomes evident that the negative existential verb plays a significant role, as shown in \REF{poss:FN:neg}. Whether the negation of sentences with quantifiers is done through the standard negation of the copula can only be speculated within the corpus, as there is no evidence for it. Following the pattern established for Tundra Nenets and the presence of similar structures in affirmative sentences, I presume that this is the case.

\ea
\label{poss:FN:neg}
\langinfo{Forest Nenets}{Agan}{AAK\_200311\_MyLife\_nar.251}\\
\gll keɬ-maʔ tʼako-štu, mʼaʔ-maʔ tʼako-štu \dots \\
oven-\textsc{nom.sg.1pl} \textsc{neg.ex-hab.3sg.s} tent-\textsc{nom.sg.1pl} \textsc{neg.ex-hab.3sg.s}\\
\glt ‘We didn't have the stove, we didn't have the tent, [we lived in a shack made of hay].’\\
\z

\subsection{Summary}
For Tundra Nenets, the primary verb used in predicative possessive sentences is the existential verb, but \textit{ŋa-} and zero copula can also occur. The animacy of the possessee does not affect the verb used. Both \textit{ŋa-} and zero copula are limited to cases with a quantifier or numeral and differ based on tense or mood markers. In negation, the negative existential verb is default, while sentences with a quantifier use standard negation, showing an interplay between existential and copula verbs in affirmative and negated forms.

Forest Nenets behaves similarly to Tundra Nenets: apart from sentences with quantifiers, the existential verb is used in predpossessive sentences. In sentences with quantifiers, the copula \textit{ŋa-} or zero copula is used. The only distinction is that in Forest Nenets, there is also the option to use the existential verb in combination with the copula \textit{me-}, which is interesting as it otherwise does not play a role in predpossessive sentences.

As was mentioned above, the focus of this study lies on the verbal element but there are a few notes on the expression of the possessor: In over 80\%, it is omitted and expressed solely through a possessive suffix. When lexically expressed, it appears either in the genitive (see \REF{poss:FN:zero2}) or nominative case (see \REF{poss:FN:zero1}). This holds true for both Tundra and Forest Nenets.

\section{Summary and outlook}\label{sec:summary}

Upon revisiting the three sentence types in the comparison for Tundra Nenets, it is evident that the provided descriptions — presented in \sectref{sec:research} — largely hold true. In predlocative clauses,  \textit{ŋa-} and \textit{me-} are employed, effectively distinguishing between animate and inanimate figures. In existential sentences, contrary to the hypothesis presented in \tabref{tab_verb}, not only the existential verb \textit{tańa-} is used, but also \textit{ŋa-}  or zero copula with quantified figures. They are also found in predpossessive sentences, which are structurally akin to existential sentences. A minor aspect to consider is the combination of \textit{tańa-}  and \textit{ŋa-}, which can occur in existential conditional clauses. 

Specifically for Tundra Nenets, \textit{me-} is exclusive to locative clauses (which is therefore the only one among the three structures where animacy plays a distinguishing role), whereas \textit{ŋa-} appears in various constructions. This aligns with \textit{ŋa-} serving as the more general copula, applicable in sentences with nominal predicates, for instance (\cite[243]{Mus2022}). Additionally, \textit{tańa-} is restricted from locative clauses, affirming its exclusive association with the existential (and consequently, predpossessive) domain.

\begin{table}
\caption{Overview of verbal elements in Tundra Nenets (affirmative) \label{tab:sum:TN_aff}}
\begin{tabularx}{\textwidth}{lQl}
\lsptoprule
structure & verbal element & usage \\
\midrule
\multirow{2}{*}{predlocative} & \textit{me-} & with animate figure \\
& \textit{ŋa-} & with inanimate figure \\
\midrule
\multirow{4}{*}{existential}  &  \textit{tańa-}         & default option \\
                              & \textit{ŋa-}            &            (mostly) with quantifiers                     \\
                              & zero           & with quantifiers                \\
                              & \textit{tańa-}   + \textit{ŋa-}      & in if-clauses                      \\
\midrule         
\multirow{3}{*}{predpossessive} &    \textit{tańa-}           &     default option      \\ 
                              &    \textit{ŋa-}            &   \multirow{2}{*}{with quantifiers}    \\
                                                               & zero                       & \\
\lspbottomrule
\end{tabularx}
\end{table}

In the examination of negation, there are fewer instances in the corpus. The line marked with * in \tabref{tab:sum:TN_neg} is not derived from the corpus but is taken from additional literature (\cite{Wagner-Nagy2011}, \cite{Mus2022}). Evidence is lacking for negated existential sentences with quantified figures, too. One can only assume that these would be negated with the standard negation pattern akin to the predpossessive negation.  What remains unclear from the data is when speakers choose to employ the emphatic standard negation of the affirmative existential verb instead of the negative existential verb. However, it is possible in any case, albeit as a subordinate variant. \tabref{tab:sum:TN_neg} presents all findings in negated clauses in Tundra Nenets.

\begin{table}
\caption{Overview of verbal elements in Tundra Nenets (negative) \label{tab:sum:TN_neg}}
\begin{tabularx}{\textwidth}{lQl}
\lsptoprule
structure    & verbal element & usage \\
\midrule
*predlocative   &     \textit{jaŋko-}           &      default option \\
\midrule
\multirow{2}{*}{existential}    &      \textit{jaŋko-}          &    default option   \\
                                &  (\textsc{emph+neg} + \textsc{cng}-form of the existential verb) & \\
\midrule
\multirow{2}{*}{predpossessive}    &      \textit{jaŋko-}          &    default option   \\
                                &  \textsc{neg} + \textsc{cng}-form of the existential verb & with quantifiers\\
\lspbottomrule
\end{tabularx}
\end{table}

In Forest Nenets, the picture appears less clear at certain points compared to Tundra Nenets. 

While in predlocative clauses, as in Tundra Nenets, a distinction is made between animate and inanimate figures using \textit{ŋa-}  and \textit{me-}, in questions, the verb is omitted. However, in existential sentences, Forest Nenets has multiple variants, including the existential verb, both copulas, a combination of both elements, and the option for the verb to be completely omitted. Thus, there are six distinct ways in which the verbal element can manifest in Forest Nenets existential sentences. The data did not provide indications of specific factors that trigger or prohibit a verbal element, and as such, further subcategories cannot be delineated at this point. However, \tabref{tab:ex:FN} in \sectref{sec:ex} reveals that while the existential verb occurs frequently, it is not significantly more common than other alternatives. What is interesting in comparison to Tundra Nenets is that, in addition to \textit{ŋa-}, \textit{me-}  also appears in existential sentences in Forest Nenets. This indicates that the distinction between animate and inanimate entities is partially maintained in this context as well. 

In Tundra Nenets, existential and predpossessive clauses were structurally similar; however, this is not necessarily the case in Forest Nenets. While there are numerous variations in existential clauses, predpossessive clauses resemble those in Tundra Nenets: the existential verb is generally used, but with quantifiers, \textit{ŋa-} is employed. This pattern is also observed in Forest Nenets. It is striking that existential and predpossessive clauses differ here, even though they should be structurally similar. The distinction regarding figures modified by a quantifier is only found in predpossessive clauses. It is noteworthy that \textit{ŋa-} in predpossessive clauses can also be used with an animate possessee, in contrast to predlocative and existential structures. However, even in predpossessive clauses there are some indications for the use of \textit{me-}: with kinship terms it is possible to employ this verbal element in Forest Nenets, implying that this type of copula always favors animate figures. In \tabref{tab:sum:FN_aff} all these findings are summarzied. 


\begin{table}
\caption{Overview of verbal elements in Forest Nenets (affirmative) \label{tab:sum:FN_aff}}
\begin{tabularx}{\textwidth}{XQll}
\lsptoprule
structure                   & verbal element & usage                           \\
\midrule
\multirow{3}{*}{predlocative} &    \textit{me-}            & with animate figure           \\
                              &    \textit{ŋa-}            & with inanimate figure         \\
                              & zero                        & possible in questions\\
\midrule
\multirow{6}{*}{existential}  &  \textit{tad’a-}         && \multirow{7}{*}{\rotatebox{90}{default options}} \\
                              & \textit{ŋa-}           & with inanimate figure                       \\
                               & \textit{tad’a-}   + \textit{ŋa-}                         & with inanimate figure\\
                               \tablevspace
                                & \textit{me-}                        & with animate figure \\
                                 & \textit{tad’a-}   + \textit{me-}     &     with animate figure                \\
                              & zero                       & \\
                             
\midrule         
\multirow{4}{*}{predpossessive} &    \textit{tad’a-}           &       default option    \\
                               \tablevspace
                              &    \textit{ŋa-}            &     \multirow{2}{*}{with quantifiers}  \\
                                        & zero                       & \\
                               \tablevspace
                                                  &  (\textit{tad’a-}  + \textit{me-})                       & with kinship terms\\
\lspbottomrule
\end{tabularx}
\end{table}

Negated sentences exhibit slightly less variation than their affirmative counterparts, but there are also deviations from expected patterns in Forest Nenets. 

For instance, in predlocative clauses, the negative existential verb is predominantly used, as anticipated. Nevertheless, there are instances where a copula is negated with standard negation in this context. 

In existential sentences, alongside the negative existential verb, the option of standard negation is present. However, similar to Tundra Nenets, it is consistently combined with the emphatic clitic in this context and is overall used only subordinately. 

In the data, there are only a few negated predpossessive sentences, all of which are negated with the negative existential verb. However, it is conceivable that, given the structure in affirmative sentences, standard negation with the affirmative existential verb would occur here if a quantified figure were present; this needs further verification with additional data. 

\tabref{tab:sum:FN_neg} presents an overview of the verbal elements in negated sentences in Forest Nenets in the corpus data.

\begin{table}
\caption{Overview of verbal elements in Forest Nenets (negative) \label{tab:sum:FN_neg}}
\begin{tabularx}{\textwidth}{lQl}
\lsptoprule
structure    & verbal element & usage \\
\midrule
\multirow{2}{*}{predlocative}   &        \textit{ďako-}         &     default option  \\
                &   (\textsc{neg} + \textsc{cng}-form of the copula) & \\
\midrule
\multirow{2}{*}{existential}   &        \textit{ďako-}         &     default option  \\
                &    \mbox{(\textsc{emph+neg} + \textsc{cng}-form of the existential verb)} & \\
\midrule
predpossessive &          \textit{ďako-}         &     default option    \\
\lspbottomrule
\end{tabularx}
\end{table}

As mentioned earlier, the focus of this study is on the verbal element within the three discussed sentence types. Nevertheless, it has been noted at the outset in \sectref{sec:research} that predlocative constructions differ from existential and predpossessive constructions in word order. Specifically it was stated that in predlocative structures, the figure precedes the ground, whereas in the other two structures, the order is reversed. Examining the corpus data under consideration, the following pattern emerges for Tundra Nenets: the statement holds true for all clauses in the data concerning locative sentences, with minimal deviations observed in existential sentences. The constructions listed here include cases where both the figure and the ground are overtly expressed. Structures with just figure or ground expressed as well as the placement of the verbal element are not taken into consideration. The findings are also summarized in \tabref{tab:sum:TN_order}.

\begin{table} 
  \caption{Word order in predlocative and existential clauses in Tundra Nenets  \label{tab:sum:TN_order}}
    \begin{tabular}{lrr}
    \lsptoprule
         & figure -- ground & ground -- figure \\
         \midrule
    predlocative    &  100\% & 0\%  \\
    existential    & 3,61\% & 96,39\%   \\
\lspbottomrule
    \end{tabular}
\end{table}

When creating the same overview (presented in \tabref{tab:sum:FN_order}) for Forest Nenets, it becomes evident that, similar to verb usage, there are greater deviations and variations. Nevertheless, the same tendencies persist: when both figure and ground are expressed, the former precedes in predlocative sentences, while in the majority of existential sentences, the order is reversed.

  \begin{table} 
  \caption{Word order in predlocative and existential clauses in Forest Nenets \label{tab:sum:FN_order}}
    \begin{tabular}{lrr}
    \lsptoprule
         & figure -- ground & ground -- figure \\
         \midrule
    predlocative     &  84,62\%& 15,38\%  \\
    existential     &  19,05\%& 80,95\%  \\
    \lspbottomrule
    \end{tabular}
  \end{table}

In predpossessive sentences, it is often not possible to discern a consistent order between the possessor (ground) and possessee (figure) since the possessor is not expressed in 80\% of the sentences. However, in cases where it is lexically expressed, it consistently appears at the beginning of the sentence, preceding the possessee. Predpossessive sentences, in terms of word order, thus align with existential sentences, as anticipated. This is again true for both Tundra and Forest Nenets.

In conclusion, the comparative analysis of Tundra Nenets and Forest Nenets identifies distinct patterns in the use of verbal elements across predlocative, existential, and predpossessive sentence types. Tundra Nenets exhibits clear preferences in verb choice, with \textit{me-} exclusive to predlocative clauses, \textit{ŋa-} as a general copula, and \textit{tańa-} reserved for existential contexts. Additionally, the combination of \textit{tańa-} and \textit{ŋa-} emerges in existential conditional clauses. Despite some deviations in negation, the patterns remain consistent. The analysis further emphasizes the differentiated role of animacy in verb selection.

Forest Nenets, on the other hand, presents more variability, particularly in existential sentences, where six distinct verbal patterns are identified. The inclusion of \textit{me-} in existential sentences indicates a partial retention of the animacy-based distinction. While predpossessive sentences align with existential structures, predlocative clauses and negation show noteworthy deviations.

In both languages, certain nuances, such as the interplay of quantifiers and the alternation between existential and predpossessive structures, add layers of complexity to the analysis. The findings contribute to a more precise understanding of verb usage in the Nenets languages and the complexity of their linguistic structures. Based on this study here, several topics could be further explored -- particularly the quantified structures, which, apart from \textcite{Hengeveld1992} and a passing mention in  \textcite{Stassen2009}, have not been systematically explored. Additionally, addressing the evident significant role of TAME and considering potential influences of language contact and Russian on these phenomena could be very interesting.

\section*{Acknowledgment}
This publication has been produced in the context of the joint research funding of the German Federal Government and Federal States in the Academies’ Programme, with funding from the Federal Ministry of Education and Research and the Free and Hanseatic City of Hamburg. The Academies’ Programme is coordinated by the Union of the German Academies of Sciences and Humanities.

This publication has been produced in the context of the research project ``Locative and existential predication in languages of the Ob-Yenisei area: typology and information structure'' funded by the Deutsche Forschungsgemeinschaft (DFG, German Research Foundation) -- project number 490822200.

I would like to express my sincere gratitude to Chris Lasse Däbritz and Rodolfo Basile for co-organizing the workshop “Locative and existential predication – Core and periphery”, held in Athens at SLE2023. I am also grateful to Anna Kampanarou and Gerson Klumpp for their insightful and constructive reviews.

\section*{Abbreviations}
\begin{tabularx}{.45\textwidth}{lQ}
\textsc{ade} & adessive\\
\textsc{adjz} & adjectivizer\\
\textsc{aug} & augmentative\\
\textsc{cng} & connegative \\
\end{tabularx}
\begin{tabularx}{.45\textwidth}{lQ}
\textsc{co} & co-affix\\
\textsc{dim} & diminutive\\
\textsc{dub} & dubitative\\
\textsc{emph} & emphatic\\
\end{tabularx}

\begin{tabularx}{.45\textwidth}{lQ}
\textsc{ep} & epenthetic\\
\textsc{ex} & existential\\
\textsc{hab} & habituative\\
\textsc{hort} & hortative\\
\textsc{lat} & lative\\
\end{tabularx}
\begin{tabularx}{.45\textwidth}{lQ}
\textsc{lim} & limitative\\
\textsc{m} & medial\\
\textsc{nar} & narrative\\
\textsc{ord} & ordinal\\
\textsc{s} & subjective\\
\end{tabularx}

\sloppy
\printbibliography[heading=subbibliography,notkeyword=this]
\end{document}
