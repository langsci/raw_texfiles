\documentclass[output=paper,colorlinks,citecolor=brown]{langscibook}
\ChapterDOI{10.5281/zenodo.16838072}
\author{Chris Lasse Däbritz\affiliation{Universität Hamburg \& Head Office of the German Science and Humanities Council}}
\title[\textsc{have}-verbs in Siberian Uralic languages]{Transitive \textsc{have}-verbs in possessive and existential clauses in Siberian Uralic languages}
\abstract{The present paper discusses and analyses the use of transitive \textsc{have}-verbs in possessive and existential clauses in three Siberian Uralic languages (Khanty, Mansi, Nganasan). It shows that all languages display etymologically unrelated \textsc{have}-verbs originating from agentive verbs. These \textsc{have}-verbs most frequently occur in possessive clauses, but to a lesser extent, they also spread to existential clauses, which may lead to ambiguities. Having described the language-specific patterns, the paper evaluates them and their grammaticalisation pathway against a broader theoretical and typological background, arguing that the Uralic cases entirely fit into cross-linguistically observed patterns.}



\IfFileExists{../localcommands.tex}{
   \addbibresource{../localbibliography.bib}
   % add all extra packages you need to load to this file

\usepackage{tabularx,multicol}
\usepackage{url}
\urlstyle{same}

\usepackage{listings}
\lstset{basicstyle=\ttfamily,tabsize=2,breaklines=true}

\usepackage{langsci-basic}
\usepackage{langsci-optional}
\usepackage{langsci-lgr}
\usepackage{langsci-osl}
% \usepackage{./langsci/styles/langsci-lgr}
% \usepackage{./langsci/styles/langsci-osl}
% \usepackage{langsci-gb4e}

\usepackage{tikz}
\usetikzlibrary{patterns,calc}
\pgfdeclarepatternformonly{south east lines}{\pgfqpoint{-0pt}{-0pt}}{\pgfqpoint{3pt}{3pt}}{\pgfqpoint{3pt}{3pt}}{
    \pgfsetlinewidth{0.6pt}
    \pgfpathmoveto{\pgfqpoint{0pt}{3pt}}
    \pgfpathlineto{\pgfqpoint{3pt}{0pt}}
    \pgfpathmoveto{\pgfqpoint{.2pt}{-.2pt}}
    \pgfpathlineto{\pgfqpoint{-.2pt}{.2pt}}
    \pgfpathmoveto{\pgfqpoint{3.2pt}{2.8pt}}
    \pgfpathlineto{\pgfqpoint{2.8pt}{3.2pt}}
    \pgfusepath{stroke}}
    
\usepackage{stmaryrd}
\usepackage{wasysym}
\usepackage{multirow}
\usepackage{caption}
\usepackage{subcaption}
\usepackage{mathrsfs}
\usepackage{qtree}

\usepackage{linguex}


   %pminos do not split footnotes
% \interfootnotelinepenalty=10000 %Footnote in Laporte chapters has to be split SN


%\DeclareIndexNameFormat{default}{%
%\nameparts{#1}%
%\usebibmacro{index:name}%
%{\index[names]}%
%{\namepartfamily}%
%{\namepartgiveni}%
% {}% L1
% {}% L2
%{\namepartprefix}% generates spurious space L3
%{\namepartsuffix}% generates spurious space L4
%}

%  {\DeclareIndexNameFormat{default}{%
%     \usebibmacro{index:name}{\index[names]}{#1}{#3}{#5}{#7}}}

%\DeclareIndexNameFormat{default}{%
%  \usebibmacro{index:name}{\sindex[nom]}{#1}{#3}{#5}{#7}}

%\DeclareIndexNameFormat{default}{%
%  \usebibmacro{index:name}{\sindex[person]}{#1}{#3}{#5}{#7}}
%\DeclareIndexNameFormat{default}{%
%\nameparts{#1} \usebibmacro{index:name}{\sindex[person]]}{\namepartfamily}{‌​\namepartgiven}{\nam‌​epartprefix}{\namepa‌​rtsuffix}}

%\newcommand{\smiley}{:)}

%\renewbibmacro*{index:name}[5]{%
%\usebibmacro{index:entry}{#1}%
%{\iffieldundef{usera}{}{\thefield{usera}\actualoperator}\mkbibindexname{#2}{#3}{#4}{#5}}}

% \newcommand{\noop}[1]{}

%remove for final
%\overfullrule=1mm

\newcommand{\tobi}[2]}}
\renewcommand{\S}[1]{\tobi{#1}{\textsc{*}}}

% this volume references
% puts: [this volume]
% already defined: \citetv
%\newcommand{\citepv}[1]{(\citeauthor{#1} \citeyear*{#1} [this volume])}
\newcommand{\citealtv}[1]{\citeauthor{#1} \citeyear*{#1} [this volume]}

%parentheses around example number
\newcommand{\pref}[1]{(\ref{#1})}

% in-text examples

\newcommand{\lnex}[1]{\textit{#1}} %target lang word
\newcommand{\lnlit}[1]{(lit.: `#1')} %literal reading
\newcommand{\lnlat}[1]{(#1)} % latinization
\newcommand{\lntrans}[1]{`#1'} %translation
\newcommand{\lnexl}[2]%
{\lnex{#1}{} \lnlat{#2}} % ex with latinization
\newcommand{\lnexlat}[3]{\lnex{#1}{} \lnlat{#2}{} \lntrans{#3}} % ex with latinization and tranl.

%ch01
\newcommand{\co}[1]{\mbox{\textbf{#1}}}

%ch09

\newcommand{\cyrbulg}[1]{\begin{otherlanguage*}{bulgarian}#1\end{otherlanguage*}}


%ch10
\newcommand{\nlp}{{\small NLP}}
\newcommand{\mwe}{{\small MWE}}
\newcommand{\rae}{{\small RAE}}
\newcommand{\lvc}{{\small LVC}}
\newcommand{\pos}{{\small P}o{\small S}}
%\newcommand{\todo}[1]{ \textcolor{red}{#1} }

%\renewcommand{\labelenumi}{\theenumi}
%\ainamefmt{{vv}{ll}{, ff}{, jj}} % fullname

\newcommand{\biberror}[1]{{\color{red}#1}}

\newcommand{\osenovaitem}{--~}
   %% hyphenation points for line breaks
%% Normally, automatic hyphenation in LaTeX is very good
%% If a word is mis-hyphenated, add it to this file
%%
%% add information to TeX file before \begin{document} with:
%% %% hyphenation points for line breaks
%% Normally, automatic hyphenation in LaTeX is very good
%% If a word is mis-hyphenated, add it to this file
%%
%% add information to TeX file before \begin{document} with:
%% %% hyphenation points for line breaks
%% Normally, automatic hyphenation in LaTeX is very good
%% If a word is mis-hyphenated, add it to this file
%%
%% add information to TeX file before \begin{document} with:
%% \include{localhyphenation}
\hyphenation{
    Beck-man
    Ngu-yen
    back-chan-nel
    back-chan-nels
    mo-not-o-nous
    ste-reo-typ-i-cal
}

\hyphenation{
    Beck-man
    Ngu-yen
    back-chan-nel
    back-chan-nels
    mo-not-o-nous
    ste-reo-typ-i-cal
}

\hyphenation{
    Beck-man
    Ngu-yen
    back-chan-nel
    back-chan-nels
    mo-not-o-nous
    ste-reo-typ-i-cal
}

   \boolfalse{bookcompile}
   \togglepaper[9]%%chapternumber
}{}

\begin{document}
\maketitle

\section{Introduction}
It has long been noted that possessive and existential clauses are tightly interwoven in many languages of the world (cf., e.g., \cite{Clark1978}, \cite{Heine1997}, \cite{Koch2012}, \cite{Creissels2023}, \cite{Karakoc2025}). For example, Turkish (Turkic, [nucl1301]) possessive clauses can roughly be paraphrased as ``X's Y exists'', cf. examples (\ref{turkish1}--\ref{turkish2}). 

\ea 
\label{turkish1}
\langinfo{Turkish}{}{personal knowledge}\\
\gll {Masa-da}   {kitap}  {var}.\\
     table-\textsc{loc} book    \textsc{ex}.\textsc{3sg}\\
\glt `There is a book on the table.'
\z 

\ea 
\label{turkish2}
\langinfo{Turkish}{}{personal knowledge}\\
\gll {Hasan'-ın}   {kitab-ı}  {var}.\\
     table-\textsc{gen} book-\textsc{poss3sg}    \textsc{ex}.\textsc{3sg}\\
\glt `Hasan has a book.'
\z 

In Turkish and other Turkic languages, the existential pattern has spread to possessive clauses, which can be proven by the inherent existential semantics of \textit{var}, originally being a nominal meaning `existing' that can occur in argument positions too \citep[817]{Johanson2021}. In contrast, there are also many languages in which transitive \textsc{have}-verbs have spread to existential clauses, e.g. Brazilian Portuguese (Indo-European, [braz1246]). Here, the existential clause in example (\ref{brazilian}) can be paraphrased as ``[It] has a book on the table''. Note that there is no semantic possessor anymore in the clause filling the subject position, but there is a locative adverbial indicating the place where the book is found. 

\ea 
\label{brazilian}
\langinfo{Brazilian Portuguese}{}{\cite[542]{Koch2012}}\\
\gll {Tem} {um} {livr-o} {sobre} {a} {mes-a}.\\
     have.\textsc{prs.3sg} \textsc{indf.m} book-\textsc{m} upon \textsc{def.f} table-\textsc{f}\\
\glt `There is a book on the table.'
\z 

Uralic languages usually lack transitive \textsc{have}-verbs and thus hardly form transitive possessive clauses, the only known exceptions being South Saami [sout2674], Khanty [khan1279], Mansi [mans1269] and Nganasan [ngan1291] \citep[977]{LaaksoWagner-Nagy2022}. The North Saami [nort2671] example (\ref{NorthSaami}) shows a possessive clause based on an existential clause, thus displaying the ``typical'' Uralic pattern, opposed to the South Saami example (\ref{SouthSaami}) which includes a form of the verb \textit{utnedh} `have'. 

\ea 
\label{NorthSaami}
\langinfo{North Saami}{}{\cite[978]{LaaksoWagner-Nagy2022}}\\
\gll {Lásse-s}   {lea}  {beana}.\\
     Lasse-\textsc{loc} be.\textsc{3sg} dog\\
\glt `Lasse has a dog.'
\z 

\ea 
\label{SouthSaami}
\langinfo{South Saami}{}{\cite[36]{Kowalik2016}}\\
\gll {manne}   {akte-m}  {peanna-m} {atnam}.\\
     \textsc{1sg.pro} one-\textsc{acc} pen-\textsc{acc} have.\textsc{prs.1sg}\\
\glt `I have a/one pen.'
\z 

Having these patterns in mind, the question arises whether the named Uralic languages use their transitive \textsc{have}-verbs also in existential clauses, similarly to Brazilian Portuguese. Relevant literature on existential clauses in Uralic languages (\cite{LaaksoWagner-Nagy2022}) does not name such usages, but one grammatical description of the Obdorsk Khanty variety [obdo1234] points to this phenomenon, stating that ``[...] in existential constructions the copula \textit{tajl} `there is/are' is used [...] although the copula \textit{uː(l)-} is also possible'' \citep[41]{Nikolaeva1999}. The named form \textit{tajl} is the third-person singular form of the verb \textit{taj-} `have' (see below for details). 

The aim of this paper is to provide a systematic analysis of \textsc{have}-verbs appearing in possessive and existential clauses in Khanty, Mansi and Nganasan, thus focusing on Uralic languages spoken in Siberia. 
The paper is structured as follows: \sectref{theory} introduces the needed theoretical background. \sectref{analysis} analyses \textsc{have}-verbs and their occurrence in the named languages. \sectref{grammaticalisation} discusses the results from \sectref{analysis} against typological and cross-linguistic approaches to the grammaticalisation of transitive \textsc{have}-verbs in existential clauses. \sectref{conclusion} summarises the paper and points to follow-up questions and topics. 

\section{Theoretical background} \label{theory}

This paper understands existential and possessive clauses as the linguistic expressions of the comparative concepts of \textsc{existential predication} and \textsc{possessive predication}, respectively. From the functional-semantic point of view taken here, existential predication cannot be discussed without considering locative predication. Both express the presence or absence of a \textsc{figure} (a.k.a. \textsc{theme}, \textsc{pivot}) at a \textsc{ground} (a.k.a. \textsc{location}, \textsc{coda}), differing only in their cognitive perspectivisation linguistically instantiated via information structure (\cite[94--100]{Hengeveld1992}, \cite[38]{Creissels2019}, among others). The terms \textsc{figure} and \textsc{ground} go back to Talmy's (\citeyear{Talmy1983}) seminal work on the cognitive structure of space: Whereas the figure is a movable referent whose site, orientation etc., are variable, the ground is the reference object for the site, orientation etc., of the figure (\cite[232]{Talmy1983}). Turning to locative and existential predication, suppose the speaker intends to describe the figure and its position at the ground. In that case, they use a locative predication that ``zooms'' on the figure element, so the perspectivisation starts from the figure and moves to the ground. In contrast, if the speaker wishes to tell something about the situation, location or circumstances in general, they use an existential predication, which is perspectivised from the ground to the figure. In terms of film language, locative predications thus provide a close-up view of the figure, whereas existential predications provide a total view of both the figure and ground. 

As for possessive predication, there is hardly a meaningful definition of possession to account for all its aspects properly, as convincingly shown by \citet[Ch. 1]{Heine1997}, \citet[10--25]{Stassen2009} and \citet[537--538]{Koch2012}. Still, as a minimal definition, one can say that possessive predication expresses an asymmetric and usually unidirectional relation of two entities, the possessor and the possessee, the latter belonging to the former. 
Like locative and existential predication, possessive predication can also be perspectivised in two ways. Perspectivisation from the possessor to the possessee yields the more frequent \textsc{(pred)possessives} (a.k.a. \textsc{have-possessives}; \textit{I have a dog}), whereas perspectivisation from the possessee to the possessor yields the less frequent \textsc{appertentives} (a.k.a. \textsc{belong-possessives}; \textit{the book is mine} $\sim$ \textit{the book belongs to me}). Note that the label \textsc{have-possessive} might be ambiguous, pointing to either the (pred)possessive construction or the transitive strategy in possessive clauses. Therefore, I use \textsc{(pred)possessive} and \textsc{appertentive} (following \cite{HaspelmathNonverbal}) when talking about the constructions, and leave the term \textsc{have-possessive} for the transitive strategy expressing predicative possession. 

In this context, it is worth noting that \citet[291--292, 306, 317]{Croft2022} and \citet[8--9, 21]{HaspelmathNonverbal} do not account for existential and (pred)possessive clauses as predications, arguing that they lack a topic-comment structure. From my point of view, however, this does not hold when accounting for information structure as being composed of the three levels \textsc{topic-comment} (what the predication is about), \textsc{focus-background} (what the speaker wants to emphasise) and \textsc{aforementionedness} or \textsc{information status} (what is known to the hearer) (cf. \cite{Molnár1991}, \cite{Lambrecht1994}, \cite{Däbritz2021}). Existential and (pred)possessive clauses certainly serve to introduce a new referent into the discourse (figure and possessee, respectively), which usually correlates with sentence focus and the relevant referents being indefinite on the latter two levels of information structure. Still, these clause types can have a topic-comment structure. In possessive clauses, the possessor frequently serves as their topic, which can be shown by cleft constructions such as \textit{as for my mother, she has a dog}, but also existential clauses may also include a topical ground element, as e.g. \textit{in the room are many candles} \citep[146--150]{Däbritz2021}. Additionally, existential clauses frequently lack an overtly expressed ground element such as in \textsc{generic existential} (a.k.a. \textsc{hyparctic}) clauses like \textit{there are particles smaller than atoms}. Still, I assume that such clauses may have the temporal/local circumstances of the described situation as the reference point for their predication, called contextual domain by \citet[70--71]{Francez2007}. This is clearer in utterances related to a concrete situation, e.g. \textit{there is no more beer}, when the speaker announces that there is no more beer in the given situation rather than that beer as a beverage is non-existent, but also generic existentials of the type shown above can be assumed to refer to the world or the universe. 

Consequently, two parameters distinguish the discussed domains: The propositional content opposes locative and existential clauses to both types of possessive clauses, whereas the perspectivisation pattern opposes locative and appertentive clauses to existential and (pred)possessive clauses (see also \citealt[216--218]{Bickerton2016} and \citealt[543--544]{Koch2012} for a concise description, the latter making up a fifth domain, namely generic existentials that by default lack an overt ground element, and arguing that their semantics differ from locational existentials). For the present paper, the relationship of locational and generic existentials need not be discussed further since the paper's crucial point of discussion is the borderline between existential and possessive predications. As for the latter, \citet[118]{Clark1978} pointedly names the animacy of the possessor and ground, respectively, as the main distinguishing feature. In possessive clauses, the possessor is prototypically animate, whereas in existential clauses, the ground element is usually inanimate, although animate ground elements are possible, too, cf. \textit{there are many people around us}. \tabref{tab:semantics} summarises the semantic features of the discussed domains, subsuming locational and generic existentials under the umbrella term \textit{existential}.

\begin{table}
\caption{Semantics of relevant domains}
\label{tab:semantics}
 \begin{tabularx}{1\textwidth}{X lll}
  \lsptoprule
            & proposition & perspectivisation  & animacy of Y\\
  \midrule
  existential  & X is at Y  & Y > X & typically inanimate\\
  locative  & X is at Y & X > Y & typically inanimate\\
  (pred)possessive  & Y has X & Y > X & typically animate\\
  appertentive  & Y has X & X > Y & typically animate\\
  \lspbottomrule
 \end{tabularx}
\end{table}


\section{Empirical analysis} \label{analysis}
\subsection{Languages and data} \label{data}

The languages under discussion here are Khanty, Mansi and Nganasan, all belonging to the Uralic language family. Khanty and Mansi form the Ob-Ugric branch of Uralic, whereas Nganasan forms part of the Samoyedic branch. All languages are severely endangered; according to the latest Russian census in 2020, there are 1,346 speakers of Mansi, 9,230 speakers of Khanty and 300 speakers of Nganasan. Whereas Nganasan is dialectally homogeneous (\cite[15]{Wagner-Nagy2019}), Mansi and Khanty exhibit large dialectal differences, so that one might even speak of several Khanty and Mansi languages (\cite[524--525]{SkribnikLaakso2022}). Usually, Mansi is divided into four dialectal groups (Northern, Western, Eastern and Southern/Tavda), and Khanty is divided into two major groups (Northern and Eastern), which in turn are divided into several dialects (ibid.). Where dialectal differences are relevant for the present study, I specifically point them out; elsewhere, I discuss all varieties together. 

The data used here come from different sources, the INEL Nganasan Corpus (\cite{BrykinaEtAl2024}) and the Ob-Ugric Database (\cite{SkribnikRiese2014}) being the most important sources. The INEL Nganasan Corpus currently includes 228 glossed texts with ca. 31,000 utterances and 189,300 tokens, recorded from 48 speakers. As for the Ob-Ugric Database, there are 101 glossed Khanty and 133 glossed Mansi texts available, coming from different sources and recorded from different speakers. Further statistics of the Ob-Ugric Database are unfortunately difficult to provide due to the structure of the database. Additionally, I took some Khanty and Mansi text collections into account in order to guarantee a dialectal balance as best as possible. In either case, the data consist of coherent, mostly monologic texts, thus, enabling to analyse relevant instances within their linguistic context. This is especially important when coming to the disambiguation of possessive and existential readings, as targeted in \sectref{ex}. 

From the described data, I extracted relevant sentences by searching texts either electronically or manually for the gloss `have' or the relevant translational equivalents in the languages. This search yielded the number of tokens displayed in \tabref{tab:tokens}. Given the size of the corpora and databases used, it should be noted that the numbers are a bit misleading since the INEL Nganasan Corpus is notably bigger than the Khanty and Mansi collections, respectively, in the Ob-Ugric Database. Thus, the frequency of \textsc{have}-verbs in Nganasan should probably not be over-exaggerated, as will be reflected also in Sections \ref{poss} and \ref{ex}. Finally, the overall number of tokens is too small and unbalanced for valid statistical analyses, which, however, can be regarded as unproblematic since this paper focuses on qualitative rather than quantitative research questions. 

\begin{table}
\caption{Forms and tokens of \textsc{have}-verbs}
\label{tab:tokens}
 \begin{tabularx}{0.8\textwidth}{XlY}
  \lsptoprule
            \textsc{language} & \textsc{lexeme} & \textsc{number of tokens}\\
  \midrule
  Nganasan  & \textit{hon-} & 126\\
  Khanty  & \textit{t\u{a}j-} \sim \textit{toj-} & 228\\
  Mansi  & \textit{on's'-} \sim \textit{än'\v{s}-} & 64\\
  \lspbottomrule
 \end{tabularx}
\end{table}

\subsection{Possessive clauses} \label{poss}

In this section, I examine how possessive clauses are built in the languages under discussion and how \textsc{have}-verbs play a role in their expression. First of all, it should be mentioned that neither of the three languages has only one single pattern to express possessive predication, but all languages exhibit both intransitive and transitive constructions. However, in Khanty, the intransitive construction is quite rare, almost exclusively occurring in the easternmost Vasyugan variety (fare1244). For the sake of completeness, I give examples of intransitive possessive clauses here as well. Following Stassen's (\citeyear{Stassen2009}) typology, the Khanty and Mansi patterns clearly represent locational possessives, i.e., the possessor is locative-marked and the possessee is the subject of the clause. The Nganasan case is more complex to analyse given the unmarked possessor (see \citealt{Wagner-Nagy2014} for details). However, irrespective of possessor coding, all structures resemble existential clauses, the possessee being the subject of the clause. 

\ea 
\label{NgPossIntr}
\langinfo{Nganasan}{INEL Nganasan Corpus}{ChND$\_$080719$\_$Life$\_$nar.026}\\
\gll {mɨŋ} {ŋonə-nuʔ} {labku-muʔ} {təi-t'u}.\\
     \textsc{1pl.pro} self-\textsc{gen.poss1pl} shop-\textsc{poss1pl} \textsc{ex-aor.3sg}\\
\glt `We have our own shop.'
\z 

\ea 
\label{KhaPossIntr}
\langinfo{Vasyugan Khanty}{}{\cite[128]{FilchenkoPotanina2010}}\\
\gll {mən-nə} {mərəm} {əŋk-äm} {wəl-wəl}.\\
     \textsc{1sg.pro-loc} only mother-\textsc{poss1sg} be-\textsc{prs.3sg}\\
\glt `I have only a mother.'
\z 

\ea 
\label{MnsPossIntr}
\langinfo{Western Mansi}{OUDB Northern Vagilsk Mansi Corpus}{Text ID 1263, 002}\\
\gll {tan} {pal-tæn-t} {nʲilə} {isʲyø̯} {oːl-i}.\\
     \textsc{3pl.pro} to-\textsc{poss3pl-loc} four daughter be-\textsc{prs.3sg}\\
\glt `They have four daughters.'
\z 

Coming to \textsc{have}-verbs occurring in possessive clauses, all languages exhibit ample evidence, although the share of \textsc{have}-possessives as opposed to existential possessives varies significantly over languages and dialects. Formally, in all languages, the possessor forms the subject of the clause, the possessee forms its direct object, and the \textsc{have}-verb serves as the linking element indexing person and number of the subject. 

\ea 
\label{NgPossHave1}
\langinfo{Nganasan}{INEL Nganasan Corpus}{ZhLH-TLJ$\_$0509XX$\_$PeopleOfTaimyr$\_$conv.ZhLH.070}\\
\gll {bən'd'ə-ʔ} {maðə-j} {ho-ndɨ-ʔ}.\\
     all-\textsc{pl} house-\textsc{acc.pl} have-\textsc{aor-3pl}\\
\glt `All [those people] have apartments.'
\z 

\ea 
\label{KhaPossHave1}
\langinfo{Sherkaly Khanty}{}{\cite[238--239]{Steinitz1975}}\\
\gll {mā} {tāpət} {poχ} {taj-t-əm}.\\
     \textsc{1sg.pro} seven son have-\textsc{prs-1sg}\\
\glt `I have seven sons.'
\z

\ea 
\label{MnsPossHave1}
\langinfo{Eastern Mansi}{OUDB Eastern Mansi Corpus}{Text ID 1557, 008}\\
\gll {møæ̯n} {woj} {kʷæl} {ənʲsʲ-ow}.\\
     \textsc{1pl.pro} bear hut have-\textsc{prs.1pl}\\
\glt `We have a bear hut.'
\z 

In examples (\ref{NgPossHave1})--(\ref{MnsPossHave1}), we see prototypical possessive relationships with the possessor being human and the possessee being inanimate in the case of ownership relations but human in the case of kinship relations. Since person-number indexing of the subject is obligatory and unambiguous in all languages, it is not surprising that the possessor can be omitted from the clause if retrievable from the context. Also in this case, there is often no doubt about a possessive reading given the animacy patterns named above and the (extra-)linguistic context of the relevant clause. Examples (\ref{NgPossHave2})--(\ref{MnsPossHave2}) show possessive clauses with omitted possessors in all languages. 

\ea 
\label{NgPossHave2}
\langinfo{Nganasan}{INEL Nganasan Corpus}{ChNS$\_$080214$\_$MountainRidges$\_$nar.013}\\
\gll {ma-bta} {s'iba-j} {hoŋ-h{ɨa͡}-ŋɨ-ʔ}, [...].\\
     what-\textsc{indf.acc} worker-\textsc{acc.pl} have-\textsc{int-inter-3pl}\\
\glt `Did they have workers, [or what did they have]?'
\z 

\ea 
\label{KhaPossHave2}
\langinfo{Yugan Khanty}{OUDB Yugan Khanty (1901) Corpus}{Text 1314, 017}\\
\gll {tʲeːt} {qoːtɐ} {poriɬ} {tojj-əm}.\\
     here somewhere drill have.\textsc{pst-1sg}\\
\glt `I had a drill somewhere here.'
\z

\ea 
\label{MnsPossHave2}
\langinfo{Western Mansi}{OUDB Pelym Mansi Corpus}{Text 1286, 002}\\
\gll {æk} {oɒ̯mp} {ænʲsʲ-iː-ɣə}.\\
     one dog have-\textsc{prs-3du}\\
\glt `They two have a dog.'
\z 

Generally, it can be said that the possessive reading of \textsc{have}-verbs is most salient in the target languages, roughly nine tenths of the analysed tokens unambiguously suggesting a possessive reading. However, the Khanty example (\ref{KhaPossHave2}) warrants discussion in more detail. The sentence comes from a tale in which two birds are haunted by an evil spirit and one bird just makes it home. In order to fight with the evil spirit, it wants to heat up a drill and poke the spirit with it. So, it is somehow presupposed in the given context that the bird owns such a drill, but it just does not know where it is. Consequently, the possessive reading is less salient here than in the Mansi example (\ref{MnsPossHave2}), among others, where the central message of the predication is that two aforementioned people own a dog. The lesser degree of salience of the possessive reading in (\ref{KhaPossHave2}), combined with the omission of the possessor, can be seen as a ``door-opener'' for existential readings of clauses formed with transitive \textsc{have}-verbs, which I will discuss in the following \sectref{ex}.

\subsection{Existential clauses} \label{ex}

As noted in \sectref{poss}, the majority of \textsc{have}-verbs occurring in the analysed material are involved in building possessive clauses. It was also shown that the possessive reading can become less salient under certain circumstances, namely the omission of the possessor from the clause and a fitting extra-linguistic context. However, clauses including both a formal possessor and possessee can also be ambiguous to some extent, as exemplified in (\ref{NgAmbigHave1}) through (\ref{MnsAmbigHave1}). 

\ea 
\label{NgAmbigHave1}
\langinfo{Nganasan}{INEL Nganasan Corpus}{ZhLH-TLJ$\_$0509XX$\_$PeopleOfTaimyr$\_$conv.TLJ.032}\\
\gll {tɨŋ} {korutə-ruʔ} {d'izel'nɨj} {elektrostancɨj} {ho-ŋɨ}?\\
     \textsc{2pl.pro} house-\textsc{poss2pl} diesel.\textsc{adj.acc} power.plant.\textsc{acc} have-\textsc{inter.3sg}\\
\glt `Does your village have a diesel generator?’ \sim `Is there a diesel generator in your village?'
\z 

\ea 
\label{KhaAmbigHave1}
\langinfo{Kazym Khanty}{OUDB Kazym Khanty Corpus}{Text 878, 005}\\
\gll {sʲi} {joxan-ət} {aːr} {xuɬ} {taj-s-ət}.\\
     that river-\textsc{pl} many fish have-\textsc{pst-3pl}\\
\glt `Those rivers had a lot of fish.' \sim `There were a lot of fish in those rivers.'
\z

\ea 
\label{MnsAmbigHave1}
\langinfo{Northern Mansi}{OUDB Northern Mansi Corpus}{Text 745, 047}\\
\gll {kol} {ala} {aːwi} {onʲsʲi}.\\
     hut roof door have.\textsc{prs-3sg}\\
\glt `The roof of the hut has a door/an opening.' \sim `There is a door/an opening in the roof of the hut.'
\z 
 
The examples (\ref{NgAmbigHave1})--(\ref{MnsAmbigHave1}) do not display prototypical possessive relationships since the possessor is an inanimate referent in either of them. This seems to evoke the possible existential readings given in the translations. From world knowledge, it appears to be at least as logical to say that there is an opening in a roof as that the roof has an opening. However, given the perfect formal resemblance to the unambiguous possessive examples (\ref{NgPossHave1})--(\ref{MnsPossHave1}) in \sectref{poss}, these examples must be accounted for as ambiguous with regard to their reading. 

Judging from the material analysed here, there are several indicators that allow dissolving the ambiguity between a possessive and an existential reading. First, there can occur number mismatches of the factual possessor of an entity and the invariant third-person singular form of the \textsc{have}-verb. In example (\ref{numbermismatch}), there are many people who own the wood in question; still, the \textsc{have}-verb displays third-person singular morphology, which favours the existential over a possessive reading. Theoretically, there could be a contextually available third-person singular possessor, either animate or inanimate, but in the given example, this is not the case. For the sake of transparency, it should be mentioned that such examples are very rare in the analysed Nganasan material. Thus, example (\ref{numbermismatch}) should not be taken as a reason for typologising Nganasan as a language regularly displaying \textsc{have}-existentials, but still it illustrates the possible grammaticalisation pathway. 

\ea 
\label{numbermismatch}
\langinfo{Nganasan}{INEL Nganasan Corpus}{MVL$\_$090807$\_$Hungabtadja$\_$flks.NN.152}\\
\gll {maa} {h{u͡a}a} {hoŋ-h{i͡a}ðɨ} {təʔ}, [...].\\
     what.\textsc{acc} wood.\textsc{acc} have-\textsc{infer.\textbf{3sg}} you.know\\
\glt `Of course, there is firewood, you know, [how would \textbf{they} be without wood].'
\z 

Second, the (extra-)linguistic context can lack any potential possessor. Example (\ref{nopossessor}) is from a text about the origin of the Khanty people. In the immediate left context, the Khanty leader is told by Komi-Zyryans that the conditions for reindeer herding are better on the other side of the Ob river. No potential (singular) possessor of the named pastures occurs in the given context. Hence, the reading is again existential rather than possessive. 

\ea 
\label{nopossessor}
\langinfo{Kazym Khanty}{OUDB Kazym Khanty Corpus}{Text 1024, 028}\\
\gll {jux-aŋ} {taxi} {taj-əɬ}.\\
     tree-\textsc{propr} place have-\textsc{prs.3sg}\\
\glt `[For winter pasture a place with trees is needed.] There is a place with trees there.'
\z

Given that in both examples (\ref{numbermismatch}) and (\ref{nopossessor}), there is no possessor expressed, one could label them as ``impersonal'' possessives which receive an existential reading, to which I come back in \sectref{grammaticalisation}. Finally, Northern Khanty dialects allow even the insertion of locative-marked ground elements in the clause which ``replace'' the would-be possessor. Examples (\ref{adverbial1}) and (\ref{adverbial2}) illustrate this. 

\ea 
\label{adverbial1}
\langinfo{Kazym Khanty}{OUDB Kazym Khanty Corpus}{Text 1024, 022}\\
\gll {aː} {sʲi} {jisən} \textbf{{tam}} \textbf{{muβ-ən}} {ɬowattaɬn} {βuɬi} {taj-əs}.\\
    but that long.ago \textbf{this} \textbf{land-\textsc{loc}} completely reindeer have-\textsc{pst.3sg}\\
\glt `But at that time, there were a lot of reindeer in this land.'
\z

\ea 
\label{adverbial2}
\langinfo{Obdorsk Khanty}{\cite{NikolaevaEtAl2019}}{Text ``Fox''}\\
\gll {amŏla} \textbf{{lip-el-na}} {tăj-əl} {pa}.\\
     what \textbf{inside-\textsc{poss3sg-loc}} have-\textsc{prs.3sg} and\\
\glt `And there is something inside.'
\z

As can be seen in the examples discussed in this section, there is thus hardly a clear-cut boundary between possessive and existential clauses. Instead, the overwhelming majority of the instances in the analysed material have a clear possessive reading and a small minority of instances have a clear existential reading. Between these two extremes, there is a continuum characterised by ambiguity and/or vagueness. After a short disgression into the negation of possessive and existential clauses in the analysed languages, \sectref{grammaticalisation} will discuss this pattern against a theoretical and typological background. 

\subsection{Negation} \label{neg}

In the preceding section, I analysed only affirmative clauses for the sake of clarity. However, it has been amply shown that non-verbal predication is prone to exhibiting special negative constructions (\citealt[13]{Croft1991}; \citealt[45]{Miestamo2005}; \citealt[2]{VeselinovaHamari2022}), so it is worth discussing it here, too. The analysed languages do not behave uniformly with regard to negative possessive and existential clauses. Khanty and Mansi regularly apply their standard negation pattern (negation particle + finite verb form) to possessive clauses, as shown by (\ref{KhaNeg1}) and (\ref{MnsNeg1}). 

\ea 
\label{KhaNeg1}
\langinfo{Sherkaly Khanty}{}{\cite[137]{Steinitz1989}}\\
\gll {tīɣ} {pu\v{s}kan} {ănt} {tăj-s-ət}.\\
   \textsc{3pl.pro} rifle \textsc{neg} have-\textsc{pst-3pl}\\
\glt `They did not have rifles.'
\z

\ea 
\label{MnsNeg1}
\langinfo{Northern Mansi}{OUDB Northern Mansi Corpus}{Text 889, 002}\\
\gll {sʲanʲ} {at} {onʲsʲ-e-əɣ}, {asʲ} {at} {onʲsʲ-e-əɣ}.\\
   mother \textsc{neg} have-\textsc{prs-3du} father \textsc{neg} have-\textsc{prs-3du}\\
\glt `They have no mother, they have no father.'
\z

Besides these patterns, possessive clauses can also be negated using a negative existential particle. If so, the clause is not transitive anymore, but rather an intransitive existential clause, as shown by (\ref{KhaNeg2}) and (\ref{MnsNeg2}). Like in the case of the affirmative intransitive possessive clauses discussed in \sectref{poss}, the possessee is the subject the clause, and the possessor is cross-referred to via possessive suffixes added to the possessee. The negative existential particle functions as the independent linking element in the indicative present tense (\ref{KhaNeg2}), whereas TAME categories call for a copula verb to be added (\ref{MnsNeg2}). Except for Southern Mansi dialects (sout3253), however, Khanty and Mansi use this strategy less frequently than the standard negation of the transitive \textsc{have}-verb. 

\ea 
\label{KhaNeg2}
\langinfo{Obdorsk Khanty}{\cite{NikolaevaEtAl2019}}{Text ``The cuckoo''}\\
\gll {mǟ} {wäj-l-am} {äntam-ət}.\\
   \textsc{1sg.pro} boot-\textsc{pl-poss1sg} \textsc{neg.ex-pl}\\
\glt `I have no boots.'
\z

\ea 
\label{MnsNeg2}
\langinfo{Southern Mansi}{INEL Mansi Corpus}{MAKh$\_$1903$\_$VillageEvsekova$\_$nar.002}\\
\gll {ɛw-i-püw-i} {iikəm} {aal-s}.\\
   girl-\textsc{poss3sg}-boy-\textsc{poss3sg} \textsc{neg.ex} be-\textsc{pst.3sg}\\
\glt `He had no children.'
\z

In Nganasan, possessive clauses are regularly negated by applying negative existentials (\ref{NgNeg1}). Though the standard negation of the transitive construction is possible and produced by speakers in elicitation (\cite[419]{Wagner-Nagy2019}; (\ref{NgNeg2})), this pattern is rarely observed in natural speech. 

\ea 
\label{NgNeg1}
\langinfo{Nganasan}{INEL Nganasan Corpus}{JMD$\_$080219$\_$MyLife$\_$nar.164}\\
\gll {mənə} {tə} {d'aŋku} {kuəd'ümu-mə}.\\
     \textsc{1sg.pro} well \textsc{neg.ex} man-\textsc{poss1sg}\\
\glt `So, I have no husband.'
\z 

\ea 
\label{NgNeg2}
\langinfo{Nganasan}{}{\cite[419]{Wagner-Nagy2019}}\\
\gll {ńuə} {ńi-sɨə} {sani-j} {honə-ʔ}.\\
     child \textsc{neg.aux-aor.3sg} toy-\textsc{acc.pl} have-\textsc{cng}\\
\glt `The child did not have toys.'
\z 

Coming to existential predication, the observed patterns thus suggest that transitive \textsc{have}-verbs may appear in negative existential predication in Khanty and Mansi, but not in Nganasan. However, this is not the case, with all three languages applying solely the intransitive existential construction as displayed in (\ref{KhaNegEx})--(\ref{NgNegEx}). 

\ea 
\label{KhaNegEx}
\langinfo{Kazym Khanty}{OUDB Kazym Khanty Corpus}{Text 1027, 001}\\
\gll {jisən} {kasəm-ən} {βuɬi} {antɵːm} {βɵːs}.\\
   in.ancient.times Kazym-\textsc{loc} reindeer \textsc{neg.ex} be.\textsc{pst.3sg}\\
\glt `In ancient times, there were no reindeer on the Kazym River.'
\z

\ea 
\label{MnsNegEx}
\langinfo{Eastern Mansi}{OUDB Eastern Mansi Corpus}{Text 1570, 058}\\
\gll [...], {mætə} {lʲəm} {pəl} {øæ̯tʲi}.\\
   \textit{} some bird.cherry \textsc{emph} \textsc{neg.ex}\\
\glt `[He goes around], there are no bird cherries.'
\z

\ea 
\label{NgNegEx}
\langinfo{Nganasan}{INEL Nganasan Corpus}{JMD$\_$080219$\_$MyLife$\_$nar.074}\\
\gll [...], {d'aŋuru-ʔ} {ńi-ni} {d'aŋguj-kə-ndu-ʔ} {sv'ad'ba-ʔ} {maa-ʔ}.\\
     \textit{} tundra-\textsc{gen.pl} on-\textsc{loc.adv} \textsc{neg.ex-iter-aor-3pl} wedding-\textsc{pl} what-\textsc{pl}\\
\glt `There are no weddings and so on in the tundra.'
\z 

To sum up this section, it can be observed that transitive \textsc{have}-verbs regularly occur in negative possessive clauses in Khanty and Mansi, whereas Nganasan shows this pattern only very infrequently. In negative existential clauses, the transitive \textsc{have}-verbs are not used in any of the languages. I will come back to this polarity-based constraint and discuss it in \sectref{grammaticalisation} in the context of generalisation patterns and grammaticalisation pathways. 

\section{Grammaticalisation pathway} \label{grammaticalisation}

\textsc{Grammaticalisation} is surely one of the key notions in any study dealing with diachronic changes going beyond phonetics and the lexicon. Not unexpectedly, there are many approaches to grammaticalisation and likewise many attempts to define it. In this paper, I conceive the concept of grammaticalisation from a functional-linguistic perspective, understanding it as a process in which a given linguistic element gains grammatical meaning. This can mean that the given element ``enters'' the grammatical space from the lexicon or that it becomes ``more'' grammaticalised by expanding its functional domains. (See \citealt{NarrogHeine2021} and \citealt{HarderBoye2021} for details). Here, I will discuss both aspects subsequently, first dealing with the lexical origins of transitive \textsc{have}-verbs in the Uralic languages and then discussing their spread to existential clauses. 

First, it should be noted that the \textsc{have}-verbs in Khanty, Mansi and Nganasan are, according to the best of today's knowledge, etymologically unrelated (\citealt[977]{LaaksoWagner-Nagy2022}). Their diachronic origin can thus not be investigated by applying the comparative method, but one has to rely on the relevant language-specific data available. The Khanty verb \textit{tăj}- \sim \textit{toj}- has the following meanings as attested by \citet[1400--1401]{Steinitz1966}: `have', `hold', `hold in hand', `keep', `give birth', `wear'. Similarly, Mansi \textit{onʲsʲ}- \sim \textit{änʲ\v{s}}- means `have', `hold', `contain', `carry' (\cite[172]{Honti2008}). These meanings can regularly be observed in the Ob-Ugric Database (\citealt{SkribnikRiese2014}). The Nganasan verb \textit{hon-} means `have' and `carry' $\sim$ `wear' (Russian: носить) according to \citet[197]{KosterkinaEtAl2001}. In the INEL Nganasan Corpus (\cite{BrykinaEtAl2024}), the latter meaning is only very sparsely attested, if at all, mostly in the seemingly derived form \textit{honsəi-}. Interestingly, Forest Enets [fore165], a close linguistic relative of Nganasan, displays the verb \textit{pɔnʼir-} whose initial part is regularly cognate to Nganasan \textit{hon-}, whereas \textit{-ir} is a multiplicative suffix (\citealt[828]{KhaninaShluinsky2023}). According to \citet[352--353]{SorokinaBolina2009}, this verb has the meanings `make', `use', `keep', among others, all of them being attested in the INEL Enets Corpus (\cite{KhaninaEtAl2024}). Additionally, the verb can also be used in a \textsc{have}-like fashion in possessive clauses in Forest Enets, cf. example (\ref{FEPoss}), although this usage is marginal from a quantitative point of view. Thus, there is at least indirect evidence that Nganasan \textit{hon-} may trace back to an agentive verb similar to its partial cognate in Forest Enets. 

\ea 
\label{FEPoss}
\langinfo{Forest Enets}{INEL Enets Corpus}{BVN$\_$1969$\_$HowWeLived$\_$nar.025}\\
\gll {ɔnɛj} {mɛz} {pɔnʼiŋa-aʔ}, {bɔlko-xuru-ʔ} {bunʼi-aʔ} {pɔnʼir-ʔ} {tɔ} {dʼubo-n}.\\
     true tent make-\textsc{1pl} balok-\textsc{emph-pl} \textsc{emph.neg.aux-1pl} do-\textsc{cng} that during-\textsc{loc.adv}\\
\glt `We had a tent, we even had baloks at that time.'
\z 

The observed polysemy is cross-linguistically common, and agentive verbs meaning `grasp', `seize', `take', `hold' etc. often serve as lexical sources for transitive \textsc{have}-verbs (\cite[47--48]{Heine1997}, \cite[54]{ChappellLü2022}). The apparent motivation for applying verbs with the given meaning lies in their semantics, namely the semantics of physically having, holding or taking something. Besides the physical contact of ``holder'' and ``holdee'', it is worth noting that the former is certainly prototypically human or animate, whereas the latter is prototypically inanimate. This concrete relationship of ``holder'' and ``holdee'' is then transferred to the more abstract relation of possession as defined and described above. The Khanty and Mansi \textsc{have}-verbs perfectly fit the cross-linguistic data regarding their lexical source as well as the given functional explanation. Only the available meanings `give birth' and `wear' in Khanty diverge to some extent but do not pose any greater problems either for establishing the grammaticalisation of the named lexical verbs into transitive \textsc{have}-verbs. As for the Nganasan \textsc{have}-verb, however, this conclusion can be drawn only with some reservation since there is only very sparse language-internal evidence for the described polysemy. The comparison with Forest Enets surely stands to reason that Nganasan \textit{hon-} might have (had) an agentive meaning as well, but synchronically, this cannot be proven. However, in all three languages, it is clear that the \textsc{have}-verbs indeed display the meaning `have', so it is not surprising at all that these verbs are used in possessive clauses. 

With regard to \textsc{have}-verbs used in existential clauses, it must first be shown that this usage is indeed secondary against their application in possessive clauses. There are at least two arguments in favour of this analysis. On the one hand, the lexical sources of the Uralic verbs -- as far as they are attested -- are also very common for transitive \textsc{have}-verbs, which are used in possessive clauses but not necessarily in existential clauses; an example for the latter pattern would be English \textit{have}. On the other hand, \textsc{have}-verbs occur much more frequently in possessive clauses than in existential clauses in the analyzed languages (see \sectref{poss}), and their occurrence is basically restricted to affirmative clauses in existential predication in all languages. Both aspects taken together, as well as the easily observable transitive syntax, leave no doubt about the primariness of the possessive construction against the existential one. 

The grammaticalisation of transitive possessive clauses to existential clauses has been amply discussed by \citet{ChappellLü2022} and \citet{Creissels2019,Creissels2023}. Studies all assume that the initial step of grammaticalisation is the impersonalisation of possessive clauses. As shown in Sections \ref{poss} and \ref{ex}, this holds for the Siberian Uralic languages as well: First, the possessor can be easily dropped since it is consistently indexed by person-number suffixes at the verb. Second, in appropriate contexts, no third-person possessor may be determined in the given context, as e.g. in the Khanty example (\ref{nopossessor2}). Formally, the main protagonist of the story, a girl, a house and the girl's room could be eligible for being the table's ``possessor'', but the overall context rather suggests an existential reading, so that the clause is semantically impersonal. 

\ea 
\label{nopossessor2}
\langinfo{Yugan Khanty}{OUDB Yugan (2010-) Khanty Corpus}{Text 1619, 094}\\
\gll {ottə} {pɐsən}, {pɐsən} {taj-ət}.\\
   ehm table table have-\textsc{prs.3sg}\\
\glt `[He went into the house, he entered the girl's room.] Ehm a table, there is a table.'
\z

The second and crucial step is the conventionalisation of the impersonalised possessive clause for expressing existential predication (\cite[73]{Creissels2019}; \cite[50]{Creissels2023}). This step can hardly be shown by linguistic material, but one can imagine that an intermediate stage is the interpretation `[somewhere], they\textsubscript{impers} have X' (\cite[50]{Creissels2023}). Likewise, it is conceivable that the omitted possessor is conceptually replaced by a contextually available ground element, as e.g. the house or the room in example (\ref{nopossessor2}). In either case, the construction becomes a conventionalised expression for the existential predication `there is X'. 

Finally, the ground element can be expressed by a locative complement, as in the Kazym Khanty example (\ref{adverbialbear}). However, it should be noted that this pattern is restricted to Northern Khanty varieties in the analysed data. Neither other Khanty varieties nor Mansi and Nganasan display it. 

\ea 
\label{adverbialbear}
\langinfo{Kazym Khanty}{OUDB Kazym Khanty Corpus}{Text 1091, 003}\\
\gll {in} {jɛməŋ} {nʲoɬ} {kɵrt-ən} {pupe-ɬ} {taj-s-eɬ}, {taj-s-eɬ}.\\
   now sacred cape village-\textsc{loc} bear-\textsc{poss3sg} have-\textsc{pst-3sg>sg} have-\textsc{pst-3sg>sg}\\
\glt `Now there was a bear in the village of the sacred cape.'
\z

Hence, the Siberian Uralic data neatly illustrate different steps of the grammaticalisation pathway proposed by \citet{ChappellLü2022} and \citet{Creissels2019,Creissels2023}. From a typological and worldwide perspective, existential clauses formed with transitive \textsc{have}-verbs are not as rare as one might suspect: Apart from the well-known European cases, \citet[74]{Creissels2019} accounts for a number of West African languages, and \citet{ChappellLü2022} describe them in many Mainland East and South East Asian languages. Additionally, pidgins and creoles are a group with a strong predominance of existential clauses formed with \textsc{have}-verbs, although most lexifier languages do not exhibit this pattern (\cite[74--75]{Creissels2019} and \cite{Michaelis2023}). Thus, Khanty, Mansi and -- to a significantly lesser extent -- Nganasan are no extra-ordinary cases from a worldwide perspective, but certainly so far the only Siberian languages, in which transitive \textsc{have}-verbs are attested in both possessive and existential clauses. 

\section{Conclusion and outlook} \label{conclusion}

The present paper started with the observation that Uralic languages usually do not display transitive \textsc{have}-verbs, the exceptions being the Siberian Uralic languages Khanty, Mansi and Ngansan, as well as South Saami varieties. Considering the cross-linguistically common spread of \textsc{have}-verbs to existential clauses, the initial question was whether the Siberian Uralic languages likewise exhibit their transitive \textsc{have}-verbs in existential clauses. It could be shown that this is indeed the case, although the possessive uses of the \textsc{have}-verbs are clearly more frequent in natural speech. Comparing the languages and varieties, Northern Khanty shows most ample evidence of \textsc{have}-existentials, followed by Eastern Khanty (except for the Vasyugan dialect) and Mansi (except for Southern Mansi varieties); Nganasan displays only a few unambiguous cases, but still provides interesting insights regarding the grammaticalisation pathway, as discussed above. Evaluating the data against a theoretical and typological background, the paper could illustrate that the Uralic data unequivocally support previous work and claims made by, inter alia, \citet{ChappellLü2022} and \citet{Creissels2023}. The Khanty and Mansi \textsc{have}-verbs clearly trace back to agentive verbs meaning `hold', `keep' or `carry', whereas this cannot be proven for Nganasan, but only indirectly stipulated when taking into account a partial cognate of the \textsc{have}-verb in Forest Enets. Furthermore, the verbs spread from possessive to existential clauses via conventionalising their impersonal usages in potentially ambiguous contexts. From a language-specific perspective, one could ask whether the grammaticalisation process is still ongoing since (1) negative clauses favour negative existential items over negated \textsc{have}-verbs, and (2) only Northern Khanty varieties allow inserting a locative-marked ground element in existential clauses formed with \textsc{have}-verbs. From an areal and typological perspective, one could surely ask whether there are more Siberian languages that use \textsc{have}-verbs at least as a side strategy in expressing possessive and existential predication. So far, this is hardly attested in the literature, but to me it seems not be excluded that a careful analysis of empirical language data reveals similar cases. 

Finally, the present paper shows how fruitful the work with language corpora can be for language typology, especially when it comes to co-expression patterns and ambiguities. Without their linguistic context, many of the above examples probably would have been significantly less expressive. Though acknowledging that large-scale typological studies can hardly be corpus-based, the study thus shows that linguistic typology and corpus linguistics can surely profit from each other. Consequently, further methodologically similar studies on co-expression patterns and grammaticalisation pathways in the realm of non-verbal predication seem highly promising. 


\section*{Abbreviations}
\begin{tabularx}{.5\textwidth}{@{}lQ@{}}
\textsc{aor} & aorist\\
\textsc{cng} & connegative\\
\textsc{emph} & emphasis\\
\textsc{ex} & existential\\
\textsc{infer} & inferential\\
\end{tabularx}%
\begin{tabularx}{.5\textwidth}{@{}lQ@{}}
\textsc{int} & intentional\\
\textsc{inter} & interrogativ\\
\textsc{iter} & iterative\\
\textsc{pro} & pronoun\\
\textsc{propr} & proprietive\\
\end{tabularx}

\section*{Acknowledgements}

This publication has been produced in the context of the research project ``Locative and existential predication in languages of the Ob-Yenisei area: typology and information structure'' funded by the Deutsche Forschungsgemeinschaft (DFG, German Research Foundation) -- project number 490822200. In addition, I thank the various commenters at SLE56 in Athens and ICHL26 in Heidelberg, the two reviewers and Beáta Wagner-Nagy for their inspiring ideas and suggestions.  

\sloppy
\printbibliography[heading=subbibliography,notkeyword=this]
\end{document}
