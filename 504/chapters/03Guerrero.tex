\documentclass[output=paper,colorlinks,citecolor=brown]{langscibook}
\ChapterDOI{10.5281/zenodo.16838058}
\author{Lilián Guerrero\orcid{}\affiliation{Universidad Nacional Autónoma de México}}
%\ORCIDs{}
\title{Locative/existential constructions in Southern Uto-Aztecan languages}
\abstract{Prior typological studies dealing with locative/existential constructions have ignored languages spoken in Mexico outside the Mesoamerican area. The present study fills this gap and examines constructions that code figure-ground relationships in Southern Uto-Aztecan languages. The aim of this chapter is threefold: to establish the number of locative predicates in each language under analysis, to determine the typology of locative predications in this language family, and to look at the semantic properties of locative predicates in one of these languages (Yaqui).

I show that Southern Uto-Aztecan languages display not only a rich inventory of locative predicates, but also variation in the use of postural (`sit', `stand', `lie'), general locative (`be located'), existential (`exist'), and positional verbs (`be inserted', `be stacked', `be leaning', and so on). Furthermore, I propose that, because Southern Uto-Aztecan languages typically use postural verbs in locative descriptions, they are examples of the locative type in \citegen{Stassen1997} typology of locative predication, and the small-set language type in \citegen{AmekaLevinson2007} typology. There are two major exceptions to this pattern. On the one hand, Mayo and some Nahuatl languages can be characterized as single-set language type since they only use a general locative verb. On the other hand, Yaqui is a good example of \citegen{Grinevald2006} intermediate-set language type, since it uses postural, general locative, existential, and positional verbs in locative constructions. After a thorough discussion of relevant Yaqui data, I reveal that the selection of locative/existential verbs is semantically motivated by the animacy, number, and shape of the figure, the properties of the ground, and spatial configurations. I also demonstrate that, at least in Yaqui, there is no syntactic or pragmatic evidence for a dedicated existential or locative-existential construction; only locative constructions are documented.}

\IfFileExists{../localcommands.tex}{
   \addbibresource{../localbibliography.bib}
   % add all extra packages you need to load to this file

\usepackage{tabularx,multicol}
\usepackage{url}
\urlstyle{same}

\usepackage{listings}
\lstset{basicstyle=\ttfamily,tabsize=2,breaklines=true}

\usepackage{langsci-basic}
\usepackage{langsci-optional}
\usepackage{langsci-lgr}
\usepackage{langsci-osl}
% \usepackage{./langsci/styles/langsci-lgr}
% \usepackage{./langsci/styles/langsci-osl}
% \usepackage{langsci-gb4e}

\usepackage{tikz}
\usetikzlibrary{patterns,calc}
\pgfdeclarepatternformonly{south east lines}{\pgfqpoint{-0pt}{-0pt}}{\pgfqpoint{3pt}{3pt}}{\pgfqpoint{3pt}{3pt}}{
    \pgfsetlinewidth{0.6pt}
    \pgfpathmoveto{\pgfqpoint{0pt}{3pt}}
    \pgfpathlineto{\pgfqpoint{3pt}{0pt}}
    \pgfpathmoveto{\pgfqpoint{.2pt}{-.2pt}}
    \pgfpathlineto{\pgfqpoint{-.2pt}{.2pt}}
    \pgfpathmoveto{\pgfqpoint{3.2pt}{2.8pt}}
    \pgfpathlineto{\pgfqpoint{2.8pt}{3.2pt}}
    \pgfusepath{stroke}}
    
\usepackage{stmaryrd}
\usepackage{wasysym}
\usepackage{multirow}
\usepackage{caption}
\usepackage{subcaption}
\usepackage{mathrsfs}
\usepackage{qtree}

\usepackage{linguex}


   %pminos do not split footnotes
% \interfootnotelinepenalty=10000 %Footnote in Laporte chapters has to be split SN


%\DeclareIndexNameFormat{default}{%
%\nameparts{#1}%
%\usebibmacro{index:name}%
%{\index[names]}%
%{\namepartfamily}%
%{\namepartgiveni}%
% {}% L1
% {}% L2
%{\namepartprefix}% generates spurious space L3
%{\namepartsuffix}% generates spurious space L4
%}

%  {\DeclareIndexNameFormat{default}{%
%     \usebibmacro{index:name}{\index[names]}{#1}{#3}{#5}{#7}}}

%\DeclareIndexNameFormat{default}{%
%  \usebibmacro{index:name}{\sindex[nom]}{#1}{#3}{#5}{#7}}

%\DeclareIndexNameFormat{default}{%
%  \usebibmacro{index:name}{\sindex[person]}{#1}{#3}{#5}{#7}}
%\DeclareIndexNameFormat{default}{%
%\nameparts{#1} \usebibmacro{index:name}{\sindex[person]]}{\namepartfamily}{‌​\namepartgiven}{\nam‌​epartprefix}{\namepa‌​rtsuffix}}

%\newcommand{\smiley}{:)}

%\renewbibmacro*{index:name}[5]{%
%\usebibmacro{index:entry}{#1}%
%{\iffieldundef{usera}{}{\thefield{usera}\actualoperator}\mkbibindexname{#2}{#3}{#4}{#5}}}

% \newcommand{\noop}[1]{}

%remove for final
%\overfullrule=1mm

\newcommand{\tobi}[2]}}
\renewcommand{\S}[1]{\tobi{#1}{\textsc{*}}}

% this volume references
% puts: [this volume]
% already defined: \citetv
%\newcommand{\citepv}[1]{(\citeauthor{#1} \citeyear*{#1} [this volume])}
\newcommand{\citealtv}[1]{\citeauthor{#1} \citeyear*{#1} [this volume]}

%parentheses around example number
\newcommand{\pref}[1]{(\ref{#1})}

% in-text examples

\newcommand{\lnex}[1]{\textit{#1}} %target lang word
\newcommand{\lnlit}[1]{(lit.: `#1')} %literal reading
\newcommand{\lnlat}[1]{(#1)} % latinization
\newcommand{\lntrans}[1]{`#1'} %translation
\newcommand{\lnexl}[2]%
{\lnex{#1}{} \lnlat{#2}} % ex with latinization
\newcommand{\lnexlat}[3]{\lnex{#1}{} \lnlat{#2}{} \lntrans{#3}} % ex with latinization and tranl.

%ch01
\newcommand{\co}[1]{\mbox{\textbf{#1}}}

%ch09

\newcommand{\cyrbulg}[1]{\begin{otherlanguage*}{bulgarian}#1\end{otherlanguage*}}


%ch10
\newcommand{\nlp}{{\small NLP}}
\newcommand{\mwe}{{\small MWE}}
\newcommand{\rae}{{\small RAE}}
\newcommand{\lvc}{{\small LVC}}
\newcommand{\pos}{{\small P}o{\small S}}
%\newcommand{\todo}[1]{ \textcolor{red}{#1} }

%\renewcommand{\labelenumi}{\theenumi}
%\ainamefmt{{vv}{ll}{, ff}{, jj}} % fullname

\newcommand{\biberror}[1]{{\color{red}#1}}

\newcommand{\osenovaitem}{--~}
   %% hyphenation points for line breaks
%% Normally, automatic hyphenation in LaTeX is very good
%% If a word is mis-hyphenated, add it to this file
%%
%% add information to TeX file before \begin{document} with:
%% %% hyphenation points for line breaks
%% Normally, automatic hyphenation in LaTeX is very good
%% If a word is mis-hyphenated, add it to this file
%%
%% add information to TeX file before \begin{document} with:
%% %% hyphenation points for line breaks
%% Normally, automatic hyphenation in LaTeX is very good
%% If a word is mis-hyphenated, add it to this file
%%
%% add information to TeX file before \begin{document} with:
%% \include{localhyphenation}
\hyphenation{
    Beck-man
    Ngu-yen
    back-chan-nel
    back-chan-nels
    mo-not-o-nous
    ste-reo-typ-i-cal
}

\hyphenation{
    Beck-man
    Ngu-yen
    back-chan-nel
    back-chan-nels
    mo-not-o-nous
    ste-reo-typ-i-cal
}

\hyphenation{
    Beck-man
    Ngu-yen
    back-chan-nel
    back-chan-nels
    mo-not-o-nous
    ste-reo-typ-i-cal
}

   \boolfalse{bookcompile}
   \togglepaper[3]%%chapternumber
}{}

%\pretocmd{\gll}{\def\eachwordone{\itshape}\def\eachwordtwo{\normalfont}}{}{}

\begin{document}
\maketitle

\section{Introduction} \label{Introduction}

Previous typological studies have demonstrated that the conceptual notions of location and existence are closely related, and consequently, many languages of the world use similar or even the exact same predication to express these semantic domains (\cite{Clark1978}; \cite{Freeze1992}; \cite{Hengeveld1992}; 
\cite{Stassen1997}, \citeyear{Stassen2009}; \cite{Koch2012}; \cite{Creissels2019}; \cite{McNally2016}; \cite{ChappellCreissels2019}; \cite{ChappellLü2022}; \cite{Däbritz2022LocativeSamoyedic}). Most of these studies conceive locative and existential predications as structures expressing the presence/absence of a figure (a.k.a. theme, pivot) in a ground (a.k.a. location, coda). The clauses in (\ref{Guerrero1}) and (\ref{Guerrero2}), from English and Spanish, respectively, encode prototypical figure-ground relationships: the figure is ‘the cup’ and the ground corresponds to ‘the table’.\footnote{This paper focuses on locative (\textit{The cat is on the bed}) and locative existential (\textit{There is a cat on the bed}) predications. In the literature, it is common for each predication type to be defined based on pragmatic factors. For instance, a locative clause is “a clause in which a non-focal referent is said to be in some location”, while an existential clause is “a clause in which a focal referent is said to be in some location, being (re-)introduced in the discourse” (\cite{Däbritz2023}). Since little is known about focal and non-focal domains in Uto-Aztecan languages at this point, I decided to use “locative” and “existential” as descriptive terms. As such, I use “locative description” to identify a structure for coding a figure-ground relationship, “locative predication” to indicate the predicative element in such descriptions, and “locative/existential verbs” to specify the lexical element linking the figure and the ground.}

\ea \label{Guerrero1}
\ea \label{Guerrero1a}
\textit{The cup \textbf{is} on the table.}\\
\ex \label{Guerrero1b}
\textit{There \textbf{is} a cup on the table.}\\
\z
\z

\ea \label{Guerrero2}
\langinfo{Spanish}{Indo-European, [stan1288]}{personal knowledge}\\
\ea \label{Guerrero2a}
\textit{La taza \textbf{está} en la mesa}.\\
\ex \label{Guerrero2b}
\textit{\textbf{Hay} una taza en la mesa.}\\
\z
\z

In English, the two entities involved in the locative description are linked by the multifunctional copula \textit{be}, although the syntactic structure for the existential predication (\ref{Guerrero1b}) is different from the locative (\ref{Guerrero1a}). In Spanish, each interpretation correlates with a different copula: \textit{estar} for location (\ref{Guerrero2a}), and \textit{haber} for existence (\ref{Guerrero2b}). Lezgian (Nakh-Daghestanian, [lezg1247]) also has several copula verbs, including the locatives \textit{gwa} ‘be at’, \textit{gala} ‘be behind’, \textit{kwa} ‘be below’, \textit{ala} ‘be on’, and \textit{awa} ‘be in’ (\cite{Haspelmath1993Lezgian}). Like English \textit{be}, the Lezgian copula \textit{awa \sim ’wa \sim wa} can co-express location (\ref{Guerrero3a}) and existence (\ref{Guerrero3b}).

\newpage

\ea  \label{Guerrero3}
\langinfo{Lezgian}{}{\cite[317--318]{Haspelmath1993Lezgian}}
\ea \label{Guerrero3a}
\gll Req’-e-\textbf{wa}-j daǧwi. \\
road-\textsc{iness}-be.in-\textsc{ptcp} mountaineer \\
\glt ‘A mountaineer (who is) on the road.’ \\

\ex\label{Guerrero3b}
\gll Tükwend-a	gzaf mal \textbf{awa}. \\
store-\textsc{iness} many good  be.in \\
\glt ‘There are many goods in the store.’ \\
\z
\z

Other languages may use other means to express figure-ground relationships, including postural, positional, and dispositional locative verbs (\cite{Newman2002}; \cite{Levinson2003}; \cite{Grinevald2006}; \cite{AmekaLevinson2007}). The central meaning of postural verbs such as ‘sit’, ‘lie’, and ‘stand’ is to denote the actual posture of humans and other animates that may change their posture; in some languages these verbs may also be used to describe the location of an inanimate object, in which case the selection of one verb over the other may be determined by the geometrical features of the object, e.g., apples and balls are conceived as ‘sit’, blankets and knives as ‘lie’, and trees and bottles as ‘stand’. In addition to ‘sit’, ‘lie’, and ‘stand’, positional verbs encode specific spatial configurations like ‘hang’, ‘be tied’, ‘be inserted’, ‘be stacked’, ‘lean’, ‘be upside down’, ‘be piled up’, and so on.\footnote{In the literature, the term ``positional'' has been used in different ways: as a general term for locative verbs (vs. no verb), as a term for postural verbs, as a term for dispositional verbs, and as a different verb type along with postural and dispositional verbs. In this chapter I use it in the last sense.} Lastly, dispositional verbs include postural and positional verbs, as well as a large set of verbs that denote spatial features like orientation, direction, support, suspension, and blockage of motion, along with configurations of parts of the figure with respect to parts of the ground.

Mesoamerican languages spoken in southern Mexico and Central America are well known in the literature due to their variation regarding locative predication. For instance, Chiapas Zoque typically uses the existential verb \textit{’it} (\ref{Guerrero4a}), though postural (\ref{Guerrero4b}) and dispositional verbs (\ref{Guerrero4c}) can also be used in some contexts (\cite{DeLaCruz2016}; \cite{McDermott2017}).
The choice of postural and dispositional verbs is an alternative (pragmatically marked) construction, and these verbs provide additional information regarding the disposition of the figure with respect to the ground (\cite[52]{McDermott2017}).

\newpage
\ea \label{Guerrero4}
\langinfo{Chiapas Zoque}{Mixe-Zoquean, [chia1265]}{\cite[49, 82, 130]{McDermott2017}}

\ea \label{Guerrero4a}
\gll Nas=’omo Ø-\textbf{’it}-u pelota. \\
ground=\textsc{loc}3 3\textsc{b}-exist-\textsc{cmpl} ball \\
\glt ‘There is a ball on the ground.’ \\

\ex \label{Guerrero4b}
\gll Jǝti te’ pǝnn-tsǝnki tsima=kǝ’mǝn Ø-\textbf{ten}-u. \\
now	\textsc{det} man-figure	glass=\textsc{loc}5	3\textsc{b}-standing-\textsc{cmpl} \\
\glt ‘Now the toy man is standing by the glass.’ \\

\ex \label{Guerrero4c}
\gll Te’ pǝn Ø-\textbf{ten-ki’m}-u te’ tǝk-kpak=’omo.  \\
\textsc{det} man 3\textsc{b}-standing-ascend-\textsc{cmpl} \textsc{det}	house-head=\textsc{loc}3 \\
\glt ‘The man is standing upright on the roof of the house.’ \\
\z
\z

Zapotec languages (Oto-Manguean) have around 20 positional verbs to express figure-ground relationships (\cite{LillehaugenSonnenschein2012}; \cite{Lopez2015}; \cite{ForemanLillehaugen2017}; \cite{AlonsoOrtiz2020}). The locative descriptions in (\ref{Guerrero5a}--\ref{Guerrero5b}) show the Yalálag Zapotec postural verbs \textit{zú} ‘stand’ and \textit{kúà} ‘lie’. The meaning of some of these verbs may be extended to denote existential predication, especially when the ground is implicit (\cite{AlonsoOrtiz2020}). The clause in (\ref{Guerrero5c}) takes \textit{dè} ‘lie’, but it is interpreted as ‘exist’.

\ea \label{Guerrero5}
\langinfo{Yalálag Zapotec}{Oto-Manguean, [yala1267]}{\cite[93, 91, 85]{AlonsoOrtiz2020}}

\ea \label{Guerrero5a}
\gll \textbf{Zú-llà}=n lù yù=nhà. \\
standing-\textsc{adv}=3\textsc{inan} \textsc{rn} ground=\textsc{dem.dist}\\
\glt ‘[The bottle] is standing on the ground.’ \\

\ex \label{Guerrero5b}
\gll Sà-dáò-kà	\textbf{n-kúà}-n	lhàò	xhbà=nhà. \\
bean-\textsc{dim-pl}	\textsc{sta}-lying-3\textsc{inan} \textsc{rn} table=\textsc{dem.dist} \\
\glt ‘The beans are lying on the table.’ \\

\ex \label{Guerrero5c}
\gll Nbàlhàs-ùlhè	\textbf{dè} 	yâj 	nhìsèn 	kà-nhà.  \\
pretty-\textsc{aum}	\textsc{cop.exist} rain	water	\textsc{adv.t-dem.dist} \\
\glt ‘There was abundant water at that time.’  \\
\z
\z

Mayan languages are characterized by their hundreds of dispositional verbs, but they also use existential verbs (\cite{Brown1994}; \cite{LevinsonHaviland1994}; \cite{Grinevald2006}; \cite{BohnemeyerBrown2007}; \cite{BelloroEtAl2008}; \cite{SáintzGómez2010}).  In fact, some languages in this family differ in their preferences for using existential verbs over dispositional verbs. For instance, it is said that Tzeltal speakers use dispositional verbs (\ref{Guerrero6a}) more often than existential verbs (\ref{Guerrero6b}), while Yucatec speakers prefer the existential verbs (\ref{Guerrero6d}) over dispositional verbs (\ref{Guerrero6c}) (\cite[1137]{BohnemeyerBrown2007}).

\ea \label{Guerrero6}

\ea \label{Guerrero6a}
\langinfo{Tzeltal}{Maya, [tzel1254]}{\cite[1115]{BohnemeyerBrown2007}}\\
\gll \textbf{Bech’-el-Ø} ta ch’ujt ton. \\
wind.around-\textsc{disp.s-b}.3	\textsc{prep} belly	stone\\
\glt ‘It [rope] is wound around the belly of the stone.’ \\

\ex \label{Guerrero6b}
\langinfo{Tzeltal}{}{\cite[1122]{BohnemeyerBrown2007}}\\
\gll Te	timbre,	\textbf{ay-Ø} ta s-xujk ala  karta. \\
\textsc{det} stamp	exist-\textsc{b}.3 \textsc{prep}	\textsc{a}.3-edge \textsc{dim} letter \\
\glt ‘The stamp, it is at the edge of the letter.’ \\

\ex \label{Guerrero6c}
\langinfo{Yucatec}{Maya, [yuca1254]}{\cite[1116]{BohnemeyerBrown2007}}\\
\gll Ti’ \textbf{bak’-akal-Ø} te tunich-o’.   \\
there wind.around-\textsc{disp.s-b}.3 \textsc{prep}	stone-\textsc{d}2 \\
\glt ‘It [rope] is wound around the belly of the stone.’  \\

\ex \label{Guerrero6d}
\langinfo{Yucatec}{}{\cite[1123]{BohnemeyerBrown2007}}\\
\gll Le	lùuch-o’, ti’ \textbf{yàan-Ø} y-óok’ol le mèesa-o’.   \\
\textsc{det}	cup-\textsc{d}2	there	exist-\textsc{b}.3	\textsc{a}.3-on	\textsc{det}	table-\textsc{d}2 \\
\glt ‘The cup, it’s there on the table.’  \\

\z
\z

Several typologies have been proposed to capture cross-linguistic variation regarding locative predication. First, Stassen (\citeyear{Stassen1997}) proposed a typology with three basic types (\tabref{tab:GuerreroTable1}): the verbal type (the locative phrase is treated as a verb), the nominal type (with a non-locative copula), and the locative type (with a locative copula or verb). In this proposal, English is considered a nominal type, whereas Spanish, Lezgian, and Mixe-Zoquean, Zapotec, and Mayan languages are grouped into the locative type.

\begin{table}
    \begin{tabularx}{\textwidth}{llQ}
        \lsptoprule
        \textsc{Type} & \textsc{Description} & \textsc{Languages}\\
        \midrule
        Verbal & Zero copula (with PNG marking)  & \\
        Nominal & Non-locative copula (covert or overt) & English\\
        Locative & Locative copula or verb & Spanish, Lezgian, Mixe-Zoquean, Zapotec, Mayan\\
        \lspbottomrule
    \end{tabularx}
    \caption{Types of contrastive locative predicates (\cite{Stassen1997})}
    \label{tab:GuerreroTable1}
\end{table}

\citet{AmekaLevinson2007} proposed a typology with four basic types based on the number and semantics of locative predicates (\tabref{tab:GuerreroTable2}), reaching from no verb (Type 0) or a single verb (Type I) to a small contrastive set of verbs (Type II) or a large set of dispositional verbs (Type III). Note that the locative type in Stassen’s proposal correlates to four types in Ameka and Levinson’s typology: general copulas (Type Ia), locative/existential copulas (Type Ib), postural verbs (Type II), and dispositional verbs (Type III). Accordingly, English, Spanish and Lezgian are single-set languages, while Mixe-Zoquean, Mayan and Zapotec languages are multiverb-set languages since they have more than 9 locative verbs.


\begin{table}
    \begin{tabularx}{\textwidth}{lQp{2.65cm}}
    \lsptoprule
    \textsc{Type} & \textsc{Description} & \textsc{Languages}\\
    \midrule
    Type 0 & No verb & \\
    \midrule
    Type I & A single locative verb & \\
    Ia & Multifunctional copula & English \\
    Ib & Locative/existential verb & \mbox{Spanish, Lezgian}\\
    \midrule
    Type II & A small contrastive set of locative verbs (3--7 verbs) & \\
    IIa & Postural verbs & \\
    IIb & Ground space indicating verbs & \\
    \midrule
    Type III & Multiverb positional verbs; a large set of dispositional verbs (9--100) & Mixe-Zoquean, \mbox{Mayan, Zapotec}\\
    \lspbottomrule
    \end{tabularx}
    \caption{Types of contrastive locative predicates (\cite[863--864]{AmekaLevinson2007})}
    \label{tab:GuerreroTable2}
\end{table}

As for Amerindian languages, Grinevald (\citeyear{Grinevald2006}) introduced two changes to Ameka and Levinson’s proposal (\tabref{tab:GuerreroTable3}). First, she characterized Type 0 languages as those with no verbal element and languages that use non-locative copula verbs. Second, she proposed a continuum of locative predication with postural verbs (Type II) and dispositional verbs (Type IV) at the two ends of the continuum, and positional verbs (Type III) as the intermediate point. This classification distinguishes Mixe-Zoquean and Zapotec languages (Type III) from Mayan languages (Type IV).

\begin{table}
\begin{tabularx}{\textwidth}{llQ}    \lsptoprule
    \textsc{Type} & \textsc{Description} & \textsc{Languages}\\
    \midrule
    Type 0 & No locative information (zero or existential copula) & English \\
    Type I & One locative verb (distinct from existential copula) & Spanish, Lezgian \\
    Type II & Prototypical postural verbs (3--4 verbs) & \\
    Type III & Positional verbs & Mixe-Zoquean,\newline Zapotec \\
    Type IV & Dispositional verbs (several hundred verbs) & Mayan\\
    \lspbottomrule
    \end{tabularx}
    \caption{Types of contrastive locative predicates (\cite[4]{Grinevald2006})}
    \label{tab:GuerreroTable3}
\end{table}

My goal in this chapter is to contribute to cross-linguistic studies dealing with locative descriptions with further data from Southern Uto-Aztecan languages. Unlike Mesoamerican languages, members of the Southern Uto-Aztecan family typically use postural verbs (‘sit’, ‘lie’, ‘stand’, ‘hang’), but they may also use general locative (‘be located’) and/or existential verbs (‘exist’). Therefore, my aim in this study is threefold. First, I aim to identify the inventory of locative/existential verbs in each Southern Uto-Aztecan language. Second, I attempt to lay out the semantic properties that each of these locative/existential verbs may describe in one of these languages (Yaqui). Finally, I seek to establish the typological classification of locative predications of this family language.

The rest of the chapter is organized as follows: after a brief introduction to the Southern Uto-Aztecan language family, and the methodology used for data collection (§\ref{UtoAztecanLanguageCorpus}), the inventory of locative/existential verbs in each of these languages is established (§\ref{LocativeDescriptionsInSouthernUto}). I demonstrate that the number of verbs varies from one to eight depending on the prompt used to collect the data. Based on the verbal inventory, I propose that all Southern Uto-Aztecan languages fit into the Locative type in Stassen’s typology, and most of them can be classified as small-set languages as per Ameka and Levinson’s proposal, except for Mayo and two Nahuatl languages which can be considered single-set languages. 

The next section (§\ref{LocativeDescriptionsYaqui}) is devoted to Yaqui, the language I am most familiar with. Yaqui shows a rich inventory of locative predicates: two existential verbs, two general locative verbs, four postural verbs, as well as a dozen positional verbs. After examining the semantic motivations behind the selection of these verbs (e.g., the animacy, number, and shape of the figure, properties of the ground, and spatial configurations), I propose that Yaqui does not belong to a small-set language type (as claimed in \cite{Guerrero2012, Guerrero2017}), but it is better characterized as an intermediate-set language in terms of Grinevald’s typology. Finally, with respect to the so-called “existential” verbs in Yaqui (§\ref{ExistentialConstructionsInYaqui}), I provide syntactic and pragmatic evidence that there is no dedicated existential construction (e.g., \textit{a book exists}) or dedicated locative-existential construction (e.g., \textit{there is a book (on the table)}). There are only locative constructions. 

\section{Uto-Aztecan languages and corpus} \label{UtoAztecanLanguageCorpus}

\begin{figure}[b]
\includegraphics[width=\textwidth]{figures/guerrero-Map1.png}
\caption{Southern Uto-Aztecan languages under study}
\label{GuerreroFig1}
\end{figure}


The Uto-Aztecan family is divided into Northern and Southern branches. The Northern branch is spoken in the United States, while the Southern branch is mainly spoken in Mexico and Central America. The analysis focuses on languages from the Southern branch. Although the internal organization of this branch is still under discussion (\cite{Miller1984}; \cite{CortinoValinas1989}; \cite{Dakin1994}; \cite{Campbell1997}; \cite{Moctezuma2012}) five major groups are generally recognized:

i) Tepiman (Southern Tepehuan, Northern Tepehuan, Mountain Pima, Lower Pima); ii) Taracahitan (Tarahumara varieties, Guarijio, Yaqui, Mayo), iii) Tubar (extinct), iv) Corachol (Cora, Huichol), and v) several Nahuatl varieties (including Classic Nahuatl and Pipil). 	

The analysis of locative descriptions in the present study includes data from all (living) Uto-Aztecan languages spoken in Mexico (cf. \figref{GuerreroFig1}): Yaqui [yaqu1251], Mayo [mayo1264], Guarijio [huar1255], Tarahumara [tara1321], Lower Pima [pima1248], Northern Tepehuan [nort2959], Southern Tepehuan [sout2975], Cora [cora1260], and Huichol [huic1243]. It also includes data from three Nahuatl languages of the Aztecan branch: Hidalgo [east2538], Mexicanero [dura1246], and Huastec Nahuatl [huas1257].\footnote{Three languages are excluded from the present analysis: Classic Nahuatl [clas1250] and Tubar [tuba1279], since both are extinct, and Mountain Pima, which is mostly spoken in Arizona. There is little documentation on the Sonoran variant of Mountain Pima [sono1245].} Geographically, Corachol and Nahuatl languages border on the Mesoamerican area.


The data discussed in the following sections come from several sources (\tabref{tab:GuerreroTable4}). First, I examine data exemplified in previous grammatical studies. Second, I analyze data from the \textit{Archivo de Lenguas Indígenas de México} (Archive of Mexican Indigenous Languages). Third, following the methodology used in several cross-linguistic studies, I include data collected with visual stimuli such as the “Topological Relations Picture Series” (BowPed, \cite{BowermanPeterson1992}), the “Picture Series for Positional Verbs” (PSPV, \cite{AmekaEtAl1999}), and “Locative Descriptions of Animate and Inanimate Entities” (DELOCA, \cite{Guerrero2022}). And fourth, the analysis of Yaqui also includes natural data from oral texts and from fieldwork conducted with Yaqui speakers.\footnote{If an example was elicited with a visual stimulus, the name of the picture series, the number of the picture, and the speaker identification are indicated after the free translation; if the example comes from previous publications or oral texts, the source is then indicated; if the example does not show the source, it means it was collected in the field. In most of the cases, I followed the transcription and glossing conventions given in the original source, including the translation of the verbal element.}


\begin{table}[b]
    \begin{tabularx}{\columnwidth}{lQlQQQ}
\lsptoprule
       {Language} & {Previous works} & {Archivo} & {Visual stimuli} & {Oral texts} & {First-hand data} \\
\midrule
        \textsc{Taracahita}  &  &  &  &  & \\
        Yaqui & \langscicheckmark & \langscicheckmark & \langscicheckmark & \langscicheckmark & \langscicheckmark \\
        Mayo & \langscicheckmark & \langscicheckmark &  &  & \\
        Guarijio & \langscicheckmark & \langscicheckmark &  &  & \\
        Tarahumara & \langscicheckmark &  & \langscicheckmark &  & \\
        \midrule
        \textsc{Tepiman} &  &  &  &  & \\
        Lower Pima & \langscicheckmark & \langscicheckmark & \langscicheckmark &  & \\
        Northern Tepehuan & \langscicheckmark &  &  &  & \\
        Southern Tepehuan & \langscicheckmark & \langscicheckmark & \langscicheckmark &  & \\
        \midrule
        \textsc{Corachol} &  &  &  &  & \\
        Huichol & \langscicheckmark & \langscicheckmark & \langscicheckmark &  & \langscicheckmark \\
        Cora & \langscicheckmark &  & \langscicheckmark &  & \\
        \midrule
        \textsc{Nahuatl} &  &  &  &  & \\
        Hidalgo &  & \langscicheckmark &  &  & \\
        Mexicanero &  & \langscicheckmark &  &  & \\
        Huasteca & \langscicheckmark &  & \langscicheckmark &  & \\
\lspbottomrule
    \end{tabularx}
    \caption{Southern Uto-Aztecan languages under analysis}
    \label{tab:GuerreroTable4}
\end{table}

\largerpage
With their own particularities regarding word order and argument coding, speakers of Southern Uto-Aztecan languages use the same construction when asserting the location of a figure in reference to the ground regardless of whether or not the figure is selected as the ‘perspectival center’ (\cite{Creissels2019}). The Huichol examples in (\ref{Guerrero7a}--\ref{Guerrero7b}) illustrate the basic properties of a locative construction in this family: the figure is coded as the intransitive nominative subject, the ground can be realized as an oblique (postpositionally marked) phrase, an adverb, or both, and there is a verbal element linking both entities, either a postural verb (\ref{Guerrero7a}) or an existential verb (\ref{Guerrero7b}). The ground of existential clauses can be omitted in Huichol (\ref{Guerrero7c}), while its omission is rare with postural verbs.


\ea \label{Guerrero7}
\langinfo{Huichol}{}{BowPed\_16\_DA (a), \cite[113]{RamosBierge2017} (b--c)}

\ea \label{Guerrero7a}
\gll Peruta ’ipari-ta \textbf{pa-a-ye-ka}. \\
ball chair-\textsc{loc}	\textsc{as-loc-loc}-sitting.\textsc{vl.sg}\\
\glt 	‘The ball is sitting below the chair.’ \\

\ex \label{Guerrero7b}
\gll Ta=kie	hikɨ wiesta \textbf{m-au-xuawe}	 ta=teimauneiya. \\
\textsc{1pl.poss}=house	now	ceremony \textsc{as}-all-exist \textsc{1pl-poss}=ceremony.drum  \\
\glt ‘There is a ritual at our home, the ritual of the drum…’ \\

\ex \label{Guerrero7c}
\gll  \textbf{Me-te-xuawe }	kawaya-tsixi 	puritu-tsixi 	waka-tsixi.  \\
\textsc{3pl.sbj-dist}-exist 	horse-\textsc{pl} 	donkey-\textsc{pl} 	cow-\textsc{pl} \\
\glt ‘There are horses, donkeys, cows.’  \\
\z
\z

Clauses with postural verbs in Northern Tepehuan can be translated as coding location (\ref{Guerrero8a}) or existence (\ref{Guerrero8b}). In this language, the notion of existence can be also expressed by the \textit{t\textsuperscript{y}iípu/oid\textsuperscript{y}ága} ‘exist (\textsc{sg/pl})’ verb (\ref{Guerrero8c}).

\ea \label{Guerrero8}
\langinfo{Northern Tepehuan}{}{\cite[275, 307, 281--282]{Bascom1982}}

\ea \label{Guerrero8a}
\gll Tán \textbf{dáha} ááni giñ-kii-í-ri. \\
here sit.\textsc{sg} \textsc{1sg.nom} \textsc{1sg.poss}-house-\textsc{loc}-in\\
\glt 	‘Here I am in my house.’ \\

\ex \label{Guerrero8b}
\gll Gi-vanámo-aba \textbf{dáha} imó nakásirai. \\
\textsc{2sg.poss}-hat-on sit.\textsc{sg} one scorpion   \\
\glt ‘There is a scorpion on your hat.’ \\

\ex \label{Guerrero8c}
\gll Múí-dyu kií-ki. \\
many-quantity \textsc{red}-house \\
\glt ‘There are a lot of houses.’  \\
\z
\z

In most languages of the family, locative/existential verbs are formally different from nominal copula verbs.\footnote{\citet{Stassen2013}  studied the possible relationships between the encoding of nominal predication (e.g., \textit{John is a tailor}) and locative predication (e.g., \textit{John is in Paris}). He found that some languages use the same verb type for both predications (e.g., English) whereas other languages use different verb types (e.g., Spanish). The latter is the most common strategy in his sample, and it is also observed in the Southern Uto-Aztecan family.}  For instance, in Huichol there is a zero copula (\ref{Guerrero9a}) or a copulative-verbalizer element (\ref{Guerrero9b}). The semantically empty copulative element occurs in non-present-tense clauses, serving as a support element to carry tense-aspect markers. In Northern Tepehuan, the zero copula also occurs in nominal clauses in the present tense (\ref{Guerrero9c}), but \textit{ir} ‘be’ is used in the past tense (\ref{Guerrero9d}). The existential verbs \textit{xuawe} and \textit{t\textsuperscript{y}iípu/oid\textsuperscript{y}ága} are not used in nominal clauses. As I show later in this chapter, the use of zero copula is rare in locative descriptions.

\ea \label{Guerrero9}


\ea \label{Guerrero9a}
\langinfo{Huichol}{}{\cite[339]{RamosBierge2017}} \\
\gll Aame te-pɨ-wixari-tari. \\
\textsc{1pl} \textsc{1pl.sbj-as}-huichol-\textsc{pl}\\
\glt ‘We are Huicholes.’ \\

\ex \label{Guerrero9b}
\langinfo{Huichol}{}{\cite[111]{Gomez1999}} \\
\gll Wani ’aiki tupiri-ti-kai. \\
John last\_year topil-\textsc{vblz-ipfv} \\
\glt ‘John was topil [sheriff] last year.’ \\

\ex \label{Guerrero9c}
\langinfo{Northern Tepehuan}{}{\cite[281]{Bascom1982}} \\
\gll Kiíli ááni.   \\
man	\textsc{1sg.nom} \\
\glt ‘I am a man.’  \\

\ex \label{Guerrero9d}
\langinfo{Northern Tepehuan}{}{\cite[327]{Bascom1982}} \\
\gll I-Sireñio \textbf{ir} kápigi-ka-tadai.   \\
the-Sirenio	be governor-\textsc{stat-pastc} \\
\glt ‘Sirenio was governor.’  \\

\z
\z

In Southern Uto-Aztecan languages, the word order of major constituents may vary, as can be seen in (\ref{Guerrero7b}--\ref{Guerrero7c}) and (\ref{Guerrero8a}--\ref{Guerrero8b}). Moreover, nominal phrases may or may not have a determiner. Accordingly, the notions of “definiteness” and “focal/non-focal” elements usually used to distinguish location and existential meanings might be captured by means other than word order and definiteness marking.

\section{Locative descriptions in Southern Uto-Aztecan languages} \label{LocativeDescriptionsInSouthernUto}

In this section, the number of different verbs asserting the location of a figure in reference to the ground is determined in each Southern Uto-Aztecan language based on data from the \textit{Archivo de Lenguas Indígenas de México} (hereafter, \textit{Archivo}) (§\ref{DataFromArchivo}) and from visual stimuli (§\ref{DataVisualStimuli}). I show that the inventory varies from one to eight locative/existential verbs.  When there is more than one verb, the preference for selecting one verb type over the others may depend on the semantic features of the figure, but also on the methodology used for collecting the data.

\subsection{Data from \textit{Archivo}} \label{DataFromArchivo}
The \textit{Archivo} is a collection of volumes describing the basic linguistic structures of Mexican indigenous languages. This collection includes studies on Yaqui (\cite{Estrada2009}), Mayo (\cite{Freeze1989}), Guarijio (\cite{Miller1993}), Lower Pima (\cite{Estrada1998}), Southern Tepehuan (\cite{GarciaReyes2023}), Huichol (\cite{Gomez1999}), Acaxochitlán, Hidalgo Nahuatl (\cite{Lastra1980}), and Mexicanero (\cite{Canger2001}). The data for each language was obtained via a syntactic questionnaire aimed at eliciting intransitive and transitive sentences and simple and complex constructions. Around 40 of these clauses express the location or existence of an entity at a certain place. There are a few clause pairs where the first sentence asks for the location of an entity (e.g., \textit{Where is your father?}) and the second (and third) provide a locative description as an answer (e.g., \textit{My father is at home, My father is with Manuel}). Other clauses are plain locatives (e.g., \textit{The bird is in the tree}) or existentials (e.g., \textit{There are beans in the pot, There is no water here}). A few negative existential clauses are also included (e.g., \textit{There is no water in the pot}). In this corpus, there are human, animate (e.g., \textit{dog, bird, fly}), and inanimate entities (e.g., \textit{pot, machete, church, market, corral, clothes, beans, tortillas, water}) serving as the figure, as well as different entities (e.g., \textit{human, house, wall, corner, milpa, monte, tree, tree trunk, basket, pot, fire}) and deictic points (e.g., \textit{here, there, far away}) denoting the ground.
The questionnaire was conducted in Spanish, and the speakers translated the Spanish clause into their own language. 

Two different prompts were used to collect this data. In the first one, the speakers were asked to translate from Spanish \textit{¿Dónde está X?} (‘Where is X?’) and \textit{X está en Y} (‘X is at Y’). Table (\ref{tab:GuerreroTable5}) presents the verbs used by the speakers in this task. In the second prompt, speakers were asked to translate \textit{Hay X en Y} (‘There is X at Y’) and \textit{No hay X en Y} (‘There is no X at Y’). Table (\ref{tab:GuerreroTable6}) lists the verbs used for this task. In both tables, the last row includes the copulative linking element found in nominal clauses, to contrast them with locative/existential verbs.

From Table (\ref{tab:GuerreroTable5}) we can see that the Taracahita, Tepiman and Corachol language groups typically use postural verbs when translating a Spanish locative clause, while Mayo, Mexicanero and Hidalgo Nahuatl systematically use a general locative verb. A general locative verb is a verb meaning ‘be located’ and does not provide any information about the figure, the ground, or the spatial configuration between them. Notice that existential verbs are used sporadically when translating a Spanish locative clause. In contrast, Table (\ref{tab:GuerreroTable6}) reveals that existential verbs are the default choice when translating a Spanish existential clause.

\begin{sidewaystable}
\footnotesize
\caption{\label{tab:GuerreroTable5}Copula and locative/existential verbs in Southern Uto-Aztecan languages (Translation of ‘X is at Y’)}
\begin{tabularx}{\textwidth}{lQQ@{}QQQ@{}lQ@{}Q}
\lsptoprule
\textsc{figure-ground} & Yaqui & Mayo & Guarijio & Lower Pima & Southern Tepehuan & Huichol & Hidalgo \mbox{Nahuatl} & Mexicanero \\
\midrule
human-ANY & \textit{aane} ‘exist’ \textit{katek} ‘sit’ & \textit{aane}  ‘exist’ & \textit{kahti}  ‘sit’ & \textit{daha}  ‘sit’ & \textit{əirɨ}  ‘be\textsubscript{around}’ \textit{daa}  ‘sit’ & \textit{ka}  ‘sit’ & \textit{ka}  ‘be\textsubscript{\textsc{loc}}’ & \textit{onka}  ‘be\textsubscript{\textsc{loc}}’ \\

dog-house & \textit{bo’oka}  ‘lie’ \mbox{\textit{weyek}  ‘stand’} & \textit{aane}  ‘exist’ & \textit{chuku}  ‘kneel’ & \textit{daha}  ‘sit’ & \textit{əirɨ}  ‘be\textsubscript{around}’ & \textit{we}  ‘stand’ & \textit{ka}  ‘be\textsubscript{\textsc{loc}}’ & \textit{onka}  ‘be\textsubscript{\textsc{loc}}’ \\

bird-tree & \textit{katek}  ‘sit’ & \textit{katek}  ‘sit’ & \textit{weri}  ‘stand’ & \mbox{\textit{kiika}  ‘stand’} & \textit{daa}  ‘sit’ & \textit{ka}  ‘sit’ & \textit{ka}  ‘be\textsubscript{\textsc{loc}}’ & \textit{onka}  ‘be\textsubscript{\textsc{loc}}’\\

fly-tree trunk & \textit{katek}  ‘sit’ & \textit{katek}  ‘sit’ & \mbox{\textit{chuku}  ‘kneel’} & \mbox{\textit{kiika}  ‘stand’} & \textit{daa}  ‘sit’ & \textit{ka}  ‘sit’ & \textit{sewi}  ‘rest’ & \textit{onka}  ‘be\textsubscript{\textsc{loc}}’ \\

pot-ground & \mbox{\textit{manek}  ‘be\textsubscript{\textsc{loc}}’} & \textit{o:rek}  ‘be\textsubscript{\textsc{loc}}’ & \textit{weri}  ‘stand’ & \textit{daha}  ‘sit’ & \textit{daa}  ‘sit’ & \textit{ka}  ‘sit’ & \textit{ka}  ‘be\textsubscript{\textsc{loc}}’ & \textit{onka}  ‘be\textsubscript{\textsc{loc}}’ \\

machete-corner & \textit{weyek}  ‘stand’ \textit{bo‘oka}  ‘lie’ & \textit{o:rek}  ‘be\textsubscript{\textsc{loc}}’ & \textit{poI}  ‘lie’ & \mbox{\textit{kiika}  ‘stand’} \textit{kat}  ‘lie’ & \textit{kaat}  ‘lie’ & \textit{ka}  ‘sit’ & \textit{ka} ‘be\textsubscript{\textsc{loc}}’ & \textit{onka}  ‘be\textsubscript{\textsc{loc}}’\\

church-house & \textit{katek}  ‘sit’ & \textit{o:rek}  ‘be\textsubscript{\textsc{loc}}’ & \textit{weri}  ‘stand’ & \mbox{\textit{kiika}  ‘stand’} & \textit{kɨɨk}  ‘stand’ & \textit{we}  ‘stand’ & \textit{ka}  ‘be\textsubscript{\textsc{loc}}’ & \textit{onka}  ‘be\textsubscript{\textsc{loc}}’ \\

market-there & \textit{katek}  ‘sit’ & \textit{o:rek}  ‘be\textsubscript{\textsc{loc}}’ & \textit{weri}  ‘stand’ & \mbox{\textit{kiika}  ‘stand’} & \textit{kɨɨk}  ‘stand’ & \textit{we}  ‘stand’ & \textit{ka}  ‘be\textsubscript{\textsc{loc}}’ & \textit{onka}  ‘be\textsubscript{\textsc{loc}}’\\

fence-house & \textit{katek} ‘sit’ & \textit{o:rek}  ‘be\textsubscript{\textsc{loc}}’ & \textit{weri} ‘stand’ & \mbox{\textit{kiika} ‘stand’} & \textit{kɨɨk} ‘stand’ & \textit{ma}  ‘be\textsubscript{\textsc{loc}}’ & \textit{ka}  ‘be\textsubscript{\textsc{loc}}’ & \textit{onka}  ‘be\textsubscript{\textsc{loc}}’ \\

clothes-wardrobe & \mbox{\textit{aayuk}  ‘exist’} & \textit{o:rek}  ‘be\textsubscript{\textsc{loc}}’ & \textit{mani}  ‘be\textsubscript{\textsc{loc}}’ & \mbox{\textit{vivuta}  ‘be\textsubscript{\textsc{loc}}’} & \textit{Ø} & \textit{ka}  ‘sit’ & \textit{ka}  ‘be\textsubscript{\textsc{loc}}’ & \textit{onka}  ‘be\textsubscript{\textsc{loc}}’ \\

beans-pot & \mbox{\textit{manek} ‘be\textsubscript{\textsc{loc}}’} & \textit{o:rek} ‘be\textsubscript{\textsc{loc}}’ & \textit{mani}  ‘be\textsubscript{\textsc{loc}}’ & \mbox{\textit{vivuta}  ‘be\textsubscript{\textsc{loc}}’} & \textit{jim}  ‘go’ & \textit{ma}  ‘be\textsubscript{\textsc{loc}}’ & \textit{ka}  ‘be\textsubscript{\textsc{loc}}’ & \textit{onka}  ‘be\textsubscript{\textsc{loc}}’ \\

town-there & \textit{ta:wa} ‘stay’ & \textit{o:rek}  ‘be\textsubscript{\textsc{loc}}’ \textit{ta:wa} ‘stay’ & \textit{a:ka} ‘stand.\textsc{pl}’ & \textit{daha} ‘sit’ & \textit{jir} ‘sit’ & \textit{Ø} & \textit{Ø} & \textit{Ø} \\

\textsc{nominal copula} & \textit{Ø, -tu} & \textit{Ø, -tu} & \textit{hu, ine, gari} & \textit{Ø} & \textit{jir, jix} & \textit{Ø, -tɨ(a), -ya} & \textit{Ø, wawi, katka, taw} & \textit{Ø, ye, čiwa katka}\\
\lspbottomrule
\end{tabularx}
\end{sidewaystable}

\begin{sidewaystable}
\footnotesize
\caption{\label{tab:GuerreroTable6} Copula and locative/existential verbs in Southern Uto-Aztecan languages (Translation of ‘There is X at Y’ and ‘There is no X at Y’)}
\begin{tabularx}{\textwidth}{lQQQQQQp{12mm}p{15mm}@{}}
\lsptoprule
\textsc{figure-ground} & Yaqui & Mayo & Guarijio & Lower Pima & Southern Tepehuan & Huichol & Hidalgo  Nahuatl & Mexicanero \\
\midrule
human-house & \textit{aane} ‘exist’ \textit{kaita} ‘not.exist’ & \textit{aane}  ‘exist’ \textit{ka:habe} ‘not.exist’ & \mbox{\textit{mociwi} ‘sit.\textsc{pl}’} \textit{kaita} ‘not.exist’ & \textit{amik} ‘exist’ \mbox{\textit{amit} ‘not.exist’} & \textit{ja’ich} ‘exist’ & \textit{xuawe} ‘exist’ & \mbox{\textit{ka}  ‘be\textsubscript{\textsc{loc}}'} & \textit{onka}  ‘be\textsubscript{\textsc{loc}}' \\
\midrule
deer-monte & \textit{aane}  ‘exist’ \textit{kaita} ‘not.exist’ & \textit{aane} ‘exist’ \textit{ka:habe} ‘not.exist’ & \textit{kaita} ‘not.exist’ & \textit{amik} ‘exist’ \mbox{\textit{amit} ‘not.exist’} & \textit{ja’ich} ‘exist’ & \textit{xuawe} ‘exist’ & \textit{kate} ‘be\textsc{\textsubscript{loc}.pl}’ & \textit{onka}  ‘be\textsubscript{\textsc{loc}}'\\

trees-there & \textit{jaabwek} ‘stand.\textsc{pl}’ & \textit{aika} ‘exist’ \mbox{\textit{ka:ita} ‘not.exist’} & \textit{ahawi} ‘stand.\textsc{pl}’ & \textit{amik} ‘exist’ \mbox{\textit{amit} ‘not.exist’} & \textit{ja’ich} ‘exist’ & \textit{xuawe} ‘exist’ & \mbox{\textit{ka}  ‘be\textsubscript{\textsc{loc}}'} & \textit{kate} ‘be\textsc{\textsubscript{loc}.pl}'\\

market-town & \textit{jippue} ‘have’ & \textit{aika} ‘exist’ \mbox{\textit{ka:ita} ‘not.exist’} & \textit{weri} ‘stand’ & \textit{amik} ‘exist’ \mbox{\textit{amit} ‘not.exist’} & \textit{ja’ich} ‘exist’ & \textit{xuawe} ‘exist’ & \textit{ka} ‘be\textsc{\textsubscript{loc}}' & \textit{onka}  ‘be\textsc{\textsubscript{loc}}'\\

sickness-town & \mbox{\textit{aayuk} ‘exist’} & \textit{aika} ‘exist’ \mbox{\textit{ka:ita} ‘not.exist’} & \textit{cee} ‘have’ & \textit{amik} ‘exist’ \mbox{\textit{amit} ‘not.exist’} & \textit{ja’ich} ‘exist’ & \textit{’axiya} ‘attack’ & \mbox{\textit{ka}  ‘be\textsubscript{\textsc{loc}}'} & \textit{onka}  ‘be\textsubscript{\textsc{loc}}'\\

4 tortillas-basket & \mbox{\textit{aayuk} ‘exist’} & \textit{aika} ‘exist’ \mbox{\textit{ka:ita} ‘not.exist’} & \textit{mani} ‘be\textsubscript{\textsc{loc}}' & \textit{amik} ‘exist’ \mbox{\textit{amit} ‘not.exist’} & \textit{ja’ich} ‘exist’ & \mbox{\textit{mane} ‘be\textsc{\textsubscript{loc}.pl}'} \textit{ma} ‘be\textsubscript{\textsc{loc}}' & \textit{ka}  ‘be\textsc{\textsubscript{loc}}' & \textit{onka}  ‘be\textsc{\textsubscript{loc}}' \\

1 tortilla-basket & \textit{katek}  ‘sit’ & \textit{aika} exist \mbox{\textit{ka:ita} ‘not.exist’} & \textit{kahti} ‘sit’ & \textit{amik} ‘exist’ \mbox{\textit{amit} ‘not.exist’} & \textit{ja’ich} ‘exist’ & \textit{mane} ‘be\textsc{\textsubscript{loc}.pl}' \textit{ma} ‘be\textsubscript{\textsc{loc}}' & \mbox{\textit{ka}  ‘be\textsubscript{\textsc{loc}}'} & \textit{onka}  ‘be\textsubscript{\textsc{loc}}' \\

beans-pot & \mbox{\textit{aayuk} ‘exist’} \textit{kaita} ‘not.exist’ & \textit{aika} ‘exist’ \textit{ka:ita} ‘not.exist’ & \textit{mani} ‘be\textsubscript{\textsc{loc}}' \textit{kaita} ‘not.exist’ & \textit{amik} ‘exist’ \mbox{\textit{amit} ‘not.exist’} & \textit{jim} ‘go \textit{ja’ich} ‘exist’ & \textit{mane} ‘be\textsc{\textsubscript{loc}.pl}' \textit{ma} ‘be\textsubscript{\textsc{loc}}' & \mbox{\textit{ka}  ‘be\textsubscript{\textsc{loc}}'} & \textit{onka}  ‘be\textsubscript{\textsc{loc}}' \\

water-pot & \mbox{\textit{aayuk} ‘exist’} \textit{kaita} ‘not.exist’ & \textit{aika} ‘exist’ \textit{ka:ita} ‘not.exist’ & \textit{mani} ‘be\textsubscript{\textsc{loc}}' \textit{kaita} ‘not.exist’ & \textit{amik} ‘exist’ \mbox{\textit{amit} ‘not.exist’} & \textit{jim} ‘go \textit{ja’ich} ‘exist’ & \mbox{\textit{mane} ‘be\textsc{\textsubscript{loc}.pl}'} \textit{xuawe} ‘exist’ \mbox{\textit{mawe} ‘not.exist’} & \mbox{\textit{ka}  ‘be\textsubscript{\textsc{loc}}'} & \textit{onka}  ‘be\textsubscript{\textsc{loc}}'\\

houses-town & \mbox{\textit{aayuk} ‘exist’} \mbox{\textit{kaita} ‘not.exist’} & \textit{Ø} & \textit{ahawi} ‘stand.\textsc{pl}’ & \textit{amik} ‘exist’ \mbox{\textit{amit} ‘not.exist’} & \textit{jim} ‘go’ \textit{ja’ich} ‘exist’ & \textit{xuawe} ‘exist’ & \textit{kate} ‘be\textsc{\textsubscript{loc}.pl}' & \textit{onka}  ‘be\textsubscript{\textsc{loc}}' \\

\textsc{nominal copula} & \textit{Ø, -tu} & \textit{Ø, -tu} & \textit{hu, ine,}  \textit{gari} & \textit{Ø} & \textit{jir, jix} & \textit{Ø, -tɨ(a),}  \textit{-ya} & \textit{Ø, wawi,}  \mbox{\textit{katka,}  \textit{taw}} & \textit{Ø, ye,}  \textit{čiwa}  \textit{katka}\\
\lspbottomrule
\end{tabularx}
\end{sidewaystable}

The distribution of each verb type can be motivated by the nature of figure. First, in this questionnaire, the location of humans is preceded by the question \textit{Where is your father?}, meaning the person does not know the actual position of the figure. From Table (\ref{tab:GuerreroTable5}), I observe four patterns in the answers to this question: Mayo uses \textit{aane} ‘exist’ (\ref{Guerrero10a}); Nahuatl varieties use \textit{ka} $\sim$  \textit{onka} ‘be located’ (\ref{Guerrero10b}); Yaqui and Southern Tepehuan alternate between ‘sit’, ‘be around’ or ‘exist’ (\ref{Guerrero10c}--\ref{Guerrero10d}), and the remaining languages make use of ‘sit’ as the default choice (\ref{Guerrero10e}). Dogs, cats, and other animals that can change their posture voluntarily are conceived as human figures when describing their location.

\ea \label{Guerrero10}
\ea \label{Guerrero10a}
\langinfo{Mayo}{}{\cite[80]{Freeze1989}} \\
\gll In pa ka: howa-po a:ne. \\
\textsc{1sg.poss} father \textsc{neg} house-\textsc{loc} exist\\
\glt ‘My father is not in the house.’ \\

\ex \label{Guerrero10b}
\langinfo{Hidalgo Nahuatl}{}{\cite[47]{Lastra1980}} \\
\gll No-tata ompa ka teč in	tiankis-λe. \\
\textsc{1sg.poss}-father there  be.located \textsc{loc} \textsc{det} tianguis-\textsc{abs} \\
\glt ‘My father is at the market.’ \\

\ex \label{Guerrero10c}
\langinfo{Southern Tepehuan}{}{\cite[90]{GarciaReyes2023}} \\
\gll Gu-ñ chat  bhammɨ  əirɨ gaa-tɨr.   \\
\textsc{det-1sg.poss} father \textsc{dir} be.around	field-\textsc{loc} \\
\glt ‘My father is at the field.’  \\

\ex \label{Guerrero10d}
\langinfo{Southern Tepehuan}{}{\cite[91]{GarciaReyes2023}} \\
\gll Gu-ñ chat mi’ daa ki-cham.   \\
\textsc{det-1sg.poss} father \textsc{dir} sitting.\textsc{sg} house-\textsc{loc} \\
\glt ‘My father is sitting in the house.’  \\

\ex \label{Guerrero10e}
\langinfo{Huichol}{}{\cite[91]{Gomez1999}} \\
\gll Ne-papa Manuweri-tsia p-e-ka.   \\
\textsc{1sg.poss}-father Manuel-with \textsc{as-loc}-sitting\textsc{.vl.sg} \\
\glt ‘My father is sitting with Manuel.’  \\

\z
\z

Second, it seems there is a tendency to classify non-human entities based on the geometric (inherent) features of the figure and its disposition with respect to the ground. For example, birds and flies are classified as ‘sit’ or ‘stand’, while churches and houses are seen as ‘stand’ (except in Yaqui). However, the locative description of pots, machetes, and beans varies (\ref{Guerrero11}): pots are considered as ‘stand’ in Guarijio, ‘sit’ in Lower Pima, Southern Tepehuan and Huichol, and ‘be located’ in Yaqui, while machetes are seen as ‘stand’ or ‘lie’, and beans are conceived as simply ‘be located’. Yet, Mayo, Hidalgo Nahuatl, and Mexicanero are exceptions to this pattern, since churches, houses, pots, machetes, and beans are considered to ‘be located’. That is, even those languages that use postural verbs for animate and inanimate figures may use a general locative verb with certain entities (clothes and beans in this corpus). Verbs like \textit{manek, mani,} and \textit{ma} in Yaqui, Guarijio and Huichol, \textit{o:rek} in Mayo, \textit{vivuta} in Lower Pima, and \textit{ka} and \textit{onka} in Nahuatl languages express the location of the figure without any indication of topological relations (e.g. they are unspecified for posture).

\ea \label{Guerrero11}

\ea \label{Guerrero11a}
\langinfo{Guarijio}{}{\cite[69--71]{Miller1993}}\\
\gll Weh-či \textbf{weri} sigori=ga. \\
ground-\textsc{loc} standing.\textsc{sg} pot=\textsc{emph}\\
\glt ‘The pot is standing on the ground.’\\


\ex \label{Guerrero11b}
\langinfo{Guarijio}{}{\cite[69--71]{Miller1993}}\\
\gll Sigori-ci \textbf{mani} muni=ga. \\
pot-\textsc{loc} be.located	beans=\textsc{emph} \\
\glt‘The beans are in the pot.’ \\

\ex \label{Guerrero11c}
\langinfo{Pima bajo}{}{\cite[55--57]{Estrada1998}} \\
\gll Ha’a divor-tam \textbf{dah}.   \\
pot ground-\textsc{loc}	sitting.\textsc{sg} \\
\glt ‘The pot is sitting on the ground.’  \\

\ex \label{Guerrero11d}
\langinfo{Pima bajo}{}{\cite[55--57]{Estrada1998}} \\
\gll Vav-tam ha’a-tam \textbf{vivuta}.   \\
beans\textsc{pos} pot-\textsc{loc} be.located.\textsc{ipfv} \\
\glt ‘The beans are in the pot.’  \\

\ex \label{Guerrero11e}
\langinfo{Huichol}{}{\cite[91--93]{Gomez1999}} \\
\gll Xari kwie-po \textbf{p-u-ka}.   \\
pot ground-\textsc{loc} \textsc{as-loc}-sitting.\textsc{vl.sg} \\
\glt ‘The pot is sitting on the ground.’  \\

\ex \label{Guerrero11f}
\langinfo{Huichol}{}{\cite[91--93]{Gomez1999}} \\
\gll Mume xari-ta \textbf{pi-ye-ma}.    \\
clothes	pot-\textsc{loc} \textsc{as-loc}-lying.\textsc{sg} \\
\glt ‘The beans are in the pot.’  \\

\ex \label{Guerrero11g}
\langinfo{Mexicanero}{}{\cite[72--74]{Canger2001}} \\
\gll In komih \textbf{Ø-onka} pa yelin rinkon.\\
\textsc{det} pot 3\textsc{sg}-be.located \textsc{loc} \textsc{dem} corner \\
\glt ‘The pot is in this corner.’  \\

\ex \label{Guerrero11h}
\langinfo{Mexicanero}{}{\cite[72--74]{Canger2001}} \\
\gll In išoh \textbf{ Ø-onka} pa komih. \\
\textsc{det} beans	3\textsc{sg}-be.located \textsc{loc} pot \\
\glt ‘The beans are in the pot.’   \\

\z
\z

From Table (\ref{tab:GuerreroTable6}) it is clear that Southern Uto-Aztecan languages have existential verbs and their use increases when speakers are asked to translate ‘There is an X at Y’. However, Guarijio is an outlier since it only has a negative existential verb, whereas Nahuatl languages preserve the general ‘be located’ verb. According to this data, Taracahita languages distinguish the existence of animate vs. inanimate entities: in Yaqui and Mayo, humans are combined with \textit{a:ane} ‘exist’, and non-humans with  \textit{aayuk} and \textit{aika}, whereas Guarijio extends the use of postural verbs to humans, trees, markets, and houses, as well as \textit{mani} ‘be located’ to beans and water (\ref{Guerrero12a}--\ref{Guerrero12b}). In opposition, Tepiman and Corachol languages use the same existential verb regardless of the animacy of the figure. Moreover, some languages may use a different verb for plural entities; for instance, Yaqui and Guarijio use a postural verb when there is one tortilla, but an existential or ‘be located’ verb when there is more than one. Additionally, Nahuatl languages may use \textit{kate} ‘sit’ for certain plural animate figures (\ref{Guerrero12c}--\ref{Guerrero12d}).

\ea \label{Guerrero12}

\ea \label{Guerrero12a}
\langinfo{Guarijio}{}{\cite[79]{Miller1993}} \\
\gll I’wa \textbf{kahti} pie temei. \\
here sitting.\textsc{sg} one tortilla\\
\glt ‘One tortilla is sitting here.’ \\

\ex \label{Guerrero12b}
\langinfo{Guarijio}{}{\cite[79]{Miller1993}} \\
\gll Nao temei \textbf{mani} i’wa=go. \\
four tortilla be.located here=\textsc{emph}\\
\glt ‘There are four tortillas here.’  \\

\ex \label{Guerrero12c}
\langinfo{Hidalgo Nahuatl}{}{\cite[59]{Lastra1980}} \\
\gll Iteč in kwauλa-λi a’mo	\textbf{kate} masa:-me’?   \\
in \textsc{det}	monte-\textsc{abs} \textsc{neg}	sitting.\textsc{pl}	deer-\textsc{pl} \\
\glt ‘Aren’t there any deer on the monte?’  \\

\ex \label{Guerrero12d}
\langinfo {Mexicanero}{}{\cite[85]{Canger2001}} \\
\gll Pa-in monte amo \textbf{onka} masa-t? \\
\textsc{loc-det} monte \textsc{neg} be.located deer-\textsc{abs} \\
\glt ‘Aren’t there any deer on the monte?’\\

\z
\z

Some languages use a different verb to express the absence of an entity in a certain place. The Taracahita group uses \textit{aayuk} and \textit{aika} for ‘exist’ and \textit{kaita} $\sim$ \textit{ka:ita} $\sim$ \textit{kaite}  for ‘not exist’. Huichol may use \textit{ma} for ‘be located’ and \textit{mawe} for ‘not be located’. Lower Pima uses \textit{amik} for ‘exist’ and \textit{amit} for ‘not exist’, whereas Southern Tepehuan describes the existence of beans and water with \textit{jim} ‘go’ (i.e., they ‘go in that place’) and their absence by negating the existential verb \textit{ja’ich} (\ref{Guerrero13}). In the \textit{Archivo}, negative polarity was part of the second prompt (when translating existential clauses), hence it is unclear whether postural verbs can be used both in affirmative and negative locative descriptions.

\ea \label{Guerrero13}
\langinfo{Southern Tepehuan}{}{\cite[105]{GarciaReyes2023}} \\

\ea \label{Guerrero13a}
\gll Bha ja’a-ta’m  jim  gu  sudai’. \\
\textsc{dir} pot-inside	go	\textsc{det} water \\
\glt ‘There is water in the pot.’ \\

\ex \label{Guerrero13b}
\gll Bha ja’a-ta’m  cham \textbf{jai’ch} gu  sudai’. \\
\textsc{dir} pot-inside \textsc{neg} exist \textsc{det} water \\
\glt ‘There is no water in the pot.’ \\

\z
\z

Therefore, based on data from the \textit{Archivo}, I have shown that Mexicanero and Hidalgo Nahuatl languages tend to use one and the same general locative verb in locative descriptions, whereas other Southern Uto-Aztecan languages embrace both postural and existential predicates depending on the prompt used to collect the data. From a diachronic perspective, the general locative verbs \textit{ka} and \textit{onka} in Mexicanero and Hidalgo Nahuatl are clearly related to ‘sit’ in Yaqui (\textit{katek}), Guarijio (\textit{kahti}), and Huichol (\textit{ka}), and ‘lie’ in Southern Tepehuan (\textit{kaat}). Interestingly, Mayo exhibits more variation than expected, since it systematically uses \textit{o:rek} ‘be located’ with inanimate figures, while the use of postural and existential verbs are limited to some animate entities. Mayo and Yaqui are the closest languages within the Taracahita group, but they behave differently in terms of coding figure-ground relationships.

\subsection{Data from visual stimuli} \label{DataVisualStimuli}

Since 2011, a group of linguists has been studying locative descriptions based on first-hand data for Yaqui (\cite{Gutierrez2011}; \cite{Guerrero2012}, \citeyear{Guerrero2017}, \citeyear{Guerrero2023}; \cite{O’mearaGuerrero2015}), Cora (\cite{VazquezSoto2013}), Ra’ichaala Tarahumara (\cite{Moreno2017}), Southern Tepehuan (\cite{GarciaSalido2017}; \cite{GuerreroGarciaSalido2019}), Lower Pima (\cite{Valenzuela2017}), Huichol (\cite{Guerrero2017}; \cite{GuerreroGarciaSalido2019}; \cite{Hernandez2024}; \cite{GuerreroRamosUnderReview}), and Huasteca Nahuatl (Author's unpublished corpus data). The data was collected in the field using visual stimuli: BowPed (with black and white line drawings), PSPV and DELOCA (with color photos). Each visual stimulus contains a figure and a ground that illustrate a particular topological relation (e.g., sitting on the chair, hanging from the tree, stuck to a stick). After being shown the image, native speakers of each language were asked to describe the location of the figure with respect to the ground by answering the question \textit{Where is the [figure]?} The figure object involves human, animate (dogs, cats, fishes, rabbits, insects, owls), and inanimate entities (fruits, trees, ladders, houses, fences, pencils, cups, pots, bottles, phones, lamps, balls, clothes, rags, ropes, sticks, branches, beans, and so on); the BowPed also includes some body parts. Due to the wide variety of objects included in the visual questionnaires, the data collected with this methodology not only allows the verbal inventory in each language to be reviewed, but it also permits an examination of their distribution based on the semantic features of the figure as well as the spatial configuration of the figure on the ground.
The data collected using a visual stimulus reveal the prominence of postural verbs in locative descriptions, whereas existential verbs are rather infrequent. 

Hence, it is confirmed that Yaqui and Tarahumara (Taracahita), Lower Pima and Southern Tepehuan (Tepiman), and Cora and Huichol (Corachol) all use ‘sit’, ‘stand’, ‘lie’, and ‘hang’ as the default choice to locate animate and inanimate entities. The verbs ‘sit’, ‘stand’, and ‘lie’ (and ‘hang’ in some languages) show singular and plural suppletive forms depending on the number of the figure. For most inanimate figure objects, the use of these verbs is obligatory, and their selection mainly reflects inherent features of the figure and the spatial configuration. For example, objects with a compact, rounded shape are associated with ‘sit’; objects conceived as being upright and having vertical elongation are associated with ‘stand’; thin objects having horizontal elongation are associated with ‘lie’; finally, objects that are suspended from the bottom but firmly attached to the ground or that are floating on water are conceived as ‘hang’. Furthermore, the use of non-postural verbs is rare and limited to certain objects, e.g., beans on a table.

The basic inventory in each language includes additional verbs alongside ‘sit’, ‘stand’, ‘lie’, and ‘hang’. For example, Tarahumara has a ‘kneel’ verb and a verb meaning ‘be parallel’. Lower Pima has two ‘lie’ verbs, one for animate and the other for inanimate entities. Northern Tepehuan also has two ‘lie’ verbs. Southern Tepehuan has two ‘lie’ and two ‘stand’ verbs that also vary depending on animacy. Huichol has two ‘sit’ verbs, one for compact spherical objects and the other for flexible ones, two ‘lie’ verbs, one for objects placed horizontally, and the other for thin/rigid figures extended along the ground, as well as \textit{ma/mane} ‘be located horizontally (\textsc{sg/pl})’. Likewise, Cora has seven postural verbs (\cite{VazquezSoto2013}): \textit{ka/tei} ‘sit (\textsc{sg/pl})’ (\ref{Guerrero14a}), \textit{bee/uu} ‘stand (\textsc{sg/pl})’ (\ref{Guerrero14b}), \textit{ka’a/he’e} ‘lie (\textsc{sg/pl})’ (\ref{Guerrero14c}), \textit{re’eka/he’e} ‘lie (\textsc{sg/pl})’, \textit{pii/kaa} ‘lie, extended (\textsc{sg/pl})’ (\ref{Guerrero14d}), \textit{kaabe’e/kaabibi-hme} ‘hang (\textsc{sg/pl})’ (\ref{Guerrero14e}), and finally, \textit{mwaa/mwan} ‘be located (\textsc{sg/pl})’ (\ref{Guerrero14f}). Additionally, there are two existential verbs, \textit{piriki} and \textit{e’en}, but only the latter may occur in locative descriptions (\ref{Guerrero14g}).



\ea \label{Guerrero14}
\langinfo{Cora}{}{\cite[160--161, 180]{VazquezSoto2013}} \\

\ea \label{Guerrero14a}
\gll Báaso=pu hetsé hu-ráh-\textbf{ka} mansáan. \\
bowl=3.\textsc{sbj} in \textsc{inside}-\textsc{inside}-sitting.\textsc{sg} apple\\
\glt ‘The apple is sitting inside the bowl.’ \\

\ex \label{Guerrero14b}
\gll Puéerta hetsé=pu hah-téh-\textbf{bee} í námi’i-ra’an. \\
door in=3.\textsc{sbj}	\textsc{prox}-\textsc{prox}-standing.\textsc{sg} 	\textsc{top} lock-3\textsc{sg.poss} \\
\glt ‘The lock is standing in the door.’  \\

\ex \label{Guerrero14c}
\gll Serbiyéeta heté=pu be’e-ráa-\textbf{ka’a} í kucháara.  \\
napkin	below=3.\textsc{sbj} \textsc{below}-\textsc{below}-lying.\textsc{sg} \textsc{top} spoon \\
\glt ‘The spoon is lying below the napkin.’  \\

\ex \label{Guerrero14d}
\gll Kiyéh hapwá=pu húh-\textbf{pii} hí káunari. \\
trunk above=3.\textsc{sbj} \textsc{over}-extended.\textsc{sg} \textsc{top} rope \\
\glt ‘The rope is lying above the tree trunk.’  \\

\ex \label{Guerrero14e}
\gll Téete’e=pu \textbf{káabe’e} háh hetsé í tsáapwa. \\
inside=3.\textsc{sbj} hanging.\textsc{sg} water in \textsc{top} mojarra \\
\glt ‘The mojarra fish is hanging inside the water.’ \\

\ex \label{Guerrero14f}
\gll Mée=pu wa-tá-\textbf{mwaa} í múuku’u. \\
there=3.\textsc{sbj} 3.\textsc{ext-sop}-be.located.\textsc{sg} \textsc{top} 	hat \\
\glt ‘The hat is located over there.’ \\

\ex \label{Guerrero14g}
\gll Méesa hapwá=pu h-\textbf{é’en} hí báaso.	   \\
table above=3.\textsc{sbj} 3.\textsc{sbj}-exist \textsc{top} cup \\
\glt ‘The cup is above the table.’   \\

\z
\z

Preliminary observations on data from Huasteca Nahuatl show that the general locative verb \textit{onkak} ‘be located’ is present, but its use is limited (\ref{Guerrero15a}). However, there are two copula verbs: \textit{eltok} ‘be’ for certain inanimate objects (\ref{Guerrero15b}), and \textit{iktok $\sim$ ihtok} ‘be’ for animate objects (\ref{Guerrero15c}) (\cite[214--215]{BellerBeller1979}). Additionally, there are several examples of postural verbs with humans (\ref{Guerrero15d}), and animate entities (\ref{Guerrero15e}). Note that copula and postural verbs are marked by the stative suffix \textit{-ok}.

\ea \label{Guerrero15}
Huasteca Nahuatl \\
\ea \label{Guerrero15a}

\gll Ipan se kuatetomitl \textbf{onkak} kamoj-tli. \\
3\textsc{sg.loc} \textsc{det} trunk be.located sweet\_potato-\textsc{abs}\\
\glt ‘The sweet potatoes are located by the trunk.’ \\
(DELOCA\_29: PA) \\

\ex \label{Guerrero15b}
\gll Se kuetlaxolo-li \textbf{el-t-ok} ipan tlali. \\
\textsc{det} ball-\textsc{abs} be-\textsc{linker-sta} 3\textsc{sg.loc} table \\
\glt ‘The ball is located on the table.’  \\
(PSPV\_7: PA) \\

\ex \label{Guerrero15c}
\gll Itsala se silla \textbf{is-tok} se cuatochi.  \\
below \textsc{det} chair be-\textsc{linker-sta} \textsc{det} rabbit \\
\glt ‘The rabbit is located under the chair.’  \\
(DELOCA\_35: PA) \\

\ex \label{Guerrero15d}
\gll Se telpoka-tl \textbf{ye-t-ok} ipan silla. \\
\textsc{det} man-\textsc{abs} sitting-\textsc{linker-sta} 3\textsc{sg.loc} chair \\
\glt ‘The man is sitting on the chair.’  \\
(DELOCA\_37: PA) \\

\ex \label{Guerrero15e}
\gll Se uakax \textbf{uilan-t-ok} tlalchi. \\
\textsc{det} cow lying-\textsc{linker-sta} ground \\
\glt ‘The cow is lying on the ground.’  \\
(DELOCA\_59: PA) \\
\z
\z

In summary, based on the above data, Yaqui, Guarijio, Tarahumara, Lower Pima, Southern Tepehuan, Huichol, and Cora can be characterized as small-set languages in Ameka and Levinson’s typology, since they use 5--7 contrastive postural verbs to encode figure-ground relationships. The available data regarding Mayo, Hidalgo Nahuatl, and Mexicanero suggest that they can be considered as single-set languages since they typically use a general locative verb. Still, these languages may use postural verbs as residual verbs in some contexts, e.g., to locate animate entities. More data is needed for Huastec Nahuatl. In the next section, I revisit the Yaqui data to carefully examine the semantic properties that motivate the selection of one verb over the others. I also demonstrate that Yaqui exhibits a classificatory system regarding locative predication, which is not limited to postural verbs as suggested before.

\section{Locative descriptions in Yaqui} \label{LocativeDescriptionsYaqui}

Yaqui is a synthetic/agglutinating, dependent-marking, head-final language (\cite{DedrickCasad1999}; \cite{Guerrero2006}). Yaqui keeps track of syntactic functions of nominal arguments using case markers: the nominative is unmarked (Ø), the accusative is marked by the suffix \textit{-ta}, and the oblique is marked by postpositions. Plural nominals are marked by \textit{-(i)m}, excluding nominative/accusative case marking. Previous studies dealing with some aspects of locative verbs include the brief mentions in \citet{DedrickCasad1999} and some monographic works claiming that Yaqui typically uses four postural verbs in locative descriptions (\cite{Guerrero2006}, \citeyear{Guerrero2012}, \citeyear{Guerrero2017}, \citeyear{Guerrero2023}; \cite{Gutierrez2011}; \cite{BelloroGuerrero2012}; \cite{O’mearaGuerrero2015}). However, new data reveal that Yaqui also uses general locative, existential, and positional verbs (\tabref{tab:GuerreroTable7}) when describing the location of the figure with respect to the ground. Except for ‘hang’, postural verbs have suppletive singular/plural forms, whereas the so-called existential verbs distinguish animacy. Positional verbs correspond to the resultative version of several causative predicates. Note that several verbs have grammaticalized the stative/perfective \textit{-k(a)} and cannot receive any other tense-aspect morpheme.

In what follows, I describe postural (§\ref{GuerreroPosturalVerbs}), general locative (\ref{GeneralLocativeVerbs}), existential (\ref{ExistentialVerbsGuerrero}) and positional verbs (§\ref{PositionalVerbsGuerrero}) based on data collected with visual stimuli and oral texts. I then discuss the distribution of these locative/existential verbs according to the figure-ground spatial configuration they denote (§\ref{LocExVerbsDistribution}).

\begin{table}
\begin{tabularx}{\textwidth}{llQ}
\lspbottomrule
 {postural} & \textit{bo’oka/to’oka} & ‘lie (\textsc{sg/pl})’ \\
& \textit{weyek/ja’abwek} & ‘stand (\textsc{sg/pl})’\\
& \textit{katek/jo’oka} & ‘sit (\textsc{sg/pl})’ \\
& \textit{cha’aka} & ‘hang (\textsc{sg/pl})’ \\
\midrule
 {general locative}  & \textit{o’orek} & ‘be located (\textsc{sg/pl})’ \\
& \textit{manek} & ‘be located (\textsc{sg/pl})' \\
\midrule
 {existential} & \textit{aane} & ‘exist (human)’ \\
& \textit{aayuk} & ‘exist (non-human)’ \\
\midrule
 {positional}  & \textit{kechia} & ‘be placed in standing position (\textsc{sg})’\\
& \textit{yechai} & ‘be placed in sitting position (\textsc{sg})’ \\
& \textit{kibachai/kiimari} & ‘be placed inside (\textsc{sg/pl})' \\
& \textit{pitti} & ‘be nailed’ \\
& \textit{go’oti} & ‘be scattered’ \\
& \textit{suma’i} & ‘be tied’ \\
& \textit{su’uti} & ‘be inserted’ \\
& \textit{chuakti} & ‘adhered/be stuck’ \\
& \textit{cha’ati} & ‘adhered/ be hanged’ \\
& \textit{tutti} & ‘be kept inside/stored face down’ \\
& \textit{mochila, montoi} & ‘be piled up’ \\
& \textit{po’ola} & ‘be inclined’ \\
& \textit{benti} & ‘adhered/ be spread’ \\
& \textit{patti} & ‘be covered’ \\
\lspbottomrule
\end{tabularx}
\caption{Locative/existential verbs in Yaqui}
\label{tab:GuerreroTable7}
\end{table}

\subsection{Postural verbs} \label{GuerreroPosturalVerbs}
Verbs such as ‘sit’, ‘stand’, and ‘lie’ indicate a particular body orientation, prototypically for animate entities that may change their posture voluntarily. The examples below illustrate the actual posture of humans and dogs while they are ‘standing’ (\ref{Guerrero16a}--\ref{Guerrero16b}), ‘sitting’ (\ref{Guerrero16c}), and ‘lying’ (\ref{Guerrero16d}).

\ea \label{Guerrero16}

\ea \label{Guerrero16a}
\gll Ju’u-Ø yoeme-Ø kari-bepa \textbf{weyek}. \\
\textsc{det-nom} man-\textsc{nom} house-on.top	standing.\textsc{sg.pfv} \\
\glt ‘The man is standing on top of the house.’ \\
(BowPed\_34: AV) \\



\ex \label{Guerrero16b}
\gll U-me chu’u-im  bankoa naat \textbf{ja’abwek}. \\
\textsc{det-pl}	dog-\textsc{pl}	chair	side.\textsc{locc} standing.\textsc{pl.pfv} \\
\glt ‘The dogs are standing close to the chair.’  \\
(Fieldwork notes)\\

\ex \label{Guerrero16c}
\gll U-Ø chu’u-Ø kari-po pa’aku \textbf{katek}. \\
\textsc{det-nom} dog-\textsc{nom} house-\textsc{loc} outside	sitting.\textsc{sg.pfv} \\
\glt ‘The dog is sitting outside the house.’  \\
(BowPed\_6: ML) \\

\ex \label{Guerrero16d}
\gll U-Ø chu’u-Ø kari-po \textbf{bo’oka}. \\
\textsc{det-nom} dog-\textsc{nom} house-\textsc{loc} lying.\textsc{sg.pfv} \\
\glt ‘The dog is lying inside the house.’   \\
(BowPed\_71: AM) \\
\z
\z


When describing the location of concrete and movable inanimate objects, Yaqui shows many of the conventional collocations of figure-posture observed in other languages, in which these verbs function as a classificatory-like system (\cite{Newman2002}; \cite{AmekaLevinson2007}). Among these verbs, \textit{bo’oka/to’oka} ‘lie (\textsc{sg/pl})’ is the easiest to describe since it expresses the location of entities that are longer than they are wide and that are spread out on a flat surface. The types of entities that are categorized as being in a lying position generally illustrate a longer horizontal axis than vertical one, in the sense that they can be located at the top (\ref{Guerrero17a}) or be extended over a surface (\ref{Guerrero17b}). This predicate is also used to describe the inherent position of certain animals, including reptiles, lizards, snakes, worms, and insects whose bodies are in full contact with the ground (\ref{Guerrero17c}).

\ea \label{Guerrero17}

\ea \label{Guerrero17a}
\gll Laapis-Ø eskritorio-po \textbf{bo’oka}. \\
pencil-\textsc{nom}	desk-\textsc{loc} lying.\textsc{sg.pfv}\\
\glt ‘The pencil is lying on the desk.’ \\
(BowPed\_59: ML) \\

\ex \label{Guerrero17b}
\gll U-Ø wikia-Ø teta-t jikat \textbf{bo’oka}. \\
\textsc{det-nom} rope-\textsc{nom}	stone-\textsc{locc}	up.\textsc{locc}	lying.\textsc{sg.pfv} \\
\glt ‘The rope is lying above the stone.’   \\
(PSPV\_3:GF) \\

\ex \label{Guerrero17c}
\gll U-Ø bakot-Ø wia-po \textbf{bo’oka}. \\
\textsc{det-nom} snake-\textsc{nom}	ground-\textsc{loc}	lying.\textsc{sg.pfv} \\
\glt ‘The snake is lying on the rock.’  \\
(BowPed\_15: BV)  \\
\z
\z

The pair \textit{weyek}/\textit{ja’abwek} ‘stand (\textsc{sg/pl})’ demands a greater degree of control and volition on the part of the entity to maintain a standing position. In Yaqui, horses, cows, deer, and hens are conceived as ‘standing’ (\ref{Guerrero18a}). The types of inanimate figures that inherently ‘stand’ can be characterized as being longer and thinner entities, e.g., trees, houses, churches, walls, candles, brooms, ladders, flagpoles, bottles, and fences (\ref{Guerrero18b}--\ref{Guerrero18c}).

\ea \label{Guerrero18}

\ea \label{Guerrero18a}
\gll Toto’i-m korral-po \textbf{ja’abwek}. \\
hen-\textsc{pl} corral-\textsc{loc} standing.\textsc{sg.pfv}\\
\glt ‘The hens are standing in the corral.’  \\
(DELOCA\_18: ML) \\

\ex \label{Guerrero18b}
\gll 	Juupa-Ø juya-po \textbf{weyek}. \\
mesquite-Ø	hill-\textsc{loc} standing.\textsc{sg.pfv} \\
\glt ‘The mesquite tree is standing on the hill.’  \\
(BowPed\_17: AV) \\

\ex \label{Guerrero18c}
\gll U-Ø sapti-Ø kari naapo \textbf{weyek}. \\
\textsc{det-nom} reed.fence-\textsc{nom} house side.\textsc{loc} standing.\textsc{sg.pfv} \\
\glt ‘The reed fence is standing near the house.’  \\
(DELOCA\_63\_AM) \\
\z
\z

The third postural verb \textit{katek}/\textit{jo’oka} ‘sit (\textsc{sg/pl})’ is a little more complex to define because it can be associated with several types of entities without an apparent common geometrical property. For instance, ‘sit’ is the default position for birds (\ref{Guerrero19a}), amphibians, turtles, and chameleons. The locative description of inanimate objects with a round shape like squash, tomatoes, onions, lemons, eggs, stones, balls, stars, and the Milky Way are all characterized as ‘sit’ (\ref{Guerrero19b}). Other entities that can be conceived to be ‘sitting’ include towns (\ref{Guerrero19c}), mountains, water springs and nets; furniture like tables, chairs, beds, and latrines; figure objects like pitchers, pots, cars, boats, and wheelbarrows, as well as some body parts like eyes, ears, the nape of the neck, the crown of the head, ankles, brains, intestines, gallbladders, and livers. A book on a shelf, bugs on a wall, and tortillas in a napkin are also associated with ‘sit’ even though their geometrical features are very different.

\ea \label{Guerrero19}
\ea \label{Guerrero19a}
\gll Wiikit-Ø juya-t \textbf{katek}. \\
bird-\textsc{nom} tree-\textsc{locc} sitting.\textsc{sg.pfv}\\
\glt ‘The bird is sitting in the tree.’   \\
(DELOCA\_32: AM) \\

\ex \label{Guerrero19b}
\gll U-me’e chokim mek-jikat \textbf{jo’oka}. \\
\textsc{det-pl} star.\textsc{pl} far-up.\textsc{locc} sitting.\textsc{sg.pfv} \\
\glt ‘The stars are sitting over there.’  \\
(Fieldwork notes)\\

\ex \label{Guerrero19c}
\gll Wiibisim Rajum-met	cha’aka	\textbf{katek}. \\
Wiibisim.\textsc{pl} Rajum-\textsc{pl.locc} after sitting.\textsc{sg.pfv} \\
\glt ‘Huirivis is sitting before Rahum.’  \\
(Fieldwork notes)\\

\ex \label{Guerrero19d}
\gll U-me librom tabla-po jiikat \textbf{katek}. \\
\textsc{det-pl}	book.\textsc{pl}	shelf-\textsc{loc}	up.\textsc{locc} 	sitting.\textsc{sg.pfv} \\
\glt ‘The book is on the shelf.’   \\
(BowPed\_8: ML) \\

\z
\z

The last postural verb corresponds to \textit{cha’aka} ‘hang’. This verb describes an object that is hanging from or attached to the ground, such as clothes, lamps, phones, flags, portraits, fruits, and leaves (\ref{Guerrero20a}), as well as ropes or clothes hanging down from an upper position (\ref{Guerrero20b}). It can be also used with human entities, as shown in (\ref{Guerrero20c}).

\largerpage
\ea \label{Guerrero20}
\ea \label{Guerrero20a}
\gll Telefono-Ø sami-t \textbf{cha’aka}. \\
phone-\textsc{nom} wall-\textsc{locc} hanging.\textsc{pfv}\\
\glt ‘The phone is hanging on the wall.’   \\
(BowPed\_25: AM) \\

\ex \label{Guerrero20b}
\gll Tajorim juya-bepa \textbf{cha’aka}. \\
clothes.\textsc{pl}	tree-on.top hanging.\textsc{pfv}\\
\glt ‘The clothes are hanging on top of the tree.’   \\
(PSPV\_59: ML) \\

\ex \label{Guerrero20c}
\gll Ili uusi-Ø	juya-po	mam-mea \textbf{cha’aka}. \\
\textsc{dim} child-\textsc{nom}	tree-\textsc{loc}	hand-\textsc{pl.ins}		hanging.\textsc{pfv} \\
\glt ‘The child is hanging from the tree with her hands.’  \\
(DELOCA\_41: AV) \\
\z
\z

\subsection{General locative verbs} \label{GeneralLocativeVerbs}

There are two locative verbs in Yaqui that do not lexicalize postural meanings: \textit{o’orek} and \textit{manek}. Both can be freely translated as ‘be located’ (\textit{estar} in Spanish). The first one is also attested in Mayo, whereas the second one is lexically related to \textit{maní} ‘be located’ in Guarijio, \textit{mwaa/mwan} ‘be located (\textsc{sg/pl})’ in Cora, and \textit{ma/mane} ‘lie’ in Huichol. In Yaqui, these verbs are limited to inanimate entities with or without a concrete shape, i.e., items that have a mass interpretation. For instance, seeds, liquids, and containers used to hold liquids are characterized by \textit{manek} (\ref{Guerrero21a}), while rope, balloons, waste, clothes, beans, keys, money, and concrete objects carefully placed on the ground are described by \textit{o’orek} (\ref{Guerrero21b}--\ref{Guerrero21c}). In my corpus, \textit{o’orek} is more frequent than \textit{manek}.

\ea \label{Guerrero21}
\ea \label{Guerrero21a}
\gll Tasa-Ø	mesa-po	\textbf{manek}. \\
cup-\textsc{nom} table-\textsc{loc}	be.located.\textsc{pfv} \\
\glt ‘The cup is located on the table.’   \\
(BowPed\_1: GF) \\

\ex \label{Guerrero21b}
\gll U-Ø wikiat-Ø kuta-t kom \textbf{o’orek}.\\
\textsc{det-nom} rope-\textsc{nom}	tree-\textsc{locc} down be.located.\textsc{pfv}\\
\glt ‘The rope is located down on the tree.’   \\
(PSPV\_68: AM) \\

\ex \label{Guerrero21c}
\gll U-Ø xabe-Ø mesa-po \textbf{o’orek}. \\
\textsc{det-nom} key-\textsc{nom} table-\textsc{loc} be.located.\textsc{pfv}\\
\glt ‘The keys are located on the table.’  \\
(DELOCA\_51: AV) \\
\z
\z

\subsection{Existential verbs} \label{ExistentialVerbsGuerrero}

Yaqui grammar identifies two “existential” verbs (\cite{DedrickCasad1999}): \textit{aane} for humans, and \textit{aayuk} for non-human animate and inanimate entities. There are no examples of \textit{aane} in data collected with visual stimuli since in the few pictures involving a human figure, the person maintains a specific posture (e.g., standing on the roof, sitting in front of the fire, sitting behind a sofa, sitting on a chair, lying on a bed). However, this verb occurs easily in natural conversation and oral texts (\ref{Guerrero22a}--\ref{Guerrero22b}). Since humans are conceived as volitionally and freely moving entities, \textit{aane} is usually translated as either ‘exist’, ‘be located’, ‘be around’, while \textit{aayuk} rarely occurs in BowPed and PSPV responses, but it is present in DELOCA’s description. In my data, \textit{aayuk} describes the location of grains, mass objects, and liquid objects, such as ashes, money, matches, garlic, and so on (\ref{Guerrero22c}). The “existential” meaning of these verbs is discussed in (§\ref{ExistentialConstructionsInYaqui}).

\ea \label{Guerrero22}
\ea \label{Guerrero22a}
\gll In maala-Ø	kosina-po \textbf{aane-n}.\\
\textsc{1sg.poss} mother-\textsc{nom} kitchen-\textsc{loc}	exist.\textsc{anim-ipfv} \\
\glt ‘My mother was in the kitchen.’  \\
(Conv2: 67) \\

\ex \label{Guerrero22b}
\gll Senu wasuktia-po=ne ama \textbf{aane-k}.\\
one year-\textsc{loc=1sg.nom} there exist.\textsc{anim-pfv}\\
\glt ‘I was there for one year.’    \\
(\cite{Buitimea2007}; \textit{tenku\_ania}: 72) \\

\ex \label{Guerrero22c}
\gll Munim wari-po \textbf{aayuk}. \\
beans.\textsc{pl} basket-\textsc{loc} exist.\textsc{inanim.pfv}\\
\glt ‘The beans are in the basket.’   \\
(DELOCA\_23: AV) \\
\z
\z

\subsection{Positional verbs} \label{PositionalVerbsGuerrero}
Finally, some pictures from BowPed were designed to also collect elaborated (less typical) locative situations. For instance, a paper pierced by a nail, a hose tied around a tree trunk, a ring on a finger, and a cup with a crack. In Yaqui, these and several other peripheral scenes were described with a resultative construction, which involves the stative version of placement verbs and other change-of-state causative verbs. As shown in (\ref{Guerrero23a}--\ref{Guerrero23b}), Yaqui resultative locative verbs are marked by the stative suffixes \textit{-i} or \textit{-la}. When there is a body part as the figure or the ground, a possessive verb may occur; compare (\ref{Guerrero23c}--\ref{Guerrero23d}).

\ea \label{Guerrero23}
\ea \label{Guerrero23a}
\gll Jikiam-mechi jiosiam \textbf{kiima-ri}. \\
nail.\textsc{pl-pl.locc} paper.\textsc{pl} put.into.\textsc{pl-sta} \\
\glt ‘The papers are pierced by the nail.’    \\
(BowPed\_22: BV) \\

\ex \label{Guerrero23b}
\gll Boomba kuta-t \textbf{suma’-i}. \\
balloon-\textsc{nom} stick-\textsc{locc} tie-\textsc{sta} \\
\glt ‘The balloon is tied to the stick.’ \\
(BowPed\_33: AV) \\

\ex \label{Guerrero23c}
\gll Sinto-Ø wikosa-t \textbf{yecha-i}. \\
bell-\textsc{nom} waist-\textsc{locc} sit.\textsc{sg-sta} \\
\glt ‘The bell is on the waist.’   \\
(BowPed\_42: AM) \\

\ex \label{Guerrero23d}
\gll Jamut-Ø sinto-ta wikosa-po	\textbf{jippue}. \\
woman-\textsc{nom}	bell-\textsc{acc} waist-\textsc{loc}	have\\
\glt ‘The woman has a bell on her waist.’    \\
(BowPed\_42: BV) \\

\z
\z

\subsection{Locative/existential verbs distribution} \label{LocExVerbsDistribution}
Yaqui postural, general locative, existential and positional verbs are all attested in data collected by visual stimuli. In the locative descriptions collected by BowPed and PSPV, postural verbs are the most common and productive verbs, while the occurrence of non-postural verbs is occasional. Figures \ref{GuerreroFig2} and \ref{GuerreroFig3} show the distribution of ‘sit’, ‘stand’, ‘lie’, and ‘hang’ according to the locative situation (pictures) from these two stimuli. DELOCA was designed as a complementary stimulus to collect data with animate and inanimate objects not considered by the other two instruments (plural, mass, and non-countable entities). \figref{GuerreroFig4} shows the distribution of postural verbs in DELOCA responses. Unlike with other verbs, the use of positional verbs is easier to predict given the spatial configuration of the figure with respect to the ground. As such, these verbs are not included in Figures \ref{GuerreroFig2}--\ref{GuerreroFig4}.

\begin{figure}
% \includegraphics[width=\textwidth]{figures/guerrero-Fig2.jpg}
\includegraphics[width=\textwidth]{figures/guerrero-Fig2mod.pdf}
\caption{Yaqui postural verbs in Bowped}
\label{GuerreroFig2}
\end{figure}

\begin{figure}
% \includegraphics[width=\textwidth]{figures/guerrero-Fig3.jpg}
\includegraphics[width=\textwidth]{figures/guerrero-Fig3mod.pdf}
\caption{Yaqui postural verbs in PSPV}
\label{GuerreroFig3}
\end{figure}

\begin{figure}
% \includegraphics[width=\textwidth]{figures/guerrero-Fig4.jpg}
\includegraphics[width=\textwidth]{figures/guerrero-Fig4mod.pdf}
\caption{Yaqui postural verbs in DELOCA}
\label{GuerreroFig4}
\end{figure}



\begin{figure}[b]
% \includegraphics[width=\textwidth]{figures/guerrero-Fig5.jpg}
\includegraphics[width=\textwidth]{figures/guerrero-Fig5mod.pdf}
\caption{Yaqui postural and general locative verb alternations}
\label{GuerreroFig5}
\end{figure}

Nevertheless, the most interesting cases in Yaqui involve inanimate objects with inherent features that are harder to categorize with postural verbs, such as plural objects of different sizes or textures, pairs and collective objects, grains, liquids, and other mass-like entities. These objects tend to be conceived as ‘be located’ or ‘exist’. Accordingly, Yaqui postural verbs strongly describe the location of typical, concrete, and movable entities, whereas \textit{aayuk}, \textit{manek}, and \textit{o’orek} locate less typical entities. In fact, Yaqui speakers show little consistency in their use of non-postural verbs, and it’s common for more than one verb to be used for the same figure-ground relationship. For instance, a cemetery can be conceived as \textit{o’orek} ‘be located’ or \textit{bo’oka} ‘lie’ (\ref{Guerrero24a}), some garlic in a basket can be described as \textit{manek} ‘be located’ or \mbox{\textit{aayuk} ‘exist’} (\ref{Guerrero24b}), and a town or group of houses can be described as \textit{o’orek} ‘be located’, \mbox{\textit{aayuk} ‘exist’}, or \textit{bo’oka} ‘lie’ (\ref{Guerrero24c}). Figures \ref{GuerreroFig5} and \ref{GuerreroFig6} try to capture these alternations when asserting the location of less typical figures.



\ea \label{Guerrero24}
\ea \label{Guerrero24a}
\gll Sementerio-Ø teo-po bichapo \textbf{o’orek} \textup{//} \textbf{bo’oka}.\\
cemetery-\textsc{nom} church-\textsc{loc} in.front	be.located.\textsc{pfv} // lying.\textsc{sg.pfv}	 \\
\glt ‘The cemetery is located/lying in front of the church.’	  \\
(DELOCA\_33) \\

\ex \label{Guerrero24b}
\gll Poato-po ajosim \textbf{manek} \textup{//} \textbf{ayuk}.\\
dish-\textsc{loc} garlic.\textsc{pl} be.located.\textsc{pfv} // exist.\textsc{inanim}\\
\glt ‘The garlic is located in the dish/there is garlic on the dish.’  \\
(DELOCA\_20) \\

\ex \label{Guerrero24c}
\gll U-me kari-m kawi-bepa \textbf{o’orek} \textup{//} \textbf{ayuk} \textup{//} \textbf{bo’oka}. \\
\textsc{det-pl}	house-\textsc{pl} hill-on.the.top 	be.located.\textsc{pfv} // exist.\textsc{inanim.pfv} // lying.\textsc{sg.pfv} \\
\glt ‘The town is located at the top of the hill/there is a town at the top of the hill.’ \\
(DELOCA\_53) \\
\z
\z

\begin{figure}[t]
% \includegraphics[width=\textwidth]{figures/guerrero-Fig6_new.pdf}
% \includegraphics[width=\textwidth]{figures/guerrero-Fig6_newmod.pdf}
\includegraphics[width=\textwidth]{figures/guerrero-Fig6_newa.pdf}
\caption{Yaqui general locative and existential verb alternations}
\label{GuerreroFig6}
\end{figure}

Although the analysis of Yaqui non-postural verbs is ongoing, it is evident that \textit{o’orek} is the most productive verb in the corpus, whereas \textit{manek} is the least common. \textit{O’orek} is associated with several concrete figures carefully located on the ground; the spatial configuration of the figure can be considered as canonical (natural) or not, e.g., a ball located on the top of stones. \textit{Manek} usually describes the location of containers (e.g., a cup) or entities that are contained by the ground, whereas \textit{aayuk} is combined with collective and mass-like entities.  Still, several figures associated with \textit{manek} can also be described with \textit{aayuk}, but some entities characterized with \textit{aayuk} cannot be described using \textit{manek} (e.g., towns, houses, money, ashes, beans on the table).

Finally, I observed that the use of postural verbs in negative locative clauses is rare. For humans, I found a couple of examples negating a postural verb in oral narratives (\ref{Guerrero25a}), but the most common situation is to use negative copulative-like clauses with \textit{kaabe} ‘none, nobody’ for humans (\ref{Guerrero25b}) and \textit{kaita} ‘nothing’ for non-human entities (\ref{Guerrero25c}).\footnote{One of the reviewers mentioned that clauses with a negated verb (\ref{Guerrero25a}) and clauses with ‘nobody/nothing’ (\ref{Guerrero25b}--\ref{Guerrero25c}) cannot be directly compared as alternative strategies since they may mean different things. I agree. The analysis of negative locative descriptions requires more data, and so it is out of the scope of the present study.} 

\ea \label{Guerrero25}
\ea \label{Guerrero25a}
\gll Sestul ta’a-po=bea=ne  aman noite-k, jo’awa-u=ne yepsa-k kaa ama a \textbf{katek}-o.	\\
one	day-\textsc{loc=dm=1sg.nom}	there return-\textsc{pfv}	house-\textsc{dir=1sg.nom} arrive.\textsc{sg-pfv} \textsc{neg} there		3\textsc{sg.acc}	sitting.\textsc{sg-clm}\\
\glt  ‘One day I returned there, I arrived at (his) house when he 
		wasn’t there.’	  \\
(\cite{Buitimea2007}; ko’okoi: 27) \\

\ex \label{Guerrero25b}
\gll Ju’u Suawaka ase’ebwa-wa sisiwokjikuchiam yeu=wike-k, ketun a am \textbf{kaabe}-tu-k-o.\\
\textsc{det} Suawaka father.in.law-\textsc{gen}	sword.\textsc{pl} out=take-\textsc{pfv} \textsc{neg.yet} 3\textsc{sg.acc}	there nobody-\textsc{vblz-pfv-clm}\\
\glt ‘Suawaka took his father-in-law’s sword when he wasn’t there.’   \\
(\cite{Buitimea2007}; suawaka: 76) \\

\ex \label{Guerrero25c}
\gll Junak=into	kia	jita	juya	juni’i	\textbf{kaita}		junama’a.		 \\
then=\textsc{dm}	only	thing	tree	\textsc{adv} nothing	over.there\\
\glt ‘In that moment, there was not a single tree over there [to 
	hide myself].’  \\
 (\cite{Silva2004}; HVH:142) \\
\z
\z

\section{Existential constructions in Yaqui} \label{ExistentialConstructionsInYaqui}

In the previous section, I introduced \textit{aane} and \textit{aayuk} as “existential” verbs following both their description in Yaqui grammar and their most common translation into Spanish. However, these verbs do not behave as plain existential predicates. In the literature, the notions of location and existence have been distinguished based on pragmatic factors. It has been said, for instance, that in locative descriptions the figure tends to be definite and correlates to the ‘topic’, whereas the ground may be either definite or indefinite and tends to relate to the function ‘focus’. On the other hand, in existential descriptions, the figure tends to be indefinite, and the ground may be optional, e.g., \textit{There is a cup} (\cite{WardBirner1995}; \cite{Koch2012}; \cite{BentleyEtAl2015}; \cite{Creissels2019}). That is, functionally, an existential construction is “a clause in which an indefinite and discourse-new nominal phrase is said to be in some location” (\cite{HaspelmathNonverbal}).

As is the case in other Southern Uto-Aztecan languages, the use of determiners is optional in Yaqui. The noun phrases introducing the figures in (\ref{Guerrero26a}) and (\ref{Guerrero27}) have determiners, but those in (\ref{Guerrero26b}--\ref{Guerrero26d}) do not. Yaqui communities are organized into eight towns; after mentioning the names of each town in the narrative, the clause in (\ref{Guerrero26d}) reintroduces the phrase \textit{goi naiki pueplom} ‘eight towns’ without a determiner, so it does not express the existence of the towns, but rather their location. Therefore, the presence of determiners is not crucial to determining ‘definiteness’ in Yaqui.

\ea \label{Guerrero26}
\ea \label{Guerrero26a}
\gll Si’ime	ju-me’e	wikichi-m ama \textbf{aane}-me … \\
all \textsc{det-pl} bird-\textsc{pl} there exist.\textsc{anim-nmlz} \\
\glt ‘All the birds were there …’    \\
(\cite{SilvaAlvarezBuitimea1998}; taawe: 70) \\

\ex \label{Guerrero26b}
\gll Gente-Ø=be	junum bicha saja-ka-me Jori-po \textbf{aane}-me. \\
people-\textsc{nom=dm} there toward go.\textsc{pl-pfv-nmlz} Jori-\textsc{loc}  exist-\textsc{nmlz} \\
\glt ‘People who went over there, stayed there…’  \\
(\cite{Guerrero2019}; HVF: 57) \\

\ex \label{Guerrero26c}
\gll Si	kaa	wakas-Ø	aman \textbf{aayuk}. \\
\textsc{int} \textsc{neg} meat-\textsc{nom}	there exist.\textsc{inanim.pfv} \\
\glt ‘There is no meat there.’    \\
(\cite{Buitimea2007}; suawaka: 98) \\

\ex \label{Guerrero26d}
\gll ’um  itom hiak bwia-po goi\_naiki pueplo-m \textbf{aayuk}. \\
here \textsc{1pl.poss} Yaqui land-\textsc{loc} two-four town-\textsc{pl} exist.\textsc{inanim.pfv}\\
\glt ‘The eight towns are here in our Yaqui land.’     \\
(\cite{FelixND}; HVC: 44) \\
\z
\z

\newpage
In fact, the alleged “existential” verbs in Yaqui are used with both discourse-new (\ref{Guerrero27a}) and discourse-known entities (\ref{Guerrero27b}--\ref{Guerrero27c}); note that in (\ref{Guerrero27d}), there is a first person pronoun functioning as the located/existing figure. Given the nature of oral narratives, these verbs are more frequent with known participants. In this respect, there seems to be no basis for categorizing them as existential.

\ea \label{Guerrero27}
\ea \label{Guerrero27a}
\gll Kaa=ne inepo-la im jo’ak, aman wate-Ø \textbf{aane}. \\
\textsc{neg=1sg.nom} \textsc{1sg.nom-emph} here live.\textsc{pfv} there	others-\textsc{nom} exist.\textsc{anim} \\
\glt ‘I don’t live alone here, there are others over there.’    \\
(\cite{SilvaAlvarezBuitimea1998}; toad: 20) 



\ex \label{Guerrero27b}
\gll Ian tajti junum \textbf{aayuk} nu’u-Ø Ten\_Jawe-Ø. \\
now	until there exist.\textsc{inanim.pfv} \textsc{det-nom} Ten\_Jawe-\textsc{nom} \\
\glt ‘Until now it has been there, the Ten Jawe mountain.’   \\
(\cite{SilvaAlvarezBuitimea1998}; surem: 111) \\

\ex \label{Guerrero27c}
\gll Em achai Konichukui-Ø ket ama \textbf{aane-k}. \\
2\textsc{sg.pos} father	Konichukui-\textsc{nom}	also there exist.\textsc{anim-pfv}	 \\
\glt ‘Your father Konichukui was also there.’    \\
(\cite{Buitimea2007}; nassuawao: 24) \\

\ex \label{Guerrero27d}
\gll In huubi temai-ne inim ne \textbf{aane}-tiu-ne. \\
\textsc{1sg.pos} wife notify-\textsc{pot} here	\textsc{1sg.nom} exist.\textsc{anim}-tell-\textsc{pot}\\
\glt ‘Tell my wife that I am here.’     \\
(\cite{Johnson1962}; yoooni: 55) \\

\z
\z

With respect to the coding of the ground, in most cases, the location is explicit and coded as a deictic demonstrative, a phrase marked by locative postpositions, or a combination of both. The latter situation is the most common in Yaqui narratives (\ref{Guerrero28a}). In addition to physical location, the expression of temporal location is also very common in Yaqui narratives (\ref{Guerrero28b}).

\ea \label{Guerrero28}
\ea \label{Guerrero28a}
\gll Junama	teebat-po into kaa bu’u	na’asom ama \textbf{ayuk-an}. \\
there	ground-\textsc{loc} \textsc{dm} \textsc{neg} big orange.tree.\textsc{pl} \textsc{dem.ds.l} exist.\textsc{inanim-ipfv} \\
\glt ‘There in the ground there are no big orange trees.’     \\
(\cite{Buitimea2007}; yo’otui: 18) \\

\ex \label{Guerrero28b}
\gll Beja	goi 	ta’a-po 	into 	goi 	tuuka-po aman 	ae-mak 	\textbf{ane-ka.} \\
\textsc{dm}	two	day-\textsc{loc}	and	two	night-\textsc{loc} there	3\textsc{sg.obl}-with	exist.\textsc{anim-ipfv}   \\
\glt ‘He [the child] was with him [Suawaka] over there two days and 
		two nights.’  \\
(\cite{SilvaAlvarezBuitimea1998}; suawaka: 72) \\

\z
\z

Importantly, ‘general-existential’ readings (e.g., \textit{There are books}) arise rarely in examples such as  (\ref{Guerrero29}), where there is no specific location. In other words, locative constructions in Yaqui can be understood as simply ‘exist’ or ‘exist at an unknown place’ when the ground is left implicit.

\ea \label{Guerrero29}
\ea \label{Guerrero29a}
\gll \textbf{Aane} juna-me yoawa emo’i’i’a-me. \\
exist.\textsc{anim} \textsc{dem-pl}	yoawa 	bad-\textsc{nmlz} \\
\glt 	‘There are bad \textit{yoawas}.’     \\
(\cite{Buitimea2007}; suawaka: 141) \\

\ex \label{Guerrero29b}
\gll ¿Jai=saka	ket 	into 	\textbf{aane}	ju-me	yoawa-m 	emo’i’i’ame? – Jeewi,	\textbf{aane}	juna-me 	yoawa	emo’i’i’ame. \\
how=\textsc{emph} also	\textsc{dm} exist.\textsc{anim}	\textsc{det-pl}	yoawa-\textsc{pl}	bad.\textsc{pl} \textsc{} yes	exist.\textsc{anim}	\textsc{dem-pl}	yoawa	bad.\textsc{pl}   \\
\glt ‘Do bad \textit{yoawas} really also exist? - Yes, bad \textit{yoawas} also exist.’   \\
(\cite{Buitimea2007}; jaboi: 140--141) \\

\z
\z

The same construction is used discourse-initially for ‘existential-presentative’ functions. The clause in (\ref{Guerrero30a}) is the first line of a story and introduces the ‘existence’ of animals in God’s kingdom; (\ref{Guerrero30b}) shows an alternative coding with a ‘dwell’ stative verb in the same discourse context.

\ea \label{Guerrero30}
\ea \label{Guerrero30a}
\gll Dios-ta    reino-po         si      	bu’u   	animaal-im  	ama  	\textbf{aane-n}. \\
God-\textsc{acc} kingdom-\textsc{loc} \textsc{int}	big	animal-\textsc{pl}	there		exist.\textsc{anim-ipfv}		 \\
\glt 	‘In God’s kingdom there were a lot of big animals.’      \\
(\cite{SilvaAlvarezBuitimea1998}; gato montes: 1) \\

\ex \label{Guerrero30b}
\gll Sestul	ta’a-po 	kaa 	mekka	junum 	batwe-wi bwe’u 	bakot 	\textbf{jiapsi-su-k}	bwe’u	wojo’o-ku.  \\
one 	day-\textsc{loc} \textsc{neg}	far	over.there	river-\textsc{dir} big	snake	live-\textsc{cmpl-pfv}	big	hole-\textsc{loc}  \\
\glt ‘One day, over there not far from the river, a big snake lived inside a big hole.’   \\
(\cite{Buitimea2007}; suakawa: 1) \\


\z
\z

Consequently, there is no syntactic nor pragmatic evidence at this point for a dedicated existential or locative-existential construction in Yaqui, but only for locative constructions asserting the location of the figure with respect to the ground, i.e., \textit{aane} and \textit{aayuk} should be considered another type of general locative verb.

\section{Final comments} \label{FinalComments}

One of the main goals of this paper was to establish the inventory of locative and existential verbs expressing figure-ground relationships in Southern Uto-Aztecan languages. I have shown that languages of this family can be arranged along a continuum with respect to the number of contrastive verbs linking the figure and the ground (\tabref{tab:GuerreroTable8}).

\begin{table}
    \begin{tabularx}{\textwidth}{lXXXXr}
    \midrule
    \multicolumn{1}{l}{\textsc{Single-set}} & \multicolumn{1}{c}{\textsc{} \Leftarrow \textsc{} \Leftarrow \textsc{}} & \multicolumn{2}{c}{\textsc{Small-set}} & \multicolumn{1}{c}{\Rightarrow \textsc{} \Rightarrow \textsc{}} & \multicolumn{1}{r}{\textsc{Intermediate-set}}\\
    \midrule
    Mexicanero & \multicolumn{2}{l}{Huastec Nahuatl} & \multicolumn{2}{l}{Tarahumara} & \multicolumn{1}{r}{Yaqui} \\
    \multicolumn{2}{l}{Hidalgo Nahuatl} & & \multicolumn{2}{l}{Guarijio} & \\
    \multicolumn{1}{r}{Mayo} & & & \multicolumn{2}{l}{Lower Pima} & \\
    & & & \multicolumn{2}{l}{Northern Tepehuan} &\\
    & & & \multicolumn{2}{l}{Southern Tepehuan} &\\
    & & & \multicolumn{2}{l}{Cora} &\\
    & & & \multicolumn{2}{l}{Huichol} & \\
    \midrule
    \end{tabularx}
    \caption{Continuum of locative predication in Southern Uto-Aztecan languages}
    \label{tab:GuerreroTable8}
\end{table}

At one end, Mexicanero and Hidalgo Nahuatl languages are of the single-set language type since, in the analyzed corpus, they use one general locative verb which is neutral for posture and spatial disposition. The ‘be located’ verbs \textit{ka, onka, onkak} are historically derived from a postural ‘sit’ verb. Huastec Nahuatl seems to use more than one general locative verb (e.g., other locative copulas or postural verbs), but more data is needed. Mayo deserves special attention since it deviates from other Taracahita languages when selecting a general ‘be located’ verb over postural verbs. Still, Mayo and Nahuatl languages may use postural verbs as residual verbs in some context, e.g., to locate animate entities. 
In the middle of the continuum are Tarahumara, Guarijio, Lower Pima, Northern Tepehuan, Southern Tepehuan, Huichol, and Cora, which are of the small-set language type since they use 4--8 contrastive locative/existential verbs that are sensitive to the semantic properties of the figure. If available, these languages may use an existential verb in certain contexts, e.g., when translating Spanish existential clauses and when coding negative location. 
At the other end of the continuum is Yaqui, which uses around 20 locative/existential verbs. In that sense, Yaqui is better characterized as an intermediate-set language. Furthermore, Yaqui ‘existential’ verbs can be further analyzed as general locatives since they do not form existential or locative-existential constructions. Further studies may reveal that other Southern Uto-Aztecan languages can use also positional verbs to express elaborated locative situations.

Finally, previous studies have observed that the domains of location and existence can be co-expressed by the same predicates or morphosyntactic constructions in the languages of the world (\cite{Clark1978}; \cite{Hengeveld1992}; \cite{Koch2012}; \cite{ChappellLü2022}; \cite{Däbritz2023}). Yaqui, and presumably all other Southern Uto-Aztecan languages as well, does provide evidence for such co-expression patterns since locative/existential verbs (and structures) encode the domain of location, which can be understood as locative-existential in certain contexts.

\section*{Acknowledgements}
I would like to thank the anonymous reviewers and editors for their helpful comments and suggestions on earlier versions of this paper. All remaining errors are exclusively my responsibility.

This study was partially supported by the grant ``La codificación del espacio (y otras nociones cognitivas y/o estructuralmente relacionadas) en lenguas yutoaztecas sureñas: estudio descriptivo, comparativo y tipológico'' (UNAM-DGAPA-PAPIIT IG400225).

 
\section*{Abbreviations}

\begin{tabularx}{.45\textwidth}{lQ}
\textsc{a} & cross-reference set A (‘ergative’)\\
\textsc{adv.t} & temporal adverb\\
\textsc{anim} & animate\\
\textsc{as} & assertion\\
\textsc{aum} & augmentative\\
\textsc{b} & cross-reference set B (‘absolutive’)\\
\textsc{d2} & deictic\\
\textsc{dim} & diminutive\\
\end{tabularx}
\begin{tabularx}{.45\textwidth}{lQ}
\textsc{dir} & directional\\
\textsc{dist} & distal\\
\textsc{disp.s} & dispositional stative\\
\textsc{dm} & discourse marker\\
\textsc{emph} & emphatic\\
\textsc{evid.ind} & indicative evidential\\
\textsc{ext} & extension\\
\textsc{inanim} & inanimate\\
\textsc{iness} & inessive\\
\textsc{int} & intensifier\\
\end{tabularx}

\begin{tabularx}{.45\textwidth}{lQ}
\textsc{locc} & contact locative\\
\textsc{mir} & mirative\\
\textsc{narr} & narrative\\
\textsc{pastc} & continuous past\\
\textsc{prep} & preposition\\
\textsc{red} & reduplication\\
\end{tabularx}
\begin{tabularx}{.45\textwidth}{lQ}
\textsc{rn} & relational noun\\
\textsc{sop} & supported\\
\textsc{sta} & stative\\
\textsc{vis} & visible, speaker’s area\\
\textsc{vblz} & verbalizer\\
\textsc{vl} & volume\\
\end{tabularx}

\sloppy
\printbibliography[heading=subbibliography,notkeyword=this]
\end{document}
