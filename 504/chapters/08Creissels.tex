\documentclass[output=paper,colorlinks,citecolor=brown]{langscibook}
\ChapterDOI{10.5281/zenodo.16838070}
\author{Denis Creissels\affiliation{Université Lumière (Lyon 2)}}
\title[‘Be/have’ verbs in historical perspective]{‘Be/have’ verbs in historical perspective}
\abstract{It is cross-linguistically common that verbs whose use in possessive clauses is comparable to that of English \textit{have} are also used impersonally in existential and inverse-locational clauses, but in most of the languages in which a ‘have’ verb also acts as an existential predicator, its possible functions do not include that of copula in plain-locational clauses. The term ‘be/have’ verb refers to verbs involved in a much less common configuration, in which a verb found in possessive clauses like English \textit{have} is also found not only in inverse-locational and existential clauses, but also in the role of copula in plain-locational clauses. After sketching a typology of the verbs projecting possessive clauses and discussing the distribution of ‘be/have’ verbs in the world’s languages, this article analyzes the scenarios that may lead to the emergence of ‘be/have’ verbs in the history of a language.}


\IfFileExists{../localcommands.tex}{
   \addbibresource{../localbibliography.bib}
   % add all extra packages you need to load to this file

\usepackage{tabularx,multicol}
\usepackage{url}
\urlstyle{same}

\usepackage{listings}
\lstset{basicstyle=\ttfamily,tabsize=2,breaklines=true}

\usepackage{langsci-basic}
\usepackage{langsci-optional}
\usepackage{langsci-lgr}
\usepackage{langsci-osl}
% \usepackage{./langsci/styles/langsci-lgr}
% \usepackage{./langsci/styles/langsci-osl}
% \usepackage{langsci-gb4e}

\usepackage{tikz}
\usetikzlibrary{patterns,calc}
\pgfdeclarepatternformonly{south east lines}{\pgfqpoint{-0pt}{-0pt}}{\pgfqpoint{3pt}{3pt}}{\pgfqpoint{3pt}{3pt}}{
    \pgfsetlinewidth{0.6pt}
    \pgfpathmoveto{\pgfqpoint{0pt}{3pt}}
    \pgfpathlineto{\pgfqpoint{3pt}{0pt}}
    \pgfpathmoveto{\pgfqpoint{.2pt}{-.2pt}}
    \pgfpathlineto{\pgfqpoint{-.2pt}{.2pt}}
    \pgfpathmoveto{\pgfqpoint{3.2pt}{2.8pt}}
    \pgfpathlineto{\pgfqpoint{2.8pt}{3.2pt}}
    \pgfusepath{stroke}}
    
\usepackage{stmaryrd}
\usepackage{wasysym}
\usepackage{multirow}
\usepackage{caption}
\usepackage{subcaption}
\usepackage{mathrsfs}
\usepackage{qtree}

\usepackage{linguex}


   %pminos do not split footnotes
% \interfootnotelinepenalty=10000 %Footnote in Laporte chapters has to be split SN


%\DeclareIndexNameFormat{default}{%
%\nameparts{#1}%
%\usebibmacro{index:name}%
%{\index[names]}%
%{\namepartfamily}%
%{\namepartgiveni}%
% {}% L1
% {}% L2
%{\namepartprefix}% generates spurious space L3
%{\namepartsuffix}% generates spurious space L4
%}

%  {\DeclareIndexNameFormat{default}{%
%     \usebibmacro{index:name}{\index[names]}{#1}{#3}{#5}{#7}}}

%\DeclareIndexNameFormat{default}{%
%  \usebibmacro{index:name}{\sindex[nom]}{#1}{#3}{#5}{#7}}

%\DeclareIndexNameFormat{default}{%
%  \usebibmacro{index:name}{\sindex[person]}{#1}{#3}{#5}{#7}}
%\DeclareIndexNameFormat{default}{%
%\nameparts{#1} \usebibmacro{index:name}{\sindex[person]]}{\namepartfamily}{‌​\namepartgiven}{\nam‌​epartprefix}{\namepa‌​rtsuffix}}

%\newcommand{\smiley}{:)}

%\renewbibmacro*{index:name}[5]{%
%\usebibmacro{index:entry}{#1}%
%{\iffieldundef{usera}{}{\thefield{usera}\actualoperator}\mkbibindexname{#2}{#3}{#4}{#5}}}

% \newcommand{\noop}[1]{}

%remove for final
%\overfullrule=1mm

\newcommand{\tobi}[2]}}
\renewcommand{\S}[1]{\tobi{#1}{\textsc{*}}}

% this volume references
% puts: [this volume]
% already defined: \citetv
%\newcommand{\citepv}[1]{(\citeauthor{#1} \citeyear*{#1} [this volume])}
\newcommand{\citealtv}[1]{\citeauthor{#1} \citeyear*{#1} [this volume]}

%parentheses around example number
\newcommand{\pref}[1]{(\ref{#1})}

% in-text examples

\newcommand{\lnex}[1]{\textit{#1}} %target lang word
\newcommand{\lnlit}[1]{(lit.: `#1')} %literal reading
\newcommand{\lnlat}[1]{(#1)} % latinization
\newcommand{\lntrans}[1]{`#1'} %translation
\newcommand{\lnexl}[2]%
{\lnex{#1}{} \lnlat{#2}} % ex with latinization
\newcommand{\lnexlat}[3]{\lnex{#1}{} \lnlat{#2}{} \lntrans{#3}} % ex with latinization and tranl.

%ch01
\newcommand{\co}[1]{\mbox{\textbf{#1}}}

%ch09

\newcommand{\cyrbulg}[1]{\begin{otherlanguage*}{bulgarian}#1\end{otherlanguage*}}


%ch10
\newcommand{\nlp}{{\small NLP}}
\newcommand{\mwe}{{\small MWE}}
\newcommand{\rae}{{\small RAE}}
\newcommand{\lvc}{{\small LVC}}
\newcommand{\pos}{{\small P}o{\small S}}
%\newcommand{\todo}[1]{ \textcolor{red}{#1} }

%\renewcommand{\labelenumi}{\theenumi}
%\ainamefmt{{vv}{ll}{, ff}{, jj}} % fullname

\newcommand{\biberror}[1]{{\color{red}#1}}

\newcommand{\osenovaitem}{--~}
   %% hyphenation points for line breaks
%% Normally, automatic hyphenation in LaTeX is very good
%% If a word is mis-hyphenated, add it to this file
%%
%% add information to TeX file before \begin{document} with:
%% %% hyphenation points for line breaks
%% Normally, automatic hyphenation in LaTeX is very good
%% If a word is mis-hyphenated, add it to this file
%%
%% add information to TeX file before \begin{document} with:
%% %% hyphenation points for line breaks
%% Normally, automatic hyphenation in LaTeX is very good
%% If a word is mis-hyphenated, add it to this file
%%
%% add information to TeX file before \begin{document} with:
%% \include{localhyphenation}
\hyphenation{
    Beck-man
    Ngu-yen
    back-chan-nel
    back-chan-nels
    mo-not-o-nous
    ste-reo-typ-i-cal
}

\hyphenation{
    Beck-man
    Ngu-yen
    back-chan-nel
    back-chan-nels
    mo-not-o-nous
    ste-reo-typ-i-cal
}

\hyphenation{
    Beck-man
    Ngu-yen
    back-chan-nel
    back-chan-nels
    mo-not-o-nous
    ste-reo-typ-i-cal
}

   \boolfalse{bookcompile}
   \togglepaper[8]%%chapternumber
}{}

%\pretocmd{\gll}{\def\eachwordone{\itshape}\def\eachwordtwo{\normalfont}}{}{}

\begin{document}
\maketitle

\section{Introduction} \label{Introduction-Cre}

The term \textsc{‘be/have' verb} is the term I propose for verbs that have the ability to act not only as ‘have’ verbs in possessive clauses (and as existential predicators), but also as copulas in plain-locational predication, i.e. in clauses denoting the spatial relationship figure-ground with the perspectivization ‘from figure to ground’, such as English \textit{John is in his office}. 

It is cross-linguistically common that, as discussed in \citet{Creissels2023}, verbs whose use in possessive clauses is comparable to that of English \textit{have} are also used impersonally in existential clauses (i.e. in clauses in which an entity of any kind is just characterized as constituting an element of some not overtly specified situation, such as English \textit{There are two ways of doing that}) and in inverse-locational clauses (i.e. in clauses that describe the spatial relationship between a figure and a ground with the perspectivization ‘from ground to figure’, such as English \textit{There is a cat in the tree}, cf. \cite{Creissels2019}).\footnote{\citet{Däbritz2025Have} addresses this question for Siberian Uralic languages.}  However, in most of the languages in which a ‘have’ verb also acts as an existential predicator, its possible functions do not include that of copula in plain-locational clauses (i.e. in clauses that describe the spatial relationship between a figure and a ground with the perspectivization ‘from figure to ground’, such as English \textit{The cat is in the tree}). Example (\ref{Creissels1}) illustrates this cross-linguistically common type of alignment between possessive predication involving a `have' verb, existential predication and locative predication.\footnote{The mention ‘personal documentation’ refers to data collected by the author on poorly documented or undocumented languages on which fieldwork was carried out, or to data taken from various sources other than descriptive grammars, or constructed according to the indications given by descriptive grammars. In these cases, the examples have been checked with the help of native speakers. For the examples taken from sources in which they are already provided with glosses, the glosses have been adapted by the author for the sake of consistency.} 

\ea \label{Creissels1}
\langinfo{Bulgarian}{Indo-European, [bulg1262]}{personal documentation}\\
\ea \label{Creissels1a}
\gll  Sestra mi \textbf{ima} kotka. \\
 sister \textsc{i}\textsubscript{\textsc{adp}}:\textsc{1sg} \textbf{have.\textsc{prs.\textsc{i}\textsubscript{s/a}:3sg}} cat\\
\trans ‘My sister has a cat.’ (possessive clause)\\
\ex \label{Creissels1b}
\gll\textbf{Ima} kotka pod masa-ta. \\
\textbf{have.\textsc{prs.\textsc{i}\textsubscript{s/a}:3sg}\textsubscript{\textsc{expl}}}	cat	under	table-\textsc{d}\\
         \trans ‘There is a cat under the table.’ (inverse-locational clause)\\
\ex \label{Creissels1c}
\gll Kotka-ta e pod masa-ta.\\
cat-D \textbf{be.\textsc{prs.\textsc{i}\textsubscript{s/a}:3sg}}	under table-\textsc{d} \\
        \trans ‘The cat is under the table.’ (plain-locational clause) \\
\z
\z

Example (\ref{Creissels2}) illustrates the same configuration in a West-African language.

\ea \label{Creissels2}
\langinfo{Wolof}{Atlantic-Congo, [nucl1347]}{personal documentation}\\
\ea \label{Creissels2a}
\gll  Astu	\textbf{am} {na}	muus. \\
\textsc{prn} \textbf{have} 	\textbf{\textsc{prf.\textsc{i}\textsubscript{s/a}:3sg}}	cat(clM)\\
\trans ‘Astou has a cat.’ (possessive clause)\\

\ex \label{Creissels2b}
\gll \textbf{Am}	\textbf{na}	a-m	muus	\textit{ci suufu}	taabal	b-i. \\
\textbf{have} \textbf{\textsc{prf.\textsc{i}\textsubscript{s/a}:3sg}\textsubscript{\textsc{expl}}}	\textsc{indf}-clM cat(clM)	under	table(clB)	clB-\textsc{d}\\
\trans ‘There is a cat under the table.’ (inverse-locational clause)\\

\ex \label{Creissels2c}
\gll Muus m-i \textbf{mu} \textbf{ngi} \textit{ci suufu} taabal b-i.\\
cat(clM) clM-D \textbf{\textsc{i}\textsc{\textsubscript{s/a}}:\textsc{3sg}} \textbf{\textsc{loc.cop}} under table(clB) clB-\textsc{d}\\
\trans ‘The cat is under the table.’ (plain-locational clause) \\
\z
\z

Example (\ref{Creissels3}) from the Na-Qiangic language Qiang (Sino-Tibetan, [qian1264]) illustrates a much less common configuration, in which a verb whose use in possessive clauses is similar to that of English \textit{have} can also be found not only in inverse-locational and existential clauses, but also in the role of copula in plain-locational clauses. Qiang is a verb-final language with a grammatical relation ‘subject’ conflating transitive A and intransitive S, manifested in obligatory indexation of subjects on the verb, and no flagging of either subjects or objects. As can be seen in (\ref{Creissels3}), in the plain-locational clauses of Puxi Qiang involving \textit{zə} in the role of copula, the subject represents the figure (hence the presence of the first person form \textit{zəʴ} in (\ref{Creissels3a})), whereas in the possessive use of the same verb, the subject represents the possessor (hence the presence of the same first person form \textit{zəʴ} in (\ref{Creissels3c})). In other words, in Puxi Qiang, the same verb occurs in possessive clauses in which it behaves like verbs commonly identified as ‘have’ verbs in language descriptions, and in plain locational clauses in which it behaves like verbs commonly designated as ‘be’ verbs. In other words, Puxi Qiang \textit{zə} is a ‘be/have’ verb.

\ea \label{Creissels3}
\langinfo{Puxi Qiang}{}{\cite[93--94]{Huang2004}}\\
\ea \label{Creissels3a}
\gll  Ŋa tso \textbf{zəʴ}. \\
1\textsc{sg.top} here \textbf{be/have.\textsc{i}\textsubscript{\textsc{s/a}}:1}\\
\trans ‘I am here.’ (plain-locational clause)\\

\ex \label{Creissels3b}
\gll Tɕi ȿkueȿkue-ta dzua \textbf{zə}. \\
house	around-\textsc{loc}	army	\textbf{be/have.\textsc{i}\textsc{\textsubscript{s/a}}:3}\\
\trans ‘There is a team of soldiers around the house.’ (inverse-locational clause)\\

\ex \label{Creissels3c}
\gll Ŋa tsutsu a-la \textbf{zəʴ}.\\
1\textsc{sg.top}	younger.brother	one-\textsc{clf}	\textbf{be/have.\textsc{i}\textsc{\textsubscript{s/a}}:1}\\
\trans ‘I have a younger brother.’ (possessive clause) \\
\z
\z

In this article, after sketching a typology of the verbs that have the ability to act as the predicative nucleus of possessive clauses (\sectref{VerbsOfPossession}) and summarizing the data I have been able to gather about the distribution of the configuration illustrated by example (\ref{Creissels3}) in the world’s languages (\sectref{BeHaveVerbsWorld}), I discuss data suggesting possible scenarios that may lead to the emergence of this configuration in the history of a language (\sectref{PossibleOrigins}). \sectref{Conclusion} summarizes the main conclusions.

\section{Verbs of possession, ‘have’ verbs and ‘be/have’ verbs} \label{VerbsOfPossession}
\subsection{Predicative possession between verbal and non-verbal predication} \label{PredicativePossessionBetween}

Two types of predicative possession can be recognized, depending on the perspectivization of the possessive relationship: either from possessor to possessee (English \textit{I have a book}) or from possessee to possessor (English \textit{The book is mine}). This article deals exclusively with predicative possession constructions expressing the perspectivization ‘from possessor to possessee’.

Building on \citet{Hengeveld1992} and \citet{Stassen1997}, I adopt the definition of non-verbal predication constructions as constructions giving rise to non-elliptical clauses analyzable as consisting of an argument phrase and a predicate phrase in which \textsc{the property- or relation-denoting element that acts as the semantic nucleus of the predicate phrase is not a verb}.  Note that this definition leaves open the possibility that clauses analyzable as instances of non-verbal predication include a verb acting as a copula, i.e. playing a role in the expression of TAM categories (and possibly other distinctions encoded by means of verbal inflection), but not in the identification of the semantic roles expressed by the nominal terms of the clause. 

Within this approach to non-verbal predication, even if some functional types of predication are more commonly expressed via constructions meeting the definition of non-verbal predication than others, functional types of predication cannot be a priori classified as being verbal or non-verbal in nature. In particular, predicative possession is typically a functional domain characterized by competition between verbal and non-verbal predication, and the data provided by \citet{Stassen2009} show that there is no obvious unbalance in the cross-linguistic distribution of the verbal and non-verbal types of predicative possession. 

 \subsubsection{Types of possessive clauses analyzable as instances of non-verbal predication} \label{TypesOfPossessiveNonVerbal}
 Example (\ref{Creissels4}) illustrates a type of possessive clauses in which a proprietive noun or adjective glossable as ‘provided with possessee’ acts as a predicate assigning the role of possessor to an unflagged noun phrase that constitutes its argument.
 
\ea \label{Creissels4}
\langinfo {Classical Nahuatl}{Uto-Aztecan, [clas1250]}{\cite{Launey1981}}\\
\gll Ca tlāca-huâ in tēuctli.\\
\textsc{asrt} man-\textsc{propr} \textsc{d} lord\\
\trans ‘The lord has slaves.' \\
\z

In the type illustrated by example (\ref{Creissels5}), an adpositional phrase or case-marked noun phrase glossable as ‘in possession of N’ acts as a predicate assigning the role of possessor to an unflagged noun phrase that constitutes its argument.

\ea \label{Creissels5}
\langinfo {Hausa}{Afroasiatic, [haus1257]}{\cite[222]{Newman2000}}\\
\gll Yaaròo yanàa dà fensìr̃. \\
boy \textsc{i}\textsubscript{\textsc{s/a}}:3\textsc{sg.m}.\textsc{loc.cop} with pencil \\
\trans ‘The boy has a pencil.’\\
\z

In the type illustrated by example (\ref{Creissels6}), an adpositional phrase or case-marked noun phrase glossable as ‘in N’s personal sphere’ acts as a predicate assigning the role of possessee to an unflagged noun phrase that constitutes its argument.

\ea \label{Creissels6}
\langinfo {Estonian}{Uralic, [esto1258]}{\cite[258]{EreltMetslang2006}}\\
\gll Mu-l oled sina.\\
1\textsc{sg-adess} be.\textsc{prs.i\textsubscript{s/a}:2sg}	2\textsc{sg} \\
\trans ‘I have you.’ \\
\z

In the type illustrated by example (\ref{Creissels7}), a possessive-marked noun phrase combines with an existential predicator into a predicate glossable as ‘X’s possessee exists’ that assigns the role of possessor to a noun phrase that constitues its argument. What is crucial for the possessive interpretation of the construction is not the existential predicator, but the possessive marking of the possessee, and it is interesting to observe that some languages have the same type of construction without any overt existential predicator, as in (\ref{Creissels8}).

\ea \label{Creissels7}
\langinfo {Turkish}{Turkic, [nucl1301]}{personal documentation}\\
\gll Murat-ın otomobil-i var.\\
\textsc{prn-gen} car-\textsc{i\textsubscript{adp}:3sg} \textsc{exist} \\
\trans ‘Murat has a car.’\\
\z

\ea \label{Creissels8}
\langinfo {Emerillon}{Tupian, [emer1243]}{\cite[325]{Rose2002}}\\
\gll e-kija\\
\textsc{i\textsubscript{adp}:1sg}-hammock \\
\trans ‘my hammock’ (noun phrase) or ‘I have a hammock’ (clause) \\
\z

\subsubsection{Types of possessive clauses analyzable as instances of verbal predication} \label{TypesOfPossessiveVerbal}

The distinctive property of the possessive clauses analyzable as instances of verbal predication is that the possessive interpretation is not determined by the coding of the nominal terms but by the lexical meaning of a verb acting as the predicative nucleus of the clause. 
	The verbs having the ability of acting as the predicative nucleus of possessive clauses may be monovalent denominal verbs (proprietive verbs), assigning to the sole core nominal term of the clause the role of possessor of an entity belonging to the category denoted by the noun from which they derive, as in (\ref{Creissels9}). Note that the same gloss \textsc{propr} is used for affixes deriving proprietive nouns or adjectives, as in (\ref{Creissels4}) above, and for affixes deriving proprietive verbs, as in (\ref{Creissels9}).\footnote{Diachronically, a relationship between transitive ‘have’ verbs and proprietive verbs is plausible, since in a language in which a transitive ‘have’ verb has the ability to incorporate its P, the generalization of the incorporating construction may lead to the obsolescence of the original transitive construction, which automatically converts the former ‘have’ verb into an affix deriving proprietive verbs from nouns.} 

\ea \label{Creissels9}
\langinfo {Kalaallisut}{Eskimo-Aleut, [kala1399]}{\cite[25]{VanGeenhoven1998}}\\
\gll Angut taana illu-qar-puq.\\
man that house-\textsc{propr-ind.i\textsubscript{s}:3sg} \\
\trans ‘That man has a house.’\\
\z

However, the verbs acting as the predicative nucleus of possessive clauses are more commonly bivalent verbs that assign the roles of possessor and possessee to two nominal terms of the clauses they project. A distinction can be made between those in which the possessee is expressed as the A term of a transitive construction or the S term of an intransitive construction (i.e., the subject, in the languages that have a grammatical relation ‘subject’), and those in which the A term of a transitive construction or the S term of an intransitive construction expresses the role of possessor.
	
The first possibility is sporadically attested among the indigenous languages of South America with the possessee and the possessor encoded as the A and P terms of a transitive construction. It is also found in Northwest Caucasian and Kartvelian languages with the possessee encoded as the S term of an intransitive construction and the possessor encoded as an indirect object. Most of the verbs projecting possessive clauses whose A or S term expresses the role of possessee derive from an existential or locational verb via applicative derivation. Such possessive clauses can consequently be explained as meaning literally ‘Possessee exists for Possessor’. For example, \citet{OverallForthcoming} shows that Chicham / Jivaroan languages have transitive possessive clauses of this type, as in (\ref{Creissels10b}), to be compared with the intransitive inverse-locational clause (\ref{Creissels10a}).

\ea \label{Creissels10}
\langinfo{Wampis}{Chicham, [huam1247]}{\cite[521, 759]{Peña2015}, quoted by \cite{OverallForthcoming}}\\
\ea \label{Creissels10a}
\glll  {Aánman}  {nápi}  {áwai}.\\
aa=numa=ni napi a-ua-i\\
outside=\textsc{loc=all} snake exist.\textsc{ipfv-i\textsubscript{s}:3-decl}\\
\trans ‘There is a snake (somewhere) outside.’\\
\ex \label{Creissels10b}
\glll  {Arútam}  {míɲa}  {arútɨawɛɛ}.\\
arutam mi=na a-ru-tu-a-ua-i\\
power.vision 1\textsc{sg=acc} exist-\textsc{appl-i\textsubscript{p}:1sg-ipfv-i\textsubscript{a}:3-decl}\\
\trans ‘I have an Arutam power.’\\
\z
\z

Transitive possessive clauses of the same type as (\ref{Creissels10b}) have also been signaled in some varieties of Quechua (\cite{Myler2016}) and in Mataguayan languages (\cite{VidalNercesianForthcoming}).

Example (\ref{Creissels11}) illustrates a possessive verb deriving from a ‘be’ verb via applicativization and projecting clauses in which the possessee is expressed as S, and the possessor as an indirect object.

\ea \label{Creissels11}
\langinfo {West Circassian}{Abkhaz-Adyge, [adyg1241]}{\cite[887]{ArkadievEtAl2024}}\\
\gll W-jane-w-jate-xe-r	w-jə-ʔe-x-a?\\
\textsc{i\textsubscript{adp}:2sg}-mother-\textsc{i\textsubscript{adp}:2sg}-father-\textsc{pl-abs}	\textsc{i\textsubscript{io}:2sg-appl}-be-\textsc{pl-q}\\
\trans ‘Do you have parents?’\\
\z

Example (\ref{Creissels12}) illustrates a similar construction, in which, however, the verb stems glossed ‘be in the sphere of’ cannot be decomposed morphologically. Note that the choice between the two verbs of possession of Georgian illustrated in (\ref{Creissels12}) depends on animacy of the possessee.

\ea \label{Creissels12}
\langinfo {Georgian}{Kartvelian, [nucl1302]}{personal documentation}\\
\ea \label{Creissels12a}
\gll Vano-s axali megobari hq’avs.\\
Vano-\textsc{dat} new friend be.in.the.sphere.of.\textsc{prs.i\textsubscript{s}:3sg.i\textsubscript{io}:3sg}\\
\trans ‘Vano has a new friend.’\\
\ex \label{Creissels12b}
\gll Vano-s axali saxli akvs.\\
Vano-\textsc{dat} new house be.in.the.sphere.of.\textsc{prs.i\textsubscript{s}:3sg.i\textsubscript{io}:3sg}\\
\trans ‘Vano has a new house.’\\
\z
\z

In contrast to the scarcity of the constructions of the type illustrated in examples (\ref{Creissels10}) to (\ref{Creissels12}), clauses projected by verbs of possession in which the A term of a transitive construction or the S term of an intransitive construction expresses the role of possessor, discussed in more detail in \sectref{HaveVerbs}, are extremely common cross-linguistically. Alongside with the type illustrated in (\ref{Creissels6}) above (characterized by adjunct-like coding of the possessor), they constitute in fact one of the two major types of predicative possession in terms of distribution across geographical areas and language families.

\subsection{‘Have’ verbs} \label{HaveVerbs}

According to the broad definition adopted in this article, a ‘have’ verb is a semantically bivalent verb projecting possessive clauses in which the possessor is coded like the agent of prototypical transitive verbs (A) or like the sole essential participant of a semantically monovalent verb (S).\footnote{For a discussion of the notion of essential participant and for details about the approach to S, A, and P underlying this paper, readers are referred to \citet[chapter 1]{Creissels2024a}. Suffice it to mention here that this approach is in line with what \citet{Haspelmath2011} characterizes as the Comrian paradigm.}  In the languages that have a grammatical relation ‘subject’ encompassing the agent of prototypical transitive verbs and the sole essential participant of semantically monovalent verbs, ‘have’ verbs can equally be characterized as bivalent possessive verbs whose subject represents the possessor. 
	Cross-linguistically, the vast majority of the verbs that meet this definition of ‘have’ verbs select a coding frame identical to that of prototypical transitive verbs, and consequently meet the narrower definition of ‘have’ verbs found in the literature (including \cite{Creissels2023}), according to which transitivity is a definitional feature of ‘have’ verbs. For example, in Mandinka, the verb \textit{sòtó} ‘have’ projects clauses expressing a variety of possessive relationships comparable to that expressed by English \textit{have}, and, as shown in (\ref{Creissels13}), its coding frame is fully aligned with that of a prototypical transitive verb such as \textit{tábì} ‘cook’. 

\ea \label{Creissels13}
\langinfo {Mandinka}{Mande, [mand1436]}{personal documentation}\\
\ea \label{Creissels13a}
\gll Fàatú yè kín-òo tábí kèê-lú yè.\\
\textsc{prn} \textsc{compl.tr} meal-\textsc{d} cook man.\textsc{d-pl} for\\
\trans ‘Fatou cooked the meal for the men.’ (prototypical transitive clause)\\
\ex \label{Creissels13b}
\gll Fàatú yè báadíŋ-ò-lú sòtó \~nĭŋ sàatêe tó.	\\
\textsc{prn} \textsc{compl.tr}	relative-\textsc{d-pl}	have	\textsc{dem}	village.\textsc{d}	\textsc{loc}\\
\trans ‘Fatou has relatives in this village.’ (possessive clause)\\
\z
\z

Example (\ref{Creissels14b}) illustrates a clause projected by a transitive ‘have’ verb in a language in which, as can be seen from (\ref{Creissels14a}), the transitive construction shows a contrast between ergative-marked A and zero-marked P.

\ea \label{Creissels14}
\langinfo {Central Basque}{isolate, [basq1248]}{personal documentation}\\
\ea \label{Creissels14a}
\gll Jon-ek bi etxe egin zituen.\\
\textsc{prn-erg} two house make.\textsc{compl} \textsc{aux.i\textsubscript{zer}:3pl.i\textsubscript{erg}:3sg}\\
\trans ‘Jon has built two houses.’ \\
\ex \label{Creissels14b}
\gll Jon-ek bi seme dauzka.	\\
\textsc{prn-erg} two	son	have.\textsc{i\textsubscript{zer}:3pl.i\textsubscript{erg}:3sg}\\
\trans ‘Jon has two sons.’ \\
\z
\z

However, the broad definition of ‘have’ verbs formulated above also encompasses verbs selecting coding frames that variously depart from the transitive construction, but sharing with transitive ‘have’ verbs the essential property of projecting possessive clauses in which the A-like or S-like coding of the possessor characterizes it as the participant having the highest degree of inherent topicality.

Verbs that meet the broad definition of ‘have’ verbs but whose coding frame variously departs from the transitive construction are attested in quite a few languages. However, their status is problematic in a general typology of predicative possession excluding them from the notion of ‘have’ verbs, since the constructions in which they are involved do not meet the definition of any of the other types of predicative possession commonly recognized. The possibility of integrating them into a general typology of predicative possession is therefore an obvious advantage of the broad definition of ‘have’ verbs adopted in this article.
	
 Arabic dialects are a case in point. In Classical Arabic and Modern Standard Arabic, the standard expression of predicative possession is a construction whose literal equivalent in English would be ‘At Possessor is Possessee’, but this situation has not been maintained in Arabic dialects. As discussed in more details in \citet{Creissels2022}, most vernacular varieties of Arabic have a predicative possession construction involving a word belonging to a category for which the label ‘pseudo-verb’ is commonly used in descriptions of Arabic dialects, and this possessive (pseudo-)verb meets the broad definition of ‘have’ verbs but not the narrow one. 
	
 In vernacular Arabic varieties, pseudo-verbs are words of non-verbal origin that have acquired uses in which they can be analyzed synchronically as projecting clauses, and in which they have acquired some properties typical of verbs in Arabic morphosyntax, in particular in the expression of negation. For example, Moroccan Arabic has a pseudo-verb \textit{ʔand} ‘have’ cognate with the preposition \textit{ʔand} ‘at’, but synchronically distinct from it in its syntactic and semantic properties. As illustrated in (\ref{Creissels15}), \textit{ʔand} ‘have’, contrary to \textit{Ɂand} ‘at’, obligatorily combines with a person-number-gender suffix indexing the possessor, but cannot be followed by a noun phrase expressing the semantic role of possessor. A possessor noun phrase can only precede \textit{ʔand}, whereas the possessee is expressed as an unflagged noun phrase that follows \textit{ʔand}. 


 \ea \label{Creissels15}
\langinfo {Moroccan Arabic}{Afroasiatic, [moro1292]}{\cite[51--52]{Caubet1993}}\\
\gll Ḥmәd ʕand-u әl-ktāb.\\
\textsc{prn} have-\textsc{i:3sg.m} book\\
\trans ‘Ahmed has the book.’ \\
\z

The coding frame of Moroccan \textit{ʔand} ‘have’ is similar to the basic transitive construction as regards constituent order and the obligatory indexation of the participant that can be expressed as a pre-verbal noun phrase. However, it cannot be analyzed as an instance of the transitive construction for the following two reasons: the paradigm of obligatory indexes suffixed to \textit{ʔand} is distinct from the standard paradigm of subject indexes and more similar (although not identical) to the paradigm of object indexes (the explanation being that it originates historically from the paradigm of indexes representing the complement of prepositions), and, contrary to the object of transitive verbs, the participant encoded as a postverbal noun phrase in possessive clauses cannot be alternatively represented by an index suffixed to the verb.

\subsection{The historical origin of ‘have’ verbs} \label{HistoricOriginHave}

A major source of ‘have’ verbs is the emergence of ‘have’ verbs as the result of a process of semantic bleaching affecting transitive verbs expressing meanings such as ‘take’, ‘grasp’, ‘hold’, ‘get’, ‘bear’ (\cite[637--659]{Creissels1979}, \cite[103]{Givon1984}, \cite[48]{Heine1997}, \cite[63]{Stassen2009}). This evolution is widely attested not only in various branches of the Indo-European family, but also in many other language families all around the world.\footnote{Interestingly, ‘have’ verbs resulting from this kind of semantic shift are particularly common in pidgins and creoles, cf. \citet{MichaelisEtAl2013}.}

A second source of ‘have’ verbs is the widespread tendency of predicative possession constructions originally belonging to other types to acquire coding properties making them similar to a ‘have’ construction. As rightly observed by \citet[208--243]{Stassen2009}, this kind of evolution, commonly designated as \textsc{have-drift}, may affect constructions that originally belong to any other type of predicative possession, and is in fact a major source of typological change in the way languages express predicative possession, cf. also \citet{Creissels2023}. The evolution of predicative possession in Arabic, evoked in \sectref{HaveVerbs}, illustrates this possibility.

A third possible source of ‘have’ verbs is the lexicalization of the applicative form of locational or existential verbs. As already mentioned in \sectref{HaveVerbs}, the applicativization of locational or existential verbs may yield verbs of possession with which the possessor is coded as the P term of a transitive construction. However, due to the remarkable semantic plasticity of applicativization, the applicativization of verbs expressing location, posture or existence may also yield ‘have’ verbs with which the possessor is coded as the A term of a transitive construction or the S term of an intransitive construction. This possibility can be illustrated by the Australian language Diyari, where a verb still identifiable formally as the applicative form of ‘sit’ has become a ‘have’ verb, cf. example (\ref{Creissels16}). This is an instance of lexicalization in the sense that, synchronically, the construction illustrated in (\ref{Creissels16}) can no longer be analyzed as an applicative construction, since the S of the base verb expresses the role of person who sits, whereas the A of the derived verb expresses the role of possessor.

 \ea \label{Creissels16}
\langinfo {Diyari}{Pama-Nyungan, [dier1241]}{\cite[402]{Austin2024}}\\
\gll Yundru karna tharla ngama-lka-yi?\\
\textsc{2sg.erg} person name.\textsc{acc} sit-\textsc{appl-prs}\\
\trans ‘Do you have an Aboriginal name?’  \\
\z

\subsection{‘Be/have’ verbs} \label{BeHaveVerbs}

\textsc{‘be/have’ verbs} have been defined in \sectref{Introduction-Cre} as verbs that have the ability to act not only as ‘have’ verbs in possessive clauses and as existential predicators, but also as copulas in plain-locational predication, i.e. in clauses denoting the spatial relationship figure-ground with the perspectivization ‘from figure to ground’, such as English \textit{John \textbf{is} in his office}. This configuration, already illustrated in the introduction by an example from the Na-Qiangic language Qiang, is also found in Indonesian (\ref{Creissels17}).

\ea \label{Creissels17}
\langinfo{Indonesian}{Austronesian, [indo1316]}{\cite[264]{Sneddon1996}}\\
\ea \label{Creissels17a}
\gll Saya tidak \textbf{ada} uang.\\
\textsc{1sg} \textsc{neg} be/have money\\
\trans ‘I don’t have any money.’ (possessive predication)\\

\ex \label{Creissels17b}
\gll Di Indonesia tidak \textbf{ada} kanguru. \\
in \textsc{prn}	\textsc{neg} be/have kangaroo\\
\trans ‘In Indonesia there are no kangaroos.’ (inverse-locational predication)\\

\ex \label{Creissels17c}
\gll Ayah tidak \textbf{ada}	di kantor.\\
father \textsc{neg}	be/have	in office\\
\trans ‘Father isn’t in the office.’ (plain-locational predication)\\
\z
\z

Some of the verbs that meet this definition of ‘be/have’ verbs also have a copular use in nominal and/or adjectival predication (i.e. in clauses such as \textit{John \textbf{is} a teacher, John \textbf{is} tall}). This is not the case in Qiang or in Indonesian, but example (\ref{Creissels18}) illustrates such a configuration in the Bantu language Kikuyu, an AVP/SV language in which A in transitive predication and S in intransitive predication are obligatorily indexed by means of the same set of verbal prefixes.\footnote{The role played by the expletive S/A index of class 16 (etymologically a locative index) in the inverse-locational clause (\ref{Creissels18b}) is comparable to that of \textit{there} in the English equivalent of this clause.}

\ea \label{Creissels18}
\langinfo{Kikuyu}{Atlantic-Congo, [kiku1240]}{\cite[86, 89, 93]{LiNavarro2015}}\\
\ea \label{Creissels18a}
\gll \textbf{Tũ-rĩ} n-gari.\\
\textbf{\textsc{i\textsubscript{s/a}:1pl}-be/have}	\textsc{sg}-car(cl9)\\
\trans ‘We have a car.’ (possessive predication)\\

\ex \label{Creissels18b}
\gll \textbf{Ha-rĩ} benjũ metha-inĩ. \\
\textbf{\textsc{i\textsubscript{s/a}}:cl16\textsc{\textsubscript{expl}}-be/have}	\textsc{sg}.pencil(cl9)	\textsc{sg}.table(cl9)-\textsc{loc}\\
\trans ‘There is a pencil on the table.’ (inverse-locational predication)\\

\ex \label{Creissels18c}
\gll I-bera \textbf{rĩ-rĩ} gĩ-kombe-inĩ.\\
\textsc{sg}-pear(cl5)	\textbf{\textsc{i\textsubscript{s/a}}:cl5-be/have} \textsc{sg}-cup(cl7)-\textsc{loc}	\\
\trans ‘The pear is in the cup.’ (plain-locational predication)\\

\ex \label{Creissels18d}
\gll \textbf{Mũ-ti-rĩ} arimũ. \\
\textbf{\textsc{i\textsubscript{s/a}}:2\textsc{pl-neg}-be/have} \textsc{pl}.teachers(cl2)\\
\trans ‘You are not teachers.’ (nominal predication)\\

\ex \label{Creissels18e}
\gll \textbf{Tũ-rĩ} a-rũaru.\\
\textbf{\textsc{i\textsubscript{s/a}}:1\textsc{pl}-be/have} cl2-sick\\
\trans ‘We are sick.’ (adjectival predication)\\
\z
\z

\section{‘Be/have’ verbs in the world’s languages} \label{BeHaveVerbsWorld}

Two hotspots of ‘be/have’ verbs can be identified: Mainland South East Asia in the first place, and to a lesser extent the Ghana-Togo region in West Africa. The other languages for which I have been able to find mentions of the existence of a ‘be/have’ verb show no genetic or areal clustering. However, in the absence of a more systematic investigation, I cannot exclude that this might be due to accidental gaps in the data.

\subsection{‘Be/have’ verbs in Mainland South East Asia} \label{BeHaveVerbsMainland}

‘Be/have’ verbs are particularly prominent in Mainland South East Asia. In this area, the data analyzed by \citet{ChappellCreissels2019} and \citet{ChappellLü2022} show the existence of ‘be/have’ verbs in the following languages:

\begin{itemize}
    \item one Hmongic language (Yanghao, [nort2747]),
    \item two Austroasiatic languages (Bugan, [buga1247] and Mang [mang1378]),
    \item among Sino-Tibetan languages, Jingpho (Brahmaputran [jing1260]), Tujia (unclassified Sino-Tibetan [tuji1244]), several languages belonging to the Burmese-Lolo, Na-Qiangic and Karenic branches of Sino-Tibetan, three Sinitic languages (Haikou Southern Min [hain1238], Linxia [hez1244] and Dabu Hakka [hakk1236]), and four varieties of Bai (unclassified Sino-Tibetan, [baic1239]).
\end{itemize}

In Bai languages (and only in Bai languages) the ‘be/have’ verb is also used as a copula in equative clauses (nominal predication).

\subsection{‘Be/have’ verbs in the Ghana-Togo region} \label{BeHaveVerbsGhanaTogo}

In West Africa, ‘be/have’ verbs are attested in several languages of the Ghana-Togo region belonging to the Kwa and Gur subfamilies of the Atlantic-Congo family: 

\begin{itemize}
    \item Akan (Tano, Kwa, Atlantic-Congo, [akan1250];  \cite{Boadi1971},  \cite{ReddenOwusu1995})
    \item Efutu (Tano, Kwa, Atlantic-Congo, [efut1241]; \cite{Agyeman2016}),
    \item Gikyode (Tano, Kwa, Atlantic-Congo, [giky1238]; Gikyode dictionary online),\footnote{\url{https://www.webonary.org/gikyode/browse}, accessed 10 April 2024.}
    \item Gonja (Tano, Kwa, Atlantic-Congo, [gonj1241]; Gonja dictionary online),\footnote{\url{https://www.webonary.org/gonja/browse}, accessed 10 April 2024.}
    \item Nkami (Tano, Kwa, Atlantic-Congo, [nkam1239]; \cite{Asante2017}),
    \item Nkonya (Tano, Kwa, Atlantic-Congo, [nkon1248]; \cite{Reineke1972} and Nkonya dictionary online),\footnote{\url{https://www.webonary.org/nkonya/browse}, accessed 10 April 2024.}
    \item Likpe (Na-Togo, Kwa, Atlantic-Congo, [sekp1241]; \cite{Ameka2007}, \cite{Ameka2009} and personal communication),
    \item Tuwuli (Ka-Togo, Kwa, Atlantic-Congo, [tuwu1238]; \cite{Harley2005}),
    \item Lama (Gurunsi, Gur, Atlantic-Congo, [lama1275]; \cite{Simnara2019}).
\end{itemize}

Tano languages are particularly well represented among the languages of the Ghana-Togo region that have ‘be/have’ verbs. Moreover, with the exception of Akan, all the Tano languages in this list belong to the Guang subgroup, which apparently shows the highest concentration of languages with ‘be/have’ verbs in this area.

\subsection{‘Be/have’ verbs in other parts of the world} \label{BeHaveVerbsOtherParts}

The other languages for which I have been able to find mentions of the existence of a ‘be/have’ verb show no areal or genetic clustering:

\begin{itemize}
    \item Indonesian (Malayo-Sumbawan; \cite{Sneddon1996}),
    \item Cheke Holo (Oceanic, Austronesian, [chek1238]; \cite{Boswell2018}),
    \item Diu Indo-Portuguese (Portuguese-based creole, [dama1278]; \cite{Cardoso2009}),
    \item Gulf Pidgin Arabic (Arabic-based pidgin, [pidg1248]; \cite{Bakir2014}),
    \item Français Tirailleur (a French-based pidgin that was used as a lingua franca by West African soldiers and their white officers in the French colonial army, [fran1267]; \cite{Skirkgård2013}),
    \item Iatmul (Ndu, Sepik, [iatm1242]; \cite{Jendraschek2012}),
    \item Kikuyu (Bantu, Benue-Congo; \cite{LiNavarro2015}),
    \item some varieties of ǃXun (Kx’a, [juho1239]; \cite{HeineKönig2015}).
\end{itemize}

\section{Possible origins of ‘be/have’ verbs} \label{PossibleOrigins}

For some of the languages mentioned in \sectref{BeHaveVerbsWorld}, the data at my disposal are not suggestive of any particular historical scenario for the emergence of ‘be/have’ verbs. In particular, all that can be said about the Tano languages listed in \sectref{BeHaveVerbsGhanaTogo} is that their ‘be/have’ verbs are certainly reflexes of the same root (they all consist of a labial consonant \textit{b} or \textit{w} and a back vowel \textit{ʊ, o} or \textit{ɔ}). In the Tano languages that do not have a ‘be/have’ verb, a reflex of the same root is found as a locational copula, e.g. Baule (Central Tano, Atlantic-Congo, [baou1238]) \textit{wo} ‘be at’, Nawuri (Guang, Atlantic-Congo, [nawu1242]) \textit{bʊ} ‘be at’, and at least some of them have a ‘have’ verb with a transparent etymology. For example Chumburung (Guang, Atlantic-Congo, [chum1261]) \textit{de} ‘have’ is also used with the meaning ‘take’. Although such observations are not fully conclusive, they at least leave open the possibility that a ‘be/have’ verb was already present in the ancestor of Tano languages, and lost its possessive use in some of them due to the development of alternative ways of expressing predicative possession.
	
For some others of the languages mentioned in \sectref{BeHaveVerbsWorld}, however, there is more or less solid evidence that the ‘be/have’ verb results from the evolution of a word that was originally either a ‘have’ verb (\sectref{HaveVerbBeHaveVerb}), a locational copula (\sectref{LocationalCopulaAtPisP} and \sectref{LocationalCopulaPiswithP}), an existential predicator (\sectref{ExPredBeHave}), or the applicative form of an existential verb (\sectref{ApplicativeForm}).

\subsection{‘Have’ verb > ‘be/have’ verb} \label{HaveVerbBeHaveVerb}
	
In some of the South East Asian languages in which the data analyzed by \citet{ChappellCreissels2019} and \citet{ChappellLü2022} show the existence of a ‘be/have’ verb, the verb in question also has transitive uses with meanings such as ‘take’. This is in particular the case for the Puxi Qiang ‘be/have’ verb \textit{zә} illustrated in example (\ref{Creissels3}) above. 
	
Since it is difficult to imagine the direct conversion of a ‘take’ verb into a locational copula, whereas ‘have’ verbs resulting from the semantic bleaching of ‘take’ verbs are cross-linguistically common, the reasonable hypothesis in such a configuration is the following three-stage evolution: 
 
\begin{itemize}
    \item a ‘take’ verb was first converted into a ‘have’ verb;
    \item subsequently, the ‘have’ verb acquired the possibility of being used as an existential predicator and as a copula in inverse-locational predication;
    \item finally its copular use extended to all locational clauses, irrespective of the distinction between plain- and inverse-locational predication.
\end{itemize}

The first stage of this evolution is a scenario particularly well-attested cross-linguistically, in particular (but not only) in various branches of Indo-European, cf. for example Indo-European \textit{*gʰabʰ} ‘seize, take’ > Latin \textit{habēre} ‘have’.

The second stage is also a well-attested type of historical change, analyzed in detail in \citet{Creissels2023}, cf. for example Old Spanish (Indo-European, [olds1249]) \textit{haber} ‘have’ > Modern Spanish (Indo-European, [stan1288]) \textit{haber} ‘there be’.

The third stage, by which an existential predicator extends its use from inverse-locational clauses to plain-locational clauses, seems to be less common in the history of languages. However, some clear cases of such an evolution can be found in the world’s languages, as in Arabic-based Pidgins or Creoles, and in some Turkic languages.
	
Many vernacular Arabic varieties have an existential predicator \textit{fī(h)}, etymologically ‘in it’. As a rule,\textit{ fī(h)} occurs only in existential and inverse-locational clauses, but the extension of its use to plain-locational clauses, illustrated in (\ref{Creissels19}), is one of the features that characterize the pidginized/creolized varieties  of Arabic spoken in Sudan.\footnote{Note that example (\ref{Creissels19}) includes both the preposition \textit{fi} ‘in’ and the predicator \textit{fí} whose historical origin is an inflected form of the same preposition meaning ‘in it’.}

\ea \label{Creissels19}
\langinfo {Sudanese Arabic-based pidgins/creoles}{}{\cite[32]{Miller2002}}\\
\gll Úwo fí fi bét.\\
3\textsc{sg} \textsc{loc.cop}	in	house\\
\trans ‘(S)he is at home.’ \\
\z

This evolution also occurred in the Turkic languages of Northeast Siberia (Sakha and Dolgan). Most Turkic languages have an existential predicator (Turkish \textit{var}), reconstructed as \textit{*bār} (with the same meaning) in Proto-Turkic. In most Turkic languages (including Turkish), the reflexes of \textit{*bār} are used in inverse-locational predication, but not in plain-locational predication, cf. (\ref{Creissels20}). However, as illustrated in (\ref{Creissels21}) for Sakha, in Sakha and Dolgan, this existential predicator has extended its use to plain-locational predication, resulting in locational clauses in which the contrast between the two possible perspectivizations of the ground-figure relationship is manifested in constituent order only.

\ea \label{Creissels20}
\langinfo{Turkish}{}{personal knowledge}\\
\ea \label{Creissels20a}
\gll Kedi bahçe-de.\\
cat	garden-\textsc{loc}\\
\trans ‘The cat is in the garden.’ (plain-locational predication)\\

\ex \label{Creissels20b}
\gll Bahçe-de bir kedi var. \\
garden-\textsc{loc}	one	cat	\textsc{exist}\\
\trans ‘There is a cat in the garden.’ (inverse-locational predication)\\
\z
\z

\ea \label{Creissels21}
\langinfo{Sakha}{Turkic, [yaku1245]}{\cite[16]{Xaritonov1987}}\\
\ea \label{Creissels21a}
\gll Xarandaas ostuol-ga baar.\\
pencil table-\textsc{loc}	\textsc{loc.cop}\\
\trans ‘The pencil is on the table.’ (plain-locational predication)\\

\ex \label{Creissels21b}
\gll Ostuol-ga xarandaas baar. \\
table-\textsc{loc} pencil \textsc{loc.cop}\\
\trans ‘There is a pencil on the table.’ (inverse-locational predication)\\
\z
\z

\subsection{Locational copula > ‘be/have’ verb as the result of changes affecting a possessive construction of the type ‘At Possessor is Possessee’} \label{LocationalCopulaAtPisP}

In some of the South East Asian languages for which the data analyzed by \citet{ChappellCreissels2019} and \citet{ChappellLü2022} show the existence of a ‘be/have’ verb, the verb in question is also attested as an intransitive posture verb, or as an intransitive verb whose meaning implies location, such as ‘dwell’ or ‘stick’. For example, Puxi Qiang has been mentioned above as having a ‘be/have’ verb illustrating the evolutionary path ‘take’ > ‘have’ > ‘there be’ > ‘be at’, but in the same Na-Qiangic group of Sino-Tibetan languages, Naxi has a ‘be/have’ verb \textit{ʑi\textsuperscript{33}} possibly cognate with \textit{ʑi\textsuperscript{55}} ‘lie’.


\ea \label{Creissels22}
\langinfo{Naxi}{Sino-Tibetan, [naxi1245]}{\cite[39--40]{ChappellLü2022}}\\
\ea \label{Creissels22a}
\gll Zue\textsuperscript{21}	tsuɑ\textsuperscript{33}=kv̩\textsuperscript{33} tʰe\textsuperscript{21} \textbf{ʑi\textsuperscript{55}}	jɤ\textsuperscript{33}.\\
child bed=on \textsc{dur} \textbf{lie} \textsc{sens}\\
\trans ‘The child is lying/sleeping in the bed.’\\

\ex \label{Creissels22b}
\gll ɳɯ\textsuperscript{21} ŋɤ\textsuperscript{33} ŋgɤ\textsuperscript{33} nv̩\textsuperscript{55}me\textsuperscript{33}=lø\textsuperscript{21}	tʰe\textsuperscript{21}	tɑ\textsuperscript{55}	\textbf{ʑi\textsuperscript{33}}.\\
2\textsc{sg} 1\textsc{sg} \textsc{gen} heart=in	\textsc{dur} always	\textbf{be/have}\\
\trans ‘You’ve always been in my heart.’ (plain-locational clause)\\

\ex \label{Creissels22c}
\gll Ze\textsuperscript{21}kʰø\textsuperscript{33} ʈʂʰɭ\textsuperscript{33} kʰø\textsuperscript{33}	bv̩\textsuperscript{21} se\textsuperscript{21},	ɲɟi\textsuperscript{21}	mɤ\textsuperscript{33}	\textbf{ʑi\textsuperscript{33}}.	\\
well this \textsc{clf} be.dry	\textsc{pfv} water	\textsc{neg} \textbf{be/have}	\\
\trans ‘The well is dry and there’s no water (inside).’ (existential clause)\\

\ex \label{Creissels22d}
\gll Çi\textsuperscript{33}	ʈʂʰɭ\textsuperscript{33} kv̩\textsuperscript{55} ciɤ\textsuperscript{55}	ŋgy\textsuperscript{33},	ti\textsuperscript{55}we\textsuperscript{55}	ŋgy\textsuperscript{33}	se\textsuperscript{21}me\textsuperscript{33}, pe\textsuperscript{33}sɿ\textsuperscript{55} \textbf{ʑi\textsuperscript{33}} mɤ\textsuperscript{55}sɿ\textsuperscript{33}. \\
person this	\textsc{clf} money have social.status have besides capacity	\textbf{be/have}	\textsc{prt}\\
\trans ‘He’s not only rich and of high social status,
but he has the capacity as well.’ (possessive clause)\\
\z
\z


Since the grammaticalization of posture verbs and of verbs expressing meanings such as ‘remain’ as locational copulas is a widely-attested phenomenon in the history of languages, it seems plausible that the first stage in the evolution leading to the emergence of a ‘be/have’ verb in the languages that have a configuration of the type illustrated in (\ref{Creissels22}) was the grammaticalization of intransitive posture verbs (or of other intransitive verbs whose meaning implies location) as locational copulas via semantic bleaching. 

Given that possessive clauses of the type ‘At Possessor is Possessee’ constitute a widespread type of predicative possession, in particular among Tibeto-Burman languages, the second stage of the evolution was probably the development of the use of the locational copula in possessive clauses of the type ‘At Possessor is Possessee’.
	
Finally, as discussed in detail by \citet[§5.3]{Creissels2023}, it is not uncommon that possessive predication constructions of the type ‘At Possessor is Possessee’ have a variant expressing topicalization of the possessor, in which the left-dislocated possessor phrase loses the adjunct-like flagging that characterizes it in the absence of topicalization, yielding constructions glossable as ‘(As for) Possessor, there is Possessee’ whose generalization may lead to the reanalysis of the construction with an unflagged possessor phrase as a construction in which the existential predicator acts as a ‘have’ verb. \citet{ChappellCreissels2019} show that Burmese (Sino-Tibetan [nucl1310]) can be analyzed as attesting a transitional stage in this evolution.

 \subsection{Locational copula > ‘be/have’ verb as the result of changes affecting a possessive construction of the type ‘Possessor is with Possessee’} \label{LocationalCopulaPiswithP}
 \subsubsection{The case of Kikuyu}

 Kikuyu has a comitative preposition \textit{na} ‘with’, a regular reflex of the Bantu comitative preposition \textit{na}, illustrated in example (\ref{Creissels23}).

\ea \label{Creissels23}
\langinfo{Kikuyu}{}{\cite[110]{LiNavarro2015}} \\

\gll Mũ-timia nĩ-a-ra-cok-ire mũ-ciĩ na ci-ana.\\
\textsc{sg}-woman(cl1)	\textsc{i\textsubscript{s/a}}:cl1-\textsc{nrpst}-return-\textsc{compl}	\textsc{sg}-home(cl3)	with	\textsc{pl}-child(cl8)\\
\trans ‘The woman returned home with the children.’\\
\z

There can be little doubt that the historical change responsible for the emergence of the Kikuyu ‘be/have’ verb \textit{rĩ} illustrated in example (\ref{Creissels18}), reproduced here as (\ref{Creissels24}), is the mere deletion of this comitative preposition in a predicative possession construction ‘NP\textsc{\textsubscript{possessor}} \textit{rĩ na} NP\textsc{\textsubscript{possessee}}’ lit. ‘Possessor is with Possessee’, where\textit{ rĩ} [rɪ] is the regular reflex of the Bantu copula \textit{*dɪ}.

\ea \label{Creissels24}
\langinfo{Kikuyu}{}{\cite[86, 89, 93]{LiNavarro2015}}\\
\ea \label{Creissels24a}
\gll \textbf{Tũ-rĩ} n-gari.\\
\textbf{\textsc{i\textsubscript{s/a}:1pl}-be/have}	\textsc{sg}-car(cl9)\\
\trans ‘We have a car.’ (possessive predication)\\

\ex \label{Creissels24b}
\gll \textbf{Ha-rĩ} benjũ metha-inĩ. \\
\textbf{\textsc{i\textsubscript{s/a}}:cl16\textsc{\textsubscript{expl}}-be/have}	\textsc{sg}.pencil(cl9)	\textsc{sg}.table(cl9)-\textsc{loc}\\
\trans ‘There is a pencil on the table.’ (inverse-locational predication)\\

\ex \label{Creissels24c}
\gll I-bera \textbf{rĩ-rĩ} gĩ-kombe-inĩ.\\
\textsc{sg}-pear(cl5)	\textbf{\textsc{i\textsubscript{s/a}}:cl5-be/have} \textsc{sg}-cup(cl7)-\textsc{loc}	\\
\trans ‘The pear is in the cup.’ (plain-locational predication)\\

\ex \label{Creissels24d}
\gll \textbf{Mũ-ti-rĩ} arimũ. \\
\textbf{\textsc{i\textsubscript{s/a}}:2\textsc{pl-neg}-be/have} \textsc{pl}.teachers(cl2)\\
\trans ‘You are not teachers.’ (nominal predication)\\

\ex \label{Creissels24e}
\gll \textbf{Tũ-rĩ} a-rũaru.\\
\textbf{\textsc{i\textsubscript{s/a}}:1\textsc{pl}-be/have} cl2-sick\\
\trans ‘We are sick.’ (adjectival predication)\\
\z
\z


The comparative data that support this analysis are that, on the one hand, constructions glossable as ‘Possessor is with Possessee’ are by far the most widespread type of predicative possession across the Bantu language family (\cite{Creissels2024b}), and on the other hand, reflexes of \textit{*dɪ} acting as copulas (and only as copulas) are pervasive across the Bantu language family. Moreover, the creation of ‘have’ verbs resulting from the fusion and reanalysis of the sequence ‘copula + comitative preposition’ in possessive constructions of the type ‘Possessor is with Possessee’ is a very common process in Bantu (\cite{Creissels2024b}). In this context, the deletion of the comitative preposition that occurred in Kikuyu (resulting in the creation of a ‘be/have’ verb) can be viewed as a borderline case of this general trend toward converting comitative-possessive constructions into have-possessive constructions.

Moreover, in Kikuyu, the original construction ‘NP\textsc{\textsubscript{possessor}} \textit{rĩ na} NP\textsc{\textsubscript{possessee}}’, where \textit{rĩ} acts as a copula, still exists in competition with the construction ‘NP\textsc{\textsubscript{possessor}} \textit{rĩ} NP\textsc{\textsubscript{possessee}}’, where \textit{rĩ} acts as a ‘have’ verb in a construction at least superficially similar to the transitive construction. I checked this with a corpus of Kikuyu proverbs available on the Internet in which possessive clauses abound,\footnote{\url{https://africanmanners.wordpress.com/2012/07/07/gikuyu-proverbs-1000-in-total}, accessed 10 April 2024.} and it turned out that in this corpus, both possibilities are widely attested. According to \citet[92]{LiNavarro2015}, in the possessive clauses of Kikuyu, “the presence of \textit{na} highlights the immediateness of the possession, so that a possessive clause lacking \textit{na} indicates more permanent possession”, cf. example (\ref{Creissels25}).

\ea \label{Creissels25}
\langinfo {Kikuyu}{}{\cite[92]{LiNavarro2015}}\\
\ea \label{Creissels25a}
\gll N-dĩ na m-buku.\\
\textsc{i\textsubscript{s/a}:1pl}-be/have with \textsc{sg}-book(cl9)\\
\trans ‘I have a book (in my possession at the moment).’ \\
(\textit{ri}̃ as a ‘be’ verb’ in a possessive clause of the be-with type)\\
\ex \label{Creissels25b}
\gll N-dĩ m-buku. \\
\textsc{i\textsubscript{s/a}:1pl}-be/have \textsc{sg}-book(cl9)\\
\trans ‘I own a book.’
(\textit{ri}̃ as a ‘have’ verb’) \\
\z
\z

I am aware of no other Bantu language in which a ‘be/have’ verb would have emerged as the result of the deletion of the comitative preposition in a construction of the type ‘Possessor is with Possessee’.

\subsubsection{The case of Lama} \label{TheCaseOfLama}



Although the available evidence is less conclusive than in the case of Kikuyu, the same historical scenario is probably responsible for the existence of a ‘be/have’ verb \textit{wɛ} in Lama. The use of this verb as a copula in plain-locational predication and as a ‘have’ verb is illustrated in (\ref{Creissels26}).

\newpage
\ea \label{Creissels26}
\langinfo {Lama}{}{\cite[198, 202]{Simnara2019}}\\
\ea \label{Creissels26a}
\gll Lóór nɖə́ cì wɛ́ lɔ́?\\
car \textsc{dem} owner be/have where\\
\trans ‘Where is the owner of this car? (plain-locational clause)\\
\ex \label{Creissels26b}
\gll Mə́ wɛ́ lóór. \\
1\textsc{sg} be/have car\\
\trans‘I have a car.’ (possessive clause) \\
\z
\z

The hypothesis that the possessive use of the ‘be/have’ verb of Lama results from the deletion of a comitative preposition in a construction that originally belonged to the type ‘Possessor is with Possessee’ is supported by data from Tem and Kabiye, two languages very closely related to Lama.\footnote{Lama, Tem and Kabiye are classified in the same subgroup of Gurunsi languages, known as Eastern Gurunsi.}  In both Tem (Atlantic Congo, [temm1241]) and Kabiye (Atlantic Congo, [kabi1261]), a verb \textit{wɛ} formally identical to the ‘be/have’ verb of Lama is attested as a copula but cannot be used as a ‘have’ verb, whereas  predicative possession involves a verb \textit{wɛna} ‘have’ whose obvious etymology is the freezing of the sequence ‘be + with’ in a predicative possession construction of the type ‘Possessor is with Possessee’. 

\subsubsection{The case of Iatmul} \label{TheCaseOfIatmul}

The creation of a ‘be/have’ verb as the consequence of the deletion of the comitative preposition in a predicative possession construction that originally belonged to the type ‘Possessor is with Possessee’ is also the scenario suggested by the data of the closely related Papuan languages Iatmul (\cite{Jendraschek2012}) and Manambu (Ndu, [mana1298]; \cite{Aikhenvald2008}). 

Iatmul and Manambu possessive clauses involve a verb (Iatmul \textit{ti’ $\sim$ li’}, Manambu \textit{tə}) which is basically a verb ‘stay’ used as a locational copula. Judging from the data provided by Aikhenvald, the possessive construction of Manambu can be straightforwardly analyzed as involving a ‘have’ verb, but the situation of Iatmul is more complex, and suggests the same development path as that already proposed for Kikuyu and Lama.
	
In Iatmul, according to Jendraschek, three distinct constructions are possible for possessive clauses. They all involve the verb \textit{ti’ $\sim$ li’} ‘stay’, and no difference in meaning is mentioned between them. The most frequent one, illustrated in (\ref{Creissels27a}), belongs to the type ‘Possessor is with Possessee’. The alternative constructions are a construction that can be glossed as ‘Of Possessor is Possessee’ (\ref{Creissels27b}), and a construction with no flagging of either the possessor or the possessee, in which consequently \textit{li’} can be analyzed as acting as a ‘have’ verb (\ref{Creissels27c}).

 \ea \label{Creissels27}
\langinfo {Iatmul}{}{\cite[215, 216]{Jendraschek2012}}\\
\ea \label{Creissels27a}
\gll Nyaan gusa okwi \textbf{li’-di’}.\\
child paddle with \textbf{be/have-3\textsc{sg.m}}\\
\trans ‘‘The child had a paddle.’ lit. ‘The child stayed with a paddle.’\\
\ex \label{Creissels27b}
\gll Wun-a saanya wugi \textbf{li’-ka}. \\
1\textsc{sg-gen} money that.which \textbf{be/have-\textsc{prs(sr)}}\\
\trans‘I have money.’ lit. ‘Of me money is that which stays.’\\
\ex \label{Creissels27c}
\gll Nyaan gusa	\textbf{li’-di’}. \\
child paddle \textbf{be/have-3\textsc{sg.m}}\\
\trans ‘The child had a paddle.’ lit. ‘The child stayed a paddle.’\\
\z
\z

I am not aware of comparative data that might support a particular scenario, but it seems difficult to imagine a motivation for the addition of a comitative preposition flagging the possessee phrase in the construction of a verb that would have been originally a ‘have’ verb. Moreover, as discussed in detail by \citet{Stassen2009}, the general trend in the structural changes that may affect predicative possession construction is that constructions that do not involve ‘have’ verbs tend to change in a way that makes them more similar to the ‘have’ type of predicative possession, but not the other way round. Consequently, a reasonable hypothesis is that, in the same way as in Kikuyu and in Lama, the construction in which \textit{li’} can be analyzed as acting as a ‘have’ verb resulted from the deletion of the comitative postposition in a possessive construction that was originally ‘Possessor is with Possessee’.

\subsection{Existential predicator > ‘be/have’ verb} \label{ExPredBeHave}

In \sectref{HaveVerbBeHaveVerb} it has been mentioned that, in the Arabic-based pidgins or creoles of Sudan, an existential predicator found as \textit{fī(h)} in many vernacular varieties of Arabic (whose ultimate origin is \textit{fī-hi} ‘in it’) has become a general locational copula marking locational clauses irrespective of the perspectivization of the figure-ground relationship. The same evolution occurred in Gulf Pidgin Arabic, as illustrated in (\ref{Creissels28b}), but in addition to that, \textit{fī} in Gulf Pidgin Arabic has acquired the function of a ‘have’ verb (\ref{Creissels28c}), and also that of an equative copula (\ref{Creissels28d}), resulting in the configuration already mentioned for Kikuyu (cf. \sectref{BeHaveVerbs}) and Bai languages (cf. \sectref{BeHaveVerbsMainland}).

\ea \label{Creissels28}
\langinfo{Gulf Pidgin Arabic}{}{\cite[418]{Bakir2014}}\\
\ea \label{Creissels28a}
\gll \textbf{Fī}	moni mā-fī muškila.\\
\textbf{be/have}	money \textsc{neg}-be/have problem\\
\trans ‘If there is money, there is no problem.’\\

\ex \label{Creissels28b}
\gll Ana bēt \textbf{fī}	wara dukkān. \\
\textsc{1sg} home \textbf{be/have} behind shop\\
\trans ‘My home is behind the shop.’\\

\ex \label{Creissels28c}
\gll Alhīn walla ana \textbf{fī}	talāta arba	baččā. \\
now	by.God \textsc{1sg}	\textbf{be/have}	three four child \\
\trans ‘I swear I have three, four children.’\\

\ex \label{Creissels28d}
\gll Ana \textbf{fī} maskīn sah walla lā? \\
\textsc{1sg} \textbf{be/have} poor right	or no\\
\trans ‘I am a poor fellow, right?’\\
\z
\z

Given that it can be taken for granted that this situation developed from the use of \textit{fī} as an existential predicator, the emergence of a ‘be/have’ verb in Gulf Pidgin Arabic can be analyzed as resulting from the conjunction of two distinct evolutions already discussed in the previous sections. The starting point of both evolutions was the use of \textit{fī} in existential and inverse-locational clauses. On the one hand, in a construction combining existential predication with a topic pragmatically interpreted as denoting a possessor (something like \textit{(As for) X, there is Y} interpreted as ‘X has Y'), the existential predicator has been reanalyzed as a ‘have’ verb. On the other hand, the existential has extended its use from inverse-locational clauses to plain-locational clauses, thus acquiring the status of general locational copula.

The same explanation applies to the distribution of \textit{y(en)a} across equative, locational, existential and possessive clauses in Français Tirailleur (a French-based pidgin that was used as a lingua franca by West African soldiers and their white officers in the French colonial army), whose origin is French \textit{il y (en) a} ‘there is (some)’.

Although the available evidence is less conclusive, comparative data suggest that the emergence of the Indonesian ‘be/have’ verb \textit{ada} might have resulted from the same conjunction of two evolutions extending the use of an existential predicator: on the one hand, emergence of a ‘have’ possessive construction via reanalysis of a construction combining existential predication with a topic interpreted pragmatically as denoting a possessor, and on the other hand, extension of the use of the same existential predicator to plain-locational clauses. The main piece of evidence is that cognates of \textit{ada} are widespread across Western Malayo-Polynesian languages, and by far their most widespread use is that of existential predicator. Additional pieces of evidence are that, on the one hand, the use of \textit{ada} is optional in plain-locational clauses, and on the other hand, in possessive clauses, the\textit{ ada} construction is in competition with a construction involving proprietive verbs derived by means of the prefix \textit{ber}, e.g. \textit{anak} ‘child’ > \textit{ber-anak} ‘have children’.

\subsection{Applicative form of an existential predicator > ‘be/have’ verb} \label{ApplicativeForm}

In general, existential predicators have the ability to occur in the same form in two types of clauses. By definition, existential predicators can be found in existential clauses in the strictest sense of this term, i.e., in clauses giving no overt indication about the situation in which the referent of the argument of the existential predicator can be found. However, existential predicators commonly also occur in clauses in which they combine with a locative expression, the S term in the construction of the existential predicator then fulfilling the semantic role of figure in a ground-figure relationship. 

A remarkable property of the Kx’a language ǃXun is that it has an existential verb \textit{gè} which in its underived form cannot combine with a phrase expressing location. In ǃXun, the use of the existential verb \textit{gè} in locational clauses requires applicative marking. Moreover, as illustrated in (\ref{Creissels29}), in some varieties of ǃXun, the applicative form \textit{gèā $\sim$ gèà} has the ability of projecting clauses that constitute the usual way of expressing not only locational predication, but also predicative possession (\cite[80--84, 233--235]{HeineKönig2015}). (\ref{Creissels29b}) is a plain-locational clause in which the applied phrase expresses the ground in a figure-ground relationship, whereas in (\ref{Creissels29c}), the applied phrase is interpreted as the possessee in a possessor-possessee relationship.

 \ea \label{Creissels29}
\langinfo{ǃXun}{}{\cite[82, 83, 233]{HeineKönig2015}}\\
\ea \label{Creissels29a}
\gll Tsȉrì(-sȉ) rē gè?\\
chair-\textsc{pl} \textsc{q} there.be\\
\trans ‘Are there chairs?’\\

\ex \label{Creissels29b}
\gll  Tsȉrì {m̄} gè-à tc'ū ńǃ{ŋ́}.\\
chair \textsc{top} there.be-\textsc{appl} house	inside\\
\trans ‘The stool is in the house.’\\

\ex \label{Creissels29c}
\gll Nā gè-ā gùmì. \\
\textsc{1sg} there.be-\textsc{appl} cattle	 \\
\trans ‘I have a cow.’\\
\z
\z

A possible explanation for this situation is that the construction in which \textit{gèā $\sim$ gèà} acts as a copula in locational clauses (including plain-locational ones) is a locative applicative (i.e., a construction in which applicative derivation licenses an applied phrase expressing the semantic role of location), whereas the use of \textit{gèā $\sim$ gèà} as a ‘have’ verb results from the lexicalization of a comitative applicative (i.e., a construction in which applicative derivation originally licensed an applied phrase expressing the semantic role of companion, subsequently reinterpreted as expressing the role of possessee).

A possible objection is that, synchronically, \textit{gèā $\sim$ gèà} does not seem to be productively used with the meaning ‘be with’. However, \citet[83]{HeineKönig2015} quote an example from an old description of a ǃXun variety in which \textit{gèā} unquestionably expresses a comitative meaning, which lends some support to this analysis.

\section{Conclusion} \label{Conclusion}


In this article, after sketching a typology of the verbs that have the ability to project possessive clauses and defining the notion of ‘be/have’ verb, I have tried to show that for at least some of the languages with ‘be/have’ verbs, there are comparative data supporting the reconstruction of four possible historical scenarios in which a word that was originally a ‘have’ verb, a locational copula, an existential predicator or the applicative form of an existential verb is converted into a ‘be/have’ verb. 

In the scenario discussed in \sectref{LocationalCopulaPiswithP}, a \textsc{single-step change}  (the deletion of the comitative preposition in a predicative possession construction of the type ‘Possessor is with Possessee’) directly results in the emergence of a ‘be/have’ verb.

The other scenarios discussed in \sectref{PossibleOrigins} are \textsc{complex scenarios} combining two changes which, with the exception of the creation of a locational copula and of a ‘have’ verb via applicativization of an existential verb, are well-attested separately in the history of locational, existential and possessive constructions:

\begin{itemize}
    \item the scenario analyzed in \sectref{HaveVerbBeHaveVerb} combines the acquisition of the function of existential predicator by a have verb and the conversion of an existental predicator into a general locational copula;
    \item the scenario analyzed in \sectref{LocationalCopulaAtPisP} combines the development of a possessive predication construction of the type ‘At Possessor is Possessee’ and the acquisition of transitive features by a possessive predication construction that originally belongs to the type ‘At Possessor is Possessee’ (\textit{have}-drift);
    \item the scenario analyzed in \sectref{ExPredBeHave} combines the conversion of an existential predicator into a general locational copula and the acquisition of transitive features by a possessive predication construction that originally belongs to the type ‘At Possessor is Possessee’ (\textit{have}-drift);
    \item the scenario analyzed in \sectref{ApplicativeForm} combines the creation of a locational copula and of a ‘have’ verb via applicativization of an existential verb in a language where the same applicative marker can be found in locative-applicative and comitative-applicative functions.
\end{itemize}

\largerpage[3]

\section*{Abbreviations}

%von Basile rüberkopiert
\begin{tabularx}{.49\textwidth}{lQ}
\textsc{a} & the nominal term of transitive clauses that represents the agent if the verb projecting the clause is a prototypical transitive verb\\
\textsc{adess} & adessive\\
\textsc{cl} & gender-number agreement pattern (class)\\
\textsc{clf} & classifier\\
\textsc{compl} & completive\\
\textsc{d} & definite determiner or default determiner\\
\textsc{decl} & declarative\\
\textsc{exist} & existential predicator\\
\textsc{expl} & expletive\\
\textsc{i} & index\\
\textsc{i\textsubscript{adp}} & index cross-referencing an adnominal possessor (possessive index)\\
\textsc{i\textsubscript{erg}} & index cross-referencing an ergative-marked term of the clause\\
\textsc{i\textsubscript{io}} & index cross-referencing an indirect object\\
\textsc{i\textsubscript{p}} & index cross-referencing the P term of a transitive clause (object index)\\
\end{tabularx}
\begin{tabularx}{.5\textwidth}{lQ}
\textsc{i\textsubscript{s/a}} & index cross-referencing the S term of a transitive clause or the A term of a transitive clause (subject index)\\
\textsc{i\textsubscript{zer}} & index cross-referencing a zero-marked term of the clause\\
\textsc{loc.cop} & locational copula\\
\textsc{nrpst} & near past\\
\textsc{p} & the nominal term of transitive clauses that represents the patient if the verb projecting the clause is a prototypical transitive verb\\
\textsc{prn} & proper name\\
\textsc{propr} & proprietive\\
\textsc{prt} & discourse particle\\
\textsc{s} & the nominal term of intransitive clauses that represents the sole essential participant if the verb projecting the clause is a semantically monovalent verb\\
\textsc{sens} & sensory\\
\textsc{sr} & subordinator\\
\end{tabularx}


\sloppy
\printbibliography[heading=subbibliography,notkeyword=this]
\end{document}
