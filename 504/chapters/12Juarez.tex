\documentclass[output=paper,colorlinks,citecolor=brown]{langscibook}
\ChapterDOI{10.5281/zenodo.16838078}

\author{Cristian R. Juárez\affiliation{Department of Humanities, Social and Political Sciences - ETH Zürich}}

\title[Multifunctional spatial markers in Mocoví]{Spatial relations and valence extension: Multifunctional spatial markers in Mocoví}
\abstract{This paper analyses the multifunctionality of spatial markers \textit{-lek} `{\LocOne}', \textit{-ɡe} `{\LocTwo}’, \textit{ɡit} `{\LocThree}' and \textit{ɡi} ‘{\LocFour}’  in Mocoví, a Guaycuruan language spoken in the Argentine side of the South American Gran Chaco region. The function of  these spatial markers is explored in the context of existential constructions and different morphologically basic one-place and two-place predicates, as well as morphologically derived intransitivized predicates. The study argues that the functional richness of spatial markers cannot be reduced to
a single major function or label. Instead, it shows that spatial markers contribute
to (i) the relational interaction between the figure and its ground in static and movement contexts, (ii) the indication of certain physical properties of a single argument in state/change-of-state predicates and certain P arguments of a two-place predicate, and (iii), the addition of a non-S or non-A argument.  Importantly, findings from this paper contribute to
Guaycuruan-internal and cross-linguistic studies, highlighting the usually unnoticed link between spatial and applicative morphology.}

\IfFileExists{../localcommands.tex}{
   \addbibresource{../localbibliography.bib}
   % add all extra packages you need to load to this file

\usepackage{tabularx,multicol}
\usepackage{url}
\urlstyle{same}

\usepackage{listings}
\lstset{basicstyle=\ttfamily,tabsize=2,breaklines=true}

\usepackage{langsci-basic}
\usepackage{langsci-optional}
\usepackage{langsci-lgr}
\usepackage{langsci-osl}
% \usepackage{./langsci/styles/langsci-lgr}
% \usepackage{./langsci/styles/langsci-osl}
% \usepackage{langsci-gb4e}

\usepackage{tikz}
\usetikzlibrary{patterns,calc}
\pgfdeclarepatternformonly{south east lines}{\pgfqpoint{-0pt}{-0pt}}{\pgfqpoint{3pt}{3pt}}{\pgfqpoint{3pt}{3pt}}{
    \pgfsetlinewidth{0.6pt}
    \pgfpathmoveto{\pgfqpoint{0pt}{3pt}}
    \pgfpathlineto{\pgfqpoint{3pt}{0pt}}
    \pgfpathmoveto{\pgfqpoint{.2pt}{-.2pt}}
    \pgfpathlineto{\pgfqpoint{-.2pt}{.2pt}}
    \pgfpathmoveto{\pgfqpoint{3.2pt}{2.8pt}}
    \pgfpathlineto{\pgfqpoint{2.8pt}{3.2pt}}
    \pgfusepath{stroke}}
    
\usepackage{stmaryrd}
\usepackage{wasysym}
\usepackage{multirow}
\usepackage{caption}
\usepackage{subcaption}
\usepackage{mathrsfs}
\usepackage{qtree}

\usepackage{linguex}


   %pminos do not split footnotes
% \interfootnotelinepenalty=10000 %Footnote in Laporte chapters has to be split SN


%\DeclareIndexNameFormat{default}{%
%\nameparts{#1}%
%\usebibmacro{index:name}%
%{\index[names]}%
%{\namepartfamily}%
%{\namepartgiveni}%
% {}% L1
% {}% L2
%{\namepartprefix}% generates spurious space L3
%{\namepartsuffix}% generates spurious space L4
%}

%  {\DeclareIndexNameFormat{default}{%
%     \usebibmacro{index:name}{\index[names]}{#1}{#3}{#5}{#7}}}

%\DeclareIndexNameFormat{default}{%
%  \usebibmacro{index:name}{\sindex[nom]}{#1}{#3}{#5}{#7}}

%\DeclareIndexNameFormat{default}{%
%  \usebibmacro{index:name}{\sindex[person]}{#1}{#3}{#5}{#7}}
%\DeclareIndexNameFormat{default}{%
%\nameparts{#1} \usebibmacro{index:name}{\sindex[person]]}{\namepartfamily}{‌​\namepartgiven}{\nam‌​epartprefix}{\namepa‌​rtsuffix}}

%\newcommand{\smiley}{:)}

%\renewbibmacro*{index:name}[5]{%
%\usebibmacro{index:entry}{#1}%
%{\iffieldundef{usera}{}{\thefield{usera}\actualoperator}\mkbibindexname{#2}{#3}{#4}{#5}}}

% \newcommand{\noop}[1]{}

%remove for final
%\overfullrule=1mm

\newcommand{\tobi}[2]}}
\renewcommand{\S}[1]{\tobi{#1}{\textsc{*}}}

% this volume references
% puts: [this volume]
% already defined: \citetv
%\newcommand{\citepv}[1]{(\citeauthor{#1} \citeyear*{#1} [this volume])}
\newcommand{\citealtv}[1]{\citeauthor{#1} \citeyear*{#1} [this volume]}

%parentheses around example number
\newcommand{\pref}[1]{(\ref{#1})}

% in-text examples

\newcommand{\lnex}[1]{\textit{#1}} %target lang word
\newcommand{\lnlit}[1]{(lit.: `#1')} %literal reading
\newcommand{\lnlat}[1]{(#1)} % latinization
\newcommand{\lntrans}[1]{`#1'} %translation
\newcommand{\lnexl}[2]%
{\lnex{#1}{} \lnlat{#2}} % ex with latinization
\newcommand{\lnexlat}[3]{\lnex{#1}{} \lnlat{#2}{} \lntrans{#3}} % ex with latinization and tranl.

%ch01
\newcommand{\co}[1]{\mbox{\textbf{#1}}}

%ch09

\newcommand{\cyrbulg}[1]{\begin{otherlanguage*}{bulgarian}#1\end{otherlanguage*}}


%ch10
\newcommand{\nlp}{{\small NLP}}
\newcommand{\mwe}{{\small MWE}}
\newcommand{\rae}{{\small RAE}}
\newcommand{\lvc}{{\small LVC}}
\newcommand{\pos}{{\small P}o{\small S}}
%\newcommand{\todo}[1]{ \textcolor{red}{#1} }

%\renewcommand{\labelenumi}{\theenumi}
%\ainamefmt{{vv}{ll}{, ff}{, jj}} % fullname

\newcommand{\biberror}[1]{{\color{red}#1}}

\newcommand{\osenovaitem}{--~}
   %% hyphenation points for line breaks
%% Normally, automatic hyphenation in LaTeX is very good
%% If a word is mis-hyphenated, add it to this file
%%
%% add information to TeX file before \begin{document} with:
%% %% hyphenation points for line breaks
%% Normally, automatic hyphenation in LaTeX is very good
%% If a word is mis-hyphenated, add it to this file
%%
%% add information to TeX file before \begin{document} with:
%% %% hyphenation points for line breaks
%% Normally, automatic hyphenation in LaTeX is very good
%% If a word is mis-hyphenated, add it to this file
%%
%% add information to TeX file before \begin{document} with:
%% \include{localhyphenation}
\hyphenation{
    Beck-man
    Ngu-yen
    back-chan-nel
    back-chan-nels
    mo-not-o-nous
    ste-reo-typ-i-cal
}

\hyphenation{
    Beck-man
    Ngu-yen
    back-chan-nel
    back-chan-nels
    mo-not-o-nous
    ste-reo-typ-i-cal
}

\hyphenation{
    Beck-man
    Ngu-yen
    back-chan-nel
    back-chan-nels
    mo-not-o-nous
    ste-reo-typ-i-cal
}

   \boolfalse{bookcompile}
   \togglepaper[12]%%chapternumber
}{}


%%%%%%%%%% MOCOVI GLOSSES%%%%%%%%%%%%%%%%%%%%%%%%%%%%%%%%%%%%
%%%%%%%Special symbols%%%%%%%%
\newleipzig{sploan}{$_{sp}$}{Spanish loanword}
\newleipzig{tbloan}{$_{tb}$}{Toba loanword}
\newleipzig{ungr}{$^{*}$}{ungrammatical}
\newleipzig{clitic}{=}{clitic ligature}
\newleipzig{dirch}{$\rightarrow$}{direction of change}
%%%%%%%Language-specific glosses%%%%%%%%%%%%%%%%%%%%%%%%%%%%%%%%%%%%%%%%%
\newleipzig{coord}{coord}{coordinant}
\newleipzig{link}{link}{linker}
\newleipzig{disc}{disc}{discourse maker}
\newleipzig{tprl}{tprl}{temporal}
\newleipzig{evid}{evid}{evidential}
\newleipzig{trvz}{trvz}{transitivizer} 
\newleipzig{III}{III}{set III bound person form}
\newleipzig{II}{II}{set II bound person form}
\newleipzig{I}{I}{set I bound person form}
\newleipzig{dir}{dir}{directional}
\newleipzig{arg}{arg}{argument}
\newleipzig{defc}{defc}{defocusing}
\newleipzig{dim}{dim}{diminutive}
\newleipzig{class}{class}{classifier}
\newleipzig{act}{act}{action}
\newleipzig{vm}{vm}{valence modifier}
\newleipzig{exist}{exist}{existential}
\newleipzig{noml}{noml}{nominal}
\newleipzig{ints}{ints}{intensifier} 
\newleipzig{rec}{rec}{recipient}
\newleipzig{causee}{causee}{causee argument}
\newleipzig{coll}{coll}{collective}
\newleipzig{pron}{pron}{pronominal}
\newleipzig{des}{des}{desiderative}
\newleipzig{val}{val}{valence}
\newleipzig{exclam}{exclam}{exclamation}
\newleipzig{inter}{inter}{interrogative}
\newleipzig{vrbz}{vrbz}{verbalizer}
\newleipzig{iter}{iter}{iterative}
\newleipzig{ep}{ep}{epenthetic segment}
\newleipzig{indet}{indet}{indeterminate}%
\newleipzig{attrb}{attrb}{attributive}
\newleipzig{ac}{ac}{active}% Grondona gloss
\newleipzig{in}{in}{inactive}% Grondona gloss
%%%%%%%%%%%%LOCATIVES%%%%%%%%%%%%%%%%%%%%%%%%
\newleipzig{locOne}{loc$_{1}$}{locative type 1}%lek
\newleipzig{locTwo}{loc$_{2}$}{locative type 2}%ge
\newleipzig{locThree}{loc$_{3}$}{locative type 3}%git
\newleipzig{locFour}{loc$_{4}$}{locative type 4}%gi
%%%%%%%%%%DIRECTIONALS
\newleipzig{dirDown}{dir$_{down}$}{directional: downwards}%
\newleipzig{dirOut}{dir$_{out}$}{directional: outwards}
\newleipzig{dirIn}{dir$_{in}$}{directional: inwards}
\newleipzig{dirUp}{dir$_{up}$}{directional: upwards}
\newleipzig{out}{out}{outwards}%wek
\newleipzig{down}{down}{downwards}%ɲi
\newleipzig{up}{up}{upwards}%ʃigem 
% %%%%%%%%%DETERMINERS%%%%%%%%%%%%%%%%%%%%%%%%
\newleipzig{detOne}{det$_{1}$}{determiner 1: no visible} %ka
\newleipzig{detTwo}{det$_{2}$}{determiner 2: going, far}%so
\newleipzig{detThree}{det$_{3}$}{determiner 3: coming, close}%na
\newleipzig{detFour}{det$_{4}$}{determiner 4: static, vertical}%ɾa
\newleipzig{detFive}{det$_{5}$}{determiner 5: static, horizontal}%dʒi
\newleipzig{detSix}{det$_{6}$}{determiner type 6: static, sitting }%ɲi
\newleipzig{deicg}{deic$_{(gng)}$}{deictic root: going} %Grondona gloss
\newleipzig{deicnex}{deic$_{(non-exst)}$}{deictic root: non-extended}
\newleipzig{inside}{inside}{locative-directional: inside} %Grondona gloss%Grondona gloss
%%%%%%%%%%%%%%%%%%OTHER
\newleipzig{asp}{asp}{aspect} 
%%%%%%%%%%%%Trees



\begin{document}
\maketitle

\section{Introduction}
In many languages, the expression of a type of locative event, such as `the cat is on the table', is semantically and structurally characterized by a specific locative predicate \citep[e.g., locative or positional, see][]{AmekaLevinson2007,Ameka2007}. Typically, this type of event is described as expressing a spatial relationship between a \textsc{figure} or \textsc{trajector} and the \textsc{ground} or \textsc{landmark} \citep{LevinsonWilkins2006Patterns,Svorou2002}. However, in some languages, the \textsc{figure-ground} relationship is primarily signaled by spatial verbal affixes, i.e. a closed class of grammatical elements characterized as ``verb-satellites'' in Talmy's (\citeyear{talmy2007}) lexical typology.

This paper explores the multifunctionality of four spatial markers -- \textit{-lek} `{\LocOne}', \textit{-ɡe} `{\LocTwo}’, \textit{ɡit} `{\LocThree}' and \textit{ɡi} ‘{\LocFour}’ --  in Mocoví [moco1246], a Guaycuruan language spoken in the Argentine side of the South American Gran Chaco region. It covers the function of these spatial markers in the context of existential constructions and different morphologically basic one-place and two-place predicates, as well as  morphologically derived intransitivized predicates. This study argues that the functional richness of spatial markers cannot be circumscribed to a single major function or label. Instead, it  shows  that spatial markers contribute to:
\begin{enumerate}[label=(\roman*)]
    \item the relational interaction between the \textsc{figure} and its \textsc{ground} in static and movement contexts,
    \item the indication of certain physical properties of a single argument in \newline state/change-of-state predicates and certain P arguments of a two-place predicate,
    \item the addition of a non-S or non-A argument.
\end{enumerate}

Although these three major functions are an oversimplification, the individual study of each spatial marker throughout this paper shows their subtle semantic contributions to the morphosyntactic contexts in which they occur. The novel data presented in this work, from a South American language, contributes to the typology of multifunctional spatial markers, highlighting relevant links between spatial morphology and valence increment. This functional link has been recently acknowledged to be a prosperous typological line of research, though it has been  well know for scholars working on Guaycuruan languages. 

This paper is structured as follows: \sectref{sec:language} provides a brief description of the language and the data employed in this study. \sectref{sec:basics-grammar} outlines basics on the morphosyntactic structure of transitive and intransitive clauses, including an overview of the relevant morphological valence changes. \sectref{sec:mutiple-functions} provides prior works on spatial markers and describes their functions in the range of constructions discussed in this paper. A short discussion is advanced in \sectref{sec:discussion}, addressing the attested patterns of use and the specific contexts in which the applicative function clearly arises. Conclusions and further lines of research are presented in \sectref{sec:conclusions}.


% \is{Cognition} %add "Cogntion" to subject index for this page


\il{Mocoví|(}



\section{The language and data}\label{sec:language}
Mocoví is part of the Guaycuruan family along with other four languages, three of which are still spoken today, i.e., Kadiwéo, Toba and Pilagá. Although the internal organization of this language family has received different representations, all of them agree on a high binary branching, placing Mocoví on the Southern branch, as in \figref{fig:guaicuruanfamily}.


\begin{figure}
\begin{forest} for tree={
    edge path={\noexpand\path[\forestoption{edge}] (\forestOve{\forestove{@parent}}{name}.parent anchor) -- +(0,-12pt)-| (\forestove{name}.child anchor)\forestoption{edge label};}}
[Proto-Guaycuruan
    [{Northern Guaycuruan}
        [\textsuperscript{$\dagger$}Mbayá [Kadiwéo ]]]
    [{Southern Guaycuruan}  
    [\textsuperscript{$\dagger$}Abipón]  
    [Qom   [\textbf{Mocoví}] [Pilagá-Toba [Toba] [Pilagá]]]]]
\end{forest}
\caption{Guaycuruan linguistic family according to Viegas Barros' (\citeyear[26--27]{ViegasBarros2013b}) classification \citep[cf.][]{Ceria1995}.}\label{fig:guaicuruanfamily} 
\end{figure}


Mocoví communities are located in northeastern Argentina, as shown in Map \ref{map:moc-communities}.\footnote{Personal elaboration based on the name of Mocoví communities as reported in different studies on Mocoví language and culture; \cite[see, among others,][]{Altman2010,Altman2011,Carrio2009,Dalla-CorteCaballero2012,Gualdieri1998,Grondona1998,iparraguire2011,Lopez2017}.} They are spread out across two neighboring provinces, Chaco and Santa Fe, creating a chain of geographically connected clusters of Mocoví families. Little is known about Mocoví dialectal differences, but some studies pointed out phonological distinctions based on the geographical location of speakers. For example, the palatalization of velar segments is a common phonological process in Chaco Mocoví, but not in Santa Fe Mocoví. Similarly, it has been noted that phonological contrasts between /ɡ/ and /ɢ/ in Chaco Mocoví tend to be lost in Santa Fe Mocoví \citep[192--194]{gualdieri.citro2006}

\begin{figure}
\captionsetup{name=Map}
    \includegraphics[width=.8\linewidth]{figures/Rplot08.png}
    \caption{Current location of Mocoví communities in Argentina (map created by the author).}
    \label{map:moc-communities}
\end{figure}


Demographically, as one moves further north within the chain of Mocoví communities, it becomes increasingly common  to find families that have been in contact with Toba-speaking communities. This is the case, for example, in the community Cacica Dominga in Colonia Aborigen, which marks the northernmost point of the Mocoví community chain in Argentina. Here, Mocoví and Toba families have been living together for over a century, fostering a complex social organization that also integrates non-indigenous people. The specific social and linguistic outcomes of this long-term contact situation are still in need of further research. 

Data from different sources are combined in this paper. Most of the examples come from what I call Northern Chaco Mocoví, which corresponds to the language as spoken in the rural area of Colonia Aborigen, and has been subject of a long-term documentation project since 2011 \citep[see][]{Juarez2019}. To distinguish Northern Chaco Mocoví data, examples are accompanied by a short metadata in parenthesis, e.g. (mocCA160720\_1: 00:31:19). Another important source of data for this study corresponds to \citet{Buckwalter1995}, a bilingual Mocoví-Spanish dictionary that was compiled primarily with data from the Mocoví spoken in Colonia el Pastoril, which is located in southeast Chaco. Data corresponding to Santa Fe Mocoví mainly comes from Carrió's work and references are given when corresponding. 

%\subsection{Methodology of comparison and distribution}\label{sec:methodology}
%In order to see the basic distribution of locative elements in combination with verb types, data for this paper combine natural discourse and elicited examples from Northern Chaco Mocoví.  Since one of the goals of this article is to show the actual actual distribution and correspondences of locative types and verbs, combining both types of data gives us a robust overview of the internal behaviour of these specific elements the lexicon and grammar of the language. Data from the dictionary additionally provides a rich source to further augment specific cases that might be missing from the actual documentation of the language. 



\section{Basic verbal clauses and prior work on locative relations}\label{sec:basics-grammar}
Understanding the basic structure of transitive and intransitive clauses clarifies how locative markers impact meaning. The following subsections outline the key morphosyntactic features of these two major clause types and discuss common valence alternations, illustrating typical changes in the expression of core arguments. Verb alternations are central to this paper, as locative markers have been argued to function as valence-increasing markers, similar to applicatives \citep[see][]{carrio2013, Carrio2011, Grondona1998}. However, as will be shown, this paper proposes that spatial markers do not always increase verb valence; in certain contexts, spatial morphology merely contributes to the spatial disposition of a single argument. By examining the morphosyntactic expression of valence alternations, we can distinguish between the valence-increasing and non-valence-increasing functions of spatial markers.  

\subsection{The morphosyntax of basic verbal clauses} \label{sec:morphosyntax-basic}
The morphosyntactic structure of transitive clauses, headed by two-place predicates such \textit{-alawat} `kill', \textit{-kon} `take', \textit{waɢɑn} `hit', and similar verbs, is exemplified in (\ref{ex:trans-clause-1p}) and (\ref{ex:trans-clause-3p}). Transitive clauses typically index the A argument and show variation in the position of P arguments. While P arguments that involve speech act participants precede the verb, non-speech act participants tend to follow the predicate or may be omitted and still be referentially recovered from prior discourse context. Although P arguments usually occur post-verbally, other word orders are possible in both elicited and natural discourse  (compare (\ref{ex:trans-clause-3p}) and (\ref{ex:typeIalternb})). The conditions that trigger these word order variations of the P argument remain to be studied and are therefore beyond the scope of this paper.    


\ea 
(mocCA191025: 00:38:34) \\
\gll \textbf{qamiɾ} \textbf{s}-alawat-\textbf{\textsc{g}} \textbf{qamiɾ} \textbf{s}-aʔɡin-\textbf{\textsc{g}} \\
  {\Second.\Pron} {\First.\II}-kill-{\Pl} {\Second.\Pron} {\First.\II}-eat{$_2$}-{\Pl} \\
  \glt `...we'll kill you and eat you...'  \label{ex:trans-clause-1p}
\z


\ea
(mocCA191127\_2: 00:08:58) \\
\gll  \textbf{i}-akon-t=oʔ \textbf{so} \textbf{qopaq} kan \textbf{i}-waɢan=oʔ [...] i-waɢan-t=oʔ \textbf{na} \textbf{l-anaɢaʔ} qo-i-nak \\
  {\Third.\II}-take-?={\Evid/\Tprl?} {\DetTwo} tree {\Link} {\Third.\II}-hit={\Evid/\Tprl?} {} {\Third.\II}-hit-?={\Evid/\Tprl?} {\DetThree} {\Third.\Poss}-nape {\Sbj.\Defc}-{\Third}-say\\
 \glt `...(the tiger) took the stick and hit (Katalina) [...] hit the nape of her head, they say...'  \label{ex:trans-clause-3p}
\z

 Intransitive clauses, headed by one-place predicates like \textit{-wiɾ} `come/go' or \mbox{\textit{-ke}} `go/leave', follow the morphosyntactic structure as in (\ref{ex:intr-basic}) and (\ref{ex:intr-obl}). Like the transitive clauses above,  only one argument is indexed in the verb complex. The syntactic position of the S argument is quite flexible and such an argument can appear before or after the verb (see \citealt[][]{califa2014} for an exploratory study of the preferred argument structure in Southern Chaco Mocoví).
 
\ea
(mocCA191127\_2: 00:08:07) \\
\gll \textbf{n}-wiɾ=oʔ so \textbf{a}-\textbf{so} \textbf{ɾeɡat} \textbf{ʔaːlo} qo-i-nak \\
    {\Third.\III}-come/go={\Tprl/\Evid?} {\DetTwo} {\F}-{\DetTwo} jaguar woman {\Sbj.\Defc}-{\Third.\II}-say \\
    \glt `...a female jaguar arrived, they say...'\label{ex:intr-basic}
\z 

\ea 
(mocCA191025: 00:14:55) \\
\gll ∅-ke=oʔ \textbf{ke}-ka l-aʔa\\
	{\Third.\II}-go/leave={\Tprl/\Evid?} {\Obl-\DetOne} {\Third.\Poss}-house \\
	\glt `...(the armadillo) went to his house...'  \label{ex:intr-obl}
\z

Although core arguments are not case-marked, nominal elements referring to locations (\ref{ex:intr-obl}), instruments (\ref{ex:tran_oblique}), time (\ref{ex:oblique_time}) and certain agent-defocused participants are \textit{ke-}marked. All these nominal elements can be categorized as adjuncts, on the understanding that the argument-adjunct distinction is rather a gradient phenomenon, as pointed out by cross-linguistic research \citep[see, e.g.,][]{Haspelmath2014,Haspelmath2015,wichmann2014}.

\ea \label{ex:tran_oblique}
(mocCA191107: 00:19:29) \\
\gll ɲi qolqaeʔ n-aʃiwi-ɢat-teɡ-taʔ na l-qaik \textbf{ke}-so waloɢ-lai\\
{\DetThree} kid.{\Pl} {\Third.\III}-be/get.dry-{\Caus}-{\Prog}-{\Rec} {\DetThree} {\Third.\Poss}-head {\Obl}-{\DetTwo} cotton-piece \\
\glt `The children are drying each other's heads with a towel.'\\
\z 


\ea \label{ex:oblique_time}
(mocCA190925: 00:20:00) \\
\gll \textbf{ke}-na naʔɢa ʔwe na ɲ-aːtʃik qanta \\
{\Obl}-{\DetThree} day {\Exist}  {\DetThree} {\First.\Poss}-thankfulness also \\
\glt `...I'm also thankful today (Lit. my thankfulness also exists today)...'\\
\z

Morphologically, the oblique marking is the main overt element distinguishing between arguments and adjuncts. However, the use of \textit{ke-} comes with a caveat in Northern Chaco Mocoví. Variation within and across speakers in the use of this marker often blurs the categorization of nominal participants as either core or adjuncts \citep[see][76]{juarez2023}. Consequently, verb semantics serve as the main criterion for reliably identifying verb-entailed participants as opposed  to non-verb-entailed participants. Despite this variability,  a general consensus exists across studies on different Mocoví variants in that \textit{ke-}marked nominals correspond to adjuncts. This view is adopted here and illustrated with examples from Santa Fe Mocoví.

\ea \label{ex:contrast-stfmoc}
Oblique contrast in Santa Fe Mocoví \citep[][52]{carrio2013}
\ea [] {\gll so qoʔo ∅-io-ta \textbf{ke}-da piɣim  \\
{\Det} bird {\Third\Sg}-fly-{\Asp} {\Obl}-{\Det} sky\\
\glt `El pájaro vuela por el cielo.'  \\
\glt  `The bird flies through the sky.'} \label{ex:oblq-santafe}
\medskip
\ex [*] {\gll so qoʔo ∅-io-ta da piɣim\\
\\
\glt \label{ex:ungrammatical-santafe}} 

\ex []
{\gll so qoʔo ∅-io-ta-\textbf{lek} da piɣim \\
{\Det} bird {\Third\Sg}-fly-{\Asp}-{\Appl} {\Det} sky \\
\glt `El pájaro vuela en el espacio (a lo largo del espacio del cielo).\footnote{To capture the subtle differences in meaning between clauses, translations in Spanish are also included as they appear in the original source.}'\\
\glt `The bird flies in the sky.'} \label{ex:core-santafe} 
\z
\z 

  A clause like (\ref{ex:oblq-santafe}) includes only one core argument and becomes ungrammatical without the oblique marker, as shown in (\ref{ex:ungrammatical-santafe}). In contrast, (\ref{ex:core-santafe}) includes two core arguments, making the oblique marking unnecessary. Importantly, the absence of the oblique marking appear to be conditioned by the presence of \textit{-lek} in the verb complex. Incidentally, contrasts of this kind support the view that a spatial marker like \textit{-lek} and other markers of the same grammatical class are applicatives. They allow the promotion of former adjuncts to the category of arguments.      

\subsection{Modification of basic clauses}\label{sec:valency-test}
Verb alternations changing the basic valence profile are primarily indicated by dedicated verb morphology and the morphologically specific indexation of core arguments. The argument indexation feature is verb-dependent, meaning that argument indexes are lexically selected. Furthermore, most valence-changing morphology typically indicates changes in the status of S and A arguments.

The relation between morphologically marked verb alternations and the indexation of S and A arguments is most clearly observed in third person. There are five verb indexes in this grammatical person and two of them are conditioned by a combination of verb types and their lexically defined valence. Here I only focus on the indexes \textit{ɾ-} and \textit{i-}, which also serve to  recognize the directionality in the valence alternation.  Typically, \textit{ɾ-} occurs with one-place predicates and those predicates call for \textit{i-} when they are causativized, as in (\ref{ex:typeIaltern}). Thus, the alternation between person indexes also signals the valence increase. 

\ea \label{ex:typeIaltern} 
Type I causative alternation \citep[examples taken from][87]{juarez2023}
\ea
{(mocCA160720\_1: 00:31:19)} \\
\gll so jale \textbf{ɾ}-ʎimi \\
{\DetTwo} man  {\Third.\II.\Intr}-be.agitated \\
\glt `The man is agitated.' \label{ex:typeIalterna} 
%\glt `El hombre está agitado.' 

\ex 
{(mocCA160720\_1: 00:25:44)} \\
\gll so jale \textbf{i}-ʎimi-\textbf{ɢat} so ɾapik  \\
{\DetTwo} man  {\Third.\II}-be.agitated-{\Caus} {\DetTwo} honey \\
 \glt `The honey agitated the man.' \label{ex:typeIalternb} 
% \glt `La miel lo agitó al hombre.' \hfill (mocCA160720\_1: 00:25:44)
\z
\z 

The opposite change is also possible, i.e., valence decreasing of bivalent predicates in which the index alternation goes from \textit{-i} to \textit{ɾ-}, as in (\ref{ex:intrgreetthird}). The former index corresponds to the basic bivalent predicate and the latter to the \textit{ɢan-}detransitivized predicate.  

\ea
Valence decreasing by \textit{-ɢan} \citep[examples taken from][87]{juarez2023} \label{ex:intrgreetthird}
\ea 
{(mocCA160725: 00:43:21)} \\
\gll so jale \textbf{i}-kin-tak so i-taʔa   \\
{\DetTwo} man {\Third.\II}-greet-{\Prog} {\DetTwo} {\First\Sg.\Poss}-father\\
\glt `The man is greeting my father.' \label{ex:intrgreetthirda}
	 
\ex 
{(mocCA160725: 00:40:08)} \\
\gll so jale \textbf{ɾ}-kin-\textbf{ɢan}-tak\\
{\DetTwo} man {\Third.\Intr.\II}-greet-{\Vm:\Intr}-{\Prog}\\
\glt `The man is greeting.'  \label{ex:intrgreetthirdb}\\
%	\glt `El hombre está saludando.’ 
\z
\z

These pairs of examples illustrate that changes in the indexation of third-person arguments help to  recognize the increment or reduction of the verb valence. This does not mean that argument indexes change the basic verb valence profile, but rather they accompany the valence changes, mainly indicated by the specific valence-changing morphology. Therefore, the argument indexation parameter becomes relevant when assessing the function of spatial markers in their applicative function. If spatial markers can increment verb valence, as it has been proposed for cases like (\ref{ex:contrast-stfmoc}), we would expect changes in third-person argument indexation, similar to cases in (\ref{ex:typeIaltern}) and (\ref{ex:intrgreetthird}). However, this is not the case. Unlike those clear cases of valence-increasing and valence-decreasing illustrated above, locative markers do not affect the expression of indexed arguments, suggesting that they do not always function as valence-increasing mechanisms.    %These two related facts suggest then that if spatial markers are also part of the applicativization domain, they do not fully fulfil one the most distinctive properties of valence modification that characterize this language. 


\ea
\label{ex:non-increase-git}
Southern Chaco Mocoví \citep[adapted from][19, 171]{Buckwalter1995} 
\ea 
reloqoguit \\
\gll \textbf{ɾ}-loqo-\textbf{ɡit}\\
{\Third.\Intr.\I}-be\textsubscript{3}-{\LocFour}\\
\glt `Está \juarezemph{frente a}.' \\
\glt  `He/She is \juarezemph{in front of it/him/her}.' \label{ex:non-increase-git-a}
\ex 
iloxoyaxañiguit \\
\gll \textbf{i}-loɢo-\textbf{ɢan}-\textbf{ɡit} \\
{\Third.\II}-be\textsubscript{3}-{\Vm:\Tr}-{\LocFour}\\
\glt `Lo presenta \juarezemph{delante de otro}.' \\
\glt `He/She introduces him/her \juarezemph{to somebody else}.' (Lit. He/she makes him/her to be in front of another one) \label{ex:non-increase-git-b}
\z
\z

Note, for example, in (\ref{ex:non-increase-git}) that \textit{-ɡit} occurs with a basic intransitive predicate, which indexes the third-person argument S via \textit{ɾ-}. That is that the verb complex remains intransitive despite the suffixation of the locative marker and the ground nominal expression is only required to narrow down the referentiality of such participant. Furthermore, when the predicate is transitivized by the valence modifier \textit{-ɢan} in its causative-like function, as in (\ref{ex:non-increase-git-b}), the predicate does take the index \textit{i-}, as a typical two-place predicate. Thus, we can see that truly valence morphology conditions the argument indexation, but not the spatial morphology. 

%he example in (\ref{ex:non-increase-git}) further supports the behaviour of spatial markers with respect to the conditioning of argument indexation. Note that the occurrence of the causative marker triggers the morphological change of S argument in (\ref{ex:non-increase-git-a}), but not the spatial marker. Furthermore, the same suffix \textit{-ɡit} can occur with both intransitive and causative-derived predicates without any constraint  

\subsection{Prior works on locatives and directionals}\label{sec:prior-work}
Since the first studies on Southern Chaco Mocoví \citep[e.g.][]{Gualdieri1998,Grondona1998}, it has been noted that the expression of location and direction in this language is morphologically verb-encoded. These studies differ in the way verb elements are segmented, glossed and particularly analyzed. 

For example,  \citet[][132--147]{Grondona1998} defined directional-locative markers as enclitics which correspond to a single paradigm (see \tabref{tab:loc-dir-grondona}).\footnote{The following tables present the directional and locative paradigms as described for Southern Chaco Mocoví and Santa Fe Mocoví. Each table re-organizes how paradigms are described in the original sources, keeping the content identical to the authors descriptions. Each table here includes a number-based \textsc{id}, \textsc{form}, \textsc{gloss}, as employed in the original source, \textsc{meaning}, the specific semantic component attributed to the spatial marker, and \textsc{sub-class}. Although the latter parameter is redundant for \tabref{tab:dir-gualdieri}-\tabref{tab:dir-carrio}, it is relevant for Grondona's paradigmatic organization, as this author classifies individual spatial markers by major categories, e.g., down/up or in/out. The column \textsc{sub-class} is kept then to maintain the same structure across the tables.} 

\begin{table}
\caption{Locative/Directional paradigm according to \citet[][]{Grondona1998}.}
\label{tab:loc-dir-grondona}
    \centering
    \begin{tabularx}{\textwidth}{lllQQ}
    \lsptoprule
         {id} &  {form} &  {gloss} &  {meaning} &  {sub-class} \\ \midrule
        1 & +ñi & \textsc{down}  & \textsc{down, downwards} & \textsc{down/up} \\ 
        2 & +sigim & \textsc{up} & \textsc{up, upwards} & \textsc{down/up} \\ 
        3 & +ot $\sim$ +oʔot & \textsc{under} & \textsc{under} & \textsc{under/on} \\ 
        4 & +leɡ & \textsc{on} & \textsc{on, over} & \textsc{under/on} \\ 
        5 & +eɡ  $\sim$ +weɡ & \textsc{out} & \textsc{out, outwards} & \textsc{in/out} \\ 
        6 & +wɡi & \textsc{inside} & \textsc{in, inwards, into} & \textsc{in/out} \\ 
        7 & +ñiɡi & \textsc{inside} & \textsc{inside} & \textsc{in/out} \\ 
        8 & +o $\sim$ +wo & \textsc{inwds} & \textsc{inwards, hither} & \textsc{in/out} \\ 
        9 & +kena & \textsc{tds.ther} & \textsc{towards here} & \textsc{towards here/there} \\ 
        10 & +iɡi & \textsc{tds} & \textsc{towards (there?)} & \textsc{towards here/there} \\ 
        11 & +aʔta & \textsc{across} & \textsc{on/to other side of, across} & \textsc{other spat. rel.} \\ 
        12 & +eʔe & \textsc{with} & \textsc{with} & \textsc{other spat. rel.} \\ 
        13 & +iɡit & \textsc{behind} & \textsc{behind} & \textsc{other spat. rel.} \\ 
        14 & +peɡeʔ $\sim$ +peʔ & \textsc{up.to} & \textsc{up to} & \textsc{other spat. rel.} \\ \lspbottomrule
    \end{tabularx}
\end{table}

She further argued that all directional-locative elements increase the verb valence.\footnote{Although Grondona mentions twice in her dissertation that there are fifteen locative-directional markers, only fourteen markers are listed in her work.} In \citet[][132--139]{Grondona1998}, we find cases like  (\ref{ex:locaapplic}) in which the number of participants increases and the marker \textit{ke} is ruled out when the suffix \textit{-lek} is added to the predicate.  


\ea
 Southern Chaco Mocoví  {\citep[examples reproduced as in the original source,][138]{Grondona1998}}\label{ex:ungramapplgrond} 
\ea 
[] {\gll s+maɢ so lwis\\
{\First\Ac}+push  {\Deicg} Luis\\
\glt `I push Luis.'} 
\ex 
[] {\gll  s+maɢ+\textbf{leɡ} so waloq so lwis\\
{\First\Ac}+push+\textsc{on}  {\Deicg}  cotton {\Deicg} Luis\\
\glt `I push Luis onto the cotton.'}
\ex
[*] {\gll s+maɢ+\textbf{leɡ} \textbf{ke} so waloq so lwis\\
{\First\Ac}+push+\textsc{on}  ke {\Deicg}  cotton {\Deicg} Luis\\
\trans `I push Luis onto the cotton.'} \label{ex:locaapplic}
\z
\z


\citet[][]{Gualdieri1998}, on the other hand, identified one paradigm of directionals and another of locative-orientative markers, and treated them as suffixes (see \tabref{tab:dir-gualdieri} and \tabref{tab:loc-orient-gualdieri}). Gualdieri also identified that directionals occur in intransitive clauses, while locative-orientatives are part of transitive clauses.   

\begin{table}
\caption{Directional paradigm according to \citet[][]{Gualdieri1998}.}
\label{tab:dir-gualdieri}
    \begin{tabularx}{\textwidth}{XXXll}
    \lsptoprule
        {id} &  {form} &  {gloss} &  {meaning} &  {sub-class}  \\ \midrule
        1 & -wek & \textsc{dir} & \textsc{‘to the exterior’} & \textsc{directional} \\ 
        2 & -o & \textsc{dir} & \textsc{‘to the interior’} & \textsc{directional} \\ 
        3 & -ni & \textsc{dir} & \textsc{‘downwards’} & \textsc{directional} \\ 
        4 & -sigim & \textsc{dir} & \textsc{‘upwards’} & \textsc{directional} \\ 
        5 & -ta & \textsc{dir} & \textsc{‘to a limit’} & \textsc{directional} \\ 
        6 & -aɢasom & \textsc{dir} & \textsc{‘to the water’} & \textsc{directional} \\ \lspbottomrule
    \end{tabularx}
\end{table}

\begin{table}
\caption{Locative-orientative paradigm according to \citet[][]{Gualdieri1998}.}
\label{tab:loc-orient-gualdieri}
\centering
\begin{tabularx}{\textwidth}{XXXll}
 \lsptoprule
        {id} &  {form} &  {gloss} &  {meaning} &  {sub-class}  \\
        \midrule
        1 & -ɡi & \textsc{loc} & \textsc{‘inside’} & \textsc{locative} \\ 
        2 & -lek & \textsc{loc} & \textsc{‘on top’} & \textsc{locative} \\ 
        3 & -ot & \textsc{loc} & \textsc{‘below’} & \textsc{locative} \\ 
        4 & -ʔot & \textsc{loc} & \textsc{‘at a limit’} & \textsc{locative} \\ 
        5 & -ɡe & \textsc{or} & \textsc{‘allative’} & \textsc{orientative} \\ 
        6 & -ɡit & \textsc{or} & \textsc{‘in front of’} & \textsc{orientative}  \\ \lspbottomrule
\end{tabularx}
\end{table}

Carrio's work (\citeyear{Carrio2011,carrio2013}) on Santa Fe Mocoví  follows an in-between approach, combining those previous descriptions. 

\begin{table}
\caption{Locative paradigm according to \citet[]{carrio2013}.}
\label{tab:loc-carrio}
    \begin{tabularx}{\textwidth}{XXXll}
    \lsptoprule
         {id} &  {form} &  {gloss} &  {meaning} &  {sub-class} \\
         \midrule
        1 & -lek & \textsc{loc} & on a flat surface & \textsc{locative} \\ 
        2 & -o & \textsc{loc} & in a closed place & \textsc{locative} \\ 
        3 & -aɢasom & \textsc{loc}& in water source & \textsc{locative} \\ 
        4 & -ʔot & \textsc{loc} & down & \textsc{locative}\\ \lspbottomrule
    \end{tabularx}
\end{table}


\begin{table}
\caption{Directional paradigm according to \citet[]{carrio2013}.}
\label{tab:dir-carrio}
    \begin{tabularx}{\textwidth}{XXXll}
   \lsptoprule
        {id} &  {form} &  {gloss} &  {meaning} &  {sub-class}  \\
        \midrule
        1 & -lek & \textsc{dir} & to the front, horizontal & \textsc{directional} \\ 
        2 & -o & \textsc{dir} & in & \textsc{directional} \\ 
        3 & -aɢasom & \textsc{dir} & to the water & \textsc{directional} \\ 
        4 & -ʃim & \textsc{dir} & Upwards & \textsc{directional} \\ 
        5 & -ni & \textsc{dir} & downwards & \textsc{directional} \\ 
        6 & -ɣe & \textsc{dir} & to an specific place & \textsc{directional} \\ 
        7 & -ɣi & \textsc{dir} & inside & \textsc{directional} \\ \lspbottomrule
    \end{tabularx}
\end{table}


In her analysis, locatives and directionals correspond to different paradigmatic classes of suffixes and only some of these suffixes occur in both paradigms (see \tabref{tab:dir-carrio} and \tabref{tab:loc-carrio}). Similar to Grondona's analysis, Carrió argues that both locatives and directionals are applicative markers. 

\section{The multiple functions of spatial markers}\label{sec:mutiple-functions}
As mentioned earlier, this paper focuses exclusively on the spatial markers illustrated in \tabref{tab:locativemarkerarguments}. My view on the function of these elements differs slightly from the prior studies mentioned above. On the one hand, I argue that these spatial markers do have multiple functions, ranging from the spatial disposition of a single argument to the addition of adjuncts as core-like arguments. Accordingly,  I propose, on the other hand, that spatial markers cannot solely be defined or glossed as applicatives, as not all of their uses increase the number of arguments in the verb framing of clauses.

\begin{table}
\caption{Paradigm of spatial markers analyzed in this study.} \label{tab:locativemarkerarguments}
\begin{tabular}{cc}
\lsptoprule
\textbf{Suffix} & \textbf{Gloss}\\\midrule
\textit{-lek} & {\LocOne}\\
\textit{-ɡe} & {\LocTwo}\\
\textit{-ɡit} & {\LocThree}\\
\textit{-ɡi} & {\LocFour}\\
\lspbottomrule
\end{tabular}
\end{table}

This also impacts on the gloss selected in this paper for Northern Chaco Mocoví spatial markers. The label `\textsc{loc}' is meant to be the descriptively simplest gloss for these suffixes, selecting the basic notion of location present across them, which paradigmatically contrast with other members of the same class -- indicated here by the sub-script included in the gloss. Given the multifunctionality of the spatial markers -- the notion of multifunctionality follows \citet[][]{haspelmath2003,haspelmath2023} --  their distribution and morphosyntactic structures in which they occur will help us understand the range of functions around these grammatical elements. The different constructions and verb types analyzed below should be taken as a preliminary \textit{type}-classification that allow us to capture the variation of functions that spatial markers have. Further research may include other types of verbs, capturing the largest extent possible of Mocoví verbal tokens.

\subsection{Combination with existentials and one-place predicates}\label{sec:one-place}

 The following subsections describe the function of spatial markers in the context of existential constructions and different types of one-place predication, capturing the semantic contributions that spatial markers have on each of these morphosyntactic structures. 

\subsubsection{Existentials}
The early works on Southern Chaco Mocoví identified two grammatical elements, dedicated to the expression of existential clauses, which have been broadly defined as establishing ``the existence of an argument'' \citep[][103]{Gualdieri1998} or expressing the ``existence of somebody or something'' \citep[][159]{Grondona1998}.  Following \citet[][103]{Gualdieri1998}, Southern Chaco  Mocoví expresses positive existence by \textit{ʔwe} and negative existence by \textit{qajka} \citep[see also][159--161]{Grondona1998}.\footnote{The unit \textit{qajka} could also be analyzed as the combination of \textit{qaj-} and the nominal determiner \textit{-ka}, as \citet[]{Grondona1998} occasionally segmented it in her work. Synchronically, other strikingly similar forms  like \textit{qaj-dʒi} or \textit{qaj-ɾa}, which do not have a negative existential meaning (see (\ref{ex:be_1})), are also present in the language. Identifying the meaning contribution of \textit{qaj-} is yet something to explore further.} Existential units seem to be a noun-verb in-between category, as they can take some verb and noun morphology \citep[see footnote 40 in][103]{Gualdieri1998} Similarly, Carrio's studies of existentials  defined these units as ``defective verbs''  for Santa Fe Mocoví \citep[][]{carrio2017,carrio2019a}. Regardless of their categorical status, the so-called existential units stand out because they do not follow any of the three inflectional verb classes, as prototypical verbs do in the language (cf. (\ref{ex:exist_neg}) and (\ref{ex:pos_exist})).     
 
 In Northern Chaco Mocoví, existentials are similar in form and function to what has been reported for Southern Chaco Mocoví.\footnote{In terms of form, however, there seems to be a free variation in the expression of the glottal stop in the positive existential, e.g., \textit{ʔwe}  $\sim$ \textit{we}.} Positive and negative existence are typically expressed as in (\ref{ex:exist_intro}) and (\ref{ex:exist_neg}), respectively.


\ea 
{(mocCA191010\_1: 00:01:43)} \\
\label{ex:exist_intro}
\gll \textbf{we} so qaɾ-qomena-pi ke-so-m paɾaɢan-aɢa-j-∅ \\
 {\Exist} {\DetTwo} {\First.\Pl}-grand\_mother-pi {\Obl}-{\DetTwo}-{\Ints} hunt-{\Val:\Intr}-{\Nmlz}-{\Attrb}-{\F} \\
\glt `...there was our grandmother, she was a gatherer...' \\
\z

\ea 
{(mocCA191010\_1: 00:14:16)} \\
\label{ex:exist_neg}
\gll qaɾa i-awek dos{\Sploan} tɾes{\Sploan} tʃaqam \textbf{qajka} ka l-iʔjaːk qo-i-nak \\
{\Link?} {\Third.\II}-bring {two} {three} {then} {\Neg.\Exist} {\DetFive} {\Third.\Poss}-load {\Sbj.\Defc}-{\Third.\II}-say\\
\glt `...but when (she) brought two or three (kids with her), then there was nothing, they say...'
\z


It has already been noted that Mocoví existentials participate in predicate possessive constructions, which aligns to a well known cross-linguistic functional and structural overlap between those types of constructions (see, for example, \citealt[][145--145]{Dryer2007}, \citealt[][6--7]{Stassen2009}, \citealt[][21--41]{VeselinovaHamari2022}). Such an overlap is also clear by the translations available for constructions like (\ref{ex:pos_exist}) and (\ref{ex:pos_exis_fut}), as both predicative and existential readings are possible. Furthermore, examples of this type suggest that existential elements might be better viewed as linking devices between participants in the predication, similar to overt copulas in other languages \citep[see][9--11]{HaspelmathNonverbal}. However, this linking function is not clear in all cases where existentials are used and it appears to emerge only in cases where the predicate possessive reading is available; otherwise there is only one participant on which the predication is about.

\ea
{(mocCA191025: 00:38:40)} \\
\label{ex:pos_exist}
\gll jim \textbf{we} na i-otawa-ʔ  dos{\Sploan} naʔ-wa ∅-kojaɾ-ta-o\\
{\First\Sg.\Pron} {\Exist} {\DetThree} {\First\Sg.\Poss}-help/partner-{\Pl} {two} {\Det}-{\Pl}  {\Third.\II}-{follow}-{\Dur}-{\Dir:\Out}
\\
\glt `... I have friends, two, who are following me...' or \\
\glt `...There are two of my friends, who are following me...' \\
\z

\ea 
(mocCA191025: 00:09:08) \\
\label{ex:pos_exis_fut}
\gll paɾa{\Sploan}  que{\Sploan}  \textbf{ʔwe} ka qo-noq maʔ \\
{for} {that} {\Exist} {\DetFive} {\First\Pl}-food {further} \\
\glt `...(I will invite you) so that we can have our lunch...' or \\
\glt `...(I will invite you) so that our lunch happens...'\\
\z 

Relevant for this paper is whether existentials can combine with spatial markers, particularly with those that are discussed in this article. Data on this regard are not conclusive. In the sources available on the language, the combination of the positive existential with spatial markers is not attested. With respect to the  negative existential, examples only include the spatial marker \textit{-ɡi}. Gualdieri's work has multiple tokens of the form \textit{qayka-ɡi}, one of which is illustrated in (\ref{ex:neg_exist_loc}), and the same form is documented in \citet[][149]{Buckwalter1995} with the meanings `there is nothing inside' and `it is empty'. 

\ea 
Southern Chaco Mocoví \citep[][192]{Gualdieri1998}\footnote{Phoneme representation and glosses are given as in the original source.} \\
\label{ex:neg_exist_loc}
\gll ni-krawm ʔiːmek \textbf{qayka}-\textbf{ɡi} \\
{\Clf}-{\Dem} house {\Neg.\Exist}-{\Loc}\\
\glt `That house (far) over there is empty.'\\
\z 

Additionally, I have not documented the combination of spatial markers with either of these existential elements in Northern Chaco Mocoví in elicited or natural speech data. This does not rule out the possibility of combining those elements, but rather demands a more controlled and detailed study of this part of Mocoví grammar. 


\subsubsection{Static locative predication}\label{sec:figure-ground-sec}
Spatial markers are quite frequent with verb forms denoting a semantically broad notion of static location. With these verbs, spatial markers contribute to delineating the relationship between a \textsc{figure} and its \textsc{ground}. There are at least four predicates to express static location and they all are glossed here as `be' (closer in meaning to `estar' in Spanish), but differentiated by a sub-script, e.g., \textit{-we(ː)} `be\textsubscript{1}', \textit{-ne(e)t} `be\textsubscript{2}', \textit{-loqo} `be\textsubscript{3}' and \textit{paʔaː} `be\textsubscript{4}'. I will start with the predicate \textit{weː} `be\textsubscript{1}', whose third person inflection is quite similar to the positive existential discussed in the previous section.\footnote{One could argue that these two lexical forms are diachronically related. However, although this could be true, the synchronic morphosyntactic properties of both units are quite different. Throughout examples in this paper, it is clear that only the verb form inflects for person and number of the subject argument. Compare examples (\ref{ex:pos_exis_fut}) and (\ref{ex:neg_exist_loc}) with examples in this section. Furthermore, semantically, the predicate possessive reading is not available for the static predicates.} Natural discourse data in (\ref{ex:be_1}) nicely illustrates the combination of the existential construction with the stative locative clause -- note that the third person S argument is expressed via ∅-. 

\ea\label{ex:be_1}
(mocCA190924\_1: 00:02:50) \\
\gll qaj-ɾa ɾa we ɾa ɲ-aʔaːtʃik ∅-\textbf{weː}-ta i-na ɲi i-jaːle-{∅} \\
{nothing?}-{\DetFour} {\DetFour} {\Exist} {\DetFour} {\First.\Poss}-thankfulness {\Third.\II}-be\textsubscript{1}-{\Dur} i-{\DetThree} {\DetFive} {\First.\Poss}-descendent-{\M} \\
\glt   `...I am thankful today because my son is here...' \\
\glt (Lit. My gratitude exists today,  my son is here) 
\z 

When the predicate \textit{-weː} combines with \textit{-lek}, as in (\ref{ex:be_1_lek}), the figure is located at the top of the ground and the entire verb complex denotes the meaning of living on a particular place. In the context of this narrative, the ground refers to the land on which the Mocoví community lives in northern Chaco. 

\newpage
\ea\label{ex:be_1_lek}
(mocCA180709: 00:13:30) \\
\gll muchos{\Sploan} años{\Sploan} na qom ∅-weː-ta-\textbf{lek} na-te i-na\\
many years {\DetThree} people {\Third.\II}-be\textsubscript{1}-{\Dur}-{\LocOne} {\DetThree}-{\Dem?} {i}-{\DetThree} \\
\glt `...people have been on this land for many years...' \\
\z

The use of \textit{-ɡe}, on the other hand, typically entails a distant location. Specifically in (\ref{ex:be_1_ge}), the ground corresponds to \textit{Villa Angela}, a small town in Southern Chaco, where the speaker spent a few years of his youth. Note that the point of reference here is the community Cacica Dominga in northern Chaco, where this recording was made. 


\ea\label{ex:be_1_ge}
(mocCA190924\_1: 00:04:03) \\
\gll tres{\Sploan} ɲaɢa-ɾi s-weː-ta-\textbf{ɡe} ɾa villa angela\\
three year-{\Pl} {\Third.\II}-be\textsubscript{1}-{\Dur}-{\LocTwo} {\DetThree} villa angela \\
\glt `...I've been in Villa Angela for three years...' 
\z
 
Furthermore, when speakers conceive the figure as being contained by the ground, the use of \textit{-ɡi} is expected. To properly understand the case in (\ref{ex:be_1_gi}) more context on the local Mocoví discourse practice is needed. This example comes from the beginning of a biographical monologue. It is typical of free speech of this kind, as well as other types, that the time and location of the speaking event is set right at the start of the discourse. Given this discourse context then, the verbal expression of \textit{-ɡi} metaphorically locates all the participants of the speaking event within the month of October. 

\ea \label{ex:be_1_gi}
(mocCA191010\_1: 00:01:26) \\
\gll diez{\Sploan} de{\Sploan} octubre{\Sploan} s-we-qo-ta-\textbf{ɡi} donde{\Sploan} está{\Sploan} octubre{\Sploan} ke-na naʔɢa \\
tenth of october {\First.\II}-be\textsubscript{1}-{\Dur}-{\LocFour} where be.{\Third.\Sg.\Prs} october {\Obl}-{\DetThree} day\\
\glt `...today is tenth of October, we reached October already...' (Lit.: `...tenth of October, we are where October is in this day...') \\
\z

Examples above came from natural discourse illustrating the use of spatial markers. In (\ref{ex:loc-stative}), data are taken from elicitation with the predicate \textit{-paʔaː} `be\textsubscript{4}' \citep[examples (\ref{ex:loc-stative}) and (\ref{ex:loc-quant}) are reproduced as in][74, 76]{juarez2023}. Minimal pairs of this kind further elaborate on the semantic contribution of each spatial marker in targeted speech. Thus, \textit{-lek} indicates that the \textsc{figure} is at the top part of the \textsc{ground}; \textit{-ɡe} denotes that the \textsc{figure} finds in a different location, which is distant from the locus of speech; \textit{-ɡit} conveys a relation of contiguity between the \textsc{figure} and the \textsc{ground}, and \textit{-ɡi} locates the \textsc{figure} at the interior of the \textsc{ground}, indicating a relation of containment between those two spatially related participants. Finally, it should be noted that the \textsc{ground} participant is syntactically expressed like any other P nominal argument of two-place predicates. In this regard, one-place and two-place clauses are similar. 

\ea
\label{ex:loc-stative}
\ea 
(mocCA210805: 00:33:55) \\
\gll ∅-paʔaː-\textbf{lek} ke-na lawa\\
	{\Third.\II}-be\textsubscript{4}-{\LocOne} {\Obl-\DetThree} soil \\
	\glt `(the tube) is on the ground.' \\

\ex
(mocCA210805: 00:10:37) \\
\gll ∅-paʔaː-ta-\textbf{ɡe} ɾa lai    \\
	{\Third.\II}-be\textsubscript{4}-{\Dur}-{\LocTwo} {\DetFour} side \\
	\glt `She/he is on the other side (of this place).' \\

 \ex
 (mocCA210805: 00:16:38) \\
 \gll ∅-paʔaː-\textbf{ɡit} dʒi Boːni na l-awa \\
	 {\Third.\II}-be\textsubscript{4}-{\LocThree} {\DetFive} Boni {\DetThree} {\Third.\Poss}-land\\
\glt `(The house) is next to Boni's field.' \\
 
\ex
(mocCA210805: 00:35:34) \\
\gll ∅-paʔaː-\textbf{ɡi} na lawa\\
	{\Third.\II}-be\textsubscript{4}-{\LocFour} {\DetThree} soil \\
	\glt `(the seed) is in the soil.' \\
\z
\z

 There is variation with respect to locative predicates that can occur without spatial makers and those that obligatorily require these suffixes. A verb of the first type is illustrated in (\ref{ex:be_1})--(\ref{ex:be_1_gi}) and of the second type is in (\ref{ex:loc-stative}). This latter case also resembles predicates employed for expressing topological relations in Amazonian languages \citep[see studies on this topic cited in][]{vallejosyopan.brown2021}, as these verb roots denote a broad positional meaning without including any semantic information on the disposition between a figure and its ground. Because of that, a root like \textit{-paʔaː} without the locative information creates ungrammatical sentences.

\subsubsection{Movement predication}\label{sec:movement-pred}
Spatial markers can also occur with movement predicates, as in (\ref{ex:movementbase-b}--\ref{ex:movementbase-e}). For the sake of consistency, minimal pairs from the Mocovi´s dictionary are presented here. Although examples below are similar to those in (\ref{ex:loc-stative}) with respect to the functions of individual spatial markers, some subtle differences in meaning are worth mentioning. As indicated by the translation of (\ref{ex:movementbase-a}), this predicate denotes displacement only, without any implication of the goal or destination. The use of \textit{-lek}, on the other hand, enriches the movement component by including the \textsc{followee} participant, which can also be animate, as in (\ref{ex:movementbase-b}). This animacy component was not present in the other contexts of use analyzed before and is also shared by  \textit{-ɡit} and \textit{-ɡi}, but not \textit{-ɡe}. Overall, the configuration of the movement event changes according to the spatial marker with which the predicate occurs; thus, the single argument of the basic predicate moves to a distant location in (\ref{ex:movementbase-c}), goes to meet someone in (\ref{ex:movementbase-d}) and goes with a group of people, as in (\ref{ex:movementbase-e}).


\ea \label{ex:movementbase}
Southern Chaco Mocoví \citep[adapted from][157]{Buckwalter1995}\footnote{The phonological representation in the second line of Buckwalter's examples is based on the grapheme-sound correspondence defined in the vocabulary.} \ea 
queeta\\
\gll ∅-ke-ta   \\
{\Third.\II}-move-{\Dur}\\
\glt `Va yendo, sigue, continúa.' \\
\glt `She/he goes to/continues to (somewhere).' \label{ex:movementbase-a}

\ex 
queetalec \\
\gll ∅-ke-ta-\textbf{lek}   \\
{\Third.\II}-move-{\Dur}-{\LocOne}\\
\glt `Le sigue, es partidario de.' \\
\glt `She/he \juarezemph{/supports him/her/(it?)}.' \label{ex:movementbase-b}


\ex 
queeta'gue \\
\gll ∅-ke-taʔ-\textbf{ɡe}  \\
{\Third.\II}-move-{\Dur}-{\LocTwo}\\
\glt `Va a tal lugar.' \\
\glt `She/he goes to \juarezemph{a place}.' \label{ex:movementbase-c}

\ex 
queeta'guit \\
\gll  ∅-ke-taʔ-\textbf{ɡit}  \\
{\Third.\II}-move-{\Dur}-{\LocThree} \\
\glt `Le va al encuentro.' \\
\glt `She/he goes \juarezemph{to meet (someone)}.' \label{ex:movementbase-d}

\newpage
\ex 
queetagui \\
\gll ∅-ke-ta-\textbf{ɡi}\\
{\Third.\II}-move-{\Dur}-{\LocFour} \\
\glt `Va entre gente, recorre los pueblos, va durante un tiempo.' \\
\glt `She/he goes (somewhere) with \juarezemph{a group/among people}.'
\glt  \label{ex:movementbase-e}
\z
\z

Since this particular predicate allows the contrast between locative-marked and non-locative-marked predicates, the question is whether this contrast also correlates with the possibility to express or not the non-figure participant. If the expression of that participant is possible, then we should ask about its encoding, namely, similar to an adjunct or similar to an argument.  Example (\ref{ex:movement-noloc}), taken from natural discourse, shows that the morphologically simplest third-person verb form of the predicate `move' does not include the goal or destination, which appears expressed as part of the second predicate.\footnote{This predicate is irregular in its third person form. The verb is articulated as \textit{xek} when there is no other morphological material to the right side of the root. Otherwise, the predicate articulates similar to the examples in (\ref{ex:movementbase}).} However, the distant location \textit{so nadʒik} naturally occurs when the predicate includes the spatial maker \textit{-ɡe} in (\ref{ex:movement-loc}). 

\ea \label{ex:movement-noloc}
(mocCA191023: 00:05:34) \\
\gll \textbf{xek} so peʔ t-aj-ɡe na n-oʔweːnaɢa \\
move.{\Third.\II} {\DetTwo} {grand\_father} {\Third.\II}-go-{\LocTwo} {\DetThree} {\Poss.\Indet}-field \\
\glt `...our grandfather left, he went to the field...' \\
\z

\ea \label{ex:movement-loc}
(mocCA120706: 00:44:46) \\
\gll jim s-ike-taʔ-\textbf{ɡe} \textbf{so} \textbf{nadʒik} peɾo{\Sploan} dʒ-oʔtʃi \\
{\First\Sg.\Pron} {\First.\II}-move-{\Dur}-{\LocTwo} {\DetTwo} track but {\First\Sg.\II}-be.afraid \\
\glt `I'm going to that road \juarezemph{over there}, but I'm scared.'
\z

All in all, examples (\ref{ex:movementbase})--(\ref{ex:movement-loc}) support the view of spatial markers functioning as argument addition devices, specifying different semantic properties of the added participant. To build a more robust case of this function, we would need to test grammaticality judgments about the expression of oblique-marked participants in all the contexts illustrated above. This task will be completed in future fieldwork.  

\subsubsection{State/change-of-state predication} \label{sec:modif-singlepart}
 Spatial markers are not always part of predications that involve a \textsc{figure-ground} relationship. They also occur with some predicates that indistinguishably denote a state or change-of-state and include only a single argument, e.g., \textit{-awiɡ} `be/get.dry' or\textit{-ʃiwi} `be/get.burned'.  In these cases, spatial markers not only indicate the specific locus of state or change-of-state, but also represent a type of quantized change, entailing that only part of the single argument displays the property denoted by the predicate.  This function of spatial markers brings them closer to the domain of affectedness \citep[see][]{Beavers2011}. 
 
 Additionally, the evidence presented in (\ref{ex:loc-burn})--(\ref{ex:noquantization}) indicates that spatial markers cannot be defined solely on their arguably applicative function. In examples (\ref{ex:loc-quant-b}) and (\ref{ex:loc-burn-lek}), for instance,   \textit{-lek} refers to an event in which the S argument is only superficially dry or burnt -- specifically,  either on the topmost layer of the soil or on the outer surface of the meat. Similarly, in (\ref{ex:loc-quant-c}), \textit{-ɡi}  indicates that the S argument is dry, but only  in its inner part. 


 \ea \label{ex:loc-burn}

 \ea
 (mocCA170803: 00:02:36) \\
 \gll so laʔat i-awiɡ \\
{\DetTwo} meat {\Third.\II}-be/get.burned \\
 \glt `The meat got burned.' \label{ex:loc-burn-base}


\ex
(mocCA210806: 00:05:06) \\
\gll so laʔat i-awiɡ-\textbf{lek} \\
 {\DetTwo} meat {\Third.\II}-be/get.burned-{\LocTwo}\\
\glt `The meat got seared on the outside.' \label{ex:loc-burn-lek}
\z
\z 


\ea
\label{ex:loc-quant}
\ea
(mocC191107: 00:03:003) \\
\gll na lawa ɾ-aʃiwi \\
{\DetThree} soil {\Third.\II.\Intr}-be/get.dry \\
\glt `The soil is dry.'  \label{ex:loc-quant-a}

\ex 
(mocC191108: 00:08:35) \\
\gll na lawa ɾ-aʃiwi-\textbf{lek} \\
  {\DetThree} soil {\Third.\II.\Intr}-be/get.dry-{\LocOne} \\
  \glt `The soil is dry in the upper part.' \label{ex:loc-quant-b}
 
\ex 
(mocC191108: 00:04:05) \\
\gll ɾ-aʃiwi-\textbf{ɡi} \\
{\Third.\II.\Intr}-be/get.dry-{\LocFour}\\
\glt `It is dry inside.'  \label{ex:loc-quant-c}
\z
\z 

However, not all of the locative markers explored in this paper have the same effect as the examples above, particularly with this kind of verb root. Locative markers \textit{-ɡe} and \textit{-ɡit} in (\ref{ex:noquantization}) exhibit more similarities in meaning to the examples in (\ref{ex:loc-stative}), rather than the cases discussed immediately above. The use of \textit{-ɡe} in (\ref{ex:noquant-a}) entails that the single argument is a location away from the speech-location -- a dimension of meaning captured by the deictic indication `over there' in the translation. The use of \textit{-ɡi} in (\ref{ex:noquan-b}) further illustrates the proximity of the plural entities included in the single argument, specifically indicating the spatial relation between these two entities, i.e., one bag being next to the other.

\ea
 \label{ex:noquantization}
\ea
(mocC191108: 00:10:50) \\
\gll ɾ-aʃiwi-ta-\textbf{ɡe}  na nadʒik \\
{\Third.\II.\Intr}-be/get.dry-{\Dur}-{\LocTwo} {\DetThree} track \\
\glt `The track \juarezemph{over there} is dry.' \label{ex:noquant-a}


\ex
(mocC191108: 00:20:43) \\
\gll s-anda-ʔ dʒi-ho so βoːlsa{\Sploan}-ɾiʔ tʃaqam  ɾ-ʃiwi-\textbf{ɡit}\\
{\First.\II}-put-{\Pl} {\DetFive}-{\Dem} {\DetTwo} bag-{\Pl} then {\Third.\II.\Intr}-be/get.dry-{\LocThree} \\
\glt `I left the bags (on the ground) and they got dry \juarezemph{one next to the other}.'  \label{ex:noquan-b}
 \z
 \z
 

\subsubsection{Other idiosyncratic uses}\label{sec:idiosincratic-use}
In most of the cases described so far, the meaning contribution of spatial markers to the verb root seems to follow a quite transparent principle of compositionality, i.e., one can more or less understand the meaning outcome based on the meaning contribution of the verb root and the spatial marker. There are cases, however, in which the outcome of the verb-suffix combination is not totally predictable or does not fall into the types of uses we have seen so far. A few verbs show this particular use of spatial markers. Although the class of predicates in (\ref{ex:crack}) belongs to those explored in \S\ref{sec:modif-singlepart}, the meaning conveyed by \textit{-ɡe} and \textit{-ɡi} differ from those. 

\ea
\label{ex:crack}
\ea 
({mocCA160710: 35:55:00}) \\
\gll dʒi ɲ-apeɡet ɾ-opoɢo-jaʔ-\textbf{ɡe} \\
{\DetFive} {\First\Poss.\II}-plate {\Third.\Intr.\I}-be/become.cracked-{\Dur}?-{\LocTwo} \\
\glt `The plate is cracked (e.g. \juarezemph{on one of the sides}).' \label{ex:crack-a}
\glt

\ex 
({mocCA160710: 32:19:00}) \\
\gll ɾ-opoqo-\textbf{ɡi} dʒi peɡet \\
{\Third\Intr.\I}-be/become.cracked-{\LocFour} {\DetFive} {\First\Poss.\II}-plate \\
\glt `The plate broke \juarezemph{completely}.' \label{ex:crack-b}
\z\z

Note that in (\ref{ex:crack-a}) the cracking event involves changes on the side of the entity referred by the single argument and there is no implication of displacement or distance from the speech event when \textit{-ɡe} is used. Likewise, \textit{-ɡi} denotes that the same event affected completely and not only a part of the single participant.\footnote{I employ the notion of  ``affectedness'' in a rather loose sense for cases like (\ref{ex:crack}). By affectedness here I mean that the spatial-marked predicates denote differences in the extent that the argument displays the specific property denoted by the predicate. Put differently, the change of state observed in the single argument can be roughly quantized, e.g. partially, totally, etc.} Additionally, these differences in meaning can be corroborated by the addition of adverbial elements. While the sentence in (\ref{ex:crack-a}) is grammatical with an adverb like `a little bit', the clause in (\ref{ex:crack-b}) turns ungrammatical with the same adverb, but grammatical with an adverb like `completely'. 


The fact that spatial markers do not function similarly to argument-addition devices with predicates in \S\ref{sec:modif-singlepart} and \S\ref{sec:idiosincratic-use} relates to the long-standing debate on  applicativization of the often-called ``unaccusative'' verbs. Although the universality of unaccusative verbs is controversial, the broad class of unaccussatives usually includes change-of-state predicates, among many other sub-classes \citep[see][]{LevinRappaporthovav1995}. Languages seem to vary with respect to the possibility to applicativize intransitive predicates with a semantically non-agentive, patient-like subject. Since at least \citet[see][254--255]{Baker1988}, it has been believed that  languages cannot applicativize ``unaccusative'' predicates; however, more recent research has shown that such a claim does not hold true \citep[e.g.][]{Bugaeva2010,denDikken2023}. It is still unclear how large the class of one-place predicates, in which spatial markers only indicate certain physical property of the single argument, is in Northern Chaco Mocoví. Nonetheless, it is typologically relevant to note that at least a few of change-of-state predicates exhibit a different semantic behavior compared to the other one-place predicates, as has been observed for other languages. However, the data call for a more fine-grained exploration of the class of non-agentive one-place predicates, as there are other predicates of the same semantic profile that can actually take spatial markers. \citet[][]{Grondona1998} provides one example with the predicate \textit{-ilew} `die', reproduced here in (\ref{ex:die-loc}).

\ea \label{ex:die-loc}
Southern Chaco Mocoví \citep[][135--136]{Grondona1998} \\
\ea [] {\gll i-ilew-\textbf{wɡi} ñi n-atarenataɢanaɢaki     \\
 {\Third.\In}-die-{\Inside} {\Deicnex} {\Abs}-hospital\\
\glt `He died \juarezemph{in the hospital}.'}
\ex [*] { \gll i-ilew-\textbf{wɡi} ke ñi n-atarenataɢanaɢaki    \\
 {\Third.\In}-die-{\Inside} {\Obl} {\Deicnex} {\Abs}-hospital\\
\glt `He died in the hospital.'}
\z
\z



\subsection{Combination with two-place predicates}\label{sec:combination-two-place}
The compatibility of spatial markers is not limited to one-place predicates; two-place predicates also take these markers. This section explores the specific function that spatial markers serve with this type of predicate. However, it is important to note that a much larger sample of two-place predicates is needed. Therefore, what is presented below is merely a preliminary examination of the combination of spatial markers and bivalent predicates. 

Let us begin with a typical two-place predicate like \textit{-waɢan} `hit'. As described before, a predicate of this kind typically indexes the A argument in the verb complex, but not the P argument. This is illustrated in (\ref{ex:cut-contrast}), where the prefix \textit{i-} indexes the A argument, whereas the nominal P argument is expressed by \textit{na lanaɢaʔ} `his/her nape'. In combination with \textit{-lek} in (\ref{ex:cut-contrast-lek}), the predicate retains the same number of arguments,  but undergoes a slight change in meaning.\footnote{Certain morphological elements do not contain a gloss because their meanings are not clear yet. This is the case of the third-person independent pronoun \textit{ɾa-maɢaɾe} in (\ref{ex:cut-contrast-lek}), where the value of \textit{-maɢaɾe} is unclear.} Specifically, we find \textit{hit} $\rightarrow$ \textit{cut}, where \textit{-lek} changes the event configuration by specifying the nature of the hitting event. The meaning of \textit{cut}, furthermore,  specializes for the domain of ``wood-cutting'', as indicated by the example translation. Such a meaning specification does not seem to occur randomly, but rather seems particular to this verb-suffix combination. This also supported by the meanings given to the verb entry \textit{ỹouaxanlec} \textit{i-waɢan-lek} in Buckwalter's dictionary: `lo derriba como a un árbol' / `He/She knocks it down like a tree', `golpea sobre' / `He/She hits on (sth.)', `corta a un árbol con hacha' / `He/She cuts a tree with an ax'.     


\ea \label{ex:cut-contrast}
 \ea (mocCA191127\_2: 00:08:58) \\
 \gll  \textbf{i}-akon-t=oʔ \textbf{so} \textbf{qopaq} kan \textbf{i}-waɢan=oʔ [...] i-waɢan-t=oʔ \textbf{na} \textbf{l-anaɢaʔ} qo-i-nak \\
        {\Third.\II}-take-?={\Evid/\Tprl?} {\DetTwo} tree {\Link} {\Third.\II}-hit={\Evid/\Tprl?} {} {\Third.\II}-hit-?={\Evid/\Tprl?} {\DetThree} {\Third.\Poss}-nape {\Sbj.\Defc}-{\Third}-say\\
        \glt `...(the tiger) took the stick and hit (Katalina) [...] hit the nape of her head, they say...' 
        
\ex 
(mocCA120706: 00:07:22) \\
\gll  ɾa-maɢaɾe i-waɢan-\textbf{lek} n-qopaɢ \\
{\DetFour}-maɢaɾe {\Third.\II}-hit-{\LocOne} {\Indet.\Poss}-wood \\
\glt `He/She \juarezemph{cuts firewood}.' (Lit. He hits on the wood.) \label{ex:cut-contrast-lek}
\glt 
\z
\z

Moreover, when the same two-place predicate occurs with the spatial marker \textit{-ɡe}, the event description changes in that the P argument is conceptualized differently from the previous example. The spatial notion of `distant location', typically expressed by this suffix, is retrieved here by the separation of portions or pieces corresponding to the P entity. Similar to the use of \textit{-lek} above, the expression of \textit{-ɡe} here modifies the event description, but does not change the basic two-place argument frame of the predicate. 

\ea Southern Chaco Mocoví \citep[in the digital version of][153]{Buckwalter1995} \\
\gll i-owaɢan-aʔ-\textbf{ɡe} \\
  {\Third.\II}-hit-{\Trvz?}-{\LocTwo} \\
  \glt `Lo descuartiza.'
  \glt `She/he \juarezemph{chop it up}.' \label{ex:cutpieces}
\z

A slightly different function is documented for the suffix \textit{-ɡi}, which contributes to the categorization of the P argument by referring to the shape-related property of the entities denoted by the predicate. In the following example, drawn from a traditional story (\ref{ex:hitinside}), the objects being hit here correspond to cylindrical elements or can simply be related by the container-like shape shared by all of them. Here again we do not find an extension of arguments, but rather a semantic specification of the basic P argument of this predicate.    

\ea
(mocC191022\_2: 00:09:12) \\
\gll so ʔwe naʔɢa peɡet jokoʔ weːna qa-i-waɢan-taʔ-pe-\textbf{ɡi} \\
 {\DetTwo} {\Exist} day plate and.also pot {\Sbj.\Defc}-{\Third.\II}-hit-{\Prog}-{\Pl.\Arg}-{\LocFour} \\
\trans `...they had plates and pots and hit \juarezemph{them} (to make noise)...' \\
\label{ex:hitinside}
\z

The function of spatial markers as in (\ref{ex:cut-contrast})--(\ref{ex:hitinside}), resembles in great deal cases of  verb-classifier combinations in, for example, Southwestern Amazonian languages. \citet[][215]{vandervoort2018} reported that when this particular combination occurs ``[classifiers] may represent incorporated arguments and at the same time function as modifiers, creating a derived verb stem or a noun''.  In what the author calls ``incorporated arguments'', classifiers may categorize a predicate argument in ``terms of shape, consistency, size, structure, animacy and position'' \citep[][217]{vandervoort2018}. Although it is not claimed here that locative markers are ``incorporated arguments'', the classification function  of \textit{-ɡi} in (\ref{ex:hitinside}) is worth noticing. Furthermore, the modification function of spatial markers appears to be recurrent with certain types of both two-place and one-place predicates (see \S\ref{sec:one-place}).

Finally, the combination of \textit{-waɢan} `hit' with \textit{-ɡit} in (\ref{ex:hitagainst}) demonstrates that this spatial marker functions slightly differently from the functions attested for the other spatial markers. In this case, \textit{-ɡit} neither modifies the verb predication nor categorizes the P argument. Instead, it introduces a locative participant against which the P argument is hit. The hitting event can be further described as a ``caused movement", in which the \textsc{figure}, \textit{so qatjaleqa} `your children', is brought into contact with the \textsc{ground}. 

\ea 
(mocC191025: 00:15:55) \\
\gll qamiɾ ɾ-owaɢan-iɾ-\textbf{ɡit} \textbf{qopaɢ} so qat-jale-qa\\
{\Second.\Pron} {\Second\Sg.\II}-hit-{\Second\Sg.\II}-{\LocThree} tree {\DetTwo} {\Second.\Poss}-descendent-{\Pl}\\
\glt `...(the armadillo said): you just hit your children \juarezemph{against the tree} (to get the honey)...' \label{ex:hitagainst}
\z

Similar to what has been shown in \S\ref{sec:one-place}, the examples in this section further illustrate that one single function of spatial markers cannot account for their multifunctionality. We have seen that spatial markers exhibit variation in their function both within individual predicates and across different predicates. 

\section{Discussion}\label{sec:discussion}
The previous sections have shown the range of functions that spatial markers have within the domain of existential constructions and verbal constructions, distinguished by verb types. The sections below summarize the patterns of functions attested for spatial markers and discusses their applicative function in light of typological criteria employed to recognize such a function. 

\subsection{Patterns of functions}


\begin{table}[t]
\caption{Overview of the distribution and attested functions of spatial makers.}
\label{tab:summary-distribution}
\fittable{
    \begin{tabular}{llll}
     \lsptoprule
         {lexical element} &  {locative type} &  {semantics} &  {valence frame} \\
         \midrule
        postv.existential & \textsc{na} & \textsc{na} & \textsc{na}\\
        neg.existential & -ɡi: {\LocFour} & S:containment & \textsc{na} \\
        be\textsubscript{4}-type & -lek: {\LocOne} & S\_LOC:top & basic \\
        be\textsubscript{4}-type & -ɡe: {\LocTwo} & S\_LOC:distant\_location & basic \\
        be\textsubscript{4}-type & -ɡit: {\LocThree} & S\_LOC:contiguity & basic \\
        be\textsubscript{4}-type &  -ɡi: {\LocFour} & S\_LOC:containment & basic \\
        \tablevspace
        be\textsubscript{1}-type & -lek: {\LocOne} & S\_LOC:top & \textbf{extended\_+1} \\
        be\textsubscript{1}-type & -ɡe: {\LocTwo} & S\_LOC:distant\_location & \textbf{extended\_+1} \\
        %be\textsubscript{1}-type & -ɡit: {\LocThree} & S\_LOC\_contiguity & \textbf{extended\_+1} \\
        be\textsubscript{1}-type &  -ɡi: {\LocFour} & S\_LOC:containment & \textbf{extended\_+1} \\
        move-type &  -lek: {\LocOne}  & A\_P:animate & \textbf{extended\_+1} \\
        move-type & -ɡe: {\LocTwo} & A\_P:distant\_location & \textbf{extended\_+1} \\
        move-type &  -ɡit: {\LocThree} & A\_P:animate\_contiguity & \textbf{extended\_+1} \\
        move-type &  -ɡi: {\LocThree} & A\_P:animate\_containment & \textbf{extended\_+1} \\
        \tablevspace
        be/get.dry-type & -lek: {\LocOne} & S:top & basic \\
        be/get.dry-type &  -ɡe: {\LocTwo} & S:distant\_location & basic \\
        be/get.dry-type & -ɡit: {\LocThree} & S:contiguity & basic \\
        be/get.dry-type & -ɡi: {\LocFour} & S:containment & basic \\
        be/get.cracked & -ɡe: {\LocTwo} & S:side & basic \\
        be/get.cracked & -ɡi: {\LocThree} & S:completely? & basic \\
        hit-type &  -lek: {\LocOne} & A\_P:upper & basic \\
        hit-type &  -ɡe: {\LocTwo} & A\_P:separation & basic \\
        \tablevspace
        hit-type & -ɡit: {\LocThree}  & A\_P\_LOC & \textbf{extended\_+1} \\
        \tablevspace
        hit-type & -ɡi: {\LocFour} & A\_P:container & basic \\
        \lspbottomrule
    \end{tabular}
    }
\end{table}
Distributionally, spatial markers combine with existential units, one-place predicates and two-place predicates; semantically, furthermore, their effect on the verb predication is quite varied. Building on these two parameters, i.e., distribution and semantic effects on predicates, function types of spatial markers, organized by types of verbs, can be grouped (see \tabref{tab:summary-distribution} below). Spatial markers refer to a locative participant rather than the S argument with predicates expressing static location. There is also a particular semantic implication when spatial markers, except for \textit{-ɡe}, occur with the move-type predicate. They express an animate participant and such an implied participant changes the meaning of the basic movement event; see examples in  (\ref{ex:movementbase}). Importantly, when combined with the group of state/change-of-state monovalent predicates, spatial markers indicate certain physical  properties of the single argument. Finally, spatial markers typically contribute to the expression of the P argument with a bivalent predicate, except for \textit{-ɡit} which adds a locative participant. 

\subsection{The applicative issue}
Another important issue for this paper is the applicative function of spatial markers. Recall that  some of the scholars working on different variants of the language have categorized spatial markers as valence increasing elements (see \S\ref{sec:prior-work}). The data discussed in this paper show that their valence extension function is just one of the many other properties of these spatial markers. Here I discuss the parameters that help us identifying the valence-related function of spatial markers, building on recent typological work on applicativization.  


\citet{zuniga.creissels2024} propose three parameters to define applicative constructions. First,  an overtly marked contrast between a basic and the argued applicative constructions should be found. Contrasts of this type are identified with `be\textsubscript{1}', `be/get.dry', `move' and `hit' verb types. In the other cases, finding a spatial-marker-free verb root is not possible, due to the ``empty" lexical nature of those roots, which need spatial markers to narrow down the verbal event. 

Second, these authors propose that participants expressed as S and A should be identically encoded in both basic and applicative constructions. This parameter presents some challenges for Mocoví due to the language-internal properties related to valence increment. As discussed in \S\ref{sec:valency-test}, the group of predicates taking \textit{ɾ-} for the third-person S argument typically shifts to \textit{i-} when the valence increases via causativization, e.g. (\ref{ex:typeIaltern}). Thus, for this group of predicates, a structural indicator of the valence change directionality exists. However, changes in the expression of third-person intransitive \textit{ɾ-} arguments are not triggered by spatial markers. This suggests that if valence increase is assumed for spatial markers, it manifests differently from the other types of morphologically marked valence changes. 

An alternative interpretation of the same phenomenon could align the function of spatial markers with true applicatives, however. If we assume that the alternation in the morphological expression of third-person S and A is modified solely by valence alternation targeting the ``subject'' argument,  it follows that spatial markers do not necessarily interact with the expression of such an argument. Spatial markers only contribute to  the addition of a non-subject argument. If this line of reasoning is taken to be more appropriate for this language, spatial markers could be considered as types of applicatives, but only within those specific contexts that have been discussed above.   

Finally, another parameter in the definition of applicatives relates to the addition of a noun phrase in the applicative construction, specifically referring to non-core S or A arguments. This applied phrase should be expressed differently from its counterpart in the non-applicative construction or may not be expressed at all in the unmarked basic construction. In principle, the criterion of noun phrase expression identifies spatial markers as true applicatives in Mocoví, particularly when they occur with both `be\textsubscript{1}' and `move' predicates. The individual use of \textit{-ɡit} with the hit-type predicate should also be included here, as an overt nominal element is allowed in the spatially marked verb (note the notation `extended\_+1' in the \textsc{valence frame} column in \tabref{tab:summary-distribution}). However, it should be noted that the addition of spatial markers naturally implies a new nominal participant, even thought this nominal participant does not necessarily need to be overtly encoded to form acceptable verb clauses in Mocoví. As mentioned in \S{\ref{sec:basics-grammar}}, the overt expression of third person P arguments in basic clauses is not obligatorily required to build a  grammatical clause. 

\subsection{Argument extension in valence-derived predicates}\label{sec:argument-extension}
Morphological markers discussed in \S\ref{sec:valency-test}, which increase or decrease verb valence, can also be stacked to modify the verb valence profile according to the specific function of each suffix. The following examples illustrate how these suffixes combine with the monovalent root \textit{-aʃiwi} `be/get dry' and the bivalent root \textit{-waɢan} `kill'. With the monovalent root,  \textit{-ɢat} increases the valence, while \textit{-ɢan} subsequently reduces the previously derived base, as in (\ref{ex:valen-change-dry}). With a bivalent root, only \textit{-ɢan} is employed, reducing the number of arguments in a way similar to antipassives.  

The morphologically derived stem \textit{-aʃiwi-ɢat-ɢan} corresponds to an intransitivized predicate, as indicated by the intransitive argument indexing \textit{ɾ-}.\footnote{In intransitivized verb stems of this type, the expression of the former P argument corresponding to the stem \textit{-aʃiwi-ɢat} is possible only if the argument corresponds to a plural referent or has a generic reading; the latter is typically accompanied by the omission of nominal determiners, see \citet[][]{juarez.alvarez-gonzalez2021}.} On this intransitivized structure, spatial markers extend the semantic number of participants entailed in the predication \citep[the notion of ``extension'' follows][3--4]{Dixon2000}, giving rise to an  ``ambitransitive'' or ``semitransitive'' construction \citep[see][]{Dryer2007,Zuniga2017}. 
\ea
(mocC191108: 00:34:51) \\
  \gll   so jale t-a-ɡe ɾa lai ɾ-aʃiwi-\textbf{ɢat}-\textbf{aɢan}\\
    {\DetTwo} man {\Third.\II.\Intr}-go-{\LocTwo} {\DetThree} side {\Third.\II.\Intr}-be/get.dry-{\Caus}-{\Vm:\Intr}\\
    \glt  `The man goes \juarezemph{to the other side} to dry (something).' \\
    \glt \label{ex:valen-change-dry}
\z

The extended participant, however, bears different semantic roles depending on the spatial marker that is employed and, in some cases, specifies the disposition of entities in the predication. In (\ref{ex:valen-change-lek}), for example, \textit{-lek} expresses a \textsc{malefactive} participant, but in (\ref{ex:valen-change-lek-mal}) it encodes a \textsc{benefactive} \citep[see also][75--76]{juarez2023}. This function of \textit{-lek} also highlights the animacy feature of the newly introduced third-person participant, resembling the case discussed in \S\ref{sec:movement-pred} with a movement predicate. In both cases, the implication of an animate, typically human, participant  arises naturally.


\ea
(mocC191107: 00:37:29) \\
\gll   so pjoɢonaɢ ɾ-aʃiwi-ɢat-ɢan-\textbf{lek} so liːja\\
    {\DetTwo} sorcerer {\Third.\II.\Intr}-be/get.dry-{\Caus}-{\Vm:\Intr}-{\LocOne} {\DetTwo} another \\
\glt `The sorcerer cast a spell \juarezemph{on someone} to become ill (Lit. to make him/her dry).' \label{ex:valen-change-lek}
\z

\ea 
(mocCA160725: 00:36:04) \\
\gll so i-taʔa ajim ɾ-alawat-ɢan-\textbf{lek} waqaeʔ \\
{\DetTwo} {\First\Sg.\Poss}-father {\First\Sg.\Pron} {\Third\Intr.\II}-kill-{\Vm:\Intr}-{\LocOne} chicken\\
\glt `My father killed a chicken \juarezemph{for me}.' \label{ex:valen-change-lek-mal}
\z

 A locative participant is added to the intransitivized predicate by means of \textit{-ɡe}, as in (\ref{ex:valenc-change-ge}). As with all the other uses of this suffix, the entailment of an inanimate distant location is quite regular, serving as the primary semantic component that coherently unifies the multiple instances described for this suffix. 


\ea
(mocC191108: 00:30:06) \\
\gll a-so ʔaːlo ɾ-aʃiwi-ɢat-ɢan-\textbf{ɡe} \\
 {\F}-{\DetTwo} woman  {\Third.\II.\Intr}-be/get.dry-{\Caus}-{\Vm:\Intr}-{\LocTwo}  \\
\glt `The woman made someone else to dry (something) \juarezemph{somewhere else}.' \\\glt \label{ex:valenc-change-ge}
\z

Finally, this intransitivized stem can also occur with \textit{-ɡit}. The argument extension effect in this case involves reintroducing into the verb  predication the P argument that was demoted by the intransitivization process. Additionally, the spatial marker specifies the disposition of entities entailed by the predicate. In (\ref{ex:valen-change-git}), \textit{-ɡit} specifically indicates that objects to be dried up are close to one another rather than being randomly distributed in space. 

\ea
(mocC191108: 00:50:19) \\
\gll   so jale ɾ-aʃiwi-ɢat-ɢan-\textbf{ɡit} \\
 {\DetTwo} man  {\Third.\II.\Intr}-be/get.dry-{\Caus}-{\Vm:\Intr}-{\LocThree}   \\
\glt `The man dries \juarezemph{mixed things}.' \label{ex:valen-change-git}
\z

The paradigmatically related near minimal pairs in  (\ref{ex:valen-change-dry})--(\ref{ex:valenc-change-ge}) do not include the combination of the same derived stems with \textit{-ɡi}. Further elicitation tasks are needed to fill this gap in the dataset. The prediction is, however, that the addition of \textit{-ɡi} would introduce a new participant, characterized as a containment object if the referent is inanimate, or aligning more closely with the notion of a group if  the entailed participant is animate.    

\section{Conclusions}\label{sec:conclusions}

Building on previous studies on Mocoví, this paper analyzed the multifunctionality of spatial markers  \textit{-lek} `{\LocOne}', \textit{-ɡe} `{\LocTwo}’, \textit{ɡit} `{\LocThree}' and \textit{ɡi} ‘{\LocFour}’, using a combination of elicited and natural speech data from Northern Chaco Mocoví, as well as data from the Mocoví documented in different regions of northeastern Argentina.  The multifunctionality of these spatial markers highlights their versatility  with respect to their distribution and their meaning contributions to a broad range of constructions, including existentials and different types of verbal constructions. Due to their synchronic multifunctional nature, defining spatial markers by only one of their functions is misleading and fails to do justice to their functional richness.

Broadly speaking, three major functions serve to organize the use of spatial markers:

\begin{enumerate}[label=(\roman*)]
    \item locative relators between a \textsc{figure} and its \textsc{ground},
    \item descriptors of physical properties of a single argument in state/change-of-state predicates,
    \item addition of a new non-S or non-A argument, with semantic roles varying according to the spatial marker selected.
\end{enumerate}

If we were to determine which of these functions might represent the most fundamental one from which the others depart, we could argue that the figure-ground function is the strongest candidate. This function corresponds to the most concrete spatial relationship, subtly present in the other functions of these elements.

One typologically significant property of spatial markers in Mocoví is their ability to extend verb valence. This function appears more abstract than the other two and can be interpreted as a metaphorical extension of locative participants into new arguments. This valence extension property seems to have developed further with certain spatial markers, which now primarily refer to animate participants. The use of spatial marking morphology to increase verb valence has long been recognized by scholars studying Mocoví and other Guaycuruan languages. However, these findings gain typological relevance when considered alongside recent comparative studies on applicativization, which reveal a cross-linguistically uncommon link between spatial marking and the applicative domain \citep{VanLinden2022,Kohlberger2022}. 

A more extensive examination of spatial marker distributions and their semantic effects on verbal and non-verbal predications is necessary to better understand the functional range of these markers within Mocoví grammar. Additionally, such research will contribute to previous  works on a similar topic within Guaycuruan languages \citep[e.g.][]{Censabella2024,gonzalez2010,carrio2013} and advance future comparative investigations.   

\section*{Abbreviations}
\begin{tabularx}{.45\textwidth}{lQ}
\textsc{ac} & active \\
\textsc{act} & action \\
\textsc{arg} & argument \\
\textsc{asp} & aspect \\
\textsc{attrb} & attributive \\
\textsc{defc} & defocusing \\
\textsc{det}\textsubscript{one} & determiner 1: no visible \\
\textsc{det}\textsubscript{two} & determiner 2: going, far \\
\textsc{det}\textsubscript{three} & determiner 3: coming, close \\
\textsc{det}\textsubscript{four} & determiner 4: static, vertical \\
\textsc{det}\textsubscript{five} & determiner 5: static, horizontal \\
\textsc{det}\textsubscript{six} & determiner 6: static, sitting \\
\textsc{deic}\textsubscript{g} & deictic root: going \\
\textsc{deic}\textsubscript{nex} & deictic root: non-extended \\
\textsc{dir} & directional \\
\textsc{dir}\textsubscript{ch} & direction of change \\
\textsc{dir}\textsubscript{down} & directional: downwards \\
\textsc{dir}\textsubscript{in} & directional: inwards \\
\textsc{dir}\textsubscript{out} & directional: outwards \\
\textsc{dir}\textsubscript{up} & directional: upwards \\
\end{tabularx}
\begin{tabularx}{.45\textwidth}{lQ}
\textsc{down} & downwards \\
\textsc{evid} & evidential \\
\textsc{exist} & existential \\
\textsc{iii} & set III bound person form \\
\textsc{ii} & set II bound person form \\
\textsc{i} & set I bound person form \\
\textsc{in} & inactive \\
\textsc{indet} & indeterminate \\
\textsc{inside} & locative-directional: inside \\
\textsc{ints} & intensifier \\
\textsc{link} & linker \\
\textsc{loc}\textsubscript{one} & locative type 1 \\
\textsc{loc}\textsubscript{two} & locative type 2 \\
\textsc{loc}\textsubscript{three} & locative type 3 \\
\textsc{loc}\textsubscript{four} & locative type 4 \\
\textsc{out} & outwards \\
\textsc{pron} & pronominal \\
\textsc{rec} & recipient \\
\textsc{tprl} & temporal \\
\textsc{trvz} & transitivizer \\
\textsc{up} & upwards \\
\textsc{val} & valence \\
\textsc{vm} & valence modifier \\
\\
\end{tabularx}

\il{Mocoví|)}
\sloppy
\printbibliography[heading=subbibliography,notkeyword=this]
\end{document}
