\documentclass[output=paper,colorlinks,citecolor=brown]{langscibook}
\ChapterDOI{10.5281/zenodo.16838056}
\author{Martin Haspelmath\orcid{}\affiliation{Max Planck Institute for Evolutionary Anthropology}}
%\ORCIDs{}

\title{Construction-functions versus construction-strategies}

\abstract{This paper discusses the contrast between two modes of typological comparison of grammatical constructions, following a distinction made by \citet{Croft2022}: We may start out from (construction-)functions and consider the diverse (construction-)\linebreak strategies that languages use, or we may focus on the strategies themselves (e.g. reflexive constructions, passives, copula constructions). Both kinds of comparative concepts are widely employed, but our traditional grammatical terms are often used in different ways. Here I suggest that we can avoid confusions and arrive at clearer terminologies if we take this distinction into account, and I discuss the concrete example of the term \textit{existential construction} (best regarded as a construction-function, not a kind of construction-strategy).}

\IfFileExists{../localcommands.tex}{
  \addbibresource{../localbibliography.bib}
  % add all extra packages you need to load to this file

\usepackage{tabularx,multicol}
\usepackage{url}
\urlstyle{same}

\usepackage{listings}
\lstset{basicstyle=\ttfamily,tabsize=2,breaklines=true}

\usepackage{langsci-basic}
\usepackage{langsci-optional}
\usepackage{langsci-lgr}
\usepackage{langsci-osl}
% \usepackage{./langsci/styles/langsci-lgr}
% \usepackage{./langsci/styles/langsci-osl}
% \usepackage{langsci-gb4e}

\usepackage{tikz}
\usetikzlibrary{patterns,calc}
\pgfdeclarepatternformonly{south east lines}{\pgfqpoint{-0pt}{-0pt}}{\pgfqpoint{3pt}{3pt}}{\pgfqpoint{3pt}{3pt}}{
    \pgfsetlinewidth{0.6pt}
    \pgfpathmoveto{\pgfqpoint{0pt}{3pt}}
    \pgfpathlineto{\pgfqpoint{3pt}{0pt}}
    \pgfpathmoveto{\pgfqpoint{.2pt}{-.2pt}}
    \pgfpathlineto{\pgfqpoint{-.2pt}{.2pt}}
    \pgfpathmoveto{\pgfqpoint{3.2pt}{2.8pt}}
    \pgfpathlineto{\pgfqpoint{2.8pt}{3.2pt}}
    \pgfusepath{stroke}}
    
\usepackage{stmaryrd}
\usepackage{wasysym}
\usepackage{multirow}
\usepackage{caption}
\usepackage{subcaption}
\usepackage{mathrsfs}
\usepackage{qtree}

\usepackage{linguex}


  %pminos do not split footnotes
% \interfootnotelinepenalty=10000 %Footnote in Laporte chapters has to be split SN


%\DeclareIndexNameFormat{default}{%
%\nameparts{#1}%
%\usebibmacro{index:name}%
%{\index[names]}%
%{\namepartfamily}%
%{\namepartgiveni}%
% {}% L1
% {}% L2
%{\namepartprefix}% generates spurious space L3
%{\namepartsuffix}% generates spurious space L4
%}

%  {\DeclareIndexNameFormat{default}{%
%     \usebibmacro{index:name}{\index[names]}{#1}{#3}{#5}{#7}}}

%\DeclareIndexNameFormat{default}{%
%  \usebibmacro{index:name}{\sindex[nom]}{#1}{#3}{#5}{#7}}

%\DeclareIndexNameFormat{default}{%
%  \usebibmacro{index:name}{\sindex[person]}{#1}{#3}{#5}{#7}}
%\DeclareIndexNameFormat{default}{%
%\nameparts{#1} \usebibmacro{index:name}{\sindex[person]]}{\namepartfamily}{‌​\namepartgiven}{\nam‌​epartprefix}{\namepa‌​rtsuffix}}

%\newcommand{\smiley}{:)}

%\renewbibmacro*{index:name}[5]{%
%\usebibmacro{index:entry}{#1}%
%{\iffieldundef{usera}{}{\thefield{usera}\actualoperator}\mkbibindexname{#2}{#3}{#4}{#5}}}

% \newcommand{\noop}[1]{}

%remove for final
%\overfullrule=1mm

\newcommand{\tobi}[2]}}
\renewcommand{\S}[1]{\tobi{#1}{\textsc{*}}}

% this volume references
% puts: [this volume]
% already defined: \citetv
%\newcommand{\citepv}[1]{(\citeauthor{#1} \citeyear*{#1} [this volume])}
\newcommand{\citealtv}[1]{\citeauthor{#1} \citeyear*{#1} [this volume]}

%parentheses around example number
\newcommand{\pref}[1]{(\ref{#1})}

% in-text examples

\newcommand{\lnex}[1]{\textit{#1}} %target lang word
\newcommand{\lnlit}[1]{(lit.: `#1')} %literal reading
\newcommand{\lnlat}[1]{(#1)} % latinization
\newcommand{\lntrans}[1]{`#1'} %translation
\newcommand{\lnexl}[2]%
{\lnex{#1}{} \lnlat{#2}} % ex with latinization
\newcommand{\lnexlat}[3]{\lnex{#1}{} \lnlat{#2}{} \lntrans{#3}} % ex with latinization and tranl.

%ch01
\newcommand{\co}[1]{\mbox{\textbf{#1}}}

%ch09

\newcommand{\cyrbulg}[1]{\begin{otherlanguage*}{bulgarian}#1\end{otherlanguage*}}


%ch10
\newcommand{\nlp}{{\small NLP}}
\newcommand{\mwe}{{\small MWE}}
\newcommand{\rae}{{\small RAE}}
\newcommand{\lvc}{{\small LVC}}
\newcommand{\pos}{{\small P}o{\small S}}
%\newcommand{\todo}[1]{ \textcolor{red}{#1} }

%\renewcommand{\labelenumi}{\theenumi}
%\ainamefmt{{vv}{ll}{, ff}{, jj}} % fullname

\newcommand{\biberror}[1]{{\color{red}#1}}

\newcommand{\osenovaitem}{--~}
  %% hyphenation points for line breaks
%% Normally, automatic hyphenation in LaTeX is very good
%% If a word is mis-hyphenated, add it to this file
%%
%% add information to TeX file before \begin{document} with:
%% %% hyphenation points for line breaks
%% Normally, automatic hyphenation in LaTeX is very good
%% If a word is mis-hyphenated, add it to this file
%%
%% add information to TeX file before \begin{document} with:
%% %% hyphenation points for line breaks
%% Normally, automatic hyphenation in LaTeX is very good
%% If a word is mis-hyphenated, add it to this file
%%
%% add information to TeX file before \begin{document} with:
%% \include{localhyphenation}
\hyphenation{
    Beck-man
    Ngu-yen
    back-chan-nel
    back-chan-nels
    mo-not-o-nous
    ste-reo-typ-i-cal
}

\hyphenation{
    Beck-man
    Ngu-yen
    back-chan-nel
    back-chan-nels
    mo-not-o-nous
    ste-reo-typ-i-cal
}

\hyphenation{
    Beck-man
    Ngu-yen
    back-chan-nel
    back-chan-nels
    mo-not-o-nous
    ste-reo-typ-i-cal
}

  \boolfalse{bookcompile}
  \togglepaper[2]%%chapternumber
}{}

%\pretocmd{\gll}{\def\eachwordone{\itshape}\def\eachwordtwo{\normalfont}}{}{}

\begin{document}
\maketitle

\section{Function-based comparison and form-based comparison} \label{HaspelmathSec1}

Cross-linguistic comparison in morphosyntax is based on comparing similar constructions across languages, and these similarities may be \textsc{functional} or \textsc{formal}. For example, the two sentences in (\ref{Haspelmath1a}) and (\ref{Haspelmath1b}) are functionally equivalent and thus comparable, even though they are rather different in terms of their form: First, the locative expression `the table' comes at the end in English but at the beginning in German; second, English has an extra (prolocative) element \textit{there}; and third, German uses the posture verb \textit{liegen} `lie' instead of the simple \textsc{be}-verb. Comparable constructions may also be similar both functionally and formally. For example, the two possessive sentences from Latin and Russian in \REF{Haspelmath2a} and \REF{Haspelmath2b} are formally quite similar in that the predicative possessor `I' is expressed as an oblique nominal (\textit{mihi}, \textit{u menja}) and there is a copula (\textit{est}, \textit{est'}). In this way, these two languages differ markedly from English, whose counterpart construction (\textit{I have a house}) has a transitive \textsc{have}-verb and a predpossessor in the nominative case.

\ea \label{Haspelmath1}
\textsc{Existential clauses}\\
\ea \label{Haspelmath1a}
\textit{There is a book on the table}.\\
\ex \label{Haspelmath1b}
\langinfo {German}{Indo-European, [stan1295]}{personal knowledge}\\
\gll Auf dem Tisch liegt ein Buch.\\
on the table lies a book\\
\glt `There is a book on the table.'\\
\z
\z

\ea \label{Haspelmath2}
\textsc{Oblique predpossessor clauses}\\
\ea \label{Haspelmath2a}
\langinfo{Latin}{Indo-European, [lati1261]}{personal knowledge}\\
\gll Mihi est domus.\\
me.\textsc{dat} is house\\
\glt `I have a house.'\\
\ex \label{Haspelmath2b}
\langinfo {Russian}{Indo-European, [russ1263]}{personal knowledge}\\
\gll U menja est' kniga.\\
at me is book\\
\glt `I have a book.'\\
\z
\z

We can thus distinguish two kinds of construction types: function-based construction types or \textsc{construction-functions}, and form-based construction types or \textsc{construction-strategies}. This distinction is conceptually obvious, but it has only recently been highlighted by \citet{Croft2016, Croft2022}.\footnote{Croft uses the term \textit{construction} for construction-functions, and the term \textit{strategy} for construction-strategies. I discuss this terminological difference briefly at the beginning of the concluding \sectref{HaspelmathSec8}.} This paper provides further general discussion, with some emphasis on locative and existential clauses (see \sectref{HaspelmathSec6} below). It should always be remembered that the concepts that I discuss here are comparative concepts, not categories of particular languages. My focus is on general linguistics, and hence on comparison, and while language-particular description is often enlightened by insights of general linguistics \citep{Haspelmath2020}, comparative concepts are special because they are defined in a language-independent way. The term \textit{construction} is usually defined as an element of a particular language \citep{Haspelmath2023Construction}, but in a comparative context, we are dealing with construction types.

Construction-functions and construction-strategies have also been called \textit{purely functional} and \textit{hybrid} comparative concepts, respectively. Construction-strategies are not purely formal (but hybrid) because they are defined as formal ways of expressing a particular meaning or function. For example, the oblique predpossessor strategy in \REF{Haspelmath2} is a way of expressing the predicative possession meaning, and it expresses only this meaning. (Purely formal comparative concepts exist as well, but mostly in phonology.\footnote{The three main purely formal comparative concepts relevant for morphosyntax are (i) precedence (A occurs before B), (ii) boundness (A does not occur in isolation), and (iii) overtness (meaning M is expressed by a form rather than by zero). Even with these, one might wonder whether they are phonological rather than morphosyntactic properties.\label{footnote2}}) By contrast, existential clauses are defined purely functionally, and can thus take diverse forms, as seen in \REF{Haspelmath1}.

While the distinction between construction-functions and construction-strategies is clear in principle, there is often terminological unclarity, because traditional grammatical terms are typically used both for functions and for strategies \citep[382]{Croft2016}. Ideally, we would always be fully explicit about the way a term is used, but this is not always practical. Thus, it is important that linguists are aware of the distinction, and of the potential polysemy of traditional terms.

\section{Construction-functions and construction-strategies in typological research} \label{HaspelmathSec2}
\largerpage
To give readers a better feel for the two kinds of construction types, I will simply list ten construction types for both, accompanied by corresponding well-known comparative studies.

\ea \label{Haspelmath3}
\textsc{Examples of construction-functions}\\
\ea \label{Haspelmath3a}
polar question constructions \citep{Dryer2005Polar}\\
\ex \label{Haspelmath3b}
comparative constructions \citep{Stassen1985, Stassen2005}\\
\ex \label{Haspelmath3c}
predpossessive constructions (\cite{Heine1997}; \cite{Stassen2009})\\
\ex \label{Haspelmath3d}
ditransitive constructions \citep{MalchukovEtAl2010}\\
\ex \label{Haspelmath3e}
reciprocal constructions \citep{Nedjalkov2007}\\
\ex \label{Haspelmath3f}
imperative constructions \citep{Aikhenvald2010}\\
\ex \label{Haspelmath3g}
causal-noncausal constructions \citep{HaspelmathEtAl2014}\\
\ex \label{Haspelmath3h}
standard negation constructions \citep{Dryer2013WALS}\\
\ex \label{Haspelmath3i}
adnominal possessive constructions \citep{Koptjevskaja-Tamm2002}\\
\ex \label{Haspelmath3j}
coordination constructions \citep{Mauri2008}\\
\z
\z

\ea \label{Haspelmath4}
\textsc{Examples of construction-strategies}\\
\ea \label{Haspelmath4a}
incorporation constructions (e.g. \cite{Olthof2020})\\
\ex \label{Haspelmath4b}
ergative constructions (e.g. \cite{Coon2017})\\
\ex \label{Haspelmath4c}
serial verb constructions (e.g. \cite{Haspelmath2016Serial})\\
\ex \label{Haspelmath4d}
causative constructions (e.g. \cite{Comrie1975})\\
\ex \label{Haspelmath4e}
antipassive constructions (e.g. \cite{JanicWitzlack2021})\\
\ex \label{Haspelmath4f}
person indexing constructions (e.g. \cite{Siewierska2005})\\
\ex \label{Haspelmath4g}
applicative constructions (e.g. \cite{ZunigaCreissels2024})\\
\ex \label{Haspelmath4h}
reflexive constructions (e.g. \cite{JanicEtAl2023})\\
\ex \label{Haspelmath4i}
switch-reference constructions (e.g. \cite{Roberts2017})\\
\ex \label{Haspelmath4j}
reduplication constructions (e.g. \cite{HurchMattes2005})\\
\z
\z

These lists make it clear that it is not true, as has sometimes been said, that typological comparison in morphosyntax must be based entirely on semantic and pragmatic function (I said this myself, erroneously, in \citealt[126]{Haspelmath2007Categories}). The construction types listed in \REF{Haspelmath4a}--\REF{Haspelmath4j} are all at least partially defined in formal terms, though they also have a functional component (they are hybrid comparative concepts, as mentioned above).\footnote{An exception are reduplication constructions \REF{Haspelmath4j}, which are defined in purely formal terms. They are thus more similar to the purely formal ``prefixing construction'', or ``zero-marking constructions'' (see \ref{footnote2}). But while one would not carry out typological studies of such extremely broad domains, typological studies of reduplication make more sense (compare also ``clitic constructions'' \citep{Haspelmath2023Clitics} and ``infixing constructions'' \citep{Yu2007}, which are defined purely formally and have been studied separately).}

Cross-linguistic research on construction-functions systematically asks different questions than cross-linguistic research on construction-strategies. We can illustrate this by contributions to the \textit{World atlas of language structures}. For a construction-function, a typology may ask about types of strategies, e.g.

\ea \label{Haspelmath5}
\textsc{examples of construction-strategies}\\
\ea \label{Haspelmath5a}
strategies for polar questions \citep{Dryer2005Polar, Dryer2005Position}\\
\ex \label{Haspelmath5b}
strategies for ditransitive constructions \citep{Haspelmath2005Ditransitive}\\
\ex \label{Haspelmath5c}
strategies for causal-noncausal constructions \citep{Haspelmath1993}\\
\ex \label{Haspelmath5d}
strategies for ordinal numerals \citep{StolzVeselinova2005}\\
\ex \label{Haspelmath5e}
strategies for plural number constructions \citep{Dryer2005Nominal}\\
\z
\z

The strategies sometimes concern the order of elements (e.g. preposed vs. postposed question particles in \citet{Dryer2005Position}), the presence or absence of a marker (e.g. double object constructions vs. indirective constructions in \citealt{Haspelmath2005Ditransitive}), or different marking strategies (e.g. affix vs. clitic vs. tonal marking in \citet{Dryer2005Polar}). The general question is: What kinds of forms can a given function take in different languages? In the \textit{World atlas of language structures}, we used different colours for the different strategies, usually including red and blue and yellow, as in \figref{HaspelmathFig1}.

\begin{figure}[ht!]
\includegraphics[width=\textwidth]{figures/HaspelmathFig1.png}
\caption{Polar question marking (different strategies, \cite{Dryer2005Polar})}
\label{HaspelmathFig1}
\end{figure}

By contrast, when we study strategies cross-linguistically, we cannot easily typologize them according to subtypes, because the strategies themselves are formally defined. Instead, we can study their coexpression patterns. So here, the general question is: What kinds of (other) functions or meanings can a given form have in different languages? In the \textit{World atlas of language structures}, there are a number of maps which address the question whether a form with one meaning can also have another meaning, and where yellow denotes coexpression of both meanings, while red denotes differentiation, as in \figref{HaspelmathFig2}.

\ea \label{Haspelmath6}
\ea \label{Haspelmath6a}
meanings of reciprocal markers (e.g. reflexive; \cite{MaslovaNedjalkov2005})\\
\ex \label{Haspelmath6b}
meanings of comitative flags (e.g. instrumental; \cite{StolzEtAl2005})\\
\ex \label{Haspelmath6c}
meanings of relativizers (e.g. genitive; \cite{Gil2005})\\
\ex \label{Haspelmath6d}
meanings of nominal coordinators (e.g. verbal conjunction; \cite{Haspelmath2005Conjunction})\\
\z
\z

\begin{figure}[ht!]
\includegraphics[width=\textwidth]{figures/HaspelmathFig2.png}
\caption{Comitative flags (identity or non-identity with instrumentals, \cite{StolzEtAl2005})}
\label{HaspelmathFig2}
\end{figure}

The contrast between a function-based perspective and a form-based perspective could also be described by the two terms \textit{onomasiological} (function-to-form) and \textit{semasiological} (form-to-function), though these terms are more often used outside of typology. Another way of putting it is that construction-functions are about ``what (is expressed)'', and construction-strategies are about ``how'' (it is expressed) \citep[19]{Croft2022}.

This description of the distinction between function-based typologies and strategy-based typologies is idealized: In practice, mixed types occur. An example is the classical typology of argument-coding alignment, where languages fall into the three broad types accusative, ergative, and neutral (omitting minor types for simplicity). This typology is illustrated by the map from \citet{Comrie2005} in \figref{HaspelmathFig3}, which shows flagging (or ``case marking'') of full nominals. On the one hand, this is a typology of strategies (see (\ref{Haspelmath4b}) above) for the expression of events with an actor and an undergoer, so it is a function-based typology. But on the other hand, the typology is also about coexpression patterns: The accusative type is defined not by having a special accusative flag, but by using the same flagging pattern for the transitive A-argument and the intransitive S-argument. So, this typology is also about coexpression of transitive and intransitive flagging patterns, not only about strategies for actor-undergoer events.

\begin{figure}[ht!]
\includegraphics[width=\textwidth]{figures/HaspelmathFig3.png}
\caption{Alignment of transitive flagging (accusative, ergative, neutral; \cite{Comrie2005})}
\label{HaspelmathFig3}
\end{figure}

Despite the existence of such mixed typologies, it is important to realize that classical grammatical typologizing takes somewhat different forms depending on whether it typologizes strategies used to express a given function or functions that a given strategy may have.

\section{Terms for functional-semantic concepts and for hybrid concepts} \label{HaspelmathSec3}

The distinction between construction-functions and construction-strategies is an instance of a broader conceptual distinction that is important in many domains of general grammar: the distinction between functional-semantic and hybrid comparative concepts. \citet{Croft2016} notes that there is no disagreement among linguists about the distinction between notional construction types (construction-functions) and hybrid construction types (construction-strategies), but often the terminology is not uniform:

\begin{quote}
  The disagreement appears to be basically about terminology: some authors use a traditional grammatical term for a construction, while others use it for a particular strategy found for that construction. One can usually figure that out by reading the supporting arguments. \citep[382]{Croft2016}
\end{quote}

Even though the context often makes it clear which meaning is intended, it is better to have different terms for the functional-semantic concepts and the hybrid concepts. While terms are often ambiguous (or vague) with respect to this distinction, there is also a tradition of distinguishing between these types of concepts by using different terms. \tabref{HaspelmathTab1} lists a number of cases where different terms have been proposed for functional-semantic concepts and hybrid concepts.


\begin{table}
  \begin{tabular}{ll}
  \lsptoprule
    \textsc{notional term} & \textsc{hybrid term} \\
  \midrule
    \textit{time} & \textit{tense} \\
    \textit{(biological) sex} & \textit{(grammatical) gender} \\
    \textit{speech-act role} & \textit{(grammatical) person} \\
    \textit{recipient} & \textit{dative} \\
    \textit{modifying possessor} & \textit{genitive} \\
    \textit{semantic role} & \textit{case marker} \\
    \textit{question} & \textit{interrogative clause} \\
    \textit{speech act} & \textit{sentence type} \\
    \textit{property concept} & \textit{adjective} \\
    \textit{causal} & \textit{causative} \citep{Haspelmath2016Causative} \\
    \textit{mutual} & \textit{reciprocal} \citep{Haspelmath2007Reciprocal} \\
    \textit{agent-patient coreference} & \textit{reflexive} \citep{Haspelmath2023Reflexive} \\
    \lspbottomrule
  \end{tabular}
  \caption{Contrasting notional and hybrid terms}
  \label{HaspelmathTab1}
\end{table}

Some of the distinctions in \tabref{HaspelmathTab1} are well-known and uncontroversial (e.g. \textit{time} vs. \textit{tense}, \textit{property concept} vs. \textit{adjective}, \textit{semantic role} vs. \textit{case marker}), others are not so well-known (e.g. \textit{question} vs. \textit{interrogative clause}, \textit{speech-act role} vs. \textit{person}), and some are fairly novel. In general, the distinctness of the concept types is evident to linguists, and they are often happy to adopt terminology that allows then to make the distinction. For example, I proposed the contrast between \textit{causal} and \textit{causative} fairly recently, but it has already been widely adopted (e.g. \citealt{AllassonniereTangEtAl2022}).

It should be noted that the contrast between the term pairs in \tabref{HaspelmathTab1} is more general than the contrast between construction-functions and construction-strategies that is the main focus of this paper. For example, `time' and `sex' are general semantic notions, not names of construction-functions. As \citet[17]{Croft2022} observes, functionally defined constructions express combinations of semantic categories and information-packaging functions, but \textit{time} and \textit{sex} are purely semantic categories. By contrast, `tense' and `(grammatical) gender' are hybrid concepts in the sense that they make reference to grammatical expression: Tense is the expression of temporal notions by grammatical markers, and gender is the classification of nouns in the context of grammatical agreement. While tense and gender are important for construction-strategies, we probably do not want to say that they are strategies themselves.

\newpage
\citet{Croft2016} says that in general, one should use the traditional terms for functional-semantic concepts:

\begin{quote}
  I would argue that it is better to use the basic traditional grammatical terms for a construction, not a strategy. The European grammatical tradition often did not make a distinction between construction and strategy, because the languages were typologically similar and used the same or similar strategies for the same constructions. Typology looks across all languages, of course, and finds other strategies used for various constructions. But the European grammatical terms are so deeply ingrained in usage that they are better used for the universal comparative concept, not a strategy that is restricted to a subset of languages. \citep[382]{Croft2016}
\end{quote}


The argumentation seems sound, but de facto linguists have often introduced new terms that are specifically reserved for functional-semantic concepts. While in the 1960s and 1970s, the terms \textit{case} and \textit{dative} were often used in a notional sense (e.g. \cite{Fillmore1968}; \cite{Givon1984DirectObject}), they have been largely replaced more recently by \textit{semantic role} and \textit{recipient}, respectively. Other novel terms for functional-semantic concepts, introduced in the 20th century or even more recently, are \textit{possessor}, \textit{speech act}, \textit{coreference}, and \textit{causal verb}.

One argument that is sometimes made in discussions of terminology is that grammatical terms should be transparent. For example, \citet[142]{Abraham2006} interprets \textit{ditransitive} as `doubly transitive', thus rejecting the use of the term for German Dative constructions, because in German ``the dative is never transitive''. And \citet[37]{Creissels2019} notes that an existential clause such as \textit{There is a bird on the roof} does not express existence, so he proposes a new term (\textit{inverse-locational construction}). However, this consideration will play no role here. What matters for terminology is precise definitions and (if possible) continuity with the tradition, but transparency is not a major concern.

\section{Function-based and form-based sense extensions of grammatical terms} \label{HaspelmathSec4}

Like the meanings of other words, the senses of many grammatical terms have been changing over the centuries and decades, and this has often led to polysemy and confusion within linguistics. In order to address such confusions, it is helpful to be aware that many of our traditional terms derive from the grammar of Latin and other important European languages (especially French, German and English).\footnote{In Western scholarship, apart from Latin, French was the most important language in the 18th century, German in the 19th, and English has been dominant since the 20th century.} These languages are typologically similar, so that Latin (or French, German or English) terms were often carried over to other languages without too many problems \citep[382]{Croft2016}.

But linguists do not have general (cross-linguistically applicable) definitions of these terms, so the sense extensions could go in multiple directions: A stereotypical use in a well-known language could be taken as a basis both for form-based extension and for function-based extension of the term. For example, the term \textit{indirect object} originally meant objects that are introduced by a preposition in French, but it was extended both to objects that are marked by dative case, as in German, and to objects that have no specific marking but denote a recipient, as in English double object constructions:

\TabPositions{3cm,6cm,9cm}

\ea \label{Haspelmath7}
\textbf{Indirect object}\\
stereotype: \tab \textit{They gave bananas \textbf{to} the children}.\\
form-based: \tab \textit{Sie gaben den Kinder-\textbf{n} Bananen}. (German, Dative \textit{-n})\\
function-based: \tab \textit{They gave the children bananas}.
\z

As a result, there is no coherent general sense of the term ``indirect object'', and recent cross-linguistic work such as \citet{MalchukovEtAl2010} uses the new term \textit{indirective} specifically for dative-like marking of R-arguments (as a hybrid concept).\footnote{In current generative grammar, \textit{IO (indirect object)} seems to be still used in the notional sense (e.g. \cite{AnagnostopoulouSevdali2020}), for what others call R-argument.} Six further examples of the same kind of sense divergence are shown in (\ref{Haspelmath8})--(\ref{Haspelmath13}) in a parallel way. 

\ea \label{Haspelmath8}
\textbf{Reflexive construction} (cf. \cite{Haspelmath2023Reflexive}; see also \sectref{HaspelmathSec5-1})
stereotype: \tab \textit{They saw \textbf{themselves} in the mirror}.\\
form-based: \tab \textit{They behaved \textbf{themselves}}.\\
function-based: \tab \textit{They dressed}. (cf. \cite{Reuland2011}: ``reflexive predicate'') 
\z

\ea \label{Haspelmath9}
\textbf{Existential construction} (see also \sectref{HaspelmathSec5-2})\\
stereotype: \tab \textit{There is money in the box}.\\
form-based: \tab \textit{There was the photo of a young couple among his \tab papers}.\\
function-based: \tab \textit{\textsc{Money} is in the box}.
\z

\ea \label{Haspelmath10}
\textbf{Adjectival construction} (see also \sectref{HaspelmathSec5-6})\\
stereotype: \tab \textit{the \textbf{new} house}\\
form-based: \tab \textit{the royal family}\\
function-based: \tab \textit{fángzi hĕn \textbf{dà}} (Mandarin (Sino-Tibetan, [mand1415]);\tab predicative verb)
\z

\ea \label{Haspelmath11}
\textbf{Nominalization construction}\\
stereotype: \tab \textit{the construct-\textbf{ion} of the city}\\
form-based: \tab \textit{the construct-ions in French}\\
function-based: \tab [\textit{\textbf{that} the city was constructed}] (cf. \cite{Shibatani2019})
\z

\ea \label{Haspelmath12}
\textbf{Impersonal construction} (cf. \cite{MalchukovOgawa2011})\\
stereotype: \tab Latin \textit{curritur} `there is running' (impersonal passive)\\
form-based: \tab Spanish \textit{llueve} `it is raining' (no subject)\\
function-based: \tab French \textit{on court} `one runs' (no specified agent)
\z

\ea \label{Haspelmath13}
\textbf{Compound construction} \citep[142]{Croft2022}\\
stereotype: \tab \textit{car key} (a typifying construction with juxtaposition)\\
form-based: \tab \textit{Yeltsin visit} (`visit by Yeltsin', but not typifying)\\
function-based: \tab \textit{bird's nest} (typifying, but no juxtaposition)
\z


Examples like these can be multiplied, and apparent disagreements in typology often boil down to different terminological choices: One and the same term can be used for a construction-strategy or for a construction-function, as pointed out by \citet[11]{Croft2022}. In the next section, I will discuss a few further terminological issues that can be resolved by making a clear distinction between functions and strategies.

\citet{Lehmann2007} was the first to highlight this contrast from the perspective of the history of linguistics. He notes that there is a general tendency for structural (construction-strategy) terms to be extended to other kinds of strategies, thus becoming terms for construction-functions. Unlike \citet{Croft2016} (as seen in the quotation in the preceding section), Lehmann prefers conservative terminological usage (e.g. for the terms \textit{case}, \textit{voice}, \textit{incorporation}, and \textit{transitivity}), but he recognizes the general tendency for terminological usage to shift from hybrid to notional.

\section{Some important distinctions that are not always made} \label{HaspelmathSec5}

In this section, I will briefly revisit a few grammatical domains where the terminology has sometimes been unclear, and where the function vs. strategy distinction can illuminate earlier misunderstandings.

\subsection{Agent-patient coreference vs. reflexive constructions} \label{HaspelmathSec5-1}

The most (stereo-)typical constructions with agent-patient coreference include a reflexivizer such as a reflexive pronoun (\ref{Haspelmath14a}) or a reflexive affix (\ref{Haspelmath14b}). 

\ea \label{Haspelmath14}
\ea \label{Haspelmath14a}
\textit{The man saw \textbf{himself} in the mirror}.\\
\ex \label{Haspelmath14b}
\langinfo {Anindilyakwa}{Gunwinyguan, [anin1240]}{\cite[530]{VanEgmond2023}}\\
\gll Dhə-dharrəngka	yingə-ngambaja-jungu-na.\\
\textsc{3.f}-female \textsc{3.f}-wash-\textsc{refl-pst}\\
\glt `The woman washed herself.'\\
\z
\z

But languages may express agent-patient coreference also without reflexivizers, as in (\ref{Haspelmath15}). The Jambi Malay pronoun \textit{dio} is neutral between disjoint reference or agent-patient coreference.

\ea \label{Haspelmath15}
\ea \label{Haspelmath15a}
\textit{The man shaved}.\\
\ex \label{Haspelmath15b}
\langinfo {Jambi Malay}{Austronesian, [jamb1236]}{\cite[147]{ColeEtAl2015}}\\
\gll Dio\textup{\textsubscript{1}} cinto dio\textup{\textsubscript{1/2}}.\\
he love he\\
\glt `He loves him.' OR `He loves himself.'\\
\z
\z

So how should we use the traditional term \textit{reflexive}? In \citet{Haspelmath2023Reflexive}, I proposed that only constructions with a reflexivizer (a grammatical marker or a reflexive noun or pronoun) should be called reflexive constructions. But alternatively, one could also extend the term \textit{reflexive} to all agent-patient coreference constructions such as (\ref{Haspelmath15a}) and (\ref{Haspelmath15b}), as is done by \citet{Reuland2011}.\footnote{If (\ref{Haspelmath15a}) were included in the definition of \textit{reflexive} and English \textit{shave} were regarded as a ``reflexive predicate'' (as in \cite{ReinhartReuland1993} and \cite{Reuland2011}), it would be difficult to provide a general definition of the term \textit{reflexive} that includes long-distance reflexive constructions. These do not involve either agent-patient coreference or ``reflexive predicates'', but they are generally included in discussions of reflexive constructions. For this reason, I chose to exclude constructions such as English (\ref{Haspelmath15a}) and to require that a reflexive construction includes a reflexivizer as in (\ref{Haspelmath14}).} The crucial point in the current context is that agent-patient coreference constructions are construction-functions, while reflexive constructions are construction-strategies. 

\subsection{Existential clause constructions} \label{HaspelmathSec5-2}

Stereotypically, an existential clause is a clause such as \textit{There is \textsc{money} in the box}, in contrast to a predicative locative clause such as \textit{The money is in the \textsc{box}}. Now again, there are (at least) two ways in which this concept could be extended: On the one hand, we could say that all sentences in which an indefinite entity is said to be in some location is an existential clause, e.g. \textit{In the box is \textsc{money}}, or \textit{\textsc{Money} is in the box}. But on the other hand, one could also require that a specific strategy such as the English \textit{there}-construction be used (as in \cite[212]{McNally2016}). If one adopts a strategy-based definition, then some clauses that are not functionally existential clauses would be included (e.g. \textit{There was the photo of a young couple among his papers}, where the located entity is not indefinite; \cite[228]{McNally2016}). So again, \textit{existential clause} can be understood either as a construction-function (as in \cite{Clark1978} and \cite{HaspelmathNonverbal}) or as a construction-strategy (as in \cite{McNally2016}). There is also a third use of the term \textit{existential}, where \textit{There is money in the box} would not count as existential but as ``inverse-locational'' \citep{Creissels2019}. For more discussion of existential clauses (in the larger context of locational and possessional clauses), see \sectref{HaspelmathSec6} below.

\subsection{Transitivity: Agent vs. A, patient vs. P} \label{HaspelmathSec5-3}

Semantic roles such as agent, patient and recipient are often used to characterize groups of syntactic arguments, but in typological discussions of the syntax of arguments, the terms \textit{A-argument} (or simply A) and \textit{P-argument} (or simply P) have proved the most useful. The difference is that only the first three terms refer to semantic functions, while the second two refer to formal strategies. As noted in \citet{Haspelmath2011}, an A-argument is generally thought of as an argument in a two-argument clause that has the same coding (in terms of flagging and/or indexing) as a typical agent (of a physical-effect verb such as `kill' or `break'), and a P-argument is an argument that has the same coding as a typical patient. A and P are thus not simply functional notions: The two arguments of a verb like \textit{see} (e.g. \textit{Kim saw Lee}) are A and P, respectively, even though they are not agent and patient. A and P are better for describing alignment types (accusative, ergative) and voice constructions than \textit{agent} and \textit{patient}, because many transitive verbs do not have ``agent'' and ``patient'' arguments in the literal sense but still behave in the same way syntactically. The notions of A- and P-arguments play a very similar role as the traditional ``subject'' and ``object'' notions, and in fact, we can basically equate the terms in many comparative contexts.\footnote{The most salient context in which a subject is not an agent is in passives (e.g. \textit{Kim was pushed by Lee}). Such voice constructions are best characterized as demoting the A to an oblique and shifting the P to an S (see \cite{Haspelmath2022Valency}), so they, too, are compatible with the equation of ``subject'' with A (and S) in comparative contexts.} So again, we have a contrast between functions (agent, patient...) and strategies (A, P...). 

Both agents/patients and A/P are often called \textit{arguments}, without distinguishing the semantic (functional) and the morphosyntactic (strategy) level. For clarity, it would seem best to call agents and patients (at the semantic level) \textit{participants}, and A and P (at the syntactic level) \textit{arguments} (or \textit{actants}) (e.g. \cite[12]{Lazard1998}; \cite[1548]{Lehmann2015}).\footnote{Etymologically, of course, the term \textit{argument} refers to the semantic level, as it comes from logic. But in linguistics, it has been used for the syntactic level since the 1980s, and the older term \textit{actant} from \citet{Tesniere1959} has remained marginal.} Similarly, it would seem ideal to distinguish between participant structure (or semantic role structure) at the semantic level, and \textit{argument structure} or \textit{valency} at the syntactic level. These distinctions are not always made, and one often needs to examine the context to find out which level is intended.\footnote{Linguists often talk about ``syntactic valency'' and ``semantic valency'' (e.g. \citealt[126]{HerbstSchüller2008}). My preference is to restrict \textit{valency} to the morphosyntactic level \citep{Haspelmath2022Valency}. (Conversely, e.g. \citealt{Melcuk2004} extends the syntactic term \textit{actant} to ``semantic actants''.)} But conceptually, there is a clear contrast between them.

Once we have the syntactic argument types (or syntactic functions) A and P, we can define the term \textit{transitive} as a comparative concept \citep[549]{Haspelmath2011}, as in (\ref{Haspelmath16}).

\ea \label{Haspelmath16}
\textbf{Transitive clause}\\
A transitive clause is a clause that has an A-argument and a P-argument.\\
\z


A transitive construction is thus a type of construction-strategy, not a type of construction-function. As the functional core of the transitive strategy, linguists have identified the functional notion of ``effective action'' \citep{Tsunoda1985, Tsunoda2015} or ``physical-effect action'' \citep{Lazard2002}. In the literature, the term \textit{transitivity} is often used in a functional-semantic sense, but as noted by \citet[§2.5]{Lehmann2007}, it is best reserved for a type of strategy.

\subsection{Causal vs. causative, noncausal vs. anticausative} \label{HaspelmathSec5-4}

Although causative verbs are not very prominent in the biggest European languages, the term \textit{causative} has been used since the 19th century for Sanskrit (Indo-European; sans1269) verbs with the suffix \textit{-aya}, so it has long been familiar to many linguists. When the first systematic cross-linguistic works appeared in the 1960s (\cite{NedjalkovSilnickij1969} and related work), morphological causatives were placed in a larger semantic context, and they were contrasted with ``lexical causatives'' such as \textit{kill} or \textit{wash}. \citet[886]{Kulikov2001} said that ``causatives can be defined as verbs which refer to a causative situation, that is, to a causal relation between two events, one of which is believed by the speaker to be caused by another''. This was a function-based sense extension of the term. 

On the other hand, \citet{NedjalkovSilnickij1969} and \citet{Nedyalkov1973} introduced the term \textit{anticausative}, but they restricted it to verbs with an overt marker that seems to ``remove'' the causative component from the verb’s meaning, e.g. Russian \textit{otkryt'-sja} `to open (intr.), to become open', derived from \textit{otkryt'} `to open (tr.), cause to become open'. So, this term referred to a specific strategy for expressing intransitive change-of-state (or otherwise non-caused) events.

The term \textit{anticausative} was first adopted in the original sense (e.g. \cite{Comrie1985, Haspelmath1993}), but some authors later generalized it to verbs such as English \textit{open} or \textit{break} which can be used intransitively or transitively (e.g. \cite{AlexiadouEtAl2006}). The resulting situation was one where both terms, the old \textit{causative} and the new \textit{anticausative}, had two meanings, referring either to a construction-function (an event with actor and undergoer and a cause component; or an event without a cause component), or to a specific strategy (a verb with an affix expressing either the presence or the absence of the cause component).

To remedy this situation, \citet{HaspelmathEtAl2014} (and \cite{Haspelmath2016Causative}) proposed a new term pair: \textit{causal} vs. \textit{noncausal}, for the construction-functions: verbs or verbal expressions that include a cause component, vs. verbs that lack a cause component. They also proposed that the terms \textit{causative} and \textit{anticausative} should be restricted to a specific type of construction-strategy, namely verbal expressions in which there is a marker that indicates causal or noncausal meaning. Thus, \textit{causative} relates to \textit{causal} as \textit{dative} relates to \textit{recipient}, or as \textit{declarative} relates to \textit{statement}. The new terminology is illustrated with examples from English and Turkish (Turkic; nucl1301) in \tabref{HaspelmathTab2}: The first two columns show verbs with noncausal function, and the second two columns show verbs with causal function. Each function can be expressed by two different strategies (basic and overtly marked). English has the causative marker \textit{make} and no anticausative marker, while Turkish has both an anticausative marker (\textit{-ıl}) and a causative marker (\textit{-dır/-dür}).


\begin{table}
  \begin{tabular}{lcccc}
  \lsptoprule
     & \multicolumn{2}{c}{\textsc{noncausal}} & \multicolumn{2}{c}{\textsc{causal}} \\
  \cmidrule(lr){2-3}\cmidrule(lr){4-5}
     & \textsc{basic} & \textsc{anticausative} & \textsc{basic} & \textsc{causative} \\
  \midrule
    English & \textit{break} (intr.) & -- & \textit{break} (tr.) & (*\textit{make break}) \\
    English & \textit{die} & -- & \textit{kill} & (*\textit{make die}) \\
    Turkish & \textit{öl-} `die' & -- & -- & \textit{öl-dür-} `kill' \\
    Turkish & -- & \textit{kır-ıl-} `break' & \textit{kır} `break' & \textit{kır-dır-} `make break' \\
    English & \textit{talk} & -- & -- & \textit{make talk}\\
  \lspbottomrule
  \end{tabular}
  \caption{Some examples of noncausal and causal forms in English and Turkish}
  \label{HaspelmathTab2}
\end{table}

Even though this new terminology was proposed fairly recently, it has already been adopted fairly widely (e.g. \cite{AralovaPakendorf2022, Creissels2024a}).

\subsection{Grammatical voice constructions} \label{HaspelmathSec5-5}

The term \textit{voice} was originally a general term for the active/passive distinction in the classical languages Latin and Greek, and it referred to an inflectional category (alongside tense, mood, and others). For example, Latin \textit{videt} means `sees' (active voice), and \textit{videtur} means `is seen' (cf. \textit{curritur} lit. `is run' in (\ref{Haspelmath12}) above). But when the term was transferred to modern languages such as French and English, which lack such an inflectional category, \textit{passive voice} was applied to a specific syntactic construction type, exemplified by \textit{The thief was caught by the police}. This construction shows a special mapping of semantic roles onto syntactic functions, and over time, other special constructions with unusual role-function mapping were included as well, in particular reflexive, middle, antipassive, applicative, and causative (e.g. \cite{Kulikov2011, ZunigaKittilä2019}).\footnote{The first paper to adopt such a conception of the notion of voice was \citet{Melcuk1970}. Kulikov’s \citeyearpar{Kulikov2011} paper clearly stands in this European tradition, and both \citet{ZunigaKittilä2019} and \citet{Haspelmath2022Valency} were strongly influenced by it (see also \cite{Creissels2024a}). (\citealt{Givon1994} and \citealt{Croft2022} represent an American tradition that was less influenced by this European tradition of voice typology; however, \citet{Comrie1978} and \citet{Farrell2005} are American authors whose conception is similar to the ``European'' one.)} In a recent paper \citep{Haspelmath2022Valency}, I characterized a voice alternation as a valency alternation with verbal coding. A valency alternation is a pair of sister valency constructions with argument positions that are linked by correspondence variables. In (\ref{Haspelmath17}), we see a labile alternation, which has no verb coding (e.g. \textit{the door opened}/\textit{we opened the door}), and (\ref{Haspelmath18}) shows two alternations with verb coding (i.e. voice alternations or voices): a passive voice alternation in (\ref{Haspelmath18a}), and an antipassive voice alternation in (\ref{Haspelmath18b}). In the uncoded alternants 〈V, A\textsubscript{x}, P\textsubscript{y}〉 of both voice alternations, there are two argument positions (A and P), of which one becomes S in the coded alternation, while the other one is downgraded (demoted to oblique or suppressed).

\largerpage
\ea \label{Haspelmath17}
labile\\
\gll {} \textup{〈V, A\textsubscript{x}, P\textsubscript{y}〉} {} \textup{`X causes Y to change'}\\
\textup{\approx} \textup{〈V, S\textsubscript{y}〉} {} \textup{`Y changes'}\\
\z

\ea \label{Haspelmath18}
\ea \label{Haspelmath18a}
passive\\
\gll {} \textup{〈V, A\textsubscript{x}, P\textsubscript{y}〉} {} {}\\
\textup{\approx} \textup{〈V-\textsc{pass}, \{obl\}\textsubscript{x}, S\textsubscript{y}〉} {} \textup{`(X) acts on Y'}\\
\ex \label{Haspelmath18b}
antipassive\\
\gll {} \textup{〈V, A\textsubscript{x}, P\textsubscript{y}〉} {} {}\\
\textup{\approx} \textup{〈V-\textsc{antipass}, S\textsubscript{x}, \{obl\}\textsubscript{y}〉} {} \textup{`X acts (on Y)'}\\
\z
\z


The passive and antipassive constructions are thus very clearly construction-strategies (not construction-functions), as they are defined through formal (or hybrid) notions such as A, P, oblique flagging, and verbal affixation.

But alternatively, the terms \textit{passive} and \textit{antipassive} can be extended in a notional sense, as we saw for the term \textit{causative} in the previous subsection. For the term \textit{passive}, a construction-function view was clearly articulated by \citet[9]{Givon1994}, who said that a passive is a construction in which ``the patient is more topical than the agent, and the agent is extremely non-topical (``suppressed'', ``demoted'').'' He thus defined \textit{passive} as a construction-function, and he was very explicit about his reasons for this. He noted that passives in different languages are structurally quite diverse (reflexive-based, nominalization-based, participle-based, etc.), and he dismissed a structural definition on the grounds that it would not be revealing: ``While the synchronic function [of the diverse passive constructions in different languages] may be that of passive voice, their structure reflects lingering vestiges of the earlier point of origin'' \citep[6]{Givon1994}. Givón was of course right that there are a wider range of structurally defined types with passive-like functions than the construction type in (\ref{Haspelmath18a}), but he did not consider the alternative possibility of choosing a different term for the functionally defined type.\footnote{He said that ``in grammatical typology, one enumerates the main structural means by which different languages code – or perform – the same function'' \citep[7]{Givon1994}, but this describes only one way of doing typology. As we saw in \sectref{HaspelmathSec2}, typologies may focus on construction-functions or on construction-strategies, and for a comprehensive picture, we need both approaches.}

Croft's \citeyearpar{Croft2022} terminology in the domain of ``voice'' is similar in that it adopts a construction-function perspective: He defines ``passive-inverse voice construction'' as expressing a situation in which the patient has higher discourse salience than the agent, and ``antipassive voice construction'' as expressing a situation in which the patient is less salient than usually (see also \cite{Vigus2018}). As a result, antipassives are represented not only by examples like (\ref{Haspelmath19b}) from Kuku Yalanji (which are antipassive construction-strategies according to the definition in (\ref{Haspelmath18b})), but also by (\ref{Haspelmath20b}) and (\ref{Haspelmath21b}), which are not antipassive construction-strategies \citep[263--265]{Croft2022}.

\ea \label{Haspelmath19}
\langinfo{Kuku Yalanji}{Pama-Nyungan, [kuku1273]}{\cite[152]{Patz2002}}
\ea \label{Haspelmath19a}
\gll Nyulu dingkar-angka minya-Ø nuka-ny.\\
\textsc{3sg.nom} man-\textsc{erg} meat-\textsc{abs} eat-\textsc{pst}\\
\glt `The man ate meat.'\\
\ex \label{Haspelmath19b}
\gll Nyulu dingkar-Ø minya-nga muka-ji-ny.\\
\textsc{3sg.nom} man-\textsc{abs} meat-\textsc{loc} eat-\textsc{antip-pst}\\
\glt `The man had a good feed of meat (he wasted nothing).'\\
\z
\z

\ea \label{Haspelmath20}
\ea \label{Haspelmath20a}
\textit{The coyote chewed the deer bone}.\\
\ex \label{Haspelmath20b}
\textit{The coyote chewed on the deer bone}.
\z
\z

\ea \label{Haspelmath21}
\ea \label{Haspelmath21a}
\textit{She ate the spaghetti}.\\
\ex \label{Haspelmath21b}
\textit{She ate}.
\z
\z

From the point of view of terminological continuity, it seems to me that it is best to avoid a broadening of the meaning of the terms \textit{passive} and \textit{antipassive} in such a way that these terms can refer to constructions which have no verbal marking at all, but in any event, this is primarily a terminological issue.\footnote{A similar case is the term \textit{inverse}, which is used for a construction-function by \citet{Givon1994}, whereas most other authors (such as \cite{JacquesAntonov2014, Creissels2024a, HaspelmathInverse}) use it for a construction-strategy (employing special inverse markers in particular person scenarios).} Givón and Croft are surely right that it is worth considering passives and antipassives in a broader perspective (involving construction-functions in which the agent or the patient have lower discourse salience than usual), and it is a secondary matter which terms to use for this broader perspective.

\subsection{Property-concept words and adjectives} \label{HaspelmathSec5-6}

Finally in this section, let us consider the terms \textit{adjective} and \textit{property-concept word}. Linguists often say that not all languages have adjectives. For example, \citet[214]{Hieber2023} observes that ``most North American languages arguably lack an adjective class, such that property concepts are a subcategory of noun or verb or divided between both''. On this view, one would be inclined to say that a property-predicating clause such as (\ref{Haspelmath22b}) from Krongo is a verbal clause (as it contains a ``Qualifying Verb''), just like the action-predicating clause in (\ref{Haspelmath22a}).

\largerpage
\ea \label{Haspelmath22}
\langinfo{Krongo}{Kadugli-Krongo, [kron1241]}{\cite[236]{Reh1985}}
\ea \label{Haspelmath22a}
\gll K-àdìyà óow.\\
\textsc{pl}-come.\textsc{prf} we\\
\glt `We have come.'\\
\ex \label{Haspelmath22b}
\gll K-àdéelá óow.\\
\textsc{pl}-beautiful we\\
\glt `We are beautiful.'\\
\z
\z


By contrast, for the English translations, one often says that \textit{we have come} is a verbal clause, while \textit{we are beautiful} is a nonverbal clause.

It may thus seem that \textit{adjective} is a strategy for expressing the property-concept function, and \textit{verb} is an alternative strategy chosen by many languages outside of Europe. However, it is unclear how \textit{noun}, \textit{verb} and \textit{adjective} can be defined as cross-linguistically applicable strategies. Linguists typically use different criteria for identifying adjectives in different languages, but this is not a way to single out a uniform cross-linguistic class. 

In the end, whether something is accepted as an adjective or not hinges on the semantic criterion of denoting a property concept. Thus, \citet[217]{Hieber2023} argues that Seneca (sene1264; Iroquoian) does not even have a verbal subclass of adjectives (following \cite{Chafe2012}), although adjectives can be said to belong to a subclass of stative verbs (e.g. \textit{otge’} `it's heavy', \textit{hohsë:h} `he's fat'). The reason is that the subclass of stative verbs includes other verbs that do not denote property concepts (e.g. \textit{otga:h} `it's making a noise', \textit{hóío’de’} `he's working'). Linguists never consider the possibility of extending the adjective concept to such event-denoting words, which shows that by \textit{adjective}, they generally mean `word denoting a property concept'. More generally, we can define \textit{noun}, \textit{verb} and \textit{adjective} as `root denoting an object, an action and a property', respectively \citep{Haspelmath2023WordClass}.

There are indeed general trends in the kinds of strategies that languages use to refer to objects (as opposed to actions and properties), to predicate actions (as opposed to objects and properties), and to modify via properties (as opposed to objects and actions), but these strategies do not apply uniformly and thus cannot be used in identifying nouns, verbs and adjectives across languages. 

Thus, the term \textit{nonverbal clause} must be defined semantically in contrast with \textit{verbal clause}, which means that in a comparative perspective, a Krongo clause such as \textit{k-àdéelá óow} `we are beautiful' is a nonverbal clause (\cite{HaspelmathNonverbal}; see also \cite[289]{Croft2022}). This is different from the way \textit{non-verbal predication} has been defined elsewhere (e.g. by \cite[27]{Hengeveld1992}), but if one wants to say that property-concept predications of the Krongo type are ``verbal'', one needs to apply language-particular criteria, and this is not how comparative concepts work (they must be based on uniform criteria across languages).

\section{Revisiting existential and hyparctic clauses} \label{HaspelmathSec6}

Let us now consider existential and related clause types, such as those in (\ref{Haspelmath23a})--(\ref{Haspelmath23d}). 

\ea \label{Haspelmath23}
\ea \label{Haspelmath23a}
\textit{There is a boat on the shore}.\\
\ex \label{Haspelmath23b}
\textit{The boat is on the shore}.\\
\ex \label{Haspelmath23c}
\textit{I have a boat}.\\
\ex \label{Haspelmath23d}
\textit{The boat is mine}.\\
\z
\z

In \citet{HaspelmathNonverbal}, I take up the discussion of \citet{Koch2012} (who was inspired by \cite{Clark1978} and \cite{Bickerton2016}) and I distinguish between four main clause types as shown in \tabref{HaspelmathTab3}.

\begin{table}
  \begin{tabularx}{\textwidth} {   Q  Q  Q}
  \lsptoprule
     & locational & possessional \\
  \midrule
    indefinite locatum/possessum & \textsc{existential} \newline (\ref{Haspelmath23a}: \textit{There is a boat on the shore}.) & \textsc{predpossessive} \newline (\ref{Haspelmath23c}: \textit{I have a boat}.)\\
  \midrule
    definite locatum/possessum & \textsc{predlocative} \newline (\ref{Haspelmath23b}: \textit{The boat is on the shore}.) & \textsc{appertentive} \newline (\ref{Haspelmath23d}: \textit{the boat is mine}.) \\
  \lspbottomrule 
  \end{tabularx}
  \caption{Four locopossessional construction-functions}
  \label{HaspelmathTab3}
\end{table}

The term \textit{existential} for (\ref{Haspelmath23a}) has been used since \citet{Jespersen1924}, and the terms \textit{predicative possession} and \textit{predicative location} are well-established, too (I abbreviate them to \textit{predpossessive} and \textit{predlocative}, because this makes it easier to be fully specific; just saying \textit{possessive} and \textit{locative} is not always clear enough).\footnote{It should also be noted that \textit{predicative possessive} (cf. \cite{Stassen2009}) is not a transparent term because existential and predpossessive clauses are not predicational. Therefore, \citet[306]{Croft2022} prefers the more transparent term \textit{predicational possession}. However, the term \textit{predicative possession} is fairly entrenched in linguistics, and the lack of transparency is not a problem for single-word terms like \textit{predpossessive}.} The term \textit{appertentive} for \textsc{belong}-constructions \citep{StolzLevkovych2019} is new but seems appropriate, as it comes from the Latin verb for ‘belong’ (cf. \textit{pertain}, \textit{appurtenance}).\footnote{\citet[212--220]{Bickerton2016} used the terms \textit{existence}, \textit{location}, \textit{possession}, and \textit{ownership}. He built on \citet{Clark1978}, who talked about \textit{existential}, \textit{locative}, \textit{possessive\textsubscript{1}} and \textit{possessive\textsubscript{2}} constructions. Her cover term for all four was ``locationals'' (cf. my new term \textit{locopossessional constructions}, \cite[§13]{HaspelmathNonverbal}).}

The clause types in \tabref{HaspelmathTab3} are construction-functions, but the terminology has not always been clear in the past. As we saw in \sectref{HaspelmathSec4}, traditional terms may be extended along the functional dimension or along the formal dimension. The first prominent paper on locopossessional constructions, \citet{Lyons1967}, started out by distinguishing three types along purely semantic lines, ignoring formal aspects and discourse pragmatics: existential (e.g. \textit{Lions exist}; \textit{There are lions (in Africa)}), locative (\textit{The book is on the table}; \textit{There is a book on the table}), and possessive (\textit{The book is John's}; \textit{John has a book}). Thus, English \textit{there}-clauses fall into two different types here (existential and locative), even though Jespersen did not limit the term \textit{existential} in this way and seemed to use the term for counterparts of \textit{there}-sentences and their presentational discourse function.\footnote{\citet[155]{Jespersen1924} characterized existential sentences as ``sentences corresponding to English sentences with \textit{there is} or \textit{there are}, in which the existence of something is asserted or denied''. Among his examples were \textit{there is no money in the box}, and \textit{there were many people present}.} However, in Clark’s \citeyearpar{Clark1978} paper, which was inspired by Lyons’s work,\footnote{An earlier version of \citet{Clark1978} was published as \citet{Clark1970}, and Clark’s dissertation was completed at the University of Edinburgh in 1969, under John Lyons’s supervision.} these terms were used in a clearly functional sense. It is this tradition of Clark which has been followed de facto by much of the literature, and which I build on in my \citeyear{HaspelmathNonverbal} paper.

The difference between Lyons’s \citeyearpar{Lyons1967} usage and Haspelmath’s \citeyearpar{HaspelmathNonverbal} can be summarized as in \tabref{HaspelmathTab4}, using Lyons’s examples.


\begin{table}
  \centering
  \begin{tabular}{lll}
  \lsptoprule
    Lyons's terms & Lyons's examples & Haspelmath's terms \\
  \midrule
    \textsc{existential}  & \textit{Lions exist}. & \textsc{hyparctic} \\
    \cmidrule{3-3}
     \textsc{existential} & \textit{There are lions in Africa}. & \textsc{existential}\\
     \cmidrule{1-1}
     \textsc{locative} & \textit{There is a book on the table}. & \textsc{existential} \\
    \cmidrule{3-3}
     \textsc{locative} & \textit{The book is on the table}. & \textsc{predlocative} \\
    \lspbottomrule
  \end{tabular}
  \caption{Hyparctic, existential and predlocative constructions}
  \label{HaspelmathTab4}
\end{table}

While Lyons’s \citeyearpar{Lyons1967} terminology is perfectly coherent in that it opposes permanent existence to temporary location, Haspelmath’s \citeyearpar{HaspelmathNonverbal} terminology has several advantages, in addition to being in line with most of the actual terminological usage in the field over the last few decades.\footnote{Of course, the term \textit{hyparctic} (based on Greek \textit{hyparxis} `existence') is an innovation, but such clauses (of ``pure existence'') have hardly been discussed by linguists.}

First, my terminology is based on two clear criteria: the semantic criterion of location (singling out existential and predlocative clauses as opposed to hyparctic clauses), and the discourse-pragmatic criterion of definiteness, which distinguishes predlocative from existential clauses, just as it distinguishes appertentive from predpossessive clauses. By contrast, the distinction between permanent location (\textit{there are lions in Africa}) and temporary or episodic location (\textit{there is a book on the table}) is much less clear. It cannot be described as ``existence'' in the literal sense because when we talk about a table having a book on it, we do of course presuppose the existence of the book. And there seem to be many intermediate cases, with different degrees of permanence, e.g. \textit{there is a sofa in the living room}, \textit{there is a restaurant in our street}, \textit{there are two bathrooms in her apartment}. Some languages make a distinction between permanent existence and temporary presence, e.g. German, which is well-known for the distinction between \textit{es gibt} (for permanent existence, e.g. \textit{es gibt Löwen in Afrika} `there are lions in Africa') and the use of posture verbs (for temporary presence, e.g. \textit{es liegt ein Buch auf dem Tisch} `there is [lit. lies] a book on the table'). However, as noted by \citet{Creissels2019}, the distinction is often related to other factors as well, such as availability. Thus, we can say \textit{es gibt heute Schokolade bei Aldi} `there is chocolate today at Aldi's', using \textit{es gibt} for temporary availability. So far, typologists have not investigated the contrast between permanent existence and temporary presence systematically for many languages.

Second, the new term \textit{hyparctic} for ``pure existence'' (with no location or possession implied) has the advantage that it assigns a clear place to sentences such as (\ref{Haspelmath24}), which have sometimes been subsumed under existential clauses on the assumption that existential clauses are about existence.

\ea \label{Haspelmath24}
\ea \label{Haspelmath24a}
\textit{God exists}.\\
\ex \label{Haspelmath24b}
\textit{There is beer without alcohol}.\\
\ex \label{Haspelmath24c}
\langinfo{Latin}{}{René Descartes, 1637}\\
\gll Cogito, ergo sum.\\
think.\textsc{1sg} therefore be.\textsc{1sg}\\
\glt `I think, therefore I am.'\\
\z
\z

However, the main features of existential clauses (in the usual sense) are location and an indefinite locatum.

Third, and most importantly in the present context, the terminology I propose does not restrict the domain of existential clauses to particular strategies. Existential clauses (as well as predpossessive clauses) are defined as construction-functions, and the kinds of construction-strategies used to express these functions may be quite diverse (this was already noted above in \sectref{HaspelmathSec1}). Thus, we can distinguish various construction-strategies (formally defined construction types) that are used to express the existential construction-function, e.g.

\ea \label{Haspelmath25}
\textbf{Existive-copula strategy}\\
\langinfo{Hungarian}{Uralic, [hung1274]}{personal knowledge}\\
\gll Van vaj a hűtő-ben.\\
there.is butter the fridge-\textsc{iness}\\
\glt `There is butter in the fridge.'\\
\z

\ea \label{Haspelmath26}
\textbf{Transpossessive-existential strategy}\\
\langinfo{Modern Greek}{Indo-European, [gree1276]}{\citealt{chapters/04Kampanarou}}\\
\gll Exi kati skilus stin peðiki xara.\\
has some dogs at.the play ground\\
\glt `There are some dogs at the playground.'\\
\z

\ea \label{Haspelmath27}
\textbf{Prolocative strategy}\\
\langinfo{Italian}{Indo-European, [ital1282]}{personal knowledge}\\
\gll Ci sono molte montagne in Svizzera.\\
\textsc{pro.loc} are many mountains in Switzerland\\
\glt `There are many mountains in Switzerland.'\\
\z

Importantly, constructions which do not have any formal specificities count as existential constructions as well. Creissels gives the examples in (\ref{Haspelmath28}) and (\ref{Haspelmath29}), which show constructions that can be used both for predlocative and for existential functions.

\ea \label{Haspelmath28}
\langinfo{Welsh}{Indo-European, [wels1247]}{\cite[691]{Feuillet1998}, cited from \cite[51]{Creissels2019}}\\
\ea \label{Haspelmath28a}
\gll Mae 'r car yma.\\
is the car here\\
\glt `The car is here.'\\
\ex \label{Haspelmath28b}
\gll Mae car yma.\\
is car here\\
\glt `There is a car here.'
\z
\z

\ea \label{Haspelmath29}
\langinfo{Mandinka}{Mande, [mand1436]}{\cite[51]{Creissels2019}}\\
\gll Wùlôo bé yíròo kótò.\\
dog.\textsc{det} \textsc{loc.cop} tree.\textsc{det} under\\
\glt `The dog is under the tree.' OR: `There is a dog under the tree.'\\
\z

In Welsh, the only difference is the presence of the definite article in the predlocative clause, while there is no difference at all in Mandinka. For Mandinka, it is probably unnecessary (and perhaps even confusing) to make a distinction between existential and predlocative clauses. However, this does not affect the way we define our comparative concepts.

There is a tendency in the literature to see an existential construction as a kind of construction-strategy (i.e. a hybrid concept) rather than as a purely notional concept. As I already noted above (\sectref{HaspelmathSec5-2}), \citet{McNally2016} restricts the term \textit{existential} to a subclass of existential clauses that are ``noncanonical'' in some way, i.e. to some types of strategies. However, this is not a homogeneous class, because ``what is canonical differs from language to language'' (as \cite[212]{McNally2016} notes). Similarly, \citet[41]{Creissels2019} restricts his term \textit{inverse-locational} to constructions that are not ``analyzable as deriving from a general-locational predication construction via the application of some morphosyntactic device generally applicable to predicative constructions'', which also means that different languages can have different ways of distinguishing such clauses from predlocative clauses. But comparative concepts must be defined in the same way in all languages, so it is better to treat existential clauses as a kind of construction-function, and to use different terms for the various construction-strategies (as seen in (\ref{Haspelmath25})--(\ref{Haspelmath27})) that can express this function.

Finally, what about ``core'' and ``periphery'' in existential constructions? In the conceptual approach that I have been pursuing, these notions play no particular role, although in practice, languages often have more commonly used and less commonly used strategies, and a multifunctional strategy typically has more commonly attested functions and less commonly attested ones. The more commonly found strategies and functions may be called ``core'' and the less frequent ones ``peripheral'', and the ``core strategies'' and ``core functions'' will be more interesting for linguists. However, this does not mean that they do not fall under the definitions. Consider the following examples, which show a less commonly used existential clause in English and a less commonly used predlocative clause in Finnish.

\ea \label{Haspelmath30}
\ea \label{Haspelmath30a}
\langinfo{English}{}{\cite[111]{Clark1978}}\\
\textit{The table has a book on it}.\\
\ex \label{Haspelmath30b}
\langinfo{Finnish}{Uralic, [finn1318]}{\cite[161]{Basile2024}}\\
\gll Löydy-n myös Facebooki-sta.\\
find.\textsc{mm-1sg} also Facebook-\textsc{ela}\\
\glt `I am also found on Facebook.'
\z
\z

English existential clauses do not often have the locatum in subject position, and Finnish predlocative clauses do not often use the verb \textit{löydy-} ‘be found’, but this does not mean that these constructions fall under these definitions in a ``peripheral'' sense. The comparative concepts have sharp boundaries and cannot be said to have a core and a periphery.

\section{Dedicated and co-opted constructions} \label{HaspelmathSec7}

Constructions of particular languages are often extended semantically to expressing further meanings, beyond the meanings they had originally. For example, it seems that the English \textit{there}-construction was originally used for presentational and existential constructions such as \textit{there is a bird on the roof}, but it was extended to hyparctic clauses (such as \textit{there is beer without alcohol}, where no location is implied). Or languages may extend an originally locational construction to be used for predicative possession, as in (\ref{Haspelmath31}) from Irish.

\ea \label{Haspelmath31}
\langinfo{Irish}{Indo-European, [iris1253]}{personal knowledge}\\
\gll Tá teach mór ag do uncail.\\
is house big at your uncle\\
\glt `Your uncle has a big house.' (Lit. `There is a big house at your uncle.')\\
\z

For such extensions of the meanings of constructions, \citet[21]{Croft2022} uses the term ``recruitment'', and others talk about ``co-opting'' constructions. 

In the terminology used in a number of works by R. M. W. Dixon and Alexandra Aikhenvald, the term \textit{strategy} is used for a construction that expresses a meaning B but also (and originally) expresses meaning A. Strategies thus contrast with constructions that only express a single meaning (that are ``dedicated'' to this meaning). Aikhenvald puts it as follows:

\begin{quote}
  [N]ot all meanings, and not all functions, have a dedicated means of expression in a language. One language may require obligatory marking of information source (known as evidentiality). Another may co-opt other means – conditional modality or reported speech – to express similar meanings as required. Many languages have a dedicated imperative paradigm used for commands. Some do not: they use a subjunctive or another verb form instead. If a language does not have a dedicated construction of a certain type, for example complement clause, or imperative, it may render the relevant meanings by using an established construction in a secondary sense. This is then a `strategy': we can have evidentiality strategies, imperative strategies, complementation strategies, and so on. \citep[53]{Aikhenvald2015}
\end{quote}

In other words, a ``strategy'' for a given meaning is a construction that has another meaning as its primary meaning. (See also \cite{Dixon2006} on ``complementation strategies'', contrasting with complement clauses, and \cite{Dixon2008} on ``comparative strategies'', contrasting with comparative constructions.) 

A problem with this distinction is that it is not clear how we can systematically distinguish the primary meaning from a ``co-opted'' (or ``recruited'') meaning. In some cases, it may seem obvious which is the primary and which is the secondary meaning (as in the examples of English hyparctic and Irish predpossessive constructions given above), but there may be other cases where it is not clear because we do not have diachronic information. And in many cases, there are three or four different meanings or functions, so that we might have ``secondary'' meanings which are ``basic'' with respect to a ``tertiary'' meaning. More generally, it is not clear to me what larger relevance the distinction between ``dedicated'', ``primary'' and ``recruited/co-opted'' strategies might have. 

But in any event, Dixon’s and Aikhenvald’s way of using the term ``strategy'' is very different from Croft’s. Croft follows the earlier usage by authors such as \citet{Keenan1977} and \citet{Stassen1997}, and in this paper, I follow Croft’s distinction and the older meaning of the term \textit{(construction-)strategy}.\footnote{\citet[21]{Croft2022} does talk about ``recruitment strategies'', and in this sense he includes a \mbox{(quasi-)} diachronic component in his definition of \textit{strategy}. This is an aspect of his treatment of strategies that I would be hesitant to adopt, because typological classification should be strictly synchronic even if the explanatory framework includes a diachronic component.} 

\section{Conclusion} \label{HaspelmathSec8}

This paper has highlighted a distinction between functionally defined comparative construction concepts and comparative construction concepts defined in a hybrid way, on the basis of both formal and functional-semantic elements. \citet{Croft2022} calls these two types \textit{constructions} and \textit{strategies}, but I find it better to distinguish between \textit{construction-functions} and \textit{construction-strategies}, because both are types of constructions. For example, an existential construction is a type of construction (defined functionally, \sectref{HaspelmathSec6}), and a reflexive construction is a type of construction (defined in a hybrid way, \sectref{HaspelmathSec5-1}). Both kinds of constructions have been important in typological research (\sectref{HaspelmathSec2}), despite what some typologists have claimed (e.g. \cite{Givon1994}, who said that typology must always focus on a functional domain). One might object to using the term \textit{construction-function} for a construction defined functionally, because a construction is not a function (but rather a pairing of a meaning/function with a formal schema; Croft, p.c.). But the alternative terms ``function-construction'' and ``strategy-construction'' do not seem very felicitous. 

A construction-strategy is definitely a strategy (and a construction), and while a construction-function is a construction and not a function in the strict sense, we often treat construction-functions as if they were functions. Thus, we sometimes talk about ``a construction A encoding (or expressing) a construction B'', or a language having ``construction A for construction B''.\footnote{\citet[21]{Croft2022} is quite explicit about the ``for'' usage when he says: ``Strategies are always strategies for a particular construction'' (by which he means that a strategy is not a purely formal pattern, but a hybrid pattern).} For example, we sometimes talk about “encoding relative clauses”, or “using verbs for adjectives”, or “expressing adnominal possession”, or we even talk about different constructions used in a language “for the comparative construction”. This way of talking may sound awkward, but is it actually in line with our general concepts: We work with language-particular constructions at the level of particular languages (they could be called \textit{p-constructions}, as they are part of p-linguistics; \citealt{Haspelmath2021General}), with comparable constructions defined functionally (\textit{construction-functions}), and comparable constructions defined in a hybrid way (\textit{construction-strategies}). All three are normally called ``constructions'', and in any particular context, we usually know what is meant, but it is still worth reflecting on the basic ontology of grammatical research in order to improve the communication in our field.

Traditional grammatical terms are based on the terms of Latin grammar, and they have often been extended from an original stereotypical (or prototypical) sense in different ways – either to formally similar constructions with different meanings, or to semantically similar constructions with different formal properties (the latter tendency was observed by \cite{Lehmann2007}). This has often led to divergent terminological usage, and also to confusion because linguists are not always aware of the divergent usage traditions. In \sectref{HaspelmathSec5}, I discussed a few terms that have been used differently in the literature (e.g. \textit{reflexive}, \textit{transitive}, \textit{causative}, \textit{antipassive}, \textit{adjective}). While I indicate my own preferences in each case, the main purpose is to make readers aware of the reasons for the divergence. Linguists do not have standardization committees, so such developments are normal, and they are surely not due to sloppiness on anyone’s part. The innovation in \citet{Croft2022} is that for each term, he says clearly whether it is a construction(-function) or a (construction-)strategy, and this information is even included in the 87-page glossary for his book. Thus, there are reasons to be confident that these problems will be fewer in the future.

While Croft says that in general, the traditional grammatical terms should be used for construction-functions (which are universal) (see the quotation from his \citeyear{Croft2016} paper in \sectref{HaspelmathSec3} above), I do not think that such a general recommendation can be given. As I mentioned earlier, linguists have repeatedly come up with novel terms that are used notionally and that contrast with the traditional terms that are used for hybrid strategy concepts, e.g. notional \textit{agent-patient coreference} (vs. hybrid \textit{reflexive}, \sectref{HaspelmathSec5-1}), notional \textit{causal}/\textit{noncausal} (vs. hybrid \textit{causative}/\textit{anticausative}, \sectref{HaspelmathSec5-3}), notional \textit{patient foregrounding}/\textit{backgrounding} (vs. hybrid \textit{passive}/\textit{antipassive}, \sectref{HaspelmathSec5-4}).\footnote{Another example of a novel notional term is \textit{binominal construction} \citep{Pepper2023}, which contrasts with the traditional term compound, used by \citet[115, 142]{Croft2022} for the juxtaposition strategy for binominals. Thus, even Croft himself does not generally follow his own recommendation.} On the other hand, for the term \textit{existential} (which was first used for English \textit{there}-clauses), I have given reasons for preferring the notional sense in \sectref{HaspelmathSec6} (as there are a range of diverse strategies in different languages to which the term has been applied, a definition in terms of a single strategy type is impossible). Thus, I think that linguists’ terminology must be opportunistic and cannot be very systematic because it should preserve continuity to the extent possible.
 
\section*{Abbreviations}
\begin{tabularx}{.45\textwidth}{lQ}
\textsc{ela} & elative \\
\textsc{iness} & inessive \\
\end{tabularx}
\begin{tabularx}{.45\textwidth}{lQ}
\textsc{mm} & middle marker \\
\textsc{pro} & pro-form \\
\end{tabularx}

\sloppy
\printbibliography[heading=subbibliography,notkeyword=this]
\end{document}
