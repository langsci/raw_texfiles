\documentclass[output=paper,colorlinks,citecolor=brown]{langscibook}
\ChapterDOI{10.5281/zenodo.16838054}

\author{Josefina Budzisch\affiliation{Universität Hamburg} and Chris Lasse Däbritz\affiliation{Universität Hamburg; Head Office of the German Science and Humanities Council} and Rodolfo Basile\affiliation{University of Tartu; Kyoto University}}
\title{Introduction} 

\abstract{This volume investigates locative and existential predication, with a focus on their structural characteristics, communicative functions, and interconnected domains across a wide range of languages. As key constructs in syntactic and typological inquiry, these predications exhibit significant formal overlap while diverging functionally: locative clauses typically anchor a definite referent to a specific spatial context, whereas existential clauses introduce new, often indefinite, referents into discourse through specialized linguistic mechanisms.

By synthesizing contributions from workshops held at the Annual SLE Meeting in Athens and the ICHL in Heidelberg, the volume critically examines the conventional dichotomy between “core” and “peripheral” coding strategies. Through the integration of diverse methodologies and theoretical perspectives, it offers novel insights into the linguistic encoding of existence, location, and possession, thereby establishing a robust foundation for future research in typology and functional linguistics.}

\IfFileExists{../localcommands.tex}{
   \addbibresource{../localbibliography.bib}
   % add all extra packages you need to load to this file

\usepackage{tabularx,multicol}
\usepackage{url}
\urlstyle{same}

\usepackage{listings}
\lstset{basicstyle=\ttfamily,tabsize=2,breaklines=true}

\usepackage{langsci-basic}
\usepackage{langsci-optional}
\usepackage{langsci-lgr}
\usepackage{langsci-osl}
% \usepackage{./langsci/styles/langsci-lgr}
% \usepackage{./langsci/styles/langsci-osl}
% \usepackage{langsci-gb4e}

\usepackage{tikz}
\usetikzlibrary{patterns,calc}
\pgfdeclarepatternformonly{south east lines}{\pgfqpoint{-0pt}{-0pt}}{\pgfqpoint{3pt}{3pt}}{\pgfqpoint{3pt}{3pt}}{
    \pgfsetlinewidth{0.6pt}
    \pgfpathmoveto{\pgfqpoint{0pt}{3pt}}
    \pgfpathlineto{\pgfqpoint{3pt}{0pt}}
    \pgfpathmoveto{\pgfqpoint{.2pt}{-.2pt}}
    \pgfpathlineto{\pgfqpoint{-.2pt}{.2pt}}
    \pgfpathmoveto{\pgfqpoint{3.2pt}{2.8pt}}
    \pgfpathlineto{\pgfqpoint{2.8pt}{3.2pt}}
    \pgfusepath{stroke}}
    
\usepackage{stmaryrd}
\usepackage{wasysym}
\usepackage{multirow}
\usepackage{caption}
\usepackage{subcaption}
\usepackage{mathrsfs}
\usepackage{qtree}

\usepackage{linguex}


   %pminos do not split footnotes
% \interfootnotelinepenalty=10000 %Footnote in Laporte chapters has to be split SN


%\DeclareIndexNameFormat{default}{%
%\nameparts{#1}%
%\usebibmacro{index:name}%
%{\index[names]}%
%{\namepartfamily}%
%{\namepartgiveni}%
% {}% L1
% {}% L2
%{\namepartprefix}% generates spurious space L3
%{\namepartsuffix}% generates spurious space L4
%}

%  {\DeclareIndexNameFormat{default}{%
%     \usebibmacro{index:name}{\index[names]}{#1}{#3}{#5}{#7}}}

%\DeclareIndexNameFormat{default}{%
%  \usebibmacro{index:name}{\sindex[nom]}{#1}{#3}{#5}{#7}}

%\DeclareIndexNameFormat{default}{%
%  \usebibmacro{index:name}{\sindex[person]}{#1}{#3}{#5}{#7}}
%\DeclareIndexNameFormat{default}{%
%\nameparts{#1} \usebibmacro{index:name}{\sindex[person]]}{\namepartfamily}{‌​\namepartgiven}{\nam‌​epartprefix}{\namepa‌​rtsuffix}}

%\newcommand{\smiley}{:)}

%\renewbibmacro*{index:name}[5]{%
%\usebibmacro{index:entry}{#1}%
%{\iffieldundef{usera}{}{\thefield{usera}\actualoperator}\mkbibindexname{#2}{#3}{#4}{#5}}}

% \newcommand{\noop}[1]{}

%remove for final
%\overfullrule=1mm

\newcommand{\tobi}[2]}}
\renewcommand{\S}[1]{\tobi{#1}{\textsc{*}}}

% this volume references
% puts: [this volume]
% already defined: \citetv
%\newcommand{\citepv}[1]{(\citeauthor{#1} \citeyear*{#1} [this volume])}
\newcommand{\citealtv}[1]{\citeauthor{#1} \citeyear*{#1} [this volume]}

%parentheses around example number
\newcommand{\pref}[1]{(\ref{#1})}

% in-text examples

\newcommand{\lnex}[1]{\textit{#1}} %target lang word
\newcommand{\lnlit}[1]{(lit.: `#1')} %literal reading
\newcommand{\lnlat}[1]{(#1)} % latinization
\newcommand{\lntrans}[1]{`#1'} %translation
\newcommand{\lnexl}[2]%
{\lnex{#1}{} \lnlat{#2}} % ex with latinization
\newcommand{\lnexlat}[3]{\lnex{#1}{} \lnlat{#2}{} \lntrans{#3}} % ex with latinization and tranl.

%ch01
\newcommand{\co}[1]{\mbox{\textbf{#1}}}

%ch09

\newcommand{\cyrbulg}[1]{\begin{otherlanguage*}{bulgarian}#1\end{otherlanguage*}}


%ch10
\newcommand{\nlp}{{\small NLP}}
\newcommand{\mwe}{{\small MWE}}
\newcommand{\rae}{{\small RAE}}
\newcommand{\lvc}{{\small LVC}}
\newcommand{\pos}{{\small P}o{\small S}}
%\newcommand{\todo}[1]{ \textcolor{red}{#1} }

%\renewcommand{\labelenumi}{\theenumi}
%\ainamefmt{{vv}{ll}{, ff}{, jj}} % fullname

\newcommand{\biberror}[1]{{\color{red}#1}}

\newcommand{\osenovaitem}{--~}
   %% hyphenation points for line breaks
%% Normally, automatic hyphenation in LaTeX is very good
%% If a word is mis-hyphenated, add it to this file
%%
%% add information to TeX file before \begin{document} with:
%% %% hyphenation points for line breaks
%% Normally, automatic hyphenation in LaTeX is very good
%% If a word is mis-hyphenated, add it to this file
%%
%% add information to TeX file before \begin{document} with:
%% %% hyphenation points for line breaks
%% Normally, automatic hyphenation in LaTeX is very good
%% If a word is mis-hyphenated, add it to this file
%%
%% add information to TeX file before \begin{document} with:
%% \include{localhyphenation}
\hyphenation{
    Beck-man
    Ngu-yen
    back-chan-nel
    back-chan-nels
    mo-not-o-nous
    ste-reo-typ-i-cal
}

\hyphenation{
    Beck-man
    Ngu-yen
    back-chan-nel
    back-chan-nels
    mo-not-o-nous
    ste-reo-typ-i-cal
}

\hyphenation{
    Beck-man
    Ngu-yen
    back-chan-nel
    back-chan-nels
    mo-not-o-nous
    ste-reo-typ-i-cal
}

   \boolfalse{bookcompile}
   \togglepaper[1]%%chapternumber
}{}

\pretocmd{\gll}{\def\eachwordone{\itshape}\def\eachwordtwo{\normalfont}}{}{}

\begin{document}
\maketitle
Locative and existential predication have long been central to syntactic-typolog\-i\-cal discussions, with their close relationship emphasized by \citet{Lyons1967} and further explored by \citet{Clark1978}, \citet{Freeze1992}, \citet{Koch2012}, and many others. Although both types of predication are semantically similar in that they locate things, animals, people, and entities in space, the primary difference between them lies in their formal and functional aspects. Specifically, locative predication is characterized by a usually definite referent, which is said to occupy a specific location (\ref{Introduction1}), whereas existential predication primarily serves to introduce a new referent by locating it in a definite place (\ref{Introduction2}).

\ea \label{Introduction1}
\textit{The students are in the class.} \\
\z

\ea \label{Introduction2}
\textit{There are students in the class.} \\
\z


In English, the construction \textit{there is/are} serves as a common formulaic device to express existential predications. However, while many languages feature similar existential elements (e.g., Italian \textit{c’è/ci sono} ‘there is/are’, Spanish \textit{hay} ‘there is/are’, Tagalog \textit{may} ‘there is/are’), not all languages exhibit the same patterns. In some languages, there is little to no formal distinction between locative and existential predication, which complicates their identification. Other languages, such as Russian, may encode existential predications through a simple change in word order, which correlates with the definiteness of the located referent: in locative predications, the referent is prototypically definite and clause-initial, while in existential predications, it is prototypically indefinite and clause-final. Furthermore, it has been argued that existential predication does not necessarily express any specific location but merely asserts the existence of a referent by introducing it into discourse. For this reason, some consider referring to it as a “predication” to be problematic.

This volume not only addresses theoretical questions but also draws on data from various languages to explore the wide range of locative and existential predications, along with their variations, as illustrated in the examples below. Some languages, such as the Australian language Kune, use posture verbs to convey a locative or existential function (\ref{Introduction3}). In (\ref{Introduction4}), a verb meaning ‘to exist’ is employed to express a possessive function, similarly to (\ref{Introduction5}), where the same function is conveyed by a ‘have’-verb. Existential and possessive functions are often expressed by the same form. In fact, existential predications frequently involve a ‘have’-verb (\ref{Introduction6}). Finally, (\ref{Introduction7}) provides an example of an invenitive-locational construction, i.e., a type of locative/existential predication characterized by a verb meaning ‘find'.

\ea \label{Introduction3}
\langinfo{Kune}{Gunwinyguan, [gune1238]}{\cite{YacopettiEtAl2025}, this volume} \\
\gll Kunj na-wern ka-ni. \\
kangaroo \textsc{ii}-many 3\textsc{min}-sit.\textsc{npst} \\
\glt ‘There are lots of kangaroos.’
\z

\ea \label{Introduction4}
\langinfo{Tundra Nenets}{Uralic, [nene1249]}{\cite{Budzisch2025}, this volume} \\
\gll N’e n’a-du tan’a-wi. \\
woman.\textsc{nom.sg} older.sister-\textsc{nom.sg.3pl} exist-\textsc{nar.3sg.s} \\
\glt ‘They had a sister.’
\z

\ea \label{Introduction5}
\langinfo{Sherkaly Khanty}{Uralic, [khan1279]}{\cite{Däbritz2025Have}, this volume} \\
\gll Mā tāpət poχ taj-t-əm. \\
1\textsc{sg.pro} seven son have-\textsc{prs-1sg} \\
\glt ‘I have seven sons.’
\z

\ea \label{Introduction6}
\langinfo{Bulgarian}{Indo-European, [bulg1262]}{\cite{Creissels2025}, this volume} \\
\gll Ima kotka pod masa-ta. \\
have.\textsc{prs.\textsc{i}\textsubscript{s/a}:3sg}\textsubscript{\textsc{expl}}	cat	under	table-\textsc{d}\\
\glt ‘There is a cat under the table.’ 
\z

\ea \label{Introduction7}
\langinfo{Maltese}{Afroasiatic, [malt1254]}{\cite{Basile2025}, this volume} \\
\gll Il-lukanda t-in-sab il-Belt. \\
\textsc{def}-hotel	3\textsc{f.sg.prs}-\textsc{drv7}-find \textsc{def}-City  \\
\glt ‘The hotel is situated/found in Valletta (the City).’
\z

The present volume, \textit{Locative and existential predication: On forms, functions, and neighboring domains}, brings together a dozen or so papers presented at two workshops held in 2023: the Annual SLE Meeting in Athens and the International Conference on Historical Linguistics in Heidelberg. The central aim of the volume, as well as the workshops, is to present typologically diverse and empirically driven research on locative and existential predication. To this end, we deliberately chose not to prepare a structured questionnaire for the contributors or impose any specific theoretical approach. Instead, we left the notions of “locative” and “existential” open to interpretation, allowing authors to make their own theoretical choices based on the specific requirements of their analyses. This decision accommodates a range of methodological approaches and highlights the diversity of perspectives within the field.

The attentive observer of the workshops, the editorial process, and, ultimately, this volume may have noticed that the project initially began with a workshop call titled \textit{Locative and existential predication: Core and periphery}. One of our initial goals was to disentangle core and peripheral coding strategies in the realm of locative and existential predication. However, the typologically diverse contributions soon made it clear that referring to “core” and “periphery” is essentially meaningless when the topic is approached from a strictly functional perspective. As a result, we chose to highlight three issues in the volume’s subtitle: first, the coding strategies of locative and existential predications, i.e., their form; second, the semantic-pragmatic aspects of the domains, i.e., their communicative function; and third, construction-functions closely related to locative and existential predication, such as possessive predication.

In this volume, the glosses adhere to the Leipzig Glossing Rules (LGR). The only glosses that are not included in the standard LGR abbreviations are listed in the individual glossaries of the respective papers. Furthermore, for each language discussed, the Glottocode is provided upon its first mention in each contribution. 

The papers included in this volume cover a wide range of topics, each focusing on a different aspect, yet all contributing to a deeper understanding of the subject at hand. 

The volume begins with a methodological and theoretical paper by \textsc{Martin Haspelmath} titled “Construction-strategies versus construction-functions”, in which he discusses William Croft’s (\citeyear{Croft2022}) concepts of “constructions” and “strategies”. Haspelmath applies these labels to a variety of linguistic phenomena, including existential clauses. In the case of existential clauses, the central argument is that they should be approached from a functional perspective, defined by their semantic-pragmatic features. Accordingly, existential clauses are seen as a construction-function rather than a construction-strategy.

Five papers in this volume address locative and existential clauses in various languages and language families around the world, each approaching the topic from different theoretical perspectives and focusing on different aspects of these construction-functions.

\textsc{Lilián Guerrero} examines locative and existential clauses in several Southern Uto-Aztecan languages, focusing on the type of predicate used in these clauses. While all the analyzed languages share the common feature of using postural and similar verbs in locative and existential clauses, they display considerable micro-typological variation, which Guerrero details extensively. She concludes that, at least in Yaqui [yaqu1251], existential clauses derive from locative clauses, meaning that there is no dedicated existential construction in the language.

\textsc{Anna Kampanarou} analyzes Modern Greek (Indo-European, [mode1248]) locative and existential clauses, with a focus on the structure of the predicate in these clauses. Since the predicate behaves differently in locative and existential clauses with respect to diagnostic parameters such as the optionality of the locative argument, preposition drop in the locative argument, and the scope of quantifiers, Kampanarou argues that the syntactic structure of locative and existential clauses cannot be identical.

\textsc{Anastasia Panova} and \textsc{Henrik Liljegren} explore variation in locative and existential clauses in Gawarbati [gawa1247], an understudied Indo-Aryan language spoken in Afghanistan and Pakistan, extending their analysis to geographically adjacent languages. They show that word order, indefiniteness marking, and the lexical choice of the predicate are the three main factors distinguishing locative from existential clauses. Additionally, they argue that information structure is not a consistent criterion for distinguishing the two clause types.

\textsc{Josefina Budzisch} examines the structure of locative, existential, and possessive clauses in Tundra and Forest Nenets [fore1274], endangered Uralic languages spoken in Western Siberia. Her analysis of corpus data reveals that locative clauses are formed by two copula verbs, displaying an animacy-based split, while existential and possessive clauses typically include an existential verb as the linking element. Furthermore, in existential and possessive clauses, quantified figure and possessee referents correlate with (zero) copula structures rather than the existential verb.

\textsc{Jorge Agulló}’s paper begins with the observation that the Definiteness Effect is strong in Spanish (Indo-European, [stan1288]) existential clauses but weak in their Catalan (Indo-European, [stan1289]) counterparts. He then discusses how Spanish varieties in contact with Catalan behave in this context, showing that the Definiteness Effect weakens in Spanish varieties that are in close contact with Catalan.

It has long been noted that locative and existential clauses share formal properties with possessive clauses in many languages around the world. 
Four papers in this volume examine various facets of these coexpression patterns and the underlying grammaticalization pathways. 

\textsc{Denis Creissels} discusses ‘be/have’-verbs as a type of linking element found in locative, existential, and possessive clauses. He defines these verbs as having the ability to function as ‘have’-verbs in possessive clauses, as existential predicators in existential clauses, and as copulas in locative clauses. In addition to providing a global typological survey of ‘be/have’-verbs, Creissels outlines five grammaticalization pathways for the emergence of these verbs.

\textsc{Chris Lasse Däbritz} analyzes transitive ‘have’-verbs in three Siberian Uralic languages (Khanty [khan1279], Mansi [mans1269], Nganasan [ngan1291]), describing their distributional patterns and arguing that at least Khanty and Mansi provide clear instances of transitive ‘have’-verbs used in existential clauses. He further demonstrates that the Siberian Uralic languages strongly support the theoretical assumptions regarding the spread of transitive ‘have’-verbs into existential clauses.

\textsc{Erin SanGregory} examines locative, existential, and possessive clauses in Wakhi [wakh1245], an Iranian language spoken in Afghanistan, showing that the three clause types share almost identical morphosyntactic structures. Additionally, she argues that it is information structure that disambiguates the three types of clauses.

\textsc{Birsel Karakoç} explores the coexpression of existential and possessive clauses in Turkic languages and the ways in which these clause types can be disambiguated. Most importantly, Karakoç argues that in existential clauses, the possessor and possessee form a single noun phrase, whereas in possessive clauses, they do not. This distinction has several syntactic consequences.

Finally, three papers address additional coexpression patterns and grammaticalization pathways that cannot be subsumed under a single overarching concept.

\textsc{Cristián Juarez} examines locative markers in Mocoví [moco1246], a Guaycuruan language spoken in Argentina, which are essential for the expression of locative clauses. Beyond this function, these markers also play a role in valence extension, demonstrating a multifunctionality that has not yet been acknowledged in typological literature.

\textsc{Rodolfo Basile} provides a description of so-called invenitive verbs in several Indo-European languages. He argues that these verbs are derived from a root meaning ‘find’, and that their middle or passive forms can function as linking elements in locative and existential clauses.

\textsc{Eleanor Yacopetti, Laurits Stapput Knudsen}, and \textsc{Tom Ennever} analyze postural verbs in three Australian languages, investigating their role in locative and existential clauses. The authors show that the use of postural verbs in these clause types depends, among other factors, on the animacy of the figure referent and varies significantly across the languages studied.

In conclusion, this volume provides a comprehensive and diverse exploration of locative and existential predication across languages, highlighting both typological variation and the intricate relationships between form, function, and grammaticalization. The contributions from different linguistic traditions and theoretical perspectives deepen our understanding of these fundamental constructions and encourage a rethinking of the boundaries between “core” and “peripheral” syntactic categories. The papers offer valuable insights into how languages express location, existence, and possession, opening new avenues for understanding the connections between linguistic meaning and grammatical structure. We sincerely thank the authors for their valuable contributions, which have made this project possible. We are especially grateful to Eva Schleitzer, whose dedicated and meticulous assistance in formatting and compiling this volume has been invaluable. Her support greatly facilitated the finalization of this project. We hope this volume will inspire further discussions and research in the field.

\sloppy
\printbibliography[heading=subbibliography,notkeyword=this]
\end{document}
