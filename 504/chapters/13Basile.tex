\documentclass[output=paper,colorlinks,citecolor=brown]{langscibook}
\ChapterDOI{10.5281/zenodo.16838080}
\author{Rodolfo Basile\orcid{}\affiliation{University of Tartu \& Kyoto University}}
%\ORCIDs{}

\title[Invenitive-locational constructions]{Invenitive-locational constructions in the languages of Europe}

\abstract{This paper presents a novel class of construction-strategies called invenitive-locational constructions. These constructions make use of a verb root meaning \textsc{find}, which becomes semantically bleached and behaves like a locational copula. Invenitive-locational constructions functionally express predlocative and existential meaning without semantically marking a specific posture. The constructions at hand are defined at the beginning of the article and then discussed with a focus on a subset of the languages of Europe that includes not only Indo-European languages, but also Maltese, Basque, and three Uralic languages (Finnish, Estonian, and Hungarian). The paper also discusses the semantic and morphosyntactic features of invenitive constructions. The main result of this paper is that invenitive-locational constructions seem to be grouped into two types, according to whether the located element (here: locatum) is animate or inanimate.}



\IfFileExists{../localcommands.tex}{
   \addbibresource{../localbibliography.bib}
   % add all extra packages you need to load to this file

\usepackage{tabularx,multicol}
\usepackage{url}
\urlstyle{same}

\usepackage{listings}
\lstset{basicstyle=\ttfamily,tabsize=2,breaklines=true}

\usepackage{langsci-basic}
\usepackage{langsci-optional}
\usepackage{langsci-lgr}
\usepackage{langsci-osl}
% \usepackage{./langsci/styles/langsci-lgr}
% \usepackage{./langsci/styles/langsci-osl}
% \usepackage{langsci-gb4e}

\usepackage{tikz}
\usetikzlibrary{patterns,calc}
\pgfdeclarepatternformonly{south east lines}{\pgfqpoint{-0pt}{-0pt}}{\pgfqpoint{3pt}{3pt}}{\pgfqpoint{3pt}{3pt}}{
    \pgfsetlinewidth{0.6pt}
    \pgfpathmoveto{\pgfqpoint{0pt}{3pt}}
    \pgfpathlineto{\pgfqpoint{3pt}{0pt}}
    \pgfpathmoveto{\pgfqpoint{.2pt}{-.2pt}}
    \pgfpathlineto{\pgfqpoint{-.2pt}{.2pt}}
    \pgfpathmoveto{\pgfqpoint{3.2pt}{2.8pt}}
    \pgfpathlineto{\pgfqpoint{2.8pt}{3.2pt}}
    \pgfusepath{stroke}}
    
\usepackage{stmaryrd}
\usepackage{wasysym}
\usepackage{multirow}
\usepackage{caption}
\usepackage{subcaption}
\usepackage{mathrsfs}
\usepackage{qtree}

\usepackage{linguex}


   %pminos do not split footnotes
% \interfootnotelinepenalty=10000 %Footnote in Laporte chapters has to be split SN


%\DeclareIndexNameFormat{default}{%
%\nameparts{#1}%
%\usebibmacro{index:name}%
%{\index[names]}%
%{\namepartfamily}%
%{\namepartgiveni}%
% {}% L1
% {}% L2
%{\namepartprefix}% generates spurious space L3
%{\namepartsuffix}% generates spurious space L4
%}

%  {\DeclareIndexNameFormat{default}{%
%     \usebibmacro{index:name}{\index[names]}{#1}{#3}{#5}{#7}}}

%\DeclareIndexNameFormat{default}{%
%  \usebibmacro{index:name}{\sindex[nom]}{#1}{#3}{#5}{#7}}

%\DeclareIndexNameFormat{default}{%
%  \usebibmacro{index:name}{\sindex[person]}{#1}{#3}{#5}{#7}}
%\DeclareIndexNameFormat{default}{%
%\nameparts{#1} \usebibmacro{index:name}{\sindex[person]]}{\namepartfamily}{‌​\namepartgiven}{\nam‌​epartprefix}{\namepa‌​rtsuffix}}

%\newcommand{\smiley}{:)}

%\renewbibmacro*{index:name}[5]{%
%\usebibmacro{index:entry}{#1}%
%{\iffieldundef{usera}{}{\thefield{usera}\actualoperator}\mkbibindexname{#2}{#3}{#4}{#5}}}

% \newcommand{\noop}[1]{}

%remove for final
%\overfullrule=1mm

\newcommand{\tobi}[2]}}
\renewcommand{\S}[1]{\tobi{#1}{\textsc{*}}}

% this volume references
% puts: [this volume]
% already defined: \citetv
%\newcommand{\citepv}[1]{(\citeauthor{#1} \citeyear*{#1} [this volume])}
\newcommand{\citealtv}[1]{\citeauthor{#1} \citeyear*{#1} [this volume]}

%parentheses around example number
\newcommand{\pref}[1]{(\ref{#1})}

% in-text examples

\newcommand{\lnex}[1]{\textit{#1}} %target lang word
\newcommand{\lnlit}[1]{(lit.: `#1')} %literal reading
\newcommand{\lnlat}[1]{(#1)} % latinization
\newcommand{\lntrans}[1]{`#1'} %translation
\newcommand{\lnexl}[2]%
{\lnex{#1}{} \lnlat{#2}} % ex with latinization
\newcommand{\lnexlat}[3]{\lnex{#1}{} \lnlat{#2}{} \lntrans{#3}} % ex with latinization and tranl.

%ch01
\newcommand{\co}[1]{\mbox{\textbf{#1}}}

%ch09

\newcommand{\cyrbulg}[1]{\begin{otherlanguage*}{bulgarian}#1\end{otherlanguage*}}


%ch10
\newcommand{\nlp}{{\small NLP}}
\newcommand{\mwe}{{\small MWE}}
\newcommand{\rae}{{\small RAE}}
\newcommand{\lvc}{{\small LVC}}
\newcommand{\pos}{{\small P}o{\small S}}
%\newcommand{\todo}[1]{ \textcolor{red}{#1} }

%\renewcommand{\labelenumi}{\theenumi}
%\ainamefmt{{vv}{ll}{, ff}{, jj}} % fullname

\newcommand{\biberror}[1]{{\color{red}#1}}

\newcommand{\osenovaitem}{--~}
   %% hyphenation points for line breaks
%% Normally, automatic hyphenation in LaTeX is very good
%% If a word is mis-hyphenated, add it to this file
%%
%% add information to TeX file before \begin{document} with:
%% %% hyphenation points for line breaks
%% Normally, automatic hyphenation in LaTeX is very good
%% If a word is mis-hyphenated, add it to this file
%%
%% add information to TeX file before \begin{document} with:
%% %% hyphenation points for line breaks
%% Normally, automatic hyphenation in LaTeX is very good
%% If a word is mis-hyphenated, add it to this file
%%
%% add information to TeX file before \begin{document} with:
%% \include{localhyphenation}
\hyphenation{
    Beck-man
    Ngu-yen
    back-chan-nel
    back-chan-nels
    mo-not-o-nous
    ste-reo-typ-i-cal
}

\hyphenation{
    Beck-man
    Ngu-yen
    back-chan-nel
    back-chan-nels
    mo-not-o-nous
    ste-reo-typ-i-cal
}

\hyphenation{
    Beck-man
    Ngu-yen
    back-chan-nel
    back-chan-nels
    mo-not-o-nous
    ste-reo-typ-i-cal
}

   \boolfalse{bookcompile}
   \togglepaper[13]%%chapternumber
}{}

%\pretocmd{\gll}{\def\eachwordone{\itshape}\def\eachwordtwo{\normalfont}}{}{}

\begin{document}
\maketitle

\section{Introduction} \label{BasileSection1}

Much of the work done on grammar during the last decades refers to linguistic structures as \textsc{constructions}. Each construction has a specific function that, according to frameworks such as (Cognitive) Construction Grammar\footnote{Variants of Construction Grammar include Radical Construction Grammar (\cite{Croft2001}, \cite{Barðdal2006}) and Diachronic Construction Grammar (\cite{NoelColleman2021}).}
(\cite{Lakoff1987}; \cite{Goldberg1995}), can be deduced from analyzing the construction as a whole. In other words, the meaning or the form of each construction is not inferable from the meanings or forms of its compositional parts. This understanding of constructions has become mainstream, and it has been adopted in typological studies. Works like Haspelmath (\citeyear{HaspelmathNonverbal}), while not explicitly defining the concept of construction or referring to Construction Grammar, use the term construction as a synonym of \textsc{clause construction} or \textsc{clause type}, referring to syntactic configurations that have a specific function.

The aim of this paper is to present a class of locational clause constructions that I call \textsc{invenitive-locational constructions}\footnote{I have already used this concept in my doctoral dissertation, which also contains an earlier draft of the paper at hand (\cite{Basile2024InvenitiveLocational}).}. This term is divided into two parts: the first part indicates the form or the strategy used (\textsc{invenitive}), and the second part refers to the construction’s function (\textsc{locational}). Invenitive-locational constructions can also be called construction-strategies, which means that they are formally defined but functionally express locational meaning (see \cite {Haspelmath2021}; \cite{Haspelmath2025}; \cite{Croft2022}). The term \textit{invenitive} comes from the Latin verb \textit{invenire} ‘to find’, because these strategies are all based on verb roots that primarily have the meaning \textsc{find}. For the verbs used in these constructions, I will use the term \textit{invenitive verbs}. Invenitive-locational constructions are common in European languages, such as Italian (\ref{Basile1}).

\ea \label{Basile1} 
\langinfo {Italian} {Indo-European, [ital1282]} {personal knowledge}\\
\gll Il gatto \textbf{si} \textbf{trova} sull’ albero. \\
\textsc{def}	cat   \textsc{mm.3sg}	find.\textsc{3sg}  	on.\textsc{def}  tree \\
\trans ‘The cat is in the tree.’ \\
\z

The example above expresses a \textsc{predlocative construction} (\cite{HaspelmathNonverbal}). Predlocative constructions have also been known as locative constructions or locative predications, terms that in line with Haspelmath (\citeyear{HaspelmathNonverbal}) do not suffice for designating these clause types, since existential constructions also express location. Because these constructions feature a definite locatum (i.e., the located element), it becomes apparent that the other part of the clause construction is a predicate, more specifically a predicative locative phrase (hence, predicative locative or predlocative). For the sake of clarity, (\ref{Basile1}) can be compared with the examples below, which instead respectively express a predlocative construction with a locational copula (\ref{Basile2a}), a predlocative construction with a posture verb phrase (\ref{Basile2b}), and an \textsc{existential construction} (\ref{Basile2c}). Existential constructions differ from predlocative constructions in that they introduce a discourse-new, indefinite locatum, and hence they do not involve a predicate (\cite{HaspelmathNonverbal}).

\newpage
\ea \label{Basile2} 
\langinfo {Italian} {} {personal knowledge}\\

\ea \label{Basile2a}
\gll Il	gatto \textbf{è} sull’ albero. \\
\textsc{def} cat  be.\textsc{3sg}  on.\textsc{def}  tree \\
\trans ‘The cat is in the tree.’ \\

\ex \label{Basile2b}
\gll Il	gatto \textbf{sta} \textbf{sdraiato}	sull’ albero. \\
\textsc{def} cat  stay.\textsc{3sg} lie.\textsc{pst.ptcp}  on.\textsc{def}  tree \\
\trans ‘The cat is lying in the tree.’  \\

\ex \label{Basile2c}
\gll \textbf{C’è} un	gatto sull’	albero. \\
\textsc{exv} \textsc{indf} cat  on.\textsc{def}  tree \\
\trans ‘There is a cat in the tree.’  \\
\z
\z

Just like in (\ref{Basile2a}), the verb in (\ref{Basile1}) does not add any semantic information to the cat being in the tree, but simply states that the cat is there (cf. (\ref{Basile2b}), where the cat has a specific posture). The verb in (\ref{Basile2a}) becomes semantically bleached, to some extent, of its original lexical meaning \textsc{find}, consequently expressing a locational function as its primary meaning. This locational function can hence be said to be coexpressed by a construction featuring a verb meaning \textsc{find}, that elsewhere would express its original meaning (for coexpression patterns, see \cite{Haspelmath2019}; \cite{Geniušienė1987}). For these reasons, such strategies may be considered locational copulas, i.e., specialized copulas featured in either a predlocative or an existential construction. Invenitive-locational constructions can be defined as being characterized by the specific use of what I call an invenitive verb, which satisfies two main criteria:

\begin{enumerate}
    \item [i.] it has a root with meaning \textsc{find} which undergoes semantic bleaching;
    \item [ii.] it features a morphological or periphrastic valency- or voice-changing marker.
    \item [] In addition, in an invenitive-locational construction, an invenitive verb: 
    \item [iii.] expresses a locational function.
\end{enumerate}

The third criterion specifies one of the possible functions invenitive verbs can have, which is that of conveying a relation between two syntactic arguments, one being the located element and the other a locative phrase, in a nonverbal predlocative or existential clause construction. However, it is worth noting that the third criterion can change, based on what kind of clause constructions invenitive verbs appear in. An example is (\ref{Basile3}).

\ea \label{Basile3} 
\langinfo {Italian} {} {personal knowledge}\\
\gll I	due	\textbf{si}		\textbf{trov-ano}	(bene). \\
\textsc{def.pl} two \textsc{mm.3pl}	find-\textsc{3pl} (well) \\
\trans ‘The two get along.’ \\
\z

The verb used in the Italian example above qualifies as invenitive: It is both semantically bleached (i.e., no actual finding is happening as another meaning takes over, criterion i.) and features a middle marker that here expresses reciprocal meaning (criterion ii.). However, it does not express a locational function. Hence, criterion iii. becomes instead:
\begin{enumerate}
    \item [iii.] expresses the meaning \textsc{get\_along}.
\end{enumerate}
In their current formulation, the definitional criteria of invenitive verbs can be divided in two parts: A formal part, represented by the first two criteria, and a functional part, represented by the third criterion. The definition considers the possibility of expressing a variety of meanings by singling out the third criterion, while the first two remain unchanged. Other examples are given in (\ref{Basile4}). 

\ea \label{Basile4} 


\ea \label{Basile4a}
\langinfo {Italian} {} {personal knowledge}\\
\gll \textbf{Mi}		\textbf{trovo}		male	(a	lavoro). \\
\textsc{mm.1sg} find.\textsc{1sg}	badly	(at	work) \\
\trans ‘I am having a bad/difficult time (at work).’ \\

\ex \label{Basile4b}
\langinfo {Spanish} {Indo-European, [stan1288]} {personal knowledge}\\
\gll \textbf{Me}		\textbf{encuentro}	mal. \\
\textsc{mm.1sg} find.\textsc{1sg}	badly \\
\trans ‘I don’t feel well.’  \\

\ex \label{Basile4c}
\langinfo {English} {} {Jay Zameska, personal communication}\\
\textit{They finally \textbf{found each other}.}  \\
\z
\z

The Spanish invenitive verb in (\ref{Basile4b}) expresses a different, yet related, meaning compared to the Italian invenitive verb in (\ref{Basile4a}) even if they are both used with the same adverb. Furthermore, it is not uncommon to hear utterances such as (\ref{Basile4c}) for expressing the meaning \textsc{fall\_in\_love}. This latter meaning may be directly connected to the idea of concretely finding a romantic partner and may work only in certain contexts, but it contributes to showing why it could be worth looking into the polysemy of \textsc{find}-based strategies more in detail, especially when they present morphosyntactic marking. In the paper at hand, I focus only on those invenitives that express a locational function. I show that invenitive-locational constructions are common in European languages, and that they are arranged according to two main types, based on the animacy of the locatum (i.e., the located element). The sample used for this investigation consists of elicited utterances collected for 13 different European languages.
In \sectref{BasileSection2}, I provide background information about nonverbal and locational predication, as well as some other key concepts. In the following \sectref{BasileSection3}, I comment on the definition proposed above and present some borderline cases of invenitive constructions. \sectref{BasileSection4} is devoted to data and methodology, and \sectref{BasileSection5} reports the main results. The paper ends with a discussion in \sectref{BasileSection6}.

\section{Key concepts} \label{BasileSection2}

This section provides an overview of some key concepts that are relevant for the analysis of invenitive-locational constructions. First, I introduce the general theoretical framework and the work done on locational constructions, since invenitive-locational constructions represent a subset of them. Second, I elaborate on three important factors that are used to analyze invenitive-locational constructions: indexing, middle markers, and mirativity.

\subsection{Locational constructions} \label{BasileSection2.1}

Locational constructions can be identified as subtypes of what Hengeveld (\citeyear{Hengeveld1992}) calls \textit{non-verbal predication}, which refers to all constructions that have a non-verbal main predicate (see also \cite{Roy2013}). Because of the difference between morphosyntactic and semantic analyses, Hengeveld specifies that non-verbal is not a synonym of nominal or verbless; hence instances of non-verbal predication can also correspond to sentences that have a verb, i.e., copula constructions (\cite[26]{Hengeveld1992}). Non-verbal predication is also known under the labels \textit{nonprototypical predication} (\cite[289]{Croft2022}), or \textit{copular clauses} (\cite{Declerck1988}; \cite{Mikkelsen2011}). In Stassen’s (\cite[55--61]{Stassen1997}) terminology, it is called \textit{intransitive predication}, which includes what he calls \textit{locational strategy} (see also \cite{Heine1997}). In Haspelmath (\citeyear{HaspelmathNonverbal}), \textit{nonverbal clause constructions}, in line with Hengeveld’s (\citeyear{Hengeveld1992}) definition, are said to lack a typical verb, i.e., they can feature a copula. Within the seven types of nonverbal clause constructions proposed by Haspelmath, two of them, predlocative constructions and existential constructions, can be subsumed by what has also been called \textit{locational predication} (\cite{Creissels2019}), \textit{locative paradigm} (\cite[554]{Freeze1992}), or \textit{location clauses} (\cite[304]{Croft2022}). Predlocative constructions have also been referred to as \textit{predicational location} (\cite[304]{Croft2022}), while existential constructions are also called \textit{presentational constructions} (\cite[304]{Croft2022}). For some types of what are traditionally considered instances of existential predication, Creissels (\citeyear{Creissels2019}) uses the term \textit{inverse-locational predication}.

The problem with Croft’s term presentational construction is that it can be confused with what Gast and Haas (\citeyear{GastHaas2011}) call \textit{presentationals} (see also \cite{Lambrecht1994}), which represent a formal structure where a discourse-new subject is introduced. This type of configuration does not, however, usually have a locational function. Creissel’s term \textit{inverse-locational predication} is contrasted with \textit{plain-locational predication} and assumes the unmarkedness of the latter, which is encoded by what he calls the prototypical \textit{figure-ground relationship}, i.e. the perspectivization ‘figure>ground’. 
Inverse-locational predication expresses instead a marked perspectivization ‘ground>figure’ (see also \cite{Koch2012}; \cite{McNally2016}; \cite{Creissels2014}; \cite{Huumo2003}). These semantic concepts date back to Talmy (\citeyear{Talmy1983}; \citeyear[311]{Talmy2000}): the figure, also called \textit{pivot} (\cite{Milsark1977}; \cite{BentleyEtAl2013}), is a rather movable entity and the ground (in some traditions called \textit{coda}, see \cite{BentleyEtAl2013}) is typically fixed and unmovable.

Haspelmath (\citeyear{HaspelmathNonverbal}) does not speak of perspectivization but refers to the pivot/figure as the \textit{locatum} in predlocative constructions. The locatum is a definite argument and is said to be in a place expressed by a locative phrase. In existential constructions, the locatum is also called the \textit{existent} and represents a discourse-new, indefinite nominal phrase. According to Haspelmath (\citeyear{HaspelmathNonverbal}), existential constructions obligatorily include a locative phrase\footnote{Existential constructions without a locative phrase are here called \textit{hyparctic} (see also \ref{BasileSection3}.)}. Existential constructions can also feature a specialized \textit{existive} (\cite{HaspelmathNonverbal}), i.e., a restricted copula such as Tagalog \textit{may} or the Italian construction \textit{c’è}, which contains the expletive locational element \textit{ci} (sometimes referred to as a \textit{dummy}, like English \textit{there} in \textit{there is/are} or Swedish \textit{det} in \textit{det finns} ‘there is/are’) and a copula. In this paper, I use the terms \textit{predlocative construction} and \textit{existential construction} as two separate functional instances of locational constructions, similarly to Haspelmath (\citeyear{HaspelmathNonverbal}).

\subsection{Indexing, middle markers, and mirativity} \label{BasileSection2.2}

As I show in Section (\ref{BasileSection5}), types of invenitive constructions are also characterized based on the indexing options of the invenitive verb, namely whether it can appear indexed for all persons or not. For this reason, I adopt the terminology in Haspelmath’s (\citeyear{Haspelmath2013}) article about argument indexing, which distinguishes two types of person forms: speech-role forms, or \textit{locuphoric} forms, that refer to both speaker (grammatically encoded as first persons) and addressee (second persons), and \textit{allophoric} forms (\cite{Dahl2000}), that refer to non-speech-role referents (i.e., third persons). Because of the intrinsically animate nature of first and second persons, we can establish a direct causal relation between animacy and locuphoric forms. Allophoric forms, on the contrary, do not correlate directly with animacy because they do not index speech-role forms. These distinctions become relevant in the typology of invenitive constructions since several languages show different morphosyntactic strategies according to the animacy of the referent and the indexing form of the invenitive verb.

Invenitive verbs are often morphologically or analytically marked via a \textit{middle marker}. I adopt the definition of middle markers (\textsc{mm}s) proposed by Inglese (\citeyear{Inglese2022b}), whose first point is that a \textsc{mm} “occurs with bivalent (or more) verbs to encode one or more of the following valency changing operations: passive, anticausative, reflexive, reciprocal, antipassive” (\cite[494]{Inglese2022b}). These are all valency-reducing operations (\cite{Inglese2022a}, \cite{Inglese2022b}; see also \cite{Kemmer1993}), which confirms the fact that \textsc{mm}s are hence “inherently polyfunctional constructions” (\cite[495]{Inglese2022b}). The underlying idea is that valency-reducing markers are polyfunctional, unlike valency increasing markers (\cite[175]{NicholsEtAl2004}; \cite[161]{Bahrt2021}; \cite{Inglese2022a}, \citeyear{Inglese2022b}). The manipulation of verb valency by \textsc{mm}s takes place in two ways, as brought to our attention by Inglese (\citeyear{Inglese2022a}): on the one hand, reflexives and reciprocals operate on the semantic structure of a two-participant event (see also \cite[95--116]{Givon2001}), on the other hand, passives and antipassives operate only on syntactic valency (\cite{ZunigaKittilä2019}). The same middle marker can coexpress several voices, as pointed out by studies focused on passive syncretism\footnote{The terms coexpression and voice syncretism can be considered synonymous in this context.} (\cite{Haspelmath1990}) or reflexive syncretism (\cite{Geniušienė1987}). These two types of voice syncretism are central in what has also been called middle syncretism (\cite[49]{Bahrt2021}). For example, the Russian suffix \textit{-sya} encodes reflexive, passive, reciprocal and anticausative functions, plus an antipassive function (\cite[140]{Bahrt2021}) that does not belong to the meanings expressed by middle syncretism. The Italian middle-voice marking particle \textit{si} can coexpress, among others, reflexive, anticausative, reciprocal, and antipassive functions, but it is also used in specific medio-transitive constructions (\cite{Masini2012}) or in impersonal constructions. As I discuss further, determining whether the particle \textit{si} expresses a middle, a reflexive or an impersonal function is usually problematic and highly context dependent. For simplicity, I use the term \textit{middle marker} (glossed \textsc{mm}) as a convenience term for all the coexpressed functions.

Some of the subtypes of invenitive constructions presented below are characterized by mirativity, a linguistic category first defined by DeLancey (\citeyear{DeLancey1997}, \citeyear{DeLancey2001}) as conveying information that is new or unexpected to the speaker. Despite its similarity with evidentiality, mirativity is considered as a separate linguistic category. In fact, even though sometimes mirative markers are coexpressed by evidential markers, it has been shown that in some languages the two categories can co-occur, occupying different slots in the verb structure, and even relate differently to other categories (\cite[436]{Aikhenvald2012}). Hengeveld and Olbertz (\citeyear[488]{HengeveldOlbertz2012}) also add that the information expressed by mirative markers can be newsworthy, unexpected or surprising not only for the speaker, but also for the addressee. Some subtypes of invenitive constructions show mirative readings, meaning that they present information that is unexpected or surprising for both participants of the speech act. These readings also seem to correlate with the degree of control that an animate speaker has over the situation and can be grammatically expressed by specialized strategies. It should be noted that mirative meaning does not arise from verb semantics, but rather from the meaning and the pragmatics of the whole construction.

\section{Features of invenitive-locational constructions and borderline cases} \label{BasileSection3}
This section introduces first the pragmatic and usage features of invenitives based on Italian data, then the typical use of invenitives in predlocative and existential constructions, comparing them to copulas. Furthermore, it elaborates on one borderline case: invenitive-hyparctic\footnote{Following Haspelmath (\citeyear{HaspelmathNonverbal}), hyparctic constructions are constructions that express “pure” existence, and do not have a locational element.}  constructions.

\subsection{Some pragmatic and usage remarks on invenitives in Italian} \label{BasileSubsection3.1}
To understand how invenitive-locational constructions behave, I here present the case of Italian, where the verb \textit{trovare} ‘to find’ expresses a variety of functions depending on its morphosyntactic marking, the number of its arguments, the semantic features of these and, ultimately, context. 

\ea \label{Basile5} 
\langinfo {Italian} {} {personal knowledge}\\

\ea \label{Basile5a}
\gll \textbf{Trovate} tutto l’occorrente	in	questa	scatola. \\
find.\textsc{2pl}	all	\textsc{def}-necessary	in	this	box \\
\trans ‘You will find all the necessary in this box.’ \\

\ex \label{Basile5b}
\gll Tutto	l’occorrente	\textbf{è}	in	questa	scatola. \\
all	\textsc{def}-necessary	be.\textsc{3sg}	in	this	box \\
\trans ‘All the necessary is in this box.’   \\

\ex \label{Basile5c}
\gll Tutto	l’occorrente	\textbf{si}	\textbf{trova}		in	questa	scatola. \\
all	\textsc{def}-necessary	\textsc{mm}	find.\textsc{3sg}	in	this	box \\
\trans ‘All the necessary is found in this box.’  \\
\z
\z

\ea \label{Basile6} 
\langinfo {Italian} {} {personal knowledge}\\
\gll Pierpaolo	\textbf{si}	\textbf{trova}		sull’albero.\\
Pierpaolo \textsc{mm}	find.\textsc{3sg} on.\textsc{def}-tree \\
\trans ‘Pierpaolo is on the tree.’ \\
\z

\ea \label{Basile7} 
\langinfo {Italian} {} {personal knowledge}\\
\gll I	due	\textbf{si}	\textbf{trovano}.\\
\textsc{def.pl} two	\textsc{mm}	find.\textsc{3pl} \\
\trans ‘Those two get along.’ \\
\z

\ea \label{Basile8} 
\langinfo {Italian} {} {personal knowledge}\\
\gll \textbf{Mi}	\textbf{trovo}		bene	qui.\\
\textsc{mm.1sg}	find.\textsc{1sg}	well	here \\
\trans ‘I am comfortable here.’ \\
\z

In (\ref{Basile5a}), the verb ‘find’ does not present any special marking, and it could be argued that it is not interpretable in any other way than in the one provided by the verb’s original meaning. However, from a usage perspective, it can be argued that there exist more common (i.e., frequent) ways of conveying the same piece of information, e.g., by using a copula (\ref{Basile5b}). Instead, (\ref{Basile5a}) specifically addresses a generic addressee, who is then given instructions about the whereabouts of something. In addition, (\ref{Basile5a}) implies pragmatic presupposition, in that what is assumed to be part of the common ground is that the addressee is supposed to be looking for the contents of the box, whereas by employing a regular \textsc{be}-copula this feature is not overt. Similarly, (\ref{Basile5c}) may also encode pragmatic presupposition, albeit through a different grammatical strategy.

In (\ref{Basile6}), what appears to be a full-fledged verb phrase is in fact a locational copula. The Italian verb \textit{trovare} ‘to find’ is bivalent, in that it encodes both an agent (commonly referred to as A) and a patient (P). When associated with the \textsc{mm} \textit{si}\footnote{Also called Italian Reflexive \textit{si}. Language-specific concepts are indicated with capital letters and are to be distinguished from comparative concepts (see, e.g., \cite{Haspelmath2019}).}, it changes its argument structure and mainly expresses anticausative, reflexive (including grooming events, such as ‘to shave’), or reciprocal meaning, but also what Inglese (\citeyear{Inglese2022b}) calls \textit{impersonal agentless passive} and what Masini (\citeyear[11]{Masini2012}) calls \textit{medio-transitive construction}\footnote{It. ”costruzione media transitiva”.}. I refer to Inglese (\citeyear[§2.1]{Inglese2022b}) for the whole range of functions encoded by the Italian \textsc{mm} \textit{si}. However, there is one specific function that the literature has so far not accounted for, namely the expression of locational meaning through the perhaps particular case provided by the combination of the verb \textit{trovare} ‘to find’ and the \textsc{mm} \textit{si}, which yields the intransitive \textit{trovarsi} ‘find.\textsc{mm}’. This strategy conveys a range of possible readings, represented by the following constructions:

\begin{itemize}
    \item[a.] A finds P in X, where A and P are coreferential (\textsc{reflexive});
    \item[b.] A’ finds P in X, where A’ refers to a generic referent (\textsc{impersonal});
    \item[c.] S finds.\textsc{mm} in X, where the verb undergoes semantic bleaching (\textsc{locational}).
\end{itemize}

In these constructions, native speaker intuition is often misleading. Because the \textsc{mm} \textit{si} is multifunctional, it is difficult to determine whether we can talk about one or the other construction. Possibly because educated Italian speakers often refer to the \textsc{mm} \textit{si} as a Reflexive marker, this interpretation is usually considered the most plausible, even though the term only refers to the formal characteristics of the marker, not necessarily to its function and meaning. This is not to say that Italian \textit{trovarsi} can never be interpreted as a reflexive: It can inasmuch as it is representing specific constructions in which it is clear that (i) the meaning \textsc{find} is maintained, and (ii) the event encoded is a proper reflexive event, where there is coreferentiality between A and P (\cite[154]{ZunigaKittilä2019}). However, most of the Italian examples are up to debate because it is unclear whether the meaning \textsc{find} is maintained. Additionally, such middle-marked forms in (\ref{Basile6}), if interpreted transitively, could convey impersonal meaning rather than reflexive meaning (people are looking for Pierpaolo > he can be found on the tree). Perhaps what comes closer to a reflexive construction is a construction involving the intransitive verb \textit{trovare} ‘to find’ and an analytical reflexive nominal (\ref{Basile9}, cf. \ref{Basile10}).

\ea \label{Basile9} 
\langinfo {Italian} {} {personal knowledge}\\
\gll Pierpaolo	\textbf{trova}	\textbf{se} \textbf{stesso}	(sull’albero).\\
Pierpaolo find.\textsc{3sg} \textsc{refl} self (on.\textsc{def}-tree) \\
\trans ‘Pierpaolo finds himself\footnote{Italian \textit{sé/se stesso} only encodes a reflexive function and is hence different from the \textsc{mm} \textit{si}. Contrarily to \textit{si}, it is always postverbal.}  (on the tree).’ \\
\z

\ea \label{Basile10} 
(CoCA)\\
\textit{I don't know why I \textbf{find} \textbf{myself} tearing up nearly every day now.}  \\
\z

(\ref{Basile9}) and (\ref{Basile10}) are fundamentally different, in that the Italian example entails the literal meaning ‘find’, while in English such uses of the verb \textit{to find} are common constructionalized ways to represent unexpectedness and encoding mirative meaning. The meaning of finding oneself is not necessarily retained, so the verb can also be considered as semantically bleached of its primary meaning\footnote{In the English example (\ref{Basile10}) there are three clues as to why we could, but should not, consider it as a reflexive construction rather than something else. First, English uses the analytical pronoun \textit{myself} as a reflexive marker, and it is undoubted that here, at least on a semantic and morphosyntactic level, the action of finding is being performed on the same referent that is performing it (coreferentiality). Second, the verb ‘find’ marks a low-control event and its semantics often correlate with mirative meaning. Finding something (and, by extension, oneself) involves a lower degree of control and intention compared to e.g., eating, sending, or looking for something. Even when the intentional process of searching precedes the finding, the latter may come as a surprise to the utterer (cf. \textit{I found my jacket in the trash} vs \textit{I found the jacket [the one I was looking for]} vs \textit{I went to pick up my jacket}). Third, the context helps confirming the low control of the action, as the speaker specifies \textit{‘I don’t know why’}.}
In (\ref{Basile6}) above it could be the case that the referent was not in full control of climbing the tree either, but this interpretation must be left aside as there are no context clues or pragmatic presupposition devices that mark whether it is true or not. Two unambiguously mirative examples that contrast (\ref{Basile6}) are given below (\ref{Basile11}).

\ea \label{Basile11} 
\langinfo {Italian} {} {personal knowledge}\\
\ea \label{Basile11a}
\gll Pierpaolo	\textbf{si}	\textbf{trova}		sull’albero	{a	sua	insaputa}.\\
Pierpaolo		\textsc{mm}	find.\textsc{3sg}	on.\textsc{def}.tree	[unbeknownst to him] \\
\trans ‘Pierpaolo finds himself on the tree, unbeknownst to himself.’ \\
\ex \label{Basile11b}
\gll Pierpaolo	\textbf{si}	\textbf{ritrova} sull’albero.\\
Pierpaolo \textsc{mm}	\textsc{mir}.find.\textsc{3sg}	on.\textsc{def}.tree \\
\trans ‘Pierpaolo finds himself on the tree.’\\
\z
\z

While (\ref{Basile11a}) uses an overt periphrastic device to specify the unwillingness of Pierpaolo being on the tree, (\ref{Basile11b}) uses a morphological one. Based on my native speaker intuition, I argue here that the prefix \textit{ri-} adds unambiguous mirative meaning to this verb, but further research is needed to assess whether this is an intrinsic characteristic of this polyfunctional morpheme or just a lexical feature of the verb \textit{ritrovarsi} ‘to end up/find oneself at/doing’ (see \cite{Cardinaletti2003} for an analysis of the Italian morpheme \textit{ri-}).

\subsection{Invenitives as copulas} \label{Basile3.2}

Invenitive strategies can equally express both predlocative and existential constructions, as shown respectively by (\ref{Basile12a}) and (\ref{Basile12b}), although they more commonly express predlocative constructions.

\ea \label{Basile12} 
\langinfo {Russian} {Indo-European, [russ1263]} {Anna Branets, personal communication}\\
\ea \label{Basile12a}
\glll Eда	наход-ит-ся	на	столе.\\
Eda	\textbf{nachod-it-sya}	na	stole.\\
food	find-\textsc{3sg-mid}	on	table.\textsc{prep} \\
\trans ‘The food is on the table.’ \\
\ex \label{Basile12b}
\glll На	столе		наход-ит-ся	еда\\
Na 	stole \textbf{nachod-it-sya}	eda\\
on	table.\textsc{prep}	find-\textsc{3sg-mid}	food \\
\trans ‘There is food on the table.’ \\
\z
\z

Because invenitive-locational constructions are coexpressed by a verb root that usually maintains its lexical meaning, semantic bleaching is a central criterion in their definition. The verb meaning \textsc{find} becomes a sort of copula. According to Haspelmath (\citeyear{HaspelmathNonverbal}), common copulas indicate “a stative link between the two argument positions of an equational, ascriptive or locational clause (i.e., a predlocative or existential clause)”. Invenitives do indicate this stative link but do so in a morphologically or periphrastically marked way, for example by employing a grammatical strategy that falls within the middle-voice spectrum. We can then say that invenitive verbs are special copulas that have a specific morphosyntactic contour. Because they are copulas that express a locational function, instances of verbal predication in which the verb retains its original meaning \textsc{find} do not belong to invenitive-locational constructions. In some cases, there is a blurry demarcation line between verbal and non-verbal predication, which makes some instances of invenitive-locational constructions difficult to identify. This problem arises from the fact that middle forms, in languages like Italian, can coexpress several functions, such as impersonals (\ref{Basile13a}; see also § \ref{BasileSubsection3.1}).

\ea \label{Basile13}
\langinfo {Italian} {} {personal knowledge}\\

\ea \label{Basile13a}
\gll Le	patate		\textbf{si}	\textbf{trovano}	al	mercato?\\
\textsc{def.pl} potato.\textsc{pl}	\textsc{mm}	find.\textsc{3pl}	at.\textsc{def}	market \\
\trans ‘Can one find potatoes at the market?’ \\

\ex \label{Basile13b}
\gll Le	patate		\textbf{ci}	\textbf{sono}		al	mercato?\\
\textsc{def.pl}	potato.\textsc{pl}	\textsc{expl}	be.\textsc{3pl}	at.\textsc{def}	market \\
\trans ‘Are there potatoes at the market?’ \\

\ex \label{Basile13c}
\gll Le	patate		\textbf{si}	\textbf{trovano}	sottoterra.\\
\textsc{def.pl}	potato.\textsc{pl}	\textsc{mm}	find.\textsc{3pl} underground \\
\trans ‘Potatoes are located underground.’\\

\ex \label{Basile13d}
\gll *Le	patate		\textbf{si}	\textbf{ritrovano}	sottoterra. \\
\textsc{def.pl}	potato.\textsc{pl}	\textsc{mm} \textsc{mir}.find.\textsc{3pl}	underground \\
\trans *‘Potatoes end up underground.’ \\

\ex \label{Basile13e}
\gll Pierpaolo	non	\textbf{si}	\textbf{trova}		nella	foto. \\
Pierpaolo \textsc{neg} \textsc{mm}	find.\textsc{3sg}	in.\textsc{def}	picture \\
\trans ‘Pierpaolo isn’t found / can’t find himself in the picture.’ \\
\z
\z

The verb in (\ref{Basile13a}) does not unambiguously express a locational function and seems to presuppose an actual search for the referent indicated by the subject, possibly also because the sentence is in the form of a question. In (\ref{Basile13a}), it would also be problematic to determine its function in the affirmative form if the utterance had a prosodic emphasis on \textit{si trovano} (personal knowledge). This may indicate that the verb should be interpreted as expressing its original meaning, and the sentence as an instance of verbal predication. However, (\ref{Basile13a}) is semantically similar to (\ref{Basile13b}), which is instead commonly considered a locational construction, hence nonverbal. Conversely, in (\ref{Basile13c}) the semantic bleaching is more evident, and the sentence can be characterized as an invenitive-locational construction \footnote{It is worth pointing out that, like in (\ref{Basile13a}), here suprasegmental features such as prosodic marking may come into play as well. A prosodic emphasis on \textit{si trovano} in (\ref{Basile13c}) would also entail a change in the utterance’s meaning towards an impersonal interpretation. Similarly, an unmarked interrogative \textit{Le patate si trovano sottoterra?} would retain a locational meaning, while a prosodically marked version with an emphasis on the verb would likely not.}. One of the reasons might be pragmatic: While it may not be obvious to the interlocutor that potatoes can be found at the market (because maybe they are not available at a certain point in time), potatoes do grow underground, where they are assumed to be permanently (until somebody picks them up). Hence, it is more likely that the speaker is not implying an actual search for the potatoes underground, but rather that she is stating a fact. This reasoning is in line with semantic nuances like the permanent (against the temporary) presence of a referent in a specific location and its movability, which could be one of the reasons why in certain cases it is more likely to consider the invenitive verb as semantically bleached (hence, as purely conveying locational information rather than giving the utterance a specific pragmatic nuance). This difference in usage might be relevant in other languages, and it certainly requires further investigation. (\ref{Basile13d}) shows that the verb \textit{ritrovarsi} ‘to end up (in/doing sth)’ only works with animate referents, since it encodes a mirative/non volitional reading. In (\ref{Basile13e}), there is again ambiguity in the sentence’s reading, since the marker \textit{si} could be interpreted as both impersonal (i.e., ‘Pierpaolo can’t be found/isn’t [found] in the picture’) and reflexive (i.e., ‘Pierpaolo can’t find himself in the picture’)\footnote{I thank Jorge Agulló for providing example (\ref{Basile13e}).}. The first reading resembles an invenitive-locational construction, while the second reading does not. (\ref{Basile13e}) shows once again that, when the located referent is animate, it becomes more difficult to determine exactly which construction we are dealing with. Pragmatics and context play once again a fundamental role in establishing whether one or the other reading arises.

\subsection{Usage of invenitives in hyparctic and possessive constructions in Finnish} \label{Basile3.3}

The term \textit{hyparctic} has first been suggested by Haspelmath (\citeyear{HaspelmathNonverbal}) to indicate all those nonverbal clause constructions that express “pure existence” without expressing a locational function. Hyparctic constructions are often coexpressed by morphosyntactic devices like existives, often also employed in existential sentences (\ref{Basile14}).

\ea \label{Basile14}
\langinfo {Italian} {} {personal knowledge}\\
\gll Dio \textbf{c’è}. \\
God	\textsc{exv} \\
\trans ‘God exists.’ \\
\z

Similarly, invenitive verbs may appear, at least in Finnish, in hyparctic constructions in lieu of the copula \textit{olla} ‘to be’. They hence fulfill all the formal criteria that define invenitive constructions, except for conveying a locational function (\ref{Basile15a}). Further evidence for why invenitives in Finnish are to be considered copulas is given by their usage in possessive constructions (\ref{Basile15b}).

\ea \label{Basile15}
\langinfo {Finnish} {} {Uralic, [finn1318]}\\
\ea \label{Basile15a}
(\cite[32]{BasileIvaska2021}) \\
\gll Erojakin			toki		\textbf{löytyy}.\\
difference.\textsc{pl.ptv.encl}	certainly	find.\textsc{mm.3sg} \\
\trans ‘There are certainly also differences.’ \\
\ex \label{Basile15b}
(\cite[2001]{Suomi24Corpus}) \\
\gll Minulla  \textbf{löytyy} kokemusta koirista. \\
1\textsc{sg.ade} find.\textsc{mm.3sg} experience.\textsc{ptv} dog.\textsc{pl.ela} \\
\trans 'I have experience with dogs.' \\
\z
\z

Using quantitative methods, Basile and Ivaska (\citeyear{BasileIvaska2021}) argue that in Finnish \textsc{find}-based strategies often replace the copula \textit{olla} ‘to be’, typically also used in hyparctic constructions and possessive constructions. Possessive and existential constructions are often investigated in tandem (see, e.g., \cite{HaspelmathNonverbal}; \cite{Creissels2023}), since in many languages the possessor can be morphosyntactically considered a locative phrase (in Finnish marked for adessive, one of the locative cases). While invenitives are not the most common strategy of expressing possession in Finnish, their usage is spread enough to be worthy being investigated as a construction(-strategy) of its own. Based on the Finnish data, it can be said that invenitive verbs can be used in three different functional types of constructions: invenitive-locational constructions (further divided into invenitive-predlocative and invenitive-existential constructions), invenitive-hyparctic constructions, and invenitive-possessive constructions.

\section{Data and methodology} \label{BasileSection4}
The aim of this paper is to show the usage of invenitive-locational constructions by analyzing first-hand data from a sample of European languages. With European languages I mean languages spoken in Europe, hence I refer not only to Indo-European languages, but also to Afro-Asiatic (Maltese) or Uralic (Finnish, Estonian, and Hungarian) languages, and to the isolate Basque. The languages analyzed are listed in \tabref{tab:BasileTable1}.

In the next Section, I give an overview of the types of invenitive-locational constructions in the languages of the sample and comment on what characterizes them. The data were either elicited from native speakers or collected on the web, and in this latter case the grammaticality of the examples was checked with native speakers and language experts. The language sample is based on availability of native speakers or experts for consultation.

\begin{table}
    \begin{tabularx}{\textwidth}{XXl}
\lsptoprule
     {Language} &  {Glottocode} &  {Language family, group} \\
    \midrule
    Albanian & alba1267 & Indo-European \\
    Basque & basq1248 & isolate \\
    English & stan1293 & Indo-European, Germanic \\
    Estonian & esto1258 & Uralic, Finnic \\
    Finnish & finn1318 & Uralic, Finnic \\
    German & stan1295 & Indo-European, Germanic \\
    Greek & mode1248 & Indo-European, Hellenic \\
    Hungarian & hung1274 & Uralic, Ugric \\
    Italian & ital1282 & Indo-European, Romance \\
    Latvian & latv1249 & Indo-European, Baltic \\
    Maltese & malt1254 & Afroasiatic, Semitic \\
    Russian & russ1263 & Indo-European, Slavic \\
    Sardinian & sard1257 & Indo-European, Romance \\
    \lspbottomrule
    \end{tabularx}
    \caption{Language sample}
    \label{tab:BasileTable1}
\end{table}

\section{Types of invenitive-locational constructions} \label{BasileSection5}

In this section, I present and analyze the two main types of invenitive-locational constructions in my sample. For each language, I have found at least one example confirming that invenitive-locational constructions are used. Although marginal in Basque and Hungarian, invenitive-locational constructions seem to be spread across the whole European linguistic area, which points to the possibility of an areal phenomenon. Invenitive constructions are arranged according to the semantic criterion of the animacy of the referent expressed by the locatum, which establishes two types. The main features of these two types are shown in \tabref{tab:BasileTable2}.

In my sample, the general tendency is that all languages feature Type 2 constructions with an inanimate locatum, a concrete locative phrase and allophoric forms, while Type 1 may also be either absent (e.g., Hungarian) or confined only to some subtypes (e.g., only with allophoric forms, see Finnish and Estonian below). As shown in the table, Type 1 has an animate locatum that can be found in a concrete or abstract/metaphorical location expressed by a locative phrase. Because the locatum is animate, the invenitive verb can appear in both locuphoric (\ref{Basile16a}--\ref{Basile16b}) and allophoric (\ref{Basile16c}) forms, as shown by Campidanese Sardinian.

\begin{table}
\begin{tabularx}{\textwidth}{QQl}
\lsptoprule
 &  {Type 1}: &  {Type 2:} \\
  {animate locatum} &  \\  {inanimate locatum} \\
  \midrule
Criterion 1 (semantic): concreteness of the locative phrase &  concrete/abstract & concrete \\
\midrule
Criterion 2 (morphosyntactic): verb indexing &  both locuphoric and allophoric forms (with language-specific differences) &  allophoric forms \\
\midrule
Language-specific subtypes: mirativity &  mirative reading possible &  N/A \\ \lspbottomrule
\end{tabularx}
    \caption{Invenitive-locational construction types}
    \label{tab:BasileTable2}
\end{table}

\ea \label{Basile16}
\langinfo {Campidanese Sardinian} {} {personal documentation}\\
\ea \label{Basile16a}
\gll \textbf{M’=agattu} in	Casteddu. \\
\textsc{mm.1sg}=find.\textsc{1sg}	in	Cagliari \\
\trans ‘I am in Cagliari.’ \\
\ex \label{Basile16b}
\gll \textbf{M’=agattu} in	una	situatzioni	malla. \\
\textsc{mm.1sg}=find.\textsc{1sg}	in	\textsc{indf} situation	bad \\
\trans ‘I find myself in a bad situation.’ \\
\ex \label{Basile16c}
\gll Su	pisci \textbf{si} \textbf{agattara}	in	s’=acqua. \\
\textsc{def} fish \textsc{mm.3sg} find.\textsc{3sg}	in	\textsc{def}=water \\
\trans ‘The fish is in the water.’ \\
\z
\z

Type 2, instead, has an inanimate locatum that can be found mainly in a concrete location expressed by a locative phrase, and the invenitive verb appears only with allophoric forms (\ref{Basile17a}). Abstract locations are rare and appear to be directly correlated with locati that encode abstract referents\footnote{See \citet[16, 33]{BasileIvaska2021} for a few examples on Finnish.} (\ref{Basile17b}).

\ea \label{Basile17}
\langinfo {Italian} {} {personal knowledge}\\
\ea \label{Basile17a}
\gll L’=Italia	\textbf{si}		\textbf{trov-a}		nel	Mediterraneo. \\
\textsc{def}=Italy	\textsc{mm.3sg}	find-\textsc{3sg}	in.\textsc{def} Mediterranean.sea \\
\trans ‘Italy is situated in the Mediterranean.’ \\
\ex \label{Basile17b}
\gll La	felicità		\textbf{si}		\textbf{trov-a}		nelle		piccole	cose. \\
\textsc{def}	happiness	\textsc{mm.3sg}	find-\textsc{3sg}	in.\textsc{def.pl}	little.\textsc{pl} thing.\textsc{pl} \\
\trans ‘Happiness is found in little things.’ \\
\z
\z

Italian represents one of the most consistent languages with regards to invenitive constructions, in that it uses the middle form not only for both types (i.e., with animate and inanimate locati), but also indiscriminately with all grammatical persons, and with both concrete and abstract locations. The same forms are also allowed, as already introduced in \sectref{BasileSection3}, for mirative readings – compare (\ref{Basile18a}) with (\ref{Basile18b}) and (\ref{Basile18c}).

\ea \label{Basile18}
\langinfo {Italian} {} {personal knowledge}\\
\ea \label{Basile18a}
\gll \textbf{Ti}	\textbf{trovi} a	casa. \\
\textsc{mm.2sg}	find.\textsc{2sg}	at	home \\
\trans ‘You are at home.’ – non-mirative reading \\
\ex \label{Basile18b}
\gll \textbf{Ti}		\textbf{ri-trovi}		a	casa. \\
\textsc{mm.2sg}	\textsc{mir}-find.\textsc{2sg}	at	home \\
\trans ‘You find yourself/end up at home.’ – mirative reading \\
\ex \label{Basile18c}
\gll Ti	\textbf{trovi}		in	una	situazione	spiacevole. \\
\textsc{mm.2sg}	find.\textsc{2sg}	in	\textsc{indf}	situation	unpleasant \\
\trans ‘You find yourself in an unpleasant situation.’ – mirative reading \\
\z
\z

The Italian prefix \textit{ri-/re-} is productive and usually expresses the meaning \textsc{again}. If used with an invenitive verb, however, it does not have any iterative meaning, at least not in the prototypical sense. Instead, it marks a mirative event. When it is absent, the only way to determine whether the utterance has a mirative reading is by looking at the pragmatic context (cf. \ref{Basile18a} and \ref{Basile18c}). \tabref{tab:BasileTable3} below summarizes the features of invenitive-locational constructions in the languages under investigation.



\begin{table}[b]
\footnotesize
\begin{tabularx}{\textwidth}{lQQQQQQQ}
\lsptoprule
       {Language} & {Animate} \mbox{locatum\footnote{In this column, Basque and English are marked ({-}) because they allow for an animate locatum only in their dedicated mirative strategies (e.g., \textit{I found myself at the hospital, *I am found at the hospital}). Similarly, Estonian and Finnish only allow animate locati if these are marked for (partitive) plural.}} & {Inanimate locatum} & {Concrete location} & {Abstract location} & {Locuphoric forms} & {Allophoric forms} & {Dedicated mirative strategy} \\
\midrule
Albanian & ({+}) & ({+}) & ({+}) & ({+}) & ({+}) & ({+}) & ({-}) \\
Basque & ({-}) only mirative & ({+}) & ({+}) & ({+}) & ({-}) only  mirative & ({+}) & ({+}) \\
English & ({-})  only mirative & ({+}) & ({+}) & ({+}) & ({-}) only mirative & ({+}) & ({+}) \\
Estonian & ({+}) \mbox{except \textsc{sg}} & ({+}) & ({+}) & ({-}) & ({-}) & ({+}) & ({+}) \\
Finnish & ({+}) \mbox{except \textsc{sg}} & ({+}) & ({+}) & ({-}) & ({+}) marginal & ({+}) & ({+}) \\
German & ({+}) & ({+}) & ({+}) & ({+}) & ({+}) & ({+}) & ({-}) \\
Greek & ({+}) & ({+}) & ({+}) & ({+}) & ({+}) & ({+}) & ({-}) \\
Hungarian & ({-}) & ({+}) & ({+}) & ({-}) & ({-}) & ({+}) & ({-}) \\
Italian & ({+}) & ({+}) & ({+}) & ({+}) & ({+}) & ({+}) & ({-})\footnote{Even though the verb \textit{ritrovarsi} ‘to end up (in/doing sth)’ always has a mirative or non-volitional reading, and hence could be thought of as a dedicated strategy, Italian invenitives do not in principle need the additional morphemic marking ri- to express mirative or non-volitional meaning. For this reason, I do not consider Italian as having a(n obligatory) dedicated mirative strategy.}  \\
Latvian & ({+}) & ({+}) & ({+}) & ({+}) & ({+}) & ({+}) & ({-}) \\
Maltese & ({+}) & ({+}) & ({+}) & ({-}) & ({+}) & ({+}) & ({+}) \\
Russian & ({+}) & ({+}) & ({+}) & ({+}) & ({+}) & ({+}) & ({-}) \\
Sardinian & ({+}) & ({+}) & ({+}) & ({+}) & ({+}) & ({+}) & ({-}) \\
\lspbottomrule
    \end{tabularx}
    \caption{Features of invenitive-locational constructions in the sample}
    \label{tab:BasileTable3}
\end{table}

The languages marked with ({+}) in the “Animate locatum” column also allow for mirative readings, and in all of them such mirative readings are conveyed by the same construction that conveys non-mirative readings. In the same column, ({-}) indicates that animate locati are completely absent (i.e., Hungarian), or that animate locati are possible only when there is a mirative reading involved (i.e., Basque and English). Estonian and Finnish are the exception, in that they allow for plural-marked animate locati only, while when they are singular they present a specialized analytical construction (cf. \sectref{BasileSection5.1}). Maltese is the only language that has an additional dedicated mirative strategy (cf. \sectref{BasileSection5.2}).

Hungarian only uses the participial form \textit{található} (find.\textsc{pot.prp}). Its usage is limited to inanimate locati and concrete locations (\ref{Basile19}).  

\ea \label{Basile19}
\langinfo {Hungarian}{}{Bogáta Timár, personal communication} \\
\gll Magyarország	Európában \textbf{talál-hat-ó}. \\
Hungary	Europe.\textsc{ine}	find-\textsc{pot-prp} \\
\trans ‘Hungary is located in Europe.’ \\
\z

Locuphoric forms and animate locati are strongly correlated categories, but not all languages of the sample allow for such pronominal uses, albeit allowing for animate locati in certain contexts (see also \sectref{BasileSection5.1}). The literature on Finnish shows that locuphoric forms of the invenitive verb \textit{löytyä} are used yet marginal (\cite{BasileIvaska2021}; \cite{Basile2024}, \cite{Basile2024InvenitiveLocational}). However, the closely related Estonian language behaves differently, as locuphoric (pronominal) forms are not allowed.

\subsection{Differences within Type 1: locuphoric and allophoric forms} \label{BasileSection5.1}

Some languages like Finnish or Estonian employ specialized grammatical strategies depending on whether the invenitive verb presents a locuphoric or an allophoric form. In fact, while the Finnish invenitive verb \textit{löytyä} (\ref{Basile20}) and the Estonian invenitive verb \textit{leiduma} (\ref{Basile21}) can be used both with inanimate (\ref{Basile20a}; \ref{Basile21a}) and animate (\ref{Basile20b}; \ref{Basile21b}) referents, they are usually not used with locuphoric forms (first or second persons), except for a few rare cases\footnote{See Basile (\citeyear{Basile2024InvenitiveLocational}) for an analysis of locuphoric forms used with the Finnish verb \textit{löytyä} in web corpora.} (\ref{Basile22}). This means that the verb is usually in the third person. Animate referents are also often limited to groups of animals or people rather than single individuals.

\ea \label{Basile20}
\langinfo {Finnish} {} {personal knowledge}\\
\ea \label{Basile20a}
\gll Jääkaapista	\textbf{löyt-y-y}	omenoita. \\
fridge.\textsc{ela}	find-\textsc{mm-3sg}	apple.\textsc{pl.ptv}\\
\trans ‘There are apples in the fridge.’ \\
\ex \label{Basile20b}
\gll Metsästä	\textbf{löyt-y-y}	erilaisia		ötököitä. \\
forest.\textsc{ela}	find-\textsc{mm-3sg}	different.\textsc{pl.ptv}	bug.\textsc{pl.ptv} \\
\trans ‘There are all sorts of bugs in the forest.’ \\
\z
\z

\ea \label{Basile21}
\langinfo {Estonian} {} {etTenTen – Web 2019\footnote{Information about the corpus can be found at \url{https://app.sketchengine.eu/\#dashboardcorpname=preloaded\%2Fettenten19_fil2&corp_info=1} (accessed on November 9th, 2022)}}\\
\ea \label{Basile21a}
\gll Metsa		all	\textbf{leid-u-b}	kukeseeni		ja metsamaasikaid. \\
forest.\textsc{gen}	under	find-\textsc{mm-3sg}	chanterelle.\textsc{pl.ptv}	and wild.strawberry.\textsc{pl.ptv} \\
\trans ‘There are chanterelles and wild strawberries on the forest floor.’
\ex \label{Basile21b}
\gll Maailmas	\textbf{leid-u-b}	veel	häid		inimesi. \\
world.\textsc{ine}	find-\textsc{mm-3sg}	still	good.\textsc{pl.ptv}	person.\textsc{pl.ptv} \\
\trans ‘There are still good people in the world.’ \\
\z
\z

\ea \label{Basile22}
\langinfo {Finnish} {} {\cite[18]{BasileIvaska2021}}\\
\gll Mistä		\textbf{löyd-y-t}		prinssini,	44–50-v.	fiksu,	pitkähkö, ulkonäkö	ok,	pilke		silmäkulmassa,	lenkkeilet	ja tanssit. \\
where.\textsc{ela}	find-\textsc{mm-2sg}		prince.\textsc{poss1sg}	44--50-y.o.	smart	tallish appearance	ok	twinkle		eye.corner.\textsc{ine}	jog.\textsc{2sg}	and dance.\textsc{2sg} \\
\trans ‘Where are you my prince, 44 to 50 years old, smart, tallish, good looking, with a twinkle in your eye, who likes to jog and dance.’ \\
\z

The specialized allophoric form and its abundant use with partitive-marked NPs (\cite{BasileIvaska2021}; see \cite{Ylikoski2023} for similar ‘inessive-subject’ constructions), which are traditionally considered existential NPs (\cite{HuumoHelasvuo2015}, \cite{HuumoLindström2014}, \cite{Metslang2012}), could be the sign of the grammaticalization of this invenitive form into a dedicated existive (like Spanish \textit{hay}), although this claim must be substantiated with further research. For the locuphoric forms, hence exclusively with animate referents, Estonian and Finnish use a dedicated analytical reflexive strategy (\ref{Basile23a}). This means that the verb does not have a reflexive marking, instead the transitive verb \textit{löytää} ‘to find’ is used with an accusative-marked reflexive pronoun (in Estonian, the verb \textit{leidma} ‘to find’ is used with the partitive-marked reflexive pronoun \textit{end/ennast}).

\ea \label{Basile23}
\langinfo {Finnish} {} {personal knowledge}\\
\ea \label{Basile23a}
\gll \textbf{Löysin}		itseni		keskeltä	metsää \\
find.\textsc{prt.1sg}	self.\textsc{acc.1sg} middle.\textsc{abl}	forest.\textsc{ptv} \\
\trans ‘I found myself in the middle of the forest.’ \\
\ex \label{Basile23b}
\gll \textbf{Olin}		yhtäkkiä	keskellä	metsää. \\
be.\textsc{prt.1sg}	suddenly	middle.\textsc{ade}	forest.\textsc{ptv}\\
\trans ‘Suddenly, I was in the middle of the forest.’ \\
\z
\z

Similarly to English (cf. \ref{Basile10}), this analytical strategy seems to be a constructionalized way used to mark a mirative event, which could in turn be directly motivated by the animacy of the referent. On the one hand, the reason why this reading arises could also lie in the higher transitivity of this type of analytical constructions, where the verb can be perceived as conveying its original meaning. On the other hand, semantically analogous mirative copula constructions like (\ref{Basile23b}) may usually be considered locational constructions.

\subsection{Differences between types: the animacy criterion} \label{BasileSection5.2}

According to the animacy criterion, invenitive verbs can have different dedicated grammatical forms. Consider English (\ref{Basile24}).

\ea \label{Basile24}
\langinfo{American English}{}{Jay Zameska, personal communication}
\ea \label{Basile24a}
\textit{Bogs and marshes \textbf{are found} throughout Estonia.}\\
\ex \label{Basile24b}
\textit{Wolves \textbf{are found} throughout the Northern hemisphere.} \\
\ex \label{Basile24c}
\textit{Ukraine \textbf{found itself} at war.} \\
\ex \label{Basile24d}
\textit{*Russia \textbf{finds itself} in Eurasia.} \\
\ex \label{Basile24e}
\textit{Jay \textbf{found himself} in the woods.} \\
\ex \label{Basile24f}
\textit{I \textbf{found myself} at the hospital.} \\
\z
\z

In English, an animate referent represented by the locatum is the criterion for grammaticality of the reflexive pronoun strategy in invenitive constructions (\ref{Basile24e}--\ref{Basile24f}), while the passive strategy can be used with both inanimate (\ref{Basile24a}) and some animate (\ref{Basile24b}) referents. When an inanimate referent does not metaphorically refer to a collectivity or a metonymic plurality of people or animate referents (\ref{Basile24c}), the reflexive strategy is perceived as ungrammatical (\ref{Basile24d}).
Construction specialization also applies to Maltese, which uses distinct grammatical strategies to differentiate between non-mirative and mirative readings. Compare the examples below (\ref{Basile25}).

\ea \label{Basile25}
\langinfo{Maltese}{}{Ray Fabri, personal communication}
\ea \label{Basile25a}
\gll Il-lukanda	\textbf{t-in-sab}	il-Belt. \\
\textsc{def}-hotel	3\textsc{f.sg.prs}-\textsc{drv7}-find \textsc{def}-City  \\
\trans ‘The hotel is situated/found in Valletta (the City).’\\
\ex \label{Basile25b}
\gll \textbf{N-in-sab}		Ruma. \\
1\textsc{sg.prs-drv7}-find	Rome \\
\trans ‘I am in Rome.’ – non-mirative reading \\
\ex \label{Basile25c}
\gll \textbf{Sib-t}		\textbf{ruħ-i}			f’=Malta. \\
find-1\textsc{sg.pst}	soul-\textsc{1sg.poss}	in=Malta \\
\trans ‘I found myself in Malta.’ – mirative reading \\
\ex \label{Basile25d}
\gll \textbf{Sib-t}		\textbf{ruħ-i}		f'=sitwazzjoni	diffiċli. \\
find-\textsc{1sg.pst}	soul-\textsc{1sg.poss}	in=situation	difficult \\
\trans ‘I found myself in a difficult situation.’ \\
\z
\z

In (\ref{Basile25a}--\ref{Basile25b}), the derivational form 7 (\textsc{drv}7) can coexpress several functions of the middle voice (in this case passive, Ray Fabri, personal communication). In (\ref{Basile25a}), the construction has a concrete location and an inanimate locatum, like in English, while in (\ref{Basile25b}), the locatum is animate. (\ref{Basile25b}) contrasts with the periphrastic reflexive strategy used in (\ref{Basile25c}), which encodes a mirative reading. The same strategy, with the same reading, can also be used with abstract locative phrases (\ref{Basile25d}).

Now, let us take a look at Basque.

\ea \label{Basile26}
\langinfo {Basque} {} {Urtzi Etxeberria, personal communication}\\
\ea \label{Basile26a}
\gll Egoera		arraro		batean		\textbf{aurki-tzen}	\textbf{naiz}. \\
situation	strange		one.\textsc{ine}	find-\textsc{prp} be.\textsc{1sg} \\
\trans ‘I find myself in a strange situation.’ \\
\ex \label{Basile26b}
\gll Webgunean	hainbat	informazio		osagarri	\textbf{aurki} \textbf{daiteke}. \\
website.\textsc{ine}	several		information.\textsc{abs}	additional	find.\textsc{inf} be.\textsc{prs.pot.3sg} \\
\trans ‘More information can be found on the website.’ \\
\z
\z

Basque presents two different morphosyntactic strategies that accurately align with the two types described above. One of them (\ref{Basile26a}) is a present participle strategy and allows for both allophoric and locuphoric forms of the verb \textit{izan} ‘to be’ but does not allow for inanimate locati (Type 1). The other one (\ref{Basile26b}) only allows the 3\textsc{sg} potential form of \textit{izan} ‘to be’ to be used with the infinitive short stem of the verb \textit{aurki} ‘to find’ (Irantzu Epelde, personal communication), and is used only with inanimate locati (Type 2).

This could be an innovation due to contact with Spanish (Irantzu Epelde, personal communication), which could confirm that (\ref{Basile26b}) is to be analyzed as an invenitive-locational construction rather than an instance of verbal predication. This argument can be deduced by observing that in Spanish invenitive-locational constructions are rather common, while according to native speakers of Basque, the most natural way of expressing a locational function in Basque is the locative copula \textit{egon} (Urtzi Etxeberria, personal communication). This fact questions the productivity of the Basque invenitive-locational construction, which should be seen as a rather marginal phenomenon that is developing thanks to language contact with its neighboring languages.
Even for a language like Basque, phylogenetically distant from the rest of the sample, the same criteria seem to apply. To distinguish between the two types of invenitive constructions, or within subtypes of the first type, specialized morphosyntactic strategies often play a role. They vary according to the animacy of the referent represented by the locatum, and to the morphological characteristics of the verb phrase.

\subsection{Marginal cases} \label{BasileSection5.3}

In Greek, middle-marked forms of the verb \textit{vrisko-} ‘to find’ are used to indicate locational meaning. However, this meaning is not available when the verb is marked for perfective aspect (\ref{Basile27}).

\ea \label{Basile27}
\langinfo{Greek}{}{Anna Kampanarou, personal communication\footnote{The Greek examples were already transliterated and provided in this form by the informant.}} \\
\gll I	Eladha		\textbf{vriskete} / *vrethike	stin Evropi. \\	
the	Greece.\textsc{nom}	find.\textsc{nact.pres.3sg} / find.\textsc{nact.past.perf.3sg} at.the Europe \\
\trans ‘Greece is / *was located in Europe.’ \\
\z

The non-active form, when marked for perfective aspect, can only be read as the literal non-active form of ‘find’, i.e., ‘be found by someone’ (\ref{Basile28}) or receive an inchoative (change-of-state) use (\ref{Basile29}; Anna Kampanarou, personal communication). The latter can be also interpreted as expressing mirative reading.

\ea \label{Basile28}
\langinfo{Greek}{}{Anna Kampanarou, personal communication} \\
\gll I	Maria		\textbf{vrethike}		apo	tin	astinomia. \\
the	Mary.\textsc{nom} 	find.\textsc{nact.perf.3sg}	by  	the	police \\
\trans ‘Mary was found by the police.’ \\
\z

\ea \label{Basile29}
\langinfo{Greek}{}{Anna Kampanarou, personal communication} \\
\gll \textbf{Vrethika}		sti	thesi		na	apologume. \\
find.\textsc{nact.past.perf.1sg}	in.the	position	to	apologize \\
\trans ‘I found myself in the position to apologize.’ \\
\z

Another marginal case is that of Albanian, where for expressing locational functions, in addition to invenitive-locational constructions, there is another strategy, based on a verb meaning \textsc{happen}. I propose that this type of construction be called \textit{evenitive-locational} (Lat. \textit{evenire} ‘to happen’). Just like in invenitive constructions, the verb \textit{ndodh-} ‘to happen’ undergoes semantic bleaching and behaves like a copula. However, it does not satisfy any other morphosyntactic or semantic criteria. Example (\ref{Basile30}) shows the Albanian invenitive-locational construction alongside the evenitive-locational construction.

\newpage
\ea \label{Basile30}
\langinfo{Albanian}{}{Nensi Islami, personal communication} \\
\ea \label{Basile30a}
\gll Italia	gje-nd-et / \textbf{ndodh-et}		në	Mesdhe. \\
Italy	find-\textsc{refl}-\textsc{3sg} / happen-\textsc{3sg}	in	Mediterranean.\textsc{acc} \\
\trans ‘Italy is situated in the Mediterranean.’ \\
\ex \label{Basile30b}
\gll \textbf{Ndodh-em} / gje-nd-em		në	një		situatë		të		keqe. \\
happen-\textsc{1sg} / find-\textsc{refl}-\textsc{1sg} in	\textsc{indef.acc}	situation.\textsc{acc} \textsc{adjart.acc}	bad.\textsc{acc} \\
\trans ‘I find myself in a bad situation.’ \\
\z
\z

This construction has not been found in any other language of the sample. Because it is so marginal and not morphologically marked, it could be a simple case of polysemy of the verb meaning \textsc{happen}. Further investigation is needed.

\section{Discussion and future research} \label{BasileSection6}

This paper has proposed a novel definition for a group of locational constructions I call invenitive-locational constructions. These constructions feature a verb with the meaning \textsc{find} which is called an invenitive verb and satisfies certain criteria. This verb has a root whose prototypical meaning is \textsc{find} and becomes semantically bleached. This verb is also morphologically marked (e.g., via a reflexive/middle marked strategy), and expresses a locational function. Semantic bleaching can be regarded as a central criterion because it makes it possible to identify invenitive constructions as having a purely locational meaning. This means that invenitive-locational constructions are instances of what is traditionally considered nonverbal predication.

Invenitive-locational constructions can be classified in two main types, which are distinct on the basis of the animacy of the referent represented by the locatum. The first type, which has an animate referent, can have language-specific subtypes that correlate not only with the animacy criterion, but also with the concreteness of the location expressed by the locative phrase and with whether the invenitive verb is indexed for a locuphoric or an allophoric form. It is still debatable whether some of these subtypes (as in Finnish or Maltese) can be distinguished on the sole basis of the animacy of the referent, or whether there is another variable that comes into play – mirativity. Mirativity could in fact be directly correlated with animacy, but Maltese, for example, uses a dedicated analytical strategy exclusively for mirative utterances. Analytical strategies like the one used by Maltese could also be interpreted as instances of verbal predication. In this case, there would be no semantic bleaching and, instead of a mirative reading, we would be dealing with a prototypical meaning \textsc{find}, which could entail per se a higher degree of unintentionality and unexpectedness of the action.

This study shows that invenitive constructions might be peculiar to European languages (see also \cite{Karakoc2025}, this volume), and that it is possible that they constitute an areal phenomenon. However, non-European languages such as Tagalog (Austronesian<Malayo-Polynesian) use forms of the verbs \textit{kita} ‘to see/find’ and \textit{hanap/tagpo} ‘to find’ to express a locational function (personal documentation). While this phenomenon could be the result of prolonged contact with European languages like Spanish and English, other non-European languages also present invenitive-locational constructions. One clear example is provided by Mandinka (Mande, \cite[145]{CreisselsSambou2013}). These recent findings call for further research on a broader spectrum of languages, that can be facilitated by the definition criteria presented above (\sectref{BasileSection1}). Such criteria make it possible to use invenitive constructions as a comparative concept. It is worth considering that there might be not only other language families in the world that use \textsc{find}-based strategies to express a locational function, but also construction-strategies based on other verbs that become semantically bleached and coexpress a locational function alongside their original meaning, in a similar way to posture verbs or to the Albanian evenitive-locational construction introduced above. I suggest that verbs like these be part of a class of construction-strategies called situative-locational constructions. Situative-locational constructions would ideally be a larger term that groups invenitive-locational constructions with other similar constructions, where semantic bleaching plays a role in conveying locational meaning.

One possible research question for the future is, of course, the grammaticalization of invenitive constructions, especially because semantic bleaching is a central component of both these constructions and grammaticalization itself (\cite[76]{HopperTraugott2003}). An example of what appears to be a fully grammaticalized invenitive strategy is given by the Swedish Existential construction \textit{det finns}. In this construction, the verb \textit{att finna} ‘to find’ presents a middle marking morpheme \textit{-s}, which is used productively in Swedish for several functions (e.g., passive, reflexive). The presence of the expletive \textit{det} is typical of existential constructions (see, e.g., English \textit{there is}). Perhaps there are traceable historical reasons behind why exactly this kind of verb has become popular to express locational meaning. This development could be directly linked to its lexical aspect, which points, in addition, toward the suitability of a cognitive approach.




\section*{Abbreviations}
\begin{tabularx}{.45\textwidth}{lQ}
\textsc{adjart} & adjectival article\\
\textsc{ade}  & adessive\\
\textsc{der} & derivational morpheme\\
\textsc{drv7} & derivational form 7\\
\textsc{ela} & elative\\
\textsc{encl} & enclitic\\
\textsc{expl} & expletive\\
\textsc{exv} & existive\\
\textsc{ine} & inessive \\
\end{tabularx}
\begin{tabularx}{.45\textwidth}{lQ}
\textsc{mir} & mirative\\
\textsc{mm} & middle markers\\
\textsc{nact} & nonactive\\
\textsc{perf} & perfective\\
\textsc{pot} & potential\\
\textsc{prep} & prepositional case\\
\textsc{prt} & preterite\\
\textsc{prp} & present participle\\
\textsc{ptv} & partitive\\
\textsc{tel} & telic\\
\end{tabularx}

\section*{Acknowledgements}
This work has been partially supported by the JSPS KAKENHI grant n. 24KF0235.
I am especially thankful to Gerson Klumpp and Mariann Bernhardt for the fruitful in-class discussions that, I think, ultimately led me to come up with this topic. Gerson was also the first person to ever encourage me to investigate invenitives more in detail. Martin Haspelmath deserves my most sincere thanks for providing valuable feedback at various stages of this work. He suggested that I use the term \textit{invenitive} (instead of my less transparent term \textit{situative}) after an online meeting we had in 2021. I also thank John Beavers for sharing his thoughts on the initial stages of this work, and for providing a different perspective over this topic. Many thanks to Chris Lasse Däbritz and Josefina Budzisch for co-organizing the workshop “Locative and existential predication – Core and periphery”, held in Athens at SLE2023, and to Jorge Agulló and Laurits Stapput Knudsen for their valuable reviews.



\section*{Appendix}

This appendix illustrates the data used in this article. \textit{Type 1} can be followed by a comment, which indicates that the data collected shows the usage of Type 1 only in certain conditions (e.g. “Type 1, abstract \textsc{loc}” means that the form appears with an animate referent located in an abstract location only).

\newpage
\subsection*{Albanian}

\begin{itemize}
    \item invenitive verb form: \textit{gjend-}
    \item source: consultation with native speaker (Nensi Islami)
\end{itemize}

\ea \label{BasileApp1}
\ea \label{BasileApp1a}
\gll Italia	\textbf{gje-nd-et} në	Mesdhe. \\
Italy	find-\textsc{refl}-\textsc{3sg}	in	Mediterranean.\textsc{acc} \\
\trans ‘Italy is situated in the Mediterranean.’ \\
\ex \label{BasileApp1b}
\gll \textbf{Gje-nd-em} në	një		situatë		të		keqe. \\
find-\textsc{refl}-\textsc{1sg} in	\textsc{indef.acc}	situation.\textsc{acc} \textsc{adjart.acc}	bad.\textsc{acc} \\
\trans ‘I find myself in a bad situation.’ \\
\z
\z

\subsection*{Basque}

\begin{itemize}
    \item invenitive verb form: \textit{aurkitzen naiz} (Type 1, abstract \textsc{loc}, mirative), \textit{aurki daiteke} (Type 2)
    \item source: Irantzu Epelde, Urtzi Etxeberria, personal communication
\end{itemize}


\ea \label{BasileApp2}
\ea \label{BasileApp2a}
\gll Egoera		arraro		batean		\textbf{aurki-tzen}	\textbf{naiz}. \\
situation	strange		one.\textsc{ine}	find-\textsc{prp} be.\textsc{1sg} \\
\trans ‘I find myself in a strange situation.’ \\
\ex \label{BasileApp2b}
\gll Webgunean	hainbat	informazio		osagarri	\textbf{aurki} \textbf{daiteke}. \\
website.\textsc{ine}	several		information.\textsc{abs}	additional	find.\textsc{inf} be.\textsc{prs.pot.3sg} \\
\trans ‘More information can be found on the website.’ \\
\z
\z

\subsection*{English}

\begin{itemize}
    \item invenitive verb form: \textit{to find oneself} (Type 1), \textit{to be found} (Type 2)
    \item source: consultation with native speaker (Jay Zameska)
\end{itemize}

\ea \label{BasileApp3}
\ea \label{BasileApp3a}
\textit{Bogs and marshes \textbf{are found} throughout Estonia.}\\
\ex \label{BasileApp3b}
\textit{Wolves \textbf{are found} throughout the Northern hemisphere.} \\
\ex \label{BasileApp3c}
\textit{Ukraine \textbf{found itself} at war.} \\
\ex \label{BasileApp3d}
\textit{Jay \textbf{found himself} in the woods.} \\
\ex \label{BasileApp3e}
\textit{I \textbf{found myself} at the hospital.} \\
\z
\z

\subsection*{Estonian}

\begin{itemize}
    \item invenitive verb form: \textit{ennast leidma} (mirative), \textit{leiduma}
    \item source: Liina Lindström, personal communication
\end{itemize}

\ea \label{BasileApp4}
\ea \label{BasileApp4a}
\gll \textbf{Leidsin}	\textbf{ennast} haiglast. \\
find.\textsc{prt.1sg}	self.\textsc{ptv} hospital.\textsc{ela} \\
\trans ‘I found myself at the hospital.’
\ex \label{BasileApp4b}
\gll Maailmas	\textbf{leid-u-b}	veel	häid		inimesi. \\
world.\textsc{ine}	find-\textsc{mm-3sg}	still	good.\textsc{pl.ptv}	person.\textsc{pl.ptv} \\
\trans ‘There are still good people in the world.’ \\
\z
\z

\subsection*{Finnish}

\begin{itemize}
    \item invenitive verb form: \textit{löytää itsensä} (mirative), \textit{löytyä}
    \item source: Ilmari Ivaska, personal communication
\end{itemize}

\ea \label{BasileApp5}
\ea \label{BasileApp5a}
\gll \textbf{Löysin}		\textbf{itseni}		keskeltä	metsää \\
find.\textsc{prt.1sg}	self.\textsc{acc.1sg} middle.\textsc{abl}	forest.\textsc{ptv} \\
\trans ‘I found myself in the middle of the forest.’ \\
\ex \label{BasileApp5b}
\gll Metsästä	\textbf{löyt-y-y}	erilaisia		ötököitä. \\
forest.\textsc{ela}	find-\textsc{mm-3sg}	different.\textsc{pl.ptv}	bug.\textsc{pl.ptv} \\
\trans ‘There are all sorts of bugs in the forest.’ \\
\z
\z

\subsection*{German}

\begin{itemize}
    \item invenitive verb form: \textit{sich befinden}
    \item source: Gerson Klumpp, personal communication
\end{itemize}

\ea \label{BasileApp6}
\gll Der	Bahnhof	\textbf{be-finde-t}	\textbf{sich}	zwischen	zwei	Städten. \\
\textsc{def}	station \textsc{der}-find-\textsc{3sg} \textsc{refl}	between	two	cities \\
\trans ‘The station is located between two cities.’
\z

\subsection*{Greek}

\begin{itemize}
    \item invenitive verb form: \textit{vriskome-}
    \item source: Anna Kampanarou, personal communication
\end{itemize}

\ea \label{BasileApp7}
\gll I	Eladha		\textbf{vriskete}	stin Evropi. \\	
the	Greece.\textsc{nom}	find.\textsc{nact.pres.3sg} at.the Europe \\
\trans ‘Greece is located in Europe.’ \\
\z

\subsection*{Hungarian}

\begin{itemize}
    \item invenitive verb form: \textit{található} (Type 2, concrete \textsc{loc})
    \item source: Bogáta Timár, Kata Kubínyi, personal communication
\end{itemize}

\ea \label{BasileApp8}
\gll Magyarország	Európában	\textbf{talál-hat-ó}. \\
Hungary	Europe.\textsc{ine}	find-\textsc{pot-prp} \\
\trans ‘Hungary is located in Europe.’ \\
\z

\subsection*{Italian}

\begin{itemize}
    \item invenitive verb form: \textit{trovarsi}
    \item source: personal knowledge, consultation with native speakers
\end{itemize}

\ea \label{BasileApp9}
\ea \label{BasileApp9a}
\gll Il gatto \textbf{si} \textbf{trova} sull’ albero. \\
\textsc{def}	cat   \textsc{mm.3sg}	find.\textsc{3sg}  	on.\textsc{def}  tree \\
\trans ‘The cat is in the tree.’ \\

\ex \label{BasileApp9b}
\gll \textbf{Ti}	\textbf{trovi} a	casa. \\
\textsc{mm.2sg}	find.\textsc{2sg}	at	home \\
\trans ‘You are at home.’ \\

\ex \label{BasileApp9c}
\gll \textbf{Ti}	\textbf{trovi}		in	una	situazione	spiacevole. \\
\textsc{mm.2sg}	find.\textsc{2sg}	in	\textsc{indf}	situation	unpleasant \\
\trans ‘You find yourself in an unpleasant situation.’ \\
\z
\z

\subsection*{Latvian}

\begin{itemize}
    \item invenitive verb form: \textit{atrasties}
    \item source: Daiki Horiguchi, personal communication, consultation with anonymous native speakers
\end{itemize}

\ea \label{BasileApp10}
\ea \label{BasileApp10a}
\gll Alus	\textbf{atrodas}	ledusskapī. \\
beer find.\textsc{3sg.refl} fridge.\textsc{loc}	\\
\trans ‘The beer is found in the fridge.’ \\

\ex \label{BasileApp10b}
\gll Jūs	\textbf{atrodaties}	centrā. \\
\textsc{2pl}	find.\textsc{2pl.refl}	center.\textsc{loc} \\
\trans ‘You are (located) in the center.’ \\
\z
\z

\subsection*{Maltese}

\begin{itemize}
    \item invenitive verb form: \textit{-sab}  (Type 1, non-mirative; Type 2), \textit{-sab ruħ-} (Type 1, mirative)
    \item source: Ray Fabri, Kirsty Azzopardi, personal communication
\end{itemize}

\ea \label{BasileApp11}
\ea \label{BasileApp11a}
\gll Il-lukanda	\textbf{t-in-sab}	il-Belt. \\
\textsc{def}-hotel	3\textsc{f.sg.prs}-\textsc{drv7}-find \textsc{def}-City  \\
\trans ‘The hotel is situated/found in Valletta (the City).’\\
\ex \label{BasileApp11b}
\gll \textbf{N-in-sab}		Ruma. \\
1\textsc{sg.prs-drv7}-find	Rome \\
\trans ‘I am in Rome.’ – non-mirative reading \\
\ex \label{BasileApp11c}
\gll \textbf{Sib-t}		\textbf{ruħ-i}			f’=Malta. \\
find-1\textsc{sg.pst}	soul-\textsc{1sg.poss}	in=Malta \\
\trans ‘I found myself in Malta.’ – mirative reading \\
\ex \label{BasileApp11d}
\gll \textbf{Sib-t}		\textbf{ruħ-i}		f'=sitwazzjoni	diffiċli. \\
find-\textsc{1sg.pst}	soul-\textsc{1sg.poss}	in=situation	difficult \\
\trans ‘I found myself in a difficult situation.’ \\
\z
\z

\subsection*{Russian}

\begin{itemize}
    \item invenitive verb form: \textit{nachodit’sya}
    \item source: Anna Branets, Denys Teptiuk, personal communication
\end{itemize}

\ea \label{BasileApp12}
\ea \label{BasileApp12a}
\gll Eda	\textbf{nachod-it-sya}	na	stole.\\
food	find-\textsc{3sg-mid}	on	table.\textsc{prep} \\
\trans ‘The food is on the table.’ \\
\ex \label{BasileApp12b}
\gll Na 	stole \textbf{nachod-it-sya}	eda.\\
on	table.\textsc{prep}	find-\textsc{3sg-mid}	food \\
\trans ‘There is food on the table.’ \\
\z
\z

\subsection*{Sardinian}

\begin{itemize}
    \item invenitive verb form: \textit{s'agattare}
    \item source: consultation with anonymous native speaker
\end{itemize}

\ea \label{BasileApp13}
\ea \label{BasileApp13a}
\gll \textbf{M’=agattu} in	Casteddu. \\
\textsc{mm.1sg}=find.\textsc{1sg}	in	Cagliari \\
\trans ‘I am in Cagliari.’ \\
\ex \label{BasileApp13b}
\gll \textbf{M’=agattu} in	una	situatzioni	malla. \\
\textsc{mm.1sg}=find.\textsc{1sg}	in	\textsc{indf} situation	bad \\
\trans ‘I find myself in a bad situation.’ \\
\ex \label{BasileApp13c}
\gll Su	pisci \textbf{si} \textbf{agattara}	in	s’=acqua. \\
\textsc{def} fish \textsc{mm.3sg} find.\textsc{3sg}	in	\textsc{def}=water \\
\trans ‘The fish is in the water.’ \\
\z
\z

\sloppy
\printbibliography[heading=subbibliography,notkeyword=this]
\end{document}
