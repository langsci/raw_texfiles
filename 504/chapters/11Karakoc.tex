\documentclass[output=paper,colorlinks,citecolor=brown]{langscibook}
\ChapterDOI{10.5281/zenodo.16838076}
\author{Birsel Karakoç\orcid{}\affiliation{Uppsala University}}
%\ORCIDs{}

\title[Existential and predicative possessive clauses in Turkic]{Distinguishing between existential and predicative possessive clauses in Turkic}

\abstract{This paper presents an analysis of the complex linguistic relationship between existential and possessive clauses in Turkic. Although there is a formal similarity through the use of the same linking device \{\textsc{bar}\} (\textit{var} in Turkish), existential and possessive clauses are clearly distinct. However, clauses with a subject that exhibits the form of a genitive-possessive construction (e.g. \textit{annem-in bardağ-ı} ‛my mom’s cup’) may allow existential and possessive readings, such as ‛There is a cup of my mom’s (somewhere)’ or ‛My mom has (a) cup(s) (somewhere)’. Thus, the aim of this paper is to discuss cases of possible ambiguity, proposing a number of syntactic and semantic criteria for distinguishing between these two readings. The presented criteria include the insertion of adverbs between the genitive and possessive nouns, the realization of information structure, specific or nonspecific references of the possessee noun, the replacement of \{\textsc{bar}\} by invenitive or postural verbs, and the derivation of relative clauses. The results show that the genitive-possessive constructions in existential and possessive clauses display a different syntactic behavior. In existential clauses, the genitive and the possessive noun display an attributive (adnominal) possessive relationship. In grammaticalized possessive clauses, in turn, the same morphological coding of the participants does not actually form a single NP unit. In other words, the genitive-possessive construction does not express attributive possession from a syntactic point of view. On the basis of these results, my basic claim is that in Turkic possessive clauses, the genitive-marked noun is grammaticalized as the clausal possessor, functionally corresponding to the subject in a ‛have’ clause in English. The possessive-marked noun expresses the clausal possessee, functionally equivalent to the object in a `have' clause in English.}

\IfFileExists{../localcommands.tex}{
   \addbibresource{../localbibliography.bib}
   % add all extra packages you need to load to this file

\usepackage{tabularx,multicol}
\usepackage{url}
\urlstyle{same}

\usepackage{listings}
\lstset{basicstyle=\ttfamily,tabsize=2,breaklines=true}

\usepackage{langsci-basic}
\usepackage{langsci-optional}
\usepackage{langsci-lgr}
\usepackage{langsci-osl}
% \usepackage{./langsci/styles/langsci-lgr}
% \usepackage{./langsci/styles/langsci-osl}
% \usepackage{langsci-gb4e}

\usepackage{tikz}
\usetikzlibrary{patterns,calc}
\pgfdeclarepatternformonly{south east lines}{\pgfqpoint{-0pt}{-0pt}}{\pgfqpoint{3pt}{3pt}}{\pgfqpoint{3pt}{3pt}}{
    \pgfsetlinewidth{0.6pt}
    \pgfpathmoveto{\pgfqpoint{0pt}{3pt}}
    \pgfpathlineto{\pgfqpoint{3pt}{0pt}}
    \pgfpathmoveto{\pgfqpoint{.2pt}{-.2pt}}
    \pgfpathlineto{\pgfqpoint{-.2pt}{.2pt}}
    \pgfpathmoveto{\pgfqpoint{3.2pt}{2.8pt}}
    \pgfpathlineto{\pgfqpoint{2.8pt}{3.2pt}}
    \pgfusepath{stroke}}
    
\usepackage{stmaryrd}
\usepackage{wasysym}
\usepackage{multirow}
\usepackage{caption}
\usepackage{subcaption}
\usepackage{mathrsfs}
\usepackage{qtree}

\usepackage{linguex}


   %pminos do not split footnotes
% \interfootnotelinepenalty=10000 %Footnote in Laporte chapters has to be split SN


%\DeclareIndexNameFormat{default}{%
%\nameparts{#1}%
%\usebibmacro{index:name}%
%{\index[names]}%
%{\namepartfamily}%
%{\namepartgiveni}%
% {}% L1
% {}% L2
%{\namepartprefix}% generates spurious space L3
%{\namepartsuffix}% generates spurious space L4
%}

%  {\DeclareIndexNameFormat{default}{%
%     \usebibmacro{index:name}{\index[names]}{#1}{#3}{#5}{#7}}}

%\DeclareIndexNameFormat{default}{%
%  \usebibmacro{index:name}{\sindex[nom]}{#1}{#3}{#5}{#7}}

%\DeclareIndexNameFormat{default}{%
%  \usebibmacro{index:name}{\sindex[person]}{#1}{#3}{#5}{#7}}
%\DeclareIndexNameFormat{default}{%
%\nameparts{#1} \usebibmacro{index:name}{\sindex[person]]}{\namepartfamily}{‌​\namepartgiven}{\nam‌​epartprefix}{\namepa‌​rtsuffix}}

%\newcommand{\smiley}{:)}

%\renewbibmacro*{index:name}[5]{%
%\usebibmacro{index:entry}{#1}%
%{\iffieldundef{usera}{}{\thefield{usera}\actualoperator}\mkbibindexname{#2}{#3}{#4}{#5}}}

% \newcommand{\noop}[1]{}

%remove for final
%\overfullrule=1mm

\newcommand{\tobi}[2]}}
\renewcommand{\S}[1]{\tobi{#1}{\textsc{*}}}

% this volume references
% puts: [this volume]
% already defined: \citetv
%\newcommand{\citepv}[1]{(\citeauthor{#1} \citeyear*{#1} [this volume])}
\newcommand{\citealtv}[1]{\citeauthor{#1} \citeyear*{#1} [this volume]}

%parentheses around example number
\newcommand{\pref}[1]{(\ref{#1})}

% in-text examples

\newcommand{\lnex}[1]{\textit{#1}} %target lang word
\newcommand{\lnlit}[1]{(lit.: `#1')} %literal reading
\newcommand{\lnlat}[1]{(#1)} % latinization
\newcommand{\lntrans}[1]{`#1'} %translation
\newcommand{\lnexl}[2]%
{\lnex{#1}{} \lnlat{#2}} % ex with latinization
\newcommand{\lnexlat}[3]{\lnex{#1}{} \lnlat{#2}{} \lntrans{#3}} % ex with latinization and tranl.

%ch01
\newcommand{\co}[1]{\mbox{\textbf{#1}}}

%ch09

\newcommand{\cyrbulg}[1]{\begin{otherlanguage*}{bulgarian}#1\end{otherlanguage*}}


%ch10
\newcommand{\nlp}{{\small NLP}}
\newcommand{\mwe}{{\small MWE}}
\newcommand{\rae}{{\small RAE}}
\newcommand{\lvc}{{\small LVC}}
\newcommand{\pos}{{\small P}o{\small S}}
%\newcommand{\todo}[1]{ \textcolor{red}{#1} }

%\renewcommand{\labelenumi}{\theenumi}
%\ainamefmt{{vv}{ll}{, ff}{, jj}} % fullname

\newcommand{\biberror}[1]{{\color{red}#1}}

\newcommand{\osenovaitem}{--~}
   %% hyphenation points for line breaks
%% Normally, automatic hyphenation in LaTeX is very good
%% If a word is mis-hyphenated, add it to this file
%%
%% add information to TeX file before \begin{document} with:
%% %% hyphenation points for line breaks
%% Normally, automatic hyphenation in LaTeX is very good
%% If a word is mis-hyphenated, add it to this file
%%
%% add information to TeX file before \begin{document} with:
%% %% hyphenation points for line breaks
%% Normally, automatic hyphenation in LaTeX is very good
%% If a word is mis-hyphenated, add it to this file
%%
%% add information to TeX file before \begin{document} with:
%% \include{localhyphenation}
\hyphenation{
    Beck-man
    Ngu-yen
    back-chan-nel
    back-chan-nels
    mo-not-o-nous
    ste-reo-typ-i-cal
}

\hyphenation{
    Beck-man
    Ngu-yen
    back-chan-nel
    back-chan-nels
    mo-not-o-nous
    ste-reo-typ-i-cal
}

\hyphenation{
    Beck-man
    Ngu-yen
    back-chan-nel
    back-chan-nels
    mo-not-o-nous
    ste-reo-typ-i-cal
}

   \boolfalse{bookcompile}
   \togglepaper[11]%%chapternumber
}{}

% \pretocmd{\gll}{\def\eachwordone{\itshape}\def\eachwordtwo{\normalfont}}{}{}

\begin{document}
\maketitle

\section{Introduction} \label{KarakocSec1}

The present paper investigates the morphosyntactic and semantic relationship between existential and predicative possessive clauses in Turkic, focusing for the most part on testing ambiguous examples in Turkish. Turkic languages usually lack a verb corresponding to `have' in English. Both concepts, existence and predicative possession, are typically expressed by clauses based on a nonverbal predicate of the type \{\textsc{bar}\} `existing', negated \{\textsc{yok}\} `non-existing'. Consider the Turkish (Turkic, [nucl1301]) examples\footnote{If not marked otherwise, the Turkish examples are constructed examples relying on the author's native speaker competence (see below).} in (\ref{Karakoc1}) (existential clause) and (\ref{Karakoc2}) (predicative possessive clause), in which the \{\textsc{bar}\} predicate is realized as \textit{var}. Following \citet{HaspelmathNonverbal}, I label this predicate with the term \textsc{existive} (abbreviated as \textsc{exv}).

\ea 
\label{Karakoc1}
\gll Dolap-ta bardak var.\\
     cupboard-\textsc{loc} cup \textsc{exv}\\
\glt `There is/are (a) cup(s) in the cupboard.' \\ 
Literally: `In the cupboard, cup existing.'\\
Existent: \textit{bardak} `cup'\\ Location: \textit{dolap} `cupboard'
\z

\ea 
\label{Karakoc2}
\gll Anne-m-in bardağ-ı var.\\
     mother-\textsc{poss1sg-gen} cup-\textsc{poss3sg} \textsc{exv}\\
\glt `My mom has (a) cup(s).' \\ 
Literally: `My mom’s, her cup, existing.'\\
Possessor (PR): \textit{annem} `my mom'\\ Possessee (PE): \textit{bardak} `cup'
\z 

In the existential clause in (\ref{Karakoc1}), the entity present in the cupboard is ‛(a) cup(s)’. The locative-marked noun \textit{dolapta} `in the cupboard' placed in the clause-initial position denotes location. In the predicative possessive clause in (\ref{Karakoc2}), the possessor \textit{annem} ‛my mom’ takes the genitive marker (\textit{annemin} ‛my mom’s’) while the possessee \textit{bardak} ‛cup’ is followed by a possessive suffix indexing person and number of the possessor (\textit{bardağı} ‛her cup’). In (\ref{Karakoc2}), location is not expressed. Despite the use of the same existive \textit{var}, the clauses in (\ref{Karakoc1}) and (\ref{Karakoc2}) are clearly distinct.

An existing entity can also appear in the form of a genitive-possessive noun phrase (NP), such as \textit{annemin bardağı} ‛my mom’s cup’ in example (\ref{Karakoc3}). Thus, depending on the context, the clause \textit{Dolapta annemin bardağı var} can be ambiguous between the readings ‛There is a cup of my mom’s in the cupboard’, as an answer to the question ‛What is in the cupboard?’ (\ref{Karakoc3a}), and ‛My mom has (a) cup(s) in the cupboard’, as an answer to ‛What does my mom have in the cupboard?’ (\ref{Karakoc3b}). In a genitive-possessive noun phrase, the attributive possessor takes the genitive case while the possessee usually carries a possessive suffix. 

In the existential reading of this clause, \textit{annemin bardağı} ‛my mom’s cup’ is a possessed noun phrase expressing attributive (or adnominal) possession. I call this type of syntactic relationship between the genitive possessor and the possessee noun \textsc{attributive PR-PE configuration}. The clause conveys the existence, presence, or location of the entity referred to in this noun phrase in some place (\ref{Karakoc3a}). The same morphological coding of the participants can also indicate predicative possession, i.e., clausal possession: ‛My mom has (a) cup(s) in the cupboard’ (\ref{Karakoc3b}). Based on a number of criteria, this paper claims that in grammaticalized predicative possessive clauses, the genitive and the possessive noun do not form a single noun phrase expressing attributive (or adnominal) possession. The genitive-marked noun phrase (\textit{annemin} ‛my mom’s’) functions as the clausal possessor on its own, for example functionally corresponding to the subject in a ‛have’ clause in English. The possessive-marked noun (\textit{bardağı} ‛her cup’) indicates the clausal possessee, functionally equivalent to the object in a ‛have’ clause in English. I call this type of relationship between the genitive possessor and the possessee noun \textsc{clausal PR-PE configuration}. Thus, in the possessive reading, the genitive-possessive structure \textit{annemin bardağı} ‛my mom’s cup’ provides an answer to the question \textit{Dolapta annemin nesi var?} ‛What does my mom have in the cupboard?’ with a possessed interrogative pronoun ‘(her) what’ (\textit{ne-si}) (\ref{Karakoc3b}).

\ea \label{Karakoc3}

\ea \label{Karakoc3a}
\gll Dolap-ta ne var? -- Dolap-ta \textup{[}anne-m-in bardağ-ı\textup{]} var. \\
cupboard-\textsc{loc} what \textsc{exv} \textsc{} cupboard-\textsc{loc} mother-\textsc{poss1sg-gen} cup-\textsc{poss3sg} \textsc{exv}\\
\glt ‛What is in the cupboard?' -- `There is a cup of my mom’s in the cupboard.'\\
Literally: ‛In the cupboard, my mom’s, her cup existing.’\\
Existent: \textit{annemin bardağı} ‛my mom’s cup’ (\textit{annem}: attributive possessor)\\ Location: \textit{dolap} ‛cupboard’


\ex \label{Karakoc3b}
\gll Dolap-ta anne-m-in ne-si var? -- Dolap-ta anne-m-in \textup{[}bardağ-ı\textup{]} var. \\
cupboard-\textsc{loc} mother-\textsc{poss1sg-gen} what-\textsc{poss3sg} \textsc{exv} \textsc{} cupboard-\textsc{loc} mother-\textsc{poss1sg-gen} cup-\textsc{poss3sg} \textsc{exv}\\
\glt ‛What does my mom have in the cupboard?’ -- ‛My mom has (a) cup(s) in the cupboard.’\\
Literally: ‛In the cupboard, my mom’s, her cup existing.’\\
Possessor: \textit{annem} ‛my mom’\\ Possessee: \textit{bardak} ‛cup’
\z
\z

Apart from a concise account in \citet{Karakoc2019}, this kind of structural ambiguity between existential and possessive clauses in Turkic, and possible criteria for determining their differences, have not been the topic of comprehensive research. The aim of the present paper is thus to focus on their differentiation by testing some syntactic and semantic criteria. Issues that will be taken into consideration include (1) the possibility to insert an adverb between the genitive and the possessive noun, (2) the realization of information structure, (3) the readings of a possessee noun, (4) question-answer patterns, (5) the possibility to replace the nonverbal existive \{\textsc{bar}\} by invenitive verbs which are based on roots meaning ‛find’ \citep{Basile2024InvenitiveLocational} or by postural verbs, and (6) strategies of clausal relativization and complementation. The paper argues that the possessor (PR) and the possessee (PE) nouns in a genitive-possessive construction exhibit a different behavior depending on whether the clause displays an existential or a predicative possessive reading. To be more precise, the basic claim is that a genitive-possessive construction in an existential clause indicates attributive (or adnominal) possession (X’s Y; \textit{annemin bardağı} ‛my mom’s cup’ as in (\ref{Karakoc3a})). In a grammaticalized possessive clause, on the other hand, the seemingly same surface structure does not involve attributive possession. Thus, the possessor is not an adjunct to the possessee, but an NP of its own (‛X’s // his/her Y’ as in (\ref{Karakoc3b})).

The present paper is not intended to be a data-driven study. As a theoretical contribution, it compares a small number of recurring examples, mostly from Turkish, taken from the literature or constructed by me, in order to check the proposed criteria for identifying subtle differences between the types of clauses in question. For the sake of methodological soundness, the examples were discussed with other native speakers, too. Formally, the paper consists of two parts. The first part (§\ref{KarakocSec2}) gives a brief overview of structural properties of nonverbal clauses in Turkic. The second part (§\ref{KarakocSec3}--\ref{KarakocSec4}) deals with the formal coincidence of existential and predicative possessive clauses and with the question of how to differentiate them. Finally, §\ref{KarakocSec5} concludes the paper.

\section{A brief overview of the typology of nonverbal clauses in Turkic} \label{KarakocSec2}

Nonverbal clauses, including copular clauses, locational clauses, existential and possessive clauses, and their complex relationships have been the topic of much research from both general-typological points of view, and regarding individual languages. Relevant accounts are \citet{Higgins1973}, \citet{Clark1978}, \citet{Hengeveld1992}, \citet{Heine1997}, \citet{Stassen1997,Stassen2009}, \citet{Creissels2013,Creissels2019}, \citet{Pustet2003}, \citet{Mikkelsen2005}, \citet{Heycock2012}, \citet{Koch2012}, \citet{Roy2013}, \citet{Croft2022}, \citet{CreisselsEtAlToAppear}, \citet{HaspelmathNonverbal}, among others. 
The majority of the studies on nonverbal clauses in Turkic deal with specific copular morphemes in individual languages. Extensive research examining the structure and semantic properties of these clauses in Turkic languages from comparative and historical points of view is rather rare. 

\subsection{Typical structures} \label{KarakocSec2-1}

The relationship between the subject and the predicate in a copular clause can be formulated as [X \textsc{be} Y]. This indicates a static use of a copular marker with the meaning of ‛be’. As is well-known, copular clauses can express a wide range of notions such as class membership, characterization, specification, identification or the like (see, e.g., \cite{Hengeveld1992}, \cite{Mikkelsen2005}). In the history of Turkic languages, various kinds of copular devices can be found, some of which have disappeared to different degrees in later languages. In particular, elaborate systems of formal markers are found in non-past copular clauses (see \cite{Karakoc2011,Karakoc2013,Karakoc2014}).

Structurally, locational clauses in which the nominal predicate contains a locative case marker are a subtype of copular clauses. This relation can be formulated as [X \textsc{be} somewhere]. In a locational clause, the \textsc{locatum}, i.e. the entity located, present or available in some place, is the subject of the clause. The locative NP indicating the location is the predicate. Example (\ref{Karakoc4}) illustrates a typical locational copular clause. The subject noun \textit{bardak} ‛cup’, placed in the initial position of the clause, denotes the locatum with a specific reference. From an information-structure point of view, the locatum provides old or given information and is therefore the topic [\textsc{top}] of the sentence. Giving new information about the topic, the location is the focus [\textsc{foc}] of the sentence. Since the subject in this example has a third-person singular reference in present tense, there is no overt copular marker. 

\ea 
\label{Karakoc4}
\gll Bardak dolap-ta.\\
     cup cupboard-\textsc{loc}\\
\glt `The cup is in the cupboard.' \\ 
Locatum (subject): \textit{bardak} `cup' [\textsc{top}]\\ Location (predicate noun): \textit{dolap} `cupboard' [\textsc{foc}]\\ Linking device: Copular marker (not realized here)
\z

An existential clause also informs about the location, presence/absence or existence/nonexistence of a subject referent, in the sense of [\textsc{there is} X somewhere]. Formally, an existential clause consists at least of three entities. The entity that is existent/nonexistent or present/absent somewhere is the subject of the clause. This entity does not occupy the clause-initial position, but follows after the location. The entity placed clause-finally is the existive marker. In example (\ref{Karakoc1}), repeated here as (\ref{Karakoc5}) for the sake of convenience, the existing entity, \textit{bardak}, is a subject with a nonspecific reference and can be interpreted as ‛a cup’ or ‛cups’ in a generic meaning. In comparison to the locational clause in example (\ref{Karakoc4}), the information structure in (\ref{Karakoc5}) is realized differently. The existent provides new information and is the focus of the sentence, whilst the clause-initial location expresses the topic, i.e., a given piece of information that the clause is intended to be about. The sentence literally means ‘In the cupboard, cup existing’.

\ea 
\label{Karakoc5}
\gll Dolap-ta bardak var.\\
     cupboard-\textsc{loc} cup \textsc{exv}\\
\glt `There is/are (a) cup(s) in the cupboard.' \\ 
Literally: `In the cupboard cup existing.'\\
Existent (subject): \textit{bardak} `cup' [\textsc{foc}]\\ Location: \textit{dolap} `cupboard' [\textsc{top}]\\ Linking device: Existive \textit{var}
\z

In cases, where the existent entity has a nonspecific reference, the clause is usually translated into English using the construction ‛There is/are …’, where the subject is an indefinite noun. This can be seen in the translation of (\ref{Karakoc5}): ‛There is/are (a) cup(s) in the cupboard’. An existent entity in the form of a genitive-possessive NP (attributive possession), however, inherently implies a specific reference. This kind of existential clauses can also be translated in English with a locational copular clause, such as \textit{Dolapta annemin bardağı var} ‘There is a cup of my mom’s in the cupboard’ (cf. example (\ref{Karakoc3a}) above) or ‛My mom’s cup is in the cupboard’.

A clause without any syntactic constituent referring to the location often indicates ‛absolute existence’ \citep[128]{Karakoc2019}, cf. example (\ref{Karakoc6}). Such clauses, in which the location is not revealed by the immediate context, are called \textsc{generic existentials} \citep{Koch2012} or \textsc{hyparctics} \citep{HaspelmathNonverbal}. 

\ea 
\label{Karakoc6}
\gll İklim değişikliğ-i var.\\
     climate change-\textsc{poss3sg} \textsc{exv}\\
\glt `There is climate change.' $\sim$ `Climate change exists.' \\ 
Existent (subject): \textit{iklim değişikliği} ‛climate change’\\ Location: --\\ Linking device: Existive \textit{var}
\z

A predicative possessive clause which denotes possession in the sense of [X \textsc{have} something] bears a formal similarity to existential clauses since the same linking device, the existive \textit{var}, is used. See example (\ref{Karakoc2}), repeated as (\ref{Karakoc7}) below. Formally, the clausal possessor (PR) takes the genitive suffix, while the possessed NP, the possessee (PE), gets a possessive suffix indexing person and number of the possessor. The possessee functions as the grammatical subject of the clause. From the point of view of information structure, the possessor NP is the topic, whereas the possessee NP gives new information, functioning as the focus. This type of possessive clause is subsumed, as a non-standard variant, under \textsc{locational possessives} in Stassen’s \citeyearpar[70--79]{Stassen2009} typology of predicative possession. \citet{Stassen2013} calls it a \textsc{genitive possessive} and states that ``the possessor NP is commonly (though not necessarily) constructed as an adnominal modifier to the possessed NP''. Similarly, \citet[58]{Heine1997} calls it \textsc{genitive schema}. The present paper, however, argues against the view that the clausal possessor is an adnominal modifier to the clausal possessee. 

\ea 
\label{Karakoc7}
\gll Anne-m-in bardağ-ı var.\\
     mother-\textsc{poss1sg-gen} cup-\textsc{poss3sg} \textsc{exv}\\
\glt `My mom has (a) cup(s).' \\ 
Literally: `My mom’s, her cup, existing.'\\
Possessor: \textit{annem} `my mom' [\textsc{top}]\\ Possessee: \textit{bardak} `cup' [\textsc{foc}]\\ Linking device: Existive \textit{var}
\z 

In locational possessives, according to \citet[50]{Stassen2009}, ``the possessor NP (PR) is constructed in some oblique, adverbial case form. As such, the PR may be marked by any formal device that the language employs to encode adverbial relations in general, such as case affixes or adpositions''. This pattern, exhibiting exactly the same form of an existential clause, such as the one given in (\ref{Karakoc5}), is also found in Turkic languages. Basically, existential constructions with a locative NP (location) referring to an animate entity allow possessive interpretations (see, e.g., \cite{Clark1978}). The possessee NP, which does not display a possessive suffix, functions as the subject of the clause. In comparison to the above-mentioned possessive clauses with a genitive-marked possessor, this locative pattern usually denotes temporary possession or availability with alienable possessees; see example (\ref{Karakoc8}) from Turkish. Further restrictions include that inalienable nouns, body parts, and kinship terms are unlikely to be found in this pattern \citep[132--134]{Karakoc2019}. However, in Turkic languages that have been in contact with Slavic, the locational possessive pattern frequently expresses clausal possession (cf. \cite{Nevskaja1997}). For instance, the endangered Turkic language Karaim [kara1464], which has undergone numerous contact-induced typological changes, uses both the typical possessive clause with a genitive-marked possessor, and the pattern with a locative adverbial to code the possessor, as equally frequent alternatives with the same meaning (\cite[16]{Csato2001}; \cite[134]{Karakoc2019}). 

\ea 
\label{Karakoc8}
\gll Anne-m-de bardak var.\\
     mother-\textsc{poss1sg-loc} cup \textsc{exv}\\
\glt `My mom has (a) cup(s) (right now).' \\ 
Literally: `At my mom’s, (a) cup(s) existing.'\\
Possessor: \textit{annem} `my mom' [\textsc{top}]\\ Possessee: \textit{bardak} `cup' [\textsc{foc}]\\ Linking device: Existive \textit{var}
\z 

Conceptually, the domains of location, existence, and possession are closely linked to one another (\cite{Clark1978}, \cite{Heine1997}, \cite{Stassen2009,Stassen2013}, \cite{Koch2012}, among others). The relations in nonverbal clauses in Turkic can be summarized as in \tabref{TabKarakoc1}.

\begin{table}
    \begin{tabularx}{.8\textwidth}{Xl}
        \lsptoprule
         {Clause type} &  {Linking device} \\
        \midrule
        Copular clause & \multirow{4}{2cm}{Copular\newline marker} \\
        \rightarrow{ }{[}X \textsc{be} Y{]} & \\
        Locational clause & \\
        \rightarrow{ }{[}X \textsc{be} somewhere{]} & \\
        \midrule
        Existential clause & \multirow{4}{2cm}{Existive\newline marker} \\
        \rightarrow{ }{[}\textsc{there is} X somewhere{]} & \\
        Possessive clause & \\
        \rightarrow{ }{[}X \textsc{have} something{]} & \\
        \lspbottomrule
    \end{tabularx}
    \caption{The relations and linking devices in nonverbal clauses in Turkic}
    \label{TabKarakoc1}
\end{table}

Alongside the coding strategies for possessive clauses sketched above, also \textsc{with}-possessives in Stassen’s \citeyearpar{Stassen2009} terms are represented in Turkic. Especially Siberian Turkic languages are rich in \textsc{with}-possessives, where the possessed entity is an adjective containing a proprietive suffix (\cite{Däbritz2018,Däbritz2022Dolgan,DäbritzTL2024}; \cite{Karakoc2019}; \cite{Johanson2021}). Possession can also be expressed with copular clauses, such as Turkish \textit{Bardak anne-m-in} ‛cup mom-\textsc{poss1sg-gen}' (literally: the cup, my mom’s) or \textit{Bardak anne-m-e ait} ‛cup mom-\textsc{poss1sg-dat} belonging’ (literally: the cup, to my mom, belonging) ‛The cup belongs to my mom’. In this paper, these constructions (\textsc{thematic possession} after \cite{Koch2012} and \textsc{appertentive clauses} after \cite{HaspelmathNonverbal}) are of no further relevance. 

The structures addressed so far concern the static types of nonverbal clauses. In my research, I systematically distinguish between static [\textsc{sta}] and dynamic types [\textsc{dyn}] of nonverbal constructions. This paper, however, focuses on the static existential and possessive clauses color-coded in gray, leaving aside the static copular and locational clauses, as well as all dynamic variants (\tabref{TabKarakoc2}).


\begin{table}
    \begin{tabularx}{\textwidth}{Xll}
    \lsptoprule
         {Clause type} & \textsc{[dyn]} & \textsc{[sta]} \\
        \midrule
        Copular & ‛become’ & ‛be’ \\
        Locational & ‛come about’, ‛come into being’ & ‛be found’, ‛be located’ \\
        \midrule
        Existential & ‛come about’, ‛come into being’ & \cellcolor{lightgray}`there is/are', `exist' \\
        Possessive & ‛acquire’, ‛take into possession’ & \cellcolor{lightgray}`have', `own', `possess' \\
        \lspbottomrule
    \end{tabularx}
    \caption{Dynamic and static meanings in nonverbal clauses in Turkic (cf. \cite{Karakoc2019})}
    \label{TabKarakoc2}
\end{table}

In Turkic languages, there are at least two basic types of existives in static existential and possessive clauses: the nonverbal existive \{\textsc{bar}\}, occurring as \textit{var} in Turkish and illustrated in examples (\ref{Karakoc1}--\ref{Karakoc3}) and (\ref{Karakoc5}--\ref{Karakoc8}), and the verbal existive \{\textsc{bol}\}. The \{\textsc{bar}\} type, which exclusively conveys static meanings, is a typical marker that is widespread in all Turkic languages. The verbal existive \{\textsc{bol}\} mainly expresses dynamic meanings, but is also used as a static existive to varying extents in different languages \citep{Karakoc2000,Karakoc2005,Karakoc2014,Karakoc2017,Karakoc2019}. The nonverbal existive \{\textsc{bar}\} takes copular markers with various functions, such as the copular marker denoting past tense, e.g. Turkish \textit{var-dı} (< \textit{var idi}), Bashkir [bash1264] \textit{bar ine}, Kazakh [kaza1248] \textit{bar edi}, Kirghiz [kirg1245] \textit{bar ele} ‘there was/were’, or an evidential marker, e.g., Turkish \textit{var-mış} (< \textit{var imiş}), Tatar [tata1255] \textit{bar iken}, Noghay [noga1249] \textit{bar eken} ‘there is/are evidently’. The verbal existive \{\textsc{bol}\} takes viewpoint aspect markers, e.g., Noghay \textit{bol-a-dï} `exist-\textsc{prs-3sg}' ‘there is/are’. In addition, the existive \{\textsc{bar}\} cannot express prospective existence or possession in the sense of ‘there will be’, ‘X will have’. In this meaning, the verbal existive \{\textsc{bol}\} is the default existive, for instance with the prospective marker \{-(y)AcAK\} in Turkish: \textit{ol-acak} ‘there will be’, ‘X will have’ \citep{Karakoc2019}. The present paper focuses on clauses based on the static existive \{\textsc{bar}\}, leaving aside the instances formed with the dynamic existive \{\textsc{bol}\} (marked in gray color in \tabref{TabKarakoc3}). The table provides the possible English translations of these markers.

\begin{table}
    \begin{tabularx}{\textwidth}{p{15mm}l@{~~}lQQ@{}}
    \lsptoprule
       \multicolumn{3}{l}{}     & \cellcolor{lightgray} \mbox{The nonverbal}   existive \{\textsc{bar}\}  & The verbal   existive  \{\textsc{bol}\}  \\
        \midrule
      Existential clause &  [\textsc{sta}] & ‛there is/are’, ‛exist’  & \cellcolor{lightgray} +   & +   \\
      \midrule
Possessive clause  & [\textsc{sta}] & ‛have’, ‛own’, ‛possess’ & \cellcolor{lightgray} +   & +  \\
        \lspbottomrule
    \end{tabularx}
    \caption{Static existives in Turkic and their English translations (cf. \cite{Karakoc2019})}
    \label{TabKarakoc3}
\end{table}

\subsection{Contact-induced deviations} \label{KarakocSec2-2}

Turkic varieties that developed under strong Persian influence have undergone several typological changes. Among other contact-induced deviations, these varieties exhibit different patterns of locational and existential clauses \citep{Karakoc2009,Karakoc2017,Karakoc2019}. Compare the contact-induced pattern in Irano-Turkic (\ref{Karakoc9}) with the typical Turkic pattern in Kazakh (\ref{Karakoc10}). See \citet{Däbritz2022Dolgan,DäbritzTL2024} for Dolgan constructions similar to example (\ref{Karakoc9}).

\ea 
\label{Karakoc9}
\langinfo{Khorasan Turkic}{[khor1269]}{\cite[182]{Bozkurt1975}}\\
\gll Ev-dä baːr-am.\\
     house-\textsc{loc} \textsc{exv-1sg}\\
\glt `I am at home.' \\ 
Locatum (subject): `I' (omitted here) [\textsc{top}]\\ Location (predicate noun): \textit{evde} [\textsc{foc}]\\ Linking device: Existive \textit{var}
\z 

\ea 
\label{Karakoc10}
\langinfo{Kazakh}{}{personal knowledge}\\
\gll Üy-de-mịn.\\
     house-\textsc{loc-1sg}\\
\glt `I am at home.' \\ 
Locatum (subject): `I' (omitted here) [\textsc{top}]\\ Location (predicate noun): \textit{üyde} [\textsc{foc}]\\ 
Linking device: \textsc{1sg} copular marker \{+MỊn\} 
\z 

Furthermore, in these contact varieties, the existive \{\textsc{bar}\} can be missing in existential clauses. Compare (\ref{Karakoc11}) from Southern Azeri with its equivalent in Turkish (\ref{Karakoc12}) \citep{Karakoc2009,Karakoc2019}. 

\ea 
\label{Karakoc11}
\langinfo{Southern Azeri}{[sout2697]}{\cite[66]{Kiral2001}}\\
\gll Orda bi(r) ġïz-di.\\
     here a girl-\textsc{cop3sg}\\
\glt `There is a girl here.' \\ 
Existent (subject): \textit{bir ġïz} [\textsc{foc}]\\ Location: \textit{orda} [\textsc{top}]\\ Linking device: \textsc{3sg} copular marker {\textit{-di}}
\z 

\ea 
\label{Karakoc12}
\gll Orada bir kız var.\\
     here a girl \textsc{exv}\\
\glt `There is a girl here.' \\ 
Existent (subject): \textit{bir kız} [\textsc{foc}]\\ Location: \textit{orada} [\textsc{top}]\\ Linking device: Existive \textit{var} 
\z 

Examples (\ref{Karakoc2}) and (\ref{Karakoc7}) above illustrate a typical possessive clause with the nonverbal existive \{\textsc{bar}\}. Once again, we observe deviations in Azeri varieties, in the sense that the possessive suffix that indicates the possessor is often not attached to the possessee NP but to the existive \textit{var}. This construction clearly goes back to a Persian pattern \citep{Karakoc2019}; see example (\ref{Karakoc13}). 

\ea 
\label{Karakoc13}
\langinfo{Southern Azeri}{}{\cite[288]{Karini2009}}\\
\gll Čörey var-ïmïz-ïdï.\\
     bread \textsc{exv-poss1pl-dist.cop}\\
\glt `We had bread.'
\z 

Interestingly, an opposite direction of copying is observed in colloquial Persian varieties, where the possessee carries the person marking as in Turkic, e.g. Tajik (Indo-European, [taji145]) \textit{Pul-äm häy} `money-\textsc{poss1sg} is' = ‘I have money’ \citep[969]{Johanson2021}.

\section{Structural ambiguity between existential and possessive clauses} \label{KarakocSec3}

Although existential and possessive clauses share a semantic affinity and exhibit morphological similarities, the two concepts are distinct. The idea of possession per se implies the notion of existence, but existence does not necessarily indicate possession. In other words, something possessed exists in some place. But something that exists or is found somewhere is not necessarily in someone’s possession. 

As already pointed out in connection with example (\ref{Karakoc3}), morphological identity can be observed in cases, where the existing entity has the form of a genitive-possessive construction. Consider the construction \textit{sopanın iki uǰu} in (\ref{Karakoc14}), where the clause may be ambiguous between two readings: ‛There are two ends of the stick’ or ‛A stick has two ends’. The question is whether the genitive-possessive construction, \textit{sopanın iki uǰu}, has to be analyzed as an attributive (or adnominal) noun phrase within an existential clause, ‛the two ends of the stick’, or as consisting of a possessor NP, ‛the stick’, and a possessee NP, ‛two ends’, involving clausal possession. In this example, the location of the existent entity is not overtly expressed, but is contextually identifiable.

\ea 
\label{Karakoc14}
\langinfo{Gagauz}{[gaga1249]}{\cite[132]{Karakoc2019}}\\
\gll Sopa-nïn iki uǰ-u var.\\
     stick-\textsc{gen} two end-\textsc{poss3sg} \textsc{exv}\\
\glt Existential reading: ‛There are two ends of the stick.’ \\
Possessive reading: ‛A stick has two ends.’ 
\z 

Unlike in typical Turkic languages, there is usually no formal identity between these clause types in Turkic varieties that developed in contact with Persian, as the contact-induced possessive clauses can be formed differently (see example \ref{Karakoc13}). Thus, in an existential clause, the possessive suffix is attached to the possessee, as in (\ref{Karakoc15}), while in a possessive clause the existive \textit{var} carries this suffix (\ref{Karakoc16}).

\ea 
\label{Karakoc15}
\langinfo{Southern Azeri}{}{personal information}\\
\gll Ġonšu-m-un ev-i var.\\
     neighbor-\textsc{poss1sg-gen} house-\textsc{poss3sg} \textsc{exv}\\
\glt ‛There is a house of my neighbor’, ‛There is a house belonging to my neighbor’ or ‛My neighbor’s house exists/is available (in some place).’ 
\z 

\ea 
\label{Karakoc16}
\langinfo{Southern Azeri}{}{personal information}\\
\gll Ġonšu-m-un ev var-ï-dï.\\
     neighbor-\textsc{poss1sg-gen} house \textsc{exv-poss3sg-cop3}\\
\glt ‛My neighbor has (a) house(s).’ 
\z 

The Turkish clause corresponding to (\ref{Karakoc15}), however, displays structural ambiguity, as illustrated in (\ref{Karakoc17}). The genitive noun, \textit{komşumun} ‛my neighbor’s’, can either be the genitive attribute in an existential clause or the possessor in a possessive clause. Thus, without further context, the clause \textit{Komşumun evi var} can be conceived of as ‛There is a house of my neighbor (in some place)’, ‛My neighbor’s house exists/is available (in some place)’ or ‛My neighbor has (a) house(s)’.

\ea 
\label{Karakoc17}
\gll Komşu-m-un ev-i var.\\
     neighbor-\textsc{poss1sg-gen} house-\textsc{poss3sg} \textsc{exv}\\
\glt Existential reading: ‛There is a house of my neighbor’, ‛My neighbor’s 							house exists/is available (in some place).’\\
Possessive reading: ‛My neighbor has (a) house(s).’
\z 

\section{Testing criteria for discrimination between existential and possessive readings} \label{KarakocSec4}

The question is how to understand whether something simply exists or is available somewhere, or whether someone possesses something, and whether there are any syntactic and semantic criteria to distinguish them from one another. In what follows, I propose six criteria to be tested. These include (1) the possibility to insert an adverb between the genitive and the possessive noun, (2) the realization of information structure, (3) available readings of the possessee noun, (4) question-answer patterns, (5) the possibility to replace the nonverbal existive \{\textsc{bar}\} by invenitive verbs which are based on roots meaning ‛find’ \citep{Basile2024InvenitiveLocational} or by postural verbs, and (6) strategies of clausal relativization and complementation.

\subsection{Intervention of an adverb between the genitive and the possessive nouns} \label{KarakocSec4-1}

Both kinds of sentences can contain a locative or temporal adverb. The adverbial element often occurs as the clause-initial topic. Example (\ref{Karakoc18}), containing the locative adverb \textit{Ankara’da} ‛in Ankara’ and example (\ref{Karakoc19}), with the adverb \textit{evde} ‛at home’, allow both existential and possessive readings.

\ea 
\label{Karakoc18}
\gll Ankara'-da komşu-m-un ev-i var.\\
     Ankara-\textsc{loc} neighbor-\textsc{poss1sg-gen} house-\textsc{poss3sg} \textsc{exv}\\
\glt Existential reading: ‛There is a house of my neighbor in Ankara.’\\
Possessive reading: ‛My neighbor has (a) house(s) in Ankara.’
\z 

\ea 
\label{Karakoc19}
\gll Ev-de (benim) kedi-m var.\\
     house-\textsc{loc} I.\textsc{gen} cat-\textsc{poss1sg} \textsc{exv}\\
\glt Existential reading: ‛There is my cat at home.’\\
Possessive reading: ‛I have (a) cat(s) at home.’
\z 

However, the intervention of a locative adverb such as \textit{Ankara’da} ‛in Ankara’ or a temporal adverb such as \textit{her zaman} ‛always’ or \textit{bu sene} ‛this year’ between the genitive and the possessive NP renders only a possessive meaning. Thus, the clause \textit{Komşumun Ankara'da evi var} must be interpreted as ‛My neighbor has (a) house(s) in Ankara’, for instance as an answer to the question ‛What does your neighbor have in Ankara?’ (\ref{Karakoc20}). Similarly, the clause \textit{Komşumun her zaman/bu sene evi var} ‛My neighbor always/this year has (a) house(s)’ can be the answer to the question ‛What does your neighbor always/this year have?’ (\ref{Karakoc21}). This phenomenon is also observed in Kazakh (\ref{Karakoc22}--\ref{Karakoc23}).

\ea 
\label{Karakoc20}
\gll Komşu-m-un Ankara'-da ev-i var.\\
     neighbor-\textsc{poss1sg-gen} Ankara-\textsc{loc} house-\textsc{poss3sg} \textsc{exv}\\
\glt ‛My neighbor has (a) house(s) in Ankara.’
\z 

\ea 
\label{Karakoc21}
\gll Komşu-m-un {her zaman} ev-i var.\\
     neighbor-\textsc{poss1sg-gen} always house-\textsc{poss3sg} \textsc{exv}\\
\glt ‛My neighbor always has (a) house(s).’
\z 

\ea 
\label{Karakoc22}
\langinfo{Kazakh}{}{personal information}\\
\gll Körši-m-niŋ Ankara-da üy-i bar.\\
     neighbor-\textsc{poss1sg-gen} Ankara-\textsc{loc} house-\textsc{poss3sg} \textsc{exv}\\
\glt ‛My neighbor has (a) house(s) in Ankara.’
\z 

\ea 
\label{Karakoc23}
\langinfo{Kazakh}{}{personal information}\\
\gll Körši-m-niŋ {är ḳašan} üy-i bar.\\
     neighbor-\textsc{poss1sg-gen} always house-\textsc{poss3sg} \textsc{exv}\\
\glt ‛My neighbor always has (a) house(s).’
\z 

In existential clauses, in contrast, an adverbial cannot intervene between the genitive and the possessive NP. The reason for this restriction is that the nouns build a genuine noun phrase as one syntactic unit, in which only the insertion of an adjective is allowed. The use of adjective forms \textit{Ankara’daki} ‛the one in Ankara’ or \textit{her zamanki} meaning ‛regular’, ‛ordinary’, ‛usual’ in Turkish (\ref{Karakoc24}--\ref{Karakoc25}), and \textit{Anḳara’daɣï} ‛the one in Ankara’ or \textit{är ḳašanɣï} ‛regular, ordinary, usual’ in Kazakh (\ref{Karakoc26}--\ref{Karakoc27}) naturally renders existential meanings, though a specific possessive reading may also be possible.

\ea 
\label{Karakoc24}
\gll Komşu-m-un Ankara'-da-ki ev-i var.\\
     neighbor-\textsc{poss1sg-gen} Ankara-\textsc{loc-rel} house-\textsc{poss3sg} \textsc{exv}\\
\glt Existential reading: ‛My neighbor’s house in Ankara exists/is available.’\\
Possessive reading: ‛My neighbor has his/her house in Ankara.’

\z 

\ea 
\label{Karakoc25}
\gll Komşu-m-un {her zamanki} ev-i var.\\
     neighbor-\textsc{poss1sg-gen} regular house-\textsc{poss3sg} \textsc{exv}\\
\glt Existential reading: ‛My neighbor’s regular house exists/is available (somewhere).’\\
Possessive reading: ‛My neighbor has his/her regular house (somewhere).’
\z 

\ea 
\label{Karakoc26}
\langinfo{Kazakh}{}{personal information}\\
\gll Körši-m-niŋ Anḳara’da-ɣï üy-i bar.\\
     neighbor-\textsc{poss1sg-gen} Ankara-\textsc{loc-rel} house-\textsc{poss3sg} \textsc{exv}\\
\glt Existential reading: ‛My neighbor’s house in Ankara exists/is available.’\\
Possessive reading: ‛My neighbor has his/her house in Ankara.’
\z 

\ea 
\label{Karakoc27}
\langinfo{Kazakh}{}{personal information}\\
\gll Körši-m-niŋ {är ḳašanɣï} üy-i bar.\\
     neighbor-\textsc{poss1sg-gen} regular house-\textsc{poss3sg} \textsc{exv}\\
\glt Existential reading: ‛My neighbor’s regular house exists/is available (somewhere).’\\
Possessive reading: ‛My neighbor has his/her regular house (somewhere).’
\z 

In general, adjectives may intervene between the genitive and the possessive NP in both clause types, yielding ambiguities (\ref{Karakoc28}).

\ea 
\label{Karakoc28}
\gll Komşu-m-un büyük ev-i var.\\
     neighbor-\textsc{poss1sg-gen} big house-\textsc{poss3sg} \textsc{exv}\\
\glt Existential reading: ‛There is a big house of my neighbor (in some place)’, ‛My 				neighbor’s big house exists/is available (somewhere).’\\
Possessive reading: ‛My neighbor has (a) big house(s).’
\z 

\subsection{Realization of information structure} \label{KarakocSec4-2}

As already pointed out, the information structure is realized differently in existential and possessive clauses. An existential clause (\ref{Karakoc29a}) consists of two syntactic constituents: \textit{komşumun evi} ‛my neighbor’s house’, which is the subject of the clause, and the existive \textit{var}. The attributive possession within this clause, i.e. the genitive-possessive NP ‛my neighbor’s house’ as a whole, denotes the focus. A possessive clause (\ref{Karakoc29b}) that exhibits the same surface structure, on the other hand, should be segmented differently, because it consists of three syntactic entities: \textit{komşum} ‛my neighbor’, the clausal possessor (PR) that denotes the topic; \textit{ev} ‛house’, the clausal possessee (PE) that denotes the focus; and the existive \textit{var}. 

\ea \label{Karakoc29}

\ea \label{Karakoc29a}
\gll \textup{[}Komşu-m-un ev-i\textup{]} var.\\
     neighbor-\textsc{poss1sg-gen} house-\textsc{poss3sg} \textsc{exv}\\
\glt ‛There is a house belonging to my neighbor (in some place).’\\
Location: missing, but identifiable from the context [\textsc{top}]\\ Existent: \textit{komşumun evi} ‛my neighbor’s house’ [\textsc{foc}]\\ Linking device: Existive \textit{var}

\ex \label{Karakoc29b}
\gll Komşu-m-un \textup{[}ev-i\textup{]} var.\\
     neighbor-\textsc{poss1sg-gen} house-\textsc{poss3sg} \textsc{exv}\\
\glt ‛My neighbor has (a) big house(s).’\\
Clausal possessor (PR): \textit{komşum} ‛my neighbor’ [\textsc{top}]\\ Clausal possessee (PE): \textit{evi} ‛his/her house’ [\textsc{foc}]\\ Linking device: Existive \textit{var}

\z 
\z

\subsection{References of a possessee noun} \label{KarakocSec4-3}

The deceptiveness of the formal identity of these clause types can be further illustrated when looking at the semantic features of the possessed entities. Consider the ambiguous clause in (\ref{Karakoc18}), repeated as (\ref{Karakoc30}). In the existential reading, the possessee noun \textit{ev} ‛house’ appears in attributive PR-PE configuration, where it has a specific singular reference, because the genitive possessor makes the NP specific. To mark a specific plural possessee in existentials, the plural suffix \{+lAr\} is attached to the possessee. In clausal possession, on the other hand, the possessee can be interpreted as either a nonspecific singular or a nonspecific plural noun. In this configuration the genitive possessor does not make the possessed NP specific. The plural suffix can be used to clearly indicate plurality.  

\ea 
\label{Karakoc30}
\gll Ankara'-da komşu-m-un ev-i var.\\
     Ankara-\textsc{loc} neighbor-\textsc{poss1sg-gen} house-\textsc{poss3sg} \textsc{exv}\\
\glt Existential reading: ‛There is a house of my neighbor in Ankara.’ (\textit{ev} ‛a specific house’ [\textsc{sg}], \textit{ev-ler} ‛specific houses’ [\textsc{pl}])\\
Possessive reading: ‛My neighbor has (a) house(s) in Ankara.’ (\textit{ev} ‛nonspecific house’ [\textsc{sg}] or ‛nonspecific houses’ [\textsc{pl}])
\z 

\subsection{Question-answer patterns} \label{KarakocSec4-4}

As has become obvious from the explanations so far, the intended meaning of an ambiguous structure can be determined by asking questions. The clause \textit{Ankara’da komşumun evi var} (repeated below as example (\ref{Karakoc31})) can be understood as an answer to four different questions: 

\begin{enumerate}
    \item \textit{Ankara’da ne var?} ‛\uline{What} is in Ankara?’ The interrogative pronoun \textit{ne} ‛what’ requires the whole genitive-possessive NP in the answer: \textit{komşumun evi} ‛my neighbor’s house’.
    \item \textit{Ankara’da kim-in ne-si var?} ‛\uline{Who} has got \uline{what} in Ankara?’ The interrogative words \textit{kimin} ‛whose’ and \textit{nesi} ‛his/her/its what’ require two entities (possessor and possessee) to be given in the answer.
    \item \textit{Ankara’da kim-in ev-i var?} ‛\uline{Who} has got (a) house(s) in Ankara?’ or ‛\uline{Whose} house exists in Ankara?’ The interrogative word \textit{kimin} ‛whose’ only needs the entity referred to by the possessor noun to be specified in the answer, that is, \textit{komşumun} ‛my neighbor’s’.
    \item \textit{Ankara’da komşu-m-un ne-si var?} ‛\uline{What} does my neighbor have in Ankara?’ The interrogative word \textit{nesi} ‛his/her/its what’ only needs the entity referred to by the possessee noun to be specified in the answer, that is \textit{evi} ‛his/her/its house’.
\end{enumerate}

If the sentence \textit{Ankara’da komşumun evi var} is understood as an answer to the first question \textit{Ankara’da ne var?} ‛What is in Ankara?’, then only an existential interpretation is possible. If, however, the sentence is conceived of as an answer to the questions (2–4), both existential and possessive interpretations are in principle possible, although a possessive interpretation would be more natural. Similarly to the previous tests, this test also shows that only in an existential clause do the possessor (\textit{komşumun}) and the possessee (\textit{evi}) nouns represent a single syntactic unit (attributive possession) providing an answer to the question \textit{ne} ‛what’.

\ea 
\label{Karakoc31}
\gll Ankara'-da komşu-m-un ev-i var.\\
     Ankara-\textsc{loc} neighbor-\textsc{poss1sg-gen} house-\textsc{poss3sg} \textsc{exv}\\
\glt Existential reading: ‛There is a house belonging to my neighbor in Ankara’, ‛My neighbor’s house exists/is available in Ankara.’\\
Possessive reading: ‛My neighbor has (a) house(s) in Ankara.’
\z

\begin{enumerate}
    \item \textit{Ankara’da \uline{ne} var?} ‛\uline{What} is/exists in Ankara?’ \rightarrow{ } Only an existential reading is possible.
    \item \textit{Ankara’da \uline{kim}-in \uline{ne}-si var?} ‛\uline{Who} has got \uline{what} in Ankara?’ \rightarrow{ } Both existential and possessive readings are possible.
    \item \textit{Ankara’da \uline{kim}-in ev-i var?} ‛\uline{Who} has got (a) house(s) in Ankara?’ $\sim$ `\uline{Whose} house exists in Ankara?' \rightarrow{ }  Both existential and possessive readings are possible.
    \item \textit{Ankara’da komşu-m-un \uline{ne}-si var?} ‛\uline{What} does my neighbor have in Ankara?’ \rightarrow{ }  Both existential and possessive readings are possible.
\end{enumerate}

\subsection{The possibility to replace the nonverbal existive \{\textsc{bar}\} with invenitive or postural verbs} \label{KarakocSec4-5}

Another criterion concerns the replacement of the nonverbal existive \{\textsc{bar}\} with an invenitive verb which is based on a root meaning ‛find’ \citep[this volume]{Basile2024InvenitiveLocational, Basile2025}, or with a postural verb (cf. \cite{BasileEtAl2023}). The invenitive verb \textit{bulun-} ‛be found’, derived from the verb \textit{bul-} ‛find’, negated \textit{bulun-ma-} ‛not be found’, ‛be absent’ in Turkish, can replace the nonverbal static existive \textit{var} in existential clauses (\ref{Karakoc32}--\ref{Karakoc34}). Similarly, in Kazakh, the postural verb \textit{žat-} ‛lie’ can be used if the sentence has an existential meaning (\ref{Karakoc35}). 

\ea 
\label{Karakoc32}
\gll Ankara'-da komşu-m-un ev-i bulun-uyor.\\
     Ankara-\textsc{loc} neighbor-\textsc{poss1sg-gen} house-\textsc{poss3sg} be.found-\textsc{prs3sg}\\
\glt ‛There is a house belonging to my neighbor in Ankara’, ‛My neighbor’s house is available in Ankara.’\\
Q: \uline{What} is found/available in Ankara?
\z

\ea 
\label{Karakoc33}
\gll Dolap-ta anne-m-in bardağ-ı bulun-uyor.\\
     cupboard-\textsc{loc} mother-\textsc{poss1sg-gen} cup-\textsc{poss3sg} be.found-\textsc{prs3sg}\\
\glt ‛There is a cup of my mom’s in the cupboard.’ \\
Q: \uline{What} is in the cupboard?
\z

\ea 
\label{Karakoc34}
\gll Ev-de (benim) kedi-m bulun-uyor.\\
     house-\textsc{loc} I.\textsc{gen} cat-\textsc{poss1sg} be.found-\textsc{prs3sg}\\
\glt ‛There is my cat at home.’ \\
Q: \uline{What} is at home?
\z

\ea 
\label{Karakoc35}
\langinfo{Kazakh}{}{personal information}\\
\gll Ankara-da körši-m-niŋ üy-i žat-ïr.\\
     Ankara-\textsc{loc} neighbor-\textsc{poss1sg-gen} house-\textsc{poss3sg} lie-\textsc{aor3sg}\\
\glt ‛There is a house belonging to my neighbor in Ankara’, ‛My neighbor’s house exists/is 				available in Ankara.’\\
Q: \uline{What} is found/available in Ankara?
\z

The invenitive verb \textit{bulun-} is only used if the location in the existential clause is overtly expressed or identifiable in the given context. It cannot be used in clauses expressing absolute existence (like example (\ref{Karakoc6}) above), also called \textsc{generic existentials} \citep{Koch2012} or \textsc{hyparctics} \citep{HaspelmathNonverbal}. I have the impression that \textit{bulun-} implies the more permanent presence of the existing entity. But, this intuitive impression, as well as other oppositions between the existive \textit{var} and the invenitive verb \textit{bulun-}, is a subject for further research. 

\subsection{Subordination strategies} \label{KarakocSec4-6}

The final test I propose in this paper concerns the formation of subordinate clauses. If the clause \textit{Komşumun evi var} (example (\ref{Karakoc17}) above) has an existential meaning, it only allows the relativization of the genitive-possessive NP as a whole. In other words, only one relative clause is possible to derive, with \textit{ev} ‛house’ as its head noun: ‛the house belonging to my neighbor...’ (\ref{Karakoc36}). The genitive noun cannot be relativized separately. It must be noted here that in Turkish nonfinite clauses, the existive \textit{var} is regularly replaced by the verbal existive \textit{ol}- ‛exist’ (the \{\textsc{bol}\} type, see §\ref{KarakocSec2-1}). For the use of \{\textsc{bar}\} existives in embedded clauses in Kipchak Turkic languages, see \citet{Karakoc2017}.

\ea \label{Karakoc36}

\ea \label{Karakoc36a}
\gll komşu-m-un ol-an ev \textup{[...]}.\\
     neighbor-\textsc{poss1sg-gen} be-\textsc{ptcp} house\\
\glt ‛The house belonging to my neighbor [...].’\\

\ex \label{Karakoc36b}
\gll komşu-m-a {ait ol-an} ev \textup{[...]}.\\
     neighbor-\textsc{poss1sg-gen} belong-\textsc{ptcp} house\\
\glt ‛The house belonging to my neighbor [...].’\\

\z 
\z

If \textit{Komşumun evi var} denotes a possessive clause, however, it is possible to derive two different relative clauses, one with the head noun \textit{komşum} ‛my neighbor’ (\ref{Karakoc37}), and one with the head noun \textit{ev} ‛the house’ (\ref{Karakoc38}). In other words, the two referents, the genitive NP (PR) and the possessive NP (PE), can be relativized separately: ‛my neighbor who has (a) house(s) [...]’, and ‛the house belonging to my neighbor [...]’.

\ea \label{Karakoc37}

\ea \label{Karakoc37a}
\gll ev-i ol-an komşu-m \textup{[...]}.\\
     house-\textsc{poss3sg} be-\textsc{ptcp} neighbor-\textsc{poss1sg}\\
\glt ‛My neighbor who has (a) house(s) [...].’\\

\ex \label{Karakoc37b}
\gll ev-in sahib-i ol-an komşu-m \textup{[...]}.\\
     house-\textsc{gen} owner-\textsc{poss3sg} be-\textsc{ptcp} neighbor-\textsc{poss1sg}\\
\glt ‛My neighbor who owns the house [...].’\\

\z 
\z

\ea \label{Karakoc38}

\ea \label{Karakoc38a}
\gll komşu-m-un ol-an ev \textup{[...]}.\\
     neighbor-\textsc{poss1sg-gen} be-\textsc{ptcp} house\\
\glt ‛The house that my neighbor owns [...].’\\

\ex \label{Karakoc38b}
\gll komşu-m-a {ait ol-an} ev \textup{[...]}.\\
     neighbor-\textsc{poss1sg-gen} belong-\textsc{ptcp} house\\
\glt ‛The house belonging to my neighbor [...].’\\

\z 
\z

Once again, this phenomenon in relative clauses shows that in an existential clause, the genitive-possessive NP has a single reference as a whole (attributive PR-PE configuration). In a possessive clause, on the other hand, the two NPs have been grammaticalized as separate syntactic units in the role of clausal possessor and clausal possessee, respectively (clausal PR-PE configuration). 

Finally, I shall briefly mention one phenomenon in Turkish complement clauses that requires more careful investigation. The subject in nonfinite complement clauses usually takes a genitive marker. If the subject is a genitive-possessive NP, such as \textit{komşumun evi} (\ref{Karakoc39}) or \textit{Hasan’ın annesi} (\ref{Karakoc40}), the possessee (\textit{evi} in (\ref{Karakoc39}) and \textit{annesi} in (\ref{Karakoc40})) takes the genitive marker, indicating an existential or possessive reading. If the possessee does not take a genitive marker, the complement clause usually should be interpreted as a possessive clause (\ref{Karakoc41}--\ref{Karakoc42}).

\largerpage
\ea 
\label{Karakoc39}
\gll Komşu-m-un ev-i-nin ol-duğ-un-u bil-iyor-um.\\
     neighbor-\textsc{poss1sg-gen} house-\textsc{poss3sg-gen} be-\textsc{actn-poss3sg-acc} know-\textsc{prs-1sg}\\
\glt Existential reading: ‛I know that there is a house belonging to my neighbor (in some place).’\\
Possessive reading: ‛I know that my neighbor has (a) house(s).'
\z

\ea 
\label{Karakoc40}
\gll Hasan’-ın anne-si-nin ol-ma-dığ-ın-ı duy-du-m.\\
     Hasan-\textsc{gen} mother-\textsc{poss3sg-gen} be-\textsc{neg-actn-poss3sg-acc} hear-\textsc{pst-1sg}\\
\glt Existential reading: ‛I heard that Hasan’s mom is not (there).’\\
Possessive reading: ‛I heard that Hasan does not have a mom.'
\z

\ea 
\label{Karakoc41}
\gll Komşu-m-un ev-i ol-duğ-un-u bil-iyor-um.\\
     neighbor-\textsc{poss1sg} house-\textsc{poss3sg-gen} be-\textsc{actn-poss3sg-acc} know-\textsc{prs-1sg}\\
\glt ‛I know that my neighbor has (a) house(s).'
\z

\ea 
\label{Karakoc42}
\gll Hasan’-ın anne-si ol-ma-dığ-ın-ı duy-du-m.\\
     Hasan-\textsc{gen} mother-\textsc{poss3sg} be-\textsc{neg-actn-poss3sg-acc} hear-\textsc{pst-1sg}\\
\glt ‛I heard that Hasan does not have a mom.'\footnote{Since the verb \textit{ol-} can express various meanings (cf. \cite{Karakoc2014,Karakoc2017,Karakoc2019}), this sentence can also be read as ‛I heard that she is not Hasan’s mom’. But this reading is not relevant here.}
\z

\section{Summary and outlook} \label{KarakocSec5}

This paper has presented an analysis of the morphosyntactic and semantic relationships between existential and possessive clauses in Turkic. Although they exhibit a formal similarity through the use of the same linking device, the existive \{\textsc{bar}\}, these two types of clauses are clearly distinct, as shown in the introductory examples (\ref{Karakoc1}) and (\ref{Karakoc2}) in §\ref{KarakocSec1}. However, clauses with a subject that has the form of a genitive-possessive construction (genitive PR-possessed PE), as shown in example (\ref{Karakoc3}), may allow existential and possessive interpretations. The aim of the paper was therefore to discuss cases of possible ambiguity, and to propose syntactic and semantic criteria for distinguishing between the two readings. 

The results of the analysis suggest that the genitive-possessive constructions in existential and possessive clauses display different characteristics. In an existential clause, the genitive and the possessive nouns involve attributive (adnominal) possession. I call this relation \textsc{attributive PR-PE configuration}. In this configuration, intervention of adverbs between PR-PE is not possible. The entire PR-PE appears as the focus [\textsc{foc}] of the given existential clause. Because of the genitive PR, a singular PE has a specific reference. A specific plural reference is achieved by using the plural suffix \{+lAr\}. As a whole, the genitive-possessive NP can provide an answer to the question \textit{ne} ‛what’, which exhibits a single reference. Furthermore, the nonverbal existive \{\textsc{bar}\} can be replaced by an invenitive (Turkish) or a postural (Kazakh) verb. In subordination, only the PE can be relativized. The PR constituent of an existential clause cannot appear separately as the head noun of a relative clause. 

In a grammaticalized possessive clause, the same morphological coding of the participants does not form a single NP. In other words, the genitive and the possessive NP do not express attributive possession. I call this type \textsc{clausal PR-PE configuration}. This configuration is characterized by the possibility to insert adverbs between PR-PE. From an information-structure point of view, the PR appears as the topic [\textsc{top}], whereas the PE gives new information and is the focus [\textsc{foc}] of the sentence. A singular PE is ambiguous between nonspecific singular and nonspecific plural reference. In this configuration, the genitive-possessive NP can provide answers to different questions, formed with the interrogatives \textit{kimin} ‛whose’ and/or \textit{nesi} ‛his/her/its what’. But it does not answer a question formed with the interrogative \textit{ne} ‛what’, which has a single reference. The existive \{\textsc{bar}\} is usually not replaced by invenitive or postural verbs in this configuration. Finally, the PR and PE in a grammaticalized possessive clause can be relativized separately. \tabref{TabKarakoc4} gives a summary of these issues.


\begin{table}
    \begin{tabularx}{\textwidth}{lQQQ}
    \lsptoprule
         & \textsc{Criterion} & \textsc{Existential} & \textsc{Possessive} \\
         \midrule
        (1) & constituents between the genitive and possessive NP & only adjectives & adjectives and \newline adverbs\\

        (2) & realization of information structure & entire genitive-\newline possessive NP is [\textsc{foc}] & genitive NP is [\textsc{top}], possessive NP is [\textsc{foc}]\\

        (3) & reference of a singular possessee noun & specific singular & nonspecific singular \newline or plural \\

        (4) & question-answer patterns & genitive-possessive NP can provide the answer to the  question \textit{ne} ‛what’ & genitive-possessive NP does not provide the answer to the question \textit{ne} ‛what’ \\

        (5) & replacement of the existive \{\textsc{bar}\} & invenitive \textit{bulun-} in Turkish or postural \textit{žat-} in Kazakh & not possible \\

        (6) & relative clauses & only possessive NP can be relativized & genitive NP and possessive NP can be separately relativized\\
        \lspbottomrule
    \end{tabularx}
    \caption{Summary of the presented criteria}
    \label{TabKarakoc4}
\end{table}

The grammaticalization of genitive NPs in genitive-possessive constructions as the subjects of nonfinite subordinate clauses is a widespread phenomenon in Turkic languages, cf. example (\ref{Karakoc43}).

\ea 
\label{Karakoc43}
\gll Ahmet'-in dün Ankara'-dan buraya gel-diğ-in-i duy-du-m.\\
     Ahmet-\textsc{gen} yesterday Ankara-\textsc{abl} hither come-\textsc{actn-poss3sg-acc} hear-\textsc{pst-1sg}\\
\glt ‛I heard that Ahmet came here from Ankara yesterday.’
\z

The genitive noun \textit{Ahmet’in} in the genitive-possessive structure \textit{Ahmet’in dün Ankara’dan buraya geldiği} functions as the subject of the complement clause. The genitive-possessive NP does not syntactically form an attributive construction. On the basis of the results presented in this paper, I argue that the genitive-marked noun is along similar lines grammaticalized as the clausal possessor, functionally corresponding to the subject in a ‛have’ clause in English. The possessive-marked noun expresses the clausal possessee, functionally equivalent to the object in a ‛have’ clause in English. Thus, I claim that the genitive-possessive structure of a possessive clause is rather to be conceived of as ‛X’s // his/her Y’. 

The proposed criteria need to be checked on a larger set of data, preferably from different branches of Turkic, and by drawing on instances with different semantic types of genitive and possessive nouns, such as animate, inanimate, body parts, etc. My general impression is that in spoken language, there is a short pause after the possessor NP in possessive clauses. But for practical reasons, it was not possible to investigate the role of intonation in this paper. This important aspect of distinguishing between the two clause types in spoken language should be a topic of further research.
 
\section*{Abbreviations}
\begin{tabularx}{.45\textwidth}{lQ}
\textsc{actn} & action nominal\\
\textsc{aor} & aorist\\
\end{tabularx}
\begin{tabularx}{.45\textwidth}{lQ}
\textsc{exv} & existive\\
\\
\end{tabularx}

\section*{Acknowledgements}
I wish to thank Martin Haspelmath and Chris Lasse Däbritz for their invaluable comments and suggestions.

%\section*{Contributions}
%John Doe contributed to conceptualization, methodology, and validation.
%Jane Doe contributed to writing of the original draft, review, and editing.

\sloppy
\printbibliography[heading=subbibliography,notkeyword=this]
\end{document}
