%% -*- coding:utf-8 -*-

\documentclass{scrarticle}

\usepackage{langsci-forest-setup}



\usepackage{german}

% does not work for the macros, strange.
% \usepackage[ngerman]{babel}
% \useshorthands{"}


\usepackage{./styles/abbrev,./styles/merkmalstruktur,./styles/my-xspace,./styles/makros.2e,./styles/article-ex,langsci-gb4e,oneline,langsci-lgr}


\newcommand{\danish}{\jambox{({Danish})}}
\newcommand{\dutch}{\jambox{({Dutch})}}
\newcommand{\english}{\jambox{({English})}}
\newcommand{\german}{\jambox{({German})}}
\newcommand{\yiddish}{\jambox{({Yiddish})}}
\newcommand{\icelandic}{\jambox{({Icelandic})}}

\usepackage{unified-biblatex}
\let\citew\citealp
\newcommand{\page}{}
\bibliography{bib-abbr,biblio}

\begin{document}



% \begin{forest}
% sm edges
% [1 %{V[\spr \eliste, \comps \eliste]}
%    [2 %{NP[\type{nom}]} 
%       [Sandy] ]
%    [3 %V\feattab{
%       %\spr \sliste{ NP[\type{nom}] }, \comps \sliste{}}
%      [4 % V\feattab{
%         % \spr \sliste{ NP[\type{nom}] },\\
%         % \comps \sliste{ PP[\type{to}] }} 
%         [6 %V\feattab{
%            %\spr \sliste{ NP[\type{nom}] },\\
%            %\comps \sliste{ NP[\type{acc}], PP[\type{to}] }} 
%            [donates] ]
%         [7 %{NP[\type{acc}]} 
%            [one Million,roof] ] ]
%      [5 %{PP[\type{to}]} 
%         [to the children,roof ] ] ] ]
% \end{forest}

\section{Valenz, Adjunktion und Scrambling}

\begin{enumerate}

%\if0

\item Which of the following structures is suitable for the sentence in (\mex{1}).

Welche der folgenden Strukturen ist für den Satz in (\mex{1}) sinnvoll?
\ea
Sandy donates one Million to the children.
\z


\begin{forest}
sm edges
[1 
   [2 
     [4 
       [6 [Sandy] ]
       [7 [donates]]]
     [5 [one Million,roof] ] ]
   [3 
     [to the children,roof ] ] ]
\end{forest}

\begin{forest}
sm edges
[1 
   [2 
      [Sandy] ]
   [3 
     [4 
        [6 
           [donates] ]
        [7 
           [one Million,roof] ] ]
     [5 
        [to the children,roof ] ] ] ]
\end{forest}


\begin{forest}
sm edges
[1 
  [2 
    [4 [Sandy] ]
    [5 [donates]]]
  [3
    [6 [one Million,roof] ] 
    [7 [to the children,roof ] ] ] ]
\end{forest}


\begin{forest}
sm edges
[1 
    [2 [Sandy] ]
    [3 [donates]]
    [4 [one Million,roof] ] 
    [5 [to the children,roof ] ] ]
\end{forest}

\begin{forest}
sm edges
[1 
  [2
    [4 [Sandy] ]
    [5 [6 [donates]]
       [7 [one Million,roof] ] ] ]
  [3 [to the children,roof ] ] ]
\end{forest}

Antwort: 2


\item Assign categories to the numbered nodes.

Ordnen Sie den Knotenzahlen Kategorien zu.

Dabei können alle Kategorien, die im Lösungsbaum vorkommen, angeklickt werden. Aber auch Kategorien
mit NP[dat] und NP[gen] und absurde Kategorien mit zwei Sachen im SPR oder acc im SPR und nom in
COMPS, zwei acc in COMPS usw. Auch die Lösung der folgenden Aufgabe mit Adjunktion sollte hier mit
aufgezählt werden, d.h. die Valenzlisten für ein strict transitives Verb und \modm für die PP.



\begin{forest}
sm edges
[{V[\spr \eliste, \comps \eliste]}
   [{NP[\type{nom}]} [Sandy] ]
   [V\feattab{
      \spr \sliste{ NP[\type{nom}] }, \comps \sliste{}}
     [V\feattab{
         \spr \sliste{ NP[\type{nom}] },\\
         \comps \sliste{ PP[\type{to}] }} 
        [V\feattab{
           \spr \sliste{ NP[\type{nom}] },\\
           \comps \sliste{ NP[\type{acc}], PP[\type{to}] }} [donates] ]
        [{NP[\type{acc}]} [one Million,roof] ] ]
     [{PP[\type{to}]} [to the children,roof ] ] ] ]
\end{forest}


\item Which of the following structures is suitable for the sentence in (\mex{1}).

Welche der folgenden Strukturen ist für den Satz in (\mex{1}) sinnvoll?
\ea
Sandy read the book in the garden.
\z


\begin{forest}
sm edges
[1 
   [2 
     [4 
       [6 [Sandy] ]
       [7 [read]]]
     [5 [the book,roof] ] ]
   [3 
     [in the garden,roof ] ] ]
\end{forest}

\begin{forest}
sm edges
[1 
   [2 
      [Sandy] ]
   [3 
     [4 
        [6 
           [read] ]
        [7 
           [the book,roof] ] ]
     [5 
        [in the garden,roof ] ] ] ]
\end{forest}


\begin{forest}
sm edges
[1 
  [2 
    [4 [Sandy] ]
    [5 [read]]]
  [3
    [6 [the book,roof] ] 
    [7 [in the garden,roof ] ] ] ]
\end{forest}


\begin{forest}
sm edges
[1 
    [2 [Sandy] ]
    [3 [read]]
    [4 [the book,roof] ] 
    [5 [in the garden,roof ] ] ]
\end{forest}

\begin{forest}
sm edges
[1 
  [2
    [4 [Sandy] ]
    [5 [6 [read]]
       [7 [the book,roof] ] ] ]
  [3 [in the garden,roof ] ] ]
\end{forest}

\begin{forest}
sm edges
[1
  [2 [Sandy] ]
  [3 
    [4 [read]]
    [5 
      [6 [the]]
      [7 [8 [book] ]
         [9 [in the garden,roof ] ] ] ]]]
\end{forest}


Antwort: 2 und 6

\item Ordnen Sie den Knotenzahlen Kategorien zu.

Dabei können alle Kategorien, die im Lösungsbaum vorkommen, angeklickt werden. Aber auch Kategorien
mit NP[dat] und NP[gen] und absurde Kategorien mit zwei Sachen im SPR oder acc im SPR und nom in
COMPS, zwei acc in COMPS usw. Auch die Lösung der vorigen Aufgabe sollte hier mit
aufgezählt werden, d.h. die Valenzlisten für ein Verb mit Akkusativ- und Präpositionalobjekt.


\begin{forest}
sm edges
[{V[\spr \eliste, \comps \eliste]}
          [{NP[\type{nom}]} [Sandy] ]
          [V\feattab{
              \spr \sliste{ NP[\type{nom}] }, \comps \sliste{  } }
            [V\feattab{
                \spr \sliste{ NP[\type{nom}] }, \comps \sliste{  } }
              [V\feattab{
                \spr \sliste{ NP[\type{nom}] },\\
                \comps \sliste{ NP[\type{acc}] }} [reads] ]
              [{NP[\type{acc}]} [the book,roof] ] 
               ]
            [{PP[\textsc{mod} VP]} [in the garden,roof] ]
] ]
\end{forest}

%\fi
\item Which of the following structures is suitable for the sentence in (\mex{1}), provided phrases
  may be abbreviated?

Welche der folgenden Strukturen ist für den Satz in (\mex{1}) sinnvoll, wenn Phrasen abgekürzt
  werden können?
\ea
\gll dass die Vorsitzende des Vereins Conny der Urkundenfälschung bezichtigt\\
     that the chairwoman  of.the society Conny the forgery accuses\\
\glt `that the chairwoman of the society accuses Conny of forgery'
\z


\begin{forest}
sm edges
[x
  [x [dass;that]]
  [x
    [x [die Vorsitzende;the chairwoman,roof]]
    [x
      [x [des Vereins;of.the society,roof]]
      [x
        [x [Conny;Conny]]
        [x
          [x [der Urkundenfälschung;the forgery,roof]]
          [x [bezichtigt;accuses]]]]]]]
\end{forest}

\begin{forest}
sm edges
[x
  [x [dass;that]]
  [x
    [x [die Vorsitzende;the chairwoman,roof]]
    [x
      [x [des Vereins;of.the society,roof]]
      [x%, calign child=2
        [x [Conny;Conny]]
        [x [der Urkundenfälschung;the forgery,roof, calign with current]]
        [x [bezichtigt;accuses]]]]]]
\end{forest}

\begin{forest}
sm edges
[x
  [x [dass;that]]
  [x
    [x [die Vorsitzende des Vereins;the chairwoman of.the society,roof]]
    [x
      [x [Conny;Conny]]
      [x
        [x [der Urkundenfälschung;the forgery,roof]]
        [x [bezichtigt;accuses]]]]]]
\end{forest}

\begin{forest}
sm edges
[x
  [x [dass;that]]
  [x
    [x [die Vorsitzende;the chairwoman,roof]]
    [x
      [x [des Vereins;of.the society,roof]]
      [x [Conny;Conny]]
      [x [der Urkundenfälschung;the forgery,roof, calign with current]]
      [x [bezichtigt;accuses]]]]]
\end{forest}

\begin{forest}
sm edges
[x
  [x [dass;that]]
  [x 
    [x
      [x
        [x [die Vorsitzende des Vereins;the chairwoman of.the society,roof]]
        [x [Conny;Conny]]]
      [x [der Urkundenfälschung;the forgery,roof]]]
    [x [bezichtigt;accuses]]]]
\end{forest}

Antwort: 3

\clearpage

\item Ordnen Sie den Knotenzahlen Kategorien zu. Gehen Sie dabei davon aus, dass eine unmarkierte
  Stellung vorliegt. %NPen können abgekürzt werden. 

\begin{forest}
sm edges
[1
  [2 [dass;that]]
  [3
    [4 [die Vorsitzende des Vereins;the chairwoman of.the society,roof]]
    [5
      [6 [Conny;Conny]]
      [7
        [8 [der Urkundenfälschung;the forgery,roof]]
        [9 [bezichtigt;accuses]]]]]]
\end{forest}


Dabei können alle Kategorien, die im Lösungsbaum vorkommen, angeklickt werden. Aber auch Kategorien
mit NP[dat] und NP[gen]. Auch die Lösung der vorigen Aufgabe sollte hier mit
aufgezählt werden, d.h. die Valenzlisten für ein Verb mit Akkusativ- und Präpositionalobjekt.

Hier können \comps \sliste{ \npnom, \npgen, \npacc, \npgen{}  } mit aufgeführt werden.

\oneline{%
\begin{forest}
sm edges
[C\feattab{ \spr \sliste{},\\
            \comps \sliste{} }
  [C\feattab{ \spr \sliste{},\\
              \comps \sliste{ S } } [dass;that]]
  [V\feattab{ \spr \sliste{},\\
              \comps \sliste{  } }
    [{NP[\type{nom}]} [die Vorsitzende des Vereins;the chairwoman of.the society,roof]]
    [V\feattab{ \spr \sliste{},\\
                \comps \sliste{ \npnom{} } }
      [{NP[\type{acc}]} [Conny;Conny]]
      [V\feattab{ \spr \sliste{},\\
                  \comps \sliste{ \npnom, \npacc{}  } }
        [{NP[\type{gen}]} [der Urkundenfälschung;the forgery,roof]]
        [V\feattab{ \spr \sliste{},\\
                    \comps \sliste{ \npnom, \npacc, \npgen{}  } } [bezichtigt;accuses]]]]]]
\end{forest}}


\item Welche der folgenden Strukturen ist für den Satz in (\mex{1}) sinnvoll?
\ea
\gll dass dem Kind jemand schnell hilft\\
     that the child somebody quickly helps\\
\glt `that somebody helps the child quickly'
\z

\begin{forest}
sm edges
[x
  [x  [dass;that]]
  [x
    [x [dem Kind;the child,roof]]
    [x
      [x [jemand;somebody]]
      [x
        [x [schnell;quickly]]
        [x [hilft;helps]]]]]]
\end{forest}


\begin{forest}
sm edges
[x,before drawing tree={x-=1em}
  [x [dass;that]]
  [x,name=s,before drawing tree={x-=1.4em}
    [x,name=vp,before drawing tree={x+=3.1em}
      [x
        [x [dem Kind;the child,roof]]
        [jemand;somebody, no edge,name=jemand]
        [schnell;quickly, no edge,name=schnell]
        [x [hilft;helps]]]]]]
\draw (vp.south) -- (schnell.north);
\draw (s.south) -- (jemand.north);
\end{forest}



\begin{forest}
sm edges
[x
  [x
    [x
      [x
        [x  [dass;that]]
        [x [dem Kind;the child,roof]]]
      [x [jemand;somebody]]]
    [x [schnell;quickly]]]
  [x [hilft;helps]]]
\end{forest}


\begin{forest}
sm edges
[x
  [x  [dass;that]]
    [x
      [x
        [x [dem Kind;the child,roof]]
        [x,calign with current [jemand;somebody]]
        [x [schnell;quickly]]]
    [x [hilft;helps]]]]
\end{forest}



\item Ordnen Sie den Knotenzahlen Kategorien zu. NPen können abgekürzt werden. 

Bei den Knotenbeschriftungen auch Adj[\textsc{mod} VP] mit dazu nehmen.

\begin{forest}
sm edges
[1
  [2 [dass;that]]
  [3
    [4 [dem Kind;the child,roof]]
    [5
      [6 [jemand;somebody]]
      [7
        [8 [schnell;quickly]]
        [9 [hilft;helps]]]]]]
\end{forest}

%\oneline{%
\begin{forest}
sm edges
[C\feattab{ \spr \sliste{},\\
            \comps \sliste{} }
  [C\feattab{ \spr \sliste{},\\
              \comps \sliste{ S } } [dass;that]]
  [V\feattab{ \spr \sliste{},\\
              \comps \sliste{  } }
    [\npdat [dem Kind;teh child,roof]]
    [V\feattab{ \spr \sliste{},\\
                \comps \sliste{ \npdat{} } }
      [\npnom [jemand;somebody]]
      [V\feattab{ \spr \sliste{},\\
                  \comps \sliste{ \npnom, \npdat{}  } }
        [{Adj[\textsc{mod} V[\textsc{ini}$-$]]} [schnell;quickly]]
        [V\feattab{ \spr \sliste{},\\
                    \comps \sliste{ \npnom, \npdat{}  } } [hilft;helps]]]]]]
\end{forest}
%}

\end{enumerate}

\section{Verbalkomplex}

\begin{enumerate}


\item
Welche der folgenden Strukturen ist für den Satz in (\mex{1}) sinnvoll?
\ea
\gll dass sie das Lied singen dürfen wird\\
     that she the song sing   be.allowed.to will\\
\glt `that she will be allowed to sing the song'
\z


\begin{forest}
sm edges
[x
  [x [dass;that]]
  [x
     [x [sie;she]]
     [x
       [x
         [x
           [x [das Lied;the song,roof]]
           [x [singen;sing]]]
         [x [dürfen;be.allowed.to]]]
       [x [wird;will]]]]]
\end{forest}


\begin{forest}
sm edges
[x
  [x
    [x
      [x
        [x
          [x [dass;that]]
          [x [sie;she]]]
        [x [das Lied;the song,roof]]]
      [x [singen;sing]]]
    [x [dürfen;be.allowed.to]]]
  [x [wird;will]]]
\end{forest}

\begin{forest}
sm edges
[x
      [x
        [x
          [x [dass;that]]
          [x [sie;she]]]
        [x [das Lied;the song,roof]]]
      [x [x [singen;sing]]
         [x [dürfen;be.allowed.to]]
         [x [wird;will]]]]]
\end{forest}


\begin{forest}
sm edges
[x
  [x [dass;that]]
  [x
     [x [sie;she]]
     [x
       [x [das Lied;the song,roof]]
       [x
         [x
           [x [singen;sing]]
           [x [dürfen;be.allowed.to]]]
         [x [wird;will]]]]]]
\end{forest}

\begin{forest}
sm edges
[x
   [x [dass;that]]
   [x
     [x [sie;she]]
     [x
       [x [das Lied;the song,roof]]
       [x [x [singen;sing]]
          [x [dürfen;be.allowed.to]]
          [x [wird;will]]]]]]
\end{forest}

Antwort: 4

\item Ordnen Sie den Knotenzahlen Kategorien zu. NPen können abgekürzt werden. 

Hier wieder Kategorien wie bei \spr machen, also mit mehreren NPen in \subj und leerer \subjl.


\oneline{%
\begin{forest}
sm edges
[1
  [2 [dass;that]]
  [3
     [4 [sie;she]]
     [5
       [6 [das Lied;the song,roof]]
       [7
         [8
           [10 [singen;sing]]
           [11 [dürfen;be.allowed.to]]]
         [9 [wird;will]]]]]]
\end{forest}}


\oneline{%
\begin{forest}
sm edges
[C\feattab{%\spr \sliste{},\\
             \comps \sliste{ }}
  [C\feattab{%\spr \sliste{},\\
             \comps \sliste{ S }} [dass;that]]
  [V\feattab{%\spr \sliste{ },\\
             \comps \sliste{ }}
     [\npnom [sie;she]]
     [V\feattab{%\spr  \sliste{ },\\
                \comps \sliste{ \npnom{} }},s sep+=1em
       [\npacc [das Lied;the song,roof]]
       [V\feattab{%\spr \sliste{ },\\
                  \comps \sliste{ \npnom, \npacc{} }}
           [V\feattab{\subj \sliste{ \npnom{} },\\
                   %            \spr \sliste{ },\\
                               \comps \sliste{ \npacc{} }}
             [V\feattab{\subj \sliste{ \npnom{} },\\
                    %             \spr \sliste{ },\\
                                 \comps \sliste{ \npacc{} }} [singen;sing]]
             [V\feattab{\subj \sliste{ \npnom{} },\\
                     %   \spr \sliste{ },\\
                        \comps \sliste{ \npacc, V[\lex+] }} [dürfen;be.allowed.to]]]
         [V\feattab{%\spr \sliste{ },\\
                    \comps \sliste{ \npnom, \npacc, V[\lex+] }} [wird;will]]]]]]
\end{forest}}


\item
Welche der folgenden Strukturen ist für den Satz in (\mex{1}) sinnvoll?
\ea
\gll dass sie das Lied wird singen dürfen\\
     that she the song will sing   be.allowed.to\\
\glt `that she will be allowed to sing the song'
\z


\begin{forest}
sm edges
[x
  [x [dass;that]]
  [x
     [x [sie;she]]
     [x
       [x
         [x
           [x [das Lied;the song,roof]]
           [x [wird;will]]]
         [x [singen;sing]]]
       [x [dürfen;be.allowed.to]]]]]
\end{forest}


\begin{forest}
sm edges
[x
  [x
    [x
      [x
        [x
          [x [dass;that]]
          [x [sie;she]]]
        [x [das Lied;the song,roof]]]
      [x [wird;will]]]
    [x [singen;sing]]]
  [x [dürfen;be.allowed.to]]]
\end{forest}

\begin{forest}
sm edges
[x
  [x [dass;that]]
  [x
     [x [sie;she]]
     [x
       [x [das Lied;the song,roof]]
       [x
         [x [wird;will]]
           [x
             [x [singen;sing]]
             [x [dürfen;be.allowed.to]]]]]]]
\end{forest}

\begin{forest}
sm edges
[x
      [x
        [x
          [x [dass;that]]
          [x [sie;she]]]
        [x [das Lied;the song,roof]]]
      [x [x [wird;will]]
         [x [singen;sing]]
         [x [dürfen;be.allowed.to]]]]]
\end{forest}

\begin{forest}
sm edges
[x,before drawing tree={x-=.7em}
  [x [dass;that]]
  [x,before drawing tree={x-=1.7em}
     [x [sie;she]]
     [x,name=wird-vp,before drawing tree={x-=3.9em}
       [x
         [x,before drawing tree={x-=1.2em}
           [x [das Lied;the song,roof]]
           [wird;will,no edge,name=wird]
           [x [singen;sing]]]
         [x [dürfen;be.allowed.to]]]]]]
\draw (wird-vp.south) -- (wird.north);
\end{forest}

Antwort: 3

\item Ordnen Sie den Knotenzahlen Kategorien zu. NPen können abgekürzt werden. 

Hier wieder Kategorien wie bei \spr machen, also mit mehreren NPen in \subj und leerer \subjl.


\oneline{%
\begin{forest}
sm edges
[1
  [2 [dass;that]]
  [3
     [4 [sie;she]]
     [5
       [6 [das Lied;the song,roof]]
       [7
         [8 [wird;will]]
           [9
             [10 [singen;sing]]
             [11 [dürfen;be.allowed.to]]]]]]]
\end{forest}}


\oneline{%
\begin{forest}
sm edges
[C\feattab{%\spr \sliste{},\\
             \comps \sliste{ }}
  [C\feattab{%\spr \sliste{},\\
             \comps \sliste{ S }} [dass;that]]
  [V\feattab{%\spr \sliste{ },\\
                       \comps \sliste{ }}
     [\npnom [sie;she]]
     [V\feattab{%\spr  \sliste{ },\\
                 \comps \sliste{ \npnom{} }}
       [\npacc [das Lied;the song,roof]]
       [V\feattab{%\spr \sliste{ },\\
                  \comps \sliste{ \npnom, \npacc{} }}
         [V\feattab{%\spr \sliste{ },\\
                    \comps \sliste{ \npnom, \npacc, V[\lex+] }} [wird;will]]
           [V\feattab{\subj \sliste{ \npnom{} },\\
                     %          \spr \sliste{ },\\
                               \comps \sliste{ \npacc{} }}
             [V\feattab{\subj \sliste{ \npnom{} },\\
                      %           \spr \sliste{ },\\
                                 \comps \sliste{ \npacc{} }} [singen;sing]]
             [V\feattab{\subj \sliste{ \npnom{} },\\
                       % \spr \sliste{ },\\
                        \comps \sliste{ \npacc, V[\lex+] }} [dürfen;be.allowed.to]]]]]]]
\end{forest}}

\end{enumerate}


\section{Verb position and fronting}

\begin{enumerate}


\item Which of the following structures is appropriate for English? (answer may be any number
  including zero) S stands for \spr \sliste{}, \comps \sliste{}, VP
for \spr \sliste{ \npnom{} }, \comps \sliste{}  and V$'$ for all other V projections.

Welche der folgenden Strukturen ist für englische Sätze angemessen? (mehrere Strukturen oder keine
Struktur sind mögliche Antworten) In den Abbildungen stehen S für \spr \sliste{}, \comps \sliste{}, VP
für \spr \sliste{ \npnom{} }, \comps \sliste{}  und V$'$ für alle anderen V-Projektionen.

\begin{forest}
[S
  [NP]
  [VP
    [V]
    [NP]]]
\end{forest}

\begin{forest}
[S
  [NP]
  [S/NP
    [NP]
    [VP/NP
      [V]
      [NP/NP [\trace]]]]]
\end{forest}

\begin{forest}
[S
  [NP]
  [S/NP
    [V]
    [S//V/NP
      [NP/NP [\trace]]
      [VP//V
        [V//V [\trace]]
        [NP]]]]]
\end{forest}


\begin{forest}
[S
  [NP]
  [S/NP
    [V]
    [S//V/NP
      [NP]
      [VP//V/NP
        [V//V [\trace]]
        [NP/NP [\trace]]]]]]
\end{forest}

\begin{forest}
[S
  [NP]
  [S/NP
    [V]
    [S//V/NP
      [NP]
      [V$'$//V/NP
        [NP/NP [\trace]]
        [V//V [\trace]]]]]]
\end{forest}

\begin{forest}
[S
  [NP]
  [S/NP
    [V]
    [S//V/NP
      [NP/NP [\trace]]
      [V$'$//V
        [NP]
        [V//V [\trace]]]]]]
\end{forest}

Antwort: die erste und die zweite    Bäume bitte mischen, \dash nicht in dieser Reihenfolge.

\item Which of the following structures is appropriate for Icelandic? (answer may be any number
  including zero) S stands for \spr \sliste{}, \comps \sliste{}, VP
for \spr \sliste{ \npnom{} }, \comps \sliste{}  and V$'$ for a V projection with empty \spr list.

Welche der folgenden Strukturen ist für isländische Sätze angemessen? (mehrere Strukturen oder keine
Struktur sind mögliche Antworten) In den Abbildungen stehen S für \spr \sliste{}, \comps \sliste{}, VP
für \spr \sliste{ \npnom{} }, \comps \sliste{}  und V$'$ für eine V-Projektion mit leerer \sprl.

Bäume wie oben. Lösung: dritter und vierter Baum.

\item Which of the following structures is appropriate for Dutch? (answer may be any number
  including zero) S stands for \spr \sliste{}, \comps \sliste{}, VP
for \spr \sliste{ \npnom{} }, \comps \sliste{}  and V$'$ for a V projection with empty \spr list.

Welche der folgenden Strukturen ist für niederländische Sätze angemessen? (mehrere Strukturen oder keine
Struktur sind mögliche Antworten) In den Abbildungen stehen S für \spr \sliste{}, \comps \sliste{}, VP
für \spr \sliste{ \npnom{} }, \comps \sliste{}  und V$'$ für eine V-Projektion mit leerer \sprl.

Bäume wie oben. Lösung: fünfter und sechster Baum.

\item Which of the following structures is suitable for the sentence in (\mex{1}). \emph{these
    delicious bagels} is suposed to correspond to one x. Please, assume the analysis layed out in \citet{MuellerGermanic}.

Welche der folgenden Strukturen ist für den Satz in (\mex{1}) sinnvoll? \emph{these
    delicious bagels} entspricht dabei einem x. Bitte legen Sie die Analyse von
  \citet{MuellerGermanic} zugrunde.
\ea
These delicious bagels, Sandy ordered yesterday.
\z

\begin{forest}
sm edges
[x 
  [x
    [x [x [x]]
       [x [x]]]
    [x [x]]]
  [x [x]]]]
\end{forest}
\begin{forest}
sm edges
[x
  [x [x [x]]
     [x [x]]]
  [x
    [x [x]]
    [x [x]]]]
\end{forest}
\begin{forest}
sm edges
[x
  [x [x [x]]
     [x [x]]]
  [x
    [x [x]]
    [x [x]]
    [x [x]]]]
\end{forest}
\begin{forest}
sm edges
[x
  [x [x]]
  [x [x [x]]
     [x
       [x [x]]
       [x [x]]]]]
\end{forest}
\begin{forest}
sm edges
[x
  [x [x]]
  [x [x [x]]
     [x [x
          [x [x]]
          [x [x]]]
        [x [x]]]]]
\end{forest}
\begin{forest}
sm edges
[x
  [x [x]]
  [x [x [x]]
     [x 
        [x [x]]
        [x
          [x [x]]
          [x [x]]]]]]
\end{forest}

Antwort: (5)



\item Assign categories to the numbered nodes.

Ordnen Sie den Knotenzahlen Kategorien zu.

Dabei gibt es auch die Wörter zur Auswahl, weil Teil der Aufgabe ist, die Spur richtig zu platzieren.

Labels: NP, PP, AP, VP, S, CP, Adv, S/NP, VP/NP NP/NP, AP/NP, V$'$, CP/NP

\begin{forest}
sm edges
[1
  [2 [10] ]
  [3 
    [4 [11] ]
    [5 
      [6  
        [8 [12] ]
        [9 [13] ] ]
      [7 [14] ] ] ]]
\end{forest}

\begin{forest}
sm edges
[S
  [NP$_i$ [these delicious bagels,roof] ]
  [S/NP 
    [NP [Sandy] ]
    [VP/NP 
      [VP/NP  
        [V [ordered] ]
        [NP/NP [\_$_i$] ] ]
      [Adv [yesterday] ] ] ]]
\end{forest}


%\fi
\item Which of the following structures is suitable for the sentence in (\mex{1}). \emph{the
    children} is suposed to correspond to one x. 

Welche der folgenden Strukturen ist für den Satz in (\mex{1}) sinnvoll? \emph{the children} entspricht dabei einem x.
\ea
Kim bought the children books.
\z

\begin{forest}
sm edges
[x
  [x [x]]
  [x
    [x 
      [x [x]]
      [x [x]]]
    [x [x]]]]
\end{forest}
\hspace{2em}
\begin{forest}
sm edges
[x
   [x [x] ]
   [x
      [x
        [x] ]
      [x
          [x [x] ]
          [x
             [x
               [x  [x] ]
               [x [x] ]]
             [x [x ] ] ] ] ] ]
\end{forest}
\hspace{2em}
\begin{forest}
sm edges
[x
   [x [x] ]
   [x
      [x
        [x] ]
      [x
          [x [x] ]
          [x
             [x [x ] ] 
             [x
               [x  [x] ]
               [x [x] ]]] ] ] ]
\end{forest}
\hspace{2em}
\begin{forest}
sm edges
[x
  [x [x]]
  [x
    [x [x]]
    [x 
      [x [x]]
      [x [x]]]]]
\end{forest}

Answer: 1

%\if0
\item Assign categories to the numbered nodes.

Ordnen Sie den Knotenzahlen Kategorien zu.

Dabei gibt es auch die Wörter zur Auswahl, weil Teil der Aufgabe ist, die Spur richtig zu platzieren.

Labels: NP, PP, AP, VP, S, CP, Adv, S/NP, VP/NP NP/NP, AP/NP, V$'$, CP/NP

\begin{forest}
sm edges
[1
  [2 [8]]
  [3
    [4 
      [6 [9]]
      [7 [10]]]
    [5 [11]]]]
\end{forest}


\begin{forest}
sm edges
[S
  [NP [Kim]]
  [VP
    [V$'$ 
      [V [bought]]
      [NP [the children,roof]]]
    [NP [books]]]]
\end{forest}

%\fi

\item Which of the following structures is suitable for the sentence in (\mex{1}).

Welche der folgenden Strukturen ist für den Satz in (\mex{1}) sinnvoll?
\ea
\gll Attackiert der        Delphin jetzt den        Hai?\\
     attacs     the.\NOM{} dolphin now   the.\ACC{} shark\\\german
\glt `Does the dolphin attack the shark now?'
\z


\begin{forest}
sm edges
[x
  [x
    [x] ]
  [x
    [x [x] ]
    [x
       [x [x]]
       [x
          [x [x] ]
          [x [x] ] ] ] ] ] 
\end{forest}
\hspace{2em}
\begin{forest}
sm edges
[x
  [x 
    [x 
      [x [x] ]
      [x [x] ] ]
    [x [x]]]
  [x [x]]]
\end{forest}
\hspace{2em}
\begin{forest}
sm edges
[x
  [x
    [x] ]
  [x
    [x [x] ]
    [x
       [x [x]]
       [x [x] ] ] ] ]
\end{forest}
\hspace{2em}
\begin{forest}
sm edges
[x
  [x
    [x 
      [x 
        [x [x] ]
        [x [x] ] ]
      [x [x]]]
    [x [x]]]
  [x [x]]]
\end{forest}

Antwort: 1

\item Assign categories to the numbered nodes.

Ordnen Sie den Knotenzahlen Kategorien zu.

Dabei gibt es auch die Wörter zur Auswahl, weil Teil der Aufgabe ist, die Spur richtig zu platzieren.

Labels: NP, PP, AP, VP, S, CP, Adv, S/NP, VP/NP NP/NP, AP/NP, V$'$, CP/NP

\begin{forest}
sm edges
[1
  [2
    [4 [11] ] ]
  [3
    [5 [12] ]
    [6
       [7 [13]]
       [8
          [9 [14] ]
          [10 [15] ] ] ] ] ] 
\end{forest}

\begin{forest}
sm edges
[S
  [{V \sliste{ S$/\!/$V }} 
    [V [attackiert$_k$;attacks] ] ]
       [{S$/\!/$V}
           [NP [der Delphin;the dolphin, roof] ]
           [{V$'$$\!/\!/$V}
             [Adv [jetzt;now]]
             [{V$'$$\!/\!/$V}
               [NP [den Hai;the shark, roof] ]
               [{V$\!/\!/$V} [\_$_k$] ] ] ] ] ] 
\end{forest}

\item Which of the following structures is suitable for the sentence in (\mex{1}).

Welche der folgenden Strukturen ist für den Satz in (\mex{1}) sinnvoll?
\ea
\gll Billedet             viser  Kim nu  Gert.\\
     picture.\textsc{def} shows Kim now Gert\\\danish
\glt `Kim now shows Gert the picture.'
\z

\begin{forest}
sm edges
[x
   [x [x] ]
   [x
      [x
        [x] ] 
      [x
          [x [x] ]
          [x
             [x
               [x
                 [x  [x] ]
                 [x [x] ]]
               [x [x ] ] ]
             [x [x]] ] ] ] ]
\end{forest}
\hspace{2em}
\begin{forest}
sm edges
[x
   [x [x] ]
   [x
     [x
       [x
         [x [x] ] 
         [x [x] ] ]
       [x [x] ]]
     [x [x] ]]]
\end{forest}
\hspace{2em}
\begin{forest}
sm edges
[x
   [x [x] ]
   [x
     [x
       [x
         [x
           [x [x] ] 
           [x [x] ] ]
         [x [x] ]]
       [x [x] ]]
     [x [x] ]]]
\end{forest}
\hspace{2em}
\begin{forest}
sm edges
[x
   [x [x] ]
   [x
      [x [x] ]
      [x
          [x [x] ]
          [x
             [x [x]]
             [x
               [x
                 [x [x] ]
                 [x [x] ]]
               [x [x ] ] ] ] ] ] ]
\end{forest}

\item Assign categories to the numbered nodes.

Ordnen Sie den Knotenzahlen Kategorien zu.

Dabei gibt es auch die Wörter zur Auswahl, weil Teil der Aufgabe ist, die Spur richtig zu platzieren.

Labels: NP, PP, AP, VP, S, CP, Adv, S/NP, VP/NP NP/NP, AP/NP, V$'$, CP/NP


\begin{forest}
sm edges
[1
   [2 [15] ]
   [3
      [4
        [6 [16] ] ]
      [5
          [7 [17] ]
          [8
             [9 [18]]
             [10
               [11
                 [13  [19] ]
                 [14 [20] ]]
               [12 [21 ] ] ] ] ] ] ]
\end{forest}


\begin{forest}
sm edges
[S
   [NP$_i$ [Billedet;picture.\textsc{def}] ]
   [S/NP
      [V \sliste{ S$/\!/$V }
        [V [viser$_j$;shows] ] ]
        [S$/\!/$V$\!$/NP
          [NP [Kim;Kim] ]
          [VP$\!/\!/$V$\!$/NP
             [Adv [nu;now]]
             [VP$\!/\!/$V$\!$/NP
               [V$'$$\!/\!/$V
                 [V$\!/\!/$V  [\_$_j$] ]
                 [NP [Gert;Gert] ]]
               [NP/NP [\_$_i$ ] ] ] ] ] ] ]
\end{forest}



\item Danish allows the following two orderings. Dänisch erlaubt die beiden folgenden Abfolgen:
\eal
\ex 
\gll Billedet             viser Kim nu  Gert.\\
     picture.\textsc{def} shows Kim now Gert\\\danish
\glt `Kim now shows Gert the picture.'
\ex 
\gll Billedet             viser Kim Gert nu.\\
     picture.\textsc{def} shows Kim Gert now\\
\glt `Kim shows Gert the picture now.'
\zl

Which one coresponds to the following structure? Welche entspricht der folgenden Struktur?


\begin{forest}
sm edges
[1
   [2 [15] ]
   [3
      [4
        [6 [16] ] ]
      [5
          [7 [17] ]
          [8
             [9
               [11
                 [13  [18] ]
                 [14 [19] ]]
               [12 [20 ] ] ]
             [10 [21]] ] ] ] ]
\end{forest}

Answer: (\mex{0}b) The full tree is:

\begin{forest}
sm edges
[S
   [NP$_i$ [Billedet;picture.\textsc{def}] ]
   [S/NP
      [V \sliste{ S$/\!/$V }
        [V [viser$_j$;shows] ] ]
        [S$/\!/$V$\!$/NP
          [NP [Kim;Kim] ]
          [VP$\!/\!/$V$\!$/NP
             [VP$\!/\!/$V$\!$/NP
               [V$'$$\!/\!/$V
                 [V$\!/\!/$V  [\_$_j$] ]
                 [NP [Gert;Gert] ]]
               [NP/NP [\_$_i$ ] ] ]
             [Adv [nu;now]] ] ] ] ]
\end{forest}

%\fi
\end{enumerate}


\end{document}