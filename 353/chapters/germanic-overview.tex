\chapter{A general overview of the Germanic languages}

This chapter provides an overview of general facts about the \ili{Germanic} languages. It derives from
slides for teaching courses about \ili{Germanic} languages that were used by Ekkehard König and passed on
to Matthias Hüning and via Matthias to me, which explains the similarity to the introductory
chapter by \citet{HvDA94a} in the book \emph{The Germanic Languages} edited by \citet{KvdA94a-ed}.



\section{Languages and speakers}

Depending on whom one asks, there are between 5000 and 7000 languages spoken worldwide currently. The
\ili{Germanic} languages are a small subset of these, 15--20 languages depending on the counting because the distinction between language and language variety is not always made according to the same criteria (\eg varieties of \ili{Dutch}). 
%AZ: (\eg varieties of \ili{Frisian}).AZ: I would not choose \ili{Frisian} as an example because some people might think of it as a variety of \ili{Dutch}.  
According to Max \citet[\page 13]{Weinreich45a-u}, a language is a dialect with an army and a
navy. According to this ``definition'', neither \ili{Yiddish} nor \ili{Faroese} would be a language.\footnote{%
See \citet[\page 13]{Weinreich45a-u} on \ili{Yiddish}: \url{https://en.wikipedia.org/wiki/A_language_is_a_dialect_with_an_army_and_navy}.
} 
%AZ The Swiss army has a bicycle group but the country has no navy.
%  https://de.wikipedia.org/wiki/Motorbootkompanie
%That the Swiss army has a bicycle group instead of a navy does not help. 
%This brief discussion should indicate that 
It is often a political question whether two closely related variants of a language are treated as different languages or not (\ili{Slovak} vs.\ \ili{Czech}, \ili{Serbian} vs.\ \ili{Croatian}, \ili{Danish} vs.\ \ili{Norwegian}).
Altogether the \ili{Germanic} languages have almost 500 million native speakers, which is 1/12 of the whole population of
the world. \ili{English} is especially widespread in terms of regions in which the language is spoken.


\section{Historical remarks and relatedness between the languages}


The \ili{Germanic} languages constitute a separate branch of the tree representing the \ili{Indo-European}
language family \citep[\page 665]{Fitch2007a-u}.
%(see Figure~\vref{fig-indo-european-fitch}).\footnote{
% The tree is a simplification ignoring many many languages. The sizes of the branches do not
% correspond to the number of speakers.
% \begin{figure}
% \includegraphics[width=\textwidth]{Pictures/indoeuropaeisch}
% \caption{\label{fig-indo-european-fitch}Language tree according to \citet[\page 665]{Fitch2007a-u}}
% \end{figure}
\ili{Proto-Germanic} formed between 2000 and
1000 BCE. Its origins are in the Baltic region, that is, in northern Germany and southern
Scandinavia. About 500 BCE, the area where it was spoken extended from the North Sea to Poland.
The first written documents are runes from about 300 CE and the \ili{Gothic} Bible translation in the fourth century.
The First \ili{Germanic} Sound Shift took place before the second century BCE. In that millennium the
\ili{Germanic} languages developed different consonants from the other \ili{Indo-European} languages. 

\citet[\page 8]{Harbert2006a-u} provides the Figure~\vref{fig-history-relations-germanic} that depicts the development of
the \ili{Germanic} languages.
\begin{figure}
%\begin{sideways}
%\includegraphics[width=.9\textheight]{Pictures/germanic-wikipedia}
%\end{sideways}
\begin{sideways}
\scalebox{.72}{%
% \begin{forest}forked edges, for tree={grow'=0, child anchor=west, parent anchor=east, align=center}
% [Proto-Germanic\il{Germanic!Proto-}
%     [Northwest Germanic\il{Germanic!Northwest}
%         [North Germanic\il{Germanic!North}, tier=preroman
%             [Old Norse\il{Norse!Old}
%                 [West Norse\il{Norse!West}, tier=premodern
%                     [\ili{Icelandic}, tier=modern]
%                     [\ili{Farocese}, tier=modern]
%                     [\ili{Norwegian}, tier=modern
%                         [\ili{Nynorsk}]
%                     ]
%                 ]
%                 [East Norse\il{Norse!East}
%                     [\ili{Danish}, tier=modern
%                         [\ili{Bokmål}]
%                     ]
%                     [\ili{Swedish}, tier=modern]
%                 ]
%              ]   
%         ]
%         [West Germanic\il{Germanic!West}, tier=preroman
%             [North Sea Germanic\il{Germanic!North Sea}\\(Ingvaeonic)
%                 [Anglo-Frisian\il{Frisian!Anglo-}
%                         [\ili{English}, tier=modern]
%                         [\ili{Frisian}, tier=modern]
%                 ]        
%                 [Old Saxon\il{Saxon!Old}
%                     [Low German\il{German!Low}, tier=modern]
%                 ]
%             ]
%             [\ili{Franconian}\\(Istvaeonic)
%                 [Low Franconian\il{Franconian!Low}, tier=middleages
%                     [Middle Dutch\il{Dutch!Middle}, tier=latemiddleages
%                         [\ili{Dutch}, tier=modern]
%                         [\ili{Afrikaans}, tier=modern]
%                     ]
%                 ]
%                 [High Franconian\il{Franconian!High}, tier=middleages, name=franconian
%                 ]
%             ]
%             [Alpine Germanic\il{Germanic!Alpine}\\(Irminonic)
%                 [\ili{Alemannic}\\\small{(Upper German\il{German!Upper})}, tier=middleages, name=alemannic
%                 	[Middle High\\German\il{German!Middle High}, tier=latemiddleages, no edge, name=mhg 
%                 		[High German\il{German!High}, tier=modern]
%                 		[PA German\il{Pennsylvania German}, tier=modern]
%                 		[\ili{Yiddish}, tier=modern]
%                 	]
%                 ]               
%                 [\ili{Bavarian}\\\small{(Upper German)\il{German!Upper}}, tier=middleages, name=bavarian]
%             ]
%         ]
%     ]    
%     [East Germanic\il{Germanic!East},  tier=preroman
%         [\ili{Gothic} (and others)]
%     ]
% ]
% {\draw[black] (franconian.east)--(mhg.west);}
% {\draw[black] (bavarian.east)--(mhg.west);}
% {\draw[black] (alemannic.east)--(mhg.west);}
% \end{forest}}
\begin{forest}forked edges, for tree={grow'=0, child anchor=west, parent anchor=east, align=center}
[Proto-Germanic\il{Germanic!Proto-}
    [Northwest Germanic\il{Germanic!Northwest}
        [North Germanic\il{Germanic!North}, tier=preroman
            [Old Norse\il{Norse!Old}
                [West Norse\il{Norse!West}, tier=premodern
                    [\ili{Icelandic}, tier=modern]
                    [\ili{Farocese}, tier=modern]
                    [\ili{Norwegian}, tier=modern
                        [\ili{Nynorsk}]
                    ]
                ]
                [East Norse\il{Norse!East}
                    [\ili{Danish}, tier=modern
                        [\ili{Bokmål}]
                    ]
                    [\ili{Swedish}, tier=modern]
                ]
             ]   
        ]
        [West Germanic\il{Germanic!West}, tier=preroman
            [North Sea Germanic\il{Germanic!North Sea}\\(Ingvaeonic)
                [Anglo-Frisian\il{Frisian!Anglo-}
                        [\ili{English}, tier=modern]
                        [\ili{Frisian}, tier=modern]
                ]        
                [Old Saxon\il{Saxon!Old}
                    [Low German\il{German!Low}, tier=modern]
                ]
            ]
            [\ili{Franconian}\\(Istvaeonic)
                [Low Franconian\il{Franconian!Low}, tier=middleages
                    [Middle Dutch\il{Dutch!Middle}, tier=latemiddleages
                        [\ili{Dutch}, tier=modern]
                        [\ili{Afrikaans}, tier=modern]
                    ]
                ]
                [High Franconian\il{Franconian!High}, tier=middleages, name=franconian
                ]
            ]
            [Alpine Germanic\\(Irminonic)
                [\ili{Alemannic}\\\small{(Upper German)}, tier=middleages, name=alemannic
                	[Middle High\\German, tier=latemiddleages, no edge, name=mhg 
                		[High German\il{German!High}, tier=modern]
                		[PA German\il{Pennsylvania German}, tier=modern]
                		[\ili{Yiddish}, tier=modern]
                	]
                ]               
                [\ili{Bavarian}\\\small{(Upper German)}, tier=middleages, name=bavarian]
            ]
        ]
    ]    
    [East Germanic\il{Germanic!East},  tier=preroman
        [\ili{Gothic} (and others)]
    ]
]
{\draw[black] (franconian.east)--(mhg.west);}
{\draw[black] (bavarian.east)--(mhg.west);}
{\draw[black] (alemannic.east)--(mhg.west);}
\end{forest}}
\end{sideways}
% Index entries have to be outside the figure if/since memoize is used
\il{German!Middle High}\il{German!Upper}\il{German!Middle High}\il{German!Upper}\il{Bavarian}\il{Germanic!Alpine}\il{Afrikaans}%
\il{Alemannic}\il{Germanic!Proto-}\il{Germanic!Northwest}\il{Germanic!North}\il{Norse!Old}\il{Norse!West}\il{Icelandic}%
\il{Farocese}\il{Norwegian}\il{Nynorsk}\il{Norse!East}\il{Danish}\il{Bokmål}\il{Swedish}\il{Germanic!West}\il{Germanic!North Sea}%
\il{Frisian!Anglo-}\il{English}\il{Frisian}\il{Saxon!Old}\il{German!Low}\il{Franconian}\il{Franconian!Low}\il{Dutch!Middle}%
\il{Dutch}\il{Afrikaans}\il{Franconian!High}\il{Alemannic}\il{German!High}\il{Pennsylvania German}\il{Yiddish}\il{Bavarian}%
\il{Germanic!East}\il{Gothic}%
\caption{\label{fig-history-relations-germanic}Development of Germanic languages according
  to \citew[\page 8]{Harbert2006a-u}}
\end{figure}
\ili{Germanic} is divided into East\il{Germanic!East}, West\il{Germanic!West}, and
North\il{Germanic!North} Germanic. East Germanic existed in the form of \ili{Gothic}
until about 1800 in Crimea (Crimean Gothic\il{Gothic!Crimean}) and is now totally extinct. 

West \ili{Germanic} consists of 
\begin{itemize}
\item \ili{German}, 
\item \ili{Yiddish}, 
\item \ili{Luxembourgish}, 
\item Pennsylvania \ili{Dutch}, 
\item Low German\il{German!Low}, % Plattdeutsch or Niederdeutsch
\item \ili{Plautdietsch} (also called Mennonite Low German\il{German!Mennonite Low}), %
\item \ili{Dutch}, 
\item \ili{Afrikaans}, 
\item \ili{Frisian}, and 
\item \ili{English}.
\end{itemize}
\largerpage
The North \ili{Germanic} languages are:
\begin{itemize}
\item \ili{Danish},
\item \ili{Swedish},
\item \ili{Norwegian},
\item \ili{Icelandic}, and 
\item \ili{Faroese}.
\end{itemize}

\noindent
Table~\vref{tab-words-germanic} shows how similar the words from the main vocabulary of the \ili{Germanic}
languages are:
\begin{table}
\centerfit{%
\begin{tabular}{lllllllll}
\hline\hline
\ili{Dutch} & vader  & vier    & vol    & huis  & bruin  & uit & kruid     & muis\\
\ili{German}        & Vater  & vier    & voll   & Haus  & braun  & aus & Kraut     & Maus\\
\ili{English}       & father & four    & full   & house & brown  & out & crowd (?) & mouse\\
\ili{Frisian}      & –      & fjouwer & fol    & hûs   & brún   & út  & krûd      & mûs\\
\ili{Swedish}     & fader  & fyra    & full   & hus   & brun   & ut  & krut      & mus\\
\ili{Danish}        & fader  & fire    & fuld   & hus   & brun   & ud  & krudt     & mus\\
\ili{Norwegian}     & far    & fire    & full   & hus   & brun   & ut  & krydder   & mus\\
\ili{Icelandic}     & faðir  & fjórir  & fullur & hús   & brúnn  & út  & –         & mús\\
\hline\hline
\end{tabular}
}
\caption{\label{tab-words-germanic}Words from the main vocabulary of some Germanic languages}
\end{table}

%% \section{Stammbaum der germanischen Sprachen}
%% \includegraphics[width=\textwidth]{Pictures/stammbaum-germanisch}
%% aus \citew[S.\,251]{Bussmann2002a}





\section{The three branches of the Germanic family}

\ili{Proto-Germanic} developed into the three main branches East, West, and North \ili{Germanic}, approximately in
the first century CE. The reasons for this development were inherent variations in the respective
dialects, migration (language contact) and standardization. This book treats the structure of the
\ili{Germanic} standard languages. This section is divided into three subsections that correspond to the
three main \ili{Germanic} branches. I will sketch the historical developments that lead to the
languages spoken today. Many of the details that are covered in Figure~\ref{fig-history-relations-germanic} will be ignored.



\subsection{East Germanic}

It\il{Germanic!East|(} is often claimed that the Goths emigrated to mainland Europe from the \ili{Danish} islands and South
Sweden, but more recent research taking archaeological findings into account assumes that they lived
on the European mainland opposite Scandinavia around the Vistula in the first century CE and later moved south to the hinterland of
the Black Sea \citep[20--30]{Heather1999a-u}. During this time they were in contact with the Vandals\il{Vandalic} and other tribes.
%emigrated from the \ili{Danish} islands and South Sweden around 100 BCE and met the
%Vandals and other tribes.\itdopt{%
%Walkden: Most current historical research (since e.g. Heather 1998, The Goths, 25–30) rejects the
%idea that the Goths originated in Scandinavia, and put their earliest origin around the Vistula. Not
%a crucial point in the context of this book, but may be worth changing.
%}
\ili{Gothic}, \ili{Vandalic}, and \ili{Burgundian} and some smaller languages constituted the East Germanic branch, of which only \ili{Gothic} got passed on.
After the decay of the \ili{Gothic} empires \ili{Gothic} died out. There were some remnants on the Crimean
peninsula until about 1800. The West \ili{Gothic} bishop Wulfila (or rather a team lead by him, see
\citealt{Ratkus2018a-u}) translated the Bible into \ili{Gothic}. The
best-known version of it is the fragment Codex Argenteus, which belongs to the university library of
Uppsala. Figure~\vref{fig-Wulfila-Bibel} shows a picture of it.\footnote{
Taken from Wikipedia: \url{http://de.wikipedia.org/wiki/Bild:Wulfila_bibel.jpg}. 19.10.2014.
}
\begin{figure}
\includegraphics[width=53mm]{Pictures/Wulfila_bibel}
\caption{\label{fig-Wulfila-Bibel}The Wulfila Bible (Codex Argenteus), picture from Wikipedia}
\end{figure}
\il{Germanic!East|)}


\subsection{North Germanic}

The\il{Germanic!North|(} first writings on runestones date back to the sixth century. The language of the Vikings
(800--1050) was rather homogeneous and it was only after this era that two branches started to
develop: the East Scandinavian\il{Scandinavian!East} branch with Old Danish\il{Danish!Old} and Old Swedish\il{Swedish!Old} and the West Scandinavian\il{Scandinavian!West} one
with Old Norwegian\il{Norwegian!Old} and Old Icelandic\il{Icelandic!Old}.


\subsubsection{Danish}

\ili{Danish} (dansk) is the official language of the
\href{https://en.wikipedia.org/wiki/Denmark}{Kingdom of Denmark} and the second official language of the
Faroe Islands and of Greenland, \ili{Inuit} being the first official language of
Greenland. \ili{Danish} has about 5.5 million speakers. About 50,000 speakers live in \href{https://en.wikipedia.org/wiki/Schleswig-Holstein}{Schleswig-Holstein}, the northernmost of the federal states of Germany.
\ili{Danish} is the \ili{Scandinavian} language that drifted furthest away from the common \ili{Scandinavian} roots.


\subsubsection{Swedish}

\ili{Swedish} (svenska) is the official language in Sweden with about 8.5 million native speakers. It is
the first language of about 300,000 \ili{Swedish}-speaking Finns in Finland. Until the times of the
Vikings \ili{Danish} and \ili{Swedish} were almost indistinguishable. Starting from about 800 they started to
diverge. Since about 1300 there have been obvious differences.



\subsubsection{Icelandic}


\ili{Icelandic} (íslenska) is the West \ili{Scandinavian} language of Iceland since its settlement over 1000 years ago. 
% Wikipedia 23.10.2014
There are about 325,000 native speakers. 97\,\% of the \ili{Icelandic} population (325,000) has \ili{Icelandic} as their
mother tongue and there are large groups of native speakers in Denmark, the USA, and Canada
(about 15,000 in total). 
There is less variation than in other \ili{Germanic} languages (no dialects). 
%AZ that is too strong there is actually quite a bit of variation but maybe it is correct to say that there is less variation than in other \ili{Germanic} languages
The language is conservative, in the sense that \ili{Icelandic}
is the language among the \ili{Germanic} languages that best preserved \ili{Germanic} vocabulary and inflection.
In the beginning there were almost no differences between \ili{Norwegian} and \ili{Icelandic} but
% The \ili{Dano-Norwegian}, then later \ili{Danish} rule of Iceland from 1536 to 1918 had little effect on the
% evolution of \ili{Icelandic} (in contrast to the \ili{Norwegian} language), which remained in daily use among
% the general population. 
% AZ: when?
starting about 1100 the languages diverged. This process continued also due to the \ili{Danish}
influence and nowadays \ili{Norwegian} is more similar to \ili{Danish} than to \ili{Icelandic}. There are many written documents in \ili{Icelandic}.

\subsubsection{Norwegian}

\largerpage
There are two standard varieties of \ili{Norwegian} (norsk): Danish-Norwegian\il{Norwegian!Danish-} (bokmål) and New-Norwegian\il{Norwegian!New} (nynorsk, landsmål). 
Both are official languages of Norway.
% and are used in parallel\itdopt{AZ: not clear what that means}. 
There are about 4,3 million speakers.
From 1380 to 1814 \ili{Danish} was the written language and local dialects were spoken in Norway. It developed into the bokmål standard.
A standard that is less influenced by \ili{Danish} was also developed. This was done by Ivar Aasen (1813--1896), 
who developed \ili{Nynorsk}. \ili{Nynorsk} got an official status in 1885. \ili{Bokmål} `book tongue' is the first language of most
of the Norwegians.

\subsubsection{Faroese}

\ili{Faroese} (føroyskt) is -- alongside \ili{Danish} -- an official language of the Faroe Islands. There
are about 47,000 speakers. The Faroe Islands have belonged to Denmark since 1816. Since 1948 they
have been a self-governing country within the \href{https://en.wikipedia.org/wiki/Danish_Realm}{\ili{Danish} Realm}.
\ili{Faroese} has a strong \ili{Danish} influence. The first manuscript transmission is as recent as 1773 and
even after that date there are not many written documents.% 
%AZ i would leave this out: (in contrast to \ili{Icelandic}).And put something in the \ili{Icelandic} section
\il{Germanic!East|)}

\subsection{West Germanic}


Opinions\il{Germanic!West|(} on the question whether West Germanic developed from a single source or not differ. Some authors
assume that the West \ili{Germanic} languages do not have a common root, but instead developed from the
following three unrelated branches of dialect groups (for instance \citealp[\page 17--18]{Robinson1992a-u} and
\citealp[\page 9]{HvDA94a}):
\begin{itemize}
\item North Sea Germanic\il{Germanic!North Sea} %(Ingvaeonic, ancestral to \ili{Anglo-Frisian} and also Old Saxon)
\item Weser-Rhine Germanic\il{Germanic!Weser-Rhine} %(Istvaeonic, ancestral to Old \ili{Frankish},
                                %its successors \ili{Low Franconian} and several dialects of Old High \ili{German})
\item Elbe Germanic\il{Germanic!Elbe} %(Irminonic, ancestral to several dialects of Old High \ili{German}, most probably including the extinct \ili{Langobardic} language).
\end{itemize}
Other authors assume that these three branches had a common ancestor (see
Figure~\ref{fig-history-relations-germanic}) and some disagree with dividing West Germanic into
three subbranches altogether \citep{Stiles2013a-u}.
There is no unique mapping of these dialect groups to the languages spoken today.%

\largerpage
\subsubsection{German}

\ili{German} is the official language of
\begin{itemize}
\item Germany (about 80 million speakers), 
\item Austria (about 7.5 million speakers), 
\item Liechtenstein (about 15,000 speakers), 
\item Switzerland (4.2 million of 6.4 million Swiss residents),
\item Northern Italy/South Tyrol (about 270,000 speakers), 
\item Belgium (about 65,000 speakers), and
\item Luxembourg (about 360,000 speakers).
\end{itemize}
In addition to \ili{German}, Luxembourg also has \ili{Luxembourgish} and \ili{French} as official
languages. 
%Poland 50,000 native speakers.
There are further countries in which \ili{German} is a national language or a national or regional
minority language: Brazil, Czech Republic, Denmark, Hungary, Namibia, Poland, \ili{Romania}, Russia, and Slovakia.

% There are about 97 million speakers of \ili{German}, about, 90 of those are native speakers and 7 million
% have \ili{German} as their second language.
% % = Migrationshintergrund
% % Quelle Wikipedia 15.10.2013
% Approximately 80 million speakers speak \ili{German} as a foreign language, about 55 Million of these live
% in the \ili{European} Union.%AZ It is a bit awkward to get these details for \ili{German} but not for all the other languages

There are three main national variants (Germany, Austria, Switzerland). In other countries \ili{German} is
%AZ leave out usually 
a minority language. There are two large dialect groups: German\il{German!Low} (Plattdüütsch,
Nedderdüütsch; in Standard \ili{German}: Platt\-deutsch or Nie\-der\-deutsch%
%AZ this is clear as mud: do you
%want to say that \ili{Dutch} is part of low Saxon or as most people do
%that low saxon are the dialects of \ili{German} that are spoken in in the Netherlands? I would say:
%Nedersaksisch or Low Saxon, a group of dialects spoken in the Netherlands, southern Danmark and
%northwestern Germany
% Well, the areas in which Low Saxon is spoken overlap with national border. I speak about langauges
% spoken in these national borders here.
) and High German\il{German!High} (varieties of \ili{German} spoken south of the Benrath and Uerdingen
isoglosses).
%\itdopt{Stefan: fix}
%\todostefan{more}

% Wikipedia
%% The High \ili{German} languages or High \ili{German} dialects (\ili{German}: Hochdeutsche Dialekte) comprise the varieties of \ili{German} spoken south of the Benrath and Uerdingen isoglosses in central and southern Germany, Austria, Liechtenstein, Switzerland, and Luxembourg as well as in neighboring portions of Belgium (Eupen-Malmedy) and the Netherlands (Southeast Limburg), France (Alsace and northern Lorraine), Italy (South Tyrol), and Poland (Upper Silesia). They are also spoken in diaspora in \ili{Romania}, Russia, the United States, Brazil, Argentina, Chile, and Namibia.

%% The High \ili{German} languages are marked by the High \ili{German} consonant shift, separating them from Low \ili{German} and Low \ili{Franconian} (\ili{Dutch}) within the continental West \ili{Germanic} dialect continuum.


%% Wikipedia

%% \ili{Middle} Low \ili{German} was the lingua franca of the Hanseatic League. It was the predominant language in \ili{Northern} Germany. This changed in the 16th century: in 1534 the Luther Bible was published. This translation is considered to be an important step towards the evolution of the Early New High \ili{German}. It aimed to be understandable to a broad audience and was based mainly on Central and Upper \ili{German} varieties. The Early New High \ili{German} language gained more prestige than Low \ili{German} and became the language of science and literature. Around the same time, the Hanseatic league, based around northern ports, lost its importance as new trade routes to Asia and the Americas were established, while the most powerful \ili{German} states of that period were located in \ili{Middle} and Southern Germany.

%% The 18th and 19th centuries were marked by mass education in Standard \ili{German} in schools. Gradually Low \ili{German} came to be politically viewed as a mere dialect spoken by the uneducated. Today Low Saxon can be divided in two groups: Low Saxon varieties with a reasonable standard \ili{German} influx[clarification needed] and varieties of Standard \ili{German} with a Low Saxon influence known as Missingsch. Sometimes, Low Saxon and Low \ili{Franconian} varieties are grouped together because both are unaffected by the High \ili{German} consonant shift. However, the proportion of the population who can understand and speak it has decreased continuously since World War II.


%% Wikipedia
%% High \ili{German} is divided into Central \ili{German}, High \ili{Franconian} (a transitional dialect), and Upper \ili{German}. Central \ili{German} dialects include \ili{Ripuarian}, Moselle \ili{Franconian}, Rhine \ili{Franconian}, Central Hessian, East Hessian, North Hessian, Thuringian, \ili{Silesian} \ili{German}, Lorraine \ili{Franconian}, \ili{Mittelalemannisch}, North Upper Saxon, High \ili{Prussian}, Lausitzisch-Neumärkisch and Upper Saxon. It is spoken in the southeastern Netherlands, eastern Belgium, Luxembourg, parts of France, and parts of Germany roughly between the River Main and the southern edge of the Lowlands. Modern Standard \ili{German} is mostly based on Central \ili{German}, although the common (but not linguistically correct) \ili{German} term for modern Standard \ili{German} is Hochdeutsch, that is, High \ili{German}.

%% The Moselle \ili{Franconian} varieties spoken in Luxembourg have been officially standardised and institutionalised and are usually considered a separate language known as \ili{Luxembourgish}.

%% The two High \ili{Franconian} dialects are East \ili{Franconian} and South \ili{Franconian}.

%% Upper \ili{German} dialects include \ili{Northern} Austro-Bavarian, Central Austro-Bavarian, Southern Austro-Bavarian, Swabian, East \ili{Franconian}, High \ili{Alemannic} \ili{German}, Highest \ili{Alemannic} \ili{German}, Alsatian and Low \ili{Alemannic} \ili{German}. They are spoken in parts of the Alsace, southern Germany, Liechtenstein, Austria, and the \ili{German}-speaking parts of Switzerland and Italy.

%% Wymysorys is a High \ili{German} dialect of Poland native to Wilamowice, and Sathmarisch and Siebenbürgisch are High \ili{German} dialects of \ili{Romania}. The High \ili{German} varieties spoken by Ashkenazi Jews (mostly in the former \ili{Russian} Empire) have several unique features, and are usually considered as a separate language, \ili{Yiddish}. It is the only \ili{Germanic} language that does not use the \ili{Latin} script as the basis of its standard alphabet.


\subsubsection{Yiddish}

\ili{Yiddish} is one of many languages spoken in the Jewish diaspora. There are no reliable numbers
as far as the number of speakers is concenred: some sources assume that
1.5 million people speak this language actively or passively with just 500.000 speakers using the language
actively in everyday life. Other sources assume that there are 4 million
speakers. In any case, the number of \ili{Yiddish} speakers was much
higher 100 years ago: \citet[\page 6]{Birnbaum1915a-u} estimated the number of speakers at up to 12 million. Most of
them lived in the area of the Soviet Union (over 7 million) and there were larger communities in the United States (over 2 million), Austria-Hungary
(1.5 to 2 million), \ili{Romania} (over 250,000), Great Britain
(250,000), Palestine, Argentina, and Canada (about 100,000 each). See also \citew{Schaefer2023a} for
estimations of numbers of speakers and further references.

%7 million speakers of \ili{Yiddish} lived in Europe, most of them in the area of the Sovjet Union and in
%Austria-Hungary. At most 75,000 \ili{Yiddish} speakers are left in Western Europe. 
\ili{Yiddish} has its roots
in medieval \ili{German} with influences from \ili{Hebrew} and \ili{Aramaic}. It is also influenced by
Roman languages, especially by Old French\il{French!Old} and \ili{Italian} varieties.


\subsubsection{Pennsylvania German}

\ili{Pennsylvania German} (Pensilfaanish, Deitsch), which is also known as Pennsylvania \ili{Dutch}, has about 300,000
native speakers, who mainly live in the USA. The most important regions are Pennsylvania, Ohio, and
Indiana. Pennsylvania \ili{German} is the result of immigration in the 17th and 18th century. Members of
various protestant religions (Mennonites, Pietists and so on) left Europe for religious reasons  
but later immigrants that came for economic reasons followed.  The language is
based on Palatine dialects and is nowadays mainly spoken by Amish and Mennonites.




\subsubsection{Dutch}

\ili{Dutch} (Nederlands) is the  official language of the Netherlands and has about 15 million speakers in that country. \ili{Dutch} is also one of the official languages of Belgium with about 6 million speakers.
%AZleave out: ; almost 4 million Walloones). 
\ili{Dutch} is the sole official language and teaching language in Suriname, 
%AZ leave out: which is independent since 1975, 
% 60% of the population speak it as their mother tongue % wikipedia
% 24% speak it as a second language
and in \href{https://en.wikipedia.org/wiki/Aruba}{Aruba} and the \href{https://en.wikipedia.org/wiki/Netherlands_Antilles}{Netherlands Antilles}.

%In Aruba, Curaçao and Sint Maarten, all parts of the Kingdom of the Netherlands, \ili{Dutch} is the official language but spoken as a first language by only 7% to 8% of the population,[56] although most native-born people on the islands can speak the language since the education system is in \ili{Dutch} at some or all levels.[57] The lingua franca of Aruba, Bonaire and Curaçao is Papiamento, a creole language that originally developed among the slave population. The population of the three northern Antilles, Sint Maarten, Saba, and Sint Eustatius, is predominantly \ili{English}-speaking.[44]

\subsubsection{Afrikaans}

\ili{Afrikaans} has been one of the official languages of South Africa since 1925, which has eleven official languages.
There are about 6.4 million native speakers in South Africa, which is about 15\,\% of the population,
and 150,000 in Namibia. \ili{Afrikaans} has developed since the 17th century from \ili{Dutch} dialects and has
been seen as an independent language since somewhere between 1775 and 1850 
%\parencites[\page 99]{denBesten2012b-u}
\citep[\page 272]{denBesten2012a-u}.
% Oft wird gesagt, dass ab ca. 1775 von '\ili{Afrikaans}' gesprochen werden kann (Den Besten spricht aber noch vom 'Proto-Afrikaans'). Die Übergänge vom Kap-Niederländischen zum \ili{Afrikaans} sind natürlich fließend, aber um die Zeit herum, gibt es doch schon genügend dokumentierte Unterschiede, um zu sagen, dass \ili{Afrikaans} eine eigenständige Sprache geworden ist. Allerspätestens ab Mitte des 19. Jahhunderts ist dann aber auf jeden Fall eine eigene Sprache anzusetzen; da sind sich alle einig. Und 1925 wird das dann ja auch offizielle Amtssprache. 
The various African languages spoken in the area interacted with \ili{Afrikaans}.
%AZ leave out: and its origins and resulted in a structural simplification in comparison to \ili{Dutch}
Today \ili{English} has a strong influence.



\subsubsection{Frisian}

The three varieties of \ili{Frisian} are not mutually intelligible. There is North Frisian\il{Frisian!North} spoken by
about 10,000 speakers mainly on the north \ili{Frisian} islands Amrum, Sylt, and Helgoland. East Frisian\il{Frisian!East}
is extinct with the exception of \ili{Saterlandic}, which is spoken in the three villages of Saterland
(Landkreis Cloppenburg) by between 1,000 and 2,500 people.
% https://www.taz.de/!5834717 Inseln
West Frisian\il{Frisian!West} is spoken in the northern \ili{Dutch} province Fryslân (Friesland) and has about 350,000
native speakers.



\subsubsection{English}

All over the world \ili{English} had about 570 million speakers at the end of the 20th century 
(337 million native speakers, 235 million speakers with \ili{English} as a second language\footnote{%
  The term \emph{Zweitsprache} `second language' is used to refer to speakers using a language in
  everyday life since this language is spoken in the environment in which they live. An example would be speakers with
  Turkish as their first language living in Germany. The \ili{German} tradition distinguishes
  \emph{Zweitsprache} from \emph{Fremdsprache} `foreign language'.%
}). The countries with the most native speakers are listed below: 
\begin{itemize}
\item USA: 227 million,
\item Great Britain: 57 million,
\item Nigeria: 43 million, 
\item Canada: 24 million,
\item Australia: 17 million,
\item Ireland: 3.5 million,
\item New Zealand: 3.2 million.
\end{itemize}
There are many national variants, which differ mostly in pronunciation.
Between 1 and 1.5 billion people have active or passive knowledge of \ili{English}.
\ili{English} is an official language in 59 states. It is the most important scientific language.%
\il{Germanic!West|)}




%      <!-- Local IspellDict: en_US-w_accents -->
