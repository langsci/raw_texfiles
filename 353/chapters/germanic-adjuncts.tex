%% -*- coding:utf-8 -*-

\chapter{Adjuncts}

\if0

\subsection{Adjunkte}

\frame{
\frametitle{Adjunkte}

\begin{itemize}
\item Argumente werden von ihrem Kopf ausgewählt
\pause
\item Adjunkte wählen sich ihren Kopf
\pause
\item Deutsch, Niederländisch, \ldots:\\
      Adjunkte im Satz gehen an irgendeine Verbprojektion (Verb in Endstellung)
\pause
\item \ili{English}, Dänisch, \ldots: Adjunkte gehen an VP (\citet{Wechsler2015a})

\eal
\ex Kim will have been [promptly [removing the evidence]].
\ex Kim will have been [[removing the evidence] promptly].
\zl
\pause
\item Wie das \spr- oder \comps-Merkmal gibt es auch ein \modm.\\
      Wert von {\sc mod} ist eine Beschreibung des zu modifizierenden Kopfes:
\begin{itemize}
\item Deutsch: {\sc mod} V[{\sc ini}$-$]
\item Englisch: {\sc mod} VP
\end{itemize}

\end{itemize}

}

\frame{
\frametitle{Freie Position der Adjunkte im Deutschen}



\scalebox{0.9}{\begin{tikzpicture}
\tikzset{level 1+/.style={level distance=3\baselineskip}}
\tikzset{frontier/.style={distance from root=14\baselineskip}}
\Tree[.{V[\spr \eliste, \comps \eliste]}
        [.Adv morgen ]
        [.{V[\spr \eliste, \comps \eliste]}
          [.{NP[\type{nom}]} {jeder} ]
          [.V\feattab{
              \spr \sliste{ }, \comps \sliste{ NP[\type{nom}] } }
            [.{NP[\type{acc}]} \edge[roof]; {das Buch} ] 
            [.V\feattab{
              \spr \sliste{  },\\
              \comps \sliste{ NP[\type{nom}], NP[\type{acc}]}} liest ] ]
] ]
\end{tikzpicture}}

{}[dass] morgen jeder das Buch liest


}


\frame{
\frametitle{Freie Position der Adjunkte im Deutschen}



\scalebox{0.9}{\begin{tikzpicture}
\tikzset{level 1+/.style={level distance=3\baselineskip}}
\tikzset{frontier/.style={distance from root=14\baselineskip}}
\Tree[.{V[\spr \eliste, \comps \eliste]}
          [.{NP[\type{nom}]} {jeder} ]
          [.V\feattab{
              \spr \sliste{ }, \comps \sliste{ NP[\type{nom}] } }
            [.Adv morgen ]
            [.V\feattab{
                \spr \sliste{ }, \comps \sliste{ NP[\type{nom}] } }
              [.{NP[\type{acc}]} \edge[roof]; {das Buch} ] 
              [.V\feattab{
                \spr \sliste{  },\\
                \comps \sliste{ NP[\type{nom}], NP[\type{acc}]}} liest ] ]
] ]
\end{tikzpicture}}

{}[dass] jeder morgen das Buch liest


}



\frame{
\frametitle{Freie Position der Adjunkte im Deutschen}



\scalebox{0.9}{\begin{tikzpicture}
\tikzset{level 1+/.style={level distance=3\baselineskip}}
\tikzset{level 3+/.style={level distance=4\baselineskip}}
\tikzset{frontier/.style={distance from root=14\baselineskip}}
\Tree[.{V[\spr \eliste, \comps \eliste]}
          [.{NP[\type{nom}]} {jeder} ]
            [.V\feattab{
                \spr \sliste{ }, \comps \sliste{ NP[\type{nom}] } }
              [.{NP[\type{acc}]} \edge[roof]; {das Buch} ] 
              [.V\feattab{
                \spr \sliste{  },\\
                \comps \sliste{ NP[\type{nom}], NP[\type{acc}]}} 
                [.Adv morgen ]
                [.V\feattab{
                  \spr \sliste{  },\\
                  \comps \sliste{ NP[\type{nom}], NP[\type{acc}]}} liest ] ] ] ]
\end{tikzpicture}}

{}[dass] jeder das Buch morgen liest


}

\frame{
\frametitle{Feste Position der Adjunkte im Englischen}



\scalebox{0.9}{\begin{tikzpicture}
\tikzset{level 1+/.style={level distance=3\baselineskip}}
\tikzset{frontier/.style={distance from root=14\baselineskip}}
\Tree[.{V[\spr \eliste, \comps \eliste]}
          [.{NP[\type{nom}]} {Peter} ]
          [.V\feattab{
              \spr \sliste{ NP[\type{nom}] }, \comps \sliste{  } }
            [.Adv often ]
            [.V\feattab{
                \spr \sliste{ NP[\type{nom}] }, \comps \sliste{  } }
              [.V\feattab{
                \spr \sliste{ NP[\type{nom}] },\\
                \comps \sliste{  NP[\type{acc}]}} reads ]
              [.{NP[\type{acc}]} books ] 
               ]
] ]
\end{tikzpicture}}


}

\frame{
\frametitle{Feste Position der Adjunkte im Englischen}



\scalebox{0.9}{\begin{tikzpicture}
\tikzset{level 1+/.style={level distance=3\baselineskip}}
\tikzset{frontier/.style={distance from root=14\baselineskip}}
\Tree[.{V[\spr \eliste, \comps \eliste]}
          [.{NP[\type{nom}]} {Peter} ]
          [.V\feattab{
              \spr \sliste{ NP[\type{nom}] }, \comps \sliste{  } }
            [.V\feattab{
                \spr \sliste{ NP[\type{nom}] }, \comps \sliste{  } }
              [.V\feattab{
                \spr \sliste{ NP[\type{nom}] },\\
                \comps \sliste{  NP[\type{acc}]}} reads ]
              [.{NP[\type{acc}]} books ] 
               ]
            [.Adv often ]
] ]
\end{tikzpicture}}


}

\fi
