\documentclass[output=paper,colorlinks,citecolor=brown]{langscibook}
\ChapterDOI{10.5281/zenodo.14922293}

\author{Sharon O'Brien\affiliation{都柏林城市大学}}
\title{如何处理机器翻译的错误:译后编辑}
\abstract{机器译文可能出现需要修改的错误,尤其是如果译文用于发布或绝对不能出错。识别并修改这些错误的工作就称为“译后编辑”(post-editing,PE)。我会在本章中借助学术界和业界资料,概述译后编辑的过程,还会解释它大致分为“轻度”译后编辑和“完整”译后编辑,并分别介绍对应的操作指南,同时还会关注应用中出现的问题。本章还概述了译后编辑使用的各类界面(包括文字处理软件、电子表格软件和专业的计算机辅助翻译工具),以及交互模式(传统、自适应或交互式)。最后介绍译后编辑研究者所使用的概念和工具,并重点关注如何测量时间、技术和认知努力。}

\IfFileExists{../localcommands.tex}{%hack to check whether this is being compiled as part of a collection or standalone
   \usepackage{langsci-optional}
\usepackage{langsci-gb4e}
\usepackage{langsci-lgr}

\usepackage{listings}
\lstset{basicstyle=\ttfamily,tabsize=2,breaklines=true}

%added by author
% \usepackage{tipa}
\usepackage{multirow}
\graphicspath{{figures/}}
\usepackage{langsci-branding}

   
\newcommand{\sent}{\enumsentence}
\newcommand{\sents}{\eenumsentence}
\let\citeasnoun\citet

\renewcommand{\lsCoverTitleFont}[1]{\sffamily\addfontfeatures{Scale=MatchUppercase}\fontsize{44pt}{16mm}\selectfont #1}
  
   %% hyphenation points for line breaks
%% Normally, automatic hyphenation in LaTeX is very good
%% If a word is mis-hyphenated, add it to this file
%%
%% add information to TeX file before \begin{document} with:
%% %% hyphenation points for line breaks
%% Normally, automatic hyphenation in LaTeX is very good
%% If a word is mis-hyphenated, add it to this file
%%
%% add information to TeX file before \begin{document} with:
%% %% hyphenation points for line breaks
%% Normally, automatic hyphenation in LaTeX is very good
%% If a word is mis-hyphenated, add it to this file
%%
%% add information to TeX file before \begin{document} with:
%% \include{localhyphenation}
\hyphenation{
affri-ca-te
affri-ca-tes
an-no-tated
com-ple-ments
com-po-si-tio-na-li-ty
non-com-po-si-tio-na-li-ty
Gon-zá-lez
out-side
Ri-chárd
se-man-tics
STREU-SLE
Tie-de-mann
}
\hyphenation{
affri-ca-te
affri-ca-tes
an-no-tated
com-ple-ments
com-po-si-tio-na-li-ty
non-com-po-si-tio-na-li-ty
Gon-zá-lez
out-side
Ri-chárd
se-man-tics
STREU-SLE
Tie-de-mann
}
\hyphenation{
affri-ca-te
affri-ca-tes
an-no-tated
com-ple-ments
com-po-si-tio-na-li-ty
non-com-po-si-tio-na-li-ty
Gon-zá-lez
out-side
Ri-chárd
se-man-tics
STREU-SLE
Tie-de-mann
}
    \bibliography{../localbibliography}
    \togglepaper[5]
}{}

\begin{document}
\maketitle


\section{定义}

机器翻译技术还不完善,有时能产出意思准确、符合语境的译文,有时却出现误译、漏译、增译或文体不当等严重错误。如果使用机器翻译只为了理解文本大意,或许无需修改这些错误。但是,如果要使用机器翻译来生成可供出版或在组织内外广泛传播的译文,通常则需要修改文中的错误。这种错误的识别和改正被称为“译后编辑”(post-editing)。在机器翻译系统开发的早期阶段,就已经有人开始使用这个术语。比起现在,当时的技术发展更落后,肯定也也不能随用随取,即时输出译文。文本先以电子版发送到机器翻译系统进行翻译,然后将译文返回给发送者,再由发送者对自动翻译好的文本进行编辑。“译后编辑”这一术语一直沿用至今,甚至也用来指代修改机器译文错误的过程(参见 \sectref{sec:obrien:3.1})。

译后编辑是双语处理任务,通常由经验丰富的专业译者来完成。担任此工作的人常被称为“译后编辑者”(post-editor)。然而, 这个职业名词还存在争议,尤其是因为专业译者的日常工作也会使用计算机辅助翻译工具,这类工具结合了翻译记忆库、术语管理和机器翻译等技术(参见下文的\citetv[§4]{chapters/kenny} 和 \sectref{sec:obrien:3})。采用这种工作模式的译者可能会先处理某个句子的模糊匹配,然后翻译下一句,再接着对下一句进行译后编辑。在这个过程中,译者的角色是不是从审校,变成译者,再变成了译后编辑者?其实不然!因为这项工作从根本上来说,仍然是翻译和审校。主要区别在于译者在处理不同的句子时,使用了不同的技术支持和输入来源。

进行译后编辑时,译者必须同时理解原文和机器译文,然后识别错误,思考并制定校正策略,最后进行修改。从根本上说,译后编辑就是审校工作。

机器译文可能会出现各类错误,如语法错误、句法错误、非必要增译漏译、词汇错误、术语不符、搭配错误或风格不当等。机器译文的错误类型和数量受很多因素影响,如语言对和翻译方向、内容类型以及机器翻译引擎的训练数据或技术。

以下以英文的机器译文及其修改版本为例介绍几种错误类型。

\ea
语法错误
\ea The cat is very protective of her kittens. She scratches anyone \textit{which} tries to touch them.
\ex The cat is very protective of her kittens. She scratches anyone \textit{who} tries to touch them.
\z
\ex
词汇错误
\ea The cat is very protective of her \textit{pups}. She scratches anyone who tries to touch them.
\ex The cat is very protective of her \textit{kittens}. She scratches anyone who tries to touch them.
\z
\ex
句法错误
\ea The new-born cygnets on the lake \textit{swam}.
\ex The new-born cygnets \textit{swam} on the lake.
\z
\ex
搭配错误
\ea The house had no \textit{flowing water}.
\ex The house had no \textit{running water}.
\z
\z

\section{译后编辑的不同程度和指南}\label{sec:obrien:2}

机器翻译的主要目标之一是,尽可能在提高效率和降低成本的同时,实现更多语言之间的信息转换。以这些目标为导向,译后编辑主要分为“轻度译后编辑”或“完整译后编辑”。

进行轻度译后编辑只需修改关键错误,追求速度;而完整译后编辑则需要修改机器译文的所有错误,比轻度译后编辑更耗时。但无论是哪种程度的译后编辑翻译,译后编辑的译文产出速度都应该比不借助计算机辅助翻译工具的翻译速度更快。国际标准化组织(ISO)制定了一份译后编辑标准\citep{ISO2017},编号为ISO18857:2017。根据该标准,轻度译后编辑被定义为“只需产出意思正确的文本,无需追求可比拟人工翻译的译后编辑过程”(第2页)。完整译后编辑被定义为“以产出可比拟人工翻译的译文为目标的译后编辑过程”(同上)。

\begin{sloppypar}
然而,从概念上讲,这些定义是有问题的。首先,除了以上非常笼统的描述方式,我们很难明确两者的区别。到底什么是“关键”错误?这个问题视需求因人而异。“意思正确”是什么意思,如何衡量?轻度译后编辑真的能够不以人工翻译的质量为对标吗?此外,很难算清深度译后编辑要比轻度译后编辑所需时间差值。这些问题促使翻译行业更规范地描述不同程度的译后编辑。例如,翻译自动化用户协会(TAUS)制定了操作指南,提出完整译后编辑包括文体层面的修改,而轻度译后编辑则不然。
\end{sloppypar}

\tabref{tab:obrien:1}列出了翻译自动化用户协会关于轻度和完整译后编辑的操作指南\citep{TAUS2010},其中并列内容可进行比较,空白单元格表示该套指南中无可比较内容:


\begin{table}[t]
\begin{tabularx}{\textwidth}{QQ}
\lsptoprule
轻度译后编辑 & 完整译后编辑\\
\midrule
确保译文的语义正确。& 确保译文的语法、句法和语义正确。\\
\tablevspace
& 确保正确翻译关键术语,且未翻译的术语包含在客户的“无需翻译”(Do Not Translate)术语列表中。\\
\tablevspace
确保无增译或漏译信息。& 确保无增译或漏译信息。
 \\
\tablevspace
修改所有含冒犯之意或文化不容的不当内容。& 修改所有含冒犯之意或文化不容的不当内容。
\\
\tablevspace
尽量保留机器译文。 & 尽量保留机器译文。\\
\tablevspace
遵从基本的拼写规则。& 遵从基本的拼写、标点符号和连字符规则。
\\
\tablevspace
无需修改文体问题。& \\
\tablevspace
无需仅为了提高文本流畅度而重组句子。& \\
\tablevspace
& 确保格式正确。\\
\lspbottomrule
\end{tabularx}
\caption{TAUS译后编辑操作指南\label{tab:obrien:1}}
\end{table}

根据ISO18857号标准(第6页),译后编辑的目标是确保:

\begin{itemize}
\item 经译后编辑的译文可以理解;
\item 源语和目标语的内容相符合;
\item 遵从TSP对译后编辑的要求。
\end{itemize}

TSP指“翻译服务提供商”(translation service provider),意为“提供翻译服务的语言服务提供商”(第4页)。

要实现这些目标,译后编辑需遵从以下标准(第6页):
\begin{itemize}
\item 术语/词汇必须保持一致,且符合专业领域的术语表达;
\item 使用目标语的标准句法、拼写、标点符号、音调符号、特殊符号和缩写等拼写惯例;
\item 遵守任何适用标准;
\item 格式正确;
\item 适用于目标受众和目标语内容;
\item 遵从客户与TSP之间的协议。
\end{itemize}

这两套指南虽然侧重点不同,但有部分内容重叠,共同组成了译后编辑的常用指南。TAUS指南鼓励尽量保留机器译文,而ISO指南更关注协议、标准和对目标受众的适用性。尽量保留机器译文是译后编辑的重要方面。译者往往很容易忽视机器译文,直接弃用并重新翻译。其实,很多译者都有这么做的倾向,认为自己自己翻得更好更快,而且比译后编辑更高效。虽然在“更好”的问题上,过去机器译文质量的确不尽如人意,但神经机器翻译的发展已经大大提高了机器翻译的整体质量,使其更加便于使用。而在“更快”的问题上,人工翻译是否比译后编辑用时更少,仍未有定音。即使译者持相反看法,但有研究表明译后编辑肯定比人工翻译更快(如\citealt{GuerberofArenas2014})。我们可以通过实践找到如何快速评测机器译文质量,从而决定其可用性和所需修改的方法。

从概念上讲,译后编辑程度与质量水平相关联,尽管下文例子说明这种联系存在问题。经过轻度译后编辑的译文“质量恰好”或“可以理解”,也就是说,要准确传达原义,但不要求通顺流畅或文体使用考究。另一方面,经过完整译后编辑的译文“质量堪比人工翻译”。但问题又来了,这种观点的内在逻辑是人工翻译的质量总是很高,可坦白讲,实际情况并非如此。

译后编辑的程度和译文质量水平的关系有时关系复杂,让人捉摸不透,为了便于理解,我们来看个例子:
\ea
\ea In a new report on the quality of teaching practice, inspectors said pre-school teachers could be \textit{training} to integrate languages such as French, German and Polish in early \textit{learn} settings.
\ex In a new report on the quality of teaching practice, inspectors said pre-school teachers could be \textit{trained} to integrate languages such as French, German and Polish in early \textit{learning} settings.
\z
\z

快速改正例句(a)的两处错误后得到例句(b)。一方面,我们可以说对例句(a)进行了“轻度”编辑,因为只涉及两处快速修改。但如果根据轻度译后编辑指南的建议,我们无需进行任何改动。另一方面,按照完整译后编辑指南的要求,为了确保译文的语义、句法和语法正确,这两处错误必须改正。这么说来,这两处简单的错误修改,究竟是轻度译后编辑,还是完整译后编辑呢?

再看另一个例子。

\ea
\ea In addition, it \textit{reiterates the instinct} of modern \textit{language} into the primary school curriculum \textit{as a fitting step}.
\ex In addition, it says \textit{re-instating} modern \textit{languages} to the primary school curriculum \textit{would be a timely move}.\footnote{\url{https://www.irishtimes.com/news/education/foreign-languages-could-be-taught-in-preschool-and-primary-department-1.4270886}}
\z
\z

例句(a)表意模糊,需要进行大幅修改才能至少准确表达原义。从根本上说,译后编辑的程度视机器译文(即机器翻译系统产出的每句译文质量)和目标质量而定。译后编辑者会根据每句译文和目标质量,在简单编辑和深入编辑之间切换。无论每一句质量如何如何,最重要的是终稿的质量一致,且符合翻译委托人和最终用户的期望。

轻度译后编辑和完整译后编辑的最后一个概念问题是,专业译者通常不愿意只提供“易于理解”的译文质量,也没有翻译专员愿意承认自己只会选择轻度译后编辑。

对这些概念及其指南的了解非常重要。显然,这些方面都还面临一些挑战。

\section{译后编辑的操作界面}\label{sec:obrien:3}

基本上,只要能在阅读原文的同时编辑机器译文,译后编辑就可以借助任何文本编辑器完成,甚至只需通过能并排显示原文和译文的电子表格就能进行编辑。不过目前专业译者常用计算机辅助翻译(CAT)工具,尤其是翻译记忆库(TM)。\textcitetv[§4]{chapters/kenny}指出,翻译记忆库是个存储已翻译过的文本句段的数据库。翻译记忆库软件是用于访问、编辑和更新该数据库中文本的软件应用程序。从本书其他章节可以看出,翻译技术和机器翻译显然是不同的技术,尽管二者关系密切,因为当代数据驱动的机器翻译系统常用存储于翻译记忆库的数据作为机器学习的重要输入。此外,由于译者普遍使用翻译记忆库,机器翻译技术就算还没完全嵌入,也已经与很多翻译记忆库工具相关联,如Trados Studio、MQM、MateCat等等。从实践的角度来看,这意味着译后编辑工作常常在翻译记忆库编辑环境中完成。

机器翻译可以用不同方式嵌入翻译记忆库环境。例如,如果记忆库中没有可用的匹配句段,那么译者就可以调用机器翻译系统。或者,译者可以自定义翻译记忆库设置,在在特定条件下(如无法在记忆库中找到匹配句段),则自动显示机器译文。

大家普遍认为,如果翻译记忆库维护良好并持续更新,其数据库提供的精确匹配比机器翻译系统对译者更有价值,因为后者可能包含错误,而前者(理想情况下)则不然。同理,以往的业内共识是,相似度为75\%以上的翻译记忆库“模糊匹配”要优于机器翻译。然而,近来的机器翻译技术更加先进,生成的译文可用性甚至大于相似度超过75\%的翻译记忆模糊匹配,更加便于译者使用。因此,译者可以根据翻译记忆库、机器翻译系统、语言对和文本主题来自定义翻译记忆库的软件设置,以便同时看到来自翻译记忆库的匹配结果和机器译文。其实,译者还可以同时获得多个机器翻译系统的译文,择优选用,然后根据需要进行编辑。

结合使用翻译记忆库和机器翻译的好处是译者在翻译时可以获得更多的选择和帮助。但是,这种工作模式也不乏挑战。这样的设置将大量信息呈现给译者,意味着他们必须处理所有这些信息(记忆库匹配及其匹配度、机器翻译建议,可能还有术语和元数据,如记忆库匹配的创建者和创建时间等(参见\citealt{TeixeiraOBrien2017}),同时从语言的角度决定使用哪个匹配以及需要如何修改。这可能导致很高甚至超负荷的认知需求,或许正因如此,才会有译者表示译后编辑比其他形式的翻译和修改工作的要求更高且更耗费精力。

这种混合型交互界面带来的另一个挑战就是前文提到的译后编辑指南。译者在处理机器译文时需要牢记这些指南,但如果处理的是记忆库的匹配结果,那么他们编辑的则是(与机器翻译系统相对的)人工翻译,因此不能算作真正的“译后编辑”。这种结合了译后编辑、审校和翻译的工作模式可能会使翻译任务变得相当复杂,更别提译者总是顶着时间压力执行翻译任务!

\subsection{传统、自适应和交互式译后编辑}\label{sec:obrien:3.1}

在上文中,我们已经概述了译后编辑的常用交互界面,在传统翻译工作环境中,译者使用翻译记忆工具的惯用翻译设置,机器翻译匹配静态显示,供选择编辑或弃用,以支持翻译记忆库匹配或原文句子的完整翻译。与所有技术一样,新发明的出现——自适应机器翻译和交互式机器翻译——使传统的默认设置有了变化。

“自适应MT”(Adaptive MT)是翻译记忆库/机器翻译集成者开发的一项功能。有了这项功能,机器翻译系统可以实时学习译者的编辑结果。这解决了一个曾让译者为之苦恼的机器翻译缺陷,即系统无法学习译者已经修改过的机器译文错误。因此,如果同一句话再次出现,译后编辑者必须再修改一次。结合翻译记忆库能在一定程度上解决这个问题——在记忆库环境中修改的某个句子会被保存到记忆库的数据库中。如果同一句话再次出现,译者看到的会是这句话在翻译记忆库的数据库中的精确匹配。但是,如果没有使用精确匹配的数据库的翻译记忆库环境,那么机器翻译系统就会出现同样的错误。使用自适应机器翻译系统能“学习”修改结果,并且理论上应该能自行适应译者的修改,以确保不再重蹈覆辙。Trados Studio这款工具便很好地发挥了这项功能。

“交互式机器翻译”(Interactive MT)可以看作是自适应机器翻译的特殊形式。这种形式与机器翻译的默认结合方式如上所述——机器翻译系统先预翻译整个句段,然后完整呈现给译者,而交互式机器翻译则会根据译者所选的每个词或短语调整后续输出。当译者接受或确认一个词时,机器翻译系统会即时调整输出。这种与机器翻译输出交互的方式独具特色,不过,它类似许多人已经习以为常的预测文本的概念。Lilt平台便以交互式机器翻译为主要特点。其实,Lilt的定位就是自适应和交互式机器翻译系统界面。\footnote{\url{https://lilt.com},最后访问于2022年6月} (其他交互式机器翻译的例子可参见 \citealt{TorregrosaRivero2018}。)

交互式机器翻译令人对“译后编辑”提出了质疑。如前所述,“译后编辑”一词出现在几十年前,当时是机器翻译系统翻译整篇文本,译者进行编辑,再返回给翻译委托人/客户。但是,在交互模式下,机器翻译随着译者当下的决定实时产出译文。因此,编辑似乎并不发生在翻译之“后”,称之为“交互式机器翻译”更加准确。和许多领域一样,“译后编辑”一词已普遍为人接受,所以不会短时间内被弃用,但或许会逐渐消失。而与机器译文交互并修改其错误这项工作在不久的将来消失的可能性很小。

\subsection{研究界面}

上文介绍了不同的译后编辑主流界面以及交互模式。还有很多其他可供使用的界面,但功能基本一致。由于译后编辑对专业译者来说是一项相对较新的任务,因此相关学术研究方兴未艾。研究主题涉及译后编辑的常见类型、是否比翻译省时、产品质量及其认知过程。

为了收集以上主题的数据,有研究人员自行开发了译后编辑界面。这样做主要是因为,对于研究项目来说,商业工具可能过于昂贵且功能相对复杂,有时反而阻碍研究目标的实现,又或者这些工具不利于控制实验条件。我们要介绍的两款为译后编辑实验开发的研究工具分别是Casmacat和Translog。Casmacat是一项欧盟资助研究项目的成果。\footnote{受到欧盟第七框架计划项目287576(ICT-2011.4.2, Seventh Framework Programme Project)的共同资助-\url{http://www.casmacat.eu}} 该项目旨在使用交互式自适应机器翻译,搭建下一代译者的工作平台,并建立交互式机器翻译的认知过程模型。Translog最初由哥本哈根商学院为研究翻译过程而开发。Translog II为最新版本,支持对翻译和译后编辑的研究。\footnote{\url{https://sites.google.com/site/centretranslationinnovation/translog-ii?authuser=0}} 这两款工具都可以结合其他有利于研究的技术,如键盘记录(记录键盘使用)和眼动追踪(记录翻译或译后编辑过程中的眼球运动和认知负荷)。这些工具的使用有助于深入理解译后编辑这项任务。不过,这些工具虽然适用于科研目的,但其功能与标准的商业级翻译记忆库工具一比较就相形见绌了。因此,它们无法完整呈现真实的编辑工作过程。


\section{译后编辑工作量的测量}

技术取得进展后,会从实验室走向公共领域,以检验其有用性。最初,新技术可能会是破坏因素——多年来使用和所接受的流程被打乱,因此引发质疑、担忧或恼怒,这是可以理解的。当习惯了某种工作方式,我们会发现很难快速接受新的方式。而且,如果新技术带来新问题,我们就不太可能完全接受它。从职业角度来看,机器翻译和译后编辑便是如此。因此,人们对译后编辑工作量的研究热度大增,以证实或反驳“译后编辑比人工翻译更高效”的说法。

约十五年以来,致力于探讨这一问题\citep{Koponen2016}的学术研究数不胜数,在此无法一一列举。但是,有一篇开创性文章非常值得一提,因为它为译后编辑工作量的测量设定了标准。该文章作者是汉斯-彼得·克林斯(Hans-Peter Krings),原文为德语,其英文版于2001年出版\citep{Krings2001}。克林斯对译后编辑和人工翻译的工作量进行了对比研究。他在进行这项研究时,机器译文的质量远比现在的差得多,而且当时的实验环境非常简易——实验任务用纸笔完成,克林斯用摄像机记录整个过程。不过,克林斯这项研究的重要性并不体现在实验条件或研究发现,而在于他提出了包含三个维度的译后编辑工作量衡量法——时间、技术和认知。

译后编辑工作量的测量通常只关注时间维度,即这项任务对比其他任务所花费的时长。时间维度当然是最重要的维度之一,尤其是在商业环境中。而且时间维度也相对容易测量,因此当 商业机构在测量译后编辑工作量时主要考量时间维度。

然而,克林斯的研究表明,还需要考虑其他两个维度。技术维度测量键盘和鼠标的操作,即删除或添加的词或字符数量,选择的短语数量,文本剪切和粘贴的词数等。前文提到的Translog就是支持键盘日志记录的工具。译后编辑就是文本修订,因此了解技术操作的工作量非常重要。不仅如此,译后编辑这种审校工作模式涉及删除、重新键入、复制和粘贴等操作,需要大量使用键盘和鼠标,使身体产生疲惫感,甚至导致手和手腕劳损。另一方面,如果机器译文质量相对较好,就可以减少译者的打字量。

除了键盘记录,技术维度也可以通过“编辑距离”(edit distance)衡量。简单来说,编辑距离计算将文本的字符串变成另一个字符串所需的最少操作步数。这种“操作”包括词的删除、插入或位置移动。编辑距离的测量指标分为几种,计算操作步数的方法略有不同。基础的测量指标之一是“莱文斯坦距离”(Levenshtein distance),计算不同单词、短语或句子之间转换所需的字符插入、删除或替换的最少次数。

以单词drink为例,要改变多少个字符才能将其变成drunk?答案是一个——只需将i替换成u。举个更复杂一点的例子,把短语He drinks变成He is drinking的莱文斯坦距离是6(在He后插入字母i和s,以及一个空格符,然后将drinks结尾的s换成i,并插入n和g)\footnote{可以使用在线计算工具,如\url{https://planetcalc.com/1721/}。}我们还可以采用更复杂的编辑距离指标,其中常用于测量译后编辑的编辑距离的指标是“翻译编辑率”(Translation Edit Rate, TER)\citep{snover-etal-2006-study}。该指标的结果范围是0-1或0\%到100\%,分数越低,则译后编辑工作量就越少,如结果为30\%表示需要对机器译文内容的30\%进行编辑才能获得译后编辑版本对应的字符串。如何更好地计算编辑距离仍面临挑战,因此存在几种不同的方法,经常使用不同的指标。

时间和技术这两个维度相对容易测量,而第三个维度——认知努力——的测量要复杂得多。认知努力是隐藏的认知过程,包括阅读、理解、比较原文和机器译文的意思、决策,同时还要考虑编辑指南和客户的期待,边修改边在编辑的同时监控文本质量。这些过程发生在大脑中,无法直接观察或测量。尽管如此,认知努力仍然是需要考虑的重要方面。有人说,译后编辑有时比人工翻译的认知负荷更高。这可能是因为译后编辑要顾及上述提及的一系列认知过程,且有些人对此还不太熟悉。即便译者借助机器翻译可以提高工作效率,但译者可能会觉得译后编辑比手动翻译更让人疲惫。提高工作效率虽然符合商业生产逻辑,但是造成译者过度疲劳的话未必是好事。正因如此,为什么在测量译后编辑工作量时,认知努力是重要的考虑因素。

但是,如何测量认知努力呢?其实,这是所有希望量化认知努力的人都想回答的问题。有时,认知努力可以通过执行者在完成任务时的“有声思维”来估算。这种方法能让执行者指出自己的认知困难。当然,在过程中进行有声思维会对任务造成干扰且降低速度,这也是该方法的缺点。第二种方法是对任务过程进行电脑录屏,待完成后让执行者边观看回放视频,边回想当时遇到的问题。这样做的好处是不会影响完成速度,但缺点是执行者可能想不起所有问题。第三种方法是研究者试图采用眼动追踪(eye tracking)技术来测量译后编辑的认知努力。这项技术用于记录眼睛注视屏幕的位置,并记录眼睛在文本上停留的时间(称为“注视时间”),甚至测量“瞳孔放大”(即测量瞳孔大小)。这几种都是公认的认知努力测量的有效方法。但是,这些方法也面临许多挑战,如眼动设备花费昂贵,具体操作和数据分析复杂;还要控制好数据收集环境,避免受试者过度频繁地移动头部,或者确保光线不会有大幅变化,以免引起瞳孔大小改变等。认知努力的测量困难重重,因此不难理解为什么很少有人从这个维度来测量译后编辑的工作量。尽管如此,认识到认知努力对译后编辑工作必不可少。

关于译后编辑工作量,还有最后一点要补充。译后编辑工作量可以间接体现某个机器翻译系统针对某个语言对或主题而产出的译文质量。因此,译后编辑工作量可以用作机器翻译质量测评。机器翻译系统的质量越低,则译后编辑所需操作和时间就越多。机器翻译质量还可以用其他方法来测量,如通过识别、分类和计算错误数量,这些方法虽然行之有效,但译后编辑工作量能提供的信息或许更多,因为它反映了想要在机器翻译的基础上获得达到质量要求的译文的难易程度。

\section{译后编辑者的资质与培训}

优秀的译后编辑者应该具备什么特质?应为其提供哪些培训?随着机器翻译逐渐发展成主流技术,这两个问题一直困扰着语言行业和学术界(参见\citealt{NitzkeHansen-Schirra2021})。

关于第一个问题,普遍的建议是,优秀的译后编辑者首先应该是优秀的译者。我们凭直觉就知道,有些人做得好翻译,但做不好审校,反之亦然。同理可得,并非人人都能胜任译后编辑的工作。

但怎样才能称作优秀的译后编辑者?这个问题已经引起了一些关注。例如,\citet{deAlmeidaO’Brien2010}认为优秀的译后编辑者应该能够:

\begin{enumerate}
\item 识别并修改机器译文的问题;
\item 以合理的速度进行后期编辑的能力,以满足这类工作的日产量要求;
\item 遵从指南准则,尽量减少“偏好性”或严格来说不必要的修改,且修改要保持在正常的译后编辑范围内。
\end{enumerate}

具体来看,第(1)点要求译后编辑者首先必须掌握翻译技能。第(2)点和第(3)点表明,译后编辑者需要能够高效工作并遵从指南准则,避免过度编辑。归根结底,能否成为“优秀的”译后编辑者与其对机器翻译技术的态度密切相关(深入讨论可参见\citealt{GuerberofArenas2013})。如果对机器翻译技术很反感,那译者可能会删除或直接忽视机器译文。这反过来会导致翻译任务更加耗时,对委托人造成更高的成本投入。这样的译者就不能算是“优秀的”译后编辑者。不过,这在很大程度上也要取决于具体情况和机器翻译质量。

现在我们来讨论第二个问题。在过去十年里,人们越来越重视培训工作。 随着机器翻译逐步融入其他计算机辅助翻译工具和翻译流程,专业译者需要继续接受职业发展培训,如机器翻译和译后编辑的专题工作坊。机器翻译和译后编辑已被纳入大学的翻译教学计划,但教学方法各有不同。有些大学单独开设译后编辑课,有些大学将译后编辑嵌入审校课程,还有些大学则是将其并入翻译技术课 (参见 \citealt{OBrienVázquez2019})。

培训的核心重点都是确保翻译专业学生了解机器翻译的最新方法、优势和局限、如何评测其质量以及如何进行译后编辑。无论是对于翻译专业学生,还是对于没有接受过翻译培训的人,培训的重点都在于了解何时以及如何使用机器翻译(关于“机器翻译素养”的讨论可参见\citet{BowkerCiro2019})。

\sloppy
\printbibliography[heading=subbibliography,notkeyword=this]
\end{document}
