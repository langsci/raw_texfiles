\documentclass[output=paper]{langscibook}
\ChapterDOI{10.5281/zenodo.14922285}
\author{Olga Torres-Hostench\affiliation{巴塞罗那自治大学}}
\title{欧洲、多语制与机器翻译}
\abstract{‌本章将介绍作为欧盟基本原则的多语制,说明如何将其付诸实践,以及如何通过语言学习和翻译来支持多语制。作者以高等院校为例,认为机器翻译能促进多语制在高校的实践。}


\begin{document}
\AffiliationsWithoutIndexing{}
\maketitle

\section{前言}\largerpage
‌欧盟以“和而不同”(united in diversity)为创立理念,寓意“语言多样性和语言学习对欧洲项目具有重要意义”\citep{EuropeanCommission2021}。‌但是,欧盟的多语制政策大多基于语言学习和人员流动,两者都相当耗时。‌而且人类的语言学习颇具挑战性。毕竟,欧盟普通公民能够学习的语言数量有限。本章将为解决以上问题提供建议,指出机器翻译有助于促进欧洲的多语制,从而促进欧洲的语言多样性。

\section{‌多语并用的欧盟}
\begin{quote}
‌欧盟所谓基于人道主义的多语制不过只是幻想。对此,大家都心照不宣。\citep[561]{House2003}
\end{quote}

‌ISO639-3是由国际标准化组织( International Organization for
Standardization,ISO)开发的一组代码,为所有已知人类语言都分配了由三个字母组成的识别码。‌截至2020年1月30日,该标准收录了7868种语言\citep{Wikizero2020},其中约600种在欧洲使用,24种为欧盟官方语言。欧盟官方语言包括:‌荷兰语、法语、德语、意大利语(始于1958年)、丹麦语、英语(始于1973年)、希腊语(始于1981年)、葡萄牙语、西班牙语(始于1986年)、芬兰语、瑞典语(始于1995年)、捷克语、爱沙尼亚语、匈牙利语、拉脱维亚语、立陶宛语、马耳他语、波兰语、斯洛伐克语、斯洛文尼亚语(始于2004年)、保加利亚语、爱尔兰语、罗马尼亚语(始于2007年)和克罗地亚语(始于2013年)。

‌语言多样性是欧洲文化遗产的一部分。‌在欧洲,有些语言具有国家级官方地位,而有些则是区域性土著语言和/或少数民族语言,受认可程度各有不同。‌《欧洲区域或少数民族语言宪章》(以下简称《宪章》)是于1998年通过的欧洲条约,旨在保护和促进传统少数民族使用的语言。‌2019年,《宪章》在监管机制指导下进行了改革与加强。《宪章》涵盖201个少数民族或语言群体的79种语言\citep{CoE2020}。‌这79种语言如表‌1‌所示\tabref{tab:torres:1}‌。

\begin{table}
\begin{tabularx}{\textwidth}{l}
\lsptoprule
\parbox{\textwidth}{
\begin{multicols}{4}
阿尔巴尼亚语\\
阿拉贡语\\
阿兰语\\
亚美尼亚语\\
亚述语\\
阿斯图里亚斯语\\
巴斯克语\\
贝亚斯语\\
白俄罗斯语\\
波斯尼亚语\\
保加利亚语\\
布涅瓦茨语\\
加泰罗尼亚语\\
康沃尔语\\
克里米亚鞑靼语\\
克罗地亚语\\
塞浦路斯马龙派\hspace*{2 mm}阿拉伯语\\
捷克语\\
丹麦语\\
芬兰语\\
法兰克-普罗旺斯语\\
法语\\
弗里斯兰语\\
加告兹语\\
加利西亚语\\
德语\\
希腊语\\
匈牙利语\\
伊纳里-萨米语\\
爱尔兰语\\
伊斯特拉-罗马尼亚语\\
意大利语\\
卡拉伊姆语\\
卡累利阿语\\
卡舒比语\\
克里姆查克语\\
库尔德语\\
克文语/芬兰语\\
拉定语\\
兰科语\\
莱昂语\\
林堡语\\
立陶宛语\\
低地德语\\
下萨克森语\\
下索布语\\
吕勒萨米语\\
马其顿语\\
曼岛语\\
梅安语\\
摩尔多瓦语\\
北弗里斯兰语\\
北萨米语\\
波兰语\\
罗姆语\\
罗马尼亚语\\
罗曼什语\\
俄语\\
鲁塞尼亚语\\
萨特弗里斯兰语\\
低地苏格兰语\\
苏格兰-盖尔语\\
塞尔维亚语\\
斯科尔特萨米语\\
斯洛伐克语\\
斯洛文尼亚语\\
南萨米语\\
瑞典语\\
鞑靼语\\
土耳其语\\
乌克兰语\\
阿尔斯特苏格兰语\\
上索布语\\
巴伦西亚语\\
弗拉赫语\\
威尔士语\\
叶尼什语\\
雅兹迪语\\
意第绪语
\end{multicols}
}\\
\lspbottomrule
\end{tabularx}
\caption{《欧洲区域或少数民族语言宪章》所涵盖的语言}
\label{tab:torres:1}
\end{table}

‌根据《宪章》,其中一些语言只会在一个国家受到保护,如芬兰的斯科尔特萨米语,而其他语言则应在多个国家受到保护,如斯洛文尼亚语在奥地利、波黑、克罗地亚和匈牙利都受到保护。除了《宪章》提及的语言,还有一些语言得到不同程度的认可。例如,意大利的撒丁岛自治区承认撒丁语为官方语言,而在意大利北部山区的某些市镇使用的罗曼什语、拉定语、辛布里语和莫其诺语也得到地方认可认。

然而,《宪章》只保障地区少数群体,而非移民群体的权利。‌此外,《宪章》还遗漏了某些语言,如法国西北部的布列塔尼语,尽管布列塔尼地区为促进该语言的日常使用,于2010年创建了一所布列塔尼语语言机构。

\begin{sloppypar}
‌欧洲范围内的移民和人口流动也强化了多语制。‌例如,欧洲国家之间的移民使得有人在安道尔讲葡萄牙语,在爱尔兰讲波兰语,甚至讲欧盟以外的语言,如汉语(普通话)或阿拉伯语。《多语种城市计划》(\citealt{ExtraYagmur2005})对比利时布鲁塞尔、德国汉堡、法国里昂、西班牙马德里、荷兰海牙和瑞典哥德堡这几座城市的中小学学生开展了家庭语言使用调查。‌收集到的语言清单如下:‌罗姆语、土耳其语、乌尔都语、亚美尼亚语、俄语、塞尔维亚语/克罗地亚语/波斯尼亚语、阿尔巴尼亚语、越南语、汉语、阿拉伯语、波兰语、索马里语、葡萄牙语、柏柏尔语、库尔德语、西班牙语、法语、意大利语、英语和德语。该研究的作者提出了清晰但具有挑衅性的结论:
\end{sloppypar}

\begin{quote}
‌在参与调查城市的20种主要语言中,10种源自欧洲,其他10种源自其他国家。调查结果表明,我们应该重新考量并扩大传统认知中的欧洲语言多样性(\citealt{ExtraYagmur2005})。
\end{quote}

可是,什么叫做语言多样性的“传统认知”呢?

\subsection{‌象征欧洲语言多样性的24种官方语言}\largerpage
欧盟将其语言多样性视为宝贵资产。‌《欧洲区域或少数民族语言宪章》(成员国,2012年)第22条规定:“欧盟应尊重文化、宗教和语言多样性”。‌然而,欧盟成员国独有界定和承认国家和地区少数民族语言的权利,且语言政策也极具争议性。

‌可与此同时,欧盟也以其主要机构使用24种官方语言来支持语言多样性而感到自豪。‌从实践的角度来看,这种做法带来一个值得深思的重大挑战。\largerpage

‌例如,欧洲议会的所有文件都以所有官方语言出版,“因为欧盟公民必须能够以本国语言阅读切身的立法”\citep{EuropeanParliament2020},而且欧洲议会议员有权以任何官方语言发言和撰写文件。‌《欧洲议会议事规则》第167条与语言相关,规定:(一)所有议会文件应以官方语言编写;(二)所有议员均有权自行选择官方语言在议会发言;(三)应提供口译服务;(四)议会议长应该对不同官方语言版本之间提出存在的任何差异作出裁决\citep{EuropeanParliament2021}。

‌至于欧盟公民,根据《欧盟运作条约》(Treaty on the Functioning of the European Union,TFEU\footnote{《欧盟运作条约》的最新版本请参考\url{https://eur-lex.europa.eu/legal-content/EN/TXT/PDF/?uri=CELEX:12012E/TXT\&from=EN}。‌除非另有说明,本章中所有网址的最后访问日期均为2022年1月。}),所有欧洲公民都有权以欧盟的任何官方语言向欧盟机构提出问题,并得到以同种语言写就的答复。这是为了使欧盟机构更加民主,使欧盟公民更容易接触。《欧盟运作条约》中与多语制有关的其他规定可参考第20条、第24条和第342条。

‌有人认为24种官方语言太多,有人则认为不够。‌有些国家尝试采取其他办法。‌例如,西班牙人使用的加泰罗尼亚语、巴斯克语和加利西亚语被欧盟认为是“附加语言”(和西班牙语一样,这些语言在各自使用的地区均属于官方语言)。‌这种地位意味着,在西班牙,欧盟公民用上述语言进行的任何沟通都必须翻译成欧盟的“议事语言”,而欧盟机构的答复也必须从议事语言翻译成附加语言。所产生的的翻译成本由西班牙承担。

‌使用英语、法语和德语这三种议事语言旨在简化欧盟的多语种沟通——对欧盟来说,24种官方语言就意味着存在552种翻译语言对组合,“因为每种语言都可以翻译成其他23种语言”\citep{EuropeanParliament2020} ,这样一来,文件翻译的工作量十分惊人。‌因此,有规定指明哪些文件需要翻译成其他23种语言,哪些文件只需翻译成上述三种议事语言。

‌欧盟委员会翻译总司(European Commission’s Directorate-General for Translation,DGT)为欧盟各机构和公民翻译文本。‌截至2022年,每年翻译的文件页数超过275万页,其中91\%为英语原文,2\%为法语原文,西班牙语原文不足1\%,最少的是德语原文。‌其他源语言合计约占翻译服务的5\%。‌在所有翻译文件中,63\%由翻译总司的内部翻译完成,37\%外包给其他公司。‌翻译内容涉及多个领域,包括欧盟法律制定(约55\%)、对外沟通和网络文本(22\%)、与其他欧盟机构和成员国议会的沟通(12\%)、与欧盟公民的沟通(5\%)、其他官方文件(4\%)以及欧盟政策公众咨询(2\%)。‌2022年的翻译预算为3.55亿欧元,占整个欧盟预算的0.2\%\citep{DGT2022}。


\subsection{‌机器翻译的加入}
‌翻译总司拥有2000名内部员工,包括语言专家和技术支持人员,并与数千名选定外部翻译合作\citep{DGT2022}。‌所有语言的译文都存于Euramis系统(EURopean
Advanced Multilingual Information System,“欧洲高级多语言信息系统”),如欧盟24种官方语言的立法文件汇编语料库《共同法律总汇》(\textit{Acquis Communautaire})。为了提高生产率和降低成本,欧盟委员会翻译总司在其部分工作流程中引入了机器翻译,目前使用的机器翻译系统名为eTranslation\footnote{\url{https://ec.europa.eu/info/resources-partners/machine-translation-public-administrations-etranslation_en}}。启用该系统的初衷是为欧盟节省时间和资金,但并不仅限于此。‌最后,随着机器翻译的进一步发展,它不仅能满足文件翻译需求的增长,翻译从未提上翻译日程的文件,将来还能增加翻译的语言对,从而更好地体现欧洲语言的多样性。不过短短的几年时间,机器翻译的发展让这一美好愿景不再是天方夜谭。‌对于原住民或地区少数民族、“非领土”或移民少数民族的语言来说,这是它们与国家官方语言一样,在欧盟机构获得代表性的唯一途径。

\subsection{多语制对欧盟意味着什么?}
多语制政策是维护和保障前文所述的语言多样性的一种方式。‌在欧盟,多语制也被视为加强社会凝聚力和促进劳动力流通的一种手段:‌“[l]语言能力有助于提高欧洲公民的流动性,就业能力和个人发展”\citep{CounciloftheEU2014}。

之所以采用多语制的语言政策,是为了在多语通用的国家和组织促进语言发展。多语制有多种形式,如:

\begin{itemize}
\item 这是一种多语言政策,即组织、公司或机构在对内和对外沟通中使用两种以上语言的策略。
\item ‌也可以指使用多种语言的欧盟,即不同的语言在欧盟共存。
\item ‌能够使用多种语言的多语种公民。
\item ‌使用多种语言的医疗体系,即为新居民提供更好的医疗服务而融合语言多样性的医疗体系。
\end{itemize}

欧盟主要的多语制政策可参考相关文件\citep{EuropeanCouncil2002, EuropeanCommission2008, CounciloftheEU2008may, CounciloftheEU2008dec, CounciloftheEU2011, CounciloftheEU2014, EuropeanCouncil2017}。

‌欧盟的多语制政策和声明经常提到语言学习,特别是“母语加两种外语”政策。根据该政策,欧盟公民“从小应至少学习两种外语”\citep{EuropeanCouncil2002}。 ‌但是,这一政策的实施就真的足以培养出多语公民吗?‌我认为,有必要将更多的内容纳入欧盟的多语政策。

多语制关乎语言政策,但并不仅限于此。\citet{Cenoz2013}提出了多语制的广泛定义,提醒我们,多语制是多维的,例如涉及个人与社会层面、能力与使用层面、双语制与多语制等。多语制也适用于地理区域或社会领域。它也可以应用于地理区域或社会领域。‌此外,多语制还可以从不同角度进行研究,包括认知、社会构建、身份、语言实践、多模态和技术等。而我在本章中使用的既不是维基百科的简单定义(“一个或多个发言者使用一种以上的语言”),也不是\citet{Cenoz2013}‌的复杂多维定义,而是欧盟委员会对多语制的定义,即“社会、机构、团体和个人在日常生活中经常使用一种以上语言的能力”(‌欧盟委员会,2007‌年)。‌“接触”一词使我们能够结合本书中有用的细微差别。‌通过撰写本章,我谨请读者思考是否可以从技术角度来定义多语制,从而调整欧盟委员会使用的上述定义。

\begin{tblsframed}{话题讨论}
‌是否存在“技术层面的多语制”,即在日常生活中,企业、机构、团体和个人借助多语翻译工具经常接触一种以上语言的能力?
\end{tblsframed}

有趣的是,我们使用的多语制定义出自多语制高级小组的上述报告,提到将“多语电子工具或能辅助非专家使用者学习第二和第三外语”作为研究领域\citep{EuropeanCommission2007}。‌同样,欧盟委员会在《多语制:既是财富,又是承诺》(‌欧盟委员会,2014年‌)的函件中称,“欧盟的语言差距可以通过媒体、新型技术和翻译服务来缩小”。‌本书旨在为这一领域作出贡献。


\subsection{‌欧盟的语言多样性促进行动}
欧盟关于语言多样性的网页\footnote{\url{https://ec.europa.eu/education/policies/language-diversity_en}}上提到了以下促进语言多样性的举措:欧洲语言日、“伊拉斯谟计划”交换项目、“欧洲文化之都”活动和“创意欧洲计划”:

\begin{itemize}
\item 欧洲语言日\footnote{\url{https://edl.ecml.at}}由欧洲委员会于2001年设立,于每年9月26日\textsuperscript{th}举行。‌这一天,欧盟国家会开展促进语言多样性和提高其他语言能力的活动。
\item “伊拉斯谟计划”交换项目。‌2014年至2020年期间,该计划为400多万个交流项目拨款147亿欧元,其中包括200万个大学生交流项目。
\item “欧洲文化之都”活动强调包括语言在内的欧洲文化多样性。‌例如,2020年的“欧洲文化之都”为克罗地亚的里耶卡市(Rijeka,通用语言为克罗地亚语)和爱尔兰的高威市(Galway,通用语言为爱尔兰语和英语)。
\item ‌创意欧洲计划\footnote{\url{http://www.creativeeuropeuk.eu/funding-opportunities/literatory-translation-0}}由欧盟委员会推出。这项计划支持包括文学翻译在内的文化和音像行业。‌具体而言,该计划为欧洲语言之间的文学作品翻译提供资助。
\item ‌还有一个有意思的语言多样性促进倡议,那就是欧盟资助项目下的“欧洲语言标签”\footnote{\url{https://ec.europa.eu/education/initiatives/label/label_public/index.cfm}}。这些项目虽然大多面向语言学习,但也有为语言多样性而专门设立的。
\end{itemize}

\begin{tblsframed}{讨论}
就目前看来,欧盟在谈论语言多样性和多语制时,对机器翻译还未给予足够的关注。如何将机器翻译纳入语言学习项目?
\end{tblsframed}

将机器翻译纳入多语制举措,或能帮助欧洲公民增加可以掌握的语言数量。还能帮助他们提升兴趣,克服畏难心理,轻松进入不熟悉的语言环境,尊重当地语言。‌此外,机器翻译还可用于帮助使用者理解未学习过的语言文本。


\subsection{欧盟的多语制和语言学习}
根据欧洲委员会语言政策门户网站:\footnote{\url{https://www.coe.int/en/web/language-policy/home}}

\begin{quote}语言学习者/使用者是语言政策方案工作的核心。‌无论其地位,所有语言都包括在内:外语、主要的教学语言、家庭使用语言、少数民族或地区语言,以及关于移民和难民的语言融合的特别计划。\end{quote}

\hspace*{-1mm}促进多语制的倡议数不胜数,但语言学习值得密切注意,特别是考虑到上述的欧盟“母语加两种外语”政策。

欧盟近期的语言技能提升措施包括成立欧洲现代语言中心(European Centre for Modern Languages, \url{www.ecml.at};欧洲教育系统和政策网络报告\citep{Eurydice2019}),该中心提供有关给予欧洲学校教授地区或少数民族语言的政策支持的信息,介绍了欧洲给予学校教授地区或少数民族语言的政策支持)、在线语言支持(OLS)平台\footnote{\url{https://erasmusplusols.eu/en/about-ols/}};欧洲语言共同参考框架(CEFR);“伊拉斯谟计划”交换项目。

这些为促进语言学习欧洲受资项目值得我们特别关注。项目所涉及的方法、语言和国家都各不相同。具体案例请参考表\tabref{tab:torres:2}。


\begin{table}
\small
\begin{tabularx}{\textwidth}{Qp{2.5cm}Q}
\lsptoprule
{项目} & {语言} & {特点}\\
\midrule
\href{https://www.itongue.eu/}{{iTongue: Our Multilingual Future (2013)}} & 未指定 & 用于外语学习的副语言学数字工具\\
\tablevspace
\href{http://medlang.eu/videos.php}{{Massive open online courses with videos for palliative clinical field and intercultural and multilingual medical communication (2014)}} & 德语、英语、法语、意大利语、西班牙语、罗马尼亚语 & 姑息疗法的20项基本程序的多语版本\\
\tablevspace
\href{https://languages4work.eu/}{{Crafting Employability Strategies for HE Students of Languages in Europe (2015).}} & 未指定 & 将就业能力纳入语言教学\\
\tablevspace
\href{http://www.mooc2move.eu/}{{LMOOCs for university students on the move (2018)}} & 法语、西班牙语 & 面向大学生的公开教育资源\\
\tablevspace
\href{http://stratapp.eu/}{{Gamifiying Academic English Skills in Higher Education: Reading Academic English App (2016).}} & 未指定 & 基于游戏的应用程序,以提高大学生的学术英语阅读技能\\
\tablevspace
\href{http://elengua.usal.es/}{{E-LENGUA: E-Learning Novelties towards the Goal of a Universal Acquisition of Foreign and Second Languages (2015).}} & 英语、阿拉伯语、西班牙语、法语、德语、意大利语、葡萄牙语 & 将数字能力纳入语言教学的最佳实践\\
\tablevspace
\href{https://eulaliaproject.eu/es/}{{EULALIA: Enhancing University Language courses with an App powered by game-based Learning and tangible user Interfaces Activities (2019).}} & 意大利语、波兰语、西班牙语、马耳他语 & 基于移动学习模式和游戏式学习方法以及有形用户界面(TUI)应用的包容性学习工具\\
\tablevspace
\lspbottomrule
\end{tabularx}
\caption{‌关注语言学习的欧洲项目案例}
\label{tab:torres:2}
\end{table}

欧盟统计局的官方网站Eurostat\footnote{\url{https://appsso.eurostat.ec.europa.eu}}提供了欧盟国家不同年级学生的第二语言和外语学习统计数据。该网站2019年的数据显示,初中生学习最多的外语是英语(将近86.8\%),其次是法语(19.4\%)、德语(18.3\%)和西班牙语(17.5\%)\citep{Eurostat2022}。

\begin{tblsframed}{话题讨论}
提到欧洲的第二语言学习,“多语言”是否暗指“英语”?
\end{tblsframed}

还有一个有趣的问题,有多少学生依照欧洲理事会\citet{EuropeanCouncil2002}的建议,学习两种或两种以上的外语?据了解,2019年,有89.9\%的中学生(近1400万人)学习了一门以上的外语\citep{Eurostat2022}。‌其中700多万人(48.1\%)学习了两种或两种以上的外语

总的来说,有7亿多来自欧盟或欧盟以外国家的人\citep{WorldBank2020}在欧洲使用的语言多达600种,而大多数欧盟国家学生学习的第一或第二外语主要有四种:英语、法语、德语或西班牙语。

\subsection{外语水平}
欧洲语言能力调查\citep{EuropeanCommission2019}选择在16个教育系统中最广泛教授的两门外语,对其54000名初中生(14至15岁)进行了水平测试。测试内容包括写作、阅读理解和听力理解,不设口语测试。调查的主要结论是,只有42\%的受试学生能自主运用其第一外语(即达到欧洲语言共同参考框架的B1和B2级别),而第二外语达到同等水平的学生只占25\%。‌此外,还有很多受试学生甚至没有达到初级水平:其中第一外语未到初级水平的占14\%,第二外语未到初级水平的占20\%\citep{EuropeanCommission2019}。

‌该调查报告还指出,2018年《欧洲晴雨表快讯》(flash Eurobarometer)对15至30岁人群进行调查的数据显示,85\%的受访者表示希望能提高自己的外语水平(主要是英语):

\begin{quote}
这表明,受访者对自己在义务教育阶段的外语水平并不满意,或无法继续保持在校水平。‌接受调查的欧洲年轻人中,有三分之一表示不能使用除了在校使用语言(即通常是母语)以外的语言学习。\citep[102]{EuropeanCommission2019}  
\end{quote}


%%[Warning: Draw object ignored]

\subsection{机器翻译能助力语言学习吗?}
我们已经看到,欧盟国家中的语言学习往往集中在少数常见语种上,而学习者常常对自己的外语水平并不满意。这些情况表明,需要对语言学习提供进一步的支持,而我们有责任研究机器翻译在其中的作用。‌由于神经机器翻译的学习速度比任何外语学习者都快,理论上,它可以用来帮助学习者理解复杂文本,提高他们的第二语言书写技能。他们可以了解如何充分利用机器翻译来学习第二语言,并依靠自己的语言水平来识别和改正机器翻译的错误。虽然有关机器翻译如何运用于语言课程的实证研究仍然不足,但还是有一些资料介绍了这方面的研究路径。‌Carré等人\textcitetv{chapters/carre}讨论了相关研究,其中包括可用于语言学习课堂的进一步想法和策略。‌还有一些被列入MultiTraiNMT项目的活动数据库\citep{MultiTraiNMT2020}。我认为,在语言课上使用机器翻译的方法有很多。如果有意识地加以批判使用,就没有必要禁止。

‌然而,有时的确没有足够的时间来培训第二语言的学生。‌其实,在机器翻译的发展历程中,很多研究的部分起因是找不到会说某种外语的人。‌冷战时期的俄英机器翻译研究正是如此 \citep{Gordin2016}。最近,2020年日本东京奥运会的组织者(由于新冠爆发,实际上于2021年举行)意识到,不可能让大多数外国人学习日语,因此需要一个更快的方法来克服奥运会期间的语言障碍。‌因此,日本内务省拨款13.8亿日元用于机器翻译研究,以提高实时语音翻译技术的质量,目的是覆盖90\%的奥运团队和游客的语言需求,希望他们能在奥运期间前往日本\citep{Murai2015}。‌日本政府为东京奥运会专用机器翻译系统的研究提供资金,私营公司则负责研发运行该系统的设备和移动应用程序。‌该计划是,公司将通过向用户出售设备和应用程序订阅来收回投资成本。‌在这种情况下,机器翻译便成为日本快速营造多语环境的方法,而非语言学习。

\section{案例研究:多语言大学}
\subsection{‌‌国际化与多语制}
欧盟委员会关于“全球欧洲高等教育”的通报确定了欧洲大学国际化的重点是‌访学交流、数字学习和加强战略合作。‌‌通报中有关语言的内容如下:

\begin{quote}
‌对于语言学习者、教师和机构来说,英语水平的确是任何国际化战略的一部分。部分欧盟成员国已经或正在引入有针对性的英语课程(尤其是在硕士阶段),作为其人才吸引战略的一部分,否则他们不会来欧洲。\citep{EuropeanCommission2013}
\end{quote}
    
\begin{sloppypar}
‌同时,多语制是欧洲的重要资产:它受到国际学生的高度重视,应该在整个高等教育课程的教学和研究中得到支持\citep{EuropeanCommission2013}。
\end{sloppypar}

事实上,欧盟仍然致力于在大学校园内推广多语制实践。原因有三。首先,因多语言校园反映了欧洲的语言多样性;其次,这为学生提供了更多的出国交流和就业机会;第三,这能促进与不同文化和学习方法的交流。与此类似,\citet{Gao2019}列出了大学参与国际化的不同原因,包括国际化可以帮助学生准备好与来自不同文化的人互动,以此来促进文化理解,减少国家间的敌意。‌然而,国际化给大学带来了挑战(同上),特别是与多语制有关的挑战。‌首先,翻译资源(包括人力资源)的缺乏有碍于大学成为完全使用英语的场所。其次,国际化可能涉及到母语地位的下降和语言多样性的丧失。‌不过,我认为机器翻译有助于应对这些困境。‌鉴于对国际化理解的转变,这项技术似乎颇具前景:一直以来,大学都会制定“国内国际化”(吸引外国学生)和“国外国际化”(将学生送往国外)的计划。在此基础上,\citet{MittelmeierRaghuram2020}纳入了“远程国际化”的概念,为以校园为基础的机构开发在线国际远程学习模式。‌‌新冠疫情的爆发无疑加速了第三种方式的发展。技术可能会改变国际化的概念,而机器翻译是一种有可能对国际化作出贡献的技术。

\subsection{英语通用语(ELF)、本地语言(LLS)与机器翻译}
‌高校的国际化战略多种多样,但最常见的可能是访学交流、开设英语或双语课程。后者有不同的名称,例如,英语通用语、英语化和全英语教学(EMI)。它被定义为 "在非英语国家的高等教育机构中使用英语作为教学媒介语言"\citep{MultilingualHigherEducation2016}。

‌英语的使用便于很多国际学生听课。但是,如果本地大学出于经济原因(因为国际学生的学费比本地学生的高)而更照顾国际学生的话,英语水平不够好的本地学生可能会有被辜负的感觉。‌有些大学在慕课(译者注:大规模开放在线课堂)上提供英语授课,想以此吸引国际学生,但最近研究表明,开设慕课对于吸引国际学生收效甚微。\citep{Zakharova2019}.

‌为了克服作为通用语的英语和本地语言之间紧张关系,House\citet{House2003}将其区分为“交流语”和“认同语”。豪斯\citet[560]{House2003}认为,交流语有助于母语不同的人进行交流。而认同语

\begin{quote}
 {通常是地方语言,特别是个人的母语,这可能是决定身份认同的主要因素,意味着在界定母语群体及其成员的集体语言文化资本中发挥作用……以及认同中涉及的情感-情绪素质类型。\citep[561]{House2003}}
\end{quote}

\begin{sloppypar}
根据这一分类,英语是母语不同的人的交流语,是一种“专业知识”\citep[561]{House2003},而认同语在母语相同的人群内部使用。
\end{sloppypar}

豪斯\citet{House2003}在德国汉堡大学对使用英语作为教学语言进行了一项案例研究,观察英语如何与当地语言互动,以及国际学生如何看待和应对这种“双层语言(diglossic)场景”\citep[570]{House2003}。结果显示,人们并不认为英语和德语存在竞争关系。英语“自成一类”,是一种超国家的辅助交流手段。在该研究中,“(目前)没有迹象表明当地语言(德语)和多语制受到威胁”\citep[574]{House2003},而且学校鼓励国际学生留学期间学习德语。

\subsection{一所真正的多语大学}
目前的多语大学是指为国际学生提供英语授课的大学,以及边境地区或使用两种以上官方语言的地区的大学。这些大学使用多种语言可能是出于历史、政治、地理或经济等原因,因此难以在语言政策上找到平衡点。国际化和多语言政策都不可能一蹴而就。\citet[12]{Knight1994}提出了机构国际化进程的六个阶段:

\begin{itemize}
\item 认清需求、目的和益处 
\item 所有大学行为者承诺参与
\item 规划:确定资源、目的和目标,以及优先事项和战略 
\item 实施:开展学术活动和服务 
\item 评审:评估和提高质量、影响和进度 
\item 强化:设置激励、认可和奖励方法
\end{itemize}

这项与多语制特别相关的计划要求我们回答以下问题:英语授课是否仅限于国际学生或课程?如果只面向国际学生,这所大学是否能称为多语大学?如果国际学生在“多语”大学里只使用英语,那他们还算是多语使用者吗?如果一所大学只在网站上使用官方语言和英语,但用当地语言授课,这样就够了吗?授课使用多种语言,还是只一种语言?学生使用的本地语言(可能不是大学的官方语言)能够融入课堂吗?授课语言是判断“多语言大学”的唯一标准吗?提供哪些语言的教学资料?语言能力是否与非语言内容一同评估?本地学生如何为英语授课做准备?多语种是否意味着使用英语授课?多语种是否意味着使用当地语言和英语?若是,分别占比多少?要有多少种语言才足以成为多语种大学?需要融入哪些语言来保证大学生进一步融入地理区域或当地经济?国际学生会生活在只有英语的舒适圈里吗?

在一个真正的多语言校园里,大学既欢迎来自国际学生的语言,也欢迎来自社会或文化边缘化群体的语言。许多地方语言在学术界历来被低估,有人对其学术词汇的丰富程度表示怀疑。一个极端的例子是克丘亚语(Quechua):第一篇完全用克丘亚语撰写的博士论文于2019年进行答辩,这年离秘鲁第一所大学成立大约过去了468年\footnote{\url{https://www.theguardian.com/world/2019/oct/27/peru-student-roxana-quispe-collantes-thesis-inca-language-quechua}}。

作为“伊拉斯谟计划”多语言高等教育\citet{MultilingualHigherEducation2016}的一部分,网络课程《多语制和教育中的多语言》(Multilingualism and plurilingualism in education)介绍了不同多语言大学的语言政策。例如,瑞士弗里堡大学设有法语和德语课程;芬兰赫尔辛基大学设有芬兰语、瑞典语和英语授课;意大利博尔扎诺自由大学设有德语、意大利语和拉定语授课,并以英语作为通用语;卢森堡大学至少20\%的课程同时以法语、英语和德语这三种语言授课。

\citet[89]{Gao2019} 提出了用以区分国际化的战略愿望和现实的衡量维度和指标。针对多语制,我们对这些建议做出了调整。

在大学治理方面,促进采用多语制的行动可包括:
(一)支持性的多语种政策框架/组织结构;
(二)语言办公室/翻译部门;
(三)机器翻译基础设施;
(四)大学内的多语种展示/标识;
(五)培养教工的多语种意识和技能;
(六)用于支持多语制的预算;
(七)多语情况监测/评价系统。

从学术角度来看,促进采用多语制的行动可包括:
(一)多语课程(为什么课程必须使用一种且只有一种语言?);
(二)多语教学,规范多语课堂;
(三)多语研究和多语会议;
(四)班上多语学生,包括国际学生和本地学生之间的互动;
(五)多语访问学者;
(六)多语教学大纲;
(七)多语研究期刊;
(八)多语课外活动和
(九)文化多样性的可见性。

最后,大学可以提供多语言定向计划、多语言支持和多语言图书馆。


%%[Warning: Draw object ignored]

\subsection{在多语言高校教学、科研、管理中使用机器翻译的思考}
多语言大学可以使用英语和当地语言互译最多的机器翻译系统。第一种策略是使用免费在线机器翻译服务,但用户应该识别并能够纠正机器翻译错误,或自行纠错,或求助专业的译后编辑服务。关于译后编辑的更多信息可参阅\textcitetv{chapters/obrien}。

第二种策略是开发大学定制的本地语言和英语之间的高质量机器翻译资源,详见\textcitetv{chapters/ramirez}。如果资源允许,欧洲和欧洲以外的大学可以共享定制的机器翻译系统。

在未来真正的多语言校园里:

\begin{itemize}
\item 国际学生能和当地学生打成一片,且能获取机器翻译资源,以便使用任何一种本地语言听课,因为会有:(一)不同语言的教学材料(假设版权问题已经解决);(二)使用多语种词汇表和专门术语数据库;(三)使用语音识别、转录和机器翻译功能等对课程进行录音。
\end{itemize}
\begin{itemize}
\item 学生将学习译后编辑的技能,以查看英语和/或其他语言的机器翻译结果,以便能够正确使用机器翻译输出。
\item 大学将为高质量的教学指南和教材提供译后编辑服务。
\item 将鼓励多语种研究传播和多语种出版物,提供嵌入式机器翻译功能。
\item 希望用母语发言的客座教授将会得到翻译服务,无论是人工翻译(若资金允许的话),还是机器翻译,而不是用蹩脚的英语费力地完成演讲和会议论文。
\item 将在校园内组织多语言活动,涉及音乐、戏剧、烹饪、政治、文学、团结、社会服务等各个领域。
\end{itemize}

\begin{sloppypar}
在一所真正的多语言大学里,当地语言、英语和其他由留学生带来的语言应该共存。这一策略有助于国际学生融入多语环境。
\end{sloppypar}

\section{结语}\largerpage
本章支持机器翻译作为推进欧洲多语制的工具。如本章所示,欧盟发布了宪章、条约和会议文件,宣传多语制是欧洲必须培养和维护的核心价值观。然而,尽管在语言学习项目上投入了很多努力和资源,“母语加两门外语”的学习目标依然很难达到。一方面,现实中大多数欧盟公民的外语只学英语;另一方面,语言学习是个漫长的过程。因此,机器翻译似乎为那些没有时间或资源继续学习更多语言的人提供了一些支持。

此外,本章还以小型多语言社群——大学——为案例,讨论了大学校园如何制定语言政策推行多语制。。语言政策可能会出于多种原因在校园内产生矛盾,但如今大多数校园实际上都是多语言的,或是通过大学的国际化/英语化,或是由于留学生的到来。在此背景下,本章探讨了设计语言政策的必要性,这些政策应当承认机器翻译推进多语制的潜力,同时不忘机器翻译带来的挑战,特别是译文质量和道德问题。在此,我想借用西班牙语当中的一个表达——“打开机器翻译的甜瓜”(abrir el melón),“打开甜瓜”的意思是解决一个迟早需要处理的问题,尽管没有人愿意这样做,因为后果未知。换句话说,没有人知道这个瓜够不够甜,但只有打开才能知道。即使现有的机器翻译系统无法表达出这个短语的隐喻意义,能看懂机器翻译字面意思的读者还是能学到一个有用的西班牙语隐喻。指不定这个比喻甚至可以传播到新的语言和文化中,因为它仅用三个词就能表达出复杂的含义。而这就是多语制的具体体现。\largerpage

\printbibliography[heading=subbibliography,notkeyword=this]

\end{document}
