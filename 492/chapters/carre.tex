\documentclass[output=paper,colorlinks,citecolor=brown]{langscibook}
\ChapterDOI{10.5281/zenodo.14922301}

\author{Alice Carré\affiliation{格勒诺布尔-阿尔卑斯大学} and Dorothy Kenny\affiliation{都柏林城市大学} and Caroline Rossi\affiliation{格勒诺布尔-阿尔卑斯大学} and  Pilar Sánchez-Gijón\affiliation{巴塞罗那自治大学} and Olga Torres-Hostench\affiliation{巴塞罗那自治大学}}

\title{机器翻译之于语言学习者}
\abstract{机器翻译在二语和外语的学习中一直存在争议,但在某些情况下,有策略地使用机器翻译可能有利于语言学习。我们会在本章讨论语言学习过程中如何运用机器翻译,提出机器翻译的数字替代方案,并举例说明机器翻译如何为语言学习者提供帮助。}


\begin{document}
\lehead{A. Carré, D. Kenny, C. Rossi, P. Sánchez Gijón \& O. Torres-Hostench}
\AffiliationsWithIndexing{}
\maketitle

\section{引言}\label{sec:carre:1}

在二语和外语学习中\footnote{请注意,本章以通用术语“语言学习”(language learning)和“语言学习者”(language learner)来指代外语学习、第二语言和随后的语言学习。学习者的第一语言为L1,其他语言则相应为L2、L3
、L\textit{n}。},机器翻译的使用一直存在争议,部分学者认为这会助长抄袭行为、造成更多错误或影响正常的学习轨迹。但是,机器翻译有时却能帮助学习者完成某些学习任务,可以视作当代语言学习者的众多数字资源之一。机器翻译与语言学习的成功结合要求我们了解这项技术的基本工作原理、机器译文质量测评方法、如何通过译前和译后编辑来提高输出质量,以及机器翻译使用的伦理问题等。这些知识和技能通常属于“机器翻译素养”\citep{BowkerCiro2019}的范围,已在本书的第2章到第6章中做了详细阐述。与此同时,\textcitetv{chapters/torres}还就“为什么机器翻译和语言学习一样,都应被视为多语言社会的重要组成部分”提出了具有说服力的观点。到目前为止,本书还未深入探究如何在语言学习中使用机器翻译,而本章的目的正是填补这一空缺。本章先介绍翻译在语言学习中的作用,然后讨论机器翻译是否也能用于语言学习,以及这种做法有何好处。接着,我们会根据使用场景,为语言学习者提出机器翻译适用和不适用的语言学习场景,并介绍更适用的数字资源。最后,我们会举例说明如何在语言学习环境中使用机器翻译。


\section{语言学习与翻译}\label{sec:carre:2}
在讨论机器翻译对语言学习的作用之前,值得一提的是,这个话题长期以来饱受争议。“语法翻译法”(即要求学习者先学习词汇和语法规则,再翻译没有给出语境的句子)备受批评,人们认为借助翻译进行语言教学具有局限性,但这种看法本身对翻译的理解就很狭隘。与此同时,\citet{Cook2010}重新激发了学界对“语言教学中的翻译”(Translation in Language Teaching,TILT)的关注,而且“在过去十年,各学科中有关翻译在语言课堂上的运用的研究成倍增加”\citep[12]{PintadoGutiérrez2018}。研究发现,借助翻译学习语言不仅有助于提高“多语言、多文化和语言交流的能力”、写作技能、语言意识和掌控能力(同上:13),还能减轻语言学习者的焦虑和认知负荷(\citealt{KellyBruen2017})。更多有关TILT的运用案例可参阅\citet{CarreresGutiérrez2021},作者介绍了如何将翻译相关活动融入语言教学的最新动态。

\section{语言学习与机器翻译}\label{sec:carre:3}
\subsection{运用机器翻译学习语言可取吗}
2010年左右的研究表明,机器翻译完全被排斥在当时的语言课堂之外。\citet{PymMalmkjaer2013}对全球教师的调查结果发现,极少教师会在语言课上使用机器译文。其中一名研究对象表示,机器翻译与不愿使用翻译进行语言教学之间似乎也存在因果关系——在中学,不使用翻译的原因往往是“害怕学生依赖机器翻译工具”\citep[93]{PymMalmkjaer2013}。

\begin{sloppypar}
不过,无论老师怎么想,使用免费在线机器翻译(Free Online MT,FOMT)的语言学习者越来越多,其使用效果值得研究。\citet{Lee2021}和\citet{JolleyMaimone2022}等学者对机器翻译在语言教学中的使用进行了综述,发现很多人将使用机器翻译视为作弊行为。于是便有学者撰写如何“检测”学生是否在L2写作中使用机器翻译。以前,机器翻译的质量一般较差,可以通过识别机翻译文的典型错误轻松判断学生是否使用了机器翻译。但如今,研究人员认为:\end{sloppypar}

\begin{quote}
随着机器翻译技术的不断进步,对于语言教师来说,识别翻译“错误”可能愈发困难。不过,帮助识别机器翻译使用的,正是这项技术的成功,而非缺点。(\citealt{DucarSchocket2018}: 787)
\end{quote}

换句话说,被发现使用机器翻译的学生露出马脚并不是因为他们的写作错误百出,而是因为机器译文质量超出了学生的实际水平。例如,若法语初学者使用了高阶课程才学到的虚拟动词形式,他就有可能使用了机器翻译。

但一个人是否作弊的判断标准不在于所使用的技术,而在于“游戏规则”。如果在L2写作中不允许使用机器翻译,但学习者还是偷偷使用,这就是作弊行为。即便没有明确禁止使用机器翻译,但学生还是在老师不知情的情况下使用,并将机器译文当做自己的写作成果,这也是投机取巧的做法。事实上,将他人的观点据为己有属于“剽窃”行为,\citet{MundtGroves2016}就机器翻译在语言学习的使用这一背景探讨了这个话题。然而,许多研究(如\citealt{Correa2011, Clifford2013, DucarSchocket2018})表明,根据机器翻译的使用程度等诸多因素,学生和教师对在语言学习中使用机器翻译持不同态度,因此情况并不是非黑即白。

如果你是在正规的环境中学习语言,最好先向老师了解清楚哪些是作弊行为。如果你是语言老师,最好先跟学生明确学习要求。无论如何,明确在完成语言类学习任务时是否能够使用机器翻译,避免作弊只是需要考量因素中的一个。其他需要考虑的因素请参阅本章\sectref{sec:carre:3}的“情境参数”和\textcitetv{chapters/moorkens}有关道德规范的阐述。我们接下来的讨论重点在于,无论老师是否同意,使用机器翻译会有哪些好处。

\subsection{在语言学习中使用机器翻译的好处}
有证据表明,使用机器翻译来完成特定任务的学习者能在短期内取得进步。例如,在\citet{O’Neill2019}的研究中,310名具备中级水平的美国大学生在不同的实验条件下用法语和西班牙语进行写作,分为“可使用谷歌翻译且事先经培训”、“可使用谷歌翻译但事先无培训”、“可使用一款在线词典且事先经培训”、“可使用一款在线词典但事先无培训”和“无任何技术支持”。其中可使用谷歌翻译且事先经培训的学生的作文得分最高,其次是可使用一款在线词典且事先经培训的学生。这些学生在一周和三到四周后又分别进行了两次测试,期间不能使用上述工具,结果显示“谷歌翻译+培训”组的学生得分并不高于其他组。这说明他们凭借工具取得的优势只能短暂存在,且还要取决于能继续使用工具。

在另一项研究中,\citet{Fredholm2019}对31名外语为西班牙语的瑞典高中生进行了为期一学年的跟踪调查,研究其作文的词汇多样性,其中大约一半的学生使用纸质词典,另一半使用谷歌翻译。他发现,机器翻译的使用与更高的词汇多样性有关,只要学生能够继续使用谷歌翻译,得分就更高,可一旦不能使用,相对优势就消失殆尽。同样,机器翻译的益处似乎要靠一直使用才能获得。

这是否意味着学习语言没必要使用机器翻译呢?不完全是。在上述两项研究中,从长远来看,使用机器翻译工具对学生没有坏处,只不过一旦不能使用工具,使用和不使用机器翻译的学生在写作方面的差别就不明显了。不过在短期内,使用机器翻译的学生的文章写得更好。因此,使用机器翻译是否有好处,似乎取决于是从短期还是长期的角度来看,以及使用目的是为了完成某个特定任务,还是持续学习语言。

通过\citegen{O’Neill2019}的研究还发现,(至少在短期内)工具使用的培训很重要。受过机器翻译工具使用训练的学生,即便只是短暂的训练,也比没有受过训练的写作更好。

\subsection{机器翻译对语言学习到底有何益处}
我们已经看到,使用机器翻译可以帮助一些学生提高整体写作水平,或者有助于提高写作中的词汇多样性。有关使用机器翻译对二语(或三语)写作的影响的研究重点关注学习者对词汇、语法和句法的使用。\citet{Lee2021}指出:

\begin{quote}
大量研究证实,机器翻译有助于学生减少拼写、词汇和语法错误,更专注于内容,因此,使用机器翻译的学生的修改成功编辑次数更多,二语写作质量更高。\citep[4]{Lee2021}
\end{quote}

但应该注意的是,个别研究的结果却截然不同。例如,\citet{Fredholm2015}发现,使用免费在线机器翻译的瑞典学生在进行西班牙语写作时,比起没有使用的学生,出现的拼写和冠词/名词/形容词配合方面的错误更少,但句法和动词变化方面的错误更多。

其他研究还关注学生的“元语言意识”(metalinguistic awareness),即:

\begin{quote}
 {将语言当做客体,关注、思考并评价语言本身的能力。(\citet[531]{Thomas1988} in \citealt[67]{ThueVold2018})}
\end{quote}

\citet{EnkinMejías-Bikandi2016}在一项研究中设计了一个练习(但并未进行测试),在这些练习中,学生接触到将英语容易出错的特定句法结构,翻译成西班牙语的机器翻译,而由于这两种语言的对比差异,机器翻译的效果通常较差。例如,非限定从句翻译成西班牙语时,最好使用限定从句。这个练习的目的是,学生可以思考机器翻译的错误,从而强化元语言意识,尤其是对比意识。然而,作者指出,随着机器翻译质量的提高,“教学资料或许也要跟上步伐”\citep[145]{EnkinMejías-Bikandi2016}。不过客观来讲,英语-西班牙语这类语言对的神经网络机器翻译引擎已经达到了很高的水平,已经不太可能出现误译重要句法结构的情况,因此,上述学者建议的练习还需重新审视,鼓励学生反思机器翻译的优点,而非缺点。\largerpage

让学习者给机器译文纠错的研究也得出同样的结论。不仅接触“低质模型”\citep{Niño2009} 在语言学习中会引发争议,而且一眼识别出神经网络机器翻译的错误可能会越来越困难(\citealt{CastilhoWay2017, LoockLéchauguette2021}),这使得译后编辑之类的任务(参见\citetv{chapters/obrien})与以往相比不再适合在某些语言学习环境中使用(参见\citealt{ZhangTorres-Hostench2019})。\footnote{话虽如此,\citet{LoockLéchauguette2021}等的最近研究可能更关注语言学习者机器翻译素养的培养,而不是元语言意识本身。教师指导下的机器译文错误分析也是出于这一目的。} 

\citet{ThueVold2018}开展了另一项关于元语言意识的研究,对象是挪威一所高中将法语作为第三语言的学生。他们要阅读同一文本的两种机器译文(分别来自谷歌翻译和微软必应翻译),然后说出哪种译文更好,并解释原因。虽然作者\citep[73]{ThueVold2018}表示这些学生的法语水平不高于欧洲共同语言参考框架的B1等级\footnote{\url{https://www.coe.int/en/web/common-european-framework-reference-languages}},但他们的具体水平并不确定。Thue Vold总结道,虽然使用机器译文来培养学习者的元语言意识“大有可为”,但“培训、技术支持和教师的引导至关重要”(同上:89),因为如果听之任之,学习者可能无法作出富有成效的分析,学生群体之间交流甚至可能强化对语言的误解(同上)。

\subsection{在语言学习中使用机器翻译的技巧}\largerpage
在语言学习中使用机器翻译具体有哪些好处,对此还没有定论。然而,这种好处很可能取决于一系列因素,包括学生的语言水平、二语写作中的文本体裁(参见\citealt{ChungAhn2021}),以及相关的语言对(参见下文\sectref{sec:carre:4.1})。然而,目前越来越多研究都一致认为:

\begin{itemize}
\item 语言学习者使用机器翻译,与其试图禁止其使用,不如根据机器翻译的有用程度进行区别化教学,这可能取决于使用的程度和场景。
\item 受过适当训练的语言学习者能更好地使用机器翻译。
\item 如果语言学习者已经熟练掌握外语,他们通常会从机器翻译中受益更多 (\citealt{O’Neill2012, ResendeWay2021})。
\end{itemize}

如果学习语言时的确使用了机器翻译,有几点并不显而易见的事项需要老师和学生注意 (还可参见 \citealt{Bowker2020})。

首先,学生能使用的机器翻译工具不止一种。许多人似乎只知道谷歌翻译(可参见\citealt{DorstBouman2022}),但学生也可以通过比较不同的免费在线机器翻译工具,如必应翻译、DeepL或百度翻译,学生可以从中获益不浅。

其次,由于机翻引擎在使用过程中会反复训练或改进,机翻译文也会随之改变。这意味着不应仅凭一次使用就否定某个机翻系统,也意味着在文章中使用免费在线翻译工具的研究人员应该说明生成机器译文的确切时间,但少有学者做到这一点。

最后,如果你用免费在线机器翻译工具来创建语言教学练习、阐述关于机器翻译技术的观点,或者辅助写作,就要给这个系统好好发挥的机会。输入文本的微小变化可能会对输出产生巨大影响。一项关于机器翻译在二语学习中的运用的研究中,{DucarSchocket2018}举了谷歌翻译的一个误译例子,如图\figref{fig:carre:1}所示,该译文基于“milk这个词的常用字面意思,而非隐喻意义”。

\begin{figure}
\includegraphics[width=\textwidth]{figures/MultiTraiNMTChapter9MTforLanguageLearners-img001.png}
\caption{\label{fig:carre:1}: 输入谷歌翻译的句子首字母小写且句尾无标点符号,译文生成于2021年10月19日}
\end{figure}

这里值得注意的是,输入句子首字母小写,且句尾无句号。但是,—如\figref{fig:carre:2}所示,若补充这些标准书面语言的特点,输出译文也有所改变——变得更好。

\begin{figure}
\includegraphics[width=\textwidth]{figures/MultiTraiNMTChapter9MTforLanguageLearners-img002.png}
 \caption{输入谷歌翻译的句子首字母大写且句尾有标点符号,译文生成于2021年10月19日}
\label{fig:carre:2}
\end{figure}

学生将文本复制粘贴到免费在线机器翻译工具时,也观察到了类似的问题。他们没有意识到复制的文本每行行末都有一个换行符,因此工具识别的是独立的行,每行都单独进行翻译。最先进的机器翻译引擎以句子为单位来训练翻译,只在能够识别和翻译完整的句子时的表现最好。因此,确保不要将带有换行符的文本“输入”到机器中,这点很重要。

\begin{itemize}
\item 正如不同的机器翻译引擎或系统的输出可以有效对比,机器翻译的有用性也可以与语料库或在线词典等工具进行对比。详细阐述请看下一节。
\end{itemize}


\section{何时使用机器翻译:语言和情景要素}\label{sec:carre:4}
\subsection{我只要一点帮助} \label{sec:carre:4.1}

如果你正在学外语,或者遇到自己不懂的语言,机器翻译是你最好的帮手吗?机器翻译不仅看似方便快捷,甚至可以帮你蒙骗老师或交际对象,让对方以为你的外语水平精湛,以至于现在的语言类学生经常分享对使用免费机器翻译引擎的依赖,例如承认这个工具能让他们做“一小时的双语者”。\footnote{这只是其中一名学生的机器翻译使用经历:\url{https://mtt.hypotheses.org/our-students-mt-stories}}  但是,你能使用机器翻译来提高外语的听、说、读和写吗?上述的研究表明,借助机器翻译提高外语技能的条件有二:第一,良好的外语水平;第二,扎实的机器翻译知识和“机器翻译素养”\citep{BowkerCiro2019}。前者只能靠反复练习,但后者的培养可以参考本章提出的准则和建议。

\subsection{语言对与体裁} 
首先,要知道机器翻译在不同语言对上的表现有好有坏。事实上,因为神经网络机器翻译系统是基于语料库的(如第2章和第7章所示),若训练数据不足,系统的输出质量会更差。本节内容给出的例子为法英语言对(包括英法双向的翻译),目前这个语言对的机器翻译质量不错,但我们欢迎读者举出自己语言对的翻译例子,与本节示例进行对比。

体裁也会造成差异。例如,你可能会发现,免费在线机器翻译系统更擅长翻译散文,而非诗歌或你喜欢的歌词。这可能是因为用于训练机器翻译系统的数据与前者更相似,并且训练数据中罕有歌曲和诗歌译文。诗歌和歌曲的翻译要求严苛,译文必须也能够朗诵或吟唱,需要特定的押韵或韵律。虽然可以训练机器撰写甚至翻译诗歌\citep{VandeCruys2018, VandeCruys2019, VandeCruys2020},但通用型免费在线机器翻译系统可能无法胜任这项任务。不过,试试也无妨,你可以选一首母语或二语的流行歌曲、诗歌或童谣,找到优质的人工译文\footnote{你也许可以在网上找到已出版的诗歌、歌曲和童谣的人工翻译。}尝试用目标语来翻出有韵律和节奏的译文。然后,借助免费在线机器翻译系统来翻译原文,将其与人工翻译进行比较。对比结果可能有助于你思考机器翻译的优点,以及人工翻译的出彩之处。

\subsection{请教语言学家}
其次,要了解你自己、老师和/或对话者的期望。不妨问问他们是否认为机器翻译可取,以及这种工具是否干扰学习语言。在某些情况下,老师使用的材料或许不能输入在线免费机器翻译引擎,因为其中包含个人或机密数据(参阅\citetv{chapters/moorkens})。语言老师可能还会说,机器翻译不利于你学习语法规则,因为它让你失去了主动学习句子结构和措辞的积极性。如果需要熟练掌握语法知识才能修改机器翻译的输出,那么使用机器翻译不一定能提高你的语法水平。恰恰相反,学生暴露在错误中却不自知,这会带来不利影响。然而,至少在某些特定情况下,通过反复使用流畅的机器翻译输出来学习外语是可以实现的:在最近的一项研究中,\citet{ResendeWay2021}证明了,部分学生可以通过神经网络机器翻译的输出间接学习句法。但学习者是否也会受到机器翻译错误的影响还有待观察。如果没有良好的机器翻译素养和足够好的目的语知识,他们很可能根本识别不出错误。(例子可参见\citealt{LoockLéchauguette2021}。)

因此,我们建议学生时常请教语言学家——先请教老师,因为他们不仅了解你的二语水平,还知道使用免费在线机器翻译工具来翻译某个语言对的特定内容有何优缺点。


\subsection{情景参数}
表\tabref{tab:carre:1}简要列出了建议考虑的情景决定参数,以帮助语言学习者判断何时可以或不应使用机器翻译。我们的建议基于免费在线神经网络机器翻译工具的使用。(注意:若表\tabref{tab:carre:1}中参数1的答案为否,则不要使用机器翻译。)

\begin{table}
\begin{tabularx}{\textwidth}{QQ}
\lsptoprule
{情景} & {决定参数(根据重要程度排序)}\\
\midrule
二语文本理解 & 1、文本无个人或机密数据
2、神经网络机器翻译输出的质量(无需完美,够好就行)\\
\tablevspace

二语写作 & 1、教师同意
2、文本无个人或机密数据
3、二语水平够好(CEFR的B1或B2水平)
4、神经网络机器翻译输出的质量\\
\tablevspace

翻译作业 & 1、教师同意(不太可能,因为使用机器翻译的话,翻译作业就变成了译后编辑)
2、文本无个人或机密数据
3、神经网络机器翻译输出的质量\\
\tablevspace

二语口头汇报筹备 & 1、教师同意
2、内容不含个人或机密数据
3、二语水平够好(CEFR的B1或B2水平)
4、神经网络机器翻译输出的质量以及能够实现文本转语音输出\\
\lspbottomrule
\end{tabularx}
\caption{何时使用或不使用机器翻译}
\label{tab:carre:1}
\end{table}


\section{机器翻译与同类数字资源}
免费在线机器翻译工具使用便捷,还可以全文输入,你可能对结果满意,不想尝试其他方法了。但通常来说,机器翻译是不够的,甚至某些情况不适合使用这种工具。机器翻译与其他数字资源相比如何?接下来,我们会依次讨论下列的四个问题,解释机器翻译指的是什么。讨论机器翻译和其他在线工具的使用,不妨先从回答以下问题开始:

\begin{itemize}
\item 你用不用在线词典,用的话,是哪些词典?
\item 你如何定义语料库?
\item 你用过在线语料库吗?
\item 你知道什么是索引(concordance)吗?
\end{itemize}

下文将机器翻译与其他工具进行对比,目的在于帮助二语学习者和一般的外语使用者了解机器翻译除即时翻译以外的特性。我们想通过对比说明的关键点之一是,虽然机器翻译使用便捷,可进行即时翻译,但这并不意味着神经网络机器翻译工具在任何时候都是首选。


\subsection{机器翻译与在线词典}
一般来说,字典是“按字母顺序列出词语及其含义,或用另一种语言给出对应词语的书”(\citetitle{Cambridge2020}, \citeyear{Cambridge2020})。这个定义可延伸至电子词典,如在线词典或应用程序,这些产品通常为语言学习者所熟知。在线词典包括传统词典,如牛津英语词典(Oxford English Dictionary)\footnote{\url{https://www.oed.com/}} ,以及至少部分信息由群体创作的新形式,如维基词典(Wiktionary)\footnote{\url{https://en.wiktionary.org/},最后访问于2022年6月20日} 或市井词典(Urban Dictionary)。\footnote{\url{https://www.urbandictionary.com/},最后访问于2022年6月20日。}

不管是否经常使用这些在线词典,在如此丰富的资源面前,每个用户都会遇到的问题是:如何判断一本词典是否可靠?对此,辞书学(即词典的编纂)提供了最佳答案,并不断发展以考虑到新词典形式和格式的变化。现在许多词典都是基于大量可用作参考的文本集合(也称为语料库),但词典编纂者永远不会满足于仅仅引用语料库中的内容。语料库反而经常被用来检查语言的运用和寻找相关的例子。在这之前,词典编纂者的工作主要包括为给定的词典找到相关的词条,并给出恰当的释义。要更好地理解这一点,看看为学习者词典就知道了,里面包含了精挑细选的词条、定义和示例。现在,大多数词典都有免费在线版本,你应该很容易就找到一两本学习词典,查看最近学过或特别喜欢的单词。

还有一点很重要:Linguee网站声称内含一本词典,但具体是哪类词典呢?我们可以区分为“基于语料库”和“由语料库驱动”这两类词典。前者仍然依赖于词典编纂者的直觉,并根据词典编纂的方法创建,而后者则自动提取自语料库。Linguee内含的是由语料库驱动的双语词典,并不提供单词的定义或精心挑选的例子。

在查看在线语料库之前,我们先总结一下神经网络机器翻译译文和词典词条之间的主要区别:\largerpage[-2]

\begin{itemize}
\item  词典词条基于单个单词,而神经机器译文的文本长度不限。神经网络机器翻译引擎远不如字典有用,因为它针对单个单词的翻译通常不可靠。脚注{之所以特别指出这一点是因为,实际上,语言学习者和大学生经常使用免费在线机器翻译工具来查找单个单词的翻译(例子参见\citealt{JolleyMaimone2022, DorstBouman2022}}。
\item  词典提供定义,这可能是确保你理解词义的唯一可靠方法。
\item  词典词条基于人类的直觉,并且(大多时候)由词典编纂者设计和/或检查。
\item  神经机器译文基于语料库,但并非照搬训练语料库的内容(如\citetv{chapters/perez}所解释)。下节内容会更详细地阐述这种差异。
\end{itemize}



\subsection{机器翻译与在线语料库} 
在开始解释机器翻译和在线语料库的区别之前,至少要弄清楚两个定义。

首先,平行语料库是原文和译文的对齐文本集合。这种平行或对齐意味着每个片段(通常是一个句子)与其译文相对照。

索引工具用于在语料库中查找数据并显示结果。表\tabref{tab:carre:2}的示例选自英国议会议事录语料库(Hansard),即加拿大议会参议院和众议院辩论的平行语料库(英语和法语)。

首先要注意,索引是一种搜索功能。例如,我们查找一个有英语译文的法语短语,先出现的是法语(位于左侧),尽管语料库的这部分内容原文是英语(在线语料库不一定提供这一信息)。同样值得一提的是,双语索引通常会在原文语段中高亮搜索词(见表\tabref{tab:carre:2}加粗部分)。很多双语索引还会高亮与原文语段相对应的目的语语段,但这种基于概率的识别通常并不准确。事实上,我们已经“清理”了表\tabref{tab:carre:2}的目的语一栏,使得索引结果“对等”(参见\citetv[§1]{chapters/kenny})部分更明了。简而言之,句子层面的对齐或能精准,但双语或平行语料索引中更精细的语言单元对照有时无法自动识别。

还要注意,在线索引通常会显示每个索引所在的完整段落或文本,如表\tabref{tab:carre:2}所示,或者链接到语料库的相关部分。

相比之下,机器翻译只会提供输入内容的建议译文,没有其他语境元素。

\begin{table}
\begin{tabularx}{\textwidth}{QQ}
\lsptoprule
{法语译文} & {英语原文}\\
\midrule
Nous vous disons que \textit{nous allons faire le nécessaire}, mais aidez-nous à nous assurer que tout le monde respecte les règles.

Monsieur le Président, nous avons promis que \textit{nous allions faire le nécessaire} pour ratifier l'accord. & But we're saying, hey, \textit{we'll do it, we'll set it up}, but help us to make sure everybody abides by the rules.

Mr. Speaker, we promised that \textit{we were going to do what was required} to ratify the agreement.\\
\lspbottomrule
\end{tabularx}
\caption{语段 \textit{nous allons faire le nécessaire}在英国议会议事录语料库中的索引结果\label{tab:carre:2}}
\end{table}

\begin{table}
\begin{tabularx}{\textwidth}{QQ}
\lsptoprule
{法语检索} & {英语神经机器译文}\\
\midrule
Nous allons faire le nécessaire.

Nous allons faire le nécessaire pour ratifier l’accord. & We will do what is necessary.

We will take the necessary steps to ratify the agreement.\\
\lspbottomrule
\end{tabularx}
\caption{语段\textit{nous allons faire le nécessaire}的神经机器译文示例(机器译文于2021年11月1日出自 \url{https://www.bing.com/translator})}
\label{tab:carre:3}
\end{table}

表\tabref{tab:carre:3}比较两次检索,前者比后者更短,且表意更模糊。例子表明神经网络机器翻译引擎能根据句子语境作出调整,即有明确的目的状语时(\textit{pour ratifier l’accord}),英语机器译文使用不同的句子结构,其中\textit{faire le nécessaire}翻译成\textit{take the necessary steps to}。

总的来说,神经网络机器翻译引擎更适用于完整的句子(参见\citetv[§7]{chapters/kenny})或文本;而查找单词、搭配或短语时,词典和/或语料库能提供更有效可靠的受控结果。平行语料库能提供多种语境易于获取的译文;而机器译文则基于对训练数据的复杂计算,但用户有时无法访问这些数据(通常对免费在线机器翻译工具的用户不可见)。这使得用户很难确定建议译文是否真的可靠。


\section{错误分析}
利用机器翻译学习第二语言或外语所需的关键技能之一是能敏锐地识别错误。为培养这种技能而设计的活动有很多,但因为建议还很少,因此我们在下文为读者提供一个带有评论的例子。\footnote{有关如何将机器翻译融入语言教学的更多最新观点,请参阅\citet{VinallHellmich2022}。}

学生给一段文字及其译文列出自己能够识别并改正的错误类型,然后将列表交给老师。老师对此给出反馈,并指出学生没有识别出的所有错误,给出解释的同时,提出改进建议和有用的例子。如果目的语文本(机器译文)为第二语言,这项练习会有难度,因此我们建议老师从母语的机器译文开始。虽然如本章开头所述,此类任务很长一段时间都不用于语言课堂,但最近有学者提出将翻译融入情境任务,以期将学生成为“在日益多语言的世界中自我反思、具有跨文化能力和有责任的意义构建者”\citep[105]{Laviosa2014}。

表\tabref{tab:carre:4}所示为英翻法的神经机器译文示例。\footnote{文本选自法国学习者的英语教科书(\citealt{Joyeux2019}:22)。为了将此活动变成情境任务,学习者要给一个几乎不懂英语的法国人(例如特地来课堂观摩的人)提供合适译文。他们尽量不要从机器译文中获取信息,也不要对其进行修改。}机器译文中的错误为加粗部分,注释见下方。\footnote{机器译文的错误分析通常基于“错误类型”(error typologies),其中包括机器译文会出现的各类错误。这些错误通常包括准确度错误(如目标语段的意思与源语段的意思不一致)和影响目的语的流畅度或语言规范的错误(如语法一致、词序、搭配等错误)。更多信息请参阅\textcitetv{chapters/rossi}对机器翻译评测,以及\textcitetv{chapters/obrien}对译后编辑的阐述。}


\begin{table}
\begin{tabularx}{\textwidth}{L{5.1cm}L{6.5cm}}

\lsptoprule
{英语原文} & {法语神经机器译文}\\
\midrule
Hi there!

My teacher asked me to write and tell you what a typical day in my life looked like so I’ll do my best to give you an idea! I get up around 7:00 am, I have breakfast (two slices of toast and a cup of tea) then I get ready and put on my uniform. I make my lunch, and I double-check that my bag is packed. I leave my house at around 8:00 am. I’ve only got a 15-minute walk to school, so I arrive early. I usually chat with my friends, or listen to music with my headphones. Classes begin at 8:45 am. & Bonjour à tous !

Mon professeur m'a demandé de vous écrire et de vous raconter à quoi ressemblait \textit{une journée typique de ma vie}, alors je vais faire de mon mieux \textit{pour vous donner une idée} ! Je me lève vers 7 heures du matin, je prends mon petit déjeuner (deux tranches de pain grillé et une tasse de thé) puis je me prépare et je mets mon uniforme. Je prépare mon déjeuner, et je vérifie que mon sac \textit{est bien emballé.} Je quitte \textit{ma} maison vers 8 heures. \textit{Je n'ai que 15 minutes de marche pour me rendre à l'école, donc j'arrive tôt}. \textit{J'ai l'habitude} de discuter avec mes amis ou d'écouter de la musique avec mes écouteurs. Les cours commencent à 8h45.\\
\lspbottomrule
\end{tabularx}
\caption{教材片段摘录的NMT输出示例} 
\label{tab:carre:4}
\end{table}

例句的翻译错误包括过度直译,如\textit{une journée typique de ma vie}。法语译文最好使用复数形式,且typique应译成mes journées de lycéen(字面意思是“我当学生的日常”)。你可能注意到, a typical day这个短语不适合直译,因为\textit{a typical day}在源语中可算作固定表达。对于习语\textit{to give you an idea},也要在目的语中找个习语,如\textit{pour vous en donner un aperçu}。

还有些语法和语言使用的错误。法语需要en这类接语代词(clitic pronouns),如可以将donc j’arrive tôt变成donc j’y arrive en avance(意为“所以我提前到”)来改进译文。另外,在英语中,即便文中的所属关系不明显,也会使用所有格,但这种情况下,法语更喜欢使用限定词(如\textit{je quitte \textbf{la} maison}更倾向于译成I leave the house,而不是my house)。语言使用也与词汇有关,虽然英语通常指移动方式(\textit{a 15-minute walk to school},意为“15分钟步行到学校”),但法语的用法更中性(\textit{15 minutes de trajet pour l’école} ,意为“到学校要15分钟的路程”),只有在必要时才添加细节(如\textit{à pied},意为“步行”)。

\begin{sloppypar}
类似的例子不胜枚举,但希望以上例子足以说明,尽管法语机器译文看起来够好了,没有严重的语法或词汇错误,但仍有很大的改进空间。找出学习者能够和不能纠正的错误,肯定会对教师有所启发(关于这点可参阅 \citet{LoockLéchauguette2021})。
\end{sloppypar}

\newpage
我们鼓励读者获取自己语言对的机器译文,先译成母语。在情景任务中使用第二语言的神经机器译文的活动,可以用于后期的语言学习,尤其是识别和纠正译文错误的练习。

\section{结语}\label{sec:carre:7}
本章介绍了目前关于机器翻译在语言学习中的应用的主要研究发现,重点关注自神经网络机器翻译出现以来,有关机器翻译技术进步的研究成果。在建立\textcitetv{chapters/rossi}提出的质量测评的语用方法之前,我们还提供了一些在语言学习中使用机器翻译的基本技巧,重点关注它对第二语言和外语学习者的作用。与在翻译任务中可能无法选用哪种工具的专业译者不同,语言学习者有多种工具可供选用,但他们首先得决定是否和何时使用机器翻译。为此,我们提供了情景参数列表,帮助这些学习者做出选择。我们还将机器翻译与词典和语料库等在线辅助工具进行了对比,重点介绍两者的优点。最后,我们提出了如何利用神经网络机器翻译,并将其纳入第二语言或外语课堂的活动建议。

\printbibliography[heading=subbibliography,notkeyword=this]
\end{document}
