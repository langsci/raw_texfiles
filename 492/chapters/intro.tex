\documentclass[output=paper]{langscibook}
\ChapterDOI{10.5281/zenodo.14922283}
\author{Dorothy Kenny\orcid{0000-0002-4793-9256}\affiliation{都柏林城市大学}}
\title{前言}
\abstract{笔者在前言中阐述出版本书的缘由动机,并为读者提供使用建议。}

\begin{document}
\maketitle

\section{写作背景}
多语制是欧盟的基本价值观,实际上有赖于语言学习和翻译等多个支柱。近年来,这两大支柱都受到了机器翻译持续发展的深刻影响。如果使用得当,机器翻译不仅有助于语言学习,还有助于更多用户阅读更多语言的文本。条件适宜的情况下,机器翻译能显著提高职业译者的工作效率。但是,若不加批判地使用,这项技术不但有碍于个人学习语言,还会掩盖机器译文的潜在问题,导致用户无法意识到阅读的文本中包含机器翻译造成的误译或偏见。过度吹捧机器翻译,可能会吓退想要从事翻译工作的人。

为了利用好机器翻译,又无损多语制的其他支柱,我们认同\citet{BowkerCiro2019}的观点,即优秀多语公民的身份和翻译职业必须以\textit{机器翻译素养}为基础。针对不同群体,机器翻译素养的要求也不同。例如,\citet{BowkerCiro2019}着眼国际学术界来探讨机器翻译素养。本书的目标读者包括两个群体:第一,偶尔使用机器翻译的用户,即利用这项技术获取信息,或无意中用到机器译文,或以语言学习为目的的用户;第二,职业译者或正在接受翻译培训的人。我们认为,所有机器翻译使用者都应基本了解机器翻译技术的重要性及其对维护多语言制度的作用,还应该对其工作原理有所认识,这样才能避免犯常见错误,从而有效利用这项技术。而希望进一步掌握这项技术的用户,可以通过阅读本书学习如何使机器翻译的价值最大化,例如如何进行译前编辑以获取更优质的译文。这类用户可能还对如何改进机器译文感兴趣。而已经或即将进入翻译产业的用户会特别关注机器译文评测,以衡量译文是否“适用”。他们还可能参与将机器翻译融入公司工作流程,或者需要了解如何定制机器翻译引擎,以满足特定客户群体的需求。此外,他们还会关注机器翻译如何影响自己的工作条件。这类读者需要更深入了解机器翻译技术及其辅助技术和工具。所有用户都应该对在机器翻译使用过程中,由不同原因导致的各种道德伦理问题有所了解。有些用户担心滋生与机器翻译相关的作弊行为,例如学生在什么情况下使用机器翻译构成作弊?以专业译者为主的其他用户要考虑到,使用机器翻译可能造成违约。另外,如今大家都必须保护他人的隐私和数据权利。和许多其他技术一样,机器翻译也可能损害自然环境。而且众所周知,机器译文会带有偏见,例如阳性变体多于阴性变体。和所有通信技术一样,机器翻译技术也具有两面性。以上都是我们关心的问题。

\section{本书使用指南}
本书将引导读者探究上述所有问题。阅读本书不要求读者事先对普通翻译或专业机器翻译有所了解。涉及机器翻译的技术问题,特别是神经机器翻译(基于人工神经网络的最先进的翻译技术)时,我们会对这些技术进行描述,避免使用读者可能不熟悉的数学概念。本书尝试提供形象化的解释,特别是隐喻,以帮助读者理解这个领域的常见概念。这样一来,我们就能帮助读者轻松入门机器学习,展现这一领域的丰富内涵。总的来说,读者会觉得相比前几章,后面的章节及其后半部分的专业性更强。有些读者可以跳过其中一些章节,但即便机器翻译专家,或许也能从“非技术性”章节中受益,如讨论机器翻译与道德伦理的章节。

\section{本书结构}
本书以Olga Torres-Hostench在第一章中对多语制的讨论开篇,阐述其意义和实施方式,尤其是在欧盟内部的实施。她说明了为什么要将机器翻译整合到语言学习和翻译之中。第二章为Dorothy Kenny对翻译定义的概述,尤其是机器翻译,旨在对翻译进行祛魅。随后,她用简单易懂的方式介绍当代机器翻译方法的基本概念,如人工智能和机器学习。第三章由Caroline Rossi和Alice Carré共同撰写,主要介绍机器翻译测评这一具有重要学术和经济价值的领域。在第四章中,Pilar Sánchez-Gijón和Dorothy Kenny阐述了如何使文章更适用于机器翻译。Sharon O'Brien则在第五章介绍如何改进机器译文,即译后编辑。第六章由Joss Moorkens撰写,讨论了机器翻译使用过程中出现的道德伦理问题。第七章是本书专业性最强的一章。由Juan Antonio Pérez-Ortiz, Mikel Forcada和Felipe Sánchez-Martínez解释了神经机器翻译的工作原理,涵盖目前机翻系统最常用的基本技术。Gema Ramírez-Sánchez在第八章中介绍如何定制化神经机器翻译。最后一章则专门讨论机器翻译与语言学习,由Alice Carré、Dorothy Kenny、Caroline Rossi、Pilar Sánchez-Gijón和Olga Torres-Hostench共同撰写。

\section*{配套资源}
本书每章都设有互动练习,可访问MultiTraiNMT网站\url{http://www.multitrainmt.eu/}获取。这些活动的永久链接为\url{https://ddd.uab.cat/record/257869}。大多数活动可以自主完成,有些则在教师指导下进行为佳。

MultiTraiNMT项目还创建了MutNMT教学平台,旨在帮助用户学习如何训练、定制和测评神经机器翻译系统。用户可通过MultiTraiNMT网站进入该平台,以便更好学习本书的第七章和第八章。

\printbibliography[heading=subbibliography,notkeyword=this]

\end{document}
