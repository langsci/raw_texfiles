\title{机器翻译知识普及}
\subtitle{为人工智能时代的用户赋能}
\BackBody{语言学习与翻译素为欧盟多语言政策的核心支柱,二者相辅相成。正如意大利作家、符号学家翁贝托·埃科(Umberto Eco)所言:“欧洲的语言就是翻译”。然而,随着机器翻译和生成式人工智能等技术的不断发展和广泛运用,传统的语言学习和翻译模式正面临挑战。如今,语言学习者可以借助免费在线机器翻译工具来帮助理解和生成外语文本。但如果学习者盲目使用这些技术,反而可能弊大于利。尽管机器翻译被视为颠覆翻译行业的革命性技术,但其现有的工作原理犹如“黑匣子”,难以理解。此外,机器翻译还涉及法律和道德等问题,使得职业译者难以融入机器翻译的工作流程。

在这一背景下,本书面向语言学习者、语言教师、翻译学习者、翻译教师以及职业译者等广泛读者群体,旨在推动机器翻译技术在教学与实践中的应用。全书由多位作者共同编撰,首先阐明了学习机器翻译的理论基础,接着介绍了机器学习的基本原理,并深入分析了神经网络机器翻译的相关内容。书中还讨论了机器翻译技术所引发的道德问题,并提出了其在语言学习中的应用建议。此外,作者还探讨了通过译前编辑、译后编辑和定制化机器翻译等方法,帮助用户最大化地发挥机器翻译技术的潜力。}
\author{Dorothy Kenny}
\renewcommand{\lsSeries}{tmnlp}% use series acronym in lower case
\renewcommand{\lsSeriesNumber}{23}

\renewcommand{\lsID}{492}
\renewcommand{\lsISBNdigital}{978-3-96110-501-4}
\renewcommand{\lsISBNhardcover}{978-3-98554-130-0}
\BookDOI{10.5281/zenodo.14886855}
\typesetter{Sebastian Nordhoff}
\proofreader{洪文杰, Wilson Lui}
\translator{钟欣奕}
\lsCoverTitleSizes{47pt}{16mm}% Font setting for the title page
