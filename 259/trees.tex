%% -*- coding:utf-8 -*-
%%%%%%%%%%%%%%%%%%%%%%%%%%%%%%%%%%%%%%%%%%%%%%%%%%%%%%%%%
%%   $RCSfile: grammatiktheorie.tex,v $
%%  $Revision: 1.3 $
%%      $Date: 2010/01/18 14:55:27 $
%%     Author: Stefan Mueller (CL Uni-Bremen)
%%    Purpose: 
%%   Language: LaTeX
%%%%%%%%%%%%%%%%%%%%%%%%%%%%%%%%%%%%%%%%%%%%%%%%%%%%%%%%%


 
% Everything following a % is ignored
% Some lines start with %. Remove the % to include them

\documentclass[output=book
		,modfonts
		,nonflat
		,multiauthors
	        ,collection
	        ,collectionchapter
	        ,collectiontoclongg
 	        ,biblatex  
                ,babelshorthands
%                ,showindex
                ,newtxmath
                ,colorlinks, citecolor=brown % for drafts
                ,draftmode
% 	        ,coverus
		  ]{./langsci/langscibook}                              
%%%%%%%%%%%%%%%%%%%%%%%%%%%%%%%%%%%%%%%%%%%%%%%%%%%%

% put all additional commands you need in the 
% following files. If you do not know what this might 
% mean, you can safely ignore this section

\title{Methods in prosody:\, \newlineCover A Romance language perspective}  
\author{Ingo Feldhausen\and Jan Fliessbach\lastand Maria del Mar Vanrell} 
\renewcommand{\lsSeriesNumber}{6}  
% \renewcommand{\lsCoverTitleFont}[1]{\sffamily\addfontfeatures{Scale=MatchUppercase}\fontsize{38pt}{12.75mm}\selectfont #1}

\renewcommand{\lsISBNdigital}{978-3-96110-104-7}
\renewcommand{\lsISBNhardcover}{978-3-96110-105-4}

\renewcommand{\lsSeries}{silp}          
\renewcommand{\lsSeriesNumber}{6}
\renewcommand{\lsURL}{http://langsci-press.org/catalog/book/183}
\renewcommand{\lsID}{183}
\renewcommand{\lsBookDOI}{10.5281/zenodo.1471564}

\typesetter{Jan Fliessbach\lastand Felix Kopecky}
\proofreader{Adrien Barbaresi, Amir Ghorbanpour, Aysel Saricaoglu, Brett Reynolds, Conor Pyle, Daniela Kolbe-Hanna, Jeroen van de Weijer, Sebastian Nordhoff\lastand Varun deCastro-Arrazola}

\BackBody{This book presents a collection of pioneering papers reflecting current methods in prosody research with a focus on Romance languages. The rapid expansion of the field of prosody research in the last decades has given rise to a proliferation of methods that has left little room for the critical assessment of these methods. The aim of this volume is to bridge this gap by embracing original contributions, in which experts in the field assess, reflect, and discuss different methods of data gathering and analysis. The book might thus be of interest to scholars and established researchers as well as to students and young academics who wish to explore the topic of prosody, an expanding and promising area of study.}

\usepackage{langsci-optional}
\usepackage{langsci-gb4e}
\usepackage{langsci-lgr}

\usepackage{listings}
\lstset{basicstyle=\ttfamily,tabsize=2,breaklines=true}

%added by author
% \usepackage{tipa}
\usepackage{multirow}
\graphicspath{{figures/}}
\usepackage{langsci-branding}

%% hyphenation points for line breaks
%% Normally, automatic hyphenation in LaTeX is very good
%% If a word is mis-hyphenated, add it to this file
%%
%% add information to TeX file before \begin{document} with:
%% %% hyphenation points for line breaks
%% Normally, automatic hyphenation in LaTeX is very good
%% If a word is mis-hyphenated, add it to this file
%%
%% add information to TeX file before \begin{document} with:
%% %% hyphenation points for line breaks
%% Normally, automatic hyphenation in LaTeX is very good
%% If a word is mis-hyphenated, add it to this file
%%
%% add information to TeX file before \begin{document} with:
%% \include{localhyphenation}
\hyphenation{
affri-ca-te
affri-ca-tes
an-no-tated
com-ple-ments
com-po-si-tio-na-li-ty
non-com-po-si-tio-na-li-ty
Gon-zá-lez
out-side
Ri-chárd
se-man-tics
STREU-SLE
Tie-de-mann
}
\hyphenation{
affri-ca-te
affri-ca-tes
an-no-tated
com-ple-ments
com-po-si-tio-na-li-ty
non-com-po-si-tio-na-li-ty
Gon-zá-lez
out-side
Ri-chárd
se-man-tics
STREU-SLE
Tie-de-mann
}
\hyphenation{
affri-ca-te
affri-ca-tes
an-no-tated
com-ple-ments
com-po-si-tio-na-li-ty
non-com-po-si-tio-na-li-ty
Gon-zá-lez
out-side
Ri-chárd
se-man-tics
STREU-SLE
Tie-de-mann
}

\newcommand{\sent}{\enumsentence}
\newcommand{\sents}{\eenumsentence}
\let\citeasnoun\citet

\renewcommand{\lsCoverTitleFont}[1]{\sffamily\addfontfeatures{Scale=MatchUppercase}\fontsize{44pt}{16mm}\selectfont #1}
   


% if you want externalize graphics, compile files in chapters directly
% \forestset{external/readonly}
% you may switch this off in the final run since eternalization may cut off overlays.
\tikzexternalize

%\forestset{external/extract=extracted-trees}


\bibliography{bib-abbr,biblio}

%\bibliography{gt}

\begin{document}
\input extracted-trees.tex
\end{document}



\end{document}

%%% Local Variables:
%%% mode: latex
%%% TeX-master: t
%%% End:
