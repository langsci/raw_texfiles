\documentclass[output=paper,biblatex,babelshorthands,newtxmath,draftmode,colorlinks,citecolor=brown]{langscibook}
\ChapterDOI{10.5281/zenodo.13645043}

\author{Danièle Godard\orcid{0000-0001-7354-6264}\affiliation{Université de Paris, Centre national de la recherche scientifique
    (CNRS)} and Pollet Samvelian\orcid{0000-0002-6622-4117}\affiliation{Université Sorbonne Nouvelle}}

\title{Complex predicates}


% \chapterDOI{} %will be filled in at production

%\epigram{Change epigram in chapters/03.tex or remove it there }
\abstract{Complex predicates are constructions in which a head attracts arguments from its predicate complement. Auxiliaries, copulas, predicative verbs, certain control or raising verbs, perception verbs, causative verbs and light verbs can head complex predicates. This phenomenon has been studied in HPSG in different languages, including Romance and Germanic languages, Korean and Persian. They each illustrate different aspects of complex predicate formation. Romance languages show that argument inheritance is compatible with different phrase structures. German, Dutch and Korean show that argument inheritance can induce different word order properties, and Persian shows that a complex predicate can be preserved by a derivation rule (nominalization from a verb), and, most importantly in Persian, which has relatively few simplex verbs, that light verb constructions are used to turn a noun into a verb.} 

\IfFileExists{../localcommands.tex}{%hack to check whether this is being compiled as part of a collection or standalone
  \usepackage{../nomemoize}
  \usepackage{langsci-optional}
\usepackage{langsci-gb4e}
\usepackage{langsci-lgr}

\usepackage{listings}
\lstset{basicstyle=\ttfamily,tabsize=2,breaklines=true}

%added by author
% \usepackage{tipa}
\usepackage{multirow}
\graphicspath{{figures/}}
\usepackage{langsci-branding}

  
\newcommand{\sent}{\enumsentence}
\newcommand{\sents}{\eenumsentence}
\let\citeasnoun\citet

\renewcommand{\lsCoverTitleFont}[1]{\sffamily\addfontfeatures{Scale=MatchUppercase}\fontsize{44pt}{16mm}\selectfont #1}
    
  ../locallangscifixes.tex

  \togglepaper[11]
}{}


\IfFileExists{../localcommands.tex}{
\input complex-predicates-include.tex}{\input chapters/complex-predicates-include.tex}


%      <!-- Local IspellDict: en_US-w_accents -->
