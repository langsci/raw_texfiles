\addchap{List of abbreviations and feature names used in the book}

\begin{refsection}


\begin{longtable}{@{}p{3cm}p{9cm}@{}}
\feat{1st-pc} & first position class \\
\feat{accent} & accent \\
\feat{act(or)} & actor argument \\
\feat{addressee} & index for addressee \\
\feat{aff} & affixes \\
\feat{agr} & agreement \\
\feat{anaph} & anaphora \\
\feat{ancs} & anchors \\
\feat{antec} & antecedent referent markers \\
\feat{arg} & semantic argument of a relation \\
\feat{arg-st} & argument Structure \\
\feat{aux} & auxiliary verb (or not) \\
\feat{background} (\feat{backgr}) & background assumptions \\
\feat{bd} & boundary tone \\
\feat{bg} & background (in information structure) \\
\feat{body} & body (nuclear scope) of quantifier \\
\feat{case} & case \\
\feat{category} & syntactic category information \\
\feat{c-indices} (\feat{c-inds}) & contextual indices \\
\feat{cl} & inflectional class \\
\feat{clitic} (\feat{clts}) & clitics \\
\feat{conds} & predicative conditions \\
\feat{cluster} & cluster of phrases \\
\feat{coll} & collocation type \\
\feat{comps} & complements \\
\feat{concord} & concord information \\
\feat{content} (\feat{cont}) & lexical semantic content \\
\feat{context} (\feat{ctxt}) & contextual information \\
\feat{coord} & coordinator \\ 
\feat{correl} & correlative marker \\
\feat{det} & semantic determiner (a.k.a. quantifier force) \\
\feat{dsl} & double slash \\
\feat{deps} & dependents \\
\feat{dom} & order Domain \\
\feat{dr} & discourse referent \\
\feat{dte} & designated terminal element \\
\feat{dtrs} & daughters \\
\feat{econt} & external content \\
\feat{embed} & embedded (or not) \\
\feat{ending} & inflectional ending \\
\feat{exp} & experiencer \\
\feat{excont} (\feat{exc}) & external content (in LRS) \\
\feat{extra} & extraposed syntactic argument \\
\feat{fc} & focus-marked lexical item \\
\feat{fcompl} & functional complement \\
\feat{fig} & figure in a locative relation \\
\feat{first} & first member of a list \\
\feat{focus} & focus \\
\feat{form} & form of a lexeme \\
\feat{fpp} & focus projection potential \\
\feat{gend} & gender \\
\feat{given} & given information \\
\feat{grnd} & ground in a locative relation \\
\feat{ground} & ground (in information structure) \\
\feat{gtop} & global top \\
\feat{harg} & hole argument of handle constraints \\
\feat{hcons} & handle constraints (to establish relative scope in MRS) \\
\feat{head} (\feat{hd}) & head features\\
\feat{hd-dtr} & head-daughter \\
\feat{hook} & hook (relevant for scope relations in MRS) \\
\feat{ic} & inverted clause (or not) \\
\feat{icons} & individual constraints \\
\feat{icont} & internal content \\
\feat{i-form} & inflected form \\
\feat{index} (\feat{ind}) & semantic index \\
\feat{incont} (\feat{inc}) & internal content (in LRS) \\
\feat{infl} & inflectional features \\
\feat{info-struc} & information structure \\
\feat{inher} & inherited non-local features \\
\feat{inst} & instance (argument of an object category) \\
\feat{inv} & inverted verb (or not) \\
\feat{ip} & intonational phrase \\
\feat{key} & key semantic relation \\
\feat{lagr} & left conjunct agreement \\
\feat{larg} & label argument of handle constraints \\
\feat{lbl} & label of elementary predications \\
\feat{lex-dtr} & lexical daughter \\
\feat{lexeme} & lexeme identifier \\
\feat{lf} & logical form \\
\feat{lid} & lexical identifier \\
\feat{light} & light expressions (or not) \\
\feat{link} & link (in information structure) \\
\feat{listeme} & lexical identifier \\
\feat{liszt} & list of semantic relations \\
\feat{local} & syntactic and semantic information relevant in local contexts \\
\feat{l-periph} & left periphery \\
\feat{ltop} & local top \\
\feat{major} & major part of speech features  \\
\feat{major} & major or minor part of speech \\
\feat{main} & main semantic contribution of a lexeme \\
\feat{marking} (\feat{mrkg}) & marking \\
\feat{max-qud} & maximal question under discussion \\
\feat{mc} & main clause (or not) \\
\feat{$\mu$-feat} & morphological features \\
\feat{minor} & minor part of speech features \\
\feat{mkg} & information structure properties (marking) of lexical items \\
\feat{mod} & modified expression \\
\feat{modal-base} & modal modification of situation core \\
%\feat{mtr} & Mother \\
\feat{mood} & mood \\
\feat{morph} & morphology \\
\feat{morph-b} & morphological base \\
\feat{mp} & morphophonology \\
\feat{mph} & morphs \\
\feat{ms} & morphosyntactic (or morphosemantic) property set \\
\feat{mud} & morph under discussion \\
\feat{n} & nominal part of speech \\
\feat{neg} & negative expression \\
\feat{non-head-dtrs} (\feat{nh-dtrs}) & non-head daughters \\
\feat{nonlocal} & syntactic and semantic information relevant for non-local dependencies \\ 
\feat{nucl} (\feat{nuc}) & nucleus of a state of affairs  \\
\feat{numb} & number \\
\feat{params} & parameters (restricted variables) \\
\feat{pa} & pitch accent \\
\feat{parts} & list of meaningful expressions \\
\feat{pers} & person \\
\feat{pc} & position class \\
\feat{pform} & preposition form \\
\feat{phon} (\feat{ph}) & phonology \\
\feat{phon-string} & phonological string \\
\feat{php} & phonological phrase \\
\feat{pol} & polarity \\
\feat{pool} & pool of quantifiers to be retrieved \\
\feat{prd} & predicative (or not) \\
\feat{pred} & predicate \\
\feat{pref} & prefixes \\
\feat{pre-modifier} &  modifiers before the modified (or not) \\
\feat{prop} & proposition \\
\feat{quants} & list of quantifiers \\
\feat{qstore} & quantifier store \\
\feat{qud} & question under discussion \\
\feat{ques} & question \\ %Not sure what this does, p.396
\feat{ragr} & right conjunct agreement \\
\feat{realized} & realized syntactic argument \\
\feat{rel} & indices for relatives \\
\feat{rln} (\feat{reln}) & semantic relation \\
\feat{rels} & list or set of semantic relations \\
\feat{rest} & non-first members of a list \\
\feat{restr} & restriction of quantifier (in MRS) \\
\feat{restrictions} (\feat{restr}) & restrictions on index \\
\feat{retrieved} & retrieved quantifiers  \\
\feat{r-mark} & reference marker \\
\feat{root} & root clause or not \\
\feat{rr} & realizational Rules \\
\feat{sal-utt} & salient Utterance \\
\feat{select} (\feat{sel}) & selected expression \\
\feat{sit} & situation \\
\feat{sit-core} & situation core \\
\feat{slash} & set of locally unrealized arguments \\
\feat{soa} (\feat{soa-arg}) & state Of Affairs \\
\feat{speaker} & index for the Speaker \\
\feat{spec} & specified \\
\feat{spr} & specifier \\
\feat{status} & information structure status \\
\feat{stem} & stem phonology \\
\feat{stm-pc} & stem position class \\
\feat{store} & same as \feat{q-store} \\ %Check GS
\feat{struc-meaning} & structured meaning \\
\feat{subj-agr} & subject agreement \\
\feat{subcat} & subcategorization \\
\feat{synsem} & syntax/ Semantics features \\
\feat{subj} & subject \\
\feat{tail} & tail (in information structure) \\
\feat{tam} & tense, aspect, mood \\
\feat{tns} & tense \\
\feat{topic} & topic \\
\feat{tp} & topic-marked lexical item \\
\feat{und} & undergoer argument \\
\feat{ut} & phonological utterance \\
\feat{v} & verbal part of speech \\
\feat{val} & valence \\
\feat{var} & variable (bound by a quantifier) \\
\feat{vform} & verb form \\
\feat{weight} & expression weight \\
\feat{wh} & \emph{wh}-expression (for questions) \\
\feat{xarg} & extra-argument \\	
\end{longtable}



%\printbibliography[heading=subbibliography]
\end{refsection}

