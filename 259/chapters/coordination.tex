\documentclass[output=paper,biblatex,babelshorthands,newtxmath,draftmode,colorlinks,citecolor=brown]{langscibook}
\ChapterDOI{10.5281/zenodo.5599848}

\author{Anne Abeillé\orcid{0000-0002-9187-2298}\affiliation{Université de Paris} and Rui P. Chaves\orcid{0000-0003-0820-6145}\affiliation{University at Buffalo, SUNY}}

\title{Coordination} 

\abstract{Coordination is a central topic in theoretical linguistics. Following GPSG, which provided the first formal analysis of unlike coordination, HPSG has developed detailed analyses of different coordination constructions in a variety of unrelated languages. Central to the HPSG analyses are two main ideas: (i) coordination structures are non-headed phrases, and (ii) coordinate daughters display some kind of parallelism, which is captured by feature sharing. From these ideas, specific properties can be derived, regarding extraction and agreement, for instance. Many HPSG analyses also agree that coordination is a cover term for a wide variety of different constructions which can be viewed as different subtypes of coordinate phrases, and which can be cross-classified with other subtypes of the grammar (nominal or not, with ellipsis or not, etc.). We present the description of various coordination phenomena and show that HPSG can account for their subtle properties, while integrating them into the general organization of the grammar.}


\IfFileExists{../localcommands.tex}{%hack to check whether this is being compiled as part of a collection or standalone
  \usepackage{../nomemoize}
  \usepackage{langsci-optional}
\usepackage{langsci-gb4e}
\usepackage{langsci-lgr}

\usepackage{listings}
\lstset{basicstyle=\ttfamily,tabsize=2,breaklines=true}

%added by author
% \usepackage{tipa}
\usepackage{multirow}
\graphicspath{{figures/}}
\usepackage{langsci-branding}

  
\newcommand{\sent}{\enumsentence}
\newcommand{\sents}{\eenumsentence}
\let\citeasnoun\citet

\renewcommand{\lsCoverTitleFont}[1]{\sffamily\addfontfeatures{Scale=MatchUppercase}\fontsize{44pt}{16mm}\selectfont #1}
  
  ../locallangscifixes.tex

  \togglepaperminimal[16]
}{}

% you may switch off externalization of changed files here:
%\forestset{external/readonly}





\IfFileExists{../localcommands.tex}{
\input coordination-include.tex}{\input chapters/coordination-include.tex}
