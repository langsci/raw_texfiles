\documentclass[output=paper,biblatex,babelshorthands,newtxmath,draftmode,colorlinks,citecolor=brown]{langscibook}
\ChapterDOI{10.5281/zenodo.13644981}

\IfFileExists{../localcommands.tex}{%hack to check whether this is being compiled as part of a collection or standalone
  \usepackage{../nomemoize}
  \usepackage{langsci-optional}
\usepackage{langsci-gb4e}
\usepackage{langsci-lgr}

\usepackage{listings}
\lstset{basicstyle=\ttfamily,tabsize=2,breaklines=true}

%added by author
% \usepackage{tipa}
\usepackage{multirow}
\graphicspath{{figures/}}
\usepackage{langsci-branding}

  
\newcommand{\sent}{\enumsentence}
\newcommand{\sents}{\eenumsentence}
\let\citeasnoun\citet

\renewcommand{\lsCoverTitleFont}[1]{\sffamily\addfontfeatures{Scale=MatchUppercase}\fontsize{44pt}{16mm}\selectfont #1}
  
  %% hyphenation points for line breaks
%% Normally, automatic hyphenation in LaTeX is very good
%% If a word is mis-hyphenated, add it to this file
%%
%% add information to TeX file before \begin{document} with:
%% %% hyphenation points for line breaks
%% Normally, automatic hyphenation in LaTeX is very good
%% If a word is mis-hyphenated, add it to this file
%%
%% add information to TeX file before \begin{document} with:
%% %% hyphenation points for line breaks
%% Normally, automatic hyphenation in LaTeX is very good
%% If a word is mis-hyphenated, add it to this file
%%
%% add information to TeX file before \begin{document} with:
%% \include{localhyphenation}
\hyphenation{
affri-ca-te
affri-ca-tes
an-no-tated
com-ple-ments
com-po-si-tio-na-li-ty
non-com-po-si-tio-na-li-ty
Gon-zá-lez
out-side
Ri-chárd
se-man-tics
STREU-SLE
Tie-de-mann
}
\hyphenation{
affri-ca-te
affri-ca-tes
an-no-tated
com-ple-ments
com-po-si-tio-na-li-ty
non-com-po-si-tio-na-li-ty
Gon-zá-lez
out-side
Ri-chárd
se-man-tics
STREU-SLE
Tie-de-mann
}
\hyphenation{
affri-ca-te
affri-ca-tes
an-no-tated
com-ple-ments
com-po-si-tio-na-li-ty
non-com-po-si-tio-na-li-ty
Gon-zá-lez
out-side
Ri-chárd
se-man-tics
STREU-SLE
Tie-de-mann
}
  ../locallangscifixes.tex

  \togglepaper[32]
}{}
% you may switch off externalization of changed files here:
%\forestset{external/readonly}


\author{Stephen Wechsler\orcid{0000-0003-4753-0212}\affiliation{The University of Texas} and Ash Asudeh\orcid{0000-0002-0552-5231}\affiliation{University of Rochester\\ \& Carleton University}}
\title{HPSG and Lexical Functional Grammar}

\abstract{This chapter compares two closely related grammatical frameworks, Head-Driven Phrase Structure Grammar (HPSG) and Lexical Functional Grammar (LFG).   Among the similarities: both frameworks  draw a lexicalist distinction between morphology and syntax, both associate certain words with lexical argument structures, both employ semantic theories based on underspecification, and both are fully explicit and computationally implemented.  The two frameworks make available many of the same representational resources. Typical differences between the analyses proffered under the two frameworks can often be traced to concomitant differences of emphasis in the design orientations of their founding formulations: while HPSG's origins emphasized the formal representation of syntactic locality conditions,  those of LFG emphasized the formal representation of functional equivalence classes across grammatical structures. Our comparison of the two theories includes a point by point syntactic comparison, after which we turn to an exposition of Glue Semantics, a theory of semantic composition closely associated with LFG.}



\IfFileExists{../localcommands.tex}{
\input lfg-include.tex}{\input chapters/lfg-include.tex} 
