\documentclass[output=paper,biblatex,babelshorthands,newtxmath,draftmode,colorlinks,citecolor=brown]{langscibook}
\ChapterDOI{10.5281/zenodo.5599840}

\IfFileExists{../localcommands.tex}{%hack to check whether this is being compiled as part of a collection or standalone
  \usepackage{../nomemoize}
  \usepackage{langsci-optional}
\usepackage{langsci-gb4e}
\usepackage{langsci-lgr}

\usepackage{listings}
\lstset{basicstyle=\ttfamily,tabsize=2,breaklines=true}

%added by author
% \usepackage{tipa}
\usepackage{multirow}
\graphicspath{{figures/}}
\usepackage{langsci-branding}

  
\newcommand{\sent}{\enumsentence}
\newcommand{\sents}{\eenumsentence}
\let\citeasnoun\citet

\renewcommand{\lsCoverTitleFont}[1]{\sffamily\addfontfeatures{Scale=MatchUppercase}\fontsize{44pt}{16mm}\selectfont #1}
  
  ../locallangscifixes.tex
  \togglepaper[13]
}{}
%\forestset{external/readonly}


\author{Anne Abeillé\affiliation{Université de Paris}}
\title{Control and raising}

% \chapterDOI{} %will be filled in at production

%\epigram{Change epigram in chapters/03.tex or remove it there }
\abstract{The distinction between raising predicates and control predicates has been a hallmark of syntactic theory since the 60s. Unlike transformational analyses, HPSG treats the difference as mainly a semantic one: raising verbs (\word{seem}, \word{begin}, \word{expect}) do not semantically select their subject (or object) nor assign them a semantic role, while control verbs (\word{want}, \word{promise}, \word{persuade}) semantically select all their syntactic arguments. On the syntactic side, raising verbs share their subject (or object) with the unexpressed subject of their non"=finite complement, while control verbs only coindex them. The distinction between raising and control lexeme types is also relevant for non"=verbal predicates such as adjectives (\word{likely} vs.\ \word{eager}). The analysis of the complement of both control and raising verbs as phrasal, rather than clausal (or small clause), will be supported by creole data. The distinction between subject and first syntactic argument will be discussed together with data from ergative languages, and the HPSG analysis will be extended to cover cases of obligatory control of the expressed subject of some finite clausal complements in certain languages. The raising analysis naturally extends to copular constructions (\word{become}, \word{consider}) and most auxiliary verbs.} 


\IfFileExists{../localcommands.tex}{
\input control-raising-include.tex}{\input chapters/control-raising-include.tex}
