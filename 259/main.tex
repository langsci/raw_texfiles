%%%%%%%%%%%%%%%%%%%%%%%%%%%%%%%%%%%%%%%%%%%%%%%%%%%%
%%%                                              %%%
%%%     Language Science Press Master File       %%%
%%%         follow the instructions below        %%%
%%%                                              %%%
%%%%%%%%%%%%%%%%%%%%%%%%%%%%%%%%%%%%%%%%%%%%%%%%%%%%
 
% Everything following a % is ignored
% Some lines start with %. Remove the % to include them

\documentclass[output=book
                ,multiauthors
	        ,collection
                ,collectionchapter
 	        ,biblatex  
                ,babelshorthands
%                ,showindex
                ,newtxmath
                ,uniformtopskip % manual adjustment of pagebreaks
%                ,colorlinks, citecolor=brown % for drafts
%                ,draftmode
		  ]{langscibook}                              
%%%%%%%%%%%%%%%%%%%%%%%%%%%%%%%%%%%%%%%%%%%%%%%%%%%%

% put all additional commands you need in the 
% following files. If you do not know what this might 
% mean, you can safely ignore this section

\usepackage{langsci-optional}
\usepackage{langsci-gb4e}
\usepackage{langsci-lgr}

\usepackage{listings}
\lstset{basicstyle=\ttfamily,tabsize=2,breaklines=true}

%added by author
% \usepackage{tipa}
\usepackage{multirow}
\graphicspath{{figures/}}
\usepackage{langsci-branding}

\title{Methods in prosody:\, \newlineCover A Romance language perspective}  
\author{Ingo Feldhausen\and Jan Fliessbach\lastand Maria del Mar Vanrell} 
\renewcommand{\lsSeriesNumber}{6}  
% \renewcommand{\lsCoverTitleFont}[1]{\sffamily\addfontfeatures{Scale=MatchUppercase}\fontsize{38pt}{12.75mm}\selectfont #1}

\renewcommand{\lsISBNdigital}{978-3-96110-104-7}
\renewcommand{\lsISBNhardcover}{978-3-96110-105-4}

\renewcommand{\lsSeries}{silp}          
\renewcommand{\lsSeriesNumber}{6}
\renewcommand{\lsURL}{http://langsci-press.org/catalog/book/183}
\renewcommand{\lsID}{183}
\renewcommand{\lsBookDOI}{10.5281/zenodo.1471564}

\typesetter{Jan Fliessbach\lastand Felix Kopecky}
\proofreader{Adrien Barbaresi, Amir Ghorbanpour, Aysel Saricaoglu, Brett Reynolds, Conor Pyle, Daniela Kolbe-Hanna, Jeroen van de Weijer, Sebastian Nordhoff\lastand Varun deCastro-Arrazola}

\BackBody{This book presents a collection of pioneering papers reflecting current methods in prosody research with a focus on Romance languages. The rapid expansion of the field of prosody research in the last decades has given rise to a proliferation of methods that has left little room for the critical assessment of these methods. The aim of this volume is to bridge this gap by embracing original contributions, in which experts in the field assess, reflect, and discuss different methods of data gathering and analysis. The book might thus be of interest to scholars and established researchers as well as to students and young academics who wish to explore the topic of prosody, an expanding and promising area of study.}

%% hyphenation points for line breaks
%% Normally, automatic hyphenation in LaTeX is very good
%% If a word is mis-hyphenated, add it to this file
%%
%% add information to TeX file before \begin{document} with:
%% %% hyphenation points for line breaks
%% Normally, automatic hyphenation in LaTeX is very good
%% If a word is mis-hyphenated, add it to this file
%%
%% add information to TeX file before \begin{document} with:
%% %% hyphenation points for line breaks
%% Normally, automatic hyphenation in LaTeX is very good
%% If a word is mis-hyphenated, add it to this file
%%
%% add information to TeX file before \begin{document} with:
%% \include{localhyphenation}
\hyphenation{
affri-ca-te
affri-ca-tes
an-no-tated
com-ple-ments
com-po-si-tio-na-li-ty
non-com-po-si-tio-na-li-ty
Gon-zá-lez
out-side
Ri-chárd
se-man-tics
STREU-SLE
Tie-de-mann
}
\hyphenation{
affri-ca-te
affri-ca-tes
an-no-tated
com-ple-ments
com-po-si-tio-na-li-ty
non-com-po-si-tio-na-li-ty
Gon-zá-lez
out-side
Ri-chárd
se-man-tics
STREU-SLE
Tie-de-mann
}
\hyphenation{
affri-ca-te
affri-ca-tes
an-no-tated
com-ple-ments
com-po-si-tio-na-li-ty
non-com-po-si-tio-na-li-ty
Gon-zá-lez
out-side
Ri-chárd
se-man-tics
STREU-SLE
Tie-de-mann
}

\newcommand{\sent}{\enumsentence}
\newcommand{\sents}{\eenumsentence}
\let\citeasnoun\citet

\renewcommand{\lsCoverTitleFont}[1]{\sffamily\addfontfeatures{Scale=MatchUppercase}\fontsize{44pt}{16mm}\selectfont #1}
   

../locallangscifixes.tex
%% -*- coding:utf-8 -*-

% St. Mü. 13.06.2021
%
% This is a file containing temporary fixes.
%
% They should be removed for further releases when the packages have been updated.
%

\makeatletter
% fix for https://github.com/plk/biblatex/issues/1149
% https://tex.stackexchange.com/q/600943/35864
% remove this when biblatex is updated to v3.17
\renewbibmacro*{related:default}[1]{%
  \entrydata*{#1}{%
    \usedriver
      {\ifnameundef{savedauthor}
         {\ifnameundef{savededitor}
            {}
            {\ifnamesequal{editor}{savededitor}
               {\clearname{editor}}
               {}}}
         {\ifnamesequal{author}{savedauthor}
            {\clearname{author}}
            {}}%
       \DeclareNameAlias{sortname}{default}%
       % from authortitle and authoryear
       \ifbibmacroundef{bbx:dashcheck}
         {}
         {\renewbibmacro*{bbx:dashcheck}[2]{##2}}%
       % authoryear
       \ifbibmacroundef{labeltitle}
         {}
         {\renewbibmacro*{labeltitle}{}}%
       \ifbibmacroundef{date+extradate}
         {}
         {\renewbibmacro*{date+extradate}{}%
          \renewbibmacro*{bbx:ifmergeddate}{\@secondoftwo}}%
       \renewbibmacro*{pageref}{}%
       \renewbibmacro*{related:init}{}}
      {\thefield{entrytype}}}}
\makeatother





% This should go to the langsci citation file langsci-unified.cbx
\DeclareCiteCommand{\citeauthor}
  {\boolfalse{citetracker}%
   \boolfalse{pagetracker}%
   \usebibmacro{prenote}}
  {\usebibmacro{citeindex}%
   \printtext[bibhyperref]{\printnames{labelname}}}
  {\multicitedelim}
  {\usebibmacro{postnote}}


% This should be in langscibook.cls
% Necessary for Tsujii, Jun’ichi
\renewbibmacro*{citeindex}{%
  \ifciteindex
    {\iffieldequalstr{labelnamesource}{shortauthor} % If biblatex uses shortauthor as the label of a bibitem
      {\ifnameundef{authauthor}                     % we check whether there is something in authauthor
        {\indexnames{author}}                       % if not, we use author
        {\indexnames{authauthor}}}                  % if yes, we use authauthor
      {\iffieldequalstr{labelnamesource}{author}    % if biblatex uses author we similarly test for
                                                    % authauthor and use this field
        {\ifnameundef{authauthor}% if defined use authauthor
          {\indexnames{author}}
          {\indexnames{authauthor}}} % if defined use this field
% same for editor
        {\iffieldequalstr{labelnamesource}{shorteditor} % If biblatex uses shorteditor as the label of a bibitem
          {\ifnameundef{autheditor}                     % we check whether there is something in autheditor
            {\indexnames{editor}}                       % if not, we use editor
            {\indexnames{autheditor}}}                  % if yes, we use autheditor
          {\iffieldequalstr{labelnamesource}{editor}    % if biblatex uses editor we similarly test for
                                                        % auteditor and use this field
            {\ifnameundef{autheditor}% if defined use autheditor
              {\indexnames{editor}}
              {\indexnames{autheditor}}}} % if defined use this field
          {\indexnames{labelname}}}}}               % as a fallback we index on whatever biblatex used.
    {}}


%\renewcommand{\itsdb}{}

%% -*- coding:utf-8 -*-

\bibliography{Bibliographies/stmue,localbibliography}




\usepackage{nomemoize} 
\memoizeset{
  memo filename prefix={chapters/hpsg-handbook.memo.dir/},
  register=\todo{O{}+m},
  prevent=\todo,
}

%\memoizeset{readonly}

% This adds a checked field to the automatically generated footer items for crossreference. If they
% are cited, one can print the checked field (as done in check-hpsg) to see quickly that these items
% are ok.
% This is not needed for main, but without it it destroys information needed for other tasks in the project.
\patchcmd{\lsCollectionMetadataToBibliography}{\immediate\write\tempfile{@incollection{#1,author={\authorTemp},title={{\expandonce{\titleTemp}}},booktitle={{\expandonce{\lsCollectionTitle}}},editor={\editorTemp},publisher={Language Science Press.},Address={Berlin},year={\lsYear},pages={\lsCollectionPaperFirstPage--\lsCollectionPaperLastPage},doi={\lsChapterDOI},keywords={withinvolume}}}}{\immediate\write\tempfile{@incollection{#1,author={\authorTemp},title={{\lsCollectionPaperFooterTitle}},booktitle={{\lsCollectionTitle}},editor={\editorTemp},publisher={Language Science Press.},Address={Berlin},series={Empirically Oriented Theoretical Morphology and Syntax},year={2021},pages={\lsCollectionPaperFirstPage--\lsCollectionPaperLastPage},checked={auto created},doi={\lsChapterDOI},keywords={withinvolume}}}}



%% -*- coding:utf-8 -*-


\usepackage{xassoccnt}
\newcounter{realpage}
\DeclareAssociatedCounters{page}{realpage}
\AtBeginDocument{%
  \stepcounter{realpage}
}


%%%%%%%%%%%%%%%%%%%%%%%%%%%%%%%%%%%%%%%%%%%%%%%%%%%%
%%%                                              %%%
%%%             Frontmatter                      %%%
%%%                                              %%%
%%%%%%%%%%%%%%%%%%%%%%%%%%%%%%%%%%%%%%%%%%%%%%%%%%%%

\begin{document}

\memoizedisable
\maketitle
\frontmatter
\memoizeenable

% %% uncomment if you have preface and/or acknowledgements

\currentpdfbookmark{Contents}{name} % adds a PDF bookmark
\tableofcontents
\addchap{Preface}
\begin{refsection}

%content goes here
 
% \printbibliography[heading=subbibliography]
\end{refsection}


%% -*- coding:utf-8 -*-
\section*{Foreword of the second edition}

The second edition comes with a lot of small improvements: the index has been improved, typos have
been fixed, and ORCIDs were added to authors and are displayed on the title pages of the papers now.

% order: 04.01.22
% Gray -> gray
% der Frau -> dem Kind
% added \ref{ex-schema-hc-flat-synsem-sign}

% complex-predicates 04.01.22
% unified synsems2signs. The relation has the same name now in order.tex and complex-predicates.tex

% relative-clauses.tex 05.01.22
% \trace -> \trace{}
% glosses aligned in {x:rc-129}
% added language tag
% fixed index entry for Bavarian German


% 18.01.22 added language info for German examples

% 25.01.22 Footnote~\ref{fn-hf-schema} was missing. % in udc

% 03.02.22 Idioms: NP in (8) too much, REL bad feature name, ref to Krenn&Erbach added

% 08.02.22 Information structure: added page numbers for Bildhauer & Cook 2010
%          fixed layout issue with Head-Dislocation Schema for Catalan
% 09.02.22 Added sentence about diff-list and reference to copestake2002.
%
% 14.02.22 Added glosses to helfen in chapter on processing
%
% 30.03.22 Figure 4, Mary is NP not N
%
% 26.10.22 (38) used to be phrase => but since the constraint referred to HD-DTR this would cause a
% conflict for unheaded phrases. Noticed by student.
% The left-hand daughter in (38b) must be SYNSEM X, noted by St.Mü.


~\medskip

\noindent
Berlin, \today\hfill Stefan Müller, Anne Abeillé, Robert D. Borsley \& Jean-​Pierre Koenig


%      <!-- Local IspellDict: en_US-w_accents -->

\addchap{Acknowledgments} 
%content goes here
The help and support of Martin Haspelmath and Sebastian Nordhoff in the preparation of this volume is gratefully acknowledged. 

We would also like to thank the authors of the chapters in this volume for their cooperation during the editing process and especially for their input to the reviewing of chapters by their peers. 

We especially thank the following additional external reviewers, %individuals, 
who contributed their time and expertise to provide independent peer review for the papers in this collection: Lisa Bonnici, Jason Brown, Elisabet Engdahl, Marieke Hoetjes, Beth Hume, Anne O'Keefe, Adam Schembri, Thomas Stolz, Andy Wedel and Shuly Wintner.
 

\addchap{Abbreviations}
\begin{tabular}{ll}
CR & Common Room     \\
NCR & non-Common Room        \\
SGH & Selwyn Girls' High     \\
The & BBs: The Blazer Brigade \\
\isi{The PCs} & The Palms Crew  \\
\end{tabular}

%% Additional prefaces and/or introductions that also have authors
%\lsCollectionPaperFrontmatterMode % Enter the Frontmatter Mode. 
%\includepaper{chapters/prefaceEd}
% \includepaper{chapters/prefaceEd2}
%\lsCollectionPaperMainmatterMode % Leave the Frontmatter Mode from pre 2020 version
\setcounter{chapter}{0} % Reset the chapter counter so that preceeding prefaces are not counted
%%
\mainmatter          
\typeout{mainmatter starts at \therealpage}

%%%%%%%%%%%%%%%%%%%%%%%%%%%%%%%%%%%%%%%%%%%%%%%%%%%%
%%%                                              %%%
%%%             Chapters                         %%%
%%%                                              %%%
%%%%%%%%%%%%%%%%%%%%%%%%%%%%%%%%%%%%%%%%%%%%%%%%%%%%
 
\part{Introduction}  

\includepaper{chapters/properties} %add a percentage sign in front of the line to exclude this chapter from book
\includepaper{chapters/evolution} 
\includepaper{chapters/formal-background}
\includepaper{chapters/lexicon}
\includepaper{chapters/understudied-languages}
 
\part{Syntactic phenomena}

\includepaper{chapters/agreement}
\includepaper{chapters/case}
\includepaper{chapters/np}
\includepaper{chapters/arg-st}
\includepaper{chapters/order}
%\includepaper{chapters/clitics}
\includepaper{chapters/complex-predicates}
\includepaper{chapters/control-raising}
\includepaper{chapters/udc}
\includepaper{chapters/relative-clauses}
\includepaper{chapters/islands}
\includepaper{chapters/coordination}
\includepaper{chapters/idioms}
\includepaper{chapters/negation}
\includepaper{chapters/ellipsis}
\includepaper{chapters/binding}
%
\part{Other levels of description}

%\includepaper{chapters/phonology}
\includepaper{chapters/morphology}
\includepaper{chapters/semantics}
\includepaper{chapters/information-structure}
%
\part{Other areas of linguistics}

%\includepaper{chapters/acquisition}
\includepaper{chapters/processing}
\includepaper{chapters/cl}
\includepaper{chapters/pragmatics}
%\includepaper{chapters/sign-lg}
\includepaper{chapters/gesture}
%
\part{The broader picture}
\includepaper{chapters/minimalism}
\includepaper{chapters/cg}
\if0
\fi
\includepaper{chapters/lfg}
\includepaper{chapters/dg}
\includepaper{chapters/cxg}
%
% copy the lines above and adapt as necessary

%%%%%%%%%%%%%%%%%%%%%%%%%%%%%%%%%%%%%%%%%%%%%%%%%%%%
%%%                                              %%%
%%%             Backmatter                       %%%
%%%                                              %%%
%%%%%%%%%%%%%%%%%%%%%%%%%%%%%%%%%%%%%%%%%%%%%%%%%%%%

% There is normally no need to change the backmatter section
\memoizedisable

\backmatter 
\phantomsection 
\addcontentsline{toc}{chapter}{\lsIndexTitle} 
\addcontentsline{toc}{section}{\lsNameIndexTitle}
\ohead{\lsNameIndexTitle} 
\printindex 
\cleardoublepage
  
\phantomsection 
\addcontentsline{toc}{section}{\lsLanguageIndexTitle}
\ohead{\lsLanguageIndexTitle} 
\printindex[lan] 
\cleardoublepage
  
\phantomsection 
\addcontentsline{toc}{section}{\lsSubjectIndexTitle}
\ohead{\lsSubjectIndexTitle} 
\printindex[sbj]
\ohead{} 


\end{document} 


%%%%%%%%%%%%%%%%%%%%%%%%%%%%%%%%%%%%%%%%%%%%%%%%%%%%
%%%                                              %%%
%%%                  END                         %%%
%%%                                              %%%
%%%%%%%%%%%%%%%%%%%%%%%%%%%%%%%%%%%%%%%%%%%%%%%%%%%%

% you can create your book by running
% xelatex main.tex
%
% you can also try a simple 
% make
% on the commandline

%%% Local Variables:
%%% mode: latex
%%% TeX-master: t
%%% TeX-engine: xetex
%%% End:
