%% -*- coding:utf-8 -*-
\documentclass[output=book
		,modfonts
		,nonflat
	        ,collection
	        ,collectionchapter
	        ,collectiontoclongg
 	        ,biblatex  
                ,babelshorthands
%                ,showindex
                ,newtxmath
                ,colorlinks, citecolor=brown 
                ,draftmode
% 	        ,coverus
		  ]{langscibook}                              
%%%%%%%%%%%%%%%%%%%%%%%%%%%%%%%%%%%%%%%%%%%%%%%%%%%%

% put all additional commands you need in the 
% following files. If you do not know what this might 
% mean, you can safely ignore this section

\title{Methods in prosody:\, \newlineCover A Romance language perspective}  
\author{Ingo Feldhausen\and Jan Fliessbach\lastand Maria del Mar Vanrell} 
\renewcommand{\lsSeriesNumber}{6}  
% \renewcommand{\lsCoverTitleFont}[1]{\sffamily\addfontfeatures{Scale=MatchUppercase}\fontsize{38pt}{12.75mm}\selectfont #1}

\renewcommand{\lsISBNdigital}{978-3-96110-104-7}
\renewcommand{\lsISBNhardcover}{978-3-96110-105-4}

\renewcommand{\lsSeries}{silp}          
\renewcommand{\lsSeriesNumber}{6}
\renewcommand{\lsURL}{http://langsci-press.org/catalog/book/183}
\renewcommand{\lsID}{183}
\renewcommand{\lsBookDOI}{10.5281/zenodo.1471564}

\typesetter{Jan Fliessbach\lastand Felix Kopecky}
\proofreader{Adrien Barbaresi, Amir Ghorbanpour, Aysel Saricaoglu, Brett Reynolds, Conor Pyle, Daniela Kolbe-Hanna, Jeroen van de Weijer, Sebastian Nordhoff\lastand Varun deCastro-Arrazola}

\BackBody{This book presents a collection of pioneering papers reflecting current methods in prosody research with a focus on Romance languages. The rapid expansion of the field of prosody research in the last decades has given rise to a proliferation of methods that has left little room for the critical assessment of these methods. The aim of this volume is to bridge this gap by embracing original contributions, in which experts in the field assess, reflect, and discuss different methods of data gathering and analysis. The book might thus be of interest to scholars and established researchers as well as to students and young academics who wish to explore the topic of prosody, an expanding and promising area of study.}

\usepackage{langsci-optional}
\usepackage{langsci-gb4e}
\usepackage{langsci-lgr}

\usepackage{listings}
\lstset{basicstyle=\ttfamily,tabsize=2,breaklines=true}

%added by author
% \usepackage{tipa}
\usepackage{multirow}
\graphicspath{{figures/}}
\usepackage{langsci-branding}

%% hyphenation points for line breaks
%% Normally, automatic hyphenation in LaTeX is very good
%% If a word is mis-hyphenated, add it to this file
%%
%% add information to TeX file before \begin{document} with:
%% %% hyphenation points for line breaks
%% Normally, automatic hyphenation in LaTeX is very good
%% If a word is mis-hyphenated, add it to this file
%%
%% add information to TeX file before \begin{document} with:
%% %% hyphenation points for line breaks
%% Normally, automatic hyphenation in LaTeX is very good
%% If a word is mis-hyphenated, add it to this file
%%
%% add information to TeX file before \begin{document} with:
%% \include{localhyphenation}
\hyphenation{
affri-ca-te
affri-ca-tes
an-no-tated
com-ple-ments
com-po-si-tio-na-li-ty
non-com-po-si-tio-na-li-ty
Gon-zá-lez
out-side
Ri-chárd
se-man-tics
STREU-SLE
Tie-de-mann
}
\hyphenation{
affri-ca-te
affri-ca-tes
an-no-tated
com-ple-ments
com-po-si-tio-na-li-ty
non-com-po-si-tio-na-li-ty
Gon-zá-lez
out-side
Ri-chárd
se-man-tics
STREU-SLE
Tie-de-mann
}
\hyphenation{
affri-ca-te
affri-ca-tes
an-no-tated
com-ple-ments
com-po-si-tio-na-li-ty
non-com-po-si-tio-na-li-ty
Gon-zá-lez
out-side
Ri-chárd
se-man-tics
STREU-SLE
Tie-de-mann
}

\newcommand{\sent}{\enumsentence}
\newcommand{\sents}{\eenumsentence}
\let\citeasnoun\citet

\renewcommand{\lsCoverTitleFont}[1]{\sffamily\addfontfeatures{Scale=MatchUppercase}\fontsize{44pt}{16mm}\selectfont #1}
   

\usepackage{lipsum}

\begin{document}

\maketitle                
\frontmatter

\currentpdfbookmark{Contents}{name} % adds a PDF bookmark
%%
\mainmatter          


\avm{
[ 
            test & x\\
            fin  & $+$ ]
}\avm{
[ subj & [ pred & \predvall{{pro}} \\ 
                     num & \vall{sg} \\ 
                     pers & \vall{3} \\ 
                     gend & \vall{fem} ] \\
            aspect  & present ]
}


\avm{
[ pred & \predvall{chase $\leftangle$(\lfgfst{f} \LFGfeat{subj})(\lfgfst{f} \LFGfeat{obj})$\rightangle$}\smallskip\\
            subj & [  pred & \predvall{child} \\
		      num  & \vall{dual} \\
		      case & \vall{erg} \\
		      adj  & \{ [  pred & \predvall{small} ] \} ] \\
obj & [pred & \predvall{dog} \\
       num  & \vall{sg}\\
       case & \vall{abs} ] \\
tense  & \vall{nonpast} \\
aspect & \vall{present.imperfect} ]
}

\avm{
\lfgfst{f}[ pred & \predvall{chase $\leftangle$(\lfgfst{f} \LFGfeat{subj})(\lfgfst{f} \LFGfeat{obj})$\rightangle$}\smallskip\\
            subj & [  pred & \predvall{child} \\
		      num  & \vall{dual} \\
		      case & \vall{erg} \\
		      adj  & \{ [  pred & \predvall{small} ] \} ] \\
obj & [pred & \predvall{dog} \\
       num  & \vall{sg}\\
       case & \vall{abs} ] \\
tense  & \vall{nonpast} \\
test & $+$\\
aspect & present.imperfect\\
aspect & $-$\\
%test & $+$ 
]
}

\avm{
[ aspect & $-$\\
  fin & $+$ ]} \avm{
[ aspect & perfect\\
  fin & $+$ ]} 



\if0

AVMs are feature value pairs:
\ea
\avm{
[
a1 & v1\\
a2 & v2\\
a3 & v3 ]}
\z
The values can be simple as in (\mex{0}) or complex as in (\mex{1}):
\ea
\avm{
[
a1 & [ a11 & v11\\
       a12 & v12]\\
a2 & [ a21 & v21\\
       a22 & v22]\\
a3 & v3 ]}
\z

AVMs may have a type. The type is like an atomic value and typeset in italics.
\ea
\avm{
[\type*{t1}
a1 & v1\\
a2 & v2\\
a3 & v3 ]}
\z

\ea
\avm{
[
a1 & [\type*{t11} 
       a11 & v11\\
       a12 & v12]\\
a2 & [ a21 & v21\\
       a22 & v22]\\
a3 & v3 ]}
\z
Values can be lists:
\ea
\avm{
[
a1 & [\type*{t11} 
       a11 & < v111, v112 >\\
       a12 & v12]\\
a2 & [ a21 & v21\\
       a22 & v22]\\
a3 & v3 ]}
\z

The elements in the list can be quite large and the brackets should adapt to this:
\ea
	\scalebox{.95}{
\avm{
	[\type*{mrs}
		hook &	[\type*{hook}
			ltop & \1 \\
			index & \2 ] \\
		rels & <[\type*{relation}
				pred & \_the\_q \\
				lbl & \3 \\
				arg0 & \4 \\
				rstr & \5 \\
				body & \6 ],
			[\type*{relation}
				pred & compound \\
				lbl & \7 \\
				arg0 & \8 \\
				arg1 & \4 \\
				arg2 & \9 ],
			[\type*{relation}
				pred & udef\_q \\
				lbl & \tag{10} \\
				arg0 & \9 \\
				rstr & \tag{11} \\
				body & \tag{12} ], \\
			[\type*{relation}
				pred & \_cherry\_n\_1 \\
				lbl & \tag{13} \\
				arg0 & \9 ],
			[\type*{relation}
				pred & \_tree\_n\_of \\
				lbl & \7 \\
				arg0 & \4 ],
			[\type*{relation}
				pred & \_blossom\_v\_1 \\
				lbl & \1 \\
				arg0 & \2 \\
				arg1 & \4 ] 
				\\ > \\
		hcons & <[\type*{qeq}
				harg & \5 \\
				larg & \7 ],
			[\type*{qeq}
				harg & \tag{11} \\
				larg & \tag{13} ] > ]
}
}
\z


Values can be structure shared by tags. These tags are visualized by numbered tags:
\ea
\avm{
[
a1 & [\type*{t11} 
       a11 & \1\\
       a12 & v12]\\
a2 & [ a21 & \2\\
       a22 & \1]\\
a3 & \2 ]}
\z


Values can be related by functions:
\ea
\avm{
[
a1 & [\type*{t11} 
       a11 & f(\1)\\
       a12 & v12]\\
a2 & [ a21 & \2\\
       a22 & \1]\\
a3 & \2 ]}
\z
 
It is also possible to combine values with operators:
\ea
\avm{
[
a1 & [\type*{t11} 
       a11 & \1\\
       a12 & v12]\\
a2 & [ a21 & \1 \+ \2]\\
a3 & \2 ]}
\z

Lists can be combined by operators, usually $\oplus$ (append) is used, but $\bigcirc$ (shuffle) is also sometimes used:
\ea
\avm{
[
a1 & [\type*{t11} 
       a11 & < v111 > \+ < v112 >\\
       a12 & v12] ]}
\z

Relations are usually typeset in a special font:
\ea
\avm{
[
a1 & [\type*{t11} 
       a11 & \texttt{collect-indices}(\1, \2)\\
       a12 & v12] ]}
\z

Relations can also be put next to the AVM:
\ea
\avm{
[
a1 & [\type*{t11} 
       a11 & \3\\
       a12 & v12] ] $\wedge$ \texttt{collect-indices}(\1, \2, \3)}
\z

It may be good to experiment with embedding brackets as in (\mex{1}b). It seems that this is
possible with delimfactor and delimfall, but maybe this can be discussed in the manual
\eal
\ex
\avm{
[
a1 & [ a11 & [ a111 & v111 ] ] ]}
\ex
\ms{
a1 & \ms{ a11 & \ms{ a111 & v111 } 
} 
}
\zl



The following AVMs are complex AVMs the type names should not sit too low at the opening
bracket. The AVMs in (\mex{1}) are typeset with different settings for delimfactor and
delimfall. These keys should be explained in the maual in more detail. delimfactor=980 and
delimfall=8pt seem to work best. They should be made the default.
\ea
\avm{
	[\type*{sign}
	phon & \list*(phon)\\
	synsem &	[\type*{synsem}
				local &	[\type*{local}
						category &	[\type*{category}
									head & part-of-speech\\
									subj & \list*(synsem)\\
									comps & \list*(synsem)]\\
						content & \ldots\\
						context & \ldots]\\
				nonlocal \ldots]
	]
}
\avm[delimfactor=950, delimfall=5pt]{
	[\type*{sign}
	phon & \list*(phon)\\
	synsem &	[\type*{synsem}
				local &	[\type*{local}
						category &	[\type*{category}
									head & part-of-speech\\
									subj & \list*(synsem)\\
									comps & \list*(synsem)]\\
						content & \ldots\\
						context & \ldots]\\
				nonlocal \ldots]
	]
}
\avm[delimfactor=980, delimfall=8pt]{
	[\type*{sign}
	phon & \list*(phon)\\
	synsem &	[\type*{synsem}
				local &	[\type*{local}
						category &	[\type*{category}
									head & part-of-speech\\
									subj & \list*(synsem)\\
									comps & \list*(synsem)]\\
						content & \ldots\\
						context & \ldots]\\
				nonlocal \ldots]
	]
}
\z

The last example seems to be about right. The manual should explain what the delimfactor and
delimfall values do and give some examples.


Special cases:


\begin{enumerate}
\item % Issue \#49 (order.tex)
The distance after numerated examples with AVMs should be the same as with other numbered examples.

\begin{exe}\ex\label{x:rc-79}  Lexical rule for propositional attitude predicates in \ili{French}\\
\avm{
	[ % cat & \[ head & v \]\\
	comps & < CP[slash & \{ \1 [\type*{prl}
		                    cont|index & \2 ] \} $\cup$ \3 ]  > \+ \ldots\\
        slash & \{ \1 \} $\cup$ \4 ]
}
$\mapsto$\\
\flushright
\avm{
	[slash & \{ [\type*{nprl}
		     cat & \upshape PP\subscr{de \2} ] \} $\cup$ \4 ]
}
\end{exe}%
\lipsum[1][1-4]
\ea
This is a text example.
\z
\lipsum[1][1-4]


\item 
\label{brackets-too-large}
%\#59, \#115: 
The brackets () should have the size of the element they are enclosing. 
%In the example (\mex{1}) they are too tall.
\ea
\avm{
[\type*{passive-verb}
 arg-st  & \2 ( \+ < ! PP[\type{by}]$_i$ ! > )]
}
\z

\ea 
\avm{
\type{coord-phrase} \impl\\
	[\punk{synsem|cat}{\1} \\
	dtrs & <[synsem|loc|cat & \1 ]> \+ ne-list([synsem|loc|cat & \1 ]) ]
}\label{coordparam2}
\z

\eal
\ex
\avm{
\emph{simple-coord-phrase} \impl \\
	[dtrs & ne-list(<[coord & none]>) \+ ne-list([coord & \1 crd ]) ]
}

\ex
\avm{
\emph{omnisyndetic-coord-phrase}  \impl \\
	[dtrs & ne-list([coord & \1 crd ]) ]
}\label{omni}
\ex
\avm{
\emph{asyndetic-coord-phrase}  \impl \\
	[dtrs & ne-list([coord & none ]) ]
}
\zl



\item 
%\#64: spacing between values is different when \verb+\punk+ is involved

The space after an attribute should be the same independent of the usage of \verb+\punk+.
\ea
\avm{
\type{nom-coord-phrase} \impl
	[\punk{synsem|loc|cont|index}{ \1} \\
	 dtrs & < [ ], \ldots, [synsem|loc|cont|index & \1 ] > ]
}
\z

\item 
%\#65:

There should be a way to have three lines in curly brackets with the brackets being adjusted to the
longest line rather to the last one.
\ea
\avm{
  [
   constraints : & \{overlap(\7,\tag{10}) $\lor$ follow(\7,\tag{10},\texttt{4}s) \\
                     total-time(\7,\tag{10},\3) \\ 
                     assign-modality(\6,\9,\2)
                    \}
   ]
}
\z

\item 
%\#70: 
AVMs without alignment are needed very often. There can be cases where the alignment is needed
on some levels but not at all levels. For example, the following AVM cannot be properly done with
\verb+punk+. And \verb+punk+ is in the way of a WYSIWYG encoding.
\eal
\ex 
% \avm{
% [\punk{mother|syn|val|comps}{< >} \\
% head-dtr & \0	[\type*{word}
%                 syn|val|comps & \tag{A} ] \\
% dtrs & < \0 > \+ \tag{A} nelist ]
% }
\avm{
[mother|syn|val|comps < > \\
 head-dtr  \0 [\type*{word}
               syn|val|comps & \tag{A} ] \\
 dtrs  < \0 > \+ \tag{A} \type{nelist}~~~~~~~~~~~~~~~~~~~~ ]
}
\ex
\avm{
[mother|syn|val|comps < > \\
 head-dtr  \0 [\type*{word}
               syn|val|comps & \tag{A} ] \\
 dtrs  < \0 > \+ \tag{A} \type{nelist} ]
}

% \ex \avm{
% [\type*{sem-obj}
% index  & s \\
% frames & <[\type*{eating-fr}
% 	   label & l\\
% 	   sit   & s\\
% 	   ingestor & i\\
% 	   ingestible & j]> ]
% }
% \ex \avm{
% [\phon < I > \\
% content|index & \1 [per & 1st \\
% 		    num & sg ] \\
% context &  !$speaker(\1)$! 
% ]
% }

% \ex 

% \avm{
% [ head & \1\\
%   head-dtr & [ head & \1 ] ]
% }


\zl

% Desired encoding: (with \verb+\phon+ working)
% \eal
% \ex \avm{
% [phon < I > \\
% content|index  \1 [per & 1st \\
% 		   num & sg ] \\
% context   !$speaker(\1)$! ]
% }


% \ex \avm{
% [\type*{sign}
% phonology list!(phon\_string)! \\
% synsem 	[local &	[category & category \\
% 						content & content \\
% 						qstore & set!(quantifier)! \\
% 						pool & set!(quantifier)! ] ] \\
% retrieved  list!(quantifier)! ]
% }
%\zl

\item 
%\#71 alignment and escape via !

%The ! was used to escape the normal layout. I used it to get normal text brackets.

The ! should work even when \verb+punk+ is used. Maybe this feature is not
necessary when \verb+punk+ is not needed anylonger.
\ea
\avm{
	[\type*{reg-bar-adj}
	\punk{phon}{\1 \+ \phonliste{ bar }} \\
	\punk{morph-b}{<[\type*{trans-verb}
			  phon \; \1 \\
			  synsem|loc  [cat|comps & < NP ![\type{acc}]!\ind{2} > \+ \3 \\
			               cont|nuc  & \4 [act & index \\
			                               und & \2] ] ] >} \\
	synsem|loc &	[cat	[head  & adj \\
				 subj  & < NP\ind{2} > \\
				 comps & \3 ] \\
			 cont|nuc  [reln    & $\diamond$ \\
				    und     & \2 \\
				    soa-arg & \4 ] ] ]
}
\z

\item %\#78 
The distance between two avms below each other should be 2pts. The concernes the two AVMs of type
\type{fact} in the example below:

\ea
\avm{
	[\type*{scope-obj}
	index & \1 \\
	restr &	\{
			[\type*{fact}
			prop|soa  &	[\type*{soa}
						nuc &	[\type*{person}
								instance & \1 ] ] ], \\
            [\type*{fact}
            prop|soa &	[\type*{soa}
            			nuc &	[\type*{speak\_to}
            					speaker & Kim\\
            					addressee & \1 ] ] ]
      		\} 
	] 
}
\z

\item 
%\#79: 
The box around numbers in tags is considered for the alignment. Not the content of the box.

\ea
\avm{\emph{head-filler-phrase} \impl\\
[slash  & \1 \\
 hd-dtr & [head & verbal \\
 	   slash & \{ \2 \} $\uplus$ \1 ] ]
% \punk{non-hd-dtrs}{ <[local &  \2 ]>} ]
}
\z
(Siede remark by Stefan: I am not sure whether I want the box to be aligned or the number.)


\item 
%\#81: The text is too high in \verb+< >+.

The text in \verb+< >+ should be on the same baseline as the sourrounding text and the brackets
should be centered. The expected layout is shown in (\mex{1}b). This is also related to point
\ref{brackets-too-large}. In general brackets should be as small as possible.
\eal
\ex \avm{
[subj  & < PRO >]
}
\ex
\avm{
[subj  & \sliste{ PRO }]
}
\zl
This also applies to empty brackets.
\eal
\ex \avm{
[subj  & < >]
}
\ex
\avm{
[subj  & \sliste{}]
}
\zl

\eal
\ex
\avm{
[pre & :	[qud=< > & : poset!(Question)!] ]
}
\ex
\avm{
[pre & :	[qud=\sliste{} & : poset!(Question)!] ]
}
\zl



\item %\#85: 
AVMs in type hierarchies should be aligned at the top. \type{noun} should be at the same height as \type{verb}.

\begin{forest} 
type hierarchy
[object,
    calign=midpoint, calign children={2}{3},
    [\avm{[\type*{substantive} 
           prd & boolean]}
      [\avm{[\type*{verb}
             vform & vform\\
             prd & plus]}]
      [\avm{[\type*{noun} 
             case & case]}]
    ]
    [case]
    [vform]
    [boolean
      [plus]
      [minus]
    ]
  ]
\end{forest}

\item 
% \#92: 
The angle brackets should consider space needed for commas if there are linebreaks. If possible the
angle brackets should span the whole enclosd AVMs.
\ea
\avm{
[\punk{synsem|loc|cont|index}{[num & pl \\
	                  gend & [m & \1 \+ \2 ] \\
                          per & [me  & \3 \+ \5\\
                              you & \4 \+ \6 ] ]} \\
 dtrs &	< [synsem|loc|cont|index & [gend & [m & \1] \\
                       	            per & [me  & \3 \\
                                           you & \4] ] ],\\
	  [synsem|loc|cont|index & [gend & [m & \2] \\
                                    per & [me  & \5 \\
                                           you & \6] ] ]  >
	]
}
\z
%The brackets seem to not see the comma. Apart from this they are too small (should span the whole AVMs)


\item 
%\#93 
Something is needed for aligning AVM at top. \verb+\atop+ does not seem to work for all cases.

There has to be a way to make both AVMs in the following example to appear at the top right below the heading. \verb+\attop+ does not work since it puts the following AVM on the next line.
\ea
\label{entryfori}
Simplified lexical signs for  \textit{I} and \textit{she}: \\*
\avm{
[\phon < I > \\
head &	[\type*{noun}
		pers & 1st \\
		num & sg ] ] 
}\hspace{1cm}
\avm{
[\phon  < she > \\
head &	[\type*{noun}
		pers & 3rd \\
		num & sg \\
		gen & fem ] ] 
}
\z
Felix provided solution with minipage, but could there be an option for calling \verb+\avm+? And if
so, could forest enable this by default?


\item 
%\#94: 

There should be a special italics correction for italics text if brackets are following.

The \emph{I} in (\mex{1}) should not touch the bracket.

\ea
\avm{
[\phon < I > \\
head &	[\type*{noun}
		pers & 1st \\
		num & sg ] ] 
}
\z

Felix: This is normal in \LaTeX. Stefan: Yes, but there is italics correction and it has to be
applied here. Always. Maybe this can be doen automatically.


\item 
%\#97: 
There should be a default behaviour in which type names are moved up a bit. This should not apply to types including letters like \emph{t} or \emph{l}.

\ea
% \avm[delimfactor=950, delimfall=5pt]{
% [\type*{sign}
% phonology & list!(phon\_string)!  ]
% }
\avm{
[\type*{sign}
phonology & list!(phon\_string)!  ]
}
\avm{
[\type*{head}
 vform & fin ]
}
\z

Felix: ``I don't think this is a good idea. For one, consider what would happen if two types are
placed in one line (e.g. two AVMs beside each other) and then the baselines of these types would not
be identical. This would be terrible.''

``Maybe what you really want isn't so much messing around with individual letters, but rather setting tighter options for the enclosing bracket delimiter, e.g.: \verb+\avm[delimfactor=950, delimfall=5pt] {}+.''

Interestingly the delimfactor does not seem to make a difference for simple AVMs like the ones in
(\mex{0}), but it does affect the ones at the very beginning.


\item 
%\#99: 
There should not be more space after operators like \verb+\uplus+.


\ea
\avm{
\emph{wh-inter-cl}:
[cont & [params & \{ \1 \} $\uplus$ set] ]
}
\z

%And this is \#106:
The space after operators should not depend on where they appear in the AVM (here \verb-\+-):
\ea
\avm{
[comps & \1 \+ <[loc & \4 [cat [head & verb $\lor$ prep \\
			         comps & < > ] ] ] > \+ \2  ]}
\z
%There was different space after the two $\oplus$. This is fixed. But the space before and after
%$\oplus$ is still different.

% This probably is due to the space added by brackets.


The space before and after $\lor$ should not differ:
\ea
\avm{
[ subcat & < > $\lor$ < \type{synsem} > ]}
\z



\item 
%\#101: space is missing due to exclamation marks in avms

The use of exclamation marks should not remove spacing before the exclamation mark.
\ea
\avm{
	[phon & < \rm{every, dog, barks} >\\
	\ldots\ cont & [main & \tag{4a} {\normalfont \sffamily bark} ] \\
	sem &	[exc & \2 !{\normalfont \sffamily $\forall$x ($\phi$,$\psi$)}! \\
			inc & \4 !{\normalfont \sffamily bark(x)}!\\
			pts & <\2, \tag{2a}, \tag{2b}, \3, \tag{3a}, \4, \tag{4a}>]]
}
\z
%The exclamation mark seems to eat the space. Felix: Hardcode space.




\item 
%\#108: 
Punk should not cause extra newlines:

\ea
\avm{
[arg-st & <[arg & \upshape NP![\type{str}]! \\
            \punk{realized}{$+$} ]> ] }
\z 

\item 
%\#109: 
The distance betweeen lines sould be eaqual when normal brackets are used. The brackets should have
the size of normal brackets in the respective font and hence the distance between the lines should
be harmonic.

\ea
\avm{
		[\type*{phrase}
		phon & !\list*(phon)!\\
		synsem & synsem\\
		dtrs & !\list*(sign)! ]
	}
\z

What I want here is the normal brackets: (). The distance between \phon and \synsem is bigger than
between \synsem and \dtrs. This is how it should look like:

\ea
\avm{
		[\type*{phrase}
		phon & \upshape !\textit{list}(\textit{phon})!\\
		synsem & synsem\\
		dtrs & \upshape !\textit{list}(\textit{sign})! ]
	}
\z
I guess the problem is due to the fact that normal brackets are not used but bigger brackets.


\item The space between \ibox{1} and NP should not be larger than the space between bracket and \ibox{1}.

\ea
\avm{
	[arg-st & < \1 NP, \2 NP > \\
	valence & < \1, \2 > \\
	gap & < > ]
}
\z

This is probably related to the fact that the box is ignored for layout purposes.

\item The brackets used by \verb+\list*+ should be the normal text brackets if the content is a single
  line.
% #126

\ea
\avm{
[\type*{sign}
phonology & \list*(phon\_string) \\
synsem    &	[local & [category & category \\
			  content  & content \\
			  qstore   & \avmset*(quantifier) \\
			  pool     & \avmset*(quantifier) ] ] \\
retrieved & \list*(quantifier) ]
}
\z

\end{enumerate}

\fi

\end{document} 

%%%%%%%%%%%%%%%%%%%%%%%%%%%%%%%%%%%%%%%%%%%%%%%%%%%% 
%%%                  END                         %%% 
%%%%%%%%%%%%%%%%%%%%%%%%%%%%%%%%%%%%%%%%%%%%%%%%%%%% 

%%% Local Variables:
%%% mode: latex
%%% TeX-master: t
%%% TeX-engine: xetex
%%% End:
