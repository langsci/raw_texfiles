\Phantom
\addcontentsline{toc}{part}{Vorbemerkungen zur ersten Auflage}

\chapter*{Vorbemerkungen zur ersten Auflage}
\markboth{Vorbemerkungen zur ersten Auflage}{Vorbemerkungen zur ersten Auflage}

\section*{Über dieses Buch}

Typische Leser dieser Einführung sind Studierende der Germanistik bzw.\ der Deutschen Philologie, die sich vor dem Studium oder währenddessen einen Überblick über wichtige grammatische Phänomene des Deutschen verschaffen möchten.%
\footnote{Im gesamten Buch wird das generische Maskulinum (\textit{Leser}, \textit{Sprecher}, \textit{Hörer} usw.) verwendet.
Als grammatische Form soll es dabei immer Personen aller Gender bezeichnen.}
Darüber hinaus ist dieses Buch für jeden geeignet, der sich für die deutsche Sprache und ihre Grammatik interessiert.
Eine sehr gute (erstsprachliche oder nahezu erstsprachliche) Kenntnis des Deutschen ist dabei angesichts der Anlage des Buchs von Vorteil.
Es ist also kein ideales Buch für all jene, die Deutsch lernen wollen und dabei noch nicht sehr weit fortgeschritten sind.

Man kann sich nun die Frage stellen, warum man überhaupt eine Einführung in die Grammatik einer Sprache benötigt, die man bereits sehr gut (ggf.\ sogar als Erstsprache) beherrscht.
Es gibt mehrere Antworten auf diese Frage.
Einerseits ist die Beschäftigung damit, wie wir Sprache benutzen, warum und wozu wir das tun, und wie die Sprache genau beschaffen ist, von grundlegendem Interesse.
Die Linguistik betreibt sozusagen in vielen Bereichen Grundlagenforschung.
Das Interesse für diese grundlegenden Fragen ist ein guter Grund, sich die eigene Sprache, die man täglich meist ohne große Reflexion benutzt, einmal analytisch und genau anzusehen.

Andererseits reden Menschen oft und gerne über die Grammatik ihrer eigenen Sprache -- nicht nur, aber natürlich besonders häufig, wenn die Frage nach dem \textit{richtigen Sprachgebrauch} aufkommt.
Gerade Studierende und Lehrende der Germanistik sollten von Berufs wegen in der Lage sein, an solchen Diskursen informiert teilzunehmen.
Dabei ist es hilfreich, zu wissen, was im System der Grammatik die Zusammenhänge (\zB zwischen Formenbildung und Satzbau) sind, was die Regeln und was die Ausnahmen sind.
Schon die Frage, warum es denn \textit{ein roter Ballon} aber \textit{der rote Ballon} heißt, ist nicht trivial zu beantworten, denn immerhin lautet es einmal \textit{roter} und einmal \textit{rote}, obwohl es sich doch in beiden Fällen um einen Nominativ Singular Maskulinum handelt.
Der Bedarf an einem guten Überblick über die Grammatik wird größer, wenn typische Berufsfelder für Germanisten ins Auge gefasst werden, \zB der Lehrberuf an Schulen, die Fremdsprachendidaktik oder das Verlagswesen.
Gerade wenn es darum geht, anderen Menschen zu vermitteln, wie bestimmte Dinge auf Deutsch gesagt oder geschrieben werden, oder darum, Texte anderer Personen grammatisch zu überprüfen, kann man sich nicht einfach darauf zurückziehen, dass man ja selber weiß, wie es richtig heißt.
Es entsteht der Bedarf an einer \textit{Erklärung}.

Daher wird in dieser Einführung ein Überblick über die wichtigen grammatischen Phänomene des Deutschen gegeben.
Gleichzeitig wird vermittelt, wie man Laute, Buchstaben, Wörter und Sätze einer Sprache, die man bereits beherrscht, so in ein System bringen kann, dass man mit anderen darüber reden kann.
Es geht sozusagen darum, aus sprachlichem Material die Regeln und die Ausnahmen zu den Regeln zu extrahieren und sie uns bewusst zu machen.
Ohne Vorkenntnisse durch bloßes Nachdenken in vertretbarer Zeit Fragen wie die folgenden präzise zu beantworten, ist vergleichsweise schwer:
Wieviele und welche Silben hat das Deutsche?
Warum gibt es Wörter wie \textit{bohrt}, aber prinzipiell keine Wörter wie \textit{bohrp}?
Was sind die typischen Betonungsmuster der Wörter?
Wie kann man am einfachsten beschreiben, wie bei Nomina der Kasus markiert wird?
Warum gibt es überhaupt verschiedene Kasus?
Was ist der Unterschied zwischen starken und schwachen Verben?
Nach welchen Regeln funktioniert die Satzgliedstellung im Deutschen?
Welche Verben kann man ins Passiv setzen?
Schreiben wir so, wie wir sprechen?
Warum gibt es in der deutschen Schrift doppelte Konsonanten?
Was ist die Funktion der Satzzeichen?

Dieses Buch ergänzt andere Literatur, die oft im Studium eingesetzt wird, oder bereitet auf diese vor.
Anspruchsvolle Grammatiken wie der \textit{Grundriss der deutschen Grammatik} von Peter Eisenberg \citep{Eisenberg1,Eisenberg2} oder die Duden"=Grammatik \citep{Duden8} werden durch die Lektüre dieses Buchs besser lesbar, weil große Teile der grammatischen Terminologie bereits bekannt sind, und weil der Umgang mit sprachlichem Material bereits kleinteiliger geübt wurde, als es in wissenschaftlichen Grammatiken möglich und zielführend ist.
Hat man einen solchen Überblick nicht, kann es bereits Schwierigkeiten bereiten, in umfangreichen Grammatiken wie der Duden-Grammatik den richtigen Abschnitt zu einer konkreten grammatischen Fragestellung zu finden.
Einführungen, die einen Überblick über viele Teildisziplinen der germanistischen Linguistik geben -- \zB \citealp{EgL07,SgL07} --, oder Einführungen in spezifische Grammatiktheorien -- \zB \citealp{Grewendorf02, Mueller08, Sternefeld2008, Sternefeld2009} -- sind nach der Lektüre dieser Einführung besser verständlich, da grammatische Kategorien detailliert an konkretem Material bereits erklärt wurden und dadurch auf andere Phänomene, andere Teilbereiche der Linguistik oder formalere Theorien bezogen werden können.

\section*{Benutzung dieses Buchs}

Das Buch ist für einen primären Arbeitszyklus \textit{Lesen--Unterrichten--Üben} konzipiert.
Das Gelesene soll durch den Unterricht gefestigt werden, die wichtigsten Punkte dabei herausgestellt werden und Probleme und Fragen erörtert werden.
Die in kurzen Sätzen gehaltenen Zusammenfassungen am Ende jedes Kapitels sollen vor allem eine Gedächtnisstütze zwischen dem Lesen und dem Unterrichten sein.

Praxis in der Anwendung der beschriebenen Analyseverfahren kann durch die zahlreichen Übungen im Anschluss an die Kapitel erworben werden.
Die Übungen zu den einzelnen Kapiteln werden fast alle im Anhang gelöst.%
\footnote{Anmerkung zur zweiten Auflage:
Der Anteil der Aufgaben, für die eine Lösung angeboten wird, ist in der zweiten Auflage rückläufig, insbesondere bei Transferaufgaben.
Auch wenn dies für das Selbststudium evtl.\ ungünstig ist, ist es für die Arbeit im Seminar zuträglich.}
Dadurch wird es im Selbststudium ermöglicht, den eigenen Lernerfolg zu kontrollieren.
Das Niveau der Übungen wird dabei durch \onestar\ (einfache Übungen), \twostar\ (Übungen auf Klausurniveau) und \tristar\ (Transferaufgaben) gekennzeichnet.

Bezüglich der Unterscheidung von hervorgehobenen Definitionen und Sätzen gilt, dass Definitionen für den argumentativen Gesamtaufbau grundlegend sind, während Sätze nur hilfreiche Schlussfolgerungen, nicht-definierende Eigenschaften und Tendenzen usw.\ formulieren.

Mit fortgeschrittenen Argumentationen bzw.\ interessanten, aber in der gegebenen Situation weniger zentralen Informationen wird folgendermaßen verfahren:
Bis zu einer Länge von ungefähr drei Sätzen werden sie in Fußnoten gesetzt.
Darüber hinaus und bis zu einer Länge von ungefähr einer Seite werden sie in deutlich abgesetzte Vertiefungsblöcke gesetzt.
Noch längere Vertiefungen werden als optionale Abschnitte des Textes gesetzt und sind im Titel mit \Opsional markiert.%
\footnote{Anmerkung zur zweiten Auflage:
Es gibt jetzt nur noch ein einheitliches Format für vertiefende Abschnitte.}
Das Buch kann auch selektiv gelesen werden, zumal wenn ein bisschen Hin- und Herblättern zwischen den Kapiteln in Kauf genommen wird.
Besonders die fünf Teile sind sehr gut einzeln lesbar.

Die Abschnitte mit weiterführender Literatur am Ende jedes der fünf Teile des Buchs stellen keine exhaustive Bibliographien dar.
Es werden spezialisierte Einführungen und exemplarisch einige (möglichst gut lesbare und möglichst auf Deutsch verfasste) Fachartikel und Monographien genannt, die interessierten Lesern den weiteren Einstieg in die Lektüre wissenschaftlicher Primärtexte ermöglichen sollen.%
\footnote{Leser sollten damit rechnen, dass in den angeführten Werken ggf.\ andere Positionen als in diesem Buch vertreten werden.}
Im Text wird aus Gründen der Übersichtlichkeit nur in Einzelfällen auf weitere Literatur verwiesen, \zB wenn nicht-triviale Beispiele, Listen, Paradigmen usw.\ direkt aus anderen Werken übernommen wurden.

\section*{Danksagungen zur ersten Auflage}

Ich möchte zunächst den Initiatoren und Unterstützern von Language Science Press, den Reihenherausgebern -- Martin Haspelmath und Stefan Müller -- sowie Sebastian Nordhoff als Koordinator von Language Science Press dafür danken, dass sie die Bedingungen dafür geschaffen haben, dass das Buch als Open-Access-Publikation erscheinen kann.
Stefan Müller danke ich darüber hinaus für eine sehr gründliche Durchsicht und eine Fülle von Kommentaren und Anmerkungen, die erheblich zur Verbesserung des Buchs beigetragen haben.
Besonderer Dank gilt Eric Fuß, der als Gutachter nicht nur einem offenen Begutachtungsprozess zugestimmt hat, sondern auch zahlreiche Anmerkungen und Kommentare gegeben hat, die ich fast vollständig übernehmen konnte.

Weiterhin danke ich den Teilnehmern und Tutorinnen meiner Lehrveranstaltungen an der Freien Universität Berlin in den Jahren 2008--2014, die viel Rückmeldung zu früheren Versionen des Textes gegeben haben.
Insbesondere sind zu nennen (in alphabetischer Reihenfolge) Sarah Dietzfelbinger, Ana Draganovi\'{c}, Lea Helmers, Theresia Lehner, Kaya Triebler, Samuel Reichert, Johanna Rehak, Cynthia Schwarz, Jana Weiß.
Für ihre Hilfe bei der Endredaktion in letzter Minute danke ich Kim Maser.
Für ausführliche Kommentare zu diversen Fassungen danke ich (in alphabetischer Reihenfolge) Götz Keydana, Michael Job, Bjarne Ørsnes, Ulrike Sayatz, Julia Schmidt, Nicolai Sinn.
Götz danke ich auch besonders für sie slawischen Beispiele, Nicolai besonders für die Diskussionen und Anregungen zu Kapitel~\ref{sec:nominalflexion}.
Ulrike danke ich vor allem für kontroverse Diskussionen zur und Nachhilfe in Graphematik, ohne die ich Teil~\ref{part:schrift} nicht hätte schreiben können.
Bjarne und Nicolai waren wichtige moralische Unterstützer und inhaltliche Ratgeber in der Frühphase des Projektes.
Felix Bildhauer danke ich für die Zusammenstellung der \textit{Außenseiter}-Sätze in Kapitel~\ref{sec:grammatik} (S.\ \pageref{ex:dg9113}).

Beim Redigieren des fertigen Buchs leuchtete Thea Dittrich.

Für alle Fehler und Inadäquatheiten, die jetzt noch im Buch verblieben sind, trage ich die alleinige Verantwortung.
Ich hätte eben besser auf all die oben genannten Personen hören sollen.

Für eine prägende Einführung in deskriptive grammatische Analysen und Argumentationen möchte ich bei dieser Gelegenheit meinen Lehrern in japanologischer Sprachwissenschaft, Thomas M.\ Groß und Iris Hasselberg, besonders danken.
Ohne diese beiden wäre ich nicht Linguist geworden.


\Phantom
\addcontentsline{toc}{part}{Vorbemerkungen zur zweiten Auflage}
\chapter*{Vorbemerkungen zur zweiten Auflage}
\markboth{Vorbemerkungen zur zweiten Auflage}{Vorbemerkungen zur zweiten Auflage}

\section*{Über die zweite Auflage}

Für jede Überarbeitung eines Buchs gibt es einen Grund.
Vor allem auf dem angelsächsischen Markt werden Lehrwerke oft jährlich aktualisiert, um Neuanschaffungen zu erzwingen und damit Umsätze zu generieren.
Bei einer nichtkommerziellen Open-Access-Veröffentlichung fällt dieser Grund aus, und es müssen also inhaltliche Gründe her.
Natürlich gibt es an einem Buch mit vielen hundert Seiten immer Kleinigkeiten zu korrigieren und verbessern, und das ist hier auch in teilweise erheblichem Ausmaß geschehen.
Einige Abschnitte bedurften aber einer größeren Überarbeitung.
Dies betraf insbesondere die Anpassung einiger Themen an das methodisch-didaktische Gesamtkonzept, wofür ich jetzt ein Beispiel gebe.

Die segmentale Phonologie folgte in der ersten Auflage weitgehend dem zehn Jahre alten Unterrichtsskript zu meiner ersten Lehrveranstaltung.
In dieser Zeit war ich der Überzeugung, dass ich in jedem Fall auch in Einführungen die generative Phonologie der 1980er und 1990er Jahre unterrichten müsse, gegebenenfalls in Form ihrer optimalitätstheoretischen Umformulierung.
Diese Entscheidung zwang mich dann unter anderem dazu, die Theorie der distinktiven Merkmale und ein bestimmtes Inventar solcher Merkmale einzuführen.
Durch dieses Vorgehen, das sich bis in die erste Auflage von \textit{Einführung in die grammatische Beschreibung des Deutschen} vererbt hat, ergab sich aber ein gewisser Ballast sowie theorieinterne Probleme, die das Unternehmen komplizierter als nötig gemacht haben.
Dies betraf zum Beispiel die Merkmalssausstattung des \textipa{[5]}, die mit dem üblicherweise verwendeten Inventar schwierig gegen die der umliegenden Vokale abzugrenzen ist.
Außerdem habe ich in der ersten Auflage das vokalische Merkmal \textsc{Gespannt} mit dem phonetisch begründeten Merkmal ATR (\textit{Advanced Tongue Root}) identifiziert.
Das funktioniert aber nicht, und die eigentliche Interaktion von Gespanntheit, Akzent und Vokallänge kann so nicht abgebildet werden.
Wenn man \textsc{Gespannt} für das Deutsche annehmen möchte, dann eher als abstraktes Merkmal, das kein vordergründiges phonetisches Korrelat hat.
Der diesen Nachteilen gegenüberstehende explanatorische Gewinn aus der Entscheidung, die generative Merkmalstheorie zu verwenden, fiel im Rahmen der Einführung in die beschreibende Grammatik des Deutschen zudem gering aus.
Im Grunde dienten die phonologischen Merkmale nur der Formulierung phonologischer Prozesse als Regeln.
Statt im Pseudoformalismus der ersten Auflage können solche Regeln aber besser und mit gleicher Genauigkeit in natürlicher Sprache formuliert werden.
Hier habe ich also grundlegend eingegriffen und einiges entfernt, dafür aber Wichtiges zur Form der Silben nachgereicht.
Abschnitt~\ref{sec:silbenundwoerter} wurde dementsprechend komplett neu geschrieben.

Auf Basis ähnlicher Überlegungen habe ich zum Beispiel die Darstellung adverbiell verwendeter Adjektive verändert, die jetzt ohne Konversion auskommt.
Gleichermaßen habe ich in der Graphematik (in diesem Fall auch aus meiner eigenen Überzeugung als spätberufener Graphematiker) die Darstellung der primären Interpunktionszeichen zu einer sogenannten \textit{konstruktionsbasierten} umgebaut.
Außerdem hatte ich in der Graphematik die Gelegenheit, grobe Fehler in der Analyse des \textit{ß} auszumerzen.
In der Wortbildung wird jetzt mehr dazu gesagt, dass Verbpräfixe und Verbpartikeln unterschiedliche Affinität zur Valenzänderung haben.
Dies sind nur einige der inhaltlich relevanten Änderungen.
Im Großen und Ganzen bleibt \textit{Einführung in die grammatische Beschreibung des Deutschen} aber das gleiche Buch.

\section*{Danksagungen zur zweiten Auflage}

Die erste Auflage dieses Buchs ist bis zur Jahresmitte 2016 ungefähr 8.500 Mal heruntergeladen worden und war damit im Juni 2016 das am häufigsten heruntergeladene Buch bei Language Science Press.
In einem guten Jahr auch nur eine halb so hohe Auflage zu erreichen, ist mit teuren traditionellen Printausgaben sehr schwer, und ich danke allen Lesern und Leserinnen der ersten Fassung, die zu diesem Erfolg beigetragen haben.

Den Herausgebern der Reihe \textit{Textbooks in Language Sciences}, Martin Haspelmath und Stefan Müller sowie dem Verlag Language Science Press danke ich einmal mehr und umso mehr dafür, dass dieses Buch als Open-Access-Publikation erscheinen kann und damit überhaupt erst sein inzwischen großes Publikum erreicht.
Felix Kopecky von Language Science Press gilt mein besonderer Dank für das Lösen von \LaTeX-Problemen, die ich selber nicht so gut lösen gekonnt hätte.
Dem Koordinator von Language Science Press, Sebastian Nordhoff, danke ich besonders nachdrücklich für seine Geduld und die schnelle Hilfe bei allen Problemen mit der Veröffentlichung.

\Np

Bei Geza Lebro bedanke ich mich für seine Hinweise auf relevante Mängel und Fehler im Buch im Rahmen einer Lehrveranstaltung an der Freien Universität Berlin im Sommer 2016.
Für inhaltliche Rückmeldung aus dem Einsatz in der Lehre danke ich Hans-Joachim Particke.
Ulrike Sayatz hat das Buch sehr ausführlich in der Lehre getestet, wofür ich ihr besonders danke.
Ulrike danke ich zudem dafür, dass wir Phonologie, Wortbildung und Graphematik nochmals im Ganzen durchdiskutieren konnten, und dass sie das Buch noch einmal vollständig gelesen hat.
Die größeren Verbesserungen von der ersten zur zweiten Auflage gehen zu einem wesentlichen Teil auf die Diskussionen mit Ulrike zurück.
Götz Keydana danke ich für Hinweise zu den Neufassungen der Kapitel über Phonetik und Phonologie.
Kim Maser und Luise Rißmann danke ich für das vollständige Durchlesen und Kommentieren des Buchs und das Überprüfen der Übungsaufgaben.
Bose für Türkisch (\textit{sap.\ sat.}).
Thea Dittrich danke ich dafür, dass sie wieder alles gegen meine Tipp- und Rechtschreibfehler getan, das man angesichts meines Arbeitsprozesses dagegen tun kann.

Selbstverständlich bin ich alleine für die verbleibenden Mängel und Schwächen verantwortlich.

\section*{Webseite, Rückmeldungen und Diskussionen}

Es gibt eine Webseite zu diesem Buch mit zusätzlichen Materialien und Diskussionen über Grammatik:

\begin{center}
  \url{http://grammatick.de/} 
\end{center}

\noindent Von nichts können zukünftige Auflagen dieses Buches (bzw.\ jedes Buches) mehr profitieren als von Anregungen durch seine Leserinnen und Leser.
Daher bitte ich ausdrücklich um Rückmeldungen formaler und inhaltlicher Natur über die Online-Plattform \textit{Paperhive}:

\begin{center}
  \url{https://paperhive.org/documents/cdutBQB09Hbc}
\end{center}

\noindent Für Diskussionen und sonstige Rückmeldungen gibt es außerdem die folgenden Möglichkeiten, mich zu kontaktieren:

\begin{center}
  \begin{tabular}{lp{2mm}l}
    Email && mail@rolandschaefer.net \\
    ORCID && \url{http://orcid.org/0000-0003-3233-7874} \\
    Twitter && @codeslapper \\
  \end{tabular}
\end{center}
