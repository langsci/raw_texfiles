\chapter{Grammatik}

\label{sec:grammatik}

\section{Sprache und Grammatik}

\label{sec:spracheundgrammatik}

\subsection{Sprache als Symbolsystem}

\label{sec:sprachealssymbolsystem}

\index{Sprache}

Sprache kann unter sehr verschiedenen Blickwinkeln wissenschaftlich betrachtet werden.
Man kann Sprache als \textit{kognitive Aktivität} des Menschen ansehen, denn offensichtlich bilden und verstehen Menschen sprachliche Äußerungen mittels kognitiver Vorgänge im Gehirn.
Mit gleichem Recht könnte man Sprache als \textit{soziale Interaktion} (Kommunikation) charakterisieren und unter diesem Aspekt untersuchen.
Sprache wird tatsächlich in Teildisziplinen der Linguistik aus solchen und vielen anderen Perspektiven betrachtet, und jede Teildisziplin hat eine andere, dem Blickwinkel angepasste Definition von Sprache.
Hier beschränken wir uns so weit wie möglich auf einen ganz bestimmten, eng definierten Aspekt von Sprache, nämlich den Charakter von \textit{Sprache als symbolisches System}.\index{Symbolsystem}
Wir gehen dabei davon aus, dass Sprache unabhängig von der Art ihrer Verarbeitung im Gehirn, ihren sozialen Funktionen usw.\ einen solchen Charakter hat.
Damit ist gemeint, dass Sprache aus Symbolen und Symbolverbindungen (Lauten, Buchstaben, Silben, Wörtern usw.) aufgebaut ist, die in systematischen Beziehungen zueinander stehen, und die auf regelhafte Weise zusammengesetzt sind.
Welches Medium dafür verwendet wird (\zB gesprochene Laute, Gebärden oder Schrift) ist erst einmal nachrangig.

Als Sprecher des Deutschen kann man \zB sofort erkennen, dass (\ref{ex:gb2957a}) eine akzeptable Symbolfolge des Symbolsystems \textit{Deutsch} ist.
Satz (\ref{ex:gb2957b}) besteht zwar aus Symbolen dieses Systems, aber diese sind falsch kombiniert.
Sätze (\ref{ex:gb2957c}) und (\ref{ex:gb2957d}) hingegen enthalten gar keine Symbole (zumindest Symbole im Sinne von Wörtern) dieses Systems.

\begin{exe}
  \ex
  \begin{xlist}
    \ex{\label{ex:gb2957a} Dies ist ein Satz.}
    \ex{\label{ex:gb2957b} Satz dies ein ist.}
    \ex{\label{ex:gb2957c} Kno kna knu.}
    \ex{\label{ex:gb2957d} This is a sentence.}
  \end{xlist}
\end{exe}

\index{Regularität}

Bezüglich (\ref{ex:gb2957a}) und (\ref{ex:gb2957b}) sind nun zwei Dinge bemerkenswert.
Erstens können wir sofort erkennen, dass die Symbolfolge in (\ref{ex:gb2957a}) konform zu einem System von Regularitäten ist, auch wenn wir diese Regularitäten nicht immer -- sogar meistens nicht -- explizit benennen können.
Dass dies bei (\ref{ex:gb2957b}) nicht der Fall ist, erkennen wir auch unverzüglich und ohne explizit nachzudenken.
Im Fall von (\ref{ex:gb2957d}) erkennen die meisten sicher sofort, dass es sich um einen Satz handelt, der zu einem anderen Symbolsystem -- dem Englischen -- konform ist.
Wir haben also offensichtlich ein System von Regularitäten verinnerlicht, das es uns ermöglicht, zu beurteilen, ob eine Symbolfolge diesem System entspricht oder nicht.%
\footnote{Weiter unten und vor allem in Abschnitt~\ref{sec:empirie} wird diskutiert, dass die eindeutige Trennung eine Vereinfachung darstellt.
In erster Näherung funktioniert sie aber recht gut.}
Außerdem können wir aus den Bedeutungen der einzelnen bedeutungstragenden Symbole (der Wörter) und der Art, wie diese zusammengesetzt sind, unverzüglich die Bedeutung der Symbolfolge (des Satzes) erkennen.
Die zuletzt genannte Eigenschaft von Sprache nennt man \textit{Kompositionalität}.

\Definition{Kompositionalität}{\label{def:kompositionalitaet}%
Die Bedeutung komplexer sprachlicher Ausdrücke ergibt sich aus der Bedeutung ihrer Teile und der Art ihrer grammatischen Kombination.
Diese Eigenschaft von Sprache nennt man \textit{Kompositionalität}.
\index{Kompositionalität}
}

Das Symbolsystem mit seinen Regularitäten und die Art der kompositionalen Konstruktion von Bedeutung sind dabei in gewissem Maß unabhängig voneinander, wie man an Satz (\ref{ex:dg7717}) zeigen kann.

\begin{exe}
  \ex{\label{ex:dg7717} Dies ist ein Kneck.}
\end{exe}

Satz (\ref{ex:dg7717}) hat sicherlich für keinen Leser dieses Buchs eine vollständig er\-schließ\-bare Bedeutung.
Dies liegt aber nicht daran, dass die Symbolfolge inkorrekt konstruiert wäre, sondern nur daran, dass wir nicht wissen, was ein \textit{Kneck} ist.
Unter der Annahme (die wir implizit sofort machen, wenn wir den Satz lesen), dass es sich bei dem Wort \textit{Kneck} um ein Substantiv handelt, kategorisieren wir den Satz als akzeptabel.
Wir können sogar sicher sagen, dass wir die Bedeutung des Satzes kennen würden, sobald wir erführen, was ein Kneck genau ist.
In einer gegebenen Kommunikationssituation könnte der Satz verwendet werden, um Sprecher eben genau darüber zu informieren, was ein Kneck ist, \zB durch gleichzeitiges Zeigen auf einen Gegenstand.

Paralleles gilt für widersinnige oder widersprüchliche Sätze wie die in (\ref{ex:dg7718}), die ebenfalls grammatisch völlig in Ordnung sind.
Gerade weil wir ein implizites Wissen davon haben, wie man aus Bedeutungen von Wörtern und der Art ihrer Zusammensetzung Bedeutungen von Sätzen ermittelt, können wir feststellen, dass die Sätze widersinnig bzw.\ widersprüchlich sind.

\begin{exe}
  \ex\label{ex:dg7718}
  \begin{xlist}
    \ex{Jede Farbe ist ein Kurzwellenradio.}
    \ex{Der dichte Tank leckt.}
  \end{xlist}
\end{exe}

Es zeigt sich also, dass die Symbole der Sprache (Laute, Wörter usw.) ein eigenes \textit{Kombinationssystem} (eine \textit{Grammatik}) haben.\index{Grammatik!als Kombinationssystem}
Dieses System ist dafür verantwortlich, dass wir Bedeutungen von komplexen Symbolfolgen verstehen (interpretieren) können.
Gleichzeitig ist das System selber aber bis zu einem gewissen Grad unabhängig davon, ob die Interpretation tatsächlich erfolgreich ist.
Wenn es dieses Sprachsystem und die Kompositionalität nicht gäbe, wäre es äußerst schwer, eine Sprache zu erlernen, sowohl als Erstsprache im Kindesalter als auch als Zweit- bzw.\ Fremdsprache.

Wegen der (partiellen) Unabhängigkeit des Symbolsystems von der Interpretation ist es legitim und sogar strategisch sinnvoll, zunächst nur das Symbolsystem zu beschreiben, ohne sich über die Bedeutung zu viele Gedanken zu machen.
Daher wird in diesem Buch die Bedeutung aus der grammatischen Analyse weitgehend ausgeklammert.%
\footnote{Kognitiv ausgerichtete linguistische Theorieansätze gehen auf Basis einer inzwischen relativ breiten Datenlage davon aus, dass das formale System und die Bedeutung nur eingeschränkt die hier angenommene Unabhängigkeit aufweisen.
Es kann \zB experimentell gezeigt werden, dass die Entscheidung zwischen wohlgeformten und nicht-wohlgeformten Symbolfolgen durch Menschen nicht allein von formalen grammatischen Bedingungen abhängt, sondern auch von der Bedeutung.
Für viele der in diesem Buch beschriebenen Phänomene kommt man allerdings recht weit, auch ohne Einbeziehung der Bedeutungsseite.
Außerdem ist es möglich, den Systembegriff auf die Semantik auszuweiten, was allerdings einen erheblichen formalen Aufwand nach sich zieht.}
Es gibt einen wesentlichen praktischen Vorteil der formalen Herangehensweise:
Definitionen und Beschreibungen, die sich an der Form orientieren, sind meist viel einfacher nachzuvollziehen und anzuwenden, als solche, die semantische Beurteilungen erfordern.
Auch wenn die Vermittlungspraxis an Schulen \idR den umgekehrten Weg geht und grammatische Kategorien über Bedeutungen einführt, taugen die formalen Eigenschaften von Sprache oft mehr, wenn systematisch nachvollziehbare Kategorisierungen gewünscht werden.
In diesem Punkt sind wir aber nicht dogmatisch und berücksichtigen die Bedeutungsseite der Sprache immer dann, wenn bei einem gegebenen Phänomen die Trennung von Grammatik und Bedeutung besonders schwierig ist, oder wenn die Berücksichtigung der Bedeutung die Argumentation wesentlich verkürzt und vereinfacht.
Dieses pragmatische Vorgehen deutet darauf hin, dass die starke Reduktion auf die Form (bzw.\ auf einen engen Begriff von Grammatik im Sinne einer Formgrammatik) nicht meiner theoretischen Position entspricht.\index{Grammatik!formbasiert}

\Stretch[2]

\subsection{Grammatik}

\label{sec:grammatikbegriff}

Wie verhält sich nun der Begriff \textit{Grammatik} zu dem oben beschriebenen Verständnis von Sprache?
Er wird stark mehrdeutig verwendet, und wir legen die relativ unspezifische Definition~\ref{def:grammatik} zugrunde.

\Definition{Grammatik}{\label{def:grammatik}%
Eine \textit{Grammatik} ist ein System von Regularitäten, nach denen aus einfachen Einheiten komplexe Einheiten einer Sprache gebildet werden.
\index{Grammatik!Sprachsystem}
\index{Regularität}
}

Wir gehen also davon aus, dass die zugrundeliegende Grammatik (das System von Regularitäten) für die Form der sprachlichen Äußerungen (\zB Sätze) verantwortlich ist, und dass Grammatiker diese Regularitäten durch Beobachtungen dieser Äußerungen zu erkennen versuchen.
Wenn man diese Regularitäten aufschreibt bzw.\ formalisiert, liegt eine wissenschaftliche Grammatik als Modell für die beobachteten Daten vor.
Davon grundsätzlich zu unterscheiden ist natürlich eine Grammatik als Artefakt (\zB ein Buch), in dem grammatische Regularitäten beschrieben werden.
Ebenso unabhängig ist die Annahme einer mentalen Grammatik in verschiedenen Richtungen der Linguistik, also einer Repräsentation der sprachlichen Regularitäten im Gehirn.
Abgesehen davon bezeichnet der Begriff \textit{Grammatik} natürlich auch die Wissenschaft, die sich mit grammatischen Regularitäten einzelner oder aller Sprachen beschäftigt.

\subsection{Akzeptabilität und Grammatikalität}

\label{sec:akzeptabilitaetgrammatikalitaet}

\index{Akzeptabilität}

Der Begriff der \textit{Grammatikalität} ist zentral für die Grammatikforschung und die theoretische Linguistik.
Sprachliche Einheiten und Konstruktionen aller Art (\zB Wörter oder Sätze) sind \textit{grammatisch} oder \textit{ungrammatisch}.
Wir bauen den Begriff der \textit{Grammatikalität} hier auf den der \textit{Akzeptabilität} auf, dessen Definition sich nicht auf ein abstraktes Symbolsystem, sondern einem kompetenten Sprachbenutzer bezieht.

\Definition{Akzeptabilität}{\label{def:akzeptabilitaet}%
Jede sprachliche Einheit (\zB jeder Satz), die von einem kompetenten Sprachbenutzer als konform zur eigenen Grammatik eingestuft wird, ist \textit{akzeptabel}.
}

Ein kompetenter Sprachbenutzer muss also gemäß dieser Definition entscheiden können, ob ein Satz, den man ihm präsentiert, ein akzeptabler Satz ist oder nicht.
Kompetent ist ein Sprachbenutzer sehr vereinfacht gesagt, wenn er die betreffende Sprache im frühen Kindesalter gelernt hat, sie seitdem kontinuierlich benutzt hat und an keiner Sprachstörung (\textit{Aphasie}) leidet.
Dass Akzeptabilitätsurteile nicht unproblematisch sind, demonstrieren die Sätze in (\ref{ex:dg9113}).

\Enl

\begin{exe}
  \ex\label{ex:dg9113}
  \begin{xlist}
    \ex{\label{ex:dg9113a} Bäume wachsen werden hier so schnell nicht wieder.}
    \ex{\label{ex:dg9113b} Touristen übernachten sollen dort schon im nächsten Sommer.}
    \ex{\label{ex:dg9113c} Schweine sterben müssen hier nicht.}
    \ex{\label{ex:dg9113d} Der letzte Zug vorbeigekommen ist hier 1957.}
    \ex{\label{ex:dg9113e} Das Telefon geklingelt hat hier schon lange nicht mehr.}
    \ex{\label{ex:dg9113f} Häuser gestanden haben hier schon immer.}
    \ex{\label{ex:dg9113g} Ein Abstiegskandidat gewinnen konnte hier noch kein einziges Mal.}
    \ex{\label{ex:dg9113h} Ein Außenseiter gewonnen hat hier erst letzte Woche.}
    \ex{\label{ex:dg9113i} Die Heimmannschaft zu gewinnen scheint dort fast jedes Mal.}
    \ex{\label{ex:dg9113j} Ein Außenseiter gewonnen zu haben scheint hier noch nie.}
    \ex{\label{ex:dg9113k} Ein Außenseiter zu gewinnen versucht hat dort schon oft.}
    \ex{\label{ex:dg9113l} Einige Außenseiter gewonnen haben dort schon im Laufe der Jahre.}
  \end{xlist}
\end{exe}

Die Komplikationen, die hier auftreten, liegen einerseits darin begründet, dass es in einer Sprachgemeinschaft nicht nur einen, sondern viele kompetente Sprecher gibt, die sich nicht immer bezüglich ihrer Akzeptabilitätsurteile einig sind.  
Andererseits sind sich auch einzelne Sprecher nicht immer so sicher in ihrem Urteil, wie es Definition~\ref{def:akzeptabilitaet} voraussetzt.
Bei (\ref{ex:dg9113a}) sind sich die meisten Sprecher des Deutschen einig, dass der Satz akzeptabel ist.
Genauso wird die Entscheidung, dass (\ref{ex:dg9113l}) nicht akzeptabel ist, meist eindeutig gefällt.
Die Sätze dazwischen führen in unterschiedlichem Maß zu Unsicherheiten bezüglich ihrer Akzeptabilität, und größere Gruppen von Sprachbenutzern sind sich selten über die genauen Urteile einig.
Dennoch ist es aus Sicht der Grammatik sinnvoll, als Anfangshypothese davon auszugehen, dass eine eindeutige Entscheidung möglich ist.
Für viele Strukturen ist die Entscheidung auch tatsächlich unproblematisch, und um diese Strukturen geht es primär in diesem Buch.
Diese Grundsatzprobleme werden in Abschnitt~\ref{sec:empirie} nochmals vertiefend diskutiert.

Definition~\ref{def:grammatikalitaet} abstrahiert vom Sprachbenutzer und bezieht sich nur auf eine Grammatik als System von Regularitäten.

\Definition{Grammatikalität}{\label{def:grammatikalitaet}%
Jede von einer Grammatik (im Sinne von Definition~\ref{def:grammatik}) beschriebene Symbolfolge ist \textit{grammatisch} bezüglich dieser Grammatik, alle anderen Symbolfolgen sind \textit{ungrammatisch} bezüglich dieser Grammatik.
\index{Grammatik}
\index{Grammatikalität}
}

Die Grammatik ist in dieser Definition ein explizit spezifiziertes System von Regularitäten, das definiert, wie aus einfachen Elementen (Symbolen) komplexere Strukturen (Symbolfolgen) zusammengesetzt werden.
Mit \textit{Symbolen} können dabei Laute, Buchstaben, Wörter, Satzteile oder sonstige Größen der Grammatik gemeint sein.
Wo und wie die Grammatik definiert ist oder sein kann, sagt Definition~\ref{def:grammatikalitaet} nicht.
Es könnte sein, dass es sich wiederum um eine im Gehirn verankerte Sammlung von Regularitäten handelt, also eine Grammatik, die das in Definition~\ref{def:akzeptabilitaet} beschriebene Verhalten eines Sprachbenutzers -- seine Akzeptabilitätsurteile -- steuert.
Definition~\ref{def:grammatikalitaet} kann aber auch auf eine Grammatik bezogen sein, die ein Linguist definiert und niedergeschrieben hat, so wie in diesem Buch.
Für natürliche Sprachen verwendet man Akzeptabilitätsurteile ihrer Sprecher, um indirekt darauf zu schließen, welche Strukturen grammatisch und ungrammatisch sind.
Da man die Sprechergrammatik aber nirgends direkt einsehen kann, ist es zielführend, zwischen Akzeptabilität und Grammatikalität zu trennen.\index{Akzeptabilität}\index{Grammatikalität}

Man setzt * (den \textit{Asterisk}) vor solche Strukturen, die relativ zu einer bestimmten Grammatik ungrammatisch sind.
Da der Asterisk alleine noch keine Information enthält, bezüglich welcher Grammatik ein Satz oder eine andere Einheit ungrammatisch ist, müsste man diese Information eigentlich zusätzlich angeben.
Wenn der Satz in (\ref{ex:grammi2types}) von den Sprechern einer Sprache nicht akzeptiert wird, wäre es korrekt, ihn mit einem Asterisk zu markieren, der sich auf die Grammatik dieser Sprache bezieht.%
\footnote{Zum hier illustrierten Phänomen vgl.\ Abschnitt~\ref{sec:ersiofu}.}

\begin{exe}
  \ex{\label{ex:grammi2types}$\ast_{\mathrm{\textnormal{Standarddeutsch}}}$ Ich glaube, dass Alma die Bücher lesen gewollt hat.}
\end{exe}

Wenn man anzeigen möchte, dass eine theoretische Grammatik den entsprechenden Satz nicht beschreibt (unabhängig davon, was Sprachbenutzer dazu sagen), weil sie vielleicht noch nicht vollständig oder nicht exakt genug formuliert ist, wäre eine Markierung wie in (\ref{ex:dg9108}) korrekt.
Diese zeigt an, dass die Theorie den Satz als ungrammatisch einstuft, auch wenn dies bedeutet, dass die Beurteilung durch die formale Theorie von den Urteilen der Sprachbenutzer abweicht.
\begin{exe}
  \ex{\label{ex:dg9108} $\ast_{\mathrm{\textnormal{Theorie}}}$ Ich glaube, dass Alma die Bücher lesen gewollt hat.}
\end{exe}

In diesem Buch markiert der Asterisk Ungrammatikalität relativ zur Grammatik der in Deutschland benutzten Standardvarietät des Deutschen (s.\ auch Abschnitt~\ref{sec:deskriptivenormativegrammatik}).
Theoretisch müssten also alle Sprecher dieser Sprache die Beispiele ohne * akzeptieren und alle Beispiele mit * als inakzeptabel ablehnen.
Dass die Annahme von einheitlich urteilenden kompetenten Sprachbenutzern genauso wie die Annahme einer wohldefinierten standardnahen Varietät des Deutschen Illusionen sind, sollte nach der bisherigen Argumentation bereits klar sein.%
\footnote{Abgesehen davon orientieren wir uns hier sehr stark an der geschriebenen Sprache, die sich wesentlich von der gesprochenen unterscheidet.
Das ist teilweise der methodisch-didaktischen Reduktion, teilweise aber auch dem Forschungsstand in der Linguistik geschuldet.
Die Erforschung der gesprochenen Sprache ist de facto ein Spezialgebiet, auch wenn Linguisten gerne behaupten, die Erforschung der gesprochenen Sprache habe ganz allgemein Vorrang vor der der geschriebenen (s.\ auch Abschnitt~\ref{sec:statusgraphematik}).}
Unter den in diesem Buch beschriebenen Phänomenen sind allerdings hoffentlich wenige, bei denen größere Gruppen von Sprechern der in Deutschland gesprochenen deutschen Verkehrssprache bezüglich der angenommenen Akzeptabilitätsurteile uneinig sind.%
\footnote{Die Reduktion auf den in Deutschland verwendeten Standard ist aus Sicht des Autors bedauerlich, zumal (neben dialektaler Variation) in Österreich und der Schweiz auch etablierte abweichende Standards existieren.
Der Platz reicht aber schlicht nicht aus, um andere Standards oder gar dialektale Variation zu berücksichtigen.}

\subsection{Ebenen der Grammatik}

\label{sec:ebenendergrammatik}

\index{Grammatik!Ebene}

Grammatikalität betrifft verschiedene Faktoren sprachlicher Strukturen (\zB die lautliche Gestalt, die Form der Wörter, den Satzbau), die man meistens verschiedenen \textit{Ebenen der Grammatik} zuordnet.
Die Ebenen, mit denen wir uns in diesem Buch beschäftigen, sind diejenigen, die vor allem die rein formalen Eigenschaften von Sprache beschreiben.
Die \textit{Phonetik} (Kapitel~\ref{sec:phonetik}) beschreibt die rein lautliche Ebene der Sprache.
Die typische Fragestellung der Phonetik ist:
Welche Laute kommen überhaupt in einer Sprache vor, und wie werden sie mit den Sprechorganen gebildet?
Die \textit{Phonologie} (Kapitel~\ref{sec:phonologie}) beschreibt die systematischen Zusammenhänge in Lautsystemen sowie die lautlichen Regularitäten, die zur Anwendung kommen.\index{Phonologie}
So eine Regularität kann sich \zB darauf beziehen, in welchen Reihenfolgen die Laute einer Sprache vorkommen können.
Die \textit{Morphologie} (Teil~\ref{part:wort}) analysiert sowohl den Aufbau von Wörtern als auch die Beziehungen zwischen verschiedenen Wörtern und verschiedenen Formen eines Wortes.
Die Morphologie teilt sich in zwei Gebiete, die getrennt behandelt werden:\index{Morphologie}
Die \textit{Wortbildung} (Kapitel~\ref{sec:wortbildung}) beschreibt, wie aus bestehenden Wörtern neue Wörter gebildet werden (\zB \textit{Fußball} aus \textit{Fuß} und \textit{Ball} oder \textit{fraulich} aus \textit{Frau} und \textit{lich}).
Die \textit{Flexion} (Kapitel~\ref{sec:nominalflexion} und~\ref{sec:verben}) beschreibt die Bildung der Formen eines Wortes (also \zB \textit{gehen} und \textit{ging}). 
Die \textit{Syntax} (Teil~\ref{part:syntax}) beschäftigt sich mit der Frage, wie Wörter zu größeren Gruppen und schließlich zu Sätzen zusammengefügt werden.\index{Syntax}
In der \textit{Graphematik} (Teil~\ref{part:schrift}) geht es darum, wie die Schrift sprachliche Einheiten kodiert.\index{Graphematik}
Warum die Graphematik ganz am Ende des Buchs steht, wird dort einleitend diskutiert.

Auch wenn in der Linguistik andere Ebenen wie die \textit{Semantik} (Bedeutungslehre), die \textit{Pragmatik} (Lehre vom Sprachgebrauch und vom sprachlichen Handeln) usw.\ intensiv erforscht werden, ist die Beschreibung der formalen Ebenen ein guter Ausgangspunkt jeder Sprachbetrachtung.
Damit ist nicht gesagt, dass es sich hier um den wichtigsten Teil der Sprachbeschreibung bzw.\ Linguistik handelt, wohl aber um den, der nach meiner Auffassung zuerst behandelt werden sollte.
Es wäre schwierig, zum Beispiel den Aufbau von Texten zu erforschen, bevor geklärt ist, wie die Bestandteile des Textes (die Sätze) zu analysieren sind.

\subsection{Kern und Peripherie}

\label{sec:kernundperipherie}

\index{Kern}
\index{Peripherie}

Bisher ging es um das grammatische System als Ganzes.
Im Verlauf des Buchs wird aber immer wieder vom \textit{Kern} und von der \textit{Peripherie} des Systems die Rede sein, \zB in Form des \textit{Kernwortschatzes} oder der Flexionstypen, die den Kern der Flexion ausmachen.
Diese Begriffe werden hier kurz eingeführt, vor allem um Missverständnissen vorzubeugen.
In (\ref{ex:gr123239})--(\ref{ex:gr123241}) sind einige Beispiele aufgeführt.
Im Folgenden wird erklärt, warum die Beispiele in (a) jeweils näher am Systemkern sind als die in (b).
Dabei ist zwar oft von Kern und Peripherie wie von zwei streng getrennten Bereichen die Rede, eigentlich muss aber von einem Kontinuum zwischen Kern und Peripherie ausgegangen werden.

\begin{exe}
	\ex\label{ex:gr123239}
	\begin{xlist}
		\ex\label{ex:gr123239a} Baum, Haus, Matte, Döner, Angst, Öl, Kutsche, \ldots
		\ex\label{ex:gr123239b} System, Kapuze, Bovist, Schlamassel, Marmelade, Melodie, \ldots
	\end{xlist}
	\ex\label{ex:gr123240}
	\begin{xlist}
		\ex\label{ex:gr123240a} geht, läuft, lacht, schwimmt, liest, \ldots
		\ex\label{ex:gr123240b} kann, muss, will, darf, soll, mag 
	\end{xlist}
	\ex\label{ex:gr123241}
	\begin{xlist}
		\ex\label{ex:gr123241a} des Hundes, des Geistes, des Tisches, des Fußes, \ldots
		\ex\label{ex:gr123241b} des Schweden, des Bären, des Prokuristen, des Phantasten, \ldots
	\end{xlist}
\end{exe}

\index{Kern!Wortschatz}
\index{Erbwort}
\index{Lehnwort}
\index{Fremdwort}

Besonders bei (\ref{ex:gr123239}) könnte man nun vermuten, dass sich im Kern eher alte germanische Wörter befinden, in der Peripherie hingegen sogenannte \textit{Fremdwörter}.
Zunächst einmal ist die korrekte Bezeichnung für Wörter, die aus anderen Sprachen übernommen (\textit{entlehnt}) wurden, \textit{Lehnwort} und nicht \textit{Fremdwort}.
Vereinfacht gesagt nennt man Wörter, die sich nach unserem Wissen seit vorhistorischer Zeit im Wortschatz befinden, und die nicht erkennbar aus einer anderen Sprache entlehnt wurden, \textit{Erbwörter}.
Die Gesamtheit der Erbwörter ist der \textit{Erbwortschatz}.\label{abs:erbwortschatz}
Vor allem stimmt aber diese einfache Zuordnung von Lehnwort und peripherem Wortschatz in (\ref{ex:gr123239}) nicht, denn \textit{Döner}, \textit{Öl} und \textit{Kutsche} sind allesamt mehr oder weniger rezent aus anderen Sprachen entlehnt worden.
Das Wort \textit{Bovist} hingegen ist ein Erbwort und trotzdem im System ein recht fremder Einzelgänger.
Es ist vielmehr so, dass alle Wörter in (\ref{ex:gr123239a}) entweder im Nominativ Singular einsilbig sind, oder aber zweisilbig und dabei auf der ersten Silbe betont.%
\footnote{Die übliche aus dem Latein entlehnte grammatische Terminologie wird hier bei der Diskussion illustrativer Beispiele von Anfang an konsequent benutzt.
Für Leser, die nicht mehr ganz firm darin sind, oder die in ihrer Schulzeit konsequent deutsche Schultermini wie \zB \textit{Namenwort}, \textit{Hauptwort} oder \textit{Dingwort} statt \textit{Substantiv} gelernt haben, sollten entweder im Buch zu den entsprechenden Themen weiterblättern oder zur Online"=Recherche greifen.
Wenn es inhaltlich wirklich darauf ankommt, werden die Begriffe natürlich jeweils hinreichend präzise eingeführt.}
Die Zweisilbler bilden hier einen sogenannten \textit{trochäischen Fuß} oder einfach einen \textit{Trochäus}, also eine Folge von einer betonten und einer unbetonten Silbe (Details dazu in Kapitel~\ref{sec:phonologie}).\index{Fuß}
Einsilbigkeit oder trochäische \textit{Fußstruktur} sind für nicht zusammengesetzte Substantive der Normalfall im Deutschen.
Die mehrsilbigen Wörter in (\ref{ex:gr123239b}) sind nun deshalb ungewöhnlich, weil sie nicht auf der ersten Silbe betont werden und teilweise mehr als zwei Silben haben (bis zu vier im Fall von \textit{Marmelade}).
Einsilbige und trochäische Substantive bilden also den Kernwortschatz der Substantive, und abweichende Substantive wie in (\ref{ex:gr123239b}) befinden sich in der Peripherie.
Das Wort \textit{Döner} ist also entlehnt, aber nicht fremd, weil es in der üblichen Aussprache trochäisch ist und seine Formen nach den allgemeinen Regularitäten deutscher maskuliner Substantive bildet.
Der Plural ist \textit{Döner} und nicht etwa türkisch \textit{dönerler}.
Ebenso ist der (definite) Akkusativ Singular \textit{den Döner} und nicht etwa türkisch \textit{döneri}.
Wäre dieses Wort nicht durch seine Bedeutung auf besondere Weise mit türkischer bzw.\ deutsch-türkischer Kultur verbunden, hätte vermutlich schon Jahrzehnte nach der Entlehnung kaum ein Sprecher mehr Anlass zur Vermutung, es könne sich um ein \textit{Lehnwort} oder gar ein \textit{Fremdwort} handeln.
Die Aussage, das Wort \textit{Döner} in einem deutschen Satz sei ein \textit{türkisches Wort}, ist also in jeder Hinsicht falsch.

\index{Verb!Modal--!Flexion}
\index{Verb!Flexionsklassen}

In (\ref{ex:gr123240}) geht es um die Verbformen der dritten Person Singular im Präsens.
Die Verben in (\ref{ex:gr123240a}) enden in dieser Form alle mit einem \textit{-t}, die in (\ref{ex:gr123240b}) nicht.
Interessant ist außerdem, dass die Liste in (\ref{ex:gr123240b}) nicht mit `\textit{\ldots}' endet.
Damit wird signalisiert, dass es genau diese sechs Verben und nicht noch mehr sind, die sich so verhalten (s.\ auch Abschnitt~\ref{sec:kleineverbklassen}).
Es gilt, dass der Kern der Verbalflexion aus Verben besteht, die in der dritten Person Präsens auf \textit{-t} enden.
Die Klasse der sogenannten \textit{Modalverben} wie \textit{können} ist hingegen im peripheren Bereich angesiedelt.

\index{Substantiv!schwach}

Die Beispiele in (\ref{ex:gr123241}) illustrieren die Formenbildung der Substantive in verschiedenen Klassen von Substantiven.
Der Genitiv Singular wird bei den maskulinen Substantiven fast immer wie in (\ref{ex:gr123241a}) mit \textit{-s} oder \textit{-es} gebildet.
Die maskulinen Substantive in (\ref{ex:gr123241b}) verhalten sich diesbezüglich anders, denn der Genitiv Singular wird hier mit \textit{-n} oder \textit{-en} gebildet.
Von diesen sogenannten \textit{schwachen Substantiven} gibt es nur gut fünfhundert im aktiven Sprachgebrauch (s.\ Abschnitt~\ref{sec:schwachsubst}).
Die Klasse der schwachen Substantive ist im Bereich der Formenbildung der Substantive peripher.

\index{Häufigkeit}
\index{Typ}
\index{Token}

% This is for page after next page!
\Enl[1]

Wie stellt man nun fest, was zum Kern gehört und was zur Peripherie?
Mit wenigen (schwer begründbaren) Ausnahmen hat die Argumentation über die Häufigkeit zu erfolgen.
Dabei ist \textit{Häufigkeit} allerdings nicht gleich \textit{Häufigkeit}, wie jetzt gezeigt wird.
Das schwache Substantiv \textit{Mensch} oder die Modalverben wie \textit{können} zum Beispiel sind zwar peripher, aber trotzdem sehr häufig in dem Sinn, dass man Formen dieser Wörter oft begegnet, wenn man deutsche Texte liest oder gesprochener deutscher Sprache zuhört.
Im DECOW14A-Korpus (zu Korpora s.\ Abschnitt~\ref{sec:empirie}) sind 0,08\% aller Wörter irgendwelche Formen des Wortes \textit{Mensch}.\index{Korpus}
Diese Häufigkeit ist für ein Substantiv vergleichsweise hoch. 
Vier der sechs Modalverben sind noch häufiger.
Tabelle~\ref{tab:modvfreq} zeigt, wieviel Prozent der Wörter in DECOW14A auf Formen dieser Verben entfallen.%
\footnote{Alle Angaben gemäß den Frequenzlisten von \url{http://corporafromtheweb.org/decow14/}}
Die dritte Spalte der Tabelle gibt an, wieviele Wörter man in diesem Korpus im Mittel lesen muss, bevor man einer Form des jeweiligen Modalverbs begegnet.
Eins von 189 deutschen Wörtern (im genannten Korpus) ist also eine Form von \textit{können}.
Zum Vergleich: Das häufigste Wort überhaupt ist der definite Artikel (\textit{der}\slash\textit{die}\slash\textit{das}), der 7.88\% aller Formen ausmacht, so dass nahezu jedes dreizehnte Wort eine seiner Formen ist.
Die Häufigkeit normaler Substantive wie \textit{Tisch} (0,006\%) oder \textit{Tankstelle} (0,0008\%) ist differenziert gestaffelt, aber verglichen mit den Modalverben niedrig.
Die drei häufigsten Verben, die keine Hilfsverben oder Modalverben sind (s.\ Abschnitt~\ref{sec:vunterklass}), sind \textit{geben} (0,21\%), \textit{machen} (0,17\%) und \textit{kommen} (0.16\%).

\begin{table}[!htbp]
  \begin{tabular}{lrr}
    \lsptoprule
    \multirow{2}{*}{\textbf{Modalverb}} & \textbf{Anteil an} & \textbf{eine Form pro}\\
    & \textbf{allen Wortformen} & \textbf{Textwörter (Mittel)}\\
    \midrule
    \textit{können} & 0,53\% &   189 \\
    \textit{müssen} & 0,21\% &   476 \\
    \textit{sollen} & 0,19\% &   526 \\
    \textit{wollen} & 0,13\% &   769 \\
    \textit{mögen}  & 0,06\% & 1.666\\
    \textit{dürfen} & 0,05\% & 2.000\\
    \lspbottomrule
  \end{tabular}
  \caption{Häufigkeit aller Formen der Modalverben im DECOW14A-Korpus}
  \label{tab:modvfreq}
\end{table}

\Enl[-1]

Wenn man sich die konkreten Vorkommen von Formen der Modalverben ansieht, sind sie also sehr häufig.
Allerdings gibt es nur sechs verschiedene Modalverben, und von den gewöhnlichen Verben wie in (\ref{ex:gr123240a}) gibt es Tausende.
Wenn man also einfach nur in Texten alle Vorkommnisse von Formen der Modalverben -- die sogenannten \textit{Tokens} -- zählt, dann sind es sehr viele.
Wenn man nur zählt, wieviele voneinander \textit{verschiedene} Formen von Modalverben -- die sogenannten \textit{Typen} -- es gibt, dann sind es ziemlich wenige.
Ähnliches gilt für die schwachen Substantive.
Einige von ihnen kommen sehr häufig vor (wie \textit{Mensch}), aber es gibt insgesamt nur gut fünfhundert verschiedene schwache Substantive.
Die Zahl der anderen Substantive im normalen Sprachgebrauch ist hingegen nach oben offen und geht in die Zehntausende.
Diese Verhältnisse lassen sich mit den Begriffen der \textit{Token-} und \textit{Typenhäufigkeit} gemäß Definition~\ref{def:toktyp} beschreiben.%
\footnote{Zur weiteren Illustration kann der folgende Satz betrachtet werden:
\textit{Eine Hose ist eine Hose.}
In diesem Satz kommen fünf Tokens vor, nämlich (in der gegebenen Reihenfolge und ohne Beachtung der satzeinleitenden Großschreibung) \textit{eine}, \textit{Hose}, \textit{ist}, \textit{eine} und \textit{Hose}.
Diese fünf Tokens entsprechen aber nur drei Typen, nämlich \textit{eine}, \textit{Hose} und \textit{ist}.
Das erste und das vierte sowie das zweite und fünfte Token sind identisch und zählen daher nur als je ein Typ.
}

\Definition{Token- und Typenhäufigkeit}{\label{def:toktyp}%
Die \textit{Tokenhäufigkeit} ist die Anzahl der Wörter (oder Konstruktionen) in Texten, egal ob es dieselben sind oder nicht.
Die \textit{Typenhäufigkeit} ist die Anzahl der voneinander verschiedenen Wörter oder Konstruktionen.
Die Typenhäufigkeit ist theoretisch maximal genauso hoch wie die Tokenhäufigkeit, in längeren Texten aber in der Praxis immer deutlich niedriger als die Tokenhäufigkeit.
}

Die Typenhäufigkeit der Modalverben und der schwachen Substantive ist gering.
Das ist unabhängig davon, ob die Wörter eine hohe Tokenhäufigkeit haben.
Der Kern wird durch die Klassen von Wörtern und Konstruktionen besetzt, die eine hohe Typenhäufigkeit haben (also große Klassen).
Typische für den Kern sind dabei regelmäßige und einheitliche Bildungen.
Die Peripherie ist durch kleine Klassen mit niedriger Typenhäufigkeit gekennzeichnet.
Besonders, wenn periphere Klassen eine hohe Tokenhäufigkeit aufweisen (wie die Hilfs- und Modalverben), sind sie anfällig für die Konservierung historischer Muster.

\Enl[1]

\Definition{Kern und Peripherie}{\label{def:kernperi}%
Der \textit{Kern} des Sprachsystems wird durch Klassen mit hoher Typenhäufigkeit gebildet.
Die \textit{Peripherie} bilden Wörter oder Konstruktionen in Klassen, die eine geringe Typenhäufigkeit haben.
Periphere Wörter und Konstruktionen können durchaus eine hohe Tokenhäufigkeit aufweisen.
}

\index{Akzeptabilität}
\index{Grammatikalität}

Der Unterschied von Kern und Peripherie darf auf keinen Fall mit Grammatikalität oder Akzeptabilität verwechselt oder vermengt werden.
Wenn nicht alle Sprecher eine bestimmte Konstruktion als akzeptabel einstufen bzw.\ ein einzelner Sprecher eine Konstruktion als weniger akzeptabel ansieht, ist diese Konstruktion damit nicht automatisch peripher.
Im Fall von solcher Variation in der Akzeptabilität geht es grundsätzlich darum, ob eine Konstruktion überhaupt zu einem grammatischen System gehört.
Die Unterscheidung nach Kern und Peripherie ist nur relevant für alle Wörter und Konstruktionen, die auf jeden Fall zum System gehören.
Es geht bei der Frage nach Kern und Peripherie einfach gesagt nur um die Größe von Klassen innerhalb des Systems.%
\footnote{Es mag sein, dass eine niedrige Typenhäufigkeit gepaart mit einer niedrigen Tokenhäufigkeit dazu führt, dass Wörter oder Konstruktionen über kurz oder lang aus dem System verschwinden.
Das ist aber eine Frage, die explizit historisch ausgerichtete oder kognitive bzw.\ psycholinguistische Theorien beantworten müssen -- und nicht etwa die deskriptive Grammatik.}

\Stretch[0.5]

\Zusammenfassung{%
Wir konzentrieren uns auf die Aspekte von Sprache, die sich als Symbolsystem beschreiben lassen.
Eine Grammatik beschreibt bestimmte Kombinationen von Symbolen auf verschiedenen Ebenen (Laute, Wörter usw.).
Diese Kombinationen -- und nur diese -- sind grammatisch.
Der Kern des Systems wird von den Klassen von Einheiten gebildet, die eine hohe Typenhäufigkeit haben.
}

\Stretch[1]

\section{Deskriptive und präskriptive Grammatik}

\label{sec:deskriptivenormativegrammatik}

\subsection{Beschreibung und Vorschrift}

In diesem Abschnitt wird die \textit{deskriptive} (beschreibende) Grammatik von jeweils anderen Arten der Grammatik abgegrenzt.
Als erstes wird eine Definition der deskriptiven Grammatik als Ausgangsbasis gegeben, s.\ Definition~\ref{def:deskgr}.

\Np

\Definition{Deskriptive Grammatik}{\label{def:deskgr}%
\textit{Deskriptive Grammatik} ist die wertneutrale Beschreibung von Sprachsystemen.
Sie beschreibt Sprachen so, wie sie beobachtet werden.
\index{Grammatik!deskriptiv}
}

\Stretch[0.5]

Wichtig ist nun die Abgrenzung zur \textit{präskriptiven Grammatik}.
Die Duden"=Grammatik \citep{Duden8} wird in ihrer aktuellen Auflage mit dem Slogan \textit{Unentbehrlich für richtiges Deutsch} beworben.
Dieser Slogan könnte so verstanden werden, dass in der Duden"=Grammatik Vorschriften für die korrekte Bildung von grammatischen Strukturen des Deutschen beschrieben werden.
Während im Duden zur Rechtschreibung also die Schreibung der Wörter in ihrer verbindlich korrekten Form festgelegt ist, könnte im Grammatik"=Band der Duden"=Redaktion der korrekte Bau von Wörtern, Sätzen und vielleicht sogar größeren Einheiten wie Texten verbindlich festgelegt sein.
Der Slogan spielt mit einem normativen oder präskriptiven Anspruch:
Was in dieser Grammatik steht, definiert richtiges Deutsch.
Ein solcher Anspruch unterscheidet die präskriptive Grammatik prinzipiell von der deskriptiven, die stets nur möglichst genau beschreiben möchte, wie bestimmte Sprachen oder alle Sprachen beschaffen sind.
Betrachtet man die Liste der Autoren der Duden"=Grammatik, die durchweg renommierte Linguisten sind, die keine stark präskriptiven Ansichten vertreten, ist im Übrigen davon auszugehen, dass der hier diskutierte Slogan vom Verlag und nicht von den Autoren stammt. 
Es handelt sich bei der Duden"=Grammatik zweifelsohne um eine der wichtigen \textit{deskriptiven Grammatiken} des Deutschen.
Wir definieren die präskriptive Grammatik in Definition~\ref{def:praegram}.

\Stretch[0.5]

\Definition{Präskriptive Grammatik}{\label{def:praegram}%
Die \textit{präskriptive} (auch \textit{normative}) \textit{Grammatik} will verbindliche Regeln festlegen, die korrekte von inkorrekter Sprache trennen.
Sie beschreibt eine Sprache, die erwünscht ist bzw.\ gefordert wird.
\index{Grammatik!präskriptiv}
}

Definition~\ref{def:praegram} verlangt bei genauem Hinsehen sofort nach einem Zusatz.
Während es bei Gesetzen meistens klar geregelt ist, wer das Recht hat, sie zu erlassen, in welchem Bereich sie gelten, und was bei Zuwiderhandlung geschieht, ist dies bei normativen grammatischen Regeln überhaupt nicht klar.
Auf diese Frage kommen wir in Abschnitt~\ref{sec:normalsbeschreibung} nach einigen weiteren terminologischen Klärungen zurück.


\Stretch[2]

\subsection{Regel, Regularität und Generalisierung}

\label{sec:regulgen}

In einer Grammatik der gegenwärtigen deutschen Standardsprache, die einen präskriptiven Anspruch erhebt, würde man vielleicht Regeln wie in (\ref{ex:gb002}) erwarten.\index{Standarddeutsch}

\begin{exe}
  \ex\label{ex:gb002}
  \begin{xlist}
    \ex{Relativsätze und eingebettete \textit{w}-Sätze werden nicht durch Komplementierer eingeleitet.}
    \ex{\textit{fragen} ist ein schwaches Verb.}
    \ex{\label{ex:gb002c} \textit{zurückschrecken} bildet das Perfekt mit dem Hilfsverb \textit{sein}.}
    \ex{Im Aussagesatz steht vor dem finiten Verb genau ein Satzglied.}
    \ex{In Kausalsätzen mit \textit{weil} steht das finite Verb an letzter Stelle.}
  \end{xlist}
\end{exe}

Man kann sich nun fragen, ob man den Regeln in (\ref{ex:gb002}) irgendwie ansieht, dass sie präskriptiv sein sollen.
Die Antwort muss \textit{Nein} lauten, denn es könnte sich auch einfach um die Beschreibungen von Beobachtungen handeln.
Im Kontext einer präskriptiven Grammatik werden solche Sätze allerdings nicht als Beobachtungen, sondern als Regeln mit Verbindlichkeitscharakter vorgetragen.
Ob die Beschreibung eines grammatischen Phänomens deskriptiv (als Beschreibung) oder präskriptiv (als Regel) verstanden werden soll, kann man nicht an der Art ihrer Formulierung ablesen, sondern nur an dem Kontext, in dem sie vorgetragen wird.
Zunächst benötigen wir jetzt Definitionen der Begriffe \textit{Regularität} (Definition~\ref{def:regularitaet}) und \textit{Regel} (Definition~\ref{def:regel}).%
\footnote{Es gibt auch andere nicht-präskriptive Verwendungen des Regelbegriffs in der Linguistik.
Oft wird einfach \textit{Regel} für \textit{Regularität} gebraucht, weil die Verwechslungsgefahr mit einem präskriptiven Vorgehen sowieso nicht besteht.
Außerdem gibt es technische Definitionen davon, was Regeln sind, die aber in entsprechenden Texten auch hinreichend eingeführt werden.}
Dem Begriff der Regel steht dann noch der Begriff der \textit{Generalisierung} (Definition~\ref{def:generalisierung}) gegenüber.

\Np

\Definition{Regularität}{\label{def:regularitaet}%
Eine grammatische \textit{Regularität} innerhalb eines Sprachsystems liegt dann vor, wenn sich Klassen von Symbolen unter vergleichbaren Bedingungen gleich (und damit vorhersagbar) verhalten.
\index{Regularität}
}

\Definition{Regel}{\label{def:regel}%
Eine grammatische \textit{Regel} ist die Beschreibung einer Regularität, die in einem normativen Kontext geäußert wird.
\index{Regel}
}

\Definition{Generalisierung}{\label{def:generalisierung}\index{Generalisierung}%
Eine grammatische \textit{Generalisierung} ist eine durch Beobachtung zustandegekommene Beschreibung einer Regularität.
}

Eine Regularität ist also ein Phänomen des Betrachtungsgegenstandes \textit{Sprache}, das Vorhandensein von Regularitäten in sprachlichen Daten ergibt sich aus dem Systemcharakter von Sprache (Definition~\ref{def:grammatik}).
Gäbe es keine Regularitäten, könnte man zugespitzt nur von einem \textit{Sprachchaos} statt von einem \textit{Sprachsystem} sprechen.
Dagegen sind Regeln und Generalisierungen vom Menschen gemacht und werden im Prinzip auf identische Weise formuliert.
Während eine Regel dabei Ansprüche an die Eigenschaften einer Sprache stellt, stellt die Generalisierung das Vorhandensein von Eigenschaften nur fest.

Wichtig ist nun, dass es sowohl von Regeln als auch von Generalisierungen nahezu immer Abweichungen gibt.
Im Fall der Regel handelt es sich bei jeder Abweichung um eine Zuwiderhandlung, im Fall der Generalisierung ist eine Abweichung nur eine Beobachtungstatsache, die von der Generalisierung nicht adäquat vorhergesagt wird.
Die Sätze in (\ref{ex:gb003}) wurden in verschiedenen Formen von Sprechern des Deutschen gesprochen oder geschrieben.
Sie stellen jeweils eine Abweichung zu (\ref{ex:gb002}) dar.

\begin{exe}
  \ex\label{ex:gb003}
  \begin{xlist}
    \ex{\label{ex:gb003a} Dann sieht man auf der ersten Seite wann, wo und wer dass kommt.%
      \footnote{\raggedright{\url{http://www.caliberforum.de/} (25.01.2010)}}}
    \ex{\label{ex:gb003b} Er frägt nach der Uhrzeit. \footnote{\raggedright{DeReKo, A99\slash NOV.83902}}}
    \ex{\label{ex:gb003c} Man habe zu jener Zeit nicht vor Morden zurückgeschreckt.%
      \footnote{\raggedright{DeReKo, A98\slash APR.20499}}}
    \ex{\label{ex:gb003d} Der Universität zum Jubiläum gratulierte auch Bundesminister Dorothee Wilms, die in den fünfziger Jahren in Köln studiert hatte.%
      \footnote{\raggedright{Kölner Universitätsjournal, 1988, S.~36, zitiert nach \citealp{Mueller03}}}}
    \ex{\label{ex:gb003e} Das ist Rindenmulch, weil hier kommt noch ein Weg.%
      \footnote{\raggedright{RTL2 \textit{Big Brother VI -- Das Dorf} (20.04.2005)}}}
  \end{xlist}
\end{exe}

Aus einer präskriptiven Perspektive kann man feststellen, dass diese Sätze in (\ref{ex:gb003}) alle falsch sind, wenn man (\ref{ex:gb002}) als Regeln aufgestellt hat.%
\footnote{Wir nehmen hier im Sinne der Argumentation an, dass dies der Fall ist.
Es soll damit nicht unterstellt werden, dass irgendeine auf dem Markt befindliche Grammatik solche Regeln aufstellt.
Es ist jedoch davon auszugehen, dass für jeden Satz Sprecher zu finden sind, die ihn für normativ falsch halten.}
Aus Sicht der deskriptiven Grammatik fängt mit dem Auffinden solcher Sätze (also mit der Feststellung von \textit{grammatischer Variation}) die eigentliche Arbeit und der Erkenntnisprozess erst an, denn keiner der Sätze ist willkürlich falsch.
Viele Abweichungen von der Norm oder von bereits aufgestellten Generalisierungen zeigen nämlich strukturelle Möglichkeiten auf, die das Sprachsystem anbietet, und die \zB in Dialekten (bzw.\ von Sprechergruppen jeder Art) genutzt werden.\index{Dialekt}
Ob diese Abweichungen dann zur sogenannten \textit{Standardsprache} gehören oder nicht, ist eine davon unabhängige Frage, die vor allem von der Definition des Standards abhängt.
Zu dieser Frage folgt weiter unten noch mehr.
Jetzt geht es erst einmal nur um die deskriptive Einordnung der Phänomene in (\ref{ex:gb003}).

Beispiel (\ref{ex:gb003a}) zeigt die Konstruktion eines eingebetteten \textit{w}-Fragesatzes mit einem Komplementierer (\textit{dass}), die nicht nur systematisch in vielen südlichen regionalen Varietäten des Deutschen vorkommt, sondern die auch aus grammatiktheoretischen Überlegungen durchaus interessant ist.\index{Satz!w-Frage--}
Die Häufung von Fragepronomina ist davon unabhängig, macht den Satz aber umso interessanter.
Beispiel (\ref{ex:gb003b}) zeigt \textit{fragen} als starkes Verb mit Umlaut in der 3.\ Person Singular Präsens Indikativ.
Aus deskriptiver Sicht schwankt hier ein Verb im gegenwärtigen Sprachgebrauch zwischen starker und schwacher Flexion (Abschnitt~\ref{sec:vvflex}).
Weiterhin ist die häufig vorkommende Alternation von \textit{sein} und \textit{haben} bei der Perfektbildung wie in (\ref{ex:gb003c}) ein theoretisch relevantes Phänomen, weil es bei der Beantwortung der Frage hilft, welche grundsätzliche Systematik hinter der Wahl des Hilfsverbs (abhängig vom Vollverb) steckt.
Beispiel (\ref{ex:gb003d}) illustriert ein syntaktisches Phänomen, nämlich das der \textit{doppelten Vorfeldbesetzung}.\index{Vorfeld}
Hier stehen scheinbar zwei Satzglieder vor dem finiten Verb (\textit{der Universität} und \textit{zum Jubiläum}), wobei die etablierte Generalisierung eigentlich die ist, dass dort nur ein Satzglied stehen kann (vgl.\ Abschnitt~\ref{sec:konstituententestsimeinzelnen} und Kapitel~\ref{sec:saetze}).
Die Beschreibung dieser Sätze in bestehende Theorien zu integrieren, ist aber durchaus möglich, und man erhält dabei eine hervorragende Möglichkeit, die Flexibilität und Adäquatheit der entsprechenden Theorien zu überprüfen.%
\footnote{Das Phänomen der doppelten Vorfeldbesetzung wird in \citet{Mueller03} diskutiert, wo auch auf Lösungsansätze verwiesen wird.
Es wird in dem vorliegenden Buch wegen seiner Komplexität nicht ausführlich besprochen.}

Dass geschriebene Sätze wie (\ref{ex:gb003e}) oft als inakzeptabel (bzw.\ \textit{falsch}) wahrgenommen werden, liegt oft daran, dass sie in der geschriebenen Sprache selten, dafür in der gesprochenen Sprache umso häufiger sind.
Nach Komplementierern (\textit{obwohl}, \textit{dass}, \textit{damit} usw.) steht im Nebensatz normalerweise das finite Verb (hier \textit{kommt}) an letzter Stelle, was in (\ref{ex:gb003e}) nicht der Fall ist.
Aus deskriptiver Perspektive fällt vor allem auf, dass \textit{weil} hier wie \textit{denn} verwendet wird.
Außerdem hat \textit{weil} mit der Nebensatz-Wortstellung wie in (\ref{ex:gb003e}) Verwendungsbesonderheiten, die es auch funktional plausibel machen, zwischen zwei verschiedenen syntaktischen Mustern in \textit{weil}-Nebensätzen zu unterscheiden.
In all diesen Fällen einfach von falschem oder richtigem Sprachgebrauch zu sprechen, wäre ganz einfach nicht angemessen.

Es sollte klar geworden sein, warum für eine wissenschaftliche Betrachtung die normative Vorgehensweise nicht infrage kommt.
Stattdessen widmen wir uns in diesem Buch der deskriptiven Grammatik und beschreiben, welche sprachlichen Konstrukte Sprecher systematisch produzieren, einschließlich eventueller systematischer Alternativen und Schwankungen.
Durch genau diesen Anspruch handeln wir uns allerdings gleich ein ganzes Bündel von praktischen Problemen ein.
Welche systematischen Phänomene suchen wir aus?
Wie systematisch muss ein Phänomen beobachtbar sein, damit es in die Beschreibung aufgenommen wird?
Welche dialektalen bzw.\ regionalen Varianten des Deutschen wollen wir mit unserer Beschreibung abdecken?
Beschreiben wir auch Konstruktionen, die zwar systematisch vorkommen, aber nur in der gesprochenen Sprache?
Neben der genannten \textit{dialektalen Variation} gibt es mindestens auch noch Variation zwischen sozialen Gruppen (\zB Jugend- oder Kiezsprachen) -- sogenannte \textit{diastratische Variation} -- und \textit{individuelle Variation}.\index{Variation}\index{Dialekt}
Da sich Sprache auch über die Zeit wandelt (\textit{diachrone Variation}), muss außerdem der Zeitraum festgelegt werden, den man beschreiben möchte.
Weil bei genauem Hinsehen Sprache ein ausuferndes Maß an Variation aufweist, ist das Grundproblem, nämlich die Definition des zu beschreibenden Gegenstandes, nicht systematisch, sondern nur pragmatisch lösbar.
 
Ganz pragmatisch orientieren wir uns daher bei unserer Beschreibung an einer quasi normierten deutschen Standardsprache, wie sie zum Beispiel in der Duden-Grammatik \citep{Duden8} oder in Peter Eisenbergs \textit{Grundriss der deutschen Grammatik} \citep{Eisenberg1,Eisenberg2} beschrieben wird.\index{Standarddeutsch}
Nur so ist überhaupt ein systematischer Einstieg in die Sprachbeschreibung möglich.
Der nächste Abschnitt diskutiert die Gründe, warum dieser vermeintliche Rückzug nach allem, was wir kritisch über normative Grammatik gesagt haben, trotzdem zulässig ist.
Abschnitt~\ref{sec:empirie} behandelt diese Frage in einem größeren theoretischen Kontext.

\subsection{Norm als Beschreibung}

\label{sec:normalsbeschreibung}

Bisher wurde nicht geklärt, ob es eine Institution gibt, die für das Deutsche irgendwelche Sprachnormen (also Regeln für den zulässigen Gebrauch von Grammatik) erlässt.
Es gibt sie nicht.
Während es \zB in Frankreich die Französische Akademie (Académie française) gibt, die einen staatlich legitimierten Normierungsauftrag hat, existiert eine vergleichbare Institution in Deutschland nicht.%
\footnote{\raggedright{\url{http://www.academie-francaise.fr/}}}
Die Kultusministerkonferenz (das Gremium, das für die bundesweite Normierung von Bildungsfragen zuständig ist) beschäftigt sich nicht intensiv mit Fragen der Grammatik, wohl aber mit Fragen der Orthographie.%
\footnote{\url{http://www.rechtschreibrat.com/}}
Das staatlich finanzierte Institut für Deutsche Sprache (IDS) könnte zunächst für eine normative Organisation gehalten werden, aber schon der zweite Satz der Selbstdarstellung des IDS lässt erkennen, dass dies nicht der Fall ist:

\begin{quote}
  [Das IDS] ist die zentrale außeruniversitäre Einrichtung zur Erforschung und Dokumentation der deutschen Sprache in ihrem gegenwärtigen Gebrauch und in ihrer neueren Geschichte.%
    \footnote{\raggedright{\url{http://www.ids-mannheim.de/} (21.09.2010)}}
\end{quote}

Außerdem wird oft, wie bereits erwähnt, die Duden"=Grammatik als normierend angesehen, auch wenn dem Duden"=Verlag dafür kein staatlicher oder gesellschaftlicher Auftrag erteilt wurde.
Die aktuelle Duden"=Grammatik wurde von Linguisten verfasst, die selber deskriptiv arbeiten und sehr wahrscheinlich den Anspruch haben, diejenige Sprache zu beschreiben, die von den Sprechern mehrheitlich als Standard akzeptiert wird (mit allen oben angedeuteten unvermeidbaren Unschärfen).
Insofern ist die Duden-Grammatik (bzw.\ jede gute deskriptive Grammatik) auch durchaus \textit{unentbehrlich für richtiges Deutsch}.
Eine solche Grammatik beschreibt eine Sprache, die von vielen Sprechern als natürlich und wenig dialektal geprägt empfunden wird.
Unentbehrlich ist eine solche Beschreibung, wenn Deutsch zum Beispiel als Fremdsprache gelernt wird, oder wenn in formeller Kommunikation eine möglichst neutrale Sprache erforderlich ist.
Von einer zweifelsfreien Unterscheidung von falsch und richtig in allen Details kann aber keine Rede sein.
Insofern richten wir ohne schlechtes Gewissen unsere Beschreibung an einer Quasi"=Norm aus, die letztlich durch Beobachtung zustande gekommen ist.
Einige elementare Probleme mit der empirischen Ermittlung dieser Quasi"=Norm bzw.\ dieses grammatischen Grundkonsenses werden im nächsten Abschnitt besprochen. 

\subsection{Empirie}

\label{sec:empirie}

\index{Empirie}

Dieser Abschnitt diskutiert zum Abschluss des Kapitels einige weiterreichende Überlegungen zu empirischen Methoden in der Linguistik.
Mit dem eindeutigen Bekenntnis zur deskriptiven Grammatik wird die Grammatik zu einer empirischen Wissenschaft, die sich dann auch an üblichen Standards empirischer Wissenschaften messen lassen muss.     
In jeder empirischen Wissenschaft stellt sich die Frage:
\textit{Woher wissen wir das alles?}
Naturwissenschaften können diese Frage in der Regel mit dem Verweis auf eine Jahrhunderte lange Tradition in Theoriebildung und experimenteller Überprüfung dieser Theorien beantworten.%
\footnote{Für die Physik gibt Harald Lesch in der 72.\ Sendung von \textit{Alpha Centauri} (BR\slash ARD alpha, Erstausstrahlung 14. Juni 2001) eine populärwissenschaftlich aufbereitete, aber souveräne Antwort: \url{https://youtu.be/gqQCrKVD2-s}}
Es gilt dann \zB in der Physik, dass Theorien wie die Allgemeine Relativitätstheorie oder die Quantenmechanik (jeweils in ihrem wohldefinierten Gültigkeitsbereich) angemessene Beschreibungen der Wirklichkeit darstellen.
Die Feinheit der Experimente und Beobachtungen sowie die mathematische Präzision aktueller Theorien erlaubt es Physikern, mit sehr hoher Sicherheit anzunehmen, dass die Theorien tatsächlich in diesem Sinn \textit{adäquat} sind.

Wo steht die Grammatik bzw.\ Linguistik in dieser Hinsicht?
In Abschnitt~\ref{sec:normalsbeschreibung} wurde die Quasi"=Norm, um die es hier gehen soll, als Beschreibung enttarnt und zu einem grammatischen Grundkonsens innerhalb der Sprechergemeinschaft des in Deutschland gesprochenen Deutsch entschärft.
Der Grundkonsens wird existierenden Grammatiken entnommen, so dass eine eigene Empirie hier nicht stattfindet.
Das hat in erster Linie damit zu tun, dass dieses Buch eine Einführung in die grammatische Beschreibung und damit weder ein eigenständiges wissenschaftliches Werk, noch eine Einführung in die linguistische Methodenlehre ist.
Die Grundsatzfrage wird damit aber nur auf die Autoren der Duden"=Grammatik oder Einzelautoren wie Peter Eisenberg verschoben, denn die konsensuellen Erkenntnisse über die Grammatik des Deutschen müssen von irgendjemandem ursprünglich geschöpft und im besten Fall immer wieder empirisch überprüft worden sein.
Eine solide deskriptive Grammatik kann nur auf empirisch gewonnenen Daten basieren, und linguistische Theorien müssen anhand solcher Daten überprüfbar sein.
Es muss also wenigstens kurz diskutiert werden, wie eine deskriptive Grammatik zum gegenwärtigen Stand der Forschung empirisch abgesichert werden kann.

Es haben sich verschiedene Methoden etabliert, um innerhalb der Linguistik empirisch zu arbeiten und Daten zu erheben.
Die drei wichtigsten sind das \textit{Experiment}, die \textit{Befragung} und die \textit{Korpusstudie}.
Bei einem \textit{Experiment} werden Sprecher unter kontrollierten Bedingungen mit sprachlichen Reizen konfrontiert.\index{Experiment}
Das Ziel ist dabei, ihre Reaktion zu messen oder sie zur Sprachproduktion zu animieren, um aus den Reaktionen auf Eigenschaften ihres mental repräsentierten Sprachsystems zu schließen.
Dabei wissen Probanden normalerweise nicht explizit, welcher Aspekt ihrer Sprache untersucht werden soll.
Bei der \textit{Befragung} werden mehr oder weniger direkt Urteile über sprachliche Phänomene von Sprechern erbeten, \zB ob ein Ausdruck akzeptabel ist oder ob zwei Ausdrücke im gegebenen Kontext gleichermaßen verwendbar sind.\index{Befragung}

Eine Sonderform der Befragung stellt die \textit{Introspektion} dar, bei der Linguisten bzw.\ Grammatiker sich selbst oder einander befragen.\index{Introspektion}
Typischerweise sieht das so aus, dass Wörter, Konstruktionen oder Sätze von Linguisten, die die betreffende Sprache nicht einmal unbedingt als Erstsprache sprechen, als grammatisch oder ungrammatisch klassifiziert werden.
Zunächst erscheint diese Methode hochgradig manipulativ und eine vollständige Karikatur empirischer Methoden zu sein.
Das empirische Vorgehen dient schließlich dazu, Forschungsergebnisse vom Individuum und seiner persönlichen Bewertung unabhängig zu machen.
Außerdem sollen empirisch gewonnene Ergebnisse prinzipiell reproduzierbar sein.
Die persönliche Einschätzung eines einzelnen Linguisten kann weder reproduziert werden, noch ist sie unabhängig von seiner Person.
Ein introspektives Vorgehen ist allerdings vergleichsweise unproblematisch, wenn es um sprachliche Strukturen geht, die trivialerweise ungrammatisch oder grammatisch sind.
Die Probleme kommen durch die Hintertür, wenn die Grenze zwischen trivialerweise Grammatischem und trivialerweise Ungrammatischem nicht mehr klar ist.
In (\ref{ex:gr9821544}) ist die Situation wohl eindeutig.

\begin{exe}
	\ex\label{ex:gr9821544}
	\begin{xlist}
		\ex[ ]{\label{ex:gr9821544a} Tania sprang vom Einmeterbrett.}
		\ex[*]{\label{ex:gr9821544b} Tania springte vom Einmeterbrett.}
	\end{xlist}
\end{exe}

Dass (\ref{ex:gr9821544a}) für Erstsprecher des Deutschen grammatisch ist, bedarf keiner empirischen Überprüfung.
Ebenso ist (\ref{ex:gr9821544b}) trivialerweise ungrammatisch.
Selbst wenn es Sprecher gäbe, die \textit{springte} statt \textit{sprang} verwenden, würde man diese wahrscheinlich besser einer Varietät jenseits des Standards zuordnen, als das System des Standarddeutschen umzuschreiben.\index{Variation}\index{Standarddeutsch}
Das Problem ist der Übergang von trivialen zu weniger trivialen Entscheidungen über Grammatikalität.
Bereits für Sätze wie (\ref{ex:dg9113}) auf S.~\pageref{ex:dg9113} ist die Entscheidung erfahrungsgemäß alles andere als trivial.
Beispiel (\ref{ex:gr9821545}) treibt die Frage noch ein Stück weiter.

\begin{exe}
	\ex[?]{\label{ex:gr9821545} Tania vom Einmeterbrett sprang.}
\end{exe}

Viele Sprecher und Linguisten würden hier klar Ungrammatikalität diagnostizieren, weil das Verb hier nicht am Satzende stehen kann (s.\ Kapitel~\ref{sec:saetze}).
Trotzdem besteht ein Unterschied zu Sätzen wie (\ref{ex:gb2957b}) auf S.~\pageref{ex:gb2957b}, der einem Wortsalat nahekommt.
Für das gegebene Beispiel wäre eine entsprechende Variante (als Wortsalat) \zB (\ref{ex:gr9821545555555}).

\begin{exe}
	\ex[*]{\label{ex:gr9821545555555} Vom sprang Tania Einmeterbrett.}
\end{exe}

Manche würden vielleicht argumentieren, dass nur in poetischer oder (vermeintlich) altertümlicher Sprache eine Satzgliedstellung wie in (\ref{ex:gr9821545}) möglich ist.%
\footnote{Auf ähnliche Weise könnte man sich in dialektale Variation oder andere Dimensionen der Variation (vgl.\ Abschnitt~\ref{sec:regulgen}) retten.}
Das passt dazu, dass die Akzeptanz für (\ref{ex:gr9821545}) wahrscheinlich größer wird, wenn ein zweiter Satz hinzukommt, der metrisch ähnlich ist und sich reimt.
Ideal wäre in dieser Hinsicht die Version in (\ref{ex:gr9821546}), zumindest wenn wir die ästhetische Qualität außer Acht lassen.

\begin{exe}
	\ex[?]{\label{ex:gr9821546} Tania vom Einmeter sprang\\und die Konkurrenz bezwang.}
\end{exe}

Lässt man sich auf solche Argumentationen ein, erklärt man aber entweder, dass poetische Sprache eine eigenständige, abweichende Grammatik hat, oder dass die Bedingungen für Grammatikalität in dieser Sprache gelockert sind.
Im Extremfall führt dieser Ansatz dazu, dass für jede Textsorte, jedes Register usw.\ eigene Grammatiken definiert werden müssen.%
\footnote{Es ist nicht auszuschließen, dass dies langfristig das angemessene Vorgehen ist, um Variation in Sprache adäquat zu modellieren.
Allerdings schrumpft der Unterschied zwischen der Bewertung von Fällen wie (\ref{ex:gr9821544b}) und Fällen wie (\ref{ex:gr9821545}) dann stark zusammen und ist nur noch ein gradueller und kein kategorischer mehr.}
Wenn Grammatiken und Theorien von Grammatik aber gar nicht zwischen Registern, Textsorten und anderen Einflussquellen unterscheiden bzw.\ ihren Gültigkeitsbereich nicht spezifisch auf solche Register, Textsorten usw.\ einschränken, kann das implizite Hin- und Herspringen zwischen ihnen ausgenutzt werden, um eigentlich inadäquate Theorien als adäquat hinzustellen.
Entweder beschreibt man also einen breiten Grundkonsens, kann dabei aber notgedrungen nicht sehr genau werden und sich nur begrenzt in der Beschreibung bzw.\ der theoretischen Modellierung festlegen, oder man macht sehr feine Unterscheidungen bezüglich des Gültigkeitsbereichs einzelner grammatischer Generalisierungen.%
\footnote{Andere Einflussquellen, die man berücksichtigen müsste, sind \zB der größere Informationskontext einzelner Äußerungen und letztlich auch immer individuelle Schwankungen.}
Die üblichen Auseinandersetzungen zwischen Linguisten, ob irgendein theorieentscheidender Satz \textit{im Deutschen} (bzw.\ einer anderen Sprache) grammatisch ist oder nicht, sind auf jeden Fall ein untrügliches Anzeichen dafür, dass der Grundkonsens verlassen wurde und damit auch der Bereich, der mit Introspektion abgedeckt werden darf.%
\footnote{Es wurde festgestellt, dass die Grammatikalitätsmarkierungen in linguistischen Artikeln mit experimentell gewonnenen Ergebnissen vergleichsweise gut, aber nicht perfekt übereinstimmen \citep{SprouseEa2013}.
Das heißt noch nicht, dass die introspektive Methode an sich gerettet ist und aufwändigere Empirie nicht sein muss.
Es geht vielmehr darum, Akzeptabilitätsurteile als Datenquelle an sich zu erhalten \citep{SchuetzeSprouse2014}.}

	Die dritte wichtige empirische Methode (und die, bei der die größten Datenmengen berücksichtigt werden können) ist die \textit{Korpusstudie}.\index{Korpus}
Ein \textit{Korpus} ist ganz allgemein gesprochen eine Sammlung von Texten aus einer oder mehreren Sprachen, ggf.\ auch aus verschiedenen Epochen und Regionen.%
\footnote{Fachsprachlich ist das Wort \textit{Korpus} immer ein Neutrum, also niemals *\textit{der Korpus}.
Der Plural lautet \textit{Korpora}.}
Man könnte unter anderem Korpora mit folgenden Inhalten erstellen:

\begin{itemize}\Lf
  \item möglichst alle Texte aus Berliner Lokalzeitungen von 1890--1910,
  \item Interviews von Bundesliga-Fußballerinnen aus der Spielzeit 2010\slash 2011,

    \Np
  
  \item eine Stichprobe von Texten deutscher Webseiten,%
		\footnote{Ein sehr großes Korpus aus deutschen Internettexten (21 Mrd.\ Wörter und Satzzeichen), die naturgemäß viel nicht-standardsprachliche Variation enthalten, ist DECOW14 \citep{SchaeferBildhauer2012a}.
		Es kann online eingesehen werden: \url{http://corporafromtheweb.org/}.}
  \item eine nach genau definierten Kriterien zusammengestellte Auswahl deutscher Texte aus den Gattungen Belletristik, Gebrauchstext, wissenschaftlicher Text und Zeitungstext aus dem zwanzigsten Jahrhundert.%
		\footnote{Ein solches Korpus wird von den Machern des Digitalen Wörterbuchs der deutschen Sprache (DWDS) erstellt: \url{http://www.dwds.de/}.}
\end{itemize}

In solchen Korpora kann man gezielt nach Material zu bestimmten grammatischen Phänomenen suchen und sowohl die Variation innerhalb des Phänomens beschreiben, aber natürlich auch die statistisch dominanten Muster herausarbeiten.
Letztere eignen sich dann zur Darstellung in einer deskriptiven (wenn man möchte auch normativ interpretierbaren) Grammatik.
Zusätzlich erlauben es Korpora oft, den Sprachgebrauch mit bestimmten Texttypen in Beziehung zu setzen, \zB Zeitungsartikel, wissenschaftliche Texte, gesprochene Sprache.
Da es heutzutage möglich ist, sehr große Korpora (prinzipiell im Bereich von Hunderten von Milliarden Textwörtern) zu erstellen, die eine enorme Variationsbreite enthalten, eignen sich Korpora besonders für das Herausarbeiten des inzwischen viel besprochenen Grundkonsenses.

Nur zur Illustration werden in diesem Buch gelegentlich Beispiele aus dem Deutschen Referenz-Korpus (DeReKo) des Instituts für Deutsche Sprache (IDS) in Mannheim zitiert.\label{abs:dereko}
Dieses Korpus enthält vor allem Zeitungstexte jüngeren Datums und kann online benutzt werden.%
\footnote{\url{http://www.ids-mannheim.de/cosmas2/}}
Gelegentlich wird das DeReKo fälschlicherweise als COSMAS bezeichnet.
Bei COSMAS (bzw.\ COSMAS2) handelt es sich aber nur um das Recherchesystem, nicht um das Korpus selber.

Auf Basis der grundsätzlichen Überlegungen in diesem Kapitel werden im weiteren Verlauf des Buchs die wichtigen Einzelthemen der Grammatik des gegenwärtigen Deutsch besprochen.
Die diskutierten Einschränkungen (keine eigene Empirie, keine Berücksichtigung regionaler Variation, Tendenz zur Beschreibung der verschrifteten Standardsprache usw.) sollten dabei bewusst bleiben.
Es geht also bei allem Bezug auf den Standard keinesfalls um \textit{das Deutsche an sich}.\index{Standarddeutsch}

\Zusammenfassung{%
Es ist nicht die Aufgabe der Sprachwissenschaft, den richtigen Sprachgebrauch zu definieren.
Einen richtigen Sprachgebrauch gibt es nicht, sondern nur historisch gewachsene grammatische Konventionen innerhalb von Sprechergemeinschaften.
Die Konventionen erlauben durchaus Variation, \zB zwischen verschiedenen Regionen.
Die wichtigste und schwierigste Aufgabe der Grammatik als Wissenschaft ist es, durch empirische Verfahren herauszufinden, was diese Konventionen sind und sie zu beschreiben.
}
