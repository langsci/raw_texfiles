\chapter{Nouns and pronominal prefix selection} \label{ch:Nouns and pronominal prefix selection}
This section describes pronominal prefix selection for nouns, beginning with \textsc{unpossessed} nouns \xref{ex:possunpossa}, and then \textsc{possessed} nouns \xref{ex:possunpossb}. Unpossessed nouns do not denote ownership, while possessed nouns do. 

\ea\label{ex:possunposs} 
\ea ganǫ́hsaˀ ‘house(s)’ (unpossessed noun)\label{ex:possunpossa}
\ex aknǫ́hsaˀ ‘my house(s)’ (possessed noun)\label{ex:possunpossb}
\z
\z

\tabref{figtab:1:pronselnouns} (page \pageref{figtab:1:pronselnouns}) summarizes pronominal prefix selection for nouns. 

\begin{table}
\caption{Pronominal prefix selection, nouns}
\label{figtab:1:pronselnouns}

\begin{tabularx}{\textwidth}{XXXX}

\lsptoprule
grammatical category & prefix type & prefix choice\\
\midrule
unpossessed & {}•basic nouns

{}•body part nouns inflected like basic nouns

& \stem{ga-} \textsc{3s.a}

\stem{o-} \textsc{3s.p}

∅ (no prefix)

& The type of prefix (\stem{ga-}, \stem{o-} or none) must be memorized for each for each word.\\
\midrule 


possessed& {}•basic nouns 

{}•body part nouns inflected like basic nouns

& 

\textsc{p}-series & \textsc{p}-series pronominal prefixes denote possession or ownership.\\

\midrule 
possessed& {}•body part nouns & \textsc{a}-series & \textsc{a}-series pronominal prefixes denote the person or being who has the body part in question. \\
\lspbottomrule
\end{tabularx}
\end{table}

\section{Unpossessed nouns} \label{ch:Unpossessed nouns}
Unpossessed basic nouns either take the \stem{ga-} \textsc{3s.a} (\textsc{a}-series) pronominal prefix \xref{ex:unpossnounppexa}, the \stem{o-} \textsc{3s.p} (\textsc{p}-series) pronominal prefix \xref{ex:unpossnounppexb}, or no prefix. Nouns without a prefix typically begin with A \xref{ex:unpossnounppexc}.

\ea\label{ex:unpossnounppex}
\ea\label{ex:unpossnounppexa}\exemph{ga}nǫ́hsaˀ ‘house(s)’ \textsc{a}-series
\ex\label{ex:unpossnounppexb}\exemph{o}ˀnhǫ́hsaˀ ‘egg(s)’ \textsc{p}-series
\ex\label{ex:unpossnounppexc}\exemph{a}dáhdi̱ˀtraˀ ‘sock(s)’ (no pronominal prefix)
\z
\z

While the \stem{ga-} and \stem{o-} prefixes both mean ‘it’, they are generally not interchangeable. For example, the word meaning ‘house’ always begins with \stem{ga-}, never with \stem{o-}. That being said, some nouns can take either \stem{ga-} or \stem{o-} \xref{ex:unpossnounppex2}, and still others take \stem{o-} or begin without a prefix \xref{ex:unpossnounppex3}. (See \sectref{ch:Basic nouns} for more examples.) (Dropping the \stem{o-} prefix in words like \cayuga{hǫ́na̱ˀdaˀ / ohǫ́na̱ˀdaˀ} ‘potato’ was common in the variety of Gayogo̱ho:nǫˀnéha:ˀ spoken in Oklahoma, see \sectref{ch:Gayogohó:nǫˀ, the Cayuga language}.)


\ea\label{ex:unpossnounppex2} \textsc{a}- or \textsc{p}-series\\
\exemph{ga}jíhoha:ˀ, \exemph{o}jíhyo̱ha:ˀ ‘straight pin’, ‘pin’, ‘brooch’, ‘safety pin’
\z

\FloatBarrier
\ea\label{ex:unpossnounppex3} \textsc{p}-series or no pronominal prefix\\
\ea \exemph{o}hsgwáęˀdaˀ, sgwáęˀdaˀ ‘coltsfoot’
\ex \exemph{o}gyáǫhsraˀ, \exemph{a}gyáǫhsraˀ ‘a trick’ 
\z
\z 



\subsection{Noun suffixes do not affect prefix choice} \label{ch:Noun suffixes do not affect prefix choice}
Noun suffixes do not affect the choice of \stem{ga-} or \stem{o-} pronominal prefix. For example, the nouns in \xref{ex:unpossnounppex8} retain their \stem{ga-} prefix regardless of whether or not a suffix is present, and similarly for the nouns with \stem{o-} in \xref{ex:unpossnounppex9}, and the noun without a pronominal prefix in \xref{ex:unpossnounppex7}.

\ea\label{ex:unpossnounppex8}
\ea ganǫhsáǫweh \\
\gll ga-nǫhs-á-ǫweh\\
\textsc{3s.a}-house-{\joinerA}-{\typicalizer}\\
\glt ‘cook-house’ (at the longhouse)
\cfex{ganǫ́hsaˀ\\
	\gll ga-nǫ́hs-aˀ\\
	\textsc{3s.a}-house-{\nounstemformer}\\
	\glt `house'}
\ex ganaˀjáǫweh \\
\gll ga-naˀj-á-ǫweh\\
\textsc{3s.a}-pot-{\joinerA}-{\typicalizer}\\
\glt ‘cooking pots’ (used at the longhouse) 
\cfex{ganáˀjaˀ\\
\gll  ga-náˀj-aˀ \\
\textsc{3s.a}-pot-{\nounstemformer}\\
\glt ‘pail’, ‘pot’, etc.}
\z
\z

\ea\label{ex:unpossnounppex9} 
\ea oyęhsráǫweh \\
\gll o-yęhsr-á-ǫweh \\
\textsc{3s.p}-blanket-{\joinerA}-{\typicalizer}\\
\glt ‘shawl’ (for dancing, or the type put on a corpse at a funeral)
\cfex{oyę́hsraˀ\\
\gll o-yę́hsr-aˀ\\
\textsc{3s.p}-blanket-{\nounstemformer}\\
\glt ‘blanket’ }
\ex onęhęˀǫ́:weh \\
\gll o-nęhęˀ-ǫ́:weh \\
\textsc{3s.p}-corn-{\typicalizer}\\
\glt ‘corn’ (flint corn) 
\cfex{onę́hę:ˀ\\
	\gll o-nę́hę:-ˀ\\
	\textsc{3s.p}-corn-{\nounstemformer}\\
	\glt `corn'}
\z
\z

\ea\label{ex:unpossnounppex7}
ahdahgwáǫweh \\
ahdahgw-á-ǫweh \\
ø.shoe-{\joinerA}-{\typicalizer}\\
\glt ‘shoe’, ‘moccasin’
\cfex{ahdáhgwaˀ\\
	\gll ahdáhgw-aˀ\\
	ø.shoe-{\nounstemformer}\\
	\glt `shoe'}
\z

\subsection{Body part nouns inflected as unpossessed basic nouns} \label{ch:Body part nouns inflected as unpossessed basic nouns}
While body part nouns generally take different prefixes and suffixes than basic nouns do (as described in \sectref{ch:Possessed body part nouns (a series)}), they can also be inflected just like \emph{unpossessed} basic nouns (see \sectref{ch:Body part nouns inflected as possessed basic nouns}). In such cases, they take the \stem{o-} \textsc{3s.p} prefix and the \stem{-aˀ} \textsc{\nsf} suffix. Such nouns tend to denote “detachable” body parts or elements \xref{ex:unpossnounppex4}.

\ea\label{ex:unpossnounppex4}
\ea onóˀjaˀ\\
\gll o-nóˀj-aˀ\\
 \textsc{3s.p}-tooth-{\nounstemformer}\\
\glt `tooth'
\ex ojíˀehdaˀ, ojíˀohdaˀ\\
\gll o-jíˀehd/jiǫˀd-aˀ\\
 \textsc{3s.p}-fingernail-{\nounstemformer}\\
\glt `fingernail'
\ex ogéˀa:ˀ \\
\gll o-géˀa:-ˀ\\
 \textsc{3s.p}-hair-{\nounstemformer}\\
\glt ‘hair’, `a rag'
\ex oˀdaˀ\\
\gll o-ˀd-aˀ\\
 \textsc{3s.p}-feces-{\nounstemformer}\\
\glt `feces'
\ex otsgraˀ\\
\gll o-tsgr-aˀ\\
 \textsc{3s.p}-saliva-{\nounstemformer}\\
\glt `saliva'
\ex ojínǫ̱hgraˀ\\
\gll o-jínǫ̱hgr-aˀ\\
 \textsc{3s.p}-mucus-{\nounstemformer}\\
\glt `mucus'
\ex onyáˀgwaˀ \\
\gll o-nyáˀgw-aˀ\\
 \textsc{3s.p}-vomit-{\nounstemformer}\\
\glt ‘vomit’, `vomitus'
\z
\z

Consistent with the detached meaning, unpossessed body part nouns can refer to (dismembered) body parts or toys (\ref{ex:unpossnounppex6}a, b) or to objectified body parts (\ref{ex:unpossnounppex6c},d). 

\newpage
\ea\label{ex:unpossnounppex6} 
\ea oˀyó:tsaˀ\\\label{ex:unpossnounppex6a}
\gll o-ˀyó:ts-aˀ\\
 \textsc{3s.p}-chin-{\nounstemformer}\\
\glt `a chin'
\ex onóˀa:ˀ \\\label{ex:unpossnounppex6b}
\gll o-nóˀa:-ˀ\\
\textsc{3s.p}-head-{\nounstemformer}\\
\glt ‘a head’ (owner unknown) 
\ex onę́:tsaˀ \\\label{ex:unpossnounppex6c}
\gll o-nę́:ts-aˀ \\
\textsc{3s.p}-arm-{\nounstemformer}\\
\glt ‘arm’ (said, for example, when holding up a doll’s arm for show)
\ex ohnáˀtsaˀ\\\label{ex:unpossnounppex6d}
\gll o-hnáˀts-aˀ\\
 \textsc{3s.p}-buttock-{\nounstemformer}\\
\glt `a bare butt'
\z
\z

As shown in \xref{ex:unpossnounppex5}, unpossessed body part nouns also appear in compound noun constructions (\sectref{ch:Compound nouns}). 

\ea\label{ex:unpossnounppex5} 
\ea gwihsgwíhs \exemph{onǫ́ˀa:ˀ}  \\
\gll gwihsgwíhs onǫ́ˀa:ˀ\\
 pig head\\
\glt ‘pig’s head’
\ex gwihsgwíhs \exemph{ohsíˀdaˀ}  \\
\gll gwihsgwíhs ohsíˀdaˀ\\
pig foot\\
\glt ‘pig’s feet’
\ex gwihsgwíhs \exemph{oˀwáhǫh}  \\
\gll gwihsgwíhs oˀwáhǫh\\
pig meat\\
\glt  ‘pig meat’, ‘pork chop’
\z
\z


\section{Possessed nouns} \label{ch:Possessed nouns}
Possessed nouns convey a relationship of ownership. Both basic nouns and body part nouns can denote possession, but are inflected in different ways, as described in \sectref{ch:Possessed basic nouns (o series)}.


\subsection{Possessed basic nouns (\textsc{p}-series)} \label{ch:Possessed basic nouns (o series)}
Possessed basic nouns take \textsc{p}-series pronominal prefixes to denote the possessor. A full paradigm is shown in \xref{ex:possnounppex}.\footnote{\textsc{p}-series pronominal prefixes do not distinguish between \textsc{inclusive} and \textsc{exclusive}.}


\ea\label{ex:possnounppex}
\ea aknǫ́hsaˀ\\
\gll ak-nǫ́hs-aˀ\\
 \textsc{1s.p}-house-{\nounstemformer}\\
\glt `my house'
\ex ǫknínǫ̱hsaˀ\\
\gll ǫkní-nǫ̱hs-aˀ\\
 \exsc{1d.p}-house-{\nounstemformer}\\
\glt `our house (two people)'
\ex ǫgwánǫ̱hsaˀ\\
\gll ǫgwá-nǫ̱hs-aˀ\\
 {1p.p}-house-{\nounstemformer}\\
\glt `our house (more than two people)'
\ex sanǫ́hsaˀ\\
\gll sa-nǫ́hs-aˀ\\
 \textsc{2s.p}-house-{\nounstemformer}\\
\glt `your house (one person)'
\ex sninǫ́hsaˀ\\
\gll sni-nǫ́hs-aˀ\\
 \textsc{2d.p}-house-{\nounstemformer}\\
\glt `your house (two people)'
\ex swanǫ́hsaˀ\\
\gll swa-nǫ́hs-aˀ\\
 \exsc{2p.p}-house-{\nounstemformer}\\
\glt `your house (more than two people)'
\ex honǫ́hsaˀ\\
\gll ho-nǫ́hs-aˀ\\
 \textsc{3s.m.p}-house-{\nounstemformer}\\
\glt `his house'
\newpage
\ex gonǫ́hsaˀ\\
\gll go-nǫ́hs-aˀ\\
 \textsc{3s.fi.p}-house-{\nounstemformer}\\
\glt `her house'
\ex onǫ́hsaˀ\\
\gll o-nǫ́hs-aˀ\\
 \textsc{3s.p}-house-{\nounstemformer}\\
\glt `its house'
\ex hodínǫ̱hsaˀ \\
\gll hodí-nǫ̱hs-aˀ \\
\textsc{3ns.m.p}-house-{\nounstemformer}\\
\glt ‘their (males’) house’ 
\ex godínǫ̱hsaˀ \\
\gll godí-nǫ̱hs-aˀ\\
\exsc{3ns.fi.p}-house-{\nounstemformer}\\
\glt ‘their (females’ or mixed) house’ 
\ex odínǫ̱hsaˀ \\
\gll odí-nǫ̱hs-aˀ \\
\exsc{3p.p}-house-{\nounstemformer}\\
\glt ‘their (animals’) house 
\z
\z

\subsection{Possessed body part nouns (\textsc{a}-series)} \label{ch:Possessed body part nouns (a series)}
Possessed body part nouns take \textsc{a}-series pronominal prefixes to denote the possessor (and require the \stem{-ˀgeh} \textsc{on} suffix). A full paradigm is shown in \xref{ex:possnounppex3}.

\ea\label{ex:possnounppex3}
\ea knętsáˀgeh\\
\gll k-nęts-áˀgeh\\
 \textsc{1s.a}-arm-{\on}\\
\glt `on my arm'
\ex kninę́tsa̱ˀgeh \\
\gll kni-nę́ts-a̱ˀgeh\\
\exsc{1d.in.a}-arm-{\on}\\
\glt ‘on our arm’ (two people, including listener) 
\ex akninętsáˀgeh \\
\gll akni-nęts-áˀgeh\\
\exsc{1d.ex.a}-arm-{\on}\\
\glt ‘on our arm’ (two people, excluding listener) 
\ex dwanę́tsa̱ˀgeh \\
\gll dwa-nę́ts-a̱ˀgeh\\
\exsc{1p.in.a}-arm-{\on}\\
\glt ‘on our arm’ (more than two people, including listener(s)) 
\ex agwanętsáˀgeh \\
\gll agwa-nęts-áˀgeh\\
\exsc{1p.ex.a}-arm-{\on}\\
\glt ‘on our arm’ (more than two people, excluding listener(s)) 
\ex hanę́tsa̱ˀgeh\\
\gll ha-nę́ts-a̱ˀgeh\\
 \textsc{3s.m.a}-arm-{\on}\\
\glt `on his arm'
\ex enę́tsa̱ˀgeh\\
\gll e-nę́ts-a̱ˀgeh\\
 \textsc{3s.fi.a}-arm-{\on}\\
\glt `on her arm'
\ex ganę́tsa̱ˀgeh\\
\gll ga-nę́ts-a̱ˀgeh\\
 \textsc{3s.a}-arm-{\on}\\
\glt `on its arm'
\ex snętsáˀgeh\\
\gll s-nęts-áˀgeh\\
 \textsc{2s.a}-arm-{\on}\\
\glt `on your arm (one person)'
\ex sninę́tsa̱ˀgeh\\
\gll sni-nę́ts-a̱ˀgeh\\
 \exsc{2d.a}-arm-{\on}\\
\glt `on your arm (two people)'
\ex swanę́tsa̱ˀgeh\\
\gll swa-nę́ts-a̱ˀgeh\\
 \textsc{2p.a}-arm-{\on}\\
\glt `on your arm (more than two people)'
\ex hadinętsáˀgeh\\
\gll hadi-nęts-áˀgeh\\
 \textsc{3ns.m.a}-arm-{\on}\\
\glt `on their arm (males only)'
\ex gaenętsáˀgeh\\
\gll gae-nęts-áˀgeh\\
 \textsc{3ns.fi.a}-arm-{\on}\\
\glt `on their arm (females or mixed)'
\ex gadinętsáˀgeh \\
\gll gadi-nęts-áˀgeh \\
\textsc{3p.a}-arm-{\on}\\
\glt ‘on their arm (animals) 
\z
\z

\subsection{Body part nouns inflected as possessed basic nouns} \label{ch:Body part nouns inflected as possessed basic nouns}
Possessed body part nouns can also be inflected like possessed basic nouns, taking \textsc{p}-series pronominal prefixes (\sectref{ch:Body part nouns inflected as unpossessed basic nouns}). In such cases, the body part can be interpreted as detached or detachable (as in \ref{ex:possnounppex2}b, c).

\ea\label{ex:possnounppex2}
\ea gonę́:tsaˀ\\
\gll go-nę́:ts-aˀ\\
\textsc{3s.fi.p}-arm-{\nounstemformer}\\
\glt ‘her arm’ (i.e. a doll’s)
\ex honę́:tsaˀ\\
\gll ho-nę́:ts-aˀ\\
\textsc{3s.m.p}-arm-{\nounstemformer}\\
\glt ‘his arm’ (i.e. said when holding up a Ken doll’s arm)
\z
\z
