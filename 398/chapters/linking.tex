\chapter[Clauses with linking words \textup{(}\textsc{particles}\textup{)}]{Clauses with linking words (\textsc{particles})} \label{ch:Clauses with linking words}
\textsc{Independent} and \textsc{dependent clauses} that begin with linking words (\textsc{particles}) are described next. (For independent and dependent clauses, see \sectref{Independent and dependent clauses, relative clauses}.)


\section{Clauses with \textit{shęh} ‘that’, \stem{shęh ni-} ‘how, what’, \textit{ne:ˀ} ‘it is’} \label{ch:Clauses with [shęh] ‘that’, [shęh ni-] ‘how, what’, [ne:ˀ] ‘it is’}
In clauses beginning with \textit{shęh} ‘that’, \stem{shęh ni-} ‘how, what’, or \textit{ne:ˀ} ‘it is’, the linking words serve to make such clauses dependent on another clause in the same \textsc{complex sentence}: as shown in \xref{ex:shesuborex}, \textit{shęh} links a dependent clause (such as \textit{shęh sadaˀgáideˀ} ‘that you are well’) with an independent one (such as \textit{agatsęnǫ́:nih giˀ} ‘I am just happy’). \textit{Shęh} is also optional -- the dependent clause can begin without it. (Examples are provided later in this section. For complex sentences, see \pageref{Simple and complex sentences, clause types})

\ea\label{ex:shesuborex}
\gll Agatsęnǫ:ní: giˀ [\exemph{shęh} sadaˀgáideˀ]. \\
I’m.happy just that you.are.well\\
\glt ‘I am just happy that you are well.’ 
\z

Phrases with \stem{shęh … ni-} ‘how, what’ introduce a type of dependent clause that functions as a “noun” (naming a person, place, thing, or concept). For example, in \xref{ex:howwhatex4}, the phrase \textit{shęh niyo:wéˀ gakwi:yó: agadekǫ́:niˀ} ‘the amount of food that I ate’ functions as a noun.

\ea\label{ex:howwhatex4}
\gll Ęyonishéˀ hęwagi̱ˀdrǫ́:ˀ  [\exemph{shę}  \exemph{ni}yo:wéˀ  gakwi:yó:  agadekǫ́:niˀ]. \\
it.will.take.time it.will.last.me that a.certain.distance good.food I.ate\\
\glt ‘The amount of food that I ate will last me a long time.’ (\cite[294]{mithun_watewayestanih_1984}, Dwade:kǫ:nih dialogue)
\z

The particle \textit{ne:ˀ} ‘it is’ can also introduce a dependent clause that functions as a “noun”. In \xref{ex:howwhatexx2}, \textit{né:ˀ desadǫ̱hwę:jǫ́:nih} denotes ‘something wanted’ or a ‘wanted item’. 

\ea\label{ex:howwhatexx2}
\gll Agyę́ˀ  [\exemph{né:ˀ}  desadǫ̱hwę:jǫ́:nih]. \\
I.have it.is you.want.it\\
\glt ‘I have what you want.’
\z

\textit{Shęh} can also introduce a type of \textsc{relative clause}, which further specifies the meaning of some noun, \sectref{Independent and dependent clauses, relative clauses}. Example \xref{ex:whoexsen4} illustrates a phrase that functions as a noun. It begins with \textit{neˀ hęnǫ́:gweh} ‘the men’ plus a relative clause, \textit{shęh ená:greˀ} ‘that live there’. The relative clause modifies the meaning of the noun by specifying a group of men, as opposed to any other group. 
 
\ea\label{ex:whoexsen4}
\gll O:nę́h hniˀ ohsǫdagwe:gǫ́h de̱hodinęhę́:ˀ neˀ hęnǫ:gwéh  [\exemph{shęh}  ená:greˀ]. \\
now and all.night they.guard the men that someone.lives.there\\
\glt ‘Now too all night they are guarding, the men who live there.’ (\cite{keye_hnyagwaidatgigowah_2012})
\z

 As mentioned earlier, relative and dependent clauses do not always begin with \textit{shęh}. Instead pronominal prefixes alone can provide the needed link: in \xref{ex:whoexsen5}, the phrase \textit{neˀ haksotgęhę́:ˀ togyę́h da̱hayagę́ˀ} functions as a noun (denoting a person). The pronominal prefix \stem{ha-} \textsc{3s.m.a} in \textit{da\textbf{há}yagęˀ} links the relative clause \textit{togyę́h daháyagęˀ} ‘he was going out from there’ to the noun it modifies, \textit{neˀ haksótgę̱hę:ˀ} ‘my late grandfather’.

\ea\label{ex:whoexsen5} 
\gll Onę́h gwaˀ, nę́-gwaˀ ahsha:kni:gę́ˀ neˀ haksotgęhę́:ˀ  [\exemph{togyę́h}  \exemph{da̱hayagę́ˀ}],\\
Now right.now now-right.now we.saw.him the my.late.grandfather there he.went.out\\
\glt ‘And then we saw my late grandfather (who) was going out from there.’ (\cite{henry_de_2005})
\z

More examples of relative clauses with \textit{shęh} and \textit{ne:ˀ} are provided in the “Related” sections.

\begin{CayugaRelated}
\item{}\textit{Ne:ˀ} ‘it is’, \sectref{p:[ne:ˀ] `it is’}

\item{}\textit{Shęh} ‘that’, \sectref{p:[shęh] `that’}

\end{CayugaRelated}



\section{Conditional clauses with \textit{gyę:gwaˀ}, \textit{gęh}, \textit{hę:gyeh} ‘if, whether’} \label{ch:Conditional clauses with [gyę:gwaˀ], [gęh], [hę:gyeh] ‘if, whether’}
\textsc{Conditional clauses} are dependent clauses describing hypothetical situations or prerequisite conditions. The linking word appears first in the clause \xxref{ex:ifclause2}{ex:ifclause}, or more accurately, \textit{gyę́:gwaˀ} and \textit{hę́:gyęh} appear at the beginning of the clause \xxref{ex:ifclause2}{ex:ifclause3}, while \textit{gęh} occurs after the first word of its clause \xref{ex:ifclause}. (In example \ref{ex:ifclause}, \textit{gęh} means ‘whether, if’, rather than ‘Q(uestion) word’, see \sectref{p:[gyę:gwaˀ]}) More examples of conditional clauses are provided in the “Related” sections.

\ea\label{ex:ifclause2}
\gll Tęˀ dejǫ́hsdi:s o:nę́h  [\exemph{gyę:gwáˀ} a:yáihe:]\\
not she.didn’t.pay.attention now if she.would.die\\
\glt ‘She didn’t care if she was going to her death.’ (\cite{carrier_legends_2013})
\z

\ea\label{ex:ifclause3} 
\gll Tę́ˀ dˀáǫ a:wadahsgwi̱ˀtrǫ́:niˀ, [\exemph{hę:gyę́h} neˀ haesatroni̱há:k aesę́:daˀ].\\
not not.at.all it.won’t.wrinkle, even.if the you.would.have.clothes.on you.would.sleep\\
\glt ‘It won’t wrinkle, even if you sleep with your clothes on.’ (\cite[225]{mithun_watewayestanih_1984}, Agyaˀdawíˀtraˀ dialogue)
\z

\ea\label{ex:ifclause} 
\gll Daskro:wíh  [sanǫhǫkdá:niˀ \exemph{gęh}]? \\
you.tell.me you.are.sick Q\\
\glt ‘Tell me whether you are sick.’ 
\z



\begin{CayugaRelated}
\item{}\textit{Gęh} ‘whether’, ‘if’, \sectref{p:[gęh] ‘whether’, ‘if’}

\item{}\textit{Gyę:gwaˀ} ‘if’, \sectref{p:[gyę:gwaˀ]}

\item{}\textit{Gyę:gwaˀ ta:- } ‘if not’, ‘if it hadn’t been’, \sectref{p:[gyę:gwaˀ ta:- ]}

\item{}\textit{Hę:-gyęh shęh} ‘no matter how much’, ‘whether or not’, ‘even if’, \sectref{p:[hę:-gyęh shęh]}

\item{}\textit{To gęh ǫ ne:ˀ} ‘is it that?’, ‘if it is that’, \sectref{p:[to gęh ǫ ne:ˀ]}
\end{CayugaRelated}

\largerpage
\section{Causative clauses with \textit{dęˀ ni:yoht shęh} ‘why’, \textit{neˀ hǫ:niˀ} ‘how’} \label{ch:Causative clauses with [dęˀ ni:yoht shęh] ‘why’, [neˀ hǫ:niˀ] ‘how’}
\textsc{Causative clauses} are dependent clauses describing cause or effect. They can function as \textsc{direct questions} \xref{ex:caussent1}, \textsc{indirect questions} \xref{ex:caussent2}, or \textsc{dependent statements} \xref{ex:caussent3}. (For these clause types, see \sectref{Independent and dependent clauses, relative clauses}.) The linking words appear at the beginning of the clause, sometimes intermixed with other particles. 

\ea\label{ex:caussent1}
\gll  [\exemph{Dęˀ}  \exemph{hné:ˀ}  \exemph{ni:yóht}  \exemph{shęh} tóh nahsye:ˀ]?\\
why in.fact the.way.it.is that that.one you.did.it\\
\glt ‘Why in fact did you do that?’ 
\z

\ea\label{ex:caussent2} 
\gll Honǫhdǫ́ˀ diˀ gęh Gwí:deh  [\exemph{dęˀ}  \exemph{ni:yóht}  \exemph{shęh} Tina gowanawę́ˀdagaˀs?]\\
he.knows so Q Peter what the.way.it.is that Tina she.likes.the.taste.of.candy\\
\glt ‘So does Peter know why Tina likes candy?’
\z

\ea\label{ex:caussent3} 
\gll Agahsę́: [\exemph{shęh} age:jí:yoh]. \\
I’m.slow that I’m.lame\\
\glt ‘I am slow because I am lame.’ 
\z

Causation may also implied just by \textsc{juxtaposing} (placing side by side) two clauses without using linking words \xxref{ex:trehssuborex9}{ex:trehssuborex11}.

\ea\label{ex:trehssuborex9} 
\gll [Agyǫ:díh] [degakegáhneˀ niga:gú:sˀuh].\\
I’m.smiling I’m.looking.at.them children\\
\glt ‘I’m smiling (because) I’m looking at the children.’ 
\z

\ea\label{ex:trehssuborex10} 
\gll [Tiga:gwe:gǫ́h tsǫ: agahyagwę́hęgyeˀs,] [aknǫháˀ a:knǫnhéht agáˀahdraˀ]. \\
all.over.the.place just I’m.going.along.picking.fruit [I’m.unable I.would.fill.it my.basket]\\
\glt ‘I am just going along picking fruit here and there (because) I am unable to fill my basket.’
\z

\ea\label{ex:trehssuborex11} 
\gll [De̱hadawę:nyéˀ haya:sǫ́h ǫgwehǫwéhne̱ha:ˀ,] [gyotgǫ́:t gyę:ˀ nǫ́ne:ˀ ę:-tsǫ́: isheˀ de̱hadawę́:nyeˀ].\\
he.wanders he.is.named Cayuga.language, always the.one you.know there.just he.is.here he.walks.about\\
\glt ‘De̱hadawę:nyéˀ, (‘he wanders’), is his Cayuga name, (because), you know, he always wanders around here and there.’ (\cite[317]{mithun_watewayestanih_1984}, Sanahsgwaęˀ Gęh Sga̱hoˀdę:ˀęh? dialogue)
\z

More examples of linking words and phrases used with causative sentences are provided in the “Related” sections.

\begin{CayugaRelated}
\item{}\textit{Dęˀ ni:yoht shęh} ‘why?’, \sectref{p:[dęˀ ni:yoht shęh]}

\item{}\textit{Hne:ˀ shęh} ‘because’, \sectref{p:[hne:ˀ shęh]}

\item{}\textit{Ji trehs} ‘because’, ‘overly’, \sectref{p:[ji trehs] ‘because’}

\item{}\textit{Ne:ˀ dagaihǫ:niˀ} ‘the reason why’, ‘that’s why’, ‘because’, \sectref{p:[ne:ˀ dagaihǫ:niˀ]}

\item{}\textit{Ne:ˀ tsǫ:} ‘because’, ‘it’s just’, \sectref{p:[ne:ˀ tsǫ:]}

\item{}\textit{Shęh} ‘that, because’, \sectref{p:[shęh] `that’}
\end{CayugaRelated}


\section{Manner clauses with \stem{dęˀ hoˀdęˀ ni-} ‘how’, \textit{shęh ni:yoht} ‘how so’} \label{ch:Manner clauses with [dęˀ ni-] ‘how’, [shęh ni:yoht] ‘how so’}
\textsc{Manner clauses} ask for or provide information about the manner in which an action is performed. Example \xref{ex:mannerclause1} illustrates a direct question or independent clause, while example \xref{ex:mannerclause2} shows a dependent clause. The linking words occur at the beginning of the clause. More examples are provided in the “Related” sections.

\ea\label{ex:mannerclause1} 
\gll Asadadwęˀna̱hsáik? [\exemph{Dęˀ} hne:ˀ  \exemph{hoˀdę́ˀ}  \exemph{n}a̱ˀsye:ˀ?]\\
you.bit.your.tongue what in.fact kind how.you.did.it]\\
\glt ‘You bit your tongue? How did you do that?’ (\cite[386]{mithun_watewayestanih_1984}, Ga̱há:gǫ: dialogue)
\cfex{\gll n-a̱ˀ-s-ye:-ˀ\\
{\partitive}-{\factual}-{2\sga}-do-{\punctual}\\
}
\z

\ea\label{ex:mannerclause2}
\gll Knigǫhá:ˀ [\exemph{shęh}  \exemph{ni:yóht} dęgátahahk]. \\
I.watch that how.it.is I.will.walk\\
\glt ‘I watch how I walk.’
\cfex{\gll ni:-yó-ht\\
        {\partitive}-{3\sgo}-resemble.{\stative}\\
        }
\z

\begin{CayugaRelated}
\item{}\textit{Dęˀ hoˀdęˀ ni-} ‘how’, ‘what way’, \sectref{p:[dęˀ hoˀdęˀ ni-]}

\item{}\textit{Shęh ni:yoht ni-} ‘how’, ‘the manner in which’, \sectref{p:[shęh ni-, shęh ni:yoht ni-]}
\end{CayugaRelated}


\largerpage
\section{Measuring clauses with \stem{do: … ni-} ‘how much, many’} \label{ch:Measuring clauses with [do: … ni-] ‘how much, many’}
\textsc{Measuring clauses} describing measurements, degrees, or amounts. The linking word \textit{do:} ‘how’ begins the clause, and is followed by a verb with the \stem{ni-} \textsc{\partitive} prefix. Measuring clauses can function as direct questions \xref{ex:howquesex}, indirect questions \xref{ex:doexx}, and dependent statements \xref{ex:doexx2}. (For clause types, see \sectref{Independent and dependent clauses, relative clauses}.)

Other examples of \stem{do: … ni-} are provided in the “Related” sections.

\ea\label{ex:howquesex} 
\gll [\exemph{Do:}  \exemph{ni}dihsé:no:]? \\
how where.you.come.from\\
\glt ‘How old are you (singular)?’
\cfex{\gll ni-di-hs-é:no-:\\
    {\partitive}-{\cislocative}-\textsc{2s.p}-originate.from-{\stative}\\
    }
\z

\ea\label{ex:doexx}
\gll Tę́ˀ gyę́:ˀ de̱ˀsgahǫ:dǫ́ˀ [\exemph{dó:}  \exemph{ni}yagonohsriyáˀgǫh]. \\
not the.one you.didn’t.ask.me how how.old.they.are\\
\glt ‘You didn’t ask me how old they are.’ (\cite[88]{mithun_watewayestanih_1984}, Eksaˀgó:wah dialogue)
\cfex{\gll ni-yagon-ohsr-iyáˀg-ǫh\\
{\partitive}-\exsc{3ns.fi.p}-winter-cross-{\stative}\\
}
\z

\ea\label{ex:doexx2} 
\gll Ęgatgęˀsé:ˀ [\exemph{dó:}  \exemph{n}i:s  \exemph{ni}swahwají:yaˀ].\\
I.will.go.and.see how you how.your.family.is\\
\glt ‘I will see how big your family is.’ (\cite[257]{mithun_watewayestanih_1984}, Ga̱hwajiyá:deˀ dialogue)
\cfex{\gll ni-swa-hwají:y-aˀ\\
            {\partitive}-\exsc{2p.p}-family-{\nounstemformer}\\}
\z



\begin{CayugaRelated}
\item{}\textit{Do:} ‘how’, \sectref{p:[do:]}

\item{}\textit{Do: ni-…nisheˀ} ‘how long’, ‘how much time’, \sectref{p:[do: ni-…nisheˀ]}

\item{}\textit{Do: ni+…ǫ:} ‘how many people’, \sectref{p:[do: ni+…ǫ:]}
\end{CayugaRelated}


\section{Clauses with \textit{sǫ́: … (nˀaht)} ‘who’} \label{ch:Clauses with [sǫ́: … (nˀaht)] ‘who’}
Clauses with \textit{sǫ́:… (nˀaht}) ‘who’ ask for or provide information about people. \textit{Sǫ:} is at the beginning of the clause, \textit{nˀaht} is optional, and other particles can intervene. \textit{Sǫ:} clauses appear as direct questions \xref{ex:whquestex3}, indirect questions \xref{ex:whquestex4} and dependent statements \xref{ex:whoexsen}. (For clause types, see \sectref{Independent and dependent clauses, relative clauses}.)

\ea\label{ex:whquestex3} 
\gll [\exemph{Sǫ:} hne:ˀ  \exemph{nˀáht} to:gyę́h eksá:ˀah]?\\
who in.fact person that.one girl\\
\glt ‘Who is that girl?’ (\cite[71]{mithun_watewayestanih_1984}, Sǫ: Hne:ˀ Nˀaht To:gyęh? dialogue)
\z

If the actual person is unknown, the verb in a \textit{sǫ:} clause typically takes a \textsc{3s.fi} pronominal prefix meaning ‘she, someone, people’ \xxref{ex:whquestex4}{ex:whoexsen}.

\ea\label{ex:whquestex4}
\gll Daskro:wíh [\exemph{sǫ:}  \exemph{nˀáht} daǫdekǫ́nya̱hneˀ]. \\
you.tell.me who person she’ll.come.and.eat\\
\glt ‘Tell me who’s coming to eat.’ 
\cfex{\gll da-ǫ-d-e-kǫ́ny-a̱-hn-e-ˀ\\
    {\cislocativefactual}\exsc{-3s.fi.a-{\semireflexive}}-{\joinerE}-eat-{\joinerA}-{\dislocative}-go-{\punctual}\\}
\z

\ea\label{ex:whoexsen}
\gll Honǫhdǫ́ˀ Gwí:deh [\exemph{sǫ} ˀǫ  \exemph{nˀáht} gowanawę́ˀdagaˀs]. \\
he.knows Peter who I.guess person she.likes.the.taste.of.sugar\\
\glt ‘Peter knows who likes candy.’ 
\cfex{\gll go-wanawę́ˀd-a-gaˀ-s\\
    \textsc{3s.fi.p}-sugar-{\joinerA}-like.the.taste.of-{\habitual}\\
    }
\z

\textit{Sǫ́: (nˀaht)} can be omitted when the context makes the meaning clear. For example, it is omitted in \xref{ex:whoexsen3}, resulting in juxtaposed clauses (\sectref{Sentences with side-by-side clauses}). The same sentence with \textit{sǫ: nˀaht} is shown in \xref{ex:whoexsen300}.

\ea\label{ex:whoexsen3}
\gll Daskro:wíh [gado:gę́: dędwá:do:t]. \\
 you.tell.me together we.will.eat\\
\glt ‘Tell me (who) will eat together with us.’ 
\z

\ea\label{ex:whoexsen300}
\gll Daskro:wiˀ [\exemph{sǫ:}  \exemph{nˀaht} gado:gę́: dęwá:dǫ:t]. \\
 you.tell.me who person together we.will.eat\\
\glt ‘Tell me who is going to dine with us.’ 
\z

More examples of clauses with \textit{sǫ́:} ‘who’ are provided in the “Related” sections.

\begin{CayugaRelated}
\item{}\textit{Sǫ: nˀaht} ‘who’, \sectref{p:[sǫ:, sǫ: nˀaht]}

\item{}\textit{Sǫ: go:węh} ‘whose’, \sectref{p:[sǫ: go:węh] ‘whose’ (possessive)}

\end{CayugaRelated}


\largerpage
\section{Clauses with \textit{dęˀ … (hoˀdęˀ)} ‘what’} \label{ch:Clauses with [dęˀ … (hoˀdęˀ)] ‘what’}
Clauses with \textit{dę́ˀ … (hoˀdęˀ)} ‘what’ ask for or provide information about objects or ideas. \textit{Dęˀ} is at the beginning of the clause, \textit{hoˀdęˀ} is optional, and other particles can intervene between them. \textit{Dę́ˀ … (hoˀdęˀ)} appears in direct questions \xref{ex:whatquesex}, and in dependent clauses \xref{ex:whatquesex2}. (It likely also appears in indirect questions.) More examples of \textit{dęˀ} clauses are provided in the “Related” sections.

\ea\label{ex:whatquesex}
\gll  [\exemph{Dę́ˀ} ní:s  \exemph{hoˀdę́ˀ} sniya:sǫh]?\\
what you kind you.two.are.called\\
\glt ‘What are you two called?’
\z

\ea\label{ex:whatquesex2}
\gll Haoˀ dahskro:wíh [\exemph{dęˀ} ni:ˀ nęga:gye:ˀ]. \\
o.k., you.tell.me what I I.will.do.it\\
\glt ‘O.k., you tell me what to do!’ 
\z

\begin{CayugaRelated}
\item{}\textit{Dęˀ-} element meaning ‘what’, ‘how’, \sectref{p:[dęˀ-]}

\item{}\textit{Dęˀ hoˀdęˀ} ‘what’ (‘indefinite pronoun’), \sectref{p:[dęˀ, dęˀ hoˀdęˀ]}

\item{}\textit{Dęˀ hne:ˀ hoˀdęˀ} ‘what’, \sectref{p:[dęˀ hne:ˀ hoˀdęˀ]}

\item{}\textit{Dęˀ ǫh ne:ˀ hoˀdęˀ} ‘what on earth?!’, \sectref{p:[dęˀ ǫh hne:ˀ hoˀdęˀ]}
\end{CayugaRelated}


\section{Clauses with \textit{hǫ:(weh)} ‘the place where’} \label{ch:Clauses with [hǫ:(weh)] ‘the place where’}
Clauses with \textit{hǫ́:(weh)} include ones beginning with \textit{gaę … hǫ:(weh)}, \textit{shęh … hǫ:(weh)} or \textit{tohgeh … hǫ:(weh)}. Such clauses ask for or provide information about locations. \textit{Gaę}, \textit{shęh}, and \textit{tohgeh} appear at the beginning of the clause and are followed by \textit{hǫ:(weh)}. Other particles can intervene. (\textit{Hǫ:weh} also has the short forms \textit{hǫ:} or \textit{nhǫ:}.) 

Examples \xxref{ex:where1}{ex:where3} illustrate independent clauses, and \xref{ex:where2} is a dependent clause functioning as a “noun” denoting a location.

\ea\label{ex:where1}
\gll  [\exemph{Gaę:}  \exemph{nhǫ:} tsiˀdrǫˀ]?\\
Which place you.live.there\\
\glt ‘Where do you live?’ 
\z

\ea\label{ex:where2}
\gll Hętsyę́:ˀ [\exemph{shęh}  \exemph{hǫ:} hesá:gwęh]. \\
you.will.put.it.back.there that where you.picked.it.up.there\\
\glt ‘You will put it back where you got it.’
\z

\ea\label{ex:where3}
\gll [\exemph{Toh-géh} gęh  \exemph{hǫ:wéh} toh naˀá:weh?]\\
that.one-on Q place there something.happened\\
\glt ‘Where did it happen?’
\z



More examples are provided in the “Related” sections. 

\begin{CayugaRelated}
\item{}\textit{Gaę hǫ:weh} ‘which place’, ‘where’, \sectref{p:[gaę hǫ:weh]}

\item{}\textit{Shęh hǫ:weh} ‘the place where’, ‘whereabouts’, \sectref{p:[shęh hǫ:weh]}

\item{}\textit{Toh-geh hǫ:weh} ‘where’, \sectref{p:[toh-geh hǫ:weh]}
\end{CayugaRelated}


\section{Clauses with \textit{hwę:dǫh}, \textit{nęh}, \textit{nę:gyęh hwaˀ}, \textit{ne:ˀ hwaˀ}, etc. ‘when’} \label{ch:Clauses with [hwę:dǫh], [nęh], [nę:gyęh hwaˀ], [ne:ˀ hwaˀ], etc. ‘when’}
Clauses with \textit{hwę́:dǫh} ‘when’ \xref{ex:when1} and \textit{nę́h} ‘when’ \xref{ex:when2} ask for or provide information about the time of an event. \textit{Hwę́:dǫh} is only used in direct and indirect questions, and \textit{nę́h} is only used in dependent statements.

\ea\label{ex:when1} 
\gll \exemph{Hwę:dǫ́h} ne:ˀ to nęya:węh? \\
when it.is that it.will.happen\\
\glt ‘When is that going to happen?’ 
\z

\ea\label{ex:when2} 
\gll [\exemph{Nęh} toh hędwá:yǫˀ] nę:dáh nędwá:ye:ˀ. \\
when there we.will.arrive.there this.way we.will.do.something\\
\glt ‘When we arrive there, we will do it this way.’\footnote{In this example and in \xref{ex:when5}, the dependent clause appears before the independent one.}
\z

 Other phrases such as \textit{do: niyowi̱hsdáˀe:ˀ} ‘what time’ \xref{ex:when3}, \textit{nę:gyę́h hwaˀ} ‘this time’ \xref{ex:when4}, and \textit{né:ˀ hwaˀ} ‘this time’ \xref{ex:when5} also ask for or provide information about the time of an event, and also appear at the beginning of direct questions, indirect questions, and dependent clauses.
 
\ea\label{ex:when3}
\gll Daskro:wíˀ [\exemph{do:}  \exemph{niyowi̱hsdaˀe:ˀ} ęsahdę́:diˀ]. \\
you.tell.me how it.is.a.certain.time you.will.leave\\
\glt ‘You tell me when you are going to leave.’ 
\z

\ea\label{ex:when4}
\gll Dęˀ diˀ hoˀdę́ˀ nę̱hsa:gyé:ˀ [\exemph{nę:gyę́h}  \exemph{hwaˀ} dęjogęnhǫ́:diˀ]?\\
What so what.kind you.will.do.something this this.time it.will.be.summer\\
\glt ‘What will you do when summer comes?’ (\cite[349]{mithun_watewayestanih_1984}, Dęˀ Hoˀdęˀ Nǫ̱hsa:gye:ˀ? dialogue)
\z

\ea\label{ex:when5}
\gll [\exemph{Ne:ˀ}  \exemph{hwaˀ} nę:gyę́h sá:dǫh: ‘dęyǫkidé:niˀ’] neˀ gę́:dǫh neˀ tęˀ-da̱ˀǫ́: ǫgwehǫ:wéh ǫ:sawá:dǫˀ.\\
it.is this.time this you.say they.will.change.us the it.means the definitely.not real.people it.will.no.longer.become\\
\glt ‘When you say ‘they will change us,’ it means that you’ll no longer be Indian.’ (\cite{henry_de_2005})
\z

For more examples, see the “Related” sections.

\begin{CayugaRelated}
\item{}\textit{Do: niyowi̱hsda̱ˀe:ˀ} ‘what time is it?’, ‘when?’, \sectref{p:[do: niyowihsdaˀe:ˀ]}

\item{}\textit{Hwę:dǫh} ‘ever’, ‘when’, \sectref{p:[hwę:dǫh]}

\item{}\textit{Ne:ˀ hwaˀ} ‘this (coming) time’, ‘when’, \sectref{p:[ne:ˀ hwaˀ]}

\item{}\textit{Nęh, ne:ˀ nęh} ‘when’, ‘once’, ‘as soon as’, \sectref{p:[nęh, ne:ˀ nęh]}

\item{}\textit{Nę:-gyęh hwaˀ} ‘this time’, ‘when’, \sectref{p:[nę:-gyęh hwaˀ] ‘this time’}
\end{CayugaRelated}


\section{Clauses with \textit{(gaoˀ) shęh niyo:weˀ} ‘before, until’} \label{ch:Clauses with [(gaoˀ) shęh niyo:weˀ] ‘before, until’}
Dependent clauses beginning with \textit{(gaoˀ) shęh niyó:(weˀ)} ‘before, until’ describe events that happened up until the time of the event described in the independent clause \xref{ex:before20}. (\textit{Niyó:weˀ} also has the short forms \textit{ní:yo:ˀ} and \textit{nyó:}.)

\textit{Shęh niyó:weˀ} occurs at the beginning of dependent clauses, but is also omitted when the context makes the meaning clear \xref{ex:beforuntilex2}. More examples with \textit{shęh niyó:weˀ} ‘before, until’ are provided in \sectref{p:[gaoˀ shęh niyo:weˀ]}.

\ea\label{ex:before20} 
\gll O:nę́h aˀa:gę́ˀ, “Waˀgyę́h,” aˀa:gę́ˀ, “ęgehsrǫ:níˀ i:wí: tó: neˀ sageˀaˀgéh [\exemph{gaoˀ}  \exemph{shę́h}  \exemph{nyó:ˀ} ęhsę́da̱ˀdraˀ.]” \\
now she.said just.now she.said I.will.fix.it I.want that the on.your.hair which that it.is.a.certain.distance you.will.go.to.bed \\
\glt ‘“Just a minute,” she said, “I want to fix your hair right there before you go to bed.”’ (\cite{henry_de_2005})
\z

\ea\label{ex:beforuntilex2}
\gll [Ęhseganyáˀk] [ęhskéya̱htgaˀ]. \\
you.will.pay I.will.release.her\\
\glt ‘You will pay (before) I let her go.’ 
\z




\section{Clauses with \textit{shęh naˀonisheˀ}, \textit{tsaˀonisheˀ}, \stem{tsi-} ‘while, when’} \label{ch:Clauses with [shęh naˀonisheˀ], [tsaˀonisheˀ], [tsi-] ‘while, when’}
Clauses with \textit{shęh naˀónisheˀ} ‘while’, ‘when’ \xref{ex:whilewhenex}, \textit{tsaˀónisheˀ} ‘while’, ‘during’, ‘when’ \xref{ex:whilewhenex2}, and clauses beginning with \stem{tsi-} \textsc{\coincident} verbs \xref{ex:whilewhenex3}, describe events that occur at the same time as the event in the independent clause. The linking words appear at the beginning of the dependent clauses. More examples are provided in \sectref{p:[shęh naˀonisheˀ]}.

\ea\label{ex:whilewhenex} 
\gll Agahyagóˀ [\exemph{shęh}  \exemph{na̱ˀonishéˀ} odahyǫ́:ni:]\\
I.picked.fruit that it.is.a.certain.time fruit.is.making\\
\glt ‘I did pick fruit while it was plentiful.’
\z

\ea\label{ex:whilewhenex2} 
\gll Hohsę́: [\exemph{tsaˀonihseˀ} hǫgwéˀdase:]. \\
he.was.fat while he.is.a.young.man\\
\glt ‘While he was young, he was fat.’ 
\z

\ea\label{ex:whilewhenex3} 
\gll Deˀagowihsrá:t [\exemph{ts}aˀǫ́nagra:t]. \\
she.had.no.breath when.she.was.born\\
\glt ‘She had no breath when she was born.’
\cfex{\gll ts-aˀ-ǫ́-nagra:t\\
        {\coincident}-{\factual}-\textsc{3s.fi.a}-caused.to.dwell.{\zeropunctual}\\
        }
\z
