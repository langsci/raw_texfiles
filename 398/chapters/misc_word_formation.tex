\chapter{Miscellaneous word formation topics} \label{ch:Miscellaneous word formation topics}
The following sections summarize where to find out about the various constructions that denote possession, plurality, and location.


\section{Possession} \label{ch:Possession}
The concept of possession or ownership is described in the sections listed below.

\begin{CayugaRelated}
\item “Possessive pronouns”, \sectref{ch:’Possessive pronouns’}

\item Possessed nouns, \sectref{ch:Possessed nouns}

\item Possessed basic nouns (\textsc{p}-series), \sectref{ch:Possessed basic nouns (o series)}

\item Possessed body part nouns (\textsc{a}-series), \sectref{ch:Possessed body part nouns (a series)}

\item Interactive kinship terms, \textsc{interactive} pronominal prefixes, \sectref{ch:Interactive kinship terms, interactive pronominal prefixes}

\item Pronominal prefix choice for stative-only verbs, \sectref{Pronominal prefix choice for stative-only verbs}

\item Neuter stative-only verbs, incorporating, conveying possession, \sectref{Neuter stative-only verbs, incorporating, conveying possession}
\end{CayugaRelated}



\section{Pluralizing} \label{ch:Pluralizing}
There are many ways to \emph{pluralize} or convey the idea that more than one object is involved. They are described in the sections listed below.

\begin{CayugaRelated}
\item{}Pluralizers, \sectref{Pluralizers}

\item{}\stem{-sǫ:ˀǫh} (pluralizer), \sectref{-sǫ:ˀǫh}

\item{}\stem{-sǫˀ} (pluralizer), \sectref{-sǫˀ}

\item{}Meaning of \stem{-sǫˀ} versus \stem{-sǫ:ˀǫh}, \sectref{Meaning of -sǫˀ versus -sǫ:ˀǫh}

\item{}Pluralizing nouns, \sectref{Pluralizing nouns}

\item{}The \stem{s-, j-, ji-} (repetitive) prefix, \sectref{[s-, j-, ji-] (repetitive)}

\item{}\stem{haˀde-pronominal prefix-\textsc{incorporated noun}-age:} ‘every, many, a variety of \textsc{incorporated nouns}’, \sectref{[haˀde-pronominal prefix-incorporated noun-age:]}

\item{}\stem{-ǫ}, \stem{-nyǫ}, \stem{-ǫ-nyǫ}, \stem{-hnǫ}, \stem{-hsǫ}, \stem{-drǫ}, \stem{-srǫ} (distributives), \sectref{[-ǫ], [-nyǫ], distributives}

\item{}\stem{-ˀs} (stative verb pluralizer), \sectref{[-ˀs] (stative verb pluralizer)}

\item{}Comparisons, counting, measuring, \sectref{ch:Comparisons, counting, measuring}

\item{}Numbers and money, \sectref{numbers and money}
\end{CayugaRelated}


\section{Location} \label{ch:Location}
The concept of location is mainly conveyed through the suffixes described in the sections listed below.

\begin{CayugaRelated}
\item{}\stem{-ˀgeh} (external locative), \sectref{-ˀgeh}

\item{}\stem{-hneh} (external locative), \sectref{-hneh}

\item{}\stem{-ˀgeh} versus \stem{-hneh}, \sectref{-ˀgeh versus -hneh}

\item{}\stem{-gǫ:} (internal locative), \sectref{[-gǫ:]}

\item{}\stem{-:kˀah} (‘beside’ locative), \sectref{[-:kˀah]}

\item{}\stem{-kdagyeˀ} (‘alongside’ locative), \sectref{[-kdagyeˀ]}
\end{CayugaRelated}

The prefixes listed below also refer to types of locations, for example \stem{ni-} {\partitive} ‘somewhere’, \stem{d-} {\cislocative} ‘here’ and \stem{heˀ-} {\translocative} ‘there’.


\begin{CayugaRelated}
\item{}The \stem{ni-} (partitive) prefix, \sectref{[ni-] (partitive)}

\item{}\stem{d-} (cislocative) and \stem{heˀ-} (translocative) prefixes, \sectref{[d-] (cislocative) and [heˀ-] (translocative)}
\end{CayugaRelated}

Several verbs also inherently imply a direction (‘down’, ‘up’, ‘from’). Some examples are provided in \xref{ex:preposex3}.

\ea\label{ex:preposex3}
\ea sa̱hsnęht\\
\gll s-a̱hsnęht\\
 \textsc{2s.a}-descend.{\noaspect}\\
\glt `get down from there'
\ex satgęh\\
\gll s-at-gęh\\
 \textsc{2s.a}-{\semireflexive}-awaken.{\noaspect}\\
\glt `get up'
\ex dwagáhdęgyǫ:\\
\gll d-wag-áhdęgy-ǫ:\\
 {\cislocative}-\textsc{1s.p}-leave-{\stative}\\
\glt `(where) I come from'
\ex dwada:dǫ́:nih\\
\gll d-w-ada:d-ǫ́:ni-h\\
 {\cislocative}-\textsc{3s.a}-{\reflexive}-make-{\habitual}\\
\glt `where it starts from'
\ex dęyehgwa̱ˀdáhnǫ:ˀ\\
\gll d-ę-ye-hgw-a̱-ˀd-á-hnǫ-:ˀ\\
 \textsc{\dualic-\future-3s.fi.a}-lift.up-\textsc{\JoinerA-\causative-\JoinerA-\distributive-\punctual}\\
\glt `she will raise or lift things up'
\z
\z

And finally, the following sections also convey concepts related to location.

\begin{CayugaRelated}
\item “Demonstrative pronouns”, \sectref{ch:’Demonstrative pronouns’}

\item “Adverbs” of place, \sectref{‘Adverbs’ of place}

\item Neuter stative-only positional verbs, with incorporated noun, \sectref{Neuter stative-only positional verbs, with incorporated noun}
\end{CayugaRelated}
