

\begin{table}
\caption{Pronominal prefix selection, nouns}
\label{tab:1:pronselnouns}
\scriptsize{
\begin{tabularx}{\textwidth}{XXX}

\lsptoprule
grammatical category & prefix type & prefix choice\\
\midrule
{}•unpossessed basic nouns

{}•unpossessed body part nouns that are inflected like basic nouns

& [ga-] \textit{it.a}

[o-] \textit{it.o}

∅ (no prefix)

& The type of prefix ([ga-], [o-] or none) must be memorized for each for each word.\\


{}•possessed basic nouns 

{}•possessed body part nouns that are inflected like basic nouns

& 

\textsc{p}-series & \textsc{p}-series pronominal prefixes can denote possession or ownership.\\
{}•possessed body part nouns & \textsc{a}-series & \textsc{a}-series pronominal prefixes denote the person or being who has the body part in question. \\
\lspbottomrule
\end{tabularx}}
\end{table}


\begin{table}
\caption{Pronominal prefix selection, kinship terms}
\label{tab:1:pronselkinship}
\tiny{
\begin{tabularx}{\textwidth}{XXX}

\lsptoprule
grammatical category & prefix type & prefix choice\\
\midrule
verbs or nouns functioning as kinship terms & \textsc{p}{}-series

\textsc{a}{}-series & •\textsc{p}-prefix or \textsc{a}{}-prefix refers to point-of-view (or ‘possessor’)

-Stem specifies the type of relative\\
same-generation & \textsc{interactive} & •1\textsuperscript{st} role refers to point-of-view (or ‘possessor’).


•2\textsuperscript{nd} role specifies the number and gender of the kin or relative

•Stem specifies a same-generation type of relative\\
different-generation & \textsc{interactive} & •1\textsuperscript{st} role refers to older generation, regardless of point-of-view

•2\textsuperscript{nd} role refers to younger generation, regardless of point-of-view

Stem specifies a different-generation type of relative or (reversed roles); like same-generation kinship terms

•1\textsuperscript{st} role refers to point-of-view (or ‘possessor’)

•2\textsuperscript{nd} role specifies the number and gender of the kin or relative

•Stem specifies a different-generation type of relative\\
different-generation 

(in-laws) & \textsc{interactive} & •1\textsuperscript{st} role refers to older generation, regardless of point-of-view

•2\textsuperscript{nd} role refers to younger generation, regardless of point-of-view

•Stem specifies a different-generation type of relative or (reversed roles) ), like same-generation kinship terms

•1\textsuperscript{st} role refers to point-of-view (or ‘possessor’)

•2\textsuperscript{nd} role specifies the number and gender of the kin or relative

•Stem specifies a different-generation type of relative

•In either case, if 1\textsuperscript{st} role refers to ‘she’, then the overall word refers to a male’s mother-in-law\\
different-generation & \textsc{p}-series & •\textsc{p}-prefix refers to younger person, and the overall word must refer to an older-generation female (or mixed group of people)\\
different-generation & /k-/ \textit{I.a}, /ha-/ \textit{he.a} & •/k-/ means ‘my older female relative’; /ha-/ means ‘my older male relative’. These words can also be used as terms of address\\
\lspbottomrule
\end{tabularx}}
\end{table}


\lipsum[1-1]

\begin{table}
\caption{Pronominal prefix selection, single-aspect verbs}
\label{tab:1:pponeaspv}
\scriptsize{
\begin{tabularx}{\textwidth}{XXXXXX}

\lsptoprule
stative-only & possession & subtypes & without inc & with inc & inc noun type\\
\midrule
personal & NO &  & \textit{A}{}-, \textsc{p}- &  & \\
\midrule
neuter & NO & V, no inc; positional V, no inc; NV (fixed)
& ga-, o- &  & \\
\midrule
neuter & NO & (N)-V; +V & o-V

+V & ga-N-V & ga-N\\

&  &  &  & o-N-V & o-N\\
&  &  & ga-V & ga-N-V & ga-/o-N\\
\midrule
neuter & NO & (N)-V, 

positional V & o-V

many ga-V & ga-N-V & ga-N\\
&  &  &  & o-N-V & o-N\\
&  &  & some ga-V & ga-N-V & ga-/o-N\\
\midrule
neuter & NO & N+V, counting V; [s-…+t] ‘one object’; […+age:] ‘two or more objects’ &  & ga-N-V & ga-N\\
&  &  &  & o-N-V & o-N\\
&  &  &  & ga-/o-N-V & o-N\\
\midrule
neuter & YES & (N)-V, 

possession & \textsc{p}-V

\textit{A}{}-V & \textsc{p}-N-V & BN, 

detachable BPN\\
&  &  &  & \textit{A}{}-N-V & non-detachable BPN\\
\midrule
neuter & YES &  [N+o:t] ‘have something attached’; [N+ęˀ] ‘have, own something’ &  & \textsc{p}-N+o:t/+ęˀ & BN, 

detachable BPN\\
&  &  &  & \textsc{a/p}{}-N+o:t/+ęˀ & non-detachable BPN\\
\lspbottomrule
\end{tabularx}}
\end{table}


\begin{table}
\caption{Pronominal prefix selection, three-aspect verbs}
\label{tab:1:ppthreeaspv}
\scriptsize{
\begin{tabularx}{\textwidth}{XXXX}
\lsptoprule
three-aspect V &  & prefix choice & roles\\
\midrule
one-role & type 1 & \textit{A}{}-V \textit{hab} and \textit{punc}

\textsc{p}-V \textit{stat} & Often expresses a 'do-er' or 'agent' role.\\
& type 2 & \textsc{p}-V \textit{hab}, \textit{punc}, and \textit{stat} & Often expresses an 'experiencer' or 'undergoer' role.\\
\midrule
two-role & type 1 & \textsc{interactive}-V \textit{hab}, \textit{punc}, and \textit{stat}

\textit{A:(it-something)}{}-V \textit{hab} and \textit{punc}

\textit{O:(it-something)}{}-V \textit{stative} OR

\textit{(it-animal):O}{}-V \textit{stat} & \textsc{interactive} prefixes express 'do-er' or 'agent' and 'experiencer / undergoer' roles.

\textsc{p}- prefixes in the stat are ambiguous: the implied ‘it-animal’ either be the 'do-er'/'agent' or the ‘experiencer / undergoer’ \\
\midrule
& type 2 & \textit{O:(it-something)}{}-V \textit{stat}, \textit{punc}, and \textit{stat} & \textsc{p}- prefixes only; prefix often expresses an experiencer / undergoer role. The second role is always an implied 'it' – often something being 'perceived'\\
\midrule
three-role &  & Like Type 1 two-role verbs & Like type 1 two-role verbs.\\
\lspbottomrule
\end{tabularx}}
\end{table}


\begin{table}
\caption{Pronominal prefix selection, e-verbs}
\label{tab:1:ppseleverb}
\scriptsize{
\begin{tabularx}{\textwidth}{XX}
\lsptoprule
e-verb type & prefix choice\\
\midrule
simple’ motion verb [e] ‘to go’

& \textit{A}{}-V only; (no \textit{stative} form)\\
complex motion verbs [V-e] & \textit{A}{}-V

\textsc{p}-V

\\
dislocative e-verbs [V-dislocative-e] & \textit{A}{}-V

\textsc{p}-V

\\
progressive verbs [V-gy-eˀ] & \textit{A}{}-V

\textsc{p}-V

\\
stative progressive verbs 

{} [V-ǫh-ǫ{}-gy-eˀ] & O-V only (like the \textit{stative} verbs they are based on)\\
\lspbottomrule
\end{tabularx}}
\end{table}

\begin{verbatim}%%move bib entries to  localbibliography.bib
\end{verbatim} 
