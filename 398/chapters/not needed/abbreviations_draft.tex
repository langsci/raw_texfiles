\begin{table}
\begin{tabularx}{\textwidth}{XX}
{}ø.punctual & punctual suffix has been deleted\\
{}3s	&	3s.zon (3rd singular zoic-neuter) \\
{}3p	&	3ns.zon	(3rd non-singular zoic-neuter) \\
a	&	agent-like argument of canonical transitive verb, \textsc{a}-series prefix, as in \textsc{3ns.m.a} ‘they (males)’  \\
alongside           &      \stem{-kdagyeˀ} “alongside” locative         \\
at           &      \stem{-hneh} external locative               \\
aug             &      augmentative (either \stem{-ge:} or \stem{-go:wah}       \\
ben             &      benefactive                           \\
beside          &      \stem{-:kˀah} “beside” locative              \\
caus            &      causative                             \\
caus-instr      &      causative-instrumental                \\
cis             &      cislocative                           \\
coin            &      coincident                            \\
completely &	\stem{-jihwęh} ‘completely, fully’ \\
contr            &      contrastive                           \\
customary             &      customary (characterizer)             \\
d	&	dual (number) \\
(d)	&	one argument must be dual (number) \\
declare.kinship		&	\stem{-shęˀ, -tsęˀ} kinship declaration suffix \\
dim             &      diminutive                            \\
disl            &      dislocative                           \\
distr           &      distributive                          \\
du            &      dualic (prepronominal prefix)         \\
ex	&	exclusive (pronominal prefix distinction) \\
fac             &      factual                               \\
facil           &      habitual facilitative                 \\
fi	&	feminine-indefinite \\
former             &      former                                \\
fut             &      future                                \\
hab             &      habitual                              \\
in        &      \stem{-gǫ:} internal locative                \\
in	&	inclusive (pronominal prefix distinction) \\
inch            &      inchoative                            \\
increment            &      noun increment                        \\
indef             &      indefinite  (optative)                          \\
instr           &      instrumental                          \\
interact	\&	textsc{interactive}-series pronominal prefix, as in \textsc{2s>1s}   ‘you>me’\\
ints            &      intensifier                           \\
length	&	suffix consisting of \stem{-:}, agnostically named \\
modz            &      modalizer                             \\
neg             &      negative                              \\
nmlz             &      nominalizer                           \\
noun	&	(in glosses) placeholder incorporated noun \\
no_aspect	&	verb lacking an aspect suffix \\
ns	&	non-singular (number) \\
(+ns)	&	both arguments must be non-singular (number) \\	
nsf             &      noun stem former                      \\
on           &      \stem{-ˀgeh} external locative               \\
opp	&	the opposite transitive arguments are included in the meaning. E.g. 2p>3s.m(opp) also means 3s.m>2p   \\
p	&	patient-like argument of canonical transitive verb, \textsc{p}-series prefix, as in \textsc{1s.p} ‘me’ \\
p   &   plural (number), as in \textsc{1p.a} ‘we’ \\
(p)	&	one argument must be plural (number) \\
part            &      partitive                             \\
past            &      past                                  \\
pl            &      plural (stative verb suffix)   \\
plrz            &      pluralizer (word suffix)   \\
pop             &      populative                            \\
purp	&	purposive \\
prog            &      progressive                           \\
proth           &      prothetic                             \\
punc            &      punctual                              \\
rec	&	reciprocal \\
refl            &      reflexive                             \\
rem             &      remote                                \\
rep             &      repetitive                            \\
rev             &      reversive                             \\
s	&	singular \\
srf             &      semireflexive                         \\
stat            &      stative                               \\
transl          &      translocative                         \\
typ & typicalizer \\
zon	&	zoic-neuter \\
\end{tabularx}
\end{table}