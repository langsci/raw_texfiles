
%<\$img:prepplegend;ebook=50\%>

\begin{table}
\caption{Legend, prepronominal prefix pronunciation}
\label{tab:1:ppplegend}
\begin{tabularx}{\textwidth}{lQ}
\lsptoprule
Abbreviation & Legend\\
\midrule
C & the following verb begins with a consonant\\
V & the following verb begins with a vowel\\
w & the following verb begins with a W\\
y &  the following verb begins with a Y \\
(w) & the following verb begins with a W, but the W deletes. (It is only found in other forms of the verb.  \\
(y) &  the following verb begins with a Y, but the Y deletes. (It is only found in other forms of the verb.)\\
(sy) & the [s-] \textit{rep} merges with the Y of the following pronominal prefix; the two sounds become J\\
(st) & the [s-] \textit{rep} becomes T when the following sound is S: [s-s] becomes [t-s]\\
(sa) & the [s-] \textit{rep} becomes [sa-] when the following sound is S: [s-s] becomes [sa-s]\\
I.o & the [a-] \textit{fact} or [a:-] \textit{indef} merge with [-wag] \textit{I.o}. The result is [ǫg-] \textit{fact.I.o} or [aǫg-] \textit{indef.I.o}\\
2 & the prefix is used before \textit{you.a} or \textit{inclusive} pronominal prefixes\\
imp & the prefix is used with imperatives (command)\\
a-person & the prefix used before \textit{you.a} ‘a-person’ pronominal prefixes\\
e-person & the prefix is used before \textit{you.a}, ‘e-person’ pronominal prefixes\\
let’s & the prefix used with \textit{let’s} commands (as in \textit{let’s do it})\\
\lspbottomrule
\end{tabularx}
\end{table}



%<\$img:AEpersons;ebook=50\%>
\begin{table}
\caption{A- and E-persons}
\label{tab:1:aepersons}
\scriptsize{
\begin{tabularx}{\textwidth}{XXX}

\lsptoprule
a-person & aˀa:kní:gęˀ, a:kní:gęˀ & ‘s/he.and.I.a:(it)’\\
& aˀ\textbf{ágwa}gęˀ, \textbf{ágwa}gęˀ & ‘s/he/they.and.I/we.a:(it)’\\
& a\textbf{kní:}gęˀ & ‘we.two(not.you):

you(one.or.two’)

‘I:you(two)’\\
& a\textbf{gwá:}gęˀ & ‘we.all(not.you):

you(one.two.all)’

‘we.two(not.you):you(all)’

‘I:you(all)’\\
& a\textbf{sgé:}gę & ‘you(one):me’ \\
\midrule
e-person & e\textbf{tní:}gęˀ, e\textbf{kní:}gęˀ & ‘you(one).and.I.a:(it)’\\
& \textbf{edwá:}gęˀ & ‘you(one/two/all).and.I/

we.a:(it)’\\
& a\textbf{hsé:}gęˀ, \textbf{es}áhdǫ:ˀ ‘you lost it’ & ‘you(one).a:(it)’\\
& \textbf{ehsní:}gęˀ & ‘you(two).a:(it)’\\
& \textbf{esa}ˀdra̱hehs ‘you are exaggerating’ & ‘it:you(one).o’\\
& \textbf{ehswá:}gęˀ & ‘you(all):(it)’\\
& e\textbf{hyá:}gęˀ, a\textbf{hyá:}gęˀ & ‘he:you(one)’\\
& \textbf{eskní:}gęˀ & ‘you(two):me/us(two)’

‘you(one):us(two)’\\
& \textbf{esgwá:}gęˀ & ‘you(all):me/us’

‘you(one,two):us(all)’\\
& \textbf{ehswá:}gęˀ & ‘you(all):(it)’\\
& e\textbf{hyá:}gęˀ, a\textbf{hyá:}gęˀ & ‘he:you(one)’\\
& \textbf{eskní:}gęˀ & ‘you(two):me/us(two)’

‘you(one):us(two)’\\
& \textbf{esgwá:}gęˀ & ‘you(all):me/us’

‘you(one,two):us(all)’\\
\lspbottomrule
\end{tabularx}}
\end{table}

\lipsum[1-1]

%<\$img:last-prefix-invariant;ebook=50\%>

\begin{table}
\caption{Invariant prefixes}
\label{tab:1:invariant}
\scriptsize{
\begin{tabularx}{\textwidth}{XX}
\lsptoprule
prefix & prefix name \\
\midrule
tsi- & {\itshape coin}\\
ti- & {\itshape contr}\\
de- & {\itshape dual}\\
ę- & {\itshape fut}\\
ni- & {\itshape part}\\
i- & {\itshape proth}\\
he- & {\itshape trans}\\
\lspbottomrule
\end{tabularx}}
\end{table}




%<\$img:last-single-prefix-pronunciation;ebook=50\%>

\begin{table}
\caption{Final prefix of combination}
\label{tab:1:finalsingleprefix}
\scriptsize{
\begin{tabularx}{\textwidth}{XXX}
\lsptoprule
last prefix & becomes & before\\
\midrule
{\itshape cisl} & t 

d 

g or d

di or de & C

w, V

y

2

\\
{\itshape fact} & a

e

ǫ

aˀ & C

e-person 

I.o 

(w), (y), V

\\
{\itshape indef} & a:

aǫ

ae & C

I.o 

2\\
{\itshape neg} & de

deˀ & C

(w), (y), V

\\
{\itshape rep} & s

t

j

ji & C

(st)

(sy)

2\\
\lspbottomrule
\end{tabularx}}
\end{table}



%<\$img:last-double-prefix-pronunciation;ebook=50\%>

\begin{table}
\caption{Final two prefixes of combination}
\label{tab:1:finaldoubleprefix}
\scriptsize{
\begin{tabularx}{\textwidth}{XXX}
\lsptoprule
prefix combo & ends with & before\\
\midrule
{\itshape fact-cisl} & ǫda

ǫdǫ

ǫdi & C

I.o 

2 

\\
{\itshape fact-dual} & adi

at

edi

ad

ag or ad & a-person 

C

e-person 

w, V

y

\\
{\itshape fact-rep} & ǫsa

ǫsǫ

ǫse & C

I.o 

2\\
{\itshape indef-cisl} & aǫda, ǫ:da

aǫdǫ, ǫ:dǫ

aǫde, ǫ:de & C

I.o 

2

\\
{\itshape indef-rep} & aǫsa, ǫ:sa

aǫsǫ, ǫ:sǫ

aǫse, ǫ:se & C

I.o 

2

\\
{\itshape rep-fact} & ǫsa

ǫsǫ

ǫsaˀ

ǫse & C

I.o 

(w), (y), V 

2\\
\lspbottomrule
\end{tabularx}}
\end{table}


%<\$img:prepronominal-prefix-pronunciation-A1;ebook=50\%>

\begin{table}
\caption{A prepronominals with a,aˀ,ǫ,e (fact)}
\label{tab:1:2finaldoubleprefix}
\scriptsize{
\begin{tabularx}{\textwidth}{XXXX}
\lsptoprule
C & (w), (y), V & I.o & 2  \\
\midrule
{}[a-], \textbf{a}gaedá:graˀ 

‘they fell down’ & [aˀ-], \textbf{aˀ}é:yǫˀ 

‘she arrived’ & [ǫ-], \textbf{ǫ}gáhdrǫˀk 

‘it frightened me’, 

‘I got frightened’ & [e-] e{}-person \textbf{e}sáhdǫ:ˀ 

‘you lost it’

aˀ-∅* 

a-person 

\textbf{aˀ}a:kní:gęˀ, a:kní:gęˀ 

‘s/he and I saw it’\\
\lspbottomrule
\end{tabularx}}
\end{table}
* The \textit{fact} optionally deletes before 2(a-persons), but the word is stressed as if the prefix were still there.

\lipsum[1-1]


%<\$img:prepronominal-prefix-pronunciation-A2;ebook=50\%>

\begin{table}
\caption{A prepronominals with a:,aǫ,ae (indef)}
\label{tab:1:finalindefprefix}
\scriptsize{
\begin{tabularx}{\textwidth}{XXX}
\lsptoprule
C & I.o & 2  \\
\midrule
{}[a:-], \textbf{a:}ga̱hyá:goˀ

‘I would pick fruit’ & [aǫ-], \textbf{aǫ}gé:gęˀ 

‘she should / would see me’ & {}[ae-] \textbf{ae}swá:gęˀ 

‘you all should / would see it’\\
\lspbottomrule
\end{tabularx}}
\end{table}

%<\$img:prepronominal-prefix-pronunciation-A3;ebook=50\%>

\begin{table}
\caption{A prepronominals ending with dǫ,de,sǫ,se}
\label{tab:1:finalindefcislrepprefix}
\scriptsize{
\begin{tabularx}{\textwidth}{XXXX}
\lsptoprule
C & I.o & 2 & \\
\midrule
{}[aǫda-, 

{}ǫ:da-, 

{}aǫda:-, 

{}ǫ:da:-], \textbf{aǫda}sagyǫ̱ˀséhaˀ ‘you would come and visit & [aǫdǫ-],

{}[ǫ:dǫ-] & [aǫde-], [ǫ:de-] & {\itshape indef-cisl}\\
{}[aǫsa-], 

{}[ǫ:sa-] & [aǫsǫ-], 

{}[ǫ:sǫ-] & [aǫse-], 

{}[ǫ:se-] & {\itshape indef-rep}\\
\lspbottomrule
\end{tabularx}}
\end{table}

\lipsum[1-1]


%<\$img:prepronominal-prefix-pronunciation-A4;ebook=50\%>

\begin{table}
\caption{A prepronominals ending with t,d,g,di (fact-dual)}
\label{tab:1:2finalindefcislrep}
\scriptsize{
\begin{tabularx}{\textwidth}{XXXX}
\lsptoprule
C & w, V & y & 2 \\
\midrule
{} [at-], \textbf{at}gaeyé:naˀ ‘they (fe/males) did it together’ & [ad-], \textbf{ad}waga̱hsíha:s ‘I choked’ & [ag-],

\textbf{ag}yagwatnę:tsí:yaˀk 

‘we all (excl.) broke our arm’ & [adi-] a-person 

\textbf{adi}gyatnę:tsí:yaˀk 

‘we two (excl.) broke our arm’ 

{} [edi-] 

e-person \textbf{edi}dwatnę:tsí:yaˀk 

‘we all (incl.) broke our arm’ \\
\lspbottomrule
\end{tabularx}}
\end{table}



%<\$img:prepronominal-prefix-pronunciation-D1;ebook=50\%>

\begin{table}
\caption{D prepronominals ending with e,ę}
\label{tab:1:dualanddualfut}
\scriptsize{
\begin{tabularx}{\textwidth}{XX}
\lsptoprule
C & \\
\midrule
{}[de-], \textbf{de}dwadagyénawahs 

‘let us all help one another’ & {\itshape dual}\\
{}[dę-], \textbf{dę}hsnáˀnetˀa:ˀ 

‘you will double it, reinforce something’ & {\itshape dual-fut}\\
\lspbottomrule
\end{tabularx}}
\end{table}


%<\$img:prepronominal-prefix-pronunciation-D2;ebook=50\%>

\begin{table}
\caption{D prepronominals ending with t,d,g,di}
\label{tab:1:dualnegfutcisl}
\scriptsize{
\begin{tabularx}{\textwidth}{XXXXX}
\lsptoprule
C & w, V & y & 2 & \\
\midrule
{}[dęt-], \textbf{dęt}ga:dǫ:goht 

‘I will pass that way’ & [dęd-] & [dęg-], \textbf{dęg}yá:kne:ˀ 

‘we two (incl.) 

will come 

back’ & [dędi-] \textbf{dędí}hsa̱ˀdre:ˀ

‘you will drive over there’ & {\itshape dual-fut-cisl}\\

{}[det-], \textbf{det}geˀ 

‘I am coming back’ & [ded-] & [deg-], 

\textbf{Deg}yotnǫ̱-hsá:kdǫ:

‘St. Catharine’s, Ont. & [dede-] imp 

(e-person) 

[dedi-] 

[dǫda-], 

imp (a-person) & {\itshape dual-cisl}\\

{}[deˀt-] \textbf{det}gá:yęˀ 

‘it is not laying there’ & [deˀd-] & [deˀg-] & [dedi-] \textbf{dedí}hsawihs 

‘you’re not wanted’ & {\itshape neg-cisl}\\
\lspbottomrule
\end{tabularx}}
\end{table}


%<\$img:prepronominal-prefix-pronunciation-D3;ebook=50\%>

\begin{table}
\caption{D prepronominals ending with a,aˀ,ǫ,e (cisl-fact)}
\label{tab:1:cislfac}
\scriptsize{
\begin{tabularx}{\textwidth}{XXXXX}
\lsptoprule
C & (w), (y), V & I.o & 2 \\
\midrule
{}[da-], \textbf{da}ga:dí:yǫˀ ‘they (animals) came in’ & [daˀ-] & [dǫ-], 

\textbf{dǫ}gáhdǫ:ˀ 

‘(where) I lost it’ & [de-] e-person\\
\lspbottomrule
\end{tabularx}}
\end{table}

\lipsum[1-1]


%<\$img:prepronominal-prefix-pronunciation-D4;ebook=50\%>

\begin{table}
\caption{D prepronominals ending with e,eˀ (neg)}
\label{tab:1:neg}
\scriptsize{
\begin{tabularx}{\textwidth}{XXX}
\lsptoprule
C & (w), (y), V \\
\midrule
{}[de-], \textbf{de}yohnegi:yo: 

‘it is not good water’ & [deˀ-] \textbf{deˀ}agadǫtgadeˀ 

‘I’m not happy’\\
\lspbottomrule
\end{tabularx}}
\end{table}



%<\$img:prepronominal-prefix-pronunciation-D5;ebook=50\%>

\begin{table}
\caption{D prepronominals ending with a:,aǫ,ae (dual-indef)}
\label{tab:1:dualindef}
\scriptsize{
\begin{tabularx}{\textwidth}{XXXX}
\lsptoprule
C & I.o & 2 \\
\midrule
{}[da:-], \textbf{da:}gatnę:tsí:yaˀk 

‘I might break my arm’ & [daǫ-] & [dae-] e-person \\
\lspbottomrule
\end{tabularx}}
\end{table}



%<\$img:prepronominal-prefix-pronunciation-D6;ebook=50\%>

\begin{table}
\caption{D prepronominals ending with sa,sǫ,se,da,dǫ,de}
\label{tab:1:dualfacindefrep}
\scriptsize{
\begin{tabularx}{\textwidth}{XXXX}
\lsptoprule
C & I.o & 2 & \\
\midrule
{}[dǫsa-], \textbf{dǫ:sá}edaˀ 

‘she stood up again’ & {\itshape dual-fact-rep}\\

{}[daǫda-], 

[dǫ:da-], \textbf{dǫ:}dá:ge:ˀ 

‘I would come back’ & [daǫdǫ-], 

[dǫ:dǫ-] & [daǫde-], 

[dǫ:de-] & {\itshape dual-indef-cisl}\\

{}[daǫsa-], 

[dǫ:sa-] & [daǫsǫ-], 

[dǫ:sǫ] & [daǫse-], 

[dǫ:se-] & {\itshape dual-indef-rep}\\
\lspbottomrule
\end{tabularx}}
\end{table}



%<\$img:prepronominal-prefix-pronunciation-D7;ebook=50\%>

\begin{table}
\caption{D prepronominals ending with s,j,t,ji}
\label{tab:1:dualfutnegrep}
\scriptsize{
\begin{tabularx}{\textwidth}{XXXXX}
\lsptoprule
C & (sy) & s & 2 & \\
\midrule
{}[dęs-] \textbf{dęs}haih-wáętwaht 

‘he will bring forth a message’ & [dęj-], \textbf{dęj}ǫtnǫ̱-hsǫ́:goht
‘she will go in one door and out the other door’ (said at a funeral) & [dęt-] & [dęji-] \textbf{dęji}dwa-nǫ́hǫnyǫ:ˀ 

‘we all (incl.) will thank again’ & {\itshape dual-fut-rep}\\
\midrule
{}[des-], \textbf{des}ga:ní:yǫ:t 

‘it is not hanging’ & [deˀj-], \textbf{dej}a:gó:yǫ: 

‘she didn’t return’ & [det-] & [deˀji-] & {\itshape neg-rep}\\
\midrule
{}[des-], 

Tęˀ gęh 

\textbf{des}hó:yǫ:? 

‘Did he not come home?’ & [dej-] & [det-] & [deji-] imp (e-person) [deji-] \textbf{deji}dwa-yę:dí: 

‘we all (incl.) do not know any longer’ [dǫsa-] imp 

(a-person) & {\itshape dual-rep}\\
\lspbottomrule
\end{tabularx}}
\end{table}


%<\$img:prepronominal-prefix-pronunciation-E1;ebook=50\%>

\begin{table}
\caption{E prepronominals with ę (fut)}
\label{tab:1:eprepronfut}
\scriptsize{
\begin{tabularx}{\textwidth}{X}
\lsptoprule
C \\
\midrule
{}[ę-], \textbf{ę}gátnǫ̱hga:ˀ ‘I’m going to cut my hair’ \\
\lspbottomrule
\end{tabularx}}
\end{table}


%<\$img:prepronominal-prefix-pronunciation-E2;ebook=50\%>

\begin{table}
\caption{E prepronominals with ę(h)s,ęj,ęt,ęji (fut-rep)}
\label{tab:1:eprepronfutrep}
\scriptsize{
\begin{tabularx}{\textwidth}{XXXXX}
\lsptoprule
C & (sy) & s & 2 \\
\midrule
{}[ęhs-], \textbf{ęs}hahé:waht 

‘he will punish again’ & [ęj-], \textbf{ęj}é:kse:k 

‘she can eat it again’ & [ęt-], \textbf{ęt}sęˀnigǫhoˀne:k 

‘you will remove yourself’ & [ęji-] 

\textbf{ęji}dwanaˀdáiksǫ:ˀ ‘we all (incl.) will snack’\\
\lspbottomrule
\end{tabularx}}
\end{table}

\lipsum[1-1]

%<\$img:prepronominal-prefix-pronunciation-E3;ebook=50\%>

\begin{table}
\caption{E prepronominals with ęt,ęd,ęg,ędi (fut-cisl)}
\label{tab:1:eprepronfutcisl}
\scriptsize{
\begin{tabularx}{\textwidth}{XXXXX}
\lsptoprule
C & w, V & y & 2\\
\midrule
{}[ęt-] \textbf{ęt}gaę:ˀ ‘they (f/m) will come’ & [ęd-] & [ęg-], 

\textbf{ęg}yoˀgá:hah ‘early evening’ & [ędi-] 

\textbf{ędi}sa̱ˀnigǫ̱hí:yoh ‘your mind will become adjusted’\\
\lspbottomrule
\end{tabularx}}
\end{table}


%<\$img:prepronominal-prefix-pronunciation-H1;ebook=50\%>

\begin{table}
\caption{H prepronominals ending with ę}
\label{tab:1:transdualfut}
\scriptsize{
\begin{tabularx}{\textwidth}{XX}
\lsptoprule
C & \\
\midrule
{}[haˀdę-], 

\textbf{haˀd\’{ę}:}syehs 

‘you will put them all together’ & {\itshape trans-dual-fut}\\

{}[hę-], 

\textbf{hę}syę:ˀ 

‘you will put it there’ & {\itshape trans-fut}\\
\lspbottomrule
\end{tabularx}}
\end{table}


%<\$img:prepronominal-prefix-pronunciation-H2;ebook=50\%>

\begin{table}
\caption{H prepronominals ending with e,i,eˀ}
\label{tab:1:epreprontrans}
\scriptsize{
\begin{tabularx}{\textwidth}{XXX}
\lsptoprule
C & 2\\
\midrule
{}[haˀde-], \textbf{haˀde}ga̱ˀdréhdage: ‘all kinds of cars’ & [haˀde-], [haˀdi-]

imp (e-person) 

{}[haˀt-] imp 

(a-person) & {\itshape trans-dual}\\

{}[he-] & [heˀ-] imp 

(eperson), \textbf{heˀ}sá:dih 

‘throw it away from me!’ & {\itshape trans}\\
\lspbottomrule
\end{tabularx}}
\end{table}

\lipsum[1-1]


%<\$img:prepronominal-prefix-pronunciation-H3;ebook=50\%>

\begin{table}
\caption{H prepronominals ending with aˀ,ǫ,eˀ (trans-fact)}
\label{tab:1:transfact}
\scriptsize{
\begin{tabularx}{\textwidth}{XXXX}
\lsptoprule
C & I.o & 2\\
\midrule
{}[haˀ{}-], \textbf{haˀ}geˀ ‘I am going there’ & [hǫ-] & [heˀ-]\\
\lspbottomrule
\end{tabularx}}
\end{table}

%<\$img:prepronominal-prefix-pronunciation-H4;ebook=50\%>

\begin{table}
\caption{H prepronominals ending with t,d,g (trans-fact-dual)}
\label{tab:1:transfactdual}
\scriptsize{
\begin{tabularx}{\textwidth}{XXXX}
\lsptoprule
C & w, V & y\\
\midrule
{}[haˀt-], 

\textbf{haˀt}gęnętsáǫnyǫ:ˀ 

‘I waved my arms’ & [haˀd-] & [haˀg-]\\
\lspbottomrule
\end{tabularx}}
\end{table}

%<\$img:prepronominal-prefix-pronunciation-H5;ebook=50\%>

\begin{table}
\caption{H prepronominals ending with a:,aǫ,ae,e:}
\label{tab:1:transdualindef}
\scriptsize{
\begin{tabularx}{\textwidth}{XXXX}
\lsptoprule
C & I.o & 2 & \\
\midrule
{}[ha:-], 
to hǫ: 

\textbf{ha:}gi̱ˀdrǫ:daˀk


‘where I would dwell’ & [haǫ-] & [hae-] [haˀ-] 

imp (a-person) & {\itshape trans-indef}\\
\midrule
{}[haˀda:-] & [haˀdaǫ-] & [haˀdae-], [haˀde:-] & {\itshape trans-dual-indef}\\
\lspbottomrule
\end{tabularx}}
\end{table}

%<\$img:prepronominal-prefix-pronunciation-H6;ebook=50\%>

\begin{table}
\caption{H prepronominals ending with sa(ˀ),sǫ,se}
\label{tab:1:transindefrepfact}
\scriptsize{
\begin{tabularx}{\textwidth}{XXXXX}
\lsptoprule
C & (w), (y), V & I.o & 2 & \\
\midrule
{}[haǫsa-], [hǫ:sa-], 

to e: \textbf{hǫsá:}ge:ˀ 

‘I would go again’ & [hǫ:saˀ-] & [haǫsǫ-], {}[hǫ:sǫ-] & [haǫse-], [hǫ:se-] & {\itshape trans-indef-rep}\\
\midrule
{}[hǫsa-], \textbf{hǫ:sa}há:yǫˀ ‘he went back inside’ & [hǫsaˀ-] & [hǫsǫ-] & [hǫse-] & {\itshape trans-rep-fact}\\
\lspbottomrule
\end{tabularx}}
\end{table}

\lipsum[1-1]

%<\$img:prepronominal-prefix-pronunciation-H7;ebook=50\%>

\begin{table}
\caption{H prepronominals ending with (h)s,j,t,ji}
\label{tab:1:transfutrep}
\scriptsize{
\begin{tabularx}{\textwidth}{XXXXX}
\lsptoprule
C & (sy) & s & 2 & \\
\midrule
{}[hęhs-], 

\textbf{hęs}geˀ ‘I am going back there’ & [hęj-], \textbf{hęj}a:gǫ:nęht ‘they (f / m) shall swallow again’ & [hęt-], 

\textbf{hęt}syę:ˀ 

‘you will put it back in its place’ & [hęji-] 

\textbf{hęjí}sasha:ˀ 

‘you will remember back in time’ & {\itshape trans-fut-rep}\\
\midrule
{}[hehs-], I:wa:kˀáh \textbf{hes}hoh-dri̱hsdǫhǫgyeˀ.
‘He’s getting nearer and nearer.’ & [hej-], \textbf{hej}ónǫ̱hsǫ:t ‘the next room’ & [het-] & [heji-]

[heji-] imp 

(e-person), 

Gaę hwaˀ nhǫ: \textbf{heji}sáihoˀdeˀ? ‘Where do you work again?’ 

{}[hǫsa-] imp 

(a-person) & {\itshape trans-rep}\\
\lspbottomrule
\end{tabularx}}
\end{table}


%<\$img:prepronominal-prefix-pronunciation-I;ebook=50\%>

\begin{table}
\caption{I prepronominals with i(proth) or ending with s,j,t}
\label{tab:1:prothrep}
\scriptsize{
\begin{tabularx}{\textwidth}{XXXX}
\lsptoprule
C & (sy) & s & \\
\midrule
{}[i-], \textbf{í:}geˀs ‘I am here’ &  &  & {\itshape proth}\\
\midrule
{}[is-], \textbf{is}geˀs ‘I’ve returned home’ & [ij-] & [it-] & {\itshape proth-rep}\\
\lspbottomrule
\end{tabularx}}
\end{table}


%<\$img:prepronominal-prefix-pronunciation-I2;ebook=50\%>

\begin{table}
\caption{I prepronominals ending with t,d,g (proth-cisl)}
\label{tab:1:prothcisl}
\scriptsize{
\begin{tabularx}{\textwidth}{XXXX}
\lsptoprule
C & w, V & y\\
\midrule
{}[it-] & [id-] & [ig-] \textbf{í:g}ye:t ‘she is standing over there’\\
\lspbottomrule
\end{tabularx}}
\end{table}


%<\$img:prepronominal-prefix-pronunciation-N1;ebook=50\%>

\begin{table}
\caption{N prepronominals ending with t,d,g,di}
\label{tab:1:partothercisl}
\scriptsize{
\begin{tabularx}{\textwidth}{XXXXX}
\lsptoprule
C & w, V & y & 2 & \\
\midrule
{}[naˀdęt-], \textbf{naˀd\’{ę}:t}ge:ˀ 

‘I’ll come back over here, return’ & [naˀdęd-] & [naˀdęg-] &  & {\itshape part-dual-fut-cisl}\\
\midrule
{}[naˀdet-], \textbf{naˀdé:t}geˀ 

‘I am coming back over here, returning’ & [naˀded-] & [naˀdeg-], 


sh\’{ę}h \textbf{nadeg}yadǫdai ‘during the week’ &  & {\itshape part-dual-cisl}\\
\midrule
{}[nęt-] & [nęd-] & [nęg-] & [nędi-] & {\itshape part-fut-cisl}\\
\midrule
{}[nit-] \textbf{nit}ha:wé:nǫh ‘male stranger’ & [nid-] & [nig-], \textbf{nig}yagawé:nǫh ‘she comes from’ 

(‘she is X years old’) & [nidi-] \textbf{nidi}hsé:nǫh ‘you (singular) 

come from’ (‘you are X years old’) & {\itshape part-cisl}\\
\lspbottomrule
\end{tabularx}}
\end{table}


%<\$img:prepronominal-prefix-pronunciation-N2;ebook=50\%>

\begin{table}
\caption{N prepronominals ending with aˀ,ǫ/wag,eˀ}
\label{tab:1:parttransfact}
\scriptsize{
\begin{tabularx}{\textwidth}{XXXX}
\lsptoprule
C & I.o & 2 & \\
\midrule
{}[naˀ-], sh\’{ę}h \textbf{naˀ}á:węh ‘how it happened’ & [nǫ-], [niwag-] 

\textbf{nǫ:}ge̱ˀsyáoˀdęh,

niwage̱ˀsyáoˀdęh

‘I’m… clan’ & [neˀ-] & {\itshape part-fact}\\
\midrule
{}[nhaˀ-] & [nhǫ-] & [nheˀ-] & {\itshape part-trans-fact}\\
\lspbottomrule
\end{tabularx}}
\end{table}


%<\$img:prepronominal-prefix-pronunciation-N3;ebook=50\%>

\begin{table}
\caption{N prepronominals ending with da,dǫ,di,de,sa,sǫ,se}
\label{tab:1:partothercislrep}
\scriptsize{
\begin{tabularx}{\textwidth}{XXXX}
\lsptoprule
C & I.o & 2 & \\
\midrule
{}[naˀdǫda-], 

sh\’{ę}h \textbf{naˀdǫ:dá:}ge:ˀ 

‘where I’m coming from’ & [naˀdǫdǫ-] & [naˀdǫdi-] & {\itshape part-fact-dual-cisl}\\
\midrule
{}[nǫda-], \textbf{nǫdá:}ge:ˀ ‘I come from’ & [nǫdǫ-] & [nǫdi-] 

Gaę nhǫ \textbf{nǫdi:}sé:nǫh? ‘Where are you from?’ & {\itshape part-fact-cisl}\\
\midrule
{}[naǫda-], [nǫ:da-] & [naǫdǫ-], 

{}[nǫ:dǫ-] & [naǫda-], [naǫde-], [nǫ:de-] & {\itshape part-indef-cisl}\\
\midrule
{}[naˀdaǫda-], 

{}[naˀdǫ:da-], 

\textbf{naˀdǫ:dá:}geˀ ‘I should come this way’ & [naˀdaǫdǫ-], 

[naˀdǫ:dǫ-] &  & {\itshape part-dual-indef-cisl}\\
\midrule
{}[naǫsa-], [nǫ:sa-], 

\textbf{naǫsa}ǫgway\’{ę}hę:k ‘what we all should be doing’ & [naǫsǫ-], [nǫ:sǫ-] & [naǫse-] & {\itshape part-indef-rep}\\
\midrule
{}[nǫsaˀ-] & [nǫsǫ-] & [nǫse-] & {\itshape part-fact-rep}\\
\lspbottomrule
\end{tabularx}}
\end{table}


%<\$img:prepronominal-prefix-pronunciation-N4;ebook=50\%>

\begin{table}
\caption{N prepronominals ending with t,d,g,de/di}
\label{tab:1:partfactdual}
\scriptsize{
\begin{tabularx}{\textwidth}{XXXXX}
\lsptoprule
C & w, V & y & 2 & \\
\midrule
{}[naˀt-] & [naˀd-] & [naˀg-] & [naˀde-], [naˀdi-] & {\itshape part-fact-dual}\\
\lspbottomrule
\end{tabularx}}
\end{table}


%<\$img:prepronominal-prefix-pronunciation-N5;ebook=50\%>

\begin{table}
\caption{N prepronominals ending with e,ę}
\label{tab:1:partother}
\scriptsize{
\begin{tabularx}{\textwidth}{XX}
\lsptoprule
C & \\
\midrule
{}[naˀde-], \textbf{naˀde}gáǫdreˀ 

‘how far apart they are’ & {\itshape part-dual}\\

{}[naˀdę-] & {\itshape part-dual-fut}\\

{}[nę-], \textbf{nę}wú:kˀuh, ‘it will be small’ & {\itshape part-fut}\\

{}[nhę-] & {\itshape part-trans-fut}\\

{}[nhe-], \textbf{nhé:}yoht ‘suddenly it occurred’ & {\itshape part-trans}\\
\lspbottomrule
\end{tabularx}}
\end{table}


%<\$img:prepronominal-prefix-pronunciation-N6;ebook=50\%>

\begin{table}
\caption{N prepronominals ending with a:,aǫ,ae}
\label{tab:1:partdualtransindef}
\scriptsize{
\begin{tabularx}{\textwidth}{XXXX}
\lsptoprule
C & I.o & 2 & \\
\midrule
{}[na:-], I:wí: to 

\textbf{na:}yá:węh 

‘I want, intend that 

to happen’ & [naǫ-] & [nae-] & {\itshape part-indef}\\
\midrule
{}[naˀda:-] & [naˀdaǫ-] & [naˀdae-] & {\itshape part-dual-indef}\\
\midrule
{}[nha:-] & [nhaǫ-] & [nhae-] & {\itshape part-trans-indef}\\
\lspbottomrule
\end{tabularx}}
\end{table}


%<\$img:prepronominal-prefix-pronunciation-N7;ebook=50\%>

\begin{table}
\caption{N prepronominals ending with ni,naˀ (part)}
\label{tab:1:prepronpart}
\scriptsize{
\begin{tabularx}{\textwidth}{XXX}
\lsptoprule
C & 2 & \\
\midrule
{}[ni-], \textbf{ni}gahn\’{ę}:yaˀ ‘it is short’ & [naˀ-] imp, let’s & {\itshape part}\\
\lspbottomrule
\end{tabularx}}
\end{table}


%<\$img:prepronominal-prefix-pronunciation-N8;ebook=50\%>


\begin{table}
\caption{N prepronominals ending with (h)s,j,t,ji}
\label{tab:1:partfutrep}
\scriptsize{
\begin{tabularx}{\textwidth}{XXXXX}
\lsptoprule
C & (sy) & s & 2 & \\
\midrule
{}[nęhs-] & [nęj-] & [nęt-] & [nęji-] & {\itshape part-fut-rep}\\
\midrule
{}[nis-], \textbf{Nis}hayé:no:s ‘Hold Unto’ (Seneca Chief Title) & [nij-], \textbf{nij}agowé:nǫh ‘female stranger’ & [nit-] & [niji-] & {\itshape part-rep}\\
\lspbottomrule
\end{tabularx}}
\end{table}


%<\$img:prepronominal-prefix-pronunciation-S1;ebook=50\%>

\begin{table}
\caption{S prepronominals with s,j,sa,ji (rep)}
\label{tab:1:srep}
\scriptsize{
\begin{tabularx}{\textwidth}{XXXXX}
\lsptoprule
C & (sy) & s & 2 & \\
\midrule
{}[s-], 

\textbf{s}gayáˀda:t 

‘one (living thing)’ & [j-], 

\textbf{j}odr\’{ę}:no:t 

‘it is singing again’ & [sa-] (imp) 

\textbf{sa}sęhni̱hs-gahsrǫ:nih

‘fix the wheel’

\textbf{sa}sahdę:dih 

‘go home!’ & [ji-] 

(imp) 

\textbf{ji}dwahshé:t e:ˀ 

‘let’s count again!’ & {\itshape rep}\\
\lspbottomrule
\end{tabularx}}
\end{table}


%<\$img:prepronominal-prefix-pronunciation-S2;ebook=50\%>

\begin{table}
\caption{S prepronominals with sa,sǫ,se (rep-fact)}
\label{tab:1:srepfact}
\scriptsize{
\begin{tabularx}{\textwidth}{XXXX}
\lsptoprule
C & I.o & 2\\
{}[sa-], 

\textbf{sa}gęˀnhoˀtrá:twaht 

‘I missed the ball’ & [sǫ-], 

\textbf{sǫ}gi:daˀ 

‘I went back to sleep’ & [se-]\\
\lspbottomrule
\end{tabularx}}
\end{table}

%<\$img:prepronominal-prefix-pronunciation-T1;ebook=50\%>

\begin{table}
\caption{T prepronominals ending with t,d,g,di}
\label{tab:1:coinfutdualcisl}
\scriptsize{
\begin{tabularx}{\textwidth}{XXXXX}
\lsptoprule
C & w, V & y & 2 & \\
{}[tsaˀdet-] & [tsaˀded-] & [tsaˀdeg-], \textbf{tsadeg}ya̱-ˀdǫdadrehk ‘the week before last’ &  & {\itshape coin-dual-cisl}\\
{}[tsęt-] & [tsęd-] & [tsęg-] & [tsędi-] & {\itshape coin-fut-cisl}\\
{}[tsit-] & [tsid-] & [tsig-] & [tsidi-] & {\itshape coin-cisl}\\
\lspbottomrule
\end{tabularx}}
\end{table}


%<\$img:prepronominal-prefix-pronunciation-DGT;ebook=50\%>

\begin{table}
\caption{T prepronominals with t,d,g,de,di (cisl)}
\label{tab:1:scisl}
\scriptsize{
\begin{tabularx}{\textwidth}{XXXXX}
\lsptoprule
C & w, V & y & 2 & \\
{}[t-], 

\textbf{t}hadínagreˀ 

‘that’s where they live over there’ & [d-], 

né:ˀ gwáhs 

textbf{d}wakyęda̱-hgówanęh

‘the biggest chair’ & [g-], 

\textbf{g}ye̱ˀdrǫˀ

‘she’s at home’ & [de-] imp 

(e-person) 

\textbf{de}sáˀdre:, 

‘drive over here’ 

[di-] 

\textbf{di}sáhdęgyǫ: 

‘you come from there’ & {\itshape cisl}\\
\lspbottomrule
\end{tabularx}}
\end{table}


%<\$img:prepronominal-prefix-pronunciation-T2;ebook=50\%>

\begin{table}
\caption{T prepronominals ending with i,aˀ}
\label{tab:1:coinorcontr}
\scriptsize{
\begin{tabularx}{\textwidth}{XXX}
\lsptoprule
C & 2 & \\
{}[tsi-], 

\textbf{tsi}yeksa̱ˀda:sé:ˀah

‘when she was a teenager’ & [tsaˀ-] imp, let’s & {\itshape coin}\\
{}[ti-], 

\textbf{ti}gaˀdre̱hdá:deˀ

‘a different car’ & [taˀ-] imp, let’s & {\itshape contr}\\
\lspbottomrule
\end{tabularx}}
\end{table}


%<\$img:prepronominal-prefix-pronunciation-T3;ebook=50\%>
\begin{table}
\caption{T prepronominals ending with e(ˀ),ę}
\label{tab:1:coincontrother}
\scriptsize{
\begin{tabularx}{\textwidth}{XX}
\lsptoprule
C & \\
\midrule
{}[taˀde-] & {\itshape contr-dual}\\
{}[tsaˀde-], 

\textbf{tsaˀdé:}waˀs 

‘they are the same size’ & {\itshape coin-dual}\\
{}[taˀdę-] 

\textbf{taˀdę}jatáhahk 

‘you two will walk side by side’ & {\itshape contr-dual-fut}\\
{}[tę-], 

\textbf{tę}hayętó:ˀah 

‘he’ll just plant it any old way or helter-skelter’ & {\itshape contr-fut}\\
{}[tsaˀdę-], 

\textbf{tsaˀd\’{ę}}hsyę:ˀ 

‘you will lay them side by side’ & {\itshape coin-dual-fut}\\
{}[tsę-] & {\itshape coin-fut}\\
{}[taˀde-] 

\textbf{taˀde}gawá:yǫ:t 

‘it has no wings’ & {\itshape contr-neg}\\
{}[teˀ-], 

tęˀ toh \textbf{teˀ}gáhe:ˀ 

‘it’s not sitting way over there’ & {\itshape contr-trans}\\
\lspbottomrule
\end{tabularx}}
\end{table}


%<\$img:prepronominal-prefix-pronunciation-T4;ebook=50\%>

\begin{table}
\caption{T prepronominals ending with aˀ,ǫ,e(ˀ),i}
\label{tab:1:contrcoinfactdual}
\scriptsize{
\begin{tabularx}{\textwidth}{XXXX}
\lsptoprule
C & I.o & 2 & \\
\midrule
{}[taˀ-] & [tǫ-] & [teˀ-] & {\itshape contr-fact}\\
{}[tsaˀ-], 

\textbf{tsaˀ}ga:y\’{ę}:daˀ 

‘when it became’ & [tsǫ-] & [tseˀ-] & {\itshape coin-fact}\\
{}[tsaˀde-] & [tsǫde-] & [tsaˀdi-] & {\itshape coin-fact-dual}\\
\lspbottomrule
\end{tabularx}}
\end{table}


%<\$img:prepronominal-prefix-pronunciation-T5;ebook=50\%>

\begin{table}
\caption{T prepronominals ending with de/di,t,d,g,di}
\label{tab:1:othercisldual}
\scriptsize{
\begin{tabularx}{\textwidth}{XXXXX}
\lsptoprule
C & w, V & y & 2 & \\
{}[tǫde-], 

tęˀ \textbf{tǫde}sa̱ˀdré:

tó:gyęh 

‘you won’t drive that over here’ &  &  & [tǫdi-] & {\itshape dual-fact-cisl}\\
{}[tsaˀt-] & [tsaˀd-] & [tsaˀg-] & [tseˀd-] V 

{}[tseˀg-] y 

{}[tseˀt-] C & {\itshape coin-fact-cisl}\\
{}[tsa:t-] & [tsa:d-] & [tsa:g-], 

[tsaeg-] y & [tsaed-] y, V 

[tsaet-] C & {\itshape coin-indef-cisl}\\
{}[teˀt-] & [teˀd-] & [teˀg-] &  & {\itshape contr-trans-dual}\\
{}[taˀt-] & [taˀde-], 

{} [teˀd-] y, V & [taˀg-], 

{}[teˀg-] y & [teˀdi-] & {\itshape contr-fact-dual}\\
\lspbottomrule
\end{tabularx}}
\end{table}


%<\$img:prepronominal-prefix-pronunciation-T6;ebook=50\%>

\begin{table}
\caption{T prepronominals ending with a:,aǫ,ae}
\label{tab:1:contrcoinindefcislrep}
\scriptsize{
\begin{tabularx}{\textwidth}{XXXX}
\lsptoprule
C & I.o & 2 & \\
{}[ta:-] & [taǫ-] & [tae-], 2 & {\itshape contr-trans-indef}\\
{}[ta:-], tęˀ 

\textbf{ta:}hay\’{ę}:toh 

‘no, he won’t plant’ & [taǫ-] & [tae-], [te:-] 

tęˀ \textbf{tae}swá:gęh, 

\textbf{te:}swá:gęh 

‘you all shouldn’t see it’ 

\textbf{te:}sagáę 

‘you are not willing’ & {\itshape contr-indef}\\
{}[taˀda:-] & [taˀdaǫ-] & [taˀdae-] & {\itshape contr-dual-indef}\\
{}[tsa:-] & [tsaǫ-] & [tsae-], [tse:-] & {\itshape coin-indef}\\
{}[tsaˀda:-] & [tsaˀdaǫ-] & [tsaˀdae-] & {\itshape coin-dual-indef}\\
{}[taǫda-], [tǫ:da-] & [taǫdǫ-], [tǫ:dǫ-] & [taǫde-] & {\itshape contr-indef-cisl}\\
{}[taǫsa-] & [taǫsǫ-] & [taǫse-] & {\itshape contr-indef-rep}\\
\lspbottomrule
\end{tabularx}}
\end{table}


%<\$img:prepronominal-prefix-pronunciation-T7;ebook=50\%>

\begin{table}
\caption{T prepronominals ending with s,j,t,ji}
\label{tab:1:coincontrotherrep}
\scriptsize{
\begin{tabularx}{\textwidth}{XXXXX}
\lsptoprule
C & (sy) & s & 2 & \\
{}[tsa:s-], [tsaǫs-] & [tsa:j-] & [tsa:t-] & [tsaes-] & {\itshape coin-indef-rep}\\
{}[tsaˀs-] & [tsaˀj-] & [tsaˀt-] & [tseji-] & {\itshape coin-fact-rep}\\
{}[taˀdes-] & [taˀdej-] &  &  & {\itshape contr-dual-rep}\\
{}[tęs-] &  &  &  & {\itshape contr-fut-rep}\\
{}[tis-], &  &  &  & {\itshape contr-rep}\\
{}[tsęhs-] & [tsęj-] & [tsęt-] & [tsęji-] & {\itshape coin-fut-rep}\\
{}[tsihs-] & [tsij-] & [tsit-] & [tsiji-] & {\itshape coin-rep}\\
\lspbottomrule
\end{tabularx}}
\end{table}

\begin{verbatim}%%move bib entries to  localbibliography.bib
\end{verbatim}
