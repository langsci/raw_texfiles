\addchap{Acknowledgements}
Nyá:węh to the coauthors, the late Frances Froman, the late Lottie Key, and the late Alfred Keye, to whom I am greatly indebted. Starting in the late 1970’s and early 1980’s, these people dedicated their working and post-retirement lives and intellects to teaching \textit{Gayogo̱ho:nǫhnéha:ˀ} ‘the Cayuga language’ and \textit{Shęh Niyǫgwaihóˀdę:} ‘our ways, beliefs, doings’ -- years before others thought it was important to do so. Several other kind speakers, including Tom and Tracy Deer, Kehte Deer, and the late Janie Johnson, also contributed greatly to informing this grammar. The value of the grammar is in their shared wisdom. 

Nya:wę́h hniˀ to Amos Key Jr. (formerly the Director of the Language Program, Woodland Cultural Centre), who has been a driving force in Gayogo̱honǫˀnéha:ˀ language advocacy for over 40 years. He assembled the team for this grammar (and a previously-published dictionary) and obtained the funding and in-kind support for the project. Amos also contributed to language sessions, asked thought\-provoking questions and added wonderful examples of language and culture. 

Nia:wenh to Angie Monture (former Executive Assistant, Language Program, Woodland Cultural Centre), who provided excellent logistical and moral support. Her skills, kindness, patience, and dedication ensured the success of this language project, and many other projects besides. And finally, nyá:węh to Deb Lickers for quietly and patiently taking care of finances.

Everyone just acknowledged contributed a sense of humour and an ethos of  humanity to the grammar project. For this and a great many other reasons, 

\begin{center}
\textit{Nya:wę́h swagwé:gǫh} ‘thank you all’
\end{center}

\addchap{Funding acknowledgements}
The following funders were integral to the Gayogo̱honǫˀnéha:ˀ dictionary and grammar projects: 

\begin{itemize}
\item The Woodland Cultural Centre provided in-kind support for the Gayogo̱honǫˀnéha:ˀ dictionary and grammar projects between 1992--2017.

\item The Six Nations Polytechnical Institute provided in-kind support for the Gayogo̱honǫˀnéha:ˀ grammar project between 2010--2015.

\item The Government of Ontario, Ministry of Education and Training, Workplace Preparation Branch provided the following funding:
\begin{itemize}
\item 1993, Project no. 4 Aboriginal Language Standardization Project, Cayuga Dictionary and Grammar. 
\item 1997, Project no. 4A, Aboriginal Language Standardisation Project, Cayuga Dictionary and Grammar. 
\end{itemize}

\item The Social Sciences and Humanities Research Council provided the following funding:
\begin{itemize}
\item 2005--2008, A new research paradigm for setting down the Cayuga oral tradition. SSHRC Strategic Research Grant no. 856-2004-1082 
\item 2010--2015, Cayuga Language Maintenance. Community-University Research Alliance (CURA) SSHRC grant no. 833-2009-1001
\end{itemize}

\item Memorial University of Newfoundland provided cash and in-kind support, including:
\begin{itemize}
\item 2010, the VP Research, the Dean of the School of Graduate Studies, and the Dean of Arts, matching funding for the Cayuga Language Maintenance SSHRC grant no. 833-2009-1001.
\end{itemize}

\end{itemize}
