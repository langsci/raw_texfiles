\addchap{Abbreviations}

\begin{tabularx}{\textwidth}{lQ}
\textsc{ø.punctual} & punctual suffix has been deleted / is not pronounced\\
\textsc{3s}	&	3s.zon (3rd singular zoic-neuter), as in \textsc{3s.a} ‘it’ (\textsc{a}-series)  \\
\textsc{3p}	&	3ns.zon	(3rd non-singular zoic-neuter), as in \textsc{3p.a} ‘they (animals, things)’ (\textsc{a}-series) \\
\textsc{a}	&	agent-like argument of canonical transitive verb, \textsc{a}-series prefix, as in \textsc{3ns.m.a} ‘they (males)’ (\textsc{a}-series)  \\
\textsc{alongside}           &      \stem{-kdagyeˀ} “alongside” locative         \\
\textsc{at}           &      \stem{-hneh} external locative               \\
\textsc{aug}             &      augmentative (either \stem{-ge:} or \stem{-go:wah})      \\
\textsc{ben}             &      benefactive                           \\
\textsc{beside}          &      \stem{-:kˀah} “beside” locative              \\
\textsc{caus}            &      causative                             \\
\textsc{caus-instr}      &      causative-instrumental                \\
\textsc{cis}             &      cislocative                           \\
\textsc{coin}            &      coincident                            \\
\textsc{completely} &	\stem{-jihwęh} ‘completely, fully’ \\
\textsc{contr}            &      contrastive                           \\
\textsc{customary}             &      customary            \\
\textsc{d}	&	dual (number), as in \textsc{2d.a} ‘you two’ (\textsc{a}-series)  \\
\textsc{(d)}	&	one argument must be dual, as in \textsc{1>2(d)} ‘I>you two’, ‘we two>you’, etc.   \\
\textsc{declare.kinship}		&	\stem{-shęˀ, -tsęˀ} kinship declaration suffix \\
\textsc{dim}             &      diminutive                            \\
\textsc{disl}            &      dislocative                           \\
\textsc{distr}           &      distributive                          \\
\textsc{du}            &      dualic (prepronominal prefix)         \\
\textsc{ex}	&	exclusive, as in \textsc{1p.ex.a} ‘we all (exclusive)’ (\textsc{a}-series) \\
\textsc{fac}             &      factual                               \\
\textsc{facil}           &      habitual facilitative                 \\
\textsc{fi}	&	feminine-indefinite, as in \textsc{3s.fi.a} ‘she, someone’ (\textsc{a}-series) \\
\textsc{former}             &      former                                \\
\textsc{fut}             &      future                                \\
\textsc{hab}             &      habitual                              \\
\textsc{in}        &      \stem{-gǫ:} internal locative (suffix)                \\
\textsc{in}	&	inclusive, as in \textsc{1d.in.a} ‘we two (inclusive)’ (\textsc{a}-series)  \\
\end{tabularx}

\begin{tabularx}{\textwidth}{lQ}
\textsc{inch}            &      inchoative                            \\
\textsc{increment}            &      noun increment                        \\
\textsc{indef}             &      indefinite  (optative)                          \\
\textsc{instr}           &      instrumental                          \\
\textsc{int}	& \textsc{interactive} series pronominal prefix, as in \textsc{2s>1s} ‘you>me’\\
\textsc{ints}            &      intensifier                           \\
\textsc{length}	&	suffix consisting of \stem{-:}, agnostically named \\
\textsc{modz}            &      modalizer                             \\
\textsc{neg}             &      negative                              \\
\textsc{nmlz}             &      nominalizer                           \\
\textsc{noun}	&	used in glosses to denote where a noun should be inserted \\
\textsc{no\_aspect}	&	verb lacking an aspect suffix \\
\textsc{ns}	&	non-singular, as in \textsc{3ns.m.a} ‘they (males)’ (\textsc{a}-series) \\
\textsc{(+ns)}	&	both arguments must be non-singular, as in \textsc{3fi>3fi(+ns)} ‘they (fe/males)>them(fe/males)’ \\	
\textsc{nsf}             &      noun stem former                      \\
\textsc{on}           &      \stem{-ˀgeh} external locative               \\
\textsc{opp}	&	the opposite transitive arguments are included in the gloss, e.g. \textsc{2p>3s.m(opp)} means either \textsc{2p>3ms} ‘you all>him’ or \textsc{3ms>2p} ‘he>you all’  \\
\textsc{p}	&	patient-like argument of canonical transitive verb, \textsc{p}-series prefix, as in \textsc{1s.p} ‘I’, ‘me’ (\textsc{p}-series)  \\
\textsc{p}   &   plural, as in \textsc{2p.a} ‘you all’ (\textsc{a}-series) \\
\textsc{(p)}	&	one argument must be plural, as in \textsc{1>2(p)} ‘I>you all’, ‘we>you’, etc.  \\
\textsc{part}            &      partitive                             \\
\textsc{past}            &      past                                  \\
\textsc{pl}            &      plural (stative verb suffix)   \\
\textsc{plrz}            &      pluralizer (word suffix)   \\
\textsc{pop}             &      populative                            \\
\textsc{purp}	&	purposive \\
\textsc{prog}            &      progressive                           \\
\textsc{proth}           &      prothetic                             \\
\textsc{punc}            &      punctual                              \\
\textsc{rec}	&	reciprocal \\
\textsc{refl}            &      reflexive                             \\
\textsc{rem}             &      remote                                \\
\textsc{rep}             &      repetitive                            \\
\end{tabularx}

\begin{tabularx}{\textwidth}{lQ}
\textsc{rev}             &      reversive                             \\
\textsc{s}	&	singular, as in \textsc{1s.a} ‘I’ (\textsc{a}-series) \\
\textsc{srf}             &      semireflexive                         \\
\textsc{stat}            &      stative                               \\
\textsc{transl}          &      translocative                         \\
\textsc{typ} & typicalizer \\
\textsc{zon}	&	zoic-neuter \\
\end{tabularx}



%\begin{tabularx}{.6\textwidth}[t]{@{}>{\scshape}lQ}
%on           &      \stem{-ˀgeh} external locative               \\
%at           &      \stem{-hneh} external locative               \\
%decla.kin       &      kinship.declaration                   \\
%in        &      \stem{-gǫ:} internal locative                \\
%beside          &      \stem{-:kˀah} “beside” locative              \\
%alongside           &      \stem{-kdagyeˀ} “alongside” locative         \\
%aug             &      augmentative (either \stem{-ge:} or \stem{-go:wah}       \\
%ben             &      benefactive                           \\
%caus            &      causative                             \\
%caus-instr      &      causative-instrumental                \\
%cis             &      cislocative                           \\
%coin            &      coincident                            \\
%contr            &      contrastive                           \\
%customary             &      customary (characterizer)             \\
%dim             &      diminutive                            \\
%disl            &      dislocative                           \\
%distr           &      distributive                          \\
%du            &      dualic (prepronominal prefix)         \\
%fac             &      factual                               \\
%facil           &      habitual facilitative                 \\
%former             &      former                                \\
%fut             &      future                                \\
%hab             &      habitual                              \\
%inch            &      inchoative                            \\
%\end{tabularx}%
%\begin{tabularx}{.4\textwidth}[t]{>{\scshape}lQ@{}}
%increment            &      noun increment                        \\
%indef             &      indefinite  (optative)                          \\
%instr           &      instrumental                          \\
%ints            &      intensifier                           \\
%modz            &      modalizer                             \\
%neg             &      negative                              \\
%nmlz             &      nominalizer                           \\
%nsf             &      noun stem former                      \\
%part            &      partitive                             \\
%past            &      past                                  \\
%pl            &      plural (stative verb suffix)   \\
%plrz            &      pluralizer (word suffix)   \\
%pop             &      populative                            \\
%prog            &      progressive                           \\
%proth           &      prothetic                             \\
%punc            &      punctual                              \\
%refl            &      reflexive                             \\
%rem             &      remote                                \\
%rep             &      repetitive                            \\
%rev             &      reversive                             \\
%srf             &      semireflexive                         \\
%stat            &      stative                               \\
%transl          &      translocative                         \\
%typ             &      typicalizer                           \\
%\\
%\end{tabularx}
