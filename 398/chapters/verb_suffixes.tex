\chapter{Verb suffixes} \label{Verb suffixes}
Verb suffixes perform a number of functions (described in the following sections). The order of verb suffixes is summarized in \tabref{figtab:1:verbsuffixorder}.


\begin{table}
\caption{Verb suffix order}
\label{figtab:1:verbsuffixorder}
\scriptsize{
\begin{tabularx}{\textwidth}{XXXX}
\lsptoprule
1 \distributive & 2 \textsc{role-adding}, 

\textsc{change of state} & 3 \textsc{aspect} & 4 \textsc{post-aspect}\\
\midrule
\stem{-ˀse} {\distributive} & \stem{-hd}, \stem{-ˀd}, \stem{-d} {\causative} & 3a  & \textsc{Tense} \\
\stem{-ǫ} {\distributive}& \stem{-hsd} {\causative-\instrumental} & \stem{{\stative}.verb-gy} {\progressive} & \stem{-gęhę:ˀ} {\past} \\
\stem{-nyǫ} {\distributive} & \stem{-ę, -ni} {\benefactive} & \stem{-hs} {\dislocative} & \stem{-hne:ˀ} {\remote} \\
\stem{-hnǫ} {\distributive} & \stem{-hs} {\benefactive} & \stem{-hn} {\dislocative} & \stem{-hk} {\former} \\
\stem{-sǫ} {\distributive}& \stem{-(ę)ˀ}, \stem{-(n)heˀ} \textsc{inch} & \stem{-ˀn} {\dislocative} & \stem{-(e:)k} {\modalizer} \\
\stem{-(d)rǫ} {\distributive}& \stem{-gw, -go} {\reversive} & \stem{-h} {\dislocative} & \\
\stem{-srǫ} {\distributive}& \stem{-hsi} {\reversive} & \stem{-ˀh} {\dislocative} & \\
\stem{(i)nyǫ-ˀse} {\distributive}-

{\distributive} &  & \stem{-dr} {\dislocative}&  \\
\stem{-ǫ-nyǫ} {\distributive}-{\distributive} &  & &  \\
\tablevspace
  &  & 3b & \\
 & & \stem{e}‛go'  & \\
 & & \stem{e-:} ‛go-{\purposive}' & \\
 \tablevspace
 & & 3c & \\
 & & \stem{-haˀ, -ˀs, -hs, -h} {\habitual} & \textsc{Non-tense}\\
& & \stem{-ǫh, -ęh, -ˀ} {\stative} & \stem{-ge:} {\augmentative}\\
& & \stem{-ˀ} {\punctual} & \stem{-ˀs} {\plural}\\
& &  & \stem{-sgǫ:} {\facilitative}\\
& &  & \stem{-jihwęh} \textsc{completely}\\
\lspbottomrule
\end{tabularx}}
\end{table}

Distributive suffixes appear immediately after the verb stem in slot 1. Role-adding and change-of-state suffixes appear in slot 2. The suffixes relevant for aspect all appear in slot 3. Meanwhile, the post-aspect suffixes in slot 4 convey past tense and other meanings. 

Slot 3 summarizes additional order constraints on the suffixes relevant for aspect. Regular verbs simply take a \textit{3c} \habitual, \stative, or {\punctual} suffix. \textsc{Progressive e-verbs} take the \textit{3a} \stem{\textsc{\stative.verb}-gy} {\progressive} combination, followed by the \textit{3b} verb \stem{e-ˀ} ‛go-\punctual' or by a \textit{3b} verb \stem{e-:} ‛go-{\purposive}' combination. \textsc{Dislocative} verbs take a \textit{3a} {\dislocative} suffix. \textsc{Dislocative e-verbs} take a \textit{3a} {\dislocative} suffix and then either the \textit{3b} verb \stem{e} ‛go' or \textit{3b} verb \stem{e-:} ‛go-{\purposive}' combination. All progressive and dislocative verbs then take a \textit{3c} aspect suffix. (Some details have been omitted. See \textit{E-verbs}, \sectref{ch:E-verbs}.) 

Slot 4 suffixes occur after Slot 3c suffixes. The \textsc{past}, \textsc{remote}, \textsc{former}, and \textsc{modalizer} suffixes, grouped together as \textsc{tense} suffixes, constitute further refinements of the aspect distinctions. The remaining suffixes (the \textsc{augmentative}, \textsc{plural}, \textsc{facilitative}, and \textsc{completely} suffixes) are grouped together as \textsc{non-tense} suffixes.


\section{Verb distributive suffixes} \label{Verb pluralizer and distributive suffixes}
The family of verb distributives \stem{-ˀse}, \stem{-ǫ}, \stem{-nyǫ}, \stem{-hnǫ}, \stem{-sǫ}, \stem{-drǫ}, \stem{-srǫ}, and the double distributives \stem{-ǫ-nyǫ} and \stem{(i)nyǫ-ˀse} convey the repetition of an action or state over several objects, times, or places (\cite[505]{mithun_watewayestanih_1984}). The distributive suffixes appear in slot 1 (see \tabref{figtab:1:verbsuffixorder}, page \pageref{figtab:1:verbsuffixorder}). All of these affixes are described in the following sections.


\subsection{\stem{-ˀse} \textsc{distributive} (\distributive)} \label{[-ˀse] (pluralizer)}
The \stem{-ˀse} {\distributive} suffix immediately follows the verb stem (slot 1 in \tabref{figtab:1:verbsuffixorder}). For example, \stem{-ˀse} occurs before a slot 2 \stem{-hd} {\causative} in \xref{ex:eventex7}.

\ea\label{ex:eventex7} otgę́ˀseht\\
\gll o-t-gę-ˀse-ht \\
\textsc{3s.p}-{\semireflexive}-see-{\distributive}-\exsc{\causative}.{\stative}\\
\glt ‘it is nice to see, interesting’ 
\z

The \stem{-ˀse} ending also appears before a slot 3 dislocative suffix (\ref{ex:eventex5}, see \sectref{Dislocative suffixes [-hs, -hn] etc}).

\ea\label{ex:eventex5} Dislocative verb \stem{verb-ˀse-h-} verb-{\distributive}-{\dislocative} and dislocative e-verb \stem{verb-ˀse-h-e-} verb-{\distributive}-\exsc{\dislocative}-go
\ea ęgagyǫ̱ˀséhaˀ\\
\gll ę-g-ag-yǫ̱-ˀsé-h-a-ˀ\\
 \fut-\textsc{1s.a}-{\semireflexive}-arrive-{\distributive}-{\dislocative}-{\joinerA}-{\punctual}\\
\glt `I am going to go and visit'

\newpage
\cfex{ęgágyǫ̱ˀse-:ˀ\\
\gll ę-g-ag-yǫ̱-ˀse-:ˀ\\
\fut-\textsc{1s.a}-{\semireflexive}-arrive-{\distributive}--{\punctual}\\
\glt ‘I will visit’
}

\ex agagyǫ̱ˀséhe-ˀ\\
\gll ę-g-ag-yǫ̱-ˀsé-h-e-ˀ\\
 \fut-\textsc{1s.a}-{\semireflexive}-arrive-{\distributive}-{\dislocative}-go-\exsc{\punctual}\\
\glt `I am going to go visit'

\cfex{hogyǫˀsé: tsǫ:\\
	\gll ho-g-yǫ-ˀsé-: tsǫ: \\
	\textsc{3s.m.p}-{\semireflexive}-arrive-{\distributive}-{\stative} just\\
	\glt `he is a live-in'}
\z
\z

For habitual aspect verbs, the \stem{-ˀse} suffix extends the time frame of a state of affairs (which seems to also intensify the meaning of the verb, \ref{ex:eventex}). For three-aspect (activity or happening) verbs, the \stem{-a-ˀse} {\joinerA}-\exsc{\distributive} combination possibly also denotes an extended timeframe or a series of actions \xxref{ex:eventex2}{ex:eventex3}.

\ea\label{ex:eventex}  habitual \stem{verb-ˀse-h} verb-{\distributive}-{\habitual}
\ea hoˀdaiháˀseh\\
\gll ho-ˀdaih-á-ˀse-h\\
 \textsc{3s.m.p}-hot-{\joinerA}-{\distributive}-{\habitual}\\
\glt `he is too hot'

\cfex{ oˀdáihę:\\
\gll o-ˀdáih-ę:\\
\textsc{3s.p}-hot-\textsc{\stative}\\
\glt ‘it is hot’
}

\ex dewagadawęnyáˀseh\\
\gll de-wag-ad-awęny-á-ˀse-h\\
 {\dualic}-\textsc{1s.p}-{\semireflexive}-stir-{\joinerA}-{\distributive}-{\habitual}\\
\glt `I am confused and doubtful'

\cfex{deyagodawęnyeˀ\\
	\gll de-yago-d-awęny-eˀ\\
	{\dualic}-\textsc{3s.fi.a}-stir-{\stative}\\
	\glt `she is walking about'}
\ex gaǫdagyenawáˀseh\\
\gll gaǫ-dag-yenaw-á-ˀse-h\\
 \textsc{3ns.fi.a}-{\reflexive}-catch-{\joinerA}-{\distributive}-{\habitual}\\
\glt `they help her'
\cfex{agaǫda:gyé:na:ˀ\\
	\gll a-gaǫ-da:g-yé:na:-ˀ\\
	\textsc{\factual-3ns.fi.a}-{\reflexive}-catch-{\punctual}\\
	\glt `they caught her'}
\z
\z

\ea\label{ex:eventex2} three-aspect \stem{verb-ˀse-:ˀ} verb-\textsc{\distributive-\punctual}
\ea a:gakeyatgę́ˀse:ˀ\\
\gll a:-gakey-at-gę́-ˀse-:ˀ\\
 {\indefinite}-\exsc{1s:3ns}-{\semireflexive}-see-{\distributive}-{\punctual}\\
\glt `I should watch them'
\cfex{haˀǫ́tgę̱ˀse:ˀ\\
	\gll haˀ-ǫ́-t-gę̱-ˀse-:ˀ\\
	\textsc{\translocative-3s.fi.a}-see-\textsc{\distributive-\punctual}\\
	\glt `she is watching over there'}
\ex jǫkihędǫ́ˀse:ˀ\\
\gll j-ǫki-hędǫ́-ˀse-:ˀ\\
 {\repetitive}-\exsc{3fis/3ns:1ns}-lead-{\distributive}-{\punctual}\\
\glt `she has gone on before us'
\cfex{hahę́:dǫ:\\
	\gll ha-hę́:dǫ-:\\
	\textsc{3s.m.a}-lead-{\stative}\\
	\glt `he is the front, the leader'}
\z
\z

\ea\label{ex:eventex3} three-aspect \stem{ę-verb-ˀse-:k} {\future}-verb-\textsc{\distributive-\modalizer}\\
ęhsáhsha̱ˀse:k\\
\gll ę-hs-áhsha-ˀse-:k\\
 \fut-\textsc{2s.p}-remember-{\distributive}-{\modalizer}\\
\glt `you will remember'
\cfex{ęwága̱hsha:ˀ \\
\gll ę-wag-ahsha:-ˀ\\
\textsc{\future-1s.p}-remember-\textsc{\punctual}\\
\glt ‘I will remember’ }
\z

Verbs ending with the \stem{-ˀse} {\distributive} suffix (and followed immediately by an aspect suffix) belong to the \textsc{mb1} conjugation class (\sectref{Aspect conjugation classes}), taking an \stem{-h} \textsc{\habitual} \xref{ex:eventex}, \stem{-:} \textsc{\stative}, and \stem{-:ˀ} \textsc{\punctual} suffix \xref{ex:eventex2}. 



\subsection{\stem{-ǫ}, \stem{-nyǫ}, \stem{-ǫ-nyǫ}, \stem{-hnǫ}, \stem{-hsǫ}, \stem{-drǫ}, \stem{-srǫ} \textsc{distributives} (\distributive)} \label{[-ǫ], [-nyǫ], distributives}
The rest of the family of {\distributive} suffixes appears immediately after the verb stem in slot 1 (\tabref{figtab:1:verbsuffixorder}, page \pageref{figtab:1:verbsuffixorder}). For stative verbs and kinship terms, the distributive conveys the idea that the quality or relationship described by the verb holds of several objects or people. For three-aspect (activity or happening) verbs, the distributive contributes the idea that the action is repeated several times. The choice of distributive suffix is unique to each verb.

\ea\label{ex:distrex} \stem{-ǫ-ˀ} {\distributive}-{\stative} with {\stative} verbs

\ea ageˀdre̱hdáędǫˀ\\
\gll ag-e-ˀdre̱hd-á-ęd-ǫ-ˀ\\
\textsc{1s.p}-{\joinerE}-car-{\joinerA}-lie.stative-{\distributive}-{\stative}\\
\glt `I have several cars'

\cfex{ageˀdre̱hdáęˀ\\
	\gll ag-e-ˀdre̱hd-á-ęˀ\\
	\textsc{1s.p}-{\joinerE}-car-{\joinerA}-lie.{\stative}\\
	\glt `I have a car'}

\ex aknǫhsó:dǫˀ\\
\gll ak-nǫhs-ó:d-ǫ-ˀ\\
 \textsc{1s.p}-house-stand.{\stative}-{\distributive}-{\stative}\\
\glt `I have several houses'

\cfex{aknǫ́hso:t\\
	\gll ak-nǫ́hs-o:t\\
	\textsc{1s.p}-house-stand.{\stative}\\
	\glt `I have a house'}

	\ex Gani:yǫ́:dǫˀ\\
\gll ga-ni:yǫ́:d-ǫ-ˀ\\
 \textsc{3s.a}-hang.{\stative}-{\distributive}-{\stative}\\
\glt `name of the white dog who went to heaven'

\cfex{ganí:yǫ:t\\
	\gll ga-ní:yǫ:t\\
	\textsc{3s.a}-hang.{\stative}\\
	\glt `it is hanging'}
\z
\z


\ea\label{ex:distrex2} \stem{-nyǫ-ˀ} {\distributive}-{\stative}, \stem{-ǫ-nyǫ-ˀ} {\distributive}-{\distributive}-{\stative} and \stem{-nyǫ-:} \exsc{\distributive-\length.\noaspect} with three-aspect verbs
\ea Dęˀ ni:s hoˀdęˀ swayasǫ̱hǫ́:nyǫˀ\\
\gll dęˀ ni:s hoˀdęˀ swa-yas-ǫ̱h-ǫ́:-nyǫ-ˀ\\
what you kind \exsc{2p.p}-named-{\stative}-{\distributive}-{\distributive}-{\stative}\\
\glt ‘What are your names?’ (each individual’s name is requested)

\cfex{Dęˀ ni:s hoˀdęˀ swayá:sǫh?  \\
	\gll Dęˀ ni:s hoˀdęˀ swa-yá:s-ǫh\\
	what you kind \exsc{2p.p}-named-{\stative}\\
	\glt ‘What is your name?’ (asking people for their team or group name)}

\ex dwakdǫ́:nyǫ:\\
\gll dwa-kdǫ́:-nyǫ:\\
\textsc{1p.in.a}-look.at-{\distributive}.{\noaspect}\\
\glt ‘let’s examine’

\cfex{dwakdǫ:\\
	\gll dwa-kdǫ:\\
	\textsc{1p.in.a}-look.at.{\noaspect}\\
	\glt `let us look'}
\z
\z

\ea\label{ex:distrex3} \stem{-hnǫ-ˀ} {\distributive}-{\stative} with stative verbs
\ea ga̱háˀdǫ̱hnǫˀ\\
\gll ga̱-háˀd-ǫ̱-hnǫ-ˀ\\
 \textsc{3s.a}-dry-{\stative}-{\distributive}-{\stative}\\
\glt `a clothesline (several objects drying)'

\cfex{oháˀdǫh\\
\gll o-háˀd-ǫh\\
\textsc{3s.p}-dry-{\stative}\\
\glt ‘it is dry’ (a field, the weather, etc.)
}

\ex ohnawáodǫnyǫˀ\\
\gll o-hnaw-á-od-ǫ-nyǫ-ˀ\\
 \textsc{3s.p}-rapids-{\joinerA}-stand.{\stative}-{\distributive}-{\distributive}-{\stative}\\
\glt `springs'

\cfex{ohnáwaot\\
	\gll o-hnáw-a-ot\\
	\textsc{3s.p}-rapids-{\joinerA}-stand.{\stative}\\
	\glt `well, spring'}
\z
\z


\ea\label{ex:distrex4} \stem{-hsǫ-ˀ} {\distributive}-{\stative} and \stem{-hsǫ-:ˀ} {\distributive}-{\punctual} with three-aspect verbs
\ea hoyę́twa̱hsǫˀ\\
\gll ho-yę́tw-a̱-hsǫ-ˀ\\
 \textsc{3s.m.p}-plant-{\joinerA}-{\distributive}-{\stative}\\
\glt `he planted several things, a variety of things'
\cfex{hoyę́:twęh\\
	\gll ho-yę́:tw-ęh\\
	\textsc{3s.m.p}-plant-{\stative}\\
	\glt `he has planted'}
\ex shǫgwaęnáwi̱hsǫˀ\\
\gll shǫgwa-ęn-áwi̱-hsǫ-ˀ\\
 \textsc{3ms:1p}-song-give-{\distributive}-{\stative}\\
\glt `he has given us songs'

\cfex{agwe:gǫ́h shǫgwaęná:wih\\
\gll agwe:gǫ́h  shǫgwa-ęn-á:wi-h\\
all 	\textsc{3ms:1p}-song-give-{\stative}\\
\glt ‘he has given all of us songs’
}
\ex ędwana̱ˀdá:ksǫ:ˀ\\
\gll ę-dwa-na̱ˀdá:-k-sǫ-:ˀ\\
 \fut-\exsc{1p.in.a}-bread-eat-{\distributive}-{\punctual}\\
\glt `we will have a snack'
\z
\z

\ea\label{ex:distrex5} \stem{-(d)rǫ-ˀ} {\distributive}-{\stative} with stative verbs and kinship terms
\ea deswatnętsáˀdrǫˀ\\
\gll de-sw-at-nęts-á-ˀ-drǫ-ˀ\\
 {\dualic}-\exsc{2p.p}-{\semireflexive}-arm-{\joinerA}-{\nsf}-\exsc{\distributive-\stative}\\
\glt `you all have your arms crossed'
\ex degaǫdę́hnǫdrǫˀ\\
\gll de-gaǫ-dę́-hnǫd-rǫ-ˀ\\
 {\dualic}-\textsc{3ns.fi.a}-{\semireflexive}-follow-{\distributive-\stative}\\
\glt `they are sisters'
\cfex{degadęhnǫ́:de:ˀ\\
	\gll de-g-adę-hnǫ́:d-e-:-ˀ\\
	{\dualic}-\textsc{1s.a}-{\semireflexive}-follow-go-{\purposive}-{\stative}\\
	\glt `my brother or sister'}
\ex de̱hęnadę́hnǫdrǫˀ\\
\gll de̱-hęn-adę́-hnǫd-rǫ-ˀ\\
 {\dualic}-\textsc{3ns.m.a}-{\semireflexive}-follow-{\distributive-\stative}\\
\glt `they are brothers'
\z
\z

\ea\label{ex:distrex6} \stem{-srǫ-:-ˀ} {\distributive}-{\punctual} and \stem{-srǫ:} {\distributive}.{\noaspect} with \textsc{three aspect} verbs
\ea ęhsrihwanéˀaksrǫ:ˀ\\
\gll ę-hs-rihw-a-néˀak-srǫ-:ˀ\\
\fut-\textsc{2s.a}-matter-{\joinerA}-sin-{\distributive}-{\punctual}\\
\glt ‘you will swear, use profane language'

\cfex{ęgrihwané:ˀa:k\\
	\gll ę-g-rihw-a-né:ˀa:k\\
	\fut-\textsc{1s.a}-matter-{\joinerA}-sin.{\zeropunctual}\\
	\glt `I will sin'}

\newpage
\ex desahjiyóhsrǫ:\\
\gll de-s-ahj-iyó-hsrǫ:\\
 {\dualic}-\textsc{2s.a}-hand-good-{\distributive}.{\noaspect}\\
\glt `you really scratch!'

\cfex{desahjí:yo:\\
	\gll de-s-ahj-í:yo:\\
	{\dualic}-\textsc{2s.a}-hand-good.{\noaspect}\\
	\glt `you scratch!'}
\z
\z

The \stem{-(i)nyǫ-ˀse} and \stem{-ǫ-nyǫ} double distributive suffixes appear to extend the time frame of a state of affairs.

\ea\label{ex:innerpl2} \stem{verb-inyǫ-ˀse-:} verb-{\distributive}-{\distributive}-{\stative}\\
agatowínyǫ̱ˀse:\\
\gll ag-atow-inyǫ-ˀse-:\\
 \textsc{1s.p}-cold-{\distributive}-{\distributive}-{\stative}\\
\glt `I have a cold'
\z


\ea\label{ex:innerpl3} \stem{verb-ǫ-nyǫ-ˀs} {\distributive}-{\distributive}-{\plural}\\
adwagagyáǫnyǫˀs \\
\gll ad-wag-ag-ya-ǫ-nyǫ-ˀs\\
 {\factualdualic}\exsc{-1s.p}-{\semireflexive}-otherness-{\distributive}-{\distributive}-{\plural}\\
\glt ‘I had a vision’, `I got spooked'
\z


Tentatively speaking, the distributives belong to the \textsc{h3} conjugation class (\ref{ex:distrex7}, \sectref{Aspect conjugation classes}): as seen in the above examples, the word-final distributive suffixes end with \stem{…ˀ} for stative verbs\footnote{Recall that the \stem{-ˀse} {\distributive} suffix in example \xref{ex:innerpl2} takes a \stem{-:} {\stative} suffix.}, with \stem{…:ˀ} {\punctual} for punctual verbs, and with \stem{-:} \exsc{\noaspect} for imperative or no-aspect verbs. 

\ea\label{ex:distrex7} aspect suffixes and distributives
\ea \textsc{stative} \stem{-distr-ˀ} {\distributive}-{\stative}, e.g. \stem{…ǫ-ˀ}
\ex \textsc{punctual} \stem{-distr-:ˀ} {\distributive}-{\punctual}, e.g. \stem{…ǫ-:ˀ}
\ex \textsc{no-aspect} \stem{-distr.:} {\distributive}.{\noaspect}, e.g. \stem{…ǫ:}
\z
\z






\section{Role-adding suffixes} \label{Role-adding suffixes}
The suffixes described next share the function of adding a \textsc{role} (\sectref{Pronominal prefixes and role}) to the verb they modify: \textsc{causative} suffixes add a “causer”, and \textsc{benefactive} suffixes add a beneficiary.


\subsection{\textsc{Causative} (\textsc{caus}) suffixes (\stem{-hd}, \stem{-ˀd}, and \stem{-d})} \label{Causative suffixes}
The three causative suffixes are \stem{-hd}, \stem{-ˀd}, and \stem{-d}. They mean ‘to cause someone to do something’ or ‘to make someone do something’ and add the new role of “causer” to the verb. To illustrate, in \xref{ex:causexa}, \stem{o-tgiˀ} has just one role (‘it’, referring to the item that is dirty). In contrast, the same verb with a causative suffix \xref{ex:causexb} adds a role for the “causer” (the person dirtying things up, in this example). Examples of each of the causative suffixes are provided in the following sections.

\ea\label{ex:causex}
\ea otgiˀ\\\label{ex:causexa}
\gll o-tgi-ˀ\\
 \textsc{3s.p}-dirty-{\stative}\\
\glt `it is dirty'
\ex ęhsé:tgiht \\\label{ex:causexb}
\gll ę-hs-e-tgi-ht\\
 \fut-\textsc{2s.a}-{\joinerE}-dirty-\exsc{cause}.{\zeropunctual}\\
\glt ‘you will dirty it up’, `you will cause it to be dirty'
\z
\z


\subsubsection{\stem{-hd} \textsc{causative}} \label{[-ht, -hd] (causative)}
The verbs in \xref{ex:causex2} take the \stem{-hd} {\causative}. To illustrate the added “causer” role, the examples below include pairs of related verbs, ones with and without the causative.

\ea\label{ex:causex2} \stem{verb-{\causative}-\exsc{\aspect}}
\ea gadenya̱ˀgwáhtaˀ \\
\gll g-ad-e-nya̱ˀgw-á-ht-haˀ\\
 \exsc{1s.a-{\semireflexive}}-{\joinerE}-vomit-{\joinerA}-{\causative}-{\habitual}\\
\glt ‘I make or cause myself to vomit’, `I am bulemic'

\cfex{agénya̱ˀgwahs \\
	\gll ag-é-nya̱ˀgw-a-hs\\
	\textsc{1s.p}-{\joinerE}-vomit-{\joinerA}-{\habitual}\\
	\glt ‘I am a vomiter’, `I vomit all the time'}

\ex ohdaht \\
\gll o-hd-a-ht\\
\textsc{3s.p}-fill.up-{\causativestative}\\
\glt ‘it is filling’ (or, ‘it makes someone get full’)

\cfex{aˀóhdaˀ\\
	\gll aˀ-ó-hd-aˀ\\
	{\factual}-\textsc{3s.p}-fill.up-{\punctual}\\
	\glt `it got full'}

\ex onéhagwaht\\
\gll o-néhagw-a-ht\\
\textsc{3s.p}-amazed-{\joinerA}-\exsc{\causativestative}\\
\glt ‘it is amazing’ (or ‘it causes people to be amazed’)\\


\cfex{ǫknéhagoˀ\\
	\gll ǫk-néhago-ˀ\\
	\exsc{\factual.1s.p}-amazed-{\punctual}\\
	\glt `I was amazed'}

\ex ohsgá:naht, ohsgá:neht\\
\glt‘it is enticing, alluring, attractive’ (or, ‘it causes longings’)\\
\gll o-hsgá:n-a/hsgáne-ht\\
\textsc{3s.p}-long.for-(\joinerA)-{\causativestative} {}\\

\cfex{hahsgá:ne:s\\
	\gll ha-hsgá:ne:-s\\
	\textsc{3s.m.a}-long.for-{\habitual}\\
	\glt `he has longings for'}

\ex hodrihwatgíhdǫh \\
\gll ho-d-rihw-a-tgí-hd-ǫh\\
 \textsc{3s.m.p}-{\semireflexive}-matter-{\joinerA}-\exsc{ugly-\causative}-{\stative}\\
\glt ‘he is talking dirty’, `he is making dirty words'

\cfex{oíhwatgiˀ\\
	\gll o-íhw-a-tgi-ˀ\\
	\textsc{3s.p}-matter-{\joinerA}-ugly-{\stative}\\
	\glt `dirty language'}

\ex haˀwá:tsˀaht\\
\gll h-aˀ-w-á:tsˀ-a-ht\\
{\translocative}-{\factual}-\textsc{3s.a}-use.up-{\joinerA}-{\causative}.{\zeropunctual}\\
\glt ‘it is all gone’ (or, ‘something made it go down to nothing’)\\

\cfex{haˀwá:tsˀa:ˀ\\
	\gll h-aˀ-w-á:tsˀ-a-:ˀ\\
	{\translocative}-{\factual}-\textsc{3s.a}-use.up-{\joinerA}-{\punctual}\\
	\glt `it went down to nothing'}
\z
\z

Verbs with the \stem{-hd, -ht} endings belong to the \textsc{ma2} conjugation class (\sectref{Aspect conjugation classes}), and take the following aspect suffixes \xref{ex:causex3}.

\ea\label{ex:causex3}
\ea \textsc{habitual} \phonet{-htaˀ}, consisting of \stem{-hd-haˀ} {\causative}-{\habitual}
\ex \textsc{punctual} \phonet{-ht}, consisting of \stem{hd.ø} {\causative}.{\zeropunctual}
\ex \textsc{stative} \phonet{hd-ǫh} {\causative}-{\stative}
\ex \textsc{stative} \phonet{ht}, consisting of \stem{hd.ø} {\causative}.{\stative}
\ex \textsc{no-aspect} \phonet{ht}, consisting of \stem{hd.ø} {\causative}.{\noaspect}
\z
\z


\subsubsection{\stem{-ˀd} \textsc{causative}} \label{[-ˀt, -ˀd] (causative)}
The verbs in \xref{ex:causex4} take the \stem{-ˀt, -ˀd} {\causative} suffix. To illustrate the added “causer” role, the examples below include pairs of related verbs, ones with and without the \stem{-ˀt, -ˀd} {\causative} suffix.

\ea\label{ex:causex4} \stem{verb-{\causative}-\exsc{\aspect}}
\ea ęhshenǫgéhaˀt \\
\gll ę-hshe-nǫg-éh-a-ˀt\\
\fut-\textsc{2s:3fi}-milk-drink-{\joinerA}-{\causative}.{\zeropunctual}\\
\glt ‘you will breast-feed someone’ (or ‘cause someone to drink milk’)

\cfex{ęknǫ́ˀge̱haˀ\\
	\gll ę-k-nǫ́ˀg-e̱h-aˀ\\
	\fut-\textsc{1s.a}-milk-drink-{\punctual}\\
	\glt `I will drink milk'}

\ex ogę́hęˀt\\
\gll o-gę́hę-ˀt\\
\textsc{3s.p}-mean-{\causativestative}\\
\glt ‘it is disgusting’ (or, ‘it causes disgust in people’)

\cfex{shegę́hęnih \\
	\gll she-gę́hę-ni-h\\
	\textsc{2s:3fi}-mean-{\benefactive}-{\habitual}\\
	\glt ‘you abuse people’, `you are mean to someone'}

\ex ęhsheya̱hdó:gaˀt\\
\gll ę-hshey-a̱hdó:g-a-ˀt\\
\fut-\textsc{2s:3fi}-mature-{\joinerA}-{\causative}.{\zeropunctual}\\
\glt ‘you will raise someone’  (or ‘you will cause someone to grow’)

\cfex{ęhsáhdo:k\\
	\gll ę-hs-áhdo:k\\
	\fut-\textsc{2s.a}-mature.{\zeropunctual}\\
	\glt `you will grow'}

\ex ęhsgáeyǫˀt\\
\gll ę-hs-gáe-yǫ-ˀt\\
\fut-{\repetitive}-\textsc{3ns.fi.a}-arrive-{\causative}.{\zeropunctual}\\
\glt ‘they will bring it back’ (or, ‘they will cause it to return’)

\cfex{sagáeyǫˀ\\
	\gll s-a-gáe-yǫ-ˀ\\
	{\repetitive}-{\factual}-\textsc{3ns.fi.a}-arrive-{\punctual}\\
	\glt `they returned'}
\z
\z

Some verbs appear to have the same meaning either with or without a \linebreak \stem{-ˀt, -ˀd} {\causative} \xref{ex:causex5}. However, there may be a subtle difference in meaning: possibly, the ones with the \stem{-ˀt, -ˀd} {\causative} are \emph{caused} or \emph{done on purpose}, while the verbs without the ending denote an \emph{unintentional action}.

\ea\label{ex:causex5}
\ea ęhsáhdǫˀt \\
\gll ę-hs-áhdǫ-ˀt\\
\fut-\textsc{2s.a}-lose-{\causative}.{\zeropunctual}\\
\glt ‘you will lose it’ (on purpose)

\ex ęhsáhdǫ:ˀ \\
\gll ę-hs-áhdǫ-:ˀ\\
\fut-\textsc{2s.a}-lose-{\punctual}\\
\glt ‘you will lose it’ (accidentally)
\z
\z

For some verbs, the \stem{-ˀt, -ˀd} {\causative} suffix changes an “adjectival” (stative-only) verb into an action verb \xref{ex:causex6}.

\ea\label{ex:causex6}
\ea ęhsnę́hętaˀt \\
\gll ę-hs-nę́hę-t-a-ˀt\\
{\future}-\textsc{2s.a}-corn-dry-{\joinerA}-{\causative}.{\zeropunctual}\\
\glt ‘you will dry corn’ (or, ‘you will cause corn to dry’)

\cfex{ohéhdatę:\\
	\gll o-héhd-a-t-ę:\\
	\textsc{3s.p}-earth-{\joinerA}-dry-{\stative}\\
	\glt `dry dirt'}

\ex otsę́nǫnyaˀt \\
\gll o-tsę́nǫny-a-ˀt\\
\textsc{3s.p}-happy-{\joinerA}-\exsc{\causative.\stative}\\
\glt ‘gratefulness, thankfulness, joy’ (or ‘it causes happiness’)

\cfex{agatsęnǫ́:ni:\\
	\gll ag-atsęnǫ́:ni-:\\
	\textsc{1s.p}-happy-{\stative}\\
	\glt `I am happy'}
\ex deyeihwagenháˀtaˀ\\
\gll de-ye-ihw-a-genh-á-ˀt-haˀ\\
{\dualic}-\textsc{3s.fi.a}-matter-{\joinerA}-argue-{\joinerA}-{\causative}-{\habitual}\\
\glt ‘that’s debatable’ (or, ‘someone causes arguments’) 

\cfex{degaihwágenhęh\\
	\gll de-ga-ihw-á-genh-ęh\\
	{\dualic}-\textsc{3s.a}-matter-{\joinerA}-argue-{\stative}\\
	\glt `there is an argument'}
\z
\z

Three-aspect verbs with the \stem{-ˀd, -ˀt} {\causative} belong to the \textsc{ma2} conjugation class (\sectref{Aspect conjugation classes}), and take the following aspect suffixes \xref{ex:causex7}.

\ea\label{ex:causex7}
\ea \textsc{habitual} \stem{-ˀtaˀ}, consisting of [-ˀd-haˀ] {\causative}-{\habitual}
\ex \textsc{punctual} \stem{-ˀt}, consisting of [ˀd.ø] {\causative}.{\zeropunctual}
\ex \textsc{stative} \stem{ˀd-ǫh} {\causative}-{\stative}
\ex \textsc{stative} \stem{ˀt}, consisting of [ˀd.ø] {\causative}.{\stative}
\ex \textsc{no-aspect} \stem{ˀt}, consisting of [ˀd.ø] {\causative}-{\noaspect}
\z
\z


\subsubsection{\stem{-d} \textsc{causative}} \label{[-t, -d] (causative)}
The following verb takes the \stem{-d} {\causative} suffix \xref{ex:causex8}.

\ea\label{ex:causex8}
haǫhwęjádadǫh\\
\gll ha-ǫhwęˀ-a-d-a-d-ǫh \\
\textsc{3s.m.a}-earth-{\joinerA}-stand-{\joinerA}-{\causative}-{\stative}\\
\glt ‘he created the earth’ (he caused the earth to exist)
\citep{foster_course_1993}

\cfex{ohwę́jadeˀ\\
	\gll o-hwęj-a-deˀ\\
	\textsc{3s.p}-earth-{\joinerA}-exist. {\stative}\\
	\glt `existing earth'}
\z


\subsection{\textsc{Benefactive} (\benefactive) suffixes (\stem{-hs} and \stem{-ę, -nih})} \label{Benefactive suffixes}
The family of \textsc{benefactive} (also called \textsc{dative}) suffixes add the meaning of ‘doing something for someone’s benefit’. (In other words, they add a new role for the beneficiary of the verb’s action.) For example, in \xref{ex:benefexa}, \cayuga{age:kǫ́:niˀ} expresses just two roles -- the “doer” (the speaker, in this case), and the object of the action (‘a meal, food’). In contrast, the same verb with a benefactive suffix \xref{ex:benefexb} conveys three roles, the “doer”, the object, and the beneficiary.

\ea\label{ex:benefex}
\ea age:kǫ́:niˀ\\\label{ex:benefexa}
\gll a-g-e:-k-ǫ́:ni-ˀ\\
 {\factual}-\textsc{1s.a}-{\joinerE}-food-make-{\punctual}\\
\glt `I cooked a meal'
\ex ahékǫnyęˀ\\\label{ex:benefexb}
\gll a-hé-k-ǫny-ę-ˀ\\
 {\factual}-\exsc{1s>3s.m}-food-make-{\benefactive}-{\punctual}\\
\glt `I cooked a meal \emph{for him}'
\z
\z

Verbs with a benefactive suffix take an interactive pronominal prefix, such as \stem{he-} \textsc{1s>3s.m} in \xref{ex:benefexb}, to express both the “doer” and the beneficiary. However, when the “doer” and beneficiary are the same person, the verb instead takes the \stem{-adad} {\semireflexive} prefix to denote that the “doer” and beneficiary are the same \xref{ex:benefex2}.

\ea\label{ex:benefex2}
ęhsadadrohé:gęˀ\\
\gll ę-hs-adad-rohé:g-ę-ˀ\\
 \fut-\textsc{2s.a}-{\reflexive}-gather-{\benefactive}-{\punctual}\\
\glt `you will accumulate (things, ideas, etc.) for yourself'
\cfex{ęhsróhe:k\\
	\gll ę-hs-róhe:k\\
	{\future}-\textsc{2s.a}-gather.{\zeropunctual}\\
	\glt `you will gather'}
\z


The benefactive suffixes include \stem{-hs}, and the \stem{-ę, -nih} family of suffixes. Examples are provided in the following sections.

\subsubsection{\stem{-hs} \textsc{benefactive}} \label{[-hs] (benefactive)}
The following verbs take the \stem{-hs} \textsc{\benefactive} suffix \xref{ex:benefex3}.


\ea\label{ex:benefex3}
\ea gǫ̱hyádǫ̱hahs\\
\gll gǫ̱-hyádǫ̱-h-a-hs\\
 \textsc{1s>2s}-write-\textsc{euph.h}-{\joinerA}-\exsc{\benefactive}.{\noaspect} \\
\glt `let me write for you' (\cite{mithun_watewayestanih_1984})
\cfex{kyá:dǫ:\\
	\gll k-hyá:dǫ:\\
	\textsc{1s.a}-write.{\noaspect}\\
	\glt `let me write'}
\ex hǫwayętwáhse:h\\
\gll hǫwa-yętw-á-hs-e:h\\
\textsc{3ms/3fis:3ms}-plant-{\joinerA}-{\benefactive}-{\stative}\\
\glt ‘he has planted for him’
(\cite{foster_course_1993})\\

\cfex{hoyę́:twęh\\
	\gll ho-yę́:tw-ęh\\
	\textsc{3s.m.p}-plant-{\stative}\\
	\glt `he has planted'}
\ex haˀhohahóˀktahs\\
\gll h-aˀ-ho-hah-óˀkd-a-hs\\
 {\translocative}-{\factual}-\textsc{3s.m.p}-road-end-{\joinerA}-{\benefactive}.{\zeropunctual}\\
\glt `the road ran out for him'
\cfex{haˀgahahóˀkdęˀ\\
	\gll h-aˀ-g-ahah-óˀkd-ęˀ\\
	{\translocative}-{\factual}-\textsc{1s.a}-road-end-{\punctual}\\
	\glt `I finished a row'}
\ex ęhsadadrá:gwahs\\
\gll ę-hs-adad-rá:gw-a-hs\\
 \fut-\textsc{2s.a}-{\reflexive}-choose-{\joinerA}-{\benefactive}.{\zeropunctual}\\
\glt `you will choose for yourself'
\cfex{ęhsrá:goˀ\\
	\gll ę-hs-rá:go-ˀ\\
	{\future}-\textsc{2s.a}-choose-{\punctual}\\
	\glt `you will choose or take out'}
\ex deshagodihwagé:nhahs\\
\gll de-shagodi-ihw-a-gé:nh-a-hs\\
 {\dualic}-\exsc{\textsc{3ns(nfi):3fi}}-matter-{\joinerA}-argue.for-{\joinerA}-\exsc{\benefactive.\habitual}\\
\glt `they argue for someone, people'

\cfex{a:hyayaˀdagénhaˀ\\
	\gll a:-hya-yaˀd-a-génh-aˀ\\
	{\indefinite}-\exsc{3ms:1s}-body-{\joinerA}-argue.for-{\punctual}\\
	\glt `he would help you'}
\z
\z

\subsubsection{\stem{-ę, -ni} \textsc{benefactive}} \label{[-ę, -ni] (benefactive)}
The following verbs take the \stem{-ę, -ni} family of benefactive suffixes \xref{ex:benefex5}.

\ea\label{ex:benefex5}
\ea hǫwayętwę́:nih\\
\gll hǫwa-yętw-ę́:-ni-h\\
 \textsc{3ms/3fis:3ms}-plant-{\joiner}-{\benefactive}-{\habitual} (Foster, 1993)\\
\glt `he plants for him'
\cfex{hoyę́:twęh\\
	\gll ho-yę́:tw-ęh\\
	\textsc{3s.m.p}-plant-{\stative}\\
	\glt `he planted'}
\ex agegę̱hę́ˀdanih\\
\gll ag-e-gę̱hę́ˀd-a-ni-h\\
 \exsc{ 1s.p}-{\joinerE}-sick.of-{\joinerA}-{\benefactive}-{\habitual}\\
\glt `I am annoyed, sick of it'
\cfex{ahsgegę̱hę́ˀdęˀ\\
	\gll a-hsg-e-gę̱hę́ˀd-ę-ˀ\\
	{\factual}-\exsc{2s:1s}-sick.of-{\benefactive}-{\punctual}\\
	\glt `you got sick of me', `you are bored with me'}
\ex ahékǫnyęˀ\\
\gll a-hé-k-ǫny-ę-ˀ\\
 {\factual}-\exsc{1s:3ms}-food-make-{\benefactive}-{\punctual}\\
\glt `I cooked a meal for him'
\cfex{age:kǫ́:niˀ\\
	\gll a-g-e:-k-ǫ́:ni-ˀ\\
	{\factual}-\textsc{1s.a}-{\joinerE}-food-make-{\punctual} (Foster, 1993)\\
	\glt `I cooked a meal'}
\ex ashagonǫhsǫnyęˀ\\
\gll a-shago-nǫhs-ǫny-ę-ˀ\\
 {\factual}-\exsc{3ms:3fi/3p.p}-house-make-{\benefactive}-{\punctual}\\
\glt `he built her a house'
\cfex{ashagonǫhsǫ́:niˀ\\
    \gll a-shago-nǫhs-ǫ́:ni-ˀ\\
    {\factual}-\exsc{3ms:3fi/3p.p}-house-make-{\punctual}\\
    \glt ‘he built a house for her’ \citep{foster_course_1993}
    }
\ex ahehswa̱ˀné:dęˀ\\
\gll a-he-hswa̱ˀn-é:d-ę-ˀ\\
 {\factual}-\exsc{1s:3ms}-upper.back-stand-{\benefactive}-{\punctual}\\
\glt `I supported him'
\cfex{hehswáˀne:t\\
	\gll he-hswáˀn-e:t\\
	\textsc{1s:3ms}-upper.back-stand.{\stative}\\
	\glt `I support or back him'}
\ex ahadadrihwagwé:nyęˀ\\
\gll a-ha-dad-rihw-a-gwé:ny-ę-ˀ\\
 {\factual}-\textsc{3s.m.a}-{\reflexive}-matter-{\joinerA}-succeed-{\benefactive}-{\punctual}\\
\glt `he had earned it for himself'
\cfex{ahsadadrihwagwé:nyaˀs\\
	\gll a-hs-adad-rihw-a-gwé:ny-a-ˀs\\
	{\factual}-\textsc{2s.a}-{\reflexive}-matter-{\joinerA}-succeed-{\joinerA}-{\plural}.{\zeropunctual}\\
	\glt `you are a success'}
\ex ęhshewayę́:sdęˀ\\
\gll ę-hshe-wayę́:sd-ę-ˀ\\
 \fut-\textsc{2s:3fi}-learn-{\benefactive}-{\punctual}\\
\glt `you will train, educate, teach someone'
\cfex{ęhsadewá:yę:s\\
	\gll ę-hs-ad-e-wá:yę:s\\
	{\future}-\exsc{2s.a-{\semireflexive}}-{\joinerE}-learn.{\zeropunctual}\\
	\glt `you will learn'}
\ex ęhsgǫganyáˀgęˀ\\
\gll ę-hs-gǫ-ganyáˀg-ę-ˀ\\
 \fut-{\repetitive}-\exsc{1s:2s}-pay-{\benefactive}-{\punctual}\\
\glt `I will repay you for your actions'
\cfex{ęhséganyaˀk\\
	\gll ę-hs-é-ganyaˀk\\
	{\future}-\textsc{2s.a}-{\joinerE}-pay.{\zeropunctual}\\
	\glt `you will pay'}
\z
\z

For the \stem{-ę, -ni} family of suffixes, \stem{-ni} is used in habitual- and stative-aspect verbs, and \stem{-ę}, in punctual-aspect verbs. The aspect endings shown in \xref{ex:benefex7} and \xref{ex:benefex8} belong to the \textsc{s1} conjugation class (\sectref{Aspect conjugation classes}).

\ea\label{ex:benefex7}
\ea \textsc{habitual} \stem{-ni-h} {\benefactive}-{\habitual}
\ex \textsc{punctual} \stem{-ę-ˀ} {\benefactive}-{\punctual}
\ex \textsc{stative} \stem{-ni-:} {\benefactive}-{\stative}
\ex \textsc{no-aspect} \stem{-ę-h} {\benefactive}-\textsc{euph.h}/{\noaspect}
\z
\z


\ea\label{ex:benefex8}
\ea ǫdadri̱hǫ́nyanih\\
\gll ǫ-dad-ri̱h-ǫ́ny-a-ni-h\\
 \textsc{3s.fi.a}-{\reflexive}-matter-make-{\joinerA}-{\benefactive}-{\habitual}\\
\glt `she is reading'
\ex ęhsadadríhǫnyęˀ\\
\gll ę-hs-adad-ríh-ǫny-ę-ˀ\\
 \fut-\textsc{2s.a}-{\reflexive}-matter-make-{\benefactive}-{\punctual}\\
\glt `you will read'
\ex wadadri̱hǫ́nyani:\\
\gll w-adad-ri̱h-ǫ́ny-a-ni-:\\
 \textsc{3s.a}\exsc{}-{\reflexive}-matter-make-{\joinerA}-{\benefactive}-{\stative}\\
\glt `reading material'
\ex sadadri̱hǫ́:nyęh\\
\gll s-adad-ri̱h-ǫ́:ny-ę-h\\
 \textsc{2s.a}-{\reflexive}-matter-make-{\benefactive}-\textsc{euph.h}/{\noaspect}\\
\glt `read!'
\z
\z

\section{Change-of-state suffixes} \label{Change-of-state suffixes}

The change-of-state suffixes are added to “adjectival” (stative-only) verbs, which describe states or qualities, and to “positional” (stative-only) verbs, which describe an object’s position. The result is a three-aspect verb denoting a change in state or change in position. (Also see \sectref{Potential change-of-state suffixes}.)

\subsection{\stem{-hsd} \textsc{causative-instrumental} (\textsc{caus}-\instrumental)} \label{[-hsd] (causative-instrumental)}
The \stem{-hsd} {\causative}-{\instrumental} suffix can be added to stative-only verbs, which become three-aspect verbs as a result. The \stem{-hsd} suffix means ‘become’ or ‘get’ \xref{ex:causinex}.\footnote{The name of this suffix, specifically the “instrumental” part, is possibly because of homophony with noun stem \stem{hsd} ‘tool, instrument’ and the verb \stem{hsd} ‘to use’.}
 
\ea\label{ex:causinex}
\ea owáno̱hsdǫh\\
\gll o-wá-no̱-hsd-ǫh\\
 \textsc{3s.p}-air-cold-{\cause}-\exsc{\stative}\\
\glt `it (weather) got cold'
\cfex{owá:no:ˀ\\
	\gll o-wá:-no-:ˀ\\
	\textsc{3s.p}-air-cold-{\stative}\\
	\glt `it (weather) is cold'}
\ex sagána̱ˀnohs\\
\gll s-a-gá-na̱ˀ-no-hs\\
 {\repetitive}-{\factual}-\textsc{3s.a}-object-cold-{\causative}.{\zeropunctual}\\
\glt `it (an object) became cold'
\cfex{onáˀno:ˀ\\
	\gll o-náˀ-no-:ˀ\\
	\textsc{3s.p}-object-cold-{\stative}\\
	\glt `it (an object) is cold, cool'}
\ex sawęhni̱srí:yohs\\
\gll s-a-w-ęhni̱sr-í:yo-hs\\
 {\repetitive}-{\factual}-\textsc{3s.a}-day-good-{\causative}.{\zeropunctual}\\
\glt `it became a nice day again'
\cfex{wę̱hnihsrí:yo:\\
	\gll w-ę̱hnihsr-í:yo:\\
	\textsc{3s.a}-day-good.{\stative}\\
	\glt `it is a nice day'}
\ex gakwáˀ ganáˀno̱hstaˀ\\
\gll gakwáˀ ga-náˀ-no̱-hst-haˀ\\
 food \textsc{3s.a}-object-cold-{\cause}-{\habitual}\\
\glt ‘refrigerator’
\z
\z

Verbs with the \stem{-hsd} ending belong to the \textsc{ma1} conjugation class (\ref{ex:causinex3}, \sectref{Aspect conjugation classes}).

\ea\label{ex:causinex3} 
\ea \textsc{habitual} \phonet{-hstaˀ}, consisting of \stem{hsd-haˀ} {\causative}-{\habitual}
\ex \textsc{punctual} \phonet{-hs}, consisting of \stem{hsd-ø} {\causative}-{\zeropunctual}
\ex \textsc{stative} \phonet{hsd-ǫh} {\causative}-{\stative}
\ex \textsc{no-aspect} \phonet{hs} consisting of \stem{hsd-ø} {\causative}-{\noaspect}
\z
\z




\subsection{\stem{-(ę)ˀ} \textsc{inchoative} (\textsc{inch})} \label{[-ˀ] (inchoative)}

The \stem{-(ę)ˀ} \textsc{inch} is added to stative-only verbs (specifically, the thematic sub-class of positional, stative-only verbs), to create three-aspect, punctual forms. Examples \xxref{ex:inchex}{ex:inchex4}, illustrate pairs of stative-only verbs and their inchoative/punctual counterparts.\footnote{Apart from example \xref{ex:inchex2}, the inchoative suffix in the remaining examples appears to be homophonous with the \stem{-(ę)ˀ} {\punctual} (and so has been glossed as \textsc{inch.\zeropunctual} in the examples). Possibly, the \stem{-(ę)ˀ} \textsc{inch} could simply be reanalysed as a \stem{-(ę)ˀ} {\punctual} suffix, in which case, it would have to be the case that the punctual suffix could occasionally appear without a mood ({\future}, {\factual} or {\indefinite}) prepronominal prefix, to impart an inchoative meaning.}


\ea\label{ex:inchex} \stem{-d} ‘stand’, \stem{daˀ} ‘stand up’\\
atá:daˀ\\
\gll at-há:-d-a-ˀ\\
 {\factual-\dualic}-\textsc{3s.m.a}-stand-{\joinerA}-\textsc{inch}.{\zeropunctual}\\
\glt `he stood up'
\cfex{ita:t\\
	\gll i-t-ha:-t\\
	\textsc{proth-\dualic-3s.m.a}-stand.{\stative}\\
	\glt `he is standing there'}
\z 


\ea\label{ex:inchex2} \stem{od} ‘stand’, \stem{odęˀ} ‘attach/put on for someone’\\
snętsó:dęˀ\\
\gll s-nęts-ó:d-ęˀ\\
 \textsc{2s.a}-arm-stand-{\benefactive}-\textsc{inch}\\
\glt `put on the arm!' (i.e. attach a doll's arm)
\cfex{snę́:tso:t\\
	\gll s-nę́:ts-o:t \\
	\textsc{2s.a}-arm-stand.{\stative}\\ 
	\glt `your attached arm'}
\z 

\newpage
\ea\label{ex:inchex3} \stem{od} ‘stand’, \stem{odaˀ} ‘come to stand’\\
ęyagonǫhsó:daˀ\\
\gll ę-yago-nǫhs-ó:d-a-ˀ\\
  {\future}-\textsc{3s.fi.p}-house-stand-{\joinerA}-\textsc{inch}.{\zeropunctual}\\
\glt `she will come to have a (standing) house'
\cfex{ganǫ́hsotge:\\
	\gll ga-nǫ́hs-ot-ge: \\
	\textsc{3s.a}-house-stand.{\stative-\augmentative} \\ 
	\glt `a standing big house'}
\z 

\ea\label{ex:inchex31} \stem{+ǫd} ‘attached’, \stem{+ǫdęˀ} ‘get attached’ \\
agagwiyǫ́:dęˀ\\
\gll a-ga-gwiy-ǫ́:d-ęˀ\\
    {\factual-3s.a}-bud-attached-\textsc{inch.\zeropunctual}\\
\glt `it got buds'
\cfex{ogwí:yǫ:t\\
	\gll o-gwí:y-ǫ:t \\
	 \textsc{3s.p}-bud-attached.{\stative}\\ 
	\glt `(attached) buds'}
\z 


\ea\label{ex:inchex4} \stem{+ęˀ} ‘have, own’, \stem{+ędaˀ} ‘come to have, own’  \\
ęyagowiyáędaˀ\\
\gll ę-yago-wiy-á-ę-d-a-ˀ\\
    \textsc{\future-3s.fi.p}-child-have-\textsc{euph.d-\joinerA-inch.\zeropunctual} \\
\glt `she will have a baby'
\cfex{dewagwiyáęˀ\\
	\gll de-wag-wiy-á-ęˀ \\
	  \textsc{\negative-1s.p}-child-{\joinerA}-have.{\stative}\\ 
	\glt `I don't have a child'}
\z



\subsection{\stem{-(n)heˀ} \textsc{inchoative} (\textsc{inch})} \label{[-heˀ] (inchoative)}

The \stem{-(n)heˀ} \textsc{inch} suffix, illustrated in examples \xxref{ex:heinchex}{ex:heinchex2}, changes stative-only verbs into three-aspect verbs.

\ea\label{ex:heinchex}
\ea agówanheˀ\\
\gll a-gówan-heˀ\\
 {\factual}-\textsc{ø.prefix}.big-\textsc{inch}.{\zeropunctual}\\
\glt `it became big'
\cfex{gowá:nęh\\
	\gll gowá:n-ęh\\
	\textsc{ø.prefix}.big-{\stative}\\
	\glt `it is big'}
 \ex ahsehsęnowanheˀ\\
 \gll a-hs-e-hsęn-owan-heˀ\\
 {\factual}-\textsc{2s.a-\joinerE}-name-big-\textsc{inch}.\zeropunctual\\
 \glt `you became famous' (literally, `your name got big')
\z 
\z 

\ea\label{ex:heinchex2}
\ea awagyesę́ˀnheˀ\\
\gll a-wa-g-yes-ę́-nheˀ\\
{\factual}-\textsc{3s.a-\semireflexive}-easy-\textsc{\stative-inch}\\
\glt `it became easy'
\cfex{wagyé:sęh\\
\gll wa-g-yé:s-ęh\\
\textsc{3s.a-\semireflexive}-easy-{\stative}\\
\glt `it is easy'}
\z
\z

\section{Movement suffixes} \label{Movement suffixes}
Suffixes denoting various kinds of movement appear in slot 2 of the verb (see \tabref{figtab:1:verbsuffixorder}, page \pageref{figtab:1:verbsuffixorder}). The \textsc{dislocative} suffixes mean ‘go and do something’ and the \textsc{progressive} means ‘go along doing something’. These suffixes interact with the \textsc{purposive} aspect (see \textit{The purposive}, \sectref{ch:The purposive}). 

The \textsc{reversive} suffix does not behave the same way, but is included in this section because it expresses a type of movement, turning an activity or happening into its opposite. 



\subsection{\stem{-hs, -hn, -ˀn, -h-, -ˀh, -ˀdr} \textsc{dislocative} (\dislocative)} \label{Dislocative suffixes [-hs, -hn] etc}
The dislocative family of suffixes includes \stem{-hs, -hn, -ˀn, -h-, -ˀh, -ˀdr}. Dislocatives add the meaning of ‘going and doing something’. The choice of suffix likely has to be learned along with each verb. Examples of all of the dislocatives except for \stem{-ˀh} are shown in \xxref{ex:dislocendex}{ex:dislocendex5}.

\ea\label{ex:dislocendex} \stem{-ˀdr} {\dislocative}\\
sędáˀdrah\\
\gll s-ęd-á-ˀdr-a-h\\
 \textsc{2s.p}-sleep-{\joinerA}-{\dislocative}-{\joinerA}-\textsc{euph.h}/{\noaspect}\\
\glt `you go to bed, you go to sleep!'
\cfex{sę́:daˀ\\
	\gll s-ę́:daˀ\\
	\textsc{2s.p}-sleep.{\noaspect}\\
	\glt `you sleep!'}
\z


\ea\label{ex:dislocendex2} \stem{-ˀn} {\dislocative}\\
ęyǫdawę́ˀnaˀ\\
\gll ę-yǫ-d-awę́-ˀn-a-ˀ\\
 \fut-\textsc{3s.fi.a}-{\semireflexive}-swim-{\dislocative}-{\joinerA}-{\punctual}\\
\glt `she will go swimming'
\cfex{ęyǫ́dawę:ˀ\\
	\gll ę-yǫ́-d-awę-:ˀ\\
	{\future}-\textsc{3s.fi.a}-{\semireflexive}-swim-{\punctual}\\
	\glt `she will swim'}
\z


\ea\label{ex:dislocendex3} \stem{-hn} {\dislocative}
\ea ahęnadáhnyo̱hnaˀ\\
\gll a-hęn-ad-áhny-o̱-hn-a-ˀ\\
 {\factual}-\textsc{3ns.m.a}-{\semireflexive}-hook-submerge-{\dislocative}-{\joinerA}-{\punctual}\\
\glt `they went fishing'
\cfex{ahęnadáhnyo:ˀ\\
	\gll a-hęn-ad-áhny-o-:ˀ\\
	{\factual}-\textsc{3ns.m.a}-{\semireflexive}-hook-submerge-{\punctual}\\
	\glt `they (males) fished'}

\ex ęgékdǫ̱hnaˀ \\
\gll ę-g-é-kdǫ̱-hn-a-ˀ\\
\fut-\textsc{1s.a}-{\joinerE}-examine-{\dislocative}-{\joinerA}-{\punctual}\\
\glt ‘I will go see’

\cfex{ęgé:kdǫ:ˀ\\
	\gll ę-g-é:-kdǫ-:ˀ\\
	{\future}-\textsc{1s.a}-{\joinerE}-examine-{\punctual}\\
	\glt `I will see, look'}

\ex dwakdǫ́hnah\\
\gll dwa-kdǫ́-hn-a-h\\
 \textsc{1p.in.a}-examine-{\dislocative}-{\joinerA}-\textsc{euph.h}/{\noaspect}\\
\glt `let us go look'

\cfex{dwakdǫ:\\
	\gll dwa-kdǫ:\\
	\exsc{1p.in.a}-examine.{\noaspect}\\
	\glt `let us look'}

\ex a:gatrǫ́nya̱hnaˀ\\
\gll a:-g-at-hrǫ́ny-a̱-hn-a-ˀ\\
 {\indefinite}-\textsc{1s.a}-{\semireflexive}-tell-{\joinerA}-{\dislocative}-{\joinerA}-{\punctual}\\
\glt `I should go and tell'

\cfex{a:gatrǫ́:niˀ\\
	\gll a:-g-at-hrǫ́:ni-ˀ\\
	{\indefinite}-\textsc{1s.a}-{\semireflexive}-tell-{\punctual}\\
	\glt `I should tell'}

 \ex agatrǫnyáhnǫh\\
\gll ag-at-hrǫny-á-hn-ǫh\\
 \exsc{1s.p}-{\semireflexive}-tell-{\joinerA}-{\dislocative}-{\stative}\\
\glt `I am telling now'
\z
\z

\ea\label{ex:dislocendex4} \stem{-h } {\dislocative}
\ea ęhsahǫgáhaˀ\\
\gll ę-hs-ahǫg-á-h-a-ˀ\\
{\future}-\textsc{2s.a}-understand-{\joinerA}-{\dislocative}-{\joinerA}-{\punctual}\\
\glt ‘you’ll be invited or asked to go’

\cfex{ęhsáhǫ:k\\
	\gll ę-hs-áhǫ:k\\
	{\future}-\textsc{2s.a}-understand.{\zeropunctual}\\
	\glt `you will understand a language'}

\ex si̱hsá:kah\\
\gll s-i̱hsá:k-h-a-h\\
 \textsc{2s.a}-seek-{\dislocative}-{\joinerA}-\textsc{euph.h}/{\noaspect}\\
\glt `go and look for it!'


\cfex{si̱hsa:k\\
	\gll s-i̱hsa:k\\
	\textsc{2s.a}-seek.{\noaspect}\\
	\glt `look for it!'}
\z
\z

\ea\label{ex:dislocendex5} \stem{-hs} {\dislocative}
\ea ęhayętwáhsaˀ\\
\gll ę-ha-yętw-á-hs-a-ˀ\\
 \fut-\textsc{3s.m.a}-plant-{\joinerA}-{\dislocative}-{\joinerA}-{\punctual} (\cite{foster_course_1993}) \\
\glt `he will go planting'
\cfex{ęháyętoˀ\\
	\gll ę-há-yęto-ˀ\\
	{\future}-\textsc{3s.m.a}-plant-{\punctual}\\
	\glt `he will plant'}
\ex ęsyęˀgǫtwáhsaˀ\\
\gll ę-s-yęˀg-ǫtw-á-hs-a-ˀ\\
 \fut-\textsc{2s.a}-tobacco-burn.up-{\joinerA}-{\dislocative}-{\joinerA}-{\punctual}\\
\glt `you will go burn tobacco'
\cfex{ wadǫ́:twahs\\
\gll w-ad-ǫ́:tw-a-hs\\
\textsc{3s.a}-{\semireflexive}-burn.up-{\joinerA}-{\habitual}\\
\glt ‘(what) it burns’ (in the way of fuel) }
\ex agáhyaksǫ:\\
\gll ag-áhy-a-k-hs-ǫ:\\
 \textsc{1s.p}-fruit-{\joinerA}-eat-{\dislocative}-{\stative}\\
\glt `I am going and eating fruit'
\z
\z

Example \xref{ex:dislocendex6} illustrates the aspect endings for dislocative suffixes.

\ea\label{ex:dislocendex6} 
\ea \textsc{habitual} (no examples of habitual forms)
\ex \textsc{punctual} \stem{-\dislocative-a-ˀ} {\dislocative}-{\joinerA}-{\punctual}
\ex \textsc{stative} \stem{-\dislocative-ǫh} {\dislocative}-{\stative}
\ex \textsc{stative} \stem{-\dislocative-ǫ:} {\dislocative}-{\stative}
\ex \textsc{no-aspect} \stem{-\dislocative-a-h} {\dislocative}-{\joinerA}-\textsc{euph.h}/{\noaspect}
\z
\z


\subsection{Dislocative stative-aspect verbs with \stem{-hk} {\former} or \stem{-:k} {\modalizer} suffixes} \label{{Dislocative with [-ǫh, -ǫ:] stative suffixes}}
Dislocative, stative-aspect verbs can also take a \stem{-hk} {\former} suffix (\ref{ex:disstatex2}, see \sectref{Statives with [-gęhę:ˀ] past, [-hne:ˀ] remote, or [-hk] former}). The resulting verbs take \textsc{p}-series pronominal prefixes, as with the original stative verbs.


\ea\label{ex:disstatex2} \stem{\textsc{p}-verb-{\dislocative}-ǫh-hk} {\dislocative}-{\stative} with {\former}\\
agatrǫnyáhnǫhk\\
\gll ag-at-hrǫny-á-hn-ǫ-hk\\
\textsc{1s.p}-{\semireflexive}-tell-{\joinerA}-{\dislocative}-{\stative}-{\former}\\
\glt ‘I went there and told’ (but I am back)
\z


Dislocative, stative-aspect verbs can also take a \stem{-:k} {\modalizer} suffix, along with either an \stem{ę-} {\future} or \stem{a:-} \textsc{\indefinite} prefix \xxref{ex:disstatex3}{ex:disstatex4} (see \sectref{Statives with [-:k] modalizer}). The resulting verbs take \textsc{p}-series pronominal prefixes, as with the original stative verbs.

\ea\label{ex:disstatex3} \stem{ę-\textsc{p}-verb-{\dislocative}-ǫ-:k} {\dislocative}-{\stative} with {\modalizer}\\
ęwagatrǫ́nya̱hnǫ:k\\
\gll ę-wag-at-hrǫ́ny-a̱-hn-ǫ-:k\\
\fut-\textsc{1s.p}-{\semireflexive}-tell-{\joinerA}-{\dislocative}-{\stative}-{\modalizer}\\
\z


\ea\label{ex:disstatex4} \stem{a:-\textsc{p}-verb-{\dislocative}-ǫ-:k} {\dislocative}-{\stative} with {\modalizer}\\
a:wagatrǫnyáhnǫ:k\\
\gll a:-wag-at-hrǫny-á-hn-ǫ-:k\\
{\indefinite}-\textsc{1s.p}-{\semireflexive}-tell-{\joinerA}-{\dislocative}-{\stative}-{\modalizer}\\
\z


\subsubsection{\stem{+ędaˀdr-}, \stem{+odaˀn-}, \stem{+ǫdaˀn-} (positional verbs with {\dislocative} suffixes)} \label{[+ędaˀdr-], [+odaˀn-], [+ǫdaˀn-]}
Three positional (stative-only) verbs often take dislocative suffixes, and then combine with incorporated nouns to create new words. Their structure is shown in the following examples.\footnote{Example \xref{ex:disposex} illustrates an e-verb with \stem{e} ‘go’ (see \sectref{ch:Dislocative-[e] verbs}). The remaining examples are of dislocative verbs without this element.}

The verbs in \xxref{ex:disposex}{ex:disposex3} are part of fixed expressions (see \sectref{Fixed and transparent expressions}), and so the meaning varies with the incorporated noun.

\ea\label{ex:disposex} \stem{{}+ę-d-a-dr} lie-\textsc{euph.d}-{\joinerA}-\exsc{\dislocative} (dislocative form of \stem{+ęˀ} ‘be lying on the ground, have’)
\ea gowiyaę́da̱ˀdreˀ\\
\gll go-wiy-a-ę́d-a̱-ˀdr-e-ˀ\\
 \textsc{3s.fi.p}-offspring-{\joinerA}-lie-\textsc{euph.d}-{\joinerA}-{\dislocative}-\exsc{go-\punctual}\\
\glt `she is with child'

\cfex{agwíyaęˀ\\
	\gll ag-wíy-a-ęˀ\\
	\textsc{1s.p}-offspring-{\joinerA}-lie.{\stative}\\
	\glt `I have a child'}

\ex dęgaihwaędáˀdreˀ \\
\gll d-ę-ga-ihw-a-ęd-á-ˀdr-e-ˀ\\
 {\dualic}-{\future}-\textsc{3s.a}-matter-{\joinerA}-lie-\textsc{euph.d}-{\joinerA}-{\dislocative}-\exsc{go-\punctual}\\
\glt ‘harmony’, `every idea will always come together'

\cfex{dęgaihwáędaˀ\\
	\gll d-ę-ga-ihw-á-ęd-a-ˀ\\
	{\dualic}-{\future}-\textsc{3s.a}-matter-{\joinerA}-lie-\textsc{euph.d}-{\joinerA}-{\punctual}\\
	\glt `it will come to an agreement, a consensus'}
\z
\z

\ea\label{ex:disposex2} \stem{{}+od-a-ˀn} stand-{\joinerA}-\exsc{\dislocative} (dislocative form of \stem{+od} ‘be standing’)\\
ęgadręnóda̱ˀnaˀ\\
\gll ę-g-ad-ręn-ód-a̱-ˀn-a-ˀ\\
 \fut-\textsc{1s.a}-{\semireflexive}-song-stand-{\joinerA}-{\dislocative}-{\joinerA}-{\punctual}\\
\glt `I will go sing'

\cfex{ęgadręnó:dęˀ\\
	\gll ę-g-ad-ręn-ó:d-ęˀ\\
	{\future}-\textsc{1s.a}-{\semireflexive}-song-stand-{\punctual}\\
	\glt `I will sing'}
\z


\ea\label{ex:disposex3} \stem{+ǫd-a-ˀn-} attached-{\joinerA}-\exsc{\dislocative} (dislocative form of \stem{+ǫd} ‘be attached’, ‘have’)\\
ęgade̱ˀsgǫ́daˀnaˀ\\
\gll ę-g-ad-e̱-ˀsg-ǫ́d-a-ˀn-a-ˀ\\
 \fut-\exsc{1s.a-{\semireflexive}}-{\joinerE}-roast-attached-{\joinerA}-{\dislocative}-{\joinerA}-{\punctual}\\
\glt `I will go fry it'

\cfex{ęgade̱ˀsgǫ́:dęˀ\\
	\gll ę-g-ad-e̱-ˀsg-ǫ́:d-ęˀ\\
	{\future}-\exsc{1s.a-{\semireflexive}}-{\joinerE}-roast-attached-{\punctual}\\
	\glt `I will fry, roast it'}
\z


\subsection{\stem{-gy} \textsc{progressive} (\progressive)} \label{Progressive [-gy] suffix}
The \stem{-gy} {\progressive} suffix adds the meaning of ‘going along doing something’ or ‘progressing in doing something’. It is always followed by \stem{e-ˀ} ‘go’-{\stative} or \stem{e-:-ˀ} ‘go’-{\purposive}-{\stative} with the commonest combinations being \stem{-a-gy-e-ˀ}, \stem{-ę-gy-e-ˀ} and \stem{-ǫ-gy-e-ˀ}.

There are two types of progressive verbs: the first (possibly less common) type, illustrated in \xref{ex:dislprogex2}, consists of a verb followed directly by the \stem{-a-gy-e-ˀ} {\joinerA}-{\progressive}-go-{\stative} combination.

\ea\label{ex:dislprogex2} \stem{verb-a-gy-e-ˀ} {\joinerA}-{\progressive}-go-{\stative}
\ea gayę́twagyeˀ\\
\gll ga-yę́tw-a-gy-e-ˀ\\
 \textsc{3s.a}-plant-{\joinerA}-{\progressive}-go-{\stative}\\
\glt `it is planted along'
\cfex{gayę́:twęh\\
	\gll ga-yę́:tw-ęh\\
	\textsc{3s.a}-plant-{\stative}\\
	\glt `it is planted'}
\ex hǫwahnǫdrá:gyeˀ\\
\gll hǫwa-hnǫdr-á:-gy-e-ˀ\\
 \exsc{3ms/3fis:3ms}-follow-{\joinerA}-{\progressive}-go-{\stative}\\
\glt `someone is following him along'
\cfex{hǫwáhnǫdreˀ\\
	\gll hǫwá-hnǫdr-e-ˀ\\
	\exsc{3ms/3fis:3ms}-follow-go-{\stative}\\
	\glt `someone is following him'}
\ex haˀdehodinęhę́dagyeˀ\\
\gll haˀ-de-hodi-nęhę́d-a-gy-e-ˀ\\
 {\translocative}-{\dualic}-\textsc{3ns.m.p}-guard-{\joinerA}-{\progressive}-go-{\stative}\\
\glt `they (m) are guarding as they are on their way over, in transit'
\cfex{todinę̱hę́:deˀ\\
	\gll t-hodi-nę̱hę́:d-e-ˀ\\
	{\cislocative}-\textsc{3ns.m.p}-guard-go-{\stative}\\
	\glt  ‘they are standing there’ (in a line formation)’, `they are guarding'}
\ex ohsáwagyeˀ\\
\gll o-hsáw-a-gy-e-ˀ\\
 \textsc{3s.p}-begin-{\joinerA}-{\progressive}-go-{\stative}\\
\glt `it is beginning'
\cfex{ętsáhsawęˀ\\
	\gll ę-t-s-áhsaw-ęˀ\\
	{\future}-{\cislocative}-\textsc{2s.a}-begin-{\punctual}\\
	\glt `you will begin'}
\z
\z

The second (possibly more common) type consists of a verb with an \stem{-ǫ, -ę} {\stative} suffix followed by the \stem{-ǫ-gy-e-ˀ} combination \xref{ex:dislprogex3a} (or \stem{-ę-gy-e-ˀ}, not shown). This type of verb is described in \sectref{ch:Progressive verbs}.


\ea\label{ex:dislprogex3} \stem{verb-ǫh-ǫ-gy-e-ˀ} {\stative}-{\joinerA}-{\progressive}-go-\textsc{\stative}\\
otsihsˀǫ̱hǫ́:gyeˀ\\\label{ex:dislprogex3a}
\gll o-t-hs-ihsˀ-ǫ̱h-ǫ́:-gy-e-ˀ\\
 \textsc{3s.p}-{\semireflexive}-noun-finish-{\stative}-{\joinerA}-{\progressive}-go-{\stative}\\
\glt `it is getting mature'
\cfex{otsíhsˀǫh \\
\gll o-t-hs-íhsˀ-ǫh\\
 \exsc{3s.p-{\semireflexive}-noun}-finish-{\stative}\\
\glt ‘it is done for the season’, ‘it has gone full cycle’, ‘it is mature’, `they (plants) have finished out'}
\z


\subsection{\stem{-gw, -go}, \stem{-hsi} \textsc{reversive} (\reversive)} \label{[-gw, -go], [-hsi] (reversives)}
The \stem{-gw, -go} {\reversive} \xref{ex:reverex} and \stem{-hsi} \xref{ex:reverex2} \textsc{\reversive} suffixes turn the meaning of the verb into its opposite. Unlike the dislocative or purposive suffixes, the reversive suffixes cannot be followed by \stem{e} ‘to go’.

The \stem{-gw, -go} suffix is possibly more common than the \stem{-hsi} suffix.

\ea\label{ex:reverex} \stem{-gw, -go} (\cite{foster_course_1993})
\ea hayę́twagwahs\\
\gll ha-yę́tw-a-gw-a-hs\\
 \textsc{3s.m.a}-plant-{\joinerA}-{\reversive}-{\joinerA}-{\habitual}\\
\glt `he is harvesting'
\cfex{hayę́:twahs\\
	\gll ha-yę́:tw-a-hs\\
	\textsc{3s.m.a}-plant-{\joinerA}-{\habitual}\\
	\glt `he is planting'}
\ex ahayętwá:goˀ\\
\gll a-ha-yętw-á:-go-ˀ\\
 {\factual}-\textsc{3s.m.a}-plant-{\joinerA}-{\reversive}-{\punctual}\\
\glt `he did harvest'
\cfex{aháyętoˀ\\
	\gll a-há-yęto-ˀ\\
	{\factual}-\textsc{3s.m.a}-plant-{\punctual}\\
	\glt `he did plant'}
\ex hoyę́twagwęh\\
\gll ho-yę́tw-a-gw-ęh\\
 \textsc{3s.m.p}-plant-{\joinerA}-{\reversive}-{\stative}\\
\glt `he harvested it'
\cfex{hoyę́:twęh\\
	\gll ho-yę́:tw-ęh\\
	\textsc{3s.m.p}-plant-{\stative}\\
	\glt `he planted it'}
\z
\z

\ea\label{ex:reverex2} \stem{-hsi} (\cite{mithun_watewayestanih_1984})\\
sasatrǫnyáhsih\\
\gll sa-s-at-hrǫny-á-hsi-h\\
 {\repetitive}-\textsc{2s.a}-{\semireflexive}-dress-{\joinerA}-{\reversive}-\textsc{euph.h}/{\noaspect}\\
\glt `get undressed again'
\cfex{sasa:trǫ́:nih\\
	\gll sa-s-a:t-hrǫ́:ni-h\\
	{\repetitive}-\textsc{2s.a}-{\semireflexive}-dress-\textsc{euph.h}/{\noaspect}\\
	\glt `get dressed again'}
\z


The \stem{-gw} {\reversive} endings belong to the \textsc{s10} conjugation class (\sectref{Aspect conjugation classes}), as shown in \xref{ex:reverex} and \xref{ex:reverex3}.

\ea\label{ex:reverex3} 
\ea \textsc{habitual} \stem{gw-a-hs} {\reversive}-{\joinerA}-{\habitual}
\ex \textsc{punctual} \stem{go-ˀ} {\reversive}-{\punctual}
\ex \textsc{stative} \stem{gw-ęh} {\reversive}-{\stative}
\ex \textsc{no-aspect} \stem{go-h} {\reversive}-\textsc{euph.h}/{\noaspect} 
\z
\z


\section{Aspect suffixes} \label{Aspect suffixes}
The aspect suffixes appear in slot 3, after the movement suffixes and before any post-aspect suffixes, \tabref{figtab:1:aspectaffixorder}, next page (also see \tabref{figtab:1:verbsuffixorder}, page \pageref{figtab:1:verbsuffixorder}). The three main types of aspect suffix are the \textsc{habitual} \xref{ex:aspectex2}, \textsc{stative} \xref{ex:aspectex3}, and \textsc{punctual} \xref{ex:punctaspectex}. Punctual verbs also require a \textsc{mood} prefix (either the \textsc{factual}, \textsc{indefinite}, or \textsc{future} \xref{ex:punctaspectex}.


\begin{table}
\caption{Aspect affix order}
\label{figtab:1:aspectaffixorder}
\begin{tabularx}{\textwidth}{|l|l|l|l|Q|}
% \lsptoprule
mood & pron & slot 1,2 & slot 3a,3b & slot 3c aspect \\
\hline
 & &  &  & \stem{-hs}, \stem{-haˀ} (etc.) \textsc{\habitual}\strut\\
 \cline{1-1}\cline{5-5}
{}\stem{aˀ-} \textsc{\factual} &  &  &  & \stem{-ˀ} \textsc{\punctual}\strut\\
{}\stem{ę-} \textsc{\future} &  &  &  & \stem{-ˀ} \textsc{\punctual}\strut\\
{}\stem{a:-} \textsc{\indefinite} &  &  &  & \stem{-ˀ} \textsc{\punctual}\strut\\
\cline{1-1}\cline{5-5}
 &  &  &  & \stem{-ęh}, \stem{-ǫh}, \stem{-ˀ}, \stem{-:} (etc.) \textsc{\stative}\strut\\
 \cline{5-5}
% \lspbottomrule
\end{tabularx}
\end{table}

\ea\label{ex:aspectex2} habitual aspect \stem{-h} etc.\\
gé:gęh, gé:gęhs\\
\gll g-é:-gę-h/hs\\
\textsc{1s.a}-{\joinerE}-see-{\habitual}\\
\glt ‘I see’
\z

\FloatBarrier
\ea\label{ex:aspectex3} stative aspect \stem{-:}, etc.\\
agé:gę: \\
\gll ag-é:-gę-:\\
 \textsc{1s.p}-{\joinerE}-see-{\stative}\\
\glt ‘I have seen it’, `my having seen it'
\z


\ea\label{ex:punctaspectex} punctual aspect \stem{-ˀ}
\ea agé:gęˀ \\
\gll a-g-é:-gę-ˀ\\
 {\factual}-\textsc{1s.a}-{\joinerE}-see-{\punctual}\\
\glt ‘I saw it’, `I did (indeed) see it'
\ex ęgé:gęˀ\\
\gll ę-g-é:-gę-ˀ\\
 \fut-\textsc{1s.a}-{\joinerE}-see-{\punctual}\\
\glt `I will see it'
\ex a:gé:gęˀ\\
\gll a:-g-é:-gę-ˀ\\
 {\indefinite}-\textsc{1s.a}-{\joinerE}-see-{\punctual}\\
\glt `I should, could, would, might see it'
\z
\z
 

\subsection{Aspect meaning} \label{Aspectual meaning}
Aspect conveys information about the duration of an event, activity, or characteristic. The punctual aspect denotes an event that has no inherent \textsc{duration}. In contrast, the habitual and stative aspects denote events with duration, and possibly even internal structure (for example, stages or repetitions). The distinction between the \textsc{non-durative} (punctual) and \textsc{durative} (habitual and stative) aspects runs through the Gayogo̱honǫˀnéha:ˀ grammar: for example, some post-aspect suffixes only attach to durative (habitual or stative) verbs because their meaning only makes sense for actions or happenings that extend over time.

The three main aspects are described next. A fourth type of aspect, the \textsc{purposive}, is described separately because it operates  differently from the three main aspects (see \textit{E-verbs}, \sectref{ch:E-verbs}).  

Finally, for several relevant (but technical) background concepts (lexical and sentential aspect), see \sectref{Lexical aspect and sentential aspect}.



\subsection{The habitual, punctual, and stative aspects} \label{The habitual, punctual and stative aspects}
All verbs take an aspect suffix (except for no-aspect verbs, see \sectref{No-aspect and punctual aspect verbs}). \textsc{Three-aspect} verbs take a habitual, punctual, or stative aspect suffix. In contrast, \textsc{stative-only} verbs just take a stative suffix, and \textsc{habitual-only} verbs just take a habitual suffix.

There is only one \stem{-ˀ} {\punctual} ending. In contrast, the habitual family of suffixes includes \stem{-h, -hs, -ˀs, -haˀ}, and the stative family includes \stem{-ǫh, -ęh, -ˀ, -:}.

Any given verb stem normally only takes one habitual, punctual, or stative suffix, as shown in \xref{ex:aspectex3} and \xref{ex:punctaspectex}. However, as seen in \xref{ex:aspectex2} (page \pageref{ex:aspectex2}), some verbs can take more than one habitual suffix (\sectref{Meaning of the habitual aspect} ).

The aspect suffixes for a given verb make up its \textsc{conjugation class}. For example, the verb stem \stem{gę-} ‘to see’ in \xref{ex:aspectex2} can take an \stem{-hs} \textsc{\habitual} ending, and so belongs to one of the \textsc{s} conjugation classes. In contrast, the verb \stem{yaˀdowehd} ‘to think about, consider’ takes a \stem{-haˀ} \textsc{\habitual} ending (as in \cayuga{deyeyaˀdówe̱htaˀ} ‘she is a thinker or seer’) and so belongs to one of the \textsc{h} conjugation classes. Verb conjugation classes are described in \sectref{Aspect conjugation classes}.

The meaning of the habitual, punctual, and stative aspect suffixes is described in the following sections. 


\subsection{Meaning of the habitual aspect} \label{Meaning of the habitual aspect}
The habitual aspect forms of certain verbs can denote an event happening in the present moment \xref{ex:habitaspectex}. (Meanwhile, for other verbs, the \emph{stative} aspect forms have this meaning, see below and \sectref{Meaning of the stative aspect}).

\ea\label{ex:habitaspectex}
\ea agáhdrǫhs\\
\gll ag-áhdrǫ-hs\\
 \textsc{1s.p}-frighten-{\habitual}\\
\glt `I am frightened of it'
\ex kehswáhęhs\\
\gll ke-hswáhę-hs\\
 \textsc{1s:3fis}-hate-{\habitual}\\
\glt `I hate her'
\ex gadejiˀóhgyaˀs\\
\gll g-ade-jiˀóh-gyak-ˀs\\
 \textsc{1s.a}-{\semireflexive}-nail-cut-{\habitual}\\
\glt `I am cutting my nails'
\ex ehé:yǫhs\\
\gll e-hé:yǫ-hs\\
 \textsc{3s.fi.a}-die-{\habitual}\\
\glt `she is dying, on her deathbed'
\ex segé:tsgwahs \\
\gll s-e-gé:tsgw-a-hs\\
 \textsc{2s.a}-{\joinerE}-raise.to.vertical-{\joinerA}-{\habitual}\\
\glt ‘you are lifting it to a vertical position’, `you give parties'\\
\ex gragé:wahs\\
\gll g-ragé:w-a-hs\\
 \textsc{1s.a}-wipe-{\joinerA}-{\habitual}\\
\glt `I am erasing or wiping it'
\ex satgáhǫhaˀ\\
\gll s-at-gáhǫ-haˀ\\
 \textsc{2s.a}-{\semireflexive}-watch-{\habitual}\\
\glt \trs{you are paying attention}, \trs{you are watching right now}
\ex tgyahsǫ́haˀ\\
\gll t-g-yahsǫ́-haˀ\\
 {\cislocative}-\textsc{1s.a}-call-{\habitual}\\
\glt  ‘I call them’, `I am a bingo caller'
\z
\z

The “present tense” interpretation of habitual verbs appears to depend on the \textsc{lexical aspect} of the verb, specifically on the \textsc{consequential} versus \textsc{non-consequential} distinction (see \sectref{Lexical aspect and sentential aspect}.) For verbs with an end-point (consequential verbs), the habitual aspect form tends to describe an event taking place in the present \xref{ex:habitstataspectex}, while the three-aspect \emph{stative} form tends to describe a resulting state (see the “cf.” examples in \ref{ex:habitstataspectex}).


\ea\label{ex:habitstataspectex}
\ea agé:nya:s\\ 
\gll ag-é:-nya:-s\\
 \textsc{1s.p}-{\joinerE}-married-{\habitual}\\
\glt `I am getting married right now'
\cfex{age:nyá:gǫh\\
	\gll ag-e:-nyá:g-ǫh\\
	\textsc{1s.p}-{\joinerE}-married-{\stative}\\
	\glt `I am married'}
\ex wahsdéhstaˀ \\
\gll w-ahsdéhsd-haˀ\\
 \textsc{3s.a}-dry.up-{\habitual}\\
\glt ‘it is evaporating’, `it evaporates'
\cfex{ohsdéhsdǫh \\
	\gll o-hsdéhsd-ǫh\\
	\textsc{3s.p}-dry.up-{\stative}\\
	\glt ‘it has evaporated’, `it is all dried up'}
\ex gatgę́hęhs\\
\gll g-at-gę́h-ę-hs\\
 \textsc{1s.a}-{\semireflexive}-get.up-{\joinerA}-{\habitual}\\
\glt `I get up all the time'
\cfex{agá:tgęh\\
	\gll ag-á:t-gęh\\
	\textsc{1s.p}-{\semireflexive}-get.up.{\stative}\\
	\glt `I got up'}
\ex edá:graˀs \\
\gll e-dá:graˀ-s\\
\textsc{3s.fi.a}-fall-{\habitual}\\
\glt ‘she is forever falling’ (for example, an old woman) 
\cfex{agidagráˀǫh\\
	\gll ag-idagráˀ-ǫh\\
	\textsc{1s.p}-fall-{\stative}\\
	\glt `I have fallen down'}
\ex é:yǫhs \\
\gll é:-yǫ-hs\\
\textsc{3s.fi.a}-arrive-{\habitual}\\
\glt ‘she arrives’ (at the same time as) 
\cfex{gó:yǫ:\\
	\gll gó:-yǫ-:\\
	\textsc{3s.fi.p}-arrive-{\stative}\\
	\glt `she has arrived'}
\ex degáhiˀs \\
\gll de-gá-hiˀ-s\\
 {\dualic}-\textsc{3s.a}-break.up-{\habitual}\\
\glt ‘it breaks’, `it is breakable'
\cfex{deyóhiˀǫh\\
	\gll de-yó-hiˀ-ǫh\\
	{\dualic}-\textsc{3s.p}-break.up-{\stative}\\
	\glt `it is broken'}
\z
\z

In contrast, for verbs without a natural end-point (non-consequential verbs), the habitual form tends to describe a habitual activity, a profession, or a repetitive event \xref{ex:habitstataspectex2}. (Meanwhile, the \emph{stative} form of such verbs tend to describe an event taking place in the present - see the “cf.” examples in \ref{ex:habitstataspectex2}.)

\ea\label{ex:habitstataspectex2}
\ea hadędǫnyáˀtaˀ\\
\gll h-adę-dǫnyáˀt-haˀ\\
 \textsc{3s.m.a}-{\semireflexive}-joke-{\habitual}\\
\glt `he is a joker'
\cfex{sadędǫnyáˀdǫh\\
	\gll s-adę-dǫnyáˀd-ǫh\\
	\textsc{2s.p}-{\semireflexive}-joke-{\stative}\\
	\glt `you are joking'}
 \newpage
\ex agénya̱ˀgwahs \\
\gll agé-nya̱ˀgwah-s\\
 \textsc{1s.p}-vomit-{\habitual}\\
\glt ‘I am vomiting’, ‘I am a vomiter’, `I vomit all the time'
\cfex{agenya̱ˀgwáhǫh\\
\gll age-nya̱ˀgwáh-ǫh\\
\textsc{1s.p}-vomit-{\stative}\\
\glt ‘I am vomiting’ (right now) }
\ex tgagyę́hętwahs\\
\gll t-g-ag-yę́hętw-a-hs\\
 {\cislocative}-\textsc{1s.p}-{\semireflexive}-pull-{\joinerA}-{\habitual}\\
\glt `I am a puller'
\cfex{gyagogyęhę́:twęh\\
	\gll g-yago-g-yęhę́:tw-ęh\\
	{\cislocative}-\textsc{3s.fi.p}-{\semireflexive}-pull-{\stative}\\
	\glt `she is pulling it'}
\ex ogyá:nǫˀs\\
\gll o-g-yá:nǫ-ˀs\\
 \textsc{3s.p}-{\semireflexive}-dream-{\habitual}\\
\glt `it dreams'
\cfex{hogyánǫ̱ˀdrǫh\\
	\gll ho-g-yánǫ̱-ˀdr-ǫh\\
	\textsc{3s.m.p}-{\semireflexive}-dream-{\dislocative}-{\stative}\\
	\glt `he is dreaming'}
\ex gaǫdewáyęhstaˀ \\
\gll gaǫ-de-wáyęhst-haˀ\\
 \exsc{3ns.fi.a}-{\semireflexive}-learn-{\habitual}\\
\glt ‘they are apprentices’, `they are learning together'
\cfex{sadewayę́hsdǫh\\
\gll s-ade-wayę́hsd-ǫh\\
\textsc{2s.p}-{\semireflexive}-learn-{\stative}\\
\glt ‘you are learning’ (now) }
\z
\z

Finally, some verbs are ambiguous -- they can denote an event either with or without an end-point or resulting change of state. The habitual aspect form of such verbs can have either of the readings described above (\ref{ex:habitstataspectex2}b, e).

More examples of habitual-aspect verbs are provided next. As mentioned previously, the non-consequential habituals can denote an activity or event that extends over time, occurring regularly, at frequent intervals, intermittently, or constantly \xref{ex:habitaspectex2}. 

\ea\label{ex:habitaspectex2}
\ea agǫ́:nihs \\
\gll ag-ǫ́:ni-hs\\
 \textsc{3s.fi.p}-make-{\habitual}\\
\glt ‘she makes it into something’, `she earns it'
\ex okahs\\
\gll o-ka-hs\\
 \textsc{3s.p}-leak-{\habitual}\\
\glt `it leaks'
\ex swatgę́nihsˀahs\\
\gll sw-at-gę́nihsˀ-a-hs\\
 {2\pla}-{\semireflexive}-meet-{\joinerA}-{\habitual}\\
\glt `you have meetings all the time'
\ex hehahé:haˀ\\
\gll he-ha-hré:-haˀ\\
 {\translocative}-\textsc{3s.m.a}-place-{\habitual}\\
\glt `he places it there all the time'

\ex deknége̱haˀ \\
\gll de-k-hnége̱-haˀ\\
{\dualic}-\textsc{1s.a}-drink-{\habitual}\\
\glt ‘I don’t drink any more’

\ex edá:graˀs\\
\gll e-dá:graˀ-s\\
\textsc{3s.fi.a}-fall-{\habitual}\\
\glt ‘she is forever falling’ (i.e. an old person)


\ex dekdaˀs\\
\gll de-k-d-a-ˀs\\
{\dualic}-\textsc{1s.a}-stand-{\joinerA}-{\habitual}\\
\glt ‘I stop here’ (whenever...)

\ex gadáhnyoˀs\\
\gll g-ad-áhny-o-ˀs\\
 \textsc{1s.a}-{\semireflexive}-hook-submerge-{\habitual}\\
\glt `I fish continually'

\ex aga:tǫ́:dęhs\\
\gll ag-a:t-hǫ́:dę-hs\\
\textsc{1s.p}-{\semireflexive}-hear-{\habitual}\\
\glt ‘I hear it all the time’ (continually or off-and-on, like the sound of a regular train whistle)
\z
\z

Non-consequential habituals can also denote a profession (an activity done on a regular basis, \ref{ex:habitaspectex4}).

\ea\label{ex:habitaspectex4}
\ea ehyádǫ̱haˀ\\
\gll e-hyádǫ̱-haˀ\\
 \textsc{3s.fi.a}-write-{\habitual}\\
\glt `she is a writer or secretary'

\ex shagóhse̱htaˀ\\
\gll shagó-hse̱ht-haˀ\\
 \exsc{3ms:3fi/3p.p}-hide-{\habitual}\\
\glt `he is a robber, stealer'

\ex hatgéhǫhaˀ\\
\gll h-at-géhǫ-haˀ\\
 \textsc{3s.m.a}-{\semireflexive}-sell-{\habitual}\\
\glt `he is an auctioneer, a seller'
\z
\z

Finally, for a handful of verbs, the habitual aspect describes a quality or characteristic \xref{ex:habitaspectex6} - a function more commonly performed by stative-only verbs (see \sectref{Meaning of the stative aspect}).

\ea\label{ex:habitaspectex6} gagrahs\\
\gll ga-gra-hs \\
\textsc{3s.a}-stink-{\habitual}\\
\glt ‘it stinks’, ‘it is strong-smelling’ 
\z


\subsubsection{Meaning of various habitual aspect suffixes}
Several verbs can take more than one type of habitual aspect suffix \xxref{ex:morehabitualex1}{ex:morehabitualex3}. 

\ea\label{ex:morehabitualex1}
\ea ga̱ˀníkǫ̱haˀ\\
\gll ga̱-ˀníkǫ̱-haˀ\\
 \textsc{3s.a}-sew-{\habitual}\\
\glt `sewing machine'
\ex eˀní:kǫhs\\
\gll e-ˀní:kǫ-hs\\
 \textsc{3s.fi.a}-sew-{\habitual}\\
\glt `seamstress'
\z
\z

\ea\label{ex:morehabitualex2}
\ea Ǫdada̱hnyóhaˀ\\
\gll ǫ-dad-e̱-hnyó-haˀ\\
 \exsc{3fis}-{\reflexive}-fish-{\habitual}\\
\glt `Fishing Dance'
\ex gadáhnyoˀs\\
\gll g-ad-áhnyo-ˀs\\
 \textsc{1s.a}-{\semireflexive}-fish-{\habitual}\\
\glt `I fish continually'
\ex hadáhnyoˀ\\
\gll ha-d-áhnyo-ˀ\\
 \textsc{3s.m.a}-{\semireflexive}-fish-{\habitual}\\
\glt `he is a fisherman'
\ex hęnáda̱hnyoh\\
\gll hęn-ád-a̱hnyo-h\\
 \textsc{3ns.m.a}-{\semireflexive}-fish-{\habitual}\\
\glt `they are fishing'
\z
\z

In such cases, the various habitual suffixes probably convey different meanings, as hinted by the translations in \xref{ex:morehabitualex2} and \xref{ex:morehabitualex3}. Such subtle distinctions would be easier to determine in the context of a conversation -- a topic for further research.

\ea\label{ex:morehabitualex3}
\ea gé:gęh ‘I see’ (regularly, but perhaps at irregular intervals)’
\ex gé:gęhs ‘I see’ (i.e. I see apple trees, orange trees, ... and sometimes, I also see cherry trees)’\footnote{Based on \xref{ex:morehabitualex3}, the \stem{-hs} \textsc{\habitual} could be reanalysed as \stem{-h-s} {\habitual}-{\plural}, because the habituals ending with \stem{-s} seem to imply plural objects. Perhaps this reanalysis also extends to  \stem{-ˀs} \textsc{\habitual} endings.}
\z
\z

\subsection{Meaning of the stative aspect} \label{Meaning of the stative aspect}
The meaning of the stative aspect depends on whether the  verb in question is a stative-only or three-aspect verb. Stative-only verbs just take a stative aspect suffix, and never take habitual or punctual suffixes. In contrast, three-aspect verbs can take habitual, punctual, and stative aspect suffixes. 

Stative-only verbs tend to describe states, characteristics, or attributes. In contrast, three-aspect statives either denote an event occurring at the present time, or a state resulting from an event having taken place. Examples are provided next.




\subsubsection{Meaning of stative-only verbs}
Stative-only verbs tend to describe states, characteristics, or attributes \xref{ex:stataspectex2}. As well, some might describe temporary states (\ref{ex:stataspectex2}a, b), and some, more permanent states (\ref{ex:stataspectex2}l, m). Stative-only verbs do not describe events or happenings.

\ea\label{ex:stataspectex2}
\ea gotowinyǫ́ˀse:\\\label{ex:stataspectex2a}
\gll go-t-howinyǫ́ˀse:\\
 \textsc{3s.fi.p}-{\semireflexive}-have.a.cold.{\stative}\\
\glt `she has a cold'
\ex de̱howę:sǫ:\\\label{ex:stataspectex2b}
\gll de̱-ho-wę́:sǫ:\\
 \exsc{\dualic-3s.m.p}-be.shod.{\stative}\\
\glt `he has shoes on'
\ex ó:gę:t\\\label{ex:stataspectex2c}
\gll ó:-gę:t\\
 \textsc{3s.p}-visible.{\stative}\\
\glt `it is visible'
\ex niyohá:kˀah\\\label{ex:stataspectex2d}
\gll ni-yo-há:kˀah\\
 {\partitive}-\textsc{3s.p}-short.{\stative}\\
\glt `it is short'
\ex ohóˀdę:ˀ\\\label{ex:stataspectex2e}
\gll o-hóˀdę:ˀ\\
 \textsc{3s.p}-soft.{\stative}\\
\glt `it is soft'
\ex ohní:yǫh\\\label{ex:stataspectex2f}
\gll o-hní:y-ǫh\\
 \textsc{3s.p}-hard-{\stative}\\
\glt `it is hard'
\ex agatganǫ́:ni:\\\label{ex:stataspectex2g}
\gll ag-at-ganǫ́:ni-:\\
 \textsc{1s.p}-{\semireflexive}-wealthy-{\stative}\\
\glt `I am wealthy'
\ex agatsenǫ́:ni:\\\label{ex:stataspectex2h}
\gll ag-at-hsenǫ́:ni-:\\
 \textsc{1s.p}-{\semireflexive}-happy-{\stative}\\
\glt `I am happy'
\ex gadręnagáˀǫh\\\label{ex:stataspectex2i}
\gll ga-dręn-a-gáˀ-ǫh\\
 \textsc{3s.a}-smell-{\joinerA}-sweet-{\stative}\\
\glt `it is sweet-smelling (food odours)'
\ex gadręnahshá:sdeˀ\\\label{ex:stataspectex2j}
\gll ga-dręn-a-hshá:sdeˀ\\
 \textsc{3s.a}-smell-{\joinerA}-strong.{\stative}\\
\glt `it is strong-smelling'
\ex ga̱hǫ́ˀji:\\\label{ex:stataspectex2k}
\gll ga̱-hǫ́ˀji:\\
 \textsc{3s.a}-dark.{\stative}\\
\glt `it is dark'
\ex knę́:ye:s\\\label{ex:stataspectex2l}
\gll k-hnę́:y-e:s\\
 \textsc{1s.a}-stature-long.{\stative}\\
\glt `I am tall'
\ex niknę́:yaˀ\\\label{ex:stataspectex2m}
\gll ni-k-hnę́:y-aˀ\\
 {\partitive}-\textsc{1s.a}-stature-short.{\stative}\\
\glt `I am short'
\ex syadǫ̱hsráędiˀ\\\label{ex:stataspectex2n}
\gll s-yadǫ̱hsrá-ędiˀ\\
 \textsc{2s.a}-paper-know.{\stative}\\
\glt `you are smart (educated)'
\z
\z

While stative-only verbs do not describe events \xref{ex:stataspectex3a}, they can be turned into three-aspect verbs describing events with the addition of the causative suffix or other change-of-state suffixes (\ref{ex:stataspectex3b}, see \sectref{Change-of-state suffixes}). 

\ea\label{ex:stataspectex3}
\ea otgiˀ\\\label{ex:stataspectex3a}
\gll o-tgi-ˀ\\
 \textsc{3s.p}-dirty-{\stative}\\
\glt `it is dirty'
\ex ęhsé:tgiht \\\label{ex:stataspectex3b}
\gll ę-hs-é:-tgi-ht\\
{\future}-\textsc{2s.a}-{\joinerE}-dirty-{\causative}.{\zeropunctual}\\
\glt ‘you will dirty it up’ 
\z
\z



\subsubsection{Meaning of three-aspect statives} \label{Meaning of three-aspect statives}
The stative forms of three-aspect verbs (three-aspect statives) either denote an event occurring at the present time, or a state resulting from an event that previously took place. The meaning partly depends on the lexical aspect of the verb, specifically, the \textsc{consequential} versus \textsc{non-consequential} distinction (see \sectref{Lexical aspect and sentential aspect}). 

If the event described by the verb implies an end-point, or results in a new state of affairs (consequential verbs) then the three-aspect stative tends to denote the end-result of an action and is often translated as a past-tense form \xref{ex:stataspectex5}. The three-aspect statives of consequential verbs sometimes even function as “nouns” (naming abstract concepts, \ref{ex:statasp1}).

\ea\label{ex:stataspectex5} three-aspect stative, meaning the result of an action or happening
\ea ohstéhsdǫh\\
\gll o-hstéhsd-ǫh\\
 \textsc{3s.p}-dry.up-{\stative}\\
\glt `it has evaporated'

\ex owa:nóhsdǫh\\
\gll o-wa:-nóhsd-ǫh\\
 \textsc{3s.p}-air-get.cold-{\stative}\\
\glt `it got cold (weather)'

\ex hehó:gyǫ:\\
\gll he-hó:-gy-ǫ:\\
 {\translocative}-\textsc{3s.m.p}-{\semireflexive}-throw-{\stative}\\
\glt `he has thrown it'

\ex tęˀ de̱hóyętwęh\\
\gll tęˀ de̱-hó-yętw-ęh\\
not {\negative}-\textsc{3s.m.p}-plant-{\stative}\\
\glt `he did not plant'


\ex de̱ˀagadrihoˀdá:dǫh\\
\gll de̱ˀ-ag-ad-rihoˀdá:d-ǫh\\
{\negative}-\textsc{1s.p}-{\semireflexive}-work-{\stative}\\
\glt `I did not work’

\ex agekwędáˀǫh\\
\gll ag-e-kw-ędáˀ-ǫh\\
 \textsc{1s.p}-{\joinerE}-food-finish-{\stative}\\
\glt `I have finished eating'

\ex hewagadę́nye̱hdǫh\\
\gll he-wag-adę́-nye̱hd-ǫh\\
 {\translocative}-\textsc{1s.p}-{\semireflexive}-send-{\stative}\\
\glt `I sent it'

\ex agatǫdę́ˀǫh \\
\gll ag-at-hǫdę́ˀ-ǫh\\
\textsc{1s.p}-{\semireflexive}-hear-{\stative}\\
\glt ‘I’ve heard it before’
\z
\z

\ea\label{ex:statasp1} stative-aspect verbs functioning as “nouns”
\ea wadeˀnyędę́hsdǫh\\
\gll wa-d-e-ˀnyędę́hsd-ǫh\\
 \textsc{3s.a}\exsc{-{\semireflexive}}-{\joinerE}-cause.to.measure-{\stative}\\
\glt `the act of measuring'
\ex watwihsdǫnyáˀdǫh \\
\gll wa-t-hwihsd-ǫnyáˀd-ǫh\\
 \textsc{3s.a}-{\semireflexive}-money-cause.to.make-{\stative}\\
\glt ‘profit’, `investment'
\ex wa̱hyá:gwęh\\
\gll wa̱-hy-á:-gw-ęh\\
 \textsc{3s.a}-fruit-{\joinerA}-pick-{\stative}\\
\glt `picked fruit'
\z
\z

However, if the event does not denote an end-point or does not result in a new state of affairs (non-consequential verbs), then the three-aspect stative verb  conveys the idea that an event is happening in the present \xref{ex:stataspectex4}.  

\ea\label{ex:stataspectex4}
\ea swagahshaˀdrǫ́:nyǫh\\
\gll s-wag-ahshaˀdr-ǫ́:-ny-ǫh\\
 {\repetitive}-\textsc{1s.p}-remember-{\joinerA}-{\distributive}-{\distributive}-{\stative}\\
\glt `I am recalling, remembering'
\ex odáˀgra̱hdǫh\\
\gll o-d-áˀgra̱hd-ǫh\\
 \textsc{3s.p}-{\semireflexive}-float-{\stative}\\
\glt `it is floating'
\ex dęya:wę́:dǫh\\
\gll d-ę-ya:w-ę́:d-ǫh\\
 {\dualic}-{\future}-\textsc{3s.p}-sway-{\stative}\\
\glt `it sways'
\ex sadędonyáˀdǫh\\
\gll s-adę-donyáˀd-ǫh\\
 \textsc{2s.p}-{\semireflexive}-make.fun.of-{\stative}\\
\glt `you are joking, making fun of something'
\ex deyagotgrę́grę̱hdǫh\\
\gll de-yago-t-grę́grę̱hd-ǫh\\
 {\dualic}-\textsc{3s.fi.p}-{\semireflexive}-wrinkle.up-{\stative}\\
\glt `she is frowning'
\ex agenya̱ˀgwáhǫh\\
\gll ag-e-nya̱ˀgw-á-h-ǫh\\
 \textsc{1s.p}-{\joinerE}-vomit-{\joinerA}-{\dislocative}-{\stative}\\
\glt `I am vomiting (right now)'
\ex sęnǫ́hdǫnyǫh\\
\gll s-ęnǫ́hdǫny-ǫh\\
 \textsc{2s.p}-wonder-{\stative}\\
\glt `you are wondering, thinking'
\ex gyagogyęhę́:twęh\\
\gll g-yago-g-yęhę́:tw-ęh\\
 {\cislocative}-\textsc{3s.fi.p}-{\semireflexive}-pull-{\stative}\\
\glt `she is pulling it'
\ex desaˀnigǫhaˀdó:gęh\\
\gll de-sa-ˀnigǫh-a-dó:g-ęh\\
 {\negative}-\textsc{2s.p}-mind-{\joinerA}-true-{\stative}\\
\glt `you cannot decide which way to go, you are flighty'
\ex age:nyá:gǫh\\
\gll ag-e:-nyá:g-ǫh\\
\textsc{1s.p}-{\joinerE}-marry-{\stative}\\
\glt ‘I am married’ (does not mean ‘I am marrying someone right now’)\\
\z
\z

Some verbs are ambiguous -- they can describe an event that either has or lacks an end-point. Such verbs can have either of the meanings described above. For example, \cayuga{agídaˀǫh} either applies to the result of an activity or to an ongoing activity \xref{ex:stataspectex6a}.

\ea\label{ex:stataspectex6}
\ea agídaˀǫh \\\label{ex:stataspectex6a}
\gll ag-ídaˀ-ǫh\\
 \textsc{1s.p}-sleep-{\stative}\\
\glt ‘I was asleep’, `I am sleeping'
\ex saihwa̱ˀné:gęh \\\label{ex:stataspectex6b}
\gll sa-rihw-a̱-ˀné:g-ęh\\
 \textsc{2s.p}-matter-{\joinerA}-pray-{\stative}\\
\glt ‘you are praying’, `you have prayed'
\ex desęna̱ˀsgwáhgwęh \\\label{ex:stataspectex6c}
\gll de-s-ę-na̱ˀsgw-á-hgw-ęh\\
 {\dualic}-\textsc{2s.p}-{\semireflexive}-bouncy.surface-{\joinerA}-lift-{\stative}\\
\glt ‘you have jumped’, `you are jumping'
\z
\z

The ambiguous three-aspect stative forms can possibly be rendered unambiguous in context -- for example, if a modifying word is added. Some hypothetical examples are provided in \xref{ex:stataspectex7}.

\ea\label{ex:stataspectex7} disambiguating stative-aspect meanings (hypothetical examples)
\ea \textbf{Te:dę́:ˀ} hodáˀǫh \\\label{ex:stataspectex7a}
te:dę́:ˀ ho-dáˀ-ǫh\\
yesterday \textsc{3s.m.p}-sleep-{\stative}\\
\glt ‘yesterday he did sleep’ (“adverb” \cayuga{te:dę:ˀ} favours a resultant-state reading.)  

\ex \textbf{O:nę́h} hodáˀǫh \\\label{ex:stataspectex7b}
\gll o:nę́h ho-dáˀ-ǫh\\
now \textsc{3s.m.p}-sleep-{\stative} \\
\glt ‘he is sleeping right now’ (particle \cayuga{o:nęh} favours a presently-occurring-activity reading)
\z
\z

\subsubsection{Meaning of stative nouns and agentive stative nouns} \label{Meaning of stative nouns and agentive stative nouns}
\textsc{Stative nouns} begin with \stem{ga-} \textsc{3s.a} or \stem{a-} \exsc{3s.a} (alternatively, \exsc{ø.prefix}), and end with an \stem{-ǫ-:-ˀ} {\stative}-{\length}-\exsc{\nsf} combination. These nouns also express abstract concepts \xref{ex:statvarex13}.

\ea\label{ex:statvarex13} 
\ea gáohǫ:ˀ\\
\gll gá-oh-ǫ-:-ˀ\\
 \textsc{3s.a}-gather-{\stative}-{\length}-\exsc{\nsf}\\
\glt `the act of gathering'
\ex gayaˀdowéhdǫ:ˀ\\
\gll ga-yaˀdowéhd-ǫ-:-ˀ\\
 \textsc{3s.a}-think.about-{\stative}-{\length}-\exsc{\nsf}\\
\glt `the idea of thinking'
\ex atǫ:dę́ˀǫ:ˀ\\
\gll at-hǫ:dę́-ǫ-:-ˀ\\
 \exsc{ø.prefix.\semireflexive}-hear-{\stative}-{\length}-\exsc{\nsf}\\
\glt `the act of hearing'
\ex adówadǫ:ˀ\\
\gll ad-ówad-ǫ-:-ˀ\\
 \exsc{ø.prefix.\semireflexive}-hunt-{\stative}-{\length}-\exsc{\nsf}\\
\glt `the hunt'
\z
\z

Less commonly, stative nouns take other prefixes with the \stem{-ǫ-:-ˀ} {\stative}-{\length}-\exsc{\nsf} combination \xref{ex:statvarex14}.

\ea\label{ex:statvarex14} 
\ea onǫhsodáiyǫ:ˀ\\
\gll o-nǫhsod-á-iy-ǫ-:-ˀ\\
 \textsc{3s.p}-sickness-verb-{\stative}-{\length}-\exsc{\nsf}\\
\glt `sickness, illness'
\ex ǫgyaˀditgę́hsǫ:ˀ\\
\gll ǫ-g-yaˀd-itgę́-hs-ǫ-:-ˀ\\
 \textsc{3s.fi.a}-{\semireflexive}-body-rise-{\distributive}-{\stative}-{\length}-\exsc{\nsf}\\
\glt `nightmares'
\z
\z

\textsc{Agentive stative nouns} are stative aspect verbs functioning as a kind of abstract “noun”, describing an item that results from the activity in question \xref{ex:statvarex12}. They take a \textsc{3s.a} prefix (either \stem{ga-} or \stem{wa-}), instead of the \stem{(y)o-} \textsc{3s.p} series pronominal prefix typically required of three-aspect statives.

\ea\label{ex:statvarex12} 
\ea wa̱hyá:gwęh\\
\gll w-a̱hy-á:-gw-ęh\\
 \textsc{3s.a}-fruit-{\joinerE}-pick-{\stative}\\
\glt `picked fruit'
\ex wadę́nhaˀǫh\\
\gll w-adę́-nhaˀ-ǫh\\
 \textsc{3s.a}-{\semireflexive}-hire-{\stative}\\
\glt `it is chartered, hired'
\ex wagyǫ: \\
\gll w-agyǫ-:\\
 \textsc{3s.a}-throw-{\stative}\\
\glt ‘something thrown away’, `discards'
\ex degágwatwęh\\
\gll de-gá-gwatw-ęh\\
 {\dualic}-\textsc{3s.a}-hem-{\stative}\\
\glt `a hem'
\z
\z




\subsection{Meaning of the punctual aspect} \label{Meaning of the punctual aspect}
The \stem{-ˀ} {\punctual} suffix denotes the occurrence of an event, but does not imply that the event has any duration, or that the event has any stages, repetitions, or other internal structure. The meaning of the punctual can be paraphrased as ‘this is an event’.

Verbs with a punctual suffix also require a mood prefix (either the factual, indefinite, or future \xref{ex:punctaspectex100}. 

\ea\label{ex:punctaspectex100}
\ea akní:nǫˀ\\
\gll a-k-hní:nǫ-ˀ\\
{\factual}-\textsc{1s.a}-buy-{\punctual}\\
\glt ‘I bought it’ (a fact)

\ex a:kní:nǫˀ\\
\gll a:-k-hní:nǫ-ˀ\\
{\indefinite}-\textsc{1s.a}-buy-{\punctual}\\
\glt ‘I might, should, could, or would buy it’ (a possible or desirable future)

\ex ękní:nǫˀ\\
\gll ę-k-hní:nǫ-ˀ\\
\fut-\textsc{1s.a}-buy-{\punctual}\\
\glt ‘I will buy it’ (a certain or near-certain future)
\z
\z

The meaning of the \stem{-ˀ} {\punctual} is made clearer by comparing punctual aspect verbs with no-aspect verbs, which are described next.

\subsection{No-aspect and punctual-aspect verbs} \label{No-aspect and punctual aspect verbs}
No-aspect verbs resemble punctual-aspect verbs in every respect, except that they lack a punctual suffix \xref{ex:punctendex4}. They are used in negation contexts \xref{ex:punctendex4a} and in imperatives or commands \xref{ex:punctendex4b}. 

\newpage
\ea\label{ex:punctendex4} No-aspect and punctual-aspect verbs, compared
\ea tęˀ ta:gaesayę́:de:\\\label{ex:punctendex4a}
\gll tęˀ t-a:-gaesa-yę́:de:\\
not {\contrastive}-{\indefinite}-\exsc{3ns:2s}-recognize.{\noaspect}\\
\glt ‘they will not recognize you’

\cfex{ęgyę́:de:ˀ\\
	\gll ę-g-yę́:de:-ˀ\\
	{\future}-\textsc{1s.a}-recognize-{\punctual}\\
	\glt `I will recognize it'}

\ex sadahǫ́:dǫ:\\\label{ex:punctendex4b}
\gll s-ad-ahǫ́:dǫ:\\
 \textsc{2s.a}-{\semireflexive}-ask.{\noaspect}\\
\glt `ask!'

\cfex{ęsada̱hǫ́:dǫ:ˀ\\
	\gll ę-s-ad-a̱hǫ́:dǫ:-ˀ\\
	{\future}-\textsc{2s.a}-{\semireflexive}-ask-{\punctual}\\
	\glt `you will ask'}
\z
\z

No-aspect verbs lack a punctual suffix for grammatical reasons (for example, being used in negative or imperative contexts). In contrast, \textsc{ø.punctual} verbs lack a \stem{-ˀ} punctual suffix for reasons of pronunciation.

For ø.punctual verbs, the \stem{-ˀ} {\punctual} is simply not pronounced -- because Gayogo̱honǫˀnéha:ˀ words do not end with consonant clusters. For example, the punctual verb form in \xref{ex:punctendex2a} hypothetically ends with \phonet{g-ˀ}, (resulting from \stem{ganyaˀg-ˀ} pay-{\punctual}), but \phonet{g-ˀ} is simplified to \phonet{k}  word-finally -- the [ˀ] {\punctual} is deleted or not pronounced. Similarly, in \xref{ex:punctendex3a}, the punctual verb form ends with an \phonet{h-ˀ} cluster (resulting from \stem{hrih-ˀ} pour-{\punctual}), which is simplified to \phonet{h}.

\ea\label{ex:punctendex2} stem \stem{ganyaˀg} ‘to pay’, missing \stem{-ˀ} \textsc{\punctual} suffix\\
aˀéganyaˀk\\\label{ex:punctendex2a}
\gll aˀ-e-ganyaˀk\\
 {\factual}-\textsc{3s.fi.a}-pay.{\zeropunctual}\\
\glt `she paid'
\cfex{gagánya̱ˀgǫh\\
	\gll ga-ga-nya̱ˀg-ǫh\\
	\textsc{3s.a}-price-pay-{\stative}\\
	\glt `payment'}
\z


\newpage
\ea\label{ex:punctendex3} stem \stem{hrih} ‘to pour, spill’, missing \stem{-ˀ} \textsc{\punctual} suffix\\
hękrih\\\label{ex:punctendex3a}
\gll h-ę-k-hrih\\
 {\translocative}-{\future}-\textsc{1s.a}-pour.{\zeropunctual}\\
\glt `I will pour'
\cfex{gahíhǫh\\
	\gll ga-hríh-ǫh\\
	\textsc{3s.a}-spill-{\stative}\\
	\glt `it is spilled'}
\z


\subsubsection{Meaning of no-aspect verbs} \label{Meaning of no-aspect verbs}
In contrast to ø-punctuals, no-aspect verbs lack an aspect suffix for grammatical reasons: in general no-aspect verbs denote non-events, or events that are not actualized, and these meanings are not compatible with a punctual suffix, which denotes an event.\footnote{No-aspect verbs function like \textsc{subjunctive mood} verbs in other languages, describing the speaker’s attitude that the event is not real or realized.} For example, no-aspect verbs are used with negative future events \xref{ex:noaspectex}: since these are non-events by definition (they will not happen), they do not take a \stem{-ˀ} \textsc{\punctual} suffix. In contrast, positive future events (the ‘cf.’ form in \ref{ex:noaspectex}) are likely to occur as events, and so take a  \stem{-ˀ} {\punctual} suffix.

\ea\label{ex:noaspectex} 
tęˀ tǫ:dá:ge:  \\
\gll tęˀ tǫdá:-g-e-:\\
not {\cislocative.\dualic.\indefinite}\exsc{-1s.a}-go-{\purposive}.{\noaspect}\\
\glt ‘I am not going’
\cfex{nǫdá:ge:ˀ\\
	\gll nǫdá:-ge-:-ˀ\\
	\exsc{\partitive.\dualic.\indefinite-1s.a}-go-{\purposive}-{\punctual}\\
	\glt `(where) I come from', `I came from (there)'}
\z


As mentioned previously, no-aspect verbs appear in negation contexts (\sectref{ch:Negation}) and in commands (\ref{ex:noaspectex}, \sectref{Commands (imperatives) and suggestions}.) They also occur in certain hypothetical sentences, which are described next, as a way of further illustrating the meaning of no-aspect forms.



\subsubsection{\textsc{no-aspect} hypothetical verbs} \label{No-aspect hypothetical verbs}
Some no-aspect verbs describe hypothetical events, rather than actual ones \xref{ex:noaspectex3}. The no-aspect verbs in \xref{ex:noaspectex3} describe hypothetical events in conditional clauses (shown in square brackets, see \sectref{ch:Conditional clauses with [gyę:gwaˀ], [gęh], [hę:gyeh] ‘if, whether’}), which, in turn, are about events that are possible but not realized. 

\ea\label{ex:noaspectex3} 
\ea\label{ex:noaspectex3a} 
\gll Toh giˀ [shęh há:ge:] \\
there just that I.should.go\\
\glt ‘Just maybe I should go there.’  

\cfex{\gll h-á:-g-e-:\\
\exsc{\cislocative-\indefinite-1s.a}-go-{\purposive}.{\noaspect}\\
}

\ex \label{ex:noaspectex3b}
\gll Daskro:wíh [ędwé: gęh]. \\
you.tell.me we.will.go if/whether \\
\glt ‘Tell me if/whether you are coming with us.’ 

\cfex{
\gll ę-dw-e:\\
{\future}-\exsc{1p.in.a}-go-{\purposive}.{\noaspect}\\
\glt ‘(that) we go together’
}

\cfex{ę́:dwe:ˀ\\
	\gll ę́:-dw-e:-ˀ\\
	{\future}-\exsc{1p.in.a}-go-{\punctual}\\
	\glt `we will go together'}

\ex\label{ex:noaspectex3c}
\gll Tęˀ dejǫhsdí:s o:nę́h [gyę:gwáˀ a:yáihe:] \\
not she.didn’t.pay.attention now if she.would.die\\
\glt ‘She did not care that she was going to her death.’ 

\cfex{\gll a:-ya-ihe: \\
{\indefinite}-\textsc{3s.fi.a}-die.{\noaspect}\\
\glt  ‘(that) she die’
}
\z
\z 


\section{Post-aspect (tense) suffixes} \label{Post-aspect (tense) suffixes}
The \textsc{post-aspect} (tense) suffixes (slot 4, \tabref{figtab:1:verbsuffixorder}, page \pageref{figtab:1:verbsuffixorder}) can attach to habitual or stative verbs, but not punctual ones. (Examples are provided in the following sections.) 

The \stem{-gęhę:ˀ} \textsc{\past} and \stem{-hk} \textsc{\former} endings add past tense meanings to habitual and stative verbs. The \stem{-hne:ˀ} \textsc{\remote} does so as well, but only attaches to stative verbs.

The \stem{-:k} \textsc{\modalizer} makes it possible to add \stem{ę-} \textsc{\future} or \stem{a:-} \textsc{\indefinite} mood prefixes to habitual and stative verbs. (Verbs with the \stem{-:k} \textsc{\modalizer} cannot take an \stem{aˀ-} \textsc{\factual} prefix.)


\subsection{Variations on the habitual aspect} \label{Variations on the habitual aspect}
Habitual aspect verbs can take the \stem{-gęhę:ˀ} \textsc{\past}, \stem{-hk} \textsc{\former} or \stem{-:k} \textsc{\modalizer} suffixes. \tabref{figtab:1:habbase} shows the original habitual aspect verb, while \tabref{figtab:1:habvariations} summarizes the variations that such verbs can take. The pronominal prefix of the resulting verb is the same type as the original habitual verb. Examples are provided in the next section.

\begin{table}
\caption{Habitual base\label{figtab:1:habbase}}
\begin{tabular}{l|l|l}
pronominal & stem & aspect suffix\strut\\
\hline
 & verb & \textsc{\habitual}\strut\\
\end{tabular}
\end{table}

\begin{table}
\caption{Habitual variations\label{figtab:1:habvariations}}
\begin{tabularx}{\textwidth}{l|Q|l|l}
% \lsptoprule
& mood & habitual base & post-aspect\\
&      & (\textsc{pron}-verb-{\habitual})  & \\
\hline
\textsc{habitual past} &  & \textsc{pron}-verb-\stem{(h)s} & \stem{-gęhę:ˀ} \textsc{\past}\\
&  & \textsc{pron}-verb-\stem{ha(ˀ)} & \stem{-hk} \textsc{\former}\strut\\
\hline
\strut\textsc{future habitual} and & \stem{ę-} {\future}  & \textsc{pron}-verb-\stem{(h)s} & \stem{-:k} \textsc{\modalizer}\\
\textsc{indefinite habitual} & \stem{a:-} \textsc{\indefinite} & \textsc{pron}-verb-\stem{ha(ˀ)} & \stem{-:k} \textsc{\modalizer}\\
% \lspbottomrule
\end{tabularx}
\end{table}


\subsubsection{Habituals with \stem{-gęhę:ˀ} {\past} or \stem{-hk} {\former}} \label{Habituals with [-gęhę:ˀ] past or [-hk] former}
Habitual verbs can take the \stem{-gęhę:ˀ} {\past} \xref{ex:habitvarex2} or \stem{-hk} {\former} endings \xref{ex:habitvarex3}. According to \citet{sasse_far_1998}, the \stem{-gęhę:ˀ} {\past} ending is used with \textsc{s}-conjugation verbs (which take \stem{-s} or \stem{-hs} {\habitual} endings, \ref{ex:habitvarex2}), while \stem{-hk} {\former} is used with \textsc{h}-conjugation verbs (which take the \stem{-haˀ} {\habitual} ending, \ref{ex:habitvarex3}). (See \textit{Aspect conjugation classes}, \sectref{Aspect conjugation classes}.)

\ea\label{ex:habitvarex2} \textsc{s}-conjugation verbs
\ea hadowahsgę́hę:ˀ\\
\gll ha-d-owad-hs-gę́hę:ˀ\\
 \textsc{3s.m.a}-{\semireflexive}-hunt-{\habitual}-{\past}\\
\glt `he used to be a hunter'
\cfex{hadó:wa:s\\
	\gll ha-d-ó:wa:d-s\\
	\textsc{3s.m.a}-{\semireflexive}-hunt-{\habitual}\\
	\glt `he is a hunter'}
\ex hahdóhsgę̱hę:ˀ\\
\gll ha-hdó-hs-gę̱hę:ˀ\\
 \textsc{3s.m.a}-dive-{\habitual}-{\past}\\
\glt `he used to dive'
\cfex{hahdo:s \\
	\gll ha-hdo:-s\\
	\textsc{3s.m.a}-dive-{\habitual}\\
	\glt ‘he dives’, `he is a diver'}
\ex sahtgaˀwa̱hsgę́hę:ˀ\\
\gll s-ahtgaˀw-a̱-hs-gę́hę:ˀ\\
 \textsc{2s.a}-release-{\joinerA}-{\habitual}-{\past}\\
\glt `you used to let go, you used to give up'
\cfex{sahtgaˀs\\
	\gll s-ahtgaˀw-s\\
	\textsc{2s.a}-release-{\habitual}\\
	\glt `you forfeit, you let go of things all the time'}
\z
\z

\ea\label{ex:habitvarex3} \textsc{h}-conjugation verbs
\ea sadǫhwidáˀtahk\\
\gll s-ad-ǫhwidáˀt-ha-hk\\
 \textsc{2s.a}-{\semireflexive}-swing-{\habitual}-{\former}\\
\glt `you used to swing (long ago)'
\cfex{sadǫhwidáˀtaˀ \\
	\gll s-ad-ǫhwidáˀt-haˀ\\
	\textsc{2s.a}-{\semireflexive}-swing-{\habitual}\\
	\glt ‘you swing (all the time)’, `you are a swinger'}
\ex gohsóˀkahk\\
\gll go-hsóˀk-ha-hk\\
 \textsc{3s.fi.p}-limp-{\habitual}-{\former}\\
\glt `she used to limp'
\cfex{go̱hsóˀkaˀ\\
	\gll go̱-hsóˀk-haˀ\\
	\textsc{3s.fi.p}-limp-{\habitual}\\
	\glt `she is limping'}
\ex hahsgyáǫhahk\\
\gll ha-hsgyáǫ-ha-hk\\
 \textsc{3s.m.a}-walk.quickly-{\habitual}-{\former}\\
\glt `he used to walk quickly'
\newpage 
\cfex{hahsgyáǫhaˀ \\
	\gll ha-hsgyáǫ-haˀ\\
	\textsc{3s.m.a}-walk.quickly-{\habitual}\\
	\glt ‘he walks quickly’, `he gives someone encouragement'}
\ex desahsǫhdré:hahk\\
\gll de-s-ahsǫhdré:-ha-hk\\
 {\dualic}-\textsc{2s.a}-join-{\habitual}-{\former}\\
\glt `you used to join things together, put it together'
\cfex{desahsǫdré:haˀ \\
	\gll de-s-ahsǫdré:-haˀ\\
	{\dualic}-\textsc{2s.a}-join-{\habitual}\\
	\glt ‘you join things together all the time’, `you do puzzles'}
\ex ǫdrę́notahk\\
\gll ǫ-d-rę́n-ot-ha-hk\\
 \textsc{3s.fi.a}-{\semireflexive}-song-stand-{\habitual}-{\former}\\
\glt `she used to sing'
\cfex{gaǫdręnó:taˀ\\
	\gll gaǫ-d-ręn-ó:t-haˀ\\
	\textsc{3ns.fi.a}-{\semireflexive}-song-stand-{\habitual}\\
	\glt `they (f/m) are singers'}
\z
\z

\subsubsection{Habituals with \stem{-:k} \textsc{modalizer} ({\modalizer})} \label{Habituals with [-:k] modalizer}
Habitual verbs can take the \stem{-:k} {\modalizer} suffix, which, as the name implies, enables the verb to take an \stem{ę-} {\future} prefix \xxref{ex:habitvarex4}{ex:habitvarex5} or \stem{a:-} \textsc{\indefinite} prefix \xref{ex:habitvarex6}, \xref{ex:habitvarex5a}, \xref{ex:habitvarex5d}.  (Verbs with the \stem{-:k} {\modalizer} do not take the \stem{aˀ-} \textsc{\factual} prefix.)

According to \citet{sasse_far_1998}, \textsc{h}-conjugation habituals take the \stem{-ha-:k} \textsc{\habitual-\modalizer} combination \xref{ex:habitvarex4}, \xref{ex:habitvarex6}, and \textsc{s}-conjugation ones take the \stem{-(h)s-e:k} \textsc{\habitual-\modalizer} combination \xref{ex:habitvarex5}.

\ea\label{ex:habitvarex4} \stem{ę-verb-ha-:k} \exsc{\future}-verb-{\habitual}-{\modalizer}
\ea dędwadadrihwanǫ́hkwa:k\\
\gll d-ę-dw-adad-rihw-a-nǫ́hkw-ha-:k\\
 {\dualic}-{\future}-\exsc{1p.in.a}-{\reflexive}-matter-{\joinerA}-love-{\habitual}-{\modalizer}\\
\glt `we all will show respect for one another'
\cfex{shedwánǫ̱hkwaˀ\\
	\gll shedwá-nǫ̱hkw-haˀ\\
	\textsc{1inp:3ms}-love-{\habitual}\\
	\glt `we love him'}

\ex ęwadę̱hníˀdoˀkta:k\\
\gll ę-w-ad-ę̱hníˀd-oˀkt-ha-:k\\
 \fut-\textsc{3s.a}-{\semireflexive}-month-end-{\habitual}-{\modalizer}\\
\glt `every month will end'
\cfex{hewádoˀktaˀ\\
	\gll he-w-ád-oˀkt-haˀ\\
	{\translocative}-\textsc{3s.a}-{\semireflexive}-end-{\habitual}\\
	\glt `it ends over there'}
\ex ęhsáhsoˀka:k\\
\gll ę-hs-áhsoˀk-ha-:k\\
 \fut-\textsc{2s.p}-limp-{\habitual}-{\modalizer}\\
\glt `you will limp'
\cfex{go̱hsóˀkaˀ\\
	\gll go̱-hsóˀk-haˀ\\
	\textsc{3s.fi.p}-limp-{\habitual}\\
	\glt `she is limping'}
\z
\z
 
\ea\label{ex:habitvarex6} \stem{a:-verb-ha-:k} \exsc{\indefinite}-verb-{\habitual}-{\modalizer}\\
a:gǫihwanǫ́hkwa:k \\
\gll a:-gǫ-ihw-a-nǫ́hkw-ha-:k\\
 {\indefinite}-\exsc{1s:2s}-matter-{\joinerA}-love-{\habitual}-{\modalizer}\\
\glt `I should care, respect your ideas'
\cfex{gǫnǫ́hkwaˀ\\
	\gll gǫ-nǫ́hkw-haˀ\\
	\textsc{1s:2s}-love-{\habitual}\\
	\glt `I love you'}
\z


\ea\label{ex:habitvarex5} \stem{ę/a:-verb-hs-e:k} \exsc{\future/\indefinite-}verb-{\habitual}-{\modalizer}
\ea\label{ex:habitvarex5a} ęhswagáęse:k\\
\gll ę-hswa-gáę-s-e:k\\
 \fut-\exsc{2p.p}-willing-{\habitual}-{\modalizer}\\
\glt `you all will be willing'
\cfex{sagáęs\\
	\gll sa-gáę-s\\
	\textsc{2s.p}-willing-{\habitual}\\
	\glt `you are willing'}
\ex a:hayętwa̱hsé:k tsǫ:\\
\gll a:-ha-yętw-a̱-hs-é:k tsǫ:\\
 {\indefinite}-\textsc{3s.m.a}-plant-{\joinerA}-{\habitual}-{\modalizer} just\\
\glt `just let him keep planting'
\cfex{hayę́:twahs\\
	\gll ha-yę́:tw-a-hs\\
	\textsc{3s.m.a}-plant-{\joinerA}-{\habitual}\\
	\glt `he is a planter'}
\ex dęhatganyáhse:k\\
\gll d-ę-ha-t-gany-á-hs-e:k\\
 {\dualic}-{\future}-\textsc{3s.m.a}-{\semireflexive}-want.desperately-{\joinerA}-{\habitual}-{\modalizer}\\
\glt `he will be desperate'
\cfex{de̱hátganyahs \\
	\gll de̱-há-t-gany-a-hs\\
	{\dualic}-\textsc{3s.m.a}-{\semireflexive}-want.desperately-{\joinerA}-{\habitual}\\
	\glt ‘he is desperate, desperately wanting something’, ‘he is in dire need’, `he will settle for just anyone'}
\ex\label{ex:habitvarex5d} a:ga̱hyágwa̱hse:k\\
\gll a:-g-a̱hy-á-gw-a̱-hs-e:k\\
 {\indefinite}-\textsc{1s.a}-fruit-{\joinerA}-pick-{\joinerA}-{\habitual}-{\modalizer}\\
\glt `I might be a fruit picker'
\cfex{ga̱hyá:gwahs\\
	\gll g-a̱hy-á:-gw-a-hs\\
	\textsc{1s.a}-fruit-{\joinerA}-pick-{\joinerA}-{\habitual}\\
	\glt `I am picking fruit'}
\z
\z


\subsubsection{Habituals with ‘easy to’ / ‘hard to’ constructions} \label{Habituals with ‘easy to / hard to’ constructions}
The verbs described in the previous section are used in ‘easy to’ \xref{ex:habitvarex8} and ‘hard to’ \xref{ex:habitvarex9} constructions.

\ea\label{ex:habitvarex8} 
\gll Wagyesa̱ˀgéh da:ga̱hyadǫ̱hsríyaˀksǫ:k. \\
it.is.easy the.paper.would.cut\\
\cfex{\gll d-a:-ga̱-hyadǫ̱hsr-íyaˀk-s-ǫ-:k\\
{\dualic}-{\indefinite}-\textsc{3s.a}-paper-cut-{\habitual}-{\distributive}-{\modalizer}\\
\glt ‘it is easy to cut the paper’, ‘the paper is easy to cut’
}
\z

\ea\label{ex:habitvarex9} 
\gll Wę:dó:ˀ  da:ga̱hyadǫ̱hsríyaˀksǫ:k. \\
it.is.hard the.paper.would.cut \\
\glt ‘it is hard to cut paper’, ‘the paper is hard to cut’

\cfex{\gll d-a:-ga̱-hyadǫ̱hsr-íyaˀk-s-ǫ-:k\\
{\dualic}-{\indefinite}-\textsc{3s.a}-paper-cut-{\habitual}-{\distributive}-{\modalizer}\\
}
\z


\subsection{Variations on the stative aspect} \label{Variations on the stative aspect}
Stative verbs take the \stem{-gęhę:ˀ} {\past}, \stem{-hne:ˀ} {\remote}, \stem{-hk} {\former}, or \stem{-:k} {\modalizer} suffixes. \tabref{figtab:1:statbase} illustrates the structure of the original stative verb, while \tabref{figtab:1:statvar} summarizes the variations of such verbs. The pronominal prefix of the resulting verb is the same type as the original stative verb. Examples are provided in the following sections.


\begin{table}
\caption{Stative base}
\label{figtab:1:statbase}
\begin{tabular}{l|l|l}
% \lsptoprule
pronominal & stem & aspect\strut \\
\hline
\textsc{p} & verb & {\stative}\strut\\
% \lspbottomrule
\end{tabular}
\end{table}

\begin{table}
\caption{Stative variations}
\label{figtab:1:statvar}
\begin{tabular}{l|l|l|l}
% \lsptoprule
& mood & stative base & post-aspect \\
&      & (\textsc{p}-verb-{\stative})\strut & \\
\hline
\textsc{stative past} &  &  & \stem{-gęhę:ˀ} {\past}\strut\\
                      &  &  & \stem{-hne:ˀ} {\remote}\\
                      &  &  & \stem{-hk} {\former}\\
\hline
\textsc{future stative} & \stem{ę-} {\future} &  & \stem{-:k} {\modalizer}\\
\textsc{indefinite stative} &  \stem{a:-} {\indefinite} &  & \stem{-:k} {\modalizer}\\
% \lspbottomrule
\end{tabular}
\end{table}


\subsubsection{Statives with \stem{-gęhę:ˀ} {\past}, \stem{-hne:ˀ} {\remote}, or \stem{-hk} {\former}} \label{Statives with [-gęhę:ˀ] past, [-hne:ˀ] remote, or [-hk] former}
Statives can take the \stem{-hne:ˀ} {\remote} \xxref{ex:statvarex2}{ex:statvarex3} or \stem{-hk} {\former} \xxref{ex:statvarex4}{ex:statvarex21} suffixes to denote an activity or the result of an activity that occurred in the past. According to \citet{sasse_far_1998}, statives of \textsc{s}-conjugation verbs take the \stem{-hne:ˀ} {\remote} suffix \xxref{ex:statvarex2}{ex:statvarex3}, while statives of \textsc{h}-conjugation verbs take the \stem{-hk} {\former} suffix \xxref{ex:statvarex4}{ex:statvarex21}.
 
\ea\label{ex:statvarex2} \stem{verb-{\stative}-\exsc{remote}} (three-aspect statives, \textsc{s}-conjugation)
\ea agahyagwę́hne:ˀ\\
\gll ag-ahy-a-gw-ę́-hne:ˀ\\
\textsc{1s.p}-fruit-{\joinerA}-pick-{\stative}-{\remote}\\
\glt ‘I’ve done picking fruit’

\cfex{agáhyagwęh\\
	\gll ag-áhy-a-gw-ęh\\
	\textsc{1s.p}-fruit-{\joinerA}-pick-{\stative}\\
	\glt `I have picked fruit (farther in the past)'}

\ex agekǫníhne:ˀ\\
\gll ag-e-k-ǫní-hne:ˀ\\
 \textsc{1s.p}-{\joinerE}-food-make.{\stative}-{\remote}\\
\glt `I have cooked'

\cfex{age:kǫ́:ni:\\
	\gll ag-e:-k-ǫ́:ni-:\\
	\textsc{1s.p}-{\joinerE}-food-make-{\stative}\\
	\glt `I cooked'}

\ex sajagǫ̱ˀǫ́hne:ˀ\\
\gll sa-jagǫ̱-ˀǫ́-hne:ˀ\\
 \textsc{2s.p}-persevere-{\stative}-{\remote}\\
\glt `you singular have persevered'

\cfex{agejagǫ́ˀǫh\\
	\gll ag-e-jagǫ́-ˀǫh\\
	\textsc{1s.p}-{\joinerE}-persevere-{\stative}\\
	\glt `I do persevere all the time'}

\ex sagahdrǫníhne:ˀ\\
\gll sa-gahdr-ǫní-hne:ˀ\\
 \textsc{2s.p}-eye-make.{\stative}-{\remote}\\
\glt `you used to stare all the time'

\cfex{sagahdrǫ́:ni: \\
	\gll sa-gahdr-ǫ́:ni-:\\
	\textsc{2s.p}-eye-make-{\stative}\\
	\glt ‘you are "nosy" with your eyes', ‘you are always looking'}

\ex dewagadǫhwęjóni̱hne:ˀ\\
\gll de-wag-ad-ǫhwęjóni̱-hne:ˀ\\
 {\dualic}-\textsc{1s.p}-{\semireflexive}-want.{\stative}-{\remote}\\
\glt `I wanted something in the past'

\cfex{dewagadǫhwęjǫ́:nih\\
	\gll de-wag-ad-ǫhwęjǫ́:ni-h\\
	{\dualic}-\textsc{1s.p}-{\semireflexive}-want-{\stative}\\
	\glt `I want something'}
\z
\z

\ea\label{ex:statvarex3} \stem{verb-{\stative}-\exsc{remote}} (stative-only verbs, \textsc{s}-conjugation)
\ea eksaˀgowáhne:ˀ\\
\gll e-ksaˀ-gowá-hne:ˀ\\
 \textsc{3s.fi.a}-child-great.{\stative}-{\remote}\\
\glt `she used to be pretty'
\newpage
\cfex{eksaˀgó:wah\\
	\gll e-ksaˀ-gó:wah\\
	\textsc{3s.fi.a}-child-great.{\stative}\\
	\glt `she is pretty'}
\ex saníˀǫhne:ˀ\\
\gll sa-ní-ˀǫ-hne:ˀ\\
 \textsc{2s.p}-greedy-{\stative}-{\remote}\\
\glt `you used to be stingy of it'
\cfex{saníˀǫh\\
	\gll sa-ní-ˀǫh\\
	\textsc{2s.p}-greedy-{\stative}\\
	\glt `you are stingy, greedy, cheap'}
\z
\z

\ea\label{ex:statvarex4} \stem{verb-{\stative}-{\former}} (three-aspect statives, \textsc{h}-conjugation)
\ea gonadręnó:daˀk\\
\gll gon-ad-ręn-ó:d-a-ˀ-k\\
 \exsc{3ns.fi.p}-{\semireflexive}-song-stand-{\joinerA}-{\causative}-{\former}\\
\glt `they did sing'
\cfex{hodrę́:no:t\\
	\gll ho-d-rę́:n-o:t\\
	\textsc{3s.m.p}-{\semireflexive}-song-stand.{\stative}\\
	\glt `he is singing'}
\ex gani:yǫ́:daˀk ‘it hung there’,\\
\gll ga-ni:yǫ́:d-a-ˀ-k\\
 \textsc{3s.a}-hang.{\stative}-{\joinerA}-{\causative}-{\former}\\
\glt `it used to hang there'
\cfex{ganí:yǫ:t\\
	\gll ga-ní:yǫ:t\\
	\textsc{3s.a}-hang.{\stative}\\
	\glt `it is hanging'}
\z
\z


\ea\label{ex:statvarex20} \stem{verb-{\stative}-{\former}} (stative-only verbs, \textsc{h}-conjugation)\\
tgáenagrehk\\
\gll t-gáe-nagre-hk\\
 {\cislocative}-\textsc{3ns.fi.a}-live.{\stative}-{\former}\\
\glt `they did live there, used to live there'
\cfex{tgáenagreˀ\\
	\gll t-gáe-nagre-ˀ\\
	{\cislocative}-\textsc{3ns.fi.a}-live-{\stative}\\
	\glt `where they live over there'}
\z 

\newpage
\ea\label{ex:statvarex21}  \stem{verb-{\stative}-\textsc{euph.d}-{\causative}-\exsc{former}} (stative-only verbs, \textsc{h}-conjugation)
\ea tgiˀdrǫ́:daˀk\\
\gll t-g-iˀdrǫ́:-d-a-ˀ-k\\
 {\cislocative}-\textsc{1s.a}-live.{\stative}-\textsc{euph.d}-{\joinerA}-{\causative}-\exsc{former}\\
\glt `where I lived'
\cfex{shęh tgiˀdrǫˀ\\
	\gll shęh t-g-iˀdrǫ-ˀ\\
	that {\cislocative}-\textsc{1s.a}-live-{\stative}\\
	\glt `where I live, reside’ }
\ex agadowihshę́:daˀk\\
\gll ag-ad-owihshę́:-d-a-ˀ-k\\
 \textsc{1s.p}-{\semireflexive}-rest.{\stative}-\textsc{euph.d}-{\causative}-\exsc{former}\\
\glt `I did have a rest'
\cfex{agadowíshęˀ\\
	\gll ag-ad-owíshę-ˀ\\
	\textsc{1s.p}-{\semireflexive}-rest-{\stative}\\
	\glt `I have rested'}
\z
\z

Statives can also take the \stem{-gęhę:ˀ} \textsc{past }to denote an activity or the result of an activity that occurred in the past (\ref{ex:statvarex210}a, b). As examples (\ref{ex:statvarex210}b, c), show, at least one verb can take either suffix. There may be a subtle difference in meaning that is not captured by the translations.

\ea\label{ex:statvarex210} \stem{verb-{\stative}-{\past}}
\ea ní:ˀ agawęgę́hę:ˀ\\\label{ex:statvarex210a}
\gll ní:ˀ ag-aw-ę-gę́hę:ˀ\\
 I \textsc{1s.p}-own-{\stative}-{\past}\\
\glt  ‘it used to be mine’
\cfex{agá:węh\\
	\gll ag-á:w-ęh\\
	\textsc{1s.p}-own-{\stative}\\
	\glt `mine'}
\ex howę́gę̱hę:ˀ\\\label{ex:statvarex210b}
\gll ho-wę-gę̱hę:ˀ\\
 \textsc{3s.m.p}-own-{\stative}-{\past}\\
\glt `it used to be his'
\cfex{hó:węh\\
	\gll hó:-w-ęh\\
	\textsc{3s.m.p}-own-{\stative}\\
	\glt `his'}
\ex howę́hne:ˀ\\\label{ex:statvarex210c}
\gll ho-w-ę́-hne:ˀ\\
 \textsc{3s.m.p}-own-{\stative}-{\remote}\\
\glt `it used to be his'
\cfex{hó:węh\\
	\gll hó:-w-ęh\\
	\textsc{3s.m.p}-own-{\stative}\\
	\glt `it is his'}
\z
\z



\subsubsection{Statives with \stem{-:k} \textsc{modalizer}} \label{Statives with [-:k] modalizer}
Verbs in the stative aspect can take the \stem{-:k} {\modalizer} suffix, along with the \stem{ę-} {\future} or \stem{a:-} \textsc{\indefinite} mood prefix. (Verbs with the \stem{-:k} {\modalizer} cannot take the \stem{aˀ-} \textsc{\factual} prefix.) The resulting verbs take the same kind of pronominal prefix as the original stative verb.

Based on post-aspect suffixes, there are two types of such verbs, as summarized in \tabref{figtab:1:statvar2}. (`V' denotes a verb stem ending with a vowel.)

\begin{table}
\caption{More stative variations}
\label{figtab:1:statvar2}
\begin{tabular}{l|l|l|l}
% \lsptoprule
& mood & stative base & post-aspect \\
&      & (\textsc{p}-verb-{\stative}) & \\
\hline
\textsc{future stative}  & \stem{ę-} {\future}  & \stem{\textsc{p}-verb(.V)-ę/-ǫ} & \stem{-:k} {\modalizer}\\
\textsc{indefinite stative} & \stem{a:-} {\indefinite} &  & \stem{-h-a/-ę/-ǫ-:k} \textsc{euph.h}{}-\\
                            &     &  & \textsc{\joiner-\modalizer}\\
% \lspbottomrule
\end{tabular}
\end{table}

One type just takes the \stem{-:k} {\modalizer} after the stative ending, both for three-aspect statives (\ref{ex:statvarex30}a--c) and for originally stative-only verbs (\ref{ex:statvarex30}d).

\ea\label{ex:statvarex30} \stem{\exsc{mood-…}verb-{\stative}-\exsc{:k}}
\ea ęwaga̱hyá:gwę:k\\
\gll ę-wag-a̱hy-á:-gw-ę-:k\\
 \fut-\textsc{1s.p}-fruit-{\joinerA}-pick-{\stative}-{\modalizer}\\
\glt `I will have picked fruit'
\cfex{agáhyagwęh\\
	\gll ag-áhy-a-gw-ęh\\
	\textsc{1s.p}-fruit-{\joinerA}-pick-{\stative}\\
	\glt `I have picked fruit'}
 \newpage
\ex ęgáˀnikǫ:k\\
\gll ę-gá-ˀnikǫ-:k\\
 \fut-\textsc{3s.a}-sew.{\stative}-{\modalizer}\\
\glt `it will be sewn'
\cfex{ga̱ˀní:kǫˀ\\
	\gll ga̱-ˀní:kǫ-ˀ\\
	\textsc{3s.a}-sew-{\stative}\\
	\glt `a seam'}
\ex ęwagatǫ́:de:k\\
\gll ę-wag-atǫ́:de-:k\\
 \fut-\textsc{1s.p}-hear.{\stative}-{\modalizer}\\
\glt `I will be hearing it'
\cfex{ aga:tǫ́:deˀ\\
\gll ag-a:tǫ́:de-ˀ \\
\textsc{1s.p}-hear-{\stative}\\
\glt ‘I hear it’ (right now) }
\ex ǫ:dagaená:gre:k, aǫdagaená:gre:k\\
\gll ǫ:da/aǫda-gae-ná:gr-e-:k\\
 {\indefinite.\cislocative}\exsc{-3ns.fi.a}-live-go-{\modalizer}\\
\glt `they would live there'
\cfex{tgáenagrehk\\
	\gll t-gáe-nagre-hk\\
	{\cislocative}-\textsc{3ns.fi.a}-live-go-{\former}\\
	\glt `they did or used to live there'}
\z
\z

The second type\footnote{The second type is also attested in Oneida \citep[88--89]{lounsbury_oneida_1953}.} takes a \stem{-h-V-k} \textsc{euph.h}-{\joiner}-\exsc{\modalizer} combination after the stative suffix. (The {\joiner} ‘V’ is either \phonet{-a} {\joinerA} or the same suffix pronounced as a copy of the preceding stative aspect suffix vowel (\phonet{-ę, -ǫ}, see \sectref{Joiner A pronounced as Ǫ or Ę}.)

\ea\label{ex:statvarex10}\stem{\exsc{mood-…}-verb-{\stative}-\exsc{euph.h-V-:k}}
\ea ęyagodeˀnyędę́hsdǫ̱hǫ:k\\
\gll ę-yago-d-e-ˀnyędę́hsd-ǫ̱-h-ǫ-:k\\
 \fut-\exsc{3s.fi.p-{\semireflexive}}-{\joinerE}-measure-{\stative}-\textsc{euph.h}-{\joiner}-{\modalizer}\\
\glt `she will be measuring things'
\cfex{wadeˀnyędę́hsdǫh\\
	\gll w-ad-e-ˀnyędę́hsd-ǫh\\
	\exsc{3s.a-{\semireflexive}}-{\joinerE}-measure-{\stative}\\
	\glt `the act of measuring'}
\ex dęgagwatwę́hę:k\\
\gll d-ę-ga-gwatw-ę́-h-ę-:k\\
 {\dualic}-{\future}-\textsc{3s.a}-hem-{\stative}-\textsc{euph.h}-{\joiner}-{\modalizer}\\
\glt `it will be hemmed'
\cfex{degágwatwęh\\
	\gll de-gá-gwatw-ęh\\
	{\dualic}-\textsc{3s.a}-hem-{\stative}\\
	\glt `a hem'}
\ex ęsade̱hsrǫni̱hsˀǫ́hǫ:k\\
\gll ę-s-ad-e̱-hsrǫni̱hsˀ-ǫ́-h-ǫ-:k\\
 \fut-\exsc{2s.p-{\semireflexive}}-{\joinerE}-ready-{\stative}-\textsc{euph.h}-{\joiner}-{\modalizer}\\
\glt `you will be ready'
\cfex{sadehsrǫníhsˀǫh\\
	\gll s-ad-e-hsrǫníhsˀ-ǫh\\
	\exsc{2s.p-{\semireflexive}}-{\joinerE}-ready-{\stative}\\
	\glt `you are ready'}
\ex ęyagǫnhehgǫ̱hǫ:k \\
\gll ę-yag-ǫnhehg-ǫ̱-h-ǫ-:k\\
 \fut-\textsc{3s.fi.p}-be.sustained-{\stative}-\textsc{euph.h}-{\joiner}-{\modalizer}\\
\glt ‘they (literally, she) will live on’, `what will sustain them'
\cfex{agǫ́nhehgǫh \\
	\gll ag-ǫ́nhehg-ǫh\\
	\textsc{3s.fi.p}-be.sustained-{\stative}\\
	\glt ‘someone lives on it’, `she is sustained by it'}
\z
\z

Finally, example \xref{ex:statvarex38} shows that some verbs can take either of the post-aspect suffix combinations in \tabref{figtab:1:statvar2} (page \pageref{figtab:1:statvar2}).

\ea\label{ex:statvarex38}
\ea a:waga:tró:wi:k\\\label{ex:statvarex38a}
\gll a:-wag-a:t-hró:wi-:-k\\
 {\indefinite}-\textsc{1s.p}-{\semireflexive}-tell-{\stative}-{\modalizer}\\
\glt `I should be talking about it'
\ex ęwagatrówi̱ha:k\\\label{ex:statvarex38b}
\gll ę-wag-at-hrówi̱-h-a-:k\\
 \fut-\textsc{1s.p}-{\semireflexive}-tell-\textsc{euph.h}-{\joinerA}-{\modalizer}\\
\glt `I will be talking about it'
\cfex{aga:tró:wi:\\
	\gll ag-a:t-hró:wi-:\\
	\textsc{1s.p}-{\semireflexive}-tell-{\stative}\\
	\glt `I have told'}
\z
\z


\subsubsection{Stative verbs with ‘easy to’ / ‘hard to’ constructions} \label{Stative verbs with ‘easy to / hard to’ constructions}
The verbs described in the previous section are used in ‘easy to’ \xref{ex:statvarex7} and ‘hard to’ \xref{ex:statvarex8} sentences. 

\ea\label{ex:statvarex7} 
\gll Wę:dó:ˀ da:ga̱hyadǫ̱hsriya̱ˀgǫ́ha:k \\
it.is.hard the.paper.would.be.cut \\
\glt ‘the paper is hard to cut’, ‘it is hard to cut the paper’

\cfex{
\gll d-a:-ga-hyadǫhsr-iya̱ˀg-ǫ́-h-a-:k\\
{\dualic}-{\indefinite}-\textsc{3s.a}-paper-cut-{\stative}-\textsc{euph.h}-{\joinerA}-{\modalizer}\\
}
\z

\ea\label{ex:statvarex8} 
\gll Wagyehsa̱ˀgéh da:ga̱hyadǫ̱hsriya̱ˀgǫ́ha:k \\
it.is.easy the.paper.would.be.cut \\
\glt ‘the paper is easy to cut’, ‘it is easy to cut the paper’
\cfex{d-a:-ga-hyadǫhsr-iya̱ˀg-ǫ́-h-a-:k\\
{\dualic}-{\indefinite}-\textsc{3s.a}-paper-cut-{\stative}-\textsc{euph.h}-{\joinerA}-{\modalizer}\\
}
\z


\section{Post-aspect suffixes (non-tense)} \label{Post-aspect suffixes (non-tense)}
The \textsc{post-aspect} suffixes (slot 4, \tabref{figtab:1:verbsuffixorder}, page \pageref{figtab:1:verbsuffixorder}) described in this section add nuances to the basic meanings of the three main aspects (the habitual, punctual, and stative). These suffixes have no effect on the choice of pronominal prefix.


\subsection{\stem{-ˀs} \textsc{plural} (\plural)} \label{[-ˀs] (stative verb pluralizer)}
The \stem{-ˀs} {\plural} appears at the end of stative verbs (slot 4,  \tabref{figtab:1:verbsuffixorder}, page \pageref{figtab:1:verbsuffixorder}), where it denotes a quality or characteristic that holds of several objects \xxref{ex:vpluralex2}{ex:vpluralex299}.\footnote{Note that the \textsc{plural} suffix is distinct from the \textsc{pluralizer} suffixes described in \sectref{Pluralizers}.}

\ea\label{ex:vpluralex2} \stem{+i:yo-ˀs} nice.\stative-{\plural} ‘several nice or beautiful things’\\
ohona̱ˀdí:yoˀs\\
\gll o-hona̱ˀd-í:yo-ˀs\\
 \textsc{3s.p}-potato-good.{\stative}-{\plural}\\
\glt `nice potatoes'
\cfex{ohona̱ˀdí:yo:\\
	\gll o-hona̱ˀd-í:yo:\\
	\textsc{3s.p}-potato-good.{\stative}\\
	\glt `nice potato'}
\z

\newpage
\ea\label{ex:vpluralex3} \stem{+owa:n-ę-ˀs} big-{\stative}-{\plural} ‘several big things’
\ea ohǫna̱ˀdówanęˀs\\
\gll o-hǫna̱ˀd-ówan-ę-ˀs\\
 \textsc{3s.p}-potato-big-{\stat}-{\plural}\\
\glt `big potatoes'
\cfex{ohǫna̱ˀdówanęh\\
	\gll o-hǫna̱ˀd-ówan-ęh\\
	\textsc{3s.p}-potato-big-{\stative}\\
	\glt `big potato'}
\ex ga̱ˀdrehdowá:nęˀs\\
\gll ga̱-ˀdrehd-owá:n-ę-ˀs\\
 \textsc{3s.a}-car-big-{\stat}-{\plural}\\
\glt `big cars'
\cfex{ga̱ˀdrehdowá:nęh\\
	\gll ga̱-ˀdrehd-owá:n-ęh\\
	\textsc{3s.a}-car-big-{\stative}\\
	\glt `big car'}
\ex ęgahnegowanę́ˀse:k\\
\gll ę-ga-hneg-owan-ę́-ˀs-e:k\\
 \fut-\textsc{3s.a}-water-big-{\stative}-{\plural}-{\zeropunctual.\modalizer}\\
\glt `there will be big bodies of water'
\z
\z

\ea\label{ex:vpluralex4} \stem{-a-ˀs} certain.size.stative-{\plural} ‘several objects of a certain size, age’\\
shęh ní:waˀs \\
\gll shęh ní:-w-aˀs\\
 that {\partitive}-\textsc{3s.a}-size.{\stat}-{\plural}\\
\glt ‘sizes’, `how big they are'
\cfex{ní:waˀ\\
	\gll ní:-w-aˀ\\
	{\partitive}-\textsc{3s.a}-size.{\stative}\\
	\glt `it is of a certain size, age'}
\z


\ea\label{ex:vpluralex} \stem{oˀdę-ˀs} type.of.{\stat}-{\plural}\\
oˀnigǫ̱hsadǫ̱hkgéha:ˀ nigaęnóˀdęˀs\\
\gll oˀnigǫ̱hsadǫ̱hkgéha:ˀ ni-ga-ęn-óˀdę-ˀs\\
 sad.kind {\partitive}-\textsc{3s.a}-song-type.of.{\stat}-{\plural}\\
\glt `blues music'

\cfex{nigaęnóˀdę:\\
	\gll ni-ga-ęn-óˀdę:\\
	{\partitive}-\textsc{3s.a}-song-type.of.{\stative}\\
	\glt `a type of song'}
\z


\ea\label{ex:vpluralex299} \stem{ahd-ǫ-ˀs} resemble-{\stat}-{\plural}\\
shęh niyóhdǫˀs\\
\gll shęh ni-yó-hdǫ-ˀs \\
 that {\partitive}-\textsc{3s.p}-resemble.{\stat}-{\plural}\\
\glt `descriptions'
\cfex{shęh ní:yoht\\
	\gll shęh ní:-yo-ht \\
	that {\partitive}-\textsc{3s.p}-resemble.{\stative}\\
	\glt `how, in what manner'}
\z



As shown in \xref{ex:vpluralex5}, the \stem{-ˀs} {\plural} suffix occurs before the \textsc{\diminutive} suffix. (\stem{-ˀuh} is a variant of the \stem{-ˀah} \textsc{\diminutive}).\footnote{The diminutive form \stem{-ˀuh} is interesting because the vowel is U instead of A (as in \stem{-ˀah}) due to a process of trans-laryngeal harmony (\textsc{tlh}). However, \textsc{tlh} should be blocked by the intervening \stem{-s} \textsc{\plural} in \stem{-u:-s-ˀuh}, as \textsc{tlh} typically only occurs between vowels separated by just one laryngeal sound -- glottal stop <ˀ> or H. For more examples of \textsc{tlh}, see \href{Joiner A pronounced as Ǫ or Ę}{Joiner A pronounced as Ǫ or Ę} \sectref{Joiner A pronounced as Ǫ or Ę}.}

\ea\label{ex:vpluralex5} \stem{-u:-s-ˀuh} small.{\stative}-{\plural}-{\diminutive} ‘several small things’\\
nigehsinú:sˀuh\\
\gll ni-g-e-hsin-ú:-s-ˀuh\\
 {\partitive}-\textsc{1s.a}-{\joinerE}-leg-small-{\plural}-\exsc{\diminutive}\\
\glt `I have two small legs'
\cfex{nigehsinú:ˀuh\\
	\gll ni-g-e-hsin-u:-ˀuh\\
	{\partitive}-\textsc{1s.a}-{\joinerE}-leg-small-{\diminutive}\\
	\glt `I have a small leg'}
\z



\subsection{\stem{-sgǫ:} \textsc{facilitative} ({\facilitative})} \label{[-sgǫ:] (facilitative)}
The \stem{-sgǫ:} \textsc{\facilitative} adds the meaning of ‘being prone to,’ ‘easily’ or ‘tending to’ (\cite[505]{mithun_watewayestanih_1984}). It occurs at the end of habitual or stative verbs \xref{ex:facilex}.

\ea\label{ex:facilex} \stem{verb-\exsc{\habitual/\stative-\facilitative}}
\ea osdagwáęsgǫ:\\
\gll o-sdagw-á-ę-sgǫ:\\
 \textsc{3s.p}-dirt-{\joinerA}-lie.{\stative}-\exsc{\facilitative}\\
\glt `it gets dirty easily'

\cfex{ohsdágwaęˀ\\
	\gll o-hsdágw-a-ęˀ\\
	{\factual}-\textsc{3s.p}-dirty{\joinerA}-\exsc{lie.\stative}\\
	\glt `it is soiled, dirty, stained'}

\ex hodędǫnyáˀdǫ̱hsgǫ:\\
\gll ho-dę-dǫnyáˀd-ǫ̱h-sgǫ:\\
 \textsc{3s.m.p}-{\semireflexive}-make.fun-{\stative}-\exsc{\facilitative}\\
\glt `he is a joker, or happy-go-lucky'

\cfex{hodędǫnyáˀdǫh\\
	\gll ho-dę-dǫnyáˀd-ǫh\\
	\textsc{3s.m.p}-{\semireflexive}-make.fun-{\stative}\\
	\glt `he is making fun of something'}

\ex gęˀgę́trosgǫ:\\ 
\gll ga-iˀgę́tr-o-sgǫ:\\
\textsc{3s.a}-white-submerged.{\stative}-\exsc{\facilitative}\\
\glt ‘it is all white’
\z
\z

In contrast to the above examples, the \stem{-sgǫ:} {\facilitative} just attaches to a bare verb stem in \xref{ex:facilex2}, where it possibly does double duty as a habitual ending.

\ea\label{ex:facilex2} \stem{verb-\exsc{\facilitative}}\\
sadahǫdǫ́sgǫ:\\
\gll s-ad-ahǫdǫ́-sgǫ:\\
 \textsc{2s.a}-{\semireflexive}-ask-{\habitualfacilitative}\\
\glt `you are inquisitive'
\cfex{sadahǫ́:dǫ:\\
	\gll s-ad-ahǫ́:dǫ:\\
	\textsc{2s.a}-{\semireflexive}-ask.{\noaspect}\\
	\glt `you ask'}
\z


\subsection{\stem{-ge:} \textsc{augmentative} ({\augmentative})} \label{[-ge:] (augmentative)}
The \stem{-ge:} \textsc{\augmentative} means ‘big’ or ‘really’. Alternatively, it intensifies the meaning of the verb in some way. It appears after habitual \xref{ex:augmexa}, punctual \xref{ex:augmexb}, or stative \xref{ex:augmexc} suffixes.

\ea\label{ex:augmex} \stem{verb-\exsc{aspect-\augmentative}}
\ea ohsdáhaˀge:\\\label{ex:augmexa}
\gll o-hsdá-haˀ-ge:\\
 \textsc{3s.p}-cry-{\habitual}-{\augmentative}\\
\glt `it is really crying'
\cfex{ga̱hsdá:haˀ\\
	\gll g-a̱hsdá:-haˀ\\
	\textsc{1s.a}-cry-{\habitual}\\
	\glt `I am crying'}
\ex ahęnatgwéni̱ˀge: \\\label{ex:augmexb}
\gll a-hęn-at-gwéni̱-ˀ-ge: \\
{\factual}-\textsc{3ns.m.a}-{\semireflexive}-compete-{\punctual}-{\augmentative}\\
\glt ‘the big win, victory’ (literally, ‘they (males) won big’) 
\cfex{ęhsa:tgwé:niˀ\\
	\gll ę-hs-a:t-gwé:ni-ˀ\\
	{\future}-\textsc{2s.a}-{\semireflexive}-compete-{\punctual}\\
	\glt `you will win'}
\ex ganǫ́hsotge:\\\label{ex:augmexc}
\gll ga-nǫ́hs-ot-ge:\\
 \textsc{3s.a}-house-stand.{\stative}-{\augmentative}\\
\glt `it is a standing big house'
\cfex{ganǫ́hso:t\\
	\gll ga-nǫ́hs-o:t\\
	\textsc{3s.a}-house-stand.{\stative}\\
	\glt `it is a standing house'}
\z
\z

\subsection{\stem{-jihwęh} ‘completely, fully’} \label{[-jihwęh] ‘completely, fully’}
The \stem{-jihwęh} \textsc{completely} suffix is added to stative verbs, where it means ‘completely, fully’.
 
\ea\label{ex:causinex30}
\ea ohę́hji̱hwęh\\
\gll o-hę́h-jihwęh\\
 \textsc{3s.p}-dry.{\stative}-\textsc{completely}\\
\glt `it is really dried out'
\cfex{ohę:\\
	\gll o-hę-:\\
	\textsc{3s.p}-dry-{\stative}\\
	\glt `it is dry'}
\ex ǫgyahdogaˀji̱hwęh\\
\gll ǫgy-ahdogaˀ-ji̱hwęh\\
 \textsc{1d.p}-grown.up-\textsc{completely}\\
\glt `we two are fully grown'
\cfex{ęgahdo:k \\
	\gll ę-g-ahdo:k\\
	{\future}-\textsc{1s.a}-grow.{\zeropunctual}\\
	\glt `I will grow'}
\z
\z

