\chapter{Word suffixes} \label{Word suffixes}
Word suffixes can attach to any type of word -- nouns, verbs, or particles, \tabref{figtab:1:wordsuffix}. (In contrast, noun suffixes (\sectref{Noun suffixes}) only attach to nouns and verb suffixes (\sectref{Verb suffixes}) only attach to verbs.)


In \tabref{figtab:1:wordsuffix}, \textsc{noun} refers to a noun stem potentially followed by suffixes that are unique to nouns. Similarly \textsc{verb} refers to a verb stem potentially followed by suffixes that are unique to verbs.

\begin{table}
\caption{Word suffixes}
\label{figtab:1:wordsuffix}
\begin{tabularx}{\textwidth}{XX}

\lsptoprule
word type & word suffix\\
\midrule
\textsc{noun}

\textsc{verb}

\textsc{particle} & \stem{-ˀgeh} {\on}

\stem{-hneh} \textsc{at}

\stem{-:ˀah, -ˀah} \textsc{\diminutive} 

\stem{-:hah} \textsc{\diminutive} 

\stem{-go:wah} \textsc{\augmentative}

\stem{-gęhę:ˀ} \textsc{\past} 

\stem{-neha:ˀ} \textsc{\customary} 

\stem{-geha:ˀ} \textsc{\customary} 

\stem{-ka:ˀ} \textsc{\customary} 

\stem{-ǫ:weh} \textsc{\typicalizer} 

\stem{-ho:nǫˀ} \textsc{\populative}

\stem{-jih} \textsc{\intensifier}

\stem{-shę:ˀ, -tshę:ˀ} \textsc{decla.kin}\footnote{The community spelling for this suffix is <-sę:ˀ, -tsę:ˀ>.}

\stem{-shǫ:ˀǫh, -shǫ:ˀah} \textsc{\pluralizer}\footnote{The community spelling for this suffix is <-sǫ:ˀǫh, -sǫ:ˀah>.}

\stem{-shǫˀ} \textsc{\pluralizer}\footnote{The community spelling for this suffix is <-sǫˀ>.}\\
\lspbottomrule
\end{tabularx}
\end{table}


Words can take more than one word suffix, as shown in \xref{ex:wdsuforder2}.

\ea\label{ex:wdsuforder2}
\ea Gwahsdǫhó:nǫˀ\\
\gll gwahsdǫ-hó:nǫˀ\\
 Boston-{\populative}\footnotemark{}\\
\glt `Americans'
\ex Gwahsdǫhonǫˀgeh\\
\gll gwahsdǫ-honǫ-ˀgeh\\
 Boston-{\populative}-{\on}\\
\glt `United States'
\ex Gwahsdǫhonǫˀge̱hó:nǫˀ\\
\gll gwahsdǫh-honǫh-ˀge̱-hó:nǫˀ\\
 Boston-{\populative}-{\on}-{\populative}\\
\glt `United States citizen'
\footnotetext{Gwahsdǫh originates from \textit{Boston}, but means ‘America’.}
\z
\z

The \stem{-jih} {\intensifier} and \stem{-go:wah} {\augmentative} endings tend to occur before other word suffixes \xref{ex:wdsuforder1}.

\ea\label{ex:wdsuforder1}
\ea oyajihó:nǫˀ\\
\gll o-ya-ji-hó:nǫˀ\\
 \textsc{3s.p}-other-{\intensifier}-\exsc{\populative}\\
\glt `stranger, alien'
\cfex{oyá:jih\\
	\gll o-yá:-jih\\
	\textsc{3s.p}-other-{\intensifier}\\
	\glt `another type'}
\ex gi̱hęˀgowáhneh\\
\gll gi̱hęˀ-gowá-hneh\\
 river-{\augmentative}-\exsc{at}\\
\glt `to the big river'
\cfex{gi̱hę:gó:wah\\
	\gll gi̱hę:-gó:wah\\
	river-{\augmentative}\\
	\glt `the big river'}
\z
\z

Word suffixes are described in the following sections.


\section{\stem{-ˀgeh} {\on} (external locative)} \label{-ˀgeh}
The \stem{-ˀgeh} {\on} suffix denotes an outside location (‘on’ or ‘at’). It attaches to basic nouns \xref{ex:locex}, body part nouns \xref{ex:locex2}, verbs \xref{ex:locex3}, and several particles \xref{ex:locex4}.

\ea\label{ex:locex} basic nouns with \stem{-ˀgeh}
\ea ahsgwáˀgeh\\
\gll ahsgwá-ˀgeh\\
 roof-{\on}\\
\glt `on the roof'
\cfex{ahsgwaˀ \\
\gll ahsgw-aˀ\\
roof-{\nsf}\\
\glt ‘the roof’  }
\ex oíhwa̱ˀgeh\\
\gll o-ríhwa̱-ˀgeh\\
 \textsc{3s.p}-matter-{\on}\\
\glt `the reason, idea for something'
\cfex{oíhwaˀ\\
	\gll o-ríhw-aˀ\\
	\textsc{3s.p}-matter-{\nsf}\\
	\glt `message,word, affair, business'}
\ex adę́ˀhęˀgeh\\
\gll adę́ˀhę-ˀgeh\\
 fence-{\on}\\
\glt `on the fence'
\cfex{adę́ˀhęˀ\\
	\gll adę́ˀh-ęˀ\\
	fence-{\nsf}\\
	\glt `fence'}
\z
\z

\ea\label{ex:locex2} body part nouns with \stem{-ˀgeh}
\ea snętsáˀgeh\\
\gll s-nętsá-ˀgeh\\
 \textsc{2s.a}-arm-{\on}\\
\glt `on your (sg.) arm'
\ex knętsáˀgeh\\
\gll k-nętsá-ˀgeh\\
 \textsc{1s.a}-arm-{\on}\\
\glt `on my arm'
\ex enę́tsa̱ˀgeh\\
\gll e-nę́tsa̱-ˀgeh\\
 \textsc{3s.fi.a}-arm-{\on}\\
\glt `on her arm'
\z
\z

\ea\label{ex:locex3} verbs with \stem{-ˀgeh}
\ea otówe̱ˀgeh\\
\gll o-tówe̱-ˀgeh\\
 \textsc{3s.p}-cold.{\stative}-{\on}\\
\glt `North'
\cfex{otó:weˀ\\
	\gll o-tó:weˀ\\
	\textsc{3s.p}-cold.{\stative}\\
	\glt `it is cold'}
\ex onénǫ̱ˀgeh\\
\gll o-nénǫ̱-ˀgeh\\
 \textsc{3s.p}-warm.{\stative}-{\on}\\
\glt `South'
\cfex{oné:nǫˀ\\
	\gll o-né:nǫˀ\\
	\textsc{3s.p}-warm.{\stative}\\
	\glt `it is mild, warm'}
\ex eˀnikǫhkwáˀgeh\\
\gll e-ˀnikǫhkw-há-ˀgeh\\
 \textsc{3s.fi.a}-sew-{\habitual}-{\on}\\
\glt `sewing room'
\cfex{eˀníkǫhkwaˀ\\
	\gll e-ˀníkhǫkw-haˀ\\
	\textsc{3s.fi.a}-sew-{\habitual}\\
	\glt `she sews'}
\ex ekǫnya̱ˀtáˀgeh\\
\gll e-kǫnya̱ˀt-há-ˀgeh\\
 \textsc{3s.fi.a}-cook-{\habitual}-{\on}\\
\glt `kitchen'
\cfex{ekǫ́nya̱ˀtaˀ\\
	\gll e-kǫ́nya̱ˀt-haˀ\\
        \textsc{3s.fi.a}-cook-{\habitual}\\
	\glt `she cooks'}
\ex Ga̱hnawiyóˀgeh\\
\gll ga̱-hnaw-iyó-ˀgeh\\
 \textsc{3s.a}-flowing.water-good.{\stative}-{\on}\\
\glt `Oklahoma'
\cfex{ga̱hna:wí:yo:\\
	\gll ga̱-hna:w-í:yo:\\
	\textsc{3s.a}-flowing.water-good.{\stative}\\
	\glt `nice rapids'}
\z
\z

\ea\label{ex:locex4} particles with \stem{-ˀgeh}\\
í:ˀgeh \\
\gll i:-ˀgeh\\
 I-{\on}\\
\glt ‘on my side’, `as for me'
\z


\section{\stem{-hneh} \textsc{at} (external locative)} \label{-hneh}
The \stem{-hneh} \textsc{at} suffix denotes a location in place or time. It attaches to atypical words functioning as “nouns” \xref{ex:locex8} and verbs functioning as “nouns” \xref{ex:locex9}. (Examples of \stem{-hneh} \textsc{at} attaching to basic nouns or body part nouns are not attested.)

\ea\label{ex:locex8} \textsc{atypical} words functioning as “nouns”
\ea gi̱hęˀgowáhneh\\
\gll gi̱hęˀ-gowá-hneh\\
 river-{\augmentative}-\exsc{at}\\
\glt `to the big river'
\cfex{gi̱hę:gó:wah\\
	\gll gi̱hę:-gó:wah\\
	river-{\augmentative}\\
	\glt `the big river'}
\ex hęnǫ́gwe̱hneh \\
\gll hęn-ǫ́gwe̱-hneh\\
\textsc{3ns.m.a}-people-\exsc{at}\\
\glt ‘men’s washroom’ 
\cfex{hęnǫ́:gweh\\
	\gll hęn-ǫ́:gweh\\
	\textsc{3ns.m.a}-people\\
	\glt `men'}
\ex gęnhéhneh\\
\gll g-ęnhé-hneh\\
 \textsc{3s.a}-summer-\exsc{at}\\
\glt `summer'
\ex Gwésa̱hsneh\\
\gll gwésa̱hs-hneh\\
 partridge-\exsc{at}\\
\glt `Akwesasne' (place name)
\ex go̱hsréhneh\\
\gll go̱hsré-hneh\\
 cold.place-\exsc{at}\\
\glt `winter'
\ex Hyáikneh\\
\gll hyá-ik-hneh\\
 berry-ripen-\exsc{at}\\
\glt `June'
\ex Jíhsgę̱hneh\\
\gll jíhsgę̱-hneh\\
 corn.husk-\exsc{at}\\
\glt `August'
\ex Sáˀgę̱hneh\\
\gll sáˀgę̱-hneh\\
 cough-\exsc{at}\\
\glt `September'
\ex Saˀgę̱hnehgó:wah\\
\gll saˀgę̱-hneh-gó:wah\\
 cough-\exsc{at-\augmentative}\\
\glt `October'
\z
\z

\ea\label{ex:locex9} \textsc{verbs} functioning as “nouns”
\ea Ohwęjagáyǫ̱hneh\\
\gll o-hwęja-gáyǫ̱-hneh\\
 \textsc{3s.p}-land-old.{\stative}-\exsc{at}\\
\glt `Europe'
\z
\z

\section{\stem{-ˀgeh} {\on} versus \stem{-hneh} \textsc{at}} \label{-ˀgeh versus -hneh}
The \stem{-ˀgeh} {\on} and \stem{-hneh} \textsc{at} suffixes both mean ‘on’, ‘at’, or ‘to’, but are used in different contexts. The \stem{-ˀgeh} {\on} suffix attaches to basic nouns and body part nouns \xref{ex:gehneh6} while \stem{-hneh} \textsc{at} does not appear to \xref{ex:gehneh}.\footnote{It may be that \stem{-hneh} \textsc{at} attaches to words ending with a vowel sound, while \stem{-ˀgeh} {\on} attaches to words ending with a consonant. However, some of the words in \xxref{ex:gehneh}{ex:gehneh5} are counterexamples.}

\newpage
\ea\label{ex:gehneh6} \stem{-ˀgeh} {\on} with basic and body part nouns
\ea gáǫda̱ˀgeh\\
\gll gá-ǫd-a̱-ˀgeh\\
 \textsc{3s.a}-log-{\joinerA}-{\on}\\
\glt `on the log'
\ex gwę̱ˀnóhsa̱ˀgeh\\
\gll g-wę̱ˀnóhs-a-ˀgeh\\
 \textsc{1s.a}-tongue-{\joinerA}-{\on}\\
\glt `on my tongue'
\z
\z

Meanwhile, either suffix can be used with personal names \xref{ex:gehneh}.

\ea\label{ex:gehneh} \stem{-hneh} \textsc{at} and \stem{-ˀgeh} {\on} after personal names
\ea Arniegeh, Arnihneh \\
\gll Arnie-ˀgeh/hneh \\
Arnie-\textsc{\on/at}\\
\glt ‘at Arnie’s place’


\ex Alvahneh, Alvaˀgeh\\
\gll Alva-ˀgeh/hneh\\
Alva-\textsc{\on/at}\\
\glt ‘at Alva’s place’

\ex Elsahneh\\
\gll Elsa-hneh \\
Elsa-\textsc{at}\\
\glt ‘at Elsa’s place’

\ex Mariaˀgeh \\
\gll Maria-ˀgeh\\
Maria-\textsc{\on}\\
\glt ‘at Maria’s place’

\ex Tómhneh\\
\gll Tom-hneh\\
Tom-\textsc{at}\\
\glt ‘at Tom’s place’

\ex Sawęhe:sgeh \\
\gll Sawęhe:s-ˀgeh\\
 Sawęhe:s-\textsc{\on}\\
 \glt  ‘at Sawęhe:s’s place’ (\exsc{verb} functioning as a “name”)

\ex Jiˀdá:wi:sgeh\\
\gll jiˀdá:wi:s-ˀgeh\\
  Jiˀdá:wi:s-\textsc{\on}\\
\glt ‘at Jiˀdawís’s place’ (\exsc{verb} functioning as a “name”)\\


\ex Alfredgeh\\
\gll Alfred-ˀgeh\\
Alfred-\textsc{\on}\\
\glt ‘at Alfred’s place’
\z
\z


Finally, the \stem{-hneh} \textsc{at} ending also attaches to atypical nouns \xref{ex:gehneh5a}. In (\ref{ex:gehneh5}b, c), both \stem{-hneh} \textsc{at} and \stem{-ˀgeh} {\on} appear in the same word, but in different order.

\ea\label{ex:gehneh5}
\ea hyáikneh \\\label{ex:gehneh5a}
\gll hya-ik-hneh\\
berry-ripen-\textsc{at/on}\\
\glt ‘June’ (in the past) (\exsc{atypical word} functioning as a “noun”) 
\ex hyáikhne̱hgeh \\\label{ex:gehneh5b}
\gll hya-ik-hne̱h-ˀgeh\\
berry-ripen-\textsc{at/on}\\
\glt ‘berry ripening time’ (in the past) (\exsc{atypical word} functioning as a “noun”)

\ex gęnęnagéhneh \\\label{ex:gehneh5c}
\gll g-ęnęn-agé-hneh\\
\textsc{3s.a}-fall-\exsc{on-at}\\
\glt ‘fall’ (season) 
\z
\z

\section{\stem{-:ˀah} \textsc{diminutive} \textup{(}{\diminutive}\textup{)}} \label{-:ˀah}
The  \stem{-:ˀah} {\diminutive} (denoting \textit{smallness} and related concepts described below) attaches to atypical words functioning as “nouns” \xref{ex:ahex}, verbs proper and verbs functioning as “nouns” \xref{ex:ahex7}, kinship terms \xref{ex:ahex6}, and basic nouns \xref{ex:ahex2a}. The vowel before the suffix is long, except with kinship terms \xref{ex:ahex6}.

\ea\label{ex:ahex} atypical words functioning as “nouns”\\
gwa̱ˀyę́:ˀah\\\label{ex:ahexa}
\gll gwa̱ˀyę́:-ˀah\\
 rabbit-{\diminutive}\\
\glt `cottontail rabbit'
\cfex{gwaˀyǫˀ \\
\glt ‘rabbit’}
\z


\ea\label{ex:ahex7} verbs (a-b), and verbs functioning as “nouns” (c-f)
\ea niyǫ́:sˀah\\ \label{ex:ahex7a}
\gll ni-y-ǫ́:s-ˀah\\
 {\partitive}-\textsc{3s.p}-long.{\stative}-{\diminutive}\\
\glt `just a little bit long'
\cfex{í:yǫ:s\\
	\gll í:-y-ǫ:s\\
	{\partitive}-\textsc{3s.p}-long.{\stative}\\
	\glt `it is long'}
\ex ohsno:wé:ˀah \\
\gll o-hsno:wé:-ˀah\\
 \textsc{3s.p}-fast.{\stative}-{\diminutive}\\
\glt `to arrive a little bit early'
\cfex{ohsnó:weˀ\\
	\gll o-hsnó:weˀ\\
	\textsc{3s.p}-fast.{\stative}\\
	\glt ‘it is fairly fast’, `it is fast, quick'}
\ex nigęhné:sˀah\\ \label{ex:ahex7c}
\gll ni-g-ęhn-é:s-ˀah\\
 {\partitive}-\textsc{3s.a}-cloth-long.{\stative}-{\diminutive}\\
\glt `a short length of cloth'
\cfex{tsǫ́: nigę́hne:s\\
	\gll tsǫ́: ni-g-ę́hn-e:s \\
	just {\partitive}-\textsc{3s.a}-cloth-long.{\stative}\\
	\glt `just a length of cloth'}
\ex Nigahnegá:ˀah\\
\gll ni-ga-hneg-á:-ˀah\\
 {\partitive}-\textsc{3s.a}-water-hold.{\stative}-{\diminutive}\\
\glt `Little Water Medicine' (name of a ceremony)

\ex aˀonihshé:ˀah\\
\gll aˀ-o-nihshé:-ˀah\\
 \textsc{factual-3s.p}-take.time.\textsc{\punctual-\diminutive}\\
\glt `a short duration'
\cfex{to naˀónisheˀ  \\
\gll to n-aˀ-ó-nishe-ˀ\\
that {\partitive}-{\factual}-\textsc{3s.p}-take.time-{\punctual}\\
\glt `it took that long’}
\ex haksaˀdasé:ˀah\\
\gll ha-ks-a-ˀd-asé:-ˀah\\
 \textsc{3s.m.a}-child-{\joinerA}-\textsc{increment}-new.{\stative}-{\diminutive}\\
\glt `teenager'
\cfex{haksá:ˀah\\
	\gll ha-ks-á:-ˀah\\
	\textsc{3s.m.a}-child-{\nounstemformer}-{\diminutive}\\
	\glt `a male child, boy'}
\z
\z

\ea\label{ex:ahex6} kinship terms
\ea hehjíˀah\\
\gll he-hjí-ˀah\\
 \textsc{1s:3ms}-older.sibling-{\diminutive}\\
\glt `my older brother'
\cfex{hehjiˀ\\
	\gll he-hjiˀ\\
	\textsc{1s:3ms}-older.sibling\\
	\glt `my older brother' (affectionate term)}
\ex kehjíˀah\\
\gll keh-jí-ˀah\\
 \textsc{1s:3fis}-older.sibling-{\diminutive}\\
\glt `my older sister'
\cfex{kehjiˀ \\
	\gll ke-hjiˀ\\
	\textsc{1s:3fis}-older.sibling\\
	\glt `my older sister' (affectionate term)}
\z
\z

The \stem{-:ˀah} {\diminutive} suffix conveys smallness, a small amount, a short segment of time, and related concepts \xref{ex:ahex2}, also (\ref{ex:ahex7a},c).

\ea\label{ex:ahex2} smallness
\ea owi:yá:ˀah\\\label{ex:ahex2a}
\gll o-wi:yá:-ˀah\\
 \textsc{3s.p}-offspring-{\diminutive}\\
\glt `baby'
\cfex{neˀ owí:yaˀ  \\
	\gll neˀ o-wí:y-aˀ\\
	the \textsc{3s.p}\exsc{-}offspring-{\nsf}\\
	\glt ‘its young, offspring’}
\newpage
\ex ohstwáhsˀah\\\label{ex:ahex2b}
\gll o-hstwá-hs-ˀah\\
 \textsc{3s.p}-shrink-{\habitual}-{\diminutive}\\
\glt `young animal, little animal, tadpole'
\cfex{wahstwahs\\
	\gll wa-hstwa-hs\\
	\textsc{3s.a}-shrink-{\habitual}\\
	\glt `it shrinks'}
\ex oˀgá:sˀah\\\label{ex:ahex2e}
\gll o-ˀgá:s-ˀah\\
 \textsc{3s.p}-evening.{\stative}-{\diminutive}\\
\glt `evening'
\cfex{deyóˀga:s \\
	\gll de-yó-ˀga:s\\
	{\dualic}-\textsc{3s.p}-evening.{\stative}\\
	\glt ‘the night is coming’, `early night'}
\z
\z

The \stem{-:ˀah} {\diminutive} suffix also denotes the diminished degree or importance of some activity \xref{ex:ahex4}.

\ea\label{ex:ahex4}
\ea ihé:ˀah\\
\gll i-h-é:-ˀah\\
 {\prothetic}-\textsc{3s.m.a}-go.{\stative}-{\diminutive}\\
\glt `he is just walking along'
\cfex{iheˀ\\
	\gll i-h-e-ˀ\\
	{\prothetic}-\textsc{3s.m.a}-go-{\stative}\\
	\glt `he is walking, moving'}
\ex a:wé:tˀah\\
\gll a:-w-é:t-ˀah\\
 {\indefinite}-\textsc{3s.a}-think.{\causative}-{\diminutive}\\
\glt `it is implied, pretend'
\cfex{á:we:ˀ\\
	\gll á:-w-e:-ˀ\\
	{\indefinite}-\textsc{3s.a}-think-{\punctual}\\
	\glt `it should want, think'}
\z
\z

The \stem{-:ˀah} {\diminutive} ending and the \stem{ti-} \textsc{\contrastive} prefix together denote \emph{oddness} (\ref{ex:ahex8}, see \sectref{[ti-] (contrastive)}).

\newpage
\ea\label{ex:ahex8} \stem{ti-verb-ˀah} ‘odd’, ‘strange’, ‘unusual’
\ea tiyó:tˀah\\
\gll ti-yó:-t-ˀah\\
 {\contrastive}-\textsc{3s.p}-stand.{\stative}-{\diminutive}\\
\glt `it is queer, unusual, odd'
\ex tiganǫ̱hsa:dé:ˀah\\
\gll ti-ga-nǫ̱hs-a:dé:-ˀah\\
 {\contrastive}-\textsc{3s.a}-house-exist.{\stative}-{\diminutive}\\
\glt `not a normal house'
\ex tęhayętó:ˀah\\
\gll t-ę-ha-yętó:-ˀah\\
 {\contrastive}-{\future}-\textsc{3s.m.a}-plant.{\punctual}-{\diminutive}\\
\glt `he will just plant it any old way or place, helter-skelter'
\z
\z

Finally, the \stem{-:ˀs-ˀah} \textsc{\plural-\diminutive} combination means `a diminished degree of some quality' (\ref{ex:ahex3}a, b).\footnote{The \textsc{plural} ({\plural}) and \textsc{pluralizer} ({\pluralizer}) suffixes are distinct. The \stem{-ˀs, -s} \textsc{plural} attaches only to \textsc{stative} verbs and is described in \sectref{Verb suffixes}. In contrast, the \stem{-shǫ:ˀǫh, -shǫ:ˀah} and \stem{-shǫˀ} \textsc{pluralizer} suffixes, described in this chapter, attach to any category of word.}

\ea\label{ex:ahex3} \stem{stative.only.verb-:ˀs-ˀah} \textsc{\plural-\diminutive}
\ea hahe:tgę́:sˀah\\\label{ex:ahex3a}
\gll ha-he:tgę́:-s-ˀah\\
 \textsc{3s.m.a}-ugly.{\stative}-{\plural}-\exsc{\diminutive}\\
\glt `he is unattractive'
\cfex{hahé:tgęˀ\\
	\gll ha-hé:tgęˀ\\
	\textsc{3s.m.a}-ugly.{\stative}\\
	\glt `he is ugly'}
\ex onǫhsatgí:sˀah\\\label{ex:ahex3b}
\gll o-nǫhs-a-tgí:-s-ˀah\\
 \textsc{3s.p}-house-{\joinerA}-ugly.{\stative}-{\plural}-\exsc{\diminutive}\\
\glt `ugly house'
\cfex{onǫ́hsatgiˀ\\
	\gll o-nǫ́hs-a-tgiˀ\\
	\textsc{3s.p}-house-{\joinerA}-ugly.{\stative}\\
	\glt `dirty house'}
\z
\z


\section{\stem{-:hah} \textsc{diminutive} ({\diminutive})} \label{-:hah}
The \stem{-:hah} {\diminutive} suffix means ‘fairly’, ‘somewhat’ or ‘a little bit’ \xref{ex:hahex}. It attaches to verbs \xref{ex:hahex}, atypical verbs functioning as “nouns” \xref{ex:hahex2a}, and some particles \xref{ex:hahex4}. The vowel before the suffix is always long.

\ea\label{ex:hahex} verbs
\ea ga̱hwajiyowanę́:hah\\
\gll ga-hwajiy-owanę́:-hah\\
 \textsc{3s.a}-family-big.{\stative}-{\diminutive}\\
\glt `my family is fairly big'
\cfex{ga̱hwajiyówanęh\\
	\gll ga-hwajiy-ówanęh\\
	\textsc{3s.a}-family-big.{\stative}\\
	\glt `big family'}
\ex niyǫ́:hah\\
\gll ni-y-ǫ́:-hah\\
 {\partitive}-\textsc{3s.p}-certain.amount.{\stative}-{\diminutive}\\
\glt `few, a little bit'
\cfex{ní:yǫ:\\
	\gll ní:-y-ǫ:\\
	{\partitive}-\textsc{3s.p}-certain.amount.{\stative}\\
	\glt `amount of things'}
\ex sgęnǫgowá:hah\\
\gll s-g-ęnǫ-gowá:-hah\\
 {\repetitive}-\textsc{3s.a}-originate.from-{\augmentative}-{\diminutive}\\
\glt `slowly, fairly slow'
\cfex{sgę:nǫgó:wah \\
    \gll s-g-ęnǫ-gó:wah\\
        {\repetitive}-\textsc{3s.a}-originate.from-{\augmentative}\\
    \glt ‘really well’ }
\ex gowa:nę́:hah\\
\gll g-owa:n-ę́:-hah\\
 \exsc{3s}-big-{\stative}-{\diminutive}\\
\glt `it is fairly big'
\cfex{gowá:nęh\\
	\gll g-owá:n-ęh\\
	\exsc{3s}-big-{\stative}\\
	\glt `it is big'}
\ex ihé:hah\\
\gll i-h-é:-hah\\
 {\prothetic}-\textsc{3s.m.a}-think.{\stative}-{\diminutive}\\
\glt `he thinks highly of himself'
\cfex{ ihe: \\
\gll i-h-e:\\
{\prothetic}-\textsc{3s.m.a}-think.{\stative}\\
\glt ‘he wants, thinks’}
\z
\z

With expressions of time, the \stem{-:hah} {\diminutive} ending denotes an \emph{earlier} time. \xref{ex:hahex2}

\ea\label{ex:hahex2}
\ea shede̱hjí:hah\\\label{ex:hahex2a}
\gll  shede̱h-jí:-hah \\
early.morning-{\intensifier}-{\diminutive}\\
\glt ‘early morning’ 
\cfex{tsishéhde̱hjih\\
	\gll tsi-shéde̱h-jih\\
	{\coincident}-early.morning-{\intensifier}\\
	\glt `this morning'}
\ex gagwide̱hjí:hah\\\label{ex:hahex2b}
\gll ga-gwide̱h-jí:-hah\\
\textsc{3s.a}-early.spring-{\intensifier}-{\diminutive}\\
\glt ‘early spring’ 

\cfex{gagwíde̱hneh\\
	\gll ga-gwíde̱-hneh\\
	\textsc{3s.a}-early.spring-\exsc{at}\\
	\glt `springtime, in the spring'}
\ex ęgyoˀgá:hah\\\label{ex:hahex2c}
\gll ę-g-yo-ˀgá:-hah \\
\fut-{\cislocative}-\textsc{3s.p}-evening.{\punctual}-{\diminutive}\\
\glt ‘early night’ 

\cfex{ęyóˀga:ˀ\\
	\gll ę-yó-ˀga:-ˀ\\
	\fut-\textsc{3s.p}-evening-{\punctual}\\
	\glt `tonight'}
\z
\z

The \stem{-:hah} {\diminutive} ending also denotes an \emph{approximate} time (\ref{ex:hahex20}. Note the stacked {\diminutive}-\textsc{at}-{\diminutive} suffixes in this example).

\newpage
\ea\label{ex:hahex20}
oˀga:sˀahné:hah\\
\gll o-ˀga:s-ˀah-hné:-hah\\
 \textsc{3s.p}-evening.{\stative}-{\diminutive}-\exsc{at-\diminutive}\\
\glt `eveningish'
\cfex{oˀgá:sˀah\\
	\gll o-ˀgá:s-ˀah\\
	\textsc{3s.p}-evening.{\stative}-{\diminutive}\\
	\glt `evening'}
\z


The \stem{-:hah} {\diminutive} suffix attaches to words describing various stages of being human \xref{ex:hahex3}.
 
\ea\label{ex:hahex3} verbs functioning as “nouns”
\ea hagęhjí:hah\\
\gll ha-gęhjí:-hah\\
\textsc{3s.m.a}-old.{\stative}-{\diminutive}\\
\glt ‘he is getting old’ 

\cfex{hagę́hjih\\
	\gll ha-gę́hjih\\
	\textsc{3s.m.a}-old.{\stative}\\
	\glt `old man'}
\ex nitawe:nǫ́:hah\\
\gll ni-t-haw-e:nǫ́:-hah\\
{\partitive}-{\cislocative}-\textsc{3s.m.p}-originate.from.{\habitual}-{\diminutive}\\
\glt ‘middle-aged male’\\

\cfex{nita:wé:nǫh \\
\gll ni-t-ha:w-é:nǫ-h \\
{\partitive}-{\cislocative}-\textsc{3s.m.p}-originate.from-{\habitual}\\
\glt ‘stranger’ (literally, ‘where he comes from’)
}
\ex niyagawe:nǫ́:hah\\
\gll ni-yagaw-e:nǫ́:-hah\\
{\partitive}-{\cislocative}-\textsc{3s.fi.p}-originate.from.{\habitual}-{\diminutive}\\
\glt ‘middle-aged female’
\cfex{niyagawé:nǫh\\
	\gll ni-yagaw-é:nǫ-h\\
	{\partitive}-{\cislocative}-\textsc{3s.fi.p}-originate.from-{\habitual}\\
	\glt `female stranger'}
\z
\z

Finally, when attached to particles, the \stem{-:hah} ending means `an indefinite place or amount' \xref{ex:hahex4}.

\newpage
\ea\label{ex:hahex4} particles
\ea ga̱ˀ tó:hah\\
\gll gaˀ tó:hah\\
 indefinite.location there-{\diminutive}\\
\glt `somewhere'
\cfex{ga̱ˀ toh \\
\gll gaˀ toh\\
indefinite.location there\\
\glt ‘anywhere’  }
\ex stǫ́:hah\\
\gll stǫ́:-hah\\
 particle-{\diminutive}\footnotemark{}\\
\footnotetext{While \stem{-hah} is the {\diminutive} suffix, the meaning of \stem{stǫ́:} element is unclear.}
\glt `a little bit'
\z
\z

\section{\stem{-go:wah} \textsc{augmentative} \textup{(}{\augmentative}\textup{)}} \label{-go:wah}
The \stem{-go:wah} \textsc{\augmentative} ending means ‘big’, or ‘great’. It attaches to nouns of any type \xref{ex:gowex} and to verbs functioning as “nouns” \xref{ex:gowex2}.
 
\ea\label{ex:gowex} nouns (basic and atypical)
\ea dagu:sgó:wah\\
\gll dagu:s-gó:wah\\
 cat-\textsc{\augmentative}\\
\glt `big cat'
\cfex{dagu:s\\
	\glt `cat'}
\ex onǫhdagó:wah\\
\gll o-nǫhd-a-gó:wah\\
 \textsc{3s.p}-bur-{\joinerA}-{\augmentative}\\
\glt `burdock'
\cfex{onǫ́hdaˀ\\
	\gll o-nǫ́hd-aˀ\\
	\textsc{3s.p}-bur-{\nsf}\\
	\glt `bur'}
\ex otręˀdagó:wah\\
\gll o-tręˀd-a-gó:wah\\
 \textsc{3s.p}-fly-{\joinerA}-{\augmentative}\\
\glt `horsefly'
\cfex{otrę́ˀdaˀ\\
	\gll o-trę́ˀd-aˀ\\
	\textsc{3s.p}-fly-{\nsf}\\
	\glt `fly'}
\ex gwa̱ˀyǫgó:wah\\
\gll gwa̱ˀyǫ-gó:wah\\
 rabbit-\textsc{\augmentative}\\
\glt `jackrabbit'
\cfex{gwa̱ˀyǫˀ\\
	\glt `rabbit'}
\ex Gayane̱hsraˀgó:wah\\
\gll ga-yane̱hsr-aˀ-gó:wah\\
 \textsc{3s.a}-law-{\joinerA}-{\augmentative}\\
\glt `the Great Law'
\cfex{gayáne̱hsraˀ\\
	\gll ga-yane̱hsr-aˀ\\
	\textsc{3s.a}-law-{\nsf}\\
	\glt  ‘rights’, ‘laws’, `code'}
\ex ohstowagó:wah\\
\gll o-hstow-a-gó:wah\\
 \textsc{3s.p}-feather-{\joinerA}-{\augmentative}\\
\glt `Great Feather Dance'
\ex awęheˀgó:wah\\
\gll awęheˀ-gó:wah\\
 wing-{\augmentative}\\
\glt `eagle'
\z
\z

\ea\label{ex:gowex2} \textsc{verbs} functioning as “nouns”
\ea ga̱hnyaˀsesgó:wah\\
\gll ga̱-hnyaˀs-es-gó:wah\\
 \textsc{3s.a}-neck-long.{\stative}-{\augmentative}\\
\glt `giraffe'
\cfex{ga̱hnyáˀse:s\\
	\gll ga̱-hnyáˀs-e:s\\
	\textsc{3s.a}-neck-long.{\stative}\\
	\glt `it has a long neck'}
\ex okdehatgiˀgó:wah\\
\gll o-kdeh-a-tgiˀ-gó:wah \\
\textsc{3s.p}-root-{\joinerA}-ugly.{\stative}-{\augmentative}\\
\glt ‘great distorted root’ (a herb)

\cfex{okdéhatgiˀ\\
	\gll o-kdéh-a-tgiˀ\\
	\textsc{3s.p}-root-{\joinerA}-ugly.{\stative}\\
	\glt `dirty or ugly root'}
\ex gayęˀgwasgó:wah\\
\gll ga-yęˀgw-a-k-s-gó:wah\\
 \textsc{3s.a}-tobacco-{\joinerA}-eat-{\habitual}-{\augmentative}\\
\glt `tobacco worm'
\cfex{gayę́ˀgwa:s\\
	\gll ga-yęˀgw-a-k-s\\
	\textsc{3s.a}-tobacco-{\joinerA}-eat-{\habitual}\\
	\glt `it eats tobacco'}
\ex gawęno̱hgriya̱ˀsgó:wah\\
\gll ga-węno̱hgr-iya̱ˀk-s-gó:wah\\
 \textsc{3s.a}-weed-cut-{\habitual}-{\augmentative}\\
\glt `weed or brush cutter'
\cfex{gawęno̱hgrí:yaˀs\\
	\gll ga-węno̱hgr-í:yaˀk-s\\
	\textsc{3s.a}-weed-cut-{\habitual}\\
	\glt `it cuts weeds'}
\z
\z

\section{\stem{-gęhę:ˀ} \textsc{past}} \label{-gęhę:ˀ}
The \stem{-gęhę:} \textsc{past} ending attaches to verbs \xref{ex:gehex}, verbs functioning as “nouns” \xref{ex:gehex30}, basic nouns \xref{ex:gehex2} and kinship terms \xref{ex:gehex3}. With verbs, \stem{-gęhę:} denotes a \emph{former} time \xref{ex:gehex}.

\ea\label{ex:gehex} verbs
\ea ní:ˀ agawęgę́hę:ˀ\\
\gll ní:ˀ ag-aw-ę-gę́hę:ˀ\\
 I \textsc{1s.p}-own-{\stative}-{\past}\\
\glt  ‘it used to be mine’
\cfex{agá:węh\\
	\gll ag-á:w-ęh\\
	\textsc{1s.p}-own-{\stative}\\
	\glt `mine'}
\newpage
\ex howę́gę̱hę:ˀˀ \\
\gll ho-wę-gę̱hę:ˀˀ\\
\textsc{3s.m.p}-own.{\stative}-{\stative}-{\past}\footnotemark{}\\
\footnotetext{Also, 
\ea howę́hne:ˀ\\
    \gll ho-w-ę́-hne:ˀ\\
    \textsc{3s.m.p}-own-{\stative}-{\remote}\\
    \glt ‘it used to be his’
    \z
 }
 \glt ‘it used to be his’
\cfex{hó:węh\\
	\gll hó:-w-ęh\\
	\textsc{3s.m.p}-own-{\stative}\\
	\glt `his'}
\z
\z

For verbs functioning as “nouns” \xref{ex:gehex30}, basic nouns \xref{ex:gehex2}, and kinship terms \xref{ex:gehex3}, the \stem{-gęhę:} ending tends to mean ‘former’ or ‘late’.

\ea\label{ex:gehex30} \textsc{verbs} functioning as “nouns”\\
eksaˀdasé:gę̱hę:ˀˀ\\
\gll e-ks-a-ˀd-asé:-gę̱hę:ˀˀ\\
 \textsc{3s.fi.a}-child-{\joinerA}-\textsc{increment}-new.{\stative}-{\past}\\
\glt `a former teenage girl'
\cfex{eksaˀdasé:ˀah\\
	\gll e-ks-a-ˀd-asé:-ˀah\\
	\textsc{3s.fi.a}-child-{\joinerA}-\textsc{increment}-new.{\stative}-{\diminutive}\\
	\glt `teenage girl'}
\z


\ea\label{ex:gehex2} basic nouns
\ea oˀnhǫhsagę́hę:ˀ \\
\gll o-ˀnhǫhs-a-gę́hę:ˀ \\
\textsc{3s.p}-egg-{\joinerA}-{\past}\\
\glt ‘it used to be an egg’ (e.g. a broken robin’s egg or a dinosaur’s egg)

\cfex{oˀnhǫ́hsaˀ\\
	\gll o-ˀnhǫ́hs-aˀ\\
	 \textsc{3s.p}-egg-{\nsf}\\
	\glt `egg'}
\ex akyęda̱hkwáˀgę̱hę:ˀˀ\\
\gll akyęda̱hkw-áˀ-gę̱hę:ˀˀ\\
 chair-{\joinerA}-{\past}\\
\glt `a former chair'
\ex gokyęda̱hkwáˀgę̱hę:ˀˀ\\
\gll go-kyęda̱hkw-áˀ-gę̱hę:ˀˀ\\
 \textsc{3s.fi.p}-chair-{\joinerA}-{\past}\\
\glt `it used to be her chair'
\cfex{akyę́da̱hkwaˀ\\
	\gll akyę́da̱hkw-aˀ\\
	chair-{\nsf}\\
	\glt `chair'}
\z
\z

\ea\label{ex:gehex3} kinship terms
\ea knoha:ˀahgęhę:ˀ\\
\gll k-noha:ˀah-gęhę:ˀ\\
 \textsc{1s.a}-aunt-{\past}\\
\glt `my late aunt'
\cfex{kno:há:ˀah\\
	\gll k-no:há:ˀah\\
	\textsc{1s.a}-aunt\\
	\glt `my aunt'}
\ex hehjiˀahgę́hę:ˀ\\
\gll he-hji-ˀah-gę́hę:ˀ\\
 \textsc{1s:3ms}-older.sibling-{\diminutive-\past}\\
\glt `my former older brother'
\cfex{hehjíˀah\\
	\gll he-hjí-ˀah\\
	\textsc{1s:3ms}-older.sibling-{\diminutive}\\
	\glt `my older brother'}
\ex haˀnígę̱hę:ˀ\\
\gll ha-ˀní-gę̱hę:ˀ\\ 
 \textsc{3s.m.a}-father-{\past}\\
\glt `my former father'
\cfex{haˀnih\\
	\gll ha-ˀnih\\
	\textsc{3s.m.a}-father\\
	\glt `my father'}
\ex knoháˀgę̱hę:ˀˀˀ\\
\gll k-noháˀ-gę̱hę:ˀˀˀ\\
 \textsc{1s.a}-mother-{\past}\\
\glt `my former mother'
\newpage
\cfex{knó:haˀ\\
	\gll k-nó:haˀ\\
	\textsc{1s.a}-mother\\
	\glt `my mother'}
\ex ǫgya:dáogę̱hę:ˀˀ\\
\gll ǫgy-a:d-áo-gę̱hę:ˀˀ\\
 \textsc{1p.p}-{\semireflexive}-ceremonial.friend-{\past}\\
\glt `my former ceremonial friend'
\cfex{ǫgya:dáoˀ\\
	\gll ǫgy-a:d-áoˀ\\
	\textsc{1p.p}-{\semireflexive}-ceremonial.friend\\
	\glt `my ceremonial friend'}
\ex ǫgyatsihgę́hę:ˀ\\
\gll ǫgy-atsih-gę́hę:ˀ\\
 \textsc{1p.p}-friend-{\past}\\
\glt `my former friend'
\cfex{ǫgyá:tsih\\
	\gll ǫgy-á:tsih\\
	\textsc{1p.p}-friend\\
	\glt `my friend'}
\z
\z

\section{\stem{-neha:ˀ} \textsc{customary} or \textsc{characterizer}} \label{-neha:ˀ}
The \stem{-neha:ˀ} \textsc{\customary} or \textsc{characterizer} ending attaches to atypical words \xref{ex:nehex}. It means a ‘typical way or kind’ (which speakers often translate as ‘Indian’ when the word in question describes traditional ways or objects).

\ea\label{ex:nehex} \textsc{atypical} words functioning as “nouns”
\ea ǫgwehǫwéhne̱ha:ˀ\\
\gll ǫgweh-ǫwéh-ne̱ha:ˀ\\
 people-\exsc{\typicalizer-\customary}\\
\glt `Indian way'
\cfex{ǫgwehǫ́:weh\\
	\gll ǫgweh-ǫ́:weh\\
	people-{\typicalizer}\\
	\glt `Indian'}
\ex hnyǫˀǫhnéha:ˀ\\
\gll hnyǫˀǫh-néha:ˀ\\
 white.{\stative}-\exsc{\customary}\\
\glt `white kind'
\cfex{hahnyǫ́ˀǫh \\
\gll ha-hnyǫ́ˀǫh\\
\textsc{3s.m.a}-white.{\stative}\\
\glt ‘white man’  }
\ex shede̱hjihahnéha:ˀ\\
\gll shede̱h-ji-hah-néha:ˀ\\
 early.morning-\textsc{\intensifier-\diminutive-\customary}\\
\glt `brunch time, mid-morning, early morning'
\cfex{shede̱hjí:hah\\
	\gll shede̱h-jí:-hah\\
	early.morning-{\intensifier}-{\diminutive}\\
	\glt `early morning'}
\z
\z

\section{\stem{-geha:ˀ} \textsc{customary}} \label{-geha:ˀ}
The \stem{-geha:ˀ} \textsc{customary} suffix means a ‘typical way or kind’ (which speakers often translate as ‘Indian’ when the word in question describes traditional ways or objects). It attaches to atypical words functioning as “nouns” \xxref{ex:gehax}{ex:gehax2} and to verbs \xref{ex:gehax3}. The names of many social dances (\sectref{Social Dances and Songs}) end with \stem{-geha:ˀ} \xref{ex:gehax2}.

\ea\label{ex:gehax} atypical words functioning as “nouns”\\
ǫgwehǫwéhge̱ha:ˀ\\
\gll ǫgweh-ǫwéh-ge̱ha:ˀ \\
people-\exsc{\typicalizer-\customary}\\
\glt ‘Indian kind, type’ (e.g. of clothes) 
\cfex{ǫgwehǫ́:weh\\
	\gll ǫgweh-ǫ́:weh\\
	people-{\typicalizer}\\
	\glt `Indian'}
\z


\ea\label{ex:gehax3} verbs\\
otowe̱ˀgéha:ˀ\\
\gll o-t-howe̱ˀ-géha:ˀ\\
 \textsc{3s.p}-{\semireflexive}-cold.{\stative}-\exsc{\customary}\\
\glt `the northern kind'
\cfex{otówe̱ˀgeh\\
	\gll o-t-howe̱-ˀgeh\\
	\textsc{3s.p}-{\semireflexive}-cold.{\stative}-{\on}\\
	\glt `north'}
\z

\newpage
\ea\label{ex:gehax2} social dances
\ea twę:twę́:tge̱ha:ˀ\\
\gll twę:twę́:t-ge̱ha:ˀ\\
 duck-{\customary}\\
\glt `duck dance'
\cfex{twę́:twę:t\\
	\glt `duck'}
\ex hnyagwaiˀgéha:ˀ\\
\gll hnyagwaiˀ-géha:ˀ\\
 bear-{\customary}\\
\glt `Bear Dance'
\cfex{hnyagwái:\\
	\glt `bear'}
\z
\z



\section{\stem{-ka:ˀ} \textsc{customary}} \label{-ka:ˀ}
The \stem{-ka:ˀ} \textsc{\customary} ending also means ‘a typical way or kind’. It attaches to atypical words functioning as “nouns” \xref{ex:kaex} and verbs functioning as “nouns” \xref{ex:kaex2}.

\ea\label{ex:kaex} atypical words functioning as “nouns”
\ea ahsdéhka:ˀ\\
\gll ahsdéh-ka:ˀ\\
 outside-{\customary}\\
\glt `outside type'
\cfex{ahsdeh `outside'}
\ex go̱hsréhka:ˀ\\
\gll go̱hsréh-ka:ˀ\\
 cold.place-{\customary}\\
\glt `winter kind'
\cfex{go̱hsréhneh\\
	\gll go̱hsré-hneh\\
	cold.place-\exsc{at}\\
	\glt `winter'}
\z
\z

\newpage
\ea\label{ex:kaex2} verbs functioning as “nouns” 
\ea ędéhka:ˀ\\
\gll ęd-é-h-ka:ˀ\\
 day-go-{\noaspect-\customary}\\
\glt `day kind'
\cfex{ę́:deh\\
	\gll ę́:d-e-h\\
	day-go-{\noaspect}\\
	\glt `day'}
\ex ahsǫ́hehka:ˀ\\
\gll ahsǫ́h-e-h-ka:ˀ\\
 night-go-\exsc{\noaspect-\customary}\\
\glt `night kind'
\cfex{ahsǫ́heh\\
	\gll ahsǫ́h-e-h\\
	night-go-{\noaspect}\\
	\glt `night'}
\ex ga̱hnyǫ́ˀǫhka:ˀ \\
\gll ga̱-hnyǫ́ˀǫh-ka:ˀ\\
\textsc{3s.a}-white.{\stative}-\exsc{\customary}\\
\glt ‘white man’s kind’ 
\cfex{hahnyǫ́ˀǫh\\
	\gll ha-hnyǫ́ˀǫh\\
	\textsc{3s.m.a}-white.{\stative}\\
	\glt `white man'}
\z
\z

\section{\stem{-ǫ:weh} \textsc{typicalizer} \textup{(}{\typicalizer}\textup{)}} \label{-ǫ:weh}
The \stem{-ǫ:weh} \textsc{\typicalizer} ending attaches to basic nouns \xref{ex:oweh} and atypical words functioning as “nouns” \xref{ex:oweh2}. Speakers often translate this ending as ‘Indian’, ‘traditional’, or ‘ceremonial’.

\ea\label{ex:oweh} basic nouns
\ea ahdahgwa̱ˀǫ́:weh \\
\gll ahdahgw-a̱ˀ-ǫ́:weh\\
 shoe-{\nsf}-{\typicalizer}\\
\glt ‘shoe’ (a special type for a corpse at a funeral), `a moccasin'

\newpage
\cfex{ahdáhgwaˀ\\
	\gll ahdáhgw-aˀ\\
	shoe-{\nsf}\\
	\glt `shoe'}

\ex oyęhsra̱ˀǫ́:weh \\
\gll o-yęhsr-a̱ˀ-ǫ́:weh\\
\exsc{3s.p}-blanket-{\nsf}-{\typicalizer}\\
\glt ‘shawl’ (for dancing, or the type put on a corpse at a funeral)

\cfex{oyę́hsraˀ\\
	\gll o-yę́hsr-aˀ\\
	\textsc{3s.p}-blanket-{\nsf}\\
	\glt `blanket'}
\ex ohnyǫsa̱ˀǫ́:weh\\
\gll o-hnyǫs-a̱ˀ-ǫ́weh\\
\textsc{3s.p}-squash-{\nsf}-{\typicalizer}\\
\glt ‘squash’ (used for soup at the longhouse, usually hubbard squash)

\cfex{ohnyǫ́hsaˀ\\
	\gll o-hnyǫ́h-saˀ\\
	\textsc{3s.p}-squash-{\nsf}\\
	\glt `squash'}

\ex oyęgwa̱ˀǫ́:weh\\
\gll o-yęgw-a̱ˀ-ǫ́:weh\\
\textsc{3s.p}-tobacco-{\nsf}-{\typicalizer}\\
\glt ‘tobacco’ (ceremonial, home-grown and not processed)

\cfex{oyę́ˀgwaˀ\\
	\gll o-yę́ˀgw-aˀ\\
	\textsc{3s.p}-tobacco-{\nsf}\\
	\glt `tobacco'}

\ex onęhęˀǫ́:weh\\
\gll o-nęhę-ˀ-ǫ́:weh\\
\textsc{3s.p}-corn-{\nsf}-{\typicalizer}\\
\glt ‘corn’ (flint corn)

\cfex{onę́hę:ˀ\\
	\gll o-nę́hę:-ˀ\\
	\textsc{3s.p}-corn-{\nsf}\\
	\glt `corn'}

\ex ganǫhsa̱ˀǫ́:weh\\
\gll ga-nǫhs-a̱ˀ-ǫ́:weh\\
\textsc{3s.a}-house-{\nsf}-{\typicalizer}\\
\glt ‘cookhouse’ (at the longhouse)

\cfex{ganǫ́hsaˀ\\
	\gll ga-nǫ́hs-aˀ\\
	\textsc{3s.a}-house-{\nsf}\\
	\glt `house'}

\ex ganaˀja̱ˀǫ́:weh\\
\gll ga-naˀj-a̱ˀ-ǫ́:weh\\
 \textsc{3s.a}-pot-{\nsf}-{\typicalizer}\\
\glt `cooking pots used at the longhouse'

\cfex{ganájaˀ\\
\gll ga-náj-aˀ\\
\textsc{3s.a}-pot-{\nsf}\\
\glt ‘pot’, ‘bottle’, etc.
}
\z
\z

\ea\label{ex:oweh2}atypical words functioning as “nouns”\\
ǫgwehǫ́:weh\\
\gll ǫgwe-h-ǫ́:weh\\
 person-\textsc{euph.h-typicalizer}\\
\glt `Indian'
\cfex{ǫ́:gweh\\
	\gll ǫ́:gwe-h\\
	person-\textsc{euph.h}\\
	\glt `person'}
\z


\textit{Ǫ́:weh} also appears as a stand-alone word in particle groups, where it means ‘really’ or ‘truly’ (see \sectref{p:[ǫ:weh, neˀ ǫ:weh]}).

\ea\label{ex:oweh4} \textsc{particle groups}
\ea 
\gll gwahs \exemph{ǫ́:weh}\\
really truly\\
\glt  ‘truly’\\

\ex 
\gll neˀ \exemph{ǫ́:weh}\\
the truly \\
\glt ‘that, really’
\ex 
\gll shęh \exemph{ǫ́:weh}  \\
that truly\\
\glt ‘it is really’
\z
\z

\newpage
\section{\stem{-ho:nǫˀ} \textsc{populative} \textup{(}{\populative}\textup{)}} \label{-ho:nǫˀ}
The \stem{-ho:nǫˀ} \textsc{\populative} ending means ‘people living at’. It attaches to basic nouns \xref{ex:honex} and to verbs functioning as “nouns” \xref{ex:honex2}. It usually follows the \stem{-ˀgeh} {\on} suffix (but see \ref{ex:honex2}b, c, and \ref{ex:honex4}).

\ea\label{ex:honex} \textsc{basic nouns}
\ea Ga̱hnawa̱ˀgehó:nǫˀ\\
\gll ga̱-hnaw-a̱-ˀgeh-hó:nǫˀ\\
 \textsc{3s.a}-rapids-{\joinerA}-{\on}-{\populative}\\
\glt `Caughnawagans'
\cfex{Ga̱hnáwa̱ˀgeh\\
	\gll ga̱-hnáw-a̱-ˀgeh\\
	\textsc{3s.a}-rapids-{\joinerA}-{\on}\\
	\glt `Caughnawaga'}
\ex ganyada:ˀge̱hó:nǫˀ\\
\gll ga-nyada:-ˀgeh-hó:nǫˀ\\
 \textsc{3s.a}-lake-{\on}-{\populative}\\
\glt `lake people, cottagers'

\cfex{ganyadá:ˀgeh\\
\gll ga-nyadá:-ˀgeh \\
\textsc{3s.a}-lake-{\on}\\
\glt ‘to, at the lake’}

\ex ǫhnyaˀge̱hó:nǫˀ\\
\gll ǫhny-a-ˀgeh-hó:nǫˀ\\
 river.flats-{\joinerA}-{\on}-{\populative}\\
\glt `people of the river flats'
\cfex{ǫhnyáˀgeh\\
	\gll ǫhny-á-ˀgeh\\
	river.flats-{\joinerA}-{\on}\\
	\glt `on the river flats'}
\ex oyajihó:nǫˀ\\
\gll o-ya-ji-hó:nǫˀ\\
 \textsc{3s.p}-other-{\intensifier}-\exsc{\populative}\\
\glt `stranger, alien'
\cfex{oyá:jih\\
	\gll o-yá:-jih\\
	\textsc{3s.p}-other-{\intensifier}\\
	\glt `another type'}
\ex Ganeda̱ˀgehó:nǫˀ\\
\gll ga-ned-a̱-ˀgeh-hó:nǫˀ\\
 \textsc{3s.a}-valley-{\joinerA}-{\on}-{\populative}\\
\glt `Lower End People'
\cfex{Ganéda̱ˀgeh \\
\gll ga-néd-a̱-ˀgeh\\
\textsc{3s.a}-valley-{\joinerA}-{\on}\\
\glt ‘Lower End’ (literally, ‘in the valley’)
}
\ex hadihǫwaˀge̱hó:nǫˀ\\
\gll hadi-hǫw-a-ˀge̱-hó:nǫˀ\\
 \textsc{3ns.m.a}-boat-{\joinerA}-{\on}-{\populative}\\
\glt `sailors, navy men, merchant marines'
\cfex{ga̱hǫ́:waˀ\\
	\gll ga̱-hǫ́:w-aˀ\\
	\textsc{3s.a}-boat-{\nsf}\\
	\glt `boat'}
\z
\z

\ea\label{ex:honex2} \textsc{verbs} functioning as “nouns”
\ea Ga̱hnawiyoˀge̱hó:nǫˀ\\
\gll ga̱-hnaw-iyo-ˀge̱h-hó:nǫˀ\\
 \textsc{3s.a}-rapids-beautiful.{\stative}-{\on}-{\populative}\\
\glt `Oklahomans'
\cfex{Ga̱hnawiyóˀgeh\\
	\gll ga̱-hnaw-iyó-ˀgeh\\
	\textsc{3s.a}-rapids-beautiful.{\stative}-{\on}\\
	\glt `Oklahoma'}
\ex Ganǫhgwa̱ˀtrohó:nǫˀ\\
\gll ga-nǫhgwa̱ˀtr-o-hó:nǫˀ\\
 \textsc{3s.a}-medicine-submerged.{\stative}-{\populative}\\
\glt `Sour Spring People'
\cfex{Ganǫ́hgwa̱ˀtroˀ\\
	\gll ga-nǫ́hgwa̱ˀtr-oˀ\\
	\textsc{3s.a}-medicine-submerged.{\stative}\\
	\glt `Sour Springs'}
\ex Dagęhyathó:nǫˀ\\
\gll da-g-ęhy-a-t-hó:nǫˀ\\
 {\cislocative}-\textsc{3s.a}-mountain.top-stand.{\stative}-{\populative}\\
\glt `Upper End People'
\newpage
\cfex{Dagę́hya:t\\
\gll da-g-ęhy-a:-t\\
{\cislocative}-\textsc{3s.a}-mountain.top-stand.{\stative}\\
\glt ‘Upper End’ (literally, ‘top of the mountain’)
}
\z
\z

\ea\label{ex:honex4}
\ea Gwahsdǫhó:nǫˀ\\
\gll gwahsdǫ-hó:nǫˀ\\
 America-{\populative}\footnotemark{}\\
\glt `Americans'
\footnotetext{Gwahsdǫh is a loanword adaptation of ‘Boston’.}
\ex Gwahsdǫhonǫˀge̱hó:nǫˀ\\
\gll gwahsdǫh-honǫh-ˀge̱-hó:nǫˀ\\
 America-{\populative}-{\on}-{\populative}\\
\glt `United States citizen'
\cfex{Gwahsdǫhonǫˀgeh\\
	\gll gwahsdǫ-honǫ-ˀgeh\\
	America-{\populative}-{\on}\\
	\glt `United States'}
\z
\z

\section{\stem{-jih} \textsc{intensifier} \textup{(}{\intensifier}\textup{)}} \label{-jih}
The \stem{-jih} \textsc{\intensifier} ending means ‘really so’, ‘even more so’. It attaches to verbs \xref{ex:jihex}, atypical words \xref{ex:jihex2}, and nouns (both atypical, \ref{ex:jihex3a}, and basic, \ref{ex:jihex3b}). It also attaches to several particles \xref{ex:particle-jih}.
 
\ea\label{ex:jihex} verbs
\ea ahóhda̱ˀjih\\
\gll a-hó-hda̱-ˀ-jih\\
 {\factual}-\textsc{3s.m.p}-get.full-{\punctual}-{\intensifier}\\
\glt `he really got full'
\cfex{ahóhdaˀ\\
	\gll a-hó-hda-ˀ\\
	{\factual}-\textsc{3s.m.p}-get.full-{\punctual}\\
	\glt `he got full'}
\ex Sgęnǫ́jih gęh?\\
\gll s-g-ęnǫ́-jih gęh\\
 {\repetitive}-\textsc{3s.a}-originate.from.{\stative}-{\intensifier} Q\\
\glt `are you (really) well?'
\cfex{Sgę́:nǫˀ.\\
	\gll s-g-ę́:nǫˀ\\
	{\repetitive}-\textsc{3s.a}-originate.from.{\stative}\\
	\glt `hello'}
\ex ohnáˀgę̱ˀjih\\
\gll o-hná:gę̱ˀ-jih\\
 \textsc{3s.p}-under.{\stative}-{\intensifier}\\
\glt `back then, back there, late, the bottom'
\cfex{ohnáˀgę:ˀ\\
	\gll o-hná:gę:-ˀ\\
	\textsc{3s.p}-under.{\stative}-{\nsf}\\
	\glt `behind, late'}
\z
\z

\ea\label{ex:jihex2} atypical words\\
hé:tgę̱hjih\\
\gll hé:tgę̱h-jih\\
 above-{\intensifier}\\
\glt `the very top'
\cfex{hé:tgęh\\
	\glt `above, up'}
\z


\ea\label{ex:jihex3} nouns
\ea dagú:jih\\\label{ex:jihex3a}
\gll dagú:s-jih\\
cat-\textsc{\intensifier}\\
\glt ‘cat’ (term of endearment)\footnote{The \phonet{s} deletes from \textit{dagu:s} before the \stem{-jih} {\intensifier} suffix.} 
\cfex{dagu:s\\
\glt ‘cat’
}
\ex oyájih\\\label{ex:jihex3b}
\gll o-y-á-jih\\
\textsc{3s.p}-other-{\joinerA}-\textsc{\intensifier}\\
\glt ‘another type’
\cfex{ó:yaˀ\\
\gll ó:-y-aˀ\\
\textsc{3s.p}-other-{\nsf}\\
\glt ‘another’
}
\z
\z

The particles in \xref{ex:particle-jih} are described in the \textit{Particle dictionary} (\sectref{ch:particle dictionary}).

\newpage
\ea\label{ex:particle-jih}
\ea \textit{Toh-jih} ‘just the one’, 'just such a one’
\ex \textit{Waˀ-jih} ‘after a while’, 'eventually’, 'just a while ago’
\ex \textit{Waˀ-jih} ‘wait!’
\ex \textit{Waˀ-jih gyę:ˀ} ‘almost’, 'just about’
\ex \textit{Waˀ-jih hya:ˀ} ‘wait a minute’, 'wait a while’, ‘wait!’
\z 
\z 

\section{\stem{-shę:ˀ, -tsę:ˀ} \textsc{kinship declaration}}
The \stem{-shę:ˀ, -tsę:ˀ} {\declarekinship} ending attaches to kinship terms, and to words functioning as “kinship terms”. Its function is to declare a kinship relationship.

\ea\label{ex:declare.kin}
\ea ǫgyaˀsé:ˀshęˀ\\
\gll ǫgy-aˀsé:ˀ-shęˀ\\
 \textsc{1d.p}-doubled.{\stative}-{\declarekinship}\\
\glt `our cousins'
\cfex{ǫgyáˀse:ˀ\\
	\gll ǫgy-áˀse:ˀ\\
	\textsc{1d.p}-doubled.{\stative}\\
	\glt `my cousin'}
\ex hoˀníshęˀ\\
\gll ho-ˀní-shęˀ\\
 \textsc{3s.m.p}-father-{\declarekinship}\\
\glt `he has a father'
\cfex{haˀnih \\
	\gll ha-ˀnih\\
	\textsc{3s.m.a}-father\\
	\glt ‘my father’, `Dad'}
\ex agaǫdáoˀtsę:ˀ\\
\gll a-gaǫ-d-áoˀ-tshę:ˀ\\
 {\factual}-\textsc{3ns.fi.a}-{\semireflexive}-run.{\stative}-{\declarekinship}\\
\glt `they (fe/males) became ceremonial friends'
\cfex{honadáoˀsǫˀ\\
	\gll hon-ad-áoˀ-shǫˀ\\
	\textsc{3s.m.p}-{\semireflexive}-run.{\stat}-{\pluralizer}\\
	\glt `they are friends'}
\newpage
\ex hohsó:tsęˀ\\
\gll ho-hsó:t-shęˀ\\
 \textsc{3s.m.p}-grandparent-{\declarekinship}\\
\glt `he has a grandparent'
\cfex{hohso:t\\
	\gll ho-hso:t\\
	\textsc{3s.m.p}-grandparent\\
	\glt `his Grandma'}
\cfex{hehso:t\\
	\gll he-hso:t\\
	\textsc{1s:3ms}-grandparent\\
	\glt `my grandfather'}
\ex honó:haˀtsęˀ\\
\gll ho-nó:haˀ-tshęˀ\\
 \textsc{3s.m.p}-mother-{\declarekinship}\\
\glt `he has a mother'
\cfex{honó:haˀ\\
	\gll ho-nó:haˀ\\
	\textsc{3s.m.p}-mother\\
	\glt `his mother'}
\z
\z



\section{Pluralizers ({\pluralizer})} \label{Pluralizers}
The two \textsc{pluralizer} endings, \stem{-shǫ:ˀǫh} {\pluralizer} and \stem{-shǫˀ} {\pluralizer} are described next. \sectref{Pluralizing nouns} also includes a discussion of the inherent plural meaning of basic nouns.


\subsection{\stem{-shǫ:ˀǫh} \textsc{pluralizer} ({\pluralizer})} \label{-sǫ:ˀǫh}
The \stem{-shǫ:ˀǫh} \textsc{\pluralizer} ending attaches to nouns \xxref{ex:sooex}{ex:sooex2}, verbs functioning as “nouns” \xref{ex:sooex3}, kinship terms \xref{ex:sooex20}, and numbers \xref{ex:sooex4}.

With nouns, \stem{-shǫ:ˀǫh} {\pluralizer} possibly means a variety of items, or a dissimilar group of items (\ref{ex:sooex}, see \sectref{Meaning of -sǫˀ versus -sǫ:ˀǫh} for further comments).

\ea\label{ex:sooex} nouns
\ea ga̱ˀdrehda̱ˀsǫ́:ˀǫh\footnote{The community spellings of the {\pluralizer} suffixes,  <-sǫ:ˀǫh, -sǫ:ˀah> and <-sǫˀ>, are used in the unsegmented words, while the linguistic spelling with SH is used in the morpheme breaks. See \sectref{Spelling systems} for more explanation.}\\
\gll ga̱-ˀdrehd-aˀ-shǫ́:ˀǫh\\
 \textsc{3s.a}-car-{\nsf}-{\pluralizer}\\
\glt `many, a variety of cars'

\cfex{ga̱ˀdréhdaˀ\\
	\gll ga̱-ˀdréhd-aˀ\\
	\textsc{3s.a}-car-{\nsf}\\
	\glt `car(s)'}

\ex ohǫda̱ˀsǫ́:ˀǫh\\
\gll o-hǫd-a̱ˀ-shǫ́:ˀǫh\\
 \textsc{3s.p}-sapling-{\nsf}-{\pluralizer}\\
\glt `bushes'

\cfex{ohǫ́:daˀ\\
	\gll o-hǫ́:d-aˀ\\
	\textsc{3s.p}-sapling-{\nsf}\\
	\glt `whip(s)'}

\ex jidęˀsǫ́:ˀǫh\\
\gll jidęˀ-shǫ́:ˀǫh\\
 bird-{\pluralizer}\\
\glt `birds'

\cfex{jidę́:ˀęh\\
	\gll jidę́:-ˀeh\\
	bird-{\diminutive}\\
	\glt `bird(s)'}
\ex agetgwę̱ˀdaˀsǫ́:ˀǫh\\
\gll ag-e-tgwę̱ˀd-aˀ-shǫ́:ˀǫh\\
 \textsc{1s.p}-{\joinerE}-wallet-{\nsf}-{\pluralizer}\\
\glt `my suitcases (ones that are scattered around)'
\cfex{agétgwę̱ˀdaˀ\\
	\gll ag-é-tgwę̱ˀd-aˀ\\
	\textsc{1s.p}-{\joinerE}-wallet-{\nsf}\\
	\glt `my suitcase(s), wallet(s), purse(s)'}
\ex hadiksa̱ˀsǫ́:ˀǫh\\
\gll hadi-ks-a̱ˀ-shǫ́:ˀǫh\\
 \textsc{3ns.m.a}-child-{\nsf}-{\pluralizer}\\
\glt `male children'
\cfex{hadiksá:ˀah\\
	\gll hadi-ks-á:ˀ-ˀah\\
	\textsc{3ns.m.a}-child-{\nsf}-\exsc{\diminutive}\\
	\glt `two boys'}
\ex gaeksa̱ˀsǫ́:ˀǫh\\
\gll gae-ks-a̱ˀ-shǫ́:ˀǫh\\
 \textsc{3ns.fi.a}-child-{\nsf}-{\pluralizer}\\
\glt `female or mixed children'
\cfex{gaeksá:ˀah\\
	\gll gae-ks-á:-ˀah\\
	\textsc{3ns.fi.a}-child-{\nsf}-\exsc{\diminutive}\\
	\glt `two girls'}
\z
\z

\ea\label{ex:sooex3} verbs functioning as “nouns”
\ea ehstaˀsǫ́:ˀǫh\\
\gll e-hsd-haˀ-shǫ́:ˀǫh\\
\textsc{3s.fi.a}-use-{\habitual}-{\pluralizer}\\
\glt ‘tools’ (literally, `what people use’)

\cfex{ehstaˀ ‘she uses’ or\\
	\gll e-hsd-haˀ\\
	\textsc{3s.fi.a}-use-{\habitual}\\
	\glt `people use'}

\ex ga:gǫgweˀdase:sǫ́:ˀǫh\\
\gll ga:g-ǫgweˀd-ase:-shǫ́:ˀǫh\\
\textsc{3ns.fi.a}-people-new.{\stat}-{\pluralizer}\\
\glt ‘young women’ (picked out from a group of women)

\cfex{ga:gǫgwéˀdase:\\
	\gll ga:g-ǫgwéˀd-ase:\\
	\textsc{3ns.fi.a}-people-new.{\stative}\\
	\glt `young women'}
\z
\z

\ea\label{ex:sooex20} kinship terms
\ea gakeyadreˀsǫ́:ˀǫh\\
\gll gakey-adreˀ-shǫ́:ˀǫh\\
 \textsc{1s:3ns}-join.{\stat}-{\pluralizer}\\
\glt `my grandchildren'

\cfex{gakéyadreˀ \\
\gll gakéy-adreˀ\\
\textsc{1s:3ns}-join.{\stative}\\
\glt ‘my grandchildren’ (said when reckoning a bloodline)
}
\ex gwadreˀsǫ́:ˀǫh\\
\gll gw-adreˀ-shǫ́:ˀǫh\\
\textsc{1:2(p)}-join.{\stat}-{\pluralizer}\\
\glt ‘grandchildren’ (term of address said by any old person to young children)

\cfex{gwadre:ˀ\\
\gll gw-adre:ˀ\\
\textsc{1:2(p)}-join.{\stative}\\
\glt ‘grandchild’ (term of address said by any old person to a young child)
}
\z
\z

The \stem{-shǫ:ˀǫh} {\pluralizer} ending also turns \emph{uncountable} nouns (mass nouns) into countable objects \xref{ex:sooex2}.

\ea\label{ex:sooex2} uncountable (mass) nouns and countable objects 
\ea ohnaˀsǫ́:ˀǫh \\
\gll  o-hn-aˀ-shǫ́:ˀǫh \\
\textsc{3s.p}-grease-{\nsf}-{\pluralizer}\\
\glt ‘greasy people’ {\countable}
\cfex{ohnaˀ \\
\gll o-hn-aˀ\\
\textsc{3s.p}-grease-{\nsf}\\
\glt ‘gravy’ \exsc{mass}   }
\ex gakwaˀsǫ́:ˀǫh \\ 
\gll ga-kw-aˀ-shǫ́:ˀǫh \\
\textsc{3s.a}-food-{\nsf}-{\pluralizer}\\
\glt ‘variety of foods’  {\countable}
\cfex{gakwaˀ \\ 
\gll ga-kw-aˀ\\
\textsc{3s.a}-food-{\nsf}\\
\glt ‘food’ \exsc{mass}  }

\z
\z

With numbers, the \stem{-shǫ:ˀǫh} {\pluralizer} ending means ‘a number of each type’ \xref{ex:sooex4}, ‘around a certain time’ \xref{ex:sooex45} or ‘at a time’ \xref{ex:sooex46}.

\ea\label{ex:sooex4} sgatsǫ́:ˀǫh gagéhǫˀ\\
\gll sgat-shǫ́:ˀǫh gagéhǫˀ \\
one-{\pluralizer} they.lie.around\\
\glt ‘one of each thing lying around’
\z

\ea\label{ex:sooex45} O: hwihsǫ:ˀǫ́h hęwáge̱hoˀ\\
\gll O: hwihs-shǫ:ˀǫ́h hęwáge̱hoˀ. \\
oh five-{\pluralizer} I.will.lie.around\\
\glt ‘Oh, I’ll be here around 5 o’clock.’
\z

\ea\label{ex:sooex46} Sgatsǫ:ˀǫ́h shę́h ta:dí:yǫˀ\\
\gll Sgat-shǫ:ˀǫ́h shę́h ta:dí:yǫˀ \\
one-{\pluralizer} that they.males.arrived\\
\glt ‘They came in one by one’ 
\z 


\subsection{\stem{-shǫˀ} \textsc{pluralizer} \textup{(}{\pluralizer}\textup{)}} \label{-sǫˀ}
The \stem{-shǫˀ} \textsc{\pluralizer} ending attaches to nouns \xref{ex:soex}, kinship terms \xref{ex:soex2}, numbers \xref{ex:soex6}, and verbs \xxref{ex:soex3}{ex:soex8}. It possibly means ‘a group of similar items’ (see \sectref{Meaning of -sǫˀ versus -sǫ:ˀǫh} for further comments).
 
\ea\label{ex:soex} nouns\\
agetgwę̱ˀdáˀsǫˀ \\
\gll ag-e-tgwę̱ˀd-áˀ-shǫˀ\\
\textsc{1s.p}-{\joinerE}-wallet-{\nsf}-{\pluralizer}\\
\glt ‘my suitcases’ (i.e. a bunch of suitcases, bundled together or mentally grouped as a unit)\\

\cfex{agétgwę̱ˀdaˀ\\
	\gll ag-é-twę̱ˀd-aˀ\\
	\textsc{1s.p}-{\joinerE}-wallet-{\nsf}\\
	\glt `my suitcase(s), wallet(s), purse(s)'}
\z


\ea\label{ex:soex2} kinship terms
\ea gakeyadréˀsǫˀ\\
\gll gakey-adréˀ-shǫˀ\\
 \textsc{1s:3ns}-join.{\stat}-{\pluralizer}\\
\glt `my grandchildren'

\cfex{gakéyadreˀ\\
\gll gaké-yadreˀ\\
\textsc{1s:3ns}-join.{\stative}\\
\glt ‘my grandchildren’ (said when reckoning a bloodline)
}

\ex ǫdę́nǫ̱hksǫˀ\\
\gll ǫ-dę́-nǫ̱hk-shǫˀ\\
 \textsc{3s.fi.a}-{\semireflexive}-relative-{\pluralizer}\\
\glt `relatives'

\cfex{agya:dę́:nǫhk\\
	\gll agya:-dę́:-nǫhk\\
	\textsc{1d.ex.a}-{\semireflexive}-relative\\
	\glt `my relative'}

\ex gakehji̱ˀáhsǫˀ\\
\gll gake-hji̱-ˀáh-shǫˀ\\
 \textsc{1s:3ns}-older.sibling-{\diminutive}-{\pluralizer}\\
\glt `my older siblings'

\cfex{kehjíˀah\\
	\gll keh-jí-ˀah\\
	\textsc{1s:3fis}-older.sibling-{\diminutive}\\
	\glt `my older sister'}
\z
\z

\largerpage
With numbers, \stem{-shǫˀ} {\pluralizer} denotes an approximate time \xref{ex:soex6a}, as does \stem{-shǫ́:ˀǫh} {\pluralizer} \xref{ex:sooex4}. It can also mean groups of one, two, three, (etc.) at a time (\ref{ex:soex6}b, c).

\ea\label{ex:soex6} numbers
\ea hwihsǫˀ\\\label{ex:soex6a}
\gll hwihs-shǫˀ\\
 five-{\pluralizer}\\
\glt `five-ish, around 5:00'
\cfex{hwihs ‘five’}
\ex sgatsǫˀ\\\label{ex:soex6b}
\gll sgat-shǫˀ\\
 one-{\pluralizer}\\
\glt `one at a time'
\cfex{sga:t \\
\glt ‘one’}
\ex deknísǫˀ \\\label{ex:soex6c}
\gll dekní-shǫˀ \\
two-{\pluralizer}\\
\glt ‘by twos’, ‘two at a time' 
\cfex{dekni:\\
\glt ‘two’ }
\z
\z

With counting expressions (verbs), \stem{-shǫˀ} {\pluralizer} means ‘apart’ \xref{ex:soex3a}, ‘each’ \xref{ex:soex3b}, or ‘at a time’ (\ref{ex:soex3c}, see \sectref{ch:Comparisons, counting, measuring}).

\ea\label{ex:soex3} verbs (counting expression with \stem{pronominal prefix-incorporated noun-verb-sǫˀ})
\ea\label{ex:soex3a} jo̱hsíˀdatsǫˀ\\
\gll j-o-hsíˀd-a-t-shǫˀ\\
 {\repetitive}-\textsc{3s.p}-foot-{\joinerA}-stand.\stative-\pluralizer\\
\glt `one foot apart'
\cfex{jo̱hsíˀda:t\\
	\gll j-o-hsíˀd-a:-t\\
	{\repetitive}-\textsc{3s.p}-foot-{\joinerA}-stand.{\stative}\\
	\glt `one foot'}
\ex sga̱ˀdréhdatsǫˀ \\\label{ex:soex3b}
\gll s-ga̱-ˀdrehd-a-t-shǫˀ\\
 {\repetitive}-\textsc{3s.a}-car-{\joinerA}-stand.{\stat}-{\pluralizer}\\
\glt ‘one car at a time’, ‘each car’, `per car'
\cfex{sga̱ˀdréhda:t\\
	\gll s-ga̱-ˀdrehd-a:-t\\
	{\repetitive}-\textsc{3s.a}-car-{\joinerA}-stand.{\stative}\\
	\glt `one car'}
\ex degadiyáhshesǫˀ\\\label{ex:soex3c}
\gll de-gadi-yáhshe-shǫˀ\\
 {\dualic}-\textsc{3p.a}-two.things.{\stat}-{\pluralizer}\\
\glt `two things at a time'
\cfex{degadiyáhshe: \\
	\gll de-gadi-yáhshe:\\
	{\dualic}-\textsc{3p.a}-two.things.{\stative}\\
	\glt ‘two of them’, `they are two (things)'}
\z
\z

Finally, when attached to verbs, \stem{-shǫˀ} {\pluralizer} means ‘several objects’ \xref{ex:soex8}.

\ea\label{ex:soex8} verbs (\stem{verb-shǫˀ})
\ea hoyę́twę̱hsǫˀ\\
\gll ho-yętw-ę̱h-shǫˀ\\
 \textsc{3s.m.p}-plant-{\stat}-{\pluralizer}\\
\glt `he planted several things'
\cfex{hoyę́:twęh\\
	\gll ho-yę́:tw-ęh\\
	\textsc{3s.m.p}-plant-{\stative}\\
	\glt `he planted'}

\ex sawę́hsǫˀ\\
\gll sa-węh-shǫˀ\\
 \textsc{2s.p}-own.{\stat}-{\pluralizer}\\
\glt `your belongings, property'

\cfex{sá:węh\\
	\gll sa-węh\\
	\textsc{2s.p}-own.{\stative}\\
	\glt `you own something'}

\ex ogáyǫ̱hsǫˀ\\
\gll o-gáy-ǫ̱h-shǫˀ\\
 \textsc{3s.p}-old-{\stat}-{\pluralizer}\\
\glt `old things'

\cfex{ogá:yǫh\\
	\gll o-gá:y-ǫh\\
	\textsc{3s.p}-old-{\stative}\\
	\glt `it is old'}

\ex shǫgwaęnáwisǫˀ\\
\gll shǫgwa-ęn-áwi-shǫˀ\\
 \textsc{3ms:1p}-song-give.{\stat}-{\pluralizer}\\
\glt `he has given us songs'

\cfex{agwe:gǫ́h shǫgwaęná:wih\\
	\gll agwe:gǫ́h shǫgwa-ęn-á:wi-:\\
	all.of.us \textsc{3ms:1p}-song-give-{\stative}\\
	\glt `he has given all of us songs'}

\ex de̱honę̱hsóweksǫˀ\\
\gll de̱-hon-ę̱hs-ówek-shǫˀ\\
 {\dualic}-\textsc{3ns.m.p}-shoe-be.covered.{\stat}-{\pluralizer}\\
\glt `they have their shoes on'

\ex ędwana̱ˀdá:ksǫ:ˀ\\
\gll ę-dwa-na̱ˀdá:-k-shǫ:ˀ\\
 \fut-\exsc{1p.in.a}-bread-bite.{\punctual}.{\pluralizer}\footnotemark{}\\
\glt `we all will have a snack'
\footnotetext{In punctual aspect forms, the {\pluralizer} ending is long \stem{-shǫ:ˀ}.}
\z
\z


\subsection{Meaning of \stem{-shǫˀ} {\pluralizer} versus \stem{-shǫ:ˀǫh} {\pluralizer}} \label{Meaning of -sǫˀ versus -sǫ:ˀǫh}
The \stem{-shǫˀ} {\pluralizer} and \stem{-shǫ́:ˀǫh} {\pluralizer} endings appear to have subtly different meanings \xxref{ex:sosoex}{ex:sosoex3}. Possibly, \stem{-shǫˀ} {\pluralizer} denotes ‘a group of similar items, lumped together’, while \stem{-shǫ́:ˀǫh} {\pluralizer} could mean ‘a group of dissimilar items’, or ‘similar items lying around in different locations’ \xref{ex:sosoex4}. Another difference is that \stem{-shǫ:ˀǫh} {\pluralizer} does not attach to verbs proper (only to verbs functioning as “nouns” and to kinship terms), while \stem{-shǫˀ} {\pluralizer} does attach to true verbs. Meanwhile, both endings attach to basic nouns and numbers.
 
\ea\label{ex:sosoex}
\ea hwihsǫˀ\\
\gll hwihs-shǫˀ\\
five-{\pluralizer}\\
\glt ‘five-ish’ (around 5 o’clock)

\ex hwihsǫ́:ˀǫh\\
\gll hwihs-shǫ́:ˀǫh\\
five-{\pluralizer}\\
\glt ‘five-ish’ (around 5 o’clock)\\
\z
\z

\newpage
\ea\label{ex:sosoex2}
\ea sgatsǫˀ shęh da̱ha:dí:yǫˀ\\
\gll sgat-shǫˀ shęh da̱ha:dí:yǫˀ\\
{one-\pluralizer} that they.came.in\\
\glt ‘they came in one by one’
\ex sgatsǫ́:ˀǫh shęh da̱ha:dí:yǫˀ\\
\gll sgat-shǫ́:ˀǫh shęh da̱ha:dí:yǫˀ \\
{one-\pluralizer} that they.came.in \\
\glt ‘they came in one by one’ 
\z
\z

\ea\label{ex:sosoex3}
\ea nihodi̱hoˀdę́: hadíksa̱ˀsǫˀ\\
\gll nihodi̱hoˀdę́: hadíksa̱ˀ-shǫˀ\\
what.they.are.like {boy-\pluralizer}\\
\glt ‘they are acting like boys’
\ex nihodi̱hoˀdę́: hadiksa̱ˀsǫ́:ˀǫh\\
\gll nihodi̱hoˀdę́: hadiksa̱ˀ-shǫ́:ˀǫh\\
what.they.are.like {boy-pluralizer}\\
\glt ‘they are acting like boys’
\z
\z

\ea\label{ex:sosoex4}
\ea agetgwę̱ˀdáˀsǫˀ\\
\gll age-tgwę̱ˀd-áˀ-shǫˀ \\
\textsc{1s.p}-wallet-{\nsf}-{\pluralizer}\\
\glt ‘my suitcases’ (i.e. a bunch of suitcases, bundled together or mentally grouped as a unit)

\ex agetgwę̱ˀdaˀsǫ́ˀ gaęgwaˀtsǫ́: tiyo:né:noˀ\\
\gll ag-e-tgwę̱ˀd-aˀ-shǫ́ˀ gaęgwaˀtsǫ́: tiyo:né:noˀ\\
\textsc{1s.p-\joinerE}-suitcase-{\nsf-\pluralizer} just.all they.have.disappeared \\
\glt ‘my suitcases have all disappeared’ 
\ex agetgwę̱ˀdaˀsǫ́:ˀǫh\\
\gll age-tgwę̱ˀd-aˀ-shǫ́:ˀǫh\\
 \textsc{1s.p}-suitcase-{\nsf}-{\pluralizer}\\
\glt `my suitcases (ones that are scattered around)'
\cfex{agétgwę̱ˀdaˀ\\
	\gll agé-tgwę̱ˀd-aˀ\\
	\textsc{1s.p}-wallet-{\nsf}\\
	\glt `my suitcase(s), wallet(s), purse(s)'}
\z
\z

\subsection{Pluralizing nouns} \label{Pluralizing nouns}
Even without a \stem{-shǫ:ˀǫh} {\pluralizer} or \stem{-shǫˀ} {\pluralizer} ending, basic nouns can be interpreted as plural in meaning, given the appropriate context. For example in \xref{ex:pluralizingnounex}, both \textit{ohwáhdaˀ} ‘maple tree’ and \textit{ǫ́:gweh} ‘person’ refer to more than one person or thing.

\ea\label{ex:pluralizingnounex}
 \gll Haˀhéˀsǫˀ. Tonhǫ́: o:nę́h ni̱héˀ shę nhǫ́: \exemph{ohwáhdaˀ}. \\
he.walked.on there then he.was that where maple.tree \\
\glt ‘He walked on and found himself under the tall trees.’ (Carrier et al., 2013)
\z

\ea\label{ex:pluralizingnounex2}
 \gll Oˀdǫ́hgwaˀ, ne:ˀ godí:nyohs \exemph{ǫ́:gweh}. \\
fever  it.is they.are.dying person\\
\glt ‘There is a fever that is killing your people.’ (Carrier et al., 2013)
\z

