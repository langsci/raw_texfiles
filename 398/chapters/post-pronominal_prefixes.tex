\chapter{Verb post-pronominal prefixes} \label{Verb post-pronominal prefixes}
The post-pronominal prefixes appear after the pronominal prefixes and before the verb stem, \tabref{figtab:1:postpronorder}.

\begin{table}
\caption{Post-pronominal prefix order}
\label{figtab:1:postpronorder}
\begin{tabular}{l|l|l|l}
\textsc{pre-pron} & \textsc{pron} & \textsc{post-pron} & \textsc{verb stem}\strut\\
\hline
\stem{de-} &  & \stem{adad-} \textsc{\reciprocal} (with \stem{de-}  \textsc{\dualic})\strut  & \\
& & \stem{adad-} \textsc{\reflexive} (without \stem{de-} \textsc{\dualic}) & \\
&  & \stem{ad-} \textsc{\semireflexive} & \\
\end{tabular}
\end{table}


The two post-pronominal prefixes are the \stem{adad-} \textsc{reflexive} or \textsc{reciprocal} \xref{ex:rflsrfex20a} and \stem{ad-} \textsc{semireflexive} \xref{ex:rflsrfex20b}. (The same verb without either prefix is shown in the “cf.” example in \ref{ex:rflsrfex20}.)


\ea\label{ex:rflsrfex20}
\ea dędwadatró:wiˀ\\\label{ex:rflsrfex20a}
\gll d-ę-dw-adat-hró:wi-ˀ\\
 {\dualic}-{\future}-\exsc{1p.in.a}-{\reflexive}-tell-{\punctual}\\
\glt `we all will tell one another'
\ex ędwa:tró:wiˀ\\\label{ex:rflsrfex20b}
\gll ę-dwa:-t-hró:wi-ˀ\\
 \fut-\exsc{1p.in.a-semireflexiv}e-tell-{\punctual}\\
\glt `we all will tell'
\cfex{ękehó:wiˀ\\
	\gll ę-ke-hó:wi-ˀ\\
	\fut-\textsc{1s>3s.fi}-tell-{\punctual}\\
	\glt `I will tell someone'}
\z
\z

For some verbs, the form of the semireflexive is \stem{adę-} or \stem{dę-} \xref{ex:rflsrfex100}.

\ea\label{ex:rflsrfex100}
\ea ęhsadęgahnyeˀ ‘you will comfort, rock a child’,\\
\gll ę-hs-adę-gahnye-ˀ\\
 \fut-\textsc{2s.a}-{\semireflexive}-play-{\stative}\\
\glt `you are babysitting'
\ex sadęˀnigǫhahni:ya:t\\
\gll sa-dę-ˀnigǫha-hní:ya:t\\
 \textsc{2s.p}-{\semireflexive}-mind-harden.{\noaspect}\\
\glt `keep your mind strong'
\cfex{ęhsaˀnigǫhahni:ya:t\\
	\gll ę-hsa-ˀnigǫha-hní:ya:t\\
	\fut-\exsc{2s.p-mind}-harden.{\zeropunctual}\\
	\glt `you will endure'}
\cfex{ęhsheˀnigǫhahni:ya:t\\
	\gll ę-hshe-ˀnigǫha-hní:ya:t\\
	\fut-\exsc{2s:3fis}-mind-harden.{\zeropunctual}\\
	\glt `you strengthen her mind'}
\z
\z

For some verbs, the form of the semireflexive is \stem{ęn-} or \stem{ę-} \xref{ex:rflsrfex101}.

\ea\label{ex:rflsrfex101}
\ea ęsęnidá:gręˀ\\
\gll ę-s-ęn-idá:grę-ˀ\\
 \fut-\textsc{2s.a}-{\semireflexive}-lay.down-{\punctual}\\
\glt `you will lie down'
\ex ęgęnidá:ge:ˀ\\
\gll ę-g-ęn-iˀd-age:-ˀ\\
 \fut-\textsc{1s.a}-{\semireflexive}-feces-wipe-{\punctual} \\
\glt `I will wipe myself clean of fecal matter'
\ex agęnagrá:dǫh\\
\gll ag-ę-nagrá:d-ǫh\\
 \textsc{1s.p}-{\semireflexive}-cause.to.live-{\stative}\\
\glt `I was born'
\z
\z	

\section{\stem{adad-} \textsc{reflexive} (\reflexive) or \textsc{reciprocal} (\reciprocal)} \label{[adad-] (reflexive or reciprocal)}
The \stem{adad-} prefix either has a reflexive or a reciprocal meaning. As a reflexive, the \stem{adad-} prefix means that the action reflects back on the “doer” of the action (for example, ‘seeing \emph{oneself}’, \ref{ex:rflsrfex2}. As a reciprocal, the action equally affects all parties (for ‘example, helping \emph{one another}’ or ‘helping \emph{each other}’, \ref{ex:rflsrfex3}). For the reciprocal sense, the verb also requires the \stem{de-} \textsc{dualic} prefix (\ref{ex:rflsrfex3}, \cite[127]{michelson_ontario_2011}).

\newpage
\ea\label{ex:rflsrfex2} with \stem{adad-} \textsc{\reflexive}
\ea agadadé:gęˀ\\
\gll a-g-adad-é:-gę-ˀ\\
 {\factual}-\exsc{1s.a-{\reflexive}}-{\joinerE}-see-{\punctual}\\
\glt `I saw myself'
\cfex{agé:gęˀ\\
	\gll a-g-é-gę-ˀ\\
	{\factual}-\textsc{1s.a}-{\joinerE}-see-{\punctual}\\
	\glt `I saw (it)'}
\cfex{ahé:gęˀ\\
	\gll a-hé:-gę-ˀ\\
	{\factual}-\exsc{1s>3s.m}-see-{\punctual}\\
	\glt `I saw him'}
\z
\z

\ea\label{ex:rflsrfex3} with \stem{de-} \textsc{\dualic} and \stem{adad-} \textsc{\reciprocal}
\ea dedwadagyénawaˀs\\
\gll de-dw-adag-yénawaˀs\\
 {\dualic}-\exsc{1p.in.a}-{\reflexive}-help.{\noaspect}\\
\glt `let us all help one another'
\cfex{dwayénawaˀs\\
	\gll dwa-yénawaˀs\\
	\exsc{1p.in.a}-help.{\noaspect}\\
	\glt `let all of us help (it)'}
\ex dędwadatró:wiˀ\\
\gll d-ę-dw-adat-hró:wi-ˀ\\
 {\dual}-\exsc{\future-1p.in.a}-{\reflexive}-tell-{\punctual}\\
\glt `we all will tell one another'
\cfex{ędwa:tró:wiˀ\\
	\gll ę-dw-a:t-hró:wi-ˀ\\
	\fut-\exsc{1p.in.a}-{\semireflexive}-tell-{\punctual}\\
	\glt `we all will tell'}
\ex dęgaǫdagyená:waˀs\\
\gll d-ę-gaǫ-dag-yená:waˀs\\
 {\dualic}-{\future}-\textsc{3ns.fi.a}-{\reflexive}-help.{\zeropunctual}\\
\glt `they will help each other'
\cfex{ęgǫyená:waˀs\\
	\gll ę-gǫ-yená:waˀs\\
	\fut-\textsc{1s>2s}-help.{\zeropunctual}\\
	\glt `I will help you'}

\newpage

\ex degaǫdatgǫ̱hsóhae\\
\gll de-gaǫ-dat-gǫ̱hs-óhae\\
{\dual}-\textsc{3ns.fi.a}-{\reflexive}-face-wash.{\stative}\\
\glt ‘they are washing each other’s faces’

\cfex{shagogǫ̱hsóhae\\
	\gll shago-gǫ̱hs-óhae\\
	\exsc{3ms:3fi/3p.p}-face-wash.{\stative}\\
	\glt `he is washing her face'}
\z
\z

The \stem{adad-} prefix also has a special (\emph{non}-reflexive) meaning with the pronominal prefixes meaning ‘her’ or ‘they (females or mixed group)’. In this case, it refers to two separate groups of females. For example, \textit{agaǫda:gyé:na:ˀ} means ‘they caught her’, not ‘they caught \emph{themselves}’ \xref{ex:rflsrfex4}.

\ea\label{ex:rflsrfex4}
\ea agaǫda:gyé:na:ˀ\\
\gll a-gaǫ-da:g-yé:na:-ˀ\\
 {\factual}-\textsc{3ns.fi.a}-{\reflexive}-catch-{\punctual}\\
\glt `they caught her'
\cfex{agaeyé:na:ˀ\\
	\gll a-gae-yé:na:-ˀ\\
	{\factual}-\textsc{3ns.fi.a}-catch-{\punctual}\\
	\glt `they caught, received, accepted it'}
\ex aˀǫdadé:gęˀ\\
\gll aˀ-ǫ-dad-é:-gę-ˀ\\
 {\factual}-\exsc{3s.fi.a-{\reflexive}}-{\joinerE}-see-{\punctual}\\
\glt `she saw her (some other person)'
\cfex{aˀé:gęˀ\\
	\gll aˀ-é:-gę-ˀ\\
	{\factual}-\textsc{3s.fi.a}-see-{\punctual}\\
	\glt `she saw (it)'}
\z
\z

\subsection{Verbs that require  \stem{adad-} \textsc{\reflexive} or \textsc{\reciprocal}} \label{Verbs that require the [adad-] (reflexive or reciprocal) prefix}
Fixed expressions (verbs) requiring the \stem{adad-} prefix are listed in \xref{ex:rflsrfex5}. (Because their meaning is fixed, \sectref{Fixed and transparent expressions}, the reflexive or reciprocal meaning is not immediately obvious in the translations.)

\ea\label{ex:rflsrfex5}
\ea wadatgowá:nęh\\
\gll w-adat-gowá:n-ęh\\
 \textsc{3s.a}-{\reflexive}-big-{\stative}\\
\glt `it is rash, unwise, self-important, egotistical'
\ex hadágya̱ˀda:s\\
\gll ha-dág-ya̱ˀda:-s\\
 \textsc{3s.m.a}-{\reflexive}-rely.on-{\habitual}\\
\glt `he has a high opinion of himself, he is self-centred, he is bold'
\ex wadatnigǫ́ha:ˀ\\
\gll w-adat-ˀnigǫ́ha:ˀ\\
 \textsc{3s.a}-{\reflexive}-watch.{\habitual}\\
\glt `it is wary, cautious'
\ex tęwadadǫ́:niˀ\\
\gll t-h-ę-w-adad-ǫ́:ni-ˀ\\
 {\contrastive}-\exsc{\translocative-\future-3s.a}-{\reflexive}-make-{\punctual}\\
\glt  ‘it will emerge or appear unintentionally’, `it will do it by itself'
\ex dęsádadǫˀ\\
\gll d-ę-s-ádad-ǫ-ˀ\\
 {\dualic}-{\future}-\textsc{2s.a}-{\reflexive}-give-{\punctual}\\
\glt `you will trade, exchange'
\ex desadadéhsnyeh \\
\gll de-s-adad-é-hsnye-h\\
 {\dualic}-\exsc{2s.a-{\reflexive}}-{\joinerE}-care.for-\textsc{euph.h}/{\noaspect}\\
\glt ‘tidy up!’, `groom yourself!'
\ex ęhsadagyé:na:ˀ\\
\gll ę-hs-adag-yé:na:-ˀ\\
 \exsc{ future-2s.a}-{\reflexive}-catch-{\punctual}\\
\glt `you will wrestle'
\ex dęhsádatsˀa:ˀ\\
\gll d-ę-hs-ádat-sˀa:-ˀ\\
 {\dualic}-{\future}-\textsc{2s.a}-{\reflexive}-use.up-{\punctual}\\
\glt `you will quarrel'
\ex dędwadadrihwanǫ́hkwa:k\\
\gll d-ę-dwa-dad-rihw-a-nǫ́hkw-a-:k\\
 {\dualic}-{\future}-\exsc{1p.in.a}-{\reflexive}-matter-{\joinerA}-respect-{\joinerA}-\exsc{\zeropunctual.\modalizer}\\
\glt `we all will show respect for one another'
\ex de̱hęnadátgǫ̱he:s’\\
\gll de̱-hęn-adát-gǫ̱he:g-s\\
 {\dualic}-\textsc{3ns.m.a}-{\reflexive}-punch-{\habitual}\\
\glt ‘boxer, `they are boxing'
\ex sa̱hęnada:thré:waht\\
\gll sa̱-hęn-ada:t-hré:waht\\
 {\repetitive}-\textsc{3ns.m.a}-{\reflexive}-punish.{\zeropunctual}\\
\glt `they repented'
\ex awadadrí:yoˀ \\
\gll a-w-adad-rí:yo-ˀ\\
 {\factual}-\textsc{3s.a}-{\reflexive}-kill-{\punctual}\\
\glt ‘it killed itself’, `suicide'
\ex ęhsadadrohé:gęˀ\\
\gll ę-hs-adad-rohé:g-ę-ˀ\\
 \fut-\textsc{2s.a}-{\reflexive}-gather-{\benefactive}-{\punctual}\\
\glt `you will accumulate (things, ideas, etc.) for yourself'

\ex ǫdadri̱hǫ́nyanih\\
\gll ǫ-dad-ri̱h-ǫ́ny-a-nih\\
 \textsc{3s.fi.a}-{\reflexive}-word-make-\textsc{\joinerA-\benefactive.\habitual}\\
\glt `she is reading'

\ex awadadríhwa̱hdǫˀt\\
\gll a-w-adad-ríhw-a̱hdǫˀt\\
{\factual}-\textsc{3s.a}-{\reflexive}-matter-cause.to.\-disappear.{\zeropunctual}\\
\glt ‘suicide’

\ex agadadwíya̱hdǫˀt\\
\gll a-g-adad-wíy-a̱hdǫˀt\\
 {\factual}-\textsc{1s.a}-{\reflexive}-offspring-cause.to.\-disappear.{\zeropunctual}\\
\glt `I had an abortion'
\ex dęgaǫdagya̱ˀdáhgwaˀt\\
\gll d-ę-gaǫ-dag-ya̱ˀd-áhgwaˀt\\
 {\dualic}-{\future}-\textsc{3ns.fi.a}-{\reflexive}-body-lift.up.{\zeropunctual}\\
\glt `they will do push ups'
\ex wadatgǫnyǫ́hstaˀ\\
\gll w-adat-gǫnyǫ́hst-haˀ\\
 \textsc{3s.a}-{\reflexive}-discriminating-{\habitual}\\
\glt `she is stuck up'
\z
\zlast


\largerpage[1.5]
\section{\stem{ad-} \textsc{semireflexive} (\semireflexive)} \label{[ad-] (semireflexive) prefix}
The \stem{ad-} \textsc{\semireflexive} means that an action affects or benefits the “doer” in some way \xref{ex:rflsrfex6}. Verbs without it describe an action that affects someone other than the “doer” (see the “cf.” examples in \ref{ex:rflsrfex6}).\clearpage

\ea\label{ex:rflsrfex6} 
\ea hatgǫhso̱háe\\
\gll ha-t-gǫhs-o̱háe\\
 \textsc{3s.m.a}-{\semireflexive}-face-wash.{\habitual}\\
\glt `he is washing his face'

\cfex{shagogǫ̱hsóhae\\
	\gll shago-gǫ̱hs-óhae\\
	\exsc{3ms:3fi/3p.p}-face-wash.{\habitual}\\
	\glt `he is washing her face'}

\ex ęgáda̱hseht\\
\gll ę-g-ád-a̱hseht\\
\fut-\textsc{1s.a}-{\semireflexive}-hide.{\zeropunctual}\\
\glt ‘I will hide’ (implied: ‘myself’)\\

\cfex{ęgáhseht\\
	\gll ę-g-áhseht\\
	\fut-\textsc{1s.a}-hide.{\zeropunctual}\\
	\glt `I will hide something'}
\z
\z

The \stem{ad-} \textsc{\semireflexive} can also serve to downplay or de-emphasize the \textsc{recipient} of the action. The verbs with \stem{ad-} in \xref{ex:rflsrfex10} emphasize \emph{giving thanks} or \emph{telling}, but do not focus on who is being thanked, or who is being told. In contrast, the verbs without \stem{-ad} in \xref{ex:rflsrfex11} emphasize what or who is being thanked. More examples are provided in \xref{ex:rflsrfex22}.
	
\ea\label{ex:rflsrfex10} verbs with with \stem{ad-} {\semireflexive}, recipient downplayed
\ea dędwadęnǫ́hǫnyǫ:ˀ\\
\gll d-ę-dw-adę-nǫ́hǫnyǫ-:ˀ\\
 {\dualic}-{\future}-\exsc{1p.in.a}-{\semireflexive}-thank-{\punctual}\\
\glt `we will give thanks'
\ex ędwa:tró:wiˀ\\
\gll ę-dw-a:t-hró:wi-ˀ\\
 \fut-\exsc{1p.in.a}-{\semireflexive}-tell-{\punctual}\\
\glt `we will tell, narrate'
\z
\z

\ea\label{ex:rflsrfex11} verbs without \stem{ad-} {\semireflexive}, more emphasis on recipient
\ea dędwanǫ̱hǫ́:nyǫ:ˀ\\
\gll d-ę-dwa-nǫ̱hǫ́:nyǫ-:ˀ\\
 {\dualic}-{\future}-\exsc{1p.in.a}-thank-{\punctual}\\
\glt `we will thank it'
\ex dęhshenǫ̱hǫ́:nyǫ:ˀ\\
\gll d-ę-hshe-nǫ̱hǫ́:nyǫ-:ˀ\\
 {\dualic}-{\future}-\exsc{2s:3fis}-thank-{\punctual}\\
\glt `you will welcome, greet or thank someone'
\z
\z

\ea\label{ex:rflsrfex22} 
\ea hadęyenáwa̱ˀseh\\
\gll h-adę-yenáwa̱ˀs-eh\\
 \textsc{3s.m.a}-{\semireflexive}-help-{\habitual}\\
\glt `he is a helper, assistant'
\ex gaǫdęyenawáˀseh\\
\gll gaǫ-dę-yenawáˀs-eh\\
 \textsc{3ns.fi.a}-{\semireflexive}-help-{\habitual}\\
\glt `they are helpers, assistants of hers'
\cfex{ęgǫyena:waˀs\\
	\gll ę-gǫ-yena:waˀs\\
	\fut-\textsc{1s>2s}-help.{\zeropunctual}\\
	\glt `I will help you'}
\z
\z

The \stem{ad-} \textsc{\semireflexive} can also serve to emphasize the \emph{result} of an activity or happening, instead of the activity itself. For example, \textit{agaǫtgwé:niˀ} ‘they won’ \xref{ex:rflsrfex12} describes the result of \textit{agaegwé:niˀ} ‘they are able to do it’. (For comparison, the same verb with the \stem{adad-} {\reflexive} prefix is shown in \ref{ex:rflsrfex12b}.)

\ea\label{ex:rflsrfex12} with \stem{ad-}, emphasizing the result of an action
\ea\label{ex:rflsrfex12a} agaǫtgwé:niˀ\\
\gll a-gaǫ-t-gwé:ni-ˀ\\
 {\factual}-\textsc{3ns.fi.a}-{\semireflexive}-succeed-{\punctual}\\
\glt `they won'
\cfex{agaegwé:niˀ\\
	\gll a-gae-gwé:ni-ˀ\\
	{\factual}-\textsc{3ns.fi.a}-succeed-{\punctual}\\
	\glt `they are or were able to do it'}
\ex agaǫdatgwé:niˀ\\\label{ex:rflsrfex12b}
\gll a-gaǫ-dat-gwé:ni-ˀ\\
 {\factual}-\textsc{3ns.fi.a}-{\reflexive}-succeed-{\punctual}\\
\glt `they beat her'
\z
\z

In many fixed expressions (verbs), the meaning of the \stem{ad-} \textsc{\semireflexive} is not immediately obvious \xref{ex:rflsrfex13}. For example, \textit{hadęhní:nǫh} means ‘he is a seller’, not ‘he is buying (for himself)’. That being said, the words with the \stem{ad-} {\semireflexive} still imply a benefit to the ‘doer’ that is absent from the same words without the \stem{ad-} {\semireflexive}.
	
\ea\label{ex:rflsrfex13}
\ea hadęhní:nǫh\\
\gll h-adę-hní:nǫ-h\\
 \textsc{3s.m.a}-{\semireflexive}-buy-{\habitual}\\
\glt `he is a seller or storekeeper'
\cfex{hahní:nǫh\\
	\gll ha-hní:nǫ-h\\
	\textsc{3s.m.a}-buy-{\habitual}\\
	\glt `he is a buyer'}
\ex agadekǫ́:ni:\\
\gll ag-ad-e-k-ǫ́:ni-:\\
 \exsc{1s.p-{\semireflexive}}-{\joinerE}-food-make-{\stative}\\
\glt `I am eating'
\cfex{age:kǫ́:ni:\\
	\gll ag-e:-k-ǫ́:ni-:\\
	\textsc{1s.p}-{\joinerE}-food-make-{\stative}\\
	\glt `I am cooking'}
\z
\z

\subsection{Verbs requiring \stem{ad-} {\semireflexive}} \label{Verbs that require the [ad-] (semireflexive) prefix}
Over 400 verbs require the \stem{ad-} {\semireflexive} prefix to express their fixed meaning. Examples are provided in \sectref{verb dictionary}.


\section{Pronunciation of the post-pronominal prefixes} \label{Pronunciation of the post-pronominal prefixes}
The semireflexive and reflexive prefixes undergo the following sound changes. First, Joiner E is absent when the prefixes are followed by W, Y, or a vowel, as shown in \xref{ex:semireflpronex}, \xref{ex:semireflpronex4}, and \xref{ex:semireflpronex5}. Joiner E is often present when the prefixes are followed by a consonant, as shown in \xref{ex:semireflpronex2}, \xref{ex:semireflpronex3}, \xref{ex:semireflpronex7b}. However, some counterexamples are shown in \xref{ex:semireflpronex6}, \xref{ex:semireflpronex7a}. 

\ea\label{ex:semireflpronex} \stem{-ad} {\semireflexive}, \stem{adad-} {\reflexive}
\ea ęsadwahátaˀt\\\label{ex:semireflpronexa}
\gll ę-s-ad-ˀwah-á-taˀt\\
 \fut-\textsc{2s.a}-{\semireflexive}-meat-dry.out.{\zeropunctual}\\
\glt `you will dry out the meat'
\ex awadadwę́:deht\\\label{ex:semireflpronexb}
\gll a-w-adad-wę́:deht\\
 {\factual}-\textsc{3s.a}-{\reflexive}-forfeit.{\zeropunctual}\\
\glt `it (e.g. a pet) forfeited (its life)'
\z
\z

\ea\label{ex:semireflpronex2} \stem{ad-e-} {\semireflexive}-{\joinerE}, \stem{adad-e-} \textsc{\reflexive-\joinerE}
\ea ęhsade̱ˀsgǫ́:dęˀ\\
\gll ę-hs-ade̱-ˀsg-ǫ́:dę-ˀ\\
 \fut-\exsc{2s.a-{\semireflexive}}-{\joinerE}-roast-put.in-{\punctual}\\
\glt `you will roast something'
\ex ahadadéˀo:k\\
\gll a-h-adad-é-ˀo:k\\
 {\factual}-\exsc{he.a-{\reflexive}}-{\joinerE}-axe.{\zeropunctual}\\
\glt `he axed himself'
\z
\z

The first A of each prefix is deleted when the semireflexive or reflexive appears after a vowel \xxref{ex:semireflpronex3}{ex:semireflpronex4}. (This is also the case for the \stem{(a)dę-} form of the semireflexive, described in \sectref{Verb post-pronominal prefixes}.)

\ea\label{ex:semireflpronex3} \stem{d-e-} {\semireflexive}-{\joinerE}, \stem{dad-e-} \textsc{\reflexive-\joinerE}
\ea ǫdéteˀtra:s\\
\gll ǫ-d-é-teˀt-ra:s\\
\exsc{3s.fi.a-{\semireflexive}}-{\joinerE}-powder-spread.{\zeropunctual}\\
\glt `she is powdering herself'
\ex ǫdade̱ˀgę́:ˀęh\\
\gll ǫ-dad-e̱-ˀgę́:ˀęh\\
\exsc{3s.fi.a-{\reflexive}}-{\joinerE}-younger.sister\\
\glt `her younger sister'
\z
\z

\ea\label{ex:semireflpronex4} \stem{d-} {\semireflexive}, \stem{dad-} {\reflexive}
\ea ǫdǫ́:nyeˀs\\
\gll ǫ-d-ǫ́:nye-ˀs\\
\textsc{3s.fi.a}-{\semireflexive}-breathe-{\habitual}\\
\glt `she is breathing'
\ex aˀǫdadahǫ́:dǫ:ˀ\\
\gll aˀ-ǫ-dad-ahǫ́:dǫ:-ˀ\\
 {\factual}-\textsc{3s.fi.a}-{\reflexive}-ask-{\punctual}\\
\glt `she asked her'
\z
\z

The final D in the \stem{ad-} {\semireflexive} and \stem{adad-} {\reflexive} becomes G before Y in \textit{Dagęhyatgehó:nǫˀ} (Lower Cayuga, \ref{ex:semireflpronex5}) but remains D in \textit{Ganedagehó:nǫˀ} (Upper Cayuga, not shown).

\ea\label{ex:semireflpronex5}  \stem{ag-} {\semireflexive}, \stem{adag-} {\reflexive}
\ea sagyaˀdo̱hái\\
\gll s-ag-yaˀd-o̱hái\\
 \textsc{2s.a}-{\reflexive}-body-wash.{\noaspect} \\
\glt `take a bath'
\ex ęhsadagyé:na:ˀ\\
\gll ę-hs-adag-yé:na:-ˀ\\
 \fut-\textsc{2s.a}-{\reflexive}-catch-{\punctual}\\
\glt `you will wrestle'
\z
\z

The final D also becomes T before certain consonants, including G \xref{ex:semireflpronex6}, \xref{ex:semireflpronex7a}. (However, note the alternative \stem{adad-e-} {\reflexive}-{\joinerE} pronunciation in \ref{ex:semireflpronex7b}.)

\ea\label{ex:semireflpronex6}  \stem{at-} {\semireflexive}\\
\glll satgǫhso̱hái\\
s-at-gǫhs-o̱hái\\ 
\textsc{2s.a}-{\semireflexive}-face-wash.{\noaspect}\\
\glt  ‘wash your face’ 
\z

\ea\label{ex:semireflpronex7} \stem{adat-} {\reflexive} or \stem{adad-e-} {\reflexive}-{\joinerE}
\ea\label{ex:semireflpronex7a} agádatgęˀ\\
\gll a-g-ádat-gę-ˀ\\
  {\factual}-\textsc{1s.a}-{\reflexive}-see-{\punctual}\\
\glt `I saw myself'
\ex agadadé:gęˀ\\\label{ex:semireflpronex7b}
\gll a-g-adad-é:-gę-ˀ\\
 {\factual}-\exsc{1s.a-{\reflexive}}-{\joinerE}-see-{\punctual}\\
\glt `I saw myself'
\z
\z

