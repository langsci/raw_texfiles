\chapter{Verb prepronominal prefixes (and verb affix order)} \label{Verb prepronominal prefixes}

The order of verb prefixes and suffixes is summarized in \tabref{fig:1:vpreford}. All of the affixes except for the pronominal prefixes are described in the following sections. (Pronominal prefixes are described in \sectref{Pronominal prefix meaning}.)


\begin{table}
\caption{Verb affix order}
\label{fig:1:vpreford}
\begin{tabularx}{\textwidth}{C|C|C|C|p{8mm}|C}
\textsc{pre-pron} & \textsc{pron} & \textsc{post-pron} & \textsc{inc noun} & \textsc{verb stem} & \textsc{verb suffixes}\\
\end{tabularx}
\end{table}

The prepronominal prefixes appear before the pronominal ones, and have two main functions, either conveying \textsc{mood}, or “adverb"-like concepts. The \stem{aˀ-} {\factual}, \stem{ę-} {\future}, and \stem{a:-} {\indefinite} prefixes together express \textsc{mood} (the speaker’s assessment of whether an action, etc., is likely, true, or desirable). The remaining prepronominal prefixes express concepts such as location (\stem{haˀ-} \textsc{\translocative}), negation (\stem{deˀ-} {\negative}), and repetition (\stem{s-} {\repetitive}).

When combined, the prepronominal prefixes appear in a set order, shown in \tabref{figtab:1:prepronorder}, page \pageref{figtab:1:prepronorder}. The \stem{ę-} {\future} and \stem{a:-} {\indefinite} occur after the \stem{de-} {dualic} and before the \stem{s-} {\repetitive} and \stem{d-} {cislocative}. In contrast, the \stem{aˀ-} {\factual} appears before the \stem{de-} {\dualic} and after the \stem{s-} {\repetitive} and \stem{d-} {\cislocative}.

\begin{table}
\caption{Prepronominal prefix order}
\label{figtab:1:prepronorder}
{
\begin{tabularx}{\textwidth}{QXXXX}

\lsptoprule
initial & {\translocative} & {\dual}

\exsc{fac.dual} & \exsc{mood} & \exsc{rep}

\exsc{cisl} 

\exsc{rep.fac}

\exsc{cisl.fac}\\
\midrule
\stem{i-} {\prothetic} & \stem{heˀ-} {\translocative} & \stem{de-} {\dualic} & \stem{ę-} {\future} & \stem{s-} {\repetitive}\\
\tablevspace
\stem{tsi-} {\coincident} & & \stem{ad-} {\factual.\dualic} & \stem{a:-} {\indefinite} & \stem{d-} {\cislocative}\\
\tablevspace
\stem{ti-} {\contrastive}& & & \stem{aˀ-} {\factual} & \stem{sa-} {\repetitive.\factual} \\
\tablevspace
\stem{ni-} {\partitive}& & & & \stem{da-} {\cislocative.\factual} \\
\tablevspace
\stem{deˀ-} {\negative} & & & & \\
\lspbottomrule
\end{tabularx}}
\end{table}




Many verbs require specific prepronominal prefixes or combinations to express specific or fixed meanings: for example, the combination of the \stem{ti-} {contrastive} and \stem{t-ˀah} ‘stand-\diminutive’ means ‘odd’ (\ref{ex:queer}, see \sectref{Fixed and transparent expressions}).

\ea\label{ex:queer} \exemph{ti}yó:tˀah \\
\gll ti-yó:-t-ˀah\\
{\contrastive}-\textsc{3s.p}-stand-{\diminutive}\\
\glt ‘it is queer, unusual, odd’ 
\z


\section{Mood and negation prefixes} \label{Mood and negation prefixes}
Unlike other prepronominal prefixes, the mood and negation prefixes all interact with the \stem{-ˀ} {\punctual} suffix (as described in later sections).

The mood prefixes are the \stem{aˀ-} {\factual}, \stem{ę-} {\future}, and \stem{a:-} {\indefinite}. The \stem{aˀ-} {\factual} expresses the idea that an event or happening is a fact (for example, because it has happened in the past, \ref{ex:moodasptens1a}). The \stem{ę-} {\future} conveys that an event or happening is certain to occur \xref{ex:moodasptens1b}. The \stem{a:-} {\indefinite} means that an event or happening is desirable or possible \xref{ex:moodasptens1c}. 

\ea\label{ex:moodasptens1} mood prefixes

\ea akní:nǫˀ\\\label{ex:moodasptens1a}
\gll a-k-hní:nǫ-ˀ\\
{\factual}-\textsc{1s.a}-buy-{\punctual}\\
\glt `I bought it'

\ex ękní:nǫˀ\\\label{ex:moodasptens1b}
\gll ę-k-hní:nǫ-ˀ\\
\fut-\textsc{1s.a}-buy-{\punctual}\\
\glt `I will buy it'

\ex a:kní:nǫˀ\\\label{ex:moodasptens1c}
\gll a:-k-hní:nǫ-ˀ\\
{\indefinite}-\textsc{1s.a}-buy-{\punctual}\\
\glt `I might, should, could, or would buy it'
\z
\z

The prefixes used for negation are the \stem{deˀ-} {\negative} (\ref{ex:moodasptens2}a, b) and \stem{ti-} {\contrastive} (\ref{ex:moodasptens2}c, d). 

\ea\label{ex:moodasptens2} negation prefixes

\ea tęˀ de̱hóyętwęh\\
\gll tęˀ de̱-hó-yętw-ęh\\
not {\negative}-\textsc{3s.m.p}-plant-{\stative}\\
\glt ‘he did not plant’

\ex tęˀ dejo:drę́:no:t’\\
\gll tęˀ de-j-o:-d-rę́:n-o:t\\
not {\negative}-{\repetitive}-\textsc{3s.p}-{\semireflexive}-song-stand.{\stative}\\
\glt  ‘it is not singing again’

\ex tęˀ ta:gaesayę́:de: \\
\gll tęˀ t-a:-gaesa-yę́:de:\\
not {\contrastive}-{\indefinite}-\exsc{3ns>2s}-recognize.{\noaspect}\\
\glt ‘they will not recognize you’

\ex tęˀ ta:gé:gęh \\
\gll tęˀ t-a:-gé:-gę-h\\
not {\contrastive}-{\indefinite}-\textsc{1s.a}-see-\textsc{euph.h}/{\noaspect}\\
\glt ‘I will not or should not see it’
\z
\z

\largerpage[2]
\subsection{\stem{aˀ-} \textsc{factual} ({\factual})} \label{[aˀ-] factual prefix}
The \stem{aˀ-} {\factual} means that the event described by the verb is a fact or certainty (in the speaker’s opinion). Verbs with the \stem{aˀ-} {\factual} grammatically require the \stem{-ˀ} {\punctual} suffix.\footnote{Verbs with the {\factual} are different from verbs with the \stem{a:-} {\indefinite} or \stem{ę-} {\future}. The latter do not always require the \stem{-ˀ} {\punctual} suffix.} However, this grammatical requirement can be obscured, since the \stem{-ˀ} punctual is sometimes deleted for ease of pronunciation. The term \textsc{zero punctual} ({\zeropunctual}) describes such cases (see \sectref{Simplifying two consonants to one}).

When the event described by the verb has already happened or can be observed to be a fact, the \stem{aˀ-} {\factual} can denote \textit{past} activities or events \xref{ex:moodex14}.

\ea\label{ex:moodex14} 

\ea akní:nǫˀ\\
\gll a-k-hní:nǫ-ˀ\\
{\factual}-\textsc{1s.a}-buy-{\punctual}\\
\glt `I bought it'
\ex
\gll ǫge:gáˀ giˀ!\\
I.did.like.it just\\ 
\glt ‘I \emph{did} like it!’ (said when you have just finished tasting something)
\cfex{
  \gll ǫge:-gá-ˀ\\
  \exsc{\factual.1s.p}-like.the.taste.of-{\punctual}\\
}\clearpage

\ex
\gll A:yę́ˀ {ji tręh}s tsishede̱hjíh agáhyagoˀ. \\
it.seems too.much this.morning I.picked.fruit \\
\glt ‘It seems to me I picked too much fruit this morning.’
\cfex{
  \gll a-g-áhya-go-ˀ\\
  {\factual}-\textsc{1s.a}-fruit-pick-{\punctual}\\
}

\ex
\gll Aǫgo̱hdǫ́h ǫgeˀdra̱héhs í:soˀ. \\
too.much 1s.pverdid.it a.lot \\
\glt ‘I ate too much.’ 
\cfex{
  \gll ǫg-e-ˀdra̱héhs\\
  \exsc{\factual.1s.a}-{\joinerE}-exceeded-{\zeropunctual}\\
}


\ex
\gll Dęˀ hne:ˀ ni:yóht shęh ahádo̱ˀne:k? \\
what in.fact reason that he.withdrew  \\
\glt ‘Why did he leave?’
\cfex{
  \gll a-há-d-o̱ˀne:k\\
  {\factual}-\textsc{3s.m.a}-{\semireflexive}-withdraw.{\zeropunctual} \\
}


\ex
\gll Né:ˀ he:gę́: agyǫ́hsętoˀ. \\
it.is it.is.seen she.cried \\
\glt ‘All she did was cry.’
\cfex{
  \gll a-g-yǫ́-hsęto-ˀ\\
  {\factual}-{\dualic}-\textsc{3s.fi.a}-cry-{\punctual}\\
}


\ex
\gll Neˀ giˀ gyę́:ˀ asiˀ.\\
the just that.one you.said.it \\
\glt ‘That’s just what you said.’ 
\cfex{
  \gll a-s-i-ˀ\\
  {\factual}-\textsc{2s.a}-say-{\punctual}\\
}


\ex
\gll Hotgǫˀǫ́h asyǫˀ! \\
what.the you.arrived \\
\glt ‘What the... you made it!’ ‘For heaven’s sake, you got here!’
\cfex{
  \gll a-s-yǫ-ˀ\\
  {\factual}-\textsc{2s.a}-arrive-{\punctual}\\
}
\z
\z

Verbs with the \stem{aˀ-} {\factual} can also mean (perhaps humorously) that a near-future event is all but certain \xref{ex:moodex16}.

\ea\label{ex:moodex16} 
\ea
\gll Ogwęhę:gyéˀ gę́:s asyǫˀ. \\
now.and.again usually you.arrived \\
\glt ‘Now and again (in general) you’ll be here.’ 
\cfex{
  \gll a-s-yǫ-ˀ\\
  {\factual}-\textsc{2s.a}-arrive-{\punctual}\\
}

\ex
\gll Ne:ˀ ǫh neˀ o:nę́h haˀgahé:ˀ edwadri̱hóˀda:t. \\
it.is I.guess the now the.time.has.arrived for.us.to.work \\
\glt ‘And it is now time for us to work.’
\cfex{
  \gll e-dwa-d-ri̱hóˀda:t\\
  {\factual}-\exsc{1p.in.a}-work.{\zeropunctual}\\
}
\z
\z


\subsubsection*{The \stem{aˀ-} {\factual} with present events and performatives} \label{The [aˀ-] factual with present events and performatives}
Verbs with the \stem{aˀ-} {\factual} can denote events happening at the moment of speech. They are also used while performing the action in question \xref{ex:moodex15}.

\ea\label{ex:moodex15}

\ea ǫgí:daˀ \\
\gll ǫg-í:d-aˀ\\
\exsc{\factual.1s.p}-sleep-{\punctual}\\
\glt ‘I slept’, `I am sleeping' (said when you want to be left alone)

\ex ǫknéhagoˀ \\
\gll ǫk-néhago-ˀ\\
\exsc{\factual.1s.p}-amazed-{\punctual}\\
\glt ‘I am amazed’, `I was amazed'

\ex sagádǫnhe:t\\
\gll s-a-g-ád-ǫnhe:t\\
{\repetitive}-{\factual}-\textsc{1s.a}-{\semireflexive}-live.{\zeropunctual}\\
\glt `I am alive again'

\ex
\gll O:nę́h gwaˀ edwawayę:nę́:daˀ. \\
now emphasis we.finished\\
\glt ‘We’re finished already.’	
\cfex{
  \gll e-dwa-wayę:nę́:d-aˀ\\
  {\factual}-\exsc{1p.in.a}-finish-{\punctual}\\
}

\ex
\gll Waˀhéˀ tsǫ́: ahá:tgęh.\\
just.now just he.got.up \\
\glt ‘He just now got up.’ 
\cfex{
  \gll a-h-á:t-gęh\\
  {\factual}-\textsc{3s.m.a}-{\semireflexive}-awaken.{\zeropunctual}\\
}
\z
\z


\subsubsection*{The \stem{aˀ-} {\factual} with hypotheticals} \label{The [aˀ-] factual with hypotheticals}
Verbs with the \stem{aˀ-} {\factual} also appear in a type of hypothetical clause (sentence) beginning with \textit{gyę́:gwaˀ} ‘if’. (For clause types, see Part~\ref{Sentences}.) Unlike other factuals, they function as suggestions and describe an imaginary event instead of a real one \xref{ex:factualpref1}.

\ea\label{ex:factualpref1} 
\ea
\gll Gyę:gwáˀ aha:gwé:niˀ?  \\
if he.was.able \\
\glt ‘Maybe if he could do it, if he is able?’
\cfex{
  \gll a-ha:-gwé:ni-ˀ\\
  {\factual}-\textsc{3s.m.a}-be.able-{\punctual}\\
}

\ex
\gll Gyę:gwáˀ ahógaęˀ/ahágaęˀ? \\
if he.was.willing \\
\glt ‘Maybe if he would agree, if he is willing?’
\cfex{
  \gll a-ha/ho-gaę-ˀ\\
  {\factual}-\exsc{3s.m.a/3s.m.p}-be.willing-{\punctual}\\
}

\ex
\gll Gyę:gwáˀ o:nę́h ahahdę́:diˀ?\\
if now he.left \\
\glt ‘What if he should leave now?’ 
\cfex{
  \gll a-ha-hdę́:di-ˀ\\
  \exsc{\factual-3s.m.a}-leave-{\punctual}\\
}
\z
\z

\subsection{\stem{ę-} \textsc{future} ({\future})} \label{[ę-] future mood prefix}
Verbs with the \stem{ę-} {\future} describe events that are likely, probable, or certain to happen \xref{ex:moodex100}, and events that are just about to happen (especially in sentences that question someone’s intentions, \ref{ex:moodex100d}). 

\ea\label{ex:moodex100} 
\ea\label{ex:moodex100a}
\gll Waˀjih, ętsyáˀdęˀ.\\
wait you.will.fall.into.it \\
\glt ‘Wait, you might fall’ 
\cfex{
  \gll ę-t-s-yáˀd-ę-ˀ\\
  \fut-{\cislocative}-\textsc{2s.a}-body-fall.into-{\punctual}\\
}

\ex\label{ex:moodex100b}
\gll Ahí:ˀ giˀ to nęyá:węh. \\
I.thought just that it.will.happen\\
\glt ‘I just thought that’s what would happen.’ 
\cfex{
  \gll n-ę-yá:w-ę-h\\
  {\partitive}-{\future}-\textsc{3s.p}-happen-\textsc{euph.h}\\
}

\ex\label{ex:moodex100c}
\gll O:nę́h to:háh ęshá:yǫˀ. \\
now almost he.will.go.back \\
\glt ‘He is almost ready to go back’. 
\cfex{
  \gll ę-s-há:-yǫ-ˀ\\
  \fut-{\repetitive}-\textsc{3s.m.a}-arrive-{\punctual}\\
}

\ex\label{ex:moodex100d}
\gll Ętné:ˀ gęh? \\
we.two.will.come Q \\
\glt ‘Do you want to come along with me?’ 
\cfex{
  \gll ę-tn-é:-ˀ\\
  \fut-\exsc{1d.in.a}-go.{\purposive}-{\punctual}\\
}

\ex\label{ex:moodex100e}
\gll Ęhsné:ˀ gęh diˀ? \\
we.two.will.come Q then \\
\glt ‘Are you coming along then?’ 
\cfex{
  \gll ę-hsn-é:-ˀ\\
  \fut-\exsc{2d.a}-go.{\purposive}-{\punctual}\\
}

\ex\label{ex:moodex100f}
\gll Hwę:dǫ́h ęyǫhdę́:diˀ? \\
when she.will.go.away\\
\glt ‘When is she going away?’
\cfex{
  \gll ę-yǫ-hdę́:di-ˀ\\
  \fut-\textsc{3s.fi.a}-leave-{\punctual}\\
}
\z
\z


Verbs with the \stem{ę-} {\future} grammatically require the \stem{-ˀ} {\punctual} suffix (except for verbs expressing a type of negative command, see \sectref{Negative commands with [ahgwih] ‘don’t’}). The grammatical requirement can be obscured because the \stem{-ˀ} {\punctual} is sometimes deleted for ease of pronunciation. The term ‘zero punctual' (\textsc{\zeropunctual}) describes such cases (see \sectref{Simplifying two consonants to one}.)


\subsubsection*{\stem{ę-} {\future} in dependent clauses} \label{[ę-] future in dependent clauses}
Verbs with the \stem{ę-} {\future} appear in dependent clauses (shown in square brackets) where they can denote future events. (For clause types, see Part~\ref{Sentences}.) The examples in \xref{ex:futureex20} show that the \stem{ę-} {\future} corresponds to a range of English translations, including future tense (‘you will leave’), infinitival (‘what to do’ ‘for us to quit’), and present tense (‘when I grow up’).

\ea\label{ex:futureex20} 
\ea
\gll Daskro:wíˀ [hwę:dǫ́h ęsahdę́:diˀ]. \\
tell.me when you.will.leave \\
\glt ‘Tell me [when you will leave].’ 
\cfex{
  \gll ę-s-ahdę́:di-ˀ\\
  \fut-\textsc{2s.a}-leave-{\punctual}\\
}

\ex
\gll [Nęh toh hędwa:yǫ́ˀ] nę:dáh nędwá:ye:ˀ. \\
when there we.will.arrive this.way we.will.do.it \\
\glt ‘[When we arrive there], we will do it this way.’
\cfex{
  \gll h-ę-dwa:-yǫ́-ˀ\\
  {\translocative}-{\future}-\exsc{1p.in.a}-arrive-{\punctual}\\
}

\ex
\gll Haoˀ dahskro:wíh [dęˀ ni:ˀ hoˀdę́ˀ nęga:gye:ˀ] \\
o.k. tell.me what I kind what.I.will.do\\
\glt ‘O.k., you tell me [what to do]!’ 
\cfex{
  \gll n-ę-g-a:g-ye:-ˀ\\
  {\partitive}-{\future}-\textsc{1s.a}-{\semireflexive}-do-{\punctual} \\
}

\ex
\gll Onęh diˀ to:háh hęgahé:ˀ [ędwę́ni̱hę:ˀ]. \\
now so almost the.time.will.arrive we.will.quit\\
\glt ‘Now it is almost time [for us to quit].’
\cfex{
  \gll ę-dw-ę́-ni̱hę:-ˀ\\
  \fut-\exsc{1p.in.a}-{\semireflexive}-quit-{\punctual}\\
}

\ex
\gll Agwaˀnigǫháˀ [shęh niyo:wéˀ hędwawayę:nę́:daˀ]. \\
we.are.waiting that a.certain.distance we.will.finish\\
\glt ‘We’re waiting [until we are finished].’
\cfex{
  \gll h-ę-dwa-wayę:nę́:d-aˀ\\
  {\translocative}-{\future}-\exsc{1p.in.a}-finish-{\punctual}\\
}

\ex
\gll Ęgahyagwahsé:k hęˀ ni:ˀ [nęh ęgáhdo:k] \\
I’ll.be.a.fruit.picker even me when I.will.grow.up\\
\glt ‘I will be a fruit-picker [when I grow up].’ 
\cfex{
  \gll ę-g-áhdo:k\\
  \fut-\textsc{1s.a}-mature.{\zeropunctual}\\
}
\z
\z


Verbs with the \stem{ę-} {\future} also appear in conditional dependent clauses \xref{ex:futureex24a}, including indirect questions (\ref{ex:futureex24b}, \sectref{ch:Conditional clauses with [gyę:gwaˀ], [gęh], [hę:gyeh] ‘if, whether’}). In both cases, the English translation corresponds to a present tense verb, not a future tense one. (For clause types, see Part~\ref{Sentences}.)


\ea\label{ex:futureex24} 
\ea\label{ex:futureex24a}
\gll Ahgwíh dęhsyéhs ohne:gáˀ [gyę:gwáˀ ihsé: ęsa:dó:wiˀ]. \\
don’t you.will.mix alcohol if you.want you.will.drive \\
\glt ‘Do not mix booze [if you want to drive].’ (‘Do not drink and drive.’)
\cfex{
  \gll ę-s-a:d-ó:wi-ˀ\\
  \fut-\textsc{2s.a}-{\semireflexive}-drive-{\punctual}\\
}

\ex\label{ex:futureex24b}
\gll Daskro:wíh [ędwé: gęh].\\
you.tell.me we.will.go Q\\
\glt ‘Tell me [if you are coming along].’
\cfex{
  \gll ę-dw-é-:\\
  \fut-\exsc{1p.in.a}-go-{\purposive}\\
}
\z
\z



\subsubsection*{\stem{ę-} {\future} and positive commands} \label{[ę-] future and positive commands}
Although verbs with the \stem{ę-} {\future} are generally statements of fact, they can also function as a type of command \xref{ex:futureex5}, (see \sectref{Commands (imperatives) and suggestions}). 

\ea\label{ex:futureex5}

\ea ęhsyeh\\
\gll ę-hs-yeh\\
\fut-\textsc{2s.a}-awaken.{\zeropunctual}\\
\glt `wake up!'

\ex ęhsyǫ́:gyaˀt\\
\gll ę-hs-yǫ́:gyaˀt\\
\fut-\textsc{2s.a}-smile.{\zeropunctual}\\
\glt `smile!'

\ex
\gll Gǫdagyéˀ ętsahdę́:diˀ. \\
right.away you.will.leave\\
\glt ‘You will leave right away.’ 
\cfex{
  \gll ę-t-s-ahdę́:di-ˀ\\
  \fut-{\cislocative}-\textsc{2s.a}-leave-{\punctual}\\
}

\ex
\gll Da:-nę:dáh haˀgahéˀ gaóˀ ędihswatríhs ęhswada̱hǫhsí:yohs.  \\
and-this it.is.time closer you.all.will.bump.up.against you.all.will.listen \\
\glt ‘Now is the time to come closer and listen.’
\cfex{
  \gll ę-di-hswa-t-ríhs\\
  \fut-{\cislocative}-\textsc{2p.a}-{\semireflexive}-bump.{\zeropunctual} \\}
   \cfex{ę-hswa-d-a̱hǫhs-í:yohs\\
   \textsc{\future-2p.a}-{\semireflexive}-ear-make.good.\textsc{\zeropunctual}\\}

\z
\z


\subsubsection*{Ahgwih \stem{ę-} {\future} and negative commands} \label{[ahgwih ę-] future and negative commands}
Together with \textit{ahgwih} ‘don’t’, verbs with the  \stem{ę-} {\future} can form a type of negative command (\ref{ex:futureex6}, see \sectref{Negative commands with [ahgwih] ‘don’t’}). \textit{Ahgwih} is optional or left out when the context makes the meaning clear. With these negative commands, the \stem{-ˀ} {\punctual} suffix is often missing for grammatical reasons \xref{ex:futureex6c}. Such verbs are termed \textsc{no-aspect} forms, as described in \sectref{Meaning of no-aspect verbs}.\footnote{Verbs glossed as {\noaspect} lack a \stem{-ˀ} {\punctual} suffix for grammatical reasons. Verbs glossed as {\zeropunctual} have no overt \stem{-ˀ} {\punctual} suffix for reasons of pronunciation.}

\ea\label{ex:futureex6} 
\ea\label{ex:futureex6a} 
\gll Ahgwíh dęhsnigǫ̱háęˀ. \\
don’t you.won’t.bother\\
\glt ‘Don’t bother it’
\cfex{
  \gll d-ę-hs-ˀnigǫháę-ˀ\\
  {\dualic}-{\future}-\textsc{2s.a}-annoy-{\punctual}\\
}

\ex\label{ex:futureex6b} 
\gll Ahgwíh ęhsáhsdihs. \\
don’t you.won’t.care.for.it \\
\glt ‘Don’t disturb, handle it.’
\cfex{
  \gll ę-hsá-hsdihs\\
  \fut-\textsc{2s.p}-care.for.it.\-{\zeropunctual}\\
}

\ex\label{ex:futureex6c} 
\gll Ahgwíh ędíhsa̱ˀdre:. \\
don’t you.won’t.drive \\
\glt ‘Don’t drive over here.’
\cfex{
  \gll ę-dí-hsa̱-ˀdre:\\
  \fut-\exsc{\cislocative-2s.p}-drive.{\noaspect}\\
}
\z
\z


\largerpage
\subsection{\stem{a:-} \textsc{indefinite} ({\indefinite})} \label{[a:-] indefinite prefix}
The \stem{a:-} {\indefinite} (formerly called the \textsc{optative}) conveys the idea that an event is desirable, wishful, or possible. Compared to verbs with the \stem{ę-} {\future} \xref{ex:moodex100}, verbs with \stem{a:-} {\indefinite} convey the idea that events that are desirable, wishful, or less certain to take place \xref{ex:moodex7}.

\ea\label{ex:moodex7} 

\ea a:kní:nǫˀ\\
\gll a:-k-hní:nǫ-ˀ\\
{\indefinite}-\textsc{1s.a}-buy-{\punctual}\\
\glt `I might, could, should, would buy it'
\ex
\gll Dewagegaę́hs to há:ge:ˀ.\\
I’m.unwilling there I.would.go.there\\
\glt ‘I am unwilling to go there.’ 
\cfex{
  \gll h-á:-g-e:-ˀ\\
  {\translocative}-{\indefinite}-\textsc{1s.a}-go-{\punctual}\\
}
\z
\z

Verbs with the \stem{a:-} {\indefinite} can also combine with \textit{tęˀ dáǫ} ‘definitely not’, to denote an undesirable future event \xref{ex:indefex2}.

\ea\label{ex:indefex2} 
\ea
\gll {Tęˀ daǫ́} to na:yá:węh.\\
definitely.not that it.would.happen\\
\glt ‘That will \emph{never} happen.’ 
\cfex{
  \gll n-a:-yá:w-ę-h\\
  {\partitive}-{\indefinite}-\textsc{3s.p}-happen-\textsc{euph.h}/{\zeropunctual}\\
}\clearpage

\ex
\gll {Tęˀ daǫ́} a:hǫwayená:waˀs. \\
definitely.not  they.would.help.him\\
\glt ‘Definitely not. No one will help him.’ 
\cfex{
  \gll a:-hǫwa-yená:waˀs\\
  {\indefinite}-\exsc{3s.m/3s.fi>3s.m}-help.{\zeropunctual}\\
}
\z
\z


Verbs with the \stem{a:-} {\indefinite} grammatically require the \stem{-ˀ} {\punctual} suffix \xref{ex:moodex7}. However, this grammatical requirement is obscured when the \stem{-ˀ} {\punctual} is deleted for ease of pronunciation. (The term \textsc{\zeropunctual} describes such cases, as in example \ref{ex:indefex2}, see \sectref{Simplifying two consonants to one}.) In addition, \stem{a:-} {\indefinite} verbs appear without the \stem{-ˀ} {\punctual} when they denote actions that are tentative or unlikely to materialize. In this case, they are classified as \textsc{no-aspect} verbs (\ref{ex:moodex11}, see \sectref{Meaning of no-aspect verbs}).

\ea\label{ex:moodex11}
\gll To giˀ shę́h há:ge:.\\
there just that I.should.go.there\\
\glt ‘Just maybe I should go there.’ 
\cfex{
  \gll h-á:-g-e-:\\
  {\translocative}-{\indefinite}-\textsc{1s.a}-go-{\purposive}.{\noaspect}\\
}
\z



\subsubsection*{\stem{a:-} {\indefinite} in dependent clauses} \label{[a:-] indefinite in dependent clauses}
Verbs with the \stem{a:-} {\indefinite} often appear in dependent clauses (shown in [] square brackets) -- for example, after verbs meaning ‘want’ or ‘think’ \xref{ex:moodex8}. (For clause types, see Part~\ref{Sentences}.)

\ea\label{ex:moodex8} 
\ea
\gll Ahí:ˀ giˀ [to na:yá:węh]. \\
I.thought just that what.should.happen\\
\glt ‘I just thought [that’s what should happen].’ 
\cfex{
  \gll n-a:-yá:w-ę-h\\
  {\partitive}-{\indefinite}-\textsc{3s.p}-happen-\textsc{euph.h}/{\zeropunctual}\\
}

\ex
\gll I:wí: [to na:yá:węh].\\
I.want that what.should.happen\\
\glt ‘I want or intend [for that to happen].’ 
\cfex{
  \gll n-a:-yá:w-ę-h\\
  {\partitive}-{\indefinite}-\textsc{3s.p}-happen-\textsc{euph.h}/{\zeropunctual}\\
}

\ex
\gll Gaę ni:gá: i:séˀ [á:se:k]? \\
Which a.certain.one you.want you.should.eat \\
\glt ‘Which one do you want [to eat]?’ 
\cfex{
  \gll á:-s-e:-k\\
  {\indefinite}-\textsc{2s.a}-{\joinerE}-eat.{\zeropunctual}\\
}
\z
\z


Verbs with the \stem{a:-} {\indefinite} also appear in conditional clauses with \textit{gyę́:gwaˀ} ‘if’ (\sectref{ch:Conditional clauses with [gyę:gwaˀ], [gęh], [hę:gyeh] ‘if, whether’}). The main clause (outside of the [] square brackets ) is also an \textsc{indefinite-punctual} verb \xref{ex:moodex10}. (For clause types, see Part~\ref{Sentences}.)

\ea\label{ex:moodex10}
\gll A:ga̱hya:góˀ [gyę:gwáˀ a:sgyená:wahs].\\
I.would.pick.fruit if you.would.help.me\\
\glt ‘I would pick fruit [if you would help me].’ 
\cfex{
  \gll a:-sg-yená:wahs\\
  {\indefinite}-\exsc{2s>1s}-help.{\zeropunctual} \\
}
\z


\largerpage
\subsubsection*{\stem{ta:} \textsc{contrastive-indefinite} and negative future events} \label{[ta:] contrastive-indefinite and negative future events}
Verbs with the \stem{ta:-} \textsc{\contrastive-\indefinite} combination denote negative future events \xref{ex:moodex12}. Such verbs grammatically lack a \stem{-ˀ} {\punctual} suffix -- they are \textsc{no-aspect} forms (\ref{ex:moodex12}b, e). This is because they describe non-events, or activities that by definition will not materialize. Meanwhile, the \stem{-ˀ} {\punctual} is reserved for actual events (see the ‘cf.’ example in \ref{ex:moodex12a}. Also see \sectref{Meaning of no-aspect verbs}.) 

\ea\label{ex:moodex12}

\ea tęˀ ta:hayę́:toh \\ \label{ex:moodex12a}
\gll tęˀ t-a:-ha-yę́:to-h\\
not {\contrastive}-{\indefinite}-\textsc{3s.m.a}-plant-\textsc{euph.h}/{\noaspect}\\
\glt ‘he won’t plant it’
\cfex{ęháyętoˀ \\
  \gll ę-há-yęto-ˀ\\
  \fut-\textsc{3s.m.a}-plant-{\punctual}\\
  \glt `he will plant it'
}

\ex
\gll {Tęˀ daǫˀ} toh tá:ge:.\\
definitely.not there I.wouldn’t.go.there\\
\glt ‘No, I definitely \emph{won’t} go there.’
\cfex{
  \gll t-á:-g-e-:\\
  {\contrastive}-{\indefinite}-\textsc{1s.a}-go-{\purposive}.{\noaspect}\\
}

\ex
\gll Ga̱ˀtoh tsǫ: ta:se:tsę́i. \\
nowhere just you.wouldn’t.find.it\\
\glt ‘You \emph{won’t} find it anywhere.’ 
\cfex{
  \gll t-a:-s-e:-tsęi\\
  {\contrastive}-{\indefinite}-\textsc{2s.a}-find.{\noaspect}\\
}

\ex
\gll Tęˀ niˀ ta:gye:na:. \\
not I I.wouldn’t.accept.it\\
\glt ‘No, I \emph{won’t} accept it.’ 
\cfex{
  \gll t-a:-g-ye:na:\\
  {\contrastive}-{\indefinite}-\textsc{1s.a}-accept.{\noaspect}\\
}

\newpage
\ex
\gll Tęˀ hwę:dǫh to ta:gye:. \\
not ever that I.wouldn’t.do.it\\
\glt ‘No, I \emph{won’t} ever do that.’ 
\cfex{
  \gll t-a:-g-ye:\\
  {\contrastive}-{\indefinite}-\textsc{1s.a}-do.{\noaspect}\\
}
\z
\z




\subsection{Negation prefixes} \label{Negation prefixes}
The \stem{deˀ-} {\negative} and \stem{ti-} {\contrastive} prefixes, described next, are both used in negation.


\subsubsection*{\stem{deˀ-} \textsc{negative} ({\negative})} \label{[deˀ-] (negative) prefix}
The \stem{deˀ} {\negative} conveys the absence or opposite of a quality, characteristic, or activity. It negates habitual verbs \xref{ex:negexample1}, stative-only verbs \xref{ex:negexample2}, and the stative forms of three-aspect verbs (\ref{ex:negexample3}, see \sectref{ch:Negation of verb forms based on the three major aspects}).

\ea\label{ex:negexample1} habitual\\
tęˀ deháyętwahs  \\
\gll tęˀ de-há-yętw-a-hs\\
not {\negative}-\textsc{3s.m.a}-plant-{\joinerA}-{\habitual}\\
\glt ‘he is not a planter’
\cfex{hayę́:twahs \\
  \gll ha-yę́:tw-a-hs\\
  \textsc{3s.m.a}-plant-{\joinerA}-{\habitual}\\
  \glt `he is a planter'
}
\z



\ea\label{ex:negexample2} stative-only

\ea detgáhe:ˀ\\
\gll de-t-gá-he:ˀ\\
{\negative}-{\cislocative}-\textsc{3s.a}-setting.on.top.of.{\stative}\\
\glt `it is not setting over there'
\cfex{tga̱he:ˀ\\
\gll t-ga̱-he:ˀ\\
\textsc{\cislocative-3s.a}-setting.on.top.of.{\stative}\\
\glt ‘it is setting there’ 
}


\ex tęˀ dejo:drę́:no:t\\
\gll tęˀ de-j-o:-d-rę́:n-o:t \\
not {\negative}-{\repetitive}-\textsc{3s.p}-{\semireflexive}-song-stand.{\stative}\\
\glt `it is not singing again'
\cfex{jodrę́:no:t \\
  \gll j-o-d-rę́:n-o:t\\
  {\repetitive}-\textsc{3s.p}-{\semireflexive}-song-stand.{\stative}\\
  \glt `it is singing again'
}


\ex tęˀ deyo:tó:weˀ  \\
\gll tęˀ de-yo:-t-hó:weˀ\\
not {\negative}-\textsc{3s.p}-{\semireflexive}-cold.{\stative}\\
\glt ‘it is not cold’
\cfex{otó:weˀ \\
  \gll o-t-hó:weˀ\\
  \textsc{3s.p}-{\semireflexive}-cold.{\stative}\\
  \glt `it is cold'
}
\z
\z

\ea\label{ex:negexample3} stative, three-aspect verb\\
tęˀ dehóyętwęh \\
\gll tęˀ de-hó-yętw-ęh\\
not {\negative}-\textsc{3s.m.p}-plant-{\stative}\\
\glt ‘he is not planting’, ‘he hasn’t planted’
\cfex{hoyę́:twęh \\
  \gll ho-yę́:tw-ęh\\
  \textsc{3s.m.p}-plant-{\stative}\\
  \glt `he is planting'
}
\z 



On a related theme, the words in \xref{ex:negex2} illustrate that concepts requiring the negative in Gayogo̱hó:nǫˀnéha:ˀ sometimes have no corresponding negative in the English translation. 

\ea\label{ex:negex2}

\ea tęˀ desadagáideˀ  \\
\gll tęˀ de-s-ad-agáideˀ\\
not {\negative}-\textsc{2s.p}-{\semireflexive}-well.{\stative}\\
\glt ‘you feel sick’, ‘you are not well’

\ex tęˀ degá:dę:s  \\
\gll tęˀ de-gá:-dę:s\\
not {\negative}-\textsc{3s.a}-thick.{\stative}\\
\glt ‘it is thin’, ‘it is not thick’

\ex deˀagadǫ́tgadeˀ, dewagadǫtgá:dǫˀ \\
\glll deˀ-ag-ad-ǫ́tgadeˀ\\
de-wag-ad-ǫtgá:dǫˀ\\
{\negative}-\textsc{1s.p}-{\semireflexive}-happy.{\stative}\\
\glt ‘I am unhappy’, `I am not happy'

\ex desatnigǫhahdó:gęh \\
\gll de-sa-t-ˀnigǫh-ahdó:g-ęh\\
{\negative}-\textsc{2s.p}-{\semireflexive}-mind-mature-{\stative}\\
\glt ‘you are immature in mind’, `your mind is not mature'

\ex tęˀ dehóge̱ˀo:t \\
\gll tęˀ de-hó-ge̱ˀ-o:t\\
not {\negative}-\textsc{3s.m.p}-hair-stand.{\stative}\\
\glt ‘he is bald’, ‘he has no hair’

\ex dehonǫdánhęh \\
\gll de-ho-nǫdánh-ęh\\
{\negative}-\textsc{3s.m.p}-guilty-{\stative}\\
\glt ‘he is innocent’, `he is not guilty'

\ex deˀaǫwé:saht \\
\gll deˀ-a-ǫwé:saht\\
{\negative}-\textsc{3s.p}-pleasant.{\stative}\\
\glt ‘it is unpleasant’, `it is not pleasant'

\ex dehoihwa:dó:gę: \\
\gll de-ho-ihwa:-dó:gę:\\
{\negative}-\exsc{3s.m.p-}matter-true.{\stative}\\
\glt ‘he is a traitor’, `he has not done right'

\ex sgahoˀdę́ˀ dehoihwáęˀ \\
\gll sgahoˀdę́ˀ de-ho-ihw-á-ęˀ\\
nothing {\negative}-\textsc{3s.m.p}-matter-{\joinerA}-lie.{\stative}\\
\glt ‘he is innocent’, `nothing lies on him'

\ex tęˀ detga:yéiˀ, tęˀ detga:yí:ˀ \\
\gll tęˀ de-t-ga:-yéiˀ/yí:ˀ\\
not {\negative}-{\cislocative}-\textsc{3s.a}-right.{\stative}\\
\glt ‘it is bad, false, wrong’, `it is not right'

\ex dehoˀnigǫ́ha:t \\
\gll de-ho-ˀnigǫ́h-a:-t\\
{\negative}-\textsc{3s.m.p}-mind-{\joinerA}-stand.{\stative}\\
\glt ‘he is ignorant, unthinkingly foolish’, `he is not smart'

\ex deyago̱ˀnigǫ̱há:gǫ:t \\
\gll de-yago̱-ˀnigǫ̱h-á:-gǫ:t\\
{\negative}-\textsc{3s.fi.p}-mind-{\joinerA}-persevere.{\stative}\\
\glt ‘she is uncompromising, unbending’, `her mind does not dither'

\ex dedisa̱ˀnigǫ̱hí:yo: \\
\gll de-di-sa̱-ˀnigǫ̱h-í:yo:\\
{\negative}-{\cislocative}-\textsc{2s.p}-mind-good.{\stative}\\
\glt ‘you are grumpy, grouchy’, `you are not happy'

\ex desa:tǫ́:da:s \\
\gll de-s-a:t-hǫ́:da:-s\\
{\negative}-\textsc{2s.a}-{\semireflexive}-listen-{\habitual}\\
\glt ‘you are disobedient’, `you do not listen'

\ex tęˀ deyǫtwaji:yǫ́:nih  \\
\gll tęˀ de-yǫ-t-hwaji:y-ǫ́:ni-h\\
not {\negative}-\textsc{3s.p}-{\semireflexive}-family-make-{\habitual}\\
\glt ‘she is barren’, ‘she cannot make a family’

\ex desęˀnigǫhgáę \\
\gll de-s-ę-ˀnigǫh-gáę\\
{\negative}-\textsc{2s.p}-{\semireflexive}-mind-willing.{\stative}\\
\glt ‘you suffer’, `your mind is not willing'

\ex dewagegáę \\
\gll de-wag-e-gáę\\
{\negative}-\textsc{1s.p}-{\joinerE}-willing.{\stative}\\
\glt ‘I am unwilling’, `I do not want to do it'

\ex dehsnǫ́hweˀs\\
\gll de-hs-nǫ́hwe-ˀs\\
\textsc{negative-2s.a}-like-{\habitual}\\
\glt  ‘you dislike’, you do not like'

\ex desrihwanǫ́hweˀs\\
\gll de-s-rihwa-nǫ́hwe-ˀs\\
{\negative}-\textsc{2s.a}-matter-like-{\habitual}\\
\glt  ‘you disapprove’, `you do not approve'
\z
\z

\largerpage
Finally, several particles require the following verb to begin with \stem{deˀ-} {\negative}. They are described in the \textit{Particle dictionary} \sectref{ch:particle dictionary}.

\begin{CayugaRelated}
\item{}\textit{Tęˀ gęh deˀ-} ‘didn't?’\\
\item{}\textit{Tęˀ de̱ˀgę:} ‘it isn't’\\
\item{}\textit{Tęˀ ne:ˀ de̱ˀgę:} ‘without’, ‘lacking’\\
\item{}\textit{Tęˀ dedo:gęhs} ‘it isn't true’, ‘not really’
\end{CayugaRelated}

\subsubsection*{\stem{ti-} \textsc{contrastive} ({\contrastive}) and negation} \label{[ti-] contrastive and negation}
The \stem{ti-} {\contrastive} is also used in negation \xref{ex:tireq3}. Affirmative verbs beginning with the \stem{a:-} {\indefinite} or \stem{ę-} {\future} prefixes are negated with the \stem{(tęˀ) ta:-} ‘not \textsc{\contrastive.\indefinite}’ combination. (\textit{Tęˀ} ‘not’ is optional, see \sectref{ch:Negation of verb forms based on the three major aspects}.) Such verbs grammatically lack a \stem{-ˀ} {\punctual} suffix, or are \textsc{no-aspect} forms, see \sectref{Meaning of no-aspect verbs}).

\ea\label{ex:tireq3}

(tęˀ) ta:hayę́:toh\\
\gll tęˀ t-a:-ha-yę́:to-h\\
not {\contrastive}-{\indefinite}-\textsc{3s.m.a}-plant-\textsc{euph.h}/{\noaspect}\\
\glt `he won’t plant'
\cfex{ęháyętoˀ \\
  \gll ę-há-yęto-ˀ\\
  \fut-\textsc{3s.m.a}-plant-{\punctual}\\
  \glt `he will plant'
}
\z



\largerpage
\section{Adverb-like prepronominal prefixes} \label{Adverb-like prepronominal prefixes}
Adverb-like prepronominals modify the meaning of an event or happening -- for example, describing the location of an activity with \stem{haˀ-} {\cislocative} ‘there’, or repetition with \stem{s-} {\repetitive} ‘again’. (Also see \sectref{ch:’Adverbs’ (words and affixes that function as adverbs)}.) Unlike the mood and negation prefixes described earlier, the adverb-like prefixes do not interact with the \stem{-ˀ} {\punctual} aspect (see \sectref{Mood and negation prefixes}).  


The adverb-like prepronominals figure into two types of expressions -- transparent ones (with verbs that optionally take the prepronominal) and fixed expressions (with verbs that require specific prepronominals to complete their meaning, see \sectref{Fixed and transparent expressions}). In the first case, the prefix adds a straightforward or transparent meaning. For example the \stem{he-} \textsc{\translocative} in \xref{ex:advlike1a} contributes the meaning ‘there’, which is missing from the example that does not have this prefix \xref{ex:advlike1b}.
 
\ea\label{ex:advlike1} verb with optional \stem{he-} \textsc{\translocative}\\
\ea hé:ye:t\\\label{ex:advlike1a}
\gll hé:-ye:-t\\
\exsc{\translocative-3s.fi.a}-stand.{\stative}\\
\glt `she is standing over there'
\ex\label{ex:advlike1b}
í:ye:t \\
  \gll í:-ye:-t\\
  {\prothetic}-\textsc{3s.fi.a}-stand.{\stative}\\
  \glt `she is standing (here)'
\z
\z 


The second type of verb requires a specific prepronominal (and possibly other elements) to express particular, fixed or non-transparent meanings. For example, while the verb \stem{t} without a fixed prepronominal means ‘stand’ \xref{ex:advlike1b}, the same verb with a fixed \stem{tsaˀ-…\textsc{incorporated noun}-t} combination means ‘the same kind of \textsc{incorporated noun}’ \xref{ex:advlike100a}, and with a fixed \stem{s-…incorporated noun-t} combination, means ‘one \textsc{incorporated noun}’ \xref{ex:advlike100b}. 

\ea\label{ex:advlike100} fixed expressions (verbs) requiring specific elements

\ea tsaˀgaˀdre̱hdá:t\\\label{ex:advlike100a}
\gll tsaˀ-gá-ˀdre̱hd-a:-t\\
{\coincidentfactual}\exsc{-3s.a}-car-{\joinerA}-stand.{\stative}\\
\glt `the same kind of car'

\ex sganǫ́hsa:t\\\label{ex:advlike100b}
\gll s-ga-nǫ́hs-a:-t\\
{\repetitive}-\textsc{3s.a}-house-{\joinerA}-stand.{\stative}\\
\glt `one house'
\z
\z  

Each of the following sections begins with a description of the transparent meaning of the relevant prepronominal -- the meaning expressed when the prefix is not part of a fixed expression. Then follow descriptions of fixed expressions that require the prepronominal prefix in question. Finally, several sections end with a description of particles that require a following verb to begin with the prepronominal prefix in question. 


\subsection{\stem{tsi-} \textsc{coincident} ({\coincident})} \label{[tsi-] (coincident) prefix}
The \stem{tsi-} \textsc{\coincident} means ‘while’, or ‘when’ (\ref{ex:tsiex2}a--d). Verbs with \stem{tsi-} can form a type of dependent clause  within a complex sentence (shown in [] square brackets in (\ref{ex:tsiex2}e, f), see \sectref{ch:Clauses with [shęh naˀonisheˀ], [tsaˀonisheˀ], [tsi-] ‘while, when’}). (For clause types, see Part~\ref{Sentences}.)

\newpage
\ea\label{ex:tsiex2} \stem{tsi-\textsc{verb}} ‘while, when’

\ea tsaˀónihseˀ \\
\gll ts-aˀ-ó-ni̱hse-ˀ\\
{\coincident}-{\factual}-\textsc{3s.p}-take.time-{\punctual}\\
\glt ‘it took a certain amount of time’, `while a certain amount of time passed'

\ex tsaˀta:dí:yǫˀ\\
\gll ts-aˀ-t-ha:dí:-yǫ-ˀ\\
{\coincident}-{\factual}-{\cislocative}-\textsc{3ns.m.a}-arrive-{\punctual}\\
\glt ‘while they were arriving…’

\ex tsihágę̱hjih\\
\gll tsi-há-gę̱hjih\\
{\coincident}-\textsc{3s.m.a}-old.person.{\stative}\\
\glt `when he was an old man…'

\ex tsiyagwaksaˀsǫ́:ˀǫh\\
\gll tsi-yagwa-ksaˀ-shǫ́:ˀǫh\\
{\coincident}-\exsc{1p.ex.a}-child-{\pluralizer}\\
\glt `when we were children…'

\ex \glll [tsihǫgwe̱ˀda:sé:] hohsę:\\
tsi-h-ǫgwe̱ˀd-a:sé: hohsę:\\
{\coincident}-\textsc{3s.m.a}-person-new.{\stative} he.is.fat\\
\glt `When he was young, he was fat.'
\cfex{hǫgwéˀdase: \\
  \gll h-ǫgwéˀd-ase:\\
  \textsc{3s.m.a}-person-new.{\stative}\\
  \glt `young man'
}

\ex \glll [hohsę́: tsaˀoni̱hséˀ] hǫgwéˀdase:.  \\
hohsę: ts-aˀ-o-ni̱hsé-ˀ hǫgwéˀdase:\\
he.is.fat {\coincident}-{\factual}-\textsc{3s.p}-take.time-{\punctual} he.was.a.young.man\\
\glt ‘When he was young, he was fat.’
\z
\z


\subsubsection*{\stem{tsi-\textsc{verb}} (fixed expressions)} \label{[tsi-verb]}
The fixed expression (verb) in \xref{ex:tsireq} requires \stem{tsi-}. The parts that can change are the mood prefix ({\factual}, {\indefinite}, or {\future}) and the pronominal prefix. 

\newpage
\ea\label{ex:tsireq} tsaˀgeht\\
\gll ts-aˀ-g-eht\\
{\coincident}-{\factual}-\textsc{1s.a}-do.on.\-purpose.{\zeropunctual}\\
\glt ‘I did it on purpose’
\z


\subsubsection*{\stem{tsi-\textsc{incorporated noun-verb}} ‘be the same kind of \textsc{incorporated noun}’}  \label{[tsi-incorporated noun-verb]}
The expression in \xref{ex:pppexpressionex32} is fixed, except that the incorporated noun can be changed. (Also see \textit{Comparisons}, \sectref{ch:Comparisons (more, the same, or less)}.)

\ea\label{ex:pppexpressionex32} \stem{tsi-…\textsc{incorporated noun}-t} ‘be the same kind of \textsc{incorporated noun}’\\
tsaˀgaˀdre̱hdá:t ahahní:nǫˀ\\
\gll tsaˀ-gá-ˀdre̱hd-a:-t ahahní:nǫˀ\\
{\coincident}-{\factual}-\textsc{3s.a}-car-{\joinerA}-stand.{\stative} he.bought.it\\
\glt `he bought the same kind of car (as me)’
\z


For the words in \xref{ex:tsiex3}, all of the parts are fixed. (These verbs function as adverbs of time, \sectref{ch:’Adverbs’ of time}.)

\ea\label{ex:tsiex3} \stem{tsi-…\textsc{period of time}-\textsc{verb}} ‘past \textsc{period of time}’

\ea tsiyaǫdadǫgę́hdǫh\\
\gll tsi-ya-ǫda-dǫgę́hd-ǫh\\
{\coincident}-\textsc{3s.p}-day-make.right-{\stative}\\
\glt `Sunday (past)'

\ex tsiyonakdóhaes\\
\gll tsi-yo-nakd-óhae-s\\
{\coincident}-\textsc{3s.p}-bed-clean-{\habitual}\\
\glt `Saturday (past)'

\ex tsigaǫ̱hyáhęh\\
\gll tsi-ga-ǫ̱hy-á-hęh\\
{\coincident}-\textsc{3s.a}-sky-{\joinerA}-mid.{\stative}\\
\glt `at noon (past)'

\ex tsishéhde̱hjih\\
\gll tsi-shéhde̱hjih\\
{\coincident}-\exsc{ø.prefix}-morning\\
\glt `this morning (earlier on in the day)'

\newpage
\ex tsigóhsreh\\
\gll tsi-g-óhsreh\\
{\coincident}-\textsc{3s.a}-winter/year\\
\glt `last winter'
\z
\z


\subsubsection*{\stem{tsaˀde-\textsc{verb}}} \label{[tsaˀde-verb]}
The following verbs (fixed expressions) require the \stem{tsaˀde-} \textsc{coincident.dualic} combination to express concepts like \textit{similarity} or \textit{equality}. 


\ea\label{ex:pppexpressionex20} \stem{tsaˀde-…\textsc{verb}} ‘side by side’

\ea tsaˀdę́hsyę:ˀ\\
\gll tsaˀd-ę́-hs-yę-:ˀ\\
{\coincident.\dualic}\exsc{-\future-2s.a}-lay-{\punctual}\\
\glt `you will put, lay them side by side'

\ex tsaˀdégayęˀ\\
\gll tsaˀdé-ga-yę-ˀ\\
{\coincidentdualic}\exsc{-3s.a}-lie-{\stative}\\
\glt `they are lying, setting side by side'
\ex
\gll tsaˀdeganǫhsa̱hęh tsaˀdegayę́ˀ sa̱hdáhgwaˀ\\
in.the.middle.of.the.house lying.side.by.side your.shoes\\
\glt  ‘your shoes lie side by side in the middle of the house’ 

\ex tsaˀdęjatáhahk\\
\gll tsaˀd-ę-j-at-háh-a-hk\\
{\coincident}-{\dualic}.{\future}-\exsc{2d.a}-{\semireflexive}-road-{\joinerA}-pick.up.{\zeropunctual}\\
\glt `you two will walk side by side'
\z
\z

\ea\label{ex:pppexpressionex25} \stem{tsaˀde-…ahd} ‘be the same, similar’

\ea tsaˀdé:yoht\\
\gll tsaˀdé:-yo-ht\\
{\coincidentdualic}\exsc{-3s.p}-resemble.{\stative}\\
\glt `they are the same, similar'
\ex
\gll A:yę́:ˀ neˀ to:há tsaˀde:yóht nę́:gyęh. Né:ˀ sadeˀnyę́:dęh.\\
it.seems the almost it.is.the.same this.one It.is you.try.3s.pn \\
\glt ‘It seems to be the same style as the one you were trying on.’  (\cite[225]{mithun_watewayestanih_1984}, Agyaˀdawítraˀ dialogue)
\z
\z

\newpage
\ea\label{ex:pppexpressionex26} \stem{tsaˀde-…ahshe:dęh} ‘be the same number of’\\
tsaˀdeyo̱hshé:dęh\\
\gll tsaˀde-yo̱-hshé:d-ęh\\
{\coincidentdualic}\exsc{-3s.p}-count-{\stative}\\
\glt `it is the same number of'
\z


\ea\label{ex:pppexpressionex27} \stem{tsaˀde-…aˀs} ‘be the same size’\\
tsaˀdé:waˀs \\
\gll tsaˀdé:-w-aˀs\\
{\coincidentdualic}\exsc{-3s.a}-be.a.certain.size.{\stative}\\
\glt ‘they are (literally, `it is’) the same size'
\z


\ea\label{ex:pppexpressionex31} \stem{tsaˀde-…ǫ:} ‘be of equal amount, number of items’\\
tsaˀdé:yǫ:\\
\gll tsaˀdé:-y-ǫ:\\
{\coincidentdualic}\exsc{-3s.p}-be.a.certain.number.of.{\stative}\\
\glt `of equal number, amount'
\z


\ea\label{ex:pppexpressionex21} \stem{tsaˀde-yo/wa-ahsęnǫh} ‘be half, be in the middle’\\
tsaˀdeyo̱hsę́:nǫh, tsaˀdewa̱hsę́:nǫh \\
\gll tsaˀde-yo̱/wa̱-hsę́:n-ǫh\\
{\coincidentdualic}\exsc{-3s.p/3s.a}-half-{\stative}\\
\glt ‘half’, `middle'
\z


\subsubsection*{\stem{tsaˀde-\textsc{incorporated noun}-verb} (fixed expressions)} \label{[tsaˀde-incorporated noun-verb]}
The fixed combination of \stem{tsaˀde-\textsc{incorporated noun}-hęh} means ‘half, in the middle of’. It modifies the meaning of whatever noun is incorporated \xref{ex:pppexpressionex22}.

\ea\label{ex:pppexpressionex22} \stem{tsaˀde-\textsc{incorporated noun}-hęh} ‘half \textsc{incorporated noun}, in the middle of \textsc{incorporated noun}’

\ea tsaˀdewę̱hníhsra̱hęh\\
\gll tsaˀde-w-ę̱hníhsr-a̱-hęh\\
{\coincidentdualic}\exsc{-3s.a}-day-{\joinerA}-sitting.on.top.of.{\stative}\\
\glt `half a day'

\ex tsaˀdegáǫhya̱hęh\\
\gll tsaˀde-gá-ǫhy-a̱-hęh\\
{\coincidentdualic}\exsc{-3s.a}-sky-{\joinerA}-sitting.on.top.of.{\stative}\\
\glt `high noon'

\ex tsaˀdeganǫ́hsa̱hęh\\
\gll tsaˀde-ga-nǫ́hs-a̱-hęh\\
{\coincidentdualic}\exsc{-3s.a}-house-{\joinerA}- sitting.on.top.of.{\stative}\\
\glt `in the middle of the house'

\ex tsaˀdewa̱hsǫ́:tęh\\
\gll tsaˀde-w-a̱hsǫ́:t-hęh\\
{\coincidentdualic}\exsc{-3s.a}-night- sitting.on.top.of.{\stative}\\
\glt `at midnight'
\z
\z

The fixed combination of \stem{tsaˀde-…\textsc{incorporated noun}-age:} means ‘two of the same kind of…’. It specifies the number of whatever noun is incorporated \xref{ex:pppexpressionex29}.

\ea\label{ex:pppexpressionex29} \stem{tsaˀde-…\textsc{incorporated noun}-age:} ‘be two of the same kind of \textsc{incorporated noun}’\\
tsaˀdeyoyę́hsrage:\\
\gll tsaˀde-yo-yę́hsr-age:\\
{\coincidentdualic}\exsc{-3s.p}-blanket-be.two.or.more.{\stative}\\
\glt `they are two of the same kind of blanket'
\z


The fixed combination of \stem{tsaˀde-…incorporated noun-oˀdę:} means ‘be similar’, ‘look the same’. It modifies the meaning of whatever noun is incorporated \xref{ex:pppexpressionex28a}. In addition, when describing people or animals, the incorporated noun is always \stem{yaˀd} ‘body’ \xxref{ex:pppexpressionex28b}{ex:pppexpressionex28c}.

\ea\label{ex:pppexpressionex28} \stem{tsaˀde-……yaˀd-oˀdę:} ‘look similar, be similar’

\ea tsaˀdega̱ˀdréhdo̱ˀdę:\\ \label{ex:pppexpressionex28a}
\gll tsaˀde-ga̱-ˀdréhd-o̱ˀdę:\\
{\coincidentdualic}-\exsc{3s.a}-car-be.a.kind.{\stative}\\
\glt `the cars look the same'

\ex tsaˀdegaeya̱ˀdóˀdę:\\ \label{ex:pppexpressionex28b}
\gll tsaˀde-gae-ya̱ˀd-óˀdę:\\
{\coincidentdualic}\exsc{-3ns.fi.a-}body-be.a.kind.{\stative}\\
\glt `they are a group of people that are similar, look the same' (i.e. nationalities or identical twins)

\ex tsaˀdegadiya̱ˀdóˀdę:\\ \label{ex:pppexpressionex28c}
\gll tsaˀde-gadi-ya̱ˀd-óˀdę:\\
{\coincidentdualic}\exsc{-3p.a}-be.a.kind.{\stative}\\
\glt `they look the same' (animals)
\ex
\gll Sowasǫ:ˀǫ́h tsaˀdegadiya̱ˀdóˀdę:\\
dogs they.look.the.same\\
\glt  ‘the dogs all look the same’


\z
\z

\subsubsection*{\stem{tsaˀdet/tsaˀdeg-\textsc{verb}}} \label{[tsaˀdet/tsaˀdeg-verb]}
The following fixed expressions (verbs) begin with \stem{tsaˀdet-} or \stem{tsaˀdeg-} \textsc{\coincident.\dualic.\cislocative}.

\ea\label{ex:pppexpressionex30} \stem{tsaˀdet-…deˀ} ‘same height’\\
tsaˀdétgadeˀ\\
\gll tsaˀdét-ga-deˀ\\
{\coincident.\dualic.\cislocative}\exsc{-3s.a}-exist.{\stative}\\
\glt `the same height'
\z


\ea\label{ex:pppexpressionex23} \stem{tsaˀdeg-…ad-ǫda-\textsc{verb}} ‘during the week (this week or last)’
\ea tsaˀdegyadǫ:dáik(hne:ˀ)
\gll tsaˀdeg-y-ad-ǫ:d-á-i-k-hne:ˀ\\
{\coincident.\dualic.\cislocative}\exsc{-3s.p}-{\semireflexive}-period.of.time-{\joinerA}-stuck.on.{\stative}-{\modalizer}-\exsc{\remote}\\
\glt `last week'

\ex tsaˀdegyadǫ́dai\\
\gll tsaˀdeg-y-ad-ǫ́d-a-i\\
{\coincident.\dualic.\cislocative}\exsc{-3s.p}-{\semireflexive}- period.of.time -{\joinerA}-stuck.on.{\stative}\\
\glt `during the week'

\ex tsaˀdegyadǫ́dadeˀ\\
\gll tsaˀdeg-y-ad-ǫ́d-a-deˀ\\
{\coincident.\dualic.\cislocative}\exsc{-3s.p}-{\semireflexive}-period.of.time-{\joinerA}-exist.{\stative}\\
\glt `during the week'
\z
\z

\subsection{\stem{ti-} \textsc{contrastive} ({\contrastive})} \label{[ti-] (contrastive)}
The \stem{ti-} {\contrastive} conveys the idea that actions or characteristics are out of the ordinary or random \xref{ex:tiex}. It is also used in negation (see \sectref{[ti-] contrastive and negation}).

\ea\label{ex:tiex}

\ea tisǫgwe̱ˀdá:deˀ\\
\gll ti-s-ǫgwe̱ˀd-á:deˀ\\
{\contrastive}-\textsc{2s.a}-person-exist.{\stative}\\
\glt `you are a different person'
\ex tiyoyę̱hsrá:deˀ \\ 
\gll ti-yo-yę̱hsr-á:deˀ\\
\textsc{\contrastive-3s.p}-blanket- exist.{\stative}\\
\glt ‘it is an odd-ball blanket’

\ex tihéhsǫˀ\\
\gll ti-h-é-hsǫˀ\\
{\contrastive}-\textsc{3s.m.a}-go-{\pluralizer}\\
\glt `he is a wanderer, loiterer'

\ex tigęnenǫ́:gyeˀs\\
\gll ti-gęn-enǫ́:-gye-ˀs\\
{\contrastive}-\textsc{3p.a}-originate.from-\exsc{\progressive}-{\habitual}\\
\glt `they are roaming about'
\z
\z


\subsubsection*{\stem{ti-\textsc{verb}}} \label{[ti-verb]}
The verbs in \xref{ex:tireq} require the \stem{ti-} {\contrastive} prefix to express their fixed meaning.

\ea\label{ex:tireq}

\ea tiyó:tˀah\\
\gll ti-yó:-t-ˀah\\
{\contrastive}-\textsc{3s.p}-stand.{\stative}-{\diminutive}\\
\glt `it is queer, unusual, odd'

\ex tiga:gwé:gǫh \\
\gll ti-ga:-gwé:g-ǫh\\
{\contrastive}-\textsc{3s.a}-all-{\stative}\\
\glt ‘all over the place’, `here and there'

\ex tiyogwé:gǫh\\
\gll ti-yo-gwé:g-ǫh\\
{\contrastive}-\textsc{3s.p}-all-{\stative}\\
\glt `all of it'

\ex tigǫ́:nihs\\
\gll ti-g-ǫ́:ni-hs\\
{\contrastive}-\textsc{1s.a}-make-{\habitual}\\
\glt `I use something in place of' (something else)

\ex tęwadadǫ́:niˀ\\
\gll t-ę-wa-dad-ǫ́:ni-ˀ\\
{\contrastive}-{\future}-\textsc{3s.a}-{\reflexive}-make-{\punctual}\\
\glt  ‘it will emerge or appear unintentionally’, `it will do it by itself'
\z
\z

The fixed expressions (verbs) in \xref{ex:tireq2} require the \stem{taˀde-} \textsc{\contrastive.\dualic} combination to convey their special meanings. 

\ea\label{ex:tireq2}

\ea taˀdehoˀnigǫ̱há:nih\\
\gll taˀde-ho-ˀnigǫ̱h-á:-nih\\
{\contrastivedualic}\exsc{-3s.m.p}-mind-{\joinerA}-\exsc{\benefactive.\habitual}\\
\glt `he is unconcerned, indifferent'

\ex taˀdegri̱hwáhsnyeˀ \\
\gll taˀde-g-ri̱hw-áhsnye-ˀ\\
{\contrastivedualic}\exsc{-1s.a}-matter-care.for-{\stative}\\
\glt ‘I oppose it’, `I do not agree'

\ex taˀdeyogaháędaht\\
\gll taˀde-yo-gah-á-ędaht\\
{\contrastivedualic}\exsc{-3s.p}-eye-{\joinerA}-cause.to.\-lie.{\zeropunctual}\\
\glt `it is fuzzy, out of focus, opaque, unclear'
\z
\z


\subsubsection*{\stem{ti-…\textsc{verb}-ˀah} ‘do something any old way’} \label{[ti-verb+ˀah]}
The fixed combination of \stem{ti-…\textsc{verb}-ˀah} means ‘do something any old way’ \xref{ex:pppexpressionex17}.

\ea\label{ex:pppexpressionex17} \stem{ti-…\textsc{verb}-ˀah} ‘do something any old way’

\ea tihe:ˀáh tsǫ:\\
\gll ti-h-e:-ˀáh tsǫ:\\
{\contrastive}-\textsc{3s.m.a}-go-{\diminutive} just \\
\glt `he is just walking along'
\cfex{iheˀ \\
  \gll i-h-e-ˀ\\
  {\prothetic}-\textsc{3s.m.a}-go-{\stative}\\
  \glt `he is walking, moving'
}

\ex tahayętó:ˀah\\
\gll t-a-ha-yętó:-ˀah\\
{\contrastive}-{\factual}-\textsc{3s.m.a}-plant-{\diminutive}\\
\glt `he planted it any old way'
\cfex{  aháyętoˀ \\
\gll a-há-yęto-ˀ\\
\textsc{\factual-3s.m.a}-plant-{\punctual}\\
\glt ‘he planted it’  

}
\z
\z



\subsubsection*{\stem{ti-…(\textsc{incorporated noun})-deˀ(ah)} ‘a different or odd kind of (\textsc{incorporated noun})’} \label{[ti-(incorporated noun)-deˀ(ah)]}
The fixed combination of \stem{ti-…(\textsc{incorporated noun})-deˀ(ah)} means ‘a different or odd kind of (\textsc{incorporated noun})’ \xxref{ex:pppexpressionex18}{ex:pppexpressionex19}. The \stem{-ˀah} \textsc{\diminutive} is sometimes absent \xref{ex:pppexpressionex19a}, \xref{ex:pppexpressionex18}. Finally, the incorporated noun is not required \xref{ex:pppexpressionex15}.

\ea\label{ex:pppexpressionex19} \stem{ti-…(\textsc{incorporated noun})-deˀah} ‘a different or odd \textsc{incorporated noun}’\\
tiganǫ̱hsa:dé:ˀah\\\label{ex:pppexpressionex19a}
\gll ti-ga-nǫ̱hs-a:-dé:-ˀah\\
{\contrastive}-\textsc{3s.a}-house-{\joinerA}-exist.{\stative}-{\diminutive}\\
\glt `not a normal house compared to the others'
\z


\ea\label{ex:pppexpressionex18} \stem{ti-…(\textsc{incorporated noun})-deˀ} ‘a different or odd \textsc{incorporated noun}’

\ea tiyoyę̱hsrá:deˀ\\
\gll ti-yo-yę̱hsr-á:-deˀ\\
{\contrastive}-\textsc{3s.p}-blanket-{\joinerA}-exist.{\stative}\\
\glt `it is an odd-ball blanket'

\ex tigaˀdre̱hdá:deˀ\\
\gll ti-ga-ˀdrehd-á:-deˀ\\
{\contrastive}-\textsc{3s.a}-car-{\joinerA}-exist.{\stative}\\
\glt `it is a different car from the others'

\ex tiganǫ̱hsá:deˀ\\
\gll ti-ga-nǫ̱hs-á:-deˀ\\
{\contrastive}-\textsc{3s.a}-house-{\joinerA}-exist.{\stative}\\
\glt `it is a different house from the others'
\z
\z

\ea\label{ex:pppexpressionex15}
\gll “Ó:. Tigá:déˀ gę́h diˀ ni:yóht tó:gyęh?” \\
Oh it.is.different Q so it.is.a.certain.kind that.one\\
\glt ‘“Oh, is that a different kind?”‘ (\cite{henry_de_2005})
\cfex{
  \gll ti-gá:-déˀ\\
  {\contrastive}-\textsc{3s.a}-exist.{\stative}\\
}
\z

\subsubsection*{\stem{taˀde-…\textsc{incorporated noun}-deˀ} ‘two different \textsc{incorporated nouns}’} \label{[taˀde-incorporated noun-deˀ]}
The fixed combination of \stem{taˀde-…\textsc{incorporated noun}-deˀ} means ‘two different \textsc{incorporated nouns}’ \xref{ex:pppexpressionex14}.

\ea\label{ex:pppexpressionex14} \stem{taˀde-…\textsc{incorporated noun}-deˀ} ‘two different \textsc{incorporated nouns}’\\
\gll taˀdega̱ˀdrehdadéˀ hodiˀdre̱hdáęˀ de̱hęnadade̱ˀgę́:ˀęh \\
two.different.cars they.own two.brothers\\
\glt ‘The two brothers have two different cars.’ 
\cfex{
  \gll taˀde-ga̱-ˀdrehd-a-déˀ\\
  {\contrastivedualic}\exsc{-3s.a}-vehicle-{\joinerA}-exist.{\stative} \\
}
\z

\subsection{\stem{ni-} \textsc{partitive} ({\partitive})} \label{[ni-] (partitive)}
The \stem{ni-} \textsc{\partitive} denotes \begin{quote}
    
…manner or extent. A verb form with the partitive prefix describes the manner in which an action is carried out or the extent or degree to which the condition or state described by the verb holds true (e.g., it is really expensive). (\cite[98]{michelson_ontario_2011})\end{quote}  Examples are provided in the following sections.

\subsubsection*{\stem{ni-\textsc{verb}}} \label{[ni-verb]}
Many verbs require the \stem{ni-} \textsc{\partitive} as a fixed element. Some examples are listed in \ref{ex:niex10}, and more are provided in the following sections. For a more comprehensive list, see \sectref{verb dictionary}.

\ea\label{ex:niex10}
\ea shęh ní:waˀs \\
\gll shęh ní:-w-aˀs\\
that \textsc{\partitive-3s.a}-be.a.certain.size.{\stative}\\
\glt ‘sizes’, ‘how big they are’ 

\ex ní:yoht\\
\gll ní:-yo-ht\\
{\partitive}-\textsc{3s.p}-resemble.{\stative}\\
\glt `what it is like' (preceded by a particle such as \textit{dęˀ} ‘what’, \textit{neˀ} ‘the’)

\ex ní:yǫ:\\
\gll ní:-y-ǫ:\\
{\partitive}-\textsc{3s.p}-be.a.certain.amount.{\stative}\\
\glt `an amount of things'

\ex {} \stem{ni-…+oˀdę:} ‘a type of’, as in:\\
\gll ni-ga-ęn-óˀdę:\\
{\partitive}-\textsc{3s.a}-song-be.a.type.of.{\stative}\\
\glt `a type of song'

\ex niyó:weˀ\\
\gll ni-yó:-weˀ\\
{\partitive}-\textsc{3s.p}-be.a.certain.distance.{\stative}\\
\glt `how far in distance'

\ex niwuˀdrugyé:ˀah\\
\gll ni-w-uˀdrígyé:ˀah\\
\textsc{\partitive-3s.a}-narrow.{\stative}\\
\glt `it is narrow'

\ex niyo:há:kˀah\\
\gll ni-yo:-há:-kˀah\\
{\partitive}-\textsc{3s.p}-measuring.line-short.{\stative}\\
\glt `it is short in height'

\ex niyǫ́:hah\\
\gll ni-y-ǫ́:-hah\\
{\partitive}-\textsc{3s.p}-be.a.certain.amount.{\stative}-{\diminutive}\\
\glt `a few, a little bit'

\ex niyǫ́:sˀah\\
\gll ni-y-ǫ́:s-ˀah\\
\textsc{\partitive-3s.p}-long.{\stative}-{\diminutive}\\
\glt `just a little bit long'

\ex niwagri̱hú:ˀuh\\
\gll ni-wag-ri̱h-ú:ˀuh\\
{\partitive}-\textsc{1s.p}-matter-small.{\stative}\\
\glt `I am sensitive'

\ex nitgá:deˀ \\
\gll ni-t-gá:-deˀ\\
{\partitive}-{\cislocative}-\textsc{3s.a}-exist.{\stative}\\
\glt ‘how high it is (inanimate object)’, `the height of something'

\ex nigáha:ˀ\\
\gll ni-gá-ha:-ˀ\\
{\partitive}-\textsc{3s.a}-hold-{\stative}\\
\glt `how much it holds'

\ex niyohnodá:kˀah\\
\gll ni-yo-hnod-á:-kˀah\\
{\partitive}-\textsc{3s.p}-water-{\joinerA}-short.{\stative}\\
\glt `it is shallow'

\ex niwú:ˀuh\\
\gll ni-w-ú:ˀuh\\
{\partitive}-\textsc{3s.a}-small.{\stative}\\
\glt `it is small, little'
\z
\z


\subsubsection*{\stem{ni-} \textsc{\partitive} with verbs of direction or origin} \label{[ni-] (partitive) with verbs of direction or origin}
Several verbs with the \stem{ni-} \textsc{\partitive} describe actions or states originating from or extending in a certain direction \xxref{ex:niex}{ex:niex26}.

\ea\label{ex:niex} \stem{ni-...e:no:ˀ} ‘to come from some place’ 

\ea nita:wé:nǫ:\\
\gll ni-t-ha:w-é:nǫ-:\\
{\partitive}-{\cislocative}-\textsc{3s.m.p}-originate.from-{\stative}\\
\glt `a stranger'


\ex Do: nidihsé:no:?  \\
\gll do: ni-di-hs-é:no-:\\
how {\partitive}-{\cislocative}-\textsc{2s.p}-originate.from-{\stative}\\
\glt ‘How old are you?’

\ex Gaę diˀ nhǫ:wéh niha:wé:no:?  \\
\gll Gaę diˀ nhǫ:wéh ni-ha:w-é:no-:\\
which then place {\partitive}-\textsc{3s.m.p}-originate.from-{\stative}\\
\glt ‘Where, then, did he go?’

\ex Gaę nhǫ: nǫdi:sé:nǫ:?  \\
\gll Gaę nhǫ: nǫdi:-s-é:nǫ-:\\
which place {\partitive}-{\cislocative}-\textsc{2s.p}-originate.from-{\stative}\\
\glt ‘Where are you from?’
\z
\z

\ea\label{ex:niex25} \stem{nǫda-…e:} ‘to come from some place’ 

\ea Gaę nǫdáhse:? \\
\gll Gaę nǫdá-hs-e-:\\
which {\partitive}-{\factual}-{\cislocative}-\textsc{2s.a}-go-{\purposive}\\
\glt ‘Where do you come from?’

\ex nǫdá:ge:\\
\gll nǫdá:-g-e-:\\
{\partitive}-{\factual}-{\cislocative}-\textsc{1s.a}-go-{\purposive}\\
\glt `where I came from'
\z
\z

\ea\label{ex:niex26} \stem{ni-…ohsr-iyaˀgǫˀ} ‘to be a certain age’\\
ni̱hosriyáˀgǫˀ\\
\gll ni̱-h-ohsr-iyáˀg-ǫˀ\\
{\partitive}-\textsc{3s.m.a}-winter-cross.{\stative}-\exsc{plural}\\
\glt `his age'
\z

For more examples, see \stem{naˀde-\exsc{a}-adreˀ} ‘to be a certain distance apart’, \sectref{[naˀde-A-adreˀ]} and \stem{\exsc{ni-yo}-yaˀdǫh} ‘in the direction of’, \sectref{[ni-yo-yaˀdǫh] ‘in the direction of’}.


\subsubsection*{\stem{ni-} \textsc{\partitive} with verbs describing group affiliation} \label{[ni-] (partitive) with verbs describing group affiliation}
Several verbs with the \stem{ni-} \textsc{\partitive} element convey the idea of being part of a group \xref{ex:niex2}. (Also see \textit{Clan names}, \sectref{clans}.)

\ea\label{ex:niex2} 
\ea
\gll Gayogo̱ho:nǫ́h gęh ni:s nahsyáˀdoˀdę:? \\
Cayuga Q you what.kind.of.body \\
\glt ‘Are you Cayuga?’ 
\cfex{
  \gll n-a-hs-yaˀd-oˀdę:\\
  {\partitive}-{\factual}-\textsc{2s.a}-body-be.a.certain.kind.{\stative}\\
}

\ex
\gll Gayogo̱ho:nǫ́h niwago̱hwę́joˀdę:. \\
Cayuga  what.my.kind.of.land.is\\
\glt ‘I am of the Cayuga nation.’ 
\cfex{
  \gll ni-wag-ohwęj-oˀdę:\\
  {\partitive}-\textsc{1s.p}-land-be.a.certain.kind.{\stative} \\
}

\ex
\gll Dęˀ ni:s hoˀdę́ˀ nisaˀsyáoˀdę:. \\
 what you kind what.my.clan.is \\
\glt ‘What is your clan?’ 
\cfex{
  \gll ni-sa-ˀsya-oˀdę:\\
  {\partitive}-\textsc{2s.p}-clan-be.a.certain.kind.{\stative} \\
}


\ex nǫgesyáoˀdę:\\
\gll n-ǫge-sya-oˀdę:\\
{\partitive}-\exsc{\factual.1s.p}-clan-be.a.certain.kind.{\stative}\\
\glt `I am ... clan'

\ex niwage̱hsyáoˀdę:\\
\gll ni-wage-hsya-oˀdę:\\
{\partitive}-\textsc{1s.p}-clan-be.a.certain.kind.{\stative}\\
\glt `I am ... clan'
\z
\z


\subsubsection*{\stem{ni-} {\partitive} with verbs describing measures or degrees} \label{[ni-] (partitive) with verbs describing measures or degrees}
Several verbs with the \stem{ni-} \textsc{\partitive} convey the idea that a state or activity exists in degrees and can be measured (\ref{ex:niex5}, see \textit{Comparisons}, \sectref{ch:Comparisons (more, the same, or less)}.)

\ea\label{ex:niex5} 

\ea niyó:tgiˀ\\
\gll ni-yó:-tgiˀ\\
{\partitive}-\textsc{3s.p}-ugly.{\stative}\\
\glt `how dirty or filthy it is'
\cfex{otgiˀ \\
  \gll o-tgiˀ\\
  \textsc{3s.p}-ugly.{\stative}\\
  \glt `it is dirty or filthy'
}

\ex niyéha:ˀ\\
\gll ni-yé-ha:-ˀ\\
{\partitive}-\textsc{3s.fi.a}-hold-{\habitual}\\
\glt `how much she holds'
\cfex{eha:ˀ \\
  \gll e-ha:-ˀ\\
  \textsc{3s.fi.a}-hold-{\habitual}\\
  \glt `she is holding something'
}

\ex nigá:nǫ:ˀ\\
\gll ni-gá:-nǫ:ˀ\\
{\partitive}-\textsc{3s.a}-expensive.{\stative}\\
\glt `how much it costs'
\cfex{ganǫ:ˀ \\
  \gll ga-nǫ:ˀ\\
  \textsc{3s.a}-expensive.{\stative}\\
  \glt `it is expensive'
}

\ex ní:ga:ˀ\\
\gll ní:-g-a:ˀ\\
{\partitive}-\textsc{3s.a}-hold.{\stative}\\
\glt `how much is contained in'
\cfex{í:ga:ˀ\\
\gll í:-g-a:ˀ\\
\textsc{prothetic-3s.a}-hold.{\stative}\\
\glt ‘it holds something’ (as in \textit{odí: í:ga:ˀ} ‘there is tea in there’) 
}
\z
\z



\subsubsection*{\stem{ni-} \textsc{\partitive} with verbs describing a period of time} \label{[ni-] (partitive) with verbs describing a period of time}
Several verbs with the \stem{ni-} \textsc{\partitive} describe an action that extends over a certain period of time \xref{ex:niex3}.

\ea\label{ex:niex3}

\ea na̱ˀónisheˀ\\
\gll n-a̱ˀ-ó-nishe-ˀ\\
{\partitive}-{\factual}-\textsc{3s.p}-take.time-{\punctual}\\
\glt `it took an amount of time'

\ea nęyónisheˀ\\
\gll n-ę-yó-nishe-ˀ\\
{\partitive}-{\future}-\textsc{3s.p}-take.time-{\punctual}\\
\glt `it will take an amount of time'
\z

\ex na̱ˀá:węh\\
\gll n-a̱ˀ-á:w-ę-h\\
{\partitive}-{\factual}-\textsc{3s.p}-happen-\textsc{euph.h}/{\zeropunctual}\\
\glt `it happened (a fact)'

\ea nęyá:węh\\
\gll n-ę-yá:w-ę-h\\
{\partitive}-{\future}-\textsc{3s.p}-happen-\textsc{euph.h}/{\zeropunctual}\\
\glt `it will happen'

\ex na:yá:węh\\
\gll n-a:-yá:w-ę-h\\
{\partitive}-{\indefinite}-\textsc{3s.p}-happen-\textsc{euph.h}/{\zeropunctual}\\
\glt `it might, could, or would happen'
\z

\ex nigyé:haˀ\\
\gll ni-g-yé:-haˀ\\
{\partitive}-\textsc{1s.a}-do-{\habitual}\\
\glt `I do it all the time'

\ea nę́:gye:ˀ\\
\gll n-ę́:-g-ye:-ˀ\\
{\partitive}-{\future}-\textsc{1s.a}-do-{\punctual}\\
\glt `I will do it'

\ex niyó:yę:\\
\gll ni-yó:-yę:\\
{\partitive}-\textsc{3s.p}-do.{\stative}\\
\glt `what it is doing'
\z
\z
\z

\subsubsection*{\stem{shęh na̱ˀonisheˀ} ‘while’} \label{shęh naˀonisheˀ}

The fixed combination of \stem{shęh na̱ˀonisheˀ} means ‘while’ (\ref{ex:niex4}, see \sectref{ch:Clauses with [shęh naˀonisheˀ], [tsaˀonisheˀ], [tsi-] ‘while, when’}).

\ea\label{ex:niex4}
\ea
\gll Agiˀda̱ˀǫ́h ǫ:  \exemph{shęh}  \exemph{na̱ˀonishéˀ} hohta:ˀ. \\
I.slept it.seems that during.the.time he.spoke \\
\glt ‘I slept while he spoke.’ 
\cfex{
  \gll n-a̱ˀ-o-nishé-ˀ\\
  {\partitive}-{\factual}-\textsc{3s.p}-take.time-{\punctual}\\
}

\ex
\gll Agahyagóˀ  \exemph{shęh}  \exemph{na̱ˀonishéˀ} odahyǫ́:ni:. \\
I.picked.fruit that during.the.time there.was.lots.of.fruit \\
\glt ‘I picked fruit while it was plentiful.’ 
\z
\z


\subsubsection*{\stem{ni-\textsc{a}-ǫ:} ‘how many living things’} \label{[ni-A-ǫ:]}
The fixed combination of \stem{ni-\textsc{a}-ǫ:}, with an \textsc{a}-series pronominal prefix, means ‘how many living things’ \xref{ex:niex8}. (Also see \textit{Comparisons}, \sectref{ch:Comparisons, counting, measuring}.)

\ea\label{ex:niex8} \stem{ni-\textsc{a}-ǫ:} ‘how many living things’

\ea nigę́:nǫ:\\
\gll ni-gę́:n-ǫ:\\
{\partitive}-\textsc{3p.a}-certain.number.of.living.things.{\stative}\\
\glt `how many of them' (animals)

\ex nigá:gǫ:\\
\gll ni-gá:g-ǫ:\\
{\partitive}-\textsc{3ns.fi.a}-certain.number.of.living.things.{\stative}\\
\glt `how many of them' (people, females or mixed)
\z
\z


\subsubsection*{\stem{ni-\textsc{p}-at/(\textsc{incorporated noun})-gaˀdeˀ} ‘to be/have many living things’} \label{[ni-O-at/incorporated noun-gaˀdeˀ]}
The fixed combination of \stem{ni-\textsc{p}-at/-gaˀdeˀ}, with a \textsc{p}-series pronominal prefix, means ‘to be many living things’ \xref{ex:niex9}. (Also see \textit{Comparisons}, \sectref{ch:Comparisons, counting, measuring}.)

\ea\label{ex:niex9} \stem{ni-\textsc{p}-at/at-gaˀdeˀ} ‘to be so many living things’\\
niyonatgáˀdeˀ\\
\gll ni-yon-at-gáˀdeˀ\\
{\partitive}-\exsc{3p.p}-{\semireflexive}-be.many.{\stative}\\
\glt `there are so many' (animals)
\z


The fixed combination of  \stem{ni-\textsc{p}-\textsc{incorporated noun}(living thing)-gaˀdeˀ}, with a \textsc{p}-series pronominal prefix, means ‘to have many living things’ \xref{ex:niex99}. (Also see \textit{Comparisons}, \sectref{ch:Comparisons, counting, measuring}.)

\ea\label{ex:niex99} \stem{\textsc{ni}-\textsc{p}-\textsc{incorporated noun}(living thing)-gaˀdeˀ} ‘to have many living things’\\
honahsgwagáˀdeˀ\\
\gll ho-nahsgw-a-gáˀdeˀ\\
\textsc{3s.m.p}-domestic.animal-{\joinerA}-be.many.{\stative}\\
\glt `he has many pets'
\z


\subsubsection*{\stem{ni-…\textsc{incorporated noun}-age:} ‘to be a certain amount of \textsc{incorporated nouns}’} \label{[ni-incorporated noun-age:]}
The fixed combination of \stem{ni-…\textsc{incorporated noun}-age:} means ‘to be a certain amount of \textsc{incorporated nouns}’ \xref{ex:niex6}. (Also see \textit{Comparisons}, \sectref{ch:Comparisons, counting, measuring}.)

\ea\label{ex:niex6} \stem{ni-…\textsc{incorporated noun}-age:} ‘to be a certain amount of \textsc{incorporated nouns}’

\ea niyowęyohgá:ge:\\
\gll ni-yo-węyohg-á:ge:\\
{\partitive}-\textsc{3s.p}-thumb-certain.number.of.{\stative}\\
\glt `it is so many inches'

\ex nigaˀahdrá:ge: \\
\gll ni-ga-ˀahdr-á:ge: \\
\textsc{\partitive-3s.a}-basket-certain.number.of.{\stative}\\
\glt ‘that many baskets’

\ex nigagǫ̱ˀtrá:ge:\\
\gll ni-ga-gǫ̱ˀtr-á:ge:\\
\textsc{\partitive-3s.a}-pound-certain.number.of.{\stative}\\
\glt `that many pounds'

\ex nigahǫhsrá:ge:\\
\gll ni-ga-hǫhsr-á:ge:\\
{\partitive}-\textsc{3s.a}-box-certain.number.of.{\stative}\\
\glt `amount of boxes'

\ex nigana̱ˀjá:ge:\\
\gll ni-ga-na̱ˀj-á:ge:\\
{\partitive}-\textsc{3s.a}-pail-certain.number.of.{\stative}\\
\glt `amount of pails'

\newpage

\ex nigayá:ge:\\
\gll ni-ga-yá:-ge:\\
{\partitive}-\textsc{3s.a}-bag-certain.number.of.stative\\
\glt `amount of bags'

\ex niwęˀnhotrá:ge:\\
\gll ni-w-ęˀnhotr-á:ge:\\
{\partitive}-\textsc{3s.a}-ball-certain.number.of.{\stative}\\
\glt `amount of balls'

\ex nigawęná:ge:\\
\gll ni-ga-węn-á:ge:\\
{\partitive}-\textsc{3s.a}-word- certain.number.of.{\stative}\\
\glt `as many words'

\ex niyǫgwe̱ˀdá:ge:\\
\gll ni-y-ǫgwe̱ˀd-á:ge:\\
{\partitive}-\textsc{3s.a}-person- certain.number.of.{\stative}\\
\glt `that many people'
\z
\z


\subsubsection*{\stem{\textsc{numeral} … ni-…\textsc{incorporated noun}-age:} ‘three (etc.) \textsc{nouns}’} \label{Numeral + [ni-incorporated noun-age:]}
The fixed combination of \stem{\textsc{numeral} … ni-…\textsc{incorporated noun}-age:} means ‘three (etc.) \textsc{nouns}’ \xref{ex:pppexpressionex8}. The numeral specifies the number of objects. (For pronominal prefix selection, see \textit{Comparisons}, \sectref{ch:Comparisons, counting, measuring}.)

\ea\label{ex:pppexpressionex8} \stem{\textsc{numeral} … ni-…\textsc{incorporated noun}-age:} ‘three (etc.) \textsc{nouns}’

\ea ahsę́h niganǫ̱hsá:ge:\\
\gll ni-ga-nǫ̱hs-á:ge:\\
{\partitive}-\textsc{3s.a}-house-be.a.certain.number.{\stative}\\
\glt `three houses'

\ex ahsę́h niwahda̱hgwá:ge:\\
\gll ni-w-ahda̱hgw-á:ge:\\
{\partitive}-\textsc{3s.a}-shoe-be.a.certain.number.{\stative}\\
\glt `three shoes'

\ex ahsę́h niyohsi̱ˀdá:ge:\\
\gll ni-yo-hsi̱ˀd-á:ge:\\
{\partitive}-\textsc{3s.p}-foot-be.a.certain.number.{\stative}\\
\glt `three feet'

\newpage

\ex ahsę́h niyohǫnáˀdage:, ahsę́h nigahǫnáˀdage:\\
\gll ni-yo/ga-hǫnáˀd-á:ge:\\
{\partitive}-\exsc{3s.p/3s.a}-potato-be.a.certain.number.{\stative}\\
\glt `three potatoes'
\z
\z


\subsubsection*{\stem{naˀde-\textsc{a}-adreˀ} ‘to be a certain distance apart’} \label{[naˀde-A-adreˀ]}
The fixed combination of \stem{naˀde-\textsc{a}-adreˀ}, with an \textsc{a}-series pronominal prefix, means ‘to be a certain distance apart’ \xref{ex:pppexpressionex7}.

\ea\label{ex:pppexpressionex7} \stem{naˀde-\textsc{a}-adreˀ} ‘to be a certain distance apart’
\ea
\gll Jo̱hsiˀdatsǫ́ˀ naˀdegęna:dréˀ ganáˀjo̱hkwaˀ.  \\
one.foot.apart they.are.a.distance.apart bottles \\
\glt ‘The bottles are one foot apart.’
\cfex{
  \gll naˀ-de-gęn-a:dréˀ\\
  {\partitive}-{\dualic}-\textsc{3p.a}-distance.apart.{\stative}\\
}
\ex
\gll Jo̱hsiˀdatsǫ́ˀ naˀdegaǫdréˀ todinę̱hę́:dreˀ. \\
one.foot.apart they.are.a.distance.apart they.are.standing.in.line \\
\glt ‘People are standing in a straight line one foot apart.’
\cfex{
  \gll naˀ-de-gaǫ-dréˀ\\
  {\partitive}-{\dualic}-\textsc{3ns.fi.a}-distance.apart.{\stative}\\
}
\z
\z

\subsubsection*{\stem{ni-yo-yaˀdǫh} ‘in the direction of’} \label{[ni-yo-yaˀdǫh] ‘in the direction of’}
The fixed combination of \stem{ni-yo-yaˀdǫh} means ‘in the direction of’ \xref{ex:pppexpressionex9}.

\ea\label{ex:pppexpressionex9} \stem{ni-yo-yaˀdǫh} ‘in the direction of’
\ea (shęh) niyóya̱ˀdǫh ‘a direction’ 
\cfex{
  \gll ni-yo-yaˀdǫh\\
  {\partitive}-\textsc{3s.p}-be.a.direction.{\stative} \\
}
\ex
\gll Otowe̱ˀgéh niyoya̱ˀdǫ́h haˀdehagáhne:ˀ \\
north direction he.goes.and.looks.over.there \\
\glt ‘He goes outside and looks to the north.’ 
\z
\z

\subsubsection*{Clauses and particle expressions with \stem{ni-} \textsc{\partitive}} \label{Particle expressions with [ni-] (partitive)}
Clause types requiring a verb that begins with \stem{ni-} {\partitive} are described in the following sections.

\begin{CayugaRelated}
\item{}Comparisons, counting, measuring, \sectref{ch:Comparisons, counting, measuring}
\item{}Comparisons (more, the same, or less), \sectref{ch:Comparisons (more, the same, or less)}\\
\item{}Clauses with \textit{shęh} ‘that’, \stem{shęh ni-} ‘how, what’, \textit{ne:ˀ} ‘it is’, \sectref{ch:Clauses with [shęh] ‘that’, [shęh ni-] ‘how, what’, [ne:ˀ] ‘it is’}\\
\item{}Causative clauses with \textit{dęˀ ni:yoht shęh} ‘why’, \textit{neˀ hǫ:niˀ} ‘how’, \sectref{ch:Causative clauses with [dęˀ ni:yoht shęh] ‘why’, [neˀ hǫ:niˀ] ‘how’}\\
\item{}Manner clauses with \textit{dęˀ ni-} ‘how’, \textbf{shęh ni:yoht} ‘how so’, etc., \sectref{ch:Manner clauses with [dęˀ ni-] ‘how’, [shęh ni:yoht] ‘how so’}\\
\item{}Measuring clauses with \stem{do: … ni-} ‘how much, many’, etc., \sectref{ch:Measuring clauses with [do: … ni-] ‘how much, many’}\\
\item{}Clauses with \textit{hwę:dǫh}, \textit{nęh}, \textit{do: niyowi̱hsdaˀe:ˀ}, \textit{nę:gyęh hwaˀ}, \textit{ne:ˀ hwaˀ} ‘when’, \sectref{ch:Clauses with [hwę:dǫh], [nęh], [nę:gyęh hwaˀ], [ne:ˀ hwaˀ], etc. ‘when’} \\
\item{}Clauses with \textit{(gaoˀ) shęh niyo:weˀ} ‘before’, 'until’, \sectref{ch:Clauses with [(gaoˀ) shęh niyo:weˀ] ‘before, until’}\\
\item{}Clauses with \textit{shęh naˀonisheˀ}, \textit{tsaˀonisheˀ}, \stem{tsi-} ‘while’, ‘when’, \sectref{ch:Clauses with [shęh naˀonisheˀ], [tsaˀonisheˀ], [tsi-] ‘while, when’}\\
\end{CayugaRelated}

The following particles (listed in the \textit{Particle dictionary}, \sectref{ch:particle dictionary}) require the next verb to begin with the \stem{ni-} \textsc{\partitive}.
 
\begin{CayugaRelated}
\item{}\textit{Dęˀ hoˀdęˀ ni-} ‘how’, 'what way’ (“adverb” of manner)\\
\item{}\textit{Dęˀ ni:yoht shęh} ‘why?’\\
\item{}\textit{Do:} ‘how’\\
\item{}\textit{Do: ni-…nisheˀ} ‘how long’, ‘how much time’\\
\item{}\textit{Do: ni-…ǫ:} ‘how many people\\
\item{}\textit{Do: niyowi̱hsda̱ˀe:ˀ} ‘what time is it?’, ‘when?’\\
\item{}\textit{Gaoˀ ni-} ‘less so’\\
\item{}\textit{Gaoˀ shęh niyo:weˀ} ‘before’, 'until’\\
\item{}\textit{Gwahs shęh ni-} ‘as … as’\\
\item{}\textit{Hne:ˀ shęh} ‘because’\\
\item{}\textit{Ji trehs} ‘because’, ‘overly’\\
\item{}\textit{Ne:ˀ} ‘it is’\\
\item{}\textit{Ne:ˀ dagaihǫ:niˀ} ‘the reason why’, ‘that’s why’, ‘because’\\
\item{}\textit{Ne:ˀ tsǫ:} ‘because’, ‘it is just’\\
\item{}\textit{Shęh} ‘that’\\
\item{}\textit{Shęh} ‘that’, 'because’\\
\item{}\textit{Shęh naˀonisheˀ}, \textit{tsaˀonisheˀ}, \stem{tsi-} ‘while’, 'when’\\
\item{}\textit{Shęh ni:yoht ni-} ‘how’, ‘the manner in which’\\
\item{}\textit{Shęh niyo:weˀ} ‘as far as’, 'as much as’\\
\item{}\textit{Toh shęh ni-} ‘to that degree’\\
\end{CayugaRelated}

\subsection{\stem{s-, j-, ji-} \textsc{repetitive} ({\repetitive})} \label{[s-, j-, ji-] (repetitive)}
The \stem{s-, j-, ji-} {\repetitive} means ‘again’, and also describes a repeated action \xref{ex:srepex}. Related meanings are covered in the following sections. 

\ea\label{ex:srepex}

\ea shoyę́:twęh\\
\gll s-ho-yę́:tw-ęh\\
{\repetitive}-\textsc{3s.m.p}-plant-{\stative}\\
\glt `he planted it again'
\cfex{hoyę́:twęh \\
  \gll ho-yę́:tw-ęh\\
  \textsc{3s.m.p}-plant-{\stative}\\
  \glt `he planted it'
}

\ex sawá:dǫˀ\\
\gll s-a-w-á:dǫ-ˀ\\
{\repetitive}-{\factual}-\textsc{3s.a}-become.well-{\punctual}\\
\glt `it became again’, ‘it became well again' (i.e. it was sick)
\cfex{awá:dǫˀ \\
\gll a-w-á:dǫ-ˀ\\
\textsc{\factual-3s.a}-become.well-{\punctual}\\
\glt ‘it has become’, ‘it became’ 
}


\ex sawada̱hǫ́:dǫ:ˀ\\
\gll sa-w-ad-a̱hǫ́:dǫ:-ˀ\\
{\repetitive}-{\factual}-\textsc{3s.a}-{\semireflexive}-ask-{\punctual}\\
\glt `it asked again'
\cfex{awada̱hǫ́:dǫ:ˀ \\
  \gll a-w-ad-a̱hǫ́:dǫ:-ˀ\\
  {\factual}-\textsc{3s.a}-{\semireflexive}-ask-{\punctual}\\
  \glt `it asked'
}

\ex jago:kǫ́:ni:\\
\gll j-ago:-k-ǫ́:ni-:\\
{\repetitive}-\textsc{3s.fi.p}-food-make-{\stative}\\
\glt `she is cooking again'
\cfex{gokǫ́:ni: \\
  \gll go-k-ǫ́:ni-:\\
  \exsc{3fis}-food-make-{\stative}\\
  \glt `she is cooking'
}
\ex jodrę́:no:t\\
\gll j-o-d-rę́:n-o:t\\
\textsc{\repetitive-3s.p}-{\semireflexive}-song-stand.{\stative}\\
\glt  ‘it is singing again' 
\cfex{odrę́:no:t \\
  \gll o-d-rę́:n-o:t\\
  \textsc{3s.p}-{\semireflexive}-song-stand.{\stative}\\
  \glt `it is singing'
}

\ex jotó:weˀ\\
\gll j-o-t-hó:weˀ\\
{\repetitive}-\textsc{3s.p}-{\semireflexive}-cold.{\stative}\\
\glt `it is cold again'
\cfex{otó:weˀ \\
  \gll o-t-hó:weˀ\\
  \textsc{3s.p}-{\semireflexive}-cold.{\stative}\\
  \glt `it is cold'
}

\ex jidwáhshe:t\\
\gll ji-dwá-hshe:t\\
{\repetitive}-\exsc{1p.in.a}-count.{\noaspect}\\
\glt `let’s count again'
\cfex{dwa̱hshe:t \\
  \gll dwa̱-hshe:t\\
  \exsc{1p.in.a}-count.{\noaspect}\\
  \glt `let’s count'
}
\z
\z

\subsubsection*{\stem{s-, j-, ji-} {\repetitive} with verbs of motion} \label{{[s-, j-, ji-] (repetitive) with verbs of motion}}
With verbs of motion, the \stem{s-, j-, ji-} {\repetitive} means ‘coming back to the place of origin’ or ‘returning to the place of origin’. For example, \textit{sáeyǫˀ} means ‘she returned’ \xref{ex:srepex2}, not ‘she arrived again’.

\ea\label{ex:srepex2}

\ea saˀé:yǫˀ\\
\gll s-aˀ-é:-yǫ-ˀ\\
{\repetitive}-{\factual}-\textsc{3s.fi.a}-arrive-{\punctual}\\
\glt `she returned'
\cfex{aˀé:yǫˀ \\
  \gll a-ˀé:-yǫ-ˀ\\
  {\factual}-\textsc{3s.fi.a}-arrive-{\punctual}\\
  \glt `she arrived'
}

\ex sa̱hahdę́:diˀ\\
\gll s-a̱-ha-hdę́:di-ˀ\\
{\repetitive}-{\factual}-\textsc{3s.m.a}-leave-{\punctual}\\
\glt `he went home'
\cfex{ahahdę́:diˀ ‘he left’, \\
  \gll a-ha-hdę́:di-ˀ\\
  {\factual}-\textsc{3s.m.a}-leave-{\punctual}\\
  \glt `he went away'
}


\ex hęhsge:ˀ\\
\gll h-ę-hs-g-e-:-ˀ\\
{\translocative}-{\future}-{\repetitive}-\textsc{1s.a}-go-{\purposive}-{\punctual}\\
\glt `I am going back there, returning'
\cfex{hę́:geˀ \\
  \gll h-ę́:-g-e-ˀ\\
  {\translocative}-{\future}-\textsc{1s.a}-go-{\punctual}\\
  \glt `I will go there'
}

\ex ihsgeˀs\\
\gll i-hs-g-e-ˀs\\
{\prothetic}-{\repetitive}-\textsc{1s.a}-go-{\habitual}\\
\glt `I have returned home'
\cfex{í:geˀs \\
  \gll í:-g-e-ˀs\\
  {\prothetic}-\textsc{1s.a}-go-{\habitual}\\
  \glt `I am here'
}

\ex shahá:wiˀ\\
\gll s-ha-há:wi-ˀ\\
{\repetitive}-\textsc{3s.m.a}-carry-{\punctual}\\
\glt `he brought it with him'
\cfex{hadihá:wiˀs \\
  \gll hadi-há:wi-ˀs\\
  \textsc{3ns.m.a}-carry-{\habitual}\\
  \glt `they carry it along'
}
\z
\z

\largerpage
\subsubsection*{\stem{s-, j-, ji-} {\repetitive} with proper names and titles} \label{[s-, j-, ji-] (repetitive) with proper names and titles}
The \stem{s-, j-, ji-} {\repetitive} appears in words describing unique people or places, where it means ‘The One Who…’ or ‘The Place Where…’ \xref{ex:srepex3}. Many Hereditary Chief names begin with the repetitive prefix (see \sectref{chiefs}).

\ea\label{ex:srepex3}

\ea Shoyǫ́:we:s\\
\gll s-ho-yǫ́:w-e:s\\
{\repetitive}-\textsc{3s.m.p}-wampum-long.{\stative}\\
\glt `He has Long Wampum' (Gayogo̱hó:nǫˀ Chief Title)

\ex Sga̱ˀna:wá:dih\\
\gll s-ga̱-ˀna:w-á:dih\\
{\repetitive}-\textsc{3s.a}-pond-side.{\stative}\\
\glt `Opposite Side of the Pond, Swamp' (Onondaga Chief Title)


\ex Sganyada:digó:wah\\
\gll s-ga-nyada:-di-gó:wah\\
{\repetitive}-\textsc{3s.a}-lake-side.{\stative}-{\augmentative}\\
\glt `Europe'

\ex Sganyádaes \\
\gll s-ga-nyáda:-es\\
{\repetitive}-\textsc{3s.a}-lake-long.{\stative}\\
\glt ‘Long Lake’, `Skaneateles, N.Y.'
\z
\z


\subsubsection*{\stem{s-/j-/ji-verb} (fixed expressions)} \label{[s-, j-, ji-verb] (fixed expressions)}
The verbs in \xref{ex:srepex4} require the \stem{s-, j-, ji-} {\repetitive} prefix to express a fixed meaning.

\ea\label{ex:srepex4}

\ea hęjéhe:ˀ\\
\gll h-ę-j-é-he:ˀ\\
{\translocative}-{\future}-{\repetitive}-\textsc{3s.fi.a}-sitting.on.top.of.{\punctual}\\
\glt `it will be her birthday'\footnote{This verb requires both the {\translocative} and the {\repetitive}.}

\ex sa̱hęnada:tré:waht\\
\gll sa̱-hęn-ada:t-hré:waht\\
{\repetitive}-\textsc{3ns.m.a}-{\reflexive}-punish.{\zeropunctual}\\
\glt `they repented'

\largerpage
\ex ęhsgáeyǫˀt\\
\gll ę-hs-gáe-yǫˀt\\
\fut-{\repetitive}-\textsc{3ns.fi.a}-cause.to.arrive.{\zeropunctual}\\
\glt `they will bring it back'

\ex sgadę́hda:ˀ\\
\gll s-ga-dę́hda:ˀ\\
{\repetitive}-\textsc{3s.a}-lie.spread.out.on.the.ground.{\stative}\\
\glt `to resurface something'

\ex ęjisatgánya̱ˀgęˀ\\
\gll ę-ji-s-at-gánya̱ˀg-ę-ˀ\\
\fut-{\repetitive}-\textsc{2s.a}-{\semireflexive}-pay-{\benefactive}-{\punctual}\\
\glt `you will be reimbursed, refunded'


\ex jǫkihędǫ́ˀse:ˀ\\
\gll j-ǫki-hędǫ́-ˀs-e-:-ˀ\\
{\repetitive}-\exsc{3s.fi/3ns>1ns}-lead-{\dislocative}-go-{\purposive}-{\stative}\\
\glt `she has gone on before us' (ceremonial language, said of a recently deceased woman)

\ex shonǫ́ˀne:t\\
\gll s-ho-nǫ́ˀ-ne:t\\
{\repetitive}-\textsc{3s.m.p}-head-keep.in.line.{\stative}\\
\glt `he is behind him, he is next in line'

\ex ęjijadrihwa̱hsrǫ́:niˀ\\
\gll ę-ji-j-ad-rihw-a̱-hsrǫ́:ni-ˀ\\
\fut-{\repetitive}-\exsc{2d.a}-{\semireflexive}-matter-{\joinerA}-fix-{\punctual}\\
\glt `you two will reconcile'

\ex sawę́hsga:ˀ\\
\gll sa-w-ę́hs-ga:-ˀ\\
{\repetitive}-\textsc{3s.a}-foot-make.a.rattling.noise-{\punctual}\\
\glt `it spun out, took off with a rattling sound'
\z
\z

\subsubsection*{\stem{s/j/ji-…\textsc{incorporated noun}-adih} ‘other side of \textsc{incorporated noun}’} \label{[s/j/ji-pronominal prefix-incorporated noun-adih]}
The fixed combination of \stem{\exsc{s/j/ji-}…\textsc{incorporated noun}-adih} means ‘other side of \textsc{incorporated noun}’ \xref{ex:pppexpressionex10}. (Also see \textit{Gwa:dih, gwai} ‘to one side’, \sectref{p:[gwa:dih]}.)

\ea\label{ex:pppexpressionex10} \stem{s/j/ji-…\textsc{incorporated noun}-adih} ‘other side of \textsc{incorporated noun}’

\ea jo̱háhadih\\
\gll j-o̱-háh-adih\\
{\repetitive}-\textsc{3s.p}-road-side.{\stative}\\
\glt `the other side of the road'

\ex swa̱háhadih\\
\gll s-wa̱-háh-adih\\
{\repetitive}-\textsc{3s.a}-road-side.{\stative}\\
\glt `the other side of the road'
\z
\z


\subsubsection*{\stem{s/j/ji-…\textsc{incorporated noun}-t} ‘one \textsc{incorporated noun}’} \label{[s/j/ji-pronominal prefix-incorporated noun-t]}
The fixed combination of \stem{s/j/ji-…\textsc{incorporated noun}-t} means ‘one \textsc{incorporated noun}’ \xref{ex:pppexpressionex11}. (Also see \textit{Counting with basic nouns}, \sectref{ch:Counting with basic nouns} and \textit{Neuter stative-only counting verbs, with an incorporated noun}, \sectref{Neuter stative-only counting verbs, with an incorporated noun}.)

\ea\label{ex:pppexpressionex11} \stem{\exsc{s/j/ji-}…\textsc{incorporated noun}-t} ‘one \textsc{incorporated noun}’

\ea sganǫ́hsa:t\\
\gll s-ga-nǫ́hs-a:-t\\
{\repetitive}-\textsc{3s.a}-house-{\joinerA}-stand.{\stative}\\
\glt `one house'
\cfex{ganǫ́hsaˀ \\
  \gll ga-nǫ́hs-aˀ\\
  \textsc{3s.a}-house-{\nounstemformer}\\
  \glt `house'
}

\ex swahdáhgwa:t\\
\gll s-w-ahdáhgw-a:-t\\
{\repetitive}-\textsc{3s.a}-shoe-{\joinerA}-stand.{\stative}\\
\glt `one shoe'
\cfex{ahdáhgwaˀ \\
  \gll ahdáhgw-aˀ\\
  {\noprefix}.shoe-{\nounstemformer}\\
  \glt `shoes'
}

\ex sgayę́hsra:t, joyę́hsra:t\\
\gll s-ga-/jo-yę́hsr-a:-t\\
{\repetitive}-\textsc{3s.a}/\exsc{\repetitive.3s.p}-blanket-{\joinerA}-stand.{\stative}\\
\glt `one blanket'
\cfex{oyę́hsraˀ \\
  \gll o-yę́hsr-aˀ\\
  \textsc{3s.p}-blanket-{\nounstemformer}\\
  \glt `blankets'
}

\ex sga̱hǫ́na̱ˀda:t, jo̱hǫ́na̱ˀda:t\\
\gll s-ga-̱/jo-hǫ́na̱ˀd-a:-t\\
{\repetitive}-\textsc{3s.a}/\exsc{\repetitive.3s.p}-potato-{\joinerA}-stand.{\stative}\\
\glt `one potato'
\cfex{ohǫ́na̱ˀdaˀ \\
  \gll o-hǫ́na̱ˀd-aˀ\\
  \textsc{3s.p}-potato-{\nounstemformer}\\
  \glt `potatoes'
}
\z
\z


\subsubsection*{\stem{s/j/ji-…t-sǫˀ} ‘one each, one per, one at a time’} \label{[s/j/ji-pronominal prefix-t-sǫˀ]}
The fixed combination of \stem{s/j/ji-…t-sǫˀ} means ‘one each, one per, one at a time’ \xref{ex:eachperex5}.


\ea\label{ex:eachperex5} \stem{s/j/ji-…t-sǫˀ} ‘one each, one per, one at a time’

\ea sgatsǫˀ \\
\gll s-ga-t-shǫˀ\\
{\repetitive}-\textsc{3s.a}-stand.{\stat}-{\pluralizer}\\
\glt ‘one at a time’, `one each'
    \cfex{sga:t \\
    \gll s-ga:-t \\
    \textsc{\repetitive-3s.a}-stand.{\stative}\\
    \glt ‘one thing’ }
   
    
    \ex 
    \gll sgatsǫ́ˀ ęsáhshe:t \\
    by.ones you.will.count\\
    \glt ‘count by ones, one by one’ 
    
    \ex 
    \gll sgatsǫ́ˀ shę́h da̱ha:dí:yoˀ \\
    one.by.one that they(males).came.in\\
    \glt ‘they came in one by one’ 
    \z
    \z


\subsubsection*{\stem{s/j/ji-…\textsc{incorporated noun}-t-sǫˀ} ‘each, per \textsc{incorporated noun}’, ‘one \textsc{incorporated noun} at a time’} \label{[s/j/ji-pronominal prefix-incorporated noun-t-sǫˀ]}
The fixed combination of \stem{s/j/ji-…\textsc{incorporated noun}-t-sǫˀ} means ‘each, per \textsc{incorporated noun}’, or ‘one \textsc{incorporated noun} at a time’ \xref{ex:eachperex1}.

\ea\label{ex:eachperex1} \stem{s/j/ji-…\textsc{incorporated noun}-t-sǫˀ} ‘each, per \textsc{incorporated noun}’, or ‘one \textsc{incorporated noun} at a time’

\ea jo̱hsˀdatsǫ́ˀ\\
\gll j-o̱-hsíˀd-a-t-shǫˀ\\
{\repetitive}-\textsc{3s.p}-foot-{\joinerA}-stand.{\stat}-{\pluralizer}\\
\glt `a one foot distance'
\cfex{jo̱hsíˀda:t \\
  \gll j-o̱-hsíˀd-a:-t\\
  {\repetitive}-\textsc{3s.p}-foot-{\joinerA}-stand.{\stative}\\
  \glt `one foot'
}

\newpage
\ex sga̱ˀdréhdatsǫˀ \\
\gll s-ga̱-ˀdréhd-a-t-shǫˀ\\
{\repetitive}-\textsc{3s.a}-car-{\joinerA}-stand.{\stat}-{\pluralizer}\\
\glt ‘each car’, `one car at a time'
\cfex{sga̱ˀdréhda:t \\
  \gll s-ga̱-ˀdréhd-a:-t\\
  {\repetitive}-\textsc{3s.a}-car-{\joinerA}-stand.{\stative}\\
  \glt `one car'
}

\ex
\gll Jo̱hsiˀdatsǫ́ˀ naˀdegaǫdréˀ todinę̱hę́:dreˀ\\
one.foot.apart they(fe/males).are.meeting they(males).are.standing\\
\glt ‘People are standing in a straight line one foot apart.’ 

\ex
\gll Jo̱hsiˀdatsǫ́ˀ naˀdegęna:dréˀ ganáˀjo̱hkwaˀ\\
one.foot.apart they(fe/males).are.meeting bottle\\
\glt ‘The bottles are standing one foot apart.’ 

\ex
\gll Sga̱ˀdrehdatsǫ́ˀ shę́h gęnatsahá:goˀ \\
each.car that they(things).gathered.together.on.the.road\\
\glt ‘Each car turned in (at the same place)’
\z
\z


\subsection{\stem{d-} \textsc{cislocative} ({\cislocative}) and \stem{heˀ-} \textsc{translocative} ({\translocative})} \label{[d-] (cislocative) and [heˀ-] (translocative)}
The \stem{d-} {\cislocative} and \stem{heˀ-} \textsc{\translocative} prefixes convey opposite locations or directions, as illustrated in \xxref{ex:dhex}{ex:dhex2} and described in the next sections.

\ea\label{ex:dhex}

\ea dahsá:dih\\
\gll da-hs-á:di-h\\
{\cislocative}-\textsc{2s.p}-throw-\textsc{euph.h}/{\noaspect}\\
\glt `throw it towards me'

\ex heˀsá:dih\\
\gll heˀ-s-á:di-h\\
{\translocative}-\textsc{2s.p}-throw-\textsc{euph.h}/{\noaspect}\\
\glt `throw it away from me'
\z
\z

\ea\label{ex:dhex2}

\ea tgayę́twa̱hsǫˀ\\
\gll t-ga-yę́tw-a̱-hs-ǫˀ\\
{\cislocative}-\textsc{3s.a}-plant-{\joinerA}-{\habitual}-\exsc{plural}\\
\glt `it is planted over there'

\newpage
\ex hegayętwáhsǫˀ\\
\gll he-ga-yętw-á-hs-ǫˀ\\
{\translocative}-\textsc{3s.a}-plant-{\joinerA}-{\habitual}-\exsc{plural}\\
\glt `it is planted over there'
\cfex{gayę́twa̱hsǫˀ \\
  \gll ga-yę́tw-a̱-hs-ǫˀ\\
  \textsc{3s.a}-plant-{\joinerA}-{\habitual}-\exsc{plural}\\
  \glt `it is planted (here)'
}
\z
\z

\subsubsection*{Motion verbs with \stem{d-} {\cislocative} and \stem{heˀ-} \textsc{\translocative}} \label{Motion verbs with [d-] (cislocative) and [heˀ-] (translocative)}
For verbs of motion, the \stem{d-} {\cislocative} prefix means ‘towards’ a point of reference, often the speaker. In contrast, the \stem{heˀ-} {\translocative} prefix means ‘away’ from a point of reference such as the speaker \xref{ex:dhex3}. 

\ea\label{ex:dhex3}

\ea dahsá:dih\\
\gll dah-s-á:di-h\\
{\cislocative}-\textsc{2s.p}-throw-\textsc{euph.h}/{\noaspect}\\
\glt `throw it to me'
\cfex{heˀsá:dih \\
  \gll heˀ-s-á:di-h\\
  {\translocative}-\textsc{2s.p}-throw-\textsc{euph.h}/{\noaspect}\\
  \glt `throw it away from me'
}
\ex dasa:dó:wih \\
\gll da-s-a:d-ó:wi-h \\
\textsc{\cislocative-2s.a}-{\semireflexive}-drive-\textsc{euph.h}/{\noaspect}\\
\glt ‘drive over here’

\cfex{haˀsa:dó:wih \\
  \gll haˀ-s-a:d-ó:wi-h\\
  {\translocative}-\textsc{2s.a}-{\semireflexive}-drive-\textsc{euph.h}/{\noaspect}\\
  \glt `drive it over there'
}

\ex dasádawę:\\
\gll da-s-ád-awę:\\
{\cislocative}-\textsc{2s.a}-{\semireflexive}-swim.{\noaspect}\\
\glt `swim this way'
\cfex{haˀsádawę: \\
  \gll haˀ-s-ád-awę:\\
  {\translocative}-\textsc{2s.a}-{\semireflexive}-swim.{\noaspect}\\
  \glt `swim over there'
}

\newpage
\ex desáˀdre:\\
\gll de-sá-ˀdre:\\
{\cislocative}-\textsc{2s.p}-drive.{\noaspect}\\
\glt `drive over here'
\cfex{heˀsáˀdre: \\
  \gll heˀ-sá-ˀdre:\\
  {\translocative}-\textsc{2s.p}-drive.{\noaspect}\\
  \glt `drive over there'
}

\ex dasrá:tęh, dadrá:tęh \\
\gll da-s/d-rá:tę-h\\
\textsc{\cislocative-2s.a}-climb-\textsc{euph.h}/{\noaspect}\\
\glt ‘climb (over here)!’  
\cfex{haˀsrá:tęh, haˀdrá:tęh \\
\gll haˀ-s/d-rá:tę-h\\
\textsc{\translocative-2s.a}-climb-\textsc{euph.h}/{\noaspect}\\
\glt ‘climb over there!’ 
}
\z
\z

The same is true of the verbs of motion in \xxref{ex:dhex4}{ex:dhex7} and of the verb in \xref{ex:dhex8}, which also conveys a type of motion.

\ea\label{ex:dhex4} \stem{yǫ} ‘arrive’

\ea dajǫh\\
\gll da-j-ǫ-h\\
{\cislocative}-\textsc{2s.a}-arrive-\textsc{euph.h}/{\noaspect}\\
\glt `come in' (destination is near the speaker)
\cfex{  haˀjǫh \\
\gll haˀ-j-ǫ-h \\
\textsc{\translocative-2s.a}-arrive-\textsc{euph.h}/{\noaspect}\\
\glt ‘go in, enter’ (destination is away from speaker) 
}

\ex daha:dí:yǫˀ\\
\gll da-ha:dí:-yǫ-ˀ\\
{\cislocative}-\textsc{3ns.m.a}-arrive-{\punctual} \\
\glt `they came in' (destination is near the speaker)
\cfex{  haˀha:dí:yǫˀ \\
\gll haˀ-ha:dí:-yǫ-ˀ\\
\textsc{\translocative-3ns.m.a}-arrive-{\punctual}\\
\glt ‘they went in there, they arrived’ (destination is away from the speaker)

}
\z
\z

\newpage
\ea\label{ex:dhex5} \stem{e} ‘go’\\
dagę́:neˀ\\
\gll da-gę́:n-e-ˀ\\
{\cislocative}-\textsc{3p.a}-go-{\stative}\\
\glt `they are coming' (destination is near the speaker)
\cfex{ haˀgę́:neˀ \\
\gll haˀ-gę́:n-e-ˀ \\
\textsc{\translocative-3p.a}-go-{\stative}\\
\glt ‘they are going’ (destination is away from the speaker) 
}
\z


\ea\label{ex:dhex6} \stem{ahdędi, -ahdęgy} ‘leave, go away’\\
disáhdęgyǫ:\\
\gll di-s-áhdęgyǫ-:\\
{\cislocative}-\textsc{2s.p}-leave-{\stative}\\
\glt `you come from there' (destination is near the speaker)
\cfex{  hesáhdęgyǫ: \\
\gll he-s-áhdęgyǫ-:\\
{\translocative}-\textsc{2s.p}-leave-{\stative}\\
\glt ‘you went over there’ (destination is away from the speaker) 
}
\z


\ea\label{ex:dhex7} \stem{ahshawi, ahsha:} ‘carry, bring’\\
dahsha:\\
\gll da-hs-ha:\\
{\cislocative}-\exsc{2s.a}-carry.{\noaspect} \\
\glt `bring it' (destination is near the speaker)
\cfex{  haˀsha: \\
\gll haˀ-s-ha: \\
{\translocative}-\textsc{2s.a}-carry.{\noaspect}\\
\glt ‘take it over there’ (destination is away from the speaker) 
}
\z


\ea\label{ex:dhex8} \stem{yęhętw} ‘pull’\\
gyagogyę̱hę́:twęh\\
\gll g-yago-g-yę̱hę́:tw-ęh\\
{\cislocative}-\textsc{3s.fi.p}-{\semireflexive}-pull-{\stative}\\
\glt `she is pulling it apart' (happens near the speaker)
\cfex{  heyagogyę́hętwęh \\
\gll he-yago-g-yę́hętw-ęh\\
{\translocative}-\textsc{3s.fi.p}-{\semireflexive}-pull-{\stative}\\
\glt ‘she is pulling from there’ (pulling starts away from the speaker) 
}
\z

\newpage
\subsubsection*{Positional verbs with \stem{d-} {\cislocative} and \stem{heˀ-} \textsc{\translocative}} \label{Positional verbs with [d-] (cislocative) and [heˀ-] (translocative)}
For positional verbs (a type of stative-only verb -- see \sectref{ch:Incorporating positional (stative-only) verbs}), the \stem{d-} prefix means ‘there’ (closer to the speaker and the \stem{heˀ-} prefix means ‘over there’ (farther away from the speaker, \ref{ex:dhex10ai}, \ref{ex:dhex10aii}). In contrast, positional verbs without either prefix have the implied meaning of ‘here’ (\ref{ex:dhex10a}, \ref{ex:dhex10b}).

\ea\label{ex:dhex10} 

\ea\label{ex:dhex10a} ga̱he:ˀ\\
\gll ga̱-he:ˀ\\
\textsc{3s.a}-sitting.on.top.of.{\stative}\\
\glt `it is sitting (here)'

\ea tga̱he:ˀ\\ \label{ex:dhex10ai}
\gll t-ga̱-he:ˀ\\
{\cislocative}-\textsc{3s.a}-sitting.on.top.of.{\stative}\\
\glt `it is setting there'

\ex\label{ex:dhex10aii} heˀgáhe:ˀ\\
\gll heˀ-gá-he:ˀ\\
{\translocative}-\textsc{3s.a}-sitting.on.top.of.{\stative}\\
\glt `it is sitting way over there'
\z

\ex gá:yęˀ\\ \label{ex:dhex10b}
\gll gá:-yęˀ\\
\textsc{3s.a}-lie.{\stative}\\
\glt `it is lying'

\ea tgá:yęˀ\\
\gll t-gá:-yęˀ\\
{\cislocative}-\textsc{3s.a}-lie.{\stative}\\
\glt `it is lying there'

\ex hegá:yęˀ\\
\gll he-gá:-yęˀ\\
{\translocative}-\textsc{3s.a}-lie.{\stative}\\
\glt `it is lying over there'
\z

\ex ganí:yǫ:t\\
\gll ga-ní:yǫ:t\\
\textsc{3s.a}-hang.{\stative}\\
\glt `it is hanging'

\newpage
\ea tga:ní:yǫ:t\\
\gll t-ga:-ní:yǫ:t\\
{\cislocative}-\textsc{3s.a}-hang.{\stative}\\
\glt `it is hanging there'

\ex hega:ní:yǫ:t\\
\gll he-ga:-ní:yǫ:t\\
{\translocative}-\textsc{3s.a}-hang.{\stative}\\
\glt `it is hanging way over there'
\z

\ex gá:deˀ\\
\gll gá:-deˀ\\
\textsc{3s.a}-stick.out.{\stative}\\
\glt `it is sticking out'

\ea tgá:deˀ\\
\gll t-gá:-deˀ\\
{\cislocative}-\textsc{3s.a}-stick.out.{\stative}\\
\glt `it is sticking out'

\ex hegá:deˀ\\
\gll he-gá:-deˀ\\
{\translocative}-\textsc{3s.a}-stick.out.{\stative}\\
\glt `it is sticking out over there'
\z

\ex eˀdrǫˀ\\
\gll e-ˀdrǫˀ\\
\textsc{3s.fi.a}-dwell.{\stative}\\
\glt `she is home'

\ea gye̱ˀdrǫˀ\\
\gll g-ye̱-ˀdrǫˀ\\
{\cislocative}-\textsc{3s.fi.a}-dwell.{\stative}\\
\glt `she is at home'

\ex heyéˀdrǫˀ\\
\gll he-yé-ˀdrǫˀ\\
{\translocative}-\textsc{3s.fi.a}-dwell.{\stative}\\
\glt `she is at home, way over there'
\z
\z
\z 


\subsubsection*{\stem{d-verb} or \stem{heˀ-verb} (fixed expressions)} \label{[d-verb] or [heˀ-verb] (fixed expressions)}
The verbs listed in \xref{ex:dhex13} require the \stem{d-} {\cislocative} to express their fixed meaning.

\ea\label{ex:dhex13}

\ea tgayéi, tgayí:\\
\gll t-ga-yéi/yí\\
{\cislocative}-\textsc{3s.a}-right.{\stative}\\
\glt `it is right, correct'

\ex dwagado̱ˀkdá:nih\\
\gll d-wag-ad-o̱ˀkd-á:ni-h\\
{\cislocative}-\textsc{1s.p}-{\semireflexive}-finish-{\benefactive}-{\habitual}\\
\glt `I am dissatisfied'

\ex ę: tsǫ: itseˀs  \\
\gll ę: tsǫ: i-t-s-e-ˀs\\
again just {\prothetic}-{\cislocative}-\textsc{2s.a}-go-{\habitual}\\
\glt ‘you wander (all the time)’, ‘you are over there’

\ex tgá:gwitgęˀs \\
\gll t-gá-ragw-itgę-ˀs\\
{\cislocative}-\textsc{3s.a}-sun-rise-{\habitual}\\
\glt ‘east’, `the sun rises there'

\ex ętsáˀsęht\\
\gll ę-t-s-áˀsęht\\
\fut-{\cislocative}-\textsc{2s.a}-bring.down.{\zeropunctual}\\
\glt `you will bring it down'
\ex tgaęgwá:dih \\
\gll t-gaę-gw-á:di-h \\
{\cislocative}-\textsc{3ns.fi.a}-direction-throw-{\habitual}\\
\glt ‘they  are going in a direction’

\ex gyodóˀkda̱ˀǫh\\
\gll g-yo-d-óˀkda̱ˀ-ǫh\\
{\cislocative}-\exsc{3s.p-{\semireflexive}-}cause.to.run.out-{\stative}\\
\glt `it is lacking'

\ex gyagoya̱ˀdá:gwęh\\
\gll g-yago-ya̱ˀd-á:-gw-ęh\\
{\cislocative}-\textsc{3s.fi.p}-body-{\joinerA}-gather-{\stative}\\
\glt `she has lost weight'

\ex gyogyę́hdǫh\\
\gll g-yo-g-yę́hd-ǫh\\
{\cislocative}-\textsc{3s.p}-{\semireflexive}-hit-{\stative}\\
\glt `the first one, the beginning'

\ex dawá:tgri:k\\
\gll d-a-wá:-t-gri-:k\\
{\cislocative}-{\factual}-\textsc{3s.a}-{\semireflexive}-fold.{\zeropunctual}-{\modalizer}\\
\glt `it pulled back, flinched, shrank'

\ex dawáˀsęˀ\\
\gll d-a-w-áˀsę-ˀ\\
{\cislocative}-{\factual}-\textsc{3s.a}-reduce-{\punctual}\\
\glt `it dropped, reduced'

\ex tgehdáhkwaˀ\\
\gll t-g-ehdáhkw-haˀ\\
{\cislocative}-\textsc{1s.a}-believe-{\habitual}\\
\glt `I believe'

\ex ętsadadáhshagwęˀ\\
\gll ę-t-s-adad-áhshagw-ę-ˀ\\
\fut-{\cislocative}-\textsc{2s.a}-{\reflexive}-remember-{\benefactive}-{\punctual}\\
\glt `you will remind yourself, make yourself remember'

\ex dawáda̱ˀsęht\\
\gll d-a-w-ád-a̱ˀsęht\\
{\cislocative}-{\factual}-\textsc{3s.a}-{\semireflexive}-cause.to.\-drop.{\zeropunctual}\\
\glt `it swooped down'

\ex dǫgádo̱ˀktahs \\
\gll d-ǫg-ád-o̱ˀkd-a-hs\\
{\cislocative}-\exsc{\factual.1s.a}-{\semireflexive}-lack-{\joinerA}-{\habitual}\\
\glt ‘I was not satisfied (with…)’, `I did not get enough'

\ex dawatehtgęht\\
\gll d-a-w-at-hehtgęht\\
{\cislocative}-{\factual}-\exsc{3s.a-{\semireflexive}-}cause.to.\-go.bad.{\zeropunctual}\\
\glt `it went bad'

\ex ętsa:tró:wiˀ\\
\gll ę-t-s-a:t-hró:wi-ˀ\\
\fut-{\cislocative}-\textsc{2s.a}-{\semireflexive}-tell-{\punctual}\\
\glt `you will recount, retell'

\ex ętsatsadǫ́:goˀ\\
\gll ę-t-s-at-hsadǫ́:-go-ˀ\\
\fut-{\cislocative}-\textsc{2s.a}-bury-{\reversive}-{\punctual}\\
\glt `you will unearth it'

\ex ętsa:dó:gęhs\\
\gll ę-t-s-a:dó:gęhs\\
\fut-{\cislocative}-\textsc{2s.a}-make.right.{\zeropunctual}\\
\glt `you will adjust'

\ex da:gonhę́hęhih\\
\gll d-a-ago-nhę́hęhih\\
{\cislocative}-{\factual}-\textsc{3s.fi.p}-dribble.{\zeropunctual}\\
\glt `she dribbled, peed'

\ex dagayagę́hdahk\\
\gll da-ga-yagę́hd-a-hk\\
{\cislocative}-\textsc{3s.a}-issue.from-{\joinerA}-{\zeropunctual.\former}\\
\glt `it came out suddenly'

\ex daję́:na:\\
\gll da-j-ę́:na:\\
{\cislocative}-\textsc{2s.a}-grab.{\noaspect}\\
\glt `grab it!'
\z
\z

The verbs in \xref{ex:dhex16} require the \stem{heˀ-} \textsc{\translocative} to express their fixed meaning.

\ea\label{ex:dhex16}

\ea hę́:goh\\
\gll h-ę́:-g-oh\\
{\translocative}-{\future}-\textsc{1s.a}-dip.in.\-liquid.{\zeropunctual}\\
\glt `I will dip it in, submerge it'

\ex hegáhgwęˀs\\
\gll he-gá-hgwę-ˀs\\
{\translocative}-\textsc{3s.a}-set-{\habitual}\\
\glt `to the setting sun, the direction of the sunset, west'

\ex hęhsye:ˀ\\
\gll h-ę-hs-ye:-ˀ\\
{\translocative}-{\future}-\textsc{2s.a}-touch-{\punctual}\\
\glt `you will touch'

\ex hęhsyáhnǫ:ˀ\\
\gll h-ę-hs-yá-hnǫ:-ˀ\\
{\translocative}-{\future}-\textsc{2s.a}-touch-{\distributive}-{\punctual}\\
\glt `you will grope, touch, pick at'

\newpage
\ex hęgǫ́:nęht\\
\gll h-ę-g-ǫ́:nęht-ø\\
{\translocative}-{\future}-\textsc{1s.a}-swallow-{\punctual}\\
\glt `I will swallow'

\ex heyótsˀahdǫh\\
\gll he-yó-tsˀahd-ǫh\\
{\translocative}-\textsc{3s.p}-dwindle-{\stative}\\
\glt `it is all gone'

\ex hęhsó:goht\\
\gll h-ę-hs-ó:goht\\
{\translocative}-{\future}-\textsc{2s.a}-surpass.{\zeropunctual}\\
\glt `you will exaggerate, go above and beyond'


\ex hęwáˀsęˀ\\
\gll h-ę-w-áˀsę-ˀ\\
{\translocative}-{\future}-\textsc{3s.a}-fall.in-{\punctual}\\
\glt `it will fall in'

\ex hęgǫyadę́:nyeht\\
\gll h-ę-gǫy-adę́:-nyeht-ø\\
{\translocative}-{\future}-\exsc{1s>2s}-send-{\punctual}\\
\glt `I will refer you (to someone else)'

\ex to: hé:yoht\\
\gll to:  hé:-yo-ht\\
that {\translocative}-\textsc{3s.p}-resemble.{\stative}\\
\glt `suddenly'

\ex hęhsadagwáihsaht\\
\gll h-ę-hs-ad-agwáihsaht\\
{\translocative}-{\future}-\textsc{2s.a}-{\semireflexive}-straighten.{\zeropunctual}\\
\glt `you will go straight'

\ex heyó:gęˀt\\
\gll he-yó:-gęˀt\\
{\translocative}-\textsc{3s.p}-transparent.{\stative}\\
\glt `it is transparent'

\ex haˀhoˀnigǫ́haˀehs\\
\gll h-aˀ-ho-ˀnigǫ́h-a-ˀehs\\
{\translocative}-{\factual}-\textsc{3s.m.p}-mind-{\joinerA}-cause.to.\-hit.{\zeropunctual}\\
\glt `his mind settled on'

\ex hęsadwęnǫ́:diˀ\\
\gll h-ę-s-ad-węn-ǫ́:di-ˀ\\
{\translocative}-{\future}-\textsc{2s.p}-{\semireflexive}-voice-throw-{\punctual}\\
\glt `you will throw your voice' (as a ventriloquist)

\ex hęsagya̱ˀdǫ́:diˀ\\
\gll h-ę-s-ag-ya̱ˀd-ǫ́:di-ˀ\\
{\translocative}-{\future}-\textsc{2s.p}-{\semireflexive}-body-throw-{\punctual}\\
\glt `you will pounce on it'

\ex hęhaihwę́hę:ˀ\\
\gll h-ę-ha-ihw-ę́hę:-ˀ\\
{\translocative}-{\future}-\textsc{3s.m.a}-matter-convey-{\punctual}\\
\glt `he will take the message'

\ex hękre:k\\
\gll h-ę-k-hre:k\\
{\translocative}-{\future}-\textsc{1s.a}-push.{\zeropunctual}\\
\glt `I will push it'

\ex hęgóˀkdęˀ\\
\gll h-ę-g-óˀkd-ęˀ\\
{\translocative}-{\future}-\textsc{1s.a}-finish-{\punctual}\\
\glt `I will finish something'
\z
\z

The verbs in \xref{ex:dhex14} require either the \stem{d-} {\cislocative} or the \stem{heˀ-} \textsc{\translocative} to express their fixed meaning.

\ea\label{ex:dhex14}

\ea tagihnǫ:s\\
\gll t-hag-ihnǫ:k-s\\
{\cislocative}-\exsc{3s.m>1s}-call-{\habitual}\\
\glt `he is calling me'
\cfex{hęgihnǫ:k \\
  \gll h-ę-g-ihnǫ:k\\
  {\translocative}-\textsc{1s.a}-call.{\zeropunctual}\\
  \glt `I will call'
}

\ex tgya̱hsǫ́haˀ\\
\gll t-g-ya̱hsǫ́-haˀ\\
{\cislocative}-\textsc{1s.a}-name-{\habitual}\\
\glt  ‘I call them’, `I am a bingo caller'
\newpage
\cfex{hęhshéyahsǫ:ˀ \\
  \gll h-ę-hshé-yahsǫ:-ˀ\\
  {\translocative}-{\future}-\exsc{2s>3s.fi}-name-{\punctual}\\
  \glt `you will call someone’s name (over there)'
}

\ex Waˀjih, ętsyáˀdęˀ\\
\gll waˀjih ę-t-s-yáˀd-ę-ˀ\\
wait \fut-{\cislocative}-\textsc{2s.a}-body-fall-{\punctual}\\
\glt `Wait, you’ll fall into it'
\cfex{hewagya̱ˀdę́ˀǫh \\
  \gll he-wag-ya̱ˀd-ę́-ˀǫh\\
  {\translocative}-\textsc{1s.p}-body-\exsc{fall}-{\stative}\\
  \glt `I have fallen into it'
}
\z
\z

\subsubsection*{\stem{d-…verb-{\stative}} ‘-est’} \label{[d+pronominal prefix-stative.verb]}
The fixed combination of \stem{d-…verb-{\stative}} turns a stative-only verb like \stem{gowan-ęh} ‘big-{\stative}’ into a superlative, meaning ‘the most’ or ‘-est’ (\ref{ex:pppexpressionex}).

\ea\label{ex:pppexpressionex} Í:ˀ tgegówanęh. \\
\gll í:ˀ t-g-e-gówan-ęh\\
I {\cislocative}-\textsc{1s.a}-{\joinerE}-big-{\stative}\\
\glt ‘I am the oldest (biggest)’ 
\cfex{gegówanęh \\
  \gll g-e-gówan-ęh\\
  \textsc{1s.a}-{\joinerE}-big-{\stative}\\
  \glt `I am big'
}
\z

Other comparisons requiring the \stem{d-} {\cislocative} are described in \textit{Comparisons}, \sectref{ch:Comparisons (more, the same, or less)}.


\subsubsection*{\stem{shęh \exsc{d-}…\textsc{incorporated noun}-o:t} ‘at the \textsc{incorporated noun}’} \label{[shęh d-pronominal prefix-incorporated noun-o:t]}
The fixed combination of \stem{shęh d-…\textsc{incorporated noun}-o:t} means ‘at the \textsc{incorporated noun}’ \xref{ex:pppexpressionex13}. 

\ea\label{ex:pppexpressionex13} \stem{shęh d-…\textsc{incorporated noun}-o:t} ‘at the \textsc{incorporated noun}’\\
shęh tganǫ́hso:t\\
\gll shęh t-ga-nǫ́hs-o:t\\
that {\cislocative}-\textsc{3s.a}-house-stand.{\stative}\\
\glt `at the house'
\cfex{ shęh tganǫhsó:t ita:t ‘he is standing at the house’
}
\z


\subsubsection*{\stem{haˀde-…\textsc{incorporated noun}-age:} ‘every, many, a variety of \textsc{incorporated nouns}’} \label{[haˀde-pronominal prefix-incorporated noun-age:]}
The fixed combination of \stem{haˀde-…\textsc{incorporated noun}-age:} means ‘every, many, a variety of \textsc{incorporated nouns}’ \xref{ex:pppexpressionex5}.

\ea\label{ex:pppexpressionex5} \stem{haˀde-…\textsc{incorporated noun}-age:} ‘every, many, a variety of \textsc{incorporated nouns}’

\ea haˀdewę̱hníhsrage: \\
\gll haˀde-w-ę̱hníhsr-age:\\
{\translocativedualic}\exsc{-3s.a}-day-two.or.more.{\stative}\\
\glt ‘every day’, `many days'

\ex haˀdewa̱hsǫ́dage:\\
\gll haˀde-w-a̱hsǫ́d-age:\\
{\translocativedualic}\exsc{-3s.a}-night-two.or.more.{\stative}\\
\glt `every night'

\ex haˀdega̱ˀdréhdage:\\
\gll haˀde-ga̱-ˀdréhd-age:\\
{\translocativedualic}\exsc{-3s.a}-car-two.or.more.{\stative}\\
\glt `all kinds of cars'
\z
\z

\subsubsection*{Particles with \stem{d-} {\cislocative} or \stem{heˀ-} \textsc{\translocative}} \label{Particles with [d-] (cislocative) or [heˀ-] (translocative)}
The following particles, described in the \textit{Particle dictionary}, \sectref{ch:particle dictionary}, require the following verb to begin with either \stem{d-} {\cislocative} or \stem{heˀ-} {\translocative}.

\begin{CayugaRelated}
\item{}\textit{Gaę hǫ:weh} ‘which place’, 'where’\\
\item{}\textit{Heyohe:ˀ} ‘more’\\
\item{}\textit{Neˀ aǫhę:ˀęh d-} ‘the most, -est, -er (of)’\\
\item{}\textit{Neˀ gyaǫhę:ˀęh d-} ‘the most’, 'the greatest’\\
\item{}\textit{Ne:ˀ gwahs d-} ‘the most’\\
\item{}\textit{Ne:ˀ gwahs heyohe:ˀ} ‘the most, -est’
\end{CayugaRelated}



\subsection{\stem{de-} \textsc{dualic} ({\dualic})} \label{[de-] (dualic)}
The \stem{de-} {\dualic} adds the meaning that there are two parts to an action, or two states, as described in the following sections.

\subsubsection*{\stem{de-} {\dualic} and two-part motions} \label{[de-] (dualic) and two-part motions}
The \stem{de-} {\dualic} is often required with verbs that describe a two-part motion, a change in position, or a back-and-forth motion \xref{ex:detwoex}.


\ea\label{ex:detwoex}

\ea dekdaˀs\\
\gll de-k-d-a-ˀs\\
{\dualic}-\textsc{1s.a}-stand-\exsc{join}-{\habitual}\\
\glt `I am standing'

\ex degá:dęh\\
\gll de-gá:-dę-h\\
{\dualic}-\textsc{3s.a}-fly-{\habitual}\\
\glt `airplane'

\ex dęhadagyáˀda̱hgwaˀt\\
\gll d-ę-ha-dag-yáˀd-a̱-hgwaˀt\\
{\dualic}-{\future}-\textsc{3s.m.a}-{\reflexive}-body-{\joinerA}-cause.to.\-raise.{\zeropunctual}\\
\glt `he will do push ups'

\ex degaǫnáˀsgwa̱hkwaˀ\\
\gll de-gaǫ-náˀsgw-a̱-hgw-haˀ\\
{\dual}-\exsc{3ns.fi.a}-lift.up-{\habitual}\\
\glt `they are jumping'

\ex de̱hęnáǫhaˀ\\
\gll de̱-hęn-áǫ-haˀ\\
{\dualic}-\textsc{3ns.m.a}-run-{\habitual}\\
\glt `they are racers'

\ex de̱hęnadátgǫ̱he:s\\
\gll de̱-hęn-adát-gǫ̱he:g-s\\
{\dualic}-\textsc{3ns.m.a}-{\reflexive}-punch-{\habitual}\\
\glt `boxers'

\ex dewá:tgwęh\\
\gll de-w-á:t-gw-ęh\\
{\dualic}-\textsc{3s.a}-{\semireflexive}-dance-{\stative}\\
\glt `it is dancing'

\ex da:há:dahs\\
\gll d-a:-há:-dahs\\
{\dualic}-{\indefinite}-\textsc{3s.m.a}-stop.{\noaspect}\\
\glt `he should stop'

\ex dehsdaˀ\\
\gll de-hs-daˀ\\
{\dualic}-\textsc{2s.a}-stop.{\noaspect}\\
\glt `stop!'

\ex dęgakeyáˀdahk\\
\gll d-ę-gake-yáˀd-a-hk\\
{\dualic}-{\future}-\exsc{1s>3ns}-body-{\joinerA}-pick.up.{\zeropunctual}\\
\glt `I will pick them up'

\ex dęwátahahk\\
\gll d-ę-w-át-hah-a-hk\\
{\dualic}-{\future}-\textsc{3s.a}-{\semireflexive}-road-{\joinerA}-pick.up.{\zeropunctual}\\
\glt `it will walk'

\ex deyagodáwęnyeˀ\\
\gll de-yago-d-áwęnye-ˀ\\
{\dualic}-\textsc{3s.fi.p}-{\semireflexive}-stir-{\stative}\\
\glt `she is walking about'

\ex dęhahgwadáhnǫ:ˀ\\
\gll d-ę-ha-hgwad-á-hnǫ-:ˀ\\
{\dualic}-{\future}-\textsc{3s.m.a}-raise.up-{\joinerA}-{\distributive}-{\punctual}\\
\glt `he will raise or lift things up'

\ex detodinę́hędreˀ\\
\gll de-t-hodi-nę́hę-dr-e-ˀ\\
{\dualic}-{\cislocative}-\textsc{3ns.m.p}-stand.in.a.line-{\dislocative}-go-{\stative}\\
\glt `they are standing (or guarding)'

\ex degóno̱hsgwihs\\
\gll de-g-ó-no̱hsgwi-hs\\
{\dualic}-\textsc{1s.a}-{\semireflexive}-sweep-{\habitual}\\
\glt `I am sweeping'

\ex deyoˀnhęhtsę́dǫ̱hǫh\\
\gll de-yo-ˀnhęhts-ę́dǫ̱h-ǫh\\
{\dual}-\textsc{3s.p}-tail-shake-{\stative}\\
\glt `it is wagging its tail'

\ex dewahji̱ˀdóha:s\\
\gll de-w-ahji̱ˀd-óha:g-s\\
{\dual}-\textsc{3s.a}-hand-squeeze-{\habitual}\\
\glt `pliers'

\ex degaǫdóhda:s\\
\gll de-gaǫ-d-óhda:g-s\\
{\dual}-\textsc{3ns.fi.a}-{\semireflexive}-clean-{\habitual}\\
\glt `janitors'
\z
\z

\subsubsection*{\stem{de-} {\dualic} for a change in state, coming apart} \label{[de-] (dualic) for a change in state, coming apart}
The \stem{de-} {\dualic} is often required with verbs that denote a change in state, specifically, the action of coming apart \xref{ex:detwoex2}.
 
\ea\label{ex:detwoex2}

\ea de̱há:yaˀs\\
\gll de̱-há:-yaˀg-s\\
{\dual}-\textsc{3s.m.a}-break-{\habitual}\\
\glt `he breaks it'

\ex degaihó:węhs\\
\gll de-ga-ih-ó:wę-hs\\
{\dualic}-\textsc{3s.a}-words-split-{\habitual}\\
\glt `computer'

\ex degáhihtaˀ\\
\gll de-gá-hriht-haˀ\\
{\dualic}-\textsc{3s.a}-chop.up-{\habitual}\\
\glt `grinder or cutter'

\ex degawidrágędrǫhs\\
\gll de-ga-widr-á-gędrǫ-hs\\
{\dualic}-\textsc{3s.a}-ice-{\joinerA}-skim.off-{\habitual}\\
\glt `cream separator'
\z
\z

\largerpage
\subsubsection*{\stem{de-} {\dualic} for something bent, crossed} \label{[de-] (dualic) for something bent, crossed}
The \stem{de-} {\dualic} is often required with verbs that describe something bent or crossed \xref{ex:detwoex3}.

\ea\label{ex:detwoex3}

\ea deyohahíya̱ˀgǫh\\
\gll de-yo-hah-íya̱ˀg-ǫh\\
{\dualic}-\textsc{3s.p}-road-cross-{\stative}\\
\glt `crossroad'

\ex deyótsaˀkdǫh\\
\gll de-yó-tsaˀkd-ǫh\\
{\dualic}-\textsc{3s.p}-bend-{\stative}\\
\glt `it is bent'

\ex degáhǫˀ\\
\gll de-gá-hǫˀ\\
{\dualic}-\textsc{3s.a}-lie.across.{\stative}\\
\glt `it is lying across'
\z
\z

\subsubsection*{\stem{\exsc{de}-verb} (fixed expressions)} \label{[de-verb] (fixed expressions)}
Over 300 verbs require the \stem{de-} {\dualic} prefix to express their fixed meaning. Only a few examples are listed in \xref{ex:detwoex4}. (For a longer list, see \sectref{verb dictionary}.)

\ea\label{ex:detwoex4}

\ea dęjidwadatnǫ̱hǫ́:nyǫ:ˀ\\
\gll d-ę-ji-dwa-dat-nǫ̱hǫ́:nyǫ-:ˀ\\
{\dualic}-{\future}-\exsc{1p.in.a}-{\reflexive}-greet-{\punctual}\\
\glt `we will greet each other again'

\ex degaˀęnáęhęˀ\\
\gll de-g-aˀęn-á-ę-hęˀ\\
{\dualic}-\textsc{1s.a}-pole-{\joinerA}-hit-{\habitual}\\
\glt `I am a snowsnake player'

\ex degáhshętwahs\\
\gll de-g-áhshętw-a-hs\\
{\dualic}-\textsc{1s.a}-cry-{\joinerA}-{\habitual}\\
\glt `I am crying'

\ex deyóhsdateh\\
\gll de-yó-hsd-ateh\\
{\dualic}-\textsc{3s.p}-shiny.object-clear.{\stative}\\
\glt `it is smooth'

\ex dewátęho:t\\
\gll de-wá-t-hęh-o:t\\
{\dualic}-\textsc{3s.a}-{\semireflexive}-wood-stand.{\stative}\\
\glt `pile of wood'

\ex dehsgwę́ˀnya:ˀ\\
\gll de-hsgw-ę́-ˀny-a:ˀ\\
{\dualic}-\exsc{2>1(p)}-{\semireflexive}-hand-hold.{\stative}\\
\glt `you who watch over us'

\ex desaˀdráihęh\\
\gll de-sa-ˀdráihęh\\
{\dualic}-\textsc{2s.p}-hurry.{\noaspect}\\
\glt `hurry!'
\z
\z

The \stem{de-} {\dualic} prefix often appears in verbs composed of loanwords (\sectref{ch:Loanwords (borrowed words)}), perhaps because it is such a common prefix \xref{ex:detwoex5}.


\ea\label{ex:detwoex5}

\ea dehsadesmack\\
\gll de-hs-ad-e-smack\\
{\dualic}-\exsc{2s.a-{\semireflexive}-JoinerE}-smack.{\zeropunctual}\\
\glt `smack your lips!'

\ex ato:stro:k\\
\gll a-t-ho:-stro:k\\
{\factual}-{\dualic}-\textsc{3s.m.p}-stroke.{\zeropunctual}\\
\glt `he had a stroke'

\ex deyagodehighlightǫh\\
\gll de-yago-d-e-highlight-ǫh\\
{\dualic}-\exsc{3s.fi.p-{\semireflexive}-JoinerE}-highlight-{\stative}\\
\glt `where she has it highlighted'
\ex
\gll Gwe:  gwe:  a:yęˀ  desatmonkeydǫh e:?\\
well well it.seems you’re.acting.like.a.monkey again\\
\glt ‘Are you acting like a monkey again?’ (Amos Key, p.c.)
\cfex{
  \gll de-sa-t-monkey-d-ǫh\\
  {\dualic}-\textsc{2s.p}-{\semireflexive}-monkey-\textsc{euph.d}\exsc{}-{\stative}\\
}
\z
\z


\subsubsection*{\stem{\textsc{de-a}-yahshe-sǫˀ} ‘two each, at a time’} \label{[de-A-incorporated noun-yahshe-sǫˀ]}
The combination of \stem{de-\textsc{a}-yahshe-sǫˀ}, with an \textsc{a}-series pronominal prefix, means ‘two each’, or ‘two at a time’ \xref{ex:eachperex4}. (Also see \stem{de-…\textsc{incorporated noun}-age-hagyeˀ} ‘two \textsc{incorporated nouns} at a time’, \sectref{[de+pronominal prefix+incorporated noun+age+hagyeˀ]}.)

\ea\label{ex:eachperex4} \stem{de-\textsc{a}-yahshe-sǫˀ} ‘two each’, ‘two at a time’

\ea degadiyáhshesǫˀ\\
\gll de-gadi-yáhshe-shǫˀ\\
\textsc{\dualic-3p.a}-two.living.things.\textsc{\stative-\pluralizer}\\
\glt `two things each, at a time'
\cfex{degadiyáhshe: \\
  \gll de-gadi-yáhshe:\\
  {\dualic}-\textsc{3p.a}-two.living.things.{\stative}\\
  \glt `two things'
}
\ex
\gll Degadiyahshesǫ́ˀ shę́h daga:dí:yoˀ.\\
two.at.a.time that they(things).came.in\\
\glt ‘they (things, i.e. cars) came in two at a time’ 
\z
\z 


\subsubsection*{\stem{de-…\textsc{incorporated noun}-age:} ‘two \textsc{incorporated nouns}’} \label{[de+pronominal prefix+incorporated noun+age:]}
The fixed combination of \stem{de-…\textsc{incorporated noun}-age:} is used for counting two objects (\ref{ex:pppexpressionex2}, see \sectref{ch:Counting two objects}).  

\ea\label{ex:pppexpressionex2} \stem{de-…\textsc{incorporated noun}-age:} ‘two \textsc{incorporated nouns}’

\ea degajihsdá:ge:\\
\gll de-ga-jihsd-á:ge:\\
{\dualic}-\textsc{3s.a}-lamp-be.two.or.more.{\stative}\\
\glt `two lamps'

\ex deyohsi̱ˀdá:ge:\\
\gll de-yo-hsi̱ˀd-áge:\\
{\dualic}-\textsc{3s.p}-foot-be.two.or.more.{\stative}\\
\glt `two feet'

\ex deyoˀnhǫ̱hsá:ge:\\
\gll de-yo-ˀnhǫ̱hs-á:ge:\\
{\dualic}-\textsc{3s.p}-egg-be.two.or.more.{\stative}\\
\glt `two eggs'

\ex dewahda̱hgwá:ge:\\
\gll de-w-ahda̱hgw-á:ge:\\
{\dualic}-\textsc{3s.a}-shoe-be.two.or.more.{\stative}\\
\glt `two shoes'
\z
\z


\subsubsection*{\stem{de-…\textsc{incorporated noun}-age-hagyeˀ} ‘two \textsc{incorporated nouns} at a time’} \label{[de+pronominal prefix+incorporated noun+age+hagyeˀ]}
The fixed combination of \stem{de-…\textsc{incorporated noun}-age-hagyeˀ} means ‘two at a time’ (\ref{ex:pppexpressionex3}, see \sectref{[de+pronominal prefix+incorporated noun+age+hagyeˀ]}).
\clearpage
\ea\label{ex:pppexpressionex3} \stem{de-…\textsc{incorporated noun}-age-hagyeˀ} ‘two \textsc{incorporated nouns} at a time’\\
degaˀdre̱hdage̱há:gyeˀ\\
\gll de-ga-ˀdre̱hd-age̱-h-á:-gy-e-ˀ\\
{\dualic}-\textsc{3s.a}-car-two.or.more.{\stative}-\textsc{euph.h}-{\joinerA}-\exsc{\progressive}-go-{\stative}\\
\glt ‘two cars at at time’ 
\z


\subsubsection*{\stem{de-…ogę:} ‘between, in the middle’} \label{[de+pronominal prefix+verb+ogę:]}
The fixed combination of \stem{de-…ogę:} means ‘between, in the middle’ \xref{ex:pppexpressionex4}.

\ea\label{ex:pppexpressionex4} \stem{de-…ogę:} ‘between, in the middle’

\ea deyó:gę: \\
\gll de-y-ó:gę:\\
{\dualic}-\textsc{3s.p}-between.{\stative}\\
\glt ‘it is between’, `in the middle'
\ex
\gll Detniyada:do:gę́: i:gá:t só:wa:s.\\
between.us it.stands dog\\
\glt ‘The dog is standing between us.’ 
\cfex{
  \gll de-tniy-ada:d-o:gę́:\\
  {\dualic}-\exsc{1d.in.a}-{\reflexive}-between. {\stative}\\
}
\z
\z 

\subsubsection*{Other fixed expressions with \stem{de-} {\dualic}} \label{Other fixed expressions with [de-] (dualic)}
Several fixed expressions require a combination of the \stem{de-} {\dualic} prefix in combination with an additional prepronominal prefix. These were previously described in the “Related” sections listed below.
 
\begin{CayugaRelated}
\item{}\stem{\exsc{tsaˀde}-verb}, \sectref{[tsaˀde-verb]}\\
\item{}\stem{\exsc{tsaˀde}-\textsc{incorporated noun}-verb} (fixed expressions), \sectref{[tsaˀde-incorporated noun-verb]}\\
\item{}\stem{tsaˀdet/tsaˀdeg-verb}, \sectref{[tsaˀdet/tsaˀdeg-verb]}\\
\item{}\stem{naˀde-\exsc{a}-adreˀ} ‘to be a certain distance apart’, \sectref{[naˀde-A-adreˀ]}\\
\item{}\stem{haˀde-…\textsc{incorporated noun}-age:} ‘every, many, a variety of \textsc{incorporated nouns}’, \sectref{[haˀde-pronominal prefix-incorporated noun-age:]}
\end{CayugaRelated}


\section{Pronunciation of prepronominal combinations} \label{Pronunciation of prepronominal combinations}
Prepronominal prefix pronunciation is described next. The forms with example Gayogo̱ho:nǫˀnéha:ˀ words are based on the fieldwork undertaken for the \citet{froman_english-cayugacayuga-english_2002} dictionary. The remaining forms are from \citet{foster_course_1993} and \citet{sasse_far_1998}. 


\subsection{Legend for prepronominal prefix pronunciation charts} \label{{Legend for prepronominal prefix pronunciation charts}}
Most prepronominals change in pronunciation either because of the following sound (such as a ‘C’ or consonant, \tabref{figtab:1:ppplegend}, page \pageref{figtab:1:ppplegend}), or because of the meaning of the following pronominal prefix -- including the pronominals listed as \textsc{1s.p} or \textsc{2} in \tabref{figtab:1:ppplegend} and the \textsc{a-person} and \textsc{e-person} pronominals in \tabref{figtab:1:aepersons}, page \pageref{figtab:1:aepersons}. The following tables summarize the relevant factors.

\begin{table}[b!]
\caption{Legend, prepronominal prefix pronunciation}
\label{figtab:1:ppplegend}
\begin{tabularx}{\textwidth}{lQ}
\lsptoprule
Abbreviation & Legend\\
\midrule
C & the next sound is a consonant\\
\tablevspace
V & the next sound is a vowel\\
\tablevspace
w & the next sound is a W\\
\tablevspace
y &  the next sound is a Y \\
\tablevspace
(w) & the next sound is a W, but the W deletes. (It is only found in other forms of the verb.)  \\
\tablevspace
(y) &  the next sound is a Y, but the Y deletes. (It is only found in other forms of the verb.)\\
\tablevspace
(sy) & the \stem{s-} {\repetitive} merges with the following sound, Y. The two sounds become J\\
\tablevspace
(ts) & the \stem{s-} {\repetitive} becomes T when the following sound is S: \phonet{s-s} becomes \phonet{t-s}\\
\tablevspace
(sa) & the \stem{s-} {\repetitive} becomes \stem{sa-} when the following sound is S: \phonet{s-s} becomes \phonet{sa-s}\\
\tablevspace
\textsc{1s.p} & the \stem{a-} {\factual} or \stem{a:-} {\indefinite} merge with \stem{-wag} \textsc{1s.p}. The result is \stem{ǫg-} \exsc{\factual.1s.p} or \stem{aǫg-} \exsc{\indefinite.1s.p}\\
\tablevspace
2 & the prefix used before \textsc{2s.a} or \textsc{inclusive} pronominal prefixes\\
\tablevspace
\exsc{imp} & the prefix used with imperatives (command)\\
\tablevspace
\exsc{a-person} & the prefix used before \exsc{2s.a} (\exsc{a-person}) pronominal prefixes\\
\tablevspace
\exsc{e-person} & the prefix used before \exsc{2s.a}, (\exsc{e-person}) pronominal prefixes\\
\tablevspace
let’s & the prefix used with \textit{let’s} commands (as in ‘let’s do it')\\
\lspbottomrule
\end{tabularx}
\end{table}



The terms \textsc{a-persons} and \textsc{e-persons} in \tabref{figtab:1:aepersons}, page \pageref{figtab:1:aepersons} both refer to prefixes meaning ‘we’ or ‘you’. \textsc{a-persons} take the \stem{aˀ-} or \stem{a-} {\factual}, and \textsc{e-persons} take the \stem{e-} {\factual}. (Most of the verbs in \tabref{figtab:1:aepersons}, page \pageref{figtab:1:aepersons} mean ‘to see’, with the exception of \textit{esáhdǫ:ˀ} ‘you lost it’ and \textit{esáˀdra̱hehs} ‘you are exaggerating’.)

\begin{table}
\caption{A- and E-persons}
\label{figtab:1:aepersons}
\begin{tabular}{lll}

\lsptoprule
\exsc{a-person} & aˀ\exemph{a:kní:}gęˀ, \exemph{a:kní:}gęˀ & \exsc{1d.ex.a}\\ [6pt]

& aˀ\exemph{ágwa}gęˀ, \exemph{ágwa}gęˀ & \exsc{1p.ex.a}\\ [6pt]
& a\exemph{kní:}gęˀ & \exsc{1>2(d)}\\ [6pt]
& a\exemph{gwá:}gęˀ & \exsc{1>2(p)}\\ [6pt]

& a\exemph{sgé:}gę & \exsc{2s>1s} \\
\midrule
\exsc{e-person} & e\exemph{tní:}gęˀ, e\exemph{kní:}gęˀ & \exsc{1d.in.a}\\[6pt]
& e\exemph{dwá:}gęˀ & \exsc{1p.in.a}\\ [6pt]

& a\exemph{hsé:}gęˀ, \exemph{es}áhdǫ:ˀ & \textsc{2s.a}\\ [6pt]
& \exemph{ehsní:}gęˀ & \exsc{2d.a}\\ [6pt]
& \exemph{esa}ˀdra̱hehs & \textsc{2s.p}\\ [6pt]
& \exemph{ehswá:}gęˀ & \textsc{2p.a}\\ [6pt]
& e\exemph{hyá:}gęˀ, a\exemph{hyá:}gęˀ & \exsc{3s.m>2s}\\ [6pt]
& \exemph{eskní:}gęˀ & \exsc{2>1(d)}\\ [6pt]
& \exemph{esgwá:}gęˀ & \exsc{2>1(p)}\\ [6pt]
\lspbottomrule
\end{tabular}
\end{table}


\subsection{Pronunciation changes at the end of the prepronominal prefixes} \label{Pronunciation changes at the end of the prepronominal prefixes}
Only a few pronunciation changes affect the end of prepronominals or prepronominal combinations. Such changes are described below.

The single prepronominal prefixes in \tabref{figtab:1:invariant}, page \pageref{figtab:1:invariant} have just one pronunciation.

\begin{table}
\caption{Invariant prefixes}
\label{figtab:1:invariant}
\begin{tabular}{ll}
\lsptoprule
prefix & prefix name \\
\midrule
\stem{tsi-} & \textsc{\coincident}\\
\stem{ti-} & {\contrastive}\\
\stem{de-} & {\dualic}\\
\stem{ę-} & {\future}\\
\stem{ni-} & {\partitive}\\
\stem{i-} & {\prothetic}\\
\stem{he-} & {\translocative}\\
\lspbottomrule
\end{tabular}
\end{table}



In contrast, the {\cislocative}, {\factual}, or {\indefinite} prefixes have the pronunciations listed in the “last prefix” column in \tabref{figtab:1:finalsingleprefix}, page \pageref{figtab:1:finalsingleprefix}. Moreover, any prepronominal combinations ending with these prefixes have the same pronunciation. For example, both the {\cislocative} and any combination ending with the {\cislocative} are pronounced as [t] before a following consonant (‘C’).

\begin{table}
\caption{Final prefix of combination}
\label{figtab:1:finalsingleprefix}
{
\begin{tabularx}{.75\textwidth}{QXX}
\lsptoprule
last prefix & becomes & before\\
\midrule
{\cislocative} & [t] 

[d] 

[g] or [d]

[di] or [de] & C

w, V

y

2

\\
\midrule
{\factual} & [a]

[e]

[ǫ]

[aˀ] & C

\exsc{e-person }

\textsc{1s.p} 

(w), (y), V

\\
\midrule
{\indefinite} & [a:]

[aǫ]

[ae] & C

\textsc{1s.p} 

2\\
\midrule
{\negative} & [de]

[deˀ] & C

(w), (y), V

\\
\midrule
{\repetitive} & [s]

[t]

[j]

[ji] & C

(ts)

(sy)

2\\
\lspbottomrule
\end{tabularx}}
\end{table}


The \textsc{\factual-\cislocative}, \textsc{\factual-\dualic}, or \textsc{\factual-\repetitive} combinations have the pronunciations listed in \tabref{figtab:1:finaldoubleprefix}. Any combinations ending with these prefixes have the same pronunciation. For example, combinations ending with the \textsc{\factual-\repetitive} include \stem{ǫsa-} \textsc{\factual-\repetitive}, \stem{dǫsa-} \textsc{\dualic-\factual-\repetitive}, and \stem{hǫsa-} \textsc{\translocative-\factual-\repetitive}.

\begin{table}
\caption{Final two prefixes of combination}
\label{figtab:1:finaldoubleprefix}
{
\begin{tabularx}{\textwidth}{QXX}
\lsptoprule
prefix combination & ends with & before\\
\midrule
\textsc{\factual-\cislocative} & [ǫda]

[ǫdǫ]

[ǫdi] & C

\textsc{1s.p}

2 

\\
\midrule
\textsc{\factual-\dualic} & [adi]

[at]

[edi]

[ad]

[ag] or [ad] & \exsc{a-person}

C

\exsc{e-person} 

w, V

y

\\
\midrule
\textsc{\factual-\repetitive} & [ǫsa]

[ǫsǫ]

[ǫse] & C

\textsc{1s.p} 

2

\\
\midrule
\textsc{\indefinite-\cislocative} & [aǫda], [ǫ:da]

[aǫdǫ], [ǫ:dǫ]

[aǫde], [ǫ:de] & C

\textsc{1s.p} 

2

\\
\midrule
\textsc{\indefinite-\repetitive} & [aǫsa], [ǫ:sa]

[aǫsǫ], [ǫ:sǫ]

[aǫse], [ǫ:se] & C

\textsc{1s.p} 

2

\\
\midrule
\textsc{\repetitive-\factual} & [ǫsa]

[ǫsǫ]

[ǫsaˀ]

[ǫse] & C

\textsc{1s.p} 

(w), (y), V 

2\\
\lspbottomrule
\end{tabularx}}
\end{table}


Examples of all prepronominal prefix combinations are provided in alphabetical order in the following sections.

\subsection{Prepronominals in alphabetical order} \label{Prepronominals in alphabetical order}
The following tables provide example words illustrating prepronominal prefix combinations (where such were available). The tables are organized both alphabetically and according to the conditions that determine the final segments of the prepronominal prefix combination. These conditions are listed as abbreviations in the table headers, and the abbreviations, in turn, are listed in \tabref{figtab:1:ppplegend}, page \pageref{figtab:1:ppplegend}.  
\clearpage

\begin{table}
\caption{Words beginning with \stem{a-/aˀ-/ǫ-/e-} {\factual}}
\label{figtab:1:2finaldoubleprefix}
{
\begin{tabularx}{\textwidth}{QXXX}
\lsptoprule
C & (w), (y), V & \textsc{1s.p} & 2  \\
\midrule
\stem{a-}, \exemph{a}gaedá:graˀ 

‘they fell down’ & \stem{aˀ-}, \exemph{aˀ}é:yǫˀ 

‘she arrived’ & \stem{ǫ-}, \exemph{ǫ}gáhdrǫˀk 

‘it frightened me’, 

‘I got frightened’ & \stem{e-} 

\exsc{e-person} \exemph{e}sáhdǫ:ˀ 

‘you lost it’

\stem{aˀ-∅}* 

\exsc{a-person}

\exemph{aˀ}a:kní:gęˀ, a:kní:gęˀ 

‘s/he and I saw it’\footnote{The {\factual} optionally deletes before 2 (\exsc{a-persons}), but the word is stressed as if the prefix were still there.}\\
\lspbottomrule
\end{tabularx}}
\end{table}

\begin{table}
\caption{Words beginning with \stem{a:-/aǫ-/ae-/e:-} {\indefinite}}
\label{figtab:1:finalindefprefix}
{
\begin{tabularx}{\textwidth}{QXX}
\lsptoprule
C & \textsc{1s.p} & 2  \\
\midrule
{}\stem{a:-}, \exemph{a:}ga̱hyá:goˀ

‘I would pick fruit’ & \stem{aǫ-}, \exemph{aǫ}gé:gęˀ 

‘she would see me’ & {}\stem{ae-} 

\exsc{a-person} 

\exemph{ae}swá:gęˀ 

{}\stem{e:-} 

\exsc{e-person} 

\exemph{e:}swá:gęˀ 

‘you all should see it’\\
\lspbottomrule
\end{tabularx}}
\end{table}

\begin{table}
\caption{Words beginning with [aǫ/ǫ: … da-/dǫ-/de-/sa-/sǫ-/se-]}
\label{figtab:1:finalindefcislrepprefix}
{
\begin{tabularx}{\textwidth}{QQQl}
\lsptoprule
C & \textsc{1s.p} & 2 & \\
\midrule
{}\stem{aǫda-}, \stem{ǫ:da-}, 

{}\stem{aǫda:-}, \stem{ǫ:da:-}, 

\exemph{aǫda}sagyǫ̱ˀséhaˀ ‘you would come and visit’ 

& \stem{aǫdǫ-}, \stem{ǫ:dǫ-} & \stem{aǫde-}, \stem{ǫ:de-} & \textsc{\indefinite-\cislocative}\\
{}\stem{aǫsa-}, \stem{ǫ:sa-} & \stem{aǫsǫ-}, \stem{ǫ:sǫ-} & \stem{aǫse-}, \stem{ǫ:se-} & \textsc{\indefinite-\repetitive}\\
\lspbottomrule
\end{tabularx}}
\end{table}

\begin{table}
\caption{Words beginning with \stem{at-/ad-/ag-/adi-/edi-} \textsc{\factual-\dualic}}
\label{figtab:1:2finalindefcislrep}
{
\begin{tabularx}{\textwidth}{QQQQ}
\lsptoprule
C & w, V & y & 2 \\
\midrule
{} \stem{at-}, 

\exemph{at}gaeyé:naˀ ‘they did it together’ & \stem{ad-}, 

\exemph{ad}waga̱hsíha:s 

‘I choked’ & \stem{ag-},

\exemph{ag}yagwat\-nę:tsí:yaˀk 

‘we all broke our arm’ & \stem{adi-} 

\exsc{a-person} 

\exemph{adi}gyatnę:tsí:yaˀk 

‘we two broke our arm’ 

{} \stem{edi-} 

\exsc{e-person} 

\exemph{edi}dwatnę:tsí:yaˀk 

‘we all broke our arm’ \\
\lspbottomrule
\end{tabularx}}
\end{table}


\begin{table}
\caption{Words beginning with \stem{d-/t-/g-/de-/di-} {\cislocative}}
\label{figtab:1:scisl2}
{
\begin{tabularx}{\textwidth}{QQQQ}
\lsptoprule
C & w, V & y & 2 \\
\midrule 
{}\stem{t-}, 

\exemph{t}hadínagreˀ 

‘that’s where they live over there’ & \stem{d-}, 

né:ˀ gwáhs 

\exemph{d}wakyęda̱h\-gówanęh

‘the biggest chair’ & \stem{g-}, 

\exemph{g}ye̱ˀdrǫˀ

‘she’s at home’ & \stem{de-} 

\textsc{imp (e-person) }

\exemph{de}sáˀdre:, 

‘drive over here’ 

\stem{di-} 

\exemph{di}sáhdęgyǫ: 

‘you come from there’ \\
\lspbottomrule
\end{tabularx}}
\end{table}


\begin{table}
\caption{Words beginning with \stem{de-/dę-}}
\label{figtab:1:dualanddualfut}
{
\begin{tabularx}{0.66\textwidth}{Ql}
\lsptoprule
C & \\
\midrule
{}\stem{de-}, \exemph{de}dwadagyénawahs 

‘let us all help one another’ & {\dualic}

\\ 
\midrule 
{}\stem{dę-}, \exemph{dę}hsnáˀnetˀa:ˀ 

‘you will double it, reinforce something’ & \textsc{\dualic-\future}\\
\lspbottomrule
\end{tabularx}}
\end{table}


\begin{table}
\caption{Words beginning with [de/dę … t-/d-/g-/di-/de-]}
\label{figtab:1:dualnegfutcisl}
{
\begin{tabularx}{\textwidth}{QQQQQ}
\lsptoprule
C & w, V & y & 2 & \\
\midrule
{}\stem{dęt-}, \exemph{dęt}ga:dǫ:goht 

‘I will pass that way’ & \stem{dęd-} & \stem{dęg-}, \exemph{dęg}yá:kne:ˀ 

‘we two (incl.) 

will come 

back’ & \stem{dędi-} \exemph{dędí}hsa̱ˀdre:ˀ

‘you will drive over there’ & {\dualic-\future-\cislocative}\\ 
\midrule 

{}\stem{det-}, 

\exemph{det}geˀ 

‘I am coming back’ & \stem{ded-} & \stem{deg-}, 

\exemph{Deg}yotnǫ̱-hsá:kdǫ:

‘St. Catharine’s, 

Ontario’ & \stem{dede-} 

\textsc{imp} 

\exsc{(e-person)}

\stem{dedi-} 

\stem{dǫda-}, 

\textsc{imp} 

\exsc{(a-person)} & \textsc{\dualic-\cislocative}\\ 
\midrule 

{}\stem{deˀt-} 

\exemph{det}gá:yęˀ 

‘it is not laying there’ & \stem{deˀd-} & \stem{deˀg-} & \stem{dedi-} \exemph{dedí}hsawihs 

‘you’re 

not wanted’ & \textsc{\negative-\cislocative}\\
\lspbottomrule
\end{tabularx}}
\end{table}


\begin{table}
\caption{Words beginning with \stem{da(ˀ)-/dǫ-/de-} \textsc{\cislocative-\factual}}
\label{figtab:1:cislfac}
{
\begin{tabularx}{\textwidth}{QQQQQ}
\lsptoprule
C & (w), (y), V & 1s.p & 2 \\
\midrule
{}\stem{da-}, \exemph{da}ga:dí:yǫˀ ‘they (animals) came in’ & \stem{daˀ-} & \stem{dǫ-}, 

\exemph{dǫ}gáhdǫ:ˀ 

‘(where) I lost it’ & \stem{de-} \exsc{(e-person)}\\
\lspbottomrule
\end{tabularx}}
\end{table}


\begin{table}
\caption{Words beginning with \stem{de(ˀ)-} {\negative}}
\label{figtab:1:neg}
{
\begin{tabularx}{\textwidth}{QQQ}
\lsptoprule
C & (w), (y), V \\
\midrule
{}\stem{de-}, \exemph{de}yohnegi:yo: 

‘it is not good water’ & \stem{deˀ-} \exemph{deˀ}agadǫtgadeˀ 

‘I’m not happy’\\
\lspbottomrule
\end{tabularx}}
\end{table}


\begin{table}
\caption{Words beginning with \stem{da:-/daǫ-/dae-} \textsc{dualic-indefinite}}
\label{figtab:1:dualindef}
{
\begin{tabularx}{\textwidth}{QQQQ}
\lsptoprule
C & 1s.p & 2 \\
\midrule
{}\stem{da:-}, \exemph{da:}gatnę:tsí:yaˀk 

‘I might break my arm’ & \stem{daǫ-} & \stem{dae-} \exsc{e-person} \\
\lspbottomrule
\end{tabularx}}
\end{table}


\begin{table}
\caption{Words beginning with [daǫ/dǫ … da-/dǫ-/de-/sa-/sǫ-/se-]}
\label{figtab:1:dualfacindefrep}
{
\begin{tabularx}{\textwidth}{QQQl}
\lsptoprule
C & \textsc{1s.p} & 2 & \\
\midrule
{}\stem{dǫsa-}, 

\exemph{dǫ:sá}edaˀ 

‘she stood up again’ & & &\textsc{\dualic-\factual-\repetitive}\\
\midrule 

{}\stem{daǫda-}, 

{}\stem{dǫ:da-}, 

\exemph{dǫ:}dá:ge:ˀ 

‘I would come back’ & \stem{daǫdǫ-}, 

{}\stem{dǫ:dǫ-} & \stem{daǫde-}, 

{}\stem{dǫ:de-} & \textsc{\dualic-\indefinite-\cislocative}\\
\midrule 

{}\stem{daǫsa-}, 

{}\stem{dǫ:sa-} & \stem{daǫsǫ-}, 

{}\stem{dǫ:sǫ} & \stem{daǫse-}, 

{}\stem{dǫ:se-} & \textsc{\dualic-\indefinite-\repetitive}\\
\lspbottomrule
\end{tabularx}}
\end{table}

\begin{table}
\caption{Words beginning with \stem{dę/de(ˀ) … s-/j-/t-/ji-}}
\label{figtab:1:dualfutnegrep}
{
\begin{tabularx}{\textwidth}{QQQQl}
\lsptoprule
C & (sy) & s & 2 & \\
\midrule
{}\stem{dęs-} 

\exemph{dęs}haih\-wáętwaht 

‘he will bring forth a message’ & \stem{dęj-}, 

\exemph{dęj}ǫtnǫ-hsǫ́:goht
‘she will go in one door and out the other door’ & \stem{dęt-} & \stem{dęji-} 

\exemph{dęji}dwa\-nǫ́hǫnyǫ:ˀ 

‘we all will thank again’ & \textsc{\dualic-\future-\repetitive}\\
\midrule
{}\stem{des-}, \exemph{des}ga:ní:yǫ:t 

‘it is not hanging’ & \stem{deˀj-}, 

\exemph{dej}a:gó:yǫ: 

‘she didn’t return’ & \stem{det-} & \stem{deˀji-} & \textsc{\negative-\repetitive}\\
\midrule
{}\stem{des-}, 

Tęˀ gęh 

\exemph{des}hó:yǫ:? 

‘Did he not come home?’ & \stem{dej-} & \stem{det-} & \stem{deji-} 

\textsc{imp} 

\exsc{(e-person)} 

\stem{deji-} 

\exemph{deji}dwa-yę:dí: 

‘we all do not know any longer’ 

\stem{dǫsa-} 

\textsc{imp} 

\exsc{(a-person)} & \textsc{\dualic-\repetitive}\\
\lspbottomrule
\end{tabularx}}
\end{table}



\begin{table}
\caption{Words beg. with \stem{ę-} {\future}}\footnote{For E prepronominals, see \tabref{figtab:1:2finaldoubleprefix} and \tabref{figtab:1:finalindefprefix}.}
\label{figtab:1:eprepronfut}
\begin{tabularx}{0.75\textwidth}{Q}
\lsptoprule
C \\
\midrule
\stem{ę-}, \exemph{ę}gátnǫ̱hga:ˀ ‘I’m going to cut my hair’ \\
\lspbottomrule
\end{tabularx}
\end{table}



\begin{table}
\caption{Words beginning with \stem{ę … (h)s-/j-/t-/ji-} \textsc{\future-\repetitive}}
\label{figtab:1:eprepronfutrep}
{
\begin{tabularx}{\textwidth}{QQQQQ}
\lsptoprule
C & (sy) & s & 2 \\
\midrule
{}\stem{ęhs-}, 

\exemph{ęs}hahé:waht 

‘he will punish again’ & \stem{ęj-}, 

\exemph{ęj}é:kse:k 

‘she can eat it again’ & \stem{ęt-}, 

\exemph{ęt}sęˀnigǫhoˀne:k 

‘you will remove yourself’ & \stem{ęji-} 

\exemph{ęji}dwanaˀdáiksǫ:ˀ 

‘we all will snack’\\
\lspbottomrule
\end{tabularx}}
\end{table}


\begin{table}
\caption{Words beginning with \stem{ę … t-/d-/g-/di-} \textsc{\future-\cislocative}}
\label{figtab:1:eprepronfutcisl}
{
\begin{tabularx}{\textwidth}{QQQQQ}
\lsptoprule
C & w, V & y & 2\\
\midrule
{}\stem{ęt-} \exemph{ęt}gaę:ˀ 

‘they will come’ & \stem{ęd-} & \stem{ęg-} 

\exemph{ęg}yoˀgá:hah 

‘early evening’ & \stem{ędi-} 

\exemph{ędi}sa̱ˀnigǫ̱hí:yoh 

‘your mind will become adjusted’\\
\lspbottomrule
\end{tabularx}}
\end{table}


\begin{table}
\caption{Words beginning with [haˀdę-/hę-] }
\label{figtab:1:transdualfut}
{
\begin{tabularx}{\textwidth}{QQ}
\lsptoprule
C & \\
\midrule
{}\stem{haˀdę-}, 

\exemph{haˀd{ę}:}syehs 

‘you will put them all together’ & \textsc{\translocative-\dualic-\future}\\ 
\midrule 

{}\stem{hę-}, 

\exemph{hę}syę:ˀ 

‘you will put it there’ & \textsc{\translocative-\future}\\
\lspbottomrule
\end{tabularx}}
\end{table}


\begin{table}
\caption{Words beginning with [haˀde-/haˀdi-/he(ˀ)-]}
\label{figtab:1:epreprontrans}
{
\begin{tabularx}{\textwidth}{QQl}
\lsptoprule
C & 2\\
\midrule
{}\stem{haˀde-}, 

\textit{haˀde}ga̱ˀdréhdage: 

‘all kinds of cars’ & \stem{haˀde-}, \stem{haˀdi-}

\textsc{imp} (e-person) 

{}\stem{haˀt-} \textsc{imp} (a-person) & \textsc{\translocative-\dualic}\\
\midrule 

{}\stem{he-} & \stem{heˀ-}

\textsc{imp} \exsc{(e person)}, 

\exemph{heˀ}sá:dih 

‘throw it away from me!’ & \textsc{\translocative}\\
\lspbottomrule
\end{tabularx}}
\end{table}


\begin{table}
\caption{Words beginning with \stem{haˀ-/hǫ-/heˀ-} \textsc{\translocative-\factual}}
\label{figtab:1:transfact}
{
\begin{tabularx}{\textwidth}{QQQl}
\lsptoprule
C & 1s.p & 2\\
\midrule
{}\stem{haˀ{}-}, \exemph{haˀ}geˀ ‘I am going there’ & \stem{hǫ-} & \stem{heˀ-}\\
\lspbottomrule
\end{tabularx}}
\end{table}


\begin{table}
\caption{Words beginning with \stem{haˀ … t-/d-/g-} \textsc{\translocative-\factual-\dualic}}
\label{figtab:1:transfactdual}
{
\begin{tabularx}{\textwidth}{QQQl}
\lsptoprule
C & w, V & y\\
\midrule
{}\stem{haˀt-}, 

\exemph{haˀt}gęnętsáǫnyǫ:ˀ 

‘I waved my arms’ & \stem{haˀd-} & \stem{haˀg-}\\
\lspbottomrule
\end{tabularx}}
\end{table}


\begin{table}
\caption{Words beginning with [h/haˀd … a:-/aǫ-/ae-/e:-]}
\label{figtab:1:transdualindef}
{
\begin{tabularx}{\textwidth}{QQQQ}
\lsptoprule
C & \textsc{1s.p} & 2 & \\
\midrule
{}\stem{ha:-}, 
to hǫ: 

\exemph{ha:}gi̱ˀdrǫ:daˀk


‘where I would dwell’ & \stem{haǫ-} & \stem{hae-} \stem{haˀ-} 

\textsc{imp} (a-person) & \textsc{\translocative-\indefinite}\\
\midrule
{}\stem{haˀda:-} & \stem{haˀdaǫ-} & \stem{haˀdae-}, \stem{haˀde:-} & \textsc{\translocative-\dualic-\indefinite}\\
\lspbottomrule
\end{tabularx}}
\end{table}


\begin{table}
\caption{Words beginning with [haǫ/hǫ: … sa(ˀ)-/sǫ-/se-]}
\label{figtab:1:transindefrepfact}
{
\begin{tabularx}{\textwidth}{QlQQQ}
\lsptoprule
C & (w), (y), V & \textsc{1s.p} & 2 & \\
\midrule
{}\stem{haǫsa-}, \stem{hǫ:sa-}, 

to e: \exemph{hǫsá:}ge:ˀ 

‘I would go again’ & \stem{hǫ:saˀ-} & \stem{haǫsǫ-}, {}\stem{hǫ:sǫ-} & \stem{haǫse-}, \stem{hǫ:se-} & \textsc{\translocative-\indefinite-\repetitive}\\
\midrule
{}\stem{hǫsa-}, \exemph{hǫ:sa}há:yǫˀ ‘he went back inside’ & \stem{hǫsaˀ-} & \stem{hǫsǫ-} & \stem{hǫse-} & \textsc{\translocative-\repetitive-\factual}\\
\lspbottomrule
\end{tabularx}}
\end{table}



\begin{table}
\caption{Words beginning with [hę/he … s-/j-/t-/ji-]}
\label{figtab:1:transfutrep}
{
\begin{tabularx}{\textwidth}{QQQQQ}
\lsptoprule
C & (sy) & s & 2 & \\
\midrule
{}\stem{hęhs-}, 

\exemph{hęs}geˀ 

‘I am going back there’ & \stem{hęj-}, 

\exemph{hęj}a:gǫ:nęht 

‘they will swallow again’ & \stem{hęt-}, 

\exemph{hęt}syę:ˀ 

‘you will put it back in its place’ & \stem{hęji-} 

\exemph{hęjí}sasha:ˀ 

‘you will remember back in time’ & \textsc{\translocative-\future-\repetitive}\\
\midrule
{}\stem{hehs-}, 

I:wa:kˀáh 

\exemph{hes}hohdri̱hs\-dǫhǫgyeˀ

‘He’s getting nearer and nearer.’ & \stem{hej-}, 

\exemph{hej}ónǫ̱hsǫ:t 

‘the next room’ & \stem{het-} & \stem{heji-}, \stem{heji-} 

\textsc{imp} 

(e-person), 

Gaę hwaˀ nhǫ: 

\exemph{heji}sáihoˀdeˀ? 

‘Where do you work again?’ 

{}\stem{hǫsa-} 

\textsc{imp} 

(a-person) & \textsc{\translocative-\repetitive}\\
\lspbottomrule
\end{tabularx}}
\end{table}



\begin{table}
\caption{Words beginning with [i-/is-/ij-/it-]}
\label{figtab:1:prothrep}
{
\begin{tabularx}{\textwidth}{QXXl}
\lsptoprule
C & (sy) & s & \\
\midrule
{}\stem{i-}

\exemph{í:}geˀs 

‘I am here’ &  &  & \textsc{\prothetic}\\
\midrule
{}\stem{is-}, 

\exemph{is}geˀs 

‘I’ve returned home’ & \stem{ij-} & \stem{it-} & {\prothetic.\repetitive}\\
\lspbottomrule
\end{tabularx}}
\end{table}



\begin{table}
\caption{Words beginning with \stem{it-/id-/ig-} \textsc{\prothetic-\cislocative}}
\label{figtab:1:prothcisl}
{
\begin{tabularx}{\textwidth}{XXQ}
\lsptoprule
C & w, V & y\\
\midrule
{}\stem{it-} & \stem{id-} & \stem{ig-} 

\exemph{í:g}ye:t 

‘she is standing over there’\\
\lspbottomrule
\end{tabularx}}
\end{table}



\begin{table}
\caption{Words beginning with [naˀ/nę/ni … t-/d-/g-/di-]}
\label{figtab:1:partothercisl}
{
\begin{tabularx}{\textwidth}{QlQQl}
\lsptoprule d
C & w, V & y & 2 & \\
\midrule
{}\stem{naˀdęt-}, \exemph{naˀd{ę}:t}ge:ˀ 

‘I’ll come back over here, return’ & \stem{naˀdęd-} & \stem{naˀdęg-} &  & \textsc{\partitive-\dualic-\future-\cislocative}\\
\midrule
{}\stem{naˀdet-}

\exemph{naˀdé:t}geˀ 

‘I am coming back over here, returning’ & \stem{naˀded-} & \stem{naˀdeg-} 


sh{ę}h \exemph{nadeg}yadǫdai ‘during the week’ &  & \textsc{\partitive-\dualic-\cislocative}\\
\midrule
{}\stem{nęt-} & \stem{nęd-} & \stem{nęg-} & \stem{nędi-} & \textsc{\partitive-\future-\cislocative}\\
\midrule
{}\stem{nit-} 

\exemph{nit}ha:wé:nǫh 

‘male stranger’ & \stem{nid-} & \stem{nig-}

\exemph{nig}yagawé:nǫh 

‘she comes from’ & \stem{nidi-} 

\exemph{nidi}hsé:nǫh 

‘you come from’ & \textsc{\partitive-\cislocative}\\
\lspbottomrule
\end{tabularx}}
\end{table}


\begin{table}
\caption{Words beginning with [n/nh … aˀ-/ǫ-/eˀ-] or [niwag-]}
\label{figtab:1:parttransfact}
{
\begin{tabularx}{\textwidth}{QQll}
\lsptoprule
C & \textsc{1s.p} & 2 & \\
\midrule
{}\stem{naˀ-}

sh{ę}h \exemph{naˀ}á:węh 

‘how it happened’ & \stem{nǫ-}, \stem{niwag-} 

\exemph{nǫ:}ge̱ˀsyáoˀdęh,

\exemph{niwage̱}ˀsyáoˀdęh

‘I’m… clan’ & \stem{neˀ-} & \textsc{\partitive-\factual}\\
\midrule
{}\stem{nhaˀ-} & \stem{nhǫ-} & \stem{nheˀ-} & \textsc{\partitive-\translocative-\factual}\\
\lspbottomrule
\end{tabularx}}
\end{table}


\begin{table}
\caption{Words beginning with [naˀ/nǫ(:)/naǫ … da-/dǫ-/di-/de-/sa-/sǫ-/se-]}
\label{figtab:1:partothercislrep}
{
\begin{tabularx}{\textwidth}{QQQl}
\lsptoprule
C & \textsc{1s.p} & 2 & \\
\midrule
{}\stem{naˀdǫda-}, 

sh{ę}h \exemph{naˀdǫ:dá:}ge:ˀ 

‘where I’m coming from’ & \stem{naˀdǫdǫ-} & \stem{naˀdǫdi-} & \textsc{\partitive-\factual-\dualic-\cislocative}\\
\midrule
{}\stem{nǫda-}

\exemph{nǫdá:}ge:ˀ 

‘I come from’ & \stem{nǫdǫ-} & \stem{nǫdi-} 

Gaę nhǫ \exemph{nǫdi:}sé:nǫh? 

‘Where are you from?’ & \textsc{\partitive-\factual-\cislocative}\\
\midrule
{}\stem{naǫda-}, 

{}\stem{nǫ:da-} & \stem{naǫdǫ-}, 

{}\stem{nǫ:dǫ-} & \stem{naǫda-}, 

{}\stem{naǫde-}, 

{}\stem{nǫ:de-} & \textsc{\partitive-\indefinite-\cislocative}\\
\midrule
{}\stem{naˀdaǫda-}, 

{}\stem{naˀdǫ:da-}, 

\exemph{naˀdǫ:dá:}geˀ 

‘I should come this way’ & \stem{naˀdaǫdǫ-}, 

\stem{naˀdǫ:dǫ-} &  & \textsc{\partitive-\dualic-\indefinite-\cislocative}\\
\midrule
{}\stem{naǫsa-}, 

{}\stem{nǫ:sa-}, 

\exemph{naǫsa}ǫgway{ę}hę:k 

‘what we all should be doing’ & \stem{naǫsǫ-}, 

{}\stem{nǫ:sǫ-} & \stem{naǫse-} & \textsc{\partitive-\indefinite-\repetitive}\\
\midrule
{}\stem{nǫsaˀ-} & \stem{nǫsǫ-} & \stem{nǫse-} & \textsc{\partitive-\factual-\repetitive}\\
\lspbottomrule
\end{tabularx}}
\end{table}


\begin{table}
\caption{Words beginning with [naˀ … t-/d-/g-/de-/di-]}
\label{figtab:1:partfactdual}
{
\begin{tabularx}{\textwidth}{QQQll}
\lsptoprule
C & w, V & y & 2 & \\
\midrule
{}\stem{naˀt-} & \stem{naˀd-} & \stem{naˀg-} & \stem{naˀde-}, \stem{naˀdi-} & \textsc{\partitive-\factual-\dualic}\\
\lspbottomrule
\end{tabularx}}
\end{table}


\begin{table}
\caption{Words beginning with [naˀd/n/nh … e-/ę-]}
\label{figtab:1:partother}
{
\begin{tabularx}{0.75\textwidth}{QQ}
\lsptoprule
C & \\
\midrule
{}\stem{naˀde-}

\exemph{naˀde}gáǫdreˀ 

‘how far apart they are’ & \textsc{\partitive-\dualic}\\
\midrule 

{}\stem{naˀdę-} & \textsc{\partitive-\dualic-\future}\\
\midrule 

{}\stem{nę-}

\exemph{nę}wú:kˀuh

‘it will be small’ & \textsc{\partitive-\future}\\
\midrule 

{}\stem{nhę-} & \textsc{\partitive-\translocative-\future}\\
\midrule 

{}\stem{nhe-}

\exemph{nhé:}yoht 

‘suddenly it occurred’ & \textsc{\partitive-\translocative}\\
\lspbottomrule
\end{tabularx}}
\end{table}


\begin{table}
\caption{Words beginning with [n/naˀd/nh … a:-/aǫ-/ae-]}
\label{figtab:1:partdualtransindef}
{
\begin{tabularx}{\textwidth}{Qlll}
\lsptoprule
C & \textsc{1s.p} & 2 & \\
\midrule
{}\stem{na:-}

\exemph{na:}yá:węh 

‘it should happen’ & \stem{naǫ-} & \stem{nae-} & \textsc{\partitive-\indefinite}\\
\midrule
{}\stem{naˀda:-} & \stem{naˀdaǫ-} & \stem{naˀdae-} & \textsc{\partitive-\dualic-\indefinite}\\
\midrule
{}\stem{nha:-} & \stem{nhaǫ-} & \stem{nhae-} & \textsc{\partitive-\translocative-\indefinite}\\
\lspbottomrule
\end{tabularx}}
\end{table}


\begin{table}
\caption{Words beginning with \stem{ni-/naˀ-} \textsc{\partitive}}
\label{figtab:1:prepronpart}
{
\begin{tabularx}{0.66\textwidth}{QQ}
\lsptoprule
C & 2 \\
\midrule
{}\stem{ni-}

\exemph{ni}gahn{ę}:yaˀ 

‘it is short’ & \stem{naˀ-} 

\textsc{imp}, let’s \\
\lspbottomrule
\end{tabularx}}
\end{table}


\begin{table}
\caption{Words beginning with [nę/ni … (h)s-/j-/t-/ji-]}
\label{figtab:1:partfutrep}
{
\begin{tabularx}{\textwidth}{QQlll}
\lsptoprule
C & (sy) & s & 2 & \\
\midrule
{}\stem{nęhs-} & \stem{nęj-} & \stem{nęt-} & \stem{nęji-} & \textsc{\partitive-\future-\repetitive}\\
\midrule
{}\stem{nis-}

\exemph{Nis}hayé:no:s 

‘Hold Unto’ & \stem{nij-}

\exemph{nij}agowé:nǫh 

‘female stranger’ & \stem{nit-} & \stem{niji-} & \textsc{\partitive-\repetitive}\\
\lspbottomrule
\end{tabularx}}
\end{table}


\begin{table}
\caption{Words beginning with \stem{s-/j-/sa-/ji-} {\repetitive}}
\label{figtab:1:srep}
{
\begin{tabularx}{\textwidth}{QQQQ}
\lsptoprule
C & (sy) & s & 2 \\
\midrule
{}\stem{s=}

\exemph{s}gayáˀda:t 

‘one (living thing)’ & \stem{j-} 

\exemph{j}odr{ę}:no:t 

‘it is singing again’ & \stem{sa-} 

\textsc{imp}

\exemph{sa}sęhni̱hs\-gahsrǫ:nih

‘fix the wheel’

\exemph{sa}sahdę:dih 

‘go home!’ & \stem{ji-} 

\textsc{imp}

\exemph{ji}dwahshé:t e:ˀ 

‘let’s count again!’\\
\lspbottomrule
\end{tabularx}}
\end{table}


\begin{table}
\caption{Words beginning with \stem{sa-/sǫ-/se-} \textsc{\repetitive-\factual}}
\label{figtab:1:srepfact}
{
\begin{tabularx}{\textwidth}{QQl}
\lsptoprule
C & \textsc{1s.p} & 2\\
\midrule 
\stem{sa-} 

\exemph{sa}gęˀnhoˀtrá:twaht 

‘I missed the ball’ & \stem{sǫ-}

\exemph{sǫ}gi:daˀ 

‘I went back to sleep’ & \stem{se-}\\
\lspbottomrule
\end{tabularx}}
\end{table}


\begin{table}
\caption{Words beginning with \stem{t-/d-/g-/de-/di-} {\cislocative}}
\label{figtab:1:scisl}
{
\begin{tabularx}{\textwidth}{QQQQ}
\lsptoprule
C & w, V & y & 2 \\
\midrule 
{}\stem{t-}, 

\exemph{t}hadínagreˀ 

‘that’s where they live over there’ & \stem{d-}, 

né:ˀ gwáhs 

\exemph{d}wakyęda̱h\-gówanęh

‘the biggest chair’ & \stem{g-}, 

\exemph{g}ye̱ˀdrǫˀ

‘she’s at home’ & \stem{de-} 

\textsc{imp}

\exsc{(e-person) }

\exemph{de}sáˀdre:, 

‘drive over here’ 

\stem{di-} 

\exemph{di}sáhdęgyǫ: 

‘you come from there’ \\
\lspbottomrule
\end{tabularx}}
\end{table}


\begin{table}
\caption{Words beginning with [ts … t-/d-/g-/di-]}
\label{figtab:1:coinfutdualcisl}
{
\begin{tabularx}{\textwidth}{llQll}
\lsptoprule
C & w, V & y & 2 & \\
\midrule 
{}\stem{tsaˀdet-} & \stem{tsaˀded-} & \stem{tsaˀdeg-}, \exemph{tsadeg}ya̱ˀ\-dǫdadrehk 

‘the week before last’ &  & \textsc{\coincident-\dualic-\cislocative}\\
{}\stem{tsęt-} & \stem{tsęd-} & \stem{tsęg-} & \stem{tsędi-} & \textsc{\coincident-\future-\cislocative}\\
\midrule 
{}\stem{tsit-} & \stem{tsid-} & \stem{tsig-} & \stem{tsidi-} & \textsc{\coincident-\cislocative}\\
\lspbottomrule
\end{tabularx}}
\end{table}

\begin{table}
\caption{Words beginning with [t/ts … i-/aˀ-]}
\label{figtab:1:coinorcontr}
{
\begin{tabularx}{\textwidth}{QQl}
\lsptoprule
C & 2 & \\
\midrule
{}\stem{tsi-}

\exemph{tsi}yeksa̱ˀda:sé:ˀah

‘when she was a teenager’ & \stem{tsaˀ-} 

\textsc{imp}, let’s & \textsc{\coincident}\\
\midrule 
\stem{ti-}, 

\exemph{ti}gaˀdre̱hdá:deˀ

‘a different car’ & \stem{taˀ-} 

\textsc{imp}, let’s & {\contrastive}\\
\lspbottomrule
\end{tabularx}}
\end{table}


\begin{table}
\caption{Words beginning with [t(s) … ę-/e(ˀ)-]}
\label{figtab:1:coincontrother}
{
\begin{tabularx}{0.75\textwidth}{Ql}
\lsptoprule
C & \\
\midrule
{}\stem{taˀde-} & \textsc{\contrastive-\dualic}\\
\midrule 
{}\stem{tsaˀde-} 

\exemph{tsaˀdé:}waˀs 

‘they are the same size’ & \textsc{\coincident-\dualic}\\
\midrule 
{}\stem{taˀdę-} 

\exemph{taˀdę}jatáhahk 

‘you two will walk side by side’ & \textsc{\contrastive-\dualic-\future}\\
\midrule 
{}\stem{tę-} 

\exemph{tę}hayętó:ˀah 

‘he’ll just plant it any old way or helter-skelter’ & \textsc{\contrastive-\future}\\
\midrule 
{}\stem{tsaˀdę-} 

\exemph{tsaˀd{ę}}hsyę:ˀ 

‘you will lay them side by side’ & \textsc{\coincident-\dualic-\future}\\
\midrule 
{}\stem{tsę-} & \textsc{\coincident-\future}\\
\midrule 
{}\stem{taˀde-} 

\exemph{taˀde}gawá:yǫ:t 

‘it has no wings’ & \textsc{\contrastive-\negative}\\
\midrule 
{}\stem{teˀ-} 

tęˀ toh \exemph{teˀ}gáhe:ˀ 

‘it’s not sitting way over there’ & \textsc{\contrastive-\translocative}\\
\lspbottomrule
\end{tabularx}}
\end{table}


\begin{table}
\caption{Words beginning with [t(s) … aˀ-/ǫ-/eˀ-]}
\label{figtab:1:contrcoinfactdual}
{
\begin{tabularx}{\textwidth}{QQQl}
\lsptoprule
C & \textsc{1s.p} & 2 & \\
\midrule
{}\stem{taˀ-} & \stem{tǫ-} & \stem{teˀ-} & \textsc{\contrastive-\factual}\\
\midrule
{}\stem{tsaˀ-} 

\exemph{tsaˀ}ga:y{ę}:daˀ 

‘when it became’ & \stem{tsǫ-} & \stem{tseˀ-} & \textsc{\coincident-\factual}\\

\lspbottomrule
\end{tabularx}}
\end{table}


\begin{table}
\caption{Words beginning with [t(s) … de-/di-/d-/g-/t-]}
\label{figtab:1:othercisldual}
{
\begin{tabularx}{\textwidth}{QQQQl}
\lsptoprule
C & w, V & y & 2 & \\
\midrule
{}\stem{tǫde-}, 

tęˀ \exemph{tǫde}sa̱ˀdré:

‘you won’t drive  over here’ &  &  & \stem{tǫdi-} & \textsc{\dualic-\factual-\cislocative}\\
\midrule
{}\stem{tsaˀt-} & \stem{tsaˀd-} & \stem{tsaˀg-} & \stem{tseˀd-} V 

{}\stem{tseˀg-} y 

{}\stem{tseˀt-} C & \textsc{\coincident-\factual-\cislocative}\\
\midrule
{}\stem{tsa:t-} & \stem{tsa:d-} & \stem{tsa:g-}, 

\stem{tsaeg-} y & \stem{tsaed-} y, V 

\stem{tsaet-} C & \textsc{\coincident-\indefinite-\cislocative}\\
\midrule
{}\stem{teˀt-} & \stem{teˀd-} & \stem{teˀg-} &  & \textsc{\contrastive-\translocative-\dualic}\\
\midrule
{}\stem{taˀt-} & \stem{taˀde-}, 

{} \stem{teˀd-} y, V & \stem{taˀg-}, 

{}\stem{teˀg-} y & \stem{teˀdi-} & \textsc{\contrastive-\factual-\dualic}\\
\lspbottomrule
\end{tabularx}}
\end{table}



\begin{table}
\caption{Words beginning with [t(s) … a(:)-/(a)ǫ-/ae-/e:]}
\label{figtab:1:contrcoinindefcislrep}
{
\begin{tabularx}{\textwidth}{QlQQ}
\lsptoprule
C & \textsc{1s.p} & 2 & \\
\midrule 
{}\stem{ta:-} & \stem{taǫ-} & \stem{tae-}, 2 & \textsc{\contrastive-\translocative-\indefinite}\\
\midrule 
{}\stem{ta:-}, tęˀ 

\exemph{ta:}hay{ę}:toh 

‘no, he won’t plant’ & \stem{taǫ-} & \stem{tae-}, \stem{te:-} 

tęˀ \exemph{tae}swá:gęh, 

\exemph{te:}swá:gęh 

‘you all shouldn’t see it’ 

\exemph{te:}sagáę 

‘you are not willing’ & \textsc{\contrastive-\indefinite}\\
\midrule 
{}\stem{taˀda:-} & \stem{taˀdaǫ-} & \stem{taˀdae-} & \textsc{\contrastive-\dualic-\indefinite}\\
\midrule 
{}\stem{tsa:-} & \stem{tsaǫ-} & \stem{tsae-}, \stem{tse:-} & \textsc{\coincident-\indefinite}\\
\midrule 
{}\stem{tsaˀda:-} & \stem{tsaˀdaǫ-} & \stem{tsaˀdae-} & \textsc{\coincident-\dualic-\indefinite}\\
\midrule 
{}\stem{taǫda-}, \stem{tǫ:da-} & \stem{taǫdǫ-}, \stem{tǫ:dǫ-} & \stem{taǫde-} & \textsc{\contrastive-\indefinite-\cislocative}\\
\midrule 
{}\stem{taǫsa-} & \stem{taǫsǫ-} & \stem{taǫse-} & \textsc{\contrastive-\indefinite-\repetitive}\\
\lspbottomrule
\end{tabularx}}
\end{table}



\begin{table}
\caption{Words beginning with [t(s) … s-/j-/t-/ji-]}
\label{figtab:1:coincontrotherrep}
{
\begin{tabularx}{\textwidth}{llXXl}
\lsptoprule
C & (sy) & s & 2 & \\
\midrule 
{}\stem{tsa:s-}, \stem{tsaǫs-} & \stem{tsa:j-} & \stem{tsa:t-} & \stem{tsaes-} & \textsc{\coincident-\indefinite-\repetitive}\\
\midrule 
{}\stem{tsaˀs-} & \stem{tsaˀj-} & \stem{tsaˀt-} & \stem{tseji-} & \textsc{\coincident-\factual-\repetitive}\\
\midrule 
{}\stem{taˀdes-} & \stem{taˀdej-} &  &  & \textsc{\contrastive-\dualic-\repetitive}\\
\midrule 
{}\stem{tęs-} &  &  &  & \textsc{\contrastive-\future-\repetitive}\\
\midrule 
{}\stem{tis-} &  &  &  & \textsc{\contrastive-\repetitive}\\
\midrule 
{}\stem{tsęhs-} & \stem{tsęj-} & \stem{tsęt-} & \stem{tsęji-} & \textsc{\coincident-\future-\repetitive}\\
\midrule 
{}\stem{tsihs-} & \stem{tsij-} & \stem{tsit-} & \stem{tsiji-} & \textsc{\coincident-\repetitive}\\
\lspbottomrule
\end{tabularx}}
\end{table}
