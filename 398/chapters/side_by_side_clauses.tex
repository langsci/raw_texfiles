\chapter{Sentences with juxtaposed clauses (no linking words)} \label{Sentences with side-by-side clauses}
In sentences with \textsc{juxtaposed} (collocated) clauses, two or more clauses are joined without linking words. Four types are described next. 

\section{Sentences with juxtaposed clauses sharing the same role} \label{Sentences with side-by-side clauses sharing the same role}
Sentences may consist of juxtaposed clauses (or juxtaposed verbs) that both have the same role (clauses are shown in square brackets below). (For role, see \sectref{Pronominal prefixes and role}.) For example, both clauses in \xref{ex:sidebysidex} refer to the same experiencer or actor, ‘I’, while both clauses in \xref{ex:sidebysidex2} refer to something that burned down. Other examples are shown in \xxref{ex:sidebysidex3}{ex:sidebysidex300}.

\ea\label{ex:sidebysidex} 
\gll [dewagegaę́hs] [to há:ge:ˀ] \\
I’m.unwilling there I.would.go.there\\
\glt ‘I am unwilling to go there.’ 
\z

\ea\label{ex:sidebysidex2} 
\gll [Haˀwa:tsˀá:ˀ] [aˀó:de:k].\\
it.went.down.to.nothing it.burned\\
\glt ‘It burned down to nothing.’
\z

\ea\label{ex:sidebysidex3} 
\gll [A:hęni:hę́:ˀ] [hahjáotaˀ].\\
 he.would.quit he.smokes.all.the.time\\
\glt ‘He would stop himself from smoking.’, ‘He would quit smoking.’
\z

\ea\label{ex:sidebysidex300} 
\gll [Ęgęni̱hę́:ˀ] [knegéhaˀ].\\
 I.will.quit I.drink.habitually\\
\glt ‘I’ll quit drinking.’
\z


\section{Sentences with juxtaposed clauses (different roles)} \label{Sentences with side-by-side clauses (different roles)}
Sentences may consist of juxtaposed clauses (or juxtaposed verbs) that express different roles -- for example, one describing an experiencer ‘I’, and one describing an agent, ‘her’ \xref{ex:sidebysidex4}. Other examples are provided in \xxref{ex:sidebysidex400}{ex:sidebysidex401}.  

\ea\label{ex:sidebysidex4} 
\gll [Knigǫhá:ˀ] [ęyé:yǫˀ].\\
 I’m.expecting she.will.arrive\\
\glt ‘I am expecting [her to arrive]’, ‘I am waiting [for her to arrive.]’
\z

\ea\label{ex:sidebysidex400}  
\gll  [O:nę́h giˀ haˀgahé:ˀ] [ęshétro̱hna].\\
 now just it.is.time you.will.take.him.back\\
\glt ‘It is time [for you to take him back].’
\z

\ea\label{ex:sidebysidex401} 
\gll [I:wí:] [to na:yá:węh].\\
 I.want that it.should.happen\\
\glt ‘I want, intend [(for) that to happen]’
\z

Two special types of sentences containing juxtaposed clauses and expressing different roles are described next.

\subsection{Sentences with \textit{á:węˀ, wá:dǫh, á:yęˀ, a:wé:tˀah}} \label{Sentences with á:węˀ, wá:dǫh, á:yęˀ, a:wé:tˀah}
Juxtaposed sentences can also begin with a clause from the list in \xref{ex:sidebysidex5}, followed by a clause that expresses a different role. Such sentences convey hearsay or degrees of certainty (\ref{ex:sidebysidex6}- \ref{ex:sidebysidex7}, also see \textit{Evidential markers}, \sectref{ch:Evidential markers}).

\ea\label{ex:sidebysidex5} 
\ea á:węˀ ‘it is said’
\ex agęˀ ‘it is said’
\ex wá:dǫh ‘it is said’
\ex á:yęˀ ‘it seems’
\ex a:wé:tˀah ‘it is pretend, implied’
\z
\z

\ea\label{ex:sidebysidex6} 
\gll Gwé:, [a:yéˀ] [sanǫ́hnya̱ˀgǫh]. \\
well it.seems you.are.hurt\\
\glt ‘Well, [it looks like] [you are hurt].’ 
\z

\ea\label{ex:sidebysidex7} 
\gll [A:we:tˀáh] [agída̱ˀǫh].\\
it.is.pretend I’m.sleeping\\
\glt ‘I am pretending [(that) I am asleep].’, ‘I am pretending [to be asleep].’
\z


\subsection{Sentences with \textit{wę́:do:ˀ, wagyéhsaˀgeh}} \label{Sentences with wę́:do:ˀ, wagyéhsaˀge}
Certain juxtaposed sentences begin with the clause (verb) \textit{wę́:dǫ:ˀ} ‘it is hard’ \xref{ex:sidebysidex8} or \textit{wagyéhsa̱ˀgeh} ‘it is easy’ \xref{ex:sidebysidex9}. (For the structure of the verbs in the second clause in these examples, see \sectref{Habituals with ‘easy to / hard to’ constructions}, \sectref{Stative verbs with ‘easy to / hard to’ constructions}.)

\ea\label{ex:sidebysidex8} 
\gll Wę:dó:ˀ [da:syadǫhsrí:yaˀk].\\
it.is.hard you.would.cut.paper\\
\glt ‘It’s hard [for you to cut the paper].’
\cfex{\gll d-a:-s-hyadǫhsr-í:yaˀk\\
{\cislocative}-{\indefinite}-\textsc{2s.a}-paper-cut.{\zeropunctual}\\}
\z

\ea\label{ex:sidebysidex9} 
\gll Wagyehsa̱ˀgéh [da:syadǫhsrí:yaˀk].\\
it.is.easy you.would.cut.paper\\
\glt ‘It’s easy [for you to cut the paper].’
\cfex{\gll d-a:-s-hyadǫhsr-í:yaˀk\\
{\cislocative}-{\indefinite}-\textsc{2s.a}-paper-cut.{\zeropunctual}\\}
\z

Sometimes the role (actor, experiencer, etc.) of the second verb is overtly expressed in a noun (‘John’). The noun comes either before the verb \xref{ex:sidebysidex10}, or after the verb. In the second case, the noun is preceded by \textit{neˀ} ‘the’ \xref{ex:sidebysidex11}. 

\ea\label{ex:sidebysidex10} 
\gll Ǫ:gwéh gaǫdǫ́h wę:dó:ˀ [John dahátahahk].\\
people they.say it.is.difficult John he.took.the.road\\
\glt ‘People say that it is hard [for John to walk].’
\cfex{\gll d-a-h-át-hah-a-hkw\\
{\cislocative}-{\factual}-\textsc{3s.m.a}-{\semireflexive}-road-{\joinerA}-pick.up.{\zeropunctual}\\}
\z

\ea\label{ex:sidebysidex11} 
\gll Wę:dó:ˀ agę́ˀ [dahataháhk neˀ John]\\
it.is.difficult it.is.said he.took.the.road the John\\
\glt ‘It’s said that it’s hard [for John to walk].’
\z

In the following examples, where the “doer” or agent (e.g., of \emph{cutting}) is unknown, the second clause contains a special type of habitual verb \xxref{ex:sidebysidex12}{ex:sidebysidex120} or stative verb (\ref{ex:sidebysidex13}-\ref{ex:sidebysidex130}, see \sectref{Variations on the habitual aspect}, \sectref{Variations on the stative aspect}).

\ea\label{ex:sidebysidex12} 
\gll wagyesa̱ˀgéh [da:ga̱hyadǫ̱hsríyaˀksǫ:k].\\
it.is.easy for.the.paper.to.cut\\
\glt ‘The paper [(it) is easy to cut].’
\cfex{ \gll d-a:-ga̱-hyadǫ̱hsr-íyaˀk-sǫ:-k\\ 
{\dualic}-{\indefinite}-\textsc{3s.a}-paper-cut-{\habitual}-{\pluralizer}-{\modalizer}\\}
\z

\ea\label{ex:sidebysidex120}
\gll Wę:dó:ˀ da:ga̱hyadǫ̱hsríyaˀksǫ:k.\\
it.is.hard the.paper.should.be.cut\\
\glt \trs{The paper [(it) is hard to cut].}
\z

\ea\label{ex:sidebysidex13} 
\gll Wagyehsa̱ˀgeh da:ga̱hyadǫ̱hsriya̱ˀgǫ́ha:k.\\
it.is.easy the.paper.would.be.cut\\
\glt ‘The paper is easy to cut.’
\cfex{\gll d-a:-ga̱-hyadǫ̱hsr-iya̱ˀg-ǫ́-h-a-:k\\
{\dualic}-{\indefinite}-\textsc{3s.a}-paper-cut-{\stative}-\textsc{euph.h}-{\joinerA}-\exsc{\modalizer}\\}
\z

\ea\label{ex:sidebysidex130} 
\gll Wę:do:ˀ da:ga̱hyadǫ̱hsriya̱ˀgǫ́ha:k.\\
it.is.hard the.paper.should.be.cut\\
\glt ‘The paper is hard to cut.’
\z

It may be that \textit{wę́:dǫ:ˀ} and \textit{wagyéhsa̱ˀgeh} are only followed by certain kinds of verbs -- for example, verbs describing activities: for verbs describing mental states, alternative phrases or work-arounds seem to be preferable \xxref{ex:sidebysidex14}{ex:sidebysidex145}. 

\ea\label{ex:sidebysidex14}
\gll Tihǫgwe̱ˀdoˀdę:hę́h gę́:s tǫ: John ta:hehsnǫ́hweˀ.\\
all.sorts.of.people(males) usually that.one John you.don’t.like.him\\
\glt ‘It is hard to like John.’, ‘John is hard to like.’ 
\z

\ea\label{ex:sidebysidex140}
\gll De̱hǫwadinǫ̱hwéˀs neˀ John.\\
they(males).don’t.like.him the John\\
\glt ‘It is hard to like John.’, ‘John is hard to like.’
\z

\ea\label{ex:sidebysidex141}
\gll Hǫwadi̱hswahę́hs tǫ: John.\\
They.hate.him that.one John\\
\glt ‘It is hard to like John.’, ‘John is hard to like.’ 
\z

\ea\label{ex:sidebysidex142}
\gll Hǫgweˀdi:yó: gę́:s tǫ: John.\\
he.is.a.good.person usually that.one John\\
\glt ‘It is easy to like John.’, ‘John is easy to like.’
\z

\ea\label{ex:sidebysidex143}
\gll Hǫgweˀdi:yó: neˀ John.\\
he.is.a.good.person the John\\
\glt ‘It is easy to like John.’, ‘John is easy to like.’
\z

\ea\label{ex:sidebysidex144}
\gll Hodǫgwe̱ˀdanǫ̱hwéˀ tǫ: John.\\
he.is.a.liked.person that.one John\\
\glt ‘It is easy to like John.’, ‘John is easy to like.’
\z

\ea\label{ex:sidebysidex145}
\gll Hǫwadinǫhwéˀs John hǫgweˀdí:yo:\\
they(males).like.him John he.is.a.good.person\\
\glt ‘It is easy to like John.’, ‘John is easy to like.’
\z
