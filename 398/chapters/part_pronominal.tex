\setpartpreamble{\bigskip\bigskip\noindent\textsc{Pronominal prefixes} refer to people, places, objects, or abstract concepts. All Gayogo̱ho:nǫˀnéha:ˀ verbs have pronominal prefixes, and most nouns have them as well. In contrast, particles do not. For the most part, nouns and verbs take the same pronominal prefixes, and so these prefixes express related concepts. For example, the verb \textit{aknǫhǫkdá:nih} ‘I am sick’ takes the \stem{ak-} prefix to refer to the speaker, while the noun \textit{aknǫ́hsaˀ} ‘my house’ takes the same \stem{ak-} prefix to denote the speaker’s ownership.

This chapter describes pronominal prefix meaning. The following chapter describes how the prefixes are pronounced, and the subsequent chapter describes pronominal prefix \emph{selection} -- the factors influencing the type of prefix a verb or noun takes.}
\part{Pronominal prefixes} \label{Pronominal prefixes}
