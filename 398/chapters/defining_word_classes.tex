\chapter{Defining nouns, verbs, and particles} \label{ch:Defining nouns, verbs, particles}
\textsc{nouns}, \textsc{verbs}, and \textsc{particles} are uniquely defined by the three criteria in \tabref{figtab:1:threewordclass}. 

\begin{table}
\caption{Three word classes\label{figtab:1:threewordclass}}

\begin{tabular}{l@{~~}llcc}
\lsptoprule
% word class &
% examples
\multicolumn{2}{l}{class/examples} & num. vowels & internal structure & closed class\\
\midrule
\multicolumn{2}{l}{\textsc{particles}} & max. 1 & no & yes \\
& \cayuga{i:ˀ}  ‘I, we’ \\
& \cayuga{ihs} ‘you’ & \\
& \cayuga{gę:s} ‘usually’& \\
& \cayuga{hniˀ}  ‘and’ & \\\midrule
\multicolumn{2}{l}{\textsc{nouns}} & min. 2 & yes & yes \\
& \cayuga{ganǫ́hsaˀ} ‘house’ & \\
& \cayuga{syaˀdáˀgeh} ‘on your body’ & \\
\midrule
\multicolumn{2}{l}{\textsc{verbs}} & min. 2 & yes & no\\
& \cayuga{í:ge:s} ‘I eat’ & \\
& \cayuga{godrę́:no:t} ‘she is singing’ & \\
\lspbottomrule
\end{tabular}
\end{table}

Particles have only one vowel, while verbs and nouns require at least two. Particles also have no internal structure (no \textsc{prefixes} or \textsc{suffixes}, for which see \sectref{Prefixes, suffixes, affixes, and stems}), while verbs and nouns do. (In addition, nouns have different prefixes and suffixes than verbs; see examples \ref{ex:nstr}, \ref{ex:vstr}, and Part \ref{Word Creation}). Finally, verbs form an open class, while the noun and particle classes are closed.\footnote{Open classes can grow over time, as speakers coin new words. In contrast, closed classes do not easily admit new words. For example, speakers will create new verbs to denote people, places, or things, instead of creating new nouns for this function.}

\ea\label{ex:nstr} a noun with internal structure (prefixes and suffixes)\\
ganǫhsa̱ˀsǫ́:ˀah\\
\gll ga-nǫhs-a̱ˀ-sǫ:ˀah\\
 \textsc{3s.a}-house-{\nounstemformer}-{\pluralizer}\\
\glt \trs{houses}
\z


\ea\label{ex:vstr} a verb with internal structure\\
dewage̱hęna̱ˀtraˀehsdǫ́hǫgyeˀ\\
\gll de-wage-hęnaˀtra-ˀe-hsd-ǫh-ǫgyeˀ\\
 {\dualic}-\textsc{1s.a}-knife-hit-{\causative}-{\stative}-{\progressive}\\
\glt `I am going along stabbing things.'
\z

A small group of \textsc{atypical} words do not neatly fall into the above categories. They are described in \sectref{Atypical words}.


\section{Word class versus speech function} \label{ch:Word class versus word function}
While it is true that Gayogo̱ho:nǫˀnéha:ˀ has just three word classes, two of these word classes perform many more speech \textit{functions}. The difference between speech function and word class is explained below.


The function of a word is often different from its class. To illustrate, let us take it for granted for now that \textit{degáhswa̱ˀne:t} ‘pie’ \xref{ex:halhaa} is a \textsc{verb}. Despite being a \textsc{verb}, \textit{degáhswa̱ˀne:t} functions as a “noun” in that it names an object in sentence \xref{ex:halhab} -- naming objects is one of the primary functions of nouns. 


\ea\label{ex:halha} a \textsc{verb} functioning as a “noun”
\ea degáhswa̱ˀne:t ‘pie’ (literally, ‘its upper back is doubled’ -- possibly referring to the pie plate)\label{ex:halhaa}
\ex \label{ex:halhab}
\gll Swá:yęˀ diˀ hne:ˀ sga̱hoˀdę́:ˀęh \exemph{degáhswa̱ˀne:t}? \\
you.have so in.fact some pie\\
\glt ‘Do yous have any kind of pie?’ (\cite[294]{mithun_watewayestanih_1984}, Dwade:kǫ́:nih dialogue)
\z
\z 

\tabref{tab:1:wdfun} summarizes the relationship between word class and speech function. While \textsc{nouns} only function as “nouns”, \textsc{verbs} and \textsc{particles} perform all of the remaining speech functions.

\begin{table}
\caption{word class versus word function}
\label{tab:1:wdfun}
\begin{tabularx}{\textwidth}{lQQ}
\lsptoprule
word class & speech function & word class and speech function\\
\midrule
\textsc{nouns} & “noun” & \textsc{noun} functioning as a “noun”\\
\midrule 
\textsc{verbs} & “verb”, “noun”, 

“pronoun”, “adjective”, 

“adverb” & \textsc{verb} functioning as a “verb”, “noun”, etc.\\
\midrule 
\textsc{particles} & “pronoun”, “adverb”, 

“conjunction”, 

“relativizer”, “degree word”, 

“question word”, “filler” or 

“fluency word” (discourse particle) & \textsc{particle} functioning as a “pronoun”, “adverb”, etc.\\
\lspbottomrule
\end{tabularx}
\end{table}


In this work, small caps denote word classes, and double quotes refer to speech functions --wherever the distinction is relevant. For example, \textit{degáhswa̱ˀne:t} ‘pie’ is a \textsc{verb} functioning as a “noun”, while \textit{ganǫ́hsaˀ} ‘house’ is a \textsc{noun} functioning as a “noun”.

Typically, speech functions are divided into major and minor ones (for reasons not relevant to this discussion). The five major speech functions of “noun”, “pronoun”, “adjective”, “adverb”, and “verb” are covered in the following chapters, while the minor speech functions in \figref{tab:1:wdfun} are covered in Part~\ref{Sentences} and Part~\ref{ch:Discourse and discourse signposts}. Also, since the following chapters are organized according to function (rather than word class), they often cover a range of items: for example, the section on “pronouns” describes anything functioning as a “pronoun”, including \textsc{verbs}, some \textsc{particles}, and the \textsc{pronominal prefixes}.

