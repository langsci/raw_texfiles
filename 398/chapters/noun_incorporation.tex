\chapter{Noun incorporation} \label{ch:Noun Incorporation}
Incorporated nouns appear after the post-pronominal prefixes and before the verb stem, \tabref{figtab:1:incnounposition}.
 		
\begin{table} [H]
\caption{Noun incorporation position\label{figtab:1:incnounposition}}
\begin{tabularx}{\textwidth}{@{}l|l|l|Q|l@{}}
% \lsptoprule
pre-pro\-nom\-i\-nal & pronominal & post-pro\-nom\-i\-nal & incorporated noun stem & verb stem\\
% \lspbottomrule
\end{tabularx}
\end{table}


The resulting words contain both a noun stem and a verb stem \xref{ex:nounincex}. In contrast, most other words include just one stem (noun or verb).\footnote{For examples of words with two verb stems, see E-verbs (\sectref{ch:E-verbs}) and the (reanalysed) \stem{(h)sˀ} \textsc{\eventuative} (\sectref{[-(h)sˀ] (the eventuative)}).}

\ea\label{ex:nounincex}
\ea ga̱hǫwadíhǫh\\
\gll ga̱-hǫw-adíh-ǫh\\
 \textsc{3s.a}-boat-lean-{\stative}\\
\glt `a leaning boat'
\cfex{ga̱hǫ́:waˀ\\
	\gll ga̱-hǫ́:w-aˀ\\
	\textsc{3s.a}-boat-{\nounstemformer}\\
	\glt `boat'}
\cfex{wadíhǫh\\
	\gll w-adíh-ǫh\\
	\textsc{3s.a}-lean-{\stative}\\
	\glt `it is leaning'}
\ex niyohahú:ˀuh\\
\gll ni-yo-hah-ú:ˀuh\\
 {\partitive}-\textsc{3s.p}-road-small.{\stative}\\
\glt `a small road'
\cfex{oháhaˀ\\
	\gll o-háh-aˀ\\
	\textsc{3s.p}-road-{\nounstemformer}\\
	\glt `road'}
\cfex{niwú:ˀuh\\
	\gll ni-w-ú:ˀuh\\
	{\partitive}-\textsc{3s.a}-small.{\stative}\\
	\glt `it is small'}
\z
\z

The meaning of words with noun incorporation is either transparent \xref{ex:nounincex2a} or fixed \xref{ex:nounincex2b}. (For \textit{transparent} versus \textit{literal}, see \sectref{Fixed and transparent expressions}.)

\ea\label{ex:nounincex2}
\ea agatgǫ̱hsóhaeˀ \\\label{ex:nounincex2a}
\gll a-g-at-gǫ̱hs-óhae-ˀ\\
{\factual}-\textsc{1s.a}-{\semireflexive}-face-wash-{\punctual}\\
\glt ‘I washed my face’ (transparent meaning)

\ex go̱ˀnigǫ̱háhetgęˀs\\\label{ex:nounincex2b}
\gll go̱-ˀnigǫ̱h-á-hetgę-ˀs\\
\textsc{3s.fi.p}-mind-{\joinerA}-ugly-{\habitual}\\
\glt ‘she is sobbing uncontrollably’ (literally, ‘she has a bad-turning mind’ or a ‘she is having a mind breakdown’)

\z
\z



\section{Nouns that can be incorporated} \label{ch:Nouns that can be incorporated}
Nouns that can be incorporated include basic nouns \xref{ex:nounincex4}, nouns with a  \stem{-tra}, \stem{-(h)sra} {\nominalizer} suffix \xref{ex:nounincex5}, and body part nouns \xref{ex:nounincex6}. The part that is incorporated is the stem (the noun minus any prefixes or suffixes) or a \textsc{stem-\nominalizer} combination \xref{ex:nounincex5}. (For the \textsc{nominalizer}, see \sectref{[-tr-aˀ] and [-(h)sr-aˀ]}.)

\ea\label{ex:nounincex4} incorporated basic nouns\\
ahawiyánę̱hsgoˀ\\
\gll a-ha-wiy-á-nę̱hsgo-ˀ\\
 {\factual}-\exsc{he.a}-offspring-{\joinerA}-steal-{\punctual}\\
\glt `he kidnapped a child'
\cfex{owí:yaˀ\\
	\gll o-wí:y-aˀ\\
	\textsc{3s.p}-offspring-{\nounstemformer}\\
	\glt `offspring'}
\cfex{ęhánę̱hsgoˀ\\
	\gll ę-há-nę̱hsgo-ˀ\\
	\fut-\exsc{he.a}-steal-{\punctual}\\
	\glt `he will steal'}
\z

\newpage
\ea\label{ex:nounincex5} incorporated nouns with \stem{-tra, -(h)sra} {\nominalizer}
\ea dehsehenaˀtraˀéhstaˀ\\
\gll de-hs-e-henaˀtra-ˀéhst-haˀ\\
 {\dualic}-\exsc{2s.a}-{\joinerE}-knife-stab-{\habitual}\\
\glt `you stab it repeatedly'
\cfex{ohéna̱ˀtraˀ ‘blade’,\\
	\gll o-hénaˀ-tr-aˀ\\
	\textsc{3s.p}-cut.up-{\nominalizer}-{\nsf}\\
	\glt `knife'}
\ex ga̱hyadǫ̱hsra:yé:nahs\\
\gll ga̱-hyadǫ̱hsra:-yé:na-hs\\
 \textsc{3s.a}-paper-hold.together-{\habitual}\\
\glt `paper clip'

\cfex{ga̱hyádǫ̱hsraˀ\\
\gll ga̱-hyádǫ̱-hsr-aˀ\\
\textsc{3s.a}-write-{\nominalizer}-{\nsf}\\
}
\z
\z

\ea\label{ex:nounincex6} incorporated body part nouns\\
hohsíˀdatgiˀ\\
\gll ho-hsíˀd-a-tgiˀ\\
 \textsc{3s.m.p}-foot-{\joinerA}-dirty.{\stative}\\
\glt `his feet are dirty'
\cfex{hahsíˀdaˀgeh\\
	\gll ha-hsíˀd-a̱ˀgeh\\
	\textsc{3s.m.a}-foot-{\on}\\
	\glt `on his foot'}
\cfex{otgiˀ\\
	\gll o-tgiˀ\\
	\textsc{3s.p}-dirty.{\stative}\\
	\glt `it is dirty or ugly'}
\z


Some loanwords can also be incorporated after the addition of the \stem{-tr} \textsc{\nominalizer} \xref{ex:nounincex301}. In contrast, some do not need the suffix \xref{ex:nounincex302}.

\ea\label{ex:nounincex301} incorporated loanwords\\
\ea gajobtrowá:nęh\\
\gll ga-job-tr-owá:n-ęh\\
 \textsc{3s.a}-job-{\nominalizer}-big-{\stative}\\
\glt `big job'
\ex gajobtrí:yo:\\
\gll ga-job-tr-í:yo:\\
 \textsc{3s.a}-job-{\nominalizer}-good.{\stative}\\
\glt `nice job'
\ex hojobtrí:yo:\\
\gll ho-job-tr-í:yo:\\
\textsc{3s.m.p}-job-{\nominalizer}-good.{\stative}\\
\glt ‘he has a good job’
\z
\z

\ea\label{ex:nounincex302} ęgadehórsnęht \\
\gll ę-g-ad-e-hórs-nęht \\
\fut-\textsc{1s.a}-{\semireflexive}-horse-dismount.{\zeropunctual}\\
\glt ‘I will get off a horse’ (humorous)
\z


\subsection{Placeholder incorporated nouns} \label{ch:Placeholder incorporated nouns}
Several obligatorily incorporating verbs require a \textsc{placeholder} incorporated noun when no other noun is incorporated. (In context, the placeholder noun stem means ‘it’, ‘thing’, or ‘something’.) For example, the verb \stem{+ohae} ‘to wash’ takes placeholder \stem{n} \xref{ex:incverbex9}.

\ea\label{ex:incverbex9} \stem{n+ohae}, \stem{+ohae} ‘to wash’\\
ęknóhaeˀ\\
\gll ę-k-n-óhae-ˀ\\
 \fut-\exsc{1s.a-noun}-wash-{\punctual}\\
\glt `I will wash it'
\cfex{ęgahjo̱háeˀ\\
	\gll ę-g-ahj-o̱háe-ˀ\\
	\fut-\textsc{1s.a}-hand-wash-{\punctual}\\
	\glt `I will wash my hands'}
\z


Examples of other placeholders are shown in \xxref{ex:incverbex10}{ex:incverbex14}. The type of placeholder noun is specific to each verb. For example, \stem{naˀ} is unique to \stem{+nawę:} \xref{ex:incverbex10}.

\ea\label{ex:incverbex10} \stem{naˀ+nawę:}, \stem{+nawę:} ‘to be wet, etc.’\\
onáˀnawę:\\
\gll o-náˀ-nawę:\\
 \textsc{3s.p}-\exsc{noun}-moist.{\stative}\\
\glt `it is wet, melted, moist'
\cfex{ohehdaná:wę:\\
	\gll o-hehd-a-ná:wę:\\
	\textsc{3s.p}-earth-{\joinerA}-moist.{\stative}\\
	\glt `wet, moist dirt, earth'}
\z


\ea\label{ex:incverbex11} \stem{aˀs+ęˀ}, \stem{+ęˀ} ‘to fall off, drop, reduce’\\
gyo̱ˀsę́ˀǫh\\
\gll g-yo̱-ˀs-ę́ˀ-ǫh\\
 {\cislocative}-\textsc{3s.p}-\exsc{noun}-fall-{\stative}\\
\glt `it has fallen off something'
\cfex{heyohnegę́ˀǫh\\
	\gll he-yo-hneg-ę́ˀ-ǫh\\
	{\translocative}-\textsc{3s.p}-water-fall-{\stative}\\
	\glt `falling water'}
\z


\ea\label{ex:incverbex12} \stem{d+ohag}, \stem{+ohag} ‘squeeze something’\\
dewakdo̱há:gǫh\\
\gll de-wak-d-o̱há:g-ǫh\\
 {\dualic}-\exsc{1s.p-noun}-squeeze-{\stative}\\
\glt `I am squeezing it'
\cfex{degahstǫdróha:s\\
\gll de-ga-hstǫdr-óha:g-s\\
{\dualic}-\textsc{3s.a}-straw-squeeze-{\habitual}\\
\glt ‘bailer’ (for hay, straw)
}
\z


\ea\label{ex:incverbex13} \stem{hsr+ǫni}, \stem{+ǫni} ‘create, make’\\
ęhsehsrǫ́:niˀ\\
\gll ę-hs-e-hsr-ǫ́:ni-ˀ\\
 \fut-\exsc{2s.a-joinerE-noun}-make-{\punctual}\\
\glt `you will create, make something'
\cfex{sa̱ˀahdrǫ́:nih\\
	\gll sa̱-ˀahdr-ǫ́:ni-h\\
	\exsc{2s.p}-basket-make-{\stative}\\
	\glt `you make baskets'}
\z



\ea\label{ex:incverbex14} \stem{hny+od}, \stem{+od} ‘to be standing’\\
gahnyo:t\\
\gll ga-hn-yo:t\\
 \textsc{3s.a}-\exsc{noun}-stand.{\stative}\\
\glt `it is standing'
\cfex{gajísdo:t\\
	\gll ga-jísd-o:t\\
	\textsc{3s.a}-light-stand.{\stative}\\
	\glt `a light'}
\z


\section{Nouns that cannot be incorporated} \label{ch:Nouns that cannot be incorporated}
Some nouns are not incorporable. As well, nouns are often not incorporated even when it is possible to do so (a situation known as \textsc{excorporation}). Both cases are described in the following sections.

\subsection{Nouns that are not incorporable} \label{ch:Nouns that are not incorporable}
Nouns that cannot be incorporated include compound nouns (\sectref{ch:Compound nouns}), instrumental nouns (\sectref{ch:Instrumental nouns}), and atypical nouns (\sectref{ch:Atypical nouns}). Compound nouns \xref{ex:nounincex8a} cannot be incorporated because they contain more than one noun stem. (Verbs typically only incorporate one noun stem at a time.) “Instrumental nouns” \xref{ex:nounincex8b} cannot be incorporated because they are \textsc{verbs}. Similarly, atypical nouns (such as animal names, \ref{ex:nounincex8c}) generally cannot be incorporated.\footnote{That being said, some atypical nouns can be incorporated once the \stem{tr-} {\nominalizer} is added \xref{ex:nounincex300}.

\ea\label{ex:nounincex300} ohsowastradáihę:ˀ \\
\gll o-hsowas-tr-a-dáihę:ˀ \\
\textsc{3s.p}-dog-{\nominalizer}-{\joinerA}-hot.{\stative}\\
\glt ‘hot dog’ (humorous calque)
\z} 

\ea\label{ex:nounincex8}
\ea gwihsgwíhs oˀwáhǫh ‘pig meat’, \trs{pork}, \trs{bacon} (compound noun)\label{ex:nounincex8a}
\ex ehyádǫ̱hkwaˀ ‘pencil’ (instrumental noun)\label{ex:nounincex8b}
\ex gwíhsgwihs ‘pig’ (animal name or atypical noun)\label{ex:nounincex8c}
\z
\z

Finally, \textsc{verbs} functioning as “nouns” cannot be incorporated. A two-word expression or compound noun is used instead (\ref{ex:cantincorp1}a, b). (Examples \ref{ex:cantincorp1}c--e illustrate the incorporation of various nouns into the verb \textit{gayę́:twęh}.)

\ea\label{ex:cantincorp1}
\ea ohnyǫhsgwaéˀ gayę́:twęh ‘cucumber field’
\cfex{ohnyǫhsgwáeˀ ‘cucumber’ (\textsc{verb} functioning as a “noun”)}
\ex ohyaga̱hǫ́ˀ gayę́:twęh ‘tomato field’
\cfex{ohyaga̱hǫˀ ‘tomato’ (\textsc{verb} functioning as a “noun”)}
\ex ga̱hstǫdrayę́:twęh ‘hay field’
\cfex{ohstǫ́:draˀ ‘hay’(basic noun)}
\ex ga̱hǫnadáyętwęh ‘potato field’
\cfex{ohǫ́na̱ˀdaˀ ‘potato’ (basic noun)}
\ex gayęˀgwayę́:twęh ‘tobacco field’
\cfex{ oyę́ˀgwaˀ ‘tobacco’ (basic noun)}
\z
\z


\subsection{Excorporation} \label{ch:Excorporation}
Nouns can remain unincorporated (a situation known as \textsc{excorporation}) for stylistic reasons (see \sectref{ch:Noun incorporation in discourse}). For example, while it is possible to incorporate the stem for \textit{gatgwę̱́ˀdaˀ} ‛wallet’, as in \xref{ex:nounincex90a}, the excorporated or independent noun instead appears at the end of the sentence for special effect (such as emphasis in \ref{ex:nounincex90b}).

\ea\label{ex:nounincex90}
\ea ǫgetgwę̱ˀdáhdǫ:ˀ\\\label{ex:nounincex90a}
\gll ǫg-e-tgwę̱ˀd-áhdǫ:-ˀ\\
 \exsc{\factual.1s.p}-{\joinerE}-wallet-lose-{\punctual} \\
\glt `I lost my wallet'
\cfex{gatgwę́ˀdaˀ\\
	\gll ga-tgwę̱ˀd-aˀ\\
	\textsc{3s.a}-{\joinerE}-wallet-{\nounstemformer}\\
	\glt `wallet'}
\ex \label{ex:nounincex90b}
\gll Ni: gyę:h tó-ne:ˀ ǫgahdǫ́:ˀ neˀ agétgwę̱ˀdaˀ. \\
I mind.you that-it.is I.lost.it the my.wallet\\
\glt ‘Mind you, I lost my wallet.’ (\cite[184]{mithun_watewayestanih_1984}, Gatgwęˀdaˀ dialogue)
\z
\z

\subsubsection*{Excorporation and obligatorily-incorporating verbs} \label{ch:Excorporation and obligatorily-incorporating verbs}
Excorporation creates a potential problem in the case of obligatorily-incorporating verbs like \stem{+iyo:} ‘good, nice’ (see \sectref{ch:Obligatorily incorporating verbs}). To meet the requirement for an incorporated noun, a kind of \emph{generic} noun stem can be incorporated instead. For example, the noun stem \stem{nahsgw} ‘domesticated animal’ substitutes for unincorporable \textit{só:wa:s} ‘dog’ in the verb \textit{ganahswí:yo:} ‘good pet’ in \xref{ex:nounincex10a}.

In such cases, both the incorporated generic noun stem within the verb \emph{and} the unincorporable noun outside the verb are required (\ref{ex:nounincex10}b, c).

\ea\label{ex:nounincex10}
\ea\label{ex:nounincex10a}
\gll Ganahsgwi:yó: neˀ só:wa:s.\\
it’s.a.good.pet the dog\\
\glt ‘The dog is a good pet.’
\cfex{\gll ga-nahsgw-i:yó:\\
\textsc{3s.a}-domestic.animal-good.{\stative}\\
}


\ex \label{ex:nounincex10b}
\gll So:wá:s aknáhsgwaęˀ.\\
dog I.have.a.pet\\
\glt ‘I have a dog’ 
\cfex{\gll ak-náhsgw-a-ęˀ\\
	    \textsc{1s.p}-domestic.animal-{\joinerA}-have.{\stative}\\
	    }

\ex degraheˀda:gé: ohó:draˀ\\\label{ex:nounincex10c}
\gll de-graheˀd-a:gé: ohó:draˀ \\
 {\dualic}-\exsc{ø.prefix}.tree-two.or.more.{\stative}  basswood\\
\glt `two basswood trees'

\cfex{sgra̱héˀda:t\\
	\gll s-gra̱héˀd-a:-t\\
	{\repetitive}-\exsc{ø.prefix}.tree-{\joinerA}-stand.{\stative}\\
	\glt `one tree'}

\cfex{gra̱he:t\\
	\gll gra̱he:t\\
	{\noprefix}.tree\\
	\glt `tree'}
\z
\z


\section{Verbs that cannot incorporate nouns} \label{ch:Verbs that cannot incorporate nouns}
Action or activity verbs can only incorporate when they express or require \textsc{patient} or \textsc{object} roles. (For \textsc{role}, see \sectref{Pronominal prefixes and role}.) Examples of verbs that cannot incorporate are provided in \xref{ex:nounincex12}. Such verbs express an  \textsc{agent} or “doer”, but not a patient. 

Since incorporated nouns typically express a patient role, they are absent from verbs that fail to express such a role.

\ea\label{ex:nounincex12}
\ea ga̱hnih\\
\gll ga̱-hni-h\\
 \textsc{3s.a}-bark-{\habitual}\\
\glt `it is barking'

\ex gadá:węh\\
\gll g-ad-á:wę-h\\
 \textsc{1s.a}-{\semireflexive}-swim-{\habitual}\\
\glt `I am swimming'

\ex saˀé:yǫˀ\\
\gll s-aˀ-é:-yǫ-ˀ\\
 {\repetitive}-{\factual}-\textsc{3s.fi.a}-arrive-{\punctual}\\
\glt `she returned'

\ex nǫdáhse:ˀ \\
\gll nǫdá-hs-e-:-ˀ\\
{\partitive.\cislocative.\factual}\exsc{-2s.a}-go-{\purposive}-{\punctual}\\
\glt ’you came from there’

\ex haˀgáhge:t\\
\gll haˀ-g-áhge:t\\
 {\translocative}-\textsc{1s.a}-stop.in.{\zeropunctual}\\
\glt `I stopped by there'

\ex agiˀ\\
\gll a-g-i-ˀ\\
 {\factual}-\textsc{1s.a}-say-{\punctual}\\
\glt `I said'
\z
\z



\section{Verbs that can incorporate nouns} \label{ch:Verbs that can incorporate nouns}
Verbs that can incorporate nouns include two-role verbs, and stative-only (adjectival) verbs, described next.

\subsection{Two-role verbs that can incorporate nouns} \label{ch:Two-role verbs that can incorporate nouns}
Two-role verbs that express a patient role can incorporate nouns. For example, the verb \textit{ęknóhaeˀ} ‘I will wash it’ is a two-role verb, requiring both an agent (someone doing the washing) and a patient (someone or thing being washed). It can incorporate nouns that describe what is being washed \xref{ex:incverbex}.
 
\ea\label{ex:incverbex}
\ea agatgǫ̱hsóhaeˀ\\
\gll a-g-at-gǫ̱hs-óhae-ˀ\\
 {\factual}-\textsc{1s.a}-{\semireflexive}-face-wash-{\punctual}\\
\glt `I washed my face'
\cfex{gegǫ́hsa̱ˀgeh\\
	\gll g-e-gǫ́hs-a̱ˀgeh\\
	\textsc{1s.a}-{\joinerE}-face-{\on}\\
	\glt `on my face'}
\ex agatnu̱ˀtsóhaeˀ\\
\gll a-g-at-nu̱ˀts-óhae-ˀ\\
 {\factual}-\textsc{1s.a}-{\semireflexive}-tooth-wash-{\punctual}\\
\glt `I brushed my teeth'
\cfex{knuˀtsáˀgeh\\
	\gll k-nuˀts-áˀgeh\\
	\textsc{1s.a}-tooth-{\on}\\
	\glt `on my teeth'}
 \clearpage
\ex agęhsi̱ˀdóhaeˀ\\
\gll a-g-ęhsi̱ˀd-óhae-ˀ\\
 {\factual}-\textsc{1s.a}-foot-wash-{\punctual}\\
\glt `I washed my feet'
\cfex{gę̱hsíˀda̱ˀgeh\\
	\gll g-ę̱hsíˀd-a̱ˀgeh\\
	\textsc{1s.a}-foot-{\on}\\
	\glt `on my foot'}
\z
\z

However, not all two-role verbs can incorporate a noun: for example, while the verb \stem{ahsha:} ‘to remember’ \xref{ex:incverbex2} expresses two roles (\textsc{1s>2s} in \ref{ex:incverbex2a}, and \textsc{2s.a}-{\semireflexive} in \ref{ex:incverbex2b}), there is no evidence that this verb ever incorporates a noun. The same goes for the verb in \xref{ex:incverbex3}. (For such verbs, the “second” role is a person or several people, not an object.)

\ea\label{ex:incverbex2}
\ea í:hs agǫ́ya̱hsha:ˀ\\\label{ex:incverbex2a}
\gll í:hs a-gǫ́y-a̱hsha:-ˀ\\
 you {\factual}-\exsc{1s>2s}-remember-{\punctual}\\
\glt `I thought of you'

\ex ętsadadáhshagwęˀ\\\label{ex:incverbex2b}
\gll ę-t-s-adad-áhshagw-ę-ˀ\\
 \fut-{\cislocative}-\textsc{2s.a}-{\reflexive}-remember-{\benefactive}-{\punctual}\\
\glt `you will remind yourself, make yourself remember'
\z
\z

\ea\label{ex:incverbex3} ęhéhsnǫdręˀ\\
\gll ę-héhs-hnǫdrę-ˀ\\
\fut-\textsc{2s:3ms}-follow-{\punctual}\\
\glt ‘you will follow him’
\z



\subsection{Stative-only verbs that can incorporate nouns} \label{ch:Stative-only verbs that can incorporate nouns}
Several types of stative-only verb, described next, can incorporate nouns.

\subsubsection*{Incorporating stative-only “adjectival” verbs} \label{ch:Incorporating adjectival (stative-only) verbs}
Stative-only “adjectival” verbs are a thematic (meaning-based) subclass of verbs describing typical states, qualities, or characteristics of people or objects. (For stative-only verbs, see \sectref{Verbs and aspect}.) Such verbs often incorporate the nouns they describe. Example \xref{ex:incverbex5} lists various stative-only “adjectival” verbs with the incorporated noun \stem{(a)hy} ‘fruit, berries’. (Pronominal prefix choice for such verbs is described in \sectref{Neuter stative-only verbs and pronominal prefix choice}.)

\ea\label{ex:incverbex5}
\ea ohyajiwá:gę:\\
\gll o-hy-a-jiwá:gę:\\
 \textsc{3s.p}-fruit-{\joinerA}-sour.{\stative}\\
\glt `tart, sour fruit'
\cfex{ojíwagę:\\
	\gll o-jíwagę:\\
	\textsc{3s.p}-sour.{\stative}\\
	\glt `it is sour, salty, bitter'}
\ex ohyága̱ˀǫh\\
\gll o-hy-á-ga̱ˀ-ǫh\\
 \textsc{3s.p}-fruit-{\joinerA}-good.tasting-{\stative}\\
\glt `good-tasting fruit'
\cfex{ogáˀǫh\\
	\gll o-gáˀ-ǫh\\
	\textsc{3s.p}-good.tasting-{\stative}\\
	\glt `it tastes good'}
\ex ohyówanęh\\
\gll o-hy-ówan-ęh\\
 \textsc{3s.p}-fruit-big-{\stative}\\
\glt `big fruit'
\cfex{gagówanęh\\
	\gll ga-gówan-ęh\\
	\textsc{3s.a}-big-{\stative}\\
	\glt `it is big'}
\ex niyohyú:ˀuh\\
\gll ni-yo-hy-ú:ˀuh\\
 {\partitive}-\textsc{3s.p}-fruit-small.{\stative}\\
\glt `small fruit'
\cfex{niwú:ˀuh\\
	\gll ni-w-ú:ˀuh\\
	{\partitive}-\textsc{3s.a}-small.{\stative}\\
	\glt `how small it is'}
\ex wahyaniyǫ́:taˀ\\
\gll w-ahy-a-niyǫ́:t-haˀ\\
 \textsc{3s.a}-fruit-{\joinerA}-hang-{\habitual}\\
\glt `hanging fruit'
\cfex{ganí:yǫ:t\\
	\gll ga-ní:yǫ:t\\
	\textsc{3s.a}-hang.{\stative}\\
	\glt `it is hanging'}
\ex ohyá:jih\\
\gll o-hy-á:-jih\\
 \textsc{3s.p}-fruit-{\joinerA}-dark.{\stative}\\
\glt `dark fruit'
\cfex{ga̱hǫ́ˀjih\\
	\gll ga̱-hǫ́ˀ-jih\\
	\textsc{3s.a}-{\noun}-dark.{\stative}\\
	\glt `it is dark'}
\ex ohyá:tgę:\\
\gll o-hy-á:-tgę:\\
 \textsc{3s.p}-fruit-{\joinerA}-rotten.{\stative}\\
\glt `spoiled, rotten fruit'
\cfex{otgę:\\
	\gll o-tgę:\\
	\textsc{3s.p}-rotten.{\stative}\\
	\glt `it is spoiled, rotten'}
\ex ohyáih\\
\gll o-hy-á-ih\\
 \textsc{3s.p}-fruit-{\joinerA}-ripe.{\stative}\\
\glt `ripe fruit'
\cfex{ęwáhyaiˀ\\
	\gll ę-w-áhy-a-i-ˀ\\
	\fut-\textsc{3s.a}-fruit-{\joinerA}-ripen-{\punctual}\\
	\glt `it will ripen'}
\z
\z


\subsubsection*{Incorporating stative-only “positional”  verbs} \label{ch:Incorporating positional (stative-only) verbs}
Stative-only “positional” verbs are a thematic (meaning-based) subclass of verb describing the typical position or location of an object. Many positional verbs can incorporate nouns \xref{ex:incverbex4}. (Pronominal prefix choice for such verbs is described in \sectref{Neuter stative-only verbs and pronominal prefix choice}.)

\ea\label{ex:incverbex4}
\ea oˀgráęˀ \\
\gll o-ˀgr-á-ęˀ\\
 \textsc{3s.p}-snow-lie.{\stative}\\
\glt ‘snow lying there’, `fallen snow'
\cfex{oˀgraˀ\\
	\gll o-ˀgr-aˀ\\
	\textsc{3s.p}-snow-{\nounstemformer}\\
	\glt `snow'}
\cfex{nigá:yęˀ\\
	\gll ni-gá:-yęˀ\\
	{\partitive}-\textsc{3s.a}-lie.{\stative}\\
	\glt `where it is lying'}
\ex gayęhsradę́hda:ˀ\\
\gll ga-yęhsr-a-dę́hda:ˀ\\
 \textsc{3s.a}-blanket-{\joinerA}-lie.spread.out.on.ground.{\stative}\\
\glt `blanket lying spread out'
\cfex{gayę́hsraˀ\\
	\gll ga-yę́hsr-aˀ\\
	\textsc{3s.a}-blanket-{\nounstemformer}\\
	\glt `blanket'}
\cfex{gadę́hda:ˀ\\
	\gll ga-dę́hda:ˀ\\
	\textsc{3s.a}-lie.spread.out.on.ground.{\stative}\\
	\glt `it is lying spread out'}
\ex ohǫna̱ˀdáhsi̱ha:ˀ\\
\gll o-hǫna̱ˀd-á-hsi̱ha:ˀ\\
 \textsc{3s.p}-potato-{\joinerA}-stand.in.group.{\stative}\\
\glt `potatoes standing in a group'
\cfex{ohǫ́na̱ˀdaˀ\\
	\gll o-hǫ́na̱ˀd-aˀ\\
	\textsc{3s.p}-potato-{\nounstemformer}\\
	\glt `potato'}
\cfex{gáehsi̱ha:ˀ\\
	\gll gáe-hsi̱ha:ˀ\\
	\textsc{3ns.fi.a}-stand.in.a.group.{\stative}\\
	\glt `they are standing in a group or clump'}
\z
\z


\subsection{Obligatorily-incorporating verbs} \label{ch:Obligatorily incorporating verbs}
Some stative-only “adjectival” verbs \xref{ex:incverbex6}, some stative-only “positional” verbs \xref{ex:incverbex7}, and some two-role verbs \xref{ex:incverbex8} require an incorporated noun. A plus ‘+’ sign denotes an obligatorily-incorporating verb. (Some obligatorily-incorporating verbs require a placeholder noun, if no other noun is incorporated, see \sectref{ch:Placeholder incorporated nouns}.)

\ea\label{ex:incverbex6} \stem{+iyo:} ‘to be good, nice’
\ea agǫgwe̱ˀdí:yo:\\
\gll ag-ǫgwe̱ˀd-í:yo:\\
 \textsc{3s.fi.p}-person-nice.{\stative}\\
\glt `nice person'
\ex gaihoˀdęhsrí:yo:\\
\gll ga-ihoˀdęhsr-í:yo:\\
 \textsc{3s.a}-work-nice.{\stative}\\
\glt `nice work'
\z
\z

\ea\label{ex:incverbex7} \stem{+od} ‘to stand’
\ea ga̱há:do:t\\
\gll ga̱-há:d-o:t\\
 \textsc{3s.a}-forest-stand.{\stative}\\
\glt `a forest'
\ex odrǫ́hyo:t\\
\gll o-drǫ́hy-o:t\\
 \textsc{3s.p}-ray.of.light-stand.{\stative}\\
\glt `sunbeam, ray of light, sunshine'
\ex ga̱hnyo:t\\
\gll ga̱-hny-o:t\\
 \textsc{3s.a}-stick/placeholder.noun-stand.{\stative}\\
\glt `it is standing'
\z
\z

\ea\label{ex:incverbex8} [+ohae) ‘to wash’
\ea agatgǫ̱hsóhaeˀ\\
\gll a-g-at-gǫ̱hs-óhae-ˀ\\
 {\factual}-\textsc{1s.a}-{\semireflexive}-face-wash-{\punctual}\\
\glt `I washed my face'
\ex agęhsi̱ˀdóhaeˀ\\
\gll a-g-ęhsi̱ˀd-óhae-ˀ\\
 {\factual}-\textsc{1s.a}-foot-wash-{\punctual}\\
\glt `I washed my feet'
\ex agenoháeˀ\\
\gll a-g-e-n-ohae-ˀ\\
 {\factual}-\textsc{1s.a}-{\joinerE}-placeholder.noun-wash-{\punctual}\\
\glt `I washed (it)'
\z
\z
