\chapter{\textsc{e-verbs}} \label{ch:E-verbs}
\textsc{e-verbs} are verbs which all have the verb \stem{e} ‘go’ in common. (For a list, see \sectref{List of verbs of motion}.) The classification system proposed here is new, but is based on \citet{michelson_ontario_2011} and \citet{sasse_far_1998}. 

Four main types of e-verb will be described: the \textsc{simple motion verb} just consists of the verb [e] ‘to go, move’ \xref{ex:motverbex}. In contrast, \textsc{complex motion verbs} are words with two \textsc{stems}, the second of which is \stem{e} ‘go’ \xref{ex:motverbex2}. The remaining two types of e-verb also consist of two stems: \textsc{dislocative e-verbs} include a verb stem, a \textsc{dislocative} suffix and \stem{e} ‘go’ \xref{ex:motverbex3}. \textsc{Progressive} verbs consist of a verb stem, the \stem{-gy} \textsc{progressive}, and \stem{e} ‘go’ \xref{ex:motverbex4}. E-verbs also have a unique aspect form, the \stem{-:} \textsc{purposive}, which is described in the following section. (Also see \textit{Vowel length suffixes} \sectref{[-:] (vowel length suffixes)}.)

\ea\label{ex:motverbex} simple motion verb \stem{e}\\
í:geˀ\\
\gll í:-g-e-ˀ\\
 {\prothetic}-\textsc{1s.a}-go-{\aspect}\\
\glt `I am walking, moving'
\z


\ea\label{ex:motverbex2} complex motion verb \stem{verb/stem-e}\\
hǫwáhnǫdreˀ\\
\gll hǫwá-hnǫdr-e-ˀ\\
 \exsc{3ms/3fis:3ms}-follow-go-{\aspect}\\
\glt `someone is following him'
\z




\ea\label{ex:motverbex3} dislocative e-verb \stem{verb-{\dislocative}-e}\\
agékdǫ̱hneˀ\\
\gll agé-kdǫ̱-hn-e-ˀ\\
 \textsc{1s.p}-examine-{\dislocative}-go-{\aspect}\\
\glt `I am going to see it'
\z


\newpage
\ea\label{ex:motverbex4} progressive verb \stem{verb-{\progressive}-e}\\
hǫwahnǫdrá:gyeˀ\\
\gll hǫwa-hnǫdr-á:-gy-e-ˀ\\
 \exsc{3ms/3fis:3ms}-follow-{\joinerA}-{\progressive}-go-{\aspect}\\
\glt `someone is following him along'
\z



\section{The purposive} \label{ch:The purposive}
E-verbs are unique in allowing an additional type of aspect form, the \stem{-:} \textsc{purposive} ({\purposive}) or \textsc{long-e} form. In this work, e-verbs without the \stem{-:} {\purposive} suffix will be called \textsc{short-e} forms, while e-verbs with the \stem{-:} {\purposive} will be called \textsc{long-e} forms. Long-e forms are unique to e-verbs.

As shown in \tabref{figtab:1:purposiveasp} (page \pageref{figtab:1:purposiveasp}), short-e verbs take \stem{e-ˀs} go-{\habitual}, \stem{e-ˀ} go-{\punctual}, or  \stem{e-ˀ} go-{\stative} combinations. (The punctual forms also require a mood prefix.)

Long-e forms take \stem{e-:-ˀ} go-{\purposive}-{\punctual} or \stem{e-:-ˀ} go-{\purposive}-{\stative} combinations. (The punctual forms also require a mood prefix). No long-e habitual forms are attested.


E-verbs are described in the following sections. Pronominal prefix choice for e-verbs is described in \sectref{ch:E-verbs and pronominal prefix choice}.


\begin{table}
\caption{Short-e and long-e verbs}
\label{figtab:1:purposiveasp}
{
\begin{tabularx}{\textwidth}{QlccClQ}
\lsptoprule
& mood & pron\-ominal & \stem{e} & {\purposive} \mbox{or \textsc{long-e}} & aspect & post-aspect\\
\midrule
\textsc{\habitual} &  & \textsc{a} & \stem{e} &  & \stem{-ˀs} \textsc{\habitual}& \\
\tablevspace
\textsc{\habitual} {\past} &  & \textsc{a} & \stem{e} &  & \stem{-ˀs} \textsc{\habitual} & \mbox{\stem{-gęhę:ˀ}} \textsc{\past}\\
\tablevspace
\textsc{\future} {\habitual} & \stem{ę-} \textsc{\future} & \textsc{a} & \stem{e} &  & \stem{-ˀs} \textsc{\habitual} & \mbox{\stem{-:k, -hk}} \mbox{\textsc{\modalizer}}\\
\tablevspace
\textsc{short-e} \textsc{\factual} {\punctual} & \stem{aˀ-} \textsc{\factual} & \textsc{a} & \stem{e} &  & \stem{-ˀ} \textsc{\punctual} & \\
\tablevspace
\textsc{short-e} \textsc{\future} {\punctual} & \stem{ę-} \textsc{\future}  & \textsc{a} & \stem{e} &  & \stem{-ˀ} \textsc{\punctual} & \\
\tablevspace
\textsc{short-e} \textsc{\indefinite} {\punctual}&  \stem{a:-} \textsc{\indefinite} & \textsc{a} & \stem{e} &  & \stem{-ˀ} \textsc{\punctual} & \\
\tablevspace
\textsc{short-e} \textsc{\stative} &  &  & \stem{e} &  & \stem{-ˀ} \textsc{\stative} & \\
\tablevspace
\textsc{long-e} {\factual} {\punctual} &  \stem{aˀ-} \textsc{\factual} & \textsc{a} & \stem{e} & \stem{-:} & \stem{-ˀ} \textsc{\punctual} & \\
\tablevspace
\textsc{long-e} {\future} {\punctual} &  \stem{ę-} \textsc{\future} & \textsc{a} & \stem{e} & \stem{-:} & \stem{-ˀ} \textsc{\punctual} & \\
\tablevspace
\textsc{long-e} {\indefinite} {\punctual} &  \stem{a:-} \textsc{\indefinite} & \textsc{a} & \stem{e} & \stem{-:} & \stem{-ˀ} \textsc{\punctual} & \\
\tablevspace
\textsc{long-e} {\stative} &  &  & \stem{e} & \stem{-:} & \stem{-ˀ} \textsc{\stative} & \\
\lspbottomrule
\end{tabularx}}
\end{table}


\section{Simple motion verb \stem{e} ‘go’} \label{ch:Simple motion verb [e] ‘go’}
The simple motion verb \stem{-e} ‘go’ denotes movement towards or away from a place of origin. More specific meanings are determined by the verb’s prepronominal prefixes. For example, \stem{-e} with the \stem{haˀ-} \textsc{\translocative} means ‘go away’ \xref{ex:simplemotexa}, while \stem{-e} with the \stem{d-} \textsc{\cislocative} means ‘come towards’ \xref{ex:simplemotexb}. The meanings contributed by the prepronominal prefixes are described next in \sectref{ch:Prepronominal prefixes with [e] ‘go’}, and the aspect forms for simple motion verbs are described in \sectref{ch:Simple motion verb in the habitual} - \sectref{simple motion no-aspect}.

\ea\label{ex:simplemotex} 
\ea haˀgeˀ\\\label{ex:simplemotexa}
\gll haˀ-g-e-ˀ\\
 {\translocative}-\exsc{1s.a}-go-{\aspect}\\
\glt `I am going there'


\ex dá:geˀ, dageˀ \\\label{ex:simplemotexb}
\gll d-á:-g-e-ˀ\\
 {\cislocative}-{\factual}-\textsc{1s.a}-go-{\aspect}\\
\glt `I am coming'
\z
\z



\subsection{Prepronominal prefixes with \stem{e} ‘go’} \label{ch:Prepronominal prefixes with [e] ‘go’}
The combinations of prepronominal prefixes and \stem{-e} ‘go’ are illustrated in examples \xxref{ex:ppe1}{ex:ppe9}. (The special meanings of the habitual forms are explained in the \sectref{ch:Simple motion verb in the habitual}.)

\ea\label{ex:ppe1} \stem{e} ‘go, move, be someplace’
\ea iheˀs\\
\gll i-h-e-ˀs\\
 {\prothetic}-\textsc{3s.m.a}-go-{\habitual}\\
\glt `he is here'
\ex ęhsne:ˀ\\
\gll ę-hsn-e-:-ˀ\\
 \fut-\exsc{2d.a}-go-{\purposive}-{\punctual}\\
\glt `you two will go together'
\ex í:geˀ\\
\gll í:-g-e-ˀ\\
 {\prothetic}-\textsc{1s.a}-go-{\stative}\\
\glt `I am walking, moving'
\z
\z

\ea\label{ex:ppe2} \stem{\cislocative-…e} \trs{come here or this way}, \trs{be here}\\
dagę́:neˀ\\
\gll da-gę́:n-e-ˀ\\
 {\cislocative}-\textsc{3p.a}-go-{\stative}\\
\glt `they are coming'
\z


\ea\label{ex:ppe3} \stem{ę: tsǫ: \cislocative-…e} ‘wander’, ‘be on a certain side’
\ea ę tsǫ: itseˀs \\
\gll ę tsǫ: i-t-s-e-ˀs\\
side just {\prothetic}-{\cislocative}-\textsc{2s.a}-go-{\habitual}\\
\glt ‘you wander (all the time)’, `you are over there'
\ex ę tsǫ: ętseˀ\\
\gll ę-t-s-e-ˀ\\
 \fut-{\cislocative}-\textsc{2s.a}-go-{\punctual}\\
\glt `you will wander'
\ex ę tsǫ: itseˀ\\
\gll i-t-s-e-ˀ\\
 {\prothetic}-{\cislocative}-\textsc{2s.a}-go-{\stative}\\
\glt `you are wandering' (right now)
\z
\z

\ea\label{ex:ppe4} \stem{gaǫ \partitive-\cislocative-…e} ‘come this way’, ‘be (from) here’\\
gaoˀ nǫdáhse:ˀ come this way! \\
\gll gaoˀ nǫdá-hs-e-:-ˀ\\
 somewhere {\partitivecislocative}\exsc{-2s.a}-go-{\purposive}-{\punctual}\\
\glt `Come this way!'
\z


\ea\label{ex:ppe5} \stem{\partitive-\cislocative-e} ‘come from there’, ‘be from there’\\
nǫdá:ge:ˀ \\
\gll nǫdá:-g-e-:-ˀ\\
 {\partitivecislocative}\exsc{-1s.a}-go-{\purposive}-{\punctual}\\
\glt ‘I come from, `I came from'
\z


\ea\label{ex:ppe6} \stem{\partitive-\dualic-\cislocative-…e} ‘come back’, ‘return to here from some place’\\
naˀdę́:tge:ˀ\\
\gll naˀdę́:t-g-e-:-ˀ\\
 {\partitive.\dualic.\future.\cislocative}\exsc{-1s.a}-go-{\purposive}-{\punctual}\\
\glt `I will come back over here, return'
\z


\ea\label{ex:ppe7} \stem{\repetitive-…e} ‘go back’, ‘return home’, ‘be back’
\ea ihsgeˀs\\
\gll i-hs-g-e-ˀs\\
 {\prothetic}-{\repetitive}-\textsc{1s.a}-go-{\habitual}\\
\glt `I have returned home'
\ex ihsgeˀ \\
\gll i-hs-g-e-ˀ\\
 {\prothetic}-{\repetitive}-\textsc{1s.a}-go-{\stative}\\
\glt ‘I am going back’, ‘I am on my way back’, `I am back (from where I came)'
\z
\z

\ea\label{ex:ppe8} \stem{\translocative-…e} ‘go over there’, ‘be over there’
\ea haˀge:ˀ\\
\gll haˀ-g-e-:-ˀ\\
 {\translocativefactual}\exsc{-1s.a}-go-{\purposive}-{\punctual}\\
\glt `I went there'
\ex haˀgeˀ\\
\gll haˀ-g-e-:-ˀ\\
 {\translocativefactual}\exsc{-1s.a}-go-{\punctual}\\
\glt `I am going (there)'
\z
\z

\ea\label{ex:ppe9} \stem{\translocative-\repetitive-…e} ‘go back there’, ‘return there’\\
hęhsge:ˀ\\
\gll hęhs-g-e-:-ˀ\\
 {\translocative.\future.\repetitive}\exsc{-1s.a}-go-{\purposive}-{\punctual}\\
\glt `I am going back there'
\z


\subsection{Simple motion verb \stem{e} ‘go’ in the habitual} \label{ch:Simple motion verb in the habitual}
The \stem{e-ˀs} go-{\habitual}  form of \stem{e} ‘go’ means ‘to exist’ or ‘to be somewhere’ \xref{ex:simplemotex3}. (The meanings contributed by the prepronominal prefixes were described in \sectref{ch:Prepronominal prefixes with [e] ‘go’}.)

\ea\label{ex:simplemotex3} \stem{e-ˀs} ‘to exist’
\ea iheˀs\\
\gll i-h-e-ˀs\\
 {\prothetic}-\textsc{3s.m.a}-go-{\habitual}\\
\glt `he is here'
\ex ę: tsǫ: itseˀs \\
\gll ę: tsǫ i-t-s-e-ˀs\\
side just {\prothetic}-{\cislocative}-\textsc{2s.a}-go-{\habitual}\\
\glt ‘you wander’ (all the time), `you are over there'
\ex gę́:neˀs,\\
\gll gę́:n-e-ˀs\\
 \textsc{3p.a}-go-{\habitual}\\
\glt  ‘they are around’, ‘they are here’, `they are together'
\ex tgę́:neˀs\\
\gll t-gę́:n-e-ˀs\\
 {\cislocative}-\textsc{3p.a}-go-{\habitual}\\
\glt `they are over there'
\ex itgeˀs\\
\gll i-t-g-e-ˀs\\
 {\prothetic}-{\cislocative}-\textsc{1s.a}-go-{\habitual}\\
\glt `I am here'
\ex heheˀs\\
\gll he-h-e-ˀs\\
 {\translocative}-\textsc{3s.m.a}-go-{\habitual}\\
\glt `he is there'
\ex isgeˀs\\
\gll i-s-g-e-ˀs\\
 {\prothetic}-{\repetitive}-\textsc{1s.a}-go-{\habitual}\\
\glt `I have returned home'
\ex tigáęˀs\\
\gll ti-gáę-ˀs\\
 {\contrastive}-\textsc{3s.fi.a}.go-{\habitual}\\
\glt `they are roaming about'
\ex tí:weˀs\\
\gll tí:-w-e-ˀs\\
 {\contrastive}-\textsc{3s.a}-go-{\habitual}\\
\glt `a stray (animal)'
\z
\z

Variations of \stem{e-ˀs} go-{\habitual} include an \textsc{habitual past} form \xref{ex:simplemotex4}, and a \textsc{future habitual} (\ref{ex:simplemotex5}, see \tabref{figtab:1:purposiveasp}, page \pageref{figtab:1:purposiveasp}). An \textsc{indefinite habitual} is theoretically possible but not attested \xref{ex:simplemotex6}. (For background information, see \textit{Variations on the habitual aspect} \sectref{Variations on the habitual aspect}. Also, for pronominal prefix choice, see \sectref{ch:Simple and complex motion verb prefix choice})

\ea\label{ex:simplemotex4} \stem{…e-ˀs-gęhę:ˀ} go-{\habitual}-{\past} ‘to have existed somewhere’\\
heˀsgę́hę:ˀ\\
\gll h-e-ˀs-gę́hę:ˀ\\
 \textsc{3s.m.a}-go-{\habitual}-{\past}\\
\glt ‘he was away’, `he has been there'\\
\z


\ea\label{ex:simplemotex5} \stem{ę-…e-ˀs-e:k} \fut-\exsc{…}go-{\habitual}-{\modalizer} ‘will exist somewhere’\\
hęhéˀse:k\\
\gll h-ę-h-é-ˀs-e:k\\
 \exsc{\translocative-\future-3s.m.a}-go-{\habitual}-{\modalizer}\\
\glt `he will be there'
\cfex{iheˀs\\
	\gll i-h-e-ˀs\\
	{\prothetic}-\textsc{3s.m.a}-go-{\habitual}\\
	\glt `he is here'}
\z


\ea\label{ex:simplemotex6} hypothesized \stem{a:-…e-ˀs-e:k} {\indefinite}-\exsc{…}go-{\habitual}-{\modalizer} ‘might exist somewhere’
\z


\subsection{Simple motion verb \stem{e} ‘go’ in the punctual} \label{ch:Simple motion verb in the punctual}
The short \stem{e-ˀ} go-{\punctual} and long \stem{e-:-ˀ} go-{\purposive}-{\punctual} forms of \stem{e} ‘go’ require one of the three mood prefixes (the \stem{aˀ-} \textsc{\factual}, \stem{ę-} \textsc{\future}, or \stem{a:-} \textsc{\indefinite}, see \tabref{figtab:1:purposiveasp}, page \pageref{figtab:1:purposiveasp}). The resulting punctual forms and no-aspect verbs are described next. (For background information, see \textit{No-aspect and punctual aspect verbs}, \sectref{No-aspect and punctual aspect verbs}.)


\subsubsection*{Simple motion verb \stem{e} ‘go’ with \stem{aˀ-} \textsc{\factual}}
The punctual short-e \stem{e-ˀ} and long-e \stem{e-:-ˀ} forms can occur with the \stem{aˀ-} \textsc{\factual} prefix. Such verbs have either present \stem{e-ˀ} or past \stem{e-:-ˀ} meanings or \textsc{readings} \xxref{ex:simplemotex9}{ex:simplemotex12}.

\ea\label{ex:simplemotex9} 
\ea haˀgeˀ\\
\gll haˀ-g-e-ˀ\\
{\translocativefactual}\exsc{-1s.a}-go-{\punctual}\\
\glt ‘I am going there’ (present reading)

\ex haˀge:ˀ\\
\gll haˀ-g-e-:-ˀ\\
{\translocativefactual}\exsc{-1s.a}-go-{\purposive}-{\punctual}\\
\glt ‘I went there’ (past reading)
\z
\z

\ea\label{ex:simplemotex10}
\ea dá:geˀ, dageˀ\\
\gll d-a-g-e-ˀ\\
{\cislocative}-{\factual}-\textsc{1s.a}-go-{\punctual}\footnotemark\\
\glt ‘I am coming’ (present reading)
\footnotetext{The first vowel in /da:geˀ/ is long because of a rule that lengthens A if it is the first vowel in a two-syllable word.}

\ex dá:ge:ˀ\\
\gll d-a:-g-e-:-ˀ\\
{\cislocative}-{\factual}-\textsc{1s.a}-go-{\purposive}-{\punctual}\\
\glt ‘I came’ (past reading)\\
\z
\z

\ea\label{ex:simplemotex11} \stem{aˀ-…eˀ} (present reading)
\ea dagę́:neˀ\\
\gll da-gę́:n-e-ˀ\\
 \exsc{\cislocative.\factual-3ns.a}-go-{\punctual}\\
\glt `they are coming'

\ex dǫdáheˀ\\
\gll dǫdá-h-e-ˀ\\
 {\dualic.\factual.\cislocative}\exsc{-3s.m.a}-go-{\punctual}\\
\glt `he is coming back'

\ex haˀseˀ\\
\gll haˀ-s-e-ˀ\\
 {\translocative.\factual}\exsc{-2s.a}-go-{\punctual}\\
\glt `you are going'

\ex haˀgę́:neˀ\\
\gll haˀ-gę́:n-e-ˀ\\
 {\translocative.\factual}\exsc{-3ns.a}-go-{\punctual}\\
\glt `they are going'

\ex haˀgeˀ\\
\gll haˀ-g-e-ˀ\\
{\translocative.\factual}\exsc{-1s.a}-go-{\punctual}\\
\glt ‘I am going’ (there)\\

\ex haˀgáęˀ\\
\gll haˀ-gáę-ˀ\\
 {\translocative.\factual}\exsc{-3ns.fi.a}-go-{\punctual}\footnotemark{}\\
\glt `they are going'
\footnotetext{In this example and the following one, the E of the pronominal prefix merges with the \stem{e} ‘go’ verb, resulting in [ę].}

\ex dǫdáęˀ\\
\gll dǫdá-ę-ˀ\\
 {\dualic.\factual.\cislocative}\exsc{-3s.fi.a}.go-{\punctual}\\
\glt `she is coming back'
\z
\z

\ea\label{ex:simplemotex12} \stem{aˀ-…e:ˀ} (past reading)
\ea edwe:ˀ\\
\gll e-dw-e-:-ˀ\\
 {\factual}-\exsc{1p.in.a}-go-{\purposive}-{\punctual}\\
\glt `we all went together'

\ex haˀá:kne:ˀ\\
\gll haˀ-á:kn-e-:-ˀ\\
 {\factual}-\exsc{1d.in.a}-go-{\purposive}-{\punctual}\\
\glt `we two went'

\ex to haˀge:ˀ\\
\gll to haˀ-g-e-:-ˀ\\
there {\factual}-\textsc{1s.a}-go-{\purposive}-{\punctual}\\
\glt ‘I went’

\ex hǫsá:kne:ˀ\\
\gll hǫsá:-kn-e-:-ˀ\\
{\translocative.\factual.\repetitive}\exsc{-1d.in.a}-go-{\purposive}-{\punctual}\\
\glt `we two went back'

\ex to hǫsá:ge:ˀ\\
\gll to hǫsá:-g-e-:-ˀ\\
there {\translocative.\factual.\repetitive}\exsc{-1s.a}-go-{\purposive}-{\punctual}\\
\glt `I did go again'

\ex nǫdá:kne:ˀ\\
\gll nǫdá:-kn-e-:-ˀ\\
 {\partitive.\cislocative.\factual}\exsc{-1d.in.a}-go-{\purposive}-{\punctual}\\
\glt `where we two came from'

\ex nǫdá:ge:ˀ \\
\gll nǫdá:-g-e-:-ˀ\\
 {\partitive.\cislocative.\factual}\exsc{-1s.a}-go-{\purposive}-{\punctual}\\
\glt ‘I come from’, `I came from'
\z
\z

\subsubsection*{Simple motion verb \stem{e} ‘go’ with \stem{ę-} \textsc{\future}}
The punctual short-e \stem{e-ˀ} and long-e \stem{e-:-ˀ} forms can occur with the \stem{ę-} \textsc{\future} prefix (see \tabref{figtab:1:purposiveasp}, p. \pageref{figtab:1:purposiveasp}). The difference in meaning between short-e \xref{ex:simplemotex13} and long-e \xref{ex:simplemotex14} forms is unclear out of context. However, the long-e forms do not always have a past tense reading, except in context (see \ref{ex:simplemotex140}). Perhaps such forms would describe a \emph{certain} past event.

\ea\label{ex:simplemotex13} \stem{ę-…e-ˀ} 
\ea ę: tsǫ: ętseˀ\\
\gll ę: tsǫ: ę-t-s-e-ˀ\\
side just \fut-{\cislocative}-\textsc{2s.a}-go-{\punctual}\\
\glt ‘you will wander’
\ex ęteˀ\\
\gll ę-t-h-e-ˀ\\
 \fut-{\cislocative}-\textsc{3s.m.a}-go-{\punctual}\\
\glt `he will come this way'
\ex dętgeˀ \\
\gll d-ę-t-g-e-ˀ\\
 {\dualic}-{\future}-{\cislocative}-\textsc{1s.a}-go-{\punctual}\\
\glt ‘I will come this way’, ‘I will come back’, `I am coming back'
\ex ętgeˀ\\
\gll ę-t-g-e-ˀ\\
 \fut-{\cislocative}-\textsc{1s.a}-go-{\punctual}\\
\glt `I will come'
\ex ętgaęˀ\\
\gll ę-t-gaę-ˀ\\
 \fut-{\cislocative}-\textsc{3s.fi.a}.go-{\punctual}\\
\glt `they will come'
\ex dęgyá:kneˀ\\
\gll d-ę-g-yá:kn-e-ˀ\\
 {\dualic}-{\future}-{\cislocative}-\exsc{1d.in.a}-go-{\punctual}\\
\glt `we two will come back'
\ex hę́:geˀ\\
\gll hę́:-g-e-ˀ\\
 {\translocative}-{\future}-\textsc{1s.a}-go-{\punctual}\\
\glt `I will go'
\z
\z

\ea\label{ex:simplemotex14} /ę - e:ˀ/
\ea hę́:ge:ˀ\\
\gll hę́:-g-e-:-ˀ\\
 {\translocativefuture}\exsc{-1s.a}-go-{\purposive}-{\punctual}\\
\glt `I will go there'
\ex naˀdę́:tge:ˀ\\
\gll naˀdę́:t-g-e-:-ˀ\\
 {\partitive.\dualic.\future.\cislocative}\exsc{-1s.a}-go-{\purposive}-{\punctual}\\
\glt `I will come back over here, return'
\ex hęhsge:ˀ\\
\gll hęhs-g-e-:-ˀ\\
 {\translocative.\future.\repetitive}\exsc{-1s.a}-go-{\purposive}-{\punctual}\\
\glt `I am going back there'
\ex to e: hęhsge:ˀ\\
\gll to e: hęhs-g-e-:-ˀ\\
there again {\translocative.\future.\repetitive}\exsc{-1s.a}-go-{\purposive}-{\punctual}\\
\glt `I will go again'
\newpage
\ex ędwe:ˀ\\
\gll ę-dw-e-:-ˀ\\
 \fut-\exsc{1p.in.a}-go-{\purposive}-{\punctual}\\
\glt `we will go together'
\ex ęhsne:ˀ\\
\gll ę-hsn-e-:-ˀ\\
 \fut-\exsc{2d.a}-go-{\purposive}-{\punctual} \\
\glt `you two will go together'
\z
\z

\ea\label{ex:simplemotex140} 
\gll Gyotgǫ́:t gę:s neˀ tgá:gǫ:t hęháha:ˀ neˀ gaęgwaˀ hǫ́: hęhé:ˀ. \\
always usually the it.is.necessary he.will.take.it.there the whichever place/time he.went.there \\
\glt ‘He always had to take it with him whenever he went someplace.’ (\cite{henry_de_2005})
\cfex{hęhe:ˀ\\
\gll h-ę-h-e-:-ˀ\\
{\translocative}-{\future}-\textsc{3s.m.a}-go-{\purposive}-{\punctual}\\
\glt `he will go there'}
\z

\subsubsection*{Simple motion verb \stem{e} ‘go’ with \stem{a:-} \textsc{\indefinite}}
The punctual short-e \stem{e-ˀ} and long-e \stem{e-:-ˀ} forms can also occur with the \stem{a:-} \textsc{\indefinite} prefix (see \tabref{figtab:1:purposiveasp}, page \pageref{figtab:1:purposiveasp}). The difference in meaning between the short-e \xref{ex:simplemotex15} and long-e \xref{ex:simplemotex16} forms is unclear out of context. However, the long-e forms do not appear to have a past tense reading. Perhaps such forms would describe a \emph{possible} past event.

\ea\label{ex:simplemotex15} \stem{a: - eˀ}
\ea aǫdá:geˀ\\
\gll aǫdá:-g-e-ˀ\\
 {\indefinitecislocative}\exsc{-1s.a}-go-{\punctual}\\
\glt `I should come'
\ex naˀdǫ́:da̱heˀ\\
\gll naˀdǫ́:da̱-h-e-ˀ\\
 {\partitive.\dualic.\indefinite.\cislocative}\exsc{-3s.m.a}-go-{\punctual}\\
\glt `he would come this way'
\newpage
\ex dǫdá:kneˀ\\
\gll dǫdá:-kn-e-ˀ\\
 {\dualic.\indefinite.\cislocative}\exsc{-1d.in.a}-go-{\punctual}\\
\glt `we two would come back'
\z
\z

\ea\label{ex:simplemotex16} \stem{a: - e:ˀ}
\ea há:ge:ˀ\\
\gll há:-g-e-:-ˀ\\
 {\translocativeindefinite}\exsc{-1s.a}-go-{\purposive}-{\punctual}\\
\glt `I should go there'
\ex to e: hǫsá:ge:ˀ\\
\gll to e: hǫsá:-g-e-:-ˀ\\
there again {\translocative.\repetitive.\indefinite}\exsc{-1s.a}-go-{\purposive}-{\punctual}\\
\glt `I would go again'
\ex á:yakne:ˀ\\
\gll á:-yakn-e-:-ˀ\\
 {\indefinite}-\exsc{1d.in.a}-go-{\purposive}-{\punctual}\\
\glt `we two would go together'
\z
\z

\subsubsection*{Simple motion verb \stem{e} ‘go’, \textsc{no-aspect} forms}\label{simple motion no-aspect}
The verb \stem{e} ‘go’ can also take mood prefixes without a \stem{-ˀ} \textsc{\punctual} suffix (with both short-e \stem{e-ø} and long-e \stem{e-:-ø} forms). Such no-aspect verbs (\sectref{Meaning of no-aspect verbs}) function as suggestions \xref{ex:simplemotex600a}, commands \xref{ex:simplemotex600b}, hypotheticals \xref{ex:simplemotex600c}, or negatives (\ref{ex:simplemotex600d}, see \sectref{Commands (imperatives) and suggestions} and \sectref{ch:Negation}.)

\ea\label{ex:simplemotex600} 
\ea hé:ge:\\\label{ex:simplemotex600a} 
\gll hé:-g-e-:\\
 {\translocative}-\textsc{1s.a}-go-{\purposive}.{\noaspect}\\
\glt `let me go there'
\ex dǫdáhseh\\\label{ex:simplemotex600b} 
\gll dǫdá-hs-e-h\\
 {\dualiccislocative}\exsc{-2s.a}-go-\textsc{euph.h}/{\noaspect}\footnotemark{}\\
\glt `come back!⁠'
\footnotetext{Tom Deer, p.c.}
\newpage
\ex to há:ge:\\\label{ex:simplemotex600c} 
\gll to há:-g-e-:\\
 there {\translocativeindefinite}\exsc{-1s.a}-go-{\purposive}.{\noaspect}\\
\glt `where I might go'
\ex tęˀ to de̱ˀá:ge:\\\label{ex:simplemotex600d} 
\gll tęˀ to de̱ˀ-á:-g-e-:\\
not there \exsc{\negative-\indefinite-1s.a}-go-{\purposive}.{\noaspect}\\
\glt `I am not going'
\z
\z


\subsection{Simple motion verb \stem{e} ‘go’ in the stative, with short-e} \label{ch:Simple motion verb [e] ‘go’ in the stative, with short [e-ˀ]}
The stative short-e \stem{e-ˀ} form does not take mood prefixes (see \sectref{ch:Simple motion verb in the punctual}). Examples are shown in \xref{ex:simplemotex17}. They have a present-tense reading.

\ea\label{ex:simplemotex17} \stem{e-ˀ} {\stative} (no {\mood} prefixes)
\ea isgeˀ \\
\gll i-s-g-e-ˀ\\
{\prothetic}-{\repetitive}-\textsc{1s.a}-go-{\stative}\\
\glt ‘I am going back’, ‘I am on my way back’, ‘I am back’ (from where I came)

\ex í:geˀ\\
\gll í:-g-e-ˀ\\
 {\prothetic}-\textsc{1s.a}-go-{\stative}\\
\glt `I am walking, moving'

\ex ę: tsǫ: itseˀ \\
\gll ę: tsǫ: i-t-s-e-ˀ\\
side just {\prothetic}-{\cislocative}-\exsc{you}-go-{\stative}\\
\glt ‘you are wandering’ (right now) 

\ex ohnaˀgę́:ˀ ihseˀ\\
\gll ohnaˀgę́:ˀ i-hs-e-ˀ\\
late {\prothetic}-\textsc{2s.a}-go-{\stative}\\
\glt ‘you are late again’ (said at the moment)
\z
\z

Long-e \stem{e-:-ˀ} {\stative} forms with the simple motion verb \stem{e} are unattested. However, both short-e \stem{e-ˀ} and long-e \stem{e-:-ˀ} stative forms of \emph{complex} motion verbs do exist, as shown in (\ref{ex:complexstative1}, also see \sectref{ch:Complex motion verbs in the stative aspect}). The long-e stative forms have past-tense readings.

\ea\label{ex:complexstative1} short-e \stem{e-ˀ} and long-e \stem{e-:-ˀ} stative forms of complex motion verbs
\ea\label{ex:complexstative1a} agéhseˀ\\
\gll agé-hs-e-ˀ\\
 \textsc{1s.p}-lower.back-go-{\stative} \\
\glt `I am riding'
\ex\label{ex:complexstative1b} agéhse:ˀ \\
\gll agé-hs-e-:-ˀ\\
 \textsc{1s.p}-lower.back-go-{\purposive}-{\stative} \\
\glt ‘I rode’, `I came riding'
\z
\z

\section{Complex motion verbs with \stem{e} ‘go’} \label{ch:Complex motion verbs}
Complex motion verbs are words with two (sometimes three) stems, the last of which is always \stem{e} ‘go’ \xref{ex:complmotex}. The element preceding \stem{e} is often a verb stem \xref{ex:complmotex}. (However, it is sometimes an incorporated noun, see example \ref{ex:complmotex20}. For pronominal prefix choice, see \sectref{ch:Simple and complex motion verb prefix choice}.) 

Complex motion verbs describe more specific means of moving (or existing) than the simple motion verb described in \sectref{ch:Simple motion verb [e] ‘go’}.

\ea\label{ex:complmotex} \stem{…verb.stem-e…}
\ea krę́:deˀ\\
\gll k-hrę́:d-e-ˀ\\
 \textsc{1s.a}-lead-go-{\stative}\\
\glt `I lead, go in front'
\ex kdákseˀ\\
\gll k-dáks-e-ˀ\\
 \textsc{1s.a}-run-go-{\stative}\\
\glt `I am running'
\ex gehsreˀ\\
\gll ge-hsr-e-ˀ\\
 \textsc{1s.a}-follow-go-{\stative}\\
\glt `I am chasing it'
\ex agéˀdreˀ\\
\gll agé-ˀdr-e-ˀ\\
 \textsc{1s.a}-ride-go-{\stative}\\
\glt `I am riding along'
\ex gyená:greˀ \\
\gll g-ye-ná:gr-e-ˀ\\
 {\cislocative}-\textsc{3s.fi.a}-dwell-go-{\stative}\\
\glt ‘she lives, stays, dwells over there’, `people live over there'
\ex grá:deˀ\\
\gll g-rá:d-e-ˀ\\
 \textsc{1s.a}-climb-go-{\stative}\\
\glt `I climb'
\ex hǫwáhnǫdreˀ\\
\gll hǫwá-hnǫdr-e-ˀ\\
 \exsc{3ms/3fis:3ms}-follow-go-{\stative}\\
\glt `someone is following him'

\ex deyagodáwęnyeˀ\\
\gll de-yago-d-áwęny-e-ˀ\\
 {\dualic}-\textsc{3s.fi.p}-{\semireflexive}-stir-go-{\stative}\\
\glt `she is walking'
\ex hadíhsreˀ\\
\gll had-íhsr-e-ˀ\\
 \textsc{3ns.m.a}-follow-go-{\stative}\\
\glt `they follow, chase'
\z
\z

\ea\label{ex:complmotex20} \stem{…incorporated.noun-e…}
\ea gatahí:neˀ\\
\gll g-at-hah-í:n-e-ˀ\\
 \textsc{1s.a}-{\semireflexive}-road-lead-go-{\stative}\\
\glt `I am walking'
\ex degagyohsgwí:neˀ\\
\gll de-g-ag-yohsgw-í:n-e-ˀ\\
 {\dualic}-\textsc{1s.a}-{\semireflexive}-elbow-lead-go-{\stative}\\
\glt `I am crawling'
\ex gǫnę:tsí:neˀ\\
\gll gǫ-nę:ts-í:n-e-ˀ\\
 \textsc{1s:2s}-arm-lead-go-{\stative}\\
\glt `I am leading you by the hand'
\newpage
\ex hohseˀ\\
\gll ho-hs-e-ˀ\\
 \textsc{3s.m.p}-lower.back-go-{\stative}\\
\glt `he is riding a horse'
\ex niyó:weˀ, nyo:weˀ \\
\gll ni-yó:-w-e-ˀ\\
 {\partitive}-\textsc{3s.p}-distance/amount-go-{\stative}\\
\glt ‘how far (distance)’, `to be a certain distance'
\z
\z

The verb \stem{hawi} ‘to carry’ resembles the other complex motion verbs in structure and meaning, except that its final vowel is [i] instead of [e] (\ref{ex:complmotex2}, \cite{sasse_far_1998}).


\ea\label{ex:complmotex2} \stem{haw-i} ‘to carry’\\
ká:wiˀ\\
\gll k-há:w-i-ˀ\\
 \textsc{1s.a}-hold-go-{\stative}\\
\glt `I am carrying it'
\z


The aspect forms for complex motion verbs are described in the following sections. 


\subsection{Complex motion verbs with \stem{e} ‘go’ in the habitual} \label{ch:Complex motion verbs in the habitual aspect}
Complex motion verbs take an \stem{-ˀs} \textsc{\habitual} suffix, which carries the typical range of habitual meanings \xref{ex:complmotex3}. (For comparison, stative forms, which have a present-tense reading, are also shown in example \ref{ex:complmotex3}. See \sectref{ch:Simple and complex motion verb prefix choice} for information about pronominal prefix choice.)

\ea\label{ex:complmotex3}
\ea agéhseˀs\\
\gll agé-hs-e-ˀs\\
 \textsc{1s.p}-lower.back-go-{\habitual}\\
\glt `I habitually ride'
\cfex{agéhseˀ\\
	\gll agé-hs-e-ˀ\\
	\textsc{1s.p}-lower.back-go-{\stative}\\
	\glt `I am riding'}

\ex gegyeˀs\\
\gll ge-gy-e-ˀs\\
 \textsc{1s.a}-fly-go-{\habitual}\\
\glt `I habitually fly'
\cfex{gegyeˀ\\
	\gll ge-gy-e-ˀ\\
	\textsc{1s.a}-fly-go-{\stative}\\
	\glt `I am flying'}
\ex kdakseˀs \\
\gll k-daks-e-ˀs\\
 \textsc{1s.a}-run-go-{\habitual}\\
 \glt ‘I run’ (several times)
\cfex{kdakseˀ\\
	\gll k-daks-e-ˀ\\
	\textsc{1s.a}-run-go-{\stative}\\
	\glt `I am running'}
\ex ge̱hsreˀs\\
\gll ge-hsr-e-ˀs\\
 \textsc{1s.a}-follow-go-{\habitual}\\
\glt `I habitually chase it'
\cfex{ge̱hsreˀ\\
	\gll ge-hsr-e-ˀ\\
	\textsc{1s.a}-follow-go-{\habitual}\\
	\glt `I am chasing it'}
\ex gǫnę:tsí:neˀs, gǫnę:tsí:nehs\\
\gll gǫ-nę:ts-í:n-e-ˀs/hs\\
 \textsc{1s:2s}-arm-lead-go-{\habitual}\\
\glt `I always take you by the hand'
\cfex{gǫnę:tsí:neˀ\\
	\gll gǫ-nę:ts-í:n-eˀ\\
	\textsc{1s:2s}-arm-lead-go-{\stative}\\
	\glt `I am leading you by the hand'}
\ex ká:wiˀs \\
\gll k-há:wi-ˀs\\ 
 \textsc{1s.a}-carry-go-{\habitual}\\
 \glt ‘I carry it’ (habitually, frequently)
\cfex{ká:wiˀ\\
	\gll k-há:wi-ˀ\\
	\textsc{1s.a}-carry-{\habitual}\\
	\glt `I am carrying it'}
\z
\z


\subsection{Complex motion verbs with \stem{e} ‘go’ in the punctual} \label{ch:Complex motion verbs in the punctual aspect}
Complex motion verbs take a punctual short \stem{e-ˀ} form, along with a mood prefix (such as the \textsc{\future}, shown in \ref{ex:complmotex6}).

\ea\label{ex:complmotex6}
\ea ęgǫnętsí:neˀ\\
\gll ę-gǫ-nęts-í:n-e-ˀ\\
 \fut-\textsc{1s:2s}-arm-lead-go-{\punctual}\\
\glt `I will lead you by the hand'
\ex ęká:wiˀ\\
\gll ę-k-há:wi-ˀ\\
 \fut-\textsc{1s.a}-carry-{\punctual}\footnotemark{}\\
 \footnotetext{Recall that \stem{hawi} ‘carry’ is like other \textsc{complex motion verbs}, but ends with [i] instead of [e].}
\glt `I will be carrying it'
\z
\z

The unusual punctual complex motion verbs in \xref{ex:complmotex7} take an \stem{ę-} \textsc{\future} prefix, and also  an \stem{-hk} \textsc{\former} suffix. Meanwhile, the \textsc{\former} suffix typically only appears with habitual or stative verbs. Its presence in the following examples possibly serves the same function as the \stem{-:k} \textsc{\modalizer}, namely, enabling the verb to take a mood prefix such as the \stem{ę-} \textsc{\future}.

\ea\label{ex:complmotex7} \stem{ę-…e-hk} {\future}-…go-{\former}
\ea ękrę́:dehk\\
\gll ę-k-hrę́:d-e-hk\\
 \fut-\textsc{1s.a}-lead-go-{\former}\\
\glt `I will lead, go in front'
\cfex{krę́:dehk\\
	\gll k-hrę́:d-e-hk\\
	\textsc{1s.a}-lead-go-{\former}\\
	\glt `I led, went in front'}
\ex ęgrá:dehk\\
\gll ę-g-rá:d-e-hk\\
 \fut-\textsc{1s.a}-climb-go-{\former}\\
\glt `I will climb'
\cfex{grá:dehk\\
	\gll g-rá:d-e-hk\\
	\textsc{1s.a}-climb-go-{\former}\\
	\glt `I climbed'}
\z
\z

\subsection{Complex motion verbs with \stem{e} ‘go’ in the stative} \label{ch:Complex motion verbs in the stative aspect}
Complex motion verbs in the stative aspect end with a short-e \stem{e-ˀ} go-{\stative} or long-e \stem{e-:-ˀ} go-{\purposive}-{\stative} combination. The short-e ones have a present tense reading, while long-e ones have a past tense reading \xref{ex:complmotex4}.

\ea\label{ex:complmotex4} short-e \stem{e-ˀ} go-{\stative} and long-e \stem{e-:-ˀ} go-{\purposive}-{\stative}
\ea 
\ea agéhseˀ\\
\gll agé-hs-e-ˀ\\
 \textsc{1s.p}-lower.back-go-{\stative} \\
\glt `I am riding'

  \ex agéhse:ˀ\\
    \gll agé-hs-e-:-ˀ\\
    \textsc{1s.p}-lower.back-go-{\purposive}-{\stative}\\
    \glt ‘I rode’, ‘I came riding’
\z
\ex 
\ea gegyeˀ\\
\gll ge-gy-e-ˀ\\
 \textsc{1s.a}-fly-go-{\stative}\\
\glt `I am flying'
\ex gegye:ˀ\\
    \gll ge-gy-e-:-ˀ\\
    \textsc{1s.a}-fly-go-{\purposive}-{\stative}\\
    \glt ‘I came flying’
\z

\ex 
\ea kdakseˀ\\
\gll k-daks-e-ˀ\\
 \textsc{1s.a}-run-go-{\stative}\\
\glt `I am running'
\ex kdakse:ˀ\\ 
    \gll k-daks-e-:- \\
    \textsc{1s.a}-run-go-{\purposive}-{\stative}\\
    \glt ‘I was running’
\z
\ex 
\ea ge̱hsreˀ\\
\gll ge̱-hsr-e-ˀ\\
 \textsc{1s.a}-follow-go-{\stative}\\
\glt `I am chasing it'
\ex ge̱hsre:ˀ\\
    \gll ge-hsr-e-:-ˀ\\
    \textsc{1s.a}-follow-go-{\purposive}-{\stative}\\
    \glt ‘I was chasing it’
\z 
\newpage
\ex 
\ea gatahí:neˀ\\
\gll g-at-hah-í:n-e-ˀ\\
 \textsc{1s.a}-{\semireflexive}-road-lead-go-{\stative}\\
\glt `I am walking'
\ex gatahí:ne:ˀ\\
    \gll g-at-hah-í:n-e-:-ˀ\\
    \textsc{1s.a}-{\semireflexive}-road-lead-go-{\purposive}-{\stative}\\
    \glt ‘I was walking’
\z 
\ex 
\ea degagyo̱hsgwí:neˀ\\
\gll de-g-ag-yo̱hsgw-í:n-e-ˀ\\
 {\dualic}-\textsc{1s.a}-{\semireflexive}-elbow-follow-go-{\stative}\\
\glt `I am crawling'
\ex degagyo̱hsgwí:ne:ˀ \\
    \gll de-g-ag-yo̱hsgw-í:n-e-:-ˀ\\
    {\dualic}-\textsc{1s.a}-{\semireflexive}-elbow-follow-go-{\purposive}-{\stative}\\
    \glt ‘I was crawling’
\z 
\ex 
\ea krę́:deˀ\\
\gll k-hrę́:d-e-ˀ\\
 \textsc{1s.a}-lead-go-{\stative}\\
\glt `I lead, go in front'
\ex krę́:de:ˀ\\ 
    \gll k-hrę́:d-e-:-ˀ\\ 
    \textsc{1s.a}-lead-go-{\purposive}-{\stative}\\
    \glt ‘I lead, went in front’
\z 
\ex 
\ea grá:deˀ\\
\gll g-rá:d-e-ˀ\\
 \textsc{1s.a}-climb-go-{\stative}\\
\glt `I climb'
\ex grá:de:ˀ\\
    \gll g-rá:d-e-:-ˀ\\
    \textsc{1s.a}-climb-go-{\purposive}-{\stative}\\
    \glt ‘I climbed’
\z 
\ex 
\ea ká:wiˀ \\
\gll k-há:w-i-ˀ\\
\textsc{1s.a}-carry-go-{\stative}\\
\glt ‘I am carrying it’
\ex ká:wi:ˀ\\ 
    \gll k-há:w-i-:-ˀ\\
    \textsc{1s.a}-carry-go-{\purposive}-{\stative}\\
    \glt ‘I was carrying it’
\z 
\z
\z

Several stative forms also take the \stem{e-hk} go-{\former} combination \xref{ex:complmotex5}. The difference in meaning between verbs with the \stem{e-hk} go-{\former} and \stem{e-:-ˀ} go-{\purposive}-{\stative} combinations is unclear.

\ea\label{ex:complmotex5}
\ea krę́:dehk\\
\gll k-hrę́:d-e-hk\\
 \textsc{1s.a}-lead-go-{\former}\\
\glt `I led, went in front'
\ex krę́:de:ˀ\\ 
    \gll k-hrę́:d-e-:-ˀ\\
    \textsc{1s.a}-lead-go-{\purposive}-{\stative}\\
    \glt ‘I led, went in front’
\z 
\ex 
\ea grá:dehk\\
\gll g-rá:d-e-hk\\
 \textsc{1s.a}-climb-go-{\former}\\
\glt `I climbed'
\ex grá:de:ˀ\\ 
    \gll g-rá:d-e-:-ˀ\\
    \textsc{1s.a}-climb-go-{\purposive}-{\stative}\\
    \glt ‘I climbed’
\z 
\z


Several complex motion verbs in the stative can either take \stem{e-hk} go-{\former} \xref{ex:complmotex25a} or \stem{e-ˀ-gęhę:ˀ} go-{\stative}-{\past} \xref{ex:complmotex25b}. Again, the difference in meaning is unclear. (See \sectref{ch:Simple and complex motion verb prefix choice} for information about pronominal prefix choice.)

\ea\label{ex:complmotex25}
\ea tgáenagrehk\\\label{ex:complmotex25a}
\gll t-gáe-nagre-hk\\
 {\cislocative}-\textsc{3s.fi.a}-live-{\former}\\
\glt `they did or used to live there'
\ex tgaenagréˀgę̱hę:ˀ\\\label{ex:complmotex25b}
\gll t-gae-nagré-ˀ-gę̱hę:ˀ\\
 {\cislocative}-\textsc{3s.fi.a}-live-{\stative}-{\past}\\
\glt `they used to live there'
\cfex{tgáenagreˀ\\
	\gll t-gáe-nagre-ˀ\\
	{\cislocative}-\textsc{3s.fi.a}-live-{\stative}\\
	\glt `they live there'}
\z
\z




\section{Dislocative e-verbs} \label{ch:Dislocative-[e] verbs}
Dislocative e-verbs consist of a \stem{verb.stem-{\dislocative}} combination, followed by \stem{e} ‘go’. Dislocative e-verbs are different from \emph{plain} dislocative verbs, which have a dislocative suffix but lack the \stem{e} ‘go’ element (see \sectref{Dislocative suffixes [-hs, -hn] etc}).

The dislocative suffix denotes movement from one location to another, and also “reinforces” the meaning of \stem{e} ‘go’, to convey purpose or intention.  Example \xref{ex:displurpsufex} illustrates a dislocative e-verb, and for comparison, a plain dislocative verb and a regular verb.

\ea\label{ex:displurpsufex}
agékdǫ̱hneˀ\\
\gll a-gé-kdǫ̱-hn-e-ˀ\\
 {\factual}-\textsc{1s.a}-examine-{\dislocative}-go-{\stative}\\
\glt `I am going to see it'
\cfex{ęgékdǫ̱hnaˀ\\
	\gll ę-gé-kdǫ̱-hn-aˀ\\
	\fut-\textsc{1s.a}-examine-{\dislocative}-{\punctual}\\
	\glt `I will go see'}
\cfex{ęgékdǫ:ˀ\\
	\gll ę-gé-kdǫ-:ˀ\\
	\fut-\textsc{1s.a}-examine-{\punctual}\\
	\glt `I will see, look'}
\z



The aspect forms for dislocative e-verbs are summarized in \tabref{figtab:1:dislpurp}, p. \pageref{figtab:1:dislpurp}. Dislocative e-verbs take the same types of pronominal prefix as the regular verbs they are based on (see \sectref{ch:Dislocative E-verb prefix choice}).

\begin{table}
\caption{Dislocative e-verbs}
\label{figtab:1:dislpurp}
{
\begin{tabularx}{\textwidth}{XXXXXXX}
\lsptoprule
    name & mood & pron & \textsc{verb-disl} & \stem{e} & aspect & post-aspect \\
    \midrule
    {\dislocative} e-verb, {\habitual} & &  & \textsc{verb-\dislocative} & \stem{e} &  \stem{-ˀs} {\habitual} & \\
     \midrule
     {\dislocative} e-verb, {\habitual} {\past} & &  & \textsc{verb-\dislocative} & \stem{e} &  \stem{-ˀs} {\habitual} & \stem{-gęhę:ˀ} {\past}\\
    \midrule
    {\factual}-{\punctual} {\dislocative} e-verb & \stem{aˀ-} {\factual}  &  & \textsc{verb-\dislocative} & \stem{e} &  \stem{-ˀ} {\punctual} & \\
    \midrule
    {\future}-{\punctual} {\dislocative} e-verb & \stem{ę-} {\future} &  & \textsc{verb-\dislocative} & \stem{e} & \stem{-ø} {\punctual} & \stem{-:k} {\modalizer}\\
     \midrule
     {\indefinite}-{\punctual} {\dislocative} e-verb & \stem{a:-} 
     
     {\indefinite} &  & \textsc{verb-\dislocative} & \stem{e} & \stem{-ø} {\punctual} & \stem{-:k} {\modalizer}\\
      \midrule
      {\dislocative} \textsc{short-e} {\stative} & &  & \textsc{verb-\dislocative} & \stem{e} &  \stem{-ˀ} {\stative} & \\
     \midrule
     {\dislocative} \textsc{long-e} {\stative} & &  & \textsc{verb-\dislocative} & \stem{e-:} 
     
     go-{\purposive} & \stem{-ˀ} {\stative}  & \\
     \lspbottomrule
\end{tabularx}}
\end{table}





\subsection{Dislocative e-verbs in the habitual} \label{ch:Dislocative-[e] verbs in the habitual}
Dislocative e-verbs can take a short-e \stem{e-ˀs} go-{\habitual} combination, with the usual range of habitual meanings (\ref{ex:displurpsufex5}, see \tabref{figtab:1:dislhab}, p. \pageref{figtab:1:dislhab}). As with other habitual forms, such verbs can also take the \stem{-gęhę:ˀ} \textsc{\past} suffix (example \ref{ex:displurpsufex6}, \tabref{figtab:1:dislhab}). (For pronominal prefix choice, see \sectref{ch:Dislocative E-verb prefix choice}.)

\newpage
\ea\label{ex:displurpsufex5} 
gadáhnyo̱hneˀs\\
\gll g-ad-áhny-o̱-hn-e-ˀs\\
\textsc{1s.a}-{\semireflexive}-hook-put.in.water-{\dislocative}-go-{\habitual}\\
\glt ‘to fish continuously’
\z
	
\ea\label{ex:displurpsufex6} 
hatrǫnya̱hnéˀsgę̱hę:ˀ\\
\gll h-at-hrǫny-a̱hn-é-ˀs-gę̱hę:ˀ\\
\textsc{3s.m.a}-{\semireflexive}-tell-{\dislocative}-go-{\habitual}-{\past}\\
\glt ‘he used to tell’
\z

\begin{table}
\caption{Dislocative e-verb, habitual forms}
\label{figtab:1:dislhab}
{
\begin{tabularx} {\textwidth}{X|l|l|c|l|l}
% \lsptoprule
& pron & \textsc{verb-{\dislocative}} & \stem{e} & aspect & post-aspect \\
\hline
{\dislocative} e-verb, {\habitual} &  & \textsc{verb-{\dislocative}} & \stem{e} & \stem{-ˀs} \textsc{{\habitual}}\strut & \\
{\dislocative} e-verb, {\habitual} {\past} &  & \textsc{verb-{\dislocative}} & \stem{e} & \stem{-ˀs} {\habitual} & \stem{-gęhę:ˀ} \textsc{\past}\\
% \lspbottomrule
\end{tabularx}}
\end{table}



\subsection{Dislocative e-verbs in the punctual} \label{ch:Dislocative-[e] verbs in the punctual}
Dislocative e-verbs verbs can take a short-e \stem{e-ˀ} go-{\punctual} combination and a mood prefix (either the \stem{aˀ-} \textsc{\factual}, \stem{ę-} \textsc{\future}, or \stem{a:-} \textsc{\indefinite}, see \tabref{figtab:1:dislpunc}, p. \pageref{figtab:1:dislpunc}). A verb with the \textsc{\factual} prefix is shown in \xref{ex:displurpsufex88}, page \pageref{ex:displurpsufex88}. (For pronominal prefix choice, see \sectref{ch:Dislocative E-verb prefix choice}.)

\ea\label{ex:displurpsufex88}
ahęnadáhnyo̱hneˀ\\
\gll a-hęn-ad-áhny-o̱-hn-e-ˀ\\
 {\factual}-\textsc{3ns.m.a}-{\semireflexive}-hook-put.in.water-{\dislocative}-go-{\punctual}\\
\glt  ‘they are going fishing’, `are they going fishing?'
\cfex{ahęnadáhnyohnaˀ\\
	\gll a-hęn-ad-áhny-o̱-hn-aˀ\\
	{\factual}-\textsc{3ns.m.a}-{\semireflexive}-hook-put.in.water-go-{\punctual}\\
	\glt `they went fishing'}
\cfex{ahęnadáhnyo:ˀ\\
	\gll a-hęn-ad-áhny-o-:ˀ\\
	{\factual}-\textsc{3ns.m.a}-{\semireflexive}-hook-put.in.water-{\punctual}\\
	\glt `they fished'}
\z


As summarized in \tabref{figtab:1:dislpunc}, the \stem{ę-} \textsc{\future} and \stem{a:-} \textsc{\indefinite} forms require a \stem{-:k} \textsc{\modalizer} suffix. (Factual forms cannot co-occur with the modalizer).

\begin{table}
\caption{Dislocative e-verb, punctual forms}
\label{figtab:1:dislpunc}
{
\begin{tabularx} {\textwidth}{l|X|l|l|c|X}
% \lsptoprule
& mood & pron & \textsc{verb-disl} & \stem{e} & aspect \\
\hline
{\factual}-{\punctual} {\dislocative} e-verb  &  \stem{aˀ-} \newline
\textsc{\factual} &  & \textsc{verb-\dislocative} & \stem{e} & \stem{-ˀ} \newline \textsc{\punctual}\\
\hline
{\future}-{\punctual} {\dislocative} e-verb &  \stem{ę-} \newline
\textsc{\future} &  & \textsc{verb-\dislocative} & \stem{e} & \stem{-:k} \newline 
\textsc{\punctual.\modalizer}\\
\hline
{\indefinite}-{\punctual} {\dislocative} e-verb &  \stem{a:-} \newline
\textsc{\indefinite} &  & \textsc{verb-\dislocative}  & \stem{e} & \stem{-:k} \newline
\textsc{\punctual.\modalizer}\\
% \lspbottomrule
\end{tabularx}}
\end{table}



\subsubsection*{\stem{aˀ-…\textsc{verb-disl}-e-ˀ}}
Three-aspect punctual forms of dislocative e-verbs can take the \stem{aˀ-} \textsc{\factual}. Such verbs have an intentional future reading \xref{ex:dislpurpfacex}.

\ea\label{ex:dislpurpfacex} 
\ea agatrǫnyáhneˀ\\
\gll a-g-at-hrǫny-á-hn-e-ˀ\\
 {\factual}-\textsc{1s.a}-{\semireflexive}-tell-{\dislocative}-go-{\punctual}\\
\glt `I am going to go there and tell'

\ex agékdǫ̱hneˀ\\
\gll a-gé-kdǫ̱-hn-e-ˀ\\
 {\factual}-\textsc{1s.a}-examine-{\dislocative}-go-{\punctual}\\
\glt `I am going to see it'

\ex agaesá:keˀ \\
\gll a-gae-ihsák-h-e-ˀ\\
{\factual}-\textsc{3s.fi.a}-seek-{\dislocative}-go-{\punctual}\\
\glt ‘they are going to look for it’, ‘are they going to look for it?’

\ex ahayętwáhseˀ ‘he is going to plant’,\\
\gll a-ha-yętwá-hs-e-ˀ\\
 {\factual}-\textsc{3s.m.a}-plant-{\dislocative}-go-{\punctual}\\
\glt `is he going to plant?'

\ex agagyǫ̱ˀséheˀ\\
\gll a-g-ag-yǫˀsé-h-e-ˀ\\
 {\factual}-\textsc{1s.a}-{\semireflexive}-visit-{\dislocative}-go-{\punctual}\\
\glt `I am going to go visit'
\z
\z

\newpage
These verbs can be used either for asking about someone’s actions and current intentions or to state the same \xref{ex:dislpurpfacex2a}. They can be explicitly turned into questions by adding the question word \textit{gęh} \xref{ex:dislpurpfacex2b}.

\ea\label{ex:dislpurpfacex2}
\ea ahęnadáhnyo̱hneˀ\\\label{ex:dislpurpfacex2a}
\gll a-hęn-ad-áhny-o̱-hn-e-ˀ\\
 {\factual}-\textsc{3ns.m.a}-{\semireflexive}-hook-put.in.water-{\dislocative}-go-{\punctual}\\
\glt  ‘they are going fishing’, `are they going fishing?'
\cfex{ahęnadáhnyohnaˀ\\
	\gll a-hęn-ad-áhny-o̱-hn-aˀ\\
	{\factual}-\textsc{3ns.m.a}-{\semireflexive}-hook-put.in.water-go-{\punctual}\\
	\glt `they went fishing'}
\cfex{ahęnadáhnyo:ˀ\\
	\gll a-hęn-ad-áhny-o-:ˀ\\
	\exsc{actual-3ns.m.a}-{\semireflexive}-hook-put.in.water-{\punctual}\\
	\glt `they (males) fished'}
\ex ahęnadahnyo̱hnéˀ gęh?\\\label{ex:dislpurpfacex2b}
\gll a-hęn-ad-áhny-o̱-hn-e-ˀ gęh\\
 {\factual}-\textsc{3ns.m.a}-{\semireflexive}-hook-put.in.water-{\dislocative}-go-{\punctual} Q\\
\glt `are they going fishing?' 
\z
\z

Example \xref{ex:dislpurpfacex3} contrasts dislocative e-verbs (with a \textsc{dislocative} suffix and \stem{e} ‘go’), plain dislocative verbs (with just a \textsc{\dislocative} suffix), and regular \textsc{verbs} (without either suffix).

\ea\label{ex:dislpurpfacex3}
\ea ehsę́da̱ˀdreˀ \\
\gll e-hs.ę́da̱ˀ-dr-e-ˀ\\
 {\factual}-\textsc{2s.p}.sleep-{\dislocative}-go-{\punctual}\\
\glt ‘you are going to bed’, `are you going to bed?'

\cfex{ęsę́da̱ˀdraˀ\\
\gll e-hs-ę́da̱ˀ-dr-aˀ\\
{\factual}-\exsc{2s.p.}sleep-{\dislocative}-{\punctual}\\
\glt ‘you will go to bed’, ‘you will go to sleep’
}

\cfex{ęsę́:daˀ\\
	\gll e-hs.ę́:d-aˀ\\
	{\factual}-\textsc{2s.p}.sleep-{\punctual}\\
	\glt `you will sleep'}

\newpage
\ex aˀǫdawę́ˀneˀ \\
\gll aˀ-ǫ-d-awę́-ˀn-e-ˀ\\
 {\factual}-\textsc{3s.fi.a}-{\semireflexive}-swim-{\dislocative}-go-{\punctual}\\
\glt ‘she is going swimming’, `is she going swimming?'

\cfex{ęyǫdawę́ˀnaˀ\\
	\gll ę-yǫ-d-awę́-ˀn-aˀ\\
	\fut-\textsc{3s.fi.a}-{\semireflexive}-swim-{\dislocative}-{\punctual}\\
	\glt `she will go swimming'}

\cfex{ęyǫ́dawę:ˀ\\
	\gll ę-yǫ́-d-awę-:ˀ\\
	{\factual}-\textsc{3s.fi.a}-{\semireflexive}-swim-{\punctual}\\
	\glt `she will swim'}

\ex agaesá:keˀ \\
\gll a-gae-sá:k-h-e-ˀ\\
 {\factual}-\textsc{3s.fi.a}-seek-{\dislocative}-go-{\punctual}\\
\glt ‘they are going to look for it’, `are they going to look for it?'

\cfex{agaehsá:kaˀ\\
	\gll a-gae-sá:k-h-aˀ\\
	{\factual}-\textsc{3s.fi.a}-seek-{\dislocative}-{\punctual}\\
	\glt `they went looking for it'}

\ex ahayętwáhseˀ \\
\gll a-ha-yętw-áhs-e-ˀ\\
{\factual}-\textsc{3s.m.a}-plant-{\dislocative}-go-{\punctual}\\
\glt ‘he is going to plant’, `is he going to plant?'

\cfex{ęhayętwáhsaˀ\\
	\gll ę-ha-yętw-áhs-aˀ\\
	\fut-\textsc{3s.m.a}-plant-{\dislocative}-{\punctual}\\
	\glt `he will go planting'}

\cfex{ęháyętoˀ\\
	\gll ę-há-yęto-ˀ\\
	\fut-\textsc{3s.m.a}-plant-{\punctual}\\
	\glt `he will plant'}
\z
\z

\newpage
\subsubsection*{\stem{ę-/a:-…\textsc{verb-\dislocative}-e-:k}}
Three-aspect punctual forms of dislocative e-verbs can also take \stem{ę-} \textsc{\future} \xref{ex:dislpurpfacex7} or \stem{a:-} \textsc{\indefinite} \xref{ex:dislpurpfacex8} prefixes, in which case they also require the \stem{-:k} \textsc{\modalizer} suffix (see \tabref{figtab:1:dislpunc}, p. \pageref{figtab:1:dislpunc}).\footnote{There
    exist some dislocative e-verbs in the punctual which appear to be based on \emph{stative-only} verbs. Such verbs do not appear to require the \textsc{modalizer} \xref{ex:dislpurpfacex80}.
    \ea\label{ex:dislpurpfacex80}
    \ea \stem{ę-…\textsc{stat.verb}-e-ˀ} \\
    ękǫ́ˀji̱hneˀ\\
    \gll ę-k-hǫ́ˀji̱-hn-e-ˀ\\
    \fut-\textsc{1s.a}-dark-{\dislocative}-go-{\punctual}\\
    \glt `I am going to get dark'

    \ex \stem{a-…\textsc{stat.verb}-e-ˀ} \\
    akǫ́ˀji̱hneˀ\\
	    \gll a-k-hǫ́ji̱-hn-e-ˀ\\
	    {\factual}-\textsc{1s.a}-dark-{\dislocative}-go-{\punctual}\\
	    \glt `I got dark, black (really tanned)'

    \cfex{ga̱hǫ́ˀji:\\
	    \gll ga̱-hǫ́ˀji-:\\
	    \textsc{3s.a}-dark-{\stative}\\
	    \glt `it is dark-coloured'}
    \z
    \z
}

\ea\label{ex:dislpurpfacex7} \stem{ę-…\textsc{verb-\dislocative}-e-:k}
\ea ęgahyagwáhse:k\\
\gll ę-g-ahya-gw-áhs-e-:k\\
 \fut-\textsc{1s.a}-fruit-pick-{\dislocative}-go-{\modalizer}\\
\glt `I will be a fruit picker'

\ex ętayętwáhse:k\\
\gll ę-t-ha-yętw-áhs-e-:k\\
 \fut-{\cislocative}-\textsc{3s.m.a}-plant-{\dislocative}-go-{\modalizer}\\
\glt `he will be planting over there'
\z
\z

\ea\label{ex:dislpurpfacex8} \stem{a:-…\textsc{verb-\dislocative}-e-:k}
\ea a:ga̱hyágwa̱hse:k\\
\gll a:-g-a̱hyá-gw-a̱hs-e-:k\\
 {\indefinite}-\textsc{1s.a}-fruit-pick-{\dislocative}-go-{\modalizer}\\
\glt `I might be a fruit picker'

\largerpage
\ex ǫ:tayę́twa̱hse:k\\
\gll ǫ:-t-ha-yętw-a̱hs-e-:k\\
 {\indefinite}-{\cislocative}-\textsc{3s.m.a}-plant-{\dislocative}-go-{\modalizer}\\
\glt `he would be planting over there'

\ex a:hayętwa̱hsé:k tsǫ:\\
\gll a:-ha-yętw-a̱hs-é-:k tsǫ:\\
 {\indefinite}-\textsc{3s.m.a}-plant-{\dislocative}-go-{\modalizer} just\\
\glt `just let him keep planting'
\z
\z



\subsection{Dislocative e-verbs in the stative} \label{ch:Dislocative-[e] verbs in the stative}
Three-aspect stative forms of dislocative e-verbs can take either short-e \stem{e-ˀ} go-{\stative} or long-e \stem{e-:-ˀ} go-{\purposive}-{\stative} combinations, \tabref{figtab:1:dislstat}. The short-e versions have present readings \xref{ex:displurpsufex3}, and the long-e versions have past readings \xref{ex:displurpsufex4}. For pronominal prefix choice, see \sectref{ch:Dislocative E-verb prefix choice}.

\begin{table}
\caption{Dislocative e-verb, stative forms}
\label{figtab:1:dislstat}
{
\begin{tabularx} {\textwidth}{l|l|C|C|C|l}
% \lsptoprule
 & pron & \textsc{verb-\dislocative} & \stem{e} & \textsc{\purposive} & aspect\\
\hline
{\dislocative} \textsc{short-e} {\stative} &  & \textsc{verb-\dislocative} & \stem{e} &  & \stem{-ˀ} \textsc{\stative}\\
{\dislocative} \textsc{long-e} {\stative} &  & \textsc{verb-\dislocative} & \stem{e} & \stem{-:} & \stem{-ˀ} \textsc{\stative}\\
% \lspbottomrule
\end{tabularx}}
\end{table}


\ea\label{ex:displurpsufex3} three-aspect stative, with short \stem{e-ˀ}
\ea gatrǫ́nya̱hneˀ\\
\gll g-at-hrǫ́ny-a̱hn-e-ˀ\\
 \textsc{1s.a}-{\semireflexive}-tell-{\dislocative}-go-{\stative}\\
\glt `I have come here to tell'

\ex ǫdéˀdǫ̱hneˀ\\
\gll ǫ-dé-ˀdǫ̱-hn-e-ˀ\\
 \textsc{3s.fi.a}-{\semireflexive}-be.due-{\dislocative}-go-{\stative}\\
\glt `she is about to give birth'

\cfex{ęyǫ́de̱ˀdǫ:ˀ\\
	\gll ę-yǫ́-de-ˀdǫ-:ˀ\\
	\fut-\textsc{3s.fi.a}-{\semireflexive}-be.due-{\punctual}\\
	\glt `when she will be due'}

\ex de̱hęnatgwáhneˀ\\
\gll de̱-hęn-at-gw-áhn-e-ˀ\\
 {\dualic}-\textsc{3ns.m.a}-{\semireflexive}-dance-{\dislocative}-go-{\stative}\\
\glt `they are going to dance'

\cfex{de̱hę́natkwaˀ\\
\gll de̱-hę́n-at-gw-haˀ\\
{\dualic}-\textsc{3ns.m.a}-{\semireflexive}-dance-{\habitual}\\
\glt ‘they dance’ (all the time)
}

\ex de̱hohedáhneˀ\\
\gll de̱-ho-hed-áhn-e-ˀ\\
 {\dualic}-\textsc{3s.m.p}-yell-{\dislocative}-go-{\stative}\\
\glt `he is going along hollering'

\cfex{de̱hóhetaˀ\\
	\gll de̱-hó-het-haˀ\\
	{\dualic}-\textsc{3s.m.p}-yell-{\habitual}\\
	\glt `he is hollering'}
\z
\z

\ea\label{ex:displurpsufex4} three-aspect stative, with long-e \stem{e-:-ˀ}
\ea gatrǫ́nya̱hne:ˀ\\
\gll g-at-hrǫ́ny-a̱hn-e-:-ˀ\\
 \textsc{1s.a}-{\semireflexive}-tell-{\dislocative}-go-{\purposive}-{\stative}\\
\glt `I came and told'

\ex osdéhsda̱hne:ˀ\\
\gll o-sdéhsd-a̱hn-e-:-ˀ\\
\textsc{3s.p}-evaporate-{\dislocative}-go-{\purposive}-{\stative}\\
\glt ‘it has gone down’ (i.e. the water level)\\

\cfex{ohsdéhsdǫh\\
	\gll o-hsdéhsd-ǫh\\
	\textsc{3s.p}-evaporate-{\stative}\\
	\glt `it has evaporated, all dried up'}
\z
\z



\section{Progressive verbs with \stem{e} ‘go’} \label{ch:Progressive verbs}
Progressive verbs consist of a \textsc{stative verb base}, \tabref{figtab:1:progstatbase}, followed by the \stem{-h-ę/ǫ/a-gy-e} \textsc{euph.h}-{\joiner}-{\progressive}-go combination, yet more aspect suffixes, and potentially, mood prefixes. (For the \stem{-ę/ǫ/a} {\joiner} pronunciations, see \sectref{Joiner A pronounced as Ǫ or Ę}.) The stative verb base consists of a \stem{verb-\stative} combination (either \stem{verb-ę} or \stem{verb-ǫ}), or \stem{verb.V} - a verb stem ending with a vowel other than \phonet{ę,ǫ}, \tabref{figtab:1:progstatbase}. Since progressive verbs have a stative base, they require \textsc{p}-series pronominal prefixes (also see \sectref{ch:Progressive verb prefix choice}). The aspect forms are summarized in \figref{figtab:1:progoverview}, page \pageref{figtab:1:progoverview}, and are described next.\footnote{Haǫhyagehdeˀ Deer (p.c.) finds examples of progressive verbs ending with both short-e \stem{ę-gy-e-ˀ, ǫ-gy-e-ˀ} and long-e \stem{ę-gy-e-:-ˀ}, \stem{ǫ-gy-e-:-ˀ} combinations, which resemble the short-e and long-e endings described earlier. This is the basis for assuming that progressive verbs are indeed e-verbs. (\cite[14]{steckley_words_2007} also makes a similar claim for Huron.) That being said, combinations of progressive verbs ending with long \stem{ę-gy-e-:-ˀ} or \stem{ǫ-gy-e-:-ˀ} are not attested in the database used in this book.}


\begin{table}
\caption{Progressives - stative verb base}
\label{figtab:1:progstatbase}
{
\begin{tabularx}{0.66\textwidth}{X|X|l}
% \lsptoprule
pronominal & stem & aspect\\
\hline
\textsc{p} & \textsc{verb stem} & \stem{-ǫ} \textsc{\stative}\\
\textsc{p} & \textsc{verb stem} & \stem{-ę} \textsc{\stative}\\
\textsc{p} & \textsc{verb stem.V} &  \textsc{\stative}\\
% \lspbottomrule
\end{tabularx}}
\end{table}

\begin{figure}
\caption{Progressives (overview)}
\label{figtab:1:progoverview}
{
\begin{tabularx} {\textwidth}{Q|p{14mm}|C|Q|l|c|p{12mm}}
% \lsptoprule
& mood & {\stative} base \mbox{\footnotesize{(\textsc{p}-\textsc{verb-(\stative}))}} & \textsc{\joiner} & \textsc{\progressive} & \stem{e} & aspect suffix\\
\hline
{} {\stative} {\progressive} &  & \textsc{\stative} base & {\joiner} & \stem{-gy}  & \stem{e} & \stem{-ˀ} \textsc{\stative}\\
\tablevspace
{} {\punctual} {\progressive} & \stem{aˀ-} \textsc{\factual} or \newline
		  \stem{ę-}
            \textsc{\future} \newline or \newline
		     \mbox{\stem{a:-} \footnotesize{\textsc{\indefinite}}} & \textsc{\stative} base & {\joiner} & \stem{-gy} & \stem{e} & \stem{-ˀ} \textsc{\punctual}\\
\tablevspace
\textsc{\habitual} {\progressive} &  & \textsc{\stative} base & {\joiner} & \stem{-gy} & \stem{e} & \stem{-ˀs} \textsc{\habitual}\\
% \lspbottomrule
\end{tabularx}}
\end{figure}


\subsection{Stative progressive} \label{ch:Stative progressive}

\textsc{stative progressives} consist of a stative verb base (\tabref{figtab:1:progstatbase}) followed by the \stem{-h-ę/ǫ/a-gy-e-ˀ} \textsc{euph.h}-{\joiner}-{\progressive}-go-{\stative} combination (\tabref{figtab:1:progstatbase}, \tabref{figtab:1:statprog}, and examples \ref{ex:purpprogsufex7}-\ref{ex:purpprogsufex8}, beginning on page \pageref{ex:purpprogsufex7}). Bases ending with Ǫ take the \stem{-h-ǫ-gy-e-ˀ} \textsc{euph.h-\joiner-\progressive-go-\stative} combination \xref{ex:purpprogsufex7} and ones ending with Ę take the \stem{-h-ę-gy-e-ˀ} \textsc{euph.h-\joiner-\progressive-go-\stative} combination \xref{ex:purpprogsufex8}. In contrast, verb bases ending with V take the [-h-a-gy-e-ˀ] \textsc{euph.h}-{\joinerA}-{\progressive}-go-{\stative} combination \xref{ex:purpprogsufex9}. Like other stative-aspect verbs, these verbs require \textsc{p}-series pronominal prefixes (see \sectref{ch:Progressive verb prefix choice}).


\begin{table}
\caption{stative progressive}
\label{figtab:1:statprog}
{
\begin{tabularx} {\textwidth}{X|X|c|l}
% \lsptoprule
\textsc{\stative} base & \textsc{\progressive} & \stem{e} & aspect\\
\hline
\textsc{p}-verb\stem{-ǫ} & \stem{-h-ǫ-gy} \newline
\textsc{\euphonich-\joiner-\progressive} & \stem{e} & \stem{-ˀ} \textsc{\stative}\\
\hline
\textsc{p}-verb\stem{-ę} & \stem{-h-ę-gy} \newline
\textsc{\euphonich-\joiner-\progressive} & \stem{e} & \stem{-ˀ} \textsc{\stative}\\
\hline
\textsc{p}-verb.V & \stem{h-a-gy} \newline
\textsc{\euphonich-\joiner-\progressive} & \stem{e} & \stem{-ˀ} \textsc{\stative}\\
% \lspbottomrule
\end{tabularx}}
\end{table}

Stative-progressive verbs have a “present progressive” or ‘…is -ing’ meaning. For comparison, these and other statives are shown in the following examples.

\FloatBarrier
\ea\label{ex:purpprogsufex7} \stem{-ǫ-h-ǫ-gy-e-ˀ}
\ea sa̱hohdęgyǫ́hǫgyeˀ\\
\gll sa̱-ho-hdęgy-ǫ́-h-ǫ-gy-e-ˀ\\
 {\repetitive}-\textsc{3s.m.p}-leave-{\stative}-\textsc{euph.h}-{\joiner}-{\progressive}-go-{\stative}\\
\glt `he is on his way home'
\cfex{hesáhdęgyǫ:\\
\gll he-s-áhdęgy-ǫ:\\
{\translocative}-\textsc{2s.p}-leave-{\stative}\\
\glt ‘you went over there’
}

\ex otsihsˀǫ̱hǫ́:gyeˀ\\
\gll o-tsihsˀ-ǫ̱-h-ǫ́:-gy-e-ˀ\\
 \textsc{3s.p}-mature-{\stative}-\textsc{euph.h}-{\joiner}-{\progressive}-go-{\stative}\\
\glt `it is getting mature'

\cfex{otsíhsˀǫh\\
\gll o-tsíhsˀ-ǫh\\
\textsc{3s.p}-mature-{\stative}\\
\glt ‘it is done for the season’, ‘it has gone full cycle’, ‘it is mature’, ‘they (plants) have finished out’
}

\ex hodǫgo̱hdǫ́hǫgyeˀ\\
\gll ho-d-ǫgo̱hd-ǫ́-h-ǫ-gy-e-ˀ\\
 \textsc{3s.m.p}-{\semireflexive}-surpass-{\stative}-\textsc{euph.h}-{\joiner}-{\progressive}-go-{\stative}\\
\glt `he is going along passing'
\cfex{hodǫ́go̱hdǫh \\
\gll ho-d-ǫ́go̱hd-ǫh\\
\textsc{3s.m.p}-{\semireflexive}-surpass-{\stative}\\
\glt ‘he has gone past’
}
\ex go̱hdogadǫ́hǫgyeˀ\\
\gll go̱-hdogad-ǫ́-h-ǫ-gy-e-ˀ\\
 \textsc{3s.fi.p}-cause.to.grow-{\stative}-\textsc{euph.h}-{\joiner}-{\progressive}-go-{\stative}\\
\glt `she is growing something'

\cfex{ǫgahdǫgáˀdǫh\\
\gll ǫg-ahǫgáˀd-ǫh\\
\textsc{3fis:1s}-raise-{\stative}\\
\glt ‘she (e.g. a guardian) raised me’
}

\ex niyagotgęihsdǫ́hǫgyeˀ\\
\gll ni-yago-t-gęihsd-ǫ́-h-ǫ-gy-e-ˀ\\
{\partitive}-\textsc{3s.fi.p}-{\semireflexive}-move.ahead-{\stative}-\textsc{euph.h}-{\joiner}-{\progressive}-go-{\stative}\\
\glt  `she is travelling as she is moving'

\cfex{gagę́ihsdǫh\\
\gll ga-gę́ihsd-ǫh\\
\textsc{3s.a}-move.ahead-{\stative}\\
\glt ‘postponement’
}

\ex odaˀgra̱hdǫ́hǫgyeˀ\\
\gll o-d-aˀgra̱hd-ǫ́-h-ǫ-gy-e-ˀ\\
 \textsc{3s.p}-{\semireflexive}-float-{\stative}-\textsc{euph.h}-{\joiner}-{\progressive}-go-{\stative}\\
\glt `it is surfacing'

\cfex{odáˀgra̱hdǫh\\
\gll o-d-áˀgra̱hd-ǫh\\
\textsc{3s.p}-{\semireflexive}-float-{\stative}\\
\glt ‘it is floating’
}
\z
\z

\ea\label{ex:purpprogsufex8} \stem{-ę-h-ę-gy-e-ˀ} 
\ea agade̱ˀgwę́hęgyeˀ\\
\gll ag-ade̱-ˀgw-ę́-h-ę-gy-e-ˀ\\
 \textsc{1s.p}-{\semireflexive}-run.away-{\stative}-\textsc{euph.h}-{\joiner}-{\progressive}-go-{\stative}\\
\glt `I am running away again'
\cfex{agáde̱ˀgwęh\\
\gll ag-áde̱-ˀgw-ęh\\
\textsc{1s.p}-{\semireflexive}-run.away-{\stative}\\
\glt ‘I am running away’
}
\ex da̱hodawę́hęgyeˀ\\
\gll da̱-ho-d-aw-ę́-h-ę-gy-e-ˀ\\
 {\cislocative}-\textsc{3s.m.p}-{\semireflexive}-swim-{\stative}-\textsc{euph.h}-{\joiner}-{\progressive}-go-{\stative}\\
\glt `he is swimming towards us'

\cfex{hodá:węh\\
\gll ho-d-á:w-ęh\\
\textsc{3s.m.p}-{\semireflexive}-swim-{\stative}\\
\glt ‘he did swim’
}
\z
\z

\newpage
\ea\label{ex:purpprogsufex9} \stem{V-h-a-gy-e-ˀ} 
\ea go̱hsganye̱há:gyeˀ\\
\gll go̱-hsganye̱-h-á:-gy-e-ˀ\\
 \textsc{3s.fi.p}-shuffle-\textsc{euph.h}\exsc{-\joiner-\progressive}-go-{\stative}\\
\glt `she is shuffling along'
 \cfex{gohsga:nye:ˀ\\
 \gll go-hsgá:nye-:ˀ\\
 3s.fi.p-shuffle-stative\\
  \glt ‘she has shuffled’
}

\ex sagawe̱há:gyeˀ\\
\gll sa-gawe̱-h-á:-gy-e-ˀ\\
 \textsc{2s.p}-paddle-\textsc{euph.h}\exsc{-\joiner-\progressive}-go-{\stative}\\
\glt `you are paddling along'
\cfex{agégaweˀ\\
\gll agé-gawe-ˀ\\
\textsc{1s.p}-paddle-{\stative}\\
\glt ‘I have paddled, are paddling’ 
}

\ex agyǫdi̱há:gyeˀ\\
\gll ag-yǫdi̱-h-á:-gy-e-ˀ\\
 \textsc{1s.p}-smile-\textsc{euph.h}\exsc{-\joiner-\progressive}-go-{\stative}\\
\glt `I am going along smiling'

\cfex{hoyǫ́gyǫ̱hne:ˀ\\
\gll ho-yǫgy-ǫ̱-hne:ˀ\\
\textsc{3s.m.p}-smile-{\stative}-{\remote}\\
\glt ‘he has already smiled, ‘he did smile’
}

\ex ęjisadekǫníhagyeˀ\\
\gll ę-ji-s-ade-k-ǫní-h-a-gy-e-ˀ\\
\fut-{\repetitive}-\textsc{2s.p}-{\semireflexive}-food-make-\textsc{euph.h}\exsc{-\joiner-\progressive}-go-{\stative}\\
\glt ‘you can (literally, ‘will’) eat while you are going along’

\cfex{hode:kǫ́:ni:\\
\gll ho-de:-k-ǫ́:ni-:\\
\textsc{3s.m.p}-{\semireflexive}-food-make-{\stative}\\
\glt ‘he is eating’
}
\z
\z


\subsubsection*{\stem{-dagyeˀ}, \stem{-ędagyeˀ}, \stem{-odagyeˀ}, \stem{-ǫdagyeˀ}, \stem{-oˀkd-agyeˀ}} \label{ch:[-dagyeˀ], [-ędagyeˀ], [-odagyeˀ], [-ǫdagyeˀ], [-oˀkd-agyeˀ]}
Several positional verbs (a thematic category of stative-only verb) often take stative progressive forms. Examples illustrating the range of meanings are provided below.
 
\newpage
\ea\label{ex:purpprogsufex2}
\stem{d-a-gy-e-ˀ} stand-{\joinerA}-{\progressive}-go-{\stative} ‘continue on, be ongoing’\\
gaǫhyadá:gyeˀ\\
\gll ga-ǫhya-d-á:-gy-e-ˀ\\
 \textsc{3s.a}-sky-stand.{\stative}-{\joinerA-\progressive}-go-{\stative}\\
\glt `on-going skies, heavens, skyline, horizon'
\z


\ea\label{ex:purpprogsufex3} \stem{ę-d-a-gy-e-ˀ} lie-\textsc{euph.d-\joinerA}-\exsc{\progressive}-go-{\stative} (the meaning depends on which noun is incorporated)
\ea onǫnyáędagyeˀ\\
\gll o-nǫnyá-ę-d-a-gy-e-ˀ\\
 \textsc{3s.p}-ice-lie.{\stative}-\textsc{euph.d-\joinerA}-{\progressive}-go-{\stative}\\
\glt `it is freezing'

\cfex{onǫ́nyaęˀ\\
\gll o-nǫ́ny-a-ęˀ\\
\textsc{3s.p}-ice-\joinerA-lie.{\stative}\\
\glt ‘it is frozen’
}

\ex awęnowaędá:gyeˀ\\
\gll aw-ęnowa-ę-d-á:-gy-e-ˀ\\
 \textsc{3s.p}-sink-lie.{\stative}-\textsc{euph.d-\joinerA}-{\progressive}-go-{\stative}\\
\glt `it is sinking'

\cfex{ęwęnǫwáęˀ\\
\gll ę-w-ęnǫw-á-ę-ˀ\\
\fut-\textsc{3s.a}-sink-\joinerA-lie.{\stative}-{\punctual}\\
\glt ‘it will sink’
}
\z
\z

\ea\label{ex:purpprogsufex4} \stem{od-a-gy-e-ˀ} stand-{\joinerA}-{\progressive}-go-{\stative} (the meaning depends on which noun is incorporated)
\ea degahsdę̱hódagyeˀ\\
\gll de-ga-hsdę̱h-ód-a-gy-e-ˀ\\
 {\dualic}-\textsc{3s.a}-stone-standing.{\stative}-\joinerA-{\progressive}-go-{\stative}\\
\glt `mountain range, the Rockies'

\ex gajiˀdodá:gyeˀ\\
\gll ga-jiˀd-od-á:-gy-e-ˀ\\
 \textsc{3s.a}-cry-standing.{\stative}-\joinerA-{\progressive}-go-{\stative}\\
\glt `someone is going along crying'

\cfex{gajíˀdo:t\\
\gll ga-jíˀd-o:t\\
\textsc{3s.a}-cry-stand.{\stative}\\
\glt ‘it is crying’
}

\ex desatnǫ̱hweˀtsodá:gyeˀ\\
\gll de-s-at-nǫ̱hweˀts-od-á:-gy-e-ˀ\\
 {\dualic}-\textsc{2s.a}-{\semireflexive}-noun-stand.{\stative}-\joinerA-{\progressive}-go-{\stative}\\
\glt `you are going along doing handstands'

\cfex{dęsatnǫ̱hwe:tsó:dǫˀ\\
\gll d-ę-s-at-nǫ̱hwe:ts-ó:d-ǫˀ\\
{\dualic}-{\future}-\textsc{2s.a}-{\semireflexive}-noun-stand.{\stative}-\textsc{distr}\\
\glt ‘you will do a hand stand, turn yourself upside down’
}
\z
\z

\ea\label{ex:purpprogsufex5} \stem{ǫd-a-gy-e-ˀ} attached-{\joinerA}-{\progressive}-go-{\stative} ‘a series of embedded objects’
\ea dewaˀdęhǫ́dagyeˀ\\
\gll de-w-aˀdęh-ǫ́d-a-gy-e-ˀ\\
 {\dualic}-\textsc{3s.a}-fence-rooted.{\stative}-{\joinerA-\progressive}-go-{\stative}\\
\glt `wood fence, rail and stump fence'

\ex honhǫ́dagyeˀ\\
\gll ho-nhǫ́d-a-gy-e-ˀ\\
 \textsc{3s.m.p}-opening-{\joinerA-\progressive}-go-{\stative}\\
\glt `he has it in his mouth as he moves'

\cfex{ohsgyę́ˀdanhǫ:t\\
\gll o-hsgyę́:ˀd-a-nhǫ:t\\
\textsc{3s.p}-bones-\joinerA-opening.{\stative}\\
\glt ‘false teeth' (literally, ‘a mouth full of bones’)
}
\z
\z

\ea\label{ex:purpprogsufex6}  \stem{oˀkd-a-gy-e-ˀ} end-{\joinerA}-{\progressive}-go-{\stative} ‘a running edge’\\
odóˀkdagyeˀ \\
\gll o-d-óˀkd-a-gy-e-ˀ\\
 \textsc{3s.p}-{\semireflexive}-end.{\stative}-{\joinerA-\progressive}-go-{\stative}\\
\glt ‘rim’, ‘outer and inner rim splint’, `along the edge'
\z


\subsection{Punctual progressive} \label{ch:Punctual progressive}

\textsc{punctual progressives} consist of a stative verb base (\tabref{figtab:1:progstatbase}) followed by the \stem{-h-ę/-ǫ-/a-gy-e-ˀ} \textsc{euph.h}-{\joiner}-{\progressive}-go-{\punctual} combination, \tabref{figtab:1:puncprog}. Bases ending with Ę take the \stem{-h-ę-gy-e-ˀ} combination \xref{ex:purpprogsufex11c}, and ones ending with Ǫ take the \stem{-h-ǫ-gy-e-ˀ} combination \xref{ex:purpprogsufex10a}, \xxref{ex:purpprogsufex11a}{ex:purpprogsufex11b}. In contrast, verbs ending with V take the \stem{-h-a-gy-e-ˀ} combination \xref{ex:purpprogsufex13}.

\begin{table}
\caption{Punctual progressive}
\label{figtab:1:puncprog}
{
\begin{tabularx} {\textwidth}{X|Q|X|l|X}
% \lsptoprule
mood & {\stative} base \mbox{(\textsc{p}-\textsc{verb(-\stative}))}  & \textsc{\progressive} & \stem{e} & aspect\\
\hline
\stem{ę-} \textsc{\future}

\stem{aˀ-} \textsc{\factual}

\stem{a:-} \textsc{\indefinite} & \textsc{p-verb-ǫ} & \stem{-h-ǫ-gy} 

\textsc{euph.h-\joiner-\progressive}

& \stem{e} & \stem{-ˀ} \textsc{\punctual} or 

\stem{-:k} \textsc{\punctual.\modalizer}\\
\hline
\stem{ę-} \textsc{\future}

\stem{aˀ-} \textsc{\factual}

\stem{a:-} \textsc{\indefinite} & \textsc{p-verb-ę} & \stem{-h-ę-gy}

\textsc{euph.h-\joiner-\progressive}

& \stem{e} & \stem{-ˀ} \textsc{\punctual} or

\stem{-:k} \textsc{\punctual.\modalizer}\\
\hline
\stem{ę-} \textsc{\future}

\stem{aˀ-} \textsc{\factual}

\stem{a:-} \textsc{\indefinite} & \textsc{p-verb.V} & \stem{-h-a-gy}  

\textsc{euph.h-\joinerA-\progressive}

& \stem{e} & \stem{-ˀ} \textsc{\punctual} or

\stem{-:k} \textsc{\punctual.\modalizer}\\
% \lspbottomrule
\end{tabularx}}
\end{table}

Punctual progressives require an \stem{aˀ-} \textsc{\factual}, \stem{ę-} \textsc{\future}, or \stem{a:-} \textsc{\indefinite} prefix \xxref{ex:purpprogsufex10}{ex:purpprogsufex11}. The factual punctual progressives have a “factual-present” or \textsc{performative} meaning (\ref{ex:purpprogsufex10}, see \sectref{The [aˀ-] factual with present events and performatives}). Like other stative-aspect verbs, these verbs require \textsc{p}-series pronominal prefixes (see \sectref{ch:Progressive verb prefix choice}).

In at least one instance, the \stem{a:-} \textsc{\indefinite} form also requires a \stem{-:k} \textsc{\modalizer} suffix (compare \ref{ex:purpprogsufex12} and \ref{ex:purpprogsufex13}).

\ea\label{ex:purpprogsufex10} \stem{aˀ-…\textsc{stat.base-\progressive}-e-ˀ} \\
ǫgyaˀda̱hshęˀǫhǫ́:gyeˀ\\\label{ex:purpprogsufex10a}
\gll ǫg-yaˀd-a̱hshęˀ-ǫ-h-ǫ́:-gy-e-ˀ\\
 \exsc{\factual}.\textsc{1s.p}-body-slow-{\stative}-\textsc{euph.h}-{\joiner}-{\progressive}-go-{\punctual}\\
\glt `I am arriving late'

\cfex{ǫgyáˀda̱hshęˀ\\
\gll ǫg-yáˀd-a̱hshę-ˀ\\
\exsc{factual.1s.p}-body-slow-{\punctual}\\
\glt ‘I was late’
}
\z


\ea\label{ex:purpprogsufex11} \stem{ę-…\textsc{stat.base-\progressive}-e-ˀ}
\ea\label{ex:purpprogsufex11a} ęyago̱hdogadǫ́hǫgyeˀ\\
\gll ę-yago̱-hdogad-ǫ́-h-ǫ-gy-e-ˀ\\
 \fut-\textsc{3s.fi.p}-cause.to.grow-{\stative}-\textsc{euph.h}-{\joiner}-{\progressive}-go-{\punctual}\\
\glt `she will be continually growing it or them'

\newpage
\ex ęwaga̱hyagǫ̱hǫ́:gyeˀ\\\label{ex:purpprogsufex11b}
\gll ę-wag-a̱hya-g-ǫ̱-h-ǫ́:-gy-e-ˀ\\
 \fut-\textsc{1s.p}-fruit-eat-{\stative}-\textsc{euph.h}-{\joiner}-{\progressive}-go-{\stative}\\
\glt `I will be eating fruit as I am going along'

\ex dęyago̱ˀnya:gwę́hęgyeˀ\\\label{ex:purpprogsufex11c}
\gll d-ę-yago̱-ˀnya:-gw-ę́-h-ę-gy-e-ˀ\\
 {\dualic}-{\future}-\textsc{3s.fi.p}-hand-pick-{\stative}-\textsc{euph.h}-{\joiner}-{\progressive}-go-{\punctual}\\
\glt `she will have a hand in it'
\z
\z

\ea\label{ex:purpprogsufex12} \stem{a:-…\textsc{stat.base-\progressive}-e-ˀ}\\
a:wagahyagwę́hęgyeˀ\\
\gll a:-wag-ahya-gw-ę́-h-ę-gy-e-ˀ\\
 {\indefinite}-\textsc{1s.p}-fruit-pick-{\stative}-\textsc{euph.h}-{\joiner}-{\progressive}-go-{\punctual}\\
\glt `I should be picking fruit along'
\z


\ea\label{ex:purpprogsufex13} \stem{a:-…\textsc{stat.base-\progressive}-e-:k}\\
a:wagatrowíhagye:k\\
\gll a:-wag-at-hrowí-h-a-gy-e-:k\\
{\indefinite}-\textsc{1s.p}-{\semireflexive}-tell.{\stative}-\textsc{euph.h}-{\joinerA}-{\progressive}-go-{\modalizer}\\
\z

\subsection{Habitual progressive} \label{ch:Habitual progressive}
\begin{table}[b]
\caption{Habitual progressive}
\label{figtab:1:habprog}
{
\begin{tabularx} {\textwidth}{l|X|l|l}
% \lsptoprule
{\stative} base (\textsc{p-verb-(\stative})) & \textsc{\progressive} & \stem{e} ‘go’  & aspect\\
\hline
\textsc{p-verb-ǫ} & \stem{-h-ǫ-gy}

\textsc{euph.h-\joiner-\progressive}
& \stem{e} & \stem{-ˀs} \textsc{\habitual}\\
\hline
\textsc{p-verb-ę} & \stem{-h-ę-gy}

\textsc{euph.h-\joiner-\progressive}
& \stem{e} & \stem{-ˀs} \textsc{\habitual}\\
\hline

\textsc{p-verb.V} & \stem{-h-a-gy}

\textsc{euph.h-\joinerA-\progressive}
& \stem{e} & \stem{-ˀs} \textsc{\habitual}\\
% \lspbottomrule
\end{tabularx}}
\end{table}


\textsc{Habitual progressives} consist of a consist of a stative verb base (\tabref{figtab:1:progstatbase}) followed by the \stem{-h-ę/-ǫ-/a-gy-e-ˀs} \textsc{euph.h}\exsc{-\joiner-\progressive}-go-{\habitual} combination, \tabref{figtab:1:habprog}. Bases ending with Ǫ take the \stem{-h-ǫ-gy-e-ˀs} combination \xref{ex:purpprogsufex14}, and ones ending with Ę take the \stem{-h-ę-gy-e-ˀs} combination \xref{ex:purpprogsufex15}. In contrast, bases ending with V take the \stem{-h-a-gy-e-ˀs} \textsc{euph.h}-{\joinerA}-{\progressive}-go-{\habitual} combination \xref{ex:purpprogsufex16}. Like other stative verbs, these verbs require \textsc{p}-series pronominal prefixes (see \sectref{ch:Progressive verb prefix choice}).

Habitual progressives describe an ongoing, continuous activity. 



\ea\label{ex:purpprogsufex14} \stem{-ǫ-h-ǫ-gy-e-ˀs} 
\ea deyǫkiyęˀnyadǫ́hǫgyeˀs\\
\gll de-yǫkiy-ę-ˀnya-d-ǫ́-h-ǫ-gy-e-ˀs\\
 {\dualic}-\exsc{3fis/3ns:1ns}-{\semireflexive}-hand-stand-{\stative}-\textsc{euph.h}-{\joiner}-{\progressive}-go-{\habitual}\\
\glt `they protect us'

\cfex{deyǫkiyę́ˀnyadǫˀ\\
\gll de-yǫkiy-ę́-ˀnya-d-ǫˀ\\
\textsc{dualic-they:us}-{\semireflexive}-hand-stand.\textsc{\stative-pl}\\
\glt ‘they protect us with their hands’, ‘angels’
}
\ex honadahsehdǫ̱hǫ́:gyeˀs\\
\gll hon-ad-ahsehd-ǫ̱-h-ǫ́:-gy-e-ˀs\\
 \textsc{3ns.m.p}-{\semireflexive}-hide-{\stative}-\textsc{euph.h}-{\joiner}-{\progressive}-go-{\habitual}\\
\glt `they are sneaking around'

\cfex{agada̱hséhdǫh\\
\gll ag-ad-a̱hséhd-ǫh\\
\textsc{1s.p}-{\semireflexive}-hide-{\stative}\\
\glt ‘I am hiding now’
}
\z
\z


\ea\label{ex:purpprogsufex15} \stem{…ę-h-ę-gy-e-ˀs}
\ea agahyagwę́hęgyeˀs\\
\gll ag-ahya-gw-ę́-h-ę-gy-e-ˀs\\
 \textsc{1s.p}-fruit-pick-{\stative}-\textsc{euph.h}-{\joiner}-{\progressive}-go-{\habitual}\\
\glt `I am going along picking fruit all over the place'

\cfex{agahyagwę́hęgyeˀ\\
\gll ag-ahya-gw-ę́-h-ę-gy-e-ˀ\\
\textsc{1s.p}-fruit-pick-{\stative}-\textsc{euph.h}-{\joiner}-{\progressive}-go-{\stative}\\
\glt ‘I am going along picking fruit’
}
\ex sadeˀgwę̱hę́:gyeˀs\\
\gll sa-de-ˀgw-ę̱-h-ę́:-gy-e-ˀs\\
 \textsc{2s.p}-{\semireflexive}-flee-{\stative}-\textsc{euph.h}-{\joiner}-{\progressive}-go-{\habitual}\\
\glt `you are avoiding'

\cfex{ęhsáde̱ˀgoˀ\\
\gll ę-hs-áde̱-ˀgo-ˀ\\
\fut-\textsc{2s.a}-{\semireflexive}-flee-{\punctual}\\
\glt ‘you will flee, run away’
}

\ex deyǫgwa̱hgwę́hęgyeˀs\\
\gll de-yǫgw-a̱hgw-ę́-h-ę-gy-e-ˀs\\
 \textsc{1p.p}-pick.up-{\stative}-\textsc{euph.h}-{\joiner}-{\progressive}-go-{\habitual}\\
\glt `we all are continually picking it up'

\cfex{de̱hoya̱ˀdáhgwęh\\
\gll de̱-ho-ya̱ˀd-áhgw-ęh\\
{\dualic}-\textsc{3s.m.p}-body-pick.up-{\stative}\\
\glt ‘it picked him up’
}
\z
\z

\ea\label{ex:purpprogsufex16} \stem{…V-h-a-gy-e-ˀs}\\
honǫni̱há:gyeˀs\\
\gll hon-ǫni̱-h-á:-gy-e-ˀs\\
\textsc{3ns.m.p}-make-\textsc{euph.h}\exsc{-\joinerA-\progressive}-go-{\habitual}\\
\glt ‘they are making, earning it’ (continually)

\cfex{tiwa:gǫ́:ni:\\
\gll ti-wa:g-ǫ:ni-:\\
{\contrastive}-\textsc{1s.p}-make-{\stative}\\
\glt ‘I have made it in place of something else, replaced it’
}
\z


\subsubsection*{Past habitual progressive}
\textsc{Past} habitual progressives take the [-gęhę:ˀ] \textsc{past} ending (\sectref{Post-aspect (tense) suffixes}), \tabref{figtab:1:pasthabprog}, page \pageref{figtab:1:pasthabprog}. Unlike other progressive verbs, these ones can take either a \textsc{p}-series \xref{ex:purpprogsufex17} or \textsc{a}-series \xref{ex:purpprogsufex18} pronominal prefixes. Possibly, the ones with a \textsc{p}-series prefix describe a past habitual activity, while the ones with an \textsc{a}-series prefix describe a past profession.

\begin{table}
\caption{Past habitual progressive}
\label{figtab:1:pasthabprog}
{
\begin{tabularx} {.8\textwidth}{X|l}
% \lsptoprule
\textsc{hab prog} base & post-aspect\\
\hline
\stem{\textsc{p}-\textsc{verb}-ǫ-h-ǫ-gyeˀs} \newline
\stem{\textsc{p}-\textsc{verb}-ę-h-ę-gyeˀs} \newline
\stem{\textsc{p}-\textsc{verb}.V-h-a-gyeˀs} & \stem{-gęhę:ˀ} {\past}\\
\hline
\stem{\textsc{a}-\textsc{verb}-ǫ-h-ǫ-gyeˀs} \newline
\stem{\textsc{a}-\textsc{verb}-ę-h-ę-gyeˀs} \newline
\stem{\textsc{a}-\textsc{verb}.V-h-a-gyeˀs} & \stem{-gęhę:ˀ} {\past}\\
% \lspbottomrule
\end{tabularx}}
\end{table}


\ea\label{ex:purpprogsufex17} \stem{\textsc{p}-…hab.prog.base-gęhę:ˀ}\\
agahyagwęhęgyéˀsgę̱hę:ˀ\\
\gll ag-ahya-gw-ę-h-ę-gy-é-ˀs-gę̱hę:ˀ\\
\textsc{1s.p}-fruit-pick-{\stative}-\textsc{euph.h}-{\joiner}-{\progressive}-go-{\habitual}-{\past}\\
\glt ‘I used to go along picking fruit’ (but I no longer do)
\z

\newpage
\ea\label{ex:purpprogsufex18} \stem{\textsc{a}-…hab.prog.base-gęhę:ˀ}\\
ga̱hyagwę̱hęgye̱ˀsgę́hę:ˀ\\
\gll g-a̱hya-gw-ę̱-h-ę-gy-e̱-ˀs-gę́hę:ˀ\\
 \textsc{1s.a}-fruit-pick-{\stative}-\textsc{euph.h}-{\joiner}-{\progressive}-go-{\habitual}-{\past}\\
\glt `I used to be a fruit picker'
\z

\subsubsection*{Indefinite and (possible) future habitual progressive}
\textsc{Indefinite} habitual progressive verbs consist of an \stem{a:-} \textsc{\indefinite} prefix, a habitual progressive base, and a \stem{-:k} {\modalizer} suffix (\tabref{figtab:1:futidfhabprog} and example \ref{ex:purpprogsufex19}). (Forms with the \stem{ę-} \textsc{\future} prefix -- as in hypothetical \textit{ęwaga̱hyagwę̱hę́gye̱ˀse:k} -- may be possible, because words with the \stem{ę-} \textsc{\future} prefix are typically possible when the same word with the \stem{a:-} \textsc{\indefinite} prefix is attested.)

\begin{table}
\caption{Indefinite and (possible) future habitual progressive}
\label{figtab:1:futidfhabprog}
{
\begin{tabularx} {\textwidth}{X|X|l}
% \lsptoprule
mood & {\habitual} {\progressive} base & post-aspect\\
\hline
\stem{ę-} \textsc{\future} or \newline
\stem{a:-} \textsc{\indefinite} & \stem{\textsc{p}-\textsc{verb}-ǫ-h-ǫ-gyeˀs}  \newline
\stem{\textsc{p}-\textsc{verb}-ę-h-ę-gyeˀs}  \newline
\stem{\textsc{p}-\textsc{verb}-V-h-a-gyeˀs}  & \stem{-e:k} \textsc{\modalizer}\\

% \lspbottomrule
\end{tabularx}}
\end{table}


\ea\label{ex:purpprogsufex19} \stem{a:-\textsc{p}-\textsc{\habitual.\progressive.base}-e:k} \\
a:wagahyagwęhęgyéˀse:k\\
\gll a:-wag-ahya-gw-ę-h-ę-gy-é-ˀs-e:k\\
 {\indefinite}-\textsc{1s.p}-fruit-pick-{\stative}-\textsc{euph.h}-{\joiner}-{\progressive}-go-{\habitual}-{\modalizer}\\
\glt `I would (still) have been picking fruit…'
\z
