\chapter{“Pronouns” (words and affixes functioning as pronouns)} \label{ch:’Pronouns’ (words and affixes that function as pronouns)}
“Pronouns” are words, phrases, or prefixes whose function is to substitute for nouns or noun phrases. (Noun phrases consist of a noun plus related words.) For example, the emphatic “pronoun” \textit{háǫhęˀ} \xref{pronoun.introa} can refer to or substitute for any male in a conversation, including the noun  in \xref{pronoun.introb} and the noun phrases in (\ref{pronoun.intro}c--d). The various types of pronouns are described next.

\ea\label{pronoun.intro}
\ea\label{pronoun.introa} háǫhęˀ ‘he alone’ (emphatic “pronoun”)
\ex\label{pronoun.introb} háǫhęˀ = haˀnih ‘my father’ (kinship term)
\ex\label{pronoun.introc} háǫhęˀ = neˀ Dave hyanóˀseh ‘your uncle Dave’ (noun phrase) (\cite[339]{mithun_watewayestanih_1984}, Oˀdréhdatgiˀ dialogue)
\ex\label{pronoun.introd} háǫhęˀ = neˀ hahędage̱hó:nǫˀ ‘the farmer’ (noun phrase)
\z
\z

\section{Emphatic “pronouns”} \label{ch:’Emphatic pronouns’}
Like other pronouns, emphatic “pronouns” substitute for nouns or noun phrases. However, emphatic “pronouns” also add emphasis in the context of a conversation. For example, \textit{háǫhęˀ} ‘he alone’ in \xref{ex:emph} emphasizes \emph{who} actually spoke (that is, \emph{he}, not \emph{someone else}). 

\ea\label{ex:emph}
 \gll \exemph{haǫhę́ˀ} ahęˀ. \\
he.alone he.spoke\\
\glt ‘\emph{He} spoke.’
\z

Emphatic “pronouns” can be omitted from the sentence. While this causes a change in meaning (the emphasis is lost), the result is still a complete sentence \xref{ex:emph2}. Pronouns like \textit{háǫhęˀ} are optional because verbs like \textit{ahęˀ} ‘he spoke’ already have a pronominal prefix (see Part~\ref{Pronominal prefixes}).

\ea\label{ex:emph2} Ahęˀ. ‘He spoke.’
\z

The words functioning as emphatic “pronouns” are either \textsc{particles} or \textsc{verbs}. The following emphatic “pronouns” are described in the \textit{Particle dictionary}, \sectref{ch:particle dictionary}.

\begin{CayugaRelated}
\item{} \textit{I:ˀ, Ni:ˀ} ‘I’, ‘we’ \textsc{particle}\\
\item{} \textit{I:s, Ni:s} ‘you’ (any number of people) \textsc{particle}\\
\item{} \textit{Haǫhęˀ, Haǫhaˀ} ‘he’ \textsc{verb}\\
\item{} \textit{Gaǫhęˀ, Gaǫhaˀ} ‘she’ \textsc{verb}\\
\item{} \textit{Aǫhęˀ, Aǫhaˀ} ‘it’ \textsc{verb}\\
\item{} \textit{Honǫ:hęˀ} ‘they (males)’ \textsc{verb}\\
\item{} \textit{Gonǫ:hęˀ} ‘they (females or mixed)’ \textsc{verb}\\
\item{} \textit{Onǫ:hęˀ} ‘they (animals)’ \textsc{verb}\\
\end{CayugaRelated}




The following \textsc{particle groups} include emphatic “pronouns” and are described in the \textit{Particle dictionary}, \sectref{ch:particle dictionary}. Any emphatic “pronoun” from the preceding list can be used in place of the emphatic “pronouns” shown below. 

\begin{CayugaRelated}
    
\item{} \textit{Dęˀ ni:ˀ} ‘I am for sure’\\
\item{} \textit{Seˀ hęˀ ni:ˀ} ‘me too’, 'us too’\\
\item{} \textit{I:ˀ hne:ˀ} ‘no, I am’\\
\item{} \textit{I:ˀgeh (gwa:dih)} ‘on my side’, ‘as for me’\\
\item{} \textit{I:ˀ hya:ˀ} ‘me first’\\
\item{} \textit{I:ˀ seˀ} ‘I am’\\
\item{} \textit{Nę: ne:ˀ I:s} ‘how about you?’\\
\item{} \textit{Nę: diˀ Ni:s} ‘how about you?’\\
\item{} \textit{Ni:ˀ gyę:ˀ, \textit{ni:ˀ} gę:ˀ} ‘I did it’\\
\item{} \textit{Tęˀ I:ˀ} ‘not me’, ‘not us’\\
\item{} \textit{Tęˀ giˀ ni:ˀ} ‘no, not me’\\
\end{CayugaRelated}



\section{Possessive “pronouns”} \label{ch:’Possessive pronouns’}
Like other pronouns, possessive “pronouns” substitute for nouns or noun phrases. Possessive “pronouns” specifically substitute for a possessed object \xref{ex:poss}. 

\ea\label{ex:poss}
\ea 
 \gll Tęˀ gęh neˀ \exemph{desá:węh}? \\
Not Q the it.isn’t.yours\\
\glt ‘Isn’t that yours?’

\ex 
 \gll Ne:ˀ hne:ˀ \exemph{gó:węh}.\\
it.is in.fact hers\\
\glt ‘That’s hers, in fact.’
\z
\z 

The words functioning as possessive “pronouns” are \textsc{verbs}. The following possessive “pronouns” are described in the \textit{Particle dictionary}, \sectref{ch:particle dictionary}. 

\begin{CayugaRelated}

\item{} \textit{Aga:węh} ‘it's mine’, 'it's ours’, ‘my', 'our’\\
\item{} \textit{Sa:węh} ‘it's yours’ (one person’s), ‘your’\\
\item{} \textit{Ho:węh} ‘it's his’\\
\item{} \textit{Go:węh} ‘it's hers’, 'it's someone’s’\\
\item{} \textit{O:węh} ‘it belongs to it’ (an animal), ‘its’\\
\item{} \textit{Hona:węh} ‘it's theirs’ (males only), ‘it belongs to them’ (males only), ‘their’\\
\item{} \textit{Gona:węh} ‘it's theirs’ (females or mixed group), ‘it belongs to them’ (females or mixed group), ‘their’\\
\item{} \textit{Ona:węh} ‘it's theirs’ (animals), ‘it belongs to them’ (animals), ‘their’
\end{CayugaRelated}


\section{Demonstrative “pronouns”} \label{ch:’Demonstrative pronouns’}
Like other pronouns, demonstrative “pronouns” \xref{ex:demp} also substitute for nouns or noun phrases. Demonstrative “pronouns” additionally provide information about a person’s or object’s relative distance from the speaker. 

\ea\label{ex:demp} 
 \gll Ne:ˀ hęˀ hne:ˀ ho:wę́h \exemph{nę́:-gyęh}. \\
it.is also in.fact it.is.his this.one\\
\glt ‘This one is also his.’
\z

\ea\label{ex:demp2} 
 \gll Sǫ: hne:ˀ nˀáht \exemph{tó:-gyęh}? \\
Who in.fact a.certain.person that.one\\
\glt ‘Who is that?’ (\cite[88]{mithun_watewayestanih_1984}, Eksaˀgó:wah dialogue) 
\z



The words functioning as demonstrative “pronouns” are all \textsc{particles} or \textsc{particle groups}. The following demonstrative “pronouns” are described in the \textit{Particle dictionary}, \sectref{ch:particle dictionary}. 

\begin{CayugaRelated}
    
\item{} \stem{-gyęh} element meaning ‘this, ‘that’\\
\item{} \textit{Gwa̱-toh} ‘that one’, 'just such a one’\\
\item{} \textit{Neˀ toh} ‘that is’, ‘that one’\\
\item{} \textit{Neˀ to: ne:ˀ} ‘that’s the one’\\
\item{} \textit{Nę:} ‘this’, 'these’\\
\item{} \textit{Nę:-dah} ‘this’, 'this way’\\
\item{} \textit{Nę:-gyęh} ‘this one’\\
\item{} \textit{Si:-gyęh} ‘that one over there’\\
\item{} \textit{To, To:, Toh, Tǫ:} ‘that one’\\
\item{} \textit{To:-gyęh} ‘that one’\\
\item{} \textit{Toh-jih} ‘just the one’, 'just such a one’\\
\end{CayugaRelated}


\section{Definite “pronouns”} \label{ch:’Definite pronouns’}
Like other pronouns, definite “pronouns” also substitute for nouns or noun phrases. In addition, definite “pronouns” are used when the object or person (etc.) is known to both speaker and listener. The words functioning as definite “pronouns” are all \textsc{particles} or \textsc{particle groups}. The following definite “pronouns” are described in the \textit{Particle dictionary}, \sectref{ch:particle dictionary}. 

\begin{CayugaRelated}
\item{} \textit{Dęˀ gwaˀ hoˀdęˀ} ‘whatever one (of several)’\\
\item{} \textit{Gaę} ‘which’\\
\item{} \textit{Gaę gwaˀ ni-} ‘whichever one (of several)’, 'whichever person’\\
\item{} \textit{Hoˀdęˀ} ‘kind’\\
\item{} \textit{Shęh hoˀdęˀ} ‘a certain something’, 'that which’\\
\end{CayugaRelated}


\section{Indefinite “pronouns”} \label{ch:’Indefinite pronouns’}
Like other pronouns, indefinite “pronouns” also substitute for nouns or noun phrases. In addition, indefinite “pronouns” are used when the object or person (etc.) is not known to the speaker. The indefinite “pronouns” are \textsc{particles}, \textsc{particle groups}, or combinations of \textsc{verb+particle}. The following indefinite “pronouns” are described in the \textit{Particle dictionary}, \sectref{ch:particle dictionary}. 


\begin{CayugaRelated}
\item{} \textit{Do: gwaˀ ni:yǫ:} ‘a certain amount’, 'a certain measure’, 'however much’\\
\item{} \stem{-ga:ˀ} element (referring to living beings)\\
\item{} \textit{Gwe:gǫh, Agwe:gǫh, Ogwe:gǫh} ‘all’, 'everything’\\
\item{} \textit{Hegwe:gǫh, Neˀ hegwe:gǫh} ‘the whole thing’\\
\item{} \textit{Hejo:yaˀ tsǫ:} ‘another thing again’\\
\item{} \stem{-nˀaht, -noht} element meaning ‘some person’\\
\item{} \textit{O:yaˀ} ‘another’, 'other’, 'else’\\
\item{} \textit{Sga̱hoˀdę:ˀęh} ‘something’\\
\item{} \textit{Sga̱hoˀdę:ˀęh o:yaˀ} ‘anything else’\\
\item{} \textit{Sǫ:} ‘some (person, thing)’, 'any (person, thing)’\\
\item{} \textit{Sǫ: gwaˀ nˀaht} ‘anybody at all’, 'anyone at all’, 'any living thing’, 'somebody’, 'whoever’\\
\item{} \textit{Sǫ: gwaˀ nˀaht o:yaˀ} ‘someone else’\\
\item{} \textit{Sǫ:-ga:ˀ} ‘any one’, 'any living thing’\\
\item{} \textit{Sǫ:-ga:ˀah} ‘someone’, 'anyone’, 'anything (living)’\\
\end{CayugaRelated}

\section{Interrogative “pronouns”} \label{ch:’Interrogative pronouns’}
Like other pronouns, interrogative “pronouns” substitute for nouns or noun phrases. However, interrogative “pronouns” are used in questions, while the “pronouns” described elsewhere are used in statements. The words functioning as interrogative “pronouns” are \textsc{particles}. The following interrogative “pronouns” are described in the \textit{Particle dictionary}, \sectref{ch:particle dictionary}.

\begin{CayugaRelated}
\item{} \textit{Dęˀ} ‘what’, 'how’\\
\item{} \textit{Gaę} ‘which’\\
\item{} \textit{Sǫ:} ‘some (person, thing)’, 'any (person, thing)’\\
\end{CayugaRelated}


\section{Negative “pronouns”} \label{ch:’Negative pronouns’}
Several “pronouns” have negative counterparts, formed by adding \textit{tęˀ} ‘not’ before the “pronoun”. The following negative “pronouns” are described in the \textit{Particle dictionary}, \sectref{ch:particle dictionary}.

\begin{CayugaRelated}
\item{} \textit{Tęˀ gwa̱ˀ-toh} ‘none at all’\\
\item{} \textit{Tęˀ giˀ ni:ˀ} ‘no, not me’\\
\item{} \textit{Tęˀ i:ˀ} ‘not me’, ‘not us’\\
\item{} \textit{Tęˀ sga̱hoˀdęˀ} ‘nothing’, 'not anything’\\
\item{} \textit{Tęˀ sǫ:-ga:ˀ} 'nobody’, 'no one’\\
\end{CayugaRelated}


\section{Reflexive and reciprocal “pronouns”} \label{ch:’Reflexive’ and ‘reciprocal pronouns’}
The function of reflexive “pronoun” (meaning ‘oneself’) is performed by the \linebreak \stem{adad-} \textsc{reflexive} prefix (\sectref{[adad-] (reflexive or reciprocal)}), and the function of reciprocal “pronoun” (meaning ‘one another’) is performed by the combination of the \stem{de-} \textsc{dualic} (\sectref{[de-] (dualic)}) and \stem{adad-} \textsc{reciprocal} prefixes (see \sectref{The reflexive or reciprocal prefix and roles}).\footnote{The \stem{adad-} prefix  has two linguistic names, depending on how it is used.} The \stem{adad-} prefix is used with verbs that express two roles, such as the perceiver and the percept, or the “doer” and undergoer. (For roles, see \sectref{Role and the reflexive and semireflexive prefixes}.)

The \stem{-adad} {\reflexive} prefix means that the verb’s two roles are fulfilled by the same person or group: for example, in \xref{ex:rflp} both the perceiver and the perceived are the same person.

\ea\label{ex:rflp} \stem{-adad} {\reflexive}\\
ag\exemph{adad}é:gęˀ ‘I saw myself’ (with {\reflexive})
\cfex{agé:gęˀ ‘I saw (it)’ (no {\reflexive})}
\z


The \stem{de-} \textsc{\dualic} and \stem{-adad} {\reciprocal} combination denotes an action that affects all of the participants in a similar or reciprocal manner: for example, in \xref{ex:rflp2}, everyone benefits equally.

\ea\label{ex:rflp2} \stem{-adad} {\reciprocal} (pronounced as \stem{-adag} in this example)\\
\exemph{de}dw\exemph{adag}yénawahs ‘let us all help one another’ (with \textsc{\dualic} and {\reciprocal})
\cfex{dwayénawahs ‘let all of us help (it)’ (no \textsc{\dualic} or {\reciprocal})}
\z


\section{Personal “pronouns” (pronominal prefixes)} \label{ch:’Personal pronouns’ (pronominal prefixes)}
Like other pronouns, personal “pronouns” substitute for nouns or noun phrases. However, personal “pronouns” are not emphatic in meaning, which distinguishes them from emphatic “pronouns” such as \textit{háǫhęˀ} ‘he alone’ (see \sectref{ch:’Emphatic pronouns’}.)

The function of personal “pronouns” is performed by the pronominal prefixes (bolded in \ref{ex:perp}). Pronominal prefixes are required by all verbs and most nouns (except the atypical words described in \sectref{Atypical words}).

\ea\label{ex:perp}
\ea \exemph{swa}hnę́:ye:s ‘you all are tall’
\ex \exemph{ha}hnę́:ye:s ‘he is tall’ 
\ex \exemph{e}hnę́:ye:s ‘she or someone is tall’
\ex \exemph{ga̱}hnę́:ye:s ‘it (animal) is tall’
\z
\z



\subsection{Relative “pronouns”} \label{ch:’Relative pronouns’}
The pronominal prefixes described in the previous section may sometimes function as relative “pronouns” in a specific type of noun phrase (a group of related words, functioning as a “noun”). The noun phrase in question  consists of a “noun” and a \textsc{dependent clause}  (a type of sentence) that modifies the meaning of the “noun” in some way (see \sectref{Independent and dependent clauses, relative clauses}). For example in \xref{ex:relp}, the meaning of \textit{neˀ hagehsotgę́hę:ˀ} ‘my late grandfather’ is further specified by the dependent clause \textit{to-gyę́h dahayagę́ˀ} ‘he was going out from there’ (shown in square brackets). 

The pronominal prefix in the dependent clause creates a link between the clause and the noun it modifies, thus relating the noun to the clause. In example \xref{ex:relp}, the prefix \stem{ha-} \textsc{3s.m.a} in \textit{dahayagęˀ} ‘he went out’ performs this function.

\ea\label{ex:relp}
 \gll …neˀ hagehsotgęhę́:ˀ to-gyę́h da\exemph{ha}yagę́ˀ…\\
The my.late.grandfather there he.went.out…\\
\glt …my late grandfather, (\exemph{who}) was going out from there… (Henry, 2005)
\z

