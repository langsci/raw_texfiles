\chapter{Pronominal prefix choice for stative-only verbs} \label{Pronominal prefix choice for stative-only verbs}
\textsc{Stative-only} verbs only occur in the stative aspect, and describe a state, property, or attribute. They are single-role verbs (\sectref{Pronominal prefixes and role}), taking either \textsc{a}- or \textsc{p}-series pronominal prefixes. \tabref{figtab:1:pponeaspv} (page \pageref{figtab:1:pponeaspv}) summarizes the factors influencing pronominal prefix choice for stative-only verbs. 

\begin{table}
\small
\caption{Pronominal prefix selection, single-aspect verbs.\label{figtab:1:pponeaspv}}
\begin{tabularx}{\textwidth}{lcQQQQ}

\lsptoprule
stative-only & P?\footnote{“P” denotes possession.} & subtypes & without \textsc{ni}\footnote{Noun Incorporation} & with \textsc{ni} & inc noun type\\\midrule
personal & \ding{55} &  & \stem{\textsc{a}-} or \stem{\textsc{p}-} &  & \\
\addlinespace
\midrule
neuter & \ding{55} & \stem{V}, no NI

\stem{NV} (fixed)
& \stem{ga-} or \stem{o-} &  & \\
\addlinespace
\midrule
neuter & \ding{55} & \stem{(N)-V} 

\stem{+V} & \stem{o-V} 

n/a & \stem{ga-N-V}

\stem{o-N-V} & \stem{ga-N}

\stem{o-N}\\
&&&&&\\
&  &  & \stem{ga-V} & \stem{ga-N-V} & \stem{ga-/o-N} \\
\addlinespace
\midrule
neuter & \ding{55} & \stem{(N)-V} 

(positional V) & \stem{ga-/o-V} 

 & \stem{ga-N-V} 

\stem{o-N-V} & \stem{ga-N}

\stem{o-N}\\
&&&&&\\
&  &  & some \stem{ga-V} & \stem{ga-N-V} & \stem{ga-/o-N}\\
\addlinespace
\midrule
neuter & \ding{55} & \stem{N+V} 

(counting V) &  & \stem{ga-N-V} 

\stem{o-N-V} 

\stem{ga-/o-N-V}
& \stem{ga-N}

\stem{o-N}

\stem{o-N}\\
\addlinespace
\midrule
neuter & \ding{51} & \stem{(N)-V} 

(possession) & \stem{\textsc{a}-/\textsc{p}-V}  & \stem{\textsc{p}-N-V} & {basic noun or detachable body part noun}\\
&&&&&\\
&  &  &  & \stem{\textsc{a}-N-V} & {non-detachable body part noun}\\
\addlinespace
\midrule
neuter & \ding{51} &  \stem{N+o:t} ‘have’

\stem{N+ęˀ} ‘have’ &  & \stem{\textsc{p}-N+o:t/+ęˀ} & {basic noun or detachable body part noun}\\
&&&&&\\
&  &  &  & \stem{\textsc{a/p}-N+o:t/+ęˀ} & {non-detachable body part noun}\\
\lspbottomrule
\end{tabularx}
\end{table}

The two main types of stative-only verbs are \textsc{personal} and \textsc{neuter} verbs (\textit{stative-only} column in \tabref{figtab:1:pponeaspv}). Personal verbs take a complete set of either \textsc{a}- or \textsc{p}-series pronominal prefixes, while neuter verbs only take \stem{ga-} \textsc{3s.a} or \stem{o-} \textsc{3s.p} prefixes.

Three subtypes of stative-only verbs are also shown in \tabref{figtab:1:pponeaspv}, (\textit{subtype} column): \stem{NV} refers to fixed expressions, where the incorporated noun is a fixed part of the verb’s meaning (see \sectref{Verbs and noun incorporation}). In terms of pronominal prefix choice, \stem{NV} fixed expressions are identical to non-incorporating verbs, \stem{V}. \stem{(N)-V} refers to verbs that optionally take an incorporated noun. Finally, \stem{+V} refers to verbs that require incorporated nouns (see \sectref{Verbs and noun incorporation}.)

Pronominal prefix choice for personal verbs (both \stem{V} and fixed \stem{NV} types) must be memorized: some verbs take \textsc{a}-series and some take \textsc{p}-series prefixes. For personal verbs, noun incorporation does not affect pronominal prefix choice. 

Pronominal prefix choice for neuter verbs \emph{without} incorporation (both \stem{V} and fixed \stem{NV} types) must also  be memorized: some begin with \stem{ga-} \textsc{3s.a} and some begin with \stem{o-} \textsc{3s.p}.

In contrast, for \emph{incorporating} neuter verbs (both optionally-incorporating \stem{(N)-V} and obligatorily-incorporating \stem{+V} types), pronominal prefix selection is influenced both by the type of incorporated noun, and also by possession (the incorporated noun is ‘owned’). All of these factors are described in the following sections.


\section{Personal stative-only verbs and pronominal prefix choice} \label{Personal stative-only verbs and pronominal prefix choice}
\textsc{Personal} stative-only verbs take either the \textsc{a}-series or the \textsc{p}-series of pronominal prefix. The choice has to be memorized for each verb. For example, the verb in \xref{ex:adjvbppchoicex13a} exclusively takes \textsc{a}-series prefixes, and the verb in \xref{ex:adjvbppchoicex13b} requires \textsc{p}-series prefixes.  

\ea\label{ex:adjvbppchoicex13} personal stative-only verbs
\ea\label{ex:adjvbppchoicex13a} sewáihǫh\\
\gll se-wáihǫh\\
 \textsc{2s.a}-good.at.something.{\stative}\\
\glt `you are good at something'
\ex agáhshę:\\\label{ex:adjvbppchoicex13b}
\gll ag-áhshę:\\
 \textsc{1s.p}-slow.moving.{\stative}\\
\glt `I am slow to act'
\z
\z

Incorporated nouns do not influence the choice of prefix for personal, stative-only verbs. For example, while both verbs in \xref{ex:adjvbppchoicex14} have the same incorporated noun, the verb in \xref{ex:adjvbppchoicex14a} takes an \textsc{a}-series prefix and the verb in \xref{ex:adjvbppchoicex14b} takes a \textsc{p}-series prefix. Similarly, the verbs in \xref{ex:adjvbppchoicex15} have either an \textsc{a}- or \textsc{p}-series prefix, while the incorporated noun \stem{ǫgweˀd} ‘person’ is the same in both cases.

\ea\label{ex:adjvbppchoicex14} \stem{\textsc{a/p}-haˀd-stative.only.verb}
\ea shaˀdá:tęhs\\\label{ex:adjvbppchoicex14a}
\gll s-haˀd-á:-tęhs\\
 \textsc{2s.a}-throat-{\joinerA}-dried.out. {\habitual} \\
\glt `you are thirsty' (\stem{NV} fixed expression)
\ex go̱haˀdí:yo: \\\label{ex:adjvbppchoicex14b}
\gll go̱-haˀd-í:yo:\\
 \textsc{3s.fi.p}-throat-good.{\stative} \\
\glt ‘she is a good singer’, `she has a good voice' (\stem{NV} fixed expression)
\z
\z

\ea\label{ex:adjvbppchoicex15} \stem{\textsc{a/p}-ǫgweˀd-stative.only.verb}
\ea hǫgweˀdí:yo:\\
\gll h-ǫgweˀd-í:yo:\\
 \textsc{3s.m.a}-person-good.{\stative}  \\
\glt `he is a charming or nice person' (obligatorily-incorporating \stem{+V})
\ex agǫgwe̱ˀdáhetgęˀ\\
\gll ag-ǫgwe̱ˀd-á-hetgęˀ\\
 \textsc{3s.fi.p}-person-ugly.{\stative}  \\
\glt `she is a cruel, mean person' (optionally-incorporating \stem{(N)+V})
\z
\z

\section{Neuter stative-only verbs and pronominal prefix choice} \label{Neuter stative-only verbs and pronominal prefix choice}
For neuter stative-only verbs, pronominal prefix choice is influenced by the type of incorporated noun and by possession (whether or not the incorporated noun is ‘owned’ -  see \tabref{figtab:1:pponeaspv}, page \pageref{figtab:1:pponeaspv}.) These factors are described next.

\subsection{Neuter stative-only verbs without an incorporated noun, and NV fixed expressions} \label{Neuter stative-only verbs without an incorporated noun, and NV fixed expressions}
Neuter stative-only verbs \emph{without} an incorporated noun take either \textsc{a}-series \linebreak \stem{ga-, w-} \textsc{3s.a} prefixes or the \textsc{p}-series \stem{o-} \textsc{3s.p} prefix. The type of prefix has to be memorized. For example, the verbs in \xref{ex:adjvbppchoicex3} require a \stem{ga-} or \stem{w-} \textsc{3s.a} prefix, and the verbs in \xref{ex:adjvbppchoicex4} require an \stem{o-} \textsc{3s.p} prefix.

\ea\label{ex:adjvbppchoicex3} neuter stative-only verbs, no noun incorporation, with \stem{ga-, w-} \textsc{3s.a} prefix
\ea gashá:sdeˀ\\
\gll ga-shá:sdeˀ\\
 \textsc{3s.a}-strong.{\stative}\\
\glt `it is strong'
\ex ganǫ:ˀ\\
\gll ga-nǫ:ˀ\\
 \textsc{3s.a}-costly.{\stative}\\
\glt `it is expensive, dear, precious'
\ex ga̱hǫ́ˀji:\\
\gll ga̱-hǫ́ˀji:\\
 \textsc{3s.a}-dark.coloured.{\stative}\\
\glt `it is dark-coloured'
\ex wagyé:sęh\\
\gll w-ag-yé:s-ęh\\
 \textsc{3s.a}-{\semireflexive}-easy-{\stative}\\
\glt `it is easy'
\z
\z

\ea\label{ex:adjvbppchoicex4} neuter stative-only verbs, no noun incorporation, with \stem{o-} \textsc{3s.a} prefix
\ea ohní:yǫh\\
\gll o-hní:yǫh\\
 \textsc{3s.p}-hard.{\stative}\\
\glt `it is hard'
\ex ogáhdeh\\
\gll o-gáhdeh\\
 \textsc{3s.p}-raw.{\stative}\\
\glt `it is raw'
\ex onáˀno:ˀ\\
\gll o-náˀno:ˀ\\
 \textsc{3s.p}-cold.{\stative}\\
\glt `it is cold, cool'
\z
\z

\stem{NV} fixed expressions (ones that are neuter, stative-only verbs) are like the above verbs. For example, the ones in \xref{ex:adjvbppchoicex5} require a \stem{ga-} \textsc{3s.a} prefix (or a \stem{w-} \textsc{3s.a} prefix, \ref{ex:adjvbppchoicex5b}), and the ones in \xref{ex:adjvbppchoicex6} require an \stem{o-} \textsc{3s.p} prefix. Moreover, examples \xref{ex:adjvbppchoicex5a} and \xref{ex:adjvbppchoicex6a}, which both incorporate \stem{hneg} ‘water’, show that the type of incorporated noun does not affect prefix choice in these \stem{NV} fixed expressions.

\ea\label{ex:adjvbppchoicex5} \stem{NV} fixed expressions with \stem{ga-, w-} \textsc{3s.a} prefix
\ea ga̱hné:ga:t\\\label{ex:adjvbppchoicex5a}
\gll ga̱-hné:g-a:-t\\
 \textsc{3s.a}-water-{\joinerA}-stand.{\stative}\\
\glt `it is watery'
\ex wa̱hsóhǫ:t\\\label{ex:adjvbppchoicex5b}
\gll w-a̱hsóh-ǫ:t \\
 \textsc{3s.a}-dye-attached.{\stative}\\
\glt `it is dyed'
\ex gadręnagáˀǫh\\\label{ex:adjvbppchoicex5c}
\gll ga-dręn-a-gáˀǫh\\
 \textsc{3s.a}-smell-{\joinerA}-good.tasting.{\stative}\\
\glt `it smells good, sweet, appetizing'
\ex gahsdaowá:nęh\\\label{ex:adjvbppchoicex5d}
\gll ga-hsda:-owá:nęh\\
 \textsc{3s.a}-rain-big.{\stative}\\
\glt `it is raining hard'
\z
\z

\ea\label{ex:adjvbppchoicex6} \stem{NV} fixed expressions with \stem{o-} \textsc{3s.p} prefix
\ea ohneganá:wę:\\\label{ex:adjvbppchoicex6a}
\gll o-hneg-a-ná:wę:\\
 \textsc{3s.p}-water-{\joinerA}-wet.{\stative}\\
\glt `lead, lukewarm water'
\ex oihwí:yo:\\\label{ex:adjvbppchoicex6b}
\gll o-ihw-í:yo:\\
 \textsc{3s.p}-matter-good.{\stative}\\
\glt `it is for certain, sure'
\z
\z

Positional verbs (a thematic subtype of neuter, stative-only verb) \emph{without} an incorporated noun are also like the above verbs. For example, the ones in \xref{ex:positppchoiceex2} require a \stem{ga-} or \stem{w-} \textsc{3s.a} prefix, and the ones in \xref{ex:positppchoiceex3} require an \stem{o-} \textsc{3s.p} prefix.

\ea\label{ex:positppchoiceex2} positional verbs, no noun incorporation, with \stem{ga-} \textsc{3s.a} prefix,
\ea degáhǫˀ\\
\gll de-gá-hǫˀ\\
 {\dualic}-\textsc{3s.a}-lie.across.{\stative}\\
\glt `something is lying across a path, a door, etc.'
\ex gá:yęˀ\\
\gll gá:-yęˀ\\
 \textsc{3s.a}-lie.{\stative}\\
\glt `it is lying on the ground'
\ex gadę́hda:ˀ\\
\gll ga-dę́hda:ˀ\\
 \textsc{3s.a}-spread.out.on.ground.{\stative}\\
\glt `it is lying spread out on the floor or the ground'
\ex gagéhǫˀ\\
\gll ga-géhǫˀ\\
 \textsc{3s.a}-lie.about. {\stative}\\
\glt `things are lying about, around'
\z
\z

\ea\label{ex:positppchoiceex3} positional verbs, no noun incorporation, with \stem{o-} \textsc{3s.p} prefix
\ea áǫt \\
\gll á-ǫt\\
 \textsc{3s.p}-stick.out.{\stative}\footnotemark{}\\
\glt ‘it is attached’, `it is sticking out'
\footnotetext{In \textit{áǫt} the \textsc{3s.p} prefix is \stem{a-}, which is the regular pronunciation of this prefix before o-stems.}
\ex owaˀnę́:da:ˀ\\
\gll o-waˀnę́:da:ˀ\\
 \textsc{3s.p}-stuck.{\stative}\\
\glt `it is stuck'
\ex odáhǫh\\
\gll o-dáhǫh\\
 \textsc{3s.p}-draped.{\stative}\\
\glt `it is draped'
\z
\z

\subsection{Neuter stative-only verbs, with noun incorporation} \label{Neuter stative-only verbs, with noun incorporation}
Noun incorporation can influence pronominal prefix choice for neuter stative-only verbs. In addition, \stem{o-V} verbs and \stem{ga-V} verbs behave differently with respect to noun incorporation.

\subsection{\stem{o-V} and \stem{+V}, neuter stative-only verbs with noun incorporation} \label{[o-V] and obligatorily incorporating neuter stative-only verbs with noun incorporation}
Neuter stative-only verbs that begin with \stem{o-} \textsc{3s.p} (\stem{o-V}, \ref{ex:adjvbppchoicex7}), and those that are obligatorily incorporating (\stem{+V}, \ref{ex:adjvbppchoicex8}), tend to take the same type of pronominal prefix as the non-incorporated noun. For example, the \stem{o-V} verb \textit{oˀdáihę:} \xref{ex:adjvbppchoicex7} takes \stem{ga-} when it incorporates \stem{hsǫwahd} ‘wire’ -- a noun with a \stem{ga-} prefix. Meanwhile, the same verb takes \stem{o-} when it incorporates \stem{ˀnehs} ‘sand’ -- a noun with an \stem{o-} prefix.

\ea\label{ex:adjvbppchoicex7} \stem{o-V}\\
\ea oˀdáihę: \\
\gll o-ˀdáih-ę:\\ 
\textsc{3s.p}-hot-{\stative} \\
\glt ‘it is hot’ 

\ex ga̱hsǫwa̱hdadáihę:\\
\gll ga̱-hsǫwa̱hd-a-dáih-ę:\\
 \textsc{3s.a}-wire-{\joinerA}-hot-{\stative}\\
\glt `hot plate'
\cfex{ga̱hsǫ́wa̱hdaˀ\\
	\gll ga̱-hsǫ́wa̱hd-aˀ\\
	\textsc{3s.a}-wire-{\nounstemformer}\\
	\glt `wire, needle, nails'}
\ex oˀnehsadaihę:\\
\gll o-ˀnehs-a-dáih-ę:\\
 \textsc{3s.p}-sand-{\joinerA}-hot-{\stative}\\
\glt `desert'
\cfex{oˀnéhsaˀ\\
	\gll o-ˀnéhs-aˀ\\
	\textsc{3s.p}-sand-{\nounstemformer}\\
	\glt `sand'}
\z
\z

The facts are similar for the obligatorily incorporating verb in \xref{ex:adjvbppchoicex8}.

\ea\label{ex:adjvbppchoicex8} \stem{{}+owanęh} ‘big’ (obligatorily incorporating neuter verb)\\
\ea ga̱hǫhsrowá:nęh\\
\gll ga̱-hǫhsr-owá:n-ęh\\
 \textsc{3s.a}-box-big-{\stative}\\
\glt `big box'
\cfex{gahǫ́hsraˀ\\
	\gll ga-hǫ́hsr-aˀ \\
	\textsc{3s.a}-box-{\nounstemformer}\\
	\glt `a box'}
\ex ojǫˀdowá:nęh\\
\gll o-jǫˀd-owá:n-ęh\\
 \textsc{3s.p}-fish-big-{\stative}\\
\glt `big fish'
\cfex{ojǫ́ˀdaˀ\\
	\gll o-jǫ́ˀd-aˀ\\
	\textsc{3s.p}-fish-{\nounstemformer}\\
	\glt `a fish'}
\z
\z


\subsection{Neuter stative-only positional verbs, with incorporated noun} \label{Neuter stative-only positional verbs, with incorporated noun}
Positional verbs are a thematic subtype of neuter, stative-only verbs describing the position of an object. Most positional verbs take the same type of pronominal prefix as the noun that they incorporate. For example, \textit{wadíhǫh} ‘it is leaning’ takes the same \stem{ga-} prefix as the noun \stem{hǫw} ‘boat’ in \xref{ex:positppchoiceex4a}, and takes the same \stem{o-} prefix as the noun \stem{hah} ‘road’ in \xref{ex:positppchoiceex4b}.

\newpage
\ea\label{ex:positppchoiceex4} 
\ea wadíhǫh \\\label{ex:positppchoiceex4a} 
\gll w-adíh-ǫh\\
\textsc{3s.a}-lean-{\stative}\\
\glt ‘it is leaning’ 
\ex ga̱hǫwadíhǫh\\\label{ex:positppchoiceex4b} 
\gll ga̱-hǫw-adíh-ǫh\\
 \textsc{3s.a}-boat-lean-{\stative}\\
\glt `a leaning boat'
\cfex{ga̱hǫ́:waˀ\\
	\gll ga̱-hǫ́:w-aˀ\\
	\textsc{3s.a}-boat-{\nounstemformer}\\
	\glt `a boat'}
\ex ohahaˀdíhǫh\\\label{ex:positppchoiceex4c} 
\gll o-hah-adíh-ǫh\\
 \textsc{3s.p}-road-lean-{\stative}\\
\glt `the Milky Way'
\cfex{oháhaˀ\\
	\gll o-háh-aˀ\\
	\textsc{3s.p}-road-{\nounstemformer}\\
	\glt `road'}
\z
\z

Similar examples are shown in \xref{ex:positppchoiceex5}.

\ea\label{ex:positppchoiceex5} 
\ea áǫt \\
\gll á-ǫt\\
 \textsc{3s.p}-stick.out.{\stative}\\
\glt ‘it is attached’, `it is sticking out'
\cfex{węˀníhsgaǫt\\
	\gll w-ęˀníhsg-a-ǫt\\
	\textsc{3s.a}-wheel-{\joinerA}-stick.out.{\stative}\\
	\glt `an attached wheel'}
\cfex{ohsgyę́ˀdǫ:t\\
	\gll o-hsgyę́ˀd-ǫ:t\\
	\textsc{3s.p}-bone-stick.out.{\stative}\\
	\glt `bone'}

\ex gá:hǫh\\
\gll gá-:hǫh\\
 \textsc{3s.a}-covered.{\stative}\\
\glt `it is covered'

\cfex{ga̱hnegá:hǫh\\
    \gll ga̱-hneg-á-:hǫh \\
    \textsc{3s.a}-water-{\joinerA}-covered.{\stative}\\
    \glt ‘Head Lead on a Stick’ (snowsnake game vocabulary)
    }
\cfex{onyęˀgwáohǫh\\
	\gll o-nyęˀgw-á-ohǫh\\
	\textsc{3s.p}-drifted.snow-{\joinerA}-covered.{\stative}\\
	\glt `drifted snow'}

\ex nitgá:deˀ \\
\gll ni-t-gá:-deˀ\\
 {\partitive}-{\cislocative}-\textsc{3s.a}-exist.{\stative}\\
\glt ‘how high it is’ (inanimate object), `the height of something'

\cfex{ga̱hwajiyá:deˀ\\
\gll ga̱-hwajiy-á:-deˀ\\
\textsc{3s.a}-family-{\joinerA}-exist.{\stative}\\
\glt ‘a family’ (matrilineal)
}
\cfex{ohá:deˀ\\
	\gll o-há:-deˀ\\
	\textsc{3s.p}-road-exist.{\stative}\\
	\glt `an existing road'}
\ex gadę́hda:ˀ\\
\gll ga-dę́hda:ˀ\\
 \textsc{3s.a}-lie.spread.out.{\stative}\\
\glt `it is lying spread out on the floor or the ground'
\cfex{ganehsdá:dę̱hda:ˀ\\
	\gll ga-nehsdá:-dę̱hda:ˀ\\
	\textsc{3s.a}-board-lie.spread.out.{\stative}\\
	\glt `a floor'}
\cfex{owidradę́hda:ˀ\\
	\gll o-widr-a-dę́hda:ˀ\\
	\textsc{3s.p}-ice-lie.spread.out.{\stative}\\
	\glt `ice patch'}
\ex ga̱he:ˀ\\
\gll ga̱-he:ˀ\\
 \textsc{3s.a}-sit.up.on.top.{\stative}\\
\glt  ‘it is sitting up on top of something’, `it is sitting here'
\cfex{ganáˀja̱he:ˀ\\
	\gll ga-náˀj-a̱-he:ˀ\\
	\textsc{3s.a}-pail-{\joinerA}-sit.up.on.top.{\stative}\\
	\glt `pail setting on something'}
\newpage
\cfex{ohsgę́ˀdra̱he:ˀ\\
	\gll o-hsgę́ˀdr-a̱-he:ˀ\\
	\textsc{3s.p}-rust-{\joinerA}-sit.up.on.top.{\stative}\\
	\glt `it is rusting, rusty'}
\ex gá:yęˀ\\
\gll gá:-yęˀ\\
 \textsc{3s.a}-lie.on.ground.{\stative}\\
\glt `it is lying on the ground'
\cfex{ganǫ́ˀaęˀ\\
	\gll ga-nǫ́ˀa:-ęˀ\\
	\textsc{3s.a}-head-lie.on.ground.{\stative}\\
	\glt `a head lying on the ground'}
\cfex{ohǫ́daęˀ\\
	\gll o-hǫ́d-a-ęˀ\\
	\textsc{3s.p}-sapling-{\joinerA}-lie.on.ground.{\stative}\\
	\glt `a bush, shrub'}
\ex degáhǫˀ\\
\gll de-gá-hǫˀ\\
 {\dualic}-\textsc{3s.a}-lie.across.{\stative}\\
\glt `something is lying across a path, a door, etc.'
\cfex{dewahǫhdáhǫˀ\\
	\gll de-w-ahǫhd-á-hǫˀ\\
	{\dualic}-\textsc{3s.a}-ear-{\joinerA}-lie.across.{\stative}\\
	\glt `jackass, donkey'}
\cfex{odóˀda̱hǫˀ\\
	\gll o-dóˀd-a̱-hǫˀ\\
	\textsc{3s.p}-wave-{\joinerA}-lie.across.{\stative}\\
	\glt `little waves'}
\ex gahnyo:t\\
\gll ga-hny-o:t\\
 \textsc{3s.a}-stick-stand.{\stative}\\
\glt `it is standing'
\cfex{ga̱há:do:t\\
	\gll ga̱-há:d-o:t\\
	\textsc{3s.a}-forest-stand.{\stative}\\
	\glt `forest'}
\cfex{odrǫ́hyo:t \\
	\gll o-drǫ́hy-o:t\\
	\textsc{3s.p}-beam.of.light-stand.{\stative}\\
	\glt ‘sunbeam’, ‘ray of light’, `sunshine'}
\z
\z

The obligatorily-incorporating positional verbs in \xref{ex:positppchoiceex18} also behave like the verbs in \xref{ex:positppchoiceex5}.

\ea\label{ex:positppchoiceex18} 
\ea {}\stem{{}+itgęˀǫh} ‘come out, issue out, rise from, emanate from’
\ea 
gawęnitgę́ˀǫ:ˀ\\
	\gll ga-węn-itgę́ˀǫ-:-ˀ\\
	\textsc{3s.a}-word-emanate.from-{\nominalizer}-{\stative}\\
	\glt `a speech'
\ex gyotsaditgę́ˀǫh\\
	\gll g-yo-t-hsad-itgę́ˀ-ǫh\\
	{\cislocative}-\textsc{3s.p}-{\semireflexive}-fog-emanate.from-{\stative}\\
	\glt `steam coming out'
 \z 
\ex {}\stem{{}+kahǫ} ‘adjoin, abut’
\ea sgahsakáhǫ:ˀ\\
	\gll s-ga-hs-a-káhǫ-:ˀ\\
	{\repetitive}-\textsc{3s.a}-mouth-{\joinerA}-adjoin.\stat-\nominalizer{}\\
	\glt `large mouth bass'
\ex ohyákahǫˀ\\
	\gll o-hy-á-kahǫˀ\\
	\textsc{3s.p}-berry-{\joinerA}-adjoin.{\stative}\\
	\glt `tomatoes'
\z
\z
\z 

The obligatorily-incorporating positional verbs in \xxref{ex:positppchoiceex6}{ex:positppchoiceex603} may be like the verbs in \xref{ex:positppchoiceex18}. However, only one type of incorporated noun is shown in each case, and so the evidence is inconclusive.
 
\ea\label{ex:positppchoiceex6} \stem{{}+hsiha:ˀ} ‘stand in a bunch or group’\\
ojihsǫdáhsi̱ha:ˀ\\
\gll o-jihsǫd-á-hsi̱ha:ˀ\\
\textsc{3s.p}-star-{\joinerA}-stand.in.group.{\stative}\\
\glt ‘stars showing’ (a group of stars)

\cfex{ojihsǫ́:da:ˀ\\
	\gll o-jihsǫ́:da:-ˀ\\
	\textsc{3s.p}-star-{\nounstemformer}\\
	\glt `a cluster of stars, a star'}
\z


\ea\label{ex:positppchoiceex600} \stem{{}+ęhę:} ‘direction of’
\ea nigyowáęhę:\\
\gll ni-g-yo-wá:-ęhę:\\
 {\partitive}-{\cislocative}-\textsc{3s.p}-wind-direction.of.{\stative}\\
\glt `the direction of the wind'
\cfex{ó:wa:ˀ\\
	\gll ó:-wa:-ˀ\\
	\textsc{3s.p}-air-{\nounstemformer}\\
	\glt `air, wind, a moth'}
\ex heyohnegę́hę:ˀ\\
\gll he-yo-hneg-ę́hę:ˀ\\
 {\translocative}-\textsc{3s.p}-water-direction.of.{\stative}\\
\glt `downstream'
\cfex{ohnéga̱ˀgeh\\
	\gll o-hnég-a̱ˀgeh\\
	\textsc{3s.p}-water-{\on}\\
	\glt `on the water'}
\z
\z

\ea\label{ex:positppchoiceex601} \stem{{}+iˀ} ‘be stuck to something’
\ea oˀnéhsaiˀ\\
\gll o-ˀnéhs-a-iˀ\\
 \textsc{3s.p}-sand-{\joinerA}-stuck.to.{\stative}\\
\glt `it is sandy'
\cfex{oˀnéhsaˀ\\
	\gll o-ˀnéhs-aˀ\\
	\textsc{3s.p}-sand-{\nounstemformer}\\
	\glt `sand'}
\ex ohéhdaiˀ\\
\gll o-héhd-a-iˀ\\
 \textsc{3s.p}-earth-{\joinerA}-stuck.to.{\stative}\\
\glt `earth is stuck to it'
\cfex{ohéhdaˀ\\
	\gll o-héhd-aˀ\\
	\textsc{3s.p}-earth-{\nounstemformer}\\
	\glt `dirt, earth, ground, land'}
\z
\z

\ea\label{ex:positppchoiceex602} \stem{{}+gęhyad} ‘end, edge, top’
\ea oihwagę́hya:t\\
\gll o-ihw-a-gę́hya:t\\
 \textsc{3s.p}-matter-{\joinerA}-edge.of.{\stative}\\
\glt `it is almost to the end'
\cfex{oíhwaˀ \\
	\gll o-íhw-aˀ\\
	\textsc{3s.p}-matter-{\nounstemformer}\\
	\glt ‘message’, ‘it matters’, ‘it is its fault’, `word, affair, business'}
\ex ohnegagę́hya:t\\
\gll o-hneg-a-gę́hya:t\\
 \textsc{3s.p}-water-{\joinerA}-edge.of.{\stative}\\
\glt `just above the water'
\cfex{ohnéga̱ˀgeh\\
	\gll o-hnég-a̱ˀgeh\\
	\textsc{3s.p}-water-{\on}\\
	\glt `on the water'}
\z
\z

\ea\label{ex:positppchoiceex603} \stem{{}+o:t} ‘standing objects’, \stem{{}+od-ǫˀ} ‘several standing objects’
\ea degaiˀę́ho:t\\
\gll de-ga-iˀę́h-o:t\\
 {\dualic}-\textsc{3s.a}-pile-stand.{\stative}\\
\glt `it is piled up'
\ex ga̱hsdęhó:dǫˀ\\
\gll ga̱-hsdęh-ó:d-ǫˀ\\
 \textsc{3s.a}-rock-stand.{\stat}-{\distributive}\\
\glt `mountains, pile of boulders'
\z
\z

Finally, some positional verbs always take a \stem{ga-} \textsc{3s.a} prefix, regardless of what type of prefix the non-incorporated noun takes. They are described in the following section.


\subsection{\stem{ga-V} neuter stative-only verbs with noun incorporation} \label{[ga-V] neuter stative-only verbs with noun incorporation}
Several neuter stative-only verbs that begin with \stem{ga-} \textsc{3s.a} (or \stem{w-} \textsc{3s.a}, not shown) tend to keep their original prefix, regardless of the prefix type of the non-incorporated noun. For example, the verb \textit{ganǫ:ˀ} in \xref{ex:adjvbppchoicex9} still begins with \stem{ga-} when it incorporates \textit{ohwíhsdaˀ}, and similarly for the example in \xref{ex:adjvbppchoicex10}.

\ea\label{ex:adjvbppchoicex9} 
\ea ganǫ:ˀ \\
\gll ga-nǫ:ˀ\\
\textsc{3s.a}-costly.{\stative}\\
\glt ‘it is expensive, dear, precious’ 

\ex ga̱hwíhsdanǫ:ˀ \\
\gll ga̱-hwíhsd-a-nǫ:ˀ\\
 \textsc{3s.a}-metal-{\joinerA}-costly.{\stative}\\
\glt ‘gold’, `anything expensive'
\cfex{ohwíhsdaˀ\\
	\gll o-hwíhsd-aˀ\\
	\textsc{3s.p}-metal-{\nounstemformer}\\
	\glt `money, metal'}
\z
\z

\ea\label{ex:adjvbppchoicex10} 
\ea gahshá:sdeˀ\\
\gll ga-hshá:sdeˀ \\
\textsc{3s.a}-strong.{\stative}\\
\glt ‘it is strong’ 
\ex gawa:shá:sdeˀ\\
\gll ga-wa:-shá:sdeˀ\\
 \textsc{3s.a}-air-strong.{\stative}\\
\glt `strong wind'
\cfex{ó:wa:ˀ\\
	\gll ó:-wa:-ˀ\\
	\textsc{3s.p}-air-{\nounstemformer}\\
	\glt `wind, air'}
\z
\z 


Other examples are shown in \xxref{ex:positppchoiceex7}{ex:positppchoiceex701}.
 
\ea\label{ex:positppchoiceex7} 
\ea gagéhǫˀ \\
\gll ga-géhǫˀ\\
\textsc{3s.a}-lie.about.{\stative}\\
\glt ‘things are lying about’ 
\ex gawidragéhǫˀ\\
\gll ga-widr-a-géhǫˀ\\
 \textsc{3s.a}-ice-{\joinerA}-lie.about.{\stative}\\
\glt `ice scattered'
\cfex{owí:draˀ\\
	\gll o-wí:dr-aˀ\\
	\textsc{3s.p}-ice-{\nounstemformer}\\
	\glt `ice'}
\z
\z

\ea\label{ex:positppchoiceex700} 
\ea í:ga:ˀ \\
\gll í:-ga-:ˀ\\
{\partitive}-\textsc{3s.a}-contain.{\stative}\\
\glt ‘it contains’ 
\ex do: niyǫ́: ga̱hna:ˀ \\
\gll do: niyǫ́: ga̱-hn-a-:ˀ\\
how a.certain.amount \textsc{3s.a}-oil-{\joinerA}-contain.{\stative}\\
\glt ‘how much gas is in there’
\cfex{ohnaˀ \\
	\gll o-hn-aˀ\\
	\textsc{3s.p}-oil-{\nounstemformer}\\
	\glt ‘oil’, `grease'}
\z
\z

\ea\label{ex:positppchoiceex701}
\ea ganí:yǫ:t\\
\gll ga-ní:yǫ:t \\
\textsc{3s.a}-hang.{\stative}\\
\glt ‘it is hanging’ 
\ex gayęhsraní:yǫ:t\\
\gll ga-yęhsr-a-ní:yǫ:t\\
 \textsc{3s.a}-blanket-{\joinerA}-hang.{\stative}\\
\glt `a hanging blanket'
\cfex{oyę́hsraˀ\\
	\gll o-yę́hsr-aˀ\\
	\textsc{3s.p}-blanket-{\nounstemformer}\\
	\glt `a blanket'}
\ex gayahkwaní:yǫ:t\\
\gll ga-yahkw-a-ní:yǫ:t\\
 \textsc{3s.a}-pants-{\joinerA}-hang.{\stative}\\
\glt `hanging pants'
\cfex{oyáhkwaˀ\\
	\gll o-yáhkw-aˀ\\
	\textsc{3s.p}-pants-{\nounstemformer}\\
	\glt `pants'}
\z
\z


\subsection{Neuter stative-only counting verbs, with an incorporated noun} \label{Neuter stative-only counting verbs, with an incorporated noun}
(This topic is also covered in \textit{Counting with basic nouns}, \sectref{ch:Counting with basic nouns}.) 

Two neuter stative-only verbs are used for counting. The relevant fixed expressions are \stem{s-…+t} ‘one object’, and \stem{de-/ni-…+age:} ‘two or more objects’. These expressions always incorporate a basic noun.

When they incorporate a noun beginning with \stem{ga-} \textsc{3s.a}, counting verbs tend to take \stem{ga-} \textsc{3s.a} as well \xref{ex:countingexagain2}.

\ea\label{ex:countingexagain2} \stem{s-…+t} ‘one object’, with \textsc{a}-series incorporated noun, resulting word takes \stem{ga-}\\
sganáˀja:t\\
\gll s-ga-náˀj-a:-t\\
 {\repetitive}-\textsc{3s.a}-drum-{\joinerA}-stand.{\stative}\\
\glt `one bottle'
\cfex{ganáˀja:t\\
	\gll ga-náˀj-a:-t\\
	\textsc{3s.a}-drum-{\joinerA}-stand.{\stative}\\
	\glt `a standing bottle'}
\z


In contrast, when they incorporate a noun beginning with \stem{o-} \textsc{3s.p}, the choice of pronominal prefix can vary: sometimes, such verbs take either \stem{ga-} \textsc{3s.a} or \stem{(y)o-} \textsc{3s.p} \xxref{ex:countvarex}{ex:countvarex2}, while in other cases, such verbs only take \stem{(y)o-} \textsc{3s.p} \xref{ex:countvarex3}.

\ea\label{ex:countvarex} \stem{ni-…age:} ‘a number of objects’, with \textsc{p}-series incorporated noun, resulting word takes either \stem{ga-} or \stem{(y)o-}
\ea ahsę́h niyohǫnáˀdage:\\
\gll ahsę́h ni-yo-hǫnáˀd-age:h\\
 three {\partitive}-\textsc{3s.p}-potato-two.or.more.{\stative}\\
\glt `three potatoes'
\ex ahsę́h nigahǫnáˀdage:\\
\gll ahsę́h ni-ga-hǫnáˀd-age:\\
 three  {\partitive}-\textsc{3s.a}-potato-two.or.more.{\stative}\\
\glt `three potatoes'
\cfex{ohǫ́na̱ˀdaˀ\\
	\gll o-hǫ́na̱ˀd-aˀ\\
	\textsc{3s.p}-potato-{\nounstemformer}\\
	\glt `potatoes'}
\z
\z

\ea\label{ex:countvarex2} \stem{s-… -t} ‘one object’, with \textsc{p}-series incorporated noun, resulting word takes either \stem{ga-} or \stem{(y)o-}
\ea sgayę́hsra:t, \\
\gll s-ga-yę́hsr-a:-t\\
 {\repetitive}-\textsc{3s.a}-blanket-{\joinerA}-stand.{\stative}\\
\glt `one blanket'
\ex joyę́hsra:t\\
\gll j-o-yę́hsr-a:-t\\
 {\repetitive}-\textsc{3s.p}-blanket-{\joinerA}-stand.{\stative}\\
\glt `one blanket'
\cfex{oyę́hsraˀ\\
	\gll o-yę́hsr-aˀ\\
	3s.p-blanket-\nsf\\
	\glt `blanket'}
\newpage
\ex sgahǫ́na̱ˀda:t\\
\gll s-ga-hǫ́na̱ˀd-a:-t\\
 {\repetitive}-\textsc{3s.a}-potato-{\joinerA}-stand.{\stative}\\
\glt `one potato'
\ex jo̱hǫ́na̱ˀda:t\\
\gll j-o̱-hǫ́na̱ˀd-a:-t \\
 {\repetitive}-\textsc{3s.p}-potato-{\joinerA}-stand.{\stative}\\
\glt `one potato'
\cfex{ohǫ́na̱ˀdaˀ\\
	\gll o-hǫ́na̱ˀd-aˀ\\
	\textsc{3s.p}-potato-{\nounstemformer}\\
	\glt `potato'}
\z
\z

\ea\label{ex:countvarex3} \stem{s-… -t} ‘one object’, with \textsc{p}-series incorporated noun, resulting word takes \stem{(y)o-}\\
joháha:t\\
\gll j-o-háh-a:-t\\
 {\repetitive}-\textsc{3s.p}-road-{\joinerA}-stand.{\stative}\\
\glt `one road'
\cfex{oháhaˀ\\
	\gll o-háh-aˀ\\
	\textsc{3s.p}-road-{\nounstemformer}\\
	\glt `road'}
\z


\subsubsection{Navel, belly button, and bottle}
Two easily-confused noun stems \stem{kseˀd} and \stem{(i)tseˀd} appear both in basic noun and body-part noun forms and in fixed expressions. The resulting meanings are illustrated in \xref{ex:posownex11} and \xref{ex:posownex12}. Note that \stem{(i)tseˀd} ‘bottle’ means ‘navel’ only when it takes an \textsc{a}-series prefix \xref{ex:posownex12c}.

\ea\label{ex:posownex11} \stem{kseˀd} in nouns and fixed expressions
\ea sekséˀda̱ˀgeh\\
\gll s-e-kséˀd-a̱ˀgeh\\
 \textsc{2s.a}-{\joinerE}-belly-{\on} \\
\glt `on your belly' (body part noun, \textsc{a} prefix)
\ex gikséˀdo:t\\
\gll g-ikséˀd+o:t\\
 \textsc{1s.a}-belly-attached.{\stative} \\
\glt `my navel, my belly button' (fixed expression, \textsc{a} prefix)
\z
\z

\ea\label{ex:posownex12} \stem{(i)tseˀd} in nouns and fixed expressions
\ea gętséˀdaˀ, gatséˀdaˀ\\\label{ex:posownex12a}
\gll ga/gę-tséˀd-aˀ\\
 \textsc{3s.a}-bottle-{\nounstemformer}  \\
\glt `a bottle' (basic noun, \textsc{a} prefix)
\ex otséˀdo:t\\\label{ex:posownex12b}
\gll o-tséˀd+o:t\\
 \textsc{3s.p}-bottle-stand.{\stative}  \\
\glt `hubbard squash' (fixed expression, \textsc{p}-series prefix)
\ex getséˀdo:t\\\label{ex:posownex12c}
\gll g-e-tséˀd+o:t \\
 \textsc{1s.a}-{\joinerE}-bottle-stand.{\stative} \\
\glt `my navel' (fixed expression, \textsc{a}-series prefix)
\z
\z

\subsection{Neuter stative-only verbs, incorporating, conveying possession} \label{Neuter stative-only verbs, incorporating, conveying possession}
Neuter stative-only verbs without an incorporated noun normally only take \linebreak \stem{ga-} \textsc{3s.a} or \stem{o-} \textsc{3s.p}, as described earlier. However, in order to convey \textsc{possession} or ownership  of an incorporated noun, they will take a full range of either \textsc{a}- or \textsc{p}-series pronominal prefixes. 


The choice of \textsc{a}- or \textsc{p}-series prefix depends on the type of incorporated noun: these verbs take \textsc{p}-series prefixes when the possessed, incorporated noun is a basic noun or detachable body part \xxref{ex:adjvbppchoicex12}{ex:adjvbppchoicex120}.
 
\ea\label{ex:adjvbppchoicex12} 
\ea niwú:ˀuh\\
\gll  niwú:ˀuh\\
{\partitive}-\textsc{3s.a}-small.{\stative}\\
\glt ‘it is small’
\ex nihohkyędahkú:ˀuh\\
\gll ni-ho-hkyędahk-ú:ˀuh\\
 {\partitive}-\textsc{3s.m.p}-chair-small.{\stative}\\
\glt `his small chair'
\cfex{akyę́da̱hkwaˀ\\
	\gll akyę́da̱hkw-aˀ\\
	\textsc{ø.prefix}.chair-{\nounstemformer}\\
	\glt `chair'}
\z
\z

\newpage
\ea\label{ex:adjvbppchoicex120} \stem{{}+es} ‘long’\\
sanóˀje:s\\
\gll sa-nóˀj-e:s\\
 \textsc{1s.p}-tooth-long.{\stative}\\
\glt `your tooth is long'
\cfex{onóˀjaˀ\\
	\gll o-nóˀj-aˀ\\
	\textsc{3s.p}-tooth-{\nounstemformer}\\
	\glt `tooth'}
\z


In contrast, they take \textsc{a}-series prefixes when the possessed, incorporated noun is a non-detachable body part \xxref{ex:adjvbppchoicex11}{ex:adjvbppchoicex110}.  

\ea\label{ex:adjvbppchoicex11} 
\ea niwú:ˀuh\\
\gll niwú:ˀuh\\
{\partitive}-\textsc{3s.a}-small.{\stative}\\
\glt ‘it is small’ 
\ex nihahna̱ˀtsˀú:ˀuh\\
\gll ni-ha-hna̱ˀts-ú:ˀuh\\
 {\partitive}-\textsc{3s.m.a}-buttocks-small.{\stative}\\
\glt `he has a small butt'
\cfex{hahnáˀtsa̱ˀgeh\\
	\gll ha-hnáˀts-a̱ˀgeh\\
	\textsc{3s.m.a}-buttocks-{\on}\\
	\glt `on his buttocks'}
\z
\z

\ea\label{ex:adjvbppchoicex110} \stem{{}+es} ‘long’\\
degénętse:s \\
\gll de-g-é-nęts-e:s\\
 {\dualic}-\textsc{1s.a}-{\joinerE}-arm-long.{\stative}\\
\glt ‘my arms are long’, `I have two long arms'
\cfex{knętsáˀgeh\\
	\gll k-nęts-áˀgeh\\
	\textsc{1s.a}-arm-{\on}\\
	\glt `on my arm'}
\z


\textsc{a}-series prefixes can also be used when a \textsc{basic noun} functions as a “body part noun” \xref{ex:adjvbppchoicex111}.

\newpage
\ea\label{ex:adjvbppchoicex111} 
hahnyǫhsowá:nęh\\
\gll ha-hnyǫhs-owanęh\\
\textsc{3s.m.a}-squash-big.{\stative}\\
\glt ‘he has a big head’ (not flattering)

\cfex{ohnyǫ́hsaˀ\\
	\gll o-hnyǫ́hs-aˀ\\
	\textsc{3s.p}-squash-{\nounstemformer}\\
	\glt `squash'}
\z


Neuter stative-only verbs without an incorporated noun and NV fixed expressions, see \sectref{Neuter stative-only verbs without an incorporated noun, and NV fixed expressions}

\subsubsection{\textit{Gahnyo:t, gá:yęˀ}, incorporating, conveying possession} \label{Gahnyo:t, gá:yęˀ, incorporating, conveying possession}
The following two neuter, stative-only (positional) verbs are commonly used when expressing ownership. The verb \textit{ga̱hnyo:t} ‘it is standing’ tends to be used for owned objects that are normally attached to something else (such as body parts, \ref{ex:posownex}), and \textit{gá:yęˀ} ‘it is lying on the ground’ is used in other cases \xref{ex:posownex2}. When a noun is incorporated, \textit{ga̱hnyo:t} has the form \stem{+o:t} and \textit{gá:yęˀ} has the form \linebreak \stem{+ęˀ}.

\ea\label{ex:posownex} 
\ea ga̱hnyo:t\\
\gll ga̱-hnyyo:t\\
\textsc{3s.a}-stick-stand.{\stative}\\
\glt ‘it is standing’
\ex snętso:t\\
\gll s-nęts-o:t\\
 \textsc{2s.a}-arm-stand.{\stative}\\
\glt `you have an (attached) arm'
\cfex{snętsáˀgeh\\
	\gll s-nęts-áˀgeh\\
	\textsc{2s.a}-arm-{\on}\\
	\glt `on your arm'}
\z
\z

\ea\label{ex:posownex2} 
\ea gá:yęˀ \\
\gll gá:-yęˀ\\
\textsc{3s.a}-lie.on.ground.{\stative}\\
\glt ‘it is lying on the ground’

\newpage
\ex  hoˀdréhdaęˀ\\
\gll ho-ˀdréhd-a-ęˀ\\
 \textsc{3s.m.p}-car-{\joinerA}-lie.{\stative}\\
\glt `he has a car, his car'
\ex hó:yęˀ\\
\gll hó:-yęˀ\\
 \textsc{3s.m.p}-lie.{\stative}\\
\glt `he has, owns it'
\cfex{ga̱ˀdréhdaˀ\\
	\gll ga̱-ˀdréhd-aˀ\\
	\textsc{3s.a}-car-{\nounstemformer}\\
	\glt `car'}
\z
\z

The verb \stem{{}+o:t} takes a \textsc{p}-series prefix when denoting possession of either an incorporated basic noun or a ‘detachable’ body part \xref{ex:posownex3}. In contrast, it can take either an \textsc{a}- or \textsc{p}-series prefix with body part nouns that are not normally ‘detachable’, at least for some nouns, (\ref{ex:posownex4}a, b). The difference in meaning conveyed by the pronominal prefix choice in this case is unclear.

\ea\label{ex:posownex3} \stem{\textsc{p}-\textsc{basic.noun}+o:t} ‘to have a basic noun’\\
 agége̱ˀo:t\\
\gll ag-é-ge̱ˀ-o:t\\
 \textsc{1s.p}-{\joinerE}-hair-stand.{\stative}\\
\glt `I have hair'
\cfex{ogéˀa:ˀ ‘hair’,\\
	\gll o-géˀa:-ˀ\\
	\textsc{3s.p}-hair-{\nounstemformer}\\
	\glt `a rag'}
\z

 
\ea\label{ex:posownex4} \stem{\textsc{a/p}-\textsc{body.part.noun}+o:t} ‘have a body part noun’
\ea sanę́:tso:t, snę́:tso:t \\
\gll sa-/s-nę́:ts-o:t\\
 \textsc{1s.p}/\textsc{2s.a}-arm-stand.{\stative}\\
\glt ‘you have an arm’, `your attached arm'

\cfex{snętsáˀgeh\\
	\gll s-nęts-áˀgeh\\
	\textsc{2s.a}-arm-{\on}\\
	\glt `on your arm'}

\newpage
\ex enǫ́ˀgo:t\\
\gll e-nǫ́ˀg-o:t\\
 \textsc{3s.fi.a}-breast-stand.{\stative}\\
\glt `her (attached) breast'
\cfex{enǫ́ˀgwa̱ˀgeh\\
	\gll e-nǫ́ˀgw-a̱ˀgeh\\
	\textsc{3s.fi.a}-breast-{\on}\\
	\glt `on her breast'}
\ex knoˀjo:t\\
\gll k-noˀj-o:t\\
 \textsc{1s.a}-tooth-stand.{\stative}\\
\glt `I have teeth'
\cfex{kno̱ˀjáˀgeh\\
	\gll k-no̱ˀj-áˀgeh\\
	\textsc{1s.a}-tooth-{\on}\\
	\glt `on my teeth'}
\z
\z

The verb \stem{{}+ęˀ} takes a \textsc{p}-series prefix to convey ownership of an incorporated basic noun \xref{ex:posownex5}. In contrast, it takes either an \textsc{a}- or \textsc{p}-series prefix when the incorporated noun is a non-detachable body part noun \xref{ex:posownex6}. The difference in meaning conveyed by the pronominal prefix choices is unclear.

\ea\label{ex:posownex5} \stem{\textsc{p}-\textsc{basic.noun}+ęˀ} ‘have a basic noun’\\
honáhsgwaęˀ\\
\gll ho-náhsgw-a-ęˀ\\
 \textsc{3s.m.p}-domestic.animal-{\joinerA}-lie.{\stative}\\
\glt `he has a pet'
\cfex{ganáhsgwaˀ \\
	\gll ga-náhsgw-aˀ\\
	\textsc{3s.a}-domestic.animal-{\nounstemformer}\\
	\glt ‘pet’, `domestic animal'}
\z


\ea\label{ex:posownex6} \stem{\textsc{a/p}-\textsc{body.part.noun}+ęˀ} ‘have a body part noun’
\ea ganóˀjaęˀ\\
\gll ga-nóˀj-a-ęˀ\\
 \textsc{3s.a}-teeth-{\joinerA}-lie.{\stative}\\
\glt `it has teeth'
\ex sagǫ́hsaęˀ\\
\gll sa-gǫ́hs-a-ęˀ\\
 \textsc{1s.p}-face-{\joinerA}-lie.{\stative}\\
\glt `you have a face'
\z
\z

Both \stem{{}+ęˀ} and \stem{{}+o:t} also have plural forms which denote the existence or ownership of more than one object. The obligatorily-incorporating plural form of \stem{{}+ęˀ} is \stem{+ędǫˀ} or \stem{+ędǫnyǫˀ} \xref{ex:posownex7}, and the obligatorily-incorporating plural form of \stem{{}+o:t} is \stem{+o:dǫˀ} or \stem{+odǫnyǫˀ} \xref{ex:posownex8}. Pronominal prefix selection is the same as for the singular verb forms.

\ea\label{ex:posownex8} \stem{\textsc{p}-\textsc{basic.noun}+o:dǫˀ} ‘have basic nouns’, \stem{\textsc{a}-\textsc{body.part.noun}-o:dǫˀ} ‘have body part nouns’
\ea aknǫhsó:dǫˀ \\
\gll ak-nǫhs-ó:d-ǫˀ\\
 \textsc{1s.p}-house-stand.{\stat}-{\distributive} \\
\glt ‘I have several houses’, `I have put up several houses'
\ex ehyagwiyó:dǫˀ \\
\gll e-hyagwiy-ó:d-ǫˀ\\
 \textsc{3s.fi.a}-toe-stand.{\stat}-{\distributive}\\
\glt ‘she has toes’, `her toes'
\z
\z

\ea\label{ex:posownex7} \stem{\textsc{p}-\textsc{basic.noun}+ędǫˀ} ‘have basic nouns’\\
godinǫhsáędǫˀ\\
\gll godi-nǫhs-á-ę-d-ǫˀ\\
 \exsc{3s.fi.p}-house-{\joinerA}-lie.{\stative}-\textsc{euph.d}-\exsc{\distributive}\\
\glt `they (females or mixed) have several houses'
\cfex{ganǫ́hsaˀ\\
	\gll ga-nǫ́hs-aˀ\\
	\textsc{3s.a}-house-{\nounstemformer}\\
	\glt `house'}
\z


\subsection{Neuter, stative-only verbs (fixed expressions)} \label{Neuter, stative-only verbs (fixed expressions)}
The main point in this section is that for fixed NV expressions, pronominal prefix choice is memorized, while for transparent (N)-V expressions, pronominal prefix choice is determined by the incorporated noun. 

Fixed expressions include \stem{NV} combinations whose meaning cannot be predicted from the separate meanings of the noun and verb stems. For fixed expressions, the pronominal prefix type has to be memorized, just as with non-incorporating stative-only verbs. To illustrate this point, let us compare regular, transparent \stem{N+V} combinations with fixed \stem{NV} expressions.

First, recall that the transparent expression \textit{ga̱he:ˀ} normally takes the same type of prefix as its incorporated noun \xref{ex:exceptionpp1}.

\ea\label{ex:exceptionpp1} 
\ea ga̱he:ˀ\\
\gll ga̱-he:ˀ \\
\textsc{3s.a}-sit.up.on.top. {\stative}\\
\glt ‘it is sitting up on top of something’, ‘it is sitting here’ 
\ex ganáˀja̱he:ˀ\\
\gll ga-náˀj-a̱-he:ˀ\\
 \textsc{3s.a}-pail-{\joinerA}-sit.up.on.top. {\stative}\\
\glt `pail setting on something'
\cfex{ganáˀjaˀ \\
\gll ga-náˀj-aˀ\\
\textsc{3s.a}-pail-\textsc{noun.stem.former}\\
\glt ‘pail’ }
\ex ohsgę́ˀdra̱he:ˀ\\
\gll o-hsgę́ˀdr-a̱-he:ˀ\\
 \textsc{3s.p}-rust-{\joinerA}-sit.up.on.top. {\stative}\\
\glt `it is rusting, rusty'
\cfex{ohsgę́ˀdraˀ\\
	\gll o-hsgę́ˀdr-aˀ\\
	\textsc{3s.p}-rust-{\nounstemformer}\\
	\glt `rust'}
\z
\z

However, \xref{ex:exceptionpp2a} is an apparent counterexample -- the verb’s pronominal prefix is different from the incorporated noun’s. Meanwhile, \xref{ex:exceptionpp2b} appears to be a regular case -- the verb’s pronominal prefix is the same as the incorporated noun’s. 

\ea\label{ex:exceptionpp2}
\ea onakda̱he:ˀ ‘a loft’ \label{ex:exceptionpp2a}
\cfex{ganá:kdaˀ ‘a bed’}
\ex ganakda̱he:ˀ ‘a bed is up here’ (Kehte Deer, p.c.)\label{ex:exceptionpp2b}
\z
\z

The difference between \xref{ex:exceptionpp2a} and \xref{ex:exceptionpp2b} is that \xref{ex:exceptionpp2a} is a fixed expression, while, in contrast, \xref{ex:exceptionpp2b} has a transparent meaning that is predictable from the meaning of the noun and verb stems. 

Similarly, example \xref{ex:exceptionpp3a} illustrates an incorporating verb whose choice of pronominal prefix is predictable -- the verb takes \stem{ga-} regardless of the prefix type of the incorporated noun (\ref{ex:exceptionpp3b},c). In contrast, when the same verb is part of a fixed expression \xref{ex:exceptionpp300}, the choice of pronominal prefix is not predictable. (The NV expression in \xref{ex:exceptionpp300} has no pronominal prefix, like the noun it incorporates.)

\newpage
\ea\label{ex:exceptionpp3} 
\ea ganí:yǫ:t \\\label{ex:exceptionpp3a}
\gll ga-ní:yǫ:t\\
\textsc{3s.a}-hang.{\stative}\\
\glt ‘it is hanging’ 


\ex gayęhsraní:yǫ:t\\\label{ex:exceptionpp3b}
\gll ga-yęhsr-a-ní:yǫ:t\\
 \textsc{3s.a}-blanket-{\joinerA}-hang. {\stative}\\
\glt `a hanging blanket'
\cfex{oyę́hsraˀ\\
	\gll o-yę́hsr-aˀ\\
	\textsc{3s.p}-blanket-{\nounstemformer}\\
	\glt `a blanket'}
\ex gayahkwaní:yǫ:t\\\label{ex:exceptionpp3c}
\gll ga-yahkw-a-ní:yǫ:t\\
 \textsc{3s.a}-pants-{\joinerA}-hang. {\stative}\\
\glt `hanging pants'
\cfex{oyáhkwaˀ\\
	\gll o-yáhkw-aˀ\\
	\textsc{3s.p}-pants-{\nounstemformer}\\
	\glt `pants'}
 \z
 \z

 \ea\label{ex:exceptionpp300} 
adęhęní:yǫ:t\\\label{ex:exceptionpp3d}
\gll adęhę-ní:yǫ:t\\
 \textsc{ø.prefix}.gate-hang. {\stative}\\
\glt `gate'
\cfex{adę́hęˀ\\
	\gll adę́hę-ˀ\\
	\textsc{ø.prefix}.gate-{\nounstemformer}\\
	\glt `a fence'}
\z


Examples \xref{ex:exceptionpp2}-\xref{ex:exceptionpp300} illustrate a hallmark of lexicalization (the change from a transparent to a fixed meaning): pronominal prefix choice is no longer predictable. 

