\chapter{Pronominal prefix pronunciation} \label{Pronominal prefix pronunciation}
Each pronominal prefix is actually a \emph{set} of prefixes that share one meaning but have different pronunciations. For example, while the prefixes in \xref{ex:pronominalprefixpronunex} all sound different, they are the \emph{same} prefix in the sense that they all mean \textsc{3s.fi.a} ‘she, someone’.  

\ea\label{ex:pronominalprefixpronunex} Pronunciation changes in the \textsc{a}-series prefix meaning ‘she, someone’
    \ea {}\stem{y.ag-} \textsc{3s.fi.a} (before O stems -- stems beginning with O or Ǫ)\label{ex:pronominalprefixpronunexa}
	\ea ęya:gó:da:ˀ\\  \label{ex:pronominalprefixpronunexai}
        \gll ę-ya:g-ó:da:-ˀ \\
        \fut-\textsc{3s.fi.a}-drape-{\punctual}\\
        \glt ‘she will drape it’
	    \ex a:yagó:da:ˀ \\ \label{ex:pronominalprefixpronunexaii}
                 \gll a:-yag-ó:da:-ˀ \\
                 {\indefinite}-\textsc{3s.fi.a}-drape-{\punctual}\\
                \glt ‘she might drape it’ 

	    \ex agó:da:s \\ \label{ex:pronominalprefixpronunexaiii}
                \gll agó:-da:-s \\
                 \textsc{3s.fi.a}-drape-{\habitual}\\
                \glt ‘she is draping it’ 
     
	\z
    \ex {}\stem{y.e-} \textsc{3s.fi.a} (before C stems -- stems beginning with a consonant)\label{ex:pronominalprefixpronunexb}
	\ea ęyé:gęˀ \\
             \gll ę-yé:-gę-ˀ \\ \label{ex:pronominalprefixpronunexbi}
            \fut-\textsc{3s.fi.a}-see-{\punctual}\\
             \glt ‘she will see it’ 

	    \ex a:yé:gęˀ \\ \label{ex:pronominalprefixpronunexbii}
     \gll a:-yé:-gę-ˀ \\
     {\indefinite}-\textsc{3s.fi.a}-see-{\punctual}\\
     \glt ‘she might see it’ 
\newpage
	    \ex aˀé:gęˀ \\ \label{ex:pronominalprefixpronunexbiii}
     \gll  aˀ-é:-gę-ˀ \\
     {\factual}-\textsc{3s.fi.a}-see-{\punctual}\\
     \glt  ‘she did see it’ 
   
	    \ex é:gęhs \\
     \gll  é:-gę-hs \\
     \textsc{3s.fi.a}-see-{\habitual}\\ \label{ex:pronominalprefixpronunexbiv}
     \glt  ‘she sees it’ 
   
	\z
    \z
\z

As shown in \xref{ex:pronominalprefixpronunex}, the pronunciation of each prefix depends on whatever precedes or follows. For example, the prefix in \xref{ex:pronominalprefixpronunex} is pronounced as \stem{y.ag-} -- that is, \phonet{yag} or \phonet{ag} -- before stems beginning with an O sound \xref{ex:pronominalprefixpronunexa} but as \stem{y.e-} -- that is, \phonet{ye} or \phonet{e} -- before stems beginning with a C (consonant) sound \xref{ex:pronominalprefixpronunexb}. In addition, many pronominal prefixes gain, lose, or change initial sounds, depending on what precedes them: for example, the prefix in \xref{ex:pronominalprefixpronunex} begins with Y after a vowel (\ref{ex:pronominalprefixpronunexai}, \ref{ex:pronominalprefixpronunexaii}, \ref{ex:pronominalprefixpronunexbi}, \ref{ex:pronominalprefixpronunexbii}); in other instances, the Y merges with S to become J (not shown). The prefix is otherwise pronounced without a Y (\ref{ex:pronominalprefixpronunexaiii}, \ref{ex:pronominalprefixpronunexbiii}, \ref{ex:pronominalprefixpronunexbiv}).

The pronunciation variants of each pronominal prefix are described in the following sections.


\section{Sound changes determined by C, A, E, I, and O stems} \label{Sound changes determined by C, A, E, I, and O stems}
The  pronunciation of each pronominal prefix is determined by the first sound of the following stem (even when the first sound of the stem has been deleted, as described below.) For example, the \textsc{3s.fi.a} ‘she, someone’ prefix is \stem{y.e-} before a C stem (a stem beginning with a consonant, \ref{ex:pronprefpronchangesexa}). It is also \stem{y.e-} before an I stem (a stem beginning with an I, \ref{ex:pronprefpronchangesexb}). However, it is \stem{y.ǫ-} before an A- or E stem (\ref{ex:pronprefpronchangesex}c,d). (\textsc{E stem} includes stems beginning with E or Ę). It is \stem{y.ag-} before an O stem \xref{ex:pronprefpronchangesexe}. (\textsc{O stem} includes stems beginning with O or Ǫ. Also, the stem-initial vowel (in brackets) deletes in \ref{ex:pronprefpronchangesexb}-\ref{ex:pronprefpronchangesexd}).

\ea\label{ex:pronprefpronchangesex} 
\ea \textsc{C stem} \\\label{ex:pronprefpronchangesexa}
 aˀé:gęˀ\\
\gll aˀ-e:-gę-ˀ\\
{\factual}-\textsc{3s.fi.a}-see-{\punctual}\\
\glt  ‘she saw it’
\clearpage 
\ex \textsc{I stem} \\\label{ex:pronprefpronchangesexb}
aˀédagraˀ\\
\gll aˀ-e-(i)dagraˀ\\
{\factual}-\textsc{3s.fi.a}-fall.down.{\punctual}\\
\glt  ‘she fell down’

\ex \textsc{A stem} \\\label{ex:pronprefpronchangesexc}
 aˀǫdęnaˀtragwáhaˀ\\
\gll aˀ-ǫ-(a)dę-naˀtra-gwáha-ˀ\\
{\factual}-\textsc{3s.fi.a}-{\semireflexive}-provisions-go.and.get-{\punctual}\\
\glt  ‘she went after groceries’

\ex \textsc{E-, Ę stem}\\\label{ex:pronprefpronchangesexd}
 aˀǫniˀdé:niˀ\\
\gll aˀ-ǫ-(ę)niˀde:ni-ˀ \\
{\factual}-\textsc{3s.fi.a}-fart-{\punctual}\\
\glt  ‘she farted’

\ex \textsc{O-, Ǫ stem} \\\label{ex:pronprefpronchangesexe}
dęyagodáihsiˀ\\
\gll dę-yag-odaihsi-ˀ\\
{\dualicfuture}\exsc{-3s.fi.a}-undrape-{\punctual}\\
\glt  ‘she undraped it’
\z
\z

In addition, the prefix and stem vowel sometimes merge -- A and I become Ę in \xref{ex:stemchangeex3a}. In other cases, a vowel might delete -- prefix I and stem I become a single I in (\ref{ex:stemchangeex3b}, see \sectref{Preserving, merging, and deleting vowels}.)

\ea\label{ex:stemchangeex3}
\ea ęsę́:daˀ\\\label{ex:stemchangeex3a}
\gll ę-sa-idaˀ\\
 \fut-\textsc{2s.p}-sleep.{\punctual}\\
\glt `you will sleep'

\ex ęyo:dí:daˀ\\\label{ex:stemchangeex3b}
\gll ę-yo:dí:-(i)daˀ\\
 \fut-\exsc{3p.p}-sleep.{\punctual}\footnotemark{}\\
\glt `they will sleep, hibernate'
\footnotetext{In this and similar cases, it is possible to say that either the prefix or the stem vowel was deleted. Under either analysis, a vowel-final prefix merges with a vowel-initial stem, as in \textit{ęyodi-idaˀ}, and one of the vowels is lost. The choice of analysis about which vowel is lost has no practical consequences. For consistency, however, I assume that the stem vowel deletes.}
\z
\z

It is important to know which sound occurs at the beginning of the stem, even if the stem-initial sound sometimes deletes. A method for determining the first sound of the stem is suggested in the following section.\footnote{The rules outlined below do not apply to neuter, stative-only verbs, since the latter do not reference an “I”.}


\subsection{Determining the first sound of the stem (most verbs, nouns)} \label{(most verbs, nouns)}
The first sound of the stem can be determined by examining a specific type of word chosen from a paradigm - a set of words that have the same basic meaning, but perhaps different prefixes or suffixes to denote different participants, aspects, or moods (see \sectref{Paradigm, inflection, and conjugation}). Example \xref{ex:paradigmexample} illustrates a partial paradigm of words that share the basic meaning of ‘see’. 

\ea\label{ex:paradigmexample} A partial paradigm of the verb \stem{gę} ‘see’
\ea ęhsé:gęˀ\\\label{ex:paradigmexamplea}
\gll ę-hsé:-gę-ˀ\\
 \fut-\textsc{2s.a}-see-{\punctual}\\
\glt `you will see it'
\ex ęyé:gęˀ\\\label{ex:paradigmexampleb}
\gll ę-yé:-gę-ˀ\\
 \fut-\textsc{3s.fi.a}-see-{\punctual}\\
\glt `she will see it'
\ex ęhǫwadí:gęˀ\\\label{ex:paradigmexamplec}
\gll ę-hǫwadí:-gę-ˀ\\
 \fut-\exsc{3s.m/3s.fi>3ns.m, 3ns>3m}-see-{\punctual}\\
\glt `they or s/he will see him, them (m)'
\ex aˀe:gęˀ\\\label{ex:paradigmexampled}
\gll aˀ-é:-gę-ˀ\\
 {\factual}-\textsc{3s.fi.a}-see-{\punctual}\\
\glt `she saw it'
\ex gé:gęhs\\\label{ex:paradigmexamplee}
\gll gé:-gę-hs\\
 \textsc{1s.a}-see-{\habitual}\\
\glt `I see'
\z
\z



\begin{table}
\caption{The stem sound is after the first GE}
\label{figtab:1:fstge}
{
\begin{tabularx}{\textwidth}{lQQ}
\lsptoprule
stem sound & \textsc{a}-series prefix & \textsc{p}-series prefix\\
\midrule
 G & \exemph{ge}gówanęh \newline ‘I am big’ & \exemph{age}ga̱hágwaǫt \newline ‘I have a stye’\\
 \tablevspace
 K & \exemph{ge}kǫni̱hsg{ę}hę:ˀ \newline ‘I used to cook’ & \exemph{age}kǫníhne:ˀ \newline ‘I have cooked’\\
 \tablevspace
 J & \exemph{ge}jí:nah \newline ‘I am strong’ & \exemph{age}jinyǫhgro:t \newline ‘I have a runny nose’\\
\tablevspace
ˀ & \exemph{ge̱}ˀohs \newline ‘I’m a chopper’ & \exemph{age}ˀó:gǫh \newline ‘I did chop’,  ‘I have chopped’\\
\tablevspace
T & \exemph{ge}téˀtaˀ \newline ‘I’m a pounder’ & \exemph{agé}teˀdǫh \newline ‘I did pound it’\\
\tablevspace
TS & \exemph{ge}ts{ę}:nyeˀs, \exemph{ge}ts{ę}:nyǫhs \newline ‘I’m a finder of things’ & \exemph{agé}tsęnyǫ: \newline ‘I have found it’\\
\tablevspace
 CC & \exemph{ge}nhóha:s \newline ‘I close the door (all the time)’ & \exemph{age}nhoháhǫh, \exemph{age}nhohá:hǫh \newline ‘I have closed the door’\\
\lspbottomrule
\end{tabularx}}
\end{table}


Words meaning ‘I’ or ‘me’ (as in \ref{ex:paradigmexamplee}) can help to identify the first sound of the stem, which occurs right after the prefix meaning ‘I’. (Words meaning ‘I’ or ‘me’ are referred to as \textsc{I-words} below.) 

\FloatBarrier
In some cases, the first sound of the stem is right after the first instance of GE in the I-word, \tabref{figtab:1:fstge}.

Note that it is important to consider only ‘I’ or ‘me’ forms in order to locate the first sound of the stem: as shown in \xref{ex:firstsoundexnoti}, words without the pronominal prefix meaning ‘I’ or ‘me’ can also contain a G, but the G in this case is not immediately before the stem.

\ea\label{ex:firstsoundexnoti} Not useful for determining the stem sound
\ea gyá:dǫh\\
\gll gy-á:dǫh\\
 \exsc{1d.in.a}-say.{\stative}\\
\glt `we say'
\ex gyohnegę́ˀǫh\\
\gll g-yo-hneg-ę́ˀ-ǫh\\
 {\cislocative}-\textsc{3s.p}-water-fall-{\stative}\\
\glt `falling water'
\z
\z



An exception to the above statement about GE involves a small number of E stem verbs, \tabref{figtab:1:fstestem} and \tabref{figtab:1:otherest}. Such cases are easy to identify because the E of the GE combination is present in all forms of the word paradigm -- the E is part of the stem, and  E is resistant to deletion.

\begin{table}
\caption{\textsc{1s} ‘I’ form before an E stem verb \stem{e:hah} ‘gloat’ or \stem{e:ˀǫ:} ‘think, will, decide’}
\label{figtab:1:fstestem}
{
\begin{tabularx}{\textwidth}{lXX}
\lsptoprule
stem sound & \textsc{a}-series prefix & \textsc{p}-series prefix\\
\midrule
 E & \exemph{ig}é:hah\newline  ‘I am gloating, boastful’ & \exemph{ag}é:ˀǫ: \newline  ‘I have willed it, decided it’\\
\lspbottomrule
\end{tabularx}}
\end{table}

\begin{table}
\caption{Other forms of E stem verb \stem{e:hah} ‘gloat’ or \stem{e:ˀǫ:} ‘think, will, decide’}
\label{figtab:1:otherest}
{
\begin{tabularx}{\textwidth}{lXX}
\lsptoprule
stem sound & \textsc{a}-series prefix & \textsc{p}-series prefix\\
\midrule
 E & \exemph{ihs}é:hah \newline ‘you gloat’ & \exemph{haw}é:ˀǫ: \newline ‘he willed, decided it’\\
 \tablevspace
& \exemph{gę:n}é:hah \newline ‘they are proud’ & \exemph{hon}é:ˀǫ: \newline ‘they willed, decided it’\\
\lspbottomrule
\end{tabularx}}
\end{table}

In many cases, the first sound of the stem occurs after the first G of the I-word. (The first G is part of the \textsc{1s} ‘I’ prefix, \tabref{figtab:1:firstg}, page \pageref{figtab:1:firstg}.)

\begin{table}[p]
\caption{The stem sound is after the first G}
\label{figtab:1:firstg}
{
\begin{tabularx}{\textwidth}{lXX}
\lsptoprule
stem sound & \textsc{a}-series prefix & \textsc{p}-series prefix\\
\midrule
 A & \exemph{g}atahí:neˀ \newline ‘I am walking’ & \exemph{ag}áhsę: \newline ‘I am fat’\\
 \tablevspace
 E & i\exemph{g}é:hah \newline ‘I am gloating, boastful’ & \exemph{ag}é:ˀǫ:  \newline ‘I have willed, decided it’\\
 \tablevspace
 Ę & \exemph{g}ęnǫhdǫnyǫh \newline ‘I’m wondering, thinking, feeling’ & \exemph{ag}ęnihsgyá:ge: \newline ‘I’m urinating’\\
 \tablevspace
 I & \exemph{g}i̱hsa:s \newline ‘I’m looking for it’ & \exemph{ag}ídaˀǫh \newline ‘I was asleep’, \newline ‘I am sleeping’\\
 \tablevspace
 O & \exemph{g}ó:da:s \newline ‘I drape it all the time’ & he\exemph{wág}o̱hǫh \newline ‘I have already dipped, submerged it’\\
 \tablevspace
 Ǫ & \exemph{g}ǫ:nihs \newline ‘I make, earn (it)’ & \exemph{ag}ǫni̱hne:ˀ \newline ‘I used to or have made, earned it’\\
 \tablevspace
 U & ni\exemph{g}ú:ˀuh \newline ‘I am small’\textsuperscript{⁠} & \\
 \tablevspace
 R & \exemph{g}ragé:wahs \newline ‘I am erasing or wiping (it)’ & \exemph{ag}ríhoˀdeˀ \newline ‘I am working’\\
\lspbottomrule
\end{tabularx}}
\end{table}


For some words, the first letter of the stem appears after the first K of the I-word, \tabref{figtab:1:kfirst} (page \pageref{figtab:1:kfirst}).

\begin{table}
\caption{The stem sound is after the first K}
\label{figtab:1:kfirst}
{
\begin{tabularx}{\textwidth}{XXX}
\lsptoprule
stem sound & \textsc{a}-series prefix & \textsc{p}-series prefix\\
\midrule
 D & \exemph{k}dákseˀ \newline ‘I am running’ & \exemph{ak}dǫhgwá:hǫh \newline ‘I have a fever’\\
 \tablevspace
 N & \exemph{k}nǫ̱hweˀs \newline ‘I like, admire it’ & \exemph{ak}nǫhaˀ \newline ‘I’m unable’\\
\lspbottomrule
\end{tabularx}}
\end{table}

That being said, some of the stems following the first K really begin with a glottal stop <ˀ> or H, which disappears in the I-word (due to a spelling rule which turns the letters Kˀ and KH into K). The glottal stop <ˀ> or H reappears in other forms of the verb, \tabref{figtab:1:splmrgk} (page \pageref{figtab:1:splmrgk}). 

\begin{table}
\caption{Spelling merger of Kˀ or KH to K}
\label{figtab:1:splmrgk}
{
\begin{tabularx}{\textwidth}{p{32mm}p{42mm}Q@{}}
\lsptoprule
stem sound & \textsc{a}-series prefix & \textsc{p}-series prefix\\
\midrule
ˀN\newline  \mbox{(ˀ not heard{\slash}spelled)} & \exemph{k}nigǫ:ha:ˀ \newline ‘I am expecting, watching’ & \exemph{ak}nigǫ̱háhetgęˀs \newline \mbox{‘I’m crying uncontrollably’}\\
\tablevspace
{}ˀN (ˀ audible) & \exemph{ha}ˀnígǫ̱ha:ˀ \newline ‘he is watching, expecting’ & \exemph{ho}ˀnigǫ̱háhetgęˀs \newline \mbox{‘he is crying uncontrollably’}\\
\tablevspace
HN or HR\newline \mbox{(H not heard{\slash}spelled)} & \exemph{k}n{ę}:ye:s \newline ‘I am tall’ & \exemph{ak}ré:nęh \newline ‘I did cut it’\\
HN (H audible) & \exemph{ha}hn{ę}:ye:s \newline ‘he is tall’ & \\
\tablevspace
H\newline  \mbox{(H not heard{\slash}spelled)} & \exemph{k}á:wiˀ \newline ‘I am carrying something’ & a\exemph{k}áˀjih \newline ‘I’m hoarse’\\
H (H audible) & \exemph{ha}há:wiˀ \newline ‘he is carrying it’ & \exemph{sa}háˀjih \newline ‘you’re hoarse’\\
\lspbottomrule
\end{tabularx}}
\end{table}

\largerpage[2]
The I-word in \xref{ex:exceptionalfirstsoundex} is exceptional. It has a unique \stem{w-} \textsc{1s.a} prefix. The full paradigm of this verb is listed in \sectref{figtab:1:wantthink}.


\ea\label{ex:exceptionalfirstsoundex} í:wi: \\
\gll \stem{í:-w-i:} \\
{\prothetic}-\textsc{1s.a}-think.{\stative}\\
\glt ‘I want, hope, think’ 
\z


\section{Sound changes at the beginning of the pronominal prefix} \label{Sound changes at the beginning of the pronominal prefix}
The two main types of pronunciation change at the \emph{beginning} of a pronominal prefix include the presence \xref{ex:pronominalprefixpronunex2} or absence \xref{ex:pronominalprefixpronunex3} of one or more sounds (specifically, W, Y, YA, H). Such changes are described next.
 
\ea\label{ex:pronominalprefixpronunex2} presence of \phonet{w} in \stem{(w)ag-} \textsc{1s.p}
\ea ęwa:gí:daˀ\\
\gll ę-wa:g-í:daˀ\\
 \fut-\textsc{1s.p}-sleep.{\punctual}\\
\glt `I will sleep'


\ex a:wagí:daˀ\\
\gll a:-wag-í:daˀ\\
 {\indefinite}-\textsc{1s.p}-sleep.{\punctual}\\
\glt `I should sleep'
\z
\z

\newpage
\ea\label{ex:pronominalprefixpronunex3} absence of \phonet{w} in \stem{(w)ag-} \textsc{1s.p}\\
agída̱ˀǫh ‘I was asleep’,\\
\gll ag-ída̱ˀ-ǫh\\
 \textsc{1s.p}-sleep-{\stative}\\
\glt `I am sleeping'
\z


\subsection{Prefixes that sometimes begin with a Y} \label{Prefixes that sometimes begin with a Y}
The prefixes in \tabref{figtab:1:yinit} (page \pageref{figtab:1:yinit}) begin with Y when they follow a vowel, but have no Y when they are at the beginning of the word or when they follow the \stem{aˀ-} \textsc{\factual}.

\begin{table}
\caption{Y-initial prefixes}
\label{figtab:1:yinit}
\begin{tabularx}{\textwidth}{QQQ}

\lsptoprule

prefix & \phonet{y} after \stem{ę-} \textsc{\future} or \stem{a:-} \textsc{\indefinite} & beginning of word or after \stem{aˀ-} \textsc{\factual}\\
\midrule

{}\stem{y.ǫg.e-} \newline \textsc{3s.fi>1s} & ta:\exemph{yǫgé:}gęh \newline ‘she will not see me’ & aˀ\exemph{ǫgé:}gęˀ \newline ‘she might see me’\\
\tablevspace
{}\stem{y.ǫkni-} \newline \textsc{1d.p} & de\exemph{yǫkni}ˀdráihęh \newline ‘let us hurry’ & \exemph{ǫkní}ˀdreˀ \newline ‘we two are riding along in a vehicle’\\
\tablevspace
{}\stem{y.ǫki.y-} \newline \textsc{3s.fi/3ns>1ns} & ę\exemph{yǫ:kí:y}ǫˀ \newline ‘she will give us’, \newline ‘they will give us’ & \exemph{ǫkiy}adęˀnigǫha:ˀ \newline ‘monitors’ (literally, \newline ‘they watch over us’)\\
\tablevspace
{}\stem{y.ǫgwa.y-} \newline \textsc{1p.p} & ę\exemph{yǫ:gwá:y}ǫˀ \newline ‘it will give us all’ & aˀ\exemph{ǫgwa}nhiˀk \newline ‘we all made a mistake’\\
\tablevspace
{}\stem{y.etsi.y-} \newline \textsc{3s.fi/3ns>2ns(opp)} & ę\exemph{yetsi}hwá:waˀs \newline ‘you all will back them’ & aˀ\exemph{etsi}gwé:niˀ \newline ‘you won a competition (against them)’\\
\tablevspace
{}\stem{y.akn.i-} \newline \textsc{1d.ex.a} & á:\exemph{yakhn}e:ˀ \newline ‘we two (excludes listener) would go together’ & \exemph{akní}ˀdreˀ \newline ‘we two (excludes listener) are dragging something’\\
\tablevspace
{}\stem{y.agwa-} \newline \textsc{1p.ex.a} & ę\exemph{yagwa}de:kǫ:niˀ \newline ‘we all (excludes listeners) will eat’ & \exemph{agwa}gowá:nęh \newline ‘we all (excludes listeners) are big’\\
\tablevspace
{}\stem{y.o-} \newline \textsc{3s.p} & a:\exemph{yó:}daˀ \newline ‘it would sleep’ & aˀ\exemph{ó:}de:k \newline ‘it did burn’\\
\tablevspace
{}\stem{y.e-} \newline \textsc{3s.fi} (C stem) & ę\exemph{yé:}gęˀ \newline ‘she will see (it)’ & \exemph{e}gánya̱ˀktaˀ \newline ‘what someone pays with’, \newline ‘barter’\\
\midrule
\end{tabularx}
\end{table}


\begin{table}
\begin{tabularx}{\textwidth}{QQQ}
\midrule
\tablevspace
{}\stem{y.ǫ-} \newline \textsc{3s.fi.a} (A stem) & a:\exemph{yǫ}da:wę:ˀ \newline ‘she might swim’ & aˀ\exemph{ǫ}dawęˀneˀ \newline ‘she is going swimming’\\
{}\stem{y.ǫ-} \newline \textsc{3s.fi.a} (E stem) & ę\exemph{yǫ}ni̱ˀdé:niˀ \newline ‘she will fart’ & aˀ\exemph{ǫ}ni̱ˀdé:niˀ \newline ‘she farted’\\
\tablevspace
{}\stem{y.e, y.ę-} \newline \textsc{3s.fi.a} (E stem) & ęyé:gęˀ \newline ‘she will see (it)’ & aˀ\exemph{é:}gęˀ \newline ‘she saw (it)’\\
\tablevspace
{}\stem{y.ag-} \newline \textsc{3s.fi.a} (O stem) & ę\exemph{yag}ǫ:niˀ \newline ‘she will make, earn’ & aˀ\exemph{a:g}ǫ:niˀ \newline ‘how much she made, earned’\\
\tablevspace
{}\stem{y.aki.y-} \newline \textsc{1ns.ex>3s.fi/3ns} & ę\exemph{yakiy}ęneˀwá:ʰdęˀ \newline ‘they and I will startle them’ & \exemph{akíy}a̱htgaˀ \newline ‘we let them go, released them’\\
\tablevspace
{}\stem{y.eti.y-} \newline \textsc{1ns.in>3s.fi/3ns} & ę\exemph{ye:tí:y}ǫˀ \newline ‘presents’, \newline ‘we will give something to her/them’ & aˀ\exemph{eti}gwé:niˀ \newline ‘we won a competition’\\
\tablevspace
{}\stem{y.ǫdad.e-} \newline \textsc{3s.fi>3s.fi} & aˀ\exemph{ǫdad}ahǫ:dǫ:ˀ \newline ‘she asked her’ & ę\exemph{yǫ:dá:d}ǫˀ \newline ‘she will give her’\\
\tablevspace
{}\stem{y.on-} \newline \textsc{3p.p} (V-stem) & a:\exemph{yon}adri̱hwa:gyáǫˀ \newline ‘they (animals) might have an accident & aˀ\exemph{on}adrihwagyáoˀ \newline ‘they (animals) had an accident’\\
\tablevspace
{}\stem{y.odi-} \newline \textsc{3p.p} (C stem, I stem) & ę\exemph{yódi}ˀdaˀ \newline ‘they (animals) will sleep, hibernate’ & \exemph{odí:}gaˀs \newline ‘they (animals) like the taste of it’\\
\lspbottomrule
\end{tabularx}
\end{table}



\subsection{Prefixes that sometimes begin with YA} \label{Prefixes that sometimes begin with YA}
The prefixes in \tabref{figtab:1:yainit} (page \pageref{figtab:1:yainit}) begin with YA when they follow a vowel, but YA is missing when the prefixes are at the beginning of the word or when they follow the \stem{a-} {\factual}.

\begin{table}
\caption{YA-initial prefixes}
\label{figtab:1:yainit}
{
\begin{tabularx}{\textwidth}{QQQ}

\lsptoprule
prefix & \stem{ya-} after \stem{ę-} \textsc{\future}, \stem{a:-} \textsc{\indefinite} or another vowel & beginning of word or after \stem{aˀ-} \textsc{\factual} \\
\midrule
{}\stem{ya.godi.y-, ya.gon-} \newline \textsc{3ns.fi.p} & ę\exemph{yagodí:y}ǫˀ \newline ‘it will give them (females or mixed group)’ & \exemph{gon}ęnadi:nyǫ:dǫh \newline ‘they (females or mixed group) have moved in’\\
\midrule
{}\stem{ya.go-} \newline \textsc{3s.fi.p} & de\exemph{yago}ˀáhsawˀehs \newline ‘she has asthma’ & \exemph{go}ˀdréhdaˀ \newline ‘her car’\\
\lspbottomrule
\end{tabularx}}
\end{table}


\subsection{Prefixes that sometimes begin with an H} \label{Prefixes that sometimes begin with an H}
Several 2\textsuperscript{nd} person (‘you’) prefixes optionally begin with an H when they follow a vowel, \tabref{figtab:1:hinit} (page \pageref{figtab:1:hinit}).

\begin{table}
\caption{H-initial prefixes}
\label{figtab:1:hinit}
{
\begin{tabularx}{\textwidth}{lXX}
\lsptoprule
prefix & optional \phonet{h} after a vowel, \stem{ę-} \textsc{\future}, or \stem{a:-} \textsc{\indefinite} & no \phonet{h} at beginning of word or after \stem{a-} \textsc{\factual}\\
\midrule
{}\stem{h.sa-} \textsc{2s.p} & ę\exemph{(h)sa}ǫhyágę̱dahk \newline ‘it’s going to make you groan’ & \exemph{sa}ní:yǫ:t \newline ‘you hung it up’\\
\midrule
{}\stem{h.swa-} \textsc{2p.a/o} & ę\exemph{(h)swá:y}ǫˀ \newline ‘it will give you all’ & \exemph{swa}nóˀjo:t \newline ‘you all have teeth’\\
\midrule
{}\stem{h.s.e-} \textsc{2s.a} & ę\exemph{(h)s}yǫ:gyaˀt \newline ‘you will smile!’ & \exemph{se}hsága̱hę:t \newline ‘on your mouth’\\
\midrule
{}\stem{h.she.y-} \textsc{2s>3s.fi} & ę\exemph{(h)shey}ahdrǫhgwęˀ \newline ‘you will threaten someone’ & \exemph{shey}adrǫ̱hgwá:nih \newline ‘you’re threatening her’\\
\lspbottomrule
\end{tabularx}}
\end{table}

All of the 3\textsuperscript{rd} masculine (‘he, they (males)’) prefixes lose their initial H when they follow the \stem{t-} \textsc{\cislocative} or the \stem{a-t-} \textsc{\factual-\dualic} combination. This is because in the Henry orthography, the letter T stands for two sounds, T plus H or D plus H \xref{ex:hinitialex}.


\ea\label{ex:hinitialex} Spelling merger of T and H
\ea atóhe:t\\
\gll a-t-hó-he:t\\
 {\factual}-{\dualic}-\textsc{3s.m.p}-yell.{\zeropunctual}\\
\glt `he yelled'
\ex atóhsi̱ha:ˀs\\
\gll a-t-hó-hsi̱ha:ˀs\\
 {\factual}-{\dualic}-\textsc{3s.m.p}-choke.{\zeropunctual}\\
\glt `he choked'
\ex desáhsi̱ha:ˀs\\
\gll de-sá-hsi̱ha:ˀs\\
 {\dualic}-\textsc{2s.p}-choke-{\habitual}\\
\glt `you are choking'
\z
\z

On a related theme, the ‘whispered’ syllable DE̱H sounds like T \xref{ex:hinitialex2}.

\ea\label{ex:hinitialex2}
\exemph{de̱h}óhetaˀ\\
\gll de̱-hó-het-haˀ\\
{\dualic}-\textsc{3s.m.p}-yell-{\habitual}\\
\glt ‘he is hollering’ (sounds like \exemph{tóhetaˀ})
\z

\subsection{Sound changes: prepronominal prefix-pronominal combinations} \label{Sound changes - prepronominal prefix-pronominal combinations}
Some of the pronominal prefixes merge or otherwise have special pronunciations before certain of the prepronominals. Such sound changes are described below.

The \stem{aˀ-} \textsc{\factual} and \stem{(w)ag-} \textsc{1s.p} prefixes merge together to form \stem{ǫg-} \xref{ex:pronominalprefixpronunex8}

\ea\label{ex:pronominalprefixpronunex8} merger of \stem{aˀ-wag-} \textsc{\factual-1s.p} prefixes\\
ǫgí:daˀ \\
\gll ǫg-í:daˀ\\
 \exsc{\factual.1s.p}-sleep.{\punctual}\\
\glt ‘I am sleeping’, `I slept'
\cfex{ęwa:gí:daˀ\\
	\gll ę-wa:g-í:daˀ\\
	\fut-\textsc{1s.p}-sleep.{\punctual}\\
	\glt `I will sleep'}
\z


The \stem{aˀ- or a-} \textsc{\factual} is instead pronounced as \phonet{e-} when before the pronominal prefixes listed in \tabref{figtab:1:efact} (page \pageref{figtab:1:efact}).

\begin{table}
\caption{E-factual}
\label{figtab:1:efact}
\begin{tabularx}{\textwidth}{QQQ}
\lsptoprule
prefix & \phonet{e-} \textsc{\factual} instead of \phonet{aˀ-} & after \stem{ę-} \textsc{\future} \\
\midrule
{}\stem{(eh)s-} \newline \textsc{2s.a} & \exemph{es}áhdǫ:ˀ \newline ‘you lost it’ & ę\exemph{hsé:}gęˀ \newline ‘you will see it’\\
\tablevspace
{}\stem{(e)sni-} \newline \textsc{2d.a} & \exemph{ehsní:}gęˀ \newline ‘you two saw it’ & ę\exemph{hsní:}gęˀ \newline ‘you two will see it’\\
\tablevspace
{}\stem{(e)swa-} \newline \textsc{2p.a} & \exemph{ehswá:}gęˀ \newline ‘you all saw it’ & ę\exemph{hswá:}gęˀ \newline ‘you all will see it’\\
\tablevspace
{}\stem{(eh).sgwa.y-} \newline \textsc{2>1(p)} & e\exemph{sgwa}dri̱hwatǫdá:dęˀ \newline ‘you have listened to my idea’ & ę\exemph{sgwá:y}ǫˀ \newline ‘you all will give to me’\\
\tablevspace
{}\stem{(e)dwa-} \newline \textsc{1p.in.a} & e\exemph{dwa}tgwé:niˀ \newline ‘we all (includes listener) won a competition’ & ę\exemph{dwá}dro̱he:k \newline ‘we all (includes listener) will gather together’\\
\tablevspace
{}\stem{(e)tni-}, \stem{(e)kni} \newline \textsc{1d.in.a} & e\exemph{tni}na̱ˀná:węht \newline ‘we two (includes listener) wet it’ & dę\exemph{tni}hsdá:teˀt \newline ‘we two (includes listener) will shine it’\\
\tablevspace
{}\stem{(e)hy.a-} \newline \textsc{3s.m>2s} & e\exemph{hyá:}gęˀ (also a\exemph{hyá:}gęˀ) \newline ‘he saw you (one)’ & ę\exemph{hy}{ę}hnǫksaˀ \newline ‘he is looking for you (one)’\\
\tablevspace
{}\stem{(eh)skni-} \newline \textsc{2>1(d)} & \exemph{eskní:}gęˀ \newline ‘you two saw me’ (etc.) & ę\exemph{hskní:}gęˀ \newline ‘you two will see me’ (etc.)\\
\tablevspace
\lspbottomrule
\end{tabularx}
\end{table}

\FloatBarrier
The pronominal prefixes listed in \tabref{figtab:1:efact} (page \pageref{figtab:1:efact}) also take the \phonet{ae-} form of the {\indefinite} prefix instead of \phonet{a:-} \xref{ex:pronominalprefixpronunex10}.


\ea\label{ex:pronominalprefixpronunex10} prefixes taking the \phonet{ae-} \textsc{\indefinite} pronunciation variant
\ea áeswa:k\\
\gll ae-swa:-k\\
 {\indefinite}-\textsc{2p.a}-eat.{\zeropunctual}\\
\glt `you all should eat it'
\cfex{á:se:k\\
	\gll á:-s-e:-k\\
	{\indefinite}-\textsc{2s.a}-eat.{\zeropunctual}\\
	\glt `you should eat it'}
\ex áedwe:ˀ\\
\gll áe-dw-e:-ˀ\\
 {\indefinite}-\textsc{2p.a}-think-{\punctual}\\
\glt `we all should want, think'
\cfex{á:yagwe:ˀ\\
	\gll á:-yagw-e:-ˀ\\
	{\indefinite}-\exsc{1p.ex.a}-think-{\punctual}\\
	\glt `we all should want, think'}
\z
\z

And finally, several \textsc{p}-series prefixes take the \phonet{ędi-} \textsc{\future.\cislocative} prefix combination instead of the \phonet{ęt-} or \phonet{ęd-} variants \xref{ex:pronominalprefixpronunex11}. The same prefixes take the \phonet{adi-} \textsc{\factual.\dualic} prefix combination instead of the \phonet{ad-} or \phonet{at-} variants \xref{ex:pronominalprefixpronunex12}.

\ea\label{ex:pronominalprefixpronunex11} prefixes taking \phonet{ędi-} \textsc{\future.\cislocative}
\ea ędihsa̱hsíˀgyaˀk\\
\gll ędi-s-ahsíˀg-yaˀk\\
 {\futurecislocative}\exsc{-2s.p}-foot-break.{\zeropunctual}\\
\glt `you will stumble, stub your toe'
\ex ędisanaˀgyę:ˀ \\
\gll ędi-sa-naˀgyę:-ˀ \\
{\futurecislocative}\exsc{-2s.p}-mock-{\punctual}\\
\glt ‘it will mock, imitate you’  

\cfex{ętsnaˀgyę:ˀ\\
	\gll ę-t-s-naˀgyę:-ˀ\\
	\fut-{\cislocative}-\textsc{2s.a}-mock-{\punctual}\\
	\glt `you will  imitate, mock, mimic, something'}
\z
\z

\clearpage
\ea\label{ex:pronominalprefixpronunex12} prefixes taking \phonet{adi-} \textsc{\factual.\dualic}
\ea adísaˀtsǫhs\\
\gll adi-sa-ˀtsǫhs\\
 {\factualdualic}\exsc{-2s.p}-sneeze.{\zeropunctual}\\
\glt `you sneezed'
\cfex{dęsáˀtsǫhs\\
	\gll d-ę-sá-ˀtsǫhs\\
	{\dualic}-{\future}-\textsc{2s.p}-sneeze.{\zeropunctual}\\
	\glt `you will sneeze'}

\ex adígya̱ˀdraˀ\\
\gll adí-gy-a̱ˀdraˀ\\
 {\factualdualic}\exsc{-1d.in.a}-meet.{\zeropunctual}\\
\glt `we two met'
\cfex{atgáǫˀdraˀ\\
	\gll a-t-gáǫ-ˀdraˀ\\
	{\factual}-{\dualic}-\textsc{3ns.fi.a}-meet.{\zeropunctual}\\
	\glt `they met'}
\z
\z


\subsection{Explanation of pronominal prefix tables} \label{Explanation of pronominal prefix tables}
The following sections each include two tables, designed to illustrate pronunciation changes at the beginning and end of each pronominal prefix. (To save space, translations are omitted from some tables.) The first type is exemplified in \tabref{figtab:1:prontblfst}.  

\begin{table}
\caption{Sample table (beginning of prefix)}
\label{figtab:1:prontblfst}
{
\begin{tabularx}{\textwidth}{p{25mm}XX}
\lsptoprule
gloss & beginning of prefix & environment\\
\midrule
\textsc{1s.p} &  \exemph{ǫgé:}gęˀ& merged \textsc{\factual} and \textsc{1s.p} prefix\\
\tablevspace
&\exemph{agé:}gę:& at the beginning of the word\\
\tablevspace
& ę\exemph{wa:gé:}gęˀ & after \stem{ę-} \textsc{\future} or \stem{aǫ-} \textsc{\indefinite}\\

\lspbottomrule
\end{tabularx}}
\end{table}


The gloss column lists the meaning of the prefix (\textsc{1s.p}). The second and third columns describe pronunciation changes at the \emph{beginning} of the prefix: for example, the \textsc{1s.p} prefix either begins with \phonet{ǫ}, \phonet{a}, or \phonet{wa}.

The second type of chart, shown in \tabref{figtab:sample.they.animals} (page \pageref{figtab:sample.they.animals}), provides more information about any changes between the last sound of the pronominal prefix and the first sound of the stem. 

\begin{table}
\caption{Sample table (end of prefix)}
\label{figtab:sample.they.animals}
{
\begin{tabularx}{\textwidth}{lQlQ}
\lsptoprule
\multicolumn{2}{c}{\textsc{3p.a}}&\multicolumn{2}{c}{\textsc{3p.p}}\\
\midrule
\multicolumn{2}{l}{\bfseries A E, Ę, O, Ǫ, U stem}\\
\midrule
\stem{gęn-}&ę\exemph{gęn}adéˀgoˀ\newline ‘they  will run away’& \stem{y.on-}&\exemph{on}adagáideˀ\newline ‘they are well’\\
\tablevspace
\multicolumn{2}{l}{\bfseries I stem}\\
\midrule
\stem{gadi-(i)}&\exemph{gadí}dagrǫˀ\newline ‘they are lying around’&\stem{y.odi-(i)}&\exemph{odi}dagraˀǫh\newline ‘they have fallen down’\\
\tablevspace
\multicolumn{2}{l}{\bfseries C stem}\\
\midrule
\stem{gadi-}&\exemph{gadi:}tsgó:dǫˀ\newline ‘they are sitting’&\stem{y.odi-}&\exemph{odi}nǫ̱hǫ́kdanih\newline ‘they are sick’\\
\lspbottomrule
\end{tabularx}}
\end{table}

For example, the \textsc{3p.a} prefix is \phonet{gęn-} before any stem beginning with A, E, Ę, O, Ǫ, or U. The same prefix is pronounced as \phonet{gadi-} before stems beginning with I, and the I of the stem also deletes. (Deleted vowels are shown between parentheses.) The prefix is \phonet{gadi-} before stems beginning with any consonant (C). Similarly, the \textsc{3p.p} prefix is \stem{y.on-} before any stem beginning with A, E, Ę, O, Ǫ, or U. The same prefix is pronounced as \stem{y.odi-} before stems beginning with I, and the I of the stem also deletes. (Deleted vowels are shown between parentheses.) Finally, the prefix is \stem{y.odi-} before stems beginning with any consonant (C).





\section{Non-interactive (\textsc{a}- and \textsc{p}-series) prefix pronunciation} \label{Non-interactive (A and O series) prefix pronunciation and meaning}
The following sections describe the pronunciation of non-interactive (\textsc{a}- and \textsc{p}-series) 1st person, 2nd person, and 3rd person pronominal prefixes. Recall that these same prefixes can be used in an interactive sense, with an implied ‘it’. For example, \textsc{1s.a} can mean ‘I’ or ‘I>(it)’.
\newpage

\subsection{1st person, non-interactive (or interactive with implied ‘it’)}
~\largerpage[2]


\begin{table}[h]
\caption{\textsc{1s} (beginning of prefix)}
\label{figtab:I.beginning}
{
\begin{tabularx}{\textwidth}{p{25mm}XX}
\lsptoprule
gloss&pronominal&environment\\
\midrule
\textsc{1s.a}&a\exemph{gé:}gęˀ& (no change)\\
\tablevspace
&\exemph{gé:}gęhs& \\
\tablevspace
&ę\exemph{gé:}gęˀ& \\
\midrule
\textsc{1s.p}&\exemph{ǫgé:}gęˀ&merged {\factual} and \textsc{1s.p} prefix\\
\tablevspace
&\exemph{agé:}gę:&at the beginning of the word\\
\tablevspace
&ę\exemph{wa:gé:}gęˀ&after a vowel- \stem{ę-} {\future},

\stem{a:-} {\indefinite}\\
\lspbottomrule
\end{tabularx}}
\end{table}

~

\begin{table}[b!]
\caption{\textsc{1s} (prefix-stem boundary)}
\label{figtab:I}
{
\begin{tabularx}{\textwidth}{QQp{30mm}QQ}
\lsptoprule
&\multicolumn{2}{c}{\textsc{1s.a}}&\multicolumn{2}{c}{\textsc{1s.p}}\\
\midrule
A, E, Ę, I, O, Ǫ, U stem& \stem{g-}&\exemph{g}adá:węhs 
\newline ‘I’m a swimmer’& \stem{w.ag-}&\exemph{ag}adagaideˀ
\newline ‘I am well’\\
\tablevspace
C stem & \stem{k-\textsuperscript{(h),(h)n, (ˀ)n}} 

\stem{k-\textsuperscript{d}}&\exemph{k}nę́:ye:s
\newline ‘I’m tall’& \stem{w.ak-\textsuperscript{(h),(h)n, (ˀ)n}}

\stem{w.ak-\textsuperscript{d}}&d\exemph{wak}nigǫhí:yo:
\newline ‘I am satisfied, peaceful’\\
\tablevspace
C stem&\stem{g-\textsuperscript{r,y}}&de\exemph{g}rihwanǫ́hweˀs
\newline ‘I disapprove’& \stem{w.ag-\textsuperscript{r,y}}&\exemph{ag}yáˀdadeht
\newline ‘I am nimble, active, energetic’\\
\tablevspace
C stem& \stem{ge-\textsuperscript{g,s,w,CC}}&\exemph{ge}gę́hjih
\newline ‘I am old’& \stem{w.age-\textsuperscript{g,s,w,CC}}&\exemph{agé}gaˀs
\newline ‘I like the taste of it’\\
\lspbottomrule
\end{tabularx}}
\end{table}

\clearpage
\begin{table}
\caption{\textsc{1d} (beginning of prefix)}
\label{figtab:we.two.beginning}
{
\begin{tabularx}{\textwidth}{lXX}
\lsptoprule
gloss&pronominal&environment\\
\midrule
\textsc{1d.in.a}&e\exemph{tní:}gęˀ, e\exemph{kní:}gę&after \stem{e-} {\factual}\\
\tablevspace
&\exemph{tní:}gęhs, \exemph{kní:}gęhs&at the beginning of the word\\
\tablevspace
&ę\exemph{tní:}gęˀ, ę\exemph{kní:}gęˀ, ae\exemph{tní:}gęˀ, ae\exemph{kní:}gęˀ& after a vowel - \stem{ę-} {\future},

\stem{ae-} {\indefinite}\\
\midrule 
\tablevspace
\newline \textsc{1d.ex.a}&aˀ\exemph{a:kní:}gęˀ, *\exemph{a:kní:}gęˀ&after \stem{aˀ-} {\factual}, (*{\factual} often deletes)\\
\tablevspace
&\exemph{akní:}gęhs&at the beginning of the word\\
\tablevspace
&ę\exemph{ya:kní:}gęˀ&after a vowel- \stem{ę-} {\future},

\stem{a:-} {\indefinite}\\
\midrule 
\tablevspace

\newline \textsc{1d.p} &aˀ\exemph{ǫ:kní:}gęˀ&after \stem{aˀ-} {\factual}\\
\tablevspace
&\exemph{ǫkní:}gę:&at the beginning of the word\\
\tablevspace
&ę\exemph{yǫ:kní:}gęˀ&after a vowel - \stem{ę-} {\future},

\stem{a:-} {\indefinite}\\
\lspbottomrule
\end{tabularx}}
\end{table}

\begin{sidewaysfigure}
\caption{\textsc{1d} (prefix-stem boundary)}
\label{figtab:we.two}
{
\begin{tabularx}{\textwidth}{lQ lQ lQ}
\lsptoprule
\multicolumn{2}{c}{\textsc{1d.in.a}}&\multicolumn{2}{c}{\textsc{1d.ex.a}}&\multicolumn{2}{c}{\textsc{1d.p}}\\
\cmidrule(r){1-2}\cmidrule(lr){3-4}\cmidrule(l){5-6}
% \midrule
\multicolumn{5}{l}{\bfseries A stem}\\
\midrule
\stem{gy-}&\exemph{gy}á:dǫh \newline ‘you and I say’& \stem{y.agy-}&\exemph{agy}á:dǫh \newline ‘we two say’& \stem{y.ǫgy-}&\exemph{ǫgy}adagáideˀ \newline ‘we two are healthy’\\
\tablevspace
\multicolumn{5}{l}{\bfseries E, Ę, I, O, Ǫ, U stem}\\
\midrule
\stem{kn-, 
tn-}&e\exemph{kn}íˀ, e\exemph{tn}íˀ  \newline ‘you and I said’&\stem{y.akn-}&ni\exemph{yakn}ú:ˀuh \newline ‘we two are small’&\stem{y.ǫkn-}&\exemph{ǫkn}é:ˀǫ: \newline ‘we two have \mbox{willed, decided it’}\\
\tablevspace
\multicolumn{5}{l}{\bfseries C stem}\\
\midrule
\mbox{\stem{tni-,

kni-}}&\exemph{tni}gę́hjih, \exemph{kni}gę́hjih \newline \mbox{‘we two are old’}&\stem{y.akni-}&\exemph{akni}hnę́:ye:s \newline ‘we two are tall’&\stem{y.ǫkni-}&\exemph{ǫkni}nǫ̱\-hǫ́kdanih \newline ‘we two are sick’\\
\lspbottomrule
\end{tabularx}}
\end{sidewaysfigure}

\begin{table}
\caption{\textsc{1p} (beginning of prefix)}
\label{figtab:we.all.beginning}
{
\begin{tabularx}{\textwidth}{lXX}
\lsptoprule
gloss&pronominal&environment\\
\midrule
\textsc{1p.ex.a}&aˀ\exemph{ágwa}gęˀ, *\exemph{ágwa}gęˀ*&after \stem{aˀ-} {\factual} (*{\factual} often deletes)\\
\tablevspace
&\exemph{agwá:}gęhs&at the beginning of the word\\
\tablevspace
&ę\exemph{yágwa}gęˀ& after a vowel - \stem{ę-} {\future},

\stem{a:-} {\indefinite}\\
\tablevspace
\midrule 
\newline \textsc{1p.in.a}&e\exemph{dwá:}gęˀ&after \phonet{e-} {\factual}\\
\tablevspace
&\exemph{dwá:}gęhs&at the beginning of the word\\
\tablevspace
&ę\exemph{dwá:}gęˀ& after a vowel - \stem{ę-} {\future},

\stem{ae-} {\indefinite}\\
\tablevspace
\midrule 
\newline \textsc{1p.p}&aˀ\exemph{ǫ:gwá:}gęˀ&after \stem{aˀ-} {\factual}\\
\tablevspace
&\exemph{ǫgwá:}gęhs&at the beginning of the word\\
\tablevspace
&ę\exemph{yǫ:gwá:}gęˀ& after a vowel - \stem{ę-} {\future},

\stem{a:-} {\indefinite}\\
\lspbottomrule
\end{tabularx}}
\end{table}





\begin{table}
\small
\caption{\textsc{1p} (prefix-stem boundary)}
\label{figtab:we.all}
\begin{tabularx}{\textwidth}{lQ lQ lQ}
\lsptoprule
\multicolumn{2}{c}{\textsc{1p.ex.a}}&\multicolumn{2}{c}{\textsc{1p.in.a}}&\multicolumn{2}{c}{\textsc{1p.p}}\\
\cmidrule(r){1-2}\cmidrule(lr){3-4}\cmidrule(l){5-6}
\multicolumn{5}{l}{\bfseries A, E, Ę, stem}\\
\midrule
\stem{y.agw-}&\exemph{agw}á:dǫh ‘we all say’&\stem{dw-}&\exemph{dw}á:dǫh ‘we  all say’&\stem{y.ǫgw-}&\exemph{ǫgw}ada\-gaideˀ ‘we all are healthy’\\
\multicolumn{5}{l}{\bfseries O, Ǫ, U stem}\\
\midrule
\stem{y.agy-} &ni\exemph{yagy}\-u:ˀuh ‘we are small’& \stem{gy-} &ni\exemph{gy}u:ˀuh ‘we all are small’& \stem{y.ǫgy-} &\exemph{ǫgy}ǫ:dá:hǫh ‘we all have put it in’\\

\multicolumn{5}{l}{\bfseries I stem}\\
\midrule
\stem{y.agwę-\textsuperscript{(i)}}&\exemph{á:gwę}ˀ, aˀ\exemph{á:gwę}ˀ ‘we all said’& \stem{dwę-\textsuperscript{(i)}}, \stem{twę-\textsuperscript{(i)}} &e\exemph{dwę́}ˀ, e\exemph{twę́}ˀ  ‘we all said’&\stem{y.ǫgwę-\textsuperscript{(i)}}&\exemph{ǫgwę}da\-graˀǫh ‘we all have fallen’\\

\multicolumn{5}{l}{\bfseries C stem}\\
\midrule
\stem{y.agwa-}&\exemph{agwa}\-hnę́:ye:s ‘we all are tall’& \stem{dwa-} & \exemph{dwa̱}hnę́:ye:s ‘we all are tall’& \stem{y.ǫgwa-} &\exemph{ǫgwa}nǫ̱h\-ǫ́kdanih ‘we all are sick’\\
\lspbottomrule
\end{tabularx}
\end{table}


\clearpage
\subsection{2nd person, non-interactive (or interactive with implied ‘it’)}
\begin{table}
\caption{\textsc{2s} (beginning of prefix)}
\label{figtab:you.one.beginning}
{
\begin{tabularx}{\textwidth}{lp{30mm}X}
\lsptoprule
gloss&pronominal&environment\\
\midrule
\textsc{2s.a}&a\exemph{hsé:}gęˀ, e\exemph{s}áhdǫ:ˀ&after \stem{a-} {\factual}, after \stem{e-} {\factual}\\
\tablevspace
&\exemph{sé:}gęhs&at the beginning of the word\\
\tablevspace
&ę\exemph{hsé:}gęˀ& after a vowel - \stem{ę-} {\future}, \stem{ae-} {\indefinite}\\
\midrule
\textsc{2s.p}&a\exemph{hsá:}gęˀ&after \stem{a-} {\factual}\\
\tablevspace
&\exemph{sá:}gę:&at the beginning of the word\\
\tablevspace
&ę\exemph{hsá:}gęˀ& after a vowel - \stem{ę-} {\future}, \stem{a:-} {\indefinite}\\
\lspbottomrule
\end{tabularx}}
\end{table}

\begin{table}
\caption{\textsc{2s} (prefix-stem boundary)}
\label{figtab:you(one)}
% \begin{tabularx}{\textwidth}{XXXXXX}
% \lsptoprule
% &\textsc{2s.a}&&&\textsc{2s.p}&\\
% \midrule
% V stem&(h)s-&ni\textit{s}ú:ˀuh
% \newline ‘you are small’&A, E, Ę, O, Ǫ, U stem&(h)s-&\textit{s}ęnó:wę:
% \newline ‘you are a liar’\\
% \midrule
% R stem&(h)s-r, d-r&da\textit{hs}rá:tęh, da\textit{dr}á:tęh
% \newline ‘climb (over here)!’&I stem&(h)sę-(i)&\textit{sę}dáǫ
% \newline ‘you are fortunate’\\
% \midrule
% W, G, K, CC stem&(h)se-&\textit{se}kdǫ:
% \newline ‘examine it!’&C stem&(h)sa-&\textit{sa}nǫhǫkdá:nih
% \newline ‘you are sick’\\
% \midrule
% (H)N, N stem&(h)s-(h)n, n&\textit{s}hnę́:ye:s
% \newline ‘you are tall’&&&\\
% \lspbottomrule
% \end{tabularx}}

\begin{tabularx}{\textwidth}{QQQQ}
\lsptoprule
\multicolumn{2}{c}{\textsc{2s.a}}&\multicolumn{2}{c}{\textsc{2s.p}}\\
\midrule
\multicolumn{2}{l}{\bfseries V stem} & \multicolumn{2}{l}{\bfseries A, E, Ę, O, Ǫ, U stem}\\
\cmidrule(r){1-2}\cmidrule(l){3-4}
\stem{(h)s-} &ni\exemph{s}ú:ˀuh \newline ‘you are small’& \stem{(h)s-} &\exemph{s}ęnó:wę: \newline ‘you are a liar’\\

\tablevspace
\multicolumn{2}{l}{\bfseries R stem} & \multicolumn{2}{l}{\bfseries I stem}\\
\cmidrule(r){1-2}\cmidrule(l){3-4}
\stem{(h)s-\textsuperscript{r}}, \stem{d-\textsuperscript{r}} &da\exemph{hs}rá:tęh, da\exemph{dr}á:tęh \newline ‘climb (over here)!’& \stem{(h)sę-\textsuperscript{(i)}}&\exemph{sę}dáǫ \newline ‘you are fortunate’\\

\tablevspace
\multicolumn{2}{l}{\bfseries W, G, K, CC stem} &\multicolumn{2}{l}{\bfseries C stem}\\
\cmidrule(r){1-2}\cmidrule(l){3-4}
\stem{(h)se-} &\exemph{se}kdǫ: \newline ‘examine it!’& \stem{(h)sa-}&\exemph{sa}nǫhǫkdá:nih  \newline ‘you are sick’\\

\tablevspace
\multicolumn{2}{l}{\bfseries (H)N, N stem}\\
\cmidrule(r){1-2}
\stem{(h)s-\textsuperscript{(h)n, n}} &\exemph{s}hnę́:ye:s \newline ‘you are tall’&\\
\lspbottomrule
\end{tabularx}
\end{table}

\begin{table}
\caption{\textsc{2d} (beginning of prefix)}
\label{figtab:you.two.beginning}
{
\begin{tabularx}{\textwidth}{lXX}
\lsptoprule
gloss&pronominal&environment\\
\midrule
\textsc{2d.a}&e\exemph{hsní:}gęˀ& \stem{e-} {\factual}\\
\tablevspace
&\exemph{sní:}gęhs&at the beginning of the word\\
\tablevspace
&ę\exemph{hsní:}gęˀ& after a vowel - \stem{ę-} {\future},

\stem{ae-} {\indefinite}\\
\midrule
\textsc{2d.p}&a\exemph{hsní:}gęˀ&after \stem{a-} {\factual}\\
\tablevspace
&\exemph{sní:}gę:&at the beginning of the word\\
\tablevspace
&ę\exemph{hsní:}gęˀ& after a vowel - \stem{ę-} {\future},

\stem{a:-} {\indefinite}\\
\lspbottomrule
\end{tabularx}}
\end{table}

\begin{table}
\caption{\textsc{2d} (prefix-stem boundary)}
\label{figtab:you.two}
{
\begin{tabularx}{\textwidth}{lQlQ}
\lsptoprule
\multicolumn{2}{c}{\textsc{2d.a}}&\multicolumn{2}{c}{\textsc{2d.p}}\\
\cmidrule(r){1-2}\cmidrule(l){3-4}
\multicolumn{2}{l}{\bfseries A stem}\\
\midrule
\stem{j-} &\exemph{j}á:dǫh
\newline ‘you two say’&\stem{j-} &\exemph{j}adagaideˀ
\newline ‘you two are well’\\
\midrule

\tablevspace
\multicolumn{2}{l}{\bfseries E, Ę, O, Ǫ, U stem}\\
\midrule
\stem{(h)sn-} &ę\exemph{hsn}e:ˀ
\newline ‘you two will want, think’& \stem{(h)sn-} &\exemph{sn}é:ˀǫ:, \exemph{sn}eˀǫ:
\newline ‘you two have willed it’\\

\tablevspace
\multicolumn{2}{l}{\bfseries I stem}\\
\midrule
\stem{(h)sni-\textsuperscript{(i)}} &e\exemph{sní}ˀ \newline ‘you two said’&\stem{(h)sni-\textsuperscript{(i)}} &\exemph{sni}dáǫ
\newline ‘you two are fortunate’\\

\tablevspace
\multicolumn{2}{l}{\bfseries C stem}\\
\midrule
\stem{(h)sni-}&\exemph{sni}hnę́:ye:s
\newline ‘you two are tall’&\stem{(h)sni-}&\exemph{sni}nǫhǫkdá:nih
\newline ‘you two are sick’\\
\lspbottomrule
\end{tabularx}}
\end{table}

\begin{table}
\caption{\textsc{2p} (beginning of prefix)}
\label{figtab:you.all.beginning}
{
\begin{tabularx}{\textwidth}{p{35mm}XX}
\lsptoprule
gloss&pronominal&environment\\
\midrule
\textsc{2p.a}&e\exemph{hswá:}gęˀ& after \stem{e-} {\factual}\\
\tablevspace
&\exemph{swá:}gęhs&at the beginning of the word\\
\tablevspace
&ę\exemph{hswá:}gęˀ& after a vowel - \stem{ę-} {\future},

\stem{ae-} {\indefinite}\\
\midrule
\textsc{2p.p}&a\exemph{hswá:}gęˀ&after \stem{a-} {\factual}\\
\tablevspace
&\exemph{swá:}gę:&at the beginning of the word\\
\tablevspace
&ę\exemph{hswá:}gęˀ& after a vowel - \stem{ę-} {\future},

\stem{a:-} {\indefinite}\\
\lspbottomrule
\end{tabularx}}
\end{table}


\begin{table}
\caption{\textsc{2p} (prefix-stem boundary)}
\label{figtab:you.all}
\begin{tabularx}{\textwidth}{lQlQ}
\lsptoprule
\multicolumn{2}{c}{\textsc{2p.a}}&\multicolumn{2}{l}{\textsc{2p.p}}\\
\midrule
\multicolumn{2}{l}{\bfseries A stem}\\
\midrule
\stem{(h)swa-\textsuperscript{(a)}} &\exemph{swá:}dǫh \newline ‘you all say’& \stem{(h)swa-\textsuperscript{(a)}} &\exemph{swa}dagáideˀ \newline ‘you all are healthy’\\
\tablevspace
\multicolumn{2}{l}{\bfseries E, Ę  stem}\\
\midrule
\stem{(h)sw-} &ę\exemph{hsw}e:ˀ \newline ‘you all will want, think’& \stem{(h)sw-} &i\exemph{hsw}é: \newline ‘you all want, think’\\
\tablevspace
\multicolumn{2}{l}{\bfseries I stem}\\
\midrule
\stem{(h)swę-\textsuperscript{(i)}} &e\exemph{swę́}ˀ\newline ‘you all said’& \stem{(h)swę-\textsuperscript{(i)}} &i\exemph{hswę́}:\newline ‘you all have said’\\
\tablevspace
\multicolumn{2}{l}{\bfseries O, Ǫ, U stem}\\
\midrule
\stem{j-} &ni\exemph{j}ú:ˀuh \newline ‘you all are small’& \stem{j-} &do: ni\exemph{j}ohsriyáˀgǫh \newline ‘how old are you all’\\
\tablevspace
\multicolumn{2}{l}{\bfseries C-stem}\\
\midrule
\stem{(h)swa-} &\exemph{swa}gę́hjih \newline ‘you all are old’& \stem{(h)swa-}&\exemph{swa}nǫhǫkdá:nih \newline ‘you two are sick’\\
\lspbottomrule
\end{tabularx}
\end{table}

\FloatBarrier
\subsection{3rd person singular, non-interactive (or interactive with implied ‘it’)}

The initial H deletes after a T in the prefixes shown in \tabref{figtab:he.beginning} (page \pageref{figtab:he.beginning}) and \tabref{figtab:he} (page \pageref{figtab:he}).

\begin{table}
\caption{\textsc{3s.m} (beginning of prefix)}
\label{figtab:he.beginning}
{
\begin{tabularx}{\textwidth}{lXX}
\lsptoprule
gloss&pronominal&environment\\
\midrule
\textsc{3s.m.a} &a\exemph{há:}gęˀ&no change, takes \stem{a-} {\factual}\\
& \exemph{há:}gęhs & \\
&ę\exemph{há:}gęˀ& \\
\midrule
\textsc{3s.m.p}&a\exemph{hó:}gęˀ&no change, takes \stem{a-} {\factual}\\
&\exemph{hó:}gę:& \\
&ę\exemph{hó:}gęˀ&  \\
\lspbottomrule
\end{tabularx}}
\end{table}

\begin{table}
\caption{\textsc{3s.m} (prefix-stem boundary)}
\label{figtab:he}
{
\begin{tabularx}{\textwidth}{lQlQ}
\lsptoprule
\multicolumn{2}{c}{\textsc{3s.m.a}}&\multicolumn{2}{c}{\textsc{3s.m.p}}\\
\cmidrule(r){1-2}\cmidrule(l){3-4}
\multicolumn{2}{l}{\bfseries A stem}& \multicolumn{2}{l}{\bfseries A, I stem}\\
\cmidrule(r){1-2}\cmidrule(l){3-4}
\stem{ha-\textsuperscript{(a)}} &\exemph{há:}dǫh \newline ‘he says’& \stem{ho-\textsuperscript{(V)}} &\exemph{ho}dagáideˀ \newline ‘he is well’\\

\tablevspace
\multicolumn{2}{l}{\bfseries E, Ę, O, Ǫ, U stem} &\multicolumn{2}{l}{\bfseries E, Ę stem} \\
\cmidrule(r){1-2}\cmidrule(l){3-4}
\stem{h-} &a\exemph{h}ę́ˀ\newline ‘he said’& \stem{haw-} &\exemph{há:w}ę: \newline ‘he has said’\\

\tablevspace
\multicolumn{2}{l}{\bfseries I stem}&\multicolumn{2}{l}{\bfseries O, Ǫ stem}\\
\cmidrule(r){1-2}\cmidrule(l){3-4}
\stem{hę-\textsuperscript{(i)}} &\exemph{hę}hsa:s \newline ‘he is looking for it’& \stem{ha-} &\exemph{ha}ǫtsánǫ̱hwa:s \newline ‘his knee hurts’\\

\tablevspace
\multicolumn{2}{l}{\bfseries C-stem} & \multicolumn{2}{l}{\bfseries C-stem}\\
\cmidrule(r){1-2}\cmidrule(l){3-4}
\stem{ha-} &\exemph{ha}gę́hjih\newline ‘he is old’& \stem{ho-} &\exemph{ho}nǫhǫkdá:nih \newline ‘he is sick’\\
\lspbottomrule
\end{tabularx}}
\end{table}

\begin{table}
\caption{\textsc{\textsc{3s.fi}} (beginning of prefix)}
\label{figtab:she.beginning}
{
\begin{tabularx}{\textwidth}{lp{30mm}X}
\lsptoprule
gloss&pronominal&environment\\
\midrule
\textsc{3s.fi.a}&aˀ\exemph{é:}gęˀ&after \stem{aˀ-} {\factual}\\
\tablevspace
&\exemph{é:}gęhs&at the beginning of the word\\
\tablevspace
&ę\exemph{yé:}gęˀ& after a vowel - \stem{ę-} {\future}, \stem{a:-} {\indefinite}\\
\midrule
\textsc{3s.fi.p}&aˀ\exemph{a:gó:}gęˀ&after \stem{aˀ-} {\factual}\\
\tablevspace
&\exemph{gó:}gę:&at the beginning of the word\\
\tablevspace
&ę\exemph{ya:gó:}gęˀ& after a vowel - \stem{ę-} {\future}, \stem{a:-} {\indefinite}\\
\lspbottomrule
\end{tabularx}}
\end{table}


\begin{table}
\caption{\textsc{3s.fi} (prefix-stem boundary)}
\label{figtab:she}
\begin{tabularx}{\textwidth}{lQlQ}
\lsptoprule
\multicolumn{2}{c}{\textsc{3s.fi.a}}&\multicolumn{2}{c}{\textsc{3s.fi.p}}\\
\cmidrule(r){1-2}\cmidrule(l){3-4}
\multicolumn{2}{l}{\bfseries A stem}&\multicolumn{2}{l}{\bfseries A, I stem}\\
\cmidrule(r){1-2}\cmidrule(l){3-4}
\stem{y.ǫ-\textsuperscript{(a)}}&\exemph{ǫ́:}dǫh\newline ‘she, someone says’& \stem{ya.go-\textsuperscript{(a),(i)}} &de\exemph{yago}dáwęnyeˀ \newline ‘she is walking about’\\

\tablevspace
\multicolumn{2}{l}{\bfseries Ę, O, Ǫ, U stem}&\multicolumn{2}{l}{\bfseries E, Ę stem}\\
\cmidrule(r){1-2}\cmidrule(l){3-4}
\stem{y.ag-} &ni\exemph{ya:g}u:ˀuh \newline ‘she is small’& \stem{ya.gaw-} &\exemph{gá:w}ę: \newline ‘she has said’\\

\tablevspace
\multicolumn{2}{l}{\bfseries E stem}&\multicolumn{2}{l}{\bfseries O, Ǫ stem}\\
\cmidrule(r){1-2}\cmidrule(l){3-4}
\stem{y.ę-\textsuperscript{(e)}} &í:\exemph{yę:} \newline ‘she wants, thinks’& \stem{ya.ga-}&\exemph{ga}ǫtsanǫ́hwa:s \newline ‘her knee hurts’\\

\tablevspace
\multicolumn{2}{l}{\bfseries I stem}&\multicolumn{2}{l}{\bfseries C stem}\\
\cmidrule(r){1-2}\cmidrule(l){3-4}
\stem{y.e-\textsuperscript{(i)}} &ę\exemph{yé}dagraˀ \newline ‘she will fall down’&\stem{ya.go-}& \exemph{go}nǫhǫkdá:nih \newline ‘she is sick’\\

\tablevspace
\multicolumn{2}{l}{\bfseries C stem}\\
\stem{y.e-}&\exemph{e}gęhjih \newline ‘she is old’\\
\lspbottomrule
\end{tabularx}
\end{table}
\clearpage

The \phonet{e-} of the \stem{(y)e-} prefix merges with the initial \phonet{e-} of the verb stem to create a single vowel, \phonet{ę} \xref{ex:enasalemerger}. (Examples b-d are from \cite[113]{sasse_far_1998}.)

\ea\label{ex:enasalemerger}
\ea dǫdá\exemph{ę}ˀ\\
\gll dǫda-(y)e-e-ˀ\\
 {\dualic}-{\factual}-{\cislocative}-\textsc{3s.fi.a}-go-{\punctual}\\
\glt `she is coming back'
\cfex{dǫdá:geˀ\\
\gll dǫdá:-g-e-ˀ\\
{\dualic}-{\factual}-{\cislocative}-\textsc{1s.a}-go-{\punctual}\\
\glt ‘I am coming back’ }
\ex \exemph{ę}htaˀ\\
\gll e-eht-haˀ\\
\textsc{3s.fi.a}-go-{\habitual}\\
\glt `she usually goes'
\cfex{gehtaˀ\\
	\gll g-eht-haˀ\\
	\textsc{1s.a}-go-{\habitual}\\
	\glt `I usually go'}
\ex í:y\exemph{ę}ˀs\\
\gll í:-ye-e-ˀs\\
 {\prothetic}-\textsc{3s.fi.a}-go-{\habitual}\\
\glt `she is here'
\cfex{i:geˀs\\
	\gll í:-g-e-ˀs\\
	{\prothetic}-\textsc{1s.a}-go-{\habitual}\\
	\glt `I am here'}
\ex hę́:y\exemph{ę}:ˀ\\
\gll h-ę́:-ye-e-:-ˀ\\
 {\translocative}-{\future}-\textsc{3s.fi.a}-go-{\purposive}-{\punctual}\\
\glt `she will go there'
\cfex{hę́:ge:ˀ\\
	\gll h-ę́:-g-e-:-ˀ\\
	{\translocative}-{\future}-\textsc{1s.a}-go-{\purposive}-{\punctual}\\
	\glt `I will go there'}
\z
\z

\begin{table}
\caption{\textsc{3s} (beginning of prefix)}
\label{figtab:it.beginning}
{
\begin{tabularx}{\textwidth}{p{20mm}p{30mm}X}
\lsptoprule
gloss&pronominal&environment\\
\midrule
\textsc{3s.a}&a\exemph{gá:}gęˀ& no change\\
&\exemph{gá:}gęhs&\\
&ę\exemph{gá:}gęˀ& \\
\midrule
\textsc{3s.p}&aˀ\exemph{ó:}gęˀ&after \stem{aˀ-} {\factual}\\
\tablevspace
&\exemph{ó:}gę:&at the beginning of the word\\
\tablevspace
&ę\exemph{yó:}gęˀ& after a vowel - \stem{ę-} {\future}, \stem{a:-} {\indefinite}\\
\lspbottomrule
\end{tabularx}}
\end{table}


\begin{table}
\caption{\textsc{3s} (prefix-stem boundary)}
\label{figtab:it}
\begin{tabularx}{\textwidth}{lQlQ}
\lsptoprule
\multicolumn{2}{c}{\textsc{3s.a}}&\multicolumn{2}{c}{\textsc{3s.p}}\\
\cmidrule(r){1-2}\cmidrule(l){3-4}
\multicolumn{2}{l}{\bfseries A, E, Ę, U stem}&\multicolumn{2}{l}{\bfseries A, I stem}\\
\stem{w-} &\exemph{w}a̱hsohǫt\newline ‘it is coloured’& \stem{y.o-}\textsuperscript{(a),(i)} &ní:\exemph{yo}ht \newline ‘what it is like\\

\tablevspace
\multicolumn{2}{l}{\bfseries I-stem}&\multicolumn{2}{l}{\bfseries E, Ę stem}\\
\cmidrule(r){1-2}\cmidrule(l){3-4}
\stem{gę-\textsuperscript{(i)}} &dehs\exemph{gę́:}ne:s\newline ‘it is extinct’& \stem{aw-, y.o-\textsuperscript{(e,ę)}} &\exemph{aw}é:ˀǫ: \newline ‘it has willed, decided it’ \newline \exemph{o}neˀwaǫ: \newline ‘it was startled’\\

\tablevspace
\multicolumn{2}{l}{\bfseries O, Ǫ stem}&\multicolumn{2}{l}{\bfseries O, Ǫ stem}\\
\cmidrule(r){1-2}\cmidrule(l){3-4}
\stem{y-} &í:\exemph{y}ǫ:s \newline ‘it is long’& \stem{y.a-}&\exemph{a}ǫˀwé:sęht \newline ‘it is enjoyable’\\

\tablevspace
\multicolumn{2}{l}{\bfseries C stem} &\multicolumn{2}{l}{\bfseries  C stem}\\
\cmidrule(r){1-2}\cmidrule(l){3-4}
\stem{ga-} &\exemph{ga}ní:yǫ:t \newline ‘it is hanging’& \stem{y.o-} &\exemph{o}náˀno:ˀ \newline ‘it is cold, cool’\\
\lspbottomrule
\end{tabularx}
\end{table}

\FloatBarrier
\subsection{3rd person plural, non-interactive (or interactive with implied ‘it’)}

The initial H deletes after a T in the prefixes shown in \tabref{figtab:they.males.beginning} and \tabref{figtab:they.males}.

\begin{table}
\caption{\textsc{3ns.m} (beginning of prefix)}
\label{figtab:they.males.beginning}
{
\begin{tabularx}{\textwidth}{lXX}
\lsptoprule
gloss&pronominal&environment\\
\midrule
\textsc{3ns.m.a}&a\exemph{ha:dí:}gęˀ&no change, takes \stem{a-} {\factual}\\
&\exemph{hadí:}gęhs& \\
&ę\exemph{ha:dí:}gęˀ& \\
\midrule
\textsc{3ns.m.p}&a\exemph{ho:dí:}gęˀ&no change, takes \stem{a-} {\factual}\\
&\exemph{hodí:}gę:& \\
&ę\exemph{ho:dí:}gęˀ&  \\
\lspbottomrule
\end{tabularx}}
\end{table}

\begin{table}
\caption{\textsc{3ns.m} (prefix-stem boundary)}
\label{figtab:they.males}
{
\begin{tabularx}{\textwidth}{lQlQ}
\lsptoprule
\multicolumn{2}{c}{\textsc{3ns.m.a}}&\multicolumn{2}{c}{\textsc{3ns.m.p}}\\
\cmidrule(r){1-2}\cmidrule(l){3-4}
\multicolumn{4}{l}{\bfseries A, E, Ę, O, Ǫ, U stem}\\
\midrule
\stem{hęn-} &ni̱\exemph{hę:n}ú:sˀuh \newline ‘they are small’& \stem{hon-} &de̱\exemph{hon}adráˀǫh\newline ‘they are meeting right now’\\

\tablevspace
\multicolumn{4}{l}{\bfseries I stem}\\
\midrule
\stem{hadi-\textsuperscript{(i)}} &\exemph{hadi}ˀdrǫ́:dǫˀ\newline ‘how they are placed’& \stem{hodi-\textsuperscript{(i)}}&\exemph{hodi}dagráˀǫh \newline ‘they have fallen down’\\

\tablevspace
\multicolumn{4}{l}{\bfseries C stem}\\
\midrule
\stem{hadi-} &de̱\exemph{hadi}yáhshe:\newline ‘two males’&\stem{hodi-}&\exemph{hodi}nǫ̱hǫ́kdanih\newline ‘they are sick’\\
\lspbottomrule
\end{tabularx}}
\end{table}

\begin{table}
\caption{\textsc{3ns.fi} (beginning of prefix)}
\label{figtab:they.females.beginning}
{
\begin{tabularx}{\textwidth}{p{32mm}p{25mm}X}
\lsptoprule 
gloss&pronominal&environment\\
\midrule
\textsc{3ns.fi.a}&a\exemph{gáe}gęˀ&no change\\
&\exemph{gáegęhs}& \\
&ę\exemph{gáe}gęˀ&  \\

\tablevspace
\midrule
\textsc{3ns.fi.p}&aˀ\exemph{agodí:}gęˀ&after \stem{aˀ-} {\factual}\\
\tablevspace
&\exemph{godí:}gę:&at the beginning of the word\\
\tablevspace
&ę\exemph{yagodí:}gęˀ& after a vowel - \stem{ę-} {\future}, \stem{a:-} {\indefinite}\\
\lspbottomrule
\end{tabularx}}
\end{table}


\FloatBarrier
The \phonet{e-} of the \stem{gae-} \textsc{3ns.fi.a} prefix merges with the \phonet{e-} of the verb stem to create a single vowel, \phonet{ę} \xref{ex:enasalemerger11}.


\ea\label{ex:enasalemerger11}
\ea ętgáęˀ\\
\gll ę-t-gae-e-ˀ\\
 \fut-{\cislocative}-\textsc{3ns.fi.a}-go-{\punctual}\\
\glt `they will come'
\cfex{ęteˀ\\
	\gll ę-t-h-e-ˀ\\
	\fut-{\cislocative}-\exsc{3s.m.a}-go-{\punctual}\\
	\glt `he will come this way'}
\ex haˀgáęˀ\\
\gll haˀ-gae-e-ˀ\\
 {\translocativefactual}\exsc{-3ns.fi.a}-go-{\punctual}\\
\glt `they are going'
\cfex{haˀseˀ\\
	\gll haˀ-s-e-ˀ\\
	{\translocative.\factual}-\exsc{2s.s}-go-{\punctual}\\
	\glt `you are going'}
\ex tigáęˀs\\
\gll ti-gae-e-ˀs\\
 {\contrastive}-\textsc{3ns.fi.a}-go-{\habitual}\\
\glt `they are roaming about'
\cfex{tí:wehs\\
	\gll tí-w-e-hs\\
	{\contrastive}-\textsc{3s.a}-go-{\habitual}\\
	\glt `a stray animal'}
\z
\z
\clearpage

\begin{table}[p]
\caption{\textsc{3ns.fi} (prefix-stem boundary)}
\label{figtab:they.fe/males}
{
\begin{tabularx}{\textwidth}{lQlQ}
\lsptoprule
\multicolumn{2}{c}{\textsc{3ns.fi.a}}&\multicolumn{2}{c}{\textsc{3ns.fi.p}}\\
\cmidrule(r){1-2}\cmidrule(l){3-4}
\multicolumn{2}{l}{\bfseries A stem}&\multicolumn{2}{l}{\bfseries A, E, Ę, O, Ǫ, U stem}\\
\cmidrule(r){1-2}\cmidrule(l){3-4}
\stem{gaǫ-\textsuperscript{(a)}} &\exemph{gaǫ}gá:dǫh \newline ‘they say’& \stem{ya.gon-} &dó: ni\exemph{yagon}ohsriyáˀgǫh \newline ‘how old they are’\\

\tablevspace
\multicolumn{2}{l}{\bfseries E, Ę, O, Ǫ, U stem}&\multicolumn{2}{l}{\bfseries I stem}\\
\cmidrule(r){1-2}\cmidrule(l){3-4}
\stem{ga:g-} &\exemph{ga:g}ǫgwéˀdase: \newline ‘young women’& \stem{ya.godi-\textsuperscript{(i)}} &\exemph{godi}dagráˀǫh \newline ‘they have fallen down’\\

\tablevspace
\multicolumn{2}{l}{\bfseries E stem}&\multicolumn{2}{l}{\bfseries C stem}\\
\cmidrule(r){1-2}\cmidrule(l){3-4}
\stem{gaę-\textsuperscript{(e)}} &ę\exemph{gáę:}ˀ \newline ‘they will want’& \stem{ya.godi-}&\exemph{godi}nǫ̱hǫ́kdanih \newline ‘they are sick’\\

\tablevspace
\multicolumn{2}{l}{\bfseries I stem}\\
\cmidrule(r){1-2}
\stem{gae-\textsuperscript{(i)}} &\exemph{gáe}dagrǫˀ  \newline ‘they are lying around’\\

\tablevspace
\multicolumn{2}{l}{\bfseries C stem}\\
\cmidrule(r){1-2}
\stem{gae-}&\exemph{gáe}gę̱hjih\newline ‘they are old’\\
\lspbottomrule
\end{tabularx}}
\end{table}

\begin{table}[p]
\caption{\textsc{3p} (beginning of prefix)}
\label{figtab:they.animals.beginning}
\begin{tabularx}{0.66\textwidth}{@{~}l@{~~}p{18mm}X@{}}
\lsptoprule
gloss&pronominal&environment\\
\midrule
\textsc{3p.a}&a\exemph{ga:dí:g}ęˀ&no change, takes \stem{a-} {\factual}\\
&\exemph{gadí:}gęhs& \\
&ę\exemph{ga:dí:}gęˀ& \\
\midrule
\textsc{3p.p}&aˀ\exemph{o:dí:}gęˀ&after \stem{aˀ-} {\factual}\\
\tablevspace
&\exemph{odí:}gę:&at the beginning of the word\\
\tablevspace
&ę\exemph{yo:dí:}gęˀ& \mbox{after a vowel - \stem{ę-} {\future}, \stem{a:-} {\indefinite}}\\
\lspbottomrule
\end{tabularx}
\end{table}
\clearpage

\begin{table}[t]
\caption{\textsc{3p} (prefix-stem boundary)}
\label{figtab:they.animals}
\begin{tabularx}{\textwidth}{lQlQ}
\lsptoprule
\multicolumn{2}{l}{\textsc{3p.a}}&\multicolumn{2}{l}{\textsc{3p.p}}\\
\cmidrule(r){1-2}\cmidrule(l){3-4}
\multicolumn{2}{l}{\bfseries A E, Ę, O, Ǫ, U stem}\\
\midrule
\stem{gęn-} &ę\exemph{gęn}adéˀgoˀ \newline ‘they  will run away’& \stem{y.on-} &\exemph{on}adagáideˀ\newline ‘they are well’\\

\tablevspace
\multicolumn{2}{l}{\bfseries I stem}\\
\midrule
\stem{gadi-\textsuperscript{(i)} }&\exemph{gadí}dagrǫˀ  \newline ‘they are lying around’& \stem{y.odi-\textsuperscript{(i)}}&\exemph{odi}dagraˀǫh \newline ‘they have fallen down’\\

\tablevspace
\multicolumn{2}{l}{\bfseries C stem}\\
\midrule
\stem{gadi-}&\exemph{gadi:}tsgó:dǫˀ \newline ‘they are sitting’& \stem{y.odi-}&\exemph{odi}nǫ̱hǫ́kdanih\newline ‘they are sick’\\
\lspbottomrule
\end{tabularx}
\end{table}


\section{Interactive prefix pronunciation} \label{Interactive (GǪ) prefix meaning and pronunciation}
The pronunciation of the \textsc{interactive} pronominal prefixes is summarized in the following sections. (The meaning of each prefix is also listed, but the meaning distinctions are more fully described in \sectref{Pronominal prefix meaning}.)


\newpage
\subsection{1st and 2nd person interactions}

\begin{table}
\caption{\textsc{1>2}, interactive (beginning of prefix)}
\label{figtab:1st-2nd.beginning}
{
\begin{tabularx}{\textwidth}{QQl}
\lsptoprule
gloss& no changes\newline (beg of prefix)&environment\\
\midrule 
\textsc{1s>2s} & a\exemph{gǫ́:}gęˀ&(no change)\\
&\exemph{gǫ́:}gęhs&\\
&ę\exemph{gǫ́:}gęˀ&\\

\tablevspace
\midrule 
\textsc{1>2(d)}&a\exemph{kní:}gęˀ, \exemph{kní:}gęhs, \exemph{kní:}gęˀ &(no change)\\

\tablevspace
\midrule 
\textsc{1>2(p)}&a\exemph{gwá:}gęˀ&(no change)\\
&\exemph{gwá:}gęhs&\\
&ę\exemph{gwá:}gęˀ&\\
\lspbottomrule
\end{tabularx}}
\end{table}


\begin{table}
\caption{\textsc{1>2}, interactive (prefix-stem boundary)}
\label{figtab:1st-2nd.end}
{
\begin{tabularx}{\textwidth}{lQp{29mm}p{30mm}}
\lsptoprule
&\textsc{1s>2s}&\textsc{1>2(d)}&\textsc{1>2(p)}\\
\midrule 
C stem&\stem{gǫ-, go-\textsuperscript{n}}\newline a\exemph{gǫ́:}gęˀ&\stem{kni-}\newline a\exemph{kní:} gęˀ&\stem{gwa-}\newline a\exemph{gwá:}gęˀ\\

\tablevspace
A stem&\stem{gǫy-}\newline \exemph{gǫy}ahǫdǫ́haˀ&\stem{gy-}\newline \exemph{gy}ahǫ́dǫ̱haˀ&\stem{gwa-(a)}\newline\exemph{gw}ahǫ́dǫhaˀ\\

\tablevspace
I stem&\stem{gǫ-(i)}\newline \exemph{gǫ}hnǫ:s&\stem{kni-(i)}\newline \exemph{kni}hnǫ:s&\stem{gwę-(a)}\newline \exemph{gwę}hnǫ:s\\

\tablevspace
E, Ę stem&\stem{gǫ-(e)}\newline ę\exemph{gǫ}ne̱ˀwá:ʰdęˀ&\stem{kn-} ę\exemph{kn}ęne̱ˀwá:ʰdęˀ&\stem{gw-}\newline ę\exemph{gw}ęne̱ˀwá:ʰdęˀ\\

\tablevspace
O, Ǫ stem&\stem{gǫy-}\newline ę\exemph{gǫ́:y}ǫˀ&\stem{kn-}\newline ę\exemph{kn}ǫˀ&\stem{gy-}\newline ę\exemph{gy}ǫˀ\\
\lspbottomrule
\end{tabularx}}
\end{table}
\clearpage

\begin{table}[p!]
\caption{\textsc{2>1}, interactive (beginning of prefix)}
\label{figtab:2nd-1st.beginning}
{
\begin{tabularx}{\textwidth}{lp{30mm}Q}
\lsptoprule
gloss&changes\newline (beg of prefix)&environment\\
\midrule 
\textsc{2s>1s}&a\exemph{sgé:}gęˀ&(no change)\\
&\exemph{sgé:}gęhs&\\
&ę\exemph{hsgé:}gęˀ&\\
\midrule 
\textsc{2>1(d)}&\exemph{eskní:}gęˀ&\stem{e-} {\factual}\\
% \tablevspace
&\exemph{skní:}gęhs&at the beginning of the word\\
% \tablevspace
&ę\exemph{hskní:}gęˀ&after a vowel - \stem{ę-} {\future} or

\stem{a:-} {\indefinite}\\
\midrule
\textsc{2>1(p)}&\exemph{esgwá:}gęˀ&\stem{e-} {\factual}\\
% \tablevspace
&\exemph{sgwá:}gęhs&at the beginning of the word\\
% \tablevspace
&ę\exemph{hsgwá:}gęˀ&after a vowel - \stem{ę-} {\future} or

\stem{a:-} {\indefinite}\\
\lspbottomrule
\end{tabularx}}
\end{table}


\begin{table}[p!]
\caption{\textsc{2>1}, interactive (prefix-stem boundary)}
\label{figtab:2nd-1st.end}
{
\begin{tabularx}{\textwidth}{lQp{29mm}p{32mm}}
\lsptoprule
&\textsc{2s>1s}&\textsc{2>1(d)}&\textsc{2>1(p)} \\
\midrule
C stem&\stem{(h)sge-}

a\exemph{sgé:}gęˀ&\stem{(h)skni-}

\exemph{eskní:}gęˀ&\stem{(h)sgwa-}

\exemph{esgwá:}gęˀ\\

\tablevspace
A stem&\stem{(h)sge-}

\exemph{sg}a̱hǫ́dǫ̱haˀ&\stem{(h)sgy-}

\exemph{sgy}a̱hǫ́dǫhaˀ&\stem{(h)sgw-}

\exemph{sgw}a̱hǫ́dǫ̱haˀ\\

\tablevspace
I stem&\stem{(h)sge-}

\exemph{sg}i̱hnǫ:s&\stem{(h)skni-(i)}

\exemph{skni}hnǫ:s&\stem{(h)sgwę-(i)}

\exemph{sgwę}hnǫ:s\\

\tablevspace
E, Ę stem&\stem{(h)sge-}

ę\exemph{hsg}ęne̱ˀwá:ʰdęˀ&\stem{(h)skn-}

ę\exemph{skn}ęne̱ˀwá:ʰdęˀ&\stem{(h)sgw-}

ę\exemph{sgw}ęne̱ˀwá:ʰdęˀ\\

\tablevspace

O, Ǫ stem&\stem{(h)sge-}

ę\exemph{hsg}ǫˀ&\stem{(h)skn-}

ę\exemph{hskn}ǫˀ&\stem{(h)sgy-}

ę\exemph{sgy}ǫˀ\\
\lspbottomrule
\end{tabularx}}
\end{table}


\clearpage
\subsection{\textsc{3fi>1} and \textsc{3fi>2} interactions}
~\vspace*{-\baselineskip}\largerpage[5]

\begin{table}
\caption{\textsc{3fi>1} and \textsc{1>3fi} interactions (beginning of prefix)}
\label{figtab:1-3.beginning}
{
\begin{tabularx}{\textwidth}{lp{30mm}Q}
\lsptoprule
gloss&beginning of prefix&environment\\
\midrule 
\textsc{1s>3s.fi}&a\exemph{ké:}gęˀ&(no change)\\
&\exemph{ké:}gęhs&\\
&ę\exemph{ké:}gęˀ&\\
\midrule 
\textsc{1ns.ex>3s.fi/3ns} &aˀ\exemph{aki:}gęˀ, \exemph{akí:}gęˀ*&after \stem{a-} {\factual} (*{\factual} optionally deletes)\\
\tablevspace
&\exemph{akí:}gęhs&at the beginning of the word\\
\tablevspace
&ę\exemph{ya:kí:}gęˀ&after a vowel - \stem{ę-} {\future} or \stem{a:-} {\indefinite}\\
\midrule 
\textsc{1ns.in>3s.fi/3ns} &aˀ\exemph{e:tí:}gęˀ&after \stem{a-} {\factual}\\
\tablevspace
&\exemph{etí:}gęhs&at the beginning of the word\\
\tablevspace
&ę\exemph{ye:tí:}gęˀ&after a vowel - \stem{ę-} {\future} or \stem{a:-} {\indefinite}\\
\midrule
\textsc{3s.fi>1s} &aˀ\exemph{ǫ:gé:}gęˀ&after \stem{a-} {\factual} (*{\factual} optionally deletes)\\
\tablevspace
&\exemph{ǫgé:}gęhs&after \stem{a-} {\factual}\\
\tablevspace
&ę\exemph{yǫ:gé:}gęˀ&at the beginning of the word\\
\midrule
\textsc{3s.fi/3ns>1ns}&aˀ\exemph{ǫ:kí:}gęˀ&after a vowel - \stem{ę-} {\future} or \stem{a:-} {\indefinite}\\
\tablevspace
&\exemph{ǫkí:}gęhs&after \stem{a-} {\factual}\\
\tablevspace
&ę\exemph{yǫ:kí:}gęˀ&at the beginning of the word\\
\tablevspace
&&after a vowel - \stem{ę-} {\future} or \stem{a:-} {\indefinite}\\
\lspbottomrule
\end{tabularx}}
\end{table}
\clearpage

\begin{table}[p!]
\caption{\textsc{3fi>1} and \textsc{1>3fi} interactions (prefix-stem boundary)}
\label{figtab:1-3.end}
{\small
\begin{tabularx}{\textwidth}{lQQQQ}
\lsptoprule
&\textsc{1s>3s.fi}&\textsc{1ns.ex>3s.fi/3ns}&\textsc{1ns.in>3s.fi/3ns}&\textsc{3s.fi>1s}\\
\midrule
C stem&\stem{ke-}

a\exemph{ké:}gęˀ&\stem{y.aki-}

aˀ\exemph{a:kí:}gęˀ&\stem{y.eti-}

aˀ\exemph{e:tí:}gęˀ&\stem{y.ǫge-}

aˀ\exemph{ǫgé:}gęˀ\\
\tablevspace
A stem&\stem{key-}

\exemph{key}ahǫdǫ́haˀ &\stem{y.akiy-}

\exemph{akiy}a̱hǫ́dǫ̱haˀ&\stem{y.etiy-}

\exemph{etiy}a̱hǫ́dǫ̱haˀ&\stem{y.ǫg-}

\exemph{ǫg}ahǫdǫ́haˀ\\
\tablevspace
I stem&\stem{ke-\textsuperscript{(i)}}

\exemph{ke}hnǫ:s&\stem{y.aki-\textsuperscript{(i)}}

\exemph{ak}íhnǫ:s&\stem{y.eti-\textsuperscript{(i)}}

\exemph{etí}hnǫ:s&\stem{y.ǫg-}

\exemph{ǫg}íhnǫ:s\\
\tablevspace
E, Ę stem&\stem{key-}

ę\exemph{key}ęneˀ\-wá:ʰdęˀ&\stem{y.akiy-}

ę\exemph{yakiy}ęne̱\-ˀwá:ʰdęˀ&\stem{y.etiy-}

ę\exemph{yetiy}ęne̱ˀ\-wá:ʰdęˀ&\stem{y.ǫg-}

ę\exemph{yǫg}ęneˀ\-wá:ʰdęˀ\\
\tablevspace
O, Ǫ stem&\stem{key-}

ę\exemph{ké:y}ǫˀ&\stem{y.akiy-}

ę\exemph{ya:kí:y}ǫˀ&\stem{y.etiy-}

ę\exemph{ye:tí:y}ǫˀ&\stem{y.ǫg-}

ę\exemph{yǫ́:g}ǫˀ\\
\lspbottomrule
\end{tabularx}}
\end{table}


\begin{table}[p!]
\caption{\textsc{2>3s.fi} and \textsc{3s.fi>2} interactions (beginning of prefix)}
\label{figtab:2-3.beginning}
{
\begin{tabularx}{\textwidth}{lQQ}
\lsptoprule
gloss&beginning of prefix&environment\\
\midrule 
\textsc{2s>3s.fi}&a\exemph{shé:}gęˀ&(no change)\\
&\exemph{shé:}gęhs&\\
&ę\exemph{shé:}gęˀ&\\
\midrule 
\textsc{3s.fi>2s}&aˀ\exemph{e:sá:}gęˀ&after \stem{aˀ-} {\factual}\\
% \tablevspace
&\exemph{esá:}gęhs&at the beginning of the word\\
% \tablevspace
&ę\exemph{ye:sá:}gęˀ&after a vowel - \stem{ę-} {\future} or \stem{a:-} {\indefinite}\\
\midrule 
\textsc{3s.fi/3ns>2ns(opp)}&aˀ\exemph{e:tsí:}gęˀ&after \stem{aˀ-} {\factual}\\
% \tablevspace
&\exemph{etsí:}gęhs&at the beginning of the word\\
% \tablevspace
&ę\exemph{ye:tsí:}gęˀ&after a vowel - \stem{ę-} {\future} or \stem{a:-} {\indefinite}\\
\lspbottomrule
\end{tabularx}}
\end{table}
\clearpage

\begin{table}[t]
\caption{\textsc{2>3s.fi} and \textsc{3s.fi>2} interactions (prefix-stem boundary)}
\label{figtab:2-3.end}
{
\begin{tabularx}{\textwidth}{lQQQ}
\lsptoprule
&\textsc{2s>3s.fi}&\textsc{3s.fi>2s}&\textsc{3s.fi/3ns>2ns(opp)}\\
\midrule 
C stem&\stem{she-}

a\exemph{shé:}gęˀ&\stem{y.esa-}

aˀ\exemph{e:sá:}gęˀ&\stem{y.etsi-}

aˀ\exemph{e:tsí:}gęˀ\\
\tablevspace
A stem&\stem{shey-}

\exemph{shey}ahǫdǫ́haˀ&\stem{y.esa-\textsuperscript{(a)}}

\exemph{esa}hǫdǫ́haˀ&\stem{y.etsiy-}

\exemph{etsiy}a̱hǫ́dǫ̱haˀ\\
\tablevspace
I stem&\stem{she-(i)}

\exemph{she}hnǫ:s&\stem{y.esę-\textsuperscript{(i)}}

\exemph{esę́}hnǫ:s&\stem{y.etsi-\textsuperscript{(i)}}

\exemph{etsí}hnǫ:s\\
\tablevspace
E, Ę stem&\stem{shey-}

ę\exemph{shey}ęneˀ\-wá:ʰdęˀ&\stem{y.es-}

ę\exemph{yes}ęneˀ\-wá:ʰdęˀ&\stem{y.etsiy-}

ę\exemph{yetsiy}ęne̱ˀ\-wá:ʰdęˀ\\
\tablevspace
O, Ǫ stem&\stem{shey-}

ę\exemph{shé:y}ǫˀ&\stem{y.es-}

ę\exemph{yé:s}ǫˀ&\stem{y.etsiy-}

ę\exemph{ye:tsí:y}ǫˀ\\
\lspbottomrule
\end{tabularx}}
\end{table}


\clearpage
\subsection{\textsc{3s.m} and \textsc{1} interactions}
\begin{table}
\caption{\textsc{1>3s.m} (no changes to beginning of prefix)}
\label{figtab:I/we:him.beginning}
{
\begin{tabularx}{0.66\textwidth}{lXl}
\lsptoprule
gloss&beginning of prefix&environment\\
\midrule 
\textsc{1s>3s.m} &a\exemph{hé:}gęˀ, 

\exemph{hé:}gęhs, 

ę\exemph{hé:}gęˀ&(no change)\\
\tablevspace
\midrule
\textsc{1d.ex>3s.m}&a\exemph{sha:kní:}gęˀ, 

\exemph{shakní:}gęhs, 

ę\exemph{hsha:kní:}gęˀ&(no change)\\
\tablevspace
\midrule
\textsc{1d.in>3s.m}&a\exemph{she:tní:}gęˀ, 

\exemph{shetní:}gęhs, 

ę\exemph{she:tní:}gęˀ&(no change)\\
\tablevspace
\midrule
\textsc{1pl.ex>3s.m}&a\exemph{sha:gwá:}gęˀ, 

\exemph{shagwá:}gęhs, 

ę\exemph{sha:gwá:}gęˀ&(no change)\\
\tablevspace
\midrule
\textsc{1pl.in>3s.m}&a\exemph{she:dwá:}gęˀ, 

\exemph{shedwá:}gęhs, 

ę\exemph{she:dwá:}gęˀ&(no change)\\
\lspbottomrule
\end{tabularx}}
\end{table}

\begin{sidewaysfigure}
\caption{\textsc{1>3s.m} (prefix-stem boundary)}
\label{figtab:I/we:him.end}
{
\begin{tabularx}{\textwidth}{lQQQQQ}
\lsptoprule
&\textsc{1s>3s.m}&\textsc{1d.ex>3s.m}&\textsc{1d.in>3s.m}&\textsc{1pl.ex>3s.m}&\textsc{1pl.in>3s.m}\\
\midrule 
C stem&a\exemph{hé:}gęˀ&a\exemph{sha:kní:}gęˀ&a\exemph{she:tní:}gęˀ&a\exemph{sha:gwá:}gęˀ&a\exemph{she:dwá:}gęˀ\\
\tablevspace
A stem&\exemph{hey}a\-hǫdǫ́haˀ&\exemph{shagy}a\-hǫdǫ́haˀ&\exemph{shegy}a\-hǫdǫ́haˀ&\exemph{shagwa}\-hǫdǫ́haˀ&\exemph{shedwa}\-hǫdǫ́haˀ, \exemph{shetwa}\-hǫdǫ́haˀ\\
\tablevspace
I stem&\exemph{he}hnǫ:s&\exemph{shakní}hnǫ:s&\exemph{shetní}hnǫ:s&\exemph{shagwę́}\-hnǫ:s&\exemph{shedw}ę́hnǫ:s, \exemph{shetw}ę́hnǫ:s\\
\tablevspace
E, Ę stem&ę\exemph{hey}ęneˀ\-wá:ʰdęˀ&ę\exemph{shagy}ęneˀ\-wá:ʰdęˀ&ę\exemph{shetn}ęneˀ\-wá:ʰdęˀ&ę\exemph{shagw}ęneˀ\-wá:ʰdęˀ&ęh\exemph{shedw}ęneˀ\-wá:ʰdęˀ, ęh\exemph{shetw}ęneˀ\-wá:ʰdęˀ\\
\tablevspace
O, Ǫ stem&ę\exemph{hé:y}ǫˀ&ę\exemph{hshá:kn}ǫˀ&ę\exemph{hshé:tn}ǫˀ&ę\exemph{shá:gy}ǫˀ&ę\exemph{hshégy}ǫˀ\\
\lspbottomrule
\end{tabularx}}
\end{sidewaysfigure}

\begin{table}
\caption{\textsc{3s.m>1} (no changes to beginning of prefix)}
\label{figtab:he:me/us.beginning}
{
\begin{tabularx}{0.66\textwidth}{XXl}
\lsptoprule
gloss&beginning of prefix&environment\\
\midrule 
\textsc{3s.m>1s}&a\exemph{ha:gé:}gęˀ, 

\exemph{hagé:}gęhs, 

ę\exemph{ha:gé:}gęˀ&(no change)\\
\midrule 
\textsc{3s.m>1d} &a\exemph{shǫ:kní:}gęˀ, 

\exemph{shǫkní:}gęhs, 

ę\exemph{shǫ:kní:}gęˀ&(no change)\\
\midrule 
\textsc{3s.m>1p}&a\exemph{shǫ:gwá:}gęˀ, 

\exemph{shǫgwá:}gęhs, 

ę\exemph{shǫ:gwá:}gęˀ& (no change)\\
\lspbottomrule
\end{tabularx}}
\end{table}

\begin{sidewaysfigure}
\caption{\textsc{3s.m>1} (prefix-stem boundary)}
\label{figtab:he:me/us.end}
{
\begin{tabularx}{\textwidth}{XXXX}
\lsptoprule
&\textsc{3s.m>1s}&\textsc{3s.m>1d}&\textsc{3s.m>1p}\\
\midrule 
C stem&\stem{hage-}

a\exemph{ha:gé:}gęˀ&\stem{shǫkni-}

a\exemph{shǫ:kní:}gęˀ&\stem{shǫgwa-}

a\exemph{shǫ:gwá:}gęˀ\\
\tablevspace
A stem&\stem{hag-}

\exemph{hag}ahǫdǫ́haˀ & \stem{shǫgy-}

\exemph{shǫgy}a\-hǫdǫ́haˀ&\stem{shǫgwa-\textsuperscript{(a)}}

\exemph{shǫgwa}\-hǫdǫ́haˀ\\
\tablevspace
I stem&\stem{hag-}

\exemph{hag}íhnǫ:s&\stem{shǫkni-\textsuperscript{(i)}}

\exemph{shǫkní}hnǫ:s&\stem{shǫgwę-\textsuperscript{(i)}}

\exemph{shǫgwę́}hnǫ:s\\
\tablevspace
E, Ę stem&\stem{hag-}

ę\exemph{hag}ęne̱ˀ\-wá:ʰdęˀ&\stem{shǫgy-}

ę\exemph{shǫgy}ęneˀ\-wá:ʰdęˀ&\stem{shǫgw-}

ę\exemph{shǫgw}ęneˀ\-wá:ʰdęˀ\\
\tablevspace
O, Ǫ stem&\stem{hag-}

ę\exemph{há:g}ǫˀ&\stem{shǫkn-}

ę\exemph{shǫ́:kn}ǫˀ&\stem{shǫgy-}

ę\exemph{shǫ́:gy}ǫˀ\\
\lspbottomrule
\end{tabularx}}
\end{sidewaysfigure}


\clearpage
\subsection{\textsc{3s.m} and \textsc{2} interactions}
\begin{table}
\caption{\textsc{3s.m>2}, \textsc{2>3s.m} (beginning of prefix)}
\label{figtab:he/you:you/him.beginning}
{
\begin{tabularx}{\textwidth}{p{30mm}p{30mm}Q}
\lsptoprule
gloss&beginning of prefix&environment\\
\midrule 
\textsc{2s>3s.m} &a\exemph{hehsé:}gęˀ&(no change)\\
&\exemph{hehsé:}gęhs&\\
&ę\exemph{hehsé:}gęˀ&\\
\tablevspace
\midrule
\textsc{3s.m>2s}&e\exemph{hyá:}gęˀ, a\exemph{hyá:}gęˀ&after \stem{a-} {\factual}, or \stem{e-} {\factual}\\
\tablevspace
&\exemph{hyá:}gęhs&at the beginning of the word\\
\tablevspace
&ę\exemph{hyá:}gęˀ, áe\exemph{hya}gęˀ&after a vowel - \stem{ę-} {\future} or \stem{ae-} {\indefinite}\\
\tablevspace
\midrule
\textsc{2d>3s.m(opp)}&a\exemph{she:sní:}gęˀ&(no change)\\
&\exemph{shesní:}gęhs&\\
&ę\exemph{she:sní:}gęˀ&\\
\tablevspace
\midrule
\textsc{2d>3s.m(opp)}&a\exemph{she:swá:}gęˀ&(no change)\\
&\exemph{sheswá:}gęhs&\\
&ę\exemph{she:swá:}gęˀ&\\
\lspbottomrule
\end{tabularx}}
\end{table}

\begin{sidewaysfigure}
\caption{\textsc{3s.m>2}, \textsc{2>3s.m} (prefix-stem boundary)}
\label{figtab: he/you:you/him.end }
{
\begin{tabularx}{\textwidth}{lQQQQQ}
\lsptoprule
&\textsc{2s>3s.m}&\textsc{3s.m>2s}&\textsc{2d>3s.m(opp)}&\textsc{2p>3s.m(opp)}\\
\midrule 
C stem&\stem{hehse-}

a\exemph{hehsé:}gęˀ&\stem{hya-}

\exemph{ehyá:}gęˀ&\stem{shesni-}

a\exemph{she:sní:}gęˀ&\stem{sheswa-}

a\exemph{she:swá:}gęˀ\\
\tablevspace
A stem&\stem{hehs-}

\exemph{hehs}ahǫdǫ́haˀ&\stem{hya-\textsuperscript{(a)}}

\exemph{hy}a̱hǫ́dǫ̱haˀ&\stem{shej-}

\exemph{shej}a\-hǫdǫ́haˀ&\stem{sheswa-\textsuperscript{(a)}}

\exemph{sheswa}\-hǫdǫ́haˀ\\
\tablevspace
I stem&\stem{hehs-}

\exemph{hehs}íhnǫ:s&\stem{hyę-\textsuperscript{(i)}}

ę\exemph{hyę́}hnǫksaˀ&\stem{shesni-\textsuperscript{(i)}}

\exemph{shesní}hnǫ:s&\stem{sheswę-\textsuperscript{(i)}}

\exemph{sheswę́}hnǫ:s\\
\tablevspace
E, Ę stem&\stem{hehs-}

ę\exemph{hehs}ęneˀ\-wá:ʰdęˀ&\stem{hy-}

ę\exemph{hy}ęne̱ˀ\-wá:ʰdęˀ&\stem{shesn-}

ę\exemph{shesn}ęneˀ\-wá:ʰdęˀ&\stem{shesw-}

ę\exemph{shesw}ęneˀ\-wá:ʰdęˀ\\
\tablevspace
O, Ǫ stem&\stem{hehs-}

ę\exemph{hé:hs}ǫˀ&\stem{hyay-}

ę\exemph{hyá:y}ǫˀ&\stem{shesn-}

ę\exemph{hshé:sn}ǫˀ&\stem{shej-}

ę\exemph{hshé:j}ǫˀ\\
\lspbottomrule
\end{tabularx}}
\end{sidewaysfigure}


\clearpage
\subsection{\textsc{3s.fi/3ns} and \textsc{1} interactions}
\begin{table}
\caption{\textsc{3s.fi,3ns>1} (beginning of prefix)}
\label{figtab:they:me/us.beginning}
{
\begin{tabularx}{\textwidth}{lp{30mm}Q}
\lsptoprule
gloss&changes\newline (beg of prefix)&environment\\
\midrule 
\textsc{3ns>1s} &a\exemph{gaǫgé:}gęˀ&(no change)\\
&\exemph{gaǫgé:}gęhs&\\
&ę\exemph{gaǫgé:}gęˀ&\\
\tablevspace
\midrule 
\textsc{3s.fi/3ns>1ns}&aˀ\exemph{ǫ:kí}gęˀ&after \stem{aˀ-} {\factual}\\
\tablevspace
&\exemph{ǫkí:}gęhs&at the beginning of the word\\
\tablevspace
&ę\exemph{yǫ:kí:}gęˀ&after a vowel - \stem{ę-} {\future} or

\stem{a:-} {\indefinite}\\
\tablevspace

\lspbottomrule
\end{tabularx}}
\end{table}

\begin{table}
\caption{\textsc{3s.fi,3ns>1} (prefix-stem boundary)}
\label{figtab:they:me/us.end}
{
\begin{tabularx}{\textwidth}{lQQ}
\lsptoprule
&\textsc{3ns>1s}&\textsc{3s.fi/3ns>1ns}\\
\midrule 
C stem&\stem{gaǫge-}

a\exemph{gaǫgé:}gęˀ&\stem{y.ǫki-}

aˀ\exemph{ǫ:kí:}gęˀ\\
\tablevspace
A stem&\stem{gaǫg-}

\exemph{gaǫg}\-a̱hǫ́dǫ̱haˀ&\stem{y.ǫkiy-}

\exemph{ǫkiy}\-a̱hǫ́dǫ̱haˀ\\
\tablevspace
I stem&\stem{gaǫg-}

\exemph{gáǫg}i̱hnǫ:s&\stem{y.ǫki-\textsuperscript{(i)}}

\exemph{ǫkí}hnǫ:s\\
\tablevspace
E, Ę stem&\stem{gaǫg-}

ę\exemph{gaǫg}ęne̱ˀ\-wá:dʰęˀ&\stem{y.ǫkiy-}

ę\exemph{yǫkiy}ęne̱ˀ\-wá:ʰdęˀ\\
\tablevspace
O, Ǫ stem&\stem{gaǫg-}

ę\exemph{gáǫg}ǫˀ&\stem{y.ǫkiy-}

ę\exemph{yo:kí:y}ǫˀ\\
\lspbottomrule
\end{tabularx}}
\end{table}
\clearpage

\begin{table}[p!]
\caption{\textsc{1>3s.fi,3ns} (changes to beginning of prefix)}
\label{figtab:I/we:her/them.beginning}
{
\begin{tabularx}{\textwidth}{lQQ}
\lsptoprule
gloss&changes\newline (beg of prefix)&environment\\
\midrule 
\textsc{1s>3ns}&a\exemph{ga:ké:}gęˀ, a\exemph{ka:ké:}gęˀ&(no change)\\
&\exemph{gaké:}gęhs, \exemph{kaké:}gęhs&\\
&ę\exemph{ga:ké:}gęˀ, ę\exemph{ka:ké:}gęˀ&\\
\midrule 
\textsc{1ns.ex>3s.fi/3ns} &aˀ\exemph{a:kí:}gęˀ, \exemph{akí:}gęˀ*&after \stem{a-} {\factual}, (*{\factual} optionally deletes)\\
\tablevspace
&\exemph{akí:}gęhs&at the beginning of the word\\
\tablevspace
&ę\exemph{ya:kí:}gęˀ&after a vowel - \stem{ę-} {\future} or \stem{a:-} {\indefinite}\\
\midrule 
\textsc{1ns.in>3s.fi/3ns} &aˀ\exemph{e:tí:}gęˀ&after \stem{aˀ-} {\factual}\\
\tablevspace
&\exemph{etí:}gęhs&at the beginning of the word\\
% \tablevspace
&ę\exemph{ye:tí:}gęˀ&after a vowel - \stem{ę-} {\future} or \stem{a:-} {\indefinite}\\
\lspbottomrule
\end{tabularx}}
\end{table}
\clearpage

\begin{table}
\caption{\textsc{1>3s.fi,3ns} (prefix-stem boundary)}
\label{figtab:I/we:her/them.end}
{
\begin{tabularx}{\textwidth}{lQQQ}
\lsptoprule
&\textsc{1s>3ns}&\textsc{1ns.ex>3s.fi/3ns}&\textsc{1ns.in>3s.fi/3ns}\\
\midrule 
C stem&\stem{gake-, kake-}

a\exemph{gaké:}gęˀ, a\exemph{kaké:}gęˀ&\stem{y.aki-}

aˀ\exemph{a:kí:}gęˀ, \exemph{akí:}gęˀ&\stem{y.eti-}

aˀ\exemph{e:tí:}gęˀ\\
\tablevspace
A stem&\stem{gakey-, kakey-}

\exemph{gakey}\-a̱hǫ́dǫ̱haˀ, 

\exemph{kakey}\-a̱hǫ́dǫ̱haˀ&\stem{y.akiy-}

\exemph{akiy}\-a̱hǫ́dǫ̱haˀ&\stem{y.etiy-}

\exemph{etiy}\-a̱hǫ́dǫ̱haˀ\\
\tablevspace
I stem&\stem{gake-\textsuperscript{(i)}, kake-\textsuperscript{(i)}}

\exemph{gaké}hnǫ:s, 

\exemph{kaké}hnǫ:s&\stem{y.aki-\textsuperscript{(i)}}

\exemph{akí}hnǫ:s&\stem{y.eti-\textsuperscript{(i)}}

\exemph{etí}hnǫ:s\\
\tablevspace
E, Ę stem&\stem{gakey-, kakey-}

ę\exemph{gakey}ęne̱ˀ\-wá:ʰdęˀ, 

ę\exemph{kakey}ęne̱ˀ\-wá:ʰdęˀ&\stem{y.akiy-}

ę\exemph{yakiy}ęne̱ˀ\-wá:ʰdęˀ&\stem{y.etiy-}

ę\exemph{yetiy}ęneˀ\-wá:ʰdęˀ\\
\tablevspace
O, Ǫ stem&\stem{gakey-, kakey-}

ę\exemph{ga:ké:y}ǫˀ, 

ę\exemph{ka:ké:}yǫˀ&\stem{y.akiy-}

ę\exemph{ya:kí:y}ǫˀ&\stem{y.etiy-}

ę\exemph{ye:tí:y}ǫˀ\\
\lspbottomrule
\end{tabularx}}
\end{table}

\clearpage
\subsection{\textsc{3s.fi/3ns} and \textsc{2} interactions}

\begin{table}
\caption{\textsc{3s.fi,3ns>2} and \textsc{2>3s.fi,3ns} (changes to beginning of prefix)}
\label{figtab:you.they.beginning}
{
\begin{tabularx}{\textwidth}{QQQ}
\lsptoprule
gloss&beginning of prefix&environment\\
\midrule 
\textsc{2s>3ns} &a\exemph{ga:shé:}gęˀ, a\exemph{ka:shé:}gęˀ&(no change)\\
&\exemph{gashé:}gęhs, \exemph{kashé:}gęhs&\\
&ę\exemph{ga:shé:}gęˀ, ę\exemph{ka:shé:}gęˀ&\\
\tablevspace
\midrule
\textsc{3s.fi/3ns>2ns(opp)}&aˀ\exemph{e:tsí:}gęˀ&after \stem{aˀ-} {\factual}\\
\tablevspace

&\exemph{etsí:}gęhs&at the beginning of the word\\
\tablevspace

&ę\exemph{ye:tsí:}gęˀ&after a vowel - \stem{ę-} {\future} or \stem{a:-} {\indefinite}\\
\midrule 
\textsc{3ns>2s}&a\exemph{gaesá:}gęˀ&(no change)\\
&\exemph{gáesa}gęhs&\\
&ę\exemph{gaesá:}gęˀ&\\
\lspbottomrule
\end{tabularx}}
\end{table}


\begin{table}
\caption{\textsc{3s.fi,3ns>2} and \textsc{2>3s.fi,3ns} (prefix-stem boundary)}
\label{figtab:you.they.end}
{
\begin{tabularx}{\textwidth}{lp{32mm}QQ}
\lsptoprule
&\textsc{2s>3ns}&\textsc{3s.fi/3ns>2ns(opp)}& \textsc{3ns>2s}\\
\midrule 
C stem&\stem{gashe-, kashe-}

a\exemph{ga:shé:}gęˀ, 

a\exemph{ka:shé:}gęˀ&\stem{y.etsi-}

aˀ\exemph{e:tsí:}gęˀ& \stem{gaesa-}

a\exemph{gaesá:}gęˀ\\
\tablevspace
A stem&\stem{gashey-, kashey-}

\exemph{gashey}\-a̱hǫ́dǫ̱haˀ, 

\exemph{kashey}\-a̱hǫ́dǫ̱haˀ&\stem{y.etsiy-}

\exemph{etsiy}\-a̱hǫ́dǫ̱haˀ & \stem{gaesa-\textsuperscript{(a)}}

\exemph{gaesa}\-hǫ́dǫ̱haˀ\\
\tablevspace
I stem&\stem{gashe-\textsuperscript{(i)}, kashe-\textsuperscript{(i)}}

\exemph{gashé}hnǫ:s, 

\exemph{kashé}hnǫ:s&\stem{y.etsi-\textsuperscript{(i)}}

\exemph{etsí}hnǫ:s & \stem{gaesę-\textsuperscript{(i)}}

\exemph{gáes}ęhnǫ:s\\
\tablevspace
E, Ę stem&\stem{gashey- kashey-}

ę\exemph{gashey}ęne̱ˀ\-wá:ʰdęˀ, 

ę\exemph{kashey}ęne̱ˀ\-wá:ʰdęˀ&\stem{y.etsiy-}

ę\exemph{yetsiy}ęne̱ˀ\-wá:ʰdęˀ & \stem{gaes-}

ę\exemph{gaes}ęne̱ˀ\-wá:ʰdęˀ\\
\tablevspace
O, Ǫ stem&\stem{gashey-, kashey-}

ę\exemph{ga:shé:y}ǫˀ, 

ę\exemph{ka:shé:y}ǫˀ&\stem{y.etsiy-}

ę\exemph{ye:tsí:y}ǫˀ & \stem{gaes-}

ę\exemph{gáes}ǫˀ\\

\lspbottomrule
\end{tabularx}}
\end{table}




\clearpage
\subsection{\textsc{3>3} interactions}

\begin{table}
\caption{\textsc{3>3} (changes to beginning of prefix)}
\label{figtab:3on3.beginning}
{
\begin{tabularx}{\textwidth}{lQQ}
\lsptoprule
gloss&beginning of prefix&environment\\
\midrule 
\textsc{3s.m/3s.fi>3s.m}&a\exemph{hǫ́wa}gęˀ, \exemph{hǫwá:}gęhs, ę\exemph{hǫ́wa}gęˀ&(no change)\\
\midrule 
\textsc{3s.m/3s.fi>3ns.m, 3ns>3m}&a\exemph{hǫwadí:}gęˀ, 

\exemph{hǫwa:dí:}gęhs, 

ę\exemph{hǫwadí:}gęˀ&(no change)\\
\tablevspace
\midrule 
\textsc{3s.fi>3s.fi} &aˀ\exemph{ǫdadé}gęˀ&after \stem{aˀ-} {\factual}\\
\tablevspace
&\exemph{ǫda:dé:}gęhs&at the beginning of the word\\
\tablevspace
&ę\exemph{yǫdadé:}gęˀ&after a vowel - \stem{ę-} {\future} or \stem{a:-} {\indefinite}\\
\tablevspace
\midrule 
\textsc{3fi>3fi(+ns)} &a\exemph{gaǫda:dé:}gęˀ,

\exemph{gaǫdadé:}gęhs, 

ę\exemph{gaǫda:dé:}gęˀ&(no change)\\
\tablevspace
\midrule 
\textsc{3s.m>3fi/3p} &a\exemph{sha:gó:}gęˀ, 

\exemph{shagó:}gęhs, 

ę\exemph{sha:gó:}gęˀ&(no change)\\
\tablevspace
\midrule 
\textsc{3fi/3ns>3p}\footnote{\stem{hadi-} can also be used.}&a\exemph{gǫwadí:}gęˀ,

\exemph{gǫwa:dí:g}ęhs,

ę\exemph{gǫwadí:}gęˀ&(no change)\\
\tablevspace
\midrule 
\textsc{3ns(nfi)>3fi}&a\exemph{shagodí:g}ęˀ,

\exemph{shago:dí:}gęhs, 

ę\exemph{shagodí:}gęˀ&(no change)\\
\lspbottomrule
\end{tabularx}}
\end{table}

\begin{sidewaysfigure}
\caption{\textsc{3>3} (prefix-stem boundary)}
\label{figtab:3on3.end}
{
\begin{tabularx}{\textwidth}{QQQQQ}
\lsptoprule
&\textsc{3s.m/3s.fi>3s.m}&\textsc{3s.m/3s.fi>3ns.m, 3ns>3m}&\textsc{3s.fi>3s.fi}&\textsc{3fi>3fi(+ns)}\\
\midrule
C stem&\stem{hǫwa-}

a\exemph{hǫ́wa}gęˀ&\stem{hǫwadi-}

a\exemph{hǫwadí:}gęˀ&&\\
\tablevspace
C stem (G, K, CC)&&&\stem{y.ǫdade-}

aˀ\exemph{ǫdadé:}gęˀ &\stem{gaǫdade-}

a\exemph{gaǫda:dé:}gęˀ\\
C stem (R,Y)&&&\stem{y.ǫdag-}

haˀ\exemph{ǫdag}yaˀdatsęi:ˀ&\stem{gaǫdag-}

\exemph{gaǫdag}yenawáˀseh\\
C stem (HN, ˀN, N)&&&\stem{y.ǫdat-}

\exemph{ǫdat}no̱haˀ&\stem{gaǫdat-}

a\exemph{gaǫda:t}gwé:niˀ\\
\tablevspace
A stem&\stem{hǫwa-\textsuperscript{(a)}}

\exemph{hǫwa}hǫdǫ́haˀ&\stem{hǫwęn-}

\exemph{hǫwęn}a̱hǫ́dǫ̱haˀ&\stem{y.ǫdad-}

\exemph{ǫdad}a̱hǫ́dǫ̱haˀ&\stem{gaǫdad-}

\exemph{gaǫdad}ahǫdǫ́haˀ\\
\tablevspace
I stem&\stem{hǫwę-\textsuperscript{(i)}}

\exemph{hǫwę́}hnǫ:s&\stem{hǫwadi-\textsuperscript{(i)}}

\exemph{hǫwádi}hnǫ:s&\stem{y.ǫdad-}

\exemph{ǫdád}i̱hnǫ:s&\stem{gaǫdad-}

\exemph{gaǫdad}íhnǫ:s\\
\tablevspace
E, Ę stem&\stem{hǫw-}

ę\exemph{hǫw}ęneˀwá:ʰdęˀ&\stem{hǫwan-}

ę\exemph{hǫwan}ęne̱ˀwá:ʰdęˀ&\stem{y.ǫdad-}

ę\exemph{yǫdad}ęne̱ˀwá:ʰdęˀ&\stem{gaǫdad-}

ę\exemph{gaǫdad}ęneˀwá:ʰdęˀ\\
\tablevspace
O, Ǫ stem&\stem{hǫway-}

ę\exemph{hǫ́way}ǫˀ&\stem{hǫwęn-}

ę\exemph{hǫ:wę́:n}ǫˀ&\stem{y.ǫdad-}

ę\exemph{yǫ́dad}ǫˀ&\stem{gaǫdad-}

ę\exemph{gaǫdá:d}ǫˀ\\
\lspbottomrule
\end{tabularx}}
\end{sidewaysfigure}

\begin{table}
\caption{more \textsc{3>3} (prefix-stem boundary)}
\label{figtab:more3.on.3.end}
{
\begin{tabularx}{\textwidth}{lXXX}
\lsptoprule
&\textsc{3s.m>3fi/3p}&\textsc{3fi/3ns>3p}\footnote{\stem{hadi-} can also be used. See \tabref{figtab:they.males.beginning} (page \pageref{figtab:they.males.beginning}).}&\textsc{3ns(nfi)>3fi}\\
\midrule 
C stem&\stem{shago-}

a\exemph{sha:gó:}gęˀ&\stem{gǫwadi-}

a\exemph{gǫwadí:}gęˀ&\stem{shagodi-}

a\exemph{shagodí:}gęˀ\\
\tablevspace
A stem&\stem{shago-\textsuperscript{(a)}}

\exemph{shago}hǫdǫ́haˀ&\stem{gǫwan-}

\exemph{gǫwan}a̱hǫ́dǫ̱haˀ&\stem{shagon-}

\exemph{shagon}a̱hǫ́dǫ̱haˀ\\
\tablevspace
I stem&\stem{shago-\textsuperscript{(i)}}

\exemph{shagó}hnǫ:s&\stem{gǫwadi-\textsuperscript{(i)}}

\exemph{gǫwádi}hnǫ:s&\stem{shagodi-\textsuperscript{(i)}}

\exemph{shagódi}hnǫ:s\\
\tablevspace
E, Ę stem&\stem{shago-\textsuperscript{(ę)}}

ę\exemph{shago}neˀwá:ʰdęˀ&\stem{gǫwan-}

ę\exemph{gowan}ęne̱ˀwá:ʰdęˀ&\stem{shagon-}

ę\exemph{shagon}ęne̱ˀwá:ʰdęˀ\\
\tablevspace
O, Ǫ stem&\stem{shaga-}

ę\exemph{shága}ǫˀ&\stem{gǫwan-}

ę\exemph{gǫ́wan}ǫˀ&\stem{shagon-}

ę\exemph{sha:gó:n}ǫˀ\\
\lspbottomrule
\end{tabularx}}
\end{table}



\section{Variation in pronominal prefix pronunciation before O/Ǫ and E/Ę stems} \label{Variation in pronominal prefix pronunciation for O/Ǫ and E/Ę stems}
There is possibly some variation in the pronunciation of certain \textsc{p}-series pronominal prefixes, specifically before stems beginning with O, Ǫ, E, or Ę. The following sections describe \emph{expected} versus \emph{novel} pronominal prefixes.\footnote{Most of the novel forms were provided by one of the co-authors of this book.} The expected forms of the pronominal prefixes were also described earlier. The novel forms depart from those descriptions. Examples are provided in the following sections.


\subsection{Pronominal prefix variants before O and Ǫ stems}
Several alternative (and possibly new or recent) \textsc{p}-series pronominal prefixes are attested before O and Ǫ stems, as summarized in \tabref{figtab:1:ObeforeO} (page \pageref{figtab:1:ObeforeO}). 

\begin{table}
\caption{\textsc{p}-series pronominal prefixes before O and Ǫ stems}
\label{figtab:1:ObeforeO}
{
\begin{tabularx}{\textwidth}{XXl}
\lsptoprule
expected & novel & meaning\\
\midrule
(h)sa & (h)say & \textsc{2s.p}\\
(y)ǫgy & (y)ǫgway & \textsc{1p.p}\\
(h)sn & (h)sniy & \textsc{2d.p}\\
j & (h)sway & \textsc{2p.p}\\
(ya)ga & (ya)goy, (ya)g, (ya)gwa & \textsc{3s.fi.p}\\
ha & h, hoy & \textsc{3s.m.p}\\
(y)a & ø, (y)oy & \textsc{3s.p}\\
hon & hodin & \textsc{3ns.m.p}\\
gon & godin & \textsc{3ns.fi.p}\\
on & odin & \textsc{3p.p}\\
\lspbottomrule
\end{tabularx}}
\end{table}

Example \xref{ex:ppvariation3} (page \pageref{ex:ppvariation3}) contrasts the novel and expected forms of the prefixes. The verbs used in the examples are \stem{ǫtsanǫhwa:s} ‘have a sore knee’ and \stem{de-…odaihsi} ‘comb someone’s hair’.\footnote{Some of the examples provided are non-sensical but grammatical -- meaning the word is possible, but it is hard to imagine how or when the  word would be used.} The novel forms all tend to preserve both the prefix and the stem vowel.

\ea\label{ex:ppvariation3} 
\ea novel 
\gll \stem{dę-hsay-odáihsiˀ}\\
{\dualicfuture}\exsc{-2s.p}-hair.comb.{\punctual}\\
\cfex{expected \stem{dę-hs-odáihsiˀ}}
\ex novel 
\gll \stem{dę-yǫgway-odáihsiˀ} \\
{\dualicfuture}\exsc{-1p.p}-hair.comb.{\punctual}\\
\cfex{expected \stem{dę-yǫgy-odáihsiˀ}}
\ex novel 
\gll \stem{dę-hsniy-odáihsiˀ} \\
{\dualicfuture}\exsc{-2s.p}-hair.comb.{\punctual}\\
\cfex{expected \stem{dę-hsn-odáihsiˀ}}
\ex novel 
\gll \stem{dę-hsway-odáihsiˀ} \\
{\dualicfuture}\exsc{-2p.p}-hair.comb.{\punctual}\\
\cfex{expected \stem{dę-j-odáihsiˀ}}
\ex novel 
\gll \stem{dę-yagoy-odáihsiˀ} \\
{\dualicfuture}\exsc{-3s.fi.p}-hair.comb.{\punctual}\\
\cfex{ expected \stem{dę-yaga-odáihsiˀ}}
\ex novel 
\gll \stem{g-ǫtsánǫ̱hwa:s}, \stem{gwa-ǫtsanǫ́hwa:s} \\
\textsc{3s.fi.p}-sore.knee.{\stative} \exsc{3s.fi.p}-sore.knee.{\stative} \\
\cfex{expected \stem{ga-ǫtsanǫ́hwa:s}}
\ex novel 
\gll \stem{h-ǫtsánǫ̱hwa:s}\\
\textsc{3s.m.p}-sore.knee.{\stative}\\
\cfex{expected \stem{ha-ǫtsanǫ́hwa:s}}
\ex novel 
\gll \stem{dę-hoy-odáihsiˀ} \\
{\dualicfuture}\exsc{-3s.m.p}-hair.comb.{\punctual}\\
\cfex{expected \stem{dę-ha-odáihsiˀ}}
\ex novel 
\gll \stem{ø-ǫtsánǫ̱hwa:s} {} \\
{\noprefix}-sore.knee.{\stative} (ø = no prefix)\\
\cfex{expected \stem{a-ǫtsanǫ́hwa:s} \textsc{3s.p}-sore.knee.{\stative}}
\ex novel 
\gll \stem{dę-yoy-odáihsiˀ} \\
{\dualicfuture}\exsc{-3s.p}-hair.comb.{\punctual}\\
\cfex{expected \stem{dę-ya-odáihsiˀ}}
\ex novel 
\gll \stem{hodin-ǫtsánǫ̱hwa:s} \\
\textsc{3ns.m.p}-sore.knee.{\stative}\\
\cfex{expected \stem{hon-ǫtsanǫ́hwa:s}}
\ex novel 
\gll \stem{godin-ǫtsánǫ̱hwa:s} \\
\exsc{3ns.fi.p}-sore.knee.{\stative}\\
\cfex{expected \stem{gon-ǫtsanǫ́hwa:s}}
\ex novel 
\gll \stem{odin-ǫtsánǫ̱hwa:s} \\
\exsc{3p.p}-sore.knee.{\stative}\\
\cfex{ expected \stem{on-ǫtsanǫ́hwa:s}}
\z
\z


\subsection{Pronominal prefix variants before E and Ę stems}
Several alternative (and possibly new or recent) \textsc{p}-series pronominal prefixes are also attested before E and Ę stems, as summarized in \tabref{figtab:1:obeforee}. 

\begin{table}
\caption{\textsc{p}-series pronominal prefixes before E, Ę stems}
\label{figtab:1:obeforee}
{
\begin{tabularx}{\textwidth}{p{30mm}Xl}
\lsptoprule
expected & novel & meaning\\
\midrule
(y)ǫkn & (y)ǫgy & \textsc{1d.p}\\
(h)sn & j & \textsc{2d.p}\\
haw & ho* (*stem vowel deletes) & \textsc{3s.m.p} \\
(ya)gaw & (ya)go* (*stem vowel deletes) & \textsc{3s.fi.p} \\
(y)aw & (y)o* (*stem vowel deletes) & \textsc{3s.p}  \\
\lspbottomrule
\end{tabularx}}
\end{table}

Example \xref{ex:ppvariation13} contrasts the novel and expected pronominal prefix forms. The verbs used in the examples are E stem \stem{e:ˀǫ} ‘to will something’, Ę stem \stem{ęneˀwaǫ} ‘to be startled’, and Ę stem \stem{ęneˀwá:ʰdęˀ} ‘to startle someone’.\footnote{For the stem \stem{ęneˀwá:ʰdęˀ}, the second-last vowel is typically always long and followed by a short \phonet{h} sound.}

\ea\label{ex:ppvariation13} 
\ea novel 
\gll \stem{ǫgy-ęne̱ˀwáǫ}\\
\exsc{1d.p}-startled.{\stative}\\
\cfex{expected \stem{ǫkn-ęne̱ˀwáǫ}}
\ex novel 
\gll \stem{ę-j-ęne̱ˀwá:ʰdęˀ}\\
\fut-\exsc{1d.p}-startle.{\punctual}\\
\cfex{expected \stem{ę-hsn-ęne̱ˀwá:ʰdęˀ}}
\ex novel 
\gll \stem{ho-ne̱ˀwaǫ} \\
\textsc{3s.m.p}-startled.{\stative}\\
\cfex{expected \stem{haw-ęne̱ˀwáǫ}}
\ex novel 
\gll \stem{ę-ho-ne̱ˀwá:ʰdęˀ}\\
\fut-\textsc{3s.m.p}-startle.{\punctual}\\
\cfex{expected \stem{ę-haw-ęne̱ˀwá:ʰdę}}
\ex novel 
\gll \stem{go-néˀwaǫ} \\
{3fi.\sgo}-startled.{\stative}\\
\cfex{expected \stem{gaw-ęne̱ˀwáǫ}}
\ex novel 
\gll \stem{ę-yago-ne̱ˀwá:ʰdęˀ} \\
\fut-\textsc{3s.fi.p}-startle.{\punctual}\\
\cfex{ expected \stem{ę-yagaw-ęne̱ˀwá:ʰdęˀ}}
\ex novel 
\gll \stem{o-néˀwaǫ} \\
\textsc{3s.p}-startled.{\stative}\\
\cfex{expected \stem{aw-ęne̱ˀwáǫ}}
\ex novel 
\gll \stem{ę-yo-ne̱ˀwá:ʰdęˀ} \\
\fut-\textsc{3s.p}-startle.{\punctual}\\
\cfex{expected \stem{ę-yaw-ęne̱ˀwá:ʰdęˀ}}
\z
\z
