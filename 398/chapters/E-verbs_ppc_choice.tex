\chapter{\textsc{e-verbs} and pronominal prefix choice} \label{ch:E-verbs and pronominal prefix choice}
\textsc{e-verbs} are verbs with two stems, one of which is \stem{e} ‘go’ (see \sectref{ch:E-verbs}). Pronominal prefix choice for e-verbs is summarized in \tabref{figtab:1:ppseleverb} 

\begin{table}
\caption{Pronominal prefix selection, \textsc{e-verbs}}
\label{figtab:1:ppseleverb}
{
\begin{tabularx}{\textwidth}{XX}
\lsptoprule
\textsc{e-verb} type & prefix choice\\
\midrule
\textsc{simple motion verb} \stem{e} ‘go’

& \stem{\textsc{a}-V} only (also, no \textsc{stative} form)\\
\textsc{complex motion verbs} \stem{V-e} & \stem{\textsc{a}-V}

\stem{\textsc{p}-V}

\\
\textsc{dislocative e-verbs}\newline  \stem{V-dislocative-e} & \stem{\textsc{a}-V}

\stem{\textsc{p}-V}

\\
\textsc{progressive verbs} \stem{V-gy-eˀ} & \stem{\textsc{a}-V}

\stem{\textsc{p}-V}

\\
\textsc{stative progressive verbs} \newline
{} \stem{V-ǫ-h-ǫ{}-gy-eˀ} & \stem{\textsc{p}-V} only (like the stative verbs they are based on)\\
\lspbottomrule
\end{tabularx}}
\end{table}


\section{Simple and complex motion verb \stem{e} ‘go’ prefix choice} \label{ch:Simple and complex motion verb prefix choice}
The \textsc{simple motion verb} \stem{e} ‘go’ (\sectref{ch:Simple motion verb [e] ‘go’}) always takes \textsc{a}-series prefixes \xref{ex:motionpronchoice1}.

\ea\label{ex:motionpronchoice1} \textsc{simple motion verb} \stem{e} ‘go’
\ea itgeˀs\\
\gll i-t-g-e-ˀs\\
 {\prothetic}-{\cislocative}-\exsc{1s.a}-go-{\habitual}\\
\glt `I am here'
\ex haˀgeˀ\\
\gll haˀ-g-e-ˀ\\
 {\translocativefactual}\exsc{-1s.a}-go-{\punctual}\\
\glt `I am going there'
\ex haˀge:ˀ\\
\gll haˀ-g-e-:-ˀ\\
 {\translocativefactual}-\exsc{1s.a}-go-{\purposive}-{\punctual}\\
\glt `I went there'
\ex í:geˀ\\
\gll í:-g-e-ˀ\\
 {\prothetic}-\exsc{1s.a}-go-{\stative}\\
\glt `I am walking, moving'
\ex to há:ge:\\
\gll to h-á:-g-e-:\\
there {\translocative}-{\indefinite}-\exsc{1s.a}-go-{\purposive}.{\noaspect}\\
\glt `where I might go'
\z
\z

\textsc{complex motion verbs} (\sectref{ch:Complex motion verbs}) take either \textsc{a}- or \textsc{p}-series prefixes. The choice must be memorized for each verb \xref{ex:motionpronchoice2}.

\ea\label{ex:motionpronchoice2} \textsc{complex motion verbs}
\ea ge̱hsreˀ\\
\gll g-e̱-hsr-e-ˀ\\
 \exsc{1s.a}-{\joinerE}-chase-go-{\stative}\\
\glt `I am chasing it'
\ex agéˀdreˀ\\
\gll ag-é-ˀdr-e-ˀ\\
 \exsc{1s.p}-{\joinerE}-drag-go-{\stative}\\
\glt `I am riding along'
\z
\z


\section{Dislocative e-verb prefix choice} \label{ch:Dislocative E-verb prefix choice}
\textsc{dislocative e-verbs} (\sectref{ch:Dislocative-[e] verbs}) take the same kind of pronominal prefix as the regular verbs on which they are based: \xref{ex:motionpronchoice31} illustrates a dislocative e-verb and the corresponding regular verb, both taking \textsc{a}-series prefixes in the \textsc{habitual}. Example \xref{ex:motionpronchoice38} illustrates a \textsc{dislocative e-verb} and a corresponding regular verb, both taking \textsc{p}-series prefixes in the habitual.

\ea\label{ex:motionpronchoice31} dislocative e-verbs and regular verbs, \textsc{a}-series\\
dehęnatkwáhneˀ\\
\gll de-hęn-at-kw-á-hn-e-ˀ\\
 {\dualic}-\exsc{3ns.m.a}-{\semireflexive}-dance-{\joinerA}-{\dislocative}-\exsc{go}-{\stative}\\
\glt `they are going to dance'
\cfex{de̱hę́natkwaˀ\\
\gll de̱-hę́n-at-kw-haˀ\\
{\dualic}-\exsc{3ns.m.a}-{\semireflexive}-dance-{\habitual}\\
\glt ‘they dance’ (all the time) }
\z


\ea\label{ex:motionpronchoice38} dislocative e-verbs and regular verbs, \textsc{p}-series\\
de̱hohedáhneˀ\\
\gll de̱-ho-hed-á-hn-e-ˀ\\
 {\dualic}-\textsc{3s.m.p}-yell-{\joinerA}-{\dislocative}-go-{\stative}\\
\glt `he is going along hollering'
\cfex{de̱hóhetaˀ\\
	\gll de̱-hó-het-haˀ\\
	{\dualic}-\textsc{3s.m.p}-yell-{\habitual}\\
	\glt `he is hollering'}
\z



\section{Progressive verb prefix choice} \label{ch:Progressive verb prefix choice}
\textsc{progressive} verbs take the same type of prefix as the non-progressive verb on which they are based. Recall that there are two types of progressive verb (\sectref{ch:Progressive verbs}). 

The first type of progressive is formed by adding \stem{-gy-eˀ} to a verb stem. These verbs take the same types of pronominal prefixes as the non-progressive verbs on which they are based \xxref{ex:motionpronchoice4}{ex:motionpronchoice42}

\ea\label{ex:motionpronchoice4} \stem{\textsc{int}-V-agy-eˀ} progressive verb, based on \stem{\textsc{int}-V}\\
hǫwahnǫdrá:gyeˀ\\
\gll hǫwa-hnǫdr-á:-gy-e-ˀ\\
 \exsc{3ms/3fis:3ms}-follow-{\joinerA}-{\progressive}-go-{\stative} \\
\glt `someone is following him along'
\cfex{hǫwáhnǫdreˀ\\
	\gll hǫwá-hnǫdr-e-ˀ\\
	\exsc{3ms/3fis:3ms}-follow-go-{\stative}\\
	\glt `someone is following him'}
\z

\newpage
\ea\label{ex:motionpronchoice41} \stem{\textsc{a}-V-agy-eˀ} progressive verb, based on \stem{\textsc{a}-V}\\
gayę́twagyeˀ\\
\gll ga-yę́tw-a-gy-e-ˀ\\
 \textsc{3s.a}-plant-{\joinerA}-{\progressive}-go-{\stative}\\
\glt `it is planted along'
\cfex{gayę́:twęh\\
	\gll ga-yę́:tw-ęh\\
	\textsc{3s.a}-plant-{\stative}\footnotemark{}\\
	\glt `it is planted'}
\footnotetext{This is a special type of stative verb requiring an \textsc{a}-series prefix. See \sectref{ch:Stative nouns and agentive stative nouns}.}
\z



\ea\label{ex:motionpronchoice42} \stem{\textsc{p}-V-agy-eˀ} progressive verb, based on \stem{\textsc{p}-V}\\
ohsáwagyeˀ\\
\gll o-hsáw-a-gy-e-ˀ\\
 \textsc{3s.p}-begin-{\joinerA}-{\progressive}-go-{\stative}\\
\glt `it is beginning'
\cfex{tohsa:ˀ\\
	\gll t-ho-hsa:-ˀ\\
	{\cislocative}-\textsc{3s.m.p}-begin-{\stative}\\
	\glt `he has begun'}
\z


The second type of progressive verb is formed by adding \stem{-ǫgy-eˀ} or \stem{-ęgy-eˀ} to a \textsc{stative} verb. These \textsc{stative progressives} always take \textsc{p}-series prefixes, because they are based on the stative forms of \textsc{three-aspect verbs}, which take \textsc{p}-series prefixes in the stative \xref{ex:motionpronchoice5}.

\ea\label{ex:motionpronchoice5} \stem{\textsc{p}-V-ǫ-h-ǫgy-eˀ} or \stem{\textsc{p}-V-ę-h-ęgy-eˀ} stative progressive verbs
\ea sa̱hohdęhgyǫ́hǫgyeˀ\\
\gll sa̱-ho-hdęhgy-ǫ́-h-ǫ-gy-e-ˀ\\
 {\repetitive}-\textsc{3s.m.p}-leave-{\stative}-\textsc{euph.h}-{\joiner}-{\progressive}-go-{\stative}\\
\glt `he is on his way home'
\cfex{hesáhdęgyǫ:\\
	\gll he-sá-hdęgy-ǫ:\\
	{\translocative}-\textsc{2s.p}-leave-{\stative}\\
	\glt `you went over there'}
\ex otsihsˀǫ̱hǫ́:gyeˀ\\
\gll o-tsihsˀ-ǫ̱-h-ǫ́:-gy-e-ˀ\\
 \textsc{3s.p}-mature-{\stative}-\textsc{euph.h}-{\joiner}-{\progressive}-go-{\stative}\\
\glt `it is getting mature'
\cfex{otsíhsˀǫh \\
	\gll o-tsíhsˀ-ǫh\\
	\textsc{3s.p}-mature-{\stative}\\
	\glt ‘it is done for the season’, ‘it has gone full cycle’, ‘it is mature’, `they (plants) have finished out'}
\z
\z

