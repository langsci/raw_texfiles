\chapter{Clauses with conjunctions} \label{ch:Clauses with conjunctions}
\textsc{Conjunctions} are linking words that connect list items. Words like \textit{hniˀ} ‘and’ connect similar words or phrases in a list. \textit{Hne:ˀ} ‘in fact’, ‘but’ connects contrasting or dissimilar items. Finally, conjunctions like \textit{nigę́ˀǫh} ‘or’ connect alternatives or choices of item(s). Phrases with these linking words are described in the following sections.

\section{\textit{Hniˀ} ‘and’ conjunction} \label{ch:[hniˀ] ‘and’ conjunction}
The conjunction \textit{hniˀ} ‘and’ links similar items in a list. For example, two conjoined independent clauses (in square brackets) are linked by \textit{hniˀ} in \xref{ex:conjex3}. \textit{Hniˀ} tends to occur at the end of the list of items. 

\ea\label{ex:conjex3}
\gll [Sanaháowe:k], [sagyaˀdawíˀt] hniˀ.\\
[put.your.hat.on], [put.your.coat.on] and\\
\glt ‘Put your hat and coat on!’ (\cite[212]{mithun_watewayestanih_1984}, Satrǫ́:nih dialogue)
\z

\textit{Hniˀ} is optional in lists: for example, the conjoined word-groups in \xref{ex:conjex4} just occur side-by-side. (The conjoined items in this example function as “nouns”).

\ea\label{ex:conjex4} 
\gll A:yę́:ˀ to-ˀǫ-tsǫ: ni:yǫ́: gwahs dewagadǫhwęjó:nih, [nę:gyę́ neˀ gajihwáˀ] [né:ˀ neˀ enesdanyáˀktaˀ].\\
it.seems that-maybe-just a.certain.kind really I.want.it this.one the hammer it.is the saw. \\
\glt ‘That seems to be all I want, really, this hammer (and) the saw.’ (\cite[159]{mithun_watewayestanih_1984}, Enǫhsǫnyaˀdaˀsǫ:ˀǫh dialogue)
\z

More examples are provided in the “Related” sections.

\begin{CayugaRelated}
\item Sentences with juxtaposed clauses, \ref{Sentences with side-by-side clauses}

\item Clauses with conjunctions, \ref{ch:Clauses with conjunctions}

\item Independent and dependent clauses, relative clauses, \ref{Independent and dependent clauses, relative clauses}

\item{} \textit{Hniˀ} \trs{and}, \ref{p:[hniˀ] ‘and’}

\item{} \textit{Ne:ˀ giˀ hniˀ} \trs{and that too}, \trs{also}, \ref{p:[ne:ˀ giˀ hniˀ]}

\end{CayugaRelated}


\section{\textit{Hne:ˀ, Ne:ˀ} … (\textit{tsǫ: shęh}) ‘but’} \label{ch:[hne:ˀ], [ne:ˀ … (tsǫ: shęh)] ‘but’}
\textit{Hne:ˀ} ‘in fact’, ‘but’ and \textit{ne:ˀ … (tsǫ: shęh)} ‘it.is (just that)’ link contrasting or dissimilar ideas.\footnote{Strictly speaking, the words described here are contrastive focus markers (see “Related”), described here because they can be translated as ‘but’.} \textit{Hne:ˀ} tends to occur at the end of the list of contrasting ideas (shown in square brackets in \ref{ex:conjex8}. The first instance of \textit{hne:ˀ} in \ref{ex:conjex8} is described in \sectref{p:[hne:ˀ] ‘in fact’}.)

\ea\label{ex:conjex8}
\gll [Tęˀ hne:ˀ de̱hahǫkáˀ ǫgwehǫ́:weh], [hoˀnigǫ̱haędaˀs] hné:ˀ.\\
not in.fact he.doesn’t.speak Indian he.understands in.fact\\
\glt ‘He doesn’t speak Indian, but he does understand.’ (\cite[61]{mithun_watewayestanih_1984}, Sǫ: hne:ˀ nˀaht to:gyęh? dialogue)
\z

\textit{Ne:ˀ} (plus other optional particles) appears at the beginning of a contrasting idea or \textsc{clause} \xref{ex:conjex13}. 

\ea\label{ex:conjex13} 
\gll [Swahyo:wá:ˀ age:gáˀs], [né:ˀ giˀ heyohé: age:gáˀs neˀ jihsó:dahk]\\
apples I.like it.is just more I.like the strawberries\\
\glt ‘I like apples, (but) I just like strawberries more.’ (\cite[45]{michelson_ontario_2011})
\z

More information about such constructions can be found in the following sections.

\begin{CayugaRelated}
\item Sentences with side-by-side clauses, \ref{Sentences with side-by-side clauses}

\item Clauses with conjunctions, \ref{ch:Clauses with conjunctions}

\item Independent and dependent clauses, relative clauses, \ref{Independent and dependent clauses, relative clauses}

\item{} \textit{Deˀę: gwaˀ-heh tęˀ seˀ} \trs{but then not really}, \ref{p:[deˀę: gwaˀ-heh tęˀ seˀ]}

\item{} \textit{Hne:ˀ} \trs{but}, \ref{p:[hne:ˀ] ‘but’}

\item{} \textit{Ne:ˀ tsǫ: shęh} \trs{but}, \trs{it's just that…}, \ref{p:[ne:ˀ tsǫ: shęh]}


\end{CayugaRelated}


\section{\textit{Nigę́ˀǫh} ‘or?’, \textit{Giˀ shęh} ‘or’, etc.}\label{ch:[nigę́ˀǫh] ‘or?’, [giˀ shęh] ‘or’, etc}
A range of words and word-groups are used for lists of alternative items. Phrases with \textit{nigę́ˀǫh} ‘or…?’ tend to be used in questions \xref{ex:alternative1}, in which case, \textit{nigę́ˀǫh} is always preceded by \textit{gęh} ‘Q(uestion)’ earlier in the sentence. In contrast, phrases with \textit{giˀ shęh} ‘or’ tend appear in statements \xref{ex:alternative2}. In both cases, \textit{nigę́ˀǫh} and \textit{giˀ shęh} tend to occur after the last choice. (In \ref{ex:alternative1}-\ref{ex:alternative2}, the items of choice are in square brackets.) Examples of similar alternative choice questions and statements are provided in the “Related” sections.

\ea\label{ex:alternative1}
\gll [Hǫ:gwéh] gęh [agǫ:gwéh] nigę́ˀǫh? \\
Man Q woman or\\
\glt ‘Was it a man or a woman?’
\z

\ea\label{ex:alternative2}
\gll Tęˀ sga̱hoˀdę́ˀ neˀ [oˀwáhǫh], [ji̱ˀdę:ˀę́h], [ojǫˀdáˀ] gi̱ˀ-shęh.\\
Not anything the meat, birds, fish or \\
\glt ‘No meat, no birds, no fish.’ (\cite{carrier_legends_2013})
\z

\begin{CayugaRelated}
\item Clauses with conjunctions, \ref{ch:Clauses with conjunctions}

\item Independent and dependent clauses, relative clauses, \ref{Independent and dependent clauses, relative clauses}

\item{} \textit{Gęh ga:t giˀ shęh nigęˀǫh} \trs{or?}, \ref{p:[gęh ga:t giˀ shęh nigęˀǫh]}

\item{} \textit{Gęh tęˀ nigęˀǫh} \trs{or not?}, \ref{p:[gęh tęˀ nigęˀǫh]}

\item{} \textit{Gyę:gwaˀ giˀ shęh} \trs{or maybe}, \ref{p:[gyę:gwaˀ giˀ shęh]}

\item{} \textit{Ne:ˀ giˀ shęh neˀ … giˀ shęh} \trs{or}, \ref{p:[ne:ˀ giˀ shęh neˀ … giˀ shęh]}

\item{} \textit{Ne:ˀ/Neˀ nigęˀǫh neˀ …} \trs{or?}, \ref{p:[ne:ˀ/neˀ nigęˀǫh neˀ …]}


\end{CayugaRelated}

