\setpartpreamble{\bigskip\bigskip\noindent This part of the grammar describes sentence formation. The first chapter, \hyperlink{ch:Simple Sentences}{\textit{Simple Sentences}}, describes sentences consisting of a single clause, including commands, various types of questions, and sentences joined with the linking verbs \textit{né:ˀ} ‘it is’ or \textit{de̱ˀgę:} ‘it is not’. (See \sectref{Utterances, clauses, phrases, and sentences} for definitions of clauses, etc.) The second chapter on \hyperlink{Sentences with juxtaposed clauses}{\textit{Sentences with side-by-side clauses}} describes sentences made up of clauses that are joined without linking words (particles). The chapter on \hyperlink{ch:Clauses with linking words}{\textit{Clauses with linking words}} describes clauses beginning with particles or particle groups such as \textit{sǫ: nˀaht} ‘who’ or \textit{dęˀ hoˀdęˀ} ‘what’. The chapter on \hyperlink{ch:Clauses with conjunctions}{\textit{Clauses with conjunctions}} describes clauses (and smaller phrases) connected by words such as \textit{hniˀ} ‘and’, \textit{hne:ˀ} ‘but, in fact’, and the like. The last chapter in this part of the grammar describes phrases used for \hyperlink{ch:Comparisons, counting, measuring}{\textit{Comparisons, counting, and measuring}}.}


\part{Sentences} \label{Sentences}
