\chapter{Pronominal prefix meaning} \label{Pronominal prefix meaning}
Pronominal prefixes express the concepts of \textsc{person}, \textsc{number}, \textsc{gender}, \textsc{clusivity}, and \textsc{role}, which are defined in the following sections. The remaining sections of this chapter describe the three types of pronominal prefix (the \textsc{non-interactive} \textsc{a}-, and \textsc{p}-series, and the \textsc{interactive} series). The chapter ends with a description of the exceptional nature of the \textsc{3s} ‘it’ prefixes.


\section{Person} \label{Person}
\textsc{Person} is a general term for point of view, including the \textsc{1\textsuperscript{st} person} or speaker(s), the \textsc{2\textsuperscript{nd} person} or audience (listener or listeners), and the \textsc{3\textsuperscript{rd} person} (referring to whatever or whomever the speaker and audience are discussing). 

The main difference between the words in \xref{ex:perseg100} is the type of person expressed by the pronominal prefix. (A hyphen appears between the prefix and stem in the following examples.)

\ea\label{ex:perseg100} 
\ea k-hnę́:ye:s ‘I am tall’
\begin{itemize}
    \item  speaker(s) → 1\textsuperscript{st} \textsc{person}  (said by whomever is speaking)
\end{itemize}
\ex s-hnę́:ye:s ‘you are tall’
\begin{itemize}
    \item  audience / listener(s) → 2\textsuperscript{nd} \textsc{person} (referring to a single-person audience)
\end{itemize}
\ex ha-hnę́:ye:s ‘he is tall’
\begin{itemize}
    \item  everyone and everything else → 3\textsuperscript{rd} \textsc{person} (referring to someone the speaker is talking \emph{about}, not \emph{to})
\end{itemize}
\z
\z


\section{Number} \label{Number}
\textsc{Number} refers to the number of people or objects. For example, while the verbs in \xref{ex:numex100} all refer to the audience (\textsc{2\textsuperscript{nd}} person), they refer to different numbers of audience members -- \textsc{singular} (one person), \textsc{dual} (two people) or \textsc{plural} (more than two people). 

\ea\label{ex:numex100} \textsc{number}\\
\ea s-hnę́:ye:s ‘you are tall’ \textsc{singular}
\ex sni̱-hnę́:ye:s ‘you two are tall’ \textsc{dual}
\ex swa̱-hnę́:ye:s ‘you all are tall  \textsc{plural}
\z
\z

For \textsc{3\textsuperscript{rd}} person pronominals, the relevant distinction is between \textsc{singular} and \textsc{non-singular} -- as opposed to the three-way singular/dual/plural distinction for \textsc{1\textsuperscript{st}} and \textsc{2\textsuperscript{nd}} persons. (However, see \sectref{3rd person pronominal prefix concepts} for discussion of dual \textsc{3\textsuperscript{rd}} person forms.)

\ea\label{ex:numex101} \textsc{number}\\
\ea e-hnę́:ye:s ‘she is tall’ \textsc{singular}
\ex gae-hnę́:ye:s ‘you two are tall’ \textsc{non-singular}
\z
\z

\section{Gender} \label{Gender}
Gayogo̱ho:nǫ̱ˀnéha:ˀ \textsc{gender} categories include the \textsc{masculine} (‘he’, ‘him’), \textsc{femin\-ine-indefinite} (‘she’, ‘her’, ‘someone’), and \textsc{zoic-neuter} (‘it’ -- an object or animal). For example, while both of the words in  \xref{ex:gendex100} are \textsc{3\textsuperscript{rd}} person singular, they express different genders.

\ea\label{ex:gendex100} \textsc{gender}\\
\ea ha-hnę́:ye:s ‘he is tall’ \textsc{masculine}
\ex e-hnę́:ye:s ‘she is tall’ \textsc{feminine-indefinite}
\z
\z

Grammatical gender does not necessarily match up with biological gender. For example, the Gayogo̱ho:nǫˀnéha:ˀ prefix in \xref{ex:gendex200} can refer either to a female or to a person whose gender is not defined (in which case it means ‘someone’, ‘anyone’, or ‘people’).

\ea\label{ex:gendex200} e-hnę́:ye:s ‘she is tall, ‘someone is tall’\\
\begin{itemize}
    \item 3\textsuperscript{rd} singular feminine-indefinite reference\\
\end{itemize}
\z

Similarly, the prefix in \xref{ex:gendex300} can  refer either to a group of females or to a mixed group of males and females.

\ea\label{ex:gendex300} gae-hnę́:ye:s ‘they are tall’\\
\begin{itemize}
    \item 3\textsuperscript{rd} plural feminine-indefinite reference\\
\end{itemize}
\z

In fact, the feminine-indefinite prefixes may even refer to an all-male group: in  \xref{ex:gendex400}, \textit{ena:greˀ} refers to ‘the men’, although the word literally means ‘she or someone lives there’.

\ea\label{ex:gendex400} 
\gll O:nę́h hniˀ ohsǫdagwe:gǫ́h de̱hodinęhę́:ˀ neˀ hęnǫ:gwéh [shęh e-ná:greˀ]. \\
now and all.night they.guard the men that someone.lives.there\\
\glt ‘Now too all night they are guarding, the men [who live there].’ (Keye, 2012)
\z

The \textsc{zoic-neuter} prefix also demonstrates the difference between grammatical and biological gender. It can refer to an animal, an object, or the weather  (\ref{ex:gendex500}a--c), and to females (in a derogatory sense, \ref{ex:gendex500d}). 

\ea\label{ex:gendex500} \textsc{zoic-neuter} reference
\ea\label{ex:gendex500a} o-hsę: ‘it (animal) is fat’ (referring to an animal)
\ex\label{ex:gendex500b} o-hóˀdę:ˀ ‘it (i.e. dirt) is soft, loose’ (referring to an object)
\ex\label{ex:gendex500c} o-tó:weˀ ‘it (weather) is cold’ (referring to the weather)
\ex\label{ex:gendex500d} wa-datgehǫ́haˀ ‘she is a prostitute’ (literally, ‘it sells its private parts’)
\z
\z


\section{Clusivity (inclusive and exclusive)} \label{Clusivity}
\textsc{Clusivity} is a term relevant for ‘we’ forms. \textsc{Inclusive} ‘we’ forms group the speaker(s) and listener(s) together \xref{ex:clusex100a}, whereas \textsc{exclusive} ‘we’ forms make it clear that the speaker(s) and listener(s) are not the same group \xref{ex:clusex100b}. 

\ea\label{ex:clusex100} \textsc{clusivity}
\ea\label{ex:clusex100a} kni̱-hnę́:ye:s ‘we two are tall’ \textsc{inclusive} (includes an audience member)
\ex\label{ex:clusex100b} akni-hnę́:ye:s ‘we two are tall’ \textsc{exclusive} (excludes the audience member)
\z
\z


\section{Combining person, number, gender, and clusivity} \label{Combining person, number, gender, and clusivity}
Person, number, gender, and clusivity are combined in and expressed through two types of pronominal prefix, \textsc{interactive} and \textsc{non-interactive} ones (defined in \sectref{Three types of pronominal prefix}). The meanings of the non-interactive prefixes are described next. For the meanings conveyed by the interactive pronominal prefixes, see \sectref{a and o (non-interactive) prefixes}.

\subsection{\textsc{1\textsuperscript{st}} person pronominal prefix concepts} \label{1st person pronominal prefix concepts}
The \textsc{1\textsuperscript{st}} person pronominal prefixes express person and number, but not gender \xref{ex:1stex100}. Also, clusivity only applies to ‘we’ (\textsc{1\textsuperscript{st}} person plural) forms. (Singular \textsc{1\textsuperscript{st}} person or ‘I’ forms cannot express clusivity, since clusivity presupposes more than one person). 

\ea\label{ex:1stex100}
k-hnę́:ye:s ‘I am tall’\label{ex:1stex100a}
\begin{itemize}
    \item  person (\textsc{1\textsuperscript{st}}), gender, clusivity (absent), number (singular)
\end{itemize}
\z


Example \xref{ex:1stex200} illustrates a verb that takes the full set of inclusive and exclusive ‘we’ forms. Only verbs with \textsc{a}-series pronominal prefixes express the inclusive versus exclusive distinction. (The \textsc{a}-series is a type of non-interactive pronominal prefix, as is the \textsc{p}-series, see \sectref{a and o (non-interactive) prefixes}).

\ea\label{ex:1stex200}
\ea kni̱-hnę́:ye:s, tni̱-hnę́:ye:s ‘you (one person) and I are tall’
\begin{itemize}
    \item person (\textsc{1\textsuperscript{st}}), gender (absent), clusivity (inclusive), number (dual)
\end{itemize} 
\ex akni-hnę́:ye:s ‘we two (not you) are tall’ 
\begin{itemize}
    \item person (\textsc{1\textsuperscript{st}}), gender (absent), clusivity (exclusive), number (dual)
\end{itemize}
\ex dwa-hnę́:ye:s ‘you all and I are tall’, ‘we all (and you) are tall’
\begin{itemize}
    \item person (\textsc{1\textsuperscript{st}}), gender (absent), clusivity (inclusive), number (plural)
\end{itemize}
\ex agwa-hnę́:ye:s ‘we all (not you) are tall’
\begin{itemize}
    \item  person (\textsc{1\textsuperscript{st}}), gender (absent), clusivity (exclusive), number (plural)
\end{itemize}
\z
\z

Example \xref{ex:1stex300} illustrates a different type of verb, one that does not distinguish between inclusive and exclusive ‘we’ forms. Verbs with \textsc{p}-series pronominal prefixes do not express the inclusive versus exclusive distinction (see \sectref{a and o (non-interactive) prefixes}).

\ea\label{ex:1stex300}
\ea ǫkni-nǫ̱hǫ́kdanih ‘we two are sick’
\begin{itemize}
    \item person (\textsc{1\textsuperscript{st}}), gender and clusivity (absent), number (dual)
\end{itemize}
\ex ǫgwa-nǫ̱hǫ́kdanih ‘we all are sick’
\begin{itemize}
    \item person (\textsc{1\textsuperscript{st}}), gender and clusivity (absent), number (plural)
\end{itemize} 
\z
\z


\subsection{\textsc{2\textsuperscript{nd}} person pronominal prefix concepts} \label{2nd person pronominal prefix concepts}
The \textsc{2\textsuperscript{nd}} person pronominal prefixes express person and number (singular, dual, or plural), but not gender or clusivity \xref{ex:2ndex100}.

\ea\label{ex:2ndex100}
\ea s-hnę́:ye:s ‘you (one person) are tall’
\begin{itemize}
    \item person (\textsc{2\textsuperscript{nd}}), gender and clusivity (absent), number (singular)
\end{itemize} 
\ex sni̱-hnę́:ye:s ‘you two are tall’
\begin{itemize}
    \item person (\textsc{2\textsuperscript{nd}}), gender and clusivity (absent), number (dual)
\end{itemize} 
\ex swa̱-hnę́:ye:s ‘you all are tall’
\begin{itemize}
    \item person (\textsc{2\textsuperscript{nd}}), gender and clusivity (absent), number (plural)
\end{itemize}
\z
\z

\subsection{\textsc{3\textsuperscript{rd}} person pronominal prefix concepts} \label{3rd person pronominal prefix concepts}
The \textsc{3\textsuperscript{rd}} person pronominal prefixes express person and number (singular or non-singular). They are the only non-interactive pronominal prefixes to express gender \xref{ex:3rdex100}.

\ea\label{ex:3rdex100}
\ea ha-hnę́:ye:s ‘he is tall’
\begin{itemize}
    \item person (\textsc{3\textsuperscript{rd}}), gender (masculine), number (singular), clusivity (absent)
\end{itemize}
\ex e-hnę́:ye:s ‘she or someone is tall’
\begin{itemize}
    \item person (\textsc{3\textsuperscript{rd}}), gender (feminine-indefinite), number (singular), clusivity (absent)
\end{itemize}
\ex ga̱-hnę́:ye:s ‘it (animal) is tall’
\begin{itemize}
    \item person (\textsc{3\textsuperscript{rd}}), gender (zoic-neuter), number (singular), clusivity (absent)
\end{itemize}
\ex hadi-hnę́:ye:s ‘they (males only) are tall’
\begin{itemize}
    \item person (\textsc{3\textsuperscript{rd}}), gender (masculine), number (non-singular), clusivity (absent)
\end{itemize}
\ex gae-hnę́:ye:s ‘they (females or mixed group) are tall’ 
\begin{itemize}
    \item person (\textsc{3\textsuperscript{rd}}), gender (feminine-indefinite), number (non-singuilar), clusivity (absent)
\end{itemize}
\ex gadi-hnę́:ye:s ‘they (animals) are tall’
\begin{itemize}
    \item person (\textsc{3\textsuperscript{rd}}), gender (zoic-neuter), number (non-singular), clusivity (absent)
\end{itemize}
\z
\z

Other Ǫgwehǫwéhne̱ha:ˀ languages also  have \textsc{3\textsuperscript{rd}} person dual forms, but these are rare in Gayogo̱ho:nǫˀnéha:ˀ. The renowned orator Huron Miller used \stem{hni-}, a \textsc{3\textsuperscript{rd} masculine dual} prefix, in his recitation of \textit{Gaihwi:yo:} \xref{ex:3rdex200}.\footnote{Tom Deer, p.c. Tom also notes that Huron Miller spoke Cayuga, Onondaga, Mohawk, and Seneca.}

\ea\label{ex:3rdex200}
\ea detniyanó:we:ˀ \\
\gll det-hni-yanó:we:ˀ\\ 
{\dualiccislocative}\exsc{-3du.m.a}-fast.{\stative}\\
\glt ‘two fast male runners’
\ex haˀtní:daˀ \\
\gll haˀt-hní:-daˀ \\
{\translocativedualic}\exsc{-3du.m.a}-stop.{\punctual}\\
\glt ‘they (two males) stopped there’
\ex haˀtniyadowe̱hdáhnǫh \\
\gll haˀt-hni-yadowe̱hdáhn-ǫh \\
{\translocativedualic}\exsc{-3du.m.a}-think.over-{\stative}\\
\glt ‘they (two males) thought it over’
\ex hǫsahnihóˀkdęˀ \\
\gll hǫsa-hni-hóˀkd-ęˀ\\
{\translocative.repetitivefactual}\exsc{-3du.m.a}-end-{\punctual}\\
\glt ‘they (two males) ended’, ‘they told everything’
\z
\z

\section{Pronominal prefixes and role} \label{Pronominal prefixes and role}
In addition to person, number, gender, and clusivity, the pronominal prefixes also interact with the concept of \textsc{role}. Role refers to the parts played by the people or objects necessary to complete the meaning of a verb. To illustrate, for the verb \textit{ahágǫ̱he:k}, the male’s role (‘he’) is that of an \exsc{agent} or “doer”, performing the action. In contrast, for the verb \textit{ahógǫ̱he:k}, the male’s role (‘him’) is reversed, and the person in question is (unfortunately) undergoing the action, having the role of \textsc{patient} or {\undergoer} \xref{ex:rolex100}.

\ea\label{ex:rolex100}
\ea a-há-gǫ̱he:k ‘he punched it’
\ex a-hó-gǫ̱he:k ‘it punched him’
\z
\z

Verbs can express several different types of role, depending on the meaning of the verb: for example, activity verbs such as \textit{ę́:ge:k} ‘I will eat it’ or \textit{segǫ́he:s} ‘you hit it (all the time)’ express both a “doer” and an undergoer. In contrast,  verbs of perception such as \textit{aˀé:gęˀ} ‘she saw it’, or \textit{ęsa:tǫ́:dęh} ‘you will hear it’ require a \textsc{perceiver} and a \exsc{percept}. As well, stative-only verbs such as \textit{godinǫ̱hǫ́kdanih} ‘they are sick’ require an \exsc{attributee} -- something or someone displaying the quality or attribute in question. These are only some of the types of role.

For present purposes, we will focus on the \emph{number} of roles required by a verb, rather than the \emph{types} of role. Verbs can express one role \xref{ex:rolex200} or two \xref{ex:rolex300}. Several verbs even require three roles \xref{ex:rolex400}.

\ea\label{ex:rolex200} ha-hnę́:ye:s ‘\textbf{he} is tall’\\
\glt  (one-role verb, \textsc{3\textsuperscript{rd}} person)
\z

\ea\label{ex:rolex300} a-gǫ́:-gęˀ ‘\textbf{I} saw \textbf{you} (singular)’\\
\glt  (two-role verb, \textsc{1\textsuperscript{st}} and \textsc{2\textsuperscript{nd}} person roles)
\z

\ea\label{ex:rolex400} shǫgwa-ęnáwi̱hǫˀ ‘\textbf{he} has given \textbf{us} \textbf{a song}’\\
\glt  (three-role verb, 3\textsuperscript{rd} masculine, 1\textsuperscript{st}, and 3\textsuperscript{rd} zoic-neuter roles, where ‘a song’ is the 3\textsuperscript{rd} zoic-neuter {\undergoer} or “object”)
\z

Verbs requiring one role take \textsc{non-interactive} pronominal prefixes, while verbs expressing two or three roles take \textsc{interactive} pronominal prefixes (as well as non-interactive ones, see \sectref{Three types of pronominal prefix}).


\subsection{Role and the reflexive and semireflexive prefixes} \label{Role and the reflexive and semireflexive prefixes}
The \textsc{reflexive} (or \textsc{reciprocal}) prefix (\sectref{[adad-] (reflexive or reciprocal)}) and the \textsc{semireflexive} prefix (\sectref{[ad-] (semireflexive) prefix}) both influence the number of roles expressed by the verb. 

\subsubsection*{The \textsc{reflexive} or \textsc{reciprocal prefix} and roles} \label{The reflexive or reciprocal prefix and roles}
The \stem{-adad} \textsc{\reflexive} or \textsc{\reciprocal} prefix  occurs in two-role verbs \xref{ex:reflex100}. Used in the \emph{reflexive} sense, it means that the two roles are performed by the same person \xref{ex:reflex100}. 

\ea\label{ex:reflex100} \stem{-adad} \textsc{\reflexive}\\
agadadé:gęˀ\\
\gll ag-adad-é:-gę-ˀ\\
\exsc{1s.p-{\reflexive}}-{\joinerE}-see-{\punctual}\\
\glt ‘I saw myself’
\z


When the reflexive occurs with a \textsc{feminine-indefinite} pronominal prefix (meaning ‘she’ or ‘they (females or mixed group)’), the “self” or reflexive meaning is not automatic. For example, \textit{aˀǫdadé:gęˀ} can mean either ‘she saw herself’ or ‘she saw her (someone else)’ \xref{ex:reflex200}.

\newpage
\ea\label{ex:reflex200} \stem{-adad} \textsc{\reflexive}, feminine-indefinite
\ea aˀǫdadé:gęˀ\\
\gll aˀ-ǫ-dad-é:-gę-ˀ\\
{\factual}-\textsc{3s.fi.a}-{\reflexive}-see-{\punctual}\\
\glt ‘she saw herself’, ‘she saw her (someone else)’

\ex agaǫda:gyé:naˀ\\
\gll a-gaǫ-da:g-yé:na-ˀ\\
{\factual}-\textsc{3ns.fi.a}-{\reflexive}-catch-{\punctual}\\
\glt ‘they (females or mixed) caught her, someone, them’
\z
\z

When used in a \emph{reciprocal} sense, the \stem{-adad} prefix means that the participants fulfilling both roles are equally affected by the action \xref{ex:reflex300}. Verbs with the reciprocal meaning also require the \stem{de-} \textsc{dualic} prefix.

\ea\label{ex:reflex300} \stem{de-…adad-} \textsc{\dualic…\reflexive} = \textsc{reciprocal}\\
dedwadagyénawahs\\
\gll de-dwa-dag-yénaw-a-hs\\
{\dualic}-\exsc{1p.in.a}-{\reflexive}-help-{\joinerA}-{\habitual}\\
\glt ‘let us all help one another’
\z


\subsubsection*{The \textsc{semireflexive} prefix and roles} \label{The semireflexive prefix and roles}
The \stem{-ad} \textsc{\semireflexive} appears in two-role verbs. In general, it means that the action reflects back on or affects the “doer” \xref{ex:groomex100}.

\ea\label{ex:groomex100} 
hatgǫhso̱háe\\
\gll h-at-gǫhs-o̱háe\\
\textsc{3s.m.a}-{\semireflexive}-face-wash.{\stative}\\
\glt ‘he is washing his face’ (with \textsc{\semireflexive})

\cfex{shagogǫ̱hsóhai\\
	\gll shago-gǫ̱hs-óhae\\
	\exsc{3s.m>3fi/3p}-face-wash.{\stative}\\
	\glt `he is washing her face' 
 (the same verb without the \textsc{\semireflexive}, with {\doer} ‘he’, {\undergoer} ‘her face’)}
\z


The \stem{-ad} \textsc{\semireflexive} can also suppress or downplay an undergoer role. As shown in \xref{ex:groomex200}, while the word without the semireflexive clearly expresses both the “doer” (‘you’) and the undergoer (‘her’ or ‘someone’, \ref{ex:groomex200a}), the same word with the \stem{-ad} \textsc{\semireflexive} does not focus on \emph{who} is being thanked, but instead emphasizes the activity, ‘giving thanks’ \xref{ex:groomex200b}. As a result, it expresses just one role (“doer”-‘we’ in the example) and so only uses non-interactive pronominal prefixes.

\ea\label{ex:groomex200}
\ea dęhshenǫ̱hǫ́:nyǫ:ˀ \\\label{ex:groomex200a}
\gll d-ę-hshe-nǫ̱hǫ́:nyǫ:-ˀ\\
{\dualic}-{\future}-\exsc{2s>3s.fi}-thank-\exsc{\punctual }\\
\glt ‘you will welcome, greet or thank someone/her’ (two-role verb, no {\semireflexive}, interactive pronominal prefix)
\ex dędwadęnǫ́hǫnyǫ:ˀ \\\label{ex:groomex200b}
\gll d-ę-dw-adę-nǫ́hǫnyǫ:-ˀ \\
{\dualic}-{\future}-\exsc{1p.in.a}-{\semireflexive}-thank-{\punctual}\\
\glt ‘we all will give thanks’ (two-role verb with {\semireflexive}, non-interactive pronominal prefix)
\z
\z

Several verbs with the \stem{ad-} {\semireflexive} prefix do not appear to display role suppression \xref{ex:groomex300a}. In such cases, the \stem{ad-} {\semireflexive} is part of a fixed expression, meaning that the same verb without it has a fairly different meaning \xref{ex:groomex300b}. That being said, the verbs requiring \stem{ad-} {\semireflexive} in this way do still express an activity that somehow reflects back on the main participants.

\ea\label{ex:groomex300}
\ea agadekǫ́:ni:\\\label{ex:groomex300a}
\gll ag-ad-e-k-ǫ́:ni-:\\
\exsc{1s.p-{\semireflexive}}-{\joinerE}-food-make-{\stative}\\
\glt ‘I am eating something’ (two-role verb with obligatory {\semireflexive} as part of a fixed expression)

\ex age:kǫ́:ni:\\\label{ex:groomex300b}
\gll ag-e:-k-ǫ́:ni-:\\
{1\sgo}-{\joinerE}-food-make-{\stative}\\
\glt ‘I am cooking something’ (related two-role verb without {\semireflexive})
\z
\z



\section{Three types of pronominal prefix} \label{Three types of pronominal prefix}
The three types of pronominal prefix are the \textsc{non-interactive} \textsc{a}-series, the \textsc{non-interactive} \textsc{p}-series, and the \textsc{interactive} series. These prefix types are described in the following sections.

\section{\textsc{a}- and \textsc{p}-series (non-interactive) prefixes} \label{a and o (non-interactive) prefixes}
The \textsc{a}- and \textsc{p}-series of pronominal prefix express only one role, such as the attributee (\ref{ex:haho100}a, b), the agent \xref{ex:haho100c}, or the experiencer \xref{ex:haho100d}. Because they only convey one role, \textsc{a}- and \textsc{p}-series prefixes are \exsc{non-interactive}.

\ea\label{ex:haho100}
\ea hahnę́:ye:s \\\label{ex:haho100a}
\gll ha-hnę́:ye:s\\
\textsc{3s.m.a}-tall.{\stative}\\
\glt ‘he is tall’ (\textsc{a}-series, \exsc{attributee})

\ex hohsę: \\\label{ex:haho100b}
\gll ho-hsę:\\
\textsc{3s.m.p}-fat.{\stative}\\
\glt ‘he is fat’ (\textsc{p}-series, \exsc{attributee})

\ex hadá:kseˀ \\\label{ex:haho100c}
\gll ha-dakse-ˀ \\
\textsc{3s.m.a}-run.go-{\stative}\\
\glt ‘he is running’ (\textsc{a}-series, \textsc{agent} or “doer”)

\ex ęhó:daˀ \\\label{ex:haho100d}
\gll ę-hó:-da-ˀ\\
\fut-\textsc{3s.m.p}-sleep-{\punctual}\\
\glt ‘he will sleep’ (\textsc{p}-series, \textsc{experiencer})
\z
\z

That being said, \textsc{a}- and \textsc{p}-series prefixes can also be used interactively, in verbs that express two (or three) roles. In such cases, one of the roles is always an implied ‘it’ \xref{ex:haho200}. For example \textit{ahá:gęˀ} overtly expresses the perceiver (‘he’), and implies the percept (‘it’ or ‘something’) -- there is no specific part of the prefix that corresponds to the implied percept. To show that the ‘it’ role is only implied in context and not overtly expressed in the pronominal prefix, ‘it’ appears in brackets in the translation. 

\ea\label{ex:haho200}
\ea ahá:gęˀ\\
\gll a-há:-gę-ˀ\\
{\factual}-\textsc{3s.m.a}-see-{\punctual}\\
\glt ‘he saw (it)’ (\textsc{a}-series, used interactively in verbs requiring two roles)

\ex ahó:gęˀ \\
\gll a-hó:-gę-ˀ\\
{\factual}-\textsc{3s.m.p}-see-{\punctual}\\
\glt ‘(it) saw him’ (\textsc{p}-series, used interactively in verbs requiring two roles)
\z
\z

A complete list of the non-interactive and interactive prefixes is provided in the following section. The meaning relationships between the non-interactive and interactive prefixes are described in \sectref{Summary charts, pronominal prefix pronunciation and meaning}.


\subsection{Terminology for \textsc{a}- and \textsc{p}-series (non-interactive) pronominal prefixes} \label{Terminology for non-interactive (A or O) pronominal prefixes}
\largerpage

The linguistic terminology for the non-interactive pronominal prefixes is summarized in \xxref{ex:pplingex200}{ex:pplingex100}.

\ea\label{ex:pplingex200} \textsc{a}-series pronominal prefixes (with verb \stem{hnęye:s} ‘to be tall’, noun \stem{nęts-aˀgeh} ‘(on) someone’s arm’
\ea \textsc{1s.a} ‘I’, ‘my (body part)’
\begin{itemize}
    \item   \exemph{k}hnę́:ye:s ‘I am tall’
    \item   \exemph{k}nętsáˀgeh ‘(on) my arm’
\end{itemize}
\ex \textsc{1d.ex.a} ‘he and I’ or ‘she and I’, ‘our (body part)’
\begin{itemize}
    \item   \exemph{akni}hnę́:ye:s ‘we two are tall’
    \item   \exemph{akni}nętsáˀgeh ‘(on) our arm’
\end{itemize}
\ex \textsc{1p.ex.a} ‘he and we’, ‘she and we’, ‘they and I’, ‘they and we’, ‘our (body part)’
\begin{itemize}
    \item   \exemph{agwa}hnę́:ye:s ‘we all are tall’
    \item  \exemph{agwa}nętsáˀgeh ‘(on) our arm’
\end{itemize}
\ex \textsc{1d.in.a} ‘you (one person) and I’, ‘our (body part)’
\begin{itemize}
    \item  \exemph{kni̱}hnę́:ye:s, \exemph{ tni̱}hnę́:ye:s ‘you and I are tall’
    \item   \exemph{kni}nę́tsaˀge, \exemph{tni}nę́tsaˀgeh ‘(on) our arm’
\end{itemize}
\ex \textsc{1p.in.a} ‘you (two, all) and I’, or ‘you(one person) and we’, ‘our (body part)’
\begin{itemize}
    \item   \exemph{dwa}hnę́:ye:s ‘we all are tall’
    \item   \exemph{dwa}nę́tsaˀgeh ‘(on) our arm’
\end{itemize}
\ex \textsc{2s.a} ‘you (one person)’, ‘your (body part)’
\begin{itemize}
    \item   \exemph{s}hnę́:ye:s ‘you are tall’
    \item   \exemph{s}nętsáˀgeh ‘(on) your arm’
\end{itemize}
\ex \textsc{2d.a} ‘you two’, ‘your (body part)’
\begin{itemize}
    \item   \exemph{sni̱}hnę́:ye:s ‘you two are tall’
    \item   \exemph{sni}nę́tsaˀgeh ‘(on) your arm’
\end{itemize}
\ex \textsc{2p.a} ‘you all (three or more people)’, ‘your (body part)’
\begin{itemize}
    \item   \exemph{swa̱}hnę́:ye:s ‘you all are tall’
    \item   \exemph{swa}nę́tsaˀgeh ‘(on) your arm’
\end{itemize}
\ex \textsc{3s.zon.a} (abbreviated as \textsc{3s.a}) ‘it (animal, object)’, ‘its (body part)’
\begin{itemize}
    \item   \exemph{ga̱}hnę́:ye:s ‘it is tall’
    \item   \exemph{ga}nę́tsaˀgeh ‘(on) its arm’
\end{itemize}
\ex \textsc{3s.m.a} ‘he’, ‘his (body part)’
\begin{itemize}
    \item   \exemph{ha}hnę́:ye:s ‘he is tall’
    \item   \exemph{ha}nę́tsaˀgeh ‘(on) his arm’
\end{itemize}
\ex \textsc{3s.fi.a} ‘she, someone, somebody’, ‘hers, someone’s, somebody’s (body part)’
\begin{itemize}
    \item   \exemph{e}hnę:ye:s ‘she or someone is tall’
    \item   \exemph{e}nę́tsaˀgeh ‘(on) her, someone’s arm’
\end{itemize}
\ex \textsc{3ns.zon.a}, (abbreviated as \textsc{3p.a}) ‘they (animals, objects)’, ‘their (body part)’
\begin{itemize}
    \item   \exemph{gadi}hnę́:ye:s ‘they are tall’
    \item   \exemph{gadi}nętsáˀgeh ‘(on) their arm’
\end{itemize}
\ex \textsc{3ns.m.a} ‘they (males only)’, ‘their (body part)’ 
\begin{itemize}
    \item   \exemph{hadi}hnę́:ye:s ‘they are tall’
    \item   \exemph{hadi}nętsáˀgeh ‘(on) their arm’
\end{itemize}
\ex \textsc{3ns.fi.a} ‘they (females only, or a mixed group of males and females)’, ‘their (body part)’
\begin{itemize}
    \item   \exemph{gae}hnę́:ye:s ‘they are tall’
    \item   \exemph{gae}nętsaˀgeh ‘(on) their arm’
\end{itemize}
\z
\z

\ea\label{ex:pplingex100} \textsc{p}-series pronominal prefixes (with verb \stem{nǫhǫkdanih} ‘to be sick’, basic noun \stem{nǫhs-aˀ} ‘house’)
\ea \textsc{1s.p} ‘I’, ‘my (basic noun)’
\begin{itemize}
    \item  \exemph{ak}nǫhǫkdá:nih ‘I am sick’
    \item  \exemph{ak}nǫ́hsaˀ ‘my house’
\end{itemize}
\ex \textsc{1d.p} ‘we two’, ‘you (one person) and I’, ‘our (basic noun)’
\begin{itemize}
    \item  \exemph{ǫkni}nǫ̱hǫ́kdanih ‘we two are sick’
    \item  \exemph{ǫkní}nǫ̱hsaˀ ‘our house’
\end{itemize}
\ex \textsc{1p.p} ‘we all’, ‘you (two or more) and I’, ‘you and we’, ‘our (basic noun)’
\begin{itemize}
    \item  \exemph{ǫgwa}nǫ̱hǫ́kdanih ‘we all are sick’
    \item  \exemph{ǫgwá}nǫ̱hsaˀ ‘our house’
\end{itemize}
\ex \textsc{2s.p} ‘you (one person)’, ‘your (basic noun)’
\begin{itemize}
    \item   \exemph{sa}nǫhǫkdá:nih ‘you are sick’
    \item   \exemph{sa}nǫ́hsaˀ ‘your house’
\end{itemize}
\ex \textsc{2d.p} ‘you two’, ‘your (basic noun)’
\begin{itemize}
    \item   \exemph{sni}nǫhǫkdá:nih ‘you two are sick’
    \item  \exemph{sni}nǫ́hsaˀ ‘your house’
\end{itemize}
\ex \textsc{2p.p} ‘you all (more than two)’, ‘your (basic noun)’
\begin{itemize}
    \item   \exemph{swa}nǫhǫkdá:nih ‘you all are sick’
    \item   \exemph{swa}nǫ́hsaˀ ‘your house’
\end{itemize}
\ex \textsc{3s.zon.p} (abbreviated as \textsc{3s.p}) ‘it’, ‘its (basic noun)’
\begin{itemize}
    \item   \exemph{o}nǫhǫkdá:nih ‘it is sick’
    \item   \exemph{o}nǫ́hsaˀ ‘its house’
\end{itemize}
\ex \textsc{3s.m.p} ‘he’, ‘his (basic noun)’
\begin{itemize}
    \item  \exemph{ho}nǫhǫkdá:nih ‘he is sick’
    \item  \exemph{ho}nǫ́hsaˀ ‘his house’
\end{itemize}
\ex \textsc{3s.fi.p} ‘she’, ‘someone’, ‘anyone’, ‘her’, ‘someone’s’, ‘anyone’s (basic noun)’
\begin{itemize}
    \item   \exemph{go}nǫhǫkdá:nih ‘she, someone is sick’
    \item  \exemph{go}nǫ́hsaˀ ‘her, someone’s house’
\end{itemize}
\ex \textsc{3ns.zon.p}, (abbreviated as \textsc{3p.p}) ‘they (animals, objects), ‘their (basic noun)’
\begin{itemize}
    \item   \exemph{odi}nǫ̱hǫ́kdanih ‘they are sick’
    \item   \exemph{odí}nǫ̱hsaˀ ‘their house’
\end{itemize}
\ex \textsc{3ns.m.p} ‘they (males only), ‘their (basic noun)’
\begin{itemize}
    \item  \exemph{hodi}nǫ̱hǫ́kdanih ‘they are sick’
    \item  \exemph{hodí}nǫ̱hsaˀ ‘their house’
\end{itemize}
\ex \textsc{3ns.fi.p} ‘they’ (females only, or a mixed group of males and females), ‘their (basic noun)’
\begin{itemize}
    \item   \exemph{godi}nǫ̱hǫ́kdanih ‘they are sick’
    \item   \exemph{godí}nǫ̱hsaˀ ‘their house’
\end{itemize}
\z
\z



\subsection{The special status of the \textsc{3s} ‘it’ role and prefixes} \label{The special status of the neuter singular ‘it’ role and prefixes}
The \textsc{3s} ‘it’ role and prefixes, shown in \xref{ex:haho400}, are different from the other \textsc{a}- and \textsc{p}-series prefixes in several important ways. 

\ea\label{ex:haho400} \textsc{3s} prefixes
\ea \textsc{a}-series \stem{ga-}, \stem{wa-}, \stem{a-} or \stem{ø-} \textsc{3s.a}
\ex \textsc{p}-series \stem{(y)o-} \textsc{3s.p}
\z
\z

First, as mentioned in the previous section, the \textsc{3s} ‘it’ role is implied but not explicitly represented (pronounced) in the pronominal prefix of verbs that use  \textsc{a}- or \textsc{p}-series prefixes in an interactive sense -- as in \textit{ahá:gęˀ} ‘he saw (it)’. This property is unique to the \textsc{3s} ‘it’ prefix.

Second, the \textsc{3s} ‘it’ prefixes are also relevant for the distinction between \textsc{personal} verbs and \textsc{neuter} verbs. Personal verbs take a full range of pronominal prefixes -- either all of the \textsc{a}-series prefixes \xref{ex:haho800a} or all of the \textsc{p}-series prefixes \xref{ex:haho800b}. In contrast, neuter verbs only take one of the \textsc{3s} ‘it’ prefixes \xref{ex:haho900}. 

\ea\label{ex:haho800} \textsc{personal} verbs (taking a full range of \textsc{a}- or \textsc{p}-series prefixes)
\ea\label{ex:haho800a} hahnę́:ye:s ‘he is tall’, ehnę́:ye:s ‘she is tall’, ga̱hnę́:ye:s ‘it (animal) is tall’, khnę́:ye:s ‘I am tall’ (etc.)
\ex\label{ex:haho800b} hohsę: ‘he is fat’, gohsę: ‘she is fat’, ohsę: ‘it (animal) is fat’, agáhsę: ‘I am fat’ (etc.)
\z
\z

\ea\label{ex:haho900} \textsc{neuter} verbs (only taking one of the \textsc{3s} ‘it’ prefixes in \ref{ex:haho400})
\ea\label{ex:haho900a} \textsc{a}-series: gá:dę:s ‘it is thick, dense’, wagyé:sęh ‘it is easy’
\ex\label{ex:haho900b} \textsc{p}-series: otó:weˀ ‘it is cold (weather)’, áogaˀt ‘it is rough’
\z
\z

Finally, uniquely for \textsc{neuter}, \textsc{stative-only} verbs the choice of pronominal prefix often depends on the pronominal prefix of the incorporated noun (see \sectref{Pronominal prefix choice for stative-only verbs}). 


\section{\textsc{interactive} prefixes} \label{gǫ (interactive) prefixes}
The \textsc{interactive} prefixes express an interaction between groups, referring to two roles at once -- for example, the \textsc{perceiver} (‘I’) and the \textsc{percept} (‘you singular’) in \xref{ex:interex100}.

\ea\label{ex:interex100} agǫ́:gęˀ\\
\gll a-gǫ́:-gę-ˀ\\
{\factual}-\exsc{1s>2s-}see-{\punctual}\\
\glt ‘I saw you (singular)’
\z

Interactive prefixes also refer to three roles, for the handful of verbs that require such. In \xref{ex:interex200}, the interactive prefix \stem{shǫgwa-} refers to a “doer” (‘he’), a \textsc{recipient} (‘us all’), and an item being transferred (‘it’, or ‘a song’). For verbs requiring three roles, the item being transferred is always an object.

\ea\label{ex:interex200} shǫgwaęnáwi̱hǫˀ\\
\gll shǫgwa-ęn-áwi̱-h-ǫˀ\\
\exsc{3s.m>1p}-song-give-{\stat}-{\distributive}\\
\glt ‘he has given each of us a song’
\z

A complete list of interactive prefixes, with an explanation of their meaning and how they relate to the non-interactive prefixes, is provided in the following section. Section \sectref{Interactive (GǪ) prefix meaning and pronunciation} describes their pronunciation.


\subsection{Summary charts, pronominal prefix pronunciation and meaning} \label{Summary charts, pronominal prefix pronunciation and meaning}
The figures in this section are based on a table in \citet{foster_course_1993}, which in turn is based on the one in \citet{lounsbury_oneida_1953}. However, the information has been reorganized in order to better reveal the structure and meaning of the pronominal prefixes. (That being said, \tabref{figtab:1:2 interactions} and \tabref{figtab:2:1 interactions} retain the original organization of part of the table in \cite{foster_course_1993}.) The pronunciation variants of each prefix are also summarized in these tables, and are explained in later sections.

\tabref{figtab:1:2 interactions} and \tabref{figtab:2:1 interactions} (page \pageref{figtab:2:1 interactions}) summarize the prefixes that describe 1\textsuperscript{st} and 2\textsuperscript{nd} person interactions.  To read \tabref{figtab:1:2 interactions} and \tabref{figtab:2:1 interactions}, replace the “X” in the first column with the item listed in the top row. For example, \stem{gǫ-, go-} means \textsc{1s>2s}, as in \textit{gonǫ́hkwaˀ} ‘I love you (one person)’. In contrast, \stem{kni-, kn-}, meaning \textsc{1>2(d)} has a wider range of potential meanings, (including \textsc{1d>2s} and \textsc{2du>2du)}, as in \textit{kninǫ́hkwaˀ} ‘we two love you (one)’ or ‘we two love you two’: any combination of \textsc{singular} or \textsc{dual} \textsc{1>2} is implied -- excluding \textsc{plural} meanings -- as long as the \textsc{dual} is expressed.

\begin{table}
\caption{1>2 interactions}
\label{figtab:1:2 interactions}

\begin{tabular}{|l|ccc|}
\hline
& \multicolumn{1}{c|}{X>\textsc{2s `you'}} & \multicolumn{1}{c|}{X>\textsc{2d `you two'}} & X>\textsc{2p `you all'} \\
\hline
\textsc{1s `I'}>X & \multicolumn{1}{c|}{gǫ-C,(i),V }& \multicolumn{1}{c|}{} & \\
& \multicolumn{1}{c|}{go-n} &\multicolumn{1}{c|}{} & \\
& \multicolumn{1}{c|}{\textbf{\textsc{1s>2s}}} &\multicolumn{1}{c|}{} & \\
\cline{1-2}
\textsc{1d `we two'}>X & & \multicolumn{1}{c|}{kni-C} & \\
 & & \multicolumn{1}{c|}{gy-a} & \\
 & & \multicolumn{1}{c|}{kn-e,i,o} & \\
 & & \multicolumn{1}{c|}{\textbf{\textsc{1>2(d)}}} & \\
\cline{1-3}
\textsc{1p `we all'}>X & & & gwa-C \\
& & & gw-a,e \\
& & & gwę-(i) \\
& & & gy-o\\
& & & \textbf{\textsc{1>2(p)}}\\
\hline
\end{tabular}
\end{table}

\begin{table}
\caption{2>1 interactions}
\label{figtab:2:1 interactions}
\begin{tabular}{|l|ccc|}
\hline
& \multicolumn{1}{c|}{X>\textsc{1s `me'}} & \multicolumn{1}{c|}{X>\textsc{1d `us two'}} & \multicolumn{1}{c|}{X>\textsc{1p `us all'}}\\
\hline
\textsc{2s `you'}>X & \multicolumn{1}{c|}{sg-V} & \multicolumn{1}{c|}{} & \\
& \multicolumn{1}{c|}{sk-n} & \multicolumn{1}{c|}{} & \\
& \multicolumn{1}{c|}{sge-k,g,C} & \multicolumn{1}{c|}{} & \\
& \multicolumn{1}{c|}{\textbf{\textsc{2s>1s}}} & \multicolumn{1}{c|}{} & \\
\cline{1-2}
\textsc{2d `you two'}>X & & \multicolumn{1}{c|}{skni-C} & \\
& & \multicolumn{1}{c|}{skn-e,i,o} & \\
& & \multicolumn{1}{c|}{sgy-a} & \\
& & \multicolumn{1}{c|}{\textbf{\textsc{2>1(d)}}}& \\
\cline{1-3}
\textsc{2p `you all}>X & & & sgwa-C\\
& & & sgw-a,e \\
& & & sgwę-(i) \\
& & & sgy-o\\
& & & \textbf{\textsc{2>1(p)}}\\
\hline
\end{tabular}
\end{table}

\FloatBarrier
\tabref{figtab:a,o,3s.m interactions} (page \pageref{figtab:a,o,3s.m interactions}) summarizes the meaning and pronunciation of the non-interactive, 1\textsuperscript{st} and 2\textsuperscript{nd} person \textsc{a}- and \textsc{p}-series pronominal prefixes. It also includes the \textsc{3s.m>x}, \textsc{x>3s.m} \textsc{interactive} prefixes because these prefixes uniquely share the person, clusivity, and number distinctions expressed by the 1\textsuperscript{st} person \textsc{a}- and \textsc{p}-series prefixes. (The remaining non-masculine 3\textsuperscript{rd} person \textsc{a}- and \textsc{p}-series prefixes are listed later in \tabref{figtab:3:3 interactions}, page \pageref{figtab:3:3 interactions}.)

To read \tabref{figtab:a,o,3s.m interactions} (page \pageref{figtab:a,o,3s.m interactions}) replace the \textsc{a}- or \textsc{p}-series “X” in the top row with an item from the first column. For example, the \stem{g-, k-, ge-} prefix means \textsc{1s>(it)} -- that is, ‘I’ or ‘I>(it)’. Similarly, the \stem{\textsuperscript{w}ag-, \textsuperscript{w}ak-, \textsuperscript{w}age-, ǫg-} (etc.) prefix means \textsc{(it)>1s} -- ‘I’ or ‘(it)>I,me’. (Recall that (it) signifies an implied role.) For the interactive \textsc{3s.m} prefixes, replace the “X” with an item in the first column. For example the \stem{hi-, he-, hiy-, hey-, h-} prefix means \textsc{1s>3s.m} or ‘I>him’, and the \stem{hag-, hak-, hage-} prefix means \textsc{3s.m>1s} or ‘he>me’.

\begin{table}
\caption{\textsc{a}- and \textsc{p}-series, with \textsc{3s.m} interactions}
\label{figtab:a,o,3s.m interactions}
\small
\begin{tabular}{lllll}
\lsptoprule
 & \textsc{a}-series & \textsc{x}>\textsc{3s.m} & \textsc{3s.m}>\textsc{x}  & \textsc{p}-series\\
 &  & \textsc{`him'} & \textsc{`he'} & \\
\midrule
\textsc{1s>(x) `I'} & g & hi-/he-C & hag-V & \textsuperscript{W}ag-\\
 & k-n,(h)n,(ˀ)n & hiy-/hey-V & hak-n, (h)n, (ˀ)n & \textsuperscript{W}ak-n,(h)n,(ˀ)n \\
 & ge-k,g,CC & hi/he-(i) & hage-k,g,CC & \textsuperscript{W}age-k,g,CC \\
 & \textbf{\textsc{1s.a}} & \textbf{\textsc{1s>3s.m}} & \textbf{\textsc{3s.m>1s}} & ǫg/ǫk/\\
  & & & & ǫge-\textsc{fact.Io}\\
 & & & & \textbf{\textsc{(x)>1s.p}}\\
\midrule
 \textsc{1d>(x) `we two'}& kni-, tni-C & shetni-C & shǫkni-C &  \textsuperscript{Y}ǫkni-C\\
\textsc{inclusive} & kn-, tn-e,i,o & shetn-e,i,o & shǫkn-e,i,o & \textsuperscript{Y}ǫkn-e,i,o \\
 & gy-a & shegy-a & shǫgy-a & \textsuperscript{Y}ǫgy-a\\
 & \textbf{\textsc{1d.in.a}} & \textbf{\textsc{1d.in>3s.m}} & \textbf{\textsc{3s.m>1d}} & \textbf{\textsc{(x)>1d.p}} \\
\cmidrule{1-3}
\textsc{1d>(x) `we two'} & \textsuperscript{Y}akni-C & shakni-C & & \\
\textsc{exclusive} & \textsuperscript{Y}akn-e,i,o & shakn-e,i,o & & \\
 & \textsuperscript{Y}agy-a & shagy-a & & \\
 & \textbf{\textsc{1d.ex.a}} & \textbf{\textsc{1d.ex>3s.m}} & & \\
\midrule
\textsc{1p>(x) `we all'} & dwa-C & shedwa-C & shǫgwa-C & \textsuperscript{Y}ǫgwa-C\\
\textsc{inclusive} & dw-a,e & shedw-a,e & shǫgw-a,e &
\textsuperscript{Y}ǫgw-a,e \\
 & dwę-(i) & shedwę-(i) & shǫgwę-(i) & \textsuperscript{Y}ǫgwę-(i)\\
 & gy-o & shegy-o & shǫgy-o & \textsuperscript{Y}ǫgy-o\\
  & \textbf{\textsc{1p.in.a}} & \textbf{\textsc{1p.in>3s.m}} & \textbf{\textsc{3s.m>1p}} & \textbf{\textsc{(x)>1p.p}}\\
\cmidrule{1-3}
\textsc{1p>(x) `we all'}& \textsuperscript{Y}agwa-C & shagwa-C & & \\
\textsc{exclusive} & \textsuperscript{Y}agw-a,e & shagw-a,e & & \\
 & \textsuperscript{Y}agwę-(i) & shagwę-(i) & & \\
 & \textsuperscript{Y}agy-o & shagy-o & & \\
 & \textbf{\textsc{1p.ex.a}} & \textbf{\textsc{1p.ex>3s.m}} & & \\
 \midrule
\textsc{2s>(x) `you'} & (eh)s-C,V & hehs-C,V & hya-C& sa-C \\
 & (eh)se-k & hehse-k,g & hy-a,e,o & s-a,e,o \\
 & d-r & \textbf{\textsc{2s>3s.m}} & hyę-(i) & sę-(i)\\
 & j-y & & \textbf{\textsc{3s.m>2s}} & \textbf{\textsc{(x)>2s.p}}  \\
 & \textbf{\textsc{2s.a}} & & & \\
\midrule
 \textsc{2d>(x) `you two'} & (eh)sni-C & \multicolumn{2}{c}{shesni-C} & sni-C \\
& (eh)sn-e,i,o & \multicolumn{2}{c}{shesn-e,i,o} & sn-e,i,o \\
& j-a & \multicolumn{2}{c}{shej-a} & j-a\\
& \textbf{\textsc{2d.a}} & \multicolumn{2}{c}{\textbf{\textsc{2d>3s.m(opp)}}} & \textbf{\textsc{(x)>2d.p}} \\
\midrule
\textsc{2p>(x) `you all'} & (eh)swa-C & \multicolumn{2}{c}{sheswa-C} & swa-C \\
 & (eh)sw-a,e & \multicolumn{2}{c}{shesw-a,e} & sw-a,e \\
 & (eh)swę-(i) & \multicolumn{2}{c}{sheswę-(i)} & swę-(i)\\
 & j-o & \multicolumn{2}{c}{shej-o} & j-o\\
  & \textbf{\textsc{2p.a}} & \multicolumn{2}{c}{\textbf{\textsc{2p>3s.m(opp)}}} & \textbf{\textsc{(x)>2p.p}}\\
\lspbottomrule
\end{tabular}
\end{table}

The prefixes in \tabref{figtab:1,2, 3fi, 3nsg interactions} (page \pageref{figtab:1,2, 3fi, 3nsg interactions}) are grouped together because they all express 1\textsuperscript{st} and 2\textsuperscript{nd} interactions with \textsc{3s.fi} and \textsc{3ns} meanings in the same way. To read \tabref{figtab:1,2, 3fi, 3nsg interactions} (page \pageref{figtab:1,2, 3fi, 3nsg interactions}) replace the “X” in the top row with an item from the first column. For example, the \stem{\textsuperscript{y}ǫki-, \textsuperscript{y}ǫkiy-, \textsuperscript{y}ǫk-} prefix means \textsc{3s.fi/3ns>1ns(opp)}, which expands to \textsc{3s.fi>1ns} ‘she>us (two or more)’  or \textsc{3ns>1ns} ‘they>us (two or more)’. (Clusivity, gender, and the dual/plural distinction are not relevant here.)

\begin{table}
\caption{\textsc{1,2, 3fi}, and \textsc{3ns} interactions}
\label{figtab:1,2, 3fi, 3nsg interactions}

\small
\begin{tabular}{lcccc}
\lsptoprule
& \textsc{3s.fi>x} & \textsc{3ns.fi>x `they'} & \textsc{x}:\textsc{3s.fi} & \textsc{x}:\textsc{3ns.fi `them'}\\
& \textsc{`she'} & \textsc{3ns.zon>x `they'} & \textsc{`her'} & \textsc{x}:\textsc{3ns.zon `them'}\\
& & \textsc{3ns.m>x `they'} & & \textsc{x}:\textsc{3ns.m `them'} \\
& & \textsc{(3ns)} & & \textsc{(3ns)} \\
\midrule
\textsc{1s `I'} & \textsuperscript{Y}ǫg- & gaǫg-& ke-C,(i) & gake-/kake-C,(i) \\
& \textsuperscript{Y}ǫk-n & gaǫk-n & key-V & gakey-/kakey-V\\
& \textsuperscript{Y}ǫge-k,g,C & gaǫge-k,g,C & \textbf{\textsc{1s>3s.fi}} & \textbf{\textsc{1s>3ns}} \\
& \textbf{\textsc{3s.fi>1s}} & \textbf{\textsc{3ns>1s}} &  &  \\
\midrule
\textsc{1ns.ex} & & & \multicolumn{2}{c}{\textsuperscript{Y}aki-C,(i)}\\
\textsc{`we exclusive'} & & & \multicolumn{2}{c}{\textsuperscript{Y}akiy-V} \\
& \multicolumn{2}{c}{\textsuperscript{Y}ǫki-C,(i)} & \multicolumn{2}{c}{\textbf{\textsc{1ns.ex>3s.fi/3ns}}}\\
& \multicolumn{2}{c}{\textsuperscript{Y}ǫkiy-V} & \multicolumn{2}{c}{\textbf{\textsc{}}}\\
\cmidrule{1-1}\cmidrule{4-5}
\textsc{1ns.in} & \multicolumn{2}{c}{\textbf{\textsc{3s.fi/3ns>1ns}}} & \multicolumn{2}{c}{\textsuperscript{Y}eti-C,(i) }\\
\textsc{`we inclusive'} & \multicolumn{2}{c}{\textbf{\textsc{}}} & \multicolumn{2}{c}{\textsuperscript{Y}etiy-V}\\
 & & & \multicolumn{2}{c}{\textbf{\textsc{1ns.in>3s.fi/3ns}}}\\
 & & & \multicolumn{2}{c}{\textbf{\textsc{}}}\\
\midrule
\textsc{2s `you'} & \textsuperscript{Y}esa-C & gaesa-C & (h)she-C,(i) & gashe-/kashe-C,(i) \\
& \textsuperscript{Y}es-V & gaes-V & (h)shey-V & gashey-/kashey-V\\
& \textsuperscript{Y}esę-(i) & gaesę-(i)  & \textbf{{\textsc{2s>3s.fi}}} & \textbf{{\textsc{2s>3ns}}} \\
& \textbf{{\textsc{3s.fi>2s}}} & \textbf{{\textsc{3ns>2s}}} &  &  \\
\midrule
\textsc{2ns} & & \multicolumn{2}{c}{\textsuperscript{Y}etshi-/\textsuperscript{Y}etsi-C,(i)} & \\
\textsc{`you two, all'} & & \multicolumn{2}{c}{\textsuperscript{Y}etshiy-/\textsuperscript{Y}etsiy-V} & \\
& & \multicolumn{2}{c}{\textbf{\textsc{3s.fi/3ns>2ns(opp)}}} & \\
& & \multicolumn{2}{c}{\textbf{\textsc{}}} & \\
\lspbottomrule
\end{tabular}
\end{table}

\FloatBarrier
\tabref{figtab:3:3 interactions} (page \pageref{figtab:3:3 interactions}) lists the remaining pronominal prefixes, which all express 3>3 interactions in similar ways. (Also shown are the non-interactive 3\textsuperscript{rd} person prefixes, which can be used in an interactive sense with an implied ‘it’). To read \tabref{figtab:3:3 interactions} (page \pageref{figtab:3:3 interactions}) replace the “X” in the first column with an item from the top row. For example, the \stem{ho-, haw-, ha-} prefix means \textsc{(3s)>3s.m} -- either \textsc{3s.m.p} or ‘(it)>him’, depending on the verb. Similarly, the \stem{ha-, h-, hę-} prefix means \textsc{3s.m>(3s)} -- either \textsc{3s.m.a} or ‘he>(it)’. The abbreviation \textsc{(+ns)} denotes that both arguments must be \textsc{non-singular}, and the abbreviation \textsc{(nfi)} denotes that neither argument can be \textsc{feminine indefinite}. Note that the \textsc{3s>X} row lists all of the \textsc{p}-series pronominal prefixes and that the \textsc{X>3s} column lists all of the \textsc{a}-series pronominal prefixes.


\begin{sidewaysfigure}
\caption{3>3 interactions}
\label{figtab:3:3 interactions}
\resizebox{.8\textwidth}{!}{
\small
\begin{tabular}{|l|l|l|l|l|l|l|l|}
\hline
 & & \textsc{x>3s.zon} & \textsc{x>3s.m} & \textsc{x>3ns.m} & \textsc{x>3s.fi} & \textsc{x>3ns.fi} & \textsc{x>3ns.zon}\\
 & & \textsc{`it'} & \textsc{`him'} & \textsc{`them'} & \textsc{`her'} & \textsc{`them'} & \textsc{`them'} \\
 \hline
\textsc{3s.zon>x} & ga-C & \textsuperscript{Y}o-C,(a),(i) & ho-C,(a),(i) & hodi-C,(i) & \textsuperscript{YA}go-C,(a),(i) & \textsuperscript{YA}godi-C,(i) & \textsuperscript{Y}odi-C,(i)\\
\textsc{`it'} & gę-(i) & \textsuperscript{Y}aw-e & haw-e/i & hon-V & \textsuperscript{YA}gaw-e & \textsuperscript{YA}gon-V & \textsuperscript{Y}on-V\\
& w-a,e & \textsuperscript{Y}a-o & ha-o & \textbf{\textsc{3ns.m.p}}& \textsuperscript{YA}ga-o & \textbf{\textsc{3ns.fi.p}} & \textbf{\textsc{3p.p}} \\
& \textsuperscript{Y}{}-o & \textbf{\textsc{3s.p}} & \textbf{\textsc{3s.m.p}} & & \textbf{\textsc{3s.fi.p}} & & \\
& \textbf{\textsc{3s.a}} & & & & & & \\
\hline
\textsc{3s.m>x} & {ha-C} & & \multicolumn{1}{c|}{hǫwa-C} & & \multicolumn{3}{c|}{shago-C,(a),(i)} \\
\textsc{`he'} & {h-V} & & \multicolumn{1}{c|}{hǫw-a,e} & & \multicolumn{3}{c|}{shagaw-e} \\
& {hę-(i)} & & \multicolumn{1}{c|}{hǫwę-(i)} & & \multicolumn{3}{c|}{shaga-o} \\
& \textbf{\textsc{3s.m.a}} & & \multicolumn{1}{c|}{hǫway-o} & & \multicolumn{3}{c|}{\textbf{\textsc{3s.m>3fi/3p}}} \\
& & & \multicolumn{1}{c|}{\textbf{\textsc{3s.m/3s.fi>3s.m}}} & & \multicolumn{3}{c|}{\textbf{\textsc{}}} \\
& & & \multicolumn{1}{c|}{\textbf{\textsc{}}} & & \multicolumn{3}{c|}{\textbf{\textsc{}}} \\
& & & \multicolumn{1}{c|}{\textbf{\textsc{}}} & & \multicolumn{3}{c|}{\textbf{\textsc{}}} \\
\cline{1-3}\cline{6-8}
\textsc{3s.fi>x} & {\textsuperscript{Y}e-C,(i)} & & & & \textsuperscript{Y}ǫdade-C & & \\
\textsc{`she'} & {\textsuperscript{Y}ǫ-(a)} & & & & \textsuperscript{Y}ǫdad-V & & \\
& {\textsuperscript{Y}ag-e,o} & & & & \textsuperscript{Y}ǫdag-y & & \\
& {\textsuperscript{Y}ę-(e)} & & & & \textsuperscript{Y}ǫdat-n & & \\
& \textbf{\textsc{3s.fi.a}} & & & & \textbf{\textsc{3s.fi>3s.fi}} & & \\
\cline{1-4}\cline{6-6}
\textsc{3ns.fi>x} & gae-C,(i) & & \multicolumn{2}{l|}{} & \multicolumn{1}{l}{} & gaǫdade-C & \\
\textsc{`they'} & gaǫ-(a) & & \multicolumn{2}{l|}{} & \multicolumn{1}{l}{} & {gaǫdad-V} & \multicolumn{1}{c|}{gǫwadi-C,(i)} \\
& {ga:g-e,o} & & \multicolumn{2}{l|}{\textbf{\textsc{}}} & \multicolumn{1}{l}{} & gaǫdag-y & \multicolumn{1}{c|}{gǫwan-V} \\
& {gaę-(e)} & & \multicolumn{2}{l|}{\textbf{\textsc{}}} &\multicolumn{1}{l}{} & {{gaǫdat-n}} & \multicolumn{1}{c|}{(hadi-)}\\
& \textbf{\textsc{3ns.fi.a}} & & \multicolumn{2}{l|}{{}} & \multicolumn{1}{l}{} & \textbf{\textsc{3fi>3fi(+ns)}} & \multicolumn{1}{c|}{\textbf{\textsc{3fi/3ns>3p}}} \\
& \textbf{\textsc{}} & & \multicolumn{2}{c|}{hǫwadi-C,(i)} & \multicolumn{1}{l}{} & \textbf{\textsc{}} & \multicolumn{1}{c|}{\textbf{\textsc{}}} \\
& \textbf{\textsc{}} & & \multicolumn{2}{c|}{hǫwan-V} & \multicolumn{1}{l}{} & \textbf{\textsc{}} & \multicolumn{1}{c|}{\textbf{\textsc{}}}\\
\cline{1-3}\cline{6-7}
\textsc{3ns.zon>x} & gadi-C,(i) & & \multicolumn{2}{c|}{\textbf{\textsc{3s.m/3s.fi>3ns.m,}}} & \multicolumn{2}{c|}{} & \multicolumn{1}{c|}{\textbf{\textsc{}} }\\
\textsc{`they'} & gęn-V & & \multicolumn{2}{c|}{\textbf{\textsc{3ns>3m}}} & \multicolumn{2}{c|}{} & \multicolumn{1}{c|}{\textbf{\textsc{}}} \\
& \textbf{\textsc{3p.a}} & & \multicolumn{2}{l|}{} & \multicolumn{2}{c|}{shagodi-C,(i)} & \\
\cline{1-3}
\textsc{3ns.m>x} & {hadi-C,(i)} & & \multicolumn{2}{l|}{\textbf{\textsc}} & \multicolumn{2}{c|}{{shagon-V}} & \\
\textsc{`they'} & {hęn-V} & & \multicolumn{2}{l|}{\textbf{\textsc{}}} & \multicolumn{2}{c|}{\textbf{\textsc{3ns(nfi)>3fi}}}  & \\
& \textbf{\textsc{3ns.m.a}} & & \multicolumn{2}{l|}{} & \multicolumn{2}{c|}{\textbf{\textsc{}}} & \\
& & & \multicolumn{2}{c|}{} & \multicolumn{2}{c|}{\textbf{\textsc{}}} & \\
\hline
\end{tabular}
}
\end{sidewaysfigure}


\subsection{Recap: pronominal prefix meaning} \label{Recap -- pronominal prefix meaning}
The concepts of person, number, gender, and clusivity are expressed in the \textsc{a}-series, \textsc{p}-series, and interactive pronominal prefixes. Verbs express up to three roles with these pronominal prefixes. Verbs with one role take either the \textsc{a}- or \textsc{p}-series of non-interactive prefix (see \sectref{Pronominal prefix choice for stative-only verbs}, \sectref{Pronominal prefix choice for one-role three-aspect verbs}, and \sectref{ch:E-verbs and pronominal prefix choice}), whereas verbs with two or three roles take both the interactive type and the non-interactive \textsc{a}- and \textsc{p}-series prefixes (see \sectref{Pronominal prefix choice for two- and three-role, three-aspect verbs}). In the latter case, the non-interactive \textsc{a}- and \textsc{p}-series prefixes express a non-zoic-neuter role plus an implied ‘it’ (zoic-neuter) role.
