\chapter{The Hodinǫ̱hsǫ́:nih}
\label{ch:The Hodinǫhsǫ́:nih}
The \textit{Hodinǫ̱hsǫ́:nih} ‘People of the Longhouse’ once lived in the Mohawk River valley and around the Finger Lakes district in present-day New York State. The \textit{Ganyęˀge̱hó:nǫˀ} ‘Mohawk people’ resided between the Allegheny and Catskill Mountains, while the \textit{Onǫdowáˀga:ˀ} ‘Seneca people’ were settled along the Gene\-ssee River. In between, the \textit{Onǫda̱ˀgehó:nǫˀ} ‘Onondaga people’ lived near Syracuse, New York, and the \textit{Ohnyahęhó:nǫˀ} ‘Oneida people’ lived near Lake Oneida, New York. The \textit{Gayogo̱hó:nǫˀ} ‘Cayuga people’ lived in an area between Lake Cayuga and Lake Owasco in present-day Cayuga County, New York (\cite[3--6]{michelson_comparative_1988}, see \figref{map:original.home}, page \pageref{map:original.home}).

\begin{figure}
\includegraphics[width=\textwidth]{figures/Six Nations c 1720.jpg}
\caption{The original homelands of the Ǫgwehó:weh (based on \citet{morgan_lewis_henry_map_1720}, \citet{gisgeography_new_2013}, \citet{nonenmacher_iroquois_2004}.}
\label{map:original.home}
\end{figure}

The \textit{Ǫgwehǫ́:weh} ‘Real People’ or ‘Iroquois’\footnote{The word ‘Iroquois’ is likely a “…pidginized Basque word learned by the French via some native intermediary language.” (\cite[1122]{bakker_basque_1991}). The Basques had extensive trading relationships with the Mi’kmaq, Innu, and Maliseet “…from the late 1540’s to the first decades of the seventeenth century in the Gulf of Saint Lawrence and the Atlantic seacoast.” (\cite[1120]{bakker_basque_1991}). Consequently, a Basque-Algonquian pidgin or trade language developed to facilitate communication. In this pidgin, the word from which ‘Iroquois’ derives was something like \textit{(h)ilo\-koa} ‘killer-person’ (\cite[1122--1123]{bakker_basque_1991}), a name reflecting the Iroquois’ reputation for war-mongering. The original Basque roots are \stem{hil} ‘kill’ and \stem{-ko-a} ‘person originating from’ (\cite[1121]{bakker_basque_1991}).} formed the original \textit{Hwíhs Niyohwęjá:ge:} ‘Five Nations Iroquois Confederacy or League of the Iroquois’. The League came about some time before European contact. According to the oral tradition recorded in \citet[xix-xxxi]{gibson_concerning_1992}, \textit{Deganawí:daˀ} ‘the Peacemaker’, aided by \textit{Hayę́hwataˀ} ‘He Who Makes a Wampum Belt’, first persuaded the \textit{Ga\-nyęˀge̱hó:nǫˀ} and \textit{Ohnyahęhó:nǫˀ} to join together in alliance, followed by the \textit{Onǫda̱ˀgehó:nǫˀ}, \textit{Gayogo̱hó:nǫˀ}, and \textit{Onǫdowáˀga:ˀ}. The central concepts of the Peacemaker’s message (or \textit{Gayanę̱hsraˀgó:wah} ‘the Great Law’) were \textit{Gaihwí:yo:} ‘the Good Message’, \textit{Ga̱hsháhsdę̱hsraˀ} ‘power, strength’, and \textit{Sgę:nǫˀ} ‘peace’.

The \textit{Ganyęˀge̱hó:nǫˀ}, \textit{Onǫda̱ˀgehó:nǫˀ}, and \textit{Onǫdowáˀga:ˀ} formed the senior “side” (moiety or phratry) of the League, while the \textit{Ohnyahęhó:nǫˀ} and \textit{Gayogo̱hó:nǫˀ} constituted the junior side. The moieties were metaphorically conceived as kin belonging to the same \textit{Ganǫhsǫ́:nih} ‘Longhouse’ or ‘household’: the senior moiety were “fathers” (father or father’s brother) to the junior moiety, and were also known as the ‘Three Brothers’. The junior moiety stood as “sons” (a man’s brother’s son or a woman’s brother’s son) to the senior moiety (H. Woodbury, in \cite[54--55]{fenton_great_1998}), and were known as the ‘Four Brothers’ (for reasons described next). 

In 1722 -- after losing their homelands in present-day North Carolina in 1712 -- the \textit{Dahsgáowęˀ} ‘Tuscarora people’ were adopted into the League by the \textit{Gayogo̱hó:nǫˀ} (\cite[54]{fenton_great_1998}, \cite[26]{schoolcraft_notes_1846}). The League was afterwards known as the \textit{Hyeí Niyǫhwęjá:ge:} ‘Six Nations’. Meanwhile, the \textit{Gayogo̱hó:nǫˀ} had also adopted the \textit{Ewáˀganhaˀ} ‘Delaware’. Consequently, the junior moiety was known as the ‘Four Brothers’.\footnote{\citet[117]{shimony_conservatism_1994} points out that the Nanticoke, Tutelo, and Wyandot had also previously been adopted.} 

The \textit{Ganyęˀge̱hó:nǫˀ} were the ‘Keepers of the Eastern Door’ and the \textit{Onǫdowáˀ\-ga:ˀ}, the ‘Keepers of the Western Door’ -- terms which referred to the boundaries of \textit{Ǫgwehǫ́:weh} country. The \textit{Onǫda̱ˀgehó:nǫˀ} were \textit{Honadejíhsdane:t} ‘Keepers of the Fire’ -- referring to the Council Fire of the Confederacy -- at \textit{Onǫ́da̱ˀgeh} (\cite[xxvi]{gibson_concerning_1992}). The Grand Council consisted of 50 \textit{Hodiyanéhsǫˀ} (hereditary chiefs, whose titles are listed in \sectref{chiefs}). To maintain peace between \textit{Hodinǫ̱hsǫ́:nih} nations, the Grand Council debated League affairs with the goal of reaching \textit{Sga̱ˀnígǫ̱ha:t} ‘one mind’ or ‘consensus’.

After the American Revolution, the Council Fire and Grand Council were re\-established at \textit{Tahnawá:deˀ} ‘Tonawanda’. A twin Council was also established at Six Nations. It operates alongside the Six Nations Council, which was imposed by the Government of Canada in 1924.

\section{Gaihwí:yo: and related matters}
\label{gaihwiyo}

\textit{Gaihwí:yo:} (the ‘Good Message’, referred to earlier), can be characterized as the League’s “…body of dicta, anecdotes, and laws…” (\cite[192]{shimony_conservatism_1994}). It includes the relatively newer prophecies of the Seneca prophet \textit{Sganyadáiyoˀ} ‘Handsome Lake’ (1735--1815). For this reason, \textit{Gaihwí:yo:} is also known as the ‘Code of Handsome Lake’ or ‘the Code’. 

\textit{Gaihwí:yo:} is preached in two types of events, the Six Nations Convention and the Chiefs’ Convention. (‘Six Nations’ in this context refers to the \textit{Hodinǫ̱hsǫ́:nih} in general.) The Six Nations Convention is a series of meetings performed during “…the biennial circuit which begins every fall at Tonawanda and which is observed in turn at each of the other … participating Longhouses …” in New York state, Quebec, and Ontario (\cite[192]{shimony_conservatism_1994}). The Chiefs’ Conventions, in contrast, are recitations of \textit{Gaihwí:yo:} sponsored by individual \textit{Ganǫhsé:sǫˀs} ‘Longhouses’ (singular \textit{Ganǫ́hse:s} ‘Longhouse’). Each event lasts for four days. It includes morning recitations of \textit{Gaihwí:yo:}, followed by afternoon sessions with elements such as \textit{Gaihwatéˀdǫh} ‘explanation, clarification’, \textit{Ga̱hsgyáǫnyǫhk} ‘words of encouragement’, and \textit{Adatrewáhdǫ:ˀ} ‘repentance, apology’. The orators who recite \textit{Gaihwí:yo:} perform these and other incredibly long speeches from memory, having learned this oral tradition directly from earlier preachers.

The \textit{Ganǫhsé:sǫˀs} referred to above are congregations of \textit{Godinǫ̱hsǫ́:nih} ‘Longhouse followers’ who live according to the moral and ethical teachings of \textit{Sganya\-dáiyoˀ}. The main components of the \textit{Hodinǫ̱hsó:nih} Longhouse tradition are \textit{Gaihwí:yo:}, the \textit{Geí Niyoihwá:ge:} ‘Four Ceremonies’, the \textit{Ga̱he:yǫ́ˀ Oíhwa̱ˀgeh} ‘funeral rites’, \textit{Gayęˀgǫtwę́:ˀ neˀ Onǫ́hgwa̱ˀtraˀ} ‘tobacco burnings and ceremonies for Medicine Societies’, and \textit{Ohwęjagehgehá:ˀ Gáęnaˀ} ‘Social dances and earth songs’.  

The \textit{Godinǫ̱hsǫ́:nih} also observe “doings” tied to the agricultural cycle, for example, the \textit{Adahyáohǫ:ˀ} ‘Gathering of Fruit’ or ‘Strawberry Ceremony’ and the \textit{Shadiyáǫdataˀ} ‘Dry Up the Trees’ Ceremony (done at the end of the maple sap run). The highlight of the year is the \textit{Tsaˀdego̱hsrahę́h Gaihwayáǫni:} ‘Midwinter Ceremony’, during which the \textit{Geí Niyoihwá:ge:} are performed. The different Longhouses at Six Nations put on \textit{Geí Niyoihwá:ge:}, not necessarily at the same time, but sometime during January or February of each year. 

Many occasions (including ceremonies, and even run-of-the-mill meetings) begin and end with \textit{Ganǫ́hǫnyǫh} or\textit{ Ganǫ́hǫnyǫhk} ‘The Thanksgiving Address’ (described in \cite{foster_earth_1974}), during which an orator takes between several minutes to several hours to thank the Creator for everything under the sun.


\section{Land}
\label{land}
The dispossession of the \textit{Hodinǫhsǫ́:nih} from their traditional lands before and after the American Revolution is well-documented elsewhere. Included in this section are simplified maps for Mohawk, Oneida, Onondaga, and Seneca migrations, based on the synopsis in \citet[3--6]{michelson_comparative_1988}, which will provide some notion of the changes from the original and present \textit{Hodinǫ̱hsǫ́:nih} lands. 

\begin{figure}
\includegraphics[width=.9\textwidth]{figures/Mohawk_Migrations.jpg}
\caption{Ganyęˀge̱hó:nǫˀ ‘Mohawk’ migrations}
\label{map:Mohawk.migrations}
\end{figure}


\begin{figure}
\includegraphics[width=.9\textwidth]{figures/Oneida_Migrations.jpg}
\caption{Ohnyahęhó:nǫˀ ‘Oneida’ migrations}
\label{map:Oneida.migrations}
\end{figure}

\begin{figure}
\includegraphics[width=.9\textwidth]{figures/Onondaga_Migrations.jpg}
\caption{Onǫda̱ˀgehó:nǫˀ ‘Onondaga’ migrations}
\label{map:Onondaga.migrations}
\end{figure}

\begin{figure}
\includegraphics[width=.9\textwidth]{figures/Seneca_Migrations.jpg}
\caption{Onǫdowáˀga:ˀ ‘Seneca’ migrations}
\label{map:Seneca.migrations}
\end{figure}

The \textit{Gayogo̱hó:nǫˀ} villages (and other \textit{Hodinǫ̱hsǫ́:nih} villages) were destroyed in the Sullivan Campaign of 1779, in retaliation for the \textit{Gayogo̱hó:nǫˀ} having sided with the British during the American Revolution. Many \textit{Gayogo̱hó:nǫˀ} then removed to Fort Niagara and ultimately to Six Nations of the Grand River. 

Before the American Revolution, some of the \textit{Gayogo̱hó:nǫˀ} had settled with a group of \textit{Onǫdowáˀga:ˀ} on the Lower Sandusky River in Ohio. They were collectively known as the ‘Sandusky Senecas’. In 1831, they moved to Miami, Oklahoma (\cite[5]{michelson_comparative_1988}, \cite[149]{mithun_iroquoian_1979}), which is shown as the ‘Seneca-Cayuga Tribe of Oklahoma’ in \figref{map:Cayuga.migrations} (page \pageref{map:Cayuga.migrations}). Differences between Oklahoma and Six Nations \textit{Gayogo̱honǫˀnéha:ˀ} ‘Cayuga language’ are summarized in \sectref{ch:Gayogohó:nǫˀ, the Cayuga language}.

\begin{figure}
\includegraphics[width=.9\textwidth]{figures/Cayuga_Migrations.jpg}
\caption{\textit{Gayogo̱hó:nǫˀ} ‘Cayuga’ migrations}
\label{map:Cayuga.migrations}
\end{figure}

The British Crown granted lands to the \textit{Hodinǫ̱hsǫ́:nih} in the 1784 Haldimand Proclamation. For a map showing the approximate dates and locations of the original \textit{Hodinǫ̱hsǫ́:nih} settlements on the lower Haldimand Tract, see \citet[140]{hill_clay_2017}. For a map of the original Haldimand Tract and the current extent of Six Nations, see \citep{six_nations_lands_and_resources_haldimand_2008}. 

See \figref{map:6nations.today} (page \pageref{map:6nations.today}) for  \textit{Gayogo̱honǫˀnéha:ˀ} place names at Six Nations and \figref{map:place.names} (page \pageref{map:place.names}) for \textit{Gayogo̱honǫˀnéha:ˀ} place names in Ontario and Quebec, Canada, and in New York State, USA.



\section{Gayogohonǫˀnéha:ˀ, the Cayuga language} \label{ch:Gayogohó:nǫˀ, the Cayuga language}

\textit{Gayogo̱honǫˀnéha:ˀ} belongs to the \textit{Ǫgwehǫwéhne̱ha:ˀ} ‘Iroquoian language’ family.\footnote{\textit{Gayogo̱hó:nǫˀ} refers to the people, and \textit{Gayogo̱honǫˀnéha:ˀ} ‘the Cayuga way’ refers to the language. However, \textit{Gayogo̱hó:nǫˀ} is often used in reference to the language.} It is closely related to  \textit{Onǫda’gegá’} ‘Onondaga’ and \textit{Onödowá’ga:’} ‘Seneca’ and more distantly, to \textit{Ska:rù:rę’} ‘Tuscarora’, \textit{Kanien’kéha} ‘Mohawk’, \textit{Onʌyotaʼa:ká:} ‘Oneida’, and \textit{Tsalagi} ‘Cherokee’.\footnote{The autonyms for each language are used in this paragraph. However, in the figures, the \textit{Gayogo̱honǫˀnéha:ˀ} words are used instead.} A simplified family tree is provided in \figref{map:Iroq.language.family}, page \pageref{map:Iroq.language.family}.

\begin{figure}
%\includegraphics[width=\textwidth]{figures/Iroquoian language family, simplified}
\small\resizebox{\textwidth}{!}{
\begin{forest}
for tree={grow'=east, forked edge, anchor=base west}
[Ǫgwehǫwéhne̱ha:ˀ
  [Otowe̱ˀgéha:ˀ
    [,shape=coordinate, l=2pt
      [,shape=coordinate, l=2pt
        [,shape=coordinate, l=2pt
          [Onǫdowáˀga:ˀ,tier=langgroup1]
          [Gayogo̱hó:nǫˀ,tier=langgroup1
            [Dagęhyatgehó:nǫˀ]
            [Ganedagehó:nǫˀ]
          ]
        ] [Onǫdagehó:nǫˀ, tier=langgroup1]
      ] [Ohnyahęhó:nǫˀ, tier=langgroup1] 
        [Ganyęˀge̱hó:nǫˀ, tier=langgroup1]
    ] [Dahsgáowęˀ, tier=langgroup1]
  ]
  [Onenǫ̱ˀgéha:ˀ [Tsalagi, tier=langgroup1]]
]
\end{forest}}
\caption{\textit{Ǫgwehǫwéhne̱ha:ˀ} language family (simplified). Based on \citet{schillaci_linguistic_2017}, \citet{eberhard_ethnologue_2022}. \textit{Otowe̱ˀgéha:ˀ} ‘Northern kind’, \textit{Onęnǫ̱ˀgéha:ˀ} ‘Southern kind’. Strictly speaking, many of the language names should end with \stem{-neha:ˀ}, without which, the words represent the names of peoples, not languages. The suffix has been omitted to save space.}
\label{map:Iroq.language.family}
\end{figure}

\textit{Ǫgwehǫ́:weh} ‘Iroquoian’ speakers can often speak or understand more than one language, at least with practice, because of similarities among the languages, which are illustrated in \tabref{figtab:1:twofive}, page \pageref{figtab:1:twofive}.

\begin{table}
\caption{Words for ‘two’ and ‘five’\label{figtab:1:twofive}}
\begin{tabular}{llll} 
\lsptoprule
Autonym & English name & `two' & `five'\\
\midrule
Gayogo̱honǫ\-ˀnéha:ˀ & Cayuga & tekni: & hwihs\\
Kanien’kéha & Mohawk & tékeni & wisk\\
Onʌyotaʼa:ká: & Oneida & tékni & wisk\\
Onǫda’gegá’ & Onondaga & teknih & hwiks\\
Onödowá’ga:’ & Seneca & tekni:h & wis\\
Ska:rù:rę’ & Tuscarora & ne:kti:h & wisk\\
Tsalagi & Cherokee & ta’li & hiski\\
\lspbottomrule
\end{tabular}
\end{table}

\textit{Gayogo̱ho:nǫˀnéha:} is currently spoken at Six Nations of the Grand River near Brantford, Ontario, Canada. At least in the past, the variety of \textit{Ganedagehonǫ́ˀ\-ne̱ha:ˀ} ‘Lower Cayuga’ was spoken in the Lower End (\textit{Ganédageh} ‘in the valley’), while \textit{Dagęhyatgehonǫ́ˀne̱ha:ˀ} ‘Upper Cayuga’ was spoken around the Upper End (\textit{Dagę́hya:t} ‘top of the mountain’).

\textit{Ganedagehonǫ́ˀne̱ha:ˀ} speakers use GY and KY where \textit{Dagęhyatgehonǫ́ˀne̱ha:ˀ} speakers use DY and TY instead \xref{ex:tydy}. \textit{Ganedagehonǫ́ˀne̱ha:ˀ} speakers also pronounce underlined vowels \xref{ex:tydya} differently than \textit{Dagęhyatgehonǫ́ˀne̱ha:ˀ} speakers do (\ref{ex:tydyb}, see \sectref{Sound changes in word formation}). 

\ea\label{ex:tydy}
\ea gyo̱hdǫ: ‘nine’ (\textit{Ganedagehonǫ́ˀne̱ha:ˀ})\label{ex:tydya}
\ex dyohdǫ: ‘nine’ (\textit{Dagęhyatgehonǫ́ˀne̱ha:ˀ})\label{ex:tydyb}
\z
\z

A variety of \textit{Gayogo̱ho:nǫˀnéha:}ˀ was also spoken in Oklahoma. The speech of one Oklahoman speaker is described in \citet{mithun_incipient_1989}. The main changes are summarized in this section. For the pronunciation of \textit{Ganedage\-honǫ́ˀ\-ne̱ha:ˀ} see \sectref{sounds and spelling}, \sectref{Sound changes in word formation}, \sectref{Sound changes - prepronominal prefix-pronominal combinations}, and \sectref{Sound changes determined by C, A, E, I, and O stems}.

Differences between Oklahoma and Six Nations \textit{Gayogo̱ho:nǫˀnéha:} include vocabulary items, as shown in \xref{ex:word.for.car}. 

\FloatBarrier
\ea \label{ex:word.for.car} The word for ‘car’ (\cite[247]{mithun_incipient_1989})
\ea gagá̱howanęˀs  (Oklahoma speaker)\\
\gll ga-gá̱h-owanę-ˀs\\
\textsc{3s.a}-eye-big.{\stat}-\plural\\
\ex ga̱ˀdréhdaˀ (Six Nations speakers)\\
\gll ga̱-ˀdréhd-aˀ\\
\textsc{3s.a}-drag-{\nounstemformer}\\
\z
\z

Noun incorporation is also different: Oklahoma \textit{Gayogo̱honǫˀnéha:ˀ} speakers tend to avoid pronominal prefixes denoting possession in noun incorporation words \xref{ex:big.housea}, and also tended to avoid incorporation  \xref{ex:big.oniona}.

\ea \label{ex:big.house}
‘She has a big house’ (\cite[249]{mithun_incipient_1989})
\ea ganǫhsuwá:nęh (Oklahoma speaker)\\\label{ex:big.housea}
\gll ga-nǫhs-uwá:n-ęh\\
\textsc{3s.a}-house-big-{\stative} \\
\ex gonǫhsowá:nęh (Six Nations speakers)\\\label{ex:big.houseb}
\gll go-nǫhs-owá:n-ęh\\
\textsc{3s.fi.p}-house-big-{\stative} \\
\z
\z

\ea \label{ex:big.onion} ‘She has a big onion’ (\cite[250]{mithun_incipient_1989})
\ea kuwá:nęh ˀnúhsaˀ (Oklahoma speaker)\\\label{ex:big.oniona}
\gll k-uwá:n-ęh ˀnúhs-aˀ\\
\textsc{3s.a}-big-{\stative} onion-{\nounstemformer} \\
\ex go̱ˀnǫhsowá:nęh  (Six Nations speakers)\\ \label{ex:big.onionb}
\gll go̱-ˀnǫhs-owá:n-ęh \\
\textsc{3s.fi.p}-onion-big-{\stative}\\ 
\z
\z

As shown in examples \xref{ex:big.oniona} and \xref{ex:beans}, Oklahoma \textit{Gayogo̱honǫˀnéha:ˀ} omitted most instances of the /o-/ \textsc{3s.p} prefix in nouns (\ref{ex:beansa}, \cite[251--252]{mithun_incipient_1989}), and shortened some words even further \xref{ex:buttera}, (but without changing accent placement).


\ea\label{ex:beans} ‘beans’ (\cite[254]{mithun_incipient_1989}) 
\ea\label{ex:beansa} sáheˀdaˀ (Oklahoma speaker)\\
\gll sáheˀd-aˀ\\
beans-{\nounstemformer}\\

\ex\label{ex:beansb} osáheˀdaˀ (Six Nations speakers)\\
\gll o-sáheˀd-aˀ\\
\textsc{3s.p}-beans-{\nounstemformer}\\
\z
\z

\ea \label{ex:butter} ‘butter’ (\cite[255]{mithun_incipient_1989})
\ea \label{ex:buttera}drá:htaˀ (Oklahoma speaker)\\
\gll drá:ht-aˀ\\
butter-{\nounstemformer}\\

\ex \label{ex:butterb}owidrá:htaˀ (Six Nations speakers)\\
\gll o-widrá:ht-aˀ\\
\textsc{3s.p}-butter-{\nounstemformer}\\
\z
\z

Words with an O-like sound in Six Nations \textit{Gayogo̱honǫˀnéha:ˀ} instead have a U-like sound in Oklahoma \textit{Gayogo̱honǫˀnéha:ˀ} \xref{ex:colda}. (In contrast, Six Nations \textit{Gayogo̱honǫˀnéha:ˀ} only has a few words with an U-like sound (see \sectref{vowel.pronunciation}). 

\ea \label{ex:cold} ‘it is cold’ (\cite[250--251]{mithun_incipient_1989})
\ea utú:weˀ  (Oklahoma speaker)\\\label{ex:colda}
\gll u-tú:weˀ\\
\textsc{3s.p}-cold.{\stative}\\
\ex otó:weˀ (Six Nations speakers)\\\label{ex:coldb}
\gll o-tó:weˀ\\
\textsc{3s.p}-cold.{\stative} \\
\z
\z

Six Nations and Oklahoma \textit{Gayogo̱honǫˀnéha:}ˀ had slightly different rules for accent placement and related sound changes. For example, the accent was changed for some Oklahoma words \xref{ex:eyea}. 

\ea\label{ex:eye} ‘in my eye’ (\cite[252]{mithun_incipient_1989})
\ea gegahá:gǫ: (Oklahoma speaker)\\\label{ex:eyea}
\gll g-e-gah-á:gǫ:\\
\exsc{\textsc{1s.a}}-{\joinerE}-eye-in\\

\ex\label{ex:eyeb}
gegáhagǫ:  (Six Nations speakers)\\
\gll g-e-gáh-agǫ:\\
\exsc{\textsc{1s.a}}-{\joinerE}-eye-in \\
\z
\z

Also, while whisper (devoicing) is obligatory for the first vowel of the word in \xref{ex:handa} in \textit{Ganedagehonǫ́ˀne̱ha:ˀ},\footnote{In \textit{Ganedagehonǫ́ˀne̱ha:ˀ} the first vowel is devoiced in words beginning with a consonant-vowel-H combination.}, it was optional in Oklahoma \textit{Gayogo̱ho:\-nǫhnéha:ˀ} \xref{ex:handb}. (Laryngealization, the sound change shown in the third vowel of both examples in \xref{ex:hand}, is obligatory in both \textit{Ganedagehonǫ́ˀne̱ha:ˀ} and in Oklahoma \textit{Gayogo̱honǫˀnéha:ˀ}, \cite[254]{mithun_incipient_1989}).

\ea \label{ex:hand} ‘on my hand’ (\cite[253]{mithun_incipient_1989})
\ea ge̱hsóhda̱ˀgeh (Six Nations speakers)\\\label{ex:handa}
\gll g-e̱-hsóhd-a̱-ˀgeh\\
\exsc{\textsc{1s.a}}-{\joinerE}-hand-{\joinerA}-{\on}\\

\ex gehsúhda̱ˀgeh (Oklahoma speaker)\\\label{ex:handb}
\gll g-e-hsúhd-a̱-ˀgeh\\
\exsc{\textsc{1s.a}}-{\joinerE}-hand-{\joinerA}-{\on}\\
\z
\z

Similarly, devoicing does not affect the first vowel of the word in \textit{Ganedage\-honǫ́ne̱ha:ˀ} \xref{ex:threea}, but could do so in Oklahoma \textit{Gayogo̱honǫˀnéha:}ˀ \xref{ex:threeb}.\footnote{In \textit{Ganedagehonǫ́ˀne̱ha:ˀ}, words beginning with vowels are not devoiced.}

\ea \label{ex:three} ‘three’ (\cite[254]{mithun_incipient_1989})
\ea ahsǫ́h (Six Nations)\label{ex:threea} 
\ex a̱hsųh (Oklahoma)\label{ex:threeb} 
\z
\z
