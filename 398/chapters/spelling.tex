\chapter{Sounds and spelling} \label{sounds and spelling}
\section{Vowels and consonants}
Gayogo̱honǫˀnéha:ˀ has seven vowels (with underlyingly long counterparts, see \sectref{longvsshortvowels})  and thirteen consonants.

\begin{table}
\begin{floatrow}
\captionsetup{margin=.05\linewidth}
\ttabbox
{\begin{tabularx}{\linewidth}{XXXXX}
\lsptoprule
i(:) & & & & u(:)\\
e(:) & ę(:) & & o(:) & ǫ(:)\\
 & & a(:) & & \\
\lspbottomrule
\end{tabularx}}
{\caption{Vowels}\label{figtab:1:vowels}}%
\ttabbox
{\begin{tabularx}{\linewidth}{XXXX}
\lsptoprule
t,d & & k,g & ˀ\\
s &  & &  h\\
ts & j &  & \\
n & & &  \\
r & & &  \\
 &  y & w & \\
\lspbottomrule
\end{tabularx}}
{\caption{Consonants}
\label{figtab:1:consonants}}
\end{floatrow}
\end{table}


The following descriptions use symbols from the International Phonetic Alphabet (IPA) for clarity. IPA letters are in square brackets.

\section{Vowel pronunciation} \label{vowel.pronunciation}

The Gayogo̱honǫˀnéha:ˀ vowel letters represent just one sound each. That being said,  vowels can also be modified – lengthened, whispered (devoiced), or glottalized (produced with creaky voice) –  depending on factors described in \sectref{ch:Pronunciation changes related to the syllable count}.

\begin{description}[font=\normalfont]
\item[I] sounds like the I \phonet{i} in \textit{magazine}. 

\item[E] sounds like the E \phonet{e} in \textit{hey}.

\item[A] sounds like the A \phonet{ɑ} in \textit{father}.

\item[O] sounds like the O \phonet{o} in \textit{so}.

\item[U] sounds like the U \phonet{u} in \textit{blue}. Only a few words have this sound, including, \textit{niwú:ˀuh} ‘it is small’, \textit{niwuˀdrugye:ˀah} ‘it is narrow’, and \textit{kyohsaˀgeh, kyuhsaˀgeh} ‘on my elbow’. 

\item[Ę] sounds like the nasal E \phonet{ɛ̃} in English \textit{men} or the nasal vowel in French \textit{frein} ‘brake’. It can also sound more like the nasal O \phonet{ʌ̃} in \textit{money} or like the U in \textit{pun}. Long Ę: can sound like \phonet{ɛ̃n}, as in \textit{nę́:dah} \phonet{nɛ̃́ndah} ‘hear, take it!’ This happens especially when long Ę: is before D, G, J, K, T, or TS.

\item[Ǫ] sounds like the nasal O \phonet{õ} in English \textit{known} or the nasal vowel in French \textit{don} ‘gift’. It can also sound more like the nasal OO \phonet{ũ} in \textit{noon}. Long Ǫ: can sound like \phonet{õn}, as in \textit{ǫ́:dǫh} \phonet{ṍndõh} ‘she says’. This happens especially when long Ǫ: is before D, G, J, K, T, or TS.

\end{description}


\subsection{Long versus short vowels}\label{longvsshortvowels}

Gayogo̱honǫˀnéha:ˀ distinguishes between long and short vowels. Long vowels are nearly always long. 

\ea\label{long.vowels} Long vowels
\ea onǫ́ˀ\exemph{a:}ˀ ‘a head’
\cfex{o-nǫ́ˀa:-ˀ \textsc{3.p}-head-{\nounstemformer}}
\ex sanǫˀ\exemph{á:}ˀgeh ‘on your head’
\cfex{sa-nǫˀá:-ˀgeh \textsc{2s.a}-head-{\on}}
\ex honǫˀ\exemph{á:}dę:s ‘he’s thick-headed’
\cfex{ho-nǫˀá:-dę:s \textsc{3s.m.p}-head-thick.\stative}
\z
\z 

Long vowels are only shortened when another vowel follows.

\ea\label{shortening.long.vowels} Long vowel shortening\\
ęgatnǫ̱ˀ\exemph{a}oháiˀ ‘I will wash my head’
\cfex{ę-g-at-nǫ̱ˀa-ohái-ˀ {\future}-\textsc{1s.a}-{\semireflexive}-head-wash-{\punctual}}
\z


In contrast, short vowels are typically short.

\ea\label{short.vowels} Short vowels\\
ag\exemph{a}t\exemph{ǫ}dę́ˀǫh ‘I have heard it before’
\cfex{ag-at-hǫdę́ˀ-ǫh \textsc{1s.p}-{\semireflexive}-hear-{\stative}}
\z 


However, short vowels can be lengthened according to rules described in \sectref{Non-final accent and lengthening}.

\ea\label{lengthening.short.vowels} Short vowel lengthening\\
ag\exemph{a:}t\exemph{ǫ́:}deˀ ‘I hear it (right now)’
\cfex{ag-a:t-hǫ́:de-ˀ \textsc{1s.p}-{\semireflexive}-hear-{\stative}}
\z 
 

\section{Consonant pronunciation}
The consonant letters often represent several sounds. This is explained below and in \sectref{syllable.structure} (also see \tabref{figtab:1:consspell}, page \pageref{figtab:1:consspell}.)

\begin{description}[font=\normalfont]
\item[T] sounds like the T \phonet{tʰ} in \textit{ten}. T either syllabifies as two separate consonants \phonet{t.h} or as a single consonant \phonet{tʰ}.

\item[D] sounds like the D \phonet{d} in \textit{den}. In whispered syllables, it also sounds like T [t] (see \sectref{ch:Pronunciation changes related to the syllable count}). D syllabifies as a single consonant \phonet{d}. 

\item[K] sounds like the K \phonet{kʰ} in \textit{Ken}. K either syllabifies as two separate consonants \phonet{k.h} or as a single consonant \phonet{kʰ}.

\item[G] sounds like the hard G \phonet{g} in \textit{gum}. In whispered syllables, it also sounds like K \phonet{k} (see \sectref{ch:Pronunciation changes related to the syllable count}). G syllabifies as a single consonant \phonet{g}. In Gayogo̱honǫˀnéha:ˀ, G is never pronounced like the “soft G” in \textit{gem}.

\item[S] sounds like the S \phonet{s} in \textit{sun}. In some words, it sounds like the Z \phonet{z} in \textit{zoo}, for example in \textit{í:soˀ} ‘many, much’. S syllabifies as a single consonant \phonet{s, z} or as part of another consonant (as in \phonet{kˢ}).

\item[SR] – as in \textit{wę̱hnihSRí:yo:} ‘nice day’ –  sounds like the SHR \phonet{ʃɹ} in \textit{shrink}. Some speakers pronounce SR as FR \phonet{fɹ} instead, for example in words like \textit{ganǫ́hkwasraˀ} (ganǫ́hkwaFRaˀ) ‘love’. SR syllabifies as two separate consonants, \phonet{ʃ.ɹ} or \phonet{f.ɹ}.

\item[S] sometimes represents two sounds, S and H, as in \textit{less heat}. For example, the word meaning ‘that’ is often spelled as \textit{sęh} \phonet{sʰɛ̃h} instead of \textit{shęh}. To more closely mirror the pronunciation, the SH spelling is used in morpheme breaks in this work, as in \textit{shęh} ‘that’.\footnote{Morpheme breaks show a word’s prefixes, stem, and suffixes.} SH either syllabifies as two separate consonants \phonet{s.h} or as a single consonant \phonet{sʰ}.

\item[TS] sounds like T, S, and H, as in \textit{let’s hide}. TS either syllabifies as two separate consonants \phonet{t.sʰ} or as a single consonant \phonet{tˢʰ}.

\item[J] sounds like the J \phonet{dʒ} sound in \textit{judge} or like the DZ \phonet{ʣ} in \textit{adze}  or \textit{gadzooks}. In whispered syllables (see \sectref{ch:Pronunciation changes related to the syllable count}), it sounds like the CH \phonet{ʧ} in \textit{chair}, or like \phonet{ts} as in \textit{let’s}. J either syllabifies as two separate consonants \phonet{d.ʒ} or \phonet{d.z} or as a single consonant, \phonet{dʒ, dz}.

\item[N] sounds like the N \phonet{n} in \textit{no}. In whispered syllables (see \sectref{ch:Pronunciation changes related to the syllable count}), it sounds like the N (devoiced \phonet{n̥}) in \textit{snore}. N syllabifies as a single consonant.

\item[R] sounds like the R \phonet{ɹ} in \textit{raw}. In whispered syllables (see \sectref{ch:Pronunciation changes related to the syllable count}), it sounds like the R (devoiced \phonet{ɹ̥}) in \textit{three}. R syllabifies as a single consonant, but is always preceded by another consonant such as \phonet{t,d,k,g,s,n}.

\item[W] sounds like the W \phonet{w} in \textit{we}. In whispered syllables (see \sectref{ch:Pronunciation changes related to the syllable count}), it sounds like the W (devoiced \phonet{w̥}) in \textit{sway}. W syllabifies as a single consonant \phonet{w} or as part of another consonant (such as \phonet{dʷ, gʷ}).

\item[Y] sounds like the Y \phonet{j} in \textit{you}. In whispered syllables (see \sectref{ch:Pronunciation changes related to the syllable count}), it sounds like the Y sound (devoiced \phonet{j̥}) that can be heard (but is not spelled) after F in \textit{few}, or after P in \textit{pure}. Y syllabifies as a single consonant \phonet{j} or as part of another consonant (such as \phonet{dʲ, gʲ}).

\item[H] sounds like the H in \textit{he}. For syllabification purposes, when H is between two vowels (syllables), it is part of both the first syllable (vowel) and the second syllable (vowel).\footnote{Technically speaking, intervocalic H and intervocalic glottal stop (described next) are ambisyllabic.} Examples are provided in the following sections.

\item[H] sounds are heard but not spelled in letters like T \phonet{th, tʰ}, K \phonet{kh, kʰ}, TS \phonet{tˢʰ} and S (when S spells \phonet{sʰ}).

\item[<ˀ>] (glottal stop)\footnote{Gayogo̱honǫˀnéha:ˀ language teachers refer to glottal stop <ˀ> as a “slow marker”.} is a true consonant in Gayogo̱honǫˀnéha:ˀ (Although itis not spelled in English, glottal stop can be heard in several English expressions, including  uh-uh \phonet{ˈʔʌʔʌ} ‘no’). For syllabification purposes, when <ˀ> occurs between two vowels (syllables), it is part of both the first syllable (vowel) and the second syllable (vowel). Examples are provided in the following sections.
\end{description}


\section{Alphabetization}

In this work, words are alphabetized as in the roman alphabet with a few exceptions: alphabetization ignores the glottal stop <ˀ> and the lengthener < : >. For example, items beginning with \phonet{ˀni} or \phonet{ni} are both listed under NI. Similarly, \phonet{aˀa} is alphabetized as AA, and \phonet{a:} is treated as A. Alphabetization also ignores the distinction between E and Ę, and between O and Ǫ. For example, items beginning with \phonet{od} or \phonet{ǫd} are alphabetized under OD.


\section{Accent and related pronunciation changes}
The unique rhythm of Gayogo̱honǫˀnéha:ˀ arises from a combination of accent, vowel lengthening and syllable shortening (described in the following sections). Accented vowels have a higher pitch (musical tone) than unaccented vowels, and appear near the end of the word. They are spelled with an acute accent mark <´>\footnote{Gayogo̱honǫˀnéha:ˀ language teachers also call the accent mark a “stress marker” or “stress point”.}. Lengthened or long vowels take twice as long to produce as short vowels. Vowel lengthening (or length) is spelled with a colon <:>.\footnote{Gayogo̱honǫˀnéha:ˀ language teachers refer to the colon < : > as a “lengthener”.} Syllable shortening is a cover term for a set of pronunciation changes, described later, which affects certain syllables ending with H or glottal stop <ˀ>. 

Accent is either \textsc{final} or \textsc{non-final}. Final accent falls on the last syllable (vowel) of the word and non-final accent falls on the second-last, third-last, or fourth-last syllable. Final accent is described in \sectref{ch:Pronunciation changes in sentences}. (Briefly stated, non-final accent switches to final accent under certain conditions.) Non-final accent placement and the related rules of vowel lengthening and syllable shortening are described below. All of these rules refer to the syllable count, described next.

\subsection{Counting syllables (or vowels)}
\label{counting.syllables}
The rules of accent placement, vowel lengthening, and syllable shortening make reference to two types of syllable count \xref{ex:syllcountex1}. (Since every syllable includes a vowel, the words \textit{vowel} and \textit{syllable} are used interchangeably in this section. For syllable structure, see \sectref{syllable.structure}.)

\ea\label{ex:syllcountex1} Two types of syllable count
\ea \textsc{the odd/even count}: starting from the beginning of the word, vowels (syllables) are either odd-numbered (first, third, etc.) or even-numbered (second, fourth, etc.).\label{ex:syllcountex1a}
\ea The odd/even count is irrelevant for final syllables.
\z
\ex \textsc{the word-final count}: starting from the end of the word, vowels (syllables) are either:\label{ex:syllcountex1b}
\ea final\label{ex:syllcountex1c}
\ex second-last\label{ex:syllcountex1d}
\ex third-last, or\label{ex:syllcountex1e}
\ex fourth-last.\label{ex:syllcountex1f}
\z
\z
\z

The second-last syllable is also called the \textsc{penultimate} vowel/syllable or the \textsc{penult}.

The odd/even count starts at the beginning of the word, as shown with a rightwards arrow ‘→’ in \figref{figtab:1:syllcount}. The word-final count starts from the end of the word instead, as shown with a leftwards arrow ‘←’.

\begin{figure}
\caption{The syllable count - \textit{ęsadejęhęné:dahk} ‘you will prepare your fire’ }
\label{figtab:1:syllcount}
\begin{tabularx}{\textwidth}{XXXXXXXXX} 
\lsptoprule
& ę & sa & de & ję & hę & né: & dahk & \\
\midrule

 → & 1 & 2 & 3 & 4 & 5 & 6 & 7 & \\
 \midrule
&  &  &  & 4th-last & 3rd-last & 2nd-last

(penult) & last & ←\\
\lspbottomrule
\end{tabularx}
\end{figure}

In most cases, long vowels count as one unit for the odd/even count \xref{ex:syllcountex1a}. However, the long \stem{a:-} {\indefinite} prefix always counts as two (\figref{figtab:1:syllindef}), as do underlyingly (or usually) long vowels, \figref{figtab:1:longvsyll} (page \pageref{figtab:1:longvsyll}).

\begin{figure}
\caption{The syllable count and the \stem{a:-} {\indefinite}. \textit{a:yetsę́iˀ} ‘she might find it’}
\label{figtab:1:syllindef}
\begin{tabularx}{\textwidth}{XXXXXXX} 
\lsptoprule
& a & : & yet & sę́ & iˀ & \\
\midrule

 → & 1 & 2 & 3 & 4 & 5 & \\
 \midrule
&  &  &  & 2nd-last 

(penult) & last & ←\\
\lspbottomrule
\end{tabularx}
\end{figure}

\begin{figure}
\caption{The syllable count and underlying long vowels - \textit{aknǫˀá:nǫ̱hwa:s} ‘I have a headache’}
\label{figtab:1:longvsyll}
\begin{tabularx}{\textwidth}{XXXXXXXX}
\lsptoprule
& ak & nǫˀ & á & : & nǫ̱h & wa:s & \\
\midrule
 → & 1 & 2 & 3 & 4 & 5 & 6 & \\
 \midrule
&  &  &  &  & 2nd-last 

(penult) & last & ←\\
\lspbottomrule
\end{tabularx}
\end{figure}


\section{Non-final accent and lengthening} \label{Non-final accent and lengthening}
Non-final accent and lengthening are determined by making reference to the odd/even and word-final counts described in \sectref{counting.syllables}. In a nutshell, even-numbered penults are always accented (and lengthened, if possible). In contrast, odd-num\-bered penults are only sometimes accented (and lengthened, if possible). More often than not, odd-numbered penults cannot be accented (or lengthened) and the third-last vowel is accented instead.

The phrase \textit{lengthen (if possible)} refers to the condition in \xref{vowellengthcondition}. Examples will be provided below.


\ea\label{vowellengthcondition} \textsc{condition on vowel lengthening}: vowels cannot be lengthened if they are followed by H, glottal stop <ˀ>, or another vowel. 
\z


\subsection{Accenting and lengthening even-numbered penults}
Even-numbered penults are always accented \xref{ex:nonfinacex}. They are also lengthened (if possible, \ref{ex:nonfinacexa}). The even-numbered penults in \xref{ex:nonfinacexb} and \xref{ex:nonfinacexc} are accented but cannot be lengthened because they are followed by H \xref{ex:nonfinacexb} or glottal stop <ˀ> \xref{ex:nonfinacexc}.

\ea\label{ex:nonfinacex} accent and lengthening of even-numbered, second-last vowels
\ea\label{ex:nonfinacexa} de̱hǫgwe̱ˀdí:yo: (de̱h-hǫg-we̱ˀ-\exemph{dí:}-yo:) ‘he is not a nice man’ (lengthened)
\ex\label{ex:nonfinacexb} taˀdewagéhdǫ:ˀ (taˀ-de-wa-\exemph{géh}-dǫ:ˀ) ‘I did not mean or intend it’ (not lengthened, because followed by H)
\ex\label{ex:nonfinacexc} ęyóˀga:ˀ (ę-\exemph{yóˀ}-ga:ˀ) ‘it will be night’ (not lengthened, because followed by <ˀ>)
\z
\z

\subsection{Accenting and lengthening odd-numbered penults}
Odd-numbered penults are sometimes accented but more often than not, the third-last vowel must be accented instead.

Odd-numbered penults followed by a single consonant are accented and also lengthened (if possible, \ref{ex:nonfinacex6}, \ref{ex:nonfinacex5}). When odd-numbered penults are lengthened, the even-numbered, third-last vowel is also lengthened (if possible,  \ref{ex:nonfinacex6}). Example \xref{ex:nonfinacex5} shows a case where the even-numbered third-last vowel cannot be lengthened because it is followed by glottal stop <ˀ>.

\ea\label{ex:nonfinacex6} Accented odd-numbered penult, followed by a single consonant (third-last vowel is also lengthened)
\ea aga:tǫ́:deˀ (a-\exemph{ga:t-hǫ́:}-deˀ) ‘I heard it’ 
\z
\z 

\ea\label{ex:nonfinacex5} Accented odd-numbered penult, followed by a single consonant (third-last vowel is not lengthened)
\ea ęhsnaˀjó:dęˀ (ęhs-\exemph{naˀ-jó:}-dęˀ) ‘you will boil it in a pot’
\z
\z 


Odd-numbered penults with the vowel \textit{A} cannot be accented \xref{ex:nonfinacex2a}. (An asterisk denotes an ungrammatical form.) The even-numbered, third-last vowel is accented instead \xref{ex:nonfinacex2b}. Neither vowel is lengthened \xref{ex:nonfinacex2b}. 

\ea\label{ex:nonfinacex2} Unaccented odd-numbered penult with the vowel \textit{A}
\ea\label{ex:nonfinacex2a} \ding{55} *asatgǫ̱hsoháeˀ (a-sat-gǫ̱h-soh-\exemph{há}-eˀ)
\ex\label{ex:nonfinacex2b} \ding{51} asatgǫ̱hsóhaeˀ (a-sat-gǫ̱h-\exemph{sóh}-ha-eˀ) ‘you have washed your face’\footnote{For the doubled H and other details about syllabification, see \sectref{syllable.structure}.}
\z
\z 

Odd-numbered penults followed by two or more consonants cannot be accented \xref{ex:nonfinacex3a}. An even-numbered, third-last vowel is accented instead \xref{ex:nonfinacex3b}. Neither vowel is lengthened \xref{ex:nonfinacex3b}. 

\ea\label{ex:nonfinacex3} Unaccented odd-numbered penult followed by two or more consonants
\ea\label{ex:nonfinacex3a} \ding{55} *gatgęhétsaˀ (gat-gęh-\exemph{hét}-sʰaˀ)
\ex\label{ex:nonfinacex3b} \ding{51} gatgę́hetsaˀ (gat-\exemph{gę́h}-het-sʰaˀ) ‘a handle’ 
\z
\z

The accent rule illustrated in \xref{ex:nonfinacex3} also applies to odd-numbered penults followed by T \phonet{th}, K \phonet{kh}, and J \phonet{dz, tʃ}, which represent two consonants \xref{ex:nonfinacex4}.

\ea\label{ex:nonfinacex4} Unaccented odd-numbered penult followed by T, K, J… (representing two or more consonants)
\ea \ding{55} *hoyanétaˀ (ho-ya-\exemph{nét}-haˀ) 
\ex \ding{51} hoyánetaˀ (ho-\exemph{yá}-net-haˀ) ‘the chief’s clan mother’
\z
\z 

Finally, odd-numbered penults followed by H or glottal stop <ˀ> are not accented \xref{ex:newnonfinacex1a}. The third-last vowel is accented instead \xref{ex:newnonfinacex1b}. Neither vowel is lengthened \xref{ex:newnonfinacex1b}. 

\ea\label{ex:newnonfinacex1} 
\ea\label{ex:newnonfinacex1a} \ding{55} *Gayogáhneh (Ga-yo-\exemph{gáh}-neh)
\ex\label{ex:newnonfinacex1b} \ding{51} Gayóga̱hneh (Ga-\exemph{yó}-ga̱h-neh) ‘Cayuga, Ontario’
\z 
\z

\section{Accent shift}
A non-final accent can shift from the predicted vowel (V2) to a preceding vowel (V1) if there is no consonant between V1 and V2. For example, in \xref{ex:nonfinacex7a}, the even-numbered penult O should be accented. However, it is immediately preceded by A, which is accented instead \xref{ex:nonfinacex7b}. 


\ea\label{ex:nonfinacex7} Accent shift from penult to third-last vowel
\ea \ding{55} *sayaˀdodrǫhgwaónihs (sa-yaˀ-dod-rǫhg-\exemph{wa-ó}-nihs\label{ex:nonfinacex7a}
\ex \ding{51} sayaˀdodrǫhgwáonihs (sa-yaˀ-dod-rǫhg-\exemph{wá-o}-nihs) ‘you are always shivering’ \label{ex:nonfinacex7b}
\z
\z 

In \xref{ex:nonfinacex700a}, the odd-numbered penult cannot be accented and so the even-numbered third-last vowel should be accented instead. However, it is immediately preceded by A, which ultimately receives the accent because of accent shift \xref{ex:nonfinacex700b}. This is the only instance where accent will fall on the fourth-last vowel.

\ea\label{ex:nonfinacex700} Accent shift from third-last to fourth-last vowel vowel
\ea \ding{55} *wanahaótrahǫˀ (wa-nah-\exemph{ha-ót}-rah-hǫˀ \label{ex:nonfinacex700a}
\ex \ding{51} wanaháotrahǫˀ (wa-nah-\exemph{há-ot}-rah-hǫˀ) ‘hats lying on the ground’ (\cite[225]{mithun_watewayestanih_1984}, Agyaˀdawíˀtraˀ dialogue)\label{ex:nonfinacex700b} 
\z
\z 


\section{Exceptions to non-final accent placement rules} \label{ch:Exceptions to non-final accent placement rules}
Several exceptions to the rules of non-final accent placement (\sectref{Non-final accent and lengthening}) are described in this and the following section. 

The vowels before certain endings are always accented and long. The relevant endings are the \stem{-kˀah} \textsc{beside} locative suffix, and the \stem{-:ˀah, -:ˀǫh} and \stem{-:hah} {\diminutive} suffixes \xref{ex:nonfinacex12}.

\ea\label{ex:nonfinacex12}
\ea ganǫhsá:kˀah (ga-nǫh-\exemph{sá:k}-ˀah) ‘beside the house’ 
\cfex{ganǫ́hsaˀ (ga-nǫ́h-saˀ) ‘house’}
\ex ohsno:wé:ˀah (ohs-no:-\exemph{wé:ˀ}-ˀah) ‘it is fairly fast’, ‘to arrive a little bit early’
\cfex{ohsnó:weˀ (ohs-nó:-weˀ) ‘it is fast, quick’}
\ex gowa:nę́:hah (go-wa:-\exemph{nę́:h}-hah) ‘it is fairly big’ 
\cfex{gowá:nęh (go-wá:-nęh) ‘it is big’}
\z
\z

A number of words have underlyingly (typically) long vowels (see \sectref{longvsshortvowels}). If the long vowel falls on the penult, it is accented -- even if it is odd-numbered \xref{ex:nonfinacex13}.

\ea\label{ex:nonfinacex13} 
\ea knó:haˀ (\exemph{knó:h}-haˀ) ‘my mother’
\ex kré:haˀ (\exemph{kré:h}-haˀ) ‘I am setting it on something’ 
\z
\z

Contrary to the rules of non-final accent placement (\sectref{Non-final accent and lengthening}, example \ref{ex:nonfinacex2}), an odd-numbered penultimate A \emph{can} be accented and lengthened if it is followed by a single W or Y \xref{ex:nonfinacex14}.

\ea\label{ex:nonfinacex14}
\ea gá:yęˀ (\exemph{gá:}-yęˀ) ‘it is lying there’
\ex sá:węh (\exemph{sá:}-węh) ‘it is yours’ 
\ex hadihá:wiˀs (ha-dih-\exemph{há:}-wiˀs) ‘they carry along’
\z
\z 

The initial \stem{o-} \textsc{3s.p} prefix of some nouns can optionally be deleted. However, the noun is still accented as if the prefix were present. (Also see the Oklahoma Gayogo̱honǫˀnéha:ˀ examples in \sectref{ch:Gayogohó:nǫˀ, the Cayuga language}.)

\ea\label{ex:nonfinacex15}
\ea ohǫ́na̱ˀdaˀ, \exemph{hǫ́na̱ˀdaˀ} ‘potatoes’
\ex ohsháihsdaˀ, \exemph{sháihsdaˀ} ‘snake’
\z
\z 

The \stem{aˀ-} {\factual} prefix is optionally deleted if the following vowel is A (see \sectref{Pronominal prefix pronunciation}). However, the resulting word is still accented and lengthened as if the prefix were present. Example \xref{ex:nonfinacex16a} illustrates a word with the factual prefix. The odd-numbered penult is accented and lengthened, and the third-last vowel is also lengthened (\sectref{Non-final accent and lengthening}, example \ref{ex:nonfinacex6}). Example \xref{ex:nonfinacex16b} shows the same word without a factual prefix. It is still accented and lengthened as if it had an odd-numbered penult. For comparison, the ungrammatical example in \xref{ex:nonfinacex16c} illustrates the accent and lengthening that would be predicted if the word had an even-numbered penult. 

\ea\label{ex:nonfinacex16}
\ea \ding{51} \exemph{aˀ}a:kní:gęˀ ‘she or he and I saw it’ \label{ex:nonfinacex16a}
\cfex{aˀ-a:kní:-gę-ˀ {\factual}-\textsc{1d.ex.a}-see-{\punctual}} 
\ex \ding{51} a:kní:gęˀ ‘she or he and I saw it’ \label{ex:nonfinacex16b}
\cfex{a:kní:-gę-ˀ \textsc{deleted.factual}-\textsc{1d.ex.a}-see-{\punctual}} 
\ex \ding{55} *akní:-gę-ˀ *\textsc{1d.ex.a}-see-{\punctual} \label{ex:nonfinacex16c}
\z
\z

\subsection{Accenting short words}
The non-final-accent rules described in \sectref{Non-final accent and lengthening} do not necessarily apply to short words with either one or two syllables. Accenting \emph{two}-vowel words is described in this section. See \sectref{ch:Accenting particles and particle groups in sentences} for accenting particles or one-vowel words. 

Recall than in general, if an odd-numbered penult cannot be accented, the third-last vowel is accented instead (\sectref{Non-final accent and lengthening}, examples \ref{ex:nonfinacex2}-\ref{ex:newnonfinacex1}). However, while two-vowel words have an odd-numbered penult, they do not have a third-last vowel by definition, and so many two-vowel words cannot be accented according to the rules in \sectref{Non-final accent and lengthening}. For example, the first vowel in the word in \figref{figtab:1:shortwdsyll} is an odd-numbered penult. Since it is an A (and is also followed by H), it cannot be accented. Meanwhile, the accent cannot shift further leftwards, as happens in longer words, because the odd-numbered, second-last vowel is also the first vowel of the word. 

\begin{figure}
\caption{The syllable count with short words - \textit{hahdo:s, hahdó:s} ‘he dives’}
\label{figtab:1:shortwdsyll}
\begin{tabularx}{\textwidth}{XXXX} 
\lsptoprule
& hah & do:s & \\
\midrule
 → & 1 & 2 & \\
& 2nd-last & last & ←\\
\lspbottomrule
\end{tabularx}
\end{figure}

These two-vowel words are sometimes pronounced without an accent – that is, with a low pitch on both vowels. Alternatively, they can be pronounced with a final accent \xref{ex:nonfinacex9}.

\FloatBarrier
\ea\label{ex:nonfinacex9}
\ea hahdo:s, hahdó:s (hah-do:s, hah-dó:s) ‘he dives’
\ex ohyaˀ, ohyáˀ (oh-yaˀ, oh-yáˀ) ‘berry, fruit ‘
\ex dasha:, dashá: (das-ha:, das-há:) ‘here, take this!’ 
\z
\z 


\section{Pronunciation changes related to the syllable count} \label{ch:Pronunciation changes related to the syllable count}
In \textit{Ganedagehonǫ́ˀne̱ha:ˀ} (Lower Cayuga), odd-numbered syllables ending with H or glottal stop <ˀ> are pronounced differently than their even-numbered counterparts.\footnote{The pronunciation changes are collectively known as \textsc{laryngeal metathesis}, and are described in \citet{foster_alternating_1982} and \citet{doherty_acoustic-phonetic_1993}.} (The best way to understand the following description is to listen to a speaker pronouncing all of the words in the following examples.) 

Odd-numbered syllables ending with H are often whispered (\ref{ex:lmex}-\ref{ex:lmex4}). (Technically, the entire syllable is \textsc{devoiced}.) Vowel underlining identifies the relevant syllables, which often sound as if they have been shortened to a single consonant. For example, the syllables \textit{dǫ̱h} and \textit{de̱h} can simply sound like a T \xxref{ex:lmex}{ex:lmex2}. 

\ea\label{ex:lmex} 
ehyádǫ̱hkwaˀ (eh-yá-\exemph{dǫh}-kwaˀ) ‘pencil’ (sounds like \textit{ehyáTkwaˀ})
\cfex{ehyá:dǫh (eh-yá:-\exemph{dǫh}) ‘she writes’ (all three sounds in \textit{dǫh} are clearly audible)}
\z

\ea\label{ex:lmex2} 
de̱hęnaǫháˀ (\exemph{de̱h}-hę-ná-ǫh-haˀ) ‘they (m) race’ (sounds like \textit{Tęnáǫhaˀ})
\cfex{degęnáǫhaˀ (\exemph{de}-gę-ná-ǫh-haˀ) ‘they (animals) are racers’  (both sounds in \textit{de} are clearly audible)}
\z

Whispered vowels are sometimes hard to hear, especially in the first syllable in a word \xxref{ex:lmex3}{ex:lmex4}.

\ea\label{ex:lmex3} 
gǫ̱hswáhęhs (\exemph{gǫ̱hs}-wáh-hęhs) ‘I hate you’ (sounds like \textit{KHswáhęhs})
\cfex{gǫnǫ́hweˀs (\exemph{gǫ}-nǫ́h-weˀs) ‘I like you’ (both sounds in \textit{gǫ} are clearly audible)}
\z


\ea\label{ex:lmex4} 
sa̱hsgá:ne:s (\exemph{sa̱hs}-gá:-ne:s) ‘you long for something’ (sounds like \textit{SSgá:ne:s})
\cfex{hahsgá:ne:s (\exemph{hahs}-gá:-ne:s) ‘he longs for something’ (all sounds in \textit{hahs} are clearly audible)}
\z


Odd-numbered syllables ending with glottal stop <ˀ> often sound quite different from their even-numbered counterparts. (Technically, the entire syllable is \textsc{creaky-voiced} or \textsc{glottalized}.) The pronunciation changes are described next. 

First, the glottal stop <ˀ> sound might disappear \xref{ex:lmex12}.

\ea\label{ex:lmex12}
ga̱ˀdréhdaˀ (\exemph{ga̱ˀd}-réh-daˀ) ‘car, vehicle’ (odd-numbered \textit{ga̱ˀ} sounds more like \textit{ga}. It is hard to hear any \phonet{ˀ} sound.)
\cfex{agéˀdre̱hdaˀ (a-\exemph{géˀd}-re̱h-daˀ) ‘my car, vehicle’ (in even-numbered \textit{géˀ},  the <ˀ> sound is clearly audible)}
\z


Alternatively, the glottal stop <ˀ> sound might be heard before the odd-numbered vowel (whereas in related words, it sounds as if it is after the vowel \xref{ex:lmex22}.

\ea\label{ex:lmex22}
ęhsade̱ˀnyę́:dęˀ (ęh-sa-\exemph{de̱ˀn}-yę́:-dęˀ) ‘you will try’ (odd-numbered \textit{de̱ˀ} sounds more like \textit{dˀe} or \textit{de})
\cfex{sadeˀnyę́:dęh (sa-\exemph{deˀn}-yę́:-dęh) ‘sample it! try it!’ (in even-numbered \textit{deˀ}, all three sounds are ‘in place’, and clearly audible)}
\z


The odd-numbered vowel might also disappear entirely, leaving just the glottal stop <ˀ> sound \xref{ex:lmex5}.

\ea\label{ex:lmex5}
sayę́de̱ˀǫh (sa-yę́-\exemph{de̱ˀ}-ˀǫh)\footnote{This example is from \citet{mithun_watewayestanih_1984}.} ‘you are really good at something’ (odd-numbered \textit{de̱ˀ} sounds like \phonet{tˀ}, an \textsc{ejective} consonant, and no vowel can be heard)
\cfex{ęsayędéˀǫhǫ:k (ę-sa-yę-\exemph{déˀ}-ˀǫh-ǫ:k) ‘you will be really good at it’ (in even-numbered \textit{deˀ}, all three sounds are ‘in place’, and clearly audible)}
\z


Odd-numbered syllables ending with glottal stop <ˀ> have been spelled in various ways, summarized in \xref{ex:lmex6}. In this book, the underlining convention shown in \xref{ex:lmex6d} is used, in order to make the spelling of odd-numbered syllables ending with H and glottal stop <ˀ> consistent (see \sectref{Spelling systems}).

\ea\label{ex:lmex6} ways of spelling shortened syllables ending with glottal stop <ˀ> 
\ea without <ˀ>: gadréhdaˀ (\exemph{gad}-réh-daˀ)\label{ex:lmex6a}
\ex by moving <ˀ>: gˀadréhdaˀ (\exemph{gˀad}-réh-daˀ)\label{ex:lmex6b}
\ex by moving <ˀ> and underlining the vowel gˀa̱dréhdaˀ (\exemph{gˀa̱d}-réh-daˀ)\label{ex:lmex6c}
\ex by underlining the vowel: ga̱ˀdréhdaˀ (\exemph{ga̱ˀd}-réh-daˀ)\label{ex:lmex6d}
\z
\z

\subsection{Exceptions to pronunciation changes related to the syllable count}

Contrary to the rules just described, \emph{all} odd-numbered syllables remain unchanged or fully pronounced if they \emph{begin} with H (\ref{ex:lmex8}a, b, c), glottal stop <ˀ> (\ref{ex:lmex8}d, e), or with consonants containing H, such as T (\phonet{tʰ},  \ref{ex:lmex8f}), or K (\phonet{kʰ}, \ref{ex:lmex8g}).

\ea\label{ex:lmex8} no pronunciation changes, because the syllable begins with H or <ˀ>
\ea onóhaˀtaˀ (o-nóh-\exemph{haˀt}-haˀ) ‘soap’\label{ex:lmex8a}
\ex sheˀshę́nyǫgwęh (\exemph{sʰeˀs}-hę́n-yǫg-węh) ‘you have overpowered someone’\label{ex:lmex8b}
\ex heˀsáˀdre: (\exemph{heˀ}-sáˀd-re:) ‘drive over there’\label{ex:lmex8c}
\ex heyótsˀahdǫh (he-yót-\exemph{sˀah}-dǫh) ‘it is all gone’\label{ex:lmex8d}
\ex agáˀahdraˀ (a-gáˀ-\exemph{ˀahd}-raˀ) ‘my basket’\label{ex:lmex8e}
\ex ęwáhtehgyaˀk (ę-wáh\exemph{t-hehg}-yaˀk) ‘it will erode’\label{ex:lmex8f}
\ex koˀáˀgeh (\exemph{kʰoˀ}-ˀáˀ-geh) ‘on my lap’\label{ex:lmex8g}
\ex dęhsekahsǫ́:goˀ (dęh-se\exemph{k-hah}-hsǫ́:-goˀ) ‘you will divide it into parts’\label{ex:lmex8h}
\z
\z

Similarly all odd-numbered syllables remain fully pronounced when they are the \emph{first} syllable of the word and not preceded by a consonant \xref{ex:lmex9}.

\ea\label{ex:lmex9} 
\ea oˀdáihę: (\exemph{oˀ}-dá-ih-hę:) ‘it is hot’\label{ex:lmex9a}
\ex oháhaˀ (\exemph{oh}-háh-haˀ) \trs{road}
\z
\z 

And lastly, final syllables ending with <ˀ> or H are always fully pronounced, whether they are odd- or even-numbered \xref{ex:lmex10}. 


\ea\label{ex:lmex10} 
\ea ga̱ˀdréhdaˀ (gad-réh-\exemph{daˀ}) ‘car, vehicle’
\ex oˀwáhǫh (oˀ-wáh-\exemph{hǫh}) \trs{meat}
\z
\z

\section{Pronunciation changes in sentences} \label{ch:Pronunciation changes in sentences}

Whereas \sectref{Non-final accent and lengthening} described the kinds of pronunciation changes (accent and lengthening) affecting words in isolation, the following sections describe the pronunciation changes that affect connected words (words in sentences). 

\subsection{Non-final and final accent in neutral sentences} \label{Non-final and final accent in neutral sentences}
In \emph{neutral} sentences or ordinary statements, words have either a non-final or final accent. Non-final accent (\sectref{Non-final accent and lengthening}) applies to words in isolation \xref{ex:finalacex100} or sentence-final words \xref{ex:finalacex2}. In contrast, final accent applies to words in mid-sentence (for which reason, it is also called \textsc{continuing} accent \xref{ex:finalacex3}.

\ea\label{ex:finalacex100} word in isolation, non-final accent\\
Aga:\exemph{tǫ́:}deˀ. ‘I heard it’
\z


\ea\label{ex:finalacex2} sentence-final word, non-final accent\\
\gll Neˀ  gi̱ˀ  tsǫ́:  aga:\exemph{tǫ́:}deˀ \\
The just only  I.heard.it\\
\glt ‘I just heard it’ 
\z


\ea\label{ex:finalacex3} mid-sentence word, final (continuing) accent\\
\gll Aga:tǫ:\exemph{déˀ} tsǫ:, tę́ˀ ní:ˀ degé:gę:ˀ. \\
I.heard.it just not I I.didn’t.see.it\\
\glt ‘I just heard it, I didn’t see it.’ 
\z


When accent placement changes from non-final to final, the vowel lengthening described in \sectref{Non-final accent and lengthening} and \sectref{vowellengthcondition} is not affected. For example, the vowels that are lengthened in non-finally-accented \textit{aga:tǫ́:deˀ} \xref{ex:finalacex2} will remain long in finally-accented \textit{aga:tǫ:déˀ} \xref{ex:finalacex3}.


\subsection{Accent in non-neutral sentences} \label{Accent in ‘non-neutral’ sentences}
Other types of accent, for example, accent in non-neutral sentences, is understudied.\footnote{\citet{hatcher_intonation_2022} will add greatly to our knowledge of this topic.} Some preliminary information about emphatic statements and questions is provided here.\footnote{For the accent patterns in speeches, see \citet{foster_earth_1974}.}

For emphasis, speakers sometimes use final accent instead of the expected non-final accent. For example, the single-word sentence \textit{dó:gaˀ} ‘I do not know’ would normally have a non-final accent, but in \xref{ex:nonneutracex}, it has a final accent, which seems to convey emphasis.

\ea\label{ex:nonneutracex} Do:gáˀ! ‘I do not know!’ (emphatic sentence)
\z

A special accent pattern is used in “WH questions”.\footnote{The term “WH questions” refers to questions which in English begin with ‘who’, ‘what’, ‘when’, ‘where’, or ‘why’.} The words after the question word(s) are accented \emph{non-neutrally}: either all of the vowels are accented (they all have a high pitch), or all are unaccented (they all have a low pitch, \cite{mithun_watewayestanih_1984}, \ref{ex:nonneutracex2}). 

\ea\label{ex:nonneutracex2} WH-question accent (WH words are italicized)
\ea 
 \gll \textit{Gaę} \textit{nhǫ́:} \exemph{dísáhdę́gyǫ́:/disahdęgyǫ:}? \\
 Which place you.come.from.there \\
\glt ‘Where do you come from?’ 

\ex 
 \gll \textit{Dęˀ} \textit{hoˀdę́ˀ} \exemph{syá:sǫ́h/sya:sǫh}?  \\
 What kind you.are.called \\
\glt  ‘What is your name?’


\ex 
 \gll \textit{Dęˀ} \textit{hoˀdę́ˀ} \exemph{ę́hsnége̱há/ęhsnege̱ha}ˀ?  \\
 What kind you.will.drink\\
\glt ‘What will you drink?’ 

\z
\z

Another accent pattern is found in “yes-no” questions with \textit{gęh} ‘Q’.\footnote{“Yes-no” questions elicit answers that may begin with “yes” or “no”.} The word before \textit{gęh} is often pronounced with a non-final accent (as if it were sentence-final, \ref{ex:nonneutracex3}). The particle  \textit{gęh} often remains unaccented, and the words after \textit{gęh} are accented like a regular sentence.

\ea\label{ex:nonneutracex3} 
\gll \exemph{Wadę́ganyahs} gęh satse:nę́ˀ só:wa:s? \\
it.bites  Q your.pet dog\\
\glt ‘Does your dog bite?’ (\cite[335]{mithun_watewayestanih_1984})
\z

\subsection{Accenting particles and particle groups in sentences} \label{ch:Accenting particles and particle groups in sentences}
Particles are accented differently from other words, mainly because they are short words with only one vowel.

Particles tend to be unaccented when they occur next to longer words, which makes them sound like part of the longer word \xref{ex:partaccex}. 

\ea\label{ex:partaccex}
Heshawęhę́ \exemph{tsǫ:} ‘he speaks without thinking’
\cfex{hehsawęhę́\exemph{tsǫ:} (alternative spelling, reflecting the pronunciation)}
\z


Particles often occur in groups, which tend to share one accent (as if they were one word, not several). The particle group (bolded and surrounded by []) at the beginning of \xref{ex:partaccex2a} shares a final accent and the one at the beginning of \xref{ex:partaccex2b} shares a non-final accent. Also (but not illustrated here) particle groups are less obviously accented than other words (as described in \cite{rueentan_prosody_2014}), and may even remain unaccented.

\ea\label{ex:partaccex2}  Particle groups
\ea\label{ex:partaccex2a} 
\gll [\exemph{dęˀ} \exemph{hoˀdę́ˀ}] niyawę̱ˀǫ́h neˀ swéˀgeh hne:ˀ\\
 What kind it.happened the long.ago in.fact\\
\glt ‘…what happened a long time ago.’ (\cite{henry_de_2005})
\ex\label{ex:partaccex2b}
\gll [\exemph{dęˀ} \exemph{hoˀdę́ˀ} \exemph{gę:s}] nigagye̱háˀ neˀ swéˀgeh.\\
 What emphasis kind usually  I.did  the long.ago \\
\glt ‘…what I used to do a long time ago.’ (\cite{henry_de_2005})
\z
\z


\subsection{Accent and Euphonic H in sentences} \label{Accent and pronunciation H in sentences}
\textsc{Euphonic H} is described in \sectref{Word-final pronunciation H}. The short version is that certain words end with an H when they are sentence-final \xref{ex:accexs3}, but lack the final H when not sentence-final \xref{ex:accexs}.

\ea\label{ex:accexs3} 
\gll Ó:. Dę́ˀ hoˀdę́ˀ \exemph{haya:sǫh}? \\
Oh, what kind he.is.called\\
\glt ‘Oh. What is his name?’ (\cite[88]{mithun_watewayestanih_1984}, Eksaˀgó:wah dialogue; sentence-final H)
\z


\ea\label{ex:accexs} 
 \gll Elsie \exemph{eya:sǫ́}  neˀ gonǫhgę́:t, … \\
Elsie she.is.called the she.has.fair.hair\\
\glt ‘The fair one is named Elsie…’ (\cite[88]{mithun_watewayestanih_1984}, Eksaˀgó:wah dialogue; not sentence-final, no H)
\z
