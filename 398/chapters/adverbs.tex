\chapter{“Adverbs” (words and affixes functioning as adverbs)} \label{ch:’Adverbs’ (words and affixes that function as adverbs)}
Words functioning as “adverbs” modify the meaning of verbs or sentences by specifying (or questioning) the time, manner, place, or degree of an activity, happening, or state. The order of “adverbs” in phrases is described next, and then a description of the words, prefixes, and suffixes that function as “adverbs” follows.


\section{“Adverb” order in phrases} \label{ch:’Adverb’ order in phrases}
“Adverbs” tend to occur in specific positions in a phrase. For example, some must be first in the phrase (or before the verb, as in \textit{honákwę̱ˀǫh} in \ref{ex:advorder1}). Others have to occur after another word, but close to the beginning of the sentence, \xref{ex:advorder2}. Many other “adverbs” can appear wherever they make sense in the sentence \xref{ex:advorder3}. The typical positions of \textsc{particles} functioning as “adverbs” are described in \sectref{ch:particle order}.

\ea\label{ex:advorder1} clause-initial\\
 \gll \exemph{ahsǫh} ne:ˀ honákwę̱ˀǫh. \\
still it.is he.is.angry\\
\glt ‘He is still angry.’
\z

\ea\label{ex:advorder2} after another word, close to the beginning of the sentence
\ea
 \gll jidwáhshe:t \exemph{e:ˀ} \\
let’s.count again\\
\glt ‘Let’s count again!’

\ex
 \gll I:ˀ gęh \exemph{e:ˀ} sgwatró:wi:? \\
I Q again you.talk.about.me\\
\glt ‘Are you talking about me again?’ (said jokingly)
\z
\z

\ea\label{ex:advorder3} wherever relevant in the sentence
\ea
 \gll \exemph{Gǫdagyeˀ} ętsahdę́:diˀ. \\
right.away you.will.leave\\
\glt ‘You will leave right away!’

\ex
 \gll Tę́ˀ. Hęgyę́ˀ-tsǫ:. Shede̱hjí:hah ní:ˀ ęgahdę:díˀ \exemph{gǫdagyeˀ}.\\
no, never.mind-just early.morning I I.will.leave right.away\\
\glt ‘No, never mind, I’ll go there right away, early in the morning.’ (\cite[494]{mithun_watewayestanih_1984}, Ękníyętoˀ dialogue)
\z 
\z



\section{“Adverbs” of time} \label{ch:’Adverbs’ of time}
“Adverbs” of time either describe or question \emph{when} an activity or action takes place \xref{ex:advtimeex}. 

\ea\label{ex:advtimeex}
\gll Neˀ giˀ ę:ˀ \exemph{e:ˀ} toh iheˀs. \\
The just affirm again that.one he.is\\
\glt ‘He is here again!’
\z


\subsection{“Adverbs” of time (particles)}
The following \textsc{particles} function as “adverbs” of time and are described in the \textit{Particle dictionary}, \sectref{ch:particle dictionary}.

\begin{CayugaRelated}
    
\item{}\textit{Ahsǫh} \trs{still}, \trs{yet}

\item{}\textit{Daji:hah gwaˀ} \trs{soon}, \trs{a short while}

\item{}\textit{E:ˀ} \trs{again}, \trs{still}

\item{}\textit{Gaę niyo:weˀ tsǫ:} \trs{whenever}

\item{}\textit{Gaoˀ nawahtgeh} \trs{the time before then}

\item{}\textit{Gę:s} \trs{generally}, \trs{used to}, \trs{usually}, \trs{normally}

\item{}\textit{Giˀ} \trs{just}

\item{}\textit{Gǫdagyeˀ} \trs{immediately}, \trs{right away}

\item{}\textit{Gwaˀ} \trs{immediately}, \trs{right then}, \trs{just then}, \trs{finally}

\item{}\textit{Gwahs hwaˀ} \trs{this time for sure}

\item{}\textit{Gwahs waˀ-heh tsǫ:} \trs{just now}, \trs{just a few seconds ago}

\item{}\stem{\textit{-heh}} element related to time

\item{}\textit{Hwaˀ} \trs{this time}, \trs{next}

\item{}\textit{Hwę:dǫh} \trs{when}

\item{}\textit{Hwę:dǫh gwaˀ} \trs{sometime}, \trs{whenever}, \trs{once}

\item{}\textit{Hya:ˀ} \trs{first}, \trs{before anything else}

\item{}\textit{Naˀgę:ˀ, ohnaˀgę:ˀ} \trs{late}

\item{}\textit{Ne:ˀ giˀ gyę:ˀ hya:ˀ} \trs{before all else}, \trs{first}

\item{}\textit{Ne:ˀ hwaˀ} \trs{this (coming) time}

\item{}\textit{Ne:ˀ seˀ gę:s neˀ} \trs{back then}, \trs{a long time ago}

\item{}\textit{Nę: hwaˀ waˀ-ne:ˀ} \trs{nowadays}

\item{}\textit{Nę:-gyęh hwaˀ} \trs{this time}

\item{}\textit{Ogwęhę:gyeˀ} \trs{now and then}, \trs{now and again}

\item{}\textit{Ohę:dǫ:} \trs{early}, \trs{first}

\item{}\textit{Ohnaˀgę:ˀ, naˀgę:ˀ} \trs{late}

\item{}\textit{Ohnaˀgęhjih} \trs{late}, \trs{back then}

\item{}\textit{O:nęh, neˀ o:nęh} \trs{now}, \trs{when}, \trs{then}, \trs{at this time}

\item{}\textit{O:nęh e:ˀ} \trs{again!}

\item{}\textit{O:nęh gwaˀ} \trs{suddenly}, \trs{already}, \trs{finally}, \trs{all at once}

\item{}\textit{O:nęh to:hah} \trs{soon}, \trs{almost}

\item{}\textit{Tęˀ ahsǫh} \trs{not yet}

\item{}\textit{Tęˀ hwę:dǫh} \trs{never}, \trs{not ever}

\item{}\textit{Tgǫhaǫgyeˀ} \trs{sometimes}

\item{}\textit{Tgwahaǫ:ˀ} \trs{sometimes}

\item{}\textit{Tgwęhę:ˀ} \trs{sometimes}

\item{}\textit{To: niyonisheˀ to:} \trs{a certain amount of time}

\item{}\textit{To: niyo:weˀ ne:ˀ} \trs{when}

\item{}\textit{Tǫ:-hwaˀ} \trs{that time}

\item{}\textit{To:hah} \trs{a place}, \trs{a time}

\item{}\textit{Waˀ-} element meaning \trs{current moment}

\item{}\textit{Waˀ-gyęh} \trs{presently}, \trs{so now}, \trs{then}

\item{}\textit{Waˀ-heh} \trs{just now}, \trs{finally}

\item{}\textit{Waˀ-heh-geha:ˀ} \trs{finally}, \trs{at long last}

\item{}\textit{Waˀ-jih} \trs{after a while}, \trs{eventually}, \trs{just a while ago}

\item{}\textit{Waˀ-ji-:hah} \trs{presently}, \trs{a little later}, \trs{after a bit}, \trs{after a while}, \trs{afterwards}, \trs{later}

\item{}\textit{Waˀ-ne:ˀ} \trs{today}, \trs{now}

\end{CayugaRelated}


\subsection{Prefixes and suffixes functioning as “adverbs” of time}
Several verb prefixes and suffixes function as “adverbs” of time. For example, the \stem{ts(i)-} \textsc{\coincident} prefix means ‘at the same time’ as another activity (\ref{ex:advex4}, see \sectref{[tsi-] (coincident) prefix}).

\ea\label{ex:advex4} 
 \gll De̱ˀagowihsrá:t \exemph{tsaˀ}ǫ́nagraˀt. \\
she.had.no.breath while.she.was.born\\
\glt ‘She had no breath when she was born.’
\z

Similarly, the combination \stem{shęh n(i)-\textsc{verb}} ‘that \textsc{\partitive}-\textsc{verb}’ can describe actions extending over a certain period of time, or occurring while some other action happens (\ref{ex:advex5}, see \sectref{[ni-] (partitive)}).

\ea\label{ex:advex5} 
 \gll Agiˀda̱ˀǫ́h ǫ \exemph{shęh} \exemph{naˀonishéˀ} hohta:ˀ. \\
I.was.sleeping I.guess that it.took.time he.was.speaking\\
\glt ‘I slept while he spoke.’
\z

The \stem{s-, j-} \textsc{\repetitive} prefix can also function as an “adverb” of time, describing a repeated action (\ref{ex:advex6}, see \ref{[s-, j-, ji-] (repetitive)}).

\ea\label{ex:advex6}
\ea \exemph{s}awada̱hǫ́:dǫ:ˀ ‘it asked again’
\cfex{awada̱hǫ́:dǫ:ˀ ‘it asked’}
\ex \exemph{j}ago:kǫ́:ni: ‘she is cooking again’
\cfex{ gokǫ́:ni: ‘she is cooking’}
\z
\z



\subsection{Verbs functioning as “adverbs” of time}
Several \textsc{verbs} function as “adverbs” of time, identifying the time of an activity, happening, or state \xref{ex:advex7}. For more examples, see “Related”.

\ea\label{ex:advex7}
\ea sǫ́:deˀ ‘last night’ (atypical \textsc{verb})
\ex Awędędáˀǫh ‘Monday (present)’ (literally, ‘the day was finished’, \textsc{verb})
\z
\z


\begin{CayugaRelated}
\item Months, \ref{months}

\item Periods of time in the day, \ref{periods of time in the day}

\item Yesterday, today, and tomorrow, \ref{yesterday, today, and tomorrow}

\item Seasons and years, \ref{seasons and years}

\item Weekdays, \ref{weekdays}

\item Clock time, \ref{clock time}
\end{CayugaRelated}

\section{“Adverbs” of place} \label{‘Adverbs’ of place}
“Adverbs” of place specify (or question) where an activity or action takes place \xxref{ex:advex19}{ex:advex190}. For more examples, see “Related”.

\ea\label{ex:advex19}
 \gll \exemph{To} \exemph{hǫ:} hayę́hęˀ. \\
there place he.puts.it\\
\glt ‘He is always putting it there.’
\z

\ea\label{ex:advex190}
 \gll \exemph{Tohgeh} gęh \exemph{nhǫ:wéˀ} toh naˀá:weh? \\
there-on Q place that it.happened\\
\glt ‘Where did that happen?’
\z

\begin{CayugaRelated}
\item Location, \ref{ch:Location}

\item Place names, \ref{place names}
\end{CayugaRelated}

\subsection{“Adverbs” of place (particles)} \label{‘Adverbs’ of place (particles)}
\textsc{particles}, \textsc{particle groups}, and \textsc{particle}-\textsc{verb} combinations may function as “adverbs” of place, specifying the location of an action or activity. The following “adverbs” of place are described in the \textit{Particle dictionary}, \sectref{ch:particle dictionary}.

\begin{CayugaRelated}
    \item{} \textit{Ahsdeh} \trs{outside}, \trs{outdoors}

\item{} \textit{Akda:gyeˀ} \trs{beside}, \trs{the edge}

\item{} \textit{Da: gwa:dih} \trs{over here}, \trs{this side}

\item{} \textit{Da: hǫ:weh hǫ:} \trs{this is where}

\item{} \textit{E: gwa:dih} \trs{on the other side}

\item{} \textit{E: ni-\textsc{noun}+adih} \trs{on the other side of \textsc{noun}}

\item{} \textit{Gaę gwaˀ gwa:dih} \trs{whichever way}, \trs{whichever side}

\item{} \textit{Gaę gwaˀ hǫ:weh} \trs{somewhere}, \trs{someplace}, \trs{wherever}, \trs{around}

\item{} \textit{Gaę gwaˀ…ni-\textsc{noun}} \trs{wherever \textsc{noun} is}

\item{} \textit{Gaę gwaˀ tsǫ:} \trs{wherever}, \trs{somewhere}

\item{} \textit{Gaę niyó:weˀ} \trs{how far}, \trs{which distance}

\item{} \textit{Gaoˀ} \trs{this side}, \trs{this way}

\item{} \textit{Gaˀ-} element referring to a location

\item{} \textit{Gaˀ-to:hah} \trs{somewhere}, \trs{someplace}, \trs{around}, \trs{anywhere}, \trs{thereabouts}

\item{} \textit{Gaˀ-to:hah tohgeh} \trs{thereabouts}

\item{} \textit{Gwa:dih, gwai} \trs{to one side}

\item{} \textit{Hehdaˀgeh gwa:dih} \trs{below}, \trs{low}

\item{} \textit{Hehdaˀgeh hǫ:} \trs{downstairs}

\item{} \textit{Hehdage̱hjih} \trs{the bottom}, \trs{low}

\item{} \textit{Hejo:yaˀ tsǫ:} \trs{elsewhere}

\item{} \textit{He:tgęh} \trs{above}, \trs{up}, \trs{superior}

\item{} \textit{He:tgę̱hjih} \trs{the very top}, \trs{high up}

\item{} \textit{He:yo: dagwaishǫ:} \trs{straight ahead}

\item{} \textit{Hę:gyeh gaę hǫ:weh} \trs{no matter where}, \trs{no matter which place}

\item{} \textit{Hǫ:weh} \trs{where}

\item{} \textit{I:nǫh} \trs{far}

\item{} \textit{I:wa:kˀah} \trs{near}

\item{} \textit{Neˀ gwaˀ toh} \trs{here (rather than there)}

\item{} \textit{Neˀ to gyę:ˀ hǫ:weh} \trs{it is where}

\item{} \textit{Nę: toh gwa:dih} \trs{on this side}

\item{} \textit{Nę:-gyęh gwa:dih} \trs{here}, \trs{this side}, \trs{over here}

\item{} \textit{Nę: toh} \trs{here}

\item{} \textit{Nigwa:dih} \trs{-ward, direction}

\item{} \textit{Niyo:weˀ, ni:yo:ˀ, nyo:ˀ} \trs{a certain distance}, \trs{a certain time}

\item{} \textit{Ohę:dǫ: shęh} \trs{ahead}, \trs{in front}, \trs{forward}

\item{} \textit{Ohnaˀgę: gwa:dih shęh} \trs{behind}

\item{} \textit{Ohnaˀgęhjih} \trs{late}, \trs{back then}, \trs{the bottom}

\item{} \textit{Ohnaˀgǫ:} \trs{underneath}, \trs{beneath}, \trs{under}

\item{} \textit{Senyęˀsgwadih} \trs{to your left}

\item{} \textit{Sewaihǫhsdǫh gwa:dih} \trs{to your right}

\item{} \textit{Sganyęˀsgwadih} \trs{to its left}

\item{} \textit{Shęh hǫ: heyodokdaˀǫh} \trs{the bottom}

\item{} \textit{Si:} \trs{over there}

\item{} \textit{Si gwa:dih} \trs{over there}

\item{} \textit{Si gwa:di:hah} \trs{just this side of}

\item{} \textit{Si hne:ˀ si gwa:dih} \trs{over there}

\item{} \textit{Si: hǫ:weh} \trs{way over there}

\item{} \textit{Sǫ: gwadih} \trs{on some other side}

\item{} \textit{Tęˀ gaˀ-toh} \trs{nowhere}, \trs{not anywhere}

\item{} \textit{To:, toh, tǫ:} \trs{there}, \trs{that place}

\item{} \textit{To: gwa:dih} \trs{on that side}

\item{} \textit{To: hǫ:weh} \trs{there}, \trs{where}

\item{} \textit{Toh-geh} \trs{there}
\end{CayugaRelated}



\subsection{Prefixes that function as “adverbs” of place} \label{Prefixes and suffixes that function as ‘adverbs’ of place}
Verb prefixes functioning as “adverbs” of place include the \textsc{partitive} \xref{ex:advex230a}, \textsc{cislocative} \xref{ex:advex230b}, \textsc{translocative} \xref{ex:advex230c}, and \textsc{coincident-dualic} \xref{ex:advex230d} prepronominal prefixes, which are described in the “Related” sections.

\ea\label{ex:advex230} 
\ea\label{ex:advex230a}\exemph{ni}ta:wé:nǫ: ‘a stranger’ (literally, ‘he comes from somewhere’) 
\ex\label{ex:advex230b}\exemph{de}sáˀdre: ‘drive over here’ 
\ex\label{ex:advex230c}\exemph{heˀ}sáˀdre: ‘drive over there’ 
\ex\label{ex:advex230d}\exemph{tsaˀd}ę́hsyę:ˀ ‘you will put, lay them side by side’
\z
\z

\begin{CayugaRelated}
\item{}\stem{tsi-} (coincident) prefix, \ref{[tsi-] (coincident) prefix}

\item{}\stem{ni-} (partitive) prefix, \ref{[ni-] (partitive)}

\item{}\stem{d-} (cislocative) and \stem{heˀ-} (translocative) prefixes, \ref{[d-] (cislocative) and [heˀ-] (translocative)}
\end{CayugaRelated}




\section{“Adverbs” of manner} \label{ch:’Adverbs’ of manner}
“Adverbs” of manner describe (or question) how an activity takes place \xref{ex:advex8}. 

\ea\label{ex:advex8} 
 \gll \exemph{Sgę:nǫ:ˀǫ́h} hodrihsdǫ̱hǫ́:gy.eˀ \\
slowly he.is.sneaking.around\\
\glt ‘He is sneaking around slowly.’ 
\z

\subsection{Prefixes and suffixes that function as “adverbs” of manner}
Several verb prefixes and suffixes function as “adverbs” of manner, describing or questioning how an activity takes place. For example, the combination \stem{shęh ni-} ‘that \partitive’ means something like ‘how’, ‘the manner in which’ \xref{ex:advex11}.
	
\ea\label{ex:advex11} 
 \gll Dękde:níˀ \exemph{shęh} \exemph{ni}wagri̱hóˀdę:\\
I.will.change that how.I.do.things\\
\glt ‘I am going to change my outlook’
\z

Similarly, verbs with \stem{ti- … -ˀah} \textsc{\contrastive-…-\diminutive} have the “adverb”-like meaning of ‘any old way’ \xref{ex:advex12}.

\ea\label{ex:advex12} 

 \gll \exemph{ti}he:\exemph{ˀáh} tsǫ: \\
he.is.walking.along-ish just\\
\glt ‘He is just walking along any old way.’
\cfex{ iheˀ ‘he is walking, moving’}

\z 

The \stem{de-} \textsc{\dualic} prefix can function as an “adverb” of manner describing a two-part motion \xref{ex:advex13}.

\ea\label{ex:advex13} \exemph{d}ęhadagyáˀda̱hgwaˀt ‘he will do push ups’ 
\z
	
The \stem{-ge:} \textsc{\augmentative} suffix can function as an “adverb” of manner, describing an activity done ‘in a big way’ \xref{ex:advex14}.

\ea\label{ex:advex14} 
ahęnatgwéni̱ˀ\exemph{ge:} ‘the big win’, ‘victory’ (literally, ‘they won big’)
\cfex{ęhsa:tgwé:niˀ ‘you will win’}
\z

The \stem{-sgǫ:} \textsc{\facilitative} suffix also functions as an “adverb” of manner \xref{ex:advex15}.

\ea\label{ex:advex15} 
sadahǫdǫ́\exemph{sgǫ:} ‘you are nosy’ (literally, ‘you ask easily', ‘you are always asking questions') 
\cfex{sadahǫ́:dǫ: ‘you ask’}
\z

The above affixes are described in the “Related”  sections listed below.

\begin{CayugaRelated}
\item{}The \stem{ti-} (contrastive) prefix, \ref{[ti-] (contrastive)}

\item{}The \stem{ni-} (partitive) prefix, \ref{[ni-] (partitive)}

\item{}The \stem{de-} (dualic) prefix, \ref{[de-] (dualic)}

\item{}\stem{-ge:} (augmentative), \ref{[-ge:] (augmentative)}

\item{}\stem{-sgǫ:} (facilitative), \ref{[-sgǫ:] (facilitative)}
\end{CayugaRelated}

\subsection{Verbs functioning as “adverbs” of manner}
Several \textsc{verbs} function as “adverbs” of manner, describing how an activity takes place \xxref{ex:advex16}{ex:advex191}. In \xref{ex:advex16}, \textit{ohsnó:weˀ} ‘it is fast’ modifies the meaning of \textit{adwákyuˀkdęˀ} ‘it gets dull’.

\ea\label{ex:advex16}
 \gll \exemph{Ohsno:wéˀ} gę:s adwákyuˀkdęˀ.\\
it.is.fast usually it.gets.dull\\
\glt ‘It gets dull fast.’ (\cite[159]{mithun_watewayestanih_1984}, Enǫhsǫnyaˀdaˀsǫ:ˀǫh dialogue) 
\z

\ea\label{ex:advex17}
 \gll …\exemph{tręhs} gę:s \exemph{wagyesˀagéh} aˀosdagwá:ęh. \\
…too.much usually it.is.easy it.gets.dirty\\
\glt ‘…because it gets dirty too easily.’ (\cite[225]{mithun_watewayestanih_1984}, Agyaˀdawíˀtraˀ dialogue)
\z

\ea\label{ex:advex191} 
 \gll \exemph{Wę:dó:ˀ} da:ga̱hyadǫ̱hsríyaˀksǫ:k.\\
it.is.hard the.paper.should.be.cut\\
\glt ‘The paper is hard to cut.’
\z
	
Some verbs also express an “adverb”-like meaning that is not specifically spelled out in a prefix or suffix \xref{ex:advex18}.
	
\ea\label{ex:advex18}
\ea ętsę́hsga:ˀ ‘you take off \textit{abruptly, quickly, suddenly}’
\ex hahsgyáǫhaˀ ‘he walks \textit{quickly}’, ‘he gives someone encouragement’
\z
\z



\section{“Adverbs” of degree} \label{ch:’Adverbs’ of degree}
“Adverbs” of degree describe the extent or degree of a characteristic or activity \xref{ex:advex27}. For more examples, see \textit{Comparisons}, \sectref{ch:Comparisons (more, the same, or less)}.

\ea\label{ex:advex27} 
\ea 
 \gll \exemph{Stǫ:háh} segę́i:s.\\
a.little move\\
\glt ‘Move a little bit!’

\ex
 \gll  \exemph{Trǫhgeh} \exemph{tsǫ:} jǫgwaihóˀdeˀ. \\
barely just we.are.working\\
\glt ‘We’re barely working.’
\z
\z



\subsection{Particles functioning as “adverbs” of degree} \label{Particles and verbs functioning as ‘adverbs’ of degree}
\textsc{particles} may function as “adverbs” of degree, specifying the degree or extent of an action or activity⁠. The following “adverbs” of degree are described in the \textit{Particle dictionary}, \sectref{ch:particle dictionary}.

\begin{CayugaRelated}
 \item{}  \textit{Ahsǫh} \trs{more}

\item{} \textit{Aǫgo̱hdǫh} \trs{exceptional}, \trs{over the top}, \trs{extremely}, \trs{too much so}

\item{} \textit{Do:gęhs} \trs{really}, \trs{very}

\item{} \textit{Do̱hgá:ˀah} \trs{a few}

\item{} \textit{Do:s} \trs{really}, \trs{very}

\item{} \textit{Giˀ gyę:ˀ} \trs{quite}, \trs{kind of}

\item{} \textit{Gwahs} \trs{really}, \trs{just}, \trs{quite}, \trs{intensifier}

\item{} \textit{Gwahs ǫ:weh} \trs{really}

\item{} \textit{Gwe:gǫh, agwe:gǫh, ogwe:gǫh} \trs{completely}, \trs{totally}

\item{} \textit{Heyohe:ˀ} \trs{more}

\item{} \textit{I:soˀ} \trs{much}, \trs{many}, \trs{lots}, \trs{very}

\item{} \textit{I:so:ˀah} \trs{a fairly big bit}

\item{} \textit{Ji} \trs{too much}

\item{} \textit{Ji aǫgo̱hdǫh} \trs{too much so}

\item{} \textit{Ji trehs} \trs{more (than usual)}, \trs{too much so}

\item{} \textit{Ne:ˀ tsǫ:} \trs{that is only}, \trs{that is all}


\item{} \textit{Nę: tsǫ: gwaˀ toh ni-} \trs{just a little bit}, \trs{very little}

\item{} \textit{Nę: tsǫ: ni-} \trs{just}

\item{} \textit{O:weh, neˀ ǫ:weh} \trs{really}

\item{} \textit{Stǫ:hah, stǫ:hǫh} \trs{a little bit}

\item{} \textit{Tęˀ degę:hęh} \trs{not too many}

\item{} \textit{To hę:gyeh shęh} \trs{really}

\item{} \textit{Trǫhgeh tsǫ:} \trs{barely}

\item{} \textit{Tsǫ:} \trs{just}, \trs{only}

\item{} \textit{Waˀ-jih gyę:ˀ} \trs{almost}, \trs{just about}
\end{CayugaRelated}




\subsection{Suffixes functioning as “adverbs” of degree}
Several suffixes may function as “adverbs” of degree. The \stem{-ˀah} and \stem{-hah} \textsc{\diminutive} suffixes convey the idea that a quality exists to a more modest degree \xref{ex:advex31}.

\ea\label{ex:advex31}
\ea sgęnǫgowá:\exemph{hah} ‘slowly’, ‘\textit{fairly} slow’
\cfex{sgęnǫ:ˀǫh ‘slowly’}
\ex gowa:nę́:\exemph{hah} ‘it is \textit{fairly} big’
\cfex{ gowá:nęh ‘it is big’}
\ex niyǫ́:s\exemph{ˀah} ‘just \textit{a little bit} long’
\cfex{í:yǫ:s ‘it is long’}
\z
\z

The \stem{-jih} \textsc{\intensifier} and \stem{-jihwęh} \textsc{completely} suffixes convey the idea that a quality exists to a more extreme degree \xref{ex:advex30}.

\ea\label{ex:advex30}
\ea owíh\exemph{jih} ‘it is \textit{over}cooked’
\cfex{ deyó:wi:h ‘it is undercooked’, ‘it is not ripe’}
\ex ohę́h\exemph{ji̱hwęh} `it is \textit{really} dried out'\\
\cfex{ohę: `it is dry'}
\z
\z


\subsection{Verbs functioning as “adverbs” of degree}
Several \textsc{verbs} may function as “adverbs” of degree, describing the extent of a quality or activity: when \textit{áǫgo̱hdǫh} modifies the meaning of other verbs, it means ‘extremely’ or ‘too much’ \xref{ex:advex28}. For more examples, see \sectref{Particles and verbs functioning as ‘adverbs’ of degree}
\ea\label{ex:advex28} 
\ea
 \gll \exemph{Aǫgohdǫ́h} ohdrǫhk.\\
it.surpasses it.is.dangerous\\
\glt ‘It is extremely dangerous.’
\ex
 \gll \exemph{Aǫgohdǫ́h} ǫgeˀdra̱héhs tsęh nagadekǫ́:niˀ.\\
it.surpasses I.went.overboard that how.much.I.ate\\
\glt ‘I ate too much.’
\z
\z




\section{Negative “adverbs”} \label{ch:Negative ‘adverbs’}
Negative “adverbs” begin with \textit{tęˀ} \trs{not}. The following negative “adverbs” are described in the \textit{Particle dictionary}, \sectref{ch:particle dictionary}.

\begin{CayugaRelated}
    \item \textit{Tęˀ ahsǫh} \trs{not yet}

\item{} \textit{Tęˀ hwę:dǫh} \trs{never}, \trs{not ever}

\item{} \textit{Tęˀ gaˀ-toh} \trs{nowhere}, \trs{not anywhere}
\end{CayugaRelated}


