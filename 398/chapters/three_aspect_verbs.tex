\chapter{Pronominal prefix choice for three-aspect verbs} \label{Pronominal prefix choice for three-aspect verbs}
\textsc{Three-aspect} verbs occur in the habitual, punctual, and stative aspects. They tend to describe a happening or event. There are three types of such verbs. (For \textsc{role}, see \sectref{Pronominal prefixes and role}.)

\begin{itemize}
    \item \textsc{one-role} verbs, taking non-interactive prefixes (either the \textsc{a}- or \textsc{p}-series);
    \item \textsc{two-role} verbs, taking interactive prefixes, as well as \textsc{a}- or \textsc{p}-series, used in an interactive sense, with implied ‘it’;
    \item \textsc{three-role} verbs, which take interactive prefixes, as well as \textsc{a}- or \textsc{p}-series, used in an interactive sense, with implied ‘it’.
\end{itemize} 

As shown in \tabref{figtab:1:ppthreeaspv} (page \pageref{figtab:1:ppthreeaspv}), aspect is the main factor influencing pronominal prefix choice for three-aspect verbs. 

\begin{table}
\caption{Pronominal prefix selection, three-aspect verbs} \label{Pronominal prefix selection, three-aspect verbs}
\label{figtab:1:ppthreeaspv}
\begin{tabularx}{\textwidth}{llQQ}
\lsptoprule
three-aspect V & subclass & prefix choice & roles\\\midrule
one-role & type 1 & \stem{\textsc{a}-V} \textsc{\habitual}, \textsc{\punctual}

\stem{\textsc{p}-V} \textsc{\stative} & •typically, a “doer”/agent role\\\addlinespace
\midrule
one-role & type 2 & \stem{\textsc{p}-V} \textsc{\habitual}, \textsc{\punctual}, \textsc{\stative} & •typically, an experiencer or undergoer role\\\addlinespace
\midrule
two-role & type 1 & \stem{\textsc{int}-V} \textsc{\habitual}, \textsc{\punctual}, \textsc{\stative}

\stem{\textsc{a:(it-obj)}-V} \textsc{\habitual}, \textsc{\punctual}

\stem{\textsc{p:(it-obj)}-V} \textsc{\stative} or

\stem{\textsc{(it-animal):p}-V} \textsc{\stative} & •\textsc{int}-prefixes express “doer”/agent and experiencer or undergoer roles.

•\textsc{p}-prefixes in the \textsc{stat} are ambiguous: an implied ‘it-animal’ is either the “doer”/agent or the experiencer or undergoer \\
\addlinespace
\midrule
two-role & type 2 & \stem{\textsc{p:(it-obj)}-V} in the \textsc{\habitual}, \textsc{\punctual}, and \textsc{\stative} & •\textsc{p}-prefixes only; prefix often expresses an experiencer or undergoer role. The second role is always an implied ‘it’ -- often the percept\\
\addlinespace
\midrule
three-role &  & Like type 1 two-role verbs & Like type 1 two-role verbs.\\
\lspbottomrule
\end{tabularx}
\end{table}


\section{Pronominal prefix choice for one-role, three-aspect verbs} \label{Pronominal prefix choice for one-role three-aspect verbs}
As shown in \tabref{figtab:1:ppthreeaspv} (following page), there are two types of one-role, three-aspect verb: the first type takes \textsc{a}-series prefixes in the habitual or punctual, but switches to the \textsc{p}-series in the stative aspect. The second type takes the \textsc{p}-series prefixes in all three aspects.

\subsection{One-role, three-aspect verbs, \textsc{a}-series prefixes in the habitual and punctual, \textsc{p} prefixes in the stative} \label{One-role, three-aspect verbs, A prefixes in the habitual and punctual, O prefixes in the stative}
Type 1 one-role, three-aspect verbs (see \tabref{figtab:1:ppthreeaspv}, following page) take an \textsc{a}-series prefix in the habitual or punctual (\ref{ex:3aspswitex}a, b), (\ref{ex:3aspswitex2}a, b), but change to a \textsc{p}-series prefix in the stative \xref{ex:3aspswitexc}, \xref{ex:3aspswitex2c}. Many of these verbs describe activities or actions. The examples in \xref{ex:3aspswitex2} also show that noun incorporation is irrelevant to pronominal prefix choice for such verbs.

\ea\label{ex:3aspswitex} {} \stem{{\dualic}-\exsc{pronominal prefix}-verb-{\aspect}}
\ea degaǫwę́hda:s\\\label{ex:3aspswitexa}
\gll de-gaǫ-wę́hda:d-s\\
 {\dualic}-\textsc{3ns.fi.a}-run-{\habitual}\\
\glt `they run'

\ex da:gáǫwę̱hda:t\\\label{ex:3aspswitexb}
\gll d-a:-gáǫ-wę̱hda:t\\
 {\dualic}-{\indefinite}-\textsc{3ns.fi.a}-run.{\zeropunctual}\\
\glt `they might run'

\ex dewagę̱hdá:dǫh\\\label{ex:3aspswitexc}
\gll de-wag-ę̱hdá:d-ǫh\\
 {\dualic}-\textsc{1s.p}-run-{\stative}\\
\glt `I have run'
\z
\z

\ea\label{ex:3aspswitex2} {} \stem{{\dualic}-\exsc{pronominal prefix}-incorporated noun-verb-{\aspect}}
\ea de̱haihwágenhahs\\\label{ex:3aspswitex2a}
\gll de̱-ha-ihw-á-genh-a-hs\\
 {\dualic}-\textsc{3s.m.a}-matter-{\joinerA}-advocate-{\joinerA}-{\habitual}\\
\glt `he is a lawyer'

\ex dęhsrihwagé:nhaˀ\\\label{ex:3aspswitex2b}
\gll d-ę-hs-rihw-a-gé:nh-a-ˀ\\
 {\dualic}-{\future}-\textsc{2s.a}-matter-{\joinerA}-advocate-{\joinerA}-{\punctual}\\
\glt `you will argue, debate, protest'

\ex deyagodihwágenhęh\\\label{ex:3aspswitex2c}
\gll de-yagodi-ihw-á-genh-ęh\\
 {\dualic}-\exsc{3ns.fi.p}-matter-{\joinerA}-advocate-{\stative}\\
\glt `they are arguing'
\z
\z


\subsection{One-role, three-aspect verbs that always take \textsc{p}-series prefixes} \label{One-role, three-aspect verbs that always take o prefixes}
Type 2 one-role, three-aspect verbs (see \tabref{figtab:1:ppthreeaspv}, page \pageref{figtab:1:ppthreeaspv}) take a \textsc{p}-series  prefix in all three aspects. Many of these verbs describe actions that \emph{happen} to someone rather than being \emph{caused} by someone. However, there are obvious exceptions, including the verb in \xref{ex:3aspswitex6}.

\ea\label{ex:3aspswitex3}
\ea i:sóˀ agida̱hsgęhę́:ˀ gę:s  \\
\gll i:sóˀ ag-ida̱-hs-gęhę́:  gę:sˀ \\
a.lot \textsc{1s.p}-sleep-{\habitual}-{\past} usually\\
\glt ‘I used to sleep a lot’

\ex ęwa:gí:daˀ\\
\gll ę-wa:g-í:da-ˀ\\
 \fut-\textsc{1s.p}-sleep-{\punctual}\\
\glt `I will sleep'

\ex ǫgí:daˀ\\
\gll ǫg-í:da-ˀ\\
\exsc{factual.1s.p}-sleep-{\punctual}\\
\glt ‘I slept’, ‘I’m sleeping’

\ex agída̱ˀǫh\\
\gll ag-ída̱ˀ-ǫh\\
 \textsc{1s.p}-sleep-{\stative}\\
\glt `I am sleeping'
\z
\z

\ea\label{ex:3aspswitex4}
\ea hoyǫ́:dih\\
\gll ho-yǫ́:di-h\\
 \textsc{3s.m.p}-smile-{\habitual}\\
\glt `he is smiling'

\ex aho:yǫ́:diˀ\\
\gll a-ho:-yǫ́:di-ˀ\\
 {\factual}-\textsc{3s.m.p}-smile-{\punctual}\\
\glt `he smiled'
\z
\z

\ea\label{ex:3aspswitex5}
\ea agénya̱ˀgwahs\\
\gll ag-é-nya̱ˀgw-a-hs\\
 \textsc{1s.p}-{\joinerE}-vomit-{\joinerA}-{\habitual}\\
\glt `I vomit all the time'

\ex ǫgénya̱ˀgoˀ\\
\gll ǫg-é-nya̱ˀgo-ˀ\\
 \exsc{factual.1s.p}-{\joinerE}-vomit-{\punctual}\\
\glt `I vomited'

\ex agenya̱ˀgwáhǫh\\
\gll ag-e-nya̱ˀgw-á-h-ǫh\\
\textsc{1s.p}-{\joinerE}-vomit-{\joinerA}-{\dislocative}-{\stative}\\
\glt ‘I am vomiting’ (right now)\\
\z
\z

\newpage

\ea\label{ex:3aspswitex6}
\ea de̱hóhetaˀ\\
\gll de̱-hó-het-haˀ\\
 {\dualic}-\textsc{3s.m.p}-yell-{\habitual}\\
\glt `he is hollering'
\ex atóhe:t\\
\gll a-t-hó-he:t\\
 {\factual}-{\dualic}-\textsc{3s.m.p}-yell.{\zeropunctual}\\
\glt `he hollered or yelled'

\z
\z


\section{Pronominal prefix choice for two- and three-role, three-aspect verbs} \label{Pronominal prefix choice for two- and three-role, three-aspect verbs}
As summarized in (see \tabref{figtab:1:ppthreeaspv}, page \pageref{figtab:1:ppthreeaspv}), there are two types of two-role, three-aspect verb, both of which are described below. (Three-role, three-aspect verbs also pattern the same way and so are described below.)

\subsection{Two- and three-role verbs, pronominal prefix choice} \label{Two- and three-role verbs, A prefixes in the habitual and punctual, O prefixes in the stative}
Recall that two- and three-role verbs take interactive prefixes, but also use \emph{non}-interactive (\textsc{a}- or \textsc{p}-series) prefixes in an interactive sense, with an implied ‘it’ role (see \sectref{Three types of pronominal prefix}.) 

When such verbs take an interactive prefix, the pronominal prefix does not change with aspect: for example, \stem{shǫgwa-} \textsc{3s.m>1p} remains the same in the punctual and stative examples in \xref{ex:3aspswitex7}.

\ea\label{ex:3aspswitex7}
\ea ęhshǫ:gwá:yǫˀ\\
\gll ę-hshǫ:gwá:y-ǫ-ˀ\\
 \fut-\exsc{3ms:1p}-give-{\punctual}\\
\glt `he will give (it) to us all'
\ex shǫgwá:wi:\\
\gll shǫgw-á:wi-:\\
 \exsc{3ms:1p}-give-{\stative}\\
\glt `he has given (it) to us all'
\z
\z

When two- and three-role verbs use a non-interactive prefix in an interactive sense, the interpretation of the implied ‘it’ depends on aspect.\footnote{The examples in \xref{ex:3aspswitex9} demonstrate that incorporated nouns are irrelevant to pronominal prefix choice for this type of verb. The only factor affecting pronominal prefix choice is aspect.

\ea\label{ex:3aspswitex9} \stem{\exsc{pronominal prefix}-incorporated noun-verb-{\aspect}}
\ea hehaihwę́hęhs ‘he always takes the message’,\\
\gll he-ha-ihw-ę́hę-hs\\
 {\translocative}-\textsc{3s.m.a}-matter-convey-{\habitual}\\
\glt `he is a messenger'

\ex hęhaihwę́hę:ˀ\\
\gll h-ę-ha-ihw-ę́hę-:ˀ\\
 {\translocative}-{\future}-\textsc{3s.m.a}-matter-convey-{\punctual}\\
\glt ‘he will take the message’

\ex hehoihwę́hę:\\
\gll he-ho-ihw-ę́hę-:\\
{\translocative}-\textsc{3s.m.p}-matter-convey-{\stative}\\
\glt ‘he has taken a message’

\z
\z} First, in the habitual and punctual, an \textsc{a}-series prefix  overtly expresses a “\emph{doer}”, ‘I’,  and also implies a \emph{non-“doer”-it (animal)} \xref{ex:3aspswitex15a}. In contrast, a \textsc{p}-series prefix used in an interactive sense implies a \emph{“doer”-it (animal)} and overtly expresses a \emph{non-“doer”} role, ‘me’ \xref{ex:3aspswitex15b}. 

\ea\label{ex:3aspswitex15}
\ea ę́:gǫˀ\\\label{ex:3aspswitex15a}
\gll ę-g-ǫ-ˀ\\
 \fut-\textsc{1s.a}-give-{\punctual}\\
\glt `\emph{I} (doer) will give it to \emph{it} (non-doer, animal)'
\ex ęwá:gǫˀ ‘\emph{it} (doer, animal) will give (it) \emph{to me} (non-doer)’, ‘I will be given’,\\\label{ex:3aspswitex15b}
\gll ę-wá:g-ǫ-ˀ\\
 \fut-\textsc{1s.p}-give-{\punctual}\\
\glt `something will be given to me'
\z
\z

However, in the \emph{stative} aspect, \textsc{p}-series prefixes used in an interactive sense are ambiguous: the implied role can refer to a \textit{“doer” (animal)} or a \textit{non-“doer” (animal)} role \xref{ex:3aspswitex20}.\footnote{With only one example to bear it out, the analysis in \xref{ex:3aspswitex20} is tentative.}

\newpage
\ea\label{ex:3aspswitex20} hewagi̱hnǫ́:gǫh\\
\gll he-wag-i̱hnǫ́g-ǫh\\
{\translocative}-\textsc{1s.p}-call-{\stative}\\
\glt ‘\emph{I} have called \emph{it} (non-doer, animal), ‘\emph{it} (doer, animal) has called \emph{me} (non-doer)’
\z


\subsection{Two-role verbs that always take \textsc{p}-series prefixes} \label{Two-role verbs that always take o prefixes}
Type 2 two-role verbs (see \tabref{figtab:1:ppthreeaspv}, page \pageref{figtab:1:ppthreeaspv}) take a \textsc{p}-series pronominal prefix in all three aspects \xxref{ex:3aspswitex10}{ex:3aspswitex13}. Unlike other two-role verbs, these verbs do not take interactive prefixes at all, but only \textsc{p}-series prefixes (with an implied ‘it’ for the second role). Many of these verbs describe activities that \emph{happen} to someone rather than being \emph{caused} by someone. However, there are exceptions, such as the verb in \xref{ex:3aspswitex13}.

\ea\label{ex:3aspswitex10}
\ea agé:gaˀs\\
\gll ag-é:-gaˀ-s\\
 \textsc{1s.p}-{\joinerE}-like.the.taste.of-{\habitual}\\
\glt `I like the taste (of it)'
\ex ęwa:gé:gaˀ\\
\gll ę-wa:g-é:-gaˀ\\
 \fut-\textsc{1s.p}-{\joinerE}-like.the.\-taste.of.{\zeropunctual}\\
\glt `I will like the taste'

\ex ǫge:gáˀ giˀ\\
\gll ǫg-e:-gáˀ giˀ\\
 \textsc{1s.p}-{\joinerE}-like.the.taste.of.{\stative} just\\
\glt `I did like (it)!' (said when you just finished tasting something)
\z
\z

\ea\label{ex:3aspswitex11}
\ea agé:swahs\\
\gll ag-é:-sw-a-hs\\
 \textsc{1s.p}-{\joinerE}-smell-{\joinerA}-{\habitual}\\
\glt `I smell (it) right now'


\ex ęwágeshoˀ\\
\gll ę-wág-e-sho-ˀ\\
\fut-\textsc{1s.p}-{\joinerE}-smell-{\punctual}\\
\glt ‘I’ll smell (it)’

\newpage
\ex ǫgé:shoˀ\\
\gll ǫg-é:-sho-ˀ\\
 \exsc{\factual.1s.p}-{\joinerE}-smell-{\punctual}\\
\glt `I did smell (it)'
\z
\z

\ea\label{ex:3aspswitex12} 
\ea aga:tǫ́:dęhs\\
\gll ag-a:tǫ́:dęh-s\\
\textsc{1s.p}-hear-{\habitual}\\
\glt ‘I hear (it) all the time’ (continually or off-and-on, i.e. the sound of a regular train going by one’s house)\\


\ex ǫga:tǫ́:dęh\\
\gll ǫg-a:tǫ́:dęh\\
\textsc{1s.p}-hear.{\zeropunctual}\\
\glt `I heard (it)'

\ex agatǫdę́ˀǫh\\
\gll ag-atǫdę́ˀ-ǫh\\
\textsc{1s.p}-hear-{\stative}\\
\glt ‘I’ve heard (it) before’
\z
\z

\ea\label{ex:3aspswitex13} 
\ea hehó:gyeˀs \\
\gll he-hó:-gy-eˀs\\
 {\translocative}-\textsc{3s.m.p}-throw-{\habitual}\\
\glt ‘he throws (it)’ (all the time), `he is a pitcher'

\ex haˀhó:diˀ\\
\gll haˀ-hó:-di-ˀ\\
 {\translocative}-\textsc{3s.m.p}-throw-{\punctual}\\
\glt `he threw (it)'

\ex hehó:gyǫ:\\
\gll he-hó:-gy-ǫ:\\
 {\translocative}-\textsc{3s.m.p}-throw-{\stative}\\
\glt `he has thrown (it)'
\z
\z

\section{Pronominal prefix choice and variations on aspectual forms} \label{Pronominal prefix choice and variations on aspectual forms}
Many two- and three-role verbs take variations of the habitual and stative aspect (see \sectref{Post-aspect (tense) suffixes}). For such verbs, pronominal prefix choice is the same as for the regular verb. For example, the habitual past variant in \xref{ex:3aspvarexa} takes the same kind of \textsc{p} prefix as the plain habitual in \xref{ex:3aspvarexb}. More examples are provided in the following sections.

\ea\label{ex:3aspvarex}
\ea ageswa̱hsgę́hę:ˀ\\\label{ex:3aspvarexa}
\gll ag-e-sw-a̱-hs-gę́hę:ˀ\\
\textsc{1s.p}-{\joinerE}-smell-{\joinerA}-{\habitual}-{\past}\\
\glt ‘I used to be able to smell’ (but my nose quit working)

\ex agé:swahs\\\label{ex:3aspvarexb}
\gll ag-é:-sw-a-hs\\
\textsc{1s.p}-{\joinerE}-smell-{\joinerA}-{\habitual}\\
\glt ‘I smell it’ (at the time when this is being said)
\z
\z


\subsection{Pronominal prefix choice for variations on the habitual aspect} \label{Pronominal prefix selection for variations on the habitual aspect}
For variations of habitual aspect verbs (\sectref{Variations on the habitual aspect}), the pronominal prefix is always of the same type as the original habitual verb.

\ea\label{ex:3aspvarex3} \textsc{a}-series habituals
\ea hadó:wa:s\\
\gll h-adó:wa:d-s\\
 \textsc{3s.m.a}-hunt-{\habitual}\\
\glt `he is a hunter'
\cfex{hadowasgę́hę:ˀ\\
	\gll h-adowad-s-gę́hę:ˀ\\
	\textsc{3s.m.a}-hunt-{\habitual}-{\past}\\
	\glt `he used to be a hunter'}
\cfex{hadówa̱hstahk\\
	\gll h-adówa̱hst-ha-hk\\
	\textsc{3s.m.a}-hunt-{\habitual}-{\former}\\
	\glt `he used to be a hunter (long ago)'}
\ex hahdo:s\\
\gll h-ahdo:-s\\
 \textsc{3s.m.a}-dive-{\habitual}\\
\glt `he dives, he is a diver'
\cfex{hahdóhsgę̱hę:ˀ\\
	\gll h-ahdó-hs-gę̱hę:ˀ\\
	\textsc{3s.m.a}-dive-{\habitual}-{\past}\\
	\glt `he used to dive'}
\z
\z

\newpage
\ea\label{ex:3aspvarex100} \textsc{p}-series habituals
\ea agé:swahs\\
\gll ag-é:-sw-a-hs\\
\textsc{1s.p}-{\joinerE}-smell-{\joinerA}-{\habitual}\\
\glt ‘I smell it’ (at the time when this is being said)

\cfex{ageswa̱hsgę́hę:ˀ\\
\gll ag-e-sw-a̱-hs-gę́hę:ˀ\\
\textsc{1s.p}-{\joinerE}-smell-{\joinerA}-{\habitual}-{\past}\\
\glt ‘I used to be able to smell’ (but my nose quit working)
}
\ex go̱hsóˀkaˀ\\
\gll go̱-hsóˀk-haˀ\\
 \textsc{3s.fi.p}-limp-{\habitual}\\
\glt `she is limping'

\cfex{go̱hsóˀkahk\\
	\gll go̱-hsóˀk-ha-hk\\
	\textsc{3s.fi.p}-limp-{\habitual}-{\former}\\
	\glt `she used to limp'}

\cfex{ęhsáhso̱ˀka:k\\
	\gll ę-hs-áhso̱ˀk-ha-:k\\
	\fut-\textsc{2s.p}-limp-{\habitual}-{\modalizer}\\
	\glt `you will limp'}

\ex agé:gaˀs\\
\gll ag-é:-gaˀ-s\\
 \textsc{1s.p}-{\joinerE}-like.the.taste.of-{\habitual}\\
\glt `I like the taste of it'

\cfex{agega̱ˀsgę́hę:ˀ\\
\gll ag-e-ga̱ˀ-s-gę́hę:ˀ\\
\textsc{1s.p}-{\joinerE}-like.the.taste.of-{\habitual}-{\past}\\
\glt ‘I used to like the taste of it’ (but I don’t anymore)
}
\z
\z


\subsection{Pronominal prefix choice for variations on the stative aspect} \label{Pronominal prefix selection for variations on the stative aspect}
For variations of stative aspect verbs (\sectref{Statives with [-gęhę:ˀ] past, [-hne:ˀ] remote, or [-hk] former}), the pronominal prefix is always the same as the regular stative verb.
\newpage

\ea\label{ex:3aspvarex5} \textsc{p}-series statives
\ea saníˀǫh\\
\gll sa-níˀ-ǫh\\
 \textsc{2s.p}-cheap-{\stative}\\
\glt `you are stingy, greedy, cheap'

\cfex{saníˀǫhne:ˀ\\
	\gll sa-níˀ-ǫh-ne:ˀ\\
	\textsc{2s.p}-cheap-{\stative}-{\past}\\
	\glt `you used to be stingy'}
\ex deyagodáwęnyeˀ\\
\gll de-yago-d-áwęnye-ˀ\\
 {\dualic}-\textsc{3s.fi.p}-{\semireflexive}-stir-{\stative}\\
\glt `she is walking about'

\cfex{dęyagodawęnyéha:k\\
	\gll d-ę-yago-d-awęnyé-h-a-:k\\
	{\dualic}-{\future}-\textsc{3s.fi.p}-{\semireflexive}-stir-\textsc{euph.h}-{\joinerA}-{\modalizer}\\
	\glt `she will be walking about'}
\z
\z

\ea\label{ex:3aspvarex7} \textsc{a}-series statives
\ea hahę́:dǫ:\\
\gll ha-hę́:d-ǫ:\\
 \textsc{3s.m.a}-lead-{\stative}\\
\glt `he is the front, leader'

\cfex{hahę́dǫ̱hne:ˀ\\
	\gll ha-hę́d-ǫ̱h-ne:ˀ\\
	\textsc{3s.m.a}-lead-{\stative}-{\past}\\
	\glt `he used to be a leader'}


\ex degágwatwęh\\
\gll de-gá-gwatw-ęh\\
 {\dualic}-\exsc{3s.a}-hem-{\stative}\\
\glt `a hem'

\cfex{dęgagwatwę́hę:k\\
	\gll dę-ga-gwatw-ę́h-ę-:k\\
	\fut-\exsc{3s.a}-hem-{\stative}-{\joinerA}-{\modalizer}\\
	\glt `it will be hemmed'}
\z
\z
