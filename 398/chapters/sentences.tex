\chapter{Simple sentences} \label{Simple sentences}
Various types of simple sentence (which contain just one independent clause) are described in this chapter. For background concepts, see the following sections.

\begin{CayugaRelated}
\item Utterances, clauses, phrases, and sentences, \sectref{Utterances, clauses, phrases, and sentences}\\
\item Simple and complex sentences, clause types, \sectref{Simple and complex sentences, clause types}\\
\item Independent and dependent clauses, relative clauses, \sectref{Independent and dependent clauses, relative clauses}
\end{CayugaRelated}

\section{Commands (imperatives) and suggestions} \label{Commands (imperatives) and suggestions}
Commands (imperatives) and suggestions are used for asking someone to do (or not do) something. Several types are described next. 

\subsection{2nd person (‘you’) commands} \label{2nd person (‘you’) commands}
2\textsuperscript{nd} person commands begin with a 2\textsuperscript{nd} person (‘you’) pronominal prefix. They do not have an aspect suffix \xref{ex:commandex} -- they are no-aspect verbs -- and most do not have a mood prefix (but see below). (For no-aspect, see \sectref{No-aspect verbs} and for euphonic H (\textsc{euph.h}), see \sectref{Word-final pronunciation H}).\footnote{In translations, an exclamation point ‘!’ denotes that the verb in question is a command. Also in translations, “you” is in parentheses because the “you” concept is specifically expressed in the Gayogo̱ho:nǫˀnéha:ˀ prefix but is typically not expressed in English commands.}

\ea\label{ex:commandex} {}\stem{\textsc{2\textsuperscript{nd} person prefix-verb}-{\noaspect}}
\ea sejá:gǫh \\
\gll s-e-já:gǫ-h \\
\textsc{2s.a}-{\joinerE}-persevere-\textsc{euph.h}/{\noaspect}\\
\glt ‘persevere!’, ‘keep it up!’ (a word of encouragement) 


\cfex{ęhse:já:gǫ:ˀ\\
	\gll ę-hs-e:-já:gǫ-:ˀ\\
	\fut-\textsc{2s.a}-{\joinerE}-persevere-{\punctual}\\
	\glt `you will persevere'}

\ex dehsáhsawęh\\
\gll de-hs-áhsawę-h\\
{\cislocative}-\textsc{2s.a}-start-\textsc{euph.h}\\
\glt ‘(you) start!’

\cfex{ ętsáhsawęˀ\\
\gll ę-t-s-áhsawę-ˀ\\
\fut-{\cislocative}-\textsc{2s.a}-begin-{\punctual}\\
\glt ‘you will begin’
}
\z
\z

2\textsuperscript{nd} person commands can also begin with interactive pronominal prefixes, as long as the prefix in question references a 2nd person ‘you’ \xref{ex:commandex3}.

\ea\label{ex:commandex3} \stem{\textsc{interactive-verb}-{\noaspect}}
\ea sheyénawaˀs\\
\gll she-yénawaˀs\\
\textsc{2s:3fis}-help.{\noaspect}\\
\glt ‘(you) help her!’

\ex hehsyénawaˀs\\
\gll hehs-yénawaˀs\\
\textsc{2s:3ms}-help.{\noaspect}\\
\glt ‘(you) help him!’
\z
\z

A third type of 2\textsuperscript{nd} person command takes an \stem{a:-} {\indefinite} prefix, but again has no aspect suffix \xref{ex:commandex13}.

\ea\label{ex:commandex13} {}\stem{\textsc{indefinite-2\textsuperscript{nd} person prefix}-verb-{\noaspect}}\\
dǫdasa̱hsá:węh\\
\gll dǫda-s-a̱hsá:wę-h\\
{\dualic.\indefinite.\cislocative}\exsc{-2s.a}-start-\textsc{euph.h}/{\noaspect}\\
\glt ‘you should start again!’ 
\z


As noted earlier, 2\textsuperscript{nd} person commands often lack an aspect suffix. Such verbs end with euphonic H if they would otherwise end with a short vowel \xref{ex:commandex6}. 

\newpage
\ea\label{ex:commandex6} {}\stem{\textsc{2\textsuperscript{nd} person prefix-verb}-\textsc{euph.h}/{\noaspect}}
\ea sadeˀnyę́:dęh\\
\gll s-ade-ˀnyę́:dę-h\\
 \textsc{2s.a}-{\semireflexive}-try-\textsc{euph.h}/{\noaspect}\\
\glt `try it!'

\cfex{ęhsade̱ˀnyę́:dęˀ\\
	\gll ę-hs-ade̱-ˀnyę́:dę-ˀ\\
	\fut-\textsc{2s.a}-{\semireflexive}-try-{\punctual}\\
	\glt `you will try it'}

\ex Gǫdagyéˀ desahdę́:dih.\\
\gll gǫdagyéˀ de-s-ahdę́:di-h \\
right.away {\dualic}-\textsc{2s.a}-leave-\textsc{euph.h}/{\noaspect} \\
\glt `Leave right away!'

\cfex{ ęsahdę́:diˀ\\
\gll ę-s-ahdę́:di-ˀ\\
\fut-\textsc{2s.a}-leave-{\punctual}\\
\glt ‘you will leave’

}
\z
\z

However, euphonic H does not appear in words ending with a long vowel \xref{ex:commandex4} or a consonant \xref{ex:commandex5}.

\ea\label{ex:commandex4} {}\stem{\textsc{2\textsuperscript{nd} person prefix-verb}-{\noaspect}}\\
sadahǫ́:dǫ:\\
\gll s-ad-ahǫ́:dǫ:\\
 \textsc{2s.a}-{\semireflexive}-ask.{\noaspect} \\
\glt `ask!'

\cfex{ęsada̱hǫ́:dǫ:ˀ\\
	\gll ę-s-ad-a̱hǫ́:dǫ:-ˀ\\
	\fut-\textsc{2s.a}-{\semireflexive}-ask-{\punctual}\\
	\glt `you will ask'}
\z


\ea\label{ex:commandex5} {}\stem{\exsc{2nd person prefix-verb}-{\noaspect}}\\
segá:nę:t\\
\gll s-e-gá:nę:t\\
 \textsc{2s.a}-{\joinerE}-lick.{\noaspect}\\
\glt `lick it!'

\cfex{ęhséganę:t\\
	\gll ę-hs-é-ganę:t\\
	\fut-\textsc{2s.a}-{\joinerE}-lick.{\zeropunctual}\\
	\glt `you will lick it'}
\z



\subsection{1st person (‘let me’ or ‘let us’) commands (suggestions)} \label{1st person (‘let me’ or ‘let us’) commands (suggestions)}
1\textsuperscript{st} person (‘let me’ or ‘let us’) commands take 1\textsuperscript{st} person pronominal prefixes. They do not have a mood prefix or an aspect suffix. They sound more like suggestions \xref{ex:commandex7}. 

\ea\label{ex:commandex7} {}\stem{\exsc{1\textsuperscript{st} person prefix-verb}-{\noaspect}}
\ea hé:ge: \\
\gll hé:-g-e-:\\
{\translocative}-\textsc{1s.a}-go-{\purposive.{\noaspect}}\\
\glt ‘let me go there’ 


\ex deyǫkni̱ˀdráihęh\\
\gll de-yǫkni̱-ˀdráihę-h\\
 {\dualic}-\exsc{1d.p}-hurry-\textsc{euph.h}/{\noaspect}\\
\glt `let us two hurry'

\ex deyǫgwa̱ˀdráihęh\\
\gll de-yǫgwa̱-ˀdráihę-h\\
{\dualic}-\textsc{1p.p}-hurry-\textsc{euph.h}/{\noaspect}\\
\glt ‘let us all hurry’
\z
\z

1\textsuperscript{st} person (‘let me’ or ‘let us’) commands can also take interactive pronominal prefixes, as long as the prefix references a 1\textsuperscript{st} person ‘I’ or ‘we’ \xref{ex:commandex9}.

\ea\label{ex:commandex9} {}\stem{\textsc{interactive-verb}-{\noaspect}}
\ea etiyená:waˀs\\
\gll eti-yená:waˀs\\
 \exsc{1ns.in>3s.fi/3ns}-help.{\noaspect}\\
\glt `let us all help her/them'

\ex dedwadagyénawaˀs\\
\gll de-dw-adag-yénawaˀs\\
 {\dualic}-\exsc{1inp.a}-{\reflexive}-help.{\noaspect}\\
\glt `let us all help one another'
\z
\z


1\textsuperscript{st} person commands do not have an aspect suffix. However, some 1\textsuperscript{st} person imperatives can end with euphonic H \xref{ex:commandex12}. 

\newpage
\ea\label{ex:commandex12} {}\stem{\textsc{1\textsuperscript{st} person prefix-verb}-\textsc{euph.h}/{\noaspect}}\\
ó: gadeˀyę́:dęh  \\
\gll ó: g-ad-e-ˀnyę́:dę-h\\
 oh \exsc{1s.a-{\semireflexive}}-{\joinerE}-try-\textsc{euph.h}/{\noaspect} \\
\glt ‘Oh, let me try!’

\cfex{a:gadeˀnyę́:dęˀ\\
	\gll a:-g-ad-e-ˀnyę́:dę-ˀ\\
	{\indefinite}-\exsc{1s.a-{\semireflexive}}-{\joinerE}-try-{\punctual}\\
	\glt `I might try or attempt it'}
\z


However, euphonic H does not appear in words ending with a long vowel \xref{ex:commandex10} or a consonant \xref{ex:commandex11}\footnote{The \stem{-ˀ} {\punctual} suffix also deletes after consonant-final stems. See \sectref{Sound changes in word formation}.}. Such verbs just lack a final \stem{-ˀ} \textsc{\punctual} ending.

\ea\label{ex:commandex10} {}\stem{\textsc{1\textsuperscript{st} person prefix-verb}-{\noaspect}}
\ea dedwayenáwa̱ˀkǫ:\\
\gll de-dwa-yenáwa̱ˀkǫ:\\
{\dualic}-\exsc{1inp.a}-work.together.{\noaspect}\\
\glt ‘let’s all work together’


\ex dwagyę:\\
\gll dwa-g-yę:\\
\textsc{1inp.a}-{\semireflexive}-sit.{\noaspect}\\
\glt ‘let’s all sit down’
\z
\z

\ea\label{ex:commandex11} {}\stem{\textsc{1\textsuperscript{st} person prefix-verb}-{\noaspect}}
\ea dwa̱hshe:t \\
\gll dw-a̱hshe:t\\
 \textsc{1inp.a}-count.{\noaspect} \\
\glt `let’s all count'

\ex dedwadagyénawahs\\
\gll de-dw-adag-yénawaˀs\\
 {\dualic}-\exsc{1inp.a}-{\reflexive}-help.{\noaspect}\\
\glt `let’s all help each other'
\z
\z


\subsection{3rd person (‘someone’) commands} \label{3rd person (‘someone’) commands}
3\textsuperscript{rd} person (‘someone’) commands begin with an interactive pronominal prefix, which includes a \textsc{3fi.s} reference to ‘she, someone’ - an audience member or potential listener who isn't known to the speaker. These commands have no aspect suffix \xref{ex:commandex14}. 

\ea\label{ex:commandex14} {}\stem{\textsc{3\textsuperscript{rd} person prefix-verb}-{\noaspect}}\\
Sǫga:ˀáh ǫgyénawaˀs  \\
\gll sǫga:ˀáh ǫg-yénawaˀs\\
someone \textsc{3s.fi>1s}-help.{\noaspect}\\
\glt ‘Someone help me!’
\z


\subsection{Statements functioning as commands or suggestions} \label{Statements functioning as commands or suggestions}
2\textsuperscript{nd} person \emph{statements} can be interpreted as commands in context. For example, 2\textsuperscript{nd} person verbs with the \stem{ę-} {\future} prefix and \stem{-ˀ} {\punctual} suffix can be interpreted as suggestions \xref{ex:commandex15}.

\ea\label{ex:commandex15} \stem{\textsc{future-2\textsuperscript{nd} person prefix-verb}-{\punctual}}
\ea ęhsade̱ˀnyę́:dęˀ\\
\gll ę-hs-ad-e̱-ˀnyę́:dę-ˀ\\
\fut-\exsc{2s.a-{\semireflexive}}-{\joinerE}-try-{\punctual}\\
\glt ‘you will try it’


\ex ęsada̱hǫ́:dǫ:ˀ\\
\gll ę-s-ada̱hǫ́:dǫ:-ˀ\\
 \fut-\textsc{2s.a}-ask-{\punctual}\\
\glt `you will ask'

\ex Gǫdagyéˀ ętsahdę́:diˀ’\\
\gll Gǫdagyéˀ ę-t-s-ahdę́:di-ˀ\\
right.away \fut-{\dualic}-\textsc{2s.a}-leave-{\punctual}\\
\glt  ‘You will leave right away.’

\ex Sgatsǫ́ˀ ętsáhshe:t  \\
\gll sgatsǫ́ˀ ę-t-s-áhshe:t\\
 ones \fut-{\dualic}-\textsc{2s.a}-count.{\zeropunctual}\\
\glt ‘You will count by ones.’, ‘Count one by one!’
\z
\z

Similarly, 2\textsuperscript{nd} person statements with an \stem{a:-} {\indefinite} prefix and \stem{-ˀ} {\punctual} suffix can be interpreted as suggestions \xref{ex:commandex150}.

\ea\label{ex:commandex150} {}\stem{\textsc{indefinite-2\textsuperscript{nd} person prefix-verb}-{\punctual}}
\ea a:yagwahsę́:nǫˀ\\
\gll a:-yagwa-hsę́:n-ǫ-ˀ\\
 {\indefinite}-\exsc{1p.ex.a}-name-give-{\punctual}\\
\glt `we all should give it a name'

\ex ǫ:dasáhsawęˀ\\
\gll ǫ:da-s-áhsawę-ˀ\\
 {\indefinitecislocative}\exsc{-2s.a}-begin-{\punctual}\\
\glt `you should begin'

\ex ǫ:tahá:wiˀ\\
\gll ǫ:t-ha-há:wi-ˀ\\
 {\indefinitecislocative}\exsc{-3s.m.a}-carry-{\punctual}\\
\glt `he should bring something'

\ex áeswa:k\\
\gll áe-swa:-k\\
 {\indefinite}-\textsc{2p.a}-eat.{\zeropunctual}\\
\glt `you all should eat it'
\z
\z

\subsection{Particles and particle groups used with commands} \label{Particles and particle groups used with commands}
The following particles and particle groups either function as commands in their own right or are used with commands. The following examples are described in the \textit{Particle dictionary}, \sectref{ch:particle dictionary}.

\begin{CayugaRelated}
\item{}\textit{A:we:tˀah} ‘it is pretend’, 'it's implied’\\
\item{}\textit{Aweˀ hę:gyeh tsǫ:, Awęˀ hę:gyeh tsǫ:} ‘let it go!’\\
\item{}\textit{Do: i:ˀ} ‘let me!’, ‘how about me?’\\
\item{}\textit{Giˀ gyę:ˀ} ‘just do it!’\\
\item{}\textit{Gwaˀ} ‘immediately’, ‘do it now!’\\
\item{}\textit{Hanyoh, Hanyohanyoh} ‘do it!’, ‘come on!’, ‘you go!’\\
\item{}\textit{Haoˀ} ‘come on!’\\
\item{}\textit{Hę:gyeh giˀ} ‘leave well enough alone!\\
\item{}\textit{Nę:} ‘look!’, ‘say!’, ‘see!’\\
\item{}\textit{Nę:-dah} ‘here, take this!’\\
\item{}\textit{Si gwa:dih} ‘move it!’\\
\item{}\textit{Waˀ-gyęh} ‘listen!’, ‘excuse me!’, ‘would you’\\
\item{}\textit{Waˀ-jih} ‘wait!’\\
\item{}\textit{Waˀ-jih hya:ˀ} ‘wait a minute’, 'wait a while’, ‘wait!’
\end{CayugaRelated}

\subsection{Negative commands with \textit{ahgwih} ‘don’t’} \label{Negative commands with [ahgwih] ‘don’t’}
Negative 2\textsuperscript{nd} person commands optionally begin with \textit{ahgwih} ‘don’t’, followed by a verb that begins with an \stem{ę-} {\future} mood prefix and a 2\textsuperscript{nd} person ‘you’ prefix. Negative commands are often no-aspect verbs (\ref{ex:basicnegex}, see \sectref{No-aspect verbs}). However, sometimes negative commands have a \stem{-ˀ} {\punctual} suffix \xref{ex:basicnegex2}.

\ea\label{ex:basicnegex} \stem{ahgwih \textsc{\future-2\textsuperscript{nd} person prefix-verb}-{\noaspect}}
\ea Ahgwíh ęhsa:tró:wih\\
\gll ahgwíh ę-hs-a:t-hró:wi-h \\
don’t \fut-\textsc{2s.a}-{\semireflexive}-tell-\textsc{euph.h}/{\noaspect}\\
\glt ‘Don’t tell anything!’ 

\ex Ahgwíh ędíhsa̱ˀdre: \\
\gll ahgwíh ę-dí-hsa̱-ˀdre:\\ 
don’t \fut-{\cislocative}-\textsc{2s.p}-drive.{\noaspect}\\
\glt ‘Don’t drive over there.’ 

\ex Ahgwíh hęhsáˀdre: \\
\gll ahgwíh h-ę-hsá-ˀdre: \\
don’t {\translocative}-{\future}-\textsc{2s.p}-drive.{\noaspect}\\
\glt ‘Don’t drive over there.’ 


\ex Ahgwíh ęhsá:sdi:s \\
\gll ahgwíh ę-hs-á:sdi:s \\
don’t \fut-\textsc{2s.a}-care.for.{\noaspect}\\
\glt ‘Don’t disturb it!’, ‘Don’t handle it!’ 
\z
\z

\ea\label{ex:basicnegex2} \stem{ahgwih \textsc{\future-2\textsuperscript{nd} person prefix-verb}-{\punctual}}
\ea Ahgwíh dęhsnigǫ̱háęˀ.\\
\gll ahgwíh d-ę-hs-ˀnigǫ̱h-á-ęˀ \\
don’t {\dualic}-{\future}-\textsc{2s.a}-mind-{\joinerA}-lie.{\zeropunctual}\\
\glt ‘Don’t bother it!’ 


\ex Ahgwíh ęhsatwi̱hsdę́ˀda:ˀ \\
\gll ahgwíh ę-hs-at-hwi̱hsd-ę́ˀda:-ˀ\\
don’t \fut-\textsc{2s.a}-{\semireflexive}-money-burn.up-{\punctual}\\
\glt ‘Don’t waste your money!’ 
\z
\z

Additional examples of negative commands can be found in the \textit{Particle dictionary}, \sectref{ch:particle dictionary}.

\begin{CayugaRelated}
\item{}\textit{Ahgwih gwaˀ} ‘don’t!’\\
\item{}\textit{Ahgwih hwaˀ} ‘don’t!’\\
\item{}\textit{Ahgwih hwę:dǫh} ‘don’t ever’
\end{CayugaRelated}



\subsection{Negative suggestions} \label{Negative suggestions}
Negative suggestions optionally begin with \textit{tęˀ} ‘no’, plus a no-aspect verb that begins with the \stem{ta:-, tae-, te:-, etc.} \textsc{\contrastive.\indefinite} combination \xref{ex:negsuggex}.
 
\ea\label{ex:negsuggex} {} \stem{tęˀ \textsc{\contrastive.\indefinite-2\textsuperscript{nd} person prefix-verb}-{\noaspect}}\\
tęˀ taeswá:gę: \\
\gll tęˀ t-ae-swá:-gę:\\
not  {\contrastive}-{\indefinite}-\textsc{2p.a}-see.{\noaspect}\\
\glt ‘You all shouldn’t/won’t see it.’ \\
\z


Another type of negative suggestion begins with the phrase \textit{tęˀ ta:wá:dǫh} ‘you may not’, followed by a verb with an \stem{a:-} {\indefinite} prefix and \stem{ -ˀ} {\punctual} suffix \xref{ex:negsuggex2}. (\textit{Tęˀ ta:wá:dǫh} is related to the phrase \textit{ęwá:dǫˀ} ‘yes, you may…’, both of which are described in \sectref{ch:Possibility markers}.)
	
\ea\label{ex:negsuggex2} \stem{tęˀ ta:wa:dǫ́h \textsc{\indefinite-2\textsuperscript{nd} person prefix-verb}-{\punctual}}\\
tęˀ  ta:wa:dǫ́h  a:sejáodęˀ \\
\gll tęˀ t-a:-w-a:dǫ́-h a:-s-e-j-á-odę-ˀ\\
not {\contrastive}-{\indefinite}-\textsc{3s.a}-become-\textsc{euph.h} {\indefinite}-\textsc{2s.a}-{\joinerE}-smoke-{\joinerA}-put.in-{\punctual}\\
\glt ‘No smoking!’

\cfex{ęwá:dǫˀ\\
	\gll ę-w-á:dǫ-ˀ\\
	\fut-\textsc{3s.a}-become-{\punctual}\\
	\glt `yes, you may'}
\z



\section{Yes-no questions with \textit{gęh}, \textit{ę:ˀ}, and \textit{do:gęhs}} \label{Yes-no questions with [gęh], [ę:ˀ], and [do:gęhs]}
Yes-no questions prompt for answers that optionally begin with \textit{ęhę:ˀ} ‘yes’ or \textit{tęˀ} ‘no’. (In contrast, questions beginning with words like \textit{sǫ́:} ‘who’ or \textit{dęˀ hoˀdęˀ} ‘what’, require answers other than \textit{ęhęˀ} or \textit{tęˀ}.) Types of yes-no questions are described next.



\subsection{Yes-no questions with \textit{gęh}, \textit{tęˀ gęh}} \label{Yes-no questions with [gęh]}

Yes-no questions require either \textit{gęh} ‘Q(uestion)’ \xref{ex:tagquesex} or \textit{tęˀ gęh} ‘isnʼt it’ \xref{ex:tagquesex2}. (Also see tag questions, \sectref{Tag questions (asking for confirmation from the listener)}.)

\ea\label{ex:tagquesex} 
\gll Sgęnǫ́jih \exemph{gęh}? \\
it.is.really.well Q\\
\glt ‘Are you well?’
\z

\ea\label{ex:tagquesex2} 
\gll \exemph{Tęˀ} \exemph{gęh} deshó:yǫ:? \\
Not Q he.doesn’t.come.back\\
\glt ‘Did he not come home?’
\z

Negative yes-no questions begin with \textit{tęˀ gęh} ‘isn't, didn't…etc.’, followed by a negative verb \xref{ex:tagquesex5}.

\ea\label{ex:tagquesex5} 
\exemph{Tęˀ} \exemph{gęh} neˀ desá:węh?\\
\gll Tęˀ gęh neˀ de-sá:-w-ęh \\
not Q the {\negative}-\textsc{2s.p}-own-{\stative}\\
\glt ‘Isn’t it yours?’
\z

In positive questions, \textit{gęh} has to occur after another word, for example after the first word in the simple sentence in \xref{ex:tagquesex4} or after the first word in the utterance in \xref{ex:tagquesex400}. 

\ea\label{ex:tagquesex4}
\gll Enǫ́hweˀs \exemph{gęh}? \\
she.likes.it Q\\
\glt ‘Does she like it?’ 
\z

\ea\label{ex:tagquesex400}
\gll Ęhę́ˀ \exemph{gęh}? \\
Yes Q\\
\glt ‘Is that right?’ 
\z

In addition to having to appear after another word, \textit{gęh} must appear as close to the beginning of the clause as possible: for example, it occurs as part of a particle group \textit{diˀ gęh} in \xref{ex:tagquesex7} but instead, immediately after the verb in \xref{ex:tagquesex700} and \xref{ex:tagquesex6}.

\ea\label{ex:tagquesex7}
\gll Dó:gęhs diˀ \exemph{gęh}? \\
true so Q\\
\glt ‘Is that true?’
\z

\ea\label{ex:tagquesex700}
\gll Sanǫhǫkdá:niˀ \exemph{gęh} diˀ? \\
you.are.sick Q so \\
\glt ‘Are you sick?’ 
\z

In \xref{ex:tagquesex6} and \xref{ex:tagquesex600}, \textit{gęh} appears after the third word in a complex sentence. However it is still as close to the beginning of its clause as possible -- it appears after the first word of the dependent clause (shown in square brackets).

\ea\label{ex:tagquesex6}
\gll Daskro:wíh ędwé: \exemph{gęh}. \\
you.tell.me we.will.go.together Q\\
\glt ‘Tell me if you are coming along.’
\z


\ea\label{ex:tagquesex600}
\gll Daskro:wíh sanǫhǫkdá:niˀ \exemph{gęh}? \\
you.tell.me you.are.sick Q\\
\glt ‘Tell me whether you are sick.’
\z

\textit{Gęh} does not translate into a specific word, but turns a statement \xref{ex:tagquesex301} into a question \xref{ex:tagquesex300}.

\ea\label{ex:tagquesex301}
\gll Sanǫhǫkdá:niˀ.\\
you.are.sick\\
\glt ‘You are sick.’ 
\z

\ea\label{ex:tagquesex300}
\gll Sanǫhǫkdá:niˀ \exemph{gęh} diˀ? \\
you.are.sick Q so\\
\glt ‘So, are you sick?’ 
\z

Statements without \textit{gęh} can also function as \textit{yes-no questions}: in the following examples, there is no explicit question word \xref{ex:statquesex}.

\ea\label{ex:statquesex}
\ea Sanǫhǫkdá:niˀ?\\
\glt ‘Are you sick?’

\ex Gwé:, odeˀdre̱hdágadeˀ? \\
\glt ‘Well, were there a lot of cars?’
\z
\z

The following questions with \textit{gęh} or \textit{tęˀ gęh} are described further in the \textit{Particle dictionary}, \sectref{ch:particle dictionary}.

\begin{CayugaRelated}
\item{}\textit{Gęh} ‘Q’ (question marker)\\
\item{}\textit{Gęh hne:ˀ} ‘how about this one?’\\
\item{}\textit{Oˀǫ:, O: ǫh} ‘Oh really?’\\
\item{}\textit{O: gęh} ‘really?’\\
\item{}\textit{Tęˀ gęh deˀ-} ‘didn't?’\\
\item{}\textit{To gęh ǫ ne:ˀ} ‘I wonder if it is…’
\end{CayugaRelated}

\subsection{Tag questions (asking for confirmation from the listener)} \label{Tag questions (asking for confirmation from the listener)}
Tag questions prompt for confirmation from the listener, and end with short words/phrases (or \textsc{tags}) such as \textit{ę:ˀ} ‘is it so?, isn’t it so?’, \textit{tęˀ gęh} ‘isn't it?’, \textit{ęhę:ˀ gęh} ‘is it?’, and \textit{dó:gęhs} ‘sure, truly’ (\ref{ex:tagquesex205}-\ref{ex:tagquesex203}, examples from \cite{deer_cayuga_2015}.)


\ea\label{ex:tagquesex205} 
\gll Enǫ́hweˀs\exemph{ ę:ˀ}? \\
she.likes.it affirm\\
\glt ‘She likes it, doesn’t she?’
\z

\ea\label{ex:tagquesex200} 
\gll Wę̱hnihsri:yó: waˀneˀ, \exemph{dó:gęhs}? \\
nice.day today truly\\
\glt ‘It’s a nice day today, innit?’
\z


\ea\label{ex:tagquesex201} 
\gll Wę̱hnihsri:yó: waˀneˀ, \exemph{dó:gęhs ę:ˀ}? \\
nice.day today truly affirm\\
\glt ‘It’s a nice day, innit?’
\z


\ea\label{ex:tagquesex202} 
\gll Wę̱hnihsri:yó: waˀneˀ, \exemph{tęˀ gęh}? \\
nice.day today not Q\\
\glt ‘It’s a nice day, innit?’
\z


\ea\label{ex:tagquesex203} 
\gll Wę̱hnihsri:yó: waˀneˀ, \exemph{neˀ ę:ˀ}?\\
nice.day today the affirm\\
\glt ‘It’s a nice day, innit?’ 
\z


As shown in \xref{ex:tagquesex3}, \textit{ę:ˀ} can also be used in statements, to signal agreement (\sectref{ch:Agreement markers}). 

\ea\label{ex:tagquesex3}
\gll Háe. Wę̱hnisri:yó: \exemph{ę́:ˀ}.\\
Hi. nice.day affirm\\
\glt ‘Hi. Nice day, isn’t it.’

\gll Ęhę́ˀ \exemph{ę́:ˀ}.\\
Yes affirm\\
\glt ‘Yes, it is.’ (\cite[339]{mithun_watewayestanih_1984}, Oˀdréhdatgiˀ dialogue)
\z


The following examples of tag questions and statements are described in the \textit{Particle dictionary}, \sectref{ch:particle dictionary}.

\begin{CayugaRelated}
\item{}\textit{Ę:ˀ}, \textit{Neˀ ę:ˀ} ‘isn't it so?’, ‘yes?’, ‘no?’, ‘innit?’\\
\item{}\textit{Ęhęˀ gęh} ‘is that right?’\\
\item{}\textit{Do:gęhs} ‘it's true’, ‘isn’t it true?’\\
\item{}\textit{Do:gęhs diˀ gęh} ‘it's true’, ‘isn’t it true?’\\
\item{}\textit{Do:gęhs ę:} ‘it's true’, ‘isn't it true?’\\
\item{}\textit{Gęh} ‘didn’t I’\\
\item{}\textit{Tęˀ gęh} ‘no?’, ‘isn’t it?’
\end{CayugaRelated}

\section{Negative questions} \label{Negative question}

The following examples of negative questions can be found in the \textit{Particle dictionary}, \sectref{ch:particle dictionary}.

\begin{CayugaRelated}
\item{}\textit{Gęh tęˀ nigęˀǫh} ‘or not?’\\
\item{}\textit{Tęˀ gęh deˀ-} ‘didn't?’\\
\item{}\textit{Tęˀ gęh} ‘no?’, ‘isn't it?’
\end{CayugaRelated}



\section{Equative sentences with linking verbs \textit{né:ˀ}, \textit{deˀgę:}} \label{Equative sentences with linking verbs [né:ˀ], [deˀgę:]}
Equative sentences link two nouns (or noun phrases) -- shown in [] square brackets -- implying that the nouns are equal or somehow the same. Positive equative sentences optionally include the linking word \textit{né:ˀ} ‘it is’ (\ref{ex:equate1}. 

\ea\label{ex:equate1} {} \stem{N … (ne:ˀ) N}\\

\gll [Ganyó:ˀ] hęˀ hne:ˀ (\exemph{ne:ˀ}) [gwíhsgwihs].\\
animal also in.fact (it.is) pigs\\
\glt ‘Pigs are animals.’
\z


Negative equative sentences include the words \textit{tęˀ … de̱ˀgę:} ‘it isn't’. The linked nouns (in square brackets) are shown in \xref{ex:equate2}. (The ellipsis ‘…’ in \textit{tęˀ … de̱ˀgę:} shows where other particles can appear, to dress up the phrase.)

\ea\label{ex:equate2} {} \stem{tęˀ … N de̱ˀgę: N}\\

\gll \exemph{Tęˀ} hne:ˀ [ganyó:ˀ] \exemph{de̱ˀgę:} [neˀ awę́hęˀ] \\
Not in.fact animal it.is.not the flower(s)\\
\glt ‘Flowers are not animals.’
\z


The following sections contain examples of equative sentences and are described further in the \textit{Particle dictionary}, \sectref{ch:particle dictionary}.

\begin{CayugaRelated}
\item{}\textit{Ę:, E:} `atypical verb’\\
\item{}\textit{Ne:ˀ} ‘it is’ (equative)\\
\item{}\textit{Tęˀ de̱ˀgę:} ‘it isn't’\\
\item{}\textit{Tęˀ ne:ˀ de̱ˀgę:} ‘without’, 'lacking’
\end{CayugaRelated}

\subsubsection*{Verbs of existence} \label{Verbs of existence}
Verbs of existence are thematically related to the linking words used in equative sentences, and are listed below for interest’s sake. (To save space, many examples only contain links to other sections.)

Verbs meaning ‘to be, to exist’ are listed in \xref{ex:vexist}. 

\ea\label{ex:vexist}
\ea ne:ˀ ‘it is’ (\sectref{p:[ne:ˀ] `it is’})

\ex tęˀ … de̱ˀgę:\\
\glt ‘it isn't’ (\sectref{p:[tęˀ deˀgę:]})

\ex iheˀs\\
\gll i-h-e-ˀs\\
 {\prothetic}-\textsc{3s.m.a}-go-{\habitual} (\sectref{ch:Simple motion verb in the habitual})\\
\glt `he is here'
\z
\z


Other verbs of existence have more specific meanings \xxref{ex:vexist2}{ex:vexist4}.

\ea\label{ex:vexist2} \stem{iˀdrǫˀ, iˀdrǫd, iˀdrǫdǫˀ} ‘live, dwell someplace, be placed’
\ea shęh tgiˀdrǫˀ\\
\gll t-g-iˀdrǫˀ\\
 {\cislocative}-\textsc{1s.a}-dwell.{\stative}\\
\glt `where I live'

\ex hadiˀdrǫ́:dǫˀ\\
\gll had-iˀdrǫ́:-d-ǫˀ\\
 \textsc{3ns.m.a}-dwell-\textsc{euph.d}-{\distributive}\\
\glt `how they are placed'

\ex ęhsniˀdrǫ́:daˀk\\
\gll ę-hsn-iˀdrǫ́:-d-a-ˀk\\
 \textsc{\future-2d.a}-dwell-\textsc{\causative-\joinerA-\modalizer}\\
\glt `you two will stay home'

\ex nęgiˀdrǫ́:daˀk\\
\gll n-ę-g-iˀdrǫ́:-d-a-ˀk\\
\textsc{\partitive-\future-1s.a}-dwell-\textsc{\causative-\joinerA-\modalizer}\\
    \glt ‘where I will dwell, live’
\z
\z

\ea\label{ex:vexist4} \stem{nagreˀ} ‘live someplace’
\ea tgáenagreˀ\\
\gll t-gáe-nagreˀ\\
 {\cislocative}-\textsc{3ns.fi.a}-dwell.{\stative}\\
\glt `where they live over there'

\ex knagreˀ\\
\gll k-nagreˀ\\
 \textsc{1s.a}-dwell.{\stative}\\
\glt `I live, stay'
\z
\z

Positional verbs (a thematic type of stative-only verb) describe the typical position of an object \xref{ex:vexist5}.

\ea\label{ex:vexist5} positional verbs
\ea shęh dwaknǫ́hso:t  \\
\gll shęh d-wak-nǫhs-o:t\\
that  {\cislocative}-\textsc{1s.p}-house-stand.{\stative}\\
\glt ‘where I dwell’

\ex  nigá:yęˀ \\
\gll ni-gá:-yęˀ\\
 {\partitive}-\textsc{3s.a}-lie.on.ground.{\stative}\\
\glt ‘where it is at’, `where it is placed'

\ex ohwę́jadeˀ\\
\gll o-hwę́j-a-deˀ\\
 \textsc{3s.p}-land-{\joinerA}-exist.{\stative}\\
\glt `existing earth, land'
\z
\z

Finally, counting verbs (a thematic type of stative-only verb) describe the number of items that exist (\ref{ex:vexist6}, see \textit{Comparisons, counting, measuring}, \sectref{ch:Comparisons, counting, measuring}).

\ea\label{ex:vexist6} counting verbs\\
To: gitsǫ: ni:yǫ:.  \\
\gll To: gi-tsǫ: ni:-y-ǫ:\\
that just-only {\partitive}-\textsc{3s.p}-certain.amount.{\stative}\\
\glt ‘That’s just all there is.’
\z

