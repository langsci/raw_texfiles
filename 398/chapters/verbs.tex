\chapter{Verbs} \label{Verbs}
Verbs describe actions \xref{ex:verbex}, types of movement \xref{ex:verbex2}, states or characteristics \xref{ex:verbex3}, ways of existing \xref{ex:verbex4}, mental activities \xref{ex:verbex5}, perceptions \xref{ex:verbex6}, and object locations \xref{ex:verbex7}. They also function as “nouns” (\sectref{“Nouns” (words functioning as nouns)}), “adjectives” (\sectref{ch:’Adjectives’ (words that function as adjectives)}) and “adverbs” (\sectref{ch:’Adverbs’ (words and affixes that function as adverbs)}).

\ea\label{ex:verbex} actions
\ea kdakseˀ ‘I am running’
\ex ę́:ge:k ‘I will eat’
\ex dęyǫ́na̱hsgwahk ‘she will jump’
\ex ęgaǫgidagráhdęˀ ‘they are going to trip me, make me fall’
\z
\z

\ea\label{ex:verbex2} movement
\ea dagę́:neˀ ‘they are coming
\ex ǫkníˀdreˀ ‘we two are riding along in a vehicle’
\ex swatahí:neˀ ‘you all are walking’
\z
\z

\ea\label{ex:verbex3} states or characteristics
\ea hahnę́:ye:s ‘he is tall’
\ex aknó:shę: ‘I am envious’
\ex aknǫhǫkdá:nih ‘I am sick’
\z
\z

\ea\label{ex:verbex4} existence (or non-existence)
\ea 
\gll To: gi̱ˀ tsǫ: ní:yǫ:.\\
that just only it.is\\
\glt ‘That’s just all there is.’
\ex iheˀs ‘he is here’
\z
\z

\ea\label{ex:verbex5} mental activities, feelings
\ea ihse: ‘you want, hope, think’
\ex gonǫ́hkwaˀ ‘I love you’
\ex desaˀnigǫ́hęhdǫh ‘you are sad’
\z
\z

\ea\label{ex:verbex6} perceptions
\ea ęsé:gęˀ ‘you will see’
\ex ęsa:tǫ́:dęh ‘you will hear it’
\ex ęwágeshoˀ ‘I will smell it’ (unintentionally)
\ex ęwádehswaht ‘I will smell it’ (on purpose)
\z
\z

\ea\label{ex:verbex7} object location
\ea gá:yę:ˀ ‘it is lying’
\ex ganí:yǫ:t ‘it is hanging’
\z
\z

Verbs are the only required element in a sentence. As such, they often express the meaning of an entire sentence \xref{ex:verbex8}.

\ea\label{ex:verbex8} verbs as sentences
\ea degaǫdatgǫhsóhaeˀ ‘they are washing each other’s faces’
\ex ahadadrihwagwé:nyęˀ ‘he had earned it for himself’
\z
\z

Verbs minimally require a pronominal prefix. They also usually require an aspect suffix (\cite{chafe_seneca_1967}). However, \textsc{no-aspect} verbs, including “commands” and “suggestions” (described in \sectref{Commands (imperatives) and suggestions}) do not have an aspect suffix (for which, see \sectref{Meaning of no-aspect verbs}).

Verbs will be classified according to the criteria in \xref{ex:verbcriteria}, which are defined in the following sections.

\ea\label{ex:verbcriteria} criteria for verb classification
\ea noun incorporation 
\ex transparent or fixed expressions
\ex aspect
\ex pronominal prefix type
\z
\z


\section{Verbs and noun incorporation} \label{Verbs and noun incorporation}
Verbs can be classified according to their ability to \textsc{incorporate} (include) a noun stem (\sectref{ch:Noun Incorporation}). This criterion yields three types of verbs, described next.


\subsection{Verbs that optionally have an incorporated noun} \label{Verbs that optionally have an incorporated noun}
Some verbs can take an \textsc{incorporated noun} but do not require one. For example, the verbs in \xref{ex:verbex21} and \xref{ex:verbex22} have an incorporated noun and are well-formed -- they sound like words. However, the same verbs -- \textit{ęhsní:nǫˀ} and \textit{ęgyę́:to} -- lack an incorporated noun and are still perfectly good words. 

\ea\label{ex:verbex21}
ęyagw\exemph{ahgwęnya}hnínǫnyǫ:ˀ ‘we all will buy clothes’
\cfex{ \exemph{ahgwę́ny}aˀ ‘clothes’}
\cfex{ ęhsní:nǫˀ ‘you will buy it’}
\z


\ea\label{ex:verbex22}
hęk\exemph{hǫnaˀdá}yętoˀ ‘I will plant potatoes there’
\cfex{ o\exemph{hǫ́naˀda}ˀ ‘potatoes’}
\cfex{ ęgyę́:toˀ ‘I will plant it or something’}
\z


\subsection{Verbs that cannot have an incorporated noun} \label{Verbs that cannot have an incorporated noun}
Some verbs cannot take an incorporated noun because their meaning does not allow for it. These are “activity” or “action” verbs that only express one \textsc{role} (\sectref{Pronominal prefixes and role}). For example, the verbs shown in \xref{ex:verbex23} express just one role, that of “doer” (either a single “doer” or multiple “doers”). However, their meaning does not include that of an undergoer. In contrast, the verb in \xref{ex:verbex22} above, \textit{hękǫna̱ˀdáyętoˀ}, requires both a “doer” (‘I’) and an undergoer (‘potatoes’).

\ea\label{ex:verbex23}
\ea\label{ex:verbex23a} ga̱hnih ‘it is barking’
\ex\label{ex:verbex23b} gadá:węh ‘I am swimming’
\ex\label{ex:verbex23c} hęnadawęhs ‘they (males) are swimming’
\ex\label{ex:verbex23d} saˀe:yǫˀ ‘she returned’
\ex\label{ex:verbex23e} nǫdáhse:ˀ ‘you came from there’
\ex\label{ex:verbex23f} haˀgáge:t ‘I stopped by there’
\ex\label{ex:verbex23g} agiˀ ‘I said’
\z
\z


\subsection{Verbs requiring an incorporated noun} \label{Verbs requiring an incorporated noun}
Some verbs require an incorporated noun. (As such, they are unlike \textsc{minimal} verbs, which only require a pronominal prefix, a verb stem, and an aspect suffix.) For example, the verb \stem{+i:yo:} ‘be good, nice’ never occurs without an incorporated noun. It incorporates the noun \stem{nahsgw} ‘domestic animal’ in \xref{ex:verbex24a} but instead takes the noun \stem{yęhsr} ‘blanket’ in \xref{ex:verbex24b}. A plus ‘+’ sign before a verb stem means that the verb in question requires an incorporated noun.

Many obligatorily incorporating verbs function as “adjectives”, describing attributes of the incorporated noun \xxref{ex:verbex24}{ex:verbex25}. 

\ea\label{ex:verbex24} \stem{+i:yo:} ‘be good, nice’
\ea ganahsgw\exemph{í:yo:} ‘a nice pet’\label{ex:verbex24a}
\cfex{ga\exemph{náhsgw}aˀ ‘tame animal’, ‘pet’, ‘domestic animal’}
\ex oyęhsr\exemph{í:yo:} ‘nice blanket’\label{ex:verbex24b}
\cfex{o\exemph{yę́hsr}aˀ ‘blanket(s)’}
\z
\z


\ea\label{ex:verbex25} \stem{+owa:nęh} ‘be big’
\ea ga̱hǫhsr\exemph{owá:nęh} ‘big box’\label{ex:verbex25a}
\cfex{ ga̱\exemph{hǫ́hsr}aˀ ‘box’}
\ex ojǫˀd\exemph{owá:nęh} ‘big fish’\label{ex:verbex25b}
\cfex{ o\exemph{jǫ́ˀd}aˀ ‘fish’}
\z
\z

\largerpage
\section{Transparent vs fixed expressions} \label{Fixed and transparent expressions}
Verbs can be classified as either \textsc{transparent} or \textsc{fixed expressions}.\footnote{Fixed expressions are the verb \textsc{bases} described in \citet{foster_base_1989}.} 

Transparent expressions just have one straightforward (literal) meaning. For example, in example (\ref{ex:verbex24a}, previous page), the transparent verb \textit{ɡanahsgwí:yo:} has parts meaning ‘pet’ \stem{nahsgw} and ‘nice’ \stem{+iyo:}, and the word straightforwardly means ‘nice pet’. As well, substituting the pronominal prefix and the incorporated noun results in a new word, \textit{oyęhsrí:yo:}, which also has a straightforward meaning, ‘nice blanket’ \xref{ex:verbex24b}.

In contrast, fixed expressions have both a transparent (or literal) meaning and a \emph{non}-transparent meaning. The non-trans\-parent meaning is dominant. For example, in \xref{ex:verbex26a}, the intended, fixed meaning is ‘sandpiper’ (a type of shore bird) and the transparent (and not usually relevant) meaning of the word is ‘good sand’. 

\largerpage
\ea\label{ex:verbex26} 
\ea o\exemph{ˀnehs}í:yo: \\\label{ex:verbex26a} 
\glt ‘sandpiper’ (fixed meaning); ‘good sand’ (transparent meaning)
\cfex{ o\exemph{ˀnéhs}aˀ ‘sand’}

\ex ga\exemph{niga:hęhsr}í:yo: \\\label{ex:verbex26b} 
\glt ‘cotton’, ‘silk’ (fixed meaning); ‘good material’ (transparent meaning)
\cfex{ o\exemph{nigá:hęhsr}aˀ ‘material’, ‘cloth’}
\z
\z

In order to convey the fixed meaning of ‘sandpiper’ in \xref{ex:verbex26a}, the pronominal prefix, incorporated noun, and verb must remain constant or fixed, while other parts can change as needed. (For example, one could add the \stem{-ˀs} {\plural} suffix to create \textit{oˀnehsí:yoˀs} ‘sandpipers’.) Similarly, all other fixed expressions require specific prefixes, incorporated nouns, or suffixes in order to fully express their non-transparent meaning.


\largerpage
\subsection{Types of fixed expression} \label{Types of fixed expression}
For fixed expressions, the verb stem and at least one other part remain fixed or constant. The various types of fixed expression are described in this section.

In some cases, the combination of \stem{incorporated noun-verb} stem conveys a fixed meaning. For example, in \xref{ex:verbex28a}, the meaning of ‘offending someone’ is conveyed by the combination of \stem{ˀnigǫha+ˀe:k} ‘mind+hit’. 

\ea\label{ex:verbex28} fixed expressions with incorporated noun \stem{ˀnigǫha} ‘mind’
\ea\label{ex:verbex28a} ęhshe\exemph{ˀnigǫ́ha}ˀe:k ‘you will offend someone’ (literally, ‘you will hit someone’s mind’) 
\ex\label{ex:verbex28b} ho\exemph{ˀnigǫhá}ędaˀs ‘he understands’ (literally, ‘his mind settles’)
\ex\label{ex:verbex28c} ęhshe\exemph{ˀnigǫh}ǫ́:niˀ ‘you will influence, persuade someone’ (literally, ‘you will make someone’s mind’)
\z
\z

In other cases, the combination of \stem{prefix-…verb stem} conveys a fixed meaning. Example \xref{ex:verbex29a} illustrates the verb \stem{yena:} ‘to catch or receive something’. In the related expression in \xref{ex:verbex29b} the fixed parts are the \stem{t-} \textsc{\dualic} prepronominal prefix and \stem{yena:}; this combination means ‘to do something together’. In \xref{ex:verbex29c}, the fixed parts include the incorporated noun \stem{rihwa} ‘word’ and \stem{yena:}, which together mean ‘to accept advice’. Finally, in \xref{ex:verbex29d}, the \stem{adag-} \textsc{\reflexive} prefix and \stem{yena:} together mean ‘to wrestle’.

\ea\label{ex:verbex29} the transparent expression \stem{yena:ˀ} ‘to catch, receive, accept’, and related fixed expressions
\ea\label{ex:verbex29a} ag\exemph{yé:na:ˀ }’I caught, received it’ (minimal verb, \stem{yena:ˀ} transparent expression)
\ex\label{ex:verbex29b} a\exemph{t}hadi\exemph{yé:na:ˀ} ‘they did it together’, ‘they were accomplices’ (\stem{\dualic-…yena:} fixed expression)
\ex\label{ex:verbex29c} ęhs\exemph{rihwayé:na:ˀ} ‘you will accept advice, a suggestion’ (\stem{rihwa-yena:} fixed expression)
\ex\label{ex:verbex29d} ęhs\exemph{adagyé:na:ˀ} ‘you will wrestle’ (\stem{\reflexive-yena:} fixed expression)
\z
\z

Like other verbs, fixed expressions may also have free elements: for example both of the words in \xref{ex:tsiex} begin with \stem{tsi-} \textsc{\coincident}, which can be freely added to the relevant verbs to transparently mean ‘when’. Meanwhile, the words in \xref{ex:tsiex} also include fixed \stem{incorporated noun-verb stem} combinations -- the combination \stem{ǫgweˀd-ase:} \xref{ex:tsiexa} means ‘young person’, while \stem{ksaˀd-ase:} \xref{ex:tsiexb} means ‘teenager’.

\ea\label{ex:tsiex} \stem{tsi-verb} ‘while, when’
\ea\label{ex:tsiexa} 
 \gll \exemph{tsi}hǫgwe̱ˀda:sé: hohsę: \\
while.he.was.young he.is.fat\\
\glt ‘When he was young, he was fat.’ 
\cfex{hǫgwéˀdase:\\
	\gll h-ǫgwéˀd-ase:\\
	\textsc{3s.m.a}-person-new.{\stative}\\
	\glt `young man'}
\ex \exemph{tsi}yeksa̱ˀdá:se: ‘when she was a teenager…’ \label{ex:tsiexb}
\cfex{eksaˀdasé:ˀah\\
	\gll e-ksaˀd-asé:-ˀah\\
	\textsc{3s.fi.a}-child-new.{\stative}-{\diminutive}\\
	\glt `teenage girl'}
\z
\z


\section{Verbs and aspect} \label{Verbs and aspect}
Verbs can be classified according to the types of aspect suffixes they take. Before getting into details, some background information is provided next.

The three main types of aspect suffix are the punctual, habitual, and stative (see \sectref{Aspect suffixes}). While there is only one punctual suffix, \stem{-ˀ} {\punctual}, there are several habitual and stative suffixes. For example, the habitual endings include \stem{-s}, \stem{-haˀ}, and \stem{-h}. As shown in \xref{ex:habitexampleagain}, a given verb normally takes just one of the possible habitual (or stative, not shown) aspect endings.

\ea\label{ex:habitexampleagain} verbs in the habitual aspect
\ea degáˀswe:\exemph{s} ‘I am going deaf’
\ex hęnáˀswa̱ht\exemph{haˀ} ‘they are firemen’
\ex sgegę̱hęˀda:ni\exemph{h} ‘you are sick of me’
\z
\z

In this work, verbs with the punctual suffix are described as “punctual verbs”, while verbs with any of the habitual suffixes are “habitual verbs”, and verbs with any of the stative suffixes are “stative verbs”.


\subsection{Verbs occurring only in one aspect (stative or habitual)} \label{Verbs occurring only in one aspect (stative or habitual)}
Some verbs occur only in one aspect, either the stative or the habitual. \textsc{Stative-only} verbs take only a stative aspect suffix \xref{ex:verbex12}. A smaller set of \textsc{habitual-only} verbs exclusively occur in the habitual aspect \xref{ex:verbex13}.

\ea\label{ex:verbex12} stative-only verbs
\ea ohsnó:weˀ ‘it is fast, quick’ 
\ex go̱haˀdí:yo: ‘she is a good singer, she has a good voice’ 
\ex ogá:yǫh ‘it is old’ 
\z
\z

\ea\label{ex:verbex13} habitual-only verbs
\ea gagrahs ‘it stinks’
\ex agadǫ̱hswéˀdanih ‘I am hungry’ 
\ex gotgǫ́:nyohs ‘she has high standards’ 
\ex akneˀdraˀdá:nih ‘I feel nausea’, ‘I am nauseous’
\z
\z

\subsection{Three-aspect verbs (habitual, punctual, stative)} \label{Three-aspect verbs (habitual, punctual, stative)}
\textsc{Three-aspect} verbs can take three all aspects -- the habitual, punctual, and stative. The three-aspect verb in \xref{ex:verbex93} takes the \stem{-h} \textsc{\habitual} \xref{ex:verbex93a}, the \stem{-ˀ} \textsc{\punctual} \xref{ex:verbex93b} and the \stem{-:} \textsc{\stative} \xref{ex:verbex93c} aspect endings.

\ea\label{ex:verbex93} \stem{hsęnǫni} ‘store something’
\ea\label{ex:verbex93a} se̱hsę:nǫ́:nih\\
\gll s-e̱-hsę:nǫ́:ni-h\\
 \exsc{2s.a-joinerE-}store-{\habitual}\\
\glt `you are storing it right now'
\ex ęhsehsęnǫ́:niˀ\\\label{ex:verbex93b}
\gll ę-hs-e-hsęnǫ́:ni-ˀ\\
 \fut-\exsc{2s.a-joinerE-}store-{\punctual}\\
\glt `you will store it'
\ex ga̱hsę:nǫ́:ni:\\\label{ex:verbex93c}
\gll ga̱-hsę:nǫ́:ni-:\\
 \textsc{3s.a}-store-{\stative}\\
\glt `stored items'
\z
\z 

Verbs that take other types of habitual and stative endings are shown in \xref{ex:verbex91} and \xref{ex:verbex92}. (The punctual suffix only has one form, \stem{-ˀ}.\footnote{The punctual suffix is not pronounced in words ending with a consonant, which is why the verbs in \xref{ex:verbex91b} and \xref{ex:verbex92b} do not have an overt \stem{-ˀ} \textsc{\punctual} ending. The term \textsc{zero punctual} ({\zeropunctual}) describes this state of affairs.})

\ea\label{ex:verbex91} \stem{ˀhoweg} ‘cover something’
\ea geˀhó:we:s\\\label{ex:verbex91a}
\gll g-e-ˀhó:we:-s\\
 \textsc{1s.a}-{\joinerE}-cover-{\habitual}\\
\glt `I am covering something'
\ex ęhseˀhó:we:k\\\label{ex:verbex91b}
\gll ę-hs-e-ˀhó:we:k\\
 \fut-\textsc{2s.a}-{\joinerE}-cover.{\zeropunctual}\\
\glt `you will cover something'
\ex ageˀhowé:gǫh\\\label{ex:verbex91c}
\gll ag-e-ˀhowé:g-ǫh\\
 \textsc{1s.p}-{\joinerE}-cover-{\stative}\\
\glt `I did cover something'
\z
\z

%
%
% \ea\label{ex:verbex91} \stem{-ˀhoweg} ‘cover something’
%     \ea   geˀhó:we:s\\
%     \gll g-e-ˀhó:we:-s\\
%     1s.a-joinerE-cover-\textsc{habit}\\
%     \glt ‘I am covering something’
%
%     \ex    ęhseˀhó:we:k\\
%     \gll ę-hs-e-ˀhó:we:k\\
%     \textsc{fut-2s}-{\joinerE}-cover.\textsc{ø.punct} \\
%     \glt ‘you will cover something’
%
%     \ex    ageˀhowé:gǫh \\
%     \gll ag-e-ˀhowé:g-ǫh\\
%     \textsc{1s.p}-{\joinerE}-cover-\textsc{stat}\\
%     \glt ‘I did cover something’
%     \z
% \z


\ea\label{ex:verbex92} \stem{de-…-ęnahsgwahgw} ‘jump’
\ea deyǫna̱hsgwáhkwhaˀ\\
\gll de-yǫ-na̱hsgwáhkw-haˀ\\
 {\dualic}-\textsc{3s.fi.a}-jump-{\habitual}\\
\glt `she is jumping'
\ex dęyǫ́na̱hsgwahk\\\label{ex:verbex92b}
\gll d-ę-yǫ́-na̱hsgwahk\\
 {\dualic}-{\future}-\textsc{3s.fi.a}-jump.{\zeropunctual}\\
\glt `she will jump'
\ex desęna̱hsgwáhgwęh\\
\gll de-s-ę-na̱hsgwáhgw-ęh\\
 {\dualic}-\textsc{2s.a}-{\semireflexive}-jump-{\stative}\\
\glt `you have jumped'
\z
\z



\subsection{No-aspect verbs} \label{No-aspect verbs}
\textsc{No-aspect} verbs lack an aspect suffix. The example shown in \xref{ex:commandexample} is a type of command. In contrast, the comparative example in \xref{ex:commandexample} illustrates a verb with a punctual suffix, which functions as a statement. For the meaning of no-aspect verbs, see \sectref{Meaning of no-aspect verbs}.

\ea\label{ex:commandexample}
sadahǫ́:dǫ:\\\label{ex:commandexamplea}
\gll s-ad-ahǫ́:dǫ:\\
 \textsc{2s.a}-{\semireflexive}-ask.{\noaspect} (a command)\\
\glt `ask!'
\cfex{ęsada̱hǫ́:dǫ:ˀ\\
	\gll ę-s-ad-a̱hǫ́:dǫ:-ˀ\\
	\fut-\textsc{2s.a}-{\semireflexive}-ask-{\punctual} (a statement)\\
	\glt `you will ask'}
\z


\subsection{E-verbs} \label{Verbs of motion}
\textsc{E-verbs} \xref{ex:motionverbsagain} are different from the three-aspect verbs described earlier. The differences are too great to be summarized here and are covered in \sectref{ch:E-verbs}.

\ea\label{ex:motionverbsagain}
\ea í:geˀ ‘I am walking, moving’ 
\ex dagę́:neˀ ‘they (z) are coming’ 
\z
\z


\section{Verb classes and pronominal prefix type} \label{Verb classes and pronominal prefix type}
Verbs are also classified by the type of pronominal prefix they take \xref{ex:verbex80}. Verbs either take interactive (\textsc{int}-, \textsc{a}- and \textsc{p}-series) or non-interactive (\textsc{a}- or \textsc{p}-series) prefixes. The verbs taking non-interactive prefixes also subdivide into personal and neuter types \xref{ex:verbex80b}. (For definitions of \textsc{int}, \textsc{a}- and \textsc{p}-series prefixes, see \sectref{Three types of pronominal prefix}.) 

\ea\label{ex:verbex80}Verb classes, distinguished by pronominal prefix type
\ea\label{ex:verbex80a}\textsc{interactive} -- taking \textsc{int}-, \textsc{a}- and \textsc{p}-series prefixes
\ex\label{ex:verbex80b}\textsc{non-interactive} -- exclusively taking either \textsc{a}- or \textsc{p}-series prefixes 
\ea\label{ex:verbex80c}\textsc{personal} -- either taking all of the \textsc{a}-series prefixes or all of the \textsc{p}-series prefixes
\ex\label{ex:verbex80d}\textsc{neuter} -- taking only the neuter \stem{ga-} \textsc{3s.a} or \stem{(y)o-} \textsc{3s.p} prefix
\z
\z
\z



Interactive verbs \xref{ex:verbex80a} can take all of the interactive pronominal prefixes -- the \textsc{int} series \xref{ex:verbex90a}, the \textsc{a}-series \xref{ex:verbex90b} and the \textsc{p}-series \xref{ex:verbex90c}. (For this type of verb, the \textsc{a}- and \textsc{p}-series prefixes have a special interactive meaning, with an implied ‘it’ (shown in brackets in \ref{ex:verbex90}).

\newpage
\ea\label{ex:verbex90}
\ea\label{ex:verbex90a} a\exemph{gǫ́:}gęˀ ‘I saw you’ (singular) (\textsc{int} prefix)
\ex\label{ex:verbex90b} a\exemph{há:}gęˀ ‘he saw (it)’ (\textsc{a}-series prefix, used in an interactive sense with implied object ‘it’)
\ex\label{ex:verbex90c} a\exemph{hó:}gęˀ ‘(it) saw him’, ‘he was seen’ (\textsc{p}-series prefix, used in an interactive sense with implied subject ‘it’)
\z
\z

In contrast, non-interactive verbs \xref{ex:verbex80b} exclusively take either the \textsc{a}-series \xref{ex:verbex910a} or the \textsc{p}-series \xref{ex:verbex910b}. 

\ea\label{ex:verbex910}
\ea\label{ex:verbex910a}\exemph{ha}hnę́:ye:s ‘he is tall’ (\textsc{a}-series exclusively)
\ex\label{ex:verbex910b}\exemph{ho}hsę: ‘he is fat’ (\textsc{p}-series exclusively)
\z
\z


Non-interactive verbs \xref{ex:verbex80b} further divide into \textsc{personal} or \textsc{neuter} types. Personal verbs take a full range of \textsc{a}-series \xref{ex:verbex920a} or \textsc{p}-series \xref{ex:verbex920b} prefixes. In contrast, neuter verbs only take neuter ‘it’ prefixes \xref{ex:verbex930}. 


\ea\label{ex:verbex920} personal verbs 
\ea\label{ex:verbex920a} \exemph{ha}hnę́:ye:s ‘he is tall’, \exemph{e}hnę́:ye:s ‘she is tall’, \exemph{ga}hnę́:ye:s ‘it (animal) is tall’, \exemph{k}hnę́:ye:s ‘I am tall’ (etc.) (\textsc{a}-series exclusively)
\ex\label{ex:verbex920b} \exemph{ho}hsę: ‘he is fat’, \exemph{go}hsę: ‘she is fat’, \exemph{o}hsę: ‘it is fat’, \exemph{ag}áhsę: ‘I am fat’ (etc.) (\textsc{p}-series exclusively)
\z
\z


\ea\label{ex:verbex930} neuter verbs 
\ea \exemph{gá:}dę:s ‘it is thick, dense’ (\textsc{a}-series prefix)
\ex \exemph{w}agyé:sęh ‘it is easy’ (\textsc{a}-series prefix)
\ex \exemph{o}tó:weˀ ‘it is cold (weather)’ (\textsc{p}-series prefix)
\z
\z

That being said, neuter verbs can also take non-neuter, \textsc{p}-series prefixes, but only in order to denote the ownership of an incorporated noun (\ref{ex:verbex94}, see \sectref{Neuter stative-only verbs, incorporating, conveying possession}). 


\ea\label{ex:verbex94} \stem{+es} ‘long’, i:yǫ:s ‘it is long’ (\textsc{a}-series prefix)\\
\exemph{sa}nóˀje:s ‘your tooth is long’ (non-neuter \textsc{p}-series prefix, denoting possession)
\cfex{\exemph{o}nóˀjaˀ ‘tooth’ (\textsc{p}-series prefix)}
\z



\section{Verb classes (subdivided by aspect, pronominal prefix type)} \label{Verb classes (subdivided by aspect, pronominal prefix type)}
To summarize, verbs can be classified according to the criteria of noun incorporation, meaning (fixed or transparent), aspect, and pronominal prefix type. The main criteria used in this work are aspect and pronominal prefix type.\footnote{In the Verb dictionary (\sectref{verb dictionary}) information about noun incorporation and fixed expressions is also provided where relevant.} The verb classes defined by these criteria are summarized in \xxref{ex:verbex361}{ex:verbex362}. 

The stative-only and habitual-only verbs in \xref{ex:verbex361}, described earlier in \sectref{Verbs occurring only in one aspect (stative or habitual)}, either take \textsc{a}- or \textsc{p}-series prefixes exclusively. Their prefix type, together with the neuter vs. personal distinction, results in four sub-classes of verb.

\ea\label{ex:verbex361} stative-only or habitual-only verbs (including \textsc{NV} fixed expressions)\footnote{\textsc{NV} is an abbreviation for fixed expressions that require an \stem{incorporated noun-verb} combination.}
\ea with \textsc{a}-series prefixes
\ea \textsc{neuter} verbs (only taking \stem{ga-} \textsc{3s.a} or \stem{w(a)-} \textsc{3s.p} prefix)
\ex \textsc{personal} verbs (taking any personal prefix)
\z
\ex with \textsc{p}-series prefixes
\ea \textsc{neuter} verbs (only taking \stem{o-} \textsc{3s.p} prefix)
\ex \textsc{personal} verbs (taking any personal prefix)
\z
\z
\z

The three-aspect verbs in \xref{ex:verbex362}, described earlier in \sectref{Three-aspect verbs (habitual, punctual, stative)}, fall into two major types. The first type \xref{ex:verbex362a} takes \textsc{a}-series prefixes in the habitual and punctual, but \textsc{p}-series prefixes in the stative. The second type \xref{ex:verbex362b} takes \textsc{p}-series prefixes in all three aspects. The pronominal prefix type, together with the neuter, personal, and interactive distinction, results in the following sub-classes in \xref{ex:verbex362}. 

\ea\label{ex:verbex362} three-aspect verbs (including \textsc{NV} fixed expressions)
\ea with \textsc{a}-series prefixes in the habitual and punctual and \textsc{p}-series prefixes in the stative\label{ex:verbex362a}
\ea \textsc{neuter} verbs
\ex \textsc{personal} verbs
\ex \textsc{interactive} verbs
\z
\ex with \textsc{p}-series prefixes in the habitual, punctual, and stative\label{ex:verbex362b}
\ea \textsc{neuter} verbs 
\ex \textsc{personal} verbs
\ex \textsc{interactive} verbs
\z
\z 
\z

Pronominal prefix choice is described further in \sectref{Pronominal prefix choice for stative-only verbs} (stative-only and habitual-only verbs) and in \sectref{Pronominal prefix choice for three-aspect verbs} (three-aspect verbs).

Finally, for \textsc{e-verbs}, see \sectref{ch:E-verbs}. 


