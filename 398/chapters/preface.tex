\addchap{Foreword}

The \textit{Gayogo̱hó:nǫˀ} were a highly evolved \textit{Ǫgwehǫ́:weh} civilization that gave birth to a social democratic society and confederacy of five distinct Nations, \textit{Hwíhs Niyǫhwęjá:ge:}. \textit{Ǫgwehǫ́:weh} translates spiritually as the Divine People, having descended from the Sky World as Star People.

The social and spiritual order had women in the centre of this universe in a matrilineal society – a dichotomy, opposite to that of patrilineal European societies. \textit{Ǫgwehǫ́:weh} theology centred on this social and spiritual order. The \textit{Ǫgwehǫ́:weh} civilization established this highly evolved social democracy and confederation of nations when the rest of the world were led by kings, queens and popes.

The \textit{Ǫgwehǫ́:weh} had in their theology that God or the Creator loved all, with no fear of God. Their theology did not possess the abstract concepts of the fear of God, original sin, baptism, a devil with cloven hoofs; and the idea that men would be damned to hard labour every six days for their \emph{original sin} and, women would suffer with hard labour when delivering children, for their original sin. The children were sacred and cherished, with no concept of having been born with or from sin. \textit{Ǫgwehǫ́:weh} children did not have to be baptized because of sin. Women possessed stature and agency, simply because like the Creator they could bring forth life. Man does not possess this biological trait.

The \textit{Ǫgwehǫ́:weh} social democracy became known as the \emph{Haudenosaunee} Confederacy (\textit{Hodinǫ̱sǫ́:nih}, or ‘Men who build Longhouses’). Their language carried all the traits and characteristics of their civilization, including virtues, ethics, emotional intelligence (\emph{EQ-i}), linguistic sovereignty, horticulture, a moral compass, philosophies, psychologies, earth sciences, a cosmology, biology, homeopathic sciences, the arts (music, dance, story telling, visual art, etc.), prophecies, taboo manners and protocols, highly evolved rituals for mortal thanksgiving to the Creator, knowledges of biodiversity, respect for the ecology / horticulture / agriculture and its conservation, rituals for naming, and the idea that at the time of ‘passing on’ or death, the people returned home to be with family and to be with the Creator.

This Social Democracy evolved in the Finger Lake region of what is now New York State in the USA. Geographically, these Five Nations all lived around a Finger Lake in a peaceful existence, governed by the protocols of a highly evolved democracy and a sacred trust in providence with the Creator. After the Five Nation confederation, they had no use for a militia and army. They were living in peace at the time of Contact. They are truly Divine Star People... (my bias).\\\\
\noindent --- Dehaeho:węhs / Taeho:węhs / Amos Key Jr.


\addchap{Preface}

In 1993, the Government of Ontario’s Ministry of Education provided funding that led to a series of Iroquoian dictionaries and grammars. The Woodland Cultural Centre (Brantford, Ontario, Canada) undertook a \textit{Gayogo̱honǫˀnéha:ˀ} (Cayuga) dictionary and grammar project, under the guidance of the then-Director of the Language Program, Amos Key, Jr. I was invited to join as a project linguist and undertook fieldwork with the co-authors – Frances Froman, Lottie Keye, and Alfred Keye – mainly between 1993 and 2002. I inherited and added to a Gayogo̱honǫˀnéha:ˀ database, originally curated by Cindy John (former Coordinator with the Language Program at the Woodland Cultural Centre). The database eventually became a dictionary, (\citep{froman_english-cayugacayuga-english_2002}), and is also the basis for this grammar. The latter took longer to produce, in part because I needed to rethink the grammar’s organization. 

\begin{figure}
\begin{subfigure}{0.3\textwidth}
\includegraphics[height=3cm]{figures/FranFroman.jpg}
\caption{Frances Froman}
\end{subfigure}\hfill
\begin{subfigure}{0.3\textwidth}
\centering
\includegraphics[height=3cm]{figures/AlfredKeye.jpg}
\caption{Alfred Keye}
\end{subfigure}\hfill
\begin{subfigure}{0.3\textwidth}
\includegraphics[height=3cm]{figures/LottieKeye.jpg}
\caption{Lottie Keye}
\end{subfigure}
\caption{The co-authors}
\end{figure}



My main contribution to this grammar is to describe not only form or structure, but also function or usage: with respect to form, Gayogo̱honǫˀnéha:ˀ has just three classes of words: \textsc{nouns}, \textsc{verbs}, and \textsc{particles}. However, the three word classes have many more functions in Gayogo̱honǫˀnéha:ˀ: for example, particles can function as “pronouns”, “adverbs”, “conjunctions”, “relativizers”, “degree words”, “question words”, “evidentials”, “topic managers”, and more. I have organized much of the grammar according to word function, which is why, for example, the grammar has a section on words that function like “adjectives”, although Gayogo̱honǫˀnéha:ˀ does not have a word class of adjective (see \sectref{ch:Defining nouns, verbs, particles}).

While the present grammar is founded on the intellectual contributions of Frances Froman, Lottie Keye, and Alfred Keye, they did not intend to make any claims on behalf of other speakers. Nor would they necessarily agree with my analyses. For example, while I enjoy making generalizations or rules, I have also observed that Gayogo̱honǫˀnéha:ˀ speakers place equal weight on exceptions. Respecting this value, I have tried to comment on exceptions, mysteries, and areas requiring further investigation where relevant, especially since their presence can lead to a rethinking of any generalizations. I hope that the Gayogo̱honǫˀnéha:ˀ examples are enough in number and organized well enough to be treated as the ultimate authority. (The Gayogo̱honǫˀnéha:ˀ language examples without explicit attribution are from my fieldwork notes with the coauthors.)

Speaking of sources, I also relied on other examples of natural or constructed speech, including audio transcripts and translations of stories and conversations. Written transcripts and translations of Gayogo̱honǫˀnéha:ˀ are a new type of source not available even forty years before the publication of this grammar. One of the principal sources is \citet{mithun_watewayestanih_1984}, which has a rich set of conversations constructed by the late Reg Henry and by Dr. Marianne Mithun. Other sources are stories written or translated by Lottie Keye (\cite{keye_hnyagwaidatgigowah_2012}, \cite{keye_circle_2016}), and Gayogo̱honǫˀnéha:ˀ legends from a project with the Canadian Broadcasting Corporation (see \cite{carrier_legends_2013}). I also relied on several unpublished grammatical descriptions, including \citet{foster_course_1993}, \citet{sasse_far_1998}, and \citet{michelson_ontario_2011}.

Part~\ref{introduction} of the grammar is a brief introduction to Gayogo̱honǫˀnéha:ˀ, spelling, and sounds. Part~\ref{Word Classification} describes the three Gayogo̱honǫˀnéha:ˀ word classes – \textsc{nouns}, \textsc{verbs}, and \textsc{particles} – and also the functions (labelled as major “parts of speech”) performed by these word classes. In Part~\ref{Word Classification}, (since it is an introductory section), words are not segmented into morphemes (word parts), insofar as possible. Morphemes are instead covered in Part~\ref{Word Creation}, which describes how \textsc{prefixes}, \textsc{suffixes}, and \textsc{stems} combine in word creation. 


Part~\ref{Pronominal prefixes} describes \textsc{pronominal prefixes} separately, since pronominal prefix selection and pronunciation are large topics. Part~\ref{Sentences} describes how words combine into sentences, with a focus on describing the fixed ordering of clause-initial particles. Part~\ref{ch:Discourse and discourse signposts} explores \textsc{discourse} - how sentences combine into larger units, in particular examining how word order, noun incorporation, and particles convey information in discourse or connected speech. Part VII contains some technical (linguistic) notes, which were removed from the main text to avoid detracting from the flow of the descriptions of Gayogo̱honǫˀnéha:ˀ. Part~\ref{appendices} (Appendices) includes a \hyperlink{ch:noun dictionary}{noun dictionary}, \hyperlink{verb dictionary}{verb dictionary}, and \hyperlink{ch:particle dictionary}{particle dictionary}, organized according to the systems of classification outlined in the previous parts of the grammar (particularly Parts \ref{Word Classification} and \ref{Word Creation}). Part~\ref{appendices} ends with a \hyperlink{thematic dictionary}{thematic dictionary}. The dictionaries in Part~\ref{appendices} were separated out from the previous sections to improve the flow in Parts~\ref{introduction}--\ref{Word Creation}, and so that teachers and learners would have useful collections of organized Gayogo̱honǫˀnéha:ˀ words.

Gayogo̱honǫˀnéha:ˀ pronunciation and sound changes are described where most relevant throughout the grammar, beginning with a basic introduction in \chapref{sounds and spelling}, followed by sound changes in word formation, \chapref{Sound changes in word formation}, pronunciation of prepronominal combinations, \chapref{Pronunciation of prepronominal combinations}, pronunciation of the post-pronominal prefixes, \chapref{Pronunciation of the post-pronominal prefixes}, and pronominal prefix pronunciation, \chapref{Pronominal prefix pronunciation}. Much more could be said on the topic of Gayogo̱honǫˀnéha:ˀ phonology. 

Parts~\ref{introduction}--\ref{Pronominal prefixes} of the grammar are the best-developed sections. However, Sections~\ref{Aspect suffixes}--\ref{Post-aspect (tense) suffixes} (on aspect), \chapref{ch:E-verbs} (on \textsc{e-verbs} and the \textsc{purposive}), and \chapref{ch:Negation} (on \textsc{negation}) in Part~\ref{Word Creation}, really require further investigation. As well, Parts~\ref{Sentences} (sentence formation or syntax) and~\ref{ch:Discourse and discourse signposts} (discourse are just the tip of the iceberg. I hope that teachers and postsecondary learners find this grammar to be a useful starting point for further investigation.\\\\
\noindent --- Carrie Dyck, 2023
