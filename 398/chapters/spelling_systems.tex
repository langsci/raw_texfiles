\chapter{Spelling systems} \label{Spelling systems}
There are two spelling systems for Gayogo̱honǫˀnéha:ˀ -- the Linguistic and the Henry orthographies. The Linguistic system was originally developed in the 1970s for Native Language Teacher Certification programs, and is used in academic articles and in major works such as \citet{mithun_watewayestanih_1984}. The Linguistic system more closely resembles the orthographies of the other \textit{Ǫgwehǫ́:weh} languages. Some examples of other orthographies are shown in \tabref{figtab:1:twofive} (page \pageref{figtab:1:twofive}). 

The Henry orthography was developed in the early 1980s by the late Reginald Henry (credited as being the “father” of language revitalization at Six Nations). It is used in educational settings at Six Nations, in \citet{froman_english-cayugacayuga-english_2002}, and in this grammar. The main difference between the two writing systems is in how consonants are represented, as shown in \tabref{figtab:1:consspell}. 

\begin{table}
\caption{Consonant spellings compared\label{figtab:1:consspell}}
\fittable{
\begin{tabular}{lllll}
\lsptoprule
Linguistic & Henry & IPA & gloss & location\\
\midrule
\exemph{th}ęˀ & \exemph{t}ęˀ & \phonet{tʰ} & ‘no’ & not word-final\\
\exemph{t}aku:s & \exemph{d}agu:s & \phonet{d} & ‘cat’ & not word-final\\
ská:\exemph{t} & sgá:\exemph{t} & \phonet{tʰ} & ‘one’ & word-final\\
\exemph{t}káhe:ˀ & \exemph{t}gáhe:ˀ & \phonet{tʰ} & ‘it is setting 

over there’ & before a consonant\\
ka\exemph{kh}waˀ & ga\exemph{k}waˀ & \phonet{kʰ} & ‘food’ & not word-final\\
\exemph{k}ęh & \exemph{g}ęh & \phonet{g} & ‘question word’ & not word-final\\
khehá:wah\exemph{k} & kehá:wah\exemph{k} & \phonet{kʰ} & ‘my daughter’ & word-final\\
\exemph{k}nó:haˀ & \exemph{k}nó:haˀ & \phonet{kʰ} & ‘mother’ & before a consonant\\
\exemph{sh}ęh & \exemph{s}ęh & \phonet{sʰ} & ‘that’ & anywhere\\
\exemph{tsh}ǫ: & \exemph{ts}ǫ: & \phonet{tsʰ} & ‘just’ & anywhere\\
\exemph{tsy}á:tahk\footnote{Also spelled as \exemph{ts}á:tahk in the Linguistic orthography.} & \exemph{j}á:dahak & \phonet{dʒ, dz} & ‘seven’ & anywhere\\
\lspbottomrule
\end{tabular}
}
\end{table}

\section{A modified Henry orthography} \label{A modified Henry orthography}
A modified version of the Henry orthography is used in this book. This section describes the departures from the original orthography.\footnote{Also, language examples originally written in the linguistic orthography have been changed to the modified Henry spelling system in this grammar.} 

The original spelling system has a “lengthener” (length mark) <:> both after long vowels and double vowels \xref{ex:mod2a}. In contrast, the modified orthography does not use a lengthener after double vowels \xref{ex:mod2b}, but only after long or lengthened vowels \xref{ex:mod3c}.

\ea\label{ex:mod2} 
\ea\label{ex:mod2a} deg\exemph{aǫ:}doˀjinéhtaˀ ‘they figure-skate’ (original)
\ex\label{ex:mod2b} deg\exemph{aǫ}doˀjinéhtaˀ ‘they figure-skate’ (modified)
\z
\z

The original (Henry and Linguistic) systems have two ways of spelling accent-related syllable pronunciations (which are described in \pageref{ch:Pronunciation changes in sentences}). The original orthographies switch the glottal stop <ˀ> and vowel in certain syllables ending with glottal stop <ˀ> \xref{ex:mod3a}, but underline the vowel in certain syllables ending with H \xref{ex:mod3c}. In contrast, the modified orthography uses the underline convention for both types of pronunciation change (\ref{ex:mod3}b, c). (The result is that words belonging to the same paradigm are spelled the same way. For paradigm, see \pageref{Paradigm, inflection, and conjugation}.)

\ea\label{ex:mod3} 
\ea\label{ex:mod3a}\exemph{gˀa}nígǫ̱haˀ ‘the mind’ (original)
\ex\label{ex:mod3b}\exemph{ga̱ˀ}nígǫ̱haˀ (modified)
\ex\label{ex:mod3c}\exemph{ga̱}hǫ́:ga:ˀ ‘elm’ (original and modified)
\z
\z
One of the main differences between the Henry and Linguistic orthographies is the use of H in consonant spellings. For example, Linguistic <th> corresponds to Henry <t>, and Linguistic <sh>, to Henry <s>, \tabref{figtab:1:consspell}, \xref{ex:mod4a}. In this grammar, an extra <h> is added in the linguistic analysis for clarity, where relevant (\ref{ex:mod4}b, c). 

The accent mark is also omitted from stand-alone particles \xref{ex:mod4b}. 

\ea\label{ex:mod4}
\ea\label{ex:mod4a} sę́h ‘that’ (original)
\ex\label{ex:mod4b} shęh ‘that’ (modified)\footnote{\textit{Shęh} is also spelled and pronounced as \textit{tsęh} (Henry orthography), \textit{tshęh} (Linguistic orthography.)}
\ex\label{ex:mod4c} ehstaˀ \\
\gll e-hst-haˀ\\
\textsc{3fis.a}-use-{\habitual}\\
\glt ‘she uses’, ‘people use’ (H is added in the linguistic analysis)
\z
\z 



\section{Spelling particles} \label{Spelling particles}
\textsc{Particles} (see \sectref{ch:Defining nouns, verbs, particles}) often undergo pronunciation changes (which are described in \pageref{ch:Accenting particles and particle groups in sentences}.) As a result, particles and particle groups are typically spelled several different ways. To standardize the spelling, particles are spelled in their full or stand-alone form in this work, as shown in \tabref{figtab:1:particlepronun}. (Often, however, the common spellings for particle groups are also shown.) In spelling particles, the accent mark is often omitted, partly because it is less prominent in particles and particle groups (\cite{rueentan_prosody_2014}). 

\begin{table}
\caption{Particle pronunciation}
\label{figtab:1:particlepronun}
\begin{tabularx}{\textwidth}{lll}
\lsptoprule
stand-alone (modified) 

spelling & common spellings & translation \\
\midrule
haoˀ dęˀ nyoh & haoˀ dę̱ˀnyó & “o.k.!”\\
toh gęh ǫh ne:ˀ & to gęǫ ne:ˀ, to gaǫ ne:ˀ & “I wonder if it is” \\
\lspbottomrule
\end{tabularx}
\end{table}


\section{Brackets}

The bracket conventions used in this work are shown in \tabref{figtab:1:bracket_usage}.
\begin{table}
\caption{Bracket usage}
\label{figtab:1:bracket_usage}
\begin{tabularx}{\textwidth}{XXX}
\lsptoprule
bracket type & usage & example \\
\midrule
\phonet{…} & International Phonetic Alphabet (IPA) & [tʰ]\\
\midrule
\stem{…} & underlying representation, 

allomorph, 

spelling variant & \stem{a-}, \stem{aˀ-}, \stem{e-} {\factual} \\
\midrule
<…> & spelling variant & <ˀ> glottal stop

<s>, S (alternative)\\
\lspbottomrule
\end{tabularx}
\end{table}

Symbols from the International Phonetic Alphabet (IPA) are enclosed between square brackets \phonet{…}. 

Morphemes and allomorphs are enclosed between floor brackets \stem{…} instead of the usual slash brackets /…/ mainly to avoid the extensive use of slash brackets. As well, unlike slash brackets, the floor brackets represent not only morphemes, but also allophonic and spelling variants. For example, the reader might see three \textsc{factual} allomorphs referred to in the text, \stem{a-}, \stem{aˀ-}, or \stem{e-}. Such representations more closely align with the spelling and also serve to avoid unnecessarily detailed underlying representations such as /a(ˀ)-, e-/ -- or even “archiphonemic” representations such as /A(ˀ)-/. The use of floor brackets for stems such as \stem{ihsag} ‘look for’ also signals the potential for spelling changes such as \textit{agíhsa\textbf{g}ǫh} ‘I’m looking for it’ versus \textit{ęhsíhsa:\textbf{k}} ‘you will look for it’.

Orthographic representations are sometimes enclosed between angle brackets for clarity (often just to highlight the presence of a glottal stop <ˀ>). Otherwise, uppercase letters are used for denoting orthographic representations.
