\chapter{Comparisons, counting, measuring} \label{ch:Comparisons, counting, measuring}
This chapter describes the various phrases used for making comparisons, counting objects, and measuring. (For related information, see \textit{Pluralizing} \sectref{ch:Pluralizing}, \textit{Numbers and money}, \sectref{numbers and money}.)

\section{Comparisons (more, the same, or less)} \label{ch:Comparisons (more, the same, or less)}
Comparisons apply to qualities or attributes, and convey concepts like degrees of sameness or difference \xref{ex:comparisonclause}. Comparative phrases typically include stative-only verbs beginning with \stem{ni-} \textsc{\partitive}, which function as “adjectives”. 

\ea\label{ex:comparisonclause}
\gll Aǫhę:ˀę́h hniˀ gyogaˀǫ́h shęh gaoˀ nitode̱ˀnyę́:dę: \\
the.most and it.tastes.good that even.so so.much.he.had.tasted \\
\glt ‘And it was sweeter than anything he had ever tasted.’ (\cite{carrier_legends_2013})
\cfex{\gll ni-t-ho-d-e̱-ˀnyę́:dę-:\\
    {\partitive}-{\cislocative}-\exsc{3s.m.p-{\semireflexive}}-{\joinerE}-try-{\stative}\\
    }
\z

Comparative constructions are described in the following sections.

\begin{CayugaRelated}
\item{}“Adjectives” (words that function as adjectives), \sectref{ch:’Adjectives’ (words that function as adjectives)}

\item{}\stem{tsaˀde-verb}, \sectref{[tsaˀde-verb]}

\item{}\stem{tsi-\textsc{incorporated noun}-verb} ‘be the same kind of \textsc{incorporated noun}’, \sectref{[tsi-incorporated noun-verb]}

\item{}\stem{ti-\textsc{incorporated noun}-deˀ(ah)} ‘a different \textsc{incorporated noun}’, \sectref{[ti-(incorporated noun)-deˀ(ah)]}

\item{}\stem{ni-} (partitive) with \textsc{verbs} describing measures or degrees, \sectref{[ni-] (partitive) with verbs describing measures or degrees}

\item{}\stem{d-pronominal prefix-stative.verb} ‘-est’, \ref{[d+pronominal prefix-stative.verb]}

\end{CayugaRelated}

Particles used in comparative constructions are listed below and are described in the \textit{Particle dictionary}, \sectref{ch:particle dictionary}.

\begin{CayugaRelated}


\item{}\textit{Gaoˀ ni-} \trs{less so}

\item{}\textit{Gwahs shęh ni-} \trs{as…as}

\item{}\textit{Heyohe:ˀ shęh } \trs{more than}

\item{}\textit{I:nǫh gaoˀ neˀ} \trs{far from enough}

\item{}\textit{Ji trehs shęh} \trs{too much so for}, \trs{so much so}

\item{}\textit{Neˀ aǫhę:ˀęh d-} \trs{the most, -est, -er (of)}

\item{}\textit{Neˀ gyaǫhę:ˀęh d-} \trs{the most}, \trs{the greatest}

\item{}\textit{Neˀ heyohe:ˀ} \trs{more, -er}

\item{}\textit{Ne:ˀ gwahs d-} \trs{the most}

\item{}\textit{Ne:ˀ gwahs heyohe:ˀ} \trs{the most, -est}

\item{}\textit{Shęh niyo:weˀ} \trs{as far as}, \trs{as much as}

\item{}\textit{Toh niyo:weˀ} \trs{that far} 

\item{}\textit{Toh shęh ni-} \trs{to that degree}
\end{CayugaRelated}

\section{Counting with basic nouns} \label{ch:Counting with basic nouns}
Counting with basic nouns is summarized next. (For basic nouns, see \sectref{ch:Basic nouns}.) 


\subsection{Counting one object} \label{ch:Counting one object}
The template for the verb meaning ‘one object’ is shown in (\ref{ex:countingex}, also see \sectref{[s/j/ji-pronominal prefix-incorporated noun-t]}). 

\ea\label{ex:countingex} \stem{\exsc{s-pronominal prefix}-\textsc{incorporated noun}-{\joinerA}-t}
\z

For basic nouns beginning with \stem{ga-} \textsc{3s.a}, the corresponding counting verb begins with \stem{s-ga-} \textsc{\repetitive-3s.a} \xref{ex:countingex2}.

\ea\label{ex:countingex2} sganáˀja:t\\
\gll s-ga-náˀj-a:-t\\
 {\repetitive}-\textsc{3s.a}-drum-{\joinerA}-stand.{\stative}\\
\glt `one bottle'
\cfex{ganáˀja:t\\
	\gll ga-náˀj-a:-t\\
	\textsc{3s.a}-drum-{\joinerA}-stand.{\stative}\\
	\glt `a standing bottle'}
\z


Basic nouns beginning with \stem{o-} \textsc{3s.p} are less straightforward: for some, the corresponding counting verb begins with \stem{s-ga-} \textsc{\repetitive-3s.a} \xref{ex:countingex3}. However, for others, the corresponding counting verb begins either with \stem{s-ga-} \textsc{\repetitive-3s.a} or \stem{j-o-} \textsc{\repetitive-3s.p} \xref{ex:countingex4}. For yet another set of nouns beginning with \stem{o-}, the counting verb can only begin with \stem{j-o-} \xref{ex:countingex5}.

\ea\label{ex:countingex3} sgayę́hsra:t\\
\gll s-ga-yę́hsr-a:-t\\
 {\repetitive}-\textsc{3s.a}-blanket-{\joinerA}-stand.{\stative}\\
\glt `one blanket'
\cfex{oyę́hsraˀ\\
	\gll o-yę́hsr-aˀ\\
	\textsc{3s.p}-blanket-{\nounstemformer}\\
	\glt `blanket'}
\z


\ea\label{ex:countingex4}
\ea sgaˀnhǫ́hsa:t\\
\gll s-ga-ˀnhǫ́hs-a:-t\\
 {\repetitive}-\textsc{3s.a}-egg-{\joinerA}-stand.{\stative}\\
\glt `one egg'
\ex joˀnhǫ́hsa:t\\
\gll j-o-ˀnhǫ́hs-a:-t\\
 {\repetitive}-\textsc{3s.p}-egg-{\joinerA}-stand.{\stative}\\
\glt `one egg'
\cfex{oˀnhǫ́hsaˀ\\
	\gll o-ˀnhǫ́hs-aˀ\\
	\textsc{3s.p}-egg-{\nounstemformer}\\
	\glt `egg'}
\z
\z

\ea\label{ex:countingex5} 
joháha:t\\
\gll j-o-háh-a:-t\\
 {\repetitive}-\textsc{3s.p}-road-{\joinerA}-stand.{\stative}\\
\glt `one road'
\cfex{oháhaˀ\\
	\gll o-háh-aˀ\\
	\textsc{3s.p}-road-{\nounstemformer}\\
	\glt `road'}
\z


For basic nouns beginning with [a], the corresponding counting verb begins with \stem{s-wa-} \textsc{\repetitive-3s.a} \xref{ex:countingex6}.

\newpage
\ea\label{ex:countingex6} 
swanaháotra:t\\
\gll s-w-anaháotr-a:-t\\
 {\repetitive}-\textsc{3s.a}-hat-{\joinerA}-stand.{\stative}\\
\glt `one hat'
\cfex{anaháotraˀ\\
	\gll anaháotr-aˀ\\
	hat-{\nounstemformer}\\
	\glt `hat'}
\z


At least one \textsc{instrumental noun} (\sectref{ch:Instrumental nouns}) can be enumerated just like basic nouns beginning with \stem{ga-} (\ref{ex:countingex7}).

\ea\label{ex:countingex7} 
sgahyádǫ̱hkwa:t\\
\gll s-ga-hyádǫ̱hkw-a:-t\\
 {\repetitive}-\textsc{3s.a}-pencil-{\joinerA}-stand.{\stative}\\
\glt `one pencil'
\cfex{ehyádǫ̱hkwaˀ\\
	\gll e-hyádǫ̱-hkwaˀ\\
	\exsc{she/someone.a}-write-{\instrumental}\\
	\glt `pencil'}
\z




\subsection{Counting two objects} \label{ch:Counting two objects}
The template for the verb meaning ‘two objects’ is shown in (\ref{ex:countingex800}, also see \sectref{[de+pronominal prefix+incorporated noun+age:]}).

\ea\label{ex:countingex800} \stem{\exsc{de-pronominal prefix}-\textsc{incorporated noun}-age:} ‘two \textsc{incorporated nouns}’
\z

For basic nouns beginning with \stem{ga-} \textsc{3s.a}, the corresponding counting verb begins with \stem{de-ga-} \textsc{\dualic-3s.a} \xref{ex:countingex900}.

\ea\label{ex:countingex900}
degaji̱hsdá:ge:\\
\gll de-ga-ji̱hsd-á:ge:\\
 {\dualic}-\textsc{3s.a}-lamp-certain.number.{\stative}\\
\glt `two lamps'
\cfex{gajíhsdaˀ\\
	\gll ga-jíhsd-aˀ\\
	\textsc{3s.a}-lamp-{\nounstemformer}\\
	\glt `lamp'}
\z


For basic nouns beginning with \stem{o-} \textsc{3s.p}, the corresponding counting verb begins with \stem{de-yo-} \textsc{\dualic-3s.p} \xref{ex:countingex101}.

\ea\label{ex:countingex101}
\ea deyoˀnǫ̱hsá:ge:\\
\gll de-yo-ˀnǫ̱hs-á:ge:\\
 {\dualic}-\textsc{3s.p}-egg-be.a.certain.number.{\stative}\\
\glt `two eggs'
\cfex{oˀnhǫ́hsaˀ\\
	\gll o-ˀnhǫ́hs-aˀ\\
	\textsc{3s.p}-egg-{\nounstemformer}\\
	\glt `egg'}
\ex deyohsi̱ˀdá:ge:\\
\gll de-yo-hsi̱ˀd-á:ge:\\
 {\dualic}-\textsc{3s.p}-foot-be.a.certain.number.{\stative}\\
\glt `two feet'
\cfex{ohsíˀdaˀ\\
	\gll o-hsíˀd-aˀ\\
	\textsc{3s.p}-foot-{\nounstemformer}\\
	\glt `foot'}
\z
\z

For basic nouns beginning with [a], the corresponding counting verb begins with \stem{de-wa-} \textsc{\dualic-3s.a} \xref{ex:countingex110}.

\ea\label{ex:countingex110}
dewahda̱hgwá:ge:\\
\gll de-wa-hda̱hgw-á:ge:\\
 {\dualic}-\textsc{3s.a}-shoe-be.a.certain.number.{\stative}\\
\glt `two shoes'
\cfex{ahdáhgwaˀ\\
	\gll ahdáhgw-aˀ\\
	shoe-{\nounstemformer}\\
	\glt `shoe'}
\z



\subsection{Counting three or more objects} \label{ch:Counting three or more objects}
The template for the verb meaning ‘three or more objects’ is shown in (\ref{ex:countingex}, also see \sectref{Numeral + [ni-incorporated noun-age:]}).

\ea\label{ex:countingex12} \stem{number}  \stem{\exsc{ni-pronominal prefix}-\textsc{incorporated noun}-age:} ‘number of \textsc{incorporated nouns}’
\z

For basic nouns beginning with \stem{ga-} \textsc{3s.a}, the corresponding counting verb begins with \stem{ni-ga-} \textsc{\partitive-3s.a} \xref{ex:countingex13}.

\newpage
\ea\label{ex:countingex13}
ahsęh niganǫ̱hsá:ge:\\
\gll ahsęh ni-ga-nǫ̱hs-á:ge:\\
three {\partitive}-\textsc{3s.a}-house-be.a.certain.number.{\stative}\\
\glt `three houses'
\cfex{ganǫ́hsaˀ\\
	\gll ga-nǫ́hs-aˀ\\
	\textsc{3s.a}-house-{\nounstemformer}\\
	\glt `house'}
\z


Basic nouns beginning with \stem{o-} \textsc{3s.p} are less straightforward: for some, the corresponding counting verb begins with  \stem{ni-yo-} \textsc{\partitive-3s.p} \xref{ex:countingex14}. For others, the corresponding counting verb begins with either \stem{ni-yo} \textsc{\partitive-3s.p} or \stem{ni-ga} \textsc{\partitive-3s.a} \xref{ex:countingex15}.

\ea\label{ex:countingex14} 
ahsęh niyohsi̱ˀdá:ge:\\
\gll ahsęh ni-yo-hsi̱ˀd-á:ge:\\
three {\partitive}-\textsc{3s.p}-foot-be.a.certain.number.{\stative}\\
\glt `three feet'
\cfex{ohsíˀdaˀ\\
	\gll o-hsíˀd-aˀ\\
	\textsc{3s.p}-foot-{\nounstemformer}\\
	\glt `feet'}
\z


\ea\label{ex:countingex15} 
\ea ahsęh niyohǫnáˀdage:\\
\gll ahsęh ni-yo-hǫnáˀd-age:\\
three {\partitive}-\textsc{3s.p}-potato-be.a.certain.number.{\stative}\\
\glt `three potatoes'
\ex ahsęh nigahǫnáˀdage:\\
\gll ahsęh ni-ga-hǫnáˀd-age:\\
three  {\partitive}-\textsc{3s.a}-potato-be.a.certain.number.{\stative}\\
\glt `three potatoes'
\cfex{ohǫ́na̱ˀdaˀ\\
	\gll o-hǫ́na̱ˀd-aˀ\\
	\textsc{3s.p}-potato-{\nounstemformer}\\
	\glt `potatoes'}
\z
\z

For basic nouns beginning with [a], the corresponding counting verb begins with \stem{ni-wa} \textsc{\partitive-3s.a} \xref{ex:countingex16}.

\newpage
\ea\label{ex:countingex16}
ahsęh niwahda̱hgwá:ge:\\
\gll ahsęh ni-w-ahda̱hgw-á:ge:\\
three {\partitive}-\textsc{3s.a}-shoe-be.a.certain.number.{\stative}\\
\glt `three shoes'
\cfex{ahdáhgwaˀ\\
	\gll ahdáhgw-aˀ\\
	shoe-{\nounstemformer}\\
	\glt `shoes'}
\z



\section{Counting words that are not basic nouns} \label{ch:Counting words that are not basic nouns}
Several other strategies, described next, are used for counting words that are not basic nouns. (Such alternative counting strategies are needed because most words functioning as “nouns” cannot be incorporated, unlike the basic nouns described earlier).


\subsection{Counting with placeholder words and placeholder incorporated nouns} \label{ch:Counting with placeholder words and placeholder incorporated nouns}
Many nouns, and words functioning as “nouns”, cannot be incorporated into the counting verbs described earlier. In such cases, alternative methods are used instead, as shown in \xxref{ex:countplaceholdex8}{ex:countplaceholdex9}. For these constructions, the counting verbs (described previously) are still employed. However, they do not incorporate the noun that is actually counted. Instead, they incorporate a placeholder noun, which has a generic meaning. The enumerated noun appears after the verb, as a stand-alone word.

\ea\label{ex:countplaceholdex8} \stem{\exsc{s-pronominal prefix}-\textsc{placeholder.incorporated.noun}-t \textsc{specific.noun}} ‘one \textsc{specific noun}’
\ea swa̱hdahgwá:t ahdahgwáǫweh\\
\gll s-wa̱-hdahgw-á:-t ahdahgwá-ǫweh\\
 {\repetitive}-\textsc{3s.a}-shoe-{\joinerA}-stand.{\stative}  {\noprefix}-shoe-{\typicalizer}\\
\glt `one ceremonial shoe'
\ex sganahsgwá:t dagus\\
\gll s-ga-nahsgw-á:t dagus\\
 {\repetitive}-\textsc{3s.a}-tame.animal-{\joinerA}-stand.{\stative} cat\\
\glt `one pet cat'
\z
\z

\newpage
\ea\label{ex:countplaceholdex9} \stem{\exsc{de-pronominal prefix}-\textsc{placeholder.incorporated noun}-age: \textsc{specific.noun}} ‘two \textsc{specific nouns}’\\
degrahe̱ˀda:gé: ohó:draˀ\\
\gll de-grahe̱ˀd-a:gé: o-hó:dr-aˀ\\
 {\dualic}-tree-be.a.certain.number.{\stative}  \textsc{3s.p}-basswood-{\nounstemformer}\\
\glt `two basswood trees'
\z


The special verbs used for counting living things (“nouns”) are illustrated in \xxref{ex:countplaceholdex3}{ex:countplaceholdex5}. 

\ea\label{ex:countplaceholdex3} \stem{\exsc{s-pronominal prefix}-yaˀd-{\joinerA}-t [living.thing]}  ‘one [living.thing]’
\ea sgayaˀdá:t dagu:s\\
\gll s-ga-yaˀd-á:-t dagu:s\\
 {\repetitive}-\textsc{3s.a}-body-{\joinerA}-stand.{\stative} cat\\
\glt `one cat'
\ex sgayaˀdá:t jogrihs ‘one blackbird’
\ex sgayaˀdá:t ohtahyǫ́:ni: ‘one wolf’
\ex sgayaˀdá:t twę́:twę:t ‘one duck’
\ex sgayaˀdá:t degaya̱ˀdáhkwaˀ ‘one hen hawk’
\z
\z

\ea\label{ex:countplaceholdex4} \stem{\exsc{de-pronominal prefix}-yahshe: [living.thing]}  ‘two [living.things]’\\
degadiyahshé: dagu:s\\
\gll de-gadi-yahshé: dagu:s\\
 {\dualic}-\textsc{3p.a}-two.living.things.{\stative} cat\\
\glt `two cats'
\z


\ea\label{ex:countplaceholdex5} [\textsc{number}] \stem{\exsc{ni-pronominal prefix}-ǫ: [living.thing]} ‘three or more [living.things]’\\
ahsę́h nigę:nǫ́: dagus \\
\gll ahsę́h ni-gę:n-ǫ́: dagu:s\\
three {\partitive}-\textsc{3p.a}-certain.number.{\stative} cat\\
\glt `three cats'
\z


As shown in \xxref{ex:countplaceholdex6}{ex:countplaceholdex7}, the same verbs are also used for counting people. (The words describing people (\textit{agǫgweˀdá:se:}, \textit{haksá:ˀah}, etc.) are \textsc{verbs} functioning as “nouns”, rather than true \textsc{nouns}, and so cannot be incorporated.) Additionally, however, the pronominal prefix of the counting verb has to describe the number and gender of the people referred to. 

\ea\label{ex:countplaceholdex6} \stem{counting verb [person]} (\cite{michelson_ontario_2011})
\ea jeyaˀdá:t agǫgwe̱ˀdá:se:\\
\gll j-e-yaˀd-a:-t ag-ǫgweˀd-a:se:\\
 {\repetitive}-\textsc{3s.fi.a}-body-{\joinerA}-stand.{\stative} \exsc{3s.fi.a}-person-new.{\stative}\\
\glt `one young woman'
\ex degaeyahshé: ga:gǫgwéˀdase:\\
\gll de-gae-yahshé: ga:g-ǫgwéˀd-ase:\\
 {\dualic}-\textsc{3ns.fi.a}-two.{\stative}  \exsc{3ns.fi.a}-person-new.{\stative}\\
\glt `two young women'
\ex ahsę́h niga:gǫ́: ga:gǫgwéˀdase:  \\
\gll ahsę́h ni-ga:g-ǫ́: ga:g-ǫgwéˀd-ase:\\
 three {\partitive}-\textsc{3ns.fi.a}-be.a.certain.number.{\stative}  \exsc{3ns.fi.a}-person-new.{\stative}\\
\glt ‘three young women’
\z
\z

\ea\label{ex:countplaceholdex7} \stem{counting verb [person]} (\cite{michelson_ontario_2011})
\ea shayaˀdá:t haksá:ˀah\\
\gll s-ha-yaˀd-á:t ha-ksá:ˀah\\
 {\repetitive}-\textsc{3s.m.a}-body-stand.{\stative}  \exsc{3s.m.a}-young.child\\
\glt `one boy'
\ex de̱hadiyahshé: hadiksá:ˀah\\
\gll de̱-hadi-yahshé: hadi-ksá:ˀah\\
 {\dualic}-\textsc{3ns.m.a}-two.living.things.{\stative}  \exsc{3ns.m.a}-young.child\\
\glt `two boys'
\ex ahsę́h ni̱hę:nǫ́: hadiksa̱ˀsǫ́:ˀǫh \\
\gll ahsę́h ni̱-hę:n-ǫ́: hadi-ksa̱ˀsǫ́:ˀǫh\\
three  {\partitive}-\textsc{3ns.m.a}-certain.number.{\stative}  \exsc{3ns.m.a}-young.child-{\pluralizer}\\
\glt ‘three boys’
\z
\z

\subsection{Counting without noun incorporation} \label{ch:Counting without noun incorporation}
The counting verbs \stem{yahshe:} ‘two living things’ and \stem{ǫ:} ‘a number of things’ cannot incorporate nouns. Instead the nouns they refer to appear after the verb \xref{ex:countplaceholdex10}. Similar examples were provided in (\ref{ex:countplaceholdex6}-\ref{ex:countplaceholdex7} above).

\newpage
\ea\label{ex:countplaceholdex10} \stem{\textsc{number} \stem{\exsc{ni-y}-ǫ:} [thing]]}\\
ahsę́h ni:yǫ́: oná:tsih  \\
\gll ahsę́h ni:-y-ǫ́: on-á:tsih\\
three  {\partitive}-\textsc{3s.p}-be.a.number.of.things.{\stative} \exsc{3ns.o}-friends.{\stative}\\
\glt ‘three pairs of socks’
\z


The following example has the same basic structure as \xref{ex:countplaceholdex10}. However, the noun after \stem{ní:yǫ:} is a long-ish phrase functioning as a “noun” (shown in square brackets in \xref{ex:countplaceholdex11}.

\ea\label{ex:countplaceholdex11} \stem{[number] \stem{\exsc{ni-y}-ǫ:} [thing/phrase]}\\
geí ni:yǫ́: [geí nigętse̱ˀda:gé: wadǫ:dá:taˀ] \\
\gll geí ni:-y-ǫ́: geí ni-ga-itse̱ˀd-á:ge: w-ad-ǫ:dá:t-haˀ\\
four {\partitive}-\textsc{3s.p}-be.a.number.of.things.{\stative} four \exsc{partitive-3s.a}-bottle-two.or.more.{\stative} \textsc{3s.a}-{\semireflexive}-contain-{\habitual}\\
\glt ‘four gallons’
\z


A less common method of counting is just to place a number before the relevant word. This strategy is used for several unincorporable nouns \xref{ex:countplaceholdex12a} and several \textsc{verbs} functioning as “nouns” \xref{ex:countplaceholdex12b}.

\ea\label{ex:countplaceholdex12} \stem{[number] [word/phrase]}
\ea sgá:t awę́hęˀ ‘one flower’\label{ex:countplaceholdex12a}
\ex sgá:t heyótga̱htwęh ‘one mile’\label{ex:countplaceholdex12b}
\z
\z

And finally, nouns that are already incorporated into verbs (regular ones, not counting verbs) can be enumerated by just adding a number before the relevant incorporating verb \xref{ex:countplaceholdex13}.
 
\ea\label{ex:countplaceholdex13} \stem{[number] \exsc{prefix}-\textsc{incorporated noun}-\textsc{verb}} ‘do something to [number] of [incorporated nouns]’\\
sgá:t shayę́dǫtaˀ  \\
\gll sgá:t s-ha-yę́d-ǫt-h-a-ˀ\\
one  {\repetitive}-\textsc{3s.m.a}-firewood-place-{\dislocative}-{\joinerA}-{\punctual}\\
\glt ‘He put one more piece of wood on the fire.’
\cfex{oyę́:daˀ\\
	\gll o-yę́:d-aˀ\\
	\textsc{3s.p}-firewood-{\nounstemformer}\\
	\glt `firewood'}
\z

