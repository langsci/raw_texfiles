\chapter{Technical notes}\label{ch:technical notes}
This section includes various explanations of technical linguistic terms and concepts which are useful -- even necessary -- background information. They are placed here, rather than in the main text, where the concepts were deemed to detract from the main message.

\section{Syllable structure} \label{syllable.structure}
Gayogo̱honǫˀnéha:ˀ syllables require a vowel, \textsc{v}, and optionally begin or end with a consonant, \textsc{c}. The largest syllable is a \textsc{cvc} syllable, one with a consonant, vowel, and consonant. Smaller syllables include \textsc{v}, \textsc{cv} or \textsc{vc}, as shown in \xref{ex:1syl}. 

\ea\label{ex:1syl} 
\ea ó: ‘oh’ (\textsc{v} syllable)

\ex dó: ‘how’ (\textsc{cv} syllable)

\ex í:s ‘you’ (\textsc{vc} syllable)

\ex nę́h ‘when, then’ (\textsc{cvc} syllable)
\z
\z
Words consist of strings of syllables. \tabref{figtab:1:wordshape} summarizes the basic shape of  Gayogo̱honǫˀnéha:ˀ words in schematic form. (Optional consonants are shown in parentheses \textsc{(c)}, and so \textsc{(c)v(c)} is an abbreviation for 4 syllables: ones with \textsc{v}, \textsc{cv}, \textsc{vc}, or \textsc{cvc} shapes.)


\begin{table}
\caption{Word shape}
\label{figtab:1:wordshape}
\begin{tabular}{lll}
\lsptoprule
Extra word-initial \textsc{c} & Syllable & More syllables\\\midrule
\textsc{c} & \textsc{(c)v(c)} & \textsc{(c)v(c)}\\
\lspbottomrule
\end{tabular}
\end{table}

The sounds that can appear in each \textsc{c} or \textsc{v} slot are listed in \xref{ex:pos}. Some example words are provided in \xref{ex:sylex}. Finally, an extra consonant (which does not fit into the \textsc{cvc} template) can appear at the beginning of some words (as shown in \tabref{figtab:1:wordshape} and example \ref{ex:sylex}a–c).

\ea\label{ex:pos} (a hyphen denotes a syllable boundary)
\ea extra \textsc{c} at the beginning of the word = s, t \phonet{tʰ}, k \phonet{kʰ}\label{ex:posa}

\ex \textsc{c} = h, ˀ, t \phonet{tʰ}, d, k \phonet{kʰ}, g, s, ts \phonet{tˢʰ}, n, w, y, r, \textsc{c}w \phonet{gʷ, kʷ, sʷ}, \textsc{c}y \phonet{gʲ, kʲ, nʲ}\label{ex:posb}

\ex \textsc{v} = \phonet{a, ah, aˀ, a:, a:h, a:ˀ} (where \exemph{a} = any vowel. For \phonet{ah, aˀ}, etc., see the next section)\label{ex:posc}

\ex \textsc{cc}\footnote{(\textsc{cc} is a consonant cluster, originating from having two \textsc{cvc-cvc} syllables in a row.)} = t \phonet{t-h}, k \phonet{k-h}, ts \phonet{t-sʰ}, j \phonet{d-ʒ, d-z}, s-h , g-w, k-w, s-w, g-y, k-y, s-n, k-n, t-n, n-y, g-r, k-r \phonet{kʰ-r}, d-r, t-r \phonet{tʰ-r}, s-r, s-k, s-g, s-t, t-ˀ, k-ˀ, s-ˀ, n-ˀ, t-k, k-t\label{ex:posd}
\z
\z

\largerpage
\ea\label{ex:sylex} Example syllabifications
\ea kdagǫˀ (k-da-gǫˀ) ‘grey squirrel, black squirrel’

\ex snętsáˀgeh (s-nęt-sʰáˀ-geh) ‘on your arm’

\ex tganǫ́hso:t (t-ga-nǫ́h-so:t) ‘a standing house’

\ex diˀdi:ˀ (diˀ-di:ˀ) ‘blue jay’

\ex oˀnéhsaˀ (oˀ-néh-saˀ) ‘sand’

\ex gaˀgaˀ: (gaˀ-ga:ˀ) ‘crow, raven’

\ex haˀhó:diˀ (haˀ-hó:-diˀ) ‘he threw it’

\ex ó:gę:ˀt (ó:-gęˀt) ‘it is visible’

\ex oháˀkdaˀ (oh-háˀk-daˀ) ‘soot’\footnote{In this and later examples, a double <h-h> or double <ˀ-ˀ> denotes that a single consonant, H or <ˀ>, affiliates with both the preceding and following vowel (i.e. it is \textsc{ambisyllabic}).}

\ex aˀǫda̱hǫhdo̱háeˀ (aˀ-ǫ-da̱h-hǫh-do̱h-há-eˀ) ‘she washed her ears’

\ex goyéhtaˀ (go-yéht-haˀ) ‘alarm clock’

\ex ehstaˀ (ehs-tʰaˀ) ‘she or someone uses it’

\ex otéˀtraˀ  (o-téˀtʰ-raˀ) ‘flour, powder’

\ex wa̱hdǫhs (wa̱h-dǫhs) ‘it disappears’

\ex a:hagwé:niˀ (a-ah-hag-wé:-niˀ) ‘he was able to do it’ \footnote{Words like \cayuga{a:hagwé:niˀ} begin with the \stem{a:-} {\indefinite} prefix. This prefix counts as two syllables for the syllable count, which is described in \sectref{counting.syllables}.}

\ex só:wa:s (só:-wa:s) ‘dog’

\ex eksá:ˀah (ek-sá:ˀ-ˀah) ‘girl’

\ex onohotsgę́ˀęh (o-noh-hotˢʰ-gę́ˀ-ˀęh) ‘beech tree’

\ex tsahgó:wah (tˢʰah-gó:-wah) ‘pigeon’\footnote{Recall that TS stands for \phonet{tˢʰ} in the Henry orthography.}
\z
\z

Some words have three consonants in a row, rather than the maximal number of two \textsc{c}’s between vowels predicted by a hypothetical string of [\textsc{cvc-cvc}] syllables. In such cases, one of the consonants is always S, W, or Y \xref{ex:cont}, analysed here as consonant secondary articulations.

\ea\label{ex:cont}
\ea ęhsrihwanéˀak\textbf{s}rǫ:ˀ (ęhs—rih—wa—néˀ—akˢ—rǫ:ˀ) ‘you will swear’\label{ex:conta}
\ex de̱hę́natk\textbf{w}aˀ (de̱h—hę́—nat—kʷaˀ) ‘they dance’\label{ex:contb}
\ex ętg\textbf{y}ǫˀ (ęt—gyǫˀ) ‘I will come in’\label{ex:contc}
\ex g\textbf{w}é: (gʷé:) ‘hello’, ‘well!’\label{ex:contd}
\ex gatg\textbf{w}ę́ˀdaˀ (gat-gʷę́ˀ-daˀ) ‘wallet’, ‘purse’, ‘suitcase’, etc.\label{ex:conte}
\z
\z


\section{Prefixes, suffixes, affixes, and stems} \label{Prefixes, suffixes, affixes, and stems}
Many Gayogo̱honǫˀnéha:ˀ words consist of smaller parts, namely  \textsc{prefixes}, \textsc{suffixes}, \textsc{affixes}, and \textsc{stems}. (“Affix” is a cover term for prefixes and suffixes.) The stem is the core or substantive part of any word. Prefixes occur before the stem, and suffixes occur after the stem, \tabref{figtab:1:prefsufstem}. Affixes tend to contribute more abstract meanings to words, while stems contribute the main meaning. For example, the prefix \stem{ga-} \textsc{3s.a} in \tabref{figtab:1:prefsufstem} denotes that the word refers to a \emph{thing}, and also that the noun is \emph{unpossessed}. In turn, the  \stem{-aˀ} \textsc{\nsf} suffix identifies the word as a noun. Finally, the stem \stem{nǫhs} ‘house’ is the meaningful “heart” of the word \textit{ga-nǫhs-aˀ} ‘house(s)’.

\begin{table}
\caption{Prefixes, stems, and suffixes}
\label{figtab:1:prefsufstem}
\begin{tabular}{lll}
\lsptoprule
prefix & stem & suffix\\\midrule
ga & nǫ́hs & aˀ\\
\textsc{3s.a} & house & \textsc{\nsf}\\
\lspbottomrule
\end{tabular}
\end{table}

Some affixes are optional, as with the \stem{-gó:wah} suffix shown in \xref{ex:prefsufstemidex2a}. Suffixes are optional if their absence still results in a stand-alone word. For example, \stem{-gó:wah} is optional because \cayuga{otrę́ˀdaˀ} \xref{ex:prefsufstemidex2b} is a stand-alone word without the suffix. (The word without the suffix has a different meaning, but it is still a word).

\ea\label{ex:prefsufstemidex2}
\ea otręˀda\exemph{gó:wah} ‘horsefly’\label{ex:prefsufstemidex2a}
\ex otrę́ˀdaˀ ‘fly’\label{ex:prefsufstemidex2b}
\z
\z

More often than not, however, Gayogo̱honǫˀnéha:ˀ affixes are obligatory: their absence results in an incomplete word. When the prefixes in (\ref{ex:prefsufstemidex3}a, b) are removed, the result \stem{nǫhǫkda:nih} does not make sense. In fact, \stem{nǫhǫkda:nih} requires the prefix in order to be a word.

\ea\label{ex:prefsufstemidex3}
\ea aknǫhǫkdá:nih\\\label{ex:prefsufstemidex3a}
\gll ak-nǫhǫkdá:nih\\
 \textsc{1s.p}-sick.{\stative}\\
\glt `I am sick'

\ex sanǫhǫkdá:nih\\\label{ex:prefsufstemidex3b}
\gll ak-nǫhǫkdá:nih\\
 \textsc{2s.p}-sick.{\stative}\\
\glt `you are sick'

\cfex{\exemph{*nǫhǫkdá:nih} (not a word)}
\z
\z

In words with obligatory affixes, one affix can still be substituted for another of the same type. For example, \stem{nǫhǫkda:nih} can take either /ak-/ \textsc{1s.p} or /sa-/ \textsc{2s.p}, or any other pronominal prefix \xref{ex:prefsufstemidex3}.

The following words were suggested as candidates for Gayogo̱honǫˀnéha:ˀ grammatical terms.

\ea\label{ex:cayugaprefsufstemidex3}
\ea owę́:naˀ	‘word’\label{ex:cayugaprefsufstemidex3a}

\ex deganǫ̱hsó:taˀ	‘particle’ (literally, ‘it connects’)\label{ex:cayugaprefsufstemidex3b}

\ex gayaˀda̱hsró:nih	‘adjective’ (literally, ‘it dresses things up’)\label{ex:cayugaprefsufstemidex3c}

\ex gawęnagwe:ní:yo: ‘base, stem’ (literally, ‘the main word’)\label{ex:cayugaprefsufstemidex3d}

\ex dwasáwa̱hkwaˀ ‘prefix’ (literally, ‘it begins’)\label{ex:cayugaprefsufstemidex3e}

\ex ohę́:dǫh ‘pronominal prefix’ (literally, ‘in front of’)\label{ex:cayugaprefsufstemidex3f}

\ex ohę:dǫ́h dwasáwa̱hkwaˀ ‘prepronominal prefix’ (‘literally in front of the prefix’)\label{ex:cayugaprefsufstemidex3g}

\ex hewádoktaˀ ‘suffix’ (literally, ‘it ends’)\label{ex:cayugaprefsufstemidex3h}

\ex ęsehsgo̱haowá:neht ‘branch it out’ (i.e. add particles, etc. to ‘dress up’ speech)\label{ex:cayugaprefsufstemidex3i}
\z
\z 

\largerpage
\section{Paradigm, inflection, and conjugation} \label{Paradigm, inflection, and conjugation}
The concepts of \textsc{paradigm}, \textsc{inflection}, and \textsc{conjugation} describe relationships between words.

A paradigm is a group of words that have the same stem but different affixes. For example, the basic noun paradigm in \xref{ex:prefsufstemidex4} consists of a noun stem that has been \textsc{inflected} for person pronominal prefixes. 

Both verb and noun stems can be inflected or systematically changed by substituting prefixes or suffixes of the same type -- for example, by changing the pronominal prefix, as in \xref{ex:prefsufstemidex4}. 
 
\ea\label{ex:prefsufstemidex4} A basic noun inflected for pronominal prefixes
\ea \exemph{ak}nǫ́hsaˀ ‘my house’

\ex \exemph{ǫkní}nǫ̱hsaˀ ‘our house (2 of us)’

\ex \exemph{ǫgwá}nǫ̱hsaˀ ‘our house (more than 2)’

\ex \exemph{sa}nǫ́hsaˀ ‘your (singular) house’

\ex \exemph{sni}nǫ́hsaˀ ‘your house (2 of you)’

\ex \exemph{swa}nǫ́hsaˀ ‘your house (all of you)’

\ex \exemph{ho}nǫ́hsaˀ ‘his house’

\ex \exemph{go}nǫ́hsaˀ ‘her house’

\ex \exemph{o}nǫ́hsaˀ ‘its house’

\ex \exemph{hodí}nǫ̱hsaˀ ‘their (males) house’

\ex \exemph{godí}nǫ̱hsaˀ ‘their (females) house’

\ex \exemph{odí}nǫ̱hsaˀ ‘their (animals) house’

\z
\z

The term \textsc{conjugation} is relevant for verbs. An example is shown in \xref{ex:prefsufstemidex5}, where the verbs are conjugated for the three major aspect suffixes (the {\habitual}, {\punctual}, and {\stative}, see \sectref{Aspect conjugation classes}). 

\ea\label{ex:prefsufstemidex5}
\ea sga̱hdę́:gyeˀs\\
\gll s-g-a̱hdę́:gy-eˀs\\
 {\repetitive}-\textsc{1s.a}-arrive-{\habitual}\\
\glt `I go home all the time'

\ex ęsgahdę́:diˀ\\
\gll ęs-g-ahdę́:di-ˀ\\
 {\futurerepetitive}\exsc{-1s.a}-arrive-{\punctual}\\
\glt `I will go home'

\ex swagáhdęgyǫ: \\
\gll s-wag-áhdęgy-ǫ:\\
 {\repetitive}-\textsc{1s.p}-arrive-{\stative}\\
\glt ‘I am going home’, `I have gone home'
\z
\z


\section{Atypical words} \label{Atypical words}
Atypical words cannot be categorized into one of the three classes of \cayuga{Gayogo̱honǫˀnéha:ˀ} words -- nouns, verbs or particles (defined in \sectref{ch:Defining nouns, verbs, particles}). They include atypical nouns, “suffix-words” (\textsc{enclitics}), atypical verbs, and kinship terms. With the exception of kinship terms, atypical words lack pronominal prefixes and sometimes other obligatory affixes as well.

Atypical nouns (atypical words functioning as “nouns”) are unlike other nouns in that they have no prefixes or suffixes \xref{ex:non.canonical1}.

\ea\label{ex:non.canonical1} atypical nouns
\ea diˀdi:ˀ ‘blue jay’

\ex dó:dihs ‘salamander’

\ex ǫ́:gweh ‘person’
\z
\z

“Suffix-words” (\textsc{enclitics}) can appear either as suffixes \xref{ex:non.canonical2a} or as stand-alone words \xref{ex:non.canonical2b}.


\ea\label{ex:non.canonical2} suffix-words
\ea ohahakdá:gyeˀ\\\label{ex:non.canonical2a}
\gll o-hah-akdá:gyeˀ\\
 \textsc{3s.p}-road-\exsc{alongside} \\
\glt `along the edge of the road'

\ex akdá:gyeˀ ‘the edge, beside’\label{ex:non.canonical2b}
\z
\z

Atypical verbs lack the pronominal prefixes normally required of verbs, but have verbal suffixes. The ones in \xref{ex:non.canonical4} function as “nouns”.

\ea\label{ex:non.canonical4} atypical \textsc{verbs} functioning as “nouns”
\ea gi̱hę:k\\
\gll g̱ih-ę:-k\\
 \textsc{ø.prefix}.river-lie.{\stative}-{\modalizer}\\
\glt `river, stream, creek'

\ex nigihú:ˀuh\\
\gll ni-gih-ú:ˀuh\\
 \exsc{\partitive}-\textsc{ø.prefix.}river-small.{\stative}\\
\glt `small stream'

\ex gi̱hę́:deˀ\\
\gll gi̱hę:-deˀ\\
 \textsc{ø.prefix}.river-exist.{\stative}\\
\glt  ‘creek, river, stream’, `it is a river'
\z
\z

\textsc{Kinship terms} are atypical. Their structure is described in greater detail in \sectref{ch:Kinship terms (atypical words)}.


\section{No prepositions} \label{No prepositions}
Gayogo̱honǫˀnéha:ˀ does not have prepositions: instead, concepts such as ‘for’, ‘to’, ‘from’, ‘up’, and ‘down’ are an inherent part of the meaning of some verbs. For example, some verbs take a \textsc{benefactive} suffix (\sectref{Benefactive suffixes}) to convey the meaning of doing something \emph{for} someone’ \xref{ex:preposex}.

\ea\label{ex:preposex} ahékǫnyęˀ\\
\gll a-he-kǫny-ę-ˀ\\
{\factual}-\exsc{1s:3ms}-cook-{\benefactive}-{\punctual}\\
\glt ‘I cooked a meal for him’
\z

Similarly, the verbs in \xref{ex:preposex2} refer to a source, a recipient, and an object being transferred. In this case, the concepts of the recipient (‘\emph{to} me’) or source (‘\emph{from} her’) are encoded in the interactive pronominal prefix.

\ea\label{ex:preposex2} 
\ea  ęsgwá:yǫˀ\\
\gll ę-sgwa-yǫ-ˀ\\
 \fut-\exsc{2:1(p)}-give-{\punctual}\\
\glt `you all will give it to me'

\ex ahsagokwáhkwaˀ \\
\gll a-hsago-kwá-hkwa-ˀ\\
 {\factual}-\exsc{3ms:3fi/3p.p}-food-take.from-{\punctual}\\
\glt ‘he grabbed the food from her’, `he took the food from her'
\z
\z



\section{Potential change-of-state suffixes} \label{Potential change-of-state suffixes}
The following notes speculate about the possible existence of causative suffixes pronounced as \phonet{-:} (added vowel length), and about the nature of the \stem{-(h)sˀ)} \textsc{eventuative} (\textsc{ev}) suffix.


\subsection{\stem{-:} (vowel length suffixes)} \label{[-:] (vowel length suffixes)}
This section is about a group of suffixes -- or perhaps, one suffix -- pronounced as \phonet{-:} (vowel length).

Vowel length possibly serves as a causative suffix \stem{-:} that changes an unintentional happening into an intentional action. To illustrate, \textit{ęhsáhdǫˀ} ‘you will disappear’ \xref{ex:vlengsufa} would not have the suffix, while \textit{esáhdǫ:ˀ} ‘you lost it’ \xref{ex:vlengsufb} might have the suffix -- \xref{ex:vlengsufb} could be translated as ‘you caused it to disappear’. Similarly, \xref{ex:vlengsufc}, without the \stem{-:} suffix, would describe an accidental happening, while \xref{ex:vlengsufd}, with the suffix, would describe a deliberate action.

\ea\label{ex:vlengsuf} 
\ea ęhsáhdǫˀ\\\label{ex:vlengsufa}
\gll ę-hs-áhdǫ-ˀ\\
 \fut-\textsc{2s.p}-lose-{\punctual}\\
\glt `you will disappear'

\ex esáhdǫ:ˀ\\\label{ex:vlengsufb}
\gll e-sá-hdǫ-:-ˀ\\
 {\factual}-\textsc{2s.p}-lose-{\causative}-{\punctual}\\
\glt `you lost it'

\ex ęhsaˀnigǫ́hahdǫˀ\\\label{ex:vlengsufc}
\gll ę-hsa-ˀnigǫ́h-ahdǫ-ˀ\\
\fut-\textsc{2s.p}-mind-lose-{\punctual}\\
\glt ‘you will faint’ (literally, ‘your mind will disappear’)

\ex esahwi̱hsdáhdǫ:ˀ\\\label{ex:vlengsufd}
\gll e-sa-hwi̱hsd-áhdǫ-:-ˀ\\
 {\factual}-\textsc{2s.p}-money-lose-{\causative}-{\punctual}\\
\glt `you lost money'
\z
\z

The same proposed \stem{-:} \textsc{\causative} suffix possibly distinguishes between many pairs of punctual verb forms with \stem{-:} and stative verb forms without \stem{-:} \xref{ex:vlengsuf2}. Like other causative suffixes (see \sectref{Causative suffixes}), it would have the effect of changing a state into an activity. For example, \stem{tsaˀdę́hsyę:ˀ} \xref{ex:vlengsuf2a} describes an activity (‘you will cause them to be lying side-by-side’), while \stem{tsaˀdégayęˀ} (also in \ref{ex:vlengsuf2a}) describes the state of ‘lying side-by-side’.

\ea\label{ex:vlengsuf2} 
\ea tsaˀdę́hsyę:ˀ\\\label{ex:vlengsuf2a}
\gll tsaˀd-ę́-hs-yę-:-ˀ\\
 {\coincident.\factual.\dualic}\exsc{-future-2s.a}-lay-{\causative}-{\punctual}\\
\glt `you will lay them side-by-side'

\cfex{tsaˀdégayęˀ\\
\gll tsaˀdé-ga-yęˀ\\
{\coincident}-{\dualic}.\textsc{3s.a}-lie.{\stative}\\
\glt ‘they are (literally, ‘it is’) lying or setting side-by-side’
}

\ex ęhsyę:ˀ\\
\gll ę-hs-yę-:-ˀ\\
\fut-\textsc{2s.a}-lie-{\causative}-{\punctual}\\
\glt ‘you will put it there’ (or, ‘you will cause it to be placed there’)
\cfex{nigá:yęˀ\\
	\gll ni-gá:-yę-ˀ\\
	{\partitive}-\textsc{3s.a}-lie.{\stative}\\
	\glt `where it is at, where it is placed'}

\ex ęgátge̱hǫ:ˀ\\
\gll ę-g-át-ge̱hǫ-:-ˀ\\
\fut-\textsc{1s.a}-{\semireflexive}-sell-{\causative}-{\punctual}\\
\glt ‘I will have for sale’ (or, ‘I will cause to sell’)

\cfex{honátge̱hǫˀ\\
	\gll hon-át-ge̱hǫ-ˀ\\
	\textsc{3ns.m.p}-{\semireflexive}-sell-{\habitual}\\
	\glt `they (m) are selling something'}
\z
\z


In the same vein, a length \stem{-:} suffix relevant to the purposive aspect possibly occurs in e-verbs (see \sectref{ch:E-verbs}). When e-verbs end with short \stem{e}, they appear to describe a state (which translates as a present tense). In contrast, when they take long \stem{e-:} forms, they appear to describe an activity or happening (often translated as past tense, \ref{ex:vlengsuf3}).

\ea\label{ex:vlengsuf3} 
\ea haˀgeˀ\\
\gll h-aˀ-g-e-ˀ\\
 {\translocative}-{\factual}-\textsc{1s.a}-go-{\punctual}\\
\glt `I am going there'

\cfex{haˀge:ˀ\\
	\gll h-aˀ-g-e-:-ˀ\\
	{\translocative}-{\factual}-\textsc{1s.a}-go-{\purposive}-{\punctual}\\
	\glt `I went there'}

\ex hadíhsreˀ\\
\gll hadí-hsre-ˀ\\
 \textsc{3ns.m.a}-follow-{\stative}\\
\glt `they follow'

\cfex{ęséhsre:ˀ\\
	\gll ę-s-é-hsre-:-ˀ\\
	\fut-\textsc{2s.a}-{\joinerE}-chase-{\purposive}-{\punctual}\\
	\glt `you will chase or follow'}

\ex géhseˀ\\
    \gll ag-é-hs-e-ˀ\\
    \textsc{1s.p}-{\joinerE}-lower.back-go-{\stative}\\
        \glt ‘I am riding'


\cfex{agéhse:ˀ ‘I rode’,\\\largerpage
	\gll ag-é-hs-e-:-ˀ\\
	\textsc{1s.p}-{\joinerE}-lower.back-go-{\purposive}-{\stative}\\
	\glt `I came riding'}

\ex gegyeˀ\\
\gll g-e-gye-ˀ\\
 \textsc{1s.a}-{\joinerE}-fly-{\stative}\\
\glt `I am flying'


\cfex{gegye:ˀ\\
	\gll g-e-gy-e-:-ˀ\\
	\textsc{1s.a}-{\joinerE}-fly-go-{\purposive}-{\stative}\\
	\glt `I came flying'}
\z
\z

Finally, a set of suffixes, tentatively analysed as consisting of \stem{-:-ˀ} \textsc{\causative-\nsf} potentially appears at the end of stative nouns (a type of noun based on stative verb forms (\ref{ex:vlengsuf4}, see \sectref{ch:Stative nouns and agentive stative nouns}). It would turn a stative verb into a noun, which would then require the \stem{-ˀ} \textsc{\nsf} suffix. The words in \xref{ex:vlengsuf4} describe an ‘object’ resulting from an activity.

\ea\label{ex:vlengsuf4} 
\ea adrihwa̱hséhdǫ:ˀ\\
\gll ad-rihw-a̱hséhd-ǫ-:-ˀ\\
 \exsc{ø.prefix.\semireflexive-matter}-hide-\stat-\nominalizer{-\nsf}\\
\glt `a secret (was caused)'

\cfex{odrihwa̱hséhdǫh\\
	\gll o-d-rihw-a̱hséhd-ǫh\\
	\textsc{3s.p}-{\semireflexive}-matter-hide-{\stative}\\
	\glt `it is secret'}

\ex adówadǫ:ˀ\\
\gll adówad-ǫ-:-ˀ\\
 \textsc{ø.prefix}.hunt-\stat-\nominalizer{-\nsf}\\
\glt `the hunt'


\cfex{agadowá:dǫh\\
	\gll ag-adowá:d-ǫh\\
	\textsc{1s.p}-hunt-{\stative}\\
	\glt `I have hunted', `I am hunting'}
\z
\z


\subsection{\stem{-(h)sˀ} (the eventuative) ({\eventuative})} \label{[-(h)sˀ] (the eventuative)}
An \stem{-(h)sˀ} \textsc{\eventuative} suffix was previously posited for Gayogo̱honǫˀnéha:ˀ (and \textit{Onöndowaˀga:ˀ} / Seneca). It is said to occur before the \stem{-ǫh} \textsc{\stative} suffix, where it “…signif[ies] an event that is finally complete.” (\cite[283]{charles_history_2010}) The verb provided as an example of the \stem{-ˀs} \textsc{\eventuative} is shown in \xref{ex:eventuative1a}.

However, it may be that \stem{hsˀ, ihsˀ, at.hsˀ} is instead the verb meaning ‘finish’, ‘use up’, in which case, \xref{ex:eventuative1a} should be reanalysed as a compound construction with two verb stems, as in \xref{ex:eventuative1b}.

\ea\label{ex:eventuative1} 
\ea with {\eventuative}\\
agade̱hsrǫ́ni̱hsˀǫh\\\label{ex:eventuative1a}
\gll ag-ad-e̱-hsr-ǫ́ni̱-hsˀ-ǫh\\
 \exsc{1s.p-{\semireflexive}}-{\joinerE}-tool-make-\exsc{\eventuative}-{\stative} \\
\glt `I am ready'

\ex with \stem{hsˀ, ihsˀ, at.hsˀ} ‘finish’\\
agade̱hsrǫ́ni̱hsˀǫh\\\label{ex:eventuative1b}
\gll ag-ad-e̱-hsr-ǫ́ni̱-hsˀ-ǫh\\
 \exsc{1s.p-{\semireflexive}}-{\joinerE}-tool-make-finish-{\stative} \\
\glt `I am ready'
\z
\z

Example \xref{ex:eventuative6} illustrates a verb with a structure similar to the one proposed in \xref{ex:eventuative1b}.

\ea\label{ex:eventuative6} \stem{at-geni-hsˀ} \textsc{\semireflexive-}compete-finish ‘have a meeting’
\ea swatgénihsˀahs\\\label{ex:eventuative6a}
\gll sw-at-géni-hsˀ-a-hs\\
 {2\pla}-{\semireflexive}-compete-finish-{\joinerA}-{\habitual}\\
\glt `you have meetings all the time'

\ex ęhswatgeníhsˀa:ˀ\\\label{ex:eventuative6b}
\gll ę-hsw-at-gení-hsˀ-a:ˀ\\
 \fut-{2\pla}-{\semireflexive}-compete-finish-{\punctual}\\
\glt `you all will have a meeting'

\ex swatgénihsˀǫh\\\label{ex:eventuative6c}
\gll sw-at-géni-hsˀ-ǫh\\
 \exsc{2p.p}-{\semireflexive}-compete-finish-{\stative}\\
\glt `you are having a meeting right now'
\z
\z

The remaining examples in \xxref{ex:eventuative8}{ex:eventuative12} provide evidence that a verb \stem{hsˀ, ihsˀ, at.hsˀ} ‘finish, use up’ exists independently. As a member of the \textsc{S13} conjugation class (\sectref{Aspect conjugation classes}), it would take the  \stem{-a-hs} {\habitual}, the \stem{-a-:ˀ} {\punctual}, and the \stem{-ǫh} {\stative}. Examples supporting the S13 conjugation class analysis, and exemplifying other compound structures with \stem{hsˀ, ihsˀ, at.hsˀ} ‘finish’, ‘use up’, are provided in \xxref{ex:eventuative8}{ex:eventuative12}.

\ea\label{ex:eventuative8} \stem{rihw-ihsˀ} matter-finish ‘promise, make an agreement’
\ea haíhwi̱hsˀahs\\
\gll ha-íhw-i̱hsˀ-a-hs\\
 \textsc{3s.m.a}-matter-finish-{\joinerA}-{\habitual}\\
\glt `he is making promises', `he is making agreements'

\ex ęgríhwihsˀa:ˀ\\
\gll ę-g-ríhw-ihsˀ-a:ˀ\\
 \fut-\textsc{1s.a}-matter-finish-{\punctual}\\
\glt `I will promise'
\z
\z

\ea\label{ex:eventuative11} \stem{-at-noun-ihˀs} \textsc{\semireflexive-noun}-finish ‘finish something’
\ea ęgatędíhsˀa:ˀ\\
\gll ę-g-at-hęd-íhsˀ-a:ˀ\\
 \fut-\textsc{1s.a}-{\semireflexive}-field-finish-{\punctual}\\
\glt `I will finish gardening', I will finish the field'

\ex agataháhsˀa:ˀ\\
\gll a-g-at-hah-á-hsˀ-a:ˀ\\
 {\factual}-\textsc{1s.a}-{\semireflexive}-road-{\joinerA}-finish-{\punctual}\\
\glt `I finished a road'
\z
\z

\ea\label{ex:eventuative10} \stem{ag-yaˀd-ihsˀ} {\semireflexive}-body-finish ‘mature’\\
ęyǫgya̱ˀdíhsa:ˀ\\
\gll ę-yǫ-g-ya̱ˀd-íhs-a:ˀ\\
 \fut-\textsc{3s.fi.a}-{\semireflexive}-body-finish-{\punctual}\\
\glt `she has matured', `she has completed her life cycle'
\z


\ea\label{ex:eventuative7} \stem{ad-rihw-a-hsˀ} {\semireflexive}-matter-{\joinerA}-finish ‘plan’\\
ędwadri̱hwáhsˀa:ˀ\\
\gll ę-dw-ad-ri̱hw-á-hsˀ-a:ˀ\\
 \fut-\exsc{1p.in.a}-{\semireflexive}-matter-{\joinerA}-finish-{\punctual}\\
\glt `we all will plan something'
\z


\ea\label{ex:eventuative13} \stem{-hsˀ-a-hd} finish-{\joinerA}-\exsc{causative} ‘use up’
\ea hęgéhsˀaht\\
\gll h-ę-g-é-hsˀ-a-ht\\
 {\translocative}-{\future}-\textsc{1s.a}-{\joinerE}-finish-{\joinerA}-{\causative}.{\zeropunctual}\\
\glt `I will use up'
\z
\z

\ea\label{ex:eventuative3} \stem{-at/adat-hsˀ} {\semireflexive}/{\reflexive}-finish ‘get used up’
\ea awá:tsˀa:ˀ \\
\gll a-w-á:t-hsˀ-a:ˀ\\
 {\factual}-\textsc{3s.a}-{\semireflexive}-use.up-{\punctual}\\
\glt ‘it is worn out, all gone, burnt up’, `it went down to nothing'

\ex haˀwádatsˀǫh\\
\gll haˀ-w-ádat-hsˀ-ǫh\\
 {\translocative}-\textsc{3s.a}-{\reflexive}-use. up-{\stative}\\
\glt `it is empty, burnt up, used up'
\z
\z

\ea\label{ex:eventuative4} \stem{de … adęt-hsˀ} \textsc{\dualic…\reflexive}-finish ‘earn, deserve’
\ea desádętsˀahs \\
\gll de-s-ádęt-hsˀ-a-hs\\
 {\dualic}-\textsc{2s.a}-{\reflexive}-finish-{\joinerA}-{\habitual}\\
\glt ‘your salary’, `what you make'

\ex dęhsádętsˀa:ˀ\\
\gll d-ę-hs-ádęt-hsˀ-a:ˀ\\
 {\dual}-\exsc{\future-2s.a}-{\reflexive}-finish-{\punctual}\\
\glt `you will earn it (money)'

\ex desádętsˀǫh \\
\gll de-s-ádęt-hsˀ-ǫh\\
{\dualic}-\textsc{2s.p}-{\reflexive}-finish-{\stative}\\
\glt ‘you’ve earned or deserve it’, ‘you’ve paid your dues’ 
\z
\z

\ea\label{ex:eventuative12} \stem{de-…-adat-hsˀ} {\dualic}-\exsc{\reflexive}-finish ‘quarrel’\\
dęhsádatsˀa:ˀ\\
\gll d-ę-hs-ádat-hsˀ-a:ˀ\\
 {\dualic}-{\future}-\textsc{2s.a}-{\reflexive}-finish-{\punctual}\\
\glt `you will quarrel'
\z


Example \xref{ex:eventuative5} appears to consist of \stem{at-hs-ihsˀ} ‘\textsc{\semireflexive-noun}-finish’, where the \stem{-hs} element is a type of placeholder incorporated noun (see \sectref{ch:Placeholder incorporated nouns}).

\ea\label{ex:eventuative5} \stem{-at-hs-ihsˀ} {\semireflexive}-noun-finish ‘ripen, mature’
\ea watsíhsˀahs\\
\gll w-at-hs-íhsˀ-a-hs\\
 \textsc{3s.a}-{\semireflexive}-noun-finish-{\joinerA}-{\habitual}\\
\glt `it is ripening, maturing'

\ex ęwátsihsˀa:ˀ\\
\gll ę-w-át-hs-ihsˀ-a:ˀ\\
 \fut-\textsc{3s.a}-{\semireflexive}-noun-finish-{\punctual}\\
\glt `it will mature, ripen'

\ex tsíhsˀǫh \\
\gll o-t-hs-íhsˀ-ǫh\\
 \textsc{3s.p}-{\semireflexive}-noun-finish-{\stative}\\
\glt ‘it is done for the season’, ‘it has gone full cycle’, ‘it is mature’, `they (plants) have finished out'
\z
\z

And finally, in \xref{ex:eventuative2} and \xref{ex:eventuative9}, it may be that the \stem{at} {\semireflexive} has been reanalyzed as part of the verb stem and can thus occur after an incorporated noun.
 


\ea\label{ex:eventuative2} \stem{-(noun)-at.hsˀ-a-hd} {\semireflexive}.finish-{\joinerA}-\exsc{causative} ‘be all gone’
\ea haˀwá:tsˀaht\\
\gll h-aˀ-w-á:t.hsˀ-a-ht\\
 {\translocative}-{\factual}-\textsc{3s.a}-use.up-{\causative}.{\zeropunctual}\\
\glt `it is all gone'

\ex haˀwatnegá:tsˀaht\\
\gll h-aˀ-w-at-hneg-á:t.hsˀ-a-ht\\
 {\translocative}-{\factual}-\textsc{3s.a}-{\semireflexive}-water-{\semireflexive}.finish-{\joinerA}-{\causative}.{\zeropunctual}\\
\glt `water is all gone'

\ex heyótsˀahdǫh\\
\gll he-yó-t.hsˀ-a-hd-ǫh\\
 {\translocative}-\textsc{3s.p}-{\semireflexive}-finish-{\joinerA}-{\causative}-{\stative}\\
\glt `it is all gone'
\z
\z

\ea\label{ex:eventuative9} \stem{d…ad-rihw-at.hsˀ} \exsc{\cislocative…\semireflexive}-matter-{\semireflexive}.finish ‘earn, fulfill’\\
atadri̱hwa:tsˀa:ˀ \\
\gll a-t-ha-d-ri̱hw-a:t.hsˀ-a:ˀ\\
{\factual}-{\cislocative}-\textsc{3s.m.a}-{\semireflexive}-matter-{\semireflexive}.finish-{\punctual}\\
\glt ‘he earned it’, ‘he fulfilled it’ 
\z

\section{Aspect conjugation classes} \label{Aspect conjugation classes}
Every verb takes a set of aspect endings (\sectref{Verbs and aspect}) known as a \textsc{conjugation class}, which is relevant for choosing the appropriate aspect and post-aspect (tense) suffixes. \citet{sasse_far_1998} describe three types of aspect conjugation classes, the \textsc{s}, \textsc{h}, and \textsc{m} or \textsc{mixed} class. These classes are defined by the pairings of habitual and stative endings. For example, \textsc{s} class verbs take \stem{-(h)s} \textsc{\habitual} and \stem{-ǫh} \textsc{\stative} endings \xxref{ex:aspclass1}{ex:aspclass2}, while \textsc{h} class verbs take \stem{-haˀ} \textsc{\habitual} and \stem{-ˀ} \textsc{\stative} endings \xref{ex:aspclass3}. (The punctual forms are also shown below, but are not relevant to the classification system.)

The aspect conjugation classes are also subdivided according to the last sound of the verb stem. For example, while the \textsc{s} classes all take an \stem{-s} {\habitual} ending \xxref{ex:aspclass1}{ex:aspclass2}, the \textsc{s3} subclass of verb stems end in T/D \xref{ex:aspclass1} and \textsc{s4} verb stems all end in K/G \xref{ex:aspclass2}.\footnote{The description in this section is the analysis in \citet{sasse_far_1998}. Also see \citet[85]{lounsbury_oneida_1953} and \citet[20, 27]{michelson_oneida-englishenglish_2002} for Oneida, and \citet[89]{woodbury_reference_2018} for Onondaga conjugation classes. The other Iroquoian languages have fewer aspect conjugation classes, in part because they use just one criterion -- the pairings of habitual and stative endings. In contrast, \citet{sasse_far_1998} further subdivide the same classes according to the final sound(s) of the stem, following the insights in \citet{michelson_mohawk_1975}. Finally, see \sectref{Simplifying two consonants to one} regarding the simplification of word-final consonant clusters, which occurs in many of the following examples.}

\ea\label{ex:aspclass1} \textsc{s3} verb stem ends with T/D, and takes \stem{-s} {\habitual}, \stem{-ø} {\zeropunctual}, and \stem{-ǫh} {\stative} endings
\ea gadó:wa:s\\
\gll g-ad-ó:wa:d-s\\
 \textsc{1s.a}-{\semireflexive}-hunt-{\habitual}\\
\glt `I am a hunter'

\ex ęga:dó:wa:t\\
\gll ę-g-a:d-ó:wa:t-ø\\
 \fut-\textsc{1s.a}-{\semireflexive}-hunt-{\zeropunctual}\\
\glt `I will hunt'

\ex agadowá:dǫh\\
\gll ag-ad-owá:d-ǫh\\
 \exsc{ 1s.p}-{\semireflexive}-hunt-{\stative}\\
\glt `I am hunting'
\z
\z

\ea\label{ex:aspclass2} \textsc{s4} verb stem ends with K/G, and takes \stem{-s} {\habitual}, \stem{-ø} {\zeropunctual}, and \stem{-ǫh} {\stative} endings
\ea degáˀswe:s\\
\gll de-g-áˀswe:g-s\\
 {\dualic}-\textsc{1s.a}-deaf-{\habitual}\\
\glt `I am going deaf'

\ex dęgáˀswe:k\\
\gll d-ę-g-áˀswe:k-ø\\
 {\dualic}-{\future}-\textsc{1s.a}-deaf-{\zeropunctual}\\
\glt `I will go deaf'

\ex dewaga̱ˀswé:gǫh\\
\gll de-wag-a̱ˀswé:g-ǫh\\
 {\dualic}-\textsc{1s.p}-deaf-{\stative}\\
\glt `I went deaf'
\z
\z


\ea\label{ex:aspclass3} \textsc{h3} verb stem ends with a vowel and takes \stem{-haˀ} {\habitual}, \stem{-:ˀ} {\punctual}, and \stem{-ˀ} {\stative} endings
\ea gadǫtgadǫ́haˀ\\
\gll g-ad-ǫtgadǫ́-haˀ\\
 \textsc{1s.a}-{\semireflexive}-have.fun-{\habitual}\\
\glt `I am fun-loving'

\ex ęhswadǫtgá:dǫ:ˀ\\
\gll ę-hsw-ad-ǫtgá:dǫ-:ˀ\\
 \fut-{2\pla}-{\semireflexive}-have.fun-{\punctual}\\
\glt `you all will have a good time'

\ex ǫgwadǫtgá:dǫˀ\\
\gll ǫgw-ad-ǫtgá:dǫ-ˀ\\
 \textsc{1inp.o}-{\semireflexive}-have.fun-{\stative}\\
\glt `we all are having fun'
\z
\z

The \textsc{m} class verbs have mixed pairings of habitual and stative endings: the \textsc{ma} classes take \stem{-haˀ} {\habitual} endings with \stem{-ǫh} {\stative} endings \xref{ex:aspclass4} and the \textsc{mb} classes take \stem{-(h)s} {\habitual} endings with \stem{-ˀ} {\stative} endings \xref{ex:aspclass5}.

\ea\label{ex:aspclass4} \textsc{ma1} class verb stem ends with HD or ˀD and takes \stem{-haˀ} {\habitual}, \stem{-ø} {\zeropunctual}, and \stem{-ǫh} {\stative} endings
\ea hęnáˀswa̱htaˀ\\
\gll hęn-áˀswa̱ht-haˀ\\
 \textsc{3ns.m.a}-extinguish-{\habitual}\\
\glt `they are firemen'

\ex ęgáˀswaht\\
\gll ę-g-áˀswah-t-ø\\
 \fut-\textsc{1s.a}-extinguish-{\zeropunctual}\\
\glt `I will extinguish it'

\ex agáˀswa̱hdǫh\\
\gll ag-áˀswa̱hd-ǫh\\
 \textsc{1s.p}-extinguish-{\stative}\\
\glt `I am extinguishing it'
\z
\z


\ea\label{ex:aspclass5} \textsc{mb1} verb stem ends with a vowel and takes \stem{-hs} {\habitual}, \stem{-:ˀ} {\punctual}, and \stem{-ˀ} {\stative} endings
\ea degadawę́:nyehs\\
\gll de-g-ad-awę́:nye-hs\\
 {\dualic}-\textsc{1s.a}-{\semireflexive}-stir-{\habitual}\\
\glt ‘I wander about all the time’, `I am a wanderer'

\ex dęgadawę́:nye:ˀ\\
\gll d-ę-g-ad-awę́:nye-:ˀ\\
 {\dualic}-{\future}-\textsc{1s.a}-{\semireflexive}-stir-{\punctual}\\
\glt `I will wander, travel, walk about'

\ex deyagodáwęnyeˀ\\
\gll de-yago-d-áwęnye-ˀ\\
 {\dualic}-\textsc{3s.fi.p}-{\semireflexive}-stir-{\stative}\\
\glt `she is walking about'
\z
\z

Finally, stative-only and habitual-only verbs take a subset of the three-aspect conjugation class endings. For example, the stative-only verb in \xref{ex:aspclass6} belongs to the \textsc{s20a} conjugation class, like its “parent” verb \stem{gri} ‘wrinkle, fold’.

\ea\label{ex:aspclass6} An \textsc{s20a}, stative-only verb
\ea degéˀnyǫgri:\\
\gll de-g-é-ˀnyǫ-gri-:\\
 {\dualic}-\textsc{1s.a}-{\joinerE}-nose-fold-{\stative}\\
\glt `my nose is stuffed up'

\cfex{dawá:tgri:k \\
	\gll d-a-wá:-t-gri-:k\\
	{\cislocative}-{\factual}-\textsc{3s.a}-{\semireflexive}-fold-{\punctualmodalizer}\\
	\glt `it pulled back, flinched, shrank' (three-aspect verb)}

\cfex{totgri: \\
	\gll t-ho-t-gri-:\\
	{\cislocative}-\textsc{3s.m.p}-{\semireflexive}-fold-{\stative}\\
	\glt ‘he’s a wimp’, `he pulls back' (three-aspect verb)}
\z
\z


\subsection{List of aspect conjugation classes} \label{List of aspect conjugation classes}
The aspect conjugation classes are listed in \tabref{figtab:1:Hclass} (page \pageref{figtab:1:Hclass}), \tabref{figtab:1:Sclass} (page \pageref{figtab:1:Sclass}), and \tabref{figtab:1:MBclass} (page \pageref{figtab:1:MBclass}). These have been modified from \citet{sasse_far_1998} in order to more clearly show the sound changes that occur when the verb stem and aspect suffix are joined. (Also, some smaller subclasses have been removed from the tables.) \emph{Sounds in uppercase letters in the tables are deleted or not pronounced.}


\begin{table}
\caption{H-class (after \cite{sasse_far_1998})} 
\label{figtab:1:Hclass}
\begin{tabular}{lllll}
\lsptoprule
stem-final sound & hab &  class & stat & punc \\\midrule
d & \stem{-haˀ} & H1 & \stem{-ø} & \stem{-ęˀ}\\
aw, ow & \stem{-ehaˀ} & H2 & \stem{a:W-ˀ}, \stem{o:W-ˀ}\footnote{Uppercase letters DELETE.} & \stem{-ęˀ}\\
V\footnote{V = any vowel.} & \stem{-haˀ} & H3 & \stem{-ˀ} & \stem{-:ˀ}\\
\stem{+ę} ‘lie’ & \stem{+ę-haˀ}, \stem{+ę-hęˀ} & H3 & \stem{+ęˀ} & \stem{+ę-:ˀ}\\
\stem{yę} ‘lie’ & \stem{yę-haˀ}, \stem{yę-hęˀ} & H3 & \stem{yęˀ} & \stem{yę-:ˀ}\\
dr, gr & \stem{-ehaˀ} & H4 & \stem{e:ˀ} & \stem{-ęˀ}\\
hR\footnote{R deletes if between vowels.} & \stem{-ehaˀ} & H4 & \stem{-e:ˀ} & \stem{-ęˀ}\\
a: & \stem{-haˀ} & H5 & \stem{-ˀ} & \stem{aę-ˀ} instead of \stem{a:-ˀ}\\
\lspbottomrule
\end{tabular}
\end{table}


\begin{table}
\caption{S-class (after \cite{sasse_far_1998})\label{figtab:1:Sclass}}
\fittable{\begin{tabular}{lllll}
\lsptoprule
stem-final sound & hab & class & stat & punc \\
\midrule
i, e, ę, o, ǫ, a, a: & \stem{-s}, \stem{-h} & S1 & \stem{-:} & \stem{-ˀ}\\
i, e, ę, o, ǫ, a, a: & \stem{-s}, \stem{-h} & S1 & \stem{-ˀ-ǫh} (\textsc{inch}-{\stative}) & \stem{-ˀ}\\
{}ˀ & \stem{-s} & S2 & \stem{-ǫh} & \stem{-ø}\\
d & \stem{D-s}\footnote{Uppercase letters DELETE.} & S3 & \stem{-ǫh} & \stem{-ø}\\
g & \stem{G-s} & S4 & \stem{-ǫh} & \stem{-ø}\\
h & \stem{-s} & S5 & \stem{-ǫh} & \stem{-ø}\\
h & \stem{-s} & S5 & \stem{H-ˀ-ǫh} (\textsc{inch}-{\stative}) & \stem{-ø}\\
a:h & \stem{a:H-s} & S5 & \stem{-ǫh} & \stem{a:H-ø}\\
s & \stem{-ahs} & S6 & \stem{-ęh} & \stem{-ø}\\
kh, nh & \stem{-ahs} & S7a & \stem{-ęh} & \stem{-aˀ}\\
h & \stem{-ahs} & S7b & \stem{-ęh} & \stem{-aˀ}\\
Vn\footnote{V = any vowel.} & \stem{-ahs} & S8 & \stem{-ęh} & \stem{N-:ˀ}\\
Vw & \stem{-ahs} & S9 & \stem{-ęh} & \stem{W-:ˀ}\\
Cw & \stem{-ahs} & S10 & \stem{-ęh} & \stem{o-ˀ}\footnote{W becomes O.}\\
aˀw & \stem{-ahs} & S11 & \stem{-ęh} & \stem{aWˀ}\\
    & \stem{aWˀ-s} & \\
a:ˀw & \stem{a:ˀW-s} & S11 & \stem{-ęh} & \stem{a:Wˀ}\\
d, g & \stem{-ahs} & S12 & \stem{-ǫh}, \stem{ęh} & \stem{-ø}\footnote{D,G pronounced as T,K respectively.}\\
sˀ & \stem{-ahs} & S13 & \stem{-ǫ} & \stem{-a:ˀ}\\
kd & \stem{-ǫhs} & S14 & \stem{-ǫ:} & \stem{kD-ø}\\
aǫ, ęǫ & \stem{-s} & S15 & \stem{-ø} & \stem{aǪ:-ˀ}\\
       &      &     &      & \stem{ęǪ:-ˀ}\\
ey & \stem{-ǫhs} & S16 & \stem{-ǫ:} & \stem{eY:-ˀ}\\
Cy (dy, gy, ny) & \stem{-eˀs}, \stem{-ǫhs} & S17 & \stem{-ǫ:} & \stem{Ci-ˀ} instead of \stem{Cy-ˀ}\\
Cy (sy) & \stem{-ǫhs} & S17 & \stem{sY-ǫ:} & \stem{si-ˀ} instead o: \stem{sy-ˀ}\\
dr & \stem{-ǫhs} & S18 & \stem{-ǫ:} & \stem{DRs-e:ˀ}\\
Ry\footnote{Stem historically ended in R but now ends in NY/WI or NY/I.}  & \stem{ny-eˀs} & S19 & \stem{ny-ǫ:} & \stem{wi-ˀ} or \stem{i-ˀ} instead of \stem{Ri-ˀ}\\
V:G & \stem{VG-hs} & S20a & \stem{V:G-ø} or \stem{V:g-ǫh} & \stem{V:k-ø}\\
VˀG & \stem{VˀG-s} & S20b & \stem{VˀG-ǫh} & \stem{Vˀk-ø}\\
\lspbottomrule
\end{tabular}}
\end{table}


\begin{table}
\caption{M-class (after \cite{sasse_far_1998})}
\label{figtab:1:MBclass}
\begin{tabular}{lllll}
\lsptoprule
stem-final sound & hab &  class & stat & punc \\\midrule
hd, ˀd & \stem{-haˀ}\footnote{Pronounced as \phonet{ht-haˀ} and \phonet{ˀt-haˀ} respectively.} & MA1 & \stem{-ǫh} & \stem{-ø}\\
hsd & \stem{-haˀ}\footnote{Pronounced as \phonet{-hst-haˀ}.} & MA1 & \stem{-ǫh} & \stem{hsD-ø}\footnote{Uppercase letters DELETE.}\\
d, g & \stem{-haˀ}\footnote{Pronounced as \phonet{t-haˀ} and \phonet{k-haˀ} respectively} & MA2 & \stem{-ǫh} & \stem{-ø}\\
hgw & \stem{-haˀ} \footnote{Pronounced as \phonet{-hkw-haˀ}.} & MA3 & \stem{-ęh} & \stem{-ø}\\
a: & \stem{-haˀ} & MA4 & \stem{-ø} & \stem{-ˀ}\\
eh & \stem{H-haˀ} & MA5 & \stem{-ęh} & \stem{-aˀ}\\
V\footnote{V = any vowel.} & \stem{-hs} & MB1 & \stem{-ˀ} & \stem{-:ˀ}\\
(3 verbs) & \stem{-ę-hs}, \stem{-ę:-s} & MB2 & \stem{-e-ˀ} & \stem{-ę-h}, \stem{ę-ø}\\
\lspbottomrule
\end{tabular}
\end{table}



\section{Lexical aspect and sentential aspect} \label{Lexical aspect and sentential aspect}
This section introduces the concepts of lexical and sentential aspect, in order to provide more context for the discussion of the “present tense” (see \sectref{The habitual, punctual and stative aspects}). 

A type of aspect known as lexical aspect or \textsc{Aktionsart} (\cite{vendler_verbs_1957}) is inherent to the meaning of verb stems. For example, certain verbs describe an event that naturally has no duration, one that begins and ends at (virtually) the same time \xref{ex:aktionex}. 

\ea\label{ex:aktionex} instantaneous events
\ea aˀé:yǫˀ\\
\gll aˀ-é:-yǫ-ˀ\\
 {\factual}-\textsc{3s.fi.a}-arrive-{\punctual}\\
\glt `she arrived'

\ex ęga:dé:gaˀt\\
\gll ę-g-a:dé:g-a-ˀt-ø\\
 \fut-\textsc{1s.a}-burn-{\joinerA}-{\causative}-{\zeropunctual}\\
\glt `I will start a fire'
\z
\z

Other verbs denote an activity that naturally extends or repeats over time, but without any implied end-point \xref{ex:aktionex2}, page \pageref{ex:aktionex2}.

\FloatBarrier
\ea\label{ex:aktionex2} durative events with no end-point
\ea ęhségawe:ˀ\\
\gll ę-hs-é-gawe-:ˀ\\
 \fut-\textsc{2s.a}-{\joinerE}-paddle-{\punctual}\\
\glt `you will row'

\ex sagawe̱há:gyeˀ\\
\gll sa-gawe̱-h-á:-gy-e-ˀ\\
 \textsc{2s.p}-paddle-\textsc{euph.h}-{\joinerA}-{\progressive}-go-{\stative}\\
\glt `you are paddling along'
\z
\z

Finally, other verbs describe a state of affairs (as opposed to an event, happening, or activity, \ref{ex:aktionex3}). 

\ea\label{ex:aktionex3} states
\ea agadǫ̱hswéˀdanih\\
\gll ag-ad-ǫ̱hswéˀd-a-ni-h\\
 \textsc{1s.p}-{\semireflexive}-hunger-{\joinerA}-{\benefactive}-{\habitual}\\
\glt `I am hungry'

\ex ogé:draˀ \\
\gll o-gé:draˀ\\
 \textsc{3s.p}-unripe.{\stative}\\
\glt ‘it is green, unripe’, `raw fruit'
\z
\z

These are just a few examples of the possible types of lexical aspect.\footnote{\citet{vendler_verbs_1957}’s original Aktionsart categories are listed below. More categories have been proposed in later works on aspect.


\begin{itemize}
\item \textsc{achievement} verbs describe an event that has no duration or that occurs in an instant. The beginning and end-points are nearly simultaneous; the event results in a new state of affairs;
\item \textsc{accomplishment} verbs describe events that take some time and that have a natural end-point. The event results in a new state of affairs;
\item \textsc{activity} verbs describe events that take some time, but which do not necessarily have an end-point. The event does not result in a new state of affairs;
\item \textsc{states} describe a state of affairs (and not an event).
\end{itemize}}


The main lexical aspect distinction in \textit{Ǫgwehǫwéhne̱ha:ˀ} ‘Iroquoian’ languages is between \textsc{nonconsequential} and \textsc{consequential} verbs (\cite{chafe_consequential_1980}).\footnote{\citet{chafe_consequential_1980}’s \textsc{consequential} category includes Vendler’s \textsc{accomplishment} and \textsc{achievement} types, while the non-consequential category corresponds to Vendler’s \textsc{activity} type.} Consequential verbs imply an end-point or result in a new state of affairs. In contrast, nonconsequential verbs have no end-point or fail to result in a new state of affairs. The consequential/nonconsequential distinction is relevant for interpreting the meaning (including the time-frame) of the habitual and stative forms of verbs of three-aspect verbs (see \sectref{Meaning of the habitual aspect}, \sectref{Meaning of the stative aspect}). 

Another type of aspect known as \textsc{sentential} aspect is conveyed by \emph{groups} of words (specifically, verbs and the words that modify verbs in sentences, including particles that function as “adverbs” (described earlier) and the objects -- incorporated or not -- that undergo the verb’s action). For example, the activity (verb) in \xref{ex:aktionex5} does not imply a natural end-point. However, when the same verb incorporates a noun (an object undergoing the action of the verb), the activity gains an end-point \xref{ex:aktionex4}. 

\ea\label{ex:aktionex5} no end-point, no change of state
\ea dęgéhsnyeˀ\\
\gll d-ę-gé-hsnye-ˀ\\
{\dualic}-{\future}-\textsc{1s.a}-care.for-{\punctual}\\
\glt ‘I will look after it’
\z
\z

\ea\label{ex:aktionex4} end-point, with change of state
\ea atgatnǫ̱hsáhsnyeˀ\\
\gll a-t-g-at-nǫ̱hsá-hsnye-ˀ\\
{\factual}-{\dualic}-\textsc{1s.a}-{\semireflexive}-house-care.for-{\punctual}\\
\glt ‘I cleaned up the house’ 
\z
\z 



\section{Utterances, clauses, phrases, and sentences} \label{Utterances, clauses, phrases, and sentences}
The concepts of \textsc{utterance}, \textsc{clause}, \textsc{phrase}, and \textsc{sentence} are defined in this section. All of these terms refer to groups of related or mutually-relevant words. 

An \textsc{utterance} is any group of related words uttered by a speaker. Utterances can include any type of word (nouns, verbs, or particles), as long as the words make sense together. Some utterances just include particles \xref{ex:sentencex5}, while others include verbs and other related words (shown in later examples). 

\ea\label{ex:sentencex5} all-particle utterances 
\ea 
\gll Dęˀ hne:ˀ.\\
what in.fact\\
\glt ‘That’s why.’

\ex
\gll Dó: í:ˀ!\\
how I\\
\glt ‘Let me!’

\ex
\gll Nę́:dah. \\
here.this\\
\glt ‘Here, take this.’

\ex
\gll Trehs giˀ gyę:ˀ! \\
too.much just the.one\\
\glt ‘My goodness!’
\z
\z 

\textsc{Clauses} obligatorily contain a single verb, and can optionally include other relevant nouns or particles. The clause (and utterance) in \xref{ex:sentencetypeexagaina} only contains the obligatory verb. (The clause is between square brackets, and the verb is shown in \textbf{bold}.) The clause in \xref{ex:sentencetypeexagainb} includes a verb and a noun phrase (defined later). In contrast, the utterance in \xref{ex:sentencetypeexagainc} contains two clauses, the second of which has two particles and a verb. The utterance in \xref{ex:sentencetypeexagaind} also includes two clauses, the second of which contains a verb, followed by a noun (technically, \cayuga{niga:gú:sˀuh} is a \textsc{verb} functioning as a “noun”). 

\ea\label{ex:sentencetypeexagain} utterances and sentences
\ea {}[\textbf{Agatsęnǫ́:ni:.}] \label{ex:sentencetypeexagaina}
\glt ‘I am happy.’
\ex  {}[\textbf{Godiˀgrǫ́ˀ} neˀ Mary.]\\\label{ex:sentencetypeexagainb}
\gll godiˀgrǫ́ˀ neˀ Mary\\
she.is.shy the Mary\\
\glt ‘Mary is shy.’

\ex {}[\textbf{Hętsyę́:ˀ}] [shęh hǫ: \textbf{hesá:gwęh}.]\\\label{ex:sentencetypeexagainc}
\gll Hętsyę́:ˀ shęh hǫ: hesá:gwęh\\
you.will.put.it.back.there that where you.gathered.it.there\\
\glt ‘You will put it back where you got it.’

\ex {}[\textbf{Agyǫ:díh}] [\textbf{degakegáhneˀ} niga:gú:sˀuh].\\\label{ex:sentencetypeexagaind}
\gll Agyǫ:díh degakegáhneˀ niga:gú:sˀuh\\
I’m.smiling I’m.looking.at.them children\\
\glt ‘I am smiling (because) I am looking at the children.’ 
\z
\z

\textsc{Phrases} are groups of words which minimally include either a noun, verb, or particle. A noun with related words is a \textsc{noun phrase}. For example, the noun phrase \textit{neˀ Mary}, \xref{ex:sentencetypeexagainb}, consists of a particle \textit{neˀ} ‘the’ and a noun (Mary). A verb with related words is a \textsc{verb phrase} (which is also the smallest type of clause). An example is the second verb phrase (also a clause) in \xref{ex:sentencetypeexagainc}. Finally a \textsc{particle phrase} (or \textsc{particle group}), is a group of related particles, such as \textit{shęh hǫ:} in \xref{ex:sentencetypeexagainc}. (Particle groups typically occur at the beginning of clauses.)

The utterances in \xref{ex:sentencetypeexagain} are also \textsc{sentences}. Sentences consist of one or more clauses, and therefore contain one or more verbs, plus other related words. (Unlike sentences, utterances do not require verbs, as shown in \ref{ex:sentencex5}.) The sentences in (\ref{ex:sentencetypeexagain}a, b) each have one clause, while the sentences in (\ref{ex:sentencetypeexagain}c, d) each have two clauses. The number of clauses per sentence is relevant for the distinction between simple and complex sentences, which is described next.  


\subsection{Simple and complex sentences, clause types} \label{Simple and complex sentences, clause types}
Sentences are either \textsc{simple} or \textsc{complex}, depending on the number of clauses. Simple sentences have just one clause, and complex sentences have more than one. 

The simple sentence in \xref{ex:sentencetypeexa} contains one clause, which in turn is so small that it only consists of the obligatory element, a verb (in \textbf{bold}). The simple sentence in \xref{ex:sentencetypeexb} also contains just one clause, consisting of a verb  plus a particle.
 
\ea\label{ex:sentencetypeex}
\ea \textbf{Agatsęnǫ́:ni:} ‘I am happy’\label{ex:sentencetypeexa}

\ex \textbf{Ęwá:dǫh} gęh? ‘May I?’\label{ex:sentencetypeexb}
\z
\z
xa 
Complex sentences contain more than one clause, and therefore, more than one verb plus associated words. The second clause of each sentence is shown between square brackets in \xref{ex:subordsentex}. 

\ea\label{ex:subordsentex}
\ea \textbf{Agatsęnǫ:ní:} [shęh \textbf{ahsyǫˀ}].\\\label{ex:subordsentexa}
\gll agatsęnǫ:ní: shęh ahsyǫˀ  \\
I.am.happy that you.arrived\\
\glt ‘I am happy [that you’ve arrived].’

\ex \textbf{Daskro:wíh} [sǫ: nˀaht \textbf{daǫdekǫ́nya̱hneˀ}].\\\label{ex:subordsentexb}
\gll Daskro:wíh sǫ: nˀaht daǫdekǫ́nya̱hneˀ\\
tell.me who person someone.will.come.and.eat\\
\glt ‘Tell me [who’s coming to eat].’
\z
\z 

The complex sentences in \xref{ex:subordsentex} each consist of an independent and dependent clause, described next. 

\subsection{Independent and dependent clauses, relative clauses} \label{Independent and dependent clauses, relative clauses}
Complex sentences can include either independent and dependent clauses. Independent clauses can stand on their own as complete thoughts. Examples include \textit{agatsęnǫ́:ni:} ‘I am happy’ or \textit{daskro:wíh} ‘tell me’, both from example \xref{ex:subordsentex} in the previous section. In contrast, dependent clauses such as \textit{shęh ahsyǫˀ} ‘that you’ve arrived’ \xref{ex:subordsentexa} or \textit{sǫ: nˀaht daǫdekǫ́nya̱hneˀ} ‘who’s coming to eat’ \xref{ex:subordsentexb} sound incomplete or normally cannot stand alone (unless the context makes their meaning clear). In \xref{ex:subordsentex}, the dependent clauses occur \emph{after} the independent ones. However, dependent clauses also occur \emph{before} independent ones (examples are provided in later sections). 

A distinct type of complex sentence, shown in \xref{ex:dependentclausesoundsindependent}, consists of two clauses, each of which could stand alone as independent. However, the second clause is clearly relevant to the first one, even though it has no special words linking it to the first clause. This type of complex sentence contains \textsc{juxtaposed} clauses.

\ea\label{ex:dependentclausesoundsindependent} 
\gll [A:hęni:hę́:ˀ] [hahjáotaˀ].\\
he.would.quit he.smokes.all.the.time\\
\glt ‘He would quit smoking.’ 
\z
	
Independent and dependent clauses also have distinct \emph{functions}, either as “statements” or as “questions”. This results in four types of complex sentence \xref{ex:clausetypeex}. 

\ea\label{ex:clausetypeex} 
\ea Simple sentences, consisting of\label{ex:clausetypeexa}
\begin{itemize}
    \item an \textsc{independent clause} functioning as a “stand-alone statement” \xref{ex:sentencetypeexa}
    \item an \textsc{independent clause} functioning as a “direct question” \xref{ex:sentencetypeexb}
\end{itemize}

\ex Complex sentences, consisting of an independent clause \emph{and}\label{ex:clausetypeexb}
\begin{itemize}
    \item a \textsc{dependent clause} functioning as a “dependent statement” \xref{ex:subordsentexa}
    \item a \textsc{dependent clause} functioning as an “indirect question” \xref{ex:subordsentexb}
\end{itemize} 
\z
\z

Finally, dependent \emph{statements} -- \xref{ex:clausetypeexb} or \xref{ex:subordsentexa} -- also have two distinct functions \xref{ex:clausetypeex100}. 

\ea\label{ex:clausetypeex100}
\ea a dependent statement functioning to complete the meaning of an independent clause;\label{ex:clausetypeex100a}

\ex a dependent statement functioning to complete the meaning of a noun.\label{ex:clausetypeex100b}
\z
\z 

The first function, \xref{ex:clausetypeex100a}, was illustrated in \xref{ex:subordsentexa}, where the dependent statement completes the meaning of the independent clause. The second type, \xref{ex:clausetypeex100b}, is relevant for nouns, and is described next. 

\subsubsection{Dependent statements functioning as “nouns”, and relative clauses}
Certain dependent statements function as “nouns” in their own right, \xref{ex:nounclause}.

\ea\label{ex:nounclause}
\gll shęh ní:waˀs\\
that it.is.a.certain.size\\
\glt ‘sizes’

\cfex{\gll ní:-w-aˀs\\
	{\partitive}-\textsc{3s.a}-certain.size.{\stative}\\
	}
\z

Other dependent statements (technically known as \textsc{relative clauses}) modify or complete the meaning of nouns. Example \xref{ex:sentencetypeex6} includes the noun phrase \textit{Gwi:déh hǫwa:yę:dí: hǫ́:gweh} ‘the man that Peter knows’, which functions as a type of multi-word “noun” for the rest of the sentence (describing the person who likes sugar). It includes a relative clause, \textit{Gwi:déh hǫwa:yę́:di:} ‘Peter he.knows.him’, plus the noun whose meaning is modified, \textit{hǫ́:gweh} ‘the man’.

\ea\label{ex:sentencetypeex6}
\gll [Gwi:déh hǫwa:yę:dí: hǫ́:gweh] ho:gáˀs  nawę́ˀdaˀ. \\
[Peter he.knows.him man] he.likes.the.taste.of.it sugar\\
\glt ‘[The man that Peter knows] likes candy.’
\z
