\chapter{Sound changes in word formation} \label{Sound changes in word formation}
Sound changes in word formation involve adjusting prefixes, suffixes, and stems for ease of pronunciation (or \textsc{euphony}, adj. \textsc{euphonic}). The types of sound change are described in the following sections.

\section{Simplifying two consonants to one} \label{Simplifying two consonants to one}
For ease of pronunciation, two consonants are simplified to one at the end of verbs. For example, the \stem{-ˀ} {\punctual} deletes after consonant-final verb stems (\stem{ganyaˀg} in \ref{ex:clustersimpex4a}, or \stem{hrih} in \ref{ex:clustersimpex5a}), but is pronounced after verb stems ending with a vowel (\stem{hní:nǫ} in \ref{ex:clustersimpex3}).

\ea\label{ex:clustersimpex4} \stem{…ganyaˀg-ˀ} simplified to \stem{…ganyaˀk-ø}\\
\ea aˀéganyaˀk\\\label{ex:clustersimpex4a}
\gll aˀ-e-ganyaˀg-ˀ\\
 {\factual}-\textsc{3s.fi.a}-pay-{\punctual}\\
\glt `she paid'

\ex gagánya̱ˀgǫh\\\label{ex:clustersimpex4b}
\gll ga-ga-nya̱ˀg-ǫh\\
 \textsc{3s.a}-price-pay-{\stative}\\
\glt `payment'
\z
\z

\ea\label{ex:clustersimpex5} \stem{hrih-ˀ} simplified to \stem{hrih-ø}
\ea hękrih\\\label{ex:clustersimpex5a}
\gll h-ę-k-hrih-ˀ\\
 {\translocative}-{\future}-\textsc{1s.a}-pour-{\punctual}\\
\glt `I will pour'


\ex gahíhǫh\\\label{ex:clustersimpex5b}
\gll ga-hríh-ǫh\\
 \textsc{3s.a}-spill-{\stative}\\
\glt `it is spilled'
\z
\z

\ea\label{ex:clustersimpex3} no simplification for \stem{…ǫ-ˀ} (etc.)
\ea akní:nǫˀ\\
\gll a-k-hní:nǫ-ˀ\\
 {\factual}-\textsc{1s.a}-buy-{\punctual}\\
\glt `I did buy it'

\ex ękní:nǫˀ\\
\gll ę-k-hní:nǫ-ˀ\\
\fut-\textsc{1s.a}-buy-{\punctual}\\
\glt ‘I will buy it'
\z
\z

To avoid \phonet{ks} and \phonet{ts} at the end of a word, \phonet{k} or \phonet{t} deletes before the \stem{-s} \textsc{\habitual} suffix \xref{ex:clustersimpex2a}, \xref{ex:clustersimpexa}.\footnote{In contrast, speakers of other \textit{Ǫgwehǫ́:weh} languages keep word-final \phonet{ts} or \phonet{ks}, as shown by the following \textit{Kanien’kéha} (Mohawk) examples.

\ea\label{ex:footnoteMohawk}
\ea rató:rats\\
\glt ‘he is a hunter’
\ex í:keks
\glt ‘I am eating something’
\z
\z}

\ea\label{ex:clustersimpex2} \phonet{ts} simplified to \phonet{s}\\
hadó:wa:s\\\label{ex:clustersimpex2a}
\gll h-adó:wa:t-s\\
 \textsc{3s.m.a}-hunt-{\habitual} (\phonet{t} deleted)\\
\glt `he is a hunter'
\cfex{ęha:dó:wa:t\\
	\gll ę-h-a:dó:wa:t-ˀ\\
	\fut-\exsc{\textsc{3s.m.a}}-hunt-{\punctual} (\phonet{t} is not deleted, but the \phonet{-ˀ} {\punctual} is deleted)\\
	\glt `he will hunt'}
\z


\ea\label{ex:clustersimpex} \phonet{ks} simplified to \phonet{s}\\
gi̱hsa:s\\\label{ex:clustersimpexa}
\gll g-i̱hsa:k-s\\
 \textsc{1s.a}-look.for-{\habitual} (\phonet{k} deleted)\\
\glt `I am looking for it'
\cfex{ęgíhsa:k\\
	\gll ę-g-íhsa:k-ˀ\\
	\fut-\textsc{1s.a}-look-{\punctual} (\phonet{k} is not deleted, but the \phonet{-ˀ} {\punctual} is deleted)\\
	\glt `I will look for it'}
\z


\section{Preserving, merging, and deleting vowels} \label{Preserving, merging, and deleting vowels}
Two vowels in a row are avoided, particularly between a pronominal prefix and stem. Vowels are either merged or deleted to achieve this result. For example, prefix-final \phonet{i} and stem \phonet{i} simplify to just one \phonet{i} sound \xref{ex:wdsoundchangex5a}. In contrast, prefix-final \phonet{i} remains when the following stem begins with a consonant \xref{ex:wdsoundchangex5b}, and stem-initial \phonet{i} is preserved after prefixes ending with a consonant \xref{ex:wdsoundchangex5c}. 

\ea\label{ex:wdsoundchangex5} \phonet{i-i} becomes \phonet{i}
\ea gadíˀdrǫˀ\\\label{ex:wdsoundchangex5a}
\gll gadi-iˀdrǫ-ˀ\\
\textsc{3ns.a}-live-{\stative}\\
\glt ‘they live (designates a shed, dog house, etc.)’

\ex gadinǫ̱hweˀs\\\label{ex:wdsoundchangex5b}
\gll gadi-nǫ̱hwe-ˀs\\
 \textsc{3ns.a}-like-{\habitual}\\
\glt `they like it'

\ex hęgiˀdrǫ́:daˀk\\\label{ex:wdsoundchangex5c}
\gll h-ę-g-iˀdrǫ́:-d-aˀk\\
 {\translocative}-{\future}-\textsc{1s.a}-live-{\causative}-{\modalizer} \\
\glt `where I will live'
\z
\z

\phonet{e-i} are simplified to \phonet{i} to avoid two vowels in a row \xref{ex:wdsoundchangex6a}. In contrast, stem \phonet{i} remains after prefixes ending with a consonant \xref{ex:wdsoundchangex6b}. For comparison, \xref{ex:wdsoundchangex6c} shows prefix \phonet{e} before stems beginning with a consonant.

\ea\label{ex:wdsoundchangex6} \phonet{e-i} becomes \phonet{e}
\ea ehé:yǫhs\\\label{ex:wdsoundchangex6a}
\gll e-ihey-ǫhs\\
 \textsc{3s.fi.a}-die-{\habitual} \\
\glt `she is dying, on her death bed'

\ex gi̱hé:yǫhs\\\label{ex:wdsoundchangex6b}
\gll g-ihey-ǫhs\\
 \textsc{1s.a}-die-{\habitual}\\
\glt `I am dying'

\ex egeˀahsrǫni:ˀ\\\label{ex:wdsoundchangex6c}
\gll e-geˀa-hsrǫ́:ni-h\\
 \textsc{3s.fi.a}-hair-fix-{\habitual}\\
\glt `she is a hairdresser'
\z
\z

Instead of deletion, two vowels \emph{merge} into a new vowel in several contexts. First, prefix \phonet{a} and stem \phonet{i} merge to \phonet{ę} \xref{ex:wdsoundchangex7a}. In contrast, stem \phonet{i} remains after prefixes ending with a consonant \xref{ex:wdsoundchangex7b} and prefix \phonet{a} remains before stems beginning with a consonant \xref{ex:wdsoundchangex7c}. 

\ea\label{ex:wdsoundchangex7} \phonet{a-i} become \phonet{ę}
\ea hęhé:yǫhs\\\label{ex:wdsoundchangex7a}
\gll ha-ihé:y-ǫhs\\
 \textsc{3s.m.a}-die-{\habitual}\\
\glt `he is dying'
\ex gi̱hé:yǫhs\\\label{ex:wdsoundchangex7b}
\gll g-ihé:y-ǫhs\\
 \textsc{1s.a}-die-{\habitual}\\
\glt `I am dying'
\ex hanǫhsǫ:nih\\\label{ex:wdsoundchangex7c}
\gll ha-nǫhs-ǫ́:ni-h\\
 \textsc{3s.m.a}-house-build-{\habitual}\\
\glt `he is a carpenter'
\z
\z

Prefix \phonet{e} and stem \phonet{e} also merge into a single \phonet{ę} \xref{ex:wdsoundchangex8}. 

\ea\label{ex:wdsoundchangex8} \phonet{e-e} become \phonet{ę}
\ea dǫdáęˀ\\
\gll dǫdá-e-e-ˀ\\
 {\dualicfactualcislocative}\exsc{-3s.fi.a}-go-{\punctual}\\
\glt `she is coming back'
\cfex{dǫdá:geˀ\\
	\gll dǫdá:-g-e-ˀ\\
	{\dualicfactualcislocative}\exsc{-1s.a}-go-{\punctual}\\
	\glt `I am coming back'}
\ex ętgáęˀ\\
\gll ę-t-gáe-e-ˀ\\
 \fut-{\cislocative}-\exsc{3ns.fi.a}-go-{\punctual}\\
\glt `they will come'
\cfex{ęteˀ\\
	\gll ę-t-h-e-ˀ\\
	\fut-{\cislocative}-\exsc{\textsc{3s.m.a}}-go-{\punctual}\\
	\glt `he will come this way'}
\ex í:yę:\\
\gll í:-ye-e-:\\
 {\prothetic}-\textsc{3s.fi.a}-want-{\stative}\\
\glt `she wants'
\cfex{ihse:\\
	\gll i-hs-e-:\\
	{\prothetic}-\textsc{2s.a}-want-{\stative}\\
	\glt `you want'}
\z
\z

Another strategy for avoiding two vowels in a row is to add a \textsc{euphonic} \phonet{y} between certain pronominal prefixes and stem vowels \xref{ex:wdsoundchangex9}. (\textsc{euphonic} elements are sounds inserted for ease of pronunciation, see \sectref{Euphonic sounds}.)
 
\ea\label{ex:wdsoundchangex9} \phonet{ǫ-ǫ} tends to become \phonet{ǫyǫ}\\
gǫ́:yǫˀ\\
\gll gǫ́:-y-ǫ-ˀ\\
 \exsc{1s:2s}-\textsc{euph.y}-give-{\punctual}\\
\glt `I will give you'
\z


\phonet{y}-insertion sometimes occurs between prefixes like \phonet{swa-} \textsc{3p.a}, and a following stem-initial \phonet{ę} or \phonet{ǫ} \xref{ex:wdsoundchangex10a}. Again, this strategy prevents the presence of two vowels in a row. The strategy in \xref{ex:wdsoundchangex10a} is possibly a recent development: the more conservative strategy is to instead use a different pronunciation like \phonet{j-} \textsc{3p.a} before stem-initial \phonet{ę} or \phonet{ǫ} \xref{ex:wdsoundchangex10b}.

\ea\label{ex:wdsoundchangex10} treatment of \phonet{a-ǫ}
\ea ęhswá:yǫˀ\\\label{ex:wdsoundchangex10a}
\gll ę-hswá:-y-ǫ-ˀ\\
 \fut-\exsc{2p.a}-\textsc{euph.y}-give-{\punctual}  (innovative extra \phonet{y})\\
\glt `you will give it'
\ex ęjǫˀ\\\label{ex:wdsoundchangex10b}
\gll ę-j-ǫ-ˀ\\
\fut-\exsc{2p.a/2pl.o}-give-{\punctual}\\
\glt ‘it will give you all’, ‘you all will give it’ (conservative, no extra \phonet{y})
\z
\z

Prefixes like \phonet{hęn-} \xref{ex:wdsoundchangex11a} and \phonet{hon-} \xref{ex:wdsoundchangex11c} are used when the verb stem begins with a vowel, but \phonet{hadi-} \xref{ex:wdsoundchangex11b} or \phonet{hodi-} \xref{ex:wdsoundchangex11d} are used before stems beginning with a consonant. The use of \phonet{hęn-} and \phonet{hon-} before vowel-initial verb stems again prevents having two vowels in a row.

\ea\label{ex:wdsoundchangex11}
\ea ahęnéhehk\\\label{ex:wdsoundchangex11a}
\gll hęn-é-he-hk\\
 \textsc{3ns.m.a}-think-{\habitual}-{\former} \\
\glt `they (males) were thinking'
\ex hadíˀnhahgyaˀs\\\label{ex:wdsoundchangex11b}
\gll hadí-ˀnhahg-yaˀk-s\\
 \textsc{3ns.m.a}-log-cut-{\habitual}\\
\glt `loggers'
\ex honé:ˀǫ:\\\label{ex:wdsoundchangex11c}
\gll hon-é:ˀ-ǫ:\\
 \textsc{3ns.m.p}-will-{\stative}\\
\glt `they have willed, decided it'
\ex hodí:yęˀ \\\label{ex:wdsoundchangex11d}
\gll hodí:-yę-ˀ\\
 \textsc{3ns.m.p}-lie-{\stative}\\
\glt ‘they have’, `they have lain'
\z
\z


\section{\phonet{r} and \phonet{hr}} \label{R and HR}
When noun or verb stems begin with \phonet{r}, the \phonet{r} is missing after a vowel \xref{ex:rdelex}, but retained after a consonant \xref{ex:rdelex2}.

\ea\label{ex:rdelex} \phonet{r} deleted between vowels
\ea gáęnaˀ\\
\gll gá-ręn-aˀ\\
 \textsc{3s.a}-song-{\nsf}\\
\glt `song'

\ex shǫgwaęnáwi̱hshǫˀ\\
\gll shǫgwa-ręn-áwi̱h-shǫˀ\\
\textsc{3ms:1p}-song-give-{\pluralizer}\\
\glt `he has given us songs'
\z
\z

\ea\label{ex:rdelex2} \phonet{r} retained after a consonant
\ea wadrę́:no:t\\
\gll w-ad-rę́:n-o:t\\
\textsc{3s.a}-{\semireflexive}-song-stand.{\stative}\\
\glt ‘it is singing’ (or purring, in the case of a cat)

\ex ęhsadręnatǫdá:taˀ\\
\gll ę-hs-ad-ręn-atǫdá:t-h-a-ˀ\\
\fut-\textsc{2s.a}-{\semireflexive}-song-listen-{\dislocative}-{\joinerA}-{\punctual}\\
\glt ‘you’ll go listen to the songs’ (said as an invitation)
\z
\z

Similarly, for noun or verb stems beginning with \phonet{hr}, the \phonet{hr} is missing after a vowel \xref{ex:rdelex3a}, but present after a consonant \xref{ex:rdelex3b}. 

\ea\label{ex:rdelex3}
\ea shehó:wih\\\label{ex:rdelex3a}
\gll she-hró:wi-h\\
\textsc{2s:3fis}-tell-\textsc{euph.h}\\
\glt `tell her'
\ex ęga:tró:wiˀ\\\label{ex:rdelex3b}
\gll ę-g-a:t-hró:wi-ˀ\\
 \fut-\textsc{1s.a}-{\semireflexive}-tell-{\punctual}\\
\glt `I will tell'
\z
\z

\section{\phonet{dy/gy} and \phonet{ty/ky}} \label{DY/GY and TY/KY}
\textit{Ganedagehonǫ́ˀne̱ha:ˀ} (Lower Cayuga) speakers pronounce certain prefixes differently than \textit{Dagęhyatgehonǫ́ˀne̱ha:ˀ} (Upper Cayuga) speakers do, before stems beginning with \phonet{y} or \phonet{hy}. Ganedagehonǫ́ˀne̱ha:ˀ speakers use \phonet{ag-} \textsc{\semireflexive} \xref{ex:gyexa} and \phonet{adag-} \textsc{\reflexive} (not shown) before \phonet{y}, and also \phonet{ak-} \xref{ex:gyexb} and \phonet{adak-} (not shown) before \phonet{hy}. Both Ganedagehonǫ́ˀne̱ha:ˀ and Dagęhyatgehonǫ́ˀne̱ha:ˀ speakers use \phonet{ad-} (or \phonet{adad-}) before other sounds \xref{ex:gyexc}, except that Dagęhyatgehonǫ́ˀne̱ha:ˀ speakers say \phonet{at-} or \phonet{adat-} before stems beginning with \phonet{hy} \xref{ex:gyexd}.

\ea\label{ex:gyex}
\ea sagyaˀdo̱hái\\\label{ex:gyexa}
\gll s-ag-yaˀd-ohái\\
 \textsc{2s.a}-{\semireflexive}-body-wash.{\noaspect} \\
\glt `take a bath'
\ex ęgákyadǫ:ˀ\\\label{ex:gyexb}
\gll ę-g-ák-hyadǫ-:ˀ\\
 \fut-\textsc{1s.a}-{\semireflexive}-write-{\punctual}\\
\glt `I will have a reading'
\ex sadahǫhdóhai\\\label{ex:gyexc}
\gll s-ad-ahǫhd-óhai\\
 \textsc{2s.a}-{\semireflexive}-ears-wash.{\noaspect}\\
\glt `wash your ears'
\ex ęgátyadǫ:ˀ\\\label{ex:gyexd}
\gll ę-g-át-hyadǫ-:ˀ\\
 \fut-\textsc{1s.a}-{\semireflexive}-write-{\punctual}\\
\glt `I will have a reading'
\z
\z

\begin{sloppypar}
Ganedagehonǫ́ˀne̱ha:ˀ speakers use \phonet{g-} \textsc{\cislocative} before \phonet{y} \xref{ex:gyex3a}, while Dagęhyatgehonǫ́ˀne̱ha:ˀ speakers use \phonet{d-} in the same instance. Both Ganedagehonǫ́ˀne̱ha:ˀ and Dagęhyatgehonǫ́ˀne̱ha:ˀ speakers use the \phonet{d-} \cislocative form in the remaining cases \xref{ex:gyex3b}.
\end{sloppypar}

\ea\label{ex:gyex3}
\ea gyodóˀkda̱ˀǫh\\\label{ex:gyex3a}
\gll g-yo-d-óˀkd-a̱-ˀǫh\\
 {\cislocative}-\textsc{3s.p}-{\semireflexive}-end-{\joinerA}-{\stative}\\
\glt `it is lacking'
\ex dawádo̱ˀkdęˀ ‘it lacked’\\\label{ex:gyex3b}
\gll d-a-w-ád-o̱ˀkd-ęˀ\\
 {\cislocative}-{\factual}-\textsc{3s.a}-{\semireflexive}end-{\punctual}\\
\glt `it was not enough'
\z
\z

\section{Euphonic sounds} \label{Euphonic sounds}
Euphonic sounds exist for ease of pronunciation and convey no meaning -- a fact which distinguishes them from, for example, single-sound prefixes like \phonet{d-} {\cislocative}, meaning ‘here’, or \phonet{g-} \textsc{1s.a}, meaning ‘I’. The various types of euphonic sound are describe next.

\subsection{Joiner A \phonet{a}} \label{Joiner A}
\textsc{Joiner A} (\phonet{a}) prevents (breaks up) too-large groups of consonants. (In a different context, \textsc{Joiner E} \phonet{e} is used instead. It is described in \sectref{Joiner E}.) 

Joiner A appears between an incorporated noun ending with a consonant and a verb beginning with a consonant \xref{ex:epenvexa}. No Joiner A appears if the incorporated noun already ends with a vowel \xref{ex:epenvex2} or if the following verb begins with a vowel \xref{ex:epenvex3}.

\ea\label{ex:epenvex} Joiner A
\ea ga̱ˀdrehdagwé:gǫh\\\label{ex:epenvexa}
\gll ga̱-ˀdrehd-a-gwé:gǫh\\
 \textsc{3s.a}-car-{\joinerA}-all.{\stative}\\
\glt `all the cars'
\ex oˀdréhdatgiˀ\\\label{ex:epenvexb}
\gll o-ˀdréhd-a-tgiˀ\\
\textsc{3s.p}-car-{\joinerA}-ugly.{\stative}\\
\glt `ugly car'
\z
\z

\ea\label{ex:epenvex2} no Joiner A, incorporated noun ends with a vowel
\ea onę́hętgiˀ\\\label{ex:epenvex2a}
\gll o-nę́hę-tgiˀ\\
\textsc{3s.p}-corn-dirty.{\stative}\\
\glt `dirty corn'
\z
\z

\ea\label{ex:epenvex3} no Joiner A, verb begins with a vowel
\ea ga̱ˀdréhde:s\\
\gll ga̱-ˀdréhd-e:s\\
 \textsc{3s.a}-car-long.{\stative}\\
\glt `it is a long car'
\ex ga̱ˀdrehdí:yo:\\
\gll ga̱-ˀdrehd-í:yo:\\
 \textsc{3s.a}-car-nice.{\stative}\\
\glt `it is a nice car'
\ex nigaˀdre̱hdú:ˀuh\\
\gll ni-ga-ˀdre̱hd-ú:ˀuh\\
 {\partitive}-\textsc{3s.a}-car-small.{\stative}\\
\glt `it is a small car'
\z
\z

Joiner A also appears between verb stems ending with a consonant and suffixes beginning with a consonant \xref{ex:epenvex6a}, but not otherwise \xref{ex:epenvex6b}.

\ea\label{ex:epenvex6} 
\ea akyę́da̱hkwaˀ\\\label{ex:epenvex6a}
\gll akyę́d-a-hkwaˀ\\
 \textsc{ø.prefix}.sit-{\joinerA}-{\instrumental}\\
\glt `chair'
\ex ehyádǫ̱hkwaˀ\\\label{ex:epenvex6b}
\gll e-hyádǫ̱-hkwaˀ\\
 \textsc{3s.fi.a}-write-{\instrumental}\\
\glt `writing implement (pen, pencil, etc.)'
\z
\z

Unexpectedly, Joiner A appears between an incorporated noun and a small set of vowel-initial, obligatorily-incorporating verbs, including \stem{{}+ę} ‘lying on the ground’ \xref{ex:epenvex4}. (The non-incorporating form of the same verb, \stem{yę}, begins with a consonant,  \ref{ex:epenvex4b}.)

\ea\label{ex:epenvex4} 
\ea ga̱ˀdréhdaęˀ\\\label{ex:epenvex4a}
\gll ga̱-ˀdréhd-a-ęˀ\\
 \textsc{3s.a}-car-{\joinerA}-lie.{\stative}\\
\glt `one parked car'
\ex gá:yęˀ\\\label{ex:epenvex4b}
\gll gá:-yęˀ\\
 \textsc{3s.a}-lie.{\stative}\\
\glt `it is lying on the ground'
\z
\z

The examples in \xref{ex:epenvex5} look similar to (\ref{ex:epenvex4}a, c). However, the noun-final long \phonet{a:} shown in \xref{ex:epenvex5} shortens to \phonet{a} before vowel-initial verbs (\ref{ex:epenvex5}b, d). This prevents a series of a long vowel followed by a short vowel, but results in a sequence of two short vowels in a row (which is otherwise avoided).
 
\ea\label{ex:epenvex5} 
\ea sga:t wę̱ˀníhsgaǫt \\
\gll sga:t w-ę̱ˀníhsga-ǫt\\
one  \textsc{3s.a}-wheel-attached \\
\glt ‘it has one wheel’ (\phonet{a} is shortened /a:/, not \exsc{Joiner A})
\cfex{ęˀníhsga:ˀ, węˀníhsga:ˀ\\
	\gll (w)-ęˀníhsga:-ˀ\\
	(\textsc{3s.a})-wheel-{\nounstemformer} \\
	\glt `a wheel' (illustrating long /a:/)}
\ex ogaot\\
\gll o-gá-ot\\
\textsc{3s.p}-price-stand.{\stative} \\
\glt `a debt' (\phonet{a} is shortened /a:/, not \exsc{Joiner A})
\cfex{ó:ga:ˀ\\
	\gll ó:-ga:-ˀ\\
	\textsc{3s.p}-price-{\nounstemformer} \\
	\glt `a price' (illustrating long /a:/)}
\z
\z

\subsubsection*{Joiner A pronounced as \phonet{ǫ} or \phonet{ę}} \label{Joiner A pronounced as Ǫ or Ę}
Joiner A is instead pronounced as \phonet{ǫ} after \phonet{ǫh} and \phonet{ę} after \phonet{ęh} \xref{ex:tlhex}, but not otherwise \xref{ex:tlhex2}.

\newpage
\ea\label{ex:tlhex} 
\ea sa̱hohdęgyǫ́hǫgyeˀ\\
\gll sa̱-ho-hdęgy-ǫ́-h-ǫ-gy-e-ˀ\\
 {\repetitive}-\exsc{\textsc{3s.m.p}}-leave-{\stative}-\textsc{euph.h}-{\joinerA}-{\progressive}-go-{\stative}\\
\glt `he is on his way home'
\ex agade̱ˀgwę́hęgyeˀ\\
\gll ag-ade̱-ˀgw-ę́-h-ę-gy-e-ˀ\\
 \textsc{1s.a}-{\semireflexive}-run.away-{\stative}-\textsc{euph.h}-{\joinerA}-{\progressive}-go-{\stative}\\
\glt `I am running away again'
\z
\z

\ea\label{ex:tlhex2}
\ea go̱hsganye̱há:gyeˀ\\
\gll go̱-hsganye̱-h-á:-gy-e-ˀ\\
 \textsc{3s.fi.p}-shuffle-\textsc{euph.h}-{\joinerA}-{\progressive}-go-{\stative}\\
\glt `she is shuffling along'
\ex agyǫdi̱há:gyeˀ\\
\gll ag-yǫdi̱-h-ágy-e-ˀ\\
 \textsc{1s.a}-smile-\textsc{euph.h}-{\joinerA}-{\progressive}-go-{\stative}\\
\glt `I am going along smiling'
\z
\z

\subsubsection*{Other instances of \phonet{a} being changed to \phonet{ǫ}} \label{Other instances of A being changed to Ǫ}
The second vowel of the \stem{-shǫ:ˀǫh} \textsc{\pluralizer} suffix can either be pronounced as \phonet{a} or \phonet{ǫ} \xref{ex:tlhex3}.

\ea\label{ex:tlhex3}
\ea ganyoˀsǫ́:ˀ\exemph{a}h, ganyoˀsǫ́:ˀ\exemph{ǫ}h \\
\gll ga-nyoˀ-shǫ́:ˀah/-shǫ́:ˀǫh\\
 \textsc{3s.a}-wild.game-{\pluralizer}\\
\glt ‘wild game’, `wild animals'
\ex gaęnasǫ́:ˀ\exemph{a}h, gaęnasǫ́:ˀ\exemph{ǫ}h \\
\gll ga-ęn-a-shǫ́:ˀah/-shǫ́:ˀǫh\\
 \textsc{3s.a}-song-{\pluralizer}\\
\glt ‘songs’, `songs for Our Sustenance'
\z
\z

\subsection{Joiner E \phonet{e}} \label{Joiner E}
As with Joiner A, \textsc{Joiner E} \phonet{e} is added in order to avoid too-long or unacceptable consonant clusters. While Joiner A and Joiner E serve a similar purpose, they appear in different places in the word. Joiner A occurs anywhere after the reflexive prefix. In contrast, Joiner E appears between pronominal, semireflexive, or reflexive prefixes ending with a consonant, and a consonant-initial verb or noun stem \xref{ex:epenvex7}. 

\ea\label{ex:epenvex7} Joiner E\\
gaǫdadé:gęhs\\
\gll gaǫ-dad-é:-gę-hs\\
 \exsc{3ns.fi.a}-{\reflexive}-{\joinerE}-see-{\habitual}\\
\glt `they see her'
\cfex{ gaǫdadahǫdǫ́haˀ\\
\gll gaǫ-dad-ahǫdǫ́-haˀ\\
\textsc{3ns.fi.a}-{\reflexive}-ask-{\habitual}\\
\glt ‘they are asking her’ (etc.) 
}
\z


The \phonet{(h)s-} \textsc{2s.a} prefix tends to be followed by Joiner E \xref{ex:epenvex8}, except before \phonet{n}, \phonet{y}, or a vowel \xref{ex:epenvex9}. Joiner E is unexpected in this context because clusters like \phonet{sh}, \phonet{sd,st}, \phonet{sg}, and \phonet{sw} are acceptable elsewhere in the word \xref{ex:epenvex10}.

\ea\label{ex:epenvex8} Joiner E after \phonet{(h)s-} \textsc{2s.a}
\ea dęsehehdáwęnye:ˀ\\
\gll d-ę-se-hehd-á-węnye-:ˀ\\
 {\dualic}-{\future}-\textsc{2s.a}-{\joinerE}-earth-{\joinerA}-stir-{\punctual}\\
\glt `you will tend or till your garden'
\ex sedá:kseˀ\\
\gll s-e-dá:kse-ˀ\\
 \textsc{2s.a}-{\joinerE}-run-{\stative}\\
\glt `you are running'
\ex í:se:k\\
\gll í:-s-e:-k\\
 {\prothetic}-\textsc{2s.a}-{\joinerE}-eat.{\zeropunctual}\\
\glt `eat it!'
\ex segǫ́hsa̱ˀgeh\\
\gll s-e-gǫ́hs-a̱ˀgeh\\
 \textsc{2s.a}-{\joinerE}-face-{\on}\\
\glt `on your face'
\ex ęsé:gęˀ\\
\gll ę-s-é:-gę-ˀ\\
 \fut-\textsc{2s.a}-{\joinerE}-see-{\punctual}\\
\glt `you will see'
\ex ęhsewayę:nę́:daˀ\\
\gll ę-hs-e-wayę:nę́:d-aˀ\\
 \fut-\textsc{2s.a}-{\joinerE}-complete-{\punctual}\\
\glt `you will complete'
\z
\z

\ea\label{ex:epenvex9} no Joiner E before \phonet{n}, \phonet{y}, or a vowel
\ea syę́:sehs\\
\gll s-yę́:se-hs\\
 \textsc{2s.a}-skin-{\habitual}\\
\glt `you skin animals all the time'
\ex snętsáˀgeh\\
\gll s-nęts-áˀgeh\\
 \textsc{2s.a}-arm-{\on}\\
\glt `on your arm'
\ex sǫtsáˀgeh\\
\gll s-ǫts-áˀgeh\\
 \textsc{2s.a}-knee-{\on}\\
\glt `on your knee'
\z
\z

\ea\label{ex:epenvex10} \phonet{sh}, \phonet{sd/st}, \phonet{sg}, and \phonet{sw} are otherwise allowed
\ea \exemph{sh}ehó:wih\\
\gll she-hró:wi-h\\
\textsc{2s:3fis}-tell-\textsc{euph.h}\\
\glt `tell her'
\ex \exemph{st}ǫ́:hah ‘a little bit’ 
\ex \exemph{sg}a̱ˀáhdra:t\\
\gll s-ga̱-ˀáhdr-a:-t\\
 {\repetitive}-\textsc{3s.a}-basket-{\joinerA}-stand.{\stative}\\
\glt `one basket'
\ex \exemph{sw}a̱hdegá:ˀgeh\\
\gll swa̱-hdegá:-ˀgeh\\
 {2\pla}-ribs-{\on}\\
\glt `on your ribs'
\z
\z

\subsection{Euphonic D, Y, and W} \label{Euphonic D, Y, and W}
Euphonic D \phonet{d}, Y \phonet{y}, and W \phonet{w} appear in several contexts for ease of pronunciation. Euphonic D appears between a few vowel-final verbs, and a following Joiner A \xxref{ex:euphhd2}{ex:euphhd3}, and even in some loanwords (see \ref{ex:euphhd4}, where the incorporated noun and loanword ‘monkey’ ends with a vowel sound).

\ea\label{ex:euphhd2}
\ea onǫnyáędagyeˀ\\
\gll o-nǫny-á-ę-d-a-gy-e-ˀ\\
\textsc{3s.p}-ice-{\joinerA}-lie.{\stative}-\textsc{euph.d}-{\joinerA}-{\progressive}-go-{\stative}\\
\glt `it is freezing'
\cfex{onǫ́nyaęˀ\\
	\gll o-nǫ́ny-a-ęˀ\\
	\textsc{3s.a}-ice-{\joinerA}-lie.{\stative}\\
	\glt `it is frozen'}
\ex awęnowaędá:gyeˀ\\
\gll aw-ęnowaę-d-á:-gy-e-ˀ\\
\textsc{3s.p}-sink-\textsc{euph.d}-{\joinerA}-{\progressive}-go-{\stative}\\
\glt `it is sinking'
\cfex{ęwęnǫwáęˀ\\
	\gll ę-w-ęnǫwáę-ˀ\\
	\fut-\textsc{3s.a}-sink-{\punctual}\\
	\glt `it will sink'}
\z
\z

\ea\label{ex:euphhd3}
\ea tgiˀdrǫ́:daˀk\\
\gll t-g-iˀdrǫ́:-d-a-ˀk\\
 {\cislocative}-\textsc{1s.a}-live.{\stative}-\textsc{euph.d}-{\joinerA}-\exsc{past}\\
\glt `where I lived'
\cfex{shęh tgiˀdrǫˀ\\
	\gll shęh t-g-iˀdrǫ-ˀ\\
	that {\cislocative}-\textsc{1s.a}-live-{\stative}\\
	\glt `where I live, reside’}
\ex agadowihshę́:daˀk\\
\gll ag-ad-owihshę́:-d-a-ˀk\\
 \textsc{1s.p}-{\semireflexive}-rest.{\stative}-\textsc{euph.d}-{\joinerA}-\exsc{past} \\
\glt `I did have a rest'
\cfex{agadowíshęˀ\\
	\gll ag-ad-owíshę-ˀ\\
	\textsc{1s.p}-{\semireflexive}-rest-{\stative}\\
	\glt `I have rested'}
\z
\z

\ea\label{ex:euphhd4}
Gwe: gwé: a:yę́ˀ desatmónkeydǫh e:? \\
\gll de-sa-t-mónkey-d-ǫh\\
{\dualic}-\textsc{2s.p}-{\semireflexive}-monkey-\textsc{euph.d}-\exsc{stative}\footnotemark{}\\
\glt ‘Are you acting like a monkey again?’
\footnotetext{Humorous example provided by Amos Key, Jr.}
\z

Euphonic Y appears between certain vowel-final pronominal prefixes and vowel-initial stems (\ref{ex:euphhd5}, see \sectref{Preserving, merging, and deleting vowels}.)

\ea\label{ex:euphhd5}
\ea gǫ́:yǫˀ\\
\gll gǫ́:-y-ǫ-ˀ\\
 \exsc{1s:2s}-\textsc{euph.y}-give-{\punctual}\\
\glt `I will give you'
\ex sheyahǫ́:dǫ:\\
\gll she-y-ahǫ́:dǫ:\\
 \exsc{2s:3fis}-\textsc{euph.y}-ask.{\noaspect}\\
\glt `ask her!'
\z
\z

In only a few words, Euphonic W appears after a pronominal prefix and before the verb stem \xref{ex:euphhd6}.

\ea\label{ex:euphhd6} 
\ea degaǫwę́hda:s\\
\gll de-gaǫ-w-ę́hda:-s\\
 {\dualic}-\exsc{3s.fi.a}-\textsc{euph.w}-run-{\habitual}\\
\glt `they run'
\ex da:gáǫwę̱hda:t\\
\gll d-a:-gáǫ-w-ę̱hda:t\\
 {\dualic}-{\indefinite}-\exsc{3s.fi.a}-\textsc{euph.w}-run.{\zeropunctual}\\
\glt `they might run'
\cfex{dewagę̱hdá:dǫh\\
	\gll de-wag-ę̱hdá:d-ǫh\\
	{\dualic}-\textsc{1s.p}-run-{\stative}\\
	\glt `I have run'}
\z
\z

\begin{CayugaRelated}
\item Preserving, merging, and deleting vowels \sectref{Preserving, merging, and deleting vowels}
\end{CayugaRelated}

\subsection{The prothetic vowel (euphonic \phonet{i})} \label{Euphonic I}
The \textsc{prothetic} vowel (euphonic \phonet{i}) appears at the beginning of some verbs. (It is lengthened and accented to \phonet{í:} under conditions described in \sectref{Non-final accent and lengthening}.) While it resembles a prefix, the prothetic vowel has no meaning. To illustrate, in \xref{ex:proth}, all three verbs end with \phonet{se:k}, meaning ‘you eat it’. The verb beginning with prothetic \phonet{í:} has no added meaning \xref{ex:protha}. In contrast, the verb beginning with the \stem{a:-} \textsc{\indefinite} additionally means ‘should’ \xref{ex:prothb} and the verb with the \stem{ę:-} \textsc{\future} additionally means ‘will’ \xref{ex:prothc}.

\ea\label{ex:proth}
\ea í:se:k\\\label{ex:protha}
\gll í:-s-e:-k\\
 {\prothetic}-\textsc{2s.a}-{\joinerE}-eat.{\zeropunctual}\\
\glt `you eat it!'
\ex á:se:k\\\label{ex:prothb}
\gll á:-s-e:-k\\
 {\indefinite}-\textsc{2s.a}-{\joinerE}-eat.{\zeropunctual}\\
\glt `you should eat it'
\ex ę́:se:k\\\label{ex:prothc}
\gll ę́:-s-e:-k\\
 \fut-\textsc{2s.a}-{\joinerE}-eat.{\zeropunctual}\\
\glt `you will eat it'
\z
\z

The purpose of the prothetic vowel is to ensure that verbs have at least two vowels\footnote{This is a common requirement cross-linguistically.}. The verbs with the prothetic vowel in (\ref{ex:proth2}a, c, e) have only one vowel before \phonet{i, í:} is added. In contrast, the verbs in (\ref{ex:proth2}b, d, f) do not need the prothetic vowel because the pronominal prefixes already contribute a second vowel.

\ea\label{ex:proth2}
\ea ikne:\\
\gll i-kn-e:\\
 {\prothetic}-\exsc{1d.in.a}-think.{\stative}\\
\glt `you and I want, think'
\ex akne:\\
\gll akn-e:\\
 \textsc{1d.ex.a}-think.{\stative}\\
\glt `we two want, think'
\ex idwe:\\
\gll i-dw-e:\\
 {\prothetic}-\exsc{1p.in.a}-think.{\stative}\\
\glt `we all want, think'
\ex agwe:\\
\gll agw-e:\\
 \textsc{1p.ex.a}-think.{\stative}\\
\glt `we all want, think'
\ex ihe:\\
\gll i-h-e:\\
 {\prothetic}-\exsc{3s.m.a}-think.{\stative}\\
\glt `he wants, thinks'
\ex hę́:ne:\\
\gll hę́:n-e:\\
 \textsc{3ns.m.a}-think.{\stative}\\
\glt `they want, think'
\z
\z


\subsubsection*{Three-vowel verbs beginning with the \textsc{prothetic} vowel} \label{Three-vowel verbs beginning with euphonic I}
According to the description in \sectref{Euphonic I}, verbs beginning with the prothetic vowel should never have more than two vowels, since the prothetic vowel exists to ensure that verbs become two vowels long. For the most part, this is the case \xref{ex:proth3}.

\ea\label{ex:proth3}
\ea í:geˀs\\
\gll í:-g-e-ˀs\\
 {\prothetic}-\exsc{1s.a}-go-{\habitual}\\
\glt `I am here'
\ex ihsgeˀs\\
\gll i-hs-g-e-ˀs\\
 {\prothetic}-\exsc{\repetitive-1s.a}-go-{\habitual}\\
\glt `I am back home'
\ex í:drah\\
\gll í:-d-rah\\
 {\prothetic}-\exsc{2s.a}-spread.{\noaspect}\\
\glt `spread it on'
\ex í:ga:ˀ\\
\gll í:-ga:ˀ\\
 {\prothetic}-\exsc{3s.a}-contain.{\stative}\\
\glt `it contains'
\ex í:ga:t\\
\gll í:-ga:-t\\
 {\prothetic}-\exsc{3s.a}-stand.{\stative}\\
\glt `it is standing'
\ex í:gye:t\\
\gll í:-g-ye:-t\\
 {\prothetic}-{\cislocative}-\textsc{3s.fi.a}-stand.{\stative}\\
\glt `she is standing over there'
\ex ihe:\\
\gll i-h-e:\\
 {\prothetic}-\exsc{3s.m.a}-want.{\stative}\\
\glt `he wants it'
\ex ihse:\\
\gll i-hs-e:\\
 {\prothetic}-\exsc{2s.a}-want.{\stative}\\
\glt `you want'
\ex ikda:s\\
\gll i-k-dahsd\\
 {\prothetic}-\exsc{1s.a}-string.up.{\habitual}\\
\glt `I am stringing or draping it'
\ex ita:t\\
\gll i-t-ha:-t\\
 {\prothetic}-{\cislocative}-\exsc{\textsc{3s.m.a}}-stand.{\stative}\\
\glt `he is standing over there'
\ex í:yoˀ\\
\gll í:-y-oˀ\\
 {\prothetic}-\exsc{3s.p}-submerged.{\stative}\\
\glt `it is in the water'
\ex í:yǫ:s\\
\gll í:-y-ǫ:s\\
 {\prothetic}-\exsc{3s.p}-long.{\stative}\\
\glt `it is long'
\z
\z

However, some exceptions are shown in \xref{ex:proth4}. The exceptions all end with the \stem{-:hah} \textsc{\diminutive} suffix.\footnote{Technically speaking, \stem{-:hah} is an enclitic -- a suffix, which behaves like an independent word in that it does not affect accent placement.}

\ea\label{ex:proth4}
\ea ihsé:hah\\
\gll i-hs-é-:hah\\
 {\prothetic}-\exsc{2s.a}-think.{\stative}-{\diminutive}\\
\glt `you gloat'
\ex ihé:hah\\
\gll i-h-é-:hah\\
 {\prothetic}-\exsc{3s.m.a}-think.{\stative}-{\diminutive}\\
\glt `he thinks highly of himself'
\ex igé:hah\\
\gll i-g-é-:hah\\
 {\prothetic}-\exsc{1s.a}-think.{\stative}-{\diminutive}\\
\glt `I am gloating, boastful'
\z
\z

\subsection{Euphonic H} \label{Word-final pronunciation H}
Euphonic H \phonet{h} is added to words ending with a short vowel, but only when such words are either stand-alone (not shown) or sentence-final \xref{ex:wdfinalhex}.\footnote{Technically speaking, the \stem{-ǫh} and \stem{-ęh} \textsc{\stative} endings, illustrated in \textit{hodǫ́go̱hdǫh} and \textit{hodá:węh} in \xref{ex:euphhd}, should be analysed as \stem{-ǫ-h} and \stem{-ę-h} {\stative}-\textsc{euph.h}. The final H is an instance of the rule for inserting word-final H, for which see \sectref{Accent and pronunciation H in sentences} . However, to simplify matters, the \stem{-ǫh} and \stem{-ęh} endings are simply glossed as {\stative} in this work.} Euphonic H is not added when the same words are mid-sentence \xref{ex:wdfinalhex2}. This was previously described in \sectref{Accent and pronunciation H in sentences}.

\ea\label{ex:wdfinalhex} 
 \gll Sǫ: hne:ˀ nˀáht tó:gyę\exemph{h}? \\
who in.fact person that.one\\
\glt ‘Who is that?’
\z

\ea\label{ex:wdfinalhex2} 
\gll Sǫ: hne:ˀ nˀáht to:gyę́ gaeksa̱ˀsǫ́:ˀǫh? \\
who in.fact person that.one children\\
\glt ‘Then who are those children?’ (\cite[88]{mithun_watewayestanih_1984}, Eksaˀgó:wah dialogue)
\z

Euphonic H also appears between a verb ending with a vowel and the \stem{-a-gy-} {\joinerA}-{\progressive} suffix combination \xref{ex:euphhd}. (For the structure of progressive verbs, see \sectref{ch:Progressive verbs}.)

\newpage
\ea\label{ex:euphhd}
\ea honǫni̱há:gyeˀs \\
\gll hon-ǫni̱-h-á:gy-e-ˀs\\
\textsc{3ns.m.p}-make-\textsc{euph.h}\exsc{-progressive}-go-{\habitual}\\
\glt ‘they are making, earning it’ (continually)


\cfex{tiwa:gǫ́:ni:\\
	\gll ti-wa:g-ǫ́:ni-:\\
	{\contrastive}-\textsc{1s.a}-make-{\stative}\\
	\glt `I have made it in place of something else, replaced it'}
\ex hodǫgo̱hdǫ́hǫgyeˀ\\
\gll ho-d-ǫgo̱hd-ǫ́-h-ǫ-gy-e-ˀ\\
 \textsc{3s.m.p}-{\semireflexive}-surpass-{\stative}-\textsc{euph.h}-{\joiner}-{\progressive}-go-{\stative}\\
\glt `he is going along passing'
\cfex{hodǫ́go̱hdǫh\\
	\gll ho-d-ǫ́go̱hd-ǫh\\
	\textsc{3s.m.p}-{\semireflexive}-surpass-{\stative}\\
	\glt `he has gone past'}
\ex da̱hodawę́hęgyeˀ\\
\gll da̱-ho-d-aw-ę́-h-ę-gy-e-ˀ\\
 {\cislocative}-\exsc{\textsc{3s.m.p}}-{\semireflexive}-swim-{\stative}-\textsc{euph.h}-{\joiner}-{\progressive}-go-{\stative}\\
\glt `he is swimming towards us'
\cfex{hodá:węh\\
	\gll ho-d-á:w-ęh\\
	\textsc{3s.m.p}-{\semireflexive}-swim-{\stative}\\
	\glt `he did swim'}
\z
\z


\begin{CayugaRelated}
\item Accent and Euphonic H in sentences \sectref{Accent and pronunciation H in sentences}
\item Progressive verbs \sectref{ch:Progressive verbs}
\end{CayugaRelated}
