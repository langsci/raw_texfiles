\chapter{“Nouns” (words functioning as nouns)}\label{“Nouns” (words functioning as nouns)}
Words functioning as “nouns” name people, places, things, ideas, and abstractions. Both \textsc{nouns} proper and words functioning as “nouns”, are described next.

\largerpage
\section{Basic nouns} \label{ch:Basic nouns}
The function of \textsc{basic nouns} is to name people, places, objects, and abstract concepts. Basic nouns consist of a prefix, a stem, and a suffix (defined in \sectref{Prefixes, suffixes, affixes, and stems}). Most basic nouns end with the \stem{-aˀ} \textsc{noun stem former} (\textsc{\nsf}) suffix. Sometimes this suffix is \stem{-ęˀ} or \stem{-a:ˀ} instead. (For examples, see \sectref{ch:basic nouns ending in [-aˀ], [-tr-aˀ] or [-hsr-aˀ]}). The suffix just identifies the word as a noun, while the stem carries the core meaning.

Most basic nouns begin with a prefix, either \stem{ga-} or \stem{o-} \xref{ex:baspre}, while some begin with \stem{ę-} or \stem{wę-} \xref{ex:baspre2}, and some with \stem{a-} \xref{ex:baspre6}.\footnote{It is possible to analyse \phonet{a} in this type of word as the noun prefix counterpart \xref{ex:ppselex2a} of the verb prefix /wa-/ \textsc{3s.a} ‘it’ \xref{ex:ppselex2b}. Alternatively, \phonet{a} may be analysed as part of the stem, in which case, nouns beginning with \phonet{a} have no prefix and are \textsc{atypical} (which is defined in \sectref{ch:Atypical nouns}.)

\ea\label{ex:ppselex2} 
\ea akyęda̱hkwaˀ\\\label{ex:ppselex2a}
\gll a/ø-kyęda̱hkw-aˀ\\
\textsc{3s.a}-/{\noprefix.}chair-{\nounstemformer}\\
\glt ‘chair’
\ex wakyęda̱hgówanęh \\\label{ex:ppselex2b}
\gll w-akyęda̱hg-owanęh \\
\textsc{3s.a}-chair-big.{\stative}\\
\glt ‘it is a big chair’
\z
\z
} 

\ea\label{ex:baspre}
\ea ga̱há:daˀ ‘forest, bush’
\ex onę́nyo̱ˀgwaˀ ‘pills’
\z
\z

\ea\label{ex:baspre2}
\ea ęhníˀda:ˀ ‘a month, a moon’
\ex ęˀníhsga:ˀ, węˀníhsga:ˀ ‘wheel, circle, hoop’
\z
\z



\ea\label{ex:baspre6}
\ea adáˀditraˀ ‘cane’
\ex adęnidéǫsraˀ ‘the act of kindness’
\z
\z



The type of pronominal prefix must be memorized for nouns (\sectref{ch:Nouns and pronominal prefix selection}): for example, \textit{ga̱há:daˀ} ‘forest, bush’ always begins with \stem{ga-}, and \textit{onę́nyo̱ˀgwaˀ} ‘pills’ can only begin with \stem{o-}. However, a few basic nouns can take either \stem{ga-} or \stem{o-} \xref{ex:baspre3}. Still others either begin with \stem{o-} or are missing a prefix \xref{ex:baspre4}. Finally, some begin either with \stem{o-} or with \stem{a-} (\ref{ex:baspre7}, see previous footnote about the analysis of \phonet{a}).

\ea\label{ex:baspre3} 
\ea ga̱hnyę́da̱hsaˀ, ohnyę́da̱hsaˀ ‘beak’
\ex ga̱hǫ́ˀji̱hsdaˀ, ohǫ́ˀji̱hsdaˀ ‘a motor, engine’
\ex ga̱hsdágwaˀ, ohsdágwaˀ ‘dirty clothes’
\ex gajíhoha:ˀ, ojíhoha:ˀ ‘straight pin, pin, brooch, safety pin’
\ex ga̱héhnaˀ, ohéhnaˀ ‘cargo, bundle, load’
\ex gáidrę̱hdaˀ, owídrę̱hdaˀ \trs{sleep}, \trs{a dream}
\z
\z

\ea\label{ex:baspre4}
\ea ohsgwáęˀdaˀ, sgwáęˀdaˀ ‘coltsfoot’
\ex ohǫ́na̱ˀdaˀ, hǫ́na̱ˀdaˀ ‘potato’
\ex ohtahyǫ́:ni:, tahyǫ́:ni: ‘wolf’
\z
\z

\ea\label{ex:baspre7}
\ea ogyáǫhsraˀ, agyáǫhsraˀ ‘a trick’
\ex odǫ́hne̱ˀtsaˀ, adǫ́hne̱ˀtsaˀ ‘ladder or stairs’
\z
\z
The basic noun template in \tabref{figtab:1:basicnoun} summarizes the order and types of prefix, stem, and suffix characteristic of basic nouns.

\begin{table}
\caption{Basic nouns (alienable)}
\label{figtab:1:basicnoun}
\begin{tabularx}{.75\textwidth}{XXX}
\lsptoprule
pronominal 

prefix & stem & \nounstemformer\\
\midrule
\stem{ga-} \textsc{3s.a}

\stem{o-} \textsc{3s.p}

\stem{(a-)} \textsc{3s.a} & \textsc{noun} & \stem{-aˀ} \\
\lspbottomrule
\end{tabularx}
\end{table}



\subsection{Basic nouns with \textsc{nominalizer} ({\nominalizer}) suffix} \label{ch:Basic nouns with nominalizer suffix}
Several basic nouns have an additional suffix (the \textsc{nominalizer}, \nominalizer), which appears after the stem and before the \textsc{\nsf} suffix \xref{ex:baspre10}. The nominalizer allows a verb stem like \stem{tgiˀ} ‘ugly’ to become a basic noun.

\ea\label{ex:baspre10}
gatgíˀtraˀ\\
\gll ga-tgíˀ-tr-aˀ\\
 \textsc{3s.a}-ugly.\stat-\nominalizer{-\nsf}\\
\glt `junk'
\z


The template for such nouns is shown in \tabref{figtab:1:bn.nominalizer}. For more examples, see \sectref{ch:basic nouns ending in [-aˀ], [-tr-aˀ] or [-hsr-aˀ]}.

\begin{table}
\caption{Basic noun with \textsc{\nominalizer}}
\label{figtab:1:bn.nominalizer}
\begin{tabularx}{\textwidth}{XXXX}
\lsptoprule
pronominal 

prefix & stem & {\nominalizer} & {\nsf}\\
\midrule
\stem{ga-} \textsc{3s.a}

\stem{o-} \textsc{3s.p}

\stem{(a-)} \textsc{3s.a} & \textsc{verb} & \stem{-tr} 

\stem{-hsr} & \stem{-aˀ} \\
\lspbottomrule
\end{tabularx}
\end{table}



\section{Body part nouns} \label{ch:Body part nouns}
\textsc{Body part nouns} function to name body parts.\footnote{Also known as \textsc{inalienable} nouns, they describe parts that are not normally separable from the body.} They begin with a pronominal prefix (in bold) describing the possessor of the body part, and end with \stem{-aˀgeh} {\on}.

\ea\label{ex:bodyn}
\ea \textbf{k}yohsáˀgeh, \textbf{k}yuhsáˀgeh ‘on my elbow’
\ex \textbf{s}ǫtsáˀgeh ‘on your (s.) knee’
\ex \textbf{e}kséˀda̱ˀgeh ‘on her belly’
\ex \textbf{e}nę́tsa̱ˀgeh ‘on her arm’
\ex \textbf{ga̱}hnyęda̱hsáˀgeh ‘on its beak’
\ex \textbf{ga̱}hsíˀda̱ˀgeh ‘on my foot’
\ex \textbf{ga̱}hyagwiyáˀgeh ‘on my toes’
\z
\z


While such nouns end with \stem{-aˀgeh} \on, the suffix meaning is often not prominent: in \xref{ex:bodyn2}, \textit{ge̱hsína̱ˀgeh} is understood to mean ‘my leg’, rather than ‘on my leg’.

\ea\label{ex:bodyn2} 
\gll Awatǫgaék ge̱hsína̱ˀgeh.\\
it.stiffened.up on.my.leg\\
\glt ‘My leg stiffened up.’ 
\z

\tabref{figtab:1:inalienable.noun} shows the order and types of prefix, stem, and suffix characteristic of body part nouns. For more examples, see \sectref{ch:list of body part nouns}. 

\begin{table}
\caption{Body part (inalienable) nouns}
\label{figtab:1:inalienable.noun}
\begin{tabularx}{.75\textwidth}{XXX}
\lsptoprule
pronominal 

prefix & stem & external locative

\on\\
\midrule
\textsc{a} series\footnote{For \textsc{a} series, see \sectref{ch:Nouns and pronominal prefix selection}.} & \textsc{noun} & \stem{-aˀgeh} \\
\lspbottomrule
\end{tabularx}
\end{table}


\subsection{‘Detached’ or unpossessed body part nouns} \label{ch:’Detached’ or unpossessed body part nouns}
Body part noun stems (described in the previous section) can also be inflected (\sectref{Paradigm, inflection, and conjugation}) with the same prefixes and suffixes as basic nouns \xref{ex:bodybas}. Nouns inflected in this way tend to have the special meaning of being detached (or \textsc{alienable}, \ref{ex:bodybas2}). They also appear in a type of \textsc{compound noun} (\ref{ex:bodybas3}, \sectref{ch:Compound nouns}). For more examples, see \sectref{ch:(detached or unpossessed)}.

\ea\label{ex:bodybas} body part noun inflected as a basic noun
\ea á:gwa̱hdaˀ, ogwáhdaˀ ‘sole, ball of foot’
\ex awę́yǫ̱hsaˀ ‘heart’
\ex awę́ˀna̱hsaˀ ‘tongue’
\ex awę́ˀyo̱hga:ˀ ‘thumb’
\ex ogáhaˀ ‘eye’
\ex ogǫ́daˀ ‘bridge of one’s nose’
\ex ogǫ́hsaˀ ‘face’
\z
\z

\ea\label{ex:bodybas2} detached meaning
\ea onę́:tsaˀ ‘arm’ (said, for example, when holding doll’s arm for show) 
\ex onǫ́ˀa:ˀ ‘a severed head’ 
\z
\z

\ea\label{ex:bodybas3} compound noun\\
gwihsgwíhs onǫ́ˀa:ˀ ‘pig’s head’
\z


\section{Compound nouns} \label{ch:Compound nouns}
\textsc{Compound nouns} consist of two words (or more) that function together to name a single object, idea, abstraction, person, etc. The last word of the compound describes a general category (as in \textit{oˀwáhǫh} ‘meat’), while the non-final words denote a sub-category (a ‘type of’ something, as in \textit{gwihsgwíhs oˀwáhǫh} ‘pig meat’, \xref{ex:compn} - \xref{ex:compn5} and \tabref{figtab:1:compoundnoun}. 

\ea\label{ex:compn}
\gll gwihsgwíhs oˀwáhǫh\\
pig meat\\
\glt ‘pig meat, pork chop, bacon’
\z

\ea\label{ex:compn2}
\gll dewahǫhdé:s oˀwáhǫh \\
deer meat\\
\glt ‘deer meat, venison’
\z

\ea\label{ex:compn4}
\gll dakshaeˀdóhs oˀwáhǫh\\
chicken meat\\
\glt ‘chicken’ 
\z

\ea\label{ex:compn5}
\gll ohsohgwi:yóˀs watgǫhsgwa̱ˀtróˀ onǫhsagahę́:dǫˀ \\
nice.colours window.pane windows\\
\glt ‘stained glass windows’ 
\z

\begin{table}
\caption{compound nouns}
\label{figtab:1:compoundnoun}
\begin{tabularx}{.66\textwidth}{XX}
\lsptoprule
non-final word(s) & final word\\
\midrule
an instance of a 

category & a general category \\
\lspbottomrule
\end{tabularx}
\end{table}


\section{Verbs that function as nouns} \label{ch:Verbs that function as nouns}
\textsc{verbs} often function as “nouns”, naming a person, place, thing, or abstraction. Two types are described next.

\subsection{“Instrumental nouns”} \label{ch:Instrumental nouns}
“Instrumental nouns” are \textsc{verbs} functioning as “nouns”. They end with the \stem{-(h)kw-haˀ} \textsc{\instrumental}-{\habitual} suffix combination, and typically denote an object used as an instrument. (Related verbs are shown in the comparative “cf.” examples in \ref{ex:instrn}.)

\ea\label{ex:instrn}
\ea ehyádǫ̱hkwaˀ ‘pencil’ (literally, ‘instrument that people write with’)
\cfex{ehyá:dǫh ‘she or someone writes’}
\ex ǫtgahiˀdáhkwaˀ ‘toy’ (literally, ‘instrument that people play with’)
\cfex{ǫtgáhiˀtaˀ ‘what she or someone plays with’}
\z
\z

Instrumental nouns most commonly have the structure shown in \tabref{figtab:1:instrumentalnoun}. For a list of examples, see \sectref{ch:instrumental nouns ending in [-(h)kwaˀ]}.

\begin{table}
\caption{Instrumental nouns}
\label{figtab:1:instrumentalnoun}
\begin{tabularx}{.75\textwidth}{XXX}
\lsptoprule
pronominal 

prefix & stem type & \instrumental-\habitual\\
\midrule
\stem{e-}, \stem{ǫ-}, etc.

\textsc{3s.fi.a}
& \textsc{verb} & \stem{-hkw-haˀ} 

(\phonet{-hkwaˀ})\\
\lspbottomrule
\end{tabularx}
\end{table}



\subsection{Words for human beings} \label{ch:Words for human beings} 
Several \textsc{verbs} function as “nouns” in naming human beings, or the stages of being human. As verbs, such words often have a sentence-like meaning (as shown by the translations in \ref{ex:humn2}). A list of example words is provided in \sectref{ch:Words for human beings}.

\ea\label{ex:humn2}
\ea nihú:ˀuh ‘little boy’ (noun-like meaning), ‘he is small’ (verbal or sentence-like meaning)
\ex hǫgwéˀdase:ˀ ‘young man’ (noun-like meaning), ‘he is a new person’ (verbal or sentence-like meaning)
\ex nitawenǫ́:hah ‘middle-aged male’ (noun-like meaning), ‘he has come from somewhere’ (verbal or sentence-like meaning)
\z
\z


\subsection{“Meaningful” verbal nouns} \label{ch:Meaningful verbal nouns}
“Meaningful” verbal nouns are \textsc{verbs} that function as “nouns”, naming objects, places, and abstract concepts \xref{ex:meann}. As verbs, they have sentence-like meanings.\footnote{This observation may explain why speakers say that Gayogo̱ho:nǫˀnéha:ˀ words mean more than English words do: “You don’t need a thesaurus in Gayogo̱ho:nǫnéha:ˀ.” (unattributed quotation).} Many of the words listed in the “Related” sections are \textsc{verbs} functioning as “nouns”. 

\ea\label{ex:meann}
\ea degahęnáˀtra̱ˀse:ˀ ‘scissors’ (literally, ‘it has two blades’)
\ex ga̱hnyaˀsesgó:wah ‘giraffe’ (literally, ‘it has a great big long neck’)
\ex hadihnyǫˀǫ́hsratęhs ‘ironworkers’ (literally, ‘they climb iron (for a living)’)
\ex degaihó:węhs ‘computer’ (literally, ‘it splits words’)
\z
\z

\begin{CayugaRelated}
\item Clans, \sectref{clans}

\item Nations, \sectref{nations}

\item Chiefs, \sectref{chiefs}
\end{CayugaRelated}



\section{Stative nouns and agentive stative nouns} \label{ch:Stative nouns and agentive stative nouns}
\textsc{Stative nouns} are related to verbs in the stative aspect (see \sectref{Meaning of stative nouns and agentive stative nouns}). They often denote the result of an action (see example \ref{ex:statnom1}). 

There are two types of stative nouns. The first type, \textsc{stative nouns} proper \xref{ex:statnom2}, either begin with \phonet{a} \xref{ex:statnom2a}\footnote{As mentioned in an earlier footnote, nouns beginning with \phonet{a} can be analyzed as beginning with \stem{ø-} (no pronominal prefix), or as beginning with \stem{a-} \textsc{3s.a} (a prefix unique to nouns).} or with the \stem{ga-} \textsc{3s.a} prefix \xref{ex:statnom2b}. They end with the \stem{-ǫ-:-ˀ} \textsc{\stative-\nominalizer-\nsf} combination. 

\ea\label{ex:statnom2} stative nouns
\ea\label{ex:statnom2a} adówadǫ:ˀ ‘the hunt’ 
\ex\label{ex:statnom2b} gayaˀdowéhdǫ:ˀ ‘the idea of thinking’ 
\z
\z

The template for stative nouns is shown in \tabref{figtab:1:stativenoun}. A list is provided in \sectref{ch:list of stative nouns}.

\begin{table}[H]
\caption{Stative nouns}
\label{figtab:1:stativenoun}
\begin{tabular}{lll}
\lsptoprule
pronominal prefix & stem & stative nominal combo \\
\midrule
\stem{ga-} \textsc{3s.a}

\stem{wa-} 

\stem{(a-)}& \textsc{verb} & \stem{-ǫ-:-ˀ} 

\textsc{\stative-\nominalizer-\nsf} \\
\lspbottomrule
\end{tabular}
\end{table}

The second type, \textsc{agentive stative nouns}, are \textsc{verbs} in the stative aspect, functioning as “nouns”. They take a \stem{ga-} or \stem{wa-} \textsc{3s.a} pronominal prefix, and end with a \textsc{stative} suffix such as \stem{-ǫh} or \stem{-ǫ:} \xref{ex:statnom1}.\footnote{Agentive stative verbs functioning as “nouns” take \textsc{3s.a} ‘it’ prefixes. In contrast, regular stative aspect verbs typically take \textsc{3s.p} ‘it’ prefixes, as in the following example, which is related to \xref{ex:statnom1b}.

\ea\label{ex:statnom100} heyó:gyǫ:\\
\gll  he-y-ó:gy-ǫ: \\
{\cislocative}-\exsc{3s.p}-throw-{\stative}\\
\glt ‘it has thrown it’
\z}

\ea\label{ex:statnom1} agentive stative “nouns”
\ea\label{ex:statnom1a} gagánya̱ˀgǫh ‘payment’
\ex\label{ex:statnom1b} wagyǫ: ‘something thrown away’, ‘discards’ 
\z
\z

The template for \textsc{agentive stative nouns} is shown in \tabref{figtab:1:agentive.stative.noun}. A list is provided in \sectref{ch:list of agentive stative nouns}.

\begin{table}[H]
\caption{Agentive stative nouns}
\label{figtab:1:agentive.stative.noun}
\begin{tabular}{lll}
\lsptoprule
pronominal prefix & stem & stative aspect \\
\midrule
\stem{ga-} \textsc{3s.a}

\stem{wa-} \textsc{3s.a} & verb & \stem{-ǫh}, \stem{-ǫ:}, (etc.) \\
\lspbottomrule
\end{tabular}
\end{table}


\section{Incorporated noun stems} \label{ch:Incorporated noun stems}
Incorporated noun stems are basic or body-part noun \textsc{stems} (for stems, see \sectref{Prefixes, suffixes, affixes, and stems}). They appear within a verb -- a state of affairs known as \textsc{noun incorporation} (\sectref{ch:Noun Incorporation}).

The following examples illustrate two independent nouns and their related noun stems \xref{ex:incnnex1}, \xref{ex:incnnex2a}, two incorporating verbs \xref{ex:incnnex1}, \xref{ex:incnnex2b}, and the resulting words in which the noun is incorporated into the verb \xref{ex:incnnex1}, \xref{ex:incnnex2c}.\footnote{English  words with a  structure similar to \xref{ex:incnnex1}, \xref{ex:incnnex2c} include ‘brain-wash’ and ‘white-wash’.} 

\ea\label{ex:incnnex1} incorporated basic nouns
\ea o-wí:y-aˀ ‘offspring’ (stand-alone noun) → \stem{wiy} (noun stem)
\ex ahá-nę̱hsgoˀ ‘he will steal’ (incorporating verb)
\ex aha-wiy-á-nę̱hsgoˀ ‘he kidnapped a child’ (incorporated noun + verb)
\z
\z

\ea\label{ex:incnnex2}incorporated body part nouns
\ea ha-hsíˀd-a̱ˀgeh ‘on his foot’ (stand-alone noun) → \stem{hsiˀd} (noun stem)\label{ex:incnnex2a}
\ex o-tgiˀ ‘it is dirty’, ‘it is ugly’ (incorporating verb)\label{ex:incnnex2b}
\ex ho-hsíˀd-a-tgiˀ ‘his feet are dirty’ (incorporated noun + verb)\label{ex:incnnex2c}
\z
\z

Some incorporated nouns also take an extra suffix when they incorporate -- a \textsc{noun increment} (\textsc{incr}) \xref{ex:nounincrement}.


\ea\label{ex:nounincrement}
 hǫgweˀdí:yo:\\
\gll h-ǫgwe-ˀd-í:yo:\\
\textsc{3s.m.a}-person-\textsc{incr}-nice.{\stative}\\
\glt `he is a nice person'
\cfex{hǫ́:gweh (atypical word) \\
    \gll h-ǫ́:gwe-h\\
   \textsc{3s.m.a}-person-\textsc{euph.h}\\
     \glt ‘man’
     }
\z

A list of frequently-incorporated nouns can be found in \sectref{ch:frequently-incorporated nouns}.

\section{Atypical nouns} \label{ch:Atypical nouns}
\textsc{Atypical nouns} lack a pronominal prefix, which is required for all other nouns, or are otherwise odd. They function as “nouns” (naming people, objects, and abstract concepts).

Some atypical nouns consist of two repeated syllables (\textsc{reduplication}) and in many cases, they imitate the sound of the animal they name (\textsc{onomatopoeia} \ref{ex:atypn1}). Such nouns have no prefixes or suffixes.

\newpage
\ea\label{ex:atypn1}
\ea dihsdihs ‘house woodpecker’
\ex diˀdi:ˀ ‘blue jay’
\ex duwísduwi:ˀ ‘killdeer’
\ex gáˀga:ˀ ‘crow’, ‘raven’
\z
\z

Like the examples in \xref{ex:atypn1}, the atypical nouns in \xref{ex:atypn2} also have no prefixes (or suffixes).

\ea\label{ex:atypn2}
\ea dago:s, dagu:s ‘cat’
\ex só:wa:s ‘dog’
\z
\z

Some atypical nouns end with the \stem{-aˀ} {\nsf} suffix \xref{ex:atypn3}, but are missing a pronominal prefix (\noprefix). (Similar examples were described in the section on basic nouns, \sectref{ch:Basic nouns}.)

\ea\label{ex:atypn3}
\ea nawę́ˀdaˀ\\
\gll ø-nawęˀd-aˀ\\
 {\noprefix}-sugar-{\nounstemformer}\\
\glt `sugar'
\ex tsaˀgę́:daˀ\\
\gll ø-tsaˀgę́:d-aˀ\\
 {\noprefix}-corn.tassel.{\nounstemformer}\\
\glt `corn tassel'
\ex yahgęhdaˀ\\
\gll ø-yahgęhd-aˀ\\
 {\noprefix}-morel-{\nounstemformer}\\
\glt `morel, black type of mushroom'
\z
\z

Some atypical nouns begin with GW and mostly refer to birds or other animals that have some ability to fly or jump high. Like other atypical nouns, these nouns have no pronominal prefix \xref{ex:atypn4}.

\ea\label{ex:atypn4}
\ea gwáoh ‘screech owl’
\ex gwáˀda: ‘flying squirrel’
\ex gwaˀyǫˀ ‘rabbit’
\ex gwę̱ˀdihs ‘night hawk’
\z
\z

Some atypical nouns begin with JI or J, and mainly denote animals (including some birds) \xref{ex:atypn5}. Many such nouns have a verb-like internal structure, but like other atypical nouns, do not have a pronominal prefix.

\ea\label{ex:atypn5}
\ea jidę́:ˀęh\\
\gll ji-dę́:-ˀęh\\
\exsc{ji-}stem-{\diminutive}\\
\glt ‘bird’ 
\ex jíhnyo̱ˀgęˀ\\
\gll ji-hny-oˀgęˀ\\
 \exsc{ji}-stick-together.{\stative}\\
\glt `chipmunk'
\ex jihsda: ‘grasshopper’ (possibly related to \stem{ga-jihsd-aˀ} ‘lamp, light’)
\z
\z

Some atypical nouns begin with SGWA, which possibly has something to do with frogs. Again, these atypical nouns have no pronominal prefix \xref{ex:atypn6}.

\ea\label{ex:atypn6}
\ea sgwagwáǫdǫˀ ‘toad’
\cfex{(possibly, \\
sgwagwá-ǫd-ǫˀ\\
frog-rooted.\textsc{\distributive.\stative})\\}
\ex sgwá:yęh ‘otter’ 
\cfex{(possibly,\\
sgwa-yęh\\
frog-verb.{\stative})\\}
\ex sgwáˀahdaˀ ‘frog’ 
\cfex{(possibly,\\
sgwáˀahd-aˀ\\
frog-{\nounstemformer})\\}
\cfex{(possibly, \\
sgwáˀ-ahdaˀ\\
frog-be.full.of.food.{\stative})\\}
\z
\z

Finally, the atypical nouns in \xref{ex:atypn7} resemble verbs but otherwise -- like other atypical nouns -- lack a pronominal prefix.

\ea\label{ex:atypn7}
\ea dogriyaˀgǫˀ ‘buffalo’ 
\cfex{(possibly includes\\
\stem{+iyaˀg-ǫh} \\
cut.across-{\stative})\\}
\ex drę́:na: ‘skunk’ 
\cfex{(possibly, \\
ø-drę́:n-a: \\
{\noprefix}-smell-hold.{\stative})\\}
\newpage
\ex hehshai: ‘fox’ 
\cfex{(possibly,\\
ø-hehsa-i:ˀ\\
{\noprefix}-decayed.tree-coloured.{\stative})\\}
\ex tgwiyó:gęˀ ‘channel catfish’ 
\cfex{(possibly, \\
ø-t-gwiy-ó:gę: \\
{\noprefix}-{\semireflexive}-twig-together.{\stative})\\}
\ex tsahgó:wa:ˀ ‘pigeon’
\cfex{(possibly includes\\
\stem{-go:wah})\\
big)}
\z
\z

The template for atypical nouns is shown in \tabref{figtab:1:atypical}. For a list of atypical nouns, see \sectref{ch:atypical nouns (list)}.

\begin{table}[H]
\caption{Atypical nouns}
\label{figtab:1:atypical}
\begin{tabularx}{.66\textwidth}{XX}
\lsptoprule
pronominal prefix & stem\\
\midrule
ø (\noprefix) & \textsc{noun} (+suffixes)

\textsc{verb} (+suffixes)\\
\lspbottomrule
\end{tabularx}
\end{table}


\section{Kinship terms (atypical words)} \label{ch:Kinship terms (atypical words)}
\textsc{Kinship terms} function to name family members, and (more metaphorically) sides or moieties in the League, the longhouse, or clans \xref{ex:kinn}. 

\ea\label{ex:kinn}
\ea heˀgę́:ˀęh ‘my younger brother’
\ex gakéno̱haˀahsǫ́:ˀǫh ‘my aunts’ 
\ex hehjíˀah ‘my older brother’
\ex ǫgyá:gyoh ‘my brother-in-law, sister-in-law’ 
\ex haˀnih ‘my father’ 
\ex ǫgyá:tsih ‘my friend, my boyfriend, my girlfriend’ 
\ex haˀní:hah ‘my godfather’ 
\ex hehso:t ‘my grandfather’
\ex agya:dę́:nǫhk ‘my relative’
\z
\z

Many kinship terms are atypical, in the sense that they have both verb-like and noun-like properties (\cite{koenig_argument_2010}). For example, the \textsc{interactive} type of pronominal prefix (\sectref{Three types of pronominal prefix}) present in many kinship terms is otherwise used exclusively with verbs.

In verbs, interactive pronominal prefixes refer to two people or groups of people (such as the speaker and a male, for the verb in \ref{ex:kinn2a}). However for kinship terms, the same kind of pronominal prefix refers to just one person or group (such as a male in \ref{ex:kinn2b}. Also see \cite[170--171]{koenig_argument_2010}). (For kinship term pronominal prefix choice, see \sectref{ch:Kinship terms and pronominal prefix choice}.)

\ea\label{ex:kinn2}
\ea\label{ex:kinn2a}\textbf{he}nǫ́hkwaˀ ‘I love him’, with \stem{he-} \textsc{1s>3s.m} referring to the speaker and a male (verb)
\ex\label{ex:kinn2b}\textbf{he}ˀgę́:ˀęh ‘my younger brother’, with \stem{he-} \textsc{1s>3s.m} referring to a male (kinship term)
\z
\z

While kinship terms take verb-like pronominal prefixes, they are not otherwise verbal. For example, they cannot be negated in the same way as verbs (\cite[180]{koenig_argument_2010}). Verbs are negated with \stem{tęˀ de-}, while kinship terms are negated with \stem{tęˀ … deˀgę:} \xref{ex:kinn3}.\footnote{Kehte Deer, p.c.}

\ea\label{ex:kinn3}
\gll \exemph{Tęˀ} \exemph{deˀ}óyanreˀ.\\
not it.is.not.good\\
\glt ‘it is not good.’ (negated verb) 
\z

\ea\label{ex:kinn34}
\gll \exemph{Tęˀ} kehawahk \exemph{deˀgę:}. \\
not my.daughter it.is.not\\
\glt ‘she is not my daughter’ (negated kinship term)
\z

The template for kinship terms is shown in \tabref{figtab:1:kinshipterm}. For a list of kinship terms, see \sectref{kinship terms (list)}.

\begin{table}[H]
\caption{Kinship terms}
\label{figtab:1:kinshipterm}
\begin{tabularx}{.5\textwidth}{XX}
\lsptoprule
interactive 

pronominal 

prefix & kinship stem\\
\midrule
\stem{he-}, etc. & \stem{ˀgę:ˀęh}, etc.\\
\lspbottomrule
\end{tabularx}
\end{table}


\subsection{Terms of address} \label{ch:Terms of address}
\textsc{Terms of address} are similar to kinship terms. However, their function is to address a person by their kinship status instead of using their name \xref{ex:kinn300}. For a list of such words, see \sectref{terms of address (list)}.

\ea\label{ex:kinn300} Sgę:nǫ́:, gwadreˀ! ‘Hello, granddaughter!’ 
\z


\section{Loanwords (borrowed words)} \label{ch:Loanwords (borrowed words)}
\textsc{Loanwords}, adapted from other languages, function as “verbs” \xref{ex:loann2} and as “nouns” \xxref{ex:loann}{ex:loann4}. In Gayogo̱ho:nǫˀnéha:ˀ, loanwords are often used jokingly (\ref{ex:loann2}c, d).

\ea\label{ex:loann2}
\ea dehsáde\textbf{smack} ‘smack your lips!’
\ex ató:\textbf{strok} ‘he had a stroke’
\ex e\textbf{boss}gó:wah ‘she is the big boss’
\ex ęgade\textbf{hórse}nęht ‘I will get off a horse’ (humorous expression)
\ex deyagode\textbf{highlight}dǫh ‘where she has it highlighted’
\z
\z


Some loanwords sound similar to the original word in the source language \xxref{ex:loann2}{ex:loann}. 

\ea\label{ex:loann} ji\textbf{gup}sǫ́:ˀǫh ‘cups’ (said to be a “slang” word for ‘tea cups’)
\z

Other loanwords have been adapted to sound better in Gayogo̱ho:nǫˀnéha:ˀ. For example, P in the source languages in \xref{ex:loann3} became GW or G in Gayogo̱ho:nǫˀnéha:ˀ \xref{ex:loann3}.

\ea\label{ex:loann3}
\ea \exemph{gw}ę́:niˀ ‘penny’ (from English ‘penny’)
\ex da\exemph{g}u:s ‘cat’ (from Dutch \textit{Der Poes} ‘the cat’)
\z
\z

Some loanwords look even more like regular Gayogo̱ho:nǫˀnéha:ˀ words because they have acquired Gayogo̱ho:nǫˀnéha:ˀ prefixes or suffixes. The nouns in (\ref{ex:loann4}a, b) begin with \stem{o-} or \stem{ga-}, like basic nouns, and the ones in (\ref{ex:loann4}b--e) are incorporated into verbs and either have \textsc{noun increments} (\textsc{incr}) \xref{ex:loann4b} or \textsc{nominalizer} (\nominalizer) suffixes (\ref{ex:loann4}c--e).

\ea\label{ex:loann4}
\ea o\textbf{di:}\\\label{ex:loann4a}
\gll o-di:\\
\textsc{3s.p}-tea\\
\glt `tea' \textsc{basic noun}
\ex o\textbf{gwę́ni̱}ˀdaˀ\\\label{ex:loann4b}
\gll o-gwęni-ˀd-aˀ\\
\textsc{3s.p}-penny-\exsc{incr-\nsf}\\
\glt `penny' \textsc{basic noun}
\ex ga\textbf{job}trowá:nęh\\\label{ex:loann4c}
\gll ga-job-tr-owá:n-ęh\\
 \textsc{3s.a}-job-{\nominalizer}-big-{\stative}\\
\glt `big job' \textsc{verb}
\ex ga\textbf{job}trí:yo:\\\label{ex:loann4d}
\gll ga-job-tr-í:yo:\\
 \textsc{3s.a}-job-{\nominalizer}-good.{\stative}\\
\glt `nice job' \textsc{verb}
\ex ho\textbf{job}trí:yo:\\\label{ex:loann4e}
\gll ho-job-tr-í:yo:\\
 \textsc{3s.m.p}-job-{\nominalizer}-good.{\stative}\\
\glt `he has a good job' \textsc{verb}
\z
\z


