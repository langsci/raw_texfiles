\chapter{Kinship terms and pronominal prefix choice} \label{ch:Kinship terms and pronominal prefix choice}
\textsc{Kinship terms} describe blood relations, marriage relations, relationships between clans, sides (or moieties, including longhouse moieties), relationships between friends and neighbours, and relationships among \textit{Ǫgwehǫ́:weh} peoples.\footnote{The kinship terms listed in this section were compiled from \citet{deer_cayuga_2015}, \citet{foster_course_1993}, Foster (p.c.), \citet{froman_english-cayugacayuga-english_2002}, \citet{mithun_watewayestanih_1984}, and \citet{sasse_far_1998}.} For a list of kinship terms, see \sectref{kinship terms (list)}.


Based on pronominal prefix selection, kinship terms can be divided into two types. The first type takes \textsc{non-interactive} (\textsc{a} or \textsc{p}) prefixes. They include \textsc{verbs} and \textsc{nouns} functioning as “kinship terms”. The second type take \textsc{interactive} prefixes. They are “kinship terms proper” or \textsc{interactive kinship terms}.\footnote{Kinship terms are \textsc{atypical} words, for reasons described in \sectref{ch:Kinship terms (atypical words)}.} 


In several cases, the same kinship term can be inflected two ways (either as non-interactive or interactive): example \xref{ex:twotypesa} shows a \textsc{verb} functioning as a “kinship term”, inflected with a non-interactive, \textsc{p}-series pronominal prefix. In contrast, example \xref{ex:twotypesb} shows the same stem, inflected with an interactive prefix. More examples are provided in later sections.
 
\ea\label{ex:twotypes} 
\ea\label{ex:twotypesa} ǫgyáˀse:ˀ (\stem{\textsc{p}-aˀse:ˀ} ‘cousin’, with \textsc{p}-series prefix)\\
\gll ǫgy-áˀse:ˀ\\
 \exsc{1d.p}-doubled.{\stative} \\
\glt `my cousin'
\ex gya̱ˀse:ˀ (\stem{\textsc{interactive}-aˀse:ˀ} ‘cousin’, with interactive prefix)\\\label{ex:twotypesb} 
\gll gy-aˀse:ˀ\\
\exsc{1>2(d)}-doubled.{\stative} \\
\glt ‘cousin!’ (when directly addressing a cousin)  
\z
\z

\tabref{figtab:1:pronselkinship} summarizes pronominal prefix selection and meaning for kinship terms.

\begin{table}
\caption{Pronominal prefix selection, kinship terms}
\label{figtab:1:pronselkinship}
{
\begin{tabularx}{\textwidth}{p{2.5cm}lQ}

\lsptoprule
grammatical category & prefix type & meaning of prefix choice\\
\midrule
\mbox{\textsc{verbs} or \textsc{nouns}} functioning as\newline “kinship terms” & \textsc{p}-series or

\textsc{a}{} series & • \textsc{p} prefix or \textsc{a} prefix refers to point-of-view (or “possessor”)

• Stem specifies the type of relative\\
\midrule
\textsc{same-generation} kinship term & \textsc{interactive} & • 1\textsuperscript{st} role refers to point-of-view (or “possessor”)


• 2\textsuperscript{nd} role specifies the number and gender of the kin or relative

• Stem specifies a same-generation relative\\

\midrule
\textsc{different-generation} kinship term & \textsc{interactive} & • 1\textsuperscript{st} role refers to older generation, regardless of point-of-view

• 2\textsuperscript{nd} role refers to younger generation, regardless of point-of-view

• Stem specifies a different-generation relative

• or, with \emph{reversed roles}, works like same-generation kinship terms\\
\midrule
\end{tabularx}
}
\end{table}

\begin{table}[t]
{
\begin{tabularx}{\textwidth}{p{2cm}lQ}

\lsptoprule
grammatical category & prefix type & meaning of prefix choice\\
\midrule


\textsc{different-generation} kinship term

(in-laws) & \textsc{interactive} & • 1\textsuperscript{st} role refers to older generation, regardless of point-of-view

• 2\textsuperscript{nd} role refers to younger generation, regardless of point-of-view

• Stem specifies a different-generation relative

•or, with \emph{reversed} roles, works like same-generation kinship terms

• In either case, if 1\textsuperscript{st} role refers to ‘she’, then the overall word refers to a male’s mother-in-law\\

\midrule
\textsc{different-generation} kinship term & \textsc{p}{} series & • \textsc{p}{}-prefix refers to a younger person, and the overall word must refer to an older-generation female (or mixed group of people)\\

\midrule
\textsc{different-generation} kinship term & /k-/ \exsc{1s.a}, /ha-/ \textsc{3s.m.a} & • /k-/ means ‘my older female relative’

/ha-/ means ‘my older male relative’ and such words can also be used as \textsc{terms of address}\\
\lspbottomrule
\end{tabularx}}
\end{table}


\clearpage
\section{Verbs functioning as “kinship terms”, \textsc{a}-series prefixes} \label{ch:Verbs functioning as ‘kinship terms’, a-series pronominal prefixes}
Several verbs with \textsc{a}-series pronominal prefixes function as “kinship terms” \xxref{ex:kintermppex}{ex:kintermppex4}. The pronominal prefix expresses the 1\textsuperscript{st}, 2\textsuperscript{nd}, or 3\textsuperscript{rd} person point-of-view (or the “owner” of the relative), and the verb stem specifies the type of relative. Such kinship terms tend to describe relations between people of the same approximate age.


\ea\label{ex:kintermppex}
\stem{de-\textsc{a}-adęhnǫde:ˀ} ‘siblings’ (related to verb \stem{hnǫd(r)} ‘follow someone’)
\ea degaǫdęhnǫ́:de:ˀ\\
\gll de-gaǫ-dę-hnǫ́:de:ˀ\\
 {\dualic}-\textsc{3ns.fi.a}-{\semireflexive}-sibling\\
\glt `his/her sister/brother' (literally, ‘they follow each other’)

\ex de̱hęnadęhnǫ́:de:ˀ\\
\gll de̱-hęn-adę-hnǫ́:de:ˀ\\
 {\dualic}-\textsc{3ns.m.a}-{\semireflexive}-sibling\\
\glt `his brother'

\ex de̱hęnadęhnǫ́:drǫˀ\\
\gll de̱-hęn-adę-hnǫ́:dr-ǫˀ\\
 {\dualic}-\textsc{3ns.m.a}-{\semireflexive}-sibling-{\distributive}\\
\glt `his brothers'
\z
\z

\ea\label{ex:kintermppex2} \stem{de-\textsc{a}-(r)ihwawaˀkǫˀ} ‘spouse’ (related to verb \stem{rihwawaˀs} ‘support someone’)
\ea degaeihwawáˀkǫˀ ‘her husband’,\\
\gll de-gae-ihwawáˀkǫˀ\\
 {\dualic}-\textsc{3ns.fi.a}-support.{\distributive}\\
\glt `his wife'

\ex desni:hwáwaˀkǫˀ\\
\gll de-sni:-hwáwaˀkǫˀ\\
 {\dualic}-\exsc{2d.a}-support.{\distributive}\\
\glt `your wife/husband'

\ex deyakni:hwawáˀkǫ:ˀ\\
\gll de-yakni:-hwawáˀkǫ:ˀ\\
 {\dualic}-\exsc{1d.ex.a}-support.{\distributive}\\
\glt `my wife/husband'
\z
\z
\newpage
\ea\label{ex:samegeninlawex4} \stem{\textsc{a}-adę-nǫhk(sǫˀ)} ‘to be mutually related’
\ea agya:dę́:nǫhk\\
\gll agy-a:dę́:-nǫhk\\
 \exsc{1d.ex.a}-{\semireflexive}-related.{\stative}\\
\glt `my relative'

\ex ǫgya:dę́:nǫhk\\
\gll ǫgy-a:dę́:-nǫhk\\
 \exsc{1d.in.a}-{\semireflexive}-related.{\stative}\\
\glt `my relative'

\ex ǫgwa:dę́:nǫhk\\
\gll ǫgwa:-dę́:-nǫhk\\
 \exsc{1p.in.a}-{\semireflexive}-related.{\stative}\\
\glt `our relative'

\ex ǫgwadęnǫhksǫˀ\\
\gll ǫgwa-dę-nǫhksǫˀ\\
 \exsc{1p.in.a}-{\semireflexive}-related.{\pluralizer}	\\
\glt `our relatives'

\ex ja:dę́:nǫhk\\
\gll j-a:dę́:-nǫhk\\
 \exsc{2d.a}-{\semireflexive}-related.{\stative}\\
\glt `your relative'

\ex gęna:dę́:nǫhk\\
\gll gęn-a:dę́:-nǫhk\\
{3n.\sga}-{\semireflexive}-related.{\stative}\\
\glt ‘they are related’

\ex ǫdę́nǫ̱hksǫˀ\\
\gll ǫ-dę́-nǫ̱hksǫˀ\\
{3fi.\sga}-{\semireflexive}-related.{\pluralizer}\\
\glt ‘relatives, kin’, etc.
\z
\z

\ea\label{ex:kintermppex5} \stem{t-\textsc{a}-gowanę(ˀs)} ‘older sibling’ (verb \stem{t-…-gowanę} ‘biggest’, with \stem{ˀs} {\plural})\\
tgaegówanęˀs\\
\gll t-gae-gówanęˀs\\
 {\cislocative}-\textsc{3ns.fi.a}-big.\plural\\
\glt `my older siblings'
\z

\newpage
\ea\label{ex:kintermppex6} \stem{\textsc{a}-ad-riyahsǫˀ} ‘older siblings’\\
gaǫdriyáhsǫˀ\\
\gll gaǫ-d-riyáhsǫˀ\\
 \textsc{3ns.fi.a}-{\semireflexive}-older.sibling.{\pluralizer}\\
\glt `my older siblings'
\z


\ea\label{ex:kintermppex3} \stem{de-\textsc{a}-nǫhsakahǫh} ‘neighbour’ (consists of \stem{nǫhsa-kahǫ} ‘adjoining houses’)\\
deyagwanǫhsakáhǫh\\
\gll de-yagwa-nǫhsa-káhǫh\\
 {\dualic}-\exsc{1p.ex.a}-house-adjoin.{\stative}\\
\glt `my neighbours'
\z


\ea\label{ex:kintermppex4} \stem{de-\textsc{a}-nǫhsane:gę:} ‘neighbour’ (contains \stem{nǫhsa-negę:} ‘side-by-side houses’)\\
deyagwanǫhsané:gę:\\
\gll de-yagwa-nǫhs-a-né:gę:\\
 {\dualic}-\exsc{1p.ex.a}-house-{\joinerA}-side.by.side.{\stative}\\
\glt `my neighbour'
\z


\section{Verbs and nouns functioning as “kinship terms”, \textsc{p}-series prefixes} \label{ch:Verbs and nouns functioning as ‘kinship terms’, o-series prefixes}
Several \textsc{verbs} with \textsc{p}-series pronominal prefixes function as “kinship terms” \xxref{ex:kintermppex7}{ex:kintermppex14}, as do some \textsc{nouns} \xxref{ex:kintermppex9}{ex:interkintermppex30}. The pronominal prefix expresses the 1\textsuperscript{st}, 2\textsuperscript{nd}, or 3\textsuperscript{rd} person point-of-view (or the “owner(s)” of the relative(s)), and the stems specify the type of relative. These kinship terms tend to describe relations between people of the same approximate age.

\ea\label{ex:kintermppex7} \stem{\textsc{p}-adaoˀ} ‘ceremonial friend’ (related to verb \stem{ao, ęo} ‘to run, race’)
\ea ǫgya:dáoˀ\\
\gll ǫgya:-d-áoˀ\\
 \exsc{1d.p}-{\semireflexive}-run.{\stative} \\
\glt `my ceremonial friend'

\ex hona:dáoˀ\\
\gll hon-a:d-áoˀ\\
 \textsc{3ns.m.p}-{\semireflexive}-run.{\stative}\\
\glt `his ceremonial friend'

\ex ǫgwadáoˀsǫˀ\\
\gll ǫgwa-d-áoˀ-sǫˀ\\
 {1\plo}-{\semireflexive}-run.{\stat}-{\pluralizer}\\
\glt `all of our ceremonial friends'
\z
\z

\ea\label{ex:kintermppex8} \stem{\textsc{p}-aˀse:ˀ} ‘cousin’ (related to verb \stem{aˀse:ˀ} ‘doubled’)
\ea ǫgyáˀse:ˀ\\
\gll ǫgy-áˀse:ˀ\\
 \exsc{1d.p}-doubled.{\stative}\\
\glt `my cousin'

\ex honáˀse:ˀ\\
\gll hon-áˀse:ˀ\\
 \textsc{3ns.m.p}-doubled.{\stative}\\
\glt `his cousin'

\ex ǫgwaˀse:ˀsǫ́:ˀah\\
\gll ǫgwa-ˀse:ˀ-sǫ́:-ˀah\\
 {1\plo}-doubled.{\stative}-{\pluralizer}-{\diminutive}\\
\glt `all of my cousins'
\z
\z

\ea\label{ex:kintermppex12} \stem{\textsc{p}-atsih} ‘friend’ (related to verb \stem{atsih} ‘paired’)
\ea ǫgyá:tsih\\
\gll ǫgy-á:tsih\\
 \exsc{1d.p}-paired.{\stative}\\
\glt `my friend'

\ex honá:tsih\\
\gll hon-á:tsih\\
 \textsc{3ns.m.p}-paired.{\stative}\\
\glt `his friend'

\ex ǫgwátsi̱hsǫˀ\\
\gll ǫgw-átsi̱h-sǫˀ\\
 {1\plo}-paired.{\stat}-{\pluralizer}\\
\glt `all of our friends'
\z
\z

\ea\label{ex:kintermppex13} \stem{\textsc{p}-agyoh} ‘sibling-in-law, in-laws’
\ea ǫgyá:gyoh\\
\gll ǫgy-á:g-yoh\\
 \exsc{1d.p}-sibling.in.law\\
\glt `my brother-in-law, sister-in-law'

\ex honá:gyoh\\
\gll hon-á:g-yoh\\
 \textsc{3ns.m.p}-sibling.in.law\\
\glt `her/his brother-in-law'
\z
\z

\ea\label{ex:kintermppex14} \stem{de-\textsc{p}-hsnyeˀǫh} ‘adopted one’ (related to verb \stem{hsnyeˀ} ‘care for, look after’)\\
deyago̱hsnyéˀǫh\\
\gll de-yago̱-hsnyéˀǫh\\
 {\dualic}-\textsc{3s.fi.p}-cared.for.{\stative}\\
\glt `her adopted daughter'
\z


\ea\label{ex:kintermppex9} \stem{\textsc{p}--hwajiy-aˀ} ‘family’ (\textsc{noun} \stem{hwajiy} ‘family’)\\
akwa:jí:yaˀ\\
\gll ak-hwa:jí:y-aˀ\\
 \textsc{1s.p}-family-{\nounstemformer}\\
\glt `my family'
\z


\ea\label{ex:kintermppex10} \stem{\textsc{p}-ǫgweˀd-aˀ} ‘relatives, people’ (\textsc{noun} \stem{ǫgwe-ˀd} ‘person’)\\
agǫ́gwe̱ˀdaˀ\\
\gll ag-ǫ́gwe̱ˀd-aˀ\\
 \textsc{1s.p}-people-{\nounstemformer}\\
\glt `my relatives'
\z


\ea\label{ex:kintermppex310} \stem{\textsc{p}-adreˀtr-aˀ} ‘grandchild(ren)’ (\textsc{noun} \stem{adreˀtr} ‘grandchild(ren)’)\\
agádre̱ˀtraˀ \\
\gll ag-adreˀtr-aˀ\\
\textsc{1s.p}-grandchildren-{\nounstemformer}\\
\glt ‘my grandchild(ren)’
\z


\ea\label{ex:interkintermppex30} \stem{\textsc{p}-ksaˀgowahsr-aˀ} ‘someone’s boyfriend, girlfriend’ (\textsc{noun} \stem{ksaˀgowahsr} ‘beautiful one’)\footnote{In context, the boy/girl-friend is taken to be the opposite sex to the person referred to by the pronominal prefix.}
\ea ageksaˀgówa̱hsraˀ\\
\gll age-ksaˀgówa̱hsr-aˀ\\
 \textsc{1s.p}-beautiful.one-{\nounstemformer}\\
\glt `my boyfriend or girlfriend'
\ex hoksaˀgohwáhsraˀ\\
\gll ho-ksaˀgohwáhsr-aˀ\\
\textsc{3s.m.p}-beautiful.one-{\nounstemformer}\\
\glt `his girlfriend'
\ex goksaˀgowáhsraˀ\\
\gll go-ksaˀgowáhsr-aˀ\\
 \textsc{3s.fi.p}-beautiful.one-{\nounstemformer}\\
\glt `her boyfriend'
\cfex{keksaˀgowáhsraˀ ‘my girlfriend’ with an \textsc{interactive} prefix\\
\gll ke-ksaˀgowáhsr-aˀ\\
\exsc{1s>3s.fi}-beautiful.one-{\nounstemformer} (see \ref{ex:interkintermppex29})\\ }
\z
\z

\section{Interactive kinship terms, interactive pronominal prefixes} \label{ch:Interactive kinship terms, interactive pronominal prefixes}
Interactive kinship terms consist of a stem with an interactive pronominal prefix. For comparison, example \xref{ex:interkintermppex} illustrates an interactive kinship term \xref{ex:interkintermppexa} and two verbs (not kinship terms, \ref{ex:interkintermppex}b,c) that take the same type of interactive prefix.  

\ea\label{ex:interkintermppex}
\ea heyá:dreˀ\\\label{ex:interkintermppexa}
\gll hey-á:dreˀ\\
 \textsc{1s>3s.m}-grandchild \\
\glt `my grandson' (kinship term with interactive prefix)
\ex henǫ́hweˀs\\
	\gll he-nǫ́hweˀ-s\\
	\textsc{1s>3s.m}-like-{\habitual} \\
	\glt `I like him' (verb with interactive prefix)
\ex hehswáˀne:t\\
	\gll he-hswáˀn-e:t\\
	\textsc{1s>3s.m}-back-stand.{\stative} \\
	\glt `I support or back him' (verb with interactive prefix)
\z
\z 


Recall that interactive pronominal prefixes express two sets of person, number, and gender features. The two sets are referred to below as \emph{first} and \emph{second} roles. (The format \textsc{first>second} role is used in the translations.) For verbs (\ref{ex:interkintermppex}b, c), the first role generally refers to the “doer” or \textsc{agent} and the second role refers to the “recipient” or \textsc{patient} of the action. However, for interactive kinship terms, the interactive prefixes work differently, as described in the following sections.

The two types of interactive kinship terms to be described next are \textsc{same-generation} versus \textsc{different-generation} ones.  


\subsection{Same-generation kinship terms, interactive pronominal prefixes} \label{ch:Same-generation kinship terms, interactive pronominal prefixes}
\textsc{same-generation kinship terms} denote a relationship between people (or groups of people) of approximately the same generation. For such words, the first role of the interactive prefix refers to the point-of-view or “possessor” of the relative: for example, the prefix \stem{he-} \textsc{1s>3s.m} refers to \emph{my} relative, as does \stem{ke-} \exsc{1s>3s.fi} \xref{ex:interkintermppex29}. The second role specifies the number and gender of the kin or relative being referred to: for example, \stem{he-} \exsc{1s>3s.m} refers to one \emph{male} relative, while \stem{ke-} \exsc{1s>3s.fi} refers to one \emph{female} relative \xref{ex:interkintermppex29}. Other kinship terms that work the same way are listed below.

\ea\label{ex:interkintermppex29} \stem{\textsc{int}-ksaˀgowahsraˀ} ‘boyfriend, girlfriend’ 
\ea heksaˀgowáhsraˀ\\
\gll he-ksaˀgowáhsr-aˀ\\
 \textsc{1s>3s.m}-beautiful.one-{\nounstemformer}\\
\glt `my boyfriend'
\ex keksaˀgowáhsraˀ\\
\gll ke-ksaˀgowáhsr-aˀ\\
 \exsc{1s>3s.fi}-beautiful.one-{\nounstemformer}\\
\glt `my girlfriend'
\z
\z

\ea\label{ex:samegeninlawex5} \stem{\textsc{int}-nǫhk(sǫˀ)} ‘close relative, to be closely related to someone’ (related to verb \stem{nǫhkw} ‘love someone’)
\ea kenǫ́hksǫˀ\\
\gll ke-nǫ́hksǫˀ\\
 \exsc{1s>3s.fi}-related.{\pluralizer}\\
\glt `my relatives'

\ex shenǫ́hksǫˀ\\
\gll she-nǫ́hksǫˀ\\
\exsc{2s>3s.fi}-related.{\pluralizer}\\
\glt  ‘your relatives’ (high language, describing our relationship to all people)
\z
\z

\newpage
\ea\label{ex:samegeninlawex6}
\stem{\textsc{int}-gęhjih} ‘spouse’ (literally, ‘old person’)\\
\ea hegę́hjih \\
\gll  he-gę́hjih \\
\textsc{1s>3s.m}-old.one.{\stative}\\
\glt  ‘my husband’  ‘my old man’ (informal)

\ex kegę́hjih \\
\gll  ke-gę́hjih \\
\exsc{1s>3s.fi}-old.one.{\stative}\\
\glt  ‘my wife’, ‘my old lady’ (informal)

\ex hehségę̱hjih \\
\gll  hehsé-gę̱hjih\\
\textsc{2s>3s.m}-old.one.{\stative}\\
\glt  ‘your husband’, ‘your old man’ (informal)

\ex shegę́hjih \\
\gll  she-gę́hjih\\
\exsc{2s>3s.fi}-old.one.{\stative}\\
\glt  ‘your wife’, ‘your old lady’ (informal)

\ex hǫwágę̱hjih \\
\gll  hǫwá-gę̱hjih\\
\exsc{3s.m/3s.fi>3s.m}-old.one.{\stative}\\
\glt  ‘her husband’, ‘her old man’ (informal)

\ex shagógę̱hjih \\
\gll  shagó-gę̱hjih\\
\exsc{3s.m>3fi/3p}-old.one.{\stative}\\
\glt  ‘his wife’, ‘his old lady’ (informal)
\z
\z

The following \textsc{terms of address} (words used when speaking directly to the relative in question, or words used instead of a name) are also inflected like kinship terms. They refer to same-generation relatives and use interactive prefixes \xref{ex:termaddppex7}. 

\ea\label{ex:termaddppex7} \stem{\textsc{int}-aˀse:ˀ} ‘cousin!’ (term of address)
\ea gya̱ˀse:ˀ\\
\gll gy-aˀse:ˀ\\
\exsc{1>2(d)}-doubled\\
\glt ‘cousin!’ (when directly addressing a cousin)

\ex gwa̱ˀse:ˀ\\
\gll gw-aˀse:ˀ\\
\exsc{1>2(p)}-doubled\\
\glt ‘cousins!’ (when directly addressing one’s cousins)
\z
\z

In addition, examples \xref{ex:termaddppex7} and \xref{ex:twotypes2} (repeated from page \pageref{ex:kintermppex8}) together show that the same stem can sometimes be inflected in two different ways. Example \xref{ex:twotypes2a} shows a \textsc{verb} functioning as a “kinship term”, inflected with \textsc{p}-series pronominal prefixes. In contrast, example \xref{ex:twotypes2b} illustrates the same stem as an interactive kinship term, inflected with interactive pronominal prefixes.

\ea\label{ex:twotypes2} 
\ea\label{ex:twotypes2a} \stem{\textsc{p}-aˀse:ˀ} ‘cousin’ \\
\glll ǫgyáˀse:ˀ \\
ǫgy-áˀse:ˀ\\
 {\wetwo}-doubled.{\stative} \\
\glt `my cousin'

\ex \stem{\textsc{int}-aˀse:ˀ} ‘cousin’ \\\label{ex:twotypes2b}
\glll gya̱ˀse:ˀ \\
gy-aˀse:ˀ \\
\exsc{1>2(d)}-doubled.{\stative} \\
\glt ‘cousin!’ (when directly addressing a cousin)
\z
\z


\subsection{Different-generation kinship terms, interactive pronominal prefixes} \label{ch:Different-generations kinship terms, interactive pronominal prefixes}
\textsc{different-generation} kinship terms describe a relationship between people (or groups of people) of different generations. They prioritize the older generation over the younger generation: the first role of the pronominal prefix refers to the older generation, and the second role refers to the younger generation, regardless of point-of-view. For example, in \textit{heyá:dreˀ} ‘my grandson’ \xref{ex:interkintermppex3a}, the first role of the \stem{he-} \textsc{1s>3s.m} prefix refers to the older relative (the grandparent/“possessor” ‘I, my’) and the second role refers to the younger relative (the grandson, ‘he, him’). Meanwhile, in \textit{hagá:dreˀ} ‘his grandson, me’\footnote{Thanks to Karin Michelson for this wording.} or ‘I am his grandson’ \xref{ex:interkintermppex3b}, the first role of the \stem{hag-} \textsc{3s.m>1s} prefix \emph{still} refers to the older relative (the grandparent/“possessor”, ‘he, his’), and the second role still refers to the younger relative (the grandson, ‘I, me’).

\newpage
\ea\label{ex:interkintermppex3} \stem{\textsc{int}-adreˀ} ‘grandchild relationship’
\ea heyá:dreˀ\\\label{ex:interkintermppex3a}
\gll hey-á:dreˀ \\
 \textsc{1s>3s.m}-grandchild\\
\glt `my grandson'

\ex hagá:dreˀ \\\label{ex:interkintermppex3b}
\gll hag-á:dreˀ\\
 \textsc{3s.m>1s}-grandchild\\
\glt ‘his grandson, me’, `I am his grandson'

\ex keyá:dreˀ\\\label{ex:interkintermppex3c}
\gll key-á:dreˀ\\
 \exsc{1s>3s.fi}-grandchild\\
\glt `my granddaughter'

\ex hǫwá:dreˀ\\\label{ex:interkintermppex3d}
\gll hǫwá:-dreˀ\\
 \exsc{3s.m/3s.fi>3s.m}-grandchild\\
\glt `his/her grandson'

\ex sheyá:dreˀ\\\label{ex:interkintermppex3e}
\gll shey-á:dreˀ\\
 \exsc{2s>3s.fi}-grandchild\\
\glt `your granddaughter'

\ex ǫdádadreˀ\\\label{ex:interkintermppex3f}
\gll ǫdád-adreˀ\\
 \exsc{3s.fi>3s.fi}-grandchild\\
\glt `her granddaughter'

\ex shagó:dreˀ\\\label{ex:interkintermppex3g}
\gll shagó:-dreˀ\\
 \exsc{3s.m>3fi/3p}-grandchild \\
\glt `his granddaughter'

\ex gaǫdadreˀsǫ́:ˀah\\\label{ex:interkintermppex3h}
\gll gaǫd-adreˀ-shǫ́:ˀah\\
 \exsc{3fi>3fi(+ns)}-grandchild-{\pluralizer}\\
\glt `her grandchildren'

\newpage
\ex shagódre̱ˀsǫˀ\\\label{ex:interkintermppex3i}
\gll shagó-dre̱ˀ-shǫˀ\\
 \exsc{3s.m>3fi/3p}-grandchild-{\pluralizer}\\
\glt `his grandchildren'

\ex gakeyadréˀsǫˀ\\
\gll gakey-adréˀ-shǫˀ\\
 \exsc{1s>3ns}-grandchild-{\pluralizer}\\
\glt `my grandchildren'

\ex gasheyadréˀsǫˀ\\
\gll gashey-adréˀ-shǫˀ\\
\exsc{2s>3ns}-grandchild-{\pluralizer}\\
\glt `your grandchildren'

\ex hehsá:dreˀ\\
\gll hehs-á:dreˀ\\
 \textsc{2s>3s.m}-grandchild \\
\glt `your grandson'

\ex gwadre:ˀ\\
\gll gw-adreˀ\\
 \exsc{1>2(p)}-grandchild \\
\glt `you are my grandchild' (term of address)
\z
\z

Other kinship terms that work the same way are listed below. 

\ea\label{ex:samegeninlawex} \stem{\textsc{int}-ˀgę:ˀęh} ‘younger sibling’
\ea heˀgę́:ˀęh\\
\gll he-ˀgę́:ˀęh\\
 \textsc{1s>3s.m}-younger.sibling\\
\glt `my younger brother'
\ex hesheˀgę́:ˀęh\\
\gll heshe-ˀgę́:ˀęh\\
 \textsc{2s>3s.m}-younger.sibling\\
\glt `your younger brother'
\newpage
\ex keˀgę́:ˀęh\\
\gll khe-ˀgę́:ˀęh\\
 \exsc{1s>3s.fi}-younger.sibling\\
\glt `my younger sister'
\ex ǫdade̱ˀgę́:ˀęh\\
\gll ǫdade̱-ˀgę́:ˀęh\\
 \exsc{3s.fi>3s.fi}-younger.sibling\\
\glt `her younger sister'
\ex shagoˀgę́:ˀęh\\
\gll shago-ˀgę́:ˀęh\\
 \exsc{3s.m>3fi/3p}-younger.sibling\\
\glt `his younger sister'
\ex hǫwaˀgę́:ˀęh\\
\gll hǫwa-ˀgę́:ˀęh\\
 \exsc{3s.m/3s.fi>3s.m}-younger.sibling\\
\glt `his/her younger brother'
\ex sheˀgę́:ˀęh\\
\gll she-ˀgę́:ˀęh\\
 \exsc{2s>3s.fi}-younger.sibling\\
\glt `your younger sister'
\ex etiˀgę́:ˀah\\
\gll eti-ˀgę́:ˀah\\
 \exsc{1ns.in>3s.fi/3ns}-younger.sibling\\
\glt `our younger sister'
\ex shǫgwaˀgę́:ˀah\\
\gll shǫgwa-ˀgę́:ˀah\\
\exsc{3s.m>1p}-younger.sibling\\
\glt `our younger brother'
\z
\z

\ea\label{ex:interkintermppex4} \stem{\textsc{int}-hawahk(sǫˀ)} ‘child(ren)’
\ea hehá:wahk\\
\gll he-há:wahk\\
 \textsc{1s>3s.m}-child\\
\glt `my son'
\newpage
\ex kehá:wahk\\
\gll ke-há:wahk\\
 \exsc{1s>3s.fi}-child\\
\glt `my daughter'
\z
\z

\ea\label{ex:interkintermppex6} \stem{\textsc{int}-hawa:kˀah} ‘(maternal) niece’ (i.e. the child of one’s mother’s brother or mother’s sister)
\ea ǫdatawá:kˀah\\
\gll ǫdat-hawá:kˀah\\
 \exsc{3s.fi>3s.fi}-niece\\
\glt `her maternal niece'

\ex kehawá:kˀah\\
\gll ke-hawá:kˀah\\
 \exsc{1s>3s.fi}-niece\\
\glt `my maternal niece'
\z
\z

\ea\label{ex:interkintermppex7} \stem{\textsc{int}-ǫhwadęˀ(sǫˀ)} ‘niece, nephew’
\ea gakeyǫ̱hwá:dęˀ \\
\gll gakey-ǫ̱hwá:dęˀ\\
 \exsc{1s>3ns}-niece/nephew\\
\glt ‘my brother’s children’, `my nieces and nephews'

\ex keyǫhwá:dęˀ\\
\gll key-ǫhwá:dęˀ\\
 \exsc{1s>3s.fi}-niece/nephew\\
\glt `my niece'

\ex gasheyǫ̱hwádę̱ˀsǫˀ\\
\gll gashey-ǫ̱hwádę̱ˀsǫˀ\\
\exsc{2s>3ns}-niece/nephew\\
\glt `your nieces and nephews'
\z
\z

\ea\label{ex:interkintermppex8} \stem{\textsc{int}-yaˀdawęh} ‘niece, nephew’ (literally, ‘to support someone’)
\ea gakeya̱ˀdá:węh \\
\gll gake-ya̱ˀdá:węh \\
  \exsc{1s>3ns}-niece/nephew \\
\glt ‘my brother’s children’ 

\newpage

\ex heyáˀdawęh\\
\gll he-yáˀdawęh\\
 \textsc{1s>3s.m}-niece/nephew\\
\glt `my nephew'

\ex keyáˀdawęh\\
\gll ke-yáˀdawęh\\
\exsc{1s>3s.fi}-niece/nephew\\
\glt ‘my brother’s kids’
\z
\z

\ea\label{ex:samegeninlawex2} \stem{\textsc{int}-hjiˀah} ‘older sibling’
\ea hehshéhjiˀah\\
\gll hehshé-hjiˀah\\
 \textsc{2s>3s.m}-older.sibling\\
\glt `your older brother'

\ex kehjíˀah\\
\gll khe-hjíˀah\\
 \exsc{1s>3s.fi}-older.sibling\\
\glt `my older sister'
\z
\z


\ea\label{ex:interkintermppex10} \stem{\textsc{int}-hsot} ‘grandparent’
\ea hagéhso:t, hakso:t\\
\gll hagé/hak-hso:t \\
 \textsc{3s.m>1s}-grand.parent\\
\glt `my grandfather'

\ex hǫwáhso:t\\
\gll hǫwá-hso:t\\
 \exsc{3s.m/3s.fi>3s.m}-grand.parent\\
\glt `his/her grandfather'
\z
\z

\ea\label{ex:interkintermppex13} \stem{\textsc{int}-no:haˀ} ‘mother’ (also see \ref{ex:interkintermppex14}, \ref{ex:interkintermppex15})\\
ǫkíno̱haˀ\\
\gll ǫkí-no̱haˀ\\
 \exsc{3s.fi/3ns>1ns}-mother\\
\glt `our mother'
\z

\newpage
\ea\label{ex:interkintermppex16} \stem{\textsc{int}-nohá:ˀah} ‘(maternal) aunt’\footnote{This word used to refer only to one’s mother’s sister. Nowadays, it can also refer to one’s father’s sister.} (also see \ref{ex:interkintermppex17}, \ref{ex:interkintermppex18})\\
ǫkino̱há:ˀah\\
\gll ǫki-no̱há:ˀah\\
 \exsc{3s.fi/3ns>1ns}-aunt\\
\glt `our (plural) aunt'
\z


\ea\label{ex:interkintermppex19} \stem{\textsc{int}-noˀsęh} ‘(maternal) uncle’\footnote{This word used to refer only to one’s mother’s brother. Nowadays, it can also refer to one’s father’s brother.} (also see \ref{ex:interkintermppex20})
\ea hǫwáno̱ˀsęh\\
\gll hǫwá-no̱ˀsęh\\
 \exsc{3s.m/3s.fi>3s.m}-uncle\\
\glt `his/her uncle'
\ex hyanóˀsęh\\
\gll hya-nóˀsęh\\
 \exsc{3s.m>2s}-uncle\\
\glt `your uncle'
\ex shǫkníno̱ˀsęh\\
\gll shǫkní-no̱ˀsęh\\
 \exsc{3s.m>1d}-uncle\\
\glt `our uncle'
\z
\z

\ea\label{ex:interkintermppex21} \stem{\textsc{int}-ˀnih} ‘father’ (also see \ref{ex:interkintermppex22})
\ea hǫwáˀnih\\
\gll hǫwá-ˀnih\\
 \exsc{3s.m/3s.fi>3s.m}-father\\
\glt `his/her father'

\ex hya̱ˀnih\\
\gll hya̱-ˀnih\\
 \exsc{3s.m>2s}-father\\
\glt `your father'
\z
\z

\ea\label{ex:interkintermppex23}
\stem{\textsc{int}-naˀehs} ‘step-mother’ (also see \ref{ex:interkintermppex28})\\
ǫknáˀehs\\
\gll ǫk-náˀehs\\
\exsc{3s.fi>1s}-step.mother\\
\glt ‘my step-mother’
\z


\ea\label{ex:interkintermppex25}
\stem{\textsc{int}-ˀni:hah} ‘godfather, step-father’ (also see \ref{ex:interkintermppex26})\\
hǫwaˀní:hah\\
\gll hǫwa-ˀní:hah\\
 \exsc{3s.m/3s.fi>3s.m}-step.father\\
\glt `her godfather/step-father'
\z


\ea\label{ex:interkintermppex5} \stem{\textsc{int}-no:ˀ} ‘step-parent/child’
\ea hakno:ˀ\\
\gll hak-no:ˀ\\
 \textsc{3s.m>1s}-step.parent/child\\
\glt `my step-father'
\ex hehsno:ˀ\\
\gll hehs-no:ˀ\\
 \textsc{2s>3s.m}-step.parent/child\\
\glt `your step-son'
\ex shé:no:ˀ\\
\gll shé:-no:ˀ\\
 \exsc{2s>3s.fi}-step.parent/child\\
\glt `your step-daughter'
\z
\z

\subsubsection*{Different-generation kinship terms, reversed roles} \label{ch:Different-generation kinship terms, reversed roles}
Several of the different-generation kinship terms described in the last section can instead be inflected just like same-generation kinship terms (see \sectref{ch:Same-generation kinship terms, interactive pronominal prefixes}): the first role denotes the point-of-view (the “possessor” of the relative) and the second role specifies the number and gender of the relative in question. We can think of these as “reversed role” kinship terms. Examples contrasting the two types are shown below.

\ea\label{ex:interkintermppex16a} \stem{\textsc{int}-nohá:ˀah} ‘(maternal) aunt’ (reversed roles, like same-generation kinship terms)\\
etino̱há:ˀah\\
\glt ‘our aunt’ (two of us)\\
\gll eti-no̱há:ˀah\\
\exsc{1ns.in>3s.fi/3ns}-aunt\\
\cfex{ǫkino̱há:ˀah\\
	\gll ǫki-no̱há:ˀah\\
	\exsc{3s.fi/3ns>1ns}-aunt \\
	\glt `our aunt' (non-reversed roles, like regular different-generation kinship terms)}
\z


\ea\label{ex:interkintermppex85}
\stem{\textsc{int}-noˀsęh} ‘(maternal) uncle’ \\

\glll Etino̱ˀsę́h Ęhęnatnoˀáowanaht\\
eti-no̱ˀsę́h {}\\ 
  \exsc{1ns.in>3s.fi/3ns}-uncle they.become.bigheads\\
\glt ‘Our Uncles ceremony’ (reversed roles, like same-generation kinship terms)\\

\cfex{shǫkníno̱ˀsęh\\
	\gll shǫkní-no̱ˀsęh\\
	\exsc{3s.m>1d}-uncle \\
	\glt `our (dual) uncle' (non-reversed roles, like regular different-generation kinship terms)}
\z



\subsubsection*{Different-generation kinship terms, in-laws} \label{ch:Different-generation kinship terms, in-laws}
The two kinship terms described in this section both refer to different-generation in-laws (the parent-in-law / child-in-law relationship). These terms work differently than other different-generation kinship terms (\sectref{ch:Different-generations kinship terms, interactive pronominal prefixes}), partly because of the unique meaning of the stems. 

For the kinship term \stem{nenhǫs}, the first role can refer to the older generation and the second role can refer to the younger generation (\ref{ex:diffgeninlawex2a}a--c), which is mostly like the different-generation kinship terms described on page \pageref{ch:Different-generations kinship terms, interactive pronominal prefixes}. However, if the first role and point-of-view (the “possessor”) coincide, then the stem means ‘son-in-law’ \xref{ex:diffgeninlawex2aa}, and if the first role and point-of-view (the “possessor”) do not coincide, then the stem means ‘father-in-law’ (\ref{ex:diffgeninlawex2a}b, c). Lastly, if the first role is \textsc{3s.fi} ‘she’, the word refers only to a male speaker’s mother-in-law \xref{ex:diffgeninlawex2ad}.

\ea\label{ex:diffgeninlawex2a} \stem{\interactive-nenhǫs} ‘someone’s father/son-in-law’, ‘a male’s mother-in-law’
\ea hené:nhǫ:s\\\label{ex:diffgeninlawex2aa}
\gll he-né:nhǫ:s\\
 \textsc{1s>3s.m}-in.law\\
\glt `my son-in-law'
\ex hakné:nhǫ:s\\\label{ex:diffgeninlawex2ab}
\gll hak-né:nhǫ:s\\
 \textsc{3s.m>1s}-in.law\\
\glt `my father-in-law'
\ex hyané:nhǫ:s\\\label{ex:diffgeninlawex2ac}
\gll hya-né:nhǫ:s\\
 \exsc{3s.m>2s}-in.law\\
\glt `your father-in-law'
\ex ǫkné:nhǫ:s\\\label{ex:diffgeninlawex2ad}
\gll ǫk-né:nhǫ:s\\
\exsc{3s.fi>1s}-in.law\\
\glt ‘my mother-in-law’ (a man speaking)\\
\z
\z

Alternatively, \stem{nenhǫs} can work like the same-generation kinship terms described in \sectref{ch:Same-generation kinship terms, interactive pronominal prefixes}: the first role can just refer to the point-of-view (the “possessor” of the relative) while the second role will describe the number and gender of the relative \xref{ex:diffgeninlawex2ba}. That being said, if the second role is \textsc{3s.fi} ‘she’, the word refers only to a male’s mother-in-law (\ref{ex:diffgeninlawex2b}b, c).

\ea\label{ex:diffgeninlawex2b} \stem{interactive-nenhǫs} ‘someone’s father/son-in-law’, ‘a male’s mother-in-law’
\ea hǫwánenhǫ:s\\\label{ex:diffgeninlawex2ba}
\gll hǫwá-nenhǫ:s\\
 \exsc{3s.m/3s.fi>3s.m}-in.law\\
\glt `his/her father-in-law, son-in-law'
\ex shagónenhǫ:s\\\label{ex:diffgeninlawex2bb}
\gll shagó-nenhǫ:s\\
 \exsc{3s.m>3fi/3p}-in.law\\
\glt `his mother-in-law'
\ex shené:nhǫ:s \\\label{ex:diffgeninlawex2bc}
\gll  she-né:nhǫ:s \\
 \exsc{2s>3s.fi}-in.law\\
\glt ‘your mother-in-law’ (when speaking to a man)
\z
\z

For \stem{hseyǫh}, the first role can refer to the older generation and the second role can reference the younger generation, as long as both roles refer to females (\ref{ex:diffgeninlawex3a}a, b).

\ea\label{ex:diffgeninlawex3a} \stem{\interactive-hse-yǫh} ‘a female’s mother/daughter-in-law’
\ea ǫgehsé:yǫh\\
\gll ǫge-hsé:yǫh\\
\exsc{3s.fi>1s}-in.law\\
\glt ‘my mother-in-law’ (a woman speaking)

\ex kehsé:yǫh\\
\gll khe-hsé:yǫh\\
\exsc{1s>3s.fi}-in.law\\
\glt ‘my daughter-in-law’ (a woman speaking)
\z
\z

Alternatively (as with same-generation kinship terms, \sectref{ch:Same-generation kinship terms, interactive pronominal prefixes}), the first role for \stem{hseyǫh} can just refer to the point-of-view (the “possessor” of the relative) while the second role denotes the relative. Both roles still only refer to females \xref{ex:diffgeninlawex3b}.

\ea\label{ex:diffgeninlawex3b} \stem{\interactive-hse-yǫh} ‘a female’s mother/daughter-in-law’
\ea ǫdade̱hsé:yǫh\\
\gll ǫdade̱-hsé:yǫh\\
 \exsc{3s.fi>3s.fi}-in.law\\
\glt `her mother-in-law'
\ex shehsé:yǫh \\
\gll she-hsé:yǫh\\
\exsc{2s>3s.fi}-in.law\\
\glt ‘your mother-in-law, daughter-in-law' (when speaking to a woman)
\z
\z

\subsection{Different-generation kinship terms taking non-interactive \textsc{a}- or \textsc{p}-series prefixes} \label{ch:Different-generation kinship terms taking non-interactive a or o series pronominal prefixes}
The different-generation kinship terms described in \sectref{ch:Interactive kinship terms, interactive pronominal prefixes} normally (or maybe traditionally) take interactive prefixes. However, several of these terms can also be inflected with non-interactive, \textsc{a}- or \textsc{p}-series prefixes. They are described next.


\subsubsection*{Different-generation kinship terms taking \textsc{p}-series prefixes} \label{ch:Different-generation kinship terms taking o series prefixes}
While different-generation kinship terms normally take interactive pronominal prefixes (\sectref{ch:Different-generations kinship terms, interactive pronominal prefixes}), they can take \textsc{p}-series prefixes instead. In such cases, the pronominal prefix refers to the younger member(s) of the relationship, and the stem refers to older-generation females \xxref{ex:samegeninlawex3}{ex:interkintermppex28}. If interactive prefixes are used, the same stem can refer to older-generation males (as shown by the forms labelled “cf.” in \ref{ex:samegeninlawex3}-\ref{ex:interkintermppex28}). 

\ea\label{ex:samegeninlawex3} \stem{\textsc{p}-hjiˀah} ‘to have an older sister’\\
hohjíˀah\\
\gll ho-hjíˀah\\
\textsc{3s.m.p}-older.sibling\\
\glt `his older sister'
\cfex{hehshéhjiˀah\\
	\gll hehshé-hjiˀah\\
	\textsc{2s>3s.m}-older.sibling \\
	\glt `your older brother' (from example \ref{ex:samegeninlawex2})}
\z


\ea\label{ex:interkintermppex11} \stem{\textsc{p}-hsot} ‘to have a grandmother’
\ea hohso:t\\
\gll ho-hso:t\\
\textsc{3s.m.p}-grand.parent\\
\glt `his Grandma'
\ex sa̱hso:t\\
\gll sa̱-hso:t\\
 \exsc{2s.p}-grand.parent\\
\glt `your Grandmother'
\cfex{hagéhso:t, hakso:t\\
    \gll  hagé/hak-hso:t\\
	\textsc{3s.m>1s}-grand.parent\\
	\glt `my grandfather' (from example \ref{ex:interkintermppex10})}
\z
\z

\ea\label{ex:interkintermppex28} \stem{\textsc{p}-no:ˀ} ‘have a step-mother’
\ea sano:ˀ\\
\gll sa-no:ˀ\\
 \textsc{2s.p}-step.parent/child\\
\glt `your step-mother'
\ex hono:ˀ\\
\gll ho-no:ˀ\\
\textsc{3s.m.p}-step.parent/child\\
\glt `his step-mother'
\cfex{hakno:ˀ\\
	\gll hak-no:ˀ\\
	\textsc{3s.m>1s}-step.parent/child \\
	\glt `my step-father' (from example \ref{ex:interkintermppex5})}
\z
\z

\ea\label{ex:interkintermppex14} \stem{\textsc{p}-no:haˀ} ‘to have a mother’
\ea honó:haˀ\\
\gll ho-nó:haˀ\\
\textsc{3s.m.p}-mother\\
\glt `his mother'
\ex sanó:haˀ\\
\gll sa-nó:haˀ\\
 \textsc{2s.p}-mother\\
\glt `your mother'
\cfex{ǫkíno̱haˀ\\
	\gll ǫkí-no̱haˀ\\
	\exsc{3s.fi/3ns>1ns}-mother \\
	\glt `our mother' (from example \ref{ex:interkintermppex13})}
\z
\z

\ea\label{ex:interkintermppex17} \stem{\textsc{p}-no:ha:ˀah} ‘to have a (maternal) aunt’
\ea gono:há:ˀah\\
\gll go-no:há:ˀah\\
 \textsc{3s.fi.p}-aunt\\
\glt `her aunt'

\ex hono:há:ˀah\\
\gll ho-no:há:ˀah\\
\textsc{3s.m.p}-aunt\\
\glt ‘his aunt’ (on the maternal side)\\

\ex sano:há:ˀah\\
\gll sa-no:há:ˀah\\
 \textsc{2s.p}-aunt\\
\glt `your aunt'

\cfex{ǫkino̱há:ˀah\\
	\gll ǫki-no̱há:ˀah\\
	\exsc{3s.fi/3ns>1ns}-aunt \\
	\glt `our aunt' (from example \ref{ex:interkintermppex16})}
\z
\z

\ea\label{ex:interkintermppex24} \stem{\textsc{p}-naˀehs} ‘to have a step-mother’
\ea aknáˀehs\\
\gll ak-náˀehs\\
 \textsc{1s.p}-step.mother\\
\glt `I have a step-mother'

\ex sanáˀehs\\
\gll sa-náˀehs\\
 \textsc{2s.p}-step.mother\\
\glt `your step-mother'

\cfex{ǫknáˀehs\\
	\gll ǫk-náˀehs\\
	\exsc{3s.fi>1s}-step.mother \\
	\glt `my step-mother'} (from example \ref{ex:interkintermppex23})
\z
\z

\newpage
\ea\label{ex:kintermppex15} \stem{\textsc{p}-haˀk} ‘to have an aunt’ (archaic word, possibly related to verb \textit{ga̱haˀk} ‘it did sit on something’)\\
agé:haˀk\\
\gll agé:-haˀk\\
\textsc{1s.p}-aunt\\
\glt ‘my aunt’ (archaic word)
\z


\ea\label{ex:kintermppex11}
\stem{\textsc{p}-adę-ni-ho:nǫˀ} ‘to have relatives on one’s father’s side’. (Related to \stem{ˀnih} ‘father’)\\
sadęni̱hó:nǫˀ\\
\gll s-adę-ni̱-hó:nǫˀ\\
\textsc{2s.p}-{\semireflexive}-father-{\populative}\\
\glt ‘your relatives on your father’s side’
\z


\subsubsection*{Different-generation kinship terms taking \textsc{a}-series prefixes} \label{ch:Different-generation kinship terms taking a series prefixes}
While different-generation kinship terms normally take interactive pronominal prefixes (\sectref{ch:Different-generations kinship terms, interactive pronominal prefixes}), they can instead take the \textsc{a}-series \stem{k-} \textsc{1s.a} prefix or the \textsc{a}-series \stem{ha-} \textsc{2ms.a} prefix. In such cases, \stem{k-} \textsc{1s.a} means ‘my maternal relative’ \xxref{ex:interkintermppex15}{ex:interkintermppex20} and \stem{ha-} \textsc{2ms.a} means ‘my paternal relative’ \xxref{ex:interkintermppex22}{ex:interkintermppex26}. Some of these same words can also be used as terms of address (see \sectref{ch:Terms of address}).

\ea\label{ex:interkintermppex15} \stem{k-nó:haˀ} ‘(my) mother’ (optionally also a term of address)\\
knó:haˀ \\
\gll k-nó:haˀ\\
 \exsc{1s.a}-mother\\
\glt ‘my mother’, `Mom'
\cfex{ǫkíno̱haˀ\\
	\gll ǫkí-no̱haˀ\\
	\exsc{3s.fi/3ns>1ns}-mother \\
	\glt `our mother' (from example \ref{ex:interkintermppex13})}
\z


\ea\label{ex:interkintermppex18} \stem{k-no:ha:ˀah} ‘(my) (maternal) aunt’ (optionally also a term of address)\\
kno:há:ˀah \\
\gll k-no:há:ˀah\\
 \exsc{1s.a}-aunt\\
\glt ‘my (maternal) aunt’, `Auntie'
\cfex{ǫkino̱há:ˀah\\
	\gll ǫki-no̱há:ˀah\\
	\exsc{3s.fi/3ns>1ns}-aunt \\
	\glt `our aunt' (from example \ref{ex:interkintermppex16})}
\z


\ea\label{ex:interkintermppex20} \stem{k-noˀsęh} ‘my (maternal) uncle’ (optionally also a term of address)\\
knóˀsęh\\
\gll k-nóˀsęh\\
 \exsc{1s.a}-uncle\\
\glt `my uncle'
\cfex{hǫwáno̱ˀsęh\\
	\gll hǫwá-no̱ˀsęh\\
	\exsc{3s.m/3s.fi>3s.m}-uncle \\
	\glt `his/her uncle' (from example \ref{ex:interkintermppex19})}
\z


\ea\label{ex:interkintermppex22} \stem{ha-ˀnih} ‘my father’ (optionally a term of address)\\
haˀnih \\
\gll ha-ˀnih\\
 \textsc{3s.m.a}-father\\
\glt ‘my father’, `Dad'
\cfex{hǫwáˀnih\\
	\gll hǫwá-ˀnih\\
	\exsc{3s.m/3s.fi>3s.m}-father \\
	\glt `his/her father' (from example \ref{ex:interkintermppex21})}
\z


\ea\label{ex:interkintermppex26} \stem{ha-ˀni:hah} ‘my godfather’ (optionally a term of address)\\
haˀní:hah\\
\gll ha-ˀní:hah\\
 \textsc{3s.m.a}-godfather\\
\glt `my godfather'
\cfex{hǫwaˀní:hah\\
	\gll hǫwa-ˀní:hah\\
	\exsc{3s.m/3s.fi>3s.m}-step.father \\
	\glt `her godfather/step-father' (from example \ref{ex:interkintermppex25})}
\z


Similarly, stems like \stem{hsot} ‘grandparent’ only refer to older-generation \textit{females} when \stem{k-} \textsc{1s.a} is used \xref{ex:interkintermppex12}. For such words, in order to refer to an older-generation male, an interactive prefix is used instead (see the “cf.” examples in \ref{ex:interkintermppex12}, as previously described in \sectref{ch:Different-generations kinship terms, interactive pronominal prefixes}).

\ea\label{ex:interkintermppex12} \stem{k-so:t} ‘grandmother’ (optionally also a term of address)\\
 kso:t \\
\gll k-hsot \\
\exsc{1s.a}-grand.parent \\
\glt ‘my grandmother’, ‘Grandma’ 
\cfex{hagéhso:t, hakso:t\\
	\gll hagé/hak-hso:t\\
	\textsc{3s.m>1s}-grand.parent \\
	\glt `my grandfather' (from example \ref{ex:interkintermppex10})}
\cfex{hehso:t\\
	\gll he-hso:t\\
	\textsc{1s>3s.m}-grand.parent\\
	\glt `Grandpa'}
\z

