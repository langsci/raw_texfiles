\chapter{Particle dictionary} \label{ch:particle dictionary}
\section{Particle order} \label{ch:particle order}
Many particles occur in fixed positions. For example, particles such as \cayuga{gęh} must appear after another word, but also close to the beginning of the clause that they occupy (\ref{ex:complsentex}, square brackets denote clause boundaries). Other particles, including \textbf{dęˀ} (also shown in \ref{ex:complsentex}) appear at the beginning of the clause. Yet others, including \cayuga{neˀ} occur before the word they modify (for example, before \cayuga{kso:t} in \ref{ex:complsentex}). Finally some particles can appear wherever relevant (see \sectref{Free-standing particles and groups}). Particle order is described in the following sections.


\ea
\label{ex:complsentex}
\gll [Sęnǫ́hdǫh \exemph{gęh}] [dęˀ niyǫgyeháˀ \exemph{neˀ} kso:t?]\\
you.know Q what what.she.is.doing the grandmother\\
\glt “Do you know what our grandma is doing?” (\cite{henry_de_2005})
\z

\subsection*{Initial particles and groups}
The following particles and groups occur before the verb whose meaning they modify, specifically at the beginning of independent clauses (\ref{ex:particleclauseorder8}, see \sectref{Independent and dependent clauses, relative clauses}.) 

\ea
\label{ex:particleclauseorder8}
\begin{multicols}{2}

\textit{A:we:tˀah} \trs{pretend to…}

\textit{A:yę:ˀ} \trs{it seems}, \trs{I guess}

\textit{Ahgwih} \trs{don’t}

\textit{Ahgwih…hwę:dǫh} \trs{don’t ever}

\textit{Ahgwih gwaˀ} \trs{don’t}

\textit{Awęˀ} \trs{it is said}

\textit{Da ne:ˀ} \trs{and}

\textit{Da: nę: dah} \trs{and now}

\textit{(da:) ne:ˀ hniˀ} \trs{also}, \trs{and}, \trs{too}

\textit{Da: (ne:ˀ) onęh} \trs{and now}

\textit{Ewa:dǫˀ} \trs{Yes, you may}, \trs{it is permissible}

\textit{Hę:-dah} ‘and’ (often used at the beginning of lines in speeches)

\textit{Gyę:gwaˀ hne:ˀ hwaˀ} \trs{maybe this time}, \trs{just maybe}

\textit{(g)yę:gwaˀ giˀ shęh hwaˀ} \trs{maybe this time}

\textit{Ne:ˀ giˀ} \trs{just}

\textit{Ne:ˀ giˀ gyę:ˀ ǫh} \trs{I guess}

\textit{Ne:ˀ giˀ …, … hniˀ} \trs{and that too}, \trs{and that also}

\textit{Ne:ˀ giˀ shęh hwaˀ} \trs{maybe this time}

\textit{Ne:ˀ giˀ tsǫ: gwahs} ‘that’s basically all’

\textit{Ne:ˀ he:gę:} \trs{just}, \trs{only}, \trs{all}

\textit{Ne:ˀ hęˀ hne:ˀ} \trs{also too}

\textit{Ne:ˀ hne:ˀ ne:ˀ} \trs{in fact, it is!}, \trs{It \emph{is}}

\textit{Ne:ˀ hniˀ ne:ˀ} \trs{and that also}

\textit{Ne:ˀ (neˀ)} \trs{it is}, \trs{that is}, \trs{that’s what}

\textit{Ne:ˀ nęh} \trs{it is when}

\textit{Ne:ˀ ǫh} \trs{I guess it is}

\textit{Ne:ˀ seˀ gyę:ˀ} \trs{you know}

\textit{Ne:ˀ tsǫ: shęh} \trs{but}

\textit{Neˀ gwa̱ˀ-toh} \trs{also}

\textit{Neˀ gyę:ˀ nę ne:ˀ} (emphasis)

\textit{Neˀ gyę:ˀ neˀ gyę:ˀ / negęˀnagęˀ} \trs{that is what}

\textit{Neˀ seˀ} ‘that’s just the one’, 'that’s just who’

\textit{Neˀ to gyę:ˀ} ‘that’s what’

\textit{(neˀ) to: … ne:ˀ} ‘that’s it’, 'that’s’

\textit{Neˀ toh} \trs{that is}

\textit{Neˀ (tsǫ:) gwaˀ toh} \trs{and also just this}

\textit{Ni:ˀ gyę:ˀ, ni:ˀ gę:ˀ} \trs{I did}

\textit{Tęˀ gyę:ˀ nę neˀ} \trs{what on earth?}, \trs{emphatically no}

\textit{Tęˀ hne:ˀ} \trs{definitely not}

\textit{Tęˀ hne:ˀ (neˀ)} \trs{not \emph{that one}}

\textit{Tęˀ to ne:ˀ} \trs{not really}

\textit{Tęˀǫ:, tęˀ ǫh} \trs{maybe not}

\textit{Toh-geh} \trs{and then}

\textit{To gę-ǫ / gaǫ (ne:ˀ), to ǫ} \trs{whether}, \trs{if}, \trs{I wonder if}

\textit{Waˀ gyęh} \trs{listen!}, \trs{excuse me}, \trs{would you…?}

\textit{Waˀ-jih} \trs{wait!}
\end{multicols}
\z

The following particles or groups occur at the beginning of special clause types -- either independent clauses (i), dependent clauses (d), direct questions (dq), or indirect questions (iq). The relevant clause types are listed after each particle in (\xref{ex:particleclauseorder1} (For clause types, see \sectref{Independent and dependent clauses, relative clauses}.)


\ea
\label{ex:particleclauseorder1}
particles that appear at the beginning of the clause
\begin{multicols}{2} 

\textit{Dęˀ … hoˀdęˀ ni-} ‘how’, 'in what way’ (d, iq, dq)

\textit{Dęˀ … ni:yoht} \trs{why} (i, d)

\textit{Dęˀ …(hoˀdęˀ)} \trs{what} (d, iq, dq)

\textit{Do: … ni-} ‘how much’, 'how many’ (d, iq, dq)

\textit{Do: niyowi̱hsda̱ˀe:ˀ} \trs{when}, `at what time’ (i, d)

\textit{Gaę … hǫ:weh} \trs{where, which place} (d, dq)

\textit{Gaoˀ … ni-} ‘less so, -er’ (i, d)

\textit{(gaoˀ) shęh niyo:weˀ} \trs{until, before} (d)

\textit{Gwahs heyohe:} \trs{the most, -est} (i, d)

\textit{(gwahs) shęh ni-} ‘as…as’ (i, d)

\textit{Gyę:gwaˀ a:-} ‘if’ (i, d)

\textit{Gyę:gwaˀ ta:-} ‘if not’ (i, d)

\textit{Hę:gyęh, hę:gyeh (shęh) …} \trs{no matter how much}, \trs{whether or not}, \trs{even if} (d)

\textit{Heyohe:ˀ} \trs{more, -er} (i, d)

\textit{Hne:ˀ (shęh)} \trs{because} (d)

\textit{Hwę:dǫh} \trs{when} (dq, iq)

\textit{I:nǫh … gaoˀ neˀ} \trs{far from}, \trs{inadequate amount} (i, d)

\textit{Ji … trehs … shęh} \trs{too much so for}, \trs{too much so} (i, d)

\textit{(ji) trehs} \trs{because} (d)

\textit{(neˀ) aǫhę:ˀęh d-} ‘the most, -est, -er (of)’ (i, d)

\textit{Neˀ gyaǫhę:ˀęh d-} ‘the most’, 'the greatest’ (i, d)

\textit{Ne:ˀ …} \trs{because}, \trs{it is} (d, i)

\textit{Ne:ˀ gwahs d-} ‘the most, -est’ (i, d)

\textit{Ne:ˀ … hǫ:niˀ / dagaihǫ:niˀ} \trs{why} (d, i)

\textit{Ne:ˀ hwaˀ} \trs{this (coming) time}, \trs{when} (d, i)

\textit{Nę:-gyęh hwaˀ} ‘this time’, 'when’ (d, i)

\textit{Nęh} \trs{when} (d)

\textit{Shęh} \trs{because} (d)

\textit{Shęh} \trs{that} (d)

\textit{Shęh hǫ:weh} \trs{the place where}, \trs{whereabouts} (d)

\textit{Shęh naˀonisheˀ / tsaˀonisheˀ / tsi-} ‘while’, 'when’ (d)

\textit{Shęh ni:yoht} \trs{how}, \trs{the manner in which} (d)

\textit{Shęh niyo:weˀ} \trs{as far as}, \trs{as much as} (i, d)

\textit{Shęh noh-geh} ‘even’ (emphasizing the unexpected) (i, d)

\textit{Sǫ: … (nˀaht/nˀoht)} \trs{who} (d, iq, dq )

\textit{Sǫ: go:węh} \trs{whose} (d, iq, dq)

\textit{Tęˀ gęh deˀ-} ‘isn’t it? ` (i, d)

\textit{To … ni-} ‘to that degree’ (i, d)

\textit{To niyo:weˀ} \trs{that far}, \trs{that much} (i, d)

\textit{Tohgeh … hǫ:weh} \trs{where} (dq)
\end{multicols}
\z

The particles and particle groups in \xref{ex:particleclauseorder10} appear directly before nouns or before verbs functioning as “nouns”. Alternatively, they are free-standing when they function as “pronouns” (see \sectref{ch:’Pronouns’ (words and affixes that function as pronouns)}).

\ea
\label{ex:particleclauseorder10}
\textit{Nę:} \trs{this}, \trs{these}

\textit{Gaę gwaˀ…ni-}\textsc{noun} \trs{wherever \textsc{noun} is}

\textit{Nę:-gyęh} \trs{this}, \trs{this one}

\textit{To:-gyęh} \trs{that}, \trs{that one}
\z

The following particles (proclitics) must precede another word \xref{ex:particleclauseorder6}. An example is shown in \xref{ex:particleclauseorder61}.

\ea
\label{ex:particleclauseorder6}
proclitic particles

\textit{Da:} \trs{and}, \trs{here}

\textit{E:} \trs{some direction}, \trs{towards}

\textit{Neˀ} \trs{the} (etc.)
\z

\ea
\label{ex:particleclauseorder61} proclitic particle placement\\
\gll \exemph{ę:} nahahá:dih\\
towards the.side.of.the.road\\
\glt ‘on the other side of the road’
\z

Particles functioning as “adverbs” tend to appear clause-initially, or before the verb whose meaning they modify. For these, see \sectref{ch:’Adverbs’ (words and affixes that function as adverbs)}.


\subsubsection*{Enclitic particles}

Enclitic particles must follow another word -- typically, the word or phrase whose meaning they modify. Simultaneously, these particles or groups are as close to the beginning of their clause as possible (but obviously, cannot be first). Example \xref{ex:particleclauseorder3} lists most of the enclitic particles and groups.

\ea
\label{ex:particleclauseorder3}

enclitic particles 

\begin{multicols}{2}


\textit{…dęˀ hniˀ} \trs{for sure}

\textit{… dęˀ ni:ˀ} \trs{me, for sure}

\textit{… deˀęgwaheh tęˀ seˀ} \trs{but then not really}

\textit{… diˀ} \trs{so}, \trs{then}

\textit{… e:ˀ} \trs{again}, \trs{still}

\textit{… ę:ˀ} \trs{yes indeed}, \trs{isn’t it?}

\textit{…(ga:t) giˀ-shęh} \trs{maybe}, \trs{or maybe}

\textit{…(ga:t) giˀ shęh tęˀ} \trs{maybe not}, \trs{or not}

\textit{… gę-ǫ} \trs{whether}, \trs{if}, \trs{I wonder if…?}

\textit{… gę:s} \trs{generally}, \trs{used to}, \trs{usually}, \trs{normally}

\textit{… gęh} \trs{Q}, \trs{whether}, \trs{if}, \trs{mind you}, \trs{didn’t I?}, \trs{is it?}

\textit{… gęh hne:ˀ} \trs{how about this one?}

\textit{… giˀ} \trs{just now}

\textit{… giˀ gyę:ˀ} \trs{quite}, \trs{kind of}, \trs{just do it!}

\textit{… giˀ hne:ˀ} \trs{but}, \trs{however}

\textit{… giˀ ne:ˀ} \trs{just}

\textit{… giˀ (tsǫ:)} \trs{just}, \trs{really}

\textit{… gwa:dih, gwai} \trs{to one side}

\textit{… gwahs} \trs{anyway}

\textit{… gwaˀ} \trs{immediately}, \trs{right then}, \trs{just then}, \trs{finally}, \trs{do it now!}

\textit{… gyę:ˀ} (emphasis)

\textit{… gyę:ˀ hne:ˀ} \trs{it was \emph{this one}, (not that one)}

\textit{… gyę:ˀ nę ne:ˀ} (emphasis)

\textit{… gyę:ˀ ǫh, … gęˀ-ǫh} \trs{maybe}, \trs{I guess}, \trs{I wonder}

\textit{… hęˀ} \trs{also}, \trs{too}

\textit{… hęˀ hne:ˀ} \trs{also}, \trs{too}

\textit{… hne:ˀ} \trs{in fact}, \trs{but}

\textit{… hniˀ} \trs{also}, \trs{and}, \trs{too}

\textit{… hwaˀ} \trs{this time}, \trs{next}

\textit{… hya:ˀ} \trs{first}, \trs{before anything else}

\textit{… (neˀ) ę:ˀ} \trs{isn’t it so?}, \trs{yes?}, \trs{no?}, \trs{innit?}

\textit{… ǫh ne:ˀ} \trs{maybe}

\textit{… ǫh, ǫ:} \trs{I guess}, \trs{I wonder (if)}

\textit{… (seˀ) hęˀ ni:ˀ} \trs{me too}
\end{multicols}
\z

\subsection*{Final particles and groups (tags)}
The following particles or particle groups tend to appear as tags at the end of utterances (see \sectref{Tag questions (asking for confirmation from the listener)}). They can also be independent utterances, as in \xref{ex:particleclauseorder4}.

\ea
\label{ex:particleclauseorder4}

\ea \textit{Do:gęhs} \trs{isn’t it true?}

\ex \textit{Do:gęhs ę:} \trs{isn’t it true?}

\ex \textit{Ehę:ˀ gęh} \trs{is that right?}

\ex \textit{Tęˀ gęh} \trs{isn’t that right?}
\z
\z 


\subsection*{Free-standing particles and groups}\label{Free-standing particles and groups}

The following particles and particle groups appear wherever relevant or independently \xref{ex:particleclauseorder7}. (An exclamation mark in the translations denotes either an exclamation or a command.)


\ea
\label{ex:particleclauseorder7}

\begin{multicols}{2}

\textit{A:wetˀahshǫ́:ˀǫh} \trs{it’s pretend}, \trs{it’s make-believe}

\textit{Agí:} \trs{ouch!} (expressing pain)

\textit{Ahgwih gwaˀ} \trs{don’t do it!}

\textit{Ahsdeh} \trs{get outside!}

\textit{Ahsǫh} \trs{more!} (asking for food or drink)

\textit{Ahsǫh gęh} \trs{do you want more?} (offering food or drink)

\textit{Ajú:} \trs{yikes!} (said in response to an unpleasant sensation)

\textit{Aweˀ hęgyeh tsǫ:, awęˀ hęgyeh tsǫ:} \trs{let it go}

\textit{Da: neˀ toh} \trs{that’s all}

\textit{Daji:hah (gwaˀ)} \trs{soon}, \trs{in a short while}

\textit{Do: i:ˀ} \trs{let me!}, \trs{how about me?}

\textit{Dó:gaˀ} \trs{I don’t know}

\textit{Dó:gęhs} \trs{exactly}, \trs{for sure}

\textit{Do:gęhs diˀ gęh} \trs{isn’t it true?}

\textit{Do:s giˀ} \trs{just so}, \trs{indeed}

\textit{Ehęˀ} \trs{yes}

\textit{Ehęˀ ę:ˀ} \trs{yes indeed}

\textit{Ehęˀ gyę:ˀ} \trs{yes indeed}

\textit{Ehęˀ seˀ} \trs{yes indeed}

\textit{Ewa:dǫˀ gęh} \trs{may I?}, \trs{may we?}

\textit{Ewa:dǫˀ giˀ-shęh} \trs{maybe}, \trs{possibly}

\textit{Gwahs ǫ:węh seˀ} \trs{yes indeed}

\textit{Gwaˀ ti:gę:} \trs{plainly}, \trs{clearly}, \trs{as it is}

\textit{Gwa̱ˀ-toh} \trs{exactly that}

\textit{Gwé:} \trs{well!}, \trs{hello}

\textit{Háiˀ, háeˀ} \trs{hi} (a word attributed to Oneida or Tutelo)

\textit{Hanyoh, hanyohanyoh} \trs{do it!}, \trs{come on}

\textit{Haoˀ} \trs{come on}, \trs{o.k.}

\textit{Haoˀ dęˀ nyoh} \trs{o.k. then.}

\textit{Haoˀ diˀ sah} \trs{alright}, \trs{o.k.}

\textit{Hę:gyeh (tsǫ:)} \trs{no matter}, \trs{not likely}, \trs{not for long}, \trs{never mind}

\textit{Hę:gyeh giˀ} \trs{leave well enough alone}

\textit{Hęˀęh} \trs{no}

\textit{Hoh, ho:} \trs{what the…?}

\textit{Hoho:} \trs{aha!}, \trs{oh no!}

\textit{Hotgǫˀǫh, otgǫˀ} \trs{what the…?}, \trs{for heaven’s sake!}

\textit{Hoˀdęˀ} \trs{kind (of thing)}

\textit{I:ˀ hne:ˀ} \trs{I am}

\textit{I:ˀ hya:ˀ} \trs{me first}

\textit{I:ˀ seˀ} \trs{I am}

\textit{Nę:} \trs{look!}, \trs{say,…}

\textit{Nę:-gyęh hwaˀ} \trs{this here}

\textit{Nę:-dah} \trs{here, take this}

\textit{Nę:-dah} \trs{this way}

\textit{Nę: diˀ ni:s/i:ˀ} \trs{how about you/me?}

\textit{Nę: ne:ˀ i:s/i:ˀ} \trs{how about you/me?}

\textit{Nę: toh} \trs{here}, \trs{this many}

\textit{Ne:ˀ} \trs{yes indeed}

\textit{(Neˀ) gwa̱ˀ-toh} \trs{here} (rather than there)

\textit{None:ˀ, nǫne:ˀ} \trs{mind you}, \trs{you know}

\textit{Nyá:węh} \trs{thanks}

\textit{Nyoh} \trs{you’re welcome}, \trs{alright}, \trs{o.k.}

\textit{O: gęh?} \trs{really?}

\textit{O: tęˀ ǫh … (gat) giˀ-shęh gyę:gwaˀ/ne:ˀ hwaˀ} \trs{maybe, maybe not}

\textit{O:, ó:ò:} \trs{oh!}

\textit{Oˀ-ǫ:, o: ǫh} \trs{oh really?}

\textit{Sgę:nǫ:ˀǫ́h} \trs{slowly}

\textit{Sgę́:nǫˀ} \trs{hello}

\textit{Si … gwa:dih} \trs{move it!} (said to a dog)

\textit{Tęˀ} \trs{no}

\textit{Tęˀ dedó:gęhs} \trs{not really}, \trs{it isn’t true}

\textit{Tęˀ giˀ ni:ˀ/ni:s} \trs{no, me/you!}

\textit{Tęˀ giˀ shęh hwaˀ (daǫ:ˀ)} \trs{maybe not}

\textit{Tęˀ (gwahs) ǫ:weh} \trs{not really}

\textit{Tęˀ gyę:ˀ ǫh, tęˀ gęˀǫh} \trs{not really}

\textit{Tęˀ i:ˀ} \trs{not me}

\textit{Tęˀ ne:ˀ de̱ˀgę:} \trs{it isn’t the one}

\textit{Tęˀ seˀ} \trs{but then, not really}

\textit{To tsǫ:} \trs{that’s enough}, \trs{that’s all for now}

\textit{Trehs giˀ gyęˀ} \trs{my goodness!}, \trs{too bad!}, \trs{that’s amazing!}

\textit{Tsę́:} \trs{oh my!}

\textit{Waˀ-jih hya:ˀ} \trs{wait a minute}, \trs{wait a while}, \trs{wait!}

\end{multicols}

\z



Particles or particle groups functioning as “pronouns” are also independent, occurring wherever nouns can (see \sectref{ch:’Pronouns’ (words and affixes that function as pronouns)}). The word order for nouns and pronouns was described in the section on word order effects (see \sectref{ch:Word order effects}).

\section{A particles}

\subsection*{\textbf{Aga:węh} \trs{it's mine}, \trs{it's ours}, \trs{my}, \trs{our}} \label{p:[aga:węh]}

\CayugaRemark{Verb functioning as a “possessive pronoun” (\sectref{ch:’Possessive pronouns’}); free-standing.}

\ea
\label{ex:apart1}
agá:węh\\
\gll ag-á:w-ęh\\
\textsc{1s.p}-own-{\stative}\\
\glt ‘mine’, ‘I own it’
\z



\subsection*{\textbf{Agi:} \trs{ouch!}} \label{p:[agi:]}
\CayugaRemark{Particle functioning as an “exclamation” (\sectref{ch:Exclamations}), in response to pain.}

\ea
\label{ex:apart17}
\gll “\textbf{Agí:!}” agyohé:t onę́h awádegoˀ.\\
ow she.cried now she.ran.away\\
\glt “Ow!” she cried and ran away. (\cite{keye_circle_2016}, Circle Book 10, The Magic Chair)
\z


\subsection*{\textbf{Ahgwih gwaˀ} \trs{don’t!}} \label{p:[ahgwih gwaˀ]}
\CayugaRemark{Particle group functioning as an “adverb of negation” (\sectref{ch:Negative ‘adverbs’}, \sectref{Negative commands with [ahgwih] ‘don’t’}); emphatic; clause-initial.}

\ea
\label{ex:apart2}
\gll Snigǫ:há:k giˀ gyę́:ˀ shę nyó: hętse:ˀ. \textbf{ Ahgwíh} \textbf{gwaˀ} ęjisa̱hnǫhnyáˀk é:ˀ!\\
watch.out just emphasis that place you.will.go do.not emphasis you.will.be.hurt again\\
\glt ‘Watch out as you go. Don’t get hurt again!’ (\cite[386]{mithun_watewayestanih_1984}, Ga̱há:gǫ: dialogue)
\z


\subsection*{\textbf{Ahgwih hwaˀ} \trs{don’t!}} \label{p:[ahgwih hwaˀ]}
\CayugaRemark{Particle group functioning as an “adverb of negation” (\sectref{ch:Negative ‘adverbs’}, \sectref{Negative commands with [ahgwih] ‘don’t’}); emphatic; clause-initial.}

\ea
\label{ex:apart4}
\gll \textbf{Ahgwih} \textbf{hwaˀ} a:satró:wih.\\
do.not this.time you.should.tell\\
\glt ‘You shouldn’t tell!’
\z


\subsection*{\textbf{Ahgwih hwę:dǫh} \trs{don’t ever}} \label{p:[ahgwih hwę:dǫh]}
\CayugaRemark{Particle group \textit{ahgwih … hwę:dǫh} functions as an adverb of negation (\sectref{ch:Negative ‘adverbs’}); emphatic; clause-initial.}

\ea
\label{ex:apart3}
\gll “Do:gę́hs oyoˀtí:yeht,” aˀa:gę́ˀ, “\textbf{ahgwíh} ni:s \textbf{hwę:dǫ́h} ęhsnihs tó:gyęh.\\
it.is.true it.is.sharp she.said don’t you ever you.will.use.it that\\
\glt ‘“It is really sharp,” she said, “don’t you ever use it.”’ (\cite{henry_de_2005})
\z


\subsection*{\textbf{Ahsdeh} \trs{outside}, \trs{outdoors}} \label{p:[ahsdeh]}
\CayugaRemark{Particle functioning as an “adverb of place” (\sectref{‘Adverbs’ of place}); clause-initial.}

\ea
\label{ex:apart9}
\gll Eksaˀshǫ:ˀǫ́h hniˀ ǫ:nęh tęˀ \textbf{ahsdeh} de̱ˀjǫ́tga̱hnyeh.\\
children and now not outside they.didn’t.play\\
\glt ‘The children no longer played in the woods.’ (Carrier et al., 2013)
\z

\ea
\label{ex:apart10}
\gll “O: tsé:,” aˀa:gę́ˀ, “\textbf{ahsdéh} jatgáhnye̱hah!\\
oh oh.my she.said outside you.two.play\\
\glt ‘“Oh my,” she said, “go play outside!”’ (Henry, 2005)
\z


\subsection*{\textbf{Ahsǫh} \trs{still}, \trs{yet}} \label{p:[ahsǫh] ‘still’, ‘yet’}
\CayugaRemark{Particle functioning as an “adverb of time” (\sectref{ch:’Adverbs’ of time}); clause-initial.}

\ea
\label{ex:apart5}
\gll \textbf{Ahsǫ́h} ne:ˀ honákwę̱ˀǫh.\\
still it.is he.is.angry\\
\glt ‘he is still angry.’
\z

\ea
\label{ex:apart6}
\gll \textbf{Ahsǫ́h} hodrę́:nǫ:t.\\
still he.is.singing\\
\glt ‘he is still singing.’
\z

\begin{CayugaRelated}
\item \textit{Ahsǫh} \trs{more}, p. \pageref{p:[ahsǫh] ‘more’}\\
\item \textit{Ahsǫh gęh} \trs{Do you want some more?}, p. \pageref{p:[ahsǫh gęh]}\\
\item \textit{Ahsǫh} \trs{more} (asking food or drink), p. \pageref{p:[ahsǫh] ‘more’ (asking food or drink)}
\end{CayugaRelated}

\subsection*{\textbf{Ahsǫh} \trs{more}} \label{p:[ahsǫh] ‘more’}
\CayugaRemark{Particle functioning as an “adverb of degree” (\sectref{ch:’Adverbs’ of degree}); clause-initial.}

\ea
\label{ex:apart7}
\gll \textbf{Ahsǫ́h} e:ˀ nihs desatwęjó:nih!\\
more again you you.want.it\\
\glt ‘You still want more!’
\z

\begin{CayugaRelated}
\item \textit{Ahsǫh gęh} \trs{Do you want some more?}, p. \pageref{p:[ahsǫh] ‘more’}\\
\item \textit{Ahsǫh} \trs{more} (asking food or drink), p. \pageref{p:[ahsǫh] ‘more’ (asking food or drink)}
\end{CayugaRelated}


\subsection*{\textbf{Ahsǫh} \trs{more} (asking food or drink)} \label{p:[ahsǫh] ‘more’ (asking food or drink)}
\CayugaRemark{Particle; free-standing; used when asking for more food or drink.}

\begin{CayugaRelated}
\item \textit{Ahsǫh} \trs{still}, \trs{yet}, p. \pageref{p:[ahsǫh] ‘still’, ‘yet’}\\
\item \textit{Ahsǫh gęh} \trs{Do you want some more?}, p. \pageref{p:[ahsǫh] ‘more’}
\end{CayugaRelated}


\subsection*{\textbf{Ahsǫh gęh} \trs{Do you want some more?}} \label{p:[ahsǫh gęh]}
\CayugaRemark{Particle group; a question asked when serving food or drink.}

\ea
\label{ex:apart8}
\gll Ahsǫh gęh?\\
more Q\\
\glt ‘Do you want some more?’
\z

\begin{CayugaRelated}
\item \textit{Ahsǫh} \trs{still}, \trs{yet}, p. \pageref{p:[ahsǫh] ‘still’, ‘yet’}\\
\item \textit{Ahsǫh} \trs{more} (asking food or drink), p. \pageref{p:[ahsǫh] ‘more’ (asking food or drink)}
\end{CayugaRelated}

\subsection*{\textbf{Aju:} \trs{yikes!}} \label{p:[aju:]}
\CayugaRemark{Particle functioning as an “exclamation” (\sectref{ch:Exclamations}), in response to an unpleasant sensation, such as being splashed with cold water, ice, or snow, etc.}


\subsection*{\textbf{Akda:gyeˀ} \trs{beside}, \trs{the edge}} \label{p:[akda:gyeˀ]}
\CayugaRemark{Particle functioning as an “adverb of place” (\sectref{‘Adverbs’ of place}), clause-initial \xref{ex:apart11}; can also be a suffix, with a similar meaning (\ref{ex:apart12}, also see \sectref{[-kdagyeˀ]}).}

\ea
\label{ex:apart11}
\gll \textbf{Akda:gyéˀ} hé:ye̱:t.\\
the.edge she.stands.there\\
\glt ‘She is standing on the edge.’
\z

\ea\label{ex:apart12} 
\ea ohah\exemph{akdá:gyeˀ} `along the edge of the road’
\ex ganyad\exemph{akdá:gyeˀ} `along the lakeside or shoreline’
\z
\z 


\subsection*{\textbf{Aǫgohdǫh} \trs{exceptional}, \trs{over the top}, \trs{extremely}, \trs{too much so}} \label{p:[aǫgohdǫh]}
\CayugaRemark{Verb functioning as an “adverb of degree” (\sectref{ch:’Adverbs’ of degree}); clause-initial.}

\ea
\label{ex:apart13}
\gll \textbf{Aǫgohdǫ́h} oné:nǫˀ.\\
it.is.surpassed it.is.warm/hot.weather\\
\glt ‘It is exceptionally hot weather.’
\z

\begin{CayugaRelated}
\item \textit{Ji aǫgo̱hdǫh} \trs{too much so}, p. \pageref{p:[ji aǫgohdǫh]}
\end{CayugaRelated}

\subsection*{\textbf{Aǫhęˀ, Aǫhaˀ} \trs{it}} \label{p:[aǫhęˀ]}
\CayugaRemark{Verb functioning as an “emphatic pronoun” (\sectref{ch:’Emphatic pronouns’}); free-standing.}

\ea
\label{ex:apart14}
áǫhęˀ\\
\gll á-ǫhęˀ\\
\textsc{3s.p}-alone.{\stative}\\
\glt ‘it is alone’
\z

\begin{CayugaRelated}
\item \textit{Neˀ aǫhę:ˀęh d-stative.verb … (shęh gaoˀ ni-stative.verb)} \trs{the most, -est, -er (of)}, p. \pageref{p:[neˀ aǫhę:ˀęh d-]}\\
\item \textit{Neˀ gyaǫhę:ˀęh d-stative-verb… (shęh ni-stative.verb)} \trs{the most}, \trs{the greatest}, p. \pageref{p:[neˀ gyaǫhę:ˀęh d-]}
\end{CayugaRelated}

\subsection*{\textbf{Aweˀ hę:gyeh tsǫ:, Awęˀ hę:gyeh tsǫ:} \trs{let it go}} \label{p:[aweˀ hę:gyeh tsǫ:]}
\CayugaRemark{Verb and particle group; free-standing expression functioning as a “command” (\sectref{Particles and particle groups used with commands}).}

\ea
\label{ex:apart15}
\gll Aweˀ hę:gyeh tsǫ:.\\
it.is.said no.matter just\\
\glt ‘Let it go!’ `Never mind that!’
\z

\begin{CayugaRelated}
\item \textit{Hę:-gyęh, Hę:-gyeh} \trs{no matter}, p. \pageref{p:[hę:-gyęh]}\\
\item \textit{Hę:-gyeh gaę hǫ:weh} \trs{no matter where}, \trs{no matter which place}, p. \pageref{p:[hę:-gyeh gaę hǫ:weh]}\\
\item \textit{Hę:-gyeh giˀ} \trs{leave well enough alone}, p. \pageref{p:[hę:-gyeh giˀ]}\\
\item \textit{Hę:-gyęh shęh} \trs{no matter how much}, \trs{whether or not}, \trs{even if}, p. \pageref{p:[hę:-gyęh shęh]}\\
\item \textit{Hę:-gyęh tsǫ:} \trs{it doesn’t matter}, \trs{never mind}, p. \pageref{p:[hę:-gyęh tsǫ:]}\\
\item \textit{To hę:gyeh shęh} \trs{really}, p. \pageref{p:[to hę:gyeh shęh]}
\end{CayugaRelated}

\subsection*{\textbf{A:we:tˀah} \trs{it is pretend}, \trs{it's implied}} \label{p:[a:we:tˀah]}
\CayugaRemark{Verb, used here as a command (\sectref{Particles and particle groups used with commands}); appears before another clause or verb.}

\ea
\label{ex:apart16}
\gll \textbf{A:we:tˀáh} tsǫ: dęsáˀtsǫhs.\\
it.is.pretend just you.will.sneeze\\
\glt ‘Pretend to sneeze!’
\z



\subsection*{\textbf{A:yę:ˀ} \trs{I guess}, \trs{it seems}} \label{p:[a:yę:ˀ]}
\CayugaRemark{Verb functioning as an “evidential marker” (\sectref{ch:Evidential markers}); appears before another clause or verb.}

\ea
\label{ex:apart18}
\gll Gwé:, \textbf{a:yéˀ} sanǫ́hnya̱ˀgǫh.\\
well, it.seems you.are.hurt\\
\glt ‘Well, it looks like you are hurt.’
\z


\section{D particles}

\subsection*{\textbf{Da:} \trs{and}} \label{p:[da:]}
\CayugaRemark{Particle or sentence connector (\cite[189]{foster_earth_1974}), signaling the continuation of a previous topic (\sectref{ch:Topic continuation markers}); clause-initial.}

\begin{CayugaRelated}
\item \textit{Da: gwa:dih} \trs{over here}, \trs{this side}, p. \pageref{p:[da: gwa:dih]}\\
\item \textit{Da: hǫ:weh hǫ:} \trs{this is where}, p. \pageref{p:[da: hǫ:weh hǫ:]}\\
\item \textit{Da: ne:ˀ, da: ne:ˀ hniˀ} \trs{too}, \trs{also}, \trs{and}, p. \pageref{p:[da: ne:ˀ, da: ne:ˀ hniˀ]}\\
\item \textit{Da: ne:ˀ onęh} \trs{and now}, p. \pageref{p:[da: ne:ˀ onęh]}\\
\item \textit{Da: neˀ toh} \trs{that’s all}, p. \pageref{p:[da: neˀ toh]}\\
\item \textit{Da: nę: dah} \trs{and now}; p. \pageref{p:[da: nę: dah]}\\
\item \textit{Nę:-dah} \trs{this}, \trs{this way}, p. \pageref{p:[nę:-dah] `this’, ‘this way’}\\
\item \textit{Nę:-dah} \trs{here, take this}, p. \pageref{p:[nę:-dah] ‘Here!’}
\end{CayugaRelated}

\subsection*{\textbf{Da: gwa:dih} \trs{over here}, \trs{this side}} \label{p:[da: gwa:dih]}
\CayugaRemark{Particle and atypical verb \textit{da:…gwa:dih} functions as an “adverb of place” (\sectref{‘Adverbs’ of place}); clause-initial; \textit{gwa:dih} can also be spelled or pronounced as \textit{gwai}.}

\ea
\label{ex:dpart71}
\gll Sǫ:, di̱ˀ  hne:ˀ nˀáht nę:gyę́h \textbf{da-gwa:díh} gaet?\\
Who, so  in.fact person this.one over.here someone.is.standing\\
\glt ‘So then who is this standing over here?’ (\cite[257]{mithun_watewayestanih_1984}, Ga̱hwajiyá:deˀ dialogue)
\z

\ea
\label{ex:dpart72}
\gll \textbf{Da:} giˀ gyę:ˀ \textbf{gwa:díh} wanaháotrahǫˀ.\\
here just this.one side hats.are.lying\\
\glt ‘Over here are the hats.’ (\cite[225]{mithun_watewayestanih_1984}, Agyaˀdawíˀtraˀ dialogue)
\z

\begin{CayugaRelated}
\item \textit{Da:} \trs{and}, p. \pageref{p:[da:]}\\
\item \textit{Gwa:dih, Gwai} \trs{to one side}, p. \pageref{p:[gwa:dih]}
\end{CayugaRelated}

\subsection*{\textbf{Da: hǫ:weh hǫ:} \trs{this is where}} \label{p:[da: hǫ:weh hǫ:]}

\CayugaRemark{Particle group \textit{da:…hǫ:weh (hǫ:)} functions as an “adverb of place” (\sectref{‘Adverbs’ of place}); clause-initial; \textit{hǫ:weh} can be spelled or pronounced as \textit{hǫ:}, \textit{nhǫ:}, \textit{hǫ:weh}, \textit{nhǫ:weh}.}

\ea
\label{ex:dpart1}
\gll \textbf{Da:} \textbf{nhǫ:weh} \textbf{hǫ:} hęˀdrǫˀ.\\
here place place he.lives.here\\
\glt ‘This is where he lives.’
\z

\begin{CayugaRelated}
\item \textit{Da:} \trs{and}, p. \pageref{p:[da:]}\\
\item \textit{Hǫ:weh} \trs{where}, p. \pageref{p:[hǫ:weh]}
\end{CayugaRelated}



\subsection*{\textbf{Da: ne:ˀ, Da: ne:ˀ hniˀ} \trs{too}, \trs{also}, \trs{and}} \label{p:[da: ne:ˀ, da: ne:ˀ hniˀ]}
\CayugaRemark{Particle group \textit{da: ne:ˀ (hniˀ)} signals the continuation of a previous topic and the addition of new information (\sectref{ch:Topic continuation markers}); clause-initial.} 

\ea
\label{ex:dpart5}
\gll \textbf{Da} \textbf{ne:ˀ} \textbf{hniˀ} dwę́:dǫh. \\ 
and it.is also we.all.mean.it\\
\glt ‘That is also what we mean.’
\z

\begin{CayugaRelated}
\item \textit{Da:} \trs{and}, p. \pageref{p:[da:]}\\
\item \textit{Hniˀ} \trs{and}, p. \pageref{p:[hniˀ] ‘and’}\\
\item \textit{Ne:ˀ} \trs{just so}, \trs{indeed}, p. \pageref{p:[ne:ˀ] ‘just so’}
\end{CayugaRelated}

\subsection*{\textbf{Da: ne:ˀ onęh} \trs{and now}} \label{p:[da: ne:ˀ onęh]}
\CayugaRemark{Particle group, \textit{da: (ne:ˀ) onęh} is used at the beginning of a performance (a speech); clause-initial.}

\ea
\label{ex:dpart2}
\gll \textbf{Da} \textbf{ne:ˀ} \textbf{onę́h} toh niyáwę̱hdreˀ.\\
and it.is now that.one what.is.going.to.happen\\
\glt ‘and now this is what is going to happen.’
\z

\begin{CayugaRelated}
\item \textit{Da:} \trs{and}, p. \pageref{p:[da:]}\\
\item \textit{Ne:ˀ} \trs{just so}, \trs{indeed}, p. \pageref{p:[ne:ˀ] ‘just so’}
\end{CayugaRelated}



\subsection*{\textbf{Da: neˀ toh} \trs{that’s all}} \label{p:[da: neˀ toh]}
\CayugaRemark{Particle group and free-standing expression signaling the end of a topic (\sectref{ch:Topic starters and conclusions}); often said at the end of a speech.}

\ea
\label{ex:dpart4}
\gll Da neˀ toh.\\
and the that\\
\glt ‘That’s all.’
\z

\begin{CayugaRelated}
\item \textit{Da:} \trs{and}, p. \pageref{p:[da:]}\\
\item \textit{Da: neˀ toh} \trs{that’s all}, p. \pageref{p:[da: neˀ toh]}\\
\item \textit{Ne:ˀ} \trs{just so}, \trs{indeed}, p. \pageref{p:[ne:ˀ] ‘just so’}\\
\item \textit{To:}, \textit{Toh}, \textit{Tǫ:} \trs{that one}, p. \pageref{p:[to:] ‘that one’}
\end{CayugaRelated}


\subsection*{\textbf{Da: nę: dah} \trs{and now}} \label{p:[da: nę: dah]}
\CayugaRemark{Particle group, signaling the beginning of a speech, etc.; clause-initial.}

\ea
\label{ex:dpart3}
\gll \textbf{Da:} \textbf{nę:} \textbf{dah} haˀgahéˀ gaoˀ ędihswatríhs ęhswada̱hǫhsí:yohs. \\ 
and this.one and it.is.time closer you.will.bump.together you.will.listen\\
\glt ‘Now is the time to come closer and listen.’
\z

\begin{CayugaRelated}
\item \textit{Da:} \trs{and}, p. \pageref{p:[da:]}\\
\item \textit{Da: nę: dah} \trs{and now}, p. \pageref{p:[da: nę: dah]}\\
\item \textit{Nę:} \trs{this}, \trs{these}, p. \pageref{p:[nę:] ‘this, these’}\\
\item \textit{Nę:-dah} \trs{this}, \trs{this way}, p. \pageref{p:[nę:-dah] `this’, ‘this way’}
\end{CayugaRelated}


\subsection*{\textbf{Daji:hah gwaˀ} \trs{soon}, \trs{a short while}} \label{p:[daji:hah gwaˀ]}
\CayugaRemark{Particle group \textit{daji:hah (gwaˀ)} functions as an “adverb of time” (\sectref{ch:’Adverbs’ of time}); clause-initial.}

\ea
\label{ex:dpart6}
\gll \textbf{Daji:hah} \textbf{gwaˀ} giˀ tsǫ: o:nęh sagaǫdagaidá:t neˀ háǫgwe̱ˀdaˀ. \\
early right.then just only now they.were.feeling.better.again the his.people\\
\glt ‘Soon his people were starting to feel better.’ (\cite{carrier_legends_2013})
\z

\ea
\label{ex:dpart7}
\gll Toh giˀ hęˀ hne:ˀ haˀgaeyǫ́ˀ, haˀgaǫgyǫˀsé:ˀ \textbf{dají:hah}.\\
there just also in.fact they.arrived.there, they.visited.there a.while\\
\glt ‘Also when they arrived, they visited for a little while.’ (\cite{henry_de_2005})
\z

\ea
\label{ex:dpart8}
\gll A:yę:́ˀ \textbf{dají:hah} tsǫ: o:nę́h adekwa̱hasraˀgéh aˀehę́ˀ…\\
it.seemed a.little.while just now on.the.table she.placed.it\\
\glt ‘It seemed only a little while before she put it (bread) on the table…’ (\cite{henry_de_2005})
\z

\ea
\label{ex:dpart9}
\gll Aˀa:gę́ˀ, “Waˀgyę́h de̱hsni:yǫ́h \textbf{dají:hah} o:nę́h!” \\
she.said wait, you.two.come.in a.little.while now\\
\glt ‘She said, “It is time for you to come in for a little while!”’ (\cite{henry_de_2005})
\z

\begin{CayugaRelated}
\item \textit{Gwaˀ} \trs{immediately}, \trs{right then}, \trs{just then}, \trs{finally}, p. \pageref{p:[gwaˀ] ‘immediately’}
\end{CayugaRelated}

\subsection*{\textbf{Deˀę: gwaˀ-heh tęˀ seˀ} \trs{but then not really}} \label{p:[deˀę: gwaˀ-heh tęˀ seˀ]}
\CayugaRemark{Particle group; tag (\sectref{ch:Clauses with conjunctions}).}

\ea
\label{ex:dpart29}
\gll deˀ-ę: gwaˀ-heh tęˀ seˀ\\
not-it.is.so intensifier-heh not you.know\\
\glt ‘but then, not really’
\z

\ea
\label{ex:dpart30}
\gll I:wíˀ gę́:s agatganǫ́:niˀ, \textbf{deˀę:-gwahéh tęˀ seˀ}. \\
I.want usually I.am.wealthy, it.is.not-just.then not you.know\\
\glt ‘I want to be wealthy, but then not really.’
\z

\begin{CayugaRelated}
\item \textit{E:ˀ}, \textit{Neˀ ę:ˀ} \trs{isn’t it so?}, \trs{yes?}, \trs{no?}, \trs{innit?}, p. \pageref{p:[ę:ˀ], [neˀ ę:ˀ] `isn’t it so?’}\\
\item \textit{Gwaˀ} \trs{immediately}, \trs{right then}, \trs{just then}, \trs{finally}, p. \pageref{p:[gwaˀ] ‘immediately’}\\
\item \stem{-heh} \trs{element (related to time)}, p. \pageref{p:[-heh]}\\
\item\textit{Seˀ} \trs{you know}, p. \pageref{p:[seˀ]}\\
\item\textit{Tęˀ} \trs{no}, \trs{not}, p. \pageref{p:[tęˀ]}
\end{CayugaRelated}

\subsection*{\textbf{Dęˀ} \trs{what}, \trs{how}} \label{p:[dęˀ-]}
\CayugaRemark{Particle modifying nouns, or words functioning as nouns (\sectref{ch:Clauses with [dęˀ … (hoˀdęˀ)] ‘what’}); clause-initial or phrase-initial.}


\begin{CayugaRelated}
\item \textit{Dęˀ} \trs{what}, \trs{how}, p. \pageref{p:[dęˀ-]}\\
\item \textit{Dęˀ, dęˀ hoˀdęˀ} \trs{what}, p. \pageref{p:[dęˀ, dęˀ hoˀdęˀ]}\\
\item \textit{Dęˀ gwaˀ hoˀdęˀ} \trs{whatever one (of several)}, p. \pageref{p:[dęˀ gwaˀ hoˀdęˀ]}\\
\item \textit{Dęˀ hne:ˀ hoˀdęˀ} \trs{what} (emphatic), p. \pageref{p:[dęˀ hne:ˀ hoˀdęˀ]}\\
\item \textit{Dęˀ hniˀ} \trs{for sure}, p. \pageref{p:[dęˀ hniˀ]}\\
\item \textit{Dęˀ hoˀdęˀ ni-} \trs{how}, \trs{what way}, p. \pageref{p:[dęˀ hoˀdęˀ ni-]}\\
\item \textit{Dęˀ ni:ˀ} \trs{I am for sure}, p. \pageref{p:[dęˀ ni:ˀ]}\\
\item \textit{Dęˀ ni:yoht shęh} \trs{why?}, p. \pageref{p:[dęˀ ni:yoht shęh]}\\
\item \textit{Dęˀ ǫh hne:ˀ hoˀdęˀ} \trs{what on earth?}, `I wonder what?’, p. \pageref{p:[dęˀ ǫh hne:ˀ hoˀdęˀ]}\\
\item \textit{Haoˀ dęˀ nyoh} \trs{o.k. then}, p. \pageref{p:[haoˀ dęˀ nyoh]}
\end{CayugaRelated}


\subsection*{\textbf{Dęˀ, Dęˀ hoˀdęˀ} \trs{what}} \label{p:[dęˀ, dęˀ hoˀdęˀ]}
\CayugaRemark{Particle group \textbf{dęˀ …(hoˀdęˀ)} functions as an indefinite (\sectref{ch:’Indefinite pronouns’}) or interrogative (\sectref{ch:’Interrogative pronouns’}) pronoun; clause-initial in direct questions \xxref{ex:dpart33}{ex:dpart36}, indirect questions (not shown), and dependent clauses \xref{ex:dpart37}, also see \sectref{ch:Clauses with [dęˀ … (hoˀdęˀ)] ‘what’}.}

\ea
\label{ex:dpart33}
\gll \textbf{Dęˀ} ni:s \textbf{hoˀdę́ˀ} sniya:sǫh? \\
what you kind you.are.called\\
\glt ‘What are you two called?’
\z

\ea
\label{ex:dpart34}
\gll \textbf{Dęˀ} hniˀ \textbf{hoˀdę́ˀ} hoihoˀdęhsro̱ˀdę:? \\
what and kind his.kind.of.work\\
\glt ‘What does he do?’
\z

\ea
\label{ex:dpart35}
\gll \textbf{Dęˀ} ní:s ętsahtgaˀ? \\
what you you.will.give.up \\
\glt ‘What will you give, donate?’
\z

\ea
\label{ex:dpart36}
\gll \textbf{Dęˀ} ní:s sa̱ˀnigǫ̱hoˀdę:? \\
what you your.kind.of.thoughts\\
\glt ‘What are your thoughts?’
\z

\ea
\label{ex:dpart37}
\gll Gwi:déh honǫhdǫ́ˀ [\textbf{dęˀ} \textbf{hoˀdę́ˀ} gáęnǫ̱hweˀs]. \\
Peter he.knows what kind they.like.the.taste.of.it \\
\glt ‘Peter knows what they like.’
\z

\begin{CayugaRelated}
\item \textit{Dęˀ} \trs{what}, \trs{how}, p. \pageref{p:[dęˀ-]}\\
\item \textit{Hoˀdęˀ} \trs{kind}, p. \pageref{p:[hoˀdęˀ]}
\end{CayugaRelated}

\subsection*{\textbf{Dęˀ gwaˀ hoˀdęˀ} \trs{whatever}} \label{p:[dęˀ gwaˀ hoˀdęˀ]}
\CayugaRemark{Particle group functioning as an “indefinite pronoun” (\sectref{ch:’Indefinite pronouns’}); free-standing.}

\ea
\label{ex:dpart10}
\gll Tę́ˀ de̱ˀago̱hsdǫ́:ˀ neˀ \textbf{dęˀ} \textbf{gwaˀ} \textbf{hóˀdęˀ}…\\
not she.didn’t.use.it the what right.then kind\\
`She never used whatever (it was called)…’ (\cite{henry_de_2005})
\z

\ea
\label{ex:dpart11}
\gll “Neˀ gę:s he:gę́: ohneˀdra̱ˀgéh toh gę:s a:yę́:ˀ gashá:s \textbf{dęˀ} \textbf{gwaˀ} \textbf{hóˀdęˀ}.”\\
the usually it.exists.there on.the.ground there usually it.seems it.looks.for what right.then kind\\
\glt ‘It usually sees on the ground whatever will make it strong.’ (speaking of chickens pecking at the ground) (\cite{henry_de_2005})
\z

\begin{CayugaRelated}
\item\textit{Dęˀ} \trs{what}, \trs{how}, p. \pageref{p:[dęˀ-]}\\
\item\textit{Dęˀ, Dęˀ hoˀdęˀ} \trs{what}, p. \pageref{p:[dęˀ, dęˀ hoˀdęˀ]}\\
\item\textit{Dęˀ hoˀdęˀ ni-} \trs{how}, \trs{what way}, p. \pageref{p:[dęˀ hoˀdęˀ ni-]}\\
\item\textit{Gwaˀ} \trs{immediately}, \trs{right then}, \trs{just then}, \trs{finally}, p. \pageref{p:[gwaˀ] ‘immediately’}\\
\item\textit{Hoˀdęˀ} \trs{kind}, p. \pageref{p:[hoˀdęˀ]}
\end{CayugaRelated}

\subsection*{\textbf{Dęˀ hne:ˀ hoˀdęˀ} \trs{what} (emphatic)} \label{p:[dęˀ hne:ˀ hoˀdęˀ]}
\CayugaRemark{Particle group \textit{dęˀ hne:ˀ (hoˀdęˀ)} functions as an indefinite (\sectref{ch:’Indefinite pronouns’}) or interrogative (\sectref{ch:’Interrogative pronouns’}) pronoun; emphatic; clause-initial in direct questions \xref{ex:dpart38}, indirect questions (not shown), and dependent clauses (not shown).}

\ea
\label{ex:dpart38}
\gll \textbf{Dęˀ} \textbf{hné:ˀ} í:se:ˀ? \\
what in.fact you.want.it\\
\glt ‘What do you want?’
\z

\begin{CayugaRelated}
\item Clauses with [dęˀ … (hoˀdęˀ)] \trs{what}, p. \pageref{ch:Clauses with [dęˀ … (hoˀdęˀ)] ‘what’}\\
\item \textit{Dęˀ} \trs{what}, \trs{how}, p. \pageref{p:[dęˀ-]}\\
\item \textit{Dęˀ, dęˀ hoˀdęˀ} \trs{what}, p. \pageref{p:[dęˀ, dęˀ hoˀdęˀ]}\\
\item \textit{Dęˀ hoˀdęˀ ni-} \trs{how}, \trs{what way}, p. \pageref{p:[dęˀ hoˀdęˀ ni-]}\\
\item \textit{Gwaˀ} \trs{immediately}, \trs{right then}, \trs{just then}, \trs{finally}, p. \pageref{p:[gwaˀ] ‘immediately’}\\
\item \textit{Hne:ˀ} \trs{in fact}, p. \pageref{p:[hne:ˀ] ‘in fact’}\\
\item \textit{Hoˀdęˀ} \trs{kind}, p. \pageref{p:[hoˀdęˀ]}
\end{CayugaRelated}

\subsection*{\textbf{Dęˀ hniˀ} \trs{for sure}} \label{p:[dęˀ hniˀ]}
\CayugaRemark{Particle group \textit{…dęˀ hniˀ} emphasizes a fact (\sectref{ch:Focus, contrastive focus, and emphasis markers}); enclitic.}

\ea
\label{ex:dpart12}
\gll Agyaˀda̱hsdeˀ \textbf{dęˀ} \textbf{hniˀ}. \\
I’m.heavy what and\\
\glt ‘I am heavy for sure.’
\z

\ea
\label{ex:dpart13}
\gll Nę́: swatgahtóh gra̱he:t. Weˀsgęhę́: neˀ hnyagwái:. Heˀtgę́h \textbf{dęˀ} \textbf{hniˀ} hegano̱ˀjoyaǫní: tó:gyęh! Nę́: ne:ˀ hǫ:níˀ gonahdrǫ́ˀs neˀ ǫ́:gweh. Oyaˀdanehagwáht dęˀ hniˀ nęgyę́h hnyagwái dwadó:wa:s! \\
look you.all.look.at.it tree it.was.here the bear high what and the.marks.are.made.up.there those.ones see it.is the.reason they.are.afraid the people an.amazing.body what and this.one bear we.are.hunting.it \\
\glt ‘Look at this pine tree. The bear has been here. See how high up those marks are! See – that’s what frightens the people. This bear we are hunting is Nyah-gwaheh, a monster bear.’ (\cite{carrier_legends_2013})
\z

\begin{CayugaRelated}
\item \textit{Dęˀ} \trs{what}, \trs{how}, p. \pageref{p:[dęˀ-]}\\
\item \textit{Hniˀ} \trs{and}, p. \pageref{p:[hniˀ] ‘and’}
\end{CayugaRelated}

\subsection*{\textbf{Dęˀ hoˀdęˀ ni-} \trs{how}, \trs{what way}} \label{p:[dęˀ hoˀdęˀ ni-]}
\CayugaRemark{Particle group \textit{dęˀ … (hoˀdęˀ) ni-} introduces a manner clause (\sectref{ch:Manner clauses with [dęˀ ni-] ‘how’, [shęh ni:yoht] ‘how so’}); clause-initial in direct questions \xref{ex:dpart14}, indirect questions (not shown), and dependent clauses [ ], \xref{ex:dpart15}}

\ea
\label{ex:dpart14}
\gll Asadadwęˀna̱hsáik? \textbf{Dęˀ} hne:ˀ \textbf{hoˀdę́ˀ} \textbf{naˀ}sye:ˀ? \\
you.bit.your.tongue what in.fact kind how.you.did.that \\
\glt ‘You bit your tongue? How did you do that?’ (Mithun and Henry, 1984, 386) (Ga̱há:gǫ: dialogue)
\z

\ea
\label{ex:dpart15}
\gll Tęˀ néˀ sga̱hoˀdę́ˀ dehoˀnigǫhaędáˀs [\textbf{dęˀ} \textbf{hoˀdę́ˀ} \textbf{ni}yo:gyę́: neˀ ga̱hwísda̱ˀe:s]. \\
not the nothing he.doesn’t.understand what kind how.it.works the clock\\
\glt ‘No, he doesn’t understand how time works.’ (Mithun and Henry, 1984, 508) (Aǫhdęgyǫ́heˀ dialogue)
\z

\begin{CayugaRelated}
\item \textit{Dęˀ} \trs{what}, \trs{how}, p. \pageref{p:[dęˀ-]}\\
\item \textit{Dęˀ, Dęˀ hoˀdęˀ} \trs{what}, p. \pageref{p:[dęˀ, dęˀ hoˀdęˀ]}
\end{CayugaRelated}

\subsection*{\textbf{Dęˀ ni:ˀ} \trs{I am, for sure}} \label{p:[dęˀ ni:ˀ]}
\CayugaRemark{Particle group \textit{…dęˀ ni:ˀ} functions as an emphatic pronoun (\sectref{ch:’Emphatic pronouns’}); tag.}

\ea
\label{ex:dpart28}
\gll Agyáˀdahsdeˀ \textbf{dęˀ} \textbf{ni:ˀ}.\\
I’m.heavy what the.me\\
\glt ‘I’m heavy.’
\z

\begin{CayugaRelated}
\item\textit{Dęˀ} \trs{what}, \trs{how} \pageref{p:[dęˀ-]}\\
\item \textit{I:ˀ}, \textit{Ni:ˀ} \trs{I}, \trs{we}, p. \pageref{p:[i:ˀ]}
\end{CayugaRelated}

\subsection*{\textbf{Dęˀ ni:yoht shęh} \trs{why?}} \label{p:[dęˀ ni:yoht shęh]}
\CayugaRemark{Particles and verb \textit{dęˀ … ni:yoht (shęh)} introduces a causative phrase (\sectref{ch:Causative clauses with [dęˀ ni:yoht shęh] ‘why’, [neˀ hǫ:niˀ] ‘how’}); emphatic (\sectref{ch:Focus, contrastive focus, and emphasis markers}); clause-initial in direct questions \xxref{ex:dpart16}{ex:dpart18} and indirect questions \xref{ex:dpart19}. Not used in dependent clauses, where phrases like \textbf{ne:ˀ dagaihǫ:niˀ} ‘the reason why’, ‘that’s why’, ‘because’ are used instead.}

\ea
\label{ex:dpart16}
\gll \textbf{Dęˀ} \textbf{hné:ˀ} \textbf{ni:yóht} \textbf{shęh} ahádo̱ˀne:k? \\
what in.fact the.way.it.is that he.withdrew\\
\glt ‘Why did he leave?’
\z

\ea
\label{ex:dpart17}
\gll \textbf{Dęˀ} \textbf{hné:ˀ} \textbf{ni:yóht} í:se:ˀ. \\
what in.fact the.way.it.is you.want.it\\
\glt ‘Why do you want it that way?’
\z

\ea
\label{ex:dpart18}
\gll \textbf{Dęˀ} diˀ \textbf{hoˀdę́ˀ} \textbf{ni:yóht} tréhs hnaˀgę:ˀ? \\
what so what the.way.it.is too late\\
\glt ‘Why was it so late?’ (\cite[508]{mithun_watewayestanih_1984}, Aǫhdęgyǫ́heˀ dialogue)
\z

\ea
\label{ex:dpart19}
\gll Honǫhdǫ́ˀ diˀ gęh Gwí:deh [\textbf{dęˀ} \textbf{ni:yóht} \textbf{shęh} Tina gowanawę́ˀdagaˀs]? \\
he.knows so Q Peter what the.way.it.is that Tina she.likes.the.taste.of.sugar\\
\glt ‘Does Peter know why Tina likes candy?’
\z

\begin{CayugaRelated}
\item\textit{Dęˀ} \trs{what}, \trs{how} \pageref{p:[dęˀ-]}\\
\item\textit{Shęh} \trs{because}, p. \pageref{p:[shęh] `because’}
\end{CayugaRelated}


\subsection*{\textbf{Dęˀ ǫh hne:ˀ hoˀdęˀ} \trs{what on earth?}, \trs{I wonder what?}} \label{p:[dęˀ ǫh hne:ˀ hoˀdęˀ]}
\CayugaRemark{Particle group \textit{dęˀ oh ne:ˀ (hoˀdęˀ)} functions as an “indefinite pronoun” (\sectref{ch:’Indefinite pronouns’}); clause-initial in direct questions \xref{ex:dpart39}, indirect questions \xref{ex:dpart40}, and dependent clauses (not shown).}

\ea
\label{ex:dpart39}
\gll Ogaˀǫ́h nę́:gyęh! \textbf{Dęˀ} \textbf{ǫ-hne:ˀ} \textbf{hoˀdęˀ}? Dó:s nawę́ˀdaweht! \\
it.is.good this! What I.wonder-in.fact kind? Very it.is.sweet! \\
\glt ‘This is good! What is this? This is sweet!’ (Carrier et al., 2013)
\z

\ea
\label{ex:dpart40}
\gll [\textbf{Dęˀ} \textbf{ǫ-hné:ˀ} na:ya:wę́h] gyę:gwáˀ gwe:gǫ́h hęhné:ˀ a:gaǫgya̱ˀdáhk neˀ agǫ́gwe̱ˀdaˀ? \\
What I.wonder-in.fact it.would.happen if all also.in.fact I.could.share.with.them the my.people? \\
\glt ‘I wonder what would happen if I could share this with my people?’ (\cite{carrier_legends_2013})
\z

\begin{CayugaRelated}
\item Clauses with \textit{Dęˀ … (hoˀdęˀ)} \trs{what}, p. \pageref{ch:Clauses with [dęˀ … (hoˀdęˀ)] ‘what’}\\
\item \textit{Dęˀ} \trs{what}, \trs{how} \pageref{p:[dęˀ-]}\\
\item \textit{Hoˀdęˀ} \trs{kind}, p. \pageref{p:[hoˀdęˀ]}\\
\item \textit{Ne:ˀ} \trs{it is}, p. \pageref{p:[ne:ˀ] `it is’}\\
\item \textit{Oh ne:ˀ} \trs{maybe}, p. \pageref{p:[ǫh ne:ˀ]}\\
\item \textit{Oh, Ǫ:, Ǫ} \trs{I guess}, \trs{I wonder (if)}, p. \pageref{p:[ǫh, ǫ:, ǫ]}
\end{CayugaRelated}

\subsection*{\textbf{Diˀ} \trs{so}, \trs{then}} \label{p:[diˀ]}
\CayugaRemark{Particle signaling acknowledgement of something just said (\sectref{ch:Acknowledgement markers}); enclitic.}

\ea
\label{ex:dpart20}
\gll  Dó:gęhs \textbf{diˀ} gęh? \\
true so Q\\
\glt ‘Is that true then?’
\z

\ea
\label{ex:dpart21}
\gll Gaę \textbf{diˀ} nhǫ:wéh nihsweˀs o:nęh? \\
which then place you.are.someplace now\\
\glt ‘Where then are you now?’
\z

\ea
\label{ex:dpart22}
\gll Hwę:dǫ́h \textbf{diˀ} ęje:yǫˀ? \\
when then she.will.return\\
\glt ‘When then is she going to return?’
\z

\ea
\label{ex:dpart23}
\gll I:s \textbf{diˀ} gęh hne:ˀ? \\
you then Q in.fact\\
\glt ‘How about you then?’
\z

\ea
\label{ex:dpart24}
\gll Ęhę́ˀ sgę́:nǫˀ. Nęˀ \textbf{diˀ} ní:s? \\
yes fine and so you\\
\glt ‘I am fine thanks. And you then?’
\z

\ea
\label{ex:dpart25}
\gll Sanǫhǫkdá:niˀ gęh \textbf{diˀ}? \\
you are.sick Q so\\
\glt ‘Are you sick then?’
\z

\ea
\label{ex:dpart26}
\gll Ęhsnéˀ gęh \textbf{diˀ}? \\
we.two.will.go.together Q then\\
\glt ‘Are you coming along then?’
\z

\ea
\label{ex:dpart27}
\gll Do: \textbf{diˀ} nęyónisheˀ? \\ 
how then it.will.take.some.time\\
\glt ‘So how long will it take?’
\z


\subsection*{\textbf{Do:} \trs{how}} \label{p:[do:]}
\CayugaRemark{Particle \textit{do:} followed by verb beginning with \stem{ni-} {\partitive} prefix and functioning as an “adverb of manner” (\sectref{ch:’Adverbs’ of manner}, \sectref{ch:Measuring clauses with [do: … ni-] ‘how much, many’}); clause-initial.}

\ea
\label{ex:dpart41}
\gll \textbf{Do:} \textbf{ni}gá:nǫ:? \\
how it.costs.a.certain.amount \\
\glt ‘How much does it cost?’
\z

\ea
\label{ex:dpart42}
\gll \textbf{Do:} \textbf{ní:}waˀs?\\
how it.is.a.certain.size\\
\glt ‘How big is it?’
\z

\ea
\label{ex:dpart43}
\gll \textbf{Do:} \textbf{ni}yá:gaˀ?\\
how she.is.a.certain.size \\
\glt ‘How big is she / it?’
\z

\ea
\label{ex:dpart44}
\gll \textbf{Do:} \textbf{ni}tgá:deˀ? \\
how it.stands.out \\
\glt ‘How high is it?’
\z

\ea
\label{ex:dpart45}
\gll \textbf{Do:} \textbf{ni}yohwihsdáˀe:ˀ? \\
how the.way.it.strikes.the.metal \\
\glt ‘What time is it?’
\z

\ea
\label{ex:dpart46}
\gll \textbf{Do:} \textbf{ni}sohsriyáˀgǫh?> \\
how you.have.crossed.a.number.of.winters \\
\glt ‘How old are you?’
\z

\ea
\label{ex:dpart47}
\gll \textbf{Do:} diˀ \textbf{ni}yó:weˀ? \\
how so it.is.a.certain.distance \\
\glt ‘How far is it then?’
\z

\ea
\label{ex:dpart48}
\gll \textbf{Do:} gwaˀ \textbf{ni}yóhsrage:? \\
how this.time it.is.two.or.more.winters \\
\glt ‘How many years is it?’
\z

\begin{CayugaRelated}
\item \textit{Do:} \trs{how}, p. \pageref{p:[do:]}\\
\item \textit{Do: gwaˀ ni:yǫ:} \trs{a certain amount}, \trs{a certain measure}, \trs{however much}, p. \pageref{p:[do: gwaˀ ni:yǫ:]}\\
\item \textit{Do: i:ˀ} \trs{let me} \trs{how about me?}, p. \pageref{p:[do: i:ˀ]}\\
\item \textit{Do: ni-…nisheˀ} \trs{how long}, \trs{how much time}, p. \pageref{p:[do: ni-…nisheˀ]}\\
\item \textit{Do: ni+…ǫ:} \trs{how many people} p. \pageref{p:[do: ni+…ǫ:]}\\
\item \textit{Do: niyowi̱hsda̱ˀe:ˀ} \trs{what time is it?}, \trs{when?}, p. \pageref{p:[do: niyowihsdaˀe:ˀ]}
\end{CayugaRelated}

\subsection*{\textbf{Do: gwaˀ ni:yǫ:} \trs{a certain amount}, \trs{a certain measure}, \trs{however much}} \label{p:[do: gwaˀ ni:yǫ:]}
\CayugaRemark{Particle group and verb functioning as an “indefinite pronoun” (\sectref{ch:’Indefinite pronouns’}); free-standing.}

\ea
\label{ex:dpart49}
\gll \textbf{Do:} \textbf{gwaˀ} \textbf{ni:yǫ́:} ęsagá:dęˀ. \\
how.much right.then a.certain.amount it.will.cost.you\\
\glt ‘It is going to cost you a certain amount.’
\z

\ea
\label{ex:dpart50}
\gll “\textbf{Do:} \textbf{gwaˀ} \textbf{ní:yǫ:} neˀ, neˀ giˀ ęhadigá:nyaˀk. \\
how.much right.then a.certain.amount the the just they(males).will.pay.for.it\\
\glt ‘However much they will use to pay for it.’ (\cite{henry_de_2005})
\z

\begin{CayugaRelated}
\item \textit{Do:} \trs{how}, p. \pageref{p:[do:]}\\
\item \textit{Gwaˀ} \trs{immediately}, \trs{right then}, \trs{just then}, \trs{finally}, p. \pageref{p:[gwaˀ] ‘immediately’}
\end{CayugaRelated}

\subsection*{\textbf{Do: i:ˀ} \trs{let me!}, \trs{how about me?}} \label{p:[do: i:ˀ]}
\CayugaRemark{Particle group and free-standing expression; also see (\sectref{Particles and particle groups used with commands}).}

\ea
\label{ex:dpart51}
\gll Do: i:ˀ! \\
how me\\
\glt ‘Let me!’ `How about me!’
\z

\begin{CayugaRelated}
\item \textit{Do:} \trs{how}, p. \pageref{p:[do:]}\\
\item \textit{I:ˀ}, \textit{Ni:ˀ} \trs{I}, \trs{we}, p. \pageref{p:[i:ˀ]}
\end{CayugaRelated}

\subsection*{\textbf{Do: ni-…nisheˀ} \trs{how long}, \trs{how much time}} \label{p:[do: ni-…nisheˀ]}
\CayugaRemark{Particle plus verb, \textit{do: … ni-…nisheˀ} `how long’ functions as an “adverb of time” (\sectref{ch:’Adverbs’ of time}, \sectref{ch:Measuring clauses with [do: … ni-] ‘how much, many’}); clause-initial.}

\ea
\label{ex:dpart52}
\gll \textbf{Dó:} gwaˀ \textbf{n}ęyóni̱sheˀ? \\
how emphasis it.will.take.time\\
\glt ‘How long will it take then?’
\z

\ea
\label{ex:dpart53}
\gll \textbf{Dó:} gwaˀ \textbf{naˀ}óni̱hsheˀ? \\
how emphasis it.took.time \\
\glt ‘How long did it take?’
\z

\begin{CayugaRelated}
\item \textit{Do:} \trs{how}, p. \pageref{p:[do:]}
\end{CayugaRelated}

\subsection*{\textbf{Do: ni+…ǫ:} \trs{how many}, \trs{how much}} \label{p:[do: ni+…ǫ:]}
\CayugaRemark{Particle plus verb \stem{ni+…ǫ:} `be a certain number of’, used with counting and measuring (\sectref{ch:Measuring clauses with [do: … ni-] ‘how much, many’}, \sectref{ch:Comparisons, counting, measuring}); clause-initial.}

\ea
\label{ex:dpart54}
\gll \textbf{Do:} \textbf{ni}gá:gǫ:? \\
how a.number.of.people \\
\glt ‘How many people (females or mixed group)?’
\z

\ea
\label{ex:dpart55}
\gll \textbf{Do:} \textbf{ni}hę́:nǫ:? \\
how a.number.of.males \\
\glt ‘How many men?’
\z

\ea
\label{ex:dpart56}
\gll \textbf{Do:} \textbf{ní:}yǫ:? \\
how a.number.of.things \\
\glt ‘How many?’, `How much?’
\z

\ea
\label{ex:dpart57}
\gll \textbf{Do:} \textbf{ni:}yǫ́: ga̱hna:ˀ? \\
how a.number.of.things gas\\
\glt ‘How much gas is in there?’
\z

\begin{CayugaRelated}
\item \textit{Do:} \trs{how}, p. \pageref{p:[do:]}
\end{CayugaRelated}

\subsection*{\textbf{Do: niyowihsdaˀe:ˀ} \trs{what time is it?}, \trs{when?}} \label{p:[do: niyowihsdaˀe:ˀ]}
\CayugaRemark{Particle and verb functioning as an “adverb of time” (\sectref{ch:’Adverbs’ of time}, \sectref{ch:Clauses with [hwę:dǫh], [nęh], [nę:gyęh hwaˀ], [ne:ˀ hwaˀ], etc. ‘when’}); clause-initial in independent clauses (direct questions, \xref{ex:dpart31} and dependent clauses \xref{ex:dpart32}}

\ea
\label{ex:dpart31}
\gll \textbf{Do:} \textbf{niyowihsdaˀe:ˀ}?\\
how it.is.a.certain.time\\
\glt ‘What time is it?’
\z

\ea
\label{ex:dpart32}
\gll Daskro:wíˀ [\textbf{do:} \textbf{niyowihsdaˀe:ˀ} ęsahdę́:diˀ]. \\
you.tell.me how it.is.a.certain.time you.will.leave\\
\glt ‘You tell me when you are going to leave.’
\z

\begin{CayugaRelated}
\item \textit{Do:} \trs{how}, p. \pageref{p:[do:]}
\end{CayugaRelated}

\subsection*{\textbf{Do:gaˀ} \trs{I don’t know}} \label{p:[do:gaˀ]}
\CayugaRemark{Particle functioning as an “evidential marker” (\sectref{ch:Evidential markers}); free-standing.}

\ea
\label{ex:dpart58}
Sam:\\
\gll Sǫ: diˀ hne:ˀ nˀaht? \\
who so in.fact who \\
\glt `Who is it?’ \\
Lila: \\
\gll \textbf{Dó:gaˀ}. \\
I.don’t.know \\
\glt ‘I don’t know.’ (\cite[441]{mithun_watewayestanih_1984}, Dwęnǫhsanékahǫˀ dialogue)
\z


\subsection*{\textbf{Do:gęhs} \trs{really}, \trs{very}} \label{p:[do:gęhs]}
\CayugaRemark{Particle functioning as an “adverb of degree” (\sectref{ch:’Adverbs’ of degree}); clause-initial.}

\ea
\label{ex:dpart59}
\gll …tréhs a:yę́:ˀ o:nę́h [\textbf{do:gę́hs} ganǫ́:ˀ ohǫ́na̱ˀdaˀ].\\
…because it.seems now [really it.is.expensive potatoes] \\
\glt ‘…because it seems like potatoes are really expensive now.’ \cite[420]{mithun_watewayestanih_1984}, Ęswayętoˀ Gęh? dialogue)
\z

\ea
\label{ex:dpart60}
\gll A:yę́:ˀ [\textbf{do:gę́hs} desawayęnhá:ˀǫh]. \\
it.seems [really you.are.busy] \\
\glt ‘You really looked busy.’ (\cite[494]{mithun_watewayestanih_1984}, Ękníyętoˀ dialogue)
\z

\begin{CayugaRelated}
\textit{Do:gęhs} \trs{it’s true}, \trs{exactly}, \trs{just so}, \trs{isn’t it true?}, p. \pageref{p:[do:gęhs]}\\
\textit{Do:gęhs diˀ gęh} \trs{it’s true}, \trs{isn’t it true?}, p. \pageref{p:[do:gęhs diˀ gęh]}\\
\textit{Do:gęhs ę:} \trs{it’s true}, \trs{exactly}, \trs{just so}, \trs{isn’t it true?}, p. \pageref{p:[do:gęhs], [do:gęhs ę:ˀ]}
\end{CayugaRelated}



\subsection*{\textbf{Do:gęhs, Do:gęhs ę:ˀ} \trs{it’s true}, \trs{exactly}, \trs{just so}, \trs{isn’t it true?}} \label{p:[do:gęhs], [do:gęhs ę:ˀ]}
\CayugaRemark{Particle asking for or providing confirmation (\sectref{ch:Agreement markers}); tag (\sectref{Tag questions (asking for confirmation from the listener)}) or free-standing.}

\ea
\label{ex:dpart61}
\gll Wę̱hnihsri:yó: wáˀneˀ, \textbf{dó:gęhs}? \\
it.is.a.nice.day today, true\\
\glt ‘It’s a nice day today, innit?’
\z

\ea
\label{ex:dpart62}
\gll Wę̱hnihsri:yó: wáˀneˀ, \textbf{dó:gęhs} \textbf{ ę:ˀ}? \\
it.is.a.nice.day today, true affirm\\
\glt ‘It’s a nice day, innit?’
\z

\begin{CayugaRelated}
\item \textit{Do:gęhs} \trs{really}, \trs{very}, p. \pageref{p:[do:gęhs]}\\
\item \textit{Do:gęhs diˀ gęh} \trs{it’s true}, \trs{isn’t it true?}, p. \pageref{p:[do:gęhs diˀ gęh]}\\
\item \textit{Do:gęhs ę:} \trs{it’s true}, \trs{exactly}, \trs{just so}, \trs{isn’t it true?}, p. \pageref{p:[do:gęhs], [do:gęhs ę:ˀ]}
\end{CayugaRelated}

\subsection*{\textbf{Do:gęhs diˀ gęh} \trs{it’s true}, \trs{isn’t it true?}} \label{p:[do:gęhs diˀ gęh]}
\CayugaRemark{Particle group asking for confirmation (\sectref{Tag questions (asking for confirmation from the listener)}); free-standing.}

\ea
\label{ex:dpart63}
\gll Dó:gęhs diˀ gęh? \\
True so Q\\
\glt ‘Is that true then?’
\z


\subsection*{\textbf{Dohga:ˀah} \trs{a few}} \label{p:[dohga:ˀah]}
\CayugaRemark{Atypical verb functioning as an “adverb of degree” (\sectref{ch:’Adverbs’ of degree}).}

\ea
\label{ex:dpart580}
do̱hga:ˀah {}\\
\gll do̱hg-a:ˀah\\
noun-small.stative\\
\glt ‘a few’
\z



\subsection*{\textbf{Do:s} \trs{really}, \trs{very}} \label{p:[do:s]}
\CayugaRemark{Particle group functioning as an “adverb of degree” (\sectref{ch:’Adverbs’ of degree}); clause-initial.}

\ea
\label{ex:dpart66}
\gll O:nę́h \textbf{do:s} aˀohdrǫ́:k neˀ hnyagwáiˀ. \\
now really it.became.fearful the bear\\
\glt ‘Fear filled the heart of the great bear for the first time.’ (\cite{carrier_legends_2013})
\z

\ea
\label{ex:dpart67}
\gll Ogaˀǫ́h nę́:gyęh! Dęˀ ǫhne:ˀ hoˀdęˀ? \textbf{Do:s} nawę́ˀdaweht! \\
it.tastes.good this what I.wonder thing really sweet \\
\glt ‘This is good! What is this? This is sweet!’ (\cite{carrier_legends_2013})
\z

\ea
\label{ex:dpart68}
\gll Ó:, neˀ giˀ \textbf{do:s} aˀagyatgęˀsé:ˀ. \\
oh the just really we.looked.at.it \\
\glt ‘Oh, then we really had a good look.’ (\cite{henry_de_2005})
\z

\ea
\label{ex:dpart69}
\gll Akeyatgahtóˀ, o:nę́h \textbf{do:s} agyǫ́:gyaˀt. \\
I.looked.at.her now really I.laughed\\
\glt ‘I looked at her, and really started laughing.’ (\cite{henry_de_2005})
\z

\begin{CayugaRelated}
\item \textit{Do:s giˀ} \trs{just so}, \trs{indeed}, p. \pageref{p:[do:s giˀ]}
\end{CayugaRelated}

\subsection*{\textbf{Do:s giˀ} \trs{just so}, \trs{indeed}} \label{p:[do:s giˀ]}
\CayugaRemark{Particle group functioning as an “agreement marker” (\sectref{ch:Agreement markers}); free-standing.}

\ea
\label{ex:dpart70}
\gll Do:s giˀ.\\
really just\\
\glt ‘Just so.’, ‘Indeed.’
\z

\begin{CayugaRelated}
\item \textit{Do:s} \trs{really}, \trs{very}, p. \pageref{p:[do:s]}\\
\item \textit{Giˀ} \trs{just}, p. \pageref{p:[giˀ]}
\end{CayugaRelated}

\section{E, Ę particles}

\subsection*{\textbf{E:ˀ} \trs{again}, \trs{still}} \label{p:[e:ˀ]}
\CayugaRemark{Particle functioning as an “adverb of time” (\sectref{ch:’Adverbs’ of time}); enclitic.}

\ea
\label{ex:epart1}
\gll Jidwahshé:t giˀ gyę:ˀ \textbf{e:ˀ}.\\
let’s.count just that.one again\\
\glt ‘Let’s count it again then!’
\z

\ea
\label{ex:epart2}
\gll Hehshę:da:gé:ˀ \textbf{e:ˀ}.\\
he.is.lying.over.there again\\
\glt ‘He is lying over there again!’
\z

\ea
\label{ex:epart3}
\gll Ahsǫ́h \textbf{e:ˀ} nihs desatwę:jó:nih!\\
more again you you.want.it\\
\glt ‘You still want more!’
\z

\ea
\label{ex:epart4}
\gll Neˀ giˀ ę:ˀ \textbf{e:ˀ} toh iheˀs.\\
it just affirm again that.one he.is\\
\glt ‘He is here again!’
\z

\ea
\label{ex:epart5}
\gll Tsę: \textbf{e:ˀ} neˀ satró:wi:.\\
Oh.my, again the you.talk.about.it\\
\glt ‘You are talking about that again!’ (said in exasperation)
\z


\subsection*{\textbf{Ę:ˀ, Neˀ ę:ˀ} \trs{isn’t it so?}, \trs{yes?}, \trs{no?}, \trs{innit?}} \label{p:[ę:ˀ], [neˀ ę:ˀ] `isn’t it so?’}
\CayugaRemark{Particle group \textit{…(neˀ) ę:ˀ} functions as a tag question or, in answers, to convey affirmation (\sectref{Tag questions (asking for confirmation from the listener)}); enclitic.}

\ea
\label{ex:enpart2}
\gll Enǫhwéˀs \textbf{ę́:ˀ}?\\
She.likes.it affirm\\
\glt ‘She likes it, doesn’t she?’
\z

\ea
\label{ex:enpart3}
\gll Wę̱hnihsri:yó: wáˀneˀ, \textbf{neˀ} \textbf{ę́:ˀ}?\\
it.is.a.nice.day today, the affirm\\
\glt ‘it is a nice day, innit?’
\z

\begin{CayugaRelated}
\item \textit{Ę:, E:} (possible atypical verb), p. \pageref{p:[ę:, e:] `atypical verb’}
\end{CayugaRelated}

\subsection*{\stem{ę:, e:} (possible atypical verb)} \label{p:[ę:, e:] `atypical verb’}
\CayugaRemark{\stem{ę:, e:} may be an atypical verb. The following words may be related.}

\ea
\label{ex:enpart1}
\ea
 ne:ˀ `it is’
\ex de̱ˀgę: `it isn’t’
\ex niwa:gęˀ `so many’ (\cite{henry_de_2005})
\ex gę:s `usually’
\ex he:gę: `all’
\ex ę:ˀ `it is so’, `affirmative’
\ex ti:gę: `something odd’
\z
\z

\begin{CayugaRelated}
\item Equative sentences with linking verbs \textit{né:ˀ} \trs{it is} or \textit{de̱ˀgę:} \trs{it isn’t}, p. \pageref{Equative sentences with linking verbs [né:ˀ], [deˀgę:]}\\
\item \textit{E:ˀ, Neˀ ę:ˀ} \trs{isn’t it so?}, \trs{yes?}, \trs{no?}, \trs{innit?}, p. \pageref{p:[ę:ˀ], [neˀ ę:ˀ] `isn’t it so?’}\\
\item \textit{Gę:s} \trs{generally, used to, usually, normally}, p. \pageref{p:[gę:s]}\\
\item \textit{Ne:ˀ} equative, p. \pageref{p:[ne:ˀ] equative}\\
\item \textit{Ne:ˀ he:gę:} \trs{just}, \trs{only}, \trs{all}, p. \pageref{p:[ne:ˀ he:gę:]}\\
\item \textit{Otgaˀdeˀ neˀ niwa:gęˀ ni-} \trs{as many as}, p. \pageref{p:[otgaˀdeˀ neˀ niwa:gęˀ ni-]}\\
\item \textit{Tęˀ de̱ˀgę:} \trs{it isn’t}, p. \pageref{p:[tęˀ deˀgę:]}
\end{CayugaRelated}

\subsection*{\textbf{Ę: gwa:dih} \trs{on the other side}} \label{p:[ę: gwa:dih]}
\CayugaRemark{Particle group functioning as an “adverb of place” (\sectref{‘Adverbs’ of place}, \sectref{p:[ę: ni-noun+adih]}); clause-initial; \textit{gwa:dih} can also be spelled or pronounced as \textit{gwai}.}

\largerpage
\ea
\label{ex:enpart4}
\gll Ne: tęhséhsage:t \textbf{ę:} \textbf{gwa:dih} hęhsóda: ę: na̱ˀohahá:dih degyohǫ́:do:t to hǫ: hęhsóda:.\\
it.is you.will.bend.it other side you.will.hook.it other side.of.the.road another.whip there where you.will.hook.it\\
\glt ‘You will bend it and hook it onto another whip on the other side of the road.’ (\cite{mithun_how_1980})
\z

\begin{CayugaRelated}
\item \textit{Gwa:dih, gwai} \trs{to one side}, p. \pageref{p:[gwa:dih]}
\end{CayugaRelated}

\subsection*{\textbf{Ę: ni-\textsc{noun}+adih} \trs{on the other side of \textsc{noun}}} \label{p:[ę: ni-noun+adih]}
\CayugaRemark{Particle and incorporating verb \stem{ę: ni-\textsc{noun}+adih} functions as an “adverb of place” (\sectref{‘Adverbs’ of place}); clause-initial.}

\ea
\label{ex:enpart5}
\gll Ne: tęhséhsage:t ę: gwa:dih hęhsóda: \textbf{ę:} \textbf{n}a̱ˀohah\textbf{á:dih} degyohǫ́:do:t to hǫ: hęhsóda:.\\
it.is you.will.bend.it other side you.will.hook.it other side.of.the.road another.whip there where you.will.hook.it\\
\glt ‘You will bend it and hook it onto another whip on the other side of the road.’ (\cite{mithun_how_1980})
\z

\ea
\label{ex:enpart6}
\gll \textbf{Ę:} \textbf{na}ganǫ̱hs\textbf{a:díh} ita:t.\\
other side.of.the.house he.is.standing\\
\glt ‘he is standing on the other side of the house.’
\z

\begin{CayugaRelated}
\item \textit{E: gwa:dih} \trs{on the other side}, p. \pageref{p:[ę: gwa:dih]}
\end{CayugaRelated}

\subsection*{\textbf{Ęhęˀ} \trs{yes}} \label{p:[ęhęˀ]}
\CayugaRemark{Particle functioning as an “agreement marker” (\sectref{ch:Agreement markers}).}

\begin{CayugaRelated}
\item \textit{Ehęˀ ę:ˀ} \trs{yes indeed}, p. \pageref{p:[ęhęˀ ę:ˀ]}\\
\item \textit{Ehęˀ gęh} \trs{is that right?}, p. \pageref{p:[ęhęˀ gęh]}\\
\item \textit{Ehęˀ gyę:ˀ} \trs{yes indeed}, p. \pageref{p:[ęhęˀ gyę:ˀ]}\\
\item \textit{Ehęˀ seˀ} \trs{it is so}, \trs{yes indeed}, p. \pageref{p:[ęhęˀ seˀ]}
\end{CayugaRelated}

\subsection*{\textbf{Ęhęˀ ę:ˀ} \trs{yes indeed}} \label{p:[ęhęˀ ę:ˀ]}
\CayugaRemark{Particle group \textit{(ęhęˀ) ę:ˀ} functions as an agreement marker (\sectref{ch:Agreement markers}); enclitic. A speaker can use \textit{ę:ˀ} to affirm or emphasize her statement; the listener can use \textit{ęhęˀ ę:ˀ} to agree.}

\ea
\label{ex:enpart7}
\gll Neˀ giˀ \textbf{ę:ˀ} e:ˀ toh iheˀs.\\
neˀ just affirm again there he.is\\
\glt ‘he is here again!’
\z

\ea
\label{ex:enpart8}
Sam:\\
\gll Háe. Wę̱hnisri:yó: \textbf{ę́:ˀ}. \\
hi nice.day affirm\\
\glt `Hi. Nice day, isn’t it.’\\
Neil: \\
\gll \textbf{Ęhę́ˀ} \textbf{ę́:ˀ}. \\
yes affirm\\
\glt `Yes it is, isn’t it.’ (Mithun and Henry, 1984, 339, Oˀdréhdatgiˀ dialogue)
\z



\begin{CayugaRelated}
\item \textit{E:ˀ, Neˀ ę:ˀ} \trs{isn’t it so?}, \trs{yes?}, \trs{no?}, \trs{innit?}, p. \pageref{p:[ę:ˀ], [neˀ ę:ˀ] `isn’t it so?’}\\
\item \textit{Ehęˀ} \trs{yes}, p. \pageref{p:[ęhęˀ]}\\
\item \textit{Ehęˀ gęh} \trs{is that right?}, p. \pageref{p:[ęhęˀ gęh]}\\
\item \textit{Ehęˀ gyę:ˀ} \trs{yes indeed}, p. \pageref{p:[ęhęˀ gyę:ˀ]}\\
\item \textit{Ehęˀ seˀ} \trs{it is so}, \trs{yes indeed}, p. \pageref{p:[ęhęˀ seˀ]}
\end{CayugaRelated}


\subsection*{\textbf{Ęhęˀ gęh} \trs{is that right?}} \label{p:[ęhęˀ gęh]}
\CayugaRemark{Particle group and free-standing expression functioning as a “tag question” (\sectref{Tag questions (asking for confirmation from the listener)}).}

\ea
\label{ex:enpart9}
\gll Ęhęˀ gęh?\\
yes Q\\
\glt ‘Is that right?
\z

\begin{CayugaRelated}
\item \textit{Ehęˀ} \trs{yes}, p. \pageref{p:[ęhęˀ]}\\
\item \textit{Ehęˀ ę:ˀ} \trs{yes indeed}, p. \pageref{p:[ęhęˀ ę:ˀ]}\\
\item \textit{Ehęˀ gyę:ˀ} \trs{yes indeed}, p. \pageref{p:[ęhęˀ gyę:ˀ]}\\
\item \textit{Ehęˀ seˀ} \trs{it is so}, \trs{yes indeed}, p. \pageref{p:[ęhęˀ seˀ]}
\end{CayugaRelated}


\subsection*{\textbf{Ęhęˀ gyę:ˀ} \trs{yes indeed}} \label{p:[ęhęˀ gyę:ˀ]}
\CayugaRemark{Particle group and free-standing expression, functioning as an “agreement marker” (\sectref{ch:Agreement markers}).}

\ea
\label{ex:enpart10} Neighbour: \\
\gll Sgę́:nǫˀ. Se̱ˀsgęhę́:ˀ ę:ˀ gwa̱ˀtóh waˀjih?. \\
Hello, you.were.here affirmation just.now not.long.ago \\
\glt `Hello. Weren’t you here a while ago?’\\
Sam: \\
\gll \textbf{Ęhę́ˀ} \textbf{gyę́:ˀ}.\\
yes this.one\\
\glt  `Yes I was.’ (\cite[441]{mithun_watewayestanih_1984}, Dwęnǫhsanekahǫˀ dialogue)
\z

\begin{CayugaRelated}
\item \textit{Ehęˀ} \trs{yes}, p. \pageref{p:[ęhęˀ]}\\
\item \textit{Ehęˀ ę:ˀ} \trs{yes indeed}, p. \pageref{p:[ęhęˀ ę:ˀ]}\\
\item \textit{Ehęˀ gęh} \trs{is that right?}, p. \pageref{p:[ęhęˀ gęh]}\\
\item \textit{Ehęˀ seˀ} \trs{it is so}, \trs{yes indeed}, p. \pageref{p:[ęhęˀ seˀ]}
\end{CayugaRelated}


\subsection*{\textbf{Ęhęˀ seˀ} \trs{it is so}, \trs{yes indeed}} \label{p:[ęhęˀ seˀ]}
\CayugaRemark{Particle group and free-standing expression, functioning as an “agreement marker” (\sectref{ch:Agreement markers}).}

\ea
\label{ex:enpart11}
\gll Ęhęˀ seˀ.\\
Yes you.know\\
\glt ‘Yes indeed.
\z

\begin{CayugaRelated}
\item \textit{Ehęˀ} \trs{yes}, p. \pageref{p:[ęhęˀ]}\\
\item \textit{Ehęˀ ę:ˀ} \trs{yes indeed}, p. \pageref{p:[ęhęˀ ę:ˀ]}\\
\item \textit{Ehęˀ gęh} \trs{is that right?}, p. \pageref{p:[ęhęˀ gęh]}\\
\item \textit{Ehęˀ gyę:ˀ} \trs{yes indeed}, p. \pageref{p:[ęhęˀ gyę:ˀ]}\\
\item \textit{Seˀ} \trs{you know}, p. \pageref{p:[seˀ]}
\end{CayugaRelated}

\subsection*{\textbf{Ęwa:dǫˀ} \trs{yes, you may}, \trs{it is permissible}} \label{p:[ęwa:dǫˀ]}
\CayugaRemark{Verb functioning as an evidential marker (\sectref{ch:Evidential markers}).}

\ea
\label{ex:enpart12}
\gll Ęhęˀ, \textbf{ęwá:dǫˀ}.\\
yes it.will.be.possible\\
\glt ‘Yes, you may.’
\z

\ea
\label{ex:enpart13}
\gll Né:ˀ giˀ aˀa:gę́ˀ, “\textbf{Ęwa:dǫ́ˀ}, \textbf{ęwa:dǫ́ˀ} ojikeˀdáˀ sdǫ́:hah ęhsnihs, deyohsaít hniˀ.”\\
it.is just she.said, it.is.possible, it.is.possible salt a.little you.will.use, pepper also\\
\glt ‘And then she said, “Use a little salt, and pepper too.”’
\z

\ea
\label{ex:enpart14}
\gll Ó:, neˀ gíˀ gyę:ˀ aˀa:gę́ˀ, “Owidra:htáˀ hniˀ \textbf{ęwa:dǫ́ˀ} ęhsráh neˀ onáˀda:ˀ.”\\
Oh, the just this.one she.said, butter and it.is.possible you.will.spread the bread\\
\glt ‘Oh, and she also said, “There is butter to spread on your bread.”’ (\cite{henry_de_2005})
\z

\ea
\label{ex:enpart15}
\gll Swasha:ˀsé:k shęh taˀdewę̱hnihsragé: \textbf{ęwa:dǫ́ˀ} daedwadęnǫ̱hǫ́:nyǫ:ˀ.\\
you.remember that every.day it.is.possible. we.should.give.thanks\\
\glt ‘Remember every day is a good day, and we can give thanks for that.’ (\cite{carrier_legends_2013})
\z

\ea
\label{ex:enpart16}
\gll O:nę́h agasdáędaˀ. \textbf{Ęwa:dǫ́ˀ} ganadagǫ́: hękné: ó:nęh.\\
now it.stopped.raining. it.is.possible to.town we.will.go.there now\\
\glt ‘Now the rain has stopped. We can go to town now.’ (\cite[212]{mithun_watewayestanih_1984}, Satrǫ́:nih dialogue)
\z

\begin{CayugaRelated}
\item \textit{Ewa:dǫˀ gęh} \trs{may I}, \trs{may we}, p. \pageref{p:[ęwa:dǫˀ gęh]}\\
\item \textit{Ewa:dǫˀ giˀ shęh} \trs{maybe}, \trs{a possibility}, p. \pageref{p:[ęwa:dǫˀ giˀ shęh]}
\end{CayugaRelated}

\subsection*{\textbf{Ęwa:dǫˀ gęh} \trs{may I}, \trs{may we}} \label{p:[ęwa:dǫˀ gęh]}
\CayugaRemark{Particle group functioning as question ; clause-initial; (also see \sectref{ch:Evidential markers}).}

\ea
\label{ex:enpart17}
\gll \textbf{Ęwa:dǫ́ˀ} \textbf{gęh} ęgatgǫhsóhai?\\
it.is.possible Q I.will.wash.my.face\\
\glt ‘May I wash my face?’
\z

\ea
\label{ex:enpart18}
\gll \textbf{Ęwa:dǫ́ˀ} \textbf{gęh} neˀ ohnégagriˀ?\\
it.is.possible Q the soup\\
\glt ‘May we have some soup?’ (\cite{carrier_legends_2013})
\z

\begin{CayugaRelated}
\item \textit{Ewa:dǫˀ} \trs{yes, you may}, \trs{it is permissible}, p. \pageref{p:[ęwa:dǫˀ]}\\
\item \textit{Ewa:dǫˀ giˀ shęh} \trs{maybe}, \trs{a possibility}, p. \pageref{p:[ęwa:dǫˀ giˀ shęh]}
\end{CayugaRelated}

\subsection*{\textbf{Ęwa:dǫˀ giˀ shęh} \trs{maybe}, \trs{a possibility}} \label{p:[ęwa:dǫˀ giˀ shęh]}
\CayugaRemark{Particle group functioning as an “evidential marker” (\sectref{ch:Evidential markers}).}

\ea
\label{ex:enpart19}
\gll ęwa:dǫˀ giˀ shęh\\
it.will.be.possible just that\\
\glt ‘maybe’
\z

\begin{CayugaRelated}
\item \textit{Ewa:dǫˀ} \trs{yes, you may}, \trs{it is permissible}, p. \pageref{p:[ęwa:dǫˀ]}\\
\item \textit{Ewa:dǫˀ gęh} \trs{may I}, \trs{may we}, p. \pageref{p:[ęwa:dǫˀ gęh]}
\end{CayugaRelated}


\section{G particles}

\subsection*{\stem{gaˀ-} element (referring to an approximate location)} \label{p:[gaˀ-]}
\CayugaRemark{Element appearing in several particle groups; describes an approximate location.}

\begin{CayugaRelated}
\item \textit{Gaˀ-to:hah} \trs{somewhere}, \trs{someplace}, \trs{around}, \trs{anywhere}, \trs{thereabouts}, p. \pageref{p:[gaˀ-to:hah]}\\
\item \textit{Gaˀ-to:hah tohgeh} \trs{thereabouts}, p. \pageref{p:[gaˀ-to:hah tohgeh]}\\
\item \textit{Tęˀ gaˀ-toh} \trs{nowhere}, \trs{not anywhere}, p. \pageref{p:[tęˀ gaˀ-toh]}
\end{CayugaRelated}

\subsection*{ \stem{-ga:ˀ} element (referring to living beings)} \label{p:[-ga:ˀ]}
\CayugaRemark{\stem{-ga:ˀ} element appearing in several particle combinations; enclitic; refers to living beings.}

\begin{CayugaRelated}
\item \textit{Sǫ:-ga:ˀ} \trs{anyone}, \trs{any living thing}, p. \pageref{p:[sǫ:-ga:ˀ]}\\
\item \textit{Sǫ:-ga:ˀah} \trs{someone}, \trs{anyone}, \trs{anything}, p. \pageref{p:[sǫ:-ga:ˀah] ‘someone’}\\
\item \textit{Tęˀ sǫ:-ga:ˀ} \trs{nobody}, \trs{no one}, p. \pageref{p:[tęˀ sǫ:-ga:ˀ]}
\end{CayugaRelated}

\subsection*{\textbf{Gaˀ-toh, Tęˀ gaˀ-toh} \trs{nowhere}, \trs{not anywhere}} \label{p:[gaˀ-toh, tęˀ gaˀ-toh]}
\CayugaRemark{Particle group \textit{(tęˀ) ga̱ˀ-toh} functions as an “adverb of place” (\sectref{‘Adverbs’ of place}); clause-initial.}

\ea
\label{ex:gpart1}
\gll \exemph{Ga̱ˀ-toh} tsǫ: ta:setsę́i:.\\
nowhere only you.will.find.it\\
\glt ‘You just won’t find it anywhere.’
\z

\begin{CayugaRelated}
\item \stem{Gaˀ-} element (referring to a location), p. \pageref{p:[gaˀ-]}\\
\item \textit{Gaˀ-to:hah} \trs{somewhere}, \trs{someplace}, \trs{around}, \trs{anywhere}, \trs{thereabouts}, p. \pageref{p:[gaˀ-to:hah]}\\
\item \textit{Gaˀ-to:hah tohgeh} \trs{thereabouts}, p. \pageref{p:[gaˀ-to:hah tohgeh]}\\
\item \textit{Tęˀ} \trs{no}, \trs{not}, p. \pageref{p:[tęˀ]}\\
\item \textit{Tęˀ gaˀ-toh, Gaˀ-toh} \trs{nowhere}, \trs{not anywhere}, p. \pageref{p:[tęˀ gaˀ-toh]}
\end{CayugaRelated}

\subsection*{\textbf{Gaˀ-to:hah} \trs{somewhere}, \trs{someplace}, \trs{around}, \trs{anywhere}, \trs{thereabouts}} \label{p:[gaˀ-to:hah]}
\CayugaRemark{Particle group functioning as an “adverb of place” (\sectref{‘Adverbs’ of place}); clause-initial.}

\ea
\label{ex:gpart2}
\gll \exemph{Ga̱ˀto:háh} todáhsdǫh.\\
somewhere he.is.hiding.there\\
\glt ‘he is hiding somewhere, someplace.’
\z

\ea
\label{ex:gpart3}
\gll \exemph{Ga̱ˀtoháh} tsǫ: ęse:tsę́iˀ.\\
somewhere only you.will.find.it\\
\glt ‘You will just find it someplace.’
\z

\ea
\label{ex:gpart4}
\gll Si gi̱ˀ-shę́h hwaˀ gwa:díh \exemph{ga̱ˀtoháh} tganí:yǫ:t.\\
over.there maybe this.time side somewhere it.is.hanging\\
\glt ‘Maybe it is hanging somewhere over there.’ (\cite[142]{mithun_watewayestanih_1984}, Satgęh dialogue)
\z

\begin{CayugaRelated}
\item \stem{Gaˀ-} element (referring to a location), p. \pageref{p:[gaˀ-]}\\
\item\textit{Gaˀ-to:hah tohgeh} \trs{thereabouts}, p. \pageref{p:[gaˀ-to:hah tohgeh]}\\
\item\textit{Tęˀ gaˀ-toh} \trs{nowhere}, \trs{not anywhere}, p. \pageref{p:[tęˀ gaˀ-toh]}\\
\item\textit{To:hah} \trs{a place}, \trs{a time}, p. \pageref{p:[to:hah]}
\end{CayugaRelated}

\subsection*{\textbf{Gaˀ-to:hah tohgeh} \trs{thereabouts}} \label{p:[gaˀ-to:hah tohgeh]}
\CayugaRemark{Particle group \textit{ga̱ˀ-to:hah … tohgeh} functions as an “adverb of place” (\sectref{‘Adverbs’ of place}); when used figuratively, it refers to an approximate amount; clause-initial.}

\ea
\label{ex:gpart5}
\gll Ó:, agiˀ giˀ gyę:ˀ a:yę́:ˀ degrǫ́ˀ nigahwi̱hsda:gé: gęh, \textbf{ga̱ˀtó:hah} neˀ enesdanyaˀktaˀ \textbf{tohgéh} degahwi̱hsda:gé: hne:ˀ neˀ gajíhwaˀ.\\
oh, I.said just that.one I.guess eight dollars Q, somewhere the saw there two.dollars also the hammer\\
\glt ‘Oh, I guess I said about eight dollars for the saw, didn’t I, and two dollars for the hammer.’ (\cite[159]{mithun_watewayestanih_1984}, Enǫhsǫnyaˀdaˀsǫ́:ˀǫh dialogue)
\z

\begin{CayugaRelated}
\item \stem{Gaˀ-} element (referring to a location), p. \pageref{p:[gaˀ-]}\\
\item\textit{Gaˀ-to:hah} \trs{somewhere}, \trs{someplace}, \trs{around}, \trs{anywhere}, \trs{thereabouts}, p. \pageref{p:[gaˀ-to:hah]}\\
\item\textit{Tęˀ gaˀ-toh} \trs{nowhere}, \trs{not anywhere}, p. \pageref{p:[tęˀ gaˀ-toh]}\\
\item\textit{To:hah} \trs{a place}, \trs{a time}, p. \pageref{p:[to:hah]}
\end{CayugaRelated}

\subsection*{\textbf{Gaę} \trs{which}} \label{p:[gaę-]}
\CayugaRemark{Particle modifying nouns or words functioning as nouns; does not occur alone; phrase- or clause-initial; asks for information about a specific person, place, or object (from among a set of people, places, or objects).}

\begin{CayugaRelated}
\item \textit{Gaę gwaˀ gwa:dih} \trs{whichever way}, \trs{whichever side}, p. \pageref{p:[gaę gwaˀ gwa:dih]}\\
\item \textit{Gaę gwaˀ hǫ:weh} \trs{somewhere}, \trs{someplace}, \trs{wherever}, \trs{around}, p. \pageref{p:[gaę gwaˀ hǫ:weh]}\\
\item \textit{Gaę gwaˀ ni-} \trs{whichever one (of several)}, \trs{whichever person}, p. \pageref{p:[gaę gwaˀ ni- ]}\\
\item \textit{Gaę gwaˀ ni-noun} \trs{wherever \textsc{noun} is}, p. \pageref{p:[gaę gwaˀ ni-noun]}\\
\item \textit{Gaę gwaˀ tsǫ:} \trs{wherever}, \trs{somewhere}, p. \pageref{p:[gaę gwaˀ tsǫ:]}\\
\item \textit{Gaę hǫ:weh} \trs{which place}, \trs{where}, p. \pageref{p:[gaę hǫ:weh]}\\
\item \textit{Gaę niyó:weˀ} \trs{how far}, \trs{which distance}, p. \pageref{p:[gaę niyó:weˀ]}\\
\item \textit{Hę:gyeh gaę hǫ:weh} \trs{no matter where}, \trs{no matter which place}, p. \pageref{p:[hę:-gyeh gaę hǫ:weh]}
\end{CayugaRelated}

\subsection*{\textbf{Gaę gwaˀ gwa:dih} \trs{whichever way}, \trs{whichever side}} \label{p:[gaę gwaˀ gwa:dih]}
\CayugaRemark{Particle group \textbf{gaę gwaˀ gwa:dih} functions as an “adverb of place” (\sectref{‘Adverbs’ of place}); clause-initial; \textit{gwa:dih} can also be spelled or pronounced as \textit{gwai}; a following verb likely requires the \stem{ni-} {\partitive}, \stem{d-} {\cislocative}, or \stem{heˀ-} {\translocative} prefix.}

\ea
\label{ex:gpart6}
\gll Ętsatwada:séˀ \textbf{gaę-gwaˀ} \textbf{gwai} \textbf{h}ęhse:ˀ dęhsatga̱hdǫ́:nyǫˀ.\\
you.will.come.around whichever way you.will.go you.will.take.a.look\\
\glt ‘Whichever way you go, look around.’ (\cite{mithun_how_1980})
\z

\begin{CayugaRelated}
\item \textit{Gaę} \trs{which}, p. \pageref{p:[gaę-]}\\
\item \textit{Gaę gwaˀ hǫ:weh} \trs{somewhere}, \trs{someplace}, \trs{wherever}, \trs{around}, p. \pageref{p:[gaę gwaˀ hǫ:weh]}\\
\item \textit{Gaę gwaˀ ni-} \trs{whichever one (of several)}, \trs{whichever person}, p. \pageref{p:[gaę gwaˀ ni- ]}\\
\item \textit{Gaę gwaˀ ni-noun} \trs{wherever \textsc{noun} is}, p. \pageref{p:[gaę gwaˀ ni-noun]}\\
\item \textit{Gaę gwaˀ tsǫ:} \trs{wherever}, \trs{somewhere}, p. \pageref{p:[gaę gwaˀ tsǫ:]}\\
\item \textit{Gwaˀ} \trs{immediately}, \trs{right then}, \trs{just then}, \trs{finally}, p. \pageref{p:[gwaˀ] ‘immediately’}\\
\item \textit{Gwa:dih, Gwai} \trs{to one side}, p. \pageref{p:[gwa:dih]}
\end{CayugaRelated}


\subsection*{\textbf{Gaę gwaˀ hǫ:weh} \trs{somewhere}, \trs{someplace}, \trs{wherever}, \trs{around}} \label{p:[gaę gwaˀ hǫ:weh]}
\CayugaRemark{Particle group \textit{gaę gwaˀ…hǫ:(weh)} functions as an “adverb of place” (\sectref{‘Adverbs’ of place}); clause-initial; the following verb requires the \stem{ni-} {\partitive}, \stem{d-} {\cislocative}, or \stem{heˀ-} {\translocative} prefix; \textit{hǫ:weh} can be spelled or pronounced as \textit{hǫ:}, \textit{nhǫ:}, \textit{hǫ:weh}, or \textit{nhǫ:weh}. \textit{Gaęgwaˀ hǫ:weh} can be used in a figurative sense to mean ‘around’ or ‘about’ \xxref{ex:gpart12}{ex:gpart13}}

\ea
\label{ex:gpart7}
\gll \textbf{Gaę-gwaˀ} \textbf{nhǫ:weh} tǫwaˀnigǫ́ha:ˀ.\\
which-just.there place she.waits.for.him.there\\
\glt ‘She waits for him somewhere, someplace.’
\z

\ea
\label{ex:gpart8}
\gll To hǫ́: iwá:kˀah to hǫ́: gętsgo:t \textbf{gaę-gwa}ˀ \textbf{nhǫ́:}.\\
there place near there place it.sits which-just.there place\\
\glt ‘Near there somewhere is where it will be.’ (\cite{mithun_how_1980})
\z


\ea
\label{ex:gpart9}
\gll O:, tę́ˀ giˀ gwahs a:yę́:ˀ de̱ˀagęnǫhdǫ́ˀ gaé nhǫ́: dǫgáhdǫ:ˀ, \textbf{gaę-gwáˀ} giˀ \textbf{nhǫ́:} nę́:-toh.\\
oh, not just really it.seems I.do.not.know which place I.lost.it, which-just.there just place that.one-there.\\
\glt ‘Oh, I don’t really seem to know where I lost it, somewhere.’ (\cite[184]{mithun_watewayestanih_1984}, Gatgwęˀdaˀ dialogue)
\z

\ea
\label{ex:gpart11}
\gll Gyotgǫ́:t gę:s neˀ tgá:gǫ:t hęháha:ˀ neˀ \textbf{gaę-gwaˀ} \textbf{hǫ́:} hęhé:ˀ.\\
always usually the it.is.necessary he.will.take.it.there the which-just.there place he.went.there\\
\glt ‘He always had to take it with him wherever he went.’ (\cite{henry_de_2005})
\z

\ea
\label{ex:gpart12} Saleslady: \\
\gll Do: niyohshe:dę́h se̱hstaˀ?\\ 
{} how it.is.numbered you.use.it \\
\glt ‘What size do you wear?’ 

Lila: \\
\gll Ó:, \textbf{gaę-gwaˀ} \textbf{nhǫ́:} neˀ géi sga̱heˀ ǫ: niyohshé:dęh. \\
Oh, which-just.there place the four tens speculate a.certain.number \\
\glt `Oh, somewhere around size fourteen.’ (\cite[225]{mithun_watewayestanih_1984}, Agyaˀdawíˀtraˀ dialogue)
\z


\ea
\label{ex:gpart13}
\gll Haoˀ-dę̱ˀnyóh. Ętgǫhnǫkséˀ giˀ gyę́:ˀ \textbf{gaę-gwaˀ-nhǫ́:} neˀ jadahkshǫ́:ˀǫh.\\
ok. I’ll.pick.you.up just then which-just.there-place the seven-ish\\
\glt ‘All right. I’ll pick you up somewhere around sevenish.’ (\cite[349]{mithun_watewayestanih_1984}, Dęˀ Hoˀdęˀ Nǫ̱sa:gye:ˀ? dialogue)
\z

\begin{CayugaRelated}
\item \textit{Gaę} \trs{which}, p. \pageref{p:[gaę-]}\\
\item \textit{Gaę gwaˀ gwa:dih} \trs{whichever way}, \trs{whichever side}, p. \pageref{p:[gaę gwaˀ gwa:dih]}\\
\item \textit{Gaę gwaˀ ni-} \trs{whichever one (of several)}, \trs{whichever person}, p. \pageref{p:[gaę gwaˀ ni- ]}\\
\item \textit{Gaę gwaˀ ni-noun} \trs{wherever \textsc{noun} is}, p. \pageref{p:[gaę gwaˀ ni-noun]}\\
\item \textit{Gaę gwaˀ tsǫ:} \trs{wherever}, \trs{somewhere}, p. \pageref{p:[gaę gwaˀ tsǫ:]}\\
\item \textit{Gwaˀ} \trs{immediately}, \trs{right then}, \trs{just then}, \trs{finally}, p. \pageref{p:[gwaˀ] ‘immediately’}\\
\item \textit{Gwa:dih, Gwai} \trs{to one side}, p. \pageref{p:[gwa:dih]}\\
\item \textit{Hǫ:weh} \trs{where}, p. \pageref{p:[hǫ:weh]}
\end{CayugaRelated}

\subsection*{\textbf{Gaę gwaˀ ni-} \trs{whichever one (of several)}, \trs{whichever person}} \label{p:[gaę gwaˀ ni- ]}
\CayugaRemark{Particle group \stem{gaę (gwaˀ) … ni-} functions as a definite pronoun (\sectref{ch:’Definite pronouns’}).}

\ea
\label{ex:gpart14}
\gll \textbf{gaę} \textbf{(gwaˀ)} \textbf{ní:}waˀ\\
which right.then it.is.a.certain.size\\
\glt ‘whichever one (object)’
\z

\ea
\label{ex:gpart15}
\gll \textbf{gaę} \textbf{(gwaˀ)} \textbf{ní:}ga:ˀ\\
which right.then it.is.contained.in.something\\
\glt ‘whichever one (object)’
\z

\ea
\label{ex:gpart16}
\gll \textbf{gaę} \textbf{ni}yéya̱ˀda:ˀ\\
which someone’s.body.is.contained.in.it\\
\glt ‘which person’, ‘which woman’
\z

\ea
\label{ex:gpart17}
\gll Ó:, a:yę́:ˀ ní:ˀ né:ˀ gwahs knǫ̱hweˀs neˀ hehsháęˀ niyohso̱hgoˀdę:, né:ˀ gi̱ˀ-shę́h neˀ ojiˀtgwa:gę́:tˀah, jiˀtgwá:ˀ gi̱ˀ-shę́h, otgwęhji̱ˀa:gę́:t gi̱ˀ-shę́h. \textbf{Gaę} \textbf{gwáˀ} giˀ \textbf{ni:}gá: nę́:gyęh.\\
oh, I.guess I it.is really I.like t he brown colour, it.is maybe the yellow.one, yellow maybe, pink maybe. whichever really just it.is.contained.in.something this.one\\
\glt ‘Oh, the one I like the best is brown, or perhaps a light yellow, or maybe yellow, or maybe pink. Anyway, one of these.’ (\cite[225]{mithun_watewayestanih_1984}, Agyaˀdawíˀtraˀ dialogue)
\z

\ea
\label{ex:gpart18}
\gll To: neˀ gayá:ˀ, neˀ gę:s aˀeno̱hái:ˀ, tohgéh \textbf{gaę} \textbf{gwaˀ} \textbf{ní:}waˀ to: neˀ gayá:ˀ aˀehsrǫ́:niˀ.\\
there the bag, the usually she.washed.it, then whichever right.then it.is.a.certain.size there the bag she.fixed.it\\
\glt ‘…she would wash the flour bag and fix it.’ (use it for a tick/mattress) (\cite{henry_de_2005})
\z

\ea
\label{ex:gpart19}
\gll \textbf{Gaę} \textbf{ní:}ga:ˀ i:séˀ á:se:k?\\
which it.is.contained.in.something you.want you.should.eat\\
\glt ‘Which one do you want to eat?’
\z

\begin{CayugaRelated}
\item \textit{Gaę} \trs{which}, p. \pageref{p:[gaę-]}\\
\item \textit{Gaę gwaˀ gwa:dih} \trs{whichever way}, \trs{whichever side}, p. \pageref{p:[gaę gwaˀ gwa:dih]}\\
\item \textit{Gaę gwaˀ hǫ:weh} \trs{somewhere}, \trs{someplace}, \trs{wherever}, \trs{around}, p. \pageref{p:[gaę gwaˀ hǫ:weh]}\\
\item \textit{Gaę gwaˀ ni-noun} \trs{wherever \textsc{noun} is}, p. \pageref{p:[gaę gwaˀ ni-noun]}\\
\item \textit{Gaę gwaˀ tsǫ:} \trs{wherever}, \trs{somewhere}, p. \pageref{p:[gaę gwaˀ tsǫ:]}\\
\item \textit{Gwaˀ} \trs{immediately}, \trs{right then}, \trs{just then}, \trs{finally}, p. \pageref{p:[gwaˀ] ‘immediately’}\\
\item \textit{Gwa:dih, Gwai} \trs{to one side}, p. \pageref{p:[gwa:dih]}
\end{CayugaRelated}


\subsection*{\textbf{Gaę gwaˀ ni-\textsc{noun}} \trs{wherever \textsc{noun} is}} \label{p:[gaę gwaˀ ni-noun]}
\CayugaRemark{Particle group \stem{gaęgwaˀ … ni-noun} functions as an “adverb of place” (\sectref{‘Adverbs’ of place}); means that an object exists ‘wherever’; clause-initial.}

\ea
\label{ex:gpart20}
\gll I:wí: neˀ \textbf{gaę-gwáˀ} \textbf{ni}gátseˀdáˀ to: aˀakninǫ́:nheht,\\
I.think the which-just.there some.bottle there we.filled.it\\
\glt ‘I think that wherever the bottle (was), we filled it there,…’ (\cite{henry_de_2005})
\z


\begin{CayugaRelated}
\item \textit{Gaę} \trs{which}, p. \pageref{p:[gaę-]}\\
\item \textit{Gaę gwaˀ gwa:dih} \trs{whichever way}, \trs{whichever side}, p. \pageref{p:[gaę gwaˀ gwa:dih]}\\
\item \textit{Gaę gwaˀ hǫ:weh} \trs{somewhere}, \trs{someplace}, \trs{wherever}, \trs{around}, p. \pageref{p:[gaę gwaˀ hǫ:weh]}\\
\item \textit{Gaę gwaˀ ni-} \trs{whichever one (of several)}, \trs{whichever person}, p. \pageref{p:[gaę gwaˀ ni- ]}\\
\item \textit{Gaę gwaˀ tsǫ:} \trs{wherever}, \trs{somewhere}, p. \pageref{p:[gaę gwaˀ tsǫ:]}\\
\item \textit{Gwaˀ} \trs{immediately}, \trs{right then}, \trs{just then}, \trs{finally}, p. \pageref{p:[gwaˀ] ‘immediately’}\\
\item \textit{Gwa:dih, Gwai} \trs{to one side}, p. \pageref{p:[gwa:dih]}
\end{CayugaRelated}


\subsection*{\textbf{Gaę gwaˀ tsǫ:} \trs{wherever}, \trs{somewhere}} \label{p:[gaę gwaˀ tsǫ:]}
\CayugaRemark{Particle group functioning as an “adverb of place” (\sectref{‘Adverbs’ of place}); likely clause-initial. A following verb likely requires the \stem{ni-} {\partitive}, \stem{d-} {\cislocative}, or \stem{heˀ-} {\translocative} prefix.}

\ea
\label{ex:gpart29}
\gll gaę gwaˀ tsǫ:\\
which emphasis just\\
\glt ‘wherever’, ‘somewhere’
\z

\begin{CayugaRelated}
\item \textit{Gaę} \trs{which}, p. \pageref{p:[gaę-]}\\
\item \textit{Gaę gwaˀ gwa:dih} \trs{whichever way}, \trs{whichever side}, p. \pageref{p:[gaę gwaˀ gwa:dih]}\\
\item \textit{Gaę gwaˀ hǫ:weh} \trs{somewhere}, \trs{someplace}, \trs{wherever}, \trs{around}, p. \pageref{p:[gaę gwaˀ hǫ:weh]}\\
\item \textit{Gaę gwaˀ ni-} \trs{whichever one (of several)}, \trs{whichever person}, p. \pageref{p:[gaę gwaˀ ni- ]}\\
\item \textit{Gaę gwaˀ ni-noun} \trs{wherever \textsc{noun} is}, p. \pageref{p:[gaę gwaˀ ni-noun]}\\
\item \textit{Gwaˀ} \trs{immediately}, \trs{right then}, \trs{just then}, \trs{finally}, p. \pageref{p:[gwaˀ] ‘immediately’}\\
\item \textit{Gwa:dih, Gwai} \trs{to one side}, p. \pageref{p:[gwa:dih]}\\
\item \textit{Tsǫ:} \trs{just}, \trs{only}, p. \pageref{p:[tsǫ:]}
\end{CayugaRelated}

\subsection*{\textbf{Gaę hǫ:weh} \trs{which place}, \trs{where}} \label{p:[gaę hǫ:weh]}
\CayugaRemark{\textsc{Particle group} \textit{gaę … (hǫ:weh)} functions as an “adverb of place” (\sectref{‘Adverbs’ of place}), clause-initial in direct questions \xxref{ex:gpart21}{ex:gpart25} and dependent clauses \xxref{ex:gpart26}{ex:gpart27}; \textit{hǫ:weh} can be spelled or pronounced as \textit{hǫ:}, \textit{nhǫ:}, \textit{hǫ:weh}, or \textit{nhǫ:weh}; (also see \sectref{ch:Clauses with [hǫ:(weh)] ‘the place where’}).}

\ea
\label{ex:gpart21}
\gll \exemph{Gaę} hne:ˀ sagyaˀdawiˀtraˀ? \\
Where in.fact your.coat\\
\glt ‘Where is your coat?’ (\cite[142]{mithun_watewayestanih_1984}, Satgęh dialogue)
\z

\ea
\label{ex:gpart22}
\gll \exemph{Gaę} nǫdahse:? \\
Where you.come.from.a.certain.place\\
\glt ‘Where do you come from?’
\z

\ea
\label{ex:gpart23}
\gll \exemph{Gaę} diˀ \exemph{nhǫ:wéh} ihsweˀs o:nęh?\\
Which so place you.are now\\
\glt ‘Where then are you now?’
\z

\ea
\label{ex:gpart24}
\gll \exemph{Gaę:} \exemph{nhǫ:} tsiˀdrǫˀ?\\
Which place you.live.there\\
\glt ‘Where do you live?’
\z

\ea
\label{ex:gpart25}
\gll \exemph{Gaę} hwaˀ \exemph{nhǫ:} hejisaihoˀdeˀ? \\
Which this.time place you.work.over.there\\
\glt ‘Where do you work?’
\z

\ea
\label{ex:gpart26}
\gll Tę́ˀ de̱ˀagęnǫhdǫ́ˀ [\exemph{gaę} \exemph{nhǫ́:} nigá:yęˀ]. \\
Not I.don’t.know which place it.is.placed.somewhere\\
\glt ‘I don’t know where it is.’ (\cite[142]{mithun_watewayestanih_1984}, Satgęh dialogue)
\z

\ea
\label{ex:gpart27}
\gll Tę́ˀ giˀ ní:ˀ de̱ˀge:gę́: [\exemph{gaę}-ˀǫ \exemph{nhǫ́:} desáhdǫ:ˀ].\\
Not just I I.didn’t.see.it which-I.wonder place you.lost.it\\
\glt ‘Well, I didn’t see where you lost it.’ (\cite[184]{mithun_watewayestanih_1984}, Gatgwęˀdaˀ dialogue)
\z

\begin{CayugaRelated}
\item \textit{Gaę} \trs{which}, p. \pageref{p:[gaę-]}\\
\item \textit{Hǫ:weh} \trs{where}, p. \pageref{p:[hǫ:weh]}
\end{CayugaRelated}

\subsection*{\textbf{Gaę niyó:weˀ} \trs{how far}, \trs{which distance}} \label{p:[gaę niyó:weˀ]}
\CayugaRemark{Particle and verb functioning as an “adverb of place” (\sectref{‘Adverbs’ of place}); clause-initial in direct questions.}

\ea
\label{ex:gpart28}
\gll \textbf{Gaę} \textbf{niyo:wéˀ} heha:wé:nǫ:.\\
which it.is.a.certain.distance he.has.gone.there\\
\glt ‘How far has he gone?’
\z

\begin{CayugaRelated}
\item \textit{Gaę} \trs{which}, p. \pageref{p:[gaę-]}
\end{CayugaRelated}

\subsection*{\textbf{Gaę niyo:weˀ tsǫ:} \trs{whenever}} \label{p:[gaę niyo:weˀ tsǫ:]}
\CayugaRemark{Particle group functioning as an “adverb of time” (\sectref{ch:’Adverbs’ of time}); likely clause-initial in dependent clauses. (Reported in (\cite{sasse_far_1998}), who attribute the phrase to a list provided by Michael Foster.)}

\ea
\label{ex:gpart285}
\gll gaę niyo:weˀ tsǫ:\\
which it.is.a.certain.distance just\\
\glt ‘whenever’
\z


\begin{CayugaRelated}
\item \textit{Gaę} \trs{which}, p. \pageref{p:[gaę-]}
\end{CayugaRelated}

\subsection*{\textbf{Gaoˀ} \trs{this side}, \trs{this way}} \label{p:[gaoˀ]}
\CayugaRemark{Particle functioning as an “adverb of place” (\sectref{‘Adverbs’ of place}); clause-initial.}

\ea
\label{ex:gpart30}
\gll \textbf{Gaoˀ} nǫdáhse:ˀ.\\
this.way you.will.come.a.certain.way\\
\glt ‘Come this way.’
\z

\ea
\label{ex:gpart31}
\gll \textbf{Gaoˀ} dǫdé:swe:!\\
this.way you.all.return\\
\glt ‘Come back!’ (\cite{carrier_legends_2013})
\z

\ea
\label{ex:gpart32}
\gll Da:nę:dah haˀgahéˀ \textbf{gaoˀ} ędihswatríhs ęhswada̱hǫhsí:yohs.\\
and.now it.is.time this.way you.all.will.come.closer you.all.will.listen\\
\glt ‘Now is the time to come closer and listen.’
\z

\begin{CayugaRelated}
\item \textit{Gaoˀ nawahtgeh} \trs{the time before then}, p. \pageref{p:[gaoˀ nawahtgeh]}\\
\item \textit{Gaoˀ ni-} \trs{less so}, p. \pageref{p:[gaoˀ ni-]}\\
\item \textit{Gaoˀ shęh niyo:weˀ} \trs{before}, \trs{until}, p. \pageref{p:[gaoˀ shęh niyo:weˀ]}\\
\item \textit{I:nǫh gaoˀ neˀ} \trs{far from enough}, p. \pageref{p:[i:nǫh gaoˀ neˀ]}
\end{CayugaRelated}

\subsection*{\textbf{Gaoˀ nawahtgeh} \trs{the time before then}} \label{p:[gaoˀ nawahtgeh]}
\CayugaRemark{Particle group functioning as an “adverb of time” (\sectref{ch:’Adverbs’ of time}); clause-initial.}

\ea
\label{ex:gpart33}
\gll \textbf{Gaoˀ} \textbf{nawahtgéh} tsaˀgaǫdrí:yoˀ…\\
towards the.time.before when.they.fought\\
\glt ‘The time before, when they fought…’
\z


\begin{CayugaRelated}
\item \textit{Gaoˀ} \trs{this side}, \trs{this way}, p. \pageref{p:[gaoˀ]}\\
\item \textit{Gaoˀ ni-} \trs{less so}, p. \pageref{p:[gaoˀ ni-]}\\
\item \textit{Gaoˀ shęh niyo:weˀ} \trs{before}, \trs{until}, p. \pageref{p:[gaoˀ shęh niyo:weˀ]}
\end{CayugaRelated}


\subsection*{\textbf{Gaoˀ ni-} \trs{less so}} \label{p:[gaoˀ ni-]}
\CayugaRemark{Particle group \stem{gaoˀ … ni-stative.verb} is used in comparisons (\sectref{ch:Comparisons (more, the same, or less)}); \textit{gaoˀ} precedes a stative verb beginning with \stem{ni-} {\partitive}.}

\ea
\label{ex:gpart34}
\gll \textbf{gaoˀ} \textbf{ní:}waˀ\\
less.so it.is.small\\
\glt ‘it is smaller than…’
\cfex{ní:-w-aˀ {\partitive}-\textsc{3s.a}-be.a.certain.size.{\stative}}
\z


\ea
\label{ex:gpart35}
\gll \textbf{gaoˀ} \textbf{ni}wakyędáhkwaˀ\\
less.so a.certain.size.of.chair\\
\glt ‘a smaller chair’
\cfex{ni-w-akyędáhkw-aˀ {\partitive}-\textsc{3s.a}-chair-be.a.certain.size.{\stative}}
\z


\ea
\label{ex:gpart36}
\gll Ne:ˀ \textbf{gaoˀ} \textbf{ni:}yǫ́: dęhsátahahk\\
it.is less.so a.certain.number you.will.walk\\
\glt ‘You have to walk a lot less.’ (\cite{mithun_how_1980})
\cfex{ni:-y-ǫ́: {\partitive}-\textsc{3s.p}-certain.number.{\stative}}
\z


\ea
\label{ex:gpart37}
\gll Ne:ˀ tsǫ́: hné:ˀ sa:dǫ́h shęh [\textbf{gaoˀ} ahsǫ́h \textbf{ni}wago̱hsríya̱ˀgǫh]…\\
it.is just in.fact you.say that less.so still I.am.some.years.old\\
\glt ‘That’s not fair. Just because I am younger than you.’ (\cite{carrier_legends_2013})
\cfex{ni-wag-o̱hsr-íya̱ˀg-ǫh {\partitive}-\textsc{1s.p}-winter-cross-{\stative}}
\z


\ea
\label{ex:gpart38}
\gll Aǫhę:ˀę́h hniˀ gyogaˀǫ́h shęh [\textbf{gaoˀ} \textbf{ni}tode̱ˀnyę́:dę:]\\ 
the.most and it.tastes.good that less.so how.he.tasted.it\\
\glt ‘And it was sweeter than anything he had ever tasted.’ (\cite{carrier_legends_2013})
\cfex{ni-t-ho-d-e̱-ˀnyę́:dę-: \partitive-\cislocative-\textsc{3s.m.p}-{\semireflexive}-{\joinerE}-try-{\stative}}
\z


\begin{CayugaRelated}
\item \textit{Gaoˀ} \trs{this side}, \trs{this way}, p. \pageref{p:[gaoˀ]}\\
\item \textit{Gaoˀ nawahtgeh} \trs{the time before then}, p. \pageref{p:[gaoˀ nawahtgeh]}\\
\item \textit{Gaoˀ shęh niyo:weˀ} \trs{before}, \trs{until}, p. \pageref{p:[gaoˀ shęh niyo:weˀ]}
\end{CayugaRelated}


\subsection*{\textbf{Gaoˀ shęh niyo:weˀ, Shęh niyo:weˀ} \trs{before}, \trs{until}} \label{p:[gaoˀ shęh niyo:weˀ]}
\CayugaRemark{Particle group \textit{(gaoˀ) shęh niyo:weˀ} functions as an “adverb of time” (\sectref{ch:’Adverbs’ of time}); clause-initial in dependent clauses; \textit{niyó:weˀ} also has the short forms \textit{ní:yo:ˀ} or \textit{nyo:}. For more examples, see \sectref{ch:Clauses with [(gaoˀ) shęh niyo:weˀ] ‘before, until’}.}

\ea
\label{ex:gpart39}
\gll \textbf{gaoˀ} \textbf{shę} \textbf{nyó:ˀ} to: nęyá:węh…\\
which that it.is.a.certain.distance that it.will.happen…\\
\glt ‘before that happens…’ (sentence fragment)
\z


\ea
\label{ex:gpar40}
\gll O:nę́h aˀa:gę́ˀ, “Waˀgyę́h,” aˀa:gę́ˀ, “ęgehsrǫ:níˀ i:wí: tó: neˀ sageˀaˀgéh [\textbf{gaoˀ} \textbf{shę} \textbf{nyó:ˀ}  ęhsę́da̱ˀdraˀ.]”\\
now she.said just.now she.said I.will.fix.it I.want that the on.your.hair which that it.is.a.certain.distance you.will.go.to.bed\\
\glt ‘“Just a minute,” she said, “I want to fix your hair right there before you go to bed”.’ (\cite{henry_de_2005})
\z


\begin{CayugaRelated}
\item \textit{Gaoˀ} \trs{this side}, \trs{this way}, p. \pageref{p:[gaoˀ]}\\
\item \textit{Gaoˀ nawahtgeh} \trs{the time before then}, p. \pageref{p:[gaoˀ nawahtgeh]}\\
\item \textit{Gaoˀ ni-} \trs{less so}, p. \pageref{p:[gaoˀ ni-]}\\
\item \textit{Niyo:weˀ, Ni:yo:ˀ, Nyo:ˀ} \trs{a certain distance}, p. \pageref{p:[niyo:weˀ]}\\
\item \textit{Shęh niyo:weˀ} \trs{as far as}, \trs{as much as}, p. \pageref{p:[shęh niyo:weˀ]}\\
\item \textit{Shęh niyo:weˀ, Gaoˀ shęh niyo:weˀ} \trs{before}, \trs{until}, p. \pageref{p:[shęh niyo:weˀ, gaoˀ shęh niyo:weˀ]}\\
\item \textit{Toh niyo:weˀ} \trs{that far}, p. \pageref{p:[toh niyo:weˀ]}\\
\item \textit{To: niyo:weˀ ne:ˀ} \trs{when}, p. \pageref{p:[to: niyo:weˀ ne:ˀ]}
\end{CayugaRelated}


\subsection*{\textbf{Gaǫhęˀ, Gaǫhaˀ} \trs{she}} \label{p:[gaǫhęˀ]}
\CayugaRemark{Verb functioning as an “emphatic pronoun” (\sectref{ch:’Emphatic pronouns’}); free-standing.}

\ea
\label{ex:gpar41}
\glll gáǫhęˀ\\
ga-ǫhęˀ\\
\textsc{3s.fi.p}-alone.stative\\
\glt ‘she is alone’
\z


\subsection*{\textbf{Ga:t giˀ shęh} \trs{maybe}, \trs{or maybe}} \label{p:[ga:t giˀ shęh]}
\CayugaRemark{Particle group \textit{…(ga:t)-giˀ-shęh} functions as a doubt or certainty marker (\sectref{ch:Evidential markers}); enclitic.}

\ea
\label{ex:gpar42}
\gll ga:t-giˀ shęh\\
just that\\
\glt ‘maybe’, ‘or maybe’
\z


\ea
\label{ex:gpar43}
\gll To \textbf{giˀ} \textbf{shęh} há:ge:.\\
there just that I.should.go.there\\
\glt ‘Maybe I should go there.’
\z


\ea
\label{ex:gpar44}
\gll Dęgyadawęnyeháˀ [gyę:ˀgwáˀ \textbf{giˀ} \textbf{shęh} hniˀ ęgyádawę:ˀ].\\
we.will.walk.about if just that and we.will.swim\\
\glt ‘We could go for a walk or maybe go for a swim.’ (\cite[45]{michelson_ontario_2011})
\z


\begin{CayugaRelated}
\item \textit{Ga:t giˀ shęh tęˀ} \trs{maybe not}, \trs{or not}, p. \pageref{p:[ga:t giˀ shęh tęˀ]}\\
\item \textit{Gęh ga:t giˀ shęh nigęˀǫh} \trs{or?}, p. \pageref{p:[gęh ga:t giˀ shęh nigęˀǫh]}\\
\item \textit{Giˀ} \trs{just}, p. \pageref{p:[giˀ]}\\
\item \textit{O: tęˀ ǫh ga:t giˀ shęh gyę:gwaˀ} \trs{maybe, maybe not}, p. \pageref{p:[o: tęˀ ǫh ga:t giˀ shęh gyę:gwaˀ]}\\
\item \textit{O: tęˀ ǫh ga:t giˀ shęh ne:ˀ hwaˀ} \trs{maybe, maybe not}, p. \pageref{p:[o: tęˀ ǫh ga:t giˀ shęh ne:ˀ hwaˀ]}\\
\item \textit{Shęh} \trs{that}, \trs{because}, p. \pageref{p:[shęh] `because’}\\
\item \textit{Tęˀ} \trs{no}, \trs{not}, p. \pageref{p:[tęˀ]}\\
\item \textit{Tęˀ giˀ shęh hwaˀ daˀǫ} \trs{maybe not}, p. \pageref{p:[tęˀ giˀ shęh hwaˀ daˀǫ]}
\end{CayugaRelated}


\subsection*{\textbf{Ga:t giˀ shęh tęˀ} \trs{maybe not}, \trs{or not}} \label{p:[ga:t giˀ shęh tęˀ]}
\CayugaRemark{Particle group \textit{(ga:t) giˀ shęh tęˀ} functions as a doubt or certainty marker (\sectref{ch:Evidential markers}); free-standing expression.}

\ea
\label{ex:gpar45}
\gll ga:t giˀ shęh tęˀ\\
{} just that not\\
\glt ‘maybe not’, ‘or not’
\z


\begin{CayugaRelated}
\item \textit{Ga:t giˀ shęh} \trs{maybe}, \trs{or maybe}, p. \pageref{p:[ga:t giˀ shęh]}\\
\item \textit{Gęh ga:t giˀ shęh nigęˀǫh} \trs{or?}, p. \pageref{p:[gęh ga:t giˀ shęh nigęˀǫh]}\\
\item \textit{Giˀ} \trs{just}, p. \pageref{p:[giˀ]}\\
\item \textit{O: tęˀ ǫh ga:t giˀ shęh gyę:gwaˀ} \trs{maybe, maybe not}, p. \pageref{p:[o: tęˀ ǫh ga:t giˀ shęh gyę:gwaˀ]}\\
\item \textit{O: tęˀ ǫh ga:t giˀ shęh ne:ˀ hwaˀ} \trs{maybe, maybe not}, p. \pageref{p:[o: tęˀ ǫh ga:t giˀ shęh ne:ˀ hwaˀ]}\\
\item \textit{Shęh} \trs{that}, \trs{because}, p. \pageref{p:[shęh] `because’}\\
\item \textit{Tęˀ} \trs{no}, \trs{not}, p. \pageref{p:[tęˀ]}\\
\item \textit{Tęˀ giˀ shęh hwaˀ daˀǫ} \trs{maybe not}, p. \pageref{p:[tęˀ giˀ shęh hwaˀ daˀǫ]}
\end{CayugaRelated}


\subsection*{\textbf{Gęh} \trs{Q} (question marker)} \label{p:[gęh] `Q’}
\CayugaRemark{Particle \textit{… gęh} `Q’ is used with questions that imply \textit{ęhęˀ} `yes’ or \textit{tęˀ} `no’ answers; used with direct questions (shown below), indirect questions, and dependent conditional clauses (\sectref{Yes-no questions with [gęh]}); enclitic. Translated as ‘whether’ or ‘if’ when used in indirect yes-no questions, and dependent clauses.}

\ea
\label{ex:gpar46}
\gll Sgęnǫ́jih \textbf{gęh}?\\
you.are.well Q\\
\glt ‘Are you well?’
\z


\ea
\label{ex:gpar47}
\gll Ędwé:ˀ \textbf{gęh}?\\
we.will.go.together Q\\
\glt ‘Are you coming along with us?’
\z


\ea
\label{ex:gpar48}
\gll Ęhsnéˀ \textbf{gęh} diˀ?\\
we.will.go Q so\\
\glt ‘Are you coming along?’
\z


\ea
\label{ex:gpar49}
\gll I:ˀ \textbf{gęh} sgí:dǫh?\\
I Q you.mean.me\\
\glt ‘Do you mean me?’
\z


\ea
\label{ex:gpar50}
\gll Ahsǫ́h \textbf{gęh}?\\
more Q\\
\glt ‘Do you want some more?’
\z


\begin{CayugaRelated}
\item \textit{Ahsǫh gęh} \trs{Do you want some more?}, p. \pageref{p:[ahsǫh gęh]}\\
\item \textit{Do:gęhs diˀ gęh} \trs{it is true}, \trs{isn’t it true?}, p. \pageref{p:[do:gęhs diˀ gęh]}\\
\item \textit{Ehęˀ gęh} \trs{is that right?}, p. \pageref{p:[ęhęˀ gęh]}\\
\item \textit{Ewa:dǫˀ gęh} \trs{may I}, \trs{may we}, etc., p. \pageref{p:[ęwa:dǫˀ gęh]}\\
\item \textit{Gęh} \trs{whether}, \trs{if}, p. \pageref{p:[gęh] ‘whether’, ‘if’}\\
\item \textit{Gęh} \trs{didn’t I}, p. \pageref{p:[gęh] `didn’t I’}\\
\item \textit{Gęh} \trs{mind you}, p. \pageref{p:[gęh] ‘mind you’}\\
\item \textit{Gęh ga:t giˀ shęh nigęˀǫh} \trs{or?}, p. \pageref{p:[gęh ga:t giˀ shęh nigęˀǫh]}\\
\item \textit{Gęh hne:ˀ} \trs{how about this one?}, p. \pageref{p:[gęh hne:ˀ]}\\
\item \textit{Gęh tęˀ nigęˀǫh} \trs{or not?}, p. \pageref{p:[gęh tęˀ nigęˀǫh]}\\
\item \textit{Ne:ˀ diˀ gęh} \trs{is that it then?}, \trs{is that}, p. \pageref{p:[ne:ˀ diˀ gęh]}\\
\item \textit{Ne:ˀ gęh … ne:ˀ / neˀ nigęˀǫh neˀ …} \trs{or?}, p. \pageref{p:[ne:ˀ gęh … ne:ˀ]}\\
\item \textit{O: gęh} \trs{really?}, p. \pageref{p:[o: gęh]}\\
\item \textit{Tęˀ gęh} \trs{no?, isn’t it?}, p. \pageref{p:[tęˀ gęh]}\\
\item \textit{Tęˀ gęh deˀ-} \trs{didn’t?}, p. \pageref{p:[tęˀ gęh deˀ-]}\\
\item \textit{To gęh ǫ ne:ˀ} \trs{I wonder if it is}, p. \pageref{p:[to gęh ǫ ne:ˀ]}
\end{CayugaRelated}


\subsection*{\textbf{Gęh} \trs{whether}, \trs{if}} \label{p:[gęh] ‘whether’, ‘if’}
\CayugaRemark{Particle \textit{… gęh} ‘whether, if’ introduces indirect yes-no questions \xref{ex:gpar51} and dependent conditional clauses (\ref{ex:gpar52}, see \sectref{ch:Conditional clauses with [gyę:gwaˀ], [gęh], [hę:gyeh] ‘if, whether’}); enclitic.}

\ea
\label{ex:gpar51}
\gll Daskro:wíh [sanǫhǫkdá:niˀ \textbf{gęh}]?\\
you.tell.me you.are.sick Q\\
\glt ‘Tell me whether you are sick.’
\z


\ea
\label{ex:gpar52}
\gll Ęsgoho:wíˀ [to \textbf{gęh} ne: hę́:yę:ˀ].\\
I.will.tell.you there Q it.is she.will.go.there\\
\glt ‘I will tell you if she is going.’
\z


\begin{CayugaRelated}
\item \textit{Gęh} \trs{Q} (question marker), p. \pageref{p:[gęh] `Q’}\\
\item \textit{Gęh} \trs{didn’t I}, p. \pageref{p:[gęh] `didn’t I’}\\
\item \textit{Gęh} \trs{mind you}, p. \pageref{p:[gęh] ‘mind you’}
\end{CayugaRelated}



\subsection*{\textbf{Gęh} \trs{didn’t I}} \label{p:[gęh] `didn’t I’}
\CayugaRemark{Particle \textit{…gęh} `didn’t I’ is used as a tag in rhetorical questions (which do not require an answer; see \sectref{Tag questions (asking for confirmation from the listener)}); enclitic.}

\ea
\label{ex:gpar53}
\gll Ó:, agiˀ giˀ gyę:ˀ a:yę́:ˀ degrǫ́ˀ nigahwi̱hsda:ge: \textbf{gęh}, ga̱ˀtó:ha neˀ enesdanyaˀktaˀ toh-gé degahwi̱hsda:gé: hne:ˀ neˀ gajíhwaˀ.\\
Oh I.said just that.one it.seems eight dollars, Q, about the saw that.one-on two.dollars in.fact the hammer\\
\glt ‘Oh, I guess I said eight dollars, didn’t.I, for the saw, and two dollars for the hammer.’ (\cite[159]{mithun_watewayestanih_1984}, Enǫhsǫnyaˀdaˀsǫ́:ˀǫh dialogue)
\z

\begin{CayugaRelated}
\item \textit{Gęh} \trs{Q} (question marker), p. \pageref{p:[gęh] `Q’}\\
\item \textit{Gęh} \trs{whether}, \trs{if}, p. \pageref{p:[gęh] ‘whether’, ‘if’}\\
\item \textit{Gęh} \trs{mind you}, p. \pageref{p:[gęh] ‘mind you’}
\end{CayugaRelated}



\subsection*{\textbf{Gęh} \trs{mind you}} \label{p:[gęh] ‘mind you’}
\CayugaRemark{Particle \textit{…gęh} `mind you’ signals shared knowledge (known to both speaker and listener, \sectref{ch:Shared knowledge markers}); enclitic.}

\ea
\label{ex:gpar54}
\gll Ni: \textbf{gę:}(gęh) tó-ne:ˀ ǫgahdǫ́:neˀ agétgwę̱ˀdaˀ.\\
I mind.you it.is.that I.lost.it my.wallet…\\
\glt ‘Mind you, I lost my wallet.’ (\cite[184]{mithun_watewayestanih_1984}, Gatgwę́ˀdaˀ dialogue)
\z


\begin{CayugaRelated}
\item \textit{Gęh} \trs{Q} (question marker), p. \pageref{p:[gęh] `Q’}\\
\item \textit{Gęh} \trs{whether}, \trs{if}, p. \pageref{p:[gęh] ‘whether’, ‘if’}\\
\item \textit{Gęh} \trs{mind you}, p. \pageref{p:[gęh] ‘mind you’}
\end{CayugaRelated}



\subsection*{\textbf{Gęh ga:t giˀ shęh nigęˀǫh} \trs{or?}} \label{p:[gęh ga:t giˀ shęh nigęˀǫh]}
\CayugaRemark{Particle group \textit{gęh …, (ga:t giˀ shęh) … nigęˀǫh} ‘or?’ functions as a conjunction, connecting a list of alternative choices, (\sectref{ch:Clauses with conjunctions}). The alternatives are given between [ ] in the examples below; \textit{gęh} is enclitic and \textit{nigę́ˀǫh} tends to appear at the end of the list of choices.}

\ea
\label{ex:gpar55}
\gll [Hǫ:gwéh] \textbf{gęh} [agǫ:gwéh] \textbf{nigęˀǫ́h} Dan eyá:sǫh?\\
man Q woman or Dan someone.is.named\\
\glt ‘Was it a man or a woman whose name is Dan?’
\z


\ea
\label{ex:gpar56}
\gll Sá:gaˀs \textbf{gęh} shęh niyogaˀǫ́hsro̱ˀdę: neˀ [daksaeˀdóhs] [gyǫnǫhsgwaǫ́t] [gwihsgwíhs oˀwahǫ́h] \textbf{nigę́ˀǫh}?\\
you.like.the.taste.of.it Q that what.it.tastes.like the chicken cow pig meat or\\
\glt ‘Do you like the taste of chicken, beef or pork?’ (p.c., Alfred Keye and Tom Deer)
\z


\ea
\label{ex:gpar57}
\gll [Desatahahkwáˀ] \textbf{gęh} neˀ o:nę́h ęhsadadrihǫnyani̱háˀ [goyaˀdanęhgwíh ęhsatnǫ̱hdá: ęhsehdáhk] \textbf{nigę́ˀǫh}?\\
you.walk Q the when you.will.go.to.school bus you.will.embark you.will.go.by.vehicle or\\
\glt ‘Do you walk or take the bus to get to school?’ (p.c., Alfred Keye and Tom Deer)
\z


\ea
\label{ex:gpar58}
\gll [Desadǫ̱hwę:jo:níh] \textbf{gęh} [trehs] \textbf{nigęˀǫ́h} [gano:ˀ]?\\
you.want.it Q too or expensive\\
\glt ‘Do you want it or is it too expensive (and so you don’t want it)?’ (Michelson and Price, 2011, 45)
\z


\ea
\label{ex:gpar59}
\gll [Seko:níhs \textbf{gęh} neˀ ęhsadekǫ:níˀ oˀga:sˀáh] [(\textbf{ga:t} {\textbf{giˀ} \textbf{shę́h}}) ęhsni:nǫ́ˀ] \textbf{nigę́ˀǫh}?\\
you.cook Q the you.will.eat evening ({} or.maybe) you.will.buy or\\
\glt ‘Do you cook your evening meal or buy it?’ (p.c., Alfred Keye and Tom Deer)
\z


\begin{CayugaRelated}
\item \textit{Ga:t giˀ shęh} \trs{maybe}, \trs{or maybe}, p. \pageref{p:[ga:t giˀ shęh]}\\
\item \textit{Ga:t giˀ shęh tęˀ} \trs{maybe not}, \trs{or not}, p. \pageref{p:[ga:t giˀ shęh tęˀ]}\\
\item \textit{Gęh ga:t giˀ shęh nigęˀǫh} \trs{or?}, p. \pageref{p:[gęh ga:t giˀ shęh nigęˀǫh]}\\
\item \textit{Gęh tęˀ nigęˀǫh} \trs{or not?}, p. \pageref{p:[gęh tęˀ nigęˀǫh]}\\
\item \textit{Giˀ} \trs{just}, p. \pageref{p:[giˀ]}\\
\item \textit{Ne:ˀ gęh … ne:ˀ/neˀ nigęˀǫh neˀ …} \trs{or?}, p. \pageref{p:[ne:ˀ gęh … ne:ˀ]}\\
\item \textit{…Nigęˀǫh} \trs{or} (conjunction), p. \pageref{p:[…nigęˀǫh]}\\
\item \textit{O: tęˀ ǫh ga:t giˀ shęh gyę:gwaˀ} \trs{maybe, maybe not}, p. \pageref{p:[o: tęˀ ǫh ga:t giˀ shęh gyę:gwaˀ]}\\
\item \textit{O: tęˀ ǫh ga:t giˀ shęh ne:ˀ hwaˀ} \trs{maybe, maybe not}, p. \pageref{p:[o: tęˀ ǫh ga:t giˀ shęh ne:ˀ hwaˀ]}\\
\item \textit{Shęh} \trs{that}, \trs{because}, p. \pageref{p:[shęh] `because’}\\
\item \textit{Tęˀ} \trs{no}, \trs{not}, p. \pageref{p:[tęˀ]}\\
\item \textit{Tęˀ giˀ shęh hwaˀ daˀǫ} \trs{maybe not}, p. \pageref{p:[tęˀ giˀ shęh hwaˀ daˀǫ]}
\end{CayugaRelated}


\subsection*{\textbf{Gęh hne:ˀ} \trs{how about this one?}} \label{p:[gęh hne:ˀ]}
\CayugaRemark{Particle group \textit{… gęh hne:ˀ} used with questions that imply \textit{ęhęˀ} \trs{yes} or \textit{tęˀ} \trs{no} answers (\sectref{Yes-no questions with [gęh]}); enclitic; emphatic.}

\ea
\label{ex:gpar60}
\gll Í:s diˀ \textbf{gęh} \textbf{hne:ˀ}?\\
you so Q in.fact\\
\glt ‘How about you?’ `Was that you?’
\z



\begin{CayugaRelated}
\item \textit{Gęh} \trs{Q} (question marker), p. \pageref{p:[gęh] `Q’}\\
\item \textit{Hne:ˀ} \trs{in fact}, p. \pageref{p:[hne:ˀ] ‘in fact’}
\end{CayugaRelated}

\subsection*{\textbf{Gęh oh, Gę-ǫ:, Ga-ǫ: ne:ˀ} \trs{is it that?}, \trs{if it is that}} \label{p:[gęh oh, gę-ǫ:]}
\CayugaRemark{Particle group \textit{gęh ǫh (ne:ˀ)} introduces conditional clauses (\sectref{ch:Conditional clauses with [gyę:gwaˀ], [gęh], [hę:gyeh] ‘if, whether’}); enclitic. \textit{Gęh ǫh} is also spelled as \textit{gęˀǫ}, \textit{gęǫ} or \textit{gaǫ:}.}

\ea
\label{ex:gpar61}
\gll Sanǫhǫkda:níˀ {\textbf{gá-ǫ: / gę́-ǫ:}}.\\
you.are.sick Q-I.guess\\
\glt ‘I wonder if you are sick!’
\z


\begin{CayugaRelated}
\item \textit{Gęh} \trs{Q} (question marker), p. \pageref{p:[gęh] `Q’}\\
\item \textit{Ǫh ne:ˀ} \trs{maybe}, p. \pageref{p:[ǫh ne:ˀ]}\\
\item \textit{Ǫh, ǫ:, ǫ} \trs{I guess}, \trs{I wonder (if)}, p. \pageref{p:[ǫh, ǫ:, ǫ]}\\
\item \textit{To:, Toh, Tǫ:} \trs{that one}, p. \pageref{p:[to:] ‘that one’}\\
\item \textit{To gęh ǫ ne:ˀ} \trs{is it that?}, \trs{if it is that}, p. \pageref{p:[to gęh ǫ ne:ˀ]}
\end{CayugaRelated}

\subsection*{\textbf{Gęh tęˀ nigęˀǫh} \trs{or not?}} \label{p:[gęh tęˀ nigęˀǫh]}
\CayugaRemark{Particle group \textit{… gęh (…) tęˀ nigęˀǫh} ‘or not?’ functions as a conjunction, connecting alternative choices (\sectref{ch:Clauses with conjunctions}); \textit{gęh} is enclitic and \textit{tęˀ nigę́ˀǫh} appears at the end of the list of choices []. In these examples, the second choice is omitted (because it is obvious or repetitive).}

\ea
\label{ex:gpar62}
\gll [Da̱héˀ] \textbf{gęh} \textbf{tęˀ} \textbf{nigę́ˀǫh}?\\
he.is.coming Q not or\\
\glt ‘Is he coming or not?’
\z


\ea
\label{ex:gpar63}
\gll [Sanǫhǫkda:níh] \textbf{tęˀ} \textbf{nigę́ˀǫh}?\\
you.are.sick not or\\
\glt ‘Are you sick or not?’
\z


\begin{CayugaRelated}
\item \textit{Gęh} \trs{Q} (question marker), p. \pageref{p:[gęh] `Q’}\\
\item \textit{…Nigęˀǫh} \trs{or} (conjunction), p. \pageref{p:[…nigęˀǫh]}\\
\item \textit{Tęˀ} \trs{no}, \trs{not}, p. \pageref{p:[tęˀ]}
\end{CayugaRelated}

\subsection*{\textbf{Gę:s} \trs{generally}, \trs{used to}, \trs{usually, normally}} \label{p:[gę:s]}
\CayugaRemark{Particle functioning as an “adverb of time” (\sectref{ch:’Adverbs’ of time}); enclitic.}

\ea
\label{ex:gpar64}
\gll I:sóˀ agida̱hsgęhę́:ˀ \textbf{gę:s}.\\
much I.used.to.sleep generally\\
\glt ‘I generally used to sleep a lot.’
\z


\ea
\label{ex:gpar65}
\gll Jiˀ \textbf{gę:s} trehs shenó:wę:.\\
because generally too.much you.lie\\
\glt ‘You lie too much’; `you are too much of a liar (in general).’
\z


\ea
\label{ex:gpar66}
\gll Ne:ˀ \textbf{gę:s} gá:dǫh.\\
it.is usually I.say\\
\glt ‘That’s what I usually say.’
\z


\ea
\label{ex:gpar67}
\gll Ji \textbf{gę:s} trehs ja̱hsę: tsęh nisnihnę́:ye:s.\\
too usually too.much you.two.are.fat that you.two.are.tall\\
\glt ‘you are generally too fat for your height.’
\z


\ea
\label{ex:gpar68}
\gll Ahsǫhéh \textbf{gę:s} agahdrǫ́ˀni:.\\
darkness usually I’m.afraid\\
\glt ‘I am generally afraid of the dark.’
\z


\begin{CayugaRelated}
\item \textit{Ę:, E:} (possible atypical verb), p. \pageref{p:[ę:, e:] `atypical verb’}
\end{CayugaRelated}

\subsection*{\textbf{Giˀ} \trs{just}, \trs{really}} \label{p:[giˀ]}
\CayugaRemark{Particle \textit{… giˀ} is used for emphasis (\sectref{ch:Focus, contrastive focus, and emphasis markers}); enclitic.}

\ea
\label{ex:gpar69}
\gll Ne:ˀ \textbf{giˀ} é:dǫh.\\
it.is just she.means.it\\
\glt ‘That’s just what she means.’
\z

\ea
\label{ex:gpar70}
\gll Sanǫhǫkda:níˀ  \textbf{giˀ} gęh?\\
you are.sick just Q\\
\glt ‘Are you really sick?’ `Are you sick for sure?’
\z

\ea
\label{ex:gpar71}
\gll Tęˀ ga̱ˀtoh. Tęˀ  \textbf{giˀ} dehé:gę:\\
not anywhere not just I.did’t.see.him\\
\glt ‘Nowhere. I just didn’t see him.’
\z

\ea
\label{ex:gpar72}
\gll Hoihoˀdeˀsri:yó: \textbf{giˀ} ne:ˀ.\\
he.has.a.good.job just it.is\\
\glt ‘He just has a good job.’
\z

\ea
\label{ex:gpar73}
\gll Waheˀ  \textbf{giˀ} gatahí:neˀ.\\
now just I.am.walking\\
\glt ‘I am just getting on my way.’
\z

\begin{CayugaRelated}
\item \textit{Do:s giˀ} \trs{just so}, \trs{indeed}, p. \pageref{p:[do:s giˀ]}\\
\item \textit{Ewa:dǫˀ giˀ shęh} \trs{maybe}, \trs{a possibility}, p. \pageref{p:[ęwa:dǫˀ giˀ shęh]}\\
\item \textit{Ga:t giˀ shęh} \trs{maybe}, \trs{or maybe}, p. \pageref{p:[ga:t giˀ shęh]}\\
\item \textit{Ga:t giˀ shęh tęˀ} \trs{maybe not}, \trs{or not}, p. \pageref{p:[ga:t giˀ shęh tęˀ]}\\
\item \textit{Gęh ga:t giˀ shęh nigęˀǫh} \trs{or?}, p. \pageref{p:[gęh ga:t giˀ shęh nigęˀǫh]}\\
\item \textit{Giˀ gyę:ˀ} \trs{just do it!}, p. \pageref{p:[giˀ gyę:ˀ] ‘just do it’}\\
\item \textit{Giˀ gyę:ˀ} \trs{quite}, \trs{kind of}, p. \pageref{p:[giˀ gyę:ˀ] ‘quite’}\\
\item \textit{Giˀ hne:ˀ} \trs{but}, \trs{however}, p. \pageref{p:[giˀ hne:ˀ]}\\
\item \textit{Giˀ ne:ˀ} \trs{it is just}, p. \pageref{p:[giˀ ne:ˀ]}\\
\item \textit{Giˀ tsǫ:} \trs{just}, \trs{really}, p. \pageref{p:[giˀ tsǫ:]}\\
\item \textit{Gyę:gwaˀ giˀ shęh} \trs{or maybe}, p. \pageref{p:[gyę:gwaˀ giˀ shęh]}\\
\item \textit{Gyę:gwaˀ giˀ shęh hwaˀ} \trs{maybe this time}, p. \pageref{p:[gyę:gwaˀ giˀ shęh hwaˀ]}\\
\item \textit{Hę:gyeh giˀ} \trs{leave well enough alone}, p. \pageref{p:[hę:-gyeh giˀ]}\\
\item \textit{Ne:ˀ giˀ} \trs{just} (emphasis), p. \pageref{p:[ne:ˀ giˀ]}\\
\item \textit{Ne:ˀ/Neˀ giˀ gyę:ˀ} \trs{that’s just it}, \trs{that’s it for sure}, p. \pageref{p:[ne:ˀ/neˀ giˀ gyę:ˀ]}\\
\item \textit{Ne:ˀ giˀ gyę:ˀ hya:ˀ} \trs{before all else}, \trs{first}, p. \pageref{p:[ne:ˀ giˀ gyę:ˀ hya:ˀ]}\\
\item \textit{Ne:ˀ giˀ gyę:ˀ ǫh} \trs{I guess}, p. \pageref{p:[ne:ˀ giˀ gyę:ˀ ǫh]}\\
\item \textit{Ne:ˀ giˀ hniˀ} \trs{and that too}, \trs{and that also}, p. \pageref{p:[ne:ˀ giˀ hniˀ]}\\
\item \textit{Ne:ˀ giˀ shęh hwaˀ} \trs{maybe this time}, p. \pageref{p:[ne:ˀ giˀ shęh hwaˀ]}\\
\item \textit{Ne:ˀ giˀ shęh neˀ … giˀ shęh} \trs{or}, p. \pageref{p:[ne:ˀ giˀ shęh neˀ … giˀ shęh]}\\
\item \textit{Ne:ˀ giˀ tsǫ: gwahs} \trs{that’s really all}, p. \pageref{p:[ne:ˀ giˀ tsǫ: gwahs]}\\
\item \textit{O: tęˀ ǫh ga:t giˀ shęh gyę:gwaˀ} \trs{maybe, maybe not}, p. \pageref{p:[o: tęˀ ǫh ga:t giˀ shęh gyę:gwaˀ]}\\
\item \textit{O: tęˀ ǫh ga:t giˀ shęh ne:ˀ hwaˀ} \trs{maybe, maybe not}, p. \pageref{p:[o: tęˀ ǫh ga:t giˀ shęh ne:ˀ hwaˀ]}\\
\item \textit{Tęˀ giˀ ni:ˀ} \trs{no, not me}, p. \pageref{p:[tęˀ giˀ ni:ˀ]}\\
\item \textit{Tęˀ giˀ shęh hwaˀ daˀǫ} \trs{maybe not}, p. \pageref{p:[tęˀ giˀ shęh hwaˀ daˀǫ]}\\
\item \textit{Trehs giˀ gyę:ˀ} \trs{my goodness}, \trs{too bad}, \trs{that’s amazing}, p. \pageref{p:[trehs giˀ gyę:ˀ]}
\end{CayugaRelated}

\subsection*{\textbf{Giˀ gyę:ˀ} \trs{quite}, \trs{kind of}} \label{p:[giˀ gyę:ˀ] ‘quite’}
\CayugaRemark{Particle group functioning as an “adverb of degree” (\sectref{ch:’Adverbs’ of degree}); enclitic.}

\ea
\label{ex:gpar74}
\gll “O:,” aˀa:gę́ˀ, “ohsno:wé:ˀah  \textbf{giˀ} \textbf{gyę:ˀ} hwaˀ sa̱hsyǫˀ!”\\
oh she.said quicklyish just this.one this.time you.arrived.home\\
\glt “Oh,” she said, “you came home kind of early this time!”(\cite{henry_de_2005})
\z

\begin{CayugaRelated}
\item \textit{Giˀ} \trs{just}, p. \pageref{p:[giˀ]}\\
\item \textit{Giˀ gyę:ˀ} \trs{just do it!}, p. \pageref{p:[giˀ gyę:ˀ] ‘just do it’}\\
\item \textit{Gyę:ˀ} \trs{just the one}, p. \pageref{p:[gyę:ˀ]}
\end{CayugaRelated}

\subsection*{\textbf{Giˀ gyę:ˀ} \trs{just do it!}} \label{p:[giˀ gyę:ˀ] ‘just do it’}
\CayugaRemark{Particle group \textit{… giˀ gyę:ˀ} adds emphasis to a command (\sectref{Particles and particle groups used with commands}); enclitic.}

\ea
\label{ex:gpar75}
\gll Jidwahshé:t \textbf{giˀ} \textbf{gyę:ˀ} e:ˀ.\\
let’s.all.count just this.one again\\
\glt ‘Let’s count it again then!’
\z

\ea
\label{ex:gpar76}
\gll Desaˀdraihę́h \textbf{giˀ} \textbf{gyę:ˀ}!\\
you(one).hurry just this.one\\
\glt ‘Hurry up then!’ (\cite[142]{mithun_watewayestanih_1984}, Satgęh dialogue)
\z

\ea
\label{ex:gpar77}
\gll Haoˀ-dęˀ-nyóh. Sasęda̱ˀdráh \textbf{giˀ} \textbf{gyę:ˀ} hya:ˀ.\\
ok-what-acknowledge go.back.to.sleep just this.one this.time\\
\glt ‘O.k. Go back to bed.’ (\cite[212]{mithun_watewayestanih_1984}, Satrǫ́:nih dialogue)
\z

\ea
\label{ex:gpar78}
\gll Hó:! Heje:kni:yǫ́h \textbf{giˀ} \textbf{gyę:ˀ}!\\
Hoh! let’s.us.two.go.back.in just this.one\\
\glt ‘Hoh! Let’s go in!’ (\cite{henry_de_2005})
\z

\ea
\label{ex:gpar79}
\gll Sadeˀnyę:dę́h \textbf{giˀ} \textbf{gyę:ˀ}.\\
you(one).try just this.one\\
\glt ‘Try this one.’ (\cite[212]{mithun_watewayestanih_1984}, (Agyaˀdawíˀtraˀ dialogue)
\z

\begin{CayugaRelated}
\item \textit{Giˀ} \trs{just}, p. \pageref{p:[giˀ]}\\
\item \textit{Giˀ gyę:ˀ} \trs{quite}, \trs{kind of}, p. \pageref{p:[giˀ gyę:ˀ] ‘quite’}\\
\item \textit{Gyę:ˀ} \trs{just the one}, p. \pageref{p:[gyę:ˀ]}
\end{CayugaRelated}

\subsection*{\textbf{Giˀ hne:ˀ} \trs{but}, \trs{however}} \label{p:[giˀ hne:ˀ]}
\CayugaRemark{Particle group \textit{… giˀ hne:ˀ} introduces new information contrasting with something said previously (\sectref{ch:Focus, contrastive focus, and emphasis markers}); enclitic.}

\ea
\label{ex:gpar80}
\gll Ǫgwa:yę́ˀ \textbf{giˀ} \textbf{hné:ˀ} neˀ wagye:sę́h heyóhe:, wagyesę̱hsǫ́:ˀǫh.\\
we.have just in.fact the cheap.ones more, they.are.cheap\\
\glt ‘But we do have cheaper ones, the cheaper kind.’ (\cite[159]{mithun_watewayestanih_1984}, Enǫhsǫnyaˀdaˀsǫ́:ˀǫh dialogue)
\z

\ea
\label{ex:gpar81}
\gll Tgwęhǫ: \textbf{giˀ} \textbf{hné:ˀ} gę́s tęˀ.\\
sometimes just but usually not\\
\glt ‘Sometimes this isn’t the case.’ (\cite{mithun_how_1980})
\z

\begin{CayugaRelated}
\item \textit{Giˀ} \trs{just}, p. \pageref{p:[giˀ]}\\
\item \textit{Hne:ˀ} \trs{in fact}, p. \pageref{p:[hne:ˀ] ‘in fact’}
\end{CayugaRelated}

\subsection*{\textbf{Giˀ ne:ˀ} \trs{it’s just}} \label{p:[giˀ ne:ˀ]}
\CayugaRemark{Particle group \textit{… giˀ ne:ˀ} is used for emphasis (\sectref{ch:Focus, contrastive focus, and emphasis markers}); enclitic.}

\ea
\label{ex:gpar82}
\gll Hoihoˀdęhsri:yó: \textbf{giˀ} \textbf{ne:ˀ}.\\
he.has.a.good.job just it.is\\
\glt ‘He just has a good job.’
\z

\begin{CayugaRelated}
\item \textit{Giˀ} \trs{just}, p. \pageref{p:[giˀ]}
\end{CayugaRelated}

\subsection*{\textbf{Giˀ shęh} \trs{maybe}, \trs{or maybe}} \label{p:[giˀ shęh]}
\CayugaRemark{Particle group.}

\begin{CayugaRelated}
\item \textit{Ga:t giˀ shęh} \trs{maybe}, \trs{or maybe}, p. \pageref{p:[ga:t giˀ shęh]}
\end{CayugaRelated}

\subsection*{\textbf{Giˀ shęh, Ne:ˀ giˀ shęh neˀ … giˀ shęh} \trs{or}} \label{p:[giˀ shęh], [ne:ˀ giˀ shęh neˀ … giˀ shęh]}
\CayugaRemark{Particle group \textit{(ne:ˀ giˀ shęh neˀ) … giˀ shęh} functions as a conjunction, connecting an additional list of choices, [ ]. \textit{Ne:ˀ giˀ shęh neˀ} optionally appears before the first item in the list; the other choices are followed by \textit{giˀ shęh}. }

\begin{CayugaRelated}
\item \textit{Ne:ˀ giˀ shęh neˀ … giˀ shęh} \trs{or}, p. \pageref{p:[ne:ˀ giˀ shęh neˀ … giˀ shęh]}
\end{CayugaRelated}

\subsection*{\textbf{Giˀ tsǫ:} \trs{just}, \trs{really}} \label{p:[giˀ tsǫ:]}
\CayugaRemark{Particle group \textit{… giˀ (tsǫ:)} is used for emphasis (\sectref{ch:Focus, contrastive focus, and emphasis markers}); enclitic.}

\ea
\label{ex:gpar83}
\gll To: \textbf{giˀ} \textbf{tsǫ:} ní:yǫ:.\\
that just only it.is.a.certain.amount\\
\glt ‘That’s just all there is.’
\z

\begin{CayugaRelated}
\item \textit{Giˀ} \trs{just}, \trs{really}, p. \pageref{p:[giˀ]}\\
\item \textit{Tsǫ:} \trs{just}, \trs{only}, p. \pageref{p:[tsǫ:]}
\end{CayugaRelated}


\subsection*{\textbf{Gona:węh} \trs{it's theirs (females or mixed group)}, \trs{it belongs to them (females or mixed group)}, \trs{their}} \label{p:[gona:węh]}
\CayugaRemark{Verb functioning as a “possessive pronoun” (\sectref{ch:’Possessive pronouns’}); free-standing.}

\ea
\label{ex:gpar86}
\glll goná:węh\\
gon-á:w-ęh\\
\textsc{3ns.fi.p}-own-\stative\\
\glt ‘they (fe/males) own it’
\z


\subsection*{\textbf{Gonǫ:hęˀ} \trs{they (females or mixed group)}} \label{p:[gonǫ:hęˀ]}
\CayugaRemark{Verb functioning as an “emphatic pronoun” (\sectref{ch:’Emphatic pronouns’}); free-standing.}

\ea\label{ex:gpar84}
gonǫ́:hęˀ\\
\gll gon-ǫ́:hęˀ\\
\textsc{3ns.fi.p}-alone.\stative\\
\glt ‘they are alone’
\z



\subsection*{\textbf{Go:węh} \trs{it's hers}, \trs{it's someone’s}} \label{p:[go:węh]}
\CayugaRemark{Verb functioning as a “possessive pronoun” (\sectref{ch:’Possessive pronouns’}); free-standing.}

\ea\label{ex:gpar85} 
gó:węh\\
\gll gó:-w-ęh\\
\textsc{3s.fi.p}-own-stative\\
\glt ‘she owns it’
\z


\subsection*{\textbf{Gǫdagyeˀ} \trs{immediately}, \trs{right away}} \label{p:[gǫdagyeˀ]}
\CayugaRemark{Atypical word functioning as an “adverb of time” (\sectref{ch:’Adverbs’ of time}); clause-initial.}

\ea
\label{ex:gpar87}
\gll Háoˀ desaˀdráihęh, \textbf{gǫdagyeˀ} dętseˀ.\\
come.on you.hurry, right.away you.will.return\\
\glt ‘Come on, hurry, and come right back.’ (\cite[441]{mithun_watewayestanih_1984}, Dwęnǫhsanékahǫˀ dialogue)
\z


\subsection*{\textbf{Gwaˀ} \trs{immediately}, \trs{right then}, \trs{just then}, \trs{finally}} \label{p:[gwaˀ] ‘immediately’}
\CayugaRemark{Particle functioning as an “adverb of time” (\sectref{ch:’Adverbs’ of time}); enclitic.}

\ea
\label{ex:gpar88}
\gll …o:nęh \textbf{gwaˀ} ahá:nyoˀ.\\
now right.then he.died\\
\glt ‘…finally he died.’(\cite{carrier_legends_2013})
\z

\ea
\label{ex:gpar89}
\gll Sa̱hohsa:ˀ \textbf{gwaˀ} shęh toh tsitga:yę́ˀ neˀ gajęˀ gaǫda:kˀáh tgra̱he:t.\\
He.remembered right.then that there while.it.was.lying.there the container beside.the.log there.in.the.tree\\
\glt ‘One day he remembered he had left his birchbark container at the bottom of the tree.’ (\cite{carrier_legends_2013})
\z

\ea
\label{ex:gpar90}
\gll Ęsadra̱ˀswahetgę́ˀs \textbf{gwaˀ} shęh gyotgǫ́:t ihsé: a:sade:kǫ́:niˀ.\\
your.luck.will.turn.bad right.then that always you.want you.should.eat\\
\glt ‘One day you’ll get into trouble, always wanting to eat!’ (\cite{carrier_legends_2013})
\z

\begin{CayugaRelated}
\item \textit{Ahgwih gwaˀ} \trs{don’t!}, p. \pageref{p:[ahgwih gwaˀ]}\\
\item \textit{Daji:hah gwaˀ} \trs{soon}, \trs{a short while}, p. \pageref{p:[daji:hah gwaˀ]}\\
\item \textit{Deˀę: gwaˀ-heh tęˀ seˀ} \trs{but then not really}, p. \pageref{p:[deˀę: gwaˀ-heh tęˀ seˀ]}\\
\item \textit{Dęˀ gwaˀ hoˀdęˀ} \trs{whatever one (of several)}, p. \pageref{p:[dęˀ gwaˀ hoˀdęˀ]}\\
\item \textit{Do: gwaˀ ni:yǫ:} \trs{a certain amount}, \trs{a certain measure}, \trs{however much}, p. \pageref{p:[do: gwaˀ ni:yǫ:]}\\
\item \textit{Gaę gwaˀ gwa:dih} \trs{whichever way}, \trs{whichever side}, p. \pageref{p:[gaę gwaˀ gwa:dih]}\\
\item \textit{Gaę gwaˀ hǫ:weh} \trs{somewhere}, \trs{someplace}, \trs{wherever}, \trs{around}, p. \pageref{p:[gaę gwaˀ hǫ:weh]}\\
\item \textit{Gaę gwaˀ ni-} \trs{whichever one (of several)}, \trs{whichever person}, p. \pageref{p:[gaę gwaˀ ni- ]}\\
\item \textit{Gaę gwaˀ ni-noun} \trs{wherever \textsc{noun} is}, p. \pageref{p:[gaę gwaˀ ni-noun]}\\
\item \textit{Gaę gwaˀ tsǫ:} \trs{wherever}, \trs{somewhere}, p. \pageref{p:[gaę gwaˀ tsǫ:]}\\
\item \textit{Gwaˀ} \trs{immediately}, p. \pageref{p:[gwaˀ] ‘immediately’}\\
\item \textit{Gwaˀ} \trs{do it now!}, p. \pageref{p:[gwaˀ] `do it now’}\\
\item \textit{Gwaˀ ti:gę:} \trs{plainly}, \trs{clearly}, \trs{as it is}, p. \pageref{p:[gwaˀ ti:gę:]}\\
\item \textit{Gyę:gwaˀ a:-verb} \trs{if}, p. \pageref{p:[gyę:gwaˀ]}\\
\item \textit{Gyę:gwaˀ giˀ shęh} \trs{or maybe}, p. \pageref{p:[gyę:gwaˀ giˀ shęh]}\\
\item \textit{Gyę:gwaˀ giˀ shęh hwaˀ} \trs{maybe this time}, p. \pageref{p:[gyę:gwaˀ giˀ shęh hwaˀ]}\\
\item \textit{Gyę:gwaˀ hne:ˀ hwaˀ} \trs{just maybe}, p. \pageref{p:[gyę:gwaˀ hne:ˀ hwaˀ]}\\
\item \textit{Gyę:gwaˀ ta:-} \trs{if not}, \trs{if it hadn’t been}, p. \pageref{p:[gyę:gwaˀ ta:- ]}\\
\item \textit{Hwę:dǫh gwaˀ} \trs{sometime}, \trs{whenever}, \trs{once}, p. \pageref{p:[hwę:dǫh gwaˀ]}\\
\item \textit{Neˀ gwaˀ toh} \trs{here (rather than there)}, p. \pageref{p:[neˀ gwaˀ toh]}\\
\item \textit{Neˀ tsǫ: gwaˀ toh} \trs{also}, p. \pageref{p:[neˀ tsǫ: gwaˀ toh]}\\
\item \textit{Nę: tsǫ: gwaˀ toh ni-} \trs{just a little bit}, \trs{very little}, p. \pageref{p:[nę: tsǫ: gwaˀ toh ni-]}\\
\item \textit{O: tęˀ ǫh ga:t giˀ shęh gyę:gwaˀ} \trs{maybe, maybe not}, p. \pageref{p:[o: tęˀ ǫh ga:t giˀ shęh gyę:gwaˀ]}\\
\item \textit{O:nęh gwaˀ} \trs{suddenly}, \trs{already}, \trs{finally}, \trs{all at once}, p. \pageref{p:[o:nęh gwaˀ]}\\
\item \textit{Sǫ: gwaˀ, sǫ: gwaˀ nˀaht} \trs{anybody at all}, \trs{anyone at all}, \trs{any living thing}, \trs{somebody}, \trs{whoever}, p. \pageref{p:sǫ: gwaˀ, sǫ: gwaˀ nˀaht]}\\
\item \textit{Sǫ: gwaˀ nˀaht o:yaˀ} \trs{someone else}, p. \pageref{p:[sǫ: gwaˀ nˀaht o:yaˀ]}
\end{CayugaRelated}


\subsection*{\textbf{Gwaˀ} \trs{immediately}, \trs{do it now!}} \label{p:[gwaˀ] `do it now’}
\CayugaRemark{Particle group functioning as an “adverb of time” (\sectref{ch:’Adverbs’ of time}); also functions as an emphasis marker, making a command more urgent (\sectref{Particles and particle groups used with commands}); enclitic.}

\ea
\label{ex:gpar91}
\gll Osháisdaˀ! Gatgáhtwahs shęh nisá:yę:, sęní:hę:ˀ \textbf{gwaˀ}.\\
Snake! I.see that where.you.are.lying, quit.it right.now\\
\glt ‘I can see you serpent. Don’t go into that water.’ (\cite{carrier_legends_2013})
\z

\begin{CayugaRelated}
\item \textit{Gwaˀ} \trs{immediately}, \trs{right then}, \trs{just then}, \trs{finally}, p. \pageref{p:[gwaˀ] ‘immediately’}
\end{CayugaRelated}

\subsection*{\textbf{Gwaˀ ti:gę:} \trs{plainly}, \trs{clearly}, \trs{as it is}} \label{p:[gwaˀ ti:gę:]}
\CayugaRemark{Particle and verb functioning as an “evidential marker” (\sectref{ch:Evidential markers}); free-standing.}

\ea
\label{ex:gpar92}
\gll gwaˀ ti:gę:\\
immediately something.odd\\
\glt ‘plainly’, ‘clearly’, ‘as it is’
\z

\begin{CayugaRelated}
\item \textit{Gwaˀ} \trs{immediately}, \trs{right then}, \trs{just then}, \trs{finally}, p. \pageref{p:[gwaˀ] ‘immediately’}
\end{CayugaRelated}

\subsection*{\textbf{Gwaˀ toh} \trs{that one}, \trs{just such a one}} \label{p:[gwaˀ toh]}
\CayugaRemark{Particle group functioning as a “demonstrative pronoun” (\pageref{ch:’Demonstrative pronouns’}); free-standing.}

\ea
\label{ex:gpar93}
\gll gwaˀ toh\\
right.then that.one\\
\glt ‘that one’, ‘just such a one’
\z

\ea
\label{ex:gpar94}
\gll Ęhę́ˀ, ganǫ́:ˀ, né:ˀ tsǫ: shęh oya:nréˀ giˀ gyę́:ˀ \textbf{gw̱aˀ-toh}.\\
yes it.is.expensive it.is just that it.is.good just emphasis right.then-that.one\\
\glt ‘Yes, it is expensive, but it is a good one.’ (\cite[159]{mithun_watewayestanih_1984}, Enǫhsǫnyaˀdaˀsǫ́:ˀǫh dialogue)
\z

\ea
\label{ex:gpar95}
\gll \textbf{Neˀ} \textbf{gwaˀ-toh} gí:dǫh.\\
neˀ right.then-that.one I.mean.it\\
\glt ‘I also mean that.’
\z

\ea
\label{ex:gpar96}
\gll \textbf{Neˀ} \textbf{gwaˀ-toh} haˀhǫwadíha:ˀ.\\
neˀ right.then-that.one they.took.him.over.there\\
\glt ‘They also took him.’
\z

\begin{CayugaRelated}
\item \textit{Gwaˀ} \trs{immediately}, \trs{right then}, \trs{just then}, \trs{finally}, p. \pageref{p:[gwaˀ] ‘immediately’}\\
\item \textit{To:, Toh, Tǫ:} \trs{that one}, p. \pageref{p:[to:] ‘that one’}
\end{CayugaRelated}

\subsection*{\textbf{Gwaˀ toh, Neˀ gwaˀ toh} \trs{here (rather than there)}} \label{p:[gwaˀ toh, neˀ gwaˀ toh]}
\CayugaRemark{Particle group \textit{(neˀ) gwa̱ˀ-toh} functions as an “adverb of place” (\sectref{‘Adverbs’ of place}); emphatic; clause-initial.}


\begin{CayugaRelated}
\item \textit{Neˀ gwaˀ toh} \trs{here (rather than there)}, p. \pageref{p:[neˀ gwaˀ toh]}
\end{CayugaRelated}

\subsection*{\textbf{Gwa:dih, Gwai} \trs{to one side}} \label{p:[gwa:dih]}
\CayugaRemark{Atypical verb functioning as an “adverb of place” (\sectref{‘Adverbs’ of place}); enclitic to words describing the actual direction.}


\begin{CayugaRelated}
\item \textit{Da: gwa:dih} \trs{over here}, \trs{this side}, p. \pageref{p:[da: gwa:dih]}\\
\item \textit{E: gwa:dih} \trs{on the other side}, p. \pageref{p:[ę: gwa:dih]}\\
\item \textit{Gaę gwaˀ gwa:dih} \trs{whichever way}, \trs{whichever side}, p. \pageref{p:[gaę gwaˀ gwa:dih]}\\
\item \textit{Hehdaˀgeh gwa:dih} \trs{below}, \trs{low}, p. \pageref{p:[hehdaˀgeh gwa:dih]}\\
\item \textit{I:ˀgeh (gwa:dih)} \trs{on my side}, \trs{as for me}, p. \pageref{p:[i:ˀgeh gwa:dih]}\\
\item \textit{Nę: toh gwa:dih} \trs{on this side}, p. \pageref{p:[nę: toh gwa:dih]}\\
\item \textit{Nę:-gyęh gwa:dih} \trs{here}, \trs{this side}, \trs{over here}, p. \pageref{p:[nę:-gyęh gwa:dih]}\\
\item \textit{Nigwa:dih} \trs{-ward, direction}, p. \pageref{p:[nigwa:dih]}\\
\item \textit{Ohnaˀgę: gwa:dih shęh} \trs{behind}, p. \pageref{p:[ohnaˀgę: gwa:dih shęh]}\\
\item \textit{Sewaihǫhsdǫh gwa:dih} \trs{to your right}, p. \pageref{p:[sewaihǫhsdǫh gwa:dih]}\\
\item \textit{Si gwa:dih} \trs{over there}, p. \pageref{p:[si gwa:dih] ‘over there’}\\
\item \textit{Si gwa:dih} \trs{move it!}, p. \pageref{p:[si gwa:dih] ‘move it!’}\\
\item \textit{Si gwa:di:hah} \trs{just this side of}, p. \pageref{p:[si gwa:di:hah]}\\
\item \textit{Si hne:ˀ si gwa:dih} \trs{over there}, p. \pageref{p:[si hne:ˀ si gwa:dih]}\\
\item \textit{To: gwa:dih} \trs{on that side}, p. \pageref{p:[to: gwa:dih]}
\end{CayugaRelated}

\subsection*{\textbf{Gwahs} \trs{really}, \trs{quite}} \label{p:[gwahs] ‘really’}
\CayugaRemark{Particle functioning as an “adverb of degree” (\sectref{ch:’Adverbs’ of degree}); clause-initial.}

\ea
\label{ex:gpar97}
\gll Ęhę́ˀ, \textbf{gwáhs} giˀ ǫgwatga̱ˀdé:ˀah.\\
yes quite just we.are.many-ish\\
\glt ‘Yes, there are quite a few of us.’(\cite[257]{mithun_watewayestanih_1984}, Ga̱hwajiyá:deˀ dialogue)
\z

\ea
\label{ex:gpar98}
\gll \textbf{Gwahs} sgę:nǫ:ǫ́h awádesgoh oshaisdagó:wah ohnegagǫ́: haˀweˀ.\\
quite slowly it.was.escaping the.big.snake in.the.water it.went.there\\
\glt ‘The giant snake was sliding slowly across towards the water.’ (\cite{carrier_legends_2013})
\z

\ea
\label{ex:gpar99}
\gll \textbf{Gwahs} tęˀ sga̱hǫˀdęˀ o:nęh de̱ˀsgá:yęˀ.\\
really not anything now nothing.is.left\\
\glt ‘There is really nothing left.’ (\cite{carrier_legends_2013})
\z

\ea
\label{ex:gpar100}
\gll \textbf{Gwahs} hniˀ agadagáida:t.\\
really and I’m.feeling.better\\
\glt ‘I am feeling so much better.’ (\cite{carrier_legends_2013})
\z

\ea
\label{ex:gpar101}
\gll …\textbf{gwahs} tsaˀdegaǫhya̱hę́hneh…\\
…right middle.of.the.sky.place\\
\glt ‘…right above us…’ (\cite{carrier_legends_2013})
\z

\begin{CayugaRelated}
\item \textit{Gwahs} \trs{anyway}, p. \pageref{p:[gwahs] ‘anyway’}\\
\item \textit{Gwahs hwaˀ} \trs{this time for sure}, p. \pageref{p:[gwahs hwaˀ]}\\
\item \textit{Gwahs ǫ:weh} \trs{really}, p. \pageref{p:[gwahs ǫ:weh]}\\
\item \textit{Gwahs ǫ:węh seˀ} \trs{indeed}, \trs{for sure}, p. \pageref{p:[gwahs ǫ:węh seˀ]}\\
\item \textit{Gwahs shęh ni-} \trs{as…as}, p. \pageref{p:[gwahs shęh ni-]}\\
\item \textit{Gwahs waˀ-heh tsǫ:} \trs{just now}, \trs{just a few seconds ago}, p. \pageref{p:[gwahs waˀ-heh tsǫ:]}\\
\item \textit{Ne:ˀ giˀ tsǫ: gwahs} \trs{that’s really all}, p. \pageref{p:[ne:ˀ giˀ tsǫ: gwahs]}\\
\item \textit{Ne:ˀ gwahs d-} \trs{the most}, p. \pageref{p:[ne:ˀ gwahs d-]}\\
\item \textit{Ne:ˀ gwahs heyohe:ˀ} \trs{the most, -est}, p. \pageref{p:[ne:ˀ gwahs heyohe:ˀ]}\\
\item \textit{Tęˀ gwahs ǫ:weh} \trs{not really}, \trs{not quite}, p. \pageref{p:[tęˀ gwahs ǫ:weh]}
\end{CayugaRelated}


\subsection*{\textbf{Gwahs} \trs{anyway}} \label{p:[gwahs] ‘anyway’}
\CayugaRemark{Particle conveying focus or emphasis (\sectref{ch:Focus, contrastive focus, and emphasis markers}); enclitic.}

\ea
\label{ex:gpar102}
\gll Sǫ: ní:s nˀaht \textbf{gwahs}? Tę́ˀ \textbf{gwáhs} ǫ:wi: de̱ˀgǫyędéi. Dave Maracle gęh haya:sǫ́h neˀ hya̱ˀnih?\\
who you someone anyway not anyway really I.don’t.know.you Dave Maracle Q he.is.called the your.father\\
\glt ‘Who are you, anyway? I don’t really know you. Is Dave Maracle your father?’ (\cite[339]{mithun_watewayestanih_1984}, Oˀdréhdatgiˀ dialogue)
\z



\subsection*{\textbf{Gwahs heyohe:ˀ, Ne:ˀ gwahs heyohe:ˀ} \trs{the most, -est}} \label{p:[gwahs heyohe:ˀ]}
\CayugaRemark{Particle group \stem{(ne:ˀ) gwahs stative.verb (heyohe:)} is used in comparisons (\sectref{ch:Comparisons (more, the same, or less)}). }


\begin{CayugaRelated}
\item \textit{Ne:ˀ gwahs heyohe:ˀ} \trs{the most, -est}, p. \pageref{p:[ne:ˀ gwahs heyohe:ˀ]}
\end{CayugaRelated}


\subsection*{\textbf{Gwahs hwaˀ} \trs{this time for sure}} \label{p:[gwahs hwaˀ]}
\CayugaRemark{Particle group \textit{gwahs hwaˀ} functions as an “adverb of time” (\sectref{ch:’Adverbs’ of time}); emphatic; clause-initial.}

\ea
\label{ex:gpar103}
\gll \textbf{Gwahs} \textbf{hwaˀ} eságaę.\\
emphasis this.time you.give.permission\\
\glt ‘You are willing or giving permission this time.’
\z

\begin{CayugaRelated}
\item \textit{Gwahs} \trs{really}, \trs{just}, \trs{quite}, p. \pageref{p:[gwahs] ‘really’}\\
\item \textit{Hwaˀ} \trs{this time}, \trs{next}, p. \pageref{p:[hwaˀ]}\\
\item \textit{Ne:ˀ} \trs{it is}, p. \pageref{p:[ne:ˀ] `it is’}
\end{CayugaRelated}


\subsection*{\textbf{Gwahs ǫ:weh} \trs{really}} \label{p:[gwahs ǫ:weh]}
\CayugaRemark{Particle group functioning as an “adverb of degree” (\sectref{ch:’Adverbs’ of degree}); clause-initial.}

\ea
\label{ex:gpar104}
\gll \textbf{Gwahs} \textbf{ǫ:weh} tsaˀho:yá:t honíˀǫh.\\
really truly he.cannot.help.it he.is.stingy\\
\glt ‘He just cannot help himself, he is really really stingy.’
\z

\ea
\label{ex:gpar105}
\gll \textbf{Gwahs} \textbf{ǫ:weh} agadatsę́hse:!\\
really truly I’m.tired\\
\glt ‘I am so tired!’ (\cite{carrier_legends_2013})
\z

\ea
\label{ex:gpar106}
\gll \textbf{Gwahs} \textbf{ǫ:wéh} a:yę́ˀ tę́ˀ desho:tǫ́:deˀ.\\
really truly it.seems not he.doesn’t.hear\\
\glt ‘It really seems like he doesn’t hear.’ (That is, he has selective hearing.)
\z

\begin{CayugaRelated}
\item \textit{Gwahs} \trs{really}, \trs{just}, \trs{quite}, p. \pageref{p:[gwahs] ‘really’}\\
\item \textit{Gwahs ǫ:węh seˀ} \trs{indeed}, \trs{for sure}, p. \pageref{p:[gwahs ǫ:węh seˀ]}\\
\item \textit{O:weh, Neˀ ǫ:weh} \trs{really}, p. \pageref{p:[ǫ:weh, neˀ ǫ:weh]}
\end{CayugaRelated}


\subsection*{\textbf{Gwahs ǫ:węh seˀ} \trs{indeed}, \trs{for sure}} \label{p:[gwahs ǫ:węh seˀ]}
\CayugaRemark{Particle group functioning as an “agreement marker” (\sectref{ch:Agreement markers}); free-standing.}

\ea
\label{ex:gpar107} Speaker A: \\
\gll Trehs taˀdeyodriˀsdagé: ga̱ha:gǫ́: waˀne:ˀ. \\
too not.much.noise in.woods today\\
\glt `The woods are too quiet.’\\
Speaker B: \\
\gll Ęhę:ˀ, \textbf{gwahs} \textbf{ǫ́:weh} \textbf{seˀ}.\\
yes very it.is.true you.know \\
\glt `Yes, much too quiet.’ (\cite{carrier_legends_2013})
\z


\begin{CayugaRelated}
\item \textit{Gwahs} \trs{really}, \trs{just}, \trs{quite}, p. \pageref{p:[gwahs] ‘really’}\\
\item \textit{Gwahs ǫ:weh} \trs{really}, p. \pageref{p:[gwahs ǫ:weh]}\\
\item \textit{O:weh, Neˀ ǫ:weh} \trs{really}, p. \pageref{p:[ǫ:weh, neˀ ǫ:weh]}\\
\item \textit{Seˀ} \trs{you know}, p. \pageref{p:[seˀ]}
\end{CayugaRelated}


\subsection*{\textbf{Gwahs shęh ni-} \trs{as…as}} \label{p:[gwahs shęh ni-]}
\CayugaRemark{Particle group \stem{(gwahs) shęh ni-stative.verb} is used in comparisons (\sectref{ch:Comparisons (more, the same, or less)}); it precedes a stative verb beginning with \stem{ni-} {\partitive}, and is clause-initial in dependent clauses.}

\ea
\label{ex:gpar108}
\gll Enagre̱hsra:gwe:gǫ́h giˀ o:nę́h agaehne:ga:gó:ˀ \textbf{gwahs} \textbf{shęh} \textbf{n}agaegwé:niˀ.\\
all.the.inhabitants just now they.collected.water so.much that how.much.they.could\\
\glt ‘So all the village began collecting as much liquid as they could.’ (\cite{carrier_legends_2013})
\z

\ea
\label{ex:gpar109}
\gll Atę:ná:t giˀ. [\textbf{Gwahs} \textbf{shęh} \textbf{ni}hodigwe:nyǫ́:] atę́:na:t.\\
they.danced just so.much that how.they.were.able.to they.danced\\
\glt ‘The boys danced. They danced like they’d never danced before.’ (\cite{carrier_legends_2013})
\z

\ea
\label{ex:gpar110}
\gll Sgę:nǫ́:ˀǫh \textbf{shęh} \textbf{n}ęhse:gwé:níˀ dęhsátahahk\\
slow that how.you.will.be.able.to you.will.walk\\
\glt ‘Walk as slowly as you can.’ (\cite{mithun_how_1980})
\z

\begin{CayugaRelated}
\item \textit{Gwahs} \trs{anyway}, p. \pageref{p:[gwahs] ‘anyway’}\\
\item \textit{Shęh} \trs{that}, \trs{because}, p. \pageref{p:[shęh] `because’}
\end{CayugaRelated}


\subsection*{\textbf{Gwahs waˀ-heh tsǫ:} \trs{just now}, \trs{just a short while ago}} \label{p:[gwahs waˀ-heh tsǫ:]}
\CayugaRemark{Particle group functioning as an “adverb of time” (\sectref{ch:’Adverbs’ of time}); emphatic; clause-initial.}

\ea
\label{ex:gpar111}
\gll \textbf{Gwahs} \textbf{waˀhéh} \textbf{tsǫ:} to: naˀá:węh.\\
really today only that it.happened\\
\glt ‘That just happened.’
\z

\ea
\label{ex:gpar112}
\gll \textbf{Gwahs} \textbf{waˀhéh} \textbf{tsǫ:} agatna̱ˀdaǫdá:gwęh.\\
really today only I.took.bread.out\\
\glt ‘I just took it out of the oven.’ (\cite[376]{mithun_watewayestanih_1984}, Dajǫh dialogue)
\z

\ea
\label{ex:gpar113}
\gll Ęhę́ˀ, o:nę́h giˀ agade̱hsrǫ́ni̱hsˀǫh. \textbf{Gwahs} \textbf{waˀheh-tsǫ́:} agagya̱ˀdoháesiˀ.\\
yes now just I’m.getting.ready really today-only I’ve.finished.bathing\\
\glt ‘Yes, I am ready now. I just finished having a bath.’ (\cite[294]{mithun_watewayestanih_1984}, Dwade:kǫ́:nih dialogue)
\z

\begin{CayugaRelated}
\item \textit{Gwahs} \trs{anyway}, p. \pageref{p:[gwahs] ‘anyway’}\\
\item \textit{Tsǫ:} \trs{just}, \trs{only}, p. \pageref{p:[tsǫ:]}\\
\item \textit{Waˀ-heh} \trs{just now}, \trs{finally}, p. \pageref{p:[waˀ-heh]}
\end{CayugaRelated}


\subsection*{\textbf{Gwe:} \trs{hello}} \label{p:[gwe:] `hello!’}
\CayugaRemark{Particle functioning as a greeting.}

\begin{CayugaRelated}
\item \textit{Gwe:} \trs{well}, p. \pageref{p:[gwe:] ‘well, …’}
\end{CayugaRelated}


\subsection*{\textbf{Gwe:} \trs{well!}} \label{p:[gwe:] ‘well!’}
\CayugaRemark{Particle functioning as an “exclamation” (\sectref{ch:Exclamations}); conveys surprise or other emotion.}

\ea
\label{ex:gpar114}
\gll \textbf{Gwé:} a:yę́ˀ sanǫ́hnya̱ˀgǫh!\\
well it.seems you.are.hurt\\
\glt ‘Well, it looks like you are hurt!’
\z

\begin{CayugaRelated}
\item \textit{Gwe:} \trs{well}, p. \pageref{p:[gwe:] ‘well, …’}
\end{CayugaRelated}


\subsection*{\textbf{Gwe:} \trs{well}} \label{p:[gwe:] ‘well, …’}
\CayugaRemark{Particle signaling a change in topic (\sectref{ch:Topic changers}); clause-initial.}

\ea
\label{ex:gpar115} 
\gll David:  Háe Pete. \\
{} hi Pete\\
\glt ‘Hi Pete’\\
 
\gll Pete: Háe. \\
{} hi\\
\glt ‘Hi.’\\

\gll David: \textbf{Gwé:}, í:s gęh satse:nę́ˀ to:gyę́h so:wa:s?\\
{} so, you Q its.your.pet that.one dog\\
\glt ‘Well, is that your dog?’ (\cite[317]{mithun_watewayestanih_1984}, Sanahsgwaę́ˀ Gęh Sga̱hoˀdę́:ˀęh dialogue)
\z

\ea
\label{ex:gpar116}
\gll \textbf{Gwé:}, dędwaˀęnáęˀ agę́ˀ ęyó:hęˀ. Oyanra̱hsdǫ́h giˀ hne:ˀ shęh i:sóˀ oˀgráęˀ.\\
well we.will.play.snowsnake it.is.said tomorrow. It.is.ideal just in.fact that a.lot there.is.snow\\
\glt ‘Well, they say we’re going to play snowsnake tomorrow. It is ideal because there is a lot of snow.’ (\cite[481]{mithun_watewayestanih_1984}, Dedwaˀęnáęˀ dialogue)
\z


\begin{CayugaRelated}
\item \textit{Gwe:} \trs{well!}, p. \pageref{p:[gwe:] ‘well!’}
\end{CayugaRelated}


\subsection*{\textbf{Gwe:gǫh, Agwe:gǫh, Ogwe:gǫh} \trs{all}, \trs{everything}} \label{p:[gwe:gǫh] ‘all’}
\CayugaRemark{Verb (atypical, when without \stem{o-} \textsc{3s.p} prefix), functioning as an “indefinite pronoun” (\sectref{ch:’Indefinite pronouns’}); free-standing.}

\ea
\label{ex:gpar117}
\gll O:nę́h giˀ adyǫkni̱ˀdraihę́h, aˀakniksaohé:k \textbf{gwé:gǫh} hoˀdę́ˀ, hoˀdę́ˀ ęyá:gwahs.\\
now just we.hurried, we.gathered all the.kind, the.kind we.will.use\\
\glt ‘Then we hurried up and gathered up all the dishes that we would use.’ (\cite{henry_de_2005})
\z

\begin{CayugaRelated}
\item \textit{Gwe:gǫh, Agwe:gǫh, Ogwe:gǫh} \trs{completely}, \trs{totally}, p. \pageref{p:[gwe:gǫh] ‘totally’}\\
\item \textit{Hegwe:gǫh, Neˀ hegwe:gǫh} \trs{the whole thing}, p. \pageref{p:[neˀ hegwe:gǫh]}
\end{CayugaRelated}


\subsection*{\textbf{Gwe:gǫh, Agwe:gǫh, Ogwe:gǫh} \trs{completely}, \trs{totally}} \label{p:[gwe:gǫh] ‘totally’}
Verb (atypical, when without \stem{o-} prefix) functioning as an “adverb of degree” (\sectref{ch:’Adverbs’ of degree}); clause-initial.

\ea
\label{ex:gpar118}
\gll A:yę́:ˀ hniˀ \textbf{gwe:gǫ́h} agadranawę̱ˀdoˀkdáˀǫh hniˀ.\\
it.seems and it.is.all we.have.run.out.of.sugar and\\
\glt ‘I seem to have completely run out of sugar, too.’ (\cite[441]{mithun_watewayestanih_1984}, Dwęnǫhsanékahǫˀ dialogue)
\z

\begin{CayugaRelated}
\item \textit{Gwe:gǫh, Agwe:gǫh, Ogwe:gǫh} \trs{all}, \trs{everything}, p. \pageref{p:[gwe:gǫh] ‘all’}\\
\item \textit{Hegwe:gǫh, Neˀ hegwe:gǫh} \trs{the whole thing}, p. \pageref{p:[neˀ hegwe:gǫh]}
\end{CayugaRelated}


\subsection*{\textbf{Gyę:ˀ} \trs{just the one}} \label{p:[gyę:ˀ]}
\CayugaRemark{Particle \textit{… gyę:ˀ} conveys emphasis (\sectref{ch:Focus, contrastive focus, and emphasis markers}); enclitic.}

\ea
\label{ex:gpar119}
\gll Honǫksǫ́ˀ seˀ neˀ Hagówanęh shęh haǫháˀ \textbf{gyę:ˀ} gaihwaędahgǫ́h dęshago̱hsnyéˀ neˀ háǫgwe̱ˀdaˀ.\\
he.knew you.know the he.is.big that he.alone emphasis the.responsibility he.will.lead.them the his.people\\
\glt ‘Their leader knew he was responsible for his people.’ (\cite{carrier_legends_2013})
\z

\ea
\label{ex:gpar120}
\gll Hę́:gyeh. Tęˀ \textbf{gyę:ˀ} ta:yoyęnawáˀs neˀ otgǫ́ˀtraˀ neˀ o:nę́h dwa̱hsreˀ.\\
no.matter not emphasis it.wouldn’t.be.helped the magic the now we.all.follow.it\\
\glt ‘Not for long. His magic is no good once we have his trail.’ (referring to the Ugly Bear having magic) (\cite{carrier_legends_2013})
\z

\ea
\label{ex:gpar121}
\gll O:nę́h diˀ ihse:t \textbf{gyę:ˀ} nóne:ˀ dehsatga̱hdǫ́:nyǫhs.\\
now then you.are.standing emphasis you.know you.look.around\\
\glt ‘And now you will stand there looking.’ (\cite{mithun_how_1980})
\z

\ea
\label{ex:gpar122}
\gll Sǫheh \textbf{gyę:ˀ} nóne:ˀ gá:dǫh deyonadáwęnyeˀ.\\
nighttime emphasis you.know I.say they.walk.around\\
\glt ‘As I said, at night, they walk around.’ (\cite{mithun_how_1980})
\z

\ea
\label{ex:gpar123}
\gll Ne:ˀ seˀ gę:s neˀ sweˀgé:hah hęnadęhní:nǫh gę:s neˀ gwaˀyǫˀ \textbf{gyę:ˀ}.\\
it.is you.know usually the long.ago they.sold us ually the rabbits emphasis\\
\glt ‘A long time ago, they used to sell rabbits.’ (\cite{mithun_how_1980})
\z

\begin{CayugaRelated}
\item \textit{Ehęˀ gyę:ˀ} \trs{yes indeed}, p. \pageref{p:[ęhęˀ gyę:ˀ]}\\
\item \textit{Giˀ gyę:ˀ} \trs{just do it!}, p. \pageref{p:[giˀ gyę:ˀ] ‘just do it’}\\
\item \textit{Giˀ gyę:ˀ} \trs{quite}, \trs{kind of}, p. \pageref{p:[giˀ gyę:ˀ] ‘quite’}\\
\item \textit{Gyę:ˀ hne:ˀ tęˀ neˀ} \trs{it was this one, (not that one)}, p. \pageref{p:[gyę:ˀ hne:ˀ]}\\
\item \textit{Gyę:ˀ ǫh} \trs{maybe}, \trs{I guess}, \trs{I wonder}, p. \pageref{p:[gyę:ˀ ǫh]}\\
\item \textit{Neˀ gyę:ˀ nę ne:ˀ} (emphasis), p. \pageref{p:[neˀ gyę:ˀ nę ne:ˀ]}\\
\item \textit{Neˀ gyę:ˀ neˀ gyę:ˀ, Negęˀnagęˀ} \trs{that is what}, p. \pageref{p:[neˀ gyę:ˀ neˀ gyę:ˀ]}\\
\item \textit{Neˀ to gyę:ˀ} \trs{that’s what}, p. \pageref{p:[neˀ to gyę:ˀ]}\\
\item \textit{Neˀ to gyę:ˀ hǫ:weh} \trs{it is where}, p. \pageref{p:[neˀ to gyę:ˀ hǫ:weh]}\\
\item \textit{Ne:ˀ/neˀ giˀ gyę:ˀ} \trs{that’s just it}, \trs{that’s it for sure}, p. \pageref{p:[ne:ˀ/neˀ giˀ gyę:ˀ]}\\
\item \textit{Ne:ˀ giˀ gyę:ˀ hya:ˀ} \trs{before all else}, \trs{first}, p. \pageref{p:[ne:ˀ giˀ gyę:ˀ hya:ˀ]}\\
\item \textit{Ne:ˀ giˀ gyę:ˀ ǫh} \trs{I guess}, p. \pageref{p:[ne:ˀ giˀ gyę:ˀ ǫh]}\\
\item \textit{Ne:ˀ gyę:ˀ neˀ} \trs{that really is}, p. \pageref{p:[ne:ˀ gyę:ˀ neˀ]}\\
\item \textit{Ne:ˀ seˀ gyę:ˀ} \trs{you know}, p. \pageref{p:[ne:ˀ seˀ gyę:ˀ]}\\
\item \textit{Ni:ˀ gyę:ˀ, Ni:ˀ gę:ˀ} \trs{I did it}, p. \pageref{p:[ni:ˀ gyę:ˀ]}\\
\item \textit{Tęˀ gyę:ˀ nę neˀ} \trs{not}, \trs{what on earth?}, p. \pageref{p:[tęˀ gyę:ˀ nę neˀ]}\\
\item \textit{Tęˀ gyę:ˀ ǫh, Tęˀ gęˀǫh} \trs{not really}, p. \pageref{p:[tęˀ gyę:ˀ ǫh]}\\
\item \textit{Waˀ-jih gyę:ˀ} \trs{almost}, \trs{just about}, p. \pageref{p:[waˀ-jih gyę:ˀ]}
\end{CayugaRelated}


\subsection*{\stem{-gyęh} \trs{this}, \trs{that}} \label{p:[-gyęh]}
\CayugaRemark{\stem{-gyęh} is an enclitic element occurring in the following particle combinations.}

\begin{CayugaRelated}
\item \textit{Hę:-gyęh, hę:gyeh} \trs{no matter}, p. \pageref{p:[hę:-gyęh]}\\
\item \textit{Hę:-gyęh shęh} \trs{no matter how much}, \trs{whether or not}, \trs{even if}, p. \pageref{p:[hę:-gyęh shęh]}\\
\item \textit{Hę:-gyęh tsǫ:} \trs{it doesn’t matter}, \trs{never mind}, p. \pageref{p:[hę:-gyęh tsǫ:]}\\
\item \textit{Nę:-gyęh} \trs{this one}, p. \pageref{p:[nę:-gyęh hwaˀ] ‘this one’}\\
\item \textit{Nę:-gyęh gwa:dih} \trs{here}, \trs{this side}, \trs{over here}, p. \pageref{p:[nę:-gyęh gwa:dih]}\\
\item \textit{Nę:-gyęh hwaˀ} \trs{this time}, p. \pageref{p:[nę:-gyęh hwaˀ] ‘this time’}\\
\item \textit{Si:-gyęh} \trs{that one over there}, p. \pageref{p:[si:-gyęh]}\\
\item \textit{To:-gyęh} \trs{that one}, p. \pageref{p:[to:-gyęh]}\\
\item \textit{Waˀ-gyęh} \trs{presently}, \trs{so now}, \trs{then}, p. \pageref{p:[waˀ-gyęh] ‘presently’}\\
\item \textit{Waˀ-gyęh} \trs{listen}, \trs{excuse me}, \trs{would you}, p. \pageref{p:[waˀ-gyęh] ‘Listen!’}
\end{CayugaRelated}


\subsection*{\textbf{Gyę:ˀ hne:ˀ, Gyę:ˀ hne:ˀ tęˀ neˀ} \trs{it was this one, (not that one)}} \label{p:[gyę:ˀ hne:ˀ]}
\CayugaRemark{Particle group \textit{… gyę:ˀ hne:ˀ … (tęˀ ne)} conveys contrastive focus (\sectref{ch:Focus, contrastive focus, and emphasis markers}); enclitic after the item contrasted.}

\ea
\label{ex:gpar129}
\gll I:s \textbf{gyęˀ} \textbf{hne:ˀ}.\\
you this.one in.fact\\
\glt ‘That’s you!’
\z

\ea
\label{ex:gpar130}
\gll I:s \textbf{gyęˀ} \textbf{hne:ˀ} tsǫ: e:ˀ toh nahsye:ˀ\\
you this.one in.fact just again that.one you.did.something\\
\glt ‘you are the only one that did that again’
\z

\ea
\label{ex:gpar131}
\gll Wa:li \textbf{gyęˀ} \textbf{hne:ˀ}, tęˀ neˀ John.\\
Mary this.one in.fact not the John\\
\glt ‘It was Mary, not John.’
\z

\begin{CayugaRelated}
\item \textit{Gyę:ˀ} \trs{just the one}, p. \pageref{p:[gyę:ˀ]}\\
\item \textit{Hne:ˀ} \trs{in fact}, p. \pageref{p:[hne:ˀ] ‘in fact’}\\
\item \textit{Neˀ} \trs{that is}, \trs{the}, p. \pageref{p:[neˀ]}\\
\item \textit{Tęˀ} \trs{no}, \trs{not}, p. \pageref{p:[tęˀ]}
\end{CayugaRelated}


\subsection*{\textbf{Gyę:ˀ nę ne:ˀ, Neˀ gyę:ˀ nę ne:ˀ} (emphasis)} \label{p:[gyę:ˀ nę ne:ˀ]}
\CayugaRemark{Particle group \textit{(neˀ) gyę:ˀ nę ne:ˀ} conveys emphasis (\sectref{ch:Focus, contrastive focus, and emphasis markers}).}

\ea
\label{ex:gpar132}
\gll Tsaˀdeyǫ̱hwęjaga̱hę́h \textbf{gyę:ˀ} \textbf{nę} \textbf{ne:ˀ} hǫ́:weh to:gyę́h hǫ: heyagwę̱ˀdrǫ́ˀ tohgéh hǫ́:weh\\
as.the.land.rises.up emphasis here it.is where there where we.live.there that’s.where place\\
\glt ‘Where the land rises up, there where we live, that’s the place (where he went).’ (\cite{henry_de_2005})
\z

\ea
\label{ex:gpar133}
\gll Ohnegatgíˀ \textbf{gyę:ˀ} \textbf{nę́} \textbf{ne:ˀ} ęwá:dǫˀ.\\
dirty.water emphasis here it.is it.will.become\\
\glt ‘The water will get dirty (because it will get riled up)!’ (\cite{henry_de_2005})
\z

\ea
\label{ex:gpar134}
\gll Aˀa:gę́ˀ, “Hotgęni̱hsaˀnhǫ́: \textbf{gyę:ˀ} \textbf{nę́} \textbf{ne:ˀ}.”\\
she.said he.went.to.council emphasis here it.is\\
\glt ‘She said, “He went to Council again, as usual.”’ (\cite{henry_de_2005})
\z

\ea
\label{ex:gpar135}
\gll O:nę́h \textbf{gyę:ˀ} \textbf{nę́} \textbf{ne:ˀ} tó:hah ęyagwade:kǫ́:niˀ.\\
now emphasis here it.is almost we.will.eat\\
\glt ‘And then it was almost time for us to eat.’ (\cite{henry_de_2005})
\z

\ea
\label{ex:gpar136}
\gll “Ó:,” aˀa:gę́ˀ, “agatna̱ˀdaǫ́t \textbf{gyę:ˀ} \textbf{nę́} \textbf{ne:ˀ} hniˀ!\\
oh, she.said I.made.bread emphasis here it.is and\\
\glt “Ah,” she said, “I made pan bread too!” (\cite{henry_de_2005})
\z

\ea
\label{ex:gpar137}
\gll Deyakigahné:ˀ, \textbf{gyę:ˀ} \textbf{nę́} \textbf{ne:ˀ} hǫ́: dyǫknida̱ˀǫ́h héˀtgęh.\\
we.were.looking.at.her emphasis here it.is there we.were.sleeping upstairs\\
\glt ‘We were looking at her where we were asleep upstairs.’ (\cite{henry_de_2005})
\z

\begin{CayugaRelated}
\item \textit{Neˀ gyę:ˀ nę ne:ˀ} (emphasis), p. \pageref{p:[neˀ gyę:ˀ neˀ gyę:ˀ]}
\end{CayugaRelated}


\subsection*{\textbf{Gyę:ˀ ǫh, gęˀ ǫh} \trs{maybe}, \trs{I guess}, \trs{I wonder}} \label{p:[gyę:ˀ ǫh]}
\CayugaRemark{Particle group \textit{… gyę:ˀ ǫh, gęˀǫh} functions an evidential marker (\sectref{ch:Evidential markers}); emphatic; enclitic. Possibly related to \textit{gęh ǫh}.}

\ea
\label{ex:gpar124}
\gll Ó:, degyega̱hné:ˀ giˀ. O:nę́h \textbf{gyę́:ˀ} \textbf{ǫh} ǫkwái.\\
oh, she.is.looking.out just now this.one I.guess the.food.is.cooked\\
\glt ‘Oh, she is looking out. The food must be cooked now.’ (\cite[294]{mithun_watewayestanih_1984}, Dwade:kǫ́:nih dialogue)
\z

\ea
\label{ex:gpar125}
\gll O:nę́h giˀ \textbf{gyę́:ˀ} \textbf{ǫh} ęsgahdę́:diˀ. O:nę́h gyę́:ˀ nǫ́ne:ˀ hné:ˀ haˀwa:jˀáht degáhswa̱ˀne:t!\\
now just this.one I.guess I.will.go.home now emphasis you.know in.fact it.is.all.gone the.pie\\
\glt ‘Well, I might as well go home now, because the pie is all gone, you know!’ (\cite[376]{mithun_watewayestanih_1984}, Dajǫh dialogue)
\z

\ea
\label{ex:gpar126}
\gll Né:ˀ \textbf{gyę́:ˀ} \textbf{ǫh} hné:ˀ nęh agídagraˀ agadadwęˀna̱hsáik.\\
it.is this.one I.guess in.fact when I.fell I.bit.my.tongue\\
\glt ‘I guess it was when I fell that I bit my tongue.’ (\cite[386]{mithun_watewayestanih_1984}, Ga̱há:gǫ: dialogue)
\z

\ea
\label{ex:gpar127}
\gll Tę́ˀ giˀ hné:ˀ de̱ˀagęnǫhdǫ́ˀ a:ge̱ˀnhǫhsaniháˀ \textbf{gę́ˀ} \textbf{ǫh} hniˀ.\\
not just in.fact I.don’t.know I.should.borrow.eggs this.one I.guess and\\
\glt ‘I don’t really know whether I should borrow eggs too.’ \cite[441]{mithun_watewayestanih_1984}, Dwęnǫhsanékahǫˀ dialogue)
\z

\ea
\label{ex:gpar128}
\gll To \textbf{gęˀ} ǫhní:yoht?\\
that I.guess it.resembles.something\\
\glt ‘I wonder if it is like that?’
\z

\begin{CayugaRelated}
\item \textit{Gęh oh, Gę-ǫ:, Ga-ǫ: ne:ˀ} \trs{is it that?}, \trs{if it is that}, p. \pageref{p:[gęh oh, gę-ǫ:]}\\
\item \textit{Gyę:ˀ} \trs{just the one}, p. \pageref{p:[gyę:ˀ]}\\
\item \textit{Oh, Ǫ:, Ǫ} \trs{I guess}, \trs{I wonder (if)}, p. \pageref{p:[ǫh, ǫ:, ǫ]}
\end{CayugaRelated}


\subsection*{\textbf{Gyę:gwaˀ} \trs{if}} \label{p:[gyę:gwaˀ]}
\CayugaRemark{Particle \textit{gyę:gwaˀ} introduces conditional clauses; clause-initial in dependent clauses [ ], implying a prerequisite condition or hypothetical situation (\sectref{ch:Conditional clauses with [gyę:gwaˀ], [gęh], [hę:gyeh] ‘if, whether’}). \textit{Gyę:ˀgwaˀ} requires a following verb beginning with an [a:-] {\indefinite} prefix, and the verb often lacks an aspect suffix.}

\ea
\label{ex:gpar138}
\gll Tęˀ dejǫ́hsdi:s o:nę́h \textbf{gyę:gwáˀ} \textbf{a:}yáihe:\\
not she.didn’t.pay.attention now if she.would.die\\
\glt ‘She didn’t care that she was going to her death.’ (\cite{carrier_legends_2013})
\z

\ea
\label{ex:gpar139}
\gll A:ga̱hya:góˀ \textbf{gyę:gwáˀ} \textbf{a:}sgyená:waˀs\\
I.would.pick.fruit if you.would.help.me\\
\glt ‘I would pick fruit if only you would help me.’
\z

\ea
\label{ex:gpar140}
\gll Ahgwíh dęhsyéhs ohné:gaˀ \textbf{gyę:gwáˀ} \textbf{ihsé:} ęsa:dó:wiˀ.\\
don’t you.will.mix water if you.want you.will.drive\\
\glt ‘Don’t mix booze if you want to drive.’ (`Don’t drink and drive.’)
\z

\begin{CayugaRelated}
\item \textit{Gyę:gwaˀ giˀ shęh} \trs{or maybe}, p. \pageref{p:[gyę:gwaˀ giˀ shęh]}\\
\item \textit{Gyę:gwaˀ giˀ shęh hwaˀ} \trs{maybe this time}, p. \pageref{p:[gyę:gwaˀ giˀ shęh hwaˀ]}\\
\item \textit{Gyę:gwaˀ hne:ˀ hwaˀ} \trs{just maybe}, p. \pageref{p:[gyę:gwaˀ hne:ˀ hwaˀ]}\\
\item \textit{Gyę:gwaˀ ta:-} \trs{if not}, \trs{if it hadn’t been}, p. \pageref{p:[gyę:gwaˀ ta:- ]}\\
\item \textit{O: tęˀ ǫh ga:t giˀ shęh gyę:gwaˀ} \trs{maybe, maybe not}, p. \pageref{p:[o: tęˀ ǫh ga:t giˀ shęh gyę:gwaˀ]}
\end{CayugaRelated}


\subsection*{\textbf{Gyę:gwaˀ giˀ shęh} \trs{or maybe}} \label{p:[gyę:gwaˀ giˀ shęh]}
\CayugaRemark{Particle group \textit{…, … gyę:gwaˀ giˀ shęh (…)} functions as a conjunction, connecting alternative choices [ ]; appears between or after the two choices (\sectref{ch:Clauses with conjunctions}). (For \textit{ne:ˀ giˀ shęh neˀ}, also shown below, see “Related”.)}

\ea
\label{ex:gpar141}
\gll [Dęgyadawęnye̱háˀ] \textbf{gyę:ˀ-gwáˀ} \textbf{giˀ-shę́h} [ne:ˀ gi̱ˀ-shę́h neˀ ęgyádawę:ˀ].\\
we.will.walk.about if maybe [it.is maybe the we.will.swim]\\
\glt ‘We could go for a walk or maybe go for a swim.’ (\cite[45]{michelson_ontario_2011})
\z

\begin{CayugaRelated}
\item \textit{Giˀ} \trs{just}, p. \pageref{p:[giˀ]}\\
\item \textit{Gyę:gwaˀ a:-verb} \trs{if}, p. \pageref{p:[gyę:gwaˀ]}\\
\item \textit{Ne:ˀ giˀ shęh neˀ … giˀ shęh} \trs{or}, p. \pageref{p:[ne:ˀ giˀ shęh neˀ … giˀ shęh]}\\
\item \textit{Shęh} \trs{that}, \trs{because}, p. \pageref{p:[shęh] `because’}
\end{CayugaRelated}


\subsection*{\textbf{Gyę:gwaˀ giˀ shęh hwaˀ} \trs{maybe this time}} \label{p:[gyę:gwaˀ giˀ shęh hwaˀ]}
\CayugaRemark{Particle group \textit{(g)yę:gwaˀ giˀ shęh hwaˀ} functions as a doubt or certainty marker (\sectref{ch:Evidential markers}); free-standing; \textit{gyę:gwaˀ} can also be pronounced as \textit{yę:gwaˀ}.}

\ea
\label{ex:gpar142}
\gll “\textbf{Yę́:gwaˀ} \textbf{yę́:gwaˀ} \textbf{giˀ} \textbf{shę́h} \textbf{hwaˀ},” aˀa:gę́ˀ.\\
if if just that this.time she.said\\
\glt ‘“If you do, maybe this time,” she said.’ (\cite{henry_de_2005})
\z

\begin{CayugaRelated}
\item \textit{Giˀ} \trs{just}, p. \pageref{p:[giˀ]}\\
\item \textit{Gyę:gwaˀ a:-verb} \trs{if}, p. \pageref{p:[gyę:gwaˀ]}\\
\item \textit{Gyę:gwaˀ giˀ shęh} \trs{or maybe}, p. \pageref{p:[gyę:gwaˀ giˀ shęh]}\\
\item \textit{Gyę:gwaˀ hne:ˀ hwaˀ} \trs{just maybe}, p. \pageref{p:[gyę:gwaˀ hne:ˀ hwaˀ]}\\
\item \textit{Hwaˀ} \trs{this time}, \trs{next}, p. \pageref{p:[hwaˀ]}\\
\item \textit{Shęh} \trs{that}, \trs{because}, p. \pageref{p:[shęh] `because’}
\end{CayugaRelated}


\subsection*{\textbf{Gyę:gwaˀ hne:ˀ hwaˀ} \trs{just maybe}} \label{p:[gyę:gwaˀ hne:ˀ hwaˀ]}
\CayugaRemark{Particle group functioning as a “possibility marker” (\sectref{ch:Evidential markers}).}

\ea
\label{ex:gpar143}
\gll gyę:gwaˀ hne:ˀ hwaˀ\\
if in.fact this.time\\
\glt ‘just maybe’
\z

\begin{CayugaRelated}
\item \textit{Gyę:gwaˀ a:-verb} \trs{if}, p. \pageref{p:[gyę:gwaˀ]}\\
\item \textit{Hne:ˀ} \trs{in fact}, p. \pageref{p:[hne:ˀ] ‘in fact’}\\
\item \textit{Hwaˀ} \trs{this time}, \trs{next}, p. \pageref{p:[hwaˀ]}
\end{CayugaRelated}


\subsection*{\textbf{Gyę:gwaˀ ta:-} \trs{if not}, \trs{if it hadn’t been}} \label{p:[gyę:gwaˀ ta:- ]}
\CayugaRemark{Particle, clause-initial in dependent conditional clauses [ ], (\sectref{ch:Conditional clauses with [gyę:gwaˀ], [gęh], [hę:gyeh] ‘if, whether’}); \textit{yę:gwaˀ} requires a following verb beginning with \stem{ta:-} \textsc{\contrastive-\indefinite} prefixes; the verb often lacks an aspect suffix.}

\ea
\label{ex:gpar144}
\gll Ahsǫ́h awaga̱hyagwę̱hęgye̱ˀsé:k [gyę:gwáˀ ta:waknǫ́hnya̱ˀgǫ:k]\\
Still I.would.have.been.going.along.picking.fruit if I.hadn’t.been.hurt\\
\glt ‘I would still have been picking fruit if I hadn’t been hurt.’
\z

\begin{CayugaRelated}
\item \textit{Gyę:gwaˀ a:-verb} \trs{if}, p. \pageref{p:[gyę:gwaˀ]}
\end{CayugaRelated}

\section{H particles}

\subsection*{\textbf{Hanyoh, Hanyo-hanyoh} \trs{do it!}, \trs{come on}, \trs{you go!}} \label{p:[hanyoh]}
\CayugaRemark{Particle group, preceding a command (\sectref{Particles and particle groups used with commands}) or stand-alone phrase. \textit{Hanyo-hanyoh} is used as a word of encouragement.}

\ea
\label{ex:hpar100}
\gll Dwadęhnǫdrǫ́ˀ, \textbf{hanyóh} sheswa:yé:na:!\\
brothers, come.on let.us.all.catch.it\\
\glt ‘Brothers, let’s hurry up and catch him!’ (\cite{carrier_legends_2013})
\z


\ea
\label{ex:hpar2}
\gll \textbf{Hányoh}. Dé:dwa:t.\\
come.on let’s.all.of.us.dance\\
\glt ‘Come on. We will dance.’ (\cite{carrier_legends_2013})
\z


\subsection*{\textbf{Haoˀ} \trs{o.k.}} \label{p:[haoˀ] ‘o.k.!’}
\CayugaRemark{Particle signaling acknowledgement (\sectref{ch:Acknowledgement markers}); free-standing.}

\ea
\label{ex:hpar3}
\gll O:nę́h aˀa:gę́ˀ, “Ja̱ˀsnęht to:gyę́h waˀjíh neˀ sǫ:gwaˀ nóht ęyagonǫ́hnyaˀk.”\\
now she.said you.two.descend there right.now the anybody person she.will.get.hurt\\
\glt ‘Then she said, “Both of you get down before somebody gets hurt.” (said by the mother)’\\
\gll “Ó:, \textbf{háoˀ}.”\\
“Oh, o.k.”\\
\glt ‘“Oh, o.k.” (said by the children)’ (\cite{henry_de_2005})
\z



\begin{CayugaRelated}
\item \textit{Haoˀ} \trs{come on}, p. \pageref{p:[haoˀ] ‘come on!’}\\
\item \textit{Haoˀ dęˀ nyoh} \trs{o.k. then}, p. \pageref{p:[haoˀ dęˀ nyoh]}\\
\item \textit{Haoˀ diˀ sah} \trs{alright}, \trs{o.k.}, p. \pageref{p:[haoˀ diˀ sah]}
\end{CayugaRelated}

\subsection*{\textbf{Haoˀ} \trs{come on}} \label{p:[haoˀ] ‘come on!’}
\CayugaRemark{Particle, preceding a command in these examples (\sectref{Particles and particle groups used with commands}).}

\ea
\label{ex:hpar4}
\gll \textbf{Haoˀ} dahskro:wíh dę́ˀ ní:ˀ hoˀdę́ˀ nęgá:gye:ˀ.\\
O.k., tell.me what I what I.will.do.it\\
\glt ‘O.k., you tell me what to do!’
\z

\ea
\label{ex:hpar5}
\gll \textbf{Haoˀ} gya̱hdę́:dih.\\
come.on let.us.two.go\\
\glt ‘Come on, let’s go.’ (\cite[142]{mithun_watewayestanih_1984}, Satgęh dialogue)
\z

\begin{CayugaRelated}
\item \textit{Haoˀ} \trs{o.k.}, p. \pageref{p:[haoˀ] ‘o.k.!’}
\end{CayugaRelated}

\subsection*{\textbf{Haoˀ dęˀ nyoh} \trs{o.k. then}} \label{p:[haoˀ dęˀ nyoh]}
\CayugaRemark{Particle group signaling acknowledgement (\sectref{ch:Acknowledgement markers}); free-standing.}

\ea
\label{ex:hpar6}
\gll \textbf{Haoˀ} \textbf{dęˀ} \textbf{nyóh} ó:nęh, o:nę́h agekwędáˀǫh.\\
ok what acknowledge now now I’ve.finished.eating\\
\glt ‘O.k., I have finished now.’ (\cite[142]{mithun_watewayestanih_1984}, Satgęh dialogue)
\z

\begin{CayugaRelated}
\item \textit{Dęˀ} \trs{what}, \trs{how}, p. \pageref{p:[dęˀ-]}\\
\item \textit{Haoˀ} \trs{o.k.}, p. \pageref{p:[haoˀ] ‘o.k.!’}\\
\item \textit{Nyoh} \trs{you’re welcome}, \trs{alright}, \trs{o.k.}, p. \pageref{p:[nyoh]}
\end{CayugaRelated}

\subsection*{\textbf{Haoˀ diˀ sah} \trs{alright}, \trs{o.k.}} \label{p:[haoˀ diˀ sah]}
\CayugaRemark{Particle group signaling acknowledgement (\sectref{ch:Acknowledgement markers}); free-standing. (\textit{Saˀ} means ‘look!’ or ‘see!’ in Tuscarora, \citealt{rudes_tuscarora-englishenglish-tuscarora_1999}.)}

\ea
\label{ex:hpar7}
\gll \textbf{Haoˀ} \textbf{diˀ} \textbf{sáh}. O:nę́h giˀ hyá:. Nya:wę́h giˀ gyę́:ˀ shęh nyó: dasknˀǫ́hda:.\\
ok so you.know now just time acknowledge just this.one that how.it.is you.put.me.in\\
\glt ‘Alright. So long. Thanks for the ride.’ (\cite[339]{mithun_watewayestanih_1984}, Oˀdréhdatgiˀ dialogue)
\z

\ea
\label{ex:hpar8}
\gll \textbf{Haoˀ} \textbf{diˀ} \textbf{sáh}, ękni:nǫ́ˀ giˀ gyę́:ˀǫh.\\
o.k. so you.know I.will.buy.it just I.guess\\
\glt ‘O.k., I guess I’ll buy it then.’ (\cite[225]{mithun_watewayestanih_1984}, Agyaˀdawíˀtraˀ dialogue)
\z

\begin{CayugaRelated}
\item \textit{Diˀ} \trs{so}, \trs{then}, p. \pageref{p:[diˀ]}\\
\item \textit{Haoˀ} \trs{o.k.}, p. \pageref{p:[haoˀ] ‘o.k.!’}\\
\item \textit{Sah} \trs{you know}, p. \pageref{p:[sah]}\\
\item \textit{Seˀ} \trs{you know}, p. \pageref{p:[seˀ]}
\end{CayugaRelated}


\subsection*{\textbf{Haǫhęˀ, Haǫhaˀ} \trs{he}} \label{p:[haǫhęˀ]}
\CayugaRemark{Verb functioning as an “emphatic pronoun” (\sectref{ch:’Emphatic pronouns’}); free-standing.}
\clearpage
\ea
\label{ex:hpar9} háǫhęˀ\\
\gll ha-ǫhęˀ\\
\textsc{3s.m.p}-alone.stative\\
\glt ‘he is alone’
\z



\subsection*{\stem{-heh} element (related to time)} \label{p:[-heh]}
\CayugaRemark{Element, enclitic in the following particle combinations, which function as adverbs of time (\sectref{ch:’Adverbs’ of time}).}

\begin{CayugaRelated}
\item \textit{Deˀę: gwaˀ-heh tęˀ seˀ} \trs{but then not really}, p. \pageref{p:[deˀę: gwaˀ-heh tęˀ seˀ]}\\
\item \textit{Gwahs waˀ-heh tsǫ:} \trs{just now}, \trs{just a few seconds ago}, p. \pageref{p:[gwahs waˀ-heh tsǫ:]}\\
\item \textit{Waˀ-heh} \trs{just now}, \trs{finally}, p. \pageref{p:[waˀ-heh]}\\
\item \textit{Waˀ-heh-geha:ˀ} \trs{finally}, \trs{at long last}, p. \pageref{p:[waˀ-heh-geha:ˀ]}
\end{CayugaRelated}

\subsection*{\textbf{Hehdaˀgeh gwa:dih} \trs{below}, \trs{low}} \label{p:[hehdaˀgeh gwa:dih]}
\CayugaRemark{Phrase \textit{hehdaˀgeh (gwa:dih)} functions as an “adverb of place” (\sectref{‘Adverbs’ of place}); clause-initial; \textit{gwa:dih} can also be spelled or pronounced as \textit{gwai}. \textit{Hehdaˀgeh} is possibly related to \textit{ohéhda̱ˀgeh} ‘on dirt, earth, ground, land’.}

\ea
\label{ex:hpar11}
\gll \textbf{hehdaˀgeh} \textbf{gwai}\\
on.the.ground side\\
\glt ‘the bottom’
\z

\ea
\label{ex:hpar1}
\gll \textbf{hehdaˀgéh} nitgá:deˀ\\
below it.exists.there.in.a.certain.place\\
\glt ‘it is low, it is below.’
\z

\begin{CayugaRelated}
\item \textit{Gwa:dih, Gwai} \trs{to one side}, p. \pageref{p:[gwa:dih]}
\end{CayugaRelated}

\subsection*{\textbf{Hehdaˀgeh hǫ:} \trs{downstairs}} \label{p:[hehdaˀgeh hǫ:]}
\CayugaRemark{Phrase functioning as an “adverb of place” (\sectref{‘Adverbs’ of place}); clause-initial. \textit{Hehdaˀgeh} is possibly related to \textit{ohéhda̱ˀgeh} ‘on dirt, earth, ground, land’.}

\ea
\label{ex:hpar12}
\gll \textbf{Hehdáˀgeh} \textbf{hǫ́:} hęhse:ˀ.\\
below place you.will.go.there\\
\glt ‘You go downstairs!’
\z

\begin{CayugaRelated}
\item \textit{Hehdaˀgeh gwa:dih} \trs{below}, \trs{low}, p. \pageref{p:[hehdaˀgeh gwa:dih]}\\
\item \textit{Hǫ:weh} \trs{where}, p. \pageref{p:[hǫ:weh]}
\end{CayugaRelated}

\subsection*{\textbf{Hehdagehjih} \trs{the bottom}, \trs{low}} \label{p:[hehdagehjih]}
\CayugaRemark{Atypical noun functioning as an “adverb of place” (\sectref{‘Adverbs’ of place}); clause-initial. \textit{Hehdaˀgeh} is possibly related to \textit{ohéhda̱ˀgeh} ‘on dirt, earth, ground, land’.}

\ea
\label{ex:hpar13}
\gll \textbf{Hehdagęhjíh} niga:gyéˀ degá:dęhs.\\
low it.is.flying plane\\
\glt ‘The plane is flying low.’
\z

\begin{CayugaRelated}
\item \textit{Hehdaˀgeh gwa:dih} \trs{below}, \trs{low}, p. \pageref{p:[hehdaˀgeh gwa:dih]}\\
\item \textit{Hehdaˀgeh hǫ:} \trs{downstairs}, p. \pageref{p:[hehdaˀgeh hǫ:]}
\end{CayugaRelated}

\largerpage[1.5]
\subsection*{\textbf{Hejo:yaˀ tsǫ:} \trs{another thing again}, \trs{on a tangent}} \label{p:[hejo:yaˀ tsǫ:] ‘another thing again’}
\CayugaRemark{Noun and particle functioning as an “indefinite pronoun” (\sectref{ch:’Indefinite pronouns’}); free-standing. Also metaphorically means, ‘on a tangent’, in which case it does not function as an “indefinite pronoun”. Also see \textit{o:yaˀ} `another, other, else’.}

\ea
\label{ex:hpar14}
\gll \textbf{hejo:yaˀ} \textbf{tsǫ:}\\
another just\\
\glt ‘another thing again’
\cfex{he-j-o:-y-aˀ {} \translocative-\repetitive-it-other-\nsf}
\z\clearpage

\begin{CayugaRelated}
\item \textit{Hejo:yaˀ tsǫ:} \trs{elsewhere}, p. \pageref{p:[hejo:yaˀ tsǫ:] ‘elsewhere’}\\
\item \textit{O:yaˀ} \trs{another}, \trs{other}, \trs{else}, p. \pageref{p:[o:yaˀ]}\\
\item \textit{Tsǫ:} \trs{just}, \trs{only}, p. \pageref{p:[tsǫ:]}
\end{CayugaRelated}

\subsection*{\textbf{Hejo:yaˀ tsǫ:} \trs{elsewhere}} \label{p:[hejo:yaˀ tsǫ:] ‘elsewhere’}
\CayugaRemark{Noun and particle functioning as an “adverb of place” (\sectref{‘Adverbs’ of place}); free-standing. Also see \textit{o:yaˀ} `another, other, else’.}


\begin{CayugaRelated}
\item \textit{Hejo:yaˀ tsǫ:} \trs{another thing again}, p. \pageref{p:[hejo:yaˀ tsǫ:] ‘another thing again’}\\
\item \textit{O:yaˀ} \trs{another}, \trs{other}, \trs{else}, p. \pageref{p:[o:yaˀ]}\\
\item \textit{Tsǫ:} \trs{just}, \trs{only}, p. \pageref{p:[tsǫ:]}
\end{CayugaRelated}

\subsection*{\textbf{He:tgęh} \trs{above}, \trs{up}, \trs{superior}} \label{p:[he:tgęh]}
\CayugaRemark{Atypical noun functioning as an “adverb of place” (\sectref{‘Adverbs’ of place}); also metaphorically means `superior’; clause-initial.}

\ea
\label{ex:hpar18}
\gll \textbf{hé:tgęh}\\
above\\
\glt ‘above, up, aloft’
\z

\ea\label{ex:hpar19}
\gll \textbf{he:tgę́h} tgá:deˀ\\
above it.exists\\
\glt ‘it is superior’
\z

\begin{CayugaRelated}
\item \textit{He:tgę̱hjih} \trs{the very top}, \trs{high up}, p. \pageref{p:[he:tgęhjih]}
\end{CayugaRelated}

\subsection*{\textbf{He:tgęhjih} \trs{the very top}, \trs{high up}} \label{p:[he:tgęhjih]}
\CayugaRemark{Atypical noun functioning as an “adverb of place” (\sectref{‘Adverbs’ of place}); clause-initial.}

\ea
\label{ex:hpar20}
\gll \textbf{Hetgęhjíh} niga:gyéˀ degá:dęhs.\\
high.up it.is.flying plane\\
\glt ‘The plane is flying high.’
\z

\begin{CayugaRelated}
\item \textit{He:tgęh} \trs{above}, \trs{up}, \trs{superior}, p. \pageref{p:[he:tgęh]}
\end{CayugaRelated}


\subsection*{\textbf{He:yo: dagwaishǫ:} \trs{straight ahead}} \label{p:[he:yo: dagwaishǫ:]}
\CayugaRemark{Phrase functioning as an “adverb of place” (\sectref{‘Adverbs’ of place}); clause-initial.}

\ea
\label{ex:hpar21}
\gll \textbf{He:yó:} \textbf{dagwaishǫ́:} haˀohahá:deˀ.\\
a.far.distance straight the.road.goes\\
\glt ‘The road goes straight ahead.’
\z

\ea
\label{ex:hpar22}
\gll \textbf{He:yó:} \textbf{dagwaishǫ́:} haˀohahadagyéˀ hęyǫ́gwadre:ˀ.\\
a.far.distance straight the.road.goes.there we.all.will.drive.there\\
\glt ‘We’ll drive on the straight road.’
\z

\begin{CayugaRelated}
\item \textit{Gwa:dih, Gwai} \trs{to one side}, p. \pageref{p:[gwa:dih]}\\
\item \textit{Senyęˀsgwadih} \trs{to your left}, p. \pageref{p:[senyęˀsgwadih]}\\
\item \textit{Sewaihǫhsdǫh gwa:dih} \trs{to your right}, p. \pageref{p:[sewaihǫhsdǫh gwa:dih]}
\end{CayugaRelated}

\subsection*{\textbf{Heyohe:ˀ} \trs{more}} \label{p:[heyohe:ˀ]}
\CayugaRemark{Verb functioning as an “adverb of degree” (\sectref{ch:’Adverbs’ of degree}); clause-initial.}

\ea
\label{ex:hpar15}
\gll O:nęh \textbf{heyohé:} agadǫ̱hswéˀdanih!\\
now it.is.beyond I’m.hungry\\
\glt ‘I am really hungry!’ (\cite{carrier_legends_2013})
\z

\ea
\label{ex:hpar16}
\gll \textbf{Heyohé:} ęhsnigǫ̱há:k tó:gyęh.\\
it.is.beyond you.will.take.care that.one\\
\glt ‘Now you will take care to walk more carefully.’ (\cite{henry_de_2005})
\z

\begin{CayugaRelated}
\item \textit{Heyohe:ˀ shęh} \trs{more than}, p. \pageref{p:[heyohe:ˀ shęh]}\\
\item \textit{Neˀ heyohe:ˀ} \trs{more}, ‘-er’, p. \pageref{p:[neˀ heyohe:ˀ]}\\
\item \textit{Ne:ˀ gwahs heyohe:ˀ} \trs{the most, -est}, p. \pageref{p:[ne:ˀ gwahs heyohe:ˀ]}
\end{CayugaRelated}

\subsection*{\textbf{Heyohe:ˀ shęh} \trs{more than}} \label{p:[heyohe:ˀ shęh]}
\CayugaRemark{Particle group \stem{heyohe:ˀ (pro)noun} … \stem{shęh (pro)noun} is used in comparisons (\sectref{ch:Comparisons (more, the same, or less)}).}

\ea
\label{ex:hpar17}
\gll \textbf{Heyohé:ˀ} [ní:ˀ] na:wagekwaędáˀk \textbf{shęh} [háǫhaˀ].\\
more I what.I.eat that him\\
\glt ‘\emph{I} need more food than \emph{him}.’ (\cite{carrier_legends_2013})
\z

\begin{CayugaRelated}
\item \textit{Heyohe:ˀ} \trs{more}, p. \pageref{p:[heyohe:ˀ]}\\
\item \textit{Neˀ heyohe:ˀ} \trs{more, -er}, p. \pageref{p:[neˀ heyohe:ˀ]}\\
\item \textit{Ne:ˀ gwahs heyohe:ˀ} \trs{the most, -est}, p. \pageref{p:[ne:ˀ gwahs heyohe:ˀ]}\\
\item \textit{Shęh} \trs{that}, \trs{because}, p. \pageref{p:[shęh] `because’}
\end{CayugaRelated}

\largerpage
\subsection*{\textbf{Hęˀ} \trs{also}, \trs{too}} \label{p:[hęˀ]}
\CayugaRemark{Particle signaling the continuation of a previous topic and the addition of new information (\sectref{ch:Topic continuation markers}); enclitic.}

\begin{CayugaRelated}
\item \textit{Ne:ˀ hęˀ hne:ˀ} \trs{too}, \trs{also}, p. \pageref{p:[ne:ˀ hęˀ hne:ˀ]}\\
\item \textit{Seˀ hęˀ ni:ˀ} \trs{me too}, \trs{us too}, p. \pageref{p:[seˀ hęˀ ni:ˀ]}
\end{CayugaRelated}

\subsection*{\textbf{Hę:} syllable} \label{p:[hę:] syllable}
\CayugaRemark{Syllable often appearing at the beginning of speech lines, and often with the sentence connector \textit{dah} `and’, as in the phrase \textit{hę:dah} `and now’. An example from \textit{Ganǫ́hǫnyǫhk} ‘Thanksgiving Address’ in (\cite{foster_earth_1974}) is shown below.}

\ea
\label{ex:hpar28}
\gll \textbf{hę:}-né: neˀ í: nę:-gyę́h, agaihwáyędáˀk, nę:dá niyóihwáˀ.\\
hę-it’s the me this the.matter.has.been.laid this speech.\\
\glt ‘The responsibility for this speech has been given to me. (\cite{foster_earth_1974})
\z


\begin{CayugaRelated}
\item \textit{Da:} \trs{and}, p. \pageref{p:[da:]}\\
\item \textit{Hę:-dah} \trs{and now}, p. \pageref{p:[hę:-dah]}
\end{CayugaRelated}


\subsection*{\textbf{Hęˀ hne:ˀ, Ne:ˀ hęˀ hne:ˀ} \trs{too}, \trs{also}} \label{p:[hęˀ hne:ˀ]}
\CayugaRemark{Particle group \textit{(ne:ˀ) hęˀ hne:ˀ} emphasizes the continuation of a previous topic and the addition of new information (\sectref{ch:Topic continuation markers}); \textit{ne:ˀ hęˀ hne:ˀ} is clause-initial; \textit{…hęˀ hne:ˀ} is enclitic. }

\begin{CayugaRelated}
\item \textit{Ne:ˀ hęˀ hne:ˀ} \trs{too}, \trs{also}, p. \pageref{p:[ne:ˀ hęˀ hne:ˀ]}
\end{CayugaRelated}

\subsection*{\textbf{Hęˀ ni:ˀ / ni:hs, Seˀ hęˀ ni:ˀ / ni:hs} \trs{me too}, \trs{us too}, \trs{you too}} \label{p:[hęˀ ni:ˀ]}
\CayugaRemark{Particle group \textit{… (seˀ) hęˀ ni:ˀ} functions as an emphatic pronoun (\sectref{ch:’Emphatic pronouns’}); enclitic.}

\ea
\label{ex:hpar23}
\gll A:ga̱hyagwa̱hsé:k \textbf{seˀ} \textbf{hęˀ} \textbf{ni:ˀ} gyę:gwáˀ a:wagadagáide:k.\\
I.would.pick.fruit you.know also the.me if I.would.be.well\\
\glt ‘I would be a fruit picker too if I were well.’
\z

\ea
\label{ex:hpar24}
\gll Agahyagwęhné:ˀ \textbf{seˀ} \textbf{hęˀ} \textbf{ni:ˀ}.\\
I’ve.picked.fruit you.know also the.me\\
\glt ‘I’ve also picked fruit.’ (I’ve experienced this)
\z


\ea
\label{ex:hpar25}
\gll Ęgahyagwahsé:k \textbf{hęˀ} \textbf{ni:ˀ} nęh ęgáhdo:k.\\
I’ll.pick.fruit also the.me when I.will.mature\\
\glt ‘I’ll be a fruit picker when I grow up.’
\z


\ea
\label{ex:hpar26}
\gll Ó:, gǫdagyéˀ \textbf{hęˀ} \textbf{ni:ˀ} haˀa:kni:yǫ́ˀ!\\
oh right.away also the.us we.two.went.in\\
\glt ‘Oh, right away we both got up and went in!’ (\cite{henry_de_2005})
\z


\ea
\label{ex:hpar27}
\gll Ęhę́ˀ, ǫgwatgaˀdé:ˀah. Swa̱hwajiyowanę́h di̱ˀ-gęh \textbf{hęˀ} \textbf{ni:s}?\\
yes we.are.many your.family.is.big so-Q also the.you\\
\glt ‘Yes, there are a lot of us. Do you have a big family too?’(\cite[257]{mithun_watewayestanih_1984}, Ga̱hwajiyá:deˀ dialogue)
\z


\begin{CayugaRelated}
\item \textit{Hęˀ} \trs{also}, \trs{too}, p. \pageref{p:[hęˀ]}\\
\item \textit{I:ˀ, Ni:ˀ} \trs{I}, \trs{we}, p. \pageref{p:[i:ˀ]}\\
\item \textit{I:s, Ni:s} \trs{you} (any number of people), p. \pageref{p:[i:s]}\\
\item \textit{Seˀ} \trs{you know}, p. \pageref{p:[seˀ]}
\end{CayugaRelated}

\subsection*{\textbf{Hę:-dah} \trs{and now}} \label{p:[hę:-dah]}
\CayugaRemark{Particle group functioning as a sentence connector; used at the beginning of lines in speeches.}

\ea
\label{ex:hpar29}
\gll \textbf{hę}-dá: o:nę́h, to niyó:wéˀ, nigahá:wíˀ.\\
hę-and now that is.how.far it.is.carried.\\
\glt ‘And now the time has come.’ (\cite{foster_earth_1974})
\z


\begin{CayugaRelated}
\item \textit{Da:} \trs{and}, p. \pageref{p:[da:]}\\
\item \textit{Hę:} (syllable), p. \pageref{p:[hę:] syllable}
\end{CayugaRelated}

\subsection*{\textbf{Hę:-gyęh, Hę:-gyeh} \trs{no matter}} \label{p:[hę:-gyęh]}
\CayugaRemark{Particle group \textit{hę:-gyęh} or \textit{hę:-gyeh} functions as an evidential marker (\sectref{ch:Evidential markers}); free-standing or clause-initial.}

\ea
\label{ex:hpar30}
\gll \textbf{Hę:gyę́h}. Tęˀ gyę́:ˀ ta:yoyęnawáˀs neˀ otgǫˀtraˀ neˀ o:nę́h dwa̱hsreˀ.\\
no.matter not emphasis it.won’t.help.it the magic the now we.are.following.him\\
\glt ‘Not for long. His magic is no good once we have his trail.’ (\cite{carrier_legends_2013})
\z


\begin{CayugaRelated}
\item \textit{Aweˀ hę:gyeh tsǫ, Awęˀ hę:gyeh tsǫ:} \trs{let it go}, p. \pageref{p:[aweˀ hę:gyeh tsǫ:]}\\
\item \stem{-gyęh} \trs{this}, \trs{that}, p. \pageref{p:[-gyęh]}\\
\item \textit{Hę:gyeh gaę hǫ:weh} \trs{no matter where}, \trs{no matter which place}, p. \pageref{p:[hę:-gyeh gaę hǫ:weh]}\\
\item \textit{Hę:gyeh giˀ} \trs{leave well enough alone}, p. \pageref{p:[hę:-gyeh giˀ]}\\
\item \textit{Hę:-gyęh shęh} \trs{no matter how much}, \trs{whether or not}, \trs{even if}, p. \pageref{p:[hę:-gyęh shęh]}\\
\item \textit{Hę:-gyęh tsǫ:} \trs{it doesn’t matter}, \trs{never mind}, p. \pageref{p:[hę:-gyęh tsǫ:]}\\
\item \textit{To hę:gyeh shęh} \trs{really}, p. \pageref{p:[to hę:gyeh shęh]}
\end{CayugaRelated}

\subsection*{\textbf{Hę:-gyeh gaę hǫ:weh} \trs{no matter where}, \trs{no matter which place}} \label{p:[hę:-gyeh gaę hǫ:weh]}
\CayugaRemark{Particle group \textit{hę:-gyęh / hę:-gyeh gaę hǫ:weh} functions as an “adverb of place” (\sectref{‘Adverbs’ of place}); clause-initial; a following verb requires the \stem{ni-} {\partitive}, \stem{d-} {\cislocative}, or \stem{heˀ-} {\translocative} prefix; \textit{hǫ:weh} can be spelled or pronounced as \textit{hǫ:}, \textit{nhǫ:}, \textit{hǫ:weh}, or \textit{nhǫ:weh}.}

\ea
\label{ex:hpar31}
\gll Jiˀdę:ˀę́h ni:ˀ, ę: tsǫ: ęhsgé:gyeˀ \textbf{hę:gyéh} \textbf{gaę} \textbf{hǫ:wéh} ęwa:dǫ́ˀ hę́:ge:ˀ.”\\
birds I, indeed just I.go.along.seeing.them.again no.matter which place it.is.possible I.will.go.there\\
\glt ‘Birds are all I see no matter where I may go.’ (\cite{henry_de_2005})
\z


\begin{CayugaRelated}
\item \textit{Aweˀ hę:gyeh tsǫ, Awęˀ hę:gyeh tsǫ:} \trs{let it go}, p. \pageref{p:[aweˀ hę:gyeh tsǫ:]}\\
\item \textit{Hę:-gyęh, hę:gyeh} \trs{no matter}, p. \pageref{p:[hę:-gyęh]}\\
\item \textit{Hę:gyeh giˀ} \trs{leave well enough alone}, p. \pageref{p:[hę:-gyeh giˀ]}\\
\item \textit{Hę:-gyęh shęh} \trs{no matter how much}, \trs{whether or not}, \trs{even if}, p. \pageref{p:[hę:-gyęh shęh]}\\
\item \textit{Hę:-gyęh tsǫ:} \trs{it doesn’t matter}, \trs{never mind}, p. \pageref{p:[hę:-gyęh tsǫ:]}\\
\item \textit{To hę:gyeh shęh} \trs{really}, p. \pageref{p:[to hę:gyeh shęh]}
\end{CayugaRelated}

\subsection*{\textbf{Hę:-gyeh giˀ} \trs{leave well enough alone}} \label{p:[hę:-gyeh giˀ]}
\CayugaRemark{Particle group \textit{hę:gyeh / hę:gyęh giˀ} functions as a “command” (\sectref{Particles and particle groups used with commands}).}

\ea
\label{ex:hpar32}
\gll \textbf{Hę:-gyęh} \textbf{giˀ}!\\
no.matter just\\
\glt ‘Leave well enough alone!’
\z


\begin{CayugaRelated}
\item \textit{Aweˀ hę:gyeh tsǫ, Awęˀ hę:gyeh tsǫ:} \trs{let it go}, p. \pageref{p:[aweˀ hę:gyeh tsǫ:]}\\
\item \textit{Hę:-gyęh, hę:gyeh} \trs{no matter}, p. \pageref{p:[hę:-gyęh]}\\
\item \textit{Hę:gyeh gaę hǫ:weh} \trs{no matter where}, \trs{no matter which place}, p. \pageref{p:[hę:-gyeh gaę hǫ:weh]}\\
\item \textit{Hę:-gyęh shęh} \trs{no matter how much}, \trs{whether or not}, \trs{even if}, p. \pageref{p:[hę:-gyęh shęh]}\\
\item \textit{Hę:-gyęh tsǫ:} \trs{it doesn’t matter}, \trs{never mind}, p. \pageref{p:[hę:-gyęh tsǫ:]}\\
\item \textit{To hę:gyeh shęh} \trs{really}, p. \pageref{p:[to hę:gyeh shęh]}
\end{CayugaRelated}

\subsection*{\textbf{Hę:-gyęh shęh} \trs{no matter how much}, \trs{whether or not}, \trs{even if}} \label{p:[hę:-gyęh shęh]}
\CayugaRemark{Particle group \textit{hę:-gyęh / hę:-gyeh (shęh) …} is clause-initial in conditional clauses (\sectref{ch:Conditional clauses with [gyę:gwaˀ], [gęh], [hę:gyeh] ‘if, whether’}).}

\ea
\label{ex:hpar33}
\gll Jiˀdę:ˀę́h ní:ˀ, ę: tsǫ: ęhsge:gyéˀ [\textbf{hę:gyéh} gaę hǫ:wéh ęwa:dǫ́ˀ hę́:ge:ˀ].”\\
birds I, affirm just I.will.fly.again [no.matter which way it.may.be I.will.go.there]\\
\glt ‘Birds are all I see no matter where I may go.’ (\cite{henry_de_2005})
\z


\ea
\label{ex:hpar34}
\gll Tę́ˀ dˀáǫ a:wadahsgwi̱ˀtrǫ́:niˀ, [\textbf{hę:gyę́h} neˀ haesatroni̱há:k aesę́:daˀ].\\
not not.at.all it.won’t.wrinkle, even.if the you.would.have.clothes.on you.would.sleep\\
\glt ‘It won’t wrinkle, even if you sleep with your clothes on.’ (\cite[225]{mithun_watewayestanih_1984}, Agyaˀdawíˀtraˀ dialogue)
\z


\ea
\label{ex:hpar35}
\gll [\textbf{Hę:gyę́h} \textbf{shęh} agadatsę́hse:], tga:gǫ́:t ęge:tsę́iˀ.\\
no.matter that I’m.tired, it.must.be I.will.find.it\\
\glt ‘Although I am tired, I have to find it.’ (\cite{carrier_legends_2013})
\z


\ea
\label{ex:hpar36}
\gll [\textbf{Hę:gyéh} \textbf{shę́h} de̱ˀǫgwadǫtga:déˀ] “onę́h-gi̱ˀhyá:” ayagwę́ˀ waˀjíh hné: tsǫ́: ęja:kí:gęh.\\
no.matter that we.are.not.happy, “good-bye” we.say in.a.while in.fact just we.will.see.them.again\\
\glt ‘And though we’re sad to say good-bye, we’ll see them in a while.’ (\cite{keye_circle_2016}, Circle Book 10, On Our Way)
\z


\begin{CayugaRelated}
\item \textit{Aweˀ hę:gyeh tsǫ, Awęˀ hę:gyeh tsǫ:} \trs{let it go}, p. \pageref{p:[aweˀ hę:gyeh tsǫ:]}\\
\item \stem{-gyęh} \trs{this}, \trs{that}, p. \pageref{p:[-gyęh]}\\
\item \textit{Hę:-gyęh, Hę:gyeh} \trs{no matter}, p. \pageref{p:[hę:-gyęh]}\\
\item \textit{Hę:gyeh gaę hǫ:weh} \trs{no matter where}, \trs{no matter which place}, p. \pageref{p:[hę:-gyeh gaę hǫ:weh]}\\
\item \textit{Hę:gyeh giˀ} \trs{leave well enough alone}, p. \pageref{p:[hę:-gyeh giˀ]}\\
\item \textit{Hę:-gyęh tsǫ:} \trs{it doesn’t matter}, \trs{never mind}, p. \pageref{p:[hę:-gyęh tsǫ:]}\\
\item \textit{To hę:gyeh shęh} \trs{really}, p. \pageref{p:[to hę:gyeh shęh]}
\end{CayugaRelated}

\subsection*{\textbf{Hę:-gyęh tsǫ:} \trs{it doesn’t matter}, \trs{never mind}} \label{p:[hę:-gyęh tsǫ:]}
\CayugaRemark{Particle group \textit{hę:-gyęh / hę:-gyeh tsǫ:} functions as an “evidential marker” (\sectref{ch:Evidential markers}); free-standing, or clause-initial.}

\ea
\label{ex:hpar37}
\gll Tę́ˀ. \textbf{Hęgyę́h} \textbf{tsǫ:}. Sede̱hjí:hah ní:ˀ ęgahdę:díˀ gǫ́dagyeˀ.\\
no no.matter just early.morning I I.will.leave immediately\\
\glt ‘No, never mind, I’ll go there right away, early in the morning.’ (\cite[494]{mithun_watewayestanih_1984}, Ękníyętoˀ dialogue)
\z


\begin{CayugaRelated}
\item \textit{Aweˀ hę:gyeh tsǫ, Awęˀ hę:gyeh tsǫ:} \trs{let it go}, p. \pageref{p:[aweˀ hę:gyeh tsǫ:]}\\
\item \stem{-gyęh} \trs{this}, \trs{that}, p. \pageref{p:[-gyęh]}\\
\item \textit{Hę:-gyęh, Hę:gyeh} \trs{no matter}, p. \pageref{p:[hę:-gyęh]}\\
\item \textit{Hę:gyeh gaę hǫ:weh} \trs{no matter where}, \trs{no matter which place}, p. \pageref{p:[hę:-gyeh gaę hǫ:weh]}\\
\item \textit{Hę:gyeh giˀ} \trs{leave well enough alone}, p. \pageref{p:[hę:-gyeh giˀ]}\\
\item \textit{Hę:-gyęh shęh} \trs{no matter how much}, \trs{whether or not}, \trs{even if}, p. \pageref{p:[hę:-gyęh shęh]}\\
\item \textit{To hę:gyeh shęh} \trs{really}, p. \pageref{p:[to hę:gyeh shęh]}
\end{CayugaRelated}


\subsection*{\textbf{Hęˀęh} \trs{no}} \label{p:[hęˀęh]}
\CayugaRemark{Particle signaling disagreement (\sectref{ch:Disagreement markers}); said to be a slang-like expression.}

\subsection*{\textbf{Hne:ˀ} \trs{but}} \label{p:[hne:ˀ] ‘but’}
\CayugaRemark{Particle functioning as a “conjunction” (\sectref{ch:Clauses with conjunctions}); enclitic.}

\ea
\label{ex:hpar38}
\gll Tęˀ hne:ˀ [de̱hahǫkáˀ ǫgwehǫ́:weh], [hoˀnigǫ̱haędaˀs] \textbf{hné:ˀ}.\\
not in.fact [he.doesn’t.speak Indian], [he.understands] in.fact\\
\glt ‘He doesn’t speak Indian, but he does understand.’ (\cite[61]{mithun_watewayestanih_1984}, Sǫ: hne:ˀ nˀaht to:gyęh? dialogue)
\z


\begin{CayugaRelated}
\item \textit{Hne:ˀ} \trs{in fact}, p. \pageref{p:[hne:ˀ] ‘in fact’}
\end{CayugaRelated}

\subsection*{\textbf{Hne:ˀ} \trs{in fact} (emphatic)} \label{p:[hne:ˀ] ‘in fact’}
\CayugaRemark{Particle used for emphasis or focus (\sectref{ch:Focus, contrastive focus, and emphasis markers}, also see \sectref{ch:[hne:ˀ], [ne:ˀ … (tsǫ: shęh)] ‘but’}); enclitic.}

\ea
\label{ex:hpar39}
\gll Ne:ˀ \textbf{hne:ˀ} gó:węh.\\
it.is in.fact she.owns.it\\
\glt ‘That’s hers in fact.’ `That’s hers.’
\z


\ea
\label{ex:hpar40}
\gll Dęˀ \textbf{hne:ˀ} í:se:ˀ?\\
what in.fact you.want.it\\
\glt ‘What in fact do you want?’ `What do you want?’
\z


\ea
\label{ex:hpar41}
\gll Dęˀ \textbf{hne:ˀ} hoˀdęˀ?\\
what in.fact kind\\
\glt ‘What kind in fact?’ `What kind?’
\z


\ea
\label{ex:hpar42}
\gll Dęˀ \textbf{hne:ˀ} ni:yóht shęh toh nahsye:ˀ.\\
what in.fact a.certain.way that that.one you.did.it.a.certain.way\\
\glt ‘Why in fact did you do that?’ ``Why did you do that?’’
\z


\ea
\label{ex:hpar43}
\gll I:s gęh \textbf{hne:ˀ}?\\
you Q in.fact\\
\glt ‘Was that in fact you?’ `Was that you?’
\z


\ea
\label{ex:hpar44}
\gll I:s diˀ gęh \textbf{hne:ˀ}?\\
you so Q in.fact\\
\glt ‘How about you then?’ `How about you?’
\z


\begin{CayugaRelated}
\item \textit{Dęˀ hne:ˀ hoˀdęˀ} \trs{what}, p. \pageref{p:[dęˀ hne:ˀ hoˀdęˀ]}\\
\item \textit{Gęh hne:ˀ} \trs{how about this one?}, p. \pageref{p:[gęh hne:ˀ]}\\
\item \textit{Giˀ hne:ˀ} \trs{but}, \trs{however}, p. \pageref{p:[giˀ hne:ˀ]}\\
\item \textit{Gyę:ˀ hne:ˀ tęˀ neˀ} \trs{it was this one, (not that one)}, p. \pageref{p:[gyę:ˀ hne:ˀ]}\\
\item \textit{Gyę:gwaˀ hne:ˀ hwaˀ} \trs{just maybe}, p. \pageref{p:[gyę:gwaˀ hne:ˀ hwaˀ]}\\
\item \textit{Hne:ˀ} \trs{but}, p. \pageref{p:[hne:ˀ] ‘but’}\\
\item \textit{Hne:ˀ shęh} \trs{because}, p. \pageref{p:[hne:ˀ shęh]}\\
\item \textit{I:ˀ hne:ˀ} \trs{no, \emph{I} am}, p. \pageref{p:[i:ˀ hne:ˀ]}\\
\item \textit{Ne:ˀ hęˀ hne:ˀ} \trs{too}, \trs{also}, p. \pageref{p:[ne:ˀ hęˀ hne:ˀ]}\\
\item \textit{Ne:ˀ hne:ˀ ne:ˀ} \trs{in fact, it is}, \trs{it \emph{is}}, p. \pageref{p:[ne:ˀ hne:ˀ ne:ˀ]}\\
\item \textit{Si hne:ˀ si gwa:dih} \trs{over there}, p. \pageref{p:[si hne:ˀ si gwa:dih]}\\
\item \textit{Tęˀ hne:ˀ} \trs{definitely not}, p. \pageref{p:[tęˀ hne:ˀ]}\\
\item \textit{Tęˀ hne:ˀ neˀ} \trs{…not \emph{that one}}, p. \pageref{p:[tęˀ hne:ˀ neˀ]}
\end{CayugaRelated}

\subsection*{\textbf{Hne:ˀ shęh} \trs{because}} \label{p:[hne:ˀ shęh]}
\CayugaRemark{Particle group \textit{hne:ˀ (shęh)} is clause-initial in a causative dependent clause (\sectref{ch:Causative clauses with [dęˀ ni:yoht shęh] ‘why’, [neˀ hǫ:niˀ] ‘how’}).}

\ea
\label{ex:hpar45}
\gll O:nę́h giˀ gyę́:ˀ-ǫh ęsgahdę́:diˀ. O:nę́h gyę́:ˀ nǫ́ne:ˀ [\textbf{hné:ˀ} haˀwa:jˀáht degáhswa̱ˀne:t]!\\
Now just emphasis-I.guess I.will.go.home. Now emphasis you.know [in.fact it.has.gone.down.to.nothing pie]\\
\glt ‘Well, I might as well go home now, [because the pie is all gone], you know!’ (\cite[376]{mithun_watewayestanih_1984}, Dajǫh dialogue)
\z


\ea
\label{ex:hpar46}
\gll Oyanra̱hsdǫ́ giˀ [\textbf{hne:ˀ} \textbf{shęh} i:sóˀ oˀgráęˀ].\\
It.is.ideal just in.fact that a.lot there.is.snow\\
\glt ‘It is ideal because there is a lot of snow.’ (Mithun and Henry 1982, 481, Dędwaˀęnáę dialogue)
\z


\begin{CayugaRelated}
\item \textit{Hne:ˀ} \trs{in fact}, p. \pageref{p:[hne:ˀ] ‘in fact’}\\
\item \textit{Shęh} \trs{that}, \trs{because}, p. \pageref{p:[shęh] `because’}
\end{CayugaRelated}

\subsection*{\textbf{Hniˀ} \trs{and}} \label{p:[hniˀ] ‘and’}
\CayugaRemark{Particle functioning as a “conjunction” (\sectref{ch:Clauses with conjunctions}), linking a list of similar items [ ]; enclitic; tends to occur at the end of the list of items.}

\ea
\label{ex:hpar47}
\gll De̱hęnadatnǫ̱hwéˀs [Gwi:déh] [Éd] \textbf{hniˀ} shę́h hona:dáoˀ\\
they.like.each.other [Peter] [Ed] and that they.are.friends\\
\glt ‘Peter and Ed like each other because they are friends.’
\z


\ea
\label{ex:hpar48}
\gll Ó:, haˀdeyo̱hsóhgwage:, [otgwę́hji̱ˀaˀ], [ǫhyˀáęˀ], [swęˀdˀáęˀ], [nrahdˀáęˀ], dewagye̱sahsdǫ́ \textbf{hniˀ} [neˀ gę:ˀgę́:] niyohso̱hgoˀdęˀs.\\
Oh, all.sorts.of.colours, [red], [blue], [black], [green] all.mixed.up and [the white] what.kinds.of.colours\\
\glt ‘Oh, all sorts of colors, red, blue, black, (and) green, and some that are white.’ (\cite[225]{mithun_watewayestanih_1984}, Agyaˀdawíˀtraˀ dialogue)
\z


\begin{CayugaRelated}
\item \textit{Da: ne:ˀ hniˀ} \trs{too}, \trs{also}, \trs{and}, p. \pageref{p:[da: ne:ˀ, da: ne:ˀ hniˀ]}\\
\item \textit{Dęˀ hniˀ} \trs{for sure}, p. \pageref{p:[dęˀ hniˀ]}\\
\item \textit{Hniˀ} \trs{and}, \trs{also}, \trs{too}, p. \pageref{p:[hniˀ] ‘and’, ‘also’, ‘too’}\\
\item \textit{Ne:ˀ giˀ hniˀ} \trs{and that too}, \trs{and that also}, p. \pageref{p:[ne:ˀ giˀ hniˀ]}\\
\item \textit{Ne:ˀ hniˀ ne:ˀ} \trs{and that also}, p. \pageref{p:[ne:ˀ hniˀ ne:ˀ]}
\end{CayugaRelated}

\subsection*{\textbf{Hniˀ} \trs{and}, \trs{also}, \trs{too}} \label{p:[hniˀ] ‘and’, ‘also’, ‘too’}
\CayugaRemark{Particle signaling the continuation of a previous topic and the addition of new information (\sectref{ch:Topic continuation markers}); enclitic.}

\ea
\label{ex:hpar49}
\gll Dęˀ \textbf{hniˀ} hoˀdę́ˀ hoihoˀdę́hsro̱ˀdeˀ?\\
What and what his.kind.of.work\\
\glt ‘And what does he do?’
\z


\begin{CayugaRelated}
\item \textit{Da: ne:ˀ hniˀ} \trs{too}, \trs{also}, \trs{and}, p. \pageref{p:[da: ne:ˀ, da: ne:ˀ hniˀ]}\\
\item \textit{Dęˀ hniˀ} \trs{for sure}, p. \pageref{p:[dęˀ hniˀ]}\\
\item \textit{Hniˀ} \trs{and}, p. \pageref{p:[hniˀ] ‘and’}\\
\item \textit{Ne:ˀ giˀ hniˀ} \trs{and that too}, \trs{and that also}, p. \pageref{p:[ne:ˀ giˀ hniˀ]}\\
\item \textit{Ne:ˀ hniˀ ne:ˀ} \trs{and that also}, p. \pageref{p:[ne:ˀ hniˀ ne:ˀ]}
\end{CayugaRelated}

\subsection*{\textbf{Hoˀdęˀ} \trs{kind}} \label{p:[hoˀdęˀ]}
\CayugaRemark{Particle functioning as a “definite pronoun” \sectref{ch:’Definite pronouns’}; free-standing.}

\ea
\label{ex:hpar52}
\gll O:nę́h giˀ adyǫkniˀdraihę́h, aˀakniksaohé:k gwé:gǫh \textbf{hoˀdę́ˀ}, \textbf{hoˀdę́ˀ} ęyá:gwahs.\\
now then we.hurried.up we.gathered all the.kind, the.kind we.will.use\\
\glt ‘Then we hurried up and gathered up all the dishes that we would use.’  (\cite{henry_de_2005})
\z


\ea
\label{ex:hpar53}
\gll Pete: A:yę́:ˀ gajihyó:t \textbf{hoˀdę́ˀ} neˀ onáˀda:ˀ.\\
{} it.seems oven.bread the.kind the bread\\
\glt ‘It looks like homemade biscuits.’

\gll Sam: Né:ˀ tó ne:ˀ do:gę́hs \textbf{hoˀdęˀ}. Gonaˀda:yędeiˀ-ǫ́: gyę́:ˀ kegę́jih.\\
{} it.is that it.is it.is.true the.kind she.is.a.good.baker-I.suppose that.one my.wife\\
\glt ‘They sure are. She is a good baker, my wife.’ (\cite[294]{mithun_watewayestanih_1984}, Dwadé:kǫ́:nih dialogue)
\z

\ea
\label{ex:hpar54}
\gll Ęhę́ˀ, né:ˀ giˀ neˀ gwáhs gę:s stǫ:hǫh oya:nré:ˀah \textbf{hóˀdęˀ}.\\
yes, it.is just the more usually a.little.bit better-ish the.kind\\
\glt ‘Yes, it is one of the better ones.’ (\cite[456]{mithun_watewayestanih_1984}, Oyę́hsraˀ dialogue)
\z


\begin{CayugaRelated}
\item \textit{Dęˀ, Dęˀ hoˀdęˀ} \trs{what}, p. \pageref{p:[dęˀ, dęˀ hoˀdęˀ]}\\
\item \textit{Dęˀ gwaˀ hoˀdęˀ} \trs{whatever one (of several)}, p. \pageref{p:[dęˀ gwaˀ hoˀdęˀ]}\\
\item \textit{Dęˀ hne:ˀ hoˀdęˀ} \trs{what}, p. \pageref{p:[dęˀ hne:ˀ hoˀdęˀ]}\\
\item \textit{Dęˀ hoˀdęˀ ni-} \trs{how}, \trs{what way}, p. \pageref{p:[dęˀ hoˀdęˀ ni-]}\\
\item \textit{Dęˀ ǫh hne:ˀ hoˀdęˀ} \trs{what on earth?}, p. \pageref{p:[dęˀ ǫh hne:ˀ hoˀdęˀ]}\\
\item \textit{Sga̱hoˀdę:ˀęh} \trs{something}, p. \pageref{p:[sgahoˀdę:ˀęh]}\\
\item \textit{Sga̱hoˀdę:ˀęh o:yaˀ} \trs{anything else}, p. \pageref{p:[sgahoˀdę:ˀęh o:yaˀ]}\\
\item \textit{Shęh hoˀdęˀ} \trs{a certain something}, \trs{that which}, p. \pageref{p:[shęh hoˀdęˀ]}
\end{CayugaRelated}


\subsection*{\textbf{Hoh, Ho:} \trs{what the?}, \trs{yay!}} \label{p:[hoh, ho:]}
\CayugaRemark{Particle functioning as an “exclamation” (\sectref{ch:Exclamations}), conveying surprise or enthusiasm.}

\ea
\label{ex:hpar50}

\gll “I:yę́: ǫ:wí: ęyǫtna̱ˀdáǫdęˀ,” aˀa:gę́ˀ, “to:gyę́h ogaˀǫ́h ehstaˀ.”\\
she.wants I.think she.will.bake.bread she.said this.one it.tastes.good she.makes\\
\glt ‘“I do believe she is going to bake,” she said, “that good-tasting bread she makes.”’


\gll \textbf{Hó:}! Heje:kni:yǫ́h giˀ gyę:ˀ!\\
ho let’s.go.in just emphasis\\
\glt ‘Ho! Let’s go in!’
(\cite{henry_de_2005})

\z

\ea
\label{ex:hpar51}
\gll \textbf{Hó:} oya:di:yó: giˀ!\\
ho, good.track just\\
\glt ‘Oh, it is a good track.’ (\cite[481]{mithun_watewayestanih_1984}, Dędwaˀęnáęˀ dialogue)
\z


\subsection*{\textbf{Hoho:} \trs{aha!}, \trs{oh no!}} \label{p:[hoho:]}
\CayugaRemark{Particle functioning as an “exclamation” (\sectref{ch:Exclamations}); said in anticipation of something bad or good (for example, someone’s about to win at bingo, or have a fight).}




\subsection*{\textbf{Hona:węh} \trs{it's theirs (males only)}, \trs{it belongs to them (males only)}, \trs{their (males only)}} \label{p:[hona:węh]}
\CayugaRemark{Verb functioning as a “possessive pronoun” (\sectref{ch:’Possessive pronouns’}); free-standing.}

\ea
\label{ex:hpar55} \textbf{honá:węh}\\
\gll hon-á:w-ęh\\
\textsc{3ns.m.p}-own-{\stative}\\
\glt ‘they (males) own it’
\z



\subsection*{\textbf{Honǫ:hęˀ} \trs{they (males)}} \label{p:[honǫ:hęˀ]}
\CayugaRemark{Verb functioning as an “emphatic pronoun” (\sectref{ch:’Emphatic pronouns’}); free-standing.}

\ea
\label{ex:hpar56} \textbf{honǫ́:hęˀ}\\
\gll hon-ǫ́:hęˀ\\
\textsc{3ns.m.p}-alone.{\stative}\\
\glt ‘they (males) are alone’
\z



\subsection*{\textbf{Hotgǫˀǫh, Otgǫˀ} \trs{what the?}, \trs{for Heaven’s sake!}} \label{p:[hotgǫˀǫh]}
\CayugaRemark{Verb \textit{hotgǫ́ˀǫh} or \textit{otgǫˀ} functions as an exclamation (\sectref{ch:Exclamations}); said when something is out of the ordinary or not right; literally, \textit{otgǫˀ} means `strong or ominous medicine’. }



\begin{CayugaRelated}
\item \textit{Otgǫˀ, hotgǫˀǫh} \trs{what the?}, \trs{for Heaven’s sake!}, p. \pageref{p:[otgǫˀ]}
\end{CayugaRelated}


\subsection*{\textbf{Ho:węh} \trs{it's his}, \trs{his}} \label{p:[ho:węh]}
\CayugaRemark{Verb functioning as a “possessive pronoun” (\sectref{ch:’Possessive pronouns’}); free-standing.}

\ea
\label{ex:hpar57}\textbf{hó:węh}\\
\gll hó:-w-ęh \\
\textsc{3s.m.p}-own-{\stative}\\
\glt ‘he owns it’
\z



\subsection*{\textbf{Hǫ:weh} \trs{where}} \label{p:[hǫ:weh]}
\CayugaRemark{Particle functioning as an “adverb of place” (\sectref{‘Adverbs’ of place}); enclitic; \textit{hǫ:weh} can be spelled or pronounced as \textit{hǫ:}, \textit{nhǫ:}, \textit{hǫ:weh}, or \textit{nhǫ:weh}.}



\begin{CayugaRelated}
\item \textit{Da: hǫ:weh hǫ:} \trs{this is where}, p. \pageref{p:[da: hǫ:weh hǫ:]}\\
\item \textit{Gaę gwaˀ hǫ:weh} \trs{somewhere}, \trs{someplace}, \trs{wherever}, \trs{around}, p. \pageref{p:[gaę gwaˀ hǫ:weh]}\\
\item \textit{Gaę hǫ:weh} \trs{which place}, \trs{where}, p. \pageref{p:[gaę hǫ:weh]}\\
\item \textit{Hę:gyeh gaę hǫ:weh} \trs{no matter where}, \trs{no matter which place}, p. \pageref{p:[hę:-gyeh gaę hǫ:weh]}\\
\item \textit{Hǫ:weh} \trs{where}, p. \pageref{p:[hǫ:weh]}\\
\item \textit{Neˀ to gyę:ˀ hǫ:weh} \trs{it is where}, p. \pageref{p:[neˀ to gyę:ˀ hǫ:weh]}\\
\item \textit{Shęh hǫ:weh} \trs{the place where}, \trs{whereabouts}, p. \pageref{p:[shęh hǫ:weh]}\\
\item \textit{Si: hǫ:weh} \trs{way over there}, p. \pageref{p:[si: hǫ:weh]}\\
\item \textit{To: hǫ:weh} \trs{there}, \trs{where}, p. \pageref{p:[to: hǫ:weh]}\\
\item \textit{Toh-geh hǫ:weh} \trs{where}, p. \pageref{p:[toh-geh hǫ:weh]}
\end{CayugaRelated}


\subsection*{\textbf{Hwaˀ} \trs{this time}, \trs{next}} \label{p:[hwaˀ]}
\CayugaRemark{Particle functioning as an “adverb of time” (\sectref{ch:’Adverbs’ of time}); enclitic.}

\ea
\label{ex:hpar58}
\gll “O:,” aˀa:gę́ˀ, “ohsno:wé:ˀah giˀ gyę:ˀ \textbf{hwaˀ} sa̱hsyǫˀ!”\\
oh she.said quickly just emphasis this.time you.arrived.home\\
\glt “Oh,” she said “you came home kind of early this time!” (\cite{henry_de_2005})
\z

\ea
\label{ex:hpar59}
\gll Junior: Gwé:, dęˀ ní:s \textbf{hwaˀ} hoˀdę́ˀ nęswa:gyé:ˀ nę:gyę́h \textbf{hwaˀ} dęyogęnhǫ:diˀ?\\
{} well, what you this.time kind you.will.do.something this this.time it.will.be.summer\\
\glt ‘Well, what are yous going to do this summer?’

\gll Sam: Ó:. Ęgyagwa̱hdę:díˀ ni:ˀ \textbf{hwaˀ}.\\
{} oh, we’re.going.away we this.time\\
\glt ‘Oh, we’re going away this time.’

\gll Junior: Ó:o:, tę́ˀ \textbf{hwaˀ} taeswayę́twa̱hsǫ: gęh? \\
{} oh, not this.time you.won’t.plant Q\\
\glt ‘Oh, you are not going to plant this time?’

\gll Sam: Tę́ˀ. Ęyagwa̱hdę:díˀ ni:ˀ \textbf{hwaˀ}.\\
{} no, we’re.going.away we this.time\\
\glt ‘No, we’re going away this time.’ (\cite[402]{mithun_watewayestanih_1984},  Ęyagwa̱hdę́:diˀ dialogue)
\z



\ea
\label{ex:hpar60}
\gll Ne:ˀ seˀ gyę:ˀ gę:s to:gyę́h \textbf{hwaˀ} nęh sǫheh nęh gadidaksénǫgyeˀs shęh nhǫ: ohádenyǫˀ.\\
it.is you.know emphasis usually that.one time when night when they.run.around that where there.are.roads\\
\glt ‘This is how it is, you know, at night when they run around their roads.’ (\cite{mithun_how_1980})
\z


\begin{CayugaRelated}
\item \textit{Ahgwih hwaˀ} \trs{don’t!}, p. \pageref{p:[ahgwih hwaˀ]}\\
\item \textit{Gwahs hwaˀ} \trs{this time for sure}, p. \pageref{p:[gwahs hwaˀ]}\\
\item \textit{Gyę:gwaˀ giˀ shęh hwaˀ} \trs{maybe this time}, p. \pageref{p:[gyę:gwaˀ giˀ shęh hwaˀ]}\\
\item \textit{Gyę:gwaˀ hne:ˀ hwaˀ} \trs{just maybe}, p. \pageref{p:[gyę:gwaˀ hne:ˀ hwaˀ]}\\
\item \textit{Ne:ˀ giˀ shęh hwaˀ} \trs{maybe this time}, p. \pageref{p:[ne:ˀ giˀ shęh hwaˀ]}\\
\item \textit{Ne:ˀ hwaˀ} \trs{this (coming) time}, \trs{when}, p. \pageref{p:[ne:ˀ hwaˀ]}\\
\item \textit{Nę: hwaˀ waˀ-ne:ˀ} \trs{nowadays}, p. \pageref{p:[nę: hwaˀ waˀ-ne:ˀ]}\\
\item \textit{Nę:-gyęh hwaˀ} \trs{this time}, p. \pageref{p:[nę:-gyęh hwaˀ] ‘this time’}\\
\item \textit{O: tęˀ ǫh ga:t giˀ shęh ne:ˀ hwaˀ} \trs{maybe, maybe not}, p. \pageref{p:[o: tęˀ ǫh ga:t giˀ shęh ne:ˀ hwaˀ]}\\
\item \textit{Tęˀ giˀ shęh hwaˀ daˀǫ} \trs{maybe not}, p. \pageref{p:[tęˀ giˀ shęh hwaˀ daˀǫ]}\\
\item \textit{To:-hwaˀ, Tǫ:-hwaˀ} \trs{that time}, p. \pageref{p:[tǫ:-hwaˀ]}
\end{CayugaRelated}

\subsection*{\textbf{Hwę:dǫh} \trs{ever}, \trs{when}} \label{p:[hwę:dǫh]}
\CayugaRemark{Particle functioning as an “adverb of time” (\sectref{ch:’Adverbs’ of time}), clause-initial in direct \xxref{ex:hpar61}{ex:hpar62} and indirect \xref{ex:hpar63} questions [ ] (see \sectref{ch:Clauses with [hwę:dǫh], [nęh], [nę:gyęh hwaˀ], [ne:ˀ hwaˀ], etc. ‘when’}).}

\ea
\label{ex:hpar61}
\gll \textbf{Hwę:dǫ́h} ne:ˀ to nęya:węh?\\
when it.is that it.will.happen\\
\glt ‘When is that going to happen?’
\z


\ea
\label{ex:hpar62}
\gll \textbf{Hwę:dǫ́h} diˀ ęje:yǫˀ?\\
when so she.will.return\\
\glt ‘When then is she going to return?’
\z


\ea
\label{ex:hpar63}
\gll Daskro:wíh [\textbf{hwę:dǫ́h} ęsahdę:diˀ].\\
you.tell.me when you.will.leave\\
\glt ‘Tell me when you are going to leave.’
\z


\begin{CayugaRelated}
\item \textit{Ahgwih hwę:dǫh} \trs{don’t ever}, p. \pageref{p:[ahgwih hwę:dǫh]}\\
\item \textit{Hwę:dǫh} \trs{when}, p. \pageref{p:[hwę:dǫh]}\\
\item \textit{Hwę:dǫh gwaˀ} \trs{sometime}, \trs{whenever}, \trs{once}, p. \pageref{p:[hwę:dǫh gwaˀ]}\\
\item \textit{Tęˀ hwę:dǫh} \trs{never}, p. \pageref{p:[tęˀ hwę:dǫh]}
\end{CayugaRelated}

\subsection*{\textbf{Hwę:dǫh gwaˀ} \trs{sometime}, \trs{whenever}, \trs{once}} \label{p:[hwę:dǫh gwaˀ]}
\CayugaRemark{Particle group functioning as an “adverb of time” (\sectref{ch:’Adverbs’ of time}); clause-initial.}

\ea
\label{ex:hpar64}
\gll \textbf{Hwę:dǫ́h} \textbf{gwaˀ} to niyáwę̱ˀǫh.\\
when right.then that something.happened\\
\glt ‘It happened sometime in the past.’
\z


\ea
\label{ex:hpar65}
\gll \textbf{Hwę:dǫ́h} \textbf{gwaˀ} to niyáwę̱hdreˀ.\\
when right.then that something.is.going.to.happen\\
\glt ‘Sometime it is going to happen.’
\z


\ea
\label{ex:hpar66}
\gll \textbf{Hwę:dǫ́h} \textbf{gwaˀ} dętseˀ.\\
when right.then you.will.return\\
\glt ‘Come back sometime!’ (\cite[39]{michelson_ontario_2011})
\z


\ea
\label{ex:hpar67}
\gll \textbf{hwę:dǫ́h} \textbf{gwaˀ} neˀ shę́h nidwa:gé:nǫ:\\
when right.then the that I.will.be.a.certain.number\\
\glt ‘…maybe once I have reached a certain age,…’ (\cite{henry_de_2005})
\z


\begin{CayugaRelated}
\item \textit{Gwaˀ} \trs{immediately}, \trs{right then}, \trs{just then}, \trs{finally}, p. \pageref{p:[gwaˀ] ‘immediately’}\\
\item \textit{Hwę:dǫh} \trs{ever}, \trs{when}, p. \pageref{p:[hwę:dǫh]}
\end{CayugaRelated}

\subsection*{\textbf{Hya:ˀ} \trs{first}, \trs{before anything else}} \label{p:[hya:ˀ]}
\CayugaRemark{Particle functioning as an “adverb of time” (\sectref{ch:’Adverbs’ of time}); enclitic.}

\ea
\label{ex:hpar68}
\gll Ne:ˀ giˀ hǫ:niˀ e:dwade:kǫ́:niˀ \textbf{hya:ˀ}!\\
it.is just the.reason we.should.eat first\\
\glt ‘That’s why I think we should eat something first brothers!’ (\cite{carrier_legends_2013})
\z


\begin{CayugaRelated}
\item \textit{I:ˀ hya:ˀ} \trs{me first}, p. \pageref{p:[i:ˀ hya:ˀ]}\\
\item \textit{Ne:ˀ giˀ gyę:ˀ hya:ˀ} \trs{before all else}, \trs{first}, p. \pageref{p:[ne:ˀ giˀ gyę:ˀ hya:ˀ]}\\
\item \textit{O:nęh giˀ hya:ˀ} \trs{good-bye}, p. \pageref{p:[o:nęh giˀ hya:ˀ]}\\
\item \textit{Waˀ-jih hya:ˀ} \trs{wait a minute}, \trs{wait a while}, \trs{wait!}, p. \pageref{p:[waˀ-jih hya:ˀ]}
\end{CayugaRelated}

\section{I particles}

\subsection*{\textbf{I:ˀ, Ni:ˀ} \trs{I}, \trs{we}} \label{p:[i:ˀ]}
\CayugaRemark{Particle functioning as an “emphatic pronoun” (\sectref{ch:’Emphatic pronouns’}). \textit{Ní:ˀ} is said to be a combination of \textit{neˀ} and \textit{i:ˀ}; free-standing. \textit{I:ˀ} and \textit{ni:ˀ} can occur anywhere that \textit{i:s} and \textit{ni:s} do.}


\begin{CayugaRelated}
\item \textit{Dęˀ ni:ˀ} \trs{I am for sure}, p. \pageref{p:[dęˀ ni:ˀ]}\\
\item \textit{Do: i:ˀ} \trs{let me!}, \trs{how about me?}, p. \pageref{p:[do: i:ˀ]}\\
\item \textit{I:ˀ hne:ˀ} \trs{no, \emph{I} am}, p. \pageref{p:[i:ˀ hne:ˀ]}\\
\item \textit{I:ˀ, Ni:ˀ}, \trs{I}, \trs{we}, p. \pageref{p:[i:ˀ]}\\
\item \textit{I:ˀ hya:ˀ} \trs{me first}, p. \pageref{p:[i:ˀ hya:ˀ]}\\
\item \textit{I:ˀ seˀ} \trs{I am}, p. \pageref{p:[i:ˀ seˀ]}\\
\item \textit{I:ˀgeh (gwa:dih)} \trs{on my side}, \trs{as for me}, p. \pageref{p:[i:ˀgeh gwa:dih]}\\
\item \textit{Nę: diˀ ni:s}, p. \pageref{p:[nę: diˀ ni:s]}\\
\item \textit{Ni:ˀ gyę:ˀ, ni:ˀ gę:ˀ} \trs{I did it}, p. \pageref{p:[ni:ˀ gyę:ˀ]}\\
\item \textit{Seˀ hęˀ ni:ˀ} \trs{me too}, \trs{us too}, p. \pageref{p:[seˀ hęˀ ni:ˀ]}\\
\item \textit{Tęˀ giˀ ni:ˀ} \trs{no, not me}, p. \pageref{p:[tęˀ giˀ ni:ˀ]}\\
\item \textit{Tęˀ i:ˀ} \trs{not me}, \trs{not us}, p. \pageref{p:[tęˀ i:ˀ]}
\end{CayugaRelated}


\subsection*{\textbf{I:ˀ hne:ˀ} \trs{no, \emph{I am}}} \label{p:[i:ˀ hne:ˀ]}
\CayugaRemark{Particle group functioning as an “emphatic pronoun” (\sectref{ch:’Emphatic pronouns’}); free-standing.}


\ea
\label{ex:ipar2}
\gll \textbf{I:ˀ} \textbf{hne:ˀ}.\\
I in.fact\\
\glt ‘No, \emph{I am}.’ (answering the question, “Is she leaving?”)
\z


\subsection*{\textbf{I:ˀ hya:ˀ} \trs{me first}, \trs{us first}} \label{p:[i:ˀ hya:ˀ]}
\CayugaRemark{Particle group functioning as an “emphatic pronoun” (\sectref{ch:’Emphatic pronouns’}); free-standing.}

\ea
\label{ex:ipar3}
\gll \textbf{I:ˀ} \textbf{hya:ˀ}.\\
I first\\
\glt ‘Me first.’
\z



\subsection*{\textbf{I:ˀ seˀ} \trs{\emph{I am}}} \label{p:[i:ˀ seˀ]}
\CayugaRemark{Particle group functioning as an “emphatic pronoun” (\sectref{ch:’Emphatic pronouns’}) with contrastive focus; free-standing.}


\ea
\label{ex:ipar4}
\gll \textbf{I:ˀ} \textbf{seˀ}.\\
I you.know\\
\glt ‘No, \emph{I am}.’ (answering the question, “Is she leaving?”)
\z

\begin{CayugaRelated}
\item \textit{I:ˀ, Ni:ˀ}, \trs{I}, \trs{we}, p. \pageref{p:[i:ˀ]}\\
\item \textit{Seˀ} \trs{you know}, p. \pageref{p:[seˀ]}
\end{CayugaRelated}




\subsection*{\textbf{I:ˀgeh gwa:dih} \trs{on my side}, \trs{as for me}} \label{p:[i:ˀgeh gwa:dih]}
\CayugaRemark{Particle group, \textit{i:ˀgeh (gwa:dih)} functions as an emphatic pronoun (\sectref{ch:’Emphatic pronouns’}); free-standing; \textit{gwa:dih} can also be spelled or pronounced as \textit{gwai}.}

\ea
\label{ex:ipar1}
\gll \textbf{i:ˀ-geh} \textbf{gwa:dih}\\
I-on side\\
\glt ‘on my side’, ‘as for me’
\z


\begin{CayugaRelated}
\item \textit{Gwa:dih, gwai} \trs{to one side}, p. \pageref{p:[gwa:dih]}\\
\item \textit{I:ˀ, Ni:ˀ}, \trs{I}, \trs{we}, p. \pageref{p:[i:ˀ]}
\end{CayugaRelated}


\subsection*{\textbf{I:nǫh} \trs{far}} \label{p:[i:nǫh]}
\CayugaRemark{Atypical verb functioning as an “adverb of place” (\sectref{‘Adverbs’ of place}); clause-initial.}

\ea
\label{ex:ipar5}
\gll \textbf{I:nǫ́h} tgidrǫˀ.\\
it.is.far I.live.there\\
\glt ‘I live far away.’
\z

\begin{CayugaRelated}
\item \textit{I:nǫh gaoˀ neˀ} \trs{far from}, \trs{not all}, p. \pageref{p:[i:nǫh gaoˀ neˀ]}
\end{CayugaRelated}


\subsection*{\textbf{I:nǫh gaoˀ neˀ} \trs{far from}, \trs{not all}} \label{p:[i:nǫh gaoˀ neˀ]}
\CayugaRemark{Atypical verb and particle group \stem{i:nǫh … gaoˀ neˀ stative.verb} is used in comparisons (\sectref{ch:Comparisons (more, the same, or less)}).}

\ea
\label{ex:ipar6}
\gll \textbf{I:nǫ́h} giˀ \textbf{gaǫ} \textbf{neˀ} gwę:gǫh hęyodiyéna̱ha:k\\
it.is.far just so.much the all they.will.be.caught\\
\glt ‘But I wouldn’t very often catch all of them.’ (\cite{mithun_how_1980})
\z

\begin{CayugaRelated}
\item \textit{Gaoˀ} \trs{this side}, \trs{this way}, p. \pageref{p:[gaoˀ]}\\
\item \textit{I:nǫh} \trs{far}, p. \pageref{p:[i:nǫh]}\\
\item \textit{Neˀ} \trs{that is}, \trs{the}, p. \pageref{p:[neˀ]}
\end{CayugaRelated}


\subsection*{\textbf{I:s, Ni:s} \trs{you} (any number of people)} \label{p:[i:s]}
\CayugaRemark{Particle functioning as an “emphatic pronoun” (\sectref{ch:’Emphatic pronouns’}); \textit{ní:s} is said to be a combination of \textit{neˀ} and \textit{i:s}; free-standing. \textit{I:s} and \textit{ni:s} can occur anywhere that \textit{i:ˀ} and \textit{ni:ˀ} do.}

\begin{CayugaRelated}
\item \textit{Dęˀ ni:ˀ} \trs{I am for sure}, p. \pageref{p:[dęˀ ni:ˀ]}\\
\item \textit{Do: i:ˀ} \trs{let me!}, \trs{how about me?}, p. \pageref{p:[do: i:ˀ]}\\
\item \textit{I:ˀ hne:ˀ} \trs{no, \emph{I} am}, p. \pageref{p:[i:ˀ hne:ˀ]}\\
\item \textit{I:ˀ, Ni:ˀ}, \trs{I}, \trs{we}, p. \pageref{p:[i:ˀ]}\\
\item \textit{I:ˀ hya:ˀ} \trs{me first}, p. \pageref{p:[i:ˀ hya:ˀ]}\\
\item \textit{I:ˀ seˀ} \trs{I am}, p. \pageref{p:[i:ˀ seˀ]}\\
\item \textit{I:ˀgeh (gwa:dih)} \trs{on my side}, \trs{as for me}, p. \pageref{p:[i:ˀgeh gwa:dih]}\\
\item \textit{Nę: diˀ ni:s} \trs{how about you?}, p. \pageref{p:[nę: diˀ ni:s]}\\
\item \textit{Ni:ˀ gyę:ˀ, ni:ˀ gę:ˀ} \trs{I did it}, p. \pageref{p:[ni:ˀ gyę:ˀ]}\\
\item \textit{Seˀ hęˀ ni:ˀ} \trs{me too}, \trs{us too}, p. \pageref{p:[seˀ hęˀ ni:ˀ]}\\
\item \textit{Tęˀ giˀ ni:ˀ} \trs{no, not me}, p. \pageref{p:[tęˀ giˀ ni:ˀ]}\\
\item \textit{Tęˀ i:ˀ} \trs{not me}, \trs{not us}, p. \pageref{p:[tęˀ i:ˀ]}
\end{CayugaRelated}


\subsection*{\textbf{I:soˀ} \trs{much}, \trs{many}, \trs{lots}, \trs{very}} \label{p:[i:soˀ]}
\CayugaRemark{Atypical verb functioning as an “adverb of degree” (\sectref{ch:’Adverbs’ of degree}); clause-initial.}

\ea
\label{ex:ipar7}
\gll Dó:gęhs \textbf{í:soˀ} gotsę:nǫ:níˀ Hiˀnóˀ ashagoyénawaˀs.\\
truly lots she.is.happy Hiˀnoˀ he.helped.her\\
\glt ‘She was very grateful that Hiˀnoˀ helped her.’ (\cite{carrier_legends_2013})
\z

\ea
\label{ex:ipar8}
\gll \textbf{I:sóˀ} ga̱ˀdrehda̱ˀsǫ́:ˀǫh hó:yęˀ.\\
many cars he.owns\\
\glt ‘He has many cars.’
\z

\begin{CayugaRelated}
\item \textit{I:so:ˀah} \trs{a fairly big bit}, p. \pageref{p:[i:so:ˀah]}
\end{CayugaRelated}


\subsection*{\textbf{I:so:ˀah} \trs{a fair amount}} \label{p:[i:so:ˀah]}
\CayugaRemark{Atypical verb functioning as an “adverb of degree” (\sectref{ch:’Adverbs’ of degree}); clause-initial.}

\ea
\label{ex:ipar9}
\gll …gwahs \textbf{i:só:ˀah} ęyagwayę́twahsǫ:ˀ…,\\
…really a.fair.amount we.will.plant.things\\
\glt ‘… we’ll plant a lot,…’(\cite[420]{mithun_watewayestanih_1984}, Ęswayętoˀ Gęh? dialogue)
\z

\begin{CayugaRelated}
\item \textit{I:soˀ} \trs{much}, \trs{many}, \trs{lots}, \trs{very}, p. \pageref{p:[i:soˀ]}
\end{CayugaRelated}


\subsection*{\textbf{I:wa:kˀah} \trs{near}} \label{p:[i:wa:kˀah]}
\CayugaRemark{Verb functioning as an “adverb of place” (\sectref{‘Adverbs’ of place}); clause-initial.}

\ea
\label{ex:ipar10}
\gll \textbf{I:wa:kˀáh} dǫdahodrihsdǫ̱hǫ́:gyeˀ.\\
it.is.near he.is.sneaking.around.ever.closer\\
\glt ‘He is getting nearer and nearer again.’
\z


\section{J particles}

\subsection*{\textbf{Ji} \trs{overly}, \trs{too}} \label{p:[ji]}
\CayugaRemark{Element occurring in the following particle groups; proclitic.}

\begin{CayugaRelated}
\item \textit{Ji aǫgo̱hdǫh} \trs{too much so}, p. \pageref{p:[ji aǫgohdǫh]}\\
\item \textit{Ji trehs} \trs{more (than usual)}, \trs{too much so}, p. \pageref{p:[ji trehs] ‘more (than usual)’}\\
\item \textit{Ji trehs} \trs{because}, \trs{overly}, p. \pageref{p:[ji trehs] ‘because’}\\
\item \textit{Ji trehs shęh} \trs{too much so for}, \trs{so much so}, p. \pageref{p:[ji trehs shęh]}
\end{CayugaRelated}


\subsection*{\textbf{Ji aǫgohdǫh} \trs{too much so}} \label{p:[ji aǫgohdǫh]}
\CayugaRemark{Particle and verb \textit{ji … aǫgo̱hdǫh} functions as an “adverb of degree” (\sectref{ch:’Adverbs’ of degree}); clause-initial before an “adjectival” (stative-only) verb.} 

\ea
\label{ex:jpar1}
\gll Ęhę́ˀ ji giˀ \textbf{aǫgohdǫ́h} oyęhsrí:yoˀ, neˀ gwahs ǫ:wéh á:gehs.\\
yes too.much just it.is.surpassed good.blanket, the really truly I.should.use.it\\
\glt ‘Yes, it is too good a quilt to really use.’ (\cite[456]{mithun_watewayestanih_1984}, Oyę́hsraˀ dialogue)
\z

\begin{CayugaRelated}
\item \textit{Aǫgo̱hdǫh} \trs{exceptional}, \trs{over the top}, \trs{extremely}, \trs{too much so}, p. \pageref{p:[aǫgohdǫh]}\\
\item \textit{Ji} \trs{too much}, p. \pageref{p:[ji]}
\end{CayugaRelated}


\subsection*{\textbf{Ji trehs} \trs{more (than usual)}, \trs{too much so}} \label{p:[ji trehs] ‘more (than usual)’}
\CayugaRemark{Particle group \textit{(ji)…trehs} functions as an “adverb of degree” (\sectref{ch:’Adverbs’ of degree}); clause-initial.}

\ea
\label{ex:jpar2}
\gll Dęˀ diˀ hoˀdę́ˀ ni:yóht \textbf{tréhs} hnaˀgę:ˀ? \\
what then thing the.reason too late\\
\glt ‘Why was it so late?’ (\cite[508]{mithun_watewayestanih_1984}, Aǫhdęgyǫ́heˀ dialogue)
\z

\ea
\label{ex:jpar3}
\gll …\textbf{tréhs} a:yę́:ˀ ǫknigǫ̱hsá:dǫˀk. \\
…too.much.so it.seems I.got.lonesome\\
\glt ‘ …I seem to get too lonesome. (\cite[88]{mithun_watewayestanih_1984}, Eksaˀgó:wah dialogue)
\z


\ea
\label{ex:jpar4}
\gll Tęˀ, \textbf{tręhs} do:gęhs a:yę́:ˀ gowá:nęˀs. \\
No, too.much truly it.seems it.is.big.items\\
\glt ‘It really seems to be too big for me.’ (\cite[225]{mithun_watewayestanih_1984}, Agyaˀdawíˀtraˀ dialogue)
\z

\ea
\label{ex:jpar5}
\gll \textbf{Jiˀ} gę:s \textbf{trehs} shenó:wę:. \\
too usually too.much.so you.lie\\
\glt ‘You lie too much.’ `you are too much of a liar.’
\z

\ea
\label{ex:jpar6}
\gll A:yeˀ \textbf{ji} \textbf{tręhs} tsishede̱hjíh agáhyagoˀ \\
it.seems too too.much this.morning I.picked.fruit\\
\glt ‘It seems to me I picked too much fruit this morning.’
\z

\begin{CayugaRelated}
\item \textit{Ji} \trs{too much}, p. \pageref{p:[ji]}\\
\item \textit{Ji trehs} \trs{because}, \trs{overly}, p. \pageref{p:[ji trehs] ‘because’}\\
\item \textit{Ji trehs shęh} \trs{too much so for}, \trs{so much so}, p. \pageref{p:[ji trehs shęh]}\\
\item \textit{Shęh} \trs{that}, \trs{because}, p. \pageref{p:[shęh] `because’}\\
\item \textit{Trehs} \trs{too much}, p. \pageref{p:[trehs, ji trehs]}
\end{CayugaRelated}


\subsection*{\textbf{Ji trehs} \trs{because}, \trs{overly}} \label{p:[ji trehs] ‘because’}
\CayugaRemark{Particle group \textit{(ji) trehs} is clause-initial in causative dependent clauses (\sectref{ch:Causative clauses with [dęˀ ni:yoht shęh] ‘why’, [neˀ hǫ:niˀ] ‘how’}).}

\ea
\label{ex:jpar7}
\gll Ętsgihnǫkséˀ giˀ gyę:ˀ, [\textbf{tréhs} tę́ˀ de̱ˀage̱ˀdréhdaęˀ]. \\
you.will.come.and.seek.me just emphasis because not I.don’t.have.a.car\\
\glt ‘Will you come and get me, because I don’t have a car?’ (\cite[279]{mithun_watewayestanih_1984}, Ǫdwęnǫdáhtaˀ dialogue)
\z

\ea
\label{ex:jpar8}
\gll Agahdrǫ́ˀs onę́h e:  {[\textbf{ji} \textbf{trehs}} satsę:nǫ́:nih]. \\
I’m.frightened now again because you.are.happy\\
\glt ‘I am frightened now because you are too happy again.’
\z

\begin{CayugaRelated}
\item \textit{Ji} \trs{too much}, p. \pageref{p:[ji]}\\
\item \textit{Ji trehs} \trs{more (than usual)}, \trs{too much so}, p. \pageref{p:[ji trehs] ‘more (than usual)’}\\
\item \textit{Ji trehs shęh} \trs{too much so for}, \trs{so much so}, p. \pageref{p:[ji trehs shęh]}\\
\item \textit{Shęh} \trs{that}, \trs{because}, p. \pageref{p:[shęh] `because’}\\
\item \textit{Trehs} \trs{too much}, p. \pageref{p:[trehs, ji trehs]}
\end{CayugaRelated}


\subsection*{\textbf{Ji trehs shęh} \trs{too much so for}, \trs{so much so}} \label{p:[ji trehs shęh]}
\CayugaRemark{Particle group \stem{(ji) … trehs \textsc{stative.verb} shęh \textsc{verb}} or \stem{(ji) … trehs shęh \textsc{stative.verb}} is used in comparisons (\sectref{ch:Comparisons (more, the same, or less)}).}

\ea
\label{ex:jpar9}
\gll \textbf{Ji} gę́:s \textbf{trehs} ja̱hsę́: \textbf{shęh} nisnihnę́:ye:s. \\
too usually too.much you.are.fat that how.tall.you.are\\
\glt ‘You are generally too fat for your height.’
\z

\ea
\label{ex:jpar10}
\gll \textbf{Tréhs} \textbf{shę́h} i:sóˀ agwánę̱hgwiˀ. \\
too.much that it.is.a.lot we.move.it\\
\glt ‘There is such a lot for us to move.’ (\cite{keye_circle_2016}, Circle Book 11, On Our Way)
\z

\begin{CayugaRelated}
\item \textit{Ji} \trs{too much}, p. \pageref{p:[ji]}\\
\item \textit{Ji trehs} \trs{more (than usual)}, \trs{too much so}, p. \pageref{p:[ji trehs] ‘more (than usual)’}\\
\item \textit{Ji trehs} \trs{because}, \trs{overly}, p. \pageref{p:[ji trehs] ‘because’}\\
\item \textit{Shęh} \trs{that}, \trs{because}, p. \pageref{p:[shęh] `because’}\\
\item \textit{Trehs} \trs{too much}, p. \pageref{p:[trehs, ji trehs]}
\end{CayugaRelated}

\section{N particles}

\subsection*{\textbf{Nˀaht, Noht} meaning \trs{some person}} \label{p:[-nˀaht, -noht]}
\CayugaRemark{Particle functioning as a “pronoun”; always appears after \textit{sǫ:} `who’.}

\begin{CayugaRelated}
\item \textit{Sǫ:, sǫ: nˀaht} \trs{who}, p. \pageref{p:[sǫ:, sǫ: nˀaht]}\\
\item \textit{Sǫ: gwaˀ nˀaht} \trs{anybody at all}, \trs{anyone at all}, \trs{any living thing}, \trs{somebody}, \trs{whoever}, p. \pageref{p:sǫ: gwaˀ, sǫ: gwaˀ nˀaht]}\\
\item \textit{Sǫ: gwaˀ nˀaht o:yaˀ} \trs{someone else}, p. \pageref{p:[sǫ: gwaˀ nˀaht o:yaˀ]}
\end{CayugaRelated}


\subsection*{\textbf{Naˀgę:ˀ, Ohnaˀgę:ˀ} \trs{late}} \label{p:[naˀgę:ˀ]}
\CayugaRemark{Verb \textit{(oh)naˀgę:ˀ} functions as an “adverb of time” (\sectref{ch:’Adverbs’ of time}); clause-initial.}

\ea
\label{ex:npar1}
\gll \textbf{Ohnaˀgę́:ˀ} ihseˀ\\
late you.are.here\\
\glt ‘You are late again.’ (said at the moment)
\z


\begin{CayugaRelated}
\item \textit{Ohnaˀgę: gwa:dih shęh} \trs{behind}, p. \pageref{p:[ohnaˀgę: gwa:dih shęh]}\\
\item \textit{Ohnaˀgęhjih} \trs{late}, \trs{back then}, p. \pageref{p:[ohnaˀgęhjih] ‘late’}\\
\item \textit{Ohnaˀgęhjih} \trs{late}, \trs{back then}, \trs{the bottom}, p. \pageref{p:[ohnaˀgęhjih] ‘late’}\\
\item \textit{Ohnaˀgǫ:} \trs{underneath}, \trs{beneath}, \trs{under}, p. \pageref{p:[ohnaˀgǫ:]}
\end{CayugaRelated}

\subsection*{\textbf{Neˀ} \trs{that is}, \trs{the}} \label{p:[neˀ]}
\CayugaRemark{Particle signaling that the item it modifies constitutes shared knowledge (known to both speaker and listener, \sectref{ch:Shared knowledge markers}); \textit{neˀ} also identifies a generic category of event or state (\cite[155]{chafe_discourse_1994}). Enclitic to the phrase it modifies (shown in bold).}

\ea
\label{ex:npar2}
\gll Dęˀ hoˀdę́ˀ eya:sǫ́h \textbf{neˀ} \textbf{sanǫ́:haˀ}?\\
what kind she.is.called the your.mother\\
\glt ‘What is the name of your mother?’ (asking about someone who is the current topic of conversation)
\z

\ea
\label{ex:npar3}
\gll Honahsę́: hęˀ \textbf{neˀ} \textbf{dehęnadęhnǫ́:drǫˀ}.\\
they.are.fat also the your.brothers\\
\glt ‘Your brothers are also fat.’ (talking about people known to both speaker and listener)
\z

\ea
\label{ex:npar4}
\gll Elsie eya:sǫ́h \textbf{neˀ} \textbf{gonǫhgę́:t}, Esther hné:ˀ eya:sǫ́h \textbf{neˀ} \textbf{gogéˀaji:}.\\
Elsie she.is.named the she.is.fair-haired, Esther in.fact she.is.named the she.is.dark-haired\\
\glt ‘The fair-haired one is named Elsie, and the dark-haired one is Esther.’ (\cite[88]{mithun_watewayestanih_1984}, Eksaˀgó:wah dialogue) (speaking of two women whom both speaker and listener can see)
\z

\ea
\label{ex:npar5}
\gll Go:gáˀs gęh nawę́ˀdaˀ \textbf{neˀ} \textbf{Tina}?\\
she.likes.the.taste.of.it Q candy the Tina\\
\glt ‘Does Tina like candy?’ (identifying people known by their proper names to both speaker and listener)
\z

\ea
\label{ex:npar6}
\gll Hona:dáoˀ hǫwanǫ̱hwéˀs hęˀ \textbf{neˀ} \textbf{Gwí:deh}.\\
his.friend he.likes.him also the Peter\\
\glt ‘His friend likes Peter also.’
\z

\ea identifying a generic state of ownership\\
\label{ex:npar7}
\gll
Tęˀ gęh \textbf{neˀ} \textbf{desá:węh}?\\
not Q the you.don’t.own.it\\
\glt ‘Isn’t that yours?’ 
\z

\ea identifying a category of event, such as ‘things I remember’ or ‘her breadmaking’\\
\label{ex:npar8}
\gll \textbf{Neˀ} giˀ \textbf{agása:ˀs} \textbf{neˀ} dyotgǫ́:t gę:s gonaˀdaę́ˀ \textbf{neˀ} ksotgę̱hę́:ˀ \textbf{neˀ} \textbf{gotnaˀdaǫ́daˀk}.\\
the just I.remember the always usually she.made.bread the late.grandmother the she.made.bread.\\
\glt ‘What I remember is that she always had bread, that grandma; she made bread. (i.e. ‘her breadmaking’) (\cite{henry_de_2005}) 
\z

\ea
\label{ex:npar9} identifying a category of event\\
\gll Ahę́ˀ \textbf{neˀ} Hahnę:yé:s, “Sgęnǫˀ:ǫ́h hędwá:trihs. Ędwadenyę:dę́ˀ \textbf{neˀ} \textbf{tawęnihná:do:k}\\
he.said the tall.one slowly we.all.will.approach we.all.will.try the it.shouldn’t.notice\\
\glt ‘The Tall One said, “Go slowly, let us get near. We will try for it not to notice.”’ (referring to the magical bear Hnyagwaiˀdatgiˀgó:wah) (\cite{keye_hnyagwaidatgigowah_2012})
\z





\begin{CayugaRelated}
\item \textit{Neˀ gyaǫhę:ˀęh d-} \trs{the most}, \trs{the greatest}, p. \pageref{p:[neˀ gyaǫhę:ˀęh d-]}\\
\item \textit{Da: neˀ toh} \trs{that’s all}, p. \pageref{p:[da: neˀ toh]}\\
\item \textit{E:ˀ, Neˀ ę:ˀ} \trs{isn’t it so?}, \trs{yes?}, \trs{no?}, \trs{innit?}, p. \pageref{p:[ę:ˀ], [neˀ ę:ˀ] `isn’t it so?’}\\
\item \textit{Gyę:ˀ hne:ˀ tęˀ neˀ} \trs{it was this one, (not that one)}, p. \pageref{p:[gyę:ˀ hne:ˀ]}\\
\item \textit{I:nǫh gaoˀ neˀ} \trs{far from enough}, p. \pageref{p:[i:nǫh gaoˀ neˀ]}\\
\item \textit{Neˀ} \trs{that is}, \trs{the}, p. \pageref{p:[neˀ]}\\
\item \textit{Neˀ aǫhę:ˀęh d-} \trs{the most, -est, -er (of)}, p. \pageref{p:[neˀ aǫhę:ˀęh d-]}\\
\item \textit{Neˀ gwaˀ toh} \trs{here (rather than there)}, p. \pageref{p:[neˀ gwaˀ toh]}\\
\item \textit{Neˀ gyę:ˀ neˀ gyę:ˀ, negęˀnagęˀ} \trs{that is what}, p. \pageref{p:[neˀ gyę:ˀ neˀ gyę:ˀ]}\\
\item \textit{Neˀ gyę:ˀ nę ne:ˀ} (emphasis), p. \pageref{p:[neˀ gyę:ˀ nę ne:ˀ]}\\
\item \textit{Neˀ hegwe:gǫh} \trs{the whole thing}, p. \pageref{p:[neˀ hegwe:gǫh]}\\
\item \textit{Neˀ heyohe:ˀ} \trs{more, -er}, p. \pageref{p:[neˀ heyohe:ˀ]}\\
\item \textit{Neˀ seˀ} \trs{that’s just the one}, \trs{that’s just who}, p. \pageref{p:[neˀ seˀ]}\\
\item \textit{Neˀ to gyę:ˀ} \trs{that’s what}, p. \pageref{p:[neˀ to gyę:ˀ]}\\
\item \textit{Neˀ to gyę:ˀ hǫ:weh} \trs{it is where}, p. \pageref{p:[neˀ to gyę:ˀ hǫ:weh]}\\
\item \textit{Neˀ to: ne:ˀ} \trs{that’s the one}, p. \pageref{p:[neˀ to: ne:ˀ]}\\
\item \textit{Neˀ toh} \trs{that is}, \trs{that one}, p. \pageref{p:[neˀ toh]}\\
\item \textit{Neˀ tsǫ: gwaˀ toh} \trs{also}, p. \pageref{p:[neˀ tsǫ: gwaˀ toh]}\\
\item \textit{Ne:ˀ gęh … ne:ˀ/neˀ nigęˀǫh neˀ …} \trs{or?}, p. \pageref{p:[ne:ˀ gęh … ne:ˀ]}\\
\item \textit{Ne:ˀ/neˀ giˀ gyę:ˀ} \trs{that’s just it}, \trs{that’s it for sure}, p. \pageref{p:[ne:ˀ/neˀ giˀ gyę:ˀ]}\\
\item \textit{Ne:ˀ giˀ shęh neˀ … giˀ shęh} \trs{or}, p. \pageref{p:[ne:ˀ giˀ shęh neˀ … giˀ shęh]}\\
\item \textit{Ne:ˀ gyę:ˀ neˀ} \trs{that really is}, p. \pageref{p:[ne:ˀ gyę:ˀ neˀ]}\\
\item \textit{Ne:ˀ neˀ} \trs{it is}, \trs{that is}, \trs{that’s what}, p. \pageref{p:[ne:ˀ neˀ]}\\
\item \textit{Ne:ˀ seˀ gę:s neˀ} \trs{back then}, \trs{a long time ago}, p. \pageref{p:[ne:ˀ seˀ gę:s neˀ]}\\
\item \textit{O:nęh, neˀ o:nęh} \trs{now}, \trs{when}, \trs{then}, \trs{at this time}, p. \pageref{p:[o:nęh]}\\
\item \textit{Otgaˀdeˀ neˀ niwa:gęˀ ni-} \trs{as many as}, p. \pageref{p:[otgaˀdeˀ neˀ niwa:gęˀ ni-]}\\
\item \textit{O:weh, neˀ ǫ:weh} \trs{really}, p. \pageref{p:[ǫ:weh, neˀ ǫ:weh]}\\
\item \textit{Tęˀ gyę:ˀ nę neˀ} \trs{\emph{not}}, \trs{what on earth?}, p. \pageref{p:[tęˀ gyę:ˀ nę neˀ]}\\
\item \textit{Tęˀ hne:ˀ neˀ} \trs{…not \emph{that one}}, p. \pageref{p:[tęˀ hne:ˀ neˀ]}
\end{CayugaRelated}

\subsection*{\textbf{Neˀ aǫhę:ˀęh d-stative.verb … (shęh gaoˀ ni-stative.verb)} \trs{the most, -est, -er (of)}} \label{p:[neˀ aǫhę:ˀęh d-]}
\CayugaRemark{Phrase \stem{(neˀ) aǫhę:ˀęh d-\textsc{stative.verb} (ni-d-\textsc{stative.verb})} is used in comparisons (\sectref{ch:Comparisons (more, the same, or less)}); \textit{(neˀ) aǫhę:ˀęh} precedes a stative verb beginning with \stem{d-} {\cislocative}.}

\ea
\label{ex:npar11}
\gll \textbf{neˀ} \textbf{aǫhę:ˀę́h} \textbf{d}wakyęda̱hgówanęh\\
the the.most very.big.chair\\
\glt ‘the greatest chair’
\cfex{a-ǫhę:ˀ-ęh \textsc{3s.p}-alone.{\stative}-{\diminutive}}
\cfex{d-w-akyęda̱hg-ówan-ęh \textsc{\cislocative-3s.a}-chair-big-{\stative}}
\z

\ea
\label{ex:npar12}
\gll \textbf{Aǫhę:ˀę́h} hniˀ \textbf{g}yogaˀǫ́h \textbf{shęh} \textbf{gaóˀ} \textbf{ni}tode̱ˀnyę́:dę:.\\
the.most and more.than.sweet that more.so more.than.he.had.tasted\\
\glt ‘And it was sweeter than anything he had ever tasted.’ (\cite{carrier_legends_2013})
\cfex{g-yo-gaˀ-ǫ́h \textsc{\cislocative-3s.p}-taste.good-{\stative}}
\cfex{ni-t-ho-d-e̱-ˀnyę́:dę-: \textsc{\partitive-\cislocative-\textsc{3s.m.p}-{\semireflexive}}-{\joinerE}-try-{\stative}}
\z

\ea
\label{ex:npar13}
\gll Í:ˀ \textbf{aǫhę:ˀę́h} \textbf{t}knę:yé:s \textbf{shęh} \textbf{nig}yǫ:.\\
I the.most I’m.more.than.tall that a.certain.amount\\
\glt ‘I am the tallest one here.’ (\cite{carrier_legends_2013})
\cfex{t-k-hnę:yé:s \textsc{\cislocative-1s.a}-tall.{\stative}}
\cfex{ni-g-y-ǫ: \textsc{\partitive-\cislocative-3s.p}-a.\-certain.\-amount.{\stative}}
\z

\begin{CayugaRelated}
\item \textit{Neˀ gyaǫhę:ˀęh d-} \trs{the most}, \trs{the greatest}, p. \pageref{p:[neˀ gyaǫhę:ˀęh d-]}\\
\item \textit{Aǫhęˀ, Aǫhaˀ} \trs{it}, p. \pageref{p:[aǫhęˀ]}
\end{CayugaRelated}


\subsection*{\textbf{Neˀ gwaˀ toh} \trs{here (rather than there)}} \label{p:[neˀ gwaˀ toh]}
\CayugaRemark{Particle group \textit{(neˀ) gwa̱ˀ-toh} functions as an “adverb of place” (\sectref{‘Adverbs’ of place}); emphatic; clause-initial.}

\ea
\label{ex:npar26}
\gll \textbf{Neˀ} \textbf{gwaˀ} \textbf{toh} haˀhǫwadíha:ˀ.\\
the just.there here they.took.him.over.there\\
\glt ‘They also took him here.’
\z

\ea
\label{ex:npar27}
\gll Sgę́:nǫˀ. Se̱ˀsgęhę́:ˀ ę:ˀ \textbf{gwaˀ-toh} waˀjih?\\
hello you.were.again again just.there-here a.while.ago\\
\glt ‘Hello. Weren’t you here a while ago?’ (\cite[441]{mithun_watewayestanih_1984}, Dwęnǫhsanekahǫˀ dialogue)
\z

\begin{CayugaRelated}
\item \textit{Gwaˀ} \trs{immediately}, \trs{right then}, \trs{just then}, \trs{finally}, p. \pageref{p:[gwaˀ] ‘immediately’}\\
\item \textit{Neˀ} \trs{that is}, \trs{the}, p. \pageref{p:[neˀ]}\\
\item \textit{To:, Toh, Tǫ:} \trs{there}, \trs{that place}, p. \pageref{p:[to:] ‘there’, ‘that place’}
\end{CayugaRelated}

\subsection*{\textbf{Neˀ gyaǫhę:ˀęh d-stative.verb … (shęh ni-stative.verb)} \trs{the most}, \trs{the greatest}} \label{p:[neˀ gyaǫhę:ˀęh d-]}
\CayugaRemark{Particle group \stem{(neˀ) g-y-aǫhę:-ˀęh d-\textsc{stative.verb} (shęh) ni-\textsc{verb})} is used in comparisons (\sectref{ch:Comparisons (more, the same, or less)}); \textit{(neˀ) gyaǫhę:ˀęh} precedes a verb beginning with \stem{d-/g-} {\cislocative}.}

\ea
\label{ex:npar1110}
\gll \textbf{neˀ} \textbf{gyaǫhę:ˀę́h} \textbf{g}yonǫˀshéhsde:ˀ\\
the the.most very.sickening\\
\glt ‘the most sickening, tiring, aggravating’
\cfex{g-ya-ǫhę:ˀ-ęh \textsc{\cislocative-3s.p}-alone.{\stative}-{\diminutive}}
\cfex{g-yo-nǫˀshéhsde:ˀ \textsc{\cislocative-3s.p}-sickening.{\stative}}
\z

\ea
\label{ex:npar1112}
\gll \textbf{neˀ} \textbf{gyaǫhę:ˀę́h} \textbf{d}wakyęda̱hgówanęh\\
the the.most very.big.chair\\
\glt ‘the greatest chair’
\cfex{g-ya-ǫhę:ˀ-ęh \textsc{\cislocative-3s.p}-alone.{\stative}-{\diminutive}}
\cfex{d-w-akyęda̱hg-ówan-ęh \textsc{\cislocative-3s.a}-chair-big-{\stative}}
\z

\ea
\label{ex:npar1130}
\gll O:nę́h ahǫwawę:ná:ˀ \textbf{neˀ} \textbf{gyaǫhę:ˀę́h} \textbf{t}hahnę:yé:s \textbf{shęh} \textbf{ni̱}hę́:nǫ:.\\
then he.had.words.with.him the most he.was.so.tall that they.were.a.certain.number\\
\glt ‘I am the tallest one here.’ (\cite{carrier_legends_2013})
\cfex{thahnę:yé:s \textsc{\cislocative-3s.m.a}-tall.{\stative}}
\cfex{ni̱-hę́:-nǫ: \textsc{\partitive-3ns.m.a}-a.\-certain.\-amount.{\stative}}
\z

\begin{CayugaRelated}
\item \textit{Aǫhęˀ, Aǫhaˀ} \trs{it}, p. \pageref{p:[aǫhęˀ]}\\
\item \textit{Neˀ aǫhę:ˀęh d-} \trs{the most}, \trs{the greatest}, p. \pageref{p:[neˀ aǫhę:ˀęh d-]}
\end{CayugaRelated}


\subsection*{\textbf{Neˀ gyę:ˀ neˀ gyę:ˀ, Negęˀnagęˀ} \trs{that's what}} \label{p:[neˀ gyę:ˀ neˀ gyę:ˀ]}
\CayugaRemark{Particle group \textit{neˀ gyę:ˀ neˀ gyę:ˀ} or \textit{negęˀnagęˀ} conveys emphasis or focus (\sectref{ch:Focus, contrastive focus, and emphasis markers}); clause-initial.}

\ea
\label{ex:npar28}
\gll \textbf{neˀ} \textbf{gyę:ˀ} \textbf{neˀ} \textbf{gyę:ˀ} {(possible analysis)}\\
the this.one the this.one {}\\
\glt ‘that’s what…’
\z

\ea
\label{ex:npar29}
\gll \textbf{Negęˀnageˀ} gi:dǫh.\\
that’s.what I.mean\\
\glt ‘That’s what I meant to say.’
\z

\begin{CayugaRelated}
\item \textit{Gyę:ˀ} \trs{just the one}, p. \pageref{p:[gyę:ˀ]}\\
\item \textit{Neˀ} \trs{that is}, \trs{the}, p. \pageref{p:[neˀ]}
\end{CayugaRelated}


\subsection*{\textbf{Neˀ gyę:ˀ nę ne:ˀ} (emphasis)} \label{p:[neˀ gyę:ˀ nę ne:ˀ]}
\CayugaRemark{Particle group \textit{(neˀ) gyę:ˀ nę ne:ˀ} conveys emphasis (\sectref{ch:Focus, contrastive focus, and emphasis markers}); clause-initial.}

\ea
\label{ex:npar30}
\gll “Ne:ˀ,” aˀa:gę́ˀ, “\textbf{neˀ} \textbf{gyę:ˀ} \textbf{nę} \textbf{ne:ˀ} dęyǫ́kiˀnya:ˀ.”\\
it.is she.said the emphasis here it.is they.govern.us\\
\glt ‘“That’s it,” she said, “that’s because they govern us.”’ (\cite{henry_de_2005})
\z

\ea
\label{ex:npar31}
\gll \textbf{Neˀ} \textbf{gyę:ˀ} \textbf{nę} \textbf{ne:ˀ} toh hǫ́: hęnatgęníhsˀanhaˀ, ahatgęnihsˀanháˀ neˀ hodiyanéhsǫˀ\\
the emphasis here it.is that.one place they.will.meet he.holds.council the they.call.it\\
\glt ‘He went to where they hold council. it is called Ahatgęnihsˀanhaˀ (`he holds council’).’ (\cite{henry_de_2005})
\z

\begin{CayugaRelated}
\item \textit{Gyę:ˀ} \trs{just the one}, p. \pageref{p:[gyę:ˀ]}\\
\item \textit{Gyę:ˀ nę ne:ˀ, Neˀ gyę:ˀ nę ne:ˀ} (emphasis), p. \pageref{p:[gyę:ˀ nę ne:ˀ]}\\
\item \textit{Neˀ} \trs{that is}, \trs{the}, p. \pageref{p:[neˀ]}\\
\item \textit{Ne:ˀ} \trs{it is}, p. \pageref{p:[ne:ˀ] `it is’}\\
\item \textit{Nę:} \trs{this}, \trs{these}, p. \pageref{p:[nę:] ‘this, these’}
\end{CayugaRelated}


\subsection*{\textbf{Neˀ hegwe:gǫh} \trs{the whole thing}} \label{p:[neˀ hegwe:gǫh]}
\CayugaRemark{Particle and atypical verb \textit{(neˀ) hegwe:gǫh} functioning as an “indefinite pronoun” (\sectref{ch:’Indefinite pronouns’}); free-standing.}

\ea
\label{ex:hpar10}
\gll Dó: diˀ niga:nǫ́:ˀ \textbf{neˀ} \textbf{hegwé:gǫh}?\\
how then it.costs.a.certain.amount the all.of.it\\
\glt ‘How much is the whole thing?’ (\cite[159]{mithun_watewayestanih_1984}, Enǫhsǫnyaˀdaˀsǫ́:ˀǫh dialogue)
\z

\begin{CayugaRelated}
\item \textit{Gwe:gǫh, Agwe:gǫh, Ogwe:gǫh} \trs{all}, \trs{everything}, p. \pageref{p:[gwe:gǫh] ‘all’}\\
\item \textit{Gwe:gǫh, Agwe:gǫh, Ogwe:gǫh} \trs{completely}, \trs{totally}, p. \pageref{p:[gwe:gǫh] ‘totally’}
\end{CayugaRelated}


\subsection*{\textbf{Neˀ heyohe:ˀ} \trs{more, -er}} \label{p:[neˀ heyohe:ˀ]}
\CayugaRemark{Particle group \stem{(neˀ) heyohe:ˀ \textsc{stative.verb}} or \stem{(neˀ) \textsc{stative.verb} heyohe:ˀ} is used in comparisons (\sectref{ch:Comparisons (more, the same, or less)}). \textit{heyohe:ˀ} literally means \trs{it is excessive}, \trs{it is overboard}.}

\ea
\label{ex:npar17}
\glll \textbf{heyohé:ˀ} wakyęda̱hgówanęh\\
it.is.overboard w-akyęda̱hg-ówan-ęh\\
{} \textsc{3s.a}-chair-big-{\stative}\\
\glt ‘a bigger chair’
\z

\ea
\label{ex:npar18}
\glll Ęhsyáni̱hsak seˀ-gye:ˀ waˀhéh o:nęh ne:ˀ hǫ:niˀ \textbf{heyohé:ˀ} wagyésa̱ˀgeh.\\
you.will.look.for.tracks you.know-emphasis before now this-is because more w-ag-yésa̱ˀgeh\\
{} {} {} {} {} {} {} \textsc{3s.a}-{\semireflexive}-easy.{\stative}\\
\glt ‘You look for the tracks first, because it is easier.’ (\cite{mithun_how_1980})
\z

\ea
\label{ex:npar19}
\glll Ǫgwa:yę́ˀ giˀ hné:ˀ \textbf{neˀ} wagye:sę́h \textbf{heyóhe:ˀ}, wagyesę̱hsǫ́:ˀǫh.\\
we.have just in.fact the w-ag-ye:s-ę́h more w-ag-yes-ę̱h-sǫ́:ˀǫh\\
{} {} {} {} \textsc{3s.a}-{\semireflexive}-cheap-{\stative} {} \textsc{3s.a}-{\semireflexive}-cheap-stative-pluralizer\\
\glt ‘But we do have cheaper ones, the cheaper kind.’ (\cite[159]{mithun_watewayestanih_1984}, Enǫhsǫnyaˀdaˀsǫ́:ˀǫh dialogue)
\z

\ea
\label{ex:npar20}
\gll Ęhatgę̱ˀsé:ˀ gę́:s neˀ joni:tsgrǫ́:t \textbf{heyohé:ˀ} i:sóˀ agahnegeháˀ aǫgágriˀ, haǫháˀ giˀ hęˀ hne:ˀ toh na̱há:ye:ˀ.\\
he.went.looking usually the squirrel more it.is.a.lot he.drank.sap, it.was.good.tasting.sap he.alone just also in.fact that.one he.did.it\\
\glt ‘Just like the squirrel, he kept going back for more.’ (\cite{carrier_legends_2013})
\z

\begin{CayugaRelated}
\item \textit{Heyohe:ˀ} \trs{more}, p. \pageref{p:[heyohe:ˀ]}\\
\item \textit{Heyohe:ˀ shęh} \trs{more than}, p. \pageref{p:[heyohe:ˀ shęh]}\\
\item \textit{Ne:ˀ gwahs heyohe:ˀ} \trs{the most, -est}, p. \pageref{p:[ne:ˀ gwahs heyohe:ˀ]}
\end{CayugaRelated}

\subsection*{\textbf{Neˀ ǫ:weh, Ǫ:weh} \trs{really}} \label{p:[neˀ ǫ:weh]}
\CayugaRemark{Particle group functioning as an “adverb of degree” (\sectref{ch:’Adverbs’ of degree}); clause-initial. }

\ea
\label{ex:npar21}
\gll \textbf{Neˀ} \textbf{ǫ:weh} gaǫdé:nǫhk.\\
the truly they.are.related\\
\glt ‘They really are related.’
\z

\begin{CayugaRelated}
\item \textit{O:weh, Neˀ ǫ:weh} \trs{really}, p. \pageref{p:[ǫ:weh, neˀ ǫ:weh]}
\end{CayugaRelated}

\subsection*{\textbf{Neˀ seˀ} \trs{that’s just the one}, \trs{that’s just who}} \label{p:[neˀ seˀ]}
\CayugaRemark{Particle group used for emphasis or focus (\sectref{ch:Focus, contrastive focus, and emphasis markers}); free-standing or clause-initial.}

\ea
\label{ex:npar32}
\gll \textbf{Neˀ} \textbf{seˀ} hǫwę́:dǫh.\\
the you.know she.means.him\\
\glt ‘He is just the one she means.’
\z

\begin{CayugaRelated}
\item \textit{Neˀ} \trs{that is}, \trs{the}, p. \pageref{p:[neˀ]}\\
\item \textit{Seˀ} \trs{you know}, p. \pageref{p:[seˀ]}
\end{CayugaRelated}


\subsection*{\textbf{Neˀ to gyę:ˀ} \trs{that’s what}} \label{p:[neˀ to gyę:ˀ]}
\CayugaRemark{Particle group, used for emphasis or focus (\sectref{ch:Focus, contrastive focus, and emphasis markers}); clause-initial.}

\ea
\label{ex:npar33}
\gll \textbf{Neˀ} \textbf{to} \textbf{gyę:ˀ} haˀhoya̱ˀdá:węh.\\
the that emphasis it.happened.to.his.body.over.there\\
\glt ‘That’s what happened to him over there.’
\z

\begin{CayugaRelated}
\item \textit{Gyę:ˀ} \trs{just the one}, p. \pageref{p:[gyę:ˀ]}\\
\item \textit{Neˀ} \trs{that is}, \trs{the}, p. \pageref{p:[neˀ]}\\
\item \textit{To:, Toh, Tǫ:} \trs{that one}, p. \pageref{p:[to:] ‘that one’}
\end{CayugaRelated}

\subsection*{\textbf{Neˀ to gyę:ˀ hǫ:weh} \trs{it’s where}} \label{p:[neˀ to gyę:ˀ hǫ:weh]}
\CayugaRemark{Particle group functioning as an “adverb of place” (\sectref{‘Adverbs’ of place}); emphatic; likely clause-initial; \textit{hǫ:weh} can be spelled or pronounced as \textit{hǫ:}, \textit{nhǫ:}, \textit{hǫ:weh}, or \textit{nhǫ:weh}.}

\begin{CayugaRelated}
\item \textit{Gyę:ˀ} \trs{just the one}, p. \pageref{p:[gyę:ˀ]}\\
\item \textit{Hǫ:weh} \trs{where}, p. \pageref{p:[hǫ:weh]}\\
\item \textit{Neˀ} \trs{that is}, \trs{the}, p. \pageref{p:[neˀ]}\\
\item \textit{To:, Toh, Tǫ:} \trs{there}, \trs{that place}, p. \pageref{p:[to:] ‘there’, ‘that place’}
\end{CayugaRelated}

\subsection*{\textbf{Neˀ to: ne:ˀ} \trs{that’s the one}} \label{p:[neˀ to: ne:ˀ]}
\CayugaRemark{Particle group \textit{(neˀ) to: … ne:ˀ} functions as a “demonstrative pronoun” (\sectref{ch:’Demonstrative pronouns’}); emphatic; clause-initial.}

\ea
\label{ex:npar23}
\gll \textbf{Neˀ} \textbf{to:} \textbf{ne:ˀ} shagóhsgane:s.\\
the that.one it.is she.desires.him\\
\glt ‘She is the one he desires.’
\z

\ea
\label{ex:npar24}
\gll Ihsé: gęh [\textbf{to:} \textbf{ne:ˀ} dwá:ye:]?\\
you.think Q [that.one it.is we.do.it]\\
\glt ‘Do you think we should do it that way?’
\z

\ea
\label{ex:npar25}
\gll [\textbf{To:} ni:yóˀ \textbf{ne:ˀ} heseˀs].\\
that.one where it.is you.are.there\\
\glt ‘That’s your birthday.’ (literally, `that’s when you come around again.’)
\z

\begin{CayugaRelated}
\item \textit{Neˀ} \trs{that is}, \trs{the}, p. \pageref{p:[neˀ]}\\
\item \textit{Ne:ˀ} \trs{it is}, p. \pageref{p:[ne:ˀ] `it is’}\\
\item \textit{To:, Toh, Tǫ:} \trs{that one}, p. \pageref{p:[to:] ‘that one’}
\end{CayugaRelated}

\subsection*{\textbf{Neˀ toh} \trs{that is}, \trs{that one}} \label{p:[neˀ toh]}
\CayugaRemark{Particle group functioning as a “demonstrative pronoun” (\sectref{ch:’Demonstrative pronouns’}); emphatic; clause-initial.}

\ea
\label{ex:npar22}
\gll \textbf{Neˀ} \textbf{toh} ni:yǫ́: hó:yęˀ.\\
it that.one a.certain.number he.owns.it\\
\glt ‘That’s how many he has.’
\z

\begin{CayugaRelated}
\item \textit{Neˀ} \trs{that is}, \trs{the}, p. \pageref{p:[neˀ]}\\
\item \textit{To:, Toh, Tǫ:} \trs{that one}, p. \pageref{p:[to:] ‘that one’}
\end{CayugaRelated}

\subsection*{\textbf{Neˀ tsǫ: gwaˀ toh} \trs{also}} \label{p:[neˀ tsǫ: gwaˀ toh]}
\CayugaRemark{Particle group, \textit{neˀ (tsǫ:) gwaˀ toh} signals the continuation of a previous topic and the addition of new information (\sectref{ch:Topic continuation markers}); clause-initial.}

\ea
\label{ex:npar34}
\gll [\textbf{Neˀ} \textbf{gwaˀ} \textbf{toh} haˀhǫwadíha:ˀ.]\\
the emphasis that.one they.took.him.away\\
\glt ‘They also took him.’
\z

\ea
\label{ex:npar35}
\gll Hohsę́: [\textbf{neˀ} \textbf{tsǫ́:} \textbf{gwaˀ} \textbf{toh} nihahnę́:ye:s]\\
he.is.fat the only emphasis that.one he.is.short\\
\glt ‘He is stout and he is also short.’
\z

\begin{CayugaRelated}
\item \textit{Gwaˀ} \trs{immediately}, \trs{right then}, \trs{just then}, \trs{finally}, p. \pageref{p:[gwaˀ] ‘immediately’}\\
\item \textit{Neˀ} \trs{that is}, \trs{the}, p. \pageref{p:[neˀ]}\\
\item \textit{To:, Toh, Tǫ:} \trs{that one}, p. \pageref{p:[to:] ‘that one’}\\
\item \textit{Tsǫ:} \trs{just}, \trs{only}, p. \pageref{p:[tsǫ:]}
\end{CayugaRelated}

\subsection*{\textbf{Ne:ˀ} \trs{it is}} \label{p:[ne:ˀ] `it is’}
\CayugaRemark{Particle or atypical verb, conveys contrastive focus (\sectref{ch:Focus, contrastive focus, and emphasis markers}, \sectref{ch:Clauses with [shęh] ‘that’, [shęh ni-] ‘how, what’, [ne:ˀ] ‘it is’}); clause-initial in a dependent clause. Other uses are listed in the “Related” sections.}

\ea
\label{ex:npar36}
\gll Hwę:dǫ́h [\textbf{né:ˀ} to nęyá:węh]?\\
when [it.is that it.will.happen]\\
\glt ‘When [is it that that’s going to happen]?’
\z

\ea
\label{ex:npar37}
\gll Tigaˀdre̱hda:déˀ [\textbf{né:ˀ} ahahní:nǫˀ].\\
a.different.kind.of.car [it.is he.bought.it]\\
\glt ‘He bought a different car.’ `A different kind of car [is what he bought].’
\z

\begin{CayugaRelated}
\item \textit{Hne:ˀ, Ne:ˀ … (tsǫ: shęh)} \trs{but}, p. \pageref{ch:[hne:ˀ], [ne:ˀ … (tsǫ: shęh)] ‘but’}\\
\item \textit{Da: ne:ˀ onęh} \trs{and now}, p. \pageref{p:[da: ne:ˀ onęh]}\\
\item \textit{Da: ne:ˀ, da: ne:ˀ hniˀ} \trs{too}, \trs{also}, \trs{and}, p. \pageref{p:[da: ne:ˀ, da: ne:ˀ hniˀ]}\\
\item \textit{Dęˀ ǫh ne:ˀ hoˀdęˀ} \trs{what on earth?}, p. \pageref{p:[dęˀ ǫh hne:ˀ hoˀdęˀ]}\\
\item \textit{Ę:, E:} (possible atypical verb), p. \pageref{p:[ę:, e:] `atypical verb’}\\
\item \textit{Giˀ ne:ˀ} \trs{it is just}, p. \pageref{p:[giˀ ne:ˀ]}\\
\item \textit{Ne:ˀ} \trs{just so}, \trs{indeed}, p. \pageref{p:[ne:ˀ] ‘just so’}\\
\item \textit{Ne:ˀ} \trs{it is}, p. \pageref{p:[ne:ˀ] `it is’}\\
\item \textit{Ne:ˀ} \trs{it is} (equative), p. \pageref{p:[ne:ˀ] equative}\\
\item \textit{Ne:ˀ dagaihǫ:niˀ} \trs{the reason why}, \trs{that’s why}, \trs{because}, p. \pageref{p:[ne:ˀ dagaihǫ:niˀ]}\\
\item \textit{Ne:ˀ diˀ gęh} \trs{is that it then?}, \trs{is that}, p. \pageref{p:[ne:ˀ diˀ gęh]}\\
\item \textit{Ne:ˀ gęh … ne:ˀ/neˀ nigęˀǫh neˀ …} \trs{or?}, p. \pageref{p:[ne:ˀ gęh … ne:ˀ]}\\
\item \textit{Ne:ˀ giˀ} \trs{just} (emphasis), p. \pageref{p:[ne:ˀ giˀ]}\\
\item \textit{Ne:ˀ giˀ gyę:ˀ hya:ˀ} \trs{before all else}, \trs{first}, p. \pageref{p:[ne:ˀ giˀ gyę:ˀ hya:ˀ]}\\
\item \textit{Ne:ˀ giˀ gyę:ˀ ǫh} \trs{I guess}, p. \pageref{p:[ne:ˀ giˀ gyę:ˀ ǫh]}\\
\item \textit{Ne:ˀ giˀ hniˀ} \trs{and that too}, \trs{and that also}, p. \pageref{p:[ne:ˀ giˀ hniˀ]}\\
\item \textit{Ne:ˀ giˀ shęh hwaˀ} \trs{maybe this time}, p. \pageref{p:[ne:ˀ giˀ shęh hwaˀ]}\\
\item \textit{Ne:ˀ giˀ shęh neˀ … giˀ shęh} \trs{or}, p. \pageref{p:[ne:ˀ giˀ shęh neˀ … giˀ shęh]}\\
\item \textit{Ne:ˀ giˀ tsǫ: gwahs} \trs{that’s really all}, p. \pageref{p:[ne:ˀ giˀ tsǫ: gwahs]}\\
\item \textit{Ne:ˀ gwahs d-} \trs{the most}, p. \pageref{p:[ne:ˀ gwahs d-]}\\
\item \textit{Ne:ˀ gwahs heyohe:ˀ} \trs{the most, -est}, p. \pageref{p:[ne:ˀ gwahs heyohe:ˀ]}\\
\item \textit{Ne:ˀ gwahs hwaˀ} \trs{this time for sure}, p. \pageref{p:[gwahs hwaˀ]}\\
\item \textit{Ne:ˀ gyę:ˀ neˀ} \trs{that really is}, p. \pageref{p:[ne:ˀ gyę:ˀ neˀ]}\\
\item \textit{Ne:ˀ he:gę:} \trs{just}, \trs{only}, \trs{all}, p. \pageref{p:[ne:ˀ he:gę:]}\\
\item \textit{Ne:ˀ hęˀ hne:ˀ} \trs{too}, \trs{also}, p. \pageref{p:[ne:ˀ hęˀ hne:ˀ]}\\
\item \textit{Ne:ˀ hne:ˀ ne:ˀ} \trs{in fact, it is}, \trs{it \emph{is}}, p. \pageref{p:[ne:ˀ hne:ˀ ne:ˀ]}\\
\item \textit{Ne:ˀ hniˀ ne:ˀ} \trs{and that also}, p. \pageref{p:[ne:ˀ hniˀ ne:ˀ]}\\
\item \textit{Ne:ˀ hwaˀ} \trs{this (coming) time}, \trs{when}, p. \pageref{p:[ne:ˀ hwaˀ]}\\
\item \textit{Ne:ˀ neˀ} \trs{it is}, \trs{that is}, \trs{that’s what}, p. \pageref{p:[ne:ˀ neˀ]}\\
\item \textit{Ne:ˀ ǫh} \trs{I guess it is}, p. \pageref{p:[ne:ˀ ǫh]}\\
\item \textit{Ne:ˀ seˀ gę:s neˀ} \trs{back then}, \trs{a long time ago}, p. \pageref{p:[ne:ˀ seˀ gę:s neˀ]}\\
\item \textit{Ne:ˀ seˀ gyę:ˀ} \trs{you know}, p. \pageref{p:[ne:ˀ seˀ gyę:ˀ]}\\
\item \textit{Ne:ˀ tsǫ:} \trs{because}, \trs{it’s just}, p. \pageref{p:[ne:ˀ tsǫ:]}\\
\item \textit{Ne:ˀ tsǫ: shęh} \trs{but}, \trs{it’s just that}, p. \pageref{p:[ne:ˀ tsǫ: shęh]}\\
\item \textit{Ne:ˀ tsǫ:} \trs{that’s only}, \trs{that’s all}, p. \pageref{p:[ne:ˀ tsǫ:] ‘that’s only’}\\
\item \textit{Ne:ˀ/neˀ giˀ gyę:ˀ} \trs{that’s just it}, \trs{that’s it for sure}, p. \pageref{p:[ne:ˀ/neˀ giˀ gyę:ˀ]}\\
\item \textit{Nę: hwaˀ waˀ-ne:ˀ} \trs{nowadays}, p. \pageref{p:[nę: hwaˀ waˀ-ne:ˀ]}\\
\item \textit{Nę: ne:ˀ i:s} \trs{how about you}, p. \pageref{p:[nę: ne:ˀ i:s]}\\
\item \textit{Nęh, ne:ˀ nęh} \trs{when}, \trs{once}, \trs{as soon as}, p. \pageref{p:[nęh, ne:ˀ nęh]}\\
\item \textit{Neˀ gyę:ˀ nę ne:ˀ} (emphasis), p. \pageref{p:[neˀ gyę:ˀ nę ne:ˀ]}\\
\item \textit{Neˀ to: ne:ˀ} \trs{that’s the one}, p. \pageref{p:[neˀ to: ne:ˀ]}\\
\item \textit{O: tęˀ ǫh ga:t giˀ shęh ne:ˀ hwaˀ} \trs{maybe, maybe not}, p. \pageref{p:[o: tęˀ ǫh ga:t giˀ shęh ne:ˀ hwaˀ]}\\
\item \textit{Oh ne:ˀ} \trs{maybe}, p. \pageref{p:[ǫh ne:ˀ]}\\
\item \textit{Tęˀ ne:ˀ de̱ˀgę:} \trs{without, lacking}, p. \pageref{p:[tęˀ ne:ˀ deˀgę:]}\\
\item \textit{Tęˀ tǫ ne:ˀ, Tęˀ to ne:ˀ} \trs{not really}, p. \pageref{p:[tęˀ tǫ ne:ˀ]}\\
\item \textit{To gęh ǫ ne:ˀ} \trs{I wonder if it is}, p. \pageref{p:[to gęh ǫ ne:ˀ]}\\
\item \textit{To: niyo:weˀ ne:ˀ} \trs{when}, p. \pageref{p:[to: niyo:weˀ ne:ˀ]}\\
\item \textit{Waˀ-ne:ˀ} \trs{today}, \trs{now}, p. \pageref{p:[waˀ-ne:ˀ]}
\end{CayugaRelated}

\subsection*{\textbf{Ne:ˀ} \trs{it is} (equative)} \label{p:[ne:ˀ] equative}
\CayugaRemark{Particle or atypical verb \stem{noun … (ne:ˀ) noun} optionally links two nouns (or noun phrases), and carries the idea that the two nouns are at least partly equivalent. The first noun `[ ]’ describes a category, and the second noun `[ ]’ belongs to the category denoted by the first noun. See \sectref{Equative sentences with linking verbs [né:ˀ], [deˀgę:]}.}

\ea
\label{ex:npar38}
\gll [Ganyó:ˀ] hęˀ hne:ˀ (\textbf{ne:ˀ}) [gwíhsgwihs].\\
animal also in.fact (it.is) pigs\\
\glt ‘Pigs are animals.’
\z

\begin{CayugaRelated}
\item \textit{Ne:ˀ} \trs{it is}, p. \pageref{p:[ne:ˀ] `it is’}
\end{CayugaRelated}

\largerpage
\subsection*{\trs{Ne:ˀ} \trs{just so}, \trs{indeed}} \label{p:[ne:ˀ] ‘just so’}
\CayugaRemark{Particle or atypical verb; free-standing expression, functioning as an “agreement marker” (\sectref{ch:Agreement markers}).}

\ea
\label{ex:npar39}
\gll “\textbf{Ne:ˀ},” aˀa:gę́ˀ, “neˀ gyę:ˀ nę \textbf{ne:ˀ} dęyǫ́kiˀnya:ˀ.”\\
it.is she.said the emphasis that.one it.is they.govern.us\\
\glt ‘“That’s it”, she said, “that’s because they govern us.”’ (\cite{henry_de_2005})
\z

\ea
\label{ex:npar40}
\gll “\textbf{Ne:ˀ} tó:gyęh!” ahęˀ.\\
it.is that.one he.said\\
\glt ‘“That’s right!” he said.’ (\cite{henry_de_2005})
\z

\ea
\label{ex:npar41}
\gll Pete: Ne:ˀ diˀ gęh? A:yę́:ˀ hné:ˀ gáegę̱hjih.\\
{} it.is so Q it.seems in.fact they.are.old\\
\glt ‘Is that them? They look old!’


\gll Junior: Ęhę́ˀ \textbf{né:ˀ}.\\
{} yes it.is\\
\glt ‘Yes, that is them.’ (\cite[88]{mithun_watewayestanih_1984}, Eksaˀgó:wah dialogue)
\z

\begin{CayugaRelated}
\item \textit{Ne:ˀ} \trs{it is}, p. \pageref{p:[ne:ˀ] `it is’}\\
\item \textit{Ne:ˀ} \trs{it is} (equative), p. \pageref{p:[ne:ˀ] equative}
\end{CayugaRelated}

\largerpage
\subsection*{\textbf{Ne:ˀ dagaihǫ:niˀ} \trs{the reason why}, \trs{that’s why}, \trs{because}} \label{p:[ne:ˀ dagaihǫ:niˀ]}
\CayugaRemark{Particle group \textit{ne:ˀ … (dagai)hǫ:niˀ} is clause-initial to an independent \xxref{ex:npar42}{ex:npar44} or dependent \xxref{ex:npar45}{ex:npar46} causative clause (\sectref{ch:Causative clauses with [dęˀ ni:yoht shęh] ‘why’, [neˀ hǫ:niˀ] ‘how’}).}

\ea
\label{ex:npar42}
\gll \textbf{Ne:ˀ} seˀ \textbf{hǫ:niˀ} ogyanahséhdǫh.\\
it.is you.know the.reason it-has-hidden-its-tracks\\
\glt ‘That is because it has hiddens its tracks there.’ (\cite{mithun_how_1980})
\z

\ea
\label{ex:npar43}
\gll \textbf{Ne:ˀ} \textbf{hǫ:niˀ} to: ní:yoht.\\
it.is the.reason that it.is.a.certain.way\\
\glt ‘That’s why it is that way.’
\z

\ea
\label{ex:npar44}
\gll Ęhę́ˀ, gye̱ˀdrǫ́ˀ giˀ. \textbf{Né:ˀ} giˀ \textbf{ho:níˀ} to itgéˀs wę̱hnihsragwé:gǫh.\\
yes, she.is.home just. it.is just the.reason there I.am.there all.day\\
\glt ‘Yes, she is home. That’s why I was there all day.’ (\cite[456]{mithun_watewayestanih_1984}, Oyęhsraˀ dialogue)
\z

\ea
\label{ex:npar45}
\gll Ahsǫ́h ní:ˀ agǫtsanǫhwá:s [\textbf{neˀ} \textbf{dagaihǫ:níˀ} gyotedę̱hsǫ:ˀǫ́h agahyagwę́hne:ˀ]\\
still I my.knees.are.sore the the.reason this.other.day I.picked.fruit\\
\glt ‘My knees are still sore because of the other day (when) I picked fruit.’
\z

\ea
\label{ex:npar46}
\gll [\textbf{Ne:ˀ} \textbf{hǫ́:niˀ} í:wi: ęsadewá:yę:s], a:gasheya̱ˀdágenhaˀ sǫgwéˀdasǫˀ í:soˀ godinǫ̱hǫ́kdani:.\\
it.is the.reason I.want you.will.learn you.should.help.them your.people much they.are.sick\\
\glt ‘Because I want you to learn how to help your people, they are very sick.’ (\cite{carrier_legends_2013})
\z

\begin{CayugaRelated}
\item \textit{Ne:ˀ} \trs{it is}, p. \pageref{p:[ne:ˀ] `it is’}
\end{CayugaRelated}

\subsection*{\textbf{Ne:ˀ diˀ gęh} \trs{is that it then?}, \trs{is that…?}} \label{p:[ne:ˀ diˀ gęh]}
\CayugaRemark{Particle group and free-standing expression or tag (\sectref{Tag questions (asking for confirmation from the listener)}).}

\ea
\label{ex:npar47}
\gll \textbf{Ne:ˀ} \textbf{diˀ} \textbf{gęh}?\\
it.is so Q\\
\glt ‘Is that it then?’ (that is, is that what you were referring to?)
\z

\ea
\label{ex:npar48}
\gll Junior: Si giˀ gyę:ˀ dagáęˀ ó:nęh!\\
{} there just this.one they.are.coming now\\
\glt “Well here they come now!”

\gll Pete: \textbf{Ne:ˀ} \textbf{diˀ} \textbf{gęh}? A:yę́:ˀ hné:ˀ gáegę̱hjih.\\
{} it.is so Q? it.seems in.fact they.are.old.women\\
\glt ‘Is that them? They look old!” (\cite[88]{mithun_watewayestanih_1984}, Eksaˀgó:wah dialogue)
\z

\begin{CayugaRelated}
\item \textit{Diˀ} \trs{so}, \trs{then}, p. \pageref{p:[diˀ]}\\
\item \textit{Gęh} \trs{Q} (question marker), p. \pageref{p:[gęh] `Q’}\\
\item \textit{Ne:ˀ} \trs{it is}, p. \pageref{p:[ne:ˀ] `it is’}
\end{CayugaRelated}

\subsection*{\textbf{Ne:ˀ gęh … ne:ˀ/neˀ nigęˀǫh neˀ …} \trs{or?}} \label{p:[ne:ˀ gęh … ne:ˀ]}
\CayugaRemark{Particle group \textit{(ne:ˀ gęh) … ne:ˀ/neˀ nigęˀǫh neˀ …} functions as a conjunction (\sectref{ch:Clauses with conjunctions}), for connecting a list of choices [ ]; \textit{ne:ˀ gęh} appears before the list, and \textit{neˀ/ne:ˀ nigęˀǫh neˀ}, appears after the first item in a set of two choices.}

\ea
\label{ex:npar49}
\gll \textbf{Né:ˀ} \textbf{gęh} [neˀ gwahs ǫ:wéh oya:nréˀ desadǫ̱hwę:jó:nih], \textbf{né:ˀ} \textbf{nigęˀǫ́h} \textbf{neˀ} [haˀdewę̱hnihsragehká:ˀ] tsǫ: hoˀdę́ˀ ęhsehs?\\
it.is Q the really really it.is.good you.want.it, it.is or the everyday.kind just what you.will.use\\
\glt ‘Do you want a really good one, or will you just use the everyday kind?’(\cite[225]{mithun_watewayestanih_1984}, Agyaˀdawíˀtraˀ dialogue)
\z

\ea
\label{ex:npar50}
\gll [Onǫˀgwáˀ] \textbf{neˀ} \textbf{nigęˀǫh} \textbf{neˀ} [ohneganóhs] desadǫhwę:jó:nih.\\
milk the or the water you.want.it\\
\glt ‘Do you want milk or water?’(\cite[45]{michelson_ontario_2011})
\z

\begin{CayugaRelated}
\item \textit{Gęh} \trs{Q} (question marker), p. \pageref{p:[gęh] `Q’}\\
\item \textit{Ne:ˀ} \trs{it is}, p. \pageref{p:[ne:ˀ] `it is’}
\end{CayugaRelated}

\subsection*{\textbf{Ne:ˀ giˀ} \trs{just} (emphasis)} \label{p:[ne:ˀ giˀ]}
\CayugaRemark{Particle group, used for emphasis or focus (\sectref{ch:Focus, contrastive focus, and emphasis markers}); clause-initial.}

\ea
\label{ex:npar51}
\gll \textbf{Ne:ˀ} \textbf{giˀ} é:dǫh.\\
it.is just she.means.it\\
\glt ‘\emph{That’s} what she means.’
\z

\ea
\label{ex:npar52}
\gll \textbf{Ne:ˀ} \textbf{giˀ} to:gyę́h agǫ́:gweh.\\
it.is just that.one woman\\
\glt ‘\emph{She’s} the one.’
\z

\begin{CayugaRelated}
\item \textit{Giˀ} \trs{just}, p. \pageref{p:[giˀ]}\\
\item \textit{Ne:ˀ} \trs{it is}, p. \pageref{p:[ne:ˀ] `it is’}
\end{CayugaRelated}

\subsection*{\textbf{Ne:ˀ/neˀ giˀ gyę:ˀ} \trs{that’s just it}, \trs{that’s it for sure}} \label{p:[ne:ˀ/neˀ giˀ gyę:ˀ]}
\CayugaRemark{Particle group used for emphasis or focus (\sectref{ch:Focus, contrastive focus, and emphasis markers}); free-standing or clause-initial.}

\ea
\label{ex:npar53}
\gll \textbf{Neˀ} \textbf{giˀ} \textbf{gyę́:ˀ} asiˀ.\\
the just this.one you.said\\
\glt ‘That’s what you said.’
\z

\ea
\label{ex:npar54}
\gll \textbf{Neˀ} \textbf{giˀ} \textbf{gyę́:ˀ} gáǫdǫh.\\
the just this.one they(females).mean\\
\glt ‘That’s what the women say.’
\z


\ea
\label{ex:npar55}
\gll “Tga:gǫ́:t gyęˀǫˀǫh hya̱hsó:t, neˀ shęh nyo:ˀ ęhshá:yǫˀ, waˀhéh ó:nęh. \textbf{Néˀ} \textbf{giˀ} \textbf{gyę:ˀ}ˀ.\\
it.is.obligatory I.guess your.grandfather the that a.certain.time he.will.arrive.home, just.now then it.is just this.one\\
\glt “It will have to be when your grandpa arrives home. That’s how it will be.” (\cite{henry_de_2005}) (speaking about when to do something…) 
\z


\ea
\label{ex:npar56}
\gll \textbf{Neˀ} \textbf{giˀ} \textbf{gyę:ˀ} dó:gęhs.\\
the just this.one it.is.true\\
\glt ‘That’s just what we did…’ (\cite{henry_de_2005}) (referring to cleaning up after playing)
\z

\ea
\label{ex:npar57}
\gll “\textbf{Ó:}, \textbf{neˀ} \textbf{giˀ} \textbf{gyę:ˀ}”, aˀa:gę́ˀ, “owidra:htaˀ hniˀ ęwa:dǫ́ˀ ęhsrá neˀ onaˀda:ˀ.”\\
oh the just this.one she.said butter also it.is.possible you.will.spread the bread\\
\glt ‘“Oh, and of course,” she said, “there is butter to spread on your bread.”’ (Henry, 2005)
\z

\begin{CayugaRelated}
\item \textit{Giˀ} \trs{just}, p. \pageref{p:[giˀ]}\\
\item \textit{Giˀ gyę:ˀ} \trs{quite}, \trs{kind of}, p. \pageref{p:[giˀ gyę:ˀ] ‘quite’}\\
\item \textit{Gyę:ˀ} \trs{just the one}, p. \pageref{p:[gyę:ˀ]}\\
\item \textit{Neˀ} \trs{that is}, \trs{the}, p. \pageref{p:[neˀ]}\\
\item \textit{Ne:ˀ} \trs{it is}, p. \pageref{p:[ne:ˀ] `it is’}
\end{CayugaRelated}

\subsection*{\textbf{Ne:ˀ giˀ gyę:ˀ hya:ˀ} \trs{before all else}, \trs{first}} \label{p:[ne:ˀ giˀ gyę:ˀ hya:ˀ]}
\CayugaRemark{Particle group functioning as an “adverb of time” (\sectref{ch:’Adverbs’ of time}); clause-initial.}

\ea
\label{ex:npar58}
\gll Háoˀ. \textbf{Neˀ} \textbf{giˀ} \textbf{gyę́:ˀ} \textbf{hya:ˀ} aekninaˀdá:k agatna̱ˀdáǫt degáhswa̱ˀne:t.\\
OK. it.is just emphasis for.now/this.time we.should.eat I.baked.it pie\\
\glt ‘Mmm. Before anything else, we should eat the pie I baked.’ (\cite[376]{mithun_watewayestanih_1984}, Dajǫh dialogue)
\z

\ea
\label{ex:npar59}
\gll \textbf{Neˀ} \textbf{giˀ} \textbf{gyę́:ˀ} \textbf{hya:ˀ}  ędwa:gyę́ht ęyagwa̱ˀgranhohsro:dę́ˀ nę́:-toh.\\
it.is just emphasis for.now/this.time we.will.knock.down we.will.pile.snow this.one-here\\
\glt ‘Well first, we’ll pile some snow here.’ (\cite[481]{mithun_watewayestanih_1984}, Dędwaˀęnáęˀ dialogue)
\z

\begin{CayugaRelated}
\item \textit{Giˀ} \trs{just}, p. \pageref{p:[giˀ]}\\
\item \textit{Gyę:ˀ} \trs{just the one}, p. \pageref{p:[gyę:ˀ]}\\
\item \textit{Hya:ˀ} \trs{first}, \trs{before anything else}, p. \pageref{p:[hya:ˀ]}\\
\item \textit{Ne:ˀ} \trs{it is}, p. \pageref{p:[ne:ˀ] `it is’}
\end{CayugaRelated}

\subsection*{\textbf{Ne:ˀ giˀ gyę:ˀ ǫh} \trs{I guess}} \label{p:[ne:ˀ giˀ gyę:ˀ ǫh]}
\CayugaRemark{Particle group, emphatic, functioning as an “evidential marker” (\sectref{ch:Evidential markers}); clause-initial.}

\ea
\label{ex:npar60}
\gll Ęhę́ˀ. \textbf{né:ˀ} \textbf{giˀ} \textbf{gyę:ˀ} \textbf{ǫ́h} nę:gyę́h néˀ. Degahwi̱hsda:gé: nigaji̱hwa:nǫ́:ˀ. Né:ˀ ękní:nǫˀ.\\
yes it.is just this.one I.wonder this.one the two.dollars the.hammer.costs it.is I.will.buy.it\\
\glt ‘Yes. I guess I will buy the two-dollar hammer.’ (\cite[158]{mithun_watewayestanih_1984}, Enǫhsǫnyaˀdaˀsǫ́:ˀǫh dialogue)
\z

\begin{CayugaRelated}
\item \textit{Giˀ} \trs{just}, p. \pageref{p:[giˀ]}\\
\item \textit{Gyę:ˀ} \trs{just the one}, p. \pageref{p:[gyę:ˀ]}\\
\item \textit{Ne:ˀ} \trs{it is}, p. \pageref{p:[ne:ˀ] `it is’}\\
\item \textit{Oh, ǫ:, ǫ} \trs{I guess}, \trs{I wonder (if)}, p. \pageref{p:[ǫh, ǫ:, ǫ]}
\end{CayugaRelated}

\subsection*{\textbf{Ne:ˀ giˀ hniˀ} \trs{and that too}, \trs{and that also}} \label{p:[ne:ˀ giˀ hniˀ]}
\CayugaRemark{Particle group \stem{ne:ˀ giˀ \textsc{item} \textsc{item} hniˀ} `also (etc.)’ functions as an emphatic conjunction (\sectref{ch:Clauses with conjunctions}) linking a list of similar items [ ]. \textit{Ne:ˀ giˀ} appears at the beginning of the clause, and the linked items occur between it and \textit{hniˀ}.}

\ea
\label{ex:npar61}
\gll \textbf{Né:ˀ} \textbf{giˀ} [Sampsón], [Deliláh] \textbf{hniˀ} gaǫdatáwa̱hksǫˀ. Ahsę́h nigá:gǫ:.\\
it.is just [Sampson], [Delilah] and their.children. Three a.certain.number\\
\glt ‘Those are Sampson and Delilah’s children. There are three of them.’ (\cite[68]{mithun_watewayestanih_1984}, Eksaˀgó:wah dialogue)
\z

\begin{CayugaRelated}
\item \textit{Giˀ} \trs{just}, p. \pageref{p:[giˀ]}\\
\item \textit{Hniˀ} \trs{and}, p. \pageref{p:[hniˀ] ‘and’}\\
\item \textit{Ne:ˀ} \trs{it is}, p. \pageref{p:[ne:ˀ] `it is’}
\end{CayugaRelated}


\subsection*{\textbf{Ne:ˀ giˀ shęh hwaˀ} \trs{maybe that’s the one}} \label{p:[ne:ˀ giˀ shęh hwaˀ]}
\CayugaRemark{Particle group functioning as a “doubt or certainty marker” (\sectref{ch:Evidential markers}); clause-initial.}

\ea
\label{ex:npar62}
\gll \textbf{Ne:ˀ} \textbf{giˀ} \textbf{shęh} \textbf{hwaˀ} é:dǫh.\\
it.is just that this.time she.means.it\\
\glt ‘Maybe that’s the one she means.’
\z

\begin{CayugaRelated}
\item \textit{Giˀ} \trs{just}, p. \pageref{p:[giˀ]}\\
\item \textit{Hwaˀ} \trs{this time}, \trs{next}, p. \pageref{p:[hwaˀ]}\\
\item \textit{Ne:ˀ} \trs{it is}, p. \pageref{p:[ne:ˀ] `it is’}\\
\item \textit{Shęh} \trs{that}, \trs{because}, p. \pageref{p:[shęh] `because’}
\end{CayugaRelated}

\subsection*{\textbf{Ne:ˀ giˀ shęh neˀ … giˀ shęh} \trs{or}} \label{p:[ne:ˀ giˀ shęh neˀ … giˀ shęh]}
\CayugaRemark{Particle group \textit{(ne:ˀ giˀ shęh neˀ) … giˀ shęh} functions as a conjunction (\sectref{ch:Clauses with conjunctions}), connecting an additional list of choices, [ ]; \textit{ne:ˀ giˀ shęh neˀ} optionally appears before the first item in the list; the other choices are followed by \textit{giˀ shęh}.}

\ea
\label{ex:npar63}
\gll Ó:, a:yę́:ˀ ní:ˀ né:ˀ gwahs knǫ̱hweˀs neˀ hehsá̱ˀęˀ niyohso̱hgoˀdę:, \textbf{né:ˀ} \textbf{giˀ-shę́h} \textbf{neˀ} [ojiˀtgwa:gę́:tˀah], [jiˀtgwá:ˀ \textbf{giˀ-shęh}], [otgwęhji̱ˀa:gę́:t \textbf{giˀ-shęh}].\\
Oh, it.seems I it.is really I.like the brown it.is.coloured, it.is maybe the [light.yellow], [yellow or], [pink or].\\
\glt ‘Oh, the one I like the best is brown, or perhaps a light yellow, or maybe yellow, or maybe pink.’ (\cite[225]{mithun_watewayestanih_1984}, Agyaˀdawíˀtraˀ dialogue)
\z

\ea
\label{ex:npar64}
\gll [Hnaˀgǫ̱hká:ˀ] \textbf{giˀ-shę́h} [sga̱hoˀdę:ˀęh]?\\
underclothes or anything\\
\glt ‘Underclothes or something?’ (\cite[225]{mithun_watewayestanih_1984}, Agyaˀdawíˀtraˀ dialogue)
\z

\ea
\label{ex:npar65}
\gll Tę́ˀ gęh taˀdesadǫhwęjo:níh sga̱hoˀdę:ˀę́h o:yaˀ? [Adáhdi̱ˀtraˀ] [ahdahgwáˀ] \textbf{giˀ-shęh}?.\\
not Q you.don’t.want.it something else? [socks], [shoes] or\\
\glt ‘There was not anything else you wanted? Stockings, or shoes?’ (\cite[225]{mithun_watewayestanih_1984}, Agyaˀdawíˀtraˀ dialogue)
\z


\ea
\label{ex:npar66}
\gll Tęˀ sga̱hoˀdę́ˀ [neˀ oˀwáhǫh], [ji̱ˀdę:ˀę́h], [ojǫˀdáˀ] \textbf{giˀ-shęh}.\\
Not anything [the meat], [birds], [fish] or\\
\glt ‘No meat, no birds, no fish.’ (\cite{carrier_legends_2013}) (Men went hunting, but returned with nothing.)
\z

\begin{CayugaRelated}
\item \textit{Giˀ} \trs{just}, p. \pageref{p:[giˀ]}\\
\item \textit{Giˀ shęh} \trs{maybe}, \trs{or maybe}, p. \pageref{p:[giˀ shęh]}\\
\item \textit{Hwaˀ} \trs{this time}, \trs{next}, p. \pageref{p:[hwaˀ]}\\
\item \textit{Neˀ} \trs{that is}, \trs{the}, p. \pageref{p:[neˀ]}\\
\item \textit{Ne:ˀ} \trs{it is}, p. \pageref{p:[ne:ˀ] `it is’}\\
\item \textit{Shęh} \trs{that}, \trs{because}, p. \pageref{p:[shęh] `because’}
\end{CayugaRelated}

\subsection*{\textbf{Ne:ˀ giˀ tsǫ: gwahs} \trs{that’s really all}} \label{p:[ne:ˀ giˀ tsǫ: gwahs]}
\CayugaRemark{Particle group functioning as a “demonstrative pronoun” (\sectref{ch:’Demonstrative pronouns’}); emphatic; clause-initial.}

\ea
\label{ex:npar67}
\gll Ęyǫgwadáhnyo:k, [\textbf{né:ˀ} \textbf{giˀ} \textbf{tsǫ́:} \textbf{gwahs} ęyagyonhéhgǫ̱hǫ:k],\\
we.will.be.fishing, it.is just only really we.will.live.on.it\\
\glt ‘We’ll be fishing, and that’s all we’ll live on,…’ (\cite[402]{mithun_watewayestanih_1984}, Ęyagwa̱hdę́:diˀ dialogue)
\z

\begin{CayugaRelated}
\item \textit{Giˀ} \trs{just}, p. \pageref{p:[giˀ]}\\
\item \textit{Giˀ tsǫ:} \trs{just}, \trs{really}, p. \pageref{p:[giˀ tsǫ:]}\\
\item \textit{Gwahs} \trs{really}, \trs{just}, \trs{quite}, p. \pageref{p:[gwahs] ‘really’}\\
\item \textit{Ne:ˀ} \trs{it is}, p. \pageref{p:[ne:ˀ] `it is’}\\
\item \textit{Tsǫ:} \trs{just}, \trs{only}, p. \pageref{p:[tsǫ:]}
\end{CayugaRelated}

\subsection*{\textbf{Ne:ˀ gwahs d-} \trs{the most}} \label{p:[ne:ˀ gwahs d-]}
\CayugaRemark{Particle group \stem{neˀ gwahs d-\textsc{stative.verb}} is used in comparisons (\sectref{ch:Comparisons (more, the same, or less)}); clause-initial before a stative verb beginning with \stem{d-/g-} {\cislocative}.}

\ea
\label{ex:npar68}
\gll \textbf{ne:ˀ} \textbf{gwahs} \textbf{d}wakyęda̱hgówanęh\\
it.is emphasis very.big.chair\\
\glt ‘the biggest chair’
\cfex{d-w-akyęda̱hg-ówan-ęh \textsc{\cislocative-3s.a}-chair-big-{\stative}}
\z

\begin{CayugaRelated}
\item \textit{Gwahs} \trs{really}, \trs{just}, \trs{quite}, p. \pageref{p:[gwahs] ‘really’}\\
\item \textit{Ne:ˀ} \trs{it is}, p. \pageref{p:[ne:ˀ] `it is’}
\end{CayugaRelated}

\subsection*{\textbf{Ne:ˀ gwahs heyohe:ˀ} \trs{the most, -est}} \label{p:[ne:ˀ gwahs heyohe:ˀ]}
\CayugaRemark{Particle group \stem{(ne:ˀ) gwahs \textsc{stative.verb} (heyohe:)} is used in comparisons (\sectref{ch:Comparisons (more, the same, or less)}).}

\ea
\label{ex:npar69}
\gll Tohgéh nę:gyę́h \textbf{gwahs} gajihyowa:nę́h \textbf{heyóhe:}.\\
here this.one the.most very.big.hammer it.is.beyond\\
\glt ‘Here is the biggest hammer.’ (\cite[159]{mithun_watewayestanih_1984}, Enǫhsǫnyaˀdaˀsǫ́:ˀǫh dialogue)
\cfex{ga-jihy-owa:n-ę́h \textsc{3s.a}-hammer-big-{\stative}}
 \z

\ea
\label{ex:npar70}
\gll Dá giˀ gwa:díh tga̱hǫˀ \textbf{ne:ˀ} \textbf{gwahs} oyá:nreˀs…\\
Over just here it.rests it.is the.most good.ones\\
\glt ‘Over here are the best (ones)…’ (\cite[159]{mithun_watewayestanih_1984}, Enǫhsǫnyaˀdaˀsǫ́:ˀǫh dialogue)
\cfex{\textsc{3s.p}-good.\textsc{\stative-\plural}}

\z

\begin{CayugaRelated}
\item \textit{Gwahs} \trs{really}, \trs{just}, \trs{quite}, p. \pageref{p:[gwahs] ‘really’}\\
\item \textit{Heyohe:ˀ} \trs{more}, p. \pageref{p:[heyohe:ˀ]}\\
\item \textit{Heyohe:ˀ shęh} \trs{more than}, p. \pageref{p:[heyohe:ˀ shęh]}\\
\item \textit{Neˀ heyohe:ˀ} \trs{more, -er}, p. \pageref{p:[neˀ heyohe:ˀ]}\\
\item \textit{Ne:ˀ} \trs{it is}, p. \pageref{p:[ne:ˀ] `it is’}
\end{CayugaRelated}

\subsection*{\textbf{Ne:ˀ gyę:ˀ neˀ} \trs{that really is}} \label{p:[ne:ˀ gyę:ˀ neˀ]}
\CayugaRemark{Particle group \textit{ne:ˀ gyę:ˀ (neˀ)} is used for emphasis or focus (\sectref{ch:Focus, contrastive focus, and emphasis markers}); clause-initial.}

\ea
\label{ex:npar71}
\gll \textbf{Ne:ˀ} \textbf{gyę:ˀ} gí:dǫh.\\
it.is this.one I.mean\\
\glt ‘That is what I mean.’
\z

\ea
\label{ex:npar72}
\gll \textbf{Ne:ˀ} \textbf{gyę:ˀ} \textbf{neˀ} tó:gyęh.\\
it.is this.one the that.one\\
\glt ‘That \emph{is} the one.’
\z

\begin{CayugaRelated}
\item \textit{Gyę:ˀ} \trs{just the one}, p. \pageref{p:[gyę:ˀ]}\\
\item \textit{Neˀ} \trs{that is}, \trs{the}, p. \pageref{p:[neˀ]}\\
\item \textit{Ne:ˀ} \trs{it is}, p. \pageref{p:[ne:ˀ] `it is’}
\end{CayugaRelated}

\subsection*{\textbf{Ne:ˀ he:gę:} \trs{just}, \trs{only}, \trs{all}} \label{p:[ne:ˀ he:gę:]}
\CayugaRemark{Particle group \textit{ne:ˀ … he:gę:} is used for emphasis or focus (\sectref{ch:Focus, contrastive focus, and emphasis markers}); clause-initial.}

\ea
\label{ex:npar73}
\gll \textbf{Ne:ˀ} \textbf{he:gę:} ęhsyaˀdaniyǫdago̱háǫˀ.\\
it.is all you.will.unhang.bodies\\
\glt ‘All you have to do is take the bodies down.’ (\cite{henry_de_2005}) (speaking of removing rabbits from a snare)
\z

\ea
\label{ex:npar74}
\gll \textbf{Né:ˀ} tsǫ́: \textbf{he:gę́} neˀ ahatsęnǫ:níˀ neˀ o:nę́h saeyǫ́ˀ neˀ knó:haˀ.\\
it.is just all the he.was.happy the now she.returned.home the mother\\
\glt ‘He is just happy that my mother is home.’(\cite[508]{mithun_watewayestanih_1984}, Aǫhdęgyǫheˀ dialogue)
\z

\begin{CayugaRelated}
\item \textit{Ę:, E:} (possible atypical verb), p. \pageref{p:[ę:, e:] `atypical verb’}
\end{CayugaRelated}

\subsection*{\textbf{Ne:ˀ hęˀ hne:ˀ} \trs{too}, \trs{also}} \label{p:[ne:ˀ hęˀ hne:ˀ]}
\CayugaRemark{Particle group \textit{(ne:ˀ) hęˀ hne:ˀ} emphasizes the continuation of a previous topic and the addition of new information (\sectref{ch:Topic continuation markers}); \textit{ne:ˀ hęˀ hne:ˀ} is clause-initial; \textit{…hęˀ hne:ˀ} is enclitic.}

\ea
\label{ex:npar75}
\gll Honahsę́: \textbf{hęˀ} \textbf{hne:ˀ} de̱hęnadęhnǫ́:drǫˀ.\\
they.are.fat also in.fact your.brothers\\
\glt ‘Your brothers are also fat.’
\z

\ea
\label{ex:npar76}
\gll Ganyó:ˀ \textbf{hęˀ} \textbf{hne:ˀ} neˀ/ne:ˀ gwíhsgwihs.\\
Animal also in.fact the/it.is pig\\
\glt ‘Pigs are animals too.’
\z

\ea
\label{ex:npar77}
\gll Toh hǫ́: aˀagǫdá: onę́h \textbf{hęˀ} \textbf{hne:ˀ} aˀeji̱ho:dę́ˀ, gaya:gǫ́: \textbf{hęˀ} \textbf{hné:ˀ} aˀagǫdá:.\\
that.one place she.put.it.in now also in.fact she.closed.it in.the.bag also in.fact she.put.it.in\\
\glt ‘She put it (tea) in the bag too and then she closed the bag too.’ (\cite{henry_de_2005})
\z

\ea
\label{ex:npar78}
\gll \textbf{Ne:ˀ} \textbf{hęˀ} \textbf{hne:ˀ} gáǫgwe̱ˀdaˀ.\\
it.is also in.fact her.people.\\
\glt ‘That is her family.’
\z

\begin{CayugaRelated}
\item \textit{Hęˀ} \trs{also}, \trs{too}, p. \pageref{p:[hęˀ]}\\
\item \textit{Hne:ˀ} \trs{in fact}, p. \pageref{p:[hne:ˀ] ‘in fact’}\\
\item \textit{Ne:ˀ} \trs{it is}, p. \pageref{p:[ne:ˀ] `it is’}\\
\item \textit{Tęˀ hne:ˀ neˀ} \trs{…not \emph{that one}}, p. \pageref{p:[tęˀ hne:ˀ neˀ]}
\end{CayugaRelated}

\subsection*{\textbf{Ne:ˀ hne:ˀ ne:ˀ} \trs{in fact}, \trs{it is}, \trs{it \emph{is}}} \label{p:[ne:ˀ hne:ˀ ne:ˀ]}
\CayugaRemark{Particle group \textit{ne:ˀ hne:ˀ (neˀ)} emphasizes new information contrasting with something said previously (\sectref{ch:Focus, contrastive focus, and emphasis markers}); clause-initial.}

\ea
\label{ex:npar79}
\gll \textbf{Ne:ˀ} \textbf{hne:ˀ} \textbf{neˀ} gaihwagwę:ní:yo:.\\
it.is in.fact the it.is.the.main.idea\\
\glt ‘That in fact is the main thing, the main item. That is the main thing.’
\z

\ea
\label{ex:npar80}
\gll \textbf{Ne:ˀ} \textbf{hne:ˀ} gó:węh.\\
it.is in.fact she.owns.it\\
\glt ‘That’s \emph{hers}.’ (not someone else’s)
\z

\begin{CayugaRelated}
\item \textit{Hne:ˀ} \trs{in fact}, p. \pageref{p:[hne:ˀ] ‘in fact’}\\
\item \textit{Ne:ˀ} \trs{it is}, p. \pageref{p:[ne:ˀ] `it is’}
\end{CayugaRelated}

\subsection*{\textbf{Ne:ˀ hniˀ ne:ˀ} \trs{and that also}} \label{p:[ne:ˀ hniˀ ne:ˀ]}
\CayugaRemark{Particle group signaling the continuation of a previous topic and the addition of new information (\sectref{ch:Topic continuation markers}); clause-initial.}

\ea
\label{ex:npar81}
\gll \textbf{ne:ˀ} \textbf{hniˀ} \textbf{ne:ˀ}\\
it.is in.fact it.is\\
\glt ‘and that also’
\z

\begin{CayugaRelated}
\item \textit{Hniˀ} \trs{and}, p. \pageref{p:[hniˀ] ‘and’}\\
\item \textit{Ne:ˀ} \trs{it is}, p. \pageref{p:[ne:ˀ] `it is’}
\end{CayugaRelated}


\subsection*{\textbf{Ne:ˀ hwaˀ} \trs{this (coming) time}, \trs{when}} \label{p:[ne:ˀ hwaˀ]}
\CayugaRemark{Particle group functioning as an “adverb of time” (\sectref{ch:’Adverbs’ of time}, \sectref{ch:Clauses with [hwę:dǫh], [nęh], [nę:gyęh hwaˀ], [ne:ˀ hwaˀ], etc. ‘when’}); clause-initial.}

\ea
\label{ex:npar85}
\gll [\textbf{Ne:ˀ} \textbf{hwaˀ} nę:gyę́h sá:dǫh: ‘dęyǫkidé:niˀ’] neˀ gę́:dǫh neˀ tęˀ-da̱ˀǫ́: ǫgwehǫ:wéh ǫ:sawá:dǫˀ. \\
it.is this.time this you.say: they.will.change.us the it.means the definitely.not real.people it.will.no.longer.become\\
\glt ‘When you say ‘they will change us,’ it means that you’ll no longer be Indian.’ (\cite{henry_de_2005})
\z

\ea
\label{ex:npar86}
\gll \textbf{Ne:ˀ} \textbf{hwaˀ} ęyotshęnǫ́:ni.ˀ \\
it.is time she.will.be.happy\\
\glt ‘This time she will be happy.’
\z

\ea
\label{ex:npar87}
\gll \textbf{Ne:ˀ} \textbf{hwaˀ} gí:dǫh.\\
it.is time I.mean.it\\
\glt ‘I mean that this time.’
\z

\begin{CayugaRelated}
\item \textit{Hwaˀ} \trs{this time}, \trs{next}, p. \pageref{p:[hwaˀ]}
\end{CayugaRelated}

\subsection*{\textbf{Ne:ˀ neˀ} \trs{it is}, \trs{that is}, \trs{that’s what}} \label{p:[ne:ˀ neˀ]}
\CayugaRemark{Particle group \textit{Ne:ˀ (neˀ)} that singles out the following items as being in focus (\cite{keusen_focus_1994}, see \sectref{ch:Focus, contrastive focus, and emphasis markers}); clause-initial.}

\ea
\label{ex:npar82}
\gll “Ó:, \textbf{ne:ˀ} ni:ˀ tó:gyęh!” \\
oh, it.is I that\\
\glt “Oh, I want (to do) that!” (Henry, 2005) (mend clothes) 
\z


\ea
\label{ex:npar83}
\gll \textbf{Ne:ˀ} gę:s gá:dǫh. \\
it.is usually I.say\\
\glt ‘That’s what I usually say.’
\z

\ea
\label{ex:npar84}
\gll \textbf{Ne:ˀ} gó:węh. \\
it.is she.owns.it\\
\glt ‘That’s \emph{hers}.’ (Speakers said that this phrase was ‘a little more definite’ than “ne:ˀ hne:ˀ gó:węh”.)
\z

\begin{CayugaRelated}
\item \textit{Neˀ} \trs{that is}, \trs{the}, p. \pageref{p:[neˀ]}\\
\item \textit{Ne:ˀ} \trs{it is}, p. \pageref{p:[ne:ˀ] `it is’}
\end{CayugaRelated}


\subsection*{\textbf{Ne:ˀ nęh, Nęh} \trs{when}, \trs{once}, \trs{as soon as}} \label{p:[ne:ˀ nęh, nęh]}
\CayugaRemark{Particle group \textit{(ne:ˀ) nęh} functions as an “adverb of time” (\sectref{ch:’Adverbs’ of time}); clause-initial to dependent clauses. }

\begin{CayugaRelated}
\item \textit{Nęh, ne:ˀ nęh} \trs{when}, \trs{once}, \trs{as soon as}, p. \pageref{p:[nęh, ne:ˀ nęh]}
\end{CayugaRelated}

\subsection*{\textbf{Ne:ˀ/neˀ nigęˀǫh neˀ …, Ne:ˀ gęh … ne:ˀ/neˀ nigęˀǫh neˀ …} \trs{or?}} \label{p:[ne:ˀ/neˀ nigęˀǫh neˀ …]}
\CayugaRemark{Particle group \textit{(ne:ˀ gęh) … ne:ˀ/neˀ nigęˀǫh neˀ …} functions as a conjunction (\sectref{ch:Clauses with conjunctions}), for connecting a list of choices; \textit{ne:ˀ gęh} appears before the list, and \textit{neˀ/ne:ˀ nigęˀǫh neˀ}, appears after the first item in a list of two choices. }


\begin{CayugaRelated}
\item \textit{Ne:ˀ gęh … ne:ˀ/neˀ nigęˀǫh neˀ …} \trs{or?}, p. \pageref{p:[ne:ˀ/neˀ nigęˀǫh neˀ …]}
\end{CayugaRelated}

\subsection*{\textbf{Ne:ˀ ǫh} \trs{I guess it is}} \label{p:[ne:ˀ ǫh]}
\CayugaRemark{Particle group, emphatic, functioning as an “evidential marker” (\sectref{ch:Evidential markers}); clause-initial.}

\ea
\label{ex:npar88}
\gll \textbf{Ne:ˀ} \textbf{ǫh} neˀ o:nę́h haˀgahé:ˀ edwadri̱hóˀda:t. \\
it.is I.guess the.now it.is time we.should.work\\
\glt ‘I guess it is now time for us to work.’
\z

\begin{CayugaRelated}
\item \textit{Ne:ˀ} \trs{it is}, p. \pageref{p:[ne:ˀ] `it is’}\\
\item \textit{Oh, ǫ:, ǫ} \trs{I guess}, \trs{I wonder (if)}, p. \pageref{p:[ǫh, ǫ:, ǫ]}\\
\item \textit{Oh ne:ˀ} \trs{maybe}, p. \pageref{p:[ǫh ne:ˀ]}
\end{CayugaRelated}

\subsection*{\textbf{Ne:ˀ seˀ gę:s neˀ} \trs{generally}, \trs{used to}, \trs{usually}, \trs{normally}} \label{p:[ne:ˀ seˀ gę:s neˀ]}
\CayugaRemark{Particle group \textit{ne:ˀ (seˀ) gę:s neˀ} functions as an “adverb of time” (\sectref{ch:’Adverbs’ of time}); clause-initial.}

\ea
\label{ex:npar89}
\gll \textbf{Ne:ˀ} \textbf{seˀ} \textbf{gę:s} \textbf{neˀ} sweˀgé:hah hęnadę̱hní:nǫh gę:s neˀ gwaˀyǫˀ gyę:ˀ. \\
that you.know usually the long.ago they.sell used.to this rabbit then\\
\glt ‘A long time ago, they used to sell rabbits.’ (\cite{mithun_how_1980})
\z

\begin{CayugaRelated}
\item \textit{Gę:s} \trs{generally, used to, usually, normally}, p. \pageref{p:[gę:s]}\\
\item \textit{Neˀ} \trs{that is}, \trs{the}, p. \pageref{p:[neˀ]}\\
\item \textit{Ne:ˀ} \trs{it is}, p. \pageref{p:[ne:ˀ] `it is’}\\
\item \textit{Seˀ} \trs{you know}, p. \pageref{p:[seˀ]}
\end{CayugaRelated}

\subsection*{\textbf{Ne:ˀ seˀ gyę:ˀ} \trs{you know}} \label{p:[ne:ˀ seˀ gyę:ˀ]}
\CayugaRemark{Particle group signaling shared knowledge (known to both speaker and listener, \sectref{ch:Shared knowledge markers}); or just a means of keeping the floor in a conversation; clause-initial.}

\ea
\label{ex:npar90}
\gll \textbf{Ne:ˀ} \textbf{séˀ} \textbf{gyę:ˀ} gę:s to:gyę́h hwaˀ nęh sǫhéh nęh gadidaksénǫgyeˀs shęh nhǫ́: ohádenyǫˀ. \\
it.is you.know this.one usually that.one this.time once nighttime once they.roamed.about that where roads.are\\
\glt ‘This is how it is, you know, at night when they run around their roads.’ (\cite{mithun_how_1980})
\z

\ea
\label{ex:npar91}
\gll Ahę́ˀ, “\textbf{Ne:ˀ} \textbf{séˀ} \textbf{gyę:ˀ} shęh tęˀ ní:s desana̱hsgwáęˀ toh nagana̱hsgwiyohá:k shęh niyóht neˀ Oditragáˀǫh.” \\
he.said it.is you.know this.one that not you you.have.no.pet that.one how.nice.a.pet.it.is that how.so the good.tea\\
\glt ‘He said, “Because you don’t have a pet as nice as Good Tea.”’ (\cite{keye_hnyagwaidatgigowah_2012})
\z

\begin{CayugaRelated}
\item \textit{Gyę:ˀ} \trs{just the one}, p. \pageref{p:[gyę:ˀ]}\\
\item \textit{Ne:ˀ} \trs{it is}, p. \pageref{p:[ne:ˀ] `it is’}\\
\item \textit{Seˀ} \trs{you know}, p. \pageref{p:[seˀ]}
\end{CayugaRelated}

\subsection*{\textbf{Ne:ˀ tsǫ:} \trs{that’s only}, \trs{that’s all}} \label{p:[ne:ˀ tsǫ:] ‘that’s only’}
\CayugaRemark{Particle group functioning as an “adverb of degree” (\sectref{ch:’Adverbs’ of degree}); clause-initial.}

\ea
\label{ex:npar92}
\gll \textbf{Ne:ˀ} \textbf{tsǫ:} hǫwáya̱ˀda:ˀs. \\
it.is just she.depends.on.him\\
\glt ‘He is the only one she depends on.’
\z

\ea
\label{ex:npar93}
\gll \textbf{Ne:ˀ} \textbf{tsǫ:} dehoya̱ˀdówę̱hdǫh. \\
it.is just he.is.thinking.about.it\\
\glt ‘That’s all he is thinking about.’ (He is preoccupied.)
\z

\begin{CayugaRelated}
\item \textit{Ne:ˀ} \trs{it is}, p. \pageref{p:[ne:ˀ] `it is’}\\
\item \textit{Ne:ˀ tsǫ:} \trs{because}, \trs{it’s just}, p. \pageref{p:[ne:ˀ tsǫ:]}\\
\item \textit{Ne:ˀ tsǫ: shęh} \trs{it’s just that}, p. \pageref{p:[ne:ˀ tsǫ: shęh]}\\
\item \textit{Tsǫ:} \trs{just}, \trs{only}, p. \pageref{p:[tsǫ:]}
\end{CayugaRelated}


\subsection*{\textbf{Ne:ˀ tsǫ:} \trs{because}, \trs{it's just}} \label{p:[ne:ˀ tsǫ:]}
\CayugaRemark{Particle group \textit{ne:ˀ (tsǫ:) …} is clause-initial in a causative independent \xref{ex:npar94} or dependent \xref{ex:npar95} clause (\sectref{ch:Causative clauses with [dęˀ ni:yoht shęh] ‘why’, [neˀ hǫ:niˀ] ‘how’}).}

\ea
\label{ex:npar94}
\gll \textbf{Né:ˀ} gyę:ˀ tréhs⁠1 degahǫ̱hstǫ́ˀe:s. \\
it.is emphasis too.much it.has.long.whiskers\\
\glt ‘Because its whiskers are too long.’ (\cite[317]{mithun_watewayestanih_1984}, Sanahsgwaęˀ Gęh Sga̱hoˀdę:ˀęh? dialogue) (Pete is answering David’s question, “Why is it called that?”)
\z

\ea
\label{ex:npar95}
\gll Gowę́dagę: [\textbf{ne:ˀ} \textbf{tsǫ:} niyésgyędaˀ]. \\
she.is.timid it.is just she.is.small-boned\\
\glt ‘She is timid because she is just a small person.’
\z

\begin{CayugaRelated}
\item \textit{Ne:ˀ} \trs{it is}, p. \pageref{p:[ne:ˀ] `it is’}\\
\item \textit{Ne:ˀ tsǫ:} \trs{that’s only}, \trs{that’s all}, p. \pageref{p:[ne:ˀ tsǫ:] ‘that’s only’}\\
\item \textit{Ne:ˀ tsǫ: shęh} \trs{it’s just that}, p. \pageref{p:[ne:ˀ tsǫ: shęh]}\\
\item \textit{Tsǫ:} \trs{just}, \trs{only}, p. \pageref{p:[tsǫ:]}
\end{CayugaRelated}

\subsection*{\textbf{Ne:ˀ tsǫ: shęh} \trs{but}, \trs{it’s just that}} \label{p:[ne:ˀ tsǫ: shęh]}
\CayugaRemark{Particle group functioning as a “conjunction” (\sectref{ch:Clauses with conjunctions}); clause-initial.}

\ea
\label{ex:npar96}
\gll Sǫgweˀdi:yó: giˀ hné:; \textbf{né:ˀ} \textbf{tsǫ́:} \textbf{shęh} [tę́ˀ gwahs ǫ:wé de̱ˀaknǫ̱hweˀǫ́:ˀ shę nhó: snagreˀ], tréhs a:yę́:ˀ ǫknigǫ̱hsá:dǫˀk. \\
you.are.nice just in.fact, it.is just that [not really really I.don’t.like that where you.live], too it.seems I’m.lonesome\\
\glt ‘Well you are a nice person, but I don’t really like your neighborhood. I seem to get too lonesome.’ (\cite[88]{mithun_watewayestanih_1984}, Eksaˀgó:wah dialogue)
\z

\ea
\label{ex:npar97}
\gll Ęhę́ˀ, ganǫ́:ˀ, [\textbf{né:ˀ} \textbf{tsǫ:} \textbf{shęh} oya:nréˀ giˀ gyę́:ˀ gwa̱ˀtoh.] \\
yes it.is.expensive it.is just that good.one just this.one that.one\\
\glt ‘Yes, it is expensive, but it is a good one.’ (\cite[159]{mithun_watewayestanih_1984}, Enǫhsǫnyaˀdaˀsǫ́:ˀǫh dialogue)
\z

\begin{CayugaRelated}
\item \textit{Ne:ˀ} \trs{it is}, p. \pageref{p:[ne:ˀ] `it is’}\\
\item \textit{Ne:ˀ tsǫ: shęh} \trs{it’s just that}, p. \pageref{p:[ne:ˀ tsǫ: shęh]}\\
\item \textit{Shęh} \trs{that}, \trs{because}, p. \pageref{p:[shęh] `because’}\\
\item \textit{Tsǫ:} \trs{just}, \trs{only}, p. \pageref{p:[tsǫ:]}
\end{CayugaRelated}

\subsection*{\textbf{Nę:} \trs{this}, \trs{these}} \label{p:[nę:] ‘this, these’}
\CayugaRemark{Particle functioning as a “demonstrative pronoun” (\sectref{ch:’Demonstrative pronouns’}); free-standing or phrase-initial (as in \textit{nę: atsógwa̱hdaˀ} ‘these pipes, this pipe’).}

\ea
\label{ex:npar98}
\gll Ęhę́ˀ, ohya:jíˀ hoˀdę́ˀ degahswaˀné:t ǫgwá:yęˀ. \textbf{Nę:} giˀ gyę́:ˀ sna̱ˀdá:k. \\
Yes, berry kind pie we.have.it. this.one just emphasis eat.it. \\
\glt ‘Yes, we have blueberry pie. Here then, eat this (pie).’ (\cite[294]{mithun_watewayestanih_1984}, Dwade:kǫ́:nih dialogue)
\z

\ea
\label{ex:npar99}
\gll O:nę́h giˀ, gado:gę́: gaetsgó:t, gado:gę́: hniˀ agaejaodę́ˀ \textbf{nę:} atsógwa̱hdaˀ. \\
Now just, together they.sat, together and they.smoked this.one pipe\\
\glt ‘And then they both sat together and smoked these pipes.’ (\cite{henry_de_2005})
\z

\ea
\label{ex:npar100}
\gll Gaę gwaˀ giˀshęh nhǫ:wéh a:gadǫ́hsę:ˀ \textbf{nę:} niyónisheˀ? \\
which right.then maybe place I.would.rest this.one how.it.takes.time\\
\glt ‘Maybe a place to rest before that time?’ (\cite{carrier_legends_2013})
\z

\begin{CayugaRelated}
\item \textit{Da: nę: dah} \trs{and now}, p. \pageref{p:[da: nę: dah]}\\
\item \textit{Nę:} \trs{look!}, \trs{say!}, \trs{see}, p. \pageref{p:[nę:] ‘look!’, ‘say!’, ‘see’}\\
\item \textit{Nę: diˀ ni:s} \trs{how about you?}, p. \pageref{p:[nę: diˀ ni:s]}\\
\item \textit{Nę: hwaˀ waˀ-ne:ˀ} \trs{nowadays}, p. \pageref{p:[nę: hwaˀ waˀ-ne:ˀ]}\\
\item \textit{Nę: ne:ˀ i:s} \trs{how about you}, p. \pageref{p:[nę: ne:ˀ i:s]}\\
\item \textit{Nę: toh} \trs{here}, p. \pageref{p:[nę: toh]}\\
\item \textit{Nę: toh gwa:dih} \trs{on this side}, p. \pageref{p:[nę: toh gwa:dih]}\\
\item \textit{Nę: tsǫ: gwaˀ toh ni-} \trs{just a little bit}, \trs{very little}, p. \pageref{p:[nę: tsǫ: gwaˀ toh ni-]}\\
\item \textit{Nę: tsǫ: ni-} \trs{just}, p. \pageref{p:[nę: tsǫ: ni-]}\\
\item \textit{Nę:-dah} \trs{this}, \trs{this way}, p. \pageref{p:[nę:-dah] `this’, ‘this way’}\\
\item \textit{Nę:-dah} \trs{here, take this}, p. \pageref{p:[nę:-dah] ‘Here!’}\\
\item \textit{Nę:-gyęh} \trs{this one}, p. \pageref{p:[nę:-gyęh]}\\
\item \textit{Nę:-gyęh gwa:dih} \trs{here}, \trs{this side}, \trs{over here}, p. \pageref{p:[nę:-gyęh gwa:dih]}\\
\item \textit{Nę:-gyęh hwaˀ} \trs{this time}, p. \pageref{p:[nę:-gyęh hwaˀ] ‘this time’}
\end{CayugaRelated}

\subsection*{\textbf{Nę:} \trs{look!}, \trs{say!}, \trs{see!}} \label{p:[nę:] ‘look!’, ‘say!’, ‘see’}
\CayugaRemark{Particle functioning as a “command” (\sectref{Particles and particle groups used with commands}), said when drawing attention to something.}

\ea
\label{ex:npar101}
\gll \textbf{Nę:}. Tęnénǫgyeˀs. Háe! háe! \\
Look they.originate.from.there hi hi\\
\glt ‘Look. There they are. Hi! Hi!’ (\cite[481]{mithun_watewayestanih_1984}, Dędwaˀęnáęˀ dialogue)
\z

\ea
\label{ex:npar102}
\gll \textbf{Nę́:} haˀsatga̱htóh o:yáˀ tganǫ́hso:t. A:yę́:ˀ sǫgwaˀ nˀáht gonęna̱ˀdínyǫ̱ˀdǫh. \\
say have.a.look other house.standing.over.there it.seems someone person they.have.moved.in\\
\glt ‘Say, have a look at the next house. It seems like somebody has moved in.’ (\cite[441]{mithun_watewayestanih_1984}, Dwęnǫhsanékahǫˀ dialogue)
\z

\ea
\label{ex:npar103}
\gll \textbf{Nę:} swatgahtóh gra̱he:t. \\
Look! take.a.look tree\\
\glt ‘Look at this pine tree.’ (\cite{carrier_legends_2013})
\z

\ea
\label{ex:npar104}
\gll \textbf{Nę:} - ne:ˀ hǫ:niˀ gonahdrǫ́ˀs neˀ ǫ́:gweh. \\
see - the reason.why they.are.afraid the people\\
\glt ‘See – that’s what frightens the people.’ (Carrier et al., 2013)
\z

\begin{CayugaRelated}
\item \textit{Nę:} \trs{this}, \trs{these}, p. \pageref{p:[nę:] ‘this, these’}
\end{CayugaRelated}

\subsection*{\textbf{Nę: diˀ ni:s} \trs{how about you?}} \label{p:[nę: diˀ ni:s]}
\CayugaRemark{Particle group; free-standing.}

\ea
\label{ex:npar105}
\gll \textbf{Nę:} \textbf{diˀ} \textbf{ni:s}?\\
this(one) so you\\
\glt ‘How about you?’
\z

\begin{CayugaRelated}
\item \textit{Diˀ} \trs{so}, \trs{then}, p. \pageref{p:[diˀ]}\\
\item \textit{I:s, Ni:s} \trs{you} (any number of people), p. \pageref{p:[i:s]}\\
\item \textit{Nę:} \trs{this}, \trs{these}, p. \pageref{p:[nę:] ‘this, these’}
\end{CayugaRelated}

\subsection*{\textbf{Nę: hwaˀ waˀ-ne:ˀ} \trs{nowadays}} \label{p:[nę: hwaˀ waˀ-ne:ˀ]}
\CayugaRemark{Particle group functioning as an “adverb of time” (\sectref{ch:’Adverbs’ of time}); clause-initial.}

\ea
\label{ex:npar113}
\gll \textbf{Nę:} \textbf{hwaˀ} \textbf{waˀne:ˀ} tęˀ de̱ˀsgánagreˀ. \\
now just.now today not they-no-longer-live\\
\glt ‘And nowadays, there are not very many around anymore.’ (\cite{mithun_how_1980})
\z

\begin{CayugaRelated}
\item \textit{Hwaˀ} \trs{this time}, \trs{next}, p. \pageref{p:[hwaˀ]}\\
\item \textit{Nę:} \trs{this}, \trs{these}, p. \pageref{p:[nę:] ‘this, these’}\\
\item \textit{Waˀ-ne:ˀ} \trs{today}, \trs{now}, p. \pageref{p:[waˀ-ne:ˀ]}
\end{CayugaRelated}

\subsection*{\textbf{Nę: ne:ˀ i:s} \trs{how about you?}} \label{p:[nę: ne:ˀ i:s]}
\CayugaRemark{Particle group; free-standing.}

\ea
\label{ex:npar106}
\gll \textbf{Nę:} \textbf{ne:ˀ} \textbf{i:s}? Dęˀ nis hoˀdęˀ nisaˀnigǫ́hoˀdę:? \\
this.one it.is you what you kind what.your.thoughts.are\\
\glt ‘How about you? What are your thoughts?
\z

\begin{CayugaRelated}
\item \textit{I:s, Ni:s} \trs{you} (any number of people), p. \pageref{p:[i:s]}\\
\item \textit{Ne:ˀ} \trs{it is}, p. \pageref{p:[ne:ˀ] `it is’}\\
\item \textit{Nę:} \trs{this}, \trs{these}, p. \pageref{p:[nę:] ‘this, these’}
\end{CayugaRelated}

\subsection*{\textbf{Nę: toh} \trs{this place (here)}, \trs{this one}} \label{p:[nę: toh]}
\CayugaRemark{Particle group functioning as an “adverb of place” (\sectref{‘Adverbs’ of place}) or as a “demonstrative pronoun” (\sectref{ch:’Demonstrative pronouns’}). Clause-initial for “adverb of place” and free-standing for “demonstrative pronoun” function.}

\ea
\label{ex:npar119}
\gll \textbf{Nę:-tóh} giˀ nhǫ́: ęgatnˀǫhdáhgoˀ. \\
this.one-here just where I.will.debark\\
\glt ‘I’ll get out here.’ (\cite[339]{mithun_watewayestanih_1984}, Oˀdréhdatgiˀ dialogue)
\z

\ea
\label{ex:npar120}
\gll Tę́ˀ gyę́:ˀ hné:ˀ de̱hadi̱ˀdrehdǫ́:nihs \textbf{nę:-toh}. \\
Not this.one in.fact they.don’t.repair.cars this.one-here\\
\glt ‘They don’t repair cars here.’(\cite[339]{mithun_watewayestanih_1984}, Oˀdréhdatgiˀ dialogue)
\z

\ea
\label{ex:npar121}
\gll \textbf{Nę: toh} ni:yǫ́: hó:yęˀ. \\
this.one here a.certain.number he.has\\
\glt ‘That’s how many he has.’
\z

\begin{CayugaRelated}
\item \textit{Nę:} \trs{this}, \trs{these}, p. \pageref{p:[nę:] ‘this, these’}\\
\item \textit{Nę: toh gwa:dih} \trs{on this side}, p. \pageref{p:[nę: toh gwa:dih]}\\
\item \textit{To:, Toh, Tǫ:} \trs{there}, \trs{that place}, p. \pageref{p:[to:] ‘there’, ‘that place’}
\end{CayugaRelated}

\subsection*{\textbf{Nę: toh gwa:dih} \trs{on this side}} \label{p:[nę: toh gwa:dih]}
\CayugaRemark{Particle group functioning as an “adverb of place” (\sectref{‘Adverbs’ of place}); clause-initial; \textit{gwa:dih} can also be spelled or pronounced as \textit{gwai}.}

\ea
\label{ex:npar114}
\gll \textbf{Nę:} \textbf{toh} \textbf{gwaih} hęˀdrǫˀ Allan. \\
this.one that.one side he.lives.there Allan\\
\glt ‘Allan lives on this side.’
\z

\begin{CayugaRelated}
\item \textit{Gwa:dih, gwai} \trs{to one side}, p. \pageref{p:[gwa:dih]}\\
\item \textit{Nę:} \trs{this}, \trs{these}, p. \pageref{p:[nę:] ‘this, these’}\\
\item \textit{Nę: toh} \trs{here}, p. \pageref{p:[nę: toh]}\\
\item \textit{To:, Toh, Tǫ:} \trs{there}, \trs{that place}, p. \pageref{p:[to:] ‘there’, ‘that place’}
\end{CayugaRelated}

\subsection*{\textbf{Nę: tsǫ: gwaˀ toh ni-} \trs{just a little bit}, \trs{very little}} \label{p:[nę: tsǫ: gwaˀ toh ni-]}
\CayugaRemark{Particle group functioning as an “adverb of degree” (\sectref{ch:’Adverbs’ of degree}); clause-initial. A following stative-only verb begins with \stem{ni-} {\partitive}.}

\ea
\label{ex:npar115}
\gll \textbf{Nę:} \textbf{tsǫ:} \textbf{gwaˀ} \textbf{toh} \textbf{ni}yǫ́: hohwíhsdaęˀ.\\
this.here just even that a.certain.amount he.has.money\\
\glt ‘Also, right now he has very little money.’
\z

\ea
\label{ex:npar116}
\gll Hohsę́: \textbf{nę:} \textbf{tsǫ:} \textbf{gwaˀ} \textbf{toh} \textbf{ni}hahnę́:ye:s. \\
he.is.fat this.here just even that he.has.some.height\\
\glt ‘He is stout and he is also short (literally, not even tall).’
\z

\begin{CayugaRelated}
\item \textit{Gwaˀ} \trs{immediately}, \trs{right then}, \trs{just then}, \trs{finally}, p. \pageref{p:[gwaˀ] ‘immediately’}\\
\item \textit{Nę:} \trs{this}, \trs{these}, p. \pageref{p:[nę:] ‘this, these’}\\
\item \textit{To:, Toh, Tǫ:} \trs{that one}, p. \pageref{p:[to:] ‘that one’}\\
\item \textit{Tsǫ:} \trs{just}, \trs{only}, p. \pageref{p:[tsǫ:]}
\end{CayugaRelated}

\subsection*{\textbf{Nę: tsǫ: ni-} \trs{just}} \label{p:[nę: tsǫ: ni-]}
\CayugaRemark{Particle group functioning as an “adverb of degree” (\sectref{ch:’Adverbs’ of degree}); precedes a stative verb beginning with \stem{ni-} {\partitive}; clause-initial.}

\ea
\label{ex:npar117}
\gll \textbf{Nę:} \textbf{tsǫ́:} \textbf{ni}yǫ́: hohwíhsdaęˀ. \\
this.here just a.certain.amount he.has.money\\
\glt ‘He has very little money.’
\z

\ea
\label{ex:npar118}
\gll \textbf{Nę:} \textbf{tsǫ:} \textbf{ni}watonáˀdaˀ. \\
this.here just a.certain-sized.potato\\
\glt ‘It’s just a small, puny potato.’
\z

\begin{CayugaRelated}
\item \textit{Nę:} \trs{this}, \trs{these}, p. \pageref{p:[nę:] ‘this, these’}\\
\item \textit{Tsǫ:} \trs{just}, \trs{only}, p. \pageref{p:[tsǫ:]}
\end{CayugaRelated}

\subsection*{\textbf{Nę:-dah} \trs{this}, \trs{this way}} \label{p:[nę:-dah] `this’, ‘this way’}
\CayugaRemark{Particle group functioning as a “demonstrative pronoun” (\sectref{ch:’Demonstrative pronouns’}); free-standing.}

\ea
\label{ex:npar107}
\gll Nęh toh hędwa:yǫ́ˀ \textbf{nę:-dáh} nędwá:ye:ˀ. \\
when that.place we.will.arrive.there this.one-and we.will.do.something\\
\glt ‘When we arrive there, we will do it this way.’
\z

\begin{CayugaRelated}
\item \textit{Da:} \trs{and}, p. \pageref{p:[da:]}\\
\item \textit{Nę:} \trs{this}, \trs{these}, p. \pageref{p:[nę:] ‘this, these’}\\
\item \textit{Nę:-dah} \trs{here, take this}, p. \pageref{p:[nę:-dah] ‘Here!’}
\end{CayugaRelated}


\subsection*{\textbf{Nę:-dah} \trs{here, take this}} \label{p:[nę:-dah] ‘Here!’}
\CayugaRemark{Particle group functioning as a “command” (\sectref{Particles and particle groups used with commands}), said when handing an item to someone.}

\ea
\label{ex:npar108}
\gll \textbf{Nę:-dáh} giˀ gyę:ˀ satgę:ˀsé: agwáya̱ˀda:ˀ. \\
this.here-and just this.one take.a.look our.pictures\\
\glt ‘In fact here, have a look at our pictures.’ (\cite[257]{mithun_watewayestanih_1984}, Ga̱hwajiyá:deˀ dialogue)
\z

\begin{CayugaRelated}
\item \textit{Da:} \trs{and}, p. \pageref{p:[da:]}\\
\item \textit{Nę:} \trs{this}, \trs{these}, p. \pageref{p:[nę:] ‘this, these’}
\end{CayugaRelated}

\subsection*{\textbf{Nę:-gyęh} \trs{this one}} \label{p:[nę:-gyęh]}
\CayugaRemark{Particle group functioning as a “demonstrative pronoun” (\sectref{ch:’Demonstrative pronouns’}); free-standing or preceding the word it modifies (as in \textit{nę:gyęh hwaˀ} ‘this time’).}

\ea
\label{ex:npar109}
\gll Ne:ˀ hęˀ hne:ˀ ho:wę́h \textbf{nę́:-gyęh}. \\
it.is also in.fact it.is.his this.one\\
\glt ‘This is also his.’
\z

\begin{CayugaRelated}
\item \stem{-gyęh} `this, `that’, p. \pageref{p:[-gyęh]}\\
\item \textit{Nę:} \trs{this}, \trs{these}, p. \pageref{p:[nę:] ‘this, these’}\\
\item \textit{Nę:-gyęh gwa:dih} \trs{here}, \trs{this side}, \trs{over here}, p. \pageref{p:[nę:-gyęh gwa:dih]}\\
\item \textit{Nę:-gyęh hwaˀ} \trs{this time}, p. \pageref{p:[nę:-gyęh hwaˀ] ‘this one’}
\end{CayugaRelated}


\subsection*{\textbf{Nę:-gyęh gwa:dih} \trs{here}, \trs{this side}, \trs{over here}} \label{p:[nę:-gyęh gwa:dih]}
\CayugaRemark{Particle group functioning as an “adverb of place” (\sectref{‘Adverbs’ of place}); clause-initial; \textit{gwa:dih} can also be spelled or pronounced as \textit{gwai}.}

\ea
\label{ex:npar110}
\gll Né:ˀ giˀ \textbf{nę:gyę́h} \textbf{gwai} iha:t ne:ˀ hne:ˀ hehso:t,… \\
it.is just this side he.stands it.is in.fact my.grandfather\\
\glt ‘This one standing over here is my grandfather,…’ (\cite[257]{mithun_watewayestanih_1984}, Ga̱hwajiyá:deˀ dialogue)
\z

\begin{CayugaRelated}
\item \textit{Gwa:dih, gwai} \trs{to one side}, p. \pageref{p:[gwa:dih]}\\
\item \textit{Nę:-gyęh} \trs{this one}, p. \pageref{p:[nę:-gyęh]}
\end{CayugaRelated}

\subsection*{\textbf{Nę:-gyęh hwaˀ} \trs{this one}} \label{p:[nę:-gyęh hwaˀ] ‘this one’}
\CayugaRemark{Particle group functioning as “demonstrative pronoun” (\sectref{ch:’Demonstrative pronouns’}); free-standing.}

\ea
\label{ex:npar111}
\gll Ne:ˀ nęh toh nęyá:węh hęwátgatsaˀt shęh nhǫ́: hesodá:hǫh \textbf{nę́:-gyęh} \textbf{hwaˀ} gáęho:t\\
that-is when that.one so.it.will.happen it.will.come.off that place you.have.hooked.it this.one this.time trap\\
\glt ‘And when that happens, the part of the trap you hooked on will come right off.’ (\cite{mithun_how_1980})
\z

\begin{CayugaRelated}
\item \textit{Hwaˀ} \trs{this time}, \trs{next}, p. \pageref{p:[hwaˀ]}\\
\item \textit{Nę:-gyęh}  \trs{this one}, p. \pageref{p:[nę:-gyęh]}
\end{CayugaRelated}


\subsection*{\textbf{Nę:-gyęh hwaˀ} \trs{this time}, \trs{when}} \label{p:[nę:-gyęh hwaˀ] ‘this time’}
\CayugaRemark{Particle group functioning as an “adverb of time” (\sectref{ch:’Adverbs’ of time}, \sectref{ch:Clauses with [hwę:dǫh], [nęh], [nę:gyęh hwaˀ], [ne:ˀ hwaˀ], etc. ‘when’}); clause-initial in independent and dependent clauses [].}

\ea
\label{ex:npar112}
\gll Dęˀ diˀ hoˀdę́ˀ nę̱hsa:gyé:ˀ [\textbf{nę:gyę́h} \textbf{hwaˀ} dęjogęnhǫ́:diˀ]? \\
What so what.kind you.will.do.something this this.time it.will.be.summer\\
\glt ‘What will you do when summer comes?’ (\cite[349]{mithun_watewayestanih_1984}, Dęˀ Hoˀdęˀ Nǫ̱hsa:gye:ˀ? dialogue)
\z

\begin{CayugaRelated}
\item \textit{Nę:-gyęh} \trs{this one}, p. \pageref{p:[nę:-gyęh]}\\
\item \textit{Nę:-gyęh hwaˀ} \trs{this time}, p. \pageref{p:[nę:-gyęh hwaˀ] ‘this time’}
\end{CayugaRelated}


\subsection*{\textbf{Nęh, Ne:ˀ nęh} \trs{when}, \trs{once}, \trs{as soon as}} \label{p:[nęh, ne:ˀ nęh]}
\CayugaRemark{Particle group \textit{(ne:ˀ) nęh} functions as an “adverb of time” (\sectref{ch:’Adverbs’ of time}); clause-initial in dependent clauses []. In contrast, \textit{o:nęh} `when’ is used in independent clauses (see \sectref{ch:Clauses with [hwę:dǫh], [nęh], [nę:gyęh hwaˀ], [ne:ˀ hwaˀ], etc. ‘when’}).}

\ea
\label{ex:npar122}
\gll Ęgahyagwahsé:k hęˀ ni:ˀ [\textbf{nęh} ęgáhdo:k]. \\
I.will.habitually.pick.fruit also me when I.will.grow.up\\
\glt ‘I’ll be a fruit picker when I grow up.’
\z

\ea
\label{ex:npar123}
\gll \textbf{ne:ˀ} \textbf{nęh} dwa̱hdę́:gyǫh \\
it.is when we.will.arrive.there\\
\glt ‘when we get there’
\z

\ea
\label{ex:npar124}
\gll [\textbf{Nęh} toh hędwá:yǫˀ] nę:dáh nędwá:ye:ˀ. \\
when there we.will.arrive.there this.way we.will.do.something\\
\glt ‘When we arrive there, we will do it this way.’
\z

\ea
\label{ex:npar125}
\gll Ęyosdaęda̱ˀǫhǫ́:k gi̱ˀ-shę́-hwaˀ hne:ˀ [\textbf{nę́h} ętsyeh]. \\
it.will.have.stopped.raining maybe indeed when you.will.awaken\\
\glt ‘Perhaps the rain will have stopped when you wake back up.’ (\cite[212]{mithun_watewayestanih_1984}, Satrǫ́:nih dialogue)
\z

\begin{CayugaRelated}
\item \textit{Ne:ˀ} \trs{it is}, p. \pageref{p:[ne:ˀ] `it is’}\\
\item \textit{O:nęh, neˀ o:nęh} \trs{now}, \trs{when}, \trs{then}, \trs{at this time}, p. \pageref{p:[o:nęh]}
\end{CayugaRelated}


\subsection*{\textbf{Nęh to:hah, O:nęh to:hah} \trs{soon}, \trs{almost}} \label{p:[nęh to:hah, o:nęh to:hah]}
\CayugaRemark{Particle group \textit{(o:)nęh…to:hah} functions as an “adverb of time” (\sectref{ch:’Adverbs’ of time}); clause-initial. }

\begin{CayugaRelated}
\item \textit{O:nęh to:hah} \trs{soon}, \trs{almost}, p. \pageref{p:o:nęh to:hah]}
\end{CayugaRelated}

\subsection*{\textbf{Ni:ˀ gyę:ˀ, Ni:ˀ gę:ˀ} \trs{I did it}} \label{p:[ni:ˀ gyę:ˀ]}
\CayugaRemark{Particle group functioning as an “emphatic pronoun” (\sectref{ch:’Emphatic pronouns’}); free-standing or clause-initial.}

\ea
\label{ex:npar126}
\gll \textbf{Ni:} \textbf{gę:} tóh-ne:ˀ ǫgahdǫ́: neˀ agétgwę̱ˀdaˀ. Tę́ˀ ní:s ga̱ˀ-tóh de̱ˀse:gę:? \\
the.me emphasis this.one I.lost.it the my.wallet not you anywhere you.didn’t.see\\
\glt ‘…mind you, \emph{I} lost my wallet. You haven’t seen it?’ (\cite[184]{mithun_watewayestanih_1984}, Gatgwę́ˀdaˀ dialogue)
\z

\ea
\label{ex:npar127}
\gll \textbf{Ni:} \textbf{gę:} giˀ ahí:ˀ a:gęniháˀ sa:wę́ˀ degahenáˀtraˀse:ˀ. \\
the.me emphasis just I.thought I.should.borrow you.own.it scissors\\
\glt ‘\emph{I} thought I might borrow your scissors.’ (Mithun and Henry, 1984, 368) (Degahenáˀtra̱ˀse:ˀ dialogue)
\z

\begin{CayugaRelated}
\item \textit{Gyę:ˀ} \trs{just the one}, p. \pageref{p:[gyę:ˀ]}\\
\item \textit{I:ˀ, Ni:ˀ}, \trs{I}, \trs{we}, p. \pageref{p:[i:ˀ]}
\end{CayugaRelated}

\subsection*{\textbf{Nigęˀǫh} (conjunction)} \label{p:[…nigęˀǫh]}
\CayugaRemark{Particle \textit{…nigęˀǫh} ‘or’ appears in conjunction phrases (\sectref{ch:Clauses with conjunctions}, \sectref{ch:[nigę́ˀǫh] ‘or?’, [giˀ shęh] ‘or’, etc}); enclitic.}

\begin{CayugaRelated}
\item \textit{Gęh ga:t giˀ shęh nigęˀǫh} \trs{or?}, p. \pageref{p:[gęh ga:t giˀ shęh nigęˀǫh]}\\
\item \textit{Gęh tęˀ nigęˀǫh} \trs{or not?}, p. \pageref{p:[gęh tęˀ nigęˀǫh]}\\
\item \textit{Ne:ˀ gęh … ne:ˀ/neˀ nigęˀǫh neˀ …} \trs{or?}, p. \pageref{p:[ne:ˀ/neˀ nigęˀǫh neˀ …]}
\end{CayugaRelated}

\subsection*{\textbf{Nigwa:dih} \trs{-ward}, \trs{direction}} \label{p:[nigwa:dih]}
\CayugaRemark{Atypical verb functioning as an “adverb of place” (\sectref{‘Adverbs’ of place}); preceded by a word that describes the actual direction; clause-initial. Also see \textit{gwa:dih, gwai} ‘to one side’}


\ea
\label{ex:npar128}
\gll Ó:, tę́ˀ gyę́:ˀ-ǫh \textbf{onenǫˀgéh} \textbf{nigwa:díh} haˀgeˀ. \\
oh not that-I.guess south side I’m.going.there\\
\glt ‘Oh, I don’t think so. I am going south.’ (\cite[349]{mithun_watewayestanih_1984}, Dęˀ Hoˀdęˀ Nǫ̱sa:gye:ˀ dialogue)
\z

\begin{CayugaRelated}
\item \textit{Gwa:dih, gwai} \trs{to one side}, p. \pageref{p:[gwa:dih]}
\end{CayugaRelated}

\subsection*{\textbf{Niyo:weˀ, Ni:yo:ˀ, Nyo:ˀ} \trs{a certain distance}, \trs{a certain time}} \label{p:[niyo:weˀ]}
\CayugaRemark{Atypical verb functioning as an “adverb of place” (\sectref{‘Adverbs’ of place}); typically preceded by another word; \textit{niyó:weˀ} also has the short forms \textit{ní:yo:ˀ} or \textit{nyo:ˀ}.}

\begin{CayugaRelated}
\item \textit{Gaoˀ shęh niyo:weˀ} \trs{before}, \trs{until}, p. \pageref{p:[gaoˀ shęh niyo:weˀ]}\\
\item \textit{Niyo:weˀ, ni:yo:ˀ, nyo:ˀ} \trs{a certain distance}, \trs{a certain time}, p. \pageref{p:[niyo:weˀ]}\\
\item \textit{Shęh niyo:weˀ} \trs{as far as}, \trs{as much as}, p. \pageref{p:[shęh niyo:weˀ]}\\
\item \textit{Toh niyo:weˀ} \trs{that far}, p. \pageref{p:[toh niyo:weˀ]}\\
\item \textit{To: niyo:weˀ ne:ˀ} \trs{when}, p. \pageref{p:[to: niyo:weˀ ne:ˀ]}
\end{CayugaRelated}


\subsection*{\textbf{None:ˀ, Nǫne:ˀ} \trs{mind you}, \trs{you know}} \label{p:[none:ˀ]}
\CayugaRemark{Particle signaling shared knowledge (known to both speaker and listener, \sectref{ch:Shared knowledge markers}); free-standing.}

\ea
\label{ex:npar129}
\gll Sǫheh gyę:ˀ \textbf{nóne:ˀ} gá:dǫh deyonadáwęnyeˀ. \\
nighttime this.one you.know I.say they.walk.about\\
\glt ‘As I said, at night, you know, they walk around.’ (\cite{mithun_how_1980})
\z

\ea
\label{ex:npar130}
\gll O:nę́h giˀ gyę́:ˀǫh ęsgahdę́:diˀ. O:nę́h gyę́:ˀ \textbf{nǫ́ne:ˀ} hné:ˀ haˀwa:jˀáht degáhswa̱ˀne:t! \\
now just I.guess I.will.go.home now emphasis you.know in.fact it.has.gone.down.to.nothing pie\\
\glt ‘Well, I might as well go home now, because the pie is all gone, you know!’ (\cite[376]{mithun_watewayestanih_1984}, Dajǫh dialogue)
\z

\begin{CayugaRelated}
\item \textit{Ne:ˀ} \trs{it is}, p. \pageref{p:[ne:ˀ] `it is’}\\
\item \textit{Nę:} \trs{this}, \trs{these}, p. \pageref{p:[nę:] ‘this, these’}
\end{CayugaRelated}


\subsection*{\textbf{Nya:węh} \trs{thank you}, \trs{thanks}} \label{p:[nya:węh]}
\CayugaRemark{Particle signaling acknowledgement (\sectref{ch:Acknowledgement markers}); also an expression of thanks.} At the end of the meal, each person says \textit{nyá:węh} \trs{thanks} or \textit{nya:wę́h (seˀ) hę́ˀ ni:ˀ} \trs{thanks, me too} (to the Creator), and the other people say \textit{nyoh} in acknowledgement.

\ea
\label{ex:npar131}
\gll \textbf{Nya:wę́h} giˀ gyę́:ˀ adi:dwá:dǫ:t. \\
acknowledge just this.one we.all.ate.together\\
\glt ‘Thank you that we ate together.’ (\cite[294]{mithun_watewayestanih_1984}, Dwade:kǫ́:nih dialogue)
\z

\ea
\label{ex:npar132}
\gll Haoˀ diˀ sáh. O:nę́h giˀ hyá:. \textbf{Nya:wę́h} giˀ gyę́:ˀ shęh nyó: dasknˀǫ́hda:. \\
Ok so you.know now just time acknowledge just this.one that place how.it.is you.put.me.in\\
\glt ‘Alright. So long. Thanks for the ride.’ (\cite[339]{mithun_watewayestanih_1984}, Oˀdréhdatgiˀ dialogue)
\z

\ea
\label{ex:npar133}
\gll Ahęˀ neˀ hagęhjiháˀ \textbf{nya:wę́h} shęh ehswá:yǫˀ. \\
he.said the old.man acknowledge that you.all.came.here\\
\glt ‘The old man said thank you because you all came here.’ (\cite{keye_hnyagwaidatgigowah_2012})
\z

\begin{CayugaRelated}
\item \textit{Nyoh} \trs{you’re welcome}, \trs{alright}, \trs{o.k.}, p. \pageref{p:[nyoh]}
\end{CayugaRelated}

\subsection*{\textbf{Nyoh} \trs{you're welcome}, \trs{alright}, \trs{o.k.}} \label{p:[nyoh]}
\CayugaRemark{Particle signaling acknowledgement (\sectref{ch:Acknowledgement markers}); often said in response to \textit{nya:węh} ‘thanks’. One can also say it sarcastically to bug someone.}


\ea
\label{ex:npar134}
\gll \textbf{Nyoh}! Nya:wę́h hniˀ. O:nę́h giˀ ęgakeho:wíˀ akwá:ji:yaˀ, ęgakeˀnigǫhaędáhdęˀ shęh niyó:gyę:ˀ. \\
Ok! Thanks as.well now just I.will.tell.them my.family I.will.make.them.understand that how.it.is \\
\glt ‘Thank you Hi’No’. I will tell them and make them understand.’ (\cite{carrier_legends_2013}) (In response to Hi’no’ instructing the Maiden to do something)
\z

\ea
\label{ex:npar135}
\gll “\textbf{Nyóh},” ahęˀ neˀ hagę́hjih, “dęgadręnǫ:táhs giˀ neˀ ehswadra̱hswíhyo̱ha:k”.\\
o.k. he.said the old.man I.will.pray/hope just the you.come.to.have.good.luck\\
\glt ‘“O.k.”, said the old man, “I will hope you have good luck.”’ (\cite{keye_hnyagwaidatgigowah_2012})
\z

\begin{CayugaRelated}
\item \textit{Nya:węh} \trs{thank you}, \trs{thanks}, p. \pageref{p:[nya:węh]}
\end{CayugaRelated}

\section{O particles}
\subsection*{\textbf{O:, O:o:} \trs{oh}} \label{p:[o:, ó:ò:]}
\CayugaRemark{Particle functioning as an “exclamation” (\sectref{ch:Exclamations}), expressing surprise or interest.}

\ea
\label{ex:opart2}
\gll \textbf{Ó:ò:}, gaę nhǫ́: gyagohdęgyǫ:?\\
oh which place she.comes.from.there\\
\glt ‘Oh. Where does she come from?’ (\cite[61]{mithun_watewayestanih_1984}, Sǫ: Hne:ˀ Nˀaht Tó:gyęh? dialogue)
\z

\begin{CayugaRelated}
\item \textit{O: gęh} \trs{really?}, p. \pageref{p:[o: gęh]}\\
\item \textit{O: tęˀ ǫh ga:t giˀ shęh gyę:gwaˀ} \trs{maybe, maybe not}, p. \pageref{p:[o: tęˀ ǫh ga:t giˀ shęh gyę:gwaˀ]}\\
\item \textit{O: tęˀ ǫh ga:t giˀ shęh ne:ˀ hwaˀ} \trs{maybe, maybe not}, p. \pageref{p:[o: tęˀ ǫh ga:t giˀ shęh ne:ˀ hwaˀ]}
\end{CayugaRelated}

\subsection*{\trs{O ǫ:, O: ǫh} \trs{oh really?}} \label{p:[o ˀǫ:, o: ǫh]}
\CayugaRemark{Particle group functioning as a “doubt or certainty marker” (\sectref{ch:Evidential markers}), implying \textit{ęhęˀ} \trs{yes} or \textit{tęˀ} \trs{no} answers.}

\ea
\label{ex:opart3}
\gll \textbf{O:} \textbf{ǫh}?\\
oh I.guess\\
\glt ‘Oh really?’
\z


\subsection*{\trs{O: gęh} \trs{really?}} \label{p:[o: gęh]}
\CayugaRemark{Particle group functioning as a “doubt or certainty marker” (\sectref{ch:Evidential markers}) and implying \textit{ęhęˀ} \trs{yes} or \textit{tęˀ} \trs{no} answers.}

\ea
\label{ex:opart1}
\gll \textbf{O:} \textbf{gęh}?\\
oh Q\\
\glt ‘Really?’
\z


\begin{CayugaRelated}
\item \textit{Gęh} \trs{Q} (question marker), p. \pageref{p:[gęh] `Q’}\\
\item \textit{O:, ó:ò:} \trs{oh}, p. \pageref{p:[o:, ó:ò:]}
\end{CayugaRelated}

\subsection*{\textbf{O: tęˀ ǫh ga:t giˀ shęh gyę:gwaˀ} \trs{maybe, maybe not}} \label{p:[o: tęˀ ǫh ga:t giˀ shęh gyę:gwaˀ]}
\CayugaRemark{Particle group \textit{o: tęˀ ǫh … (gat) giˀ shęh gyę:gwaˀ} functions as a “doubt or certainty marker” (\sectref{ch:Evidential markers}); free-standing.}

\ea
\label{ex:opart4}
\gll \textbf{O:} \textbf{tęˀ} \textbf{ǫh} (giˀ), \textbf{(ga:t-)giˀ} \textbf{shęh} \textbf{gyę:gwaˀ}.\\
oh not I.guess just just that if\\
\glt ‘Maybe, maybe not.’
\z

\begin{CayugaRelated}
\item \textit{Ga:t giˀ shęh} \trs{maybe}, \trs{or maybe}, p. \pageref{p:[ga:t giˀ shęh]}\\
\item \textit{Ga:t giˀ shęh tęˀ} \trs{maybe not}, \trs{or not}, p. \pageref{p:[ga:t giˀ shęh tęˀ]}\\
\item \textit{Gęh ga:t giˀ shęh nigęˀǫh} \trs{or?}, p. \pageref{p:[gęh ga:t giˀ shęh nigęˀǫh]}\\
\item \textit{Giˀ} \trs{just}, p. \pageref{p:[giˀ]}\\
\item \textit{Gyę:gwaˀ a:-verb} \trs{if}, p. \pageref{p:[gyę:gwaˀ]}\\
\item \textit{Gyę:gwaˀ giˀ shęh hwaˀ} \trs{maybe this time}, p. \pageref{p:[gyę:gwaˀ giˀ shęh hwaˀ]}\\
\item \textit{O:, ó:ò:} \trs{oh}, p. \pageref{p:[o:, ó:ò:]}\\
\item \textit{O: tęˀ ǫh ga:t giˀ shęh ne:ˀ hwaˀ} \trs{maybe, maybe not}, p. \pageref{p:[o: tęˀ ǫh ga:t giˀ shęh ne:ˀ hwaˀ]}\\
\item \textit{Oh, ǫ:, ǫ} \trs{I guess}, \trs{I wonder (if)}, p. \pageref{p:[ǫh, ǫ:, ǫ]}\\
\item \textit{Shęh} \trs{that}, \trs{because}, p. \pageref{p:[shęh] `because’}\\
\item \textit{Tęˀ} \trs{no}, \trs{not}, p. \pageref{p:[tęˀ]}\\
\item \textit{Tęˀ giˀ shęh hwaˀ daˀǫ} \trs{maybe not}, p. \pageref{p:[tęˀ giˀ shęh hwaˀ daˀǫ]}
\end{CayugaRelated}



\subsection*{\textbf{O: tęˀ ǫh ga:t giˀ shęh ne:ˀ hwaˀ} \trs{maybe, maybe not}} \label{p:[o: tęˀ ǫh ga:t giˀ shęh ne:ˀ hwaˀ]}
 \CayugaRemark{Particle group \textit{o: tęˀ ǫh … (gat) giˀ shęh ne:ˀ hwaˀ} functions as a “doubt or certainty marker” (\sectref{ch:Evidential markers}); free-standing.}

\ea
\label{ex:opart5}
\gll \textbf{O:} \textbf{tęˀ} \textbf{ǫ:} (giˀ), \textbf{(ga:t-)giˀ} \textbf{shęh} \textbf{ne:ˀ} \textbf{hwaˀ}.\\
oh not I.guess just just that it.is this.time\\
\glt ‘Maybe, maybe not.’
\z

\begin{CayugaRelated}
\item \textit{Ga:t giˀ shęh} \trs{maybe}, \trs{or maybe}, p. \pageref{p:[ga:t giˀ shęh]}\\
\item \textit{Ga:t giˀ shęh tęˀ} \trs{maybe not}, \trs{or not}, p. \pageref{p:[ga:t giˀ shęh tęˀ]}\\
\item \textit{Gęh ga:t giˀ shęh nigęˀǫh} \trs{or?}, p. \pageref{p:[gęh ga:t giˀ shęh nigęˀǫh]}\\
\item \textit{Giˀ} \trs{just}, p. \pageref{p:[giˀ]}\\
\item \textit{O:, ó:ò:} \trs{oh}, p. \pageref{p:[o:, ó:ò:]}\\
\item \textit{O: tęˀ ǫh ga:t giˀ shęh gyę:gwaˀ} \trs{maybe, maybe not}, p. \pageref{p:[o: tęˀ ǫh ga:t giˀ shęh gyę:gwaˀ]}\\
\item \textit{Oh, ǫ:, ǫ} \trs{I guess}, \trs{I wonder (if)}, p. \pageref{p:[ǫh, ǫ:, ǫ]}\\
\item \textit{Shęh} \trs{that}, \trs{because}, p. \pageref{p:[shęh] `because’}\\
\item \textit{Tęˀ} \trs{no}, \trs{not}, p. \pageref{p:[tęˀ]}\\
\item \textit{Tęˀ giˀ shęh hwaˀ daˀǫ} \trs{maybe not}, p. \pageref{p:[tęˀ giˀ shęh hwaˀ daˀǫ]}
\end{CayugaRelated}




\subsection*{\textbf{Ogwęhę:gyeˀ} \trs{now and then}, \trs{now and again}} \label{p:[ogwęhę:gyeˀ]}
\CayugaRemark{Atypical verb functioning as an “adverb of time” (\sectref{ch:’Adverbs’ of time}); clause-initial.}

\ea
\label{ex:opart6}
\gll \textbf{Ogwehę:gyéˀ} gę:s ahsyǫˀ.\\
now.and.again usually you’ll.arrive\\
\glt ‘Now and again you’ll be here.’
\z

\begin{CayugaRelated}
\item \textit{Hę:-gyęh, Hę:gyeh} \trs{no matter}, p. \pageref{p:[hę:-gyęh]}
\end{CayugaRelated}


\subsection*{\textbf{Ohę:dǫ:} \trs{early}, \trs{first}} \label{p:[ohę:dǫ:]}
\CayugaRemark{Verb functioning as an “adverb of time” (\sectref{ch:’Adverbs’ of time}); clause-initial.}

\ea
\label{ex:opart7}
\gll \textbf{Ohę:dǫ́:} ihseˀ.\\
early/first you.are.here\\
\glt ‘You are the first to arrive.’, ‘You are early.’
\z

\begin{CayugaRelated}
\item \textit{Ohę:dǫ: shęh} \trs{ahead}, \trs{in front}, \trs{forward}, p. \pageref{p:[ohę:dǫ: shęh]}
\end{CayugaRelated}


\subsection*{\textbf{Ohę:dǫ: shęh} \trs{ahead}, \trs{in front}, \trs{forward}} \label{p:[ohę:dǫ: shęh]}
\CayugaRemark{Verb and particle \textit{ohę:dǫ: (shęh)} functions as an “adverb of place” (\sectref{‘Adverbs’ of place}); clause-initial.}

\ea
\label{ex:opart8}
\gll Hoˀę:náˀ atahk, o:nę́h \textbf{ohę:dǫ́:} haˀtę́hda:t.\\
his.spear he.picked.it.up now ahead he.went.ahead\\
\glt ‘He grabbed his spear and he dashed ahead of his brothers.’ (\cite{carrier_legends_2013})
\z

\ea
\label{ex:opart9}
\gll \textbf{Ohę:dǫ́:} heˀga:yę́ˀ ęˀnhó:traˀ.\\
ahead it.is.lying.there ball\\
\glt ‘The ball is lying ahead.’
\z

\ea
\label{ex:opart10}
\gll \textbf{ohę:dǫ́:} (\textbf{shęh}) ganǫ́hso:t\\
ahead (that) standing.house\\
\glt ‘the house in front’, `in front of the house’
\z

\ea
\label{ex:opart11}
\gll \textbf{Ohę:dǫ́:} heˀsénǫ̱gyeˀ.\\
forward you.are.going.along.that.way\\
\glt ‘you are moving forward.’
\z

\begin{CayugaRelated}
\item \textit{Ohę:dǫ:} \trs{early}, \trs{first}, p. \pageref{p:[ohę:dǫ:]}\\
\item \textit{Shęh} \trs{that}, \trs{because}, p. \pageref{p:[shęh] `because’}
\end{CayugaRelated}


\subsection*{\textbf{Ohnaˀgę:ˀ, Naˀgę:ˀ} \trs{late}} \label{p:[ohnaˀgę:ˀ]}
\CayugaRemark{Verb \textit{(oh)naˀgę:ˀ} functions as an “adverb of time” (\sectref{ch:’Adverbs’ of time}); clause-initial.}

\ea
\label{ex:opart12}
\gll \textbf{Ohnaˀgę́:ˀ} ihseˀ\\
late you.are.here\\
\glt ‘You are late again.’ (said at the moment)
\z

\begin{CayugaRelated}
\item \textit{Ohnaˀgę: gwa:dih shęh} \trs{behind}, p. \pageref{p:[ohnaˀgę: gwa:dih shęh]}\\
\item \textit{Ohnaˀgęhjih} \trs{late}, \trs{back then}, \trs{the bottom}, p. \pageref{p:[ohnaˀgęhjih] ‘late’}\\
\item \textit{Ohnaˀgǫ:} \trs{underneath}, \trs{beneath}, \trs{under}, p. \pageref{p:[ohnaˀgǫ:]}
\end{CayugaRelated}


\subsection*{\textbf{Ohnaˀgę: gwa:dih shęh} \trs{behind}, \trs{back}, \trs{backwards}} \label{p:[ohnaˀgę: gwa:dih shęh]}
\CayugaRemark{Verb and particle group \textit{(oh)naˀgę: (gwa:dih) (shęh)} functioning as an “adverb of place” (\sectref{‘Adverbs’ of place}); clause-initial; \textit{gwa:dih} can also be spelled or pronounced as \textit{gwai}.}

\ea
\label{ex:opart13}
\gll Ne: giˀ tsǫ: \textbf{naˀgę:} \textbf{gwá:dih} \textbf{shęh} ga̱hnáwęhtak…\\
it.is just only behind side that the.falls.rushed.out\\
\glt ‘But in a cave behind the rushing waters…’ (\cite{carrier_legends_2013})
\z

\ea
\label{ex:opart14}
\gll Ahgwíh \textbf{ohnaˀgę́:ˀ} hęjiswatgáhtoh.\\
don’t back you.all.will.look.that.way\\
\glt ‘You must not look back.’ (\cite{carrier_legends_2013})
\z

\ea
\label{ex:opart15}
\gll Gei ni̱hęnado:wá:s neˀ gwa̱ˀ-tóh neˀ hodi:tse:nę́ˀ so:wá:s \textbf{ohnaˀgę́:ˀ} hadínę̱hę:ˀ.\\
four a.number.of.hunters the that.one the their.pet dog behind they.stand.in.line\\
\glt ‘The hunters and their small dog are close behind the dipper’s handle.’ (\cite{carrier_legends_2013})  (speaking of the placement of the hunters and their dog in the handle of the Big Dipper)
\z

\ea
\label{ex:opart16}
\gll \textbf{ohnaˀgę́:ˀ/naˀgę́:ˀ} \textbf{shęh} ganǫ́hso:t\\
behind that standing.house\\
\glt ‘behind the house’
\z

\ea
\label{ex:opart17}
\gll \textbf{ohna:gę́:ˀ} ganǫ́hsadeˀ\\
behind the.existing.house\\
\glt ‘the house behind the house’
\z

\ea
\label{ex:opart18}
\gll Ne: giˀ tsǫ: \textbf{naˀgę:} gwá:dih shęh ga̱hnáwęhtak…\\
it.is just only behind side that the.waters.issue.out\\
\glt ‘But in a cave behind the rushing waters…’ (\cite{carrier_legends_2013})
\z

\ea
\label{ex:opart19}
\gll \textbf{Ohnaˀgę:} haˀdesatsáˀge:t.\\
behind bend.yourself\\
\glt ‘Bend backward!’
\z

\begin{CayugaRelated}
\item \textit{Gwa:dih, gwai} \trs{to one side}, p. \pageref{p:[gwa:dih]}\\
\item \textit{Ohnaˀgę:ˀ, naˀgę:ˀ} \trs{late}, p. \pageref{p:[ohnaˀgę:ˀ]}\\
\item \textit{Shęh} \trs{that}, \trs{because}, p. \pageref{p:[shęh] `because’}
\end{CayugaRelated}


\subsection*{\textbf{Ohnaˀgęhjih} \trs{late}, \trs{way back then}, \trs{the bottom}} \label{p:[ohnaˀgęhjih] ‘late’}
\CayugaRemark{Verb functioning as an “adverb of time” (\sectref{ch:’Adverbs’ of time}) or “adverb of place” (\sectref{‘Adverbs’ of place}); clause-initial.}

\ea
\label{ex:opart20}
\gll \textbf{Ohnaˀgęhjíh} ahá:yǫˀ.\\
late he.arrived\\
\glt ‘He arrived late.’
\z

\begin{CayugaRelated}
\item \textit{Ohnaˀgę:ˀ, naˀgę:ˀ} \trs{late}, p. \pageref{p:[ohnaˀgę:ˀ]}\\
\item \textit{Ohnaˀgę: gwa:dih shęh} \trs{behind}, p. \pageref{p:[ohnaˀgę: gwa:dih shęh]}\\
\item \textit{Ohnaˀgęhjih} \trs{late}, \trs{back then}, p. \pageref{p:[ohnaˀgęhjih] ‘late’}\\
\item \textit{Ohnaˀgǫ:} \trs{underneath}, \trs{beneath}, \trs{under}, p. \pageref{p:[ohnaˀgǫ:]}
\end{CayugaRelated}


\subsection*{\textbf{Ohnaˀgǫ:} \trs{underneath}, \trs{beneath}, \trs{under}} \label{p:[ohnaˀgǫ:]}
\CayugaRemark{Noun functioning as an “adverb of place” (\sectref{‘Adverbs’ of place}); clause-initial.}

\ea
\label{ex:opart21}
\gll \textbf{Ohnaˀgǫ́h} tgá:yęˀ.\\
below it.is.lying.there\\
\glt ‘It lies underneath, below.
\z

\begin{CayugaRelated}
\item \textit{Ohnaˀgę:ˀ, naˀgę:ˀ} \trs{late}, p. \pageref{p:[ohnaˀgę:ˀ]}\\
\item \textit{Ohnaˀgę: gwa:dih shęh} \trs{behind}, p. \pageref{p:[ohnaˀgę: gwa:dih shęh]}\\
\item \textit{Ohnaˀgęhjih} \trs{late}, \trs{back then}, \trs{the bottom}, p. \pageref{p:[ohnaˀgęhjih] ‘late’}
\end{CayugaRelated}

\subsection*{\textbf{Ona:węh} \trs{it's theirs (animals)}, \trs{it belongs to them (animals)}, \trs{their (animals)}} \label{p:[ona:węh]}
\CayugaRemark{Verb functioning as a “possessive pronoun” (\sectref{ch:’Possessive pronouns’}); free-standing.}

\ea
\label{ex:opart22}
\glll \textbf{oná:węh} {}\\
on-á:w-ęh\\
\textsc{3p.p}-own-{\stative}\\
\glt ‘they (animals) own it’
\z


\subsection*{\textbf{O:nęh} \trs{now}, \trs{when}, \trs{then}, \trs{at this time}} \label{p:[o:nęh]}
\CayugaRemark{Particle \textit{o:nęh} functions as an “adverb of time” (\sectref{ch:’Adverbs’ of time}). Clause-initial in independent clauses only. \textit{Nęh} is used in dependent clauses.}

\ea
\label{ex:opart23}
\gll Gaę diˀ nhǫ:wéh nihsweˀs \textbf{ó:nęh}?\\
which then place you.all.are.here now\\
\glt ‘Where then are you now?’
\z

\ea
\label{ex:opart24}
\gll \textbf{O:nę́h} giˀ haˀgahé:ˀ ęshétro̱hnaˀ.\\
now just the.time.has.come you.will.take.him.back\\
\glt ‘Now it is time for you to take him back.’
\z

\ea
\label{ex:opart25}
\gll \textbf{O:nę́h} giˀ hodehsroníhsˀǫh.\\
now just he.is.ready\\
\glt ‘He is ready now.’
\z

\ea
\label{ex:opart26}
\gll \textbf{O:nę́h} giˀ ęgoyęˀęsętwáhsǫˀǫh.\\
now just I.will.kick.you.around\\
\glt ‘I am now going to kick you around.’
\z

\ea
\label{ex:opart27}
\gll Ó:, hyeiˀshǫ́:ˀǫh áǫda̱hseˀ. \textbf{Onęh}-ˀǫ ęyagokwai̱hséha:k.\\
O, six-ish you.should.come then-I.guess she.will.have.cooked.the.food\\
\glt ‘Oh, you should come around six-ish. She should have the food cooked by then.’ (\cite[279]{mithun_watewayestanih_1984}, Ǫdwęnǫdáhtaˀ dialogue)
\z

\begin{CayugaRelated}
\item \textit{Nęh, ne:ˀ nęh} \trs{when}, \trs{once}, \trs{as soon as}, p. \pageref{p:[nęh, ne:ˀ nęh]}\\
\item \textit{O:nęh, neˀ o:nęh} \trs{now}, \trs{when}, \trs{then}, \trs{at this time}, p. \pageref{p:[o:nęh]}\\
\item \textit{O:nęh e:ˀ} \trs{again!}, p. \pageref{p:[o:nęh e:ˀ]}\\
\item \textit{O:nęh gwaˀ} \trs{suddenly}, \trs{already}, \trs{finally}, \trs{all at once}, p. \pageref{p:[o:nęh gwaˀ]}\\
\item \textit{O:nęh to:hah} \trs{soon}, \trs{almost}, p. \pageref{p:o:nęh to:hah]}
\end{CayugaRelated}



\subsection*{\textbf{O:nęh e:ˀ} \trs{again} (emphatic)} \label{p:[o:nęh e:ˀ]}
\CayugaRemark{Particle group functioning as an “adverb of time” (\sectref{ch:’Adverbs’ of time}); emphatic; clause-initial.}

\ea
\label{ex:opart28}
\gll \textbf{O:nęh} \textbf{é:ˀ} Hiˀnoˀ haˀhó:diˀ dewęni̱hóksǫh shęh níwe: oshaísdaˀ, o:nęh gwaˀ ahá:nyoˀ.\\
now again Hiˀnoˀ he.threw.it lightening.bolts that where.it.was snake, now right.then he.died\\
\glt ‘Again and again Hi’No’ hurled lightening bolts at the monster snake. Finally the poisoner, the giant serpent, was killed.’ (\cite{carrier_legends_2013})
\z

\ea
\label{ex:opart29}
\gll \textbf{O:nęh} \textbf{e:ˀ}  toh hodáditsgo:t.\\
now again there he.sits.himself\\
\glt ‘Now again he has himself sitting there.’
\z

\ea
\label{ex:opart30}
\gll Agahdrǫˀs \textbf{o:nęh} \textbf{e:ˀ} ji trehs satsę:nǫ́:nih.\\
I’m.frightened now again too too.much you.are.happy\\
\glt ‘I am frightened now because you are too happy.’
\z

\ea
\label{ex:opart31}
\gll \textbf{O:nęh} \textbf{e:ˀ}  agriˀsdowá:neh.\\
now again I’m.noisy\\
\glt ‘I am loud, noisy again.’
\z

\ea
\label{ex:opart32}
\gll \textbf{O:nęh} \textbf{e:ˀ} i:s toh haˀségǫ̱he:k.\\
now again you that.one you.are.fighting.there\\
\glt ‘You are getting into someone else’s fight again, butting in.’
\z

\begin{CayugaRelated}
\item \textit{E:ˀ} \trs{again}, \trs{still}, p. \pageref{p:[e:ˀ]}\\
\item \textit{O:nęh, Neˀ o:nęh} \trs{now}, \trs{when}, \trs{then}, \trs{at this time}, p. \pageref{p:[o:nęh]}
\end{CayugaRelated}


\subsection*{\textbf{O:nęh giˀ hya:ˀ} \trs{good-bye}} \label{p:[o:nęh giˀ hya:ˀ]}
\CayugaRemark{Particle group used for leave-taking; free-standing expression.}

\ea
\label{ex:opart33}
\gll O:nęh giˀ hya:ˀ\\
now just before.anything.else\\
\glt ‘Good-bye.’
\z

\begin{CayugaRelated}
\item \textit{Giˀ} \trs{just}, p. \pageref{p:[giˀ]}\\
\item \textit{Hya:ˀ} \trs{first}, \trs{before anything else}, p. \pageref{p:[hya:ˀ]}\\
\item \textit{O:nęh, Neˀ o:nęh} \trs{now}, \trs{when}, \trs{then}, \trs{at this time}, p. \pageref{p:[o:nęh]}\\
\item Greetings, origins, and affiliations, p. \pageref{ch:greetings, origins, and affiliations}
\end{CayugaRelated}



\subsection*{\textbf{O:nęh gwaˀ, Nęh gwaˀ} \trs{suddenly}, \trs{already}, \trs{finally}, \trs{all at once}} \label{p:[o:nęh gwaˀ]}
\CayugaRemark{Particle group \textit{(o:)nęh gwaˀ} functions as an “adverb of time” (\sectref{ch:’Adverbs’ of time}); clause-initial.}

\ea
\label{ex:opart34}
\gll \textbf{O:nęh} \textbf{gwaˀ} edwawayę:nę́:daˀ.\\
now just.now we’re.finished\\
\glt ‘We’re finished already.’
\z

\ea
\label{ex:opart35}
\gll \textbf{Onę́h} \textbf{gwaˀ}, \textbf{nę́-gwaˀ} ahsha:kni:gę́ˀ neˀ hagehsotgęhę́:ˀ togyę́h da̱hayagę́ˀ,\\
now just.now now-just.now we.saw.him the our.late.grandfather there he.was.going.out\\
\glt ‘And all at once we saw my late grandfather, who was going out from there.’ (\cite{henry_de_2005})
\z

\ea
\label{ex:opart36}
\gll Gyotgǫ́:t toh hehéhtaˀ, heyohé: hę hne:ˀ i:sóˀ ahahnegéhaˀ. \textbf{O:nęh} \textbf{gwaˀ} ahoˀnigǫháędaˀ, aga:gwe:níˀ ahóyǫgyaˀt.\\
always that.one he.did.it.on.purpose, even and in.fact more he.drank.it now just.now he.understood, it.was.able.to he.smiled\\
\glt ‘Just like the squirrel, he kept going back for more. Then he discovered something that made him smile.’ (\cite{carrier_legends_2013})
\z

\ea
\label{ex:opart37}
\gll Honǫˀséh hada:kséˀ neˀ hohshę:. \textbf{O:nęh} \textbf{gwaˀ} a:yę́:ˀ ahę́dagraˀ.\\
he.was.lazy he.runs the he.is.fat now just.now it.seems he.fell\\
\glt ‘But the lazy hunter was getting tired of running and he suddenly pretended to fall.’ (\cite{carrier_legends_2013})
\z

\ea
\label{ex:opart38}
\gll \textbf{O:nęh} \textbf{gwaˀ} dó:gęhs ahę́he: hagę́hjiˀ.\\
now just.now truly he.conveyed.it.to.her this.old.one\\
\glt ‘Finally the Old One (HiˀNoˀ) gave her the news she wanted.’ (\cite{carrier_legends_2013})
\z

\ea
\label{ex:opart39}
\gll O:nęh é:ˀ Hiˀnoˀ haˀhó:diˀ dewęni̱hóksǫh shęh níwe: oshaísdaˀ, \textbf{o:nęh} \textbf{gwaˀ} ahá:nyoˀ.\\
now again Hiˀnoˀ he.threw.it lightening.bolts that where.it.was snake, now just.now he.died\\
\glt ‘Again and again Hi’No’ hurled lightening bolts at the monster snake. Finally the poisoner, the giant serpent, was killed.’ (\cite{carrier_legends_2013})
\z

\begin{CayugaRelated}
\item \textit{Gwaˀ} \trs{immediately}, \trs{right then}, \trs{just then}, \trs{finally}, p. \pageref{p:[gwaˀ] ‘immediately’}\\
\item \textit{O:nęh, Neˀ o:nęh} \trs{now}, \trs{when}, \trs{then}, \trs{at this time}, p. \pageref{p:[o:nęh]}
\end{CayugaRelated}


\subsection*{\textbf{O:nęh to:hah} \trs{soon}, \trs{almost}} \label{p:o:nęh to:hah]}
\CayugaRemark{Particle group \textit{(o:)nęh…to:hah} functions as an “adverb of time” (\sectref{ch:’Adverbs’ of time}); clause-initial.}

\ea
\label{ex:opart40}
\gll \textbf{Nę:} \textbf{toháˀ} ędwahdę́:diˀ.\\
now almost we.all.will.leave\\
\glt ‘We’ll soon be on our way.’ (\cite{keye_circle_2016}, Circle Book 11, On Our Way)
\z

\ea
\label{ex:opart41}
\gll \textbf{O:nę́h} \textbf{to:háh} toh hęshe:ˀ.\\
now almost there he.will.return.there\\
\glt ‘He is almost ready to return.’
\z

\ea
\label{ex:opart42}
\gll \textbf{O:nę́h} \textbf{to:háh} John ęhshodǫ̱hswéˀdęˀ.\\
Now almost John he.will.become.hungry.again.\\
\glt ‘John is almost going to get hungry again.’
\z

\ea
\label{ex:opart43}
\gll \textbf{O:nę́h} diˀ \textbf{to:háh} hęgahé:ˀ ędwę́ni̱hę:ˀ.\\
now so almost the.time.has.come we.will.quit\\
\glt ‘Now it is almost time for us to quit.’
\z

\ea
\label{ex:opart44}
\gll \textbf{O:nę́h} hné:ˀ \textbf{to:háh} ja:dáhk niyohwi̱hsdáˀe:.\\
now in.fact almost seven o’clock\\
\glt ‘It is now almost seven o’clock.’ (\cite[142]{mithun_watewayestanih_1984}, Satgęh dialogue)
\z

\ea
\label{ex:opart45}
\gll \textbf{O:nę́h} \textbf{to:háh} ęshá:yǫˀ.\\
now almost he.will.go.back\\
\glt ‘He is almost ready to go back.’
\z

\ea
\label{ex:opart46}
\gll \textbf{O:nę́h} giˀ \textbf{to:háh} hękní:yǫh.\\
now just almost we.will.arrive.there\\
\glt ‘We’ll be there soon.’ (\cite[481]{mithun_watewayestanih_1984}, Dędwaˀęnáęˀ dialogue)
\z

\begin{CayugaRelated}
\item \textit{O:nęh, Neˀ o:nęh} \trs{now}, \trs{when}, \trs{then}, \trs{at this time}, p. \pageref{p:[o:nęh]}\\
\item \textit{To:hah} \trs{a place}, \trs{a time}, p. \pageref{p:[to:hah]}
\end{CayugaRelated}


\subsection*{\textbf{Onǫ:hęˀ} \trs{they (animals)}} \label{p:[onǫ:hęˀ]}
\CayugaRemark{Verb functioning as an “emphatic pronoun” (\sectref{ch:’Emphatic pronouns’}); free-standing.}

\ea
\label{ex:opart47} \textbf{onǫ́:hęˀ}\\
\gll on-ǫ́:hęˀ\\
\textsc{3p.p}-alone.{\stative}\\
\glt ‘they (animals) are alone’
\z


\subsection*{\textbf{Otgaˀdeˀ neˀ niwa:gęˀ ni-} \trs{as many as}} \label{p:[otgaˀdeˀ neˀ niwa:gęˀ ni-]}
\CayugaRemark{Verb and particle group \stem{otgaˀdeˀ … neˀ … \textsc{number} (niwa:gęˀ ni-\textsc{stative.verb})} figures in comparisons (\sectref{ch:Comparisons (more, the same, or less)}).}

\ea
\label{ex:opart48}
\gll Ó: \textbf{otgáˀdeˀ} giˀ ni: gę́:s \textbf{neˀ} wa̱hshę: \textbf{niwa:gęˀ} nigahaní:yo:t neˀ swa̱hsǫ́:da:t\\
oh it.is.many just I usually the ten so.many where.they.were.hanging the one.night\\
\glt ‘I would often have as many as ten hoops hanging each night.’ (\cite{mithun_how_1980})
\z

\ea
\label{ex:opart49}
\gll Ó: \textbf{otgáˀdeˀ} hne:ˀ gę:s \textbf{neˀ} tgwáhaǫˀ hyeiˀ, ó: tgwaháǫˀ já:dahk.\\
oh it.is.many in.fact usually the sometimes six, oh sometimes seven\\
\glt ‘Oh sometimes I used to catch as many as six or seven.’ (\cite{mithun_how_1980})
\z

\begin{CayugaRelated}
\item \textit{Ę:, E:} (possible atypical verb), p. \pageref{p:[ę:, e:] `atypical verb’}
\end{CayugaRelated}


\subsection*{\textbf{Otgǫˀ, Hotgǫˀǫh} \trs{what the…?}, \trs{for Heaven’s sake!}} \label{p:[otgǫˀ]}
\CayugaRemark{Verb \textit{hotgǫ́ˀǫh} or \textit{otgǫˀ} functions as an “exclamation” (\sectref{ch:Exclamations}); said when something is out of the ordinary or not right. \textit{Otgǫˀ} literally means `it is strong or ominous medicine’.}

\ea
\label{ex:opart50}
\gll \textbf{Hotgǫˀǫ́h} ahsyǫˀ!\\
for.heaven’s.sake you.arrived\\
\glt ‘For heaven’s sake, you got here!’ `What the…?! You made it!’
\z

\ea
\label{ex:opart51}
\gll Jagwadatre:waht tęˀ de̱ˀagodigaęˀǫ:ˀ akíno̱haˀ. \textit{Otgǫˀ} to:háh seˀ tsaˀgá:t a:yǫgwadri̱hwahdęgyǫ́:k shęh ni:yóht ganǫ́hsesgeh.\\
we.are.sorry not they.wouldn’t.agree our.parents it.is.not.fair almost certainly the.same.thing we.should.do.a.ceremony that what.kind at.the.longhouse\\
\glt ‘We’re sorry – our parents would not agree. it isn’t fair – it would have been like the real thing -- a real ceremony.’ (\cite{carrier_legends_2013})
\z



\subsection*{\textbf{O:węh} \trs{it belongs to it (an animal)}, \trs{its}} \label{p:[o:węh]}
\CayugaRemark{Verb functioning as a “possessive pronoun” (\sectref{ch:’Possessive pronouns’}); free-standing.}

\ea
\label{ex:opart52} \textbf{ó:węh}\\
\gll ó:-w-ęh\\
\textsc{3s.p}-own-{\stative}\\
\glt ‘it owns it’
\z


\subsection*{\textbf{O:yaˀ} \trs{another}, \trs{other}, \trs{else}} \label{p:[o:yaˀ]}
\CayugaRemark{Basic noun functioning as an “indefinite pronoun” (\sectref{ch:’Indefinite pronouns’}); free-standing.}

\ea
\label{ex:opart53}
\gll Ohgeh \textbf{o:yaˀ} hniˀ gę:s niyagwayé:haˀ.\\
then other and usually the.way.we.do.it\\
\glt ‘There is another way we used to do it.’ (\cite{mithun_how_1980})
\z

\ea
\label{ex:opart54}
\gll Onę́h néˀ \textbf{oyaˀsǫ́ˀ} ǫgwéh gonadekǫ:níˀ hę́hne:.\\
now the others people they.are.eating also\\
\glt ‘Soon other people were eating too.’ (\cite{keye_circle_2016}, Circle Book 13, Rosie went to a feast of food)
\z

\begin{CayugaRelated}
\item \textit{Hejo:yaˀ tsǫ:} \trs{another thing again}, p. \pageref{p:[hejo:yaˀ tsǫ:] ‘another thing again’}\\
\item \textit{Hejo:yaˀ tsǫ:} \trs{elsewhere}, p. \pageref{p:[hejo:yaˀ tsǫ:] ‘elsewhere’}\\
\item \textit{Sga̱hoˀdę:ˀęh o:yaˀ} \trs{anything else}, p. \pageref{p:[sgahoˀdę:ˀęh o:yaˀ]}\\
\item \textit{Sǫ: gwaˀ nˀaht o:yaˀ} \trs{someone else}, p. \pageref{p:[sǫ: gwaˀ nˀaht o:yaˀ]}
\end{CayugaRelated}

\section{Ǫ particles}

\subsection*{\textbf{Ǫh, Ǫ:, Ǫ} \trs{I guess}, \trs{I wonder (if)}} \label{p:[ǫh, ǫ:, ǫ]}
\CayugaRemark{Particle group, emphatic, \textit{…ǫh}, \textit{ǫ:}, \textit{ǫ} functions as an “evidential marker” (\sectref{ch:Evidential markers}); enclitic; also pronounced as \textit{ˀǫh}, \textit{ˀǫ:}, \textit{ˀǫ}.}

\ea
\label{ex:onpart4}
\gll To \textbf{ˀǫ:} ní:yoht. \\
that I.wonder it.is.like.that\\
\glt ‘I wonder if it is like that.’
\z

\ea
\label{ex:onpart5}
\gll Agiˀda̱ˀǫ́h \textbf{ǫh} shęh na̱ˀonishéˀ hohta:ˀ. \\
I.slept I.guess that during he.spoke\\
\glt ‘I suppose I slept while he spoke.’
\z

\ea
\label{ex:onpart6}
\gll A:yę́:ˀ gwe:gǫ́ˀ \textbf{ǫ} swayáˀda:ˀ. Swatgaˀdéˀ giˀ gyę:ˀ. \\
it.seems all I.guess your.pictures you.are.many just this.one\\
\glt ‘It seems like you all got your picture taken. There are a lot of yous.’ (\cite[257]{mithun_watewayestanih_1984}, Ga̱hwajiyá:deˀ dialogue)
\z

\ea
\label{ex:onpart7}
\gll Ó:, hyeiˀshǫ́:ˀǫ áǫda̱hseˀ. Onęh \textbf{ˀǫ} ęyagokwai̱hséha:k. \\
O, six-ish you.should.come then. I.guess she.will.have.cooked.the.food\\
\glt ‘Oh, you should come around six-ish. She should have the food cooked by then.’ (\cite[279]{mithun_watewayestanih_1984}, Ǫdwęnǫdáhtaˀ dialogue)
\z

\ea
\label{ex:onpart8}
\gll ˀǫ́h, a:yę́:ˀ i:wí: odǫtga:déˀ \textbf{ǫh}. \\
oh it.seems I.think it.will.be.fun I.guess\\
\glt ‘Oh, it seems, I think, it will be fun.’ (\cite[402]{mithun_watewayestanih_1984}, Ęyagwa̱hdę́:diˀ dialogue)
\z


\ea
\label{ex:onpart9}
\gll A:yę́:ˀ agwę́h onęhę́ˀ osaheˀdáˀ ohǫna̱ˀdáˀ \textbf{ǫ} hniˀ, \\
it.seems we.have corn beans potatoes I.guess and\\
\glt ‘… It seems like we think corn, beans, and potatoes,’ (\cite[420]{mithun_watewayestanih_1984}, Ęswayętoˀ Gęh dialogue) (listing what will be planted)
\z

\ea
\label{ex:onpart10}
\gll I:wí: ahsę́h nigaya:gé:ˀ \textbf{ǫ} a:gagwé:niˀ. \\
I.think three an.amount.of.bags I.guess it.should.be.possible\\
\glt ‘I think about three bags should do it.’ (\cite[420]{mithun_watewayestanih_1984}, Ęswayętoˀ Gęh dialogue)
\z

\begin{CayugaRelated}
\item \textit{Dęˀ ǫh ne:ˀ hoˀdęˀ} \trs{what on earth?}, p. \pageref{p:[dęˀ ǫh hne:ˀ hoˀdęˀ]}\\
\item \textit{Gyę:ˀ ǫh} \trs{maybe}, \trs{I guess}, \trs{I wonder}, p. \pageref{p:[gyę:ˀ ǫh]}\\
\item \textit{Ne:ˀ giˀ gyę:ˀ ǫh} \trs{I guess}, p. \pageref{p:[ne:ˀ giˀ gyę:ˀ ǫh]}\\
\item \textit{Ne:ˀ ǫh} \trs{I guess it is}, p. \pageref{p:[ne:ˀ ǫh]}\\
\item \textit{Oˀǫ:, o: ǫh} \trs{oh really?}, p. \pageref{p:[o ˀǫ:, o: ǫh]}\\
\item \textit{O: tęˀ ǫh ga:t giˀ shęh gyę:gwaˀ} \trs{maybe, maybe not}, p. \pageref{p:[o: tęˀ ǫh ga:t giˀ shęh gyę:gwaˀ]}\\
\item \textit{O: tęˀ ǫh ga:t giˀ shęh ne:ˀ hwaˀ} \trs{maybe, maybe not}, p. \pageref{p:[o: tęˀ ǫh ga:t giˀ shęh ne:ˀ hwaˀ]}\\
\item \textit{Oh ne:ˀ} \trs{maybe}, p. \pageref{p:[ǫh ne:ˀ]}\\
\item \textit{Tęˀ gyę:ˀ ǫh, tęˀ gęˀǫh} \trs{not really}, p. \pageref{p:[tęˀ gyę:ˀ ǫh]}\\
\item \textit{To gęh ǫ ne:ˀ} \trs{I wonder if it is}, p. \pageref{p:[to gęh ǫ ne:ˀ]}
\end{CayugaRelated}

\subsection*{\textbf{Ǫh ne:ˀ} \trs{maybe}} \label{p:[ǫh ne:ˀ]}
\CayugaRemark{Particle group \textit{ǫh ne:ˀ verb} functions as an “evidential marker” (\sectref{ch:Evidential markers}); enclitic.}

\ea
\label{ex:onpart1}
\gll Mary \textbf{ǫh} \textbf{ne:ˀ} gohwihsdagaˀdeˀ.\\
Mary I.wonder it.is she.has.a.lot.of.money\\
\glt ‘Maybe Mary has a lot of money.’
\z


\ea
\label{ex:onpart2}
\gll “Dęˀ \textbf{ǫh} \textbf{ne:ˀ} na:ya:wę́h,” aˀa:gę́ˀ, “e:gyadeˀsgóh nę:gyę́h ohnáwaot?” \\
what I.wonder it.is it.would.happen she.said we.should.immerse.ourselves this well \\
\glt ‘“What would really happen,” she said, “if we got into the well?”’ (\cite{henry_de_2005})
\z

\ea
\label{ex:onpart3}
\gll Akeyatgahtóˀ, agíˀ, “Tęˀ \textbf{ǫh} \textbf{ne:ˀ} da̱ˀǫ́ a:wá:dǫˀ.” \\
I.looked.at.her I.said not I.guess it.is definitely.not it.cannot.be.done\\
\glt ‘I just looked at her. I said, “No we cannot do that.”’ (\cite{henry_de_2005})
\z

\begin{CayugaRelated}
\item \textit{Ne:ˀ} \trs{it is}, p. \pageref{p:[ne:ˀ] `it is’}\\
\item \textit{Ne:ˀ ǫh} \trs{I guess it is}, p. \pageref{p:[ne:ˀ ǫh]}\\
\item \textit{Oh, ǫ:, ǫ} \trs{I guess}, \trs{I wonder (if)}, p. \pageref{p:[ǫh, ǫ:, ǫ]}
\end{CayugaRelated}


\subsection*{\textbf{Ǫ:weh, Neˀ ǫ:weh} \trs{really}} \label{p:[ǫ:weh, neˀ ǫ:weh]}
\CayugaRemark{Particle group functioning as an “adverb of degree” (\sectref{ch:’Adverbs’ of degree}); clause-initial.}

\ea
\label{ex:onpart12}
\gll \textbf{Neˀ} \textbf{ǫ:weh} gaǫdé:nǫhk. \\
the truly they.are.related\\
\glt ‘They really are related.’
\z

\begin{CayugaRelated}
\item \textit{Gwahs ǫ:weh} \trs{really}, p. \pageref{p:[gwahs ǫ:weh]}\\
\item \textit{Gwahs ǫ:węh seˀ} \trs{indeed}, \trs{for sure}, p. \pageref{p:[gwahs ǫ:węh seˀ]}\\
\item \textit{Shęh ǫ:weh} \trs{it is really}, p. \pageref{p:[shęh ǫ:weh] ‘it is really’}
\end{CayugaRelated}


\section{S particles}

\subsection*{\textbf{Sah} \trs{you know}} \label{p:[sah]}
\CayugaRemark{Possibly related to \textit{seˀ} `you know’. Appears in at least one particle combination. Also, possibly related to Tuscarora \textit{saˀ} ‘look!’ or ‘see!’ (\cite{rudes_tuscarora-englishenglish-tuscarora_1999}).}

\begin{CayugaRelated}
\item \textit{Haoˀ diˀ sah} \trs{alright}, \trs{o.k.}, p. \pageref{p:[haoˀ diˀ sah]}\\
\item \textit{Seˀ} \trs{you know}, p. \pageref{p:[seˀ]}
\end{CayugaRelated}

\subsection*{\textbf{Sa:węh} \trs{it's yours (one person’s)}, \trs{your}} \label{p:[sa:węh]}
\CayugaRemark{Verb functioning as a “possessive pronoun” (\sectref{ch:’Possessive pronouns’}); free-standing.}

\ea
\label{ex:spart1}
\glll \textbf{sá:węh} {}\\
s-á:w-ęh\\
\textsc{2s.p}-own-{\stative}\\
\glt ‘you own it’
\z


\subsection*{\textbf{Seˀ} \trs{you know}} \label{p:[seˀ]}
\CayugaRemark{Particle signaling shared knowledge (known to both speaker and listener, \sectref{ch:Shared knowledge markers}). }

\begin{CayugaRelated}
\item \textit{Deˀę: gwaˀ-heh tęˀ seˀ} \trs{but then not really}, p. \pageref{p:[deˀę: gwaˀ-heh tęˀ seˀ]}\\
\item \textit{Ehęˀ seˀ} \trs{it is so}, \trs{yes indeed}, p. \pageref{p:[ęhęˀ seˀ]}\\
\item \textit{Gwahs ǫ:węh seˀ} \trs{indeed}, \trs{for sure}, p. \pageref{p:[gwahs ǫ:węh seˀ]}\\
\item \textit{I:ˀ seˀ} \trs{I am}, p. \pageref{p:[i:ˀ seˀ]}\\
\item \textit{Neˀ seˀ} \trs{that’s just the one}, \trs{that’s just who}, p. \pageref{p:[neˀ seˀ]}\\
\item \textit{Ne:ˀ seˀ gę:s neˀ} \trs{back then}, \trs{a long time ago}, p. \pageref{p:[ne:ˀ seˀ gę:s neˀ]}\\
\item \textit{Ne:ˀ seˀ gyę:ˀ} \trs{you know}, p. \pageref{p:[ne:ˀ seˀ gyę:ˀ]}\\
\item \textit{Sah} \trs{you know}, p. \pageref{p:[sah]}\\
\item \textit{Seˀ hęˀ ni:ˀ} \trs{me too}, \trs{us too}, p. \pageref{p:[seˀ hęˀ ni:ˀ]}\\
\item \textit{Tęˀ seˀ} \trs{not really}, \trs{but then not really}, p. \pageref{p:[tęˀ seˀ]}
\end{CayugaRelated}

\subsection*{\textbf{Seˀ hęˀ ni:ˀ} \trs{me too}, \trs{us too}} \label{p:[seˀ hęˀ ni:ˀ]}
\CayugaRemark{Particle group \textit{… (seˀ) hęˀ ni:ˀ} functions as an emphatic pronoun (\sectref{ch:’Emphatic pronouns’}); enclitic.}

\ea
\label{ex:spart2}
\gll A:ga̱hyagwa̱hsé:k \textbf{seˀ} \textbf{hęˀ} \textbf{ni:ˀ} gyę:gwáˀ a:wagadagáide:k.\\
I.would.pick.fruit you.know also the.me if I.would.be.well\\
\glt ‘I would be a fruit picker too if I were well.’
\z

\ea
\label{ex:spart3}
\gll Agahyagwęhné:ˀ \textbf{seˀ} \textbf{hęˀ} \textbf{ni:ˀ}.\\
I’ve.picked.fruit you.know also the.me\\
\glt ‘I’ve also picked fruit.’ (I’ve experienced this)
\z

\begin{CayugaRelated}
\item \textit{Hęˀ} \trs{also}, \trs{too}, p. \pageref{p:[hęˀ]}\\
\item \textit{Hęˀ ni:ˀ} \trs{me too}, \trs{us too}, p. \pageref{p:[hęˀ ni:ˀ]}\\
\item \textit{I:ˀ, Ni:ˀ}, \trs{I}, \trs{we}, p. \pageref{p:[i:ˀ]}\\
\item \textit{Seˀ} \trs{you know}, p. \pageref{p:[seˀ]}
\end{CayugaRelated}

\subsection*{\textbf{Senyęˀsgwadih} \trs{to your left}} \label{p:[senyęˀsgwadih]}
\CayugaRemark{Atypical verb functioning as an “adverb of place” (\sectref{‘Adverbs’ of place}); clause-initial.}


\ea
\label{ex:spart400}
\glll \textbf{senyę́ˀsgwadih} {}\\
se-nyęˀs-gwadih\\
\textsc{\repetitive-2s.a-noun}-side.{\stative}\\
\glt ‘to your left’
\z

\ea
\label{ex:spart4}
\gll \textbf{Senyę́ˀsgwadíh} goyaˀdanędagǫ́h ne:ˀ hoksaˀgowáhsraˀ.\\
to.your.left.side she.is.clinging.on it.is his.beautiful.one\\
\glt ‘To your left stands his beautiful, dearly beloved woman.’
\z

\begin{CayugaRelated}
\item \textit{Gwa:dih, gwai} \trs{to one side}, p. \pageref{p:[gwa:dih]}\\
\item \textit{He:yo: dagwaishǫ:} \trs{straight ahead}, p. \pageref{p:[he:yo: dagwaishǫ:]}\\
\item \textit{Sewaihǫhsdǫh gwa:dih} \trs{to your right}, p. \pageref{p:[sewaihǫhsdǫh gwa:dih]}
\end{CayugaRelated}

\subsection*{\textbf{Sewaihǫhsdǫh gwa:dih} \trs{to your right}} \label{p:[sewaihǫhsdǫh gwa:dih]}
\CayugaRemark{Atypical verb and particle, \stem{waihǫhsdǫh gwa:dih} functions as an “adverb of place” (\sectref{‘Adverbs’ of place}); clause-initial; \textit{gwa:dih} can also be spelled or pronounced as \textit{gwai}. \stem{waihǫhsdǫh} is conjugated with \textsc{a}-series pronominal prefixes.} 

\ea
\label{ex:spart5}
\gll \textbf{Sewaihǫhsdǫ́h} \textbf{gwaí /gwa:díh} hęsatahá:goˀ.\\
your.right side you.will.get.off\\
\glt ‘You will get off on the right side of the road.’
\z

\begin{CayugaRelated}
\item \textit{Gwa:dih, gwai} \trs{to one side}, p. \pageref{p:[gwa:dih]}\\
\item \textit{He:yo: dagwaishǫ:} \trs{straight ahead}, p. \pageref{p:[he:yo: dagwaishǫ:]}\\
\item \textit{Senyęˀsgwadih} \trs{to your left}, p. \pageref{p:[senyęˀsgwadih]}
\end{CayugaRelated}

\subsection*{\textbf{Sgahoˀdę:ˀęh} \trs{something}, \trs{anything}} \label{p:[sgahoˀdę:ˀęh]}
\CayugaRemark{Particle functioning as an “indefinite pronoun” (\sectref{ch:’Indefinite pronouns’}); free-standing.}

\ea
\label{ex:spart6}
\gll Hnaˀgǫ̱hká:ˀ gi̱ˀ-shę́h \textbf{sgahoˀdę:ˀęh}?\\
Undies or something\\
\glt ‘Underclothes or something?’ (saleslady suggesting something to buy)(\cite[225]{mithun_watewayestanih_1984}, Agyaˀdawíˀtraˀ dialogue)
\z

\ea
\label{ex:spart7}
\gll A:sna̱ˀdaǫníˀ giˀ gyę́:ˀ, \textbf{sgahoˀdę́:ˀęh} neˀ waˀjíh ędwa:k.\\
you.should.bake just emphasis, something it presently we.will.eat\\
\glt ‘You should bake something for us all to eat later on.’ (\cite[225]{mithun_watewayestanih_1984}, Dwęnǫhsanékahǫˀ dialogue)
\z

\ea
\label{ex:spart8}
\gll A:yę́:ˀ \textbf{sgahoˀdę:ˀę́h} snegé:haˀ.\\
it.seems something you.are.drinking\\
\glt ‘It looks like you are drinking something.’ (\cite{carrier_legends_2013})
\z

\begin{CayugaRelated}
\item \textit{Hoˀdęˀ} \trs{kind}, p. \pageref{p:[hoˀdęˀ]}\\
\item \textit{Sga̱hoˀdę:ˀęh o:yaˀ} \trs{anything else}, p. \pageref{p:[sgahoˀdę:ˀęh o:yaˀ]}
\end{CayugaRelated}

\subsection*{\textbf{Sgahoˀdę:ˀęh o:yaˀ} \trs{anything else}} \label{p:[sgahoˀdę:ˀęh o:yaˀ]}
\CayugaRemark{Particle and noun, \textit{sga̱hoˀdę:ˀęh … o:yaˀ} functions as an “indefinite pronoun” (\sectref{ch:’Indefinite pronouns’}); free-standing.}

\ea
\label{ex:spart9}
\gll Tę́ˀ gęh taˀdesadǫhwęjo:níh \textbf{sgahoˀdę:ˀę́h} \textbf{o:yaˀ}? Adáhdi̱ˀtraˀ ahdahgwáˀ gi̱-shęh?.\\
not Q any.old.thing.you.want anything other? socks shoes maybe?\\
\glt ‘There was not anything else you wanted? Stockings, or shoes?’ (\cite[225]{mithun_watewayestanih_1984}, Agyaˀdawíˀtraˀ dialogue)
\z

\ea
\label{ex:spart10}
\gll \textbf{Sgahoˀdę:ˀę́h} diˀ gęh \textbf{o:yáˀ} sanáhsgwaęˀ?\\
anything so Q other you.have.pets\\
\glt ‘Do you have any other pets?’ (\cite[317]{mithun_watewayestanih_1984}, Sanahsgwaęˀ gęh sga̱hoˀdę́:ˀęh? dialogue)
\z

\begin{CayugaRelated}
\item \textit{Hoˀdęˀ} \trs{kind}, p. \pageref{p:[hoˀdęˀ]}\\
\item \textit{O:yaˀ} \trs{another}, \trs{other}, \trs{else}, p. \pageref{p:[o:yaˀ]}\\
\item \textit{Sga̱hoˀdę:ˀęh} \trs{something}, p. \pageref{p:[sgahoˀdę:ˀęh]}
\end{CayugaRelated}

\subsection*{\textbf{Sganyęˀsgwadih} \trs{to its left}} \label{p:[sganyęˀsgwadih]}
\CayugaRemark{Atypical verb functioning as an “adverb of place” (\sectref{‘Adverbs’ of place}); clause-initial.}

\ea
\label{ex:spart401}
\glll \textbf{sganyęˀsgwadih} {}\\
s-ga-nyęˀs-gwadih\\
\textsc{\repetitive-3s.a-noun}-side.{\stative}\\
\glt ‘to its left’
\z

\begin{CayugaRelated}
\item \textit{Gwa:dih, gwai} \trs{to one side}, p. \pageref{p:[gwa:dih]}\\
\item \textit{He:yo: dagwaishǫ:} \trs{straight ahead}, p. \pageref{p:[he:yo: dagwaishǫ:]}\\
\item \textit{Sewaihǫhsdǫh gwa:dih} \trs{to your right}, p. \pageref{p:[sewaihǫhsdǫh gwa:dih]}
\end{CayugaRelated}

\subsection*{\textbf{Shęh} \trs{that}} \label{p:[shęh] `that’}
\CayugaRemark{Particle optionally introducing a dependent clause \xxref{ex:spart11}{ex:spart14}, a dependent clause functioning as a “noun” \xxref{ex:spart15}{ex:spart24}, or a relative clause \xref{ex:spart25}, see (\sectref{ch:Clauses with [shęh] ‘that’, [shęh ni-] ‘how, what’, [ne:ˀ] ‘it is’}); clause-initial; also pronounced as \textit{tsęh}.} 

\ea
\label{ex:spart11}
\gll Degaˀdre̱hdaˀge̱ha:gyéˀ [\textbf{shęh} ahę:nǫ́:da:ˀ].\\
two.cars.at.a.time [that they.put.in]\\
\glt ‘[They put in] the cars two at a time.’
\z

\ea
\label{ex:spart12}
\gll Gwi:déh ahę́ˀ [Tina go:gáˀs nawę́ˀdaˀ].\\
Peter he.said [Tina she.likes.the.taste.of.it candy]\\
\glt ‘Peter said [Tina likes sugar/candy].’ (This sentence shows that \textbf{shęh} is optional.)
\z

\ea
\label{ex:spart13}
\gll Gwi:déh honǫhdǫ́ˀ [\textbf{shęh} Tina nawęˀdáˀ gó:gaˀs].\\
Peter he.knows [that Tina sugar/candy she.likes.the.taste.of.it]\\
\glt ‘Peter knows that Tina likes candy.’
\z

\ea
\label{ex:spart14}
\gll Gwi:déh ahę́ˀ [Mary ǫ́:dǫh], [go:gáˀs nawę́ˀdaˀ]].\\
Peter he.said [Mary she.says] [she.likes.the.taste.of.it sugar/candy]\\
\glt ‘Peter said Mary says she likes sugar/candy.’ (This sentence illustrates that \textbf{shęh} is optional.)
\z

\ea
\label{ex:spart15}
\gll A:yę́:ˀ do:gę́hs ǫ: odǫtga:déˀ [\textbf{shęh} gá:gǫnheˀ].\\
it.seems truly I.guess it.is.enjoyable [how they.live]\\
\glt ‘It sure seems enjoyable, [the way they live].’ (\cite[88]{mithun_watewayestanih_1984}, Eksaˀgó:wah dialogue)
\z

\ea
\label{ex:spart16}
\gll \textbf{shęh} \textbf{ni}yó:sdeˀ\\
that how.heavy.it.is\\
\glt ‘mass’
\z

\ea
\label{ex:spart17}
\gll \textbf{shęh} \textbf{ni}gá:dę:s\\
that how.thick.it.is\\
\glt ‘volume’, `density’, `mass’, `how thick it is’
\z

\ea
\label{ex:spart18}
\gll \textbf{shęh} \textbf{ni}yóyade:s\\
that how.deep.it.is\\
\glt ‘a hole’
\z

\ea
\label{ex:spart19}
\gll \textbf{shęh} \textbf{ni}wagadeˀdróˀdę:\\
that how.I.feel\\
\glt ‘how I feel about something, someone’
\z

\ea
\label{ex:spart20}
\gll [\textbf{shęh} \textbf{ni}yo:wéh] taná:greˀ\\
[that it.is.a.certain.distance] he.lives.there\\
\glt ‘He lives [that far (away)].’
\z

\ea
\label{ex:spart21}
\gll Sekdǫ́: [\textbf{shęh} \textbf{ni}yoga̱ˀǫ́hsro̱ˀdę:]\\
try.it [that what.kind.of.taste.it.has]\\
\glt ‘Taste it!’ ‘Try [what it tastes like]!’
\z

\ea
\label{ex:spart22}
\gll Ohsga:náht [\textbf{shęh} \textbf{ni}hahnatsí:yo:].\\
it.is.attractive [that what.a.nice.butt.he.has]\\
\glt ‘He has [an attractive butt].’
\z

\ea
\label{ex:spart23}
\gll Dękde:níˀ [\textbf{shęh} \textbf{ni}wagri̱hóˀdę:].\\
I.will.change.it [that what.my.kind.of.matter.is]\\
\glt ‘I am going to change [my outlook].’
 \z

\ea
\label{ex:spart24}
\gll Aǫgo̱hdǫ́h ǫgeˀdra̱héhs [\textbf{shęh} \textbf{n}agadekǫ́:niˀ].\\
it.is.above.average I.exceeded [that what.I.ate]\\
\glt ‘I ate too much.’
\z

\ea
\label{ex:spart25}
\gll O:nę́h hniˀ ohsǫdagwe:gǫ́h de̱hodinęhę́:ˀ \textbf{neˀ} \textbf{hęnǫ:gwéh} [\textbf{shęh} \textbf{ená:greˀ}]\\
now and all.night they.guard the men [that someone.lives.there]\\
\glt ‘Now too all night they are guarding, the men [who live there].’ (\cite{keye_hnyagwaidatgigowah_2012}) (The bolded material constitutes a noun phrase that includes a relative clause. The relative clause is shown in square brackets `[ ]’).
\z


\subsection*{\textbf{Shęh} \trs{because}} \label{p:[shęh] `because’}
\CayugaRemark{Particle, clause-initial in causative dependent clauses (\sectref{ch:Causative clauses with [dęˀ ni:yoht shęh] ‘why’, [neˀ hǫ:niˀ] ‘how’}). Also pronounced as \textit{tsęh}.} 

\ea
\label{ex:spart26}
\gll De̱hęnadatnǫhwéˀs Gwi:déh Ed hniˀ [\textbf{shęh} hona:dáoˀ]\\
They.like.each.other Peter Ed and that they.are.friends\\
\glt ‘Peter and Ed like each other because they are friends.’
\z

\ea
\label{ex:spart27}
\gll Agyǫˀnigǫho:gǫ́ˀ [\textbf{shęh} ǫdowishę́ˀ goná:jih].\\
her.mind.is.scattered that she.died her.friend\\
\glt ‘She is broken-hearted because her friend died.’
\z

\begin{CayugaRelated}
\item \textit{Nigę́ˀǫh} \trs{or?}, \textit{Giˀ shęh} \trs{or}, etc., p. \pageref{ch:[nigę́ˀǫh] ‘or?’, [giˀ shęh] ‘or’, etc}\\
\item \textit{Dęˀ ni:yoht shęh} \trs{why?}, p. \pageref{p:[dęˀ ni:yoht shęh]}\\
\item \textit{Ewa:dǫˀ giˀ shęh} \trs{maybe}, \trs{a possibility}, p. \pageref{p:[ęwa:dǫˀ giˀ shęh]}\\
\item \textit{Gaoˀ shęh niyo:weˀ} \trs{before}, \trs{until}, p. \pageref{p:[gaoˀ shęh niyo:weˀ]}\\
\item \textit{Ga:t giˀ shęh} \trs{maybe}, p. \pageref{p:[ga:t giˀ shęh]}\\
\item \textit{Ga:t giˀ shęh tęˀ} \trs{maybe not}, \trs{or not}, p. \pageref{p:[ga:t giˀ shęh tęˀ]}\\
\item \textit{Gęh ga:t giˀ shęh nigęˀǫh} \trs{or?}, p. \pageref{p:[gęh ga:t giˀ shęh nigęˀǫh]}\\
\item \textit{Gwahs shęh ni-} \trs{as…as}, p. \pageref{p:[gwahs shęh ni-]}\\
\item \textit{Gyę:gwaˀ giˀ shęh hwaˀ} \trs{maybe this time}, p. \pageref{p:[gyę:gwaˀ giˀ shęh hwaˀ]}\\
\item \textit{Gyę:gwaˀ giˀ shęh} \trs{or maybe}, p. \pageref{p:[gyę:gwaˀ giˀ shęh]}\\
\item \textit{Heyohe:ˀ shęh} \trs{more than}, p. \pageref{p:[heyohe:ˀ shęh]}\\
\item \textit{Hę:-gyęh shęh} \trs{no matter how much}, \trs{whether or not}, \trs{even if}, p. \pageref{p:[hę:-gyęh shęh]}\\
\item \textit{Hne:ˀ shęh} \trs{because}, p. \pageref{p:[hne:ˀ shęh]}\\
\item \textit{Ji trehs shęh} \trs{too much so for}, \trs{so much so}, p. \pageref{p:[ji trehs shęh]}\\
\item \textit{Ne:ˀ giˀ shęh hwaˀ} \trs{maybe this time}, p. \pageref{p:[ne:ˀ giˀ shęh hwaˀ]}\\
\item \textit{Ne:ˀ giˀ shęh neˀ … giˀ shęh} \trs{or}, p. \pageref{p:[ne:ˀ giˀ shęh neˀ … giˀ shęh]}\\
\item \textit{Ne:ˀ tsǫ: shęh} \trs{but, it’s just that…}, p. \pageref{p:[ne:ˀ tsǫ: shęh]}\\
\item \textit{O: tęˀ ǫh ga:t giˀ shęh gyę:gwaˀ} \trs{maybe, maybe not}, p. \pageref{p:[o: tęˀ ǫh ga:t giˀ shęh gyę:gwaˀ]}\\
\item \textit{O: tęˀ ǫh ga:t giˀ shęh ne:ˀ hwaˀ} \trs{maybe, maybe not}, p. \pageref{p:[o: tęˀ ǫh ga:t giˀ shęh ne:ˀ hwaˀ]}\\
\item \textit{Ohę:dǫ: shęh} \trs{ahead}, \trs{in front}, \trs{forward}, p. \pageref{p:[ohę:dǫ: shęh]}\\
\item \textit{Ohnaˀgę: gwa:dih shęh} \trs{behind}, p. \pageref{p:[ohnaˀgę: gwa:dih shęh]}\\
\item \textit{Shęh} \trs{that}, p. \pageref{p:[shęh] `that’}\\
\item \textit{Shęh hǫ: heyodokdaˀǫh} \trs{the bottom}, p. \pageref{p:[shęh hǫ: heyodokdaˀǫh]}\\
\item \textit{Shęh hoˀdęˀ} \trs{a certain something}, \trs{that which}, p. \pageref{p:[shęh hoˀdęˀ]}\\
\item \textit{Shęh hǫ:weh} \trs{the place where}, ‘whereabouts’, p. \pageref{p:[shęh hǫ:weh]}\\
\item \textit{shęh naˀonisheˀ, tsaˀonisheˀ}, \stem{tsi-} \trs{while}, \trs{when}, p. \pageref{p:[shęh naˀonisheˀ]}\\
\item \textit{Shęh ni:yoht ni-} ‘how’, ‘the manner in which’, p. \pageref{p:[shęh ni-, shęh ni:yoht ni-]}\\
\item \textit{Shęh niyo:weˀ} \trs{as far as}, \pageref{p:[shęh niyo:weˀ, gaoˀ shęh niyo:weˀ]}\\
\item \textit{Shęh nohgeh} \trs{even}, p. \pageref{p:[shęh nohgeh]}\\
\item \textit{Tęˀ giˀ shęh hwaˀ daˀǫ} \trs{maybe not}, p. \pageref{p:[tęˀ giˀ shęh hwaˀ daˀǫ]}\\
\item \textit{To hę:gyeh shęh} \trs{really}, p. \pageref{p:[to hę:gyeh shęh]}\\
\item \textit{Toh shęh ni-} ‘to that degree’, p. \pageref{p:[toh shęh ni-]}
\end{CayugaRelated}

\subsection*{\textbf{Shęh hoˀdęˀ} \trs{a certain something}, \trs{that which}} \label{p:[shęh hoˀdęˀ]}
\CayugaRemark{Particle group functioning as a “definite pronoun” \sectref{ch:’Definite pronouns’}; free-standing.}

\ea
\label{ex:spart28}
\gll Gado:gę́: \textbf{shęh} \textbf{hoˀdęˀ} aˀǫgwayǫ́dahk.\\
together that kind we.all.came.together\\
\glt ‘A certain something that brought us together.’ (a meeting, etc.)
\z

\begin{CayugaRelated}
\item \textit{Hoˀdęˀ} \trs{kind}, p. \pageref{p:[hoˀdęˀ]}\\
\item \textit{Shęh} \trs{that}, \trs{because}, p. \pageref{p:[shęh] `because’}
\end{CayugaRelated}

\subsection*{\textbf{Shęh hǫ: heyodokdaˀǫh} \trs{the bottom}} \label{p:[shęh hǫ: heyodokdaˀǫh]}
\CayugaRemark{Particle group and verb \textit{shęh hǫ: heyodokdáˀǫh} functions as an “adverb of place” (\sectref{‘Adverbs’ of place}); likely clause-initial; \textit{hǫ:weh} is spelled variously as \textit{hǫ:}, \textit{nhǫ:}, \textit{hǫ:weh}, or \textit{nhǫ:weh}.}

\ea
\label{ex:shehhoheyo}
\gll shęh hǫ: heyodokdaˀǫh\\
that place the.end\\
\glt ‘the bottom’
\z


\begin{CayugaRelated}
\item \textit{Hǫ:weh} \trs{where}, p. \pageref{p:[hǫ:weh]}\\
\item \textit{Shęh} \trs{that}, \trs{because}, p. \pageref{p:[shęh] `because’}
\end{CayugaRelated}

\subsection*{\textbf{Shęh hǫ:weh} \trs{the place where}, \trs{whereabouts}} \label{p:[shęh hǫ:weh]}
\CayugaRemark{Particle group functioning as an “adverb of place” (\sectref{‘Adverbs’ of place}), clause-initial in dependent clauses; \textit{hǫ:weh} is spelled variously as \textit{hǫ:}, \textit{nhǫ:}, \textit{hǫ:weh}, or \textit{nhǫ:weh}.}

\ea
\label{ex:spart29}
\gll Knigǫhá:ˀ [\textbf{shęh} \textbf{hǫ:wéh} dęgátahahk].\\
I.watch that where I.will.walk\\
\glt ‘I watch where I walk.’
\z

\ea
\label{ex:spart30}
\gll Hętsyę́:ˀ [\textbf{shęh} \textbf{hǫ:} hesá:gwęh].\\
you.will.put.it.back.there that where you.picked.it.up.there\\
\glt ‘You will put it back where you got it.’
\z

\ea
\label{ex:spart31}
\gll [\textbf{shęh} \textbf{hǫ:wéh} nito:né:nǫ:]\\
that where they.come.from.a.certain.place\\
\glt ‘where they come from’
\z

\largerpage
\begin{CayugaRelated}
\item Clauses with \textit{hǫ:(weh)} \trs{the place where}, p. \pageref{ch:Clauses with [hǫ:(weh)] ‘the place where’}\\
\item \textit{Hǫ:weh} \trs{where}, p. \pageref{p:[hǫ:weh]}\\
\item \textit{Shęh} \trs{that}, \trs{because}, p. \pageref{p:[shęh] `because’}\\
\item \textit{Shęh hǫ: heyodokdaˀǫh} \trs{the bottom}, p. \pageref{p:[shęh hǫ: heyodokdaˀǫh]}
\end{CayugaRelated}


\subsection*{\textbf{Shęh naˀonisheˀ, Tsaˀonisheˀ, Tsi-} \trs{while}, \trs{when}} \label{p:[shęh naˀonisheˀ]}
\CayugaRemark{Particle and verb beginning with \stem{ni-} {\partitive} or \stem{tsi-} \textsc{\coincident}, functioning as an “adverb of time” (\sectref{ch:’Adverbs’ of time}, \sectref{ch:Clauses with [shęh naˀonisheˀ], [tsaˀonisheˀ], [tsi-] ‘while, when’}); clause-initial in dependent clauses.}

\ea
\label{ex:spart32}
\gll Agiˀda̱ˀǫ́h ǫh [\textbf{shęh} \textbf{naˀonishéˀ} hohta:ˀ].\\
I.slept I.guess that it.is.a.certain.time he.is.speaking\\
\glt ‘I slept while he spoke.’
\z

\ea
\label{ex:spart33}
\gll Hohsę́: [\textbf{tsaˀonihseˀ} hǫgwéˀdase:].\\
he.was.fat while he.is.a.young.man\\
\glt ‘While he was young, he was fat.’
\z

\ea
\label{ex:spart34}
\gll [\textbf{Tsi}gǫgwe̱ˀda:sé:] gę́:s ga̱hyagwa̱hsgę́hę:ˀ.\\
when.I.was.young usually I.would.pick.berries\\
\glt ‘When I was a young person I usually picked berries.’
\z

\ea
\label{ex:spart35}
\gll Knǫhwéˀs gę:s [\textbf{tsi}gǫgwe̱ˀda:séˀ] tiga:gwe:gǫ́h gę́:s agahyagwęhęgyéˀsgę̱hę:ˀ\\
I.love.it usually when.I.was.young all.over usually I.would.go.along.picking.fruit\\
\glt ‘When I was young, I used to love going all over and picking fruit.’
\z

\begin{CayugaRelated}
\item \textit{Shęh} \trs{that}, \trs{because}, p. \pageref{p:[shęh] `because’}
\end{CayugaRelated}

\subsection*{\textbf{Shęh ni-, Gwahs shęh ni-} \trs{as…as}} \label{p:[shęh ni-], [gwahs shęh ni-]}
\CayugaRemark{Particle group \stem{(gwahs) shęh ni-\textsc{stative.verb}} is used in comparisons (\sectref{ch:Comparisons (more, the same, or less)}); it precedes a stative verb beginning with \stem{ni-} {\partitive}, and is clause-initial in dependent clauses. }

\begin{CayugaRelated}
\item \textit{Gwahs} \trs{anyway}, p. \pageref{p:[gwahs] ‘anyway’}\\
\item \textit{Gwahs shęh ni-} \trs{as…as}, p. \pageref{p:[gwahs shęh ni-]}\\
\item \textit{Shęh} \trs{that}, \trs{because}, p. \pageref{p:[shęh] `that’}
\end{CayugaRelated}

\subsection*{\textbf{Shęh ni-, Shęh ni:yoht ni-} \trs{how}, \trs{the manner in which}} \label{p:[shęh ni-, shęh ni:yoht ni-]}
\CayugaRemark{Particle and verb \stem{(shęh) ni-, shęh ni:yoht ni-} is clause-initial in a manner clause (\sectref{ch:Manner clauses with [dęˀ ni-] ‘how’, [shęh ni:yoht] ‘how so’}).}

\ea
\label{ex:spart36}
\gll Knigǫhá:ˀ [\textbf{shęh} \textbf{ni:}yóht dęgátahahk].\\
I.watch that which.way I.will.walk\\
\glt ‘I watch how I walk.’
\cfex{ni:-yó-ht \textsc{\partitive-3s.p}-resemble.{\stative}}

\z

\ea
\label{ex:spart37}
\gll \textbf{shęh} \textbf{n}a̱ˀá:węh\\
that how.it.happened\\
\glt ‘how it happened’
\cfex{n-a̱ˀ-á:-w-ęh \textsc{\partitive-\factual-3s.p}-happen-{\stative}}
\z

\ea
\label{ex:spart38}
\gll \textbf{shęh} \textbf{ni}yóyanreˀ\\
that how.good.it.is\\
\glt ‘how good it is’
\cfex{ni-yó-yanreˀ \textsc{\partitive-3s.p-}good.{\stative}}
\z

\ea
\label{ex:spart39}
\glll \textbf{ni}yá:węhs\\
ni-yá:-w-ę-hs\\
\textsc{\partitive-3s.p}-happen-\textsc{\habitual}\\
\glt ‘how it happens’
\z

\ea
\label{ex:spart40}
\glll \textbf{ni}yáwę̱ˀǫh {}\\
ni-yá:-w-ę̱-ˀǫh\\
\textsc{\partitive-3s.p}-happen-{\stative}\\
\glt ‘how it did happen’
\z

\begin{CayugaRelated}
\item \textit{Shęh} \trs{that}, \trs{because}, p. \pageref{p:[shęh] `that’}
\end{CayugaRelated}

\subsection*{\textbf{Shęh niyo:weˀ} \trs{as far as}, \trs{as much as}} \label{p:[shęh niyo:weˀ]}
\CayugaRemark{Particle group \textit{shęh niyo:weˀ} is used in comparisons (\sectref{ch:Comparisons (more, the same, or less)}); clause-initial; \textit{niyó:weˀ} also has the short forms \textit{ní:yo:ˀ} or \textit{nyo:}.}

\ea
\label{ex:spart41}
\gll “A:yę́:ˀ, \textbf{shęh} \textbf{nyó:ˀ} haˀhéˀ dedwatahaˀsé:ˀ, to nyo:wéˀ,” aˀa:gę́ˀ neˀ ǫgyaˀsé:ˀ, aˀa:gę́ˀ.\\
it.seems that a.certain.distance he.went.there the.crossroad, that a.certain.distance she.said the we.are.cousins she.said\\
\glt ‘“It seems, as far as he went on the crossroad, that far (we will go),” said my cousin.’ (\cite{henry_de_2005})
\z

\ea
\label{ex:spart42}
\gll Ęyonishéˀ hęwagi̱ˀdrǫ́:ˀ [\textbf{shę} \textbf{niyo:weˀ} gakwi:yó: agadekǫ́:niˀ].\\
it.will.be.a.while it.will.last.me that a.certain.distance good.food I.ate.it\\
\glt ‘It will last me a long time, how much good food I ate.’ (\cite[294]{mithun_watewayestanih_1984}, Dwade:kǫ́:nih dialogue)
\z

\begin{CayugaRelated}
\item \textit{Gaoˀ shęh niyo:weˀ} \trs{before}, \trs{until}, p. \pageref{p:[gaoˀ shęh niyo:weˀ]}\\
\item \textit{Niyo:weˀ, ni:yo:ˀ, nyo:ˀ} \trs{a certain distance}, \trs{a certain time}, p. \pageref{p:[niyo:weˀ]}\\
\item \textit{Shęh} \trs{that}, \trs{because}, p. \pageref{p:[shęh] `that’}\\
\item \textit{Toh niyo:weˀ} \trs{that far}, p. \pageref{p:[toh niyo:weˀ]}\\
\item \textit{To: niyo:weˀ ne:ˀ} \trs{when}, p. \pageref{p:[to: niyo:weˀ ne:ˀ]}
\end{CayugaRelated}

\subsection*{\textbf{Shęh niyo:weˀ, Gaoˀ shęh niyo:weˀ} \trs{before}, \trs{until}} \label{p:[shęh niyo:weˀ, gaoˀ shęh niyo:weˀ]}
\CayugaRemark{Particle group \textit{(gaoˀ) shęh niyo:weˀ} functions as an “adverb of time” (\sectref{ch:’Adverbs’ of time}, \sectref{ch:Clauses with [(gaoˀ) shęh niyo:weˀ] ‘before, until’}); clause-initial in dependent clauses; \textit{niyó:weˀ} also has the short forms \textit{ní:yo:ˀ} or \textit{nyo:}.}

\ea
\label{ex:spart43}
\gll Agwaˀnigǫháˀ [\textbf{shęh} \textbf{niyo:wéˀ} hędwawayę:nę́:daˀ].\\
we.are.waiting that it.is.a.certain.distance we.will.finish\\
\glt ‘We’re waiting until the time we’re finished.’
\z

\ea
\label{ex:spart44}
\gll Háoˀ o:nę́h dajagyę̱hę́:toh! Heˀsniǫdiˀdré: o:nę́h [\textbf{shęh} \textbf{niyó:} heyó:doˀk].\\
OK now you.two.pull! drag.the.log.over.there now [that it.is.a.certain.distance it.will.end]\\
\glt ‘Come on, you two pull. Drag the log now until it gets to the end.’ (\cite[481]{mithun_watewayestanih_1984}, Dędwaˀęnáęˀ dialogue)
\z

\begin{CayugaRelated}
\item \textit{Gaoˀ} \trs{this side}, \trs{this way}, p. \pageref{p:[gaoˀ]}\\
\item \textit{Gaoˀ nawahtgeh} \trs{the time before then}, p. \pageref{p:[gaoˀ nawahtgeh]}\\
\item \textit{Gaoˀ ni-} \trs{less so}, p. \pageref{p:[gaoˀ ni-]}\\
\item \textit{Gaoˀ shęh niyo:weˀ, shęh niyo:weˀ} \trs{before}, \trs{until}, p. \pageref{p:[gaoˀ shęh niyo:weˀ]}\\
\item \textit{Niyo:weˀ, ni:yo:ˀ, nyo:ˀ} \trs{a certain distance}, \trs{a certain time}, p. \pageref{p:[niyo:weˀ]}\\
\item \textit{Shęh niyo:weˀ} \trs{as far as}, \trs{as much as}, p. \pageref{p:[shęh niyo:weˀ]}\\
\item \textit{To: niyo:weˀ ne:ˀ} \trs{when}, p. \pageref{p:[to: niyo:weˀ ne:ˀ]}\\
\item \textit{Toh niyo:weˀ} \trs{that far}, p. \pageref{p:[toh niyo:weˀ]}
\end{CayugaRelated}

\subsection*{\textbf{Shęh nohgeh} \trs{even}} \label{p:[shęh nohgeh]}
\CayugaRemark{Particle group used for emphasis (\sectref{ch:Focus, contrastive focus, and emphasis markers}); possibly only appears before noun phrases [].}

\ea
\label{ex:spart45}
\gll Ó: i:soˀ gyó:doˀk waˀne:ˀ, \textbf{shę́h} \textbf{nohgeh} [neˀ ojǫ́ˀdaˀ], gá:dǫh gyę:ˀ waˀjih.\\
oh much it.is.lessened today, that even the fish, I.say emphasis a.while.ago\\
\glt ‘Oh, there is a lot less of everything nowadays even fish, as I was saying just a while ago.’ (\cite{mithun_how_1980})
\z

\begin{CayugaRelated}
\item \textit{Shęh} \trs{that}, \trs{because}, p. \pageref{p:[shęh] `that’}
\end{CayugaRelated}

\subsection*{\textbf{Shęh ǫ:weh} \trs{it is really}} \label{p:[shęh ǫ:weh] ‘it is really’}
\CayugaRemark{Particle group marking emphasis (\sectref{ch:Focus, contrastive focus, and emphasis markers}).}

\begin{CayugaRelated}
\item \textit{Shęh} \trs{that}, \trs{because}, p. \pageref{p:[shęh] `that’}\\
\item \textit{ǫ:weh, neˀ ǫ:weh} \trs{really}, p. \pageref{p:[ǫ:weh, neˀ ǫ:weh]}
\end{CayugaRelated}

\subsection*{\textbf{Si:} \trs{over there}} \label{p:[si:]}
\CayugaRemark{Particle functioning as an “adverb of place” (\sectref{‘Adverbs’ of place}); clause-initial.}

\ea
\label{ex:spart46}
\gll \textbf{Si:} tga̱heˀ.\\
over.there it.is.sitting.up.on.top.of.something\\
\glt ‘It is sitting over there.’
\z

\ea
\label{ex:spart47}
\gll \textbf{Si:} tga:ní:yǫ:t.\\
over.there it.is.hanging.there\\
\glt ‘It was hanging over there.’
\z

\ea
\label{ex:spart48}
\gll \textbf{Si:} giˀ gyę:ˀ dagáęˀ ó:nęh!\\
there just this.one they.are.coming now\\
\glt ‘Well here they come now!’ (\cite[88]{mithun_watewayestanih_1984}, Eksaˀgó:wah dialogue)
\z

\begin{CayugaRelated}
\item \textit{Si gwa:dih} \trs{move it!}, p. \pageref{p:[si gwa:dih] ‘move it!’}\\
\item \textit{Si gwa:dih} \trs{over there}, p. \pageref{p:[si gwa:dih] ‘over there’}\\
\item \textit{Si gwa:di:hah} \trs{just this side of}, p. \pageref{p:[si gwa:di:hah]}\\
\item \textit{Si hne:ˀ si gwa:dih} \trs{over there}, p. \pageref{p:[si hne:ˀ si gwa:dih]}\\
\item \textit{Si: hǫ:weh} \trs{way over there}, p. \pageref{p:[si: hǫ:weh]}\\
\item \textit{Si:-gyęh} \trs{that one over there}, p. \pageref{p:[si:-gyęh]}
\end{CayugaRelated}

\subsection*{\textbf{Si gwa:dih} \trs{over there}} \label{p:[si gwa:dih] ‘over there’}
\CayugaRemark{Particle group \textit{si…gwa:dih} functions as an “adverb of place” (\sectref{‘Adverbs’ of place}); clause-initial; \textit{gwa:dih} can also be spelled or pronounced as \textit{gwai}.}

\ea
\label{ex:spart49}
\gll \textbf{Si} \textbf{gwa:dih} tgá:yęˀ.\\
over.there side it.lies.there\\
\glt ‘It is lying over there.’
\z

\ea
\label{ex:spart50}
\gll \textbf{Si} gi̱ˀ-shę́h hwaˀ \textbf{gwa:díh} ga̱ˀto:háh tganí:yǫ:t.\\
over.there maybe this.place side somewhere it.is.hanging.there\\
\glt ‘Maybe it is hanging somewhere over there.’ (\cite[142]{mithun_watewayestanih_1984}, Satgęh dialogue)
\z

\ea
\label{ex:spart51}
\gll \textbf{Si:} hne:ˀ \textbf{gwai} hęhsá:diˀ.\\
over.there in.fact side you.will.put.it.to.one.side\\
\glt ‘Throw that over there!’
\z

\begin{CayugaRelated}
\item \textit{Gwa:dih, gwai} \trs{to one side}, p. \pageref{p:[gwa:dih]}\\
\item \textit{Si:} \trs{over there}, p. \pageref{p:[si:]}\\
\item \textit{Si gwa:dih} \trs{move it!}, p. \pageref{p:[si gwa:dih] ‘move it!’}
\end{CayugaRelated}

\subsection*{\textbf{Si gwa:dih} \trs{move it!}} \label{p:[si gwa:dih] ‘move it!’}
\CayugaRemark{Particle group \textit{si … gwa:dih} functions as a “command” (\sectref{Particles and particle groups used with commands}), used when speaking to animals; \textit{gwa:dih} can also be spelled or pronounced as \textit{gwai}.}

\ea
\label{ex:spart52}
\gll \textbf{Si} hne:ˀ \textbf{si} \textbf{gwa:díh} haˀse:.\\
over.there in.fact over.there side go.over.there\\
\glt ‘Get over there!’ (speaking to an animal)
\z

\begin{CayugaRelated}
\item \textit{Gwa:dih, gwai} \trs{to one side}, p. \pageref{p:[gwa:dih]}\\
\item \textit{Si gwa:dih} \trs{over there}, p. \pageref{p:[si gwa:dih] ‘over there’}
\end{CayugaRelated}

\subsection*{\textbf{Si gwa:di:hah} \trs{just this side of}} \label{p:[si gwa:di:hah]}
\CayugaRemark{Particle group \textit{si…gwa:di-hah} functions as an “adverb of place” (\sectref{‘Adverbs’ of place}). In the examples, it is used metaphorically to mean ‘a little more than’, ‘in the neighbourhood of’, or ‘just this side of’; clause-initial.}

\ea
\label{ex:spart53}
\gll \textbf{si} \textbf{gwa:di:-hah} {}\\
there side.\textsc{\stative-\diminutive}\\
\glt ‘just this side of’
\z

\ea
\label{ex:spart54}
\gll Géi na̱ˀdewe̱ˀnyá:wˀe: \textbf{si} \textbf{gwa:dí:hah} niga:nǫ́:ˀ.\\
four hundred over.there side-ish it.costs\\
\glt ‘It costs a little more than four hundred dollars.’ (\cite[420]{mithun_watewayestanih_1984}, Ęswayętoˀ Gęh? dialogue)
\z

\begin{CayugaRelated}
\item \textit{Gwa:dih, gwai} \trs{to one side}, p. \pageref{p:[gwa:dih]}\\
\item \textit{Si gwa:dih} \trs{over there}, p. \pageref{p:[si gwa:dih] ‘over there’}
\end{CayugaRelated}

\subsection*{\textbf{Si hne:ˀ si gwa:dih} \trs{over there}} \label{p:[si hne:ˀ si gwa:dih]}
\CayugaRemark{Particle group; \textit{si hne:ˀ (si) gwá:dih} functions as an “adverb of place” (\sectref{‘Adverbs’ of place}); emphatic; clause-initial; \textit{gwa:dih} can also be spelled or pronounced as \textit{gwai}.}

\ea
\label{ex:spart55}
\gll \textbf{Si:} \textbf{hne:ˀ} \textbf{gwai} hęhsá:diˀ!\\
over.there in.fact side you.will.throw.it.there\\
\glt ‘Throw that over there!’
\z

\ea
\label{ex:spart56}
\gll \textbf{Si} \textbf{hne:ˀ} \textbf{si-gwa:díh} haˀse:!\\
over.there in.fact side go.over.there\\
\glt ‘Get over there!’ (said to an animal)
\z

\begin{CayugaRelated}
\item \textit{Gwa:dih, gwai} \trs{to one side}, p. \pageref{p:[gwa:dih]}\\
\item \textit{Hne:ˀ} \trs{in fact}, p. \pageref{p:[hne:ˀ] ‘in fact’}\\
\item \textit{Si:} \trs{over there}, p. \pageref{p:[si:]}\\
\item \textit{Si gwa:dih} \trs{move it!}, p. \pageref{p:[si gwa:dih] ‘move it!’}
\end{CayugaRelated}

\subsection*{\textbf{Si: hǫ:weh} \trs{way over there}} \label{p:[si: hǫ:weh]}
\CayugaRemark{Particle group \textit{si:…hǫ:weh} functions as an “adverb of place” (\sectref{‘Adverbs’ of place}); clause-initial; \textit{hǫ:weh} can be spelled or pronounced as \textit{hǫ:}, \textit{nhǫ:}, \textit{hǫ:weh}, or \textit{nhǫ:weh}.}

\ea
\label{ex:spart57}
\gll \textbf{Si:} \textbf{nhǫ:weh} tgani:yǫ:t sagyaˀdawíˀtraˀ.\\
over.there place it.is.hanging.there your.coat\\
\glt ‘Your coat is hanging way over there.’
\z

\ea
\label{ex:spart58}
\gll \textbf{Si:} \textbf{hǫ:weh} haˀęˀ ęyékwaˀ, waˀjíh ędwana̱ˀdá:kshǫ:ˀ\\
over.there place she.is.there she.will.escape, later we.all.will.snack\\
\glt ‘I see her going over there; later we’ll go have a snack.’ (\cite{henry_de_2005})
\z

\begin{CayugaRelated}
\item \textit{Hǫ:weh} \trs{where}, p. \pageref{p:[hǫ:weh]}\\
\item \textit{Si:} \trs{over there}, p. \pageref{p:[si:]}
\end{CayugaRelated}


\subsection*{\textbf{Si:-gyęh} \trs{that one over there}} \label{p:[si:-gyęh]}
\CayugaRemark{Particle group functioning as a “demonstrative pronoun” (\sectref{ch:’Demonstrative pronouns’}); free-standing.}

\ea
\label{ex:spart59}
\gll \textbf{Si:-gyęh} tgá:yęˀ, dé:sehk!\\
that.one it.is.lying.there pick.it.up\\
\glt ‘That lying over there, pick it up!’
\z

\begin{CayugaRelated}
\item \stem{-gyęh} \trs{this}, \trs{that}, p. \pageref{p:[-gyęh]}\\
\item \textit{Si:} \trs{over there}, p. \pageref{p:[si:]}
\end{CayugaRelated}

\subsection*{\textbf{Sǫ:} \trs{some (person, thing)}, \trs{any (person, thing)}} \label{p:[sǫ:]}
\CayugaRemark{Particle functioning as an “indefinite” (\sectref{ch:’Indefinite pronouns’}) or “interrogative” (\sectref{ch:’Interrogative pronouns’}) pronoun; also used as a short form for \textit{sǫ: nˀaht} \trs{who}. }

\begin{CayugaRelated}
\item \textit{Sǫ, sǫ: nˀaht} \trs{who}, p. \pageref{p:[sǫ:, sǫ: nˀaht]}\\
\item \textit{Sǫ: go:węh} \trs{whose} (interrogative), p. \pageref{p:[sǫ: go:węh] ‘whose’}\\
\item \textit{Sǫ: gwaˀ nˀaht} \trs{anybody at all}, \trs{anyone at all}, \trs{any living thing}, \trs{somebody}, \trs{whoever}, p. \pageref{p:sǫ: gwaˀ, sǫ: gwaˀ nˀaht]}\\
\item \textit{Sǫ: gwaˀ nˀaht o:yaˀ} \trs{someone else}, p. \pageref{p:[sǫ: gwaˀ nˀaht o:yaˀ]}\\
\item \textit{Sǫ: gwadih} \trs{on some other side}, p. \pageref{p:[sǫ: gwadih]}\\
\item \textit{Sǫ:-ga:ˀ} \trs{anyone}, \trs{any living thing}, p. \pageref{p:[sǫ:-ga:ˀ]}\\
\item \textit{Sǫ:-ga:ˀah} \trs{someone}, \trs{anyone}, \trs{anything (living)}, p. \pageref{p:[sǫ:-ga:ˀah] ‘someone’}\\
\item \textit{Tęˀ sǫ:-ga:ˀ} \trs{nobody}, \trs{no one}, p. \pageref{p:[tęˀ sǫ:-ga:ˀ]}
\end{CayugaRelated}

\subsection*{\textbf{Sǫ:, Sǫ: nˀaht} \trs{who}} \label{p:[sǫ:, sǫ: nˀaht]}
\CayugaRemark{Particle group \textit{sǫ: … (nˀaht/nˀoht)} functions as an “indefinite pronoun” (\sectref{ch:’Indefinite pronouns’}, \sectref{ch:Clauses with [sǫ́: … (nˀaht)] ‘who’}); clause-initial in direct questions \xref{ex:spart76}, statements \xref{ex:spart77}, indirect questions \xxref{ex:spart78}{ex:spart79}, and dependent clauses \xxref{ex:spart80}{ex:spart81}.}

\ea
\label{ex:spart75}
\gll \textbf{Sǫ́:} ne:ˀ?\\
Who it.is\\
\glt ‘Who’s there?’
\z

\ea
\label{ex:spart76}
\gll \textbf{Sǫ:} hne:ˀ \textbf{nˀáht} to:gyę́h eksá:ˀah?\\
who in.fact person that.one girl\\
\glt ‘Who is that girl?’ (\cite[61]{mithun_watewayestanih_1984}, Sǫ: Hne:ˀ Nˀaht To:gyęh? dialogue)
\z

\ea
\label{ex:spart77}
\gll \textbf{Sǫ:} ˀǫh hne:ˀ \textbf{nˀaht} gaǫdę́:nǫhk.\\
who I.wonder in.fact person her.relatives\\
\glt ‘I am wondering who her people are.’
\z

\ea
\label{ex:spart78}
\gll Daskro:wíˀ [\textbf{sǫ:} \textbf{nˀaht} daǫdekǫ́nya̱hneˀ].\\
you.tell.me who person someone.is.coming.to.eat\\
\glt ‘Tell me who’s coming to eat.’
\z

\ea
\label{ex:spart79}
\gll Gwi:déh honǫhdǫ́ˀ gęh [\textbf{sǫ} ˀǫ \textbf{nˀáht} gowanawę́ˀdagaˀs]?\\
Peter he.knows Q who I.wonder person she.likes.the.taste.of.sugar\\
\glt ‘Does Peter know who likes candy?’
\z

\ea
\label{ex:spart80}
\gll Toh hǫ:wéh godaˀstáˀ gę:s [\textbf{sǫ:} \textbf{nˀáht} neˀ ęyǫnǫ̱hwé:t].\\
that.one where they.sleep usually who person the she.will.visit\\
\glt ‘That’s where they sleep when someone is visiting.’ (\cite{henry_de_2005})
\z

\ea
\label{ex:spart81}
\gll Ne:ˀ giˀ tsǫ́: agǫgwe̱ˀdá:se: tęˀ degógaę [\textbf{sǫ:} \textbf{nˀáht} ęyagodí:nya:k].\\
it.is just only young.woman not she.didn’t.want who person they.will.marry\\
\glt ‘But the young girl didn’t want to be forced by her family to marry anyone.’ (\cite{carrier_legends_2013})
\z

\begin{CayugaRelated}
\item \stem{-nˀaht, -noht} \trs{some person}, p. \pageref{p:[-nˀaht, -noht]}\\
\item \textit{Sǫ:} \trs{some (person, thing)}, \trs{any (person, thing)}, p. \pageref{p:[sǫ:]}
\end{CayugaRelated}


\subsection*{\textbf{Sǫ: go:węh} \trs{whose} (possessive)} \label{p:[sǫ: go:węh] ‘whose’ (possessive)}
\CayugaRemark{Particle and verb functioning as a “possessive pronoun” (\sectref{ch:’Possessive pronouns’}); free-standing.}

\ea
\label{ex:spart67}
\gll \textbf{sǫ:} \textbf{gó:węh}\\
someone she.owns.it\\
\glt ‘she/someone owns it’
\cfex{gó:-w-ęh \textsc{3s.fi.p}-own-{\stative}}
\z

\ea
\label{ex:spart68}
\gll \textbf{Sǫ:} \textbf{go:węh} toh ga̱ˀdréhdase:ˀ?\\
who someone’s that new.car\\
\glt ‘Whose new car is that?’
\z

\begin{CayugaRelated}
\item \textit{Go:węh} \trs{it’s hers}, \trs{it’s someone’s}, p. \pageref{p:[go:węh]}\\
\item \textit{Sǫ:} \trs{some (person, thing)}, \trs{any (person, thing)}, p. \pageref{p:[sǫ:]}
\end{CayugaRelated}

\subsection*{\textbf{Sǫ: go:węh} \trs{whose} (interrogative)} \label{p:[sǫ: go:węh] ‘whose’}
\CayugaRemark{Particle group functioning as an “interrogative possessive pronoun” (\sectref{ch:’Interrogative pronouns’}, \sectref{ch:’Possessive pronouns’}, \sectref{ch:Clauses with [sǫ́: … (nˀaht)] ‘who’}); clause-initial in direct questions (and possibly indirect questions, and dependent clauses).}

\ea
\label{ex:spart69}
\gll \textbf{Sǫ́:} \textbf{go:wę́h} tǫ ga̱ˀdréhdase:ˀ?\\
Who someone.owns.it that.one new.car\\
\glt ‘Who owns that new car?’ `Whose new car is that?’
\z

\begin{CayugaRelated}
\item \textit{Go:węh} \trs{it’s hers}, \trs{it’s someone’s}, p. \pageref{p:[go:węh]}\\
\item \textit{Sǫ:} \trs{some (person, thing)}, \trs{any (person, thing)}, p. \pageref{p:[sǫ:]}
\end{CayugaRelated}

\subsection*{\textbf{Sǫ: gwaˀ, Sǫ: gwaˀ nˀaht, Sǫ: gwaˀ nˀoht} \trs{anybody at all}, \trs{anyone at all}, \trs{any living thing}, \trs{somebody}, \trs{whoever}} \label{p:sǫ: gwaˀ, sǫ: gwaˀ nˀaht]}
\CayugaRemark{Particle group \textit{sǫ:-gwaˀ (nˀaht/nˀoht)} functions as an “indefinite pronoun” (\sectref{ch:’Indefinite pronouns’}); free-standing.}

\ea
\label{ex:spart70}
\gll Ahgwíh \textbf{sǫ:-gwaˀ} da:yǫkí:dahs.\\
don’t someone-right.then they.should.stop.us\\
\glt ‘Don’t let anyone stop us.’ (\cite{carrier_legends_2013})
\z

\ea
\label{ex:spart71}
\gll Nę́: haˀsatga̱htóh o:yáˀ tganǫ́hso:t. A:yę́:ˀ \textbf{sǫ:-gwaˀ-nˀáht} gonęna̱ˀdínyǫ̱ˀdǫh.\\
look look.at.it other the.house.there It.seems someone-right.then-person she.has.moved.in\\
\glt ‘Say, have a look at the next house. It seems like somebody has moved in.’ (\cite[441]{mithun_watewayestanih_1984}, Dwęnǫhsanékahǫˀ dialogue)
\z

\ea
\label{ex:spart72}
\gll Háoˀ \textbf{sǫ:-gwaˀ-nˀáht} gaǫda̱ˀgéh de̱há:daˀ.\\
come.on someone-right.then-person on.the.log he.stands\\
\glt ‘Come on, somebody stand on the log.’ (\cite[481]{mithun_watewayestanih_1984}, Dędwaˀęnáęˀ dialogue)
\z

\ea
\label{ex:spart73}
\gll Ahę́ˀ, “Tęˀ gwahs.” Ahę́ˀ, “Trehs \textbf{sǫ:-gwaˀ-nˀóht} giˀ ahǫwayá:sǫ:ˀ, neˀ gyęǫˀǫh honǫhsodáęˀ.”\\
he.said not this.time he.said because someone-right.then-person just he.called.him, the maybe his.sickness\\
\glt ‘He said, “Not this time, because whoever they called, maybe he was sick.”’ (\cite{henry_de_2005})
\z

\begin{CayugaRelated}
\item \textit{Gwaˀ} \trs{immediately}, \trs{right then}, \trs{just then}, \trs{finally}, p. \pageref{p:[gwaˀ] ‘immediately’}\\
\item \stem{-nˀaht, -noht} \trs{some person}, p. \pageref{p:[-nˀaht, -noht]}\\
\item \textit{Sǫ:} \trs{some (person, thing)}, \trs{any (person, thing)}, p. \pageref{p:[sǫ:]}\\
\item \textit{Sǫ: gwaˀ nˀaht o:yaˀ} \trs{someone else}, p. \pageref{p:[sǫ: gwaˀ nˀaht o:yaˀ]}
\end{CayugaRelated}


\subsection*{\textbf{Sǫ: gwaˀ nˀaht o:yaˀ} \trs{someone else}} \label{p:[sǫ: gwaˀ nˀaht o:yaˀ]}
\CayugaRemark{Particle group functioning as an “indefinite pronoun” (\sectref{ch:’Indefinite pronouns’}); free-standing.}

\ea
\label{ex:spart74}
\gll Ahí:ˀ tgá:gǫ:t \textbf{sǫ:-gwaˀ-nˀóht} \textbf{o:yáˀ} ękehó:wíˀ,\\
I.thought important someone-right.then-person other I.will.tell.someone\\
\glt ‘I thought that it was important that someone else should be told,’(\cite{henry_de_2005})
\z

\begin{CayugaRelated}
\item \textit{Gwaˀ} \trs{immediately}, \trs{right then}, \trs{just then}, \trs{finally}, p. \pageref{p:[gwaˀ] ‘immediately’}\\
\item \stem{-nˀaht, -noht} \trs{some person}, p. \pageref{p:[-nˀaht, -noht]}\\
\item \textit{O:yaˀ} \trs{another}, \trs{other}, \trs{else}, p. \pageref{p:[o:yaˀ]}\\
\item \textit{Sǫ:} \trs{some (person, thing)}, \trs{any (person, thing)}, p. \pageref{p:[sǫ:]}
\end{CayugaRelated}

\subsection*{\textbf{Sǫ: gwa:dih} \trs{on some other side}} \label{p:[sǫ: gwadih]}
\CayugaRemark{Particle group \textit{sǫ:…gwadih}, \textit{sǫ gwai} functions as an “adverb of place” (\sectref{‘Adverbs’ of place}); likely clause-initial.}
	
\begin{CayugaRelated}
\item \textit{Gwa:dih, gwai} \trs{to one side}, p. \pageref{p:[gwa:dih]}\\
\item \textit{Sǫ:} \trs{some (person, thing)}, \trs{any (person, thing)}, p. \pageref{p:[sǫ:]}
\end{CayugaRelated}

\subsection*{\textbf{Sǫ:-ga:ˀ} \trs{anyone}, \trs{any living thing}} \label{p:[sǫ:-ga:ˀ]}
\CayugaRemark{Particle group functioning as an “indefinite pronoun” (\sectref{ch:’Indefinite pronouns’}); free-standing.}

\ea
\label{ex:spart60}
\gll Hada:di:gę́ˀ hnyagwáiˀ tę́ˀ hwę:dǫ́ˀ \textbf{sǫ:gá:ˀ} degye:gę́:ˀ toh na:gagowanę́hę:k, to hniˀ na:yoyáˀdatgi:k.\\
…they.saw.it bear not ever anything it.was.seen that.one it.was.so.big, that and it.was.so.ugly\\
\glt ‘A great bear had appeared, so large and so powerful that many thought it must be some kind of a monster.’ (\cite{carrier_legends_2013})
\z

\begin{CayugaRelated}
\item \stem{-ga:ˀ} element (referring to living beings), p. \pageref{p:[-ga:ˀ]}\\
\item \textit{Sǫ:} \trs{some (person, thing)}, \trs{any (person, thing)}, p. \pageref{p:[sǫ:]}
\end{CayugaRelated}


\subsection*{\textbf{Sǫ:-ga:ˀah} \trs{someone}, \trs{anyone}, \trs{anything (living)}} \label{p:[sǫ:-ga:ˀah] ‘someone’}
\CayugaRemark{Particle group functioning as an “indefinite pronoun” (\sectref{ch:’Indefinite pronouns’}); clause-initial in direct questions \xxref{ex:spart61}{ex:spart62}, indirect questions (not shown), and dependent clauses \xxref{ex:spart65}{ex:spart66}. Also used in statements \xxref{ex:spart63}{ex:spart64} where it is free-standing.}

\ea
\label{ex:spart0}
\gll \textbf{sǫ:-ga:ˀah} {}\\
sǫ:-ga:-ˀah some(one)-\textsc{element-diminutive}\\
\glt ‘someone’, ‘anyone’, ‘anything (living)’
\z

\ea
\label{ex:spart61}
\gll \textbf{Sǫga:ˀáh} gęh hǫwa:yę́:di:?\\
someone Q he.knows.him\\
\glt ‘Does someone know him?’ `Who knows him?’
\z

\ea
\label{ex:spart62}
\gll \textbf{Sǫga:ˀáh} gę́h neˀ gagǫ:gwéh swęnǫhsanekahǫ́ˀ neˀ tę́ˀ de̱ˀagonyá:gǫh?\\
Anyone Q the women your.neighbours the not they.aren’t.married\\
\glt ‘Do you have any women neighbors who are not married?’ (\cite[88]{mithun_watewayestanih_1984}, Eksaˀgó:wah dialogue)
\z

\ea
\label{ex:spart65}
\gll Ne:ˀ giˀ hǫ:niˀ ahí:ˀ gyę́:gwaˀ hné: hwaˀ a:gatró:wiˀ gyę́:gwaˀ hné: hwaˀ \textbf{sǫgá:ˀah} ęhodi̱ˀnigǫ̱háędaˀ nę:-gyę́h hwaˀ shęh niga:yę:.\\
it.is just the.reason I.thought if in.fact this.time I.would.tell if in.fact this.time someone they.will.understand this time that how.it.is.done\\
\glt ‘That is why I thought perhaps I might tell about it so that they might understand how it is done.’ (\cite{mithun_how_1980})
\z

\ea
\label{ex:spart66}
\gll A:ya:wę́h giˀ gyę́:ˀ do:gę́hs a:setséiˀ \textbf{sǫgá:ˀah} a:heyaˀdagé:nhaˀ,\\
I.hope just emphasis sure you.would.find.it someone I.will.help.him\\
\glt ‘I sure hope you find someone who could help you, (\cite[339]{mithun_watewayestanih_1984}, Oˀdréhdatgiˀ dialogue)
\z

\ea
\label{ex:spart63}
\gll \textbf{Sǫga:ˀáh} ǫgyénawahs!\\
someone someone.help.me\\
\glt ‘Someone help me!’
\z

\largerpage
\ea
\label{ex:spart64}
\gll Ó:. Gyę́:gwaˀ giˀ gyę́:ˀ ęga:she:gę́ˀ \textbf{sǫgá:ˀah}, ęgasheho:wíˀ shęh ǫgetgwę̱ˀdáhdǫ:ˀ.\\
Oh, if just emphasis you.will.see.them someone, you.will.tell.them that I.lost.my.wallet\\
\glt ‘Oh. Well, if you see anyone, tell them I lost my wallet.’ (\cite[184]{mithun_watewayestanih_1984}, Gatgwę́ˀdaˀ dialogue)
\z


\begin{CayugaRelated}
\item \stem{-ga:ˀ} element (referring to living beings), p. \pageref{p:[-ga:ˀ]}\\
\item \textit{Sǫ:} \trs{some (person, thing)}, \trs{any (person, thing)}, p. \pageref{p:[sǫ:]}
\end{CayugaRelated}

\subsection*{\textbf{Stǫ:hah, Stǫ:hǫh} \trs{a little bit}} \label{p:[stǫ:hah]}
\CayugaRemark{Particle functioning as an “adverb of degree” (\sectref{ch:’Adverbs’ of degree}); clause-initial.}

\ea
\label{ex:spart82}
\gll \textbf{Stǫ:háh} segę́is.\\
little.bit move\\
\glt ‘Move a little bit.’
\z

\ea
\label{ex:spart83}
\gll Tę́ˀ hné:ˀ de̱ˀheˀs. Honǫhǫkda:níh gyę́:ˀ-gęh \textbf{stǫ:hǫ́h} hotowinyǫˀsé: gyę́:ˀ-ǫh.\\
no in.fact he.is.not.here he.is.sick this.one-Q a.little he.has.a.cold this.one-I.guess\\
\glt ‘No he isn’t. He seems to be sick, a bit of a cold, I guess.’ (\cite[184]{mithun_watewayestanih_1984}, Gatgwę́ˀdaˀ dialogue)
\z

\ea
\label{ex:spart84}
\gll Géi sga̱heˀ niyohshé:dęh. Gyę:gwáˀ-hne:-hwáˀ \textbf{stǫ:hǫ́h} degyáhdi̱hęh, shęh ní:waˀs.\\
four tens what.size.it.is maybe-in.fact a.little they.are.different, what size.they.are\\
\glt ‘It is a size fourteen. Maybe they are a little different in size.’ (\cite[225]{mithun_watewayestanih_1984}, Agyaˀdawíˀtraˀ dialogue)
\z


\section{T particles}

\subsection*{\textbf{Tęˀ} \trs{no}, \trs{not}} \label{p:[tęˀ]}
\CayugaRemark{Particle signaling disagreement (\sectref{ch:Disagreement markers}); optional before a negated verb.}

\ea
\label{ex:tpart1}
\gll \textbf{Tęˀ} \textbf{ta:}hayę́:toh.\\
not he.won’t.plant\\
\glt ‘No, he won’t plant.’
\z

\ea
\label{ex:tpart2}
\gll \textbf{Tęˀ} \textbf{de}wagadagáideˀ.\\
not I’m.not.well\\
\glt ‘No, I am not well.’
\z

\ea
\label{ex:tpart3}
\gll \textbf{Tęˀ} toh \textbf{de}gáhe:ˀ.\\
not there it.is.not.setting\\
\glt ‘No, it isn’t setting there.’
\z

\ea
\label{ex:tpart4}
\gll \textbf{Tęˀ} \textbf{tǫ}desa̱ˀdré: tó:gyęh!\\
not you.won’t.drive that.one\\
\glt ‘Don’t drive that over here!’
\z

\ea
\label{ex:tpart5}
\gll \textbf{Tęˀ} \textbf{taˀde}yagodawę́:nyeˀ.\\
not she.is.not.walking.about\\
\glt ‘No, she isn’t walking about.’
\z

\ea
\label{ex:tpart6}
\gll \textbf{Tęˀ} ni:ˀ toh \textbf{tá:}ge:ˀ.\\
not I there I’m.not.going\\
\glt ‘No, I am not going there.’
\z

\ea
\label{ex:tpart7}
\gll \textbf{Tęˀ} ni:ˀ \textbf{ta:}gyé:na:.\\
not I I.won’t.accept.it\\
\glt ‘No, I will not accept it.’
\z

\ea
\label{ex:tpart8}
\gll \textbf{Tęˀ} giˀ ni:ˀ to \textbf{tá:}gye:. \\
not just I there I.won’t.do.it\\
\glt ‘No, I will not do that.’
\z



\subsection*{\textbf{Tęˀ ahsǫh} \trs{not yet}} \label{p:[tęˀ ahsǫh]}
\CayugaRemark{Particle group \textit{tęˀ … ahsǫh} functions as an “adverb of time” (\sectref{ch:’Adverbs’ of time}); clause-initial.}

\ea
\label{ex:tpart9}
\gll \textbf{Tęˀ} gęh \textbf{ahsǫ́h} de̱ˀotsęhdǫ́h tó:gyęh?!\\
not Q yet it.is.not.getting.tired this.one\\
\glt ‘Isn’t that bear getting tired yet?!’ (\cite{carrier_legends_2013})
\z


\ea
\label{ex:tpart10}
\gll \textbf{Tęˀ} \textbf{ahsǫ́h} de̱ˀagatgę́hǫh.\\
not yet I’m.not.up\\
\glt ‘I am not yet up.’ (\cite[142]{mithun_watewayestanih_1984}, Satgęh dialogue)
\z

\begin{CayugaRelated}
\item \textit{Ahsǫh} \trs{still}, \trs{yet}, p. \pageref{p:[ahsǫh] ‘still’, ‘yet’}\\
\item \textit{Tęˀ} \trs{no}, \trs{not}, p. \pageref{p:[tęˀ]}
\end{CayugaRelated}

\subsection*{\textbf{Tęˀ daˀǫ} \trs{it will never happen}, \trs{definitely not}} \label{p:[tęˀ daˀǫ]}
\CayugaRemark{Particle group \textit{tęˀ … da̱ˀǫ} signals disagreement (\sectref{ch:Disagreement markers}); emphatic; clause-initial.}

\ea
\label{ex:tpart11}
\gll \textbf{tęˀ} \textbf{da} \textbf{ˀǫ}\\
not and I.wonder\\
\glt ‘it will never happen’, ‘definitely not’
\z

\ea
\label{ex:tpart12}
\gll \textbf{Tęˀ} \textbf{daˀǫ} toh tá:ge:.\\
not definitely.not there I.should.go.there\\
\glt ‘No, I will definitely not go.’
\z

\ea
\label{ex:tpart13}
\gll \textbf{Tęˀ} \textbf{daˀǫ}  a:hǫwayená:waˀs.\\
not definitely.not he.should.help.him\\
\glt ‘Definitely not; no one will help him.’
\z

\ea
\label{ex:tpart14}
\gll \textbf{Tęˀ} giˀ \textbf{daˀǫ} ta:yagógaę.\\
no just definitely.not she.would.agree\\
\glt ‘No, she will never agree.’
\z

\ea
\label{ex:tpart15}
\gll “O:nę́h giˀ gę:s \textbf{tęˀ} gwahs \textbf{daˀǫ́} a:hadijáoˀdę:.”\\
now just usually no emphasis definitely.not he.would.smoke.tobacco\\
\glt “If only they didn’t smoke as much tobacco.” (\cite{henry_de_2005})
\z

\ea
\label{ex:tpart16}
\gll Ji̱ˀtréhs sahsę:, sanǫˀséh hniˀ. \textbf{Tęˀ} \textbf{daˀǫ́} a:wadeˀgóˀ neˀ nęh ęhyá:gęˀ!\\
too.much.you are.fat you.are.lazy and not certainly.not it.would.escape the now it.will.get.away.from.you\\
\glt ‘You are too fat and lazy – it certainly won’t run from you.’ (\cite{carrier_legends_2013})
\z

\ea
\label{ex:tpart17}
\gll Tę́ˀ. Tréhs hne:ˀ a:yę́:ˀ niwú:sˀuh. A:yę́:ˀ \textbf{tę́ˀ} \textbf{daˀǫ} ha:wagí:dęˀ.\\
not too.much in.fact it.seems it.is.small it.seems not certainly.not it.should.fit.me\\
\glt ‘No. It seems to be too small for me. It doesn’t seem to fit me.’ (\cite[225]{mithun_watewayestanih_1984}, Agyaˀdawíˀtraˀ dialogue)
\z

\ea
\label{ex:tpart18}
\gll A:yę́:ˀ gyę:ˀ stǫ:hǫ́h ganǫ́:ˀ. Ne:ˀ tsǫ́: shęh oya:nréˀ se̱ˀ-gyę́: hné:ˀ gwa̱ˀ-toh. \textbf{Tę́ˀ} \textbf{daˀǫ} a:wadahsgwi̱ˀtrǫ́:niˀ, hę:gyę: neˀ haesatroni̱há:k aesę́:daˀ.\\
it.seems this.one little.bit expensive it.is just that good.ones you.know-this.one in.fact that.one not certainly.not it.won’t.get.wrinkles no.matter the you.would.wear.clothes you.would.sleep\\
\glt ‘It seems a little expensive, but it is a good one. It won’t wrinkle, even if you sleep with your clothes on.’ (\cite[225]{mithun_watewayestanih_1984}, Agyaˀdawíˀtraˀ dialogue)
\z

\ea
\label{ex:tpart19}
\gll O:nę́h ǫ hné:ˀ gwé:gǫh. \textbf{Tę́ˀ} ǫ hné:ˀ \textbf{daˀǫ} daǫsagǫˀnigǫháęˀ ó:nęh.\\
now I.guess in.fact all not I.wonder in.fact definitely.not I.shouldn’t.bother.you now\\
\glt ‘This must be everything now. I shouldn’t bother you any more now.’ (\cite[441]{mithun_watewayestanih_1984}, Dwęnǫhsanekahǫˀ dialogue)
\z

\begin{CayugaRelated}
\item \textit{Da:} \trs{and}, p. \pageref{p:[da:]}\\
\item \textit{Oh, ǫ:, ǫ} \trs{I guess}, \trs{I wonder (if)}, p. \pageref{p:[ǫh, ǫ:, ǫ]}\\
\item \textit{Tęˀ} \trs{no}, \trs{not}, p. \pageref{p:[tęˀ]}
\end{CayugaRelated}

\subsection*{\textbf{Tęˀ dedo:gęhs} \trs{it’s not true}, \trs{not really}} \label{p:[tęˀ dedo:gęhs]}
\CayugaRemark{Particle and verb signaling disagreement (\sectref{ch:Disagreement markers}); free-standing or clause-initial.}

\ea
\label{ex:tpart20}
\gll \textbf{Tęˀ} tǫ ne:ˀ \textbf{dedó:gęhs}.\\
no that.one it.is it.is.not.true\\
\glt ‘No, it isn’t really true.’
\z

\ea
\label{ex:tpart21}
\gll \textbf{Tęˀ} giˀ hne:ˀ \textbf{dedo:gę́hs} deˀsá:dǫh.\\
not just in.fact it.is.not.true you.are.not.saying.it\\
\glt ‘No, it isn’t really true what you are saying.’
\z

\begin{CayugaRelated}
\item \textit{Do:gęhs} \trs{it is true}, \trs{isn’t it true?}, p. \pageref{p:[do:gęhs]}\\
\item \textit{Tęˀ} \trs{no}, \trs{not}, p. \pageref{p:[tęˀ]}
\end{CayugaRelated}

\subsection*{\textbf{Tęˀ deˀgę:} \trs{it isn’t}} \label{p:[tęˀ deˀgę:]}
\CayugaRemark{Particle group \stem{tęˀ … \textsc{noun} de̱ˀgę: neˀ \textsc{noun}} or \stem{tęˀ … \textsc{noun noun} de̱ˀgę:} ‘noun is not a noun’ links two nouns (or noun phrases), and conveys the idea that the second noun (phrase) does not belong to the category denoted by the first noun (phrase) \xxref{ex:tpart22}{ex:tpart24}. Words or phrases functioning as “nouns” are shown between square brackets.)  Particle group \stem{tęˀ … \textsc{noun} de̱ˀgę:} ‘noun is not’ conveys the idea that the noun (in square brackets) does not exist \xref{ex:tpart25}.}

\ea
\label{ex:tpart22}
\gll \textbf{tęˀ} \textbf{deˀ-gę:}\\
not \textbf{not}-it.is\\
\glt ‘it isn’t’
\z

\ea
\label{ex:tpart23}
\gll \textbf{Tęˀ} hne:ˀ [ganyó:ˀ] \textbf{deˀgę:} [neˀ awę́hęˀ]\\
Not in.fact animal it.is.not the flower(s)\\
\glt ‘Flowers are not animals.’
\z

\ea
\label{ex:tpart24}
\gll Tę́ˀ ní:ˀ de̱ˀa:gé:gaˀs. \textbf{Tę́ˀ} [ní:ˀ] [gwaˀyǫˀ] \textbf{deˀgę:}.\\
no I I.don’t.like.the.taste.of.it not I rabbit it.is.not\\
\glt ‘No, I don’t like it. I am no rabbit, you know.’ (\cite[294]{mithun_watewayestanih_1984}, Dwade:kǫ́:nih dialogue)
\z


\ea
\label{ex:tpart25}
\gll \textbf{Tęˀ} gwahs ǫ:wéh [ǫ:gwéh] \textbf{degę:}.\\
not really truly person it.is.not\\
\glt ‘No, you aren’t really human.’
\z

\begin{CayugaRelated}
\item Equative sentences with linking verbs \textit{né:ˀ} \trs{it is} or \textit{de̱ˀgę:} \trs{it isn’t}, p. \pageref{Equative sentences with linking verbs [né:ˀ], [deˀgę:]}\\
\item \textit{Ę:, E:} (possible atypical verb), p. \pageref{p:[ę:, e:] `atypical verb’}\\
\item \textit{Tęˀ} \trs{no}, \trs{not}, p. \pageref{p:[tęˀ]}\\
\item \textit{Tęˀ degę:hęh} \trs{not too many}, p. \pageref{p:[tęˀ degę:hęh]}
\end{CayugaRelated}

\subsection*{\textbf{Tęˀ degę:hęh} \trs{not too many}} \label{p:[tęˀ degę:hęh]}
\CayugaRemark{Particle group \textit{(tęˀ) degę:hęh} likely functions as an “adverb of degree” (\sectref{ch:’Adverbs’ of degree}) and is likely clause-initial.}

\ea
\label{ex:tpart26}
\gll \textbf{tęˀ} \textbf{deˀ-gę:-hęh}\\
not \textbf{not}-it.is-\textbf{diminutive}\\
\glt ‘not too many’
\z

\begin{CayugaRelated}
\item \textit{Tęˀ} \trs{no}, \trs{not}, p. \pageref{p:[tęˀ]}\\
\item \textit{Tęˀ de̱ˀgę:} \trs{it isn’t}, p. \pageref{p:[tęˀ deˀgę:]}
\end{CayugaRelated}

\subsection*{\textbf{Tęˀ gaˀ-toh, Gaˀ-toh} \trs{nowhere}, \trs{not anywhere}} \label{p:[tęˀ gaˀ-toh]}
\CayugaRemark{Particle group \textit{(tęˀ) ga̱ˀ-toh} functions as an “adverb of place” (\sectref{‘Adverbs’ of place}); clause-initial.}

\ea
\label{ex:tpart27}
\gll \textbf{Tęˀ} \textbf{gaˀ-toh} degá:yęˀ.\\
not anywhere it.is.not.lying\\
\glt ‘It isn’t lying anywhere.’
\z

\begin{CayugaRelated}
\item \stem{gaˀ-} element (referring to a location), p. \pageref{p:[gaˀ-]}\\
\item \textit{Gaˀ-toh, tęˀ gaˀ-toh} \trs{nowhere}, \trs{not anywhere}, p. \pageref{p:[gaˀ-toh, tęˀ gaˀ-toh]}\\
\item \textit{Gaˀ-to:hah} \trs{somewhere}, \trs{someplace}, \trs{around}, \trs{anywhere}, \trs{thereabouts}, p. \pageref{p:[gaˀ-to:hah]}\\
\item \textit{Gaˀ-to:hah tohgeh} \trs{thereabouts}, p. \pageref{p:[gaˀ-to:hah tohgeh]}\\
\item \textit{Tęˀ} \trs{no}, \trs{not}, p. \pageref{p:[tęˀ]}
\end{CayugaRelated}


\subsection*{\textbf{Tęˀ gęh} \trs{no?}, \trs{isn’t it?}} \label{p:[tęˀ gęh]}
\CayugaRemark{Particle group or tag question (\sectref{Tag questions (asking for confirmation from the listener)}); optionally followed by a negative verb.}

\ea
\label{ex:tpart28}
\gll Wę̱hnihsri:yó: waˀneˀ, \textbf{tęˀ} \textbf{gęh}?\\
it.is.a.nice.day today, not Q\\
\glt ‘It’s is a nice day, innit?’
\z

\begin{CayugaRelated}
\item \textit{Gęh} \trs{Q} (question marker), p. \pageref{p:[gęh] `Q’}\\
\item \textit{Tęˀ} \trs{no}, \trs{not}, p. \pageref{p:[tęˀ]}
\end{CayugaRelated}

\subsection*{\textbf{Tęˀ gęh deˀ-} \trs{didn’t?}} \label{p:[tęˀ gęh deˀ-]}
\CayugaRemark{Particle group used with questions that imply \textit{ęhęˀ} \trs{yes} or \textit{tęˀ} \trs{no} answers (\sectref{Yes-no questions with [gęh]}); followed by a negative verb; clause-initial.}

\ea
\label{ex:tpart29}
\gll \textbf{Tęˀ} \textbf{gęh} \textbf{de}shó:yǫ:? {}\\
Not Q he.didn’t.return\\
\glt ‘Did he not come home?’
\z

\ea
\label{ex:tpart30}
\gll \textbf{Tęˀ} \textbf{gęh} \textbf{de}hsé: to: ne: dwá:ye:?\\
Not Q you.don’t.think that it.is we.do.it\\
\glt ‘Don’t you think we should do that?’
\z


\begin{CayugaRelated}
\item \textit{Gęh} \trs{Q} (question marker), p. \pageref{p:[gęh] `Q’}\\
\item \textit{Tęˀ} \trs{no}, \trs{not}, p. \pageref{p:[tęˀ]}
\end{CayugaRelated}

\subsection*{\textbf{Tęˀ giˀ ni:ˀ} \trs{no, not me}} \label{p:[tęˀ giˀ ni:ˀ]}
\CayugaRemark{Particle group functioning as an “emphatic pronoun” (\sectref{ch:’Emphatic pronouns’}); free-standing.}

\ea
\label{ex:tpart33}
\gll \textbf{Tęˀ} \textbf{giˀ} \textbf{ni:ˀ} to tá:gye:.\\
not just the.me that I.wouldn’t.do.it\\
\glt ‘No, I will not do that.’
\z

\begin{CayugaRelated}
\item \textit{Giˀ} \trs{just}, p. \pageref{p:[giˀ]}\\
\item \textit{I:ˀ, Ni:ˀ} \trs{I}, \trs{we}, p. \pageref{p:[i:ˀ]}\\
\item \textit{Tęˀ} \trs{no}, \trs{not}, p. \pageref{p:[tęˀ]}
\end{CayugaRelated}

\subsection*{\textbf{Tęˀ giˀ shęh hwaˀ daˀǫ} \trs{maybe not}} \label{p:[tęˀ giˀ shęh hwaˀ daˀǫ]}
 \CayugaRemark{Particle group \textit{tęˀ giˀ shęh (hwaˀ) (daǫ:)} functions as a “doubt or certainty marker” (\sectref{ch:Evidential markers}); free-standing or clause-initial.}

\ea
\label{ex:tpart34}
\gll \textbf{Tęˀ} \textbf{giˀ} \textbf{shęh} \textbf{hwaˀ -} waˀgyę́h dęˀ nęyá:węh.\\
not just that this.time presently what it.will.happen\\
\glt ‘Maybe not – we’ll see.’ (\cite{carrier_legends_2013})
\z

\ea
\label{ex:tpart35}
\gll [\textbf{Tęˀ} \textbf{giˀ} \textbf{shę́h} \textbf{hwaˀ} \textbf{daǫ́:} toh na:yá:węh.]\\
not just that this.time not.at.all that.one it.will.happen\\
\glt ‘Maybe it won’t happen this time.’ (\cite{henry_de_2005})
\z

\begin{CayugaRelated}
\item \textit{Ga:t giˀ shęh} \trs{maybe}, \trs{or maybe}, p. \pageref{p:[ga:t giˀ shęh]}\\
\item \textit{Ga:t giˀ shęh tęˀ} \trs{maybe not}, \trs{or not}, p. \pageref{p:[ga:t giˀ shęh tęˀ]}\\
\item \textit{Gęh ga:t giˀ shęh nigęˀǫh} \trs{or?}, p. \pageref{p:[gęh ga:t giˀ shęh nigęˀǫh]}\\
\item \textit{Giˀ} \trs{just}, p. \pageref{p:[giˀ]}\\
\item \textit{Hwaˀ} \trs{this time}, \trs{next}, p. \pageref{p:[hwaˀ]}\\
\item \textit{O: tęˀ ǫh ga:t giˀ shęh gyę:gwaˀ} \trs{maybe, maybe not}, p. \pageref{p:[o: tęˀ ǫh ga:t giˀ shęh gyę:gwaˀ]}\\
\item \textit{O: tęˀ ǫh ga:t giˀ shęh ne:ˀ hwaˀ} \trs{maybe, maybe not}, p. \pageref{p:[o: tęˀ ǫh ga:t giˀ shęh ne:ˀ hwaˀ]}\\
\item \textit{Shęh} \trs{that}, \trs{because}, p. \pageref{p:[shęh] `that’}\\
\item \textit{Tęˀ} \trs{no}, \trs{not}, p. \pageref{p:[tęˀ]}\\
\item \textit{Tęˀ daˀǫ} \trs{it will never happen}, \trs{definitely not}, p. \pageref{p:[tęˀ daˀǫ]}
\end{CayugaRelated}

\subsection*{\textbf{Tęˀ gwahs ǫ:weh} \trs{not really}, \trs{not quite}} \label{p:[tęˀ gwahs ǫ:weh]}
\CayugaRemark{Particle group \textit{tęˀ (gwahs) (ǫ:weh)} signals disagreement (\sectref{ch:Disagreement markers}); clause-initial.}

\ea
\label{ex:tpart36}
\gll \textbf{Tęˀ} \textbf{gwahs} \textbf{ǫ:wéh} ǫ:gwéh degę:.\\
not really truly person it.is.not\\
\glt ‘No, you are not really human.’
\z

\ea
\label{ex:tpart37}
\gll Sǫ: ní:s nˀaht gwahs? \textbf{Tę́ˀ} \textbf{gwáhs} \textbf{ǫ:wéh} de̱ˀgǫyędéi. Dave Maracle gęh haya:sǫ́h neˀ hya̱ˀnih?\\
who you who anyway not really truly I.do.not.know.you Dave Maracle Q he.is.named the your.father\\
\glt ‘Who are you, anyway? I don’t really know you. Is Dave Maracle your father?’ (\cite[339]{mithun_watewayestanih_1984}, Oˀdréhdatgiˀ dialogue)
\z

\ea
\label{ex:tpart38}
\gll \textbf{Tę́ˀ} giˀ hné:ˀ \textbf{gwahs} á:yę:ˀ, tréhs gyę́:ˀ ǫh do:gę́hs wa̱hdahgwadę́:s dewagę́:sǫ:.\\
not just in.fact really it.seems too.much this.one I.guess it.is.true thick.shoes I.have.shoes.on\\
\glt ‘No, it doesn’t really seem too bad, I guess, because I have really thick shoes on.’ (\cite[386]{mithun_watewayestanih_1984}, Ga̱há:gǫ: dialogue)
\z

\ea
\label{ex:tpart39}
\gll \textbf{Tę́ˀ} \textbf{gwahs} \textbf{deˀ}ǫkniwayęnę́da̱ˀǫˀ, stǫ́:hǫh daonǫ́:ˀ⁠. \\
not really we.did.not.finish.it a.little they.weren’t.a.certain.number\\
\glt ‘Not quite we didn’t. There was still a little more left to do.’ (\cite[456]{mithun_watewayestanih_1984}, Oyę́hsraˀ dialogue)
\z

\ea
\label{ex:tpart40}
\gll \textbf{Tę́ˀ} \textbf{gwáhs} \textbf{de}gowanę́h \textbf{tę́ˀ} \textbf{gwáhs} \textbf{de}wú:ˀuh.\\
not really it.is.not.big not really it.isn’t.small\\
\glt ‘Not too big, not too small.’ (\cite{keye_circle_2016}, Circle Book 10, the Magic Chair)
\z

\begin{CayugaRelated}
\item \textit{Gwahs} \trs{really}, \trs{just}, \trs{quite}, p. \pageref{p:[gwahs] ‘really’}\\
\item \textit{Gwahs ǫ:weh} \trs{really}, p. \pageref{p:[gwahs ǫ:weh]}\\
\item \textit{O:weh, neˀ ǫ:weh} \trs{really}, p. \pageref{p:[ǫ:weh, neˀ ǫ:weh]}\\
\item \textit{Tęˀ} \trs{no}, \trs{not}, p. \pageref{p:[tęˀ]}
\end{CayugaRelated}


\subsection*{\textbf{Tęˀ gwaˀ-toh} \trs{none at all}} \label{p:tęˀ gwa̱ˀ-toh}
\CayugaRemark{Particle group functioning as a “demonstrative pronoun” (\sectref{ch:’Demonstrative pronouns’}); free-standing.}

\ea
\label{ex:tpart31}
\gll \textbf{tęˀ} \textbf{gwaˀ} \textbf{toh}\\
not right.then that.one\\
\glt ‘none at all’
\z

\ea
\label{ex:tpart32}
\gll Nę: tsǫ: ni:yǫ́: ǫgwa:yę́ˀ oˀwáhǫh. \textbf{Tęˀ} \textbf{gwaˀ-toh} de̱ˀjǫ:gwa:yę́ˀ neˀ onę́hę:ˀ, osáheˀdaˀ.\\
it.is little it.is.a.certain.amount we.have meat not right.then-that.one we.do.not.have the corn beans\\
\glt ‘We have very little meat left; we have no corn or beans left.’(\cite{carrier_legends_2013})
\z

\begin{CayugaRelated}
\item \textit{Gwaˀ} \trs{immediately}, \trs{right then}, \trs{just then}, \trs{finally}, p. \pageref{p:[gwaˀ] ‘immediately’}\\
\item \textit{Gwa̱ˀ toh} \trs{that one; just such a one}, p. \pageref{p:[gwaˀ toh]}\\
\item \textit{Tęˀ} \trs{no}, \trs{not}, p. \pageref{p:[tęˀ]}\\
\item \textit{Tęˀ gwa̱ˀ-toh} \trs{none at all}, p. \pageref{p:tęˀ gwa̱ˀ-toh}\\
\item \textit{to}, \textit{To:, Toh, Tǫ:} \trs{that one}, p. \pageref{p:[to:] ‘that one’}
\end{CayugaRelated}

\subsection*{\textbf{Tęˀ gyę:ˀ nę neˀ} \trs{not}, \trs{what on earth?}} \label{p:[tęˀ gyę:ˀ nę neˀ]}
\CayugaRemark{Particle group emphasizing something negative or upsetting (\sectref{ch:Focus, contrastive focus, and emphasis markers}); clause-initial.}

\ea
\label{ex:tpart41}
\gll Tohgéh aˀa:gę́ˀ, “Dęˀ diˀ nęya:wę́h ędwáganyaˀk? \textbf{Tęˀ} \textbf{gyę:ˀ} \textbf{nę} \textbf{neˀ} - dęˀ diˀ hoˀdę́ˀ ędwahs? \textbf{Tęˀ} \textbf{gyę:ˀ} \textbf{nę} \textbf{ne:ˀ} ni:ˀ de̱ˀǫ:gwa:yę́ˀ neˀ ohwíhsdaˀ.”\\
then she.said what so something.will.happen we.will.pay not emphasis here the - what so kind we.will.use not emphasis here the we we.do.not.have the money\\
\glt ‘Then she said, “How will we pay for it? What on earth will we use? None of us have any money.”’ (\cite{henry_de_2005})
\z

\begin{CayugaRelated}
\item \textit{Gyę:ˀ} \trs{just the one}, p. \pageref{p:[gyę:ˀ]}\\
\item \textit{Neˀ} \trs{that is}, \trs{the}, p. \pageref{p:[neˀ]}\\
\item \textit{Neˀ gyę:ˀ nę ne:ˀ} (emphasis), p. \pageref{p:[neˀ gyę:ˀ nę ne:ˀ]}\\
\item \textit{Nę:} \trs{this}, \trs{these}, p. \pageref{p:[nę:] ‘this, these’}\\
\item \textit{Nę: ne:ˀ i:s} \trs{how about you}, p. \pageref{p:[nę: ne:ˀ i:s]}\\
\item \textit{Tęˀ} \trs{no}, \trs{not}, p. \pageref{p:[tęˀ]}
\end{CayugaRelated}

\subsection*{\textbf{Tęˀ gyę:ˀ ǫh, Tęˀ gęˀǫh} \trs{not really}} \label{p:[tęˀ gyę:ˀ ǫh]}
\CayugaRemark{Particle group signaling disagreement (\sectref{ch:Disagreement markers}); free-standing.}

\ea
\label{ex:tpart42}
\gll Ó:, \textbf{tęˀ} \textbf{gyę́:ˀ} \textbf{ǫ} hné:ˀ ǫ:wí: agyę́ˀ gyę́:ˀ nǫ́ne:ˀ neˀ degahehdáwenyehs. Dají:hah gę́:s tsǫ́: i:sóˀ atkehdawę́:nye:ˀ.\\
oh, not this.one I.wonder in.fact I.should.think I.have emphasis you.know the cultivator short.amount.of.time usually just much I.plow.it\\
\glt ‘Oh, not really, because I have a cultivator, you know. In just a short time I plow a lot.’ (\cite[420]{mithun_watewayestanih_1984}, Ęswayętoˀ Gęh dialogue)
\z

\begin{CayugaRelated}
\item \textit{Gyę:ˀ} \trs{just the one}, p. \pageref{p:[gyę:ˀ]}\\
\item \textit{Gyę:ˀ ǫh} \trs{maybe}, \trs{I guess}, \trs{I wonder}, p. \pageref{p:[gyę:ˀ ǫh]}\\
\item \textit{Oh, ǫ:, ǫ} \trs{I guess}, \trs{I wonder (if)}, p. \pageref{p:[ǫh, ǫ:, ǫ]}\\
\item \textit{Tęˀ} \trs{no}, \trs{not}, p. \pageref{p:[tęˀ]}
\end{CayugaRelated}

\subsection*{\textbf{Tęˀ hne:ˀ} \trs{definitely not}} \label{p:[tęˀ hne:ˀ]}
\CayugaRemark{Particle group signaling disagreement (\sectref{ch:Disagreement markers}); emphatic; clause-initial.}

\ea
\label{ex:tpart43}
\gll \textbf{Tęˀ} \textbf{hne:ˀ} hwaˀ to ta:yá:węh.\\
not in.fact this.time that it.shouldn’t.happen\\
\glt ‘No, that’ll never really happen this time.’
\z

\ea
\label{ex:tpart44}
\gll \textbf{Tęˀ} \textbf{hne:ˀ} dehóyętwęh.\\
not in.fact he.did.not.plant\\
\glt ‘No, in fact he didn’t plant.’
\z

\begin{CayugaRelated}
\item \textit{Hne:ˀ} \trs{in fact}, p. \pageref{p:[hne:ˀ] ‘in fact’}\\
\item \textit{Tęˀ} \trs{no}, \trs{not}, p. \pageref{p:[tęˀ]}
\end{CayugaRelated}

\subsection*{\textbf{Tęˀ hne:ˀ neˀ} \trs{not that one}} \label{p:[tęˀ hne:ˀ neˀ]}
\CayugaRemark{Particle group \stem{(\textsc{item}) tęˀ hne:ˀ (neˀ) \textsc{item}} is used for contrastive focus (\sectref{ch:Focus, contrastive focus, and emphasis markers}).}

\ea
\label{ex:tpart45}
\gll Ohyadǫhsra:sé: i:wí: \textbf{tęˀ} \textbf{hne:ˀ} ohyadǫ̱hsra:gá:yǫh.\\
new.book I.want not in.fact old.book\\
\glt ‘I want the \emph{new} book, not the \emph{old} one.’
\z

\ea
\label{ex:tpart46}
\gll Agadadrihǫnyę́ˀ ohyadǫhsra:sé:ˀ té:dęˀ, \textbf{tęˀ} \textbf{hne:ˀ} \textbf{neˀ} gyoté:dęht.\\
I.read.it new.book yesterday, not in.fact the day.before.yesterday\\
\glt ‘I read the new book \emph{yesterday}, not \emph{the day before}.’
\z

\ea
\label{ex:tpart47}
\gll I:ˀ ęgahdę:díˀ \textbf{tęˀ} \textbf{hne:ˀ} \textbf{neˀ} gáǫhęˀ.\\
I I.will.leave not in.fact the she.alone\\
\glt ‘It is \emph{I} who am leaving, not \emph{her}.’
\z

\ea
\label{ex:tpart48}
\gll Wa:li gó:gaˀs, \textbf{tęˀ} \textbf{hne:ˀ} \textbf{neˀ} John.\\
Mary she.likes.the.taste.of.it not in.fact the John\\
\glt ‘It is \emph{Mary} who likes the taste of it, not \emph{John}.’
\z

\ea
\label{ex:tpart49}
\gll \textbf{Tęˀ} \textbf{hne:ˀ} gaǫhę́ˀ ne:ˀ. \textbf{Tęˀ} \textbf{hne:ˀ} haǫhę́ˀ giˀ.\\
not in.fact she.alone it.isn’t in.fact he.alone just\\
\glt ‘It’s not \emph{her}. It’s \emph{him}.’
\z

\begin{CayugaRelated}
\item \textit{Hne:ˀ} \trs{in fact}, p. \pageref{p:[hne:ˀ] ‘in fact’}\\
\item \textit{Neˀ} \trs{that is}, \trs{the}, p. \pageref{p:[neˀ]}\\
\item \textit{Tęˀ} \trs{no}, \trs{not}, p. \pageref{p:[tęˀ]}
\end{CayugaRelated}

\subsection*{\textbf{Tęˀ hwę:dǫh} \trs{never}} \label{p:[tęˀ hwę:dǫh]}
\CayugaRemark{Particle group functioning as an “adverb of time” (\sectref{ch:’Adverbs’ of time}); clause-initial.}

\ea
\label{ex:tpart50}
\gll \textbf{Tęˀ} \textbf{hwę:dǫ́h} tǫ: sǫsádonhe:k.\\
not ever that.one you’ll.come.alive.again\\
\glt ‘You’ll never come alive again.’
\z

\ea
\label{ex:tpart51}
\gll \textbf{Tęˀ} \textbf{hwę:dǫ́h} to tǫ: ęshá:gyeˀ.\\
not ever that that.one he’ll.do.it.again\\
\glt ‘No, he’ll never do that.’
\z

\ea
\label{ex:tpart52}
\gll \textbf{Tęˀ} \textbf{hwę:dǫ́h} to tá:gyeˀ.\\
not ever that I.would.do.it\\
\glt ‘I would never do that.’
\z

\begin{CayugaRelated}
\item \textit{Hwę:dǫh} \trs{ever}, \trs{when}, p. \pageref{p:[hwę:dǫh]}\\
\item \textit{Tęˀ} \trs{no}, \trs{not}, p. \pageref{p:[tęˀ]}
\end{CayugaRelated}

\subsection*{\textbf{Tęˀ i:ˀ} \trs{not me}, \trs{not us}} \label{p:[tęˀ i:ˀ]}
\CayugaRemark{Particle group functioning as an “emphatic pronoun” (\sectref{ch:’Emphatic pronouns’}); free-standing.}


\ea
\label{ex:tpart53}
\gll \textbf{Tęˀ} \textbf{i:ˀ}.\\
not I\\
\glt ‘Not me.’ (Answering the question “Are you leaving?”)
\z

\begin{CayugaRelated}
\item \textit{I:ˀ, Ni:ˀ} \trs{I}, \trs{we}, p. \pageref{p:[i:ˀ]}\\
\item \textit{Tęˀ} \trs{no}, \trs{not}, p. \pageref{p:[tęˀ]}
\end{CayugaRelated}

\subsection*{\textbf{Tęˀ ne:ˀ deˀgę:} \trs{without}, \trs{lacking}} \label{p:[tęˀ ne:ˀ deˀgę:]}
\CayugaRemark{Particle group \stem{tęˀ…\textsc{noun} (ne:ˀ) de̱ˀgę:} conveys the idea that the noun or item is absent. For the positive counterpart, see \sectref{Equative sentences with linking verbs [né:ˀ], [deˀgę:]}.}

\ea
\label{ex:tpart54}
\gll \textbf{Tęˀ} (\textbf{ne:ˀ}) \textbf{deˀgę:}.\\
not (it.is) it.is.not\\
\glt ‘It isn’t the one’
\z

\ea
\label{ex:tpart55}
\gll \textbf{Tęˀ} [ohnaˀ] \textbf{deˀgę:}, shę́h ohę́: neˀ onáˀda:.\\
not grease it.is.not, that it.is.dry the bread\\
\glt ‘(The one with) no grease, because the bread is dry.’ (\cite{henry_de_2005})
\z

\begin{CayugaRelated}
\item \textit{Ne:ˀ} \trs{it is}, p. \pageref{p:[ne:ˀ] `it is’}\\
\item \textit{Tęˀ} \trs{no}, \trs{not}, p. \pageref{p:[tęˀ]}\\
\item \textit{Tęˀ de̱ˀgę:} \trs{it isn’t}, p. \pageref{p:[tęˀ deˀgę:]}
\end{CayugaRelated}


\subsection*{\textbf{Tęˀ seˀ} \trs{not really}, \trs{but then not really}} \label{p:[tęˀ seˀ]}
\CayugaRemark{Particle group signaling new information that is contrary to expectation or contrastive focus, \sectref{ch:Focus, contrastive focus, and emphasis markers}; free-standing.}

\ea
\label{ex:tpart56}
\gll I:wíˀ gę:s agatganǫ́:niˀ, de̱ˀęgwaheh \textbf{tęˀ} \textbf{seˀ}.\\
I.want usually I.am.wealthy, not.then not really\\
\glt ‘I want to be wealthy, but then not really.’
\z

\ea
\label{ex:tpart57}
\gll Wa̱ˀhéˀ ęgíhsa:k. \textbf{Tę́ˀ} \textbf{seˀ} sǫ:-gwáhs de̱ˀagatrehnagáˀ tsǫ:.\\
just.now I’ll.look.for.it not really anything-emphasis I.haven’t.unpacked just\\
\glt ‘I’ll just look for it. I am not really unpacked yet.’ (\cite[41]{mithun_watewayestanih_1984}, Dwęnǫhsanekahǫˀ dialogue)
\z

\begin{CayugaRelated}
\item \textit{Seˀ} \trs{you know}, p. \pageref{p:[seˀ]}\\
\item \textit{Tęˀ} \trs{no}, \trs{not}, p. \pageref{p:[tęˀ]}
\end{CayugaRelated}

\subsection*{\textbf{Tęˀ sgahoˀdęˀ} \trs{nothing}, \trs{not anything}} \label{p:[tęˀ sg̱ahoˀdęˀ]}
\CayugaRemark{Particle group \textit{(tęˀ) … sga̱hoˀdęˀ} functions as an “indefinite pronoun” (\sectref{ch:’Indefinite pronouns’}); free-standing.}

\ea
\label{ex:tpart58}
\gll \textbf{Tęˀ} \textbf{sgahoˀdę́ˀ} desę́nǫ̱hdǫˀ.\\
not anything you.do.not.know\\
\glt ‘No, you don’t know anything.’
\z

\ea
\label{ex:tpart59}
\gll A:yę́:ˀ giˀ hne:ˀ \textbf{Tęˀ} gwáhs \textbf{sgahoˀdęˀ}.\\
it.seems just in.fact not intensifier anything.\\
\glt ‘Oh, nothing much, I guess.’ (In response to \textbf{Dęˀ hoˀdę́ˀ ni̱hsagye̱h}aˀ? `What are you doing?’) (\cite[41]{mithun_watewayestanih_1984}, Ǫdwęnǫdáhtaˀ dialogue)
\z

\ea
\label{ex:tpart60}
\gll \textbf{Sgahoˀdę́ˀ} dawátehtgęht.\\
something it.went.bad\\
\glt ‘Something’s not right.’
\z

\subsection*{\textbf{Tęˀ sǫ:-ga:ˀ} \trs{nobody}, \trs{no one}} \label{p:[tęˀ sǫ:-ga:ˀ]}
\CayugaRemark{Particle group functioning as an “indefinite pronoun” (\sectref{ch:’Indefinite pronouns’}); free-standing.}

\ea
\label{ex:tpart61}
\gll \textbf{Tęˀ} \textbf{sǫgá:ˀ} de̱ˀagodekǫ́:nih.\\
not anyone no.one.is.eating\\
\glt ‘No one is eating.’(\cite[39]{michelson_ontario_2011})
\z

\begin{CayugaRelated}
\item \textit{Sǫ:-ga:ˀ} \trs{anyone}, \trs{any living thing}, p. \pageref{p:[sǫ:-ga:ˀah] ‘someone’}\\
\item \textit{Tęˀ} \trs{no}, \trs{not}, p. \pageref{p:[tęˀ]}
\end{CayugaRelated}

\subsection*{\textbf{Tęˀ ta:wa:dǫh} \trs{you may not}} \label{p:[tęˀ ta:wa:dǫh]}
\CayugaRemark{Particle and verb \stem{tęˀ … ta:wa:dǫh … \indefinite-\textsc{verb}} functions as a negative suggestion (\sectref{Negative suggestions}).}

\ea
\label{ex:tpart62}
\gll \textbf{Tęˀ} \textbf{ta:wa:dǫ́h} \textbf{a:}sejáodęˀ.\\
not it.shouldn’t.be.possible you.would.smoke\\
\glt ‘No smoking.’
\cfex{a:-s-e-já-odę-ˀ\\
\textsc{\indefinite-2s.a-\joinerE-}tobacco-put.in-\textsc{punctual}\\}
\z

\begin{CayugaRelated}
\item \textit{Tęˀ} \trs{no}, \trs{not}, p. \pageref{p:[tęˀ]}
\end{CayugaRelated}

\subsection*{\textbf{Tęˀ tǫ ne:ˀ, Tęˀ to ne:ˀ} \trs{not really}} \label{p:[tęˀ tǫ ne:ˀ]}
\CayugaRemark{Particle group signaling disagreement (\sectref{ch:Disagreement markers}); free-standing or clause-initial.}

\ea
\label{ex:tpart63}
\gll \textbf{Tęˀ} \textbf{tǫ} \textbf{ne:ˀ} dedó:gehs.\\
not that.one it.is it.is.not.true\\
\glt ‘No, it isn’t really true.’
\z

\ea
\label{ex:tpart64}
\gll Neˀ to ne:ˀ, [\textbf{tęˀ} \textbf{to} \textbf{ne:ˀ} detga:yéi:ˀ].\\
the that it.is, not that.one it.is it.is.not.right\\
\glt ‘That’s the one that’s ‘not all there.’’
\z

\begin{CayugaRelated}
\item \textit{Ne:ˀ} \trs{it is}, p. \pageref{p:[ne:ˀ] `it is’}\\
\item \textit{Tęˀ} \trs{no}, \trs{not}, p. \pageref{p:[tęˀ]}\\
\item \textit{To:, Toh, Tǫ:} \trs{that one}, p. \pageref{p:[to:] ‘that one’}
\end{CayugaRelated}

\subsection*{\textbf{Tgǫhaǫgyeˀ} \trs{sometimes}} \label{p:[tgǫhaǫgyeˀ]}
\CayugaRemark{Atypical verb functioning as an “adverb of time” (\sectref{ch:’Adverbs’ of time}); clause-initial.}

\ea
\label{ex:tpart65}
\gll A:yę́:ˀ gę:s \textbf{tgǫháǫgyeˀ} neˀ tęˀ gwáhs de̱ˀakniksaˀdí:yo:.\\
it.seems usually sometime the not really we.weren’t.good.children\\
\glt ‘I think that we weren’t always good children.’ (\cite{henry_de_2005})
\z


\begin{CayugaRelated}
\item \textit{Tgwahaǫ:ˀ} \trs{sometimes}, p. \pageref{p:[tgwahaǫ:ˀ]}\\
\item \textit{Tgwęhę:ˀ} \trs{sometimes}, p. \pageref{p:[tgwęhę:ˀ]}
\end{CayugaRelated}


\subsection*{\textbf{Tgwahaǫ:ˀ} \trs{sometimes}} \label{p:[tgwahaǫ:ˀ]}
\CayugaRemark{Atypical verb functioning as an “adverb of time” (\sectref{ch:’Adverbs’ of time}); clause-initial.}

\ea
\label{ex:tpart66}
\gll Ó: otgáˀdéˀ hne:ˀ gę:s neˀ \textbf{tgwáhaǫ:ˀ} hyeiˀ, ó: \textbf{tgwaháǫ:ˀ} já:dahk.\\
oh it.is.much in.fact usually the sometimes six, oh sometimes seven\\
\glt ‘Oh sometimes I used to catch as many as six or seven.’ (Henry, 2005)
\z

\begin{CayugaRelated}
\item \textit{Tgǫhaǫgyeˀ} \trs{sometimes}, p. \pageref{p:[tgǫhaǫgyeˀ]}\\
\item \textit{Tgwęhę:ˀ} \trs{sometimes}, p. \pageref{p:[tgwęhę:ˀ]}
\end{CayugaRelated}


\subsection*{\textbf{Tgwęhę:ˀ} \trs{sometimes}} \label{p:[tgwęhę:ˀ]}
\CayugaRemark{Atypical verb functioning as an “adverb of time” (\sectref{ch:’Adverbs’ of time}); clause-initial.}

\ea
\label{ex:tpart67}
\gll \textbf{Tgwęhę́:ˀ} hakdǫhs.\\
sometimes he.stops.back.in\\
\glt ‘Sometimes he comes back around.’
\z

\begin{CayugaRelated}
\item \textit{Tgǫhaǫgyeˀ} \trs{sometimes}, p. \pageref{p:[tgǫhaǫgyeˀ]}\\
\item \textit{Tgwahaǫ:ˀ} \trs{sometimes}, p. \pageref{p:[tgwahaǫ:ˀ]}
\end{CayugaRelated}


\subsection*{\textbf{To, To:, Toh, Tǫ:} \trs{that one}} \label{p:[to:] ‘that one’}
\CayugaRemark{Particle functioning as a “demonstrative pronoun” (\sectref{ch:’Demonstrative pronouns’}); free-standing.}

\ea
\label{ex:tpart68}
\gll Ahí:ˀ giˀ \textbf{toh} nęyá:węh.\\
I.thought just that.one it.will.happen\\
\glt ‘I thought that’s what would happen.’
\z

\ea
\label{ex:tpart69}
\gll Dęˀ hne:ˀ ni:yóht shęh \textbf{toh} na̱hsye:ˀ?\\
what in.fact the.reason that that.one you.do.it\\
\glt ‘Why in fact did you do that?’
\z

\begin{CayugaRelated}
\item \textit{Da: neˀ toh} \trs{that’s all}, p. \pageref{p:[da: neˀ toh]}\\
\item \textit{Gwa̱ˀ toh} \trs{that one}, \trs{just such a one}, p. \pageref{p:[gwaˀ toh]}\\
\item \textit{Neˀ gwaˀ toh} \trs{here (rather than there)}, p. \pageref{p:[neˀ gwaˀ toh]}\\
\item \textit{Neˀ to: ne:ˀ} \trs{that’s the one}, p. \pageref{p:[neˀ to: ne:ˀ]}\\
\item \textit{Neˀ toh} \trs{that is}, \trs{that one}, p. \pageref{p:[neˀ toh]}\\
\item \textit{Neˀ tsǫ: gwaˀ toh} \trs{also}, p. \pageref{p:[neˀ tsǫ: gwaˀ toh]}\\
\item \textit{Nę: toh} \trs{here}, p. \pageref{p:[nę: toh]}\\
\item \textit{Nę: toh gwa:dih} \trs{on this side}, p. \pageref{p:[nę: toh gwa:dih]}\\
\item \textit{Nę: tsǫ: gwaˀ toh ni-} \trs{just a little bit}, \trs{very little}, p. \pageref{p:[nę: tsǫ: gwaˀ toh ni-]}\\
\item \textit{Tęˀ gaˀ-toh} \trs{nowhere}, \trs{not anywhere}, p. \pageref{p:[tęˀ gaˀ-toh]}\\
\item \textit{Tęˀ gwa̱ˀ-toh} \trs{none at all}, p. \pageref{p:tęˀ gwa̱ˀ-toh}\\
\item \textit{Tęˀ tǫ ne:ˀ, Tęˀ to ne:ˀ} \trs{not really}, p. \pageref{p:[tęˀ tǫ ne:ˀ]}\\
\item \textit{To:, Toh, Tǫ:} \trs{that one}, p. \pageref{p:[to:] ‘that one’}\\
\item \textit{To:, Toh, Tǫ:} \trs{there}, \trs{that place}, p. \pageref{p:[to:] ‘there’, ‘that place’}\\
\item \textit{To gęh ǫ ne:ˀ} \trs{I wonder if it is}, p. \pageref{p:[to gęh ǫ ne:ˀ]}\\
\item \textit{To: gwa:dih} \trs{on that side}, p. \pageref{p:[to: gwa:dih]}\\
\item \textit{To hę:gyeh shęh} \trs{really}, p. \pageref{p:[to hę:gyeh shęh]}\\
\item \textit{To: hǫ:weh} \trs{there}, \trs{where}, p. \pageref{p:[to: hǫ:weh]}\\
\item \textit{Toh niyo:weˀ} \trs{that far}, p. \pageref{p:[toh niyo:weˀ]}\\
\item \textit{To: niyo:weˀ ne:ˀ} \trs{when}, p. \pageref{p:[to: niyo:weˀ ne:ˀ]}\\
\item \textit{To: niyonisheˀ to:} \trs{a certain amount of time}, p. \pageref{p:[to: niyonisheˀ to:]}\\
\item \textit{Toh shęh ni-} \trs{to that degree}, p. \pageref{p:[toh shęh ni-]}\\
\item \textit{To tsǫ:} \trs{that’s all}, p. \pageref{p:[to tsǫ:] ‘that’s all’}\\
\item \textit{To tsǫ:} \trs{that’s enough}, p. \pageref{p:[to tsǫ:] ‘that’s enough!’}\\
\item \textit{To:-gyęh} \trs{that one}, p. \pageref{p:[to:-gyęh]}\\
\item \textit{Toh-geh} \trs{there, then}, p. \pageref{p:[toh-geh] ‘there, ‘then’}\\
\item \textit{Toh-geh} \trs{and then}, p. \pageref{p:[toh-geh] ‘and then’}\\
\item \textit{Toh-geh hǫ:weh} \trs{where}, p. \pageref{p:[toh-geh hǫ:weh]}\\
\item \textit{Toh-jih} \trs{just the one}, \trs{just such a one}, p. \pageref{p:[toh-jih]}
\end{CayugaRelated}


\subsection*{\textbf{To, To:, Toh, Tǫ:} \trs{there}, \trs{that place}} \label{p:[to:] ‘there’, ‘that place’}
\CayugaRemark{Particle functioning as an “adverb of place” (\sectref{‘Adverbs’ of place}), and possibly, as “adverb of time” (\sectref{ch:’Adverbs’ of time}) meaning \trs{then}; clause-initial.}

\ea
\label{ex:tpart70}
\gll Dewagegaę́hs \textbf{to:} há:ge:ˀ.\\
I’m.unwilling that.one I.will.go.there\\
\glt ‘I am unwilling to go there.’
\z

\begin{CayugaRelated}
\item \textit{To:, Toh, Tǫ:} \trs{that one}, p. \pageref{p:[to:] ‘that one’}
\end{CayugaRelated}


\subsection*{\textbf{To gęh ǫ ne:ˀ} \trs{is it that?}, \trs{if it is that}} \label{p:[to gęh ǫ ne:ˀ]}
\CayugaRemark{Particle group \textit{to (gęh) (ǫ) (ne:ˀ)} introduces conditional clauses \xxref{ex:tpart71}{ex:tpart74}, \sectref{ch:Conditional clauses with [gyę:gwaˀ], [gęh], [hę:gyeh] ‘if, whether’}; clause-initial or free-standing. \textit{Gęh ǫ} is also spelled as \textit{gęˀǫ}, \textit{gęǫ} or \textit{gaǫ:}, and \textit{ǫ} is also spelled as \textit{ˀǫ}, \textit{ǫh}, or \textit{ǫ:}.}

\ea
\label{ex:tpart71}
\gll \textbf{O} \textbf{gá-ǫ:/gę́-ǫ:} \textbf{ne:ˀ} ní:yoht?\\
that Q-I.guess it.is it.is.a.certain.way\\
\glt ‘I wonder if it is like that?’
\z

\ea
\label{ex:tpart72}
\gll \textbf{To} \textbf{gę-ˀǫ} ní:yoht?\\
that Q-I.guess it.is.a.certain.way\\
\glt ‘I wonder if it is like that?’
\z

\ea
\label{ex:tpart73}
\gll \textbf{To} \textbf{ˀǫ} ní:yoht?\\
that I.guess it.is.a.certain.way\\
\glt ‘I wonder if it is like that?’
\z

\ea
\label{ex:tpart74}
\gll Ęsgoho:wíˀ [\textbf{to} \textbf{gęh} \textbf{ne:ˀ} hę́:yę:ˀ].\\
I.will.tell.you that Q it.is she.will.go.there\\
\glt ‘I will tell you if she is going.’
\z

\begin{CayugaRelated}
\item \textit{Gęh} \trs{Q} (question marker), p. \pageref{p:[gęh] `Q’}\\
\item \textit{Gęh oh, gę-ǫ:, ga-ǫ: ne:ˀ} \trs{is it that?}, \trs{if it is that}, p. \pageref{p:[gęh oh, gę-ǫ:]}\\
\item \textit{Oh, ǫ:, ǫ} \trs{I guess}, \trs{I wonder (if)}, p. \pageref{p:[ǫh, ǫ:, ǫ]}\\
\item \textit{Oh ne:ˀ} \trs{maybe}, p. \pageref{p:[ǫh ne:ˀ]}\\
\item \textit{To:, Toh, Tǫ:} \trs{that one}, p. \pageref{p:[to:] ‘that one’}
\end{CayugaRelated}


\subsection*{\textbf{To gwa:dih} \trs{on that side}} \label{p:[to: gwa:dih]}
\CayugaRemark{Particle group functioning as an “adverb of place” (\sectref{‘Adverbs’ of place}); likely clause-initial. \textit{Gwa:dih} can also be spelled or pronounced as \textit{gwai}.}

\ea
\label{ex:tpart81}
\gll \textbf{to:} \textbf{gwa:dih}\\
that.one side\\
\glt ‘on that side’
\z

\begin{CayugaRelated}
\item \textit{Gwa:dih, gwai} \trs{to one side}, p. \pageref{p:[gwa:dih]}\\
\item \textit{To:, Toh, Tǫ:} \trs{that one}, p. \pageref{p:[to:] ‘that one’}
\end{CayugaRelated}


\subsection*{\textbf{To hę:gyeh shęh} \trs{really}, \trs{very}} \label{p:[to hę:gyeh shęh]}
\CayugaRemark{Particle group functioning as an “adverb of degree” (\sectref{ch:’Adverbs’ of degree}); clause-initial.}

\ea
\label{ex:tpart89}
\gll \textbf{To} \textbf{hęgyeh} \textbf{shęh} age:ji:yoˀ.\\
that no.matter that I’m.lame\\
\glt ‘I am really crippled.’
\z

\begin{CayugaRelated}
\item \textit{Hę:-gyęh, hę:gyeh} \trs{no matter}, p. \pageref{p:[hę:-gyęh]}\\
\item \textit{Hę:-gyęh shęh} \trs{no matter how much}, \trs{whether or not}, \trs{even if}, p. \pageref{p:[hę:-gyęh shęh]}\\
\item \textit{To:, Toh, Tǫ:} \trs{that one}, p. \pageref{p:[to:] ‘that one’}
\end{CayugaRelated}


\subsection*{\textbf{To hǫ:weh} \trs{there}, \trs{where}} \label{p:[to: hǫ:weh]}
\CayugaRemark{Particle group functioning as an “adverb of place” (\sectref{‘Adverbs’ of place}); clause-initial. \textit{Hǫ:weh} can be spelled or pronounced as \textit{hǫ:}, \textit{nhǫ:}, \textit{hǫ:weh}, \textit{nhǫ:weh}.}

\ea
\label{ex:tpart90}
\gll \textbf{To} \textbf{hǫ:} hę́:geˀ.\\
there place I.will.go\\
\glt ‘I will go there.’
\z

\ea
\label{ex:tpart91}
\gll \textbf{To} \textbf{hǫ:} hayę́hęˀ.\\
there place he.puts.it.there\\
\glt ‘He is always putting it there.’
\z

\ea
\label{ex:tpart92}
\gll \textbf{To} \textbf{hǫ:} sré:hah.\\
there place you.place.it\\
\glt ‘That’s where you put it.’
\z

\begin{CayugaRelated}
\item \textit{Hǫ:weh} \trs{where}, p. \pageref{p:[hǫ:weh]}\\
\item \textit{To:, Toh, Tǫ:} \trs{that one}, p. \pageref{p:[to:] ‘that one’}
\end{CayugaRelated}

\subsection*{\textbf{To ne:ˀ, Neˀ to ne:ˀ} \trs{that’s the one}} \label{p:[to: ne:ˀ]}
\CayugaRemark{Particle group \textit{(neˀ) to: … ne:ˀ} functions as a “demonstrative pronoun” (\sectref{ch:’Demonstrative pronouns’}); emphatic; clause-initial.}

\begin{CayugaRelated}
\item \textit{Neˀ to: ne:ˀ} \trs{that’s the one}, p. \pageref{p:[neˀ to: ne:ˀ]}
\end{CayugaRelated}


\subsection*{\textbf{To niyo:weˀ} \trs{that far}} \label{p:[toh niyo:weˀ]}
\CayugaRemark{Particle and verb \textit{toh … niyo:weˀ} is used in comparisons (\sectref{ch:Comparisons (more, the same, or less)}). \textit{Niyó:weˀ} also has the short forms \textit{ní:yo:ˀ} or \textit{nyo:}.}

\ea
\label{ex:tpart93}
\gll \textbf{To} \textbf{niyo:wéˀ} hęˀséˀ waˀhéˀ hęhsyǫˀ.\\
that a.certain.distance you.will.be.there time you.will.arrive.there\\
\glt ‘You’ll get that far before you arrive.’
\z

\ea
\label{ex:tpart94}
\gll \textbf{Toh} tsǫ: \textbf{ni:yo:wéˀ} hęsę́ˀdrǫ:ˀ.\\
that.one just a.certain.distance you’ll.be.driven.over.there\\
\glt ‘That’s only as far as it will take you.’
\z

\ea
\label{ex:tpart95}
\gll \textbf{To:} \textbf{ni:yó:ˀ} neˀ heha:wé:noh.\\
that a.certain.distance  the he.went.over.there\\
\glt ‘That’s as far as he has gone.’
\z

\ea
\label{ex:tpart96}
\gll “A:yę́:ˀ, shęh nyó:ˀ haˀhéˀ dedwatahaˀsé:ˀ, \textbf{to} \textbf{nyo:wéˀ},” aˀa:gę́ˀ neˀ ǫgyaˀsé:ˀ, aˀa:gę́ˀ.\\
it.seems that a.certain.distance he.went.there the.crossroad, that a.certain.distance she.said the we.are.cousins she.said\\
\glt ‘“It seems, as far as he went on the crossroad, that far (we will go),” said my cousin.’ (\cite{henry_de_2005})
\z

\ea
\label{ex:tpart97}
\gll \textbf{Tó} tsǫ: \textbf{ni:yǫ:} ne:ˀ ahi:ˀ aeswadáhǫdęh.\\
that just a.certain.distance it.is I.thought you.would.hear\\
\glt ‘And that is about as much as I thought you would like to hear.’ (\cite{mithun_how_1980})
\z

\begin{CayugaRelated}
\item \textit{Gaoˀ shęh niyo:weˀ} \trs{before}, \trs{until}, p. \pageref{p:[gaoˀ shęh niyo:weˀ]}\\
\item \textit{Niyo:weˀ, ni:yo:ˀ, nyo:ˀ} \trs{a certain distance}, \trs{a certain time}, p. \pageref{p:[niyo:weˀ]}\\
\item \textit{Shęh niyo:weˀ} \trs{as far as}, \trs{as much as}, p. \pageref{p:[shęh niyo:weˀ]}\\
\item \textit{To:, Toh, Tǫ:} \trs{that one}, p. \pageref{p:[to:] ‘that one’}\\
\item \textit{To: niyo:weˀ ne:ˀ} \trs{when}, p. \pageref{p:[to: niyo:weˀ ne:ˀ]}
\end{CayugaRelated}



\subsection*{\textbf{To niyo:weˀ ne:ˀ} \trs{when}} \label{p:[to: niyo:weˀ ne:ˀ]}
\CayugaRemark{Particles and verb, \textit{to: … niyo:(weˀ) ne:ˀ} functions as an “adverb of time” (\sectref{ch:’Adverbs’ of time}); clause-initial. \textit{Niyó:weˀ} also has the short forms \textit{ní:yo:ˀ} or \textit{nyo:}.}

\ea
\label{ex:tpart98}
\gll \textbf{To:} \textbf{ni:yó:} ne:ˀ hesheˀs.\\
that a.certain.amount.of.time it.is you’ll.come.around.again\\
\glt ‘That’s when your birthday is.’
\z

\largerpage
\begin{CayugaRelated}
\item \textit{Gaoˀ shęh niyo:weˀ} \trs{before}, \trs{until}, p. \pageref{p:[gaoˀ shęh niyo:weˀ]}\\
\item \textit{Niyo:weˀ, ni:yo:ˀ, nyo:ˀ} \trs{a certain distance}, \trs{a certain time}, p. \pageref{p:[niyo:weˀ]}\\
\item \textit{Shęh niyo:weˀ} \trs{as far as}, \trs{as much as}, p. \pageref{p:[shęh niyo:weˀ]}\\
\item \textit{To:, Toh, Tǫ:} \trs{that one}, p. \pageref{p:[to:] ‘that one’}\\
\item \textit{Toh niyo:weˀ} \trs{that far}, p. \pageref{p:[toh niyo:weˀ]}
\end{CayugaRelated}


\subsection*{\textbf{To niyonisheˀ to:} \trs{a certain amount of time}} \label{p:[to: niyonisheˀ to:]}
\CayugaRemark{Particle plus verb and particle, \textit{to … niyonisheˀ to:} functions as an “adverb of time” (\sectref{ch:’Adverbs’ of time}, \sectref{ch:Clauses with [shęh naˀonisheˀ], [tsaˀonisheˀ], [tsi-] ‘while, when’}); clause-initial.}

\ea
\label{ex:tpart99}
\gll “\textbf{Toh} tsǫ́: \textbf{niyonishéˀ} to ęshekni̱ˀnigǫ̱há:k neˀ shakníhso:t.”\\
that.one just it.takes.time that.one we.will.wait.for.him the our.grandfather\\
\glt “We’ll wait for our grandpa there only as long as that.” (\cite{henry_de_2005})
\z

\begin{CayugaRelated}
\item \textit{To:, Toh, Tǫ:} \trs{that one}, p. \pageref{p:[to:] ‘that one’}
\end{CayugaRelated}


\subsection*{\textbf{To shęh ni-, To ni-} \trs{to that degree}} \label{p:[toh shęh ni-]}
\CayugaRemark{Particle group \stem{toh … (shęh) ni-\textsc{stative.verb}} is used in comparisons (\sectref{ch:Comparisons (more, the same, or less)}); precedes a stative verb beginning with \stem{ni-} {\partitive}.}

\ea
\label{ex:tpart100}
\gll “\textbf{Toh} tsǫ́: \textbf{niyonishéˀ} to ęsheknigǫ̱há:k neˀ shakníhso:t.”\\
that.one just an.amount.of.time there we.will.wait.for.him the our.grandfather\\
\glt ‘We’ll wait for our grandpa there only as long as that.’ (\cite{henry_de_2005})
\z

\ea
\label{ex:tpart101}
\gll \textbf{Tó} \textbf{shęh} \textbf{ni}wáˀ néˀ ogya:nǫ́hk akyę́da̱hkwaˀ.\\
that.one that a.certain.size the it.is.magic chair\\
\glt ‘The magic chair was just the right size for the little boy.’ (\cite{keye_circle_2016}, Circle Book 10, The Magic Chair)
\z

\begin{CayugaRelated}
\item \textit{Shęh} \trs{that}, \trs{because}, p. \pageref{p:[shęh] `that’}\\
\item \textit{To:, Toh, Tǫ:} \trs{that one}, p. \pageref{p:[to:] ‘that one’}
\end{CayugaRelated}


\subsection*{\textbf{To tsǫ:} \trs{that’s enough}} \label{p:[to tsǫ:] ‘that’s enough!’}
\CayugaRemark{Particle group and expression used when asking someone to stop pouring a drink or providing food.}

\ea
\label{ex:tpart102}
\gll \textbf{To} \textbf{tsǫ:}!\\
that just\\
\glt ‘That’s enough!’
\z

\begin{CayugaRelated}
\item \textit{To:, Toh, Tǫ:} \trs{that one}, p. \pageref{p:[to:] ‘that one’}\\
\item \textit{To tsǫ:} \trs{that’s all}, p. \pageref{p:[to tsǫ:] ‘that’s all’}\\
\item \textit{Tsǫ:} \trs{just}, \trs{only}, p. \pageref{p:[tsǫ:]}
\end{CayugaRelated}


\subsection*{\trs{To tsǫ:} \trs{that’s all}, \trs{that’s it}} \label{p:[to tsǫ:] ‘that’s all’}
\CayugaRemark{Particle group signalling the end of a topic or conversation (\sectref{ch:Topic starters and conclusions}); free-standing.}

\ea
\label{ex:tpart103}
\gll \textbf{To} \textbf{tsǫ:}.\\
that just\\
\glt ‘That’s all.’, ‘That’s it.’
\z

\begin{CayugaRelated}
\item \textit{To:, Toh, Tǫ:} \trs{that one}, p. \pageref{p:[to:] ‘that one’}\\
\item \textit{To tsǫ:} \trs{that’s enough}, p. \pageref{p:[to tsǫ:] ‘that’s enough!’}\\
\item \textit{Tsǫ:} \trs{just}, \trs{only}, p. \pageref{p:[tsǫ:]}
\end{CayugaRelated}


\subsection*{\textbf{To:-gyęh} \trs{that one}} \label{p:[to:-gyęh]}
\CayugaRemark{Particle group functioning as a “demonstrative pronoun” (\sectref{ch:’Demonstrative pronouns’}); free-standing, or preceding the word it modifies (as in \textit{to:-gyę́h hǫgweˀdasé:ˀah} ‘that young man’).}

\ea
\label{ex:tpart82}
\gll Tęˀ tǫ:desa̱ˀdré: \textbf{tó:-gyęh}!\\
not do.not.drive that.one\\
\glt ‘Don’t drive that one over here!’
\z

\ea
\label{ex:tpart83}
\gll Sęnihę́: \textbf{tó:-gyęh}!\\
quit.it that.one\\
\glt ‘Quit that!’
\z

\ea
\label{ex:tpart84}
\gll Ne:ˀ gyę:ˀ neˀ \textbf{tó:-gyęh}.\\
it.is this.one the that.one\\
\glt ‘That is the one.’
\z

\ea
\label{ex:tpart85}
\gll Sǫ: hne:ˀ nˀáht \textbf{tó:-gyęh}?\\
Who in.fact a.certain.person that.one\\
\glt ‘Who is that?’ (\cite[88]{mithun_watewayestanih_1984}, Eksaˀgó:wah dialogue)
\z

\ea
\label{ex:tpart86}
\gll \textbf{To:-gyęh} hǫgweˀdase:ˀáh shǫgwahyadǫ́:nih.\\
that.one young.man he.writes.for.us\\
\glt ‘That young man does our writing.’
\z

\ea
\label{ex:tpart87}
\gll Ne:ˀ seˀ gyę:ˀ gę:s \textbf{to:-gyę́h} hwaˀ nęh sǫheh nęh gadidaksénǫgyeˀs shęh nhǫ: ohádenyǫˀ.\\
it.is you.know this.one usually that time when night when they.run.around that where existing.roads\\
\glt ‘This is how it is, you know, at night when they run around their roads.’ (\cite{mithun_how_1980})
\z

\begin{CayugaRelated}
\item \stem{-gyęh} \trs{this}, \trs{that}, p. \pageref{p:[-gyęh]}\\
\item \textit{To:, Toh, Tǫ:} \trs{that one}, p. \pageref{p:[to:] ‘that one’}
\end{CayugaRelated}


\subsection*{\textbf{To:-hwaˀ, Tǫ:-hwaˀ} \trs{that time}} \label{p:[tǫ:-hwaˀ]}
\CayugaRemark{Particle group functioning as an “adverb of time” (\sectref{ch:’Adverbs’ of time}); emphatic; clause-initial.}

\ea
\label{ex:tpart104}
\gll \textbf{Tǫ:-hwaˀ} gí:dǫh.\\
that.one-that.time I.mean.it\\
\glt ‘I mean it that time.’
\z

\begin{CayugaRelated}
\item \textit{Hwaˀ} \trs{this time}, \trs{next}, p. \pageref{p:[hwaˀ]}\\
\item \textit{To:, Toh, Tǫ:} \trs{that one}, p. \pageref{p:[to:] ‘that one’}
\end{CayugaRelated}


\subsection*{\textbf{To:hah} \trs{a place}, \trs{a time}} \label{p:[to:hah]}
\CayugaRemark{Particle functioning as an “adverb of time” (\sectref{ch:’Adverbs’ of time}).}

\ea
\label{ex:tpart88}
\gll \textbf{to:-hah}\\
that(one.)-diminutive\\
\glt ‘a place’, ‘a time’
\z

\begin{CayugaRelated}
\item \textit{Gaˀ-to:hah} \trs{somewhere}, \trs{someplace}, \trs{around}, \trs{anywhere}, \trs{thereabouts}, p. \pageref{p:[gaˀ-to:hah]}\\
\item \textit{Gaˀ-to:hah tohgeh} \trs{thereabouts}, p. \pageref{p:[gaˀ-to:hah tohgeh]}\\
\item \textit{O:nęh to:hah} \trs{soon}, \trs{almost}, p. \pageref{p:o:nęh to:hah]}\\
\item \textit{To:, Toh, Tǫ:} \trs{that one}, p. \pageref{p:[to:] ‘that one’}\\
\item \textit{To:, Toh, Tǫ:} \trs{there}, \trs{that place}, p. \pageref{p:[to:] ‘there’, ‘that place’}
\end{CayugaRelated}

\subsection*{\textbf{Toh-geh} \trs{there}, \trs{then}} \label{p:[toh-geh] ‘there, ‘then’}
\CayugaRemark{Particle group functioning as an “adverb of place” (\sectref{‘Adverbs’ of place}) meaning \trs{there}, or as an “adverb of time” (\sectref{ch:’Adverbs’ of time}) meaning \trs{then}; clause-initial.}

\ea
\label{ex:tpart75}
\gll Ętsá:ˀ, \textbf{tohgéh} hętsá:ˀ gwa̱ˀ tóh nęh ęgéhsda̱hsiˀ.\\
you.will.take.it, there you.will.take.it.there emphasis there when I.have.finished.with.them\\
\glt ‘You can take them back when I have finished with them. (\cite[368]{mithun_watewayestanih_1984}, Degahenáˀtraˀse:ˀ dialogue)
\z

\begin{CayugaRelated}
\item \textit{To:, Toh, Tǫ:} \trs{there}, \trs{that place}, p. \pageref{p:[to:] ‘there’, ‘that place’}\\
\item \textit{Toh-geh hǫ:weh} \trs{where}, p. \pageref{p:[toh-geh hǫ:weh]}
\end{CayugaRelated}

\subsection*{\textbf{Toh-geh} \trs{and then}} \label{p:[toh-geh] ‘and then’}
\CayugaRemark{Particle signaling the continuation of a previous topic (\sectref{ch:Topic continuation markers}); clause-initial.}

\ea
\label{ex:tpart76}
\gll Eyaˀdá:t gonǫhgę́:t, [\textbf{toh-géh} jeyaˀdá:t gogéˀaji:]\\
this.one.woman she.is.fair-haired that.one-on this.one.woman she.has.dark.hair\\
\glt ‘One is fair, and the other has dark hair.’ (\cite[88]{mithun_watewayestanih_1984}, Eksaˀgó:wah dialogue)
\z

\ea
\label{ex:tpart77}
\gll \textbf{Toh-géh}, í:ˀ hné:ˀ hęgǫhawihdę́ˀ sadę́na̱ˀtraˀ.\\
that.one-on I in.fact I.will.carry.for.you your.lunch\\
\glt ‘Then I’ll carry your lunch for you.’ (\cite[494]{mithun_watewayestanih_1984}, Ękníyętoˀ dialogue)
\z

\ea
\label{ex:tpart78}
\gll \textbf{Toh-géh} neˀ agǫ:gwéh, né:ˀ hęˀ hne:ˀ deyagyadęhnǫ́:de:ˀ.\\
that.one-on the woman, it.is also in.fact she.is.my.sister\\
\glt ‘And the woman, she is my sister.’ (\cite[257]{mithun_watewayestanih_1984}, Ga̱hwajiyá:deˀ dialogue)
\z

\begin{CayugaRelated}
\item \textit{To:, Toh, Tǫ:} \trs{that one}, p. \pageref{p:[to:] ‘that one’}\\
\item \textit{Toh-geh} \trs{there}, p. \pageref{p:[toh-geh] ‘there, ‘then’}
\end{CayugaRelated}

\subsection*{\textbf{Toh-geh hǫ:weh} \trs{where}} \label{p:[toh-geh hǫ:weh]}
\CayugaRemark{Particle group functioning as an “adverb of place” (\sectref{‘Adverbs’ of place}), \sectref{ch:Clauses with [hǫ:(weh)] ‘the place where’}; clause-initial in direct questions. \textit{Hǫ:weh} can be spelled or pronounced as \textit{hǫ:}, \textit{nhǫ:}, \textit{hǫ:weh}, or \textit{nhǫ:weh}.}

\ea
\label{ex:tpart79}
\gll \textbf{Toh-géh} gęh \textbf{hǫ:wéh} toh naˀá:weh?\\
that.one-on Q place there something.happened\\
\glt ‘Where did it happen?’
\z

\begin{CayugaRelated}
\item \textit{Hǫ:weh} \trs{where}, p. \pageref{p:[hǫ:weh]}\\
\item \textit{To:, Toh, Tǫ:} \trs{there}, \trs{that place}, p. \pageref{p:[to:] ‘there’, ‘that place’}\\
\item \textit{Toh-geh} \trs{there}, p. \pageref{p:[toh-geh] ‘there, ‘then’}\\
\item \textit{Toh-geh} \trs{and then}, p. \pageref{p:[toh-geh] ‘and then’}
\end{CayugaRelated}

\subsection*{\textbf{Toh-jih} \trs{just the one}, \trs{just such a one}} \label{p:[toh-jih]}
\CayugaRemark{Particle functioning as a “demonstrative pronoun” (\sectref{ch:’Demonstrative pronouns’}); emphatic; free-standing. Includes the \stem{-jih} {\intensifier} (\sectref{-jih}).}

\ea
\label{ex:tpart80}
\gll Né:ˀ a:yę́:ˀ \textbf{toh-jíh} shęh ní:waˀs.\\
it.is it.seems that.one-intensifier that it.is.a.certain.size\\
\glt ‘It seems to be just the right size.’ (\cite[225]{mithun_watewayestanih_1984}, Agyaˀdawíˀtraˀ dialogue)
\z

\begin{CayugaRelated}
\item \textit{To:, Toh, Tǫ:} \trs{that one}, p. \pageref{p:[to:] ‘that one’}
\end{CayugaRelated}



\subsection*{\textbf{Trehs, Ji trehs} \trs{too much}} \label{p:[trehs, ji trehs]}
\CayugaRemark{Particle group \textit{(ji) trehs} functions as an “adverb of degree” (\sectref{ch:’Adverbs’ of degree}); clause-initial.}

\ea
\label{ex:tpart105}
\gll …\textbf{tréhs} a:yę́:ˀ ǫknigǫ̱hsá:dǫˀk.\\
…too.much.so it.seems I.got.lonesome\\
\glt ‘ …I seem to get too lonesome.’ (\cite[88]{mithun_watewayestanih_1984}, Eksaˀgó:wah dialogue)
\z

\ea
\label{ex:tpart106}
\gll \textbf{Jiˀ} gę:s \textbf{trehs} shenó:wę:.\\
too usually too.much.so you.lie\\
\glt ‘You lie too much.’
\z

\begin{CayugaRelated}
\item \textit{Ji} \trs{too much}, p. \pageref{p:[ji]}\\
\item \textit{Ji trehs} \trs{more (than usual)}, \trs{too much so}, p. \pageref{p:[ji trehs] ‘more (than usual)’}\\
\item \textit{Ji trehs} \trs{because}, \trs{overly}, p. \pageref{p:[ji trehs] ‘because’}\\
\item \textit{Ji trehs shęh} \trs{too much so for}, \trs{so much so}, p. \pageref{p:[ji trehs shęh]}\\
\item \textit{Trehs giˀ gyę:ˀ} \trs{my goodness}, \trs{too bad}, \trs{that’s amazing}, p. \pageref{p:[trehs giˀ gyę:ˀ]}
\end{CayugaRelated}


\subsection*{\textbf{Trehs giˀ gyę:ˀ} \trs{my goodness!}, \trs{too bad!}, \trs{that’s amazing!}} \label{p:[trehs giˀ gyę:ˀ]}
\CayugaRemark{Particle group functioning as an “exclamation” (\sectref{ch:Exclamations}), conveying surprise or disappointment.}

\ea
\label{ex:tpart107}
\gll O:, \textbf{tréhs} \textbf{giˀ} \textbf{gyę́:ˀ} agayeshaˀ, osdaǫgyǫ́: hné:ˀ ó:nęh.\\
oh, too.much just this.one it.is.too.bad, it.is.raining in.fact now\\
\glt ‘Oh, that’s too bad, it is raining now.’ (\cite[212]{mithun_watewayestanih_1984}, Satrǫ́:nih dialogue)
\z

\ea
\label{ex:tpart108}
\gll Ó:o:. \textbf{tréhs-giˀ-gyę́:ˀ} do:gę́hs sadraˀswahé:tgęˀ, shęh ne̱ˀsaya̱ˀda:wę́h wáˀne:ˀ.\\
oh, it.is.amazing it.is.true your.luck.turned.bad that something.happened.to.you today\\
\glt ‘Oh. It sure is amazing, how bad your luck turned, with all that happened to you today.’ (\cite[386]{mithun_watewayestanih_1984}, Ga̱há:gǫ: dialogue)
\z

\begin{CayugaRelated}
\item \textit{Giˀ} \trs{just}, p. \pageref{p:[giˀ]}\\
\item \textit{Gyę:ˀ} \trs{just the one}, p. \pageref{p:[gyę:ˀ]}\\
\item \textit{Ji trehs} \trs{more (than usual)}, \trs{too much so}, p. \pageref{p:[ji trehs] ‘more (than usual)’}\\
\item \textit{Ji trehs} \trs{because}, \trs{overly}, p. \pageref{p:[ji trehs] ‘because’}
\end{CayugaRelated}



\subsection*{\textbf{Trǫhgeh tsǫ:} \trs{barely}} \label{p:[trǫhgeh tsǫ:]}
\CayugaRemark{Particle group functioning as an “adverb of degree” (\sectref{ch:’Adverbs’ of degree}); emphatic; clause-initial.}

\ea
\label{ex:tpart109}
\gll \textbf{Trǫhgeh} \textbf{tsǫ:} ǫgwahdęgyáˀdǫh.\\
barely just we.begin\\
\glt ‘We’re just barely able to make it go.’ (speaking of a ceremony, etc.)
\z

\ea
\label{ex:tpart110}
\gll \textbf{Trǫhgeh} \textbf{tsǫ:} jǫgwaihóˀdeˀ.\\
barely just we.work.again\\
\glt ‘We’re barely working.’
\z

\begin{CayugaRelated}
\item \textit{Tsǫ:} \trs{just}, \trs{only}, p. \pageref{p:[tsǫ:]}
\end{CayugaRelated}


\subsection*{\textbf{Tsaˀonisheˀ, shęh naˀonisheˀ} \trs{while}, \trs{when}} \label{p:[tsaˀonisheˀ]}
\CayugaRemark{Particle and verb beginning with \stem{ni-} {\partitive} or \stem{tsi-} \textsc{\coincident} (\sectref{ch:Clauses with [shęh naˀonisheˀ], [tsaˀonisheˀ], [tsi-] ‘while, when’}); clause-initial in dependent clauses.}

\ea
\label{ex:tpart111}
\gll Agiˀda̱ˀǫ́h ǫh [\textbf{shęh} \textbf{naˀonishéˀ} hohta:ˀ].\\
I.slept I.guess that it.is.a.certain.time he.is.speaking\\
\glt ‘I slept while he spoke.’
\z

\ea
\label{ex:tpart112}
\gll Hohsę́: [\textbf{tsaˀonihseˀ} hǫgwéˀdase:].\\
he.was.fat while he.is.a.young.man\\
\glt ‘While he was young, he was fat.’
\z

\ea
\label{ex:tpart113}
\gll [\textbf{Tsi}gǫgwe̱ˀda:sé:] gę́:s ga̱hyagwa̱hsgę́hę:ˀ.\\
when.I.was.young usually I.would.pick.berries\\
\glt ‘When I was a young person I usually picked berries.’
\z

\ea
\label{ex:tpart114}
\gll Knǫhwéˀs gę:s [\textbf{tsi}gǫgwe̱ˀda:séˀ] tiga:gwe:gǫ́h gę́:s agahyagwęhęgyéˀsgę̱hę:ˀ\\
I.love.it usually when.I.was.young all.over usually I.would.go.along.picking.fruit\\
\glt ‘When I was young, I used to love going all over and picking fruit.’
\z

\begin{CayugaRelated}
\item \textit{Shęh} \trs{that}, \trs{because}, p. \pageref{p:[shęh] `that’}\\
\item \textit{shęh naˀonisheˀ, tsaˀonisheˀ}, \stem{tsi-} \trs{while}, \trs{when}, p. \pageref{p:[shęh naˀonisheˀ]}
\end{CayugaRelated}


\subsection*{\textbf{Tsę:} \trs{oh my!}} \label{p:[tsę:]}
\CayugaRemark{Particle functioning as an “exclamation” (\sectref{ch:Exclamations}), conveying mild disgust or exasperation; pronounced as [tsæ:] ([æ] sounds like the A in ‘tack’).}

\ea
\label{ex:tpart115}
\gll \textbf{Tsę́:} e:ˀ neˀ satró:wi:.\\
oh.my again the you.are.talking\\
\glt ‘You’re talking about that again!’ (expression of disgust)
\z

\ea
\label{ex:tpart116}
\gll “O: \textbf{tsę:},” aˀa:gę́ˀ, “ahsdéh jatgáhnye̱hah!”\\
oh oh.my she.said outside you.two.go.and.play\\
\glt ‘“Oh my,” she said, “go play outside!”’ (\cite{henry_de_2005})
\z


\subsection*{\trs{Tsǫ:} \trs{just}, \trs{only}} \label{p:[tsǫ:]}
\CayugaRemark{Particle functioning as an “adverb of degree” (\sectref{ch:’Adverbs’ of degree}); clause-initial.}

\ea
\label{ex:tpart117}
\gll Hwíhs agę́ˀ [\textbf{tsǫ́:} na̱ˀgahwi̱hsdáˀe:k] to nyó: haˀgáęˀ.\\
five it.is.said [only a.certain.number.of.hours] that amount.of.time they.went.there\\
\glt ‘They said it took only five hours to get there.’ (\cite[508]{mithun_watewayestanih_1984}, Aǫhdęgyǫ́heˀ dialogue)
\z

\ea
\label{ex:tpart118}
\gll \textbf{Tsǫ:} niwahonáˀde:s.\\
just a.certain.length.of.potato\\
\glt ‘They are just short spuds.’ (speaking of young children)
\z

\begin{CayugaRelated}
\item \textit{Hne:ˀ, Ne:ˀ … (tsǫ: shęh)} \trs{but}, p. \pageref{ch:[hne:ˀ], [ne:ˀ … (tsǫ: shęh)] ‘but’}\\
\item \textit{Aweˀ hę:gyeh tsǫ:, Awęˀ hę:gyeh tsǫ:} \trs{let it go}, p. \pageref{p:[aweˀ hę:gyeh tsǫ:]}\\
\item \textit{Gaę gwaˀ tsǫ:} \trs{wherever}, \trs{somewhere}, p. \pageref{p:[gaę gwaˀ tsǫ:]}\\
\item \textit{Giˀ tsǫ:} \trs{just}, \trs{really}, p. \pageref{p:[giˀ tsǫ:]}\\
\item \textit{Gwahs waˀ-heh tsǫ:} \trs{just now}, \trs{just a few seconds ago}, p. \pageref{p:[gwahs waˀ-heh tsǫ:]}\\
\item \textit{Hejo:yaˀ tsǫ:} \trs{another thing again}, p. \pageref{p:[hejo:yaˀ tsǫ:] ‘another thing again’}\\
\item \textit{Hejo:yaˀ tsǫ:} \trs{elsewhere}, p. \pageref{p:[hejo:yaˀ tsǫ:] ‘elsewhere’}\\
\item \textit{Hę:-gyęh tsǫ:} \trs{it doesn’t matter}, \trs{never mind}, p. \pageref{p:[hę:-gyęh tsǫ:]}\\
\item \textit{Ne:ˀ giˀ tsǫ: gwahs} \trs{that’s really all}, p. \pageref{p:[ne:ˀ giˀ tsǫ: gwahs]}\\
\item \textit{Ne:ˀ tsǫ:} \trs{that’s only}, \trs{that’s all}, p. \pageref{p:[ne:ˀ tsǫ:] ‘that’s only’}\\
\item \textit{Ne:ˀ tsǫ:} \trs{because}, \trs{it’s just}, p. \pageref{p:[ne:ˀ tsǫ:]}\\
\item \textit{Ne:ˀ tsǫ: shęh} \trs{but}, \trs{it’s just that}, p. \pageref{p:[ne:ˀ tsǫ: shęh]}\\
\item \textit{Neˀ tsǫ: gwaˀ toh} \trs{also}, p. \pageref{p:[neˀ tsǫ: gwaˀ toh]}\\
\item \textit{Nę: tsǫ: gwaˀ toh ni-} \trs{just a little bit}, \trs{very little}, p. \pageref{p:[nę: tsǫ: gwaˀ toh ni-]}\\
\item \textit{Nę: tsǫ: ni-} \trs{just}, p. \pageref{p:[nę: tsǫ: ni-]}\\
\item \textit{To tsǫ:} \trs{that’s all}, p. \pageref{p:[to tsǫ:] ‘that’s all’}\\
\item \textit{To tsǫ:} \trs{that’s enough}, p. \pageref{p:[to tsǫ:] ‘that’s enough!’}\\
\item \textit{Trǫhgeh tsǫ:} \trs{barely}, p. \pageref{p:[trǫhgeh tsǫ:]}
\end{CayugaRelated}


\section{W particles}

\subsection*{\stem{waˀ-} (referring to the current moment)} \label{p:[waˀ-]}
\CayugaRemark{Element occurring at the beginning of the following particle combinations and having to do with time (\sectref{ch:’Adverbs’ of time}); proclitic.}

\begin{CayugaRelated}
\item \textit{Gwahs waˀ-heh tsǫ:} \trs{just now}, \trs{just a few seconds ago}, p. \pageref{p:[gwahs waˀ-heh tsǫ:]}\\
\item \textit{Nę: hwaˀ waˀ-ne:ˀ} \trs{nowadays}, p. \pageref{p:[nę: hwaˀ waˀ-ne:ˀ]}\\
\item \textit{Waˀ-gyęh} \trs{presently}, \trs{so now}, \trs{then}, p. \pageref{p:[waˀ-gyęh] ‘presently’}\\
\item \textit{Waˀ-gyęh} \trs{listen}, \trs{excuse me}, \trs{would you}, p. \pageref{p:[waˀ-gyęh] ‘Listen!’}\\
\item \textit{Waˀ-heh} \trs{just now}, \trs{finally}, p. \pageref{p:[waˀ-heh]}\\
\item \textit{Waˀ-heh-geha:ˀ} \trs{finally}, \trs{at long last}, p. \pageref{p:[waˀ-heh-geha:ˀ]}\\
\item \textit{Waˀ-jih} \trs{after a while}, \trs{eventually}, \trs{just a while ago}, p. \pageref{p:[waˀ-jih] ‘after a while’}\\
\item \textit{Waˀ-jih} \trs{wait!}, p. \pageref{p:[waˀ-jih] ‘wait!’}\\
\item \textit{Waˀ-jih gyę:ˀ} \trs{almost}, \trs{just about}, p. \pageref{p:[waˀ-jih gyę:ˀ]}\\
\item \textit{Waˀ-jih hya:ˀ} \trs{wait a minute}, \trs{wait a while}, \trs{wait!}, p. \pageref{p:[waˀ-jih hya:ˀ]}\\
\item \textit{Waˀ-ji-:hah} \trs{presently}, \trs{a little later}, \trs{after a bit}, \trs{after a while}, \trs{afterwards}, \trs{later}, p. \pageref{p:[waˀ-ji-:hah]}\\
\item \textit{Waˀ-ne:ˀ} \trs{today}, \trs{now}, p. \pageref{p:[waˀ-ne:ˀ]}
\end{CayugaRelated}


\subsection*{\textbf{Waˀ-gyęh} \trs{presently}, \trs{so now}, \trs{then}} \label{p:[waˀ-gyęh] ‘presently’}
\CayugaRemark{Particle group functioning as an “adverb of time” (\sectref{ch:’Adverbs’ of time}); clause-initial.}

\ea
\label{ex:wpart1}
Lila: \\
\gll \textbf{Waˀgyę́h} diˀ seteˀtraníhahsah, nawęˀdáˀ hniˀ ętsę́ni̱haˀ.\\
so.now so you.will.go.and.borrow.flour sugar and you.will.borrow\\
\glt ‘So why don’t you go and borrow some flour from her, and borrow some sugar, too.’

Sam:\\
\gll Haoˀ dę̱ˀ nyóh. \textbf{waˀgyę́h} diˀ é:ˀ ne̱ˀ-tóh hesge:.\\
Ok what acknowledge so.then so again the-there I’ll.go.back.again\\
\glt ‘O.K. then I’ll go back there again.’ (\cite[441]{mithun_watewayestanih_1984}, Dwęnǫhsanékahǫˀ dialogue)
\z


\begin{CayugaRelated}
\item \stem{gyęh} \trs{this}, \trs{that}, p. \pageref{p:[-gyęh]}\\
\item \stem{waˀ-} (referring to the current moment), p. \pageref{p:[waˀ-]}\\
\item \textit{Waˀ-gyęh} \trs{listen}, \trs{excuse me}, \trs{would you}, p. \pageref{p:[waˀ-gyęh] ‘Listen!’}
\end{CayugaRelated}


\subsection*{\textbf{Waˀ-gyęh} \trs{listen!}, \trs{excuse me}, \trs{would you}} \label{p:[waˀ-gyęh] ‘Listen!’}
\CayugaRemark{Particle group functioning as a “command” (\sectref{Particles and particle groups used with commands}).}

\ea
\label{ex:wpart2}
\gll \textbf{Waˀgyęh}!\\
so.now\\
\glt ‘Listen!’, ‘Excuse me!’
\z


\ea
\label{ex:wpart3}
\gll \textbf{Waˀgyęh} desaˀdráihęh!\\
so.now hurry.up\\
\glt ‘Would you hurry up!’
\z


\begin{CayugaRelated}
\item \stem{gyęh} \trs{this}, \trs{that}, p. \pageref{p:[-gyęh]}\\
\item \stem{waˀ-} (referring to the current moment), p. \pageref{p:[waˀ-]}\\
\item \textit{Waˀ-gyęh} \trs{presently}, \trs{so now}, \trs{then}, p. \pageref{p:[waˀ-gyęh] ‘presently’}
\end{CayugaRelated}


\subsection*{\textbf{Waˀ-heh} \trs{just now}, \trs{finally}} \label{p:[waˀ-heh]}
\CayugaRemark{Particle group functioning as an “adverb of time” (\sectref{ch:’Adverbs’ of time}); clause-initial.}

\ea
\label{ex:wpart4}
\gll \textbf{Waˀhéh} giˀ gatahí:neˀ.\\
just.now just I’m.walking\\
\glt ‘I am just now getting on my way.’
\z


\ea
\label{ex:wpart5}
\gll To niyo:wéˀ hęˀséˀ \textbf{waˀhéh} hęhsyǫˀ.\\
that a.certain.distance you.will.go.there just.now you.will.arrive\\
\glt ‘You’ll get that far before you arrive.’
\z


\ea
\label{ex:wpart6}
\gll \textbf{Waˀhéh} ahsyǫˀ.\\
finally you.arrived\\
\glt ‘You finally arrived.’ (just now)
\z


\ea
\label{ex:wpart7}
\gll \textbf{Waˀhéh} \textbf{tsǫ:} ahá:tgęh.\\
just.now only he.got.up\\
\glt ‘He just now got up.’
\z

\largerpage
\begin{CayugaRelated}
\item \textit{Gwahs waˀ-heh tsǫ:} \trs{just now}, \trs{just a few seconds ago}, p. \pageref{p:[gwahs waˀ-heh tsǫ:]}\\
\item \stem{-heh} element (related to time), p. \pageref{p:[-heh]}\\
\item \stem{waˀ-} (referring to the current moment), p. \pageref{p:[waˀ-]}\\
\item \textit{Waˀ-heh-geha:ˀ} \trs{finally}, \trs{at long last}, p. \pageref{p:[waˀ-heh-geha:ˀ]}
\end{CayugaRelated}



\subsection*{\textbf{Waˀ-heh-geha:ˀ} \trs{finally}, \trs{at long last}} \label{p:[waˀ-heh-geha:ˀ]}
\CayugaRemark{Particle group functioning as an “adverb of time” (\sectref{ch:’Adverbs’ of time}); clause-initial.}

\ea
\label{ex:wpart8}
\gll \textbf{waˀ-heh-geha:ˀ}\\
current.moment-\textbf{element-former}\\
\glt ‘finally’, ‘at long last’
\z


\ea
\label{ex:wpart9}
\gll \textbf{Waˀhehgehá:ˀ} aha:kǫ́:niˀ.\\
at.long.last he.cooked\\
\glt ‘At long last he (decided to) cook.’
\z


\begin{CayugaRelated}
\item \textit{Gwahs waˀ-heh tsǫ:} \trs{only just now}, p. \pageref{p:[gwahs waˀ-heh tsǫ:]}\\
\item \stem{-heh} element (related to time), p. \pageref{p:[-heh]}\\
\item \stem{waˀ-} (referring to the current moment), p. \pageref{p:[waˀ-]}\\
\item \textit{Waˀ-heh} \trs{just now}, \trs{finally}, p. \pageref{p:[waˀ-heh]}
\end{CayugaRelated}




\subsection*{\textbf{Waˀ-jih} \trs{after a while}, \trs{eventually}, \trs{just a while ago}} \label{p:[waˀ-jih] ‘after a while’}
\CayugaRemark{Particle group functioning as an “adverb of time” (\sectref{ch:’Adverbs’ of time}); includes the \stem{-jih} {\intensifier} suffix, (\sectref{-jih}); free-standing.}

\ea
\label{ex:wpart10}
\gll “I:sóˀ toh né:ˀ tga:gǫ́:t ęga:yę:dáˀk,” aˀa:gę́ˀ, “\textbf{waˀjíh} ęhswa:tó:wa:t.”\\
a.lot that.one it.is it.is.necessary it.will.be.done she.said after.a.while it.will.get.cold\\
\glt ‘There is a lot a lot that needs to be done,” she said, “today it will get cold again.”’ (\cite{henry_de_2005})
\z


\ea
\label{ex:wpart11}
\gll Ó: i:soˀ gyó:doˀk waˀne:ˀ, shę́h nohgeh neˀ ojǫ́ˀdaˀ, gá:dǫh gyę:ˀ \textbf{waˀjih}.\\
oh it.is.much it.is.lessened today that even the fish I.say emphasis a.while.ago\\
\glt ‘Oh, there is a lot less of everything nowadays even fish, as I was saying just a while ago.’ (\cite{mithun_how_1980})
\z


\ea
\label{ex:wpart12}
\gll O:nę́h aˀa:gę́ˀ, “Ja̱ˀsnęht to:gyę́h \textbf{waˀjíh} neˀ sǫ:gwaˀ nóht ęyagonǫ́hnyaˀk.”\\
now she.said get.down there eventually the someone person she.will.get.hurt\\
\glt ‘And then she said, “Both of you get down before somebody gets hurt.”’ (\cite{henry_de_2005})
\z


\ea
\label{ex:wpart13}
\gll Honęnǫ̱hdǫ́ˀ shęh \textbf{waˀjih} \textbf{tsǫ:} o:nęh sǫ:gwaˀ noht ęyagonyóˀ neˀ nę:gyę́h Hnyagwaiˀdatgiˀgó:wah.\\
they.know that after.a.while just now someone person she.will.die the this.one Ugly.Bear\\
\glt ‘They knew Ugly Bear would soon become even more bold. (\cite{carrier_legends_2013})
\z


\ea
\label{ex:wpart14}
\gll O:nę́h giˀ aˀa:gę́ˀ, “\textbf{Waˀjíh} \textbf{tsǫ:},” aˀa:gę́ˀ, “ęgaˀdaihaˀt haˀdé:yǫ:,” aˀa:gę́ˀ.\\
now just she.said after.a.while just she.said I.will.heat.up various.things she.said\\
\glt ‘Now then she said, “In just a little while I’ll heat up various things.”’ (\cite{henry_de_2005})
\z


\ea
\label{ex:wpart15}
\gll \textbf{Waˀjih} \textbf{tsǫ́:} o:nę́h ęgyahdę́:diˀ.\\
after.a.while just now we.will.leave\\
\glt ‘You and I will leave pretty soon.’ (\cite[142]{mithun_watewayestanih_1984}, Satgęh dialogue)
\z


\begin{CayugaRelated}
\item \stem{waˀ-} (referring to the current moment), p. \pageref{p:[waˀ-]}\\
\item \textit{Waˀ-jih} \trs{wait!}, p. \pageref{p:[waˀ-jih] ‘wait!’}\\
\item \textit{Waˀ-jih gyę:ˀ} \trs{almost}, \trs{just about}, p. \pageref{p:[waˀ-jih gyę:ˀ]}\\
\item \textit{Waˀ-jih hya:ˀ} \trs{wait a minute}, \trs{wait a while}, \trs{wait!}, p. \pageref{p:[waˀ-jih hya:ˀ]}\\
\item \textit{Waˀ-ji-:hah} \trs{presently}, \trs{a little later}, \trs{after a bit}, \trs{after a while}, \trs{afterwards}, \trs{later}, p. \pageref{p:[waˀ-ji-:hah]}
\end{CayugaRelated}


\subsection*{\textbf{Waˀ-jih} \trs{wait!}, \trs{hold up!}} \label{p:[waˀ-jih] ‘wait!’}
\CayugaRemark{Particle group \textit{waˀ-jih} \trs{wait} is clause-initial in commands (\sectref{Particles and particle groups used with commands}).}

\ea
\label{ex:wpart16}
\gll \textbf{Waˀjih}, ętsyáˀdęˀ.\\
wait you.will.fall\\
\glt ‘Wait, you might fall.’
\z


\begin{CayugaRelated}
\item \stem{waˀ-} (referring to the current moment), p. \pageref{p:[waˀ-]}\\
\item \textit{Waˀ-jih} \trs{after a while}, \trs{eventually}, \trs{just a while ago}, p. \pageref{p:[waˀ-jih] ‘after a while’}
\end{CayugaRelated}


\subsection*{\textbf{Waˀ-jih gyę:ˀ} \trs{almost}, \trs{just about}} \label{p:[waˀ-jih gyę:ˀ]}
\CayugaRemark{Particle group functioning as an “adverb of degree” (\sectref{ch:’Adverbs’ of degree}); emphatic; likely clause-initial.}

\begin{CayugaRelated}
\item \textit{Gyę:ˀ} \trs{just the one}, p. \pageref{p:[gyę:ˀ]}\\
\item \stem{waˀ-} (referring to the current moment), p. \pageref{p:[waˀ-]}\\
\item \textit{Waˀ-jih} \trs{after a while}, \trs{eventually}, \trs{just a while ago}, p. \pageref{p:[waˀ-jih] ‘after a while’}\\
\item \textit{Waˀ-ji-:hah} \trs{presently}, \trs{a little later}, \trs{after a bit}, \trs{after a while}, \trs{afterwards}, \trs{later}, p. \pageref{p:[waˀ-ji-:hah]}\\
\item \textit{Waˀ-jih hya:ˀ} \trs{wait a minute}, \trs{wait a while}, \trs{wait!}, p. \pageref{p:[waˀ-jih hya:ˀ]}
\end{CayugaRelated}




\subsection*{\textbf{Waˀ-ji-:hah} \trs{presently}, \trs{a little later}, \trs{after a bit}, \trs{after a while}, \trs{afterwards}, \trs{later}} \label{p:[waˀ-ji-:hah]}
\CayugaRemark{Particle group functioning as an “adverb of time” (\sectref{ch:’Adverbs’ of time}); clause-initial.}

\ea
\label{ex:wpart17}
\gll \textbf{Waˀji:hah} to hęgyǫˀ.\\
presently there I.will.arrive.there\\
\glt ‘I will arrive there after a while.’
\z


\begin{CayugaRelated}
\item \stem{waˀ-} (referring to the current moment), p. \pageref{p:[waˀ-]}\\
\item \textit{Waˀ-jih} \trs{after a while}, \trs{eventually}, \trs{just a while ago}, p. \pageref{p:[waˀ-jih] ‘after a while’}\\
\item \textit{Waˀ-jih gyę:ˀ} \trs{almost}, \trs{just about}, p. \pageref{p:[waˀ-jih gyę:ˀ]}\\
\item \textit{Waˀ-jih hya:ˀ} \trs{wait a minute}, \trs{wait a while}, \trs{wait!}, p. \pageref{p:[waˀ-jih hya:ˀ]}
\end{CayugaRelated}



\subsection*{\textbf{Waˀ-jih hya:ˀ} \trs{wait a minute}, \trs{wait a while}, \trs{wait!}} \label{p:[waˀ-jih hya:ˀ]}
\CayugaRemark{Particle group, likely clause-initial before commands (\sectref{Particles and particle groups used with commands}).}

\ea
\label{ex:wpart18}
\gll \textbf{Waˀ-jih} \textbf{hya:ˀ}.\\
in.a.while before.anything.else\\
\glt ‘Wait a minute.’, ‘Wait a while.’, ‘Wait!’
\z


\begin{CayugaRelated}
\item \textit{Hya:ˀ} \trs{first}, \trs{before anything else}, p. \pageref{p:[hya:ˀ]}\\
\item \textit{Waˀ-jih} \trs{after a while}, \trs{eventually}, \trs{just a while ago}, p. \pageref{p:[waˀ-jih] ‘after a while’}\\
\item \textit{Waˀ-jih} \trs{wait!}, p. \pageref{p:[waˀ-jih] ‘wait!’}
\end{CayugaRelated}


\subsection*{\textbf{Waˀ-ne:ˀ} \trs{today}, \trs{now}} \label{p:[waˀ-ne:ˀ]}
\CayugaRemark{Particle group functioning as an “adverb of time” (\sectref{ch:’Adverbs’ of time}); free-standing.}

\ea
\label{ex:wpart19}
\gll Ędwahyagwáhaˀ, ędwahnyo̱ˀgwano̱hgwa:sáˀ hniˀ \textbf{waˀne:ˀ}!\\
we.will.gather.berries, we.will.gather.nuts and today\\
\glt ‘We are gathering nuts and berries today!’ (\cite{carrier_legends_2013})
\z


\begin{CayugaRelated}
\item \textit{Ne:ˀ} \trs{it is}, p. \pageref{p:[ne:ˀ] `it is’}\\
\item \textit{Nę: hwaˀ waˀ-ne:ˀ} \trs{nowadays}, p. \pageref{p:[nę: hwaˀ waˀ-ne:ˀ]}\\
\item \stem{waˀ-} (referring to the current moment), p. \pageref{p:[waˀ-]}
\end{CayugaRelated}



\section{Y particles}

\subsection*{\textbf{Yę:gwaˀ} \trs{if}} \label{p:[yę:gwaˀ]}
\CayugaRemark{Alternative way to pronounce \textit{gyę:gwaˀ} \trs{if}.}

\begin{CayugaRelated}
\item \textit{Gyę:gwaˀ} \trs{if}, p. \pageref{p:[gyę:gwaˀ]}
\end{CayugaRelated}

