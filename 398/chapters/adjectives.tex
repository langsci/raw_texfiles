\chapter{“Adjectives” (words functioning as adjectives)} \label{ch:’Adjectives’ (words that function as adjectives)}
Words functioning as “adjectives” describe qualities, characteristics, or states. \textsc{Stative-only} (or “adjectival”) verbs perform this function (\ref{ex:adjex3}, see \sectref{Verbs occurring only in one aspect (stative or habitual)}, \sectref{single-aspect verbs}). Since stative-only verbs require a pronominal prefix, they also describe who or what has the characteristic in question. 

\ea\label{ex:adjex3}
\ea ohsnó:weˀ ‘it is fast, quick’
\ex go̱haˀdí:yo: ‘she is a good singer, she has a good voice’
\ex ogá:yǫh ‘it is old’
\ex go̱hsdę́:ˀęh ‘she is old’
\ex onówęht ‘it is absurd, unbelievable’
\ex agadatsę́hse: ‘I am tired’
\ex hohsgénhaˀseh ‘he is greedy’
\ex dewadagwę́hdę: ‘it is flat’
\ex ohdaht ‘it is filling’
\ex oyá:nreˀ ‘it is nice, good, beautiful’
\z
\z


“Adjectives” often modify nouns \xref{ex:adjex2}, or noun phrases such as \textit{neˀ Mary} \xref{ex:adjex}, to further specify ‘who’ or ‘what’ has the quality or characteristic in question.

\ea\label{ex:adjex2}
 \gll \exemph{onahsę́:} daksháheˀdohs\\
they.are.fat chickens\\
\glt ‘fat chickens’, ‘The chickens are fat.’
\z

\ea\label{ex:adjex}
 \gll \exemph{Godiˀgrǫ́ˀ} neˀ Mary.\\
she.is.shy the Mary\\
\glt ‘Mary is shy.’
\z

Many adjectival verbs \textsc{incorporate} (include) the noun that they modify (\ref{ex:adjex4}, see \sectref{ch:Incorporated noun stems}, \sectref{ch:Noun Incorporation}.)

\ea\label{ex:adjex4} adjectival verb with incorporated noun \stem{di-tra-} ‘tea’\\
o\exemph{ditra}gáˀǫh ‘good tea’
\cfex{ogáˀǫh ‘it is good-tasting, it tastes good’ (adjectival verb without incorporated noun)}
\z


Finally, adjectival verbs by nature invite comparisons (\ref{ex:adjex5}, see \sectref{ch:Comparisons (more, the same, or less)}).

\ea\label{ex:adjex5}
 \gll \exemph{ne:ˀ} \exemph{gwahs} \exemph{d}wakyęda̱hgówanęh \\
it.is emphasis biggest.chair\\
\glt ‘the biggest chair’
\z

