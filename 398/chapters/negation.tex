\chapter{Negation} \label{ch:Negation}
Negation turns affirmative statements (or verbs) into negative ones. Affirmative statements are “true” in some sense, while negative statements (also verbs) express the opposite of affirmative ones. Negation is summarized in \tabref{figtab:1:negation}. (The verb categories in \tabref{figtab:1:negation} are described in \sectref{Verbs and aspect}.)

\begin{table}
\caption{Negation}
\label{figtab:1:negation}
\fittable{
\begin{tabular}{lll}
\lsptoprule
& positive & negative\\
\midrule
\textsc{habitual} & \stem{\textsc{verb}-{\habitual}} & \stem{{\negative}-\textsc{verb-\habitual}}\\
\midrule
\textsc{stative} and & \stem{\textsc{verb}-{\stative}} & \stem{{\negative}-\textsc{verb-\stative}}\\
\textsc{factual punctual} & \stem{\textsc{\factual}{}-\textsc{verb-\punctual}} & \\
\midrule
\textsc{future punctual} and & \stem{\textsc{future}{}-\textsc{verb-\punctual}} & \stem{{\contrastive-\indefinite}-\textsc{verb-\noaspect}}\\
\textsc{indefinite punctual} & \stem{{\indefinite}-\textsc{verb-\punctual}} & \\
\lspbottomrule
\end{tabular}
}
\end{table}

As shown in \tabref{figtab:1:negation}, there are two distinct ways to negate verbs: \textsc{habitual}- and \textsc{stative}-aspect verbs  just take the \stem{deˀ-} \textsc{\negative} prefix, without any further modification \xxref{ex:negationex3}{ex:negationex5}. (The particle \cayuga{tęˀ} ‘not’ is always optional.)

\ea\label{ex:negationex3} \stem{(tęˀ) {\negative}-\textsc{verb}-{\habitual}} (\textsc{three-aspect} habitual)
\ea tęˀ de̱ˀagyade:kǫ́:nih\\
\gll tęˀ de̱ˀ-agy-ade:-kǫ́:ni-h\\
not  {\negative}-\exsc{1d.ex.a}-{\semireflexive}-eat-{\habitual}\\
\glt `we two do not eat'
\cfex{gaǫdekǫ́:nih  \\
\gll gaǫ-de-kǫ́:ni-h \\
\exsc{3ns.fi.a}-{\semireflexive}-eat-{\habitual}\\
\glt ‘they (f/m) eat’ (regularly)}
\ex tęˀ de̱ˀaknigǫháędaˀs  \\
\gll tęˀ de̱ˀ-ak-nigǫháęd-a-ˀs\\
not  {\negative}-\textsc{1s.p}-understand-{\joinerA}-{\habitual}\\
\glt ‘I do not understand’
\cfex{aknigǫ̱háędaˀs\\
	\gll ak-nigǫ̱háęd-a-ˀs\\
	\textsc{1s.p}-understand-{\joinerA}-{\habitual}\\
	\glt `I understand'}
\ex tęˀ degé:gęhs  \\
\gll tęˀ de-gé:-gę-hs\\
not {\negative}-\textsc{1s.a}-see-{\habitual}\\
\glt ‘I do not see things’
\cfex{gé:gęhs \\
\gll gé:-gę-hs\\
\textsc{1s.a}-see-{\habitual}\\
\glt ‘I see things’}
\z
\z
 
\ea\label{ex:negationex4} \stem{(tęˀ) {\negative}-\textsc{verb}-{\stative}} (\textsc{three-aspect} stative)\\
tęˀ dewa:kní:nǫh, tęˀ de̱ˀa:kní:nǫh  \\
\gll tęˀ de-wa:k/ˀa:k-hní:nǫ-h\\
not {\negative}-\textsc{1s.p}-buy-{\stative}\\
\glt ‘I did not buy it’
\cfex{akní:nǫh\\
	\gll ak-hní:nǫ-h\\
	\textsc{1s.p}-buy-{\stative}\\
	\glt `I did buy it'}
\z


\ea\label{ex:negationex5} \stem{(tęˀ) {\negative}-\textsc{verb}.{\stative}} (\textsc{stative-only} verb)\\
tęˀ degá:nǫ:ˀ\\
\gll tęˀ de-gá:-nǫ:ˀ\\ 
not {\negative}-{3\sga}-costly.{\stative}\\
\glt `it is not costly'
\cfex{ganǫ:ˀ\\
	\gll ga-nǫ:ˀ\\
	\textsc{3s.a}-costly.stative\\
	\glt `it is expensive, dear, precious'}
\z


Punctual verbs (\sectref{Verbs and aspect}) are negated differently than habitual or stative ones: for factual-punctual verbs (\sectref{Aspect suffixes}), the negative counterpart is the negative \textsc{three-aspect stative} form of the same verb \xref{ex:negationex6}. In other words, affirmative factual-punctual forms \xref{ex:negationex6} and affirmative stative-aspect forms \xref{ex:negationex600} share the \emph{same} negative form.

\newpage
\ea\label{ex:negationex6} \stem{(tęˀ) {\negative}-\textsc{verb}-{\stative}} (negative of the factual-punctual)\\
\glll tęˀ de̱hóyętwęh  \\
tęˀ de̱-hó-yętw-ęh\\
not {\negative}-\textsc{3s.m.p}-plant-{\stative}\\
\glt ‘he did not plant’
\cfex{aháyętoˀ\\
	\gll a-há-yęto-ˀ\\
	{\factual}-\textsc{3s.m.a}-plant-{\punctual}\\
	\glt `he did plant'}
\z


\ea\label{ex:negationex600} \stem{(tęˀ) {\negative}-\textsc{verb}-{\stative}} (negative of the affirmative stative)\\
\glll tęˀ de̱hóyętwęh  \\
tęˀ de̱-hó-yętw-ęh\\
not {\negative}-\textsc{3s.m.p}-plant-{\stative}\\
\glt ‘he did not plant’
\cfex{hoyę́:twęh\\
	\gll ho-yę́:tw-ęh\\
	\textsc{3s.m.p}-plant-{\stative}\\
	\glt `he is planting'}
\z



For future-punctuals and indefinite-punctuals (\sectref{Aspect suffixes}), the negative counterpart begins with the \stem{ta:-} \textsc{\contrastive-\indefinite} combination and lacks an aspect suffix (\ref{ex:negationex7}-\ref{ex:negationex8}. These are \textsc{no-aspect} forms (\sectref{No-aspect verbs}).

\ea\label{ex:negationex7} \stem{(tęˀ) {\contrastive}-\exsc{indefinite}-\textsc{verb}-{\noaspect}} (negative of the future-punctual)
\ea tęˀ ta:gaesayę́:de:  \\
\gll tęˀ t-a:-gaesa-yę́:de:-ø\\
not {\contrastive}-{\indefinite}-\exsc{3ns>2s}-recognize-{\noaspect}\\
\glt ‘they will not recognize you’
\cfex{ęgyę́:de:ˀ\\
	\gll ę-g-yę́:de:-ˀ\\
	\fut-\textsc{1s.a}-recognize-{\punctual}\\
	\glt `I will recognize it'}
\ex tęˀ ta:gé:gęh\\
\gll tęˀ t-a:-gé:-gę-h\\
 not {\contrastive}-{\indefinite}-\textsc{1s.a}-see-\exsc{euph.h}/{\noaspect}\\
\glt `I will not or should not see it'
\cfex{ęgé:gęˀ\\
	\gll ę-gé:-gę-ˀ\\
	\fut-\textsc{1s.a}-see-{\punctual}\\
	\glt `I will see it'}
\z
\z

\ea\label{ex:negationex8} \stem{(tęˀ) {\contrastive}-\exsc{indefinite}-\textsc{verb}-{\noaspect}} (negative of the indefinite-punctual)
\ea tęˀ ta:gatgáhtoh\\
\gll tęˀ t-a:-g-at-gáhto-h\\
 not  {\contrastive}-{\indefinite}-\textsc{1s.a}-{\semireflexive}-see-\textsc{euph.h}/{\noaspect}\\
\glt `I should not (or will not) look at it'
\ex tęˀ ta:kní:nǫh  \\
\gll tęˀ t-a:-k-hní:nǫ-h\\
not {\contrastive}-{\indefinite}-\textsc{1s.a}-buy-\textsc{euph.h}/{\noaspect}\\
\glt ‘I should not or will not buy it’
\cfex{e:hsní:nǫˀ\\
	\gll e:-hs-hní:nǫ-ˀ\\
	{\indefinite}-\textsc{2s.a}-buy-{\punctual}\\
	\glt `you would buy, purchase it'}
\ex tęˀ ta:gé:gęh\\
\gll tęˀ t-a:-gé:-gę-h\\
 not  {\contrastive}-{\indefinite}-\textsc{1s.a}-see-\textsc{euph.h}/{\noaspect}\\
\glt `I will not or should not see it'
\cfex{a:gé:gęˀ\\
	\gll a:-gé:-gę-ˀ\\
	{\indefinite}-\textsc{1s.a}-see-{\punctual}\\
	\glt `I should (etc.) see it'}
\z
\z




\section{Negation of verb forms based on the three major aspects} \label{ch:Negation of verb forms based on the three major aspects}
Verbs variations based on the three major aspects (\sectref{Post-aspect (tense) suffixes}) also have negative versions. In general, the system is the same as for three-aspect verbs. For example, the negative of any  affirmative habitual or affirmative stative variation begins with \stem{deˀ-} \textsc{\negative} \xref{ex:no-mood-negation1}.

\ea\label{ex:no-mood-negation1} negation of variations on habitual and stative forms
\ea tęˀ degatrowíhsgę̱hę:\\
\gll tęˀ de-g-at-hrowí-hs-gę̱hę:ˀ\\
not {\negative}-\textsc{1s.a}-{\semireflexive}-tell-{\habitual}-{\past}\\
\glt  ‘I didn’t used to tell’
\newpage
\cfex{gatrowihsgę́hę:ˀ\\
	\gll g-at-hrowi-hs-gę́hę:ˀ\\
	\textsc{1s.a}-{\semireflexive}-tell-{\habitual}-{\past}\\
	\glt `I used to tell'}
\ex tęˀ dewagatrowi̱há:gyeˀs  \\
\gll tęˀ de-wag-at-hrowi̱-h-á:-gy-e-ˀs\\
not {\negative}-\textsc{1s.p}-{\semireflexive}-tell-\textsc{euph.h}-{\joinerA}-{\progressive}-go-{\habitual}\\
\glt ‘I am not going along telling here and there’
\cfex{agatrowíhagyeˀs\\
	\gll ag-at-hrowí-h-a-gy-e-ˀs\\
	\textsc{1s.p}-{\semireflexive}-tell-\textsc{euph.h}-{\joinerA}-{\progressive}-go-{\habitual}\\
	\glt `I am going along telling here and there'}
\ex tęˀ deˀagatrówi̱hne:ˀ \\
tęˀ  deˀ-ag-at-hrówi̱-hne:ˀ \\
not {\negative}-\textsc{1s.p}-{\semireflexive}-tell.{\stative}-{\past}\\
\glt ‘I haven’t/hadn’t told’ 
\cfex{agatrowíhne:ˀ\\
	\gll ag-at-hrowí-hne:ˀ\\
	\textsc{1s.p}-{\semireflexive}-tell.{\stative}-{\past}\\
	\glt `I have/had told'}
\z
\z


Similarly, any affirmative verb beginning with the \stem{ę-} \textsc{\future} or \stem{a:-} \textsc{\indefinite}, and ending with a \stem{-:k} \textsc{\modalizer} takes the \stem{ta:-} \textsc{\contrastive-\indefinite} combination \xref{ex:no-mood-negation2}.


\ea\label{ex:no-mood-negation2} negation of [ę/a:-…:k] \exsc{\future/\indefinite…\modalizer} verbs
\ea tęˀ ta:gatrówi̱ha:k\\
\gll tęˀ t-a:-g-at-hrówi̱-h-a-:k\\ 
not {\contrastive}-{\indefinite}-\textsc{1s.a}-{\semireflexive}-tell-\textsc{euph.h}-{\joinerA}-{\modalizer}\\
\glt  ‘I won’t/wouldn’t be a teller’, I won’t/wouldn’t tell all the time’
\cfex{ęgatrowíha:k \\
	\gll ę-g-at-hrowí-h-a:-k\\
	\fut-\exsc{1s.a-{\semireflexive}-}tell-\textsc{euph.h}-{\joinerA}-{\modalizer}\\
	\glt ‘I will be a teller’, `I’ll tell all the time'}
\cfex{a:gatrówi̱ha:k\\
	\gll a:-g-at-hrówi̱-h-a-:k\\
	{\indefinite}-\textsc{1s.a}-{\semireflexive}-tell-\textsc{euph.h}-{\joinerA}-{\modalizer}\\
	\glt `I would be a teller’, `I would tell all the time'}
\ex tęˀ ta:wagatrowíha:k  \\
\gll tęˀ t-a:-wag-at-hrowí-h-a-:k\\
not {\negative}-{\indefinite}-\exsc{\semireflexive}-tell-\textsc{euph.h}-{\joinerA}-{\modalizer}\\
\glt ‘I wouldn’t have told’
\cfex{ęwagatrówi̱ha:k\\
	\gll ę-wag-at-hrówi̱-h-a-:k\\
	\fut-\textsc{1s.p}-{\semireflexive}-tell-\textsc{euph.h}-{\joinerA}-{\modalizer}\\
	\glt `I will have told'}
\ex tęˀ ta:waga:tró:wi:k  \\
\gll tęˀ t-a:-wag-a:t-hró:wi-:k\\
 not {\contrastive}-{\indefinite}-\textsc{1s.p}-{\semireflexive}-tell.{\stative}-{\modalizer}\\
\glt ‘I shouldn’t have told’
\cfex{a:waga:tró:wi:k\\
	\gll a:-wag-a:t-hró:wi-:k\\
	{\indefinite}-\textsc{1s.p}-{\semireflexive}-tell.{\stative}-{\modalizer}\\
	\glt `I should have told'}
\ex tęˀ ta:gatrowi̱hágye̱ˀse:k  \\
\gll tęˀ t-a:-g-at-hrowi̱-h-á-gy-e̱-ˀs-e:k\\
not {\negative}-{\indefinite}-\textsc{1s.a}-{\semireflexive}-tell-\textsc{euph.h}-{\joinerA}-{\progressive}-go-{\habitual}-{\modalizer}\\
\glt ‘I still wouldn’t have been going along telling’
\cfex{ęwagatrowi̱hágye̱ˀse:k\\
	\gll ę-wag-at-hrowi̱-h-á-gy-e̱-ˀs-e:k\\
	 \fut-\textsc{1s.p}-{\semireflexive}-tell-\textsc{euph.h}-{\joinerA}-{\progressive}-go-{\habitual}-{\modalizer}\\
	\glt `I will still have been going along telling'}
\cfex{a:wagatrowihagyéˀse:k\\
	\gll a:-wag-at-hrowi-h-a-gy-é-ˀs-e:k\\
	{\indefinite}-\textsc{1s.p}-{\semireflexive}-tell-\textsc{euph.h}-{\progressive}-{\habitual}-{\modalizer}\\
	\glt `I should still have been going along telling'}
\z
\z


Finally, the negative of any affirmative verb beginning with the \stem{aˀ-} \textsc{\factual} and ending with the \stem{-ˀ} \textsc{\punctual}, is a stative form beginning with \stem{deˀ-} \textsc{\negative} \xref{ex:no-mood-negation3}.

\ea\label{ex:no-mood-negation3} negation of \stem{aˀ-…-ˀ} \textsc{\factual-…-\punctual} forms \\
tęˀ dewagatrowi̱há:gyeˀ  \\
\gll tęˀ de-wag-at-hrowi̱-h-á:-gy-e-ˀ\\
 not {\negative}-\textsc{1s.p}-{\semireflexive}-tell.{\stative}-\textsc{euph.h}-{\joinerA}-{\progressive}-go-{\stative}\\
\glt ‘I haven’t gone along telling’
\cfex{ǫgatrowíhagyeˀ\\
	\gll ǫg-at-hrowí-h-a-gy-e-ˀ\\
	\exsc{\factual.1s.p}-{\semireflexive}-tell-\textsc{euph.h}-{\joinerA}-{\progressive}-go-{\punctual}\\
	\glt `I went along telling'}
\z

