\chapter{Word order effects} \label{ch:Word order effects}
Gayogo̱ho:nǫˀnéha:ˀ word order is \emph{flexible}, meaning for example that a noun describing an \textsc{agent} or “doer”, such as \textit{neˀ kso:t}, can appear either before the verb \xref{ex:flexwordex} or after \xref{ex:flexwordex2}. Similarly, a noun undergoing an action, such as \textit{jihsǫ́:dahk}, can occur either before the verb \xref{ex:flexwordex3} or after \xref{ex:flexwordex4}. 

\ea\label{ex:flexwordex}
\gll Neˀ giˀ aˀa:gę́ˀ, [\exemph{neˀ} \exemph{ksó:t} aˀa:gę́ˀ], “Ęhsnigǫ̱há:k gwaˀ to:gyę́h, ahgwí, ęhsahdǫ́: neˀ gayá:ˀ.”\\
The just she.said, [the grandmother, she.said], you.two.be.careful intensifier that.one, don’t you.will.lose the bag\\
\glt \trs{So then [our grandmother said], “You two be careful, don’t lose that bag!”}(\cite{henry_de_2005})
\z

\ea\label{ex:flexwordex2} 
\gll Neˀ giˀ to: neˀ, [onę́h aˀa:gę́ˀ \exemph{neˀ} \exemph{kso:t},] “Jadohswe̱ˀdá:nih gę́h?” \\
The just that the, [now she.said the grandmother], you.two.are.hungry Q \\
\glt \trs{It’s just that then [our grandma said], “Are you both hungry?”} (\cite{henry_de_2005})
\z

\ea\label{ex:flexwordex3}
\gll \exemph{Jihsǫ:dáhk} ga̱hyá:gwahs \\
strawberries I’m.picking.fruit\\
\glt ‘I’m picking strawberries.’
\z

\ea\label{ex:flexwordex4} 
\gll Ga̱hya:gwáhs \exemph{jihsǫ́:dahk} \\
I’m.picking.fruit strawberries\\
\glt ‘I’m picking strawberries.’
\z

Since word order is flexible, speakers can and do use it to manage information flow. Some examples of the kinds of information implied by word order choice are provided next. The following descriptions are tentative, since more research is needed on this topic.

\section{First position} \label{ch:First position}
Words in \textsc{first position} are either at the very beginning of an utterance, or right after a particle group. They tend to convey significant, notable, or topical information. For example, in \xref{ex:flexwordex6b}, the speaker uses \textit{oda:dí:dęht} first in the sentence because it is the main point. (The village described in \ref{ex:flexwordex6a} was undergoing a famine.)

\ea\label{ex:flexwordex6}
\ea\label{ex:flexwordex6a}
\gll  O:nę́h haha:di:yǫ́ˀ shęh ho:wę́h hadinagrehsraę́ˀ neˀ ǫgwehǫ́:weh.\\
now they.arrived that where they.live the real.people\\
\glt ‘Now they arrived where the Real People lived.’

\ex\label{ex:flexwordex6b}
\gll \exemph{Oda:di:dę́h}t dęˀhoˀdę́ˀ ahęnatgáhtoˀ.\\
it.was.pitiful what they.looked.at.it\\
\glt  ‘It was pitiful what they saw.’ (\cite{carrier_legends_2013})
\z
\z

As another example, in answer to the salesman’s question in the dialogue in \xref{ex:flexwordex5}, Sam first mentions the item that he wants, \textit{enesdanyáktaˀ} \xref{ex:flexwordex5b}, because it is the direct, relevant response to the salesman’s question. In contrast, later in the dialogue \xref{ex:flexwordex5c}, the main point, (again expressed first), is about buying (\textit{ękni:nǫ́ˀ}), and mention of the saw (\textit{neˀ gwáhs ga̱henˀatri:yó: enesdanyáktaˀ}) is now at the end of the utterance. 

\ea\label{ex:flexwordex5}
\ea\label{ex:flexwordex5a}
\gll Salesman: Sga̱hoˀdę́:ˀęh gęh o:yáˀ desadǫ̱hwę:jo:nih? \\
{} anything Q other you.want.it\\
\glt ‘Is there something else you want?’

\ex\label{ex:flexwordex5b}
\gll Sam: Ęhę́ˀ. \exemph{Enesdanyaˀktáˀ} giˀ hniˀ dewagadǫhwęjó:nih. \\
{} yes saw just and I.want.it\\
\glt ‘Yes, I also want a saw.’

\ex\label{ex:flexwordex5c}
\gll Ne:ˀ giˀ gyę:ˀǫh ękni:nǫ́ˀ \exemph{neˀ} \exemph{gwáhs} \exemph{ga̱hena̱ˀtri:yó:} \exemph{enesdanyáˀktaˀ}. \\
{} it.is just I.guess I.will.buy.it the best good.knife saw\\
\glt ‘Then I guess I’ll buy the best cutting saw.’ (\cite[159]{mithun_watewayestanih_1984}, Enǫhsǫnyaˀdaˀsǫ́:ˀǫh dialogue)
\z
\z

From previous studies, we know that first position is for \emph{new} information (information that the speaker believes is not known to the listener(s) (\cite[29--30]{mithun_is_1992}, \cite[408]{mithun_morphological_1995}). First position is also for information contrasting with something mentioned earlier (\cite[37]{mithun_is_1992}). Finally, first position is reserved for words expressing a new topic or a shift in topic (\cite[27, 29--32]{mithun_is_1992}, \cite[405]{mithun_morphological_1995}, also see \cite{chafe_discourse_1994}).

Example \xref{ex:flexwordex7} introduces an additional point, that speakers can use several strategies at once to manage information flow. In \xref{ex:flexwordex7}, the speaker introduces a new topic (another way to hunt for rabbits) and so places \cayuga{ó:yaˀ} ‘another (way)’ in first position (the closest possible position is after \textit{tohgeh}). Meanwhile, \textit{tohgeh} also signals a shift in topic. (It functions as a \textsc{discourse marker} or \textsc{topic manager}, described in \sectref{ch:Discourse markers}).

\ea\label{ex:flexwordex7}
\gll Tohgeh o:yaˀ hniˀ gę:s niyagwayé:haˀ.\\
then other too used-to so-we-did-it.\\
\glt ‘There is another way we used to do it.’ (\cite{mithun_how_1980})
\z


\section{Last position} \label{ch:Last position}
\textsc{Last position} refers to the end of an utterance. Words in last position tend to express old information (information that the speaker believes to be already known or readily accessible from the context of the discourse, \cite[405]{mithun_morphological_1995}). For example, \textit{neˀ ksotgę̱hę́:ˀ} appears near the end of \xref{ex:flexwordex8b} because ‘the grandmother’ was previously introduced. In contrast, the most important information in \xref{ex:flexwordex8b} is \textit{neˀ dyotgǫ́:t gę:s}, which appears first in order to emphasize that grandmother \emph{always} had bread. 

\ea\label{ex:flexwordex8}
\ea\label{ex:flexwordex8a} 
\gll Neˀ giˀ agása:ˀs \\
the just I.remember \\
\glt ‘What I remember’\\

\ex \label{ex:flexwordex8b}
\gll neˀ dyotgǫ́:t gę:s gonaˀdaę́ˀ \exemph{neˀ} \exemph{ksotgę̱hę́:ˀ}.\\
the always usually she.had.bread the late.grandmother \\
\glt ‘is that she always had bread, that grandma.’

\ex \label{ex:flexwordex8c}
\gll Neˀ gotnaˀdáǫdaˀk.\\
the she.made.bread\\
\glt ‘She made bread.’ (\cite{henry_de_2005})
\z
\z

Last position also introduces sub-themes, or alternatively, reminds listener(s) about information that continues to be relevant: in \xref{ex:flexwordex18b} \textit{neˀ odiˀtragáˀǫh} introduces the subtopic of “Good Tea” (the name of the dog), who is then described in the following sentence \xref{ex:flexwordex18c}. More examples are provided in \sectref{ch:Non-incorporated nouns in final position}.

\ea\label{ex:flexwordex18}
\ea\label{ex:flexwordex18a} 
\gll Da̱hataę́: shayaˀdá:t gwahs hniˀ nę: hahayé: neˀ hotse:nę́ˀ só:wa:s\\
he.spoke one.man just.then and then he.touched the  his.pet dog\\
\glt \trs{One man spoke up and touched his pet dog.}

\ex \label{ex:flexwordex18b}
\gll Ahę́ˀ, “Ne:ˀ seh-gyę:ˀ sęh tęˀ ní:s desana̱hsgwaę́: toh nagana̱hsgwiyohá:k sęh niyóht \exemph{neˀ} \exemph{odiˀtragáˀǫh}.”\\
he.said it’s you.know-this.one that not you you.don’t.have.a.pet that.one how.good.the.pet.is that it’s.like the good.tea\\
\glt \trs{He said, “Because you don’t have a pet as good as Good Tea.”}

\ex \label{ex:flexwordex18c}
\gll “Tęˀ-hné: hwę́:dǫˀ da̱hodahdǫ́:dęˀ ó:nęh ęhayanę́hę:ˀ ga:nyo:ˀ.”\\
not-in.fact ever it.disappears.on.him now he.will.track.it wild.animal\\
\glt  \trs{“It never disappears on him when he tracks an animal.”} (He never loses the track.) (\cite{keye_hnyagwaidatgigowah_2012})
\z
\z 


\chapter{Noun incorporation in discourse} \label{ch:Noun incorporation in discourse}
Recall that \textsc{noun incorporation} is optional for some verbs. When noun incorporation is optional, speakers can make deliberate choices about whether or not to incorporate, for stylistic effect. In fact, pairs of non-incorporated and incorporated nouns often appear together in connected speech: for example, the non-incorporated noun \textit{oˀę:náˀ} appears in the sentence in \xref{ex:nounincdiscexa}, while its incorporated counterpart \stem{ˀęn} appears in \xref{ex:nounincdiscexb}. The discourse-relevant reasons for such pairings are described in the following sections. (For related information, see \sectref{Verbs and noun incorporation} and \ref{ch:Noun Incorporation}.)

\ea\label{ex:nounincdiscex} 
\ea \label{ex:nounincdiscexa} 
 \gll O:nę́h hahonadíˀ ohyoˀti:yéht \exemph{oˀę:náˀ} shęh í:ga:t.\\
now they.threw.it.there it.is.sharp spear that it.stands\\
\glt ‘Now they threw a sharp spear where he (the Bear) stood.’

\cfex{oę:náˀ\\
\gll o-ę:n-áˀ\\
        \textsc{3s.p}-spear-{\nounstemformer}\\}
\ex \label{ex:nounincdiscexb} 
 \gll Ne:ˀ neˀ hohsę́: haho\exemph{ˀęn}ǫ:díˀ gwahs tsaˀdeyonǫˀahę́ˀ haˀgahnyó:daˀ.\\
it.is the he.is.fat he.spear-threw right middle.of.its.head it.stuck.there\\
\glt ‘The fat one threw the spear right in the middle of its head, (where) it stuck.’ (\cite{keye_hnyagwaidatgigowah_2012})

\cfex{hahoˀęnǫ:díˀ\\
\gll ha-ho-ˀęn-ǫ:dí-ˀ\\
    {\translocative}-\textsc{3s.m.p}-spear-throw-{\punctual}\\
    }
\z
\z 


\section{Non-incorporated nouns in first position} \label{ch:Non-incorporated nouns in first position}
Non-incorporated nouns in first position tend to introduce prominent or new information (that is, \textsc{foregrounded} material). In contrast, incorporated nouns by definition do not occur in first position. They often refer to previously-established, non-topical, or background information (that is, \textsc{given} information), (\cite[406]{mithun_morphological_1995}). For example, the use of non-incorporated \textit{oˀnhǫhsaˀ} in first position in \xref{ex:nounincdiscex30a} signals that ‘eggs’ have become important or relevant to the point at hand. In contrast, the switch to incorporated \stem{ˀnhǫhs} ‘eggs’ in \xref{ex:nounincdiscex30b} conveys that \emph{eggs} are now old news, with the question about \emph{borrowing} becoming more relevant or topical in \xref{ex:nounincdiscex30b}, (\cite[429]{mithun_languages_1999}).

\ea\label{ex:nounincdiscex30}
\ea\label{ex:nounincdiscex30a} 
\gll \exemph{Oˀnhǫhsaˀ} hniˀ to í:wa:t.\\
eggs too there it.is.in.there\\
\glt \trs{There are some eggs in there too.}
\ex \label{ex:nounincdiscex30b}
\gll Tęˀ de̱ˀgę́: hne:ˀ a:se̱\exemph{ˀnhǫhsa}níhaˀ hniˀ?\\
not it.isn’t in.fact you.should.borrow.eggs too \\
\glt ‘Don't you think you should egg-borrow too?’ (\cite[429]{mithun_languages_1999}, modified orthography and glosses)
\z
\z 

 

For Kanien'kéha, \citet[2,8]{decaire_2017} convincingly argue that noun incorporation is obligatory when the verb allows it and that \textsc{excorporation} is a device for placing \textsc{focus} on the noun in question. Excorporated nouns expressing focus appear to the left of the verb, in first position.

\section{Excorporated nouns in final position} \label{ch:Non-incorporated nouns in final position}
Excorporated nouns can also appear in \emph{final} position, where they can express points of clarification, or can also remind the listener about concepts that continue to be relevant to the conversation (\cite{li_givenness_1976}, \cite[199--200]{mithun_languages_1999}. Also see \sectref{ch:Last position}). For example, \textit{otgwę́hsaˀ} appears in final position in \xref{ex:nounincdiscex4b}, where it clarifies ‘what is dripping’. (Also, illustrating that excorporation was a deliberate choice, \textit{otgwę́hsaˀ} could have been incorporated into \textit{ęyó:kaˀ} ‘it will drip’, as shown by words like \textit{otgwęhsokahs} ‘dripping blood’.)

\ea\label{ex:nounincdiscex4}
\ea\label{ex:nounincdiscex4a} 
\gll O:nę́h gę:s ahadinyóˀ ętgawe̱hsahíˀ neˀ hnyagwaidatgiˀgó:wah.\\
now usually they.kill.it it.will.bleed the Ugly.Bear\\
\glt ‘When they kill it, Ugly Bear bleeds.’
\ex \label{ex:nounincdiscex4b}
\gll Ne:ˀ hne:ˀ ęyo:káˀ gaǫhyada:gyéˀ \exemph{otgwę́hsa}ˀ. \\
it.is in.fact it.will.drip along.the.sky blood \\
\glt ‘That is what drips in the sky, blood.’
\ex 
\gll Ne:ˀ dęhsgade:níˀ shęh niyohso̱hgoˀdę́hs onrahdǫ:dǫ́ˀ neˀ gra̱héˀdǫˀ.\\
it.is it.will.change that what.colours.they.are the.leaves the trees\\
\glt ‘That’s what changes the colours the leaves of the trees.’ (\cite{keye_hnyagwaidatgigowah_2012})
\z
\z 

Similarly, in example \xref{ex:nounincdiscex5}, excorporated \textit{oˀwáhǫh}, \textit{neˀ onę́hę:ˀ}, and \textit{ohsáheˀdaˀ} appear in final position. (The choice of excorporation must have been deliberate, since the verbs used, \textit{ǫgwá:yęˀ} and \textit{de̱ˀjǫ:gwá:yęˀ,} tend to have incorporated nouns.) By placing excorporated nouns in final position, the speaker specifies and emphasizes all the items that are lacking.

\ea\label{ex:nounincdiscex5}
 \gll Nę: tsǫ: ni:yǫ́: ǫgwa:yę́ˀ \exemph{oˀwáhǫh}. Tęˀ gwa̱ˀtóh de̱ˀjǫ:gwa:yę́ˀ \exemph{neˀ} \exemph{onę́hę:ˀ}, \exemph{ohsáheˀdaˀ}.\\
it.is just a.certain.amount we.have meat not any.at.all we.don’t.have the corn, beans.\\
\glt ‘We have very little meat left. We have no corn or beans left.’ (\cite{carrier_legends_2013})
\z

In \xref{ex:nounincdiscex3c}, the choice to use a non-incorporated noun in final position signals a clarification, or even a punch-line, about which type of liquid or sap is under discussion, \textit{nę:gyę́h ohwahdáˀ otséhsdaˀ}. In contrast, earlier in the story in (\ref{ex:nounincdiscex3}a, b), \stem{hnega} ‘liquid, sap’ is incorporated because the main point is not about the type of sap, but instead, about collecting and heating it.

\ea\label{ex:nounincdiscex3}
\ea\label{ex:nounincdiscex3a} 
 \gll Tga:gǫ́:t heyohé: i:sóˀ ędwa\exemph{hnegá:}goˀ,\\
it.is.necessary much more  we.will.collect.liquid\\
\glt ‘We must collect the sap,’

\cfex{\gll ę-dwa-hneg-á:-go-ˀ\\
    \fut-\exsc{1p.in.a}-liquid-{\joinerA}-gather-{\punctual}\\}
\ex \label{ex:nounincdiscex3b}
 \gll ędwa\exemph{hnega}íht, \\
we.will.heat.liquid\\
\glt ‘we will heat it,’

\cfex{\gll ę-dwa-hneg-a-íht\\
    \fut-\exsc{1p.in.a}-liquid-{\joinerA}-break.into.bits.ø.{\punctual}\\
    }
\ex \label{ex:nounincdiscex3c}
 \gll ędwahsrǫ:níˀ \textbf{nę:gyę́h} \textbf{ohwahdáˀ} \exemph{otséhsdaˀ}.\\
we.will.make.it this.one maple syrup.\\
\glt ‘and make this maple syrup.’ (\cite{carrier_legends_2013})
\z
\z 


\chapter{Pronominal prefixes in discourse} \label{ch:Pronominal prefixes in discourse}
Pronominal prefixes also help to manage the flow of information in discourse. Examples are provided in the following sections.


\section{Pronominal prefixes and flexible word order} \label{ch:Pronominal prefixes and flexible word order}
Pronominal prefixes enable the kind of flexible word order (Mithun, 1995, 411) described earlier, in that they remove potential ambiguity of reference. For example, in \xref{ex:whoexsen500}, the pronominal prefix \stem{ha-} \textsc{3s.m.a} in \textit{da-há-yagęˀ} clearly links the description \textit{togyę́h daháyagęˀ} to the noun it modifies, \textit{neˀ haksótgę̱hę:ˀ}.

\ea\label{ex:whoexsen500} 
\gll Onę́h gwaˀ, nę́-gwaˀ ahsha:kni:gę́ˀ neˀ haksotgęhę́:ˀ [togyę́h \exemph{da̱hayagę́ˀ}].\\
Now right.now now-right.now we.saw.him the my.late.grandfather [there he.went.out]\\
\glt ‘And then we saw my late grandfather [(who) was going out from there].’ (\cite{henry_de_2005})
\z

Because pronominal prefixes remove ambiguity of reference, the order of phrases in sentences can be freer. To use an earlier example, the noun-group \textit{neˀ ksó:t} can appear before the verb \textit{aˀa:gę́ˀ} in \xref{ex:flexwordexpron}, but after the verb in \xref{ex:flexwordexpron2}, with no ambiguity concerning who is speaking.

\ea\label{ex:flexwordexpron}
\gll Neˀ giˀ aˀa:gę́ˀ, [\exemph{neˀ} \exemph{ksó:t} aˀa:gę́ˀ,] “Ęhsnigǫ̱há:k gwaˀ to:gyę́h, ahgwí, ęhsahdǫ́: neˀ gayá:ˀ.”\\
The just she.said, [the grandmother, she.said], you.two.be.careful intensifier that.one, don’t you.will.lose the bag\\
\glt \trs{So then our grandmother said, “You two be careful, don’t lose that bag.”} (\cite{henry_de_2005})
\z

\ea\label{ex:flexwordexpron2} 
\gll Neˀ giˀ to: neˀ, [onę́h aˀa:gę́ˀ \exemph{neˀ} \exemph{kso:t},] “Jadohswe̱ˀdá:nih gę́h?” \\
The just that the, [now she.said the grandmother], you.two.are.hungry Q \\
\glt \trs{It’s just that then [our grandma said], “Are you both hungry?”} (\cite{henry_de_2005})
\z


\section{Backgrounding and pronominal prefixes} \label{ch:Backgrounding and pronominal prefixes}
The \textsc{3s.fi} (‘she/someone’) pronominal prefix often refers to \textsc{backgrounded} people. It can even refer to more than one person, either male(s), female(s), or a mixed group. For example, in \xref{ex:indefref2}, which sets the stage for the rest of a story, the \textsc{3s.fi.a} prefix refers to people who are being terrorized by a magic bear. In \xref{ex:indefref2a}, the pronominals in \textit{go̱hdrǫˀs} ‘she/someone is scared’ and \textit{he:yęˀ} ‘she/someone is there’ refer to \textit{neˀ ǫgweh} ‘the person’ (in context, ‘the people’). In \xref{ex:indefref2b}, the pronominal in \textit{dejǫtga̱hnyeh} ‘she/someone doesn’t play’ refers to \textit{eksaˀshǫ:ˀǫh} ‘the children’. Similarly, the prefix in \textit{ęyǫtgę̱hęhsǫ:ˀ} ‘she/someone will get up severally’ refers to \textit{neˀ ǫgwehsǫ:ˀǫh} ‘the people’ in \xref{ex:indefref2d}. 

\ea\label{ex:indefref2}
\ea\label{ex:indefref2a} 
\gll \exemph{Go̱hdrǫˀs} agęˀ hniˀ i:nǫh he:yęˀ neˀ ǫgweh shęh ena:greˀ.\\
someone.is.scared it’s.said and far someone.is.over.there the person that someone.lives\\
\glt ‘They are scared to go too far, the people who live there.’
\cfex{go̱hdrǫˀs\\
go-hdrǫˀs\\
\textsc{3s.fi.a}-scared.{\stative}}
\ex \label{ex:indefref2b}
\gll Eksaˀshǫ:ˀǫh hniˀ ǫ:nęh tęˀ ahsdeh \exemph{dejǫtga̱hnyeh}.\\
children and now not outside someone.doesn’t.play\\
\glt ‘The kids too now, they do not play outside.’
\cfex{dejǫtga̱hnyeh\\
de-j-ǫ-t-ga̱hnye-h\\
{\negative-\repetitive-\textsc{3s.fi.a}-{\semireflexive}-play-{\habitual}}}
\ex \label{ex:indefref2c}
\gll O:nęh hniˀ ohsǫdagwe:gǫh dehodinęhę:ˀ neˀ hęnǫ:gweh shęh ena:greˀ.\\
now and all.night they(males).guard the men that someone.lives.there\\
\glt ‘Now too all night the men are guarding the village.’
\ex \label{ex:indefref2d}
\gll He:yǫ:ˀ \exemph{ęyǫtgę̱hęhsǫ:ˀ} neˀ ǫgwehsǫ:ˀǫh oyę:deht giˀ shęh,\\
every.time someone.will.get.up.severally the people it.is.obvious just that \\
\glt ‘Every time the people get up its obvious it was there,’
\cfex{ęyǫtgę̱hęhsǫ:ˀ\\
ę-yǫ-t-gę̱h-ę-hs-ǫ-:ˀ\\
{\future}-\textsc{3s.fi.a}-{\semireflexive}-get.up-{\joinerA}-{\dislocative}-{\pluralizer}-{\punctual}}
\ex \label{ex:indefref2e}
\gll weˀsgę̱hę:ˀ neˀ hnyagwaiˀdatgiˀgo:wah shęh niwahsǫ:di:s.\\
it.was.there the big.ugly.bear that a.certain.length.of.night\\
\glt ‘the big ugly bear, during the night.’ (\cite{keye_hnyagwaidatgigowah_2012})
\z
\z 

\section{Foregrounding and pronominal prefixes} \label{ch:Foregrounding and pronominal prefixes}
Choosing between \textsc{3m} (3\textsuperscript{rd} person male) or \textsc{3fi} (3\textsuperscript{rd} person feminine-indefinite) prefixes possibly signals that the entities referred to are either foregrounded or backgrounded.\footnote{The analysis in this section is tentative. It is based on Baldwin’s analysis of Onʌyotaʼa:ká: texts, (\cite{baldwin_indefinite_1997}).} In connected speech, \textsc{3m} prefixes meaning ‘he’ or ‘they (males)’ refer to foregrounded entities, and \textsc{3fi} prefixes meaning ‘she/someone’ or ‘they (females or mixed group)’ possibly refer to backgrounded entities (\cite{baldwin_indefinite_1997}). For example, in \xref{ex:indefref1b}, \textit{ǫtahí:neˀ} ‘she or someone walks’ refers to a backgrounded, unspecified \textit{shǫ:gwaˀ-noht} (who was introduced in (\ref{ex:indefref1}a, b). Then, this unknown \textit{shǫ:gwaˀ-noht} is revealed to be a male in \xref{ex:indefref1c}, named as ‘Grandpa’ in \xref{ex:indefref1d}. 

Crucially, in \xref{ex:indefref1c}, the speaker switches to a ‘he’ pronominal in \textit{da̱hanhohá:goˀ} when referring to \textit{shǫ:gwaˀ-noht}. The switch in prefix implies that \textit{shǫ:gwaˀ-noht} has been revealed and is now part of the new or foregrounded information in the discourse. 

\ea\label{ex:indefref1}
\ea\label{ex:indefref1a} 
\gll O:nę́h gwaˀ aˀǫgyatǫ:dę́h to: shǫ:gwaˀ-noht ahsdéh i:wa:kˀah ganǫ́hso:t,\\
now just.now we.heard that some-person outside near standing.house\\
\glt ‘And then suddenly we heard someone outside beside the house,’
\ex \label{ex:indefref1b}
\gll shǫ:gwaˀ-nóht \exemph{ǫtahí:neˀ}.\\
some-person she/someone.walks\\
\glt ‘someone walking.’
\cfex{ǫtahí:neˀ\\
ǫ-t-hah-í:ne-ˀ\\
\textsc{3s.fi.a}-{\semireflexive}-road-lead-{\punctual}}
\ex \label{ex:indefref1c}
\gll O:nę́h gwaˀ ǫgya:tǫ́:deˀ shǫgwaˀ-nóht \exemph{da̱hanhohá:goˀ}.\\
now just.now we.heard some-person he.opened.the.door\\
\glt ‘Then we also heard someone open the door.’
\cfex{da̱hanhohá:goˀ\\
d-a̱-ha-nhoh-á:-go-ˀ\\
{\cislocative}-{\future}-\textsc{3s.m.a}-door-{\joinerA}-pick.up-{\punctual}}
\ex \label{ex:indefref1d}
\gll O:, neˀ giˀ gyę-ǫ-ˀǫh ó:nęh shaknihsó:t sa̱há:yǫˀ. \\
Oh the just this.one-I.guess-I.guess now our.grandfather he.came.home\\
\glt ‘Oh, then I guess our grandpa came home.’ (\cite{henry_de_2005})
\z
\z 


\chapter{Discourse markers} \label{ch:Discourse markers}
\textsc{Discourse markers} (mostly particles) figure prominently in managing the flow of information in discourse. They convey at least four types of meaning, listed in \xref{ex:discoursemarkerintro} and described in the following sections.

\ea\label{ex:discoursemarkerintro} Types and meaning of discourse markers (after the classification scheme proposed in \cite[26]{andersen_pragmatic_2001})
\ea \textsc{exclamations} convey the speaker’s spontaneous reactions; 
\ex \textsc{evidential markers} communicate the speaker’s belief about the nature of the evidence for his/her statements;
\ex \textsc{interactional markers} help to manage turn-taking and other verbal interactions between speaker(s) and listener(s); 
\ex \textsc{topic markers} communicate how utterances relate to one another -- what is topical, what is backgrounded, etc.
\z
\z

Most of the following sections end with lists of links to other sections in this work -- the examples are in other sections. Although unsatisfying, this method of presentation results in much less duplication. (The discourse particles described next were deliberately listed elsewhere, in the Particle dictionary, \sectref{ch:particle dictionary}, in order to showcase relationships among particles.)


\section{Exclamations} \label{ch:Exclamations}
\textsc{Exclamations} are discourse markers (particles) that express emotions or reactions such as shock, surprise, pain, disgust, and anger \xxref{ex:exclamations1}{ex:exclamations2}. 

\ea\label{ex:exclamations1}
 \gll “\exemph{Agí:}!” agyohé:t onę́h awádegoˀ.\\
ow she.cried now she.ran.away\\
\glt \trs{“Ow!” she cried and ran away.} (\cite{keye_circle_2016}, Circle Book 10, The Magic Chair)
\z

\ea\label{ex:exclamations2}
 \gll “O: \exemph{tsę:},” aˀa:gę́ˀ, “ahsdéh jatgáhnye̱hah!” \\
oh my she.said outside you.two.go.and.play \\
\glt \trs{“Oh my,” she said, “go play outside!} (\cite{henry_de_2005})
\z

The following exclamations described in the Particle dictionary, \sectref{ch:particle dictionary}.

\begin{CayugaRelated}
\item{} \textit{Agi:} \trs{ouch!}\\
\item{}\textit{Aju:} \trs{yikes!}\\
\item{}\textit{Gwe:} \trs{well!}\\
\item{}\textit{Hoh, Ho:} \trs{what the…?!}\\
\item{}\textit{Hoho:} \trs{aha!}, \trs{oh no!}\\
\item{}\textit{Otgǫˀ, Hotgǫˀǫh} \trs{what the…?!}, \trs{for Heaven’s sake!}\\
\item{}\textit{O:, O:ò:} \trs{oh}\\
\item{}\textit{Trehs giˀ gyę:ˀ} \trs{my goodness!}, \trs{too bad!}, \trs{that’s amazing!}\\
\item{}\textit{Tsę:} \trs{oh my!}
\end{CayugaRelated}

\section{Evidential markers} \label{ch:Evidential markers}
\textsc{Evidential markers} express the speaker’s opinion about whether his or her statements are first-hand, hearsay, etc. Types of evidential words and phrases are described in the following sections. (Also see \textit{Mood and negation prefixes}, \sectref{Mood and negation prefixes}.) 


\subsection{Quotation markers} \label{ch:Quotation markers}
\textsc{Quotation markers} are words or phrases introducing direct quotations. These markers are generally versions of the verb ‘to say’ \xxref{ex:quotex}{ex:quotex2}.

\ea\label{ex:quotex}
 \gll Gwi:déh \exemph{ha:dǫ́h}, “Nawęˀdáˀ go:gáˀs Tina.”\\
Peter he.said sugar she.likes.it Tina\\
\glt \trs{Peter said, “Tina likes candy.”}
\z

\ea\label{ex:quotex2}
 \gll O:nę́h \exemph{aˀa:gę́ˀ}, “Ja̱ˀsnę́ht to:gyę́h waˀjíh neˀ sǫ:gwaˀnóht ęyagonǫ́hnyaˀk.”\\
now she.said you.two.get.down there right.now the someone someone.will.get.hurt\\
\glt \trs{And then she said, “Both of you get down before somebody gets hurt.”} (\cite{henry_de_2005})
\z

\subsection{Hearsay markers} \label{ch:Hearsay markers}
\textsc{Hearsay markers} express the idea that the information presented is hearsay or second-hand \xref{ex:hearsay1}. These markers are also generally versions of a verb meaning ‘it is said’. 

\ea\label{ex:hearsay1}
 \gll O:nę́”, \textbf{agęˀ}, neˀ Government nitawe:nǫ́: hǫgwéh, aha:yǫ́ˀ neˀ toh. Ahshǫgwa̱hó:wiˀ. O:nę́h, \exemph{agę́ˀ}, dęyǫkidé:niˀ. \\
now it.is.said the Government he.was.sent man he.arrived the there he.told.us now it.is.said they.will.change.us\\
\glt \trs{Now then the Government man arrived here. He said that they would change us now.} (\cite{henry_de_2005})
\z

\ea\label{ex:hearsay3}
 \gll Hǫwadigówaneh \exemph{wá:dǫh} Onǫdowáˀga:ˀ, Hiˀnoˀ hayá:sǫh. \\
he.is.their.leader it.is.said Seneca, Hiˀnoˀ he.is.called\\
\glt ‘It is said that the one called Hiˀnoˀ was the leader of the Senecas.’ (\cite{carrier_legends_2013})
\z


\subsection{Doubt or certainty markers} \label{ch:Doubt or certainty markers}
\textsc{Doubt} or \textsc{certainty markers} (particles or particle groups) express the speaker’s degree of confidence in what he or she is saying. 

\ea\label{ex:doubtcertainty}
 \gll Ó:, degyega̱hné:ˀ giˀ. O:nę́h gyę́:ˀ \exemph{ǫh} ǫkwái. \\
oh, she.is.looking.out just now this.one I.guess the.food.is.cooked \\
\glt ‘Oh, she is looking out. The food must be cooked now.’ (\cite[294]{mithun_watewayestanih_1984}, Dwade:kǫ́:nih dialogue) 
\z

\ea\label{ex:doubtcertainty2}
 \gll Mary \exemph{ǫh} ne:ˀ gohwihsdagaˀdeˀ. \\
Mary I.wonder it.is she.has.a.lot.of.money\\
\glt ‘Maybe Mary has a lot of money.’
\z

The following doubt or certainty markers are described in the \textit{Particle dictionary}, \sectref{ch:particle dictionary}.

\begin{CayugaRelated}
\item{}\textit{A:yę:ˀ} \trs{I guess}, \trs{it seems}\\
\item{}\textit{Do:gaˀ} \trs{I don’t know}\\
\item{}\textit{Ga:t giˀ shęh} \trs{maybe}, \trs{or maybe}\\
\item{}\textit{Ga:t giˀ shęh tęˀ} \trs{maybe not}, \trs{or not}\\
\item{}\textit{Giˀ shęh} \trs{maybe}, \trs{or maybe}\\
\item{}\textit{Gwaˀ ti:gę:} \trs{plainly}, \trs{clearly}, \trs{as it is}\\
\item{}\textit{Gyę:ˀ ǫh} \trs{maybe}, \trs{I guess}, \trs{I wonder}\\
\item{}\textit{Gyę:gwaˀ giˀ shęh hwaˀ} \trs{maybe this time}\\
\item{}\textit{Gyę:gwaˀ hne:ˀ hwaˀ} \trs{just maybe}\\
\item{}\textit{Hę:-gyęh, hę:gyeh} \trs{no matter}\\
\item{}\textit{Hę:-gyęh tsǫ:} \trs{it doesn’t matter}, \trs{never mind}\\
\item{}\textit{Ne:ˀ giˀ shęh hwaˀ} \trs{maybe this time}\\
\item{}\textit{Ne:ˀ giˀ gyę:ˀ ǫh} \trs{I guess}\\
\item{}\textit{Ne:ˀ ǫh} \trs{I guess it is}\\
\item{}\textit{O: tęˀ ǫh ga:t giˀ shęh gyę:gwaˀ} \trs{maybe, maybe not}\\
\item{}\textit{O: tęˀ ǫh ga:t giˀ shęh ne:ˀ hwaˀ} \trs{maybe, maybe not}\\
\item{}\textit{Oh, ǫ:, ǫ} \trs{I guess}, \trs{I wonder (if)}\\
\item{}\textit{Oh ne:ˀ} \trs{maybe}\\
\item{}\textit{Tęˀ giˀ shęh hwaˀ daˀǫ} \trs{maybe not}
\end{CayugaRelated}


\subsection{Possibility markers} \label{ch:Possibility markers}
\textsc{Possibility markers} (particles or particle groups) provide the speaker’s opinion about whether something is possible or permissible. 

\ea\label{ex:possibility1}
 \gll Swasha:ˀsé:k shęh taˀdewę̱hnihsragé: \exemph{ęwa:dǫ́ˀ} daedwadęnǫ̱hǫ́:nyǫ:ˀ. \\
you.remember that every.day it.is.possible we.should.give.thanks\\
\glt ‘Remember every day is a good day, and we can give thanks for that.’ (\cite{carrier_legends_2013})
\z

The following possibility markers are described in the \textit{Particle dictionary}, \sectref{ch:particle dictionary}.

\begin{CayugaRelated}
\item{}\textit{Ęwa:dǫˀ} \trs{yes, you may}, \trs{it is permissible}, \trs{you can}\\
\item{}\textit{Ęwa:dǫˀ gęh} \trs{may I}, \trs{may we}\\
\item{}\textit{Ęwa:dǫˀ giˀ shęh} \trs{maybe}, \trs{a possibility}\\
\item{}\textit{Tęˀ ta:wa:dǫh} \trs{you may not}
\end{CayugaRelated}



\section{Interactional markers} \label{ch:Interactional markers}
\textsc{Interactional markers} (particles or particle groups) help to manage interactions between speaker(s) and listener(s). Various types are described in the following sections.

\subsection{Agreement markers} \label{ch:Agreement markers}
\textsc{Agreement markers} (particles or particle groups) express the speaker’s agreement with previous statements or ideas. 

\ea\label{ex:agreement1}
\gll Trehs taˀdeyodriˀsdagé: ga̱ha:gǫ́: waˀne:ˀ.\\
too not.much.noise in.woods today\\
\glt ‛The woods are too quiet.’

 \gll Ęhę:ˀ, \textbf{gwahs} \textbf{ǫ́:weh} \exemph{seˀ}.\\
yes very it.is.true you.know\\
\glt ‛Yes, much too quiet.’ (\cite{carrier_legends_2013})
\z

The following agreement markers are described in the \textit{Particle dictionary}, \sectref{ch:particle dictionary}.

\begin{CayugaRelated}
\item{}\textit{Do:gęhs} \trs{exactly}, \trs{just so}\\
\item{}\textit{Do:s giˀ} \trs{just so}, \trs{indeed}\\
\item{}\textit{Ęhęˀ} \trs{yes}\\
\item{}\textit{Ęhęˀ ę:ˀ} \trs{yes indeed}\\
\item{}\textit{Ęhęˀ gyę:ˀ} \trs{yes indeed}\\
\item{}\textit{Ęhęˀ seˀ} \trs{it is so}, \trs{yes indeed}\\
\item{}\textit{Gwahs ǫ:węh seˀ} \trs{indeed}, \trs{for sure}\\
\item{}\textit{Ne:ˀ} \trs{just so}, \trs{indeed}\\
\end{CayugaRelated}


\subsection{Disagreement markers} \label{ch:Disagreement markers}
\textsc{Disagreement markers} express the speaker’s disagreement with previous statements or ideas \xref{ex:disagreement}. 


\ea\label{ex:disagreement}
 \gll \exemph{Tęˀ} giˀ hne:ˀ \exemph{dedo:gę́hs} deˀsá:dǫh. \\
not just in.fact it.isn’t.true you.are.not.saying.it\\
\glt ‘No, it isn’t really true what you are saying.’
\z

The following disagreement markers are described in the \textit{Particle dictionary}, \sectref{ch:particle dictionary}.

\begin{CayugaRelated}
\item{}\textit{Hęˀęh} \trs{no}\\
\item{}\textit{Tęˀ} \trs{no}, \trs{not}\\
\item{}\textit{Tęˀ daˀǫ} \trs{it will never happen}, \trs{definitely not}\\
\item{}\textit{Tęˀ gyę:ˀ ǫh, tęˀ gęˀǫh} \trs{not really}\\
\item{}\textit{Tęˀ gwahs ǫ:weh} \trs{not really}, \trs{not quite}\\
\item{}\textit{Tęˀ hne:ˀ} \trs{definitely not}\\
\item{}\textit{Tęˀ seˀ} \trs{not really}, \trs{but then not really}\\
\item{}\textit{Tęˀ tǫ ne:ˀ, Tęˀ to ne:ˀ} \trs{not really}\\
\item{}\textit{Tęˀ dedo:gęhs} \trs{it isn't true}, \trs{not really}\\
\end{CayugaRelated}



\subsection{Acknowledgement markers} \label{ch:Acknowledgement markers}
\textsc{Acknowledgement markers} signal that the speaker is responding to something previously stated or implied \xxref{ex:acknowledgement}{ex:acknowledgement2}. 

\ea\label{ex:acknowledgement}
 \gll Ęhsnéˀ gęh \exemph{diˀ}? \\
we.two.will.go.together Q then\\
\glt ‘Are you coming along then?’, ‘So, are you coming along?’
\z

\ea\label{ex:acknowledgement2}
 \gll \exemph{Haoˀ} \exemph{diˀ} \exemph{sáh}, ękni:nǫ́ˀ giˀ gyę́:ˀǫh.\\
o.k. so you.know I.will.buy.it just I.guess\\
\glt ‘O.k., I guess I’ll buy it then.’ (\cite[225]{mithun_watewayestanih_1984}, Agyaˀdawíˀtraˀ dialogue)
\z

The following acknowledgement markers are described in the \textit{Particle dictionary}, \sectref{ch:particle dictionary}.

\begin{CayugaRelated}
\item{}\textit{Diˀ} \trs{so}, \trs{then}\\
\item{}\textit{Haoˀ} \trs{o.k.}\\
\item{}\textit{Haoˀ dęˀ nyoh} \trs{o.k. then}\\
\item{}\textit{Haoˀ diˀ sah} \trs{alright}, \trs{o.k.}\\
\item{}\textit{Nya:węh} \trs{thank you}, \trs{thanks}\\
\item{}\textit{Nyoh} \trs{you're welcome}, \trs{alright}, \trs{o.k.}\\
\end{CayugaRelated}



\subsection{Shared knowledge markers} \label{ch:Shared knowledge markers}
\textsc{Shared knowledge markers} convey the idea that the topic at hand is known to both speaker and listener \xref{ex:sharedknowledge}. 

\ea\label{ex:sharedknowledge}
 \gll Sǫheh gyę:ˀ \exemph{nóne:ˀ} gá:dǫh deyonadáwęnyeˀ.\\
nighttime the.one you.know I.say they.walk.about\\
\glt ‘As I said, at night, you know, they walk around.’ (\cite{mithun_how_1980})
\z

The following shared knowledge markers are described in the \textit{Particle dictionary}, \sectref{ch:particle dictionary}.

\begin{CayugaRelated}
\item{}\textit{Gęh} \trs{mind you}\\
\item{}\textit{Neˀ} \trs{that is}, \trs{the}\\
\item{}\textit{None:ˀ, Nǫne:ˀ} \trs{mind you}, \trs{you know}\\
\item{}\textit{Ne:ˀ seˀ gyę:ˀ} \trs{you know}\\
\item{}\textit{Seˀ} \trs{you know}
\end{CayugaRelated}


\section{Topic markers} \label{ch:Topic markers}
\textsc{Topic markers} help speakers and listeners to identify and manage topics. Several types are described in the following sections.


\subsection{Topic starters and conclusions} \label{ch:Topic starters and conclusions}
\textsc{Topic starters} and \textsc{conclusions} signal the beginning or end of a topic \xref{ex:startendtopic}.  

\ea\label{ex:startendtopic}
 \gll \exemph{Da} \exemph{ne:ˀ} onę́h toh niyáwę̱hdreˀ. \\
and it.is now that.one what.is.going.to.happen\\
\glt ‘And now this is what is going to happen.’
\z

The following topic starters and conclusions are described in the \textit{Particle dictionary}, \sectref{ch:particle dictionary}.

\begin{CayugaRelated}
\item{}\textit{Da:} \trs{and}\\
\item{}\textit{Da: ne:ˀ onęh} \trs{and now}\\
\item{}\textit{Da: nę: dah} \trs{and now}\\
\item{}\textit{Da: ne:ˀ} \trs{that's it}\\
\item{}\textit{Da: neˀ toh} \trs{that's all}\\
\item{}\textit{Hę:-dah} \trs{and now}\\
\item{}\textit{Hę:} (syllable)\\
\item{}\textit{To tsǫ:} \trs{that’s all}\\
\end{CayugaRelated}



\subsection{Topic continuation markers} \label{ch:Topic continuation markers}
\textsc{Topic continuation markers} express the idea that the speaker is continuing a topic by adding or asking for new information \xref{ex:continuingtopic}. 

\ea\label{ex:continuingtopic}
 \gll Dęˀ \exemph{hniˀ} hoˀdę́ˀ hoihoˀdę́hsro̱ˀdeˀ? \\
What and kind his.kind.of.work\\
\glt ‘And what does he do?’
\z

The following topic continuation markers are described in the \textit{Particle dictionary}, \sectref{ch:particle dictionary}.

\begin{CayugaRelated}
\item{}\textit{Hęˀ} \trs{also}, \trs{too}\\
\item{}\textit{Hniˀ} \trs{and}, \trs{also}, \trs{too}\\
\item{}\textit{Da: ne:ˀ hniˀ} \trs{too}, \trs{also}, \trs{and}\\
\item{}\textit{Neˀ tsǫ: gwaˀ toh} \trs{also}\\
\item{}\textit{Ne:ˀ hęˀ hne:ˀ} \trs{too}, \trs{also}\\
\item{}\textit{Ne:ˀ hniˀ ne:ˀ} \trs{and that also}\\
\item{}\textit{Toh-geh} \trs{and then}
\end{CayugaRelated}



\subsection{Topic changers} \label{ch:Topic changers}
\textsc{Topic changers} signal that the speaker is changing topics or introducing new information that is thematically different from what was previously talked about. In \xref{ex:changetopic}, David and Pete have just exchanged greetings, and then David uses \textit{gwé:} to signal a change in topic. 

\ea\label{ex:changetopic}
 \gll \exemph{Gwé:}, í:s gęh satse:nę́ˀ to:gyę́h so:wa:s?\\
well, you Q its.your.pet that.one dog\\
\glt ‘Well, is that your dog?’ (\cite[317]{mithun_watewayestanih_1984}, Sanahsgwaę́ˀ Gęh Sga̱hoˀdę́:ˀęh dialogue)
\z

The following topic changers are described in the \textit{Particle dictionary}, \sectref{ch:particle dictionary}.

\begin{CayugaRelated}
\item{}\textit{Giˀ hne:ˀ} \trs{but}, \trs{however}\\
\item{}\textit{Gwe:} \trs{well}\\
\item{}\textit{Ne:ˀ tsǫ: shęh} \trs{it's just that}
\end{CayugaRelated}



\subsection{Focus, contrastive focus, and emphasis markers} \label{ch:Focus, contrastive focus, and emphasis markers}
\textsc{Focus markers} highlight new information introduced by the speaker \xref{ex:focus}. \textsc{Contrastive focus markers} signal that the new information being presented is a correction or contrast with previous information \xref{ex:contrastivefocus}. \textsc{Emphasis markers} highlight old or background information \xxref{ex:focus}{ex:emphasisingeneral}. 

\ea\label{ex:focus}
 \gll \exemph{Neˀ} \exemph{seˀ} hǫwę́:dǫh. \\
the you.know she.means.him\\
\glt ‘He is \emph{just the one} she means.’
\z

\ea\label{ex:contrastivefocus}
 \gll Ohyadǫhsra:sé: i:wí: \exemph{tęˀ} \exemph{hne:ˀ} ohyadǫ̱hsra:gá:yǫh.\\
new.book I.want not in.fact old.book\\
\glt ‘I want the \emph{new} book, not the \emph{old} one.’
\z

\ea\label{ex:emphasisingeneral}
 \gll Nę́: swatgahtóh gra̱he:t. Weˀsgęhę́: neˀ hnyagwái:. Heˀtgę́h \exemph{dęˀ} \exemph{hniˀ} hegano̱ˀjoyaǫní: tó:gyęh!\\
look you.all.look.at.it tree it.was.here the bear high what and the.marks.are.made.up.there those.ones\\
\glt ‘Look at this pine tree. The bear has been here. See \emph{just how high up} those marks are!’ (\cite{carrier_legends_2013})
\z

The following focus, contrastive focus, and emphasis markers are described in the \textit{Particle dictionary}, \sectref{ch:particle dictionary}.

\begin{CayugaRelated}
\item{}\textit{Dęˀ hniˀ} \trs{for sure}\\
\item{}\textit{Giˀ ne:ˀ} \trs{it's just}\\
\item{}\textit{Giˀ tsǫ:} \trs{just}, \trs{really}\\
\item{}\textit{Gwahs} \trs{anyway}\\
\item{}\textit{Gyę:ˀ} \trs{just the one}\\
\item{}\textit{Gyę:ˀ hne:ˀ tęˀ neˀ} \trs{it was this one, (not that one)}\\
\item{}\textit{Hne:ˀ} \trs{in fact}\\
\item{}\textit{Neˀ seˀ} \trs{that’s just the one}, \trs{that’s just who}\\
\item{}\textit{Neˀ gyę:ˀ neˀ gyę:ˀ, negęˀnagęˀ} \trs{that is what}\\
\item{}\textit{Neˀ gyę:ˀ nę ne:ˀ} \trs{emphasis}\\
\item{}\textit{Neˀ to gyę:ˀ} \trs{that’s what}\\
\item{}\textit{Ne:ˀ/Neˀ giˀ gyę:ˀ} \trs{that’s just it}, \trs{that’s it for sure}\\
\item{}\textit{Ne:ˀ diˀ gęh} \trs{is that it then?}, \trs{is that…?}\\
\item{}\textit{Ne:ˀ giˀ} \trs{just}\\
\item{}\textit{Ne:ˀ gyę:ˀ neˀ} \trs{that really is}\\
\item{}\textit{Ne:ˀ he:gę:} \trs{just}, \trs{only}, \trs{all}\\
\item{}\textit{Ne:ˀ hne:ˀ ne:ˀ} \trs{in fact, it is}, \trs{it IS}\\
\item{}\textit{Ne:ˀ neˀ} \trs{it is}, \trs{that is}, \trs{that’s what}\\
\item{}\textit{Shęh nohgeh} \trs{even}\\
\item{}\textit{Tęˀ gyę:ˀ nę neˀ} \trs{NOT}, \trs{what on earth?}\\
\item{}\textit{Tęˀ hne:ˀ neˀ} \trs{not THAT ONE}\\
\item{}\textit{tęˀ seˀ} \trs{not really}, \trs{but then, not really}
\end{CayugaRelated}


