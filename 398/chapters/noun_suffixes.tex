\chapter{Noun suffixes} \label{Noun suffixes}
Noun suffixes attach exclusively to nouns. (They are unlike the word suffixes described in \sectref{Word suffixes}, which attach to nouns, verbs, or particles). Noun suffixes appear in the order shown in \tabref{figtab:1:nounsuffix}. They can be classified into two types, “noun identifiers”, whose function is to identify the relevant words as \textsc{nouns}, and the \textsc{locative} noun suffixes.


\begin{table}
\caption{Noun suffixes}
\label{figtab:1:nounsuffix}
\begin{tabular}{lll}
\lsptoprule
stem & noun identifier & locative noun suffixes\\
\midrule
noun & \stem{-aˀ} \textsc{\nsf} & \stem{-gǫ:} \textsc{in} (\textsc{internal locative})\\
& \stem{-ˀd-aˀ} \textsc{increment-\nsf} & \stem{-:kˀah} \textsc{“beside” locative} \\
& \stem{-hsr-aˀ} \textsc{\nominalizer-\nsf} & \stem{-kdagyeˀ} \textsc{“alongside” locative}\\
& \stem{-tr-aˀ} \textsc{\nominalizer-\nsf} & \\
& \stem{-hkw-aˀ} \textsc{\instrumental} & \\
\lspbottomrule
\end{tabular}
\end{table}


\section{Noun identifier suffixes} \label{Suffixes that identify words as being nouns}
Some suffixes either identify words as nouns, or turn verbs into nouns. They are described in this and the following sections.


\subsection{\stem{-aˀ} \textsc{noun stem former} ({\nsf})} \label{[-aˀ]}
The \stem{-aˀ} \textsc{\nsf} \xref{ex:nsf} attaches to basic nouns (\sectref{ch:Basic nouns}). The purpose or meaning of the suffix is to identify the word as a noun.

\ea\label{ex:nsf}
\ea ga̱ˀdréhdaˀ\\
\gll ga̱-ˀdréhd-aˀ\\
 \textsc{3s.a}-car-{\nounstemformer}\\
\glt `car'
\cfex{ga̱ˀdréhde:s\\
	\gll ga̱-ˀdréhd-e:s\\
	\textsc{3s.a}-car-long.{\stative}\\
	\glt `long car'}
\ex awę́nǫ̱hgraˀ\\
\gll awę́nǫ̱hgr-aˀ\\
 {\noprefix}-weed-{\nounstemformer}\\
\glt `weeds'
\cfex{owę́nǫ̱hgre:s\\
	\gll o-wę́nǫ̱hgr-e:s\\
	\textsc{3s.p}-weed-long.{\stative}\\
	\glt `tall weeds'}
\ex onóˀjaˀ\\
\gll o-nóˀj-aˀ\\
 \textsc{3s.p}-tooth-{\nounstemformer}\\
\glt `teeth'
\cfex{onóˀje:s\\
	\gll o-nóˀj-e:s\\
	\textsc{3s.p}-tooth-long.{\stative}\\
	\glt `a fang'}
\z
\z

\subsection{Noun increments (\textsc{incr})} \label{Noun increments}
The \stem{-ˀd} \textsc{incr} attaches to some atypical words (\ref{Atypical words}) when such words are incorporated (\ref{ex:nounincrex}, for incorporation, see \sectref{ch:Incorporated noun stems}). 

\ea\label{ex:nounincrex} 
\ea hǫgweˀdí:yo:\\
\gll h-ǫgwe-ˀd-í:yo:\\
 \textsc{3s.m.a}-person-{increment}-nice.{\stative}\\
\glt `he is a nice person'
\cfex{hǫ́:gweh \\
\gll h-ǫ́:gweh \\
\textsc{3s.m.a}-person\\
\glt ‘man’  atypical word}
\ex sgagwéni̱ˀda:t\\
\gll s-ga-gwéni̱-ˀd-a:-t\\
 {\repetitive}-\textsc{3s.a}-penny-\exsc{increment-joinerA}-stand.{\stative}\\
\glt `one penny'
\cfex{gwé:nihs ‘penny’ (loan word, \textsc{noun})}
\ex ga̱hnyagwáiˀda̱ˀgę:t\\
\gll ga̱-hnyagwái-ˀd-a̱-ˀgę:t\\
 \textsc{3s.a}-bear-\exsc{increment-joinerA}-light.coloured.{\stative}\\
\glt `polar bear'
\cfex{ hnyagwái ‘bear’ (animal name, \textsc{noun})}
\z
\z


\subsection{\stem{-tr-aˀ} and \stem{-(h)sr-aˀ} \textsc{nominalizer-noun stem former} (\textsc{\nominalizer-\nsf})} \label{[-tr-aˀ] and [-(h)sr-aˀ]}
When added to verb stems, the \stem{-tr-aˀ} and \stem{-(h)sr-aˀ} \textsc{\nominalizer-\nsf} combination results in a basic noun (\ref{ex:traex}, also described in \sectref{ch:Basic nouns with nominalizer suffix}).

\ea\label{ex:traex}
\ea atgáhnye̱ˀtraˀ\\
\gll at-gáhnye̱ˀ-tr-aˀ\\
 \exsc{ø.prefix.\semireflexive}-play-{\nominalizer}-{\nsf}\\
\glt `sports, games'
\cfex{satgáhnyeˀ\\
	\gll s-at-gáhnyeˀ\\
	\textsc{2s.a}-{\semireflexive}-play.{\noaspect}\\
	\glt `play!'}
\ex gaihwanéˀaksraˀ\\
\gll ga-ihwa-néˀak-sr-aˀ\\
 \textsc{3s.a}-matter-sin-{\nominalizer}-{\nsf}\\
\glt `sin'
\cfex{ęgrihwanéˀa:k\\
	\gll ę-g-rihwa-néˀa:k\\
	\fut-\exsc{1s.a}-matter-sin.{\zeropunctual}\\
	\glt `I will sin'}
\ex oyę́hsraˀ\\
\gll o-yę́-hsr-aˀ\\
 \textsc{3s.p}-lie-{\nominalizer}-{\nsf}\\
\glt `blanket(s), quilt(s)'
\cfex{gá:yęˀ\\
	\gll gá:-yę-ˀ\\
	\textsc{3s.a}-lie-{\stative}\\
	\glt `it is lying on the ground'}
\ex adeˀnyędę́hsraˀ\\
\gll ade-ˀnyędę́-hsr-aˀ\\
 \exsc{ø.prefix.\semireflexive}-measure-{\nominalizer}-{\nsf}\\
\glt `measurement(s)'
\cfex{ęhsade̱ˀnyę́:dęˀ\\
	\gll ę-hs-ade̱-ˀnyę́:dę-ˀ\\
	\fut-\textsc{2s.a}-{\semireflexive}-measure-{\punctual}\\
	\glt `you will try, measure'}
\ex ga̱hyádǫ̱hsraˀ\\
\gll ga̱-hyádǫ̱-hsr-aˀ\\
 \textsc{3s.a}-write-{\nominalizer}-{\nsf}\\
\glt `paper'
\cfex{ehyá:dǫh\\
	\gll e-hyá:dǫ-h\\
	\textsc{3s.fi.a}-write-{\habitual}\\
	\glt `she or someone writes'}
\z
\z


\subsection{\stem{-ǫ-:-ˀ} \textsc{stative-nominalizer-noun stem former} ({\stative-\nominalizer-\nsf})} \label{[-ǫ-:-ˀ]}
The \stem{-ǫ-:-ˀ} \textsc{\stative-\nominalizer-\nsf} combination \xref{ex:examplestatnom} turns a stative aspect verb into a type of abstract noun (see \sectref{ch:Stative nouns and agentive stative nouns}).

\ea\label{ex:examplestatnom}
gayaˀdowéhdǫ:ˀ\\
\gll ga-yaˀdowéhd-ǫ-:-ˀ\\
 \textsc{3s.a}-think-\stat-\nominalizer{-\nsf}\\
\glt `the idea of thinking'
\cfex{dewagyaˀdowehdǫh \\
	\gll de-wag-yaˀdowehd-ǫh\\
	{\dualic}-\textsc{1s.p}-think.{\stative}\\
	\glt ‘I have already thought about it’, `I am thinking about it'}
\z


\subsection{\stem{-hkw-haˀ} \textsc{instrumental} ({\instrumental}-{\habitual} combination) } \label{[-hkwaˀ]}
The \stem{-hkw-haˀ} {\instrumental}-{\habitual} combination (spelled <-hkwaˀ> or <-hgwaˀ>) attaches to verb stems, resulting in a verb in the habitual aspect \xref{ex:kwaex}, which functions as an “instrumental noun” (for which, see \sectref{ch:Instrumental nouns}). The \stem{-hkw-haˀ} {\instrumental}-{\habitual} combination is glossed as \stem{-hkwaˀ} {\instrumental} throughout this book, except where the \stem{-hkw-haˀ} analysis is needed for clarity.

\ea\label{ex:kwaex}
\ea ehyádǫ̱hkwaˀ\\
\gll e-hyádǫ̱-hkwaˀ\\
\textsc{3s.fi.a}-write-{\instrumental}\\
\glt ‘pencil’ (literally, ‘what she or one uses for writing’)\\

\cfex{ehyá:dǫh\\
	\gll e-hyá:dǫ-h\\
	\textsc{3s.fi.a}-write-{\habitual}\\
	\glt `she writes'}

\ex ǫtgahiˀdáhkwaˀ \\
\gll ǫ-t-gahiˀd-á-hkwaˀ\\
\textsc{3s.fi.a}-{\semireflexive}-play.with-{\joinerA}-{\instrumental}\\
\glt ‘toy’ (literally, what she or one uses for playing)

\cfex{ǫtgáhiˀtaˀ\\
	\gll ǫ-t-gáhiˀt-haˀ\\
	\textsc{3s.fi.a}-{\semireflexive}-play.with-{\habitual}\\
	\glt `what she plays with'}

\ex wadǫnyedáhkwaˀ\\
\gll w-ad-ǫnyed-á-hkwaˀ\\
 \textsc{3s.a}-{\semireflexive}-cause.to.breathe-{\joinerA}-{\instrumental}\\
\glt `gill'

\cfex{wadǫ́nyetaˀ,\\
	\gll w-ad-ǫ́nyet-haˀ\\
	\textsc{3s.a}-{\semireflexive}-cause.to.breathe-{\habitual}\\
	\glt  ‘how it breathes’ `it’s a breather'}
\ex godi:tshe:nę́ˀ hadiˀdrǫdáhkwaˀ  \\
\gll godi:tshe:nę́ˀ hadi-ˀdrǫ-d-á-hkwaˀ\\
tame.animals \textsc{3p.a}-live-{\euphonicd}-\exsc{joinerA}-{\instrumental}\\
\glt ‘pasture’

\cfex{gadíˀdrǫˀ\\
\gll gadí-ˀdrǫˀ\\
\textsc{3p.a}-live.{\stative}\\
\glt ‘they (animals) live’ (describing a shed or dog house, etc.)
}

\ex ǫdręnáęda̱hkwaˀ\\
\gll ǫ-d-ręn-á-ę-d-a̱-hkwaˀ\\
 \textsc{3s.fi.a}-{\semireflexive}-song-{\joinerA}-lie-{\euphonicd}-\exsc{joinerA}-{\instrumental}\\
\glt `church'
\cfex{honadręnáęˀ \\
\gll  hon-ad-ręn-á-ęˀ\\
\textsc{3ns.m.p}-{\semireflexive}-song-{\joinerA}-lie.{\stative}\\
\glt ‘they are praying’ (Christian activity)}
\z
\z


\subsubsection*{\stem{+ędahkwaˀ}, \stem{+odahkwaˀ}, and \stem{+ǫdahkwaˀ} (\stem{verb-\instrumental})} \label{[+ędahkwaˀ], [+odahkwaˀ], and [+ǫdahkwaˀ]}
Many instrumental nouns (\sectref{ch:Instrumental nouns}) end with \stem{+ędahkwaˀ}, \stem{+odahkwaˀ}, or \stem{+ǫdahkwaˀ}, which are combinations of an obligatorily-incorporating verb and the \stem{-hkwaˀ} \textsc{\instrumental} suffix. The approximate meanings of each combination are illustrated in \xxref{ex:kwaex3}{ex:kwaex5}.

\ea\label{ex:kwaex3} \stem{+ę-d-a-hkwaˀ} lie.{\stative}-{\euphonicd}-\exsc{joinerA}-{\instrumental} ‘a place to put something’\\
ehwihsdáęda̱hkwaˀ\\
\gll e-hwihsd-á-ęda̱hkwaˀ\\
 \textsc{3s.fi.a}-money-{\joinerA}-lie.{\stative}-{\euphonicd}-\exsc{joinerA}-{\instrumental}\\
\glt `a bank'
\z


\ea\label{ex:kwaex4} \stem{+od-a-hkwaˀ } stand-{\joinerA}-{\instrumental} ‘a standing object’\\
ǫdręnodáhkwaˀ\\
\gll ǫ-d-ręn-odáhkwaˀ\\
 \textsc{3s.fi.a}-{\semireflexive}-song-stand.{\stative}-{\joinerA}-{\instrumental}\\
\glt `musical instruments'
\z


\ea\label{ex:kwaex5} \stem{+ǫd-a-hkwaˀ} attached/sticking.out-{\joinerA}-{\instrumental} ‘an object that sticks out’, ‘an attached object’\\
ǫtnaˀdáǫda̱hkwaˀ\\
\gll ǫ-t-naˀd-á-ǫda̱hkwaˀ\\
 \textsc{3s.fi.a}-{\semireflexive}-bread-{\joinerA}-attached/sticking.out.{\stative}-{\joinerA}-{\instrumental}\\
\glt `a bread pan'
\z


\section{Locative noun suffixes} \label{Locative noun suffixes}
The three locative noun suffixes (see \tabref{figtab:1:nounsuffix}) denote various types of location, including, ‘in’, ‘at’, ‘beside’, and ‘alongside’. They are described next.

\subsection{\stem{-gǫ:} \textsc{in} (internal locative)} \label{[-gǫ:]}
The \stem{-gǫ:} \textsc{in} (\textsc{internal locative}) ending means ‘in, inside, under’. It can attach to basic nouns (\ref{ex:gonex}, \sectref{ch:Basic nouns}) and is required for some body part nouns (\ref{ex:gonex3}, \sectref{ch:Body part nouns}).

\ea\label{ex:gonex} basic nouns
\ea ga̱hǫ́wagǫ:\\
\gll ga̱-hǫ́w-a-gǫ:\\
 \textsc{3s.a}-boat-{\joinerA}-\exsc{in}\\
\glt `in a boat'
\cfex{ga̱hǫ́:waˀ\\
	\gll ga̱-hǫ́:w-aˀ\\
	\textsc{3s.a}-boat-{\nounstemformer}\\
	\glt `boat'}
\ex oyádagǫ:\\
\gll o-yád-a-gǫ:\\
 \textsc{3s.p}-track-{\joinerA}-\exsc{in}\\
\glt `in the basement, track'
\cfex{oyá:daˀ\\
	\gll o-yá:d-aˀ\\
	\textsc{3s.p}-track-{\nounstemformer}\\
	\glt `basement, track'}
\ex ohǫ́dagǫ:\\
\gll o-hǫ́d-a-gǫ:\\
 \textsc{3s.p}-sapling-{\joinerA}-\exsc{in}\\
\glt `in the bushes'
\cfex{ohǫ́:daˀ\\
	\gll o-hǫ́:d-aˀ\\
	\textsc{3s.p}-sapling-{\nounstemformer}\\
	\glt `whip, sapling'}
\ex ǫgwanadá:gǫ:\\
\gll ǫgwa-nad-á:-gǫ:\\
 \textsc{1p.p}-town-{\joinerA}-\exsc{in}\\
\glt `in our community'
\cfex{ǫgwa:ná:daˀ\\
	\gll ǫgwa:-ná:d-aˀ\\
	\textsc{1p.p}-town-{\nounstemformer}\\
	\glt `our community'}
\ex adę́ˀhęgǫ:\\
\gll adę́ˀhę-gǫ:\\
 \exsc{ø.prefix.}fence-{\joinerA}-\exsc{in}\\
\glt `in the yard'
\cfex{adę́hęˀ\\
	\gll adę́hę-ˀ\\
	{\noprefix}.fence-{\nounstemformer}\\
	\glt `fence'}
\ex ohnégagǫ:\\
\gll o-hnég-a-gǫ:\\
 \textsc{3s.p}-water-{\joinerA}-\exsc{in}\\
\glt `in the lake, water'
\ex ganyadá:gǫ:\\
\gll ga-nyadá:-gǫ:\\
 \textsc{3s.a}-lake-\exsc{in}\\
\glt `underneath the lake'
\ex ǫhnyá:gǫ:\\
\gll ǫhnyá:-gǫ:\\
 {\noprefix}.river.flats-{\joinerA}-\exsc{in}\\
\glt `in the river flats'
\ex ganéˀdagǫ:\\
\gll ga-néˀd-a-gǫ:\\
 \textsc{3s.a}-valley-{\joinerA}-\exsc{in}\\
\glt `under the valley'
\ex ohéhdagǫ:\\
\gll o-héhd-a-gǫ:\\
 \textsc{3s.p}-earth-{\joinerA}-\exsc{in}\\
\glt `under the ground'
\z
\z

\ea\label{ex:gonex3} body part nouns
\ea senhǫ́hdagǫ:\\
\gll se-nhǫ́hd-a-gǫ:\\
 \textsc{2s.a}-underarm-{\joinerA}-\exsc{in}\\
\glt `your underarm'
\ex sragwáhdagǫ:\\
\gll s-ragwáhd-a-gǫ:\\
 \textsc{2s.a}-sole-{\joinerA}-\exsc{in}\\
\glt `the ball of your foot'
\ex seséhdagǫ:\\
\gll se-séhd-a-gǫ:\\
 \textsc{2s.a}-nape-{\joinerA}-\exsc{in}\\
\glt `the nape of your neck'
\z
\z


\subsection{\stem{-:kˀah} \textsc{beside} (locative)} \label{[-:kˀah]}
The \stem{-:kˀah} \textsc{beside} (\textsc{locative}) ending attaches to basic nouns (\ref{ex:kaloc}, \sectref{ch:Basic nouns}) and atypical words functioning as “nouns” (\ref{ex:kaloc2}, \sectref{Atypical words}). The vowel preceding the suffix is always long.

\newpage
\ea\label{ex:kaloc} \textsc{basic nouns}
\ea ganǫhsá:kˀah\\
\gll ga-nǫhs-á:-kˀah\\
 \textsc{3s.a}-house-{\joinerA}-\exsc{beside}\\
\glt `beside the house'
\cfex{ganǫ́hsaˀ\\
	\gll ga-nǫ́hs-aˀ\\
	\textsc{3s.a}-house-{\nounstemformer}\\
	\glt `house'}
\ex akyęda̱hkwá:ˀkˀah\\
\gll akyęda̱hkw-á:-kˀah\\
 {\noprefix}.chair-{\joinerA}-\exsc{beside}\\
\glt `beside the chair'
\cfex{akyę́da̱hkwaˀ\\
\gll akyę́da̱hkw-aˀ\\ 
{\noprefix}.chair-{\nounstemformer}\\ 
\glt ‘chair’ }
\ex ga̱hę:dá:ˀkˀah\\
\gll ga̱-hę:d-á:ˀ-kˀah\\
 \textsc{3s.a}-field-{\joinerA}-\exsc{beside}\\
\glt `by the field'
\cfex{ga̱hę́dagǫ:\\
	\gll ga̱-hę́d-a-gǫ:\\
	\textsc{3s.a}-field-{\joinerA}-\exsc{in}\\
	\glt `in the field, meadow'}
\ex wadęhę́:kˀah\\
\gll w-adęhę́:-kˀah\\
 \textsc{3s.a}-fence-{\joinerA}-\exsc{beside}\\
\glt `alongside the fence'
\cfex{adę́hęˀ\\
	\gll adę́hę-ˀ\\
	{\noprefix}.fence-{\nounstemformer}\\
	\glt `fence'}
\z
\z

\ea\label{ex:kaloc2} atypical words functioning as “nouns” \\
gi̱hę́:kˀah\\
\gll gi̱hę́:-kˀah\\
 {\noprefix}.river-\exsc{beside}\\
\glt `along the river'
\cfex{gi̱hę:\\
	\gll gi̱hę:\\
	{\noprefix}.river\\
	\glt `river'}
\z


The \stem{-:kˀah} ending also resembles a verb \xref{ex:kaloc3}. However, words with the verb proper begin with \stem{ni-} \textsc{\partitive}, while words with the \stem{-:kˀah} \textsc{beside} ending do not. Also, the verb \stem{kˀah} means ‘short’ while the \stem{-:kˀah} suffix means ‘beside’.

\ea\label{ex:kaloc3} 
nigaˀdre̱hdá:kˀah\\
\gll ni-ga-ˀdre̱hd-á:-kˀah\\
{\partitive}-\textsc{3s.a}-car-{\joinerA}-short.{\stative}\\
\glt ‘short car’
\z

\subsection{\stem{-kdagyeˀ} \textsc{alongside} (locative)} \label{[-kdagyeˀ]}
The \stem{-kdagyeˀ} \textsc{alongside} \textsc{(locative)} ending attaches to basic nouns (\ref{ex:kdagye}, \sectref{ch:Basic nouns}) and atypical words functioning as “nouns” (\ref{ex:kdagye2}, \sectref{Atypical words}).

\ea\label{ex:kdagye} basic nouns\\
ganǫhsakdá:gyeˀ\\
\gll ga-nǫhs-a-kdá:gyeˀ\\
 \textsc{3s.a}-house-{\joinerA}-\exsc{alongside}\\
\glt `along the house'
\cfex{ganǫ́hsaˀ\\
	\gll ga-nǫ́hs-aˀ\\
	\textsc{3s.a}-house-{\nounstemformer}\\
	\glt `house'}
\z


\ea\label{ex:kdagye2} \textsc{atypical words} functioning as “nouns”\\
gi̱hę́kdagyeˀ\\
\gll gi̱hę́-kdagyeˀ\\
 {\noprefix}.river-\exsc{alongside}\\
\glt `along the river'
\cfex{gi̱hę:\\
	\gll gi̱hę:\\
	{\noprefix}.river\\
	\glt `river'}
\z


\newpage
The atypical word in \xref{ex:kdagye3} also resembles the  \stem{-kdagyeˀ} suffix and denotes a location.

\ea\label{ex:kdagye3} 
akdá:gyeˀ ‘the edge’, ‘beside’
\z

\section{Verbs that resemble noun suffixes (“suffix-like” verbs)} \label{Verbs that resemble noun suffixes}
“Suffix-like” verbs are obligatorily-incorporating, stative-only verbs (see \sectref{Verbs requiring an incorporated noun}). They resemble noun suffixes in that they also appear after noun stems and modify the meaning of the noun. However suffix-like verbs describe a state or attribute (\ref{ex:fakevbsuf}, see \sectref{Verbs occurring only in one aspect (stative or habitual)} for more examples), while noun suffixes either denote the noun word class (\sectref{Suffixes that identify words as being nouns}) or various locations (\sectref{Locative noun suffixes}). 

\ea\label{ex:fakevbsuf} “suffix-like” verb stem \stem{+deˀ} ‘existing’, not a noun suffix
\ea ohwę́jadeˀ\\\label{ex:fakevbsufa}
\gll o-hwę́j-a-deˀ\\
 \textsc{3s.p}-earth-{\joinerA}-exist.{\stative}\\
\glt `existing earth, land'
\cfex{ohwę́jagǫ:\\
\gll o-hwę́j-a-gǫ:\\
\textsc{3s.p}-earth-{\joinerA}-\textsc{in}\\
\glt `under the earth, ground'}
\ex gáǫdadeˀ\\\label{ex:fakevbsufb}
\gll gá-ǫd-a-deˀ\\
 \textsc{3s.a}-log-{\joinerA}-exist.{\stative}\\
\glt `a beam'
\cfex{gáǫ:daˀ\\
\gll gá-ǫ:d-aˀ\\
\textsc{3s.a}-log-{\nounstemformer}\\
\glt `a log'
}
\z
\z
