\documentclass[output=paper]{langscibook}
\ChapterDOI{10.5281/zenodo.7096304}
\author{Christopher Hutton\orcid{}\affiliation{University of Hong Kong} and James McElvenny\affiliation{University of Siegen}}
\title{Linguistics under National Socialism}
\abstract{\noabstract}

\begin{document}
\maketitle 

\paragraph*{JMc:} In the most recent podcast episodes, which have focused on Central Europe in the first half of the twentieth century, we’ve met a number of figures who were forced into exile by the rise of fascism. In this episode, we turn our attention to those who stayed behind and found a place for themselves and their scholarship under the new regimes. We also take a moment to consider the parallels between this period and today. To guide us through these topics, we’re joined by Christopher Hutton, Professor of English Linguistics at the University of Hong Kong.

So, Chris, you’ve written extensively on the place of language study and anthropology in the so-called Third Reich. Your publications on this topic include the 1999 book \textit{Linguistics in the Third Reich} and the 2005 \textit{Race in the Third Reich}. Can you tell us what the main themes of Nazi language study were? How did these themes differ from language study in the democratic countries of the time?

\paragraph*{CH:} I think you have to start in the 1920s and ’30s. Remember that Germany is really the centre of linguistics internationally at that time. So German linguistics was influential around the world, but it had some peculiarities that were not adopted internationally. One of these is the centrality of the concept of \textit{Volk}. This is very different from, say, French or Anglo-American linguistics. And then you have these ideas about mother tongue and discussions of bilingualism, language islands, \textit{Sprachinselforschung}.

I think there is a contrast with what’s going on in France, and in the UK, and the US. Of course, you do have in the US the Boasian tradition – Humboldt, Boas – but it’s focused mainly on indigenous cultures of North America, so it has this kind of niche, and in there, it’s a sort of rescue operation in some ways, and in some ways politically liberal. Boas himself counts as a liberal, although there is a more complicated story there, actually.

If you think of Saussure’s \textit{langue}, a concept – whatever you make of it – which is very free of some of the ideology that sticks to the German concepts of \textit{Volk} and language community and so on, it seems almost Cartesian in its abstractness, and I think that is very significant. Saussure does have a reception in Germany, and there is structural linguistics, but the idea tends to be that, well, the conceptual structure of the language should have some basis in history, tradition, and so on. So German linguistics is very different from Saussurean structuralism, which, if you take it puristically, is entirely synchronic. There is no real narrative you can make of the history of a language, in a sort of ideological “The story of language X”.

I think there is a kind of continental sensibility because of the effect of World War I on state boundaries in Europe, and there is a level of insecurity and uncertainty which doesn’t apply in the US and the UK, which really makes a big difference. Because German linguistics falls largely under \textit{Germanistik}, which was an extremely conservative discipline, the people in \textit{Germanistik} on the whole were on the right. They didn’t necessarily become true Nazis, but they were certainly on the \textit{völkisch} side, as opposed to, say, sociology in Germany.

\paragraph*{JMc:} Can I just ask about what you said about Boas, that there’s a connection there with the German tradition but that Boas’ work was focused on American indigenous languages? Do you think that there’s still a connection there, though, with how the German nationalists in Nazi Germany conceived of themselves? Because if you go back into the nineteenth century, there’s a lot of sympathy, especially in German pop culture, with the plight of indigenous people in America – if you think of the novels of Karl May, for example. There’s also this fascination of the part of German scholars with things like Tacitus’s descriptions of the \textit{Germanen} as an indigenous people on the edge of civilization.

\paragraph*{CH:} I think it’s a very good point. Maybe you can look at it this way. There’s hostility to the Anglo-American model of the state, as well as to the French model. These are seen as assimilatory and lacking a kind of organic basis: they’re capitalist and based in law, in some kind of Common Law, which is an individualistic system and promotes, in a way, social movement and mobility, and also sees property as a resource to be exploited.

Although Boas, being Jewish, and also politically liberal, ends up attacking the Nazis, there are parallels there, and you could put it under hostility to modernity, in a way. Sapir maintains some of the same spirit: the ideal of the Native American fishing in that tranquil way, free of the pressures of the modern industrialized world, the timetable, and so on. It’s an attractive image to everybody, but I think this form of Romantic primitivism – or whatever you want to call it – was very powerful in Germany. 

And it also spills over into Celtic studies, and the affinity to Celtic music, culture, again, in opposition to this hostile Common Law English state, the colonial settler state which then threatens to obliterate diversity. It’s true that Common Law gobbles up diversity. Look at Australia and the \textit{terra nullius} doctrine. Once you’re inside the Common Law it may protect you, but if you’re faced with it coming at you, it’s actually really brutal. They had a point, I think.

\paragraph*{JMc:} So on this point of Celtic studies, one of the major areas of applied linguistics that thrived under the National Socialist regime – because it aligned very well with the regime’s interests – was the issue of minority language rights. This was very prominent in Celtic studies, as you mentioned.

So, first of all, in \textit{Germanistik}, there was the issue of \textit{Auslandsdeutsche} – that is, German speakers who were living outside the political boundaries of Germany, predominantly in Eastern Europe, but also in migrant communities in North America. But the issue of minority language rights was also deployed against the enemies of Nazi Germany – and this is where Celtic studies comes in – in alleged solidarity with oppressed ethnic groups such as the Bretons in France, the Welsh and the Highland Scots in Britain, and the Irish in Ireland. The Republic of Ireland was already an independent country by this stage, but the historical tensions between the Irish, who traditionally spoke a Celtic language, and the English colonial regime were still there, and Ireland itself was, of course, neutral in World War II. 

But was this scholarship in \textit{Germanistik} and Celtic studies really entwined with the Nazi ideology, or was it just an opportunistic appeal to the interests of the regime in order to secure funding and political support?

\paragraph*{CH:} Well, I think the affinity was sincere. There’s figures like Ludwig Mühlhausen, there’s Leo Weisgerber, and Willy Krogmann. They had very deep affinities to this Celtic culture, and they were very hostile to what the British had done or were doing in Ireland. So I think there is a sincere element to it. There is also an opportunistic element if you look at Heinz Kloss, who was much more concerned with Germans, overseas Germans, or Germans outside the Reich, but he did get a lot of funding in relation to independent research institutes (\textit{Forschungsstellen}).

Another way to look at this question is to look at the east. Michael Burleigh wrote a brilliant book called \textit{Germany turns eastwards}, and it’s about the scholarship of the predominantly Slavic east. What you can see there is a mixture, in policy terms, of getting people on board – so appropriating, assimilating – and also settler colonial ambitions. Some Ukrainians were working with the Nazis, and then you have the Latvian SS, you have collaboration, but in the long run, I guess there was a plan, for the whole of Europe, a mixture of ethnic states in the west and settler colonialism in the east. 

How exactly that would have worked is unclear, but Alfred Rosenberg was saying to Hitler, “The Ukrainians hate Stalin.” Rosenberg was from the east so he was familiar with the situation. And I think Hitler was, on the other hand, much more insistent on a kind of scorched earth policy because of this settler ambition. But they did have a European plan, and it did include a more “natural” ethnic ecology of Western Europe which would have been, I presume, ethnic states under Nazi tutelage, so sort of patron states. 

Certainly, Leo Weisgerber was active in Brittany. And there was an attempt to use Flemish nationalism. Certainly in the case of the academics, they were sincerely interested in the project because they basically distrusted the modern state, the nation-state form, because it’s not organic, but I think there was an overriding cynicism in the higher levels of the Nazi Party. 

It wouldn’t have been a great deal for the ethnic minorities in the end. The ruthlessness of it is such that the kind of autonomy they would have got would have been very, very thin. So again, I think the idea of drawing clean lines is underlying all of this, and the return to the organic state. But the academics didn’t have the intellectual answers, really. And then there’s the overriding technocratic, brutal nature of the project – which becomes stronger and stronger as the war goes on. This re-engineering project is secondary, I think, at a certain point because it’s a brutal battle for survival. 


\largerpage

But a lot of the academics are sincerely invested in these projects, so back to your original question, especially with the Celts, I think there are a lot of affinities, and the academic links went back way before the war, and they still continue, actually. There’s still a Celtic Romanticism in Germany. It’s nothing like it was, but I noticed that when I lived in Germany. There is this Romantic attachment to a particular form of Celtic imagery and way of being as opposed to the kind of hard capitalist modernity of England or the US. So I think that ethos remains – stripped, I should add, of its nasty toxic elements.

\paragraph*{JMc:} OK, so that brings us to the present day. Minority language rights are of course a major issue in mainstream linguistics today, but the focus today is perhaps on indigenous languages in places that have been subject to settler colonialism, such as North and South America and Australia, so the sort of project that Boas was engaged in back in these days. But also in Britain and France, the rights of speakers of Celtic languages are very much on the agenda and have managed to win some government support, and even in Germany, some small minorities such as the Sorbs in the Lausitz, in Brandenburg and Saxony, who speak a Slavic language, have been able to gain official support.

But today, minority language rights are usually considered a progressive issue, an effort to counteract the deleterious effects of colonialism and the aggressive spread of hegemonic cultures. How can an issue like this have such different, even diametrically opposed, political associations in different historical contexts?

\paragraph*{CH:} I think one of the keys to this is that the language minority politics of Europe between the wars and into the war is about territory. So the whole tension underlying it is the question, whose territory is this? And basically – back to the organic state – if you want to consolidate and survive and not to lose parts of your \textit{Volk}, then it seems that you need political power in those regions in order to protect that.

Obviously, the Germans are hurting because they’ve lost a lot of territory and a lot of their speakers are now citizens of other states, so the whole issue is explosive at the level where people are going to be killed. To bring about this kind of ideal state, you’re going to have to move people or kill them. So it’s very different from the post-war US where it’s an argument about cultural space or about legitimacy or access to social mobility. There’s no underlying murderous potential to that, but of course there’s a lot of social tension around it. So I think that’s one difference.

I think that sociolinguistics has suffered from a single model of this, so if you say “mother tongue language rights”, everyone goes, “Great!” Language politics should include politics. If you look at the politics of these states, it becomes a much more muddled and complicated story. Back in the 1990s, Robert Philipson would go around the world telling everyone to use their mother tongues – but he did it in English, of course – and in a way, it was a one-size-fits-all solution emanating from northern Europe. So my problem, in a way, is that we don’t look enough at the actual politics, the real governmental system, the structures, the resourcing, and all the effects. 

People can pat themselves on the back for saying, “I support language rights,” but they don’t actually cost it in any way, politically or economically. Maybe it’s the problem with the identity left now that it’s not interested in economics. You know, when I was young, Marxists and leftists would talk about economics all the time. Now, they only talk about identity, and it seems to me this is a problem for sociolinguistics.

The situation today is of course better in many ways, more progressive. Take the example of Welsh. Welsh is now enjoying quite a strong degree of official recognition, and that’s great. I don’t see any problem with that, and I think this can keep going further. But every speaker of Welsh is a native speaker of English as well, so it’s a very unusual situation, and I think that’s really beneficial to the kind of possibilities of this situation. 

But in other situations, people are on the edge of these modern states, like in South America. It is a difficult issue. It’s very easy to sit here and go, “They should keep their languages and cultures,” but modernity can be brutal. The Welsh are in modernity, whereas in Brazil – or these Amazonian peoples – getting into modernity will destroy their cultures. I don’t see any easy point of view from here. 

Another huge block of states are the Leninist states or the former Leninist states, which is a vast percentage of the world population – so China, Vietnam, Laos, Burma to a degree, and even India, in a funny way – where you have official minority classifications centrally organized, and the politics of that are very, very different from the minority policies of, say, the US. 

If we think back in the US context, both Uriel Weinrich and Joshua Fishman, in many ways the founders of modern sociolinguistics in the 1950s and ’60s, have a whole list of Nazis in their references. I mean, not one or two, maybe 20 or 25. So how is that possible? Weinrich’s \textit{Languages in Contact}, if you look in the bibliography, there’s a bunch of really nasty, toxic people there, one of whom was executed for war crimes. So how is that possible? It’s because, well, one, I think in Fishman’s case, he just was not interested in the problematic nature of minority politics in the interwar era, and he didn’t understand Kloss, who was both his close collaborator and a former collaborator of the Nazis, and he was worried about protecting the program that he had, which was to promote ethnic revival in the US and globally in the decolonizing world, a kind of rational language politics or language engineering.

\paragraph*{JMc:} But your average sociolinguist – so someone like Weinreich or Fishman – who might be citing lots of Nazis, maybe their principle would be, don’t say that they’re hypocritical, say rather that they’re apolitical – that the ideas are separate from the politics that they were used to support.

\paragraph*{CH:} Well, my theory with Weinreich was that he was trying to protect the discipline, and he did his fieldwork in Switzerland, so he was in the only bit of continental Europe which was still intact in some sense after the war, and I think he was such a straight and high-minded guy that he felt it beneath him to lay into these others.

But Max Weinreich, his father, wrote one of the first books on Nazi scholarship. And in his private correspondence, as Kalman Weiser showed in his \citeyear{weiser2018a} paper, Max Weinreich was scathing about Franz Beranek, one of the Germans who worked on Yiddish. Max Weinreich called Beranek complicit in murder and so on. So there is something strange about that. 

As for Fishman, it is possible he was protecting people. Or maybe he didn’t know. I don’t know whether it was Weinreich who gave him the references. Fishman certainly knew about Georg Schmidt-Rohr. Schmidt-Rohr had a complicated evolution: in \citeyear{schmidt-rohr1932a} he got into political trouble for seemingly suggesting that language could create \textit{Volk}, and then he reoriented himself to get past the Nordicist attacks on him. But he was no liberal. 

Then Fishman gradually stops citing German sources. In a way that’s mapping the end of German dominance in academia after the war, and the rise of the US as the preeminent linguistics power.

\paragraph*{JMc:} What a claim to fame, preeminent linguistics power. It’s not quite as impressive as being the greatest military power or economic power.

\paragraph*{CH:} True, but it goes together a little bit because look at the US university system, and then because of the expansion in the 1960s, American linguistics really took off. American sociolinguistics has a kind of virgin birth in the ’60s. They act as if there was never a European background. There’s something slightly odd about it, even though Kloss is there in meetings with leading figures in America like Dell Hymes and John J. Gumperz, and so on. They seem to forget all the literature from Britain – the British Empire was a key place for linguistic research – as well as all the material on the ethnic politics of Central and Eastern Europe. Sociolinguistics comes along and it’s a very US thing.

\paragraph*{JMc:} That’s probably a good note to end the interview on, so thank you very much for your answers to those questions.

\paragraph*{CH:} Thanks very much. It was good fun. I enjoyed that.

\nocite{boas1911a}
\nocite{boas1911b}
\nocite{fishman1964a}
\nocite{kloss1941a}
\nocite{muehlhausen1939a}
\nocite{philipson1992a}
\nocite{sapir1949a}
\nocite{schmidt-rohr1932a}
\nocite{schmidt-rohr1933a}
\nocite{weinreich1946a}
\nocite{weinreich1953a}
\nocite{weisgerber1939a}
\nocite{burleigh1988a}
\nocite{hutton1999a}
\nocite{hutton2005a}
\nocite{knobloch2005a}
\nocite{weiser2018a}


\sloppy
\PrintPrimarySources{}
\PrintSecondarySources{}

\end{document}
