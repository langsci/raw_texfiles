\documentclass[output=paper]{langscibook}
\ChapterDOI{10.5281/zenodo.7096298}
\author{Chloé Laplantine\orcid{}\affiliation{University of Paris, CNRS Histoire des théories linguistiques} and James McElvenny\affiliation{University of Siegen}}
\title{Émile Beneveniste}
\abstract{\noabstract}
\begin{document}
\maketitle 


\paragraph*{JMc:} In recent interviews, we’ve been talking about the history of linguistic structuralism in Europe. We’ve mentioned that it was above all in France where structuralism really took hold. By the middle of the twentieth century, structuralism in France had become something of an official doctrine underpinning the humanities and social sciences. To get a better idea of the career of French structuralism, we’re joined today by Chloé Laplantine from the CNRS Laboratory for the History of Linguistic Theories in Paris. She’s going to tell us in particular about the life and work of Émile Benveniste, a key figure in French linguistics, who did much to elaborate structuralist thought.

So, Chloé, tell us: Who was Émile Benveniste? How did he become one of the leading French linguists of the twentieth century?

\paragraph*{CL:} Thank you very much, James, for inviting me to answer your questions. It’s a pleasure to talk today with you about Émile Benveniste, who is indeed considered an important linguist of the twentieth century. I’ll try today to shed light on his original contributions to reflection on language.

Let’s first say a few words about his life and career. Benveniste was born in Aleppo, Syria, in 1902. His parents where teachers for the Alliance israélite internationale. He was sent to Paris in 1913 to pursue rabbinic studies, to become a rabbi, at the Petit séminaire. There, he met Sylvain Lévi, who was filling in for another teacher during the war. Sylvain Lévi – who belonged to the same generation as Ferdinand de Saussure – was an important figure in Oriental studies, particularly interested in Sanskrit, in the history of Indian religion and culture, teaching Sanskrit language and literature at the Collège de France. 

Sylvain Lévi apparently found in Benveniste a promising student, and sent him to the Sorbonne. At the Sorbonne, Benveniste attended the classes of Joseph Vendryès, with whom he studied Celtic linguistics, and under whose direction he wrote his first essay in 1920, “The sigmatic futures and subjunctives in Archaic Latin”. Benveniste also attended the classes in comparative grammar given by Antoine Meillet at the Collège de France, as well as frequenting the École des langues orientales, where he studied Sanskrit with Jules Bloch and Vedic with Louis Finot. Rounding things out, he also studied Latin paleography with Émile Chatelain at the École des hautes études. 

Benveniste was one of the young and brilliant students who were gathering around Antoine Meillet. Others we should also mention include Louis Renou, Pierre Chantraine, Jerzy Kuryłowicz, and Marie-Louise Sjoestedt. As we can already see, Benveniste’s work originated in the French tradition of Oriental studies, comparative grammar, philology, and within the framework of existing institutions like the École des hautes études, the Collège de France, the Société de linguistique de Paris, the Sorbonne and the École des langues orientales.

In 1927, Meillet invited Benveniste, then aged only 25, to replace him at the École des hautes études, and 10 years later, in 1937 he was named to the chair of comparative grammar at the prestigious Collège de France, again replacing Meillet who had died the previous year.

Now that we have seen the institutional background to Benveniste’s work, let’s go into details. What strikes me the most when looking at the classes Benveniste gave at the Collège de France – when reading their summaries or consulting his manuscripts – is his understanding of the notion of “comparative grammar”. We can see that from the beginning, that is to say 1937, he examined general problems in linguistics on the empirical basis of a great variety of languages, which was something quite new at the time. Meillet, teaching comparative grammar before Benveniste, was already looking for data in non-Indo-European language families, but with Benveniste – who was trained as an Indo-Europeanist – we see clearly that linguistics is not only Indo-European linguistics, or even more so that our knowledge about languages can be refined or even renewed in light of non-Indo-European languages. 

This might make us think of Franz Boas or Edward Sapir in America. Just to give an example, one of Benveniste’s first lectures in 1937 was devoted to the notion of negation; a look at the manuscripts shows us that he was particularly interested in the system of negation in Greek, but also collected quite a bit of information on negation in many different languages – Chinook, Inuit (which back then was usually called “Eskimo”), Khoekhoe (back then called “Hottentot”), Yakut, German, etc. What is more, his research doesn’t consist in a collection of facts but leads to the formulation of a general theory of negation.  We also see from his notes that, while preparing his class, he was reading Jespersen on negation in English, Jacob van Ginneken’s \textit{Principes de linguistique psychologique}, but also Hegel, Henri Bergson on the idea of “nothingness”, and Heidegger.

I think this example gives us a good idea of the originality of Benveniste’s approach, his openness to the empirical diversity of languages, and the constant tension between this empirical diversity and the formulation of a general linguistic theory. We might quote here a passage from one of his articles, “Coup d’œil sur le développement de la linguistique”, published in 1963. He writes: “It is with languages that the linguist deals, and linguistics is primarily the theory of languages. But […] the infinitely diverse problems of particular languages have in common that, when stated to a certain degree of generality, they always have a bearing on language in general.” 

I think, in this passage, we can hear something characteristic of Benveniste’s approach, which is to consider that knowledge may always be called into question – and this is not a structuralist attitude. This attitude of critical distance appears clearly in the notion of \textit{problème}, which he frequently uses in his writings, and which he chose for the title of his volume of collected papers, \textit{Problèmes de linguistique générale}, published in \citeyear{benveniste1966a}. 

Most of Benveniste’s writings are devoted to problems in Indo-European linguistics. But these articles or books, as specialized as they may sometimes look -- if you consider their titles -- have in common that they are not confined to a purely linguistic analysis. When Benveniste works on the system of tenses in Latin, or on the distinction between nouns for agents and nouns for actions in Indo-European, his analysis of the formal system of the languages brings to light unconscious cultural representations. 

We can offer another example: in his article “Two different models of the city”, Benveniste compares two ways to conceive of the politics involved in the relation of the citizen to the city. He shows that the Latin \textit{civis} is a term of reciprocity and mutuality – one is the \textit{civis} only of another \textit{civis} – and that the derived term \textit{civitas} is the whole of these relations of reciprocity. The equivalent Greek term, \textit{polis}, is quite different: \textit{polis} is an abstract concept from which the term \textit{polites} is derived, the citizen being then only a part of a preconceived whole.

In the same way, when Benveniste works on the notion of rhythm, or on the notion of eternity, by examining the history of linguistic forms through examples taken from philosophers, historians, or poets, he brings to light conceptions specific to particular societies, like an ethnographer would do, and at the same time unveils an archaeology of our conceptions. This is precisely what he did with his book \textit{Le vocabulaire des institutions indo-européennes}, which can be considered a book of linguistic ethnography, a very different approach from that of ethnologists who would generally consider language as something contained within the society. For Benveniste, language is not contained \textit{within} the society; it is the interpreter of society.

\paragraph*{JMc:} What were the main contributions of Benveniste to structuralist theory and what impact did his work have on the development of structuralism, both within disciplinary linguistics and more broadly?

\paragraph*{CL:} We see in many of his articles that Benveniste considers Saussure as a starting point for the study of language – not the only one, of course, but an important starting point – and this for several reasons, among which we can mention the idea that language is a \textit{form}, not a \textit{substance}, that language is never given as a physical object would be, but only exists in one’s point of view, and thus the necessity for the linguist to acquire a critical distance and consciousness of his or her own practice. Saussure speaks of the necessity of \textit{showing the linguist what he or she does}.

Benveniste recognizes everywhere the importance of Saussure, but also says that what proves the fertility of a theory lies in the contradictions to which it gives rise. In “La nature du signe linguistique” published in the first issue of \textit{Acta Linguistica} in 1939, he argues, against Saussure, that the relation between the concept and the acoustic image is not \textit{arbitrary} but \textit{necessary}, the idea of arbitrariness being, according to Benveniste, a residue of substantialist conceptions of language. In articles such as “La forme et le sens dans le langage”, in 1966, or “Sémiologie de la langue”, in 1968, Benveniste invites us to go beyond Saussure and the dimension of the sign, which, according to him, is only one aspect of the problem of language and doesn’t do justice to its living reality. He suggests a tension between two dimensions: one that he calls ``semiotic'' which is the dimension of the \textit{sign}, and involves the faculty of recognition (a sign exists or does not exist); the other dimension is called ``semantic'', it is the universe of \textit{discourse} and \textit{meaning}, its unity being the \textit{sentence} and the faculty involved being \textit{comprehension}.

Here we find not only something new in comparison with Saussure, but also something that does not match at all with structuralist presuppositions. This point of view on language is totally different as it is now conceived as an activity. Each enunciation is a unique event which vanishes as soon as it is uttered. It is never predictable; the universe of discourse is infinite. Benveniste writes that “[t]o say ‘hello’ to somebody every day is each time a reinvention”, and you’ll notice that he chooses a sentence word as an example. You can repeat the same word; it is never the same enunciation.

Another notion that goes with enunciation is that of subjectivity. Benveniste criticizes the reduction of language to an instrument of communication which supposed the separation of language from the human speaker. For Benveniste, the speaker is \textit{in} language, and even more constitutes themselves in and through language as a \textit{subject}. We can quote here a manuscript note: “Language as lived[.] Everything depends on that: in language taken on and lived as a human experience, nothing has the same meaning as with language viewed as a formal system and described from the outside.”

In 1967 Benveniste undertook research on the French poet Charles Baudelaire. Maybe it was an answer to Jakobson and Levi-Strauss’s structuralist analysis of Baudelaire’s poem \textit{Les Chats} published in \citeyear{jakobson1962b}. When Jakobson and Levi-Strauss take the poem to pieces, analyse it with the tools of structuralist linguistics, nothing remains of the originality of Baudelaire’s poem. Their analysis could be repeated indifferently with any poem. What Benveniste tries to do in opposition to this is to show how Baudelaire re-invents language in his poems, how he invents an original experience or vision that he shares with the reader. This research on Baudelaire’s language, which was never published, develops an important reflection on meaning. A poem by Baudelaire doesn’t work the same way as ordinary language. For Benveniste, Baudelaire creates a \textit{new semiology}, a language that escapes the conventions of discourse.

So I think we’ve seen that Benveniste’s work extends far beyond the framework of structuralist thought. I mentioned earlier his curiosity about linguistic diversity. I could have said a few words about the research he did in 1952 and ’53 on the Northwest Coast of America on the Haida, Tlingit, and Gwich’in languages. His curiosity about these languages and cultures was motivated, among other reasons, by an interrogation of \textit{meaning}: he wanted to investigate the ways language signifies and symbolizes. And he had the feeling that linguistics, in particular in America, didn’t care about meaning anymore. But for Benveniste, much more than a means of communication, language is a means of living: \textit{Bien avant de servir à communiquer, le langage sert à vivre}.

\paragraph*{JMc:} That’s great. Thank you very much, Chloé, for talking to us today.

\paragraph*{CL:} Thank you very much, James!

\nocite{unknown-a}
\nocite{benveniste1937a}
\nocite{benveniste1966a}
\nocite{benveniste1971a}
\nocite{benveniste1969a}
\nocite{benveniste1974a}
\nocite{benveniste2011a}
\nocite{benveniste2012a}
\nocite{benveniste2015a}
\nocite{jakobson1962b}
\nocite{jespersen1917a}
\nocite{adam2012a}
\nocite{dessons2006a}
\nocite{laplantine2011a}
\nocite{laplantine2019a}


\sloppy
\PrintPrimarySources{}
\PrintSecondarySources{}

\end{document}
